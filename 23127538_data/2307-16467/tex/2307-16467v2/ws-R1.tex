%%%%%%%%%%%%%%%%%%%%%%%%%%%%%%%%%%%%%%%%%%%%%%%%%%%%%%%%%%%%%%%%%%%%%%%%%%%%
%% Trim Size: 9.75in x 6.5in
%% Text Area: 8in (include Runningheads) x 5in
%% ws-ijmpd.tex   :     19-4-07
%% Tex file to use with ws-ijmpd.cls written in Latex2E.
%% The content, structure, format and layout of this style file is the
%% property of World Scientific Publishing Co. Pte. Ltd.
%% Copyright 1995, 2002 by World Scientific Publishing Co.
%% All rights are reserved.
%%%%%%%%%%%%%%%%%%%%%%%%%%%%%%%%%%%%%%%%%%%%%%%%%%%%%%%%%%%%%%%%%%%%%%%%%%%%
%

%\documentclass[draft]{ws-ijmpd}
\documentclass{ws-ijmpd}
\usepackage{tabularx}
\usepackage{multirow}
\usepackage{url}
\usepackage[super,compress]{cite}
\begin{document}

\markboth{Xie et al.}
{Constraints on Cosmological Models with Gamma-Ray Bursts in Cosmology-Independent Way}

%%%%%%%%%%%%%%%%%%%%% Publisher's Area please ignore %%%%%%%%%%%%%%%
%
\catchline{}{}{}{}{}
%
%%%%%%%%%%%%%%%%%%%%%%%%%%%%%%%%%%%%%%%%%%%%%%%%%%%%%%%%%%%%%%%%%%%%

\title{Constraints on Cosmological Models with Gamma-Ray Bursts in Cosmology-Independent Way}

\author{Hanbei Xie, Xiaodong Nong, Huifeng Wang, Bin Zhang, Zihao Li, and Nan Liang$^*$}

\address{Key Laboratory of Information and Computing Science Guizhou Province, Guizhou Normal University, Guiyang, Guizhou 550025, China\\
School of Cyber Science and Technology, Guizhou Normal University, Guiyang, Guizhou 550025, China\\
School of Mathematical Sciences,Guizhou Normal University, Guiyang, Guizhou 550025, China\\
Joint Center for FAST Sciences Guizhou Normal University Node, Guiyang, Guizhou 550025, China\\
$^*$E-mail:liangn@bnu.edu.cn}



\maketitle

\begin{history}
\received{Day Month Year}
\revised{Day Month Year}
\end{history}

\begin{abstract}
In this paper, we present a cosmology-independent method to constrain cosmological models from the latest 221 gamma-ray bursts (GRBs) sample, including 49 GRBs from Fermi catalog with the Amati relation  (the $E_{\rm p}$-${E}_{\rm iso}$ correlation), which calibrated by using a Gaussian process from the Pantheon+ type Ia supernovae (SNe Ia) sample. %Then we combine the GRB data in the redshift range of ${z \geq 0.8}$, ${z \geq 1.0} $ and ${z \geq 1.4}$ with the OHD data separately, and utilize the MCMC numerical method to fit the parameters of the cosmological model.
%We construct the Hubble diagram and constrain cosmological models  with the GRBs at high redshift and the latest observational Hubble data (OHD)  by the Markov Chain Monte Carlo (MCMC) method.
With 182 GRBs at $0.8\le z\le8.2$ in the Hubble diagram and the latest observational Hubble data (OHD) by the Markov Chain Monte Carlo (MCMC) method, we obtained $\Omega_{\rm m}$ = $0.348^{+0.048}_{-0.066}$ and $h$ = $0.680^{+0.029}_{-0.029}$  for the flat $\Lambda$CDM model, and  $\Omega_{\rm m}$ = $0.318^{+0.067}_{-0.059}$, $h$ = $0.704^{+0.055}_{-0.068}$, $w$ = $-1.21^{+0.32}_{-0.67}$ for the flat $w$CDM model.
%With the  GRBs at high redshift in the constructed Hubble diagram and the latest observational Hubble data (OHD), we constrain cosmological models in a flat space with the Markov Chain Monte Carlo (MCMC) method, we obtained $\Omega_{\rm m}$ = $0.314^{+0.046}_{-0.063}$ and $h$ = $0.681^{+0.029}_{-0.029}$  for the flat $\Lambda$CDM model, and  $\Omega_{\rm m}$ = $0.269^{+0.10}_{-0.055}$, $h$ = $0.683^{+0.042}_{-0.072}$, $w$ = $-1.00^{+0.63}_{-0.29}$ for the flat $w$CDM model at the 1$\sigma$ confidence level.
These results are consistent with those %from the simultaneous fitting method
in which the coefficients of the Amati relation and the cosmological parameters fitted simultaneously.
%and we find that the $\Lambda$CDM model is favoured respect to the $w$CDM model and the CPL model with the selection criteria.

\end{abstract}

\keywords{gamma-ray bursts, general - cosmology, observations.}

\ccode{PACS numbers:}

%\tableofcontents

\section{Introduction}	
%Type Ia supernovae (SNe Ia) are a unique type of supernova explosion known for their consistent brightness and duration, making them valuable objects for studying the early evolution of the universe \citep{Riess1998,Perlmutter1999}. They are often referred to as the "standard candles" of the universe. However, the maximum redshift at which SNe Ia have been detected is currently around $z\sim2.3$ \citep{Scolnic2018}. To investigate cosmic evolution at higher redshift ranges, it becomes necessary to observe celestial objects with even higher redshifts than SNe Ia.
Long Gamma-ray bursts (GRBs) are the most intense and energetic bursts of gamma rays from the cosmic space within a short period of time. Currently, the maximum observable redshift of GRBs is estimated to be around $z=9.4$ \cite{Cucchiara2011}, which is significantly greater than Type Ia supernovae (SNe Ia), with the maximum observable redshift $z\sim2.3$ \cite{Scolnic2018}.
%SNe Ia are a unique type of supernova explosion known for their consistent brightness and duration, making them valuable objects for studying the early evolution of the universe \citep{Riess1998,Perlmutter1999}. They are often referred to as the "standard candles" of the universe. However, the maximum redshift at which SNe Ia have been detected is currently around $z\sim2.3$ \citep{Scolnic2018}.
%Similar to SNe Ia, GRB also exhibit luminosity relationships, and the photons  emitted by GRBs are not affected by interstellar dust extinction. Consequently, GRBs provide a means to explore cosmic evolution at higher redshift ranges.
Several empirical GRB luminosity relations, which are connections between measurable properties of the instantaneous gamma-ray emission and the luminosity or energy, have been proposed to standardize GRBs \cite{Fenimore2000,Norris2000,Amati2002,Ghirlanda2004a,Yonetoku2004,Liang2005,Firmani2006,Ghirlanda2006,Schaefer2007,Tsutsui2009a,Izzo2015}. %see e.g., \cite{Ghirlanda2006}, and \cite{Schaefer2007} for reviews.
However, the early studies had usually calibrated the luminosity relations of  GRBs  by assuming a FIDUCIAL cosmological model \cite{Dai2004,Schaefer2007}. %Due to a lack of low-redshift samples at $z<0.1$,
Therefore, using these model-dependent GRB data to constrain cosmological models leads to the circularity problem \cite{Ghirlanda2006}.
Liang \emph{et al.}\cite{Liang2008} proposed  a model-independent method  to calibrate the luminosity relations of GRBs with SNe Ia data by the interpolation method and construct the GRB Hubble diagram, which can be used to constrain cosmological models \cite{Capozziello2008,Capozziello2009,Wei2009,Wei2010,Liang2010,Liang2011,Wang2016,Liu2022b}. The luminosity relations of GRBs can be calibrated with SNe Ia data by the similar methods \cite{LZ2008,Kodama2008,Capozziello2010,Gao2012,Liu2015,Izzo2015,Demianski2017a,Demianski2021,Shirokov2020,Muccino2021,Lovyagin2022,Tang2022,Liang2022}.%The luminosity distances at the redshift of the low-redshift GRB data can be derived by interpolating the SNe Ia data directly, and then the values of the coefficients of the GRB luminosity relation can be determined from these low-redshift GRB data. Extrapolating these results on the high-redshift GRB data can constructing the GRB Hubble diagram.


Furthermore, the observational Hubble data (OHD) obtained with the cosmic chronometers (CC) method, which related the evolution of differential ages of passive galaxies at different redshifts \cite{Jimenez2002,Jimenez2003},
have unique advantages to calibrate GRBs in a model-independent way.
Amati \emph{et al.} \cite{Amati2019} proposed an alternative  method to calibrate GRB correlations by using the OHD through the B\'ezier parametric curve and built up a Hubble diagram consisting of 193 GRBs with the Amati relation (the $E_{\rm p}$-${E}_{\rm iso}$ correlation) \cite{Amati2002}.
Following this method \cite{Amati2019}, % from OHD fitted by the B\'ezier parametric to calibrate GRBs \citep{Amati2019},
several works have  constrained  cosmological models with the Amati relation %of the latest GRB data
\cite{Montiel2021,Luongo2021,Luongo2023,Muccino2023}.




On the other hand, the simultaneous fitting method, in which the coefficients of relations and the parameters of the cosmological model are constrained simultaneously, has been proposed to avoid the circularity problem  \cite{Amati2008}.
%However, the circularity problem cannot be circumvented completely by means of  statistical approaches that rely on a specific cosmological model for joint fitting \citep{Amati2008,Li2008,Wang2008}.
Khadka \emph{et al.} \cite{Khadka2021} compile a data set of 118 GRBs (the A118 sample \cite{Khadka2020}, including 25 Fermi GRB sample \cite{Dirirsa2019}) with the smallest intrinsic dispersion from the total 220 GRBs (the A220 sample) with the Amati relation to derive the correlation and cosmological model parameters simultaneously. %With the A220 and the A118 GRB samples,
Cao \emph{et al.} \cite{Cao2022MN510,Cao2022MN512,Cao2022MN516} used the Amati relation \cite{Amati2002} with the A220 and the A118 GRB samples in conjunction with \textbf{the Dainotti relation}\footnote{\textbf{Compared to GRB relations of the prompt emission phase, the relations involving the X-ray afterglow plateau phase \cite{Dainotti2008,Cardone2009,Cardone2010} exists less variability in its features. Dainotti \emph{et al.} proposed the relation between the plateau luminosity and the end time of the plateau in X-ray afterglows (2D Dainotti relation\cite{Dainotti2008}), %$L_0$-$t_b$),
which have been used to cosmological constraint \cite{Dainotti2010,Dainotti2011a,Dainotti2011b,Dainotti2013a,Dainotti2013b,Dainotti2015a,Dainotti2015b,Dainotti2017a}. Furthermore, the GRB  Fundamental Plane relation (the 3D Dainotti relation) among the rest-frame time and X-ray luminosity at the end of the plateau emission and the peak prompt luminosity %($T^*_{X}$-$L_{X}$-$L_{\rm{peak}}$)
with small intrinsic scatter has been found \cite{Dainotti2016,Dainotti2017b,Dainotti2020a,Dainotti2021a,Srinivasaragavan2020}. Some similar 2D and 3D relations  with the plateau in the X-ray afterglows has also been found \cite{Hu2021,Wang2022,Li2023}.
Recently, the relationship in optical wavelengths between the optical rest-frame end time and the optical luminosity at the end of the plateau has been found \cite{Dainotti2020b}.
Very recently, the GRB relation  in radio plateau phase afterglows  has also been investigated \cite{Levine2021,Tian2023}.}} %\textbf{compiled by Wang et al. \cite{Hu2021} and Hu et al. \cite{Wang2022}}
to constrain cosmological model parameters by the simultaneous fitting method.
\textbf{They showed that Platinum sample including 50 GRB data can be standardized with a cosmological-model-independent 3D Dainotti relation\cite{Cao2022MN512}, and  the 3D Dainotti is strongly favoured over the 2D one with different GRB data compilation \cite{Cao2022MN516}. %Platinum GRB data compilation and the LGRB95 data compilation, as well as the GRB combination of the two sample.
Dainotti \emph{et al.} \cite{Dainotti2022MN514,Dainotti2022ApJS261} usd optical and X-ray GRB fundamental planes as cosmological distance indicators.
Dainotti \emph{et al.} \cite{Dainotti2023MN518} correctted the 3D relation by considering the selection and evolutionary effects with a reliable statistical method to obtain a lower central value for the intrinsic scatter.
The 3D Dainotti relation have also been used  with a binned analysis with GRBs, SNe Ia, and baryonic acoustic oscillations (BAOs)  \cite{Dainotti2022PASJ74}; and joint constraints combined GRBs with quasars, SNe Ia, and BAOs \cite{Dainotti2023ApJ951,Bargiacchi2023MN521};
as well as a robust cosmographic technique \cite{Bargiacchi2023MN525}.
}






Up to now, whether the luminosity relations of GRB  are redshift dependent or not is still under debate.
The possible evolutionary effects in GRB relations have been discussed \textbf{in many works \cite{Basilakos2008,Lin2016,Wang2017,Demianski2017a,Demianski2021,Dai2021,Tang2021,Dainotti2022MN514,Dainotti2023MN518}}.
\textbf{Regarding the luminosity function and density rate and cosmological evolution of the formation rate of GRBs, the luminosity relations of GRB could be evolution with redshift\cite{Petrosian2015,Lloyd2019,Tstetova2017,Yu2017,Dainotti2021ApJ914}}.
With the A220 sample, Khadka \emph{et al.} \cite{Khadka2021}  found that the Amati relation is independent
of redshift within the error bars; Liu \emph{et al.} \cite{Liu2022a,Liu2022b}  proposed the improved Amati relation by accounting for evolutionary effects via copula, and  %calibrated the copula relations from SNe Ia by the interpolation method to constrain cosmological models, they
found that a redshift evolutionary correlation is favored slightly;
Kumar \emph{et al.} \cite{Kumar2023}  divided the GRB data into five distinct redshift bins to calibrate the Amati relation, and found that GRBs do seem to evolve with cosmological redshift.

Recently, Jia \emph{et al.} \cite{Jia2022}  found no statistically significant evidence for the redshift evolution  with the Amati relation from the analysis of data in different redshift intervals with a long GRB sample, which contains 221 long GRBs with redshifts from 0.03 to 8.20, including 49 GRBs from Fermi catalog. %They calibrated the Amati correlation by GRBs in a small redshift range by using a model-independent method and then combined the J221 GRBs samples with SNe Ia data to constrain cosmological parameters. The results show that the combination of GRBs data and SNe Ia data constrain the cosmological parameters better. The fitting results support the $\Lambda$CDM model.
Liang \emph{et al.} \cite{Liang2022} calibrated the Amati relation with the A219 sample and the A118 sample by using a Gaussian process from the Pantheon samples with 1048 SNe Ia data points \cite{Scolnic2018}, and constrain cosmological  models in flat space with GRBs at high redshift and 31 OHD via the Markov Chain Monte Carlo (MCMC) numerical method. % obtained by the cosmic chronometers (CC) method. %, which related the evolution of differential ages of passive galaxies at different redshifts \citep{Jimenez2002}.
Li, Zhang \& Liang \cite{LZL2023} calibrated GRB from the latest 32 OHD via the Gaussian process to construct the GRB Hubble diagram with the A118 data set, and constrain Dark Energy models with GRBs at high redshift and  SNe Ia in a flat space by the MCMC method.
More recently, Mu \emph{et al.} \cite{Mu2023} reconstruct cosmography parameters up to fifth order with the Amati relation of the A219 sample \cite{Liang2022} calibrated from Pantheon+ samples \cite{Scolnic2022}, which contains 1701 SNe light curves of 1550 spectroscopically confirmed SNe Ia at redshift $z<2.26$. %The results incated that the current GRBs data points are still not give any viable prediction of the kinematics of our universe at high redshifts.

%\cite{Wang2022} used a tight correlation between the plateau luminosity and the end time of the plateau in the X-ray afterglows out to the redshift $z = 5.91$.

In this paper, we utilize the latest 221 GRB data compiled in Ref.~\refcite{Jia2022} (the J221 sample) and the Pantheon+ sample \cite{Scolnic2022} to calibrate the Amati relation by Gaussian process at low redshift, and obtained the  Hubble diagram of GRBs. %Then we combine the GRB data at high redshift %in the redshift range of ${z \geq 0.8}$, ${z \geq 1.0} $ and ${z \geq 1.4}$, respectively,with the latest OHD data \citep{LZL2023} to fit the parameters of the cosmological model by useing the MCMC numerical method.
We constrain cosmological models  with the GRBs at high redshift  and the latest 32 OHD data \cite{LZL2023}  by the MCMC method. Finally, we also use the simultaneous fitting method for constraints on cosmological models.


\section{CALIBRATION OF THE AMATI RELATION}
The Amati relation \cite{Amati2002}, which connects the spectral peak energy and the isotropic equivalent radiated energy (the $E_{\rm p}$-${E}_{\rm iso}$ correlation) of GRBs, can be expressed as
\begin{equation}y= a + bx,\end{equation}
where $y={\rm log}_{10}\frac{E_{{\rm is o}}}{{\rm 1erg}}$, $x={\rm log}_{10}\frac{E_{{\rm p}}}{{\rm 300keV}}$,
and $a$ and $b$ are free coefficients, % needed to be calibrated from  the observed data.
$E_{{\rm iso}}$ and $E_{\rm p}$ can be calculated by
\begin{equation}E_{{\rm iso}} = 4\pi d^2_L(z)S_{{\rm bolo}}(1+z)^{-1},\quad E_{\rm p} = E^{{\rm obs}}_{{\rm p}}(1+z), \end{equation}
where $E^{{\rm obs}}_{{\rm p}}$ and  $S_{{\rm bolo}}$ are the GRB spectral peak energy and bolometric fluence.
%We use the values of $E_{\rm p}$ and $E_{\rm iso}$ can be found in  Tab. 1 in  \cite{Jia2022}.
It should be noted that the values of $E_{{\rm iso}}$ from Tab. 1 in Ref.~\refcite{Jia2022} are related with luminosity distance $d_L$, which depend on cosmological models.  The luminosity distance can be calculated by $ d_{\mathrm{L}}(z)= \frac{c(1+z)}{H_{0}}\int_{0}^{z} \frac{\mathrm{d} z^{\prime}}{\sqrt{\Omega_{m}(1+z^{\prime})^{3}+\Omega_{\mathrm{\Lambda}}}}$, where $\Omega_{\rm m}$ represents the parameter for non-relativistic matter density, $\Omega_{\Lambda}$ represents the cosmological constant density, and $H_0$ represents the Hubble constant. Jia \emph{et al.} \cite{Jia2022}  used the standard cosmological parameters from Plank Collaboration \cite{Plank2020} ($\Omega_{\rm m}$ = 0.315, $\Omega_{\Lambda}$ = 0.685, and $H_0$ = 67.4 km $s^{-1}$ ${{\rm Mpc}}^{-1}$)  to obtain the values of $E_{{\rm iso}}$.

% Figure environment removed

Gaussian process is a fully Bayesian method for smoothing data, which can effectively reduce the errors of reconstructed results \cite{Seikel2012a,Seikel2012b}. Recently, Gaussian process has been widely applied to the field of cosmology \cite{Lin2018,Li2021,Sun2021,Wang2022,Liang2022,LZL2023,Benisty2023}.In order to obtain model-independent $E_{{\rm iso}}$, we use a Gaussian process to reconstruct the values of the luminosity distance ($d_L$) of GRBs  from SNe Ia data. In the Gaussian process, the function values $f(z)$  are correlated by a covariance function $k(z,\tilde z)$ to characterize the connection between the function values at different reconstructed points.
We adopt a squared exponential covariance function with the property of infinite differentiability suitable for reconstructing the shape of the function, which is given by  proposed by Seikel \emph{et al.} \cite{Seikel2012a}, $k(z,\tilde{z})=\sigma_f^2\exp\left[-\frac{(z-\tilde{z})^2}{2l^2}\right]$, where  $\sigma_f$ and $l$ are the hyperparameters need to optimize the values.

We use public python package GaPP\footnote{\url{https://github.com/astrobengaly/GaPP}} to calibrate the GRB relation from the SNe Ia. We use the J221 GRB data\footnote{We revisited the J221 sample \cite{Jia2022}, which consists of 49 GRBs from Fermi catalog, 33 GRBs from Ref.~\refcite{Amati2019} and 139 GRBs from Ref.~\refcite{Wang2016}.%, including 42 updated long GRBs, 69 GRBs from \cite{Amati2008}, 24 GRBs from \cite{Amati2009}, and another 4 GRBs are obtained from private communications.
}, and the Pantheon+ sample \cite{Scolnic2022} comprising 1701 light curves of 1550 unique spectroscopically confirmed SNe Ia\footnote{The Pantheon+ sample \cite{Scolnic2022} do not use SNe from SNLS at $z > 0.8$ due to sensitivity to the $U$ band in model training, so the Pantheon+ statistics between $0.8 < z < 1.0$ are lower than that of Pantheon \cite{Scolnic2018} and the Joint Light-curve Analysis (JLA \cite{Betoule2014}).}.
The distance modulus $\mu$ is related to the luminosity distance $d_L$: $\mu= m - M = 5\log \frac{d_L}{\textrm{Mpc}} + 25$, where $m$ and $M$ correspond to the apparent magnitude and absolute magnitude, respectively. The reconstructed apparent magnitudes from Pantheon+ sample are showed in Fig. \ref{fig/1.png}.
We find that the SNe Ia data points are sparse at $0.8\leq z\leq2.26$, the reconstruction function exhibits strange oscillations and large uncertainty.

To calibrate the Amati relation, we use GaPP to reconstruct the apparent magnitudes of SNe Ia data points  with 39 GRBs  at $z<0.8$ from the J221 sample.
In order to compare with the previous analyses\cite{Liu2022b,Liang2022,LZL2023}, we also used a subsample of SNe Ia with a redshift cutoff at $z=1.4$ to calibrate the Amati relation with 90 GRBs  at $z<1.4$ from the J221 sample. % using the luminosity distance reconstruction function of SNe Ia data at $z \le 2.26$.
% Because the Pantheon+ SNe Ia samples is relatively sparse in the redshift range of $0.8 < z < 1.0$, in \cite{Liang2022}, they  selected the redshift range of $z < 1.4$. Therefore, we  consider the redshift ranges of $z<0.8$, $z<1.0$ and $z<1.4$.
Two likelihood function methods \cite{D'Agostini2005,Reichart2001} %on the sample data of GRBs
are used to fit the parameters of Amati relation ($a$ and $b$). The likelihood function proposed by Ref.~\refcite{D'Agostini2005} is written as
\begin{eqnarray}\label{Lc}
    \mathcal{L}_{\rm D}\propto\prod_{i=1}^{N_1} \frac{1}{\sigma^2}
    \times\exp\left[-\frac{[y_i-y(x_i,z_i; a, b, M)]^2}{2\sigma^2}\right].
\end{eqnarray}
Here $\sigma=\sqrt{\sigma_{\rm int}^2+\sigma_{y,i}^2+b^2\sigma_{x,i}^2}$,  $\sigma_{\rm int}$ is the intrinsic scatter of GRBs, $\sigma_y=\frac{1}{\rm ln10}\frac{\sigma_{E_{\rm iso}}}{E_{\rm iso}},\quad \sigma_x=\frac{1}{\rm ln10}\frac{\sigma_{E_{\rm p}}}{E_{\rm p}}$, $\sigma_{E_{\rm p}}$ is the error magnitude of $E_{\rm p}$, and $\sigma_{E_{\rm iso}}=4\pi d^2_L\sigma_{S_{\rm bolo}}(1+z)^{-1}$ is the error magnitude of $E_{\rm iso}$, where $\sigma_{S_{\rm bolo}}$ is the error magnitude of $S_{\rm bolo}$. It should be noted that the use of the Ref.~\refcite{D'Agostini2005} likelihood  may introduce a subjective bias on the choice of the independent variable in the analysis. The likelihood function proposed by Ref.~\refcite{Reichart2001} has the advantage of not requiring the arbitrary choice of an independent variable among $E_{p}$ and $E_{{\rm iso}}$, which has been used to get rid of this bias \cite{Amati2013,LZL2023}.
The Ref.~\refcite{Reichart2001} likelihood function can be written as \cite{Lin2016,LZL2023}
\begin{eqnarray}\label{eqnarray6}
    \mathcal{L}_{\rm R}\propto\prod_{i=1}^{N_1} \frac{\sqrt{1+b^2}}{\sigma}
    \times\exp\left[-\frac{[y_i-y(x_i,z_i; a, b, M)]^2}{2\sigma^2}\right]
\end{eqnarray}
Here the intrinsic scatter can be calculated by $\sigma_{\rm int}=\sqrt{\sigma_{y,\rm int}^2 + b^2\sigma_{x,\rm int}^2}$, in which $\sigma_{x,\rm int}$ and $\sigma_{y,\rm int}$ are the intrinsic scatter along the $x$-axis and $y$-axis.

We used the python package \texttt{emcee}\cite{ForemanMackey2013} , which is optimized on the basis of the Metropolis-Hastings algorithm, to implement the MCMC numerical fitting method.
The parameters $a$, $b$, $\sigma_{\rm int}$, and the absolute magnitude $M$ of SNe Ia simultaneously using the MCMC method with the likelihood functions from the J221 sample with redshift  $z < 0.8$ (39 GRBs) and $z < 1.4$ (90 GRBs) are shown in Table \ref{ta1}. % The number of points used in each \textit{emcee} procedure is 5000.
%it can be observed that the absolute magnitude values obtained using the likelihood functions of \cite{D'Agostini2005} and \cite{Reichart2001} are nearly identical ($M=-19.50^{+1.40}_{-1.40}$). Furthermore,
We find that the fitting results of the intercept ($a$) with the two likelihood function methods \cite{D'Agostini2005,Reichart2001}  are consistent in 1$\sigma$ uncertainty; however,
there is a significant difference in the slope parameter $b$ with the two likelihood function methods \cite{D'Agostini2005,Reichart2001} . As pointed out in Ref.~\refcite{LZL2023}, this discrepancy arises because the likelihood employed by Ref.~\refcite{D'Agostini2005} may introduce subjective biases.% in the selection of independent variables.
To avoid any bias in the selection of independent variables, we utilize the calibration results obtained through the likelihood method proposed by Ref.~\refcite{Reichart2001} to construct the GRB Hubble diagrams.% at $z \geq 0.8$ and $z \geq 1.4$.

\begin{table}
\tbl{Calibration results (the intercept $a$, the slope $b$,  the intrinsic scatter $\sigma_{\rm int}$ and the absolute magnitude $M$) of the Amati relation in the J221 GRB sample at $z < 0.8$, $z < 1.4$ by the likelihood method Reichart 2001\cite{Reichart2001}  and the likelihood method D'Agostini 2005\cite{D'Agostini2005}.}
{\begin{tabular}{@{}cccccc@{}} \toprule
Methods & data sets &$a$& $b$& $\sigma_{{\rm int}}$& $M$ \\ \hline
\multirow{2}{*}{ D'Agostini 2005 } &39GRBs ($z < 0.8$) & $52.75^{+0.58}_{-0.58}$ & $1.50^{+0.13}_{-0.13}$ & $0.431$ & $-19.50^{+0.14}_{-0.14}$\\
\multirow{3}{*}{} & 90GRBs($z < 1.4$) & $52.83^{+0.58}_{-0.58}$ & $1.59^{+0.10}_{-0.10}$ & $0.433$ & $-19.50^{+0.14}_{-0.14}$\\
\hline
\multirow{2}{*}{ Reichart 2001} &39GRBs ($z < 0.8$) & $52.80^{+0.47}_{-0.87}$ & $1.808^{+0.094}_{-0.12}$ & $0.413$ & $-19.40^{+0.14}_{-0.14}$\\
\multirow{3}{*}{} & 90GRBs($z < 1.4$)  & $52.87^{+0.58}_{-0.58}$ & $2.026^{+0.083}_{-0.093}$ & $0.423$ & $-19.50^{+0.14}_{-0.14}$\\
\botrule
\end{tabular} \label{ta1}}
\end{table}



\section{THE HUBBLE DIAGRAM AND CONSTRAINS ON COSMOLOGICAL MODELS}
 % Figure environment removed
We construct the Hubble diagram by extrapolating the calibration results of the Amati relation at low-redshift GRBs to  high-redshift. The Hubble diagram with J221 sample are plotted in Fig. \ref{2.png}. The cosmological parameters can be fitted by minimizing the $\chi^2$ statistic.\begin{equation}\chi^2_{\rm GRB} = \sum^{N_1}_{i=1} \left[\frac{\mu_{\rm obs}(z_i)-\mu_{\rm th}(z_i;p,H_0)}{\sigma_{\mu_i}}\right]^2.
\end{equation}
Here, $N_1$ = 182 or 131 represents the number of high-redshift GRBs with $z \geq 0.8$ or $z \geq 1.4$, respectively, in the J221 sample, $\mu_{{\rm obs}}$ is the observational value of distance modulus and its error $\sigma_{\mu_i}$. The uncertainty of GRB distance modulus with the Amati relation is
\begin{equation}\sigma^2_\mu=\bigg(\frac{5}{2}\sigma_{{\rm log}\frac{E_{\rm iso}}{\rm 1erg}}\bigg)^2+ \bigg (\frac{5}{\rm 2ln10}\frac{\sigma_{S_{\rm bolo}}}{S_{\rm bolo}} \bigg)^2 \ ,\end{equation}
where
\begin{equation}\sigma^2_{{\rm log}\frac{E_{\rm iso}}{\rm 1erg}}=\sigma^2_{\rm int}+ \bigg (\frac{b}{\rm ln10}\frac{\sigma_{E_{\rm p}}}{E_{\rm p}} \bigg )^2+\sum \bigg (\frac{\partial_{y}(x;\theta_c)}{\partial \theta_i} \bigg)^2C_{ii}\, .
\end{equation}
Here $\theta_c$ = \{$\sigma_{{\rm int}}$, $a, b$\}, and $C_{ii}$ means the diagonal element of the covariance matrix of these fitting coefficients. $\mu_{{\rm th}}$ is the theoretical value of distance modulus calculated from the cosmological model,  $H_0$ is the Hubble constant,  $p$ represents the cosmological parameters.
Considering a flat space,
for the $w$CDM model which has a constant equation of state of dark energy, the theoretical value of the luminosity distance can be obtained from
\begin{equation}d_{L;{\rm th}}=\frac{{c}(1+z)}{H_{\rm 0}}\int^z_0\frac{dz}{[\Omega_{\rm m}(1+z)^3+\Omega_{\Lambda}(1+z)^{3(1+w)}]^{\frac{1}{2}}} . \end{equation}
Here   $c$ is the speed of light, and $\Omega_{\rm m}$ and $\Omega_{\Lambda}$ are the present dimensionless density parameters of matter and dark energy, respectively, which satisfy $\Omega_{\rm m}$ + $\Omega_{\Lambda}$ = 1.
For the flat $\Lambda$CDM model, $w=-1$.

We employ the Python package \texttt{emcee} \cite{ForemanMackey2013} to constrain cosmological models with  the GRB data at high-redshift. %In each \textit{emcee} iteration, we generate 5000 datasets.
The results of 182 and 131 GRBs in the J221 data set at $z \geq 0.8$ and $z \geq 1.4$ are shown in Figures \ref{constrain1} ($\Lambda$CDM model) and Figure \ref{constrain2} ($w$CDM model), respectively. Constraint results with 1$\sigma$ confidence level are summarized in Tab. \ref{tab2}. \footnote{It should be noted that GRB data
alone are unable to constrain $H_0$ because of the degeneracy between $H_0$ and the correlation intercept
parameter; therefore $H_0$ is set to be $70\ {\rm km}\ {\rm s}^{-1}{\rm Mpc}^{-1}$ for GRB-only analyses in previous works \cite{Khadka2021,Cao2022a,Liang2022,LZL2023}.} %In order to compare with the previous analyses, we also set $H_0$=$70\ {\rm km}\ {\rm s}^{-1}{\rm Mpc}^{-1}$ for the cases only with GRBs.}

% Figure environment removed

% Figure environment removed


In our analysis, we also use the latest OHD in Ref.~\refcite{LZL2023} to constrain cosmological models, %The OHD  can be obtained from the galactic age differential method \citep{Jimenez2002}, which have advantages to constrain cosmological parameters and distinguish dark energy models.
including the 31 Hubble parameter measurements at $0.07<z<1.965$ \cite{Stern2010,Moresco2012,Moresco2015,Moresco2016,Zhang2014,Ratsimbazafy2017}, and a new point at $z=0.80$ proposed by Jiao \emph{et al.} \cite{Jiao2023} in a similar approach. %In this work, we also use the 31 OHD and the point Ref.~\refcite{Jiao2023}.
It should be noted that Borghi \emph{et al.}\cite{Borghi2022} obtained  another new OHD at $z=0.75$. Considering these two measurements \cite{Jiao2023,Borghi2022} are
not fully independent and their covariance is not clear, we only use the point in Ref.~\refcite{Jiao2023},  which taking advantage of the $~1/\sqrt 2$ fraction of systematic uncertainty. One could either use the data in Ref.~\refcite{Borghi2022} alternatively  with other 31 OHD to investigate cosmology \cite{Kumar2023,Muccino2023,Lee2023,Favale2023}.
For the OHD data set, the $\chi^2$ has the form
\begin{equation}\chi^2_{{\rm {OHD}}} =
\sum^{N_3}_{i=1} \left [\frac{H_{\rm  obs}(z_i)-H_{\rm th}(z_i;p,H_0)}{\sigma_{H_i}} \right]^2 . \end{equation}
Here $N_3=32$  denotes  the number of the Hubble parameter measurements. The total $\chi^2$ of GRB and OHD data is
\begin{equation}\chi^2_{{\rm total}} = \chi^2_{{\rm GRB}} + \chi^2_{{\rm OHD}}.
\end{equation}

The constraint results of the high-redshift GRBs (182 GRBs at $z\geq 0.8$, and 131 GRBs at $z\geq 1.4$) from the J221 data set and 32 OHD, are plotted in Figures \ref{constrain3} ($\Lambda$CDM model) and Figure \ref{constrain4} ($w$CDM model), and summarized in Tab. \ref{tab2} with the 1$\sigma$ confidence level.
\textbf{With 182 GRBs at $0.8\le z\le8.2$ in the J221 sample, we obtained $\Omega_{\rm m}$ = $0.373^{+0.047}_{-0.058}$ ($\Lambda$CDM) and $\Omega_{\rm m}=0.316^{+0.19}_{-0.094}, w=-1.00^{+0.65}_{-0.28}$ ($w$CDM), which are consistent with Ref.~\refcite{Dainotti2023MN518} using the 3D GRB relation alone calibrated on SNe Ia ($\Omega_{\rm m}$ = $0.306\pm0.069$ for $\Lambda$CDM, fixing $h=70$; and $w$ = $-0.906\pm0.697$ for $w$CDM, fixing $\Omega_{\rm m}=0.3$, $h=70$). Our results are  more stringent than previous results in Ref.~\refcite{Cao2022MN516} with  the Platinum GRB and the LGRB95 sample for $\Lambda$CDM and $w$CDM model.}
With 182 GRBs at $0.8\le z\le8.2$ in the J221 sample and 32 OHD, we obtained $\Omega_{\rm m}=0.348^{+0.048}_{-0.066}$ and $h=0.680^{+0.029}_{-0.029}$  for the flat $\Lambda$CDM model, and  $\Omega_{\rm m}$ = $0.318^{+0.067}_{-0.059}$, $h$ = $0.704^{+0.055}_{-0.068}$, $w$ = $-1.21^{+0.32}_{-0.67}$ for the flat $w$CDM model, which are consistent with the results \textbf{using the  193 GRBs (Amati relation) and SNe Ia ($\Omega_{\rm m}=0.397\pm0.040$  for the $\Lambda$CDM model, and $\Omega_{\rm m}=0.34^{+0.13}_{-0.15}, w=-0.86^{+0.36}_{-0.38}$ for the $w$CDM model;  fixing $h=0.6774$) at the 2$\sigma$ level }\cite{Amati2019}; \textbf{and the result in Ref.~\refcite{Dainotti2022MN514} combining SNe Ia and GRBs with a 3D optical Dainotti correlation for a flat $\Lambda$CDM cosmology ($\Omega_{\rm m}=0.299\pm0.009)$}.
%These results are consistent with  \textbf{the results from GRBs+BAOs for a flat $\Lambda$CDM cosmology without correcting for the evolutionary effects for GRBs ($\Omega_{\rm m}=0.289\pm0.015$ and $h=0.674\pm0.011$)\cite{Dainotti2022PASJ74}; and the results from SNe Ia+BAO+GRBs using GRBs without calibration on SNe Ia and with uniform priors without the correction for the evolution ($w$ = $-1.017\pm0.014$ for $w$CDM, fixing $\Omega_{\rm m}=0.3$, $h=70$)\cite{Dainotti2023MN518}.}
With 131 GRBs at $1.4\le z\le8.2$ in the J221 sample and 32 OHD, we obtained $\Omega_{\rm m}$ = $0.314^{+0.046}_{-0.063}$ and $h$ = $0.681^{+0.029}_{-0.029}$  for the flat $\Lambda$CDM model, and  $\Omega_{\rm m}$ = $0.269^{+0.10}_{-0.055}$, $h$ = $0.683^{+0.042}_{-0.072}$, $w$ = $-1.00^{+0.63}_{-0.29}$ for the flat $w$CDM model at the 1$\sigma$ confidence level,
which are consistent with \textbf{our previous analyses with 98 GRBs at $1.4<z\leq8.2$ in the A118 sample and OHD ($\Omega_{\rm m}$=$0.346^{+0.048}_{-0.069}$, $h$=$0.677^{+0.029}_{-0.029}$ for the flat $\Lambda$CDM model, and  $\Omega_{\rm m}$=$0.314^{+0.072}_{-0.055}$, $h$=$0.705^{+0.055}_{-0.069}$, $w$=$-1.23^{+0.33}_{-0.64}$ for the flat $w$CDM model)\cite{Liang2022}}.

\begin{table}
\tbl{Constraints on the $\Lambda$CDM and $w$CDM Models at the 1$\sigma$ Confidence
Level from J221 GRBs at high redshift $z \geq 0.8$(182 GRBs), and $z \geq 1.4$(131 GRBs) with 32 OHD Data Sets. (For the cases only with GRBs, h is set to be 0.7.) }
{\begin{tabular}{@{}ccccc@{}} \toprule
Models & data sets &$\Omega_{\rm{m}}$& $h$& $w$\\ \hline
\multirow{5}{*}{$\Lambda$CDM}&182 GRBs ($z\geq0.8$)  &  $0.374^{+0.047}_{-0.058}$  & $-$& $-$\\
\multirow{6}{*}{}&131 GRBs ($z\geq1.4$)  & $0.373^{+0.047}_{-0.058}$  & $-$& $-$\\
\multirow{6}{*}{}&182 GRBs+OHD  & $0.348^{+0.048}_{-0.066}$  & $0.680^{+0.029}_{-0.029}$& $-$\\
\multirow{6}{*}{}&131 GRBs+OHD  & $0.314^{+0.046}_{-0.063}$  & $0.681^{+0.029}_{-0.029}$& $-$\\
\hline
\multirow{5}{*}{$w$CDM}&182 GRBs ($z\geq0.8$) & $0.316^{+0.19}_{-0.094}$& $-$&$-1.00^{+0.65}_{-0.28}$\\
\multirow{6}{*}{}&131 GRBs ($z\geq1.4$)  & $0.21^{+0.14}_{-0.11}$ & $-$ & $-0.98^{+0.54}_{-0.20}$\\
\multirow{6}{*}{}&182 GRBs +OHD & $0.318^{+0.067}_{-0.059}$ & $0.704^{+0.055}_{-0.068}$ & $-1.21^{+0.32}_{-0.67}$\\
\multirow{6}{*}{}&131 GRBs +OHD & $0.269^{+0.10}_{-0.055}$ & $0.683^{+0.042}_{-0.072}$ & $-1.00^{+0.63}_{-0.29}$\\
\botrule
\end{tabular} \label{tab2}}
\end{table}



Finally, we also use the J221 data set to constrain the $\Lambda$CDM and $w$CDM models by using the method of simultaneous fitting, in which the parameters of cosmological models ($\Omega_{\rm m}$, $h$, and $w$) and the relation parameters ($a$ and $b$) are fitted simultaneously.
%The number of points that have been used in each \textit{emcee}  procedure is 4000.
The results from the J221 sample combined with the OHD data set are shown in Fig. \ref{constrain5}, and summarized in Table \ref{tab3} with the 1$\sigma$ confidence level.
%\textbf{With the J221 sample and 32 OHD, we obtained $\Omega_{\rm m}=0.395^{+0.054}_{-0.078}$, $h=0.651^{+0.030}_{-0.030}$, $a=52.869^{+0.035}_{-0.035}$, $b=1.453^{+0.064}_{-0.064}$  for the flat $\Lambda$CDM model, and  $\Omega_{\rm m}=0.350^{+0.11}_{-0.069}$, $h=0.648^{+0.037}_{-0.061}$, $w=-0.97^{+0.63}_{-0.30}$, $a=52.871^{+0.036}_{-0.036}$, $b=1.450^{+0.064}_{-0.064}$  for the flat $w$CDM model.}
It is found that the values of the coefficients of the Amati relation ($a$, $b$, $\sigma_{{\rm int}}$) for the flat $\Lambda$CDM model and  the flat $w$CDM model in simultaneous fitting are almost identical, which are consistent with the results calibrating from the low-redshift data.
%\textbf{The values of  the uncertainty of the relation parameters ($a$, $b$) and the intrinsic scatter $\sigma_{\rm int}$ in simultaneous fitting are smaller than those listed in Table 1, which is attributed to  the number of calibrated GRBs in the J221 data set, these are apparently larger than the 39 GRBs at $z<0.8$ and the 90 GRBs at $z<1.4$ in the J221 sample.}
\textbf{Compared to the results  of Ref.~\refcite{Dainotti2022PASJ74} from GRBs+BAOs with EV ($\Omega_{\rm m}=0.286\pm0.015$, $H_0=67.219\pm1.050\ {\rm km}\ {\rm s}^{-1}{\rm Mpc}^{-1}$), Ref.~\refcite{Dainotti2023MN518} from SNe Ia+BAO+GRBs using GRBs with the correction for the evolution indicated with EV ($\Omega_{\rm m}=0.310\pm0.007$, $H_0=67.83\pm0.16\ {\rm km}\ {\rm s}^{-1}{\rm Mpc}^{-1}$; and $w$ = $-1.017\pm0.015$ for $w$CDM, fixing $\Omega_{\rm m}=0.3$, $h=0.70$), and Ref.~\refcite{Cao2022MN512} from GRBs+BAOs ($\Omega_{\rm m}=0.299^{+0.016}_{-0.018}$, $H_0=69.4\pm1.81\ {\rm km}\ {\rm s}^{-1}{\rm Mpc}^{-1}$),
%for the values of the cosmological parameters in a flat $\Lambda$CDM model,
we  find the result of $h$ with J221+OHD  for a flat $\Lambda$CDM cosmology are consistent with Ref.~\refcite{Dainotti2022PASJ74,Dainotti2023MN518} and Ref.~\refcite{Cao2022MN512} at the 1$\sigma$ confidence level; and the result of $\Omega_{\rm m}$ is slightly different with  Ref.~\refcite{Dainotti2022PASJ74,Dainotti2023MN518} and Ref.~\refcite{Cao2022MN512} at the 1$\sigma$ confidence level. We also find that the value of $w$ for a flat $w$CDM model with  Jia221+OHD  are consistent with Ref.~\refcite{Dainotti2023MN518} at the 1$\sigma$ confidence level.}
\textbf{Following the same approach as in Ref.~\refcite{Dainotti2022MN514,Dainotti2022PASJ74,Dainotti2023ApJ951,Bargiacchi2023MN521}, we also consider the selection biases and redshift evolution for the J221 sample, and find that the fitting results of the cosmological parameters with and without correcting for the evolutionary effects for GRBs are almost identical.}





% Figure environment removed
% Figure environment removed




\begin{table}
\tbl{Simultaneous Fitting Results of $\Omega_{\rm m}$, $h$, $a$, $b$ and $\sigma_{\rm int}$ in the $\Lambda$CDM and $w$CDM Models, with  J221 GRB + 32 OHD Data Sets.}
{\begin{tabular}{@{}cccccccc@{}} \toprule
Models& Data Sets& $\Omega_{m}$& $h$& $w$& $a$& $b$& $\sigma_{\rm int}$\\ \hline
\multirow{2}{*}{$\Lambda$CDM}&\multirow{2}{*}{J221 GRB + 32 OHD}&\multirow{2}{*}{$0.395^{+0.054}_{-0.078}$}&\multirow{2}{*}{$0.651^{+0.030}_{-0.030}$}&\multirow{2}{*}{-}&\multirow{2}{*}{$52.869^{+0.035}_{-0.035}$}&\multirow{2}{*}{$1.453^{+0.064}_{-0.064}$}&\multirow{2}{*}{$0.393^{+0.019}_{-0.022}$}\\

\multirow{4}{*}{$w$CDM}&\multirow{4}{*}{J221 GRB + 32 OHD}&\multirow{4}{*}{$0.350^{+0.11}_{-0.069}$}&\multirow{4}{*}{$0.648^{+0.037}_{-0.061}$}&\multirow{4}{*}{$-0.97^{+0.63}_{-0.30}$}&\multirow{4}{*}{$52.871^{+0.036}_{-0.036}$}&\multirow{4}{*}{$1.450^{+0.064}_{-0.064}$}&\multirow{4}{*}{$0.393^{+0.020}_{-0.023}$}\\
\hline
\\
\\
\toprule
\end{tabular} \label{tab3}}
\end{table}

% Figure environment removed


\section{SUMMARY AND DISCUSSION}

In this paper, we use the Gaussian process to calibrate the Amati relation of GRBs %with $z<0.8$ and $z<1.4$
from the Pantheon+ sample \cite{Scolnic2022} by Gaussian process and obtain the GRB Hubble diagram with the latest J221 GRB sample \cite{Jia2022}. %Then, these GRB data are used to constrain the $\Lambda$CDM and $w$CDM models.
With 131 GRBs at $1.4\le z\le8.2$ in the J221 sample and 32 OHD, we obtained $\Omega_{\rm m}$ = $0.314^{+0.046}_{-0.063}$ and $h$ = $0.681^{+0.029}_{-0.029}$  for the flat $\Lambda$CDM model, and  $\Omega_{\rm m}$ = $0.269^{+0.10}_{-0.055}$, $h$ = $0.683^{+0.042}_{-0.072}$, $w$ = $-1.00^{+0.63}_{-0.29}$ for the flat $w$CDM model at the 1$\sigma$ confidence level.
We find that our results with 131 GRBs at $1.4\le z\le8.2$ are consistent with previous analyses that obtained in Ref.~\refcite{Liang2022}.
We also use GRB data sets of J221 sample to fit $\Omega_{\rm m}$, $h$, $a$, $b$, $\sigma_{\rm int}$ and $w$ parameters simultaneously. It is found that the simultaneous fitting results are consistent with those obtained from the low-redshift calibration method.


$H_0$ with a redshift evolving is an interesting idea for the $H_0$ tension~\cite{Dainotti2021ApJ912,Dainotti2022Galaxies,Colgain2022,Jia2023,Hu2023,Malekjani2023}.
Recently, \textbf{Dainotti \emph{et al.} \cite{Dainotti2021ApJ912,Dainotti2022Galaxies} fit the $H_0$ values with a function mimicking the redshift evolution to find a slowly decreasing trend.}
Jia \emph{et al.} \cite{Jia2023} find a decreasing trend in the Hubble constant with a significance of a $5.6\sigma$ confidence level with SN Ia, OHD  and baryon acoustic oscillation (BAO) data, which indicate that $H_0$ value is consistent with that measured from the local data at low redshift  and drops to the value measured from the CMB at high redshift.
Malekjani \emph{et al.} \cite{Malekjani2023} find the evolving ($H_0$,$\Omega_{\rm m}$) values above $z = 0.7$ in Pantheon+ sample.
We find that the $H_0$  value with GRBs at $0.8\le z\le8.2$ and OHD at $z\le1.975$  seems to favor the one from the Planck observations, and the $\Omega_{\rm m}$ value of our results for the flat $\Lambda$CDM model is consistent with the CMB observations at the 1$\sigma$ confidence level. A larger $\Omega_{\rm m}$ values in the $\Lambda$CDM model with GRBs at high redshift is obtained, but adding OHD at low redshit removes this trend.






\textbf{It should be note that the potential use of machine learning (ML) algorithms for reconstructing light curves could further enhance parameter determination\cite{Dainotti2023ApJS267}.
Dainotti \emph{et al.} \cite{Dainotti1907} use ML to infer redshifts from the observed temporal and spectral features of GRBs.}
\textbf{Moreover, ML have been use to calibrate the Amati relation \cite{Luongo2021,Zhang2312}. Furthermore, Dainotti \emph{et al.} \cite{Dainotti2022MN514,Dainotti2022Galaxies} investigate  perspective of the future contribution of GRB-Cosmology.}
In future, GRBs could be used to set tighter constraints on cosmological models with the GRB sample from Fermi data with much smaller scatters, as well as the data from the Chinese-French mission SVOM (the Space-based multiband astronomical Variable Objects Monitor)\cite{Bernardini2021}, which will provide a substantial enhancement of the number of GRBs  with measured redshift and spectral parameters.




\section*{ACKNOWLEDGMENTS}
We thank Guangzhen Wang, Zhiguo Xiong and Prof. Puxun Wu %,  Prof. Xiaoyao Xie, Prof.  Jianchao Feng, Prof.  Junjin Peng
for for discussions and kind help.
We also thank the referee for helpful comments and constructive suggestions.
This project was supported by the Guizhou Provincail Science and Technology Foundation (QKHJC-ZK[2021] Key 020).


%\section*{DATA AVAILABILITY}
%Data are available at the following references:
%the latest OHD obtained with the CC method from Ref.~\refcite{LZL2023}, the  GRB data set from Ref.~\refcite{Jia2022} and the Pantheon+ SNe sample from Ref.~\refcite{Scolnic2022}.
%The data underlying this article will be shared on reasonable request to the corresponding author.


%\begin{thebibliography}{000} %for 3 digits
%\begin{thebibliography}{00}  %for 2 digits
\begin{thebibliography}{200}    %for 1 digit
\bibitem{Cucchiara2011} Cucchiara, A., Levan, A., Fox, D. B., \textit{et al.}  \textit{Astrophys. J.} {\bf 736} (2011) 7.
\bibitem{Scolnic2018} Scolnic, D. M., Jones, D. O., Rest, A., \textit{et al.}  \textit{Astrophys. J.} {\bf 859} (2018) 101.
\bibitem{Fenimore2000} Fenimore, E. E., \& Ramirez-Ruiz, E., {\it preprint astro-ph/0004176 } (2000).
\bibitem{Norris2000} Norris, J. P., Marani, G. F., \& Bonnell, J. T., \textit{Astrophys. J.} {\bf 534} (2000) 248.
\bibitem{Amati2002} Amati, L., Frontera, F., Tavani, M., {\it et al}., \textit{Astron. Astrophys.} {\bf 390} (2000) 81.
\bibitem{Ghirlanda2004a} Ghirlanda, G., Ghisellini, G., \& Lazzati, D., \textit{Astrophys. J.} {\bf 616} (2004) 331.
\bibitem{Yonetoku2004} Yonetoku, D., Murakami, T., Nakamura, T., {\it et al}., \textit{Astrophys. J.} {\bf 609} (2004) 935.
\bibitem{Liang2005} Liang, E., \& Zhang, B., \textit{Astrophys. J.} {\bf 633} (2005) 611.
\bibitem{Firmani2006} Firmani, C., Ghisellini, G., Avila-Reese, V., \& Ghirlanda, G., \textit{Mon. Not. R. Astron. Soc.} {\bf 370} (2006) 185.
\bibitem{Ghirlanda2006} Ghirlanda, G., Ghisellini, G.,\& Firmani, C., \textit{New J. Phys} {\bf 8} (2006) 123.
\bibitem{Schaefer2007} Schaefer, B. E., \textit{Astrophys. J.} {\bf 660} (2007) 16.
\bibitem{Tsutsui2009a} Tsutsui, R., Nakamura, T., Yonetoku, D.,  Murakami, T., Kodama, Y., \& Takahashi, K., \textit{J. Cosmol. Astropart. Phys.} {\bf 0908} (2009) 015.
\bibitem{Izzo2015} Izzo, L., Muccino, M., Zaninoni, E., Amati, L., \& Della Valle, M., \textit{Astron. Astrophys.} {\bf 582} (2015) A115.
\bibitem{Dai2004} Dai, Z., Liang, E., \& Xu, D., \textit{Astrophys. J.} {\bf 612} (2004) L101.
\bibitem{Liang2008} Liang, N., Xiao, W. K., Liu, Y., \& Zhang, S. N., \textit{Astrophys. J.} {\bf 685} (2008) 354.
\bibitem{Capozziello2008} Capozziello, S., \& Izzo, L., \textit{Astron. Astrophys.} {\bf 490} (2008) 31.
\bibitem{Capozziello2009} Capozziello, S. \& Izzo, L., \textit{Nucl. Phys. Proc. Suppl.} {\bf 194} (2009) 206.
\bibitem{Wei2009} Wei, H., Zhang, S. N., \textit{Eur. Phys. J.} {\bf 63} (2009) 139.
\bibitem{Wei2010} Wei, H., \textit{J. Cosmol. Astropart. Phys.} {\bf 08} (2010) 020.
\bibitem{Liang2010} Liang, N., Wu, P.,  \& Zhang, S. N., \textit{Phys. Rev. D} {\bf 81} (2010) 083518.
\bibitem{Liang2011}  Liang, N., Xu, L.,  \& Zhu, Z. H., \textit{Astron. Astrophys.} {\bf 527} (2011) A11.
\bibitem{Wang2016}  Wang, J. S., Wang, F. Y., Cheng, K. S., \& Dai, Z. G., \textit{Astron. Astrophys.} {\bf 585} (2016) A68.
\bibitem{Liu2022b} Liu, Y., Liang, N., Xie, X., {\it et al}., \textit{Astrophys. J.} {\bf 935} (2022) 7.
\bibitem{LZ2008} Liang, N., \& Zhang, S., {\it AIP Conf. Proc. } Vol. 1065 (2008).
\bibitem{Kodama2008} Kodama, Y., Yonetoku, D., Murakami, T., {\it et al}., \textit{Mon. Not. R. Astron. Soc.} {\bf 391} (2008) L1.
\bibitem{Capozziello2010} Capozziello, S., \& Izzo, L., \textit{Astron. Astrophys.} {\bf 519} (2010) A73.
\bibitem{Gao2012} Gao, H., Liang, N., \& Zhu, Z.-H., \textit{Int. J. Mod. Phys. D.} {\bf 21} (2011) 1250016.
\bibitem{Liu2015} Liu, J., \& Wei, H., \textit{Gen. Rel. Grav.} {\bf 47} (2015) 141.
\bibitem{Demianski2017a} Demianski, M., Piedipalumbo, E., Sawant, D., \& Amati, L., \textit{Astron. Astrophys.} {\bf 598} (2017) A112.
\bibitem{Demianski2021} Demianski, M., Piedipalumbo, E., Sawant, D., \& Amati, L., \textit{Mon. Not. R. Astron. Soc.} {\bf 506} (2021) 903.
\bibitem{Shirokov2020} Shirokov, S. I., Sokolov, I. V., Lovyagin, N. Y., {\it et al}., \textit{Mon. Not. R. Astron. Soc.} {\bf 496} (2020) 1530.
\bibitem{Muccino2021} Muccino, M., Izzo, L., Luongo, O., {\it et al}., \textit{Astrophys. J.} {\bf 908} (2021) 181.
\bibitem{Lovyagin2022} Lovyagin, N. Y., Gainutdinov, R. I.,  Shirokov, S. I., {\it et al}., {\it Universe } {\bf 08} (2022) 334.
\bibitem{Tang2022} Tang L., Lin, H.-N., Li X., \& Liu L., \textit{Mon. Not. R. Astron. Soc.} {\bf 509} (2022) 1194.
\bibitem{Liang2022} Liang, N., Li, Z., Xie, X., \& Wu, P., \textit{Astrophys. J.} {\bf 941} (2022) 84.
\bibitem{Jimenez2002} Jimenez, R., \& Loeb, A., \textit{Astrophys. J.} {\bf 573} (2002) 37.
\bibitem{Jimenez2003} Jimenez, R., Verde, L., Treu, T. \& Stern, D., \textit{Astrophys. J.} {\bf 593} (2003) 622.
\bibitem{Amati2019} Amati, L., D'Agostino, R., Luongo, O., Muccino, M., \& Tantalo, M., \textit{Mon. Not. R. Astron. Soc.} {\bf 486} (2019) L46.
\bibitem{Montiel2021} Montiel, A., Cabrera, J. I., \&  Hidalgo, J. C., \textit{Mon. Not. R. Astron. Soc.} {\bf 501} (2021) 3515.
\bibitem{Luongo2021} Luongo, O., \& Muccino, M., \textit{Mon. Not. R. Astron. Soc.} {\bf 503} (2021) 4581.
\bibitem{Luongo2023} Luongo, O., \& Muccino, M., \textit{Mon. Not. R. Astron. Soc.} {\bf 518} (2023) 2247.
\bibitem{Muccino2023} Muccino, M., Luongo, O., \&  Jain, D., \textit{Mon. Not. R. Astron. Soc.} {\bf 523} (2023) 4938.
\bibitem{Amati2008} Amati, L., Guidorzi, C., Frontera, F., {\it et al}., \textit{Mon. Not. R. Astron. Soc.} {\bf 391} (2008) 577.
\bibitem{Khadka2021} Khadka, N., Luongo, O., Muccino, M., \& Ratra, B., \textit{J. Cosmol. Astropart. Phys.} {\bf 091} (2021) 042.
\bibitem{Khadka2020} Khadka, N. \& Ratra, B., \textit{Mon. Not. R. Astron. Soc.} {\bf 499} (2020) 391.
\bibitem{Dirirsa2019} Fana Dirirsa,  F., Razzaque, S., Piron, F., {\it et al}., \textit{Astrophys. J.} {\bf 887} (2019) 19.
\bibitem{Cao2022MN510} Cao, S., Khadka, N., \& Ratra, B., \textit{Mon. Not. R. Astron. Soc.} {\bf 510} (2022) 2928.
\bibitem{Cao2022MN512} Cao, S., Dainotti, M., \& Ratra, B., \textit{Mon. Not. R. Astron. Soc.} {\bf 512} (2022) 439.
\textbf{\bibitem{Cao2022MN516} Cao, S., Dainotti, M., \& Ratra, B., \textit{Mon. Not. R. Astron. Soc.} {\bf 516} (2022) 1386.
\bibitem{Dainotti2008} Dainotti, M. G., Cardone V. F., \& Capozziello S., \textit{Mon. Not. R. Astron. Soc.} {\bf 391} (2008) L79.
\bibitem{Cardone2009} Cardone, V. F., Capozziello, S., \&  Dainotti, M. G., \textit{Mon. Not. R. Astron. Soc.} {\bf 400} (2009) 775.
\textbf{\bibitem{Cardone2010} Cardone V. F., Dainotti, M. G., Capozziello, S., \& Willingale, R. \textit{Mon. Not. R. Astron. Soc.} {\bf 408} (2010) 1181.
\bibitem{Dainotti2010} Dainotti, M. G., Cardone V. F., \& Capozziello S., \textit{Astrophys. J.} {\bf 722} (2010) L215.
\bibitem{Dainotti2011a} Dainotti, M. G.,  Ostrowski, M., \& Willingale, R. \textit{Mon. Not. R. Astron. Soc.} {\bf 418} (2011) 2202.
\bibitem{Dainotti2011b} Dainotti, M. G.,  Cardone, V. F., Capozziello, S., {\it et al.} \textit{Astrophys. J.} {\bf 730} (2011) 135.
\bibitem{Dainotti2013a} Dainotti, M. G.,  Petrosian, V., Singal, J., {\it et al.} \textit{Astrophys. J.} {\bf 774} (2013) 157.
\bibitem{Dainotti2013b} Dainotti, M. G.,  Cardone, V. F., Piedipalumbo, E., {\it et al.} \textit{Mon. Not. R. Astron. Soc.} {\bf 436} (2013) 82.
\bibitem{Dainotti2015a} Dainotti, M. G., Petrosian, V., Willingale, R., {\it et al.} \textit{Mon. Not. R. Astron. Soc.} {\bf 451} (2015) 3898.
\bibitem{Dainotti2015b} Dainotti, M. G., Del Vecchio, R., Shigehiro, N., {\it et al.} \textit{Astrophys. J.} {\bf 800} (2015) 31.
\bibitem{Dainotti2016}  Dainotti, M. G., Postnikov, S., Hernandez, X., {\it et al.} \textit{Astrophys. J.} {\bf 825} (2016) L20
\bibitem{DelVecchio2016} Del Vecchio, R., Dainotti, M. G., \& Ostrowski, M. {\it et al.} \textit{Astrophys. J.} {\bf 828} (2016) 36.
\bibitem{Dainotti2017a} Dainotti, M. G., Nagataki, S., Maeda, K., {\it et al.} \textit{Astron. Astrophys.} {\bf 600} (2017) 98.
\bibitem{Dainotti2017b} Dainotti, M. G., Hernandez, X., Postnikov, S.,{\it et al.} \textit{Astrophys. J.} {\bf 848} (2017) 88.
\bibitem{Srinivasaragavan2020}  Srinivasaragavan, G. P., Dainotti, M. G.,  Fraija,  N. Fraija {\it et al.} \textit{Astrophys. J.} {\bf 903} (2020) 18.
\bibitem{Dainotti2020a} Dainotti, M. G., Lenart, A. L., Sarracino, G., {\it et al.} \textit{Astrophys. J.} {\bf 904} (2020) 97.
\bibitem{Dainotti2020b}  Dainotti, M. G., Livermore, S., Kann, D. A., {\it et al.} \textit{Astrophys. J.} {\bf 905} (2020) L26.
\bibitem{Dainotti2021a}  Dainotti, M. G., Lenart, A. L., Fraija, N., {\it et al.} \textit{PASJ} {\bf 73} (2021) 970.
\bibitem{Levine2021} Levine, D., Dainotti, M., Zvonarek, K. J., {\it et al.} \textit{Astrophys. J.} {\bf 925} (2021) 15.
\bibitem{Tian2023} Tian, X., Li, J.-L., Yi, S. X., {\it et al.} \textit{Astrophys. J.} {\bf 958} (2023) 74.
}
\bibitem{Hu2021} Hu, J. P., Wang, F. Y., \& Dai, Z. G., \textit{Mon. Not. R. Astron. Soc.} {\bf 507} (2021) 730.
\bibitem{Wang2022}  Wang, F. Y., Hu, J. P., Zhang, G. Q., \& Dai, Z. G., \textit{Astrophys. J.} {\bf 924} (2022) 97.
\bibitem{Li2023} Li, J.-L., Yang, Y.-P., Yi, S.-X., Hu, J.-P., Wang, F.-Y., \& Qu, Y.-K., \textit{Astrophys. J.} {\bf 953}
    (2023) 58.}
\bibitem{Dainotti2022ApJS261} Dainotti, M. G., Young, S.,  Li, L., {\it et al}., \textit{Astrophys. J. Suppl.} {\bf 261} (2022) 25.
\textbf{\bibitem{Dainotti2022MN514} Dainotti, M. G., Nielson, V., Sarracino, G., {\it et al}., \textit{Mon. Not. R. Astron. Soc.} {\bf 514} (2022) 1828.
\bibitem{Dainotti2022PASJ74} Dainotti, M. G., Sarracino G., \&  Capozziello S., \textit{PASJ} {\bf 74} (2022) 1095.
\bibitem{Dainotti2023MN518} Dainotti, M. G., Lenart, A. L., Chraya, A., {\it et al}., \textit{Mon. Not. R. Astron. Soc.} {\bf 518} (2023) 2201.
\bibitem{Dainotti2023ApJ951} Dainotti, M. G., Bargiacchi, G., \&  Bogdan, M. {\it et al.}, \textit{Astrophys. J.} {\bf 951} (2023) 63.
\bibitem{Bargiacchi2023MN521} Bargiacchi, G., Dainotti, M. G., Nagataki, S.,{\it et al.} \textit{Mon. Not. R. Astron. Soc.} {\bf 521} (2023) 3909.
\bibitem{Bargiacchi2023MN525} Bargiacchi, G., Dainotti, M. G., \&  Capozziello, S. \textit{Mon. Not. R. Astron. Soc.} {\bf 525}
    (2023) 3104.}
\bibitem{Basilakos2008}Basilakos, S., \& Perivolaropoulos., L. \textit{Mon. Not. R. Astron. Soc.} {\bf 391} (2008) 411.
\bibitem{Lin2016} Lin, H. N., Li, X.  \& Chang, Z., \textit{Mon. Not. R. Astron. Soc.} {\bf 455} (2016) 2131.
\bibitem{Wang2017} Wang, G. J., Yu, H., Li, Z. X., Xia, J. Q., \& Zhu Z.-H., \textit{Astrophys. J.} {\bf 836} (2017) 103.
\bibitem{Dai2021} Dai, Y., Zheng, X.-G., Li, Z. X., {\it et al}., \textit{Astron. Astrophys.} {\bf 651} (2000) L8.
\bibitem{Tang2021} Tang L., Li X., Lin, H.-N., \& Liu L., \textit{Astrophys. J.} {\bf 907} (2021) 121.
\textbf{\bibitem{Petrosian2015}  Petrosian, V.,  Kitanidis, E., \& Kocevski, D. \textit{Astrophys. J.} {\bf 806} (2015) 44.
\bibitem{Lloyd2019}  Lloyd-Ronning, N. M., Aykutalp, A., \& Johnson, J. L.  \textit{Mon. Not. R. Astron. Soc.} {\bf 488} (2019) 5823.
\bibitem{Tstetova2017}  Tsvetkova, A., Frederiks, D., Golenetskii, S., {\it et al.} \textit{Astrophys. J.} {\bf 850} (2017) 161.
\bibitem{Yu2017}  Yu, H., Wang, F. Y., Dai, Z. G. {\it et al.} \textit{Astrophys. J. Suppl.} {\bf 218} (2015) 13.
\bibitem{Dainotti2021ApJ914}  Dainotti, M. G., Petrosian, V., \& Bowden, L. \textit{Astrophys. J.} {\bf 914} (2021) L40.
    }
\bibitem{Liu2022a} Liu, Y., Chen, F., Liang, N., {\it et al}., \textit{Astrophys. J.} {\bf 931} (2022) 50.
\bibitem{Kumar2023} Kumar, D. et al., {\it et al}., \textit{J. Cosmol. Astropart. Phys.} {\bf 07} (2023) 021.
\bibitem{Jia2022} Jia, X. D., Hu, J. P., Yang, J., Zhang, B. B., \&  Wang, F. Y., \textit{Mon. Not. R. Astron. Soc.} {\bf 516} (2022) 2575.
\bibitem{LZL2023} Li, Z., Zhang, B.,  \& Liang, N., \textit{Mon. Not. R. Astron. Soc.} {\bf 521} (2023) 4406.
\bibitem{Mu2023} Mu, Y., Chang, B., \& Xu, L. (2023), (arXiv:2302.02559).
\bibitem{Scolnic2022} Scolnic., {\it et al}., \textit{Astrophys. J.} {\bf 938} (2022) 113.
\bibitem{Plank2020} Plank Collaboration. Aghanim, N., Akrami, Y., Arroja, F., {\it et al}., \textit{Astron. Astrophys.} {\bf 641} (2020) A1.
\bibitem{Seikel2012a} Seikel, M., Clarkson, C., \& Smith, M., \textit{J. Cosmol. Astropart. Phys.} {\bf 06} (2012) 036.
\bibitem{Seikel2012b} Seikel, M., Yahya, S., Maartens, R., \&  Clarkson, C. , \textit{Phys. Rev. D} {\bf 86} (2012) 083001.
\bibitem{Lin2018} Lin, H. N., Li, M. H., \& Li, X., \textit{Mon. Not. R. Astron. Soc.} {\bf 480} (2018) 3117.
\bibitem{Li2021} Li, X., Keeley, R. E., Shafieloo, A., {\it et al}., \textit{Mon. Not. R. Astron. Soc.} {\bf 507} (2021) 919.
\bibitem{Sun2021} Sun, W., Jiao, K., \&  Zhang, T.-J., \textit{Astrophys. J.} {\bf 915} (2021) 123.
\bibitem{Benisty2023} Benisty, D., Mifsud, J., Levi Said, J., \& Staicova, D., \textit{Physics of the Dark Universe} {\bf 39} (2023) 101160.
\bibitem{Betoule2014} Betoule, M., Kessler, R., Guy, J., {\it et al}., \textit{Astron. Astrophys.}  {\bf 568} (2014) A22.
\bibitem{D'Agostini2005} D'Agostini, G., (2005)(arXiv: physics/0511182)
\bibitem{Reichart2001} Reichart, D. E., \textit{Astrophys. J.} {\bf 553} (2001) 57.
\bibitem{Amati2013} Amati, L. \& Della Valle, \textit{Int. J. Mod. Phys. D.} {\bf 22} (2013) 1330028.
\bibitem{ForemanMackey2013} Foreman-Mackey, D., Hogg, D. W., Lang, D., \& Goodman, J., \textit{Publications of the Astronomical Society of the Pacific} {\bf 125} (2013) 306.
\bibitem{Brout2022} Brout, D., Scolnic, D., {\it et al}., \textit{Astrophys. J.}  {\bf 938} (2022) 110.
\bibitem{Riess2022} Riess, A. G., Yuan, W., Macri, L. M., {\it et al}., \textit{Astrophys. J.}  {\bf 934} (2022) L7.
\bibitem{Stern2010} Stern, D., Jimenez, R., Verde, L., Kamionkowski, M. \& Starford, S. A., \textit{J. Cosmol. Astropart. Phys.} {\bf 02} (2010) 008.
\bibitem{Moresco2012} Moresco, M., Verde, L., Pozzetti, L., Jimenez, R. \& Cimatti, A., \textit{J. Cosmol. Astropart. Phys.} {\bf 08} (2012) 006.
\bibitem{Moresco2015} Moresco, M., \textit{Mon. Not. R. Astron. Soc.} {\bf 450} (2015) L16.
\bibitem{Moresco2016} Moresco, M., Pozzetti, L., Cimatti, A., {\it et al}., \textit{J. Cosmol. Astropart. Phys.} {\bf 05} (2016) 014.
\bibitem{Zhang2014} Zhang, C., Zhang, H., Yuan, S., Liu, S., Zhang, T. \& Sun, Y., \textit{Res. Astron. Astrophys} {\bf 14} (2014) 1221.
\bibitem{Ratsimbazafy2017} Ratsimbazafy, A. L., Loubser, S. I., Crawford, S. M., {\it et al}., \textit{Mon. Not. R. Astron. Soc.} {\bf 467} (2017) 3239.
\bibitem{Jiao2023} Jiao, K., Borghi, N., Moresco, M. \& Zhang, T-J., \textit{Astrophys. J. Suppl.} {\bf 265} (2023) 48.
\bibitem{Borghi2022} Borghi, N., Moresco, M. \& Cimatti, A., \textit{Astrophys. J.} {\bf 928} (2022) L4.
\bibitem{Lee2023} Lee, S.,  \textit{Mon. Not. R. Astron. Soc.}  {\bf 522} (2023) 3248
\bibitem{Favale2023}Favale, A., Gomez-Valent, A. \& Migliaccio  M., \textit{Mon. Not. R. Astron. Soc.}  {\bf 523} (2023) 3406
\textbf{
\bibitem{Dainotti2021ApJ912} Dainotti, M. G., De Simone, B., Schiavone, T., {\it et al.} \textit{Astrophys. J.} {\bf 912} (2021) 50.
\bibitem{Dainotti2022Galaxies} Dainotti, M. G., De Simone, B., Schiavone, T., {\it et al.} \textit{Galaxies} {\bf 10} (2022) 24.
\bibitem{Colgain2022} Colgain, E. O., Sheikh-Jabbari,  M. M.,   Solomon, R.  {\it et al.} \textit{Phys. Rev. D.} {\bf 106} (2022), L041301.
}
\bibitem{Jia2023} Jia, X. D., Hu, J. P., \&  Wang, F. Y., \textit{Astron. Astrophys.} {\bf 674} (2023) A45.
\bibitem{Hu2023} Hu, J. P., \&  Wang, F. Y., {\it Universe} {\bf 9(2)} (2023) 94
\bibitem{Malekjani2023} Malekjani, M., Mc Conville, R., Colgain, E. O. {\it et al.} (2023), arXiv:2301.12725.
\textbf{
\bibitem{Dainotti2023ApJS267} Dainotti, M. G., Sharma, R., Narendra, A., {\it et al.} \textit{Astrophys. J. Suppl.} {\bf 267}
(2023) 42.
\bibitem{Dainotti1907} Dainotti, M. G., Petrosian, V., Bogdan, M. {\it et al.} (2019), arXiv:1907.05074
\bibitem{Zhang2312} Zhang, B., Xie., X. Y., Nong, X. D. {\it et al.} (2023), arXiv:2312.09440
}
\bibitem{Bernardini2021} Bernardini, M. G., Cordier, B. \& Wei., J., {\it Galaxies } {\bf 9} (2021) 113.

\end{thebibliography}

\end{document}
