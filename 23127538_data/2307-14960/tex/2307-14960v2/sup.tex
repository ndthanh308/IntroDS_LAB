\renewcommand{\thetable}{S\Roman{table}}
\setcounter{table}{0}  

\renewcommand{\theequation}{S\arabic{equation}}
\setcounter{equation}{0}  

\setcounter{secnumdepth}{2}


\renewcommand{\thesection}{S\arabic{section}}
\renewcommand{\thesubsection}{\thesection.\arabic{subsection}}
\renewcommand{\thesubsubsection}{\thesubsection.\arabic{subsubsection}}

\phantomsection
\begin{large}
\begin{center}
    \textbf{Supplemental Material} 
\end{center}
\end{large}

In this Supplemental Material, in Sec.~\ref{app:Haldane}, we derive the formula $C_{j} =(-1)^{j+1} \, \textrm{sgn}( \sin \phi )$ for the topological phase using standard geometrical definitions.
In Sec.~\ref{app:inputoutput}, we give additional information on the local probe and on the derivation of Eq. (6) in the Letter.

\section{Chern number and Berry gauge fields} \label{app:Haldane}

In this section we show a detailed analytical calculation of the Chern number, based on an approach introduced by Kohmoto \cite{Kohmoto85}, for the Haldane Hamiltonian. 
This computation relies on the definition of two distinct gauge choices $G_I$ and $G_{II}$ for the Bloch eigenvectors $
\ket{u_{i,{\bf k}}} = \left[\alpha_{i}^{1}({\bf k}) a_{1,{\bf k}}^{\dagger} + \alpha_{i}^{2}({\bf k}) a_{2,{\bf k}}^{\dagger} \right] \ket{0} \,,
$ that we respectively write $\ket{u_{i,{\bf k},I}}$ and $\ket{u_{i,{\bf k},II}}$.

(i) \textit{Gauge choice $G_I$}: the coefficient $\alpha_{i }^{1}({\bf k})$ is real. 
More precisely, we choose 
\begin{equation} \label{eig0}
\alpha_{i }^{1}({\bf k})=  \rho_i({\bf k}) \,,
\end{equation}
with 
\begin{equation}
\rho_i({\bf k}) = \dfrac{h_2({\bf k}) + (-1)^i \epsilon({\bf k}) }{2\left[\epsilon({\bf k})^2 +(-1)^i h_2({\bf k}) \epsilon({\bf k}) \right]^{1/2}},
\end{equation}
and we have
\begin{equation}
\alpha_{i }^{2}({\bf k})  =  \lambda_i({\bf k}) \textrm{e}^{-i \varphi({\bf k})} \, ,
\end{equation}
 with 
 \begin{equation}
 \lambda_i({\bf k}) = \dfrac{|h_1({\bf k}) |}{2\left[\epsilon({\bf k})^2 +(-1)^i h_2({\bf k}) \epsilon({\bf k}) \right]^{1/2}} ,
 \end{equation}
 and
\begin{equation} 
\textrm{e}^{-i \varphi({\bf k})}=\dfrac{ \textrm{e}^{i {\bf k} \cdot {\bf a}_3} h_1({\bf k})^* }{ |h_1({\bf k})| }.
\end{equation}

(ii) \textit{Gauge choice $G_{II}$}: the coefficient $ \alpha_{i }^{2}({\bf k})$ is real. 
We choose $ \ket{u_{j,{\bf k},II}}=\textrm{e}^{i \varphi({\bf k})}\ket{u_{j,{\bf k},I}}$, \textit{i.e.}
\begin{equation}
\alpha_{i }^{2}({\bf k}) =\lambda_i({\bf k}),
\end{equation}
and we have
\begin{equation}
\alpha_{i }^{1}({\bf k}) = \rho_i({\bf k}) \textrm{e}^{i \varphi({\bf k})}.
\end{equation}

%%%%%%%%%%%%%%%%%%%%%%%%%%%%%%%%%%%%%%%%%%%%%%%%%%%%%%%%%%%%%%%%%%%%%%
\begin{table}[b]
\caption{\label{tab:rho1}Table giving the values of of $\rho_i({\bf k})$ and $ \lambda_i({\bf k})$ at the Dirac points in the case $|M|< 3 \sqrt{3} t_2 |\sin \phi|$ , as a function of the sign of the Semenoff mass $M$ and of the index of the energy band $i$.}
\begin{tabular}{ccccc}
\hline
\hline
 &  $\textrm{sgn}( \sin \phi ) =(-1)^i \, $ & $\textrm{sgn}( \sin \phi ) =(-1)^{i+1} \, $  \\
\hline
$\rho_i({\bf k} \rightarrow {\bf K}) $ & $0$ 
&  $(-1)^i$ \\
$ \rho_i({\bf k} \rightarrow {\bf K}')$ &  $(-1)^i$  &  $0$\\
$  \lambda_i({\bf k} \rightarrow {\bf K}) $  &  $1$  &  $0$\\
$ \lambda_i({\bf k} \rightarrow {\bf K}')$ & $0$ &  $1$ \\
\hline
\hline
\end{tabular}
\end{table}
%%%%%%%%%%%%%%%%%%%%%%%%%%%%%%%%%%%%%%%%%%%%%%%%%%%%%%%%%%%%%%%%%%%%
%%%%%%%%%%%%%%%%%%%%%%%%%%%%%%%%%%%%%%%%%%%%%%%%%%%%%%%%%%%%%%%%%%%%%%
\begin{table}[b]
\caption{\label{tab:rho3}Table giving the values of of $\rho_i({\bf k})$ and $ \lambda_i({\bf k})$ at the Dirac points in the case $|M| > 3 \sqrt{3} t_2 |\sin \phi|$, as a function of the sign of the Semenoff mass $M$ and of the index of the energy band $i$.}
\begin{tabular}{ccccc}
\hline
\hline
 &  $\textrm{sgn}\, M  =(-1)^i \, $ & $\textrm{sgn}\, M  =(-1)^{i+1} \, $  \\
\hline
$\rho_i({\bf k} \rightarrow {\bf K}) $ & $(-1)^i$ 
&  $0$ \\
$ \rho_i({\bf k} \rightarrow {\bf K}')$ &  $(-1)^i$  &  $0$\\
$  \lambda_i({\bf k} \rightarrow {\bf K}) $  &  $0$  &  $1$\\
$ \lambda_i({\bf k} \rightarrow {\bf K}')$ & $0$ &  $1$ \\
\hline
\hline
\end{tabular}
\end{table}
%%%%%%%%%%%%%%%%%%%%%%%%%%%%%%%%%%%%%%%%%%%%%%%%%%%%%%%%%%%%%%%%%%%%

 
The phase of the wavefunction is chosen by requiring that either the coefficient $\alpha_{i }^{1}({\bf k})$ or the coefficient $ \alpha_{i }^{2}({\bf k}) $ is real (and non-zero). 
$\rho_i({\bf k})$ and $ \lambda_i({\bf k})$ may vanish only at the Dirac points and it depends on the sign of $h_2$. For $|M| <3 \sqrt{3} t_2 |\sin \phi|$ ($|M| >3 \sqrt{3} t_2 |\sin \phi|$), the values of $\rho_i({\bf k})$ and $ \lambda_i({\bf k})$ at the Dirac points are given in Table~\ref{tab:rho1} (Table~\ref{tab:rho3}).
 
(i) In the case $|M|< 3 \sqrt{3} t_2 |\sin \phi|$, we notice that (see Table~\ref{tab:rho1}), for all the values of $\phi$, none of both gauge choices $G_{I}$ and $G_{II}$ is well-defined in the entire BZ. Indeed, $\rho_i({\bf k})$ and $\lambda_i({\bf k})$ each vanish at one of the Dirac points. 
We then define two non-overlapping domains $\mathcal{D}_I$ and $\mathcal{D}_{II}$ in the BZ, each containing a different Dirac point, and we use a different gauge choice for the Bloch states in each domain \cite{Kohmoto85}. To be more specific, for $(-1)^i \, \textrm{sgn}( \sin \phi ) = +1$, we apply $G_I$ for the points contained in $\mathcal{D}_{II}$ and $G_{II}$ for the points contained in $\mathcal{D}_{I}$ while for $(-1)^i \, \textrm{sgn}( \sin \phi ) = -1$, we apply $G_{I}$ for the points contained in $\mathcal{D}_{I}$ and $G_{II}$ for the points contained in $\mathcal{D}_{II}$.
Then the phase of $\ket{u_{i,{\bf k},I}}$ (the eigenstate in $\mathcal{D}_{I}$) and $\ket{u_{i,{\bf k},II}}$ (the eigenstate in $\mathcal{D}_{II}$) and the Berry gauge fields ${\bf A}_{i,{\bf k},I} = \bra{u_{i,{\bf k},I}}{\bf \nabla}_{{\bf k}} \ket{u_{i,{\bf k},I}}$ and ${\bf A}_{i,{\bf k},II} = \bra{u_{i,{\bf k},II}}{\bf \nabla}_{{\bf k}} \ket{u_{i,{\bf k},II}}$ are uniquely and smoothly defined respectively on $\mathcal{D}_I$ and $\mathcal{D}_{II}$. 
The Chern number associated to the $j^{\textrm{th}}$ band (remind that $j=1$ or $j =2$) reads
\begin{align} 
\begin{split}
\label{chern_nunununu}
    C_j = \dfrac{1}{2 i \pi} \bigg[ &\int_{\mathcal{D}_{\textrm{I}}} d^2{\bf k} \cdot \left( {\bf \nabla}_{{\bf k}} \times {\bf A}_{j,{\bf k},\textrm{I}} \right) \\ &+\int_{\mathcal{D}_{\textrm{II}}} d^2{\bf k} \cdot \left( {\bf \nabla}_{{\bf k}} \times {\bf A}_{j,{\bf k},\textrm{II}} \right)\bigg],
\end{split}
\end{align}
where $d^2{\bf k}$ is an oriented infinitesimal surface element and ${\bf \nabla}_{{\bf k}} \times {\bf A}_{j,{\bf k},\textrm{I}}$ and ${\bf \nabla}_{{\bf k}} \times {\bf A}_{j,{\bf k},\textrm{II}}$ are the Berry curvatures respectively associated to ${\bf A}_{j,{\bf k},\textrm{I}}$ and ${\bf A}_{j,{\bf k},\textrm{II}}$. Let us define a closed path P along the boundary between $\mathcal{D}_I$ and $\mathcal{D}_{II}$, surrounding once the Dirac points. Using Stokes' theorem leads to
\begin{equation} 
	C_j =  \dfrac{1}{2 i \pi} \left( \oint_{\textrm{P}} d{\bf k}  \cdot {\bf A}_{j,{\bf k},\textrm{I}} - \oint_{\textrm{P}} d{\bf k} \cdot {\bf A}_{j,{\bf k},\textrm{II}}\right).
\end{equation}
$\oint_{\textrm{P}} d{\bf k}$ is a line integral along P, where we have $\ket{u_{j,{\bf k},I}} =\textrm{e}^{i (-1)^j \, \textrm{sgn}( \sin \phi ) \varphi({\bf k})}\ket{u_{j,{\bf k},II}}$. 
We choose $\varphi({\bf k})$ so that it is smooth along the whole P path and we obtain
\begin{equation} 
	C_{j} = \dfrac{(-1)^{j} \,\textrm{sgn}( \sin \phi )}{2 \pi} \oint_{\textrm{P}} d{\bf k} \cdot {\bf \nabla}_{{\bf k}} \varphi ({\bf k}).
\end{equation}

$C_i$ is found by studying how $\varphi({\bf k})$ evolves when moving along P. Generally speaking, when the P path surround a $\varphi ({\bf k})$'s divergence (here at the ${\bf K}$ or ${\bf K}'$ point), the accumulated phase increases or decreases by $\pm 2 \pi z, \, z \in \mathbb{Z}$, which gives a quantized value of $C$, as expected. Here, we find that $\varphi ({\bf k})$ changes by $-2 \pi$ when moving along the entire closed path P. This gives 
\begin{equation}
C_{j} =(-1)^{j+1} \, \textrm{sgn}( \sin \phi ). 
\end{equation}
One can build intuition from small displacement $\delta {\bf k}$ around $\bf K$, in which case we have $h_1({\bf K} + \delta {\bf k}) = 3 t_1(- \delta k_x + i \delta k_y)/2$ where $\delta k_x = \delta {\bf k} \cdot {\bf e}_x$, $\delta k_y = \delta {\bf k} \cdot {\bf e}_y$, ${\bf e}_x=-\left({\bf b}_2+ {\bf b}_3 \right)/\sqrt{3}$ and ${\bf e}_y=\left({\bf b}_2- {\bf b}_3 \right)/3$. In this case, we easily see that $\varphi$, which is defined by $\textrm{e}^{-i \varphi({\bf k})}=\textrm{e}^{i {\bf k} \cdot {\bf a}_3} h_1({\bf k})^* / |h_1({\bf k})| $, varies by $-2 \pi$, for one entire closed path P around $\bf K$ (oriented anticlockwise).
As mentioned in the letter, this form is equivalent to the one in Eq. (2) which is local at the Dirac points.

(iii) In the case $|M| > 3 \sqrt{3} t_2 |\sin \phi|$, we gathered the values of $\rho_i({\bf k})$ and $ \lambda_i({\bf k})$ at the Dirac points in Table~\ref{tab:rho3}. At both Dirac points, either $\rho_i({\bf k})$ or $\lambda_i({\bf k})$ are non-vanishing, therefore it is possible to find a unique and smooth phase for the ket $\ket{u_{i,{\bf k}}}$ everywhere in the BZ which leads to a unique and smooth Berry gauge field ${\bf A}_{{\bf k}}$. Depending on the value of $(-1)^i \, \textrm{sgn}( M)$, we apply gauge choice $G_{I}$ or $G_{II}$ for all the points of the BZ, and then we can show that the associated wave function $\ket{u_{i,{\bf k},I}}$ or $\ket{u_{i,{\bf k},II}}$ (and its phase) is uniquely and smoothly defined, as is the Berry gauge field. Because the BZ is a torus, the Chern numbers $C_j$ are vanishing.

\section{Local response to capacitively coupled probes in a two-dimensional lattice bosonic system}\label{app:inputoutput}

In this Section, we consider a set of bosonic probes (typically microwave light resonators) capacitively coupled to a bosonic lattice model. We derive the relation between the output voltage operator in a probe at a certain position on the lattice and the input voltage operators associated to the ensemble of probes on the lattice. We consider a 2-dimensional lattice system with periodic boundary conditions. 

\subsection{Hamiltonian} The Hamiltonian describing the lattice model with capacitively coupled probes reads
\begin{equation}
	H =  H_{\textrm{lat}} + H_{\textrm{prb}} + H_{\textrm{cpl}}, 
\end{equation}
where $H_{\textrm{lat}}$ is the (topological) lattice Hamiltonian, $H_{\textrm{prb}}$ is the Hamiltonian associated to the probe(s) and $H_{\textrm{cpl}}$ is the Hamiltonian associated to the coupling between the lattice and the probe(s).

Let us call $N_C$ the number of sites per unit cell in the lattice we consider and $N$ the total number of unit cells. We label each site within a unit cell with different colors, and two different sites belonging to the same Bravais lattice are labeled by the same color. 
We define the Fourier transform of the annihilation operator of a (bosonic) particle at position $\bf r$ and the inverse relation 
\begin{equation}
a_{j,{\bf k}} = \sum_{{\bf r} \in R_{j}} \textrm{e}^{-i {\bf k} .{\bf r}} a_{{\bf r}} \quad \textrm{and} \quad a_{{\bf r}} = \dfrac{1}{N} \sum_{{\bf k}} \textrm{e}^{i {\bf k} .{\bf r}} a_{j({\bf r}),{\bf k}},
\end{equation}
with $R_j$ the ensemble containing all the lattice positions of the color-$j$ sites and the function $j({\bf r})$ returns the color index at the ${\bf r}$ site.
We formally write
\begin{equation}
	H_{\textrm{lat}} = \sum_{\boldsymbol{k}} \Psi_{\boldsymbol{k}}^\dagger h_{\boldsymbol{k}} \Psi_{\boldsymbol{k}},
\end{equation}
with $\Psi_{{\bf k}}^\dagger = \left(a_{1,{\bf k}}^\dagger, \dots a_{N,{\bf k}}^\dagger\right)$. We write $h_{{\bf k}}$'s associated eigenvalues $E_{i,{\bf k}}$, where $i \in \{1, \dots, N_C\}$, and we write the associated eigenvectors 
\begin{equation}
\ket{\Phi_{i,{\bf k}}} = \Phi_{i,{\bf k}}^\dagger \ket{0} = \sum_{j=1}^{N_C} \alpha_{i }^{j}({\bf k}) a_{j,{\bf k}}^\dagger \ket{0},
\end{equation}
with $\alpha_{i }^{j}({\bf k}) \in \mathbb{C}$. We have
\begin{equation}
	H_{\textrm{lat}} = \sum_{{\bf k}} \sum_{i=1}^{N_C} E_{i,{\bf k}} \Phi_{i,{\bf k}}^\dagger \Phi_{i,{\bf k}}.
\end{equation}
Each probe are resonators with a certain number of relevant  modes, each mode $q$ being characterized by the frequency $\omega_q$. Therefore we write
\begin{equation}
H_{\textrm{prb}} = \sum_{{\bf r} \in R_p} \sum_q \hbar \omega_q b_{\boldsymbol{r},q}^\dagger b_{{\bf r},q},
\end{equation}
with $b_{\boldsymbol{r},q}$ the annihilation operators for the mode $q$ of the probe at $\boldsymbol{r} \in R_p$, $R_p$ being the ensemble of the positions of the nodes coupled to a probe.
We assume a capacitive coupling between each probe and a node of the lattice. The Hamiltonian reads
\begin{equation}
H_{\textrm{cpl}} = \sum_{{\bf r} \in R_p} \left( a_{\bf r}+ a_{{\bf r}}^\dagger \right)\sum_q g_q \left( b_{{\bf r},q}+ b_{{\bf r},q}^\dagger  \right).
\end{equation}
The coupling amplitude $g_q$ is assumed not to depend on the position of the probe. 

\subsection{Input-output analysis} 

Here we use the input-output formalism as reviewed in Ref.~\cite{Clerk10}.

Let us define the input voltage in the probe at $\bf r$
\begin{equation}
V_{\bf r}^{\textrm{in}}(t) =\sum_q g_q \left[ \textrm{e}^{-i \omega_q (t-t_i)}b_{{\bf r},q}(t_i) + h.c. \right],
\end{equation}
where $t_i < t$ is an initial time in the distant past, and the output voltage in the probe at $\bf r$
\begin{equation}
V_{\bf r}^{\textrm{out}}(t) = \sum_q g_q \left[ \textrm{e}^{-i \omega_q (t-t_f)} b_{{\bf r},q}(t_f) + h.c. \right],
\end{equation}
where $t_f>t$ is a final time in the distant future. Let us call $x_{\bf r} =  a_{\bf r}+ a_{{\bf r}}^\dagger$. The Heisenberg equation of motion (EOM) for the $b_{jq}$ operator reads\begin{equation} \label{eomb}
	\dot{b}_{{\bf r},q} = \dfrac{i}{\hbar}  \left[H,b_{{\bf r},q}\right] = -i \omega_q b_{{\bf r},q} - \dfrac{i  g_q}{\hbar} x_{\bf r}.
\end{equation}
The solution of this equation of motion is
\begin{equation} \label{fi0}
	 b_{{\bf r},q} (t) = \textrm{e}^{-i \omega_q (t-t_i)}b_{{\bf r},q}(t_i) - \dfrac{i  g_q}{\hbar} \int_{t_i}^t d\tau \textrm{e}^{-i \omega_q (t-\tau)} x_{\bf r}(\tau),
\end{equation}
or equivalently
\begin{equation}
	 b_{{\bf r},q} (t) = \textrm{e}^{-i \omega_q (t-t_f)}b_{{\bf r},q}(t_f) + \dfrac{i  g_q}{\hbar} \int_{t}^{t_f} d\tau \textrm{e}^{-i \omega_q (t-\tau)} x_{\bf r}(\tau).
\end{equation}
Combining the previous equations and their complex conjugate counterparts we get 
\begin{equation} \label{nan}
V_{\bf r}^{\textrm{in}}(t) - 2 \sum_q \dfrac{g_q^2}{\hbar} \int_{t_i}^{t_f} d \tau \sin \left[ \omega_q (t- \tau)\right] x_{\bf r}(\tau) = V_{\bf r}^{\textrm{out}}(t).
\end{equation}
We Fourier transform the previous equation with respect to the time variable $t$, we define $J(t) = 2 i \sum_q \dfrac{g_q^2}{\hbar} \sin \left( \omega_q t\right) $ \cite{Schiro14} and we get
\begin{equation} \label{ano}
V_{\bf r}^{\textrm{out}}(\omega) = V_{\bf r}^{\textrm{in}}(\omega) +i J(\omega) x_{\bf r}(\omega).
\end{equation}
Notice that $J(\omega) \in \mathbb{R}$. Its explicit expression is
\begin{equation}
J(\omega) = 2 \pi \sum_q \dfrac{g_q^2}{\hbar} \left[ \delta(\omega - \omega_q) - \delta(\omega + \omega_q) \right],
\end{equation}
where $\delta$ is the Dirac delta function.

Now we express $x_{\bf r}(\omega)$ as a function of $V_{\bf r}^{\textrm{in}}(\omega)$. The Heisenberg EOM for the $\Phi_{i,{\bf k}}$ modes reads
\begin{align} \label{eom}
\begin{split}
   \hbar \dot{\Phi}_{i,{\bf k}} = &-i E_{i,{\bf k}} \Phi_{i,{\bf k}} \\ &- i \sum_{{\bf r} \in R_p}  \left[\alpha_{i }^{j}({\bf k})\textrm{e}^{i {\bf k} \cdot {\bf r}}\right]^* \sum_q g_q \left( b_{{\bf r},q} + b_{{\bf r},q}^\dagger \right),
\end{split}
\end{align}
where $j$ is a function of ${\bf r}$; it returns the color index at the ${\bf r}$ site. Using equation~\ref{fi0} we have, up to second order in the coupling amplitudes,
\begin{equation}
	i \hbar \dot{\Phi}_{i,{\bf k}} (t)= E_{i,{\bf k}} \Phi_{i,{\bf k}}(t) +\sum_{{\bf r} \in R_p}  \left[\alpha_{i }^{j}({\bf k}) \textrm{e}^{i {\bf k} \cdot {\bf r}} \right]^* V_{\bf r}^{\textrm{in}}(t).
\end{equation}
Now we use the Fourier transformation with respect to the time variable to write, still up to second order in the coupling amplitudes, 
\begin{equation} 
\Phi_{i,{\bf k}} (\omega)= \dfrac{1}{-\hbar \omega - E_{i,{\bf k}} + i 0^+} \sum_{{\bf r} \in R_p}  \left[\alpha_{i }^{j}({\bf k}) \textrm{e}^{i {\bf k} \cdot {\bf r}} \right]^* V_{\bf r}^{\textrm{in}}(\omega) ,
\end{equation} 
and 
\begin{equation} 
\Phi_{i,{\bf k}}^\dagger (\omega)= -\dfrac{1}{-\hbar \omega + E_{i,{\bf k}} + i 0^+} \sum_{{\bf r} \in R_p}  \alpha_{i }^{j}({\bf k}) \textrm{e}^{i {\bf k} \cdot {\bf r}} V_{\bf r}^{\textrm{in}}(\omega).
\end{equation} 
Note that $\Phi_{i,{\bf k}} (\omega) = \textrm{T.F.}\left( \Phi_{i,{\bf k}}\right)(\omega)$, T.F. denoting the Fourier transform and $\Phi_{i,{\bf k}}^\dagger (\omega) = \textrm{T.F.}\left( \Phi_{i,{\bf k}}^\dagger\right)(\omega)$ so $ \Phi_{i,{\bf k}} (\omega) \neq \left( \Phi_{i,{\bf k}}^\dagger (\omega) \right)^\dagger$. 

We introduce the $\beta_{i }^{j}({\bf k})$ coefficients such that
\begin{equation}
a_{j,{\bf k}}^\dagger \ket{0} = \sum_{i=1}^{N_C} \beta_{i }^{j}({\bf k}) \Phi_{i,{\bf k}}^\dagger \ket{0} .
\end{equation}
Then we have
\begin{equation}
x_{\bf r} = \dfrac{1}{ N} \sum_{\bf k} \sum_{i=1}^{N_C} \textrm{e}^{i {\bf k} \cdot {\bf r}} \left[\beta_{j({\bf r})}^i({\bf k})\right]^* \Phi_{i,{\bf k}}+ h.c. ,
\end{equation}
and we obtain, up to second order in the coupling amplitudes
\begin{equation} \label{eno}
x_{\bf r}(\omega) = \sum_{{\bf r}_0 \in R_p} \chi_{{\bf r},{\bf r}_0} V_{{\bf r}_0}^{\textrm{in}}(\omega),
\end{equation}
with
\begin{align}
\begin{split}
\chi_{{\bf r},{\bf r}_0} = \dfrac{1}{N}\sum_{\substack{ i=1 \\ {\bf k} }}^{N_C}  \left( \dfrac{C_{i,{\bf k},{\bf r},{\bf r}_0}^*}{-\hbar\omega - E_{i,{\bf k}} + i 0^+} - \dfrac{C_{i,{\bf k},{\bf r},{\bf r}_0}}{-\hbar\omega + E_{i,{\bf k}} + i 0^+} \right),
\end{split}
\end{align}
and
\begin{equation}
C_{i,{\bf k},{\bf r},{\bf r}_0}=\beta_{j({\bf r})}^i({\bf k}) \alpha^{j({\bf r}_0)}_{i}({\bf k}) \textrm{e}^{-i {\bf k} \cdot ({\bf r}-{\bf r}_0)}.
\end{equation}
Let us notice from the last equation that adding a probe with no input does not influence the response at the other probes (with or without input). This is because we restricted the response to first order in $g$. 

Finally, Eq.~\eqref{ano} gives, up to fourth order in the coupling amplitudes $\{g_q\}$, 
\begin{equation} \label{eq:hzfjzd1}
V_{{\bf r}}^{\textrm{out}}(\omega) =V_{{\bf r}}^{\textrm{in}}(\omega) +i J(\omega])\sum_{{\bf r}_0 \in R_p} \chi_{{\bf r},{\bf r}_0} V_{{\bf r}_0}^{\textrm{in}}(\omega).
\end{equation}

In the simple case ${\bf r}={\bf r}_0={\bf R}_0$ we consider in the main text, for simplicity, we define $\gamma_{j_0,{\bf k}}^{i}$ such that $\gamma_{j_0,{\bf k}}^{i}=C_{i,{\bf k},{\bf R}_0,{\bf R}_0} = \beta_{j_0 }^{i}({\bf k}) \alpha_{i }^{j_0}({\bf k})$. Moreover, for the Haldane system we consider the Bloch eigenvectors
\begin{equation} \label{eq:blockvec}
\ket{u_{i,{\bf k}}} = u_{i,{\bf k}}^\dagger \ket{0} = \left[\alpha_{i }^{1}({\bf k}) a_{1,{\bf k}}^\dagger + \alpha_{i }^{2}({\bf k}) a_{2,{\bf k}}^\dagger \right] \ket{0}, 
\end{equation}
which are defined through the components $\alpha_{i }^{1}({\bf k})$ and $\alpha_{i }^{2}({\bf k})$
\begin{equation} \label{eq:blockcompo}
\alpha_{i }^{2}({\bf k}) =  \dfrac{h_1({\bf k})^*}{h_2({\bf k})-h_0({\bf k})+E_{i,{\bf k}}} \alpha_{i }^{1}({\bf k}).
\end{equation}
We have introduced $E_{i,{\bf k}} = h_0({\bf k}) + (-1)^i \epsilon({\bf k})$, $\epsilon({\bf k}) = \sqrt{|h_1({\bf k})|^2+h_2({\bf k})^2}$ and the coefficients $\beta_{i }^{j}({\bf k})$ satisfy 
\begin{equation} \label{eq:blockreverse}
a_{j,\boldsymbol{k}}^\dagger \ket{0} = \sum_{i=1}^2 \beta_{i }^{j}({\bf k}) \Phi_{i,\boldsymbol{k}}^\dagger \ket{0}, \, j=\{A,B\}.
\end{equation}
Using Eqs.~\eqref{eq:blockvec}, \eqref{eq:blockcompo} and \eqref{eq:blockreverse}, we obtain
\begin{equation}
\gamma_{j_0,{\bf k}}^{i}  = \dfrac{1}{2}+\dfrac{(-1)^{j_0+i+1} h_2({\bf k})}{2 \epsilon({\bf k})}.
\end{equation}
