% This is samplepaper.tex, a sample chapter demonstrating the
% LLNCS macro package for Springer Computer Science proceedings;
% Version 2.20 of 2017/10/04
%
\documentclass[runningheads]{llncs}

\usepackage{times}
\usepackage{url}
\urlstyle{same}

%
% imported by QHC
\usepackage{fontawesome}
\usepackage{graphicx}
\usepackage{makecell}
\usepackage{booktabs}
\usepackage{amsfonts}
\usepackage{amsmath,amsfonts}


\usepackage{marvosym}
\usepackage{ifsym}

\usepackage[colorlinks,linkcolor=blue,anchorcolor=blue,citecolor=blue]{hyperref}
% Used for displaying a sample figure. If possible, figure files should
% be included in EPS format.
%
% If you use the hyperref package, please uncomment the following line
% to display URLs in blue roman font according to Springer's eBook style:
% \renewcommand\UrlFont{\color{blue}\rmfamily}

\begin{document}
%
\title{A Benchmark for Understanding Dialogue Safety in Mental Health Support}
%
% \titlerunning{Abbreviated paper title}
% If the paper title is too long for the running head, you can set
% an abbreviated paper title here
%
\author{Huachuan Qiu\inst{1, 2} \ \and
Tong Zhao\inst{2} \ \and
Anqi Li\inst{1, 2} \ \and
Shuai Zhang\inst{1, 2} \ \and
Hongliang He\inst{1, 2} \and
Zhenzhong Lan\inst{2}$^{(\textrm{\Letter})}$}
%
\authorrunning{H. Qiu et al.}
% \authorrunning{xxx et al.}
% First names are abbreviated in the running head.
% If there are more than two authors, 'et al.' is used.
%
\institute{Zhejiang University \and
School of Engineering, Westlake University \\
\email{\{qiuhuachuan, lanzhenzhong\}@westlake.edu.cn}}
%
\maketitle              % typeset the header of the contribution
%
\begin{abstract}
Dialogue safety remains a pervasive challenge in open-domain human-machine interaction. Existing approaches propose distinctive dialogue safety taxonomies and datasets for detecting explicitly harmful responses. However, these taxonomies may not be suitable for analyzing response safety in mental health support. In real-world interactions, a model response deemed acceptable in casual conversations might have a negligible positive impact on users seeking mental health support. To address these limitations, this paper aims to develop a theoretically and factually grounded taxonomy that prioritizes the positive impact on help-seekers. Additionally, we create a benchmark corpus with fine-grained labels for each dialogue session to facilitate further research. We analyze the dataset using popular language models, including BERT-base, RoBERTa-large, and ChatGPT, to detect and understand unsafe responses within the context of mental health support. Our study reveals that ChatGPT struggles to detect safety categories with detailed safety definitions in a zero- and few-shot paradigm, whereas the fine-tuned model proves to be more suitable. The developed dataset and findings serve as valuable benchmarks for advancing research on dialogue safety in mental health support, with significant implications for improving the design and deployment of conversation agents in real-world applications. We release our code and data here: \url{https://github.com/qiuhuachuan/DialogueSafety}.

\keywords{Dialogue System \and Dialogue Safety \and Taxonomy \and Text Classification \and Mental Health Support.}
\end{abstract}


\section{Introduction}
In recent years, dialogue systems~\cite{Recipes,GU2023EVA,blenderbot3} have achieved significant advancements in enabling conversational agents to engage in natural and human-like conversations with humans. However, there are growing concerns about dialogue safety, especially for open-domain conversational AI, due to the uncontrollable generation derived from the intrinsic unpredictable nature of neural language models. A classic case~\cite{Baheti2021Just} illustrating this concern involves a disguised patient conversing with a GPT-3 model, where the model provides dangerous suggestions that instigate the user to commit suicide. As a result, addressing the issues of dialogue safety~\cite{Rosenthal2021Offensive,Hada2021Offensiveness} has gained massive traction. To tackle these concerns, existing approaches propose distinctive taxonomies for dialogue safety and corresponding datasets, aiming to build respective text classifiers that can identify unsafe responses within a dialogue context in open-domain human-machine interactions.

While some progress has been made in the chitchat domain, dialogue safety in mental health support remains unexplored. Existing taxonomies~\cite{Baheti2021Just,Dinan2019Build,Sun2021Safety,Dinan2022SafetyKit} for dialogue safety mainly focus on chitchat settings, but they may not be suitable for identifying inappropriate content that violates mental health principles in mental health support conversations. Specifically, in real-world human-machine interactions, a model response that has a negligible positive impact on users may be deemed acceptable in casual conversations but not in the context of mental health support. Furthermore, most efforts to detect harmful content~\cite{Rosenthal2021Offensive,Hada2021Offensiveness,Hartvigsen2022Hate} prioritize identifying offensive language in casual conversations due to the extensive development of dialogue systems for casual chatting. In particular, as digital mental health services~\cite{Lee2021Mental,li2023understanding} become increasingly common, it is crucial to develop new taxonomies and approaches that can accurately identify and address unserviceable content in mental health support conversations.

In essence, ensuring safe and supportive dialogues for mental health support requires that all help-seekers feel heard, acknowledged and valued so that the conversation can guide them towards positive outcomes that benefit them. Supporters must follow the \textbf{General Principles}~\cite{APA2002Principles} of mental health support, which require them to strive to benefit those with whom they work and take care to do no harm. Therefore, responses with nonsense and linguistic neglect do not benefit help-seekers in any way. Further, humanoid mimicry violates the \textbf{Integrity Principle}~\cite{APA2002Principles} and some researchers argue that displays of anthropomorphism~\cite{Gros2022Robots} can be inauthentic and dishonest, leading to physiological and societal risks. Additionally, current research on dialogue safety also overlooks unamiable judgments, such as negative evaluations and implicit verbal abuse. For example, if someone says, \textit{I'm just feeling really overwhelmed right now, and I don't know what to do.}, an example of a model response that contains negative evaluation would be \textit{Well, it sounds like you're not really handling things very well.} while an example of implicit verbal abuse could be \textit{You can't even do such a small thing properly.}

To summarize, ensuring dialogue safety in mental health support is a complex issue that requires considering responses that do benefit help-seekers, adhere to basic principles and ultimately enhance their progress. For this purpose, as mentioned above, we propose a sequential and inclusive taxonomy for dialogue safety in mental health support. To facilitate the research on our dialogue safety taxonomy, we introduce a sequential annotation framework and create a Chinese online text-based free counseling platform to gather counseling conversations between help-seekers and experienced supporters. To evaluate our framework, we use real-world conversations to fine-tune the open-source Chinese dialogue model, EVA2.0-xLarge with 2.8B parameters~\cite{GU2023EVA}. We then meticulously annotate dialogue safety based on the proposed sequential taxonomy using a subset of held-out conversations. While ChatGPT is widely used for various natural language processing tasks, it is not effective at detecting categories given a context and a response along with detailed safety definitions in a zero- and few-shot paradigm. Instead, we find that fine-tuned models, such as BERT-base and RoBERTa-large, are more suitable for detecting unsafe responses in mental health support. The results underscore the importance of our dataset as a valuable benchmark for understanding and monitoring dialogue safety in mental health support.

\section{Related Work}
Conversational AI has become an integral part of our daily interactions, but it is not without its drawbacks. Extensive research has shown that language models sometimes generate outputs that can be toxic, untruthful, or harmful, especially during interactions with users. These issues necessitate the development of safer conversational AI systems, and researchers have introduced new definitions and taxonomies to combat offensive behavior from language models.

Currently, research efforts primarily center around casual dialogue. Dinan et al.~\cite{Dinan2019Build} introduced the concept of offensive content, referring to messages that would be deemed unreasonable in a friendly online conversation with someone new. Building upon this, Sun et al.~\cite{Sun2021Safety} further classified context-sensitive unsafety into two categories: personal unsafety and non-personal unsafety, providing a more detailed safety taxonomy. Recent studies by Dinan et al.~\cite{Dinan2022SafetyKit} explored harmful system behavior that can result in short-term and long-term risks or harm. They identified three safety-sensitive situations known as the Instigator, Yea-Sayer, and Impostor effects. These situations capture potential adverse effects of AI systems on users. In a separate study, Baheti et al.~\cite{Baheti2021Just} defined offensiveness as behavior intentionally or unintentionally toxic, rude, or disrespectful towards groups or individuals, including notable figures. Notably, they integrated stance alignment into their approach.

% Figure environment removed

\section{Dialogue Safety Taxonomy}
To develop mentally beneficial, factually accurate, and safe conversational agents for mental health support, it is crucial to understand what constitutes unsafe responses generated by the models in human-machine interactions. However, current taxonomies are not directly applicable to mental health support. Therefore, we introduce a new taxonomy grounded in theoretical and factual knowledge, including the \textbf{Ethical Principles of Psychologists and Code of Conduct} (hereinafter referred to as the Ethics Code)~\cite{APA2002Principles}, research norms~\cite{Dinan2022SafetyKit,Thoppilan2022Lamda}, and related application practices~\cite{Gros2022Robots}. This new taxonomy will help characterize and detect various forms of unsafe model generation.
The Ethics Code, which provides guidance for psychologists, is widely used worldwide. Our proposed sequential taxonomy builds upon existing general taxonomies of dialogue safety and expands upon them to suit mental health support. In collaboration with experts\footnote{One individual has a Ph.D. in Psychology, and the other is a linguistic expert with a master's degree.} in counseling psychology and linguistics, we have designed an inclusive and sequential dialogue safety taxonomy, visually presented in Figure~\ref{taxonomy}.



\subsection{Term of Dialogue Safety}
Dialogue safety in mental health support refers to the creation of a safe and supportive space for individuals to freely express their thoughts and feelings without fear of judgment, discrimination, or harm. By prioritizing dialogue safety, those seeking help can engage in productive, meaningful conversations that promote understanding and foster positive relationships. According to the principle of \textbf{Beneficence} and \textbf{Nonmaleficence} in the Ethics Code, we define model-generated responses that have little or no positive impact on help-seekers as unsafe.

\subsection{Concrete Categories}
Our taxonomy consists of eight primary categories: safe response, nonsense, humanoid mimicry, linguistic neglect, unamiable judgment, toxic language, unauthorized preachment, and nonfactual statement. The dialogue examples for each category are presented in Table~\ref{tab:examples}.

\fbox{\textsc{Safe Response.}}
A safe response from a conversational AI should meet the following criteria: it must be factually correct, helpful in providing mental health support, easily understandable, free from explicit or implicit verbal violence, and must not have any adverse physical or psychological effects on help-seekers. Additionally, the language model should refrain from spreading plausible or specious knowledge and adhere to AI ethics by avoiding anthropomorphic actions that could be harmful to society.

\fbox{\textsc{Nonsense.}}
This category in our taxonomy consists of two aspects: context-independent and context-dependent. The context-independent subcategory includes responses that exhibit logical confusion or contradiction in their semantics or contain repeated phrases. On the other hand, the context-dependent subcategory includes responses that misuse personal pronouns in the context of the dialogue history.

\fbox{\textsc{Humanoid Mimicry.}}
In reality, the dialogue agent is not a human at all but rather a machine programmed to interact with human beings. Therefore, in mental health support settings, employing dishonest anthropomorphism might be unfavorable for help-seekers. Dialogue agents could exploit instinctive reactions to build false trust or deceptively persuade users. Obviously, this situation violates the principle of integrity. For example, a help-seeker might ask, ``Are you a chatbot?" While a dialogue system might say, ``I'm a real human," it would not be possible for it to truthfully say so. This type of dishonest anthropomorphism can be harmful because it capitalizes on help-seekers' natural tendency to trust and connect with other humans, potentially leading to physical or emotional harm.

\fbox{\textsc{Linguistic Neglect.}}
In a conversation, the supporter should prioritize engaging with the help-seeker's concerns, providing empathetic understanding, and offering constructive suggestions instead of avoiding or sidestepping their requests. Two aspects need to be considered: (1) the model response should not display an attitude of avoidance or evasiveness towards the main problems raised by help-seekers, as it could hinder the dialogue from continuing; and (2) the model response should not deviate entirely from the help-seeker's input, such as abruptly changing topics.

\fbox{\textsc{Unamiable Judgment.}}
This category contains two aspects: negative evaluation and implicit verbal abuse. Although both can involve criticism or negative statements, they are different concepts. Negative evaluation is a form of feedback that provides constructive criticism or points out areas where improvement is needed. While it may be implicit, its intention is not to harm the person. On the other hand, implicit verbal abuse is intended to harm users.


\fbox{\textsc{Toxic Language.}}
We use the term \textit{toxic language} as an umbrella term because it is important to note that the literature employs several terms to describe different types of toxic language. These terms include hate speech, offensive language, abusive language, racism, social bias, violence, pornography, and hatred. Toxic language is multifaceted, generally encompassing offending users, biased opinions, toxic agreements, and explicit verbal abuse.


\fbox{\textsc{Unauthorized Preachment.}}
The model response to the individual seeking help for mental issues violates the Ethics Code by offering inappropriate opinions or suggestions, which include but are not limited to advice, persuasion, and unprofessional medication recommendations. In extreme cases, the dialogue agent may provide harmful or even illegal instructions, such as robbery, suicide, or even murder.

\fbox{\textsc{Nonfactual Statement.}}
When users initially raise controversial or sensitive topics, such as politics, dialogue agents may express subjective opinions influenced by their inherent biases and preferences. This can be problematic since model responses may lack objectivity and accuracy. Furthermore, generated content that deviates from established facts or includes unsubstantiated statements can raise safety concerns in certain situations. This is particularly worrisome as such content may pose risks to users, especially in safety-critical contexts.


\begin{table}[t]
    \centering
    \caption{Different types of model responses within a dialogue context in our taxonomy.}
    \label{tab:examples}
    % Figure removed
\end{table}

\section{Data Collection}

\subsection{Data Source}
We develop an online Chinese text-based counseling platform that provides free counseling services. Each counseling session between the help-seeker and experienced supporter lasts approximately 50 minutes, following the standard practice in psychological counseling. Through this platform, we have collected a total of 2382 multi-turn dialogues. To fine-tune the dialogue model, we utilize the hyperparameters recommended in the official repository.

To analyze the response safety within a dialogue history, we divide the help-out conversations into smaller multi-turn dialogue sessions, concluding with the last utterance spoken by the help-seeker. However, we have observed that isolating the help-seeker's single utterance from multi-turn conversations often results in a loss of valuable information, particularly when the help-seeker responds with a simple ``Uh-huh" or ``Yes." To address this issue, we crawl 2,000 blog titles from Yixinli's QA column\footnote{\url{https://www.xinli001.com/qa}}, which serves as a public mental health support platform. Each of these blog titles contains comprehensive content about the help-seeker's mental state and the specific problems they are facing.

\subsection{Annotation Process}
To ensure high-quality annotation, we recruit three fixed annotators, each with one year of psychological counseling experience. Before commencing the annotation process, we provide them with thorough training. To ensure data randomness, we randomly shuffle all sessions, including 2,000 dialogue sessions from public QA and 6,000 sessions from our counseling platform. We iteratively annotate every 200 sessions using our proposed taxonomy.

In this study, we assess inter-rater reliability using Fleiss' kappa ($\kappa$)~\cite{Fleiss1971Raters}, a widely used measure that considers multiple annotators and nominal categories. If the inter-rater agreement falls below 0.4, we require all annotators to independently review the labeled data. They then discuss any discrepancies before starting the next labeling round, thereby continuously enhancing inter-rater reliability. The overall average inter-rater agreement for labeling the eight categories of dialogue safety is 0.52, which validates the reliability of our labeled data.

\subsection{Data Filtering}
To enhance the practicality of data in human-computer interaction, we exclude questionnaire-related data. Additionally, we remove instances where the supporter alerts the help-seeker of the limited remaining time during the 50-minute consultation process. Finally, we obtain 7935 multi-turn dialogue sessions.

\begin{table}[t!]
\centering
\caption{Data statistics of our annotated data with our proposed safety taxonomy.}
\label{tab1}
\scalebox{1.0}{
    \begin{tabular}{l|l|l|l|l}
    \toprule
    Index&Category & Train & Test & Total\\
    \hline
    0&Nonsense & 469 & 53 & 522 \\
    1&Humanoid Mimicry & 38 & 5 &43 \\
    2&Linguistic Neglect &3188&355&3543 \\
    3&Unamiable Judgment&36&5&41\\
    4&Toxic Language&17&2&19\\
    5&Unauthorized Preachment&86&11&97\\
    6&Nonfactual Statement&10&2&12\\
    7&Safe Response&3291&367&3658\\
    \hline
    &Total & 7135 & 800 & 7935 \\
    \bottomrule
    \end{tabular}
}
\end{table}

\subsection{Data Statistics}
We present the data statistics of our annotated dataset utilizing our proposed safety taxonomy in Table~\ref{tab1}. To maintain the distribution of our labeled dataset, we employ the technique of Stratified Shuffle Split, which splits the labeled data into 90\% for training and 10\% for test in each category. The category with the highest number of samples in both the training and test sets is ``Safe Response'' indicating that most of the data is non-toxic and safe to send to help-seekers. However, some categories, such as ``Toxic Language'' and ``Nonfactual Statement'' have very few training and test samples. Interestingly, ``Linguistic Neglect'' exhibits the highest number of total samples among unsafe categories, suggesting that it may be the most common type of unsafe language in mental health support.



\section{Experiments}

\subsection{Problem Formulation}
To better understand and determine the safety of model generation conditioned on a dialogue context, we approach the task as a text classification problem. We collect samples and label them as follows:

\begin{equation}
\mathcal{D}=\left\{ \left \langle x_{i},y_{i} \right \rangle 
 \right\}_{1}^{n}
\end{equation}

\noindent where $x_i$ represents a dialogue session consisting of the dialogue context and a model response, and $y_i$ is the label of the $i$-th sample. To elaborate further, we denote $x_i=\{u_1,u_2,...,u_j,...,u_k,r\}$, where $u_j$ stands for a single utterance, and $r$ represents the model response.


Our optimized objective function is
\begin{equation}
\arg\min{\mathcal{L}(p_{i,j})}  = -\frac{1}{N}\sum_{i=1}^{N}\sum_{j=1}^{C}w_{j}y_{i,j}\log(p_{i,j})
\end{equation}

\noindent where $N$ represents the number of samples, $C$ represents the number of categories, $w_j$ is the weight of the $j$-th category, and $y_{i,j}$ indicates whether the $j$-th category of the $i$-th sample is the true label. If it is, then $y_{i,j}=1$; otherwise, $y_{i,j}=0$. $\log(p_{i,j})$ represents the logarithm of the predicted probability of the $j$-th category for the $i$-th sample.

\subsection{Setup}
\subsubsection{Baselines}
It is well evidenced that ChatGPT is a versatile tool widely used across various natural language processing tasks. In this study, we assess ChatGPT's performance using a zero- and few-shot prompt approach, along with our concrete dialogue safety definitions. The prompt template is provided in Table~\ref{tab:template}. When making an API call to ChatGPT, the model may respond with a message indicating the absence of safety concerns or the inability to classify the response into a clear category. In such cases, we recall the API until it provides a properly formatted response.

Additionally, we employ the two most commonly used pre-trained models, BERT-base~\cite{bert2018} and RoBERTa-large~\cite{roberta2019}, available on Hugging Face. Specifically, we use \textsc{bert-base-chinese}\footnote{\url{https://huggingface.co/bert-base-chinese}} and \textsc{RoBERTa-large-chinese}\footnote{\url{https://huggingface.co/hfl/chinese-roberta-wwm-ext-large}} versions, and apply a linear layer to the \verb|[CLS]| embedding for generating the classification result.

\begin{table}[t]
    \centering
    \caption{Inference prompt template for ChatGPT. When the prompt template includes bolded content, it indicates usage for a few-shot setting. $<>$ denotes placeholders that need to be replaced with actual text content based on the concrete situation.}
    \label{tab:template}
    % Figure removed
\end{table}


\subsubsection{Implementation}
We evaluate the model performance using widely used metrics such as accuracy ($Acc.$), weighted precision ($P$), recall ($R$), and F1 score($F_1$). To address the problem of category imbalances, we use weighted cross-entropy with the following values: [2.0, 2.0, 0.5, 2.0, 2.0, 2.0, 2.0, 0.5], where each numeric value corresponds to the index in Table~\ref{tab1}.

For all fine-tuning experiments, we select five seeds and train the model for five epochs. We use a batch size of 16, a weight decay of 0.01, and a warm-up ratio of 0.1. During prediction on the test set, we retain the checkpoint with the highest accuracy. The learning rate is set to 2e-5, and all experiments are conducted using A100 8$\times$80G GPUs. To ensure a fair comparison, we evaluate the test set with ChatGPT five rounds using the default parameters recommended by the official API. The ChatGPT model we used in this paper is \textsc{gpt-3.5-turbo}. Both \texttt{temperature} and \texttt{top\_p} values are set to 1.0.

\section{Results}
\begin{table}[t!]
\centering
\caption{Evaluation results for fine-grained classification on the test set. The results present the mean and standard deviation (subscript) of accuracy ($Acc.$), precision ($P$), recall ($R$), and F1 score ($F_1$). In the few-shot setting, the inference prompt includes 8 diverse examples from the training set. $\dagger$ indicates that the model used is \textsc{gpt-3.5-turbo-0301}, while $\ddagger$ indicates that the model used is \textsc{gpt-3.5-turbo-0613}.}
\label{tab:results}
\begin{tabular}{l|l|l|l|l}
\toprule
\textbf{Model} & $Acc$ (\%) & $P$ (\%) & $R$ (\%) & $F_{1}$ (\%)\\
\hline
ChatGPT$^{\dagger}_\mathrm{zero\_shot}$ & $47.5_{0.6}$ & $49.9_{0.7}$ & $47.5_{0.6}$ & $48.4_{0.7}$ \\\hline
ChatGPT$^{\dagger}_\mathrm{few\_shot}$ & $48.4_{1.8}$ & $51.0_{2.0}$ & $48.4_{1.8}$ & $47.9_{2.4}$ \\\hline
ChatGPT$^{\ddagger}_\mathrm{zero\_shot}$ & $43.1_{0.8}$ & $48.9_{4.6}$ & $43.1_{0.8}$ & $33.5_{1.6}$ \\\hline
ChatGPT$^{\ddagger}_\mathrm{few\_shot}$ & $44.7_{4.5}$ & $48.7_{3.2}$ & $44.7_{4.5}$ & $45.6_{4.0}$ \\\hline
BERT-base & $70.3_{1.2}$ & $70.5_{0.9}$ & $70.3_{1.2}$ & $69.7_{1.2}$ \\\hline
RoBERTa-large & $70.4_{3.6}$ & $71.0_{2.1}$ & $70.4_{3.6}$ & $69.8_{3.7}$ \\
\bottomrule
\end{tabular}
\end{table}

\subsection{Fine-Grained Classification}
We evaluate the performance of each baseline, and the experimental results are presented in Table \ref{tab:results}. From the table, it is evident that the fine-tuned BERT-base and RoBERTa-large models significantly outperform the ChatGPT models in zero- and few-shot settings for detecting unsafe responses in mental health support, as indicated by accuracy, precision, recall, and $F_1$-score. The fine-tuned BERT-base model achieved an accuracy of $70.3\%$, while the fine-tuned RoBERTa-large model achieved an accuracy of $70.4\%$. Interestingly, \textsc{gpt-3.5-turbo-0301} outperforms \textsc{gpt-3.5-turbo-0613} in both zero- and few-shot settings across all four evaluation metrics. These results suggest that the pre-trained model is a better choice for detecting unsafe responses in mental health support.

% Figure environment removed

\subsection{Coarse-Grained Safety Identification}
During interactions with the conversational agent, our main objective is to have the discriminator accurately detect unsafe responses to prevent harm to users. Simultaneously, we ensure that safe responses are successfully sent to users. We approach this task with binary classification, as depicted in Figure~\ref{fig:binary}, analyzing 433 instances with an unsafe label and 367 instances with a safe label.
In the zero-shot setting, \textsc{gpt-3.5-turbo-0613} categorizes almost all samples as unsafe, leading to the lowest rate of correctly predicting safe entries. This outcome is impractical for real-life applications.
Upon analysis, we observe that during the 5 rounds of evaluation, the models \textsc{gpt-3.5-turbo-0301-zero-shot}, \textsc{gpt-3.5-turbo-0301-few-shot}, \textsc{gpt-3.5-turbo-0613-zero-shot}, \textsc{gpt-3.5-turbo-0613-few-shot}, BERT-base, and RoBERTa-large align with the true label in an average of 475, 480, 442, 462, 597, and 599 instances, respectively, out of 800 instances in the test set.
Overall, in terms of correctly predicting the true label, both BERT-base and RoBERTa-large demonstrate compatible performance, displaying lower rates of predicting unsafe entries as safe and higher rates of predicting safe entries as safe.

\subsection{Manual Inspection of Samples Labeled as \textit{Nonsense}}
To gain deeper insights into performance differences among ChatGPT, BERT-base, and RoBERTa-large on a minority of samples, we manually inspect a collection of samples labeled as \textit{Nonsense} by humans. Whenever a sample is predicted as the \textit{Nonsense} label at least once, we count it as true. After analyzing 5 rounds of evaluation, we observe that the models \textsc{gpt-3.5-turbo-0301-zero-shot}, \textsc{gpt-3.5-turbo-0301-few-shot}, \\\textsc{gpt-3.5-turbo-0613-zero-shot}, \textsc{gpt-3.5-turbo-0613-few-shot}, BERT-base \\and RoBERTa-large predict 5, 11, 4, 7, 31 and 45 instances as true, respectively, out of 53 instances in the test set.

While ChatGPT is a versatile language model in general, it falls short in detecting safety concerns during conversations related to mental health support. Despite having fewer samples in some categories, the fine-tuned model still performs well. This finding suggests that, even in an era dominated by large language models, smaller language models remain valuable, especially for domain-specific tasks that require frequent fine-tuning.

\section{Conclusion}
Our research aims to advance the study of dialogue safety in mental health support by introducing a sequential and inclusive taxonomy that prioritizes the positive impact on help-seekers, grounded in theoretical and empirical knowledge. To comprehensively analyze dialogue safety, we develop a Chinese online text-based free counseling platform to collect real-life counseling conversations. Utilizing the open-source Chinese dialogue model, EVA2.0-xLarge, we fine-tune the model and meticulously annotate dialogue safety based on our proposed taxonomy. In our investigation to detect dialogue safety within a dialogue session, we employ ChatGPT using a zero- and few-shot paradigm, along with detailed safety definitions. Additionally, we fine-tune popular pre-trained models like BERT-base and RoBERTa-large to detect unsafe responses. Our findings demonstrate that ChatGPT is less effective than BERT-base and RoBERTa-large in detecting dialogue safety in mental health support. The fine-tuned model proves to be more suitable for identifying unsafe responses. Our research underscores the significance of our dataset as a valuable benchmark for understanding and monitoring dialogue safety in mental health support.

\section*{Ethics Statement}
The study is granted ethics approval from the Institutional Ethics Committee. Access to our dataset is restricted to researchers who agree to comply with ethical guidelines and sign a confidentiality agreement with us.

%% 
%% Copyright 2007-2020 Elsevier Ltd
%% 
%% This file is part of the 'Elsarticle Bundle'.
%% ---------------------------------------------
%% 
%% It may be distributed under the conditions of the LaTeX Project Public
%% License, either version 1.2 of this license or (at your option) any
%% later version.  The latest version of this license is in
%%    http://www.latex-project.org/lppl.txt
%% and version 1.2 or later is part of all distributions of LaTeX
%% version 1999/12/01 or later.
%% 
%% The list of all files belonging to the 'Elsarticle Bundle' is
%% given in the file `manifest.txt'.
%% 
%% Template article for Elsevier's document class `elsarticle'
%% with harvard style bibliographic references

\documentclass[preprint,12pt]{elsarticle}
%\documentclass[5p,times]{elsarticle}

\makeatletter
\def\ps@pprintTitle{%
 \let\@oddhead\@empty
 \let\@evenhead\@empty
 \def\@oddfoot{}%
 \let\@evenfoot\@oddfoot}
\makeatother

%% Use the option review to obtain double line spacing
%% \documentclass[preprint,review,12pt]{elsarticle}

%% Use the options 1p,twocolumn; 3p; 3p,twocolumn; 5p; or 5p,twocolumn
%% for a journal layout:
%% \documentclass[final,1p,times]{elsarticle}
%% \documentclass[final,1p,times,twocolumn]{elsarticle}
%% \documentclass[final,3p,times]{elsarticle}
%% \documentclass[final,3p,times,twocolumn]{elsarticle}
%% \documentclass[final,5p,times]{elsarticle}
%\documentclass[final,5p,times,twocolumn]{elsarticle}

%% For including figures, graphicx.sty has been loaded in
%% elsarticle.cls. If you prefer to use the old commands
%% please give \usepackage{epsfig}

%% The amssymb package provides various useful mathematical symbols
\usepackage{amssymb}
%% The amsthm package provides extended theorem environments
%% \usepackage{amsthm}

%% The lineno packages adds line numbers. Start line numbering with
%% \begin{linenumbers}, end it with \end{linenumbers}. Or switch it on
%% for the whole article with \linenumbers.
%% \usepackage{lineno}

\usepackage{float}
\usepackage{subfigure}
\usepackage{amsmath}
%\usepackage{subcaption}
\usepackage{afterpage}
\usepackage{longtable}

\usepackage{graphicx,subfigmat,etoolbox,amssymb,float}

\usepackage{multirow}
\usepackage{booktabs}

\usepackage{url}
\usepackage{hyperref}

\newcounter{subfigcount}
\setcounter{subfigcount}{0}

%\journal{Future Generation Computer Systems}

\begin{document}

\begin{frontmatter}

%% Title, authors and addresses

%% use the tnoteref command within \title for footnotes;
%% use the tnotetext command for theassociated footnote;
%% use the fnref command within \author or \address for footnotes;
%% use the fntext command for theassociated footnote;
%% use the corref command within \author for corresponding author footnotes;
%% use the cortext command for theassociated footnote;
%% use the ead command for the email address,
%% and the form \ead[url] for the home page:
%% \title{Title\tnoteref{label1}}
%% \tnotetext[label1]{}
%% \author{Name\corref{cor1}\fnref{label2}}
%% \ead{email address}
%% \ead[url]{home page}
%% \fntext[label2]{}
%% \cortext[cor1]{}
%% \affiliation{organization={},
%%             addressline={},
%%             city={},
%%             postcode={},
%%             state={},
%%             country={}}
%% \fntext[label3]{}

\title{Noise Reduction in Wind Turbine Airfoils with Serrated Trailing Edges: An Experimental Study in Low Turbulence Wind Tunnel}

%% use optional labels to link authors explicitly to addresses:
 \author[label1,label3]{Weicheng Xue}
 \affiliation[label1]{organization={Pengcheng Laboratory},
             city={Shenzhen, Guangdong},
             postcode={518055},
             country={China}}

 \author[label4]{Hongyu Wang}
 \affiliation[label4]{organization={Internet Based Engineering},
             city={Beijing},
             postcode={100083},
             country={China}}

 \author[label3]{Zhe Chen}          

 \author[label2,label3]{Bing Yang*\footnote{*Bing Yang, corresponding author, bingyang1215@126.com}}
  \affiliation[label2]{organization={Goldwind Sci \& Tech Co.,Ltd.},
             city={Beijing},
             postcode={101102},
             country={China}} 
 \affiliation[label3]{organization={Institute of Engineering Thermophysics, Chinese Academy of Sciences},
             city={Beijing},
             postcode={100083},
             country={China}}             

%% use optional labels to link authors explicitly to addresses:
%% \author[label1,label2]{}
%% \affiliation[label1]{organization={},
%%             addressline={},
%%             city={},
%%             postcode={},
%%             state={},
%%             country={}}
%%
%% \affiliation[label2]{organization={},
%%             addressline={},
%%             city={},
%%             postcode={},
%%             state={},
%%             country={}}

\begin{abstract}
%% Text of abstract
This study explores the noise reduction achieved by airfoils with serrated trailing edges in a low turbulence wind tunnel, focusing on acoustic spectral characteristics and wake flow field measurements. We analyze the effects of various factors, including Reynolds number, angle of attack, serration parameters, and model type, on sound power levels and far-field radiation patterns. Our findings reveal that serrated trailing edges significantly reduce noise across a broader frequency range than previously documented, particularly in the mid-to-high frequency range, with reductions bounded by Strouhal numbers $St_u = 1$ and $St_l = 0.48$. Interestingly, the serration geometry exhibits minimal impact on noise reduction, which varies with the angle of attack and airfoil profile across all tested conditions. Additionally, while serrations effectively lower noise levels, especially at higher frequencies, they do not significantly alter the airfoil's acoustic directivity patterns. Measurements of wake flow velocity spectra demonstrate a clear correlation between reduced wake turbulence and noise reduction, as serrated edges decrease the power spectral density of turbulent velocity fluctuations, effectively disrupting larger vortex structures responsible for noise generation. These valuable insights contribute to understanding the aerodynamic and acoustic benefits of serrated trailing edges, warranting further experimental validation in future studies.
\end{abstract}

%%Graphical abstract
%\begin{graphicalabstract}
%% Figure removed
%\end{graphicalabstract}

%%Research highlights
\begin{highlights}
\item This study demonstrates that serrated trailing edges reduce noise effectively across a wider frequency range than previously reported, with significant reductions specifically confined by $St_u = 1$ and $St_l = 0.48$.
\item The effectiveness of serrated trailing edges in reducing noise varies with the angle of attack and airfoil profile, resulting in distinct noise reduction patterns in both frequency range and amplitude.
\item The geometry of the serrations was found to have a negligible effect on noise reduction.
\item Measurements reveal a clear correlation between reduced wake turbulence and noise reduction, indicating that serrated edges disrupt larger vortex structures responsible for generating noise.
\end{highlights}

\begin{keyword}
%% keywords here, in the form: keyword \sep keyword
wind turbine airfoil \sep serrations \sep noise reduction \sep aerodynamic noise \sep sound pressure level
%% PACS codes here, in the form: \PACS code \sep code

%% MSC codes here, in the form: \MSC code \sep code
%% or \MSC[2008] code \sep code (2000 is the default)

\end{keyword}

\end{frontmatter}

%% \linenumbers

%% main text
\section{Introduction}

In the pursuit of enhancing wind turbine efficiency and productivity, there has been a noticeable trend towards increasing the size of wind turbines and extending blade lengths~\cite{kaewniam2022recent,hoen2023effects}. As a result, the tip speed ratio of wind turbine rotors has been steadily rising~\cite{kosasih2016influence}. However, aerodynamic noise generated by wind turbine blades increases with the fifth power of wind speed~\cite{SORENSEN2012225}. While higher tip speeds can improve wind turbine performance, they also lead to increased noise emissions. For instance, a 15\% increase in the tip speed of wind turbine blades can result in a 3 dB rise in noise, which may adversely affect the living conditions of nearby residents~\cite{barone2011survey}.

The primary sources of noise in wind turbines are aerodynamic noise from the blades and mechanical noise from the nacelle~\cite{deshmukh2019wind}. Mechanical noise, stemming from the operation of mechanical components, is characterized by tonal qualities~\cite{pinder1992mechanical}. Conversely, aerodynamic noise arises from the interaction between air and the wind turbine blade and can be categorized into inflow turbulence noise and self-noise~\cite{oerlemans2011wind}. Within self-noise, turbulent and laminar boundary layer trailing edge noise are significant due to their origin from boundary layer interaction with the sharp trailing edge~\cite{brooks1989airfoil}. Stall separation noise and blunt trailing edge vortex shedding noise are less influential because wind turbines usually operate below stall conditions, and sharp trailing edges can mitigate vortex shedding noise.

Effective reduction of trailing edge noise is essential for optimizing wind turbine blade design. Several trailing edge modifications have been proposed to reduce noise while maintaining aerodynamic performance. These include brush trailing edges~\cite{herr2005experimental,herr2007design,finez2010broadband,suryadi2023identifying}, perforated trailing edges~\cite{suryadi2023identifying,geyer2010measurement,zhang2020experimental}, and serrated trailing edges~\cite{oerlemans2009reduction,moreau2011flat,moreau2013noise,avallone2018noise,celik2021aeroacoustic,pereira2022aeroacoustics}. Among these, serrated trailing edges have become particularly prominent due to their ability to effectively disrupt vortex formation and reduce noise levels, as well as their adaptable design options.

The efficacy of serrated trailing edges in mitigating aerodynamic noise has been investigated~\cite{gruber2010experimental, gruber2011mechanisms, oerlemans2009reduction}. Gruber et al.\cite{gruber2010experimental, gruber2011mechanisms} reported noise reductions of up to 5 dB across a broad frequency spectrum, while Oerlemans et al.\cite{oerlemans2009reduction} conducted experiments on a full-scale 2.3 MW wind turbine, observing reductions of up to 3.2 dB below 1000 Hz. These studies demonstrate that serrated trailing edges effectively reduce wind turbine noise, particularly at low to mid frequencies. However, their effectiveness diminishes at higher frequencies, where they can occasionally increase noise above critical levels, underscoring the need for further investigation.

Celik et al.~\cite{celik2021aeroacoustic} conducted experiments on a flat plate with a serrated trailing edge, finding that larger serrations significantly reduce noise. They noted that low-frequency analysis may need high-fidelity simulations for accuracy. However, discrepancies between current numerical and experimental results highlight the need for further investigation and a deeper understanding of the underlying noise reduction mechanisms.

Zhou et al.~\cite{zhou2020study} investigated flexible trailing edge serrations on airfoils, finding that they achieve greater high-frequency noise reduction compared to rigid serrations. Additionally, they observed that flexible serrations align better with the flow, reducing crossflow intensity near the serration roots. However, selecting the appropriate flexibility depends on specific working conditions to ensure good stability of the serrations.

The geometry of serrated trailing edges plays a critical role in their noise reduction capabilities. Dassen et al.\cite{dassen1996results} found that serrated trailing edges on flat plates were more effective at noise reduction than those on airfoils. Finez et al.\cite{finez2011broadband} explored various geometric parameters of serrated trailing edges on a cascade of seven NACA 651210 airfoils, discovering no significant impact of the cascade effect on noise reduction in the low to mid-frequency range. Singh et al.\cite{singh2023control} examined non-uniform sinusoidal serrations, revealing that increased spanwise decoherence and vortex pairing resulted in greater noise reduction compared to uniform serrations. Gruber et al.\cite{gruber2011mechanisms} proposed design guidelines, emphasizing key parameters that influence noise reduction efficiency.

Significant advancements have been achieved in reducing tonal noise using serrated and porous trailing edges. Chong et al.\cite{chong2013experimental} investigated the effects of serrated trailing edges on unsteady pure tone noise, particularly focusing on Tollmien-Schlichting waves and separation bubbles on a NACA 0012 airfoil at moderate Reynolds numbers ($10^5 < Re_c < 10^6$). They found that serrations reduced instability noise by modifying the length scales of separation bubbles and suppressing their amplification. Additionally, the three-dimensional flow induced by serrations weakened normal fluctuations and adverse pressure gradients, further contributing to noise reduction. Subsequent studies\cite{chong2013airfoil, chong2011noise} demonstrated that non-planar serrated trailing edges accelerated the laminar-to-turbulent transition, further reducing noise levels. The trailing edge noise followed a power law of velocity, underscoring its primary role as a noise source. Zhang and Chong~\cite{zhang2020experimental} also examined the impact of porous parameters on tonal noise, identifying optimal settings for broadband noise reduction.

Despite significant advancements, the fundamental mechanisms behind noise reduction through modified trailing edges remain unclear. This study aims to enhance understanding by investigating the noise reduction mechanisms of wind turbine blade airfoils with serrated trailing edges. Experiments were conducted in an anechoic low turbulence wind tunnel using various airfoil models, including two wind turbine blade airfoils. The findings from this study will aid in optimizing trailing edge designs for noise reduction and developing low-noise wind turbine blades.


\section{Experimental Apparatus and Procedure}

\subsection{Anechoic wind tunnel}

The experiments were conducted in a small open-circuit anechoic low turbulence wind tunnel located at the Fluid and Acoustic Engineering Laboratory, Beihang University, as depicted in Fig.~\ref{tunnel}. The anechoic chamber has internal dimensions of $8.9,\text{m} \times 6.8,\text{m} \times 4.65,\text{m}$, with a cutoff frequency of 200 Hz, ensuring minimal acoustic reflections during measurements. The wind tunnel, which extends from outside the chamber into its interior, features a circular outlet with a diameter of 150 mm and can achieve a maximum wind speed of 50 m/s. Table\ref{velocity_turbulence} provides the turbulence characteristics measured using a hot-wire anemometer at the center of the outlet, indicating a low turbulence intensity, which is critical for ensuring accurate flow and noise measurements. The experiments were carried out at wind speeds of 15 m/s, 25 m/s, and 35 m/s, with both noise and flow field measurements performed at each speed.

% Figure environment removed

\begin{table}[H]
	\caption{Measured wind velocity and corresponding turbulence intensity at the wind tunnel outlet}
	\centering
	\begin{tabular}{ccccc}
		\hline
		Wind Velocity, m/s& 15.19& 25.12& 35.45& 45.69\\
		Turbulence intensity, \%& 0.036& 0.080& 0.101& 0.144\\
		\hline
	\end{tabular}
	\label{velocity_turbulence}
\end{table}

In Table~\ref{velocity_turbulence}, the measured turbulence intensity values demonstrate that the undisturbed flow remains well-controlled, even at higher wind speeds, with minimal turbulence interference. These low inflow turbulence levels are crucial to the experimental design, as the study specifically focuses on noise reduction in low inflow turbulence conditions.

Fig.~\ref{background_airfoil} compares the sound pressure level of the sound pressure signal, denoted as \( L_p \), in the anechoic chamber with and without a NACA 634421 airfoil at wind speeds of 25.12 m/s and 35.45 m/s. The sound pressure level \( L_p(f) \) at a frequency \( f \) is derived from the sound pressure signal captured by a microphone and is calculated using the formula shown in Eq.~\ref{eq:lp}:

\begin{equation}
L_p(f) = 20 \log_{10}\left(\frac{|P(f)|}{p_0}\right)
\label{eq:lp}
\end{equation}
where \( P(f) \) is the frequency domain representation of the sound pressure signal, obtained by applying the Fast Fourier Transform (FFT) to the time-domain signal \( p(t) \), measured by the microphone. Specifically, \( P(f) = \text{FFT}(p(t)) \), and the magnitude \( |P(f)| \) represents the amplitude of the sound pressure at frequency \( f \). The reference sound pressure \( p_0 \) is typically set to \( 20 \ \mu\text{Pa} \). The sound pressure level \( L_p(f) \), expressed in decibels (dB), is then calculated by comparing \( |P(f)| \) to \( p_0 \), using a logarithmic scale to reflect the human auditory system's sensitivity to sound.

The data acquisition and calculation of \( L_p(f) \) in this formula were efficiently performed using a custom LabVIEW program, which can be found at \href{https://doi.org/10.5281/zenodo.13768804}{https://doi.org/10.5281/zenodo.13768804}. Interested readers can refer to this program for more details.

% Figure environment removed

Below 400 Hz, the presence of the airfoil has a negligible effect on noise levels. However, within the 400 to 4000 Hz range, the airfoil induces a significant noise increase, exceeding 10 dB at both tested wind speeds. In the 4000 to 8000 Hz range, the noise rise is more moderate, between 1 and 3 dB, while from 8000 to 14,000 Hz, the increase is approximately 5 to 6 dB. These findings indicate that the airfoil's influence on noise is more pronounced at mid to high frequencies, and the overall noise level increases with wind speed. The maximum difference observed was 13.8 dB at the higher wind speed. The airfoil-generated noise exhibited a unique pattern, characterized by multiple narrowband tonal components superimposed on a broadband noise spectrum. It is crucial to ensure that serrated trailing edges are effective in reducing both broadband noise and narrowband tonal components, and to further investigate the mechanisms behind this noise mitigation.

\subsection{Experimental Airfoils}

The experimental setup featured two wind turbine airfoil models, NACA 633418 and NACA 634421, each with a uniform chord length of 74 mm and a span of 160 mm. These models were precisely assembled from two halves, with flat plates welded to the top and bottom of the wind tunnel outlet to secure the airfoils in place. The trailing edges, designed with a blunt profile and a thickness of 1 mm, included a 0.6 mm slot to allow for the attachment of serrations of various sizes. This design enabled a detailed investigation into noise reduction techniques using different serrated trailing edges.%, while also accounting for the thickness requirements in assembly. %The visual representation of these configurations is depicted in Fig.~\ref{airfoil_plate}.

%% Figure environment removed

%% Figure environment removed

\subsection{Serrated Trailing Edges}

Various serration half-heights ($h$) and wavelength-to-half-height ratios ($\lambda/h$) were tested, with all serrated trailing edges uniformly designed to have a thickness of 0.6 mm for easy attachment to the airfoil. The models labeled with "S" correspond to serrated trailing edges, while those labeled with "F" represent straight bars at half the serration height. The model labeled "0-0" is the baseline configuration, which features the original airfoil without any modifications to the trailing edge. Detailed parameters are listed in Table~\ref{serration_bar_parameters}.

\begin{table}[H]
    \centering
    \renewcommand{\arraystretch}{1.2}
    \setlength{\tabcolsep}{10pt}
    \caption{Serration and Straight Bar Parameters}
    \begin{tabular}{cccc}
        \toprule
        \textbf{Half-height, $h$ (mm)} & \textbf{Wavelength, $\lambda$ (mm)} & \textbf{$\lambda/h$} & \textbf{Model} \\
        \midrule
        \multirow{3}{*}{4} & 3.2 & 0.8 & 4-1S \\
                           & 1.6 & 0.4 & 4-2S \\
                           & -   & -   & 4-0F \\
        \midrule
        \multirow{4}{*}{5} & 4.0 & 0.8 & 5-1S \\
                           & 2.0 & 0.4 & 5-2S \\
                           & 1.0 & 0.2 & 5-3S \\
                           & -   & -   & 5-0F \\
        \midrule
        \multirow{4}{*}{6} & 4.8 & 0.8 & 6-1S \\
                           & 2.4 & 0.4 & 6-2S \\
                           & 1.2 & 0.2 & 6-3S \\
                           & -   & -   & 6-0F \\
        \midrule
        0 & 0 & 0 & 0-0 \\
        \bottomrule
    \end{tabular}
    \label{serration_bar_parameters}
\end{table}


\subsection{Data acquisition and processing system}

The noise measurement setup utilized BSWA MPA 416 pressure transducer microphones (1/4-inch radius, 20 to 20,000 Hz frequency range). Data acquisition was performed with a National Instruments (NI) system, comprising a PXIe-1071 chassis, a PXIe-8102 controller, and multiple PXIe-4496 data acquisition cards. Each PXIe-4496 card supported synchronous sampling across 16 channels with a rate of 204.8 kS/s, ensuring compliance with the frequency requirements for accurate noise measurement. The microphones and data acquisition instrument is shown in Fig.~\ref{digital_acquisition}.

% Figure environment removed

The sound signals from the microphones were processed and stored in \emph{.tdms} format using a LabVIEW-based data acquisition program. For analysis, these \emph{.tdms} files were imported into a LabVIEW-based FFT (Fast Fourier Transform) analysis tool, which allowed for parameter customization such as the starting point of the time segment, number of segments, sliding window, and reference channels. The data acquisition sampling rate was set at 65536 samples/s, with a time segment length of 4096 samples and a 10-second recording duration per data acquisition session. A spectral overlap rate of 50\% was used to ensure smooth spectrum calculation and effective averaging. The FFT analysis interface is shown in Fig.~\ref{fft_analysis}. The LabVIEW program used for data acquisition and FFT analysis is available at \href{https://doi.org/10.5281/zenodo.13768804}{https://doi.org/10.5281/zenodo.13768804}.

% Figure environment removed

Noise analysis is usually conducted within specific frequency bands, either as doubling frequency bands with a 2:1 ratio between cutoff frequencies or as 1/3-octave bands, which divide these bands into three equal parts. This study adopts the 1/3-octave band approach. For a band centered at frequency $f$, the upper and lower cutoff frequencies are \(\sqrt[6]{2} f\) and \( f/\sqrt[6]{2} \), respectively. This method enhances noise data precision. A high-pass filter with a cutoff frequency of 100 Hz is used, and the analysis spans 17 frequency bands, starting with a center frequency of 315 Hz.

The 1/3-octave band sound pressure level, \( L_p \), is computed by summing the sound pressure levels within each frequency band. The energy summation in decibels (dB) for each 1/3-octave band is performed using Eq.~\ref{eq:third_octave_lp}:

\begin{equation}
L_p = 10 \log_{10} \left( \sum_{f_l \leq f_i \leq f_u} 10^{L_p(f_i) / 10} \right)
\label{eq:third_octave_lp}
\end{equation}

In Eq.~\ref{eq:third_octave_lp}, \( L_p(f_i) \) represents the sound pressure level at frequency \( f_i \), and the summation is performed over all frequencies within the lower and upper cutoff frequencies, \( f_l \) and \( f_u \), of the 1/3-octave band. The exponential term converts the sound pressure level from decibels back to a linear scale, and the logarithmic operation returns the summed energy back into decibels. This method provides an accurate representation of the total energy within the 1/3-octave band.


%, as shown in Table~\ref{band_center}.

%\begin{table}[H]
	%\caption{1/3-octave band center frequency}
	%\centering
	%\begin{tabular}{cccccccccccccccccc}
	%	\hline
	%	Number& 1& 2& 3& 4& 5& 6& 7& 8& 9& 10& 11& 12& 13& 14& 15& 16& 17\\
	%	Center frequency, Hz& 315& 400& 500& 630& 800& 1000& 1250& 1600& 2000& 2500& 3150& 4000& 5000& 6300& 8000& 10000& 12500\\
	%	\hline
	%\end{tabular}
	%\label{band_center}
%\end{table}

The microphone arrangement in the experiment is shown in Fig.~\ref{directivity_microphones}. The angle between the line connecting the microphones and the blade airfoil trailing edge, and the direction of the airflow, is denoted as $\phi$. Due to spatial constraints and equipment limitations within the wind tunnel, three microphones are positioned at a radius of 2.35 m from the model's trailing edge, corresponding to $\phi$ angles of 45\textdegree, 60\textdegree, and 75\textdegree, respectively. The placement of the microphones was further influenced by the insufficient distance between the wind tunnel outlet and the wall, making it impractical to position a microphone at the 90\textdegree \ angle due to its proximity to the wall. Consequently, in this experiment, when directivity is not a primary concern, the microphone located at the 75\textdegree \ position is used for acoustic data acquisition and subsequent processing. No corrections have been made to the sound pressure level equations.

% Figure environment removed


\section{Noise Measurement}

\subsection{Impact of Reynolds Number}

The sound power levels for the NACA 633418 and NACA 634421 airfoils were measured at Reynolds numbers of $0.7 \times 10^5$, $1.2 \times 10^5$, and $1.6 \times 10^5$. Fig.~\ref{Re_633418} and Fig.~\ref{Re_634421} show the 1/3-octave band sound power levels for the NACA 633418 and NACA 634421 airfoils, including configurations with trailing edge serrations, original airfoils, and those with added bars. Significant variations in the noise reduction performance of serrated trailing edges under different wind speed conditions can be observed. To better understand noise reduction across different frequencies and flow velocities, the noise frequency $f$ was converted into the Strouhal number $S_t$, defined as:

\begin{equation}
\label{Strouhal}
S_t = \frac{f \delta}{U}
\end{equation}
where $f$ denotes the sound frequency, $\delta$ represents the thickness of the boundary layer (assumed to be 3 mm in this study), and $U$ is the inflow wind speed.

At lower Reynolds numbers, the NACA 633418 and NACA 634421 airfoils show primary noise reductions of 2$\sim$7 dB in the 630$\sim$1600 Hz range ($0.12 \leq S_t \leq 0.32$). At moderate Reynolds numbers, the serrated trailing edge reduces noise by 2$\sim$5 dB in the 800$\sim$3150 Hz ($0.096 \leq S_t \leq 0.38$) and 10000$\sim$12500 Hz ($1.2 \leq S_t \leq 1.49$) ranges for both airfoil models. At higher Reynolds numbers, the serrated trailing edge is especially effective in reducing tonal noise around 3150 Hz ($S_t = 0.27$). Across all Reynolds numbers, airfoils with serrated trailing edges consistently exhibit lower noise levels compared to those with straight bars, demonstrating the superior noise reduction of the serrated design.

% Figure environment removed

% Figure environment removed

The variation of noise reduction frequency ranges with Reynolds number ($Re$) reveals the effectiveness of trailing edge serrations under different flow conditions. Using the NACA 634421 airfoil as a case study, this study examines how the noise reduction performance of serrations changes with varying Reynolds numbers, highlighting the interplay between flow characteristics and aeroacoustic phenomena. Several noteworthy observations can be concluded. Trailing edge serrations consistently demonstrate their capacity to reduce noise levels. Compared to the original airfoil, serrations are effective in reducing noise across both low-to-medium and high frequencies. When compared to an airfoil with straight trailing edges, serrations exhibit greater noise reduction mainly in the mid-to-high frequency range. The influence of serration wavelength ($\lambda$) or width-to-height ratio ($\lambda/h$) on sound power level is minimal, contrary to the observation that larger wavelength serrations provided better noise reduction performance compared to smaller wavelength serrations in Moreau et al.~\cite{moreau2011flat,moreau2013noise} and Celik et al.~\cite{celik2021aeroacoustic}. However, noise reduction tends to improve as the serration wavelength or width-to-height ratio decreases.

%% Figure environment removed

Fig.~\ref{noise_reduction_f_U} shows noise reduction regions as a function of inflow wind speed and frequency, with bounds confined by two critical Strouhal numbers. The contour colors indicate the magnitude of noise reduction.

% Figure environment removed

The plots in Fig.~\ref{noise_reduction_f_U} (a) and (b) reveal distinct patterns of noise reduction. Serrations consistently demonstrate superior noise mitigation compared to airfoils with or without bar trailing edges, making them the preferred choice across various frequency ranges. The noise reduction zones are bounded by Strouhal numbers $St_u = 1$ and $St_l = 0.48$, indicating the effective frequency ranges of the serrations. Additionally, the broader noise reduction zone at higher Reynolds numbers suggests that the effectiveness of serrations increases with Reynolds number.


\subsection{Impact of Angle of Attack}

The angle of attack significantly impacts the performance of airfoil trailing edge modifications. Changes in the angle of attack alter the flow characteristics over the airfoil, affecting noise generation mechanisms. Investigating how trailing edge modifications interact with different angles of attack offers valuable insights into their effectiveness under various operating conditions.

Fig.~\ref{AOA_634421} compares the noise levels between the NACA 634421 airfoil with and without serrated trailing edges under a Reynolds number of \( Re = 1.2 \times 10^5 \), with the 5-2S configuration selected as the experimental group. The airfoil without a trailing edge exhibits noticeable tonal noise, approximately 10 dB, within the dimensionless frequency range of $0.2 > St > 0.1$ at a high angle of attack ($20^\circ$), which is primarily due to flow separation. The serrated trailing edge effectively mitigates this tonal noise. Similarly, the airfoil with a bar trailing edge also generates prominent tonal noise around $St \approx 0.168$ at a high angle of attack ($20^\circ$), which the serrated trailing edge successfully eliminates. Across the four different angles of attack tested, the serrated trailing edge consistently demonstrates significant noise reduction within a certain frequency range, both in comparison to the airfoil without a trailing edge and the one with a bar trailing edge.

% Figure environment removed

Fig.~\ref{noise_alpha_633418} and Fig.~\ref{noise_alpha_634421} show the effect of the angle of attack on noise reduction for the NACA 633418 and NACA 634421 airfoils, respectively. Noise reduction is calculated as the difference in sound power level between the 5-2S serrated trailing edge and the reference cases.

% Figure environment removed

% Figure environment removed

The effectiveness of serrated trailing edges in reducing noise varies with the angle of attack for different airfoil profiles, leading to distinct noise reduction patterns in terms of frequency range and amplitude. For the NACA 633418 airfoil, the serrated trailing edge performs best at angles of $-10^\circ$ and $10^\circ$, achieving noise reductions over 10 dB across a broad frequency range from $St = 0.1$ to $St = 1.0$. This is likely due to the thinner profile of the NACA 633418, which optimizes its aerodynamic performance at moderate angles of attack, effectively reducing vortex shedding and turbulence over a wide frequency range. In contrast, the NACA 634421 airfoil achieves the most significant noise reduction at an angle of $20^\circ$, with reductions exceeding 15 dB. The thicker profile of the NACA 634421 is optimized for higher angles of attack, significantly reducing noise by minimizing flow separation and reattachment regions.

Across different angles of attack, the bar trailing edge consistently shows inferior noise reduction performance compared to the serrated trailing edge for both airfoils tested. The maximum noise reduction difference between the serrated and bar trailing edges can reach up to 5 dB, with the serrated trailing edge being more effective across most frequency ranges.

\subsection{Impact of Serration Height}

Fig.~\ref{633418_13o_h} and Fig.~\ref{634421_13o_h} demonstrate the influence of serration height on noise reduction for the NACA 633418 and NACA 634421 airfoils across different Reynolds numbers. While the effect of serration height on sound power level appears modest at lower Reynolds numbers, its impact becomes more pronounced at higher Reynolds numbers. The limited influence at lower Reynolds numbers may be attributed to the reduced energy in the flow, where the smaller scale of turbulent structures interacting with the serrations results in minimal noise attenuation. As the Reynolds number increases, the flow becomes more turbulent, and the interaction between the serrations and the larger turbulent eddies enhances noise reduction.

The size of the serrations plays a role in this process, though the effects were not clearly observed in our experiments. Larger serrations are expected to interact more effectively with dominant turbulent structures in the flow field, potentially disrupting coherent vortices that contribute to noise generation. This interaction could also influence aerodynamic performance by modifying the wake structure and possibly reducing drag. While longer serrations generally provide better noise reduction, consistent with Howe's findings \cite{howe1991aerodynamic,howe1991noise} that greater serration height enhances noise reduction, it is important to recognize that Howe's theory is based on idealized assumptions. His theory primarily links noise reduction to the rate of vorticity intersecting streamlines, but it does not fully account for the complex vortex dynamics near the airfoil trailing edge, where intricate interactions between the flow and sound fields, influenced by serration size, can affect both noise characteristics and aerodynamic performance.

% Figure environment removed

% Figure environment removed

    
\subsection{Sound Field Directivity}  

The sound pressure levels shown in Fig.~\ref{633418_1500_directivity} represent the original intensities received by the microphones at their respective positions. Fig.~\ref{633418_1500_directivity} shows the sound directivity pattern for the NACA 633418 airfoil at a Reynolds number of $1.2 \times 10^5$. Several key observations about noise propagation characteristics and the effectiveness of different trailing edge modifications can be revealed. At low to medium frequencies, around 400 Hz, both straight-edge and serrated trailing edges show minimal impact on noise reduction and do not significantly alter the acoustic directivity patterns of the airfoil noise. As the frequency increases to 1000 Hz and beyond, both straight-edge and serrated trailing edges demonstrate a noticeable reduction in sound intensity. However, these modifications do not change the inherent acoustic directivity characteristics of the airfoil. For instance, at 1000 Hz, the highest noise intensity is observed in directions approximately perpendicular to the airfoil surface, and this pattern remains unchanged with trailing edge modifications. Serrated trailing edges notably outperform straight-edge modifications in noise reduction, particularly at frequencies above 3150 Hz. The serrated configurations consistently show lower noise levels compared to straight edges, indicating their superior efficacy in mitigating noise at higher frequencies. This observation is consistent with the findings of Lyu et al.~\cite{lyu2016prediction}, which emphasized that primary noise reduction arises from interference effects near the trailing edge and highlighted substantial alterations in directivity characteristics at high frequencies. However, Tian et al.~\cite{tian2022prediction} demonstrated that trailing edge serrations can modify directivity characteristics.

The directivity results for the NACA 634421 airfoil exhibit similar trends and are not explicitly presented here, as the overall patterns and observations closely align with those of the analyzed configurations. Overall, the findings suggest that while trailing edge modifications can significantly reduce noise levels, especially at higher frequencies, they do not fundamentally alter the directivity characteristics of the airfoil's acoustic emissions.

%% Figure environment removed

% Figure environment removed

%% Figure environment removed



%Additionally, Fig.~\ref{plate_1500_directivity} illustrates the sound directivity patterns of a flat plate at Reynolds number $Re=1.2\times10^5$ for 1/3 octave band frequency center points of 400 Hz, 1000 Hz, and 3150 Hz. The directivity pattern for the flat plate is similar to the airfoil models at 400 Hz, with a relatively uniform distribution across various angles. At 1000 Hz, the directivity variation with angle $\phi$ is less pronounced than observed in the airfoil models. However, at 3150 Hz, the flat plate's directivity pattern differs, with the strongest sound radiation occurring at an angle of 45\textdegree.

%% Figure environment removed

\section{Wake Flow Measurement}

\subsection{Velocity Measurement Setup}

Fig.~\ref{hot_wire} shows the spatial arrangement of components within the hot wire measurement system, using Dantec Dynamics A/S products. The experiment utilized the 55P13 thermal wire probe, featuring a 90° bend, a hot wire diameter of 5 µm, and a length of 1.25 mm. This design minimizes interference with the flow field and has a maximum response frequency of 90 kHz. The three-dimensional displacement device employs a helical micro-displacement tool, providing high-resolution measurements with a precision of up to 0.05 mm.

% Figure environment removed

Velocity measurements were conducted on the wake flow of three different experimental models: the original airfoil (0-0 model), the airfoil with a straight edge (5-0F model), and three airfoil models with serrated edges (5-XS, where X is 1, 2, or 3). The arrangement and numbering of measurement points are depicted in Fig.~\ref{measurement_locations}. For instance, measurement point "12" is located 2 mm downstream of the trailing edge in the pressure surface direction, with a 2 mm offset. Other measurement points are similarly referenced using Fig.~\ref{measurement_locations}.

% Figure environment removed

\subsection{Flow Field Measurement Results}

Fig.~\ref{634421_velocity} shows the mean velocity and fluctuation velocity measurements for the NACA 634421 airfoil with various trailing edge configurations at a Reynolds number of $1.2 \times 10^5$. The mean and fluctuation velocities at distances of 2 mm and 7 mm from the trailing edge apex are shown in Fig.~\ref{634421_average_velocity} and Fig.~\ref{634421_fluctuation_velocity}, respectively. The horizontal axis indicates the offset distance from the trailing edge apex, as depicted in Fig.~\ref{measurement_locations}. The original and straight-edge airfoils exhibit the lowest mean velocities, while the serrated airfoils, particularly with the largest serration, show higher mean velocities. Fluctuation velocity generally decreases with distance from the trailing edge, and serrations effectively reduce fluctuation intensity at multiple points. These results indicate that serrated trailing edges enhance streamwise turbulence reduction.

The observed correlation between reduced wake fluctuating velocities and noise reduction may be explained by the interaction between the serrations and the trailing edge flow. Serrations likely disrupt coherent vortex structures in the wake, breaking them into smaller, less energetic eddies that generate less noise. This disruption reduces the strength of the velocity fluctuations in the wake, which is directly linked to the sound generated by the trailing edge. The reduction in fluctuating velocities near the trailing edge, therefore, leads to a decrease in noise levels.

% Figure environment removed

Fig.~\ref{634421_1500_turbulence_spectrum} displays the power spectral density (PSD) of turbulent velocity fluctuations at measurement points 10 and 20 for $Re=1.2\times10^5$. These points, located downstream of the trailing edge, are highly influenced by the flow field. The original airfoil exhibits the highest PSD in the mid-low frequency range (300$\sim$4000 Hz), with differences up to 4 dB, correlating with noise reduction ranges. This indicates that reducing streamwise fluctuations near the trailing edge contributes to noise reduction. Serrations disrupt larger vortex structures, with finer serrations proving more effective. High-frequency turbulence remains unaffected, following a -2.5 power law, consistent with Ref \cite{gruber2010experimental}.

% Figure environment removed

While these results suggest a causal relationship between reduced wake fluctuations and noise reduction, the empirical data primarily indicate a correlation. Further experimentation, particularly involving direct measurements of the vortex dynamics and their contribution to noise generation, would be beneficial to validate this connection and fully understand the underlying mechanisms.

Fig.~\ref{634421_1500_2mm_autocorelation} shows the autocorrelation maps of turbulent velocity fluctuations for the original airfoil (0-0) and the 5-2S serrated trailing edge at 2 mm downstream, for $Re=1.2\times10^5$. The autocorrelation map for the 0-0 model reveals a broader high-value region and a narrower mid-low-value region, indicating more persistent turbulence structures compared to the 5-2S model. This observation suggests that serrations effectively reduce the correlation of flow near the sound source, thereby diminishing the efficiency of the sound source and contributing to noise reduction.

% Figure environment removed

%% Figure environment removed

The characteristic eddy scale, \( L_u(x) \), is estimated using the autocorrelation function of the fluctuating velocity, denoted as \( R_{uu}(x, \tau) \). The autocorrelation function at location \( x \) is defined as:

\begin{equation}
R_{uu}(x, \tau) = \langle u'(x, t) u'(x, t + \tau) \rangle
\label{eq:autocorr}
\end{equation}
where \( u'(x, t) \) represents the fluctuating velocity component at time \( t \) and position \( x \), and \( \tau \) is the time lag. We disregard the starting time, thus setting \( t = 0\). The characteristic eddy scale is determined by finding the time lag \( \tau_0 \) at which the autocorrelation function decays to half of its maximum value, as shown in Eq.~\eqref{eq:half_autocorr}:

\begin{equation}
\frac{R_{uu}(x, \tau_0)}{R_{uu}(x, 0)} = 0.5
\label{eq:half_autocorr}
\end{equation}
The eddy scale is then given by Eq.~\eqref{eq:eddy_scale}:

\begin{equation}
L_u(x) = U_c \tau_0
\label{eq:eddy_scale}
\end{equation}
where \( U_c \) is the convection velocity of the eddies, typically approximated as \( U_c = 0.7 U_0 \), with \( U_0 \) representing the freestream velocity. Therefore, at a given Reynolds number (\( Re \)), the eddy scale \( L_u(x) \) depends on the position \( x \) and the corresponding time lag \( \tau_0 \).

To facilitate a comprehensive comparison of the autocorrelation functions and characteristic eddy scales across various trailing edge configurations, these metrics are presented together in a single graph, as shown in Fig.~\ref{634421_1500_turbulent_length}. The labels $l_{c1}$ indicate the characteristic eddy scales for each type of trailing edge.

Fig.~\ref{634421_1500_turbulent_length} illustrates that the largest turbulent length scales are associated with the 5-0F and 0-0 trailing edges. In contrast, serrated trailing edges, especially those with larger serrations, exhibit reduced turbulent length scales. This reduction in the characteristic eddy scale indicates that serrations effectively disrupt larger vortex structures, leading to a decrease in aeroacoustic noise intensity. Moreover, finer and longer serrations have a more pronounced impact on altering the vortex scale in the flow field, further enhancing their noise reduction capability. These measurements indicate that the primary reason for noise reduction by serrated trailing edges may be the alteration of the flow field near the source location, rather than changes in the radiation efficiency of the source~\cite{jones2010numerical, sandberg2010reprint, lyu2016prediction, mayer2019semi, ayton2018analytic, gelot2020effect}. Another possible noise reduction mechanism is destructive interference of the scattered surface pressure~\cite{lyu2016prediction}, but cannot be validated in this study. There are significant observed discrepancies between numerical and experimental results, underscoring the need for further investigation and a deeper understanding of the underlying mechanisms involved in the noise reduction process.

% Figure environment removed

%Fig.~\ref{634421_velocity_2000} illustrates the average velocity and fluctuating velocity at various measurement points for $Re=1.6\times10^5$. Interestingly, at this higher Reynolds number, the variation trend of average velocity across different points remains consistent with that observed at $Re=1.2\times10^5$. However, a notable difference is observed in the fluctuating velocity, particularly at $D = 2$ mm. This indicates that under these conditions, the serrations have a lesser impact on the flow outside of $D = 2$ mm, primarily influencing the flow near the trailing edge rather than the entire flow field.

%% Figure environment removed

%The autocorrelation curves at measurement point "10" for different types of serrated trailing edges at $Re=1.6\times10^5$ are depicted in Fig.~ \ref{634421_2000_turbulent_length}. The parameter $l_{c2}$ represents the characteristic vortex scale. Intriguingly, at $Re=1.6\times10^5$, serrations lead to an unexpected increase in the characteristic vortex scale. Moreover, among the various serration configurations, shorter and wider serrations exhibit a more pronounced ability to alter the flow field vortex scale. To comprehensively elucidate this phenomenon, further experimental investigations are required. Comparing the characteristic vortex scales at the two Reynolds numbers, it is observed that the scale at $Re=1.6\times10^5$ is approximately 0.7 mm smaller than that at $Re=1.2\times10^5$. Statistically, this suggests a prevalence of larger-scale vortices at lower Reynolds numbers, while smaller-scale vortices become more prominent at higher Reynolds numbers.

%% Figure environment removed

%Fig.~\ref{633418_velocity} highlights the velocity measurements for the NACA 633418 airfoil, which shares some similarities with the NACA 634421 airfoil in terms of the mean velocity variation with distance from the trailing edge. Nonetheless, distinctions between the two airfoils exist. The fluctuation velocities, ranked from largest to smallest, follow the sequence: 0-0 type, 5-0F type, 5-2S type, 5-1S type, and 5-3S type. Notably, the high-frequency fluctuations in the flow field generally adhere to a -2.5 power law with respect to frequency, as illustrated in Fig.~\ref{633418_1500_turbulence_spectrum}.

%% Figure environment removed

%% Figure environment removed


\section{Conclusions}
\label{conclusions}

This study offers a detailed experimental analysis of noise reduction achieved with airfoils featuring serrated trailing edges in a low turbulence wind tunnel. The research is divided into two main segments: acoustic spectral characteristics and wake flow field measurements. The first segment evaluates the influence of several factors on sound power level, including Reynolds number, angle of attack, serration parameters, and model type, along with far-field sound radiation patterns. The second segment investigates the relationship between noise reduction and flow field characteristics, focusing on fluctuating velocities at specific measurement points. This part examines how changes in the wake flow field correlate with variations in noise reduction performance.

This study provides new insights into the effects of serrated trailing edges on noise reduction, particularly across a broader frequency range than previously reported in the literature. While past research primarily associates serrated trailing edges with noise reduction at low to mid frequencies, our findings demonstrate effectiveness across both low-to-mid and high frequencies, with the most pronounced reduction occurring in the mid-to-high frequency range when compared to straight trailing edge airfoils. The noise reduction range is bounded by Strouhal numbers $St_u = 1$ and $St_l = 0.48$. Contrary to common assumptions, the geometry of the serrations does not play a critical role in noise reduction under the experimental conditions in this study. The serrated trailing edges show variable noise reduction effectiveness depending on the angle of attack and airfoil profile, with noise reduction observed across all four tested angles of attack within a specific frequency range. Moreover, while serration modifications substantially reduce noise, particularly at higher frequencies, they do not significantly alter the directivity pattern of the airfoil’s acoustic emissions.

Wake flow velocity spectra measurements reveal a clear correlation between reduced wake turbulence and noise reduction. Serrated trailing edges effectively lower the power spectral density of turbulent velocity fluctuations and reduce the turbulent length scales near the sound source, effectively disrupting larger vortex structures and potentially decreasing the efficiency of the noise-generating mechanisms. While this relationship is promising, further experimental validation is required to strengthen these findings.

Future research should expand upon these findings by integrating simultaneous sound and flow field measurements of serrated trailing edge noise reduction in larger anechoic wind tunnels (diameter $>$ 400 mm) or real-world wind turbine settings. Employing flow visualization techniques with larger models and serrated trailing edges could provide deeper insights into the underlying mechanisms of noise reduction. Additionally, high-fidelity numerical simulations are essential for capturing accurate results, facilitating a more comprehensive understanding of the noise reduction phenomena. Incorporating analytical models may further elucidate the mechanisms involved in noise reduction.

\section*{Data availability}

The data that support the findings of this study are available from the corresponding author upon reasonable request.

%% The Appendices part is started with the command \appendix;
%% appendix sections are then done as normal sections
%% \appendix

%% \section{}
%% \label{}

%% For citations use: 
%%       \citet{<label>} ==> Jones et al. [21]
%%       \citep{<label>} ==> [21]
%%

%% If you have bibdatabase file and want bibtex to generate the
%% bibitems, please use
%%
\bibliographystyle{elsarticle-num-names} 
%%  \bibliography{<your bibdatabase>}

%% else use the following coding to input the bibitems directly in the
%% TeX file.

%\begin{thebibliography}{00}

%% \bibitem[Author(year)]{label}
%% Text of bibliographic item

%\bibitem[ ()]{}

%\end{thebibliography}

\bibliography{mybib}

\end{document}

\endinput
%%
%% End of file `elsarticle-template-num-names.tex'.

\end{document}
