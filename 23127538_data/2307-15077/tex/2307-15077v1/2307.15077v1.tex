\documentclass[11pt, article]{amsart}

%\usepackage {times}
\usepackage{amsmath}
\usepackage{latexsym}
\usepackage[all]{xy}
\usepackage{color}
\newtheorem{teorema}{Theorem}[section]
\newtheorem{definicion}[teorema]{Definition}
\include{amslatex}
\newtheorem{proposicion}[teorema]{Proposition}
\newtheorem{lema}[teorema]{Lemma}
\newtheorem{corolario}[teorema]{Corollary}
\newcommand{\Diff}[0]{{\textup{Diff}}}

\newtheorem{comentario}[teorema]{Remark}

\newtheorem{ejemplo}[teorema]{Example}
\renewcommand{\baselinestretch}{1.10}
\usepackage[margin=1.5in]{geometry}

 \numberwithin{equation}{section}

\begin{document}
\begin{title}[Effect of the maximal proper acceleration in the inertia]
{On the effect of the maximal proper acceleration in the inertia}
\end{title}
\maketitle
\begin{center}
\author{Ricardo Gallego Torrom\'e}\footnote{Email: rigato39@gmail.com}
\end{center}
\begin{center}
\address{Department of Mathematics\\
Faculty of Mathematics, Natural Sciences and Information Technologies\\
University of Primorska, Koper, Slovenia}\\
\&\\
\address{Instituto Universitario de Matem\'atica Pura y Aplicada,\\
Universitat Polit$\grave{\textrm{e}}$cnica de Val$\grave{\textrm{e}}$ncia, Valencia 46022, Spain}
\end{center}
\begin{center}
\author{Pedro Fern\'andez de C\'ordoba}\footnote{Email: pfernandez@mat.upv.es}
\end{center}
\begin{center}
\address{Instituto Universitario de Matem\'atica Pura y Aplicada,\\
Universitat Polit$\grave{\textrm{e}}$cnica de Val$\grave{\textrm{e}}$ncia, Valencia 46022, Spain}
\end{center}

\begin{abstract}
The effect of an hypothetical maximal proper acceleration on the inertial mass of charged particles is investigated in the context of particle accelerators. In particular, it is shown that maximal acceleration implies a reduction of the luminosity of the bunches relative with respect to the expected luminosity in the relativistic models of the bunches. This relative loss in luminosity is of the order $10^{-7}$ for the LHC and of order $10^{-4}$ for current laser plasma accelerators facilities. Although this effect is small, it increases with the square of the number of particles in the bunch.
\end{abstract}
\section{Introduction}
Despite Lorentzian geometry has been outstandingly successful for the description of physical spacetime arena in relativistic theories and constitutes one of the pilars of modern theoretical physics, it lays on several assumptions that, from a formal point of view, some of them remain still poorly unjustified. This is the case of the {\it clock hypothesis} in relativity. Despite its relevance in the formal construction of relativity and that it comes as an axiom, it id an assumption clearly stated only in few references. One of them it is in Einstein's own account of relativity theory \cite{Einstein1922}. Other one, even more clear and precise, is Pauli's account of relativity \cite{Pauli 1958}. The assumption can be states as the the requirement that the proper time functional along a curve depends formally only on the tangent structure along the curve. The justification in Einstein's work is none, except maybe the easiness in the formalities of the theories. Of course, a posteriori, the models based on this type of geometric elements have been tremendously successful. But this success cannot eliminate certain unsatisfactory  about the theoretical frame adopted. After all, high order jet geometry is possible where the line element depends upon high order derivatives of the coordinates. Similar reflections apply to Pauli's discussion.  was more or less because, as long as, one is has been without a better justification of the clock hypothesis it seems liberating to renounce to it.

There are reasons to generalized the standard Lorentzian spacetime structures to incorporate, not only a maximal speed of propagation of matter in vacuum, but also a maximal proper acceleration. From one side, the arguments of B. Mashhoon showing that, in the case of classical electrodynamics, in situations when radiation reaction effects are of relevance, it is necessary to abandon the assumption of the clock hypothesis  \cite{Mashhoon1990}, because in such situations it cannot hold.

The renounce to the clock hypothesis opens the door to a non-trivial dependence of the spacetime metric in acceleration. One natural way to describe this dependence is by means of a small, linear perturbation to the original Lorentzian structure that depends on the scalar value of the proper acceleration. The smallness of the perturbation implies the existence of a well-defined connection theory, similar to what happens in Randers-type spaces in Finsler geometry \cite{Ricardo2015}. Being small means an open condition on accelerations, that is, an uniformly bounded acceleration for any type of admissible physical world line. That is, the existence of a maximal acceleration.

A second motivation for the radical modification of the classical structure of spacetime that proposes the theory of maximal proper acceleration metrics consists on the resolution of the conundrum of the motion of a classical charged point particle. Indeed, the development of a model of point charged particle where, in the framework of fields depending on the second jet of classical probe particles world lines and in the setting of maximal acceleration as envisaged in \cite{Ricardo2015}, showed the existence of a consistent second order differential dynamical model for the dynamics of point charge particle \cite{Ricardo 2017}. Schematically, the theory assumed that the electromagnetic fields are not independent physical objects defined over the spacetime manifold $M_4$, but that they depend on how they are probed by using test charged particles. This is a bold idea within the standard field tradition. But it also natural one in the context of feed-back dynamical systems as a charge particle-field dynamics. Once this hypothesis on the nature of the classical fields is accepted, it is natural to think that the spacetime metric structure is a high order jet field too, thinking in that matter constrains the spacetime structure and that the spacetime structure affects the evolution of matter.

This paper explores the effect of maximal proper acceleration on different dynamical systems through the dependence of the inertial mass in the maximal acceleration. The topic is related with the effect of acceleration on inertia. Indeed, Bonnor discussed a second order differential equation where the inertial mass depends on the acceleration \cite{Bonnor}. However, despite several formal similarities, Bonnor's theory and the theory of inertia studied in this paper are rather different. Moreover, although the present investigation is restricted to the consequences for particle acceleration, in the theory discussed in this paper, the relation between inertia and acceleration is universal.

The structure of this paper is the following. First, we recall the elements of the theory of high order fields and metrics of maximal acceleration that we need. Later, the notions of $4$-momentum and the relation of mass with acceleration is discussed. After that, the electrodynamic model of point charged particle that we use is introduced, with special attention to the case of bunches of particles moving coherently. In this way, we do not consider space charge effects. Although this model is rather simple, it makes apparent that the effects of maximal acceleration on inertial mass to scales are observable with current facilities.
 Evaluated for the case of the LHC, it turns out that the effect is small, being the increase in the inertial mass of the bunch due to acceleration of order $10^{-7}$. However, for current laser-plasma acceleration the effect is of order $10^{-4}$ relative to standard relativistic mass. Remarkably, an increment in the inertial mass of the particles constituting a coherent mass implies a reduction in the luminosity of the same order, relative to the theoretical luminosity in the relativistic theories, an effect that we expect can be directly measured or traced back in the laboratory.
\section{Spacetimes with metrics of maximal acceleration and its kinematics}
\subsection{Geometries of maximal proper acceleration as high order jet geometries}Geometries of maximal proper acceleration, referred to as high-order jet geometries, constitute
a class of geometric structures that exhibit compatibility with both maximal proper acceleration and maximal speed \cite{Ricardo2007,Ricardo2012,Ricardo2015}. These metric spacetimes structures are not Lorentzian metrics or pseudo-Riemannian metrics, but metric structures defined in higher order jet bundles over the spacetime manifold $M_4$. Jet spaces provide a framework specially adapted to consider problems of formal theory of differential equations. The role played by jet theory constructions and notions in physics emerges when one realized how jet theory is specially adequate to consider back-reaction effects in a systematic way. Indeed, such a framework was used to construct a scheme of classical electrodynamics where fields depend on how they are interrogated by test charged point particles and charged point particles obey a consistent second order differential equation \cite{Ricardo2007,Ricardo2012,Ricardo2015}. Some of the consequences of that model will be discussed in this paper.

The leading order term of maximal acceleration metric is a Lorentzian structure, that here we denote by $\eta$ and with signature $(-1,1,1,1)$. The metric $\eta$ is seeing either as the limit metric of $g$ when the maximal acceleration goes to infinity or as the metric when the test particles does not experience acceleration.
 Let $(M_4,\eta)$ be a Lorentzian spacetime, $D$ the covariant derivative operator of the Levi-Civita connection of $\eta$ and $NC$ the null cone bundle.
The maximal acceleration metric is determined by the following result.
Let $x:I\to { M}_4$ be a smooth curve on $M_4$. By $^1x$ denotes the first jet at the point $x(t)$, that is, the $1$-jet whose coordinates are $(x^\mu(\tau),\dot{x}^\mu(\tau))$. Let us consider the case where $x:I\to M_4$ is such that $\eta(\dot{x}(\tau),\dot{x}(\tau))\neq 0$.
Then there is a non-degenerate, symmetric form $g$ such that acting on the tangent vector $\dot{x}$ the value is
\begin{align}
g_{\mu \nu}(\,^2x(\tau)) =\Big(1+ \frac{  \eta(D_{\dot{x}}\dot{x}(\tau)
D_{\dot{x}}\dot{x}(\tau))}{A^2 _{max}\,\eta(\dot{x},\dot{x})}\Big)\, \eta_{\mu \nu},
\label{maximalaccelerationmetric}
\end{align}
where $^2x(\tau)\in\,J^2(M_4)$ is the second jet at $x(\tau)$ with coordinates $(x^\mu(\tau),\dot{x}^\mu(\tau))$.
The bilinear form
\begin{align}
g(\,^2x)=\,g_{\mu\nu}(\,^2x)\,dx^\mu\otimes dx^\nu
\end{align}
 with components given by the covariant formula \eqref{maximalaccelerationmetric} is the {\it metric of
maximal acceleration} in general coordinates. It determines the proper
time along the curve $x:I\to M_4$ and also provides a generalization of the notion of
angle.

The metric structure defined above is of maximal acceleration in the following sense. Let $x:I \to M_4$ such that:
 \begin{itemize}
 \item It holds that
 $g(\dot{x},\dot{x})<0$, $\eta(\dot x,\dot x)<0$,

\item The covariant condition
\begin{align}
\eta(D_{\dot{x}}\,\dot{x},\,D_{\dot{x}}\,\dot{x})\,\geq 0.
\label{spacelikeaccelerations}
\end{align}
holds.

 \end{itemize}
Thus under the assumption just described, one has the natural bounds
 \begin{align}
  0 \leq g(\ddot{x},\ddot{x})\leq\,\eta(D_{\dot{x}}\dot{x},D_{\dot{x}}\dot{x})<\,A^2_{max}.
  \label{boundedconditionforacceleration}
 \end{align}
By a maximal proper acceleration $A_{max}$ we mean that the bound \eqref{boundedconditionforacceleration} holds good for any value of the four acceleration vector $D_{\dot{x}}\dot{x}$. Note that the parameter used in the definition of the acceleration is the proper parameter of $\eta$,
\begin{align}
\tau [\gamma] :=\,\int_{\gamma} \,dt\,\left( -\eta({\gamma}',{\gamma}')\right)^{1/2} .
\label{propertime eta}
\end{align}
This functional is indeed re-parametrization invariant.
$\tau [\gamma] $ is equal or larger than the proper time of the metric of maximal acceleration,
\begin{align}
s [\gamma] :=\,\int_{\gamma} \,d\tau\,\left( -g(\dot{\gamma},\dot{\gamma})\right)^{1/2} = \,\int_{\gamma} \,d\tau\,\left(1-\frac{\eta({\ddot{\gamma},\ddot{\gamma}})}{A^2_{\textrm{max}}}\right)^{1/2},
\label{propertime g}
\end{align}
where in this expression, the parametrization of the curve is done with respect to the proper time $\tau$ of $\eta$; the functional $s [\gamma]$ is not re-parametrization invariant.

\subsection*{Notions of four-velocity in a geometry of maximal acceleration}
The four-velocity is given by
\begin{align}
v^\mu(t)=\,\frac{1}{\sqrt{1-\frac{a^2(t)}{A^2_{max}}}}\,\tilde{v}^\mu(t),\quad \mu=1,...,4,
\label{fourvelocity}
\end{align}
where $\tilde{v}^\mu(t)$ is
 \begin{align*}
\tilde{v}^\mu(t)=\,\lim_{\Delta\to 0}\frac{x^\mu(t+\Delta)-x^\mu(t)}{\int^{t+\Delta}_{t}\,\sqrt{-\eta(x',x')}\,d\tilde{t}}=\,\lim_{\Delta\to 0}\frac{x^\mu(t+\Delta)-x^\mu(t)}{\Delta}.
\end{align*}
 The components $v^\mu(t)$  defined the expression \eqref{fourvelocity} determine the $4$-velocity, that is indeed a $4$-vector.
\subsection*{The case when $\eta$ is the Minkowski metric}
If the metric $\eta$ is the Minkowski metric $h=diag\,(-1,1,1,1)$, in any Fermi coordinate system
the relativistic four-velocity $\tilde{v}^\mu(t)$ is related with the coordinate velocity vector $\vec{\bf {v}}$ by the
expression
\begin{align*}
 \tilde{v}^0=\,\frac{1}{\sqrt{1-\frac{\vec{\bf {v}}^2(t)}{c^2}}}\,c,\quad
\vec{\tilde{v}}(t)=\,\frac{1}{\sqrt{1-\frac{\vec{\bf {v}}^2(t)}{c^2}}}\,\vec{\bf {v}}(t).
\end{align*}
Then  from the expression \eqref{fourvelocity}, one has the following relations for the components of the four-velocity $v^\mu(t)$,
\begin{align}
v^0(t)=\,\frac{1}{\sqrt{1-\frac{a^2(t)}{A^2_{max}}}}\frac{1}{\sqrt{1-\frac{\vec{\bf {v}}^2(t)}{c^2}}}\,c,
\label{componentv0}
\end{align}
\begin{align}
\vec{v}(t)=\,\frac{1}{\sqrt{1-\frac{a^2(t)}{A^2_{max}}}}\frac{1}{\sqrt{1-\frac{\vec{\bf {v}}^2(t)}{c^2}}}\,\vec{\bf
{v}}(t),
\label{componentv123}
\end{align}
from which follows that the components $(v^0,\vec{v})$ transform contravariantly under the action of the Lorentz group $O(1,3)$.



\subsection{Redefinition of the energy and momentum observables}
\begin{definicion}
Let $(M,g)$ be a spacetime of maximal acceleration and $O$ an observer. Then the four-momentum  of a point particle with mass $m$ and world line
$x:I\to M$ observed by $O$ is defined by the components
\begin{align}
P^\mu(t)=\,m\,v^\mu( t),\quad \mu=1,2,3,4,
\label{fourmomentum}
\end{align}
where $v^\mu(t)$ is the velocity measured by $O$.
\end{definicion}

In the case when $\eta$ is the Minkowski metric $\eta$ there are inertial coordinate systems. The components respect to an inertial coordinate system of the celerity $4$-vector $(v^0,\vec{v})$ are given by
\eqref{componentv0} and \eqref{componentv123}. Then the components of the four-momentum are
 \begin{align}
c\,P^0(t)=E(t)=\,\frac{1}{\sqrt{1-\frac{a^2(t)}{A^2_{max}}}}\,\frac{1}{\sqrt{1-\frac{\vec{\bf {v}}^2}{c^2}}}\,mc^2,
\label{modifiedE}
\end{align}
\begin{align}
\vec{\bf P}(t)=\,\frac{1}{\sqrt{1-\frac{a^2(t)}{A^2_{max}}}}\,\frac{1}{\sqrt{1-\frac{\vec{\bf {v}}^2}{c^2}}}\,m\vec{\bf {v}},
\label{threemoment}
\end{align}
where $a^2 :=\eta(\ddot{x},\ddot{x})$. Note that $g(\ddot{x},\ddot{x})\leq \eta(\ddot{x},\ddot{x})$.

According to the re-definition of energy of a point particle, in the instantaneous Lorentzian coordinate frame where the particle has associated the Lorentz factor $\gamma$, the energy of the system is given by the expression introduced in ref. \cite{Ricardo2015},
\begin{align*}
E=\,\gamma \,\frac{m_0\,c^2}{\sqrt{1-\frac{a^2}{A^2_\textrm{max}}}}.
\end{align*}
The redefinition of energy implies that the inertial mass of an accelerated particle with relativistic factor $\gamma$ is of the form
\begin{align*}
m :=\,\gamma\,\frac{m_0}{\sqrt{1-\frac{a^2}{A^2_\textrm{max}}}}.
\end{align*}
In particular, in the instantaneously co-moving frame where $\gamma =1$ it holds that
\begin{align}
m =\,\frac{m_0}{\sqrt{1-\frac{a^2}{A^2_\textrm{max}}}}.
\label{mass of an accelerated particle}
\end{align}

\section{The model of the classical point charged particle in spacetimes  of maximal acceleration}

In the theory of higher order jet electrodynamics, the spacetime structure that determines the proper time is a metric of maximal acceleration \eqref{maximalaccelerationmetric}.
Under this assumption and after some further constraints related with the conservation of energy by assuming compatibility with the covariant Larmor's radiation formula \cite{Jackson}, general arguments lead to a  model of the point charged particle described by a second order differential equation \cite{Ricardo 2017}. In local coordinates, the equation acquires the expression
\begin{align}
m\,\ddot{x}^{\mu} =\,q\,F^{\mu}\,_{\nu}\,\dot{x}^{\nu}-\,\frac{2}{3}\,{q^2}\,g_{\rho\sigma}\,
\ddot{x}^{\rho}\ddot{x}^{\sigma}\,\dot{x}^{\mu},\quad F^{\mu}\,_{\nu}=\,g^{\mu\rho}F_{\rho\nu}.
\label{equationofmotion}
\end{align}

Some remarks on the equation \eqref{equationofmotion} are in order. The first is, as it is noticeable, that it is a second order differential equation, in contrast with the Lorentz-Dirac equation. From this fact it is natural that it avoids the pre-accelerated solutions of the Lorentz-Dirac equation, that has their origin in the third order derivatives that appear in it. Second, one can also show that the equation \eqref{equationofmotion} does not have run-away solutions \cite{Ricardo 2017}.

Equation \eqref{equationofmotion} formally resembles the equation obtained by W. Bonnor \cite{Bonnor}. Besides the rotund differences in the assumptions of the theory beneath the equation \eqref{equationofmotion} and Bonnor's theory, there is an immediate difference, since in our theory the parameter $\tau$ that appears in the equation is the proper time associated with the metric of maximal acceleration, while in Bonnor's theory the spacetime is relativistic and the proper time parameter that appears in the theory is just the Minkowskian's one. Thus all these make both theories different.

The property of the equation \eqref{equationofmotion} that we are concerned in this paper is that it implies a maximal proper acceleration. Such acceleration scale can be obtained from itself as follows.
If we contract the left hand side with the left hand side and the right hand side with the right hand side using the metric of maximal acceleration $g$, then the following expression is obtained:
 \begin{align*}
 m^2\left(1-\frac{a^2}{A_\textrm{max}}\right)\,a^2\,= \,F^2_L +\,\left(\frac{2}{3}\,q^2\right)^2\,(a^2)^2\,g(\dot{x},\dot{x}).
 \end{align*}
 where
 \begin{align*}
 F^2_L =\, q\, F^{\mu}_\nu\, \dot{x}^\nu\,g_{\mu\rho}\,q\,F^{\rho\sigma}\dot{x}^\sigma\,\geq 0.
 \end{align*}
 Since $F^2_L$ is greater or equal to zero and $g(\dot{x},\dot{x})=-1$, it follows that
  \begin{align*}
 m^2\,\left(1-\frac{a^2}{A^2_\textrm{max}}\right)a^2-\left(\frac{2}{3}\,q^2\right)^2\,(a^2)^2\geq 0.
 \end{align*}
Since $a^2 >0$, then we have that
 \begin{align*}
 1> \, a^2\left(\frac{1}{A^2_{\textrm{max}}}-\left(\frac{2 q^2}{3m}\right)^2 \right)
 \end{align*}
It follows the bounds
\begin{align}
a^2\leq A^2_{\mathrm{max}}< \left(\frac{3}{2}\,\frac{m}{q^2}\right)^2 .
\end{align}
Therefore, the maximal acceleration of a point particle whose dynamics is determined by the equation \eqref{equationofmotion} is given by the expression
\begin{align}
 A^2_{\mathrm{max}}=\,\left(\frac{3}{2}\,\frac{m}{q^2}\right)^2 ,
\label{valueofthemaximalacceleration}
\end{align}
since there is no other acceleration scale between $0$ and $\frac{3}{2}\,\frac{m}{q^2}$ in the model.
This value of the maximal proper acceleration \eqref{valueofthemaximalacceleration} depends on the characteristics of charged particle, in particular, on the charge and mass of the particle. It is this property that we will show has potential phenomenological consequences for the luminosity of the bunches in particle acceleration.


\section{Phenomenological consequences of the redefinition of the energy in accelerator physics}
The difference in the definition of energy E between the theory with maximal proper acceleration
and the energy of the system according to the theory of relativity takes the form 
\begin{align}
\Delta E =\,\gamma\, \left(\frac{1}{\sqrt{1-\frac{a^2}{A^2_\textrm{max}}}}-1\right)\,m_0\,c^2 =\,\frac{1}{2}\,\gamma\,\frac{a^2}{A^2_\textrm{max}}\,m_0\,c^2+\,\mathcal{O}(a^2/A^2_{max})^2.
\end{align}
Since $a^2/A^2_{max}\ll 1$, the approximation up to second order is reasonable for our purposes. Indeed, the quotient $\mathcal{O}(a^2/A^2_\textrm{max})^2$ can be evaluated for the model of point particle proposed. For a particle of charge and mass $(q_0,m_0)$, we have that the maximal acceleration is of the form $A_\textrm{max}=\,\frac{3}{2} \frac{m_0}{q^2_0}$. For  one proton, this is of order $A_\textrm{max}(1)\sim \, 10^{35} m/s^2$. But if the bunch being accelerated contains $N$ particles and if the bunch of particles is considered as a sole particle \cite{Ricardo 2019}, then the maximal acceleration is decreased by a factor $N$. Hence the exceed in energy with respect to the relativistic energy due to a maximal acceleration is given by the expression
\begin{align*}
\Delta E^{l=1} / E_{rel} = \, \frac{1}{2}\, N^2 \,\frac{a^2}{A^2_\textrm{max}(1)},
\end{align*}
where $A^2_\textrm{max}(1)$ is the maximal proper acceleration that applies to a sole particle.
If the systems pass by an identical acceleration device $M$ times, the additive effect leads to
\begin{align}
\Delta E^{l=M} / E_{rel} = \, \frac{1}{2}\, M\,N^2\,\frac{a^2}{A^2_\textrm{max}(1)}.
\label{effect of energy in the accelartion in an accelerator}
\end{align}
In terms of the value of the maximal acceleration we have
\begin{align}
\Delta E^{l=M} / E_{rel} = \, \frac{2}{9}\,\frac{q^4_0}{m^2_0} M\,N^2\,a^2 .
\label{effects depends upon the mass}
\end{align}
As an approximation, the value of $a^2$ can be calculated using the Lorentz formula.

In order to illustrate the theory, let us consider the case of the effect of maximal acceleration at LHC. In this case, the particles accelerated are protons. The maximal number of particles per bunch is of
the form $N=\,10^{12}$. Furthermore, the number of turns before the collisions is of order $10^7$. This implies an acceleration of the order $10^{15} m/s^2$ and a maximal acceleration for an individual proton  $A_{\textrm{max}}(1)= \frac{2}{3}\frac{m_p}{e^2}\sim 10^{35} m/s^2$. In the case of the radio frequency (RF) cavity acceleration, in the co-moving system, the electric field is of order $2 \times 10^6$ Volts along $10 m$ cavities. In this case, since there are $16$ RF-cavities, we have that $M\sim 10^{7}$.  Thus we have that for the RF acceleration, the energy difference between the relativistic theory and the theory of maximal acceleration that we are discussing is of order $\Delta E^{l=10^{7}} / E_{rel} \sim 10^{-9}$.

The effect of the acceleration when the bunch passes through the electromagnetic field of the magnets is estimated as follows. Each dipole magnet generates a maximal field of $8.3\, T$. Also, for the maximal energy, the relativistic $\gamma$ factor is of order $7000$. Thus by applying the Lorentz force as a first approximation, the acceleration of each proton in the bunch is of order $10^{15} m/s^2$. However, the number of dipole magnets is of order $1300$. Thus since the number of turns is $10^7$, this implies $M\sim 10^{10}$. For bunches with $N\sim 10^{12} $ protons, this leads to discrepancy in the energy of order  $\Delta E^{l=10^{10}} / E_{rel} \sim 10^{-7}$ as a conservative estimate.

This discrepancy in energy implies a deficit in the population bunch with respect to the expected in relativistic models. Since the energy injected in the dynamical system is fixed, let say that is $E_0$, there is an extra inertia in the process of acceleration, due to the relation \eqref{mass of an accelerated particle}. The stability of the bunch of particles implies that this difference in energy, or the deficit to accelerate, is translated into a situation where fewer particles are actually accelerated to the required energy.
In other words, there is a small deficit in the population of the bunch. The ideal equipartition of
energy in the bunch implies
\begin{align*}
(E_0+\Delta E) (N+\Delta N) = \,E_0 N.
\end{align*}
Thus we have that
\begin{align}
\frac{\Delta N}{N} =\,-\frac{\Delta E}{E_0}=\,-\frac{2}{9}\,\frac{q^4_0}{m^2_0} M\,N^2\,a^2 .
\end{align}
For the example of the acceleration of magnets of proton bunches in the LHC as discussed above, we have that $\frac{\Delta N}{N} \sim \,- 10^{-7}$.

The discrepancy in energy $\frac{\Delta E}{E_0}$ depends upon the species of particle by the inverse of the mass squared in \eqref{effects depends upon the mass}. For instance, in electron bunch acceleration in laser-plasma acceleration, $N$ can be of order $10^8$, which is lower than in LHC bunches. However, the acceleration $a$ can be much higher, of order $10^{22} m/s^2$, while the maximal acceleration of the electron is of the form $A_{\textrm{max}}\sim 10^{32} m/s^2.$ These figures result in a discrepancy with relativistic energy is of order $
\Delta E^{l=1} / E_{rel}\sim 10^{-4}$, which is translated in the equivalent decrease in the population of the bunch.
\subsection{Conclusion}In this paper we have investigated one consequence of the theory of spacetimes with maximal acceleration as developed in \cite{Ricardo2015} on the inertial mass. In particular, we have observed how the existence of maximal acceleration for the model of electrodynamics discussed in \cite{Ricardo 2017} implies, for the model of bunch of a particle acceleration discussed in \cite{Ricardo 2019}, a reduction on the luminosity with respect to the relativistic theory. One lucky aspect of this effect, that at least holds in the domain where the assumptions of the models hold, is the accumulation effect in the relative deficit $\Delta E/E$ due to the factors on $N^2$, $a^2$ and $M$. This makes it even more feasible to test the effect if
such factors can be increased.





\footnotesize{
\begin{thebibliography}{22}


\bibitem{Bonnor} W. B. Bonnor, {\it A new equation of motion for a radiating charged
 particle}, Proc. R. Soc. Lond. A {\bf 337}, 591-598 (1974).



\bibitem{Einstein1922} A. Einstein, {\it The meaning of relativity}, Princenton University Press (1923).

\bibitem{Ricardo2007} R. Gallego Torrom\'e, {\it On a covariant version of Caianiello's Model}, Gen. Rel. Grav. {\bf 39}, 1833-1845 (2007).

\bibitem{Ricardo2012} R. Gallego Torrom\'e, {\it Geometry of generalized higher
order fields and applications to classical linear electrodynamics}, arXiv:1207.3791.


\bibitem{Ricardo2015} R. Gallego Torrom\'e, {\it An effective theory of metrics with maximal proper acceleration}, Class. Quantum. Grav. {\bf 32}, 2450007 (2015).

\bibitem{Ricardo 2017} R. Gallego Torrom\'e, {\it A second order differential equation for a point
Charged particle}, Int. J. Geom. Methods Mod. Phys., {\bf 14}, No. 04, 1750049 (2017).

\bibitem{Ricardo 2019} R. Gallego Torrom\'e, {\it Some consequences of theories with maximal acceleration in laser-plasma acceleration}, Modern Physics Letters A Vol. {\bf 34} (2019) 1950118.

\bibitem{Jackson} J. D. Jackson, {\it Classical Electrodynamics}, Third ed.
Wiley (1998).

\bibitem{Mashhoon1990} B. Mashhoon,
{\it The Hypothesis of Locality in relativistic physics}, Phys. Lett. A {\bf 145}, 147–153 (1990).


\bibitem{Pauli 1958} W. Pauli, {\it Theory of Relativity}, Pergamon Press (1958).










\end{thebibliography}}

\end{document} 