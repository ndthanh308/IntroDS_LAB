\section{PROBLEM STATEMENT}
\label{sec: problemstatement}
% Figure environment removed

% \wenzhen{this is not that obvious. You need to explain why. Also you can comment on the importance of measuring volume} 
\wenzhen{You need to discuss whether the method can be generalized to other containers} 


We aim to estimate the properties of solid particles in the container using only touch sensing. In this section, we define the targeting particle properties as the following: content mass $M_p$, content volume $V_p$, particle size $D_p$, and particle shape $S_p$. Some samples of the particles with different properties are shown in Fig. \ref{fig:problem_statement}. These properties are chosen because they are fundamental information to define the particles inside the container and useful perception information for manipulation tasks. For example, accurate measurement of the total volume of particles is essential for precise volume control in robotic pouring applications. 



% In this work, we assume the granular media in the container is homogeneous.
% To simplify the problem, we make several assumptions and definitions. First,\wenzhen{where are the second, third ...?} we assume the same type of particles are the same.
%For some solid particles such as candy and mints, this assumption is perfectly satisfied, while for some particles such as beans. this assumption may not be correct because there are individual differences between particles.

\noindent \textbf{Content mass $M_p$} is defined as the total mass of the particles inside the container.

%Given the same container whose mass is defined as $m_c$, the estimation of particle mass $m_p$ can be thought as equal to the estimation of the total weight $m_w = m_p + m_c$. 
% In the following experiments, we get $m_c$ by measuring the weight of the empty container and estimate $m_p$.

\noindent \textbf{Content volume $V_p$} is defined as the total volume of those particles and the room between their stacking. In the following experiment, we use the same container during training and testing. Therefore, $V_p$ is linearly correlated to the height $H_p$ of the particle stacks when holding the container vertically. To simplify, we will use content height $H_p$ to represent $V_p$. 

\noindent \textbf{Particle size $D_p$} is defined as the diameter of the sphere with the same volume as the given particle: $D_p = (\frac{6V}{\pi})^{\frac{1}{3}}$, where $V$ is the volume of the individual particle. 

\noindent \textbf{Particle shape $S_p$} is defined by the particle sphericity $\Psi_p$, which is commonly used to describe how similar the particle shape is compared to spheres. Sphericity is defined as the ratio of the surface area of a sphere with the same volume as the given particle to the surface area of the particle \cite{wadell1935volume}: $\Psi_p = \frac{\pi^{\frac{1}{3}}(6V)^{\frac{2}{3}}}{A}$, where $V$ is the volume of the individual particle and $A$ is its surface area. Sphericity is always a number between $0$ and $1$. The more spherical the shape, the closer its sphericity is to $1$. Since the sphericity distribution of common particles is skewed toward $1$ and is unbalanced, and the sphericity of powder is less meaningful, we re-define the shape descriptor to five classes using the following rules:

\begin{equation*}
S_p = \left\{
\begin{aligned}
0 & \hspace{0.5 in} (D_p \leq 1\text{mm})\\
1 & \hspace{0.5 in} (\Psi_p \leq 0.7 \hspace{0.1in} \text{and} \hspace{0.1in} D_p > 1\text{mm})\\
2 & \hspace{0.5 in} (0.7 < \Psi_p \leq 0.9  \hspace{0.1in} \text{and} \hspace{0.1in} D_p > 1\text{mm}) \\
3 & \hspace{0.5 in} (0.9 < \Psi_p \leq 0.96 \hspace{0.1in} \text{and} \hspace{0.1in} D_p > 1\text{mm}) \\
4 & \hspace{0.5 in} (0.96 < \Psi_p \leq 1 \hspace{0.1in} \text{and} \hspace{0.1in} D_p > 1\text{mm}) \end{aligned}
\right.
\end{equation*}
 
% The shape of powder particles is defined as $0$ because we observed powder usually performs like very irregular particles. \joe{I think this previous sentence can be removed. The previous sentence is clear enough.}


% For the particle whose shape is regular, we calculate their theoretical volume and surface area. For the particle whose shape is irregular, we assume its shape is an ellipsoid and measure the length of its 3 axes.\wenzhen{how is this paragraph connected to the previous one?}



% In this paper, we estimate the particle properties in multiple steps with different exploratory procedures. %In next section, we will describe how we estimate the particle property based on the tactile data acquired by some dynamic actions and the previous estimation on some properties.

