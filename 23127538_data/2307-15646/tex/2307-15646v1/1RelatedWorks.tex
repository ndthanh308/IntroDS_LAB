\section{RELATED WORKS}
\label{sec: relatedworks}

\subsection{Properties of Granular Media}

The study of granular media properties is an active research area within the physics community. Some work concluded the relationship between the granular media macroscale behavior to particle-scale properties, which inspires us when we define the robot motion. The angle of repose (AoR) is one common metric to describe particle behavior. It is commonly defined to be the steepest slope of the unconfined material, measured from the horizontal plane on which the material can be heaped without collapsing \cite{mehta1994dynamics}. AoR was found to be related to multiple particle properties of granular material, including the particle shape \cite{dai2016effect}, the particle size \cite{zhou2002experimental}, \cite{botz2003effects}, the particle friction coefficients \cite{santos2016investigation}, the content moisture \cite{zaalouk2009effect}, the number of particles \cite{miura1997method}, et al. Although multiple works concluded the relationship between AoR and particle properties, few works applied it to robot applications. In our work, we design one robot motion inspired by this and extract topple-related features which reflect the particles' AoR to estimate solid particle properties.

% \cite{do2017automated}, \cite{benvenuti2016identification} attempted to calibrate the particle simulators by measuring the particle AoR. \wenzhen{multiple grammar problems}

Very few previous works solved the problem that using robots to estimate solid particle properties. The only relevant work we found was by Matt \cite{matl2020inferring}, who used the pattern of the particle stacked rings to calibrate parameters of the particles, such as friction and restitution, for simulation. In our work, we define the particle property estimation task from a different perspective: we focus on the estimation of some explicit particle properties such as particle size and particle shape. These properties are more intuitive for humans and robots to get an understanding of those particles.



\subsection{Content Classification}

Although very few works targeted estimating particle properties, researchers have tried to recognize the particle type in opaque containers, which is a relevant easier task. Those methods provide us with some inspiration about how to distinguish particles with different properties. Those works used dynamic signals to recognize the content, which are collected with the robot motion such as shaking, dropping, and tapping \cite{sinapov2009interactive}, \cite{sinapov2011interactive}, \cite{griffith2012object}. The signals come from either audio or touch. For example, Eppe \cite{eppe2018deep} presented a strategy where the robot shakes the plastic capsules and uses auditory information to classify the contents. Jin \cite{jin2019open} collected the sound generated by rotating the bottles with different particles to classify the contents. Clarke \cite{clarke2018learning} used shaking sounds for estimating flow and amounts. Sinapov \cite{sinapov2014learning} proposed a framework to detect object categories by combining visual, auditory, and proprioceptive information. They explored 36 objects with different contents, weights, and colors by 10 behaviors. Chen \cite{chen2016learning} combined acoustic and acceleration information for content classification. There are also some works trying to estimate liquid properties and dynamic tactile sensing is found to be useful. Huang \cite{Huang-RSS-22} used dynamic tactile sensing to estimate liquid viscosity and height. Saal \cite{saal2010active} used tactile sensing to predict the liquid viscosity and optimized the shaking frequency and the rotation angle of shaking. Matl \cite{matl2019haptic} used F/T data and a model-based method for liquid property estimation and utilized the estimation for accurate pouring liquids.  Different from the above works, we target a more challenging task: estimating the quantity results of solid particle properties. Compared to semantic labels, the quantitative particle property estimation would be more helpful for following manipulation tasks and need to be generalized to unseen particles.
