%0%%%%%%%%%%%%%%%%%%%%%%%%%%%%%%%%%%%%%%%%%%%%%%%%%%%%%%%%%%%%%%%%%%%%%%%%%%%%%%%
%2345678901234567890123456789012345678901234567890123456789012345678901234567890
%        1         2         3         4         5         6         7         8

%&latex 
\usepackage{hyperref}
\usepackage{url}
\usepackage{xurl}
\usepackage{graphicx}
\usepackage{booktabs}
\usepackage{amssymb}
\usepackage{amsmath}
\usepackage{subcaption}
\usepackage{xcolor}
\usepackage{adjustbox} 
\usepackage{multirow}
\usepackage{wrapfig}
\usepackage{tikz}
\usepackage{xspace}
\usepackage{soul}
\newcommand{\ie}{\textit{i}.\textit{e}., }
\newcommand{\eg}{\textit{e}.\textit{g}., }
\usepackage{titletoc}
% \usepackage{flushend}

\definecolor{bg_blue}{RGB}{213,227,251}
\definecolor{bg_yellow}{RGB}{250,243,187}
\definecolor{bg_purple}{RGB}{177,167,207}
\definecolor{bg_red}{RGB}{200,169,188}
\definecolor{bg_green}{RGB}{192,213,175}
\definecolor{bg_skin}{RGB}{245,232,210}

\definecolor{red_color}{RGB}{255,0,0}
\newcommand{\redtext}[1]{\textcolor{red_color}{#1}}
\definecolor{yellow_color}{RGB}{255,202,47}
\newcommand{\yellowtext}[1]{\textcolor{yellow_color}{#1}}

\definecolor{purple_color}{RGB}{64,103,139}
\newcommand{\purpletext}[1]{\textcolor{purple_color}{#1}}

\definecolor{dark_red}{RGB}{153, 31, 41}
\newcommand{\drtext}[1]{\textcolor{dark_red}{#1}}
\definecolor{green_color}{RGB}{130,139,78}
\newcommand{\greentext}[1]{\textcolor{green_color}{#1}}
\definecolor{brown_color}{RGB}{205,90,161}
\newcommand{\browntext}[1]{\textcolor{brown_color}{#1}}
\definecolor{lg_color}{RGB}{63,147,139}
\newcommand{\lgtext}[1]{\textcolor{lg_color}{#1}}

\definecolor{com_color}{RGB}{0,0,139}
\newcommand{\com}[1]{\textcolor{com_color}{#1}}


\definecolor{orange_color}{RGB}{255,148,63}
\newcommand{\orangetext}[1]{\textcolor{orange_color}{#1}}

\definecolor{gray_color}{RGB}{169,169,169}
\newcommand{\graytext}[1]{\textcolor{gray_color}{#1}}


\definecolor{lightgray}{RGB}{220,220,220}
\newcommand{\codehighlight}[1]{\colorbox{lightgray}{{#1}}}

\definecolor{lightgreen}{RGB}{179,207,176}
\newcommand{\codehighlightgreen}[1]{\colorbox{lightgreen}{{#1}}}

\definecolor{lightblue}{RGB}{181,209,230}
\newcommand{\codehighlightblue}[1]{\colorbox{lightblue}{{#1}}}


\newcommand{\model}{\mbox{\sc LLM-Rec}\xspace}


\newcommand{\ctext}[3][RGB]{%
  \begingroup
  \definecolor{hlcolor}{#1}{#2}\sethlcolor{hlcolor}%
  \hl{#3}%
  \endgroup
}


\newcommand\DoToC{%
  \startcontents
  \printcontents{}{1}{\textbf{Table of Contents (Appendix)}\vskip9pt\hrule\vskip5pt}
  \vskip3pt\hrule\vskip5pt
}



\title{\LARGE \bf Estimating Properties of Solid Particles Inside Container \\Using Touch Sensing 
}


\author{Xiaofeng Guo$^{1}$, Hung-Jui Huang$^{1}$, and Wenzhen Yuan$^{1}$% <-this % stops a space
\thanks{This work is supported by the Toyota Research Institute (TRI).}% <-this % stops a space
\thanks{$^{1}$Xiaofeng Guo, Hung-Jui Huang, and Wenzhen Yuan are with the Robotics Institute, Carnegie Mellon University, 5000 Forbes Ave, Pittsburgh, PA 15213, USA {\tt\small\{xguo2, hungjuih, wenzheny\}@andrew.cmu.edu}}%
}


\begin{document}



\maketitle
\thispagestyle{empty}
\pagestyle{empty}


%%%%%%%%%%%%%%%%%%%%%%%%%%%%%%%%%%%%%%%%%%%%%%%%%%%%%%%%%%%%%%%%%%%%%%%%%%%%%%%%
\begin{abstract}
Solid particles, such as rice and coffee beans, are commonly stored in containers and are ubiquitous in our daily lives. Understanding those particles’ properties could help us make later decisions or perform later manipulation tasks such as pouring. Humans typically interact with the containers to get an understanding of the particles inside them, but it is still a challenge for robots to achieve that. This work utilizes tactile sensing to estimate multiple properties of solid particles enclosed in the container, specifically, content mass, content volume, particle size, and particle shape. We design a sequence of robot actions to interact with the container. Based on physical understanding, we extract static force/torque value from the F/T sensor, vibration-related features and topple-related features from the newly designed high-speed GelSight tactile sensor to estimate those four particle properties. We test our method on $37$ very different daily particles, including powder, rice, beans, tablets, etc. Experiments show that our approach is able to estimate content mass with an error of $1.8$ g, content volume with an error of $6.1$ ml, particle size with an error of $1.1$ mm, and achieves an accuracy of $75.6$\% for particle shape estimation. In addition, our method can generalize to unseen particles with unknown volumes. By estimating these particle properties, our method can help robots to better perceive the granular media and help with different manipulation tasks in daily life and industry.
\end{abstract}


% \begin{abstract}
% Solid particles, such as beans or rice are commonly used\wenzhen{weird wording} in our daily lives. However, it remains a challenge for robots to estimate their properties.\wenzhen{Why robot need to do that?} In fact, the definition of this task itself: selecting the proper descriptor to describe the particle, is an open-problem. \joe{I am not sure the connection between this and the next sentence.} In one hand, human could get a rough estimation about the particles enclosed in the container by interacting with the container, on the other hand, dynamic tactile sensing has been applied to achieve precise estimation on liquid properties.\wenzhen{I can't see the connection between the two sides} This work utilized dynamic tactile sensing to estimate solid particles properties, specifically, particle weight, particle volume, particle size, and particle shape.\wenzhen{You skipped the question of ``defining the task''} We proposed a pipeline to use different actions to activate the whole system to estimate the particle properties step by step. Experiments show that our method could achieve xxg, xx mm, xx, and xx mm for mass estimation, volume estimation, shape estimation, and size estimation individually. In addition, we also show that our method can be generlized to unseen daily particles with unseen volume. Our method can estimate those four properties step by step. By estimating the particle macroscopic property, we believe out method will help robots to better perceive the granular media and help for potential manipulation task for daily life and industry.\wenzhen{The structure of the abstract is confusing and the logics is not coherent. You need to rewrite it}
% \end{abstract}







%%%%%%%%%%%%%%%%%%%%%%%%%%%%%%%%%%%%%%%%%%%%%%%%%%%%%%%%%%%%%%%%%%%%%%%%%%%%%%%%



\section{INTRODUCTION}
\label{Sec: intro}

% \wenzhen{For the first paragraph, you should reduce the less important information but talk more about relevant background information}
% \joe{I think you can condense the 2.5 sentences after "which is a challenging task due to ...". The shorter, the clearer.}
% \feng{modified}

% Tactile perception is important for robot and is broadly used for many different tasks. For example, robot can use tactile sensing to estimate the object hardness \cite{yuan2017shape}, object local geometry \cite{pezzementi2011tactile}, to detect
% slip during grasping \cite{romano2011human}, and recognize material textures by
% classifying the vibration patterns during contact \cite{sinapov2011vibrotactile}.\wenzhen{this seems to be kind of disconnect to the topic. One way you can address the background is that those perception tasks are based on ``direct contact'', but bottle shaking is kind of ``indirect perception''.} 

Granular media, also known as solid particles, are widely used in daily life, from construction materials like sand and stones, to culinary ingredients like spices powder and beans. Understanding their properties is important for humans to better utilize them. However, particles are commonly stored in containers like jars or bottles in daily life, making them challenging to perceive. Instead of pouring particles out and using vision, humans often use touch sensing to tell what is inside the container by interacting with the container dynamically. For example, by shaking the container, humans could roughly tell whether the particle is large or small; whether the particle is ball-shaped or irregular-shaped. However, it remains challenging for robots to estimate the property of particles in that case. 

In this work, we target estimating the property of the solid particles inside a container. Specifically, we try to estimate the content mass, content volume, particle size, and particle shape. We choose these four properties because they describe the basic geometry information of universal solid particles. The estimation of these properties helps people recognize the particles, whether they are seen or unseen, and helps to complete the following manipulation tasks such as pouring.


Previous efforts attempt to classify different solid particles enclosed in the container using audio \cite{eppe2018deep}, tactile \cite{chen2016learning}, reflectance spectroscopy \cite{hanson2022slurp}, or multi-modalities \cite{sinapov2014learning}. They manually designed the actions to be rotation \cite{jin2019open}, shaking \cite{chen2016learning}, squeezing \cite{guler2014s}, and so on. Although most of them achieved a high classification accuracy, they have poor generalizability and are hard to handle unseen particles. Most of them also require the volume of the particle to be the same as during training. To the best of our knowledge, there are no previous works on the estimation of the properties of the solid particles enclosed in the container, especially particle size and shape. In this work, we aim to estimate some basic particle properties so the unseen particles can be better described, which is a challenging task due to the complex dynamic of solid particles. Multiple particle properties such as density, humidity, and texture jointly influence the particle macro behavior, of which a precise model is still missing. Therefore, finding a perceptible particle behavior which directly related to individual particle property remains challenging.


% Figure environment removed


In this work, we propose a pipeline to use tactile sensing to estimate different properties of solid particles enclosed in the container, including content mass, content volume, particle size, and particle shape, as shown in Fig. \ref{fig:concept}. We utilize both static tactile sensing, which focuses on the F/T at a static state, and dynamic tactile sensing, which focuses on the temporal changing of tactile signals. To achieve better fingertip dynamic tactile sensing, we design a new tactile sensor: high-speed GelSight, which has both high spatial resolution ($640\times480$) and high temporal resolution ($815$ Hz). To acquire tactile data, we design a sequence of robot motions to interact with the container. The robot first lifts and tilts the container to different angles. The content mass and content volume are first estimated by measuring the static wrist F/T values at each rotation angle. Then the robot rotates the container in two directions at different speeds. We extract the vibration-related features and topple-related features separately from the dynamic tactile signals recorded from the new-designed high-speed GelSight during these two robot motions. We utilize these features to estimate the particle size and shape. 

 
The main contribution of this work is proposing a framework to solve a newly defined task: property estimation of the solid particles inside a container. To describe the particles inside the container, we estimate the content mass, content volume, particle size, and particle shape. We also contribute to designing a new tactile sensor: high-speed GelSight, which has both high temporal resolution and high spatial resolution. We collected the data from $37$ very different daily particles with different amounts in the container. The experiments show that our approach can estimate content mass with an error of $1.8$ g, content volume with an error of $6.1$ ml, particle size with an error of $1.1$ mm, and particle shape with an estimation accuracy of $75.6$\%. We also show that our method can be generalized to unseen daily particles with unseen volume. We believe our method can improve the ability of robots to perceive solid particles and help them perform various manipulation tasks in daily life and industry.

\wenzhen{I think the summarization of the contributions looks weak. E.g., I'm not convinced that the task definition in this specific case looks like a significant contribution. Besides, it's the first time you mention the high-speed GelSight. You probably want to highlight it earlier. You want to highlight what's the major achievement you make through this work.} \joe{agreed}

% First, we systematically define the task of solid particle property estimation. Second, we design and fabricate a high-speed GelSight tactile sensor to sense the fingertip vibration signals. Third, we propose a pipeline to estimate four properties of solid particles enclosed in the container step by step. 

% define task, why we need oarticle size and shape. task. new sensor

% However, comparing with liquid, the dynamic of solid particle is much more complex. Additionally, comparing with liquid, there are more particle properties\wenzhen{The comparison with liquid perception is less important if you want to save space} would influence the particle dynamic. Therefore, how to define the task and select the proper descriptor of solid particle is also challenging. 




% 1. solid particle is commonly used.
% 2. the estimation of solid particle is important.
% 3. In this wo rk, we targets on the xx task.
% 4. This work is challenging.
% 5. other works targets on classificaiton task.
% //6. human use tactile sensing to solve the task,
% 7. Our pipeline
% 8. our contribution
\section{RELATED WORKS}
\label{sec: relatedworks}

\subsection{Properties of Granular Media}

The study of granular media properties is an active research area within the physics community. Some work concluded the relationship between the granular media macroscale behavior to particle-scale properties, which inspires us when we define the robot motion. The angle of repose (AoR) is one common metric to describe particle behavior. It is commonly defined to be the steepest slope of the unconfined material, measured from the horizontal plane on which the material can be heaped without collapsing \cite{mehta1994dynamics}. AoR was found to be related to multiple particle properties of granular material, including the particle shape \cite{dai2016effect}, the particle size \cite{zhou2002experimental}, \cite{botz2003effects}, the particle friction coefficients \cite{santos2016investigation}, the content moisture \cite{zaalouk2009effect}, the number of particles \cite{miura1997method}, et al. Although multiple works concluded the relationship between AoR and particle properties, few works applied it to robot applications. In our work, we design one robot motion inspired by this and extract topple-related features which reflect the particles' AoR to estimate solid particle properties.

% \cite{do2017automated}, \cite{benvenuti2016identification} attempted to calibrate the particle simulators by measuring the particle AoR. \wenzhen{multiple grammar problems}

Very few previous works solved the problem that using robots to estimate solid particle properties. The only relevant work we found was by Matt \cite{matl2020inferring}, who used the pattern of the particle stacked rings to calibrate parameters of the particles, such as friction and restitution, for simulation. In our work, we define the particle property estimation task from a different perspective: we focus on the estimation of some explicit particle properties such as particle size and particle shape. These properties are more intuitive for humans and robots to get an understanding of those particles.



\subsection{Content Classification}

Although very few works targeted estimating particle properties, researchers have tried to recognize the particle type in opaque containers, which is a relevant easier task. Those methods provide us with some inspiration about how to distinguish particles with different properties. Those works used dynamic signals to recognize the content, which are collected with the robot motion such as shaking, dropping, and tapping \cite{sinapov2009interactive}, \cite{sinapov2011interactive}, \cite{griffith2012object}. The signals come from either audio or touch. For example, Eppe \cite{eppe2018deep} presented a strategy where the robot shakes the plastic capsules and uses auditory information to classify the contents. Jin \cite{jin2019open} collected the sound generated by rotating the bottles with different particles to classify the contents. Clarke \cite{clarke2018learning} used shaking sounds for estimating flow and amounts. Sinapov \cite{sinapov2014learning} proposed a framework to detect object categories by combining visual, auditory, and proprioceptive information. They explored 36 objects with different contents, weights, and colors by 10 behaviors. Chen \cite{chen2016learning} combined acoustic and acceleration information for content classification. There are also some works trying to estimate liquid properties and dynamic tactile sensing is found to be useful. Huang \cite{Huang-RSS-22} used dynamic tactile sensing to estimate liquid viscosity and height. Saal \cite{saal2010active} used tactile sensing to predict the liquid viscosity and optimized the shaking frequency and the rotation angle of shaking. Matl \cite{matl2019haptic} used F/T data and a model-based method for liquid property estimation and utilized the estimation for accurate pouring liquids.  Different from the above works, we target a more challenging task: estimating the quantity results of solid particle properties. Compared to semantic labels, the quantitative particle property estimation would be more helpful for following manipulation tasks and need to be generalized to unseen particles.

\section{PROBLEM STATEMENT}
\label{sec: problemstatement}
% Figure environment removed

% \wenzhen{this is not that obvious. You need to explain why. Also you can comment on the importance of measuring volume} 
\wenzhen{You need to discuss whether the method can be generalized to other containers} 


We aim to estimate the properties of solid particles in the container using only touch sensing. In this section, we define the targeting particle properties as the following: content mass $M_p$, content volume $V_p$, particle size $D_p$, and particle shape $S_p$. Some samples of the particles with different properties are shown in Fig. \ref{fig:problem_statement}. These properties are chosen because they are fundamental information to define the particles inside the container and useful perception information for manipulation tasks. For example, accurate measurement of the total volume of particles is essential for precise volume control in robotic pouring applications. 



% In this work, we assume the granular media in the container is homogeneous.
% To simplify the problem, we make several assumptions and definitions. First,\wenzhen{where are the second, third ...?} we assume the same type of particles are the same.
%For some solid particles such as candy and mints, this assumption is perfectly satisfied, while for some particles such as beans. this assumption may not be correct because there are individual differences between particles.

\noindent \textbf{Content mass $M_p$} is defined as the total mass of the particles inside the container.

%Given the same container whose mass is defined as $m_c$, the estimation of particle mass $m_p$ can be thought as equal to the estimation of the total weight $m_w = m_p + m_c$. 
% In the following experiments, we get $m_c$ by measuring the weight of the empty container and estimate $m_p$.

\noindent \textbf{Content volume $V_p$} is defined as the total volume of those particles and the room between their stacking. In the following experiment, we use the same container during training and testing. Therefore, $V_p$ is linearly correlated to the height $H_p$ of the particle stacks when holding the container vertically. To simplify, we will use content height $H_p$ to represent $V_p$. 

\noindent \textbf{Particle size $D_p$} is defined as the diameter of the sphere with the same volume as the given particle: $D_p = (\frac{6V}{\pi})^{\frac{1}{3}}$, where $V$ is the volume of the individual particle. 

\noindent \textbf{Particle shape $S_p$} is defined by the particle sphericity $\Psi_p$, which is commonly used to describe how similar the particle shape is compared to spheres. Sphericity is defined as the ratio of the surface area of a sphere with the same volume as the given particle to the surface area of the particle \cite{wadell1935volume}: $\Psi_p = \frac{\pi^{\frac{1}{3}}(6V)^{\frac{2}{3}}}{A}$, where $V$ is the volume of the individual particle and $A$ is its surface area. Sphericity is always a number between $0$ and $1$. The more spherical the shape, the closer its sphericity is to $1$. Since the sphericity distribution of common particles is skewed toward $1$ and is unbalanced, and the sphericity of powder is less meaningful, we re-define the shape descriptor to five classes using the following rules:

\begin{equation*}
S_p = \left\{
\begin{aligned}
0 & \hspace{0.5 in} (D_p \leq 1\text{mm})\\
1 & \hspace{0.5 in} (\Psi_p \leq 0.7 \hspace{0.1in} \text{and} \hspace{0.1in} D_p > 1\text{mm})\\
2 & \hspace{0.5 in} (0.7 < \Psi_p \leq 0.9  \hspace{0.1in} \text{and} \hspace{0.1in} D_p > 1\text{mm}) \\
3 & \hspace{0.5 in} (0.9 < \Psi_p \leq 0.96 \hspace{0.1in} \text{and} \hspace{0.1in} D_p > 1\text{mm}) \\
4 & \hspace{0.5 in} (0.96 < \Psi_p \leq 1 \hspace{0.1in} \text{and} \hspace{0.1in} D_p > 1\text{mm}) \end{aligned}
\right.
\end{equation*}
 
% The shape of powder particles is defined as $0$ because we observed powder usually performs like very irregular particles. \joe{I think this previous sentence can be removed. The previous sentence is clear enough.}


% For the particle whose shape is regular, we calculate their theoretical volume and surface area. For the particle whose shape is irregular, we assume its shape is an ellipsoid and measure the length of its 3 axes.\wenzhen{how is this paragraph connected to the previous one?}



% In this paper, we estimate the particle properties in multiple steps with different exploratory procedures. %In next section, we will describe how we estimate the particle property based on the tactile data acquired by some dynamic actions and the previous estimation on some properties.


\section{METHOD}
\label{sec: method}

To estimate the content mass, volume, particle size, and shape, we utilize both static and dynamic tactile sensing. We use different robot actions to interact with the container to estimate the properties. 
%When estimating the latter properties, we assume we have the estimated result of the former property. 
We mount a F/T sensor at the robot's wrist to get precise force and torque measurements. We attach a GelSight tactile sensor \cite{dong2017improved} and a newly designed high-speed GelSight tactile sensor on a two-fingered gripper to collect the fingertip tactile signals. The grasping force is set to be $25$ N to prevent slippage between the bottle and the fingertip. The specific setup will be introduced in Sec. \ref{sec: setup}.

% Figure environment removed


\subsection{Mass Estimation and Volume Estimation}
\label{mass_volume}

We estimate the content $M_p$ and the content volume $V_p$ using the wrist-mounted F/T sensor.

To estimate $M_p$, we measure the force change in the vertical direction $\Delta F_z$ after lifting the container. Then 
\begin{equation}\label{eq:mass}
M_p = \frac{\Delta F_z}{g} - M_c
\end{equation}
where $M_c$ is container mass, which is measured in advance.
% \wenzhen{why do you need to do this?} 

% Figure environment removed


We estimate content height $H_p$ to represent $V_p$. When tilting the container, particles with different $H_p$ have different center of mass (CoM) locations, resulting in different wrist torque. Specifically, after lifting the container vertically, the robot rotates the container with $\theta$ and shakes the container to make the particle stacking surface horizontal, as shown in the second column in Fig. \ref{fig:pipeline}. After rotation with $\theta$, the measured torque $T_y$ is determined by the content mass $M_p$ and the particles CoM x-position $d_{H_p}$. Additionally, $d_{H_p}$ is monotonically correlated to content height $H_p$, as shown in Fig. \ref{fig:volumepipeline}. Therefore, we have $H_p = H_p(T_y, M_p, \theta)$. $H_p(T_y, M_p, \theta)$ is non-linear due to the irregular wedge shape of the particle pile, as discussed in \cite{matl2019haptic}. Here we utilize the Multi-Layer Perception (MLP) to learn this function.

% We estimate content height $H_p$ to represent $V_p$. When tilting the container, particles with different $H_p$ have different center of mass (CoM) locations, resulting in different torque to the fingertip. Specifically, after lifting the container vertically, the robot rotates the container with $\theta$ and shakes the container to make the particle stacking surface horizontal, as shown in Fig. \ref{fig:volumepipeline} and the second column in Fig. \ref{fig:pipeline}. After rotation with $\theta$, the torque around the grasping point $T_y$ is affected by the particle mass $M_p$ and the particle CoM position: $T_y = M_pgd_c$. The particle CoM position is related to the content height $H_p$ and the rotation angle $\theta$, so we have $T_y(\theta,M_p, H_p) = M_pgd_c(\theta,H_p)$. $d_c$ is monotonically correlated to $H_p$ when $\theta$ fixed, as shown in Fig. \ref{fig:volumepipeline}. With a fixed grasping point (fixed $l$), we have the measured torque
% \begin{equation}
% \begin{split}
% T_y' & = T_y-M_pgd_g - M_od_g \\
% & = M_pg[d_c(\theta,H_p)-l\cos{\theta}]-M_ogl\cos{\theta}
% \end{split} 
% \end{equation}
% where $l$ is the distance between the F/T sensor y-axis and the grasping point y-axis and $M_o$ is the mass of the gripper and the empty bottle, both of which are constant, so we have $H_p = H_p(T_y', M_p, \theta)$. Since $H_p(T_y', M_p, \theta)$ is highly non-linear, we utilize MLP to learn this function.

In our method, we ask the robot to rotate the container to $30$, $45$, and $60$ degrees and measure the torque $T_y$ before shaking and after shaking at each angle. Those measured torques and the estimated particle mass $\Tilde{M_p}$ are used as the features to input to a 4-layer MLP to estimate $H_p$ and $V_p$. The size of two hidden layers of the MLP is set to be $16$ and $4$. We use MSE loss and Adam optimizer to train it.


%\joe{We need to explain why we need a 3-layer MLP here. It is natural for the readers to assume a model-based method works better.}\wenzhen{Agree. You need to provide some insight of why using this method, and why the measured signal is relevant to volumn.}
%\feng{Modified}
% \joe{I think this is overwhelmingly confusing. There is too much unnecessary math. Let me think of a more straightforward way to explain it.}
% Here we don't use the theoretical equation to calculate the volume directly because the fingertip GelSight requires lots of data for calibration to calculate absolute torque.


% We measure the solid particle volume by a similar way with the liquid volume estimation method described in \cite{matl2019haptic}. As shown in Fig. \ref{fig:massandvolumepipeline}. After lifting the container, we rotate the container for 45 degree. We know that:

% $T_y_1-T_y_1^e = M_pgr_1$, $T_y_2-T_y_2^e = M_pgr_2\sin{\frac{\pi}{4}}$.

% $$\sqrt{2}T_y_1-T_y_2 = -\sqrt{2}M_pg(y_b+\frac{h}{2}-\frac{a^2}{8h})+\sqrt{2}T_y_1^e-T_y_2^e$$

% where $T_y_1$ is the wrist torque measured before rotation and $T_y_2$ is the wrist torque measured after rotation. $T_y_i^e$ is the torque measure when the container is empty. $y_b$ is the container relative position and $a$ is the width of the container. We first measured those data for calibration. Then given the prior estimation of $m_p$ and the measurement of $T_y$, we calculate the particle total volume directly. \wenzhen{it seems that you are halfly done here}


% Different with liquid,\wenzhen{grammar} the surface of granular media won't\wenzhen{abbreviation} become a simple horizontal surface after rotation. We shake the container in the direction of contact force to make the surface be a horizontal surface.

% We use two different methods to estimate the particle volume: model-based method and data-driven method. \wenzhen{why you need a data-driven method?}

% In model-based method, we assume the shape and size of the container to be known. Here we use a cuboid container.



\subsection{Size Estimation and Shape Estimation} 
\label{section: shape_size}
% Estimating particles' individual properties relies on finding the relationship between particle macro-scale behavior and particle-scale parameters, as well as obtaining accurate perceptions of the particle macro-scale behavior. Both of them increase the difficulty of the particle property estimation task.
Then we estimate $D_p$ and $S_p$ by the newly designed high-speed GelSight, a vision-based tactile sensor, which will be introduced in detail in Sec. \ref{sec: design}. This sensor mainly consists of a soft elastomer with printed markers and an embedded camera. After making contact with objects, the soft elastomer deforms and the markers move, both of which will be captured by the embedded camera. Typically, the marker's motion in the images shows the local shear force applied to the soft elastomer. Additionally, we utilize physical inspirations and experiment findings to relate particle size and shape to two macro-scale particle behaviors: vibration intensity during colliding and stacking pattern during rotation. Then we design two robot motions and utilize dynamic tactile sensing to extract vibration-related features and topple-related features to estimate particle size and shape.

% In order to estimate particles' individual properties, the tactile signal needs to be collected when there are relative movements between individual particles instead of moving as a whole.

% Therefore, the design principle of our two exploratory procedures that estimate particle size and shape is to introduce relative movements between particles. We extract useful features from the tactile signals collected under these exploratory procedures. 


The first robot motion is to rotate the container from a nearly upside-down pose to an upward pose, shown as the fast rotation motion in Fig. \ref{fig:pipeline}. It aims to initiate collisions between particles and the container wall to generate vibration signals. Intuitively, the intensity of this vibration signal is positively correlated with particle size: large particles will lose more kinetic energy after colliding with the container walls, leading to a stronger contact force, and resulting in higher vibration intensity. Specifically, we rotate the container's long axis from $-135$ to $135$ degrees relative to the upward direction with $15$ degree/s. We collect the high-speed GelSight images during the period when the container's long axis is $-60$ to $0$ degrees relative to the upward direction. This is the duration that the particles fall and collide with the bottom of the container. We extract all the markers' position sequences from the high-speed GelSight images and use the top $30$ markers with the largest motion as the markers inside the contact area. We average their motions as the principle vibration signal $s(t), t\in[1, T]$. Fig. \ref{fig:vibration_intensity} shows examples of the vibration signal collected with different particles and also shows that larger particles usually have vibrations with higher intensity. We design two vibration-related features to capture the vibration intensity: $ v_{1_a} = \sum_{t=a+1}^T \lvert s(t)-s(t-a)\rvert, a = [1,\dots 50] $ and $v_{2_a} = \sum_{t=a+1}^{T-a} \lvert 2s(t)-s(t-a)-s(t+a)\rvert, a = [1,\dots 50]$. Intuitively, these features quantify the local variability of $s(t)$ over a specific time scale defined by the step size $a$. The vibration-related feature extraction pipeline is shown in Fig. \ref{fig:vibration_intensity_pipeline}.
\wenzhen{these are too details about the method. You need to give an overview of the major algorithm of your method, explain why it works, and then go to the details} 

% Figure environment removed


% Figure environment removed

The second robot motion is to slowly tilt the container to explore the stability of the particle stack at different tilting angles, shown as the slow rotation motion of Fig. \ref{fig:pipeline}. During rotation, after the tilting angle exceeds a certain threshold (the angle of repose (AoR) introduced in Sec. \ref{sec: relatedworks}), the particle stack collapses and reaches a new stacking state. The process repeats. Different particle stacking state has different particle CoM, causing different fingertip torque. Therefore, the discontinuous particle stack collapse causes the plot of the fingertip torque to the tilting angle to have a step-like shape, as shown in Fig. \ref{fig:stack_pattern}. The lower envelope of the torque signal (dotted pink line in Fig. \ref{fig:stack_pattern}) describes the maximum tilting angle for each stacking state that maintains stability. In practice, the exact step-like torque signal differs between repeated trials since the initial particles' states can be slightly different; but its lower envelope stays similar across trials (Fig. \ref{fig:collapse_pattern}) since it reflects particle AoR that is unaffected by the initial particles' state. Similarly, the upper envelope reflects the state after the particles collapse. Intuitively, both envelopes are related to the particle AoR and therefore highly relate to the particle shape and size. For example, spherical particles collapse easier and formulate the upper and lower envelope with less difference, as shown in Fig. \ref{fig:collapse_pattern}.

\wenzhen{the description is not clear enough}

Specifically, the robot rotates the container from $-60$ to $60$ degrees (third column of Fig. \ref{fig:pipeline}) and records the high-speed GelSight images during this process. We extract the rotation patterns of the high-speed GelSight markers to estimate the fingertip torque (note that the unit of measured torque is sensor-related and is not ${N}\cdot m$). We then extract the lower and upper envelope of the step-like torque signal and equally sample $100$ points from both envelopes to represent the topple-related feature.

% Figure environment removed

% \wenzhen{Start with a transaction}
% \feng{active not passive}

Both the two features: vibration-related features and topple-related features are correlated with the particle properties. Here we utilize learning methods to model the complex relationship between the features and the particle size and shape. Specifically, we stack the vibration-related features ($100$ dimensions), and the topple-related features ($200$ dimensions), together with the estimated content mass and content volume from Sec. \ref{mass_volume} as the total feature vectors. We input those features into a 4-layer MLP to estimate $D_p$ and into the Random Forests classifier for shape classification. The size of two hidden layers of the MLP is set to be $16$ and $4$. We use MSE loss and Adam optimizer in Pytorch to train it.









% \begin{comment}

% In addition to the vibration-related feature, we design an exploratory procedure to capture the particles' Angle of Repose (AoR). AoR [citation] of a granular material is the steepest surface angle of a pile of such particles that could maintain stability. For example, spherical particles often have smaller AoR since they tend to roll down on steep pile surfaces. We programmed the robot to rotate the container from $-60$ to $60$ degrees and record the GelSight images during this process. We extract the rotation patterns of the GelSight markers to estimate the torque. As we rotate the container, the surface angle of the particles will increase; when it exceeds the AoR, the particle pile will collapse, leading to a sudden change in the center of mass of the particles. Therefore, the plot of torque to the container tilting angle has a step-like shape shown in Figure X. In practice, the exact step-like torque signal differs between trials since the initial particles' states can be slightly different. While the torque signal can be unrepeatable, its lower and upper envelope (Figure X) should be repeatable since it reflects fundamental AoR-related properties. The lower envelope reflects the boundary between the stable (left) and unstable (right) region in the plot, and the upper envelope reflects the state of particle piles after collapses. We extract the lower and upper envelope of the step-like torque signal and equally sample $100$ points from both envelopes to represent the AoR-related feature.

% In addition to this vibration-related features, we also extract low-frequency related features, which described the stacking pattern\wenzhen{what is this?} of the particles during rotation. According to \cite{}, the particle stacking pattern during rotation can be described as the Angle of Repose (AoR), the largest angle that the stack could keep stable. Once the angle of the stack is larger than AoR, the stack would collapse. Particle size and particle shape are two important properties that would influence AoR \cite{}, \cite{}. Here, we design the action to be slow rotation\wenzhen{be exact} and extract this feature to estimation particle size and shape. Without vision information, it is almost impossible to get the 2D or 3D shape of the particle stack pattern.\wenzhen{verbose} Here we extract the center of mass position of the particles to describe the particle stack pattern. The detailed signal processing to extract the particle stack pattern is described in the next section. Theoretically, different particles will have differently patterned trajectories during rotation. However, even if repeating the same rotation process on the same particles, the stack pattern trajectory may also be different, due to the complex dynamic and granular media.\wenzhen{Everything is too abstract. You need to use figures and examples} Instead of directly using the stack pattern, we fit the upper envelope and lower envelope to the particle stack pattern, the lower envelope describe the boundary of when the stack pattern would collapse, and the upper envelope describes the boundary that what's the stack pattern like after collapse. We equally sampled 100 data points from each of the envelope to represent the stack pattern during rotation of different particles. 
% \end{comment}
%The stack pattern features and vibration features are first used to estimate the particle shape. 

%Then the vibration internsity features, together with the previous estimated particle properties, are used to estiamted the particle size.




% Figure environment removed

 % The gap between the upper and lower envelope is larger for particles with larger sizes and more irregular shapes.
% \feng{replace 'repeat trials', what's the step signal mean?}
% \wenzhen{you can put a picture of each particle on the figure}


% In shape and size estimation, we extract the 'vibration intensity' and 'stacking pattern' features from the fingertip Gelsights. 

% To extract the 'vibration intensity' feature, we used the markers' position sequence extracted from the high-speed gelsight. We first extract out the dropping period based on the robot end-effector angle. Then we ranked the markers’ motion magnitude and  use the top 30 markers with the largest motion (defined as average
% displacement magnitude per frame) as the markers inside the contact area. This avoids the feature overfitting to specific markers, i.e., specific contact area. Then we average the those markers motion to represent the principle vibrations $s(t)$. Then the vibration intensity is calculated as $ v_a = \sum_i=1^T s(t)-s(t-a), a \in (1,50) $.

% To extract the 'stacking pattern' feature, we used the markers' position sequence extracted from the low-speed gelsight. The markers motions is caused by the combination of three components: the rotation of gravity, the particle center of mass movement, and some bias component. The rotation of gravity would generate a constant rotating pattern, where the rotation angle is the end-effector rotation angle. The particle center of mass movement
% would generate a time-vary rotation pattern, where the rotation center is fixed. We decompose the raw markers motion into these three components by minimizing the bias term. Then the particle center of mass movement, which is a rotation pattern, is used as the raw stacking pattern. Then we used a 5th order polynomial function to fit the upper and lower envelope of that raw stacking pattern. The lower envelope is fitted by minimizing the raw signal part below and above it. 

% \begin{equation*}
% \begin{split}
% le(t) & = \arg\min_e(t) \sum_{[e(t)-s(t)]>0} [e(t)-s(t)] \\
% & + w\times\sum_{[e(t)-s(t)]<0}[-e(t)+s(t)]
% \end{split}
% \end{equation*}

% The upper envelope is fitted in a similar way. Then we equally sampled 100 data points from each of the envelope to represent the stack pattern during rotation of different particles.

\section{HIGH-SPEED GELSIGHT TACTILE SENSOR}
\label{sec: design}

% \wenzhen{At the beginning of the section, you should explain why you do this and a brief overview of how you do it}

% \wenzhen{Use the present tense to describe the work done}


In our method described in Sec. \ref{sec: method}, we relate particle size to fingertip vibration intensity. To sense the fingertip vibration signals, we design and fabricate a new tactile sensor: the high-speed GelSight tactile sensor. GelSight is a commonly used vision-based tactile sensor \cite{dong2017improved}. In their design, the sensor captures tactile images with $640\times480$ pixels resolution at $30$ Hz. However, the vibrations signal has a higher frequency and the human finger can sense signals up to approximately $700$ Hz \cite{dahiya2009tactile}. In our experiment, we found the frequency of the vibration signals caused by particle contact is over $100$ Hz. Based on that, we request a sensor that has a higher sampling rate.


In our design, we keep the basic structure of GelSight, but use a high-speed camera (A5031CU815 from HUARAY Tech.) to capture the high-frequency tactile signals. The camera captures images as fast as $815$ Hz. In this work, we capture the images at $800$ Hz with $640\times480$ spatial resolution. Additionally, we attach a $12$ mm lens to the camera and its field of view is about $18 \text{ mm}\times14 \text{ mm}$. 3 RGB LEDs of around $100$ lm are used to the lighting the gel pad. With respect to the gel pad, we fabricate a dome-shaped silicone pad for robust grasping and transfer a marker array with $1.7$ mm spacing to the surface of the gel pad. In total, there are around 70 markers in the field of view for tracking to represent the local motion of the gel pad. Fig. \ref{fig:experimentsetup}b shows the mechanical designs and real images of our tactile sensor. The massive tactile data will bring a higher computational burden when processed online, so we take all processing offline. Experiments in Sec. \ref{sec: comparison} show the benefits of using our newly designed sensor in this task.

% \wenzhen{This should be a contribution and you should highlight it in the introduction. And you need to highlight why high-speed GelSight is needed} 
% \joe{Maybe discussed why high-speed GelSight is needed. Readers can't tell from your v(t) that these are high frequency features.}

% \feng{brief talk we cannot make it online so that the processing is offline.}

%  \wenzhen{why is this in the hardware session?}




% \wenzhen{Add dimensions in the CAD and camera images}
% \input{4Experiment Setup}
\section{EXPERIMENT RESULTS}
\label{sec: results}


We conduct experiments to evaluate our pipeline's performance in estimating the four properties of the solid particles inside the container. In this section, we show the results for both seen and unseen particles. We also evaluate the influence of three key components of our system: the choice of sensors, actions, and features to extract from the raw data. 

% the following questions: (1) How well does our approach estimate those four particle properties (content mass, volume, particle size, and particle shape)? (2) Could our method estimate unseen particles with different volumes? (3) In our method, there are three key components different with other works: the newly designed high-speed GelSight, the well-defined actions, and the physical-inspired feature extraction process. How helpful to use them for this task?
% \wenzhen{these components are confusing to me-- it seems that they are not the major results, but only some secondary conclusion} 
% \feng{How about now?}
% we will show three key components of our pipeline that contributes to the results: (1) The high-speed GelSight which can sense the fingertip vibration signals. (2) The well-defined actions to activate the whole system to get better sensory data; (3) The feature extraction process inspired by the physical understanding.

% approach the following questions: (1) How well could our approach estimate particle properties? (2) Can our approach generalize to other unseen particles?


\subsection{Experiment Setup and Dataset}
\label{sec: setup}

The robot experiment setup is shown in Fig. \ref{fig:experimentsetup}a. We use a 6 DoF robotic arm (UR5E from Universal Robotics) and a parallel gripper (WSG50 from Weiss Robotics) to grasp the bottle. One Fingertip GelSight \cite{dong2017improved} and one newly designed High-Speed GelSight are mounted on the gripper. A 6-axis Force/Torque sensor (NRS-D50 from Nordbo Robotics) is mounted at the wrist.


% Figure environment removed

% % Figure environment removed


% Figure environment removed


% \wenzhen{avoid passive voice if you can} 


We collect a dataset of $37$ different common solid particles, as shown in Fig. \ref{fig:dataset}. These particles are different in terms of density, size, and shape. They are selected to cover a wide variety of particles in daily life. The particles are stored inside a common plastic cuboid bottle. In this work, we only focus on this particular type of container. We collect data at three levels of content heights $H_p$: $30 \text{mm}, 50\text{mm}, \text{and } 70\text{mm}$.
We perform the data collection process three times for each particle-height setting, formulating a dataset with $333$ data points. 

% \wenzhen{I'm confused here: in one of the experiment, you tested on new particles. Which means the division is in a different way, right?}\feng{yes, the division od VIB and VIC are different.}
% \wenzhen{It looks weird to me if this "dataset" part is only about the dataset used in VI.B, but not VI.C. Why don't you combine them?}
% \feng{I moved the splition part to each of the subsections}\wenzhen{The same with height estimation}

\subsection{Property Estimation with Seen Particles}

We first evaluate our method for estimating particle properties on seen particles. The experiment is conducted on all the $37$ particles in the dataset. Here we randomly split the $333$ data points into $80\%$ training data points and $20\%$ testing data points. The results are shown in Fig. \ref{fig:property_estimation_result}.

\subsubsection{Mass Estimation}
% We estimate mass $M_p$ using Equation \ref{eq:mass}. 
Our method estimates $M_p$ with mean absolute error (MAE) of $1.8$g, and mean absolute percentage error (MAPE) of $1.4$\%.
% The error mainly comes from the hardware noise, such as the inconsistent cable tension.\wenzhen{The second one is confusing. I suggest you to remove it if you don't want to explain why that's causing problems-- therotically it shouldn't}

\subsubsection{Volume Estimation}
We estimate $V_p$ by estimating $H_p$. Our approach estimates $H_p$ with MAE of $2.1$mm and MAPE of $4.3$\%, which represents the $V_p$ estimation MAE is about $6.1$ml and MAPE is $4.3$\%. Note that the ground truth of $H_p$ has a measurement error of roughly $1$ mm.

\subsubsection{Size Estimation}
The particle size $D_p$ estimation MAE is about $1.1$ mm, which is around the size difference between barley and green beans. For most spherical particles, the estimation error is small, while for the irregular particle, the estimation error is larger. There is a measurement error of $0.5$ mm for the groundtruth measurement. %The ground truth of the size has a measurement error of roughly 0.5 mm due to individual particle differences. 
The complex particle dynamic increases the particle behavior variance under small disturbances. In addition, multiple other particle parameters such as texture or friction coefficient would also influence the particle behavior. Some of the daily particles such as rosemary or crushed pepper are very light, decreasing the sensory data magnitude. All the above contribute to the estimation error.


\subsubsection{Shape Estimation}
We classify the particles into different shape groups, as defined in Sec. \ref{sec: problemstatement}. The classification accuracy is $75.6$\%. Our method recognizes most of the particles well. Most of the misclassification of particles happens to the particles whose shape is closer to spherical. This is because the sphericity difference of those particles is smaller and the other particle property such as stickiness would also affect the particle behavior.

% Most of the misclassification of particles happens to the adjacent class. 
% \joe{I see class 2 to class 4 though, why?}
% We train a 3-layer MLP to estimate the particle shape $S_p$. We use MSE loss and ask the model to predict the descriptor $S_p$ instead of the class type. Our approach estimates $S_p$ with MAE of $0.53$. \joe{While I know what you are talking about, this $S_p$ estimation error is particularly confusing.} 

% Figure environment removed


% \subsection{Generalize to Unseen Particles}
\subsection{Property Estimation with Unseen Particles}
% To show the generalizability of our approach, we test our method on particles not used for training. 
To evaluate the generalizability of our approach, we utilize it to estimate the properties of unseen particles. Among our dataset, we select $5$ types of particles for testing: ground coffee, rosemary, orzo, navy beans, and green beans. We use the data collected from the other $32$ particle types for training. To test the generalizability with volume estimation, for each test particle, we also collected 6 data points with different random volumes. So we have $288$ training data points and $75$ test data points.

The results are shown in Fig. \ref{fig:generalization_result}: MAEs of $M_p$, $H_p$, and $D_p$ estimation are $1.4$g, $2.0$mm, and $0.8$ mm, respectively. The particle shape classification accuracy is $64$\%. The result shows that our approach can estimate unseen particles' properties. For most particles, the estimation error of mass, volume, and size is in a similar range to the error testing on seen particles. The algorithm works well to recognize the shape of spherical particles such as navy beans and green beans but performs worse when estimating the shape of irregular particles. Most of the misclassification happens to adjacent groups, which are harder to distinguish.  

% \wenzhen{Check whether the analysis is updated according to the latest experiments}


% Figure environment removed


% In our method, we estimate the particle size and shape based on the prior estimation of mass and volume. To show the benefit of the prior estimation, here we make a comparison on size and shape estimation. We use the same training and test set in Sec\wenzhen{?} \joe{This paragraph is in general weird to me. Maybe need to rewrite it.}\wenzhen{Agree. Can't understand it at all.} but use different features to estimate particle size and shape. 1. We don't include the mass and volume in the input features; 2. We include the estimated mass and volume in the input features; 3. We include the ground truth mass and volume in the input features. The results of these three methods are 1.7 mm, 1.6 mm, 1.4 mm, individually. The results show that the prior estimation of mass and volume is helpful for estimating particle size and shape, and how precise the prior estimation is will influence the latter estimation. 

\begin{table*}[]
    \caption{Comparison of using different sensors, different actions, and different signal processing methods}
    \label{tab:comparison}
    \centering
    \begin{tabular}{c|c|c|c|c}
    \hline
    Tactile sensor & Actions & Signal processing method & size estimation MAE (mm) & shape classification accuracy \\
    \hline \hline
    Normal GelSight & \makecell{two dynamic robot motion \\ (our action)} & \makecell{feature engineering \\ (our method)} & 1.5 & 63.3\%\\
    \hline
    High-speed GelSight & horizontal shaking \cite{chen2016learning} & spectrum features \cite{chen2016learning} & 2.3 & 36.1\% \\
    \hline 
    High-speed GelSight & \makecell{two dynamic robot motion \\ (our action)} & MFCC + LSTM \cite{jonetzko2020multimodal} & 2.9 & 37.3\% \\
    \hline
    \textbf{High-speed GelSight} & \makecell{ \textbf{two dynamic robot motion} \\ \textbf{(our action)}} & \makecell{ \textbf{feature engineering} \\ \textbf{(our method)}} & \textbf{1.1} & \textbf{75.6\%} \\
    \hline
    \end{tabular}

\end{table*}


\subsection{Comparison with Other Methods}
\label{sec: comparison}
The good performance of our method relies on three key components: the use of high-speed GelSight, the optimized actions, and the physical-inspired feature extraction process. Here we evaluate the necessity of those components by comparing the property estimation results using different settings, and the results are shown in Table \ref{tab:comparison}. %conduct multiple comparisons to show their benefits.

\noindent\textbf{Choice of Sensor: }
We first make a comparison with if we only use the traditional GelSight sensor~\cite{dong2017improved} with a sampling rate of $30$ Hz. Using this setting, we lose some high-frequency vibration features during the fast rotation action. The result shows that using high-speed GelSight improves both size estimation and shape estimation accuracy. It also shows the high-frequency vibration features benefit the task.% and the newly-designed high-speed GelSight is helpful for the particle property estimation task.

\noindent\textbf{Choice of Action: }
In our method, we designed two rotation motions based on the desired perception of particle macro-scale behavior to estimate particle size and shape. Here we compare it with the baseline motion: shaking in one direction, which is a common action used to classify solid particles in previous work \cite{chen2016learning}, \cite{eppe2018deep}. Specifically, we ask the robot to shake the container horizontally at $2$ Hz and record dynamic tactile signals by high-speed GelSight. The magnitude of each frequency band is commonly used as the spectrum features of the vibration signal \cite{chen2016learning}. Here we extract the spectrum features from tactile signals of horizontal shaking and input it to a 4-layer MLP to estimate particle size and shape. The results show that by using our actions, the property estimation results improved significantly.
% \wenzhen{Here you also use a different signal processing method. Why that one? Explain it}
%Our estimation error is smaller, showing our designed actions get more sensory data than using horizontal shaking, proving that choosing a proper action to interact with the container is important. 
% \joe{I think this section can be simplified and shortened. You can condense the first 5 sentences into 1 or 2 sentences: We use x and y actions to trigger dynamic tactile signal to estimate x. In this section, we compare it with x.}
% We make a comparison using that motion on particle size and shape estimation. Fig. \ref{} shows the results by horizontal shaking motion. The size estimation error is around 2.8 mm, which is nearly random guessing. 

\noindent\textbf{Choice of Features: }
In our method, based on the physical inspiration, we choose vibration-related features and topple-related features for estimating the particle shape and size. We manually designed the signal processing and feature extraction process. 
%If lacking the physical understanding, a common way, as done by the previous work,
Instead, a common way to process temporal data is to use recurrent neural networks. Specifically, previous works \cite{eppe2018deep}, \cite{jonetzko2020multimodal} on particle classification employed Mel Frequency Cepstral Coefficients (MFCC) preprocessing to the signal and then input the Mel coefficients to LSTM for classification. Here we compare our results with their method. We apply MFCC to the tactile signals we collected during the two exploration processes mentioned in Sec. \ref{section: shape_size} and use LSTM for the estimation of particle size and shape.
% is to input the temporal sequence directly to the neural network. Table \ref{tab:comparison} shows the particle size and shape estimation results if we input the raw signals into LSTM. 
It turns out that the estimation error is significantly larger than ours. This shows that the physics-inspired features we use effectively extracted the important information relevant to the particle dynamics from the tactile signal.
% showing our hand-designed features are proper for estimating corresponding particle properties.
% \joe{I think the first two sentences can be condensed. You just need to mention that you are comparing your processing method with LSTM.}

\subsection{Sugar Humidity Estimation}


Other than the properties we described before, some properties are specific to only certain types of particles but are highly useful at the function level. A typical example is the humidity of sugar, which helps humans to handle them properly. 
% not commonly used to describe all solid particles, such as particle stickiness and humidity, but the estimation of them to some specific particles is also important.
% \feng{jump to the most important thing}
% \wenzhen{the logic flow is weird}
% \joe{I think you can replace the previous sentence to "We show that our pipeline can estimate other particle properties such as sugar stickiness."}
In this subsection, we apply our method to estimate sugar humidity. According to \cite{fraysse1999humidity}, the granular media AoR is also related to particle humidity, inspiring us to use the topple-related features to infer the sugar humidity since these features reflect the particle AoR. Intuitively, higher humidity force sugar to be sticky and hard to collapse, while the dry sugar would flow more like the liquid.
To magnify the collapse pattern of the sticky sugar, we use a rotation motion from $-135$ to $135$ degrees for the exploratory procedure and record the tactile signals by the high-speed GelSight. Then we extract the topple-related features in the same way described in Sec. \ref{section: shape_size}. We then input those features into a 3-layer MLP to estimate the sugar humidity.

To collect a dataset for the task, we manually drop $0.1$, $0.2$, $0.3$, $0.4$, and $0.5$ ml of water into the $150$ g fine sugar powder to create the sugar with $5$ different humidity levels. We use the volume of water to denote the humidity level in this specific setting. 
% Our objective is to estimate the volume of water mixed with the sugar. 
Fig. \ref{fig:sugar_stacking_pattern} shows the step-like torque signals collected in $3$ groups of humid sugar. Sugar with higher humidity has a larger portion to be sticky to the container wall and never collapses, resulting in a lesser variation in the step-like signals. We repeat the exploratory procedure on the $5$ groups of humid sugar with 10 trials. Fig. \ref{fig:sugar_result}a shows the estimation result if we randomly split the whole dataset into $80$\% training dataset and $20$\% testing dataset. Our estimation of the added water has an MAE of $0.026$ ml. Fig. \ref{fig:sugar_result}b shows the estimation result if we train on humid sugar with $0.1$, $0.3$, $0.5$ ml of water and test on humid sugar with $0.2$, $0.4$ ml of water. Our estimation of the added water has an MAE of $0.043$ ml. This experiment shows the potential of our methodology to estimate other particle properties.

% \wenzhen{Can you describe the difference in the signal cased by the sugar humidity?}
%Our previous experiments show that the solid particle dynamic is very complex\wenzhen{no you didn't show that} and a prior estimation of some property (particle mass, volume) can help in estimating other properties (size). We wonder what if we could have stronger prior knowledge such as the specific particle type, could we reduce the variance and get a better estimation?\wenzhen{This motivation part is very weird. I suggest rewriting it.}\joe{okay}\joe{Can we motivate this from the application point of view? In the kitchen, we will need to know the sugar humidity} \wenzhen{That should work. Try to make it intuitive} \joe{It will be slightly disconnect from the two previous experiments though. Is that okay?}\wenzhen{You are doing what you can in two hours}Specifically, we target the sugar particle and try to estimate the humidity of the sugar. 
% Recall that the AoR is also related to particle humidity. Therefore, here we use the stacking pattern feature to estimate how much water is there in the sugar powder. 
% Given the fact that the humid sugar is too sticky to have movement when the end-effector only rotates from -60 degrees to 60 degrees, w
%We enlarge the rotation range to -135 degrees to 135 degrees for generating a more obvious movement. We manually drop 0.1 ml to 0.5 ml of water into the 150g fine sugar powder and shake the bottle vigorously to make the sugar with different humidity. Fig. \ref{}\wenzhen{...} shows the stacking pattern signals. We repeat each group with 10 trials. Fig. \ref{} shows the estimation result if we random split and whole dataset into 80\% training set and 20\% test. The water estimation MAE is 0.26 ml. Fig. \ref{} shows the estimation result if we trained on 0.1 ml, 0.3 ml, 0.5 ml water and tested on 0.2 ml, 0.4 ml water. The water estimation MAE is 0.43 ml.







% Figure environment removed
\feng{include the insight in the captions as well}
\feng{modified}

% Figure environment removed

\wenzhen{caption of fig.14 is unclear}
\feng{modified}
%   (a) Water amount estimation results if we randomly split training and test set. (b) Water amount estimation results if tests on the unseen humid sugar
% \subsection{Comparison}


% \subsection{Discussion}


% \subsection{Comparison with Audio Sensing}


\section{CONCLUSION}
\label{sec: conclusion}

In this work, we present an approach to estimate the properties of solid particles in the container. We combine static and dynamic tactile sensing to estimate content mass, content volume, particle size, and particle shape. We design a sequence of exploratory procedures to interact with the container to estimate those four properties. In the experiment, our approach can estimate those properties, achieving an accuracy of $1.8$ g, $6.1$ ml, $1.1$ mm, and $75.6$\% for mass, volume, size, and shape estimation. Additionally, we show that our method can generalize to unseen daily particles with unseen volumes. We also show that the physical-inspired actions and features can be used to estimate sugar humidity. We believe our explorations on solid particle property estimation will help with different manipulation tasks in daily life and industry.


% \include{7Appendix}





% \section*{ACKNOWLEDGMENT}


% \newpage

\bibliographystyle{IEEEtran}
\bibliography{root}



\end{document}
