\section{INTRODUCTION}
\label{Sec: intro}

% \wenzhen{For the first paragraph, you should reduce the less important information but talk more about relevant background information}
% \joe{I think you can condense the 2.5 sentences after "which is a challenging task due to ...". The shorter, the clearer.}
% \feng{modified}

% Tactile perception is important for robot and is broadly used for many different tasks. For example, robot can use tactile sensing to estimate the object hardness \cite{yuan2017shape}, object local geometry \cite{pezzementi2011tactile}, to detect
% slip during grasping \cite{romano2011human}, and recognize material textures by
% classifying the vibration patterns during contact \cite{sinapov2011vibrotactile}.\wenzhen{this seems to be kind of disconnect to the topic. One way you can address the background is that those perception tasks are based on ``direct contact'', but bottle shaking is kind of ``indirect perception''.} 

Granular media, also known as solid particles, are widely used in daily life, from construction materials like sand and stones, to culinary ingredients like spices powder and beans. Understanding their properties is important for humans to better utilize them. However, particles are commonly stored in containers like jars or bottles in daily life, making them challenging to perceive. Instead of pouring particles out and using vision, humans often use touch sensing to tell what is inside the container by interacting with the container dynamically. For example, by shaking the container, humans could roughly tell whether the particle is large or small; whether the particle is ball-shaped or irregular-shaped. However, it remains challenging for robots to estimate the property of particles in that case. 

In this work, we target estimating the property of the solid particles inside a container. Specifically, we try to estimate the content mass, content volume, particle size, and particle shape. We choose these four properties because they describe the basic geometry information of universal solid particles. The estimation of these properties helps people recognize the particles, whether they are seen or unseen, and helps to complete the following manipulation tasks such as pouring.


Previous efforts attempt to classify different solid particles enclosed in the container using audio \cite{eppe2018deep}, tactile \cite{chen2016learning}, reflectance spectroscopy \cite{hanson2022slurp}, or multi-modalities \cite{sinapov2014learning}. They manually designed the actions to be rotation \cite{jin2019open}, shaking \cite{chen2016learning}, squeezing \cite{guler2014s}, and so on. Although most of them achieved a high classification accuracy, they have poor generalizability and are hard to handle unseen particles. Most of them also require the volume of the particle to be the same as during training. To the best of our knowledge, there are no previous works on the estimation of the properties of the solid particles enclosed in the container, especially particle size and shape. In this work, we aim to estimate some basic particle properties so the unseen particles can be better described, which is a challenging task due to the complex dynamic of solid particles. Multiple particle properties such as density, humidity, and texture jointly influence the particle macro behavior, of which a precise model is still missing. Therefore, finding a perceptible particle behavior which directly related to individual particle property remains challenging.


% Figure environment removed


In this work, we propose a pipeline to use tactile sensing to estimate different properties of solid particles enclosed in the container, including content mass, content volume, particle size, and particle shape, as shown in Fig. \ref{fig:concept}. We utilize both static tactile sensing, which focuses on the F/T at a static state, and dynamic tactile sensing, which focuses on the temporal changing of tactile signals. To achieve better fingertip dynamic tactile sensing, we design a new tactile sensor: high-speed GelSight, which has both high spatial resolution ($640\times480$) and high temporal resolution ($815$ Hz). To acquire tactile data, we design a sequence of robot motions to interact with the container. The robot first lifts and tilts the container to different angles. The content mass and content volume are first estimated by measuring the static wrist F/T values at each rotation angle. Then the robot rotates the container in two directions at different speeds. We extract the vibration-related features and topple-related features separately from the dynamic tactile signals recorded from the new-designed high-speed GelSight during these two robot motions. We utilize these features to estimate the particle size and shape. 

 
The main contribution of this work is proposing a framework to solve a newly defined task: property estimation of the solid particles inside a container. To describe the particles inside the container, we estimate the content mass, content volume, particle size, and particle shape. We also contribute to designing a new tactile sensor: high-speed GelSight, which has both high temporal resolution and high spatial resolution. We collected the data from $37$ very different daily particles with different amounts in the container. The experiments show that our approach can estimate content mass with an error of $1.8$ g, content volume with an error of $6.1$ ml, particle size with an error of $1.1$ mm, and particle shape with an estimation accuracy of $75.6$\%. We also show that our method can be generalized to unseen daily particles with unseen volume. We believe our method can improve the ability of robots to perceive solid particles and help them perform various manipulation tasks in daily life and industry.

\wenzhen{I think the summarization of the contributions looks weak. E.g., I'm not convinced that the task definition in this specific case looks like a significant contribution. Besides, it's the first time you mention the high-speed GelSight. You probably want to highlight it earlier. You want to highlight what's the major achievement you make through this work.} \joe{agreed}

% First, we systematically define the task of solid particle property estimation. Second, we design and fabricate a high-speed GelSight tactile sensor to sense the fingertip vibration signals. Third, we propose a pipeline to estimate four properties of solid particles enclosed in the container step by step. 

% define task, why we need oarticle size and shape. task. new sensor

% However, comparing with liquid, the dynamic of solid particle is much more complex. Additionally, comparing with liquid, there are more particle properties\wenzhen{The comparison with liquid perception is less important if you want to save space} would influence the particle dynamic. Therefore, how to define the task and select the proper descriptor of solid particle is also challenging. 




% 1. solid particle is commonly used.
% 2. the estimation of solid particle is important.
% 3. In this wo rk, we targets on the xx task.
% 4. This work is challenging.
% 5. other works targets on classificaiton task.
% //6. human use tactile sensing to solve the task,
% 7. Our pipeline
% 8. our contribution