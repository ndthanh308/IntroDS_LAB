\section{HIGH-SPEED GELSIGHT TACTILE SENSOR}
\label{sec: design}

% \wenzhen{At the beginning of the section, you should explain why you do this and a brief overview of how you do it}

% \wenzhen{Use the present tense to describe the work done}


In our method described in Sec. \ref{sec: method}, we relate particle size to fingertip vibration intensity. To sense the fingertip vibration signals, we design and fabricate a new tactile sensor: the high-speed GelSight tactile sensor. GelSight is a commonly used vision-based tactile sensor \cite{dong2017improved}. In their design, the sensor captures tactile images with $640\times480$ pixels resolution at $30$ Hz. However, the vibrations signal has a higher frequency and the human finger can sense signals up to approximately $700$ Hz \cite{dahiya2009tactile}. In our experiment, we found the frequency of the vibration signals caused by particle contact is over $100$ Hz. Based on that, we request a sensor that has a higher sampling rate.


In our design, we keep the basic structure of GelSight, but use a high-speed camera (A5031CU815 from HUARAY Tech.) to capture the high-frequency tactile signals. The camera captures images as fast as $815$ Hz. In this work, we capture the images at $800$ Hz with $640\times480$ spatial resolution. Additionally, we attach a $12$ mm lens to the camera and its field of view is about $18 \text{ mm}\times14 \text{ mm}$. 3 RGB LEDs of around $100$ lm are used to the lighting the gel pad. With respect to the gel pad, we fabricate a dome-shaped silicone pad for robust grasping and transfer a marker array with $1.7$ mm spacing to the surface of the gel pad. In total, there are around 70 markers in the field of view for tracking to represent the local motion of the gel pad. Fig. \ref{fig:experimentsetup}b shows the mechanical designs and real images of our tactile sensor. The massive tactile data will bring a higher computational burden when processed online, so we take all processing offline. Experiments in Sec. \ref{sec: comparison} show the benefits of using our newly designed sensor in this task.

% \wenzhen{This should be a contribution and you should highlight it in the introduction. And you need to highlight why high-speed GelSight is needed} 
% \joe{Maybe discussed why high-speed GelSight is needed. Readers can't tell from your v(t) that these are high frequency features.}

% \feng{brief talk we cannot make it online so that the processing is offline.}

%  \wenzhen{why is this in the hardware session?}




% \wenzhen{Add dimensions in the CAD and camera images}