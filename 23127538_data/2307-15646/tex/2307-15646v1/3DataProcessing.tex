\section{METHOD}
\label{sec: method}

To estimate the content mass, volume, particle size, and shape, we utilize both static and dynamic tactile sensing. We use different robot actions to interact with the container to estimate the properties. 
%When estimating the latter properties, we assume we have the estimated result of the former property. 
We mount a F/T sensor at the robot's wrist to get precise force and torque measurements. We attach a GelSight tactile sensor \cite{dong2017improved} and a newly designed high-speed GelSight tactile sensor on a two-fingered gripper to collect the fingertip tactile signals. The grasping force is set to be $25$ N to prevent slippage between the bottle and the fingertip. The specific setup will be introduced in Sec. \ref{sec: setup}.

% Figure environment removed


\subsection{Mass Estimation and Volume Estimation}
\label{mass_volume}

We estimate the content $M_p$ and the content volume $V_p$ using the wrist-mounted F/T sensor.

To estimate $M_p$, we measure the force change in the vertical direction $\Delta F_z$ after lifting the container. Then 
\begin{equation}\label{eq:mass}
M_p = \frac{\Delta F_z}{g} - M_c
\end{equation}
where $M_c$ is container mass, which is measured in advance.
% \wenzhen{why do you need to do this?} 

% Figure environment removed


We estimate content height $H_p$ to represent $V_p$. When tilting the container, particles with different $H_p$ have different center of mass (CoM) locations, resulting in different wrist torque. Specifically, after lifting the container vertically, the robot rotates the container with $\theta$ and shakes the container to make the particle stacking surface horizontal, as shown in the second column in Fig. \ref{fig:pipeline}. After rotation with $\theta$, the measured torque $T_y$ is determined by the content mass $M_p$ and the particles CoM x-position $d_{H_p}$. Additionally, $d_{H_p}$ is monotonically correlated to content height $H_p$, as shown in Fig. \ref{fig:volumepipeline}. Therefore, we have $H_p = H_p(T_y, M_p, \theta)$. $H_p(T_y, M_p, \theta)$ is non-linear due to the irregular wedge shape of the particle pile, as discussed in \cite{matl2019haptic}. Here we utilize the Multi-Layer Perception (MLP) to learn this function.

% We estimate content height $H_p$ to represent $V_p$. When tilting the container, particles with different $H_p$ have different center of mass (CoM) locations, resulting in different torque to the fingertip. Specifically, after lifting the container vertically, the robot rotates the container with $\theta$ and shakes the container to make the particle stacking surface horizontal, as shown in Fig. \ref{fig:volumepipeline} and the second column in Fig. \ref{fig:pipeline}. After rotation with $\theta$, the torque around the grasping point $T_y$ is affected by the particle mass $M_p$ and the particle CoM position: $T_y = M_pgd_c$. The particle CoM position is related to the content height $H_p$ and the rotation angle $\theta$, so we have $T_y(\theta,M_p, H_p) = M_pgd_c(\theta,H_p)$. $d_c$ is monotonically correlated to $H_p$ when $\theta$ fixed, as shown in Fig. \ref{fig:volumepipeline}. With a fixed grasping point (fixed $l$), we have the measured torque
% \begin{equation}
% \begin{split}
% T_y' & = T_y-M_pgd_g - M_od_g \\
% & = M_pg[d_c(\theta,H_p)-l\cos{\theta}]-M_ogl\cos{\theta}
% \end{split} 
% \end{equation}
% where $l$ is the distance between the F/T sensor y-axis and the grasping point y-axis and $M_o$ is the mass of the gripper and the empty bottle, both of which are constant, so we have $H_p = H_p(T_y', M_p, \theta)$. Since $H_p(T_y', M_p, \theta)$ is highly non-linear, we utilize MLP to learn this function.

In our method, we ask the robot to rotate the container to $30$, $45$, and $60$ degrees and measure the torque $T_y$ before shaking and after shaking at each angle. Those measured torques and the estimated particle mass $\Tilde{M_p}$ are used as the features to input to a 4-layer MLP to estimate $H_p$ and $V_p$. The size of two hidden layers of the MLP is set to be $16$ and $4$. We use MSE loss and Adam optimizer to train it.


%\joe{We need to explain why we need a 3-layer MLP here. It is natural for the readers to assume a model-based method works better.}\wenzhen{Agree. You need to provide some insight of why using this method, and why the measured signal is relevant to volumn.}
%\feng{Modified}
% \joe{I think this is overwhelmingly confusing. There is too much unnecessary math. Let me think of a more straightforward way to explain it.}
% Here we don't use the theoretical equation to calculate the volume directly because the fingertip GelSight requires lots of data for calibration to calculate absolute torque.


% We measure the solid particle volume by a similar way with the liquid volume estimation method described in \cite{matl2019haptic}. As shown in Fig. \ref{fig:massandvolumepipeline}. After lifting the container, we rotate the container for 45 degree. We know that:

% $T_y_1-T_y_1^e = M_pgr_1$, $T_y_2-T_y_2^e = M_pgr_2\sin{\frac{\pi}{4}}$.

% $$\sqrt{2}T_y_1-T_y_2 = -\sqrt{2}M_pg(y_b+\frac{h}{2}-\frac{a^2}{8h})+\sqrt{2}T_y_1^e-T_y_2^e$$

% where $T_y_1$ is the wrist torque measured before rotation and $T_y_2$ is the wrist torque measured after rotation. $T_y_i^e$ is the torque measure when the container is empty. $y_b$ is the container relative position and $a$ is the width of the container. We first measured those data for calibration. Then given the prior estimation of $m_p$ and the measurement of $T_y$, we calculate the particle total volume directly. \wenzhen{it seems that you are halfly done here}


% Different with liquid,\wenzhen{grammar} the surface of granular media won't\wenzhen{abbreviation} become a simple horizontal surface after rotation. We shake the container in the direction of contact force to make the surface be a horizontal surface.

% We use two different methods to estimate the particle volume: model-based method and data-driven method. \wenzhen{why you need a data-driven method?}

% In model-based method, we assume the shape and size of the container to be known. Here we use a cuboid container.



\subsection{Size Estimation and Shape Estimation} 
\label{section: shape_size}
% Estimating particles' individual properties relies on finding the relationship between particle macro-scale behavior and particle-scale parameters, as well as obtaining accurate perceptions of the particle macro-scale behavior. Both of them increase the difficulty of the particle property estimation task.
Then we estimate $D_p$ and $S_p$ by the newly designed high-speed GelSight, a vision-based tactile sensor, which will be introduced in detail in Sec. \ref{sec: design}. This sensor mainly consists of a soft elastomer with printed markers and an embedded camera. After making contact with objects, the soft elastomer deforms and the markers move, both of which will be captured by the embedded camera. Typically, the marker's motion in the images shows the local shear force applied to the soft elastomer. Additionally, we utilize physical inspirations and experiment findings to relate particle size and shape to two macro-scale particle behaviors: vibration intensity during colliding and stacking pattern during rotation. Then we design two robot motions and utilize dynamic tactile sensing to extract vibration-related features and topple-related features to estimate particle size and shape.

% In order to estimate particles' individual properties, the tactile signal needs to be collected when there are relative movements between individual particles instead of moving as a whole.

% Therefore, the design principle of our two exploratory procedures that estimate particle size and shape is to introduce relative movements between particles. We extract useful features from the tactile signals collected under these exploratory procedures. 


The first robot motion is to rotate the container from a nearly upside-down pose to an upward pose, shown as the fast rotation motion in Fig. \ref{fig:pipeline}. It aims to initiate collisions between particles and the container wall to generate vibration signals. Intuitively, the intensity of this vibration signal is positively correlated with particle size: large particles will lose more kinetic energy after colliding with the container walls, leading to a stronger contact force, and resulting in higher vibration intensity. Specifically, we rotate the container's long axis from $-135$ to $135$ degrees relative to the upward direction with $15$ degree/s. We collect the high-speed GelSight images during the period when the container's long axis is $-60$ to $0$ degrees relative to the upward direction. This is the duration that the particles fall and collide with the bottom of the container. We extract all the markers' position sequences from the high-speed GelSight images and use the top $30$ markers with the largest motion as the markers inside the contact area. We average their motions as the principle vibration signal $s(t), t\in[1, T]$. Fig. \ref{fig:vibration_intensity} shows examples of the vibration signal collected with different particles and also shows that larger particles usually have vibrations with higher intensity. We design two vibration-related features to capture the vibration intensity: $ v_{1_a} = \sum_{t=a+1}^T \lvert s(t)-s(t-a)\rvert, a = [1,\dots 50] $ and $v_{2_a} = \sum_{t=a+1}^{T-a} \lvert 2s(t)-s(t-a)-s(t+a)\rvert, a = [1,\dots 50]$. Intuitively, these features quantify the local variability of $s(t)$ over a specific time scale defined by the step size $a$. The vibration-related feature extraction pipeline is shown in Fig. \ref{fig:vibration_intensity_pipeline}.
\wenzhen{these are too details about the method. You need to give an overview of the major algorithm of your method, explain why it works, and then go to the details} 

% Figure environment removed


% Figure environment removed

The second robot motion is to slowly tilt the container to explore the stability of the particle stack at different tilting angles, shown as the slow rotation motion of Fig. \ref{fig:pipeline}. During rotation, after the tilting angle exceeds a certain threshold (the angle of repose (AoR) introduced in Sec. \ref{sec: relatedworks}), the particle stack collapses and reaches a new stacking state. The process repeats. Different particle stacking state has different particle CoM, causing different fingertip torque. Therefore, the discontinuous particle stack collapse causes the plot of the fingertip torque to the tilting angle to have a step-like shape, as shown in Fig. \ref{fig:stack_pattern}. The lower envelope of the torque signal (dotted pink line in Fig. \ref{fig:stack_pattern}) describes the maximum tilting angle for each stacking state that maintains stability. In practice, the exact step-like torque signal differs between repeated trials since the initial particles' states can be slightly different; but its lower envelope stays similar across trials (Fig. \ref{fig:collapse_pattern}) since it reflects particle AoR that is unaffected by the initial particles' state. Similarly, the upper envelope reflects the state after the particles collapse. Intuitively, both envelopes are related to the particle AoR and therefore highly relate to the particle shape and size. For example, spherical particles collapse easier and formulate the upper and lower envelope with less difference, as shown in Fig. \ref{fig:collapse_pattern}.

\wenzhen{the description is not clear enough}

Specifically, the robot rotates the container from $-60$ to $60$ degrees (third column of Fig. \ref{fig:pipeline}) and records the high-speed GelSight images during this process. We extract the rotation patterns of the high-speed GelSight markers to estimate the fingertip torque (note that the unit of measured torque is sensor-related and is not ${N}\cdot m$). We then extract the lower and upper envelope of the step-like torque signal and equally sample $100$ points from both envelopes to represent the topple-related feature.

% Figure environment removed

% \wenzhen{Start with a transaction}
% \feng{active not passive}

Both the two features: vibration-related features and topple-related features are correlated with the particle properties. Here we utilize learning methods to model the complex relationship between the features and the particle size and shape. Specifically, we stack the vibration-related features ($100$ dimensions), and the topple-related features ($200$ dimensions), together with the estimated content mass and content volume from Sec. \ref{mass_volume} as the total feature vectors. We input those features into a 4-layer MLP to estimate $D_p$ and into the Random Forests classifier for shape classification. The size of two hidden layers of the MLP is set to be $16$ and $4$. We use MSE loss and Adam optimizer in Pytorch to train it.









% \begin{comment}

% In addition to the vibration-related feature, we design an exploratory procedure to capture the particles' Angle of Repose (AoR). AoR [citation] of a granular material is the steepest surface angle of a pile of such particles that could maintain stability. For example, spherical particles often have smaller AoR since they tend to roll down on steep pile surfaces. We programmed the robot to rotate the container from $-60$ to $60$ degrees and record the GelSight images during this process. We extract the rotation patterns of the GelSight markers to estimate the torque. As we rotate the container, the surface angle of the particles will increase; when it exceeds the AoR, the particle pile will collapse, leading to a sudden change in the center of mass of the particles. Therefore, the plot of torque to the container tilting angle has a step-like shape shown in Figure X. In practice, the exact step-like torque signal differs between trials since the initial particles' states can be slightly different. While the torque signal can be unrepeatable, its lower and upper envelope (Figure X) should be repeatable since it reflects fundamental AoR-related properties. The lower envelope reflects the boundary between the stable (left) and unstable (right) region in the plot, and the upper envelope reflects the state of particle piles after collapses. We extract the lower and upper envelope of the step-like torque signal and equally sample $100$ points from both envelopes to represent the AoR-related feature.

% In addition to this vibration-related features, we also extract low-frequency related features, which described the stacking pattern\wenzhen{what is this?} of the particles during rotation. According to \cite{}, the particle stacking pattern during rotation can be described as the Angle of Repose (AoR), the largest angle that the stack could keep stable. Once the angle of the stack is larger than AoR, the stack would collapse. Particle size and particle shape are two important properties that would influence AoR \cite{}, \cite{}. Here, we design the action to be slow rotation\wenzhen{be exact} and extract this feature to estimation particle size and shape. Without vision information, it is almost impossible to get the 2D or 3D shape of the particle stack pattern.\wenzhen{verbose} Here we extract the center of mass position of the particles to describe the particle stack pattern. The detailed signal processing to extract the particle stack pattern is described in the next section. Theoretically, different particles will have differently patterned trajectories during rotation. However, even if repeating the same rotation process on the same particles, the stack pattern trajectory may also be different, due to the complex dynamic and granular media.\wenzhen{Everything is too abstract. You need to use figures and examples} Instead of directly using the stack pattern, we fit the upper envelope and lower envelope to the particle stack pattern, the lower envelope describe the boundary of when the stack pattern would collapse, and the upper envelope describes the boundary that what's the stack pattern like after collapse. We equally sampled 100 data points from each of the envelope to represent the stack pattern during rotation of different particles. 
% \end{comment}
%The stack pattern features and vibration features are first used to estimate the particle shape. 

%Then the vibration internsity features, together with the previous estimated particle properties, are used to estiamted the particle size.




% Figure environment removed

 % The gap between the upper and lower envelope is larger for particles with larger sizes and more irregular shapes.
% \feng{replace 'repeat trials', what's the step signal mean?}
% \wenzhen{you can put a picture of each particle on the figure}


% In shape and size estimation, we extract the 'vibration intensity' and 'stacking pattern' features from the fingertip Gelsights. 

% To extract the 'vibration intensity' feature, we used the markers' position sequence extracted from the high-speed gelsight. We first extract out the dropping period based on the robot end-effector angle. Then we ranked the markers’ motion magnitude and  use the top 30 markers with the largest motion (defined as average
% displacement magnitude per frame) as the markers inside the contact area. This avoids the feature overfitting to specific markers, i.e., specific contact area. Then we average the those markers motion to represent the principle vibrations $s(t)$. Then the vibration intensity is calculated as $ v_a = \sum_i=1^T s(t)-s(t-a), a \in (1,50) $.

% To extract the 'stacking pattern' feature, we used the markers' position sequence extracted from the low-speed gelsight. The markers motions is caused by the combination of three components: the rotation of gravity, the particle center of mass movement, and some bias component. The rotation of gravity would generate a constant rotating pattern, where the rotation angle is the end-effector rotation angle. The particle center of mass movement
% would generate a time-vary rotation pattern, where the rotation center is fixed. We decompose the raw markers motion into these three components by minimizing the bias term. Then the particle center of mass movement, which is a rotation pattern, is used as the raw stacking pattern. Then we used a 5th order polynomial function to fit the upper and lower envelope of that raw stacking pattern. The lower envelope is fitted by minimizing the raw signal part below and above it. 

% \begin{equation*}
% \begin{split}
% le(t) & = \arg\min_e(t) \sum_{[e(t)-s(t)]>0} [e(t)-s(t)] \\
% & + w\times\sum_{[e(t)-s(t)]<0}[-e(t)+s(t)]
% \end{split}
% \end{equation*}

% The upper envelope is fitted in a similar way. Then we equally sampled 100 data points from each of the envelope to represent the stack pattern during rotation of different particles.
