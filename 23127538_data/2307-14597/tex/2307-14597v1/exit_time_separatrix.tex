
\begin{proposition}
\label{prop:exit_time_from_separatrix}
    Let $0<\alpha<1/2$, $V^\e=\{x:|H(x)-H(O)|<\e^\alpha\}$, and $\tau=\inf\{t:\tx_t^\e\not\in V^\e\}$. Then, uniformly in $0<\alpha<1/2$ and $(x,y)\in V^\e\times\mathbb T^m$, as $\e\downarrow0$,
    \begin{equation}
        \E_{(x,y)}\tau=O(\e^{2\alpha}|\log\e|).
    \end{equation}
\end{proposition} 
To prove Proposition~\ref{prop:exit_time_from_separatrix}, it is more convenient to consider the process $(\tbx_t^\e,\tbxi_t^\e)$, define the stopping time $\bm\tau=\inf\{t:\tbx_t^\e\not\in V^\e\}$, and then prove that $\E_{(x,y)}\bm\tau=O(\e^{2\alpha-1}|\log\e|)$.
We need careful analysis of the behavior of the processes near the saddle point and away from the saddle point. 
The latter is easier to deal with since there is no degeneracy, while the former needs us to, again, use the Morse Lemma to make concrete computations. 
For simplicity, we prove the result in the case of $(\bx_t^\e,\bxi_t^\e)$ without the additional drift $c(x,y)$ since it can be seen in the proof that the additional terms induced by $c(x,y)$ are always relatively small. 
We prove that there exist two neighborhoods, $D_1\subset D_2$, of $O$ (as shown in Figure~\ref{fig:D1D2}), such that, in $V^\e$, it takes the process $O(|\log\e|)$ time to escape from $D_2$, and $O(1)$ time to return to $D_1$.
Since $x\in V^\e$ is two-dimensional, we denote $x=(p,q)$. To avoid confusion brought by convoluted formulas, we assume the saddle point is the origin and the Hamiltonian $H(x)=pq$ in a small neighborhood of $O$.
    This assumption is not restrictive because, as in the proof of Lemma~\ref{lem:throughsaddlepoint}, we can use the Morse Lemma and perform a random change of time with the multiplier bounded from below and above, which will not change the order of the expected time. 
    For $r>0$, we denote $D_r$ to be the region in $V^\e$ with $|p|\leq r$ and $|q|\leq r$, $(\partial D_r)_{\textrm{in}}=\{|p|=r\}\cap V^\e$, $(\partial D_r)_{\textrm{out}}=\{|q|=r\}\cap V^\e$, and choose $r_3>0$ small enough that $H(x)=pq$ in $D_{r_3}$.
% Figure environment removed
\begin{lemma}
\label{lem:near}
    There exist $r_1,r_2>0$ such that, uniformly in $(x,y)\in D_{r_1}\times\mathbb T^m$, as $\e\downarrow0$,
    \begin{equation}
        \E_{(x,y)}\bar\tau=O(|\log\e|),
    \end{equation}
    where $\bar\tau=\inf\{t:\bx_t^\e\not\in D_{r_2}\}$.
\end{lemma}
\begin{proof}
    
    %Then we choose two smaller regions $D_{r_1}$ and $D_{r_2}$ with $r_1<r_2$, and we will study the exit time and exit location from $D_{r_1}$ to $D_{r_2}$.
    We denote $\eta_t^\e=(\bx_t^\e)_2$ and study the behavior of $\eta_t^\e$ inside $B(0,r_3)$. All the computations below concern $\bx_t^\e$ before leaving $D_{r_3}$. As in \eqref{eq:slowx}, we can write the equation for $\eta_t^\e$,
    \begin{equation}
    \begin{aligned}
    \label{eq:etat}
        {\eta}_t^\e&=q+\int_0^t {\eta}_s^\e ds+\sqrt{\e}\int_0^t\nabla_y u_2(\bx_s^\e,\bxi_s^\e)^{\mathsf T}\sigma(\bxi_s^\e)dW_s\\
        &\quad+\e\int_0^t\nabla_x u_2(\bx_s^\e,\bxi_s^\e)\cdot b(\bx_s^\e,\bxi_s^\e)ds+\e(u_2(x,y)-u_2(\bx_t^\e,\bxi_t^\e)).
    \end{aligned}
    \end{equation}
    Introduce $\hat\eta_t^\e$, which is close to $\eta_t^\e$:
    \begin{equation}
        \label{eq:eta}
        \hat\eta_t^\e = q+\int_0^t {\eta}_s^\e ds+\sqrt{\e}\int_0^t\nabla_y u_2(\bx_s^\e,\bxi_s^\e)^{\mathsf T}\sigma(\bxi_s^\e)dW_s+\e\int_0^t\nabla_x u_2(\bx_s^\e,\bxi_s^\e)\cdot b(\bx_s^\e,\bxi_s^\e)ds.
    \end{equation}
    Let $F(q)=\int_0^q e^{-z^2}\int_0^{z}e^{w^2}dwdz$, which satisfies $ F(q)\sim \frac{1}{2}\log q$, as $p\to\infty$, and has bounded derivatives up to the third order.
    Then we can choose a large $C>0$, such that $|F'|,|F''|,|F'''|$, $|u(x,y)|,|\nabla u(x,y)|,|\nabla^2 u(x,y)|$ are bounded by $C$.
    Recall from \eqref{eqb:nondegenerate-diffusion} that the vector-valued function $\nabla_y u_2(x,y)^{\mathsf T}\sigma(y)$ has non-degenerate average w.r.t. $\mu$, in the sense that $\bm A_{22}(x)>0$. 
    Let $A_0=\bm A_{22}(O)>0$ and let $0<r_1<r_2<r_3$ be such that $A_0(1-1/(2C))<\bm A_{22}(x)<A_0(1+1/(2C))$ in $D_{r_2}$, as shown in Figure~\ref{fig:MorseLemma}.
    Let $\bar r_2=\frac{r_3+r_2}{2}$. Define function $f(q)$ (that depends on $\e$) and compute its derivatives:
    \begin{equation}
    \label{eqb:derivatives_f}
    \begin{aligned}
        f(q)=2(F(\frac{\bar r_2}{\sqrt{A_0\e}})-F(\frac{q}{\sqrt{A_0\e}})),~~~&f'(q)=-\frac{2}{\sqrt{A_0\e}}F'(\frac{q}{\sqrt{A_0\e}}),\\
        f''(q)=-\frac{2}{A_0\e}F''(\frac{q}{\sqrt{A_0\e}}),~~~~~~~~~~~~~~&f'''(q)=-\frac{2}{(\sqrt{A_0\e})^3}F'''(\frac{q}{\sqrt{A_0\e}}).
    \end{aligned}
    \end{equation}
    Furthermore, $f$ satisfies the differential equations:
    \begin{equation}
    \label{eqb:equation_f}
        \begin{cases}
        \frac{1}{2}A_0\e f''+qf'=-1\\
        f(-\bar r_2)=f(\bar r_2)=0
        \end{cases}.
    \end{equation}
    By Lemma~\ref{lem:solution}, there is a function $v_2(x,y)$ that is bounded together with its derivatives such that
    \begin{equation}
        Lv_2(x,y)=\left|\nabla_y u_2(x,y)\sigma(y)\right|^2-\bm A_{22}(x).
    \end{equation}
    where $L$ is the generator of the process $\xi_t$ (see \eqref{eqb:def_operator_L}). 
    %Let $M$ be a large constant such that up to the tenth powers of the terms $$\sup_{x,y}|g(x,y)|,\sup_{x,y}|\nabla g(x,y)|,\sup_{x,y}|b(x,y)|,A_0, 1/A_0, C, 10$$ are bounded by $M$. 
    Since $|\eta_t^\e-\hat\eta_t^\e|=O(\e)$ and $\bar r_2>r_2$, we can apply Ito's formula to $v_2(\bx_t^\e,\bxi_t^\e)f''(\eta_t^\e)$ for $0\leq t\leq\bar\tau$:
    \begin{equation}
    \label{eq:avg_step_eg}
        \begin{aligned}
            v_2(\bx_t^\e,\bxi_t^\e)f''(\eta_t^\e)=&v_2(x,y)f''(q)+\int_0^t\nabla_x(v_2(\bx_s^\e,\bxi_s^\e)f''(\eta_s^\e))\cdot b(\bx_s^\e,\bxi_s^\e)ds\\
            &+\frac{1}{\e}\int_0^t Lv_2(\bx_s^\e,\bxi_s^\e)f''(\eta_s^\e)ds+\frac{1}{\sqrt{\e}}\int_0^t\nabla_yv_2(\bx_s^\e,\bxi_s^\e)^{\mathsf T}f''(\eta_s^\e)\sigma(\bxi_s^\e)dW_s.
        \end{aligned}
    \end{equation}
    Hence it follows that
    \begin{equation}
    \label{eqb:averaged_a22}
    \begin{aligned}
        &\int_0^t (\left|\nabla_y u_2(\bx_s^\e,\bxi_s^\e)\sigma(\bxi_s^\e)\right|^2-\bm A_{22}(\bx_s^\e))f''(\eta_s^\e)ds\\
        &\quad=\e\left(v_2(\bx_t^\e,\bxi_t^\e)f''(\eta_t^\e)-v_2(x,y)f''(q)\right) -\e\int_0^t\nabla_x(v_2(\bx_s^\e,\bxi_s^\e)f''(\eta_s^\e))\cdot b(\bx_s^\e,\bxi_s^\e)ds\\
        &\quad\quad -\sqrt{\e}\int_0^tf''(\eta_s^\e)\nabla_yv_2(\bx_s^\e,\bxi_s^\e)^{\mathsf T}\sigma(\bxi_s^\e)dW_s\\
        &\quad= O(1)+O(\frac{1}{\sqrt{\e}})\cdot t-\sqrt{\e}\int_0^tf''(\eta_s^\e)\nabla_yv_2(\bx_s^\e,\bxi_s^\e)^{\mathsf T}\sigma(\bxi_s^\e)dW_s.
    \end{aligned}
    \end{equation}
    Now apply Ito's formula to $f(\hat\eta_t^\e)$ for $0\leq t\leq \bar\tau$:
    \begin{align*}
        f(\hat\eta_t^\e)&=f(q)+\int_0^tf'(\hat\eta_s^\e)\eta_s^\e ds+\frac{\e}{2}\int_0^t f''(\hat\eta_s^\e)|\nabla_y u_2(\bx_s^\e,\bxi_s^\e)^{\mathsf T}\sigma(\bxi_s^\e)|^2ds\\
        &\quad+\e\int_0^t f'(\hat\eta_s^\e)\nabla_x u_2(\bx_s^\e,\bxi_s^\e)\cdot b(\bx_s^\e,\bxi_s^\e)ds+\sqrt{\e}\int_0^tf'(\hat\eta_s^\e)\nabla_y u_2(\bx_s^\e,\bxi_s^\e)^{\mathsf T}\sigma(\bxi_s^\e)dW_s\\
        &= f(q)+\int_0^tf'(\hat\eta_s^\e)\hat\eta_s^\e ds+O(\sqrt{\e})\cdot t+\frac{\e}{2}\int_0^t f''(\hat\eta_s^\e)\bm A_{22}(\bx_s^\e)ds\\
        &\quad+\frac{\e}{2}\int_0^t (|\nabla_y u_2(\bx_s^\e,\bxi_s^\e)^{\mathsf T}\sigma(\bxi_s^\e)|^2-\bm A_{22}(\bx_s^\e))f''(\eta_s^\e)ds+O(\sqrt{\e})\cdot t\\
        &\quad+\sqrt{\e}\int_0^tf'(\hat\eta_s^\e)\nabla_y u_2(\bx_s^\e,\bxi_s^\e)^{\mathsf T}\sigma(\bxi_t^\e)dW_s\\
        &= f(q)+\int_0^t [f'(\hat\eta_s^\e)\hat\eta_s^\e+\frac{A_0\e}{2}f''(\hat\eta_s^\e)]ds+\frac{\e}{2}\int_0^t f''(\hat\eta_s^\e)(\bm A_{22}(\bx_s^\e)-A_0)ds\\
        &\quad+\frac{\e}{2}( O(1)+O(\frac{1}{\sqrt{\e}})\cdot t-\sqrt{\e}\int_0^tf''(\eta_s^\e)\nabla_yv_2(\bx_s^\e,\bxi_s^\e)^{\mathsf T}\sigma(\bxi_s^\e)dW_s)+O(\sqrt{\e})\cdot t\\
        &\quad+\sqrt{\e}\int_0^tf'(\hat\eta_s^\e)\nabla_y u_2(\bx_s^\e,\bxi_s^\e)^{\mathsf T}\sigma(\bxi_s^\e)dW_s\\
        &\leq f(q)-t+\frac{t}{2}+O(\e)+O(\sqrt{\e})\cdot t-\frac{\sqrt{\e^3}}{2}\int_0^tf''(\eta_s^\e)\nabla_yv_2(\bx_s^\e,\bxi_s^\e)^{\mathsf T}\sigma(\bxi_s^\e)dW_s\\
        &\quad+\sqrt{\e}\int_0^tf'(\hat\eta_s^\e)\nabla_y u_2(\bx_s^\e,\bxi_s^\e)^{\mathsf T}\sigma(\bxi_s^\e)dW_s,
    \end{align*}where the equalities follow from \eqref{eqb:derivatives_f} and \eqref{eqb:averaged_a22}, and the last inequality holds since $f$ solves \eqref{eqb:equation_f} and $|\bm A_{22}(\bx_s^\e)-A_0|<A_0/(2C)$.
    Let $\Tilde \tau=\bar\tau\wedge 1/\e$. Then $\hat\eta_{\Tilde\tau}^\e\in[-\bar r_2,\bar r_2]$ because $|\hat\eta_{\Tilde\tau}^\e-\eta_{\Tilde\tau}^\e|=O(\e)$. The previous calculation reduces to
    \begin{align*}
        f(\hat\eta_{\Tilde\tau}^\e)&\leq f(q)-\Tilde\tau/2+O(\e)+O(\sqrt\e)\cdot\Tilde\tau-\frac{\sqrt{\e^3}}{2}\int_0^{\Tilde\tau}f''(\eta_s^\e)\nabla_yv_2(\bx_s^\e,\bxi_s^\e)^{\mathsf T}\sigma(\bxi_s^\e)dW_s\\
        &\quad+\sqrt{\e}\int_0^{\Tilde{\tau}}f'(\hat\eta_s^\e)\nabla_y u_2(\bx_s^\e,\bxi_s^\e)^{\mathsf T}\sigma(\bxi_s^\e)dW_s.
    \end{align*}
    By taking the expectation, we have for all $x\in D_{r_2}$, $y\in\mathbb T^m$, and $\e$ small enough
    \begin{equation}
        \E_{(x,y)}\Tilde{\tau}\leq 5\sup_{-\bar r_2\leq q'\leq\bar r_2}f(q')=O(|\log\e|).
    \end{equation}
    Then, by Markov's inequality and the Markov property, we obtain that $\E_{(x,y)}\bar\tau=O(|\log\e|)$.
\end{proof}
\begin{lemma}
\label{lem:exit_location}
    Let $r_1,r_2,\bar\tau$ be defined as in Lemma~\ref{lem:near}. Then, uniformly in $(x,y)\in D_{r_1}\times\mathbb T^m$,
    \begin{equation}
        \Prob_{(x,y)}(\bx_{\bar\tau}^\e\in(\partial D_{r_2})_{\mathrm{in}})\to0~\text{as }\e\downarrow0.
    \end{equation}
\end{lemma}
\begin{proof}
    Again, we denote $x=(p,q)$ and, for simplicity, we assume that the saddle point is the origin and that $H(x) = pq$ in a small neighborhood of $O$.
    We extend the function $b(x,y)$ in the vertical direction in such a way that it is bounded together with its partial derivatives and the first component of $\bar b(x)$ is $-p$ in the region $\{x:|p|\leq r_2\}$.
    We denote $\zeta_t^\e=(\bx_t^\e)_1$ and show that it takes significantly longer than $|\log\e|$ for $\zeta_t^\e$ to reach $\pm r_2$, hence it is unlikely for $\bx_t^\e$ to exit $D_{r_2}$ through ${(\partial D_{r_2})}_{\mathrm{in}}$. All the computations below concern $\bx_t^\e$ before $\zeta_t^\e$ reaches $\pm r_2$. As in \eqref{eq:slowx}, we can write the equation for $\zeta_t^\e$:
    \begin{equation}
    \begin{aligned}
    \label{eq:zetat}
        {\zeta}_t^\e&=p-\int_0^t {\zeta}_s^\e ds+\sqrt{\e}\int_0^t\nabla_y u_1(\bx_s^\e,\bxi_s^\e)^{\mathsf T}\sigma(\bxi_s^\e)dW_s\\
        &\quad+\e\int_0^t\nabla_x u_1(\bx_s^\e,\bxi_s^\e)\cdot b(\bx_s^\e,\bxi_s^\e)ds+\e(u_1(x,y)-u_1(\bx_t^\e,\bxi_t^\e)).
    \end{aligned}
    \end{equation}
    Introduce $\hat\zeta_t^\e$, which is close to $\zeta_t^\e$:
    \begin{equation}
        \label{eq:zeta}
        \hat\zeta_t^\e = p-\int_0^t {\zeta}_s^\e ds+\sqrt{\e}\int_0^t\nabla_y u_1(\bx_s^\e,\bxi_s^\e)^{\mathsf T}\sigma(\bxi_s^\e)dW_s+\e\int_0^t\nabla_x u_1(\bx_s^\e,\bxi_s^\e)\cdot b(\bx_s^\e,\bxi_s^\e)ds.
    \end{equation}
    Since $b(x,y)$ and its partial derivatives are bounded in $\{x:|p|\leq r_2\}\times\mathbb T^m$, we can choose $C>0$ such that
    \begin{equation}
    \label{eqb:supremums}
        \sup_{x:|p|\leq r_2,y\in\mathbb T^m}\left(|\nabla_y u_1(x,y)^{\mathsf T}\sigma(y)|^2\vee2|u_1(x,y)|\vee|\nabla_x u_1(x,y)\cdot b(x,y)|\right)<C/2.
    \end{equation}
    Let us define $\bar r_2=\frac{r_1+r_2}{2}$, $\hat{\tau}_2=\inf\{t:|\hat\zeta_t^\e|>\bar r_2\}$, and function $f(p)=\exp(p^2/(C\e))$. Then it follows that
    \begin{equation}
    \label{eq:ode_exit_loc}
        \frac{C\e}{2}f''-pf'-f=0.
    \end{equation}
    Note that $|\zeta_t^\e|\leq r_2$ for $0\leq t\leq\hat\tau_2$ since $|\zeta_t^\e-\hat\zeta_t^\e|\leq C\e/2$. Apply Ito's formula to $\exp(-t/2)f(\hat\zeta_t^\e)$ for $0\leq t\leq\hat\tau_2$ and obtain using \eqref{eqb:supremums}:
    \begin{align*}
        e^{-t/2}f(\hat\zeta_t^\e)&=f(p)-\frac{1}{2}\int_0^t e^{-s/2}f(\hat\zeta_s^\e)ds-\int_0^t e^{-s/2}f'(\hat\zeta_s^\e)\zeta_s^\e ds\\
        &\quad+\e\int_0^t e^{-s/2} f'(\hat\zeta_s^\e)\nabla_x u_1(\bx_s^\e,\bxi_s^\e)\cdot b(\bx_s^\e,\bxi_s^\e)ds\\
        &\quad+\frac{\e}{2}\int_0^t e^{-s/2}f''(\hat\zeta_s^\e)|\nabla_y u_1(\bx_s^\e,\bxi_s^\e)^{\mathsf T}\sigma(\bxi_s^\e)|^2ds\\
        &\quad+\sqrt{\e}\int_0^t e^{-s/2}f'(\hat\zeta_s^\e)\nabla_y u_1(\bx_s^\e,\bxi_s^\e)^{\mathsf T}\sigma(\bxi_s^\e)dW_s\\
        &= f(p)-\frac{1}{2}\int_0^t e^{-s/2}f(\hat\zeta_s^\e)ds\\
        &\quad+\int_0^t e^{-s/2}f'(\hat\zeta_s^\e)\left(-\frac{1}{2}\hat\zeta_s^\e+\left[(\hat\zeta_s^\e-\zeta_s^\e)+\e \nabla_x u_1(\bx_s^\e,\bxi_s^\e)\cdot b(\bx_s^\e,\bxi_s^\e)-\frac{1}{2}\hat\zeta_s^\e\right]\right) ds\\
        &\quad+\frac{\e}{2}\int_0^t e^{-s/2}f''(\hat\zeta_s^\e)|\nabla_y u_1(\bx_s^\e,\bxi_s^\e)^{\mathsf T}\sigma(\bxi_s^\e)|^2ds\\
        &\quad+\sqrt{\e}\int_0^t e^{-s/2}f'(\hat\zeta_s^\e)\nabla_y u_1(\bx_s^\e,\bxi_s^\e)^{\mathsf T}\sigma(\bxi_s^\e)dW_s\\
        &\leq f(p)-\frac{1}{2}\int_0^t e^{-s/2}f(\hat\zeta_s^\e)ds-\frac{1}{2}\int_0^t e^{-s/2}f'(\hat\zeta_s^\e)\hat\zeta_s^\e ds+\frac{C\e}{4}\int_0^t e^{-s/2}f''(\hat\zeta_s^\e)ds\\
        &\quad+\int_0^t e^{-s/2}f'(\hat\zeta_s^\e)\left[(\hat\zeta_s^\e-\zeta_s^\e)+\e \nabla_x u_1(\bx_s^\e,\bxi_s^\e)\cdot b(\bx_s^\e,\bxi_s^\e)-\frac{1}{2}\hat\zeta_s^\e\right] ds\\
        &\quad+\sqrt{\e}\int_0^t e^{-s/2}f'(\hat\zeta_s^\e)\nabla_y u_1(\bx_s^\e,\bxi_s^\e)^{\mathsf T}\sigma(\bxi_s^\e)dW_s\\
        &\leq  f(p)+18C\e(1-e^{-t/2})+\sqrt{\e}\int_0^t e^{-s/2}f'(\hat\zeta_s^\e)\nabla_y u_1(\bx_s^\e,\bxi_s^\e)^{\mathsf T}\sigma(\bxi_s^\e)dW_s.
    \end{align*}
    The last inequality follows from \eqref{eq:ode_exit_loc} and the fact that the integrand on the second line is either negative, when $|\hat\zeta_s^\e|\geq 2C\e$, or small and bounded by $9C\e e^{-s/2}$, when $|\hat\zeta_s^\e|\leq 2C\e$. By replacing $t$ by the stopping time $\hat\tau_2$ and taking expectation, we obtain
    \begin{equation}
        \E_{(x,y)} e^{-\hat\tau_2/2}\leq 2 e^{(r_1^2-{\bar r_2}^2)/(C\e)}.
    \end{equation}
    Let $\bar{\tau}_2=\inf\{t:|\zeta_t^\e|>r_2\}$. Then, since $|\zeta_t^\e-\hat\zeta_t^\e|\leq C\e/2$, it follows that
    \begin{equation}
        \Prob_{(x,y)}(\bar{\tau}_2<|\log\e|/\sqrt{\e})\leq \Prob_{(x,y)}(\hat{\tau}_2<|\log\e|/\sqrt{\e})\leq 2\exp(\frac{r_1^2-{\bar r_2}^2}{C\e}+\frac{|\log\e|}{2\sqrt{\e}})\to 0,
    \end{equation}
    as $\e\downarrow0$. However, by Lemma~\ref{lem:near} and Markov's inequality, we have
    \begin{equation}
        \Prob_{(x,y)}(\bar\tau>|\log\e|/\sqrt{\e})\to0,
    \end{equation}
    as $\e\downarrow0$. Thus, the desired result follows.
\end{proof}
\iffalse
\begin{remark}
Although we simply assumed that $H=x_1x_2$, the result can be generalized without much more effort. For instance, in Lemma~\ref{lem:near}, instead of $\int_0^t f'(\eta_s^\e)\Tilde\eta_s^\e ds$, we will have $\int_0^t f'(\eta_s^\e)c(\Tilde\zeta_s^\e,\Tilde\eta_s^\e)\Tilde\eta_s^\e ds$ with function $c(p,q)$ bounded both from below (by constant $b$) and above. Then we can modify our function $f$ such that it is a solution to
\begin{equation}
        \begin{cases}
        \frac{B\e}{2}f''+bqf'=-1\\
        f(-\bar r_2)=f(\bar r_2)=0.
        \end{cases}
    \end{equation}
Similar modifications can be applied to Lemma~\ref{lem:exit_location} to generalize the result. Namely, instead of $\int_0^t e^{-\lambda s}f'(\zeta_s^\e)\Tilde\zeta_s^\e ds$, we will have $$\int_0^t e^{-\lambda s}f'(\zeta_s^\e)c(\Tilde\zeta_s^\e,\Tilde\eta_s^\e)\Tilde\zeta_s^\e ds=\int_0^t e^{-\lambda s}f'(\zeta_s^\e)[\frac{b}{2}\zeta_s^e+c(\Tilde\zeta_s^\e,\Tilde\eta_s^\e)(\Tilde\zeta_s^\e-\zeta_s^\e)+(c(\Tilde\zeta_s^\e,\Tilde\eta_s^\e)-\frac{b}{2})\zeta_s^\e] ds.$$ We modify $f$ by replacing $C$ with $2C/b$ and change the exponent from $-t/2$ to $-bt/2$.
\end{remark}
\fi
\begin{lemma}
    \label{lem:away}
    Let $\bar{\bar\tau}=\inf\{t:\bx_t^\e\in D_{r_1}\}\wedge\bm\tau$. Then there exists $a>0$ such that, uniformly in $(x,y)\in(\partial {D_{r_2}})_{\mathrm{out}}\times\mathbb T^m$,
    \begin{equation}
    \label{eq:away_non_deg}
        \Prob_{(x,y)}(\bar{\bar\tau}<\bm\tau,\int_0^{\bar{\bar\tau}}|\nabla_y u_h(\bx_s^\e,\bxi_s^\e)^{\mathsf T}\sigma(\bxi_s^\e)|^2ds<a)\to 0~\text{as }\e\to0.
    \end{equation}
    Furthermore, $\E_{(x,y)}\bar{\bar\tau}$ is bounded uniformly in $(x,y)\in(\partial {D_{r_2}})_{\mathrm{out}}\times\mathbb T^m$.
\end{lemma}
\begin{proof}
Let $\hat t>0$ and $\check t>0$ be the lower bound and the upper bound of time spent by $\bm x_t$ to get from $(\partial {D_{r_2}})_{\mathrm{out}}$ to $D_{r_1}$, respectively. Then, similarly to \eqref{eq5:positive_variance_one_rotation}, there exists $a>0$ such that
\begin{equation}
    \Prob_{(x,y)}\left(\int_0^{\hat t/2} |\nabla_y u_h(\bx_s^\e,\bxi_s^\e)^{\mathsf T}\sigma(\bxi_s^\e))|^2ds>a,\sup_{0\leq t\leq2\hat t}|\bx_t^\e-\bm x_t|\leq\e^{\frac{1+2\alpha}{4}}\right)\to1.
\end{equation}
Hence
\begin{align*}
    &\Prob_{(x,y)}\left(\bar{\bar\tau}<\bm\tau,\int_0^{\bar{\bar\tau}}|\nabla_y u_h(\bx_s^\e,\bxi_s^\e)^{\mathsf T}\sigma(\bxi_s^\e)|^2ds<a\right)\\
    &\leq\Prob_{(x,y)}\left(\hat t/2\leq\bar{\bar\tau}<\bm\tau,\int_0^{\bar{\bar\tau}}|\nabla_y u_h(\bx_s^\e,\bxi_s^\e)^{\mathsf T}\sigma(\bxi_s^\e)|^2ds<a\right)+\Prob_{(x,y)}(\bar{\bar\tau}<\bm\tau,\bar{\bar\tau}<\hat t/2)\\
    &\leq\Prob_{(x,y)}\left(\int_0^{\hat t/2}|\nabla_y u_h(\bx_s^\e,\bxi_s^\e)^{\mathsf T}\sigma(\bxi_s^\e)|^2ds<a\right)+\Prob_{(x,y)}\left(\sup_{0\leq t\leq2\hat t}|\bx_t^\e-\bm x_t|>\e^{\frac{1+2\alpha}{4}}\right)\\
    &\to0.
\end{align*}
Similarly, it is easy to see that $\Prob_{(x,y)}(\bar{\bar\tau}>2\check t)<\Prob_{(x,y)}\left(\sup_{0\leq t\leq2\check t}|\bx_t^\e-\bm x_t|>\e^{\frac{1+2\alpha}{4}}\right)\to0$, and the desired result follows from the Markov property.
\end{proof}


\begin{proof}[Proof of Proposition~\ref{prop:exit_time_from_separatrix}]
    As in~\eqref{eq:H}:
    \begin{align*}
        H(\bx_t^\e) &= H(x)+\e\int_0^t \nabla_x u_h(\bx_s^\e,\bxi_s^\e)\cdot b(\bx_s^\e,\bxi_s^\e)ds\\
    &\quad+\sqrt{\e}\int_0^t \nabla_y u_h(\bx_s^\e,\bxi_s^\e)^{\mathsf T}\sigma(\bxi_s^\e)dW_s+\e(u_h(x,y)-u_h(\bx_t^\e,\bxi_t^\e)).
    \end{align*}
    The change in $H(\bx_t^\e)$ is mainly due to the stochastic integral while the other terms are of order $O(\e)$ and $O(t\cdot\e)$ and can be controlled. For each $t(\e)>0$,
    \begin{equation}
    \label{eq:set}
    \begin{aligned}
        \{\bm\tau<t(\e)\}\supset&\left\{\sup_{[0,t(\e)]}\left|\sqrt{\e}\int_0^{t(\e)} \nabla_y u_h(\bx_s^\e,\bxi_s^\e)^{\mathsf T}\sigma(\bxi_s^\e)dW_s\right|>3\e^\alpha\right\}\\
        &\quad\bigcap\left\{\e\int_0^{t(\e)} |\nabla_x u_h(\bx_s^\e,\bxi_s^\e)\cdot b(\bx_s^\e,\bxi_s^\e)|ds<\e^\alpha/2\right\}.
    \end{aligned}
    \end{equation}
    Note that if we choose $t(\e)=o(\e^{\alpha-1})$, then the second event is always true. Now we recursively define stopping times:
    \begin{align*}
        \theta^1_0&=0,\\
        \theta^2_k&=\inf\{t\geq\theta^1_{k-1}:\bx_t^\e\in\partial D_{r_2}\}\wedge\bm\tau,\\
        \theta^1_k&=\inf\{t\geq\theta^2_{k}:\bx_t^\e\in\partial D_{r_1}\}\wedge\bm\tau.
    \end{align*}
    We denote $D(x,y)=\nabla_y u_h(x,y)^{\mathsf T}\sigma(y)$. Note that once the process leaves $V^\e$, the stopping times stay constant afterwards. The main idea of the proof is to show that after a sufficiently long time $t(\e)$, the stochastic integral will accumulate enough variance to exit from $V^\e$. Let us bound the probability of variance being small:
    \begin{equation}
    \label{eq:top_inequality}
        \begin{aligned}
            &\Prob_{(x,y)}\left(\bm\tau\geq t(\e),\int_0^{t(\e)}|D(\bx_s^\e,\bxi_s^\e)|^2ds<9\e^{2\alpha-1}\right)\\
            &\quad\leq\Prob_{(x,y)}\left(\bm\tau\geq t(\e)>\theta^1_{n(\e)},\int_0^{t(\e)}|D(\bx_s^\e,\bxi_s^\e)|^2ds<9\e^{2\alpha-1}\right)+\Prob_{(x,y)}(\theta^1_{n(\e)}\geq t(\e)),
        \end{aligned}
    \end{equation}
    where the integer $n(\e)$ will be specified later.
    Let $\bar\tau$, $\bar{\bar\tau}$, and $a$ be defined as in Lemma~\ref{lem:near} and Lemma~\ref{lem:away}. Then
    \begin{align}
        &\Prob_{(x,y)}\left(\bm\tau\geq t(\e)>\theta^1_{n(\e)},\int_0^{t(\e)}|D(\bx_s^\e,\bxi_s^\e)|^2ds<9\e^{2\alpha-1}\right)\nonumber\\
        &\quad \leq \exp(9\e^{2\alpha-1}/a)\E_{(x,y)}\left(\chi_{\{\bm\tau>\theta^1_{n(\e)}\}}\exp\left(-\frac{1}{a}\int_0^{\theta^1_{n(\e)}}|D(\bx_s^\e,\bxi_s^\e)|^2ds\right)\right)\label{eq:rotations}\\
        &\quad\leq\exp(9\e^{2\alpha-1}/a)\left[\sup_{(x,y)\in\partial D_{r_1}\times\mathbb T^m}\E_{(x,y)}\left(\chi_{\{\bm\tau>\theta^1_1\}}\exp\left(-\frac{1}{a}\int_0^{\theta^1_1}|D(\bx_s^\e,\bxi_s^\e)|^2ds\right)\right)\right]^{{n(\e)}-1}.\nonumber
    \end{align}
    Now let us deal with one excursion from $D_{r_2}$ to $D_{r_1}$. For $(x,y)\in(\partial {D_{r_2}})_{\mathrm{out}}\times\mathbb T^m$,
    \begin{equation}
    \label{eqb:e0.99}
    \begin{aligned}
        &\E_{(x,y)}\left(\chi_{\{\bar{\bar\tau}<\bm\tau\}}\exp\left(-\frac{1}{a}\int_0^{\bar{\bar\tau}}|D(\bx_s^\e,\bxi_s^\e)|^2ds\right)\right)\\
        &\quad\leq \Prob_{(x,y)}(\bar{\bar\tau}<\bm\tau,\int_0^{\bar{\bar\tau}}|D(\bx_s^\e,\bxi_s^\e)|^2ds<a)+\Prob_{(x,y)}(\bar{\bar\tau}<\bm\tau,\int_0^{\bar{\bar\tau}}|D(\bx_s^\e,\bxi_s^\e)|^2ds\geq a)/e\\
        &\quad\leq e^{-0.99},
    \end{aligned}
    \end{equation}
    for all $\e$ sufficiently small, by Lemma~\ref{lem:away}. For $(x,y)\in\partial D_{r_1}\times\mathbb T^m$:
    \begin{align*}
        &\E_{(x,y)}\left(\chi_{\{\bm\tau>\theta^1_1\}}\exp\left(-\frac{1}{a}\int_0^{\theta^1_1}|D(\bx_s^\e,\bxi_s^\e)|^2ds\right)\right)\\
        &\leq \E_{(x,y)}\left(\chi_{\{\bm\tau>\theta^1_1,\bx^\e_{\bar\tau}\in(\partial {D_{r_2}})_{\mathrm{out}}\}}\exp\left(-\frac{1}{a}\int_0^{\theta^1_1}|D(\bx_s^\e,\bxi_s^\e)|^2ds\right)\right)+\Prob_{(x,y)}(\bx^\e_{\bar\tau}\in(\partial {D_{r_2}})_{\mathrm{in}})\\
        &\leq \sup_{(x',y')\in(\partial {D_{r_2}})_{\mathrm{out}}\times\mathbb T^m}\E_{(x',y')}\left(\chi_{\{\bar{\bar\tau}<\bm\tau\}}\exp\left(-\frac{1}{a}\int_0^{\bar{\bar\tau}}|D(\bx_s^\e,\bxi_s^\e)|^2ds\right)\right)+\Prob_{{(x,y)}}(\bx^\e_{\bar\tau}\in(\partial {D_{r_2}})_{\mathrm{in}})\\
        &\leq e^{-0.98},
    \end{align*}
    by Lemma~\ref{lem:exit_location} and \eqref{eqb:e0.99}.
    Now we can come back to \eqref{eq:rotations} and have
    \begin{equation}
    \label{eq:first_prob}
        \Prob_{(x,y)}\left(\bm\tau\geq t(\e)>\theta^1_n,\int_0^{t(\e)}|D(\bx_s^\e,\bxi_s^\e)|^2ds<9\e^{2\alpha-1}\right)\leq\exp(9\e^{2\alpha-1}/a-0.98({n(\e)}-1)).
    \end{equation}
    The second probability on the right hand side of \eqref{eq:top_inequality} can be bounded by Lemmas~\ref{lem:near} and \ref{lem:away} with certain $K>0$:
    \begin{equation}
    \label{eq:second_prob}
        \begin{aligned}
            \Prob_{(x,y)}(\theta^1_n\geq t(\e))&\leq{\E_{(x,y)}\theta^1_n}/{t(\e)}\\
            &\leq{\left(\sup_{(x',y')\in\partial D_{r_1}\times\mathbb T^m}\E_{(x',y')}\bar\tau+\sup_{{(x',y')\in\partial D_{r_2}\times\mathbb T^m}}\E_{(x',y')}\bar{\bar\tau}\right)}\cdot\frac{n(\e)}{t(\e)}\\
            &\leq\frac{n(\e)K|\log\e|}{t(\e)}.
        \end{aligned}
    \end{equation}
    Choose $n(\e)=[10\e^{2\alpha-1}/a+2]$. Then the quantity in \eqref{eq:first_prob} converges to $0$. Choose $t(\e)=100K\e^{2\alpha-1}|\log\e|/a$, then the quantity on the right hand side of \eqref{eq:second_prob} converges to $0.1$. Therefore, the quantity on the right hand side of \eqref{eq:top_inequality} converges to $0.1$. Moreover, since $t(\e)=o(\e^{\alpha-1})$, it follows from \eqref{eq:set} that, for all $x\in V^\e$, $y\in\mathbb T^m$, and $\e$ sufficiently small,
    \begin{align*}
        &\Prob_{(x,y)}(\bm\tau\geq t(\e))\\
        &=\Prob_{(x,y)}\left(\bm\tau\geq t(\e),\sup_{[0,t(\e)]}\left|\sqrt{\e}\int_0^t D(\bx_s^\e,\bxi_s^\e)dW_s\right|\leq 3\e^\alpha\right)\\
        &\leq \Prob_{(x,y)}\left(\bm\tau\geq t(\e),\int_0^{t(\e)}|D(\bx_s^\e,\bxi_s^\e)|^2ds>9\e^{2\alpha-1},\sup_{[0,t(\e)]}\left|\sqrt{\e}\int_0^t D(\bx_s^\e,\bxi_s^\e)dW_s\right|\leq 3\e^\alpha\right)\\
        &\quad+ \Prob_{(x,y)}\left(\bm\tau\geq t(\e),\int_0^{t(\e)}|D(\bx_s^\e,\bxi_s^\e)|^2ds<9\e^{2\alpha-1}\right)\\
        &\leq 0.69+0.11=0.8,
    \end{align*}
    since the stochastic integral in the last inequality can be represented as time-changed Brownian motion. Finally, we have by the Markov property
    \[
    \E_{(x,y)}\bm\tau\leq 5t(\e)=O(\e^{2\alpha-1}|\log\e|).
    \]
\end{proof}
