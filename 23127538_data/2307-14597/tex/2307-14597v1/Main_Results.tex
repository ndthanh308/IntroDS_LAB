\section{Main Results}
\label{sec:mainresult}
Throughout this article, $\Prob$ and $\E$ represent the probability and expectation, respectively, and the  subscripts pertain to initial conditions. For brevity, the stopping times' dependence on parameters and initial conditions is not always indicated in the notation when introduced (e.g. \eqref{eq:stopping_times}). $\nabla$ denotes a first order differential operator, i.e., derivative, gradient, Jacobian, etc., depending on the context. 
$\chi_A$ denotes the indicator function of the event $A$.
If $A$ and $B$ are two non-negative functions that depend on an asymptotic parameter, we write $A\lesssim B$ if $A=O(B)$.
$\bm{\mathrm C}_0(\mathbb G)$ is the space of continuous functions on the Reeb graph $\mathbb G$ that tend to zero at infinity with uniform norm.
$h$ is the projection onto $\mathbb G$.
In order to formulate the assumptions and results, we introduce some notation: 
\begin{enumerate}
    \item $O_i$'s are the vertices on the graph and are occasionally used to denote the corresponding critical points on the plane when there is no ambiguity. $I_k$'s are the edges on the graph and $U_k$'s are the corresponding two-dimensional domains. Formally, $O_\infty$ is the vertex that corresponds to infinity. 
    %$I_{k^*}$ is the semi-infinite edge that connects $O_\infty$ and $O_{i^*}$, which  is the other endpoint of $I_{k^*}$, and $U^*$ is the unbounded domain that corresponds to $I_{k^*}$. 
    A symbol $\sim$ between a vertex and an edge means that the vertex is an endpoint of the edge. 
    \item Consider the following metric on $\mathbb G$: $r(h_1,h_2)$ is the length of the shortest path connecting $h_1$ and $h_2$. For example, if $h_1=(1,H_1)$, $I_1\sim O_1$, $O_1\sim I_2$, $I_2\sim O_2$, $O_2\sim I_3$ and $h_2=(3,H_2)$, then $r(h_1,h_2)=|H_1-H(O_1)|+|H(O_1)-H(O_2)|+|H(O_2)-H_2|$. 
    \item $\gamma(h)=\{x:H(x)=h\}$ and $\gamma_k(h)$ is the connected component of $\gamma(h)$ in the domain $U_k$.
    \item $b_h(x,y)=\nabla H(x)\cdot b(x,y)$. 
    \item $\xi_t$ is the diffusion process on $\mathbb T^m$ with the generator $L$, where
    \begin{equation}
        \label{eqb:def_operator_L}
    L f(y)=v(y)\cdot\nabla_y f(y)+\frac{1}{2}\sum_{i,j}(\sigma\sigma^*)_{ij}(y)\frac{\partial^2}{\partial y_i\partial y_j}f(y).
    \end{equation}
    \item For $h$ in the interior of $I_k$, define
    \begin{align*}
            Q_k(h)&=\int_{\gamma_k(h)}\frac{dl}{|\nabla H(x)|},\\
            A_k(h)&=\frac{2}{Q_k(h)}\int_{\gamma_k(h)}\frac{1}{|\nabla H(x)|}\int_0^\infty\E_\mu b_h(x,\xi_s)b_h(x,\xi_0)dsdl,\\
            B_k(h)&=\frac{1}{Q_k(h)}\int_{\gamma_k(h)}\frac{1}{|\nabla H(x)|}\int_0^\infty\E_\mu\nabla_x b_h(x,\xi_s)\cdot(b(x,\xi_0)-\nabla^{\perp}H(x))dsdl,\\
            L_kf(h)&=\frac{1}{2}A_k(h)f''(h)+B_k(h)f'(h).
        \end{align*}
\end{enumerate}

The following conditions are assumed to hold throughout the article.

\textbf{\textit{Assumptions on the coefficients:}}
\begin{enumerate}
    \item[\hypertarget{H1}{\textit{(H1)}}] $v(y)$ and $\sigma(y)$ are $C^\infty$ functions on $\mathbb T^m$. $\sigma(y)$ is $m\times m$ matrix-valued and $\sigma(y)\sigma(y)^{\mathsf T}$ is positive-definite for all $y\in\mathbb T^m$.
    \item[\hypertarget{H2}{\textit{(H2)}}] $H(x)$ is a $C^\infty$ function from $\mathbb R^2$ to $\mathbb R$ with bounded second derivatives. $H(x)$ has a finite number of non-degenerate critical points. Each level curve corresponding to a vertex on the Reeb graph contains at most one critical point. As $|x|\to+\infty$, $H(x)/|x|\to+\infty$. 
    \item[\hypertarget{H3}{\textit{(H3)}}] $b(x,y)$ is a $C^\infty$ function from $\mathbb R^2\times\mathbb T^m$ to $\mathbb R^2$ such that the averaged process is a Hamiltonian system with $H$, i.e. $\bar b(x)=\nabla^\perp H(x)$. 
    \item[\hypertarget{H4}{\textit{(H4)}}] The fast oscillating perturbation is non-degenerate, i.e. $\{b(x,y)-\bar b(x):y\in\mathbb T^m\}$ spans $\mathbb R^2$ for each $x\in\mathbb R^2$, and is uniformly bounded together with its first derivatives.
    \item[\hypertarget{H5}{\textit{(H5)}}] For each $x$ that belongs to one of the separatrices, there exists $y\in\mathbb T^m$ such that the process in \eqref{eq:theprocess1} satisfies the parabolic H\"ormander condition at $(x,y)$. Namely, with $\e^{-1}\tilde v(y)$ being the drift term in the equation for $\bxi^\e_t$ in the Stratonovich form, we have that
    \begin{equation}
        \mathrm{Lie}\left(
        \left\{\begin{pmatrix}
            0\\ \sigma_k(y)
        \end{pmatrix},1\leq k\leq m\right\}\bigcup
        \left\{\left[\begin{pmatrix}
            b(x,y)\\ \tilde v(y)
        \end{pmatrix},
        \begin{pmatrix}
            0\\ \sigma_k(y)
        \end{pmatrix}\right],1\leq k\leq m\right\}\right)
    \end{equation} at $(x,y)$ spans $\mathbb R^{2+m}$, where $\sigma_k$ is the $k-$th column of $\sigma(y)$, $[\cdot,\cdot]$ is the Lie bracket, and Lie$(\cdot)$ is the Lie algebra generated by a set (cf. \cite{Hormander_Hairer} or Section 2.3.2 of \cite{Nualart}). 
%boundedness of all derivatives can be implied by infinite differentiability on bounded set. But uniform boundedness should be assumed separately.
\end{enumerate}

\begin{definition}
\label{def:domain_original}
    The domain $D(\mathcal L)$ consists of functions $f\in \bm{\mathrm{C}}_0(\mathbb G)$ satisfying:
    \begin{enumerate}[(i)]
        \item $f$ is twice continuously differentiable in the interior of each edge $I_k$ of $\mathbb G$;
        \item The limits $\lim_{h_k\to O_i}L_kf(h_k)$ exist and do not depend on the edge $I_k$;
        \item For interior vertex $O_i$, there are constants $p_k:=\pm\lim_{h\to O_i}A_k(h)Q_k(h)$ such that
        \begin{equation}
        \label{eq2:gluing_condition}
            \sum_{I_k\sim O_i}p_k\lim_{h_k\to O_i}f'(h_k)=0,
        \end{equation}
    \end{enumerate}
    where the sign $+$ is taken if $O_i$ is minimum on $I_k$, and the sign $-$ is taken otherwise.
    The operator $\mathcal L$ on the Reeb graph is defined by 
        \begin{equation}
            \label{def:operatorL}
            \mathcal Lf(h)=L_kf(h)
        \end{equation}
    for $f\in D(\mathcal L)$ and $h$ in the interior of $I_k$, and defined as $\lim_{h\to O_i}\mathcal Lf(h)$ at the vertex $O_i$.
\end{definition}
By the Hille-Yosida theorem (see, for example, Theorem 4.2.2 in \cite{markov_process}), one can check that there exists a unique strong Markov process on $\mathbb G$ with continuous sample paths that has $\mathcal L$ as its generator.
Now we are ready to formulate the main result of this article.
% One-dimension so itself without closure is the generator. Continuous follows from pg171. Strong Markov follows from Feller.
\begin{theorem}
\label{thm:mainresult}
    Let the process $(\bx_t^\e,\bxi_t^\e)$ be defined as in \eqref{eq:theprocess1} and the conditions \hyperlink{H1}{\textit{(H1)-(H5)}} hold.
    Then $h(\bx_{t/\e}^\e)$ converges weakly to the strong Markov process on the Reeb graph $\mathbb G$ that has the generator $(\mathcal L,D(\mathcal L))$ and the initial distribution $h(x_0)$.  
\end{theorem}
\begin{remark}
    %The assumption $H(2)$ cannot be relaxed since the place you modify the motion is not where it stays compact. You need the decaying to stay compact. And the compactness is usually much larger than the contraints on coefficients. 
    The last condition in \hyperlink{H2}{\textit{(H2)}} can be relaxed without much extra effort since the limiting process defined by $\mathcal L$ cannot reach infinity in finite time.
    In addition, as seen from the proofs in Section~\ref{sec:exponenitalconvergence} and Remark~\ref{rmk:positive_any_subset_separatrix}, assumption \hyperlink{H5}{\textit{(H5)}} can be relaxed so that it holds for at least one point (different from the saddle point) on each separatrix. Moreover, if the number of Lie brackets needed to generate $\mathbb R^{2+m}$ in the parabolic H\"ormander condition is assumed to be given, then we can relax the assumptions on smoothness of the coefficients.
\end{remark}
%Let $\mathcal D$ be the set of functions that satisfy conditions (i) and (iii) in Definition~\ref{def:domain_original}.
%and have bounded first and second derivatives. <- not really needed because of tightness
To prove the theorem, we need a result on weak convergence of processes, that is Lemma 4.1 in \cite{FreidlinKoralov2021} adapted to our case (see also the original statement in \cite{FreidlinWentzell1994}):
\begin{lemma}
\label{lem:martingale_problem}
Let $\Psi$ be a dense linear subspace of $\bm{\mathrm{C}}_0(\mathbb G)$ and $\mathcal D_{\mathcal L}$ be a linear subspace of $D(\mathcal L)$, and suppose that $\Psi$ and $\mathcal D_{\mathcal L}$ have the following properties:
\begin{enumerate}[(1)]
    \item There is a $\lambda>0$ such that for each $F\in\Phi$ the equation $\lambda f-\mathcal Lf=F$ has a solution $f\in\mathcal D_{\mathcal L}$;
    \item For each $T>0$, each $f\in\mathcal D_{\mathcal L}$, and each compact $K\subset\mathbb G$, 
    \begin{equation}
        \label{eq:mgprob}
        \E_{(x,y)}[f(h(X_{T}^\e))-f(h(x))-\int_0^T\mathcal Lf(h(X_{t}^\e))dt]\to 0,
    \end{equation}
\end{enumerate}
uniformly in $x\in h^{-1}(K)$ and $y\in\mathbb T^m$. 

Suppose that the family of measures on $\bm{\mathrm{C}}([0,\infty], \mathbb G)$ induced by the processes $h(X_t^\e)$, $\e>0$, is tight for each $(x,y)\in\mathbb R^2\times\mathbb T^m$. Then, for each $(x,y)\in\mathbb R^2\times\mathbb T^m$, $h(X_t^\e)$ converges weakly to the strong Markov process on the Reeb graph $\mathbb G$ that has the generator $(\mathcal L,D(\mathcal L))$ and the initial distribution $h(x)$.  
\end{lemma}
Here we choose $\Psi$ to be all the functions in $\bm{\mathrm{C}}_0(\mathbb G)$ that are twice continuously differentiable in the interior of each edge; $\mathcal D_{\mathcal L}$ to be all the functions in $D(\mathcal L)$ that are four times continuously differentiable in the interior of each edge. It is easy to check condition $(1)$ holds in Lemma~\ref{lem:martingale_problem}, and the tightness of distributions of $h(X_t^\e)$ for all $\e>0$ is verified in Appendix~\ref{sec:tightness}. Then the main ingredient of the proof is to verify \eqref{eq:mgprob} in condition (2) of Lemma~\ref{lem:martingale_problem}.
%which states that the martingale problem considered in $\mathcal D$ is solved ``asymptotically'' by $h(\bx_{t/\e}^\e)$.
\iffalse
\begin{lemma}
\label{lem:matingale_original_on_R2}
For all $f\in \mathcal D_{\mathcal L}$ and all $T>0$, \eqref{eq:mgprob} holds uniformly in $x$ in any compact set in $\mathbb R^2$ and $y_0\in\mathbb T^m$.
\end{lemma}
\fi


