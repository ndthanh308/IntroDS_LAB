\section{Introduction}
\label{sec:introduction}
Consider a diffusion process $(\bx_t^\e,\bxi_t^\e)$ in $\mathbb R^2\times\mathbb T^m$ satisfying 
\begin{equation}
\label{eq:theprocess1}
\begin{aligned}
    d\bm X_t^\e=&~b(\bx_t^\e,\bxi_t^\e)dt,~~~~~~~ \bx_0^\e=x_0\in\mathbb R^2,\\
    d\bxi_t^\e=&~\frac{1}{\e}v(\bxi_t^\e)dt+\frac{1}{\sqrt{\e}}\sigma(\bxi_t^\e)dW_t,~~~\bxi_0^\e=y_0\in\mathbb T^m,
\end{aligned}
\end{equation}
where $\e$ is a small positive parameter, $\mathbb T^m$ is the $m$-dimensional torus, and $W_t$ is an $m$-dimensional Brownian motion. In the coupled system, $\bx_t^\e$ is the slow motion and $\bxi_t^\e$ is the fast motion, since the generator of the diffusion in the second equation is scaled by $\e^{-1}$. In this system, all the randomness comes from the second equation and the slow motion depends on the fast one in a deterministic way, and this dependence results in fast-changing velocity for the slow motion. 
%The fast motion is considered to be one-dimensional here, for simplicity of notation, but this assumption is not essential. 
Under natural conditions, the averaging principle holds for the process in \eqref{eq:theprocess1} 
(cf. \cite{randomperturbation}). For example, if $\sigma(y)$ is non-degenerate (and thus 
$\bxi_t^\e$ has a unique invariant measure $\mu$ independent of $\e$), then $\bx_t^\e$ converges as $\e\to0$ in probability on each finite interval $[0,T]$ to an averaged process defined by the differential equation
\begin{equation}
\label{eq:averagedprocess}
    d{\bm{x}_t}=\bar b({\bm{x}_t})dt,
\end{equation}
where $\bar b(x)=\int_{\mathbb T^m}b(x,y)d\mu(y)$. 
Therefore, $\bx_t^\e$ can be viewed as a result of fast-oscillating random perturbations of the deterministic process ${\bm{x}_t}$. 
Moreover, the deviation can be described more precisely: the process $\e^{-1/2}(\bx_t^\e-{\bm{x}_t})$ converges weakly to a Gaussian Markov process on a finite interval $[0,T]$ (cf. \cite{randomperturbation}), and, if we assume a special type of vector $b(x,y)$, then the local limit theorem holds for $\e^{-1}(\bx_t^\e-{\bm{x}_t})$ at time $t$ (\cite{LLT}).


% Original Statements
\iffalse
On the other hand, if the slow motion in the system \eqref{eq:theprocess1} has a first integral, then on larger time scale of order $\e^{-1}$, one can expect that, given ergodicity of the system \eqref{eq:averagedprocess} on its level sets, the process converges weakly to a diffusion process in its first integral (\cite{AIHPB_1995__31_3_485_0}). 
Intuitively, the convergence is due to Gaussian approximation of the perturbations and two averaging principles on both the fast mixing process $\xi_t^\e$ and fast component of $X^\e_t$ on the level sets.
More precisely, suppose that \eqref{eq:averagedprocess} has a first integral $H$, i.e. $H({\bm{x}_t})=H(x_0)$ for all $t\geq0$ and the function $H$ has only one critical point in $\mathbb R^2$. 
\fi 




If the system \eqref{eq:averagedprocess} has a first integral $H$, then, by the averaging principle, $H(\bx_t^\e)$ is nearly constant on finite time intervals when $\e$ is small. Nontrivial behavior can, however, be observed on larger time intervals (of order $\e^{-1}$). Assume, momentarily, that $H$ has a single critical point. Then it was demonstrated in \cite{AIHPB_1995__31_3_485_0} that $H(\bx_{t/\e}^\e)$ converges weakly in $C([0,T])$, as $\e\to0$, to a diffusion process for any finite $T$, under additional assumptions. A similar result was obtained recently in \cite{Freidlin2021} in the case where the slow motion is an $N$-dimensional process and the averaged process has $n$ independent first integrals. The result holds in the region where no critical points of the first integrals are present and action-angle-type coordinates can be introduced. 
% Figure environment removed
    
Let us return to the two-dimensional situation. In the presence of multiple critical points, including saddle points, the problem gets more complicated as we need to consider the Reeb graph in order to describe the evolution of the first integrals denoted by $h=(k,H)$. (For instance, in Figure~\ref{fig:reeb_graph}, we have two local minima and one saddle point in the space and thus two exterior vertices and one interior vertex on the graph.) In particular, the interior vertices on the graph correspond to the level curves that contain the saddle points, and those level curves are called the separatrices. 
In this situation, the limiting behavior has already been described for the white-noise-type additive perturbations of dynamical systems: Hamiltonian systems in $\mathbb R^2$ (\cite{randomperturbation}), general dynamical systems with conservation laws in $\mathbb R^n$ (\cite{Freidlin2004}), and Hamiltonian systems with an ergodic component on two-dimensional surfaces (\cite{Dmitry2008},\cite{Dmitry2013},\cite{dolgopyat_freidlin_koralov_2012}). 

In this article, we consider fast oscillating random perturbations, as discussed above, of Hamiltonian system in $\mathbb R^2$ with multiple critical points and prove that the evolution of the first integrals $h$ converges to a diffusion process defined by an operator ($\mathcal L, D(\mathcal L))$ on the corresponding Reeb graph. In particular, the exterior vertices turn out to be inaccessible and the behavior of the process near the interior vertices is described in terms of the domain $D(\mathcal L)$ in the following way: for interior vertex $O_i$, there are constants $p_k$ such that each function $f\in D(\mathcal L)$ satisfies
\begin{equation}
\label{eq:gluing_condtion}
\sum_{I_k\sim O_i}p_k\lim_{h_k\to O_i}f'(h_k)=0.
\end{equation}
Intuitively, the absolute value of $p_k$ is proportional to the probability of entering edge $I_k$ after the process arrives at the vertex $O_i$. The relation \eqref{eq:gluing_condtion} is usually referred to as the gluing condition. 
In the next section, we will formulate the results along with the assumptions more precisely. The coefficients $p_k$ will be calculated explicitly. As we mentioned, similar results hold in case of additive perturbations of Hamiltionian systems. Now the techniques in the proof are more involved and require new ideas with analysis on multiple time scales : $O(\e^{-1})$, $O(1)$, $O(\e)$, etc.
It is worth noting that our result provides the first example where the motion on a graph and the corresponding gluing conditions appear as a result of averaging of a fast-slow system.

To start with, since our interest is in the long-time behavior of $\bx_t^\e$ on $O(\e^{-1})$ time scales, it is often convenient to consider a temporally re-scaled process $(X_t^\e,\xi_t^\e)$:
\begin{equation}
\label{eq:rescaled_process1}
\begin{aligned}
    d X_t^\e=&~\frac{1}{\e}b(X_t^\e,\xi_t^\e)dt,~~~~~~~ X_0^\e=x_0\in\mathbb R^2,\\
    d\xi_t^\e=&~\frac{1}{\e^2}v(\xi_t^\e)dt+\frac{1}{{\e}}\sigma(\xi_t^\e)dW_t,~~~\xi_0^\e=y_0\in \mathbb T^m.
\end{aligned}
\end{equation}
It is clear that $(X_t^\e,\xi_t^\e)=(\bx_{t/\e}^\e,\bxi_{t/\e}^\e)$ in distribution.
Thus, it suffices to prove the weak convergence of $h(X_t^\e)$ in the space $\bm{\mathrm{C}}([0,T],\mathbb G)$, where $\mathbb G$ is the Reeb graph.
The proof of the weak convergence relies on demonstrating that the pre-limiting process asymptotically solves the martingale problem. Namely, we will show that, for each $f$ in a sufficiently large subset of $D(\mathcal L)$ and each $T>0$,
\begin{equation}
\label{eq:martingale_problem}
    \E_{(x,y)}[f(h(X_{T}^\e))-f(h(x))-\int_0^T\mathcal Lf(h(X_{t}^\e))dt]\to 0, 
\end{equation}
as $\e\to 0$, uniformly in $x$ in any compact set in $\mathbb R^2$ and in $y\in\mathbb T^m$.
The main idea in our proof is to divide the time interval $[0,T]$ into smaller intervals between different visits to the separatrices and show that the contribution from each excursion is small and they do not accumulate. 
For example, suppose for now that there is only one saddle point, as shown in Figure~\ref{fig:markov_chain}. 
Let $O$ be the saddle point with $H(O)=0$, $\gamma$ be the separatrix, $\gamma'=\{x:|H(x)|=\e^\alpha\}$ be a set near the separatrix, where $0<\alpha<1/2$, and $\sigma\geq0$ be the first time when the process $X_{t}^\e$ reaches $\gamma$. Define inductively the two sequences of stopping times:
\begin{equation}
\label{eq:stopping_times}
   \begin{aligned}
    \sigma_0=&~\sigma,\\
    \tau_n=&\inf\{t>\sigma_{n-1}:X_t^\e\in\gamma'\},\\
    \sigma_n=&\inf\{t>\tau_{n}:X_t^\e\in\gamma\},
    \end{aligned}
\end{equation}
and consequently two Markov chains $(X_{\tau_n}^\e,\xi_{\tau_n}^\e)$ and $(X_{\sigma_n}^\e,\xi_{\sigma_n}^\e)$.
% Figure environment removed
As pointed out earlier, we wish to prove that the contributions to \eqref{eq:martingale_problem} from all individual excursions are small and the sum converges to zero as $\e\downarrow0$. 
Thus, it is sufficient to show that (a) the expectation corresponding to one excursion is exactly zero if the process starts with the invariant measure of the Markov chain $(X_{\sigma_n}^\e,\xi_{\sigma_n}^\e)$ on $\gamma\times\mathbb T^m$, and (b) the measures on $\gamma\times\mathbb T^m$ induced by $(X_{\sigma_n}^\e,\xi_{\sigma_n}^\e)$ converge exponentially, as $n\to\infty$, uniformly in $\e$ and in starting point, to the invariant one. 

%----------------------

\iffalse
The first assertion can be used to calculate the gluing conditions by relating the expectation to the invariant measure on the whole space $\mathbb R^2\times\mathbb T^m$. 
Because $f(h(X_{\sigma_n}^\e))$ is the same for all $n$, the expectation only contains that integral term, i.e.
\begin{equation}
\label{eq:mt_prob_stopped}
    \E_{\nu}\int_{0}^{\sigma}\mathcal Lf(h(X_{t}^\e))dt, 
\end{equation}
where $\nu$ is the invariant measure on $\gamma\times\mathbb T^m$ induced by $(X_{\sigma_n}^\e,\xi_{\sigma_n}^\e)$. Then, by Khas'minskii's result in \cite{Khasminskii}, this expectation is the same as the integral of $\mathcal Lf(h(x))$ w.r.t. the invariant measure on $\mathbb R^2\times\mathbb T^m$. And the constants in the gluing condition should be chosen in such a way that this space integral is zero.
\fi
%----------------------

The claim in (a) is true if the gluing conditions are chosen appropriately and there is a common invariant measure for the processes $(X_t^\e,\xi_t^\e)$ for all $\varepsilon$. In general, there is no common invariant measure for all $\e$, and we need to consider a family of auxiliary processes near the separatrix that do 
have a common invariant measure, and then use the proximity of the auxiliary and the original processes and the Girsanov theorem. 
%
%which induces a %common invariant %measure and is close %to the original one %so that we can show %that the gluing %conditions are in %fact the same by %Girsanov's Theorem. 
The assertion in (b) is harder to verify, and its proof requires new techniques, including a local limit theorem and density estimates for hypoelliptic diffusions that will be discussed in later sections.


The paper is organized as follows: In Section~\ref{sec:mainresult}, we introduce the notations, state the assumptions, and formulate the main result. 
In Section~\ref{sec:preliminaries}, we construct an auxiliary process and derive diffusion approximations of the processes. 
In Section~\ref{sec:averaging}, we prove the averaging principle up to the time when the process reaches the separatrix. 
In Section~\ref{sec:exponenitalconvergence}, we construct the Markov chain on the separatrix (see \eqref{eq:stopping_times}) and prove its mixing properties. 
In Section~\ref{sec:proofofthemainresult}, we prove the main result. A few technical results are included in the Appendix.