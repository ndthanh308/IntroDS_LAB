\section{Exponential Convergence on Separatrix}
\label{sec:exponenitalconvergence}
We fix $0<\alpha<1/2$. As in \eqref{eq:stopping_times}, we define inductively two sequences of stopping times $\sigma_n$, $n\geq0$, and $\tau_n$, $n\geq 1$, but now for the general process $(\tx_t^\e,\txi_t^\e)$ on $M\times\mathbb T^m$ with additional drift $c(x,y)$. Without loss of generality, we assume that the saddle point $O$ satisfies $H(O)=0$. Let $V^\e=\{x:|H(x)|<\e^\alpha\}$ and $U_1$, $U_2$, $U_3$ be the three domains separated by $\gamma$. We aim to prove that the distribution of Markov chain $(\tx^\e_{\sigma_n},\txi^\e_{\sigma_n})$ converges in total variation exponentially fast, uniformly in $\e$ and in the initial distribution. Namely, we have the following lemma.
%---------------------------------------------------------------------------------


\begin{lemma}
\label{lem5:expo_ergodicity}
Let $\nu^{n,\e}_{x,y}$ denote the measure on $\gamma\times\mathbb T^m$ induced by $(\tx_{\sigma_n}^\e,\txi_{\sigma_n}^\e)$ with starting point $(x,y)\in\gamma\times\mathbb T^m$. Then there exist a probability measure $\nu^\e$ on $\gamma\times\mathbb T^m$ and constants $\Xi>0$ and $0<c<1$ such that, for all $\e$ sufficiently small,
\begin{equation}
\label{eq:exponentialconvergence}
    \sup_{(x,y)\in\gamma\times\mathbb T^m}{\mathrm{TV}}(\nu^{n,\e}_{x,y},\nu^\e)<\Xi \cdot (1-c)^n,
\end{equation}
where TV is the total variation distance of probability measures. 
\end{lemma}
The rest of this section is devoted to the proof of Lemma~\ref{lem5:expo_ergodicity}. 
Let $\bm\sigma_n$, $n\geq0$, and $\bm\tau_n$, $n\geq1$, be the stopping times w.r.t. $(\tbx_t^\e,\tbxi_t^\e)$ that are analogous to $\sigma_n$, $\tau_n$ w.r.t. $(\tx_t^\e,\txi_t^\e)$.
The lemma is equivalent to the exponential convergence in total variation of $(\tbx_{\bm\tau_n}^\e,\tbxi_{\bm\tau_n}^\e)$ on $\gamma'\times\mathbb T^m$, uniformly in $\e$ and in the initial distribution. 
The proof consists of three steps:
\begin{enumerate}
    \item[\textbf{1.}] The process starting on $\gamma'\times\mathbb T^m$ hits $I\times\mathbb T^m$ before $\bm\tau_1$ with uniformly positive probability, where $I$ is an fixed interval on the separatrix.
    \item[\textbf{2.}] Let the process starting on $I\times\mathbb T^m$ evolve for a certain period of time. Then, by a local limit theorem, we can estimate from below the probabilities of hitting $O(\e)$-sized boxes in a certain $O(\sqrt{\e})$-sized region, uniformly w.r.t. the starting point on $I\times\mathbb T^m$.
    \item[\textbf{3.}] By the H\"ormander condition \hyperlink{H5}{\textit{(H5)}}, we prove a common lower bound for the density of the distribution of the process starting from each of the $O(\e)$-sized boxes after a short time.
    %Moreover, the distribution with the common lower bound as its density has a uniformly positive probability.
\end{enumerate}


Let us take care of these steps in order. 


\textbf{Step 1}. Let $0<\beta<1$, which will be specified later. We prove that the process has a uniformly positive probability of following along the averaged motion and going through a neighborhood of the saddle point without making a deviation more than $\beta\sqrt{\e}$ in terms of $H$.
\begin{lemma}
\label{lem:stay_close_to_averaged}
For each fixed $\hat t>0$, $\beta'>0$,
\[\Prob_{(x,y)}\left(\sup_{0\leq t\leq\hat t}|\tbx_t^\e-\bm x_t|\leq\beta'\sqrt{\e}\right),\]
is uniformly positive for all $(x,y)\in M\times\mathbb T^m$ and $\e$ sufficiently small.
\end{lemma}
\begin{proof}
     Let the eigenvalues of $\nabla^2 H$ be bounded by $K$.
    Recall formula \eqref{eq:slowx}. By the boundedness of the coefficients, the event
    \[E:=\left\{\sup_{0\leq t\leq\Tilde t}|\int_0^t \nabla_y u(\tbx_s^\e,\tbxi_s^\e)\sigma(\tbxi_s^\e)dW_s|\leq\frac{1}{2}\beta'e^{-K\Tilde t}\right\}\]
    has positive probability, uniformly in the starting points.
    By \eqref{eq:slowx}, we have that on the event $E$, for $t\leq\Tilde t$ and $\e$ sufficiently small,
    \begin{align*}
        |\tbx_t^\e-\bm x_t|&\leq |\int_0^t(\nabla^\perp H(\tbx_s^\e)-\nabla^\perp H(\bm x_s)ds|+\sqrt{\e}|\int_0^t \nabla_y u(\tbx_s^\e,\tbxi_s^\e)\sigma(\tbxi_s^\e)dW_s|\\
        +\e|\int_0^t &[\nabla_x u(\tbx_s^\e,\tbxi_s^\e)b(\tbx_s^\e,\tbxi_s^\e)+\nabla_y u(\tbx_s^\e,\tbxi_s^\e)c(\tbx_s^\e,\tbxi_s^\e)]ds|+\e|u(x,y)-u(\tbx_t^\e,\tbxi_t^\e)|\\
        &\leq K\int_0^t|\tbx_s^\e-\bm x_s|ds+\beta'e^{-K\Tilde t}\sqrt{\e}.
    \end{align*}
    Then Gr\"onwall's inequality implies that $|\tbx_t^\e-\bm x_t|\leq\beta'\sqrt{\e}$ for all $t\leq\Tilde{t}$. Therefore, $E$ implies $\{\sup_{0\leq t\leq\hat t}|\tbx_t^\e-\bm x_t|\leq\beta'\sqrt{\e}\}$, and the uniform positivity follows.
\end{proof}
\begin{lemma}
\label{lem:throughsaddlepoint}
     For any given $0<c<1$, there exist curves $\Gamma_1$ and $\Gamma_2$ in $U_1$ such that 
    \begin{enumerate}[(i)]
        \item $\Gamma_1$ and $\Gamma_2$ have their tangent vectors as $\nabla H$. They intersect with the separatrix on different sides of the saddle point and the averaged motion on the separatrix spends finite time from $\Gamma_2$ to $\Gamma_1$. 
        \item Let $x\in\Gamma_1$ satisfy $2\beta\sqrt{\e}\leq |H(x)|\leq 2\sqrt{\e}$ and $\tau_x=\inf\{t:\tbx_t^\e\in\Gamma_2\}$. Then for all $y\in\mathbb T^m$, $\Prob_{(x,y)}(\sup_{0\leq t\leq \tau_x}|H(\tbx^\e_t)-H(x)|\leq\beta\sqrt{\e})>c$ for all $\e$ sufficiently small.
    \end{enumerate}
\end{lemma}
% Figure environment removed
\begin{proof}
     Suppose $H(x)>0$ for all $x\in U_1$. By the Morse lemma, there exist neighborhoods $U$ and $V$ of the saddle point $O$ and the origin, respectively, and a diffeomorphism $\psi$ from $U$ to $V$ such that $H(x)=G(\psi(x))$, where $G(z)=z_1z_2$. Then consider a random change of time by dividing the generator by $D(x):=\mathrm{det}(\nabla_x\psi(x))$: 
    \begin{equation}
    \label{eq:randomchangeoftime}
    \begin{aligned}
        d\st{\tbx_t}&=\frac{b(\st{\tbx_t},\st{\tbxi_t})}{D(\st{\tbx_t})}dt,\\
        d\st{\tbxi_t}&=\frac{1}{\e}\frac{v(\st{\tbxi_t})}{D(\st{\tbx_t})}dt+\frac{1}{\sqrt\e}\frac{\sigma(\st{\tbxi_t})}{\sqrt{D(\st{\tbx_t})}}dW_t+\frac{c(\st{\tbx_t},\st{\tbxi_t})}{D(\st{\tbx_t})}dt.
    \end{aligned}
    \end{equation}
    Write the equation for $\st{\tz_t}:=\psi(\st{\tbx_t})$:
    \[
        d\st{\tz_t}=\frac{1}{D(\psi^{-1}(\st{\tz_t}))}\cdot \nabla_x\psi(\psi^{-1}(\st{\tz_t}))b(\psi^{-1}(\st{\tz_t}),\st{\tbxi_t})dt=:b^*(\st{\tz_t},\st{\tbxi_t})dt.
    \]
    It is not hard to verify that $b^*(z,y)$ satisfies
    $$\int_{\mathbb T^m}b^*(z,y)d\mu(y)=\nabla^\perp G(z).$$
    Hence, by Lemma~\ref{lem:solution}, there exists a bounded solution $u^*(z,y)$ to 
    \[Lu^*(z,y)=-(b^*(z,y)-\nabla^\perp G(z))\cdot D(\psi^{-1}(z)).\]
    Consider a local coordinate $G=z_1z_2$ and $\phi^*=\frac{1}{2}\log({z_2}/{z_1})$ in $V$.
    The averaged motion has constant speed: $0$ in $G$ and $1$ in $\phi^*$. 
    As in \eqref{eq:H} and \eqref{eq:phi}, we have the equations for $\st{\Tilde G_t}=G(\st{\tz_t})$, $\st{\Tilde \Phi_t}=\phi^*(\st{\tz_t})$, by applying Ito's formula to $u^*_g=u^*\cdot\nabla G$ and $u^*_{\phi}=u^*\cdot\nabla \phi^*$, with 
    $z=\psi(x)$, $g_0=G(z)$, and $\phi^*_0=\phi^*(z)$:
    \begin{align}
        \st{\Tilde G_t}=~&g_0+\sqrt{\e}\int_0^t \nabla_y u^*_g(\st{\tz_s},\st{\tbxi_s})^{\mathsf T}\frac{\sigma(\st{\tbxi_s})}{\sqrt{D(\psi^{-1}(\st{\tz_s}))}}dW_s-\e(u^*_g(\st{\tz_t},\st{\tbxi_t})-u^*_g(z,y))\nonumber\\
        &+\e\int_0^t\left[\nabla_z u^*_g(\st{\tz_s},\st{\tbxi_s})\cdot b^*(\st{\tz_s},\st{\tbxi_s})+\nabla_y u^*_g(\st{\tz_s},\st{\tbxi_s})\cdot \frac{c(\psi^{-1}(\st{\tz_s}),\st{\tbxi_s})}{D(\psi^{-1}(\st{\tz_s}))}\right]ds,\label{eq:G}\\
        \st{\Tilde \Phi_t}=~&\phi^*_0+t+\sqrt{\e}\int_0^t \nabla_y u^*_\phi(\st{\tz_s},\st{\tbxi_s})^{\mathsf T}\frac{\sigma(\st{\tbxi_s})}{\sqrt{D(\psi^{-1}(\st{\tz_s}))}}dW_s-\e(u^*_\phi(\st{\tz_t},\st{\tbxi_t})-u^*_\phi(z,y))\nonumber\\
        &+\e\int_0^t\left[\nabla_z u^*_\phi(\st{\tz_s},\st{\tbxi_s})\cdot b^*(\st{\tz_s},\st{\tbxi_s})+\nabla_y u^*_\phi(\st{\tz_s},\st{\tbxi_s})\cdot \frac{c(\psi^{-1}(\st{\tz_s}),\st{\tbxi_s})}{D(\psi^{-1}(\st{\tz_s}))}\right]ds.\label{eq:Phi*}
    \end{align}
    
    To get the lower bound for the desired probability, we will choose the curves $\Gamma_1$ and $\Gamma_2$ that are close enough to the saddle point. 
    The time it takes to get from $\Gamma_1$ to $\Gamma_2$ is still of order $|\log\e|$ since they are chosen independently of $\e$.
    In this way, the process starting on $\Gamma_1$ and stopped on $\Gamma_2$ will be shown to have small variance, hence it is unlikely for the process to have deviations larger than what we wish. 
    With $C>0$ to be specified later, let $l_1=\{z:\phi^*(z)=\frac{1}{4}\log\e+\frac{1}{2}\log\beta+C\}$, $l_2=\{z:\phi^*(z)=-(\frac{1}{4}\log\e+\frac{1}{2}\log\beta+C)\}$, and $l_3=\{z:\phi^*(z)=-(\frac{1}{4}\log\e+\frac{1}{2}\log\beta+C)-2\}$.
    The idea is to look at event that the process stays close to the averaged motion before the latter reaches $l_2$, which implies that the process does not make a large deviation in $G$, or equivalently, in $H$, before reaching $l_3$.
    Let $\Gamma_1^*$, $\Gamma_2^*$ be the curves that have tangent vectors as $\nabla_x\psi\circ\psi^{-1}({\nabla_x\psi\circ\psi^{-1}})^{\mathsf T}\nabla G$ and go through the points $(e^{-C},e^C\beta\sqrt{\e})$, $(e^{C+1}\beta\sqrt{\e},e^{-C-1})$, respectively.
    Since $\psi$ is a diffeomorphism, it is easy to see that each $z$ on $\Gamma_1^*$ or $\Gamma_2^*$ with $G(z)\geq\beta\sqrt{\e}$ satisfies that $\frac{1}{4}\log\e+\frac{1}{2}\log\beta+C\leq\phi^*(z)\leq-(\frac{1}{4}\log\e+\frac{1}{2}\log\beta+C)-2$.
    Let $\Gamma_1$ and $\Gamma_2$ be the pre-images of $\Gamma_1^*$ and $\Gamma_2^*$ in $U_1$. They have $\nabla H$ as tangent vectors due to the specific way we construct $\Gamma_1^*$ and $\Gamma_2^*$.
    Consider the process in \eqref{eq:randomchangeoftime} starting at $x\in\Gamma_1$ satisfying that $2\beta\sqrt{\e}\leq H(x)\leq 2\sqrt{\e}$ with an arbitrary $y\in\mathbb T^m$. 
    Let $\phi_t^*=\phi^*_0+t$.
    Define $t_x=\inf\{t:\phi_t^*=\phi^*(l_2)\}$ and $\tau_x^*=\inf\{t:|\st{\Tilde G_t}-g_0|=\beta\sqrt{\e}\}\wedge \inf\{t:|\st{\Tilde\Phi_t}-\phi_t^*|=1\}\wedge t_x$. 
    Then it is clear that $\Prob_{(x,y)}(\sup_{0\leq t\leq \tau^*_x}|H(\tbx^\e_t)-H(x)|\leq\beta\sqrt{\e})\geq\Prob_{(x,y)}(\tau_x^*=t_x)$. 
    Let $S_G$ and $S_\phi$ denote the stochastic integrals in \eqref{eq:G} and \eqref{eq:Phi*}, respectively, with $t$ replaced by $\tau_x^*$. 
    Since $\tau^*_x\lesssim|\log\e|$, $\nabla G$ is bounded, and $\nabla \phi^*\lesssim\e^{-1/2}$ before $\tau^*_x$, we see that the unwanted deviations happen only if $S_G$ and $S_\phi$ are large.
    Namely,
    \[\Prob_{(x,y)}(\tau_x^*<t_x)\leq\Prob_{(x,y)}(|S_G|\geq\beta\sqrt{\e}/2)+\Prob_{(x,y)}(|S_\phi|\geq1/2).\]
    Both terms on the right hand side can be controlled by Chebyshev's inequality. 
    Note that there exists a constant $K>0$ independent of $\e$ such that
    \begin{align*}
        \bm{\mathrm{Var}}(S_G)&\leq\e K\E\int_0^{\tau_x^*}|\nabla G(\st{\tz_s})|^2ds\\
        &=\e K\E\int_0^{\tau_x^*} \st{\Tilde G_s}(e^{2\st{\Tilde \Phi_s}}+e^{-2\st{\Tilde \Phi_s}})ds\\
        &\leq \e K\int_0^{\tau_x^*}(2+\beta)\sqrt{\e}e^2(e^{2{\phi_s^*}}+e^{-2{\phi_s^*}})ds\\
        &\leq  3K\sqrt{\e^3}e^2\int_0^{-2(\frac{1}{4}\log\e+\frac{1}{2}\log\beta+C)}(e^{2{\phi_s^*}}+e^{-2{\phi_s^*}})ds\\
        &=\ 3K\sqrt{\e^3}e^2\int_{\frac{1}{4}\log\e+\frac{1}{2}\log\beta+C}^{-(\frac{1}{4}\log\e+\frac{1}{2}\log\beta+C)}(e^{2\varphi}+e^{-2\varphi})d\varphi\\
        &\leq \ \frac{3}{\beta}Ke^{2-2C}\e,
    \end{align*}and, similarly,
    \[
        \bm{\mathrm{Var}}(S_\phi)\leq \e K\E\int_0^{\tau^*_x}|\nabla \Phi(\st{\tz_s})|^2ds
        \leq\e K\E\int_0^{\tau^*_x} \frac{1}{\st{\Tilde G}_s}(e^{2\st{\Tilde \Phi_s}}+e^{-2\st{\Tilde\Phi_s}})ds\leq  \frac{1}{\beta^2}Ke^{2-2C}.
    \]
    Then $C$ can be chosen large enough such that both variances are small enough, and hence $\Prob_{(x,y)}(|H(\tbx^\e_{\tau^*_x})-H(x)|\leq\beta\sqrt{\e})>c$. 
\end{proof}
% Figure environment removed
We can choose the corresponding curves in the other regions. As a result, we have four curves corresponding to four different directions, all with positive distance to the saddle point, as shown in Figure~\ref{fig:four curves}. Moreover, the corresponding transition probabilities near the saddle point have lower bounds analogous to that given in Lemma~\ref{lem:throughsaddlepoint} (ii). For the rotations happening away from those curves, we will prove that, before the time when the process comes back to the curves, the deviation of $H$ can be large enough to cross the separatrix with positive probability. Let $\Gamma_i(h_1,h_2)$ be the set $\{x\in\Gamma_i:h_1\leq H(x)\leq h_2\}$.
\begin{lemma}
    \label{lem:hit_separatrix}
    For each fixed $\hat t>0$,
    \[\Prob_{(x,y)}\left(\inf_{0\leq t\leq\hat t}H(\tbx_{t}^\e)\leq -\sqrt{\e},\sup_{0\leq t\leq\hat t}|\tbx_t^\e-\bm x_t|\leq\e^{\frac{1+2\alpha}{4}}\right)\]
    is positive uniformly in {$x\in\Gamma_2(0,2\sqrt{\e})$}, $y\in\mathbb T^m$, and all $\e$ sufficiently small.
\end{lemma}
\begin{proof}
    By Lemma~\ref{lem:stay_close_to_averaged} and the Markov property, it is enough to consider small $\hat t$ such that $\bm x_t$ does not reach $\Gamma_1$ before $\hat t$. Using formula \eqref{eq:slowx} again, we see that $\Prob_{(x,y)}(\sup_{0\leq t\leq\hat t}|\tbx_t^\e-\bm x_t|>\e^{\frac{1+2\alpha}{4}})\to 0$ as $\e\downarrow0$ uniformly in $(x,y)$. Use formula \eqref{eq:H} on a shorter time scale:
    \begin{align*}
        H(\tbx_{t}^\e)&=H(x)+\sqrt{\e}\int_0^{t} \nabla_y u_h(\tbx_s^\e,\tbxi_s^\e)^{\mathsf T}\sigma(\tbxi_s^\e)dW_s+\e(u_h(x,y)-u_h(\tbx_{t}^\e,\tbxi_{t}^\e))\\\
    &\quad+\e\int_0^{t}[\nabla_x u_h(\tbx_s^\e,\tbxi_s^\e)\cdot b(\tbx_s^\e,\tbxi_s^\e)+\nabla_y u_h(\tbx_s^\e,\tbxi_s^\e)\cdot c(\tbx_s^\e,\tbxi_s^\e)]ds.
    \end{align*}
    So it suffices to show the uniform positivity of 
    \[\Prob_{(x,y)}\left(\inf_{0\leq t\leq\hat t}\int_0^{t} \nabla_y u_h(\tbx_s^\e,\tbxi_s^\e)^{\mathsf T}\sigma(\tbxi_s^\e)dW_s\leq -4,\sup_{0\leq t\leq\hat t}|\tbx_t^\e-\bm x_t|\leq\e^{\frac{1+2\alpha}{4}}\right).\]
    Note that there exists another Brownian motion $\Tilde W$ such that
    \begin{equation}
    \label{eq5:time_changed_brownian_motion}
    \int_0^{t} \nabla_y u_h(\tbx_s^\e,\tbxi_s^\e)^{\mathsf T}\sigma(\tbxi_s^\e)dW_s=\Tilde W\left(\int_0^{t} |\nabla_y u_h(\tbx_s^\e,\tbxi_s^\e)^{\mathsf T}\sigma(\tbxi_s^\e)|^2ds\right).
    \end{equation}
    Recall that in Section~\ref{sec:Averaging principle before} we defined $A(x)=\int_{\mathbb T^m}|\nabla_y u_h(x,y)\sigma(y)|^2d\mu(y)$. Then, by Corollary~\ref{cor:avg},
    \begin{equation}
    \label{eq:close_in_L1}
        \E_{(x,y)}\left|\int_0^{\hat t} |\nabla_y u_h(\tbx_s^\e,\tbxi_s^\e)^{\mathsf T}\sigma(\tbxi_s^\e))|^2ds-\int_0^{\hat t} A(\tbx_s^\e)ds\right|=O(\sqrt\e).
    \end{equation}
    Note that on the event $\{\sup_{0\leq t\leq\hat t}|\tbx_t^\e-\bm x_t|\leq\e^{\frac{1+2\alpha}{4}}\}$, $A(\tbx_t^\e)$ is uniformly positive for $0\leq t\leq\hat t$. Let us denote this lower bound as $m$, which is independent of $x$, $y$, and $\e$. Then
    \begin{equation}
        \Prob_{(x,y)}(\int_0^{\hat t} A(\tbx_s^\e)ds>m\hat t,\sup_{0\leq t\leq\hat t}|\tbx_t^\e-\bm x_t|\leq\e^{\frac{1+2\alpha}{4}})\to1.
    \end{equation}
    By the $L^1$ convergence in \eqref{eq:close_in_L1}, we obtain
    \begin{equation}
    \label{eq5:positive_variance_one_rotation}
        \Prob_{(x,y)}\left(\int_0^{\hat t} |\nabla_y u_h(\tbx_s^\e,\tbxi_s^\e)^{\mathsf T}\sigma(\tbxi_s^\e))|^2ds>m\hat t/2,\sup_{0\leq t\leq\hat t}|\tbx_t^\e-\bm x_t|\leq\e^{\frac{1+2\alpha}{4}}\right)\to1.
    \end{equation}
    Suppose $0<c<\Prob(\inf_{0\leq t\leq m\hat t/2}\Tilde W_t<-4)$.
    Then, for all $\e$ sufficiently small,
    \begin{align*}
        &\Prob_{(x,y)}\left(\inf_{0\leq t\leq\hat t}\int_0^{t} \nabla_y u_h(\tbx_s^\e,\tbxi_s^\e)^{\mathsf T}\sigma(\tbxi_s^\e)dW_s\leq -4,\sup_{0\leq t\leq\hat t}|\tbx_t^\e-\bm x_t|\leq\e^{\frac{1+2\alpha}{4}}\right)\\
        &=\Prob_{(x,y)}\left(\inf_{0\leq t\leq\hat t}\Tilde W\left(\int_0^{t} |\nabla_y u_h(\tbx_s^\e,\tbxi_s^\e)^{\mathsf T}\sigma(\tbxi_s^\e))|^2ds\right)\leq -4,\sup_{0\leq t\leq\hat t}|\tbx_t^\e-\bm x_t|\leq\e^{\frac{1+2\alpha}{4}}\right)\\
        &\geq\Prob_{(x,y)}\left(\inf_{0\leq t\leq m\hat t/2}\Tilde W_t\leq -4,\int_0^{\hat t} |\nabla_y u_h(\tbx_s^\e,\tbxi_s^\e)^{\mathsf T}\sigma(\tbxi_s^\e))|^2ds>m\hat t/2,\sup_{0\leq t\leq\hat t}|\tbx_t^\e-\bm x_t|\leq\e^{\frac{1+2\alpha}{4}}\right)\\
        &\geq c/2.
    \end{align*}
\end{proof}
\begin{remark}
\label{rmk:hit_separatrix}
    The result in Lemma~\ref{lem:hit_separatrix} also holds for $x\in\Gamma_4(0,2\sqrt{\e})$. Similarly, for each fixed $\hat t>0$,
    \[\Prob_{(x,y)}\left(\sup_{0\leq t\leq\hat t}H(\tbx_{t}^\e)\geq \sqrt{\e},\sup_{0\leq t\leq\hat t}|\tbx_t^\e-\bm x_t|\leq\e^{\frac{1+2\alpha}{4}}\right)\]
    is positive uniformly in $x\in\Gamma_2(-2\sqrt{\e},0)\cup\Gamma_4(-2\sqrt{\e},0)$, $y\in\mathbb T^m$, and $\e$ sufficiently small.
\end{remark}
%--------------------------------------------------------------------------------------

%--------------------------------------------------------------------------------------
Now we can choose $\beta=1/10$. By the results in Lemma~\ref{lem:stay_close_to_averaged}, Lemma~\ref{lem:throughsaddlepoint}, Lemma~\ref{lem:hit_separatrix}, and Remark~\ref{rmk:hit_separatrix}, using the strong Markov property, we obtain the following lemma:
\begin{lemma}
\label{lem:rotation}
    There exist a closed interval $I$ on $\gamma$ that does not contain the saddle point and a constant $0<c<1$ satisfying the following property: if the system \eqref{eq:auxiliary} starts at $(x,y)\in\gamma'\times\mathbb T^m$, then for all $\e$ sufficiently small
    \begin{equation}
        \Prob_{(x,y)}(\Tilde\eta_1<\bm\tau_1)\geq c
    \end{equation}
    where $\Tilde\eta_1=\inf\{t:\tbx_t^\e\in I\}$.
\end{lemma}
\begin{remark}
\label{rmk:positive_any_subset_separatrix}
    In order for us to apply Lemma~\ref{lem:hit_separatrix}, we need to choose $I$ that contains the intersection of $\Gamma_2$ and $\gamma$ in its interior. In fact, it is not difficult to show that Lemma~\ref{lem:rotation} holds for any subset of $\gamma$ with non-empty interior.
\end{remark}

\textbf{Step 2}. 
Without loss of generality, we assume that if $\bm x_t$ starts at one endpoint of $I$, then the other endpoint is $\bm x_{1/2}$.
In the remainder of this section, $\bm x_t$ always denotes this deterministic motion, irrespective of where $\tbx_t^\e$ starts. 
We aim to study the distribution of the process $(\tbx_t^\e,\tbxi_t^\e)$ starting on $I\times\mathbb T^m$ with certain $t>0$. The choice of $t$ will depend on the initial point $x$ being considered (see Figure~\ref{fig:local_limit_theorem}), and this will be convenient as we use the strong Markov property later when combining all three steps.
% Figure environment removed
To be more precise, for $x\in I$, let $s(x)$ be such that $\bm x_{s(x)}=x$ (so $0\leq s(x)\leq1/2$). We introduce a process $\tilde\zeta_{t}^{\e}$, $0\leq t$, as the second term in the expansion of $\tbx_{t}^{\e}$ around the deterministic motion $\bm x_{(s(x)+t)}$:
\begin{equation}
\label{eq:linearize_process_tilde}
\begin{aligned}
d\tilde\zeta_{t}^{\e}&=\nabla(\nabla^\perp H)(\bm x_{s(x)+t})\tilde\zeta_{t}^{\e}dt+[b(\bm x_{s(x)+t},\tbxi_{t}^{\e})-\nabla^\perp H(\bm x_{s(x)+t})]dt,~\tilde\zeta_{0}^{\e}=0.
\end{aligned}
\end{equation}
(Note that, for finite $t$, $\tilde\zeta_{s,t}^{\e,y}$ is of order $\sqrt{\e}$.) Then, by standard perturbation arguments and Gr\"onwall's inequality, it can be shown that, uniformly in $x$ such that $0\leq s(x)\leq1/2$, $0\leq t\leq 1-s(x)$, and $y\in\mathbb T^m$,
\begin{equation}
\label{eq:xtildecloseinL1}
    \E_{(x,y)}|\tbx_{t}^{\e}-\bm x_{s(x)+t}-\tilde\zeta_{t}^{\e}|=O(\e).
\end{equation}
Therefore, understanding of the distribution of $\tilde\zeta_{1-s(x)}^{\e}$ would help one to understand the distribution of $\tbx_{1-s(x)}^{\e}$.
However, it is not straightforward to study $\tilde\zeta_{t}^{\e}$ since $(\tbxi_{t}^{\e},\tilde\zeta_{t}^{\e})$ is not a Markov process. 
We introduce a related process $\zeta_{t}^{\e}$ defined using the original Markov process $\bxi_t^\e$, apply the local limit theorem to $( \bxi_{t}^{\e}, \zeta_{t}^{\e})$, and use the Girsanov theorem to get the desired estimate. Namely, let $\zeta_{t}^{\e}$, $s\leq t$, be defined by:
\begin{equation}
\label{eq:linearizeprocess}
\begin{aligned}
d \zeta_{t}^{\e}&=\nabla(\nabla^\perp H)(\bm x_{s(x)+t}) \zeta_{t}^{\e}dt+[b(\bm x_{s(x)+t}, \bxi_{t}^{\e})-\nabla^\perp H(\bm x_{s(x)+t})]dt,~ \zeta_{0}^{\e}=0.
\end{aligned}
\end{equation}
The following result is a version of the local limit theorem \cite{LLT} adapted to our case. 
\begin{theorem}
\label{thm:locallimittheorem}
    %Let $\xi$ be the Markov process that converges to its unique invariant measure $\nu$ exponentially fast.
    Let $g:[0,1]\times \mathbb T^m\to\mathbb R^2$ be a $C^\infty$ function such that $g(t,\cdot)$ spans $\mathbb R^2$ and $\int_{\mathbb T^m} g(t,y)d\mu(y)=0$ for all $t\geq 0$, where $\mu$ is the invariant measure of $ \bxi_{t}^{\e}$. 
    Then a local limit theorem holds for the following random variable as $\e\to0$ uniformly in $(x,y)\in I\times\mathbb T^m$,
    \[S^\e:=\frac{1}{\e}\int_0^{1-s(x)} g(s(x)+t, \bxi_{t}^{\e})dt.\]
    Namely, there exists a invertible covariance matrix $B(s)$ continuous in $s$ such that
    \begin{equation}
    \label{eq5:locallimittheorem}
        \lim_{\e\to0}\left| \frac{2\pi}{\e}\sqrt{\mathrm{det}B(s(x))}\cdot\Prob_{(x,y)}(S^\e-u\in[0,1)^2)-\exp({-\frac{\e \langle B(s(x))^{-1}u,u\rangle}{2}})\right|=0,
    \end{equation}uniformly in $u\in\mathbb R^2 $, $x\in I$, and $y\in\mathbb T^m$.
\end{theorem}
The second term in \eqref{eq5:locallimittheorem} is non-trivial even when $u$ takes large values (of order $1/\sqrt{\e}$), which is exactly the situation we are dealing with.
Following \eqref{eq:linearizeprocess}, we solve explicitly
\begin{equation}
     \zeta_{1-s(x)}^{\e}=\int_0^{1-s(x)}U_{s(x)+t,1}(b(\bm x_{s(x)+t}, \bxi_{t}^{\e})-\nabla^\perp H(\bm x_{s(x)+t}))dt,
\end{equation}
where $U_{t,s}$ solves the differential equation
\[
dU_{t,s}=\nabla(\nabla^\perp H)(\bm x_{s})U_{t,s}ds,
\]
and $U_{t,t}$ is the identity matrix.
Since $\bm x_t$ is deterministic, the integrand can be treated as a function only of time $t$ and $ \bxi_{t}^{\e}$. 
Moreover, for each $t$, the integrand has zero mean w.r.t. the invariant measure and spans $\mathbb R^2$, since $U_{t,1}$ is deterministic and non-singular and, for each $x$, $\{b(x,y)-\nabla^\perp H(x):y\in\mathbb T^m\}$ spans $\mathbb R^2$ by assumption \hyperlink{H4'}{\textit{(H4$'$)}}.
Then Theorem~\ref{thm:locallimittheorem} implies that
\begin{equation}
\label{eq:A_jk}
    \Prob_{(x,y)}\left(\frac{1}{\e} \zeta_{1-s(x)}^{\e}\in[j,j+1)\times[k,k+1)\right)\geq\frac{\e}{4\pi\sqrt{\mathrm{det}B(s(x))}}\exp\left(-\frac{\e \langle B(s(x))^{-1}(j,k),(j,k)\rangle}{2}\right)
\end{equation}
for all $\e$ small enough, $-1/\sqrt{\e}\leq j,k\leq 1/\sqrt{\e}$, $x\in I$, and  $y\in\mathbb T^m$. 
Finally, we compare $(\tbx_{t}^{\e},\tbxi_{t}^{\e},\tilde\zeta_{t}^{\e})$ with $( \bx_{t}^{\e}, \bxi_{t}^{\e}, \zeta_{t}^{\e})$. 
Since the added drift $c(x,y)$ in the equation of $\tbxi_{t}^{\e}$ is small compared to the diffusion term $\frac{1}{\sqrt{\e}}\sigma(y)$, it is not hard to verify that, using the Girsanov theorem, for all $\e$ small enough, $-1/\sqrt{\e}\leq j,k\leq 1/\sqrt{\e}$, $x\in I$, and  $y\in\mathbb T^m$, 
\begin{equation}
    \label{eq:A_jk1}
    \Prob_{(x,y)}\left(\frac{1}{\e}\tilde\zeta_{1-s(x)}^{\e}\in[j,j+1)\times[k,k+1)\right)\geq \frac{1}{2}\Prob_{(x,y)}\left(\frac{1}{\e} \zeta_{1-s(x)}^{\e}\in[j,j+1)\times[k,k+1)\right).
\end{equation}

\textbf{Step 3}. We proved that $\bm x_1+\tilde\zeta_{1-s(x)}^{\e}$ reaches the $O(\e)-$sized boxes with probabilities bounded from below. 
We also proved that $\tbx_{1-s(x)}^{\e}$ is $O(\e)$-close to $\bm x_1+\tilde\zeta_{1-s(x)}^{\e}$ in $L^1$. 
Let us take one generic pair $(j,k)$, let $B^{\e,K}_{j,k}=\bm x_1+[(j-K)\e,(j+1+K)\e)\times[(k-K)\e,(k+1+K)\e)$, and study the distribution of $(\tbx_t^\e,\tbxi_t^\e)$ with the initial point in $B^{\e,K}_{j,k}\times\mathbb T^m$ after time of order $O(\e)$.
% Figure environment removed
\begin{lemma}
\label{lemma:forcetosmallerbox}
    For each $\kappa>0$, $K>0$, and $\hat y\in\mathbb T^m$, there exist $t_2>0$, $c>0$, and, for each pair $(j,k)$, a point $\hat x_{j,k}^\e$ such that, for each $(x,y)\in B^{\e,K}_{j,k}\times\mathbb T^m$ and all $\e$ sufficiently small,
    $$\Prob_{(x,y)}(\tau_\kappa<t_2\e)\geq c,$$
    where $\tau_\kappa=\inf\{t:\tbx_t^\e\in B(\hat x_{j,k}^\e,\kappa\e),~\tbxi_t^\e\in B(\hat y,\kappa)\}$.
\end{lemma}
\begin{proof}
    Recall the definition of $\bm x_1$ at the beginning of Step 2 (see Figure~\ref{fig:local_limit_theorem}).
    By assumption \hyperlink{H4'}{\textit{(H4$'$)}}, $\{b(\bm x_1,y):y\in\mathbb T^m\}$ spans $\mathbb R^2$. So there exist $y_1,y_2\in\mathbb T^m$ such that $v_1:=b(\bm x_1,y_1)$ and $v_2:=b(\bm x_1,y_2)$ span $\mathbb R^2$. 
    Let us consider the set $S_{j,k}=\bigcap_{x\in B_{j,k}^{\e,K}}\{x+av_1+bv_2:a,b\geq0\}$.
    Then it is easy to see that, there exist a constant $t_2>0$ and, for each pair $(j,k)$, a point $\hat x_{j,k}^\e\in S_{j,k}$  such that for all $x\in B_{j,k}^{\e,K}$, $\hat x_{j,k}^\e=x+ a_x \e v_1+b_x \e v_2$ and $0<a_x,b_x<t_2/5$.
    There exists $\delta>0$ such that for each $x\in B(\bm x_1,2\delta)$ and each $y$ in $B(y_i,2\delta)$, $|b(x,y)-v_i|<\kappa/t_2$, $i=1,2$.
    Let $M$ be the upper bound of vector $b(x,y)$.
    For all $\e$ sufficiently small, the probability of the following event, denoted by $E$, has a lower bound, denoted by $c$, that only depends on $t_2$, $\kappa$, $M$, $y_1$, $y_2$, $\hat y$, $\delta$, and not on the starting point $(x,y)\in B(\bm x_1,\delta)\times\mathbb T^m$, thus not on $(j,k)$: 
    \[E=\begin{Bmatrix}
    \tau_1<(t_2\wedge \kappa/M)\e/5;~\tbxi^\e_{\tau_1+t}\in B(y_1,2\delta),~t\in[0,a_x\e];~\tau_2<\tau_1+a_x+(t_2\wedge \kappa/M)\e/5; \\
    \tbxi^\e_{\tau_2+t}\in B(y_2,2\delta),~t\in[0,b_x \e];~\tau_3<\tau_2+b_x\e+(t_2\wedge \kappa/M)\e/5
    \end{Bmatrix},
    \]where $\tau_1=\inf\{t\geq0:\tbxi^\e_t\in B(y_1,\delta)\}$, $\tau_2=\inf\{t\geq\tau_1+a_{x}\e:\tbxi^\e_t\in B(y_2,\delta)\}$, and $\tau_3=\inf\{t\geq\tau_2+b_{x}\e:\tbxi^\e_t\in B(\hat{y},\kappa)\}$. If $E$ is a subset of the event  $\{\tau_\kappa<t_2\e\}$, then the lemma is proved. To show the inclusion, note that on $E$,
    \begin{align*}
        |\tbx_{\tau_3}^\e-\hat x_{j,k}^\e|&=|\tbx_{\tau_3}^\e-(x+a_{x}\e v_1+b_{x}\e v_2)|\\
        &\leq |\tbx_{\tau_3}^\e-\tbx_{\tau_2+b_{x}\e}^\e|+|\tbx_{\tau_2+b_{x}\e}^\e-(\tbx_{\tau_2}^\e+ b_{x}\e v_2)|+|\tbx_{\tau_2}^\e-\tbx_{\tau_1+a_{x}\e}^\e|\\
        &\quad\quad +|\tbx_{\tau_1+a_{x}\e}^\e-(\tbx_{\tau_1}^\e+ a_{x}\e v_1)|+|\tbx_{\tau_1}^\e-x|\\
        &\leq \kappa\e.
    \end{align*}
    Besides, by the definition of $\tau_3$, $\tbxi_{\tau_3}^\e\in B(\hat{y},\kappa)$. Thus $\tau_\kappa\leq\tau_3<t_2\e$ on $E$.
\end{proof}
From now on, let $\hat y$ be the point in assumption \hyperlink{H5}{\textit{(H5)}} such that the parabolic H\"ormander condition holds at $(\bm x_1,\hat y)$ and let $p_t^\e((x,y),\cdot)$ be the density of $(\tbx_{t\e}^\e,\tbxi_{t\e}^\e)$ starting at $(x,y)$.
\begin{lemma}
\label{lem5:density_xy}
    There exists $\kappa>0$ such that for each $\hat x\in B(\bm x_1,\kappa)$ and all $\e$ sufficiently small, there is a domain $C^\e_{\hat x,\hat y}\subset V^\e\times\mathbb T^m$ with $\lambda(C^\e_{\hat x,\hat y})>\kappa\e^2$ and $p_{1}^\e((x,y),\cdot)>\kappa/\e^2$ on $C^\e_{\hat x,\hat y}$ for $(x,y)\in B(\hat x,\kappa\e)\times B(\hat y,\kappa)$.
\end{lemma}
\begin{proof}
Consider the stochastic processes that depend on the parameters $(\e,x,y)$:
\label{eq:theta_process}
\begin{equation}
    \label{eq5:difference}
    \begin{aligned}
    d\theta_t^{\e,x,y}&=b(x+\e\theta_t^{\e,x,y},y+\eta_t^{\e,x,y})dt,~\theta_0^{\e,x,y}=0\in\mathbb R^2,\\
    d\eta_t^{\e,x,y}&= v(y+\eta_t^{\e,x,y})dt+\e c(x+\e\theta_t^{\e,x,y},y+\eta_t^{\e,x,y})dt+\sigma(y+\eta_t^{\e,x,y}) dW_t,~\eta_0^{\e,x,y}={0}\in\mathbb R^m.
    \end{aligned}
\end{equation}
Since, by assumption \hyperlink{H5}{\textit{(H5)}}, the parabolic H\"ormander condition for equation \eqref{eq:theprocess1} holds at $(\bm x_1,\hat y)$, it is not hard to see that, if $(x,y)$ is close to $(\bm x_1,\hat y)$ and $\e$ is small, the parabolic H\"ormander condition holds for \eqref{eq5:difference} at $0$ and the distribution of $(\theta_t^{\e,x,y},\eta_t^{\e,x,y})$ is absolutely continuous w.r.t. the Lebesgue measure (\cite{Nualart}). 
Moreover, if the density function, denoted by $\tilde p_1^{\e,x,y}(\theta,\eta)$, exists, it is continuous in $\e,x,y,\theta$, and $\eta$. 
Let $\hat\theta$ and $\hat\eta$ satisfy that $\tilde p_1^{0,\bm x_1,\hat y}(\hat\theta,\hat\eta)>0$. 
Then there exists $0<\delta<1$ such that $\tilde p_1^{\e,x,y}(\theta,\eta)$ exists and is greater than $\delta$ for all $0<\e<\delta$, $x\in B(\bm x_1,\delta)$, $y\in B(\hat y,\delta)$, $\theta\in B(\hat\theta,\delta)$, and $\eta\in B(\hat\eta,\delta)$. 
For $\hat x\in B(\bm x_1,\delta/2)$, define $C^\e_{\hat x,\hat y}=B(\hat x+\e\hat\theta,\e\delta/2)\times B(\hat y+\hat\eta,\delta/2)$. 
Then, for $(x,y)\in B(\hat x,\e\delta/2)\times B(\hat y,\delta/2)$, and $(x',y')\in C^\e_{\hat x,\hat y}$, and $0<\e<\delta$, we have that 
\[p^\e_1((x,y),(x',y'))=\frac{1}{\e^2}\tilde p^{\e,x,y}\left(\frac{x'-x}{\e},y'-y\right)>\frac{\delta}{\e^2}.\]
The result holds with $\kappa=(\delta/2)^{m+2}$.
\end{proof}
\begin{lemma}
\label{lem:pijk}
    For each $K>0$, there exist constants $c>0$ and $t_1>0$ such that for all $-1/\sqrt{\e}\leq j,k\leq1/\sqrt{\e}$, there exists a measure $\pi^\e_{j,k}$ and a stopping time $\Tilde\eta_3^{j,k}<t_1\e$ such that for each $(x,y)\in B^{\e,K}_{j,k}\times\mathbb T^m$, the distribution of $(\tbx_{\Tilde\eta_3^{j,k}}^\e,\tbxi_{\Tilde\eta_3^{j,k}}^\e)$ starting at $(x,y)$ has $\pi^\e_{j,k}$ as a component and $\pi^\e_{j,k}(V^\e\times\mathbb T^m)>c$ for all $\e$ sufficiently small.
\end{lemma}
\begin{proof}
    We fix constant $\kappa>0$ such that the statements in Lemma~\ref{lem5:density_xy} hold. 
    Then, for the fixed $\kappa$, by Lemma~\ref{lemma:forcetosmallerbox}, we fix $t_2>0$, $c'>0$, and the point $\hat x_{j,k}^\e$ for each pair $(j,k)$ such that for all $(x,y)\in B^{\e,K}_{j,k}\times\mathbb T^m$ and $\e$ small, $\Prob_{(x,y)}(\tau_\kappa<t_2\e)\geq c'$, where $\tau_\kappa=\inf\{t:\tbx_t^\e\in B(\hat x_{j,k}^\e,\kappa\e),~\tbxi_t^\e\in B(\hat y,\kappa)\}$.
   It follows from Lemma~\ref{lem5:density_xy} that there is a domain $C^\e_{j,k}\subset V^\e\times\mathbb T^m$ with $\lambda(C^\e_{j,k})>\kappa\e^2$ and $p_{1}^\e((x,y),\cdot)>\kappa/\e^2$ on $C^\e_{j,k}$ for all $(x,y)\in B(\hat x_{j,k}^\e,\kappa\e)\times B(\hat y,\kappa)$. 
   Then the result follows if we define $c=c'\kappa^2$, $\pi^\e_{j,k}=c'\kappa/\e^2\cdot\chi_{\{C^\e_{j,k}\}}\lambda$, $t_1=t_2+2$, and $\Tilde\eta_3^{j,k}=\tau_{\kappa}\wedge t_2\e+\e<t_1\e$.
\end{proof}
Now let us combine Step 2 and Step 3 together to get the following result concerning the total variation distance of $(\tbx_{\bm\tau_1},\tbxi_{\bm\tau_1})$ with different starting points on $I\times\mathbb T^m$:
\begin{lemma}
\label{lem5:total_variation_on_I}
    For each $(x,y)\in I\times\mathbb T^m$, let $\tilde\mu_{x,y}^\e$ be the measure induced by $(\tbx_{\bm\tau_1},\tbxi_{\bm\tau_1})$ starting at $(x,y)$. Then there exists $c>0$ such that $\mathrm{TV}(\tilde\mu_{x,y}^\e,\tilde\mu_{x',y'}^\e)<1-c$ for any $(x,y),(x',y')\in I\times\mathbb T^m$ and all $\e$ sufficiently small.
\end{lemma}
\begin{proof}
    It suffices to show that there exist $c>0$ and a stopping time $\tilde\eta\leq\bm\tau_1$ such that the total variation distance of $(\tbx_{\tilde\eta},\tbxi_{\tilde\eta})$ with different starting points on $I\times\mathbb T^m$ is no more than $1-c$. 
    Recall the definitions of $s(x)$ and $\tilde\zeta_t^\e$ in Step 2.
    For the process $(\tbx_t^\e,\tbxi_t^\e)$ starting at $(x,y)\in I\times\mathbb T^m$, define
    \[A_{j,k}^{\e}=\{\frac{1}{\e}\tilde\zeta_{1-s(x)}^{\e}\in[j,j+1)\times[k,k+1)\},\]
    \[E_{K}^{\e}=\{|\tbx_{1-s(x)}^{\e}-\bm x_1-\tilde\zeta_{1-s(x)}^{\e}|> K\e\}\cup\{\sup_{0\leq t\leq 1-s(x)}|H(\tbx_{t}^{\e})|>K\sqrt{\e}\}.\]
    Using \eqref{eq:A_jk} and \eqref{eq:A_jk1}, we can find a constant $c'>0$ such that, for all $x\in I$, $y\in \mathbb T^m$, $\e$ sufficiently small, and $-1/\sqrt{\e}\leq j,k\leq1/\sqrt{\e}$, $\Prob_{(x,y)}(A_{j,k}^{\e})\geq c'\e$.
    And using \eqref{eq:slowx} and \eqref{eq:xtildecloseinL1} we can choose $K$ large enough such that, for all $x\in I$, $y\in \mathbb T^m$, and $\e$ sufficiently small, $\Prob_{(x,y)}(E_{K}^{\e})<c'/100$.
    Let $\tilde\eta_2=1-s(x)\wedge\bm\tau_1$.
    Then it is not hard to see that \[\sum_{-1/\sqrt{\e}\leq j,k\leq1/\sqrt{\e}}\Prob_{(x,y)}(A_{j,k}^\e\cap\{\tbx_{\tilde\eta_2}\not\in B^{\e,K}_{j,k}\})<c'/100.\]
    Now let us define, for $(x,y)\in I\times\mathbb T^m$,
    \[
    R_{x,y}^\e=\{(j,k):-1/\sqrt{\e}\leq j,k\leq1/\sqrt{\e},\Prob_{(x,y)}(A_{j,k}^\e\cap\{\tbx^\e_{\Tilde\eta_2}\in B^{\e,K}_{j,k}\})<c'\e/2\}.
    \]
    Then we know that $|R_{x,y}^\e|<\frac{1}{50\e}$ since, for every $(j,k)\in R_{x,y}^\e$, 
    \[\Prob_{(x,y)}(A_{j,k}^{\e}\cap \{\tbx^\e_{\Tilde\eta_2}\not\in B^{\e,K}_{j,k}\})\geq \Prob_{(x,y)}(A_{j,k}^{\e})-\Prob_{(x,y)}(A_{j,k}^\e\cap\{\tbx^\e_{\Tilde\eta_2}\in B^{\e,K}_{j,k}\})\geq c'\e/2.\]
    Let the constants $c''>0$, $t_1>0$, the stopping time $\Tilde\eta_3^{j,k}<t_1\e$, and $\pi^\e_{j,k}$ be defined as in Lemma~\ref{lem:pijk}. 
    %Let $\Tilde\eta_3^{x,y}$ denote the corresponding stopping time with initial condition $(x,y)$. 
    Define
    \[
        \pi^\e=\frac{1}{2}c'\e\sum_{-1/\sqrt{\e}\leq j,k\leq1/\sqrt{\e}}\pi^\e_{j,k},~~~~~~~~~
        \hat\pi_{x,y,x',y'}^\e=\frac{1}{2}c'\e\sum_{(j,k)\in R^\e_{x,y}\cup R^\e_{x',y'}}\pi^\e_{j,k}.
    \]
    In order to define the desired stopping time, we first run the process starting on $I\times\mathbb T^m$ for time $\tilde\eta_2$ (with overwhelming probability, it is the time for the deterministic motion with the same stating point to reach $\bm x_1$).
    Then we use the locations of both $\tilde\zeta_{\tilde\eta_2}^\e$ and $\tx_{\tilde\eta_2}^\e$ to determine whether the process continues and, if it continues, we choose the stopping time based on Lemma~\ref{lem:pijk}.
    Namely, we define 
    \begin{equation}
        \Tilde\eta=\Tilde\eta_2+\sum_{-1/\sqrt{\e}\leq j,k\leq1/\sqrt{\e}}\chi(A_{j,k}^{\e}\cap\{\tbx^\e_{\Tilde\eta_2^{j,k}}\in B_{j,k}^{\e,K}\})\cdot \Tilde\eta_3^{j,k}(\tbx^\e_{\tilde\eta_2},\tbxi^\e_{\tilde\eta_2}),
    \end{equation}where $\Tilde\eta_3^{j,k}(x,y)$ denotes the stopping time with initial condition $(x,y)$.
    Then it follows from previous results that, for any pair $(x,y),(x',y')\in I\times\mathbb T^m$, there is a common component $\pi^\e-\hat\pi^\e_{x,y,x',y'}$ of the distributions of $(\tbx^\e_{\Tilde\eta},\tbxi^\e_{\Tilde\eta})$ starting from $(x,y)$ and $(x',y')$, respectively. Moreover, $(\pi^\e-\hat\pi^\e_{x,y,x',y'})(V^\e\times\mathbb T^m)>c'c''$ since
    $|R^\e_{x,y}|<\frac{1}{50\e}$ and $|R^\e_{x',y'}|<\frac{1}{50\e}$. Therefore, the total variation is no more than $1-c'c''$.
\end{proof}
Finally, we combine the result we just obtained with Step 1 to prove Lemma~\ref{lem5:expo_ergodicity}.
\begin{proof}[Proof of Lemma~\ref{lem5:expo_ergodicity}]
    As we discussed, the result is equivalent to the exponential convergence in total variation of $(\tbx_{\bm\tau_n}^\e,\tbxi_{\bm\tau_n}^\e)$ on $\gamma'\times\mathbb T^m$, uniformly in $\e$ and in the initial distribution. 
    Let $\mu^{\e}_{x,y}$ denote the measure on $\gamma'\times\mathbb T^m$ induced by $(\tbx_{\bm\tau_1}^\e,\tbxi_{\bm\tau_1}^\e)$ with the starting point $(x,y)\in\gamma'\times\mathbb T^m$.
    Then it suffices to prove that there exists $c>0$ such that, for every pair $(x,y)$, $(x',y')\in \gamma'\times\mathbb T^m$ and all $\e$ sufficiently small, $\mathrm{TV}(\mu^{\e}_{x,y},\mu^{\e}_{x',y'})<1-c$, which follows from Lemma~\ref{lem:rotation} and Lemma~\ref{lem5:total_variation_on_I}.
\end{proof}


