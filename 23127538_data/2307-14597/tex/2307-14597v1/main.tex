\documentclass[11pt]{article}

\usepackage{setspace}
\usepackage[english]{babel}
\usepackage[utf8]{inputenc}
\usepackage{amsmath}
\usepackage{scalerel}
\usepackage{amssymb}
\usepackage{parskip}
\usepackage{graphicx}
\usepackage{textcomp}
\usepackage{fancyhdr}
\usepackage{amsthm}
\usepackage{amssymb}
\usepackage{graphicx}
\usepackage{appendix}
\usepackage{indentfirst}
\usepackage{mathtools}
\usepackage{bm}
\usepackage{biblatex}
\usepackage{csquotes}
\usepackage{enumerate}
\usepackage{xcolor}
\usepackage{subcaption} % for side by side graph
%\usepackage{enumitem}% for label assumptions
\usepackage{hyperref}
\addbibresource{references.bib}


% Margins
\usepackage[top=2.5cm, left=3cm, right=3cm, bottom=4.0cm]{geometry}
\allowdisplaybreaks
\setlength{\topmargin}{0mm} \setlength{\oddsidemargin}{0mm}
\setlength{\textwidth}{160mm} \setlength{\textheight}{220mm}
\font\bbc=msbm10 scaled 1200
\numberwithin{equation}{section}
\setlength{\parindent}{2em}

\newcommand{\tz}{{\Tilde{Z}}}
\newcommand{\tbh}{{\Tilde{\bm{H}}}}
\newcommand{\tbq}{{\Tilde{\bm{Q}}}}
\newcommand{\tbx}{{\Tilde{\bm{X}}}}
\newcommand{\tbxi}{{\Tilde{\bm{\xi}}}}
\newcommand{\bx}{{\bm{X}}}
\newcommand{\bxi}{{\bm{\xi}}}
\newcommand{\hx}{{\hat{X}}}
\newcommand{\TX}{{\Tilde{X}}}
\newcommand{\tx}{{\Tilde{X}}}
\newcommand{\TXI}{{\Tilde{\xi}}}
\newcommand{\txi}{{\Tilde{\xi}}}
\newcommand{\Txi}{{\Tilde{\xi}}}
\newcommand{\HX}{{\hat{X}}}
\newcommand{\Hxi}{{\hat{\xi}}}
\newcommand{\nt}{{\Tilde{\eta}}}
\newcommand{\Prob}{\bm{\mathrm{P}}}
\newcommand{\Expe}{\bm{\mathrm{E}}}
\newcommand{\st}[1]{#1^{\e*}}
\newcommand{\T}{\Tilde}
\newcommand{\kp}{{\frac{k}{n}+pT^\e_k}}
\newcommand{\kpp}{{\frac{k}{n}+(p+1)T^\e_k}}
\newcommand{\givenn}{\mathrel{\stretchto{\mid}{4ex}}\lvert\varphi_\kk-H(X_\kk^\e)\rvert<h}
\newcommand{\given}{\mid\mathcal F_{\kk}}
\newcommand{\rcoe}{\nabla H(X_s^\e)^*\sigma(X_s^\e)}
\newcommand{\e}{\varepsilon}
\newcommand{\h}{\hat}
\newcommand{\kk}{\frac{k}{n}}
\newcommand{\rcoef}{\nabla H(\h X_s^\e)^*\sigma(\h X_s^\e)}
\newcommand{\kl}{\frac{k+1}{n}}
\newcommand{\pl}{\varphi^{(l)}}
\newcommand{\tablespace}{\\[1.25mm]}
\newcommand\Tstrut{\rule{0pt}{2.6ex}}     % = `top' strut
\newcommand\tstrut{\rule{0pt}{2.0ex}}     % = `top' strut
\newcommand\Bstrut{\rule[-0.9ex]{0pt}{0pt}}  % = `bottom' strut
\usepackage[active]{srcltx}

\newtheorem{theorem}{Theorem}[section]
\newtheorem{corollary}[theorem]{Corollary}
\newtheorem{conjecture}[theorem]{Conjecture}
\newtheorem{lemma}[theorem]{Lemma}
%\newtheorem{remark}[theorem]{Remark}
\newtheorem{proposition}[theorem]{Proposition}
\newtheorem{definition}[theorem]{Definition}
\newtheorem{example}[theorem]{Example}
\newtheorem{axiom}{Axiom}
\newtheorem{remark}[theorem]{Remark}
\newtheorem{exercise}{Exercise}[section]

\newcommand{\thmref}[1]{Theorem~\ref{#1}}
\newcommand{\propref}[1]{Proposition~\ref{#1}}
\newcommand{\secref}[1]{\S\ref{#1}}
\newcommand{\lemref}[1]{Lemma~\ref{#1}}
\newcommand{\corref}[1]{Corollary~\ref{#1}}
\newcommand{\remref}[1]{Remark~\ref{#1}}
\newcommand{\E}{\bm{\mathrm E}}
\newcommand{\Pro}{\bm{\mathrm P}}
%\newcommand{\R}[1]{\textcolor{red}{#1}}
%\newcommand{\B}[1]{\textcolor{blue}{#1}}
\hypersetup{
    colorlinks=true,
    linkcolor=black,
    citecolor=black,
    pdftitle={Fast oscillating random perturbations of Hamiltonian systems},
    }

%%%%%%%%%%%%%%%%%
%   Title   %
%%%%%%%%%%%%%%%%%
\title{Fast oscillating random perturbations of Hamiltonian systems}
\author{Shuo Yan\\
Department of Mathematics, University of Maryland\\ 4176 Campus Drive - William E. Kirwan Hall\\
College Park, Maryland 20742-4015,
United States\\
shuoyan@umd.edu}
\date{}

\begin{document}
\maketitle
\label{sec:abstract}
With the advent of decentralised digital currencies powered by blockchain technology, a new era of peer-to-peer transactions has commenced. The rapid growth of the cryptocurrency economy has led to increased use of transaction-enabling wallets, making them a focal point for security risks. As the frequency of wallet-related incidents rises, there is a critical need for a systematic approach to measure and evaluate these attacks, drawing lessons from past incidents to enhance wallet security.

In response, we introduce a multi-dimensional design taxonomy for existing and novel wallets with various design decisions. We classify existing industry wallets based on this taxonomy, identify previously occurring vulnerabilities and discuss the security implications of design decisions. We also systematise threats to the wallet mechanism and analyse the adversary's goals, capabilities and required knowledge. We present a multi-layered attack framework and investigate 84 incidents between 2012 and 2024, accounting for \$5.4B. Following this, we classify defence implementations for these attacks on the precautionary and remedial axes. We map the mechanism and design decisions to vulnerabilities, attacks, and possible defence methods to discuss various insights. 

The problem of the presence or absence of phase transition is central in statistical mechanics. To prove the existence of phase transition, the standard idea is to define a notion of contour and use \textit{Peierls' argument} \cite{Peierls.1936}. In the usual Ising model \cite{Ising_25}, particles of the system interact only with their nearest-neighbors. On ferromagnetic long-range Ising models \cite{Anderson_Yuval_69}, there is interaction between each pair of spins in the lattice. The Hamiltonian of the model is given formally by
\begin{equation*}
    H(\sigma) = - \sum_{x,y\in \Z^d}J_{xy}\sigma_x\sigma_y,
\end{equation*}
where $J_{xy}=J|x-y|^{-\alpha}$, $J>0$, $\alpha > d$. It is well-known that the Peierls' argument in dimension 2 implies phase transition for Ising models with nearest-neighbors or long-range interactions when $d\geq 2$, using correlation inequalities. For the unidimensional lattice, it was known that short-range models do not present phase transition. In the long-range case, a different behavior was expected depending on the exponent $\alpha$ (see \cite{Kac_Thompson_69}), but the problem was challenging since contours were first created as multidimensional objects.

In dimension $d=1$, phase transition was proved first in 1969 by Dyson \cite{Dyson.69}, for $\alpha \in (1,2)$, by proving phase transition in an auxiliary model and then using correlation inequalities. In 1982, Fr{\"o}hlich and Spencer \cite{Frohlich.Spencer.82} introduced a notion of one-dimensional contours and then applied the Peierls' argument to show phase transition for the critical value $\alpha = 2$. These contours were inspired by the multiscale techniques previously introduced to study the Berezinskii-Kosterlitz-Thouless transition in two-dimensional continuous spin systems \cite{FS81}. Later, Cassandro, Ferrari, Merola and Presutti  \cite{Cassandro.05} extended the contour argument previously available for $\alpha=2$ to exponents $\alpha\in (3-\frac{\ln 3}{\ln 2}, 2)$, with the additional restriction that the nearest-neighbor interaction is strong, i.e.,  ${J(1)\gg 1}$; this restriction was removed for a subclass of interactions in \cite{Bissacot.Endo.18}. Further results were obtained using contour arguments, such as the decay of correlations, cluster expansions, phase transition with random interactions, etc; some references with these results are \cite{ Cassandro.Merola.Picco.17, Cassandro.Merola.Picco.Rozikov.14, Imbrie.82, Imbrie.Newman.88, Johansson.91}. 

In the multidimensional setting ($d\geq 2$), Ginibre, Grossmann, and Ruelle, in \cite{Ginibre.Grossmann.Ruelle.66}, proved the phase transition for $\alpha > d+1$, using an enhanced version of Peierls' argument and the usual contours. Park proposed a different notion of contour for long-range systems in \cite{Park.88.I, Park.88.II}, extending the Pirogov-Sinai theory available for short-range interactions assuming $\alpha > 3d+1$, although he can also consider Potts models with his methods. Some results in the literature suggest that truly long-range effects appear only when $d < \alpha \leq d+1$, see for instance, \cite{Biskup_Chayes_Kivelson_07}. Recently, Affonso, Bissacot, Endo and Handa \cite{Affonso.2021}, inspired by the ideas from Fr{\"o}hlich and Spencer in \cite{FS81, Frohlich.Spencer.82}, introduced a version of multiscale multidimensional contour and proved phase transition by a contour argument in the whole region $\alpha > d$. They can consider long-range Ising models with deterministic decaying fields, first introduced in the context of nearest-neighbor interactions in \cite{Bissacot_Cioletti_10}. For these models, the lack of analyticity of the free energy does not imply phase transition since these models have the same free energy as the models with zero field. It is expected that fields decaying slowly imply uniqueness. In this setting, a contour argument is useful for proofs of phase transitions as well for uniqueness, some papers with models with deterministic decaying fields are \cite{Aoun_Ott_Velenik_23, Bissacot_Cass_Cio_Pres_15, Bissacot.Endo.18, Cioletti_Vila_2016}.

The Random Field Ising model (RFIM) \cite{Imry.Ma.75} is the nearest-neighbor Ising model with an additional external field acting on each site $(h_x)_{x\in\Z^d}$ that is a family of i.i.d. Gaussian random variable with mean 0 and variance 1. Formally, the Hamiltonian of the model is given by
\begin{equation*}
    H(\sigma) = - \sum_{\substack{x,y\in \Z^d \\|x-y|=1}}J\sigma_x\sigma_y  - \varepsilon\sum_{x\in\Z^d}h_x\sigma_x,
\end{equation*}
where $J>0$, $\varepsilon>0$, $\alpha > d$ and $d \geq 1$. A detailed account of the history of the phase transition problem for this model, as well as detailed proofs, was given in \cite{Bovier.06}. Here we present a brief overview.

During the 1980s, the question of the specific dimension where phase transition for the RFIM should happen attracted much attention and was a topic of heated debate. Two convincing arguments were dividing the physics community. One of them, due to Imry and Ma \cite{Imry.Ma.75}, was a non-rigorous application of the Peierls' argument together with the use of the isoperimetric inequality. The key idea of Peierls' argument is to define a notion of contour and calculate the energy cost of "erasing" each contour, i.e., the energy cost of flipping all spins inside the contour. When there is no external field, that energy necessary to flip the spins in a region $A\subset \Z^d$ is of the order of the boundary $|\partial A|$. When we add an external field, we get an extra cost depending on this field. Imry and Ma argued that this cost should be approximately $\sqrt{|A|}$, which is smaller than $|\partial A|$ for all regions only when $d\geq 3$, so this should be the region where phase transition occurs. The other argument, due to Parisi and Sourlas \cite{Parisi.Sourlas.79}, based on dimensional reduction, predicted that the $d$-dimensional RFIM would behave like the $d-2$-dimensional nearest-neighbor Ising model, therefore presenting phase transition only when $d\geq 4$. 

The question was settled by two celebrated papers showing that Imry and Ma's prediction was correct. First, in 1988, Bricmont and Kupiainen \cite{Bricmont.Kupiainen.88} showed that there is phase transition almost surely in $d\geq3$, for low temperatures and variance $\varepsilon$ small enough. Their proof uses a rigorous renormalization group analysis for the short-range case and it is considered involved. Still, they claimed that the result works for any model with a suitable contour representation and centered sub-gaussian external field. Later on, Aizenman and Wehr \cite{Aizenman.Wehr.90} proved uniqueness for $d\leq 2$. For detailed proofs of these results, we refer the reader to \cite{Bovier.06} (see also \cite{Berretti.85, Camia.18, Frohlich.Imbre.84,  Klein.Masooman.97} for more uniqueness results). 

Recently, Ding and Zhuang, see \cite{Ding2021}, provided a simpler proof of the phase transition, not using RGM. And in  \cite{Ding.Liu.Xia.22}, Ding, Liu and Xia proved that if $\beta_c(d)$ is the critical inverse of the temperature of the Ising model with no field, for all $\beta>\beta_c(d)$ there exists a critical value $\varepsilon_0(d, \beta)$ such that the RFIM with $\varepsilon \leq \varepsilon_0$ presents phase transition. 

In the present paper, we are considering a long-range Ising model with a random field, whose Hamiltonian is given formally by
\begin{equation*}
    H(\sigma) = - \sum_{x,y\in \Z^d}J_{xy}\sigma_x\sigma_y - \varepsilon\sum_{x\in\Z^d}h_x\sigma_x,
\end{equation*}
where $J_{xy}=J|x-y|^{-\alpha}$, $J, \varepsilon>0$, $\alpha > d$ and $h_x\in\mathbb{R}$, $d\geq 3$.
Until now, the only known result in the long-range setting is for the one-dimensional long-range Ising model with a random field, by Cassandro, Orlandi, and Picco \cite{Cassandro.Picco.09}. They used the contours of \cite{Cassandro.05} to show the phase transition for the model when $\alpha\in (3-\frac{\ln 3}{\ln 2}, \frac{3}{2})$, under the assumption $J(1) \gg 1$. We stress that, as remarked by Aizenman, Greenblatt, and Lebowitz \cite{Aizenman_Greenblatt_Lebowitz_2012}, although their argument does not work for the whole region for the exponent $\alpha$, the phase transition holds for values close to the critical value $\alpha=3/2$, since by the Aizenman-Wehr theorem we know that there is uniqueness for $\alpha>3/2$.

The argument from Ding and Zhuang in \cite{Ding2021}, for $d\geq3$, involves controlling the probability of a bad event, which is closely related to controlling the quantity $$\sup_{\substack{0\in A\subset\Z^d \\ A \text{ connected }}}\frac{\sum_{x\in A}h_x}{|\partial A|},$$ known as the greedy animal lattice normalized by the boundary. The greedy animal lattice normalized by the size, instead of the boundary, was extensively studied for general distributions of $(h_x)_{x\in\Z^d}$, see \cite{Cox_Gandolfi_Griffin_Kesten_93, Gandolfi_Kesten_94, Hammond_06, Martin_02}. When we normalize by the boundary, an argument by Fisher, Fr\"{o}hlich and Spencer \cite{FFS84} shows that the expected value of the greedy animal lattice is constant. In dimension $d=2$, the expected value is not finite, see \cite{Ding.Wirth.20}. The supremum is taken over connected regions containing the origin since the interiors of the usual Peierls contours are of this form.


For the long-range model, the interior of contours is not necessarily connected. In fact, long-range contours may have considerably large diameters with respect to their size, so their interiors can be very sparse. To avoid this, we define contours, strongly inspired by the $(M,a,r)$-partition in \cite{Affonso.2021}, using a multiscaled procedure that assures that the contours have no cluster with small density.  With them, we generalize the arguments by Fisher-Fr\"{o}hlich-Spencer \cite{FFS84}, and prove that the expected value of the greedy animal lattice is constant, even considering regions not necessarily connected in the supremum. Then, we prove the phase transition for $d\geq 3$. The main result of this paper is the following.
\begin{theorem*}Given $d\geq 3$, $\alpha>d$, there exists $\beta_c\coloneqq\beta(d, \alpha)$ and $\varepsilon_c\coloneqq\varepsilon(d, \alpha)$ such that, for $\beta >\beta_c$ and $\varepsilon\leq \varepsilon_c$, the extremal Gibbs measures $\mu_{\beta, \varepsilon}^+$ and $\mu_{\beta, \varepsilon}^-$ are distinct, that is, $\mu_{\beta, \varepsilon}^+ \neq \mu_{\beta, \varepsilon}^-$ $\mathbb{P}$-almost surely. Therefore the long-range random field Ising model presents phase transition.
\end{theorem*}

This paper is divided as follows. In Section 2, we define the model and the contours, and suitable generalizations to the constructions in \cite{Affonso.2021} are introduced.  In Section 3, we define two bad events of the external field and prove that they occur with a small probability.  In Section 4, we present the proof of the phase transition.
\section{Main Results}
\label{sec:mainresult}
Throughout this article, $\Prob$ and $\E$ represent the probability and expectation, respectively, and the  subscripts pertain to initial conditions. For brevity, the stopping times' dependence on parameters and initial conditions is not always indicated in the notation when introduced (e.g. \eqref{eq:stopping_times}). $\nabla$ denotes a first order differential operator, i.e., derivative, gradient, Jacobian, etc., depending on the context. 
$\chi_A$ denotes the indicator function of the event $A$.
If $A$ and $B$ are two non-negative functions that depend on an asymptotic parameter, we write $A\lesssim B$ if $A=O(B)$.
$\bm{\mathrm C}_0(\mathbb G)$ is the space of continuous functions on the Reeb graph $\mathbb G$ that tend to zero at infinity with uniform norm.
$h$ is the projection onto $\mathbb G$.
In order to formulate the assumptions and results, we introduce some notation: 
\begin{enumerate}
    \item $O_i$'s are the vertices on the graph and are occasionally used to denote the corresponding critical points on the plane when there is no ambiguity. $I_k$'s are the edges on the graph and $U_k$'s are the corresponding two-dimensional domains. Formally, $O_\infty$ is the vertex that corresponds to infinity. 
    %$I_{k^*}$ is the semi-infinite edge that connects $O_\infty$ and $O_{i^*}$, which  is the other endpoint of $I_{k^*}$, and $U^*$ is the unbounded domain that corresponds to $I_{k^*}$. 
    A symbol $\sim$ between a vertex and an edge means that the vertex is an endpoint of the edge. 
    \item Consider the following metric on $\mathbb G$: $r(h_1,h_2)$ is the length of the shortest path connecting $h_1$ and $h_2$. For example, if $h_1=(1,H_1)$, $I_1\sim O_1$, $O_1\sim I_2$, $I_2\sim O_2$, $O_2\sim I_3$ and $h_2=(3,H_2)$, then $r(h_1,h_2)=|H_1-H(O_1)|+|H(O_1)-H(O_2)|+|H(O_2)-H_2|$. 
    \item $\gamma(h)=\{x:H(x)=h\}$ and $\gamma_k(h)$ is the connected component of $\gamma(h)$ in the domain $U_k$.
    \item $b_h(x,y)=\nabla H(x)\cdot b(x,y)$. 
    \item $\xi_t$ is the diffusion process on $\mathbb T^m$ with the generator $L$, where
    \begin{equation}
        \label{eqb:def_operator_L}
    L f(y)=v(y)\cdot\nabla_y f(y)+\frac{1}{2}\sum_{i,j}(\sigma\sigma^*)_{ij}(y)\frac{\partial^2}{\partial y_i\partial y_j}f(y).
    \end{equation}
    \item For $h$ in the interior of $I_k$, define
    \begin{align*}
            Q_k(h)&=\int_{\gamma_k(h)}\frac{dl}{|\nabla H(x)|},\\
            A_k(h)&=\frac{2}{Q_k(h)}\int_{\gamma_k(h)}\frac{1}{|\nabla H(x)|}\int_0^\infty\E_\mu b_h(x,\xi_s)b_h(x,\xi_0)dsdl,\\
            B_k(h)&=\frac{1}{Q_k(h)}\int_{\gamma_k(h)}\frac{1}{|\nabla H(x)|}\int_0^\infty\E_\mu\nabla_x b_h(x,\xi_s)\cdot(b(x,\xi_0)-\nabla^{\perp}H(x))dsdl,\\
            L_kf(h)&=\frac{1}{2}A_k(h)f''(h)+B_k(h)f'(h).
        \end{align*}
\end{enumerate}

The following conditions are assumed to hold throughout the article.

\textbf{\textit{Assumptions on the coefficients:}}
\begin{enumerate}
    \item[\hypertarget{H1}{\textit{(H1)}}] $v(y)$ and $\sigma(y)$ are $C^\infty$ functions on $\mathbb T^m$. $\sigma(y)$ is $m\times m$ matrix-valued and $\sigma(y)\sigma(y)^{\mathsf T}$ is positive-definite for all $y\in\mathbb T^m$.
    \item[\hypertarget{H2}{\textit{(H2)}}] $H(x)$ is a $C^\infty$ function from $\mathbb R^2$ to $\mathbb R$ with bounded second derivatives. $H(x)$ has a finite number of non-degenerate critical points. Each level curve corresponding to a vertex on the Reeb graph contains at most one critical point. As $|x|\to+\infty$, $H(x)/|x|\to+\infty$. 
    \item[\hypertarget{H3}{\textit{(H3)}}] $b(x,y)$ is a $C^\infty$ function from $\mathbb R^2\times\mathbb T^m$ to $\mathbb R^2$ such that the averaged process is a Hamiltonian system with $H$, i.e. $\bar b(x)=\nabla^\perp H(x)$. 
    \item[\hypertarget{H4}{\textit{(H4)}}] The fast oscillating perturbation is non-degenerate, i.e. $\{b(x,y)-\bar b(x):y\in\mathbb T^m\}$ spans $\mathbb R^2$ for each $x\in\mathbb R^2$, and is uniformly bounded together with its first derivatives.
    \item[\hypertarget{H5}{\textit{(H5)}}] For each $x$ that belongs to one of the separatrices, there exists $y\in\mathbb T^m$ such that the process in \eqref{eq:theprocess1} satisfies the parabolic H\"ormander condition at $(x,y)$. Namely, with $\e^{-1}\tilde v(y)$ being the drift term in the equation for $\bxi^\e_t$ in the Stratonovich form, we have that
    \begin{equation}
        \mathrm{Lie}\left(
        \left\{\begin{pmatrix}
            0\\ \sigma_k(y)
        \end{pmatrix},1\leq k\leq m\right\}\bigcup
        \left\{\left[\begin{pmatrix}
            b(x,y)\\ \tilde v(y)
        \end{pmatrix},
        \begin{pmatrix}
            0\\ \sigma_k(y)
        \end{pmatrix}\right],1\leq k\leq m\right\}\right)
    \end{equation} at $(x,y)$ spans $\mathbb R^{2+m}$, where $\sigma_k$ is the $k-$th column of $\sigma(y)$, $[\cdot,\cdot]$ is the Lie bracket, and Lie$(\cdot)$ is the Lie algebra generated by a set (cf. \cite{Hormander_Hairer} or Section 2.3.2 of \cite{Nualart}). 
%boundedness of all derivatives can be implied by infinite differentiability on bounded set. But uniform boundedness should be assumed separately.
\end{enumerate}

\begin{definition}
\label{def:domain_original}
    The domain $D(\mathcal L)$ consists of functions $f\in \bm{\mathrm{C}}_0(\mathbb G)$ satisfying:
    \begin{enumerate}[(i)]
        \item $f$ is twice continuously differentiable in the interior of each edge $I_k$ of $\mathbb G$;
        \item The limits $\lim_{h_k\to O_i}L_kf(h_k)$ exist and do not depend on the edge $I_k$;
        \item For interior vertex $O_i$, there are constants $p_k:=\pm\lim_{h\to O_i}A_k(h)Q_k(h)$ such that
        \begin{equation}
        \label{eq2:gluing_condition}
            \sum_{I_k\sim O_i}p_k\lim_{h_k\to O_i}f'(h_k)=0,
        \end{equation}
    \end{enumerate}
    where the sign $+$ is taken if $O_i$ is minimum on $I_k$, and the sign $-$ is taken otherwise.
    The operator $\mathcal L$ on the Reeb graph is defined by 
        \begin{equation}
            \label{def:operatorL}
            \mathcal Lf(h)=L_kf(h)
        \end{equation}
    for $f\in D(\mathcal L)$ and $h$ in the interior of $I_k$, and defined as $\lim_{h\to O_i}\mathcal Lf(h)$ at the vertex $O_i$.
\end{definition}
By the Hille-Yosida theorem (see, for example, Theorem 4.2.2 in \cite{markov_process}), one can check that there exists a unique strong Markov process on $\mathbb G$ with continuous sample paths that has $\mathcal L$ as its generator.
Now we are ready to formulate the main result of this article.
% One-dimension so itself without closure is the generator. Continuous follows from pg171. Strong Markov follows from Feller.
\begin{theorem}
\label{thm:mainresult}
    Let the process $(\bx_t^\e,\bxi_t^\e)$ be defined as in \eqref{eq:theprocess1} and the conditions \hyperlink{H1}{\textit{(H1)-(H5)}} hold.
    Then $h(\bx_{t/\e}^\e)$ converges weakly to the strong Markov process on the Reeb graph $\mathbb G$ that has the generator $(\mathcal L,D(\mathcal L))$ and the initial distribution $h(x_0)$.  
\end{theorem}
\begin{remark}
    %The assumption $H(2)$ cannot be relaxed since the place you modify the motion is not where it stays compact. You need the decaying to stay compact. And the compactness is usually much larger than the contraints on coefficients. 
    The last condition in \hyperlink{H2}{\textit{(H2)}} can be relaxed without much extra effort since the limiting process defined by $\mathcal L$ cannot reach infinity in finite time.
    In addition, as seen from the proofs in Section~\ref{sec:exponenitalconvergence} and Remark~\ref{rmk:positive_any_subset_separatrix}, assumption \hyperlink{H5}{\textit{(H5)}} can be relaxed so that it holds for at least one point (different from the saddle point) on each separatrix. Moreover, if the number of Lie brackets needed to generate $\mathbb R^{2+m}$ in the parabolic H\"ormander condition is assumed to be given, then we can relax the assumptions on smoothness of the coefficients.
\end{remark}
%Let $\mathcal D$ be the set of functions that satisfy conditions (i) and (iii) in Definition~\ref{def:domain_original}.
%and have bounded first and second derivatives. <- not really needed because of tightness
To prove the theorem, we need a result on weak convergence of processes, that is Lemma 4.1 in \cite{FreidlinKoralov2021} adapted to our case (see also the original statement in \cite{FreidlinWentzell1994}):
\begin{lemma}
\label{lem:martingale_problem}
Let $\Psi$ be a dense linear subspace of $\bm{\mathrm{C}}_0(\mathbb G)$ and $\mathcal D_{\mathcal L}$ be a linear subspace of $D(\mathcal L)$, and suppose that $\Psi$ and $\mathcal D_{\mathcal L}$ have the following properties:
\begin{enumerate}[(1)]
    \item There is a $\lambda>0$ such that for each $F\in\Phi$ the equation $\lambda f-\mathcal Lf=F$ has a solution $f\in\mathcal D_{\mathcal L}$;
    \item For each $T>0$, each $f\in\mathcal D_{\mathcal L}$, and each compact $K\subset\mathbb G$, 
    \begin{equation}
        \label{eq:mgprob}
        \E_{(x,y)}[f(h(X_{T}^\e))-f(h(x))-\int_0^T\mathcal Lf(h(X_{t}^\e))dt]\to 0,
    \end{equation}
\end{enumerate}
uniformly in $x\in h^{-1}(K)$ and $y\in\mathbb T^m$. 

Suppose that the family of measures on $\bm{\mathrm{C}}([0,\infty], \mathbb G)$ induced by the processes $h(X_t^\e)$, $\e>0$, is tight for each $(x,y)\in\mathbb R^2\times\mathbb T^m$. Then, for each $(x,y)\in\mathbb R^2\times\mathbb T^m$, $h(X_t^\e)$ converges weakly to the strong Markov process on the Reeb graph $\mathbb G$ that has the generator $(\mathcal L,D(\mathcal L))$ and the initial distribution $h(x)$.  
\end{lemma}
Here we choose $\Psi$ to be all the functions in $\bm{\mathrm{C}}_0(\mathbb G)$ that are twice continuously differentiable in the interior of each edge; $\mathcal D_{\mathcal L}$ to be all the functions in $D(\mathcal L)$ that are four times continuously differentiable in the interior of each edge. It is easy to check condition $(1)$ holds in Lemma~\ref{lem:martingale_problem}, and the tightness of distributions of $h(X_t^\e)$ for all $\e>0$ is verified in Appendix~\ref{sec:tightness}. Then the main ingredient of the proof is to verify \eqref{eq:mgprob} in condition (2) of Lemma~\ref{lem:martingale_problem}.
%which states that the martingale problem considered in $\mathcal D$ is solved ``asymptotically'' by $h(\bx_{t/\e}^\e)$.
\iffalse
\begin{lemma}
\label{lem:matingale_original_on_R2}
For all $f\in \mathcal D_{\mathcal L}$ and all $T>0$, \eqref{eq:mgprob} holds uniformly in $x$ in any compact set in $\mathbb R^2$ and $y_0\in\mathbb T^m$.
\end{lemma}
\fi



\label{sec:localization}

Localization is the process by which the robot determines where it is with respect to a global frame of reference.
As opposed to state estimation, this typically occurs at a reduced rate and provides corrections to odometric drift.
A localization system attempts to match its sensor data to a known model of the environment or build the model simultaneously, e.g., Simultaneous Localization and Mapping (SLAM).
Commonly, localization is preformed using 2D laser scanners, 3D lidars, or cameras - though techniques exist using more exotic and/or mixtures of sensors.
Localization's output is a pose and confidence relative to a world model.
This pose solves the REP 105 portion of the \textit{tf} transformation tree from {\tt map} to {\tt odom}.

In ROS~2, the standard localization and SLAM implementations continue to be 2D laser scanner techniques due to their robustness in dynamic and changing environments over long durations.
While strides have been made in visual techniques, the field is still rapidly evolving and even relatively mature methods like ORB-SLAM3 are insufficiently robust for large classes of practical robotics tasks \cite{vslam}.
They are generally sufficient for a number of small-scale or short-duration applications, but official ROS~2 support requires further substantive improvements. 

\subsection{Localization}
\label{sec:secloc}
\label{sec:amcl}

\textbf{AMCL} One popular approach for solving the localization problem is through the use of particle filters \cite{particle_filter}, which enable robust state estimation in dynamic environments and in the presence of noisy sensor data. Particle filters represent non-Gaussian statistical distributions using a discrete set of hypotheses (particles). In the case of mobile robot localization, the distribution represents the probable pose of the robot. 

If an initial pose is not set, a particle filter may initialize the particle cloud by uniformly distributing it throughout the space.
Otherwise, the particles are distributed around the provided start pose of the robot.
At each update step, the particles are projected using a kinematic model of the robot with noise perturbations.
The particles are then weighted based on the likelihood of their sensor observations at that pose.
The filter may then resample the particles, that is, generate a new particle distribution by selecting particles from the existing distribution.

The \textit{amcl} (Adaptive Monte Carlo Localization) package contains a pure and fully-parameterized implementation of the grid particle filter localizer, Fig. \ref{fig:amcl} \cite{prob_rob}.
The package exposes numerous parameters for controlling the full behavior of the filter: minimum and maximum particle counts, resampling periods, statistical weights, motion model noises, and more.
State projection is handled via a motion model that supports differential drive or holonomic bases with configurable characteristics.
Particle weight assignment is performed by taking the current laser scan of the environment in the sensor frame, transforming it based on the pose of each particle, and then attempting to match the scan to a rastered map of the environment.
Particles that have better matches with the map will therefore have higher weights, and will be more likely to be selected in the next resampling step.

Like the odometry model, the laser model is highly configurable.
Two main models are supported: the \textit{beam model} performs ray tracing from each particle to generate a model laser beam, and then compares the ideal beam with the sensed laser measurements \cite{beammodel}.
The beam model models four sources of error due to physical phenomenon with the laser scanner and environment: sensor measurement noise, changes in the environment, failures to measure, and random noise.  
The \textit{likelihood field model} diffuses the map via a Gaussian distribution into a lookup grid such that each cell represents the range from that cell to the nearest occupied cell in the original map \cite{lhfieldmodel}.
Each laser scan point is then projected onto the lookup grid, and the score for the scan is proportional to the sum of the values of the cells into which the projected points fall.
This models sensor noise using the Gaussian distribution, but also explicitly models errors from sensor failures and random measurements in the same way as the beam model. 
This can be more optimal computationally as well as resolve lack of smoothness concerns with the beam model due to many small, dispersed occupancy grid entries in complex settings.
The performance characteristics of the \textit{amcl} package is well researched and can achieve localization accuracies of 5 cm or better in many practical environments \cite{amcl_acc}.

% Figure environment removed

\textbf{GPS Localization} Localizing a robot using GPS sensors is a popular approach for outdoor robot platforms. While subject to drift in some environments, GPS data provides a globally unambiguous positional reference and reduces the need for techniques that rely on expensive sensors. The key to integrating GPS data in a robot's state estimate is the computation of the transform between the robot's world frame (e.g., the {\tt map} frame as specified in REP-105) and the GPS frame, most commonly the WGS-84 geodetic datum \cite{wgs84}.

Generating this transform requires knowledge of the robot's pose in a world frame, the robot's latitude and longitude as provided from a GPS sensor, and the robot's heading in an earth-referenced frame, as might be obtained via an IMU with a magnetometer.

The \textit{robot\_localization} package described in Section \ref{sec:rl} provides the NavSat Transform Node, whose purpose is to calculate this transformation. It does this by converting the GPS position and earth-referenced heading into a position in the Universal Transverse Mercator (UTM) coordinate frame \cite{utm}. From there, the transform to the robot's world frame is trivial. The output of NavSat Transform Node is a pose that can be fused into the robot's state estimate or directly used for localization with respect to the robot's world frame.
The accuracy of GPS localization is limited by the environment, quality of the GPS receiver, and whether RTK (Real-time kinematic) GPS corrections are available in the area.
When RTK is available, accuracies can reach 2 cm or better \cite{rtk}.

\subsection{Mapping}

Mapping solves the fundamental problem of how a robot understands its environment.
It uses sensor data and potentially robot state estimates to generate a globally accurate model of the world.
This world representation is used for many tasks, including global planning and localization, when not using GPS or other global positioning systems.
Succinctly, mapping is the process of learning a world's contents from sensor data.
For mapping techniques in ROS~2, the map is represented as an occupancy grid, or a grid map of cells representing the probability of occupancy.
They also perform mapping and localization simultaneously, e.g. SLAM. 

\textbf{Cartographer} provides real-time simultaneous localization and mapping in both 2D and 3D for multiple platforms and sensor configurations using pose-graph optimization \cite{cartographer}.
Laser scans are added into submaps over short periods of time to create locally accurate maps with current measurements.
Scan matching is performed on these local submaps and the scan is inserted into the submap at its best estimated pose.
When a submap is completed, it takes part in a loop closure process which compares submaps and local scans to refine the global occupancy grid model.

While Cartographer is a good option for some users, it is worth noting that out-of-the-box results with Cartographer are typically poor.
It is possible to achieve 3-5 cm accuracy using Cartographer, but requires extensive professional tuning using hardware platforms with high-quality odometry.
Further, the original author's organization has halted development and support.
ROS~2 support is managed by a community effort but has not received significant attention due to the complexity of the codebase.
Thus, it is difficult to recommend the use of Cartographer. 

% Figure environment removed

% SLAM TOOLBOX
\textbf{SLAM Toolbox} is a set of tools and capabilities for 2D SLAM built upon the OpenKarto SLAM library \cite{slam}.
SLAM Toolbox is also a pose-graph optimization method.
It matches incoming sensor readings with a recent rolling buffer of scans to provide a corrected pose.
This uses a correlation grid constructed from the buffer of local measurements to match the incoming scan at a coarse half-resolution, followed up by full resolution for fine tuning the results.

The incoming scan with the initially corrected pose is then evaluated for loop closure.
Candidate scans are found which are near the incoming measurement but not connected locally in the graph.
The candidates are matched against the scan at a coarse resolution.
If the match is strong enough, a full resolution match is performed on the fine correlation grid.
If again the match is sufficiently strong, a constraint is added; then the pose-graph optimization process is performed. 

Changes were made to the OpenKarto library to increase speed of scan matching using multi-threading and replace the Sparse Bundle Adjustment with Google Ceres to provide more flexible optimization settings. It can reliably map spaces greater than 100,000 sq ft in real-time (fig. \ref{fig:slam}), enables serialization of maps for continued mapping, allows for manual manipulation of the pose-graph, and comes with several operating modes. 


\section{Averaging principle inside one domain}
\label{sec:averaging}
In this section, we consider a general process $(\tx_t^\e,\txi_t^\e)$ defined after Remark~\ref{rmk:existence_of_c}, which is a faster version of the process in \eqref{eq:auxiliary}. 
The process takes values on $M\times\mathbb T^m$.
As a result of localization, $M$ is separated into three domains, each bounded by the separatrix or a part of it.
This section is devoted to the proof of the averaging principle for $(\tx_t^\e,\txi_t^\e)$ on $M\times\mathbb T^m$ up to time when $\tx_t^\e$ exits from one of the three domains. The domain under consideration will be denoted by $U$. 
Therefore, without any ambiguity, the projection $h$ simply reduces to the Hamiltonian $H$.
Let $U(h_1,h_2)$ be the region in $U$ between $\gamma(h_1)$ and $\gamma(h_2)$, $O$ be the saddle point, and $O'$ be the extremum point, and further define stopping times $\tau(h)=\inf\{t:|H(\tx_t^\e)-H(O)|=h\}$ and $\eta(h)=\inf\{t:|H(\tx_t^\e)-H(O')|=h\}$.
Without loss of generality, we assume that $H(O)=0$ and $H(O')=1$.
\subsection{Averaging principle before \texorpdfstring{$\tau(\e^\alpha)\wedge\eta(\delta)$}{}}
\label{sec:Averaging principle before}
We aim to prove the averaging principle between $\gamma(\e^\alpha)$ and $\gamma(1-\delta)$ with constants $0<\alpha<1/4$ and $0<\delta<1$.
Notice that, for technical reasons, we assume that $0<\alpha<1/4$ in this intermediate result and in the proofs that utilize it in this subsection and the next, while we always assume that $0<\alpha<1/2$ elsewhere.
Let us further define another coordinate $\phi$ inside this domain $U$. 
Let $l$ denote the curve that is tangent to $\nabla H$ at each point and connects the saddle point $O$ and the extremum point $O'$, and let $l(h)$ be the intersection of $l$ and $\gamma(h)$.
Let $Q(h)$ denote the time it takes for the averaged process $\bm x_t$ to make one rotation on $\gamma(h)$ and $q(x)$ denote the time it takes for $\bm x_t$ starting from $l(H(x))$ to arrive at $x$. Now we define the coordinate $\phi(x)=q(x)/Q(H(x))$ whose range is $S^1:=\mathbb R\ (\mathrm{mod}\ 1)$. It is easy to see that $\bm x_t$ has constant speed $1/Q(H(\bm{x}_t))$ in $\phi$ coordinate.
Since there is logarithmic delay near the saddle point, the coordinate $\phi$ has exploding derivatives near the separatrix. However, as shown in Appendix~\ref{sec:derivatives}, the order of its derivatives w.r.t. the Euclidean coordinates is under control. 
Let us denote $\Tilde H_t^\e=H(\tx_t^\e)$ and $\Tilde\Phi_t^\e=\phi(\tx_t^\e)$. Along the same lines leading to \eqref{eq:x}, we have the following equations with $u_h=u\cdot\nabla H$, $u_\phi=u\cdot\nabla\phi$, $h_0=H(x_0)$, and $\phi_0=\phi(x_0)$:
\begin{align}
    \Tilde H_t^\e&=h_0+\int_0^t \nabla_y u_h(\tx_s^\e,\txi_s^\e)^{\mathsf T}\sigma(\txi_s^\e)dW_s+\e(u_h(x_0,y_0)-u_h(\tx_t^\e,\txi_t^\e))\nonumber\\
    &\quad+\int_0^t[\nabla_x u_h(\tx_s^\e,\txi_s^\e)\cdot b(\tx_s^\e,\txi_s^\e)+\nabla_y u_h(\tx_s^\e,\txi_s^\e)\cdot c(\tx_s^\e,\txi_s^\e)]ds,\label{eq:H}\\
    \Tilde\Phi_t^\e&=\phi_0+\int_0^t \nabla_y u_\phi(\tx_s^\e,\txi_s^\e)^{\mathsf T}\sigma(\txi_s^\e)dW_s+\frac{1}{\e}\int_0^t\frac{1}{Q(\tilde H_s^\e)}ds\nonumber\\
    &\quad+\int_0^t[\nabla_x u_\phi(\tx_s^\e,\txi_s^\e)\cdot b(\tx_s^\e,\txi_s^\e)+\nabla_y u_\phi(\tx_s^\e,\txi_s^\e)\cdot c(\tx_s^\e,\txi_s^\e)]ds\nonumber\\
    &\quad+\e(u_\phi(x_0,y_0)-u_\phi(\tx_t^\e,\txi_t^\e)),\label{eq:phi}
\end{align}
for $\e^\alpha\leq h_0\leq 1-\delta$ and $t\leq\tau(\e^\alpha)\wedge\eta(\delta)$. {The term multiplied by $1/\e$ in \eqref{eq:H} disappears since $\nabla H\cdot\nabla^\perp H=0$.} Define the following coefficients using the original coordinates for all $x\in M$:
\begin{equation}
\begin{aligned}
    \label{eq:definition_of_operator_AB_x}
    A(x)&=\int_{\mathbb T^m}|\nabla_y u_h(x,y)^{\mathsf T}\sigma(y)|^2d\mu(y),\\
    {B_c}(x)&=\int_{\mathbb T^m}[\nabla_xu_h(x,y)\cdot b(x,y)+\nabla_y u_h(x,y)\cdot c(x,y)]d\mu(y);
\end{aligned}
\end{equation}
and $(h,\phi)$ coordinates for $x=(h,\phi)$, where $\e^\alpha\leq h\leq 1-\delta$ and $\phi\in S^1$:
\begin{equation}
    \begin{aligned}
    \label{eq:definition_of_operator_AB}
    A(h,\phi)&= A(x),~~~~~~\bar A(h)=\int_{S^1} A(h,\phi)d\phi,\\
    {B_c}(h,\phi)&= {B_c}(x),~~~~\bar {B_c}(h)=\int_{S^1} {B_c}(h,\phi)d\phi.
\end{aligned}
\end{equation}
Define ${\mathcal L_c}$ by ${\mathcal L_c}f=\frac{1}{2}\bar Af''+\bar {B_c}f'$ for $f\in\mathcal D$ in the interior of each edge. In particular, when $c(x,y)=0$, this definition is consistent with that in \eqref{def:operatorL}. Introduce two processes close to $\Tilde H_t^\e,\Tilde\Phi_t^\e$:
\begin{align}
    \hat H_t^\e&=h_0+\int_0^t \nabla_y u_h(\tx_s^\e,\txi_s^\e)^{\mathsf T}\sigma(\txi_s^\e)dW_s\nonumber\\
    &\quad+\int_0^t[\nabla_x u_h(\tx_s^\e,\txi_s^\e)\cdot b(\tx_s^\e,\txi_s^\e)+\nabla_y u_h(\tx_s^\e,\txi_s^\e)\cdot c(\tx_s^\e,\txi_s^\e)]ds,\label{eq:h_hat}\\
    \hat \Phi_t^\e&=\phi_0+\int_0^t \nabla_y u_\phi(\tx_s^\e,\txi_s^\e)^{\mathsf T}\sigma(\txi_s^\e)dW_s+\frac{1}{\e}\int_0^t\frac{1}{Q(\tilde H_s^\e)}ds\nonumber\\
    &\quad+\int_0^t[\nabla_x u_\phi(\tx_s^\e,\txi_s^\e)\cdot b(\tx_s^\e,\txi_s^\e)+\nabla_y u_\phi(\tx_s^\e,\txi_s^\e)\cdot c(\tx_s^\e,\txi_s^\e)]ds.\label{eq:phi_hat}
\end{align}
For each $f\in \mathcal D$, $x\in U(\e^\alpha,1-\delta)$, $y\in\mathbb T^m$, and stopping time {$\sigma'\leq T\wedge\eta(\delta)\wedge\tau(\e^\alpha)$}, by Ito's formula applied to $f(\hat H_{\sigma'}^\e)$, we have
\begin{equation}
    \begin{aligned}
        \E_{(x,y)} f(\hat H_{\sigma'}^\e)&=f(H(x))+\E_{(x,y)}\int_0^{\sigma'}\left(\frac{1}{2} |\nabla_y u_h(\tx_s^\e,\txi_s^\e)^{\mathsf T}\sigma(\txi_s^\e)|^2f''(\hat H_s^\e)\right.\\
        &\quad\left.+ \left[\nabla_x u_h(\tx_s^\e,\txi_s^\e)\cdot b(\tx_s^\e,\txi_s^\e)+\nabla_y u_h(\tx_s^\e,\txi_s^\e)\cdot c(\tx_s^\e,\txi_s^\e)\right]f'(\hat H_s^\e)\right)ds.
    \end{aligned}
\end{equation}
Since $\sup_{0\leq t\leq\sigma'}|\Tilde H_t^\e-\hat H_t^\e|=O(\e)$,
\begin{equation}
\label{eq:f}
    \begin{aligned}
        \E_{(x,y)} f(\Tilde H_{\sigma'}^\e)&=f(H(x))+\E_{(x,y)}\int_0^{\sigma'}\left(\frac{1}{2} |\nabla_y u_h(\tx_s^\e,\txi_s^\e)^{\mathsf T}\sigma(\txi_s^\e)|^2f''(\Tilde H_s^\e)\right.\\
        &\quad\left.+ \left[\nabla_x u_h(\tx_s^\e,\txi_s^\e)\cdot b(\tx_s^\e,\txi_s^\e)+\nabla_y u_h(\tx_s^\e,\txi_s^\e)\cdot c(\tx_s^\e,\txi_s^\e)\right]f'(\Tilde H_s^\e)\right)ds+O(\e).
    \end{aligned}
\end{equation}
Combining this with \eqref{eq:definition_of_operator_AB_x} and \eqref{eq:definition_of_operator_AB}, by Lemma~\ref{lem:solution}, as in Corollary~\ref{cor:avg}, we have
\begin{equation}
\label{eq:the_result}
    \E_{(x,y)}\left[f(\Tilde H_{\sigma'}^\e)-f(H(x))-\int_0^{\sigma'}\left(\frac{1}{2}A(\Tilde H_s^\e,\Tilde \Phi_s^\e)f''(\Tilde H_s^\e)+{B_c}(\Tilde H_s^\e,\Tilde \Phi_s^\e)f'(\Tilde H_s^\e)\right)ds\right]=O(\e).
\end{equation}
\begin{lemma}
\label{lem:avg_rotation}
Let $g(h,\phi)$ be either $A(h,\phi)f''(h)$ or ${B_c}(h,\phi)f'(h)$, and $\bar g(h)=\int_{S^1}g(h,\phi)d\phi$. 
Then, for every $T>0$,
\begin{equation}
    \sup_{\substack{x\in U(\e^\alpha, 1-\delta)\\ y\in\mathbb T^m}}\sup_{\sigma'\leq T\wedge\eta(\delta)\wedge\tau(\e^\alpha)}\E_{(x,y)}\left|\int_0^{\sigma'} \left[g(\Tilde H_s^\e,\Tilde \Phi_s^\e)-\bar g(\Tilde H_s^\e)\right]ds\right|\to0,~~\text{as }\e\downarrow0,
\end{equation}where the first supremum is taken over all stopping times {$\sigma'\leq T\wedge\eta(\delta)\wedge\tau(\e^\alpha)$}.
\end{lemma}
\begin{proof}
Fix $\kappa>0$. Since, for fixed $h$, $g(h,\phi)-\bar g(h)$ is a function on $S^1$, we can approximate it by a finite sum of its Fourier series with error less than $\frac{\kappa}{2T}$:
\begin{equation}
    g(h,\phi)-\bar g(h)\approx\sum_{0<|k|\leq K(\e)}g_k(h,\phi):=\sum_{0<|k|\leq K(\e)}G_k(h)\exp(2\pi ik\phi),
\end{equation}for all $\e^\alpha\leq h\leq 1-\delta$ and $\phi\in S^1$,
where
\begin{equation}
    G_k(h)=\int [g(h,\phi)-\bar g(h)]\exp(-2\pi ik\phi)d\phi.
\end{equation}
Since, as shown in Appendix~\ref{sec:derivatives}, $g''_{\phi\phi}=O(|\log h|/h)$, we see that $K(\e)$ can be chosen as $\e^{-\alpha}|\log \e|^2$ for sufficiently small $\e$.
Then it suffices to prove that, for all $0<|k|\leq K(\e)$ and $\e$ sufficiently small,
\begin{equation}
\label{eq:suffice}
    \sup_{\substack{x\in U(\e^\alpha, 1-\delta)\\ y\in\mathbb T^m}}\sup_{\sigma'\leq T\wedge \eta(\delta)\wedge\tau(\e^\alpha)}\E_{(x,y)}\left|\int_0^{\sigma'} g_k(\Tilde H_s^\e,\Tilde \Phi_s^\e)ds\right|=o\left(\frac{\e^\alpha}{|\log \e|^2}\right).
\end{equation}
We define an auxiliary function $v$ for fixed $g_k$, where $0<|k|\leq K(\e)$:
\begin{equation}
    v=\frac{g_k(h,\phi)Q(h)}{2\pi ik},
\end{equation}
which satisfies that $v'_\phi/Q(h)=g_k(h,\phi)$. We formulate the bounds on $\phi$, $v$, $g$, and their derivatives, uniformly in all $\e^\alpha<h<1-\delta$ and $0<|k|\leq K(\e)$ (proved in the Appendix~\ref{sec:derivatives}):
\begin{equation}
\label{eq:bounds}
\begin{aligned}
    \phi&\in[0,1),~\nabla\phi=O(1/h),~\nabla^2\phi=O(1/h^2),\\
    v &= O(|\log h|),~v'_\phi=O(|\log h|),~v''_{\phi\phi}=O(|\log h|^3/h),\\
    v'_h&=O(|\log h|^2/h),~v''_{hh}=O(|\log h|^3/h^3),~v''_{\phi h}=O(|\log h|^2/h),\\
    g'_h&=O(|\log h|/h),~g''_{hh}=O(|\log h|^2/h^3).
\end{aligned}
\end{equation}
By comparing $(\Tilde H_t^\e,\Tilde\Phi_t^\e)$ and $(\hat H_t^\e,\hat \Phi_t^\e)$ in \eqref{eq:H}, \eqref{eq:phi}, \eqref{eq:h_hat}, and \eqref{eq:phi_hat}, and using the bounds in \eqref{eq:bounds}, we know that for all {$\sigma'\leq T\wedge\eta(\delta)\wedge\tau(\e^\alpha)$},
\begin{equation}
\label{eq:gk_close}
    \int_0^{\sigma'}\left|g_k(\Tilde H_s^\e,\Tilde\Phi_s^\e)-\frac{v'_{\phi}(\hat  H_s^\e,\hat \Phi_s^\e)}{Q(\tilde H_s^\e)}\right|ds=\int_0^{\sigma'}\left|\frac{v'_{\phi}(\tilde  H_s^\e,\tilde \Phi_s^\e)-v'_{\phi}(\hat  H_s^\e,\hat \Phi_s^\e)}{Q(\tilde H_s^\e)}\right|ds=O(\e^{1-2\alpha}|\log\e|^3).
\end{equation}
Apply Ito's formula to $v(\hat H_{\sigma'}^\e,\hat \Phi_{\sigma'}^\e)$ and obtain
\begin{align*}
    \frac{1}{\e}\int_0^{\sigma'} \frac{v'_{\phi}(\hat  H_s^\e,\hat \Phi_s^\e)}{Q(\tilde H_s^\e)}ds&=v(\hat H_{\sigma'}^\e,\hat \Phi_{\sigma'}^\e)-v(H(x),\phi(x))-\int_0^{\sigma'}v'_h(\hat H_s^\e,\hat \Phi_s^\e)\nabla_y u_h(\tx_s^\e,\txi_s^\e)^{\mathsf T}\sigma(\txi_s^\e)  dW_s\\
    &\quad-\int_0^{\sigma'} v'_h(\hat H_s^\e,\hat \Phi_s^\e)[\nabla_x u_h(\tx_s^\e,\txi_s^\e)\cdot b(\tx_s^\e,\txi_s^\e)+\nabla_y u_h(\tx_s^\e,\txi_s^\e)\cdot c(\tx_s^\e,\txi_s^\e)]ds\\
    &\quad-\frac{1}{2}\int_0^{\sigma'}v''_{hh}(\hat H_s^\e,\hat \Phi_s^\e)|\nabla_y u_h(\tx_s^\e,\txi_s^\e)^{\mathsf T}\sigma(\txi_s^\e)|^2ds\\
    &\quad-\int_0^{\sigma'}v'_\phi(\hat H_s^\e,\hat \Phi_s^\e)\nabla_y u_\phi(\tx_s^\e,\txi_s^\e)^{\mathsf T}\sigma(\txi_s^\e)dW_s\\
    &\quad-\int_0^{\sigma'} v'_\phi(\hat H_s^\e,\hat \Phi_s^\e)[\nabla_x u_\phi(\tx_s^\e,\txi_s^\e)\cdot b(\tx_s^\e,\txi_s^\e)+\nabla_y u_\phi(\tx_s^\e,\txi_s^\e)\cdot c(\tx_s^\e,\txi_s^\e)]ds\\
    &\quad-\frac{1}{2}\int_0^{\sigma'}v''_{\phi\phi}(\hat H_s^\e,\hat \Phi_s^\e)|\nabla_y u_\phi(\tx_s^\e,\txi_s^\e)^{\mathsf T}\sigma(\txi_s^\e)|^2ds\\
    &\quad-\int_0^{\sigma'}v_{\phi h}''\nabla_y u_h(\tx_s^\e,\txi_s^\e)^{\mathsf T}\sigma(\txi_s^\e)\sigma(\txi_s^\e)^{\mathsf T}u_\phi(\tx_s^\e,\txi_s^\e)ds.
\end{align*}
By using the estimates in \eqref{eq:bounds} and the fact that $0<\alpha<1/4$, we know that the expectation of the {right-hand side} is $o(\frac{\e^{\alpha-1}}{|\log\e|^2})$. Combining this with \eqref{eq:gk_close}, we get \eqref{eq:suffice}. {Thus, the desired result follows.}
\end{proof}
Now, applying Lemma~\ref{lem:avg_rotation} to \eqref{eq:the_result}, we get
\begin{lemma}
\label{lem:bounded}
For each $f\in \mathcal D$, $0<\alpha<1/4$, and $0<\delta<1$, as $\e \downarrow 0$,
\begin{equation}
\label{eq:bounded}
    \sup_{\substack{x\in U(\e^\alpha, 1-\delta)\\ y\in\mathbb T^m}}\sup_{\sigma'\leq T\wedge\eta(\delta)\wedge\tau(\e^\alpha)}|\E_{(x,y)}[f(H(\tx_{\sigma'}^\e))-f(H(x))-\int_0^{\sigma'}{\mathcal L_c}f(H(\tx_s^\e))ds]|\to0,
\end{equation}where the first supremum is taken over all stopping times $\sigma'\leq T\wedge\eta(\delta)\wedge\tau(\e^\alpha)$.
\end{lemma}
\begin{remark}
\label{rmk:accessibility}
The diffusion process governed by $\mathcal L_c$ can reach all points inside the edge and the interior vertex but cannot reach the exterior vertex. For example, in the case considered here, the process can reach all points in $[0,1)$ but cannot reach $1$. The reason is that, for each $\delta>0$, on $[0,1-\delta]$, $\bar B_c(h)$ is bounded and $1/\bar A(h)\lesssim|\log h|$ (see Appendix~\ref{sec:derivatives} for details). However, for $\kappa>0$ sufficiently small, on $[1-\kappa,1]$, $B_c$ is uniformly negative while $A(h)\lesssim 1-h$ due to the non-degeneracy of the maximum point (see Lemma~\ref{lemb:non-zero-drift} for details).
\end{remark}
\begin{lemma}
    \label{lem:time_bounded_wo_t}
    For each $\delta>0$ and $0<\alpha<1/4$, $\E_{(x,y)}(\eta(\delta)\wedge\tau(\e^\alpha))$ in uniformly bounded for all $x\in U(\e^\alpha, 1-\delta)$, $y\in\mathbb T^m$, and $\e$ sufficiently small.
\end{lemma}
\begin{proof}
The solution $f^\delta$ to the following equation exists on $[0,1-\delta]$ due to Remark~\ref{rmk:accessibility}:
    \begin{equation}
    \begin{cases}
    {\mathcal L_c}f^\delta=-1\\
    f^\delta(0)=f^\delta(1-\delta)=0
    \end{cases}
    \end{equation}
Let $\Tilde T>3\|f^\delta\|_{\mathrm{sup}}$, then Lemma~\ref{lem:bounded} implies that, for all $x\in U(\e^\alpha, 1-\delta)$, $y\in\mathbb T^m$, and $\e$ small enough,
\begin{equation}
    \E_{(x,y)}(\eta(\delta)\wedge\tau(\e^\alpha)\wedge \Tilde T)<\Tilde T/2.
\end{equation}
Thus, by Markov inequality and strong Markov property, $\E_{(x,y)}(\eta(\delta)\wedge\tau(\e^\alpha))\leq 2\Tilde T$.
\end{proof}
\begin{lemma}
\label{lem:bounded_wo_t}
For each $f\in \mathcal D$, $\delta>0$, and $0<\alpha<1/4$, as $\e\downarrow0$,
\begin{equation}
\label{eq:bounded_wo_t}
    \sup_{\substack{x\in U(\e^\alpha, 1-\delta)\\ y\in\mathbb T^m}}\sup_{\sigma'\leq \eta(\delta)\wedge\tau(\e^\alpha)}|\E_{(x,y)} [f(H(\tx^\e_{\sigma'}))-f(H(x))-\int_0^{\sigma'}{\mathcal L_c} f(H(\tx_s^\e))ds]|\to0,
\end{equation}where the first supremum is taken over all stopping times $\sigma'\leq \eta(\delta)\wedge\tau(\e^\alpha)$.
\end{lemma}
\begin{proof}
The result can be deduced from \lemref{lem:bounded} and \lemref{lem:time_bounded_wo_t} by choosing a sufficiently large $T$ and using the Markov property.
\end{proof}

%%--------------------------------------------------------------------------------------------------





%%-------------------------------------------------------------------------------------------------
\subsection{Averaging principle before \texorpdfstring{$\sigma$}{sigma}}
With estimates on the transition times and probabilities between level sets near the critical points, the result from last subsection can be extended to the time when the process reaches the separatrix, {which is denoted by  $\sigma$.} The main result of this subsection is
\begin{proposition}
\label{prop:up_to_separatrix}
For each $f\in \mathcal D$, as $\e\downarrow0$,
\begin{equation}
\label{eq:main_result_before_sigma}
    \sup_{\substack{x\in U, y\in\mathbb T^m}}\sup_{\sigma'\leq\sigma}|\E_{(x,y)} [f(H(\tx^\e_{\sigma'}))-f(H(x))-\int_0^{\sigma'}{\mathcal L_{c}} f(H(\tx_s^\e))ds]|\to0,
\end{equation}where the first supremum is taken over all stopping times $\sigma'\leq \sigma$.
\end{proposition}
We state a simple corollary of Proposition~\ref{prop:exit_time_from_separatrix}.
\begin{corollary}
\label{cor:exit_time_eps}
    For each $0<\alpha<1/2$, uniformly in $x\in U(0,\e^\alpha)$ and $y\in\mathbb T^m$,
    \begin{equation}
        \E_{(x,y)}\sigma\wedge\tau(\e^\alpha)=O(\e^{2\alpha}|\log\e|).
    \end{equation}
\end{corollary}
\begin{lemma}
\label{lem:lin_prob}
    For each $0<\alpha<1/2$, uniformly in $x\in U(0,\e^\alpha)$ and $y\in\mathbb T^m$,
    \begin{equation}
        |\Prob_{(x,y)}(\tau(\e^\alpha)<\sigma)-H(x)\e^{-\alpha}|=O(\e^{\alpha}|\log\e|).
    \end{equation}
\end{lemma}
\begin{proof}
    As in \eqref{eq:H}, write the equation for $H(\tx_t^\e)=\Tilde H_t^\e$ stopped at ${\sigma\wedge\tau(\e^\alpha)}$, 
    \begin{align*}
        H(\tx_{\sigma\wedge\tau(\e^\alpha)}^\e)&=H(x)+\int_0^{\sigma\wedge\tau(\e^\alpha)}\nabla_x u_h(\tx_s^\e,\txi_s^\e)\cdot b(\tx_s^\e,\txi_s^\e)+\nabla_y u_h(\tx_s^\e,\txi_s^\e)\cdot c(\tx_s^\e,\txi_s^\e)ds\\
        &\quad+\int_0^{\sigma\wedge\tau(\e^\alpha)}\nabla_y u_h(\tx_s^\e,\txi_s^\e)^{\mathsf T}\sigma(\txi_s^\e)dW_s+O(\e).
    \end{align*}
    From Corollary~\ref{cor:exit_time_eps}, it follows that
    \begin{equation*}
        |\Prob_{(x,y)}(\tau(\e^\alpha)<\sigma)-H(x)\e^{-\alpha}|=\e^{-\alpha}|\E_{(x,y)} H(\tx_{\sigma\wedge\tau(\e^\alpha)}^\e)-H(x)|=O(\e^\alpha|\log\e|).\qedhere
    \end{equation*}
\end{proof}
We prove that the process spends finite time (in expectation) inside $U$. The idea is to use the fact that the process on the graph spends little time near the vertices and exits the edge with positive probability once it gets close enough to the interior vertex.
\begin{lemma}
\label{lem:time1}
    For each $0<\alpha<1/4$, $\E_{(x,y)}\tau(\e^\alpha)$ is uniformly bounded for all $x\in U$ such that $H(x)\geq\e^\alpha$, $y\in\mathbb T^m$, and $\e$ sufficiently small.
\end{lemma}
\begin{proof}
    By Lemma~\ref{lem:near_extremum}, fix $\delta>0$ such that $\E_{(x,y)}\eta(2\delta)\leq1$  for all $\e$ sufficiently small and all $x$ satisfying $H(x)>1-2\delta$ ; By Lemma~\ref{lem:bounded_wo_t}, fix $\kappa>0$ such that $\Prob_{(x,y)}(\eta(\delta)<\tau(\e^\alpha))<1-\kappa$ all $x$ satisfying $H(x)=1-2\delta$, all $y\in\mathbb T^m$, and all $\e$ sufficiently small; By Lemma~\ref{lem:time_bounded_wo_t}, fix $T>4(1+\sup_{x\in U(\e^\alpha, 1-\delta), y\in\mathbb T^m}\E_{(x,y)}(\tau(\e^\alpha)\wedge\eta(\delta)))/\kappa$.
    For all $x$ with $H(x)>1-2\delta$ and $y\in\mathbb T^m$,
    \begin{equation}
    \begin{aligned}
        &\Prob_{(x,y)}(\tau(\e^\alpha)>2T)\\ &\leq \Prob_{(x,y)}(\eta(2\delta)>T)
        +\sup_{(x',y')\in\gamma(1-2\delta)\times\mathbb T^m}\left(\Prob_{(x',y')}(\tau(\e^\alpha)\wedge\eta(\delta)>T)+\Prob_{(x',y')}(\eta(\delta)<\tau(\e^\alpha))\right)\\
        &\leq 1-\kappa/2.
    \end{aligned}
    \end{equation}
    For all $x\in U$ with $\e^\alpha\leq H(x)\leq 1-2\delta$, the estimate above holds without the first term on the second line. Then the uniform boundedness follows from the Markov property.
\end{proof}
We can apply the similar idea near the separatrix. Namely, we choose $0<\alpha'<\alpha<1/4$. By Corollary~\ref{cor:exit_time_eps}, the process spent little time spent between $\gamma$ and $\gamma(\e^{\alpha'})$, and by Lemma~\ref{lem:lin_prob}, the process is very likely to exit through the separatrix rather than come back to $\gamma(\e^{\alpha'})$ once it reaches $\gamma(\e^\alpha)$. Then, since $\E_{(x,y)}\tau(\e^\alpha)$ is uniformly bounded, one can prove the following result:
\begin{lemma}
\label{lem:time2}
    $\E_{(x,y)}\sigma$ is uniformly bounded for all $x\in U$, $y\in\mathbb T^m$, and $\e$ sufficiently small.
\end{lemma}
\begin{proof}[Proof of Proposition~\ref{prop:up_to_separatrix}]
     Fix $\kappa>0$ and $0<\alpha<1/4$. By Lemma~\ref{lem:time1}, let $T$ be large enough such that $\Prob_{(x,y)}(\tau(\e^\alpha)>T)<\kappa$ for all $x\in U$ satisfying $H(x)\geq\e^\alpha$, $y\in\mathbb T^m$, and $\e$ sufficiently small.
    By Lemma~\ref{lem:near_extremum}, let $\delta>0$ small enough such that for $\e$ sufficiently small
     \begin{equation}
     \label{eq:cov_near_extremum}
         \sup_{\substack{x\in U:H(x)\geq 1-\delta\\ y\in\mathbb T^m}}\sup_{\sigma'\leq\eta(\delta)}|\E_{(x,y)}[f(H(\tx_{\sigma'}^\e))-f(H(x))-\int_0^{\sigma'}\mathcal L_c f(H(\tx_s^\e))ds]|<\kappa,
     \end{equation}where the first supremum is taken over all stopping times $\sigma'\leq \eta(\delta)$.
     By Remark~\ref{rmk:accessibility} and Lemma~\ref{lem:bounded_wo_t}, let $\delta'>0$ small enough such that $\Prob_{(x,y)}(\eta(\delta')<\tau(\e^\alpha))<\kappa$ for all $x\in U(\e^\alpha,1-\delta)$, $y\in\mathbb T^m$, and $\e$ sufficiently small. 
     For stopping time $\sigma'\leq\tau(\e^\alpha)$, $x\in U(\e^\alpha,1-\delta)$, and $y\in\mathbb T^m$,
     \begin{equation}
     \label{eq:cov_to_eps}
         \begin{aligned}
             &|\E_{(x,y)} [f(H(\tx^\e_{\sigma'})-f(H(x))-\int_0^{\sigma'}{\mathcal L_c} f(H(\tx_s^\e))ds]|\\
             &\leq|\E_{(x,y)} [f(H(\tx^\e_{\eta(\delta')\wedge\sigma'})-f(H(x))-\int_0^{\eta(\delta')\wedge\sigma'}{\mathcal L_c} f(H(\tx_s^\e))ds]\\\
             &\quad+\Prob_{(x,y)}(\eta(\delta')<\sigma')\sup_{\substack{x'\in U:H(x')=\delta'\\ y'\in\mathbb T^m}}|\E_{(x',y')} [f(H(\tx^\e_{\sigma'}))-f(H(x'))-\int_0^{\sigma'}{\mathcal L_c} f(H(\tx_s^\e))ds]|.
         \end{aligned}
     \end{equation}
     Note that the first term converges to $0$ as $\e\downarrow0$ by Lemma~\ref{lem:bounded_wo_t}, the probability in the second term is less than $\kappa$, and the supremum is uniformly bounded for all $\e$ by Lemma~\ref{lem:time2}. Thus, the expression on the left-hand side of \eqref{eq:cov_to_eps} converges to $0$ uniformly. Combining this with \eqref{eq:cov_near_extremum}, we obtain
     \begin{equation}
     \label{eq:cov_to_eps2}
         \sup_{\substack{x\in U:H(x)\geq\e^\alpha\\ y\in\mathbb T^m}}\sup_{\sigma'\leq\tau(\e^\alpha)}|\E_{(x,y)}[f(H(\tx_{\sigma'}^\e))-f(H(x))-\int_0^{\sigma'}\mathcal L_c f(H(\tx_s^\e))ds]|\to0.
     \end{equation}
     Finally, let us choose $0<\alpha'<\alpha$. Apply Corollary~\ref{cor:exit_time_eps} and Lemma~\ref{lem:lin_prob} to obtain that $\E_{(x,y)}\sigma\wedge\tau(\e^{\alpha'})<\e^{\alpha'}$ and $\Prob_{(x,y)}(\sigma<\tau(\e^{\alpha'}))>1/2$ for all $x\in\gamma(\e^\alpha)$, $y\in\mathbb T^m$, and $\e$ sufficiently small. As in \eqref{eq:cov_to_eps}, by stopping the process at $\tau(\e^\alpha)\wedge\sigma'$ and $\tau(\e^{\alpha'})\wedge\sigma'$ and using the strong Markov property, we can conclude that 
     \begin{equation}
         \sup_{\substack{x\in U:H(x)\geq\e^\alpha\\ y\in\mathbb T^m}}\sup_{\sigma'\leq\sigma}|\E_{(x,y)}[f(H(\tx_{\sigma'}^\e))-f(H(x))-\int_0^{\sigma'}\mathcal L_c f(H(\tx_s^\e))ds]|\to0.
     \end{equation}
     Now \eqref{eq:main_result_before_sigma} follows from this by applying Corollary~\ref{cor:exit_time_eps} again.
\end{proof}


\subsection{Averaging principle starting from \texorpdfstring{$\gamma(\e^\alpha)$}{gamma(epsilon alpha)}}
Fix $0<\alpha<\alpha_1<\alpha_2<1/2$ and $r>0$ small enough. 
{More delicate results are obtained in this subsection to describe the behavior of processes during one excursion from $\gamma(\e^\alpha)$ to $\gamma$ (such excursions, in different domains, happen during the time intervals $[\tau_n,\sigma_n]$ defined in \eqref{eq:stopping_times}). In particular, Lemma~\ref{lem:eps_avg_prin_to_sp} gives bounds on contribution to (\ref{eq:mg_problem_M}) from each such excursion.}
Recall that $Q(h)$ is the rotation time of $\bm x_t$ on $\gamma(h)$.
Our first lemma concerns the typical deviation during one rotation.
\begin{lemma}
\label{lem4:one_rotation}
    For each $\delta>0$ there is $\kappa>0$ such that for all $x\in U(\e^{\alpha_1},r)$, $y\in\mathbb T^m$, and $\e$ sufficiently small,
    \begin{equation}
        \Prob_{(x,y)}\left(\sup_{t\in[0,Q(H(x))]}|H(\tbx_t^\e)-H(\bm x_t)|>\e^{1/2-\delta}\right)<\e^\kappa.
    \end{equation}
    There exists $\delta'>0$ and $\kappa>0$ such that for all $x\in U(\e^{\alpha_1},r)$, $y\in\mathbb T^m$, and $\e$ sufficiently small,
    \begin{equation}
        \Prob_{(x,y)}\left(\sup_{t\in[0,Q(H(x))]}|\tbx_t^\e-\bm x_t|>\e^{\delta'}\right)<\e^\kappa.
    \end{equation}
\end{lemma}
\begin{proof}
    It suffices to prove the result for $\delta<1/2-\alpha_1$. Fix $0<\delta'<\delta''<1/2-\alpha_1-\delta$. Recall the definition of $q$ in Subsection~\ref{sec:Averaging principle before} and consider the coordinates $H$ and $q$ in $U(\e^{\alpha_2},2r)$. As in \eqref{eq:H} and \eqref{eq:phi}, let $q_0=q(x)$, $u_h=u\cdot\nabla H$, $u_q=u\cdot\nabla q$, and $q_t=q_0+t$, and write the equations with $\tau^0=\inf\{t:|\tbh_t^\e-h_0|>\e^{1/2-\delta}~\mathrm{or}~|\tbq_t^\e-q_t|>\e^{\delta''}\}\wedge Q(h_0)$:
    \begin{align}
    H(\tbx_{\tau^0}^\e)&=H(x)+\sqrt{\e}\int_0^{\tau^0} \nabla_y u_h(\tbx_s^\e,\tbxi_s^\e)^{\mathsf T}\sigma(\tbxi_s^\e)dW_s+\e(u_h(x,y)-u_h(\tbx_{\tau^0}^\e,\tbxi_{\tau^0}^\e))\nonumber\\
    &\quad+\e\int_0^{\tau^0}[\nabla_x u_h(\tbx_s^\e,\tbxi_s^\e)\cdot b(\tbx_s^\e,\tbxi_s^\e)+\nabla_y u_h(\tbx_s^\e,\tbxi_s^\e)\cdot c(\tbx_s^\e,\tbxi_s^\e)]ds,\label{eq4:H_tau}\\
    q(\tbx_{\tau^0}^\e)&=q_{\tau^0}+\sqrt{\e}\int_0^{\tau^0} \nabla_y u_q(\tbx_s^\e,\tbxi_s^\e)^{\mathsf T}\sigma(\tbxi_s^\e)dW_s+\e(u_q(x,y)-u_q(\tbx_{\tau^0}^\e,\tbxi_{\tau^0}^\e))\nonumber\\
    &\quad+\e\int_0^{\tau^0}[\nabla_x u_q(\tbx_s^\e,\tbxi_s^\e)\cdot b(\tbx_s^\e,\tbxi_s^\e)+\nabla_y u_q(\tbx_s^\e,\tbxi_s^\e)\cdot c(\tbx_s^\e,\tbxi_s^\e)]ds.\label{eq4:q_tau}
\end{align}
In Appendix~\ref{sec:derivatives}, we prove that $|\nabla q|=O(|\nabla H|/H)$. Thus, it is not hard to see, by looking at the inverse of the Jacobian of $(H,q)$ w.r.t. $x$, that $|H(\tbx_t^\e)-H(\bm x_t)|\leq\e^{1/2-\delta}$ and $|\tbx_t^\e-\bm x_t|\leq\e^{\delta'}$ for all $t\leq\tau^0$. Let $S_H$ and $S_Q$ denote the stochastic integrals in \eqref{eq4:H_tau} and \eqref{eq4:q_tau}. Since ${\tau^0}\lesssim|\log\e|$,
\[\Prob_{(x,y)}(\tau^0< Q(h(x)))<\Prob_{(x,y)}(|S_H|>\e^{1/2-\delta}/2)+\Prob_{(x,y)}(|S_Q|>\e^{\delta''}/2).\] The variance of $S_H$ and $S_Q$ is small:
\begin{align*}
    \bm{\mathrm{Var}}(S_H)&=\e\E(\int_0^{\tau^0} |\nabla_y u_h(\tx_s^\e,\txi_s^\e)^{\mathsf T}\sigma(\txi_s^\e)|^2ds)\lesssim\e\E(\int_0^{\tau^0}|\nabla H(\tx_s^\e)|^2ds)\lesssim\e|\log\e|,\\
    \bm{\mathrm{Var}}(S_Q)&=\e\E(\int_0^{\tau^0} |\nabla_y u_q(\tx_s^\e,\txi_s^\e)^{\mathsf T}\sigma(\txi_s^\e)|^2ds)\lesssim\e\E(\int_0^{\tau^0}|\nabla q(\tx_s^\e)|^2ds)\lesssim\e^{1-2\alpha_1}|\log\e|.
\end{align*}
Hence both results follow from Chebyshev's inequality with $\kappa<\delta$.
\end{proof}
Let $F(h)$ be the solution to
\begin{equation}
\label{eq:ode}
    \begin{cases}
    {\mathcal L_c}F=-1\\
    F(0)=F(2r)=0
    \end{cases}
\end{equation} 
Let $\tau^1$ and $\tau^2$ be the first times for $\tx_t^\e$ to exit $U(\e^{\alpha_1},r)$ and $U(\e^{\alpha_2},2r)$, respectively. Let $x_t^\e=\bm x_{t/\e}$. 
\begin{lemma}
\label{lem4:function_F}
    There exists a function $g(r)$ with $\lim_{r\to0}g(r)=0$ such that $|F'(h)|<g(r)$ for all $0<h<2r$. There exists $C>0$ such that $|F''(h)|<C|\log h|$ and $|F'''(h)|<C/h$.
\end{lemma}
\begin{proof}
The bounds can be verified with the help of estimates for $Q(h)$ in Appendix~\ref{sec:derivatives}.    
\end{proof}
\begin{lemma}
\label{lem:time_exit_alpha1}
There exists a function $g(r)$ with $\lim_{r\to0}g(r)=0$ such that for all $x\in\gamma(\e^\alpha)$, $y\in\mathbb T^m$, and $\e$ sufficiently small,
\begin{equation}
\label{eq4:epsilon_close_time}
    \E_{(x,y)}\tau^1\leq \e^\alpha g(r).
\end{equation}
\end{lemma}
\begin{proof}
For $(\tx_t^\e,\txi_t^\e)$ starting from $(x,y)$, we define $\bar\tau^2=\e Q(H(x))\wedge\tau^2$. As in \eqref{eq:the_result}:
\begin{equation}
\label{eq4:bartau2}
    \E_{(x,y)}[F(H(\tx_{\bar\tau^2}^\e))-F(H(x))-\int_0^{\bar\tau^2}(\frac{1}{2}A(\tx_s^\e)F''(H(\tx_s^\e))+B(\tx_s^\e)F'(H(\tx_s^\e)))ds]=O(\e),
\end{equation}uniformly in $x\in U(\e^{\alpha_2},2r)$ and $y\in\mathbb T^m$.
By the definition of $\bar A(h)$ and $\bar B(h)$, one can see that $\e Q(H(x))\bar A(H(x))=\int_0^{\e Q(H(x))}A(x^\e_s)ds$ and $\e Q(H(x))\bar B(H(x))=\int_0^{\e Q(H(x))}B(x^\e_s)ds$. Since $F$ solves \eqref{eq:ode}, it follows that
\begin{equation}
\label{eq4:epsilonQ}
    \e Q(H(x))=-\e Q(H(x)){\mathcal L_c}F(H(x))=-\int_0^{\e Q(H(x))}\frac{1}{2}A(x^\e_s)F''(H(x^\e_s))+B(x^\e_s)F'(H(x^\e_s))ds.
\end{equation}
We prove that there exists $K>0$ such that 
\begin{equation}
\label{eq:compare_T}
    K\E_{(x,y)}(F(H(\tx_{\bar\tau^2}^\e))-F(H(x)))\leq-\e Q(H(x))
\end{equation}
uniformly in $x\in U(\e^{\alpha_1},r)$, $y\in\mathbb T^m$, and all $\e$ sufficiently small. Then it follows that
\begin{equation}
\label{eq4:sub_solution}
    \E_{(x,y)}\tau^1\leq KF(H(x)),
\end{equation}
for $x\in U(\e^{\alpha_1},r)$, $y\in\mathbb T^m$, and all $\e$ sufficiently small.
Indeed, we can define $\bar\tau^2_k$, $k\geq0$ recursively: $\bar\tau^2_0=0$, $\bar\tau^2_{k+1}=\inf\{t\geq\bar\tau^2_{k}:\tx_t^\e\not\in U(\e^{\alpha_2},2r)\}\wedge(\e Q(H(\tx^\e_{\bar\tau^2_k}))+\bar\tau^2_k)$, and denote the first $k$ such that $\bar\tau^2_k$ exceeds $\tau^1$ as $\bm n$. Then we have
\begin{equation}
    \begin{aligned}
        &\E_{(x,y)} \left[F(H(\tx_{\bar\tau^2_{\bm n}}))-F(H(x))\right]\\
        &=\E_{(x,y)}\sum_{k=0}^\infty\chi_{\bar\tau^2_{k}<\tau^1}\left[F(H(\tx_{\bar\tau^2_{k+1}}))-F(H(\tx_{\bar\tau^2_{k}}))\right]\\
        &\leq\E_{(x,y)}\sum_{k=0}^\infty\chi_{\bar\tau^2_{k}<\tau^1}\sup_{(x',y')\in U(\e^{\alpha_1},r)\times\mathbb T^m}\E_{(x',y')}\left[F(H(\tx_{\bar\tau^2}^\e))-F(H(x'))\right]\\
        &\leq\frac{1}{K}\E_{(x,y)}\sum_{k=0}^\infty\chi_{\bar\tau^2_{k}<\tau^1}(-\e Q(\tx_{\bar\tau^2_{k}})).
    \end{aligned}
\end{equation}
Hence \begin{equation*}
    \E_{(x,y)}\tau^1\leq\E_{(x,y)}\bar\tau^2_{\bm n}=\E_{(x,y)}\sum_{k=0}^\infty\chi_{\bar\tau^2_{k}<\tau^1}(\bar\tau^2_{k+1}-\bar\tau^2_{k})\leq\e\E_{(x,y)}\sum_{k=0}^\infty\chi_{\bar\tau^2_{k}<\tau^1}Q(\tx_{\bar\tau^2_{k}})\leq KF(H(x)).
\end{equation*}Then \eqref{eq4:epsilon_close_time} follows from \eqref{eq4:sub_solution} and Lemma~\ref{lem4:function_F} by taking $x\in\gamma(\e^\alpha)$.

To prove \eqref{eq:compare_T}, it is enough to see that, for $x\in U(\e^{\alpha_1},r)$, $y\in\mathbb T^m$, and $\e$ sufficiently small,
\begin{align*}
    &\e Q(H(x))+\E_{(x,y)}F(H(\tx_{\bar\tau^2}^\e))-F(H(x))\\
    &=-\E_{(x,y)}\int_0^{\e Q(H(x))}\left(\frac{1}{2}A(x^\e_s)F''(H(x^\e_s))+B(x^\e_s)F'(H(x^\e_s))\right)ds\\
    &\quad+\E_{(x,y)}\int_0^{\bar\tau^2}\left(\frac{1}{2}A(\tx_s^\e)F''(H(\tx_s^\e))+B(\tx_s^\e)F'(H(\tx_s^\e))\right)ds+O(\e)\\
    &=\E_{(x,y)}\int_0^{\bar\tau^2}\left(\frac{1}{2}A(\tx_s^\e)F''(H(\tx_s^\e))-\frac{1}{2}A(x^\e_s)F''(H(x^\e_s))\right)ds\\
    &\quad+\E_{(x,y)}\int_0^{\bar\tau^2}\left(B(\tx_s^\e)F'(H(\tx_s^\e))-B(x^\e_s)F'(H(x^\e_s))\right)ds\\
    &\quad-\E_{(x,y)}\int_{\bar\tau^2}^{\e Q(H(x))}\left(\frac{1}{2}A(x^\e_s)F''(H(x^\e_s))+B(x^\e_s)F'(H(x^\e_s))\right)ds+O(\e)\\
    &=o(\e Q(H(x))),
\end{align*}
where the first equality is due to \eqref{eq4:bartau2} and \eqref{eq4:epsilonQ} and the last equality is due to Lemma~\ref{lem4:one_rotation} and Lemma~\ref{lem4:function_F}. 
\end{proof}
Similarly to Lemma~\ref{lem:time2}, we can look at the transitions between $\gamma(\e^\alpha)$ and $\gamma(\e^{\alpha_1})$. By the transition probabilities given in Lemma~\ref{lem:lin_prob} and transition time given in Corollary~\ref{cor:exit_time_eps} and Lemma~\ref{lem:time_exit_alpha1}, one can obtain the following result using the strong Markov property.
\begin{corollary}
\label{cor:exit_time_r}
There exists a function $g(r)$ with $\lim_{r\to0}g(r)=0$ such that for all $x\in\gamma(\e^\alpha)$, $y\in\mathbb T^m$, and $\e$ sufficiently small,
\begin{equation}
    \label{eq:g(r)}
    \E_{(x,y)}\tau(r)\wedge\sigma\leq \e^\alpha g(r).
\end{equation}
\end{corollary}
\begin{lemma}
\label{lem:eps_avg_prin_to_sp}
For each $f\in\mathcal D$, as $\e\downarrow0$,
    \begin{equation}
    \sup_{(x,y)\in\gamma(\e^\alpha)\times\mathbb T^m}\sup_{\sigma'\leq\sigma}|\E_{(x,y)} [f(H(\tx^\e_{\sigma'}))-f(H(x))-\int_0^{\sigma'}{\mathcal L_c} f(H(\tx_s^\e))ds]|=o(\e^\alpha),
\end{equation}where the first supremum is taken over all stopping times $\sigma'\leq \sigma$.
\end{lemma}
\begin{proof}
Fix $\kappa>0$. By Corollary~\ref{cor:exit_time_r}, we can choose $r$ small enough so that for stopping time $\sigma'\leq\sigma$:  $|\E_{(x,y)}[H(\tx_{\tau(r)\wedge\sigma'}^\e)-H(x)]|<\kappa\e^\alpha$ and
$|\E_{(x,y)}[f(H(\tx_{\tau(r)\wedge\sigma'}^\e))-f(H(x))]|<\kappa\e^\alpha$, and
\begin{align*}
    \sup_{(x,y)\in\gamma(\e^\alpha)\times\mathbb T^m}\sup_{\sigma'\leq\sigma}|\E_{(x,y)}\int_0^{\tau(r)\wedge\sigma'}{\mathcal L_c} f(H(\tx_s^\e))ds|<\kappa\e^\alpha,
\end{align*}for all $\e$ sufficiently small, using similar arguments leading to \eqref{eq:H} and \eqref{eq:f}. It follows that, $\Prob_{(x,y)}(H(\tx_{\tau(r)\wedge\sigma'}^\e)=r)\leq H(x)/r+ \kappa\e^\alpha/r\leq2\e^\alpha/r$.
Therefore, uniformly in all $x\in\gamma(\e^\alpha)$, $y\in\mathbb T^m$, and $\sigma'\leq\sigma$,
\begin{align*}
    &|\E_{(x,y)} [f(H(\tx^\e_{\sigma'}))-f(H(x))-\int_0^{\sigma'}{\mathcal L_c} f(\tx_s^\e)ds]|\\
    &\leq|\E_{(x,y)} f[H(\tx^\e_{\tau(r)\wedge\sigma'})-f(H(x))-\int_0^{\tau(r)\wedge\sigma'}{\mathcal L_c} f(H(\tx_s^\e))ds]|\\
    &\quad+\Prob_{(x,y)}(H(X_{\tau(r)\wedge\sigma'}^\e)=r)\sup_{\substack{x'\in\gamma(r)\\ y'\in\mathbb T^m}}|\E_{(x',y')} [f(H(\tx^\e_{\sigma'}))-f(H(x))-\int_0^{\sigma'}{\mathcal L_c} f(H(\tx_s^\e))ds]|\\
    &\leq 3\kappa\e^\alpha,
\end{align*}
for $\e$ sufficiently small, due to Proposition~\ref{prop:up_to_separatrix} and our choice of $r$. The result follows because $\kappa$ can be chosen arbitrarily small.
\end{proof}
{The last result in this subsection provides estimates that will be used later to control the number of excursions from $\gamma(\e^\alpha)$ to $\gamma$ in finite time (see Corollary~\ref{cor:num_excursion}).}
\begin{lemma}
\label{lem:number_excursion}
    There is a constant $\kappa>0$ such that, for all $\e$ sufficiently small,
    \begin{equation}
        \sup_{(x,y)\in\gamma(\e^\alpha)\times\mathbb T^m}\E_{(x,y)} e^{-\sigma}\leq1-\kappa\e^\alpha.
    \end{equation}
\end{lemma}
\begin{proof}
    By Corollary~\ref{cor:exit_time_r}, as in the proof of Lemma~\ref{lem:lin_prob}, we can fix $0<r<1/3$ such that for all $x\in\gamma(\e^\alpha)$, $y\in\mathbb T^m$, and $\e$ sufficiently small, $\Prob_{(x,y)}(\tau(r)<\sigma)\geq\e^\alpha/2r$.
    Let $F$ be defined as in \eqref{eq:ode} and $t=F(r)/3$, then it follows from Proposition~\ref{prop:up_to_separatrix}, as $\e\downarrow0$,
    \begin{equation}
        \sup_{(x,y)\in\gamma(r)\times\mathbb T^m}\E_{(x,y)}[F(H(\tx_{\sigma\wedge\tau(2r)\wedge t}^\e))-F(H(x))-\int_0^{\sigma\wedge\tau(2r)\wedge t}{\mathcal L_c}F(H(\tx_s^\e))ds]\to 0.
    \end{equation}
    Thus, we have that for all $x\in\gamma(r)$, $y\in\mathbb T^m$, and $\e$ sufficiently small,
    \begin{equation}
        \E_{(x,y)}F(H(\tx_{\sigma\wedge\tau(2r)\wedge t}^\e)>F(r)/2,
    \end{equation}
    and it follows that,
    \begin{equation}
        \Prob_{(x,y)}(\sigma>t)\geq\Prob_{(x,y)}(\sigma\wedge\tau(2r)>t)>\frac{\E_{(x,y)}F(H(\tx_{\sigma\wedge\tau(2r)\wedge t}^\e)}{\sup_{[0,2r]}F(h)}>\frac{F(r)}{2\sup_{[0,2r]}F(h)}=:c_1(r).
    \end{equation}
    Then, for all $x\in\gamma(r)$, $y\in\mathbb T^m$, and $\e$ sufficiently small,
    \begin{equation}
        \E_{(x,y)} e^{-\sigma}\leq\Prob_{(x,y)}(\sigma\leq t)+\Prob_{(x,y)}(\sigma>t)e^{-t}\leq 1-\Prob_{(x,y)}(\sigma>t)(1-e^{-t})\leq 1-c(r),
    \end{equation}
    with $c(r)=(1-\exp(-F(r)/3))c_1(r)>0$, and therefore,
    \begin{align*}
        \E_{(x,y)} e^{-\sigma}&\leq \Prob_{(x,y)}(\sigma<\tau(r))+\Prob_{(x,y)}(\sigma>\tau(r))\sup_{{x'\in\gamma(r),y'\in\mathbb T^m}}\E_{(x',y')} e^{-\sigma}\\
        &\leq 1-\Prob_{(x,y)}(\sigma>\tau(r))(1-\sup_{{x'\in\gamma(r),y'\in\mathbb T^m}}\E_{(x',y')} e^{-\sigma})\\
        &\leq 1-\frac{1}{2}c(r)\frac{\e^\alpha}{r}.
    \end{align*}
    The result holds with $\kappa=c(r)/2r$.
\end{proof}


\section{Exponential Convergence on Separatrix}
\label{sec:exponenitalconvergence}
We fix $0<\alpha<1/2$. As in \eqref{eq:stopping_times}, we define inductively two sequences of stopping times $\sigma_n$, $n\geq0$, and $\tau_n$, $n\geq 1$, but now for the general process $(\tx_t^\e,\txi_t^\e)$ on $M\times\mathbb T^m$ with additional drift $c(x,y)$. Without loss of generality, we assume that the saddle point $O$ satisfies $H(O)=0$. Let $V^\e=\{x:|H(x)|<\e^\alpha\}$ and $U_1$, $U_2$, $U_3$ be the three domains separated by $\gamma$. We aim to prove that the distribution of Markov chain $(\tx^\e_{\sigma_n},\txi^\e_{\sigma_n})$ converges in total variation exponentially fast, uniformly in $\e$ and in the initial distribution. Namely, we have the following lemma.
%---------------------------------------------------------------------------------


\begin{lemma}
\label{lem5:expo_ergodicity}
Let $\nu^{n,\e}_{x,y}$ denote the measure on $\gamma\times\mathbb T^m$ induced by $(\tx_{\sigma_n}^\e,\txi_{\sigma_n}^\e)$ with starting point $(x,y)\in\gamma\times\mathbb T^m$. Then there exist a probability measure $\nu^\e$ on $\gamma\times\mathbb T^m$ and constants $\Xi>0$ and $0<c<1$ such that, for all $\e$ sufficiently small,
\begin{equation}
\label{eq:exponentialconvergence}
    \sup_{(x,y)\in\gamma\times\mathbb T^m}{\mathrm{TV}}(\nu^{n,\e}_{x,y},\nu^\e)<\Xi \cdot (1-c)^n,
\end{equation}
where TV is the total variation distance of probability measures. 
\end{lemma}
The rest of this section is devoted to the proof of Lemma~\ref{lem5:expo_ergodicity}. 
Let $\bm\sigma_n$, $n\geq0$, and $\bm\tau_n$, $n\geq1$, be the stopping times w.r.t. $(\tbx_t^\e,\tbxi_t^\e)$ that are analogous to $\sigma_n$, $\tau_n$ w.r.t. $(\tx_t^\e,\txi_t^\e)$.
The lemma is equivalent to the exponential convergence in total variation of $(\tbx_{\bm\tau_n}^\e,\tbxi_{\bm\tau_n}^\e)$ on $\gamma'\times\mathbb T^m$, uniformly in $\e$ and in the initial distribution. 
The proof consists of three steps:
\begin{enumerate}
    \item[\textbf{1.}] The process starting on $\gamma'\times\mathbb T^m$ hits $I\times\mathbb T^m$ before $\bm\tau_1$ with uniformly positive probability, where $I$ is an fixed interval on the separatrix.
    \item[\textbf{2.}] Let the process starting on $I\times\mathbb T^m$ evolve for a certain period of time. Then, by a local limit theorem, we can estimate from below the probabilities of hitting $O(\e)$-sized boxes in a certain $O(\sqrt{\e})$-sized region, uniformly w.r.t. the starting point on $I\times\mathbb T^m$.
    \item[\textbf{3.}] By the H\"ormander condition \hyperlink{H5}{\textit{(H5)}}, we prove a common lower bound for the density of the distribution of the process starting from each of the $O(\e)$-sized boxes after a short time.
    %Moreover, the distribution with the common lower bound as its density has a uniformly positive probability.
\end{enumerate}


Let us take care of these steps in order. 


\textbf{Step 1}. Let $0<\beta<1$, which will be specified later. We prove that the process has a uniformly positive probability of following along the averaged motion and going through a neighborhood of the saddle point without making a deviation more than $\beta\sqrt{\e}$ in terms of $H$.
\begin{lemma}
\label{lem:stay_close_to_averaged}
For each fixed $\hat t>0$, $\beta'>0$,
\[\Prob_{(x,y)}\left(\sup_{0\leq t\leq\hat t}|\tbx_t^\e-\bm x_t|\leq\beta'\sqrt{\e}\right),\]
is uniformly positive for all $(x,y)\in M\times\mathbb T^m$ and $\e$ sufficiently small.
\end{lemma}
\begin{proof}
     Let the eigenvalues of $\nabla^2 H$ be bounded by $K$.
    Recall formula \eqref{eq:slowx}. By the boundedness of the coefficients, the event
    \[E:=\left\{\sup_{0\leq t\leq\Tilde t}|\int_0^t \nabla_y u(\tbx_s^\e,\tbxi_s^\e)\sigma(\tbxi_s^\e)dW_s|\leq\frac{1}{2}\beta'e^{-K\Tilde t}\right\}\]
    has positive probability, uniformly in the starting points.
    By \eqref{eq:slowx}, we have that on the event $E$, for $t\leq\Tilde t$ and $\e$ sufficiently small,
    \begin{align*}
        |\tbx_t^\e-\bm x_t|&\leq |\int_0^t(\nabla^\perp H(\tbx_s^\e)-\nabla^\perp H(\bm x_s)ds|+\sqrt{\e}|\int_0^t \nabla_y u(\tbx_s^\e,\tbxi_s^\e)\sigma(\tbxi_s^\e)dW_s|\\
        +\e|\int_0^t &[\nabla_x u(\tbx_s^\e,\tbxi_s^\e)b(\tbx_s^\e,\tbxi_s^\e)+\nabla_y u(\tbx_s^\e,\tbxi_s^\e)c(\tbx_s^\e,\tbxi_s^\e)]ds|+\e|u(x,y)-u(\tbx_t^\e,\tbxi_t^\e)|\\
        &\leq K\int_0^t|\tbx_s^\e-\bm x_s|ds+\beta'e^{-K\Tilde t}\sqrt{\e}.
    \end{align*}
    Then Gr\"onwall's inequality implies that $|\tbx_t^\e-\bm x_t|\leq\beta'\sqrt{\e}$ for all $t\leq\Tilde{t}$. Therefore, $E$ implies $\{\sup_{0\leq t\leq\hat t}|\tbx_t^\e-\bm x_t|\leq\beta'\sqrt{\e}\}$, and the uniform positivity follows.
\end{proof}
\begin{lemma}
\label{lem:throughsaddlepoint}
     For any given $0<c<1$, there exist curves $\Gamma_1$ and $\Gamma_2$ in $U_1$ such that 
    \begin{enumerate}[(i)]
        \item $\Gamma_1$ and $\Gamma_2$ have their tangent vectors as $\nabla H$. They intersect with the separatrix on different sides of the saddle point and the averaged motion on the separatrix spends finite time from $\Gamma_2$ to $\Gamma_1$. 
        \item Let $x\in\Gamma_1$ satisfy $2\beta\sqrt{\e}\leq |H(x)|\leq 2\sqrt{\e}$ and $\tau_x=\inf\{t:\tbx_t^\e\in\Gamma_2\}$. Then for all $y\in\mathbb T^m$, $\Prob_{(x,y)}(\sup_{0\leq t\leq \tau_x}|H(\tbx^\e_t)-H(x)|\leq\beta\sqrt{\e})>c$ for all $\e$ sufficiently small.
    \end{enumerate}
\end{lemma}
% Figure environment removed
\begin{proof}
     Suppose $H(x)>0$ for all $x\in U_1$. By the Morse lemma, there exist neighborhoods $U$ and $V$ of the saddle point $O$ and the origin, respectively, and a diffeomorphism $\psi$ from $U$ to $V$ such that $H(x)=G(\psi(x))$, where $G(z)=z_1z_2$. Then consider a random change of time by dividing the generator by $D(x):=\mathrm{det}(\nabla_x\psi(x))$: 
    \begin{equation}
    \label{eq:randomchangeoftime}
    \begin{aligned}
        d\st{\tbx_t}&=\frac{b(\st{\tbx_t},\st{\tbxi_t})}{D(\st{\tbx_t})}dt,\\
        d\st{\tbxi_t}&=\frac{1}{\e}\frac{v(\st{\tbxi_t})}{D(\st{\tbx_t})}dt+\frac{1}{\sqrt\e}\frac{\sigma(\st{\tbxi_t})}{\sqrt{D(\st{\tbx_t})}}dW_t+\frac{c(\st{\tbx_t},\st{\tbxi_t})}{D(\st{\tbx_t})}dt.
    \end{aligned}
    \end{equation}
    Write the equation for $\st{\tz_t}:=\psi(\st{\tbx_t})$:
    \[
        d\st{\tz_t}=\frac{1}{D(\psi^{-1}(\st{\tz_t}))}\cdot \nabla_x\psi(\psi^{-1}(\st{\tz_t}))b(\psi^{-1}(\st{\tz_t}),\st{\tbxi_t})dt=:b^*(\st{\tz_t},\st{\tbxi_t})dt.
    \]
    It is not hard to verify that $b^*(z,y)$ satisfies
    $$\int_{\mathbb T^m}b^*(z,y)d\mu(y)=\nabla^\perp G(z).$$
    Hence, by Lemma~\ref{lem:solution}, there exists a bounded solution $u^*(z,y)$ to 
    \[Lu^*(z,y)=-(b^*(z,y)-\nabla^\perp G(z))\cdot D(\psi^{-1}(z)).\]
    Consider a local coordinate $G=z_1z_2$ and $\phi^*=\frac{1}{2}\log({z_2}/{z_1})$ in $V$.
    The averaged motion has constant speed: $0$ in $G$ and $1$ in $\phi^*$. 
    As in \eqref{eq:H} and \eqref{eq:phi}, we have the equations for $\st{\Tilde G_t}=G(\st{\tz_t})$, $\st{\Tilde \Phi_t}=\phi^*(\st{\tz_t})$, by applying Ito's formula to $u^*_g=u^*\cdot\nabla G$ and $u^*_{\phi}=u^*\cdot\nabla \phi^*$, with 
    $z=\psi(x)$, $g_0=G(z)$, and $\phi^*_0=\phi^*(z)$:
    \begin{align}
        \st{\Tilde G_t}=~&g_0+\sqrt{\e}\int_0^t \nabla_y u^*_g(\st{\tz_s},\st{\tbxi_s})^{\mathsf T}\frac{\sigma(\st{\tbxi_s})}{\sqrt{D(\psi^{-1}(\st{\tz_s}))}}dW_s-\e(u^*_g(\st{\tz_t},\st{\tbxi_t})-u^*_g(z,y))\nonumber\\
        &+\e\int_0^t\left[\nabla_z u^*_g(\st{\tz_s},\st{\tbxi_s})\cdot b^*(\st{\tz_s},\st{\tbxi_s})+\nabla_y u^*_g(\st{\tz_s},\st{\tbxi_s})\cdot \frac{c(\psi^{-1}(\st{\tz_s}),\st{\tbxi_s})}{D(\psi^{-1}(\st{\tz_s}))}\right]ds,\label{eq:G}\\
        \st{\Tilde \Phi_t}=~&\phi^*_0+t+\sqrt{\e}\int_0^t \nabla_y u^*_\phi(\st{\tz_s},\st{\tbxi_s})^{\mathsf T}\frac{\sigma(\st{\tbxi_s})}{\sqrt{D(\psi^{-1}(\st{\tz_s}))}}dW_s-\e(u^*_\phi(\st{\tz_t},\st{\tbxi_t})-u^*_\phi(z,y))\nonumber\\
        &+\e\int_0^t\left[\nabla_z u^*_\phi(\st{\tz_s},\st{\tbxi_s})\cdot b^*(\st{\tz_s},\st{\tbxi_s})+\nabla_y u^*_\phi(\st{\tz_s},\st{\tbxi_s})\cdot \frac{c(\psi^{-1}(\st{\tz_s}),\st{\tbxi_s})}{D(\psi^{-1}(\st{\tz_s}))}\right]ds.\label{eq:Phi*}
    \end{align}
    
    To get the lower bound for the desired probability, we will choose the curves $\Gamma_1$ and $\Gamma_2$ that are close enough to the saddle point. 
    The time it takes to get from $\Gamma_1$ to $\Gamma_2$ is still of order $|\log\e|$ since they are chosen independently of $\e$.
    In this way, the process starting on $\Gamma_1$ and stopped on $\Gamma_2$ will be shown to have small variance, hence it is unlikely for the process to have deviations larger than what we wish. 
    With $C>0$ to be specified later, let $l_1=\{z:\phi^*(z)=\frac{1}{4}\log\e+\frac{1}{2}\log\beta+C\}$, $l_2=\{z:\phi^*(z)=-(\frac{1}{4}\log\e+\frac{1}{2}\log\beta+C)\}$, and $l_3=\{z:\phi^*(z)=-(\frac{1}{4}\log\e+\frac{1}{2}\log\beta+C)-2\}$.
    The idea is to look at event that the process stays close to the averaged motion before the latter reaches $l_2$, which implies that the process does not make a large deviation in $G$, or equivalently, in $H$, before reaching $l_3$.
    Let $\Gamma_1^*$, $\Gamma_2^*$ be the curves that have tangent vectors as $\nabla_x\psi\circ\psi^{-1}({\nabla_x\psi\circ\psi^{-1}})^{\mathsf T}\nabla G$ and go through the points $(e^{-C},e^C\beta\sqrt{\e})$, $(e^{C+1}\beta\sqrt{\e},e^{-C-1})$, respectively.
    Since $\psi$ is a diffeomorphism, it is easy to see that each $z$ on $\Gamma_1^*$ or $\Gamma_2^*$ with $G(z)\geq\beta\sqrt{\e}$ satisfies that $\frac{1}{4}\log\e+\frac{1}{2}\log\beta+C\leq\phi^*(z)\leq-(\frac{1}{4}\log\e+\frac{1}{2}\log\beta+C)-2$.
    Let $\Gamma_1$ and $\Gamma_2$ be the pre-images of $\Gamma_1^*$ and $\Gamma_2^*$ in $U_1$. They have $\nabla H$ as tangent vectors due to the specific way we construct $\Gamma_1^*$ and $\Gamma_2^*$.
    Consider the process in \eqref{eq:randomchangeoftime} starting at $x\in\Gamma_1$ satisfying that $2\beta\sqrt{\e}\leq H(x)\leq 2\sqrt{\e}$ with an arbitrary $y\in\mathbb T^m$. 
    Let $\phi_t^*=\phi^*_0+t$.
    Define $t_x=\inf\{t:\phi_t^*=\phi^*(l_2)\}$ and $\tau_x^*=\inf\{t:|\st{\Tilde G_t}-g_0|=\beta\sqrt{\e}\}\wedge \inf\{t:|\st{\Tilde\Phi_t}-\phi_t^*|=1\}\wedge t_x$. 
    Then it is clear that $\Prob_{(x,y)}(\sup_{0\leq t\leq \tau^*_x}|H(\tbx^\e_t)-H(x)|\leq\beta\sqrt{\e})\geq\Prob_{(x,y)}(\tau_x^*=t_x)$. 
    Let $S_G$ and $S_\phi$ denote the stochastic integrals in \eqref{eq:G} and \eqref{eq:Phi*}, respectively, with $t$ replaced by $\tau_x^*$. 
    Since $\tau^*_x\lesssim|\log\e|$, $\nabla G$ is bounded, and $\nabla \phi^*\lesssim\e^{-1/2}$ before $\tau^*_x$, we see that the unwanted deviations happen only if $S_G$ and $S_\phi$ are large.
    Namely,
    \[\Prob_{(x,y)}(\tau_x^*<t_x)\leq\Prob_{(x,y)}(|S_G|\geq\beta\sqrt{\e}/2)+\Prob_{(x,y)}(|S_\phi|\geq1/2).\]
    Both terms on the right hand side can be controlled by Chebyshev's inequality. 
    Note that there exists a constant $K>0$ independent of $\e$ such that
    \begin{align*}
        \bm{\mathrm{Var}}(S_G)&\leq\e K\E\int_0^{\tau_x^*}|\nabla G(\st{\tz_s})|^2ds\\
        &=\e K\E\int_0^{\tau_x^*} \st{\Tilde G_s}(e^{2\st{\Tilde \Phi_s}}+e^{-2\st{\Tilde \Phi_s}})ds\\
        &\leq \e K\int_0^{\tau_x^*}(2+\beta)\sqrt{\e}e^2(e^{2{\phi_s^*}}+e^{-2{\phi_s^*}})ds\\
        &\leq  3K\sqrt{\e^3}e^2\int_0^{-2(\frac{1}{4}\log\e+\frac{1}{2}\log\beta+C)}(e^{2{\phi_s^*}}+e^{-2{\phi_s^*}})ds\\
        &=\ 3K\sqrt{\e^3}e^2\int_{\frac{1}{4}\log\e+\frac{1}{2}\log\beta+C}^{-(\frac{1}{4}\log\e+\frac{1}{2}\log\beta+C)}(e^{2\varphi}+e^{-2\varphi})d\varphi\\
        &\leq \ \frac{3}{\beta}Ke^{2-2C}\e,
    \end{align*}and, similarly,
    \[
        \bm{\mathrm{Var}}(S_\phi)\leq \e K\E\int_0^{\tau^*_x}|\nabla \Phi(\st{\tz_s})|^2ds
        \leq\e K\E\int_0^{\tau^*_x} \frac{1}{\st{\Tilde G}_s}(e^{2\st{\Tilde \Phi_s}}+e^{-2\st{\Tilde\Phi_s}})ds\leq  \frac{1}{\beta^2}Ke^{2-2C}.
    \]
    Then $C$ can be chosen large enough such that both variances are small enough, and hence $\Prob_{(x,y)}(|H(\tbx^\e_{\tau^*_x})-H(x)|\leq\beta\sqrt{\e})>c$. 
\end{proof}
% Figure environment removed
We can choose the corresponding curves in the other regions. As a result, we have four curves corresponding to four different directions, all with positive distance to the saddle point, as shown in Figure~\ref{fig:four curves}. Moreover, the corresponding transition probabilities near the saddle point have lower bounds analogous to that given in Lemma~\ref{lem:throughsaddlepoint} (ii). For the rotations happening away from those curves, we will prove that, before the time when the process comes back to the curves, the deviation of $H$ can be large enough to cross the separatrix with positive probability. Let $\Gamma_i(h_1,h_2)$ be the set $\{x\in\Gamma_i:h_1\leq H(x)\leq h_2\}$.
\begin{lemma}
    \label{lem:hit_separatrix}
    For each fixed $\hat t>0$,
    \[\Prob_{(x,y)}\left(\inf_{0\leq t\leq\hat t}H(\tbx_{t}^\e)\leq -\sqrt{\e},\sup_{0\leq t\leq\hat t}|\tbx_t^\e-\bm x_t|\leq\e^{\frac{1+2\alpha}{4}}\right)\]
    is positive uniformly in {$x\in\Gamma_2(0,2\sqrt{\e})$}, $y\in\mathbb T^m$, and all $\e$ sufficiently small.
\end{lemma}
\begin{proof}
    By Lemma~\ref{lem:stay_close_to_averaged} and the Markov property, it is enough to consider small $\hat t$ such that $\bm x_t$ does not reach $\Gamma_1$ before $\hat t$. Using formula \eqref{eq:slowx} again, we see that $\Prob_{(x,y)}(\sup_{0\leq t\leq\hat t}|\tbx_t^\e-\bm x_t|>\e^{\frac{1+2\alpha}{4}})\to 0$ as $\e\downarrow0$ uniformly in $(x,y)$. Use formula \eqref{eq:H} on a shorter time scale:
    \begin{align*}
        H(\tbx_{t}^\e)&=H(x)+\sqrt{\e}\int_0^{t} \nabla_y u_h(\tbx_s^\e,\tbxi_s^\e)^{\mathsf T}\sigma(\tbxi_s^\e)dW_s+\e(u_h(x,y)-u_h(\tbx_{t}^\e,\tbxi_{t}^\e))\\\
    &\quad+\e\int_0^{t}[\nabla_x u_h(\tbx_s^\e,\tbxi_s^\e)\cdot b(\tbx_s^\e,\tbxi_s^\e)+\nabla_y u_h(\tbx_s^\e,\tbxi_s^\e)\cdot c(\tbx_s^\e,\tbxi_s^\e)]ds.
    \end{align*}
    So it suffices to show the uniform positivity of 
    \[\Prob_{(x,y)}\left(\inf_{0\leq t\leq\hat t}\int_0^{t} \nabla_y u_h(\tbx_s^\e,\tbxi_s^\e)^{\mathsf T}\sigma(\tbxi_s^\e)dW_s\leq -4,\sup_{0\leq t\leq\hat t}|\tbx_t^\e-\bm x_t|\leq\e^{\frac{1+2\alpha}{4}}\right).\]
    Note that there exists another Brownian motion $\Tilde W$ such that
    \begin{equation}
    \label{eq5:time_changed_brownian_motion}
    \int_0^{t} \nabla_y u_h(\tbx_s^\e,\tbxi_s^\e)^{\mathsf T}\sigma(\tbxi_s^\e)dW_s=\Tilde W\left(\int_0^{t} |\nabla_y u_h(\tbx_s^\e,\tbxi_s^\e)^{\mathsf T}\sigma(\tbxi_s^\e)|^2ds\right).
    \end{equation}
    Recall that in Section~\ref{sec:Averaging principle before} we defined $A(x)=\int_{\mathbb T^m}|\nabla_y u_h(x,y)\sigma(y)|^2d\mu(y)$. Then, by Corollary~\ref{cor:avg},
    \begin{equation}
    \label{eq:close_in_L1}
        \E_{(x,y)}\left|\int_0^{\hat t} |\nabla_y u_h(\tbx_s^\e,\tbxi_s^\e)^{\mathsf T}\sigma(\tbxi_s^\e))|^2ds-\int_0^{\hat t} A(\tbx_s^\e)ds\right|=O(\sqrt\e).
    \end{equation}
    Note that on the event $\{\sup_{0\leq t\leq\hat t}|\tbx_t^\e-\bm x_t|\leq\e^{\frac{1+2\alpha}{4}}\}$, $A(\tbx_t^\e)$ is uniformly positive for $0\leq t\leq\hat t$. Let us denote this lower bound as $m$, which is independent of $x$, $y$, and $\e$. Then
    \begin{equation}
        \Prob_{(x,y)}(\int_0^{\hat t} A(\tbx_s^\e)ds>m\hat t,\sup_{0\leq t\leq\hat t}|\tbx_t^\e-\bm x_t|\leq\e^{\frac{1+2\alpha}{4}})\to1.
    \end{equation}
    By the $L^1$ convergence in \eqref{eq:close_in_L1}, we obtain
    \begin{equation}
    \label{eq5:positive_variance_one_rotation}
        \Prob_{(x,y)}\left(\int_0^{\hat t} |\nabla_y u_h(\tbx_s^\e,\tbxi_s^\e)^{\mathsf T}\sigma(\tbxi_s^\e))|^2ds>m\hat t/2,\sup_{0\leq t\leq\hat t}|\tbx_t^\e-\bm x_t|\leq\e^{\frac{1+2\alpha}{4}}\right)\to1.
    \end{equation}
    Suppose $0<c<\Prob(\inf_{0\leq t\leq m\hat t/2}\Tilde W_t<-4)$.
    Then, for all $\e$ sufficiently small,
    \begin{align*}
        &\Prob_{(x,y)}\left(\inf_{0\leq t\leq\hat t}\int_0^{t} \nabla_y u_h(\tbx_s^\e,\tbxi_s^\e)^{\mathsf T}\sigma(\tbxi_s^\e)dW_s\leq -4,\sup_{0\leq t\leq\hat t}|\tbx_t^\e-\bm x_t|\leq\e^{\frac{1+2\alpha}{4}}\right)\\
        &=\Prob_{(x,y)}\left(\inf_{0\leq t\leq\hat t}\Tilde W\left(\int_0^{t} |\nabla_y u_h(\tbx_s^\e,\tbxi_s^\e)^{\mathsf T}\sigma(\tbxi_s^\e))|^2ds\right)\leq -4,\sup_{0\leq t\leq\hat t}|\tbx_t^\e-\bm x_t|\leq\e^{\frac{1+2\alpha}{4}}\right)\\
        &\geq\Prob_{(x,y)}\left(\inf_{0\leq t\leq m\hat t/2}\Tilde W_t\leq -4,\int_0^{\hat t} |\nabla_y u_h(\tbx_s^\e,\tbxi_s^\e)^{\mathsf T}\sigma(\tbxi_s^\e))|^2ds>m\hat t/2,\sup_{0\leq t\leq\hat t}|\tbx_t^\e-\bm x_t|\leq\e^{\frac{1+2\alpha}{4}}\right)\\
        &\geq c/2.
    \end{align*}
\end{proof}
\begin{remark}
\label{rmk:hit_separatrix}
    The result in Lemma~\ref{lem:hit_separatrix} also holds for $x\in\Gamma_4(0,2\sqrt{\e})$. Similarly, for each fixed $\hat t>0$,
    \[\Prob_{(x,y)}\left(\sup_{0\leq t\leq\hat t}H(\tbx_{t}^\e)\geq \sqrt{\e},\sup_{0\leq t\leq\hat t}|\tbx_t^\e-\bm x_t|\leq\e^{\frac{1+2\alpha}{4}}\right)\]
    is positive uniformly in $x\in\Gamma_2(-2\sqrt{\e},0)\cup\Gamma_4(-2\sqrt{\e},0)$, $y\in\mathbb T^m$, and $\e$ sufficiently small.
\end{remark}
%--------------------------------------------------------------------------------------

%--------------------------------------------------------------------------------------
Now we can choose $\beta=1/10$. By the results in Lemma~\ref{lem:stay_close_to_averaged}, Lemma~\ref{lem:throughsaddlepoint}, Lemma~\ref{lem:hit_separatrix}, and Remark~\ref{rmk:hit_separatrix}, using the strong Markov property, we obtain the following lemma:
\begin{lemma}
\label{lem:rotation}
    There exist a closed interval $I$ on $\gamma$ that does not contain the saddle point and a constant $0<c<1$ satisfying the following property: if the system \eqref{eq:auxiliary} starts at $(x,y)\in\gamma'\times\mathbb T^m$, then for all $\e$ sufficiently small
    \begin{equation}
        \Prob_{(x,y)}(\Tilde\eta_1<\bm\tau_1)\geq c
    \end{equation}
    where $\Tilde\eta_1=\inf\{t:\tbx_t^\e\in I\}$.
\end{lemma}
\begin{remark}
\label{rmk:positive_any_subset_separatrix}
    In order for us to apply Lemma~\ref{lem:hit_separatrix}, we need to choose $I$ that contains the intersection of $\Gamma_2$ and $\gamma$ in its interior. In fact, it is not difficult to show that Lemma~\ref{lem:rotation} holds for any subset of $\gamma$ with non-empty interior.
\end{remark}

\textbf{Step 2}. 
Without loss of generality, we assume that if $\bm x_t$ starts at one endpoint of $I$, then the other endpoint is $\bm x_{1/2}$.
In the remainder of this section, $\bm x_t$ always denotes this deterministic motion, irrespective of where $\tbx_t^\e$ starts. 
We aim to study the distribution of the process $(\tbx_t^\e,\tbxi_t^\e)$ starting on $I\times\mathbb T^m$ with certain $t>0$. The choice of $t$ will depend on the initial point $x$ being considered (see Figure~\ref{fig:local_limit_theorem}), and this will be convenient as we use the strong Markov property later when combining all three steps.
% Figure environment removed
To be more precise, for $x\in I$, let $s(x)$ be such that $\bm x_{s(x)}=x$ (so $0\leq s(x)\leq1/2$). We introduce a process $\tilde\zeta_{t}^{\e}$, $0\leq t$, as the second term in the expansion of $\tbx_{t}^{\e}$ around the deterministic motion $\bm x_{(s(x)+t)}$:
\begin{equation}
\label{eq:linearize_process_tilde}
\begin{aligned}
d\tilde\zeta_{t}^{\e}&=\nabla(\nabla^\perp H)(\bm x_{s(x)+t})\tilde\zeta_{t}^{\e}dt+[b(\bm x_{s(x)+t},\tbxi_{t}^{\e})-\nabla^\perp H(\bm x_{s(x)+t})]dt,~\tilde\zeta_{0}^{\e}=0.
\end{aligned}
\end{equation}
(Note that, for finite $t$, $\tilde\zeta_{s,t}^{\e,y}$ is of order $\sqrt{\e}$.) Then, by standard perturbation arguments and Gr\"onwall's inequality, it can be shown that, uniformly in $x$ such that $0\leq s(x)\leq1/2$, $0\leq t\leq 1-s(x)$, and $y\in\mathbb T^m$,
\begin{equation}
\label{eq:xtildecloseinL1}
    \E_{(x,y)}|\tbx_{t}^{\e}-\bm x_{s(x)+t}-\tilde\zeta_{t}^{\e}|=O(\e).
\end{equation}
Therefore, understanding of the distribution of $\tilde\zeta_{1-s(x)}^{\e}$ would help one to understand the distribution of $\tbx_{1-s(x)}^{\e}$.
However, it is not straightforward to study $\tilde\zeta_{t}^{\e}$ since $(\tbxi_{t}^{\e},\tilde\zeta_{t}^{\e})$ is not a Markov process. 
We introduce a related process $\zeta_{t}^{\e}$ defined using the original Markov process $\bxi_t^\e$, apply the local limit theorem to $( \bxi_{t}^{\e}, \zeta_{t}^{\e})$, and use the Girsanov theorem to get the desired estimate. Namely, let $\zeta_{t}^{\e}$, $s\leq t$, be defined by:
\begin{equation}
\label{eq:linearizeprocess}
\begin{aligned}
d \zeta_{t}^{\e}&=\nabla(\nabla^\perp H)(\bm x_{s(x)+t}) \zeta_{t}^{\e}dt+[b(\bm x_{s(x)+t}, \bxi_{t}^{\e})-\nabla^\perp H(\bm x_{s(x)+t})]dt,~ \zeta_{0}^{\e}=0.
\end{aligned}
\end{equation}
The following result is a version of the local limit theorem \cite{LLT} adapted to our case. 
\begin{theorem}
\label{thm:locallimittheorem}
    %Let $\xi$ be the Markov process that converges to its unique invariant measure $\nu$ exponentially fast.
    Let $g:[0,1]\times \mathbb T^m\to\mathbb R^2$ be a $C^\infty$ function such that $g(t,\cdot)$ spans $\mathbb R^2$ and $\int_{\mathbb T^m} g(t,y)d\mu(y)=0$ for all $t\geq 0$, where $\mu$ is the invariant measure of $ \bxi_{t}^{\e}$. 
    Then a local limit theorem holds for the following random variable as $\e\to0$ uniformly in $(x,y)\in I\times\mathbb T^m$,
    \[S^\e:=\frac{1}{\e}\int_0^{1-s(x)} g(s(x)+t, \bxi_{t}^{\e})dt.\]
    Namely, there exists a invertible covariance matrix $B(s)$ continuous in $s$ such that
    \begin{equation}
    \label{eq5:locallimittheorem}
        \lim_{\e\to0}\left| \frac{2\pi}{\e}\sqrt{\mathrm{det}B(s(x))}\cdot\Prob_{(x,y)}(S^\e-u\in[0,1)^2)-\exp({-\frac{\e \langle B(s(x))^{-1}u,u\rangle}{2}})\right|=0,
    \end{equation}uniformly in $u\in\mathbb R^2 $, $x\in I$, and $y\in\mathbb T^m$.
\end{theorem}
The second term in \eqref{eq5:locallimittheorem} is non-trivial even when $u$ takes large values (of order $1/\sqrt{\e}$), which is exactly the situation we are dealing with.
Following \eqref{eq:linearizeprocess}, we solve explicitly
\begin{equation}
     \zeta_{1-s(x)}^{\e}=\int_0^{1-s(x)}U_{s(x)+t,1}(b(\bm x_{s(x)+t}, \bxi_{t}^{\e})-\nabla^\perp H(\bm x_{s(x)+t}))dt,
\end{equation}
where $U_{t,s}$ solves the differential equation
\[
dU_{t,s}=\nabla(\nabla^\perp H)(\bm x_{s})U_{t,s}ds,
\]
and $U_{t,t}$ is the identity matrix.
Since $\bm x_t$ is deterministic, the integrand can be treated as a function only of time $t$ and $ \bxi_{t}^{\e}$. 
Moreover, for each $t$, the integrand has zero mean w.r.t. the invariant measure and spans $\mathbb R^2$, since $U_{t,1}$ is deterministic and non-singular and, for each $x$, $\{b(x,y)-\nabla^\perp H(x):y\in\mathbb T^m\}$ spans $\mathbb R^2$ by assumption \hyperlink{H4'}{\textit{(H4$'$)}}.
Then Theorem~\ref{thm:locallimittheorem} implies that
\begin{equation}
\label{eq:A_jk}
    \Prob_{(x,y)}\left(\frac{1}{\e} \zeta_{1-s(x)}^{\e}\in[j,j+1)\times[k,k+1)\right)\geq\frac{\e}{4\pi\sqrt{\mathrm{det}B(s(x))}}\exp\left(-\frac{\e \langle B(s(x))^{-1}(j,k),(j,k)\rangle}{2}\right)
\end{equation}
for all $\e$ small enough, $-1/\sqrt{\e}\leq j,k\leq 1/\sqrt{\e}$, $x\in I$, and  $y\in\mathbb T^m$. 
Finally, we compare $(\tbx_{t}^{\e},\tbxi_{t}^{\e},\tilde\zeta_{t}^{\e})$ with $( \bx_{t}^{\e}, \bxi_{t}^{\e}, \zeta_{t}^{\e})$. 
Since the added drift $c(x,y)$ in the equation of $\tbxi_{t}^{\e}$ is small compared to the diffusion term $\frac{1}{\sqrt{\e}}\sigma(y)$, it is not hard to verify that, using the Girsanov theorem, for all $\e$ small enough, $-1/\sqrt{\e}\leq j,k\leq 1/\sqrt{\e}$, $x\in I$, and  $y\in\mathbb T^m$, 
\begin{equation}
    \label{eq:A_jk1}
    \Prob_{(x,y)}\left(\frac{1}{\e}\tilde\zeta_{1-s(x)}^{\e}\in[j,j+1)\times[k,k+1)\right)\geq \frac{1}{2}\Prob_{(x,y)}\left(\frac{1}{\e} \zeta_{1-s(x)}^{\e}\in[j,j+1)\times[k,k+1)\right).
\end{equation}

\textbf{Step 3}. We proved that $\bm x_1+\tilde\zeta_{1-s(x)}^{\e}$ reaches the $O(\e)-$sized boxes with probabilities bounded from below. 
We also proved that $\tbx_{1-s(x)}^{\e}$ is $O(\e)$-close to $\bm x_1+\tilde\zeta_{1-s(x)}^{\e}$ in $L^1$. 
Let us take one generic pair $(j,k)$, let $B^{\e,K}_{j,k}=\bm x_1+[(j-K)\e,(j+1+K)\e)\times[(k-K)\e,(k+1+K)\e)$, and study the distribution of $(\tbx_t^\e,\tbxi_t^\e)$ with the initial point in $B^{\e,K}_{j,k}\times\mathbb T^m$ after time of order $O(\e)$.
% Figure environment removed
\begin{lemma}
\label{lemma:forcetosmallerbox}
    For each $\kappa>0$, $K>0$, and $\hat y\in\mathbb T^m$, there exist $t_2>0$, $c>0$, and, for each pair $(j,k)$, a point $\hat x_{j,k}^\e$ such that, for each $(x,y)\in B^{\e,K}_{j,k}\times\mathbb T^m$ and all $\e$ sufficiently small,
    $$\Prob_{(x,y)}(\tau_\kappa<t_2\e)\geq c,$$
    where $\tau_\kappa=\inf\{t:\tbx_t^\e\in B(\hat x_{j,k}^\e,\kappa\e),~\tbxi_t^\e\in B(\hat y,\kappa)\}$.
\end{lemma}
\begin{proof}
    Recall the definition of $\bm x_1$ at the beginning of Step 2 (see Figure~\ref{fig:local_limit_theorem}).
    By assumption \hyperlink{H4'}{\textit{(H4$'$)}}, $\{b(\bm x_1,y):y\in\mathbb T^m\}$ spans $\mathbb R^2$. So there exist $y_1,y_2\in\mathbb T^m$ such that $v_1:=b(\bm x_1,y_1)$ and $v_2:=b(\bm x_1,y_2)$ span $\mathbb R^2$. 
    Let us consider the set $S_{j,k}=\bigcap_{x\in B_{j,k}^{\e,K}}\{x+av_1+bv_2:a,b\geq0\}$.
    Then it is easy to see that, there exist a constant $t_2>0$ and, for each pair $(j,k)$, a point $\hat x_{j,k}^\e\in S_{j,k}$  such that for all $x\in B_{j,k}^{\e,K}$, $\hat x_{j,k}^\e=x+ a_x \e v_1+b_x \e v_2$ and $0<a_x,b_x<t_2/5$.
    There exists $\delta>0$ such that for each $x\in B(\bm x_1,2\delta)$ and each $y$ in $B(y_i,2\delta)$, $|b(x,y)-v_i|<\kappa/t_2$, $i=1,2$.
    Let $M$ be the upper bound of vector $b(x,y)$.
    For all $\e$ sufficiently small, the probability of the following event, denoted by $E$, has a lower bound, denoted by $c$, that only depends on $t_2$, $\kappa$, $M$, $y_1$, $y_2$, $\hat y$, $\delta$, and not on the starting point $(x,y)\in B(\bm x_1,\delta)\times\mathbb T^m$, thus not on $(j,k)$: 
    \[E=\begin{Bmatrix}
    \tau_1<(t_2\wedge \kappa/M)\e/5;~\tbxi^\e_{\tau_1+t}\in B(y_1,2\delta),~t\in[0,a_x\e];~\tau_2<\tau_1+a_x+(t_2\wedge \kappa/M)\e/5; \\
    \tbxi^\e_{\tau_2+t}\in B(y_2,2\delta),~t\in[0,b_x \e];~\tau_3<\tau_2+b_x\e+(t_2\wedge \kappa/M)\e/5
    \end{Bmatrix},
    \]where $\tau_1=\inf\{t\geq0:\tbxi^\e_t\in B(y_1,\delta)\}$, $\tau_2=\inf\{t\geq\tau_1+a_{x}\e:\tbxi^\e_t\in B(y_2,\delta)\}$, and $\tau_3=\inf\{t\geq\tau_2+b_{x}\e:\tbxi^\e_t\in B(\hat{y},\kappa)\}$. If $E$ is a subset of the event  $\{\tau_\kappa<t_2\e\}$, then the lemma is proved. To show the inclusion, note that on $E$,
    \begin{align*}
        |\tbx_{\tau_3}^\e-\hat x_{j,k}^\e|&=|\tbx_{\tau_3}^\e-(x+a_{x}\e v_1+b_{x}\e v_2)|\\
        &\leq |\tbx_{\tau_3}^\e-\tbx_{\tau_2+b_{x}\e}^\e|+|\tbx_{\tau_2+b_{x}\e}^\e-(\tbx_{\tau_2}^\e+ b_{x}\e v_2)|+|\tbx_{\tau_2}^\e-\tbx_{\tau_1+a_{x}\e}^\e|\\
        &\quad\quad +|\tbx_{\tau_1+a_{x}\e}^\e-(\tbx_{\tau_1}^\e+ a_{x}\e v_1)|+|\tbx_{\tau_1}^\e-x|\\
        &\leq \kappa\e.
    \end{align*}
    Besides, by the definition of $\tau_3$, $\tbxi_{\tau_3}^\e\in B(\hat{y},\kappa)$. Thus $\tau_\kappa\leq\tau_3<t_2\e$ on $E$.
\end{proof}
From now on, let $\hat y$ be the point in assumption \hyperlink{H5}{\textit{(H5)}} such that the parabolic H\"ormander condition holds at $(\bm x_1,\hat y)$ and let $p_t^\e((x,y),\cdot)$ be the density of $(\tbx_{t\e}^\e,\tbxi_{t\e}^\e)$ starting at $(x,y)$.
\begin{lemma}
\label{lem5:density_xy}
    There exists $\kappa>0$ such that for each $\hat x\in B(\bm x_1,\kappa)$ and all $\e$ sufficiently small, there is a domain $C^\e_{\hat x,\hat y}\subset V^\e\times\mathbb T^m$ with $\lambda(C^\e_{\hat x,\hat y})>\kappa\e^2$ and $p_{1}^\e((x,y),\cdot)>\kappa/\e^2$ on $C^\e_{\hat x,\hat y}$ for $(x,y)\in B(\hat x,\kappa\e)\times B(\hat y,\kappa)$.
\end{lemma}
\begin{proof}
Consider the stochastic processes that depend on the parameters $(\e,x,y)$:
\label{eq:theta_process}
\begin{equation}
    \label{eq5:difference}
    \begin{aligned}
    d\theta_t^{\e,x,y}&=b(x+\e\theta_t^{\e,x,y},y+\eta_t^{\e,x,y})dt,~\theta_0^{\e,x,y}=0\in\mathbb R^2,\\
    d\eta_t^{\e,x,y}&= v(y+\eta_t^{\e,x,y})dt+\e c(x+\e\theta_t^{\e,x,y},y+\eta_t^{\e,x,y})dt+\sigma(y+\eta_t^{\e,x,y}) dW_t,~\eta_0^{\e,x,y}={0}\in\mathbb R^m.
    \end{aligned}
\end{equation}
Since, by assumption \hyperlink{H5}{\textit{(H5)}}, the parabolic H\"ormander condition for equation \eqref{eq:theprocess1} holds at $(\bm x_1,\hat y)$, it is not hard to see that, if $(x,y)$ is close to $(\bm x_1,\hat y)$ and $\e$ is small, the parabolic H\"ormander condition holds for \eqref{eq5:difference} at $0$ and the distribution of $(\theta_t^{\e,x,y},\eta_t^{\e,x,y})$ is absolutely continuous w.r.t. the Lebesgue measure (\cite{Nualart}). 
Moreover, if the density function, denoted by $\tilde p_1^{\e,x,y}(\theta,\eta)$, exists, it is continuous in $\e,x,y,\theta$, and $\eta$. 
Let $\hat\theta$ and $\hat\eta$ satisfy that $\tilde p_1^{0,\bm x_1,\hat y}(\hat\theta,\hat\eta)>0$. 
Then there exists $0<\delta<1$ such that $\tilde p_1^{\e,x,y}(\theta,\eta)$ exists and is greater than $\delta$ for all $0<\e<\delta$, $x\in B(\bm x_1,\delta)$, $y\in B(\hat y,\delta)$, $\theta\in B(\hat\theta,\delta)$, and $\eta\in B(\hat\eta,\delta)$. 
For $\hat x\in B(\bm x_1,\delta/2)$, define $C^\e_{\hat x,\hat y}=B(\hat x+\e\hat\theta,\e\delta/2)\times B(\hat y+\hat\eta,\delta/2)$. 
Then, for $(x,y)\in B(\hat x,\e\delta/2)\times B(\hat y,\delta/2)$, and $(x',y')\in C^\e_{\hat x,\hat y}$, and $0<\e<\delta$, we have that 
\[p^\e_1((x,y),(x',y'))=\frac{1}{\e^2}\tilde p^{\e,x,y}\left(\frac{x'-x}{\e},y'-y\right)>\frac{\delta}{\e^2}.\]
The result holds with $\kappa=(\delta/2)^{m+2}$.
\end{proof}
\begin{lemma}
\label{lem:pijk}
    For each $K>0$, there exist constants $c>0$ and $t_1>0$ such that for all $-1/\sqrt{\e}\leq j,k\leq1/\sqrt{\e}$, there exists a measure $\pi^\e_{j,k}$ and a stopping time $\Tilde\eta_3^{j,k}<t_1\e$ such that for each $(x,y)\in B^{\e,K}_{j,k}\times\mathbb T^m$, the distribution of $(\tbx_{\Tilde\eta_3^{j,k}}^\e,\tbxi_{\Tilde\eta_3^{j,k}}^\e)$ starting at $(x,y)$ has $\pi^\e_{j,k}$ as a component and $\pi^\e_{j,k}(V^\e\times\mathbb T^m)>c$ for all $\e$ sufficiently small.
\end{lemma}
\begin{proof}
    We fix constant $\kappa>0$ such that the statements in Lemma~\ref{lem5:density_xy} hold. 
    Then, for the fixed $\kappa$, by Lemma~\ref{lemma:forcetosmallerbox}, we fix $t_2>0$, $c'>0$, and the point $\hat x_{j,k}^\e$ for each pair $(j,k)$ such that for all $(x,y)\in B^{\e,K}_{j,k}\times\mathbb T^m$ and $\e$ small, $\Prob_{(x,y)}(\tau_\kappa<t_2\e)\geq c'$, where $\tau_\kappa=\inf\{t:\tbx_t^\e\in B(\hat x_{j,k}^\e,\kappa\e),~\tbxi_t^\e\in B(\hat y,\kappa)\}$.
   It follows from Lemma~\ref{lem5:density_xy} that there is a domain $C^\e_{j,k}\subset V^\e\times\mathbb T^m$ with $\lambda(C^\e_{j,k})>\kappa\e^2$ and $p_{1}^\e((x,y),\cdot)>\kappa/\e^2$ on $C^\e_{j,k}$ for all $(x,y)\in B(\hat x_{j,k}^\e,\kappa\e)\times B(\hat y,\kappa)$. 
   Then the result follows if we define $c=c'\kappa^2$, $\pi^\e_{j,k}=c'\kappa/\e^2\cdot\chi_{\{C^\e_{j,k}\}}\lambda$, $t_1=t_2+2$, and $\Tilde\eta_3^{j,k}=\tau_{\kappa}\wedge t_2\e+\e<t_1\e$.
\end{proof}
Now let us combine Step 2 and Step 3 together to get the following result concerning the total variation distance of $(\tbx_{\bm\tau_1},\tbxi_{\bm\tau_1})$ with different starting points on $I\times\mathbb T^m$:
\begin{lemma}
\label{lem5:total_variation_on_I}
    For each $(x,y)\in I\times\mathbb T^m$, let $\tilde\mu_{x,y}^\e$ be the measure induced by $(\tbx_{\bm\tau_1},\tbxi_{\bm\tau_1})$ starting at $(x,y)$. Then there exists $c>0$ such that $\mathrm{TV}(\tilde\mu_{x,y}^\e,\tilde\mu_{x',y'}^\e)<1-c$ for any $(x,y),(x',y')\in I\times\mathbb T^m$ and all $\e$ sufficiently small.
\end{lemma}
\begin{proof}
    It suffices to show that there exist $c>0$ and a stopping time $\tilde\eta\leq\bm\tau_1$ such that the total variation distance of $(\tbx_{\tilde\eta},\tbxi_{\tilde\eta})$ with different starting points on $I\times\mathbb T^m$ is no more than $1-c$. 
    Recall the definitions of $s(x)$ and $\tilde\zeta_t^\e$ in Step 2.
    For the process $(\tbx_t^\e,\tbxi_t^\e)$ starting at $(x,y)\in I\times\mathbb T^m$, define
    \[A_{j,k}^{\e}=\{\frac{1}{\e}\tilde\zeta_{1-s(x)}^{\e}\in[j,j+1)\times[k,k+1)\},\]
    \[E_{K}^{\e}=\{|\tbx_{1-s(x)}^{\e}-\bm x_1-\tilde\zeta_{1-s(x)}^{\e}|> K\e\}\cup\{\sup_{0\leq t\leq 1-s(x)}|H(\tbx_{t}^{\e})|>K\sqrt{\e}\}.\]
    Using \eqref{eq:A_jk} and \eqref{eq:A_jk1}, we can find a constant $c'>0$ such that, for all $x\in I$, $y\in \mathbb T^m$, $\e$ sufficiently small, and $-1/\sqrt{\e}\leq j,k\leq1/\sqrt{\e}$, $\Prob_{(x,y)}(A_{j,k}^{\e})\geq c'\e$.
    And using \eqref{eq:slowx} and \eqref{eq:xtildecloseinL1} we can choose $K$ large enough such that, for all $x\in I$, $y\in \mathbb T^m$, and $\e$ sufficiently small, $\Prob_{(x,y)}(E_{K}^{\e})<c'/100$.
    Let $\tilde\eta_2=1-s(x)\wedge\bm\tau_1$.
    Then it is not hard to see that \[\sum_{-1/\sqrt{\e}\leq j,k\leq1/\sqrt{\e}}\Prob_{(x,y)}(A_{j,k}^\e\cap\{\tbx_{\tilde\eta_2}\not\in B^{\e,K}_{j,k}\})<c'/100.\]
    Now let us define, for $(x,y)\in I\times\mathbb T^m$,
    \[
    R_{x,y}^\e=\{(j,k):-1/\sqrt{\e}\leq j,k\leq1/\sqrt{\e},\Prob_{(x,y)}(A_{j,k}^\e\cap\{\tbx^\e_{\Tilde\eta_2}\in B^{\e,K}_{j,k}\})<c'\e/2\}.
    \]
    Then we know that $|R_{x,y}^\e|<\frac{1}{50\e}$ since, for every $(j,k)\in R_{x,y}^\e$, 
    \[\Prob_{(x,y)}(A_{j,k}^{\e}\cap \{\tbx^\e_{\Tilde\eta_2}\not\in B^{\e,K}_{j,k}\})\geq \Prob_{(x,y)}(A_{j,k}^{\e})-\Prob_{(x,y)}(A_{j,k}^\e\cap\{\tbx^\e_{\Tilde\eta_2}\in B^{\e,K}_{j,k}\})\geq c'\e/2.\]
    Let the constants $c''>0$, $t_1>0$, the stopping time $\Tilde\eta_3^{j,k}<t_1\e$, and $\pi^\e_{j,k}$ be defined as in Lemma~\ref{lem:pijk}. 
    %Let $\Tilde\eta_3^{x,y}$ denote the corresponding stopping time with initial condition $(x,y)$. 
    Define
    \[
        \pi^\e=\frac{1}{2}c'\e\sum_{-1/\sqrt{\e}\leq j,k\leq1/\sqrt{\e}}\pi^\e_{j,k},~~~~~~~~~
        \hat\pi_{x,y,x',y'}^\e=\frac{1}{2}c'\e\sum_{(j,k)\in R^\e_{x,y}\cup R^\e_{x',y'}}\pi^\e_{j,k}.
    \]
    In order to define the desired stopping time, we first run the process starting on $I\times\mathbb T^m$ for time $\tilde\eta_2$ (with overwhelming probability, it is the time for the deterministic motion with the same stating point to reach $\bm x_1$).
    Then we use the locations of both $\tilde\zeta_{\tilde\eta_2}^\e$ and $\tx_{\tilde\eta_2}^\e$ to determine whether the process continues and, if it continues, we choose the stopping time based on Lemma~\ref{lem:pijk}.
    Namely, we define 
    \begin{equation}
        \Tilde\eta=\Tilde\eta_2+\sum_{-1/\sqrt{\e}\leq j,k\leq1/\sqrt{\e}}\chi(A_{j,k}^{\e}\cap\{\tbx^\e_{\Tilde\eta_2^{j,k}}\in B_{j,k}^{\e,K}\})\cdot \Tilde\eta_3^{j,k}(\tbx^\e_{\tilde\eta_2},\tbxi^\e_{\tilde\eta_2}),
    \end{equation}where $\Tilde\eta_3^{j,k}(x,y)$ denotes the stopping time with initial condition $(x,y)$.
    Then it follows from previous results that, for any pair $(x,y),(x',y')\in I\times\mathbb T^m$, there is a common component $\pi^\e-\hat\pi^\e_{x,y,x',y'}$ of the distributions of $(\tbx^\e_{\Tilde\eta},\tbxi^\e_{\Tilde\eta})$ starting from $(x,y)$ and $(x',y')$, respectively. Moreover, $(\pi^\e-\hat\pi^\e_{x,y,x',y'})(V^\e\times\mathbb T^m)>c'c''$ since
    $|R^\e_{x,y}|<\frac{1}{50\e}$ and $|R^\e_{x',y'}|<\frac{1}{50\e}$. Therefore, the total variation is no more than $1-c'c''$.
\end{proof}
Finally, we combine the result we just obtained with Step 1 to prove Lemma~\ref{lem5:expo_ergodicity}.
\begin{proof}[Proof of Lemma~\ref{lem5:expo_ergodicity}]
    As we discussed, the result is equivalent to the exponential convergence in total variation of $(\tbx_{\bm\tau_n}^\e,\tbxi_{\bm\tau_n}^\e)$ on $\gamma'\times\mathbb T^m$, uniformly in $\e$ and in the initial distribution. 
    Let $\mu^{\e}_{x,y}$ denote the measure on $\gamma'\times\mathbb T^m$ induced by $(\tbx_{\bm\tau_1}^\e,\tbxi_{\bm\tau_1}^\e)$ with the starting point $(x,y)\in\gamma'\times\mathbb T^m$.
    Then it suffices to prove that there exists $c>0$ such that, for every pair $(x,y)$, $(x',y')\in \gamma'\times\mathbb T^m$ and all $\e$ sufficiently small, $\mathrm{TV}(\mu^{\e}_{x,y},\mu^{\e}_{x',y'})<1-c$, which follows from Lemma~\ref{lem:rotation} and Lemma~\ref{lem5:total_variation_on_I}.
\end{proof}



\section{Proof of the main result}
\label{sec:proofofthemainresult}
Since we deal with both the original and the auxiliary processes in this section, certain notation needs clarifying to avoid possible ambiguity: the process $(\tx_t^\e,\txi_t^\e)$ represents not a generic process with arbitrary bounded $c(x,y)$ but only the auxiliary process with $\tilde c(x,y)$ satisfying \eqref{eq:added term}; 
$\sigma_n$, $\tau_n$ are defined as in \eqref{eq:stopping_times}, and 
$\Tilde\sigma_n$, $\Tilde{\tau}_n$ represent the corresponding stopping time w.r.t. $(\tx_t^\e,\txi_t^\e)$. In \eqref{eq:definition_of_operator_AB}, we defined $\mathcal L_{c}$ on each edge for a generic $c(x,y)$. Here we give a more explicit definition of $\mathcal L_{\tilde c}=\tilde L_k$ on the edge $I_k$:
\begin{align*}
    \tilde L_kf(h)&=\frac{1}{2}A_k(h)f''(h)+\tilde B_k(h)f'(h),\\
    A_k(h)&=\frac{2}{Q_k(h)}\int_{\gamma_k(h)}\frac{1}{|\nabla H(x)|}\int_0^\infty\E_\mu b_h(x,\xi_s)b_h(x,\xi_0)dsdl,\\
    \Tilde B_k(h)&=\frac{1}{Q_k(h)}\int_{\gamma_k(h)}\frac{1}{|\nabla H(x)|}\int_0^\infty\E_\mu\mathrm{div}_x (b_h(x,\xi_s)(b(x,\xi_0)-\nabla^{\perp}H(x)))dsdl.
\end{align*}
One can easily check that this is consistent with the definitions of $\bar A$ and $\bar B_c$ in \eqref{eq:definition_of_operator_AB}, which are the generalizations of the coefficients defined in Section~\ref{sec:mainresult}, and
\begin{equation}
\label{eq6:derivative}
    \frac{1}{2}[A_k(h_k)Q_k(h_k)f'(h_k)]'=\frac{1}{2}A_k(h_k)Q_k(h_k)f''(h_k)+\Tilde B_k(h_k)Q_k(h_k)f'(h_k).
\end{equation}
\begin{lemma}
\label{lem:zeroexpectation}
    For each $f\in \mathcal D$ and all $\e$ sufficiently small, we have $\Expe_{\nu^\e} \int_0^{\sigma_1}\mathcal L_{\Tilde c}f(h(\tx_t^\e))dt=0$.
\end{lemma}
\begin{proof}
    Since the process $(\tx_t^\e,\txi_t^\e)$ on $M\times\mathbb T^m$ is recurrent, and the measure $\lambda\times\mu$ is the invariant measure, by Theorem~2.1 in \cite{Khasminskii}, we have that for any measurable set $A\subset M$,
    \[
        \int_M\chi_A(x)d\lambda(x)=\lambda(A)=(\lambda\times\mu)(A\times\mathbb T^m)=\E_{\nu^\e}\int_0^{\sigma_1}\chi_A(\tx_t^\e)dt.
    \]
    Thus,
    \[
        \int_M \mathcal L_{\Tilde c}f(h(x))d\lambda(x)=\E_{\nu^\e}\int_0^{\sigma_1}\mathcal L_{\Tilde c}f(h(\tx_t^\e))dt.
    \]
    So it suffices to show that the left hand side is zero. By \eqref{eq6:derivative} and \eqref{eq2:gluing_condition},
\begin{align*}
       \int_{M}\mathcal L_{\Tilde c}f(h(x))d\lambda(x)&=\sum_{k=1}^3\int_{I_k} \tilde L_kf(h_k)Q_k(h_k)dh_k\\
       &=\sum_{k=1}^3\int_{I_k}(\frac{1}{2}A_k(h_k)f''(h_k)+\Tilde B_k(h_k)f'(h_k))Q_k(h_k)dh_k\\
       &=\sum_{k=1}^3\int_{I_k}\frac{1}{2}[A_k(h_k)Q_k(h_k)f'(h_k)]'dh_k\\
       &=\frac{1}{2}\sum_{k=1}^3p_k\lim_{h_k\to O}f'(h_k)\\
       &=0.
\end{align*}
%Because $f'$ decays so fast at infinity.
%Where does the inaccessibility play its role? In this statement, seems like we only need A is zero at extremum, but the criterion needs that 1/A is not integrable.
\end{proof}

Let us verify the analogue of \eqref{eq:mg_problem_M} in the case of the auxiliary process $(\tx_t^\e,\txi_t^\e)$.
\begin{proposition}
\label{prop:auxiliary}
For each $f\in \mathcal D$ and $T>0$,
    \begin{equation}
        \E_{(x,y)}[f(h(\tx_{\eta}^\e))-f(h(x))-\int_0^{\eta}\mathcal L_{\Tilde c}f(h(\tx_t^\e))dt]\to0
    \end{equation}
    as $\e\to0$, uniformly in $x\in M$, $y\in\mathbb T^m$, and $\eta$ is a stopping time bounded by $T$.
\end{proposition}
\begin{proof}
We divide the time interval $[0,\eta]$ into visits to the separatrix. Since $\sigma_n\to\infty$,
    \begin{align}
        &|\E_{(x,y)}[f(h(\tx_{\eta}^\e))-f(h(x))-\int_0^{\eta}\mathcal L_{\Tilde c}f(h(\tx_t^\e))dt]|\nonumber\\
        &\leq|\lim_{n\to\infty}\E_{(x,y)}[f(h(\tx_{\tilde\sigma_n}^\e))-f(h(x))-\int_0^{\tilde\sigma_n}\mathcal L_{\Tilde c}f(h(\tx_t^\e))dt]|\nonumber\\
        &\quad+|\lim_{n\to\infty}\E_{(x,y)}\E_{(\tx_{\eta}^\e,\txi_{\eta}^\e)}[f(h(\tx_{\tilde\sigma_n}^\e))-f(h(\tx_0^\e))-\int_0^{\tilde\sigma_n}\mathcal L_{\Tilde c}f(h(\tx_t^\e))dt]|\nonumber\\
        &\leq 2\sup_{(x,y)\in M\times\mathbb T^m}|\E_{(x,y)}[f(H(\tx^\e_{\tilde\sigma}))-f(H(x))-\int_0^{\sigma}{\mathcal L_{\tilde c}} f(H(\tx_s^\e))ds]|\label{eq6:to_separatrx}\\
        &\quad+2\lim_{n\to\infty}\sup_{(x,y)\in\gamma\times\mathbb T^m}|\E_{(x,y)}[f(h(\tx_{\tilde\sigma_n}^\e))-f(h(x))-\int_0^{\sigma_n}\mathcal L_{\tilde c}f(h(\tx_t^\e))dt]|.\label{eq6:between_separatrices}
    \end{align}
    Note that \eqref{eq6:to_separatrx} converges to $0$ due to Proposition~\ref{prop:up_to_separatrix}, and \eqref{eq6:between_separatrices} also converges to $0$ since
    \begin{align*}
        &\lim_{n\to\infty}\sup_{(x,y)\in\gamma\times\mathbb T^m}|\E_{(x,y)}[f(h(\tx_{\tilde\sigma_n}^\e))-f(h(x))-\int_0^{\tilde\sigma_n}\mathcal L_{\tilde c}f(h(\tx_t^\e))dt]|\\
        &\leq\lim_{n\to\infty}\sup_{(x,y)\in\gamma\times\mathbb T^m}\sum_{k=0}^{n-1}|\E_{(x,y)}\int_{\tilde\sigma_k}^{\tilde\sigma_{k+1}}\mathcal L_{\tilde c}f(h(\tx_t^\e))dt|\\
        &\leq\lim_{n\to\infty}\sup_{(x,y)\in\gamma\times\mathbb T^m}\sum_{k=0}^n \left(2\cdot\mathrm{TV}(\nu_{x,y}^{k,\e},\nu^\e)\cdot\sup_{(x',y')\in\gamma\times\mathbb T^m}|\E_{(x',y')}\int_0^{\tilde\sigma_1}\mathcal L_{\tilde c}f(h(\tx_t^\e))dt|\right)\\
        &=0,
    \end{align*}
    where the second inequality is due to Lemma~\ref{lem:zeroexpectation} and the last equality follows from Proposition~\ref{prop:up_to_separatrix}, Lemma~\ref{lem5:expo_ergodicity}, and Proposition~\ref{prop:exit_time_from_separatrix}.
    Thus, the desired result holds.
\end{proof}
{To generalize the result to the original process $(X_t^\e,\xi_t^\e)$ on $M\times\mathbb T^m$, we need the next two technical results.
We start with a simple corollary of Lemma~\ref{lem:number_excursion}, which controls the number of excursions or, equivalently, the number of stopping times $\sigma_n$ and $\tau_n$ in finite time.}
\begin{corollary}
\label{cor:num_excursion}
    For a given $t>0$, the expected number of excursions before $t$ is $O(\e^{-\alpha})$:
    \begin{equation}
    \label{eq:num_excursion}
    \sum_{n=0}^\infty\Prob_{(x,y)}(\tau_{n+1}<t)\leq\sum_{n=0}^\infty\Prob_{(x,y)}(\sigma_n<t)\leq\frac{e^t}{\kappa}\e^{-\alpha},
    \end{equation}where $\kappa$ is the constant chosen in Lemma~\ref{lem:number_excursion}.
\end{corollary}
\begin{proof}
    By Lemma~\ref{lem:number_excursion} and the strong Markov property,
\begin{equation}
    \sup_{(x,y)\in M\times\mathbb T^m}\E_{(x,y)} e^{-\sigma_n}\leq(\sup_{(x,y)\in\gamma'\times\mathbb T^m}\E_{(x,y)} e^{-\sigma})^n\leq(1-\kappa\e^\alpha)^n.
\end{equation}
Thus, by Markov's inequality, for all $n>0$,
\begin{equation}
    \Prob_{(x,y)}(\tau_{n+1}<t)\leq\Prob_{(x,y)}(\sigma_n<t)\leq e^t\E_{(x,y)} e^{-\sigma_n}\leq e^t(1-\kappa\e^\alpha)^n,
\end{equation}
and \eqref{eq:num_excursion} follows by taking the sum.
\end{proof}
\begin{lemma}
\label{lem:aux_to_ori}
    For each $f\in\mathcal D$ and $\delta>0$ there is $0<\rho<1$ such that, for all $x\in\gamma$, $y\in \mathbb T^m$, and all $\e$ sufficiently small,
    \begin{equation}
    \label{eq:auxiliary_to_main}
    \begin{aligned}
        &\sup_{{\sigma'}\leq\rho}|\E_{(x,y)}\sum_{n=0}^\infty\chi_{\{\sigma_n<{\sigma'}\}}[f(h(X_{\tau_{n+1}}^\e))-f(h(X_{\sigma_n}^\e))-\int_{\sigma_n}^{\tau_{n+1}}\mathcal Lf(h(X_s^\e))ds]|\\ 
        &\quad\leq\delta\rho+\e^\alpha\delta\sum_{n=0}^\infty\Prob_{(x,y)}(\sigma_n<\rho),
    \end{aligned}
    \end{equation}
    where $\sigma'$ is a stopping time w.r.t. $\mathcal F^{X_\cdot^\e}_t$.
\end{lemma}
\begin{proof}
    The result holds either with or without the integral term since nearly all of the time is spent from $\tau_n$ to $\sigma_n$.
    To be precise, by the strong Markov property, Corollary~\ref{cor:num_excursion}, and Proposition~\ref{prop:exit_time_from_separatrix},
    \begin{equation}
    \label{eq:integral_not_matter}
    \begin{aligned}
        &\sup_{(x,y)\in\gamma\times\mathbb T^m}\sup_{{\sigma'}\leq\rho}|\E_{(x,y)}\sum_{n=0}^\infty\chi_{\{\sigma_n<{\sigma'}\}}\int_{\sigma_n}^{\tau_{n+1}}\mathcal Lf(h(X_s^\e))ds|\\
        &\lesssim \sup_{(x,y)\in\gamma\times\mathbb T^m}\sup_{{\sigma'}\leq\rho}\sum_{n=0}^\infty|\E_{(x,y)}\chi_{\{\sigma_n<{\sigma'}\}}\E_{(X_{\sigma_n}^\e,\xi_{\sigma_n}^\e)}\tau_1|=O(\e^\alpha|\log\e|).
    \end{aligned}
    \end{equation}
    Thus, it suffices to prove for all $\e$ sufficiently small
    \begin{equation}
    \label{eq:without_integral}
        \sup_{{\sigma'}\leq\rho}|\E_{(x,y)}\sum_{n=0}^\infty\chi_{\{\sigma_n<{\sigma'}\}}[f(h((X_{\tau_{n+1}}^\e))-f(h((X_{\sigma_n}^\e))|\leq\delta\rho+\e^\alpha\delta\sum_{n=0}^\infty\Prob_{(x,y)}(\sigma_n<\rho).
    \end{equation}

    Let us prove this for $\tx_t^\e$ first using Proposition~\ref{prop:auxiliary}, then apply the Girsanov theorem to get the result for $X_t^\e$.
    Let $\tilde\sigma'$ be the analogue of $\sigma'$ w.r.t. $\mathcal F^{\tx_\cdot^\e}_t$.
    Divide the time interval $[0,\tilde\sigma']$ into excursions using stopping times $\tilde\sigma_n$ and $\tilde\tau_n$:
    \begin{align}
    &\E_{(x,y)}[f(h(\tx_{\Tilde\sigma'}^\e))-f(h(x))-\int_0^{\Tilde\sigma'}\mathcal Lf(h(\tx_t^\e))dt]\label{eq:auxleft}\\
    &\quad = \E_{(x,y)}[f(h(\tx_{\Tilde\sigma'\wedge\Tilde\sigma}^\e))-f(h(x))-\int_0^{\Tilde\sigma'\wedge\Tilde\sigma}\mathcal Lf(h(\tx_t^\e))dt]\label{eq:auxfirstsum}\\
    &\quad+\sum_{n=0}^\infty \E_{(x,y)}\left(\chi_{\{\Tilde\sigma_n<\Tilde\sigma'\}}[f(h(\tx_{\Tilde\tau_{n+1}\wedge \Tilde\sigma'}^\e))-f(h(\tx_{\Tilde\sigma_n}^\e))-\int_{\Tilde\sigma_n}^{\Tilde\tau_{n+1}\wedge \Tilde\sigma'}\mathcal Lf(h(\tx_t^\e))dt]\right)\label{eq:auxsecondsum}\\
    &\quad+\sum_{n=1}^\infty \E_{(x,y)}\left(\chi_{\{\Tilde\tau_n<\Tilde\sigma'\}}[f(h(\tx_{\Tilde\sigma_{n}\wedge \Tilde\sigma'}^\e))-f(h(\tx_{\Tilde\tau_n}^\e))-\int_{\Tilde\tau_n}^{\Tilde\sigma_{n}\wedge \Tilde\sigma'}\mathcal Lf(h(\tx_t^\e))dt]\right)\label{eq:auxthirdsum}.
\end{align}
    Thus, \eqref{eq:auxsecondsum} converges to $0$ uniformly for all $x\in\gamma$ and $\Tilde\sigma'\leq\rho$ due to the convergence of \eqref{eq:auxleft}, \eqref{eq:auxfirstsum}, and \eqref{eq:auxthirdsum}, by Proposition~\ref{prop:auxiliary}, Proposition~\ref{prop:up_to_separatrix}, and Lemma~\ref{lem:eps_avg_prin_to_sp} with Corollary~\ref{cor:num_excursion}, respectively. Note that \eqref{eq:integral_not_matter} also holds for $\tx_t^\e$, hence we conclude that
    \begin{equation}
    \label{eq:aux_stop_at_sigma'}
        \sup_{(x,y)\in\gamma\times\mathbb T^m}\sup_{{\Tilde\sigma'}\leq\rho}\sum_{n=0}^\infty \E_{(x,y)}\left(\chi_{\{\Tilde\sigma_n<\Tilde\sigma'\}}[f(h(\tx_{\Tilde\tau_{n+1}\wedge \Tilde\sigma'}^\e))-f(h(\tx_{\Tilde\sigma_n}^\e))]\right)
        \to0.
    \end{equation}
    
    To apply the Girsanov theorem, we choose a sufficiently small time interval and use the fact that the transition probability of $(X_t^\e,\xi_t^\e)$ is similar to that of $(\tx_t^\e,\txi_t^\e)$ in the sense that they are absolutely continuous with density close to $1$.
    More precisely, for any fixed $\delta'>0$, by the Girsanov theorem, we can choose a constant $\rho_1$ such that for all $0<\rho<\rho_1$,
    \begin{equation}
        \mu_{x,y}^\e\left(\left|\frac{d\Tilde\mu_{x,y}^\e}{d\mu_{x,y}^\e}-1\right|<\delta'\right)\geq 1-\rho^2,
    \end{equation}
    where $\mu_{x,y}^\e$ and $\Tilde\mu_{x,y}^\e$ are the measures on $\bm{\mathrm C}[0,\rho]$ induced by $(X_t^\e,\xi_t^\e)$ and $(\tx_t^\e,\txi_t^\e)$. Define
    \[C'=\left\{\left|\frac{d\Tilde\mu_{x,y}^\e}{d\mu_{x,y}^\e}-1\right|<\delta\right\}\subset\bm{\mathrm C}[0,\rho],~\Omega'=\{(X_t^\e,~t\in[0,\rho])\in C'\}.\]
    Note that the quantity in \eqref{eq:without_integral} primarily depends on the behavior of the processes on time interval $[0,\sigma']$ and event $\Omega'$. 
    Indeed, we can replace the stopping times $\tau_n$ by $\tau_n\wedge\sigma'$ with $O(\e^\alpha)$ errors.
    To replace $\Omega$ with $\Omega'$, we need several additional results that control the difference.
    
    As in Corollary~\ref{cor:num_excursion}, we fix $\kappa>0$ and choose a large constant $C>0$ independent of $\rho$ such that
    \begin{equation}
        \sum_{n=[C\log(C/\rho)\e^{-\alpha}]}^\infty\Prob_{(x,y)}(\sigma_n<\rho)\leq\sum_{n=[C\log(C/\rho)\e^{-\alpha}]}^\infty e^{\rho}(1-\kappa\e^\alpha)^n\leq\delta'\rho\e^{-\alpha}.
    \end{equation}
    Now we choose $\rho_2>0$ such that, for all $0<\rho<\rho_2$, $C\rho\log(C/\rho)<\delta'$. Hence, for all $\sigma'\leq\rho$,
    \begin{equation}
        \sum_{n=0}^\infty\Prob_{(x,y)}(\{\sigma_n<\sigma'\}\setminus\Omega')\leq C\rho^2\log(C/\rho)\e^{-\alpha}+\delta'\rho\e^{-\alpha}\leq2\delta'\rho\e^{-\alpha}.
    \end{equation}
    Thus, with $K:=\max_{I_k\sim O}|\lim_{{h_k\in I_k, h_k\to O}}f'(h_k)|$, we obtain
    \begin{equation}
    \label{eq6:Omega'}
        |\E_{(x,y)}\sum_{n=0}^\infty\chi_{\{\sigma_n<{\sigma'}\}\setminus\Omega'}[f(h(X_{\tau_{n+1}\wedge\sigma'}^\e))-f(h(X_{\sigma_n}^\e))]|\leq 2(K+1)\delta'\rho.
    \end{equation}
    By following the same steps, we can choose $\rho_3>0$ such that for all $0<\rho<\rho_3$,
    \begin{equation}
    \label{eq6:Omega'_tilde}
        |\E_{(x,y)}\sum_{n=0}^\infty\chi_{\{\tilde\sigma_n<{\tilde\sigma'}\}\setminus\Omega'}[f(h(\tx_{\tilde\tau_{n+1}\wedge\tilde\sigma'}^\e))-f(h(\tx_{\tilde\sigma_n}^\e))]|\leq 2(K+1)\delta'\rho.
    \end{equation}
    It remains to consider
    \begin{equation}
    \label{eq6:omega_expc}
        |\E_{(x,y)}\sum_{n=0}^\infty\chi_{\{\sigma_n<{\sigma'}\}\cap\Omega'}[f(h(X_{\tau_{n+1}\wedge\sigma'}^\e))-f(h(X_{\sigma_n}^\e))]|,
    \end{equation}
    which can be written and estimated as, with $F$ denoting the functional on $\bm{\mathrm C}[0,\rho]$ found inside the expectation in \eqref{eq6:omega_expc}, 
    \begin{equation}
    \begin{aligned}
        \left|\int_{C'} F d\mu_{x,y}^\e\right|&=\left|\int_{C'} Fd\Tilde\mu_{x,y}^\e-\int_{C'} F\left(\frac{d\Tilde\mu_{x,y}^\e}{d\mu_{x,y}^\e}-1\right) d\mu_{x,y}^\e\right|\\
        &\leq \left|\int_{C'} Fd\Tilde\mu_{x,y}^\e\right|+\delta'\int_{C'} |F|d\mu_{x,y}^\e.
    \end{aligned}
    \end{equation}
    The first term is bounded by $2(K+2)\delta'\rho$ due to \eqref{eq:aux_stop_at_sigma'} and \eqref{eq6:Omega'_tilde}. The second term is simply bounded by $(K+1)\delta'\e^\alpha\sum_{n=0}^\infty\Prob_{(x,y)}(\sigma_n<\rho)$. Thus, we see that the left-hand side in \eqref{eq:without_integral} is no more than $(4K+6)\delta'(\rho+\e^\alpha\sum_{n=0}^\infty\Prob_{(x,y)}(\sigma_n<\rho))$ with finite $K$ independent of $\delta'$ and $\rho$ for all $\e$ sufficiently small. It remains to take $\delta'=\delta/(4K+6)$.
\end{proof}

\begin{proof}[Proof of Proposition~\ref{prop:main_result}]
Fix arbitrary $\delta>0$. We divide the time interval $[0,\eta]$ into excursions from $\gamma$ to $\gamma'$ and from $\gamma'$ to $\gamma$ by using stopping times $\sigma_n$ and $\tau_n$:
\begin{align}
    &\E_{(x,y)}[f(h(X_{\eta}^\e))-f(h(x))-\int_0^{\eta}\mathcal Lf(h(X_t^\e))dt]\nonumber\\
    &\quad = \E_{(x,y)}[f(h(X_{\eta\wedge\sigma}^\e))-f(h(x))-\int_0^{\eta\wedge\sigma}\mathcal Lf(h(X_t^\e))dt]\label{eq:firstsum}\\
    &\quad+\sum_{n=0}^\infty \E_{(x,y)}\left(\chi_{\{\sigma_n<\eta\}}[f(h(X_{\tau_{n+1}\wedge \eta}^\e))-f(h(X_{\sigma_n}^\e))-\int_{\sigma_n}^{\tau_{n+1}\wedge \eta}\mathcal Lf(h(X_t^\e))dt]\right)\label{eq:secondsum}\\
    &\quad+\sum_{n=1}^\infty \E_{(x,y)}\left(\chi_{\{\tau_n<\eta\}}[f(h(X_{\sigma_{n}\wedge \eta}^\e))-f(h(X_{\tau_n}^\e))-\int_{\tau_n}^{\sigma_{n}\wedge \eta}\mathcal Lf(h(X_t^\e))dt]\right)\label{eq:thirdsum}.
\end{align}
Here \eqref{eq:firstsum} converges to $0$ by Proposition~\ref{prop:up_to_separatrix} and \eqref{eq:thirdsum} converges to $0$ by Lemma~\ref{lem:eps_avg_prin_to_sp} and Corollary~\ref{cor:num_excursion}. It remains to consider \eqref{eq:secondsum} and it suffices to consider instead
\begin{equation}
\label{eq:secondsum1}
    \sum_{n=0}^\infty \E_{(x,y)}\left(\chi_{\{\sigma_n<\eta\}}[f(h(X_{\tau_{n+1}}^\e))-f(h(X_{\sigma_n}^\e))-\int_{\sigma_n}^{\tau_{n+1}}\mathcal Lf(h(X_t^\e))dt]\right)
\end{equation}
because the difference converges to $0$ by Proposition~\ref{prop:exit_time_from_separatrix}.
By Proposition~\ref{prop:auxiliary}, we choose $0<\rho<1$ such that \eqref{eq:auxiliary_to_main} holds for $\delta$ and all $\e$ sufficiently small. We introduce the stopping times $\hat\sigma_n$ by letting $\hat\sigma_0=\sigma$ and $\hat\sigma_n$ be the first of $\sigma_k$ such that $\sigma_k-\hat\sigma_{n-1}\geq\rho$. It is clear that $\hat\sigma_{[T/\rho]}\geq T\geq\eta$. {Hence, we can replace \eqref{eq:secondsum1} by}
\begin{equation}
\label{eq:secondsum2}
    \sum_{n=0}^{[T/\rho]-1}\E_{(x,y)}(\chi_{\{\hat\sigma_n<\eta\}}\E_{(X_{\hat\sigma_n}^\e,\xi_{\hat\sigma_n}^\e)}\sum_{k=0}^\infty\chi_{\{\sigma_k<\rho\}}[f(h(X_{\tau_{n+1}}^\e))-f(h(X_{\sigma_n}^\e))-\int_{\sigma_n}^{\tau_{n+1}}\mathcal Lf(h(X_t^\e))dt])
\end{equation}
and, by the strong Markov property,  the difference is no more than 
\begin{equation}
\label{eq:secondsum3}
    \sup_{(x,y)\in\gamma\times\mathbb T^m}\sup_{{\sigma'}\leq\rho}|\E_{(x,y)}\sum_{n=0}^\infty\chi_{\{\sigma_n<{\sigma'}\}}[f(h((X_{\tau_{n+1}}^\e))-f(h((X_{\sigma_n}^\e))-\int_{\sigma_n}^{\tau_{n+1}}\mathcal Lf(h(X_t^\e))dt]|,
\end{equation}where $\sigma'$ is a stopping time w.r.t. $\mathcal F^{X_\cdot^\e}_t$.
Both of them can be bounded by $O(\delta)$ due to Lemma~\ref{lem:aux_to_ori} and Corollary~\ref{cor:num_excursion}.
\end{proof}

\appendix
%----------------------------------------------------------------------------------------
\section{Derivatives of the action-angle cooridinates}
\label{sec:derivatives}
In this section, we carefully estimate the first and second derivatives of $q(x)$, $Q(h)$, $\phi(x)$, $A(h,\phi)$, and ${B_c}(h,\phi)$ in order to prove \eqref{eq:bounds}. 
Our main tool is the Morse lemma.
Note that we only need to verify the bounds near the separatrix, since the derivatives are uniformly bounded inside the domain.
For $x,y\in\mathbb R^2$, let $x\to y$ denote the line segment connecting $x$ and $y$. For $x,y\in\gamma(h)$ for certain $h$, let $x\xrightarrow{\gamma} y$ denote the piece of $\gamma(h)$ connecting $x$ to $y$ along the direction of $\nabla^\perp H$.
Recall that $q(x)=\int_{l(H(x))\xrightarrow{\gamma} x}\frac{1}{|\nabla H|}dl$. 
To start with, using the Morse lemma, one can compute that $q(x)\lesssim |\log H(x)|$ and $Q(h)\lesssim|\log h|$.
Then we make use of two special deterministic motions in the directions of $\nabla^\perp H$ and $\nabla H$ to calculate the first derivatives of $q(x)$ precisely:
\begin{equation}
    \begin{aligned}
        d\bm x_t&=\nabla^\perp H(\bm x_t)dt,\\
        d\bm y_t&=\nabla H(\bm y_t)dt.
    \end{aligned}
\end{equation}
% Figure environment removed
It follows that
\begin{equation}
\label{eqa:qxt}
    q(\bm x_t)
    %=\int_{l(H(\bm x_t))\xrightarrow{\gamma} \bm x_t}\frac{1}{|\nabla H|}dl=\int_{l(H(\bm x_0))\xrightarrow{\gamma} \bm x_0}\frac{1}{|\nabla H|}dl+\int_{\bm x_0\xrightarrow{\gamma} \bm x_t}\frac{1}{|\nabla H|}dl
    =q(\bm x_0)+t,
\end{equation}
\begin{equation}
\label{eqa:qyt}
    \begin{aligned}
        q(\bm y_t)
        =q(\bm y_0)+\int_{\partial D_t}\frac{\nabla H}{|\nabla H|^2}\cdot\bm ndl
        =q(\bm y_0)+\int_{D_t}\mathrm{div}(\frac{\nabla H}{|\nabla H|^2})dS,
    \end{aligned}
\end{equation}
where $D_t$ is the region bounded by $l$, trajectory of $\bm {y}_s$, $0\leq s\leq t$, $\gamma(H(\bm y_0))$, and $\gamma(H(\bm y_t))$, as shown in Figure~\ref{fig:coordinate}.
Thus, by differentiating \eqref{eqa:qxt} and \eqref{eqa:qyt} in $t$, we have the following equations:
\begin{equation}
\label{eqa:linear_sys}
    \begin{aligned}
        \nabla q(x)\cdot\nabla^\perp H(x)&=1,\\
        \nabla q(x)\cdot\nabla H(x)&=|\nabla H(x)|^2\int_{l(H(x))\xrightarrow{\gamma} x}\mathrm{div}(\frac{\nabla H}{|\nabla H|^2})\frac{1}{|\nabla H|}dl.
    \end{aligned}
\end{equation}
Therefore, with subscripts denoting the partial derivatives, by solving the linear system,
\begin{equation}
\label{eqa:q1q2}
    \begin{aligned}
        q'_1&=\frac{-H'_2}{{H'_1}^2+{H'_2}^2}+H'_1p,\\
        q'_2&=\frac{H'_1}{{H'_1}^2+{H'_2}^2}+H'_2p,
    \end{aligned}
\end{equation}
where $p(x)=\int_{l(H(x))\xrightarrow{\gamma} x}\mathrm{div}(\frac{\nabla H}{|\nabla H|^2})\frac{1}{|\nabla H|}dl$. Using the Morse lemma, one can compute $p(x)=O(1/H(x))$, since
\begin{equation}
    \left|\mathrm{div}(\frac{\nabla H}{|\nabla H|^2})\right|\lesssim\frac{1}{|\nabla H|^2}.
\end{equation}
Note that the non-degeneracy of the saddle point implies that $|H|\lesssim |\nabla H|^2$, and hence $\nabla q=O(|\nabla H|/H)$. The next step is to estimate $\nabla p$. For all $x,y$ close enough, with $D$ denoting the region bounded by $l$, $x\to y$, $\gamma(H(x))$, and $\gamma(H(y))$, we have
\begin{align*}
    |p(x)-p(y)|&=\left|\int_{\partial D}\mathrm{div}(\frac{\nabla H}{|\nabla H|^2})\frac{\nabla H}{|\nabla H|^2}\cdot \mathbf{n}dl-\int_{x\to y}\mathrm{div}(\frac{\nabla H}{|\nabla H|^2})\frac{\nabla H}{|\nabla H|^2}\cdot \mathbf{n}dl\right|\\
    &\leq\int_D \left|\mathrm{div}\left[\mathrm{div}(\frac{\nabla H}{|\nabla H|^2})\frac{\nabla H}{|\nabla H|^2}\right]\right|dS+\int_{x\to y}\left|\mathrm{div}(\frac{\nabla H}{|\nabla H|^2})\frac{1}{|\nabla H|}\right|dl\\
    &\lesssim\int_D \frac{1}{|\nabla H|^4}dS+\int_{x\to y}\frac{1}{|\nabla H|^3}dl.
\end{align*}
Then one can obtain $|\nabla p|\lesssim|\nabla H|/H^2$ by using the Morse lemma again and, as a result, $|q''_{ij}|\lesssim |\nabla H|^2/H^2$, $1\leq i,j\leq 2$. Similarly, we can estimate the derivatives of $Q(h)$. In fact, $Q'(h)=\int_{\gamma(h)}\mathrm{div}(\frac{\nabla H}{|\nabla H|^2})\frac{1}{|\nabla H|}dl=O(1/h)$ because of the estimate we had on $p(x)$. In addition,
\begin{equation}
    \left|Q''(h)\right|=\left|\int_{\gamma(h)}\mathrm{div}\left[\mathrm{div}(\frac{\nabla H}{|\nabla H|^2})\frac{\nabla H}{|\nabla H|^2}\right]\frac{1}{|\nabla H|}dl\right|\lesssim\int_{\gamma(h)}\frac{1}{|\nabla H|^5}dl=O(1/h^2).
\end{equation}
Since $\phi(x)=q(x)/Q(H(x))$, $|\nabla\phi|\lesssim|\nabla H|/H$ and $|\phi''_{ij}|\lesssim|\nabla H|^2/H^2$, $1\leq i,j\leq2$. Finally, we estimate the derivatives w.r.t. $h$ of a general function $F(h,\phi)=\Tilde F(x_1,x_2)$ with $\Tilde F$ having bounded first and second derivatives. By computing the inverse of the Jacobian of $(H,\phi)$ and using the first equation in \eqref{eqa:linear_sys},
\begin{equation}
\label{eqa:der_in_h}
    F'_h=\frac{\Tilde F'_1\phi'_2-\Tilde F'_2\phi'_1}{H'_1\phi'_2-H'_2\phi'_1}=(\Tilde F'_1\phi'_2-\Tilde F'_2\phi'_1)Q(H(x_1,x_2)).
\end{equation}
We deduce that $F'_h=O(|\log h|/h)$ and, using $F_h'$ instead of $F$ in {\eqref{eqa:der_in_h}, $F''_{hh}=O(|\log h|^2/h^3)$.} Similarly, one can obtain $F'_\phi=O(|\log h|)$ and $F''_{\phi\phi}=O(|\log h|/h)$.



%-----------------------------------------------------------------------------------------


\section{Exit from neighborhoods of critical points}
\label{sec:exitfromneighborhood}
In this section, we obtain estimates for the exit time from the neighborhoods of the critical points, including extremum points and saddle points. Recall the notation in Section~\ref{sec:averaging}: $O$ is a saddle point, $O'$ is an extremum point, $U$ is a domain bounded by the separatrix, and $\eta(h)=\inf\{t:|H(\tx_t^\e)-H(O')|=h\}$. Recall the function $u$ defined in \eqref{eq:u} and let us define 
\begin{equation}
    \label{eqb:nondegenerate-diffusion}
    \bm A(x)=\int_{\mathbb T^m}\nabla_y u(x,y)\sigma(y)\sigma(y)^\mathsf{T}\nabla_y u(x,y)^\mathsf{T}d\mu(y).
\end{equation}
Using assumption \hyperlink{H4'}{\textit{(H4$'$)}}, one can see that $\bm A(x)$ is positive-definite. 
Indeed, if there exist a point $x$ and a vector $v\not=0$ such that $v^\mathsf{T}\bm A(x)v=0$, then, since $\sigma\sigma^\mathsf{T}$ is positive-definite, $v^\mathsf{T}\nabla_y u(x,y)=0$ for all $y\in\mathbb T^m$.
Namely, $v^\mathsf{T} u(x,y)$ is constant, and $v^\mathsf{T}(b(x,y)-\bar b(x))=L(v^\mathsf{T}u(x,y))=0$ on $\mathbb T^m$, which contradicts with \hyperlink{H4'}{\textit{(H4$'$)}}. 
\begin{lemma}
\label{lemb:non-zero-drift}
    Recall the definition of $B_c(x)$ in \eqref{eq:definition_of_operator_AB_x}. If $O'$ is a minimum point, then $B_c(O')>0$; if $O'$ is a maximum point, then $B_c(O')<0$.
\end{lemma}
\begin{proof}
We prove the result in the case of minimum point. The other case can be treated similarly. Since $O'$ is a critical point and $Lu(x,y)=-(b(x,y)-\bar b(x))$, we have $\nabla H(O')=0$ and 
    \begin{align*}
        {B_c}(O')&=\int_{\mathbb T^m}[\nabla_xu_h(O',y)b(O',y)+\nabla_y u_h(O',y)c(x,y)]d\mu(y)\\
        &=\int_{\mathbb T^m}u(O',y)^\mathsf{T}\nabla^2 H(O')b(O',y)d\mu(y)\\
        &=-\int_{\mathbb T^m}u(O',y)^\mathsf{T}\nabla^2 H(O')Lu(O',y)d\mu(y)\\
        &=-\sum_{1\leq i,j\leq2}\frac{\partial^2}{\partial x_i\partial x_j}H(O')\int_{\mathbb T^m}u_i(O',y)Lu_j(O',y)d\mu(y)\\
        &=-\sum_{1\leq i,j\leq2}\frac{\partial^2}{\partial x_i\partial x_j}H(O')\int_{\mathbb T^m}\frac{1}{2}(u_iLu_j+u_jLu_i)(O',y)d\mu(y)\\
        &=\frac{1}{2}\sum_{1\leq i,j\leq2}\frac{\partial^2}{\partial x_i\partial x_j}H(O')\left(\bm A(O')_{i,j}-\int_{\mathbb T^m}L(u_iu_j)d\mu(y)\right)\\
        &=\frac{1}{2}\sum_{1\leq i,j\leq2}\frac{\partial^2}{\partial x_i\partial x_j}H(O')\bm A(O')_{i,j}.
\end{align*}
This is positive since both $\nabla^2 H$ and $\bm A$ are positive definite at $O'$.

\end{proof}
\begin{lemma}
\label{lem:near_extremum}
    For each $\kappa>0$, there exists $\delta>0$ such that
    \begin{equation}
    \label{eq:near_extremum}
        \E_{(x,y)}\eta(\delta)\leq\kappa
    \end{equation}
    for all $x$ satisfying $|H(x)-H(O')|<\delta$, $y\in\mathbb T^m$, and $\e$ sufficiently small.
\end{lemma}
\begin{proof}
Without loss of generality, we assume $O'$ to be a minimum point. Similarly to \eqref{eq:H}, we apply Ito's formula to $u_h(\tx^\e_{\eta(\delta)\wedge1},\txi^\e_{\eta(\delta)\wedge1})$ and we obtain
\begin{align*}
        H(\tx^\e_{\eta(\delta)\wedge1})&=H(x)+\int_0^{\eta(\delta)\wedge1} \nabla_y u_h(\tx_s^\e,\txi_s^\e)^{\mathsf T}\sigma(\txi_s^\e)dW_s+\e(u_h(x,y)-u_h(\tx_{\eta(\delta)\wedge1}^\e,\txi_{\eta(\delta)\wedge1}^\e))\\
    &\quad+\int_0^{\eta(\delta)\wedge1}[\nabla_x u_h(\tx_s^\e,\txi_s^\e)\cdot b(\tx_s^\e,\txi_s^\e)+\nabla_y u_h(\tx_s^\e,\txi_s^\e)\cdot c(\tx_s^\e,\txi_s^\e)]ds.
    \end{align*}
By taking the expectation on both sides and using Corollary~\ref{cor:avg}, we obtain
\begin{equation}
\label{eqb:exit_from_extremum}
    \E_{(x,y)}\int_0^{\eta(\delta)\wedge1}B_c(\tx_s^\e)ds<\delta+O(\e).
\end{equation}
Due to Lemma~\ref{lemb:non-zero-drift}, $B_c(O')>0$. Hence we can choose $\delta$ to be small enough such that $B_c(\tx_s^\e)>B_c(O')/2>0$ before $\eta(\delta)$. Thus, it follows from \eqref{eqb:exit_from_extremum} that, for all $\e$ sufficiently small,
\[\E_{(x,y)}(\eta(\delta)\wedge1)\leq 4\delta/B_c(O').\]
Then $\Prob_{(x,y)}(\eta(\delta)\geq1)\leq4\delta/B_c(O')$ for all $x$ satisfying $|H(x)-H(O')|<\delta$, $y\in\mathbb T^m$ by Markov's inequality. Then, one can obtain the desired result using the Markov property by the fact that $\E_{(x,y)}\eta(\delta)\leq \E_{(x,y)}(\eta(\delta)\wedge1)+\E_{(x,y)}(\eta(\delta)\chi_{\{\eta(\delta)>1\}})$.
\end{proof}



%--------------------------------------------------------------------------------------------------

\begin{proposition}
\label{prop:exit_time_from_separatrix}
    Let $0<\alpha<1/2$, $V^\e=\{x:|H(x)-H(O)|<\e^\alpha\}$, and $\tau=\inf\{t:\tx_t^\e\not\in V^\e\}$. Then, uniformly in $0<\alpha<1/2$ and $(x,y)\in V^\e\times\mathbb T^m$, as $\e\downarrow0$,
    \begin{equation}
        \E_{(x,y)}\tau=O(\e^{2\alpha}|\log\e|).
    \end{equation}
\end{proposition} 
To prove Proposition~\ref{prop:exit_time_from_separatrix}, it is more convenient to consider the process $(\tbx_t^\e,\tbxi_t^\e)$, define the stopping time $\bm\tau=\inf\{t:\tbx_t^\e\not\in V^\e\}$, and then prove that $\E_{(x,y)}\bm\tau=O(\e^{2\alpha-1}|\log\e|)$.
We need careful analysis of the behavior of the processes near the saddle point and away from the saddle point. 
The latter is easier to deal with since there is no degeneracy, while the former needs us to, again, use the Morse Lemma to make concrete computations. 
For simplicity, we prove the result in the case of $(\bx_t^\e,\bxi_t^\e)$ without the additional drift $c(x,y)$ since it can be seen in the proof that the additional terms induced by $c(x,y)$ are always relatively small. 
We prove that there exist two neighborhoods, $D_1\subset D_2$, of $O$ (as shown in Figure~\ref{fig:D1D2}), such that, in $V^\e$, it takes the process $O(|\log\e|)$ time to escape from $D_2$, and $O(1)$ time to return to $D_1$.
Since $x\in V^\e$ is two-dimensional, we denote $x=(p,q)$. To avoid confusion brought by convoluted formulas, we assume the saddle point is the origin and the Hamiltonian $H(x)=pq$ in a small neighborhood of $O$.
    This assumption is not restrictive because, as in the proof of Lemma~\ref{lem:throughsaddlepoint}, we can use the Morse Lemma and perform a random change of time with the multiplier bounded from below and above, which will not change the order of the expected time. 
    For $r>0$, we denote $D_r$ to be the region in $V^\e$ with $|p|\leq r$ and $|q|\leq r$, $(\partial D_r)_{\textrm{in}}=\{|p|=r\}\cap V^\e$, $(\partial D_r)_{\textrm{out}}=\{|q|=r\}\cap V^\e$, and choose $r_3>0$ small enough that $H(x)=pq$ in $D_{r_3}$.
% Figure environment removed
\begin{lemma}
\label{lem:near}
    There exist $r_1,r_2>0$ such that, uniformly in $(x,y)\in D_{r_1}\times\mathbb T^m$, as $\e\downarrow0$,
    \begin{equation}
        \E_{(x,y)}\bar\tau=O(|\log\e|),
    \end{equation}
    where $\bar\tau=\inf\{t:\bx_t^\e\not\in D_{r_2}\}$.
\end{lemma}
\begin{proof}
    
    %Then we choose two smaller regions $D_{r_1}$ and $D_{r_2}$ with $r_1<r_2$, and we will study the exit time and exit location from $D_{r_1}$ to $D_{r_2}$.
    We denote $\eta_t^\e=(\bx_t^\e)_2$ and study the behavior of $\eta_t^\e$ inside $B(0,r_3)$. All the computations below concern $\bx_t^\e$ before leaving $D_{r_3}$. As in \eqref{eq:slowx}, we can write the equation for $\eta_t^\e$,
    \begin{equation}
    \begin{aligned}
    \label{eq:etat}
        {\eta}_t^\e&=q+\int_0^t {\eta}_s^\e ds+\sqrt{\e}\int_0^t\nabla_y u_2(\bx_s^\e,\bxi_s^\e)^{\mathsf T}\sigma(\bxi_s^\e)dW_s\\
        &\quad+\e\int_0^t\nabla_x u_2(\bx_s^\e,\bxi_s^\e)\cdot b(\bx_s^\e,\bxi_s^\e)ds+\e(u_2(x,y)-u_2(\bx_t^\e,\bxi_t^\e)).
    \end{aligned}
    \end{equation}
    Introduce $\hat\eta_t^\e$, which is close to $\eta_t^\e$:
    \begin{equation}
        \label{eq:eta}
        \hat\eta_t^\e = q+\int_0^t {\eta}_s^\e ds+\sqrt{\e}\int_0^t\nabla_y u_2(\bx_s^\e,\bxi_s^\e)^{\mathsf T}\sigma(\bxi_s^\e)dW_s+\e\int_0^t\nabla_x u_2(\bx_s^\e,\bxi_s^\e)\cdot b(\bx_s^\e,\bxi_s^\e)ds.
    \end{equation}
    Let $F(q)=\int_0^q e^{-z^2}\int_0^{z}e^{w^2}dwdz$, which satisfies $ F(q)\sim \frac{1}{2}\log q$, as $p\to\infty$, and has bounded derivatives up to the third order.
    Then we can choose a large $C>0$, such that $|F'|,|F''|,|F'''|$, $|u(x,y)|,|\nabla u(x,y)|,|\nabla^2 u(x,y)|$ are bounded by $C$.
    Recall from \eqref{eqb:nondegenerate-diffusion} that the vector-valued function $\nabla_y u_2(x,y)^{\mathsf T}\sigma(y)$ has non-degenerate average w.r.t. $\mu$, in the sense that $\bm A_{22}(x)>0$. 
    Let $A_0=\bm A_{22}(O)>0$ and let $0<r_1<r_2<r_3$ be such that $A_0(1-1/(2C))<\bm A_{22}(x)<A_0(1+1/(2C))$ in $D_{r_2}$, as shown in Figure~\ref{fig:MorseLemma}.
    Let $\bar r_2=\frac{r_3+r_2}{2}$. Define function $f(q)$ (that depends on $\e$) and compute its derivatives:
    \begin{equation}
    \label{eqb:derivatives_f}
    \begin{aligned}
        f(q)=2(F(\frac{\bar r_2}{\sqrt{A_0\e}})-F(\frac{q}{\sqrt{A_0\e}})),~~~&f'(q)=-\frac{2}{\sqrt{A_0\e}}F'(\frac{q}{\sqrt{A_0\e}}),\\
        f''(q)=-\frac{2}{A_0\e}F''(\frac{q}{\sqrt{A_0\e}}),~~~~~~~~~~~~~~&f'''(q)=-\frac{2}{(\sqrt{A_0\e})^3}F'''(\frac{q}{\sqrt{A_0\e}}).
    \end{aligned}
    \end{equation}
    Furthermore, $f$ satisfies the differential equations:
    \begin{equation}
    \label{eqb:equation_f}
        \begin{cases}
        \frac{1}{2}A_0\e f''+qf'=-1\\
        f(-\bar r_2)=f(\bar r_2)=0
        \end{cases}.
    \end{equation}
    By Lemma~\ref{lem:solution}, there is a function $v_2(x,y)$ that is bounded together with its derivatives such that
    \begin{equation}
        Lv_2(x,y)=\left|\nabla_y u_2(x,y)\sigma(y)\right|^2-\bm A_{22}(x).
    \end{equation}
    where $L$ is the generator of the process $\xi_t$ (see \eqref{eqb:def_operator_L}). 
    %Let $M$ be a large constant such that up to the tenth powers of the terms $$\sup_{x,y}|g(x,y)|,\sup_{x,y}|\nabla g(x,y)|,\sup_{x,y}|b(x,y)|,A_0, 1/A_0, C, 10$$ are bounded by $M$. 
    Since $|\eta_t^\e-\hat\eta_t^\e|=O(\e)$ and $\bar r_2>r_2$, we can apply Ito's formula to $v_2(\bx_t^\e,\bxi_t^\e)f''(\eta_t^\e)$ for $0\leq t\leq\bar\tau$:
    \begin{equation}
    \label{eq:avg_step_eg}
        \begin{aligned}
            v_2(\bx_t^\e,\bxi_t^\e)f''(\eta_t^\e)=&v_2(x,y)f''(q)+\int_0^t\nabla_x(v_2(\bx_s^\e,\bxi_s^\e)f''(\eta_s^\e))\cdot b(\bx_s^\e,\bxi_s^\e)ds\\
            &+\frac{1}{\e}\int_0^t Lv_2(\bx_s^\e,\bxi_s^\e)f''(\eta_s^\e)ds+\frac{1}{\sqrt{\e}}\int_0^t\nabla_yv_2(\bx_s^\e,\bxi_s^\e)^{\mathsf T}f''(\eta_s^\e)\sigma(\bxi_s^\e)dW_s.
        \end{aligned}
    \end{equation}
    Hence it follows that
    \begin{equation}
    \label{eqb:averaged_a22}
    \begin{aligned}
        &\int_0^t (\left|\nabla_y u_2(\bx_s^\e,\bxi_s^\e)\sigma(\bxi_s^\e)\right|^2-\bm A_{22}(\bx_s^\e))f''(\eta_s^\e)ds\\
        &\quad=\e\left(v_2(\bx_t^\e,\bxi_t^\e)f''(\eta_t^\e)-v_2(x,y)f''(q)\right) -\e\int_0^t\nabla_x(v_2(\bx_s^\e,\bxi_s^\e)f''(\eta_s^\e))\cdot b(\bx_s^\e,\bxi_s^\e)ds\\
        &\quad\quad -\sqrt{\e}\int_0^tf''(\eta_s^\e)\nabla_yv_2(\bx_s^\e,\bxi_s^\e)^{\mathsf T}\sigma(\bxi_s^\e)dW_s\\
        &\quad= O(1)+O(\frac{1}{\sqrt{\e}})\cdot t-\sqrt{\e}\int_0^tf''(\eta_s^\e)\nabla_yv_2(\bx_s^\e,\bxi_s^\e)^{\mathsf T}\sigma(\bxi_s^\e)dW_s.
    \end{aligned}
    \end{equation}
    Now apply Ito's formula to $f(\hat\eta_t^\e)$ for $0\leq t\leq \bar\tau$:
    \begin{align*}
        f(\hat\eta_t^\e)&=f(q)+\int_0^tf'(\hat\eta_s^\e)\eta_s^\e ds+\frac{\e}{2}\int_0^t f''(\hat\eta_s^\e)|\nabla_y u_2(\bx_s^\e,\bxi_s^\e)^{\mathsf T}\sigma(\bxi_s^\e)|^2ds\\
        &\quad+\e\int_0^t f'(\hat\eta_s^\e)\nabla_x u_2(\bx_s^\e,\bxi_s^\e)\cdot b(\bx_s^\e,\bxi_s^\e)ds+\sqrt{\e}\int_0^tf'(\hat\eta_s^\e)\nabla_y u_2(\bx_s^\e,\bxi_s^\e)^{\mathsf T}\sigma(\bxi_s^\e)dW_s\\
        &= f(q)+\int_0^tf'(\hat\eta_s^\e)\hat\eta_s^\e ds+O(\sqrt{\e})\cdot t+\frac{\e}{2}\int_0^t f''(\hat\eta_s^\e)\bm A_{22}(\bx_s^\e)ds\\
        &\quad+\frac{\e}{2}\int_0^t (|\nabla_y u_2(\bx_s^\e,\bxi_s^\e)^{\mathsf T}\sigma(\bxi_s^\e)|^2-\bm A_{22}(\bx_s^\e))f''(\eta_s^\e)ds+O(\sqrt{\e})\cdot t\\
        &\quad+\sqrt{\e}\int_0^tf'(\hat\eta_s^\e)\nabla_y u_2(\bx_s^\e,\bxi_s^\e)^{\mathsf T}\sigma(\bxi_t^\e)dW_s\\
        &= f(q)+\int_0^t [f'(\hat\eta_s^\e)\hat\eta_s^\e+\frac{A_0\e}{2}f''(\hat\eta_s^\e)]ds+\frac{\e}{2}\int_0^t f''(\hat\eta_s^\e)(\bm A_{22}(\bx_s^\e)-A_0)ds\\
        &\quad+\frac{\e}{2}( O(1)+O(\frac{1}{\sqrt{\e}})\cdot t-\sqrt{\e}\int_0^tf''(\eta_s^\e)\nabla_yv_2(\bx_s^\e,\bxi_s^\e)^{\mathsf T}\sigma(\bxi_s^\e)dW_s)+O(\sqrt{\e})\cdot t\\
        &\quad+\sqrt{\e}\int_0^tf'(\hat\eta_s^\e)\nabla_y u_2(\bx_s^\e,\bxi_s^\e)^{\mathsf T}\sigma(\bxi_s^\e)dW_s\\
        &\leq f(q)-t+\frac{t}{2}+O(\e)+O(\sqrt{\e})\cdot t-\frac{\sqrt{\e^3}}{2}\int_0^tf''(\eta_s^\e)\nabla_yv_2(\bx_s^\e,\bxi_s^\e)^{\mathsf T}\sigma(\bxi_s^\e)dW_s\\
        &\quad+\sqrt{\e}\int_0^tf'(\hat\eta_s^\e)\nabla_y u_2(\bx_s^\e,\bxi_s^\e)^{\mathsf T}\sigma(\bxi_s^\e)dW_s,
    \end{align*}where the equalities follow from \eqref{eqb:derivatives_f} and \eqref{eqb:averaged_a22}, and the last inequality holds since $f$ solves \eqref{eqb:equation_f} and $|\bm A_{22}(\bx_s^\e)-A_0|<A_0/(2C)$.
    Let $\Tilde \tau=\bar\tau\wedge 1/\e$. Then $\hat\eta_{\Tilde\tau}^\e\in[-\bar r_2,\bar r_2]$ because $|\hat\eta_{\Tilde\tau}^\e-\eta_{\Tilde\tau}^\e|=O(\e)$. The previous calculation reduces to
    \begin{align*}
        f(\hat\eta_{\Tilde\tau}^\e)&\leq f(q)-\Tilde\tau/2+O(\e)+O(\sqrt\e)\cdot\Tilde\tau-\frac{\sqrt{\e^3}}{2}\int_0^{\Tilde\tau}f''(\eta_s^\e)\nabla_yv_2(\bx_s^\e,\bxi_s^\e)^{\mathsf T}\sigma(\bxi_s^\e)dW_s\\
        &\quad+\sqrt{\e}\int_0^{\Tilde{\tau}}f'(\hat\eta_s^\e)\nabla_y u_2(\bx_s^\e,\bxi_s^\e)^{\mathsf T}\sigma(\bxi_s^\e)dW_s.
    \end{align*}
    By taking the expectation, we have for all $x\in D_{r_2}$, $y\in\mathbb T^m$, and $\e$ small enough
    \begin{equation}
        \E_{(x,y)}\Tilde{\tau}\leq 5\sup_{-\bar r_2\leq q'\leq\bar r_2}f(q')=O(|\log\e|).
    \end{equation}
    Then, by Markov's inequality and the Markov property, we obtain that $\E_{(x,y)}\bar\tau=O(|\log\e|)$.
\end{proof}
\begin{lemma}
\label{lem:exit_location}
    Let $r_1,r_2,\bar\tau$ be defined as in Lemma~\ref{lem:near}. Then, uniformly in $(x,y)\in D_{r_1}\times\mathbb T^m$,
    \begin{equation}
        \Prob_{(x,y)}(\bx_{\bar\tau}^\e\in(\partial D_{r_2})_{\mathrm{in}})\to0~\text{as }\e\downarrow0.
    \end{equation}
\end{lemma}
\begin{proof}
    Again, we denote $x=(p,q)$ and, for simplicity, we assume that the saddle point is the origin and that $H(x) = pq$ in a small neighborhood of $O$.
    We extend the function $b(x,y)$ in the vertical direction in such a way that it is bounded together with its partial derivatives and the first component of $\bar b(x)$ is $-p$ in the region $\{x:|p|\leq r_2\}$.
    We denote $\zeta_t^\e=(\bx_t^\e)_1$ and show that it takes significantly longer than $|\log\e|$ for $\zeta_t^\e$ to reach $\pm r_2$, hence it is unlikely for $\bx_t^\e$ to exit $D_{r_2}$ through ${(\partial D_{r_2})}_{\mathrm{in}}$. All the computations below concern $\bx_t^\e$ before $\zeta_t^\e$ reaches $\pm r_2$. As in \eqref{eq:slowx}, we can write the equation for $\zeta_t^\e$:
    \begin{equation}
    \begin{aligned}
    \label{eq:zetat}
        {\zeta}_t^\e&=p-\int_0^t {\zeta}_s^\e ds+\sqrt{\e}\int_0^t\nabla_y u_1(\bx_s^\e,\bxi_s^\e)^{\mathsf T}\sigma(\bxi_s^\e)dW_s\\
        &\quad+\e\int_0^t\nabla_x u_1(\bx_s^\e,\bxi_s^\e)\cdot b(\bx_s^\e,\bxi_s^\e)ds+\e(u_1(x,y)-u_1(\bx_t^\e,\bxi_t^\e)).
    \end{aligned}
    \end{equation}
    Introduce $\hat\zeta_t^\e$, which is close to $\zeta_t^\e$:
    \begin{equation}
        \label{eq:zeta}
        \hat\zeta_t^\e = p-\int_0^t {\zeta}_s^\e ds+\sqrt{\e}\int_0^t\nabla_y u_1(\bx_s^\e,\bxi_s^\e)^{\mathsf T}\sigma(\bxi_s^\e)dW_s+\e\int_0^t\nabla_x u_1(\bx_s^\e,\bxi_s^\e)\cdot b(\bx_s^\e,\bxi_s^\e)ds.
    \end{equation}
    Since $b(x,y)$ and its partial derivatives are bounded in $\{x:|p|\leq r_2\}\times\mathbb T^m$, we can choose $C>0$ such that
    \begin{equation}
    \label{eqb:supremums}
        \sup_{x:|p|\leq r_2,y\in\mathbb T^m}\left(|\nabla_y u_1(x,y)^{\mathsf T}\sigma(y)|^2\vee2|u_1(x,y)|\vee|\nabla_x u_1(x,y)\cdot b(x,y)|\right)<C/2.
    \end{equation}
    Let us define $\bar r_2=\frac{r_1+r_2}{2}$, $\hat{\tau}_2=\inf\{t:|\hat\zeta_t^\e|>\bar r_2\}$, and function $f(p)=\exp(p^2/(C\e))$. Then it follows that
    \begin{equation}
    \label{eq:ode_exit_loc}
        \frac{C\e}{2}f''-pf'-f=0.
    \end{equation}
    Note that $|\zeta_t^\e|\leq r_2$ for $0\leq t\leq\hat\tau_2$ since $|\zeta_t^\e-\hat\zeta_t^\e|\leq C\e/2$. Apply Ito's formula to $\exp(-t/2)f(\hat\zeta_t^\e)$ for $0\leq t\leq\hat\tau_2$ and obtain using \eqref{eqb:supremums}:
    \begin{align*}
        e^{-t/2}f(\hat\zeta_t^\e)&=f(p)-\frac{1}{2}\int_0^t e^{-s/2}f(\hat\zeta_s^\e)ds-\int_0^t e^{-s/2}f'(\hat\zeta_s^\e)\zeta_s^\e ds\\
        &\quad+\e\int_0^t e^{-s/2} f'(\hat\zeta_s^\e)\nabla_x u_1(\bx_s^\e,\bxi_s^\e)\cdot b(\bx_s^\e,\bxi_s^\e)ds\\
        &\quad+\frac{\e}{2}\int_0^t e^{-s/2}f''(\hat\zeta_s^\e)|\nabla_y u_1(\bx_s^\e,\bxi_s^\e)^{\mathsf T}\sigma(\bxi_s^\e)|^2ds\\
        &\quad+\sqrt{\e}\int_0^t e^{-s/2}f'(\hat\zeta_s^\e)\nabla_y u_1(\bx_s^\e,\bxi_s^\e)^{\mathsf T}\sigma(\bxi_s^\e)dW_s\\
        &= f(p)-\frac{1}{2}\int_0^t e^{-s/2}f(\hat\zeta_s^\e)ds\\
        &\quad+\int_0^t e^{-s/2}f'(\hat\zeta_s^\e)\left(-\frac{1}{2}\hat\zeta_s^\e+\left[(\hat\zeta_s^\e-\zeta_s^\e)+\e \nabla_x u_1(\bx_s^\e,\bxi_s^\e)\cdot b(\bx_s^\e,\bxi_s^\e)-\frac{1}{2}\hat\zeta_s^\e\right]\right) ds\\
        &\quad+\frac{\e}{2}\int_0^t e^{-s/2}f''(\hat\zeta_s^\e)|\nabla_y u_1(\bx_s^\e,\bxi_s^\e)^{\mathsf T}\sigma(\bxi_s^\e)|^2ds\\
        &\quad+\sqrt{\e}\int_0^t e^{-s/2}f'(\hat\zeta_s^\e)\nabla_y u_1(\bx_s^\e,\bxi_s^\e)^{\mathsf T}\sigma(\bxi_s^\e)dW_s\\
        &\leq f(p)-\frac{1}{2}\int_0^t e^{-s/2}f(\hat\zeta_s^\e)ds-\frac{1}{2}\int_0^t e^{-s/2}f'(\hat\zeta_s^\e)\hat\zeta_s^\e ds+\frac{C\e}{4}\int_0^t e^{-s/2}f''(\hat\zeta_s^\e)ds\\
        &\quad+\int_0^t e^{-s/2}f'(\hat\zeta_s^\e)\left[(\hat\zeta_s^\e-\zeta_s^\e)+\e \nabla_x u_1(\bx_s^\e,\bxi_s^\e)\cdot b(\bx_s^\e,\bxi_s^\e)-\frac{1}{2}\hat\zeta_s^\e\right] ds\\
        &\quad+\sqrt{\e}\int_0^t e^{-s/2}f'(\hat\zeta_s^\e)\nabla_y u_1(\bx_s^\e,\bxi_s^\e)^{\mathsf T}\sigma(\bxi_s^\e)dW_s\\
        &\leq  f(p)+18C\e(1-e^{-t/2})+\sqrt{\e}\int_0^t e^{-s/2}f'(\hat\zeta_s^\e)\nabla_y u_1(\bx_s^\e,\bxi_s^\e)^{\mathsf T}\sigma(\bxi_s^\e)dW_s.
    \end{align*}
    The last inequality follows from \eqref{eq:ode_exit_loc} and the fact that the integrand on the second line is either negative, when $|\hat\zeta_s^\e|\geq 2C\e$, or small and bounded by $9C\e e^{-s/2}$, when $|\hat\zeta_s^\e|\leq 2C\e$. By replacing $t$ by the stopping time $\hat\tau_2$ and taking expectation, we obtain
    \begin{equation}
        \E_{(x,y)} e^{-\hat\tau_2/2}\leq 2 e^{(r_1^2-{\bar r_2}^2)/(C\e)}.
    \end{equation}
    Let $\bar{\tau}_2=\inf\{t:|\zeta_t^\e|>r_2\}$. Then, since $|\zeta_t^\e-\hat\zeta_t^\e|\leq C\e/2$, it follows that
    \begin{equation}
        \Prob_{(x,y)}(\bar{\tau}_2<|\log\e|/\sqrt{\e})\leq \Prob_{(x,y)}(\hat{\tau}_2<|\log\e|/\sqrt{\e})\leq 2\exp(\frac{r_1^2-{\bar r_2}^2}{C\e}+\frac{|\log\e|}{2\sqrt{\e}})\to 0,
    \end{equation}
    as $\e\downarrow0$. However, by Lemma~\ref{lem:near} and Markov's inequality, we have
    \begin{equation}
        \Prob_{(x,y)}(\bar\tau>|\log\e|/\sqrt{\e})\to0,
    \end{equation}
    as $\e\downarrow0$. Thus, the desired result follows.
\end{proof}
\iffalse
\begin{remark}
Although we simply assumed that $H=x_1x_2$, the result can be generalized without much more effort. For instance, in Lemma~\ref{lem:near}, instead of $\int_0^t f'(\eta_s^\e)\Tilde\eta_s^\e ds$, we will have $\int_0^t f'(\eta_s^\e)c(\Tilde\zeta_s^\e,\Tilde\eta_s^\e)\Tilde\eta_s^\e ds$ with function $c(p,q)$ bounded both from below (by constant $b$) and above. Then we can modify our function $f$ such that it is a solution to
\begin{equation}
        \begin{cases}
        \frac{B\e}{2}f''+bqf'=-1\\
        f(-\bar r_2)=f(\bar r_2)=0.
        \end{cases}
    \end{equation}
Similar modifications can be applied to Lemma~\ref{lem:exit_location} to generalize the result. Namely, instead of $\int_0^t e^{-\lambda s}f'(\zeta_s^\e)\Tilde\zeta_s^\e ds$, we will have $$\int_0^t e^{-\lambda s}f'(\zeta_s^\e)c(\Tilde\zeta_s^\e,\Tilde\eta_s^\e)\Tilde\zeta_s^\e ds=\int_0^t e^{-\lambda s}f'(\zeta_s^\e)[\frac{b}{2}\zeta_s^e+c(\Tilde\zeta_s^\e,\Tilde\eta_s^\e)(\Tilde\zeta_s^\e-\zeta_s^\e)+(c(\Tilde\zeta_s^\e,\Tilde\eta_s^\e)-\frac{b}{2})\zeta_s^\e] ds.$$ We modify $f$ by replacing $C$ with $2C/b$ and change the exponent from $-t/2$ to $-bt/2$.
\end{remark}
\fi
\begin{lemma}
    \label{lem:away}
    Let $\bar{\bar\tau}=\inf\{t:\bx_t^\e\in D_{r_1}\}\wedge\bm\tau$. Then there exists $a>0$ such that, uniformly in $(x,y)\in(\partial {D_{r_2}})_{\mathrm{out}}\times\mathbb T^m$,
    \begin{equation}
    \label{eq:away_non_deg}
        \Prob_{(x,y)}(\bar{\bar\tau}<\bm\tau,\int_0^{\bar{\bar\tau}}|\nabla_y u_h(\bx_s^\e,\bxi_s^\e)^{\mathsf T}\sigma(\bxi_s^\e)|^2ds<a)\to 0~\text{as }\e\to0.
    \end{equation}
    Furthermore, $\E_{(x,y)}\bar{\bar\tau}$ is bounded uniformly in $(x,y)\in(\partial {D_{r_2}})_{\mathrm{out}}\times\mathbb T^m$.
\end{lemma}
\begin{proof}
Let $\hat t>0$ and $\check t>0$ be the lower bound and the upper bound of time spent by $\bm x_t$ to get from $(\partial {D_{r_2}})_{\mathrm{out}}$ to $D_{r_1}$, respectively. Then, similarly to \eqref{eq5:positive_variance_one_rotation}, there exists $a>0$ such that
\begin{equation}
    \Prob_{(x,y)}\left(\int_0^{\hat t/2} |\nabla_y u_h(\bx_s^\e,\bxi_s^\e)^{\mathsf T}\sigma(\bxi_s^\e))|^2ds>a,\sup_{0\leq t\leq2\hat t}|\bx_t^\e-\bm x_t|\leq\e^{\frac{1+2\alpha}{4}}\right)\to1.
\end{equation}
Hence
\begin{align*}
    &\Prob_{(x,y)}\left(\bar{\bar\tau}<\bm\tau,\int_0^{\bar{\bar\tau}}|\nabla_y u_h(\bx_s^\e,\bxi_s^\e)^{\mathsf T}\sigma(\bxi_s^\e)|^2ds<a\right)\\
    &\leq\Prob_{(x,y)}\left(\hat t/2\leq\bar{\bar\tau}<\bm\tau,\int_0^{\bar{\bar\tau}}|\nabla_y u_h(\bx_s^\e,\bxi_s^\e)^{\mathsf T}\sigma(\bxi_s^\e)|^2ds<a\right)+\Prob_{(x,y)}(\bar{\bar\tau}<\bm\tau,\bar{\bar\tau}<\hat t/2)\\
    &\leq\Prob_{(x,y)}\left(\int_0^{\hat t/2}|\nabla_y u_h(\bx_s^\e,\bxi_s^\e)^{\mathsf T}\sigma(\bxi_s^\e)|^2ds<a\right)+\Prob_{(x,y)}\left(\sup_{0\leq t\leq2\hat t}|\bx_t^\e-\bm x_t|>\e^{\frac{1+2\alpha}{4}}\right)\\
    &\to0.
\end{align*}
Similarly, it is easy to see that $\Prob_{(x,y)}(\bar{\bar\tau}>2\check t)<\Prob_{(x,y)}\left(\sup_{0\leq t\leq2\check t}|\bx_t^\e-\bm x_t|>\e^{\frac{1+2\alpha}{4}}\right)\to0$, and the desired result follows from the Markov property.
\end{proof}


\begin{proof}[Proof of Proposition~\ref{prop:exit_time_from_separatrix}]
    As in~\eqref{eq:H}:
    \begin{align*}
        H(\bx_t^\e) &= H(x)+\e\int_0^t \nabla_x u_h(\bx_s^\e,\bxi_s^\e)\cdot b(\bx_s^\e,\bxi_s^\e)ds\\
    &\quad+\sqrt{\e}\int_0^t \nabla_y u_h(\bx_s^\e,\bxi_s^\e)^{\mathsf T}\sigma(\bxi_s^\e)dW_s+\e(u_h(x,y)-u_h(\bx_t^\e,\bxi_t^\e)).
    \end{align*}
    The change in $H(\bx_t^\e)$ is mainly due to the stochastic integral while the other terms are of order $O(\e)$ and $O(t\cdot\e)$ and can be controlled. For each $t(\e)>0$,
    \begin{equation}
    \label{eq:set}
    \begin{aligned}
        \{\bm\tau<t(\e)\}\supset&\left\{\sup_{[0,t(\e)]}\left|\sqrt{\e}\int_0^{t(\e)} \nabla_y u_h(\bx_s^\e,\bxi_s^\e)^{\mathsf T}\sigma(\bxi_s^\e)dW_s\right|>3\e^\alpha\right\}\\
        &\quad\bigcap\left\{\e\int_0^{t(\e)} |\nabla_x u_h(\bx_s^\e,\bxi_s^\e)\cdot b(\bx_s^\e,\bxi_s^\e)|ds<\e^\alpha/2\right\}.
    \end{aligned}
    \end{equation}
    Note that if we choose $t(\e)=o(\e^{\alpha-1})$, then the second event is always true. Now we recursively define stopping times:
    \begin{align*}
        \theta^1_0&=0,\\
        \theta^2_k&=\inf\{t\geq\theta^1_{k-1}:\bx_t^\e\in\partial D_{r_2}\}\wedge\bm\tau,\\
        \theta^1_k&=\inf\{t\geq\theta^2_{k}:\bx_t^\e\in\partial D_{r_1}\}\wedge\bm\tau.
    \end{align*}
    We denote $D(x,y)=\nabla_y u_h(x,y)^{\mathsf T}\sigma(y)$. Note that once the process leaves $V^\e$, the stopping times stay constant afterwards. The main idea of the proof is to show that after a sufficiently long time $t(\e)$, the stochastic integral will accumulate enough variance to exit from $V^\e$. Let us bound the probability of variance being small:
    \begin{equation}
    \label{eq:top_inequality}
        \begin{aligned}
            &\Prob_{(x,y)}\left(\bm\tau\geq t(\e),\int_0^{t(\e)}|D(\bx_s^\e,\bxi_s^\e)|^2ds<9\e^{2\alpha-1}\right)\\
            &\quad\leq\Prob_{(x,y)}\left(\bm\tau\geq t(\e)>\theta^1_{n(\e)},\int_0^{t(\e)}|D(\bx_s^\e,\bxi_s^\e)|^2ds<9\e^{2\alpha-1}\right)+\Prob_{(x,y)}(\theta^1_{n(\e)}\geq t(\e)),
        \end{aligned}
    \end{equation}
    where the integer $n(\e)$ will be specified later.
    Let $\bar\tau$, $\bar{\bar\tau}$, and $a$ be defined as in Lemma~\ref{lem:near} and Lemma~\ref{lem:away}. Then
    \begin{align}
        &\Prob_{(x,y)}\left(\bm\tau\geq t(\e)>\theta^1_{n(\e)},\int_0^{t(\e)}|D(\bx_s^\e,\bxi_s^\e)|^2ds<9\e^{2\alpha-1}\right)\nonumber\\
        &\quad \leq \exp(9\e^{2\alpha-1}/a)\E_{(x,y)}\left(\chi_{\{\bm\tau>\theta^1_{n(\e)}\}}\exp\left(-\frac{1}{a}\int_0^{\theta^1_{n(\e)}}|D(\bx_s^\e,\bxi_s^\e)|^2ds\right)\right)\label{eq:rotations}\\
        &\quad\leq\exp(9\e^{2\alpha-1}/a)\left[\sup_{(x,y)\in\partial D_{r_1}\times\mathbb T^m}\E_{(x,y)}\left(\chi_{\{\bm\tau>\theta^1_1\}}\exp\left(-\frac{1}{a}\int_0^{\theta^1_1}|D(\bx_s^\e,\bxi_s^\e)|^2ds\right)\right)\right]^{{n(\e)}-1}.\nonumber
    \end{align}
    Now let us deal with one excursion from $D_{r_2}$ to $D_{r_1}$. For $(x,y)\in(\partial {D_{r_2}})_{\mathrm{out}}\times\mathbb T^m$,
    \begin{equation}
    \label{eqb:e0.99}
    \begin{aligned}
        &\E_{(x,y)}\left(\chi_{\{\bar{\bar\tau}<\bm\tau\}}\exp\left(-\frac{1}{a}\int_0^{\bar{\bar\tau}}|D(\bx_s^\e,\bxi_s^\e)|^2ds\right)\right)\\
        &\quad\leq \Prob_{(x,y)}(\bar{\bar\tau}<\bm\tau,\int_0^{\bar{\bar\tau}}|D(\bx_s^\e,\bxi_s^\e)|^2ds<a)+\Prob_{(x,y)}(\bar{\bar\tau}<\bm\tau,\int_0^{\bar{\bar\tau}}|D(\bx_s^\e,\bxi_s^\e)|^2ds\geq a)/e\\
        &\quad\leq e^{-0.99},
    \end{aligned}
    \end{equation}
    for all $\e$ sufficiently small, by Lemma~\ref{lem:away}. For $(x,y)\in\partial D_{r_1}\times\mathbb T^m$:
    \begin{align*}
        &\E_{(x,y)}\left(\chi_{\{\bm\tau>\theta^1_1\}}\exp\left(-\frac{1}{a}\int_0^{\theta^1_1}|D(\bx_s^\e,\bxi_s^\e)|^2ds\right)\right)\\
        &\leq \E_{(x,y)}\left(\chi_{\{\bm\tau>\theta^1_1,\bx^\e_{\bar\tau}\in(\partial {D_{r_2}})_{\mathrm{out}}\}}\exp\left(-\frac{1}{a}\int_0^{\theta^1_1}|D(\bx_s^\e,\bxi_s^\e)|^2ds\right)\right)+\Prob_{(x,y)}(\bx^\e_{\bar\tau}\in(\partial {D_{r_2}})_{\mathrm{in}})\\
        &\leq \sup_{(x',y')\in(\partial {D_{r_2}})_{\mathrm{out}}\times\mathbb T^m}\E_{(x',y')}\left(\chi_{\{\bar{\bar\tau}<\bm\tau\}}\exp\left(-\frac{1}{a}\int_0^{\bar{\bar\tau}}|D(\bx_s^\e,\bxi_s^\e)|^2ds\right)\right)+\Prob_{{(x,y)}}(\bx^\e_{\bar\tau}\in(\partial {D_{r_2}})_{\mathrm{in}})\\
        &\leq e^{-0.98},
    \end{align*}
    by Lemma~\ref{lem:exit_location} and \eqref{eqb:e0.99}.
    Now we can come back to \eqref{eq:rotations} and have
    \begin{equation}
    \label{eq:first_prob}
        \Prob_{(x,y)}\left(\bm\tau\geq t(\e)>\theta^1_n,\int_0^{t(\e)}|D(\bx_s^\e,\bxi_s^\e)|^2ds<9\e^{2\alpha-1}\right)\leq\exp(9\e^{2\alpha-1}/a-0.98({n(\e)}-1)).
    \end{equation}
    The second probability on the right hand side of \eqref{eq:top_inequality} can be bounded by Lemmas~\ref{lem:near} and \ref{lem:away} with certain $K>0$:
    \begin{equation}
    \label{eq:second_prob}
        \begin{aligned}
            \Prob_{(x,y)}(\theta^1_n\geq t(\e))&\leq{\E_{(x,y)}\theta^1_n}/{t(\e)}\\
            &\leq{\left(\sup_{(x',y')\in\partial D_{r_1}\times\mathbb T^m}\E_{(x',y')}\bar\tau+\sup_{{(x',y')\in\partial D_{r_2}\times\mathbb T^m}}\E_{(x',y')}\bar{\bar\tau}\right)}\cdot\frac{n(\e)}{t(\e)}\\
            &\leq\frac{n(\e)K|\log\e|}{t(\e)}.
        \end{aligned}
    \end{equation}
    Choose $n(\e)=[10\e^{2\alpha-1}/a+2]$. Then the quantity in \eqref{eq:first_prob} converges to $0$. Choose $t(\e)=100K\e^{2\alpha-1}|\log\e|/a$, then the quantity on the right hand side of \eqref{eq:second_prob} converges to $0.1$. Therefore, the quantity on the right hand side of \eqref{eq:top_inequality} converges to $0.1$. Moreover, since $t(\e)=o(\e^{\alpha-1})$, it follows from \eqref{eq:set} that, for all $x\in V^\e$, $y\in\mathbb T^m$, and $\e$ sufficiently small,
    \begin{align*}
        &\Prob_{(x,y)}(\bm\tau\geq t(\e))\\
        &=\Prob_{(x,y)}\left(\bm\tau\geq t(\e),\sup_{[0,t(\e)]}\left|\sqrt{\e}\int_0^t D(\bx_s^\e,\bxi_s^\e)dW_s\right|\leq 3\e^\alpha\right)\\
        &\leq \Prob_{(x,y)}\left(\bm\tau\geq t(\e),\int_0^{t(\e)}|D(\bx_s^\e,\bxi_s^\e)|^2ds>9\e^{2\alpha-1},\sup_{[0,t(\e)]}\left|\sqrt{\e}\int_0^t D(\bx_s^\e,\bxi_s^\e)dW_s\right|\leq 3\e^\alpha\right)\\
        &\quad+ \Prob_{(x,y)}\left(\bm\tau\geq t(\e),\int_0^{t(\e)}|D(\bx_s^\e,\bxi_s^\e)|^2ds<9\e^{2\alpha-1}\right)\\
        &\leq 0.69+0.11=0.8,
    \end{align*}
    since the stochastic integral in the last inequality can be represented as time-changed Brownian motion. Finally, we have by the Markov property
    \[
    \E_{(x,y)}\bm\tau\leq 5t(\e)=O(\e^{2\alpha-1}|\log\e|).
    \]
\end{proof}





%--------------------------------------------------------------------------------------------------












%------------------------------------------------------------------------------------------------------
\section{Tightness}
\label{sec:tightness}
In this section, we verify the tightness for the original process \eqref{eq:rescaled_process1} on $\mathbb R^2\times\mathbb T^m$.
By the Arzelà–Ascoli theorem, it suffices to check the following two conditions 
\begin{align}
    &(\romannumeral1)\lim_{R\to+\infty}\Prob_{(x,y)}\left(\sup_{0\leq t\leq T}|H(X_t^\e)|\geq R\right)\to0,\label{eqc:bounded}\\
    &(\romannumeral2)\lim_{\delta\downarrow0}\Prob_{(x,y)}\left(\sup_{\substack{0\leq s<t\leq T\\ |s-t|<\delta}}r(h(X_t^\e),h(X_s^\e))>\kappa\right)\to0,\label{eqc:equi-cont}
\end{align}
hold for all $\kappa>0$, uniformly in all $0<\e<1$.
As in \eqref{eq:H}, we can also write the equation for $H(X_t^\e)$, where we consider $(X_t^\e,\xi_t^\e)$ to be a process on $\mathbb R^2\times\mathbb T^m$:
\begin{equation}
\label{eqc:h}
    \begin{aligned}
    H(X_t^\e)&=H(x)+\int_0^t \nabla_y u_h(X_s^\e,\xi_s^\e)^{\mathsf T}\sigma(\xi_s^\e)dW_s+\e(u_h(x_0,y_0)-u_h(X_t^\e,\xi_t^\e))\\
    &\quad+\int_0^t \nabla_x u_h(X_s^\e,\xi_s^\e)\cdot b(X_s^\e,\xi_s^\e)ds.
\end{aligned}
\end{equation}
By assumption \hyperlink{H4}{\textit{(H4)}} and Lemma~\ref{lem:solution}, $u(x,y)$ is bounded together with its first derivatives.
Besides, by assumption \hyperlink{H2}{\textit{(H2)}}, $H(x)/|x|\to+\infty$ as $|x|\to+\infty$, hence $\sup\{|x|:H(x)\leq R\}/R\to0$ as $R\to+\infty$. 
Also, there exists an $K>0$ such that  and $H(x)>-K$ for all $x\in\mathbb R^2$.
For $R>K$, let $\tau_R=\inf\{t:|H(X_t^\e)|=R\}\wedge T$. Then, by Markov's inequality and boundedness of second derivatives of $H$,
\begin{align*}
    \Prob_{(x,y)}(\sup_{0\leq t\leq T}|H(X_t^\e)|\geq R)
    &=\Prob_{(x,y)}(H(X^\e_{\tau_R})=R)\\
    &\leq \E_{(x,y)}(H(X^\e_{\tau_R})+K)/R\\
    &\lesssim ({H(x)+(\e+T)\sup\{|\nabla H(x)|:H(x)\leq R\}+T+K})/{R}\\
    &\lesssim ({H(x)+(\e+T)\sup\{|x|:H(x)\leq R\}+T+K})/{R}\to0,
\end{align*}
as $R\to+\infty$, uniformly in $0<\e<1$.
Thus, we have \eqref{eqc:bounded}, and it also follows that
\begin{equation}
\label{eqc:bounded_gradient}
    \Prob_{(x,y)}(\sup_{0\leq t\leq T}|\nabla H(X_t^\e)|\geq R)\to0,
\end{equation}
as $R\to+\infty$, uniformly in $0<\e<1$.
To verify \eqref{eqc:equi-cont}, we see that, for an arbitrary $\kappa>0$ small,
\begin{align*}
    \Prob_{(x,y)}\left(\sup_{\substack{0\leq s<t\leq T\\ |s-t|<\delta}}r(h(X_t^\e),h(X_s^\e))>\kappa\right)&\leq \sum_{k=0}^{[T/\delta]}\Prob_{(x,y)}\left(\sup_{t\leq \delta}r(h(X_{k\delta+t}^\e),h(X_{k\delta}^\e))>\kappa/4\right)\\
    &\leq \sum_{k=0}^{[T/\delta]}\Prob_{(x,y)}\left(\sup_{t\leq \delta}|H(X_{k\delta+t}^\e)-H(X_{k\delta}^\e)|>\kappa/12\right).
\end{align*}
Since \eqref{eqc:bounded_gradient} holds, it is sufficient to prove, for each $R>0$,
\begin{equation}
    \sum_{k=0}^{[T/\delta]}\Prob_{(x,y)}\left(\sup_{t\leq \delta}|H(X_{k\delta+t}^\e)-H(X_{k\delta}^\e)|>\kappa/12,\sup_{0\leq t\leq T}|\nabla H(X_t^\e)|\leq R\right)\to0.
\end{equation}
Let $K'$ be the upper bound for $|b(x,y)-\nabla^\perp H(x)|$, $|u(x,y)|$, $|\nabla_x u(x,y)|$, and eigenvalues of $\nabla^2 H(x)$ and $(\nabla_y u\sigma\sigma^\mathsf{T}\nabla_y u^\mathsf{T})(x,y)$ over $\mathbb R^2\times\mathbb T^m$. 
Here we need to deal with the cases where $\e$ is small or large compared to $\kappa$ separately. 
Namely, on the event $\{\sup_{0\leq t\leq T}|\nabla H(X_t^\e)|\leq R\}$:
\begin{enumerate}[(1)]
    \item If $\e>\frac{\kappa}{48K'R}$, we have that, by \eqref{eq:rescaled_process1}, for $\delta<(\frac{\kappa}{24K'R})^2$ and all $0\leq t\leq\delta$,
\begin{equation*}
    |H(X_{k\delta+t}^\e)-H(X_{k\delta}^\e)|=|\frac{1}{\e}\int_{k\delta}^{k\delta+t}\nabla H(X_s^\e)\cdot b(X_s^\e,\xi_s^\e)ds|\leq \frac{\delta RK'}{\e}<\kappa/12,
\end{equation*}
and the probabilities are simply zero;
    \item If $\e<\frac{\kappa}{48K'R}$, by recalling \eqref{eqc:h}, we have
\begin{align*}
    \sup_{t\leq \delta}|H(X_{k\delta+t}^\e)-H(X_{k\delta}^\e)|&\leq\sup_{t\leq \delta}|\int_{k\delta}^{k\delta+t}\nabla_y u_h(X_s^\e,\xi_s^\e)^{\mathsf T}\sigma(\xi_s^\e)dW_s|+2\e K'R+{K'}^2(K'+R)\delta\\
    &\leq\sup_{0\leq t\leq\delta K'R^2}|\tilde W_t|+{K'}^2(K'+R)\delta+\kappa/24,
\end{align*}
\end{enumerate}
where $\tilde W_t$ is another Brownian motion. Hence, for $\delta$ sufficiently small, independently of $\e$,
\begin{align*}
    &\sum_{k=0}^{[T/\delta]}\Prob_{(x,y)}\left(\sup_{t\leq \delta}|H(X_{k\delta+t}^\e)-H(X_{k\delta}^\e)|>\kappa/12,\sup_{0\leq t\leq T}|\nabla H(X_t^\e)|\leq R\right)\\
    &\leq\left[\frac{T}{\delta}+1\right]\cdot\Prob_{(x,y)}\left(\sup_{0\leq t\leq\delta K'R^2}|\tilde W_t|>\kappa/48\right)\\
    &\to0,
\end{align*}
as $\delta\downarrow0$, since each term is exponentially small as $\delta\downarrow0$.


\section*{Acknowledgment}
\label{sec:ack}
%------------------------------------------------------------------------------
We would like to thank the anonymous reviewers.
This work has been partially supported by NSF projects CCF-2217070 and CNS-1909769, the Applications Driving Architectures (ADA) Research
Center, a JUMP Center co-sponsored by SRC and DARPA, and by funding and equipment gifts from VMware and Intel.
\printbibliography
\end{document}
