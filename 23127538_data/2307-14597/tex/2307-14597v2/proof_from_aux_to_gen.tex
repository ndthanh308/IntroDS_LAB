{To generalize the result to the original process $(X_t^\e,\xi_t^\e)$ on $M\times\mathbb T^m$, we need the next two technical results.
We start with a simple corollary of Lemma~\ref{lem:number_excursion}, which controls the number of excursions or, equivalently, the number of stopping times $\sigma_n$ and $\tau_n$ in finite time.}
\begin{corollary}
\label{cor:num_excursion}
    For a given $t>0$, the expected number of excursions before $t$ is $O(\e^{-\alpha})$:
    \begin{equation}
    \label{eq:num_excursion}
    \sum_{n=0}^\infty\Prob_{(x,y)}(\tau_{n+1}<t)\leq\sum_{n=0}^\infty\Prob_{(x,y)}(\sigma_n<t)\leq\frac{e^t}{\kappa}\e^{-\alpha},
    \end{equation}where $\kappa$ is the constant chosen in Lemma~\ref{lem:number_excursion}.
\end{corollary}
\begin{proof}
    By Lemma~\ref{lem:number_excursion} and the strong Markov property,
\begin{equation}
    \sup_{(x,y)\in M\times\mathbb T^m}\E_{(x,y)} e^{-\sigma_n}\leq(\sup_{(x,y)\in\gamma'\times\mathbb T^m}\E_{(x,y)} e^{-\sigma})^n\leq(1-\kappa\e^\alpha)^n.
\end{equation}
Thus, by Markov's inequality, for all $n>0$,
\begin{equation}
    \Prob_{(x,y)}(\tau_{n+1}<t)\leq\Prob_{(x,y)}(\sigma_n<t)\leq e^t\E_{(x,y)} e^{-\sigma_n}\leq e^t(1-\kappa\e^\alpha)^n,
\end{equation}
and \eqref{eq:num_excursion} follows by taking the sum.
\end{proof}
\begin{lemma}
\label{lem:aux_to_ori}
    For each $f\in\mathcal D$ and $\delta>0$ there is $0<\rho<1$ such that, for all $x\in\gamma$, $y\in \mathbb T^m$, and all $\e$ sufficiently small,
    \begin{equation}
    \label{eq:auxiliary_to_main}
    \begin{aligned}
        &\sup_{{\sigma'}\leq\rho}|\E_{(x,y)}\sum_{n=0}^\infty\chi_{\{\sigma_n<{\sigma'}\}}[f(h(X_{\tau_{n+1}}^\e))-f(h(X_{\sigma_n}^\e))-\int_{\sigma_n}^{\tau_{n+1}}\mathcal Lf(h(X_s^\e))ds]|\\ 
        &\quad\leq\delta\rho+\e^\alpha\delta\sum_{n=0}^\infty\Prob_{(x,y)}(\sigma_n<\rho),
    \end{aligned}
    \end{equation}
    where $\sigma'$ is a stopping time w.r.t. $\mathcal F^{X_\cdot^\e}_t$.
\end{lemma}
\begin{proof}
    The result holds either with or without the integral term since nearly all of the time is spent from $\tau_n$ to $\sigma_n$.
    To be precise, by the strong Markov property, Corollary~\ref{cor:num_excursion}, and Proposition~\ref{prop:exit_time_from_separatrix},
    \begin{equation}
    \label{eq:integral_not_matter}
    \begin{aligned}
        &\sup_{(x,y)\in\gamma\times\mathbb T^m}\sup_{{\sigma'}\leq\rho}|\E_{(x,y)}\sum_{n=0}^\infty\chi_{\{\sigma_n<{\sigma'}\}}\int_{\sigma_n}^{\tau_{n+1}}\mathcal Lf(h(X_s^\e))ds|\\
        &\lesssim \sup_{(x,y)\in\gamma\times\mathbb T^m}\sup_{{\sigma'}\leq\rho}\sum_{n=0}^\infty|\E_{(x,y)}\chi_{\{\sigma_n<{\sigma'}\}}\E_{(X_{\sigma_n}^\e,\xi_{\sigma_n}^\e)}\tau_1|=O(\e^\alpha|\log\e|).
    \end{aligned}
    \end{equation}
    Thus, it suffices to prove for all $\e$ sufficiently small
    \begin{equation}
    \label{eq:without_integral}
        \sup_{{\sigma'}\leq\rho}|\E_{(x,y)}\sum_{n=0}^\infty\chi_{\{\sigma_n<{\sigma'}\}}[f(h((X_{\tau_{n+1}}^\e))-f(h((X_{\sigma_n}^\e))|\leq\delta\rho+\e^\alpha\delta\sum_{n=0}^\infty\Prob_{(x,y)}(\sigma_n<\rho).
    \end{equation}

    Let us prove this for $\tx_t^\e$ first using Proposition~\ref{prop:auxiliary}, then apply the Girsanov theorem to get the result for $X_t^\e$.
    Let $\tilde\sigma'$ be the analogue of $\sigma'$ w.r.t. $\mathcal F^{\tx_\cdot^\e}_t$.
    Divide the time interval $[0,\tilde\sigma']$ into excursions using stopping times $\tilde\sigma_n$ and $\tilde\tau_n$:
    \begin{align}
    &\E_{(x,y)}[f(h(\tx_{\Tilde\sigma'}^\e))-f(h(x))-\int_0^{\Tilde\sigma'}\mathcal Lf(h(\tx_t^\e))dt]\label{eq:auxleft}\\
    &\quad = \E_{(x,y)}[f(h(\tx_{\Tilde\sigma'\wedge\Tilde\sigma}^\e))-f(h(x))-\int_0^{\Tilde\sigma'\wedge\Tilde\sigma}\mathcal Lf(h(\tx_t^\e))dt]\label{eq:auxfirstsum}\\
    &\quad+\sum_{n=0}^\infty \E_{(x,y)}\left(\chi_{\{\Tilde\sigma_n<\Tilde\sigma'\}}[f(h(\tx_{\Tilde\tau_{n+1}\wedge \Tilde\sigma'}^\e))-f(h(\tx_{\Tilde\sigma_n}^\e))-\int_{\Tilde\sigma_n}^{\Tilde\tau_{n+1}\wedge \Tilde\sigma'}\mathcal Lf(h(\tx_t^\e))dt]\right)\label{eq:auxsecondsum}\\
    &\quad+\sum_{n=1}^\infty \E_{(x,y)}\left(\chi_{\{\Tilde\tau_n<\Tilde\sigma'\}}[f(h(\tx_{\Tilde\sigma_{n}\wedge \Tilde\sigma'}^\e))-f(h(\tx_{\Tilde\tau_n}^\e))-\int_{\Tilde\tau_n}^{\Tilde\sigma_{n}\wedge \Tilde\sigma'}\mathcal Lf(h(\tx_t^\e))dt]\right)\label{eq:auxthirdsum}.
\end{align}
    Thus, \eqref{eq:auxsecondsum} converges to $0$ uniformly for all $x\in\gamma$ and $\Tilde\sigma'\leq\rho$ due to the convergence of \eqref{eq:auxleft}, \eqref{eq:auxfirstsum}, and \eqref{eq:auxthirdsum}, by Proposition~\ref{prop:auxiliary}, Proposition~\ref{prop:up_to_separatrix}, and Lemma~\ref{lem:eps_avg_prin_to_sp} with Corollary~\ref{cor:num_excursion}, respectively. Note that \eqref{eq:integral_not_matter} also holds for $\tx_t^\e$, hence we conclude that
    \begin{equation}
    \label{eq:aux_stop_at_sigma'}
        \sup_{(x,y)\in\gamma\times\mathbb T^m}\sup_{{\Tilde\sigma'}\leq\rho}\sum_{n=0}^\infty \E_{(x,y)}\left(\chi_{\{\Tilde\sigma_n<\Tilde\sigma'\}}[f(h(\tx_{\Tilde\tau_{n+1}\wedge \Tilde\sigma'}^\e))-f(h(\tx_{\Tilde\sigma_n}^\e))]\right)
        \to0.
    \end{equation}
    
    To apply the Girsanov theorem, we choose a sufficiently small time interval and use the fact that the transition probability of $(X_t^\e,\xi_t^\e)$ is similar to that of $(\tx_t^\e,\txi_t^\e)$ in the sense that they are absolutely continuous with density close to $1$.
    More precisely, for any fixed $\delta'>0$, by the Girsanov theorem, we can choose a constant $\rho_1$ such that for all $0<\rho<\rho_1$,
    \begin{equation}
        \mu_{x,y}^\e\left(\left|\frac{d\Tilde\mu_{x,y}^\e}{d\mu_{x,y}^\e}-1\right|<\delta'\right)\geq 1-\rho^2,
    \end{equation}
    where $\mu_{x,y}^\e$ and $\Tilde\mu_{x,y}^\e$ are the measures on $\bm{\mathrm C}[0,\rho]$ induced by $(X_t^\e,\xi_t^\e)$ and $(\tx_t^\e,\txi_t^\e)$. Define
    \[C'=\left\{\left|\frac{d\Tilde\mu_{x,y}^\e}{d\mu_{x,y}^\e}-1\right|<\delta\right\}\subset\bm{\mathrm C}[0,\rho],~\Omega'=\{(X_t^\e,~t\in[0,\rho])\in C'\}.\]
    Note that the quantity in \eqref{eq:without_integral} primarily depends on the behavior of the processes on time interval $[0,\sigma']$ and event $\Omega'$. 
    Indeed, we can replace the stopping times $\tau_n$ by $\tau_n\wedge\sigma'$ with $O(\e^\alpha)$ errors.
    To replace $\Omega$ with $\Omega'$, we need several additional results that control the difference.
    
    As in Corollary~\ref{cor:num_excursion}, we fix $\kappa>0$ and choose a large constant $C>0$ independent of $\rho$ such that
    \begin{equation}
        \sum_{n=[C\log(C/\rho)\e^{-\alpha}]}^\infty\Prob_{(x,y)}(\sigma_n<\rho)\leq\sum_{n=[C\log(C/\rho)\e^{-\alpha}]}^\infty e^{\rho}(1-\kappa\e^\alpha)^n\leq\delta'\rho\e^{-\alpha}.
    \end{equation}
    Now we choose $\rho_2>0$ such that, for all $0<\rho<\rho_2$, $C\rho\log(C/\rho)<\delta'$. Hence, for all $\sigma'\leq\rho$,
    \begin{equation}
        \sum_{n=0}^\infty\Prob_{(x,y)}(\{\sigma_n<\sigma'\}\setminus\Omega')\leq C\rho^2\log(C/\rho)\e^{-\alpha}+\delta'\rho\e^{-\alpha}\leq2\delta'\rho\e^{-\alpha}.
    \end{equation}
    Thus, with $K:=\max_{I_k\sim O}|\lim_{{h_k\in I_k, h_k\to O}}f'(h_k)|$, we obtain
    \begin{equation}
    \label{eq6:Omega'}
        |\E_{(x,y)}\sum_{n=0}^\infty\chi_{\{\sigma_n<{\sigma'}\}\setminus\Omega'}[f(h(X_{\tau_{n+1}\wedge\sigma'}^\e))-f(h(X_{\sigma_n}^\e))]|\leq 2(K+1)\delta'\rho.
    \end{equation}
    By following the same steps, we can choose $\rho_3>0$ such that for all $0<\rho<\rho_3$,
    \begin{equation}
    \label{eq6:Omega'_tilde}
        |\E_{(x,y)}\sum_{n=0}^\infty\chi_{\{\tilde\sigma_n<{\tilde\sigma'}\}\setminus\Omega'}[f(h(\tx_{\tilde\tau_{n+1}\wedge\tilde\sigma'}^\e))-f(h(\tx_{\tilde\sigma_n}^\e))]|\leq 2(K+1)\delta'\rho.
    \end{equation}
    It remains to consider
    \begin{equation}
    \label{eq6:omega_expc}
        |\E_{(x,y)}\sum_{n=0}^\infty\chi_{\{\sigma_n<{\sigma'}\}\cap\Omega'}[f(h(X_{\tau_{n+1}\wedge\sigma'}^\e))-f(h(X_{\sigma_n}^\e))]|,
    \end{equation}
    which can be written and estimated as, with $F$ denoting the functional on $\bm{\mathrm C}[0,\rho]$ found inside the expectation in \eqref{eq6:omega_expc}, 
    \begin{equation}
    \begin{aligned}
        \left|\int_{C'} F d\mu_{x,y}^\e\right|&=\left|\int_{C'} Fd\Tilde\mu_{x,y}^\e-\int_{C'} F\left(\frac{d\Tilde\mu_{x,y}^\e}{d\mu_{x,y}^\e}-1\right) d\mu_{x,y}^\e\right|\\
        &\leq \left|\int_{C'} Fd\Tilde\mu_{x,y}^\e\right|+\delta'\int_{C'} |F|d\mu_{x,y}^\e.
    \end{aligned}
    \end{equation}
    The first term is bounded by $2(K+2)\delta'\rho$ due to \eqref{eq:aux_stop_at_sigma'} and \eqref{eq6:Omega'_tilde}. The second term is simply bounded by $(K+1)\delta'\e^\alpha\sum_{n=0}^\infty\Prob_{(x,y)}(\sigma_n<\rho)$. Thus, we see that the left-hand side in \eqref{eq:without_integral} is no more than $(4K+6)\delta'(\rho+\e^\alpha\sum_{n=0}^\infty\Prob_{(x,y)}(\sigma_n<\rho))$ with finite $K$ independent of $\delta'$ and $\rho$ for all $\e$ sufficiently small. It remains to take $\delta'=\delta/(4K+6)$.
\end{proof}

\begin{proof}[Proof of Proposition~\ref{prop:main_result}]
Fix arbitrary $\delta>0$. We divide the time interval $[0,\eta]$ into excursions from $\gamma$ to $\gamma'$ and from $\gamma'$ to $\gamma$ by using stopping times $\sigma_n$ and $\tau_n$:
\begin{align}
    &\E_{(x,y)}[f(h(X_{\eta}^\e))-f(h(x))-\int_0^{\eta}\mathcal Lf(h(X_t^\e))dt]\nonumber\\
    &\quad = \E_{(x,y)}[f(h(X_{\eta\wedge\sigma}^\e))-f(h(x))-\int_0^{\eta\wedge\sigma}\mathcal Lf(h(X_t^\e))dt]\label{eq:firstsum}\\
    &\quad+\sum_{n=0}^\infty \E_{(x,y)}\left(\chi_{\{\sigma_n<\eta\}}[f(h(X_{\tau_{n+1}\wedge \eta}^\e))-f(h(X_{\sigma_n}^\e))-\int_{\sigma_n}^{\tau_{n+1}\wedge \eta}\mathcal Lf(h(X_t^\e))dt]\right)\label{eq:secondsum}\\
    &\quad+\sum_{n=1}^\infty \E_{(x,y)}\left(\chi_{\{\tau_n<\eta\}}[f(h(X_{\sigma_{n}\wedge \eta}^\e))-f(h(X_{\tau_n}^\e))-\int_{\tau_n}^{\sigma_{n}\wedge \eta}\mathcal Lf(h(X_t^\e))dt]\right)\label{eq:thirdsum}.
\end{align}
Here \eqref{eq:firstsum} converges to $0$ by Proposition~\ref{prop:up_to_separatrix} and \eqref{eq:thirdsum} converges to $0$ by Lemma~\ref{lem:eps_avg_prin_to_sp} and Corollary~\ref{cor:num_excursion}. It remains to consider \eqref{eq:secondsum} and it suffices to consider instead
\begin{equation}
\label{eq:secondsum1}
    \sum_{n=0}^\infty \E_{(x,y)}\left(\chi_{\{\sigma_n<\eta\}}[f(h(X_{\tau_{n+1}}^\e))-f(h(X_{\sigma_n}^\e))-\int_{\sigma_n}^{\tau_{n+1}}\mathcal Lf(h(X_t^\e))dt]\right)
\end{equation}
because the difference converges to $0$ by Proposition~\ref{prop:exit_time_from_separatrix}.
By Proposition~\ref{prop:auxiliary}, we choose $0<\rho<1$ such that \eqref{eq:auxiliary_to_main} holds for $\delta$ and all $\e$ sufficiently small. We introduce the stopping times $\hat\sigma_n$ by letting $\hat\sigma_0=\sigma$ and $\hat\sigma_n$ be the first of $\sigma_k$ such that $\sigma_k-\hat\sigma_{n-1}\geq\rho$. It is clear that $\hat\sigma_{[T/\rho]}\geq T\geq\eta$. {Hence, we can replace \eqref{eq:secondsum1} by}
\begin{equation}
\label{eq:secondsum2}
    \sum_{n=0}^{[T/\rho]-1}\E_{(x,y)}(\chi_{\{\hat\sigma_n<\eta\}}\E_{(X_{\hat\sigma_n}^\e,\xi_{\hat\sigma_n}^\e)}\sum_{k=0}^\infty\chi_{\{\sigma_k<\rho\}}[f(h(X_{\tau_{n+1}}^\e))-f(h(X_{\sigma_n}^\e))-\int_{\sigma_n}^{\tau_{n+1}}\mathcal Lf(h(X_t^\e))dt])
\end{equation}
and, by the strong Markov property,  the difference is no more than 
\begin{equation}
\label{eq:secondsum3}
    \sup_{(x,y)\in\gamma\times\mathbb T^m}\sup_{{\sigma'}\leq\rho}|\E_{(x,y)}\sum_{n=0}^\infty\chi_{\{\sigma_n<{\sigma'}\}}[f(h((X_{\tau_{n+1}}^\e))-f(h((X_{\sigma_n}^\e))-\int_{\sigma_n}^{\tau_{n+1}}\mathcal Lf(h(X_t^\e))dt]|,
\end{equation}where $\sigma'$ is a stopping time w.r.t. $\mathcal F^{X_\cdot^\e}_t$.
Both of them can be bounded by $O(\delta)$ due to Lemma~\ref{lem:aux_to_ori} and Corollary~\ref{cor:num_excursion}.
\end{proof}
