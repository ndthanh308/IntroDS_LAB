\section{Introduction}
\label{sec:introduction}
Consider a family of diffusion processes $(\bx_t^\e,\bxi_t^\e)$ in $\mathbb R^2\times\mathbb T^m$ satisfying 
\begin{equation}
\label{eq:theprocess1}
\begin{aligned}
    d\bm X_t^\e=&~b(\bx_t^\e,\bxi_t^\e)dt,~~~~~~~ \bx_0^\e=x_0\in\mathbb R^2,\\
    d\bxi_t^\e=&~\frac{1}{\e}v(\bxi_t^\e)dt+\frac{1}{\sqrt{\e}}\sigma(\bxi_t^\e)dW_t,~~~\bxi_0^\e=y_0\in\mathbb T^m,
\end{aligned}
\end{equation}
where $\e$ is a small positive parameter, $\mathbb T^m$ is the $m$-dimensional torus, and $W_t$ is an $m$-dimensional Brownian motion. 
In the coupled slow-fast system, $\bx_t^\e$ is the slow component and $\bxi_t^\e$ is the fast component, since the generator of the diffusion in the second equation is multiplied by $\e^{-1}$.
On the space $\mathbb R^2\times\mathbb T^m$, the diffusion \eqref{eq:theprocess1} is everywhere degenerate. All the randomness comes from the second equation and is transmitted to $\bx_t^\e$ through the vector $b(\bx_t^\e,\bxi_t^\e)$, which is fast-oscillating in time.
Under natural conditions, the averaging principle holds for the process in \eqref{eq:theprocess1} 
(cf. \cite{randomperturbation}). For example, if $\sigma(y)$ is non-degenerate (and thus 
$\bxi_t^\e$ has a unique invariant measure $\mu$ independent of $\e$), then $\bx_t^\e$ converges as $\e\to0$ in probability on each finite interval $[0,T]$ to an averaged process defined by the differential equation
\begin{equation}
\label{eq:averagedprocess}
    d{\bm{x}_t}=\bar b({\bm{x}_t})dt,
\end{equation}
where $\bar b(x)=\int_{\mathbb T^m}b(x,y)d\mu(y)$. 
Therefore, $\bx_t^\e$ can be viewed as a result of fast-oscillating random perturbations{, i.e. $b(x,y)-\bar b(x)$,} of the deterministic process ${\bm{x}_t}$. 
Moreover, the deviation can be described more precisely: the process $\e^{-1/2}(\bx_t^\e-{\bm{x}_t})$ converges weakly to a Gaussian Markov process on a finite interval $[0,T]$ {(cf. \cite{MR0203789}, \cite{MR0517995})}, and, if we assume a special type of vector $b(x,y)$, then the local limit theorem holds for $\e^{-1}(\bx_t^\e-{\bm{x}_t})$ at time $t$ (\cite{LLT}).






If the system \eqref{eq:averagedprocess} has a first integral $H$, then, by the averaging principle, $H(\bx_t^\e)$ is nearly constant on finite time intervals when $\e$ is small. Nontrivial behavior can, however, be observed on larger time intervals (of order $\e^{-1}$). Assume, momentarily, that $H$ has a single critical point. Then it was demonstrated in \cite{AIHPB_1995__31_3_485_0} that $H(\bx_{t/\e}^\e)$ converges weakly in $C([0,T])$, as $\e\to0$, to a diffusion process for any finite $T$, under additional assumptions. 
A similar result in the case of multiple degrees of freedom was obtained recently in \cite{Freidlin2021} and the main goal there was to overcome difficulties related to resonances, which is typical in the case of multiple degrees of freedom.
The result holds in the region where no critical points of the first integrals are present and action-angle-type coordinates can be introduced. 
% Figure environment removed
    
Let us return to the two-dimensional situation. In the presence of multiple critical points, including saddle points, the problem gets more complicated as we need to consider the Reeb graph in order to describe the evolution of the first integrals denoted by $h=(k,H)${, where the additional discrete-valued first integral $k$ is the label of the edge on the Reeb graph. (For instance, in Figure~\ref{fig:reeb_graph}, we have one saddle point and two local minima of $H$ in the space $\mathbb R^2$. Accordingly, on the graph, we have one interior vertex, three exterior vertices, including one formally representing the infinity, and three edges connecting them.)} In particular, the interior vertices on the graph correspond to the level curves that contain the saddle points, and those level curves are called the separatrices. 
In this situation, the limiting behavior has already been described for the white-noise-type additive perturbations of dynamical systems: Hamiltonian systems in $\mathbb R^2$ (\cite{FreidlinWentzell1994}), general dynamical systems with conservation laws in $\mathbb R^n$ (\cite{Freidlin2004}), and Hamiltonian systems with an ergodic component on two-dimensional surfaces (\cite{Dmitry2008},\cite{Dmitry2013},\cite{dolgopyat_freidlin_koralov_2012}). 

In this article, we consider fast-oscillating random perturbations, as discussed above, of Hamiltonian system in $\mathbb R^2$ with multiple critical points and prove that the evolution of the first integrals $h$ converges to a diffusion process defined by an operator ($\mathcal L, D(\mathcal L))$ on the corresponding Reeb graph. In particular, the exterior vertices turn out to be inaccessible and the behavior of the process near the interior vertices is described in terms of the domain $D(\mathcal L)$ in the following way: for interior vertex $O_i$, there are constants $p_k$ such that each function $f\in D(\mathcal L)$ satisfies
\begin{equation}
\label{eq:gluing_condtion}
\sum_{I_k\sim O_i}p_k\lim_{h_k\to O_i}f'(h_k)=0,
\end{equation}{where $I_k\sim O_i$ means that $O_i$ is an endpoint of $I_k$.}
Intuitively, the absolute value of $p_k$ is proportional to the probability of entering edge $I_k$ after the process arrives at the vertex $O_i$. The relation \eqref{eq:gluing_condtion} is usually referred to as the gluing condition. 
In the next section, we will formulate the results along with the assumptions more precisely. The coefficients $p_k$ will be calculated explicitly. As we mentioned, similar results hold in case of additive perturbations of Hamiltionian systems. Now the techniques in the proof are more involved and require new ideas with analysis on multiple time scales : $O(\e^{-1})$, $O(1)$, $O(\e)$, etc.
It is worth noting that our result provides the first example where the motion on a graph and the corresponding gluing conditions appear as a result of averaging of a slow-fast system{, with a Hamiltonian structure, on a large time scale}.

{In the remainder of this section, we briefly introduce the main idea and the critical steps of the proof, and outline the plan of the paper.}
To start with, since our interest is in the long-time behavior of $\bx_t^\e$ on $O(\e^{-1})$ time scales, it is often convenient to consider a temporally re-scaled process $(X_t^\e,\xi_t^\e)$:
\begin{equation}
\label{eq:rescaled_process1}
\begin{aligned}
    d X_t^\e=&~\frac{1}{\e}b(X_t^\e,\xi_t^\e)dt,~~~~~~~ X_0^\e=x_0\in\mathbb R^2,\\
    d\xi_t^\e=&~\frac{1}{\e^2}v(\xi_t^\e)dt+\frac{1}{{\e}}\sigma(\xi_t^\e)dW_t,~~~\xi_0^\e=y_0\in \mathbb T^m.
\end{aligned}
\end{equation}
It is clear that $(X_t^\e,\xi_t^\e)=(\bx_{t/\e}^\e,\bxi_{t/\e}^\e)$ in distribution.
Thus, it suffices to prove the weak convergence of $h(X_t^\e)$ in the space $\bm{\mathrm{C}}([0,T],\mathbb G)$, where $\mathbb G$ is the Reeb graph.
The proof of the weak convergence relies on demonstrating that the pre-limiting process asymptotically solves the martingale problem. Namely, we will show that, for each $f$ in a sufficiently large subset of $D(\mathcal L)$ and $T>0$,
\begin{equation}
\label{eq:martingale_problem}
    \E_{(x,y)}[f(h(X_{T}^\e))-f(h(x))-\int_0^T\mathcal Lf(h(X_{t}^\e))dt]\to 0, 
\end{equation}
as $\e\to 0$, uniformly in $x$ in any compact set in $\mathbb R^2$ and in $y\in\mathbb T^m$.
{Note that, contrary to the standard formulation of the martingale problem, there is no conditioning in \eqref{eq:martingale_problem}. However, \eqref{eq:martingale_problem} is still enough for our purpose (see Lemma~\ref{lem:martingale_problem}), since $(X_t^\e,\xi_t^\e)_{\e>0}$ is a family of strong Markov processes.}
The main idea in our proof of \eqref{eq:martingale_problem} is to divide the time interval $[0,T]$ into excursions between different visits to the separatrices and show that the contribution from each excursion is small and they do not accumulate. 
For example, suppose for now that there is only one saddle point, as shown in Figure~\ref{fig:markov_chain}. 
Let $O$ be the saddle point with $H(O)=0$, $\gamma$ be the separatrix, $\gamma'=\{x:|H(x)|=\e^\alpha\}$ be a set near the separatrix, where $0<\alpha<1/2$, and $\sigma\geq0$ be the first time when the process $X_{t}^\e$ reaches $\gamma$. Define inductively the two sequences of stopping times:
\begin{equation}
\label{eq:stopping_times}
   \begin{aligned}
    \sigma_0=&~\sigma,\\
    \tau_n=&\inf\{t>\sigma_{n-1}:X_t^\e\in\gamma'\},\\
    \sigma_n=&\inf\{t>\tau_{n}:X_t^\e\in\gamma\},
    \end{aligned}
\end{equation}
and consequently two Markov chains $(X_{\tau_n}^\e,\xi_{\tau_n}^\e)$ and $(X_{\sigma_n}^\e,\xi_{\sigma_n}^\e)$.
% Figure environment removed
As pointed out earlier, we wish to prove that the contributions to \eqref{eq:martingale_problem} from all individual excursions are small and the sum converges to zero as $\e\downarrow0$. 
{Thus, except for the first and the last excursions, it is sufficient to show that 
(a) the expectation corresponding to one excursion $[\sigma_n,\sigma_{n+1}]$ converges to zero as $\e\downarrow0$ uniformly in initial distribution,
(b) the expectation corresponding to one excursion $[\sigma_n,\sigma_{n+1}]$ is exactly zero, for all $\e$, if the process starts with the invariant measure of the Markov chain $(X_{\sigma_n}^\e,\xi_{\sigma_n}^\e)$ on $\gamma\times\mathbb T^m$, and (c) the measures on $\gamma\times\mathbb T^m$ induced by $(X_{\sigma_n}^\e,\xi_{\sigma_n}^\e)$ converge exponentially, as $n\to\infty$, uniformly in $\e$ and in initial distribution, to the invariant one. }

{The claim in (a) is an extension of the results outside of the singularities in \cite{Freidlin2021}, and new difficulties arise due to the degenerations occurring on the boundaries.
The claim in (b) is true if there is a common invariant measure for the processes $(X_t^\e,\xi_t^\e)$ for all $\varepsilon$ and the gluing conditions are chosen appropriately. In general, there is no common invariant measure for all $\e$, and we need to consider a family of auxiliary processes that do have a common invariant measure, and then use the proximity of the auxiliary and the original processes near the separatrix, and the Girsanov theorem to show that the gluing conditions are actually the same. 
The assertion in (c) is hard to verify, and its proof requires new techniques, including a local limit theorem for time-inhomogeneous functions of Markov processes and density estimates for hypoelliptic diffusions that will be discussed in later sections.}


{\textbf{\textit{Plan of the paper.}}} 

{In Section~\ref{sec:mainresult}, we introduce the notation, state the assumptions, formulate the main result and the lemma we use to establish the weak convergence. 
In Section~\ref{sec:preliminaries}, the problem is reduced to the case we discussed where there is only one saddle point. Besides, we construct an auxiliary process with $\e$-independent invariant measure and derive diffusion approximations of the processes. 
In Section~\ref{sec:averaging}, we prove the averaging principle for the process on the Reeb graph up to the time when the process reaches an interior vertex. 
In Section~\ref{sec:exponenitalconvergence}, we construct the Markov chain on the product space of the separatrix and the $m$-torus (see \eqref{eq:stopping_times}) and prove its convergence to the invariant measure with an exponential rate uniformly in $\e$. 
In Section~\ref{sec:proofofthemainresult}, we prove the main result. 
A few technical results including time estimates near the vertices and tightness of the processes are included in the Appendix.}