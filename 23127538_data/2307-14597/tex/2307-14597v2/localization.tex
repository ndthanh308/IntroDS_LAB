\section{Preliminaries}
\label{sec:preliminaries}
In this section, we explain some technical difficulties and our approach to the proof.

\subsection{Localization}
Considering \thmref{thm:mainresult} for $\bx_t^\e$ in the state space $\mathbb R^2$ causes technical difficulties due to the presence of multiple separatrices of the Hamiltonian and to the fact that the process $\bx_t^\e$ is not positive recurrent. 
However, such difficulties can be circumvented by considering the process $\bx_t^\e$ locally. 
Namely, let us cover the plane $\mathbb R^2$ by finitely many bounded domains, each containing one of the separatrices and bounded by up to three connected components of level sets of $H$, and one unbounded domain not containing any critical points. 
For example, as shown in Figure~\ref{fig:local_graph}, we have different parts of the Reeb graph $\mathbb G$ that correspond to the domains in $\mathbb R^2$.
Every point of $\mathbb R^2$ can be assumed to be contained in the interior either one or two domains. 
Since it takes positive time to travel from the boundary of one domain to the boundary of another, it suffices to prove the result up to time of exit from one domain. 
To be more precise, let $\{V_k:1\leq k \leq K\}$ be the open cover. 
Define $\eta_0=\inf\{t\geq0:X_t^\e\in\bigcup_{1\leq k \leq K}\partial V_k\}$ and, for $k$ such that $X_{\eta_{n-1}}^\e\in V_k$, define $\eta_n=\inf\{t>\eta_{n-1}: X_t^\e\not\in V_k\}$, $n\geq 1$. 
In order to prove \eqref{eq:mgprob}, it suffices to prove instead, uniformly in $x$ in any compact set in $\mathbb R^2$ and in $y\in\mathbb T^m$, that
\begin{equation}
\label{eq:martingale_problem_stopped}
    \E_{(x,y)}[f(h(X_{T\wedge\eta_1}^\e))-f(h(x))-\int_0^{T\wedge{\eta_1}}\mathcal Lf(h(X_{t}^\e))dt]\to 0,~~{\rm as}~\varepsilon \downarrow 0,
\end{equation}
since it also implies that $\Prob(\eta_n<T)\to0$ as $n\to\infty$, uniformly in all $\e$ sufficiently small.
In the unbounded domain without critical points, \eqref{eq:martingale_problem_stopped} can be obtained using the result in bounded domain together with the tightness of $h(X_t^\e)$.
It remains to consider the bounded domains.
Let $V$ be one of the bounded domains. As explained below, the process $X_t^\e$ in $V$ can be extended beyond the time when it reaches the boundary by embedding $V$ into a compact manifold $M$ with an area form and a Hamiltonian such that there are no other separatrices.
% Figure environment removed


Consider the case where $V$ is not simply connected - for example, $V$ is the domain that contains $O_3$ in Figure~\ref{fig:local_graph}. There are three connected components of $\mathbb R^2\setminus V$ as shown in Figure~\ref{fig:three_components} (other situations can be treated similarly). Then we can modify the Hamiltonian and the vector field in $C_1$ and $C_2$ in such  a way that assumptions \hyperlink{H2}{\textit{(H2)}-\textit{(H4)}} hold locally, and there is only one extremum point of $H$ in $C_1$ and one in $C_2$ (this modification is not needed if $V$ is simply connected). 
The unbounded domain $C_3$ outside $V$ can be replaced by a compact surface $S$ so that the resulting state space of $X_t^\e$ is, topologically, a sphere $M = V \bigcup C_1 \bigcup C_2 \bigcup S$. 
Then, the vector field on the surface can be chosen as a smooth extension from $V$ so that the averaged process is a Hamiltonian system on $M$ with respect to an area form $\omega$, which is simply $dx_1\wedge dx_2$ on $V$, $C_1$, and $C_2$. 
Moreover, there exists a chart $(S,\Phi)$ such that the corresponding vector field $b(x,y)$ on $D:=\Phi(S)$ satisfies that $\{b(x,y)-\bar b(x):y\in\mathbb T^m\}$ spans $\mathbb R^2$ for each $x\in D$ and the averaged process is a Hamiltonian system with respect to $dx_1\wedge dx_2$ on $D$.
For example, as shown in Figure~\ref{fig:sphere}, we can modify the vector field on the plane outside $V$ so that there are two disks with the same center $D_1\subset D_2$, and the averaged process is a Hamiltonian system in $D_2$, in particular, rotation between $\partial D_1$ and $\partial D_2$. Then $D_2$ is smoothly glued to a hemisphere and the resulting manifold is $M$, and the vector field can be extended to the surface in such a way that the averaged process is a rotation with certain constant angular velocity on the level sets. 
% Figure environment removed

It is clear that, when restricted to $V\times\mathbb T^m$, the resulting system defined on $M\times\mathbb T^m$ has exactly the same behavior as the original process on $\mathbb R^2\times\mathbb T^m$. Therefore, it suffices to prove \eqref{eq:martingale_problem_stopped} for the new process on $M\times\mathbb T^m$.
Let us formally restate the corresponding assumptions and formulate the result on $M\times\mathbb T^m$. 
In the remainder of the paper, all the definitions (e.g. the quantities defined in Section~\ref{sec:mainresult}) and statements on $M$ are understood by locally choosing coordinates so that $\omega=dx_1\wedge dx_2$. In particular, on $S$, they are understood in the coordinate $\Phi$, while on the "flat" part that contains $V$, they are understood in the usual way.
Then the assumptions on the coefficients on $M\times\mathbb T^m$ are analogous to those introduced earlier, so we only mention the differences:
\begin{enumerate}
    \item[\hypertarget{H2'}{\textit{(H2$'$)}}]$H(x)$ is a $C^\infty$ function from $M$ to $\mathbb R$  that has three extremum points and one saddle point.
    \item[\hypertarget{H3'}{\textit{(H3$'$)}}] $b(x,y)$ is a $C^\infty$ function from $M\times\mathbb T^m$ to $TM$ such that $\bar b(x)=\nabla^\perp H(x)$. 
    \item[\hypertarget{H4'}{\textit{(H4$'$)}}] $\{b(x,y)-\bar b(x):y\in\mathbb T^m\}$ spans $TM$ for all $x\in M$.
\end{enumerate}

From this point on, we denote the process on $M\times\mathbb T^m$ as $(\bx_t^\e,\bxi_t^\e)$, and $(X_t^\e,\xi_t^\e)$ on the time scale $O(\e^{-1})$ (defined by \eqref{eq:theprocess1} and \eqref{eq:rescaled_process1} with $\mathbb R^2$ replaced by $M$), and assume that the conditions \hyperlink{H2'}{\textit{(H2$'$)-(H4$'$)}} replacing \hyperlink{H2}{\textit{(H2)-(H4)}} hold. Then \eqref{eq:mgprob} follows from the next result (see \eqref{eq:martingale_problem_stopped}).
\begin{proposition}
\label{prop:main_result}
For each $f\in \mathcal D_{\mathcal L}$ and each $T>0$,
    \begin{equation}
    \label{eq:mg_problem_M}
        \E_{(x,y)}[f(h(X_{\eta}^\e))-f(h(x))-\int_0^{\eta}\mathcal Lf(h(X_t^\e))dt]\to0,
    \end{equation}
    as $\e\to0$, uniformly in $x\in M$, $y\in\mathbb T^m$, and $\eta\leq T$ that is a stopping time w.r.t. $\mathcal F_t^{X^\e_\cdot}$.
\end{proposition}
\subsection{Auxiliary process}
It turns out that similar results hold for a more general process with a slightly perturbed fast component:
\begin{equation}
\label{eq:auxiliary}
\begin{aligned}
    d\tbx_t^\e=&\ b(\tbx_t^\e,\tbxi^\e_t)dt,\\
    d\tbxi^\e_t=&\ \frac{1}{\e}v(\tbxi^\e_t)dt+\frac{1}{\sqrt\e}\sigma(\tbxi^\e_t)dW_t+c(\tbx_t^\e,\tbxi^\e_t)dt,
\end{aligned}
\end{equation}
where $c(x,y)$ is infinitely differentiable. 
Namely, $h(\tbx_{t/\e}^\e)$ converges weakly to the Markov process defined by the operator $(\mathcal L_c, D(\mathcal L_c))$ on the Reeb graph. 
Here, the subscript $c$ indicates that $\mathcal L_c$ depends on the choice of $c(x,y)$. If $c(x,y)=0$, then it is clear that $(\tbx_t^\e,\tbxi_t^\e)=(\bx_t^\e,\bxi_t^\e)$, and thus $\mathcal L=\mathcal L_c$. 
However, if $c(x,y)\not=0$, then we have an additional drift term in \eqref{eq:auxiliary}, and thus we need an additional drift term in the generator of the limiting process. 
While a precise definition of $(\mathcal L_c, D(\mathcal L_c))$ is deferred to later sections, we observe that the operators replacing $L_k$ depend on $c(x,y)$, and the domain $D(\mathcal L_c)$ as well as the linear subspace, denoted by $\mathcal D_{\mathcal L_c}$, chosen in Lemma~\ref{lem:martingale_problem} also vary for different $c(x,y)$. 
Therefore, in order to formulate general results, we consider $\mathcal D$, the set of  continuous functions on $\mathbb G$ that are four-times continuously differentiable inside each edge and satisfy conditions (i) and (iii) in Definition~\ref{def:domain_original}, as well as a weaker form of condition (ii), namely, the limits $\lim_{h_k\to O_i} L_k f(h_k)$ exist but are not necessarily independent of the edge $I_k$. 
Note that $\mathcal D$ contains $\mathcal D_{\mathcal L_c}$ for all choices of $c(x,y)$. Define $\mathcal L_c$ on $\mathcal D$ by applying the differential operator ($L_k$ plus an additional drift corresponding to $c(x,y)$) on each edge separately, with the result not being necessarily continuous at the interior vertices.

As mentioned before, we need to construct a family of auxiliary processes that, on the one hand, have a common invariant measure for all $\e>0$ and, on the other hand, are close to the processes of interest. 
The auxiliary process on $M$ can in fact be obtained by choosing a special $c(x,y)$ in \eqref{eq:auxiliary}.
We denote this particular choice of $c(x,y)$ as $\Tilde c(x,y)$. 
Now we find $\Tilde c(x,y)$ such that $\lambda\times\mu$ is the invariant measure for the process with every $\e$, where $\lambda$ is the area measure w.r.t. $\omega$ and $\mu$ is the invariant measure for $\bxi_t^\e$ in $\mathbb T^m$. 
Let $\Tilde L^\e$ be the generator of the process $(\tbx_t^\e,\tbxi_t^\e)$:
\[
\T L^\e f(x,y)=b(x,y)\cdot\nabla_x f(x,y)+\Tilde c(x,y)\cdot\nabla_y f(x,y)+\frac{1}{\e}L f(x,y).
\]
Hence, $\lambda\times\mu$ is the invariant measure if ${\T L^{\e*}}p(y)=0$, where ${\T L^{\e*}}$ is the adjoint operator of $\T L^\e$ and $p$ is the density of $\mu$, i.e.
\begin{equation}
\label{eq:adjointoperator}
\begin{aligned}
    0={\T L^{\e*}}p(y)=&-\mathrm{div}_x (b(x,y)p(y))-\mathrm{div}_y(\Tilde c(x,y)p(y))+\frac{1}{\e}L^{*}p(y),
\end{aligned}
\end{equation}
where $L^{*}$ is the adjoint operator of $L$. Since $\mu$ is the invariant measure for $\bxi^\e_t$, the last term vanishes. Hence \eqref{eq:adjointoperator} reduces to 
\begin{equation}
\label{eq:added term}
    \mathrm{div}_x b(x,y)p(y)+\mathrm{div}_y(\Tilde c(x,y)p(y))=0.
\end{equation}
To see the existence of the solution, we need the following lemma (cf. Lemma~2.1 in \cite{Freidlin2021}).
\begin{lemma}
\label{lem:solution}
    Let $\Tilde g(x,y)$ be a bounded function on $\mathbb R^2\times\mathbb T^m$ that is infinitely differentiable, and let $\Tilde L$ be the generator of a non-degenerate diffusion on $\mathbb T^m$ with the unique invariant measure $\Tilde\mu$ and suppose that $\int_{\mathbb T^m}\Tilde g(x,y)d\Tilde{\mu}(y)=0$ for each $x\in \mathbb R^2$. Then there exists a unique solution $\Tilde{u}(x,y)$ to the equation
    \begin{equation}
    \label{eq:theequation}
        \Tilde L\Tilde{u}(x,y)=-\Tilde{g}(x,y),~~~\int_{\mathbb T^m}\Tilde u(x,y)d\Tilde{\mu}(y)=0,
    \end{equation}
    and $\Tilde{u}(x,y)$ is also bounded and infinitely differentiable.
    Moreover, if $\tilde g(x,y)$ has uniformly bounded derivatives up to order $K$ in $x$ (or $y)$, the same holds for $\tilde u(x,y)$.
\end{lemma}
\begin{remark}
\label{rmk:existence_of_c}
The same result also holds for functions on $M\times\mathbb T^m$.
Thus, the existence of the solution to \eqref{eq:added term} immediately follows from Lemma~\ref{lem:solution} applied to $\Tilde g(x,y)=\mathrm{div}_x b(x,y)p(y)$ and $\Tilde L=\Delta_y$, and by taking the gradient of the solution in \eqref{eq:theequation} w.r.t. $y$, and dividing it by $p(y)$.
\end{remark}
As in \eqref{eq:rescaled_process1}, we define $(\tx_t^\e,\txi_t^\e)=(\tbx_{t/\e}^\e,\tbxi_{t/\e}^\e)$ in distribution.
Then, a simple corollary can be obtained by using Lemma~\ref{lem:solution} and then applying Ito's formula to the corresponding solution $\Tilde u(\tbx_t^\e,\tbxi_t^\e)$ and $\Tilde u(\tx_t^\e,\txi_t^\e)$ (cf. Lemma~2.3 in \cite{Freidlin2021}).
\begin{corollary}
    \label{cor:avg}
    Let $\Tilde g$ satisfy the all the conditions in Lemma~\ref{lem:solution} with $\Tilde{L}=L$ and $K=1$, then for fixed $T>0$
    \begin{equation}
        \label{eq:avg_coef}
        \E_{(x,y)}\left|\int_0^{\eta} \Tilde g(\tbx_s^\e,\tbxi_s^\e)ds\right|=O(\sqrt\e),~~~~\E_{(x,y)}\left|\int_0^{\eta} \Tilde g(\tx_s^\e,\txi_s^\e)ds\right|=O(\e),
    \end{equation}
    uniformly in $x\in M$, $y\in\mathbb T^m$, and $\eta$ that is a stopping time bounded by $T$.
\end{corollary}

\subsection{Diffusion approximation}
{Since $\int_{\mathbb T^m}(b(x,y)-\nabla^\perp H(x))d\mu(y)=0$, by Lemma~\ref{lem:solution}, there exists a function $u$ that is bounded together with its derivatives such that
\begin{equation}
\label{eq:u}
    L u(x,y)=-(b(x,y)-\nabla^\perp H(x)).
\end{equation}}
The equation is understood element-wise. Apply Ito's formula to $u(\tbx_t^\e,\tbxi_t^\e)$:
\begin{equation}
\label{eq:ito_diffusion_approx}
    \begin{aligned}
        u(\tbx_t^\e,\tbxi_t^\e)&=u(x_0,y_0)+\frac{1}{\e}\int_0^tLu(\tbx_s^\e,\tbxi_s^\e)+\frac{1}{\sqrt{\e}}\int_0^t\nabla_y u(\tbx_s^\e,\tbxi_s^\e)\sigma(\tbxi_s^\e)dW_s\\
        &\quad+\int_0^t [\nabla_x u(\tbx_s^\e,\tbxi_s^\e)b(\tbx_s^\e,\tbxi_s^\e)+\nabla_y u(\tbx_s^\e,\tbxi_s^\e)c(\tbx_s^\e,\tbxi_s^\e)]ds.
    \end{aligned}
\end{equation}
Combining \eqref{eq:auxiliary}, \eqref{eq:u}, and \eqref{eq:ito_diffusion_approx}, we obtain
\begin{equation}
\label{eq:slowx}
    \begin{aligned}
        \tbx_t^\e &= x_0+\int_0^t\nabla^\perp H(\tbx_s^\e)ds+\e\int_0^t [\nabla_x u(\tbx_s^\e,\tbxi_s^\e)b(\tbx_s^\e,\tbxi_s^\e)+\nabla_y u(\tbx_s^\e,\tbxi_s^\e)c(\tbx_s^\e,\tbxi_s^\e)]ds\\
        &\quad+\sqrt{\e}\int_0^t \nabla_y u(\tbx_s^\e,\tbxi_s^\e)\sigma(\tbxi_s^\e)dW_s+\e(u(x_0,y_0)-u(\tbx_t^\e,\tbxi_t^\e)).
    \end{aligned}
\end{equation}
Similarly, by applying Ito's formula to $u(\tx_t^\e,\txi_t^\e)$ and repeating the steps above, we have
\begin{equation}
\label{eq:x}
    \begin{aligned}
        \tx_t^\e &= x_0+\frac{1}{\e}\int_0^t\nabla^\perp H(\tx_s^\e)ds+\int_0^t [\nabla_x u(\tx_s^\e,\txi_s^\e)b(\tx_s^\e,\txi_s^\e)+\nabla_y u(\tx_s^\e,\txi_s^\e)c(\tx_s^\e,\txi_s^\e)]ds\\
        &\quad+\int_0^t \nabla_y u(\tx_s^\e,\txi_s^\e)\sigma(\txi_s^\e)dW_s+\e(u(x_0,y_0)-u(\tx_t^\e,\txi_t^\e)).
    \end{aligned}
\end{equation}
This idea of diffusion approximation is frequently used in the remainder of the paper, and the function $u(x,y)$ always refers to the solution to \eqref{eq:u}.