\subsection{Averaging principle starting from \texorpdfstring{$\gamma(\e^\alpha)$}{gamma(epsilon alpha)}}
Fix $0<\alpha<\alpha_1<\alpha_2<1/2$ and $r>0$ small enough. 
{More delicate results are obtained in this subsection to describe the behavior of processes during one excursion from $\gamma(\e^\alpha)$ to $\gamma$ (such excursions, in different domains, happen during the time intervals $[\tau_n,\sigma_n]$ defined in \eqref{eq:stopping_times}). In particular, Lemma~\ref{lem:eps_avg_prin_to_sp} gives bounds on contribution to (\ref{eq:mg_problem_M}) from each such excursion.}
Recall that $Q(h)$ is the rotation time of $\bm x_t$ on $\gamma(h)$.
Our first lemma concerns the typical deviation during one rotation.
\begin{lemma}
\label{lem4:one_rotation}
    For each $\delta>0$ there is $\kappa>0$ such that for all $x\in U(\e^{\alpha_1},r)$, $y\in\mathbb T^m$, and $\e$ sufficiently small,
    \begin{equation}
        \Prob_{(x,y)}\left(\sup_{t\in[0,Q(H(x))]}|H(\tbx_t^\e)-H(\bm x_t)|>\e^{1/2-\delta}\right)<\e^\kappa.
    \end{equation}
    There exists $\delta'>0$ and $\kappa>0$ such that for all $x\in U(\e^{\alpha_1},r)$, $y\in\mathbb T^m$, and $\e$ sufficiently small,
    \begin{equation}
        \Prob_{(x,y)}\left(\sup_{t\in[0,Q(H(x))]}|\tbx_t^\e-\bm x_t|>\e^{\delta'}\right)<\e^\kappa.
    \end{equation}
\end{lemma}
\begin{proof}
    It suffices to prove the result for $\delta<1/2-\alpha_1$. Fix $0<\delta'<\delta''<1/2-\alpha_1-\delta$. Recall the definition of $q$ in Subsection~\ref{sec:Averaging principle before} and consider the coordinates $H$ and $q$ in $U(\e^{\alpha_2},2r)$. As in \eqref{eq:H} and \eqref{eq:phi}, let $q_0=q(x)$, $u_h=u\cdot\nabla H$, $u_q=u\cdot\nabla q$, and $q_t=q_0+t$, and write the equations with $\tau^0=\inf\{t:|\tbh_t^\e-h_0|>\e^{1/2-\delta}~\mathrm{or}~|\tbq_t^\e-q_t|>\e^{\delta''}\}\wedge Q(h_0)$:
    \begin{align}
    H(\tbx_{\tau^0}^\e)&=H(x)+\sqrt{\e}\int_0^{\tau^0} \nabla_y u_h(\tbx_s^\e,\tbxi_s^\e)^{\mathsf T}\sigma(\tbxi_s^\e)dW_s+\e(u_h(x,y)-u_h(\tbx_{\tau^0}^\e,\tbxi_{\tau^0}^\e))\nonumber\\
    &\quad+\e\int_0^{\tau^0}[\nabla_x u_h(\tbx_s^\e,\tbxi_s^\e)\cdot b(\tbx_s^\e,\tbxi_s^\e)+\nabla_y u_h(\tbx_s^\e,\tbxi_s^\e)\cdot c(\tbx_s^\e,\tbxi_s^\e)]ds,\label{eq4:H_tau}\\
    q(\tbx_{\tau^0}^\e)&=q_{\tau^0}+\sqrt{\e}\int_0^{\tau^0} \nabla_y u_q(\tbx_s^\e,\tbxi_s^\e)^{\mathsf T}\sigma(\tbxi_s^\e)dW_s+\e(u_q(x,y)-u_q(\tbx_{\tau^0}^\e,\tbxi_{\tau^0}^\e))\nonumber\\
    &\quad+\e\int_0^{\tau^0}[\nabla_x u_q(\tbx_s^\e,\tbxi_s^\e)\cdot b(\tbx_s^\e,\tbxi_s^\e)+\nabla_y u_q(\tbx_s^\e,\tbxi_s^\e)\cdot c(\tbx_s^\e,\tbxi_s^\e)]ds.\label{eq4:q_tau}
\end{align}
In Appendix~\ref{sec:derivatives}, we prove that $|\nabla q|=O(|\nabla H|/H)$. Thus, it is not hard to see, by looking at the inverse of the Jacobian of $(H,q)$ w.r.t. $x$, that $|H(\tbx_t^\e)-H(\bm x_t)|\leq\e^{1/2-\delta}$ and $|\tbx_t^\e-\bm x_t|\leq\e^{\delta'}$ for all $t\leq\tau^0$. Let $S_H$ and $S_Q$ denote the stochastic integrals in \eqref{eq4:H_tau} and \eqref{eq4:q_tau}. Since ${\tau^0}\lesssim|\log\e|$,
\[\Prob_{(x,y)}(\tau^0< Q(h(x)))<\Prob_{(x,y)}(|S_H|>\e^{1/2-\delta}/2)+\Prob_{(x,y)}(|S_Q|>\e^{\delta''}/2).\] The variance of $S_H$ and $S_Q$ is small:
\begin{align*}
    \bm{\mathrm{Var}}(S_H)&=\e\E(\int_0^{\tau^0} |\nabla_y u_h(\tx_s^\e,\txi_s^\e)^{\mathsf T}\sigma(\txi_s^\e)|^2ds)\lesssim\e\E(\int_0^{\tau^0}|\nabla H(\tx_s^\e)|^2ds)\lesssim\e|\log\e|,\\
    \bm{\mathrm{Var}}(S_Q)&=\e\E(\int_0^{\tau^0} |\nabla_y u_q(\tx_s^\e,\txi_s^\e)^{\mathsf T}\sigma(\txi_s^\e)|^2ds)\lesssim\e\E(\int_0^{\tau^0}|\nabla q(\tx_s^\e)|^2ds)\lesssim\e^{1-2\alpha_1}|\log\e|.
\end{align*}
Hence both results follow from Chebyshev's inequality with $\kappa<\delta$.
\end{proof}
Let $F(h)$ be the solution to
\begin{equation}
\label{eq:ode}
    \begin{cases}
    {\mathcal L_c}F=-1\\
    F(0)=F(2r)=0
    \end{cases}
\end{equation} 
Let $\tau^1$ and $\tau^2$ be the first times for $\tx_t^\e$ to exit $U(\e^{\alpha_1},r)$ and $U(\e^{\alpha_2},2r)$, respectively. Let $x_t^\e=\bm x_{t/\e}$. 
\begin{lemma}
\label{lem4:function_F}
    There exists a function $g(r)$ with $\lim_{r\to0}g(r)=0$ such that $|F'(h)|<g(r)$ for all $0<h<2r$. There exists $C>0$ such that $|F''(h)|<C|\log h|$ and $|F'''(h)|<C/h$.
\end{lemma}
\begin{proof}
The bounds can be verified with the help of estimates for $Q(h)$ in Appendix~\ref{sec:derivatives}.    
\end{proof}
\begin{lemma}
\label{lem:time_exit_alpha1}
There exists a function $g(r)$ with $\lim_{r\to0}g(r)=0$ such that for all $x\in\gamma(\e^\alpha)$, $y\in\mathbb T^m$, and $\e$ sufficiently small,
\begin{equation}
\label{eq4:epsilon_close_time}
    \E_{(x,y)}\tau^1\leq \e^\alpha g(r).
\end{equation}
\end{lemma}
\begin{proof}
For $(\tx_t^\e,\txi_t^\e)$ starting from $(x,y)$, we define $\bar\tau^2=\e Q(H(x))\wedge\tau^2$. As in \eqref{eq:the_result}:
\begin{equation}
\label{eq4:bartau2}
    \E_{(x,y)}[F(H(\tx_{\bar\tau^2}^\e))-F(H(x))-\int_0^{\bar\tau^2}(\frac{1}{2}A(\tx_s^\e)F''(H(\tx_s^\e))+B(\tx_s^\e)F'(H(\tx_s^\e)))ds]=O(\e),
\end{equation}uniformly in $x\in U(\e^{\alpha_2},2r)$ and $y\in\mathbb T^m$.
By the definition of $\bar A(h)$ and $\bar B(h)$, one can see that $\e Q(H(x))\bar A(H(x))=\int_0^{\e Q(H(x))}A(x^\e_s)ds$ and $\e Q(H(x))\bar B(H(x))=\int_0^{\e Q(H(x))}B(x^\e_s)ds$. Since $F$ solves \eqref{eq:ode}, it follows that
\begin{equation}
\label{eq4:epsilonQ}
    \e Q(H(x))=-\e Q(H(x)){\mathcal L_c}F(H(x))=-\int_0^{\e Q(H(x))}\frac{1}{2}A(x^\e_s)F''(H(x^\e_s))+B(x^\e_s)F'(H(x^\e_s))ds.
\end{equation}
We prove that there exists $K>0$ such that 
\begin{equation}
\label{eq:compare_T}
    K\E_{(x,y)}(F(H(\tx_{\bar\tau^2}^\e))-F(H(x)))\leq-\e Q(H(x))
\end{equation}
uniformly in $x\in U(\e^{\alpha_1},r)$, $y\in\mathbb T^m$, and all $\e$ sufficiently small. Then it follows that
\begin{equation}
\label{eq4:sub_solution}
    \E_{(x,y)}\tau^1\leq KF(H(x)),
\end{equation}
for $x\in U(\e^{\alpha_1},r)$, $y\in\mathbb T^m$, and all $\e$ sufficiently small.
Indeed, we can define $\bar\tau^2_k$, $k\geq0$ recursively: $\bar\tau^2_0=0$, $\bar\tau^2_{k+1}=\inf\{t\geq\bar\tau^2_{k}:\tx_t^\e\not\in U(\e^{\alpha_2},2r)\}\wedge(\e Q(H(\tx^\e_{\bar\tau^2_k}))+\bar\tau^2_k)$, and denote the first $k$ such that $\bar\tau^2_k$ exceeds $\tau^1$ as $\bm n$. Then we have
\begin{equation}
    \begin{aligned}
        &\E_{(x,y)} \left[F(H(\tx_{\bar\tau^2_{\bm n}}))-F(H(x))\right]\\
        &=\E_{(x,y)}\sum_{k=0}^\infty\chi_{\bar\tau^2_{k}<\tau^1}\left[F(H(\tx_{\bar\tau^2_{k+1}}))-F(H(\tx_{\bar\tau^2_{k}}))\right]\\
        &\leq\E_{(x,y)}\sum_{k=0}^\infty\chi_{\bar\tau^2_{k}<\tau^1}\sup_{(x',y')\in U(\e^{\alpha_1},r)\times\mathbb T^m}\E_{(x',y')}\left[F(H(\tx_{\bar\tau^2}^\e))-F(H(x'))\right]\\
        &\leq\frac{1}{K}\E_{(x,y)}\sum_{k=0}^\infty\chi_{\bar\tau^2_{k}<\tau^1}(-\e Q(\tx_{\bar\tau^2_{k}})).
    \end{aligned}
\end{equation}
Hence \begin{equation*}
    \E_{(x,y)}\tau^1\leq\E_{(x,y)}\bar\tau^2_{\bm n}=\E_{(x,y)}\sum_{k=0}^\infty\chi_{\bar\tau^2_{k}<\tau^1}(\bar\tau^2_{k+1}-\bar\tau^2_{k})\leq\e\E_{(x,y)}\sum_{k=0}^\infty\chi_{\bar\tau^2_{k}<\tau^1}Q(\tx_{\bar\tau^2_{k}})\leq KF(H(x)).
\end{equation*}Then \eqref{eq4:epsilon_close_time} follows from \eqref{eq4:sub_solution} and Lemma~\ref{lem4:function_F} by taking $x\in\gamma(\e^\alpha)$.

To prove \eqref{eq:compare_T}, it is enough to see that, for $x\in U(\e^{\alpha_1},r)$, $y\in\mathbb T^m$, and $\e$ sufficiently small,
\begin{align*}
    &\e Q(H(x))+\E_{(x,y)}F(H(\tx_{\bar\tau^2}^\e))-F(H(x))\\
    &=-\E_{(x,y)}\int_0^{\e Q(H(x))}\left(\frac{1}{2}A(x^\e_s)F''(H(x^\e_s))+B(x^\e_s)F'(H(x^\e_s))\right)ds\\
    &\quad+\E_{(x,y)}\int_0^{\bar\tau^2}\left(\frac{1}{2}A(\tx_s^\e)F''(H(\tx_s^\e))+B(\tx_s^\e)F'(H(\tx_s^\e))\right)ds+O(\e)\\
    &=\E_{(x,y)}\int_0^{\bar\tau^2}\left(\frac{1}{2}A(\tx_s^\e)F''(H(\tx_s^\e))-\frac{1}{2}A(x^\e_s)F''(H(x^\e_s))\right)ds\\
    &\quad+\E_{(x,y)}\int_0^{\bar\tau^2}\left(B(\tx_s^\e)F'(H(\tx_s^\e))-B(x^\e_s)F'(H(x^\e_s))\right)ds\\
    &\quad-\E_{(x,y)}\int_{\bar\tau^2}^{\e Q(H(x))}\left(\frac{1}{2}A(x^\e_s)F''(H(x^\e_s))+B(x^\e_s)F'(H(x^\e_s))\right)ds+O(\e)\\
    &=o(\e Q(H(x))),
\end{align*}
where the first equality is due to \eqref{eq4:bartau2} and \eqref{eq4:epsilonQ} and the last equality is due to Lemma~\ref{lem4:one_rotation} and Lemma~\ref{lem4:function_F}. 
\end{proof}
Similarly to Lemma~\ref{lem:time2}, we can look at the transitions between $\gamma(\e^\alpha)$ and $\gamma(\e^{\alpha_1})$. By the transition probabilities given in Lemma~\ref{lem:lin_prob} and transition time given in Corollary~\ref{cor:exit_time_eps} and Lemma~\ref{lem:time_exit_alpha1}, one can obtain the following result using the strong Markov property.
\begin{corollary}
\label{cor:exit_time_r}
There exists a function $g(r)$ with $\lim_{r\to0}g(r)=0$ such that for all $x\in\gamma(\e^\alpha)$, $y\in\mathbb T^m$, and $\e$ sufficiently small,
\begin{equation}
    \label{eq:g(r)}
    \E_{(x,y)}\tau(r)\wedge\sigma\leq \e^\alpha g(r).
\end{equation}
\end{corollary}
\begin{lemma}
\label{lem:eps_avg_prin_to_sp}
For each $f\in\mathcal D$, as $\e\downarrow0$,
    \begin{equation}
    \sup_{(x,y)\in\gamma(\e^\alpha)\times\mathbb T^m}\sup_{\sigma'\leq\sigma}|\E_{(x,y)} [f(H(\tx^\e_{\sigma'}))-f(H(x))-\int_0^{\sigma'}{\mathcal L_c} f(H(\tx_s^\e))ds]|=o(\e^\alpha),
\end{equation}where the first supremum is taken over all stopping times $\sigma'\leq \sigma$.
\end{lemma}
\begin{proof}
Fix $\kappa>0$. By Corollary~\ref{cor:exit_time_r}, we can choose $r$ small enough so that for stopping time $\sigma'\leq\sigma$:  $|\E_{(x,y)}[H(\tx_{\tau(r)\wedge\sigma'}^\e)-H(x)]|<\kappa\e^\alpha$ and
$|\E_{(x,y)}[f(H(\tx_{\tau(r)\wedge\sigma'}^\e))-f(H(x))]|<\kappa\e^\alpha$, and
\begin{align*}
    \sup_{(x,y)\in\gamma(\e^\alpha)\times\mathbb T^m}\sup_{\sigma'\leq\sigma}|\E_{(x,y)}\int_0^{\tau(r)\wedge\sigma'}{\mathcal L_c} f(H(\tx_s^\e))ds|<\kappa\e^\alpha,
\end{align*}for all $\e$ sufficiently small, using similar arguments leading to \eqref{eq:H} and \eqref{eq:f}. It follows that, $\Prob_{(x,y)}(H(\tx_{\tau(r)\wedge\sigma'}^\e)=r)\leq H(x)/r+ \kappa\e^\alpha/r\leq2\e^\alpha/r$.
Therefore, uniformly in all $x\in\gamma(\e^\alpha)$, $y\in\mathbb T^m$, and $\sigma'\leq\sigma$,
\begin{align*}
    &|\E_{(x,y)} [f(H(\tx^\e_{\sigma'}))-f(H(x))-\int_0^{\sigma'}{\mathcal L_c} f(\tx_s^\e)ds]|\\
    &\leq|\E_{(x,y)} f[H(\tx^\e_{\tau(r)\wedge\sigma'})-f(H(x))-\int_0^{\tau(r)\wedge\sigma'}{\mathcal L_c} f(H(\tx_s^\e))ds]|\\
    &\quad+\Prob_{(x,y)}(H(X_{\tau(r)\wedge\sigma'}^\e)=r)\sup_{\substack{x'\in\gamma(r)\\ y'\in\mathbb T^m}}|\E_{(x',y')} [f(H(\tx^\e_{\sigma'}))-f(H(x))-\int_0^{\sigma'}{\mathcal L_c} f(H(\tx_s^\e))ds]|\\
    &\leq 3\kappa\e^\alpha,
\end{align*}
for $\e$ sufficiently small, due to Proposition~\ref{prop:up_to_separatrix} and our choice of $r$. The result follows because $\kappa$ can be chosen arbitrarily small.
\end{proof}
{The last result in this subsection provides estimates that will be used later to control the number of excursions from $\gamma(\e^\alpha)$ to $\gamma$ in finite time (see Corollary~\ref{cor:num_excursion}).}
\begin{lemma}
\label{lem:number_excursion}
    There is a constant $\kappa>0$ such that, for all $\e$ sufficiently small,
    \begin{equation}
        \sup_{(x,y)\in\gamma(\e^\alpha)\times\mathbb T^m}\E_{(x,y)} e^{-\sigma}\leq1-\kappa\e^\alpha.
    \end{equation}
\end{lemma}
\begin{proof}
    By Corollary~\ref{cor:exit_time_r}, as in the proof of Lemma~\ref{lem:lin_prob}, we can fix $0<r<1/3$ such that for all $x\in\gamma(\e^\alpha)$, $y\in\mathbb T^m$, and $\e$ sufficiently small, $\Prob_{(x,y)}(\tau(r)<\sigma)\geq\e^\alpha/2r$.
    Let $F$ be defined as in \eqref{eq:ode} and $t=F(r)/3$, then it follows from Proposition~\ref{prop:up_to_separatrix}, as $\e\downarrow0$,
    \begin{equation}
        \sup_{(x,y)\in\gamma(r)\times\mathbb T^m}\E_{(x,y)}[F(H(\tx_{\sigma\wedge\tau(2r)\wedge t}^\e))-F(H(x))-\int_0^{\sigma\wedge\tau(2r)\wedge t}{\mathcal L_c}F(H(\tx_s^\e))ds]\to 0.
    \end{equation}
    Thus, we have that for all $x\in\gamma(r)$, $y\in\mathbb T^m$, and $\e$ sufficiently small,
    \begin{equation}
        \E_{(x,y)}F(H(\tx_{\sigma\wedge\tau(2r)\wedge t}^\e)>F(r)/2,
    \end{equation}
    and it follows that,
    \begin{equation}
        \Prob_{(x,y)}(\sigma>t)\geq\Prob_{(x,y)}(\sigma\wedge\tau(2r)>t)>\frac{\E_{(x,y)}F(H(\tx_{\sigma\wedge\tau(2r)\wedge t}^\e)}{\sup_{[0,2r]}F(h)}>\frac{F(r)}{2\sup_{[0,2r]}F(h)}=:c_1(r).
    \end{equation}
    Then, for all $x\in\gamma(r)$, $y\in\mathbb T^m$, and $\e$ sufficiently small,
    \begin{equation}
        \E_{(x,y)} e^{-\sigma}\leq\Prob_{(x,y)}(\sigma\leq t)+\Prob_{(x,y)}(\sigma>t)e^{-t}\leq 1-\Prob_{(x,y)}(\sigma>t)(1-e^{-t})\leq 1-c(r),
    \end{equation}
    with $c(r)=(1-\exp(-F(r)/3))c_1(r)>0$, and therefore,
    \begin{align*}
        \E_{(x,y)} e^{-\sigma}&\leq \Prob_{(x,y)}(\sigma<\tau(r))+\Prob_{(x,y)}(\sigma>\tau(r))\sup_{{x'\in\gamma(r),y'\in\mathbb T^m}}\E_{(x',y')} e^{-\sigma}\\
        &\leq 1-\Prob_{(x,y)}(\sigma>\tau(r))(1-\sup_{{x'\in\gamma(r),y'\in\mathbb T^m}}\E_{(x',y')} e^{-\sigma})\\
        &\leq 1-\frac{1}{2}c(r)\frac{\e^\alpha}{r}.
    \end{align*}
    The result holds with $\kappa=c(r)/2r$.
\end{proof}

