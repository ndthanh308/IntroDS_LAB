\documentclass[12pt]{amsart}
\usepackage[english]{babel}
\usepackage{amsmath}
\usepackage{amsthm}
\usepackage{anysize}
\usepackage{mathtools}
\usepackage{mathrsfs}
\usepackage{amssymb}
\usepackage{xcolor}
\usepackage{xfrac}
\usepackage{esint}
\usepackage{graphicx}
\usepackage{float}
\usepackage{array}
\usepackage{enumitem}
%\usepackage[new]{old-arrows}
%\usepackage[all,cmtip]{xy}
\usepackage{tikz-cd}
\usepackage{centernot}
\usepackage{stmaryrd}
\usepackage{comment}
\excludecomment{confidential}

\usepackage{graphicx,color}
\usepackage{tikz,pgfplots,pgf}
\usepackage[font=small]{caption}

\newtheorem{theorem}{Theorem}[section]
\newtheorem{lemma}{Lemma}[section]
\newtheorem{proposition}{Proposition}[section]
\newtheorem{corollary}{Corollary}[section]
\theoremstyle{definition}
\newtheorem{example}{Example}[section]
\theoremstyle{definition}
\newtheorem{remark}{Remark}[section]
\newtheorem{definition}{Definition}[section]

\usepackage[english]{babel}
\usepackage{amsmath}
\usepackage{amsthm}
\usepackage{mathtools}
\usepackage{mathrsfs}
\usepackage{amssymb}
\usepackage{xcolor}
\usepackage{xfrac}
\usepackage{esint}
\usepackage{graphicx}
\usepackage{float}
\usepackage{array}
\usepackage{enumitem}
\usepackage[new]{old-arrows}
\usepackage[all,cmtip]{xy}
\usepackage{tikz-cd}
\usepackage{centernot}
\usepackage{stmaryrd}
\usepackage{comment}
\usepackage{bbm}

\newcommand{\mc}{\mathcal}
\newcommand{\mf}{\mathfrak}


\newcommand{\K}{\mathbb{K}}
\newcommand{\R}{\mathbb{R}}
\newcommand{\N}{\mathbb{N}}
\newcommand{\C}{\mathbb{C}}
\renewcommand{\P}{\mathbb{P}}
\renewcommand{\l}{\lambda}
\renewcommand{\O}{\Omega}
\renewcommand{\d}{\delta}
\renewcommand{\a}{\alpha}
\renewcommand{\b}{\beta}
\hfuzz40pt
\DeclareMathOperator{\col}{col} \renewcommand{\o}{\omega}
\newcommand{\f}{\varphi}
\DeclareMathOperator{\triang}{trng}
\DeclareMathOperator{\Sub}{Sub} \DeclareMathOperator{\rank}{rank}
\DeclareMathOperator{\spn}{span} \DeclareMathOperator{\Iso}{Iso}
\DeclareMathOperator{\dom}{dom}
\renewcommand{\L}{\Lambda}
\DeclareMathOperator{\Fred}{Fred} \DeclareMathOperator{\spr}{spr}
\DeclareMathOperator{\Eig}{Eig}
\renewcommand{\subsetneq}{\varsubsetneq}
\renewcommand{\Re}{\operatorname{Re}}
\renewcommand{\Im}{\operatorname{Im}}

\DeclareMathOperator\E{E}

\DeclareMathOperator{\Ind}{Ind}
\DeclareMathOperator{\vol}{vol}
\DeclareMathOperator{\im}{im}
\DeclareMathOperator{\Aut}{Aut}
\DeclareMathOperator{\Res}{Res}
\DeclareMathOperator{\I}{Im}
\DeclareMathOperator{\Hom}{Hom}
\DeclareMathOperator{\Vol}{Vol}
\DeclareMathOperator{\colim}{colim}

\numberwithin{equation}{section}

\makeatletter
\@namedef{subjclassname@2020}{\textup{2020} Mathematics Subject Classification}
\makeatother

\begin{document}

\title{On projective limits of probability measures}

\author{Juan Carlos Sampedro} \thanks{The author has been supported by the Research Grant PID2021--123343NB--I00 of the Spanish Ministry of Science, Technology and Universities}
\address{Basque Center for Applied Mathematics (BCAM), Bilbao, Spain.
	Institute of Interdisciplinary Mathematics (IMI), Madrid, Spain.
	Departamento de Matemáticas, Universidad del País Vasco (UPV/EHU), Aptdo. 644, 48080 Bilbao, Spain.}
\email{juancarlos.sampedro@ehu.eus}


\keywords{Projective limit measures, limit, colimit, measures on vector spaces, Gaussian measures, Lebesgue spaces, Osterwalder--Schrader axioms}
\subjclass[2020]{18A30, 60A10, 46E30, 46M10}

\begin{abstract}
	The present article describes the precise structure of the $L^{p}$-spaces of projective limit measures by introducing a category theoretical perspective. This analysis is applied to measures on vector spaces and in particular to Gaussian measures on nuclear topological vector spaces. A simple application to constructive Quantum Field Theory (QFT) is given through the Osterwalder--Schrader axioms.
\end{abstract}

\maketitle

\section{Introduction}\label{section-1}

The notion of projective limit of probability spaces is fundamental in various fields of mathematics. For instance, in probability theory, the existence of stochastic processes is usually equivalent to the existence of such a limit through the celebrated Kolmogorov extension theorem \cite{K}. Moreover, the construction of various types of  measures on infinite dimensional vector spaces used in constructive Quantum Field Theory (QFT) is done through this concept, see for instance Yamasaki \cite[Ch. 3]{Y}, Glimm--Jaffe \cite[Ch. 6]{GJ} and Simon \cite{Si}. The existence of projective limits of probability spaces has been extensively studied since the foundational works of Kolmogorov \cite{K} and Bochner \cite{B} in the fifties. The greatest efforts in previous works have been devoted to establishing necessary conditions on the projective system for the existence of the projective limit measure (see, e.g., Choksi \cite{Ch}, Metivier \cite{Me}, Bourbaki \cite{Bou}, Mallory and Sion \cite{Ma}, Rao \cite{RA}, Frolík \cite{F}, Rao and Sazonov \cite{RS} and Pintér \cite{P}, among others). In this article we will change the perspective slightly. Assuming the existence of the projective limit measure, we describe the $L^{p}$-spaces of such measure by introducing a category theoretical perspective. The analogue analysis for the inductive limit measure was partially established by Macheras \cite{M}. However the projective case is more interesting from the point of view of applications where projective limits are more present that inductive ones. 
\par This paper is organized as follows. Section \ref{S2} introduces the basic notions of categorical limits that will make the article self-contained for non-experts in category theory. Section \ref{Se3} presents the main results of the article. Among the main findings, we prove that the $L^{p}$-spaces of the projective limit $(X_{\infty},\mu_{\infty})$ of the projective system $(X_{i},\mu_{i})_{i\in I}$, where $I$ is a given directed set, can be described as an explicit subspace of the product space $\prod_{i\in I}L^{p}(X_{i},\mu_{i})$. This result implies that the integration of functions on $(X_{\infty},\mu_{\infty})$ can be reduced to limits of the type
 \begin{equation*}
	\int_{X_{\infty}}f \, \mathrm{d}\mu_{\infty}=\lim_{i\in I}\int_{X_{i}}f_{i}\, \mathrm{d}\mu_{i},
\end{equation*}
where the limit is understood in the sense of net convergence. This is the content of Theorems \ref{T3.3} and \ref{C4.4}. Section \ref{Se4} applies the abstract techniques of Section \ref{Se3} to study the measures on vector spaces. This measures are commonly defined through a projective limit measure and consequently the analysis of this article is naturally applied to this case. Section \ref{Se4} is organized into three subsections. Subsection \ref{S4.1} focusses on measures on topological vector spaces. In Subsection \ref{S4.2} we obtain analogous results for Gaussian measures on nuclear spaces. This kind of measures are central in constructive (Euclidean) QFT through the program started by J. Glimm and A. Jaffe in the seventies through the Osterwalder--Schrader axioms. As a by-product, in Subsection \ref{S4.3}, using the techniques developed along Section \ref{Se4}, we give an application to the computation of Schwinger (or correlation) functions through the limit of finite dimensional integrals.
\begin{comment}
Let $\{(\Omega_{i},\Sigma_{i},\mu_{i})\}_{i\in\mathbb{N}}$ be a family of probability spaces. Kakutani's theorem \cite{}, ensures the existence of a unique probability measure $\mu_{\infty}$ defined on the measurable space $(\O,\Sigma)$, where $\O:=\bigtimes_{i\in \N}\O_{i}$ and $\Sigma$ is the $\sigma$-algebra generated by the cylinder sets
\begin{equation*}
	\mathcal{C}:=\left\{\bigtimes_{i=1}^{m}C_{i}\times\bigtimes_{i=m+1}^{\infty}\Omega_{i} :C_{i}\in\Sigma_{i}, \ \forall i\in\{1,2,...,m\} \text{ and } m\in\mathbb{N}\right\};
\end{equation*}
satisfying the identity
\begin{equation}\label{E2}
	\mu_{\infty}\left(\mathscr{C}\right)=\prod_{i=1}^{m}\mu_{i}(C_{i}), \quad \mathscr{C}=\bigtimes_{i=1}^{m}C_{i}\times\bigtimes_{i=m+1}^{\infty}\Omega_{i},
\end{equation}
for each  $\mathscr{C}\in\mathcal{C}$. In fact, the measure $\mu_{\infty}$ can be regarded as the projective limit of the finite dimensional product measures $\mu^{n}$ defined on $(\O_{n},\Sigma_{n})$, where $\O^{n}:=\bigtimes_{i=1}^{n}\O_{i}$ and $\Sigma_{n}$ is the product $\sigma$-algebra. See Section 3 for a detailed account of this fact.
In 1934, B. Jessen, as part of his PhD Thesis \cite{16}, proved that for every integrable function $f\in L^{1}(\Omega,\Sigma,\mu_{\infty})$, the identity
\begin{equation*}
	\lim_{n\to\infty}\int_{\bigtimes_{i=1}^{n}\Omega_{i}}f(\omega^{n},x_{n+1},x_{n+2},...)\ \prod_{i=1}^{n}d\mu_{i}=\int_{\Omega}f \ d\mu_{\infty}, \quad \o^{n}:=(\o_{1},\o_{2},\cdots,\o_{n}),
\end{equation*}
holds where the convergence is pointwise for $\mu_{\infty}$-a.e. $x=(x_{1},x_{2},\cdots)\in \O$. In \cite{}, the author proves that this identity induces a description of the $L^{p}$-spaces $L^{p}(\O,\Sigma,\mu_{\infty})$, $1\leq p<\infty$, as the inductive limit of $L^{p}(\O_{n},\Sigma_{n},\mu_{n})$, where $\O_{n}:=\bigtimes_{i=1}^{n}\O_{i}$, $\Sigma_{n}$ is the product $\sigma$-algebra and $\mu_{n}$ the corresponding product measure. This description expresses the exact relationship between the integration in $\O$ and in the finite dimensional $\O_{n}$.
\end{comment}

\section{Basic notions regarding categorical limits}\label{S2}

In this section we proceed to describe briefly the basic notions of projective limits (limits) and inductive limits (colimits) in category theoretical language. Let $\mathfrak{C}$ be a fixed category and $I$ a directed set. A \textit{diagram} of shape $I$ is a contravariant functor $\mf{F}:I\to \mathfrak{C}$. It can be represented as $(X_{i},\varphi_{ij})$ for a family of objects $\{X_{i}:i\in I\}$ of $\mf{C}$ indexed by $I$, and for each $i\leq j$ a morphism $\varphi_{ij}:X_{j}\to X_{i}$ such that $\varphi_{ii}$ is the identity on $X_{i}$ and $\varphi_{ik}=\varphi_{ij}\circ \varphi_{jk}$ for each $i\leq j\leq k$. In general, we can consider diagrams as functors $\mf{F}:\mathfrak{J}\to\mathfrak{C}$ indexed by a general category $\mathfrak{J}$. However, diagrams indexed by directed sets are enough for our purposes.
A \textit{cone} of $\mf{F}$ is a pair $(X,\phi_{i})$ where $X$ is an object of $\mf{C}$ and $\phi_{i}:X\to X_{i}$ is a morphism of $\mf{C}$ such that $\phi_{i}=\varphi_{ij}\circ \phi_{j}$ for every $i\leq j$. Let us denote by $\mf{K}(\mf{F})$ the category of cones of $\mf{F}$, that is, the category whose objects are cones $(X,\phi_{i})$ and given two cones $(X,\phi_{i})$, $(Y,\psi_{i})$, the morphisms $(X,\phi_{i})\to (Y,\psi_{i})$ are just morphisms $\varphi: X\to Y$ of $\mf{C}$ such that the diagram of Figure \ref{F0} commutes.
% Figure environment removed

A \textit{limit} of the diagram $(X,\varphi_{ij})$ is defined to be a cone $(Y,\psi_{i})$ characterizing by the following universal property: For every other cone $(X,\phi_{i})$, there exists a unique morphism $\varphi:X\to Y$ making the diagram of Figure \ref{F0} commutative. 
The limit of a given diagram can also be characterized as the terminal object in the category of cones of $\mf{F}$, $\mf{K}(\mf{F})$. The limit of a given diagram does not necessarily exist, however if it does exist, it is unique up to a unique isomorphism in the category of cones. Any given isomorphism class of the limit is called a \textit{realization}. The limit of a given diagram $\mf{F}:I\to \mf{C}$, $\mf{F}\equiv(X_{i},\varphi_{ij})$, is commonly denoted by $\lim \mf{F}= (\lim X_{i},\psi_{i})$.

Dually, a \textit{codiagram} is a covariant functor $\mf{F}:I\to \mathfrak{C}$. It can be represented as $(X_{i},\varphi_{ij})$ for a family of objects $\{X_{i}:i\in I\}$ of $\mf{C}$ indexed by $I$, and for each $i\leq j$ a morphism $\varphi_{ij}:X_{i}\to X_{j}$ such that $\varphi_{ii}$ is the identity on $X_{i}$ and $\varphi_{ik}=\varphi_{jk}\circ \varphi_{ij}$ for each $i\leq j\leq k$. A \textit{cocone} is a pair $(X,\phi_{i})$ where $X$ is an object of $\mathfrak{C}$ and $\phi_{i}:X_{i}\to X$ is a morphism of $\mathfrak{C}$ such that $\phi_{i}=\phi_{j}\circ \varphi_{ij}$ for every $i\leq j$. 
Let us denote by $\mf{K}^{\ast}(\mf{F})$ the category of cocones of $\mf{F}$, that is, the category whose objects are cocones $(X,\phi_{i})$ and given two cocones $(X,\phi_{i})$, $(Y,\psi_{i})$, the morphisms $(Y,\psi_{i})\to(X,\phi_{i})$ are just the morphisms $\varphi: Y\to X$ of $\mf{C}$ such that the diagram of Figure \ref{F20} commutes.
% Figure environment removed
A \textit{colimit} of the codiagram $(X_{i},\varphi_{ij})$ is defined to be a cocone $(Y, \psi_{i})$ characterized by the following universal property: For any cocone $(X,\phi_{i})$, there exists a unique morphism $\varphi:Y\to X$ making the diagram of Figure \ref{F20} commutative.

Analogously, the colimit of a given codiagram can also be characterized as the initial object in the category of cocones $\mf{K}^{\ast}(\mf{F})$. The colimit of a given codiagram does not necessarily exist, however if it does exist, it is also unique up to a unique isomorphism in the category of cocones. The colimit of a given codiagram $\mf{F}:I\to \mf{C}$, $\mf{F}\equiv(X_{i},\varphi_{ij})$, is commonly denoted by $\colim \mf{F}= (\colim X_{i},\psi_{i})$.
\par In this article, we will consider the category $\mathfrak{Ban}$ whose objects are Banach spaces and whose morphisms are linear isometries. For a deep study of this category and its colimits, we refer the reader to Wegge-Olsen \cite[App. L]{WO} and Castillo \cite{JC} and references therein.
\par Throughout this article, we will make use of the following simple lemma.
\begin{lemma}
	\label{L2.1}
	Let $(X_{i},\varphi_{ij})$ be a codiagram on $\mathfrak{Ban}$. Then a cocone $(Y,\psi_{i})$ is a realization of the colimit of $(X_{i},\varphi_{ij})$ if and only if $\bigcup_{i\in I}\psi_{i}(X_{i})$ is dense in $Y$.
\end{lemma}

\begin{proof}
	Suppose that $(Y,\psi_{i})$ defines a realization of the colimit of $(X_{i},\varphi_{ij})$. Consider the cocone $(X,\rho_{i})$ where
	\begin{equation*}
	X:=\overline{\bigcup_{i\in I}\psi_{i}(X_{i})}\subset Y,
	\end{equation*}
	and where $\rho_{i}:X_{i}\to X$ are defined by $\rho_{i}(x_{i})=\psi_{i}(x_{i})$ for each $x_{i}\in X_{i}$. Note that $X$ is a Banach space since $\bigcup_{i\in I}\psi_{i}(X_{i})$ is a vector space. By the universal property, there exists a unique isometry $\varphi:Y\to X$ making the diagram of Figure \ref{F20} commutative. It is straightforward to prove that the morphism $\varphi$ is in fact an isomorphism. Hence $\bigcup_{i\in I}\psi_{i}(X_{i})$ is dense in $Y$. On the other hand, take a cocone $(Y,\psi_{i})$ such that $\bigcup_{i\in I}\psi_{i}(X_{i})$ dense in $Y$. For any other cocone $(X,\rho_{i})$, define the morphism $\varphi:Y\to X$ by 
	\begin{equation*}
	\varphi\circ\psi_{i}:=\rho_{i}, \quad i\in I.
	\end{equation*}
	It is easy to see that the morphism $\varphi$ is the unique morphism making the diagram of Figure \ref{F20} commutative. Hence $(Y,\psi_{i})$ is a colimit.
\end{proof}

\section{Abstract results}\label{Se3}

In this section we present the main findings of the paper. Let us start by fixing some notation. As usual, we denote a measurable space by a pair $(X,\mc{B})$ where $X$ is the main set and $\mc{B}$ is a $\sigma$-algebra of subsets of $X$.  Let $\mathfrak{M}_{0}$ be the category whose objects are measurable spaces $(X,\mathcal{B})$ and whose morphisms are measurable maps $f:(X,\mathcal{B})\to (Y,\mathcal{S})$. Given a directed set $I$, a diagram $\mf{F}:I\to\mathfrak{M}_{0}$ can be described by a family $(X_{i},\mathcal{B}_{i},\pi_{ij})$ where $(X_{i},\mathcal{B}_{i})$ are measurable spaces and 
$$\pi_{ij}:(X_{j},\mathcal{B}_{j})\to (X_{i},\mathcal{B}_{i}), \quad i\leq j,$$ 
are morphisms in $\mathfrak{M}_{0}$ such that $\pi_{ii}$ is the identity and $\pi_{ik}=\pi_{ij}\circ\pi_{jk}$, for each $i\leq j\leq k$. The limit of any diagram $(X_{i},\mathcal{B}_{i},\pi_{ij})$ exists and can be described by the realization $(X_{\infty},\mathcal{B}_{\infty},\pi_{i})$ where 
\begin{equation*}
X_{\infty}:=\Big\{(x_{i})_{i\in I}\in \prod_{i\in I}X_{i}: \pi_{ij}(x_{j})=x_{i} \text{ for each } i\leq j \Big\},
\end{equation*}
the morphisms $\pi_{j}:X_{\infty}\to X_{j}$ are the projection maps given by $\pi_{j}(x_{i})_{i\in I}:=x_{j}$ and
\begin{equation*}
\mathcal{B}_{\infty}:=\sigma(\{\pi_{i}^{-1}(B): B\in\mathcal{B}_{i}, \ i\in I\}).
\end{equation*}
Given a family $B$ of subsets of a given set $X$, the notation $\sigma(B)$ stands for the $\sigma$-algebra of subsets of $X$ generated by $B$.
Let us denote by $\mathfrak{M}$ the category whose objects are probability spaces $(X,\mathcal{B},\mu)$ and whose morphisms are measure preserving maps $f:(X,\mathcal{B},\mu)\to (Y,\mathcal{S},\nu)$, i.e., measurable maps that satisfy $f_{\ast}\mu=\nu$, where $f_{\ast}\mu:\mc{S}\to[0,+\infty]$ is the push-forward measure defined by  
$$f_{\ast}\mu(B):=\mu(f^{-1}(B)), \quad B\in \mc{S}.$$
Given a directed set $I$, a diagram $\mf{F}:I\to\mathfrak{M}$ can be described by a family $(X_{i},\mathcal{B}_{i},\mu_{i},\pi_{ij})$ where $(X_{i},\mathcal{B}_{i},\mu_{i})$ are probability spaces and 
$$\pi_{ij}:(X_{j},\mathcal{B}_{j},\mu_{j})\to (X_{i},\mathcal{B}_{i},\mu_{i}), \quad i\leq j,$$
are morphisms in $\mathfrak{M}$ such that $\pi_{ii}$ is the identity and $\pi_{ik}=\pi_{ij}\circ\pi_{jk}$, for each $i\leq j\leq k$. 
\par A diagram $(X_{i},\mathcal{B}_{i},\mu_{i},\pi_{ij})$ is said to be \textit{convergent} if there exists a measure $\mu_{\infty}$ on the limit $(X_{\infty},\mc{B}_{\infty},\pi_{i})$ of $(X_{i},\mc{B}_{i},\pi_{ij})$ in $\mf{M}_{0}$ such that 
$$(\pi_{i})_{\ast}\mu_{\infty}=\mu_{i} \ \ \text{for each } i\in I.$$ Every convergent diagram $(X_{i},\mathcal{B}_{i},\mu_{i},\pi_{ij})$ has a limit in $\mf{M}$ and it is described by the realization $(X_{\infty},\mathcal{B}_{\infty},\mu_{\infty},\pi_{i})$. The characterization of convergent diagrams on $\mf{M}$ has taken a great deal of work, we refer the reader to the classical result in Yamasaki \cite[Th. 7.2, Part A]{Y} for some sufficient conditions for the convergence. A sharper result has been obtained recently by Pintér \cite{P}. 
\par Subsequently, when we cite a probability space we will omit to write the associated $\sigma$-algebra if this is clear from the context.
For each $1\leq p<\infty$, there is a contravariant functor
\begin{equation*}
L^{p}:\mathfrak{M}\longrightarrow \mathfrak{Ban},
\end{equation*}
that assign to every object $(X,\mu)$ in $\mathfrak{M}$, the object $L^{p}(X,\mu)$ in $\mathfrak{Ban}$ and to every morphism $f:(X,\mu)\to (Y,\nu)$ in $\mathfrak{M}$ the morphism $f^{\ast}:L^{p}(Y,\nu)\to L^{p}(X,\mu)$ in $\mf{Ban}$ defined by $f^{\ast}(u):=u\circ f$ for each $u\in L^{p}(Y,\nu)$. Indeed, the morphism $f^{\ast}:L^{p}(Y,\nu)\to L^{p}(X,\mu)$ is an isometry since the change of variable formula for measures (see for instance Cohn \cite[Pr. 2.6.5]{C}) and the identity $f_{\ast}\mu=\nu$ yield
$$\|f^{\ast}(u)\|_{L^{p}(\mu)}^{p}=\int_{X}|u\circ f|^{p}  \, \mathrm{d}\mu=\int_{Y}|u|^{p}  \, \mathrm{d}f_{\ast}\mu=\int_{Y}|u|^{p} \, \mathrm{d}\nu=\|u\|_{L^{p}(\nu)}^{p}.$$
Our first result states that the $L^{p}$-functor preserves limits of convergent diagrams.

\begin{theorem}
	\label{T3.1}
	Let $(X_{i},\mu_{i},\pi_{ij})$ be a convergent diagram in $\mathfrak{M}$ with limit $(X_{\infty},\mu_{\infty},\pi_{i})$. Then for each $1\leq p<\infty$, the cocone $(L^{p}(X_{\infty},\mu_{\infty}),\pi^{\ast}_{i})$ is the colimit of the codiagram $(L^{p}(X_{i},\mu_{i}),\pi_{ij}^{\ast})$ in $\mf{Ban}$.
	In other words, the $L^{p}$ functor preserves convergent limits with shape a directed set $I$, i.e., for any convergent diagram $\mf{F}:I\to \mathfrak{M}$, 
	\begin{equation*}
	L^{p}(\lim \mf{F})=\colim L^{p}\mf{F}.
	\end{equation*}
\end{theorem}

\begin{proof}
	Let us consider the family of subsets of $X_{\infty}$ given by
	\begin{equation*}
	\mathcal{R}:=\{\pi_{i}^{-1}(B): B\in\mathcal{B}_{i}, \ i\in I\}.
	\end{equation*}
	A routine calculation shows that $\mathcal{R}$ is an algebra of subsets. Moreover, by definition, $\mathcal{B}_{\infty}=\sigma(\mathcal{R})$. By Cohn \cite[Le. 3.4.6]{C} it follows that
	\begin{equation*}
	\text{span}\{\mathbbm{1}_{R}:R\in\mathcal{R}\}
	\end{equation*}
	is dense in $L^{p}(X_{\infty},\mu_{\infty})$ where $\mathbbm{1}_{R}:X_{\infty}\to\R$ stands for the characteristic function of the subset $R\in\mc{R}$. Given $R=\pi^{-1}_{i}(B)$ for some $i\in I$ and $B\in\mathcal{B}_{i}$, we can write
	\begin{equation*}
	\mathbbm{1}_{R}=\mathbbm{1}_{B}\circ\pi_{i},
	\end{equation*}
	and from this we obtain the set  inclusion
	\begin{equation}
		\label{dens}
	\text{span}\{\mathbbm{1}_{R}:R\in\mathcal{R}\}\subset\bigcup_{i\in I}\pi^{\ast}_{i}(L^{p}(X_{i},\mu_{i})).
	\end{equation}
	Consequently, the space $\bigcup_{i\in I}\pi^{\ast}_{i}(L^{p}(X_{i},\mu_{i}))$ is dense in $L^{p}(X_{\infty},\mu_{\infty})$ and by Lemma \ref{L2.1} the cocone $(L^{p}(X_{\infty},\mu_{\infty}),\pi^{\ast}_{i})$ is the colimit of $(L^{p}(X_{i},\mu_{i}),\pi_{ij}^{\ast})$.
\end{proof}

We proceed to identify the space $L^{p}(X_{\infty},\mu_{\infty})$ intrinsically in terms of the spaces $L^{p}(X_{i},\mu_{i})$. Let $(X_{i},\mu_{i},\pi_{ij})$ be a convergent diagram in $\mathfrak{M}$ with limit $(X_{\infty},\mu_{\infty},\pi_{i})$. A sequence $(f_{i})_{i\in I}\in \prod_{i\in I}L^{p}(X_{i},\mu_{i})$ is said to be co-Cauchy if for every $\varepsilon>0$, there exists $\ell\in I$ such that 
\begin{equation}
\label{coC}
\|\pi_{ij}^{\ast}(f_{i})-f_{j}\|_{L^{p}(\mu_{j})}<\varepsilon, \quad \forall (i,j)\in I\times I,  \quad \ell \leq i\leq j.
\end{equation}
Let $\mf{F}:I\to\mf{M}$ be the functor representing the diagram $(X_{i},\mu_{i},\pi_{ij})$. We consider the product space
\begin{equation*}
\mathscr{L}^{p}(\mf{F}):=\Big\{(f_{i})_{i\in I}\in \prod_{i\in I}L^{p}(X_{i},\mu_{i}): (f_{i})_{i\in I} \text{ is co-Cauchy}\Big\}\Big\slash \sim,
\end{equation*}
where two sequences $(f_{i})_{i\in I}, (g_{i})_{i\in I}\in \prod_{i\in I}L^{p}(X_{i},\mu_{i})$ are related, $(f_{i})_{i\in I}\sim(g_{i})_{i\in I}$, if by definition,
\begin{equation*}
\lim_{i\in I} \, \|f_{i}-g_{i}\|_{L^{p}(\mu_{i})}=0,
\end{equation*}
where the limit is understand in the sense of net convergence. By the elementary properties of the limit, it becomes apparent that $\sim$ is an equivalence relation. We define a norm on $\mathscr{L}^{p}(\mf{F})$ by
\begin{equation*}
\|(f_{i})_{i\in I}\|_{\mathscr{L}}:=\lim_{i\in I} \, \|f_{i}\|_{L^{p}(\mu_{i})}, \quad (f_{i})_{i\in I}\in \mathscr{L}^{p}(\mf{F}).
\end{equation*}
The convergence of the net $(\|f_{i}\|_{L^{p}(\mu_{i})})_{i\in I}\subset \R$ follows from the co-Cauchy property. Indeed, given $\varepsilon>0$, choose $\ell\in I$ such that \eqref{coC} holds for $\varepsilon/2$ and take $i,r\geq \ell$. Since $I$ is directed, there exists $j\in I$ such that $i,r\leq j$ and consequently 
$$\ell\leq i\leq j, \quad \ell\leq r\leq j.$$
Hence, by \eqref{coC} we infer that
\begin{equation}
		\label{Eqp}
\begin{split}
\left|\|f_{i}\|_{L^{p}(\mu_{i})}-\|f_{r}\|_{L^{p}(\mu_{r})}\right|&=\left|\|\pi^{\ast}_{ij}(f_{i})\|_{L^{p}(\mu_{j})}-\|\pi^{\ast}_{rj}(f_{r})\|_{L^{p}(\mu_{j})}\right|\\
&\leq \| \pi^{\ast}_{ij}(f_{i})-\pi^{\ast}_{rj}(f_{r})\|_{L^{p}(\mu_{j})}\\
&\leq \|\pi_{ij}^{\ast}(f_{i})-f_{j}\|_{L^{p}(\mu_{j})}+\|\pi^{\ast}_{rj}(f_{r})-f_{j}\|_{L^{p}(\mu_{j})}<\varepsilon.
\end{split}
\end{equation}
This shows that the net $(\|f_{i}\|_{L^{p}(\mu_{i})})_{i\in I}\subset \R$ is Cauchy and hence convergent. The next result identifies the space $\mathscr{L}^{p}(\mf{F})$ with $	L^{p}(X_{\infty},\mu_{\infty})$.
\begin{theorem}
	\label{T3.2}
	Let $(X_{i},\mu_{i},\pi_{ij})$ be a convergent diagram in $\mathfrak{M}$ with limit $(X_{\infty},\mu_{\infty},\pi_{i})$. Then for each $1\leq p<\infty$, the $L^{p}$-space $L^{p}(X_{\infty},\mu_{\infty})$ is isometrically isomorphic to $\mathscr{L}^{p}(\mf{F})$. Symbolically, 
	\begin{equation*}
	L^{p}(X_{\infty},\mu_{\infty})\simeq\mathscr{L}^{p}(\mf{F}).
	\end{equation*}
\end{theorem}

\begin{proof}
	For $1\leq p<\infty$, let us consider the operator
	\begin{equation*}
	\mf{I}_{p}:\mathscr{L}^{p}(\mf{F})\longrightarrow L^{p}(X_{\infty},\mu_{\infty}), \quad (f_{i})_{i\in I}\mapsto \lim_{i\in I} \pi_{i}^{\ast}(f_{i}).
	\end{equation*}
	We start by showing that the operator $\mf{I}_{p}$ is well defined. Let $(f_{i})_{i\in I}\in\mathscr{L}^{p}(\mf{F})$. For any given $\varepsilon>0$, taking $\ell\in I$ satisfying \eqref{coC}, we obtain
	\begin{align*}
	\|\pi^{\ast}_{i}(f_{i})-\pi_{j}^{\ast}(f_{j})\|_{L^{p}(\mu_{\infty})}&=\|(\pi^{\ast}_{j}\circ\pi^{\ast}_{ij})(f_{i})-\pi_{j}^{\ast}(f_{j})\|_{L^{p}(\mu_{\infty})}\\
	&=\|\pi_{ij}^{\ast}(f_{i})-f_{j}\|_{L^{p}(\mu_{j})}<\varepsilon, \quad \forall (i,j)\in I\times I,  \quad \ell \leq i\leq j.
	\end{align*}
	Hence $(\pi_{i}^{\ast}(f_{i}))_{i\in I}$ is a Cauchy net in $L^{p}(X_{\infty},\mu_{\infty})$, from which follows its convergence. On the other hand, given two related elements $(f_{i})_{i\in I}\sim (g_{i})_{i\in I}$, we have $\mf{I}_{p}(f_{i})_{i\in I}=\mf{I}_{p}(g_{i})_{i\in I}$ since
	\begin{align*}
	\|\mf{I}_{p}(f_{i})_{i\in I}-\mf{I}_{p}(g_{i})_{i\in I}\|_{L^{p}(\mu_{\infty})}&=\lim_{i\in I} \, \|\pi_{i}^{\ast}(f_{i})-\pi^{\ast}_{i}(g_{i})\|_{L^{p}(\mu_{\infty})}\\
	&=\lim_{i\in I} \, \|f_{i}-g_{i}\|_{L^{p}(\mu_{i})}=0.
	\end{align*}
	This proves that $\mf{I}_{p}$ is well defined. By an analogous computation we obtain that $\mf{I}_{p}$ is an isometry:
	\begin{equation*}
	\|\mf{I}_{p}(f_{i})_{i\in I}\|_{L^{p}(\mu_{\infty})}=\lim_{i\in I} \, \|\pi_{i}^{\ast}(f_{i})\|_{L^{p}(\mu_{\infty})}=\lim_{i\in I} \, \|f_{i}\|_{L^{p}(\mu_{i})}=\|(f_{i})_{i\in I}\|_{\mathscr{L}}.
	\end{equation*}
	Finally, we prove that the inverse of $\mf{I}_{p}$ is given by
	\begin{equation*}
	\mf{R}_{p}: L^{p}(X_{\infty},\mu_{\infty})\longrightarrow \mathscr{L}^{p}(\mf{F}), \quad f\mapsto ([\pi_{i}^{\ast}]^{-1}(\mathbb{E}_{i}(f)))_{i\in I},
    \end{equation*}
    where $\mathbb{E}_{i}(f):=\mathbb{E}[f \, | \, \mc{B}^{i}]$ is the conditional expectation of $f\in L^{p}(X_{\infty},\mu_{\infty})$ conditioned by the $\sigma$-algebra 
    $$\mc{B}^{i}:=\{\pi_{i}^{-1}(B):B\in\mc{B}_{i}\}, \quad i\in I.$$
	Let us show that $\mf{R}_{p}$ is well defined. Let $\mc{B}_{\R}$ be the Borel $\sigma$-algebra of $\R$ with the usual topology. Since the maps $\pi_{i}:(X_{\infty},\mc{B}^{i})\to (X_{i},\mc{B}_{i})$, $\mathbb{E}_{i}(f): (X_{\infty},\mc{B}^{i})\to (\R,\mc{B}_{\R})$ are measurable, by the Doob–Dynkin lemma (see for instance Kallenberg \cite[Le. 1.14]{OlKa}), there exists a measurable function $g_{i}:(X_{i},\mc{B}_{i})\to (\R,\mc{B}_{\R})$ such that $\mathbb{E}_{i}(f)=g_{i}\circ \pi_{i}$. In other words, there exists a measurable function $g_{i}:(X_{i},\mc{B}_{i})\to (\R,\mc{B}_{\R})$ making the diagram of Figure \ref{F990} commutative.

		% Figure environment removed

	
	\noindent By the change of variable formula for measures, we obtain
	$$\int_{X_{\infty}}|\mathbb{E}_{i}(f)|^{p}\, \mathrm{d}\mu=\int_{X_{\infty}}|g_{i}\circ \pi_{i}|^{p} \, \mathrm{d}\mu=\int_{X_{i}}|g_{i}|^{p} \, \mathrm{d}(\pi_{i})_{\ast}\mu_{\infty}=\int_{X_{i}}|g_{i}|^{p} \, \mathrm{d}\mu_{i}.$$
	Consequently $g_{i}\in L^{p}(X_{i},\mu_{i})$ and $\mathbb{E}_{i}(f)=\pi_{i}^{\ast}(g_{i})$. This implies that $$[\pi_{i}^{\ast}]^{-1}(\mathbb{E}_{i}(f))=g_{i}\in L^{p}(X_{i},\mu_{i}), \quad i\in I.$$ 
	Let $\varepsilon>0$. Since by \eqref{dens} the subspace $\bigcup_{i\in I}\pi^{\ast}_{i}(L^{p}(X_{i},\mu_{i}))$ is dense in $L^{p}(X_{\infty},\mu_{\infty})$, there exits $\ell\in I$ and $h_{\ell}\in \pi^{\ast}_{\ell}(L^{p}(X_{\ell},\mu_{\ell}))$, such that
	$$\|h_{\ell}-f\|_{L^{p}(\mu_{\infty})}<\frac{\varepsilon}{2}.$$
	Since $\mc{B}^{\ell}\subset \mc{B}^{i}$ for each $\ell\leq i$, if $h_{\ell}$ is $\mc{B}^{\ell}$-measurable, then $h_{\ell}$ is also $\mc{B}^{i}$-measurable.
	By the properties of the conditional expectation, this implies that $\mathbb{E}_{i}(h_{\ell})=h_{\ell}$ for each $\ell\leq i$.
	Consequently, we infer that
	\begin{align*}
		\|\mathbb{E}_{i}(f)-f\|_{L^{p}(\mu_{\infty})}&\leq \|\mathbb{E}_{i}(f)-\mathbb{E}_{i}(h_{\ell})\|_{L^{p}(\mu_{\infty})}+\|h_{\ell}-f\|_{L^{p}(\mu_{\infty})} \\
		&\leq 2\|h_{\ell}-f\|_{L^{p}(\mu_{\infty})}<\varepsilon.
	\end{align*}
    The last inequality follows from the fact that the conditional expectation is a norm one projection. This proves that 
    \begin{equation}
    	\label{Eq.3}
    \lim_{i\in I} \ \mathbb{E}_{i}(f)=f \quad \text{ in } L^{p}(X_{\infty},\mu_{\infty}),
    \end{equation}
    and therefore $([\pi_{i}^{\ast}]^{-1}(\mathbb{E}_{i}(f)))_{i\in I}$ is co-Cauchy. Hence $\mf{R}_{p}$ is well-defined. Finally by \eqref{Eq.3} we deduce on the one hand,
    \begin{align*}
     	(\mf{I}_{p}\circ\mf{R}_{p})(f)=\lim_{i\in I} \, \pi_{i}^{\ast}[\pi_{i}^{\ast}]^{-1}(\mathbb{E}_{i}(f))=\lim_{i\in I}  \, \mathbb{E}_{i}(f)=f,
    \end{align*}
and on the other hand,
    \begin{align*}
    	(\mf{R}_{p}\circ \mf{I}_{p})(f_{i})_{i\in I}&=\left([\pi_{j}^{\ast}]^{-1}\left[\mathbb{E}_{j}(\lim_{i\in I} \, \pi_{i}^{\ast}(f_{i}))\right]\right)_{j\in I} =\left([\pi_{j}^{\ast}]^{-1}\left[\lim_{i\in I}\, \mathbb{E}_{j}( \pi_{i}^{\ast}(f_{i}))\right]\right)_{j\in I} \\
    	&=\left([\pi_{j}^{\ast}]^{-1}\pi_{j}^{\ast}(f_{j}))\right)_{j\in I} = (f_{j})_{j\in I}.
    \end{align*}
	This proves that $\mf{I}_{p}^{-1}=\mf{R}_{p}$ and concludes the proof.
\end{proof}
Our next result uses Theorem \ref{T3.2} to give another realization of the colimit of the codiagram $(L^{p}(X_{i},\mu_{i}),\pi_{ij}^{\ast})$. To state the result, we define the isometries $(\psi_{i})_{i\in I}$, $\psi_{i}:L^{p}(X_{i},\mu_{i})\to \mathscr{L}^{p}(\mf{F})$, by
\begin{equation*}
\psi_{i}(f_{i})=(\pi^{\ast}_{ij}(f_{i}))_{j\geq i}, \quad (\pi^{\ast}_{ij}(f_{i}))_{j\geq i}:=\left\{\begin{array}{ll}
\pi^{\ast}_{ij}(f_{i}) & \text{ if } i\leq j \\
0 & \text{ else.}
\end{array}\right.
\end{equation*}
Then it is easy to see that $(\mathscr{L}^{p}(\mf{F}),\psi_{i})$ defines a cocone in $\mathfrak{Ban}$.
\begin{theorem}
	\label{T3.3}
	Let $(X_{i},\mu_{i},\pi_{ij})$ be a convergent diagram in $\mathfrak{M}$ with limit $(X_{\infty},\mu_{\infty},\pi_{i})$. Then for each $1\leq p<\infty$, the cocone $(\mathscr{L}^{p}(\mf{F}),\psi_{i})$ defines a realization of the colimit of $(L^{p}(X_{i},\mu_{i}),\pi_{ij}^{\ast})$. 
\end{theorem}
\begin{proof}
	Let $L^{p}\mf{F}:I\to\mf{Ban}$ be the functor associated to the codiagram $(L^{p}(X_{i},\mu_{i}),\pi_{ij}^{\ast})$. Since the colimit is unique up to a isomorphism on the category of cocones $\mf{K}^{\ast}(L^{p}\mf{F})$, it is enough to prove that the cocone $(\mathscr{L}^{p}(\mf{F}),\psi_{i})$ is isomorphic to $(L^{p}(X_{\infty},\mu_{\infty}),\pi^{\ast}_{i})$ in $\mf{K}^{\ast}(L^{p}\mf{F})$. By the proof of Theorem \ref{T3.2}, the map 
	\begin{equation*}
	\mathfrak{I}_{p}:\mathscr{L}^{p}(\mf{F})\longrightarrow L^{p}(X_{\infty},\mu_{\infty}), \quad (f_{i})_{i\in I}\mapsto \lim_{i\in I} \, \pi^{\ast}_{i}(f_{i}),
	\end{equation*}
	is an isometric isomorphism with inverse
		\begin{equation*}
		\mf{I}^{-1}_{p}: L^{p}(X_{\infty},\mu_{\infty})\longrightarrow \mathscr{L}^{p}(\mf{F}), \quad f\mapsto ([\pi_{i}^{\ast}]^{-1}(\mathbb{E}_{i}(f)))_{i\in I}.
	\end{equation*}
	Therefore the morphism $\mf{I}_{p}$ defines an isomorphism in $\mf{Ban}$. To prove that it defines an isomorphism of cocones, we need to verify the commutativity of the diagram of Figure \ref{F99}, as well as the commutativity of the same diagram replacing the arrow of $\mf{I}_{p}$ with the opposite arrow $\mf{I}_{p}^{-1}$.

		% Figure environment removed

	\noindent Take $f_{i}\in L^{p}(X_{i},\mu_{i})$, then by a simple computation we obtain
	\begin{equation*}
	(\mathfrak{I}_{p}\circ \psi_{i})(f_{i})=\lim_{r}h_{r}, \quad h_{r}:=\left\{
	\begin{array}{ll}
	\pi^{\ast}_{i}(f_{i}) & \text{ if } i\leq r \\
	0 & \text{ else, } 
	\end{array}
	\right.
	\end{equation*}
	and thus $(\mathfrak{I}_{p}\circ \psi_{i})(f_{i})=\pi^{\ast}_{i}(f_{i})$. Hence the diagram of Figure \ref{F99} commutes. Conversely, given $f_{i}\in L^{p}(X_{i},\mu_{i})$, we have
	\begin{equation*}
		(\mf{I}_{p}^{-1}\circ\pi_{i}^{\ast})(f_{i})=\left([\pi_{j}^{\ast}]^{-1}\mathbb{E}_{j}( \pi_{i}^{\ast}(f_{i}))\right)_{j\in I}=\left([\pi_{j}^{\ast}]^{-1}[ \pi_{i}^{\ast}(f_{i})]\right)_{j\in I}=(\pi_{ij}^{\ast}(f_{i}))_{j\geq i},
	\end{equation*}
    where the last equality follows from the identity $\pi_{j}^{\ast}\circ\pi_{ij}^{\ast}=\pi^{\ast}_{i}$ which is satisfied for each $i\leq j$.
	Thus $(\mf{I}_{p}^{-1}\circ\pi_{i}^{\ast})(f_{i})=\psi_{i}(f_{i})$ and the proof is completed.
\end{proof}
As a rather direct application of the isometric property of $\mathfrak{I}_{p}$, we obtain a integral limit approximation. Indeed, if $f\in L^{p}(X_{\infty},\mu_{\infty})$, there exists an element $(f_{i})_{i\in I}\in\mathscr{L}^{p}(\mf{F})$ such that $\|f\|_{L^{p}(\mu_{\infty})}=\|(f_{i})_{i}\|_{\mathscr{L}}$. Therefore, we have
\begin{equation*}
\int_{X_{\infty}}|f|^{p}\, \mathrm{d}\mu_{\infty}=\lim_{i\in I}\int_{X_{i}}|f_{i}|^{p}\, \mathrm{d}\mu_{i}.
\end{equation*}
\noindent Moreover, this element can be explicitly given by 
$$f_{i}:=[\pi_{i}^{\ast}]^{-1}(\mathbb{E}_{i}(f))\in L^{p}(X_{i},\mu_{i}), \quad i\in I.$$
Furthermore, if we take into account that given $f\in L^{1}(X_{\infty},\mu_{\infty})$, we can write it as $f=f^{+}-f^{-}$ with $f^{+}, f^{-}$ positive and $f^{+},f^{-}\in L^{1}(X_{\infty},\mu_{\infty})$,  we get the following result.

\begin{theorem}\label{C4.4}
	Let $(X_{i},\mu_{i},\pi_{ij})$ be a convergent diagram in $\mathfrak{M}$ with limit $(X_{\infty},\mu_{\infty},\pi_{i})$. Then for each $f\in L^{1}(X_{\infty},\mu_{\infty})$, there exists $(f_{i})_{i\in I}\in\prod_{i\in I}L^{1}(X_{i},\mu_{i})$ such that 
	\begin{equation*}
	\int_{X_{\infty}}f \, \mathrm{d}\mu_{\infty}=\lim_{i\in I}\int_{X_{i}}f_{i}\, \mathrm{d}\mu_{i}.
	\end{equation*}
Moreover the sequence $(f_{i})_{i\in I}$ can be chosen to be $f_{i}=[\pi_{i}^{\ast}]^{-1}(\mathbb{E}_{i}(f))$ for each $i\in I$.
\end{theorem}

\begin{comment}
Sometimes, it is useful to restrict the directed set to some directed subset $J\subset I$. Let us show for which directed subsets $J\subset I$ the colimit is preserved. Let $(X_{i},\mu_{i},\pi_{ij})$ be a convergent diagram in $\mathfrak{M}$ represented by the functor $F:I\to \mathfrak{M}$. We use the notation $(X_{i},\mu_{i},\pi_{ij})_{J}$ to refer to the diagram in $\mathfrak{M}$ given by the restricted functor $F:J\to \mathfrak{M}$. Clearly, if $(X_{i},\mu_{i},\pi_{ij})$ is convergent with limit $(X_{\infty},\mu_{\infty},\pi_{i})$, the diagram $(X_{i},\mu_{i},\pi_{ij})_{J}$ is also convergent with limit $(X_{\infty},\mu_{\infty},\pi_{j})_{J}$. We analogously define the codiagram $(L^{p}(X_{i},\mu_{i}),\pi^{\ast}_{ij})_{J}$.
We consider also the space
\begin{equation*}
\mathscr{L}^{p}_{J}(X_{i},\mu_{i},\pi_{ij}):=\Big\{(f_{j})_{j\in J}\in \prod_{j\in J}L^{p}(X_{j},\mu_{j}): (f_{j})_{j\in J} \text{ is co-Cauchy}\Big\}\Big\slash \sim,
\end{equation*}
under the natural definitions.

\begin{theorem}
	\label{T3.5}
	Let $(X_{i},\mu_{i},\pi_{ij})$ be a convergent diagram in $\mathfrak{M}$ with limit $(X_{\infty},\mu_{\infty},\pi_{i})$ and let $J\subset I$ be a directed subset such that
	\begin{equation}
	\label{eq3.2}
	\sigma(\{\pi_{j}^{-1}(B):j\in J, \ B\in\mathcal{B}_{j}\})=\mc{B}_{\infty}.
	\end{equation}
	Then, for each $1\leq p<\infty$, the cocone $(L^{p}(X_{\infty},\mu_{\infty}),\pi^{\ast}_{j})_{J}$ is the colimit of $(L^{p}(X_{i},\mu_{i}),\pi_{ij}^{\ast})_{J}$. Moreover, the $L^{p}$-space $L^{p}(X_{\infty},\mu_{\infty})$ is isometrically isomorphic to $\mathscr{L}^{p}_{J}(X_{i},\mu_{i},\pi_{ij})$ via the isometric isomorphism
	\begin{equation}
	\label{eq3.3}
	\mf{F}_{p}:\mathscr{L}^{p}_{J}(X_{i},\mu_{i},\pi_{ij})\longrightarrow L^{p}(X_{\infty},\mu_{\infty}), \quad (f_{j})_{j\in J}\mapsto \lim_{j\in J}\pi^{\ast}_{j}(f_{j}),
	\end{equation}
     with inverse
     \begin{equation}
	\label{eq3.3.2}
	\mf{F}^{-1}_{p}: L^{p}(X_{\infty},\mu_{\infty})\longrightarrow \mathscr{L}^{p}_{J}(X_{i},\mu_{i},\pi_{ij}), \quad f \mapsto  ([\pi_{j}^{\ast}]^{-1}(\mathbb{E}_{j}(f)))_{j\in J}.
     \end{equation}
\end{theorem}

\begin{proof}
	Let us consider the families of subsets of $X_{\infty}$ given by
	\begin{equation*}
	\mathcal{R}:=\{\pi_{i}^{-1}(B): B\in\mathcal{B}_{i}, \ i\in I\}, \quad \mathcal{R}_{J}:=\{\pi_{j}^{-1}(B): B\in\mathcal{B}_{j}, \ j\in J\}.
	\end{equation*}
	A direct calculation shows that $\mathcal{R}$ and $\mathcal{R}_{J}$ are algebras of subsets. Moreover, by definition, $\mathcal{B}_{\infty}=\sigma(\mathcal{R})$ and by \eqref{eq3.2}, $\mathcal{B}_{\infty}=\sigma(\mathcal{R}_{J})$. Hence arguing as in the proof of Theorem \ref{T3.1}, it follows that the space $\bigcup_{j\in J}\pi^{\ast}_{j}(L^{p}(X_{j},\mu_{j}))$ is dense in $L^{p}(X_{\infty},\mu_{\infty})$ and by Lemma \ref{L2.1}, the cocone $(L^{p}(X_{\infty},\mu_{\infty}),\pi^{\ast}_{j})_{J}$ is the colimit of $(L^{p}(X_{i},\mu_{i}),\pi_{ij}^{\ast})_{J}$. The proof of the identification $L^{p}(X_{\infty},\mu_{\infty})\simeq \mathscr{L}^{p}_{J}(X_{i},\mu_{i},\pi_{ij})$ via the isomorphism \eqref{eq3.3}--\eqref{eq3.3.2}, follows the same steeps of the proof of Theorem \ref{T3.2}.
\end{proof}

\end{comment}

\section{Measures on vector spaces}\label{Se4}

In this section we apply the abstract setting of Section \ref{Se3} to measures on vector spaces. Let $E$ be a real vector space. We denote the algebraic dual of $E$ by $E^{a}$, i.e., $E^{a}$ is the space consisted on linear functionals $\ell:E\to \R$. There is a natural embedding of $E$ into $[E^{a}]^{a}$ given by $\xi\mapsto \xi_{\ast}$, where $\xi_{\ast}$ is the linear functional on $E^{a}$ defined by $\ell\mapsto \ell(\xi)$ for each $\ell\in E^{a}$. We introduce the $\sigma$-algebra $\mc{B}_{E}$ on $E^{a}$ through
$$\mc{B}_{E}:=\sigma(\{\xi^{-1}_{\ast}(B):\xi\in E, \ B\in\mc{B}_{\R}\}),
$$
where $\mc{B}_{\R}$ is the Borel $\sigma$-algebra of $\R$. We denote by $\mc{R}$ the class of finite dimensional subspaces of $E$. It forms a directed set under the inclusion. A map $\chi: E \to \C$ is called \textit{characteristic function} if the following conditions are satisfied: 
\begin{enumerate}
\item \textit{Positively defined}: For each integer $n\in \N$, vectors $\xi_{1},\xi_{2},\cdots, \xi_{n}\in E$ and scalars $\alpha_{1},\alpha_{2},\cdots,\alpha_{n}\in\C$, the following positivity condition holds:
$$
\sum_{j,k=1}^{n}\alpha_{j}\bar{\alpha}_{k}\chi(\xi_{j}-\xi_{k})\geq 0.
$$
\item \textit{Continuity}: For each $R\in \mc{R}$, the restricted function $\chi|_{R}:R\to\C$ is continuous, where we are considering the usual Euclidean topology on $R$. 
\end{enumerate}
Given a measure $\mu$ on $(E^{a},\mc{B}_{E})$, the map $\chi:E\to \C$ defined by
\begin{equation}
	\label{E22}
	\chi(\xi)=\int_{E^{a}}e^{i\ell(\xi)} \, \mathrm{d}\mu(\ell), \quad \xi\in E,
\end{equation}
defines a characteristic function on $E$. If $E$ is finite dimensional, Bochner theorem (see, e.g., Yamasaki \cite[Th. 16.1]{Y})  states the converse, that is, to every characteristic function $\chi$, there corresponds a unique Borel measure $\mu$ on $E^{a}$ satisfying equation \eqref{E22}.
This result can be generalized to infinite dimensions via the Bochner--Minlos theorem \cite[Th. 16.2]{Y}. Indeed, to every characteristic function $\chi$, there corresponds a unique measure $\mu$ on the measurable space $(E^{a},\mc{B}_{E})$ satisfying \eqref{E22}. Let us give an idea of the proof of this result as it will be useful for further considerations. For every finite dimensional subspace $R\in\mc{R}$ of  $E$, the restricted characteristic function $\chi|_{R}:R\to\C$ is a finite dimensional characteristic function. By Bochner theorem, there exists a Borel measure $\mu_{R}:\mc{B}_{R}\to[0,+\infty]$ on $R^{a}$ satisfying \eqref{E22}. Then, the triple $(R^{a},\mu_{R},\pi_{RF})$, where for each $R,F\in\mc{R}$ with $R\leq F$, $\pi_{RF}$ are the measurable functions defined by
$$
\pi_{RF}:(F^{a},\mu_{F})\longrightarrow (R^{a},\mu_{R}), \quad \ell_{F}\mapsto \ell_{F}|_{R}, \quad R\leq F,$$
is a diagram on $\mf{M}$ indexed by the directed set $\mc{R}$. We represent this diagram via the functor $\mf{R}:\mc{R}\to \mf{M}$. According to Yamasaki \cite[Th. 15.1]{Y}, the diagram $\mf{R}$ is convergent with limit $(\mc{R}_{\infty},\mc{B}_{\infty},\mu_{\infty},\pi_{R})$, where
$$
\mc{R}_{\infty}:=\Big\{(\ell_{R})_{R}\in \prod_{R\in\mc{R}}R^{a}: \pi_{RF}(\ell_{F})=\ell_{R}, \ R\leq F\Big\},
$$
and the maps $\pi_{F}:\mc{R}_{\infty}\to F^{a}$ are defined by $\pi_{F}(\ell_{R})_{R}:=\ell_{F}$ for each $F\in\mc{R}$.
This space is equipped with the $\sigma$-algebra $\mc{B}_{\infty}$ defined through
$$ \mc{B}_{\infty}:=\sigma(\{\pi_{R}^{-1}(B): B\in \mc{B}_{R}, \ R\in \mc{R}\}),$$
and $\mu_{\infty}:\mc{B}_{\infty}\to [0,+\infty]$ is a measure satisfying $(\pi_{R})_{\ast}\mu_{\infty}=\mu_{R}$ for each $R\in\mc{R}$. On a further step, it is shown that the map $$I:(E^{a},\mc{B}_{E})\longrightarrow (\mc{R}_{\infty},\mc{B}_{\infty}), \quad \ell\mapsto (\ell|_{R})_{R\in\mc{R}},$$
is an isomorphism in $\mf{M}_{0}$ with inverse
$$I^{-1}:(\mc{R}_{\infty},\mc{B}_{\infty})\longrightarrow(E^{a},\mc{B}_{E}), \quad (\ell_{R})_{R\in\mc{R}}\mapsto \ell,$$
where $\ell\in E^{a}$ is defined by $\ell(\xi):=\ell_{R}(\xi)$ where $R:=\text{span}\{\xi\}$.
Hence, defining the measure 
$$\mu:\mc{B}_{E}\longrightarrow[0,+\infty], \quad \mu(B):=\mu_{\infty}(I(B)),$$ we obtain that $I:(E^{a},\mu)\to (\mc{R}_{\infty},\mu_{\infty})$ is an isomorphism in $\mf{M}$. It can be shown that this measure $\mu$ satisfies \eqref{E22}. We refer the reader to Yamasaki \cite[\S 15--16]{Y} for more details concerning the construction of the measure $\mu$.
\par We introduce the cone $(E^{a},\mu,\varphi_{R})$ of $\mf{R}$, where for each $R\in\mc{R}$, the morphism $\varphi_{R}:E^{a}\to R^{a}$ is given by 
\begin{equation}
	\label{Hom3}
\varphi_{R}:E^{a}\longrightarrow R^{a}, \quad \varphi_{R}(\ell):=\ell|_{R}.
\end{equation}
An easy computation shows that 
$$I:(E^{a},\mu,\varphi_{R})\longrightarrow (\mc{R}_{\infty},\mu_{\infty},\pi_{R}),$$
 is an isomorphism of cones in $\mf{K}(\mf{R})$. Moreover since functors preserve cone isomorphisms, we obtain that the cocones $(L^{p}(E^{a},\mu),\varphi^{\ast}_{R})$ and $(L^{p}(\mc{R}_{\infty},\mu_{\infty}),\pi^{\ast}_{R})$ are isomorphic in the category $\mf{K}^{\ast}(L^{p}\mf{R})$ via the application of the $L^{p}$ functor to $I$:
$$I^{\ast}:(L^{p}(\mc{R}_{\infty},\mu_{\infty}),\pi^{\ast}_{R})\longrightarrow (L^{p}(E^{a},\mu),\varphi^{\ast}_{R}), \quad f\mapsto f\circ I.$$ 
We start by applying the results of Section \ref{Se3} to this particular case.
\begin{theorem}
	Let $E$ be a real vector space, $\chi:E\to \C$ a characteristic function and $\mu$ be the corresponding measure on $(E^{a},\mc{B}_{E})$. Then the cone $(E^{a},\mu,\varphi_{R})$ is a realization of the limit of the convergent diagram $(R^{a},\mu_{R},\pi_{RF})$. Moreover for each $1\leq p<\infty$,
	$$\colim \ (L^{p}(R^{a},\mu_{R}),\pi^{\ast}_{RF})= (L^{p}(E^{a},\mu), \varphi^{\ast}_{R}).$$ 
\end{theorem}
\begin{proof}
	By the above construction, the cone $(\mc{R}_{\infty},\mu_{\infty},\pi_{R})$ is the limit of $(R^{a},\mu_{R},\pi_{RF})$. Since the morphism $I: (E^{a},\mu,\varphi_{R})\to (\mc{R}_{\infty},\mu_{\infty},\pi_{R})$ is a cone isomorphism and the limit is unique (up to isomorphisms) in the category of cones $\mf{K}(\mf{R})$, it follows that $(E^{a},\mu,\varphi_{R})$ defines a realization of the limit of $(R^{a},\mu_{R},\pi_{RF})$. On the other hand, Theorem \ref{T3.1} applied to the diagram $(R^{a},\mu_{R},\pi_{RF})$ assures that $(L^{p}(E^{a},\mu),\varphi^{\ast}_{R})$ is the colimit of the codiagram $(L^{p}(R^{a},\mu_{R}),\pi^{\ast}_{RF})$.
\end{proof}
Let us consider for each $1\leq p < \infty$, the normed space $\mathscr{L}^{p}(\mf{R})$. For the convenience of the reader, let us recall the definition of $\mathscr{L}^{p}(\mf{R})$. This space is given by
\begin{equation*}
	\mathscr{L}^{p}(\mf{R}):=\Big\{(f_{R})_{R\in \mc{R}}\in \prod_{R\in \mc{R}}L^{p}(R^{a},\mu_{R}): (f_{R})_{R\in \mc{R}} \text{ is co-Cauchy}\Big\}\Big\slash \sim,
\end{equation*}
where recall that two sequences $(f_{R})_{R\in \mc{R}}, (g_{R})_{R\in \mc{R}}\in \prod_{R\in \mc{R}}L^{p}(R^{a},\mu_{R})$ are related, $(f_{R})_{R\in \mc{R}}\sim(g_{R})_{R\in \mc{R}}$, if by definition,
\begin{equation*}
	\lim_{R\in \mc{R}} \, \|f_{R}-g_{R}\|_{L^{p}(\mu_{R})}=0.
\end{equation*}
The norm on $\mathscr{L}^{p}(\mf{R})$ is given by
\begin{equation*}
	\|(f_{R})_{R\in \mc{R}}\|_{\mathscr{L}}:=\lim_{R\in \mc{R}} \, \|f_{R}\|_{L^{p}(\mu_{R})}, \quad (f_{R})_{R\in \mc{R}}\in \mathscr{L}^{p}(\mf{R}).
\end{equation*}
In order to define a cocone structure on $\mathscr{L}^{p}(\mf{R})$, we introduce the morphisms
$(\psi_{R})_{R\in \mc{R}}$, $\psi_{R}:L^{p}(R^{a},\mu_{R})\to \mathscr{L}^{p}(\mf{R})$, by
\begin{equation*}
	\psi_{R}(f_{R})=(g_{F})_{F\in \mc{R}}, \quad g_{F}=(\pi^{\ast}_{RF}(f_{R}))_{F\geq R}:=\left\{\begin{array}{ll}
		\pi^{\ast}_{RF}(f_{R}) & \text{ if } R\leq F \\
		0 & \text{ else }
	\end{array}\right.
\end{equation*}
Then it is easy to see that $(\mathscr{L}^{p}(\mf{R}),\psi_{R})$ is a cocone in $\mathfrak{Ban}$. The following result identifies $L^{p}(E^{a},\mu)$ with $\mathscr{L}^{p}(\mf{R})$.
\begin{theorem}
	\label{T4.2}
	Let $E$ be an infinite dimensional real vector space,  $\chi:E\to \C$ a characteristic function and $\mu$ be the corresponding measure on $(E^{a},\mc{B}_{E})$. Then the cocone $(\mathscr{L}^{p}(\mf{R}),\psi_{R})$ defines another realization of the colimit of $(L^{p}(R^{a},\mu_{R}),\pi^{\ast}_{RF})$ and the morphism
	\begin{equation}
		\label{E4}
	\mf{L}_{p}:(\mathscr{L}^{p}(\mf{R}),\psi_{R})\longrightarrow (L^{p}(E^{a},\mu),\varphi_{R}^{\ast}), \quad (f_{R})_{R}\mapsto \lim_{R\in\mc{R}}\varphi_{R}^{\ast}(f_{R}),
\end{equation}
defines an isomorphism of cocones.
\end{theorem}

\begin{proof}
	By applying Theorem \ref{T3.3} to the diagram $\mf{R}$, we get that  $(\mathscr{L}^{p}(\mf{R}),\psi_{R})$ defines another realization of the colimit of $(L^{p}(R^{a},\mu_{R}),\pi^{\ast}_{RF})$, where the cocone isomorphism is given by
	\begin{equation}
	\label{E}
	\mf{I}_{p}:(\mathscr{L}^{p}(\mf{R}),\psi_{R})\longrightarrow (L^{p}(\mc{R}_{\infty},\mu_{\infty}),\pi^{\ast}_{R}), \quad (f_{R})_{R\in\mc{R}}\mapsto \lim_{R\in\mc{R}} \pi^{\ast}_{R}(f_{R}).
	\end{equation}
     This proves the first part of the theorem. On the other hand, since $I^{\ast}$ is a cocone isomorphism, the composition map $\mf{L}_{p}:=I^{\ast}\circ \mf{I}_{p}$ is also a cocone isomorphism. We illustrate the definition of $\mf{L}_{p}$ in the diagram of Figure \ref{F992}.

     		     	% Figure environment removed

\noindent To conclude the proof, we have to check that the isomorphism $\mf{L}_{p}$ is given by \eqref{E4}. Let $(f_{R})_{R\in\mc{R}}\in \mathscr{L}^{p}(\mf{R})$ and $\ell\in E^{a}$, then
 \begin{align*}
 	\mf{L}_{p}(f_{R})_{R}[\ell]&= (I^{\ast}\circ \mf{I}_{p})(f_{R})_{R}[\ell]=\mf{I}_{p}(f_{R})_{R}[I(\ell)] = \lim_{R\in \mc{R}} \, \pi^{\ast}_{R}(f_{R})[(\ell|_{R})_{R}] \\
 	&=\lim_{R\in \mc{R}} \, f_{R}[\pi_{R}(\ell|_{R})_{R}]= \lim_{R\in \mc{R}} \, f_{R}[\ell|_{R}]=\lim_{R\in\mc{R}} \, (f_{R}\circ \varphi_{R})[\ell]=\lim_{R\in\mc{R}} \, \varphi^{\ast}_{R}(f_{R})[\ell].
 \end{align*}
This concludes the proof.
\end{proof}

\subsection{Topological vector spaces.}\label{S4.1} Let $E$ be a real vector space. If we endow $E$ with a topology so that $E$ is a topological vector space, a natural question arises. Let $\mu$ be the measure on $(E^{a},\mc{B}_{E})$ associated with the characteristic function $\chi:E\to\C$. When does $\mu$ lie on the topological dual space $E'$? A partial answer to this question is contained in the following result whose proof can be found in Yamasaki \cite[Th. 20.1]{Y}.

\begin{theorem}
	Let $E$ be a nuclear topological vector space. If $\chi:E\to\C$ is a continuous characteristic function, the corresponding measure $\mu$ lies on $E'$.
\end{theorem}

Along this section, we will work with a nuclear space $E$ and a given continuous characteristic function $\chi:E\to\C$. Let us denote by $\mc{S}_{E}$ the $\sigma$-algebra defined by
$$\mc{S}_{E}:=\{B\cap E': B\in\mc{B}_{E}\} \ (=\mc{B}_{E}\cap E').$$
Then the measure $\mu:\mc{B}_E\to[0,\infty]$ associated with the characteristic function $\chi$ defines a measure on $(E',\mc{S}_{E})$ given by the restriction $\mu:\mc{S}_{E}\to[0,\infty]$. Subsequently, for each $R\in\mc{R}$, we use also the notation $\varphi_{R}$ to denote the restriction to $E'$ of the morphisms defined in \eqref{Hom3}, that is, $\varphi_{R}:E'\to R^{a}$, $\varphi_{R}(\ell)=\ell|_{R}$ for each $\ell\in E'$. Since $\mu(E^{a}\backslash E')=0$, the restriction operator
$$T:(L^{p}(E^{a},\mu),\varphi_{R}^{\ast})\longrightarrow (L^{p}(E',\mu),\varphi^{\ast}_{R}), \quad f\mapsto f|_{E'},$$
defines an isomorphism of cocones $\mf{K}^{\ast}(L^{p}\mf{R})$. Substituting directly the space $L^{p}(E^{a},\mu)$ by $L^{p}(E',\mu)$ in the previous section, we obtain the following result.

\begin{theorem}
	Let $E$ be a nuclear topological vector space, $\chi:E\to \C$ be a continuous characteristic function and $\mu$ be the corresponding measure on $(E',\mc{S}_{E})$. Then, for each $1\leq p<\infty$,
	$$\colim\ (L^{p}(R^{a},\mu_{R}),\pi^{\ast}_{RF})= (L^{p}(E',\mu), \varphi^{\ast}_{R}).$$ 
	Moreover, the cocone $(\mathscr{L}^{p}(\mf{R}),\psi_{R})$ defines another realization of the colimit and the morphism
	\begin{equation}
		\label{E4.3}
		\mf{L}_{p}:(\mathscr{L}^{p}(\mf{R}),\psi_{R})\longrightarrow (L^{p}(E',\mu),\varphi_{R}^{\ast}), \quad (f_{R})_{R}\mapsto \lim_{R\in\mc{R}}\varphi^{\ast}_{R}(f_{R}).
	\end{equation}
is a cocone isomorphism.
\end{theorem}

To finalize this section, we proceed to simplify the product space $\mathscr{L}^{p}(\mf{R})$ into a product of a countable number of factor spaces. This will be useful from the point of view of the applications. To this end, we will restrict to an infinite dimensional Fréchet nuclear space $E$. Under this hypothesis, the space $E$ is separable. Let $(\xi_{n})_{n\in\N}\subset E$ be a complete system of $E$, that is, $\text{span}\{\xi_{n}:n\in\N\}$ is dense in $E$. Consider the associated finite dimensional subspaces
$$H_{n}:=\text{span}\{\xi_{1},\cdots,\xi_{n}\}, \quad n\in\N,$$
with the associated directed subset $\mc{H}\equiv(H_{n})_{n\in\N}\subset \mc{R}$. Obviously, the restricted functor $\mf{R}:\mc{H}\to\mf{M}$, subsequently denoted by $\mf{H}$, is a diagram of $\mf{M}$. We denote $\mf{H}\equiv (H_{n}^{a},\mu_{n},\pi_{nm})$, where $\mu_{n}:=\mu_{H_{n}}$ and $\pi_{nm}:=\pi_{H_{n}H_{m}}$ for each $n<m$. A simple computation shows that $(L^{p}(E',\mu),\varphi^{\ast}_{n})$ is a cocone in $\mf{K}^{\ast}(L^{p}\mf{H})$, where for each $n\in\N$, $\varphi_{n}:=\varphi_{H_{n}}: E'\to H^{a}_{n}$.
The product space corresponding to $\mf{H}$ is $\mathscr{L}^{p}(\mf{H})$.	It is convenient to recall that this space is defined by
\begin{equation*}
	\mathscr{L}^{p}(\mf{H}):=\Big\{(f_{n})_{n\in \N}\in \prod_{n\in \N}L^{p}(H_{n}^{a},\mu_{n}): (f_{n})_{n\in \N} \text{ is co-Cauchy}\Big\}\Big\slash \sim,
\end{equation*}
where two sequences $(f_{n})_{n\in \N}, (g_{n})_{n\in \N}\in \prod_{n\in \N}L^{p}(H_{n}^{a},\mu_{n})$ are related, $(f_{n})_{n\in \N}\sim(g_{n})_{n\in \N}$, if by definition
\begin{equation*}
	\lim_{n\to\infty}\|f_{n}-g_{n}\|_{L^{p}(\mu_{n})}=0.
\end{equation*}
The norm on $\mathscr{L}^{p}(\mf{H})$ is defined by
\begin{equation*}
	\|(f_{n})_{n\in \N}\|_{\mathscr{L}}:=\lim_{n\to\infty}\|f_{n}\|_{L^{p}(\mu_{n})}, \quad (f_{n})_{n\in \N}\in \mathscr{L}^{p}(\mf{H}).
\end{equation*}
The final result of this section reads as follows.
\begin{theorem}
	\label{T4.5}
	Let $E$ be an infinite dimensional Fréchet nuclear space, $\chi:E\to \C$ a continuous characteristic function and $\mu$ the associated measure on $E'$. Then, for every complete system $(\xi_{n})_{n\in\N}\subset E$, the cocone $(L^{p}(E',\mu),\varphi_{n}^{\ast})$ is the colimit of $(L^{p}(H^{a}_{n},\mu_{n}), \pi^{\ast}_{nm})$. Moreover, the operator
	\begin{equation}
	\label{EqIso}
	\mf{F}_{p}:\mathscr{L}^{p}(\mf{H})\longrightarrow L^{p}(E',\mu), \quad (f_{n})_{n\in\N}\mapsto \lim_{n\to\infty} \varphi^{\ast}_{n}(f_{n}),
	\end{equation}
	defines an isometric isomorphism with inverse
	\begin{equation}
		\label{EqIso2}
		\mf{F}^{-1}_{p}: L^{p}(E',\mu) \longrightarrow \mathscr{L}^{p}(\mf{H}), \quad f \mapsto  ([\varphi_{n}^{\ast}]^{-1}(\mathbb{E}_{n}(f)))_{n\in \N}.
	\end{equation}
\end{theorem}
\begin{proof}
Let us start by proving the identity of $\sigma$-algebras
\begin{equation}
\label{E4.6}
\sigma(\{\varphi_{n}^{-1}(B): n\in \N, \ B\in\mc{B}_{n} \})=\mc{S}_{E},
\end{equation}
where $\mc{B}_{n}$ is the Borel $\sigma$-algebra of $H_{n}^{a}$ and $\varphi_{n}:=\varphi_{H_{n}}:E'\to H^{a}_{n}$ for each $n\in\N$. Note that
$$\mc{S}_{E}=\sigma(\{\xi_{\ast}^{-1}(B):\xi\in E, \ B\in \mc{B}_{\R}\}),$$
and also,
$$\sigma(\{\varphi_{n}^{-1}(B): n\in \N, \ B\in\mc{B}_{n} \})=\sigma(\{(\xi_{n})_{\ast}^{-1}(B): B\in \mc{B}_{\R}, \ n\in\N\}),$$
where for each $\xi\in E$, we are denoting by $\xi_{\ast}$ the functional 
$$\xi_{\ast}:E'\to\R, \quad  \xi_{\ast}(f):=f(\xi).$$ 
Hence, the proof of \eqref{E4.6} reduces to show the following equality of $\sigma$-algebras
\begin{equation}
	\label{E4.7}
	\mc{S}_{1}:=\sigma(\{(\xi_{n})_{\ast}^{-1}(B): B\in \mc{B}_{\R}, \ n\in\N\})=\sigma(\{\xi_{\ast}^{-1}(B):\xi\in E, \ B\in \mc{B}_{\R}\})=:\mc{S}_{2}.
\end{equation}
The inclusion $\mc{S}_{1}\subset\mc{S}_{2}$ is clear. In order to prove $\mc{S}_{2}\subset \mc{S}_{1}$, we have to show that for every $\xi\in E$, the functional $\xi_{\ast}:E^{'}\to\R$ is $\mc{S}_{1}$-measurable. Since $\text{span}\{\xi_{n}:n\in\N\}$ is dense in $E$, there exists a sequence of finite linear combinations $(w_{n})_{n\in\N}\subset E$, 
$$w_{n}:=\sum_{i=1}^{N(n)}\alpha_{i}(n) \xi_{n}, \quad n\in \N,$$
such that $w_{n}\to \xi$ as $n\to\infty$ in the metric topology of $E$ (recall that $E$ is a Fréchet space). Since $(\xi_{n})_{\ast}$ are $\mc{S}_{1}$-measurable and
$$(w_{n})_{\ast}=\sum_{i=1}^{N(n)}\alpha_{i}(n)(\xi_{n})_{\ast}, \quad n\in\N,$$
it follows that $(w_{n})_{\ast}$ are also $\mc{S}_{1}$-measurable. Moreover, $(w_{n})_{\ast}\to \xi_{\ast}$ pointwise as $n\to\infty$ since for each $f\in E'$,
$$\lim_{n\to\infty}(w_{n})_{\ast}(f)=\lim_{n\to\infty}f(w_{n})=f(\xi)=\xi_{\ast}(f).$$
Hence, $\xi_{\ast}$ is $\mc{S}_{1}$-measurable as is the pointwise limit of $\mc{S}_{1}$-measurable functions. This shows that $\mc{S}_{1}=\mc{S}_{2}$ and completes the proof of \eqref{E4.6}. Arguing as in the proof of Theorem \ref{T3.1}, it follows that the space 
$$\bigcup_{n\in\N}\varphi^{\ast}_{n}(L^{p}(H^{a}_{n},\mu_{n}))\subset L^{p}(E',\mu) ,$$
is dense in $L^{p}(E',\mu)$. By Lemma \ref{L2.1}, this implies that $(L^{p}(E',\mu),\varphi_{n}^{\ast})$ is the colimit of $(L^{p}(H^{a}_{n},\mu_{n}), \pi^{\ast}_{nm})$. The proof of the identification $L^{p}(E',\mu)\simeq \mathscr{L}^{p}(\mf{H})$ via the isomorphism \eqref{EqIso}--\eqref{EqIso2}, follows the same steeps of the proof of Theorem \ref{T3.2}.
\end{proof}

\subsection{Gaussian measures.}\label{S4.2} In this subsection we focus on the particular case of Gaussian measures. Let $E$ be an infinite dimensional Fréchet nuclear space. A measure $\mu$ on $(E',\mc{S}_{E})$ is called \textit{Gaussian} if there exists an inner product $\mf{C}(\cdot,\cdot)$ on $E$ such that 
\begin{equation}
	\label{chi}
\chi(\xi)=e^{-\frac{1}{2}\mf{C}(\xi,\xi)}, \quad \xi\in E,
\end{equation}
where $\chi: E\to\C$ is the characteristic function of $\mu$. The inner product $\mf{C}$ is usually called the covariance operator. Let $R$ be a finite dimensional subspace of $E$ and let $\mf{C}_{R}$ denote the restriction of $\mf{C}$ to $R$, i.e., $\mf{C}_{R}:R\times R\to \R$. Then, if $\{\xi_{1},\xi_{2},\cdots,\xi_{n}\}$ is a basis of $R$, $\mf{C}_{R}$ can be represented as a symmetric matrix by
$$\mf{C}_{R}\equiv \left(\mf{C}_{ij}\right)_{1\leq i,j \leq n}, \quad \mf{C}_{ij}=\mf{C}(\xi_{i},\xi_{j}).$$
We denote its inverse by $\mf{C}^{-1}_{R}\equiv (\mf{C}^{-1}_{ij})_{1\leq i,j \leq n}$.
By Bochner theorem and \eqref{chi}, it is easy to see that the measure $\mu_{R}$ is explicitly given by 
$$\mathrm{d}\mu_{R}(\mathbf{x})=\frac{1}{\sqrt{(2\pi)^{n}\det[\mf{C}_{R}]}}\exp\left\{-\frac{1}{2}\sum_{i,j=1}^{n}\mf{C}^{-1}_{ij}x_{i}x_{j}\right\} \, \mathrm{d}^{n}\mathbf{x}, \quad \mathbf{x}:=(x_{1},\cdots,x_{n})\in R^{a}\simeq \R^{n},$$
where $d^{n}\mathbf{x}$ is the Lebesgue measure on $R$. Keeping the notation of Section \ref{S4.1}, we obtain the following result as a direct application of Theorem \ref{T4.5} to this particular case.
\begin{theorem}
	\label{T4.5.2}
	Let $E$ be an infinite dimensional Fréchet nuclear space and $\mu$ a Gaussian measure on $E'$ with covariance operator $\mf{C}$. Then for every complete system $(\xi_{n})_{n\in\N}\subset E$, the operator
	$$\mf{F}_{p}:\mathscr{L}^{p}(\mf{H})\longrightarrow L^{p}(E',\mu), \quad (f_{n})_{n\in\N}\mapsto \lim_{n\to\infty} \varphi^{\ast}_{n}(f_{n}),$$
	defines an isometric isomorphism with inverse
	\begin{equation}
		\label{eq3.3.3}
		\mf{F}^{-1}_{p}: L^{p}(E',\mu) \longrightarrow \mathscr{L}^{p}(\mf{H}), \quad f \mapsto  ([\varphi_{n}^{\ast}]^{-1}(\mathbb{E}_{n}(f)))_{n\in \N}.
	\end{equation}
Consequently, for each $f\in L^{1}(E',\mu)$, there exists a sequence $(f_{n})_{n\in\N}\in \prod_{n\in \N}L^{1}(H_{n}^{a},\mu_{n})$ such that
\begin{equation}
	\int_{E'}f \, \mathrm{d}\mu=\lim_{n\to\infty} \int_{H_{n}^{a}} f_{n}(\mathbf{x})\cdot  \frac{\exp\left\{-\frac{1}{2}\sum_{i,j=1}^{n}\mf{C}^{-1}_{ij}x_{i}x_{j}\right\}}{\sqrt{(2\pi)^{n}\det[\mf{C}_{R}]}} \, \mathrm{d}^{n}\mathbf{x}.
\end{equation}
Moreover the sequence $(f_{n})_{n\in\N}$ can be chosen to be $f_{n}=[\pi_{n}^{\ast}]^{-1}(\mathbb{E}_{n}(f))$ for each $n\in \N$. 
\end{theorem}


\subsection{Applications to Quantum Field Theory}\label{S4.3}

In this section, we are mainly interested in the application of the preceding abstract results to constructive (Euclidean) Quantum Field Theory (QFT) via the Osterwalder--Schrader axioms as stated in Glimm--Jaffe \cite[Ch. 6]{GJ}. Roughly speaking, the Osterwalder--Schrader axioms are stated in terms of a Borel probability measure $\mu$ on the space of tempered distributions $\mathscr{S}'(\R^{d})$, where $d$ is the space-time dimension, satisfying a series of axioms \cite[pp. 91–92]{GJ}. The measure $\mu$ plays the same role as does the Feynman--Kac measure in quantum mechanics. One of the major concerns of QFT is the computation of the Schwinger (or correlation) functions that are given by 
$$\mf{S}_{k}:\prod_{i=1}^{k}\mathscr{S}(\R^{d})\to \R, \quad \mf{S}_{k}(f_{1},\cdots,f_{k}):=\int_{\mathscr{S}'(\R^{d})}\phi(f_{1})\cdots\phi(f_{k}) \, \mathrm{d}\mu(\phi),$$
for positive integers $k\geq 1$.
We proceed to describe the measure $\mu$ in the case of the scalar free field with mass $m\in(0,\infty)$ and spin zero. It is well known that the space of Schwartz functions $E=\mathscr{S}(\R^{d})$ is an infinite dimensional Fréchet nuclear space. Given $m\in(0,\infty)$, let $\mu_{m}$ be the Gaussian measure on $\mathscr{S}'(\R^{d})$ with covariance operator
$$\mf{C}:\mathscr{S}(\R^{d})\times \mathscr{S}(\R^{d})\longrightarrow \R, \quad  \mf{C}(f,g):=\int_{\R^{d}}f(x)(-\Delta+m^{2})^{-1}g(x) \, \mathrm{d}x.$$
The measure space $(\mathscr{S}'(\R^{d}),\mu_{m})$ is known as the path space for free particles with mass $m>0$. The computation of the Schwinger functions in this case is easy as a consequence of the Bochner theorem. Indeed, we have
\begin{align*}
\mf{S}_{k}(f_{1},\cdots,f_{k})&=\int_{\mathscr{S}'(\R^{d})}\phi(f_{1})\cdots\phi(f_{k}) \, \mathrm{d}\mu_{m}(\phi)\\
&=\int_{R^{a}}x_{1}\cdots x_{k}\frac{\exp\left\{-\frac{1}{2}\sum_{i,j=1}^{k}\mf{C}^{-1}_{ij}x_{i}x_{j}\right\}}{\sqrt{(2\pi)^{k}\det[\mf{C}_{R}]}} \, \mathrm{d}^{k}\mathbf{x},
\end{align*}
where $R:=\text{span}\{f_{1},\cdots,f_{k}\}\subset \mathscr{S}(\R^{d})$ and therefore
$$\mf{C}_{R}\equiv \left(\mf{C}_{ij}\right)_{1\leq i,j \leq k}, \quad \mf{C}_{ij}=\mf{C}(f_{i},f_{j}).$$
By the Wick--Isserlis formula the above integral is zero if $k$ odd while
$$\mf{S}_{k}(f_{1},\cdots,f_{k})=\frac{1}{2^{k/2}(k/2)!}\sum_{\sigma\in\Pi_{k}}\prod_{j=1}^{k/2}\mf{C}(f_{\sigma(2j-1)},f_{\sigma(2j)}),$$
if $k$ is even. The main interest lies in interacting fields. The program of constructing the path measure $\mu$ increases in difficulty with the dimension $d$ of space-time. Here we focus on small polynomial non-linearities in the space-time dimension $d=2$, commonly known as $\mc{P}(\phi)_{2}$-theories. The construction of the measure for this theories was completed by Glim--Jaffe \cite{GJ0} and Dimock--Glimm \cite{DG} (see also Simon \cite{Si}). In finite volume $\mc{P}(\phi)_{2}$-theories, the measure $\mu$ can be expressed as 
$$\mathrm{d}\mu=e^{-V} \, \mathrm{d}\mu_{\mf{C}},$$
where $\mu_{\mf{C}}$ is the Gaussian measure on $\mathscr{S}'(\R^{d})$ with covariance $\mf{C}$ and $e^{-V}\in L^{1}(\mathscr{S}'(\R^{d}),\mu_{\mf{C}})$.
Given a complete system $(\xi_{n})_{n\in\N}\subset \mathscr{S}(\R^{d})$, by applying Theorem \ref{T4.5.2}, we obtain that the operator 
\begin{equation}
	\label{ISOM}
	\mf{F}_{1}:\mathscr{L}^{p}(\mf{H})\longrightarrow L^{1}(\mathscr{S}'(\R^{d}),\mu_{\mf{C}}), \quad (f_{n})_{n\in\N}\mapsto \lim_{n\to\infty} \varphi^{\ast}_{n}(f_{n}),
\end{equation}
is an isometric isomorphism with inverse 
\begin{equation}
	\label{ISOM2}
\mf{F}^{-1}_{1}: L^{1}(\mathscr{S}'(\R^{d}),\mu_{\mf{C}}) \longrightarrow \mathscr{L}^{p}(\mf{H}), \quad f \mapsto  ([\varphi_{n}^{\ast}]^{-1}(\mathbb{E}_{n}(f)))_{n\in \N},
\end{equation}
where $\mf{H}\equiv (H_{n}^{a},\mu_{n},\pi_{nm})$. Consequently, given $f_{1},\cdots, f_{k}\in \mathscr{S}(\R^{d})$ and setting 
$$\xi_{1}\equiv f_{1}, \cdots, \xi_{k}\equiv f_{k},$$
the corresponding correlation functions can be computed via a finite-dimensional approximation as
\begin{align*}
\mf{S}_{k}(f_{1},\cdots,f_{k})&=\int_{\mathscr{S}'(\R^{d})}\phi(f_{1})\cdots\phi(f_{k}) \ e^{-V} \mathrm{d}\mu_{\mf{C}}(\phi)\\
&=\lim_{n\to\infty}\int_{H^{a}_{n}}f_{n}(\mathbf{x})\cdot  \frac{\exp\left\{-\frac{1}{2}\sum_{i,j=1}^{n}\mf{C}^{-1}_{ij}x_{i}x_{j}\right\}}{\sqrt{(2\pi)^{n}\det[\mf{C}_{R}]}} \, \mathrm{d}^{n}\mathbf{x},
\end{align*}
for some sequence $(f_{n})_{n\in\N}\in\prod_{n\in \N}L^{1}(H^{a}_{n},\mu_{n})$ that can be explicitly computed through \eqref{ISOM2}. This approach is different to the usual formal perturbation theory \cite[\S 8.4]{GJ} where one expands $e^{-V}$ in power series and computes the (interaction-free) integrals in each term by using Feymnan diagrams. This approach is expected to be useful for the computation of explicit correlations. This scheme will be the subject of further work.

%\subsection{Gaussian measures.} Let $E$ be a real vector space and $\mu$ be a measure on $(E^{a},\mc{B}_{E})$. The measure $\mu$ is said to be a \textit{Gaussian measure} if its characteristic function $\chi:E\to \C$ is given by
%$$\chi(\xi)=\exp\left\{-\frac{1}{2}(\xi,\xi)\right\}, \quad \xi\in E,$$
%for some inner product $(\cdot,\cdot):E\times E\to\C$ on $E$. This inner product defines a norm $\|\cdot\|$ in $E$ and therefore, the pair $(E,\|\cdot\|)$ defines a pre-Hilbert space. We denote the topological dual of $E$ under the norm topology by $E^{\ast}$. In other words, $E^{\ast}$ is the space of linear and continuous (under the norm topology) functionals $\ell:E\to \R$. As sets, we have the inclusion $E^{\ast}\subset E^{a}$. In particular, $E^{\ast}$ is a Hilbert space under the operator norm. The space $E^{\ast}$ is the so-called \textit{Cameron-Martin space}.



\begin{thebibliography}{99}% Replace 9 by 99 if 10 or more references
	%
	% Please note the use of "\and" between author names below
	%
	
	\bibitem{B}
	S. Bochner, 
	\emph{Harmonic analysis and the theory of probability.}
	University of California Press (1955).
	
	\bibitem{Bou}
	N. Bourbaki, 
	\emph{Éléments de mathématque, intégration.}
	Livre VI, Chapitre IX, Intégration sur Les Espaces Topologiques Séparés, Hermann (1969).
	
	\bibitem{JC}
	J. M. F. Castillo,
	\emph{The Hitchhiker guide to categorical Banach space theory. Part I.}
	Extracta Mathematicae \textbf{25}, 2, (2010), pp. 103--149.
	
	\bibitem{Ch}
	J. R. Choksi, 
	\emph{Inverse limits of measure spaces.}
	Proc. London Math. Soc., \textbf{8} (3)
	(1958), pp. 321–342.
	
	\bibitem{C}
	D. L. Cohn,
	\emph{Measure theory.}
	Springer, New York, (2013).
	
	\bibitem{DG}
	J. Dimock and J. Glimm,
	\emph{Measures on Schwartz distribution space and applications to $\mc{P}(\phi)_{2}$ field theories.}
	Adv. Math. \textbf{12} (1974), pp. 58–83.
	
	\bibitem{F}
	Z. Frolík,
	\emph{Projective limits of measure spaces}.
	Berkeley Symp. on Math. Statist. and Prob. (1972), pp. 67–80.
	
	\bibitem{GJ0}
	J. Glimm and A. Jaffe, 
	\emph{The $\l\varphi^{4}_{2}$ quantum field theory without cut-offs.}
	I. Phys. Rev. \textbf{176} (1968), pp. 1945–1951, II. Ann. Math. \textbf{91} (1970), pp. 362–401, III. Acta. Math. \textbf{125} (1970), pp. 203–267,	and IV. J. Math. Phys. \textbf{13} (1972), 1568–1584.
	
	\bibitem{GJ}
	J. Glimm and A. Jaffe, 
	\emph{Quantum physics.} 
	 Second Edition. Springer Verlag, (1987).
	%	\bibitem{E}
	%	\textsc{M. Emery},
	%	Stochastic Calculus in Manifolds.
	%	\emph{Springer-Verlag (1989)}.
	
	%	\bibitem[10]{DW}
	%	\textsc{C. M. DeWitt \& M. G. Laidlaw},
	%	Feynman functional integrals for systems of indistinguishable particles.
	%	\emph{Phys. Rev. D3, (1971) pp. 1375-1378}.
	
	\bibitem{OlKa}
	O. Kallenberg,
	\emph{Foundations of modern probability}.
	Third edition. Probability Theory and Stochastic Modelling \textbf{99}, Springer Nature Switzerland AG (2021).
	
	\bibitem{K}
	A. N. Kolmogorov, 
	\emph{Foundations of the theory of probability.} 
	Second English Edition, translation edited by Nathan Morrison, Chelsea Publishing Company (1956).
	
	\bibitem{M}
	N. D. Macheras,
	\emph{On inductive limits of measure spaces and projective limits of $L^{p}$-spaces.}
	Mathematika \textbf{36} (1) (1989), pp. 116–130.
	
	\bibitem{Ma}
	D. J. Mallory and M. Sion, 
	\emph{Limits of inverse systems of measures.}
	Ann. Inst. Fourier \textbf{21} (1971), pp. 25–57.
	
	\bibitem{Me}
	M. Metivier,
	\emph{Limites projectives de measures, martingales, applications.}
	Annali di Matematica, \textbf{63} (1963), pp. 225--352.
	
	\bibitem{P}
	M. Pintér,
	\emph{The existence of an inverse limit of an inverse system of measure spaces -- A purely measurable case.}
	Acta Math. Hungar., \textbf{126} (1--2) (2010), pp. 65–77.
	
	\bibitem{RA}
	M. M. Rao,
	\emph{Projective limits of probability spaces}.
	J. Multivariate Anal. \textbf{1} (1971), pp. 28–57.
	
	\bibitem{RS}
	M. M. Rao and V. V. Sazonov,
	\emph{A projective limit theorem for probability spaces and applications}.
	Theory Probab. Appl. \textbf{38} (2) (1994), pp. 307–315.
	
	\bibitem{Si}
	B. Simon, 
	\emph{The $\mc{P}(\Phi)_{2}$ (Euclidean) quantum field theory}.
	Princeton Series in Physics, Princeton Univ. Press (1974).
	
	%\bibitem{J}
	%J. C. Sampedro,
	%\emph{On the space of infinite dimensional integrable functions}.
	%J. Math. Anal. Appl. \textbf{488} (2020), no. 1.
	
	%	\bibitem[22]{SC}
	%	\textsc{L. S. Schulman}
	%	A path integral for spin.
	%	\emph{Phys. Rev. \textbf{176}, pp. 1558-1569 (1968)}.
	
	%	\bibitem{32}
	%	\textsc{S. Shirali \& H. L. Vasudeva},
	%	Metric Spaces.
	%	\emph{Springer (2006)}.
	
	%	\bibitem[24]{SP}
	%	\textsc{E. H. Spanier},
	%	Algebraic Topology.
	%	\emph{Springer-Verlag (1966)}.
	
	%	\bibitem[25]{G}
	%	\textsc{G. Springer},
	%	Introduction to Riemannian Surfaces.
	%	\emph{Addison-Wesley Inc. (1957)}.
	
	%	\bibitem[26]{U} 
	%	\textsc{T. Sunada},
	%	Trace formula, Wiener integrals and asymptotics, Spectra of Riemannian Manifolds, 
	%	\emph{Kaigai Publications, Tokyo (1983),  pp. 103--113}.
	
	\bibitem{WO}
	N. E. Wegge-Olsen,
	\emph{K-Theory and C*-Algebras, a friendly approach}.
	Oxford University Press (1993).
	
	\bibitem{Y}
	Y. Yamasaki,
	\emph{Measures on infinite-dimensional spaces}.
	World Scientific Publishing Co., Singapore, (1985).
	
	%	\bibitem[35]{47}
	%	\textsc{J. Yi},
	%	Theta-function identities and the explicit formulas for theta-function and their applications.
	%	\emph{J. Math. Anal. Appl. \textbf{292} (2), pp. 381–400}.
	
\end{thebibliography}






\end{document}
