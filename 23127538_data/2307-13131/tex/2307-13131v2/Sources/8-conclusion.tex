\section{Conclusion}
This paper introduced \namenospace, a man-in-the-middle perception attack on safety-critical cyber-physical systems. \name is the first sensor-first, dynamic adversarial machine learning framework for physical-domain attacks. \name leverages transparent displays to generate dynamic physical adversarial examples. The digital-to-physical perturbation pipeline is enabled by modeling the environmental noise due to optical transformations and environmental factors. We show the efficacy of \name on the real-world use case of traffic sign recognition, demonstrating that
% \name can dynamically determine which perturbations to generate at runtime without predetermined knowledge of which objects the victim system will encounter. Moreover, we demonstrate that 
\name can significantly outperform existing attack frameworks across varying levels of ambient illumination, including over 80\% ASR at 60,000 lux--whereas prior works can only achieve a similar ASR at less than 120 lux. \name provides a practical approach to dynamically modeling optical transformations in the context of adversarial machine learning attacks in the real world.%\nils{can we bring in some concrete performance numbers/ lux numbers here that highlight some hard evidence of success?}