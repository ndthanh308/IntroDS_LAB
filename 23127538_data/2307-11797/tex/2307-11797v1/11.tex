\UseRawInputEncoding
\documentclass[12pt]{article}
%\documentclass[11pt,letterpaper]{amsart}
%\documentclass[11pt,letterpaper]{article}

\usepackage{mathrsfs}
\usepackage{amssymb}
\usepackage{amsmath}
\usepackage{amsfonts}
\usepackage{amstext}
\usepackage{amsthm}
\usepackage{graphicx}
\usepackage{subfig}
\usepackage{color}
\usepackage{indentfirst,latexsym,bm}
\usepackage{mathrsfs}
\usepackage{cite}
\usepackage[all]{xy}
\usepackage{newlfont}
\newcommand{\tb}{\textcolor{blue}}
\newcommand{\tr}{\textcolor{red}}
\newcommand{\R}{\mathbb{R}}
\newcommand{\intn}{\int_{\R^N}}
\newcommand{\tw}{\mathcal{T}_\omega}
\usepackage{enumerate}
 \setcounter{MaxMatrixCols}{10}

% THEOREM Environments ---------------------------------------------------
\newtheorem{defn}{Definition}[section]
\newtheorem{prop}{Proposition}[section]
\newtheorem{thm}{Theorem}[section]
\newtheorem{lem}{Lemma}[section]
\newtheorem{rek}[thm]{Remark}
\numberwithin{equation}{section}

\newtheorem{theorem}{Theorem}[section]
\newtheorem{corollary}[subsection]{Corollary}
\newtheorem{lemma}[theorem]{Lemma}
\newtheorem{remark}[theorem]{Remark}
\newtheorem{definition}[theorem]{Definition}
\newtheorem{proposition}[theorem]{Proposition}
\numberwithin{equation}{section}

\newcommand{\norm}[1]{\left\Vert#1\right\Vert}

\renewcommand{\vec}[1]{\mbox{\boldmath$#1$}}
%\pdfpagewidth 8.5in
%\pdfpageheight 11in
\linespread{1.2}

% Set left margin - The default is 1 inch, so the following
% command sets a 1.25-inch left margin.
\setlength{\oddsidemargin}{0.25in}

% Set width of the text - What is left will be the right margin.
% In this case, right margin is 8.5in - 1.25in - 6.25in = 1.00in.
\setlength{\textwidth}{6.25in}

% Set top margin - The default is 1 inch, so the following
% command sets a 0.75-inch top margin.
\setlength{\topmargin}{-0.65in}

% Set height of the text - What is left will be the bottom margin.
% In this case, bottom margin is 11in - 0.25in - 9.5in = 0.75in
\setlength{\textheight}{9.5in}

% Set the beginning of a LaTeX document
\setlength{\parindent}{1em}


% Set the beginning of a LaTeX document


\begin{document}

\title{Sign-changing solutions to Schiffer's overdetermined problem on wavy cylinder %Nonsymmetric sign-changing solution for the overdetermined eigenvalue problem
\thanks{Research supported by NNSF of China (No. 11871129).}}

\author{Guowei Dai\thanks{Corresponding author.
\newline
School of Mathematical Sciences, Dalian University of Technology, Dalian, 116024, P.R. China
\newline
\text{~~~~ E-mail}: daiguowei@dlut.edu.cn.}, Yong Zhang\thanks{Institute of Applied System Analysis, Jiangsu University, Zhenjiang, 212013, P.R. China
\newline
\text{~~~~ E-mail}: 18842629891@163.com} \\
}
\date{}
\maketitle

\renewcommand{\abstractname}{Abstract}

\begin{abstract}
In this paper, we prove the existence of $k$ families of smooth unbounded domains $\Omega_s\subset\mathbb{R}^{N+1}$ with $N\geq1$, where
\begin{equation}
\Omega_s=\left\{(x,t)\in \mathbb{R}^N\times \mathbb{R}:\vert x\vert<1+s\cos \left(\frac{2\pi}{T(s)}t\right)+s w_s\left(\frac{2\pi}{T(s)}t\right)\right\},\nonumber
\end{equation}
such that
\begin{equation}
-\Delta u=\lambda u\,\, \text{in}\,\,\Omega, \,\, \partial_\nu u=0,\,\,u=\text{const}\,\,\text{on}\,\,\partial\Omega\nonumber
\end{equation}
admits a bounded sign-changing solution with exactly $k+1$ nodal domains.
These results can be regarded as counterexamples to the Schiffer conjecture. Our results also indicate that there exist non-spherical regions without Pompeiu property and show that the condition "$\partial\Omega$ is homeomorphic to the unit sphere" is necessary for Williams conjecture to hold.
\end{abstract}

\emph{Keywords:} Schiffer conjecture; Overdetermined problem; Bifurcation; Bessel function


\emph{AMS Subjection Classification(2020):} 35B32; 35N05; 35R35

\tableofcontents
% ------------------------------------------------------------------------------------------------------------Introduction


\section{Introduction}
\quad A domain $\Omega\subseteq\mathbb{R}^{N}$ is said to have the Pompeiu property if $f\equiv 0$ is the only continuous function
satisfying
\begin{equation}
\int_{\sigma(\Omega)}f(x)\,\text{d}x=0\nonumber
\end{equation}
for every rigid motion $\sigma$ of $\Omega$. The problem of classifying regions based on whether they have the Pompeiu property is called the Pompeiu problem.
The Pompeiu problem originated from harmonic analysis \cite{Pompeiu, Pompeiu1}, plasma physics \cite{Temam},
nuclear reactors \cite{Berenstein} and tomography \cite{Shepp, Smith}.
It is well known that an open ball of any radius fails to have the Pompeiu property.
However, it is  a long-standing open problem that whether the bounded region without Pompeiu property must be a ball.
In particular, Williams \cite{Williams} proposed the following conjecture.
\\ \\
\textbf{Williams conjecture.}
\emph{If $\partial\Omega$ is homeomorphic to the unit sphere in $\mathbb{R}^{N}$, then $\Omega$ has the Pompeiu
property if and only if it is not a ball.}\\
~\\
The conjecture is strongly connected to the following Schiffer conjecture \cite{Schiffer, Schiffer1}.
\\ \\
\textbf{Schiffer conjecture.} \emph{Let $\Omega\subset \mathbb{R}^N$ be a bounded regular domain and assume that $u: \Omega \rightarrow \mathbb{R}$
is a solution to the problem
\begin{equation}\label{SchifferConjecture}
\left\{
\begin{array}{ll}
\Delta u+\lambda u=0\,\, &\text{in}\,\, \Omega,\\
\partial_\nu u=0 &\text{on}\,\, \partial \Omega,\\
u=\text{const} &\text{on}\,\, \partial \Omega,
\end{array}
\right.
\end{equation}
where $\lambda$ is a parameter and $\nu$ is the unit outer normal vector on $\partial \Omega$. Then $\Omega$ is a ball and $u$ is radially symmetric.}\\
~\\
Concretely,
for a bounded regular domain $\Omega$, Williams conjecture holding is equivalent to Schiffer conjecture holding.
The Pompeiu problem or Schiffer conjecture is one of Yau's famous list of problems \cite[Problem 80]{Yau}. As far as we know, it is still a big open problem to prove the Schiffer conjecture. However, some related results have been obtained, for instance, in \cite{Berenstein1,Berenstein2} Berenstein and Yang found that the existence of infinitely many eigenvalues to (\ref{SchifferConjecture}) implies that $\Omega$ must be a round ball. In \cite{Liu}, Liu proved that the Schiffer
conjecture holds if and only if the third order interior normal derivative of the corresponding
Neumann eigenfunction $u$ is constant on the boundary of $\Omega$.
We refer to \cite{Bagchi, Deng, Friedlander} and their references for some related results in this direction.
Recently, Fall, Minlend and Weth \cite{FallMW} constructed a nontrivial family of compact subdomains of the flat cylinder $\mathbb{R}^N\times \mathbb{R}/2\pi\mathbb{Z}$ such that the problem (\ref{SchifferConjecture}) has a solution. This result can be regarded as the first counterexample to the Schiffer conjecture on unbounded domains, where they choose the radius of the ball as the bifurcation parameter and obtain a sequence of $2\pi$-periodic domains.

The natural question is whether other nontrivial unbounded domains can be constructed to support the problem (\ref{SchifferConjecture}), or in particular whether there exist domains bifurcating from cylindrical domains with any radius and more general period such that the problem (\ref{SchifferConjecture}) has solution? In this paper, we aim to give a positive answer to this question. To this end, we will choose minimum positive period $T$ as bifurcation parameter.
Although the method developed in this paper holds true for any radius, we only consider the bifurcation from the cylindrical domain with a radius of $1$ for convenience.
For any $k\in \mathbb{N}$, let $\lambda_k$ be the $k$th eigenvalue of the $0$-Neumann Laplacian on the unit ball.
The first main result is the following.
\\ \\
\textbf{Theorem 1.1.} \emph{Let} $\mathcal{C}^{2,\alpha}_{\text{even},0}\left(\mathbb{R}/2\pi \mathbb{Z}\right)$
\emph{be the space of even $2\pi$-periodic $\mathcal{C}^{2,\alpha}$ functions of mean zero and any $k\in \mathbb{N}$ with $k\geq1$.
There exist a positive number $T_{1,*}$ with
\begin{equation}
T_{1,*}\in\left(\frac{2\pi}{\sqrt{\lambda_k}},\frac{2\pi}{\sqrt{\lambda_k-\lambda_{1}}}\right)\nonumber
\end{equation}
and a smooth map}
\begin{equation}
(-\varepsilon,\varepsilon)\rightarrow \mathcal{C}^{2,\alpha}_{\text{even},0}\left(\mathbb{R}/2\pi \mathbb{Z}\right)\times \mathbb{R}:s\mapsto \left(w_s,T(s)\right)\nonumber
\end{equation}
\emph{with $w_0 = 0$, $T(0) = T_{1,*}$ such that for each $s\in  (-\varepsilon,\varepsilon)$, problem (\ref{SchifferConjecture}) has a $T(s)$-periodic solution
$u_s \in  \mathcal{C}^{2,\alpha}\left(\Omega_s\right)$ on the modified cylinder
\begin{equation}
\Omega_s=\left\{(x,t)\in \mathbb{R}^N\times \mathbb{R}:\vert x\vert<1+s\cos \left(\frac{2\pi}{T(s)}t\right)+s w_s\left(\frac{2\pi}{T(s)}t\right)\right\}.\nonumber
\end{equation}
Moreover, there exist $k$ ($T(s)$-periodic) functions $r_j\in  \mathcal{C}^{\infty}\left(\mathbb{R}\right)$ for $j\in\{1,\ldots,k\}$ with
\begin{equation}
r_j(t):\mathbb{R}\longrightarrow \left(0,1+s\cos \left(\frac{2\pi}{T(s)}t\right)+s w_s\left(\frac{2\pi}{T(s)}t\right)\right)\nonumber
\end{equation}
and
\begin{equation}
r_1(t)<\cdots<r_{k}(t),\nonumber
\end{equation}
such that $u_s\left(r_j(t),t\right)=0$.}
\\

Besides of $T_{1,*}$, we further obtain the other $k-1$ bifurcation points $T_{i,*}$ for $i\in\{2,3,...,k\}$, where these bifurcation points may be corresponding to high-dimensional kernel space.
\\ \\
\textbf{Theorem 1.2.} \emph{Under the assumptions of Theorem 1.1, there exist $k-1$ positive numbers $T_{2,*}$, $\ldots$, $T_{k,*}$ with
\begin{equation}
T_{k,*}\in\left(\frac{2\pi}{\sqrt{\lambda_k-\lambda_{k-1}}},+\infty\right)\,\,\text{and}\,\,T_{i,*}\in\left(\frac{2\pi}{\sqrt{\lambda_k-\lambda_{i-1}}},\frac{2\pi}{\sqrt{\lambda_k-\lambda_{i}}}\right)\nonumber
\end{equation}
for each $i\in\{2,\ldots,k-1\}$
and $k-1$ smooth map}
\begin{equation}
(-\varepsilon,\varepsilon)\rightarrow \mathcal{C}^{2,\alpha}_{\text{even},0}\left(\mathbb{R}/2\pi \mathbb{Z}\right)\times \mathbb{R}:s\mapsto \left(w_s,T(s)\right)\nonumber
\end{equation}
\emph{with $w_0 = 0$, $T(0) = T_{i,*}$ with $i\in\{2,\ldots,k\}$ such that for each $s\in  (-\varepsilon,\varepsilon)$, problem (\ref{SchifferConjecture}) has a sign-changing $T(s)$-periodic solution
$u_s \in  \mathcal{C}^{2,\alpha}\left(\Omega_s\right)$ on the modified cylinder
\begin{equation}
\Omega_s=\left\{(x,t)\in \mathbb{R}^{N+1}:\vert x\vert<1+s\left(\beta\cos \left(\frac{2\pi}{T(s)}t\right)+\sum_{n=1}^m\gamma_n\cos \left(\frac{2l_{j_n}\pi}{T(s)}t\right)\right)+s w_s\left(\frac{2\pi}{T(s)}t\right)\right\},\nonumber
\end{equation}
where either $\beta$, $\gamma_n$ are nonzero constants with $\beta^2+\sum_{n=1}^m\gamma_n^2=1$ if there exist $m$ elements of $\{1,\ldots,i-1\}$ with $m\leq i-1$, which are denoted by $j_n$ with $n\in\{1,\ldots,m\}$, such that $T_{i,*}= l_{j_n}T_{j_n,*}$ for some $l_{j_n}\in \mathbb{N}$ with $l_{j_n}\geq2$, or $\beta=1$ and $\gamma_n=0$ if $T_{i,*}\neq lT_{j,*}$ for every $j\in\{1,\ldots,i-1\}$ and any $l\in \mathbb{N}$ with $l\geq2$.
Moreover, there exist $k$ ($T(s)$-periodic) functions $r_j\in  \mathcal{C}^{\infty}\left(\mathbb{R}\right)$ for $j\in\{1,\ldots,k\}$ with
\begin{equation}
r_j(t):\mathbb{R}\longrightarrow \left(0,1+s\left(\beta\cos \left(\frac{2\pi}{T(s)}t\right)+\sum_{n=1}^m\gamma_n\cos \left(\frac{2l_{j_n}\pi}{T(s)}t\right)\right)+s w_s\left(\frac{2\pi}{T(s)}t\right)\right)\nonumber
\end{equation}
and
\begin{equation}
r_1(t)<\cdots<r_{k}(t),\nonumber
\end{equation}
such that $u_s\left(r_j(t),t\right)=0$.}\\

If limited to one period, it is obvious that $\Omega_s$ is bounded. Our conclusions indicate that problem (\ref{SchifferConjecture}) has a nonsymmetric sign-changing solution.
Consequently, the existence of these domains disproves Schiffer conjecture.
These results also indicate that there exist non-spherical regions without Pompeiu property, which means that the conclusion of Williams conjecture is invalid.
Note that the boundary of $\Omega_s$ we found is not homeomorphic to the unit sphere.
Thus, the condition "$\partial\Omega$ is homeomorphic to the unit sphere" is necessary for Williams conjecture to hold.

At last, we would like to mention the following overdetermined elliptic problem
\begin{equation}\label{positivesoution}
\left\{
\begin{array}{ll}
\Delta u+f(u)=0\,\, &\text{in}\,\, \Omega,\\
u>0 &\text{in}\,\, \Omega,\\
u=0 &\text{on}\,\, \partial \Omega,\\
\partial_\nu u=\text{const} &\text{on}\,\, \partial \Omega,
\end{array}
\right.
\end{equation}
which was considered by Sicbaldi \cite{Sicbaldi} where he proved that the cylinder $B_1\times \mathbb{R}$ ($B_1$ is the unit ball of $\mathbb{R}^N$ with $N\geq2$ centered at the origin) can be perturbed to an unbounded domain whose boundary is a periodic
hypersurface of revolution with respect to the $\mathbb{R}$-axis, such that problem (\ref{positivesoution}) with $f(u)=\lambda u$ has a bounded
solution.
This result can be regarded as the first counterexample to the following BCN conjecture \cite{BCN}.
\\ \\
\textbf{BCN conjecture.}
\emph{If $f$ is a Lipschitz function on a
domain $\Omega$ in $\mathbb{R}^N$ such that $\mathbb{R}^N\setminus \overline{\Omega}$ is connected, then the existence of a bounded solution
to (\ref{positivesoution}) implies that $\Omega$ is either a ball, a half-space, a generalized cylinder
$B^k\times \mathbb{R}^{N-k}$ ($B^k$ is a round ball in $\mathbb{R}^k$), or the complement of one of them.}\\
~\\
In 2012, Schlenk and Sicbaldi \cite{Schlenk} further proved that the result in \cite{Sicbaldi} is also valid for $N=1$ and the extremal domains obtained in \cite{Sicbaldi} belong to a smooth bifurcation family of domains. After this, some different types of nontrivial domains were constructed in $\mathbb{R}^N$ as negative answers to BCN conjecture, for instance nontrivial epigraph domain emanating from the half-space \cite{Del} for $N\geq9$ and the perturbations of the complement of a ball \cite{Ros1} for $N\geq 2$.
%We also mention the famous result
%of James Serrin \cite{Serrin} that if $\Omega$ is a bounded open connected set with smooth
%boundary on which there exists a function $u$ satisfying (\ref{positivesoution}) with $\lambda= s^{-1}$, then $\Omega$ must be a ball.
%Since Serrin's famous work \cite{Serrin}, overdetermined elliptic problem has attracted the attention of many mathematicians.
%In the post half century, many celebrated results have been established, for instance \cite{Aftalion, BCN, ChenLi, Gidas, Pucci, Reichel, RRS} (the lists are just a very small sample and far from being complete).

In this paper, we mainly use the ideas of \cite{Sicbaldi, Schlenk} to establish our main Theorems 1.1--1.2. However, it is worth noting that there are some essential differences between the schiffer problem (\ref{SchifferConjecture}) and the one of (\ref{positivesoution}).
First, Schlenk and Sicbaldi \cite{Sicbaldi,Schlenk} are only concerned with the positive solution to (\ref{positivesoution}) but it is necessary to consider the sign-changing solution to schiffer problem (\ref{SchifferConjecture}) due to different boundary conditions. That is, authors in \cite{Sicbaldi,Schlenk} obtained the nontrivial solution to (\ref{positivesoution}) as the perturbation of the first eigenfunction corresponding to the first eigenvalue of $0$-Dirichlet Laplacian. However, we have to choose to bifurcate from the high eigenfunction corresponding to the high eigenvalue of $0$-Neumman Laplacian for the Schiffer problem.
Second, here the new case of multidimensional kernels would take place, which is different from the one-dimensional kernel space before. Concretely, if there holds that $T_{i,*}= l_{j_n}T_{j_n,*}$ for some $l_{j_n}\in \mathbb{N}$ with $l_{j_n}\geq2$, then the kernel space corresponding to $T_{i,*}$ is high-dimensional, which leads to that the classical Crandall-Rabinowitz bifurcation theorem can not be used directly. But a new Crandall-Rabinowitz type local bifurcation theorem of high-dimensional kernels established recently in \cite{DaiZ} works here. We also refer to some interesting work \cite{DaiZ,Minlend, Ruiz}, where sign-changing solutions to some overdetermined problems are obtained. The choice in \cite{Ruiz} of trivial sign-changing solutions depend closely on the form of the equation with nontrivial domains constructed being bounded and the domains constructed in \cite{Minlend} bifurcate from suitable strips in $\mathbb{R}^2$ with the Neumann boundary condition varying from top to bottom.
At last, we want to emphasize that the handling details and conclusions in this paper are very different from those in \cite{FallMW}. Here we provide a complete and detailed characterization of the properties of the eigenvalues of linearization operator and obtain the bifurcation domains from cylindrical domains with any radius and more general period.
In particular, to obtain the monotonicity and asymptotic behavior of eigenvalues, we need to analyze the zeros distribution and the dependency on parameter of the Bessel functions very finely.
Moreover, for each $k\in \mathbb{N}$, we can obtain $k$
bifurcation points while only one bifurcation point obtained in \cite{FallMW}. In addition, we also provided the specific range of these bifurcation points

The rest of this paper is arranged as follows. In Section 2, we first show a result of eigenvalue problem on cylinder which will be used later.
Then, we transform the problem (\ref{SchifferConjecture}) into an abstract operator equation and investigate its linearization.
We end the Section 2 by recalling some properties of Bessel functions.
In Sections 3--4, we study the properties of a certain eigenvalue, which is key to obtain Theorems 1.1--1.2.
The Last Section is devoted to completing the proofs of Theorems 1.1--1.2.


\section{Preliminaries}

\quad\,  The main strategy of this paper is to apply Crandall-Rabinowitz bifurcation theorem \cite{Crandall} to obtain the nontrivial domains supporting solutions of Schiffer overdetermined problem (\ref{SchifferConjecture}). To achieve this goal, we first need to transform the problem (\ref{SchifferConjecture}) into an abstract operator equation. For convenience, let us show an elementary result on $0$-Neumann Laplacian eigenvalue problems on cylinder in the following.

\subsection{A result of eigenvalue problem on cylinder}

\quad\, We first consider the following eigenvalue problem
\begin{equation}\label{eigenvalueonball1}
\left\{
\begin{array}{ll}
\Delta u+\lambda u=0\,\, &\text{in}\,\, B_1,\\
\partial_\nu u=0 &\text{on}\,\, \partial B_1,
\end{array}
\right.
\end{equation}
where $B_1\subset \mathbb{R}^N$ is the unit ball. It follows from \cite{Coddington, Ince} that the problem (\ref{eigenvalueonball1}) possesses a sequence of eigenvalues $0=\lambda_0<\lambda_1<\lambda_2<\cdots<\lambda_k\nearrow+\infty$ and the eigenfunction corresponding to the eigenvalue $\lambda_k$ with $k\geq 1$ is sign-changing while the one with $k=0$ is any constant. In order to ensure that the whole process can be carried out, here we only consider the case of $k\geq1$. This is why we will obtain the nontrivial sign-changing solutions to Schiffer overdetermined problem (\ref{SchifferConjecture}).
Let $\overline{\phi}_k$ be the corresponding radial eigenfunction
to $\lambda_k$ with $\int_{B_1}\overline{\phi}_k^2\,\text{d}x=1/\left(2\pi\right)$ and $\overline{\phi}_k(1)>0$.

Note that $\left(\lambda_k, {\phi}_k(x,t)\right)$ with ${\phi}_k(x,t)=\overline{\phi}_k(x)$ is also a solution pair of
\begin{equation}\label{eigenvalueoncylinder1}
\left\{
\begin{array}{ll}
\Delta_{\mathring{g}}{\phi}+\lambda {\phi}=0\,\, &\text{in}\,\, C_1^T,\\
\partial_\nu{\phi}=0 &\text{on}\,\, \partial C_1^T,
\end{array}
\right.
\end{equation}
where
\begin{equation}
C_{1}^T=\left\{(x,t)\in \mathbb{R}^N\times \mathbb{R}/T\mathbb{Z}: \vert x\vert<1\right\}.\nonumber
\end{equation}
It is obvious that
\begin{equation}
\int_{C_1^{2 \pi}}{\phi}_k^2\,\text{d} x\text{d}t=1.\nonumber
\end{equation}
Since $\phi_k$ does not depend on $t$ and it is radial, we will denote $\phi_k(x,t)$ by $\phi_k(r,t)$ with $r=\vert x\vert$.
Conversely, the following result would indicate that the eigenvalue of (\ref{eigenvalueoncylinder1}) is also the eigenvalue of (\ref{eigenvalueonball1}) with the same eigenfunction (up to a constant multiple).
\\ \\
\textbf{Proposition 2.1.} \emph{If $(\kappa,w)$ with $\kappa>0$ and $w\not\equiv0$ satisfies}
\begin{equation}\label{eigenvalueoncylindermu1}
\left\{
\begin{array}{ll}
\Delta_{\mathring{g}} w+\kappa w=0\,\, &\text{in}\,\, C_1^T,\\
\partial_\nu w=0 &\text{on}\,\, \partial C_1^T,
\end{array}
\right.
\end{equation}
\emph{then $\kappa=\lambda_k$ for some $k\in \mathbb{N}$ and $w=C{\phi}_k$ for some $C\neq0$.}
\\ \\
\textbf{Proof.} Suppose, by contradiction, that $\kappa\neq \lambda_k$ for any $k\in \mathbb{N}$.
Note that
\begin{equation}\label{eigenvalueoncylinderphi1}
\left\{
\begin{array}{ll}
\Delta_{\mathring{g}} {\phi}_k+\lambda_k {\phi}_k=0\,\, &\text{in}\,\, C_1^T,\\
\partial_\nu{\phi}_k=0 &\text{on}\,\, \partial C_1^T.
\end{array}
\right.
\end{equation}
Multiplying (\ref{eigenvalueoncylindermu1}) by ${\phi}_k$, and (\ref{eigenvalueoncylinderphi1}) by $w$ and integrating, we get that
\begin{equation}
\int_{C_1^T}\nabla w\nabla \phi_k\,\text{d}x\text{d}t=\kappa\int_{C_1^T}w\phi_k\,\text{d}x\text{d}t\nonumber
\end{equation}
and
\begin{equation}
\int_{C_1^T}\nabla w\nabla \phi_k\,\text{d}x\text{d}t=\lambda_k\int_{C_1^T}w\phi_k\,\text{d}x\text{d}t.\nonumber
\end{equation}
It follows that
\begin{equation}
\left(\kappa-\lambda_k\right)\int_{C_1^T}w\phi_k\,\text{d}x\text{d}t=0.\nonumber
\end{equation}
Since $\kappa\neq \lambda_k$ for any $k\in \mathbb{N}$, we get that $\int_{C_1^T}w\phi_k\,\text{d}x\text{d}t=0$. That is to say
$w$ is orthogonal to $\phi_k$ for any $k\in \mathbb{N}$. Note that $\left\{{\phi}_k(x,t)\right\}_{k=1}^\infty$ is a basis of $L^2\left(C_1^T\right)\setminus\left\{\mathbb{R}\right\}$.
Hence $w$ is orthogonal to $L^2\left(C_1^T\right)\setminus\left\{\mathbb{R}\right\}$ and $w$ must be a constant, which contradicts the facts of $\kappa>0$ and $w\not\equiv0$.

Next, let us show that $w=C\overline{\phi}_k$ with some $C\neq0$.
It is well known that
$\phi_k$ has $k$ simple zeros in $(0,1)$, which can be denoted by $r_1,r_2,\ldots,r_{k}$. Consequently, ${\phi}_k\left(r_i,t\right)=0$
for any $i\in\{1,\ldots,k\}$ and $t\in \mathbb{R}$. From now on, $r=r_i$ is called zero line of  ${\phi}_k$ in $C_1^T$.
For convenience of expression, we use $r_0$ and $r_{k+1}$ to express $0$ and $1$, respectively.
Let $C_1^{T,i}=\left\{(x,t)\in C_1^T:r_{i-1}<\vert x\vert<r_i\right\}$ for each $i\in\{1,\ldots,k+1\}$ and
$C_1^{T,1}=\left\{(x,t)\in C_1^T:0\leq\vert x\vert<r_1\right\}$.
That is to say $C_1^{T,i}$ is the $i$-th nodal domain.
We consider (\ref{eigenvalueoncylindermu1}) restricted on $C_1^{T,1}$.
Let
\begin{equation}\label{fixedboundaryvalue}
\widetilde{w}=\left\{
\begin{array}{ll}
w\,\, &\text{in}\,\, C_1^{T,1},\\
0 &\text{on}\,\, \overline{C_1^T}\setminus C_1^{T,1}.
\end{array}
\right.\nonumber
\end{equation}
One has that $\widetilde{w}\in H^1\left(C_1^T\right)$.
Note that $\phi_k$ is not sign-changing in $C_1^{T,1}$. Multiplying (\ref{eigenvalueoncylinderphi1}) by $w^2/\phi_k$ and integrating on $C_1^{T,1}$, we obtain that
\begin{equation}
\int_{C_1^{T,1}}\left\vert \nabla \phi_k\right\vert^2\frac{w^2}{\phi_k^2}\,\text{d}x\text{d}t-2\int_{C_1^{T,1}}\nabla \phi_k  \nabla w\frac{w}{\phi_k}\,\text{d}x\text{d}t+\lambda_k\int_{C_1^{T,1}}w^2\,\text{d}x\text{d}t=0.\nonumber
\end{equation}
From (\ref{eigenvalueoncylindermu1}) we get that
\begin{equation}
\int_{C_1^{T,1}}\left\vert \nabla w\right\vert^2\,\text{d}x\text{d}t=\lambda_k\int_{C_1^{T,1}}w^2\,\text{d}x\text{d}t.\nonumber
\end{equation}
Thus, we have that
\begin{eqnarray}
0&=&\int_{C_1^{T,1}}\left\vert \nabla \phi_k\right\vert^2\frac{w^2}{\phi_k^2}\,\text{d}x\text{d}t-2\int_{C_1^{T,1}}\nabla \phi_k  \nabla w\frac{w}{\phi_k}\,\text{d}x\text{d}t+\int_{C_1^{T,1}}\left\vert \nabla w\right\vert^2\,\text{d}x\text{d}t \nonumber\\
&=&\int_{C_1^{T,1}}\left(\left\vert \nabla \phi_k\right\vert\frac{w}{\phi_k}-\left\vert \nabla w\right\vert\right)^2\,\text{d}x\text{d}t.\nonumber
\end{eqnarray}
The above equality holds if and only if $w=C\phi_k$ with some $C\neq0$.

Repeating the above argument $k$ times, we can obtain that $w=C\phi_k$ a.e. on any nodal domain $C_1^{T,i}$ for each $i\in\{2,\ldots,k+1\}$. By the standard argument of the regularity and continuity, we have that
$w=C\phi_k$ in $C_1^{T}$ with some $C\neq0$.\qed

\subsection{The operator equation and its linearization}

We now transform the aim problem (\ref{SchifferConjecture}) into an abstract operator equation.
For each $v\in \mathcal{C}^{2,\alpha}_{\text{even},0}\left(\mathbb{R}/2\pi \mathbb{Z}\right)$ with $v>-1$, define
\begin{equation}
C_{1+v}^T=\left\{(x,t)\in \mathbb{R}^N\times \mathbb{R}/T\mathbb{Z}: \vert x\vert<1+v\left(\frac{2\pi t}{T}\right)\right\}\nonumber
\end{equation}
for all $T>0$.
We consider the following eigenvalue problem
\begin{equation}\label{eigenvalueproblem}
\left\{
\begin{array}{ll}
\Delta_{\mathring{g}} {\phi}+\lambda {\phi}=0\,\, &\text{in}\,\, C_{1+v}^T,\\
\partial_\nu {\phi}=0 &\text{on}\,\, \partial C_{1+v}^T.
\end{array}
\right.
\end{equation}
It is well known that problem (\ref{eigenvalueproblem}) possesses a sequence of nontrivial (positive) eigenvalues $0<\lambda_{1,v}<\lambda_{2,v}\leq\cdots$, $\lambda_{k,v}\nearrow+\infty$
(for example, see \cite[Theorem 1.1]{Chavel}).
%It follows from \cite[Theorem 1.13]{Ambrosetti} that the problem (\ref{eigenvalueproblem}) possesses a sequence eigenvalues $0<\lambda_{1,v}<\lambda_{2,v}\leq\cdots$, $\lambda_{k,v}\nearrow+\infty$.

Let ${\phi}_{k,v}$ be an eigenfunction corresponding to $\lambda_{k,v}$ such that
\begin{equation}
\int_{C_{1+v}^{2 \pi}}{\phi}_{k,v}^2\left(x,\frac{T}{2 \pi}t\right)\,\text{d}x\text{d}t=1.\nonumber
\end{equation}
According to \cite[Theorem 1.1]{Chavel}, it can be concluded that $\phi_{k,v}\in \mathcal{C}^{\infty}\left(\overline{C_{1+v}^T}\right)$ and ${\phi}_{k,0}=\pm{\phi}_k$, $\lambda_{k,0}=\lambda_k$.
In addition, it follows from the Implicit Function Theorem that ${\phi}_{k,v}$ and $\lambda_{k,v}$ depend smoothly on $v$.
Without loss of generality, we assume ${\phi}_{k,0}={\phi}_k$.
For any fixed $t$, ${\phi}_{k,v}$ is radially symmetric with respect to the first variable.
Hence, for any fixed $t$, ${\phi}_{k,v}(r,t)$ has exactly $k$ simple zeros in $(0,1+v(t))$ which are denoted by $r_1(t), \ldots, r_{k}(t)$.
The Implicit Function Theorem implies that $r_j(t)$ with $j\in\{1,\ldots,k\}$ is $C^{\infty}$ in local. By the arbitrariness of $t$, ${\phi}_{k,v}(x,t)$ has exactly $k$ nodal domains.
That is to say, $r_j(t)$ with $j\in\{1,\ldots,k\}$ is the zero line of ${\phi}_{k,v}(r,t)$.

Define the operator
\begin{equation}
\mathcal{N}(v,T)={\phi}_{k,v}\Big|_{\partial C_{1+v}^T}-\phi_k(1).\nonumber
\end{equation}
We point out that the construction of operator $\mathcal{N}$ is different from that of \cite[Proposition 3.2]{Sicbaldi} due to the different boundary conditions.
Note that the operator $\mathcal{N}$ depends only on the variable $t$. Thus, we can define
\begin{equation}
F(v,T)=\mathcal{N}(v,T)(t)\left(\frac{T}{2\pi}t\right).\nonumber
\end{equation}
Since ${\phi}_k(1)$ is a constant, it is easy to check that $F(0,T)=0$ for any $T>0$.
Therefore, finding nontrivial domains emanating from $B_1\times \mathbb{R}$ such that problem (\ref{SchifferConjecture}) has a sign-changing solution
is equivalent to study the nontrivial solutions of $F(v,T)=0$.

To use the Crandall-Rabinowitz bifurcation theorem to $F(v,T)=0$, the key is to find the degenerate point of the linearization operator $F_{v}(0,T)$ and verify the transversality condition. Thus let us first consider the following equation
\begin{equation}\label{ckequation1}
\left(\partial_r^2+\frac{N-1}{r}\partial_r+\lambda_k\right)c-\left(\frac{2m\pi}{T}\right)^2c=0
\end{equation}
with $m\in \mathbb{N}$, $c'(1)=-\phi_k''(1)$ and $c'(0)=0$.
\\ \\
\textbf{Proposition 2.2.} \emph{There is no solution to problem (\ref{ckequation1}) when $T=2m\pi/\sqrt{\lambda_k}:=T_0$ or $T=T_i:=2m\pi/\sqrt{\lambda_k-\lambda_{i}}$ for each $i\in\{1,\ldots,k-1\}$, however problem (\ref{ckequation1}) has a unique solution $c_m$ if $T\neq T_0$ and $T\neq2m\pi/\sqrt{\lambda_k-\lambda_{i}}$ for all $i\in\{1,\ldots,k-1\}$. Moreover, $c_m$ is analytic for $T\in(0,+\infty)\setminus\left\{T_0, T_1,\ldots,T_{k-1}\right\}$.}
\\ \\
\textbf{Proof.} We first prove the nonexistence as $T=T_i$ by contradiction. For $T=T_0$, it is easy to see that
\begin{equation}
\left(\partial_r^2+\frac{N-1}{r}\partial_r\right)c=0,\nonumber
\end{equation}
which implies that
$$\left(r^{N-1}c'\right)'=0.$$
Then integrating from $0$ to $1$ and using $c'(0)=0$, we have that
\begin{equation}
c'(1)=0,\nonumber
\end{equation}
which contradicts the fact of $c'(1)=-\phi_k''(1)\neq0$. Thus the problem (\ref{ckequation1}) has no solution when $T=T_0$.
For $T=T_i$ with $i\in\{1,\ldots,k-1\}$, we have that
\begin{equation}
\left(\partial_r^2+\frac{N-1}{r}\partial_r+\lambda_{i}\right)c=0\nonumber
\end{equation}
$c'(1)=-\phi_k''(1)$ and $c'(0)=0$.
Note that $\phi_i$ satisfies
\begin{equation}
\left\{
\begin{array}{ll}
\left(r^{N-1}\phi_i'\right)'+\lambda_i r^{N-1}\phi_i=0\,\, &\text{in}\,\, (0,1),\\
\phi_i'(0)=\phi_i'(1)=0.
\end{array}
\right.\nonumber
\end{equation}
Multiplying the equation of $c$ by $\phi_i$ and the equation of $\phi_i$ by $c$ and integrating by parts, we obtain that
$$
c'(1)\phi_i(1)-\int^1_0 r^{N-1}c'\phi_i'\,\text{d}r+\lambda_i\int^1_0 r^{N-1}c \phi_i \,\text{d}r=0
$$
and
$$
\phi_i'(1)c(1)-\int^1_0 r^{N-1}c'\phi_i'\,\text{d}r+\lambda_i\int^1_0 r^{N-1}c \phi_i \,\text{d}r=0,
$$
which gives that
\begin{eqnarray}
0=\phi_i'(1)c(1)=c'(1)\phi_i(1)=-\phi_k''(1)\phi_i(1).\nonumber
\end{eqnarray}
It follows that $\phi_i(1)=0$, which is impossible due to $\phi_i'(1)=0$.
Hence, there is no solution to the problem (\ref{ckequation1}) as $T=T_i$ for each $i\in\{1,\ldots,k-1\}$.

Set $\vartheta(r):=c(r)+\phi_k''(1)r^2/2$. Then we find that
\begin{equation}\label{vckequation}
\left\{
\begin{array}{ll}
\left(\partial_r^2+\frac{N-1}{r}\partial_r+\left(\lambda_k-\left(\frac{2m\pi}{T}\right)^2\right)\right)\vartheta
=\phi_k''(1)\left(N+\frac{r^2}{2}\left(\lambda_k-\left(\frac{2m\pi}{T}\right)^2\right)\right),\\
\vartheta'(0)=\vartheta'(1)=0.
\end{array}
\right.
\end{equation}
If $T\neq T_0$ and $T\neq T_i$ for all $i\in\{1,\ldots,k-1\}$,
we see that
\begin{equation}
\lambda_k-\left(\frac{2m\pi}{T}\right)^2<\lambda_k,\,\,\,\lambda_k-\left(\frac{2m\pi}{T}\right)^2\neq0\,\,\, \text{and}\,\,\,\lambda_k-\left(\frac{2m\pi}{T}\right)^2\neq\lambda_i.\nonumber
\end{equation}
Using the Fredholm alternative theorem \cite[Theorem 2.4]{Kazdan}, we have that the problem (\ref{vckequation}) has a unique solution.
In conclusion, the problem (\ref{ckequation1}) has a unique solution if and only if $T\neq T_i$ for all $i\in\{0,1,\ldots,k-1\}$.
We use $c_m$ to denote the unique solution.

We use the following fact to show the analyticity of $c_m$:
if $G$ is an invertible operator, by the equality
\begin{equation}
(I-s G)^{-1}=\sum_{i\geq0}s^iG^i\nonumber
\end{equation}
for each $s\in \mathbb{R}$, the solution of
\begin{equation}
\left(G-\frac{\rho}{T^2} I\right)u=h\nonumber
\end{equation}
is analytic in $T$, where $I$ is the identity, $h$ is any continuous function and $\rho$ is a constant.
Let $\widetilde{X}=\left\{\varphi\in H_r^1\left(B_{1}\right):\partial_\nu \varphi=0\,\,\text{on}\,\,\partial B_{1}\right\}$ with $H_r^1=\left\{\varphi\in H^1:\varphi\,\,\text{is radially symmetric}\right\}$.
Take
\begin{equation}
G=\partial_r^2+\frac{(N-1)}{r}\partial_r+\lambda_k -\left(\frac{2m\pi}{T}\right)^2\nonumber
\end{equation}
acting on $\widetilde{X}$.
Then, for any $f\in C[0,1]$, we consider
\begin{equation}
G\vartheta=f(r).\nonumber
\end{equation}
Similarly as the above, we can obtain the existence and uniqueness of $\vartheta$.
Hence $G:\widetilde{X}\longrightarrow C[0,1]$ is invertible. Taking $\rho=0$ and
\begin{equation}
h=\phi_k''(1)\left(N+\frac{r^2}{2}\left(\lambda_k-\left(\frac{2m\pi}{T}\right)^2\right)\right),\nonumber
\end{equation}
the analyticity of $\vartheta$ is concluded. Therefore, $c_k$ is analytic with respect to $T$.
\qed\\

By Fourier expansion $v$ can be written as
\begin{equation}
v=\sum_{m\geq1}a_m\cos(mt).\nonumber
\end{equation}
Write
\begin{equation}
\sum_{m\geq1}c_m(r)a_m\cos\left(\frac{2m\pi t}{T}\right):=\psi.\nonumber
\end{equation}
In view of Proposition 2.2, then we can verify that $\psi$ satisfies the following problem
\begin{equation}\label{eigenvalueonc1cvnu=0v=0}
\left\{
\begin{array}{ll}
\Delta_{\mathring{g}}  \varphi+\lambda_k\varphi=0\,\, &\text{in}\,\, C_{1}^{T},\\
\partial_\nu\varphi=-\partial_{r}^2\phi_kv\left(\frac{2\pi t}{T}\right) &\text{on}\,\, \partial C_{1}^{T}.
\end{array}
\right.
\end{equation}
Conversely, we can show that
\\ \\
\textbf{Proposition 2.3.} \emph{The problem (\ref{eigenvalueonc1cvnu=0v=0}) has a unique solution. In particular, if $T\ne T_i$ for all $i\in\{0,1,\ldots,k-1\}$, the unique solution is just $\psi$.}
\\ \\
\textbf{Proof.} We use the Fredholm alternative to show the existence and uniqueness.
Let $w=\varphi+x\partial_r^2{\phi}_k(1)v\left(\frac{2\pi t}{T}\right)$. Then we have that
\begin{equation}
\left\{
\begin{array}{ll}
-\Delta_{\mathring{g}}  w-\lambda_kw=x\phi_k''(1)\left(\left(2m\pi/T\right)^2-\lambda_k\right)v\,\, &\text{in}\,\, C_{1}^{T},\\
\partial_\nu w=0 &\text{on}\,\, \partial C_{1}^{T}.
\end{array}
\right.\nonumber
\end{equation}
By Proposition 2.1, the homogeneous problem
\begin{equation}
\left\{
\begin{array}{ll}
-\Delta_{\mathring{g}}  w-\lambda_kw=0\,\, &\text{in}\,\, C_{1}^{T},\\
\partial_\nu w=0 &\text{on}\,\, \partial C_{1}^{T}
\end{array}
\right.\nonumber
\end{equation}
has nontrivial solution $\phi_k$ and the eigenspace is just $\left\{\phi_k\right\}$.
Since $v$ is $L^2\left(C_1^T\right)$-orthogonal to $\phi_k$, the Fredholm alternative theorem \cite[Theorem 2.4]{Kazdan} can be used to show the existence and uniqueness.
\qed\\

From now on, we always assume that $T\neq2m\pi/\sqrt{\lambda_k-\lambda_i}$ for each $m\in \mathbb{N}$ and all $i\in\{0,1,\ldots,k-1\}$.
Define
\begin{equation}
\widetilde{H}_T(v)=\psi\big|_{\partial C_{1}^{T}}\nonumber
\end{equation}
and
\begin{equation}
{H}_T(v)=\widetilde{H}_T(v)\left(\frac{T}{2\pi}t\right).\nonumber
\end{equation}
The following result indicates that the linearized operator of $F$ with respect to $v$ at point $(0,T)$ is just
${H}_T$.
\\ \\
\textbf{Proposition 2.4.} \emph{The linearization operator of $F$ at $(0,T)$ satisfies $F_{v}(0,T)w={H}_T (w)$ for any $w\in \mathcal{C}^{2,\alpha}_{\text{even},0}\left(\mathbb{R}/2\pi \mathbb{Z}\right)$.}
\\ \\
\textbf{Proof.} Although the argument is similar to that of \cite[Proposition 3.4]{Sicbaldi}, we still present it here because it is important and there are some essential differences which are arose by different boundary conditions.

For $y\in \mathbb{R}^N$ and $t\in \mathbb{R}$, then the parameterization of $C^{T}_{1+v}$ can be defined by
$$
Y(y,t):=\left( (1+s\chi(y)w)y,~~\frac{Tt}{2\pi} \right),
$$
where $\chi$ is a cutoff function that identically equals to 0 when $|y|\leq 1/2$, and identically equals to 1 when $|y|\geq 3/4$. Let $\mathring{g}$ represent the usual Euclidean metric. The metric induced by $Y$ is given by
$$
\hat{g}:=Y^* \mathring{g}
$$
such that $\hat{\phi}=Y^*\phi$ and $\hat{\lambda}=\lambda$ are solutions of
\begin{equation}
\left\{
\begin{array}{ll}
-\Delta_{\hat{g}}  \hat{\phi}-\lambda\hat{\phi}=0\,\, &\text{in}\,\, C_{1}^{2\pi},\\
\partial_\nu \hat{\phi}=0 &\text{on}\,\, \partial C_{1}^{2\pi}.
\end{array}
\right.\nonumber
\end{equation}
%with
%$$
%\int_{C_{R,1}^{2\pi}}\hat{\phi}^2~\text{dvol}_{\hat{g}}=1.
%$$
It is obvious that $\hat{\phi}_{k}=Y^*\phi_k$ is a solution of
$$
-\Delta_{\hat{g}}  \hat{\phi}_k-\lambda_k\hat{\phi}_k=0
$$
due to $\hat{g}=Y^* \mathring{g}$. Moreover,
\begin{equation} \label{eq2.8}
\hat{\phi}_k(y,t)=\phi_{k}\left( (1+sw)y,~\frac{Tt}{2\pi} \right)
\end{equation}
on $\partial C_{1}^{2\pi}$. Let $\hat{\phi}=\hat{\phi}_k+\hat{\varphi}$ and $\hat{\lambda}=\lambda_k+\mu$, we have that
\begin{equation}\label{eq2.9}
\left\{
\begin{array}{ll}
\Delta_{\hat{g}}  \hat{\varphi}+\left(\lambda_k+\mu\right)\hat{\varphi}+\mu \hat{\phi}_k=0\,\, &\text{in}\,\, C_{1}^{2\pi},\\
\partial_\nu\hat{\varphi}=-\partial_\nu\hat{\phi}_k &\text{on}\,\, \partial C_{1}^{2\pi}.
\end{array}
\right.
\end{equation}
%with
%\begin{equation}\label{eq2.10}
%\int_{C_{R,1}^{2\pi}} (2\hat{\phi}_1\hat{\varphi} +\hat{\varphi}^2) ~\text{dvol}_{\hat{g}}=\int_{C_{R,1}^{2\pi}} \phi_1^2~\text{dvol}_{\mathring{g}}-\int_{C_{1+sw}^{2\pi}} \phi_1^2~\text{dvol}_{\mathring{g}}.
%\end{equation}
It is easy to see that $\hat{\varphi}$ and $\mu$ are the smooth functions of $s$. When $s=0$, we have $\hat{\phi}=\hat{\phi_k}$, $\hat{\lambda}=\lambda_k$ and $\hat{g}=\mathring{g}$.

Define
$$
\dot{\varphi}=\partial_s \hat{\varphi}|_{s=0}, ~~~\dot{\mu}=\partial_s \mu|_{s=0}.
$$
Differentiating (\ref{eq2.9}) with respect to $s$ and evaluating the value at $s=0$, we obtain
\begin{equation}\label{eq2.10}
\left\{
\begin{array}{ll}
\Delta_{\mathring{g}}  \dot{\varphi}+\lambda_k\dot{\varphi}+\dot{\mu}\phi_k=0\,\, &\text{in}\,\, C_{1}^{2\pi},\\
\partial_\nu\dot{\varphi}=-\partial_{r}^2\phi_kw &\text{on}\,\, \partial C_{1}^{2\pi}
\end{array}
\right.
\end{equation}
due to (\ref{eq2.8}). Multiplying the first equation of (\ref{eq2.10}) by $\phi_k$, integrating over $C_{1}^{2\pi}$ and using the boundary condition and the fact that the average of $w$ is 0, we can deduce that
\begin{equation}
\dot{\mu}\int_{C_1^{2 \pi}}{\phi}_k^2\,\text{d}x\text{d}t=0.\nonumber
\end{equation}
It follows that
$\dot{\mu}=0$ and $\dot{\varphi}\left(2\pi t/T\right)$ solves (\ref{eigenvalueonc1cvnu=0v=0}). That is to say, we have that
$$
\hat{\phi}(x,t)=\hat{\phi}_k(x,t)+s\psi\left(x, Tt/2\pi\right)+\circ(s),
$$
where $\psi$ is the solution of (\ref{eigenvalueonc1cvnu=0v=0}). In particular, for $r:=|y|$ we have
$$
\hat{\phi}(y,t)=\phi_k\left(y,Tt/2\pi\right)+s\left(wr\partial_r\phi_k+\psi\left(y,Tt/2\pi\right)\right)+\circ(s)
$$
in $C_{1}^{2\pi} \setminus C^{2\pi}_{3/4}$.
Therefore, we obtain that
$$
F_{v}(0,T)w=\lim_{s\rightarrow 0}\frac{F(sw,T)-F(0,T)}{s}=w\partial_r\phi_k(1)+\psi\left(1,Tt/2\pi\right)=\psi\left(1,Tt/2\pi\right),\nonumber
$$
which is the desired conclusion.
\qed \\

Let $V_m$ be the space spanned by the function $\cos(mt)$.
Propositions 2.3--2.4 imply that $H_T$ preserves the eigenspaces $V_m$.
Let $\sigma_m(T)$ be the eigenvalues of $H_T$ with respect to the eigenfunctions $\cos(mt)$.
Similar to that of \cite{Sicbaldi}, we have that
\begin{equation}
\sigma_m(T)=c_m(1),\nonumber
\end{equation}
where $c_m$ is the continuous solution of (\ref{ckequation1}) on $[0,1]$.
Note that $\sigma_m(T)=\sigma_1\left(T/m\right)$, which indicates the property of $\sigma_m$ can be deduced from the property of $\sigma_1$. So we next only consider the case of $m=1$.
The analyticity of $c_1$ implies that $\sigma_1(T)$ is differentiable.
The zero (if it exists) of $\sigma_1(T)$ is just the degenerate point of ${H}_T$.

\subsection{Some properties of Bessel functions}
\quad\,
In this subsection, we state some important properties of Bessel functions (we refer to \cite[Chapter 10]{Olver} or \cite{Watson} for details), which would be used in the proofs of our main theorems.
For $\tau\in \mathbb{R}$, the Bessel function of the first kind is
\begin{equation}
J_\tau(s)=\sum_{m=0}^\infty\frac{(-1)^m\left(\frac{s}{2}\right)^{2m+\tau}}{m!\Gamma(\tau+m+1)},\nonumber
\end{equation}
which is a solution of the differential equation
\begin{equation}
s^2f''(s)+sf'(s)+\left(s^2-\tau^2\right)f(s)=0.\nonumber
\end{equation}
For any $\tau\in \mathbb{R}$ and $s>0$, we have the following relations
\begin{equation}\label{relationsforbesselfunction3}
sJ_{\tau}'(s)-\tau J_{\tau}(s)=-sJ_{\tau+1}(s).
\end{equation}
As $\tau$ is fixed and $s\rightarrow 0$, then
\begin{equation}\label{relationsforbesselfunctionz=0}
J_\tau(s)\thicksim\frac{\left(\frac{s}{2}\right)^\tau}{\Gamma(\tau+1)}
\end{equation}
for $\tau\neq-1,-2,-3,\ldots$.
If $\tau$ is real, then $J_\tau(s)$ has an infinite number of positive real zeros. All of
these zeros are simple, the $m$th positive zero of $J_\tau(s)$ is denoted by $j_{\tau,m}$.
When $\tau\geq0$, the zeros interlace according to the
inequalities
\begin{equation}\label{interlaceproperty}
j_{\tau,1}< j_{\tau+1,1} < j_{\tau,2} < j_{\tau+1,2} < j_{\tau,3} <\cdots.
\end{equation}
When $\tau\geq-1$, the zeros of $J_\tau(s)$ are all real.

The Bessel function of the second kind is defined by
\begin{equation}%\label{secondbesselfunctiony}
Y_\tau(s)=\left\{
\begin{array}{ll}
\frac{J_\tau(s)\cos(\tau \pi)-J_{-\tau}(s)}{\sin(\tau \pi)}\,\, &\text{if}\,\, \tau\not\in \mathbb{Z},\\
\lim_{p\rightarrow \tau}\frac{J_p(s)\cos(p \pi)-J_{-p}(s)}{\sin(p \pi)} &\text{if}\,\, \tau\in \mathbb{Z}.
\end{array}
\right.\nonumber
\end{equation}
When $\tau$ is fixed and $s\rightarrow 0$, it has that
\begin{equation}\label{Yrelationsforbesselfunctionz=0}
Y_0(s)\thicksim\frac{2}{\pi}\ln s,\,\,Y_\tau(s)\thicksim-\frac{1}{\pi}\Gamma(\tau)\left(\frac{s}{2}\right)^{-\tau}
\end{equation}
for $\tau>0$ or $\tau\neq-1/2,-3/2,-5/2,\ldots$.

For $\tau\in \mathbb{R}$, the modified Bessel function of the first kind is defined by
\begin{equation}%\label{secondbesselfunction}
I_\tau(s)=\sum_{m=0}^\infty\frac{\left(\frac{s}{2}\right)^{2m+\tau}}{m!\Gamma(\tau+m+1)},\nonumber
\end{equation}
which is the solution of the following modified Bessel's differential equation
\begin{equation}
s^2f''(s)+sf'(s)-\left(s^2+\tau^2\right)f(s)=0.\nonumber
\end{equation}
Clearly, $I_\tau(s)>0$ for all $s>0$.
For any $\tau\in \mathbb{R}$, we have that
\begin{equation}%\label{weifengongshi}
\frac{d}{d s}\left(\frac{I_\tau(s)}{s^{\tau}}\right)=\frac{I_{\tau+1}(s)}{s^{\tau}}.\nonumber
\end{equation}
It follows that
\begin{equation}\label{weifengongshi5}
sI_{\tau+1}(s)=sI_{\tau}'(s)-\tau I_{\tau}(s).
\end{equation}
As $s\rightarrow0$, the function $I_\tau(s)$ has the following asymptotic property
\begin{equation}\label{asymptoticproperties01}
I_\tau(s)\sim\frac{\left(\frac{s}{2}\right)^\tau}{\Gamma(\tau+1)}, \,\,-\tau\not\in \mathbb{Z}.
\end{equation}
While, as $s\rightarrow+\infty$, we have that
\begin{equation}\label{asymptoticproperties}
I_\tau(s)\sim \frac{e^s}{\sqrt{2\pi s}}.
\end{equation}

The modified Bessel function of the second kind is
\begin{equation}%\label{secondbesselfunctionk}
K_\tau(s)=\left\{
\begin{array}{ll}
\frac{\pi}{2}\left[\frac{I_{-\tau}(s)-I_\tau(s)}{\sin(\tau s)}\right]\,\, &\text{if}\,\, \tau\not\in \mathbb{Z},\\
\lim_{p\rightarrow \tau}\frac{\pi}{2}\left[\frac{I_{-p}(s)-I_p(s)}{\sin(ps)}\right] &\text{if}\,\, \tau\in \mathbb{Z},
\end{array}
\right.\nonumber
\end{equation}
which is also the solution of the modified Bessel's differential equation.
As $s\rightarrow0$, the function $K_\tau(s)$ has the following asymptotic properties
\begin{equation}\label{02asymptoticproperties}
K_\tau(s)\sim\frac{1}{2}\left(\frac{s}{2}\right)^\tau \Gamma(\tau), \,\,\tau>0,
\end{equation}
and
\begin{equation}\label{03asymptoticproperties}
K_0(s)\sim-\ln s.
\end{equation}

\section{The properties of $\sigma_1(T)$ for $N=1$}

\quad\, For $N=1$, we have that
\begin{equation} \label{eq5.1}
\lambda_k=k^2\pi^2
\end{equation}
and
\begin{equation}
\phi_k=(-1)^k\frac{1}{\sqrt{2\pi}}\cos(k\pi r).\nonumber
\end{equation}
It is easy to see that
\begin{equation}
\sigma(T)=c(1), \nonumber
\end{equation}
where $c$ is the continuous function on $[0,1]$ solving
\begin{equation}\label{n=1ckequation}
\left(\partial_r^2+\lambda_k\right)c-\left(\frac{2\pi}{T}\right)^2c=0
\end{equation}
with $c'(1)=-\phi_k''(1)=k^2\pi^2/\sqrt{2\pi}$ and $c'(0)=0$. Proposition 2.2 implies $\sigma(T)$ is analytic when $T\neq T_i$ for any $i\in\{0,1,\ldots,k-1\}$. Based on (\ref{eq5.1}), we deduce that $T_i=2/\sqrt{k^2-i^2}$. From now on, we use $T_{k}$ to
denote $+\infty$ for convenience.
\\ \\
\textbf{Proposition 3.1.} \emph{The function $\sigma(T):=\sigma_1(T)$ is strictly increasing in $\left(0,T_0\right)$ with $\lim_{T\rightarrow 0^+}\sigma(T)=0$. For each $i\in\{1,\ldots,k\}$, the function $\sigma:\left(T_{i-1},T_i\right)\rightarrow \mathbb{R}$ is strictly increasing with exact
one zero $T_{i,*}=4/\sqrt{(2k)^2-(2i-1)^2}$. Moreover, for each $i\in\{0,1,\ldots,k-1\}$, the $\sigma(T)$ satisfies
$\lim_{T\rightarrow T_i^-}\sigma(T)=+\infty$, $\lim_{T\rightarrow T_i^+}\sigma(T)=-\infty$ and $\lim_{T\rightarrow+\infty}\sigma(T)=+\infty$.}
\\ \\
\textbf{Proof.} Let
\begin{equation}
\alpha(T)=k^2\pi^2-\left(\frac{2\pi}{T}\right)^2.\nonumber
\end{equation}
Then we have that
\begin{equation}
\alpha\left(\frac{2}{k}\right)=0,\nonumber
\end{equation}
that is to say $T_0=2/k$.
By the Euler undetermined exponential function method, the characteristic equation of (\ref{n=1ckequation}) is $\xi^2+\alpha=0$.
Then the characteristic roots are:
\begin{equation}
\xi=\left\{
\begin{array}{ll}
\pm\sqrt{-\alpha}\,\, &\text{if}\,\, \alpha<0,\\
\pm i\sqrt{\alpha}\,\, &\text{if}\,\, \alpha>0.
\end{array}
\right.\nonumber
\end{equation}

Then the general solution is
\begin{equation}
c(r)=\left\{
\begin{array}{ll}
Ae^{\sqrt{-\alpha(T)}r}+Be^{-\sqrt{-\alpha(T)}r}\,\, &\text{if}\,\, \alpha<0,\\
A\cos\left(\sqrt{\alpha(T)}r\right)+B\sin\left(\sqrt{\alpha(T)}r\right) &\text{if}\,\, \alpha>0.
\end{array}
\right.\nonumber
\end{equation}
Then we deduce that
\begin{equation}
c'(r)=\left\{
\begin{array}{ll}
A\sqrt{-\alpha(T)}e^{\sqrt{-\alpha(T)}r}-B\sqrt{-\alpha(T)}e^{-\sqrt{-\alpha(T)}r}\,\, &\text{if}\,\, \alpha<0,\\
-A\sqrt{\alpha(T)}\sin\left(\sqrt{\alpha(T)}r\right)+B\sqrt{\alpha(T)}\cos\left(\sqrt{\alpha(T)}r\right) &\text{if}\,\, \alpha>0.
\end{array}
\right.\nonumber
\end{equation}
Combining with boundary conditions, we can obtain
\begin{equation}%\label{eq5.3}
c'(1)=\left\{
\begin{array}{ll}
A\sqrt{-\alpha(T)}e^{\sqrt{-\alpha(T)}r}-B\sqrt{-\alpha(T)}e^{-\sqrt{-\alpha(T)}r}
=\frac{k^{2}\pi^{2}}{\sqrt{2\pi}}\,\, &\text{if}\,\, \alpha<0,\\
-A\sqrt{\alpha(T)}\sin\left(\sqrt{\alpha(T)}\right)+B\sqrt{\alpha(T)}\cos\left(\sqrt{\alpha(T)}\right)
=\frac{k^{2}\pi^{2}}{\sqrt{2\pi}} &\text{if}\,\, \alpha>0
\end{array}
\right.\nonumber
\end{equation}
and
\begin{equation}
c'(0)=\left\{
\begin{array}{ll}
A\sqrt{-\alpha(T)}-B\sqrt{-\alpha(T)}=0\,\, &\text{if}\,\, \alpha<0,\\
B\sqrt{\alpha(T)}=0 &\text{if}\,\, \alpha>0.
\end{array}
\right.\nonumber
\end{equation}
It follows that $A=B$ for $\alpha<0$, and $B=0$ for $\alpha>0$.
We further infer that
\begin{equation}
A=\left\{
\begin{array}{ll}
\frac{ k^2\pi^2}{2\sqrt{-2\pi\alpha(T)}}\frac{1}{\sinh\left(\sqrt{-\alpha(T)}\right)}\,\, &\text{if}\,\, \alpha<0,\\
-\frac{ k^2\pi^2}{\sqrt{2\pi\alpha(T)}}\frac{1}{\sin\left(\sqrt{\alpha(T)}\right)} &\text{if}\,\, \alpha>0.
\end{array}
\right.\nonumber
\end{equation}
Therefore, we obtain the solution of (\ref{n=1ckequation}) is
\begin{equation}
c(r)=\left\{
\begin{array}{ll}
\frac{k^2\pi^2}{\sqrt{-2\pi\alpha(T)}}\frac{\cosh\left(\sqrt{-\alpha(T)}r\right)}{\sinh\left(\sqrt{-\alpha(T)}\right)}\,\, &\text{if}\,\, \alpha<0,\\
-\frac{k^2\pi^2}{\sqrt{2\pi\alpha(T)}}\frac{\cos\left(\sqrt{\alpha(T)}r\right)}{\sin\left(\sqrt{\alpha(T)}\right)} &\text{if}\,\, \alpha>0.
\end{array}
\right.\nonumber
\end{equation}

In particular, we have that
\begin{equation}
c(1)=\left\{
\begin{array}{ll}
\frac{k^2\pi^2}{\sqrt{-2\pi\alpha(T)}}\coth\left(\sqrt{-\alpha(T)}\right)\,\, &\text{if}\,\, \alpha<0,\\
-\frac{k^2\pi^2}{\sqrt{2\pi\alpha(T)}}\cot\left(\sqrt{\alpha(T)}\right) &\text{if}\,\, \alpha>0.
\end{array}
\right.\nonumber
\end{equation}
So we get that
\begin{equation}
\sigma(T)=\left\{
\begin{array}{ll}
\frac{k^2\pi^2}{\sqrt{-2\pi\alpha(T)}}\coth\left(\sqrt{-\alpha(T)}\right)\,\, &\text{if}\,\,T\in\left(0,\frac{2}{k}\right),\\
-\frac{k^2\pi^2}{\sqrt{2\pi\alpha(T)}}\cot\left(\sqrt{\alpha(T)}\right) &\text{if}\,\, T\in\left(\frac{2}{k},+\infty\right)\setminus\cup_{i=1}^{k-1}\left\{T_i\right\}.
\end{array}
\right.\nonumber
\end{equation}
Since $\coth (x)/x$ is strictly decreasing for $x>0$ and $\alpha(T)$ is strictly increasing, we see that $c(1)$ is strictly increasing with respect $T$ for $T\in\left(0,2/k\right)$.
According to the asymptotic behaviors of $\coth (x)$ and $\alpha(T)$, we can obtain the following asymptotic behaviors
\begin{equation}
\lim_{T\rightarrow0^+}\alpha(T)=0\,\,\text{and}\,\,\lim_{T\rightarrow (2/k)^-}\alpha(T)=+\infty.\nonumber
\end{equation}
Further, in view of the asymptotic behavior of $\cot (x)$ and the definition of $\alpha(T)$, we have that
\begin{equation}
\lim_{T\rightarrow (2/k)^+}\alpha(T)=-\infty.\nonumber
\end{equation}

Noting that
\begin{equation}
\alpha\left(T_i\right)=\lambda_i=i^2\pi^2\nonumber
\end{equation}
for each $i\in\{1,\ldots,k-1\}$, we see that $\sigma(T)$ is singular at $T=T_i$.
Therefore, for each $i\in\{1,\ldots,k-1\}$, $\sigma(T)$ satisfies $\lim_{T\rightarrow T_i^-}\sigma(T)=+\infty$ and $\lim_{T\rightarrow T_i^+}\sigma(T)=-\infty$.
Further, since $\lim_{T\rightarrow+\infty}\alpha(T)=\lambda_k=k^2\pi^2$, we have that
$\lim_{T\rightarrow+\infty}\sigma(T)=+\infty$.


The above asymptotic behavior implies that the function $\sigma$ is positive in $\left(0,2/k\right)$.
And $\sigma:\left(2/k,+\infty\right)\setminus\cup_{i=1}^{k-1}\left\{T_i\right\}\rightarrow \mathbb{R}$ for $i\in\{1,\ldots,k\}$ has at least
$k$ zeros. Furthermore, the monotonic strictly decreasing property of $\frac{\cot(x)}{x}$ for $x>0$ implies that
$\sigma$ has exactly $k$ zeros. In particular, there exist a unique $T_{i,*}\in\left(T_{i-1},T_i\right)$ such that $\sigma(T_{i,*})=0$ for each $i\in\{1,\ldots,k\}$.
\qed
\\ \\
\textbf{Remark 3.2.}
\emph{In fact, here we can further give the exact values of $T_{i,*}$.
For $T\in\left(T_{i-1},T_i\right)$ with $i\in\{1,\ldots,k\}$, we find that $\sigma(T)=0$ if and only if $\sqrt{\alpha\left(T\right)}=(2i-1)\pi/2$.
It follows from the definition of $\alpha\left(T\right)$ that
\begin{eqnarray}
T_{i,*}=\frac{4}{\sqrt{(2k)^2-(2i-1)^2}}\nonumber
\end{eqnarray}
for any $k\in \mathbb{N}$ and each $i\in\{1,\ldots,k\}$. It is not difficult to check that there exist some positive integer $l\geq 2$ such that $T_{i,*}=lT_{j,*}$ for $ i,j\in[1,k]$ and $i,j\in\mathbb{N}^+$ as in \cite{DaiZ}, then the case of multidimensional kernels must occur as stated in Theorem 1.2.
}\\
\section{The properties of $\sigma_1(T)$ for $N\geq2$}

\quad\, For the one-dimensional case, the displayed eigenvalues and eigenfunctions can be used to calculate directly the expression of $\sigma_1(T)$.
But for high-dimensional case, the displayed calculation method is no longer available. In this section, we mainly use the Bessel function method to study the characteristics of $\sigma_1(T)$.
Concretely, we shall prove the existence of zero to $\sigma_1(T)$ and study its properties at zero.
Let us first give the following asymptotic behavior of $\sigma_1$.
\\ \\
\textbf{Proposition 4.1.} \emph{Let $T_0=2\pi/\sqrt{\lambda_k}$ and $T_i=2\pi/\sqrt{\lambda_k-\lambda_i}$ for all $i\in\{1,\ldots,k-1\}$. The function $\sigma(T):=\sigma_1(T):(0,+\infty)\setminus\left\{T_1,\ldots,T_{k-1}\right\}\rightarrow \mathbb{R}$ has the following asymptotic behavior
\begin{eqnarray}
\lim_{T\rightarrow0^+}\sigma(T)=0,\,\,\lim_{T\rightarrow T_i^-}\sigma(T)=+\infty,\,\,\lim_{T\rightarrow T_i^+}\sigma(T)=-\infty,\,\,\lim_{T\rightarrow+\infty}\sigma(T)=+\infty\nonumber
\end{eqnarray}
for each $i\in\{0,1,\ldots,k-1\}$.}
\\ \\
\textbf{Proof.} For $T\in\left(0,T_0\right)$, define
\begin{equation}
\xi=\sqrt{\frac{4\pi^2}{T^2}-\lambda_k}.\nonumber
\end{equation}
Let $s=\xi r$ and $\widetilde{c}(s)=c_1(r)$. Then it satisfies
\begin{equation}
\left(\partial_s^2+\frac{N-1}{s}\partial_s-1\right)\widetilde{c}=0\nonumber
\end{equation}
with $\widetilde{c}'(0)=0$ and $\widetilde{c}'\left(\xi\right)=-\phi_k''(1)\xi^{-1}$.
Define $\widehat{c}(s)=s^\tau \widetilde{c}(s)$ with $\tau=(N-2)/2$. Then $\widehat{c}$ satisfies
\begin{equation}
\left(\partial_s^2+\frac{1}{s}\partial_r-\left(1+\frac{\tau^2}{s^2}\right)\right)\widehat{c}=0\nonumber
\end{equation}
with $\widehat{c}'(\xi)=\tau \xi^{\tau-1}c_1(1)-\xi^{\tau-1}\phi_k''(1)$, which is the modified Bessel's differential equation of order $\tau$.
The solution is given by
\begin{equation}
\widehat{c}(s)=AI_\tau(s)+BK_\tau(s)\nonumber
\end{equation}
for some constants $A$ and $B$ which are determined later.
Note that
\begin{equation}\label{limitatzero}
\lim_{s\rightarrow0}\widehat{c}(s)=\xi^{\tau}\lim_{r\rightarrow0}r^\tau c_1(r)=\xi^{\tau}\lim_{r\rightarrow0}r^\tau c_1(r)=\left\{
\begin{array}{ll}
0\,\, &\text{if}\,\, N>2,\\
c_1(0) &\text{if}\,\, N=2.
\end{array}
\right.
\end{equation}
Then, in view of (\ref{asymptoticproperties01})--(\ref{03asymptoticproperties}), we have that
\begin{equation}
B=\lim_{s\rightarrow0}\frac{\widehat{c}(s)}{K_\tau(s)}-A\lim_{s\rightarrow0}\frac{I_\tau(s)}{K_\tau(s)}=0.\nonumber
\end{equation}
So we get that
\begin{equation}
\widehat{c}(s)=AI_\tau(s),\nonumber
\end{equation}
where the constant $A$ is chosen such that
\begin{equation}
AI_\tau'(\xi)=\tau \xi^{\tau-1}c_1(1)-\xi^{\tau-1}\phi_k''(1).\nonumber
\end{equation}
Hence, we have that
\begin{eqnarray}
c_1(r)&=&\widehat{c}(s)s^{-\tau}=AI_\tau(s)s^{-\tau}=AI_\tau(\xi r)(\xi r)^{-\tau}\nonumber\\
&=&\frac{\tau \xi^{\tau-1}c_1(1)-\xi^{\tau-1}\phi_k''(1)}{I_\tau'(\xi)}I_\tau(\xi r)(\xi r)^{-\tau}\nonumber\\
&=&\frac{\tau c_1(1)-\phi_k''(1)}{\xi I_\tau'(\xi)}I_\tau(\xi r)r^{-\tau}.\nonumber
\end{eqnarray}
It follows that
\begin{eqnarray}
c_1(1)=\frac{\tau c_1(1)-\phi_k''(1)}{\xi I_\tau'(\xi)}I_\tau(\xi ).\nonumber
\end{eqnarray}
Further, we have that
\begin{eqnarray}
c_1(1)=\frac{\phi_k''(1)I_\tau(\xi )}{\tau I_\tau(\xi )-\xi I_\tau'(\xi)}.\nonumber
\end{eqnarray}
By (\ref{weifengongshi5}) we obtain that
\begin{eqnarray}
c_1(1)=-\frac{\phi_k''(1)I_\tau(\xi )}{\xi I_{\tau+1}(\xi)}.\nonumber
\end{eqnarray}
Therefore, in view of (\ref{asymptoticproperties}), we find that
\begin{eqnarray}
\lim_{T\rightarrow0^+}c_1(1)&=&-\phi_k''(1)\lim_{\xi\rightarrow+\infty}\frac{I_\tau(\xi )}{\xi I_{\tau+1}(\xi)}\nonumber\\
&=&-\phi_k''(1)\lim_{\xi\rightarrow+\infty}\frac{1}{\xi}\nonumber\\
&=&0.\nonumber
\end{eqnarray}
Since $\phi_k'(1)=0$ and $\phi_k(1)>0$, by (\ref{eigenvalueonball1}), we have that $\phi_k''(1)<0$.
Hence we get that
\begin{eqnarray}
\lim_{T\rightarrow0^+}c_1(1)=0,\nonumber
\end{eqnarray}
which is the desired asymptotic behavior of $\sigma_1$ as $T\rightarrow0^+$.

Since $I_\tau(\xi)>0$ and $I_{\tau+1}(\xi)>0$, we find that
$c_1(1)>0$ for $T\in\left(0,T_0\right)$. Based on the definition, we can deduce that $\xi$ is strictly decreasing.
It follows from \cite[Claim 6.4]{Schlenk} that
\begin{eqnarray}
f(s)=\frac{I_\tau(s)}{s I_{\tau+1}(s)}\nonumber
\end{eqnarray}
is strictly decreasing for $s>0$. Then $c_1(1)$ is strictly increasing with respect to $T\in\left(0,T_0\right)$.
Moreover, by virtue of (\ref{asymptoticproperties01}), we have that
\begin{eqnarray}
\lim_{T\rightarrow\mu^-}c_1(1)&=&\lim_{\xi\rightarrow0^+}c_1(1)=-\phi_k''(1)\lim_{\xi\rightarrow0^+}\frac{I_\tau(\xi )}{\xi I_{\tau+1}(\xi)}\nonumber\\
&=&-\phi_k''(1)\frac{\Gamma(\tau+2)}{\Gamma(\tau+1)}\lim_{\xi\rightarrow0^+}\frac{2}{\xi^2}\nonumber\\
&=&-2(\tau+1)\phi_k''(1)\lim_{\xi\rightarrow0^+}\frac{1}{\xi^2}\nonumber\\
&=&+\infty.\nonumber
\end{eqnarray}

We next study the the property of $\sigma$ when $T>T_0$. Define
\begin{equation}
\rho=\sqrt{\lambda_k-\left(\frac{2 \pi}{T}\right)^2}.\nonumber
\end{equation}
Let $s=\rho r$ and $\widehat{c}(s)=s^\tau c_1(r)$. Then we can verify that $\widehat{c}(s)$ satisfies
\begin{equation}
\left(\partial_s^2+\frac{1}{s}\partial_r+\left(1-\frac{\tau^2}{s^2}\right)\right)\widehat{c}=0\nonumber
\end{equation}
with $\widehat{c}'(\rho)=\rho^{\tau-1}\left(\tau c_1(1)-\phi_k''(1)\right)$, which is the Bessel's differential equation of order $\tau$.
Therefore,
the solution is given by
\begin{equation}
\widehat{c}(s)=AJ_\tau(s)+BY_\tau(s)\nonumber
\end{equation}
for some constants $A$ and $B$.
By (\ref{relationsforbesselfunctionz=0})--(\ref{Yrelationsforbesselfunctionz=0}) and (\ref{limitatzero}) we obtain that
\begin{equation}
B=\lim_{s\rightarrow0}\frac{\widehat{c}(s)}{Y_\tau(s)}-A\lim_{s\rightarrow0}\frac{J_\tau(s)}{Y_\tau(s)}=0.\nonumber
\end{equation}
We further get, as the above case, that
\begin{equation}
A=\frac{\rho^{\tau-1}\left(\tau c_1(1)-\phi_k''(1)\right)}{J_\tau'(\rho)}.\nonumber
\end{equation}
Therefore, we deduce that
\begin{eqnarray}
c_1(r)&=&\widehat{c}(s)s^{-\tau}=AJ_\tau(s)s^{-\tau}=AJ_\tau(\rho r)(\rho r)^{-\tau}\nonumber\\
&=&\frac{\tau c_1(1)-\phi_k''(1)}{\rho J_\tau'(\rho)}J_\tau(\rho r)r^{-\tau}.\nonumber
\end{eqnarray}
In particular, we have that
\begin{eqnarray}
c_1(1)=\frac{\tau c_1(1)-\phi_k''(1)}{\rho J_\tau'(\rho)}J_\tau(\rho ),\nonumber
\end{eqnarray}
which implies that
\begin{eqnarray}
c_1(1)=\frac{\phi_k''(1)J_\tau(\rho )}{\tau J_\tau(\rho )-\rho J_\tau'(\rho)}.\nonumber
\end{eqnarray}
Using (\ref{relationsforbesselfunction3}) we get that
\begin{eqnarray}
c_1(1)=\frac{\phi_k''(1)J_\tau(\rho )}{\rho J_{\tau+1}(\rho)}.\nonumber
\end{eqnarray}

In view of (\ref{relationsforbesselfunctionz=0}), we have that
\begin{eqnarray}
\lim_{T\rightarrow\mu^+}c_1(1)&=&\lim_{\rho\rightarrow0^+}c_1(1)=\phi_k''(1)\lim_{\rho\rightarrow0^+}\frac{J_\tau(\rho )}{\rho J_{\tau+1}(\rho)}\nonumber\\
&=&\phi_k''(1)\frac{\Gamma(\tau+2)}{\Gamma(\tau+1)}\lim_{\rho\rightarrow0^+}\frac{2}{\rho^2}\nonumber\\
&=&2(\tau+1)\phi_k''(1)\lim_{\rho\rightarrow0^+}\frac{1}{\xi^2}\nonumber\\
&=&-\infty.\nonumber
\end{eqnarray}
From \cite{FallMW} we know that
\begin{eqnarray}
J_{\tau+1}(\sqrt{\lambda_i})=0\nonumber
\end{eqnarray}
for any $i\in\{1,\ldots,k\}$ and they are all zeros of $J_{\tau+1}(s)$ for $s\in\left(0,\sqrt{\lambda_k}\right]$.
So, one has that $j_{\tau+1,i}=\sqrt{\lambda_i}$ for any $i\in\{1,\ldots,k\}$.
So $\sigma$ is singular only at
\begin{eqnarray}
T_i=\frac{2\pi}{\sqrt{\lambda_k-\lambda_i}}\nonumber
\end{eqnarray}
for each $i\in\{1,\ldots,k-1\}$.

By the interlace property (\ref{interlaceproperty}), for each $i\in\{1,\ldots,k\}$, we have that
\begin{equation}\label{interlacepropertylambdak}
s J_{\tau+1}(s)J_{\tau}(s)<0\,\,\text{for}\,\,s\in\left(j_{\tau,i},j_{\tau+1,i}\right)
\end{equation}
and
\begin{equation}
s J_{\tau+1}(s)J_{\tau}(s)>0\,\,\text{for}\,\,s\in\left(j_{\tau+1,i},j_{\tau,i+1}\right).
\nonumber
\end{equation}
It follows that
\begin{equation}
\lim_{T\rightarrow T_i^-}\frac{J_{\tau}(\rho)}{\rho J_{\tau+1}(\rho)}=\lim_{\rho\rightarrow \left(\sqrt{\lambda_i}\right)^-}\frac{J_{\tau}(\rho)}{\rho J_{\tau+1}(\rho)}=-\infty\nonumber
\end{equation}
and
\begin{equation}
\lim_{T\rightarrow T_i^+}\frac{J_{\tau}(\rho)}{\rho J_{\tau+1}(\rho)}=\lim_{\rho\rightarrow \left(\sqrt{\lambda_i}\right)^+}\frac{J_{\tau}(\rho)}{\rho J_{\tau+1}(\rho)}=+\infty\nonumber
\end{equation}
for each $i\in\{1,\ldots,k-1\}$.
Therefore, for each $i\in\{1,\ldots,k-1\}$, we obtain that
\begin{equation}
\lim_{T\rightarrow T_i^-}\sigma(T)=+\infty\nonumber
\end{equation}
and
\begin{equation}
\lim_{T\rightarrow T_i^+}\sigma(T)=-\infty.\nonumber
\end{equation}

We finally study the asymptotic behavior of $\sigma$ as $T\rightarrow+\infty$.
Using (\ref{interlacepropertylambdak}) with $i=k$, we derive that
\begin{eqnarray}
\lim_{T\rightarrow+\infty}c_1(1)&=&\lim_{\rho\rightarrow\left(\sqrt{\lambda_k}\right)^-}\frac{\phi_k''(1)J_\tau(\rho )}{\rho J_{\tau+1}(\rho)}\nonumber\\
&=&\phi_k''(1)\lim_{\rho\rightarrow\sqrt{\lambda_k}}\frac{J_\tau(\rho )}{\rho J_{\tau+1}(\rho)}\nonumber\\
&=&+\infty.\nonumber
\end{eqnarray}
Therefore, we obtain that
\begin{eqnarray}
\lim_{T\rightarrow+\infty}\sigma(T)=+\infty,\nonumber
\end{eqnarray}
which is desired asymptotic behavior.
\qed\\

Proposition 4.1 implies that $\sigma(T)$ has at least $k$ zeros, which are the degenerate points to the linearization operator of $F$ at $v=0$. To further verify the transversality condition, we need to study the monotonicity of $\sigma(T)$.
\\ \\
\textbf{Proposition 4.2.} \emph{The function $\sigma(T):(0,+\infty)\setminus\left\{T_1,\ldots,T_{k-1}\right\}\rightarrow \mathbb{R}$ satisfies $\sigma'\left(T\right)>0$ for any $T\in (0,+\infty)\setminus\left\{T_1,\ldots,T_{k-1}\right\}$. In particular, $\sigma(T)$ has exactly $k$ zeros $T_{i,*}$ with $i\in\{1,\ldots,k\}$ such that $T_{i,*}\in\left(T_{i-1},T_i\right)$ and $\sigma'\left(T_{i,*}\right)>0$.}
\\ \\
\textbf{Proof.} From the argument of Proposition 4.1 we know that $\sigma(T)$ is strictly increasing with respect to $T\in\left(0,T_0\right)$ and satisfies
\begin{eqnarray}
\lim_{T\rightarrow0^+}\sigma(T)=0\,\,\text{and}\,\,\lim_{T\rightarrow T_0}\sigma(T)=+\infty.\nonumber
\end{eqnarray}
It follows from \cite[Claim 6.6]{Schlenk} that
\begin{eqnarray}
f(s)=\frac{J_\tau(s)}{s J_{\tau+1}(s)}\nonumber
\end{eqnarray}
is strictly decreasing for $s>0$. Hence, for each $i\in\{1,\ldots,k\}$, $\sigma(T):\left(T_{i-1},T_i\right)\rightarrow \mathbb{R}$ is strictly increasing.
The other conclusions can be derived immediately from the monotonicity of $\sigma(T)$.
\qed
\\

We would like to point out that the conclusions of Proposition 3.1 can not be included by the conclusions of Propositions 4.1--4.2 because the arguments of Propositions 4.1--4.2 may be not valid for one-dimensional case. Indeed, for $N=1$, the order number $\tau$ is negative, which leads to the loss of interlace property.

\section{Proofs of Theorems 1.1--1.2}

From Proposition 3.1 and Proposition 4.1-4.2, we conclude that $\sigma(T)$ has exactly $k$ zeros $T_{i,*}$ with $i\in\{1,\ldots,k\}$ such that
\begin{equation}
T_{i,*}\in\left(T_{i-1},T_i\right)\nonumber
\end{equation}
with $\sigma'\left(T_{i,*}\right)>0$ as follows.
% Figure environment removed \label{fig1}

We further obtain the following result.
\\ \\
\textbf{Proposition 5.1.} \emph{The kernel space of $D_v F\left(T_{1,*},0\right)$ is just $V_1$, where $V_1$ is the space spanned by the function $\cos(t)$. Moreover, for each $i\in\{2,\ldots,k\}$ and every $j\in\{1,\ldots,i-1\}$, if $T_{i,*}\neq lT_{j,*}$ for any $l\in \mathbb{N}$ with $l\geq2$, the kernel space of $D_v F\left(T_{i,*},0\right)$ is still $V_1$. While, for each $i\in\{2,\ldots,k\}$, if there exist $m$ elements of $\{1,\ldots,i-1\}$ with $m\leq i-1$, which are denoted by $j_n$ with $n\in\{1,\ldots,m\}$, such that $T_{i,*}= l_{j_n}T_{j_n,*}$ for some $l_{j_n}\in \mathbb{N}$ with $l_{j_n}\geq2$ and $n\in\{1,\ldots,m\}$, the kernel space of $D_v F\left(T_{i,*},0\right)$ is just $\left(\cup_{n=1}^{m}V_{l_{j_n}}\right)\cup V_1$.}
\\ \\
\textbf{Proof.} Clearly, $V_1$ is contained in the kernel space of $D_v F\left(T_{i,*},0\right)$ for every $i\in\{1,\ldots,k\}$.
Note that
\begin{equation}
\sigma_{m}(T)=\sigma\left(\frac{T}{m}\right)\nonumber
\end{equation}
for all $m\in\mathbb{N}$.
So $\sigma_{m}(T)$ has exactly one zero $mT_{i,*}$ in $\left(mT_{i-1},mT_i\right)$.
Since $\sigma\left(T_{1,*}/j\right)>0$ with any $j\geq2$, $V_j$ is not the kernel $D_v F\left(T_{1,*},0\right)$. Thus, the kernel space of $D_v F\left(T_{1,*},0\right)$ is just $V_1$.

For each $i\in\{2,\ldots,k\}$, if $T_{i,*}\neq lT_{j,*}$ for every $j\in\{1,\ldots,i-1\}$ and any $l\in \mathbb{N}$ with $l\geq2$, $\sigma\left(T_{i,*}/l\right)\neq0$.
So $V_l$ is not the kernel $D_v F\left(T_{i,*},0\right)$ for any $l\geq2$. Therefore, the kernel space of $D_v F\left(T_{i,*},0\right)$ is still $V_1$.
For each $i\in\{2,\ldots,k\}$, if there exist $m$ elements of $\{1,\ldots,i-1\}$ with $m\leq i-1$, which are denoted by $j_n$ with $n\in\{1,\ldots,m\}$, such that $T_{i,*}= l_{j_n}T_{j_n,*}$ for some $l_{j_n}\in \mathbb{N}$ with $l_{j_n}\geq2$ and $n\in\{1,\ldots,m\}$, we have that $\sigma\left(T_{i,*}/l_{j_n}\right)=\sigma\left(T_{j_n,*}\right)=0$.
So, for every $n\in\{1,\ldots,m\}$, $V_{l_{j_n}}$ is also contained in the kernel space of $D_v F\left(T_{i,*},0\right)$.
For any $j\not\in\left\{l_{j_1},\ldots,l_{j_m}\right\}$ with $j\geq2$, we see $\sigma\left(T_{i,*}/j\right)\neq0$.
So $V_j$ is not contained in the kernel $D_v F\left(T_{i,*},0\right)$ for any $j\not\in\left\{l_{j_1},\ldots,l_{j_m}\right\}$ with $j\geq2$.
Therefore, the kernel space of $D_v F\left(T_{i,*},0\right)$ is just $\left(\cup_{n=1}^{m}V_{l_{j_n}}\right)\cup V_1$.
\qed\\

Now we prove Theorem 1.1 by verifying the hypotheses of celebrated Crandall-Rabinowitz local bifurcation theorem \cite{Crandall}.
\\ \\
\textbf{Proof of Theorem 1.1.} By Proposition 5.2, the kernel of the
linearized operator $D_vF\left(0,T_{1,*}\right)$ is $1$-dimensional and is spanned by the function $\cos (t)$.
As that of \cite[Proposition 3.2]{Sicbaldi} with obvious changes we can show that $D_vF\left(0,T_{1,*}\right)$ is a formally self-adjoint, first order elliptic operator.
For the reader's convenience, we provide the proof of self-adjoint here.
Let $\psi_1$ (or $\psi_2$) be the solution of problem (\ref{eigenvalueonc1cvnu=0v=0}) corresponding to the function $v_2$.
Set $\widetilde{\psi}_i(x,t):={\psi}_i(x,Tt/2\pi)$.
Then we have that
\begin{eqnarray}
\phi_k''(1)\int_0^{2\pi} \left(H_T\left(v_2\right)v_1- H_T\left(v_1\right)v_2\right)\,\text{d}t&=&\phi_k''(1)\int_0^{2\pi} \left(\widetilde{\psi}_2v_1- \widetilde{\psi}_1v_2\right)\,\text{d}t\nonumber\\
&=&\int_0^{2\pi} \left(\widetilde{\psi}_1\partial_\nu\widetilde{\psi}_2-\widetilde{\psi}_2\partial_\nu\widetilde{\psi}_1 \right)\,\text{d}t\nonumber\\
&=&\frac{1}{\text{Vol}\left(\mathbb{S}^{N-1}\right)}\int_{C_1^{2\pi}}\left(\widetilde{\psi}_1\Delta\widetilde{\psi}_2-\widetilde{\psi}_2\Delta\widetilde{\psi}_1 \right)\,\text{d}x\text{d}t\nonumber\\
&=&-\frac{1}{\text{Vol}\left(\mathbb{S}^{N-1}\right)}\int_{C_1^{2\pi}}\left(\lambda_k\widetilde{\psi}_1\widetilde{\psi}_2-\lambda_k\widetilde{\psi}_2\widetilde{\psi}_1 \right)\,\text{d}x\text{d}t\nonumber\\
&=&0.\nonumber
\end{eqnarray}
It follows that $D_vF\left(0,T_{1,*}\right)$ has closed range. Therefore, $D_vF\left(0,T_{1,*}\right)$ is a Fredholm operator of index zero (refer to \cite{Kubrusly}). Then its codimension is equal to $1$.
In view of Proposition 3.1 and Proposition 4.2, we obtain
\begin{equation}
D_{Tv}F\left(0,T_{1,*}\right)\cos(t)=\sigma'\left(T_{1,*}\right)\cos(t)\not\in \text{Im}\left(D_vF\left(0,T_{1,*}\right)\right).\nonumber
\end{equation}
\indent Applying the Crandall-Rabinowitz local bifurcation theorem \cite{Crandall} to $F(v,T)=0$, we obtain that there exist an open interval $I= \left(-\varepsilon,\varepsilon\right)$ and continuous
functions $T : I\rightarrow \mathbb{R}$, $w : I \rightarrow  \text{Im}\left(D_v F\left(0,T_{1,*}\right)\right)$ such that $T(0) = T_{1,*}$, $w(0) = 0$ and $F(s\cos(t) + sw(s),T(s)) = 0$ for
$s\in I$ and $F^{-1}\{0\}$ near $\left(0,T_{1,*}\right)$ consists precisely of the curves $v =0$ and
$\Gamma = \left\{(v(s),T(s)): s \in I\right\}$. Therefore, for each $s\in  (-\varepsilon,\varepsilon)$, problem (\ref{SchifferConjecture}) has a positive $T(s)$-periodic solution
$u \in  \mathcal{C}^{2,\alpha}\left(\Omega_s\right)$ on the modified cylinder
\begin{equation}
\Omega_s=\left\{(x,t)\in \mathbb{R}^N\times \mathbb{R}:R<r(x)<1+s\cos \left(\frac{2\pi }{T(s)}t\right)+s w(s)\left(\frac{2\pi }{T(s)}t\right)\right\},\nonumber
\end{equation}
which is the desired conclusion.
\qed\\

Next we will establish Theorem 1.2 by verifying the hypotheses of Crandall-Rabinowitz local bifurcation theorem \cite[Proposition 4.2]{DaiZ} with high kernel.
\\ \\
\textbf{Proof of Theorem 1.2.} For each $i\in\{2,\ldots,k\}$ and every $j\in\{1,\ldots,i-1\}$, if $T_{i,*}\neq lT_{j,*}$ for any $l\in \mathbb{N}$ with $l\geq2$, the kernel space of $D_v F\left(T_{i,*},0\right)$ is $V_1$.
Then, in view of $\sigma'\left(T_{i,*}\right)\neq0$, repeating the argument as that of Theorem 1.1 we have the desired conclusion.

For each $i\in\{2,\ldots,k\}$, if there exist $m$ elements of $\{1,\ldots,i-1\}$ with $m\leq i-1$, which are denoted by $j_n$ with $n\in\{1,\ldots,m\}$, such that $T_{i,*}= l_{j_n}T_{j_n,*}$ for some $l_{j_n}\in \mathbb{N}$ with $l_{j_n}\geq2$ and $n\in\{1,\ldots,m\}$, it follows from Proposition 5.1 that
the kernel space of $D_v F\left(T_{i,*},0\right)$ is just $\left(\cup_{n=1}^{m}V_{l_{j_n}}\right)\cup V_1$.

Reasoning as that of Theorem 1.1, $D_vF\left(0,T_{i,*}\right)$ is a formally self-adjoint, first order elliptic operator and has closed range. So $D_vF\left(0,T_{i,*}\right)$ is also a Fredholm operator of index zero (refer to \cite{Kubrusly}) with its codimension $m+1$.
Since $\sigma'\left(T_{i,*}\right)\neq0$, we obtain
\begin{equation}
D_{Tv}F\left(0,T_{i,*}\right)\cos(t)=\sigma'\left(T_{i,*}\right)\cos(t)\not\in \text{Im}\left(D_vF\left(0,T_{i,*}\right)\right).\nonumber
\end{equation}
Furthermore, since $\sigma'\left(T_{j_n,*}\right)\neq0$, we obtain that
\begin{equation}
D_{Tv}F\left(0,T_{i,*}\right)\cos\left(l_{j_n}t\right)=\sigma'\left(\frac{T_{i,*}}{l_{j_n}}\right)\cos(it)=\sigma'\left(T_{j_n,*}\right)\cos\left(l_{j_n}t\right)\not\in \text{Im}\left(D_vF\left(0,T_{i,*}\right)\right).\nonumber
\end{equation}

Applying \cite[Proposition 4.2]{DaiZ} to $F(v,T)=0$, we obtain that there exist an open interval $I= \left(-\varepsilon,\varepsilon\right)$ and continuous
functions $T : I\rightarrow \mathbb{R}$, $w : I \rightarrow  \text{Im}\left(D_v F\left(0,T_{i,*}\right)\right)$ such that $T(0) = T_{i,*}$, $w(0) = 0$ and $F\left(s\left(\beta\cos(t)+\sum_{n=1}^m\gamma_n\cos\left(l_{j_n}t\right)\right) + sw(s),T(s)\right) = 0$ for
$s\in I$ with $\beta^2+\sum_{n=1}^m\gamma_n^2=1$ and $F^{-1}\{0\}$ near $\left(0,T_{i,*}\right)$ consists precisely of the curves $v =0$ and
$\Gamma = \left\{(v(s),T(s)): s \in I\right\}$. Therefore, for each $s\in  (-\varepsilon,\varepsilon)$, problem (\ref{SchifferConjecture}) has a positive $T(s)$-periodic solution
$u \in  \mathcal{C}^{2,\alpha}$ on the modified cylinder
\begin{equation}
\left\{(x,t)\in \mathbb{R}^{N+1}:r<1+s\left(\beta\cos \left(\frac{2\pi }{T(s)}t\right)+\sum_{n=1}^m\gamma_n\cos \left(\frac{2l_{j_n}\pi }{T(s)}t\right)\right)+s w(s)\left(\frac{2\pi }{T(s)}t\right)\right\},\nonumber
\end{equation}
as desired.\qed


%\\ \\
%\textbf{Acknowledgment}
%\bigskip\\
%\indent The first author thanks Academy of Mathematics and Systems Science of CAS for the
%invitation and hospitality during his visit.

\bibliographystyle{amsplain}
\makeatletter
\def\@biblabel#1{#1.~}
\makeatother

%\bibliography{mybib2}

\providecommand{\bysame}{\leavevmode\hbox to3em{\hrulefill}\thinspace}
\providecommand{\MR}{\relax\ifhmode\unskip\space\fi MR }
% \MRhref is called by the amsart/book/proc definition of \MR.
\providecommand{\MRhref}[2]{%
  \href{http://www.ams.org/mathscinet-getitem?mr=#1}{#2}
}
\providecommand{\href}[2]{#2}
\begin{thebibliography}{10}

%\bibitem{Aftalion} A. Aftalion and J. Busca, Sym\'{e}trie radiale pour des probl\`{e}mes elliptiques surd\'{e}termin\'{e}s pos\'{e}s dans des domaines
%ext\'{e}rieurs, C. R. Acad. Sci. Paris S\'{e}r. I Math. 324 (1997), 633--638.

%\bibitem{Ambrosetti} A. Ambrosetti and A. Malchiodi,
%Nonlinear analysis and semilinear elliptic problems, Cambridge Studies in Advanced Mathematics, vol. 104, Cambridge University Press, Cambridge, 2007.

\bibitem{Bagchi} S.C. Bagchi and A. Sitaram, The Pompeiu problem revisited, L'Enseignement Mathematique, 36 (1990), 67--91.

\bibitem{Berenstein} C.A. Berenstein, An inverse spectral theorem and its relation to the Pompeiu problem, J. Anal. Math. 37 (1980), 128--144.

\bibitem{Berenstein1} C.A. Berenstein and P.C. Yang, An overdetermined Neumann problem in the unit disk, Adv. Math. 44 (1982), 1--17.

\bibitem{Berenstein2} C.A. Berenstein and P.C. Yang, An inverse Neumann problem, J. Reine Angew. Math. 382 (1987), 1--21.

\bibitem{BCN} H. Berestycki, L.A. Caffarelli and L. Nirenberg, Monotonicity for elliptic equations in unbounded Lipschitz domains,
Comm. Pure Appl. Math. 50 (1997), 1089--1111.

%\bibitem{Brezis} H. Brezis, Functional analysis, Sobolev spaces and partial differential equations, Universitext. Springer, New York, 2011.

\bibitem{Chavel} I. Chavel, Eigenvalues in Riemannian Geometry, Academic Press, 1984.

%\bibitem{ChenLi} W. Chen and C. Li, Classification of solutions of some nonlinear elliptic equations, Duke Math. J. 63 (1991), 615--622.

%\bibitem{Chen} W. Chen and C. Li, Methods on nonlinear elliptic equations, In: AIMS Ser. Differ. Equ. Dyn. Syst. vol. 4, 2010.

\bibitem{Coddington} E.A. Coddington and N. Levinson, Theory of ordinary differential oquations, McGraw-Hill, New York, 1955.

\bibitem{Crandall} M.G. Crandall and P.H. Rabinowitz, Bifurcation from simple eigenvalues, J. Funct. Anal. 8 (1971), 321--340.

%\bibitem{DaiGZ} G. Dai, S. Gao and Y. Zhang, Positive solution to an overdetermined eigenvalue problem on nontrivial hollow cylinder, Preprint.

\bibitem{DaiZ} G. Dai and Y. Zhang, Sign-changing solution for an overdetermined elliptic problem on unbounded domain, arXiv:2304.05550.

\bibitem{Del} M. Del Pino, F. Pacard and J. Wei, Serrin's overdetermined problem and constant mean curvature
surfaces, Duke Math. J. 164 (2015), 2643--2722.

\bibitem{Deng} J. Deng, Some results on the Schiffer conjecture in $\mathbb{R}^2$, J. Differential Equations 253 (2012), 2515--2526.

%\bibitem{Evans} L.C. Evans, Partial differential equations, AMS, Providence, RI, 1998.

\bibitem{FallMW} M.M. Fall, I.A. Minlend and T. Weth, The schiffer problem on the cylinder and the $2$-sphere, arXiv:2303.17036v1.

%\bibitem{Farina} A. Farina and E. Valdinoci, Flattening results for elliptic PDEs in unbounded domains with applications to overdetermined problems, Arch. Ration. Mech. Anal. 195 (2010), 1025--1058.

%\bibitem{Farina1} A. Farina and E. Valdinoci, On partially and globally overdetermined problems of elliptic type, Amer. J. Math. 135 (2013), 1699--1726.

%\bibitem{Fragala} I. Fragal\`{a} and F. Gazzola, Partially overdetermined elliptic boundary value problems, J. Differential Equations 245 (2008), 1299--1322.

%\bibitem{Fragala1} I. Fragal\`{a} et. al., Counterexamples to symmetry for partially overdetermined elliptic problems, Analysis (Munich) 29 (2009), 85--93.

\bibitem{Friedlander} L. Friedlander, Some inequalities between Dirichlet and Neumann eigenvalues, Arch. Rat. Mech.
Anal. 116 (1991), 153--160.

%\bibitem{Gaunt} R.E. Gaunt, Variance-gamma approximation via Stein's method, Electron. J. Probab. 19 (2014), no. 38, 33 pp.

%\bibitem{Gilbarg} D. Gilbarg and N.S. Trudinger, Elliptic partial differential equations of second order, Springer-Verlag, Berlin, Heidelberg, 2001.

%\bibitem{Gidas} B. Gidas, W.M. Ni and L. Nirenberg, Symmetry and related properties via the maximum principle, Comm. Math.
%Phys. 68 (3) (1979), 209--243.

\bibitem{Ince} E.L. Ince, Ordinary differential equation, Dover Publication Inc., New York, 1927.

\bibitem{Kazdan} J.L. Kazdan, Applications of partial differential equations to problems in geometry, Grad. Texts in Math., Springer, 2004.

\bibitem{Kubrusly} C.S. Kubrusly, Fredholm theory in Hilbert space-A concise introductory exposition, Bull. Belg. Math. Soc. Simon Stevin 15 (2008), 153--177.

\bibitem{Liu} G. Liu, Symmetry theorems for the overdetermined eigenvalue problems, J. Differential Equations 233 (2007), 585--600.

\bibitem{Minlend} I.A. Minlend, An overdetermined problem for sign-changing eigenfunctions in unbounded domains, arXiv:2203.15492v1.

\bibitem{Olver} F.M. Olver, D.W. Lozier, R.F. Boisvert and C.W. Clark, NIST handbook of mathematical functions hardback
and CD-ROM, Cambridge university press, 2010.

\bibitem{Pompeiu} D. Pompeiu, Sur certains syst\`{e}mes d'\'{e}quations lin\'{e}aires et sur une propri\'{e}t\'{e} int\'{e}grale des fonctions de plusieurs
variables, C. R. Acad. Sci. Paris 188 (1929), 1138--1139.

\bibitem{Pompeiu1} D. Pompeiu, Sur une propri\'{e}t\'{e} int\'{e}grale des fonctions de deux variables r\'{e}elles, Bull. Sci. Acad. Roy. Belg. 15
(1929), 265--269.

%\bibitem{Pucci} P. Pucci and J. Serrin, The maximum principle, Progr. Nonlinear Differential Equations Appl., vol. 73, Birkh\"{a}user
%Verlag, Basel, 2007.

%\bibitem{Reichel} W. Reichel, Radial symmetry for elliptic boundary-value problems on exterior domains, Arch. Ration. Mech.
%Anal. 137 (1997), 381--394.

\bibitem{RRS} A. Ros, D. Ruiz and P. Sicbaldi, A rigidity result for overdetermined elliptic problems in the
plane, Comm. Pure Appl. Math. 70 (2017), 1223--1252.

\bibitem{Ros1} A. Ros, D. Ruiz and P. Sicbaldi, Solutions to overdetermined elliptic problems in nontrivial exterior domains, J. Eur. Math. Soc. (JEMS) 22 (2020), 253--281.

\bibitem{Ruiz} D. Ruiz, Nonsymmetric sign-changing solutions to overdetermined elliptic problems in bounded domains, arXiv:2211.14014v1.

\bibitem{RSW} D. Ruiz, P. Sicbaldi and J. Wu, Overdetermined elliptic problems in onduloid-type domains with general nonlinearities, J. Funct. Anal. 283 (2022), no. 12, Paper No. 109705, 26 pp.

\bibitem{Schiffer} M. Schiffer, Variation of domain functionals, Bull. Amer. Math. Soc. 60 (1954), 303--328.

\bibitem{Schiffer1} M. Schiffer, Partial differential equations of elliptic pype, in Lecture Series of the Symposium on
PDE, Univ. of Cal Berkeley 1955, University of Kansas Press, Lawrence, KS, 1957, pp. 97--149.

\bibitem{Schlenk} F. Schlenk and P. Sicbaldi, Bifurcating extremal domains for the first eigenvalue of the Laplacian, Adv. Math. 229 (2012), 602--632.

%\bibitem{Serrin} J. Serrin, A symmetry problem in potential theory, Arch. Ration. Mech. Anal. 43 (1971), 304--318.

\bibitem{Shepp} L.A. Shepp and J. B. Kruskal, Computerized tomography: the new medical $X$-ray technology,
Amer. Math. Monthly 85 (1978), 420--439.

\bibitem{Sicbaldi} P. Sicbaldi, New extremal domains for the first eigenvalue of the Laplacian in flat tori, Calc. Var. Partial Differential
Equations 37 (2010), 329--344.

\bibitem{Smith} K.T. Smith, D.C. Solmon and S.L. Wagner, Practical and mathematical aspects of the problem
of reconstructing objects from radiographs, Bull. Amer. Math. Soc. 83 (1977), 1227--1270.

\bibitem{Temam} R. Temam, A non-linear eigenvalue problem: The shape at equilibrium of a confined plasma,
Arch. Rational Mech. Anal. 60 (1975), 51--73.

\bibitem{Watson} G.N. Watson, A Treatise on the theory of Bessel functions, Cambridge University Press/The Macmillan Company,
Cambridge, England/New York, 1944.

\bibitem{Williams} S.A. Williams, A partial solution of the Pompeiu problem, Math. Ann. 223 (1976), 183--190.

\bibitem{Yau} S.T. Yau, Problem section, in: S.-T. Yau (Ed.), Seminar on Differential Geometry, Ann. of Math. Stud., vol. 102,
Princeton Univ. Press, Princeton, 1982.

\end{thebibliography}

\end{document}
