%%%%%%%% ICML 2023 EXAMPLE LATEX SUBMISSION FILE %%%%%%%%%%%%%%%%%

\documentclass{article}

% Recommended, but optional, packages for figures and better typesetting:
\usepackage{microtype}
\usepackage{graphicx}
\usepackage{subfigure}
\usepackage{booktabs} % for professional tables

% hyperref makes hyperlinks in the resulting PDF.
% If your build breaks (sometimes temporarily if a hyperlink spans a page)
% please comment out the following usepackage line and replace
% \usepackage{icml2023} with \usepackage[nohyperref]{icml2023} above.
\usepackage{hyperref}


% Attempt to make hyperref and algorithmic work together better:
\newcommand{\theHalgorithm}{\arabic{algorithm}}

% Use the following line for the initial blind version submitted for review:
% \usepackage{assets/icml2023}

% If accepted, instead use the following line for the camera-ready submission:
\usepackage[accepted]{assets/icml2023}

% For theorems and such
\usepackage{amsmath}
\usepackage{amssymb}
\usepackage{mathtools}
\usepackage{amsthm}

\usepackage[utf8x]{inputenc}
\usepackage{amscd}
\usepackage{amsmath}
%\usepackage{amssymb}
\usepackage{amsthm}

\usepackage[colorinlistoftodos]{todonotes}

\usepackage{thmtools}
\usepackage{thm-restate}
\usepackage{mathtools}
\usepackage[full]{complexity}
\usepackage{longtable}

%\usepackage[usenames,dvipsnames]{xcolor}
\usepackage{xcolor}
% % tables
 \usepackage{array}

\usepackage{bbm}
\usepackage{comment}
\usepackage{enumerate}
\usepackage{floatrow}


\usepackage{parallel,enumitem}

\usepackage{xspace}
\usepackage{paralist}
\usepackage{xifthen}
\usepackage{url}
\usepackage{csquotes}
% \usepackage{graphicx}
\usepackage{wrapfig}
\usepackage{multirow}
\usepackage[binary-units=true]{siunitx}

\usepackage{tikz}
\usetikzlibrary{trees,decorations,arrows,automata,shadows,positioning,plotmarks,backgrounds,shapes}
\usetikzlibrary{calc,matrix,fit,petri,decorations.markings,decorations.pathmorphing,patterns,intersections,decorations.text}
\usepackage{pgfplots}
\usepackage{pgfplotstable}

\tikzstyle{mystate}=[state,inner sep=3pt,minimum size=20pt,line width=0.2mm]
\tikzstyle{fstate}=[state,accepting,inner sep=2pt,minimum size=3pt]
\tikzstyle{istate}=[state,initial,inner sep=2pt,minimum size=3pt]
\tikzstyle{mysquare}=[inner sep=3pt,minimum size=15pt,line width=0.2mm]
\tikzstyle{fmysquare}=[inner sep=3pt,minimum size=15pt,line width=0.5mm,accepting]
\newcommand{\SFSAutomatEdge}[5]{\path[->](#1) edge[#4,line width=0.2mm] node[#5] {\ensuremath{#2}} (#3);}
\usepackage{subcaption}
\usepackage{tabularx}
\usepackage{booktabs}
\usepackage{xfrac}

\usepackage{etoc}
\etocsettocdepth{3}

% \usepackage{minitoc}

% \usepackage{titletoc}
% 
% \newcommand\DoToC{%
%   \startcontents
%   \printcontents{}{2}{\textbf{Contents}\vskip3pt\hrule\vskip5pt}
%   \vskip3pt\hrule\vskip5pt
% }

\newcommand\calF{\mathcal{F}}
\newcommand\calG{\mathcal{G}}
\newcommand\calM{\mathcal{M}}
\newcommand\calV{\mathcal{V}}
\newcommand\calU{\mathcal{U}}
\newcommand\calW{\mathcal{W}}
\newcommand\calP{\mathcal{P}}
\newcommand\calD{\mathbb{D}}
%%%%%%%%%%%%%%%%%
%% macros introduced by Luke 
\newcommand\mydef[1]{{\bf\em #1}}
%%%%%%%%%%%%%%%%%

\newcommand{\numviparams}{{| \lambda |}}
\newcommand{\scoreaccvars}[1]{s_1^{#1}, \ldots, s_{\numviparams}^{#1}}
\newcommand{\scoreaccvar}[2]{s_{#1}^{#2}}
\newcommand{\isdeterm}[1]{\text{Deterministic}({#1})}


\newcommand{\expect}[1]{\mathbb{E}\left[{#1}\right]}
\newcommand{\var}[1]{\mathbb{V}\left[ {#1} \right]}
\newcommand{\expectdist}[2]{\mathbb{E}_{#1}\left[ {#2} \right]}
\newcommand{\vardist}[2]{\mathbb{V}_{#1}\left[ {#2} \right]}
\newcommand{\cov}[2]{\mathbb{C}\text{ov}[{#1}][{#2}]}
\newcommand{\covv}[1]{\mathbb{C}\text{ov}[{#1}]}
\newcommand{\corr}[1]{\mathbb{C}\text{orr}[{#1}]}

\newcommand{\fix}[1]{\mathit{fix}\left({#1}\right)}
\newcommand{\sbr}[1]{\left\llbracket {#1} \right\rrbracket}
\newcommand{\ctxtype}[3]{{#1} \cong_\text{ctx} {#2} : {#3}}
\newcommand{\bigstep}[3]{{#1} \Downarrow_{#2} {#3}}


% PCF types
\newcommand{\bool}{\mathit{bool}}
\newcommand{\nat}{\mathit{nat}}

\newcommand{\ctx}[1]{\mathcal{C}\left[ {#1}\right] }
\newcommand{\pcft}[1]{\text{PCF}_{#1}}

\newcommand{\nfl}{\mathbb{N}_\bot}
\newcommand{\bfl}{\mathbb{B}_\bot}

% PCF constructs
\newcommand{\succc}[1]{\mathbf{succ}({#1})}
\newcommand{\succcn}[2]{\mathbf{succ}^{#1}({#2})}
\newcommand{\zero}{\mathbf{0}}
\newcommand{\zerotest}[1]{\mathbf{zero}\left({#1}\right)}
\newcommand{\pred}[1]{\mathbf{pred}\left( {#1} \right)}
\newcommand{\predn}[2]{\mathbf{pred}^{#1}\left( {#2} \right)}
\def\solvable{\#}

\newcommand{\true}{\mathbf{true}}
\newcommand{\false}{\mathbf{false}}
\newcommand{\pcffix}[1]{\mathbf{fix}\left({#1}\right)}
\newcommand{\pcffn}[3]{\mathbf{fn}~{#1}:{#2}\mathpunct{.}{#3}}
\newcommand{\pairtype}[2]{{#1} * {#2}}
\newcommand{\pairexp}[2]{\mathbf{pair}({#1}, {#2})}
\newcommand{\leftexp}[1]{\mathbf{left}({#1})}
\newcommand{\rightexp}[1]{\mathbf{right}({#1})}

\newcommand{\RationalPos}{\mathbb{Q}^{+}}

\newcommand{\meas}[1]{\mathbb{M}\left( {#1} \right) }
\newcommand{\integ}[1]{\sbr{#1}_I}

\newcommand{\notbigstep}[2]{{#1}~\cancel{\Downarrow}_{#2}}
\newcommand{\subtrace}[3]{{#1}^{{#2} \ldots {#3}}}
\newcommand{\supp}[1]{\textsf{supp}\left({#1}\right)}
\newcommand{\dom}[1]{\textsf{Dom}\left({#1}\right)}
\newcommand{\suppk}[2]{\textsf{Supp}^{#1}\left({#2}\right)}
\newcommand{\tracespace}{\bigcup_{n \in \mathbb{N}}[0, 1]^n}
\newcommand{\generictracespace}{\mathbb{T}}
\newcommand{\nnreals}{\mathbb{R}_{\geq 0}}
\newcommand{\posreals}{\mathbb{R}_{> 0}}
\newcommand{\reals}{\mathbb{R}}

\newcommand{\unrollkM}[2]{\textsf{unroll}_{#1}\left({#2}\right)}
\newcommand{\nphmcint}[5]{\Psi_\textsf{NP}\left({#1}, {#2}, {#3}, {#4}, {#5}\right)}

%SPCF constructs
\newcommand{\spcfvalues}{\Lambda^0_v}

\newcommand{\prevalueM}[1]{\textsf{value}^{-1}_{#1}(\spcfvalues{})}
\newcommand{\num}[1]{\underline{#1}}

% \theoremstyle{definition}
% \newtheorem{thm}{Theorem}
% \newtheorem{lem}{Lemma}
% \newtheorem{defn}{Definition}
% \newtheorem{conj}{Conjecture}
% \newtheorem{prop}{Proposition}

%\theoremstyle{definition}
%\newtheorem{defn}{Definition}[section]
%\newtheorem{example}[defn]{Example}
%
%
%\theoremstyle{plain}
%\newtheorem{thm}{Theorem}[section]
%\newtheorem{lem}[thm]{Lemma}
%\newtheorem{cor}[thm]{Corollary}
%\newtheorem{conj}[thm]{Conjecture}
%\newtheorem{prop}[thm]{Proposition}
%\newtheorem{remark}[thm]{Remark}

%% Proofs
%\let\oldproof\proof
%\renewcommand{\proof}{\color{blue}\oldproof}


\definecolor{codegreen}{rgb}{0,0.6,0}
\definecolor{codegray}{rgb}{0.5,0.5,0.5}
\definecolor{codepurple}{rgb}{0.58,0,0.82}
\definecolor{backcolour}{rgb}{0.95,0.95,0.92}

\lstdefinestyle{myStyle}{
    belowcaptionskip=1\baselineskip,
    breaklines=true,
    frame=none,
    basicstyle=\footnotesize\ttfamily,
    keywordstyle=\bfseries\color{green!40!black},
    commentstyle=\itshape\color{purple!40!black},
    identifierstyle=\color{blue},
    backgroundcolor=\color{gray!10!white},
    %backgroundcolor=\color{backcolour}, 
    numberstyle=\tiny\color{codegray},
    stringstyle=\color{codepurple},
    breakatwhitespace=false,                          
    keepspaces=true,                 
    numbers=left,       
    numbersep=5pt,                  
    showspaces=false,                
    showstringspaces=false,
    showtabs=false,                  
    tabsize=2,
}

% argmin/argmax
\DeclareMathOperator*{\argmax}{arg\,max}
\DeclareMathOperator*{\argmin}{arg\,min}

% Concatenation of lists
\newcommand\doubleplus{+\kern-1.3ex+\kern0.8ex}

% Program configurations
\newcommand{\tuple}[1]{\ensuremath{\langle #1 \rangle}}
% Rule based definitions
\newcommand{\Rule}[4][]{\ensuremath{\inferrule*[lab={\hypertarget{#2}{(\TirName{#2})}},#1]{#3}{#4}}}

% Calligraphic symbols
\newcommand{\calI}{{\mathcal I}} 
\newcommand{\calT}{{\mathcal T}}

%  Macro for new Y operator.
\newcommand{\yBounded}[3]{\mu^{#1}_{#2}\rvert_{#3}}

%%%%%%%%%%%%%%%%%
 
%%%%%%%%%%%%%%%%%

\newcommand{\expv}{\mathbb{E}}

\newcommand{\combTr}[2]{\left[\begin{matrix}
		#1\\
		#2
	\end{matrix} \right]}

\newcommand{\exType}[2]{\left\{\begin{matrix}
		#1\\
		#2
	\end{matrix} \right\}}
\newcommand{\myint}[1]{ [#1]}
\newcommand{\Uniform}{\ensuremath{\mathrm{Uniform}}}
\newcommand{\Normal}{\ensuremath{\mathrm{normal}}}
\DeclareMathOperator{\abs}{abs}
\DeclareMathOperator{\pdf}{pdf}

\newcommand{\intConf}[1]{\lceil#1\rceil}
\newcommand{\tr}{\boldsymbol{t}}

\newcommand{\sample}{\tt{sample}}
%\newcommand{\fix}{\texttt{fix}}
%\newcommand{\num}[1]{\underline{#1}}
\newcommand{\myif}{\texttt{if}}
\newcommand{\mylet}{\texttt{let} \, }
\newcommand{\myin}{\, \texttt{in} \,}
\newcommand{\mythen}{\, \texttt{then} \,}
\newcommand{\myelse}{\, \texttt{else} \,}
\newcommand{\score}{\tt{score}}
\newcommand{\tick}{\tt{tick}}

\newcommand{\term}{\tt{term}}
\newcommand{\pv}{\mathbf{v}}
\newcommand{\rv}{\mathbf{r}}

\newcommand{\interval}{\mathfrak{I}}

\newcommand{\typeReal}{\textbf{\textsf{R}}}

\newcommand{\symbolInt}{\myint{\cdot}}

\newcommand{\LambdaInterval}{\Lambda_{\interval}}
\newcommand{\LambdaSymbolic}{\Lambda_{\text{sym}}}

\newcommand{\toIntervalTerm}[1]{#1^{2\interval}}

%Others
\newcommand{\Sset}{\mathbb{S}}
\newcommand{\Iset}{\mathbb{I}}
\newcommand{\Rset}{\mathbb{R}}
\newcommand{\Nset}{\mathbb{N}}
\newcommand{\Zset}{\mathbb{Z}}

\newcommand{\Term}{\mathbb{T}}
\newcommand{\prob}{\mathbb{P}}
\newcommand{\expt}{\mathbb{E}}


\newcommand{\Leb}{\tt{Leb}}
\newcommand{\Red}{\tt{Red}}
\newcommand{\cost}{\text{cost}}

%\newcommand{\intervalab}[2]{\underline{[#1,#2]}}
\newcommand{\intervalab}{\underline{[a,b]}}
\newcommand{\interI}{\mathcal{I}}
\newcommand{\trans}{\mathcal{T}}

\newcommand{\iv}{\mathbb{I}}

% Programming language constructs
\newcommand{\lit}[1]{\underline{#1}}
\newcommand{\letIn}[1]{\mathsf{let}\,{#1}\,\mathsf{in}\,}
\newcommand{\fixLam}[2]{\mu {#1} {#2}.}
\newcommand{\ifElse}[3]{\mathsf{if} (#1 \le \num{0}) \, {#2} \,\mathsf{else}\, {#3}}

%%Basic notions
\newcommand{\pspace}{(\Omega,\mathcal{F},\probm)}
\newcommand{\probm}{\mathbb{P}}
\newcommand{\condexpv}[2]{{\expt}{\left[{#1} \mid {#2}\right]}}

\newcommand{\stdConf}[1]{(#1)}
%\newcommand{\intConf}[1]{\lceil#1\rceil}
%\newcommand{\intConf}[1]{(#1)}
%\newcommand{\symConf}[1]{\langle\!\langle  #1 \rangle\!\rangle}
%\newcommand\symPath[1]{(#1)}
\newcommand{\symPath}[1]{\langle\!\langle  #1 \rangle\!\rangle}
\newcommand\symConf[1]{(#1)}

\newcommand{\ifSimple}[3]{\mathsf{if}(#1, #2, #3)}
%\newcommand{\ifElse}[3]{\mathsf{if} (#1 \le 0) \, \allowbreak {#2} \, \allowbreak \mathsf{else}\, {#3}}
%\newcommand{\ifElse}[3]{\ifSimple{#1}{#2}{#3}}

%\newcommand{\trace}{\mathsf{s}}
%
%\newcommand\defn[1]{{\bf \em #1}}
\newcommand{\traces}{\mathbb{T}}
%
%\newcommand{\stdConf}[1]{(#1)}
%%\newcommand{\intConf}[1]{\lceil#1\rceil}
%\newcommand{\intConf}[1]{(#1)}
%%\newcommand{\symConf}[1]{\langle\!\langle  #1 \rangle\!\rangle}
%%\newcommand\symPath[1]{(#1)}
%\newcommand{\symPath}[1]{\langle\!\langle  #1 \rangle\!\rangle}
%\newcommand\symConf[1]{(#1)}

\newcommand{\valueSem}[1]{\mathsf{val}_{#1}} % value (semantics)
\newcommand{\weightSem}[1]{\mathsf{wt}_{#1}} % weight (semantics)
\newcommand{\measureSem}[1]{\llbracket #1 \rrbracket}
\newcommand{\posterior}{\mathsf{posterior}}


%%%%%%%%%
% 
%%%%%%%%
\newcommand{\loc}{\ell}
\newcommand{\locs}{\mathit{L}}
\newcommand{\blocs}{\mathit{L}_{\mathrm{b}}}

\newcommand{\iflocs}{\mathit{L}_{\mathrm{if}}}
\newcommand{\looplocs}{\mathit{L}_{\mathrm{while}}}

\newcommand{\alocs}{\mathit{L}_{\mathrm{a}}}
\newcommand{\wlocs}{\mathit{L}_{\mathrm{w}}}
\newcommand{\rlocs}{\mathit{L}_{\mathrm{r}}}
\newcommand{\Alocs}[1]{\mathit{L}_{\mathrm{A}}^{\mathsf{#1}}}
\newcommand{\Dlocs}{\mathit{L}_{\mathrm{nd}}}
\newcommand{\transitions}{{\rightarrow}}

%%% 
\newcommand{\plocs}{\mathit{L}_{\mathrm{p}}}
\newcommand{\tlocs}{\mathit{L}_{\mathrm{t}}}

\newcommand{\lin}{\loc_\mathrm{init}}
\newcommand{\lout}{\loc_\mathrm{out}}
\newcommand{\val}[1]{\mbox{\sl Val}_{#1}}

\newcommand{\pvars}{V_\mathrm{p}}
\newcommand{\rvars}{V_{\mathrm{r}}}
\newcommand{\pre}{\mathrm{pre}}

\newcommand{\sle}{\sqsubseteq}
\newcommand{\sge}{\sqsupseteq}

\newcommand{\lfp}{\mathrm{lfp}}
\newcommand{\gfp}{\mathrm{gfp}}

\newcommand{\rdvarjdis}{\mathcal D}
\newcommand{\sampset}{\textit{supp}}

\newcommand{\upd}{\mbox{\sl upd}}
\newcommand{\wet}{\mbox{\sl wt}}
\newcommand{\transset}{\mathfrak T}
\newcommand{\valin}{\pv_{\mathrm{init}}}
\newcommand{\ret}{\mbox{\sl ret}}

\newcommand{\win}{w_{\mathrm{init}}}

\newcommand{\sampdpd}{\overline{\Upsilon}}

\newcommand{\outmap}{\text{O}}
\newcommand{\sat}[1]{\langle #1 \rangle}
\newcommand{\monoid}{\mbox{\sl Monoid}}
\newcommand{\handelmanformat}{(\dagger)}

\newcommand{\trunc}{\mathcal{B}}

\newcommand{\ewt}{\mbox{\sl ewt}}
\newcommand{\statemap}{\text{St}}

\newcommand{\valrd}{{\mathbf{r}}}
\newcommand{\frmloc}{\ell^{\mathrm{src}}}
\newcommand{\toloc}{\ell^{\mathrm{dst}}}

\newcommand{\monomials}{\mathbf{M}}
%%%%% NEW MATH DEFINITIONS %%%%%
\newtheorem{property}{Property}
\newtheorem{definition}{Definition}
\newtheorem{theorem}{Theorem}
\newtheorem{lemma}{Lemma}
\newtheorem{corollary}{Corollary}
\DeclarePairedDelimiter\abs{\lvert}{\rvert}
\DeclarePairedDelimiter\norm{\lVert}{\rVert}
\makeatletter
\let\oldabs\abs
\def\abs{\@ifstar{\oldabs}{\oldabs*}}
\let\oldnorm\norm
\def\norm{\@ifstar{\oldnorm}{\oldnorm*}}
\makeatother

% Mark sections of captions for referring to divisions of figures
\newcommand{\figleft}{{\em (Left) }}
\newcommand{\figcenter}{{\em (Center) }}
\newcommand{\figright}{{\em (Right)}}
\newcommand{\figtop}{{\em (Top) }}
\newcommand{\figbottom}{{\em (Bottom) }}
\newcommand{\captiona}{{\em (a) }}
\newcommand{\captionb}{{\em (b) }}
\newcommand{\captionc}{{\em (c) }}
\newcommand{\captiond}{{\em (d) }}

% Highlight a newly defined term
\newcommand{\newterm}[1]{{\bf #1}}


\def\figref#1{figure~\ref{#1}}
\def\Figref#1{Figure~\ref{#1}}
\def\twofigref#1#2{figures \ref{#1} and \ref{#2}}
\def\quadfigref#1#2#3#4{figures \ref{#1}, \ref{#2}, \ref{#3} and \ref{#4}}
\def\secref#1{section~\ref{#1}}
\def\Secref#1{Section~\ref{#1}}
\def\twosecrefs#1#2{sections \ref{#1} and \ref{#2}}
\def\secrefs#1#2#3{sections \ref{#1}, \ref{#2} and \ref{#3}}
\def\eqref#1{equation~\ref{#1}}
\def\Eqref#1{Equation~\ref{#1}}
% A raw reference to an equation---avoid using if possible
\def\plaineqref#1{\ref{#1}}
% Reference to a chapter, lower-case.
\def\chapref#1{chapter~\ref{#1}}
% Reference to an equation, upper case.
\def\Chapref#1{Chapter~\ref{#1}}
% Reference to a range of chapters
\def\rangechapref#1#2{chapters\ref{#1}--\ref{#2}}
% Reference to an algorithm, lower-case.
\def\algref#1{algorithm~\ref{#1}}
% Reference to an algorithm, upper case.
\def\Algref#1{Algorithm~\ref{#1}}
\def\twoalgref#1#2{algorithms \ref{#1} and \ref{#2}}
\def\Twoalgref#1#2{Algorithms \ref{#1} and \ref{#2}}
% Reference to a part, lower case
\def\partref#1{part~\ref{#1}}
% Reference to a part, upper case
\def\Partref#1{Part~\ref{#1}}
\def\twopartref#1#2{parts \ref{#1} and \ref{#2}}

% Random variables
\def\reta{{\textnormal{$\eta$}}}
\def\ra{{\textnormal{a}}}

% Random vectors
\def\rvepsilon{{\mathbf{\epsilon}}}
\def\rvtheta{{\mathbf{\theta}}}
\def\rva{{\mathbf{a}}}

% Elements of random vectors
\def\erva{{\textnormal{a}}}
\def\ervb{{\textnormal{b}}}

% Random matrices
\def\rmA{{\mathbf{A}}}
\def\rmB{{\mathbf{B}}}

% Elements of random matrices
\def\ermA{{\textnormal{A}}}
\def\ermB{{\textnormal{B}}}

\def\fvec{{\mathbf{f}}}
\def\bff{{\mathbf{f}}}
\def\bfg{{\mathbf{g}}}
% Vectors
\def\vzero{{\bm{0}}}
\def\vone{{\bm{1}}}
\def\vmu{{\bm{\mu}}}
\def\vtheta{{\bm{\theta}}}
\def\va{{\bm{a}}}
\def\vb{{\bm{b}}}
\def\vc{{\bm{c}}}
\def\vd{{\bm{d}}}
\def\ve{{\bm{e}}}
\def\vf{{\bm{f}}}
\def\vg{{\bm{g}}}
\def\vh{{\bm{h}}}
\def\vi{{\bm{i}}}
\def\vj{{\bm{j}}}
\def\vk{{\bm{k}}}
\def\vl{{\bm{l}}}
\def\vm{{\bm{m}}}
\def\vn{{\bm{n}}}
\def\vo{{\bm{o}}}
\def\vp{{\bm{p}}}
\def\vq{{\bm{q}}}
\def\vr{{\bm{r}}}
\def\vs{{\bm{s}}}
\def\vt{{\bm{t}}}
\def\vu{{\bm{u}}}
\def\vv{{\bm{v}}}
\def\vw{{\bm{w}}}
\def\vx{{\bm{x}}}
\def\vy{{\bm{y}}}
\def\vz{{\bm{z}}}

% Matrix
\def\mA{{\bm{A}}}

% Tensor
\DeclareMathAlphabet{\mathsfit}{\encodingdefault}{\sfdefault}{m}{sl}
\SetMathAlphabet{\mathsfit}{bold}{\encodingdefault}{\sfdefault}{bx}{n}
\newcommand{\tens}[1]{\bm{\mathsfit{#1}}}
\def\tA{{\tens{A}}}
\def\tB{{\tens{B}}}
\def\tC{{\tens{C}}}
\def\tD{{\tens{D}}}
\def\tE{{\tens{E}}}
\def\tF{{\tens{F}}}
\def\tG{{\tens{G}}}
\def\tH{{\tens{H}}}
\def\tI{{\tens{I}}}
\def\tJ{{\tens{J}}}
\def\tK{{\tens{K}}}
\def\tL{{\tens{L}}}
\def\tM{{\tens{M}}}
\def\tN{{\tens{N}}}
\def\tO{{\tens{O}}}
\def\tP{{\tens{P}}}
\def\tQ{{\tens{Q}}}
\def\tR{{\tens{R}}}
\def\tS{{\tens{S}}}
\def\tT{{\tens{T}}}
\def\tU{{\tens{U}}}
\def\tV{{\tens{V}}}
\def\tW{{\tens{W}}}
\def\tX{{\tens{X}}}
\def\tY{{\tens{Y}}}
\def\tZ{{\tens{Z}}}


% Graph
\def\gA{{\mathcal{A}}}
\def\gB{{\mathcal{B}}}
\def\gC{{\mathcal{C}}}
\def\dataset{{\mathcal{D}}}
\def\gE{{\mathcal{E}}}
\def\gF{{\mathcal{F}}}
\def\fourier{{\mathcal{F}}}
\def\gG{{\mathcal{G}}}
\def\gH{{\mathcal{H}}}
\def\gI{{\mathcal{I}}}
\def\gJ{{\mathcal{J}}}
\def\gK{{\mathcal{K}}}
\def\gL{{\mathcal{L}}}
\def\loss{{\mathcal{L}}}
\def\gM{{\mathcal{M}}}
\def\gN{{\mathcal{N}}}
\def\normal{{\mathcal{N}}}
\def\gaussian{{\mathcal{N}}}
\def\gO{{\mathcal{O}}}
\def\gP{{\mathcal{P}}}
\def\gQ{{\mathcal{Q}}}
\def\gR{{\mathcal{R}}}
\def\gS{{\mathcal{S}}}
\def\gT{{\mathcal{T}}}
\def\gU{{\mathcal{U}}}
\def\uniform{{\mathcal{U}}}
\def\gV{{\mathcal{V}}}
\def\gW{{\mathcal{W}}}
\def\gX{{\mathcal{X}}}
\def\gY{{\mathcal{Y}}}
\def\gZ{{\mathcal{Z}}}

\def\algebra{{\mathscr{A}}}
\def\borel{{\mathscr{B}}}
\def\manifold{{\mathscr{M}}}

% Sets
\def\sA{{\mathbb{A}}}
\def\sB{{\mathbb{B}}}
\def\complex{{\mathbb{C}}}
\def\sD{{\mathbb{D}}}
\def\expectation{{\mathbb{E}}}
\newcommand{\E}{\mathbb{E}}
\def\sF{{\mathbb{F}}}
\def\sG{{\mathbb{G}}}
\def\sH{{\mathbb{H}}}
\def\sI{{\mathbb{I}}}
\def\sJ{{\mathbb{J}}}
\def\sK{{\mathbb{K}}}
\def\sL{{\mathbb{L}}}
\def\sM{{\mathbb{M}}}
\def\natural{{\mathbb{N}}}
\def\sO{{\mathbb{O}}}
\def\sP{{\mathbb{P}}}
\def\rational{{\mathbb{Q}}}
\def\real{{\mathbb{R}}}
\newcommand{\R}{\mathbb{R}}
\def\sS{{\mathbb{S}}}
\def\sphere{{\mathbb{S}}}
\def\sT{{\mathbb{T}}}
\def\sU{{\mathbb{U}}}
\def\sV{{\mathbb{V}}}
\def\sW{{\mathbb{W}}}
\def\sX{{\mathbb{X}}}
\def\sY{{\mathbb{Y}}}
\def\integer{{\mathbb{Z}}}
\def\indicator{{\mathbbm{1}}}

% Entries of a matrix
\def\emLambda{{\Lambda}}
\def\emA{{A}}
\def\emB{{B}}
\def\emC{{C}}
\def\emD{{D}}
\def\emE{{E}}
\def\emF{{F}}
\def\emG{{G}}
\def\emH{{H}}
\def\emI{{I}}
\def\emJ{{J}}
\def\emK{{K}}
\def\emL{{L}}
\def\emM{{M}}
\def\emN{{N}}
\def\emO{{O}}
\def\emP{{P}}
\def\emQ{{Q}}
\def\emR{{R}}
\def\emS{{S}}
\def\emT{{T}}
\def\emU{{U}}
\def\emV{{V}}
\def\emW{{W}}
\def\emX{{X}}
\def\emY{{Y}}
\def\emZ{{Z}}
\def\emSigma{{\Sigma}}

% entries of a tensor
% Same font as tensor, without \bm wrapper
\newcommand{\etens}[1]{\mathsfit{#1}}
\def\etLambda{{\etens{\Lambda}}}
\def\etA{{\etens{A}}}
\def\etB{{\etens{B}}}
\def\etC{{\etens{C}}}
\def\etD{{\etens{D}}}
\def\etE{{\etens{E}}}
\def\etF{{\etens{F}}}
\def\etG{{\etens{G}}}
\def\etH{{\etens{H}}}
\def\etI{{\etens{I}}}
\def\etJ{{\etens{J}}}
\def\etK{{\etens{K}}}
\def\etL{{\etens{L}}}
\def\etM{{\etens{M}}}
\def\etN{{\etens{N}}}
\def\etO{{\etens{O}}}
\def\etP{{\etens{P}}}
\def\etQ{{\etens{Q}}}
\def\etR{{\etens{R}}}
\def\etS{{\etens{S}}}
\def\etT{{\etens{T}}}
\def\etU{{\etens{U}}}
\def\etV{{\etens{V}}}
\def\etW{{\etens{W}}}
\def\etX{{\etens{X}}}
\def\etY{{\etens{Y}}}
\def\etZ{{\etens{Z}}}

\def\ceil#1{\lceil #1 \rceil}
\def\floor#1{\lfloor #1 \rfloor}
\def\eps{{\epsilon}}

\newcommand{\pder}[1]{\frac{\partial}{\partial #1}}

\newcommand{\half}{\frac{1}{2}}
\newcommand{\limNinf}{\lim_{N \to \infty}}
\newcommand{\limTzero}{\lim_{\tau \to 0}}


\newcommand{\cmark}{\ding{51}}
\newcommand{\xmark}{\ding{55}}

\newcommand{\layer}{\mathcal{H}}
\newcommand{\defeq}{\triangleq}
%\newcommand{\defeq}{vcentcolon=}
\newcommand{\domain}{\Omega}
\newcommand{\grad}{\nabla}

\newcommand{\cin}{c_{\rm{in}}}
\newcommand{\cout}{c_{\rm{out}}}
\newcommand{\intdomain}{\int_{\domain}}
\newcommand{\network}{\gT}
\newcommand{\subnet}{\gK}
\newcommand{\map}{\gR} %\gR

\newcommand{\innerproduct}[2]{\langle #1, #2 \rangle}
\newcommand{\mcsum}[1][j]{\frac{1}{N}\sum_{#1=1}^N}

\newcommand{\inrspace}[1][c]{\gF_{#1}}

\DeclareMathOperator*{\argmax}{arg\,max}
\DeclareMathOperator*{\argmin}{arg\,min}

\let\ab\allowbreak


% if you use cleveref..
\usepackage[capitalize,noabbrev]{cleveref}

% Todonotes is useful during development; simply uncomment the next line
%    and comment out the line below the next line to turn off comments
%\usepackage[disable,textsize=tiny]{todonotes}
\usepackage[textsize=tiny]{todonotes}


% The \icmltitle you define below is probably too long as a header.
% Therefore, a short form for the running title is supplied here:
\icmltitlerunning{Interpolating Images with Diffusion Models}

\begin{document}

\twocolumn[
\icmltitle{Interpolating between Images with Diffusion Models}

% It is OKAY to include author information, even for blind
% submissions: the style file will automatically remove it for you
% unless you've provided the [accepted] option to the icml2023
% package.

% List of affiliations: The first argument should be a (short)
% identifier you will use later to specify author affiliations
% Academic affiliations should list Department, University, City, Region, Country
% Industry affiliations should list Company, City, Region, Country

% You can specify symbols, otherwise they are numbered in order.
% Ideally, you should not use this facility. Affiliations will be numbered
% in order of appearance and this is the preferred way.
\icmlsetsymbol{equal}{*}

% \begin{icmlauthorlist}
% \icmlauthor{Clinton Wang}{mit}
% \icmlauthor{Polina Golland}{mit}
% \end{icmlauthorlist}

% \icmlaffiliation{mit}{MIT CSAIL, Cambridge, USA}

\begin{icmlauthorlist}
\icmlauthor{Clinton J. Wang and Polina Golland}{} \\
MIT CSAIL
\end{icmlauthorlist}

% \icmlcorrespondingauthor{Clinton Wang}{clintonw@csail.mit.edu}

% You may provide any keywords that you
% find helpful for describing your paper; these are used to populate
% the "keywords" metadata in the PDF but will not be shown in the document
\icmlkeywords{Latent diffusion models, image interpolation, image editing, denoising diffusion model, video generation}

{%
% \renewcommand\twocolumn[1][]{#1}%
\begin{center}
\centering
\captionsetup{type=figure}
% Figure removed
\vspace{-12pt}
\captionof{figure}{\textbf{Interpolations of real images.}
By conditioning a pre-trained latent diffusion model on various attributes, we can interpolate pairs of images with diverse styles, layouts, and subjects.}
\label{fig:teaser}
\end{center}}

\vskip 0.2in
]

% this must go after the closing bracket ] following \twocolumn[ ...

% This command actually creates the footnote in the first column
% listing the affiliations and the copyright notice.
% The command takes one argument, which is text to display at the start of the footnote.
% The \icmlEqualContribution command is standard text for equal contribution.
% Remove it (just {}) if you do not need this facility.

\printAffiliationsAndNotice{}  % leave blank if no need to mention equal contribution
% \printAffiliationsAndNotice{\icmlEqualContribution} % otherwise use the standard text.

% % % Figure environment removed

% \begin{strip}\centering
% % Figure removed
% \captionof{figure}{\textbf{Interpolations of real images.}
% By conditioning a pre-trained latent diffusion model on various attributes, we can interpolate pairs of images with diverse styles, layouts, and subjects.}
% \label{fig:teaser}
% \end{strip}


% \begin{teaserfigure}
% % Figure removed
% \vspace{-8pt}
% \caption{\textbf{Interpolations of real images.}
% By conditioning a pre-trained latent diffusion model on various attributes, we can interpolate pairs of images with diverse styles, layouts, and subjects.}
% \label{fig:teaser}
% \end{teaserfigure}


\twocolumn[{%
\renewcommand\twocolumn[1][]{#1}%
\begin{center}
\centering
\captionsetup{type=figure}
% Figure removed
\vspace{-8pt}
\captionof{figure}{\textbf{Interpolations of real images.}
By conditioning a pre-trained latent diffusion model on various attributes, we can interpolate pairs of images with diverse styles, layouts, and subjects.}
\end{center}}]
\begin{abstract}
Graph Neural Networks (GNNs) have proven to be effective in processing and learning from graph-structured data.
However, previous works mainly focused on understanding single graph inputs while many real-world applications require pair-wise analysis for graph-structured data (e.g., scene graph matching, code searching, and drug-drug interaction prediction).
To this end, recent works have shifted their focus to learning the interaction between pairs of graphs.
Despite their improved performance, these works were still limited in that the interactions were considered at the node-level, resulting in high computational costs and suboptimal performance.
To address this issue, we propose a novel and efficient graph-level approach for extracting interaction representations using co-attention in graph pooling. 
Our method, Co-Attention Graph Pooling (CAGPool), exhibits competitive performance relative to existing methods in both classification and regression tasks using real-world datasets, while maintaining lower computational complexity.

\end{abstract}
\section{Introduction}

% Figure environment removed

Reinforcement Learning from Human Feedback (RLHF) has recently been used to great effect to align pretrained large language models (LLMs) to human preferences, optimizing for desirable qualities like harmlessness and helpfulness~\citep{bai2022training} and achieving state-of-the-art results across a variety of natural language tasks~\citep{openai2023gpt4}. %RLHF approaches fundamentally rely on collecting pairs of LLM outputs $(o_1, o_2)$ from a shared prompt $p$, with a human indicating which output in each pair is better on a specified attribute.
% A fundamental component of RLHF is a preference model derived from human labels, typically formatted as pairs of LLM outputs $(o_1, o_2)$ generated from a shared prompt $p$.

A standard RLHF procedure fine-tunes an initial unaligned LLM using an RL algorithm such as PPO~\citep{schulman2017proximal}, optimizing the LLM to align with human preferences. %\violet{not sure whether we need to provide this detail in the intro, especially this has nothing to do with our contribution.} % i feel like this context is useful later when e.g. explaining that context distillation is SFT
RLHF is thus critically dependent on a reward model derived from human-labeled preferences, typically \textit{pairwise preferences} on LLM outputs $(o_1, o_2)$ generated from a shared prompt $p$. % and labeled by humans. 

However, collecting human pairwise preference data, especially high-quality data, may be expensive and time consuming at scale. To address this problem, approaches have been proposed to obtain labels without human annotation, such as Reinforcement Learning from AI Feedback (RLAIF) and context distillation. 

\iffalse
raising the question of whether we can generate high-quality data for RLHF without using human labeling. %accurately-labeled preference pairs $(o_1, o_2)$
%, motivating model alignment approaches that aim to generate accurately-labeled preference pairs $(o_1, o_2)$ without human involvement. 
Two major categories of such approaches are . 
\fi

RLAIF approaches (e.g.,~\citet{bai2022constitutional}) simulate human pairwise preferences by scoring $o_1$ and $o_2$ with an LLM (Figure \ref{fig:rlcd_differences} center); the scoring LLM is often the same as the one used to generate the original pairs $(o_1, o_2)$. Of course, the resulting LLM pairwise preferences will be somewhat noisier compared to human labels. However, this problem is exacerbated by using the same prompt $p$ to generate both $o_1$ and $o_2$, causing $o_1$ and $o_2$ to often be of very similar quality and thus hard to differentiate (e.g., Table~\ref{tab:rlaif_bad_example}). Consequently, training signal can be overwhelmed by label noise, yielding lower-quality preference data. 

% While it avoids human labeling efforts, it has weakness. First, LLM preference labels will naturally be somewhat noisier compared to human labels. Furthermore, since the same prompt $p$ is used to generate both $o_1$ and $o_2$, their quality is often very similar and hard to differentiate (See Table~\ref{tab:rlaif_bad_example}). As a result, training signals can be overwhelmed by label noise, yielding lower-quality preference data. 

Meanwhile, context distillation methods (e.g., \citet{sun2023principle}) create more training signal by modifying the initial prompt $p$. 
%to create more significant training signal. 
The modified prompt $p_+$ typically contains additional context encouraging a \textit{directional attribute change} in the output $o_+$ (Figure \ref{fig:rlcd_differences} right). However, context distillation methods only generate a single output $o_+$ per prompt $p_+$, which is then used for supervised fine-tuning, losing the pairwise preferences which help RLHF-style approaches to 
%rather than using a RLHF-style preference model to 
derive signal from the contrast between outputs. 
Multiple works have observed that RL approaches using preference models for pairwise preferences can substantially improve over supervised fine-tuning by itself when aligning LLMs~\citep{ouyang2022training,dubois2023alpacafarm}. 

% conduct alignment by running supervised fine-tuning on model outputs $o_+$ generated from a modified prompt $p_+$. $p_+$ typically contains additional context encouraging desirable attributes (Figure \ref{fig:rlcd_differences} right), such as in \citet{sun2023principle}. However, multiple works have observed that RLHF-style approaches can substantially improve over supervised fine-tuning by itself when aligning LLMs~\citep{ouyang2022training,dubois2023alpacafarm}. 

Therefore, while both RLAIF and context distillation approaches have already been successfully applied in practice to align language models, we posit that it may be even more effective to combine the key advantages of both. That is, we will use RL with \textit{pairwise preferences}, while also using modified prompts to encourage \textit{directional attribute change} in outputs. %In particular, we will adapt the RLAIF data generation process with two different prompts rather than a single $p$, modifying both prompts similarly to context distillation. %\violet{this motivation is a little unexciting. I think we can more specifically discuss the potential benefits of our approach, like the benefits from RL: exploration/data generation; benefits from contrast. I don't think we get too much benefits from context distillation since we switched to the RL framework.} 

Concretely, we propose \oursfull{} (\ours{}). 
\ours{} generates preference data as follows. Rather than producing two i.i.d.\ model outputs $(o_1, o_2)$ from the same prompt $p$ as in RLAIF, \ours{} creates two variations of $p$: a \textit{positive prompt} $p_+$ similar to context distillation which encourages directional change toward a desired attribute, and a \textit{negative prompt} $p_-$ which encourages directional change \textit{against} it (Figure \ref{fig:rlcd_differences} left). We then generate model outputs $(o_+, o_-)$ respectively, and automatically label $o_+$ as preferred---that is, \ours{} automatically ``generates'' pairwise preference labels by construction. %, without further post hoc labeling.\violet{should make it clearer that our approach `generates' labels by construction} 
We then follow the standard RL pipeline of training a preference model followed by PPO. 

Compared to RLAIF-generated preference pairs $(o_1, o_2)$ from the same input prompt $p$, there is typically a clearer difference in the quality of $o_+$ and $o_-$ generated using \ours{}'s directional prompts $p_+$ and $p_-$, which may result in less label noise. %which may result in better training signal for the preference model. 
That is, intuitively, \ours{} exchanges having examples be \textit{closer to the classification boundary} for much more \textit{accurate labels} on average. Compared to standard context distillation methods, on top of leveraging pairwise preferences for RL training, \ours{} can derive signal not only from the positive prompt $p_+$ which improves output quality, but also from the negative prompt $p_-$ which degrades it. %\ours{} is not learning to imitate $o_+$, but to distill the \textit{contrast} between $o_+$ and $o_-$. 
Positive outputs $o_+$ don't need to be perfect; they only need to contrast with $o_-$ on the desired attribute while otherwise following a similar style.

% \todo{discuss our method and why intuitively it may be better.}

We evaluate the practical effectiveness of \ours{} through both human and automatic evaluations on three tasks, aiming to improve the ability of LLaMA-7B~\citep{touvron2023llama} to generate harmless outputs, helpful outputs, and high-quality story outlines. %\ours{} outperforms both RLAIF and context distillation baselines in pairwise comparisons on 
As shown in Sec. \ref{sec:experiments}, \ours{} substantially outperforms both RLAIF and context distillation baselines in pairwise comparisons when simulating preference data with LLaMA-7B, while still performing equal or better when simulating with LLaMA-30B. 
%On all three tasks, \ours{} substantially outperforms both RLAIF and context distillation baselines in pairwise comparisons---by a margin of at least 9\% and often more than 30\%---validating our method's efficacy. 
We will release all code at a later date, although in any case \ours{} is fairly easy to implement by modifying any reference RLAIF codebase. %We release all code at \todo{github link}.
\section{Related Work}
\label{sec:related}

\begin{table}[t]
\small
\centering
\caption{Comparison of our method with related settings}
\begin{tabular}{cccc}
\toprule
Setting & Detect Novel OOD Data & Semi-Supervised & Learns from Novel OOD Data \\
\midrule
SSOD & \xmark & \cmark & \xmark \\ 
Open-World OD & \cmark & \xmark & \cmark \\
Open-Set SSOD & \cmark & \cmark & \xmark \\ \midrule
\textbf{Our Method} & \cmark & \cmark & \cmark \\
\bottomrule
\end{tabular}
\label{tab:comparison}
\end{table}

\paragraph{Semi-Supervised Object Detection.} Semi-supervised object detection (SSOD) approaches have become popular to reduce the need for labeling \cite{sohn2020detection, berthelot2019mixmatch, jeong2019consistency}. Pseudo-labeling based methods such as FlexMatch \cite{zhang2021flexmatch}, TSSDL \cite{shi2018transductive}, and others \cite{iscen2019label, luo2018smooth, yan2019semi, liu2021unbiased, xu2021end}, first train a teacher model using only labeled data and then use that model to create pseudo-labels for unlabeled images. The pseudo-labels are then used along with the original labeled data to train a student model. On the other hand, consistency regularization approaches such as \cite{sajjadi2016regularization, laine2017temporal, tarvainen2017mean, liu2021certainty, luo2018smooth, jeong2019consistency, iscen2019label, liu2021unbiased, xu2021end}, aim to minimize a consistency loss between differently augmented versions of an image. All of these semi-supervised learning approaches assume a ``closed-world'' setting with a fixed set of classes in both training and testing, which is not a valid assumption in real-world applications.

\paragraph{Open-World Object Detection.} Open-world object detection enables the detection of novel objects by incrementally adding novel object classes to the set of known classes. Previous work \cite{kim2022learning, kuo2015deepbox, o2015learning, wang2020leads, Maaz2022Multimodal} has studied different methods of object proposals for novel objects by attempting to remove the notion of class (all objects are regarded the same). ORE \cite{joseph2021towards} is the first to propose an open-world object detector that identifies novel classes as ‘unknown’ and proceeds to learn the unknown classes once the labels become available. \cite{han2022expanding} aims to identify unknown objects by separating high/low-density regions in the latent space. Both these approaches work in a fully-supervised setting. Our setup goes a step further and situates the open-world problem in the context of semi-supervised learning, with limited amounts of labeled ID data \textit{only}, that more closely resembles the real-world settings. 

\paragraph{Unsupervised Object Localization.} Recently proposed methods such as CutLER \cite{wang2023cut}, FreeSolo \cite{wang2022freesolo}, LOST \cite{LOST}, and MOST \cite{rambhatla2023most} propose to localize objects in an unsupervised manner, either by segmentation masks or bounding boxes. Some of these \cite{wang2023cut, LOST, rambhatla2023most} use features from self-supervised trained transformers to localize objects in the scene. In our work, we evaluate the capabilities of such methods for localizing OOD objects, as they present open-world capabilities. Based on our evaluation (\ref{sec:expts:ablation}), we use CutLER as part of the OOD Explorer to localize OOD classes. Section \ref{sec:expts} provides the details of our evaluation. 

\paragraph{Open-Set/Open-world Semi-Supervised Object Detection.}
The open-set semi-supervised object detection problem \cite{liuopen} addressed some of the limitation of the above mentioned work. Furthermore, they address like the performance of ID classes in the presence of OOD data, but they do not learn from it or improve OOD performance. They propose an offline OOD detector to filter out OOD data, thus limiting the risk of ID performance in the presence of OOD data. In contrast, our approach \textit{both} improves performance for ID classes \textit{as well as} OOD classes, i.e., our proposed framework solves a strictly stronger problem. Specifically speaking, \cite{liuopen} solves for identifying novel classes and filters it out, but does not re-introduce the classes back into the training pipeline in order to be able to learn its features. \cite{mullappilly2024semi} addresses some of the limitations of the previous mentioned methods by extending the problem to a semi-supervised setting. However, their problem setting is similar to an incremental learning setting, access to unknown class labels is provided in subsequent tasks. Our generalized setting, on the other hand, does not require access to any unknown class labels. 

\vspace{-3pt}
\section{Preliminaries}


Let $x$ be a real image. A latent diffusion model (LDM) consists of an encoder $\gE: x \mapsto z_0$, decoder $\mathcal{D}: z_0 \mapsto \hat{x}$, and a denoising U-Net $\eps_\theta: (z_t; t, c_{\rm{text}}, c_{\rm{pose}}) \mapsto \hat{\eps}$. The timestep $t$ indexes a diffusion process, in which latent vectors $z_0$ derived from real images are mapped to a Gaussian distribution $z_T \sim \normal(0, I)$ by composing small amounts of i.i.d. noise at each step. Each noisy latent vector $z_t$ can be related to the original input as $z_t = \alpha_t z_0 + \sigma_t \eps$, $\eps \sim \mathcal{N}(0,I)$, for parameters $\alpha_t$ and $\sigma_t$. The role of the denoising U-Net is to estimate $\eps$ \cite{ddpm}. An LDM performs gradual denoising over several iterations, producing high quality outputs that faithfully incorporate conditioning information. $c_{\rm{text}}$ is text that describes the desired image (optionally including a negative prompt), and $c_{\rm{pose}}$ represents an optional conditioning pose for human or anthropomorphic subjects. The mechanics of text conditioning is described in \cite{ldm}, and pose conditioning is described in \cite{controlnet}.

% Pre-trained latent diffusion models like Stable Diffusion can produce high-quality reconstructions, $\norm{E(D(x)) - x}$ is small for a very wide range of images $x$.

\vspace{-3pt}
\section{Real Image Interpolation}
% Let $x^0, x^N$ be two real images that we want to interpolate with $N-1$ intermediate images.


\input{figs/2_pipeline}

\subsection{Latent interpolation}\label{sec:latent_interp}
Our general strategy for generating sequences of interpolations is to iteratively interpolate pairs of images, starting with the two given input images. 
For each pair of parent images, we add shared noise to their latent vectors, interpolate them, then denoise the result to generate an intermediate image. The amount of noise to add to the parent latent vectors should be small if the parents are close to each other in the sequence, to encourage smooth interpolations. If the parents are far apart, the amount of noise should be larger to allow the LDM to explore nearby trajectories in latent space that have higher probability and better match other conditioning information.

Concretely, we specify a sequence of increasing timesteps $\mathcal{T}=(t_1,\dots,t_K)$, and assign parent images using the following branching structure: images $0$ and $N$ (the input images) are diffused to timestep $t_K$ and averaged to generate image $\frac{N}{2}$, images $0$ and $\frac{N}{2}$ are diffused to timestep $t_{K-1}$ generate image $\frac{N}{4}$, images $\frac{N}{2}$ and $N$ are also diffused to timestep $t_{K-1}$ to generate image $\frac{3N}{4}$, and so on. By adding noise separately to each pair of parent images, this scheme encourages images to be close to their parents, but disentangles sibling images.  %images $kN/2^j$ and $(k+2)N/2^j$ generate image $(2k+1)N/2^{j+1}$ for all $j=$

\paragraph{Interpolation type}
We use spherical linear interpolations (\textit{slerp}) for latent space and text embedding interpolations, and linear interpolations for pose interpolations. Empirically, the difference between \textit{slerp} and linear interpolation appears to be fairly mild.

\paragraph{Noise schedule}
We perform DDIM sampling \cite{ddim}, and find that the LDM's quality is more consistent when the diffusion process is partitioned into at least 200 timesteps, and noticeably degrades at coarser schedules. Empirically, latent vectors denoised with less than 25\% of the schedule often resemble an alpha composite of their parent images, while images generated with more than 65\% of the schedule can deviate significantly from their parent images. For each interpolation we choose a linear noise schedule within this range, depending on the amount of variation desired in the output. Our approach is compatible with various stochastic samplers \cite{karras2022elucidating} which seem to yield comparable results.
% Written out explicitly, we create sequences of corresponding noisy latents $\{z_t^0\}_{t \in \mathcal{T}}, \{z_t^N\}_{t \in \mathcal{T}}$, such that:
% \begin{gather}
% z_t^i = \alpha_t z_{t-1}^i + \beta_t \eps_t,
% \end{gather}
% where $\eps_t \sim \normal(0,I)$ is shared for both images.% and $z_0^0, z_0^N$ are obtained as before.
% Each intermediate image is assigned a particular timestep $t := \texttt{image_schedule}(i)$ to generate its interpolated latent code:
% $z_t^i := \texttt{slerp}(z_t^0, z_t^N, i/N)$
% We then perform denoising with the LDM: $z_0^i := \mu_\theta(z_t^i, t)$ and use the decoder to produce the image.

% $z_0^0 := \gE(x^0)$, $z_0^N := \gE(x^N)$, and all images are generated $z_0^i = \texttt{slerp}(z_0^0, z_0^N, i/N)$, $x^i := \mathcal{D}(z_0^i)$
% We examine three different strategies for latent interpolation, which differ in how they combine diffusion with interpolation to create interpolated images.

% \paragraph{Denoise-renoise-interpolate}
% Rather than partially denoise each latent, we can fully denoise the latent, then add new noise back to the appropriate level before interpolating it. This strategy permits a much wider range of latent space to be traversed, by decoupling images $N/4$ from $3N/4$, etc., while still forcing adjacent images to be similar.
% \footnote{The interpolation of two latent vectors at a particular noise level may not remain at the same noise level due to correlations introduced during the denoising process. However, we observe empirically that the independent noise assumption.}

\subsection{Textual inversion}\label{sec:text_inversion}
Pre-trained latent diffusion models are heavily dependent on text conditioning to yield high quality outputs of a particular style. Given an initial text prompt describing the overall content and/or style of each image, we can adapt its embedding more specifically to the image by applying textual inversion. In particular, we encode the text prompt as usual, then fine-tune the prompt embedding to minimize the error of the LDM on denoising the latent vector at random noise levels when conditioned on this embedding. Specifically, we perform 100-500 iterations of gradient descent with the loss $\loss(c_{\rm{text}}) = \norm{\hat{\eps}_\theta(\alpha_t z_0 + \sigma_t \eps; t, c_{\rm{text}}) - \eps}$ and a learning rate of $10^{-4}$. The number of iterations can be increased for images with complicated layouts or styles which are harder to represent with a text prompt.

In this paper we specify the same initial prompt for both input images, although one can also substitute a captioning model for a fully automated approach. Both positive and negative text prompts are used and optimized, and we share the negative prompt for each pair of images. Since our task does not require a custom token, we choose to optimize the entire text embedding.
% We also want to interpolate the prompt between the input images so that the style and content can transition smoothly. We can either specify prompts for each of the images, or perform single-image textual inversion on the images. In our experience, the best approach is to choose an initial shared positive and negative prompt for the images, then .

\subsection{Pose guidance}\label{sec:pose_guidance}
\input{figs/4_pose_conditioning}
If the subject's pose differs significantly between the two images, image interpolation is challenging and often results in anatomical errors such as multiple limbs and faces. We obtain more plausible transitions between subjects in different poses by incorporating pose conditioning information in the LDM. We obtain poses of the input images using OpenPose \cite{openpose}, with the assistance of style transfer for cartoons or non-human subjects (see Fig. \ref{fig:openpose}). We then linearly interpolate all shared keypoint positions from the two images to obtain intermediate poses for each image. The resulting pose is provided to the LDM using ControlNet \cite{controlnet}, a powerful method for conditioning on arbitrary image-like inputs. Interestingly, we observe that \textit{even when the wrong pose is predicted} for input images, conditioning on pose still yields superior interpolations as it prevents abrupt pose changes (see Fig. \ref{fig:pose}). %Additionally, we find it helpful to increase the strength of the pose conditioning for images towards the middle of the sequence.
% If the modality of the input images is unsuitable for obtaining accurate pose information (e.g. stylized cartoons), we can first perform style transfer to a photorealistic image using the LDM, which will be more suitable as input to OpenPose even if the image quality is poor .
% Figure environment removed
%, so by choosing the LDM to use 10-20 timesteps, this step becomes very fast


\vspace{-3pt}
\subsection{CLIP ranking}\label{sec:clip_ranking}
\vspace{-3pt}
LDMs can yield outputs of widely varying quality and characteristics with different random seeds. This problem is compounded in real image interpolation since a single bad generated image compromises the quality of all other images derived from it.
Thus when quality is more important than speed, multiple candidates can be generated with different random seeds, then ranked with CLIP \cite{clip}. We repeat each forward diffusion step with different noise vectors, denoise each of the interpolated latent vectors, then measure the CLIP similarity of the decoded image with specified positive and negative prompts (e.g., positive: ``high quality, detailed, 2D'', negative: ``blurry, distorted, 3D render''). The image with the highest value of positive similarity minus negative similarity is kept. %We generate more candidates for images at higher noise levels since these images have more freedom to deviate from the desired style.
In applications requiring an even higher degree of control and quality, this pipeline can be changed into an interactive mode where users can manually select desired interpolations or even specify a new prompt or pose for a particular image.

\section{Analysis Results}
\label{sec:experiment}
%To develop a more comprehensive understanding of Python security commits, we conduct a series of analysis: (1) Dataset Characteristics, (2) Security Commits Categories to reveal what types of vulnerabilities have been fixed, (3) Security Commits Distribution over Repositories, (4) Security Commits Complexity, (5) Security Commits Locality and (6) Fix Patterns Summarization to assist in automated program repair projects. 

After constructing our datasets, we frame our evaluation into four research questions, as outlined below. 
\begin{itemize}[leftmargin=*]

\item \textbf{RQ1:} Can the graph learning-based method help improve the data collection efficiency?
\item \textbf{RQ2:} How various and representative are the collected security commits? 
\item \textbf{RQ3:} What are the unique patterns of security commits in Python? 
\item \textbf{RQ4:} How do the wild commit samples help improve \gnn{} model for downstream security commit detection? 
\end{itemize}

\subsection{Dataset Construction (RQ1)}\label{results:efficiency}

After keyword filtering and graph-based identification with humans in the loop, we collect 1,258 security commits in total. Specifically, as shown in Table~\ref{tab: dataset}, there are 729, 400, and 129 security commits in the base, pilot, and augmented datasets, respectively. Also, 2,791 non-security commits are manually labeled during the collection process.

% \begin{table}[h]
% \centering
%     \caption{The statistical information of \db{}.}
%     \setlength{\tabcolsep}{3.4mm}{
%     \begin{tabular}{c|c|c|c|c}
%     \toprule
%     {} & \multirow{2}{*}{\shortstack{\bf Base\\\bf Dataset}} & \multirow{2}{*}{\shortstack{\bf Pilot\\\bf Dataset}} & \multirow{2}{*}{\shortstack{\bf Augmented\\\bf Dataset}} & \multirow{2}{*}{\bf Total} \\
%     {} & {} & {} & {} & {} \\
%      % & \textbf{Base} & \textbf{Pilot} & \textbf{Augmented} & \multirow{2}{*}{\textbf{Total}} \\
%      % & \textbf{Dataset} & \textbf{Dataset} & \textbf{Dataset} & \\
%      % & Base Dataset & Pilot Dataset & Augmented Dataset & Total \\
%                          % & \begin{tabular}[c]{@{}c@{}}Base \\ Dataset\end{tabular} & \begin{tabular}[c]{@{}c@{}}Pilot \\ Dataset\end{tabular} & \begin{tabular}[c]{@{}c@{}}Augmented \\ Dataset\end{tabular}  & Total \\ \hline
%     \midrule
%     \multirow{2}{*}{\shortstack{\bf Security\\\bf Commits}} & \multirow{2}{*}{729} & \multirow{2}{*}{400} & \multirow{2}{*}{129} & \multirow{2}{*}{1258} \\
%     {} & {} & {} & {} & {} \\
%     \midrule
%     \multirow{2}{*}{\shortstack{\bf Non-security\\\bf Commits}} & \multirow{2}{*}{2134} & \multirow{2}{*}{535} & \multirow{2}{*}{122} & \multirow{2}{*}{2791} \\
%     {} & {} & {} & {} & {} \\
%     \bottomrule
%     \end{tabular}
%     }
%     \label{tab: dataset}
% \end{table}

\begin{table}[h]
\vspace{-0.05in}
\centering
    \caption{The composition of \db{}.}
    \setlength{\tabcolsep}{1.7mm}{
    \begin{tabular}{c|p{1.0cm}<{\centering}|p{1.0cm}<{\centering}|p{1.4cm}<{\centering}|p{1.0cm}<{\centering}}
    \toprule
    {\diagbox{\bf Commit}{\bf Dataset}} & {\bf Base} & {\bf Pilot} & {\bf Augmented} & {\bf Total} \\
     % & \textbf{Base} & \textbf{Pilot} & \textbf{Augmented} & \multirow{2}{*}{\textbf{Total}} \\
     % & \textbf{Dataset} & \textbf{Dataset} & \textbf{Dataset} & \\
     % & Base Dataset & Pilot Dataset & Augmented Dataset & Total \\
                         % & \begin{tabular}[c]{@{}c@{}}Base \\ Dataset\end{tabular} & \begin{tabular}[c]{@{}c@{}}Pilot \\ Dataset\end{tabular} & \begin{tabular}[c]{@{}c@{}}Augmented \\ Dataset\end{tabular}  & Total \\ \hline
    \midrule
    {\bf Security} & {729} &  {400} &  {129} & {1258} \\
    \midrule
    {\bf Non-Security} & {2134} & {535} & {122} &{2791} \\
    \bottomrule
    \end{tabular}
    }
    \label{tab: dataset}
\vspace{-0.05in}
\end{table}

Table~\ref{tab:spr} lists the augmentation efficiency of random selection, keyword filtering, and \gnn{}.
Compared with identifying security commits from scratch, the keyword filtering mechanism improves the collecting efficiency by over 30 percentage points and \gnn{} improves the efficiency by 40 percentage points. %It is to be noted that we only test our data augmentation methods on a small portion of commits, which has already shown effectiveness.



\begin{table}[ht]
\vspace{-0.05in}
\centering
\caption{Efficiency of keyword filtering and \gnn{}.}
\setlength{\tabcolsep}{4mm}{
\begin{tabular}{c|c|c|c}
\toprule
% \multirow{2}{*}{\textbf{Methods}} & \multirow{2}{*}{\textbf{Candidates}} & \multirow{2}{*}{\shortstack{\bf Verified\\ \bf Security Commits}} & \multirow{2}{*}{\textbf{Ratio}} \\
% {} & {} & {} & {} \\
%  \midrule
\textbf{Method}      & \textbf{\# Candidates} & \textbf{\# Verified SC$^{*}$} & \textbf{Ratio} \\
 \midrule
% Methods      & Candidates & \begin{tabular}[c]{@{}c@{}}Verified \\ Security Commits\end{tabular} & Ratio   \\ \hline
{Random~\cite{wang2021patchdb}} & {-} & {-} & {6-10\%}    \\ 
\midrule
{Keywords} & {935} & {400} & {42.70\%} \\ 
\midrule
{\gnn{}} & {251} & {129} & {51.39\%} \\ 
\bottomrule

\end{tabular}
}
\begin{tablenotes}[flushleft]
    \footnotesize
    \item $^{*}$ SC = Security Commits.
\end{tablenotes}
% \vspace{-0.1in}
\label{tab:spr}
\vspace{-0.15in}
\end{table}

\begin{table}[]
\centering
\caption{Top 5 repositories by number of security commits.}
\label{tab:repo}
\setlength{\tabcolsep}{5.4mm}{
\begin{tabular}{c|c|c}
\toprule
\textbf{Repository} & \textbf{\#SecurityCommits} & \textbf{\textbf{Proportion}} \\
\midrule
django      & 166  & 13.20\%   \\ \midrule
twisted     & 87   & 6.91\%   \\ \midrule
glance      & 54   & 4.29\%     \\ \midrule
pillow     & 41   & 3.26\%     \\ \midrule
numpy       & 39   & 3.10\%        \\ \midrule
\rowcolor{gray!10}\textbf{Total of Top 5}                   & \textbf{387}   &   \textbf{30.76\%} \\
\bottomrule
\end{tabular}
}
\vspace{-0.1in}
\end{table}


\subsection{Security Commits Categorization and Distribution (RQ2)}
% \XD{should show broad coverage and variety of DB}}

%The Common Weakness Enumeration (CWE)
NVD CWE slice~\cite{CWE_slice} associated classification taxonomy serves to identify and describe security vulnerabilities.
% in terms of CWEs. 
To understand the purpose of these commits, we investigate the CWE types associated with the CVE reports and plot the distribution of the CWE types that have been explicitly documented. Among the 729 security commits linked to 556 CVEs, due to the limited number of MITRE human analysts, only 312 (56.1\%) CVEs have been assigned CWEs. %Since the CWE taxonomy is a hierarchical structure, a CVE can be assigned with more than one CWE. 
Even so, there are already 119 distinct CWEs associated with our security commits in the base dataset, which means our \db{} contains at least 119 types of security commits in terms of corresponding vulnerabilities.
Figure~\ref{fig:cwe} enumerates the most common CWEs, including frequent security problems such as cross-site scripting (CWE-79), path traversal (CWE-22), etc. Note that we do not directly assign CWE type to security samples in the remaining base, pilot, and augmented dataset since the MITRE CWE team has its own internal process. However, based on our observation and our data collection approaches that are able to introduce wild security commits with more variance (as discussed in~\ref{exp:variance}), \db{} can encompass a broad range of security concerns with various kinds of security commits, including but not limited to above 119 CWEs. 

% % Figure environment removed

% Figure environment removed


Our collected security commits distribute among 351 popular GitHub repositories unevenly. Among them, 69 repositories provide more than two security commits, bringing a certain amount of variety. In Table~\ref{tab:repo}, the top five repositories that have the most occurrence in our dataset are django~\cite{django_2023}, twisted~\cite{twisted_2023}, glance~\cite{openstack_2023}, pillow~\cite{python-pillow_2023}, and numpy~\cite{numpy_2023}, implying that the samples in \db{} align with the popularity trend of security issue in practice.
%react to security issues on time.




% \subsection{Security Commits Complexity}

% \subsection{Security Commits Locality}

\subsection{Patch Patterns (RQ3)}
\label{rq3}

We manually go through the whole \db{} dataset, 
% we manually explore the full security commit dataset. 
% samples that have less than 200 code lines.
% We find 1,027 samples in total, which take up 81.7\% of the security commits.
% After the comprehensive analysis, 
and discover four common security fix patterns (taking up 85.85\% of all security commit samples) that may benefit software maintenance, i.e., adding or updating sanity checks,  updating APIs, updating regular expressions, and updating security properties, as listed in Table~\ref{tab:pattern}.


\begin{table}[]
\centering
%\scriptsize
\caption{The pattern types of security commits in \db{}.}
\label{tab:pattern}
\setlength{\tabcolsep}{4.0mm}{
\begin{tabular}{l|c|c}
\toprule
    \textbf{Pattern} & \textbf{\#Commits} & \textbf{Proportion} \\ 
    \midrule
    {1) Add or Update Sanity Checks} & {416} & {37.12\%} \\ 
    \midrule
    {2) Update API Usage} & {241} & {19.16\%} \\ 
    \midrule
    {3) Update Regular Expressions} & {189} & {15.02\%} \\ 
    \midrule
    {4) Restrict Security Properties} & {183} & {14.55\%} \\ 
    \midrule
    {5) Others} & {178} & {14.15\%} \\ 
    \midrule 
    \rowcolor{gray!10}{\bf Total} & {\bf 1258} & {\bf 100\%} \\
    \bottomrule
\end{tabular}
}
\vspace{-0.1in}
\end{table}


%re 168 + 21
%api 214 +27
%if 420 + 47
%property 167 +16
%others 159 + 19

%re 151
%api 190
%if 369
%property 146
%others 43


\subsubsection{Add or Update Sanity Checks}
A sanity check is a basic method to quickly evaluate if a claim or a calculation result can be true, which has been extensively applied to multiple scenarios, e.g., authentication property verification, access control, HTTP request checking~\cite{wang2020machine}. 
We summarize three representative patterns that fix the vulnerabilities via adding or updating sanity checks, which are presented by 37.12\% of security commits in \db{}.

\noindent{\bf Authentication.} Authentication is the act of proving an assertion, e.g., we need to compare the identity with the system data to verify a system user. The authentication-related vulnerabilities
% occurs when the authentication is performed improperly, which 
provide attackers the opportunities to masquerade as legitimate users. To defend them, an effective solution is to perform the additional authentication by adding more check requirements or making existing conditions more restrictive. List~\ref{lst:auth} presents an example of fixing an authentication vulnerability by narrowing down an existing restriction from \texttt{\small True} (i.e., all possible return values except \texttt{\small False}) to \texttt{\small "on"} only.


\lstdefinestyle{lst}{
    float=th,
    floatplacement=tbp,
    % abovecaptionskip=0.01in,
    numbers=left, 
    numberstyle=\scriptsize, 
    numbersep = 5pt,
    framexleftmargin = 0in,
    framexrightmargin = 0in,
    breaklines = true,
    xleftmargin = 0.18in,
    xrightmargin = 0.1in,
    basicstyle=\ttfamily\scriptsize, 
    frame=lines,
    showtabs=true,
    showspaces=true,
    showstringspaces=false,
    literate={\ }{{\ }}1,
    aboveskip=-0.00in,
    belowskip=-0.15in,
}

\begin{lstlisting}[
language=diff, 
style=lst,
caption=An example of security commit to fix authentication vulnerability (CVE-2022-0273).,
label={lst:auth},
mathescape=true
]
 $\textbf{commit 0c0313f375bed7b035c8c0482bbb09599e16bfcf}$ 
 diff --git a/cps/shelf.py b/cps/shelf.py
 @@ -248,7 +248,7 @@ def create_edit_shelf(shelf,
 ...
         $\textbf{return}$ redirect(url_for('web.index'))
-    is_public = 1 if to_save.get("is_public") else 0
+    is_public = 1 if to_save.get("is_public") == "on" else 0
     $\textbf{if}$ config.config_kobo_sync:
 ...
\end{lstlisting}

\noindent{\bf Authorization.} Authorization refers to the process of granting or denying access to certain data or actions within a system.
Authorization comes after authentication and is achieved by an access control list (ACL).
The ACL is used to check the user identity with a list of authorized operations and determine which actions a user is allowed to take, e.g., file and data permission.
% determining  to perform and which are restricted, including but not limited to file permission and data permission. 
Unrestricted authorization may lead to improper resource consumption since attackers could bypass the system to access high-security level data. List~\ref{lst:access control1} is an example that fixes an authorization bypass exploit by requiring the value of \texttt{\small os.environ.get('GITHUB\_ACTIONS')} to be \texttt{\small true}.

\lstdefinestyle{lst}{
    float=th,
    floatplacement=tbp,
    % abovecaptionskip=0.01in,
    numbers=left, 
    numberstyle=\scriptsize, 
    numbersep = 5pt,
    framexleftmargin = 0in,
    framexrightmargin = 0in,
    breaklines = true,
    xleftmargin = 0.18in,
    xrightmargin = 0.1in,
    basicstyle=\ttfamily\scriptsize, 
    frame=lines,
    showtabs=true,
    showspaces=true,
    showstringspaces=false,
    literate={\ }{{\ }}1,
    aboveskip=+0.10in,
    belowskip=-0.30in,
}

\begin{lstlisting}[
language=diff, 
style=lst,
caption=An example of security commit that fixes an authorization bypass exploit vulnerability (CVE-2022-46179).,
label={lst:access control1},
mathescape=true
]
 $\textbf{commit c658b4f3e57258acf5f6207a90c2f2169698ae22}$  
 diff --git a/core.py b/core.py
 @@ -112,7 +112,7 @@ def actualsys() :
     $\textbf{if}$ attemps == 6:
         ## Brute force protection
         $\textbf{raise}$ Exception("Too many password attempts.")
-    if os.environ.get('GITHUB_ACTIONS') != "":
+    if os.environ.get('GITHUB_ACTIONS') == "true":
         logging.warning("Running on Github Actions")
         actualsys()
     $\textbf{elif}$ uname == cred.name and pwdhash == cred.pass:
\end{lstlisting}


\noindent{\bf HTTP Request.} If the interpretation of Content-Length and/or Transfer-Encoding headers between HTTP servers are inconsistent, the attackers may take advantage of this issue and send malicious requests to the servers, i.e., HTTP request smuggling. 
A good solution is to maintain the same interpretation methods in both front-end and back-end servers. 
In this way, an effective coding practice is to add consistent sanity checks on request interpretation for both servers. 
List~\ref{lst:http} adds such a sanity check on \texttt{\small data} to determine if all characters are digits.% to avoid such exploits.

\lstdefinestyle{lst}{
    float=th,
    floatplacement=tbp,
    % abovecaptionskip=0.01in,
    numbers=left, 
    numberstyle=\scriptsize, 
    numbersep = 5pt,
    framexleftmargin = 0in,
    framexrightmargin = 0in,
    breaklines = true,
    xleftmargin = 0.18in,
    xrightmargin = 0.1in,
    basicstyle=\ttfamily\scriptsize, 
    frame=lines,
    showtabs=true,
    showspaces=true,
    showstringspaces=false,
    literate={\ }{{\ }}1,
    aboveskip=-0.00in,
    belowskip=-0.15in,
}

\begin{lstlisting}[
language=diff, 
style=lst,
caption=An example of security commit that fixes an HTTP request smuggling vulnerability (CVE-2022-24801).,
label={lst:http},
mathescape=true
]
 $\textbf{commit 8ebfa8f6577431226e109ff98ba48f5152a2c416}$ 
 diff --git a/src/twisted/web/http.py b/src/twisted/web/http.py
 @@ -2274,6 +2274,8 @@ def fail():
     $\textbf{if}$ header == b"content-length":
+        if not data.isdigit():
+            return fail()
         $\textbf{try}$:
             length = int(data)
         $\textbf{except}$ ValueError:
\end{lstlisting}


\subsubsection{Update API Usage}% Packages}
Compared with implementing the fixes from scratch, there are abundant well-formulated packages that can be adopted to realize the intended functionalities and help enforce security restrictions. 
We notice that a large number (19.16\%) of Python security commits fix vulnerabilities by imposing or substituting APIs. %packages, which is different from other languages like C/C++.
%This pattern differs from the fix patterns in other languages, e.g., C/C++.
We further categorize such security fixes % into different types 
according to their application scenarios. %scopes.

\noindent{\bf General Purpose.} There is a set of security-related modifications on built-in packages shared by applications for various purposes. For instance, \texttt{\small re.escape} is an API to escape non-alphanumerics that are not part of regular expression syntax, to avoid OS command injection, code injection, and regular expression injection. List~\ref{lst:re} is a commit example to fix regular expression injection vulnerability, which demonstrates the application of \texttt{\small re.escape} on \texttt{\small user} and \texttt{\small collection\_url}.

\lstdefinestyle{lst}{
    float=th,
    floatplacement=tbp,
    % abovecaptionskip=0.01in,
    numbers=left, 
    numberstyle=\scriptsize, 
    numbersep = 5pt,
    framexleftmargin = 0in,
    framexrightmargin = 0in,
    breaklines = true,
    xleftmargin = 0.18in,
    xrightmargin = 0.1in,
    basicstyle=\ttfamily\scriptsize, 
    frame=lines,
    showtabs=true,
    showspaces=true,
    showstringspaces=false,
    literate={\ }{{\ }}1,
    aboveskip=-0.00in,
    belowskip=-0.15in,
}

\begin{lstlisting}[
language=diff, 
style=lst,
caption=An example of security commit that fixes a regular expression injection vulnerability (CVE-2015-8748).,
label={lst:re},
mathescape=true
]
 $\textbf{commit 4bfe7c9f7991d534c8b9fbe153af9d341f925f98}$ 
 diff --git a/radicale/rights/regex.py b/radicale/rights/regex.py
 @@ -65,7 +65,10 @@ def _read_from_sections(user, collection_url, permission):
 ...
-    regex = ConfigParser({"login": user, "path": collection_url})
+    # Prevent "regex injection"
+    user_escaped = re.escape(user)
+    collection_url_escaped = re.escape(collection_url)
+    regex = ConfigParser({"login": user_escaped, "path": collection_url_escaped})
 ...
\end{lstlisting}

\noindent{\bf Web Applications.} To properly process the inputs of web applications, security commits can adopt %the existing APIs %, e.g., \texttt{\small escape\_html}, 
APIs in third-party packages for Python
(e.g., \texttt{\small parser.quote}, \texttt{\small request.server.escape}, \texttt{\small django.utils.html.escape}, and \texttt{\small html.unescape}) to escape ampersands, brackets, and quotes to the HTML/XML entities or HTTP requests for defeating cross-site scripting (XSS) and HTTP Smuggling. 
List~\ref{lst:xss2} is an example that fixes an XSS vulnerability by using the API \texttt{\small django.utils.html.escape}.

\lstdefinestyle{lst}{
    float=th,
    floatplacement=tbp,
    % abovecaptionskip=0.01in,
    numbers=left, 
    numberstyle=\scriptsize, 
    numbersep = 5pt,
    framexleftmargin = 0in,
    framexrightmargin = 0in,
    breaklines = true,
    xleftmargin = 0.18in,
    xrightmargin = 0.1in,
    basicstyle=\ttfamily\scriptsize, 
    frame=lines,
    showtabs=true,
    showspaces=true,
    showstringspaces=false,
    literate={\ }{{\ }}1,
    aboveskip=+0.00in,
    belowskip=-0.15in,
}

\begin{lstlisting}[
language=diff, 
style=lst,
caption=An example of security commit that fixes an XSS vulnerability (CVE-2022-24710).,
label={lst:xss2},
mathescape=true
]
 $\textbf{commit f6753a1a1c63fade6ad418fbda827c6750ab0bda }$
 diff --git a/weblate/trans/forms.py b/weblate/trans/forms.py
 @@ -37,6 +37,7 @@
 ...
+from django.utils.html import escape
 ...
-    label = str(unit.translation.language)
+    label = escape(unit.translation.language)
 ...
\end{lstlisting}


\noindent{\bf Shell Commands.} To handle the shell commands securely, security fixes can adopt \texttt{\small shlex.quote} and \texttt{\small subprocess} to load or execute the commands. 
With the \texttt{\small shlex.quote} API, we can have an escaped version of shell inputs, which can be safely used as tokens in a command line to avoid shell command injection.
List~\ref{lst:shell} is an example that shows the usage of \texttt{\small shlex.quote} to fix a shell injection vulnerability. 

\lstdefinestyle{lst}{
    float=th,
    floatplacement=tbp,
    % abovecaptionskip=0.01in,
    numbers=left, 
    numberstyle=\scriptsize, 
    numbersep = 5pt,
    framexleftmargin = 0in,
    framexrightmargin = 0in,
    breaklines = true,
    xleftmargin = 0.18in,
    xrightmargin = 0.1in,
    basicstyle=\ttfamily\scriptsize, 
    frame=lines,
    showtabs=true,
    showspaces=true,
    showstringspaces=false,
    literate={\ }{{\ }}1,
    aboveskip=-0.00in,
    belowskip=-0.15in,
}

\begin{lstlisting}[
language=diff, 
style=lst,
caption=An example of security commit that fixes a shell injection vulnerability (CVE-2013-7416).,
label={lst:shell},
mathescape=true
]
 $\textbf{commit 2817869f98c54975f31e2dd674c1aefa70749cca }$
 diff --git a/canto_curses/guibase.py b/canto_curses/guibase.py
 @@ -156,6 +156,11 @@ def _fork(self, path, href, text, fetch=False):
 ...
+    href = shlex.quote(href)
 ...
\end{lstlisting}


\noindent{\bf Path Name.} 
If a path name is improperly neutralized, attackers may access the files and directories outside of the restricted location. 
This vulnerability can occur by using absolute file paths or manipulating the path variables where the reference files contain ``dot-dot-slash (../)" sequences or variations.
To effectively escape such unsafe sequences, Python security commits usually adopt the secure APIs, e.g., \texttt{\small werkzeug.utils.safe\_join}, \texttt{\small yaml.safe\_load}, and \texttt{\small werkzeug.utils.secure\_filename}, to prevent the files or directories from being accessed by malicious users. 
List~\ref{lst:path traversal} is a commit example that fixes a path traversal via using the API \texttt{\small werkzeug.utils.secure\_filename}.

\lstdefinestyle{lst}{
    float=th,
    floatplacement=tbp,
    % abovecaptionskip=0.01in,
    numbers=left, 
    numberstyle=\scriptsize, 
    numbersep = 5pt,
    framexleftmargin = 0in,
    framexrightmargin = 0in,
    breaklines = true,
    xleftmargin = 0.18in,
    xrightmargin = 0.1in,
    basicstyle=\ttfamily\scriptsize, 
    frame=lines,
    showtabs=true,
    showspaces=true,
    showstringspaces=false,
    literate={\ }{{\ }}1,
    aboveskip=+0.10in,
    belowskip=-0.25in,
}

\begin{lstlisting}[
language=diff, 
style=lst,
caption=An example of security commit that fixes a path traversal vulnerability (CVE-2022-23609).,
label={lst:path traversal},
mathescape=true
]
 $\textbf{commit 1eb1e5428f0926b2829a0bbbb65b0d946e608593}$ 
 diff --git a/upload/server.py b/upload/server.py
 @@ -5,7 +5,7 @@
-
+import werkzeug.utils
 @@ -189,7 +189,7 @@ def uploadimage():
     filename = all_files[0][1] + all_files[0][2]
-    remove(filename)
+    remove(werkzeug.utils.secure_filename(filename))
     $\textbf{del}$ all_files[0]
     length = len(all_files)
\end{lstlisting}


\subsubsection{Update Regular Expressions}
Python has become a popular choice for back-end web development, and it is usually combined with some other front-end languages~\cite{python_app}. For this reason, we observe there are 15.02\% fixes that modify the regular expressions to avoid XSS, SQL injection, and open redirect vulnerabilities. 
% Python inserts itself in web development as a back-end language, and it is usually combined with some other front-end language (e.g., javascript) to build a whole website.
% We observe 15.02\% 
The regular expression patterns are tailored to match specific strings within the given text, including SQL commands, URLs, and other scripts.

\noindent{\bf SQL Commands.} The improper neutralization of SQL commands may lead to SQL injection vulnerabilities, which allow attackers to manipulate the backend database and access the information not intended to be displayed.
The corresponding fixes need to escape the unsafe characters. 
List~\ref{lst:sql} is a fixed example of SQL injection vulnerability, which substitutes the matched single and double quote characters (i.e., \texttt{\small '} and \texttt{\small "}) in the string \texttt{\small self.queueid}.

\lstdefinestyle{lst}{
    float=th,
    floatplacement=tbp,
    % abovecaptionskip=0.01in,
    numbers=left, 
    numberstyle=\scriptsize, 
    numbersep = 5pt,
    framexleftmargin = 0in,
    framexrightmargin = 0in,
    breaklines = true,
    xleftmargin = 0.18in,
    xrightmargin = 0.1in,
    basicstyle=\ttfamily\scriptsize, 
    frame=lines,
    showtabs=true,
    showspaces=true,
    showstringspaces=false,
    literate={\ }{{\ }}1,
    aboveskip=+0.0in,
    belowskip=-0.15in,
}

\begin{lstlisting}[
language=diff, 
style=lst,
caption=An example of security commit that fixes a SQL injection vulnerability (CVE-2014-125082).,
label={lst:sql},
mathescape=true
]
 $\textbf{commit fc2c1ea1b8d795094abb15ac73cab90830534e04}$
 diff --git a/.../model.py b/.../model.py
 @@ -772,13 +772,13 @@ def _get_filter(self):
 $\textbf{if}$ self.queueid:
-    ... = '%s'" % (self.queueid)
+    ... = '%s'" % (re.sub("[\"']", "", self.queueid))
\end{lstlisting}


\noindent{\bf URLs.} The improper neutralization of URLs may lead to open redirect vulnerability, which redirects an unsuspecting victim from a legitimate domain to an attacker’s phishing site. 
Effective mitigation is to replace the dangerous special characters with trusted symbols. List~\ref{lst:redirect} is an example of an open redirect vulnerability, which replaces the explicit backslash with an encoded backslash to circumvent the dangerous redirect.

\lstdefinestyle{lst}{
    float=th,
    floatplacement=tbp,
    % abovecaptionskip=0.01in,
    numbers=left, 
    numberstyle=\scriptsize, 
    numbersep = 5pt,
    framexleftmargin = 0in,
    framexrightmargin = 0in,
    breaklines = true,
    xleftmargin = 0.18in,
    xrightmargin = 0.1in,
    basicstyle=\ttfamily\scriptsize, 
    frame=lines,
    showtabs=true,
    showspaces=true,
    showstringspaces=false,
    literate={\ }{{\ }}1,
    aboveskip=-0.00in,
    belowskip=-0.15in,
}

\begin{lstlisting}[
language=diff, 
style=lst,
caption=An example of security commit that fixes an open redirect vulnerability (CVE-2019-10255).,
label={lst:redirect},
mathescape=true
]
 $\textbf{commit 08c4c898182edbe97aadef1815cce50448f975cb}$ 
 diff --git a/auth/login.py b/auth/login.py
 @@ -39,6 +39,10 @@ def _redirect_safe(self, url, ...):
+    url = url.replace("\\", "%5C")
     parsed = urlparse(url)
     $\textbf{if}$ parsed.netloc $\textbf{or not}$ (parsed.path + '/').startswith(self.base_url):
\end{lstlisting}

\noindent{\bf Scripts.} The improper input validation and encoding during web page generation may lead to XSS, which is able to reveal the cookies, session tokens, or other sensitive information retained by the browser to the attackers. A straightforward solution is to validate the matched characters of a pre-defined pattern. List~\ref{lst:xss} is an example to fix the XSS vulnerability by re-matching the characters between parentheses instead of the characters between square brackets and validating the matched pattern one by one.

\lstdefinestyle{lst}{
    float=th,
    floatplacement=tbp,
    % abovecaptionskip=0.01in,
    numbers=left, 
    numberstyle=\scriptsize, 
    numbersep = 5pt,
    framexleftmargin = 0in,
    framexrightmargin = 0in,
    breaklines = true,
    xleftmargin = 0.18in,
    xrightmargin = 0.1in,
    basicstyle=\ttfamily\scriptsize, 
    frame=lines,
    showtabs=true,
    showspaces=true,
    showstringspaces=false,
    literate={\ }{{\ }}1,
    aboveskip=+0.10in,
    belowskip=-0.25in,
}

\begin{lstlisting}[
language=diff, 
style=lst,
caption=An example of security commit that fixes an XSS vulnerability (CVE-2021-3994).,
label={lst:xss},
mathescape=true
]
 $\textbf{commit a22eb0673fe0b7784f99c6b5fd343b64a6700f06}$ 
 diff --git a/helpdesk/models.py b/helpdesk/models.py
 @@ -238 +238 @@ def cvesForCPE(cpe,
     $\textbf{if not}$ text:
         $\textbf{return}$ ""
-    pattern = fr'([\[\s\S\]]*?)\(([\s\S]*?):([\[\s\S\]]*?)\)'
+    pattern = fr'([\[\s\S\]]*?)\(([\s\S]*?):([\s\S]*?)\)'
     # Regex check
     $\textbf{if}$ re.match(pattern, text):
         # get get value of group regex
\end{lstlisting}



\subsubsection{Restrict Security Properties} 
The exploits often result from improper settings of security properties. 
14.55\% security commits in \db{} fix improper settings by updating boolean flags from \texttt{\small True} to \texttt{\small False} or vice versa, adding more arguments to methods, or adding security decorators.


\noindent{\bf Update Security Flags.} 
Security flags perform restrictions on the methods that may have access to sensitive objects. 
Improper restrictions on such flags may expose users to a risky environment and/or lead to sensitive information leakage. 
List~\ref{lst:flag} changes the flag from \texttt{\small False} to \texttt{\small True} to fix a vulnerability, where a sensitive cookie does not have a `HttpOnly' flag.


\lstdefinestyle{lst}{
    float=th,
    floatplacement=tbp,
    % abovecaptionskip=0.01in,
    numbers=left, 
    numberstyle=\scriptsize, 
    numbersep = 5pt,
    framexleftmargin = 0in,
    framexrightmargin = 0in,
    breaklines = true,
    xleftmargin = 0.18in,
    xrightmargin = 0.1in,
    basicstyle=\ttfamily\scriptsize, 
    frame=lines,
    showtabs=true,
    showspaces=true,
    showstringspaces=false,
    literate={\ }{{\ }}1,
    aboveskip=-0.00in,
    belowskip=-0.15in,
}

\begin{lstlisting}[
language=diff, 
style=lst,
caption=An example of security commit that fixes a vulnerability where the sensitive cookie does not have a `HttpOnly' flag (CVE-2019-25091).,
label={lst:flag},
mathescape=true
]
 $\textbf{commit 60a3fe559c453bc36b0ec3e5dd39c1303640a59a}$ 
 diff --git a/src/nsupdate/settings/base.py b/src/nsupdate/settings/base.py
 @@ -283,7 +283,7 @@
 ...
-CSRF_COOKIE_HTTPONLY = False
+CSRF_COOKIE_HTTPONLY = True
 ...
\end{lstlisting}

\noindent{\bf Add Restriction Arguments.} Some restriction arguments will be passed to the functions during execution. Improper argument settings may lead to a variety of mishandling. As shown in List~\ref{lst:arg}, the \texttt{\small formaction} is added to restrict the attributes of a variable to avoid XSS vulnerability.


\lstdefinestyle{lst}{
    float=th,
    floatplacement=tbp,
    % abovecaptionskip=0.01in,
    numbers=left, 
    numberstyle=\scriptsize, 
    numbersep = 5pt,
    framexleftmargin = 0in,
    framexrightmargin = 0in,
    breaklines = true,
    xleftmargin = 0.18in,
    xrightmargin = 0.1in,
    basicstyle=\ttfamily\scriptsize, 
    frame=lines,
    showtabs=true,
    showspaces=true,
    showstringspaces=false,
    literate={\ }{{\ }}1,
    aboveskip=-0.00in,
    belowskip=-0.15in,
}

\begin{lstlisting}[
language=diff, 
style=lst,
caption=An example of security commit that fixes a cross-site-scripting (XSS) vulnerability (CVE-2021-28957).,
label={lst:arg},
mathescape=true
]
 $\textbf{commit 10ec1b4e9f93713513a3264ed6158af22492f270}$ 
 diff --git a/src/lxml/html/defs.py b/src/lxml/html/defs.py
 @@ -23,6 +23,8 @@
 ...
+    # HTML5 formaction
+    'formaction'
     ])
 ...
\end{lstlisting}

\noindent{\bf Add Security Decorators.} A decorator is a function that takes another function and extends the behavior of the function without explicit modification. This mechanism has been widely adopted by security commits to add more detailed security restrictions on existing methods. List~\ref{lst:access control2} shows a security commit that fixes an access control vulnerability by adding decorator \texttt{\small security.private} to function \texttt{\small enumerateRoles}.

\lstdefinestyle{lst}{
    float=th,
    floatplacement=tbp,
    % abovecaptionskip=0.01in,
    numbers=left, 
    numberstyle=\scriptsize, 
    numbersep = 5pt,
    framexleftmargin = 0in,
    framexrightmargin = 0in,
    breaklines = true,
    xleftmargin = 0.18in,
    xrightmargin = 0.1in,
    basicstyle=\ttfamily\scriptsize, 
    frame=lines,
    showtabs=true,
    showspaces=true,
    showstringspaces=false,
    literate={\ }{{\ }}1,
    aboveskip=+0.10in,
    belowskip=-0.25in,
}

\begin{lstlisting}[
language=diff, 
style=lst,
caption=An example of security commit that fixes an access control vulnerability (CVE-2021-21336).,
label={lst:access control2},
mathescape=true
]
 $\textbf{commit 2dad81128250cb2e5d950cddc9d3c0314a80b4bb}$ 
 diff --git a/src/Products/plugins/ZODBRoleManager.py b/src/Products/plugins/ZODBRoleManager.py
 @@ -112,6 +112,7 @@ def getRolesForPrincipal(self, principal, request=None):
     #   IRoleEnumerationPlugin implementation
+    @security.private
     $\textbf{def}$ enumerateRoles(self, id=None, exact_match=False, sort_by=None, max_results=None, **kw):
         """ See IRoleEnumerationPlugin.
\end{lstlisting}



\subsection{Unique Patterns Captured from the Wild (RQ4)}\label{exp:variance}

Recall that we construct pilot and augmented datasets because the base dataset provides a limited number of security commits samples. Here, we further show the examples captured by our security commit collection approaches that introduce more variety in syntax and semantics of security-related code changes, enabling wider applications of \db{} in solving real-world Python-related security issues.

\subsubsection{Data Variety Introduced by Pilot Dataset}
We study the contribution of involving the pilot dataset for \gnn{} by comparing the model trained only on the base dataset and the model trained on the combination of the base and pilot datasets.
% first training on the base dataset and then training on the combination of the base dataset and the pilot dataset.
%Then, we analyze the samples that have not been identified by the first model but have been identified by the second model. 
We find that the pilot dataset helps the latter model to be able to identify more wild security commits. For instance, the latter \gnn{} can detect more subtle changes. % after expanding the training set with the pilot dataset. 
In List~\ref{lst:pilot}, the \texttt{\small '\%s'} has been changed to \texttt{\small ?} in a SQL query, protecting the database from being injected. 
% the condition refines the value of \texttt{\small GITHUB\_ACTIONS} from not null to true, protecting the authentication from being bypassed. 
The capability of detecting such minor changes is enabled by similar samples in the pilot dataset but not existed in the base dataset.

\lstdefinestyle{lst}{
    float=th,
    floatplacement=tbp,
    % abovecaptionskip=0.01in,
    numbers=left, 
    numberstyle=\scriptsize, 
    numbersep = 5pt,
    framexleftmargin = 0in,
    framexrightmargin = 0in,
    breaklines = true,
    xleftmargin = 0.18in,
    xrightmargin = 0.1in,
    basicstyle=\ttfamily\scriptsize, 
    frame=lines,
    showtabs=true,
    showspaces=true,
    showstringspaces=false,
    literate={\ }{{\ }}1,
    aboveskip=-0.00in,
    belowskip=-0.15in,
}

\begin{lstlisting}[
language=diff, 
style=lst,
caption=An example of security commit detected by \gnn{} trained on the base and pilot datasets.,
label={lst:pilot},
mathescape=true
]
 $\textbf{commit 9d8adbc07c384ba51c2583ce0819c9abb77dc648}$ 
 diff --git .../__init__.py .../__init__.py
 @@ -71,7 +71,7 @@ def klauen(self,
-    a = u"name == '%s' AND item =='%s'" % (name, item)
+    a = u"name == ? AND item ==?", (name, item)
\end{lstlisting}
%  $\textbf{commit c658b4f3e57258acf5f6207a90c2f2169698ae22}$ 
%  diff --git a/core.py b/core.py
%  @@ -112,7 +112,7 @@ def actualsys() :
% -    if os.environ.get('GITHUB_ACTIONS') != "":
% +    if os.environ.get('GITHUB_ACTIONS') == "true":
%          logging.warning("Running on Github Actions")
%          actualsys()

\subsubsection{Variance Introduced by Augmented Dataset}
We further evaluate to show that our augmented dataset can help train a model that is able to identify more various security commits from the wild. For example, after introducing augmented dataset into the training phase, the model detects a new escape pattern. As shown in List~\ref{lst:augmented}, the characters \texttt{\small <}, \texttt{\small >}, and \texttt{\small \&} have been escaped by being translated into Unicode, which prevents cross-site-scripting crafted with a partial JSON-serializable object. Compared with the escape expressions in Section~\ref{rq3} that only include ASCII characters, the augmented dataset help \gnn{} generalize the escapes to Unicode.


%46e95f5

\lstdefinestyle{lst}{
    float=th,
    floatplacement=tbp,
    % abovecaptionskip=0.01in,
    numbers=left, 
    numberstyle=\scriptsize, 
    numbersep = 5pt,
    framexleftmargin = 0in,
    framexrightmargin = 0in,
    breaklines = true,
    xleftmargin = 0.18in,
    xrightmargin = 0.1in,
    basicstyle=\ttfamily\scriptsize, 
    frame=lines,
    showtabs=true,
    showspaces=true,
    showstringspaces=false,
    literate={\ }{{\ }}1,
    aboveskip=-0.00in,
    belowskip=-0.15in,
}

\begin{lstlisting}[
language=diff, 
style=lst,
caption=A security commit example detected by the \gnn{} trained on the base{,} pilot{,} and augmented datasets.,
label={lst:augmented},
mathescape=true
]
 $\textbf{commit d3e428a6f7bc4c04d100b06e663c071fdc1717d9}$ 
 diff --git a/.../djblets_js.py b/.../djblets_js.py 
 @@ -28,11 +28,18 @@
+_safe_js_escapes = {
+    ord('&'): u'\\u0026',
+    ord('<'): u'\\u003C',
+    ord('>'): u'\\u003E',
+}
\end{lstlisting}

\section{Conclusion}

In this work, we present \texttt{vox2vec} --- a self-supervised framework for voxel-wise representation learning in medical imaging. Our method expands the contrastive learning setup to the feature pyramid architecture allowing to pre-train effective representations in full resolution. By pre-training a FPN backbone to extract informative representations from unlabeled data, our method scales to large datasets across multiple task domains. We pre-train a FPN architecture on more than 6500 CT images and test it on various segmentation tasks, including different organs and tumors segmentation in three setups: linear probing, non-linear probing, and fine-tuning. Our model outperformed existing methods in all regimes. Moreover, \texttt{vox2vec} establishes a new state-of-the-art result on the linear and non-linear probing scenarios. 

Still, this work has a few limitations to consider. We plan to investigate further how the performance of \texttt{vox2vec} scales with the increasing size of the pre-training dataset and the pre-trained architecture size. Another interesting research direction is exploring the effectiveness of \texttt{vox2vec} in the domain adaptation and few-shot learning scenarios.


%%%%%%%%% REFERENCES
% \clearpage
\pagebreak
% {\small
\bibliographystyle{assets/icml2023}
\bibliography{biblio}
% }

\newpage
% \newenvironment{proof}{\paragraph{Proof:}}{\hfill$\square$}
% \renewcommand\thefigure{\thesection.\arabic{figure}}
% \renewcommand\thetable{\thesection.\arabic{table}}
\renewcommand\thefigure{\thesection.\arabic{figure}}
\renewcommand\thetable{\thesection.\arabic{table}}
% \renewcommand\thealgorithm{\thesection.\arabic{algorithm}}
\appendix
\onecolumn
\setcounter{figure}{0}
\setcounter{table}{0}
% \setcounter{algorithm}{0}
% \textbf{\Large{Appendix}}
\section{Additional Figures}
\input{figs/b_more_egs}

\input{figs/5_failures}



\end{document}
