%%%%%%%%%%%%%%%%%%%%%%%%%%%%%%%%%%%%%%%%%%%%%%%%%%%%%%%%%%%%%%%%%%%%%
%%                                                                 %%
%% Please do not use \input{...} to include other tex files.       %%
%% Submit your LaTeX manuscript as one .tex document.              %%
%%                                                                 %%
%% All additional figures and files should be attached             %%
%% separately and not embedded in the \TeX\ document itself.       %%
%%                                                                 %%
%%%%%%%%%%%%%%%%%%%%%%%%%%%%%%%%%%%%%%%%%%%%%%%%%%%%%%%%%%%%%%%%%%%%%

%\documentclass[lineno, referee,sn-basic]{sn-jnl}% referee option is meant for double line spacing with lines number
\documentclass[referee,sn-basic]{sn-jnl}
\pdfsuppressptexinfo=-1


%%=======================================================%%
%% to print line numbers in the margin use lineno option %%
%%=======================================================%%

%% \documentclass[lineno,sn-basic]{sn-jnl}% Basic Springer Nature Reference Style/Chemistry Reference Style

%%======================================================%%
%% to compile with pdflatex/xelatex use pdflatex option %%
%%======================================================%%

%%\documentclass[pdflatex,sn-basic]{sn-jnl}% Basic Springer Nature Reference Style/Chemistry Reference Style

%%\documentclass[sn-basic]{sn-jnl}% Basic Springer Nature Reference Style/Chemistry Reference Style
%% \documentclass[superscriptaddress,preprint]{revtex4}
%%\documentclass[pdflatex,sn-mathphys]{sn-jnl}% Math and Physical Sciences Reference Style
%% \documentclass[sn-aps]{sn-jnl}% American Physical Society (APS) Reference Style
%%\documentclass[sn-vancouver]{sn-jnl}% Vancouver Reference Style
%%\documentclass[sn-apa]{sn-jnl}% APA Reference Style
%%\documentclass[sn-chicago]{sn-jnl}% Chicago-based Humanities Reference Style
%%\documentclass[sn-standardnature]{sn-jnl}% Standard Nature Portfolio Reference Style
%%\documentclass[default]{sn-jnl}% Default
%%\documentclass[default,iicol]{sn-jnl}% Default with double column layout

%%%% Standard Packages
%%<additional latex packages if required can be included here>
%%%%

%%%%%=============================================================================%%%%
%%%%  Remarks: This template is provided to aid authors with the preparation
%%%%  of original research articles intended for submission to journals published 
%%%%  by Springer Nature. The guidance has been prepared in partnership with 
%%%%  production teams to conform to Springer Nature technical requirements. 
%%%%  Editorial and presentation requirements differ among journal portfolios and 
%%%%  research disciplines. You may find sections in this template are irrelevant 
%%%%  to your work and are empowered to omit any such section if allowed by the 
%%%%  journal you intend to submit to. The submission guidelines and policies 
%%%%  of the journal take precedence. A detailed User Manual is available in the 
%%%%  template package for technical guidance.
%%%%%=============================================================================%%%%

\jyear{2023}%

%% as per the requirement new theorem styles can be included as shown below
\theoremstyle{thmstyleone}%
\newtheorem{theorem}{Theorem}%  meant for continuous numbers
\newtheorem{proposition}[theorem]{Proposition}% 
%%\newtheorem{proposition}{Proposition}% to get separate numbers for theorem and proposition etc.

\theoremstyle{thmstyletwo}%
\newtheorem{example}{Example}%
\newtheorem{remark}{Remark}%
\theoremstyle{thmstylethree}%
\newtheorem{definition}{Definition}%

\raggedbottom
%%\unnumbered% uncomment this for unnumbered level heads

% \newcommand{\ema}[1]{\textbf{\textcolor{teal}{XX #1 XX}}}
\renewcommand{\figurename}{Figure}
\renewcommand{\thefigure}{S\arabic{figure}}
\renewcommand{\tablename}{Table}
\renewcommand{\thetable}{S\arabic{table}}

\begin{document}

\title[Supplementary Information:
Platinum-based Catalysts for ORR
 simulated with a
Quantum Computer]{Supplementary Information: Platinum-based Catalysts for Oxygen
Reduction Reaction simulated with a
Quantum Computer}

%%=============================================================%%
%% Prefix	-> \pfx{Dr}
%% GivenName	-> \fnm{Joergen W.}
%% Particle	-> \spfx{van der} -> surname prefix
%% FamilyName	-> \sur{Ploeg}
%% Suffix	-> \sfx{IV}
%% NatureName	-> \tanm{Poet Laureate} -> Title after name
%% Degrees	-> \dgr{MSc, PhD}
%% \author*[1,2]{\pfx{Dr} \fnm{Joergen W.} \spfx{van der} \sur{Ploeg} \sfx{IV} \tanm{Poet Laureate} 
%%                 \dgr{MSc, PhD}}\email{iauthor@gmail.com}
%%=============================================================%%

\author*[1]{\fnm{Cono} \sur{Di Paola}}\email{cono.dipaola@quantinuum.com}
% \equalcont{These authors contributed equally to this work.}
\author*[1] {\fnm{Evgeny} \sur{Plekhanov}}\email{evgeny.plekhanov@quantinuum.com}
% \equalcont{These authors contributed equally to this work.}
\author[1]{\fnm{Michal} \sur{Krompiec}}\email{}
%\equalcont{These authors contributed equally to this work.}
\author[2]{\fnm{Chandan} \sur{Kumar}}\email{}
%\equalcont{These authors contributed equally to this work.}
\author[3]{\fnm{Emanuele} \sur{Marsili}}\email{}
%\equalcont{These authors contributed equally to this work.}
\author[2]{\fnm{Fengmin} \sur{Du}}\email{}
%\equalcont{These authors contributed equally to this work.}
\author[5]{\fnm{Daniel} \sur{Weber}}\email{}
%\equalcont{These authors contributed equally to this work.}

\author[4]{\fnm{Jasper Simon} \sur{Krauser}}\email{}
%\equalcont{These authors contributed equally to this work.}
\author[2]{\fnm{Elvira} \sur{Shishenina}}\email{}
%\equalcont{These authors contributed equally to this work.}
\author[1]{\fnm{David} \sur{Mu\~noz Ramo}}\email{}
%\equalcont{These authors contributed equally to this work.}

\affil[1]{\orgname{Quantinuum}, \orgaddress{\street{Terrington House, 13-15 Hills Road}, \city{Cambridge} \postcode{CB2 1NL}, \country{United Kingdom}}}

\affil[2]{\orgname{BMW Group}, \orgaddress{\city{Munich} \postcode{80788},  \country{Germany}}}

\affil[3]{\orgname{Airbus, Central Research \& Technology}, \orgaddress{\street{Pegasus House Aerospace Ave}, \city{Bristol} \postcode{BS34 7PA},  \country{United Kingdom}}}

\affil[4]{\orgname{Airbus, Central Research \& Technology}, \orgaddress{\street{Willy-Messerschmidt-Str. 1}, \city{Taufkirchen} \postcode{82024}, \country{Germany}}}

\affil[5]{\orgname{Aerostack GmbH}, \orgaddress{\city{Dettingen an der Erms} \postcode{72581}, \country{Germany}}}


\maketitle


\tableofcontents

\pagebreak

\section{Calculated statistical error on quantum device}
\begin{table}[h]
\begin{center}
\caption{VQE and VQE+PT2 statistical error $\epsilon$, calculated as a standard deviation by bootstrapping 100k and 10k shots in 10 batches each for Quantinuum H1-E noisy emulator and H1 device measurements, respectively. $E_a=E_{TS}-E_{O_2}$ and $E_d=E_{2O}-E_{O_2}$ represent the activation and dissociation energies, respectively. E$_{2O}$ is the total energy for the 'trans' configuration. All the measurements on H1-E and H1 include PMSV error mitigation correction}\label{table:vqe_deviation3}%
\begin{tabular}{@{}lrr|rr@{}}
\toprule
& \multicolumn{2}{c}{$VQE$} & \multicolumn{2}{c}{$VQE+PT2$}\\
\\
 & H1-E  &  H1  & H1-E & H1\\
\midrule
$\epsilon(E_{a})$ [eV] &  0.03993 & 0.12936 & 0.34386  &  0.26283\\
$\epsilon (E_d)$ [eV] & 0.02747  & 0.10834  & 0.03050  &  0.17411 \\
\botrule
\end{tabular}
\end{center}
\end{table}

\pagebreak

\section{Energy difference: experiments vs state vector}
\begin{table}[h]
\begin{center}
\caption{VQE and VQE+PT2 calculated distance $\Delta$ of total averaged energies with respect to StateVector ($E_{SV}$) calculations performed using Qulacs backend in eV units. Total energy calculated as an average by bootstrapping 100k and 10k shots in 10 batches each for Quantinuum H1-E noisy emulator and H1 device measurements, respectively. All the measurements on H1-E and H1 include PMSV error mitigation correction}\label{table:vqe_deviation2}%
\begin{tabular}{@{}lrr|rr@{}}
\toprule
& & \multicolumn{2}{c}{$\Delta (E-E_{SV}) \, \mathrm{[eV]}$} \\
\\
& \multicolumn{2}{c}{$VQE$} & \multicolumn{2}{c}{$VQE+PT2$}\\
\\
Species & H1-E  &  H1  & H1-E & H1\\
\midrule
O$_2$ &  0.14965 &  0.20835 &  0.04404 &  0.02131 \\
TS & 0.17522  & 0.29947 &  0.17557 &  $-$0.03943 \\
2O(`trans') &  0.10064  &  0.16303 &  $-$0.01634 &  0.05915 \\
\botrule
\end{tabular}
\end{center}
\end{table}


 \pagebreak

 \section{Density difference for fragment choice}
 
% Figure environment removed

\pagebreak

\section{Pure platinum surface: AVAS/RE threshold convergence}
 
% Figure environment removed

\pagebreak
 
\section{Pure platinum VQE and VQE+ NEVPT2 H1-E noisy emulator experiments}
 
% Figure environment removed

 \pagebreak
 

 \section{Pt/Co slabs: Optimised atomistic geometries}

% Figure environment removed


 \pagebreak


%%===========================================================================================%%
%% If you are submitting to one of the Nature Portfolio journals, using the eJP submission   %%
%% system, please include the references within the manuscript file itself. You may do this  %%
%% by copying the reference list from your .bbl file, paste it into the main manuscript .tex %%
%% file, and delete the associated \verb+\bibliography+ commands.                            %%
%%===========================================================================================%%



\section{Details of Pt/Co slabs optimisation}

\begin{table}[h]
\begin{center}
\caption{Summary of the structural properties and energetics of 1, 2 a 3ML Pt capped Co slabs without surface relaxation. Notice that the total energy difference with plus sign signifies that the energy of two dissociated O atoms is higher than the energy of the O$_2$ molecule on the surface.}
\begin{tabular}{c|c|c|c}
\toprule
& 1 ML Pt & 2 ML Pt & 3 ML Pt   \\
\midrule
O-O dist.(\AA)               & 1.23    & 1.23    & 1.23      \\
O-Pt dist. (\AA)              & 3.32    & 3.41    & 3.37      \\
%m O    (mB)           & 0.561   & 0.559   & 0.553    \\
$\Delta E$ (Ry w.r.t.   O$_2$) & $+0.0841$  & $+0.11134$ & $+0.0501$  \\ 
\bottomrule
\end{tabular}
\end{center}
\end{table}

\begin{table}[h]
\begin{center}
\caption{Summary of the structural optimization for Pt/Co/Pt slabs with 1, 2 and 3 ML Pt capping. Reported are the inter-layer distances in \AA. Pure Pt bulk inter-layer distance amounts to $\approx 2.22$\AA.}
   \label{tabSslabs}
   \begin{tabular}{c|c|c|c}
\toprule 
distance (\AA)  & 1ML Pt & 2ML Pt & 3ML Pt\\

\midrule 
Pt-Pt &    -    &   -   & 2.6083\\
Pt-Pt &    -    & 2.6736& 2.5998\\
Pt-Co & 2.2418  & 2.2197& 2.2303\\
Co-Co & 1.9568  & 1.9727& 1.9589\\
Co-Co & 2.0125  & 2.0121& 2.0164\\
Co-Co & 1.9989  & 2.0023& 1.9997\\
Co-Co & 1.9989  & 2.0023& 1.9997\\
Co-Co & 2.0125  & 2.0121& 2.0164\\
Co-Co & 1.9568  & 1.9727& 1.9589\\
Pt-Co & 2.2418  & 2.2197& 2.2303\\
Pt-Pt &    -    & 2.6736& 2.5998\\
Pt-Pt &    -    &   -   & 2.6083\\
\bottomrule
   \end{tabular}
\end{center}
\end{table}


\begin{table}[h]
\begin{center}
\caption{Summary of the magnetic polarization for Pt/Co/Pt slabs with 1, 2 and 3 ML Pt capping. Reported are the Pt (Co) magnetic moments in Bohr magneton units per atom in each layer.}
\label{tabSmag}
   \begin{tabular}{c|c|c|c}
\toprule 
  & 1ML Pt ($\mu_B$) & 2ML Pt ($\mu_B$) & 3ML Pt ($\mu_B$) \\
\midrule 
Pt  &  -    &  -    &-0.0168\\
Pt  &  -    & 0.0472& 0.0037\\
Pt  & 0.1738& 0.1712& 0.1275\\
Co  & 1.8255& 1.8279& 1.7760\\
Co  & 1.6999& 1.7394& 1.7433\\
Co  & 1.7385& 1.7666& 1.7534\\
Co  & 1.7071& 1.7430& 1.7474\\
Co  & 1.7385& 1.7666& 1.7534\\
Co  & 1.6999& 1.7394& 1.7433\\
Co  & 1.8255& 1.8279& 1.7760\\
Pt  & 0.1738& 0.1712& 0.1275\\
Pt  &  -    & 0.0472& 0.0037\\
Pt  &  -    &  -    &-0.0168\\
\bottomrule
   \end{tabular}
\end{center}
\end{table}

\pagebreak

\section{AVAS VQE$_1$ for Pt/Co}

% Figure environment removed
%\newpage
%\bibliography{references}% common bib file

\end{document}
