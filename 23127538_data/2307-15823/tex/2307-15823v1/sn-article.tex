 %%%%%%%%%%%%%%%%%%%%%%%%%%%%%%%%%%%%%%%%%%%%%%%%%%%%%%%%%%%%%%%%%%%%%
%%                                                                 %%
%% Please do not use \input{...} to include other tex files.       %%
%% Submit your LaTeX manuscript as one .tex document.              %%
%%                                                                 %%
%% All additional figures and files should be attached             %%
%% separately and not embedded in the \TeX\ document itself.       %%
%%                                                                 %%
%%%%%%%%%%%%%%%%%%%%%%%%%%%%%%%%%%%%%%%%%%%%%%%%%%%%%%%%%%%%%%%%%%%%%

%\documentclass[lineno, referee,sn-basic]{sn-jnl}% referee option is meant for double line spacing with lines number
\documentclass[referee,sn-basic]{sn-jnl}
\pdfsuppressptexinfo=-1


%%=======================================================%%
%% to print line numbers in the margin use lineno option %%
%%=======================================================%%

%% \documentclass[lineno,sn-basic]{sn-jnl}% Basic Springer Nature Reference Style/Chemistry Reference Style

%%======================================================%%
%% to compile with pdflatex/xelatex use pdflatex option %%
%%======================================================%%

%%\documentclass[pdflatex,sn-basic]{sn-jnl}% Basic Springer Nature Reference Style/Chemistry Reference Style

%%\documentclass[sn-basic]{sn-jnl}% Basic Springer Nature Reference Style/Chemistry Reference Style
%% \documentclass[superscriptaddress,preprint]{revtex4}
%%\documentclass[pdflatex,sn-mathphys]{sn-jnl}% Math and Physical Sciences Reference Style
%% \documentclass[sn-aps]{sn-jnl}% American Physical Society (APS) Reference Style
%%\documentclass[sn-vancouver]{sn-jnl}% Vancouver Reference Style
%%\documentclass[sn-apa]{sn-jnl}% APA Reference Style
%%\documentclass[sn-chicago]{sn-jnl}% Chicago-based Humanities Reference Style
%%\documentclass[sn-standardnature]{sn-jnl}% Standard Nature Portfolio Reference Style
%%\documentclass[default]{sn-jnl}% Default
%%\documentclass[default,iicol]{sn-jnl}% Default with double column layout

%%%% Standard Packages
%%<additional latex packages if required can be included here>
%%%%

%%%%%=============================================================================%%%%
%%%%  Remarks: This template is provided to aid authors with the preparation
%%%%  of original research articles intended for submission to journals published 
%%%%  by Springer Nature. The guidance has been prepared in partnership with 
%%%%  production teams to conform to Springer Nature technical requirements. 
%%%%  Editorial and presentation requirements differ among journal portfolios and 
%%%%  research disciplines. You may find sections in this template are irrelevant 
%%%%  to your work and are empowered to omit any such section if allowed by the 
%%%%  journal you intend to submit to. The submission guidelines and policies 
%%%%  of the journal take precedence. A detailed User Manual is available in the 
%%%%  template package for technical guidance.
%%%%%=============================================================================%%%%

\jyear{2023}%

%% as per the requirement new theorem styles can be included as shown below
\theoremstyle{thmstyleone}%
\newtheorem{theorem}{Theorem}%  meant for continuous numbers
%%\newtheorem{theorem}{Theorem}[section]% meant for sectionwise numbers
%% optional argument [theorem] produces theorem numbering sequence instead of independent numbers for Proposition
\newtheorem{proposition}[theorem]{Proposition}% 
%%\newtheorem{proposition}{Proposition}% to get separate numbers for theorem and proposition etc.

\theoremstyle{thmstyletwo}%
\newtheorem{example}{Example}%
\newtheorem{remark}{Remark}%

\theoremstyle{thmstylethree}%
\newtheorem{definition}{Definition}%

\raggedbottom
%%\unnumbered% uncomment this for unnumbered level heads

\begin{document}

\title[Applicability of Quantum Computing to ORR Simulations]{Applicability of Quantum Computing to Oxygen Reduction Reaction Simulations}

%%=============================================================%%
%% Prefix	-> \pfx{Dr}
%% GivenName	-> \fnm{Joergen W.}
%% Particle	-> \spfx{van der} -> surname prefix
%% FamilyName	-> \sur{Ploeg}
%% Suffix	-> \sfx{IV}
%% NatureName	-> \tanm{Poet Laureate} -> Title after name
%% Degrees	-> \dgr{MSc, PhD}
%% \author*[1,2]{\pfx{Dr} \fnm{Joergen W.} \spfx{van der} \sur{Ploeg} \sfx{IV} \tanm{Poet Laureate} 
%%                 \dgr{MSc, PhD}}\email{iauthor@gmail.com}
%%=============================================================%%

\author*[1]{\fnm{Cono} \sur{Di Paola}}\email{cono.dipaola@quantinuum.com}
% \equalcont{These authors contributed equally to this work.}
\author*[1] {\fnm{Evgeny} \sur{Plekhanov}}\email{evgeny.plekhanov@quantinuum.com}
% \equalcont{These authors contributed equally to this work.}

\author[1]{\fnm{Michal} \sur{Krompiec}}\email{}
%\equalcont{These authors contributed equally to this work.}

\author[2]{\fnm{Chandan} \sur{Kumar}}\email{}
%\equalcont{These authors contributed equally to this work.}

\author[2]{\fnm{Fengmin} \sur{Du}}\email{}
%\equalcont{These authors contributed equally to this work.}

\author[4]{\fnm{Daniel} \sur{Weber}}\email{}
%\equalcont{These authors contributed equally to this work.}


\author[3]{\fnm{Jasper Simon} \sur{Krauser}}\email{}
%\equalcont{These authors contributed equally to this work.}


\author[2]{\fnm{Elvira} \sur{Shishenina}}\email{}
%\equalcont{These authors contributed equally to this work.}

\author[1]{\fnm{David} \sur{Mu\~noz Ramo}}\email{}
%\equalcont{These authors contributed equally to this work.}

\affil*[1]{\orgname{Quantinuum}, \orgaddress{\street{Terrington House, 13-15 Hills Road}, \city{Cambridge}, \postcode{CB2 1NL},  \country{United Kingdom}}}

\affil[2]{\orgname{BMW Group}, \orgaddress{\postcode{80788} \city{Munich},  \country{Germany}}}

\affil[3]{\orgname{Airbus, Central Research \& Technology}, \orgaddress{\street{Willy-Messerschmidt-Str. 1}, \postcode{82024} \city{Taufkirchen},  \country{Germany}}}

\affil[4]{\orgname{Aerostack GmbH}, \orgaddress{ \postcode{72581} \city{Dettingen an der Erms}, \country{Germany}}}

%%==================================%%
%% sample for unstructured abstract %%
%%==================================%%

\abstract{
Hydrogen is considered a promising energy source for low-carbon and sustainable mobility. However, for wide scale adoption, improvements in the efficiency of hydrogen-to-electricity conversion are required. The kinetics of the electrocatalytic oxygen reduction reaction (ORR) is the main bottleneck of proton-exchange membrane fuel cells (PEMFC), hence development of new cathode materials is of paramount importance. Computational design of new catalysts can shorten the development time, but, due to the complexity of the ORR potential energy landscape, atomistic-level modeling of this processes is challenging. Moreover, the catalytic species may exhibit strong electronic correlations, which cannot be rigorously and accurately described with low-cost methods such as Density Functional Theory (DFT). Accurate ab-initio methods have been traditionally considered not applicable to such systems due to their high computational cost, but this soon may be possible thanks to the rapid advances in quantum computing. 

Here, we present the first classical/quantum computational study of the ORR on both pure platinum and platinum-capped cobalt surface and show the applicability of quantum computing methods to a complex catalysis problem. We show the feasibility of such a workflow implemented in InQuanto\textsuperscript{(TM)} and demonstrate it on the H1-1 trapped-ion quantum computer. Most importantly, we show that ORR on Pt/Co catalyst involves strongly-correlated species which are good test cases for future demonstration of quantum advantage. }

%After generating the reaction model at the DFT level by means of the QuantumEspresso package [2], we first
%perform HF calculations [3] and select an appropriate fragment of the model which comprises the most correlated orbitals
%of the system. We then construct an effective fermionic Hamiltonian in second quantised form in order to reduce the
%complexity of the problem. Subsequently, via Jordan-Wigner mapping [4], we translate it into a qubit Hamiltonian with a
%large enough orbital active space. We finally use VQE [5] as a state preparation with an even more reduced active space
%to retrieve the expectation value of the Hamiltonian by also mitigating the device error using Partition Measurement
%Symmetry Verification (PMSV) algorithm [6]. 
%The adopted Quantum Computing embedding workflow is as implemented by using
%InQuanto(TM), our quantum computational chemistry software.}

\keywords{Quantum computing, ab-initio simulations, PEM fuel cells, oxygen reduction reaction, low-carbon mobility }

%%\pacs[JEL Classification]{D8, H51}

%%\pacs[MSC Classification]{35A01, 65L10, 65L12, 65L20, 65L70}

\maketitle

\section{Introduction}\label{sec1}
Understanding the underlying mechanism of the oxygen reduction reaction (ORR) is key to the development of advanced and high-performing materials as catalysts, which find applications as, for example, cathodes of proton-exchange membrane fuel cells (PEMFC, schematics in Fig.~\ref{fig:pemfc})~\cite{zaman2021}.  PEMFCs are key to portable applications of fuel cell technology, such as in the transportation sector.

During this reduction reaction, coupled net transfer of four protons and electrons to the molecular oxygen ideally generates a potential of $1.23\,V_\mathrm{RHE}$ (RHE stands for reversible hydrogen electrode) in acidic media. Industrial research on ORR is focused on determining the cause of significant kinetic overpotential (over 0.3\,V) that reduces the cell voltage to less than 0.9\,V at any usable current output, and on innovating ways for mitigating this loss~\cite{Keith2010}.

Because of the very complex ORR kinetic pathway, studying this process at the atomic level is very challenging.
It is known that the ORR mechanism typically includes oxidation of the catalyst surface and/or strongly bound intermediates~\cite{zhang2007}. To understand the kinetic and thermodynamic factors driving the formation of these intermediates it is paramount to first get access to the binding energies of species involving mainly O and H atoms adsorbed on the surface. This task can be efficiently achieved by means of first-principle quantum mechanics method such as DFT or HF. Subsequently, transition state approaches like the Nudged Elastic Band (NEB) algorithm can be used to statically describe the reaction path and the activation energies of what is considered the rate-determining step of ORR, i.e. $\mathrm{O_2^* \rightarrow 2O}$: oxygen molecule dissociation on the heterogeneous catalyst surface. 

% Figure environment removed

In the difficult quest of finding a good compromise for a PEMFC cathode material with high performance, platinum and its alloys with nickel and cobalt~\cite{asara2016, shao2016, kulkarni2018, stamenkovic2007, zou2015, wang2013} have played a dominant role from the beginning, showing promising structural and electronic properties to destabilise intermediates and reduce activation energy barriers~\cite{zhang2021, molmen2021, wang2021}.

% The aim of this study is to prototype 'realistic' first-principles atomistic modelling of the Oxygen Reduction Reaction on a noble metal catalyst, by exploiting the potentiality of Quantum Computer devices.

In this context, DFT methods have been successful at simulating and modelling large systems of different nature. However, DFT suffers from a strong dependence on approximate functionals and usually exhibits poor accuracy for strongly correlated systems~\cite{cohen2012}.

On the other hand, highly accurate wave-function-based techniques, such as coupled cluster (CC), full configuration-interaction  quantum Monte Carlo (FCIQMC) and Density Matrix Renormalization Group (DMRG) have been applied to periodic systems~\cite{gruber2018a, gruber2018b, Tsatsoulis2018, booth2013, DMRG_rev_Reiher2020}. Unfortunately, their computational resource requirements extensively increase as one tries to explore more realistic systems, rendering these techniques impractical for most real-world applications in catalysis and materials science.

Quantum technology may be a good candidate to achieve a combination of high accuracy (matching the `gold standard' CCSD(T)), applicability (including strongly correlated systems), and better scalability (compared to QMC or DMRG). In particular, a class of quantum computing algorithms such as the variational quantum eigensolver (VQE) and its variants can make use of the noisy intermediate-scale quantum (NISQ) devices for simulation of eigenstates of a given Hamiltonian. 

Quantum computers naturally implement ansatze such as unitary coupled-cluster (UCC) with only a polynomial number of quantum gates. This is a big improvement if compared to the exponential increase of the computational cost in classical simulations~\cite{yoshioka2022,Gujarati2022,Yamamoto2022, Sureshbabu2021, liu2020}, although demonstration of practical advantage over classical methods will require improved quantum computing hardware and application of quantum phase estimation techniques. 

In the present study, we, for the first time, prototype a first-principles atomistic model of the ORR both on a noble metal catalyst and on its core-shell alloying arrangement with cobalt by exploiting the potential of Quantum devices. This is performed by means of a hybrid quantum-classical workflow developed at Quantinuum~\cite{krompiec2022} as implemented in InQuanto computational chemistry package~\cite{InQuanto}. We demonstrate this workflow on the H1-1 trapped-ion quantum computer. 


\section{Results and Discussion}\label{sec2}
\subsection{Catalysed oxygen reaction reduction}\label{subsec2}

In the multi-step process associated with the oxygen reduction reaction on Pt, the first one-electron transfer is generally regarded as the rate-determining step in the entire process \cite{Keith2010}. This transfer can be associated either to proton-coupled process $\mathrm{O_2^*+H^++e^-\rightarrow HO_2^*}$ \cite{Damjanovic1967, Sepa1987} or to a reductive and dissociative adsorption step $\mathrm{O_2 \rightarrow 2O}$ \cite{Clouser1993, yeager1993} catalysed by an electron exchange.

Keith and Jacob reviewed the DFT studies of competing potential- and pH-dependent mechanisms of ORR on Pt(111) surface models \cite{Keith2010}. A schematic of the network of these reactions is shown on Figure~\ref{fig:full_path}.

% Figure environment removed

In the present computational study of the ORR process, we will focus on the single-step dissociation of $\mathrm{O_2 \rightarrow 2O}$ promoted by single electron transfer, mediated by Pt(111) and Co@Pt(111) core-shell on a (5$\times$5$\times$3) and (5$\times$5$\times$5) surface slab models, respectively. 

\subsection{Classical setup calculations in periodic boundary condition regime}\label{subsec3}
The model was built by carrying out simulations for the bulk metal as well as for the multi-layer surface slabs with Quantum Espresso (QE) Density Functional Theory (DFT) package~\cite{QE_09, QE_17}. Spin polarised simulations with DFT-D3 van der Waals dispersion correction~\cite{grimme3} for geometry optimisation of structures of interest were performed. In all DFT calculations, we employed the Perdew-Burke-Ernzerhof (PBE) \cite{pbe} GGA exchange-correlation functional together with projector-augmented wave (PAW) pseudopotentials~\cite{Blochl1994}. In the case of Pt surface reaction, the wave-function and charge density cut-offs of 36-47 Ry and 221-448 Ry, respectively, together with a Brillouin zone sampling mesh of (8$\times$8$\times$8) for the bulk system and a Marzari-Vanderbilt smearing of 0.04 Ry were used.

In the Co@Pt case, a wave function cut-off of $60$ Ry with a charge density limit of $450$ Ry were applied by including a sampling k-point mesh of (10x10x1), PAW pseudopotentials and a Marzari-Vanderbilt smearing of $0.01$ Ry.
% The in-house generated PAW pseudopotentials and the plane wave cutoff of $60$ Ry were utilized.

 To avoid any density overlapping along non-periodic directions, cubic boxes of at least 10\,\AA\ for small isolated molecules in gas phase such as O$_2$ and $\sim35$\,\AA\, of vacuum on top of multi-layer surfaces ($z$-axis) were adopted.


\subsubsection{Modelling of the ORR on Pt(111)}\label{subsubsec2_1}

Assuming that the dissociative process is the central and the slowest step in ORR, an exploratory investigation has been  focused on establishing how our level of theory could reproduce the predicted adsorption sites of O$_2$ molecule as on bridge position between two nearest-neighbour platinum atoms and for atomistic oxygen in hollow FCC site. The positions of the possible four canonical (111) adsorption sites are depicted on Figure~\ref{fig:scheme_111}.

% Figure environment removed

% The model has been first simulated in the reduced (5x5x3) slab defined by the inter-layer distances $d_{12}$ and $d_{23}$ (as depicted in Figure~\ref{fig:slabs}b). The calculations were performed in spin-polarised framework and at the $\Gamma$ point, also including vand der Waals dispersion forces. 

% The topmost surface layer and the  adsorbates were allowed to move freely during the structural relaxation, while the coordinates of two remaining layers underneath, considered as ’bulk platinum’, were frozen. The calculations were performed in non-spin-polarised framework and at the $\Gamma$ point. 

We also defined the quantities characterising the process of interest (see energy profile sketched in Figure~\ref{fig:en_profile_scheme}) as: 
$$E_d=E_\mathrm{Pt/2O}^\mathrm{tot}-E_\mathrm{Pt/O_2}^\mathrm{tot}$$
that represents the thermodynamic driving force for the surface-assisted molecule dissociation and
$$E_a=E_\mathrm{Pt/O_2^*}^\mathrm{tot}-E_\mathrm{Pt/O_2}^\mathrm{tot}$$
that estimates the activation energy related to the kinetic constant of the process.

Since preliminary studies of possible deformation/reconstruction of the pure platinum surface under deposition and dissociation of one O$_2$ molecule showed negligible effects on both structural and energetic characteristic of the original slab, we decided to optimise only the forces on the adsorbate and keep the coordinates of the platinum atoms frozen.

We first simulated the model in the reduced (5x5x3) slab. In this study we were able to reproduce the preferred adsorption sites for the considered molecules on Pt(111), demonstrating that the three-layered slab structure and the approximations applied to our simulations can %qualitatively 
give a good estimate of the geometrical properties of the adsorbates. 

% But this presents also two major limits. 

% Its first inconvenience originates from the definition of the quantities that we would like to calculate (see energy profile sketched in Figure~\ref{fig:en_profile_scheme}): 
% $$E_d=E_{Pt/2O}^{tot}-E_{Pt/O_2}^{tot}$$
% that represents the thermodynamical driving force for the surface-assisted molecule dissociation and
% $$E_a=E_{Pt/O_2^*}^{tot}-E_{Pt/O_2}^{tot}$$
% that represents the activation energy related to the kinetic constant of the process.

% Figure environment removed


% These energies are calculated as energy differences between independently optimised slabs with adsorbates. This introduces additional sources of error and makes error cancellation less effective. To avoid this issue, we could optimise only the forces on the adsorbate and keep the coordinates of the platinum atoms frozen. 

Following this initial exploration of the relative stability of different arrangements considered in the dissociation process, we proceeded to  construction of a larger system to address the bulk inconsistency of the innermost layers in the smaller 5$\times$5$\times$3 structure. 

%This introduced a very large technical challenge in terms of memory consumption for InQuanto's implementation of the single impurity DMET partitioning scheme intended to be used in this study, due to the necessity of calculation of the 1-particle reduced density matrix of the whole supercell. 

Spin-polarised fully-relaxed molecule-slab structure simulations using the PBE functional together with Grimme's DFT-D3  dispersion correction were performed on the 75-atom 3-layer and 125-atom 5-layer surface models, as generally depicted on Figure~\ref{fig:slabs}. 

% Figure environment removed

As already observed in the theoretical study of the dissociation of O$_2$ in strained Pt(111)~\cite{xue2018}, we found two different dissociated states, namely the `cis' and `trans' arrangements, that were separated by $E_{trans}-E_{cis}$(5$\times$5$\times$5)$ = -0.172\,\mathrm{eV}$ in favour of the most stable `trans' configuration and with a dissociation energy for the best arrangement of $E_d$(`trans')(5$\times$5$\times$5)$ = -1.219\,\mathrm{eV}$, as shown in Fig.~\ref{fig:en_profile_scheme}. This also correctly predicted the dissociated oxygen atoms to be more stable than the corresponding adsorbed molecule on the platinum surface.

% The same workflow was applied to the palladium surface slabs and we found $E_{trans}-E_{cis}$(5x5x5)$ = -0.258 \; eV$ and $E_d('trans')$(5x5x5)$ = -1.578\; eV$ for the 'trans'/'cis' relative stability and for the 'trans' dissociation energies, respectively.

\subsubsection{Co@Pt(111) core-shell model for oxygen dissociation}\label{subsubsec2_1_1}

% Figure environment removed

In order to simulate the oxygen reduction assisted by the Co@Pt surface, we started with optimising the fcc Co slabs capped with 1, 2 and 3 atomic layers of Pt, denoted 1L, 2L and 3L hereafter. These models are built over a (1$\times$1$\times$10) optimised unit cell where a 10$\times$10$\times$1 $k$-point mesh was used.

We observed that the distribution of magnetic moments along the slabs showed a crucial dependency on the thickness of Pt capping, with the magnetism of the surface Pt atoms dramatically decaying with the distance from Co bulk, as shown in Table~\ref{tab:magnet}.

Subsequently, we proceeded to the calculation of O$_2$ bound to the surface (reagents or R) and two separate O atoms bound to the surface (products or P) systems for 1L, 2L and 3L Pt on Co slabs. 
Using the results for the optimised Co@Pt slabs, we built the O$_2$ and 2O adsorbed Co@Pt models with three Co layers and one, two and three atomic layers of Pt on the surface.
The resulting unit cell dimensions for the overall set of slabs were fixed at $2.49\times 2.49\times 64.41$ \AA$^3$.
In these calculations, we considered 5$\times$5 in-plane supercell  so that each slab layer was composed of $25$ atoms. The three Co substrate sublayer amounted to $75$ Co atoms, therefore, the total numbers of atoms in the 1L, 2L and 3L Co@Pt slabs (without oxygen) were: $100$, $125$ and $150$ respectively, as depicted in Figure~\ref{fig:slabs_copt}.
The use of PBC allowed us to reduce the influence of the boundary conditions in the surface reaction simulations.

One of the main results of the present work is the paramount importance of the surface relaxation in establishing the correct energy profile for the supported oxygen
dissociation reaction on the surface. A frozen surface would lead to a scenario where the computed energies of the P states are approximately $0.1$\,eV higher than the R states for all the Pt-layer thicknesses considered,
% 1L, 2L and 3L Co@Pt systems, 
i.e. there would be no catalytic effect at all. 
On the contrary, we were able to restore the correct energy ordering when we allowed the topmost Pt layer to relax together with the adsorbates, showing an energy differences of $1.12$, $0.28$ and $0.34$ eV for 1L, 2L and 3L, respectively. In doing this, we performed a constrained geometry optimisation by allowing the top Pt layer to relax only along the $z$ direction and by maintaining the underlying Pt and Co layers frozen at their optimised positions. Only oxygen atoms were allowed to relax in all the directions.
% The optimisation has revealed that the Pt surface shape is very sensible to the relative positions of the oxygen atoms above. 
% It is clear, therefore, that the surface relaxation plays a central role in establishing the correct energy profile for the oxygen dissociation reaction on the surface.

\begin{table}[]
    \caption{Magnetisation (in $\mu_B$) of the optimised Co@Pt slabs with one (1L), two (2L) and three (3L) atomic layers of Pt in the absence of oxygen.}
    \centering
    \begin{tabular}{c|c|c|r}
        at. layer & 1L    & 2L    & 3L \\
        \hline\hline
        Pt  &   -    &    -   & $-$0.0168 \\
        Pt  &   -    & 0.0472 &  0.0037 \\
        Pt  & 0.1738 & 0.1712 &  0.1275 \\
        Co  & 1.8255 & 1.8279 &  1.7760 \\
        Co  & 1.6999 & 1.7394 &  1.7433 \\
        Co  & 1.7385 & 1.7666 &  1.7534 \\
    \end{tabular}

    \label{tab:magnet}
\end{table}

% For further analysis of the oxygen dissociation reaction by using the Nudged Elastic Band (NEB), 
% we have chosen to bring forward the 2L system, since: i) there are experimental indications that it would be the best catalyst~\cite{osti1820884} and ii) because it represents an intermediate case between the 1L case with a lower catalytic effect and the 3L one which is closer to the bulk Pt limit.


We finally chose to adopt the 2L system for our further investigation of Co@Pt catalytic support for oxygen reduction, since: i) there are experimental indications that it would be the best catalyst~\cite{osti1820884} and ii) because it represents an intermediate case between the 1L arrangement with a lower catalytic effect and the 3L one which is closer to the pure Pt bulk limit.

\subsubsection{The transition state}\label{subsubsec2_2}

To calculate the activation energy $E_a$, we needed to determine the structure of the transition state (TS), $\mathrm O_2^*$ on Pt(111). To this end, we performed a Nudged Elastic Band calculation \cite{jonsson1998}, as implemented in Quantum Espresso.  

After a preliminary NEB simulation performed on 3-layer (5$\times$5$\times$3) models of Pt/($\mathrm{O_2 \rightarrow 2O}$), a large Pt(111) (5$\times$5$\times$5) slab was considered to address possible divergence of the most internal layers from the bulk geometry.  Only the two oxygen atoms were free to move during the optimisation and a loose force convergence threshold of about 0.05\,eV/\AA~ was used. Results for platinum are shown in Figure~\ref{fig:neb_pt}. 

We considered two minimum-energy profiles (MEPs), one leading to dissociated oxygen in `cis' (red line/dots) and one ending to the `trans' final state (green line/dots), using 6 and 10 NEB images, respectively. We also refined the transition state structures by means of the climbing image algorithm.

Interestingly, in both cases we observed an identical TS rotated geometry at a path length $\approx$ 0.4 with an $E_a = 0.687$\,eV and independent on the structure of the final state. We found the so-calculated activation energy in excellent agreement with the value of $E_a = 0.67 \pm 0.33\,\mathrm{eV}$ found in literature~\cite{Montemore2018}.

% Figure environment removed

\subsubsection{Co@Pt Minimal Energy Path}

Due to significant reconstruction of the platinum surface layer observed during the geometry relaxation, we performed $\mathrm{O_2 \rightarrow 2O}$ NEB simulation on top of Co@Pt by letting the 25 Pt atoms adapt to the oxygen molecule dissociation 
% during the energy profile optimisation 
just along the $z$ direction, since the in-plane atomic displacements are negligible due to highly packed $(111)$ arrangement. 

In the initial and final regions of the MEP, i.e. close to the P and R states, we observed diffusion effects due to the same structural `floppiness' shown by the metallic surface. For this reason, we had to refine the NEB profile several times, by zooming on the part of the path 
containing the activation energy barrier and by discarding the diffusion parts, where the energy remained essentially flat along the
reaction coordinate and so irrelevant for the determination of $E_a$.

The highly computationally challenging NEB optimisation of 27 atoms (the remaining $100$ were frozen at their optimised position) was performed by using 10 NEB images and 200 steps. Within DFT, we found a final barrier $E_a$ of about 0.46\,eV,
while the driving force was 1.30\,eV. In addition, the MEP displayed a
double peak structure with two almost equal heights. We considered the image $6$ in Figure~\ref{fig:NEBcentral} as the TS geometry, and from that we derived a smaller
system in order to finally perform the experiments on the quantum emulator.

% \textcolor{red}{
% We have also performed a NEB calculation for the determination of the structure and energy of the transition state (TS), $O_2^*$ on Co@Pt.
% The lower Pt and all Co layers were kept frozen, so that the two oxygen atoms and $25$ surface Pt atoms were allowed to relax during NEB: the former along all directions, while the latter along $z$ direction, 
% since the in plane atomic displacements are expected to be negligible due to highly packed $(111)$ arrangement. 
% We have refined the NEB profile several times, by zooming into the MEP region containing the activation energy barrier and isolating the diffusion parts close to R and P respectively, where the energy remains essentially flat along the reaction coordinate. Therefore, in what follows, we will focus on the central barrier part, since the diffusion parts are irrelevant for the determination of $E_a$.
% }


% For the barrier part, we have used $10$ NEB images and the first and last images were allowed to relax during NEB. The converged energy profile is shown in Fig.~\ref{fig:NEBcentral}. Bringing along the NEB profile optimisation with $27$ atoms moving is quite a challenging computational task which has required significant computational resources. It has required almost $200$ NEB iterations to converge. The converged energy barrier within DFT was found to be $0.46$ eV, while the driving force: $1.30$ eV. In addition, the converged MEP exhibits a double peak structure with two almost equal heights. We have taken the point number $5$ in Fig.~\ref{fig:NEBcentral} as a TS geometry and proceed to derivation of a smaller system in order to finally perform the experiments on quantum emulator.


% Figure environment removed


\subsection{From `classical' DFT models to quantum hardware experiments}\label{subsubsec3}

We decided to use the O$_2$/2O on Pt(111) (5$\times$5$\times$5) model obtained with DFT to develop a workflow for high-accuracy post-Hartree-Fock hybrid quantum-classical calculations of reaction energies. Taking advantage of the (presumed) local nature of the differences in electronic structure along the reaction path, we adopted local correlation methods, i.e. the post-Hartree-Fock step involving orbitals spatially localised around the adsorption site. 

Due to the limited size of qubit registers available in NISQ machines and quantum emulators, we employed the active space approximation. To account for the remaining correlation energy, we investigated the use of QRDM\_NEVPT2, a hybrid quantum-classical implementation of NEVPT2~\cite{krompiec2022} available in InQuanto.

\subsubsection{Single fragment and reduced active space: Regional Embedding}\label{subsubsubsec3_1}

The choice of the sub-system (fragment) where the active orbitals are localised was motivated by an analysis of the density change due to bonding with the adsorbate (see Figure~\ref{fig:density_diff} of Supplementary Information). 
Here we adopted the so-called Regional Embedding (AVAS/RE) \cite{RegionalEmbedding}, a variant of the Automatic Valence Active Space (AVAS) method \cite{AVAS}, an automatic procedure for generation of localised active spaces, developed for a use case similar to ours (i.e., adsorption energy on a periodic slab). In contrast to embedding approaches, it does not require calculation of the reduced density matrix of the whole unit cell, hence it has very low memory consumption. 

To define the AVAS/RE active space for the 3-layer atomistic model, we chose valence and higher orbitals of the `Pt19O2' fragment. This included 10 surface atoms around the adsorption site and 9 atoms just underneath, as depicted in Figure~\ref{fig:fragment_553}a. 

To select the best AVAS/RE space, we studied the convergence of the dissociation energies ($E_d$) for both `cis' and `trans' with respect to the AVAS overlap threshold. We found constant energy and reproduced DFT trend in the relative stability inside the threshold window 0.5-0.2 (see Figure~\ref{fig:fragment_553}b).

We also tested energy convergence against the number of layers in the surface model by choosing a narrower threshold window around 0.2 as in Figure~\ref{fig:fragment_555}b. We observed a shift in energies due to size effects in the smaller system. Calculations on the 5-layer surface model (as depicted in Figure~\ref{fig:fragment_555}a) clearly showed that a threshold=0.18 was able not only to reproduce the same DFT trend in the relative stability between the two final states but also to generate the same AVAS/RE active space of (310e,324o) for all four models used in this study.

\subsubsection{AVAS/RE on the core-shell system}


For the Co@Pt case, we organised the AVAS/RE workflow as follows.
A smaller finite fragment was cut out of the optimised periodic structures, as depicted schematically in Fig.~\ref{fig:small}.
% Figure environment removed
This choice of a smaller fragment was dictated by several considerations: i) the charge difference between R and P states is localised in the area containing the two oxygen atoms and in a 3$\times$3 square of surface Pt atoms, ii) the second Pt layer and at least one Co layer should be included into the small cluster in order to see the effect of magnetism on the oxygen dissociation reaction, as shown in Table~\ref{tab:magnet}, and iii) the 3$\times$3 in-plane fragment was large enough to reproduce the energetics of the PBC (5$\times$5$\times$5) slab.

The resulting model comprised $29$ atoms: $2$ oxygen, $18$ platinum and $9$ cobalt atoms. The magnetisation of each Co atom was essentially the same as in the QE calculations for the full periodic model ca. $2\mu_B$, so that the whole fragment could be easily treated within PySCF~\cite{sun2018, sun2020} ROHF calculations as a first step. 
% We have performed ROHF calculations for all of P, TS and R states with the geometries obtained by cutting from the full periodic model described in the previous subsection.



In the AVAS procedure the surface Pt $5d$, O $2p$ and $3p$ orbitals were employed as local atomic projectors, even though in an ROHF open-shell situation with finite magnetisation this was not a trivial task. This was due to the fact that half-filled Co 3d orbitals, the main source of the total magnetism in the system, contributed to populate the d-band manifold near the Fermi level. In our specific case, their correlation contribution had to be removed from the active space, since we were interested in correlations associated with the oxygen dissociation reaction and not with the Co magnetism. 



We tested the following AVAS total occupied (first number) and virtual (second number) orbitals' combinations: $\{1,2\}$, $\{2,3\}$, $\{3,3\}$, $\{4,3\}$ and $\{4,4\}$.
% With this non-standard notation, we would like to  the first number in parentheses signifies the active orbitals (states) which are occupied in the reference, while the second one stands for active virtual orbitals. 
These correspond to the following traditional CAS notations: $(2e,3o)$, $(4e,5o)$, $(6e,6o)$, $(8e,7o)$ and, $(8e,8o)$ respectively. We also checked that all the AVAS active orbitals were always localised on the two oxygen atoms and the surface platinum atoms nearest to the oxygens. 


\subsubsection{Quantum hardware experiments: the hybrid quantum-classical workflow}\label{subsubsubsec3_3}

First, we built the fermionic Hamiltonian by allowing excitations only within the subspace (310e,324o) obtained with the AVAS/RE procedure. Subsequently, we constructed a new smaller Hamiltonian where the only active space was the hardware-compatible (2e,3o) that is equivalent to 2 electrons in a circuit register of 6 qubits (6 spin-orbitals).

% Figure environment removed

% Figure environment removed


% \begin{Figure}
%     \centering
%     % Figure removed
%     \captionof{figure}{Panel a): 'Pt19O2' fragment as solid spheres where the red oxygen atoms are showing all the possible sites considered occupied in this work. Panel b): MP2 energies calculated as dissociation energies $E_d$ for 'trans' (green line/dots) and 'cis' (red line/dots) and relative stability of the two isomers $E_{trans}-E_{cis}$ (magenta line/dots). DFT reference energies as straight lines are  also included.}
%     \label{fig:fragment_553}
% \end{Figure}

% \begin{Figure}
%     \centering
%     % Figure removed
%     \captionof{figure}{Comparison of relative stability $E_{trans}-E_{cis}$ profiles as presented in panel b) while increasing the full system size from a 3- to a 5-layer model as depicted in panel a).}
%     \label{fig:fragment_555}
% \end{Figure}

% \subsection{QRDM-NEVPT2: a new framework for small optimised active space and larger periodic solid-state problems}
 
% The whole procedure started with building the fermionic Hamiltonian by activating only the sub-space (324,310) obtained with the AVAS/RE procedure. On this quantity, we constracted a new smaller Hamiltonian where the only active space was the hardware compatible (3,2) that is equivalent to 2 electrons in a circuit register of 6 qubits (6 spin-orbitals).

During the state preparation, we used a Fermionic VQE-ADAPT algorithm~\cite{Grimsley2019} that progressively builds the ansatz by sequentially incorporating operators that contribute most to lowering the VQE energy. By restricting the operator availability to those allowed in the (k=1)-UpCCGSD ansatz~\cite{Lee2019} we were able to further reduce the circuit depth with respect to the more expensive UCCSD. We also restricted the ADAPT search threshold between $2\times10^{-4}$ and $10^{-3}$ to ensure similar circuit depths for the whole set of experiments.

For all the hardware/noisy emulator data, we made use of the Partition Measurement Symmetry Verification (PMSV) error mitigation procedure~\cite{Yamamoto2022} to suppress spurious symmetries induced by the noise of the device. The symmetry verification was performed by checking that the particle number and spin parity conditions were satisfied by the qubit Hamiltonian.

After optimising the variational parameters of the VQE state entirely on a classical CPU, we measured the expectation value of the active space Hamiltonian as well as the matrix elements of the spin-traced 1- and 2-RDM operators on the quantum hardware as well as on the quantum noisy emulator. We adopted the strategy of dividing the overall Pauli words into mutually commuting sets. Each set defines a measurement circuit and in our case we obtained 12 circuits for the energy and 13 circuits for RDMs for a total of 25 per geometrical structure.

For sake of completeness, in Table~\ref{table:vqe_deviation} and Table~\ref{table:vqe_pt2_deviation} we also provide the energy distance between shot-based noisy experiments and statevector simulations, as $\epsilon (E-E_{SV})$ (also referred as $\epsilon$) for both VQE and PT2 corrected VQE, respectively. Those energy differences represent how results from the noisy device differ from the ideal statevector simulations and they also give an estimate of the quality of the performed measurements. Deviation of $E_a$ and $E_d$ from the statevector are calculated as $\epsilon (E_a) = \epsilon(TS) - \epsilon(\mathrm O_2)$ and $\epsilon (E_d) = \epsilon(2\mathrm O) - \epsilon(\mathrm O_2)$, respectively.

The mean-field and VQE results are summarised in Figure~\ref{fig:full_HF} and Figure~\ref{fig:fragments_cas}. The HF calculations for the full system were performed on the four structures as shown in Figure~\ref{fig:geometries} and evaluated as the energy difference with respect to the initial state (path length=0.0). Compared to the DFT calculations, we found an activation energy larger than the reference (DFT) and well outside the range of values published in literature (light blue area)~\cite{Montemore2018}. Also, the `cis' arrangement was found to be the most stable structural isomer, in contrast to DFT. This is a clear evidence that the atomistic models geometrically relaxed at DFT level of theory exhibits a slightly different Potential Energy Surface (PES) when calculated using the HF mean-field.



% Figure environment removed

% Figure environment removed



From the results presented in Figure~\ref{fig:fragments_cas}, the classical VQE statevector (SV) simulations (panel a) using (2e,3o) active space showed agreement with the HF results  confirming a very low correlation energy associated to the interaction between platinum and oxygen atoms at the adsorption site. The classical noisy experiments on Quantinuum `H1-2E' emulator (panel b) revealed very good agreement with the statevector counterpart and low statistical errors $\sigma \le 0.04$\,eV. However, the `cis' energy deviates by about $\epsilon(E_d) = 0.33$\,eV from SV (calculated from  Table~\ref{table:vqe_deviation}). 

% Figure environment removed


We then, performed experiments on the Quantinuum `H1-2' quantum device for the initial, transition and final `trans' states. Although they confirmed the predicted energy difference between DFT and HF, we found deviation from the SV no larger than $\epsilon(E_a) = 0.1$\,eV (calculated from Table~\ref{table:vqe_deviation}) and a statistical error on the energy average smaller than $\sigma \mathrm{[TS]} = 0.13$\,eV. 

Finally, the results for the corrected energy differences $E_a\mathrm{[TS]}$ and $E_d$[`trans'] by means of QRDM\_NEVPT2 are reported in Figure~\ref{fig:cas_pt2}. We observed that the SV simulations now display the same DFT trend for the relative stability between the two final states with $E_{trans}-E_{cis}\mathrm{[VQE+PT2]} \approx 2*(E_{trans}-E_{cis}\mathrm{[DFT]})$ and  a transition state total energy similar to that of the initial state: $E_a \approx 0.00$\,eV, in contrast with the DFT energy reference (Figure~\ref{fig:cas_pt2}, panel a). 


% Figure environment removed


The noisy emulator and the hardware experiments (panel b and panel c of Figure~\ref{fig:cas_pt2}, respectively) confirmed once again the high quality of the data by matching the SV classical simulations with a deviation smaller than $\epsilon(E_a) = 0.13\,\mathrm{eV}$ and a similar `cis' large deviation around $\epsilon(E_d) = 0.3\,\mathrm{eV}$ for the former and $\epsilon(E_d) = 0.04\,\mathrm{eV}$ for the latter (calculated from Table~\ref{table:vqe_pt2_deviation}). The statistical errors were found no larger than $\sigma(\mathrm{TS}) < 0.38\,\mathrm{eV}$ for 'H1-2E' and $\sigma(\mathrm{TS}) < 0.30$\,eV for `H1-2', respectively.

\subsubsection{Co@Pt Quantum simulations}

In order to obtain the benchmark for our quantum experiments, we performed a series of noiseless adaptive VQE~\cite{Grimsley2019} calculations and progressively increased the active space of the system. The results of this study are summarised in Fig.~\ref{fig:VQE}, where Hartree-Fock results are added as a ``zero-approximation" VQE.

While all the R, TS and P states decreased their energies in accordance with the variational principle, in contrast, their relative stability changed dramatically. Starting from $(8e,7o)$, the activation energy became positive and the driving force increased. They reached the best values with the largest active space in the present work ($(8e,8o)$ adaptive VQE) of $0.293$\,eV for the activation energy and $-1.994$\,eV for the driving force. Compared to Quantum Espresso DFT calculations, the barrier lowered by almost a factor of two, while the driving force increased again by almost a factor of two. The inversion of the trend for the activation energy, while increasing the active space size, suggests the presence of a strong static correlation in the system. From our analysis, it was also clear that most of this correlation could be captured by considering an active space with more fractionally occupied than virtual orbitals.

Given the limitation on the number of qubits, inherent for the actual noisy intermediate-scale quantum (NISQ) computers, we performed the quantum computations on a smaller $(4e,4o)$ active space by using the Quantinuum H1-1E emulator. In particular, we applied the adaptive VQE, followed by NEVPT2, as described in the Methods section. The comparison with the state vector VQE+QRDM\_NEVPT2 calculations for the same active space is also presented in Fig.~\ref{fig:H11E}.
In this case, we have performed a series of $5$ runs of  $10k$ shots each with a calculated standard deviation between $0.05$ eV (for the R state) and $0.22$ eV (for the TS state). 
 
 We found that our H1-1E simulations agree very well with the state vector benchmark for the R and P states. On the contrary, for the TS state emulator results showed a large overestimation of $\epsilon(E_a)=1.416$\,eV from the SV, a sign that its multireference nature could be more susceptible to quantum noise. 



% Figure environment removed
% Figure environment removed


\section{Conclusions}\label{sec3}

We built a realistic testbed model of the rate-limiting step of the oxygen reduction reaction on Pt(111) surfaces. The approach encompassed detailed classical DFT modelling followed by a quantum computing demonstration. 

First, we performed simulations at the DFT level to find a multi-layer slab large enough to  calculate dissociation energies of one O$_2$ molecule in periodic boundary conditions, without significant finite-size effects.  We also studied the oxygen dissociation reaction on a magnetic Co@Pt$(111)$ surface model, regarded as a better catalyst for the ORR at the cathode of membrane fuel cells. To the best of our knowledge, this is the first attempt to consider both surface relaxation and magnetism in the analysis of ORR. 

To leverage the potential of quantum computing, we addressed the strong correlation present in the system by dividing the calculation into two parts. The static correlation was captured using a complete active space approach run on quantum hardware. The dynamic correlation was taken into account using second-order perturbation theory through classical techniques.
For the active space selection, we employed the automatic AVAS/RE method and achieved accurate results with a remarkably small AS for the pure Pt catalyst. 
For the Pt/Co catalyst, despite successfully reproducing the dissociation energy results, a relatively large AS is required to fully capture the correlation energy. This outcome indirectly confirmed the strong correlation inherent in the magnetic core-shell system. This observation underscores this system's potential as an ideal platform for benchmarking and refining quantum computing methods in future research.


Our study paves the way for rigorous ab-initio studies of models of electrocatalysis and highlights the potential of early practical applications of quantum computing for fuel cell modelling. 














\backmatter

%\bmhead{Supplementary information} Here reference to SI file if needed.






\bmhead{Acknowledgments}
The authors thank Duncan Gowland, Gabriel Greene-Diniz and Emanuele Marsili for their feedback on the manuscript. 
The authors also thank Nathaniel Burdick, John Children, Vanya Eccles, Isobel Hooper, Brian Neyenhuis, Jenni Strabley for assistance in the software and hardware experiments. 
The authors performed this work partially using the mat3ra 
platform, a web-based computational ecosystem for the development of new 
materials and chemicals~\cite{Exabyte}. 
% From QUantinuumLtd., we are grateful to The numerical simulations in this work were performed on Microsoft Azure Virtual Machines provided by the program Microsoft for Startups
%Acknowledgments here!

%\section*{Declarations}

%Some journals require declarations!!
% Some journals require declarations to be submitted in a standardised format. Please check the Instructions for Authors of the journal to which you are submitting to see if you need to complete this section. If yes, your manuscript must contain the following sections under the heading `Declarations':

% \begin{itemize}
% \item Funding
% \item Conflict of interest/Competing interests (check journal-specific guidelines for which heading to use)
% \item Ethics approval 
% \item Consent to participate
% \item Consent for publication
% \item Availability of data and materials
% \item Code availability 
% \item Authors' contributions
% \end{itemize}

% \noindent
% If any of the sections are not relevant to your manuscript, please include the heading and write `Not applicable' for that section. 

% %%===================================================%%
% %% For presentation purpose, we have included        %%
% %% \bigskip command. please ignore this.             %%
% %%===================================================%%
% \bigskip
% \begin{flushleft}%
% Editorial Policies for:

% \bigskip\noindent
% Springer journals and proceedings: \url{https://www.springer.com/gp/editorial-policies}

% \bigskip\noindent
% Nature Portfolio journals: \url{https://www.nature.com/nature-research/editorial-policies}

% \bigskip\noindent
% \textit{Scientific Reports}: \url{https://www.nature.com/srep/journal-policies/editorial-policies}

% \bigskip\noindent
% BMC journals: \url{https://www.biomedcentral.com/getpublished/editorial-policies}
% \end{flushleft}

\begin{appendices}

\section{}\label{secA1}



\begin{table}[h]
\begin{center}
% \begin{minipage}{200pt}
\caption{VQE calculated distance error of measured total averaged energies with respect to StateVector ($E_{SV}$) calculations performed using Qulacs backend in eV units. Average by bootstrapping 100k and 10k shots in 10 batches each for Quantinuum H1-2E noisy emulator and H1-2 device measurements, respectively. All the measurements on H1-2E and H1-2 include PMSV error mitigation correction}\label{table:vqe_deviation}%
\begin{tabular}{@{}lrr@{}}
\toprule
& \multicolumn{2}{c}{$\epsilon (E-E_{SV}) \, \mathrm{[eV]}$} \\
\midrule
Species & H1-2E  &  H1-2  \\
\midrule
O$_2$ &  0.14965 &  0.20835 \\
\color{blue} TS & 0.17522  & 0.29947 \\
\color{green} 2O(`trans') &  0.10064  &  0.16303 \\
\color{red} 2O(`cis') &  0.48313 &  - \\
\botrule
\end{tabular}
% \end{minipage}
\end{center}
\end{table}

\begin{table}[h]
\begin{center}
% \begin{minipage}{200pt}
\caption{VQE+PT2 calculated distance error of measured total averaged energies with respect to StateVector ($E_{SV}$) calculations performed using QUlacs backend in eV units. Average by bootstrapping 100k and 10k shots in 10 batches each for Quantinuum H1-2E noisy emulator and H1-2 device measurements, respectively. All the measurements on H1-2E and H1-2 include PMSV error mitigation correction}\label{table:vqe_pt2_deviation}
\begin{tabular}{@{}lrr@{}}
\toprule
& \multicolumn{2}{c}{$\epsilon (E-E_{SV}) \, \mathrm{[eV]}$} \\
\midrule
Species & H1-2E  &  H1-2  \\
\midrule
O$_2$ &  0.04404 &  0.02131 \\
\color{blue} TS &  0.17557 &  $-$0.03943 \\
\color{green} 2O(`trans') &  $-$0.01634 &  0.05915 \\
\color{red} 2O(`cis') &  0.35133 &  - \\
\botrule
\end{tabular}
% \end{minipage}
\end{center}
\end{table}

% Figure environment removed

% An appendix contains supplementary information that is not an essential part of the text itself but which may be helpful in providing a more comprehensive understanding of the research problem or it is information that is too cumbersome to be included in the body of the paper.

%%=============================================%%
%% For submissions to Nature Portfolio Journals %%
%% please use the heading ``Extended Data''.   %%
%%=============================================%%

%%=============================================================%%
%% Sample for another appendix section			       %%
%%=============================================================%%

%% \section{Example of another appendix section}\label{secA2}%
%% Appendices may be used for helpful, supporting or essential material that would otherwise 
%% clutter, break up or be distracting to the text. Appendices can consist of sections, figures, 
%% tables and equations etc.

\end{appendices}

%%===========================================================================================%%
%% If you are submitting to one of the Nature Portfolio journals, using the eJP submission   %%
%% system, please include the references within the manuscript file itself. You may do this  %%
%% by copying the reference list from your .bbl file, paste it into the main manuscript .tex %%
%% file, and delete the associated \verb+\bibliography+ commands.                            %%
%%===========================================================================================%%
\pagebreak

%\newpage
\bibliography{references}% common bib file
%% if required, the content of .bbl file can be included here once bbl is generated
%%\input sn-article.bbl

%% Default %%
%%\input sn-sample-bib.tex%

\end{document}
