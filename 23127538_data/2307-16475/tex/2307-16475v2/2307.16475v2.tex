\documentclass[12pt, leqno]{amsart}

\usepackage[OT2,T1]{fontenc}
\DeclareSymbolFont{cyrletters}{OT2}{wncyr}{m}{n}
\DeclareMathSymbol{\Sha}{\mathalpha}{cyrletters}{"58}

\usepackage{indentfirst}
%\documentstyle[12pt]{amsart}
%\usepackage{amssymb,amsmath}
\usepackage{amstext}
% \usepackage{amsthm}
\usepackage{amsopn}
\usepackage{amsfonts}
\usepackage{amsmath}
\usepackage{latexsym}
\usepackage{amscd}
\usepackage{amssymb}
\usepackage{amsmath}
\usepackage[all,cmtip]{xy}
\usepackage{leftidx}
\usepackage{graphicx}
\usepackage{tikz}

%\usepackage[T2A]{fontenc}
%\usepackage[koi8-r]{inputenc}

\newcommand{\modif}[1]{{\color[rgb]{0,0.7,0}#1}}

%\newcommand{\fref}[1]{Figure~\ref{fig:#1}}

%\textheight     =8.85in
%\textheight     =8in
%\topmargin      =-.1in          % LaTeX uses too much space on top

\textwidth      =6in \oddsidemargin  =.25in \evensidemargin
=\oddsidemargin \font\teneufm=eufm10 \font\seveneufm=eufm7
\font\fiveeufm=eufm5
\newfam\eufmfam
\textfont\eufmfam=\teneufm \scriptfont\eufmfam=\seveneufm
\scriptscriptfont\eufmfam=\fiveeufm
\def\frak#1{{\fam\eufmfam\relax#1}}
\let\goth\mathfrak
\def\cA{\mathcal A}
\def\cB{\mathcal B}
\def\cC{\mathcal C}
\def\cD{\mathcal D}
\def\cF{\mathcal F}
\def\cI{\mathcal I}
\def\cH{\mathcal H}
\def\cI{\mathcal I}
\def\cR{\mathcal R}
\def\cO{\mathcal O}
\def\cT{\mathcal T}
\def\cK{\mathcal K}
\def\cL{\mathcal L}
\def\cE{\mathcal E}
\def\cM{\mathcal M}
\def\cX{\mathcal X}
\def\p{\frak p}
\def\zz{\frak z}
\def\m{\goth m}
\def\gm{\goth m}
\def\hh{\frak h}
\def\ee{\frak e}
\let\bb\mathbb
\def\GG{\mathbb{G}}
\def\NN{\mathbb{N}}
\def\FF{\mathbb{F}}
\def\WW{\mathbf{W}}
\def\VV{\mathbf{V}}
\def\HH{\frak H}
\def\gg{\goth g}
\def\gh{\goth h}
\def\gt{\goth t}
\def\gB{\goth B}
\def\gG{\goth G}
\def\gP{\goth P}
\def\gQ{\goth Q}
\def\gH{\goth H}
\def\gM{\goth M}
\def\gT{\goth T}
\def\gV{\goth V}
\def\gY{\goth Y}
\def\gC{\goth C}
\def\gX{\goth X}
\def\gZ{\goth Z}
\def\gk{\goth k}
\def\gl{\goth l}
\def\gi{\goth i}
\def\gc{\goth c}
\def\gb{\goth b}
\def\gp{\goth p}
\def\gr{\goth r}
\def\gg{\goth g}
\def\gu{\goth u}
\def\gq{\goth q}
\def\gz{\goth z}
\def\go{\goth o}
\def\ga{\goth a}
\def\gd{\goth d}
\def\sl{\goth{s}\goth{l}}
\def\pp{\mbox{\bf p}}
\def\1{\mbox{\bf 1}}
\def\psp{\mbox{\bf sp}}
\def\IP{\Bbb P}
\def\type{\mathrm{\bf type}}
\def\corad{\mathrm{corad}}
\def\rad{\mathrm{rad}}
\def\GR{\mathrm{GR}}
\def\rC{\dot{C}}
\def\sC{\ddot{C}}
% LaTeX begin and end of numbered equation
%

\DeclareMathOperator{\supp}{supp} \DeclareMathOperator{\Span}{Span}
\DeclareMathOperator{\tr}{tr} \DeclareMathOperator{\Hom}{Hom}
\DeclareMathOperator{\Aut}{Aut}
\DeclareMathOperator{\Autext}{Autext}
\DeclareMathOperator{\Int}{Int}
\DeclareMathOperator{\Out}{Out} \DeclareMathOperator{\Der}{Der}
\DeclareMathOperator{\Isom}{Isom}
\DeclareMathOperator{\Isomext}{Isomext}
\DeclareMathOperator{\Isomint}{Isomint}
\DeclareMathOperator{\Stab}{Stab}
\DeclareMathOperator{\im}{im} \DeclareMathOperator{\ch}{ch}
\DeclareMathOperator{\OR}{\bf O}
%\DeclareMathOperator{\char}{char}
\DeclareMathOperator{\End}{End} \DeclareMathOperator{\Id}{Id}
%\DeclareMathOperator{\deg}{deg}
%\DeclareMathOperator{\det}{det}
\DeclareMathOperator{\Sup}{Sup}
\DeclareMathOperator{\Ind}{Ind}
\DeclareMathOperator{\Ad}{Ad}
\DeclareMathOperator{\Spin}{\rm Spin}
\DeclareMathOperator{\Spec}{\rm Spec}
\DeclareMathOperator{\SU}{\rm SU}
\DeclareMathOperator{\SO}{\rm SO}
\DeclareMathOperator{\PGL}{\rm PGL}
\DeclareMathOperator{\GL}{\rm GL}
\DeclareMathOperator{\SL}{\rm SL}
\DeclareMathOperator{\Cliff}{\bf Cliff}
\DeclareMathOperator{\Gr}{Gr}


\newcommand{\incl}[1][r]
{\ar@<-0.2pc>@{^(-}[#1] \ar@<+0.2pc>@{-}[#1]}


\newcommand{\imm}[1][r]
   {\ar@{}[#1] |*[o][F]{\hbox{%
         %\vrule width 1.5mm height 0pt depth 0pt%
         %\vrule width 0pt height .75mm depth .75mm%
         }}
     \ar@{^{(}->}[#1]}


\newcommand{\uA}{{\underline{A}}}
\newcommand{\uB}{{\underline{B}}}
\newcommand{\uC}{{\underline{C}}}
\newcommand{\uE}{{\underline{E}}}
\newcommand{\uF}{{\underline{F}}}
\newcommand{\uG}{{\underline{G}}}
\newcommand{\uH}{{\underline{H}}}
\newcommand{\uM}{{\underline{M}}}
\newcommand{\uN}{{\underline{N}}}
\newcommand{\uQ}{{\underline{Q}}}
\newcommand{\uR}{{\underline{R}}}
\newcommand{\uX}{{\underline{X}}}
\newcommand{\uY}{{\underline{Y}}}
\newcommand{\uW}{{\underline{W}}}

\newcommand{\up}{{\underline{p}}}
\newcommand{\uup}{\underline{\underline{p}}}
\newcommand{\ihom}{{\underline{\rm Hom }}}
\newcommand{\inj}{\hookrightarrow}

\newcommand{\ru}{{\mathcal R}_{u,k}} % k-unipotent radical

\newcommand{\ra}{{\mathcal R}_k} % k-radical

\newcommand{\rs}{{\mathcal R}_{s,k}} % k-split  radical

\newcommand{\rus}{{\mathcal R}_{us, k}} % k-split unipotent radical

%*******
\newtheorem{theorem}{Theorem}[subsection]%Theorems et al in italic font.
\newtheorem{acknowledgement}[theorem]{Acknowledgement}
\newtheorem{algorithm}[theorem]{Algorithm}
\newtheorem{axiom}[theorem]{Axiom}
\newtheorem{case}[theorem]{Case}
\newtheorem{claim}[theorem]{Claim}
\newtheorem{conclusion}[theorem]{Conclusion}
\newtheorem{condition}[theorem]{Condition}
\newtheorem{conjecture}[theorem]{Conjecture}
\newtheorem{corollary}[theorem]{Corollary}
\newtheorem{criterion}[theorem]{Criterion}
%\newtheorem{definition}[theorem]{Definition}
\newtheorem{exercise}[theorem]{Exercise}
\newtheorem{lemma}[theorem]{Lemma}
%\newtheorem{notation}[theorem]{Notation}
\newtheorem{problem}[theorem]{Problem}
\newtheorem{proposition}[theorem]{Proposition}

\newtheorem{stheorem}{Theorem}[section]%Theorems et al in italic font.
\newtheorem{sclaim}[stheorem]{Claim}
\newtheorem{sconjecture}[stheorem]{Conjecture}
\newtheorem{scorollary}[stheorem]{Corollary}
\newtheorem{slemma}[stheorem]{Lemma}
\newtheorem{sproposition}[stheorem]{Proposition}
\newtheorem{sremark}[stheorem]{Remark}
\newtheorem{sremarks}[stheorem]{Remarks}
\newtheorem{sexample}[stheorem]{Example}
\newtheorem{sexamples}[stheorem]{Examples}
\newtheorem{sdefinition}[stheorem]{Definition}

%\theorembodyfont{\upshape}
%\newtheorem{remark}[theorem]{Remark}
%\newtheorem{example}[theorem]{Example}
\newtheorem{question}[theorem]{Question}
\newtheorem{solution}[theorem]{Solution}
\newtheorem{summary}[theorem]{Summary}
%\newenvironment{proof}[1][Proof]{\noindent\textbf{#1.} }{\ \rule{0.5em}{0.5em}}
\newtheorem*{assumption}{Assumption} %Added BNA

\theoremstyle{definition}%Theorems et al in roman font.
\newtheorem{remark}[theorem]{Remark}
\newtheorem{remarks}[theorem]{Remarks}


\newtheorem{example}[theorem]{Example}
\newtheorem{examples}[theorem]{Examples}
\newtheorem{notation}[theorem]{Notation}
\newtheorem{definition}[theorem]{Definition}
\newtheorem{construction}[theorem]{Construction}

\numberwithin{equation}{section}

%*******
\def\NN{\mathbb{N}}
\def\ZZ{\mathbb{Z}}
\def\CC{\mathbb{C}}
\def\PP{\mathbb{P}}

\def\gE{\mathfrak{E}}
\def\gF{\mathfrak{F}}
\def\gG{\mathfrak{G}}
\def\gJ{\mathfrak{J}}
\def\gP{\mathfrak{P}}
\def\gQ{\mathfrak{Q}}
\def\gL{\mathfrak{L}}
\def\gM{\mathfrak{M}}
\def\gU{\mathfrak{U}}
\def\gS{\mathfrak{S}}
\def\Par{\mathrm{Par}}


\def\QQ{\mathbb{Q}}
%\def\P{\Bbb P}
%\def\sv{{\sl v}}
\def\Z{\mathbb Z}
\def\G{\mathbb G}
\def\A{\mathbb A}
\def\C{\mathbb C}
\def\R{\mathbb R}
\def\H{\mathbb H}
\def\bB{\text{\rm \bf B}}
\def\bC{\text{\rm \bf C}}
\def\bE{\text{\rm \bf E}}
\def\bF{\text{\rm \bf F}}
\def\bH{\text{\rm \bf H}}
\def\bG{\text{\rm \bf G}}
\def\bN{\text{\rm \bf N}}
\def\bU{\text{\rm \bf U}}
\def\bW{\text{\rm \bf W}}
\def\bX{\text{\rm \bf X}}
\def\Mor{\text{\rm Mor}}
\def\isotr{\text{\rm isotr}}
\def\iso{\text{\rm iso}}
\def\bZ{\rm \bf{Z}}

\def\pr{\prime}
\def\lra{\longrightarrow}
\def\Bun{\text{\rm Bun}}
\def\Ctd{\text{\rm Ctd}}
\def\Mult{\text{\rm Mult}}
\def\rAut{\text{\rm Aut}}
\def\rEnd{\text{\rm Endt}}
\def\fet{\text{\rm f\'et}}
\def\red{\text{\rm red}}
\def\ad{\text{\rm ad}}
\def\bGL{\text{\rm \bf GL}}
\def\bPGL{\text{\rm \bf PGL}}
\def\bSL{\text{\rm \bf SL}}
\def\bT{\text{\rm \bf T}}
\def\bS{\text{\rm \bf S}}
\def\bL{\text{\rm \bf L}}
\def\cL{\mathcal{L}}
\def\cO{\mathcal{O}}
\def\bQ{\text{\rm \bf Q}}
\def\bR{\text{\rm \bf R}}
\def\bP{\text{\rm \bf P}}
\def\bY{\text{\rm \bf Y}}
\def\bK{\text{\rm \bf K}}
\def\P{\mathbb P}
\def\X{\text{\rm \bf X}}
\def\ba{\pmb a}
\def\be{\mathbf e}
\def\bg{{\bf g}}
\def\bm{\mathbf m}
\def\bn{\mathbf n}
\def\bt{\mathbf t}
\def\bk{\mathbf k}
\def\bs{{\pmb\sigma}}
\def\bx{{\pmb x}}
\def\by{{\pmb y}}
\def\bt{{\pmb\tau }}
\def\bmu{{\pmb\mu }}
\def\wh{\widehat}
\def\wt{\widetilde}
\def\us{\underset}
\def\os{\overset}
\def\ol{\overline}
\def\id{\text{\rm id}}
\def\fppf{\text{\rm fppf}}
\def\Zar{\text{\rm Zar}}
\def\ct{\text{\rm ct}}
\def\et{\text{\rm \'et}}
%\def\loop{\text{\rm loop}}
\def\finite{\text{\rm finite}}
\def\alg{\text{\rm alg}}
\def\mod{\text{\rm mod}}
\def\q{\quad}
\def\vs{\vskip.3cm}
\def\vsk{\vskip.5cm}
\def\noi{\noindent}
\def\Lie{\mathop{\rm Lie}\nolimits}
\def\idotsint{\mathop{\int\cdots\int}\nolimits}
\def\sumsum{\mathop{\sum\sum}\nolimits}
\def\limsup{\mathop{\lim\sup}\nolimits}
\def\2int{\mathop{2\int}\nolimits}
\def\Dyn{\mathrm{\bf Dyn}}
\def\uDyn{\underline{\mathrm{Dyn}}}
\def\Trans{\mathrm{Trans}}
\def\Transpt{\mathrm{Transpt}}
\def\uPsi{\underline{\Psi}}

\def\uDelta{\underline{\text{\bf Delta}}}
\def\Div{\mathop{\rm Div}\nolimits}
\def\End{\mathop{\rm  End}\nolimits}
\def\rank{\mathop{\rm rank}\nolimits}
\def\dim{\mathop{\rm dim}\nolimits}
\def\codim{\mathop{\rm codim}\nolimits}
\def\Spec{\mathop{\rm Spec}\nolimits}
\def\Sup{\mathop{\rm Sup}\nolimits}
\def\Lie{\mathop{\rm Lie}\nolimits}
\def\Ext{\mathop{\rm Ext}\nolimits}
\def\Hom{\mathop{\rm Hom}\nolimits}
\def\Der{\mathop{\rm Der}\nolimits}
\def\Stab{\mathop{\rm Stab}\nolimits}
\def\Inf{\mathop{\rm Inf}\nolimits}
\def\Ind{\mathop{\rm Ind}\nolimits}
\def\mod{\mathop{\rm mod}\nolimits}

\def\link{\mathop{\rm link}\nolimits}
\def\Mod{\mathop{\rm Mod}\nolimits}
\def\Gal{\mathop{\rm Gal}\nolimits}
\newcommand{\Hilb}{{\rm Hilb}}
\def\Irr{\mathop{\rm Irr}\nolimits}
\def\Int{\mathop{\rm Int}\nolimits}
\def\Div{\mathop{\rm Div}\nolimits}
\def\car{\mathop{\rm car}\nolimits}
\def\Pic{\mathop{\rm Pic}\nolimits}
\def\Mat{\mathop{\rm M}\nolimits}
\def\Ind{\mathop{\rm Ind}\nolimits}
\def\ord{\mathop{\rm ord}\nolimits}
\def\Coind{\mathop{\rm Coind}\nolimits}
\def\Br{\mathop{\rm Br}\nolimits}
\def\Aut{\mathrm{Aut}}
\def\uAut{\underline{\Aut}}
\def\Autext{\mathrm{Autext}}
\def\uAutext{\underline{\Autext}}
\def\Isom{\mathrm{Isom}}
\def\uIsom{\underline{\Isom}}
\def\Isomext{\mathrm{Isomext}}
\def\uIsomext{\underline{\Isomext}}

\def\uStab{\underline{\Stab}}

\def\Isomint{\mathrm{Isomint}}
\def\uIsomint{\underline{\Isomint}}

\def\Transp{\mathrm{Transp}}
\def\uTransp{\underline{\Transp}}

\def\Centr{\mathrm{Centr}}
\def\uCentr{\underline{\Centr}}

\def\Out{\text{\rm{Out}}}
\def\bAut{\text{\bf{Aut}}}
\def\bOut{\text{\bf{Out}}}
\def\Par{\text{\rm{Par}}}
\def\Int{\mathop{\rm Int}\nolimits}
\def\Rep{\mathop{\rm Rep}\nolimits}
\def\CRIS{\mathop{\rm CRIS}\nolimits}
\def\Isom{\mathop{\rm Isom}\nolimits}
\def\Isomext{\mathop{\rm Isomext}\nolimits}
\def\nrd{\mathop{\rm nrd}\nolimits}
\def\trd{\mathop{\rm trd}\nolimits}


\def\Frob{\mathop{\rm Frob}\nolimits}
\def\resp.{\mathop{\rm resp.}\nolimits}
\def\limproj{\mathop{\oalign{lim\cr
\hidewidth$\longleftarrow$\hidewidth\cr}}}
\def\limind{\mathop{\oalign{lim\cr
\hidewidth$\longrightarrow$\hidewidth\cr}}}
\def\Ker{\mathop{\rm Ker}\nolimits}
\def\proj{\mathop{\rm proj}\nolimits}
\def\Im{\mathop{\rm Im}\nolimits}
\def\Cok{\mathop{\rm Cok}\nolimits}
\def\Res{\mathop{\rm Res}\nolimits}
\def\Cor{\mathop{\rm Cor}\nolimits}
\def\bfmu{\hbox{\bfmath\char'26}}
\def\lgr{\longrightarrow}
\def\la{\longleftarrow}
\def\rg{\rightarrow}
\def\lmp{\longmapsto}
\def\card{\mathop{\rm card}\nolimits}
\font\math=cmmi10
\def\varpi{\hbox{\math\char'44}}


\def\simlgr{\buildrel\sim\over\lgr}
\def\simla{\buildrel\sim\over\la}
\def\Coker{\mathop{\rm Coker}\nolimits}
\def\Sym{\mathop{\rm Sym}\nolimits}
\def\Jac{\mathop{\rm Jac}\nolimits}
\def\pa{\S\kern.15em }
\def\tvi{\vrule height 10pt depth 5pt width 0pt}
\def\tv{\tvi\vrule}

\def\E{E^\times}
\def\tr{\mathop{\rm tr}\nolimits}
%\def\rtimes{\mathbin{{>}\!{\triangleleft}}}
\def\un{\uppercase\expandafter{\romannumeral 1}}
\def\deux{\uppercase\expandafter{\romannumeral 2}}
\def\trois{\uppercase\expandafter{\romannumeral 3}}
\def\quatre{\uppercase\expandafter{\romannumeral 4}}
\def\cinq{\uppercase\expandafter{\romannumeral 5}}
\def\six{\uppercase\expandafter{\romannumeral 6}}
\def\eskip{ \hskip .2 em}
\def\cskip{ \hskip -0.6 em}
\def\dskip{ \hskip -0.8 em}
\def\ccskip{ \hskip -1.2 em}
\def\gg{\goth g}
\def\et{\acute et}


\def\kalg{k\text{--}alg}


\title[Homogeneous spaces]{Oriented embedding functors of tori as homogeneous spaces }



\date{\today}


\author{Philippe Gille}

\address[]{P. Gille, Institut Camille Jordan - Universit\'e Claude Bernard Lyon 1
43 boulevard du 11 novembre 1918,
69622 Villeurbanne cedex - France }
\email{gille@math.univ-lyon1.fr}

\author{Ting-Yu Lee}
\address[]{T.-Y. Lee, Astronomy Mathematics Building 5F, No. 1, Sec. 4, Roosevelt Rd., Taipei 10617, Taiwan.
}
\email{tingyulee@ntu.edu.tw}



\begin{document}



 \begin{abstract} We provide a characterization of homogeneous spaces under a reductive group 
 scheme such that the geometric stabilizers are maximal tori. 
 The quasi-split case over a semilocal base (and more generally over a LG-ring)
 is of special 
 interest and permits to   answer a  question raised by Marc Levine 
 on $\SL_n$-homogeneous spaces. At the end, we provide an application to the local-global principles for embeddings of \'etale algebras with involution into central simple algebras with involution.
 
\end{abstract}



\maketitle

\bigskip


\noindent {\em Keywords:} Reductive group schemes,  tori,  torsors, homogeneous spaces, embeddings, quasi-split groups, local-global principles, \'etale algebras, central simple algebras.



\smallskip

\noindent {\em MSC 2000: 14L15, 20G35}


\bigskip


\bigskip

\section{Introduction}
Let $k$ be a field and  let $k_s$ be a separable closure of $k$;
we denote by $\Gamma_k=\Gal(k_s/k)$ the absolute Galois group of $k$.
Let $X$ be an affine $k$--variety 
that is a $\SL_n$-homogeneous space such that the stabilizer
of a geometric point is a maximal torus. Marc Levine asked whether 
$X$ is $G$-isomorphic to $\SL_n/T$ where $T$ is a maximal torus of $\SL_n$.

Our main classification result (Theorem \ref{thm_main})
tells us that $X$ is the variety\footnote{Those varieties (and their generalizations over a base) have been defined by the second author \cite{L1}.} of embeddings of a suitable oriented torus $T$ in $\SL_n$ and 
we explain now the meaning of $X(k)$.
Denoting by $\GG_m^{n,1}$ the standard split torus
of $\SL_n$, we remind  the reader that the automorphism
group of the root system $A_{n-1}=\Phi(\SL_{n}, \GG_{m}^{n,1})$
is $S_n \times \ZZ/2\ZZ$ if $n \geq 3$ and $\ZZ/2\ZZ$ for $n=2$.
We have that $T=R^1_{A/k}(\GG_m)$
for an \'etale $k$--algebra $A$ 
(see \cite[\S 7.5]{G}) and that $X(k)$ corresponds to the  set of embeddings $f: T \to \SL_n$ such that the class of the 
$S_n \times \ZZ/2\ZZ$--torsor $\mathrm{Isom}\Bigl(\Phi(\SL_{n,k_s}, \GG_{m,k_s}^{n,1}),\Phi(\SL_{n,k_s}, f(T)_{k_s})\Bigr)$ is $[A] \times 1$
if $n\geq 3$ and simply $[A]$ if $n=2$.
We used implicitly there that the Galois cohomology set $H^1(k,S_n)$ classifies isomorphism classes of 
\'etale algebras of degree $n$
and that the oriented type of $T$ is 
the class of the above torsor.

On the other hand, if we see $A$ as a $k$-vector space of dimension $n$,
we have an isomorphism $\SL(A) \simlgr \SL_n$
and the natural embedding $T = R^1_{A/k}(\GG_m) \hookrightarrow \SL_1(A)$
defines then an embedding $T \to \SL_n$ of oriented type $[A] \times 1$.
In other words, the $\SL_n$-homogeneous space $X$ carries a $k$--point whose stabilizer is $T = R^1_{A/k}(\GG_m)$. Thus $X$ is $\SL_n$-isomorphic
to $\SL_n/T$ as desired.

This statement can be strengthened in two 
ways. 
First we can replace $\SL_n$ by any quasi-split reductive
$k$-group, the presence of $k$-points on a such 
homogeneous space being the Gille-Raghunathan's theorem
\cite{Gi2004,Rg}.

With more effort, we can replace further the base field $k$ 
by an arbitrary  LG ring
(with  infinite residue fields).
 This is Corollary   \ref{cor_main} which involves
 a generalization of  the existence of maximal tori of any orientation in such group schemes (see Theorem  \ref{thm_semi_local}). 
 As in Steinberg's section theorem, the case of type $A_{2n}$
 requires an additional argument.
Also this generalization needs the notion of versal torsor
in that setting (Reichstein-Tossici \cite{RT});
we introduced a variation of this technique involving 
algebraic spaces (see \S \ref{subsec_versal}).
 
Finally we provide an arithmetic application to the embeddings of \'etale algebras with involution into central simple algebras with involution.
Let $(E,\sigma)$ be an \'etale algebra $E$  over $k$ with involution $\sigma$ and $(A,\tau)$ be a central simple algebra $A$ over $k$ with involution $\tau$.
We want to know when $(E,\sigma)$ can be embedded into $(A,\tau)$ under some constraints on the dimensions of $(E,\sigma)$ and $(A,\tau)$ (see \cite{PR1}, \cite{BLP1}, \cite{BLP2}).
In Theorem \ref{embed_algebras}, we show that when $k$ is a number field and the unitary group associated to $(A,\tau)$ is quasi-split, the local-global principle always holds for the embeddings of 
$(E,\sigma)$ into $(A,\tau)$.


 
 \medskip
 


\noindent{\bf Acknowledgments.} We thank Marc Levine for useful discussions and 
Skip Garibaldi for his suggestion to deal with LG rings. We thank the Camille Jordan  Institute for inviting  the second author.

\vskip1cm

\noindent{\bf Notation.}
We use mainly  the terminology and notations of Grothendieck-Dieudonn\'e \cite[\S 9.4  and 9.6]{EGA1}
 which agree with that  of Demazure-Grothendieck used in \cite[Exp.\ I.4]{SGA3}

(a) Let $S$ be a scheme and let $\cE$ be a quasi-coherent sheaf over $S$.
 For each morphism  $f:T \to S$,
we denote by $\cE_{T}=f^*(\cE)$ the inverse image of $\cE$
by the morphism $f$.
 We denote by $\VV(\cE)$ the affine $S$--scheme defined by
$\VV(\cE)=\Spec\bigl( \mathrm{Sym}^\bullet(\cE)\bigr)$;
it is affine  over $S$ and
represents the $S$--functor $Y \mapsto \Hom_{\cO_Y}(\cE_{Y}, \cO_Y)$
\cite[9.4.9]{EGA1}.
%This construction generalizes to algebraic spaces.

\smallskip

(b) We assume now that $\cE$ is locally free of finite rank and denote by $\cE^\vee$ its dual.
In this case the affine $S$--scheme $\VV(\cE)$ is  of finite presentation
(ibid, 9.4.11); also
the $S$--functor $Y \mapsto H^0(Y, \cE_{Y})=
\Hom_{\cO_Y}(\cO_Y, \cE_{Y} )$
is representable by the  affine $S$--scheme $\VV(\cE^\vee)$
which is also denoted by  $\WW(\cE)$  \cite[I.4.6]{SGA3}.
 %Once again this construction generalizes to algebraic spaces.

It applies to the locally free coherent sheaf
${\cE}nd(\cE) = \cE^\vee \otimes_{\cO_S} \cE$
 over $S$ so that we can consider
the affine $S$--scheme $\VV\bigl({\cE}nd(\cE)\bigr)$
that is an $S$--functor in associative commutative and unital algebras
\cite[9.6.2]{EGA1}.
Now we consider the $S$--functor $Y \mapsto \Aut_{\cO_Y}(\cE_{Y})$.
It is representable by an open $S$--subscheme of $\VV\bigl({\cE}nd(\cE)\bigr)$
which is denoted by $\GL(\cE)$ ({\it loc. cit.}, 9.6.4).



\section{Generalities on homogeneous spaces}

Let $S$ be a scheme and let $\bullet$ be the final object of the
category of fppf $S$-sheaves of groups, that is
$\bullet(T)= \Hom_{S-sch}(T,S)=h_S(T)$ for each $S$--scheme $T$.

 Let $\uG$ be an fppf $S$-sheaf of groups.
Let $\uX$ be an fppf $S$-sheaf equipped with  a left action of  $\uG$.
We say that $\uX$ is homogeneous  (resp.\, principal homogeneous) under $\uG$ if
the map $\uX \to \bullet$ is an epimorphism of fppf $S$-sheaves
and if the action map $\uG \times_S \uX \to \uX \times_S \uX$,
$(g,x) \to (x,g.x)$ is an  epimorphism (resp.\ an isomorphism) of fppf $S$-sheaves.
Similarly we have the notion of right homogeneous (resp.\, principal homogeneous) spaces.


Let $\uE$ be a right principal homogeneous  $\uG$-space and denote by $\uG'$ the twist of
$\uG$ by $\uE$ through inner automorphisms.
Then twisting by $\uE$ gives rise to an equivalence of categories
between the category of left homogeneous $\uG$--spaces and
that of  left homogeneous $\uG'$--spaces.




\iffalse
\begin{slemma}\label{lem_homogeneous}
Let $\uX$ be an fppf $S$-sheaf equipped with  a left  $\uG$-action.
Then the following are equivalent:

\smallskip

(i) $\uX$ is a homogeneous space under $\uG$;

\smallskip

(ii) There exists an fppf cover $(S_i)_{i \in I}$ of $S$ and  $S_i$-group subsheaf  $\uH_i \subset \uG \times_S S_i$
such that $\uX \times_S S_i$ is
isomorphic to the fppf quotient sheaf $(\uG \times_S S_i) / \uH_i$ with respect to the $\uG\times_S{S_i}$-action for all $i \in I$.

In particular, if $\uH$ is an fppf $S$--subsheaf of groups of $\uG$, then
the fppf quotient $\uG/\uH$ is a $\uG$--homogeneous space.
\end{slemma}

The $\uH_i$ are called the {\it local stabilizers}.

\smallskip


\begin{proof}

\noindent
$(i) \Rightarrow (ii).$ Since $\uX$ is a homogeneous $\uG$-space, the map $\uX\to\bullet$ and the action map $\uG\times_S\uX\to\uX\times_S\uX$ are epimorphisms  of fppf $S$-sheaves.
Hence there is an fppf cover $(S_i)_{i\in I}$ of $S$ such that $\uX(S_i)$ is nonempty and the action map $(\uG{\times_S}\uX)_{S_i}\to (\uX{\times}_S\uX)_{S_i}$ is surjective.

Choose $x\in \uX(S_i)$. Define the map $i_x:\uG\times_S S_i\to \uX\times_S{S_i}$ as follows.
For an $S_i$-scheme $U_i$ and $g\in\uG(U_i)\times S_i(U_i)$, we define $i_x(g)=g\cdot x.$
This map is surjective as $(\uG{\times_S}\uX)_{S_i}\to (\uX{\times}_S\uX)_{S_i}$ is surjective.
Denote the sheaf of stabilizers of $x$ by $\uH_i$.
Then $\uH_i$ is an $S_i$-group subsheaf of $\uG\times_S S_i$, and $i_x$ induces an isomorphism of sheaves between $(\uG\times_S S_i)/\uH_i$ and $\uX\times_S S_i$. 

\noindent
$(ii)\Rightarrow (i).$  Set $\uQ_i=(\uG\times_S S_i)/\uH_i$.
As $\uG(S)$ has the neutral element,  $\uQ_i\to\bullet$ is surjective.
For an $S_i$-scheme $U_i$ and $x_1$, $x_2\in \uQ_i(U_i)$, after localization we may assume that $x_1$, $x_2$ come from some $g_1$, $g_2\in \uG(U_i)$ respectively.
Clearly $g_2g_1^{-1}\cdot x_1=x_2$. Hence the action map $(\uG\times_S S_i)\times_{S_i}\uQ_i\to \uQ_i\times_{S_i} \uQ_i$ is surjective. 
As $\uX\times_S S_i$ is isomorphic to  $\uQ_i$ with respect to the $\uG\times_S S_i$-action for all $i$,
$\uX$ is a $\uG$-homogeneous space.
\end{proof}

\fi

 
\begin{sremark}{\rm For the discussion of the same material in  the representable case, see \cite[VI.1]{R}.
 }
\end{sremark}



\begin{slemma}\label{lem_quotient} Let $\uH$ be an fppf group subsheaf of $\uG$ and $\uX$ be the $\uG$-homogeneous space $\uG/\uH$.  Consider
the $S$--sheaf of groups $\uA=  \uAut_\uG(\uX)$ that acts on the left of $\uX$.

\smallskip

\noindent (1)  Let $\uN=\uN_\uG(\uH)$ be the normalizer of $\uH$ in $\uG$ (as in
\cite[\S 2.3]{Gi2015}).
Then the map $\uG \times_S \uN \to \uG$, $(g,n) \mapsto g n$
 induces an isomorphism  $\uN/\uH \simlgr \uA^{op}$.
%Furthermore $\uA$ acts freely on $\uX$ and
%we have a canonical isomorphism $\uG/ \uN  \simlgr \uX/\uA^{op}$.

\smallskip

\noindent (2) The action of $\uA$  on $\uX$ is free and the map $\uX \to \uA\backslash\uX$
is a (left) principal $\uA$--homogeneous space.

\smallskip

\noindent (3) Let $\uF$ be a right $\uA$-torsor and let $\uX'$ be the twist of $\uX$ by $\uF$.
Then $X'$ is a $G$-homogeneous space and  there is a canonical isomorphism $\uA'\backslash\uX'  \simlgr  \uA\backslash\uX$, where $\uA'=\uAut_\uG(X')$.

\smallskip

\noindent (4) The $S$--forms of $\uX$ (as $\uG$-spaces) are classified by $H^1_{right}(S,  \uA)$.


\end{slemma}

\begin{proof}
We put $\uW=\uN/\uH$. 

\noindent{(1)} The map  $\uG \times_S \uN \to \uG$
induces a map $\uX \times \uW \to \uX$ hence a group homomorphism
$\phi: \uW \to \uA^{op}$ of $S$-sheaves of groups.

\smallskip

\noindent{\it $\phi$ is a monomorphism.} Let $T$ be a $S$-scheme and let
$w \in \uW(T)$ be an element such that $\phi(w)=1$. Up to localization we can assume
that $w$ arises from an element $n \in \uN(T)$ which satisfies $1 \times n \in \uH(T)$.
Thus $w=1$.


\smallskip 

\noindent{\it $\phi$ is an epimorphism.} Let $T$ be an $S$--scheme and let
$f \in \uA(T)$.  
Denoting by $[1]$ the $S$-point
$S \xrightarrow{e_G} \uG \xrightarrow{q} \uX=\uG/\uH$
 we put  $x=f([1]) \in \uX(T)$ and up to localization we can assume
that $x$ arises from some $g \in \uG(T)$. We claim that $g \in \uN(T)$.
For each $T$-scheme $T'$ and   each $g' \in \uG(T)$  we have
$f( g'\cdot [1])= g' \cdot g_{T'}\cdot [1]$.
In particular for each $h' \in \uH(T')$, we have $h' \, g_{T'} \in g_{T'} \uH(T')$
so that $g_{T'}$ normalizes $\uH(T')$. The claim is proven
and it follows that $f= \phi(w)$ where $w$ is the image of $g \in \uN(T)$ in $\uW(T)$.

\smallskip


\noindent (2)We start by establishing that 
$\uA$ acts freely on $\uX$.
By (1), it is equivalent to prove that $\uW$ acts freely on $\uX$.
Let $T$ be a scheme and $x\in \uX(T)$.
After localization we can assume that $x$ arises from an element $g\in\uG(T)$.
Consider the $T$-sheaf of groups $\uStab_{\uW}(x)$.
For each $T$-scheme $T'$ and $w\in \uStab_{\uW}(x)(T')$, up to localization we can assume that $w$ comes from an element $n\in \uN(T')$.
Then $[g]= [g]\cdot w=[gn]$, that implies $n\in \uH(T')$ and $w$ is the identity in $\uW(T')$.
Hence $\uW$ acts freely on $\uX$.
 % and $\uX\to\uA\backslash\uX$ is a right torsor of $\uW$.
We put $\uY=\uA\backslash\uX$.
In view of \cite[III.3.1.2]{Gir},
the map $\uX\to \uY$ satisfies then  the first item
of the definition of $\uA$-torsor, that is,
the map  $\uA_{\uY}  \times_{\uY} \uX \to  
\uX \times_{\uY} \uX$, $(a,x) \mapsto (x,ax)$,  is an isomorphism.
Since $\uX \to \uY$ admits locally sections for the fppf topology,
the second requirement is satisfied as well \cite[definition III.4.1]{Gir}.
Thus $\uX\to \uY$ is a left $\uA$-torsor.


\smallskip


\noindent (3)
Let $p$ be the projection from $\uX$ to the quotient sheaf $\uA\backslash\uX$.
Consider the map $\pi:\uF\times \uX\to \uA\backslash \uX$, which projects $\uF\times \uX$ to $\uX$ and then to the quotient sheaf $\uA\backslash \uX$.
Then clearly $\pi$ induces a map from $\uX'$ to $\uA\backslash \uX$, which we still denote by $\pi$.

Note that $\uG$ acts on the $\uX$-factor of $\uF\times\uX$. As the $G$-action and the $A$-action commutes on $X$,  this defines an $\uG$-action on $\uX'$.
Choose a $fppf$-cover $\{S_i\}$of $S$ that trivializes the $\uA$-torsor $\uF$.
For each $S_i$, there is an isomorphism of right $\uA$-torsors between $\uF_{S_i}$ and $\uA_{S_i}$.
Hence $X'_{S_i}\simeq X_{S_i}$ and $\uA'_{S_i}\simeq \uA_{S_i}$.

Fix an isomorphism of $\uA_{S_i}$-torsors $\iota_i: \uF_{S_i}\to \uA_{S_i}$.
The map $\iota_i$ induces an isomorphism between $\uX'_{S_i}$ and $\uX_{S_i}$ that sends $[f, x]$ to $\iota_i(f)(x)$ for $x\in \uX(T_i)$, $f\in\uF(T_i)$ and for arbitrary $S_i$-scheme $T_i.$
We denote this induced isomorphism still by $\iota_i$.
The map $\iota_i$ induces an isomorphism between $\uA'_{S_i}$ and $\uA_{S_i}$ by sending 
$a'$ to $\iota_i\circ a'\circ \iota_i^{-1}$ for $a'\in\uA'(T_i)$ and $T_i$ an $S_i$-scheme.
Then clearly $\iota_i$ induces an isomorphism $\ol{\iota}_i: (\uA'\backslash\uX')_{S_i}\to(\uA\backslash\uX)_{S_i}$.

One checks that $p\circ \iota_i=\pi_{S_i}$.
Hence $\ol{\iota}_i\circ p'=\pi_{S_i}$ where $p'$ is the projection from $\uX'$ to $\uA'\backslash\uX'$.
This implies that $\pi(x')=\pi(a'\cdot x')$ for $x'\in \uX'(T)$, $a'\in\uA'(T)$ and for arbitrary $S$-scheme $T$.
Hence $\pi$ induces a map $\ol{\pi}$ from $\uA'\backslash\uX'$ to $\uA\backslash \uX$.
On each $S_i$,  we have $\ol{\pi}_{S_i}=\ol{\iota}_i$. As $\ol{\iota}_i$ is an isomorphism,
$\ol{\pi}$ is an isomorphism between $\uA'\backslash\uX'$ and $\uA\backslash\uX$.

\smallskip

\noindent (4) Let $X'$ be an $S$-form of $X$ as $\uG$-spaces.
Then the sheaf of isomorphisms $\uIsom_\uG(X,X')$ is a right $\uA$-torsor and
the contracted product $\uIsom_{\uG}({\uX,\uX'})\land^\uA\uX$ 
(as defined in \cite[\S III.1.3]{Gir}) is canonically isomorphic to $\uX'$.

Conversely given a right $\uA$-torsor $\uF$, let $\uX'$  be the twist of $\uX$ by $\uF$.
According to the proof of  (3) , the twist $\uX'$ is a $G$-homogeneous space and is isomorphic to $\uX$ $fppf$-locally.
Hence $\uX'$ is an $S$-form of $\uX$.
We claim that the sheaf of isomorphisms $\uIsom(X, X')$ is isomorphic to $\uF$.

Note that  $\uA\land^{\uA} \uX$ is canonically isomorphic to $\uX$ and 
the $\uA$-action on $\uX$ corresponds to the $\uA$-action on the $\uA$-factor of $\uA\land^{\uA} \uX$.
Regard $\uA$ as a right $\uA$-torsor.
Then there is a natural map $\iota$ between $\uIsom_{\uA}(\uA,\uF)$ and $\uIsom_{\uG}(\uX,\uX')=\uIsom_{\uG}(\uA\land^\uA\uX,\uF\land^\uA\uX')$.
To be precise, for an $S$-scheme $S'$ and $\phi\in\uIsom_\uA(\uA,\uF)(S')$, $\iota(\phi)([(a,x)])=[(\phi(a),x)]$ for any $S'$-scheme $S''$ and $[(a,x)]\in (\uA\land^\uA\uX)(S'')$.


The sheaf of automorphisms of $\uA$ (as a right $\uA$-torsor) is $\uA$ itself which acts on the left of $\uA$.
As $\uIsom_{\uA}(\uA,\uF)$ is a right $\uAut_\uA(\uA)$-torsor, it is  a right $\uA$-torsor. 
It is easy to check that  $\iota$ is compatible with the $\uA$-action. 
Since $\uIsom_{\uA}(\uA,\uF)$ and $\uIsom_{\uG}(\uX,\uX')$ are both right $\uA$-torsors, and $\iota$ is compatible with the $\uA$-action, $\iota$ is an isomorphism.

There is a natural map $i$ from $\uIsom_\uA(\uA,\uF)$ to $\uF$ that sends $\phi\in \uIsom_\uA(\uA,\uF)(S')$ to $\phi(1_\uA)\in\uF(S')$ for all $S$-schemes $S'$.
Since $i$ is compatible the right $\uA$-action, $i$ is an isomorphism between right $\uA$-torsors, and $i\circ\iota^{-1}$ gives the desired isomorphism between $\uIsom_\uG(\uX,\uX')$ and $\uF$. 
Therefore the map that sends an $S$-form $\uX'$ to the right $\uA$-torsor $\uIsom_{\uG}(\uX,\uX')$  is a bijection.
\end{proof}



\section{Embedding functors}
Let $S$ be a scheme. 
For a point $s\in S$, let $\kappa(s)$ be the residue field of $s$
and $\ol{\kappa(s)}$ be the algebraic closure of $\kappa(s)$. Let
$\ol{s}$ be the scheme $\Spec(\ol{\kappa(s)})$. 

\subsection{Twisted root data and Weyl groups}
Let $M$ be a $\bZ$-lattice and $M^\vee$ be its dual lattice.
Let $R$ and $R^\vee$ be  finite subsets of $M$  and $M^{\vee}$ respectively.
Suppose that  $(M,M^\vee,R, R^\vee)$  satisfy the root data axioms. 


Denote by $\uM_S$ (resp.\ $\uR_S$) the constant sheaf on $S$ associated to $M$ (resp.\ $R$).  
%We denote $\Psi_0=(\uM_S, \uM^\vee_S, \uR_S, \uR^\vee_S)$ the twisted root datum associated to $(M,M^\vee,R, R^\vee)$.
Let $T$ be an $S$-torus and denote by  $\cM$ (resp.\ $\cM^\vee$) the fppf sheaf of characters
(resp.\ cocharacters) of $T$. Let $\Psi= (\cM, \cM^\vee, \cR, \cR^\vee)$ be
a twisted root datum. (c.f.\ \cite{SGA3} Exp.\ XXII, 1.9).
We say that the twisted root datum $\Psi$ is of \emph{type} $(M,M^\vee,R, R^\vee)$ at $s\in S$ if $\Psi_{\ol{s}}\simeq (\uM_{\ol{s}},\uM_{\ol{s}}^\vee,\uR_{\ol{s}}, \uR_{\ol{s}}^\vee)$.


We denote by $W(\Psi)$ the Weyl $S$-group scheme 
of the twisted root datum $\Psi$.
If $\Psi$ is split of constant type, $\Psi$ can be written as $(\uM_{S},\uM_{S}^\vee,\uR_{S}, \uR_{S}^\vee)$ for some root datum $(M,M^\vee,R, R^\vee)$.
In this case, we have $W(\Psi) = W_S$
where $W$  is the (abstract) Weyl group  
of the root datum $(M,M^\vee,R, R^\vee)$;
we recall that $W$ is 
a finite group generated by the reflections $s_\alpha$ defined by
\[s_\alpha(x)=x-(\alpha^{\vee},x)\alpha, \mbox { for $\alpha\in R$ and $x\in M$}.\]
For $\alpha\in R$, the reflection $s_\alpha$ induces an automorphism $\tilde{s}_\alpha$ on $T$ by

\[\tilde{s}_\alpha(t)=t(\alpha^{\vee}(\alpha(t)))^{-1}\, \mbox { for any $S$-scheme $S'$, $\alpha\in R$ and $t\in T(S')$}.\]
(See \cite{SGA3} Exp.\ XXII, 3.3.)



The map sending  $s_{\alpha}$ to $\tilde{s}_\alpha$ defines an action of $W(\Psi)^{op}$ on $T$.
For $w\in W(\Psi)(S)$, we denote by $\tilde{w}$ its image in $\uAut(T)(S).$
In general, $W(\Psi)$  is a finite \'etale group scheme over $S$, which is regarded as a subgroup scheme of $\uAut(\cM)$.


Let $G$ be a reductive group scheme over $S$.
Suppose that $G$ has a maximal torus $T$ over $S$.
We denote by $\Phi(G,T)$ the twisted root datum of $G$ with respect to $T$.
Let $\uN_G(T)$  the normalizer of $T$ in $G$. The conjugation action of $\uN_G(T)/T$ on $T$ gives an isomorphic from $\uN_G(T)/T$ to $W(\Phi(G,T))^{op}$. (cf. \cite{SGA3} Exp.\ XXII, 3.1-3.4)

\medskip

\subsection{The Orientation}
Let $\Psi_1$, $\Psi_2$ be two twisted root data. Suppose that $\Psi_1$ and $\Psi_2$ 
are of the same type at each $s\in S$. 
Let $\uIsom(\Psi_1,\Psi_2)$ be the
sheaf of isomorphisms between $\Psi_1$ and $\Psi_2$. Then
$\uIsom(\Psi_1,\Psi_2)$ is a right principal homogeneous space
of $\uAut(\Psi_1)$ and a left principal homogeneous of
$\uAut(\Psi_2)$. 
Define
\[\uIsomext(\Psi_1,\Psi_2)=W(\Psi_2)\backslash\uIsom(\Psi_1,\Psi_2).\]

Write the twisted root datum $\Psi_i$ as $(\cM_i,\cM_i^\vee, \cR_i, \cR_i^{\vee}).$
For an $S$-scheme $S'$ and $f\in \uIsom(\Psi_1,\Psi_2)(S')$, $f$ induces an isomorphism between the Weyl groups $W(\Psi_{1,S'})$ and $W(\Psi_{2,S'})$.
Namely for any $S'$-scheme $S''$ and any root $\alpha$ of $\cR_{1}(S'')$, $f$ sends the reflection $s_\alpha$ to $s_{f(\alpha)}$ and   $f\circ s_{\alpha}=s_{f(\alpha)}\circ f.$
Hence  $\uIsomext(\Psi_1,\Psi_2)$ is canonically isomorphic to $\uIsom(\Psi_1,\Psi_2)/W(\Psi_1).$
As a consequence, the natural isomorphism between sheaves $\uIsom(\Psi_1,\Psi_2)$ and $\uIsom(\Psi_2,\Psi_1)$, which sends $f\in \uIsom(\Psi_1,\Psi_2)(S')$ to $f^{-1}$, gives an isomorphism between $\uIsomext(\Psi_1,\Psi_2)$ and $\uIsomext(\Psi_2,\Psi_1).$


Suppose that $\uIsomext(\Psi_1,\Psi_2)(S)$ is nonempty.
For $u\in\uIsomext(S)$, we define $\uIsomint_u(\Psi_1,\Psi_2)$ to be the fiber of $\uIsom(\Psi_1,\Psi_2)\to\uIsomext(\Psi_1,\Psi_2)$ at $u$. 


Let $G$ be a reductive group scheme  over $S$. 
The \emph{type} of
$G$ at $s$ is the isomorphism class of the root datum of $G_{\ol{s}}$ with respect to its maximal torus. (ref.~\cite{SGA3},
Exp.\ XXII, Def.\ 2.6.1, 2.7)

Suppose that $G$ has a maximal torus $T$. We define $\uIsomext(G,\Psi)$ to be  $\uIsomext(\Phi(G,T),\Psi)$.
Let  $T^\sharp$ be another maximal torus of $G$ over $S$. For an $S$-scheme $S'$, every element of the transporter $\uTransp_G(T^\sharp,T)(S')$ gives an isomorphism from $T^\sharp_{S'}$ to $T_{S}$ via conjugation, which in turns
gives an isomorphism from $\Phi(G,T)$ to $\Phi(G,T^\sharp)$. 
As $\uTransp_G(T^\sharp,T)$ is a right homogeneous space  under $\uN_G(T^\sharp)$, through the identification of $\uN_G(T^\sharp)/T^\sharp$ with $W(\Phi(G,T^\sharp))^{op}$, we see that there is 
a natural morphism between \[\uIsom(\Phi(G,T^\sharp),\Psi)\land^{\uN(T^\sharp)^{op}}\uTransp_G(T^\sharp,T)\to \uIsom(\Phi(G,T),\Psi).\]
This gives a canonical isomorphism from $\uIsomext(\Phi(G,T^\sharp),\Psi)$ to $\uIsomext(\Phi(G,T),\Psi)$.
 Hence $\uIsomext(G,\Psi)$ is well-defined. For those $G$ without maximal torus over $S$, we define $\uIsomext(G,\Psi)$
 be descent (ref. \cite[\S 1.2.1]{L1}).


We define $\uIsomext(\Psi,G)$ in a similar way.
As there is a canonical isomorphism between $\uIsomext(\Phi(G,T),\Psi)$ and $\uIsomext(\Psi,\Phi(G,T))$, 
the two functors $\uIsomext(G,\Psi)$ and $\uIsomext(\Psi,G)$ are canonically isomorphic.


An \emph{orientation} of $G$ with respect to $\Psi$ is an element of $\uIsomext(G,\Psi)(S).$

\smallskip

A twisted root datum $\Psi$ is \emph{admissible} for $G$ if at each $s\in S$,  the type of $G_{\ol{s}}$ is the same as the type of   $\Psi_{\ol s}$,
 and $\uIsomext(G,\Psi)(S) \not= \emptyset$. The \emph{admissibility} condition is a necessary condition for the existence of a maximal torus $T$ of $G$ such that $\Phi(G,T)$ is isomorphic to $\Psi$. 
 
Let $G'$ be an $S$-form of $G$.
Let $\uIsom(G,G')$ be the sheaf of group isomorphisms between $G$ and $G'$.
Note that $G$ acts on itself by conjugation. Thus we can define the right quotient of $\uIsom(G, G')$ by the adjoint quotient $G_{ad}$ of $G$ and 
denote  $\uIsom(G, G')/G_{ad}$ by $\uIsomext(G,G')$.
  
  \smallskip

 Let $\Psi_1$, $\Psi_2$ and $\Psi_3$ be twisted root data over $S$. Suppose that
all of them are of the same type at each geometric fibre of $S$. Then
we have the following morphism \label{pairing1}:
\begin{equation}
\uIsomext(\Psi_1,\Psi_2)\times\uIsomext(\Psi_2,\Psi_3)\to\uIsomext(\Psi_1,\Psi_3),
\end{equation}
that comes from the composition of isomorphisms 
$$\uIsom(\Psi_1,\Psi_2)\times\uIsom(\Psi_2,\Psi_3)\to\uIsom(\Psi_1,\Psi_3).$$



Suppose $G$ and $\Psi_1$ are of the same type at each geometric fiber of $S$.
Similarly we have the following pairings coming from the composition of isomorphisms between root data.

\begin{proposition}\label{pairing}
We have the following pairings
\begin{enumerate}
 \item $\uIsomext(G,\Psi_1)\times\uIsomext(\Psi_1,\Psi_2)\to\uIsomext(G,\Psi_2).$
 \item  $\uIsomext(G,\Psi_1)\times\uIsomext(G,\Psi_2)\to\uIsomext(\Psi_1,\Psi_2).$ 
 \item $\uIsomext(G,\Psi_1)\times\uIsomext(G',\Psi_1)\to\uIsomext(G,G').$  
 \item $\uIsomext(G',G) \times \uIsomext(G,\Psi)\to \uIsomext(G’,\Psi)$.

 \end{enumerate}
 \end{proposition}
\begin{proof}
The first two assertions come from the definition of $\uIsomext(G,\Psi)$ and the composition of isomorphisms between root data.
For (3) and (4), we notice that if $G$ and $G'$ have maximal tori $T$ and $T'$ respectively, then $\uIsomext(\Phi(G,T),\Phi(G',T'))$ is representable and isomorphic to $\uIsomext(G,G')$.
As reductive groups have maximal tori  \'etale locally, assertions (3) and (4) can be deduced from descent. 
We refer to \cite[Prop.\ 1.4]{L1} for more details.
\end{proof}

\smallskip


Another complement is the following criterion of 
reduction of torsors.
Let $G$ be a reductive group scheme over $S$ with a maximal torus $T$.
Let $E$ be right $fppf$ $G$-torsor over $S$.
Consider $G$ acts on itself by conjugation and  set $G^\sharp =E\land^G G$. 
As $G^\sharp$ is an inner twisted form of $G$, there is a canonical element $c\in\uIsomext(G,G^\sharp)(S)$.
As $\uIsomext(G,G^\sharp)$ is canonically isomorphic to $\uIsomext(\Psi,\Psi^\sharp)$, we can regard $c$ as an element in $\uIsomext(\Psi,\Psi^\sharp)(S)$ 
for any pair $(\Psi,\Psi^\sharp)$ of twisted root data
where $\Psi$ (resp.\ $\Psi^\sharp$) is admissible
for $G$ (resp.\ $G^\sharp$).

\begin{proposition}\label{prop_new}
Let $E$, $T$, $G$ and $G^\sharp$ be as above 
and consider the twisted root datum $\Psi=\Phi(G,T)$.
Then the following assertions are equivalent: 


\smallskip


(i) $E$ admits a reduction to $T$, 
i.e.\ there is a right  $T$-torsor $F$ such that $E\simeq F\land^T G$ as $G$-torsors;


\smallskip


(ii) $G^\sharp$ admits  a maximal $S$--torus $T^\sharp$ satisfying the following property:

\smallskip
the set $\uIsomint_c(\Psi,\Psi^\sharp)(S)$ is nonempty, 
with the notation $\Psi=\Phi(G^\sharp,T^\sharp)$.


\end{proposition}

\begin{proof}
Let $\uIsom_G(G,E)$ be the fppf-sheaf of isomorphisms of right $G$-torsors. Denote by $1_G$  the neutral element of $G(S)$.
There is an isomorphism $\iota$ from $\uIsom_G(G,E)$ to $E$ which sends 
$f\in\uIsom_G(G,E)(S')$ to $f(1_G)$ for any $S$-scheme $S'$. 
Regard $G$ as a right $G$-torsor,  the left multiplication of $G$ on itself forms the automorphism group $\uAut_G(G)$.
Thus $\uIsom_G(G,E)$ is a right $G$-torsor and $\iota$ is an isomorphism of right $G$-torsors.

Denote by $\uIsomint_c(G,G^\sharp)$ the fiber of $\uIsom(G,G^\sharp)\to \uIsomext(G,G^\sharp)$ at $c$.
Clearly an isomorphism between the $G$-torsors $G$ and $E$ gives an isomorphism between reductive groups $G$ and $G^\sharp$ compatible with the orientation $c$. Hence  there is a natural surjective morphism of sheaves from $\uIsom_G(G,E)$ to $\uIsomint_c(G,G^\sharp)$.


\smallskip

\noindent $(ii) \Longrightarrow (i)$. We are given a maximal $S$--torus $T^\sharp$ of $G^\sharp$
such that $\uIsomint_c(\Psi,\Psi^\sharp)(S)\neq\emptyset$.
Next consider the sheaf of isomorphisms $\uIsomint_c((G,T),(G^\sharp,T^\sharp))$, which consists of elements in $\uIsomint_c(G,G^\sharp)$   that apply $T$ to $T^\sharp$.
This is a subsheaf of $\uIsomint_c(G,G^\sharp)$. 
Set \[\cF=\uIsomint_c((G,T),(G^\sharp,T^\sharp))\underset{\uIsomint_c(G,G^\sharp)}{\times}\uIsom_G(G,E).\]
We can regard $\cF$ as a subsheaf of $\uIsom_G(G,E)$ through the second projection $p_2$.  
Note that $\uN_G(T)$ acts diagonally on the right of both factors of $\cF$.
Clearly $p_2$ is an $\uN_G(T)$-equivariant morphism. 
Let $\pi:\cF\to \uIsomint_c(\Psi,\Psi^\sharp)$ be the projection from $\cF$ to $\uIsomint_c((G,T),(G^\sharp,T^\sharp))$ composed with the natural morphism from $\uIsomint_c((G,T),(G^\sharp,T^\sharp))$ to $\uIsomint_c(\Psi,\Psi^\sharp).$
%%Under the identification of $W(\Psi)$ with $(N(T)/T)^{op}$, we see that $\pi$ is $N(T)^{op}$-equivariant.

% Suppose $\uIsomint_c(\Psi,\Psi')(S)\neq\emptyset$.
Let $f\in\uIsomint_c(\Psi,\Psi^\sharp)(S)$. Then the fiber of $\pi$ at $f$ is a right $T$-torsor and we denote it by $F$.
Note that $F$ embeds in $E$ via $\iota\circ p_2$.
Hence we have $E= F \wedge^T G$.

\smallskip

\noindent $(ii) \Longrightarrow (i)$. We suppose $E=F\land^T G$ for some $T$-torsor $F$.
Then $G^\sharp=E\land^G G=F\land^T G$, and $T^\sharp=F\land^T T$ is a maximal torus of $G^\sharp$.
% Hence (1) holds.
It remains to show that $\uIsomint_c(\Psi,\Psi^\sharp)(S)\neq\emptyset$.



Let $S'\to S$ be a $fppf$-cover that trivializes $F$.
We choose a trivialization $f:T_{S'}\xrightarrow{\sim} F_{S'}$.
Then $f$ induces an isomorphism between $(G_{S'},T_{S'})$ and $(G^\sharp_{S'},T^\sharp_{S'})$ with respect to the orientation $c$, 
which in turn gives an isomorphism $f^\diamond \in\uIsomint_c(\Psi,\Psi^\sharp)(S')$.


Set $S''=S'\times_S S'$.
Let $r_1$ and $r_2$ be the projections of $S''$  to its first and second factors respectively. 
Then $(r_2^\ast f)^{-1}\circ(r_1^\ast f)$ is an automorphism of $T_{S''}$ (as a $T_{S''}$-torsor),
which  corresponds to  the left multiplication by some element $t\in T(S'')$.
As $T$ acts trivially on itself by conjugation, $(r_2^\ast f)^{-1}\circ(r_1^\ast f)$ induces trivial automorphism on $\Psi_{S''}.$
Hence $r_2^\ast (f^\diamond )=r_1^\ast (f^\diamond )$ and $f^\diamond $ descends to an element in
$\uIsomint_c(\Psi,\Psi^\sharp)(S)$. Thus $\uIsomint_c(\Psi,\Psi^\sharp)(S) \not = \emptyset$ 
as desired.
\end{proof}


\smallskip


Denote by $G//G$ the adjoint quotient of $G$ in the GIT framework, 
its formation commutes to arbitrary base change \cite[\S 4]{L2}.
Suppose that $G$ has a maximal torus $T$.
Let $T/W(T)$ be the quotient sheaf of $T$ by the action of Weyl group $W(T)$.
Then $T/W(T)$ is isomorphic to $G//G$ (\cite{L2} Thm.\ 4.1).

Consider the quotient map (of $\fppf$-sheaves) $\pi: T\to T/W(T)$.
Let $x\in (G//G)(S)$. As $T/W(T)$ and $G//G$ are isomorphic, we can regard  $x$ as an element in $(T/W(T))(S).$
Suppose that there is a semisimple regular element $t_1\in G(S)$ whose image in $(G//G)(S)$ is $x$.
Let $T_1$ be the centralizer $\uC_G(t_1)$, that is also a maximal torus.
Let $\Psi$ (resp.\ $\Psi_1$) be the root datum associated to $T$ (resp.\ $T_1$). 
Then the natural inclusion from $T$ to $G$ gives an orientation $v \in\uIsomext(G, \Psi)(S)$.
Similarly, we get an orientation $v_1\in \uIsomext(G,\Psi_1)(S)$ from the natural inclusion of $T_1$ in $G$.
Then by  \ref{pairing} (2) we get an orientation
$u\in \uIsomext (\Psi,\Psi_1)(S)$ coming from $v\cdot v_1$.
Then we have the following proposition.
\begin{proposition}\label{W-torsor}
The left $W(T)$-torsor $\pi^{-1}(x)$ is isomorphic to $\uIsomint_u(\Psi,\Psi_1)$.
\end{proposition} 

\begin{proof}
For an $S$-scheme $S'$ and $t\in \pi^{-1}(x)(S’)$, we denote by $\uTransp(t,t_1)$ the transporter from $t$ to $t_1$.
Both $t$ and $t_1$ are mapped to the same element in $(G//G)(S)$, and hence $\uTransp(t,t_1)\to \bullet$ is surjective.
Since $t_1$ is a semisimple regular element, so is $t$.
As $t$ lies in $T$,  the centralizer $\uC_G(t)$ of $t$ is the torus $T_{S’}$ and $\uTransp(t,t_1)$ is a right $T_{S’}$-torsor.


The conjugation action of $T$ on $\Psi=\Phi(G,T)$ is trivial.
Therefore the canonical morphism $$\uTransp(t,t_1)\land^{T_{S'}}\Psi_{S'}\to \Psi_{1,S'}$$ defines an element in $\uIsomint_{u}(\Psi,\Psi_1)(S’) $.
In this way, we get a map $$i:\pi^{-1}(x)\to\uIsomint_{u}(\Psi,\Psi_1).$$
To be precise, write the root datum $\Psi$ as $(\cM,\cM^\vee,\cR,\cR^\vee)$.
For an $S'$-scheme $S''$ and $h\in\uTransp(t,t_1)(S'')$, $i(t)(m)=m\circ int (h^{-1})$ for all $m\in \cM_{S^{\prime\prime}}=\Hom_{S^{\prime\prime}}(T_{S^{\prime\prime}}, \G_{m,S^{\prime\prime}})$.
Note that $i(t)$ is independant of the choice of $h$ since $T$ acts trivially on $\cM$. 

As $\uIsomint_{u}(\Psi,\Psi_1)$ is a right $W(\Psi)$-torsor and $W(T)\simeq W(\Psi)^{op}$, we can regard it as left $W(T)$-torsor.
We write down explicitly this action.
For $w\in W(T)_{S'}$, let $n_w$ be a lift of $w$ in $N(T)(S'')$ for some $S'$-scheme $S''$.  
As $T$ acts on $\cM$ trivially,  $int (n^{-1}_w)$ induces an automorphism on $\cM_{S'}$.
 For $f\in \uIsomint_{u}(\Psi,\Psi_1)(S’)$,
 $(w\cdot f)(m):=f(m\circ int (n_w))$ for $m\in \cM_{S’}=\Hom_{S’-grp}(T_{S’},\G_{m,S’})$.

Since $\pi^{-1}(x)$ and $\uIsomint_u(\Psi,\Psi_1)$ are both $W(T)$-torsors, to check $i$ is an isomorphism, it suffices to check the map $i$ is $W(T)$-equivariant.
Let $w\in W(T)_{S'}$ and $n_w$ be a lift of $w$ in $N(T)(S'')$ for some $S'$-scheme $S''$.  
For  $h\in \uTransp(t,t_1)(S'')$,  $hn^{-1}_w\in  \uTransp(w\cdot t,t_1)(S'')$. 
Thus
\begin{equation}
\begin{aligned} 
i(w\cdot t)(m) & =m\circ int((hn^{-1}_w)^{-1})\\
&=m\circ int(n_w)\circ int(h^{-1})\\
&=i(t)(m\circ int(n_w))\\
 &=(w\cdot i(t))(m). 
\end{aligned}
\end{equation}
Hence $i$ is an isomorphism between $W(T)$-torsors.
\end{proof}








\subsection{Embedding functors}
Let $G$ be a reductive group scheme over $S$ and  \break $\Psi= (\cM, \cM^\vee, \cR, \cR^\vee)$ be an admissible root datum for $G$.
Let $T$ be the $S$-torus with sheaf of characters $\cM$.

Suppose there is an embedding $f$ of $T$ into $G$ as algebraic group schemes.
Then $f$ induces an isomorphism from the character group of $f(T)$ to $\cM$.
We call this map $f^\sharp$.
If $f^\sharp$ induces an isomorphism between $\Phi(G,f(T))$ and $\Psi$, then we say $f$ is an embedding with respect to the twisted root datum $\Psi$. 


Define the \emph{embedding
functor} $ \gE(G,\Psi)$ as follows: for each $S$-scheme $S'$,  \[ \gE(G,\Psi)(S')=\left\{\begin{array}{l}
\mbox{$f:T_{S'}\hookrightarrow G_{S'}$}\left|\begin{array}{l}\mbox{$
f$ is both a closed
immersion and a group }\\
 \mbox{homomorphism that induces an isomorphism
}\\
\mbox{$f^{\sharp}:\Phi(G_{S'},f(T_{S'}))\xrightarrow{\sim}\Psi_{S'}$
}\\
%\mbox{ $f^{\Psi}(\alpha)=\alpha\circ f^{-1}|_{f(T_{S'})}$ for
%all $\alpha\in\sM(\rS'')$,}\\\mbox{for each $\rS'$-scheme
%$\rS''$.}
\end{array}\right.\end{array}\right\}\]

Note that the fact that $\Psi$ is admissible for $G$ ensures that $\gE(G,\Psi)$ is not empty and allows us to fix an orientation $v \in \uIsomext(G,\Psi)(S)$.


The
\emph{oriented embedding functor} $\gE(G,\Psi,v)$  over $S$ is defined by 


%Given an oriented twisted root datum $(\Psi, v)$ of $\Psi$ with
%respect to $\rG$, we define the \emph{oriented embedding functor}
%as:

\[\gE(G,\Psi,v)(S')=\left\{\begin{array}{l}\mbox{$f:T_{S'}\hookrightarrow G_{S'}$}\left |
\begin{array}{l}\mbox{$f\in\gE(G,\Psi)(S')$,
and the image of $f^{\sharp}$ } \\
\mbox{ in $\uIsomext(G,\Psi)(S')$ is
$v$.}\end{array}\right.\end{array}\right\}\]

Since any two maximal tori of $G$ are \'etale locally conjugated by $G$,
the oriented embedding functor $\gE(G,\Psi,v)$ is representable by an affine $S$--scheme that  is a left homogeneous $G$-space (\cite[1.6]{L1}).  
For an $S$-scheme $S'$ and an element $f\in \gE(G,\Psi,v)(S')$, the stabilizer of $f$ in $G_{S'}$ is $f(T_{S'})$, which is a maximal torus of $G_{S'}$.
The $S$--scheme $\gE(G,\Psi,v)$ admits a right  action of
$W(T)$ and the quotient sheaf $\gE(G,\Psi,v)/W(T)$ identifies with the
$S$-scheme $\cT_G$ of maximal tori of $G$. (See \cite[\S 1.1 and \S1.2]{L1} for more details.)

\begin{sremark}{\rm
In \cite[\S 1.2.2]{L1} we define the oriented embedding functor $\gE(G,\Psi,v')$ by orientation $v'\in \uIsomext(\Psi,G)(S)$ instead of $v\in \uIsomext(G,\Psi)(S)$ here.
However when $v'$ is the image of $v$ under the canonical isomorphism $\uIsomext(G,\Psi)\to\uIsomext(\Psi,G)$, these two definitions are clearly equivalent.
}
\end{sremark}

\begin{slemma} \label{lem_compatibilities}
Keep the notation as above.
\smallskip

 (1) We have an isomorphism $W(\Psi) \simlgr \uAut_G(\gE(G,\Psi,v))$ as group schemes.

\smallskip

(2) Let $\Psi'$ be another admissible root datum for $G$,
and $v' \in \Isomext(G,\Psi')(S)$. Let $u \in \uIsomext(\Psi,\Psi')(S)$ be the orientation  $v’ \cdot v$  by the pairing (\ref{pairing}). Under the identification of $W(\Psi)$ and $\uAut_G(\gE(G,\Psi,v))$ in (1), there is a natural isomorphism between
 $\uIsomint_u( \Psi, \Psi')$ and $\uIsom_G(\gE(G,\Psi,v), \gE(G,\Psi',v'))$ as right $W(\Psi)$--torsors.

% \iffalse
%(2) The forms of $\gE(G,\Psi,v)$ are classified by $H^1_{right}( S, W(\Psi))$.


\smallskip


(3)
Let $F$ be a right ${W}(\Psi)$-torsor and $\Psi'=F\land^{W(\Psi)}\Psi$.
Let $c$ be the canonical element of $\uIsomext(\Psi,\Psi')(S)$, and
 let $v_1\in \uIsomext(G, {\Psi'})$ be  $c\cdot v$
 by the pairing (\ref{pairing}). 
By identifying $W(\Psi)$ with $\uAut_G(\gE(G,\Psi,v))$, $F\land^{W(\Psi)}\gE(G,\Psi,v)$
is isomorphic to  $\gE(G, {\Psi'},v_1)$.


\smallskip

(4)  Let $E$ be a right $G$--torsor and denote by $G'$ the twist of $G$ by $E$ via inner automorphisms.
Let $c$ be the canonical element of $\uIsomext(G',G)(S)$.
Then the homogeneous $G'$-space  $E \land^{G} \gE(G,\Psi,v)$ is isomorphic
to   $\gE(G',\Psi,v')$ where $v' \in \uIsomext(G',\Psi)(S)$ is  $c\cdot v$ by the pairing  (\ref{pairing}).
\smallskip

\iffalse


 %\iffalse

\smallskip


(5) Let $\Psi'$ be another admissible root datum for $G$ and let
$v' \in \Isomext(G,\Psi')$. Then the $S$--functor
 $\uIsom( (\Psi',c'), (\Psi,c))$ is representable by a left $W(\Psi)$--torsor $J$
and  there is a natural isomorphism  $\gE(G,\Psi,v) \land^{W(\Psi)} J  \simlgr \gE(G,\Psi',v')$.
\fi

\end{slemma}


\begin{proof}
(1) For an $S$-scheme $S'$ and $w\in W(\Psi)(S')$, we denote by $\tilde{w}$ the image of $w$ in $W(T)(S')$.
Define a morphism $\iota:W(\Psi)\to \uAut_G(\gE(G,\Psi,v))$ as follows.
For an $S$-scheme $S'$ and  $w\in W(\Psi)(S')$,
\[\iota(w)(x):=x\circ \tilde{w} \mbox{  for all $S'$-schemes $S''$,  and $x\in \gE(G,\Psi,v)(S'')$.}\]

For $w_1$ and $w_2\in W(\Psi)(S')$,
\begin{equation}
\begin{aligned}
\iota(w_1\circ w_2)(x)&=x\circ\widetilde{(w_1\circ w_2)}\\
&=x\circ\tilde w_2\circ\tilde w_1\\
&=(\iota(w_1)\circ\iota(w_2))(x).
\end{aligned}
\end{equation}
Hence $\iota$ is a group homomorphism. 

Clearly $\iota$ is monomorphism.
 To see that $\iota$ is an isomorphism, it suffices to check $\iota$ is an isomorphism of fppf-sheaves.
 Let $\{S_i\}$ be a $fppf$-cover of $S$ such that $G_{S_i}$ and $\Psi_{S_i}$ are both split.
 Then $\gE(G,\Psi,v)(S_i)$ is not empty.
 Choose $x_i\in\gE(G,\Psi,v)(S_i)$. By  Lemma \ref{lem_quotient},
 $\uAut_{G}(\gE(G,\Psi,v))(S_i)$ is isomorphic to $(\uN_G(x_i(T))/x_i(T))^{op}$, where $\uN_G(x_i(T))$ is the normalizer of $x_i(T)$ in $G$.
Since $\uN_{G}(x_i(T))/x_i(T)\simeq {W}(\Phi(G_{S_i},x_i(T_{S_i})))^{op}$,  the group $\uN_{G}(x_i(T))/x_i(T)$ is isomorphic to ${W}(\Psi_{S_i})^{op}.$
Thus $\uAut_{G}(\gE(G,\Psi,v))_{S_i}\simeq W(\Psi_{S_i})$.
As $\iota$ is a monomorphism and $\uAut_{G}(\gE(G,\Psi,v))_{S_i}$ is isomorphic to $W(\Psi_{S_i})$, $\iota$ is an isomorphism.


\smallskip

\noindent (2)
Write $\Psi'=(\cM',(\cM')^\vee, \cR',(\cR')^\vee)$ and denote by $T'$ the torus with sheaf of characters $\cM'$.
For an $S$-scheme $S'$ and $f \in\uIsomint_u(\Psi,\Psi')(S')$, we denote by $\tilde{f}$ the isomorphism from $T'_{S'}$ to $T_{S'}$ induced by $f$.
Define $\eta:\uIsomint_u(\Psi,\Psi')\to \uIsom_G(\gE(G,\Psi,v),\gE(G,\Psi',v'))$ by $\eta(f)(x)=x\circ \tilde{f} $,
for all $S'$-scheme $S''$, and $x\in\gE(G,\Psi,v)(S'')$.


Next we show that $\eta$ is compatible with the $W(\Psi)$-action under the identification of $W(\Psi)$ and $\uAut_G(\gE(G,\Psi,v))$.
Let $w\in W(\Psi)(S')$.
Then
\begin{equation}
\begin{aligned}
\eta(f\circ w)(x)&=x\circ (\widetilde{f\circ w})\\
&=x\circ \tilde{w}  \circ \tilde{f}\\
&=\eta(f)(x\circ\tilde{w})\\
&=\eta(f)( \iota(w)(x))\\
&=(\eta (f)\circ\iota(w))(x).
\end{aligned}
\end{equation}

This shows that $\eta $ is compatible with the $W(\Psi)$-action.
Since $\uIsomint_u(\Psi,\Psi')$ and $\uIsom_G(\gE(G,\Psi,v),\gE(G,\Psi',v'))$ are both right $W(\Psi)$-torsors, this implies that $\eta$ is an isomorphism.





\noindent (3)
Write $\Psi'$ as $(\cM',(\cM')^\vee,\cR',(\cR')^\vee)$ and let $T'$ be the torus with character group $\cM'$.
For an $S$-scheme $S'$ and $f\in F (S')$, we define $\varphi_f^\sharp:\cM_{S'}\to\cM'_{S'}$ as
$\varphi_f^\sharp(m)=(f,m)$ for all $S'$-scheme $S''$ and $m\in \cM_{S'}(S'')$.
Denote by $\varphi_f$ the group homomorphism from $T'$ to $T$ defined by $\varphi_f^\sharp$.

Define $\varsigma:F\times \gE(G,\Psi, v)\to\gE(G,\Psi',v')$ as follows.
For an $S$-scheme $S'$ and $(f,x)\in (F\times\gE(G,\Psi,v))(S')$, let
$\varsigma(f,x)=x\circ \varphi_f$. Clearly $x\circ \varphi_f$ is an embedding of $T'$ in $G$
with orientation $v'$.
Note that for $w\in W(\Psi)(S')$, we have
\begin{equation}
\begin{aligned}
 \varphi_{fw}^\sharp(m)&=(fw,m)\\
 &=(f,w^{-1}m)\\
 &=\varphi_f^\sharp(w^{-1}m).
\end{aligned}
\end{equation}
Therefore $\varphi_{fw}^\sharp=\varphi_f^\sharp\circ w^{-1}$ and
$\varphi_{fw}=\widetilde{w}^{-1}\circ\varphi_f$.
It follows that
\begin{equation}
\begin{aligned}
\varsigma(fw,x)&=x\circ \varphi_{fw}\\
&=x\circ \widetilde{w}^{-1}\circ \varphi_f\\
&=(w^{-1}\cdot x)\circ\varphi_f\\
&=\varsigma(f,w\cdot x).
\end{aligned}
\end{equation}
Hence $\varsigma$ induces a morphism from $F\land^{W(\Psi)}\gE(G,\Psi,v)$ to
$\gE(G,\Psi',v')$, that is clearly an isomorphism.

\smallskip

\noindent (4)
For an $S$-scheme $S'$ and $e\in E(S')$, we define $f_e: G_{S'}\to G'_{S'}$ by 
$f_e(g)=[e,g]$ for all $S'$-scheme $S''$ and $g\in G(S'')$. Then $f_e$ is an isomorphism between $G_{S'}$ and $G'_{S'}$.

For $y=(e,x)\in (E\times_S \gE(G,\Psi, v))(S')$,
we define \[\imath: E\times_S \gE(G,\Psi,v)\to \gE(G',\Psi,v')\] as
$\imath(y)=f_e\circ x$. It is clear that $\imath(y)\in\gE(G',\Psi)(S').$
As $x$ is an embedding of orientation $v$, by the definition of canonical orientation, the embedding  $\imath(y)$ is of orientation $v'$.
Since $\imath(e,x)=\imath(e\cdot g,g^{-1}\cdot x)$ for all $g\in G(S')$, $\imath$ induces an isomorhism from $E\land^G \gE(G,\Psi, v)$ to $\gE(G',\Psi,v')$.
\end{proof}

\begin{sremark}\label{rem_behaviour}{\rm
Let $G^{sc}$ be the simply connected cover of $G$
and let $G_{ad}$ be its adjoint quotient.
On the other hand let $\Psi^{sc}$ (resp.\ $\Psi_{ad}$)
the simply connected (resp.\ adjoint) root datum associated to
$\Psi$.
Then we can attach to $(\Psi,v)$ an oriented
root datum $(\Psi^{sc},v^{sc})$ (resp.\ $(\Psi_{ad},v_{ad})$)
and isomorphisms
$$
\gE(G^{sc},\Psi^{sc},v^{sc}) \simlgr \gE(G,\Psi,v) \simlgr \gE(G_{ad},\Psi_{ad},v_{ad}).
$$
It implies that we can deal in practice with semisimple simply connected
(resp.\, adjoint) group schemes.
Another advantage is that we have isomorphisms of finite \'etale
group schemes
$$
\uIsomext(G^{sc}, \Psi^{sc} ) \simlgr  \uIsomext(G_{ad}, \Psi_{ad} )
\simlgr \uIsomext( \uDyn(G), \uDyn(\Psi)).
$$
} (See \cite[Cor.\ 1.7]{L1}.)
\end{sremark}


\subsection{The main result}


\begin{stheorem}\label{thm_main}
Let $X$ be a separated $S$-scheme
satisfying one of the following conditions:

\smallskip

(i) $S$ is locally noetherian;

\smallskip

 (ii) $X$ is locally of finite type over $S$.

\smallskip

\noindent We assume that $X$ is a left homogeneous space under a reductive $S$--group 
scheme $G$
and that for each point $s \in S$, the stabilizer of $X_{\ol s}$ is a maximal torus  of $G_{\ol s}$.

\smallskip

(1) There exists a twisted root datum $\Psi= (\cM, \cM^\vee, \cR, \cR^\vee)$
that is admissible for $G$ and an orientation $v \in \uIsomext(G, \Psi)(S)$
such that $X$ is isomorphic to $\gE(G,\Psi,v)$ as left $G$-homogeneous space.

\smallskip

(2) The oriented root datum $(\Psi,v)$ in (1) is unique up to isomorphism.

\end{stheorem}


\begin{proof} We shall establish first that the local stabilizers
of $X$ are maximal tori.
Since $G$ is smooth over $S$ with connected fibers,
$X$ is smooth of finite presentation over $S$  according to \cite[prop.\ VI.1.2]{R}.

\smallskip


\noindent (1) Let $(S_i)_{i \in I}$ be an fppf cover of $S$ such that $X_{S_i}$ is $G_{S_i}$--isomorphic
to $G_{S_i}/H_i$ where $H_i$ is a $S_i$-subgroup scheme of $G_{S_i}$ (see \cite [IV 6.7.3]{SGA3}).
Since $H_i$ is the stabilizer of a point of $X(S_i)$ (which is $S_i$--separated),
$H_i$ is  a closed $S_i$--subgroup scheme of $G_{S_i}$ and is in particular affine.

Again by \cite[prop.\ VI.1.2]{R}, $H_i$ is flat locally of finite presentation over $S_i$
so is affine flat  of finite presentation over $S_i$.
%Since $X$ is smooth of finite presentation, so is $H_i$ according
%to \cite{SGA3} (\textcolor{red}{quel expos\'e?}).
% this sentence was useless
As the geometric fibers of $H_i$ are tori, it follows that $H_i$ is a $S_i$--torus
\cite[X.4.9]{SGA3}. Therefore $H_i$ is a maximal $S_i$--torus of $G_{S_i}$ for each $i \in I$.

In particular, $C(G)$ acts trivially on $X$, so
$X$ is a $G_{ad}$-homogeneous space whose local stabilizers are maximal
tori. By Remark \ref{rem_behaviour} we reduce  to the case that $G$ is
an \emph{adjoint} semisimple $S$--group scheme.
We denote by $G^q$ a quasi-split $S$--form of $G$.
Since $G$ is an inner form of $G^q$, there is a canonical orientation $c\in\uIsomext(G,G^q)(S)$.
Let $T^q$ be a maximal torus of $G^q$, $\Psi^q$ be the root datum  $\Phi(G^q,T^q)$ and $v^q\in\uIsomext(G^q,\Psi^q)(S)$ be the orientation induced by the inclusion of $T^q$ to $G^q$.
Let $u=c\cdot v^q\in \uIsomext(G,\Psi^q)(S)$ (by \ref{pairing}) and consider the homogeneous $G$--space $X^q=\gE(G,\Psi^q, u)$.
This is a left $G$ homogeneous space whose local stabilizers are maximal tori.

\smallskip
\noindent
{\bf Claim.} $X$ is an $S$-form of $X^q$.


Denote by  $\uIsom_G(X^q,X)$ the sheaf of $G$-space isomorphisms over $S$.
Since $\Phi(G_{S_i},H_i)$ is fppf-locally isomorphic to $\Psi^q_{S_i}$ over $S_i$,
we can find an fppf-cover ${S_{ij}}$ of $S_i$ such that there is $f\in X^q(S_{ij})$ with $f(T^q_{S{ij}})=H_{i,S_{ij}}$.
As the stabilizer $\uStab_{G_{S_{ij}}}(f)$ is $H_{i,S_{ij}}$, we have $X^q_{s_{ij}}\simeq G_{S_{ij}}/H_{i,S_{ij}}$ as $G_{S_{ij}}$-spaces.
Hence there is an isomorphism between $X^q$ and $X$ over $S_{ij}$ that preserves the $G$-structure.
This proves that $\uIsom_G(X^q,X)$ is nonempty.
The canonical isomorphism $\uIsom_G(X^q,X)\land^{\uAut_G(X^q)} X^q\to X$ then shows that $X$ is an $S$-form of $X^q$.
Our claim is established.

\smallskip


By identifying $W(\Psi^q)$ with $\uAut_G(X^q)$ as in Lemma \ref{lem_compatibilities} (1),
we denote by $\Psi$ the root datum $\uIsom_G(X^q,X)\land^{\uAut_G(X^q)} \Psi^q$.
Let $c'\in\uIsomext(\Psi^q,\Psi)(S)$ be the canonical orientation, and $v$ be the orientation
$u\cdot c'$ by pairing (\ref{pairing}).
We conclude that $X$ is isomorphic to the embedding functor $\gE(G,\Psi,v)$ by Lemma \ref{lem_compatibilities}(3).


\smallskip

\noindent (2)
It follows from Lemma \ref{lem_compatibilities}(2).
\end{proof}

\iffalse
\subsection{Embbedings in quasi-split groups, I}

In this section we  assume that $G$ is semisimple simply connected and quasi-split.
Let $(B,T)$ be a Killing couple of $G$. We identify the Weyl group $W(T)$ with $N_G(T)/T$ and denote by $\Dyn(G)$
the Dynkin scheme (which is finite \'etale over $S$). We have a canonical isomorphism
$\Dyn(\Phi(G,T)) \simlgr \Dyn(G)$

According to \cite[XXIV.3.13]{SGA3}, we have an isomorphism $T \simlgr R_{\Dyn(G)/S} (\GG_{m, \Dyn(G)})$.
It follows that $T$ is an open $S$--subscheme of the vector group scheme
$\WW( \Dyn(G))$ and the (left) action of $W(T)$ on $T$
extends as a linear action on  $\WW( \Dyn(G))$.
We denote by $T^{reg}$ the open subscheme of regular elements of $T$.
Then  $W(T)$ stabilizes $T^{reg}$ and acts freely on it.
According to \cite[III.2.6.1, Corollaire]{DG}, the fppf sheaf $T^{reg}/W(T)$ is representable
and the map $\pi:T^{reg} \to T^{reg}/W(T)$ is  a $W(T)$--torsor.

Denote by $G//G$ the adjoint quotient. We use now  Chevalley's theorem \cite{L2} in this setting
$u: T/W(T) \simlgr G//G$ and consider the Steinberg cross section $c: G//G \to G$ of the quotient map  $G \to G//G$.
We are given
 $x \in (T^{reg}/W(T))(S)$ and consider the semisimple regular  element
 $g =c( u(x)) \in G(S)$.
It gives rise to the maximal  $S$--subtorus $T'=C_G(g)$ of $G$
and we have the important compatibility.




\begin{slemma} \label{lem_cross}
The $W(T)$-torsors $\pi^{-1}(x)$ and  $\uIsomint_{can}( \Psi(G,T),\Psi(G,T'))$ are isomorphic.
\end{slemma}
\begin{proof}
For an $S$-scheme $S'$ and $t\in \pi^{-1}(x)(S')$, we denote by $\uTransp(t,g)$ the transporter from $t$ to $g$.
As $t$ and $g$ are mapped to the same element in $G//G$, $\uTransp(t,g)$ is not empty.
Since $t$ is a regular element,  the centralizer $\uCentr(t)$ is the torus $T_{S'}$ and $\uTransp(t,g)$ is a right $T_{S'}$-torsor.
The conjugation action of $T$ on $\Phi(G,T)$ is trivial.
Therefore the canonical morphism $$\uTransp(t,g)\land^{T_{S'}}\Phi(G_{S'},T_{S'})\to \Phi(G_{S'},T'_{S'})$$ defines an element in $\uIsomint_{can}(\Phi(G,T),\Phi(G,T))(S') $.
In this way, we get a map $$i:\pi^{-1}(x)\to\uIsomint_{can}(\Phi(G,T),\Phi(G,T')).$$
To be precise, write the root datum $\Phi(G,T)$ as $(\cM,\cM^\vee,\cR,\cR^\vee)$.
For an $S'$-scheme $S''$ and $h\in \uTransp(t,g)(S'')$, $i(t)(m)=m\circ int (h^{-1})$ for all $m\in \cM_{S''}$.

As $\uIsomint_{can}(\Phi(G,T),\Phi(G,T))$ is a right $W(\Phi(G,T))$-torsor , we can regard it as left $W(T)$-torsor.
We write down explicitly this action.
For $w\in W(T)_{S'}$, let $n_w$ be a lift of $w$ in $N(T)(S'')$ for some $S'$-scheme $S''$.  
As $T$ acts on $M$ trivially,  $int (n^{-1}_w)$ induces an automorphism on $M_{S'}$.
 For $f\in \uIsomint_{can}(\Phi(G,T),\Phi(G,T'))(S')$,
 $(w\cdot f)(m):=f(m\circ int (n_w))$ for $m\in \cM_{S'}$.

Since $\pi^{-1}(x)$ and $\uIsomint(\Phi(G,T),\Phi(G,T'))$ are both $W(T)$-torsors, to check $i$ is an isomorphism, it suffices to check the map $i$ is $W(T)$-equivariant.
For an $S'$-scheme $S''$ and $h\in \uTransp(t,g)(S'')$, $i(w\cdot t)(m)=m\circ int((hn^{-1}_w)^{-1})=(w\cdot i(t))(m)$. Hence $i$ is an isomorphism between $W(T)$-torsors.
 \end{proof}
\fi





\section{The case of a LG ring}

\subsection{LG rings}\label{subsec_LG} We call $R$ an {\em $LG$-ring\/} if for every $n\in \NN$ and every $f\in R[X_1, \ldots, X_n]$ the polynomial $f$ represents a unit if and only if one of the following obviously equivalent conditions hold:


\begin{enumerate}
 \item\label{LG-defi} $f$ represents a unit over every localization $R_\m$, $\m$ a maximal ideal of $R$;

 \item\label{LG-defii} $f$ represents a unit over every field $R/\m$, $\m$ a maximal ideal of $R$;

  \item\label{LG-defiii} $f$ represents a unit over every $R$--field $F$;

 \item \label{LG-defiv} $\sum_{(r_1, \ldots, r_n) \in R^n} \, f(r_1, \ldots, r_n) R = R$,
 \end{enumerate}
 
Clearly a semilocal ring is a LG ring.
An important fact is that locally free $R$--modules 
of rank $n$ are free as reminded in \S \ref{subsec_LG}.
One says that a ring $R$ {\em satisfies the primitive criterion\/} \cite{EG, MW}
the following equivalent conditions hold: 

\begin{enumerate}
\item for every primitive polynomial $P\in R[X]$ there exists $r\in R$ such that $P(r) \in R^\times$;

\item for every primitive $Q\in R[X_1, \ldots, X_n]$ there exists $(r_1, \ldots, r_n)\in R^n$ such that $Q(r_1, \ldots, r_n) \in R^\times$;

\item $R$ is LG and all residue fields are infinite.
\end{enumerate}

In particular a semilocal ring with infinite residue fields is a  LG ring. For more on LG rings, see \cite[\S 11.20]{GPR}.



\subsection{Schematically dominant morphisms} \label{subsec_dominant}

We recall that a morphism of schemes $f: Y \to X$
is schematically dominant if $\cO_X \to f_* \cO_Y$ is
injective; if $f$ is an immersion we say that 
$X$ is schematically dense in $Y$ \cite[11.10.2]{EGA4}.
If $g: X' \to X$ is flat,
and $f$ is schematically dominant and quasi-compact, then 
$f': Y \times_X X' \to X'$ is  schematically dominant
({\it loc. cit.}, 11.10.5).

If  $f: X \to Y$ is an $S$--morphism of schemes, we say that 
$f$ is schematically $S$-dominant (or universally 
schematically dominant with respect to $S$) if for each $S$-scheme $S'$, 
$f_{S'}: X_{S'} \to Y_{S'}$ is schematically 
dominant; similarly if $f$ is an immersion, we say that
$Y$ is schematically $S$--dense in $X$. 
Under mild assumptions, this property can be checked fiberwise ({\it ibid}, 11.10.9).
The base change property above extends mechanically:
If $g: X' \to X$ is flat  and
$f: Y \to X$ is schematically  $S$-dominant and quasi-compact, then 
$f': Y \times_X X' \to X'$ is schematically  $S$-dominant.

Rydh extended that to morphism of algebraic spaces \cite[\S 7.5]{Ry1}. A morphism of algebraic spaces $f: X \to Y$
is schematically dominant if the morphism
$\cO_Y \to f_* \cO_X$ is
injective in the small \'etale site;
if $f$ is an immersion, we say that $X$ is schematically 
dense in $Y$.

As in \cite[\S 3.5]{Ry2}  the base change property 
extends in that framework:
given a flat  morphism $g: X' \to X$
of algebraic spaces 
assuming that 
$f: Y \to X$ is schematically  dominant and quasi-compact, then  $f': Y \times_X X' \to X'$ is schematically dominant.

If $S$ is a scheme and $f: Y \to X$ is a morphism
of $S$-algebraic spaces, we say that 
$f: X \to Y$ is schematically $S$--dominant  
if for each $S$-scheme $S'$, 
$f_{S'}: X_{S'} \to Y_{S'}$ is schematically 
dominant; similarly if $f$ is an immersion, we say that
$Y$ is schematically $S$--dense in $X$.
One last time, given a flat  morphism $g: X' \to X$  and
assuming that 
$f: Y \to X$ is schematically  $S$-dominant and quasi-compact, then  $f': Y \times_X X' \to X'$ is schematically $S$--dominant.


\subsection{(4)-Versal torsors} \label{subsec_versal}

Our goal is to adapt the framework of versal torsors \cite[\S 5]{GMS} from fields to the LG
ring setting by using algebraic spaces for which we use 
Olsson's book as reference \cite{O} as well as the 
Stacks project \cite{St}. It is close to 
Reichstein-Tossici's definition of (3)-versality
for algebraic groups \cite[Def.\ 1.3]{RT}. 


Let $S=\Spec(R)$ be the spectrum of a LG-ring $R$
and let $G$ be an affine flat $S$--group scheme of finite presentation.

\begin{sdefinition} 
Let $f:E \to X$ be a $G$--torsor where $E$ is a
quasi-compact quasi-separated $S$--scheme and $X$ is a quasi-compact quasi-separated  $S$-algebraic space.
We say that $E \to X$ is (4)--versal if it satisfies
the following property:
for each open retrocompact subspace $U$ 
schematically $S$--dense in  $X$, for each  $R$--algebra $B$ 
satisfying the primitive criterion, and for each $G$--torsor $F$ over $B$, there exists $x \in U(B)$ such that $E_x \cong F$ as $G$--torsors.

\end{sdefinition}


\begin{sremarks}
{\rm

\noindent (a)  We observe that if $f:E \to X$ is a (4)--versal  $G$--torsor so is
$f^{-1}(V) \to V$ for each schematically $S$--dense  open retrocompact subspace $V$ of $X$.

\smallskip

\noindent (b) There exists a dense open 
subspace $X'$ of $X$ that is a scheme \cite[Tag 06NH]{St}. If $X'$ can be chosen furthermore schematically $S$-dense
retrocompact, 
we see then that
the open dense retrocompact subscheme $U'=U \times_X X'$ can be chosen in the definition. 
This is the case if $S$ is noetherian
of dimension $\leq 1$ and $X$ is reduced separated
according to \cite[Tag 0ADD]{St}.

}
\end{sremarks}


\begin{slemma} \label{lem_versal} Let  $G \to \GL(\cE)$  be a faithful linear representation of
$G$ where $\cE$ is a locally free $R$-module of finite rank.

\smallskip

\noindent (1) The fppf quotient  $X= \GL(\cE)/G$ is representable
by an algebraic space  and 
 the quotient map $\GL(\cE) \to X$ is a (4)-versal $G$--torsor with right $G$-action.

\smallskip

\noindent (2) Assume that the vector $S$--group $\VV(\cE)$
(defined in the notation) admits an open universally dense $S$-subscheme $V$  which is $G$--stable  and such that $G$ acts freely on $V$. Let $Y=V/G$ be the quotient algebraic space.
Then $V \to V/G$ is a (4)--versal $G$--torsor.

\end{slemma}


\begin{proof}
\noindent(1)
If $R$ is noetherian, then the fppf quotient
$X=\GL(\cE)/G$ is represented by an 
$S$--algebraic space \cite[th.\ 3.1.1]{A}.
The usual noetherian reduction trick shows that it is the case in general.
Let $B$ be an $R$--ring satisfying the primitive criterion
and $F$ be a $G$-torsor over $B$.

We assume firstly that $\cE$ is of constant rank $n$ so that 
$\cE \cong R^n$ according to \S  \ref{subsec_LG}.
Consider the exact sequence of pointed sets:
\begin{align*}
1\to G(B)\to \GL_n(B)\to X(B)\overset{\delta}{\to} H^1_{fppf}(B,G)\to H^1_{fppf}(B,\GL_n).
\end{align*}
As $B$ is an  LG ring,  we have  $H^1_{fppf}(B,\GL_n)=1$ from the same references.
Hence the torsor $F$ is a pull-back of $\GL(\cE)\to X$ at some element $x\in X(B)$.

Let $U$ be an open retrocompact subspace of $X$  that is
schematically $S$--dense in $X$. 
We have the following commutative diagram
$$\begin{CD}
U\underset{X,f_x}{\times}\GL_{n,B} @>>>  \GL_{n,B}\\
@VVV    @VV f_x V\\
U_B @>>> X_B
\end{CD}
$$
The algebraic space
$U\underset{X,f_x}{\times} \GL_{n,B}$ is representable 
by an open retrocompact subscheme $V_x$ of $\GL_{n,B}$ which is $S$-dense according to base change property listed in \ref{subsec_dominant}. 
Since $\GL_{n, R}$ is $S$-dense in the affine space $\VV(\Mat_n(R))$,
 $V_x$ is $S$-dense in $\VV(\Mat_n(R))$,
According to \cite[prop.\ 1.4]{GN}, we have that $V_x(B) \not = \emptyset$.
In other words, there exists $g \in \GL_n(B)$ such that 
$g\cdot x$ is in the image of $U(B)$.
 Since $\delta(x)=\delta(g\cdot x)$, $F$ is a pull back of $\GL(\cE)\to X$ at  $g\cdot x\in U(B)$.

The reduction to the constant rank case is standard.
Since the rank function is locally constant we have
a decomposition $R=R_1 \times \dots \times R_c$ such that
$\cE_{R_i}$ has constant rank $n_i$ for $i=1,...,c$.
The point is that each $R_i$ is an LG ring and similarly
each $B_i=B \otimes_R R_i$ satisfies the primitive criterion 
\cite[Ex.\ 1.2.(b)]{GN}.
Since $G(B)= G(B_1) \times \dots \times G(B_c)$, 
the first case is enough to complete the proof.

\smallskip

\noindent (2)
Let $B$ be a $R$--ring satisfying the primitive criterion.
Let $U$ be an open retrocompact subspace of $V/G$ which is 
$S$--schematically dense.
We consider the $G$--torsor $P= V \times_{V/G} U$
over $U$ and want to show  that it is 
(4)-versal, that is, for a given  $G$-torsor  $F$ over $B$
to find $u \in U(B)$ such that $P_u  \cong F$.
The  Chinese Remainder theorem 
 implies that $V(B) \not = \emptyset$.
We pick  $v\in V(B)$, we can define a right $G$-equivariant morphism $f_v$ from 
$\GL(\cE)$ to $\VV(\cE)$ by 
$$f_v(\sigma)=v\circ\sigma,$$
where  $\sigma$ is an element of  $\GL(\cE)(C)$ and $C$ is a $B$-algebra. We have then a commutative diagram of $G$-torsors
$$\begin{CD}
\GL(\cE)_B@>f_v>>  V_B\\
@V{q}VV    @V{q'}V V\\
\GL(\cE)_B/G_B @>\overline{f}_v>> (V/G)_B.
\end{CD}
$$
The point is that 
$\overline{f}_v^{-1}(U_B)$ is  retrocompact and 
schematically $B$-dense in $X_B=\GL(\cE)_B/G_B$
according to the preliminaries of \ref{subsec_dominant}.

According to  (1) applied to $B$, the 
$G$-torsor $\GL(\cE)\to X$ is a (4)-versal torsor.
There exists $x \in \overline{f}_v^{-1}(U_B)$ such that 
$q^{-1}(x)\cong F$. We put $u= \overline{f}_v(x)$
and conlude that $P_u={q'}^{-1}(u) \cong F$.
Thus $V\to V/G$ is also a (4)-versal torsor.
\end{proof}

In this paper we shall use only the next special case.

\begin{slemma}\label{lem_versal_finite}
Let $G$ be be an $S$--group scheme that is  finite locally free. Let $\cE$ be a $G-R$-module which is locally free of finite rank as $R$--module.
Let $U \subset \VV(\cE)$ be a $G$--stable 
retrocompact $S$-dense open subscheme such that $G$ acts freely on
$U$. Then  $U/G$ is representable by an $S$--scheme $X$ and the quotient map $U \to X$
is a versal (4)--torsor.
\end{slemma}


\begin{proof} According to \cite[III.2.6.1, Corollaire]{DG}, $U/G$ is representable
by an $S$-scheme $X$ and $U \to X$ is a $G$--torsor. Lemma \ref{lem_versal}
shows that $U \to X$ is a versal (4)--torsor.
\end{proof}



\begin{sremark}{\rm
In this setting, it is not clear whether (4)-versal torsors always exist.
}
\end{sremark}


\subsection{Embbedings in quasi-split groups}
In this section, we assume that $S$ is the spectrum of 
a ring $R$ satisfying the primitive criterion.
Let $G$ be a quasi-split reductive $S$--group scheme.

Let $(B,T)$ be a Killing couple of $G$. 
We identify the Weyl group $W(T)$ with $N_G(T)/T$ and denote by $\Dyn(G)$
the Dynkin scheme (which is finite \'etale over $S$). We have a canonical isomorphism
$\Dyn(\Phi(G,T)) \simlgr \Dyn(G)$.

According to \cite[XXIV.3.13]{SGA3}, if $G$ is simply connected, we have an isomorphism $T \simlgr R_{\Dyn(G)/S} (\GG_{m, \Dyn(G)})$.
We denote by $T^{reg}$ the open subscheme of regular elements of $T$.
Then  $W(T)$ stabilizes $T^{reg}$ and acts freely on it.
According to \cite[III.2.6.1, Corollaire]{DG}, the fppf sheaf $T^{reg}/W(T)$ is representable
and the map \break $\pi:T^{reg} \to T^{reg}/W(T)$ is  a $W(T)$--torsor.

Denote by $G//G$ the adjoint quotient. We use now  Chevalley's theorem \cite{L2} in this setting
$u: T/W(T) \simlgr G//G$.
Away of type $A_{2n}$, we can deal with
the Steinberg cross section $c: G//G \to G$ of the quotient map  $G \to G//G$ \cite[th.\ 5.7]{L2}; in that case
we can associate to an element
 $x \in (T^{reg}/W(T))(S)$  the semisimple regular  element
 $g =c( u(x)) \in G(S)$  giving rise to the maximal  $S$--subtorus $T_x=C_G(g)$ of $G$.
The following generalizes to the LG 
setting a result of Gille-Raghunathan \cite{Gi2004,Rg}. 



\begin{stheorem} \label{thm_semi_local} Let $S$ be the spectrum of 
a ring $R$ satisfying the primitive criterion.
Let $G$ be a quasi-split reductive $S$--group scheme.
Let $\Psi$ be a root data which is admissible for
$G$ and let $v \in \uIsomext( G, \Psi)(S)$ be an orientation. 
Then $\gE(G,\Psi,v)(S) \not = \emptyset$.
\end{stheorem}

\begin{sremark}\label{rem_finite} {\rm The statement holds also in the case $S$ is the spectrum
of a  finite field. Thus is obvious since all homogeneous $G$-spaces have a $k$-rational point \cite[III.2.4]{Se}.
 }
\end{sremark}


\begin{proof}
Using the same kind of reduction as in the proof of Lemma 
\ref{lem_versal}.(1) we can assume without loss of generality that $G$ is of constant type.
It is harmless to assume furthermore that 
$G$ is semisimple simply connected. Furthermore using the 
decomposition \cite[XXIV.5.5]{SGA3} and  the usual Weil restriction 
argument, we can assume
that the geometric fibers of $G$ are almost simple (note that a finite \'etale extension of $R$
satisfies the primitive criterion in view of \cite[Ex.\ 1.2.(b)]{GN}).
Let $(B,T)$ be a Killing couple of $G$. We denote  by $W(T)=N_G(T)/T$ the Weyl group and by $\Dyn(G)$
the Dynkin scheme (which is finite \'etale).


\smallskip

Let $c$ be the orientation coming from the natural inclusion $T\hookrightarrow G$, and denote by $u\in\uIsomext(\Phi(G,T),\Psi)(S)$ the image of $(c,v)$
 under the morphism in Prop.\ \ref{pairing}.
We consider the  left $W$--torsor $J=\uIsomint_u(\Phi(G,T),\Psi)$.
According to Lemma \ref{lem_compatibilities},
we have $\gE(G,\Psi,v)=J  \land^W\gE(G,\Phi(G,T),c)$.

By Lemma \ref{lem_versal_finite}, $\pi: T^{reg} \to  T^{reg}/W$ is a (4)-versal (left) $W(T)$--torsor.
In particular there exists $x \in (T^{reg}/W)(S)$ such that
$\pi^{-1}(x) \cong J$ (as left $W(T)$--torsors).

\medskip

\noindent
\textbf{Case 1.}  $G$ is  not of $A_{2n}$-type.
By Steinberg cross section, there is a semisimple regular element $t\in G(S)$ mapped to $x$.
The centralizer of $t$ is a maximal torus $T'$ of $G$ over $S$.
Let $c'\in\uIsomext(G,\Phi(G,T'))(S)$ be the canonical orientation induced by the natural inclusion from $T'$ to $G$.
Denote by $u'\in\uIsomext(\Phi(G,T),\Phi(G,T'))(S)$ be the image of $(c,c')$ under 
the pairing in Prop.\ \ref{pairing}.
By Prop.\ \ref{W-torsor} the  left $W(T)$-torsors $\pi^{-1}(x)$ and $\uIsomint_{u'}(\Phi(G,T),\Phi(G,T'))$ are isomorphic.
It follows that  $J \land^W \gE(G,\Phi(G,T),c) =\gE(G,\Psi,v)$
is isomorphic to $\pi^{-1}(x)\land^W\gE(G,\Phi(G,T),c)  \cong  \gE(G,\Phi(G,T'),c')$.
As  $\gE(G,\Phi(G,T'),c')(S)  \not = \emptyset$, we conclude that
$\gE(G,\Psi,v)(S)$ is not empty.

\medskip

\noindent
\textbf{Case 2.} $G$ is of $A_{2n}$-type. 
%First note that for type $A$, we have a Cartan involution defined over $R$, hence if we  have an embedding with respect to one orientation, we have an embedding with respect to the other orientation.
%Hence it suffices to show that $\gE(G,\Psi)(R)\neq\emptyset$.
The key point is to reduce to the case  of type $A_{2n+1}$. 
Let $R_\sharp/R$ be the \'etale quadratic algebra corresponding to the outer type of 
$G$. Then $R_\sharp/R$ splits $G$ and we denote by $\sigma$ its canonical involution. 

We consider the $R_\sharp$-module $V= (R_\sharp)^{2n+1}=V_{-} \oplus R_\sharp \oplus V_{+}$
with $V_{-}=(R_\sharp)^{n}$ and $V_{+}=(R_\sharp)^{n}$.
It is equipped with  the hermitian form $h(x,y)= \sigma(x_0) \, y_0+ \sum\limits_{i=1,..,n} \sigma(x_{-i})\,  y_i$ with the notation $x=(x_i)_{i=-n,..,n}$.
Denoting by $\HH_n$ the standard hyperbolic space of rank $2n$, we
have an orthogonal decomposition $h= \HH_n \perp \langle 1 \rangle$.

The unitary group scheme $U(V,h)$ is reductive 
and its derived group $SU(V,h)$
is a semisimple simply connected $R$--group scheme
of type  $A_{2n+1}$ \cite[prop.\ C.6]{GN}. We consider the cocharacter $\lambda: \GG_{m,R} \to U(V,h)$
defined by $\lambda(t)= \mathrm{diag}(t^{-n},t^{-n+1}, \dots, t^{-1}, 1, t, \dots, t^{2n-1}, t^{2n})$.
The associated parabolic $R$--subgroup $P=P_{U(V,h)}(\lambda)$ 
carries the Levi subgroup $C_{U(V,h)}(\lambda)$ which is the  $R$--torus
$E=U(V,h) \cap R_{R_\sharp/R}( \GG_m^{2n+1})\cong \GG_m \times_R R_{R_\sharp/R}(\GG_m^n)$ so that $P$ is an 
$R$--Borel subgroup of $U(V,h)$. 
In view of \cite[\S 2.12]{GN}, the $R$--group scheme $U(V,h)$ is then quasi-split.
By considering $E$, we see that the outer type of $SU(V,h)$ is 
$[R_\sharp] \in H^1(R, \ZZ/2\ZZ)$
so that $G \cong SU(V,h)$. 
The group $G$ carries then a Cartan involution defined over $R$, hence if we  have an embedding with respect to one orientation, we have an embedding with respect to the other orientation.
Hence it suffices to show that $\gE(G,\Psi)(R)\neq\emptyset$.

Since $U(V,h)$ and $G=SU(V,h)$ have same automorphism group, 
same scheme of tori and same isomorphism classes of root data,
we can work with $\widetilde G=U(V,h)$.
Let $\gS_{i}$ be the symmetric group of the set $\{1,...,i \}$. We have a natural injection from $\gS_{2n+1}$ to $\gS_{2n+2}$ and hence an injective homomorphism $\iota$ from
$\gS_{2n+1}\times\ZZ /2 \ZZ$ to $\gS_{2n+1}\times \ZZ/ 2\ZZ$ 
that maps $\ZZ/2\ZZ$ isomorphically to itself.
The map $\iota$ induces a map $\iota_{\ast}: H^1_{\acute{e}t}(R,\gS_{2n+1}\times \ZZ/2\ZZ )\to H^1_{\acute{e}t}(R,\gS_{2n+2}\times\ZZ/ 2\ZZ)$.
Note that $\gS_{2n+1}\times\ZZ /2 \ZZ$ is isomorphic to the Weyl group of the split root datum of $\GL_{2n}$. 
Hence $H^1_{\acute{e}t}(R,\gS_{2n+1}\times \ZZ/2\ZZ)$ classifies those root data that are \'etale locally isomorphic to the split root datum of $\GL_{2n+1,R}$.

Thus $\Psi=(\cM,\cM^\vee,\cR,\cR^\vee)$ corresponds to some   $[\alpha] \in H^1_{\acute{e}t}(R,\gS_{2n+1}\times \ZZ/2\ZZ )$.
We twist the split root datum of $\GL_{2n+2,R}$ by $\iota_\ast(\alpha)$ and denote it by $\Psi'=(\cM',\cM'^\vee,\cR',\cR'^\vee)$.
Let $T$ and $T'$ be the tori determined by $\cM$ and $\cM'$ respectively.
Note that by our construction $\cM'^\vee$ can be written as $\cM^\vee\oplus \cE^\vee$ with $\cE$ is a $fppf$ twisted form of $\ZZ$ in such a way that  $R^\vee$ is injectively sent to a subset of ${R'}^\vee$.
Fix such an isomorphism $\cM'^\vee\simeq\cM^\vee\oplus \cE^\vee$. This in turns gives an isomorphism $T'\simeq T\times \bR^{(1)}_{R_\sharp/R}(\bG_m)$.
We denote by $T_0$ the subtorus $1\times \bR^{(1)}_{R_\sharp/R}(\bG_m)\subseteq T'$ (under the above isomorphism).

%Write the Witt decomposition of  $V$ as $\langle a \rangle \perp \H_n$, where $\H_n$ is the maximal hyperbolic Hermitian subspace of $V$.
%Let $(V',h')=(V,h)\perp \langle -a \rangle$.
We put $V'= V \oplus R^\sharp$ and $h'\bigl( (v,r), (w,s) \bigr)
= h(v,w) - \sigma(r) s$. We have an orthogonal decomposition $(V',h')= \HH(V_{-}) \oplus (R_\sharp^2, h'')$
where $h''((r_1,s_1), (r_2,s_2))=  {\sigma(r_1) s_1 -\sigma(r_2) s_2}$.
Since $(1,1)$ is an isotropic unimodular vector
of  $h''$, we know that $h''$ is hyperbolic in view of \cite[prop.\ 3.7.1]{K}.
Since $R_\sharp$ is a LG ring, the locally free $R_\sharp$--modules of constant rank are free so that $h'' \cong \HH_1$ and $h' \cong \HH_{n+1}$.
The $R$--group scheme $U(V',h')$ is then a
quasi-split semisimple $R$--group scheme of type $A_{2n+1}$.


By our assumption, there is an orientation between $\Psi$ and $G$. 
Hence $\Psi$ and $G$ are both become inner forms after base change to $R$.
By our construction of $\Psi'$ and $G'$, they are also both of inner form after base change to $R$.
Hence there is an orientation $v'$ between $\Psi'$ and $G'$. 
Since $G'$ is quasi-split of type $A_{2n+1}$, by Case 1 we have $\gE(G',\Psi',v')(R)\neq\emptyset$.

Choose $f\in \gE(G',\Psi',v')(R)$. 
Consider the subgroup $f(T_0)$ in $G'$.
Let $C$ be the centralizer $\uCentr_{G'}(f (T_0))$, which is a Levi subgroup of $G'$. 
Clearly $f(T')\subseteq C$.
Let $V_0$ be the maximal $R$-submodule of $V$ fixed by $f(T_0)$. Namely, $V_0$ is the maximal $R$-submodule on which $f(T_0)$ acts trivially.
Since the action of $f(T_0)$ is $R_\sharp$-linear, 
$V_0$ is furthermore an  $R_\sharp$-submodule of $V$.
It can be checked  \'etale locally that $V=V_0 \perp V_0^\perp$
and that the $R_\sharp$--module $V_0$ is locally free of rank $2n+1$.
Let $h_0$ be the restriction of $h'$ on $V_0$.

\begin{claim} $h_0 \cong \HH_n \perp \langle a \rangle$ for some $a \in R^\times$.
\end{claim}

Since $R_\sharp$ is a LG ring, the $R_\sharp$--module $V_0$ is free of rank $2n+1$
and $V_0^\sharp$ is free of rank $1$.
We have then an orthogonal decomposition $\HH_{n+1}=h'= h_0 \perp \langle - a \rangle$
for some $a \in R^\times$. The same argument as above
shows that  $\langle a \rangle \perp \langle - a \rangle \cong\HH_1$
so that $h_0 \perp \HH \cong \HH_{n+1} \perp \langle a \rangle$.
The cancellation property \cite[App.\ 5.5]{GN} yields that
$h_0 \cong \HH_n \perp \langle a \rangle$ as desired.
The Claim is established.


The same reasoning as above shows that $U(V_0,h_0)$ 
is the quasi-split $R$--group scheme of type $A_{2n}$ of
Tits invariant $[R_\sharp] \in H^1(R,\ZZ/2\ZZ)$.
In other words we have $U(V_0,h_0)\simeq U(V,h)=\widetilde G$.  

Clearly $C$ stabilizes $V'$ and hence induces an action of $C/f(T_0)$ on  $(V',h')$.
This gives a group homomorphism of $C/f(T_0)$ to $U(V_0,h_0)$, which is an isomorphism at each geometric point $\ol{s}$ of $S$.
Therefore $C/f(T_0)\simeq U(V_0,h_0)$ and $f$ gives an embedding of $T\simeq T'/T_0$ in $\widetilde G$ with respect to $\Psi$.
\end{proof}


\begin{scorollary} \label{cor_main} Let $S$ be a scheme 
that is the spectrum of a ring $R$ satisfying the primitive criterion.
Let $G$ be a quasi-split reductive $S$--group scheme.
Let $X$ be an $S$-scheme
satisfying one of the following conditions:

\smallskip 

(i) $S$ is noetherian;

\smallskip

 (ii) $X$ is locally of finite type over $S$.

\smallskip

We assume that $X$ is a left homogeneous space under $G$
and that for each point $s \in S$, the stabilizers of $X_{\ol s}$ are maximal tori  of $G_{\ol s}$.
Then $X \cong G/T$ where $T$ is a maximal $S$--torus of $G$.
\end{scorollary}

\begin{proof}
It follows immediately from Theorem \ref{thm_main} and Theorem \ref{thm_semi_local}. 
\end{proof}


\begin{sremark}\label{rem_finite2} {\rm The statement holds also in the case $S$ is the spectrum
of a  finite field in view of Remark \ref{rem_finite}.
 }
\end{sremark}


\subsection{An application to the local-global principle for embeddings}

In this section we assume that $k$ is a global field with characteristic not $2$.
Let $K$ be a field, $A$ be a central simple algebra over $K$ with
involution $\tau$, and $E$ be an \'etale algebra over $K$ with
involution $\sigma$. Suppose that $\tau|_K=\sigma|_K$ and that $k$ is
the field of invariants $K^{\tau}$.

Let $\dim_L A=n^2$.
Assume that $\dim_K E= n$, and that if $K\neq k$, then $\dim_k E^\sigma=n$.
If $K=k$, then we assume that $\dim_k E^\sigma =[\frac{n+1}{2}]$. 

The local-global principle for embeddings of $K$-algebras with involution $(E,\sigma)$ into $(A,\tau)$ is first considered in \cite{PR1}.
One can associate a root datum $\Psi$ to $(E,\sigma)$ and a reductive group $U(A,\tau)^\circ$ to $(A,\tau)$, where $U(A,\tau)^\circ$ is the connected component of the unitary group $U(A,\tau)$. (See \cite{L1} \S 1.3.2.)
In \cite{L1} \S 1.3.2 and \cite{BLP1} \S 1.2 and 1.3, it is shown that there is a bijection between the embeddings of algebras with involutions over $k$ and  the embeddings of corresponding root data into reductive groups.

Combining with Theorem \ref{thm_semi_local}, we have the following theorem.
\begin{theorem}\label{embed_algebras}
Let $(E,\sigma)$ and $(A,\tau)$ be as above.  If $U(A,\tau)^\circ$ is quasi-split, then the local-global principle holds for the $K$-algebra embeddings of $(E,\sigma)$ into $(A,\tau)$.
\end{theorem}
\begin{proof}
Let $\Psi$ be the root datum associated to $(E,\sigma)$ (\cite{L1} \S 1.3.2).
Let $G=U(A,\tau)^\circ$. By Theorem \ref{thm_semi_local}, it suffices to show that $\uIsomext(G,\Psi)(k)\neq \emptyset$.

First we consider the case where $K\neq k$.
In this case, we can always fix an orientation by \cite{L1} Theorem 1.15 (2) and Remark 1.16.


Next we consider the case where $K=k$.
If $\tau$ is symplectic involution or $\tau$ is orthogonal with $n$ odd, then $G$ is of type $C_{n/2}$ or $B_{(n+1)/2}$.
Then there is always an orientation between $\Psi$ and $G$.

Consider the case that $\tau$ is orthogonal with $n$ even.
Suppose there is  an embedding $\eta_v$ of $(E\times_k k_v,\sigma\otimes \id_{k_v})$ into $(A\times_k k_v,\sigma\otimes\id_{k_v})$ for all places $v$ of $k$.
Then $\eta_v$ gives an isomorphism between the discriminant $\Delta(E)$  of $E$ and the center $Z(A,\tau)$ of the Clifford algebra of $(A,\tau)$ over $k_v$ (see \cite{BLP2} \S 2.3).
As they are both quadratic \'etale algebras over $k$, this means that $\Delta(E)$ is isomorphic to $Z(A,\tau)$ over $k$.
By \cite{BLP1} Proposition 1.3.1, an isomorphism between $\Delta(E)$ and $Z(A,\tau)$ gives an orientation between $\Psi$ and $G$.

Hence for $K=k$, we can always fix an orientation between $\Psi$ and $G$.
\end{proof}


 \bigskip

\begin{thebibliography}{99}

\bibitem{A}  S.~Anantharaman, {\it Sch\'emas en groupes, espaces homogènes et espaces alg\'ebriques sur une base de dimension 1},
  M\'emoires de la Soci\'et\'e Math\'ematique de France {\bf 33} (1973), 5-79. 


\bibitem{BLP1} E.~Bayer-Fluckiger, T.-Y.~Lee and R.~Parimala, {\it Embedding functor for classical groups and Brauer-Manin obstruction},
Pac. J. Math. \textbf{279} (2015), 87-100.

 \bibitem {BLP2} E.~Bayer-Fluckiger, T.-Y.~Lee, and R.~Parimala, \textit{Embeddings of maxi-mal tori in classical groups and explicit Brauer-Manin obstruction}, 
 J. Eur. Math. Soc. {\bf 20} (2018), 137-163.
 
 
 
\bibitem{Bo}  A.~Borel, {\it Linear algebraic groups}, Graduate Texts in Mathematics {\bf  126} (2nd ed.),
Berlin, New York: Springer-Verlag.

\bibitem{BLR} S.~Bosch, W.~L\"utkebohmert, M.~Raynaud,
{\it N\'eron models}, Ergebnisse der Mathematik und ihrer Grenzgebiete
{\bf 21} (1990), Springer.




\bibitem{BT} F.~Bruhat, J.~Tits, {\it
Groupes r\'eductifs sur un corps local : II. Sch\'emas en groupes.
Existence d'une donn\'ee radicielle valu\'ee},
Publications Math\'ematiques de l'IH\'ES {\bf 60} (1984), 5-184.


\bibitem{CF} B.~Calm\`es, J.~Fasel, {\it Groupes classiques},
     Autour des sch\'emas en groupes, vol II, Panoramas et Synth\`eses {\bf 46} (2015),    1-133.

%\bibitem[CTS]{CTS} J.--L. Colliot--Th\'el\`ene, J.--J. Sansuc, {\it Fibr\'es
%quadratiques et composantes connexes r\'eelles}, Math. Annalen {\bf
%244} (1979), 105--134.



%\bibitem[Co]{Co} B.~Conrad,  {\it Non-split reductive groups over
 %   $\mathbb{Z}$},
  %   Autour des sch\'emas en groupes, vol II, Panoramas
   %  et Synth\`eses {\bf 46} (2015), 193-253.



%\bibitem{CGP} B. Conrad, O. Gabber, G. Prasad, {\it
%Pseudo-reductive groups},   Cambridge University Press, second edition (2016).

%\bibitem[CLO]{CLO} B. Conrad, M. Lieblich, M. Olsson, {\it
 %Nagata compactification for algebraic spaces},
  %J. Inst. Math. Jussieu {\bf 11} (2012), 747-814.


%\bibitem[DS]{DS} V. G. Drinfeld, C. Simpson,  {\it  $B$-structures on $G$-bundles and local triviality},
%Mathematical Research Letters {\bf 2} (1995), 823-829.


\bibitem{DG} M.~Demazure, P.~Gabriel,
{\it Groupes alg\'ebriques}, Masson (1970).


\bibitem{EG} D.~R.~Estes and R.~M.~Guralnick, {\em Module equivalences: Local to global when primitive polynomials represent units}, J.~Algebra \textbf{77} (1982), 138-157.


\bibitem{GMS} S.~Garibaldi, A.~Merkurjev and J.-P.~Serre, {\it
    Cohomological invariants
 in Galois cohomology}, University Lecture Series \textbf{28} (2003),
American Mathematical Society.


\bibitem{GPR}  S.~Garibaldi,  H.P.~Petersson, M.~Racine, 
{\it  Albert algebras over commutative rings}, to appear
in  Cambridge University Press as New Mathematical Monographs vol.\ {\bf 48} (2024).


\bibitem{EGA1} A.~Grothendieck, J.-A.~Dieudonn\'e, {\it El\'ements
de g\'eom\'etrie alg\'ebrique. I}, Grundlehren der Mathematischen Wissenschaften  166; Springer-Verlag, Berlin, 1971.


%\bibitem[EGAII]{EGA2} A.\ Grothendieck (avec la collaboration de J.\ Dieudonn\'e),
%{\it El\'ements de G\'eom\'etrie Alg\'ebrique II}, Publications
%math\'ematiques de l'I.H.\'E.S.  no 8 (1961).

%\bibitem[EGAIII]{EGA3} A.\ Grothendieck (avec la collaboration de J.\ Dieudonn\'e),
%{\it El\'ements de G\'eom\'etrie Alg\'ebrique II}, Publications
%math\'ematiques de l'I.H.\'E.S.  no 11 and 17 (1961-1963).

\bibitem{EGA4} A.~Grothendieck (avec la collaboration de J.~Dieudonn\'e),
{\it El\'ements de G\'eom\'etrie Alg\'ebrique IV}, Publications math\'ematiques de l'I.H.\'E.S. no 20, 24, 28 and 32 (1964 - 1967).



%\bibitem[FGA]{FGA} A.\ Grothendieck, {\it
 %Fondements de la g\'eom\'etrie alg\'ebrique}, (Extraits du S\'eminaire Bourbaki,
 %1957-1962), Paris, Secr\'etariat math\'ematique, 1962.

% \bibitem{Gi} P. Gille, {\it Groupes alg\'ebriques semi-simples en dimension cohomologique $\leq 2$},
%Lecture Notes in Mathematics {\bf 2238} (2019), Springer.


\bibitem{Gi2004} P.~Gille, {\it  Type des tores maximaux des groupes semi-simples}, J. Ramanujan Math. Soc. {\bf 19} (2004), 213-230.


\bibitem{Gi2015} P.~Gille, {\it Sur la classification des
sch\'emas en groupes semi-simples}, ``Autour des sch\'emas en groupes, III'',
Panoramas et Synth\`eses {\bf 47} (2015), 39-110.


\bibitem{GN} P.~Gille, E.~Neher, {\it Group schemes over LG-rings  and applications to cancellation theorems and Azumaya algebras}, 
https://hal.science/hal-04631256

%\bibitem[G]{G} A. Grothendieck, {\it Technique de descente et th\'eor\`emes d'existence en g\'eom\'etrie
% alg\'ebrique.  IV. Les sch\'emas de Hilbert},  S\'eminaire Bourbaki, tome  13 (1960/61), no. 221.


%\bibitem[Gi]{Gi} P. Gille, {\it Sur la classification des sch\'emas
%en groupes semi-simples}, ``Autour des sch\'emas en groupes, III'',
%Panoramas et Synth\`eses {\bf 47} (2015), 39-110.

\bibitem{Gir} J.~Giraud,
{\em Cohomologie non-ab\'elienne}, Springer (1970).



\bibitem{G}  A.~Grothendieck, {\it Le groupe de Brauer. I. Alg\`ebres d'Azumaya et interpr\'etations diverses}, Dix expos\'es sur la cohomologie des sch\'emas, 46-66,
Adv. Stud. Pure Math., 3, North-Holland, Amsterdam, 1968.


%\bibitem[Ha]{Ha} R. Hartshorne,  {\it  Algebraic Geometry}, Graduate Texts in Mathematics, Springer.


%\bibitem[Hd]{Hd} G.Harder, {\it
%Halbeinfache Gruppenschemata \"uber vollst\"andigen Kurven},
%Inventiones Mathematicae, {\bf 6} (1968), 107-149.

%\bibitem[He]{He} J. Heinloth, {\it  Uniformization of $G$-bundles},
% Math. Annalen {\bf 347} (2010),  499-528.

%\bibitem[Hd]{Hd} G. Harder, {\it Halbeinfache Gruppenschemata \"uber vollst\"andigen %Kurven}, Inventiones mathematicae {\bf 6} (1968), 107-149.


%\bibitem{KO} M.-A. Knus, M. Ojanguren, {\em Th\'eorie de la Descente et
 %   Alg\`ebres d'Azumaya}, Lecture Notes in Mathematics \textbf{389} (1974),
  %  Springer.
  
  
\bibitem{K} M.-A.~Knus, {\it Quadratic and Hermitian Forms over
    Rings}, Grundlehren der mathematischen Wissenschaften {\bf 294}
    (1991), Springer.


%\bibitem[I]{I} L. Illusie, {\it Grothendieck's existence
%theorem in formal geometry}, with a letter  of
%Jean-Pierre Serre, Math. Surveys Monogr. {\bf  123} (2005),
%Fundamental algebraic geometry,
%179-233, Amer. Math. Soc., Providence, RI.

%\bibitem[L]{L} A. Langer, {\it Semistable principal $G$-bundles in positive characteristic},
% Duke Math. J. {\bf 128} (2005), 511-540.

%\bibitem[LMB]{LMB} G. Laumon, L. Moret-Bailly, \emph{Champs alg\'ebriques}, Ergebnisse der Mathematik und ihrer Grenzgebiete. 3. Folge. A Series of Modern Surveys in Mathematics, 39. Springer-Verlag, Berlin, 2000.


\bibitem{L1} T.-Y.~Lee, {\it Embedding functors and their arithmetic properties},
 Comment. Math. Helv. {\bf 89} (2014),  671-717.

\bibitem{L2} T.-Y.~Lee, {\it Adjoint quotients of reductive groups},
 Panoramas et  Synth\`eses {\bf 47} (2015), 131-145.


%\bibitem{M} B. Margaux, {\it  Formal torsors under reductive group schemes},
%Rev. Un. Mat. Argentina {\bf 60} (2019), 217-224.

\bibitem{M} B.~Margaux, {\it 
 Vanishing of Hochschild cohomology for affine group schemes and rigidity of homomorphisms between algebraic groups}, 
Doc. Math. {\bf 14} (2009), 653-672.


\bibitem{MW} B.~R.~McDonald and W.~C.~Waterhouse, {\em Projective modules over rings with many units}, Proc.~Amer.~Math.~Soc.~\textbf{83}(3) (1981), 455-458.

\bibitem{O} M.~Olsson, {\it Algebraic spaces and stacks}, American Mathematical Society Colloquium Publications, 62. American Mathematical Society, Providence, RI, 2016.

\bibitem{PR1} G.~Prasad and A.S.~Rapinchuk, \textit{Local-Global principles for embedding of fields with involution into semple algebras with involution},
 Comment. Math. Helv. \textbf{85} (2010), 583-645.

\bibitem{Rg} M.S.~Raghunathan, {\it
Tori in quasi-split-groups},
J. Ramanujan Math. Soc. {\bf 19} (2004), 281-287.

\bibitem{R} M.~Raynaud, {\it  Faisceaux amples sur les sch\'emas en groupes et les espaces homog\`enes},
 Lecture Notes in Mathematics {\bf  119} (1970), Springer-Verlag, Berlin-New York.


 \bibitem{RT} Z.~Reichstein, D.~Tossici, {\it
Special groups, versality and the Grothendieck-Serre conjecture}, preprint (2019),
 arXiv:1912.08109

\bibitem{Ry1}  D.~Rydh, {\it Submersions and
effective descent of \'etale morphism},
Bull. Soc. math. France
{\bf 138}(2010)  181-230.

\bibitem{Ry2}  D.~Rydh,  {\it Approximation of Sheaves on Algebraic Stacks}, 
International Mathematics Research Notices (2016),  717-737.



%\bibitem{SGA1} {\it S\'eminaire de G\'eom\'etrie
%alg\'ebrique de l'I.H.\'E.S., 1960-1961,
%Rev\^etements \'etales et groupes fondamental, dirig\'e par  A.\ Grothendieck},  Lecture Notes in Math. 151-153. Springer (1970).

\bibitem{SGA3} {\it S\'eminaire de G\'eom\'etrie
alg\'ebrique de l'I.H.\'E.S., 1963-1964, Sch\'emas en groupes, dirig\'e par M.\ Demazure et A.\ Grothendieck},  Lecture Notes in Math. 151-153.
Springer (1970).

%\bibitem[Se]{Se} J.P. Serre,   {\it Groupe de Grothendieck des sch\'emas
%en groupes r\'eductifs d\'eploy\'es},
%Publications Math\'ematiques de l'IH\'ES {\bf  34} (1968),  37-52.

\bibitem{Se} J.-P.~Serre,   {\it Cohomologie
galoisienne}, 5-i\`eme version r\'evis\'ee, Lecture Notes
in Mathematics {\bf 5}, Springer.


\bibitem{St} Stacks project, https://stacks.math.columbia.edu

%\bibitem{T} R. W. Thomason, {\it Equivariant resolution, linearization, and Hilbert's fourteenth problem over arbitrary base schemes},
%Adv. in Math. {\bf 65} (1987),  16-34.


\end{thebibliography}

\bigskip

\medskip

\end{document}

%%%%%%%%%%%%%%%%%%%%%%


\section{Answering Marc Levine's question}
 
The question was the following.

\smallskip

{\it Let $G$ be a split semi-simple algebraic group over a field $k$ of characteristic zero. Let $A$ be a local $k$-algebra. Suppose we have an $A$-linear representation of $G$ on $A^n$. Is this always isomorphic to the base extension of a $k$-linear representation of $G$ on $k^n$?
}


\medskip 

The answer is yes and works actually under slighly weaker assumptions.
We are given an $A$--representation $f: G_A \to \GL_{n,A}$.
Let $F=A / \gm$ be the residue field of $A$ and 
and denote by $f_1 :  G_F \to \GL_{n,F}$ the specialization.
According to Tits classification of the 
irreducible representations of $G$ over $F$,
there exists $u \in   \GL_n(F)$ and 
a $k$--representation $f_0: G \to \GL_n$ such
that $f=^uf_0$. Since $\GL_n(A) \to \GL_n(F)$ is onto
we can then assume without loss of generality that 
$f_1= f_{0,F}$.



According to Margaux's theorem \cite[cor.\ 4.9]{M},  $f$ and $f_0$ are $GL_n(A^{sh})$-conjugated where $A^{sh}$ stands for the 
strict henselization of $A$.  It follows that the 
transporter $$
\Bigl\{ x \in GL_n \mid x f_0 x^{-1} = f \Bigr\}
$$
is a $H:= C_{\GL_n}(f_0)$--torsor. More precisely the kernel of the map $H^1(A,H) \to  H^1(A,\GL_n)=1$ is in bijection with the set of 
 $A$-representations 
$G_A \to  \GL_{n,A}$ up to $\GL_n(A)$-conjugacy which are locally conjugated to f_0 with respect to the \'etale topology.

Since  $H$ is a product of Weil restrictions of $\GL_{n_i}$ 
we have that $H^1(O,H)=1$. Thus $f$ is $\GL_n(A)$--conjugated to $f_{0,A}$.
 


