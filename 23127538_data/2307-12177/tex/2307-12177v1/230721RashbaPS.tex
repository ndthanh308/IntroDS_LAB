\documentclass[aps,pra,twocolumn,superscriptaddress]{revtex4-2}
\usepackage{amsfonts}
\usepackage{amsmath}
\usepackage{mathrsfs}
\usepackage{amssymb}
\usepackage{graphicx}
\usepackage{float} 
\usepackage{dcolumn}
\usepackage{bm}
\usepackage{color}

%\DeclareMathOperator{\sech}{sech}
\usepackage[colorlinks=true,linkcolor=blue,anchorcolor=blue,citecolor=blue,urlcolor=blue]{hyperref}

\begin{document}

\title{Spin-orbit-coupling-induced phase separation in trapped Bose gases}

\author{Zhiqian Gui}
\affiliation{Department of Physics, Shanghai University, Shanghai 200444, China}

\author{Zhenming Zhang}
\affiliation{CAS Key Laboratory of Quantum Information, University of Science and Technology of China, Hefei 230026, China}


\author{Jin Su}
\affiliation{Department of Basic Medicine, Changzhi Medical College, Changzhi 046000, China}

\author{Hao Lyu}
\affiliation{Quantum Systems Unit, Okinawa Institute of Science and Technology Graduate University, Onna, Okinawa 904-0495, Japan}


\author{Yongping Zhang}
\email{yongping11@t.shu.edu.cn}
\affiliation{Department of Physics, Shanghai University, Shanghai 200444, China}

\begin{abstract}
In a trapped spin-$1/2$ Bose-Einstein condensate with miscible interactions, a two-dimensional spin-orbit coupling can introduce an unconventional spatial separation between the two components. We reveal the physical mechanism of such a spin-orbit-coupling-induced phase separation.  Detailed features of the phase separation are identified in a trapped Bose-Einstein condensate.  
We further analyze differences of phase separation in Rashba and anisotropic spin-orbit-coupled Bose gases. 
An adiabatic splitting dynamics is proposed as an application of the phase separation. 
%We further analyze the separation differences between Rashba and an anisotropic spin-orbit coupling. An adiabatic splitting dynamics is proposed as an application of the phase separation. 
	


	
\end{abstract}

\maketitle

\section{Introduction}
\label{introduction}

Phase separation is a generic phenomenon from classical physics to quantum physics, for example, 
the oil-water separation and spin Hall effect~\cite{Sinova}. Two-component atomic Bose-Einstein condensates (BECs) provide a tunable platform for the investigations of phase separation~\cite{Shin,Papp,Thalhammer,Tojo,Nicklas,Ota,HeL}. 
The two components can be realized by using different atomic species or same species with different electric hyperfine states.
Such a system features intra- and inter-component interactions. When the inter-component interactions dominate over the intra-component interactions, two components prefer to be phase-separated in order to minimize the inter-component interactions~\cite{Trippenbach,Petrov2015}.  The interactions for phase separation are called immiscible.  The immiscibility of two-component BECs are completely tunable in experiments. 
Phase separation effect induces rich physics in quantum gases, such as the formation of vector solitons and vortex-soliton structures, 
coherent spin dynamics, and pattern formations~\cite{Shrestha,Law,LeeKL,Roy,Eto,Bernier,Shirley,Mistakidis,Lous}.

In a two-component BEC, an artificial  spin-orbit coupling can be synthesized between different hyperfine states via Raman lasers~\cite{Lin,Goldman,Zhai2015,Zhang2016}. Such a Raman-induced spin-orbit coupling is one dimensional. Rashba spin-orbit coupling, which is two dimensional, has also been experimentally realized in BECs~\cite{WuZ,Valdes}. 
The implementation of spin-orbit coupling in BECs gives rise to exotic quantum phases and rich superfluid properties, 
which opens an avenue for simulating topological matters and exploring superfluid dynamics~\cite{LiY,Manchon,LiJR,Khamehchi,Kartashov,Hasan,Valdes,Frolian}. In a Raman-type spin-orbit-coupled BEC, a stripe phase~\cite{Wang} can exist in miscible interactions~\cite{LiY,LiJR}. In contrast, for a Rashba spin-orbit-coupled BEC, the stripe phase may appear in the immiscible regime~\cite{Ho2011}. Very interestingly, Refs.~\cite{Hu2012,Zhang2012} have numerically found a spatially phase-separated ground state in a Rashba-coupled and harmonically trapped BEC with miscible interactions. Such a state in the miscible regime is unexpected for a usual two-component BEC without spin-orbit coupling. 
Reference~\cite{Hu2012} identifies that the exotic phase separation satisfies a combined symmetry of parity and a spin flip. 
The existence of this state is attributed by Refs.~\cite{Zhang2012,Song2013} to a spin-dependent force.  The force is intrinsic in the presence of  Rashba spin-orbit coupling and drives the two components moving in opposite directions.  The force concept provides an intuitive picture for the unexpected phase separation. However, its weakness is obvious. The force is proportional to the square of Rashba spin-orbit coupling strength. Therefore, a large strength is expected to generate a larger spatial separation. In contrast, numerical results show that the separation decreases with an increasing strength~\cite{Zhang2012,Song2013}.   
So far, the physical origin of the unconventional phase separation in the miscible regime is yet to be addressed.  It calls for an unambiguous interpretation, since the phase separation has already found a broad application in other excited states.  In Ref.~\cite{Xu2015}, a spin-orbit-coupled bright soliton is found to be spatially separated in center-of-mass between the two components. Dynamics of the separation in bright solitons is analyzed by varying spin-orbit coupling strength~\cite{Wang2020}.  A spin-orbit-coupled single-vortex state, in which each component carries a singly quantized vortex, shows spatial separation between two components, and the separation is inversely proportional to spin-orbit coupling strength~\cite{Kato2017}. Recently, dynamics of the separation is triggered by a sudden quench of spin-orbit coupling strength in a trapped BEC~\cite{Ravisankar2021}.
All mentioned separations occur in the miscible regime, causing a counterintuitive expectation. 



In this paper, we provide the physical mechanism for the unconventionally spin-orbit-coupling-induced phase separation.  Eigenstates of a two-dimensional spin-orbit coupling have a momentum-dependent relative phase $\varphi(\vec{k})$ between the two components. Closely around a fixed momentum $\vec{k}_0$, the relative phase may present a linear dependence $\varphi(\vec{k})\propto (\vec{k}-\vec{k}_0)\cdot \vec{r}_0$ with a constant $\vec{r}_0$.  The linear dependence is a momentum kick to move two components relatively. The superposition of these eigenstates distributing around $\vec{k}_0$ constitutes a spatially separated wave packet.  The separation, whose amplitude can be calculated,  is a completely single-particle effect of spin-orbit coupling.  A weakly trapped BEC with two-dimensional spin-orbit coupling is a perfect platform to simulate the phase separation. The miscible interactions force atoms to condense at a certain spin-orbit-coupled momentum state with a momentum-dependent relative phase. Meanwhile, weak traps broaden the condensed momentum so that the condensation occupies momentum states in a narrow regime, which give rise to the linear dependence of the relative phase.  We numerically identify detailed features of spin-orbit-coupling-induced phase separation in a trapped BEC with miscible interactions by analyzing its ground states.  The phase separation matches with the single-particle prediction when spin-orbit coupling strength dominates. We also compare the separation differences between Rashba and an anisotropic spin-orbit coupling. Finally, as an application of the phase separation, we propose an adiabatic splitting dynamics. 





The paper is organized as follows. In Sec.~\ref{mechanism}, the physical mechanism of spin-orbit-coupling-induced phase separation is unveiled.  The separation amplitudes are predicted.  From the mechanism, we know that the separation is a single-particle effect.  In Sec.~\ref{Rashba}, we identify separated features of ground states in a trapped BEC with Rashba spin-orbit coupling by the imaginary-time evolution method and the variational method.   In Sec.~\ref{AnistropicSOC}, we reveal the effect of the anisotropy of spin-orbit coupling on the phase separation.  In Sec.~\ref{Adiabaticdynamics}, we propose an adiabatic dynamics to dynamically split two components basing on the phase separation.  For the completeness of our discussion, immiscible-interaction-induced phase separation is shown in Sec.~\ref{Immiscibilty}.
The conclusion follows in Sec.~\ref{conclusion}.






\section{Spin-orbit-coupling-induced phase separation}
\label{mechanism}

Rashba spin-orbit-coupling-induced phase separation is a completely single-particle effect. We reveal the physical origin of such phase separation.  The Rashba spin-orbit-coupled Hamiltonian is
%
\begin{equation}
\label{Hsoc}
H_\text{SOC}=\frac{p_x^2+p^2_y}{2}+\lambda (p_{x} \sigma_{y}- p_{y} \sigma_{x}),
\end{equation}
where $p_x$ and $p_y$ are the momenta along the $x$ and $y$ directions respectively, $\lambda$ is the spin-orbit coupling strength, and $\sigma_{x,y}$ are spin-1/2 Pauli matrices.  The eigenenergy of the Hamiltonian has two bands. The lower band is 
\begin{equation}\label{Band}
E = \frac{k_x^2 + k_y^2}{2} - \lambda \sqrt{k_x^2 + k_y^2},
\end{equation}
with associated eigenstates being
\begin{equation}\label{Eigenstate}
\Phi = \frac{1}{\sqrt{2}}e^{ik_x x+ik_y y}
\begin{pmatrix}
e^{i\frac{\varphi}{2}} \\ e^{-i\frac{\varphi}{2}}
\end{pmatrix}.
\end{equation}	
Since the Hamiltonian possesses continuously translational symmetry, the eigenstates are plane waves with  $k_{x,y}$ being the quasimomenta along the $x$ and $y$ directions, respectively.  The outstanding feature is that Rashba spin-orbit coupling generates a relative phase $\varphi$ between the two components, which satisfies 
\begin{equation}\label{phi}
\tan(\varphi) = \frac{k_x}{k_y}.
\end{equation}
It is noted that $(k_x,k_y)=(0,0)$ is a singularity, closely around which the relative phase cannot be defined. Therefore, the eigenstate in Eq.~(\ref{Eigenstate}) works beyond the regime around the singularity. 

We construct a wave packet by superposing these eigenstates,
\begin{equation}\label{Packet}
\Psi = \int_{-\infty}^{\infty} dk_x dk_y \mathcal{G}({\mathbf{k}}-\mathbf{\bar{k}} )\Phi,
%\frac{1}{\sqrt{2}}e^{ik_x x+ik_y y}
%\begin{pmatrix}
%e^{i\frac{\varphi}{2}} \\ e^{-i\frac{\varphi}{2}} 
%\end{pmatrix},
\end{equation}
with the superposition coefficient $\mathcal{G}$ being a momentum-dependent localized function centering around $\mathbf{\bar{k}}$. For a straightforward illustration, we take a Gaussian distribution as an example,
\begin{equation}\label{Gaussian}
\mathcal{G} ({\mathbf{k}}- \mathbf{\bar{k}}) = \frac{1}{2\pi \sqrt{\Delta_x\Delta_y}} e^{-\frac{(k_x - \bar k_x)^2}{2\Delta_x} -\frac{(k_y - \bar k_y)^2}{2\Delta_y}}.
\end{equation}		
The Gaussian distributed superposition coefficient is centered at $\mathbf{\bar{k}}=(\bar k_x,\bar k_y)$  with the packet widths $\sqrt{\Delta_{x,y}}$ along $x$ and $y$ directions. If the widths are narrow, the superposition mainly happens around $\mathbf{\bar{k}}$. Therefore, we analyze the eigenstates around $\mathbf{\bar{k}}$, and the relative phase becomes
\begin{equation}\label{Phase}
\varphi({\mathbf{k}} ) \approx \varphi(\mathbf{\bar{k}}) + \left.  (k_x - \bar k_x)\frac{\partial \varphi}{\partial k_x}\right |  _{\mathbf{\bar{k}}} + \left.  (k_y - \bar k_y)\frac{\partial \varphi}{\partial k_y}\right |_{\mathbf{\bar{k}}},
\end{equation}	
which is linearly dependent on the momenta $k_{x,y}$. This is true since expanding any continuous function around a certain 
parameter point leads to dominant linear-dependence.  Such linear dependence in Eq.~(\ref{Phase}) induced a momentum kick, generating the relative motion between the two components. 
After substituting the Gaussian distribution in Eq.~(\ref{Gaussian}) and $\varphi$ in Eq.~(\ref{Phase}) into Eq.~(\ref{Packet}) and performing integration, we get the wave packet,
\begin{align}\label{wavefunction}
\Psi = &\frac{1}{\sqrt{2}} e^{i\bar{k}_x x+i\bar{k}_y y } \notag \\
&\times
\begin{pmatrix}
e^{-\frac{\Delta_x}{2}\left [ x+ \frac{1}{2}\frac{\partial \varphi (\mathbf{\bar{k}} )}{\partial k_x}\right ]^2 -\frac{\Delta_y}{2} \left[ y+\frac{1}{2}\frac{\partial \varphi (\mathbf{\bar{k}})}{\partial k_y} \right]^2 + i\frac{\varphi(\mathbf{\bar{k}} )}{2}} \\
e^{-\frac{\Delta_x}{2}\left [ x- \frac{1}{2}\frac{\partial \varphi (\mathbf{\bar{k}})}{\partial k_x}\right ]^2 -\frac{\Delta_y}{2} \left[ y-\frac{1}{2}\frac{\partial \varphi (\mathbf{\bar{k}} )}{\partial k_y} \right]^2 - i\frac{\varphi(\mathbf{\bar{k}} )}{2}}
\end{pmatrix}.
\end{align}
The outstanding feature of the resultant wave packet is that the two components have a relative position displacement.  The displacements along the $x$ and $y$ directions are
 \begin{equation}
 \left.\frac{\partial \varphi(\mathbf{k} )}{ \partial k_x}\right|_{\mathbf{\bar{k}}}= \frac{\bar{k}_y}{ \bar{k}_x^2+\bar{k}_y^2},\ \ 
\left. \frac{\partial \varphi(\mathbf{k} )}{ \partial k_y}\right|_{\mathbf{\bar{k}}}= -\frac{\bar{k}_x}{ \bar{k}_x^2+\bar{k}_y^2}.
 \label{Shift}
 \end{equation}
The nonzero displacements give rise to a phase separation between two components. From the construction of the phase-separated wave packet, we can see that the origin of the phase separation is the existence of the momentum-dependent relative phase in eigenstates and the occupation of these eigenstates confined in a narrow momentum regime. 

Rashba spin-orbit coupled BEC is an ideal platform to generate such a phase-separated state. The lower band in Eq.~(\ref{Band}) has  infinite energy minima which locate at the quasimomenta satisfying $k_x^2+k_y^2=\lambda^2$; therefore, $k_x=\lambda \cos(\theta)$ and $k_y=\lambda \sin(\theta)$ with $\theta $ being an angle. The interacting atoms spontaneously choose one of energy minima to condense and form a BEC~\cite{Wang, Hu2012, Zhang2012}.
This means that $\theta$ is spontaneously chosen to be a value $\bar{\theta}$.  In real atomic BEC experiments, traps are inevitable. A weak harmonic trap naturally broadens the BEC momentum giving rise to a Gaussian distribution centered at $(\bar{k}_x,\bar{k}_y)=\lambda(\cos(\bar{\theta}), \sin(\bar{\theta}))$. Furthermore, the broadening is narrow so that Eq.~(\ref{Phase}) is satisfied. Consequently, the Rashba-coupled BEC  presents as a phase separated state in Eq.~(\ref{wavefunction}) with $\partial \varphi (\mathbf{k} ) / \partial k_x|_{\mathbf{\bar{k}}}=\sin(\bar{\theta})/\lambda$ and $\partial \varphi (\mathbf{k} )| / \partial k_y|_{\mathbf{\bar{k}}}=-\cos(\bar{\theta})/\lambda$.  The position displacement is inversely proportional to the spin-orbit coupling strength $\lambda$, which clearly indicates Rashba spin-orbit-coupling-induced phase separation. When the strength goes to zero ($\lambda \approx 0$), the momentum $(\bar{k}_x, \bar{k}_y)\approx (0,0)$ becomes a singularity so that the eigenstate in Eq.~(\ref{Eigenstate}) is not physical. Without the spin-orbit coupling, the BEC becomes the conventional one, and there is no phase separation between two components. 
If the strength is enhanced gradually from zero, the position displacements should continuously increase from zero to catch up with the predicted value  $(\sin(\bar{\theta})/\lambda, -\cos(\bar{\theta})/\lambda)$. When the strength $\lambda$ is large enough, the position displacements decrease towards zero again since they are inversely proportional to $\lambda$.   
In this case, the plane-wave phase $(\bar{k}_xx+\bar{k}_yy)$ dominates, while the relative phase in the eigenstates [Eq.~(\ref{Eigenstate})] is independent of $\lambda$, 
i.e., $\tan(\varphi)=\bar{k}_x/\bar{k}_y=\cot(\bar{\theta})$.
Consequently, the effect of the relative phase is obliterated by the plane-wave phase, and phase separation disappears.


According to the above mechanism of the phase separation, if there is no weak trap to broaden the condensed momentum, the spin-orbit-coupled BEC can not present the position displacement.  This is why a spatially homogeneous BEC with spin-orbit coupling does not show phase separation as studied in most literature.  Nevertheless, in order to broaden the condensed momentum without traps, we may consider spatially localized excitation states, such as bright solitons and vortices.  These spatially self-trapped states naturally broadens the condensed momentum. Therefore, the resultant phase separation between two components in Rashba spin-orbit-coupled bright solitons and quantum vortices, which have been numerically revealed in Refs.~\cite{Xu2015,Kato2017}, can be understood by a generalization of our mechanism.  Interestingly, the  position displacement of quantum vortex, which is inversely proportional to the spin-orbit coupling strength as uncovered numerically in~\cite{Kato2017}, can be explained ambiguously. 

We emphasize that the spin-orbit-coupling-induced phase separation only works for a two-dimensional spin-orbit coupling.  For an one-dimensional spin-orbit coupling, i.e., the Raman-induced one, the single-particle Hamiltonian is $H'=p_x^2/2 + \lambda p_x \sigma_z+\Omega \sigma_x$ with $\Omega$ being the Rabi frequency due to Raman lasers~\cite{Lin, Zhang2016}. The lower energy band of this system is $E=k_x^2/2 - \sqrt{\lambda^2k_x^2+\Omega^2}$ with eigenstates being $\Phi=e^{ik_xx}(-\sin(\Theta), \cos(\Theta))^T$. Here, $\tan(\Theta)=\Omega/(\lambda k_x)$, and $T$ is the transpose operator. It is noted that there is no momentum-dependent relative phase in the eigenstates. Therefore, the Raman-induced spin-orbit coupling, in principle, can not generate the phase separation. 

In above, we have revealed the physical 
mechanism of Rashba spin-orbit-coupling-induced phase separation.  We demonstrate that a weakly trapped spin-orbit-coupled BEC satisfies requirements of the mechanism.  Quantum phase in trapped spin-orbit-coupled BECs may be phase separated states. The spin-orbit-coupling-induced phase separation is a single-particle effect. The role of nonlinearity in the BEC is to spontaneously choose one energy minimum for condensation. In the following, we study ground states of a trapped spin-orbit-coupled BEC and identify features of spin-orbit-coupling-induced phase separation. 


\section{Rashba spin-orbit-coupling-induced phase separation in trapped BECs}
\label{Rashba}

We consider a quasi-two-dimensional spin-1/2 BEC with Rashba spin-orbit coupling. The trap frequency $\omega_z$ along the $z$ direction is assumed to be very large so that the dynamics is completely frozen into the ground state of the $z$-directional harmonic trap. After integrating the atomic state along $z$ direction, we are left with a quasi-two-dimensional system.  Rashba spin-orbit coupling can be artificially implemented by an optical Raman lattice~\cite{WuZ}, generating the Hamiltonian $H_\text{SOC}$ shown in Eq.~(\ref{Hsoc}). The spin-orbit-coupled BEC is described by the following Gross-Pitaevskii (GP) equation,
\begin{align}
i\frac{\partial\Psi}{\partial t}=\left( H_\text{SOC}+V+H_\text{int}\right) \Psi.
\end{align}
with  $\Psi=\left( \Psi_1,\Psi_2\right) ^T$ being the two-component wave function.
The harmonic trap in the $x-y$ plane is $V=\frac{1}{2}\omega^2(x^2+y^2)$ with $\omega$ the dimensionless trap frequency.
$H_\text{int}$ denotes nonlinear interactions,
\begin{align}
H_\text{int}=\left(\begin{matrix}
g\left|\Psi_{1}\right|^{2}+g_{12}\left|\Psi_{2}\right|^{2} & 0 \\
0 & g_{12}\left|\Psi_{1}\right|^{2}+g\left|\Psi_{2}\right|^{2}
\end{matrix}\right).
\end{align}
The GP equation is dimensionless, and the units of  length, momentum and energy as $\sqrt{\hbar/(m\omega_z)}$, $\sqrt{\hbar m\omega_z}$ and $\hbar\omega_z$, respectively. With the units, the inter- and intra-component interaction coefficients become
$g=Na\sqrt{8\pi m\omega_z/\hbar}$ and $g_{12}=Na_{12}\sqrt{8\pi m\omega_z/\hbar}$. Here, $N$ is the atom number, and $a$ and $a_{12}$ are corresponding $s$-wave scattering lengths, respectively. The wave functions satisfy the normalization condition, $\int dx dy (|\Psi_1|^2+|\Psi_2|^2)=1$.




When $g>g_{12}$, the interactions are miscible. We first study ground states of the system in this regime by performing the imaginary-time evolution of the GP equation. A typical result is shown in Fig.~\ref{fig:figure1}.  As expected from the prediction in the previous section, the ground state is phase-separated. The two components are spatially separated along the $x$ direction, as shown by Figs.~\ref{fig:figure1}(a) and~\ref{fig:figure1}(b). The ground state spontaneously chooses $\bar{\theta}=-\pi/2$ so that the atoms condense at $(\bar{k}_x,\bar{k}_y)=(0,-\lambda)$, which can be seen from the momentum-space density distributions in Figs.~\ref{fig:figure1}(c) and~\ref{fig:figure1}(d).  In this case, according to Eq.~(\ref{Shift}), the position displacement occurs along the $x$ direction, and the first component shifts by $1/(2\lambda)$ on the right side and the second component shifts oppositely by $1/(2\lambda)$ on the left side.





%%%%%%%%%%%%%%%%%%%%%%%%%%%%%%%%%%%%%%%%%%%%%
% Figure environment removed
%%%%%%%%%%%%%%%%%%%%%%%%%%%%%%%%%%%%%%%%%%%%%


%%%%%%%%%%%%%%%%%%%%%%%%%%%%%%%%%%%%%%%%%%%%%
% Figure environment removed
%%%%%%%%%%%%%%%%%%%%%%%%%%%%%%%%%%%%%%%%%%%%%



In the presence of interactions, it is impossible to construct analytical wave function of ground state from the procedure demonstrated in the previous section. Nevertheless, the single-particle wave functions in Eq.~(\ref{wavefunction}) and phase-separated results shown in Fig.~\ref{fig:figure1} stimulate us to use a trial wave function to study the phase separation by the variational method~\cite{Song2013}. The trial wave function is assumed to be
\begin{align}
\label{eq:ansatz}
\Psi(x,y)
=\frac{(\Delta'_x\Delta'_y)^{\frac{1}{4}}}{\sqrt{2\pi} }e^{i\bar{k}'_y y}
\left(\begin{array}{cc}
e^{-\frac{\Delta'_x}{2}(x-\delta_x)^2-\frac{\Delta'_y}{2}y^2} \\
e^{-\frac{\Delta'_x}{2}(x+\delta_x)^2-\frac{\Delta'_y}{2}y^2}
\end{array}\right).
\end{align}
Here, we have assumed that the atoms spontaneously condenses at $(0,\bar{k}'_y)$ in momentum space and therefore the phase separation only happens along the $x$ direction with the relative position displacement $2\delta_x$. $1/\sqrt{\Delta'_{x,y}}$ characterize the widths of the wave packet along the $x,y$ directions.
The unknown parameters  $\bar{k}'_y, \delta_x, \Delta'_{x,y}$  are to be determined by minimizing the energy functional,
\begin{align}
\mathcal{E}&= \int dxdy\Psi^* (H_\text{SOC} +V) \Psi \notag \\
&\phantom{={}}+\int dxdy \left[\frac{g}{2}(|\Psi_1|^4+|\Psi_2|^4) + g_{12} |\Psi_1|^2|\Psi_2|^2\right],
\end{align}
Substituting the trial wave function into the energy functional $\mathcal{E}$ leads to
\begin{align}
\mathcal{E}&= \frac{\bar{k}_y^{'2}}{2}
+\frac{\Delta'_{x}+\Delta'_{y}}{4}\left(1+\frac{\omega^2}{\Delta'_x\Delta'_y} \right) +\frac{1}{2}\omega^2\delta_x^{2} \notag\\
&\phantom{={}}+ \lambda (\Delta'_x\delta_x-\bar{k}'_y)e^{-\Delta'_x \delta_x^2}
+\frac{\sqrt{\Delta'_{x}\Delta'_{y}}}{8 \pi}\left( g+g_{12} e^{-2\Delta'_{x}\delta_{x}^{2} }  \right) .
\label{eq:energy}
\end{align}
By minimizing of the energy functional with respect to the unknown parameters,  $\partial \mathcal{E}/\partial X=0$ $(X=\bar{k}'_y, \delta_x, \Delta'_{x,y})$, we obtain all information of the trial wave function. The phase separation can be characterized by the center of mass of each component,
\begin{align}
\bar{\bm{r}}_{1,2}=\int \bm{r}|\Psi_{1,2}(\bm{r})|^2 d\bm{r},
\end{align}
with $\bm{r}=(x,y)$. In Fig.~\ref{fig:separation}(a), the solid lines show $\bar{x}_{1,2}=\pm \delta_x$ calculated from the variational method, 
while the results obtained by the imaginary-time evolution of the GP equation are demonstrated by the circles. 
We find that the results from the two calculation methods agree very well. Without spin-orbit coupling ($\lambda=0$), the conventional BEC has $\bar{x}_{1,2}=0$ and condensates at $\bar{k}'_y=0$, as shown in Fig.~\ref{fig:separation}(b). 
With the growth of $\lambda$, $\bar{k}'_y$ always increases linearly [see Fig.~\ref{fig:separation}(b)]. The displacement $\bar{x}$ first increases drastically to a maximum value and then declines to the predicted $\pm 1/(2\lambda)$ obtained by the single-particle model [see the dashed lines in Fig.~\ref{fig:separation}(a)]. The dependence of the displacement on $\lambda$ exactly follows the expectation in the previous section.  
In the dramatic increase regime for $\bar{x}$, the variational parameters $\Delta'_{x,y}$ also change dramatically [see Figs.~\ref{fig:separation}(c) and~\ref{fig:separation}(d)]. 


%%%%%%%%%%%%%%%%%%%%%%%%%%%%%%%%%%%%%%%%%%%%%
% Figure environment removed
%%%%%%%%%%%%%%%%%%%%%%%%%%%%%%%%%%%%%%%%%%%%%



Rashba spin-orbit coupling introduces an intrinsic force, 
\begin{align}
\bm{F}=\frac{d \bm{p}}{dt}&=-\big[[\bm{r},H_\text{SOC}],H_\text{SOC}\big] \notag \\
&=2\lambda^2(\bm{p}\times\bm{e}_z)\sigma_z,
\end{align}
with $\bm{e}_z$ being the unit vector along the $z$ direction and $\bm{p}$ the atom momentum. Considering the ground states shown in Fig.~\ref{fig:separation}, the force operator in momentum space becomes $F_x=2\lambda^2\bar{k}'_y\sigma_z$ and $F_y=0$. 
The two components feel opposite force $F_x$ along the $x$ direction. Ground states must compensate the intrinsic force to reach equilibrium. It can be implemented by displacing two component opposite to the force. Since $\bar{k}'_y<0$ in the case shown in Fig.~\ref{fig:separation}, the first component is displaced towards to the right side and the second to the left side.
The force concept has been used in Refs.~\cite{Zhang2012,Song2013} to explain the phase separation. Since the force is proportional to $\lambda^2$,
it seems that a large displacement would be induced for a large $\lambda$. However, as shown in Fig.~\ref{fig:separation}(a), the dependence of the displacement on $\lambda$ does not follow the force. 
We can see that the intrinsic force cannot explain the phase separation in the large $\lambda$ regime.
%The intrinsic force can explain that the position displacement happens along the $x$ direction, but can not catch up with the mechanism of the phase separation.






Figure~\ref{fig:separation}(a) shows that the separation follows the single-particle prediction $\pm 1/(2\lambda)$ when $\lambda$ dominates. 
When $\lambda$ is weak, the displacement also depends on other parameters, such as nonlinear coefficients and the harmonic trap. In Fig.~\ref{fig:interaction-trap}(a), we plot the displacement $\bar{x}$ as a function of the inter-component interaction coefficient $g_{12}$ for a non-dominant $\lambda$. The displacement slightly rises with an increasing $g_{12}$, and it reaches the maximum when $g_{12}=g$. If $g_{12}>g$, the interactions become immiscible, leading to ground states different from the trial wave function in Eq.~(\ref{eq:ansatz}). The dependence of the displacement on the trap frequency is shown in Fig.~\ref{fig:interaction-trap}(b). We find that the displacement decreases
as the trap frequency increases. This is because the displacement requires more kinetic energy in a tight trap. 






%%%%%%%%%%%%%%%%%%%%%%%%%%%%%%%%%%%%%%%%%%%%%
% Figure environment removed
%%%%%%%%%%%%%%%%%%%%%%%%%%%%%%%%%%%%%%%%%%%%%












\section{The anisotropic spin-orbit-coupling-induced phase separation in trapped BECs}
\label{AnistropicSOC}


Rashba-spin-orbit-coupling-induced phase separation has been analyzed in the previous section.  In the two-dimensional spin-orbit-coupled BEC experiment~\cite{WuZ}, the spin-orbit coupling strengths are tunable, which leads to an anisotropic coupling. 
It has been revealed  that the anisotropic spin-orbit coupling has a great impact on ground states of a spatially homogeneous BEC~\cite{Ozawa}.
In this section, we study anisotropic-spin-orbit-coupling-induced phase separation.
The single-particle Hamiltonian of the anisotropic spin-orbit coupling is
\begin{equation}
\label{Asoc}
H'_\text{SOC}=\frac{p_x^2+p^2_y}{2}+\lambda_1 p_{x} \sigma_{y}-\lambda_2 p_{y} \sigma_{x},
\end{equation}
with the anisotropic strengths $\lambda_1 \ne \lambda_2$. 
The lower band of $H'_\text{SOC}$ is
\begin{align}
E= \frac{k_x^2 + k_y^2}{2} - \sqrt{\lambda_1^2 k_x^2 + \lambda_2^2 	k_y^2}.
\end{align}
with the associated eigenstates being the same as Eq~(\ref{Eigenstate}) but having the different relative phase which can be written as
\begin{equation}\label{Aphi}
\tan(\varphi) = \frac{\lambda_1k_x}{\lambda_2k_y}.
\end{equation}
According to the mechanism of the spin-orbit-coupling-induced phase separation, the anisotropic coupling can generate position displacements related to the derivatives of the relative phase.  The displacements along the $x$ and $y$ directions are
 \begin{align}
&\left.\frac{\partial \varphi(\mathbf{k} )}{ \partial k_x}\right|_{\mathbf{\bar{k}}}= \frac{\lambda_1\lambda_2\bar{k}_y}{ \lambda_1^2\bar{k}_x^2+\lambda_2^2\bar{k}_y^2}, \notag\\
&\left.\frac{\partial \varphi(\mathbf{k} )}{ \partial k_y}\right|_{\mathbf{\bar{k}}}= -\frac{\lambda_1\lambda_2 \bar{k}_x}{ \lambda_1^2 \bar{k}_x^2+ \lambda_2^2\bar{k}_y^2}.
\label{AShift}
\end{align}
Here, $\mathbf{\bar{k}}=(\bar{k}_x, \bar{k}_y) $ is the momentum at which the atoms condense.  The lowest energy minima of the lower band depend on the anisotropy. When $\lambda_1< \lambda_2$, the two minima locate at $(\bar{k}_x, \bar{k}_y)=(0,\pm \lambda_2)$ [see Fig.~\ref{fig:anisotropy}(a1)]. They locate at $(\bar{k}_x, \bar{k}_y)=(\pm \lambda_1,0)$ when  $\lambda_1> \lambda_2$ [see Fig.~\ref{fig:anisotropy}(b1)].  With the miscible interactions, the BEC spontaneously chooses one of these two minima to condense.  The ground state that spontaneously condenses at  $(\bar{k}_x, \bar{k}_y)=(0, -\lambda_2)$ for  $\lambda_1< \lambda_2$ is demonstrated in Figs.~\ref{fig:anisotropy}(a2)-(a5). From the single-particle prediction in Eq.~(\ref{AShift}), the phase separation of this ground state happens only along the $x$ direction, and the center-of-mass of the first component is $\lambda_1/(2\lambda_2^2)$ and that of the second component is $-\lambda_1/(2\lambda_2^2)$. Density distributions shown in  Figs.~\ref{fig:anisotropy}(a2) and~\ref{fig:anisotropy}(a3) clearly indicate the phase separation following the predictions.  The ground state that spontaneously condenses at  $(\bar{k}_x, \bar{k}_y)=( -\lambda_1,0)$ for  $\lambda_1>\lambda_2$ is demonstrated in Figs.~\ref{fig:anisotropy}(b2)-(b5).  The single-particle mechanism in Eq.~(\ref{AShift}) predicts that for this ground state the separation happens along the $y$ direction and the center-of-mass are $\mp \lambda_2/(2\lambda_1^2)$ for two components. The results from the imaginary-time evolution shown in Figs.~\ref{fig:anisotropy}(b2) and~\ref{fig:anisotropy}(b3) match with the single-particle predictions. 





These analyses have shown that the center of mass of each component strongly depends on the ratio of the spin-orbit coupling strengths.
To reveal the dependence of  phase separation on $\lambda_2/\lambda_1$, 
we calculate ground states  with a fixed $\lambda_1$ and a changeable $\lambda_2$ by using the imaginary-time evolution. 
The results are summarized in Fig.~\ref{fig:anisotropy-com}, 
where the circles (crosses) represent the center of mass for the first (second) component.
For $\lambda_2<\lambda_1=1$, the phase separation occurs along the $y$ direction and $|\bar{y}|$ increases with the increase of $\lambda_2$ [see red circles and crosses in Fig.~\ref{fig:anisotropy-com}], 
while $\bar{x}$ is zero [see blue circles and crosses in Fig.~\ref{fig:anisotropy-com}]. When $\lambda_2=0$, the spin-orbit coupling becomes one-dimensional, there is no phase separation due to the absence of the relative phase.  
The results change for $\lambda_2>\lambda_1=1$ and the phase separation along the $x$ direction is observed.
In this case, the separation decreases with $\lambda_2$ increasing. For a very large $\lambda_2$, the separation disappears since the spin-orbit coupling effectively turns to be one-dimensional. The results in Fig.~\ref{fig:anisotropy-com} demonstrate that the maximum separation happens for $\lambda_1=\lambda_2$ which is Rashba spin-orbit coupling. This is also expected from the single-particle prediction in Eq.~(\ref{AShift}).  
%The maximum happens when $\lambda_1=\lambda_2$ and the maximized separations become the Rashba results in Eq.~(\ref{Shift}). 

%%%%%%%%%%%%%%%%%%%%%%%%%%%%%%%%%%%%%%%%%%%%%
% Figure environment removed
%%%%%%%%%%%%%%%%%%%%%%%%%%%%%%%%%%%%%%%%%%%%%
 
 %%%%%%%%%%%%%%%%%%%%%%%%%%%%%%%%%%%%%%%%%%%%%
 % Figure environment removed
 %%%%%%%%%%%%%%%%%%%%%%%%%%%%%%%%%%%%%%%%%%%%%

\section{Adiabatic splitting dynamics}
\label{Adiabaticdynamics}

We have shown that ground states of a trapped BEC with two-dimensional spin-orbit coupling and miscible interactions are phase-separated.  As an important application, we study adiabatic dynamics of the phase separation. 
As pointed out by previous works, a linear coupling between two component favors miscibility regardless of interactions~\cite{Gautam,Merhasin}.
Therefore, a miscible-to-immiscible transition may occur by decreasing the coupling. The adiabatic dynamics is stimulated by slowly switching off the linear coupling.  Theoretically, the process is described by the time-dependent GP equation, 
\begin{align}
i\frac{\partial\Psi}{\partial t}=\left[H_\text{SOC}+\Omega(t)\sigma_x+V+H_\text{int}\right] \Psi.
\label{eq:dynamics}
\end{align}
Here, $\Omega(t)\sigma_x $ represents the linear coupling between the two components, and can be experimentally achieved by using a radio-frequency coupling~\cite{Nicklas}.  The time-dependent Rabi frequency is 
\begin{equation}
\Omega(t)=\Omega_0(1-t/\tau_q),
\end{equation}
with $\Omega_0$ being the initial value of the linear coupling and $\tau_q$ is the quench duration.  At $t=0$, the presence of $\Omega_0$ greatly suppresses the ground-state phase separation. We obtain ground state by the imaginary-time evolution of Eq.~(\ref{eq:dynamics}) with $\Omega(t)=\Omega_0$. A typical ground state is shown in insets (a) and (b) of Fig.~\ref{fig:dynamics}, and the separation between two components is not obvious. Using this ground state as initial state, we evolve the time-dependent GP equation. The center-of-mass $\bar{x}$ for two components is recorded during the time evolution in Fig.~\ref{fig:dynamics}.  By decreasing the linear coupling adiabatically, the separation between two component gradually increases. When it is completely switched off, i.e., $t=\tau_q$, the separation is maximized [see corresponding density distributions in insets (c) and (d)].  
The two components can realize a dynamically spatial splitting, which move along opposite directions.
Such adiabatic splitting dynamics are reminiscent of  a kind of  ``atomic spin Hall effect''~\cite{Beeler}. 




%%%%%%%%%%%%%%%%%%%%%%%%%%%%%%%%%%%%%%%%%%%%%
% Figure environment removed
%%%%%%%%%%%%%%%%%%%%%%%%%%%%%%%%%%%%%%%%%%%%%






\section{Immiscible interactions induced phase separation}
\label{Immiscibilty} 


In all above, the interactions are miscible ($g>g_{12}$), which support atoms to condense at a particular momentum state. On the other hand, immiscible interactions ($g<g_{12}$) prefer a spatial separation between two components in order to minimize the inter-component interactions proportional to $g_{12}$. In the presence of spin-orbit coupling, the immiscible-interaction-induced phase separation presents interesting features~\cite{Zhang2012,Hu2012,Ramachandhran}.  In Fig.~\ref{fig:strong-interspecies}, we show two different kinds of immiscible-interaction-induced phase separated ground states with different values of spin-orbit coupling strength $\lambda$. For $\lambda=0.5$, the ground state obtained by the imaginary-time evolution is a half-quantum vortex state which has been first revealed in Refs.~\cite{Hu2012,Ramachandhran}. The first component distribution has a Gaussian shape [see Figs.~\ref{fig:strong-interspecies}(a1) and~\ref{fig:strong-interspecies}(a3)], and the second component is a vortex with a winding number $w=1$ [see Figs.~\ref{fig:strong-interspecies}(a2) and~\ref{fig:strong-interspecies}(a4)].  The first component is filled in the density dip of the second one, forming a spatial separation along the radial direction.  For $\lambda=1.5$, the ground state becomes a stripe state which has been first revealed in Refs.~\cite{Wang, Ho2011}.  The ground state condenses simultaneously at two different momenta [see Figs.~\ref{fig:strong-interspecies}(b3) and~\ref{fig:strong-interspecies}(b4)].  Such momentum occupation generates spatially periodic modulations in density distributions [Figs.~\ref{fig:strong-interspecies}(b1) and~\ref{fig:strong-interspecies}(b2)].  
Meanwhile, stripes of the two components are spatially separated. 

We emphasize that phase separations induced by spin-orbit coupling and immiscible interaction have different physical origins. The spin-orbit-coupling-induced phase separation only works for a two-dimensional spin-orbit coupling.  However, phase separations have been also studied for a BEC with a one-dimensional spin-orbit coupling, the mechanism of which is different.  
In the pioneered spin-orbit-coupled experiment, experimentalists observed a spatial separation between two dressed states with a Raman-induced spin-orbit coupling~\cite{Lin}.  The spin-orbit coupling generates two energy minima, whose occupations can be considered as two dressed states.  In the dressed state space, atomic interactions turn to be immiscible between two dressed states in the presence of the Rabi frequency. The phase separation happens in the dressed state space due to immiscibility. 
In addition, Ref.~\cite{Gautam} reveals the existence of phase separation in a spin-1 BEC with the Raman-induced spin-orbit coupling. 
The single-particle Hamiltonian of the system is $H=(p_x+\lambda' F_z)^2/2 +\Omega' F_x +\epsilon F_z^2$. Here,  $F_{x,y,z}$ are the spin-1 Pauli matrices, $\lambda'$ is the spin-orbit coupling strength, $\Omega'$ is the Rabi frequency, and $\epsilon$ is the quadratic Zeeman shift. The spinor interactions include density-density part with the coefficient $c_0$ and spin-spin part with the coefficient $c_2$. 
In particular, a very negative quadratic Zeeman shift $\epsilon=-\lambda^2/2$ was considered.
With such a large negative $\epsilon$, the occupation in the second component can be eliminated. The spinor only occupies the first and third components. Interestingly, the spinor interactions between the first and third components are immiscible for a negative spin-spin interaction ($c_2<0$). Different phase-separated states between the first and third component are due to immiscible interactions~\cite{Gautam} .   






 


\section{Conclusion}
\label{conclusion}

In summary, we have revealed the physical mechanism of spin-orbit-coupling-induced phase separation.  The mechanism, which is very different from the conventional immiscible-interaction-induced separation, is a complete single-particle effect of spin-orbit coupling. 
We have analyzed separation features in a trapped BEC with Rashba spin-orbit coupling and miscible interactions and studied the effects of the anisotropy of spin-orbit coupling on the separation.  All features can be explained by the single-particle mechanism. As an interesting application of the phase separation, we propose an adiabatic dynamics that can dynamically split two components spatially. 







\section*{Acknowledgments}



This work was supported by National Natural Science Foundation of China with Grants No.11974235 and 11774219. 
H.L. acknowledges support from Okinawa  Institute of Science and Technology Graduate University.



\bibliography{RashbaPS}% Produces the bibliography via BibTeX.



\end{document}