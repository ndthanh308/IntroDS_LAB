% This is samplepaper.tex, a sample chapter demonstrating the
% LLNCS macro package for Springer Computer Science proceedings;
% Version 2.20 of 2017/10/04
%
\documentclass[runningheads]{llncs}
%
\usepackage{graphicx}
\usepackage{siunitx}
\usepackage{algorithm}
%\usepackage{algorithmic}
\usepackage{algpseudocode}
\usepackage{xcolor}
\usepackage{amsmath,amssymb} % for \boldsymbol macro
\usepackage{bm}      % for \bm macro
\usepackage{booktabs}
\usepackage{multirow}
\usepackage{subcaption}
\usepackage{hyperref}
\newcommand{\enzo}[1]{\textcolor{purple}{Enzo: #1}}
\newcommand{\victor}[1]{\textcolor{blue}{#1}}
\newcommand{\marta}[1]{\textcolor{orange}{Marta: #1}}
\newcommand{\adc}[1]{\textcolor{red}{After draft completed: #1}}
\usepackage{soul}
\usepackage{enumitem}
% Used for displaying a sample figure. If possible, figure files should
% be included in EPS format.
%
% If you use the hyperref package, please uncomment the following line
% to display URLs in blue roman font according to Springer's eBook style:
% \renewcommand\UrlFont{\color{blue}\rmfamily}

\begin{document}
%
%\title{$n$-NeRF: Ensembling of low-resolution Neural Radiance Fields beats higher resolution models}
%\title{Think Small, Go Big: Achieving High-Quality 3D Scene Modeling with a NeRF Ensemble}
\title{Sparse Double Descent in Vision Transformers: real or phantom threat?}
%
\titlerunning{}
% If the paper title is too long for the running head, you can set
% an abbreviated paper title here
%
\author{Victor Qu\'etu\orcidID{0009-0004-2795-3749} \and Marta Milovanovi\'c\orcidID{0000-0002-3280-2396} \and Enzo Tartaglione\orcidID{0000-0003-4274-8298}} %\\\and Marco Grangetto$^{1}$}%\orcidID{2222--3333-4444-5555}

%
\authorrunning{V. Qu\'etu et al.}
% First names are abbreviated in the running head.
% If there are more than two authors, 'et al.' is used.
%
\institute{LTCI, T\'el\'ecom Paris, Institut Polytechnique de Paris, France
\email{\{name.surname\}@telecom-paris.fr}}
%
\maketitle              % typeset the header of the contribution
%
\begin{abstract}
Vision transformers (ViT) have been of broad interest in recent theoretical and empirical works. They are state-of-the-art thanks to their attention-based approach, which boosts the identification of key features and patterns within images thanks to the capability of avoiding inductive bias, resulting in highly accurate image analysis.
Meanwhile, neoteric studies have reported a ``sparse double descent'' phenomenon that can occur in modern deep-learning models, where extremely over-parametrized models can generalize well.
This raises practical questions about the optimal size of the model and the quest over finding the best trade-off between sparsity and performance is launched: are Vision Transformers also prone to sparse double descent? Can we find a way to avoid such a phenomenon?\\
Our work tackles the occurrence of sparse double descent on ViTs. Despite some works that have shown that traditional architectures, like Resnet, are condemned to the sparse double descent phenomenon, for ViTs we observe that an optimally-tuned $\ell_2$ regularization relieves such a phenomenon. However, everything comes at a cost: optimal lambda will sacrifice the potential compression of the ViT.

\keywords{Sparse double descent, transformers, pruning, deep learning}
\end{abstract}
%
%
%

\section{Introduction}
Deep learning models have been widely used in many applications.
For example, BERT~\citep{devlin_bert_2019}, GPT-3~\citep{brown_language_2020}, and T5~\citep{raffel_exploring_2020} achieved state-of-the-art~(SOTA) results on different natural language processing~(NLP) tasks. 
For computer vision~(CV), Transformer-like models such as ViT~\citep{dosovitskiy_image_2021} and Swin Transformer~\citep{liu_swin_2021} deliver excellent accuracy performance upon multiple tasks. 


At the same time, training deep learning models has been a critical problem troubling the community due to the long training time, especially for those large models with billions of parameters~\citep{brown_language_2020}. 
In order to enhance the training efficiency, researchers propose some manually designed parallel training strategies~\citep{narayanan_efficient_2021,shazeer_mesh-tensorflow_2018,xu_gspmd_2021}. 
However, selecting, tuning, and combining these strategies require extensive domain knowledge in deep learning models and hardware environments. With the increasing diversity of modern hardware architectures~\cite{flynn_very_1966,flynn_computer_1972} and the rapid development of deep learning models, these manually designed approaches are bringing heavier burdens to developers. 
Hence, \emph{automatic parallelism} is introduced to automate the parallel strategy searching for training models.


There are two main categories of parallelism in deep learning models: inter-layer parallelism~\citep{huang_gpipe_2019,narayanan_pipedream_2019,narayanan_memory-efficient_2021,fan_dapple_2021,li_chimera_2021,lepikhin_gshard_2021,du_glam_2022,fedus_switch_2022} and intra-layer parallelism~\citep{li_pytorch_2020,narayanan_efficient_2021,rasley_deepspeed_2020,fairscale_authors_fairscale_2021}. 
Inter-layer parallelism partitions the model into disjoint sets on different devices without slicing tensors. 
Alternatively, intra-layer parallelism partitions tensors in a layer along one or more axes and distributes them across different devices.


Current automatic parallelism techniques focus on optimizing strategies within these two categories. However, they treat these two categories separately. 
Some methods~\citep{zhao_vpipe_2022,jia_exploring_2018,cai_tensoropt_2022,wang_supporting_2019,jia_beyond_2019,schaarschmidt_automap_2021,liu_colossal-auto_2023} overlook potential opportunities for inter- or intra-layer parallelism, the others optimize inter- and intra-layer parallelism hierarchically and sequentially~\citep{narayanan_pipedream_2019,fan_dapple_2021,he_pipetransformer_2021,tarnawski_efficient_2020,tarnawski_piper_2021,zheng_alpa_2022}. 
As a result, current automatic parallelism techniques often fail to achieve the global optima and instead become trapped in local optima. 
Therefore, a unified inter- and intra-layer approach is needed to enhance the effectiveness of automatic parallelism.


This paper aims to find the optimal parallelism strategy while simultaneously considering inter- and intra-layer parallelism. 
It enables us to search in a more extensive strategy space where the globally optimal solution lurk. 
However, unifying inter- and intra-layer parallelism in automatic parallelism brings us two challenges. 
Firstly, to adopt a unified perspective on the inter- and intra-layer automatic parallelism, we should not formalize them with separate formulations as prior works. Therefore, how can we express these parallelism strategies in a unified formulation? 
Secondly, previous methods take a long time to obtain the solution with a limited strategy space. Therefore, how can we ensure that the best solution can be obtained in a reasonable time while expanding the strategy space?


To solve the above challenges, we propose UniAP. For the first challenge, UniAP adopts the mixed integer quadratic programming~(MIQP)~\citep{lazimy_mixed_1982} to search for the globally optimal parallel strategy automatically. 
It unifies the inter- and intra-layer automatic parallelism in a single MIQP formulation. 
For the second challenge, our complexity analysis and experimental results show that UniAP can obtain the globally optimal solution in a significantly shorter time.


The contributions of this paper are summarized as follows: 
\begin{itemize}
    \item We propose UniAP, the first framework to unify inter- and intra-layer automatic parallelism in model training.
    \item The optimal parallel strategies discovered by UniAP exhibit scalability on training throughput and strategy searching time.
    \item The experimental results show that UniAP speeds up model training on four Transformer-like models by up to 1.70$\times$ and reduces the strategy searching time by up to 16$\times$, compared with the SOTA method.
\end{itemize}

\section{RELATED WORK}
\label{sec:related_work}

% first approaches for trajectory forecasting
% probabilistic trajectory forecasting
% clusters based trajectory forecasting
% prob trajectory generation


Trajectory forecasting has been predominantly researched on two fronts: context-agnostic and context-aware approaches. 
Context-agnostic methods solely forecast based on observed trajectory patterns, whereas context-aware methods include social and scene layout cues. 
While context-agnostic approaches have received little attention~\cite{becker18,scholler20,giuliari20} from the research community, context-aware 
methods have been comprehensively investigated~\cite{gupta18,kosaraju19,dendorfer20,sun20,salzmann20,kothari23}. 
This paper focuses on context-agnostic approaches and breeding mechanisms for probabilistic trajectory forecasting based on encoded clustering 
information.


In the trajectory generation domain, some works tackled the problem from a probabilistic 
view in road scenarios~\cite{phan-minh20,ma21,ivanovic22,calem22} and human trajectory data~\cite{sun21,chen22,miao22}.
CoverNet~\cite{phan-minh20} comprises a Convolutional Neural Network (CNN) to extract contextual features from a road scene 
and a trajectory generator module to produce a set of possible predictions. Then, the system directly classifies the set of 
plausible trajectories yielding the score of each prediction. Conversely, we aim to produce context-agnostic samples to open 
the domain of applications of our system and reduce its requirements.
In~\cite{ma21}, the authors propose a post-hoc method named Likelihood-Based Diverse Sampling (LDS). In that paper, a novel objective function and a non-i.i.d sampling method encourage diverse predictions by suppressing similar predictions from the generated set of samples and leveraging the likelihoods from a pre-trained generative model. However, this work does not determine the scores (likelihoods) of the set of plausible trajectories, which we claim is paramount for risk-aware downstream
decision processes. 
\emph{HAICU}~\cite{ivanovic22} is a system that relies on perception and classification modules to give the 
class distribution of road agents. This system stands for a Conditional Variational Autoencoder (cVAE) conditioned on the class distribution and the observed trajectory of 
road agents to produce multimodal predictions. Therefore, this work heavily relies on upstream supervised methods to yield the class distribution, which is not easily generalizable to human trajectory data. To cope with this limitation, we propose unsupervised techniques to cluster akin trajectories, which we consider agnostic to the trajectory domain and more generalizable. 
In~\cite{calem22}, the authors enforce underlying physical admissibility constraints and diversity in a post-hoc trajectory sampling process 
based on a determinantal point process (DDP). Although it proposes a robust strategy for considering context using admissibility
constraints induced in the objective function, it does not provide a probabilistic view of the predicted trajectories.
\cite{sun21} devises a three-step method based on clustering, classification, and synthesis to predict. Contrarily to this work,
we first predict by using a cDGM and then propose a ranking proposals step. In this work, we investigate ranking proposals methods based on the distance 
to the already conceived clustering space and, therefore, not relying on a classification network. Further, \cite{chen22} 
proposes a method based on clustered goal points and a final classification step. While both \cite{sun21} and \cite{chen22}, 
at the classification step, learn the mapping between the past trajectory and the cluster class, we rank complete trajectories emphasizing the generated {\em tracklets}. Finally, in \cite{miao22}, 
the authors investigated a system with two branches: a motion pattern selector 
and a multimodal trajectory generator. The former produces a \emph{gallery} of diverse motion patterns, while the latter refines them and 
generates future trajectories. Then, a scoring method produces the most diverse predictions.


Our work encompasses mechanisms under the same umbrella as~\cite{sun21,chen22,miao22}, but our main objective is to propose a system that can 
improve current deep generative models by including information from clustered data. In our system
the clusters drive the multimodality, and the ranking proposals methods run on the generated trajectories. 
In addition, our distance-based ranking 
proposals methods does not rely on training any auxiliary neural network conversely to previous works~\cite{sun21,chen22,miao22}. In our case,
the ranking proposals step depends exclusively on the intrinsic nature of the clustering space and a similarity measure to the generated trajectories. Further, 
we propose a new deep clustering method inspired by a self-supervised deep generative model developed for the image generation task~\cite{liu20}. 
Finally, contrarily to previous works, the probabilities of each trajectory strongly affect the evaluation of our method, while in previous works, 
the probabilities only give a sense of the likelihood of each particular event.
\section{Methodology}
\label{sec:method}

\subsection{Overview}
\label{sec:method_fmwk}

As shown in~\cref{fig:method_fmwk}, the proposed unsupervised MOT framework is trained with the widely-used contrastive learning technique~\cite{chen2020simple,he2020momentum}. 
\lk{Specifically, for multi-object tracking}, objects within the tracklet ($\boldsymbol{k}_{+}$) should be pulled together and different tracklets ($\boldsymbol{k}_{-}$) should be separated. It can be mathematically formulated as:

\begin{equation}
% \begin{split}
    \mathcal{L}_{cl}( \boldsymbol{q}; \boldsymbol{k}_{+}; \boldsymbol{k}_{-} )= 
    - \log \frac{\exp(\boldsymbol{q} \cdot \boldsymbol{k}_{+} / \epsilon)}{\sum_{i}\exp(\boldsymbol{q} \cdot \boldsymbol{k}_{i} / \epsilon)}
  \label{eq:method_nce}
% \end{split}  
\end{equation}

\noindent where $\mathcal{L}_{cl}$ denotes the InfoNCE~\cite{oord2018representation} loss function, and $\epsilon$ is the temperature hyper-parameter~\cite{wu2018unsupervised}. 
Within a video, following the unsupervised tracking fashion~\cite{liu2022online,shuai2022id}, the positive and negative keys mainly come from two sources, \ie pseudo-labeled historical frame and self-augmented frame. 

\lk{However, two issues occur: (1) the uncertainty reduces the accuracy of pseudo-tracklets and (2) the randomly augmented samples fail to learn the inter-frame consistency.} 
We argue the above issues are not independent. 
\lk{By leveraging the uncertainty in turn,} the accurate pseudo-tracklets can guide the qualified positive and negative augmentations.

To address these two issues, we propose an uncertainty-aware pseudo-tracklet labeling strategy in \cref{sec:method_uoap}, which integrates a verification-and-rectification mechanism into the tracklet generation. Our method significantly improves the accuracy of pseudo-identities, especially in long-term interval. 
Then we propose a tracklet-guided augmentation strategy in \cref{sec:method_ada_aug}, which brings the temporary information into spatial augmentation. The augmented samples simulates the objects' motion. A hierarchical uncertainty-based sampling strategy is proposed for hard sample mining. More details are described in the following section.


\subsection{Uncertainty-aware Tracklet-Labeling}
\label{sec:method_uoap}

Accurate pseudo tracklet is critical in \liuk{learning feature consistency}. 
However, without manual annotation, \lk{the aggravated uncertainty makes} the tracklet-labeling a huge challenge due to various interference factors, including similar appearance among objects, frequent object cross and occlusions, \etc. 
\lk{In fact, the uncertainty can also be leveraged to improve the pseudo-accuracy in turn.}
In this section, we propose an \textbf{U}ncertainty-aware \textbf{T}racklet-\textbf{L}abeling (\textbf{UTL}) strategy for better pseudo-tracklets.

Given an input video sequence $V = \{I^{1}, I^{2}, \cdots, I^{N}\}$, each frame $I^{t}$ is annotated with the bounding boxes $B^{t} = \{b_{1}^{t}, b_{2}^{t}, \cdots, b_{M^{t}}^{t}\}$ of $M^{t}$ objects in $t_{th}$ frame, where $b_{i}^{t} = (cx_{i}^{t}, cy_{i}^{t}, w_{i}^{t}, h_{i}^{t})$ is the center coordinate and shape of the $i_{th}$ object $o_{i}^{t}$. As shown in~\cref{fig:method_fmwk}, \mywork~generates accurate pseudo-tracklets in four main steps:

1) \textbf{Association}. For a certain object $o_{i}^{t}$ in frame $I^{t}$, the $\ell_2$-normalized representation $\boldsymbol{f}_{i}^{t}$ can be expressed as $\boldsymbol{f}_{i}^{t} = {\phi}(I^{t}, b_{i}^{t})$, 
% \begin{equation}
%   \boldsymbol{f}_{i}^{t} = {\phi}(I^{t}, b_{i}^{t})
%   % / {\Vert {\phi}(I^{t}, b_{i}^{t}) \Vert}_{2}
%   \label{eq:method_feat}
% \end{equation}
where the embedding encoder is denoted as $\phi$.

To associate the objects in frame $I^{t}$ with the objects or trajectories in previous $I^{t \minus 1}$, a similarity matrix is constructed with their appearance embeddings:

\begin{equation}
  \boldsymbol{C} \in \mathbb{R}^{M^{t} \times M^{t \minus 1}}, \;
  c_{i,j} = {\boldsymbol{f}_{i}^{t}} \cdot  \boldsymbol{f}_{j}^{t \minus 1}
  \label{eq:method_matrix}
\end{equation}

\noindent where $c_{i,j}$ represents the cosine similarity between the $i_{th}$ object in frame $I^{t}$ and the $j_{th}$ object (or trajectory) in frame $I^{t \minus 1}$. Then the Hungarian algorithm~\cite{kuhn1955hungarian} is adopted to generate the identity association results.

2) \textbf{Verification}. However, the appearance representations are sometimes unreliable, especially in the unsupervised scenario. To solve this issue, an uncertainty metric is proposed to evaluate the association after the first stage.

% For an object $o_{i}^{t}$ in frame $I^{t}$, the similarities against the $M^{t \minus 1}$ objects in the previous frame can be expressed as:

% \begin{equation}
%   \boldsymbol{s}_{i} = \boldsymbol{C}_{i} = [c_{i,1}, c_{i,2}, \cdots, c_{i,M^{t \minus 1}}]^T
%   \label{eq:method_svec}
% \end{equation}

% Inspired by margin-based OOD detection~\cite{hendrycks2016baseline}, we assume that the assignment ($o_{i}^{t} \!\sim\! o_{j}^{t \minus 1}$) in the association stage is not convincing under the following circumstances:

% \begin{itemize}
%     \setlength{\itemsep}{0pt}
%     \item The assigned similarity between $o_{i}^{t}$ and $o_{j}^{t \minus 1}$ is relatively low (\ie, $c_{i,j} < m_1$).
%     \item The second-highest similarity with others ($c_{i,j_{2}}$) is close to the assigned $o_{j}^{t \minus 1}$ (\ie, $c_{i,j} - c_{i,j_{2}} < m_2$).
% \end{itemize}

% Based on these assumptions, we define an association-level uncertainty metric, which is formulated as:



Object association can be viewed as multi-category classification.
And confidence-score has been proved efficient and effective on detecting mis-classified examples~\cite{hendrycks2016baseline}.
Inspired by this, we propose to detect the mis-associated objects through the similarity-scores.


Given an object $o_{i}^{t}$ associated with $o_{j}^{t \minus 1}$ in the previous frame based on \cref{eq:method_matrix}, the association ($o_{i}^{t} \!\sim\! o_{j}^{t \minus 1}$) is unconvincing in two cases: 
1) the assigned similarity $c_{i,j}$ is relatively low (\eg, partial occlusion or motion blur) and 
2) there are other objects whose similarities are close to the assigned $c_{i,j}$ (\eg, similar appearance or indistinguishable embedding).
It can be formulated as:

\begin{equation}
  c_{i,j} < m_1; \quad c_{i,j_{2}} > c_{i,j} - m_2
  \label{eq:method_margin}
\end{equation}


\noindent 
where $m_1,m_2$ are constant margins. Only the second-highest similarity with others ($c_{i,j_{2}}$) is considered for simplicity.
In an ideal association, $c_{i,j}$ should be close to 1 and $c_{i,j_{2}}$ close to 0.
We thus proposed to estimate the association \lk{risk} as:

% \sigma_{i,j} = - \left( 
% \log c_{i,j} + \log \left( 1 - c_{i,j_{2}} \right)
% + \overline{\log \left( 1 - c_{i,l} \right) }
% \right)  
\begin{equation}
  \sigma_{i,j} = - \log c_{i,j} - \log \left( 1 - c_{i,j_{2}} \right)
  \label{eq:method_energy}
\end{equation}

Detailed derivation process refers to the supplementary materials.
Combining with \cref{eq:method_margin} and \cref{eq:method_energy} , an adaptive threshold is proposed:

\begin{equation}
  % \gamma_{i,j} = -\log \left( 1 + m_2 - c_{i,j} \right) -\log m_1 \left( 1 - m_3 \right)
  \gamma_{i,j} =  -\log m_1 - \log \left( 1 + m_2 - c_{i,j} \right)
  \label{eq:method_border}
\end{equation}

As shown in~\cref{fig:method_verify}, when the \lk{risk} $\sigma_{i,j}$ is higher than the threshold $\gamma_{i,j}$, the assignment ($o_{i}^{t} \!\sim\! o_{j}^{t \minus 1}$) should be re-considered. 
\lk{The \textbf{association uncertainty} is quantified as:}

\begin{equation}
  \delta_{i,j} = \sigma_{i,j} - \gamma_{i,j}
  \label{eq:method_uncertain}
\end{equation}

The results are not sensitive to the exact margins. We set $m_1 = 0.5$ and $m_2 = 0.05$ for a slightly better performance.
% More experimental details are shown in the supplementary materials.

The uncertain pairs after the verification stage and unmatched objects after the association stage are gathered as uncertain candidates for the rectification stage.


3) \textbf{Rectification}. 
The rectification stage is performed among the uncertain candidate. The similarities between two adjacent frames are no longer convincing.
% due to irregular motion, severe occlusion, and so on. 
More information should be taken into account, including motion \lk{estimation} and appearance \lk{variation} within a tracklet. 
% Specifically, intersection-over-union (IoU)~\cite{bewley2016simple} is the widely-used motion metric. At the same time, the tracklet embeddings can provide complementary appearance information.

For the uncertain candidates, \mywork~constructs another similarity matrix for the secondary rectification. 
First, \lk{the motion constraints should be relaxed}, so the association shares overlap \lk{higher than} $\beta$ 
% in adjacent frames 
\lk{are preserved}. 
Second, \lk{the appearance should not vary extremely fast}, so we adopt the averaged similarity between object $o_{i}^{t}$ and tracklet $trk_{j} = \{o_{j}^{t \minus K}, \cdots, o_{j}^{t \minus 1}\}$ within previous $K$ frames. 
In this stage, we solve the sub-problem of global identity assignments, which can be formulated as:

\begin{equation}
\begin{split}
  \boldsymbol{C}^\prime \in \mathbb{R}^{{M^{t}}^\prime \times {M^{t \minus 1}}^\prime} & \\
  c^\prime_{i,j} = \left( \frac{1}{K} \sum_{\hat{t} = t \minus K}^{t \minus 1} {\boldsymbol{f}_{i}^{t}} \cdot  \boldsymbol{f}_{j}^{\hat{t}} \right) 
            \times \mathbb{I} & \left( \text{IoU} \left( b_{i}^{t}, b_{j}^{t \minus 1} \right) > \beta \right) 
  \label{eq:method_recti}
\end{split}
\end{equation}

\noindent where $\mathbb{I}(*)$ is the indicator function. Then the match set is updated based on the Hungarian algorithm.

\lk{
\textit{Remark.} Our core contribution is the uncertainty-based verification mechanism, rather than the specific rectification, which shall be adjusted in practice. Empirically we set $\beta=0.1$ and $K=5$.
}

% Figure environment removed

4) \textbf{Propagation}. The pseudo-tracklets are propagated frame-by-frame. As shown in~\cref{fig:method_reidacc}, our strategy brings \lk{consistently} accurate pseudo-identities, \lk{\eg, reaching 97\% accuracy across 100 frames}.
% The pseudo-tracklets are progressively updated during the training process.
The long-term intra-tracklet consistency is successfully maintained.
% by the accurate pseudo-identities.

It is worth mentioning that the \lk{verification and rectification} stages can be naturally applied to the inference process to boost the performance, \lk{which does not conflict with existing association methods}.

\subsection{Tracklet-Guided Augmentation}
\label{sec:method_ada_aug}

The accurate pseudo-tracklets can guide the sample augmentation in the unsupervised MOT framework.
To learn the \liuk{inter-frame consistency}~\cite{chen2020simple,zhang2021fairmot}, good training samples should be diverse and \liuk{temporal-aware}. 
However, as illustrated in~\cref{fig:method_ada_aug}, existing methods usually treat augmentation and multi-object tracking as two isolated tasks, leading to ineffective augmentations. 
Instead, this paper utilizes the tracklet's spatial displacements to guide the augmentation process. 
According to a properly selected anchor pair, the proposed strategy makes the augmented frames aligned to the historical frames, simulating realistic tracklet movements. The proposed method concurrently focuses on the hard negative samples.
Details \lk{of the \textbf{T}racklet-\textbf{G}uided \textbf{A}ugmentation (TGA)} are given below.

% Figure environment removed

We introduce the temporal information into spatial transformation. 
First, given a current frame $I^{t}$ with $M^{t}$ objects, we select a source-anchor object $o_{a}^{t}$, whose bounding box is denoted as $b_{a}^{t} = (cx_{a}^{t}, cy_{a}^{t}, w_{a}^{t}, h_{a}^{t})$. Then, we choose a target-anchor $o_{a}^{t \minus \tau}$ in $(t \minus \tau)_{th}$  historical frame from the pseudo-tracklet $trk_{a} = \{o_{a}^{t_0}, o_{a}^{t_1}, \cdots, o_{a}^{t}\}$. 
Finally, to augment the current $I^{t}$ to align with historical $I^{t \minus \tau}$,  a tracklet-guided affine transformation can be expressed as:

\begin{equation}
  \begin{bmatrix}
      x^{t \minus \tau} \\ y^{t \minus \tau} \\ 1
  \end{bmatrix}
  =
  \boldsymbol{M}_{t}^{t \minus \tau}
  \begin{bmatrix}
      x^{t} \\ y^{t} \\ 1
  \end{bmatrix}
  =
  \begin{bmatrix}
      m_{11} & m_{12} & m_{13} \\
      m_{21} & m_{22} & m_{23} \\
      0      & 0      & 1
  \end{bmatrix}
  \begin{bmatrix}
      x^{t} \\ y^{t} \\ 1
  \end{bmatrix}
  \label{eq:method_affine}
\end{equation}

\noindent where $x^*,y^*$ are spatial coordinates, and $\boldsymbol{M}_{t}^{t \minus \tau}$ can be solved by direct linear transform (DLT) algorithm ~\cite{detone2016deep}. 
% with the corner locations of the anchor pair $(o_{a}^{t} \!\sim\! o_{a}^{t \minus \tau})$. 
Then an augmented frame $\tilde{I}^{t}$ is generated based on the tracklet-guided affine transformation with perspective jitter, which can be expressed as $\tilde{I}^{t} = \mathcal{T}\left(I^{t}, M_{t}^{t \minus \tau} \right)$.
% \begin{equation}
%   \tilde{I}^{t} = \mathcal{T}\left(I^{t}, M_{t}^{t \minus \tau} \right)
%   \label{eq:method_aug}
% \end{equation}

Intuitively, a proper anchor-selection is vitally important for our augmentation strategy. 

First, the identity accuracy of anchor pair $(o_{a}^{t} \!\sim\! o_{a}^{t \minus \tau})$ is important. In other words, the consistency of anchor tracklet $trk_{a}$ should be guaranteed. We thus design a tracklet-level uncertain metric based on the propagated association-level uncertainty defined in \cref{eq:method_uncertain}, which is formulated as:

\begin{equation}
  \Omega_{i} = \frac{1}{n} \sum_{s=t_0}^{t} \exp (\delta_{i}^{s})
  % \Omega_{i} = \sqrt[n]{\sigma_{i}^{t_0} \cdot \sigma_{i}^{t_1} \cdots \sigma_{i}^{t}}
  \label{eq:method_tenergy}
\end{equation}

\noindent where $\Omega_{i}$ represents the uncertainty of tracklet $trk_{i}$, \lk{and $n$ is the tracklet length}.
An uncertainty-based sampling strategy is designed to select the source anchor $o_{a}^{t}$ (along with the anchor $trk_{a}$) from the $M^{t}$ objects in frame $I^{t}$, which can be formulated as:

\begin{equation}
  p\left(a=i \mid t \right) 
  % = softmax\left(-\Omega_{i}\right)
  = \frac{\exp{\left(-\Omega_{i}\right)}}{\sum_{\hat{i}=1}^{M^{t}}\exp{\left(-\Omega_{\hat{i}}\right)}}
  \label{eq:method_sel_an_src}
\end{equation}

\noindent where $p\left(a=i \mid t \right)$ represents the probability to choose the $i_{th}$ tracklet $trk_{i}$ as the anchor $trk_{a}$.
The uncertain tracklet with high $\Omega$ is less likely to be selected, avoiding dramatic augmentations from erroneous pseudo-tracklets.

Second, hard negative samples matters in discriminablity learning. We tend to choose an indistinguishable (or, high uncertain) target anchor $o_{a}^{t \minus \tau}$ along the tracklet $trk_{i}$. The selection probability can be formulated as:

\begin{equation}
  p\left(\pi=t \minus \tau \mid a \right) 
  = \frac{\exp{\left(\delta_{a}^{t \minus \tau}\right)}}{\sum_{\hat{\tau}=t_0}^{t-1}\exp{\left(\delta_{a}^{t-\hat{\tau}}\right)}}
  \label{eq:method_sel_an_tgt}
\end{equation}

\lk{A visualization example are displayed in the supplementary material to illustrate the hierarchical sampling process.}

Compared with conventional random transformation, the proposed tracklet-guided augmentation is well-directed and tracking-related. 
\lk{Together with accurate pseudo-tracklets, \mywork~successfully improves the inter-frame consistency, as shown in \cref{fig:method_disc_vis}. }


% Figure environment removed

% \subsection{Momentum Memory Dictionary}
% \label{sec:method_md}


%To reuse the encoded samples from the intermediate mini-batches, we maintain a queue for each video in the memory dictionary by enqueueing the $M^{t}$ objects in the current frame and removing the oldest samples.
%In representation learning, high-quality negative samples play an essential role~\cite{chen2020simple,he2020momentum}. However, existing unsupervised trackers only take negative samples from adjacent frames, augmented frames, and the current frame itself. The lack of negative sample diversity prevents trackers from learning discriminative representations. \mywork~adopts a momentum dictionary mechanism to alleviate this problem.

%As shown in~\cref{fig:method_fmwk}, we build a memory dictionary for each \textit{independent} video input during training. Given an input image $I^{t}$ from video $V$, we randomly sample a number of negative object samples from other videos in the memory dictionary, so as to enrich the negative sample diversity. To reuse the encoded samples from the intermediate mini-batches, we maintain a queue for each video in the memory dictionary by enqueueing the $M^{t}$ objects in the current frame and removing the oldest samples.
\section{Results}
\label{sec:results}
\vspace{ -0.5ex}
In this section we analyze the features considered during the speech deepfake detection task and assess the performances of the proposed detector in different scenarios. 
We express these in terms of \gls{roc} curves, \gls{auc} and balanced accuracy.
Optimal performances are reached when \gls{auc} and balanced accuracy are equal to one.

\vspace{.2em}
\noindent
\textbf{Feature analysis.}
As a first experiment, we analyze the characteristics of each of the considered feature sets. 
We want to investigate how much the information they provide is correlated to avoid the computation of redundant data.
Since the proposed system performs a fusion of the three different sets of features, we want to ensure the content of the three is orthogonal so that we increase the amount of information fed to the model.
As previously mentioned, we use these features as they analyze three different aspects of a speech signal. $\mathbf{f}_{\text{FD}}$ contains information about silences, $\mathbf{f}_{\text{STLT}}$ models speech, while $\mathbf{f}_{\text{B}}$ exploits bispectral information.

To measure the correlation between the three sets, we compute the Pearson coefficient for each pair of elements of the feature vectors.
The resulting matrix describes cross-correlations between different features and auto-correlations of each vector.
Figure~\ref{fig:correlation matrix} shows the absolute values of the results of this analysis computed on the ASVspoof 2019 \textit{train} dataset.
There, we can identify different rectangular regions for each feature vector. The bicoherence features are not clearly visible since they are much less numerous.
Although the single feature vectors present a quite high degree of internal correlation, the cross coefficients between them are low. This means that the three feature vectors do not strongly correlate and do not share much information.
This motivates the joint use of these features, increasing the model's detection accuracy and robustness and the use of \gls{fc} networks to perform dimensionality reduction and drop the redundant information within each feature set.

% Figure environment removed

\iffalse
% Figure environment removed
\fi

\vspace{.2em}
\noindent
\textbf{Detection results.}
In this experiment we train and validate the proposed system respectively on the \textit{train} and \textit{dev} partitions of the ASVspoof 2019 dataset and test it on the \textit{eval} set.
We do the same with the single models that consider just one feature set at a time to verify that the fusion of the three actually improves the detection capabilities of the model.
Figure~\ref{fig:roc} shows the result of this analysis, where we can see that the fused model outperforms all the single ones and leads to better performances than the considered baselines.


Furthermore, we have also tried to see if a majority voting strategy between the predictions of the single classifiers was more effective than the proposed approach.
The end-to-end architecture shows superior performances that justify its use, with the balanced accuracy that increases by \num{6}\%.
This is probably because the end-to-end model can choose how to use and aggregate the information provided by the three feature sets, leading to better results.

% Figure environment removed


\vspace{.5em}
\noindent
\textbf{Generalization results.}
We now assess the generalization capabilities of the proposed model by testing it on unseen data during training. 
This is an important aspect in multimedia forensics, where the developed detectors are often tested in conditions other than those considered during training and must be able to provide reliable predictions.
The datasets we consider in this experiment are Cloud19~\cite{Lieto2019}, LJSpeech~\cite{ljspeech}, and VidTIMIT~\cite{sanderson2009multi}. 
To improve the system performance in this scenario, we train it on both ASVspoof 2019 and LibriSpeech, following the same approach proposed in~\cite{conti2022deepfake}.
During training, we considered the weight-based strategy presented in the previous section.

Figure~\ref{fig:barplot} 
shows the results of this analysis, where the detector proves to have good generalization capabilities by scoring high accuracy values.
Furthermore, the joint training on ASVspoof and LibriSpeech has improved the accuracy values achieved on datasets never seen before.
As a drawback, the same strategy lowers the performance on ASVspoof 2019 \textit{eval}, with a balanced accuracy value that goes from \num{83.3} to \num{78.2} on the same dataset.
This is probably due to the fact that the system improves its generalization performance and becomes less prone to overfitting on ASVspoof.

% Figure environment removed

\iffalse 

% Figure environment removed

\fi

\vspace{.5em}
\noindent
\textbf{Anti-forensics attacks.}
In this experiment we tested the robustness of the developed system to anti-forensics attacks.
Synthetic speech signals usually contain traces and artifacts left by the generators, which can be leveraged to discriminate them.
However, post-processing operations can corrupt or remove these, making the detection performance more challenging.
This is the case of media content circulating online, where the low quality and the post-processing applied to the signals make it challenging to classify them.
For this reason, being able to perform the deepfake detection task even on processed data is crucial.

We consider two anti-forensics attacks, Gaussian noise injection and MP3 compression, and evaluate how the detector performance is affected. As in the previous test, we consider the model trained on both ASVspoof 2019 and LibriSpeech.
A small change to the system must be implemented when we consider Gaussian noise injection.
Since the noise increases the power of the signal, it is unfeasible to detect the silenced regions (silence threshold needs to be automatically estimated as the analyst does not know the amount of added noise) and compute the \gls{fd} features on them. For this reason, we will consider \gls{fd} features computed on the whole signal. This should not affect the system final performance, as~\cite{mari2022sound} showed that the detector achieves equivalent scores when the \gls{fd} features are computed on the whole sample or the silenced parts only.

Figure~\ref{fig:attacks} 
shows the balanced accuracy values achieved on ASVspoof \textit{eval} in the considered cases.
These show that the proposed method is robust to MP3 compression since the accuracy does not seem to decrease even if we compress the sample considerably.
On the other hand, Gaussian noise appears to be more problematic as the prediction becomes almost random when injecting noise with a SNR value equal to \num{2}.


% Figure environment removed

\iffalse
\begin{table}[!t]
    \centering
    \caption{Accuracies on the test sets and balanced accuracies in the ASVSpoof-eval dataset with anti-forensic attacks}
    \begin{tabular}{|c|c|c|c|c|}
        \hline
        & ASVspoof \textit{eval} & Cloud19 & LJSpeech & VidTimit\\
        \hline
        Datasets & 78.2 & 85.2 & 90.7 & 100.0\\
        \hline
        & Reference & rate=128 & rate=32 & -\\
        \hline
        MP3 & 78.2 & 78.2 & 77.0 & -\\
        \hline
        & Reference & SNR=42 & SNR=22 & SNR=2\\
        \hline
        Noise & 79.9 & 74.9 & 73.0 & 50.0\\
        \hline
    \end{tabular}
    \label{tab:anti}
\end{table}

\fi

We proposed a machine-learning based method to approximate diagonal as well as non-diagonal elements of the Hessian of a molecule. The representation used is specific for every internal coordinates, and takes explicitly into account the bond order, which is sensible because a single point DFT calculation is computationally considerably less expensive that the explicit calculation of the Hessian.
We trained our ML model on a relatively small dataset (subset of QM7) of less than 7000 molecules. The Hessian was computed at the B3LYP/cc-pVDZ level of theory. 
The agreement between ML and DFT was satisfactory. In particular, the calculated MAPE for bond stretching force constant was below 2\%, and was particularly small for bonds involving hydrogen atoms because they point outwards and are less affected by the chemical environment. The MAPE for bending and torsion was of 5\% and 10\%, respectively. 
From the ML model trained on QM7 we were also able to predict the Hessian of some molecules representative of the QM9 dataset. The Hessian predicted in internal coordinates was then transformed into the mass-weighted Cartesian Hessian, the diagonalization of which yields the harmonic vibrational frequencies and normal modes, that can be compared to the ones calculated  explicitly from DFT.

High frequency vibrations and normal modes were predicted accurately, while lower frequency ones were not. This behaviour is analogous to the IR spectroscopy theory, where stretchings and bendings can be identified accurately, while torsion and delocalized vibrations are more difficult to be interpreted.

The approximate Hessian obtained with ML is computational inexpensive and can be used as an initial Hessian guess for geometry optimization, or in the context of alchemical geometry relaxation \cite{Domenichini2020,domenichini2022alchemical, shiraogawa2022exploration,shiraogawa2023optimization}. 
A good starting Hessian may speed up the convergence of the geometrical optimization. An in detail assessment of the performance of the ML Hessian proposed is not yet provided, but should carefully take into account many parameters on which the optimization depends, \textit{e.g.} the type of molecule, the initial geometry, the optimization algorithm, and the Hessian update scheme.




\noindent\textbf{Acknowledgments.} This project was provided with computer and storage resources by GENCI at IDRIS thanks to the grant 2022-AD011013930 on the supercomputer Jean Zay's the V100 partition.
%
% ---- Bibliography ----
%
% BibTeX users should specify bibliography style 'splncs04'.
% References will then be sorted and formatted in the correct style.
%
\bibliographystyle{splncs04}
\bibliography{main}
%
\end{document}
