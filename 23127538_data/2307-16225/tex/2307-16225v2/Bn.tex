%\documentclass[twocolumn,prd,amsmath,amssymb,floatfix,nofootinbib]{revtex4}
\documentclass[preprint,prd,amsmath,amssymb,floatfix,nofootinbib]{revtex4}
%\documentclass[12pt]{article}

%\documentclass[12pt]{article}
%\usepackage{epsfig}
%\textwidth18.0cm
%\textheight25.7cm
%\headheight 0 cm
%\headsep 0 cm
%\topmargin-0.9cm                  
%\oddsidemargin-0.3cm
\usepackage[colorlinks=true,urlcolor=black]{hyperref}

\hypersetup{backref = true,pagebackref = true,hyperindex = true, colorlinks = true,
  breaklinks = true, urlcolor = blue, linkcolor = blue, bookmarks = true,
  bookmarksopen = true, citecolor=red}

%\usepackage{bbold}
\usepackage{bm}
\usepackage{color}
\usepackage{dcolumn}
%\usepackage{feynmf} %for drawing Feynman diagrams
\usepackage{graphics}
\usepackage{graphicx}
\usepackage{subfigure}
\usepackage{slashed}
\usepackage{placeins}
%\usepackage{esvect}

\allowdisplaybreaks
\usepackage{pdfpages}
\usepackage{eso-pic}
\usepackage{everyshi}
\usepackage{pdflscape}


%unit of length (for drawing Feynman diagrams)
\setlength{\unitlength}{1mm}


\newcommand{\z}{&&\hspace*{-1cm}}
\newcommand{\zz}{&&\hspace*{-4cm}}
\newcommand{\ep}{\varepsilon}
\newcommand{\cita} [1] {$^{\hbox{\scriptsize \cite{#1}}}$}
\newcommand{\prepr}[1] {\begin{flushright} {\bf #1} \end{flushright} \vskip 1.5cm}
%\newcommand{\titul}[1] {\begin{center}{\large\bf #1 } \end{center}\vskip 1.cm}
%\newcommand{\autor}[1] {\begin {center} {\large \lineskip .5em #1 } \end   {center} }
%\newcommand{\lugar}[1] {\begin{center} {\it #1} \end{center}}
%\newcommand{\abstr}[1] {{\begin{center} \vskip .5cm {\bf Abstract
%                        \vspace{0pt}} \end{center}}\begin{quote} #1
%                        \end{quote}}
\newcommand{\bea}{\begin{eqnarray}}
\newcommand{\eea}{\end{eqnarray}}
\newcommand{\be}{\begin{equation}}
\newcommand{\ee}{\end{equation}}
\newcommand{\MSbar}{\overline{\rm MS}}
\newcommand{\as}{\alpha_s}
\newcommand{\asMZ}{\alpha_s(M^2_Z)}
%\newcommand{\ar}{\overline a_s}
\newcommand{\dd}{\mathrm{d}}
\newcommand{\ar}{a_s}



\begin{document}

\title{Bjorken sum rule with analytic coupling at low $Q^2$ values}
%       \author{A.\ V.~Kotikov$^{1}$ and I.A.~Zemlyakov$^{1,2}$}       }
\author{I.R. Gabdrakhmanov$^{1}$, N.A Gramotkov$^{1}$, A.V.~Kotikov$^{1}$, D.A. Volkova$^{1,2}$ and I.A.~Zemlyakov$^{1,3}$}
\affiliation{$^1$
  %Bogoliubov Laboratory of Theoretical Physics,
  Joint Institute for Nuclear Research, 141980 Dubna, Russia;\\
 $^2$Dubna State University,
%Universitetskaya str. 19 141 980,
  141980 Dubna, Moscow Region, Russia;\\
  $^3$Tomsk State University,
%Universitetskaya str. 19 141 980,
  634010 Tomsk,
  %Moscow Region,
  Russia
}

\date{\today}

\begin{abstract}

  The experimental data obtained for the polarized Bjorken sum rule $\Gamma^{p-n}_1(Q^2)$ for small values of $Q^2$
  are approximated by the predictions obtained in the framework of analytic QCD up to the 5th order perturbation theory, whose coupling constant
  does not contain the Landau pole.
  We found an excellent agreement between the experimental data and the predictions of analytic QCD, as well as a strong difference between
  these data and the results obtained in the framework of
  %\textcolor{blue}{
  perturbative
  %}
  %  standard
  QCD.
 
\end{abstract}




\maketitle

\section{Introduction}

Polarized Bjorken sum rule (BSR) $\Gamma^{p-n}_1(Q^2)$ \cite{Bjorken:1966jh,Bjorken:1969mm}, i.e. the difference between the first moments of the spin-dependent
structure functions (SFs) of a proton and neutron, is a very important space-like QCD observable \cite{Deur:2018roz,Kuhn:2008sy}.
Its isovector nature facilitates its theoretical description in perturbative QCD (pQCD) in terms of the operator product expansion (OPE),
compared to the corresponding SF integrals for each nucleon.
%Second, qualitative
Experimental results for this quantity obtained in polarized deep inelastic scattering (DIS) are currently available in a wide
range of the spacelike squared momenta $Q^2$: 0.021 GeV$^2 \leq Q^2 <$ 5 GeV$^2$
\cite{Deur:2021klh,E143:1998hbs,E154:1997xfa,E142:1996thl,E155:1999pwm,E155:2000qdr,SpinMuon:1993gcv,SpinMuonSMC:1997voo,SpinMuonSMC:1994met,SpinMuon:1995svc,SpinMuonSMC:1997ixm,SpinMuonSMC:1997mkb,COMPASS:2005xxc,Compass:2007qxf,COMPASS:2010wkz,COMPASS:2015mhb,COMPASS:2016jwv,COMPASS:2017hef,HERMES:1997hjr,HERMES:1998cbu,HERMES:2006jyl,Deur:2004ti,Deur:2008ej,Deur:2014vea,ResonanceSpinStructure:2008ceg}.
In particular, the most recent experimental results \cite{Deur:2021klh} with significantly
reduced statistical uncertainties, derived mainly from the Jefferson Lab EG4 experiment on
%CLAS EG1-DVCS experiment [11] on
polarized protons and deuterons and E97110 one on polarized ${}^3H$,
%targets,
make BSR an attractive quantity for testing various pQCD generalizations at low $Q^2$ values: $Q^2 \leq 1$ GeV$^2$.


Theoretically, pQCD (with OPE) in the $\overline{MS}$-scheme was the usual approach to describing such quantities.
This approach, however, has the theoretical disadvantage that the running coupling constant ({\it couplant}) $\alpha_s(Q^2)$
%[$\equiv \alpha_s(Q^2)/(4\pi)$]
has the Landau singularities for small $Q^2$ values: $Q^2 \leq 0.1$ GeV$^2$, which makes it inconvenient for estimating spacelike observables at small $Q^2$,
such as BSR.
%In recent years, fitting the theoretical OPE expression to the inelastic contributions to the BSR [21] with good results.
In the recent years, the extension of pQCD couplings for low $Q^2$ without Landau singularities called (fractional) analytic perturbation theory [(F)APT)]
\cite{ShS,Shirkov:1997wi,MSS,Shirkov:2000qv,Shirkov:2001sm,BMS1,Bakulev:2006ex,Bakulev:2010gm}
(or the minimal analytic (MA) theory \cite{Cvetic:2008bn}), were applied to match the theoretical OPE expression with the experimental BSR data
\cite{Pasechnik:2008th,Pasechnik:2009yc,Kotikov:2012eq,Khandramai:2011zd,Ayala:2017uzx,Ayala:2017ucf,Ayala:2018ulm,Ayala:2020scz}.



Following \cite{Cvetic:2006mk,Cvetic:2006gc}, we introduce here the derivatives (in the $k$-order of perturbation theory (PT))
\be
\tilde{a}^{(k)}_{n+1}(Q^2)=\frac{(-1)^n}{n!} \, \frac{d^n a^{(k)}_s(Q^2)}{(dL)^n},~~a^{(k)}_s(Q^2)=\frac{\beta_0 \alpha^{(k)}_s(Q^2)}{4\pi}=\beta_0\,\overline{a}^{(k)}_s(Q^2),
\label{tan+1}
\ee
which are very convenient in the case of analytic QCD. $\beta_0$ is the first coefficient of the QCD $\beta$-function:
\be
\beta(\overline{a}^{(k)}_s)=-{\left(\overline{a}^{(k)}_s\right)}^2 \bigl(\beta_0 + \sum_{i=1}^k \beta_i {\left(\overline{a}^{(k)}_s\right)}^i\bigr),
\label{bQCD}
\ee
where $\beta_i$ are known up to $k=4$ \cite{Baikov:2008jh}.

The series of derivatives $\tilde{a}_{n}(Q^2)$ can successfully replace the corresponding series of $\ar$-powers (see, e.g. \cite{Kotikov:2022swl}). Indeed, each
derivative reduces the $\ar$ power but is accompanied by an additional $\beta$-function $\sim \ar^2$.
Thus, each application of a derivative yields an additional $\ar$, and thus  it is indeed possible to use a series of derivatives instead of
a series of $\ar$-powers.


In LO, the series of derivatives $\tilde{a}_{n}(Q^2)$ are exactly the same as $\ar^{n}$. Beyond LO, the relationship between $\tilde{a}_{n}(Q^2)$
and $\ar^{n}$ was established in \cite{Cvetic:2006mk,Cvetic:2010di} and extended to the fractional case, where $n \to$ is a non-integer $\nu $, in Ref.
\cite{GCAK}.

In this short paper, we apply the inverse logarithmic expansion of the MA couplants, recently obtained in \cite{Kotikov:2022sos,Kotikov:2023meh} for any PT order
%of perturbation theory
(see Ref. \cite{Kotikov:2022vnx,Kotikov:2023nvz} for a brief introduction).
This approach is very convenient: for LO the MA couplants have simple representations (see \cite{BMS1}), while beyond LO the MA couplants are very close to LO ones,
especially for $Q^2 \to \infty$ and $Q^2 \to 0$, where the differences between MA couplants of various PT orders become insignificant.
Moreover, for $Q^2 \to \infty$ and $Q^2 \to 0$ the (fractional) derivatives of the MA couplants with $n\geq 2$ tend to zero, and therefore only the first
term in perturbative expansions makes a valuable contribution.



\section{Bjorken sum rule}

The polarized BSR
%Bjorken sum rule
is defined as the difference between the proton and neutron polarized SFs,
%structure functions,
integrated over the entire interval $x$
\be
\Gamma_1^{p-n}(Q^2)=\int_0^1 \, dx\, \bigl[g_1^{p}(x,Q^2)-g_1^{n}(x,Q^2)\bigr].
\label{Gpn} 
\ee


Theoretically, the quantity can be written in the OPE
%Operator Produxt Expansion
form
%\textcolor{blue}{
(see Ref. \cite{Shuryak:1981pi,Balitsky:1989jb})
%}
\be
\Gamma_1^{p-n}(Q^2)=
%\left|\frac{g_A}{g_V}\right| \,
\frac{g_A}{6} \, \bigl(1-D_{\rm BS}(Q^2)\bigr) + \sum_{i=2}^{\infty} \frac{\mu_{2i}(Q^2)}{Q^{2i-2}} \, ,
\label{Gpn.OPE} 
\ee
where $g_A$=1.2762 $\pm$ 0.0005 \cite{PDG20} is
%the ratio of
the nucleon axial charge, $(1-D_{BS}(Q^2))$ is the leading-twist
contribution, and $\mu_{2i}/Q^{2i-2}$ $(i\geq 1)$ are the higher-twist (HT)
contributions.
\footnote{Below, in our analysis, the so-called elastic contribution will always be excluded.}

Since we include very small $Q^2$ values here, the representation (\ref{Gpn.OPE}) of the HT contributions is inconvenient.
It is much better to use the so-called ``massive'' representation for the HT part (introduced in Ref. \cite{Teryaev:2013qba,Khandramai:2016kbh}):
\be
\Gamma_1^{p-n}(Q^2)=
%\left|\frac{g_A}{g_V}\right| \,
\frac{g_A}{6} \, \bigl(1-D_{\rm BS}(Q^2)\bigr) +\frac{\hat{\mu}_4 M^2}{Q^{2}+M^2} \, ,
\label{Gpn.mOPE} 
\ee
where the values of $\hat{\mu}_4$ and $M^2$ have been fitted in Refs. \cite{Ayala:2017uzx,Ayala:2018ulm}
in the different analytic QCD models.

In the case of MA QCD, from \cite{Ayala:2018ulm} one can see that in (\ref{Gpn.mOPE})
%\textcolor{blue}{
\be
M^2=0.439 \pm 0.012 \pm 0.463
%\textcolor{blue}{\pm 0.463},
%~~\tilde{\mu}_{\rm{MA},4}=-0.082,
~~\hat{\mu}_{\rm{MA},4}
%=\tilde{\mu}_{\rm{MA},4}/M^2
=-0.173 \pm 0.002\pm 0.666\,,
%\textcolor{blue}{\pm 0.666\,} ,
\label{M,mu} 
\ee
where
%\textcolor{blue}{
the statistical (small) and systematic (large)
%}
%only statistical
uncertainties are presented.
%(systematic uncertainties can be found in \cite{Ayala:2018ulm}).
%}

Another form, which is correct at very small $Q^2$ values, has been proposed in \cite{Gabdrakhmanov:2017dvg}
\be
\Gamma_1^{p-n}(Q^2)=%\left|\frac{g_A}{g_V}\right| \,
\frac{g_A}{6} \, \bigl(1-D_{\rm BS}(Q^2)\bigr) +\frac{\hat{\mu}_4M^2(Q^{2}+M^2)}{(Q^{2}+M^2)^2+M^2\sigma^2} \, ,
\label{Gpn.mOPE.T} 
\ee
where small value $\sigma \equiv \sigma_{\rho}  =145$ MeV (the $\rho$-meson decay width) has been used.
%{\bf We will use this expression below but with arbitrary values of $\sigma$.}


Up to the $k$-th PT order,
%of perturbation theory,
the perturbative part has the form
\be
D^{(1)}_{\rm BS}(Q^2)=\frac{4}{\beta_0} \, a^{(1)}_s,~~D^{(k\geq2)}_{\rm BS}(Q^2)=\frac{4}{\beta_0} \, a^{(k)}_s\left(1+\sum_{m=1}^{k-1} d_m \bigl(a^{(k)}_s\bigr)^m
%d_1a_s+d_2a_s^2+d_3a^3_s
\right)\,,
\label{DBS} 
\ee
where $d_1$, $d_2$ and $d_3$ are known from exact calculations (see, e.g., \cite{Chen:2006tw,Chen:2005tda}). The exact $d_4$ value is not known, but it was recently
estimated  in Ref. \cite{Ayala:2022mgz}.

Converting the powers of couplant into its derivatives, we have
\be
D^{(1)}_{\rm BS}(Q^2)=\frac{4}{\beta_0} \, \tilde{a}^{(1)}_1,~~D^{(k\geq2)}_{\rm BS}(Q^2)=
\frac{4}{\beta_0} \, \left(\tilde{a}^{(k)}_{1}+\sum_{m=2}^k\tilde{d}_{m-1}\tilde{a}^{(k)}_{m}
%\tilde{d}_1\tilde{a}_2+\tilde{d}_2\tilde{a}_3+\tilde{d}_3\tilde{a}_4
\right),
\label{DBS.1} 
\ee
where
\bea
&&\tilde{d}_1=d_1,~~\tilde{d}_2=d_2-b_1d_1,~~\tilde{d}_3=d_3-\frac{5}{2}b_1d_2-\bigl(b_2-\frac{5}{2}b^2_1\bigr)\,d_1,\nonumber \\
&&\tilde{d}_4=d_4-\frac{13}{3}b_1d_3 -\bigl(3b_2-\frac{28}{3}b^2_1\bigr)\,d_2-\bigl(b_3-\frac{22}{3}b_1b_2+\frac{28}{3}b^3_1\bigr)\,d_1
\label{tdi} 
\eea
and $b_i=\beta_i/\beta_0^{i+1}$.

For the case of 3 active quark flavors ($f=3$), we have
\footnote{The coefficients $\beta_i$ $(i\geq 0)$ of the QCD $\beta$-function (\ref{bQCD})
  and, consequently, the couplant $\alpha_s(Q^2)$ itself depend on  the number $f$ of active quark flavors, and each new quark enters/leaves
  the game at a certain threshold $Q^2_f$ according to \cite{Chetyrkin:2005ia,Schroder:2005hy,Kniehl:2006bg}. The corresponding QCD parameters
  $\Lambda^{(f)}$ in N$^i$LO of PT can be found
  in Ref. \cite{Chen:2021tjz}.}
\bea
&&d_1=1.59,~~d_2=3.99,~~d_3=15.42~~d_4=63.76, \nonumber \\ 
&&\tilde{d}_1=1.59,~~\tilde{d}_2=2.73,
~~\tilde{d}_3=8.61,~~\tilde{d}_4=21.52 \, ,
\label{td123} 
\eea
i.e., the coefficients in the series of derivatives are slightly smaller.
%\textcolor{red}{Before there were wrong numbers $\tilde{d}_2=2.51$,~ $\tilde{d}_3=10.58$.}

In MA QCD, the results (\ref{Gpn.mOPE.T}) become as follows
\be
\Gamma_{\rm{MA},1}^{p-n}(Q^2)=%\left|\frac{g_A}{g_V}\right| \,
\frac{g_A}{6} \, \bigl(1-D_{\rm{MA,BS}}(Q^2)\bigr) +\frac{\hat{\mu}_{\rm{MA},4}M^2(Q^{2}+M^2)}{(Q^{2}+M^2)^2+M^2\sigma^2},~~
%\Bigl(\tilde{\mu}_{\rm{MA},4}(Q^2)= \hat{\mu}_{\rm{MA},4}(Q^2)\,M^2\Bigr)
\,,
\label{Gpn.MA} 
\ee
where the perturbative part $D_{\rm{BS,MA}}(Q^2)$
takes the form
\be
D^{(1)}_{\rm MA,BS}(Q^2)=\frac{4}{\beta_0} \, A_{\rm MA}^{(1)},~~
D^{k\geq2}_{\rm{MA,BS}}(Q^2) =\frac{4}{\beta_0} \, \Bigl(A^{(k)}_{\rm MA}
+ \sum_{m=2}^{k} \, \tilde{d}_{m-1} \, \tilde{A}^{(k)}_{\rm MA,\nu=m} \Bigr)\,.
\label{DBS.ma} 
\ee



\section{Results}


\begin{table}[t]
\begin{center}
\begin{tabular}{|c|c|c|c|}
\hline
& $M^2$ for $\sigma=\sigma_{\rho}$ & $\hat{\mu}_{\rm{MA},4}$  for $\sigma=\sigma_{\rho}$ & $\chi^2/({\rm d.o.f.})$ for $\sigma=\sigma_{\rho}$ \\
& (for $\sigma=0$) & (for $\sigma=0$) & (for $\sigma=0$) \\
 \hline
 LO & 1.592 $\pm$ 0.300 & -0.168 $\pm$ 0.002 & 0.788  \\
 & (1.631 $\pm$ 0.301) & (-0.166 $\pm$ 0.001) & (0.789)  \\
 \hline
 NLO & 1.505 $\pm$ 0.286 & -0.157 $\pm$ 0.002 & 0.755  \\
 & (1.545 $\pm$ 0.287) & (-0.155 $\pm$ 0.001) & (0.757)  \\
 \hline
  N$^2$LO & 1.378 $\pm$ 0.242 & -0.159 $\pm$ 0.002 & 0.728  \\
 & (1.417 $\pm$ 0.241) & (-0.156 $\pm$ 0.002) & (0.728)  \\
 \hline
  N$^3$LO & 1.389 $\pm$ 0.247 & -0.159 $\pm$ 0.002 & 0.747  \\
  & (1.429 $\pm$ 0.248) & (-0.157 $\pm$ 0.002) & (0.747)  \\
   \hline
  N$^4$LO & 1.422 $\pm$ 0.259 & -0.159 $\pm$ 0.002 & 0.754  \\
 & (1.462 $\pm$ 0.259) & (-0.157 $\pm$ 0.001) & (0.754)  \\
 \hline
\end{tabular}
\end{center}
\caption{%
The values of the fit parameters with $\sigma=\sigma_{\rho}$ ($\sigma=0$).}
\label{Tab:BSR}
\end{table}



%\textcolor{blue}{
The calculation results taking into account only statistical uncertainties
%}
are presented in Table \ref{Tab:BSR} and in Fig. \ref{fig:APTHT}.
Here we use the $Q^2$-independent $M$ and $\hat{\mu}_4$ values and the twist-two parts shown in Eqs. (\ref{DBS.1}) and (\ref{DBS.ma}) for the cases
of usual PT
%perturbation theory
and APT, respectively.

In the case of using MA couplants,  we see in Table \ref{Tab:BSR} that the cases $\sigma=0$ and $\sigma=\sigma_{\rho}$ lead to very similar values for
the fitting parameters and  $\chi^2$-factor.
%results.
So, in Fig. \ref{fig:APTHT} we show only the case
with $\sigma=\sigma_{\rho}$.
%
%In the case of using MA couplants, the
The quality of the fits is very good, as evidenced quantitatively by the values of $\chi^2/({\rm d.o.f.})$. Moreover, our results obtained for different PT orders
are very similar to each others: the corresponding curves in Fig. \ref{fig:APTHT}  are indistinguishable. One can also see the important role of the
twist-four term
%\textcolor{blue}{
(see also Refs. \cite{Khandramai:2011zd} and \cite{Kataev:2005ci,Kataev:2005hv} and discussions therein).
%}.
Without it, the value of $\Gamma_1^{p-n}(Q^2)$ is about 0.16, which is very far from the experimental data.

At $Q^2 \leq 0.3$ GeV$^2$
we also see good agreement with the phenomenological models: Burkert-Ioffe one \cite{Burkert:1992tg,Burkert:1993ya} and especially
LFHQCD one \cite{Brodsky:2014yha}. For larger $Q^2$ values our results are below the results of the phenomenological models
%LFHQCD form
and at $Q^2 \geq 0.5$ GeV$^2$ are below the experimental data. We hope to improve agreement with using ``massive'' forms of HT
%higher twist
contributions $h_{2i}$ with $i\geq3$. This is a subject of future investigations.


% Figure environment removed

As seen in Fig. \ref{fig:APTHT}, the results obtained using conventional couplants are not good  and worse for the NLO case to compare to the LO one. Indeed,
the deterioration increases with the PT order in this case (see \cite{Pasechnik:2008th,Pasechnik:2009yc,Ayala:2017uzx,Ayala:2018ulm,Kotikov:2022sos}).
Thus, the use of the ``massive'' twist-four form (\ref{Gpn.mOPE}) does not improve these results, since at $Q^2 \to \Lambda_i^2$ conventional
couplants become to be singular, that leads to large and negative results for the twist-two part (\ref{DBS}). As the PT order increases,
usual couplants become singular for ever larger $Q^2$ values, while BSR tends to negative values for ever larger $Q^2$ values
(see, e.g.,  Fig. 15 in \cite{Kotikov:2022sos}).
Thus, the discrepancy between theory and experiment  increases with the PT order.
%of the perturbation theory. 


%%%%%%%%%%%%%%%%%%%%%%%%%%%%%%%%%%%%%%%%%%%%%%%%%%%%%%%%%%%%%%%%%%%%%%%%%%%%%%%%%%%%%%%%%%%%%%%%%%%%%%%%%%
\section{Conclusions}

We have considered the Bjorken sum rule in the framework of MA and
%\textcolor{blue}{
perturbative
%}
%conventional
QCD and obtained results similar to those obtained in previous
studies \cite{Pasechnik:2008th,Pasechnik:2009yc,Ayala:2017uzx,Ayala:2018ulm,Kotikov:2022sos} for the first 4 orders of PT.
%perturbation theory.
The results based on the conventional PT do not agree with the experimental data. For some $Q^2$ values, the PT results become negative, since the
high-order corrections are large and enter the twist-two term with a minus sign.
APT in the minimal version leads to a good agreement with experimental data when we used the ``massive'' version (\ref{Gpn.MA}) for
the twist-four contributions.

%\textcolor{red}{
Now we would like to discuss the photoproduction (PhP) case, i.e. the $Q^2\to0$ limit.
  In MA QCD,
  %we can consider the $Q^2 \to 0$ limit, where
  $A_{\rm MA}(Q^2=0)=1$ and $\tilde{A}^{(k)}_{\rm MA,m}=0$ for $m>1$ and we have
  \be
D_{\rm MA,BS}(Q^2=0)=\frac{4}{\beta_0} 
%\label{DBS.ma.Q0} 
%\ee
~~\mbox{and}~~
%\be
\Gamma_{\rm{MA},1}^{p-n}(Q^2=0,\sigma=0)=%\left|\frac{g_A}{g_V}\right| \,
\frac{g_A}{6} \, \bigl(1-\frac{4}{\beta_0}\bigr) +\hat{\mu}_{\rm{MA},4}
%\Bigl(\tilde{\mu}_{\rm{MA},4}(Q^2)= \hat{\mu}_{\rm{MA},4}(Q^2)\,M^2\Bigr)
\,.
\label{Gpn.MA.Q0} 
\ee
%}
%\textcolor{red}{
The finitness of cross-section in the real photon limit 
%The GDH sum rule
leads \cite{Teryaev:2013qba,Khandramai:2016kbh,Gabdrakhmanov:2017dvg}
\be
\Gamma_{\rm{MA},1}^{p-n}(Q^2=0)=0
%\label{Gpn.GDH} 
%\ee
~~\mbox{and, hence,}~~
%\be 
\hat{\mu}^{\rm php}_{\rm{MA},4}=-\frac{g_A}{6} \, \bigl(1-\frac{4}{\beta_0}\bigr).
\label{mu.GDH} 
\ee
%}
%\textcolor{red}{
In the case of 3 active quarks, i.e. $f=3$, we have
\be 
\hat{\mu}^{\rm php}_{\rm{MA},4}=-0.118
%\label{mu.GDH} 
%\ee
~~\mbox{and, hence,}~~
%\be 
|\hat{\mu}^{\rm php}_{\rm{MA},4}|< |\hat{\mu}_{\rm{MA},4}|,
\label{mu.GDH} 
\ee
shown in Table I.\\
%}
%\textcolor{red}{
  So, in our fits the finitness of cross-section in the real photon limit
  %GDH sum rule
  is violated.
  \footnote{
    %\textcolor{blue}{
    Note that the results for $\hat{\mu}_{\rm{MA},4}$ were obtained taking into account only statistical uncertainties.
    With systematic uncertainties, the results for $\hat{\mu}^{\rm php}_{\rm{MA},4}$ and $\hat{\mu}_{\rm{MA},4}$
    are in full agreement with each other, but the quality of our analysis is greatly reduced.}
  %}


%    We note that the results ontained taking into account only statistical uncertainties. With the large systematic
%    uncertainties we have an agreement

  This is a common situation that appears as a consequence of the use of analytic versions of QCD for the Bjorken sum rule
  (see, e.g., Ref. \cite{Ayala:2018ulm}). Note that our results for $\hat{\mu}_4$ shown in Table I are smaller than in \cite{Ayala:2018ulm}.\\
In our future investigations 
  we plan to improve this analysis by taking several ``massive'' twists by analogy with twist-four one shown in Eq. (\ref{Gpn.mOPE}).
  We hope that this will lead to better agrement with the real photon limit and with the studies in Ref. \cite{Soffer:1992ck,Soffer:2004ip,Pasechnik:2010fg}.\\
  %}\\


%  \section{Acknowledgments}
Authors are grateful to Alexandre P. Deur for information about new experimental data in Ref. \cite{Deur:2021klh} and discussions.
%\textcolor{blue}{
The authors are also grateful to Andrei Kataev, Nikolai Nikolaev and Oleg Teryaev for criticism, leading to a sharp
improvement in the quality of the paper.
%}
This work was supported in part by the Foundation for the Advancement of Theoretical
Physics and Mathematics “BASIS”.



%\appendix
%\def\theequation{A\arabic{equation}}
%\setcounter{equation}{0}

%\section{QCD $\beta$-function}




\begin{thebibliography}{99}

%\cite{Bjorken:1966jh}
\bibitem{Bjorken:1966jh}
J.~D.~Bjorken,
%``Applications of the Chiral U(6) x (6) Algebra of Current Densities,''
Phys. Rev. \textbf{148}, 1467-1478 (1966)
%doi:10.1103/PhysRev.148.1467

%\cite{Bjorken:1969mm}
\bibitem{Bjorken:1969mm}
J.~D.~Bjorken,
%``Inelastic Scattering of Polarized Leptons from Polarized Nucleons,''
Phys. Rev. D \textbf{1}, 1376-1379 (1970)
%doi:10.1103/PhysRevD.1.1376


%\cite{Deur:2018roz}
\bibitem{Deur:2018roz}
A.~Deur, S.~J.~Brodsky and G.~F.~De T\'eramond,
%``The Spin Structure of the Nucleon,''
%doi:10.1088/1361-6633/ab0b8f
[arXiv:1807.05250 [hep-ph]]

%\cite{Kuhn:2008sy}
\bibitem{Kuhn:2008sy}
S.~E.~Kuhn, J.~P.~Chen and E.~Leader,
%``Spin Structure of the Nucleon - Status and Recent Results,''
Prog. Part. Nucl. Phys. \textbf{63}, 1-50 (2009)
%doi:10.1016/j.ppnp.2009.02.001
%[arXiv:0812.3535 [hep-ph]].

  %\cite{Deur:2021klh}
\bibitem{Deur:2021klh}
  A.~Deur, J.~P.~Chen, S.~E.~Kuhn
  %, C.~Peng, M.~Ripani, V.~Sulkosky, K.~Adhikari, M.~Battaglieri, V.~D.~Burkert and G.~D.~Cates,
  \textit{et al.}
%``Experimental study of the behavior of the Bjorken sum at very low Q2,''
Phys. Lett. B \textbf{825}, 136878 (2022)
%doi:10.1016/j.physletb.2022.136878
%[arXiv:2107.08133 [nucl-ex]].



%\cite{E143:1998hbs}
\bibitem{E143:1998hbs}
K.~Abe, T. Akagi, P. L. Anthony \textit{et al.} [E143 Collaboration],
%``Measurements of the proton and deuteron spin structure functions g(1) and g(2),''
Phys. Rev. D \textbf{58}, 112003 (1998)
%doi:10.1103/PhysRevD.58.112003
%[arXiv:hep-ph/9802357 [hep-ph]].

%\cite{E154:1997xfa}
\bibitem{E154:1997xfa}
K.~Abe, T. Akagi, B. D. Anderson \textit{et al.} [E154 Collaboration],
%``Precision determination of the neutron spin structure function g1(n),''
Phys. Rev. Lett. \textbf{79}, 26-30 (1997)
%doi:10.1103/PhysRevLett.79.26
%[arXiv:hep-ex/9705012 [hep-ex]];


%\cite{E142:1996thl}
\bibitem{E142:1996thl}
P.~L.~Anthony, R. G. Arnold, H. R. Band \textit{et al.} [E142 Collaboration],
%``Deep inelastic scattering of polarized electrons by polarized He-3 and the study of the neutron spin structure,''
Phys. Rev. D \textbf{54}, 6620-6650 (1996)
%doi:10.1103/PhysRevD.54.6620
%[arXiv:hep-ex/9610007 [hep-ex]];

%\cite{E155:1999pwm}
\bibitem{E155:1999pwm}
P.~L.~Anthony, R. G. Arnold, T. Averett \textit{et al.} [E155 Collaboration],
%``Measurement of the deuteron spin structure function g1(d)(x) for 1-(GeV/c)**2 \ensuremath{<} Q**2 \ensuremath{<} 40-(GeV/c)**2,''
Phys. Lett. B \textbf{463}, 339-345 (1999)
%doi:10.1016/S0370-2693(99)00940-5
%[arXiv:hep-ex/9904002 [hep-ex]];

%\cite{E155:2000qdr}
\bibitem{E155:2000qdr}
P.~L.~Anthony, R. G. Arnold, T. Averett \textit{et al.} [E155 Collaboration],
%``Measurements of the Q**2 dependence of the proton and neutron spin structure functions g(1)**p and g(1)**n,''
Phys. Lett. B \textbf{493}, 19-28 (2000)
%doi:10.1016/S0370-2693(00)01014-5
%[arXiv:hep-ph/0007248 [hep-ph]].

%\cite{SpinMuon:1993gcv}
\bibitem{SpinMuon:1993gcv}
B.~Adeva, S Ahmad, A Arvidson \textit{et al.} [Spin Muon Collaboration],
%``Measurement of the spin dependent structure function g1(x) of the deuteron,''
Phys. Lett. B \textbf{302}, 533-539 (1993)
%doi:10.1016/0370-2693(93)90438-N

%\cite{SpinMuonSMC:1997voo}
\bibitem{SpinMuonSMC:1997voo}
B.~Adeva \textit{et al.} [Spin Muon Collaboration (SMC)],
%``The Spin dependent structure function g(1) (x) of the proton from polarized deep inelastic muon scattering,''
Phys. Lett. B \textbf{412}, 414-424 (1997)
%doi:10.1016/S0370-2693(97)01106-4

%\cite{SpinMuonSMC:1994met}
\bibitem{SpinMuonSMC:1994met}
D.~Adams, B. Adeva, E. Arik \textit{et al.} [Spin Muon Collaboration (SMC)],
%``Measurement of the spin dependent structure function $g_1(x)$ of the proton,''
Phys. Lett. B \textbf{329}, 399-406 (1994)
[erratum: Phys. Lett. B \textbf{339}, 332-333 (1994)]
%doi:10.1016/0370-2693(94)90793-5
%[arXiv:hep-ph/9404270 [hep-ph]];

%\cite{SpinMuon:1995svc}
\bibitem{SpinMuon:1995svc}
D.~Adams, B. Adeva, E. Arik \textit{et al.} [Spin Muon Collaboration],
%``A New measurement of the spin dependent structure function g1(x) of the deuteron,''
Phys. Lett. B \textbf{357}, 248-254 (1995)
%doi:10.1016/0370-2693(95)00898-U

%\cite{SpinMuonSMC:1997ixm}
\bibitem{SpinMuonSMC:1997ixm}
D.~Adams, B. Adeva, T. Akdogan \textit{et al.} [Spin Muon  Collaboration(SMC)],
%``The Spin dependent structure function g1(x) of the deuteron from polarized deep inelastic muon scattering,''
Phys. Lett. B \textbf{396}, 338-348 (1997)
%doi:10.1016/S0370-2693(97)00159-7

%\cite{SpinMuonSMC:1997mkb}
\bibitem{SpinMuonSMC:1997mkb}
D.~Adams, B. Adeva, E. Arik \textit{et al.} [Spin Muon Collaboration (SMC)],
%``Spin structure of the proton from polarized inclusive deep inelastic muon - proton scattering,''
Phys. Rev. D \textbf{56}, 5330-5358 (1997)
%doi:10.1103/PhysRevD.56.5330
%[arXiv:hep-ex/9702005 [hep-ex]];

%\cite{COMPASS:2005xxc}
\bibitem{COMPASS:2005xxc}
E.~S.~Ageev, V.Yu. Alexakhin, Yu. Alexandrov \textit{et al.} [COMPASS Collaboration],
%``Measurement of the spin structure of the deuteron in the DIS region,''
Phys. Lett. B \textbf{612}, 154-164 (2005)
%doi:10.1016/j.physletb.2005.03.025
%[arXiv:hep-ex/0501073 [hep-ex]];

%\cite{Compass:2007qxf}
\bibitem{Compass:2007qxf}
E.~S.~Ageev, V.Yu. Alexakhin, Yu. Alexandrov \textit{et al.} [Compass Collaboration],
%``Spin asymmetry A1(d) and the spin-dependent structure function g1(d) of the deuteron at low values of x and Q**2,''
Phys. Lett. B \textbf{647}, 330-340 (2007)
%doi:10.1016/j.physletb.2007.02.034
%[arXiv:hep-ex/0701014 [hep-ex]];

%\cite{COMPASS:2010wkz}
\bibitem{COMPASS:2010wkz}
M.~G.~Alekseev, V.Yu. Alexakhin, Yu. Alexandrov \textit{et al.} [COMPASS  Collaboration],
%``The Spin-dependent Structure Function of the Proton $g_1^p$ and a Test of the Bjorken Sum Rule,''
Phys. Lett. B \textbf{690}, 466-472 (2010)
%doi:10.1016/j.physletb.2010.05.069
%[arXiv:1001.4654 [hep-ex]];

%\cite{COMPASS:2015mhb}
\bibitem{COMPASS:2015mhb}
C.~Adolph, R. Akhunzyanov, M.G. Alexeev \textit{et al.} [COMPASS  Collaboration],
%``The spin structure function $g_1^{\rm p}$ of the proton and a test of the Bjorken sum rule,''
Phys. Lett. B \textbf{753}, 18-28 (2016)
%doi:10.1016/j.physletb.2015.11.064
%[arXiv:1503.08935 [hep-ex]];

%\cite{COMPASS:2016jwv}
\bibitem{COMPASS:2016jwv}
C.~Adolph, M. Aghasyan, R. Akhunzyanov \textit{et al.} [COMPASS  Collaboration],
%``Final COMPASS results on the deuteron spin-dependent structure function $g_1^{\rm d}$ and the Bjorken sum rule,''
Phys. Lett. B \textbf{769}, 34-41 (2017)
%doi:10.1016/j.physletb.2017.03.018
%[arXiv:1612.00620 [hep-ex]];

%\cite{COMPASS:2017hef}
\bibitem{COMPASS:2017hef}
M.~Aghasyan, R. Akhunzyanov, M.G. Alexeev \textit{et al.} [COMPASS  Collaboration],
%``Longitudinal double-spin asymmetry $A_1^{\rm p}$ and spin-dependent structure function $g_1^{\rm p}$ of the proton at small values of $x$ and $Q^2$,''
Phys. Lett. B \textbf{781}, 464-472 (2018)
%doi:10.1016/j.physletb.2018.03.044
%[arXiv:1710.01014 [hep-ex]].


%\cite{HERMES:1997hjr}
\bibitem{HERMES:1997hjr}
K.~Ackerstaff, A. Airapetian, I. Akushevich \textit{et al.} [HERMES Collaboration],
%``Measurement of the neutron spin structure function g1(n) with a polarized He-3 internal target,''
Phys. Lett. B \textbf{404}, 383-389 (1997)
%doi:10.1016/S0370-2693(97)00611-4
%[arXiv:hep-ex/9703005 [hep-ex]];

%\cite{HERMES:1998cbu}
\bibitem{HERMES:1998cbu} 
A.~Airapetian, N. Akopov, I. Akushevich \textit{et al.} [HERMES Collaboration],
%``Measurement of the proton spin structure function g1(p) with a pure hydrogen target,''
Phys. Lett. B \textbf{442}, 484-492 (1998)
%doi:10.1016/S0370-2693(98)01341-0
%[arXiv:hep-ex/9807015 [hep-ex]];

%\cite{HERMES:2006jyl}
\bibitem{HERMES:2006jyl}
  A.~Airapetian, N. Akopov, Z. Akopov \textit{et al.} [HERMES Collaboration],
%``Precise determination of the spin structure function g(1) of the proton, deuteron and neutron,''
Phys. Rev. D \textbf{75}, 012007 (2007)
%doi:10.1103/PhysRevD.75.012007
%[arXiv:hep-ex/0609039 [hep-ex]].

%\cite{Deur:2004ti}
\bibitem{Deur:2004ti}
  A.~Deur, P.~E.~Bosted, V.~Burkert
  %, G.~Cates, J.~P.~Chen, S.~Choi, D.~Crabb, C.~W.~de Jager, R.~De Vita and G.~E.~Dodge,
  \textit{et al.}
%``Experimental determination of the evolution of the Bjorken integral at low Q**2,''
Phys. Rev. Lett. \textbf{93}, 212001 (2004)
%doi:10.1103/PhysRevLett.93.212001
%[arXiv:hep-ex/0407007 [hep-ex]];

%\cite{Deur:2008ej}
\bibitem{Deur:2008ej}
  A.~Deur, P.~Bosted, V.~Burkert, D.~Crabb, V.~Dharmawardane, G.~E.~Dodge, T.~A.~Forest, K.~A.~Griffioen, S.~E.~Kuhn, R.~Minehart, and Y. Prok,
%  5\textit{et al.}
%``Experimental study of isovector spin sum rules,''
Phys. Rev. D \textbf{78}, 032001 (2008)
%doi:10.1103/PhysRevD.78.032001
%[arXiv:0802.3198 [nucl-ex]];

%\cite{Deur:2014vea}
\bibitem{Deur:2014vea}
A.~Deur, Y.~Prok, V.~Burkert, D.~Crabb, F.~X.~Girod, K.~A.~Griffioen, N.~Guler, S.~E.~Kuhn and N.~Kvaltine,
%``High precision determination of the $Q^2$ evolution of the Bjorken Sum,''
Phys. Rev. D \textbf{90}, no.1, 012009 (2014)
%doi:10.1103/PhysRevD.90.012009
%[arXiv:1405.7854 [nucl-ex]];

%\cite{ResonanceSpinStructure:2008ceg}
\bibitem{ResonanceSpinStructure:2008ceg}
K.~Slifer, O.A. Rondon, A. Aghalaryan, \textit{et al.} [Resonance Spin Structure],
%``Probing Quark-Gluon Interactions with Transverse Polarized Scattering,''
Phys. Rev. Lett. \textbf{105}, 101601 (2010)
%doi:10.1103/PhysRevLett.105.101601
%[arXiv:0812.0031 [nucl-ex]].





\bibitem{ShS} 
%\cite{Shirkov:1996cd}
D.~V.~Shirkov and I.~L.~Solovtsov,
%``Analytic QCD running coupling with finite IR behavior and universal $\alpha_s (0)$ value,''
[arXiv:hep-ph/9604363 [hep-ph]]

\bibitem{Shirkov:1997wi}
D.~V.~Shirkov and I.~L.~Solovtsov,
%``Analytic model for the QCD running coupling with universal $\alpha_s(0)$ value,''
Phys. Rev. Lett. \textbf{79} (1997), 1209-1212
%doi:10.1103/PhysRevLett.79.1209
%[arXiv:hep-ph/9704333 [hep-ph]].

\bibitem{MSS}
%\bibitem{Milton:1997mi}
K.~A.~Milton, I.~L.~Solovtsov and O.~P.~Solovtsova,
%``Analytic perturbation theory and inclusive tau decay,''
Phys. Lett. B \textbf{415} (1997), 104-110
%doi:10.1016/S0370-2693(97)01207-0
%[arXiv:hep-ph/9706409 [hep-ph]].

%\bibitem{Sh}
\bibitem{Shirkov:2000qv}
D.~V.~Shirkov,
%``Analytic perturbation theory for QCD observables,''
Theor. Math. Phys. \textbf{127} (2001), 409-423
%doi:10.1023/A:1010302206227
%[arXiv:hep-ph/0012283 [hep-ph]];

\bibitem{Shirkov:2001sm}
D.~V.~Shirkov,
%``Analytic perturbation theory in analyzing some QCD observables,''
Eur. Phys. J. C \textbf{22} (2001), 331-340
%doi:10.1007/s100520100794
%[arXiv:hep-ph/0107282 [hep-ph]].


\bibitem{BMS1}
%\bibitem{Bakulev:2005gw}
A.~P.~Bakulev, S.~V.~Mikhailov and N.~G.~Stefanis,
%``QCD analytic perturbation theory: From integer powers to any power of the running coupling,''
Phys. Rev. D \textbf{72} (2005), 074014  [Erratum-ibid.  D \textbf{72} (2005), 119908]
%doi:10.1103/PhysRevD.72.074014
%[arXiv:hep-ph/0506311 [hep-ph]].

%\cite{Bakulev:2006ex}
\bibitem{Bakulev:2006ex}
A.~P.~Bakulev, S.~V.~Mikhailov and N.~G.~Stefanis,
%``Fractional Analytic Perturbation Theory in Minkowski space and application to Higgs boson decay into a b anti-b pair,''
Phys. Rev. D \textbf{75} (2007), 056005
[erratum: Phys. Rev. D \textbf{77} (2008), 079901];
%doi:10.1103/PhysRevD.77.079901
%[arXiv:hep-ph/0607040 [hep-ph]].

%\cite{Bakulev:2010gm}
\bibitem{Bakulev:2010gm}
A.~P.~Bakulev, S.~V.~Mikhailov and N.~G.~Stefanis,
%``Higher-order QCD perturbation theory in different schemes: From FOPT to CIPT to FAPT,''
JHEP \textbf{06} (2010), 085
%doi:10.1007/JHEP06(2010)085
%[arXiv:1004.4125 [hep-ph]].

%\cite{Cvetic:2008bn}
\bibitem{Cvetic:2008bn}
G.~Cvetic and C.~Valenzuela,
%``Analytic QCD: A Short review,''
Braz. J. Phys. \textbf{38} (2008), 371-380
%[arXiv:0804.0872 [hep-ph]].


%\cite{Pasechnik:2008th}
\bibitem{Pasechnik:2008th}
R.~S.~Pasechnik, D.~V.~Shirkov and O.~V.~Teryaev,
%``Bjorken Sum Rule and pQCD frontier on the move,''
Phys. Rev. D \textbf{78} (2008), 071902
%doi:10.1103/PhysRevD.78.071902
%[arXiv:0808.0066 [hep-ph]].

%\cite{Pasechnik:2009yc}
\bibitem{Pasechnik:2009yc}
R.~S.~Pasechnik, D.~V.~Shirkov, O.~V.~Teryaev, O.~P.~Solovtsova and V.~L.~Khandramai,
%``Nucleon spin structure and pQCD frontier on the move,''
Phys. Rev. D \textbf{81} (2010), 016010
%doi:10.1103/PhysRevD.81.016010
%[arXiv:0911.3297 [hep-ph]].

%\cite{Kotikov:2012eq}
\bibitem{Kotikov:2012eq}
A.~V.~Kotikov and B.~G.~Shaikhatdenov,
%``Perturbative QCD analysis of the Bjorken sum rule,''
Phys. Part. Nucl. \textbf{45} (2014), 26-29
%doi:10.1134/S1063779614010535
%[arXiv:1212.6834 [hep-ph]].

%\cite{Khandramai:2011zd}
\bibitem{Khandramai:2011zd}
V.~L.~Khandramai, R.~S.~Pasechnik, D.~V.~Shirkov, O.~P.~Solovtsova and O.~V.~Teryaev,
%``Four-loop QCD analysis of the Bjorken sum rule vs data,''
Phys. Lett. B \textbf{706} (2012), 340-344
%doi:10.1016/j.physletb.2011.11.023
%[arXiv:1106.6352 [hep-ph]].

%\cite{Ayala:2017uzx}
\bibitem{Ayala:2017uzx}
C.~Ayala, G.~Cvetic, A.~V.~Kotikov and B.~G.~Shaikhatdenov,
%``Bjorken sum rule in QCD frameworks with analytic (holomorphic) coupling,''
Int. J. Mod. Phys. A \textbf{33} (2018) no.18n19, 1850112
%doi:10.1142/S0217751X18501129
%[arXiv:1708.06284 [hep-ph]].

%\cite{Ayala:2017ucf}
\bibitem{Ayala:2017ucf}
C.~Ayala, G.~Cveti\v{c}, A.~V.~Kotikov and B.~G.~Shaikhatdenov,
%``Bjorken sum rule in QCD with analytic coupling,''
J. Phys. Conf. Ser. \textbf{938} (2017) no.1, 012055
%doi:10.1088/1742-6596/938/1/012055
%[arXiv:1712.06048 [hep-ph]].


%\cite{Ayala:2018ulm}
\bibitem{Ayala:2018ulm}
C.~Ayala, G.~Cveti\v{c}, A.~V.~Kotikov and B.~G.~Shaikhatdenov,
%``Bjorken polarized sum rule and infrared-safe QCD couplings,''
Eur. Phys. J. C \textbf{78}, no.12, 1002 (2018)
%doi:10.1140/epjc/s10052-018-6490-9
%[arXiv:1812.01030 [hep-ph]].

%\cite{Ayala:2020scz}
\bibitem{Ayala:2020scz}
C.~Ayala, G.~Cveti\v{c}, A.~V.~Kotikov and B.~G.~Shaikhatdenov,
%``Bjorken sum rule with analytic QCD coupling,''
J. Phys. Conf. Ser. \textbf{1435} (2020) no.1, 012016
%doi:10.1088/1742-6596/1435/1/012016




%\cite{Cvetic:2006mk}
\bibitem{Cvetic:2006mk}
G.~Cvetic and C.~Valenzuela,
%``An Approach for evaluation of observables in analytic versions of QCD,''
J. Phys. G \textbf{32}, L27 (2006)
%doi:10.1088/0954-3899/32/6/L01
%[arXiv:hep-ph/0601050 [hep-ph]].

%\cite{Cvetic:2006gc}
\bibitem{Cvetic:2006gc}
G.~Cvetic and C.~Valenzuela,
%``Various versions of analytic QCD and skeleton-motivated evaluation of observables,''
Phys. Rev. D \textbf{74} (2006), 114030
[erratum: Phys. Rev. D \textbf{84} (2011), 019902]
%doi:10.1103/PhysRevD.74.114030
%[arXiv:hep-ph/0608256 [hep-ph]].


%\cite{Baikov:2008jh}
\bibitem{Baikov:2008jh}
P.~A.~Baikov, K.~G.~Chetyrkin and J.~H.~Kuhn,
%``Order alpha**4(s) QCD Corrections to Z and tau Decays,''
Phys. Rev. Lett. \textbf{101}, 012002 (2008)
%doi:10.1103/PhysRevLett.101.012002
%[arXiv:0801.1821 [hep-ph]].

%\cite{Kotikov:2022swl}
\bibitem{Kotikov:2022swl}
A.~V.~Kotikov and I.~A.~Zemlyakov,
%``About derivatives in analytic QCD,''
Pisma Zh. Eksp. Teor. Fiz. \textbf{115} (2022) no.10, 609
%doi:10.1134/S0021364022600628




%\cite{Cvetic:2010di}
\bibitem{Cvetic:2010di}
G.~Cvetic, R.~Kogerler and C.~Valenzuela,
%``Reconciling the analytic QCD with the ITEP operator product expansion philosophy,''
Phys. Rev. D \textbf{82} (2010), 114004
%doi:10.1103/PhysRevD.82.114004
%[arXiv:1006.4199 [hep-ph]].

\bibitem{GCAK}
%\bibitem{Cvetic:2011ym}
G.~Cveti\v{c} and A.~V.~Kotikov,
%``Analogs of noninteger powers in general analytic QCD,''
J. Phys. G \textbf{39} (2012), 065005
%doi:10.1088/0954-3899/39/6/065005
%[arXiv:1106.4275 [hep-ph]].

%\cite{Kotikov:2022sos}
\bibitem{Kotikov:2022sos}
A.~V.~Kotikov and I.~A.~Zemlyakov,
%``Fractional analytic QCD beyond leading order,''
J. Phys. G \textbf{50}, no.1, 015001 (2023)
%doi:10.1088/1361-6471/ac99ce
%[arXiv:2203.09307 [hep-ph]].

%\cite{Kotikov:2023meh}
\bibitem{Kotikov:2023meh}
A.~V.~Kotikov and I.~A.~Zemlyakov,
%``Fractional analytic QCD beyond leading order in the timelike region,''
Phys. Rev. D \textbf{107}, no.9, 094034 (2023)
%doi:10.1103/PhysRevD.107.094034
%[arXiv:2302.12171 [hep-ph]].

%\cite{Kotikov:2022vnx}
\bibitem{Kotikov:2022vnx}
A.~V.~Kotikov and I.~A.~Zemlyakov,
%``About Fractional Analytic QCD beyond Leading Order,''
[arXiv:2207.01330 [hep-ph]]

%\cite{Kotikov:2023nvz}
\bibitem{Kotikov:2023nvz}
A.~V.~Kotikov and I.~A.~Zemlyakov,
%``About Fractional Analytic QCD,''
[arXiv:2302.13769 [hep-ph]].


%\vskip -0.7cm
%\textcolor{blue}{
%\cite{Shuryak:1981pi}
\bibitem{Shuryak:1981pi}
E.~V.~Shuryak and A.~I.~Vainshtein,
%``Theory of Power Corrections to Deep Inelastic Scattering in Quantum Chromodynamics. 2. Q**4 Effects: Polarized Target,''
Nucl. Phys. B \textbf{201}, 141 (1982)
%doi:10.1016/0550-3213(82)90377-7

%\cite{Balitsky:1989jb}
\bibitem{Balitsky:1989jb}
I.~I.~Balitsky, V.~M.~Braun and A.~V.~Kolesnichenko,
%``Power corrections 1 / Q**2 to parton sum rules for deep inelastic scattering from polarized targets,''
Phys. Lett. B \textbf{242}, 245-250 (1990)
[erratum: Phys. Lett. B \textbf{318}, 648 (1993)]
%doi:10.1016/0370-2693(90)91465-N
%[arXiv:hep-ph/9310316 [hep-ph]].
%}


\bibitem{PDG20}
Particle Data Group collaboration, P.A. Zyla, R.M Barnett, J. Beringer et al., Review of Particle Physics, PTEP {\bf 2020} (2020) 083C01.


%\cite{Teryaev:2013qba}
\bibitem{Teryaev:2013qba}
O.~Teryaev,
%``Analyticity and higher twists,''
Nucl. Phys. B Proc. Suppl. \textbf{245} (2013), 195-198
%doi:10.1016/j.nuclphysbps.2013.10.039
%[arXiv:1309.1985 [hep-ph]].

%\cite{Khandramai:2016kbh}
\bibitem{Khandramai:2016kbh}
V.~L.~Khandramai, O.~V.~Teryaev and I.~R.~Gabdrakhmanov,
%``Infrared modified QCD couplings and Bjorken sum rule,''
J. Phys. Conf. Ser. \textbf{678} (2016) no.1, 012018
%doi:10.1088/1742-6596/678/1/012018

%\cite{Gabdrakhmanov:2017dvg}
\bibitem{Gabdrakhmanov:2017dvg}
I.~R.~Gabdrakhmanov, O.~V.~Teryaev and V.~L.~Khandramai,
%``Infrared models for the Bjorken sum rule in the APT approach,''
J. Phys. Conf. Ser. \textbf{938} (2017) no.1, 012046
%doi:10.1088/1742-6596/938/1/012046


%\cite{Chen:2006tw}
\bibitem{Chen:2006tw}
J.~P.~Chen,
%``Spin sum rules and polarizabilities: results from Jefferson lab,''
[arXiv:nucl-ex/0611024 [nucl-ex]]

%\cite{Chen:2005tda}
\bibitem{Chen:2005tda}
J.~P.~Chen, A.~Deur and Z.~E.~Meziani,
%``Sum rules and moments of the nucleon spin structure functions,''
Mod. Phys. Lett. A \textbf{20} (2005), 2745-2766
%doi:10.1142/S021773230501875X
%[arXiv:nucl-ex/0509007 [nucl-ex]].

%\cite{Ayala:2022mgz}
\bibitem{Ayala:2022mgz}
C.~Ayala and A.~Pineda,
%``Bjorken sum rule with hyperasymptotic precision,''
Phys. Rev. D \textbf{106}, no.5, 056023 (2022)
%doi:10.1103/PhysRevD.106.056023
%[arXiv:2208.07389 [hep-ph]].


%\cite{Chetyrkin:2005ia}
\bibitem{Chetyrkin:2005ia}
K.~G.~Chetyrkin, J.~H.~Kuhn and C.~Sturm,
%``QCD decoupling at four loops,''
Nucl. Phys. B \textbf{744} (2006), 121-135
%doi:10.1016/j.nuclphysb.2006.03.020
%[arXiv:hep-ph/0512060 [hep-ph]].

%\cite{Schroder:2005hy}
\bibitem{Schroder:2005hy}
Y.~Schroder and M.~Steinhauser,
%``Four-loop decoupling relations for the strong coupling,''
JHEP \textbf{01} (2006), 051
%doi:10.1088/1126-6708/2006/01/051
%[arXiv:hep-ph/0512058 [hep-ph]].

%\cite{Kniehl:2006bg}
\bibitem{Kniehl:2006bg}
B.~A.~Kniehl, A.~V.~Kotikov, A.~I.~Onishchenko and O.~L.~Veretin,
%``Strong-coupling constant with flavor thresholds at five loops in the anti-MS scheme,''
Phys. Rev. Lett. \textbf{97} (2006), 042001
%doi:10.1103/PhysRevLett.97.042001
%[arXiv:hep-ph/0607202 [hep-ph]].

%\cite{Chen:2021tjz}
\bibitem{Chen:2021tjz}
H.~M.~Chen, L.~M.~Liu, J.~T.~Wang, M.~Waqas and G.~X.~Peng,
%``Matching-invariant running of quark masses in quantum chromodynamics,''
Int. J. Mod. Phys. E \textbf{31}, no.02, 2250016 (2022)
%doi:10.1142/S0218301322500161
%[arXiv:2110.11776 [hep-ph]].

%\vskip -0.7cm
%\textcolor{blue}{
%\cite{Kataev:2005ci}
\bibitem{Kataev:2005ci}
A.~L.~Kataev,
%``Infrared renormalons and the relations between the Gross-Llewellyn Smith and the Bjorken polarized and unpolarized sum rules,''
JETP Lett. \textbf{81}, 608-611 (2005)
%doi:10.1134/1.2034588
%[arXiv:hep-ph/0505108 [hep-ph]].

%\cite{Kataev:2005hv}
\bibitem{Kataev:2005hv}
A.~L.~Kataev,
%``Deep inelastic sum rules at the boundaries between perturbative and nonperturbative QCD,''
Mod. Phys. Lett. A \textbf{20}, 2007-2022 (2005)
%doi:10.1142/S0217732305018165
%[arXiv:hep-ph/0505230 [hep-ph]].
%}

%\cite{Burkert:1992tg}
\bibitem{Burkert:1992tg}
V.~D.~Burkert and B.~L.~Ioffe,
%``On the Q**2 variation of spin dependent deep inelastic electron - proton scattering,''
Phys. Lett. B \textbf{296}, 223-226 (1992)
%doi:10.1016/0370-2693(92)90831-N

%\cite{Burkert:1993ya}
\bibitem{Burkert:1993ya}
V.~D.~Burkert and B.~L.~Ioffe,
%``Polarized structure functions of proton and neutron and the Gerasimov-Drell-Hearn and Bjorken sum rules,''
J. Exp. Theor. Phys. \textbf{78}, 619-622 (1994)
%CEBAF-PR-93-034.

%\cite{Brodsky:2014yha}
\bibitem{Brodsky:2014yha}
S.~J.~Brodsky, G.~F.~de Teramond, H.~G.~Dosch and J.~Erlich,
%``Light-Front Holographic QCD and Emerging Confinement,''
Phys. Rept. \textbf{584}, 1-105 (2015)
%doi:10.1016/j.physrep.2015.05.001
%[arXiv:1407.8131 [hep-ph]].

%\vskip -0.7cm
%\textcolor{red}{
%\cite{Soffer:1992ck}
\bibitem{Soffer:1992ck}
J.~Soffer and O.~Teryaev,
%``The Role of g-2 in relating the Schwinger and Gerasimov-Drell-Hearn sum rules,''
Phys. Rev. Lett. \textbf{70}, 3373-3375 (1993)
%doi:10.1103/PhysRevLett.70.3373

%\cite{Soffer:2004ip}
\bibitem{Soffer:2004ip}
J.~Soffer and O.~Teryaev,
%``QCD radiative and power corrections and generalized GDH sum rules,''
Phys. Rev. D \textbf{70}, 116004 (2004)
%doi:10.1103/PhysRevD.70.116004
%[arXiv:hep-ph/0410228 [hep-ph]].

%\cite{Pasechnik:2010fg}
\bibitem{Pasechnik:2010fg}
R.~S.~Pasechnik, J.~Soffer and O.~V.~Teryaev,
%``Nucleon spin structure at low momentum transfers,''
Phys. Rev. D \textbf{82}, 076007 (2010)
%doi:10.1103/PhysRevD.82.076007
%[arXiv:1009.3355 [hep-ph]].
%}

\end{thebibliography}


\end{document}


