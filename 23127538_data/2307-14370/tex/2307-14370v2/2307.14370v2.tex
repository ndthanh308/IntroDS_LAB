
%version 3
\documentclass[prb,
%twocolumn,
superscriptaddress,showpacs,amsmath,amssymb]{revtex4}
\usepackage{amsfonts}
\usepackage{bm}
\usepackage{verbatim}

\usepackage{graphicx}

%\documentstyle[12pt]{article}
%\begin{document}

%\textheight 8.9in %s7
%\oddsidemargin -0mm \evensidemargin -0mm \topmargin -1.8cm \textwidth 6.5in

 \begin{document}
\title{Views on gravity from condensed matter physics}

\author{G.E.~Volovik}
\affiliation{Low Temperature Laboratory, Aalto University,  P.O. Box 15100, FI-00076 Aalto, Finland}
\affiliation{Landau Institute for Theoretical Physics, acad. Semyonov av., 1a, 142432,
Chernogolovka, Russia}

\date{\today}








 

\begin{abstract}
{In the paper "Life, the Universe, and everything—42 fundamental questions",
Roland Allen and Suzy Lidstr\"om presented personal selection of the fundamental questions. Here, based on the condensed matter experience, we suggest the answers to some questions concerning the vacuum energy, black hole entropy and the origin of gravity. In condensed matter we know both the many-body phenomena emerging on the macroscopic level and the microscopic (atomic) physics, which generates this emergence. It appears that the same macroscopic phenomenon may be generated by essentially different microscopic backgrounds. This points to various possible directions in study of the deep quantum vacuum of our Universe.}
\end{abstract}

\maketitle

\tableofcontents




\section{Introduction: 42 fundamental questions}

Below are the questions presented in the paper by Allen and
Lidstr\"om “Life, the Universe, and everything--42 fundamental questions”.\cite{Perspective}

1. {\bf Why does conventional physics predict a cosmological constant that is vastly
too large?}

2. {\bf What is the dark energy?}

3. {\bf How can Einstein gravity be reconciled with quantum mechanics?}

4. {\bf What is the origin of the entropy and temperature of black holes?}

5. {\bf Is information lost in a black hole?}

6. {\bf Did the Universe pass through a period of inflation, and if so how and why?}

7. Why does matter still exist?

8. {\bf What is the dark matter?}

9. {\bf Why are the particles of ordinary matter copied twice at higher energy?}

10. {\bf What is the origin of particle masses, and what kind of masses do
neutrinos have?}

11. {\bf Does supersymmetry exist, and why are the energies of observed particles so
small compared to the most fundamental (Planck) energy scale?}

12. {\bf What is the fundamental grand unified theory of forces, and why?}

13. {\bf Are Einstein relativity and standard field theory always valid?}

14. Is our Universe stable?

15. Are quarks always confined inside the particles that they compose?

16. What are the complete phase diagrams for systems with nontrivial forces,
such as the strong nuclear force?

17. {\bf What new particles remain to be discovered?}

18. What new astrophysical objects are awaiting discovery?

19. What new forms of superconductivity and superfluidity remain to be
discovered?

20. What new topological phases remain to be discovered?

21. What further properties remain to be discovered in highly correlated
electronic materials?

22. What other new phases and forms of matter remain to be discovered?

23. What is the future of quantum computing, quantum information, and other
applications of entanglement?

24. What is the future of quantum optics and photonics?

25. {\bf Are there higher dimensions, and if there is an internal space, what is its
geometry?}

26. Is there a multiverse?

27. Are there exotic features in the geometry of spacetime, perhaps including
those which could permit time travel?

28. How did the Universe originate, and what is its fate?

29. {\bf What is the origin of spacetime, why is spacetime four-dimensional, and why
is time different from space?}

30. {\bf What explains relativity and Einstein gravity?}

31. {\bf Why do all forces have the form of gauge theories?}

32. {\bf Why is Nature described by quantum fields?}

33. Is physics mathematically consistent?

34. What is the connection between the formalism of physics and the reality of
human experience?

35. What are the ultimate limits to theoretical, computational, experimental, and
observational techniques?

36. What are the ultimate limits of chemistry, applied physics, and technology?

37. What is life?

38. How did life on Earth begin and how did complex life originate?

39. How abundant is life in the Universe, and what is the destiny of life?

40. How does life solve problems of seemingly impossible complexity?

41. Can we understand and cure the diseases that afflict life?

42. What is consciousness?


Condensed matter has its answers to the most of the questions, right or wrong. In some cases there are several answers to the same question, which may contradict to each other. For example, there are at least 6 different answers to the question  \#30 concerning the origin of  relativity and Einstein gravity discussed in Sec. \ref{gravity}.  However, they may open different directions in our research. 
 The questions which are (at least partially) touched in these notes are given in boldface, they are mostly related to gravity. The answers to some other questions can be found in Ref. \cite{Volovik2020}.

\section{Cosmological constant and vacuum energy}
\label{VacuumEnergy}
 

According to standard physics, the vacuum has an enormous energy density $\rho_{\rm vac}$. A positive contribution comes from the
zero-point energy of bosonic fields, such as electromagnetic field, and a negative contribution
comes from the fermionic fields -- from the so called Dirac vacuum, and there is no reason why they should cancel.\cite{Weinberg1989}
Again according to standard physics, $\rho_{\rm vac}$ should act as a gravitational source
-- effectively an enormous cosmological constant $\Lambda_{\rm vac}$. With the Planck scale providing a natural cutoff it is roughly 120 orders of magnitude
larger than is compatible with observations.
According to Bjorken, this is the oft-repeated mantra that "no one has any idea as to why the cosmological constant is so small".\cite{Bjorken2011}

However, anyone who is familiar with condensed matter physics can immediately find a loophole in this logic, simply by considering the ground state of the liquid or solid quantum material consisting for example of the atoms of one sort. This many body system is described by the thermodynamic potential $\epsilon -\mu n$, where $\epsilon$ is the energy density, $n$ is particle density and $\mu$ the chemical potential. According to the Gibbs-Duhem identity,  at zero temperature one has $\epsilon -\mu n=-P$, where $P$ is pressure.  That is why this  thermodynamic potential  is  equivalent to $\rho_{\rm vac}$, which obeys the equation of states characterizing the dark energy:
 \begin{equation}
 \rho_{\rm vac}=-P_{\rm vac}\,,
\label{EquationOfState}
\end{equation}

The low energy modes of condensed matter,  such as phonons and electronic excitations,  are described in terms of bosonic and  fermionic quantum fields.\cite{AGD} These fields have zero-point energies, which contribute to the thermodynamic potential and do not cancel each other. Nevertheless,  in the absence of environment, i.e. at zero pressure, the thermodynamic potential of the system is exactly zero,   $\epsilon -\mu n=-P=0$. 

What happens with the diverging contribution of the zero-point-energies?  It is automatically cancelled by the microscopic (atomic) degrees of freedom due to  the Gibbs--Duhem identity. According to Einstein, "thermodynamics ... is the only physical theory of universal content", and thus
the thermodynamic identity should be equally applicable to equilibrium ground state of any condensed matter system and to the physical vacuum, whatever is its content. The thermodynamics dictates that the enormous contribution of zero-point energies of quantum fields to the vacuum energy should be automatically cancelled by the microscopic (now trans-Planckian) degrees of freedom of the quantum vacuum.

If the vacuum is in full equilibrium, one has $\rho_{\rm vac}=-P_{\rm vac}=0$, and this does not depend on the physical content of the quantum vacuum.
However, in the expanding Universe, where the vacuum is out of equilibrium and/or is contaminated by matter, the vacuum energy deviates from its equilibrium value. In the early Universe the vacuum energy could be huge (of Planck scale), but in the present epoch the Universe is old and close to equilibrium:  its expansion is slow,  the matter is highly dilute, and thus the vacuum energy (the dark energy) only slightly deviates from its zero value, being on the order of the perturbations of the vacuum state caused by expansion.  
\cite{KlinkhamerVolovik2008a,KlinkhamerVolovik2008b}


Finally the answer to the question  \#1 "Why does conventional physics predict a cosmological constant that is vastly too large?" is the following. The "conventional physics" ignores the condensed matter lesson that in the full equilibrium the diverging zero-point energy of quantum fields is cancelled by  the microscopic (trans-Planckian) degrees of freedom. The reason for cancellation is the general laws of thermodynamics, which are the same for relativistic and non-relativistic vacua (ground states). 

The condensed matter also demonstrates, how the initial states with large $\rho_{\rm vac}$ relax to the equilibrium state with $\rho_{\rm vac}=0$. It is typically accompanied by rapid oscillations,\cite{VolkovKogan1974,Barankov2004,Yuzbashyan2005,Yuzbashyan2008,Gurarie2009}
which are similar to oscillations after inflation.\cite{KlinkhamerVolovik2008b} 
These spacetime-dependent perturbations behave gravitationally as a pressureless perfect fluid, which may serve as a candidate for dark matter.\cite{KlinkhamerVolovik2017}

\section{Entropy of horizon}
\label{Horizon}

\subsection{Entropy of expanding Universe}
\label{deSitter}



Condensed matter demonstrates the two-component behavior of some systems with broken symmetry.\cite{Volovik2003} These two components are: superfluid vacuum (analog of quantum vacuum) + quasiparticles (analog of matter). This can be applied to expanding Universe, where the vacuum degrees of freedom typically have much higher entropy density than the matter degrees of freedom.\cite{Volovik2023}  The de Sitter state is an example of the perturbed quantum vacuum. Its energy density is $\rho_{\rm vac}=\frac{3H^2}{8\pi G}$, where $H$ is the Hubble parameter of expansion.
This state can be considered as the thermal state of the quantum vacuum with local temperature $T_{\rm dS}=H/\pi$,\cite{Volovik2023}  which is twice the Gibbons-Hawking temperature. The local temperature determines the local processes, which are not related to horizon. For example, this temperature determines the ionization rate of atom in de Sitter environment, or the decay rate of the bound state, which is stable in flat spacetime.\cite{Maxfield2022}   The calculations of the ionization rate of atom are within the well-known physics, and do not require the existence of the horizon and the ultraviolet physics. 

In de Sitter expansion, the "superfluid" component satisfies the corresponding analog of the Gibbs-Duhem relation:
 \begin{equation}
T_{\rm dS}s_{\rm dS}=\rho_{\rm vac}+P_{\rm vac} -K{\cal R}=-K{\cal R}\,,
\label{GibbsDuhem}
\end{equation}
where the gravitational coupling $K=1/16\pi G$ and the scalar Riemann curvature ${\cal R}$ are the thermodynamically conjugate variables describing the gravitational degrees of freedom of quantum vacuum and $s_{\rm dS}$ is the entropy density of the de Sitter state, $s_{\rm dS}=3H/4G=3 \pi T/4G$. We also used the vacuum equation of states in Eq.(\ref{EquationOfState}): $P_{\rm vac}=w\rho_{\rm vac}$ with $w=-1$. 

Eq.(\ref{GibbsDuhem}) suggests that one may introduce the effective pressure, which is modified by gravitational degrees of freedom, $\tilde P_{\rm vac}= P_{\rm vac} -K{\cal R}$. Then one has 
\begin{equation}
T_{\rm dS}s_{\rm dS}=\rho_{\rm vac}+\tilde P_{\rm vac}\,,
\label{GibbsDuhem2}
\end{equation}
where the pressure $\tilde P_{\rm vac}$ corresponds to equation of state of stiff matter introduced by Zel'dovich,\cite{Zeldovich1962} $\tilde P_{\rm vac}=w\rho_{\rm vac}$ with $w=1$.
For stiff matter, the speed of sound $s$ is equal to the speed of light, $s^2=c^2 dP/d\epsilon=c^2$. 

In this vacuum thermodynamics, the total entropy in the volume $V_H$ surrounded by the cosmological horizon with radius $R=1/H$ is
\begin{equation}
S_{\rm dS}=s_{\rm dS}V_H=\frac{4\pi R^3}{3} s _{\rm dS}= \frac{\pi}{GH^2}=\frac{A}{4G} \,,
\label{dSEntropy}
\end{equation}
where $A$ is the horizon area. This agrees with the Gibbons-Hawking entropy of  the cosmological horizon. However, the origin of the entropy of horizon is the local entropy of the de Sitter quantum vacuum, rather than the entropy of the horizon degrees of freedom. This is related to the question  \#4 concerning the entropy. The application to the black hole is in the next subsection.

\subsection{Entropy of black hole}
\label{BH}

 The same thermodynamic approach can be applied to black holes.
As distinct from the de Sitter state, the black hole is the compact object, and its thermodynamics connects the global parameters, such as mass $M$, entropy of horizon $S_{\rm BH}$, total electric charge $Q$ and total angular momentum $J$. This global thermodynamics can be described by the integral form of the  Gibbs-Duhem relation. The local Gibbs-Duhem relation for the Schwarzschild black hole is: 
\begin{equation}
T_{\rm BH} s({\bf r}) =\epsilon({\bf r})- K{\cal R}({\bf r})\,,
\label{LocalGibbs}
\end{equation}
where $T_{\rm BH}=\frac{1}{8\pi MG}$ is the Hawking temperature.
Using  the  scalar Riemann curvature coming from the central singularity,\cite{Balasin1993}  
${\cal R}({\bf r})=8\pi MG\,\delta({\bf r})$, and mass density $\epsilon({\bf r})=M\,\delta({\bf r})$, one obtains the entropy density and the total entropy of the Schwarzschild black hole:
\begin{equation}
 s({\bf r}) = \frac{M}{2T_{\rm BH}}\delta({\bf r})\,\,, \,\, S_{\rm BH}=\int d^3 r\,  s({\bf r}) = \frac{A}{4G} \,,
\label{BlackGibbs}
\end{equation}
where $A$ is the area of horizon.
This demonstrates that the Bekenstein-Hawking entropy of black hole also comes from the local thermodynamics. The entropy is concentrated in the central singularity, which has the mass component ("normal component") and the gravity component ("superfluid component"). This also suggests the large information stored in the central singularity, which is the answer to the question  \#5.

\section{Scenarios of emergent gravity}
\label{gravity}

What is the origin of gravity from the point of view of condensed matter? It happens that there are at least 6 different scenarios of emergent gravity, and it is not clear which of them (if any) is preferred by Nature. 
 
 \subsection{Tetrads as bilinear forms of fermion operators}
 \label{gravity1}
 
This is the Akama-Diakonov theory, in which the gravitational tetrads $ e^a_\mu$ emerge as the order parameter of symmetry breaking phase transition, the vacuum expectation value of the bilinear form of fermionic operators:\cite{Akama1978,Diakonov2011}
\begin{equation}
 e^a_\mu=<\hat E^a_\mu>\,\,,\,\, 
 \hat E^a_\mu = \frac{1}{2}\left( \Psi^\dagger \gamma^a\partial_\mu  \Psi -  \Psi^\dagger\overleftarrow{\partial_\mu}  \gamma^a\Psi\right) \,.
\label{TetradsFermions}
\end{equation}
 This  scenario is supported by the similar  symmetry breaking scheme in superfluid $^3$He-B.\cite{Volovik1990} 
 The metric field is the bilinear combination of the tetrad fields,
$g_{\mu\nu}=\eta_{ab}e^a_\mu e^b_\nu$, 
and thus in this quantum gravity the metric is the fermionic quartet.

 \subsection{Tetrads from Weyl points}
 \label{gravity2}
 
Tetrads emerge together with Weyl fermions and gauge fields in the vicinity of the topological Weyl points in the energy spectrum of fermionic quasiparticles.\cite{Volovik2003,Horava2005}  For example, the expansion of the Hamiltonian for fermionic quasiparticles near the Weyl point at ${\bf p}={\bf p}^0$ gives 
\begin{equation}
 H \approx e^i_a \sigma^a(p_i -p_i^0) \,.
\label{TetradsWeyl}
\end{equation}
Here $\sigma^a$ are Pauli matrices, the matrix $e^i_a $ plays the role of the effective triad field, and the vector  ${\bf p}^0$ plays the role of the vector potential of the effective $U(1)$ gauge field. This scenario takes place in the Weyl superfluid $^3$He-A and now is intensively discussed in Weyl semimetals, where it is possible to simulate the black hole\cite{Volovik2016} and white hole\cite{Wilczek2020} horizons and consider analogs of gravitational anomalies.\cite{NissinenVolovik2022} The topological invariant, which supports the masslessness of fermions is expressed in terms of the matrix Green's function in the 4-dimensional momentum-frequency space:\cite{Volovik2003}
 \begin{equation}
N_3({\cal K}) = \frac{e_{\alpha\beta\mu\nu}}{24\pi^2}~
{\bf tr}\left[{\cal K}\int_\sigma   dS^\alpha
~ G\partial_{p_\beta} G^{-1}
G\partial_{p_\mu} G^{-1} G\partial_{p_\nu}  G^{-1}\right]
\label{Ksymmetry}
\end{equation}
Here the integral is over the 3D surface $\sigma$ around the singular point   in the 4-momentum space, and ${\cal K}$ is the matrix, which determines the proper discrete symmetry of the vacuum. This symmetry does not allow annihilation of the Weyl points with different chiralities.

In this scenario, the emergent space is 3-dimensional, and the spacetime is correspondingly the 4-dimensional, which follows from the topology of Weyl point. This is one of the condensed matter answers to the questions \#25 and \#29. The emergence of the gauge fields in this scenario is the possible answer to the question \#31.

Note that in this scenario the gravitational tetrads $e_a^\mu$ emerge in the form of the contravariant vectors, as distinct from the covariant tetrads $ e^a_\mu$ emerging in the first scenario in Eq.(\ref{TetradsFermions}).


 \subsection{Elasticity tetrads}
 \label{gravity3}

 Gravitational tetrads are represented by the elasticity tetrads, which describe the elasticity theory in crystals.\cite{DzyalVol1980,NissinenVolovik2019,NissinenVolovik2018} 
 In this approach an arbitrary deformed crystal structure can be described as a system of three crystallographic surfaces, Bragg planes, of constant phase $X^a(x)=2\pi n^a$, $n^a \in \mathbb{Z}$ with $a=1,2,3$. The intersection of the surfaces
\begin{equation}
X^1({\bf r},t)=2\pi n^1 \,\,, \,\,  X^2({\bf r},t)=2\pi n^2 \,\,, \,\, X^3({\bf r},t)=2\pi n^3 \,,
\label{points}
\end{equation}
represent the lattice points of a deformed crystal. In the continuum limit, the elasticity tetrads are  gradients of the phase functions:
\begin{equation}
E^{~a}_i(x)= \partial_i X^a(x)\quad i=x,y,z, \quad a=1,2,3,
\label{reciprocal}
\end{equation}
The extension to the 3+1 case produces the scenario, in which the quantum vacuum is the plastic (malleable) fermionic crystalline medium.\cite{KlinkhamerVolovik2019}  This vacuum crystal does not have the equilibrium value of lattice size, and thus all the deformations are possible. The curvature and torsion are produced by the topological defects of quantum crystals --  disclinations and dislocations correspondingly.\cite{Bilby1956,Kroner1960}

 \subsection{Rectangular tetrads}
 \label{gravity4}
 
 The metric may emerge from the non-quadratic vielbein. The dimension of spin space can be smaller or larger than the dimension of coordinate space. Example of the latter is the $4\times 5$ vielbein with dimension 5 of the spin space.\cite{Volovik2022} Such scenario may take place in the extended Akama-Diakonov gravity. In this scenario one may have continuous change of the signature of the metric. Dynamical signature has been also discussed in Ref.\cite{Zubkov2022}. 

The opposite scenario, when the dimension of spin space is smaller than the dimension of coordinate space, may  takes place for the elasticity tetrads. Examples are the lattice of linear objects -- vortices in superconductors and the systems of planes -- the smectic liquid crystal.


 \subsection{Multiple tetrads}
 \label{gravity5}

 The broken symmetry may lead to formation of different tetrads for different fermionic species.\cite{Parhizkar2022} For example, if Minkowski vacuum is degenerate with respect to discrete symmetries,\cite{Vergeles2021} one may have separate tetrads: i) tetrads for left fermions; ii) tetrads for right fermions; iii) tetrads for left untiparticles; and iv) tetrads for right antiparticles.\cite{Volovik2022a} This serves as the multi-tetrad extension of bi-metric gravity introduced by Rosen.\cite{Rosen1940}
 
 \subsection{Complex tetrads}
 \label{gravity6}

The scenario with complex tetrads takes place in the B-phase of superfluid $^3$He.\cite{Volovik1990} 
The possible realization in gravity is in Ref.\cite{Bondarenko2022}. In $^3$He-B, the collective modes of  the complex order parameter (complex tetrad) contain 14 heavy Higgs modes and one pseudo-Goldstone mode with small mass (gap).\cite{Zavjalov2016}  This suggests that in addition to the 125 GeV Higgs boson, several other Higgs bosons are waiting to be discovered in particle physics.\cite{VolovikZubkov2014,VolovikZubkov2015} 

 \subsection{Discussion}
 \label{gravity7}
 
In all six scenarios, tetrads are the primary emergent objects, while metric is the secondary object, which is the bilinear form of tetrads.  This would mean that geometry is the secondary emergent phenomenon. Whatever scenario (if any) is preferred by Nature, the gravity is not described by Einstein metric theory and requires the extended theory in terms of tetrads such as the Einstein–Cartan–Sciama–Kibble theory.\cite{Hehl2023} This is the condensed matter answer to the question  \#13.

There are the other condensed matter scenarios, in which metric emerges as primary object, such as acoustic metric in moving liquid.\cite{Unruh1981} But these analogs of gravity do not describe the interaction of gravity with fermions, and cannot serve as the guiding rule.  

In all the condensed matter scenarios, gravity emerges from the quantum vacuum degrees of freedom, which are described by the quantum mechanics and quantum field theory.
That is why, gravity is the consequence of quantum mechanics, and thus there is no contradiction between emergent gravity and quantum mechanics. This is the condensed matter answer to the question  \#3.  At the moment, condensed matter cannot say anything definite on the origin of quantum mechanics and quantum field theory (the question  \#32). There is some connection between the process of spontaneous symmetry breaking in condensed matter and the measurement process in quantum mechanics.\cite{Grady1994}  Both processes are phenomena, which take place in the limit of infinite volume $V$ of the whole system. In finite systems the quantum mechanics is reversible, and also in the finite system one cannot resolve between different phases of condensed matter systems, i.e. there is no second order phase transition. The disjoint Gibbs distributions are reached only in the limit $V\rightarrow \infty$,\cite{Sinai1983} which in quantum mechanics is the analog of the vanishing superposition of two macroscopically different states in the same limit.



Scenarios in Sections \ref{gravity1} and \ref{gravity2} suggest very different origins of quantum gravity. However, they have similar predictions for the number of fermionic degrees of freedom in quantum vacuum.
In the Diakonov theory, the grand unification with symmetry $SO(16)$ is suggested, which fits four generations of the Standard Model with 16 Weyl fermions in one generation.\cite{Diakonov2011} 
This in particular suggests that dark matter particles can be neutrino of the fourth generation,\cite{Volovik2003a} which corresponds to the scenario known as asymmetric dark matter.\cite{Petraki2013} 

The Weyl point scenario is realized, in particular, in superconductors of class $O(D_2)$.\cite {VolovikGorkov1985,Volovik2017}  In this superconductor, there are 8 Weyl points in the energy spectrum, which form cube in 3D momentum space giving rise to 8 Weyl fermions. In the 4D extension\cite{Creutz2008,Creutz2014} the Weyl nodes may form the 4D cube in the momentum-frequency space, which results in 16 Weyl fermions. All fermions become massive at low energy, when the symmetry ${\cal K}$ in Eq. (\ref{Ksymmetry}), which supports masslessness, is broken. That is why all neutrinos are massive Dirac particles, which is the condensed matter answer to question  \#10.
While the Weyl point scenario does not support the supersymmetry, the smallness of masses of observed particles follows from the the exponential suppression of the temperature of the symmetry breaking phase transitions, which is typical in condensed matter systems. This is the condensed matter answer to question  \#11.

The numbers $2^N$ also follow from the topological analysis of condensed matter systems.
If the vacuum of Standard Model is considered as topological Weyl semimetal, the maximal number of massless fermions in the symmetric phase is $16g$, where $g$ is number of generations.
\cite{VolovikZubkov2017}  The group $Z_{16}$ also appears in the classification of topological phases in 3+1 dimension.\cite{Schnyder2016} 
All this can be in favour of the Pati-Salam type extension of the Standard Model. In particular, the Standard Model with its $G(2,1,3)$ group can be extended to grand unification $G(2,2,4,4)$ group, which may include the left and right $SU(2)$ gauge fields, the $SU(4)$ color fields, and probably the $SU(4)$ family fields. 



\section{Conclusion}
\label{Conclusion}

Here we considered several answers to several fundamental questions posted in Ref.\cite{Perspective}.
These answers are based on the condensed matter experience, where we know both the many-body phenomena emerging on the macroscopic level and the microscopic (atomic) physics, which generates this emergence. It is not surprizing that the same macroscopic phenomenon may be generated by essentially different microscopic backgrounds. This points to various possible directions of studying the deep quantum vacuum.

\begin{thebibliography}{99}


 \bibitem{Perspective}
R.E. Allen  and S. Lidstr\"om,
Perspective: 
Life, the Universe, and everything -- 42
fundamental questions,
Phys. Scr. {\bf 92},  012501 (2017).

\bibitem{Volovik2020}
G.E. Volovik,
$^3$He Universe 2020,
J. Low Temp. Phys. {\bf 202}, 11--28 (2021),
%https://doi.org/10.1007/s10909-020-02538-8,
arXiv:2008.04682.

\bibitem{Weinberg1989}
S. Weinberg,
The cosmological constant problem,
 Rev. Mod. Phys. {\bf 61}, 1(1989).

\bibitem{Bjorken2011}
J. D. Bjorken,
The Dark Energy Problem: Methods and Mindsets,
afterdinner speech, March 2011.

\bibitem{AGD}
A.A. Abrikosov, L.P. Gor'kov and I.E. Dzyaloshinski,
"Methods of Quantum Field Theory in Statistical Physics",
Dover Books on Physics.

\bibitem{KlinkhamerVolovik2008a}
F.R. Klinkhamer and G.E. Volovik,
Self-tuning vacuum variable and cosmological constant,
Phys. Rev. D \textbf{77}, 085015 (2008), arXiv:0711.3170.
%%CITATION = PHRVA,D77,085015;%%

\bibitem{KlinkhamerVolovik2008b}
F.R. Klinkhamer and G.E. Volovik,
Dynamic vacuum variable and equilibrium approach in cosmology,
Phys. Rev. D \textbf{78}, 063528 (2008), arXiv:0806.2805.
%%CITATION = PHRVA,D78,063528;%%

\bibitem{VolkovKogan1974}  
A.F. Volkov and  S.M. Kogan, 
Collisionless relaxation of the energy gap in superconductors,
JETP {\bf 38}, 1018 (1974).

\bibitem{Barankov2004}
R. A. Barankov, L. S. Levitov, and B. Z. Spivak, 
Collective Rabi oscillations and solitons in a time-dependent BCS pairing problem,
Phys. Rev. Lett. {\bf 93}, 160401 (2004).

\bibitem{Yuzbashyan2005} 
A. Yuzbashyan, B. L. Altshuler, V. B. Kuznetsov, and V. Z. Enolskii,
Nonequilibrium cooper pairing in the nonadiabatic regime,
Phys. Rev. B {\bf 72}, 220503 (2005).

\bibitem{Yuzbashyan2008}  
E. A. Yuzbashyan, 
{\it Normal and anomalous solitons in the theory of dynamical Cooper pairing},
Phys. Rev. B {\bf 78}, 184507 (2008).

\bibitem{Gurarie2009}  
V. Gurarie, 
Nonequilibrium dynamics of weakly and strongly paired superconductors, 
Phys. Rev. Lett. {\bf 103}, 075301 (2009).

\bibitem{KlinkhamerVolovik2017}
F.R. Klinkhamer and G.E. Volovik,
Dark matter from dark energy in $q$-theory,
Pis'ma ZhETF {\bf 105}, 62--63  (2017), 
JETP Lett.  {\bf 105}, 74--77 (2017),
arXiv:1612.02326.


\bibitem{Volovik2003} 
G.E. Volovik, 
{\it The Universe in a Helium Droplet}, 
Clarendon Press,  Oxford (2003).

\bibitem{Volovik2023} 
G.E. Volovik, 
Analog Sommerfeld law in quantum vacuum,
Pis’ma v ZhETF {\bf 118},  (2023),
JETP Lett. {\bf 118},  (2023),
arXiv:2307.00860.

\bibitem{Maxfield2022} 
H. Maxfield and Z. Zahraee,
Holographic solar systems and hydrogen atoms: non-relativistic physics in AdS and its CFT dual,
JHEP {\bf 11} (2022) 093.

\bibitem{Zeldovich1962} 
Ya. B. Zel'dovich,
The equation of state at ultrahigh densities and its relativistic limitations,
JETP {\bf 14}, 1143 (1962).



\bibitem{Balasin1993} 
H. Balasin and H. Nachbagauer,
The energy-momentum tensor of a black hole, or what curves the Schwarzschild geometry? 
Class. Quantum Grav. {\bf 10}, 2271 (1993).

\bibitem{Akama1978}
K. Akama, 
An Attempt at Pregeometry: Gravity with Composite Metric,
Progress of Theoretical Physics, {\bf 60}, 1900--1909 (1978).

\bibitem{Diakonov2011}
D. Diakonov,
Towards lattice-regularized Quantum Gravity,
arXiv:1109.0091.

\bibitem{Volovik1990} 
G.E. Volovik, 
Superfluid $^3$He-B and gravity,
Physica B {\bf 162}, 222--230 (1990).

\bibitem{Horava2005}  
P. Ho\v{r}ava,
Stability of Fermi surfaces and $K$-theory,
Phys. Rev. Lett. \textbf{95}, 016405 (2005).

\bibitem{Volovik2016}
G.E. Volovik,
Black hole and Hawking radiation by type-II Weyl fermions,
Pis'ma ZhETF {\bf 104},  660--661 (2016),
JETP Lett.  {\bf 104},  645--648 (2016),
arXiv:1610.00521.

\bibitem{Wilczek2020}
Y. Kedem, E.J. Bergholtz and F. Wilczek,
Black and white holes at material junctions,
Physical Review Research  {\bf 2}, 043285 (2020)

\bibitem{NissinenVolovik2022}
J. Nissinen and G.E. Volovik,
Anomalous chiral transport with vorticity and torsion: Cancellation of two mixed gravitational anomaly currents in rotating chiral $p+ip$ Weyl condensates,
Phys. Rev. D {\bf 106}, 045022 (2022),
arXiv:2111.08639.

\bibitem{DzyalVol1980}
I.E. Dzyaloshinskii, and G.E. Volovick, 
Poisson brackets in  condensed matter,
Ann. Phys.  {\bf 125} 67--97 (1980).

\bibitem{NissinenVolovik2019}
J. Nissinen and G.E. Volovik,
Elasticity tetrads, mixed axial-gravitational anomalies, and (3+1)-d quantum Hall effect,
Physical Review Research {\bf 1}, 023007 (2019),
arXiv:1812.03175.

\bibitem{NissinenVolovik2018}
J. Nissinen and G.E. Volovik,
Tetrads in solids: from elasticity theory to topological quantum Hall systems and Weyl fermions,
ZhETF {\bf 154},   1051--1056 (2018),
JETP {\bf 127}, 948--957 (2018),
arXiv:1803.09234.

\bibitem{KlinkhamerVolovik2019} 
F.R. Klinkhamer and G.E. Volovik,
Tetrads and $q$-theory,
Pis'ma ZhETF  {\bf 109}, 369--370 (2019),
JETP Lett. {\bf 109},  364--367 (2019),
arXiv:1812.07046.

\bibitem{Bilby1956}
B.A. Bilby and E. Smith, 
Continuous distributions of dis- locations. III,
Proc. Roy. Sot. Sect. A {\bf 236}, 481 (1956).

\bibitem{Kroner1960}
E. Kr\"oner, 
Allgemeine Kontinuumstheorie der Versetzungen und Eigenspannungen,
Arch. Rat. Mech. Anal. {\bf 4}, 273 (1960).

\bibitem{Volovik2022}
G.E. Volovik,
From elasticity tetrads to rectangular vielbein,
Ann. Phys. {\bf 447}, 168998 (2022),
arXiv:2205.15222 [physics.class-ph].

\bibitem{Zubkov2022}
S. Bondarenko and M.A. Zubkov,
Riemann-Cartan Gravity with Dynamical Signature,
JETP Lett. {\bf 116}, 54--60 (2022).

\bibitem{Parhizkar2022}
Al. Parhizkar and V. Galitski,
Strained bilayer graphene, emergent energy scales, and Moire gravity,
Phys. Rev. Research {\bf 4}, L022027 (2022).

\bibitem{Vergeles2021}
S.N. Vergeles,
 A note on the vacuum structure of lattice Euclidean quantum gravity: birth of macroscopic space-time and PT-symmetry breaking,
 Class. Quantum Gravit. {\bf 38}, 085022 (2021)

\bibitem{Volovik2022a}
G.E. Volovik,
Combined Lorentz symmetry: lessons from superfluid $^3$He,
J. Low Temp. Phys. {\bf 206}, 1--15 (2022),
%https://doi.org/10.1007/s10909-021-02630-7,
arXiv:2011.06466.

\bibitem{Rosen1940}
N. Rosen,
General Relativity and Flat Space. I,
Phys. Rev. {\bf 57}, 147 (1940).

\bibitem{Bondarenko2022}
S. Bondarenko,
Dynamical Signature: Complex Manifolds, Gauge Fields and Non-Flat Tangent Space,
Universe {\bf 8}, 497 (2022).

\bibitem{Zavjalov2016} 
V.V. Zavjalov, S. Autti, V.B. Eltsov, P. Heikkinen, G.E. Volovik,
Light Higgs channel of the resonant decay of magnon condensate in superfluid $^3$He-B,
Nature Communications {\bf 7}, 10294 (2016),
%DOI: 10.1038/ncomms10294, www.nature.com/naturecommunications,
arXiv:1411.3983.

\bibitem{VolovikZubkov2014} 
G.E. Volovik and M.A. Zubkov,
Higgs bosons in particle physics and in condensed matter,
J. Low Temp. Phys. {\bf 175}, 486--497 (2014),
%DOI 10.1007/s10909-013-0905-7
arXiv:1305.7219.

\bibitem{VolovikZubkov2015}
G.E. Volovik and M.A. Zubkov,
Scalar excitation with Leggett frequency in $^3$He-B and the $125$ GeV Higgs particle in top quark condensation models as Pseudo - Goldstone bosons, 
Phys. Rev. D {\bf 92}, 055004 (2015),
arXiv:1410.7097.

\bibitem{Hehl2023}
Friedrich W. Hehl,
Four Lectures on Poincare Gauge Field Theory,
arXiv:2303.05366

\bibitem{Unruh1981}
W.G. Unruh, 
Experimental black hole evaporation?,
Phys. Rev. Lett. {\bf 46}, 1351--1353  (1981).

\bibitem{Volovik2003a} 
G.E. Volovik, 
Dark matter from $SU(4)$ model, 
Pisma ZhETF {\bf 78}, 1203--1206 (2003),
JETP Lett. {\bf 78},  691--694 (2003);  
hep-ph/0310006.

\bibitem{Grady1994} 
M. Grady,
Spontaneous symmetry breaking as the mechanism of quantum measurement,
hep-th/9409049.

\bibitem{Sinai1983} 
Ya.G. Sinai,
Theory of Phase Transitions, 
International series in natural philosophy (Pergamon Press, 1983).

\bibitem{Petraki2013} 
K. Petraki and R.R. Volkas,
Review of asymmetric dark matter,
Int. J. Mod. Phys. A {\bf 28}, 1330028 (2013).

\bibitem{VolovikGorkov1985}
G.E. Volovik and L.P. Gor`kov,
Superconductivity  classes in the heavy fermion systems,
ZhETFf {\bf 88}, 1412--1428 (1985), 
JETP {\bf 61}, 843--854 (1985).

\bibitem{Volovik2017} 
G.E. Volovik,
Dirac and Weyl fermions: from Gor'kov equations to Standard Model (in memory of Lev Petrovich Gor'kov),
Pis'ma ZhETF {\bf 105}, 245--246  (2017), 
JETP Lett.  {\bf 105},  273--277 (2017),
arXiv:1701.01075.

\bibitem{Creutz2008} 
M. Creutz, 
Four-dimensional graphene and chiral fermions,
J. High Energy Phys. {\bf 0804}, 017 (2008).

\bibitem{Creutz2014} 
M. Creutz, 
Emergent spin,
Ann. Phys. {\bf 342}, 21 (2014).

\bibitem{VolovikZubkov2017} 
G.E. Volovik and M.A. Zubkov,
Standard Model as the topological material,
New J. Phys. {\bf 19},  015009 (2017),
arXiv:1608.07777.

\bibitem{Schnyder2016} 
Ching-Kai Chiu, J.C.Y. Teo, A.P. Schnyder and Sh. Ryu,
Classification of topological quantum matter with symmetries,
Rev. Mod. Phys. {\bf 88}, 035005 (2016).




\end{thebibliography}
\end{document}




\bibitem{Froggatt1991}
C.D. Froggatt   and  H.B. Nielsen,
{\it Origin of Symmetry}, 
World Scientific, Singapore, 1991.
 

\bibitem{NeumannWigner1929} 
J. von Neumann J and E. Wigner,
Phys. Z. {\bf 30}, 467 (1929).

 

\bibitem{Wilczek2012}
F. Wilczek,
Introduction to quantum matter,
Phys. Scr. T{\bf 146},  014001 (2012).

\bibitem{VolovikZubkov2014}
G.E. Volovik and M.A. Zubkov,
Emergent Weyl spinors in multi-fermion systems,
Nuclear Physics B {\bf 881}, 514--538  (2014);
arXiv:1402.5700.

\bibitem{Adler1969} 
S. Adler,  
Axial-vector vertex in spinor electrodynamics,
Phys. Rev. {\bf 177}, 2426--2438 (1969).  

\bibitem{BellJackiw1969}  
J.S. Bell   and R. Jackiw,
A PCAC puzzle: $\pi_0\rightarrow\gamma\gamma$ in the $\sigma$ model,
Nuovo Cim. A {\bf 60},  47--61 (1969).

\bibitem{VilenkinLeahy1982} 
A. Vilenkin  and D.A. Leahy,
Parity non-conservation and the origin of cosmic magnetic fields, 
Astrophys. J. {\bf 254}, 77--81  (1982).

 \bibitem{NielsenNinomiya1981} 
H.B. Nielsen, M. Ninomiya: 
Absence of neutrinos on a lattice.  I - Proof by homotopy theory, 
Nucl. Phys. B \textbf{185}, 20  (1981); 
Absence of neutrinos on a lattice. II - Intuitive homotopy proof,  
Nucl. Phys. B \textbf{193}, 173 (1981). 

\bibitem{HasanKane2010}   
M.Z. Hasan and C.L. Kane, 
Topological Insulators,
Rev. Mod. Phys. {\bf 82}, 3045--3067 (2010).
% arXiv:1002.3895.

\bibitem{Xiao-LiangQi2011}   
Xiao-Liang Qi and Shou-Cheng Zhang, 
Topological insulators and superconductors,
Rev. Mod. Phys. {\bf 83}, 1057--1110 (2011).

 \bibitem{KopninHeikkilaVolovik2011} 
N.B. Kopnin, T.T. Heikkil\"a and G.E. Volovik,
High-temperature surface superconductivity in topological flat-band systems,
Phys. Rev. B {\bf 83}, 220503(R) (2011);
arXiv:1103.2033.

\bibitem{HeikkilaKopninVolovik2011} 
T.T. Heikkil\"a, N.B. Kopnin and G.E. Volovik,
Flat bands in topological media, 
JETP Lett. {\bf 94}, 233--239 (2011);
 arXiv:1012.0905.

\bibitem{VolovikMineev1982} 
G.E. Volovik, V.P. Mineev, 
Current in  superfluid Fermi liquids and the vortex core structure,
JETP {\bf 56}, 579--586 (1982).

\bibitem{VolovikVilenkin2000}  
G. E. Volovik and A. Vilenkin,
Macroscopic parity violating effects and $^3$He-A,
Phys. Rev. D {\bf 62}, 025014 (2000).

\bibitem{SonSurowka2009}  
D.T. Son and P.  Surowka,
Hydrodynamics with triangle anomalies,
Phys. Rev. Lett. {\bf 103}, 191601 (2009).

\bibitem{Polikarpov2014}
V. Braguta, M. N. Chernodub, V. A. Goy, K. Landsteiner, A. V. Molochkov, M. I. Polikarpov,
Phys. Rev. D {\bf 89}, 074510 (2014).

\bibitem{NiuThoulessWu1985} 
Qian Niu, D. J. Thouless, and Yong-Shi Wu,
Quantized Hall conductance as a topological invariant,
Phys. Rev. {\bf B~31}, 3372--3377 (1985).

\bibitem{So1985} 
H. So,
Induced topological invariants by lattice fermions in odd dimensions,
Prog. Theor. Phys. {\bf 74}, 585--593 (1985).

\bibitem{IshikawaMatsuyama1986} 
K. Ishikawa  and T. Matsuyama,
Magnetic field induced multi component QED in three-dimensions and quantum Hall effect,
Z. Phys. C {\bf 33}, 41--45 (1986). 

\bibitem{IshikawaMatsuyama1987} 
K. Ishikawa and T. Matsuyama,
A microscopic theory of the quantum Hall effect, 
Nucl. Phys. {\bf B~280}, 523--548  (1987).

\bibitem{Volovik1988}
 G.E. Volovik, 
 An analog of the quantum Hall effect in a superfluid 3He film,
JETP  {\bf 67}, 1804 (1988).
 
\bibitem{VolovikYakovenko1989}  
G.E. Volovik   and V.M. Yakovenko,  
Fractional charge, spin and statistics of solitons in superfluid $^3$He film, 
J. Phys.: Condens. Matter {\bf 1},  5263--5274 (1989).

 \bibitem{Yakovenko1989} 
V.M. Yakovenko,
Spin, statistics and charge of solitons in (2+1)-dimensional theories,   
Fizika (Zagreb) {\bf 21}, suppl. 3, 231 (1989); 
arXiv:cond-mat/9703195.

\bibitem{Golterman1993}
 M.F.L. Golterman, K.  Jansen and D.B. Kaplan,
Chern-Simons  currents and chiral  fermions on the lattice,
 Phys.Lett. B {\bf 301}, 219--223 (1993):
arXiv: hep-lat/9209003.

\bibitem{GrinevichVolovik1988} 
 P.G. Grinevich, G.E. Volovik,
 Topology of gap nodes in superfluid  3He:  $\pi_4$ homotopy group for 3He-B  disclination, 
J. Low Temp. Phys. {\bf 72}, 371  (1988).

\bibitem{EssinGurarie2011}
A.M. Essin, V. Gurarie,
Bulk-boundary correspondence of topological insulators from their Green's functions,
Phys. Rev. B {\bf 84}, 125132 (2011).

\bibitem{GurarieEssin2013} 
V. Gurarie, A.M. Essin,
Topological invariants for the fractional quantum Hall states,
 JETP Lett. {\bf 97}, 233-238 (2013);
arXiv:1301.3941.

\bibitem{Volovik1989}
 G.E. Volovik, 
 Fractional statistics and analogs of quantum Hall effect in superfluid $^3$He films, 
 AIP Conference Proceedings  {\bf 194}, 136--146 (1989).

 \bibitem{Volovik2000}
G.E. Volovik, 
Momentum-space topology of Standard Model, 
J. Low Temp. Phys.,  {\bf 119}, 241 -- 247 (2000); hep-ph/9907456.

 \bibitem{Volovik2010} 
 G.E. Volovik, 
Topological invariants  for Standard Model: from semi-metal to topological insulator,
JETP Lett. {\bf 91}, 55--61 (2010);
arXiv:0912.0502.

\bibitem{semimetal2014} 
Z. K. Liu,	 J. Jiang,	 B. Zhou,	 Z. J. Wang,	 Y. Zhang,	 H. M. Weng,	 
D. Prabhakaran,	 S-K. Mo,	 H. Peng,	 P. Dudin,	 T. Kim,	 M. Hoesch,	
 Z. Fang,	 X. Dai,	 Z. X. Shen,	 D. L. Feng,	 Z. Hussain	and  Y. L. Chen,
A stable three-dimensional topological Dirac semimetal Cd$_3$As$_2$,
Nature Materials {\bf 13}, 677--681 (2014).

\bibitem{Mermin1977} 
 N.D. Mermin,
Surface singularities and superflow in $^3$He-A', 
in: {\it Quantum Fluids and Solids}, eds.
S. B. Trickey, E. D. Adams and J. W. Dufty, Plenum, New York, pp. 3--22  (1977).

\bibitem{Volovik1978} 
G.E. Volovik, 
Topological singularities on the surface of an ordered system,
JETP Lett. {\bf 28}, 59--62 (1978).

\bibitem{Khodel1990}
V.A. Khodel  and  V.R. Shaginyan,
Superfluidity in system with fermion condensate,
JETP Lett. \textbf{51}, 553 (1990).

\bibitem{Volovik1991}
G.E. Volovik, 
A new class of normal Fermi liquids,
{\it JETP Lett.} \textbf{53}, 222 (1991).

\bibitem{KopninSalomaa1991}
N.B.  Kopnin and M.M. Salomaa, 
Mutual friction in superfluid $^3$He: Effects of bound states in the vortex core,
Phys. Rev. B {\bf 44}, 9667--9677 (1991).

\bibitem{Volovik1994}  
G.E. Volovik, 
On Fermi condensate: near the saddle point and within the vortex core, 
JETP Lett. {\bf  59}, 830--835 (1994).

\bibitem{Volovik2011}  
G.E. Volovik,
Flat band in the core of topological defects: bulk-vortex correspondence in topological superfluids with Fermi points,
 JETP Lett. {\bf 93}, 66--69 (2011);
arXiv:1011.4665.

\bibitem{Burkov2011}
A.A. Burkov and L. Balents, 
Weyl semimetal in a topological insulator multilayer,
Phys. Rev. Lett. {\bf 107}, 127205 (2011);
A.A. Burkov, M.D. Hook, L. Balents,
Topological nodal semimetals,
Phys. Rev. B {\bf 84}, 235126 (2011).

\bibitem{VolkovPankratov1985}  
B.A. Volkov and O.A. Pankratov,
Two-dimensional massless electrons in an inverted contact,
JETP Lett. {\bf 42}, 178--181 (1985).

\bibitem{SalomaaVolovik1988}
 M.M. Salomaa and  G.E. Volovik, 
 Cosmiclike domain walls in superfluid $^3$He-B: Instantons and diabolical points in (${\bf k}$,${\bf r}$) space, Phys. Rev.  {\bf B~37}, 9298--9311 (1988).

\bibitem{Schnyder2008} 
A.P. Schnyder, S. Ryu, A. Furusaki and A.W.W. Ludwig, 
Classification of topological insulators and superconductors in three spatial dimensions,
Phys. Rev. {\bf B~ 78}, 195125 (2008); A.P. Schnyder, S. Ryu, A. Furusaki and A.W.W. Ludwig, 
Classification of topological insulators and superconductors,
 AIP Conf. Proc. {\bf 1134}, 10 (2009);    
 arXiv:0905.2029.

\bibitem{Kitaev2009} 
A. Kitaev,
Periodic table for topological insulators and superconductors,
AIP Conference Proceedings, Volume {\bf 1134}, pp. 22--30 (2009);
  arXiv:0901.2686.
  
 \bibitem{Mizushima2014}
 T. Mizushima, Y. Tsutsumi, M. Sato, K. Machida,
Symmetry protected topological superfluid $^3$He-B,
 arXiv:1409.6094.

\bibitem{Haldane1988} 
 F.D.M. Haldane,
 Model for a quantum Hall effect without Landau levels: Condensed-matter realization of the "Parity Anomaly",
Phys. Rev. Lett. {\bf 61}, 2015--2018 (1988).

\bibitem{Mackenzie2003}
A.P. Mackenzie and Y. Maeno,
The superconductivity of Sr$_2$RuO$_4$ and the physics of spin-triplet pairing,
 Rev. Mod. Phys. {\bf 75}, 657--712 (2003).

\bibitem{Volovik1990} 
 G.E. Volovik,
Half quantum  vortices in the B phase of superfluid $^3$He,
JETP Lett. {\bf 52 }, 358--363 (1990).

\bibitem{KlinkhamerVolovik2005} 
F.R. Klinkhamer and G.E. Volovik, 
Emergent CPT violation from the splitting of Fermi points, 
Int. J. Mod. Phys. A {\bf 20}, 2795--2812 (2005); 
hep-th/0403037.

 \bibitem{HeikkilaVolovik2011} 
T.T. Heikkil\"a and G.E. Volovik,
Dimensional crossover in topological matter: Evolution of the multiple Dirac point in the layered system to the flat band on the surface,
Pis'ma ZhETF {\bf 93}, 63--68 (2011); JETP Lett. {\bf 93}, 59--65 (2011);
arXiv:1011.4185.


\bibitem{Yudin2014}
D. Yudin, D. Hirschmeier, H. Hafermann, O. Eriksson, A.I. Lichtenstein  and M.I. Katsnelson,
Fermi condensation near van Hove singularities within the Hubbard model on the triangular lattice,
Phys. Rev. Lett. {\bf 112}, 070403 (2014).

\bibitem{Dolgopolov2014}
A.A. Shashkin, V.T. Dolgopolov, J.W. Clark, V.R. Shaginyan, M.V. Zverev and V.A. Khodel,
Merging of Landau levels in a strongly-interacting two-dimensional electron system in silicon,
 Phys. Rev. Lett. {\bf 112}, 186402 (2014).



\bibitem{Volovik1999}
G.E. Volovik, 
Fermion zero modes on vortices in  chiral superconductors,
JETP Lett. {\bf 70}, 609--614 (1999); cond-mat/9909426.


\bibitem{ReadGreen2000}
N. Read and D. Green,
Paired states of fermions in two dimensions with breaking of parity and time-reversal symmetries and the fractional quantum Hall effect,
Phys. Rev. B {\bf 61}, 10 267--10 297 (2000).


\bibitem{Ryu2002}
S. Ryu and  Y. Hatsugai, 
Topological origin of zero-energy edge states in particle-hole symmetric systems,
 Phys. Rev. Lett. {\bf 89}, 077002 (2002).



\bibitem{SchnyderRyu2011}
 A.P.  Schnyder and S. Ryu, 
Topological phases and flat surface bands in superconductors without inversion symmetry,
Phys. Rev. B {\bf 84}, 060504(R) (2011).

\bibitem{EsquinaziHeikkila2014}
P. Esquinazi, T.T.  Heikkil\"a, Y.V. Lysogorskiy, D.A. Tayurskii, G.E. Volovik,
On the superconductivity of graphite interfaces,
Pis'ma ZhETF {\bf 100},  374--378 (2014); 
  arXiv:1407.2060.

\bibitem{TangFu2014}
E. Tang and L. Fu, 
Strain-induced partially flat band, helical snake states, and interface superconductivity in topological crystalline insulators,
Nature Physics, (2014);
arXiv:1403.7523 (2014).

\bibitem{Ballestar HeikkilaEsquinazi2014}
A. Ballestar, T.T. Heikkil\"a, P. Esquinazi,
Interface size dependence of the Josephson critical behaviour in pyrolytic graphite,
 arXiv:1401.4959.



\bibitem{Volovik1987}
G.E. Volovik,  
Zeroes in the fermionic spectrum in superfluid systems as diabolical points,
JETP Letters 46, 98 (1987).

\bibitem{VollhardtWolfle1990}
D. Vollhardt  and P.  W\"olfle,
{\it The superfluid phases of helium 3},  Taylor and Francis, London
(1990).


