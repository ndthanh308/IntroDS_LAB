\documentclass[twocolumn,aps,prl,superscriptaddress,notitlepage,longbibliography]{revtex4}
%\documentclass[aps,preprint,showpacs,superscriptaddress,groupedaddress]{revtex4}  % for double-spaced preprint
\usepackage[colorlinks=true,urlcolor=blue,citecolor=blue,linkcolor=blue]{hyperref}
\usepackage{graphicx}  % needed for figures
\usepackage{dcolumn}   % needed for some tables
\usepackage{bm}        % for math
\usepackage{CJKutf8}
\usepackage{amssymb}
\usepackage{amsfonts}
\usepackage{doi}
\usepackage{verbatim}
\usepackage{epstopdf}
\usepackage{lineno}
\usepackage{amsmath}
\usepackage{braket}
\usepackage{hyperref}
\DeclareMathOperator{\Tr}{Tr}
%\usepackage{mcite}
% avoids incorrect hyphenation, added Nov/08 by SSR
\hyphenation{ALPGEN}
\hyphenation{EVTGEN}
\hyphenation{PYTHIA}
\newcommand{\C}[1]{{\textcolor{red}{#1}}}
\newcommand{\B}[1]{{\textcolor{blue}{#1}}}
\begin{document}
\title{Dissipative Chiral Channels, Ohmic Scaling and Half-integer Hall Conductivity from the Relativistic Quantum Hall Effect}
\author{Humian Zhou}
\affiliation{International Center for Quantum Materials, School of Physics, Peking University, Beijing 100871, China}
%\author{Hailong Li}
%\affiliation{International Center for Quantum Materials, School of Physics, Peking University, Beijing 100871, China}
%\author{Dong-Hui Xu}
%\affiliation{Department of Physics, and Chongqing Key Laboratory for Strongly Coupled %Physics, Chongqing University, Chongqing 400044, China}
%\affiliation{Center of Quantum Materials and Devices, Chongqing University, Chongqing 400044, %China}
\author{Chui-Zhen Chen}
\email{czchen@suda.edu.cn}
\affiliation{School of Physical Science and Technology, Soochow University, Suzhou 215006, China}
\affiliation{Institute for Advanced Study, Soochow University, Suzhou 215006, China}
\author{Qing-Feng Sun}
\affiliation{International Center for Quantum Materials, School of Physics, Peking University, Beijing 100871, China}
\affiliation{CAS Center for Excellence in Topological Quantum Computation, University of Chinese Academy of Sciences, Beijing 100190, China}
\affiliation{Hefei National Laboratory, Hefei 230088, China}
\author{X. C. Xie}
\email{xcxie@pku.edu.cn}
\affiliation{International Center for Quantum Materials, School of Physics, Peking University, Beijing 100871, China}
\affiliation{Institute for Nanoelectronic Devices and Quantum Computing, Fudan University, Shanghai 200433, China}
\affiliation{Hefei National Laboratory, Hefei 230088, China}
\begin{abstract}
%A three-dimensional topological insulator (3D TI) hosts a 2D surface state composed of a single massless Dirac cone. %One of the most distinct quantum transport phenomenon 
The quantum Hall effect (QHE), which was observed in 2D electron gas under an external magnetic field, stands out as one of the most remarkable transport phenomena in condensed matter. However, a long standing puzzle remains regarding the observation of the relativistic quantum Hall effect (RQHE). This effect, predicted for a single 2D Dirac cone  immersed in  a magnetic field, is distinguished by the intriguing feature of  half-integer Hall conductivity (HIHC). 
In this work, we demonstrate that the condensed-matter realization of the RQHE and the direct measurement of the HIHC are feasible by investigating the underlying quantum transport mechanism.
%To this end, we systematically study the microscopic transport mechanism of HIHC for the RQHE. 
We reveal that the manifestation of HIHC is tied to the presence of dissipative half-integer quantized chiral channels circulating along the interface of the RQHE system and a Dirac metal.
%In this work, we systematically study the microscopic transport mechanism of HIHC for the RQHE, which extends beyond the realm of dissipationless quantized transport in the conventional QHE.
%In particular, we reveal that the manifestation of HIHC is tied to the presence of dissipassive half-integer chiral channels circulating along the interface of the RQHE system and a Dirac metal.
Importantly, we find that the Ohmic scaling of the longitudinal conductance of the system plays a key role in  directly measuring the HIHC in experiments.
Furthermore, we propose a feasible experimental scheme based on the 3D topological insulators to  directly measure the HIHC.
Our findings not only uncover the distinct transport mechanism of the HIHC for the RQHE, but also paves the way to the measurement of the HIHC in future experiments.



\end{abstract}
%\pacs{}

\maketitle

{\emph{Introduction.}}---
The seminal discovery of the quantum Hall effect represents a pivotal milestone in topological states of matter \cite{Klitzing1980,TKNN1982,Hasan2010,Qi2011}. 
The integer quantum Hall effect (IQHE) has been observed in conventional two-dimensional (2D) electron systems subjected to an external magnetic field, where the Hall conductivity takes integer quantized values in units of  $e^2/h$. %  $\sigma_{xy}=\nu e^2/h$ with $\nu$ an integer.
Interestingly, the relativistic quantum Hall effect (RQHE) predicted for a single 2D Dirac fermion possesses half-integer Hall conductivity
(HIHC) $\sigma_{xy}=(\nu +{1}/{2}) e^2/h$ with $\nu$ an integer \cite{Abouelsaood1985, RQHE}.
In retrospect, the initial proposal for realizing RQHE is based on the materials with 2D Dirac-like electronic excitations, such as graphene and 3D topological insulator (TI) surfaces \cite{RQHE,Zheng2002,Gusynin2005,Fu2007PRB,Qi2008,Lee2009}.
However, despite the experimental discoveries of graphene \cite{Novoselov2004} and 3D TIs \cite{Hsieh2009,Roushan2009,YLChen2009}, only an integer Hall conductance has been observed in experiments because the Dirac fermions appear in pairs in these systems \cite{Novoselov2005, ZhangNat2005, Castro2009,Molenkamp2011, Xu2014, Yoshimi2015, Xu2016}. 
Therefore, the direct observation of HIHC remains elusive, and the condensed-matter realization of RQHE is still a significant challenge.
%Recently, the experimental 
%The integer quantum Hall effect is believed to be understand via TKNN formula. The energy %spectrum of QHE forms discrete Landau levels with one-dimensional dissipationless chiral edge %modes on its boundary, giving rise to the quantized Hall conductance
%These Landau levels become deformed on the sample edge creating one-dimensional dissipationless chiral edge modes  and crossing the Fermi energy, which gives rise to the quantization of the Hall conductance.
%In retrospect, the RQHE was  predicted in a two-dimensional (2D) Dirac fermion in a %perpendicular magnetic field about four decades ago as a distinctive feature of parity anomaly %in (2+1) dimension QED. 

%Nevertheless, there is a longstanding issue that the experimental observation of the %relativistic quantum Hall effect (RQHE) for a single 2D gapless  Dirac cone  %\cite{Abouelsaood1985, RQHE}.
%Representative examples include the topological magnetoelectric effect in axion insulators, the %parity anomaly in semi-magnetic topological insulators (TIs), and the realization of Majorana %fermions in topological superconductors \cite{Qi2008, Mogi2021, Zhou2022, Fu2008}.
%Another prominent relativistic quantum phenomenon is the relativistic quantum Hall effect (RQHE) %predicted in a two-dimensional (2D) Dirac fermion system in a perpendicular magnetic field, %where the Hall conductance becomes half-integer quantized \cite{Abouelsaood1985, RQHE}.
%and  the "double" RQHE appears

Theoretically, it is widely recognized that the quantization of Hall conductivity for IQHE is intimately linked to the Chern number of the Landau levels \cite{TKNN1982}. % as elucidated by the Thouless–Kohmoto–Nightingale–den Nijs (TKNN) theory. 
According to bulk-edge correspondence, Landau levels  undergo deformation along the sample edge, resulting dissipationless chiral edge modes. These dissipationless edge modes, in turn, give rise to the observation of integer Hall conductivity in experiments.
On the other hand, tremendous efforts have also been exerted to uncover the origin of the HIHC in the RQHE.  The HIHC was obtained either by calculating the bulk electromagnetic response of a single 2D Dirac cone or by just dividing integer quantized Hall conductivity of multiple 2D Dirac cones by the number of degenerate Dirac cones \cite{Abouelsaood1985, RQHE,Nomura2008,Qi2008,Sitte2012,Zheng2002,Gusynin2005,Lee2009,Watanabe2010,Zhang_2012}.
However, the microscope edge transport theory on the RQHE remains unknown since the number of chiral edge states cannot be half-integer \cite{Shen2011,Gong2022}.
On a more fundamental level, the transport mechanism of HIHC may extend beyond the realm of dissipationless edge transport for IQHE. %and the bulk-edge correspondence of the RQHE is unclear.
So far, there are  at least three important issues not answered.
 (i)  Is it really impossible to directly measure the HIHC $\sigma_{xy}=(\nu+1/2)e^2/h$, $\nu\in \mathbb{Z}$  experimentally?
 (ii) Does the chiral edge trajectory exists in the RQHE?
 (iii) What is the transport mechanism of the RQHE?

% Figure environment removed
%uncover the transport mechanism of the HIHC.
In this Letter, we find that the HIHC can be directly measured in experimental setups through investigation of the transport properties of the RQHE.
Inspired by recent experimental observation of the half-quantized anomalous Hall conductance in a hybrid system of massive and massless Dirac cones \cite{Mogi2021}, we examine  a 2D massless Dirac cone under a nonuniform magnetic field.
This setup results in a similar hybrid system, consisting of a 2D massless Dirac metal and the RQHE system, as depicted by the yellow and blue regions in Fig.~\ref{fig1}(a).
We first reveal half-integer quantized chiral channels (CCs) located at the interface [see the red line in Fig.~\ref{fig1}(a)]. 
Then, we establish an analytic relation between the Hall conductivity and the dissipative CCs, and find that the experimental measurement of the HIHC is determined by the Ohmic scaling of the longitudinal conductance.
This relation uncovers bulk-edge correspondence of the RQHE system by elucidating how the half-integer quantized CCs on the edge results in the HIHC within the bulk.
Significantly, to achieve the HIHC in realistic systems, we employ Landauer-B{\"u}ttiker formalism to determine half-integer quantized CCs and HIHC on the surface of the 3D TI, allowing for the direct comparison with experimental observations.
Ultimately, we propose a feasible experimental scheme based on the 3D TI to directly measure the HIHC by a six-terminal device.

{\em Model Hamiltonian and half-integer quantized CCs.}---We begin with the 2D massless Dirac Hamiltonian in a nonuniform magnetic field [see Fig.~\ref{fig1}(a)] that reads:
%\begin{eqnarray}\label{eq1}
%	% \nonumber to remove numbering (before each equation)
%	H= v_F(-i\hbar {\bm \nabla}+e{\bf A})\cdot{\bm \sigma} 
%\end{eqnarray}
	\begin{equation}\label{eq1}
		H=\left\{
	\begin{array}{ll}
		-i\hbar v_F  {\bm \nabla}\cdot{\bm \sigma} &(-L_y<y< 0)\\
		v_F(-i\hbar  {\bm \nabla}+e{\bf A})\cdot{\bm \sigma} &(y\geq 0)
	\end{array}\right.
\end{equation}
where $v_F$ denotes the Fermi velocity and the Pauli matrices ${\bm \sigma}$ act in spin space.  For $-L_y< y <0$, the energy $E_{\bf k}=\pm\hbar v_F k$, with the wave vector operator  $(k_x,k_y)=(k\cos \theta,k\sin \theta)$.
For $y \geq 0$, the external magnetic field  $ B_0$ is applied along the $+z$ axis, and the vector potential is taken in the gauge ${\bf A}=-yB_0\hat{e}_x$. Then, the Landau levels forms with energy $\epsilon_n={\rm sgn}(n)\epsilon_D \sqrt{2  \left | n \right |}$ with $n \in \mathbb{Z}$ [see Fig.~\ref{eq1}(a)]. The energy and length scales are set to be $\epsilon_D=\hbar v_F/l_B$ and $l_B=\sqrt{\hbar/(eB_0)}$.
 
%The magnetic field is ${\bf B}=B(x)\hat{e}_z$ where $B(x)=0$ within the strip $-L_x\leq x\leq0$ and $B(x)=B_0$ elsewhere. 


%Here, ${\bf k}=(k_x,k_y)=(k\cos \theta,k\sin \theta)$ is the wave vector, and $\theta$ is the kinematic incidence angle.  
%The LLs have energy $\epsilon_n={\rm sgn}(n)\epsilon_D \sqrt{2  \left | n \right |}$, where integer $n$ is the LL index and ${\rm sgn}(n)$ is the sign of $n$ (positive for electrons and negative for holes).
%$\epsilon_D=\hbar v_F/l_B$ is an energy scale with the magnetic length $l_B=\sqrt{\hbar/(eB_0)}$.

%In order to calculate the equilibrium current, 
To obtain the CCs, we first derive eigenstates by solving the Dirac equation $H\psi_{\bf k}({\bf x})=E_{\bf k}\psi_{\bf k}({\bf x})$. For $-L_y < y < 0$, the eigenstate $\psi_{\bf k}({\bf x})$ reads:
\begin{eqnarray}\label{eq3}
	% \nonumber to remove numbering (before each equation)
	\psi_{\rm I}({\bf x})=
	\frac{e^{ik_x x}}{\sqrt{4 L_x L_y}}
	\left[
	\begin{pmatrix} 
		e^{-i\frac{\theta}{2}}\\
		e^{i\frac{\theta}{2}} 
	\end{pmatrix} e^{ik_y y}
	+r
	\begin{pmatrix} 
		e^{i\frac{\theta}{2}}\\
		e^{-i\frac{\theta}{2}} 
	\end{pmatrix} e^{-ik_y y}
	\right]
\end{eqnarray}
with a $\theta$-dependent reflection amplitude $r$.
%$k_y=2\pi m/L_y $  with integer $m$ since we apply the periodic boundary condition in the $y$ direction. 
For the magnetic field region ($y \geq 0$), 
%In the region $x>0$, the Dirac electrons in the magnetic field ${\bf B}=B_0\hat{e}_z$ will form Landau levels.  The landau energy $\epsilon_n={\rm sgn}(n)\epsilon_D \sqrt{2  \left | n \right |}$ with integer $n$ and ${\rm sgn}(n)$ is the sign of $n$. $\epsilon_D=\hbar v_F/l_B$ is an energy scale with the magnetic length $l_B=\sqrt{\hbar/(eB_0)}$.
\begin{eqnarray}\label{eq4}
	% \nonumber to remove numbering (before each equation)
	\psi_{\rm II}({\bf x})=
	\frac{t e^{ik_x x}}{\sqrt{4 L_x L_y}}
	\begin{pmatrix} 
		D_{(k l_B)^2/2-1}[\sqrt{2}(y/l_B-k_x l_B)]\\
		-\frac{\sqrt{2}}{k l_B} D_{(k l_B)^2/2}[\sqrt{2}(y/l_B-k_x l_B)] 
	\end{pmatrix}	
\end{eqnarray}
%with a coefficient $t$. 
$D_{\nu}(x)$ is the parabolic cylinder function \cite{Gradshteyn1980}. 
%In this text, we only consider eigenstates with eigenvalues located within the gap between LLs, and therefore 
$\psi_{\rm II}({\bf x})$ is a decaying wave as $(k l_B)^2/2 \notin \mathbb{Z}$.
%This implies that plane waves cannot transmit into the magnetic field region.
%When the eigenvalue is located at the gap between Landau levels, the plane wave cannot transmit into the magnetic field region. Then $\psi_{\bf k}(\bf x)$ in this region is a decacent wave.
%Similarly, for $x<-L_x$, the decacent wave is
By enforcing the continuity of $\psi_{\bf k}({\bf x})$ at $y=0$, we obtain 
%t =	\frac{\sqrt{2} \cos \theta e^{-i\theta/2}}{D_1+D_2e^{i(\theta+\pi/2)}},
\begin{eqnarray}\label{eq6}
	% \nonumber to remove numbering (before each equation)
	r(k,\theta)
	\equiv e^{i\phi_r(k,\theta)} 
	=
	-\frac{D_2+D_1e^{i\theta}}{D_1 +D_2e^{i\theta}}
\end{eqnarray}
and $t ={-2 i \sin \theta e^{i\theta/2}}/[D_1+D_2e^{i\theta}]$.
Here, we denote $D_1=D_{(k l_B)^2/2-1}(-\sqrt{2} k_x l_B)$ and $D_2=\sqrt{2}/(k l_B) D_{(k l_B)^2/2}(-\sqrt{2} k_x l_B)$.  $\phi_r(k,\theta)$ is the reflection phase, representing the phase shift between the reflected wave and the incident wave. 
%$\psi_{\bf k}({\bf x})$ is normalized and $(k_x,k_y)=(\pi n/L_x,2\pi m/L_y)$ with $n\in %\mathbb{N}$ and $m\in \mathbb{Z}$ as $L_x$ approaches to infinity \cite{SM}. 

Next, we calculate the number of the CCs that directly determines the Hall transport. 
The transverse equilibrium current for occupied states is given by $J_c(E_F)=\lim_{d,L_y\rightarrow\infty}\sum_{E_{\bf k}\leq E_F}\int_{-d}^{+\infty}j_{x,{\bf k}}({\bf x})\mathrm{d}y$ with $d<L_y$, where $j_{x,{\bf k}}({\bf x})=-ev_F \psi_{\bf k}^{\dagger}({\bf x}) \sigma_x \psi_{\bf k}({\bf x})$ is the transverse current density for $\psi_{\bf k}({\bf x})$. 
%$J_y(E_F,d)$ is nonzero and exhibits a chiral nature near the edge of the metallic region due to the breaking of time-reversal symmetry in the presence of a magnetic field. 
By taking the partial derivative of $J_c(E_F)$ with $E_F=\hbar v_F k_F$, we obtain
\begin{eqnarray}\label{eq7}
	% \nonumber to remove numbering (before each equation)
	\frac{h}{e} \frac{\partial J_c(E_F,d)}{\partial E_F}
	=\int_{0}^{\pi} \frac{\mathrm{d} \theta}{2 \pi} \left (-\frac{\partial \phi_r(k_F,\theta)}{\partial \theta} \right )% \nonumber
	%\\
%	+
%	\int_{-\pi/2}^{\pi/2} \frac{\mathrm{d} \theta}{2 \pi} \left (\frac{\sin [ %\phi_r(k_F,\theta)+2k_F d \cos \theta]}{\cos \theta} \right ) 
\end{eqnarray}
which is the total number of CCs. % in the region $x>-d$.
The relationship $r(k_F,0)r(k_F,\pi)=-1$ derived from Eq.~(\ref{eq6}) leads to $\phi_r(k_F ,0)-\phi_r(k_F , \pi)=2\pi(n+1/2)$ with $n\in \mathbb{Z}$. %\cite{SM}. 
This ensures the half-integer quantization of CCs [see Fig.~\ref{fig1}(b)] and the final result is given by:
\begin{eqnarray}\label{eq71}
	\frac{h}{e}\frac{\partial J_c(E_F)}{\partial E_F}=  \lfloor ( {E_F}/{\epsilon_D})^2/2 \rfloor+\frac{1}{2} 
\end{eqnarray}
where $\lfloor c \rfloor$ denotes the largest integer less than $c$.  
Note that the equilibrium current and CCs are predominantly located at the interface between the metallic region and the magnetic field region \cite{SM}. This half-integer quantized CCs manifests the bulk-edge correspondence of the RQHE system.
%This indicates there are half-integer quantized CCs coexisting with the metallic phase and  circulating around the edge of the magnetic field region, which manifests the bulk-edge correspondence of the RQHE system.
%The second term of Eq.~(\ref{eq7}) decays to zero in a power law $(k_Fd)^{-1/2}$ \cite{SM}. %This indicates that 
%where $k_F=E_F/(\hbar v_F)$.  
%$(h/e){\partial J_y(E_F,d)}/{\partial E_F}$ is the number of chiral channels in the region $x>-d$, which directly determines the Hall transport. 
%The integer $n$ is given by $n=[\nu]= [(k_F l_B)^2/2]=[(E_F/ \epsilon_D)^2/2]$ 
%The second term of Eq.~(\ref{eq7}) decays to zero in a power law $(d/l_B)^{-1/2}$ [see more details in Ref.\cite{SM}]. This indicates that the total number of chiral channels is solely determined by ${\partial J_c(E_F)}/{\partial E_F}$ and the chiral channels are predominantly located at the boundary between the metallic region ($x<0$) and the magnetic field region ($x>0$).

%The half-integer quantized $\partial J_c(E_F)/\partial E_F$ or half-integer quantized chiral channels is the origin of the HIHC.

{\em Ohmic Scaling and HIHC.}---
Now, let's establish an analytic relation between the Hall conductivity and the CCs,
and  demonstrate how  Ohmic scaling of conductance determines the existence of the HIHC. 
We consider a metallic system of size $L_x\times L_y$ (yellow region) with half-integer quantized CCs circulating around its edge, as showed in Fig.~\ref{fig1}(c).
To observe a stable HIHC in experiments,  it is essential for both the Hall resistivity and longitudinal resistivity to be independent of system's size.
For this purpose, we assume an Ohmic scaling of the conductance $G_{\text{2D}}$ in the metallic system, i.e. $G_{\text{2D}}=\sigma_nL_y/L_x$, ensuring that the conductivity $\sigma_n$ is unaffected by the system's size.
Furthermore, the dissipation also emerges in the CCs that coexist with the metallic system \cite{SM}.
To incorporate this contribution, we introduce the edge conductivity $\sigma_{\text{1D}}$ that arises from the dissipative CCs, and the edge conductance satisfies 1D Ohmic scaling law, i.e. $G_{\text{1D}}=\sigma_{\text{1D}}/L_x$.
%Note that the conductance of the metallic system satisfies the Ohmic scaling when the system %size is much larger than the dephasing length.
%In this scenario, we can consider this metallic system as composed of numerous independent %segments, each with conductance $G=\sigma_n$. %size $r_c$ and
%These segments can be viewed as independent classical resistors connected together in %accordance with Ohm's law.
%To incorporate the contribution from the CCs, the revised conductance $\tilde{G}=G+\delta G$, where $\delta G$  arises from the CCs.
%that also contributes to the longitudinal conductance in addition to the Hall conductance. 
%The massless Dirac metal is classical when its size is much larger than the dephasing length $l_{\phi}$, and then can be viewed as made up of a large number of independent boxes of size $l_{\phi}$.
%Since quantum interference effects are negligible beyond this size, we can view these boxes as independent classical resistors connected together according to Ohm's law.
%The conductance of each box in the yellow and gray region is $\sigma_0$ and $\sigma_0+\delta$, respectively. 
%Here, the extra conductance $\delta$ comes from the CCs that also contributes to the longitudinal conductance in addition to the Hall conductance.

To investigate the transport properties of Hall-bar device in Fig.~\ref{fig1}(c),
a small external bias between leads 1 and 4 is applied to generate a uniform electric field ${\bf E}=(E_x,-E_y)$ \cite{SM}. 
Then, the total current flowing from the lead 1 to 4 is given by $I_x = \sigma_n E_x L_y +\sigma_{\text{1D}}E_x+ e {\partial J_c(E_F)}/{\partial E_F} E_y L_y$. Here the first two terms are the longitudinal current, and the third term is the Hall current from the CCs.
Considering current conservation near the edge $E_y \sigma_n=E_x e {\partial J_c(E_F)}/{\partial E_F}$, the longitudianal and Hall resistivity can be obtained by $\rho_{xx}=E_x/j_x$ and  $\rho_{xy}=E_y/j_x$, with the current density $j_x=I_x/L_y$.
 %as a consequence of the uniform conductivity in the metallic region 
% validity of this uniform electric field is confirmed in .
In the Hall-bar measurement, $E_x =( V_2-V_3)/L$ and $E_y=(V_2-V_6)/ L_y$, with $V_i$ the voltage of lead $i$.  $L$ ($L_y$) is the length between the lead 2 and 3 (the lead 2 and 6).
Finally, by using the tensor relation that $\sigma_{xx}=\rho_{xx}/(\rho_{xx}^2+\rho_{xy}^2)$ and $\sigma_{xy}=\rho_{xy}/(\rho_{xx}^2+\rho_{xy}^2)$,
the longitudinal and Hall conductivity are obtained [see more details in Ref.\cite{SM}]:
\begin{eqnarray}\label{eq8}
	\sigma_{xy}(E_F)&=&\left[1+\frac{1}{1+(\sigma_c/\sigma_n)^2}\frac{\sigma_{\text{1D}}}{ \sigma_n L_y }\right] \sigma_c \nonumber \\
	\sigma_{xx}(E_F)&=& \left[1+\frac{1}{1+(\sigma_c/\sigma_n)^2}\frac{\sigma_{\text{1D}}}{\sigma_n L_y}\right]\sigma_n
\end{eqnarray}
where $\sigma_c \equiv e\partial J_c(E_F) / \partial E_F$.
%$\sigma_{xy}$ are not quantized in Eq.~\ref{eq8}, because the whole system breaks the 2D Ohmic scaling law.
%Although we assume that the longitudinal conductance of the 2D bulk and 1D edge satisfies the 2D and 1D Ohmic scaling, respectively.
Here, although $G_{\text{2D}}=\sigma_n L_y/L_x$ of the metal is size-independent (when $L_y/L_x$ is fixed),
the longitudinal edge conductance $G_{\text{1D}}=\sigma_{\text{1D}}/L_x$ arising from the CCs is size-dependent.
Therefore, the whole system breaks the 2D Ohmic scaling law, which is the reason why $\sigma_{xy}$ are not quantized in Eq.~(\ref{eq8}).
%breaks the 2D Ohmic scaling law, i.e. the conductivity $\delta G/L_y$ is size-dependent. 
In Fig.~\ref{fig1}(d), we plot the $\sigma_{xy}$ as a function of the Fermi energy $E_F/\epsilon_D$,  and  assume that $\sigma_n=\sigma_0|E_F|/\epsilon_D$ with $\sigma_0=e^2/h$ since the density of state of 2D massless Dirac cone is proposal to $|E_F|$.
Notably, in the limit that $G_{\text{1D}} \ll G_{\text{2D}}$ (or $\sigma_{\text{1D}} \ll \sigma_n L_y$ ), the Hall conductance in Eq.~(\ref{eq8}) are half-integer quantized to be $\sigma_{xy}=e\partial J_c(E_F) / \partial E_F$ [see also Fig.~\ref{fig1}(d)].
As a result, we conclude that the HIHC is tied to the half-integer quantized CCs
and the Ohmic scaling of the conductance is crucial to the direct observation of the HIHC 
in experiments. %This is main results of our work.
%Note that the expression of $I_x$ is incorrect when the conductance of metallic system does %not satisfy the Ohmic scaling, which will render  Eq.~(\ref{eq8}) invalid. 
%Thus, the Ohmic scaling is crucial to Eq.~(\ref{eq8}).


%Figure.~\ref{fig1}(d) shows
%The density of state of the massless Dirac fermion is proposal to $|E_F|$, so the %condutivity of the metallic system $\sigma_n=\sigma_0|E_F|/\epsilon_D$.  
%Here, we assume $\sigma_n=\sigma_0|E_F|/\epsilon_D$ since the density of state of the %metallic system composed of massless Dirac fermions is proposal to $|E_F|$.

% Figure environment removed  

{\em HIHC in the 3D TI.}--To illustrate the feasibility of direct experimental measurement of the HIHC, we employ the Landauer-B{\"u}ttiker formula to numerically calculate the Hall conductivity. 
We consider a 3D TI that hosts a 2D gapless surface Dirac cone \cite{Fu2007PRL,Fu2007PRB} and the effective  four-band TI Hamiltonian is given by \cite{Zhang2009} 
\begin{eqnarray}%\label{eq1}
	% \nonumber to remove numbering (before each equation)
	H_{\text{3D}}({\bf k})&=& \sum_{i=x,y,z}Ak_i\sigma_x{\otimes}s_i+(M_0-Bk^2)\sigma_z{\otimes}s_0 \nonumber
\end{eqnarray}
with model parameters $A$, $B$, and $M_0$. $\sigma_i$ and $s_i$ are Pauli matrices for the orbital and spin degrees of freedom, respectively. 
The perpendicular magnetic field is applied in the blue region with ${\bf B}=B_0\hat{e}_z$ (${\bf B}=0$ in the yellow region) [see Fig.~\ref{fig2}(a)].
We discretize the Hamiltonian $H_{\text{3D}}$ into $(L_x+2L_0) \times (L_y+2L_0) \times L_z$ cubic lattice sites.
To obtain the Ohmic scaling of conductance,
we introduce $n_x \times n_y$ B{\"u}ttiker's virtual leads on the yellow region to simulate the dephasing process \cite{SM,Buttiker1986,Buttiker1988,Xing2008,Zhou2022}. This allows the conductance  satisfies the Ohmic scaling when the system size is much larger than the dephasing length. In reality, dephasing process can be caused by electron-electron or electron-phonon interactions \cite{Chakravarty1986,A1985,Stern1990}.
%A 3D TI, hosting a 2D gapless surface Dirac cone \cite{Fu2007PRL,Fu2007PRB}, is a perfect platform for  realizing the HIHC. 
%The surface state of the 3D TI is  described by $H_{\text{2D}}=A(k_x\sigma_x+k_y\sigma_y)$.
According to Landauer-B{\"u}ttiker formula, the current flowing out from the lead $p$ is
$J_{p}=\frac{e^2}{h}\sum_{q\ne p}\left(T_{qp}V_{p}-T_{pq}V_{q}\right)$
where $V_p$ is the voltage in the lead $p$.  $T_{pq}(E_F)=\mbox{Tr}[{\bm \Gamma}_p{\bm G}^r{\bm \Gamma}_q{\bm G}^a]$ is the transmission coefficient from the lead $q$ to the lead $p$ with the linewidth function ${\bm \Gamma}_p=\Gamma_v {\bf I}_p$ and retarded (advanced) Green's function ${\bf G}^{a(r)}$ \cite{SM}. $\Gamma_v$ is the dephasing strength \cite{Y2008}, and ${\bf I}_p$ is $4n_p \times 4n_p$ unit matrix. $n_p$ is the number of the sites coupling to the lead $p$, where $n_{1(4)}=L_y$ for the real lead 1(4) and $n_p=1$ for other leads.
%The numerical method of $T_{pq}$ is given in the Ref.\cite{SM}.

In order to reveal the half-integer quantized CCs, we investigate the equilibrium current between the virtual leads with the same voltage $V_0=E_F/e$. 
The equilibrium current flowing from the box $a$ to the box $b$ is  $J_{ba}=(e^2/h) T_d(x,y) V_0$, where $T_d(x,y)=\sum_{p\in a,q\in b}\left(T_{qp}-T_{pq}\right)$ and $(x,y)$ is the space coordinate of the virtual lead in the lower right corner of the box $a$. Here, the box $a(b)$ is the left (right) black box with size $r_x \times r_y$ in Fig.~\ref{fig2}(a). 
Figure~\ref{fig2}(b) shows that $T_d$ is half-integer quantized on the upper and lower interface between the metallic region and the magnetic field region, and zero elsewhere. %[see ]. 
For simplicity, we define $t_d\equiv T_d(L_0+L_x/2,L_0+L_y-r_y+1)$ as the value of $T_d$ near the upper interface. In this case, the equilibrium current near the interface is given by $J_c=(e^2/h) t_d V_0$, and $\partial J_c(E_F) / \partial E_F=(e/h) t_d$. 
Figure~\ref{fig2}(c) shows that $t_d=\lfloor(E_F/\epsilon_D)^2/2\rfloor+1/2$ and it is independent of $\Gamma_v$.
To summarize, these findings illustrate the existence of half-integer quantized  CCs circulating around the interface between  the metallic region and magnetic field region on the surface of the 3D TI.

% Figure environment removed
Now let's come to investigate the Hall $\sigma_{xy}$ and longitudinal conductivity $\sigma_{xx}$
using the Landauer-B{\"u}ttiker formula  \cite{SM}.
% the quantization of Hall conductivity $\sigma_{xy}$.
%According to Eq.(\ref{eq8}), $\sigma_{xy}$ cannot converge to $(n +1/2)e^2/h$ unless the conductance obey the Ohmic scaling.
%To this end, we first investigate spatial distribution voltage $V(x,y)$ for a large dephasing strength $\Gamma_v=4$ with $V_1-V_4=V_0$. We show $V(x,y)$ varies uniformly with respect to spatial position in the central region of Fig.~\ref{fig3}(a). This indicate that the electric field ${\bf E}=-{\bm \nabla} V$ is uniform and thus Ohmic scaling of bulk conductance holds.
Figures ~\ref{fig3}(a) and (b) show that $\sigma_{xy}$ decreases and converges to $1.5e^2/h$ as $L_y$ increases or as $\sigma_{\text{1D}}$ decreases by increasing $\Gamma_v$, in accordance with Eq.~(\ref{eq8}).
$\sigma_{\text{1D}}$ decrease as $\Gamma_v$ increases because of the dephasing-induced momentum relaxation \cite{Datta2007}. %and $\sigma_{xx}$ 
Here, $\sigma_{xy}$ is deviated from half-integer quantization because the edge conductance $G_{\text{1D}}$ is size-dependent and breaks the 2D Ohmic scaling.
In large-size limit, $\sigma_{xy}$ is half-integer quantized  in Figs.~\ref{fig3}(c)
agreeing with $\sigma_{xy}=e^2/h(\lfloor(E_F/\epsilon_D)^2/2\rfloor+1/2)$ ,
where $\sigma_{xx}$ varies almost linearly with $E_F$  in Figs.~\ref{fig3}(d).
We stress that the remarkable consistency between the above numerical analysis and the theoretical predictions [Eq.~(\ref{eq8})] strongly demonstrates the reliability of the obtained results.                   
Therefore,  the HIHC can be observed in the 3D TI, where the Ohmic scaling plays a crucial role in the experimental measurement of the HIHC.

%Generally, the conductivity of a small metal is size-dependent due to the quantum %interference when $L_y<l_{\phi}$, and Ohmic scaling of the conductance holds only when %$L_y>l_{\phi}$.
%To this end, we first test spatial distribution voltage $V(x,y)$ for a large dephasing %strength $\Gamma_v=4$ with $V_1-V_4=V_0$. We show $V(x,y)$ varies uniformly with respect to %spatial position in the central region of Fig.~\ref{fig3}(a). This indicate that the %electric field ${\bf E}=-{\bm \nabla} V$ is uniform and thus Ohmic scaling of bulk conductance holds.
%Further, in accordance with Eq.(\ref{eq8}),
%we find that $\sigma_{xy}$ decreases and converges to the half-integer value with increasing %$L_y$ or $\Gamma_v$ in Figs.~\ref{fig3}(b) and \ref{fig3}(c).
%Here $\sigma_{xy}$ is deviated from quantized value because the extra conductance $\delta G$ %breaks Ohmic scaling. Note that as $\Gamma_v$ increases, $l_\phi$ decreases.
%Only when $L_y \gg l_{\phi}$ and $\sigma_n L_y \gg \delta G$, 
%we find $\sigma_{xy}$ is half-integer quantized 
%as $\sigma_{xy}=e^2/h(\lfloor(E_F/\epsilon_D)^2/2\rfloor+1/2)$  in Figs.~\ref{fig3}(d),
%where $\sigma_{xx}$ is almost linearly with $E_F$  in Figs.~\ref{fig3}(e).
%Therefore, the Ohmic scaling plays a key role in the transport measurement of the HIHC.

 
%The influence of $\delta G$ on $\sigma_{xy}$ can be neglected
% when $\sigma_n L_y \gg \delta G$ so that $\sigma_{xy}= e\partial J_c(E_F) / \partial E_F$.




% Figure environment removed
{\em Experimental realization of the HIHC.}--
Due to the experimental challenges in achieving the nonuniform magnetic field in Fig.~\ref{fig2}(a), we propose a more feasible approach to measure the HIHC.
%Since the nonuniform magnetic field shown in Fig.~\ref{fig2}(b) is challenging to realize %experimentally, we propose a more feasible approach to measure the HIHC.
In Fig.~\ref{fig4}(a), a 3D TI is subjected to a uniform magnetic field ${\bf B}=B_0\hat{e}_y$, and a gate voltage $V_{g1}$ ($V_{g2}$) is applied to the front (back) surface to shift the Dirac point by an energy $eV_{g1}$ ($eV_{g2}$).
These two gate voltages will introduce opposite-type (electron-hole) carriers on the front and back surfaces when $eV_{g1}+eV_{g2}=2E_F$ \cite{SM}.
Due to the opposite electromagnetic responses of holes and electrons, the effective magnetic fields on the front and back surfaces are directed inward towards these surfaces. 
In this scenario, other four surfaces act as Dirac metals, encompassed by a pair of anti-parallel CCs on the front and back surfaces. Figure~\ref{fig4}(c) demonstrates that the number of CCs ($t_d$) near the edge of the top surface is quantized in half-integer values.
This quantized CCs is also robust to the weak disorder [see Fig.~\ref{fig4}(d)], where the disorder is introduce as $H_d=V({\bf r})\sigma_0{\otimes}s_0$ and the on-site energy $V({\bf r})$ is uniformly distributed within $[-W/2,W/2]$ with disorder strength $W$. 
Now we can employ the Hall bar device to obtain the HIHC on the top Dirac surface [see Fig.~\ref{fig4}(b)]. Since lead 4 and $4'$ have the same voltage, there is no net current between them. Thus, the current $I_x$ solely arises from the top surface, enabling measurement of the transport properties exclusively on that surface.
From this perspective, the configuration in Fig.~\ref{fig4}(c) is effectively equivalent to the arrangement shown in Fig.~\ref{fig1}(c). Therefore, we can obtain the HIHC in this Hall-bar measurement according to Eq.~(\ref{eq8}). 
%In addition, the half-integer quantized CCs can be identified by measuring the nonreciprocal conductance in experiments \cite{Gong2022}.


%and the top and bottom surfaces
%the top and bottom surfaces can be viewed as separate entities and have no influence on the %transport results of each other.

%Note that the configuration in Fig.~\ref{fig4}(b)  is effectively equivalent to the arrangement  shown in Fig.~\ref{fig1}(c) because the lead 4 and lead $4'$ have the same voltage so the measured values $I_x$ is just the current flowing along the top surface from lead 1 to lead 4 [see more details in Ref.\cite{SM}].
%To demonstrate this numerically, we apply $n_x \times n_y$ virtual leads on the top surface, and define $t_{d1}\equiv T_d(n_x/2,1)$ and $t_{d2}\equiv T_d(n_x/2,n_y-r_y)$, which are the number of the CCs on the front and back surface, respectively.

%To numerically demonstrate above analysis, we shows that $t_d$, representing the number of %CCs  near the edge of the top surface, is half-integer quantized [see Fig.~\ref{fig4}(c)].
%$t_{d}= \text{sgn}(E_F-eV_{gi})[\lfloor(E_F-eV_{gi})^2/(2\epsilon_D^2)\rfloor+1/2]$ for $i=1,2$, where $\text{sgn}(E_F-eV_{gi})$ represents the direction of the CCs.
%When $t_{d1}=-t_{d2}$, we can obtain the HIHC by using the Hall bar device to measure the longitudinal and Hall resistance of the metallic system on the top surface [see Fig.~\ref{fig4}(c)].
%Note that the  configuration in Fig.~\ref{fig4}(c)  is  equivalent to the arrangement  shown in Fig.~\ref{fig1}(c) because the lead 4 and lead $4'$ have the same voltage so the measured values $I_x$ is just the current flowing on the top surface from lead 1 to lead 4.

%Due to the opposite electromagnetic responses of holes and electrons, the effective magnetic fields on the four side surfaces are directed inward towards these surfaces. 
%In this scenario, the top and bottom surfaces act as Dirac metals, while being encompassed by Landau levels on the four side surface states influenced by a magnetic field. 
%This  configuration  is topologically equivalent to the arrangement  shown in Fig.~\ref{fig2}(a).
%Moreover, the half-integer quantized CCs on the top and bottom surface are spatially separated by the insulating side surfaces. 
%This behavior is confirmed by our numerical simulations, as shown in Fig.~\ref{fig4}(b) and (c).
%Figure~\ref{fig4}(b) shows the side surfaces become gapped when the Fermi energy $\lfloor(|eV_g|/\epsilon_D)^2/2\rfloor<(E_F/\epsilon_D)^2/2<\lfloor(|eV_g|/\epsilon_D)^2/2\rfloor+1$, with periodic boundary condition in the $z$ direction and the gate voltage $V_g$.
%Therefore, we can obtain the HIHC by using the six-terminal Hall bar device to measure the longitudinal and Hall resistance of the metallic system on the top surface.
%Note that the top and bottom surface will be connected by the chiral edge states at the hinges of side surfaces if no gate is applied [see more detail in Ref.\cite{SM}]. As a result, one cannot get the signature of the HIHC without the gate voltage.


{\em Conclusion.}-- In summary, our work confirms the feasibility of directly observing the RQHE and HIHC.
We presents a transport theory of the RQHE, and demonstrates the existence of half-integer quantized CCs at the interface of the RQHE and Dirac metal. We establish a direct relationship between the HIHC and these quantized CCs and uncover the pivotal role of Ohmic scaling in experimentally measuring the HIHC.
Furthermore, we provide a feasible proposal based on the 3D TI to obtain the HIHC and realize the RQHE. 
%The transport mechanism of the HIHC discussed above is distinct from  the bulk electromagnetic response calculation of a single Dirac cone, where the longitudinal conductance is zero \cite{Nomura2008,Nomura2011}.
%The half-integer quantized CCs coexist with a metallic phase, so a non-zero longitudinal conductance appears when detecting the HIHC by using Hall bar devices.
%In realistic systems, the Dirac cones usually appear in pair.
%Therefore, when detecting the HIHC, one cone transitions to RQHE while the other remains a metallic Dirac cone.  

% However, the HIHC in an insulating phase is forbidden in realistic systems because the half-integer quantized CCs that results in the HIHC must coexist with a metallic phase. Therefore, the presence of RQHE must be accompanied with a metallic phase in realistic systems.
%investigate the transport mechanism of HIHC. 
%We demonstrate the existence of half-integer quantized CCs that coexist with a metallic %phase and circulate around the edge of a RQHE system, both theoretically in a 2D Dirac %fermion model and numerically in a 3D TI.
%In addition, We reveal that the HIHC is directly related to the half-integer quantized CCs as the conductance of the metal satisfies  Ohmic scaling.


%Importantly, we uncover the necessity of a metallic phase for the existence of the half-integer quantized CCs, and the necessity of the dephasing process for the experimental measurement of the HIHC. 



{\emph{Acknowledgement.}}---  We thank Ming Gong, Wen-Bo Dai, Hailong Li, Runjie Zheng, Dong-Hui Xu for illuminating discussions.
This work was supported by the National Key R\&D Program of China (Grants No. 2022YFA1403700), the National Basic Research Program of China (Grant No. 2015CB921102), 
NSFC (Grants No. 11921005, and No. 11974256), 
the Innovation Program for Quantum Science and Technology (Grant No. 2021ZD0302403), and the Strategic Priority Research Program of Chinese Academy of Sciences (Grant No. XDB28000000).
C.-Z.C. is also funded by the Priority Academic Program Development of Jiangsu Higher Education Institutions.

\bibliographystyle{apsrev4-2} % Tell bibtex which bibliography style to use
\bibliography{ref}
\end{document}
