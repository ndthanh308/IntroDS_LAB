\documentclass[aps,prb,onecolumn,amssymb,superscriptaddress,letterpaper,longbibliography]{revtex4-2}
%twocolumn
\usepackage{color}
% \definecolor{darkpowderblue}{rgb}{0.0, 0.2, 0.6}
\definecolor{darkblue}{rgb}{0.0, 0.0, 0.55}
\usepackage[colorlinks=true,linkcolor=darkblue,anchorcolor=red,citecolor=darkblue, urlcolor=darkblue]{hyperref}
\usepackage{mathrsfs}
\usepackage{graphicx}
\usepackage{graphics}
\usepackage{subfigure}
\usepackage{bm}
\usepackage{dcolumn}
\usepackage{amsmath,bm}
\usepackage[capitalize]{cleveref}
%\usepackage[varg]{txfonts}
\bibliographystyle{apsrev4-2}
%%Prefix a "S" to all equations, figures, tables and reset the counter %%
\setcounter{equation}{0}
% \setcounter{figure}{0}
\setcounter{table}{0}
\setcounter{page}{0}
\makeatletter
\renewcommand{\thepage}{S\arabic{page}}
 \renewcommand{\theequation}{S\arabic{equation}}
\renewcommand{\thesection}{S\arabic{section}}
\renewcommand{\thetable}{S\arabic{table}}
\renewcommand{\thefigure}{S\arabic{figure}}
%% Prefix a "S" to all equations, figures, tables and reset the counter %%


% following
\usepackage{color}
%\renewcommand\figurename{Figure S}
\begin{document}
\title{Supplementary Materials for ``Dissipative Chiral Channels, Ohmic Scaling and Half-integer Hall Conductivity from the Relativistic Quantum Hall Effect''}
\author{Humian Zhou}
\affiliation{International Center for Quantum Materials, School of Physics, Peking University, Beijing 100871, China}
\author{Chui-Zhen Chen}
\email{czchen@suda.edu.cn}
\affiliation{School of Physical Science and Technology, Soochow University, Suzhou 215006, China}
\affiliation{Institute for Advanced Study, Soochow University, Suzhou 215006, China}
\author{Qing-Feng Sun}
\affiliation{International Center for Quantum Materials, School of Physics, Peking University, Beijing 100871, China}
\affiliation{CAS Center for Excellence in Topological Quantum Computation, University of Chinese Academy of Sciences, Beijing 100190, China}
\affiliation{Hefei National Laboratory, Hefei 230088, China}
\author{X. C. Xie}
\email{xcxie@pku.edu.cn}
\affiliation{International Center for Quantum Materials, School of Physics, Peking University, Beijing 100871, China}
\affiliation{Institute for Nanoelectronic Devices and Quantum Computing, Fudan University, Shanghai 200433, China}
\affiliation{Hefei National Laboratory, Hefei 230088, China}
\date{\today }

\maketitle
\tableofcontents
\section{The eigenstates and eigenvalues}

In this section, we will derive the eigenstates and eigenvalues of $H$, where $H=v_F(-i\hbar  {\bm \nabla}+e{\bf A})\cdot{\bm \sigma}$. The perpendicular magnetic field is given by ${\bf B}=B(y)\hat{e}_z$ with $B(y)=0$ within the strip $-L_y< y<0$ and $B(y)=B_0$ elsewhere. We choose the vector potential in the gauge ${\bf A}=A(y)\hat{e}_x$ with $B(y)=-\partial_y A(y)$:
\begin{eqnarray}\label{eq1}
	% \nonumber to remove numbering (before each equation)
	A(y)=  
	\begin{cases}
		-yB_0,&y \geq 0 \\
		0,&-L_y< y<0 \\
		-(y+L_y)B_0,&y\leq-L_y \\
	\end{cases}	.
\end{eqnarray}
The eigenstate $\psi_{\bf k}({\bf x})$ and eigenvalue $E_{\bf k}$ satisfy the equation $H\psi_{\bf k}({\bf x})=E_{\bf k}\psi_{\bf k}({\bf x})$. Here, ${\bf k}=(k_x,k_y)=(k\cos \theta,k\sin \theta)$ is the wave vector, and $\theta$ is the incidence angle. The Hamiltonian $H$ is translationally  invariant along the $x$ direction, so $k_x$ is conserved. We define $\psi_{\bf k}({\bf x})=(\psi_+,\psi_{-})^\mathrm{T} e^{i k_x x}$, and obtain the coupled equation:
\begin{eqnarray}\label{eq2}
	% \nonumber to remove numbering (before each equation)
	[\pm \partial_y + k_x + (e/\hbar) A(y) ]\psi_{\pm}=k \psi_{\mp}
	,
\end{eqnarray}
where $k=E_{\bf k}/(\hbar v_F)$. We then decouple the Eq.~\ref{eq2} as:
\begin{eqnarray}\label{eq3}
	% \nonumber to remove numbering (before each equation)
	[\partial^2_y \pm (e/\hbar) \partial_y  A(y)-(k_x+(e/\hbar) A(y))^2 + k^2 ]\psi_{\pm}=0
	.
\end{eqnarray}
The solution of the Eq.~\ref{eq3} in the three regions is as follows. For $-L_y< y<0$, the Eq.~\ref{eq2} becomes 
\begin{eqnarray}\label{eq4}
	% \nonumber to remove numbering (before each equation)
	[\partial^2_y -k_x^2 + k^2 ]\psi_{\pm}=0
	.
\end{eqnarray}
 The expression of $\psi_{\bf k}({\bf x})$ in this region is 
\begin{eqnarray}\label{eq5}
	% \nonumber to remove numbering (before each equation)
	\psi_{\rm I}({\bf x})=
	\frac{e^{ik_x x}}{\sqrt{4 L_x L_y}}
	\left[
	\begin{pmatrix} 
		e^{-i\frac{\theta}{2}}\\
		e^{i\frac{\theta}{2}} 
	\end{pmatrix} e^{ik_y y}
	+r
	\begin{pmatrix} 
		e^{i\frac{\theta}{2}}\\
		e^{-i\frac{\theta}{2}} 
	\end{pmatrix} e^{-ik_y y}
	\right]
	,
\end{eqnarray}
which is the superposition of an incident plane wave and a reflected plane wave with $\theta$-dependent reflection amplitude $r$.
Here, $k_x=2\pi n/L_x $  with $n \in \mathbb{Z}$ since we apply the periodic boundary condition in the $x$ direction. 

In the region $y \geq 0$, the Eq.~\ref{eq2} becomes 
\begin{eqnarray}\label{eq6}
	% \nonumber to remove numbering (before each equation)
	[l_B^2\partial^2_y \mp 1-(y/l_B-l_B k_x)^2 + (k l_B)^2 ]\psi_{\pm}=0
	,
\end{eqnarray}
where $l_B=\sqrt{\hbar/(eB_0)}$ is the magnetic length.
Considering the eigenstate is finite at $y=+\infty$, the expression of $\psi_{\bf k}({\bf x})$ in this region is 
\begin{eqnarray}\label{eq7}
	% \nonumber to remove numbering (before each equation)
	\psi_{\rm II}({\bf x})=
	\frac{t e^{ik_x x}}{\sqrt{4 L_x L_y}}
	\begin{pmatrix} 
		D_{(k l_B)^2/2-1}[\sqrt{2}(y/l_B-k_x l_B)]\\
		-\frac{\sqrt{2}}{k l_B} D_{(k l_B)^2/2}[\sqrt{2}(y/l_B-k_x l_B)] 
	\end{pmatrix}
    ,	
\end{eqnarray}
where $D_{\nu}(x)$ is the parabolic cylinder function \cite{Gradshteyn1980}. 
Here, $t$ is a complex coefficient.
For a given uniform perpendicular magnetic field, 
the Landau levels forms with energy $\epsilon_n={\rm sgn}(n)\epsilon_D \sqrt{2  \left | n \right |}$ with $n \in \mathbb{Z}$. 
$\epsilon_D=\hbar v_F/l_B$ is an energy scale.
$\psi_{\rm II}({\bf x})$ is a decaying wave as $(k l_B)^2/2 \notin \mathbb{Z}$.

Similarly, for $x\leq-L_y$, 
\begin{eqnarray}\label{eq8}
	% \nonumber to remove numbering (before each equation)
	[l_B^2\partial^2_y \mp 1-((y+L_y)/l_B-l_B k_x)^2 + (k l_B)^2 ]\psi_{\pm}=0,
\end{eqnarray}
and $\psi_{\bf k}({\bf x})$ in this region is also a decaying wave:
\begin{eqnarray}\label{eq9}
	% \nonumber to remove numbering (before each equation)
	\psi_{\rm III}({\bf x})=
	\frac{t' e^{ik_x x}}{\sqrt{4 L_x L_y}}
	\begin{pmatrix} 
		D_{(k l_B)^2/2-1}[-\sqrt{2}((y+L_y)/l_B-k_x l_B)]\\
		\frac{\sqrt{2}}{k l_B} D_{(k l_B)^2/2}[-\sqrt{2}((y+L_y)/l_B-k_x l_B)] 
	\end{pmatrix}	.
\end{eqnarray}
The eigenstate $\psi_{\bf k}({\bf x})$ is normalized due to $\int_{-\infty}^{+\infty} \int_{0}^{L_x} \psi^{\dagger}_{\bf k}({\bf x}) \psi_{\bf k}({\bf x}) \mathrm{d} x \mathrm{d} y=1$ in the limit $L_y \rightarrow +\infty$. 

Now, we are going to determine the value of $E_{\bf k}$ by enforcing the continuity of the eigenstate at $y=0$ and $y=-L_y$.
For $\psi_{\rm I}(x,0)=\psi_{\rm II}(x,0)$ at $y=0$, we get  
\begin{eqnarray}\label{eq10}
	% \nonumber to remove numbering (before each equation)
	t =
	-\frac{2 i \sin \theta e^{i\theta/2}}{D_1+D_2e^{i\theta}},
\end{eqnarray}
\begin{eqnarray}\label{eq11}
	% \nonumber to remove numbering (before each equation)
	r(k,\theta)
	\equiv e^{i\phi_r(k,\theta)} 
	=
	-\frac{D_2+D_1e^{i\theta}}{D_1 +D_2e^{i\theta}}
	,
\end{eqnarray}
where $D_1=D_{(k l_B)^2/2-1}(-\sqrt{2} k_x l_B)$ and $D_2=\sqrt{2}/(k l_B) D_{(k l_B)^2/2}(-\sqrt{2} k_x l_B)$. $\phi_r(k,\theta)$ is the reflection phase, representing the phase shift between the reflected wave and the incident wave. At $y=-L_y$, $\psi_{\rm I}(x,-L_y)=\psi_{\rm III}(x,-L_y)$, so we get
\begin{eqnarray}\label{eq12}
	% \nonumber to remove numbering (before each equation)
	r(k,\theta)e^{i 2 k_y L_y}
	\equiv e^{i\phi'_r(k,\theta)} 
	=
	-\frac{D'_2+D'_1e^{i\theta}}{D'_1 +D'_2e^{i\theta}}
	,
\end{eqnarray}
where $D'_1=D_{(k l_B)^2/2-1}(\sqrt{2} k_x l_B)$ and $D'_2=-\sqrt{2}/(k l_B) D_{(k l_B)^2/2}(\sqrt{2} k_x l_B)$. Combining the Eq.~\ref{eq11} and the Eq.~\ref{eq12}, we get
\begin{eqnarray}\label{eq13}
	% \nonumber to remove numbering (before each equation)
	k_y
	=
	\frac{n\pi}{L_y}+\frac{\phi'_r(k,\theta)-\phi_r(k,\theta)}{2L_y} \quad (n \in \mathbb{Z})
\end{eqnarray}
For a given $k_x$, both $\theta$ and $k$ is the function of $k_y$. Therefore, one will get the values of $k_y$ by solving the Eq.~\ref{eq13}. This equation can be analytically solved in the limit $L_y \rightarrow +\infty$. We define $k_n$ as the root of Eq.~\ref{eq13} for a given $n$, and $\delta_n=k_n-k_{n-1}$. Then,
\begin{eqnarray}\label{eq14}
	% \nonumber to remove numbering (before each equation)
	\delta_n
	&=&
	\frac{\pi}{L_y}+\frac{\phi'_r(k(k_n),\theta(k_n))-\phi_r(k(k_n),\theta(k_n))}{2L_y} -\frac{\phi'_r(k(k_{n-1}),\theta(k_{n-1}))-\phi_r(k(k_{n-1}),\theta(k_{n-1}))}{2L_y} \nonumber \\
	&=&\frac{\pi}{L_y}+ \left ( \frac{\mathrm{d} \phi'_r(k(k_n),\theta(k_n)) }{\mathrm{d} k_n}-\frac{\mathrm{d} \phi_r(k(k_n),\theta(k_n)) }{\mathrm{d} k_n} \right )\frac{\pi}{2L_y^2}
	+ O(1/L_y^2)
	.
\end{eqnarray}
In the limit $L_y \rightarrow +\infty$, $\delta_n=\pi/L_y$ and so $k_n=n\pi/L_y+k_0$. Obviously, $k_0=0$ is one root of the Eq.~\ref{eq13}. So $k_n=n\pi/L_y$. Note that $\psi_{k_x,-k_y}({\bf x})=r(k,\theta)^{-1}\psi_{k_x,k_y}({\bf x})$ which means $\psi_{k_x,k_y}({\bf x})$ and $\psi_{k_x,-k_y}({\bf x})$ represent the same eigenstate, so $k_y$ only takes the positive values with
\begin{eqnarray}\label{eq15}
	% \nonumber to remove numbering (before each equation)
	k_y
	=
	\frac{n\pi}{L_y} \quad (n \in \mathbb{N}).
\end{eqnarray}
In summary, the eigenstate is:
\begin{eqnarray}\label{eq151}
	% \nonumber to remove numbering (before each equation)
	\psi_{\bf k}({\bf x})=  
	\begin{cases}
		\psi_{\rm II}({\bf x}),&y \geq 0 \\
		\psi_{\rm I}({\bf x}),&-L_y< y<0 \\
		\psi_{\rm III}({\bf x}),&y\leq-L_y \\
	\end{cases}	.
\end{eqnarray}
and the eigenvalue $E_{\bf k}=\pm\hbar v_F \sqrt{k_x^2+k_y^2}$.  $(k_x,k_y)=(2\pi m/L_x,\pi n/L_y)$ with $m\in \mathbb{Z}$ and $n\in \mathbb{N}$ as $L_y \rightarrow +\infty$. Note that the $\psi_{\rm III}({\bf x})$ has no contribution for the calculation of equilibrium current and the number of chiral channels, so we do not show the expression of $\psi_{\rm III}({\bf x})$ in the main text.




\section{Half-integer quantized chiral channels}
In this section, we will calculate the equilibrium current and determine the number of chiral channels (CCs). For 2D Dirac fermions, the electric current density in the $x$ direction for $\psi_{\bf k}({\bf x})$ is given by $j_{x,{\bf k}}({\bf x})=-ev_F \psi_{\bf k}^{\dagger}({\bf x}) \sigma_x \psi_{\bf k}({\bf x})$. Then, the equilibrium current flowing through the section $y>-d$ $(0<d<L_y)$ for the occupied states is $J_x(E_F,d)=\sum_{E_{\bf k}\leq E_F}
\int_{-d}^{+\infty}j_{x,{\bf k}}({\bf x})\mathrm{d}y$.
We split the integral into two parts and calculate them separately:
\begin{eqnarray}\label{eq16}
	% \nonumber to remove numbering (before each equation)
	\int_{-d}^{+\infty}j_{x,{\bf k}}({\bf x})\mathrm{d}y
	&=
	-\int_{-d}^{0}
	ev_F \psi_{{\rm I},\bf k}^{\dagger}({\bf x}) \sigma_x \psi_{{\rm I},\bf k}({\bf x})
	\mathrm{d}y
	-
	\int_{0}^{+ \infty}
	ev_F \psi_{{\rm II},\bf k}^{\dagger}({\bf x}) \sigma_x \psi_{{\rm II},\bf k}({\bf x})
	\mathrm{d}y
	.
\end{eqnarray}
For the first part, we use Eq.~\ref{eq5} and obtain
\begin{eqnarray}\label{eq17}
	% \nonumber to remove numbering (before each equation)
	\int_{-d}^{0}
	ev_F \psi_{{\rm I},\bf k}^{\dagger}({\bf x}) \sigma_y \psi_{{\rm I},\bf k}({\bf x})
	\mathrm{d}y
	&=\frac{e v_F}{2 L_x L_y} \left [ 2 d \cos \theta +\frac{\sin (\phi_r +2dk_y)-\sin\phi_r}{k_y} \right ]
	.
\end{eqnarray}
For the second part, we use Eq.~\ref{eq7} and obtain
\begin{eqnarray}\label{eq18}
	% \nonumber to remove numbering (before each equation)
	\int_{0}^{+ \infty}
	ev_F \psi_{{\rm II},\bf k}^{\dagger}({\bf x}) \sigma_y \psi_{{\rm II},\bf k}({\bf x})
	\mathrm{d}y
	&=&-\frac{e v_F |t|^2}{2 L_x L_y} \frac{\sqrt{2}}{kl_B}
		\int_{0}^{+ \infty}	
		D_{(k l_B)^2/2-1}[\sqrt{2}(y/l_B-k_x l_B)]
		D_{(k l_B)^2/2}[\sqrt{2}(y/l_B-k_x l_B)]
		\mathrm{d}x \nonumber \\
		&=&-\frac{e v_F |t|^2 l_B^2}{4 L_x L_y}
		\left[ k (D_1^2+D_2^2)+2 k_x  D_1 D_2 \right] \nonumber
		\\
		&=&\frac{e v_F}{2 L_x L_y}
		\left [ \frac{\partial \phi_r(k,\theta)}{k \partial \theta}
		+\frac{\sin \phi_r}{k_y} \right ]
\end{eqnarray}
where we use the relation
\begin{eqnarray}\label{eq19}
	% \nonumber to remove numbering (before each equation)
	\frac{\partial \phi_r(\theta)}{k \partial \theta}
		&=&
		\frac{D_1^2-D_2^2}{k |D_1+D_2e^{i\theta}|^2}-\frac{2\sin^2 \theta l_B^2 }{|D_1+D_2 e^{i\theta}|^2}[k (D_1^2+D_2^2)+2 k_x D_1 D_2],
\end{eqnarray}
and
\begin{eqnarray}\label{eq20}
	% \nonumber to remove numbering (before each equation)
	\frac{\sin \phi_r}{k_y}
	&=&
	-\frac{D_1^2-D_2^2}{k|D_1+D_2e^{i\theta}|^2}
	.
\end{eqnarray}
Then, we calculate $J_x(E_F,d)$ by converting the sum into an integral as $L_x$ and $L_y$ approach to infinity:
\begin{eqnarray}\label{eq21}
	% \nonumber to remove numbering (before each equation)
	J_x(E_F,d)
	&=&-
	\lim_{L_x  \rightarrow +\infty} 
	\lim_{L_y \rightarrow +\infty}
	\sum_{E_{\bf k}\leq E_F}
    \frac{e v_F}{2 L_x L_y}
    \left [ 2 d \cos \theta  +\frac{\partial \phi_r(k,\theta)}{k \partial \theta}+
    \frac{\sin (\phi_r +2dk_y)}{k_y}
    \right ] \nonumber \\
    &=&
    -\int_{0}^{+\infty}
    \frac{\mathrm{d} k_y}{\pi/L_y} 
    \int_{-\infty}^{+\infty} 
    \frac{\mathrm{d} k_x}{2\pi/L_x}
    \Theta(E_F-E_{\bf k})
    \frac{e v_F}{2 L_x L_y}
    \left [ 2 d \cos \theta  +\frac{\partial \phi_r(k,\theta)}{k \partial \theta}+
    \frac{\sin (\phi_r +2dk_y)}{k_y}
    \right ]\nonumber \\
    &=&
    -\int_{0}^{+\infty}
    \mathrm{d} k
    \int_{0}^{\pi} 
    k \mathrm{d} \theta
    \Theta(E_F-E_{\bf k})
    \frac{e v_F}{(2 \pi)^2}
    \left [ 2 d \cos \theta  +\frac{\partial \phi_r(k,\theta)}{k \partial \theta}+
    \frac{\sin (\phi_r +2dk_y)}{k_y}
    \right ]
\end{eqnarray}
where the $\Theta(x)$ is the step function. 
$J_x(E_F,d)$ is nonzero and exhibits a chiral nature near the edge of the metallic region due to the breaking of time-reversal symmetry in the presence of a magnetic field.
By taking the partial derivative of $J_x(E_F,d)$, one can get
%So the total current of the eigenstates with energy $E_F$ is
\begin{eqnarray}\label{eq22}
	% \nonumber to remove numbering (before each equation)
	\frac{\partial J_x(E_F,d)}{\partial E_F}
	&=&
	-\int_{0}^{+\infty}
	\mathrm{d} k
	\int_{0}^{\pi} 
	k \mathrm{d} \theta
	\delta(E_F-E_{\bf k})
	\frac{e v_F}{(2 \pi)^2}
	\left [ 2 d \cos \theta  +\frac{\partial \phi_r(k,\theta)}{k \partial \theta}+
	\frac{\sin (\phi_r +2dk_y)}{k_y}
	\right ] \nonumber \\
	&=&
	\frac{e}{h} 
	\int_{0}^{\pi} 
	\frac{\mathrm{d} \theta}{2\pi}
	\left ( -
	\frac{\partial \phi_r(k_F ,\theta)}{ \partial \theta}
	\right )
	-
	\frac{e}{h} 
	\int_{0}^{\pi} 
	\frac{\mathrm{d} \theta}{2\pi}
	\frac{\sin (\phi_r +2dk_F \sin \theta)}{\sin \theta}
	.
\end{eqnarray}
Here, we use the relation $E_{\bf k}=\hbar v_F k$ and  define $ k_F =E_F/(\hbar v_F)$.
$[(h/e){\partial J_x(E_F,d)}/{\partial E_F}]$ is the number of CCs in the region $y>-d$, which directly determines the Hall transport. 
The first term of Eq.\ref{eq22} is independent of the distance $d$.
 The second term of the Eq.~\ref{eq22} is approximately equal to $1/\sqrt{4\pi k_F d} \cos (2k_F d +\phi_0(k_F))$ as $ d/l_B $ tends to the infinity [see Fig.~\ref{figs1}(a)]. 
 So this term decays to zero in a power law $(d/l_B)^{-1/2}$, which means that the equilibrium current is mainly located at the boundary between the metallic region ($y<0$) and the magnetic field region ($y>0$). 
 %The first term is independent of $d$, and completely determines the value of the chiral current.
 
 For convenience, we define $J_c(E_F)=\lim_{d  \rightarrow +\infty} J_x(E_F,d)$ as the $d$-independent part of the equilibrium current $J_x(E_F,d)$. $J_c(E_F)$ is the total equilibrium current near the edge of metallic region.
 Therefore, $[(h/e){\partial J_c(E_F)}/{\partial E_F}]$ is the total number of CCs located at edge of the metallic region. Then, we obtain
\begin{eqnarray}\label{eq23}
  	% \nonumber to remove numbering (before each equation)
  	\frac{h}{e}\frac{\partial J_c(E_F)}{\partial E_F}
  	&=&
  	\int_{0}^{\pi} 
  	\frac{\mathrm{d} \theta}{2\pi}
  	\left ( -
  	\frac{\partial \phi_r(k_F ,\theta)}{ \partial \theta}
  	\right ) \nonumber \\
  	&=&
  	\frac{\phi_r(k_F ,0)-\phi_r(k_F , \pi)}{2\pi}  	
  	.
  \end{eqnarray}
The relationship $r(k_F,0)r(k_F,\pi)=-1$ derived from Eq.~\ref{eq12} leads to $\phi_r(k_F ,0)-\phi_r(k_F , \pi)=2\pi(n+1/2)$ with $n\in \mathbb{Z}$.
The exact value of $n$ is determined by the Fermi energy $E_F$ [see Fig.~\ref{figs1}(b)], and $n=\lfloor (k_F l_B)^2/2 \rfloor=\lfloor ( {E_F}/{\epsilon_D})^2/2 \rfloor$ where $\lfloor x \rfloor$ denotes the largest integer less than $x$. So
\begin{eqnarray}\label{eq24}
	% \nonumber to remove numbering (before each equation)
		\frac{h}{e}\frac{\partial J_c(E_F)}{\partial E_F}	
	&=&
	 \lfloor ( {E_F}/{\epsilon_D})^2/2 \rfloor+\frac{1}{2}.  	
\end{eqnarray}
Half-integer quantized $\partial J_c(E_F)/\partial E_F$ implies that there are half-integer quantized CCs near the edge of the magnetic field region, which is the origin of the half-integer Hall conductance.

% Figure environment removed

 

%$J_y(E_F,d)$ is an equilibrium current circulating around the edge of magnetic field region, and has a chiral nature. Meanwhile, ${\partial J_y(E_F,d)}/{\partial E_F}$ represents the chiral channel, which directly determines the Hall transport property.




\section{The relation between the CCs and the Hall conductance  }
% Figure environment removed
In this section, we will establish an analytic relation between the CCs and the Hall conductance. 
We consider a metallic system (yellow region) with half-integer quantized CCs circulating around its edge, as showed in Fig.~\ref{figs2}(a).
To observe a stable HIHC in experiments,  it is essential for both the Hall resistivity and longitudinal resistivity to remains independent of system's size.
For this purpose, we assume an Ohmic scaling of the conductance $G_{\text{2D}}$ in the metallic system, i.e. $G_{\text{2D}}=\sigma_nL_y/L_x$, ensuring that the conductivity $\sigma_n$ is unaffected by the system's size.
In this scenario, we can consider this metallic system as composed of numerous independent boxes, each with size $r_c$ and conductance $G=\sigma_n$. These boxes can be viewed as independent classical resistors connected together in accordance with Ohm's law.
To incorporate the contribution from the CCs near the edge, we introduce the 1D conductivity $\sigma_{\text{1D}}$ that arises from the CCs, and assume the corresponding conductance satisfies 1D Ohmic scaling law, i.e. $G_{\text{1D}}=\sigma_{\text{1D}}/L_x$.
We assume that there is a slowly varying electric potential with a uniform electric field ${\bf E}=(E_x,-E_y)$ in metallic region.
The electric voltage distribution is $V=-x E_x + y E_y$ for given ${\bf E}=(E_x,-E_y)$.

Considering the current conservation in each gray box such as the box 1, we obtain:
\begin{eqnarray}\label{eq25}
	[\sigma_n E_x r_c+\sigma_{\text{1D}} E_x+J_c(eV_{12})]=
	[\sigma_n E_x r_c+\sigma_{\text{1D}} E_x+J_c(eV_{13})]+E_y \sigma_n r_c
\end{eqnarray}
The left-hand side represents the coming current, while the right-hand side represents the outgoing current. The  left-hand side is the current that flows from box 2 to box 1. $[\sigma_n E_x r_c+\sigma_{\text{1D}} E_x]$ is the longitudianal current, and $J_c(eV_{12})$ is the Hall current flowing through the CCs. Here, $V_{12}$ is the voltage at the boundary between the box 1 and the box 2, and similarly for $V_{13}$. The first term of the right-hand side is the longitudinal current flowing from the box 1 to the box 3. The second term of the left-hand side is the current flowing from the box 1 to the box 4 thought the normal channels. 
Considering the voltage varies very slowly with space ($|eV-E_F|\ll \epsilon_D$), we get $J_c(eV)=\partial J_c(E_F) / \partial E_F (eV-E_F) +J_c(E_F)$. So the Eq.~\ref{eq25} becomes:
\begin{eqnarray}\label{eq26}
 E_y \sigma_n=E_x e \frac{\partial J_c(E_F)}{\partial E_F}
\end{eqnarray}
Additionally, the total current flowing in the horizontal ($x$) direction is:
\begin{eqnarray}\label{eq27}
 I_x &=& \sigma_n E_x L_y+\sigma_{\text{1D}} E_x l_{\phi}+J_c(eV_{\text{up}})-J_c(eV_{\text{down}}) \nonumber \\
 &=& \sigma_n E_x L_y+\sigma_{\text{1D}} E_x+ e \frac{\partial J_c(E_F)}{\partial E_F} E_y L_y
\end{eqnarray}
where the $V_{\text{up}}$($V_{\text{down}}$) is the voltage of the upper (lower) edge,  and $V_{\text{up}}-V_{\text{down}}=E_y L_y$. 
the first two terms are the longitudinal current, and the third term is the Hall current from the CCs.
Then the longitudianal resistance $\rho_{xx}=E_x L_y/I_x$ and Hall resistance $\rho_{xy}=E_y L_y/I_x$. 
The longitudinal conductance $\sigma_{xx}$ and Hall conductance $\sigma_{xy}$ are obtained from the tensor relation: $\sigma_{xx}=\rho_{xx}/(\rho_{xx}^2+\rho_{xy}^2)$, and $\sigma_{xy}=\rho_{xy}/(\rho_{xx}^2+\rho_{xy}^2)$.
\begin{eqnarray}\label{eq28}
	\sigma_{xy}(E_F)=\left[1+\frac{1}{1+(\sigma_c/\sigma_n)^2}\frac{\sigma_{\text{1D}}}{ \sigma_n L_y }\right] \sigma_c \nonumber \\
		\sigma_{xx}(E_F)= \left[1+\frac{1}{1+(\sigma_c/\sigma_n)^2}\frac{\sigma_{\text{1D}}}{\sigma_n L_y}\right]\sigma_n
\end{eqnarray}
where we define $\sigma_c \equiv e\partial J_c(E_F) / \partial E_F$.

\section{Dissipation from the CCs}
In this section, we will demonstrate how the dissipation emerges in the CCs. we consider $r_c^2$ virtual leads attached on the box 1 and box 2 in Fig.~\ref{figs2}(b). We assume that the transition coefficient between the lead $p$ and the $q$ is $T_{pq}$ and the CCs are located at the edge. 
%The transition coefficient difference between  $f_n({\bf r})$ and $f_c({\bf r})$ represent part of the normal and chiral channels, respectively. 
%Considering the chiral channels are located at the edge, it is reasonable to assume $f_c(x,y=1)=f_c(x)$ and $f_c(x,y)=0$ for $y \geq 2$. 
The voltage of each virtual leads is $V({\bf r})=-E_x x +E_y y$,  where ${\bf r}_p=(x_p,y_p)$ is the coordinate of the virtual lead. 
We denotes $V_{\text{up}}=-E_x x_0+E_y y_0$ where $(x_0,y_0)$ is the space coordinate of the virtual lead in the top left corner of the box 1.
Then the current flowing from the box 1 to the box 2 is
\begin{eqnarray}\label{eq281}
	I_{21}&=&\frac{e^2}{h}\sum_{p\in 2,q\in 1}\left(T_{pq} V_q-T_{qp}V_p\right) \nonumber \\
	&=&\frac{e^2}{h}\sum_{p\in 2,q\in 1}\left(T_{pq} -T_{qp}\right)V_q+\frac{e^2}{h}\sum_{p\in 2,q\in 1}T_{qp}(V_q-V_p) \nonumber \\
	&=&\frac{e^2}{h}\sum_{p\in 2',q\in 1'}\left(T_{pq} -T_{qp}\right)(x_0-x_q)E_x+\frac{e^2}{h}\sum_{p\in 2',q\in 1'}\left(T_{pq} -T_{qp}\right)V_{\text{up}}+\frac{e^2}{h}\sum_{p\in 2,q\in 1}T_{qp}(x_p-x_q)E_x \nonumber \\
	&=&\sigma_{\text{1D}}E_x+\sigma_c V_{\text{up}}+\sigma_n E_x
\end{eqnarray}
Here, $\sigma_{\text{1D}}={e^2}/{h}\sum_{p\in 2',q\in 1'}\left(T_{pq} -T_{qp}\right)(x_0-x_q)$ obviously comes from the chiral channels located the edge, which provides the longitudinal current. $\sigma_c={e^2}/{h}\sum_{p\in 2',q\in 1'}\left(T_{pq} -T_{qp}\right)$ represents the half-integer quantized chiral channels, which provides the Hall currents. Meanwhile, $\sigma_n={e^2}/{h}\sum_{p\in 2,q\in 1}T_{qp}(x_p-x_q)$ represents the normal channels, which provides the longitudinal currents.
There is a voltage drop along the CCs when the CCs coexist with the metallic system, which leads to dissipation in the CCs. So we refer to the CCs as dissipative CCs.


\section{Lattice model Hamiltonian and dephasing}
We consider a 3D topological insulator (TI) with a perpendicular magnetic field applied in the blue region as showed in the Fig.~\ref{figs3}(a). 
Here, ${\bf B}=B(x,y)\hat{e}_z$, and $B(x,y)=0$ in the yellow region ($L_0<x<L_0+L_x$, and $L_0<y<L_0+L_y$), and $B(x,y)=B_0$ elsewhere. We choose the vector potential in the gauge ${\bf A}=A(x,y)\hat{e}_x$. $A(x,y)=-yB_0$ for $0<x<L_0$ and $L_0+L_x<x<2L_0+L_x$; For $L_0<x<L_0+L_x$, we have: 
\begin{eqnarray}\label{eq30}
	% \nonumber to remove numbering (before each equation)
	A(x,y)=  
	\begin{cases}
		-yB_0,&0<y<L_0 \\
		-L_0 B_0,&L_0\leq y\leq L_0+L_y \\
		-(y-L_y)B_0,&L_0+L_y <y< 2L_0+L_y\\
	\end{cases}	.
\end{eqnarray}

We discretize the Hamiltonian $H_{\text{3D}}$ in the main text on cubic lattices:
\begin{eqnarray}\label{eq31}
	%\nonumber to remove numbering (before each equation)
	H_{\text{3D}}=&\left[
	\sum_{{\bf x}}{\bf c}_{\bf x}^{\dagger}{\mathcal M}_0{\bf c}_{\bf x}
	+\left(\sum_{{\bf x},j=x,y,z}{\bf c}_{\bf x}^{\dagger}{\mathcal T}_{{\bf x},j}{\bf c}_{{\bf x}+{\bf \hat{e}}_j}+\text{H.c.}\right)
	\right]
 +H_{\text{lead}}
\end{eqnarray}
where ${\mathcal M}_0=\left(M_0-6B/a^2\right)\sigma_z{\otimes}s_0$, and ${\mathcal T}_{{\bf x},j}=(B/a^2\sigma_z{\otimes}s_0-iA/(2a)\sigma_x{\otimes}s_j)e^{i(e/\hbar)a A(x,y)\delta_{y,j}}$.
The magnetic field ${\bf B}$ is included by doing the Peierls
substitution \cite{Hofstadter1976} as a phase of the hopping term.
Here, $a=1$ is the cubic lattice constant.
${\bf c}_{\bf x}$ (${\bf c}_{\bf x}^{\dagger}$) is the annihilation (creation) operator at site ${\bf x}$, and ${\bf \hat{e}}_j$ is a unit vector in the $j$ direction for $j=x,y,z$.
$\delta_{y,j}=1$ for $j=y$ and is zero for $j=x,z$.
The first and the second terms in Eq.~\ref{eq31} describe the 3D TI in the central region. The last term $H_{\text{lead}}=
\sum_{{\bf x},{\bf k}}\left[\epsilon_{{\bf k}}{\bf a}_{{\bf x},{\bf k}}^{\dagger}{\bf a}_{{\bf x},{\bf k}}+(t_{{\bf x},{\bf k}}{\bf a}_{{\bf x},{\bf k}}^{\dagger}{\bf c}_{{\bf x}}+\text{H.c.})\right]$ represents the Hamiltonian of virtual leads and real leads, and their couplings to the central sites.

Then, we can calculate the transmission coefficient $T_{pq}(E_F)$.
$T_{pq}(E_F)=\mbox{Tr}[{\bm \Gamma}_p{\bm G}^r{\bm \Gamma}_q{\bm G}^a]$, where the linewidth function ${\bm \Gamma}_p=i\left({\bm \Sigma}_p^r-{\bm \Sigma}_p^{r\dagger}\right)$ and the Green's function ${\bm G}^r=[{\bm G}^a]^{\dagger}=[E_F{\bm I}-H_{\mbox{cen}}-\sum_{p}{\bm \Sigma}_p^{r}]^{-1}$. ${\bm \Sigma}_p^r$ is the retarded self-energy due to the coupling to the lead $p$ and $H_{\mbox{cen}}$ is the lattice Hamiltonian of the TI [the first and the second term of Eq.~\ref{eq31}].
Note that we use the wide-band approximation for the real leads in the main text with ${\bm \Gamma}_p=\Gamma_v {\bf I}_p$, and $\Gamma_v=2\pi \rho t_k^2$  is the dephasing strength and $\rho$ is the density of state in the virtual lead \cite{Y2008}.
% In fact, the virtual leads can be used to mimic the dephasing process and serve as voltage probes.
Only electrons in the gapless surface states participate in the electrical transport when $E_F$ is located within the Landau levels at zero or low temperature. Therefore, we just need to consider the dephasing process on the metallic region.
We simulate the dephasing process by using $n_x\times n_y$ uniformly distributed B{\"u}ttiker's virtual leads on the metallic region [yellow region in Fig.~\ref{figs3}(a)].
%We select one site for every four sites in the x direction and the y direction on the bottom surface (or in the x direction and the z direction on the two sides surfaces)[see Fig.~\ref{fig1}(b)]. 
%Each selected site is attached by four virtual leads since every site has two spin and two orbital degrees of freedom.
%$N_i$ and $n_i$ are the number of total sites and selected sites in the $i$ direction for $i=x,y,z$, respectively.


In our calculations, the real lead 1 and lead 4 act as current electrodes and other real leads act as voltage electrodes. Thus, the longitudinal current $I_1=-I_4\equiv I_x$ when a external bias is applied between lead 1 and lead 4 with $V_1=V_0$ and $V_4=0$. Virtual leads act as dephasing, so the net current of each virtual lead is zero. 
Combining Landauer-B{\"u}ttiker formula\cite{Buttiker1988,Data1995} with those boundary conditions in the real and virtual leads, one can calculate the voltage of each real and virtual lead and the longitudinal current $I_x$. We define $V(x,y)$ as the voltage of the virtual leads, and $(x,y)$ is the coordinate of the virtual leads. We show $V(x,y)$ varies uniformly with respect to spatial position in the central region of Fig.~\ref{figs3}(b). This indicate that the electric field ${\bf E}=-{\bm \nabla} V$ is uniform and thus Ohmic scaling of bulk conductance holds.
%Then the Hall resistance $R_{xy}=(V_2-V_6)/I_x$ and longitudinal resistance $R_{xx}=(V_2-V_3)/I_x$ will be obtained. 
The longitudinal and Hall resistance are given by $\rho_{xx}=[(V_2-V_3)/I_x]/(L/W)$ and $\rho_{xy}=(V_2-V_6)/I_x$, respectively. We then obtain the longitudinal and Hall conductance by using $\sigma_{xx}=\rho_{xx}/(\rho_{xx}^2+\rho_{xy}^2)$ and $\sigma_{xy}=\rho_{xy}/(\rho_{xx}^2+\rho_{xy}^2)$. 
$L$ is the length between the lead 2 and 3, and $W=L_y$ is the width of the metallic region.

% Figure environment removed

\section{The Hamiltonian of TI with the gate voltages and the Hall bar device}
In the main text, we apply two gate voltages on the front and back surface of the 3D TI, respectively[see Fig.~4(a) in the main text]. Here, we show the Hamiltonian of 3D TI when applying the gate voltages.
We choose the vector potential in the gauge ${\bf A}=A(x)\hat{e}_z=-x B_0\hat{e}_z$ and discretize the Hamiltonoan $H_{\text{3D}}$ on cubic lattices:
\begin{eqnarray}\label{eq32}
	%\nonumber to remove numbering (before each equation)
	H_{\text{3D}}=&
	\sum_{{\bf x}}{\bf c}_{\bf x}^{\dagger}{\mathcal M}_0{\bf c}_{\bf x}
	+\left(\sum_{{\bf x},j=x,y,z}{\bf c}_{\bf x}^{\dagger}{\mathcal T}_{{\bf x},j}{\bf c}_{{\bf x}+{\bf \hat{e}}_j}+\text{H.c.}\right)
	+ H_{g1}+H_{g2}
\end{eqnarray}
where ${\mathcal M}_0=\left(M_0-6B/a^2\right)\sigma_z{\otimes}s_0$, and ${\mathcal T}_{{\bf x},j}=(B/a^2\sigma_z{\otimes}s_0-iA/(2a)\sigma_x{\otimes}s_j)e^{i(e/\hbar)a A(x)\delta_{z,j}}$.
The magnetic field ${\bf B}$ is included by doing the Peierls substitution \cite{Hofstadter1976} as a phase of the hopping term.
$H_{g1}=\sum_{{\bf x} \in \text{front surfaces}}{\bf c}_{\bf x}^{\dagger}{\mathcal M}_{g1}{\bf c}_{\bf x}$ with ${\mathcal M}_{g1}=eV_{g1}\sigma_0{\otimes}s_0$ and $H_{g2}=\sum_{{\bf x} \in \text{back surfaces}}{\bf c}_{\bf x}^{\dagger}{\mathcal M}_{g2}{\bf c}_{\bf x}$ with ${\mathcal M}_{g2}=eV_{g2}\sigma_0{\otimes}s_0$.
Here, $a=1$ is the cubic lattice constant.
${\bf c}_{\bf x}$ (${\bf c}_{\bf x}^{\dagger}$) is the annihilation (creation) operator at site ${\bf x}=(x,y)$, and ${\bf \hat{e}}_j$ is a unit vector in the $j$ direction for $j=x,y,z$.
$\delta_{z,j}=1$ for $j=z$ and is zero for $j=x,y$.

The gate voltage $V_{g1}$ ($V_{g2}$) is applied to the front (back) surface to shift the Dirac point by an energy $eV_{g1}$ ($eV_{g2}$). Therefore, the number and direction of the CCs on the front (back) surface can be tuned by the gate voltage $V_{g1}$ ($V_{g2}$).
To demonstrate this numerically, we apply $n_x \times n_y$ virtual leads on the top surface, and define $t_{d1}\equiv T_d(n_x/2,1)$ and $t_{d2}\equiv T_d(n_x/2,n_y-r_y)$, which are the number of the CCs on the front and back surface, respectively.
Fig.~\ref{figs4}(b) shows that $t_{di}= \text{sgn}(E_F-eV_{gi})[\lfloor(E_F-eV_{gi})^2/(2\epsilon_D^2)\rfloor+1/2]$ for $i=1,2$, where $\text{sgn}(E_F-eV_{gi})$ represents the direction of the CCs.
The gate voltages $V_{g1}$ and $V_{g2}$ can introduce opposite-type (electron-hole) carriers on the respective opposite side surfaces, and lead to $t_{d1}=-t_{d2}$ when $eV_{g1}+eV_{g2}=2E_F$.
$t_{d1}=-t_{d2}$ means that there are the same number but opposite direction of CCs  located at the edge of the front and back surface, respectively.
Actually, as long as $V_{g1}$ and $V_{g2}$ satisfy the relation $\lfloor(E_F-eV_{g2})^2/(2\epsilon_D^2)\rfloor<(E_F-eV_{g2})^2/(2\epsilon_D^2)<\lfloor(E_F-eV_{g1})^2/(2\epsilon_D^2)\rfloor+1$, one can get $t_{d1}=-t_{d2}$.

When $t_{d1}=-t_{d2}$, we can obtain the HIHC by utilizing the Hall bar device to measure the longitudinal and Hall resistance of the metallic system on the top surface [see Fig.~\ref{figs4}(b)].
The current flowing out from lead 1 will be divided into two components. One component flows along the top surface and enters lead 4, while the other component flows along the bottom surface and enters lead $4'$. Since lead 4 and lead $4'$ have the same voltage, there is no net current between them. Therefore, the top and bottom surfaces can be viewed as separate entities and do not have any influence on the transport results of each other.
From this perspective, the configuration depicted in Figure~\ref{figs4}(b) is equivalent to the arrangement shown in Figure.~1(c) in the main text.

% Figure environment removed

% Figure environment removed


\section{An alternative approach to measure the HIHC}

In this section, we will present an alternative approach for measuring the HIHC.
In Fig.~\ref{figs5}(a), a 3D TI is subjected to a uniform magnetic field ${\bf B}=(B_0,B_0,0)$, and gate voltages are applied to the side surfaces, introducing opposite-type (electron-hole) carriers on the respective opposite side surfaces.
Due to the opposite electromagnetic responses of holes and electrons, the effective magnetic fields on the four side surfaces are directed inward towards these surfaces. 
In this scenario, the top and bottom surfaces act as Dirac metals, while being encompassed by Landau levels on the four side surface states influenced by a magnetic field. 
This  configuration  is topologically equivalent to the arrangement  shown in Fig.1(c) in the main text.
Moreover, the half-integer quantized CCs on the top and bottom surface are spatially separated by the insulating side surfaces. 
This behavior is confirmed by our numerical simulations, as shown in Fig.~\ref{figs5}(b) and (c).
Figure~\ref{figs5}(c) shows the side surfaces become gapped when the Fermi energy satisfies the relation that $td_1=-td_2$, with periodic boundary condition in the $z$ direction and the gate voltage $eV_{g1}=-eV_{g2}=0.1$.
Therefore, we can obtain the HIHC by using the six-terminal Hall bar device to measure the longitudinal and Hall resistance of the metallic system on the top surface.
Compared with the approach in the main text [see Fig.~(4) in the main text], this approach does not need the lead $4'$ in Fig.~\ref{figs4}(b). However, this approach need to apply gate on the four side surfaces, which may be more difficult to realize in experiments.

Next, we will present the details of the calculation of the band structure of the TI in a uniform magnetic field ${\bf B}=(B_0,B_0,0)$ and two different gates on the opposites side surfaces [see Fig.~\ref{figs5}(a)]. 
To make the Hamiltonian is translationally invariant along the $z$ direction, we choose the vector potential in the gauge ${\bf A}=A(x,y)\hat{e}_z=B_0(y-x)\hat{e}_z$. 
%Considering this system  is translationally  invariant along the $z$ direction,
Considering the open boundary condition in the $x$ and $y$ direction and periodic boundary condition in the $z$ direction, we discretize the Hamiltonoan $H_{\text{3D}}$ on cubic lattices:
\begin{eqnarray}\label{eq32}
	%\nonumber to remove numbering (before each equation)
	H(k_z)=&
	\sum_{{\bf x}}{\bf c}_{\bf x}^{\dagger}{\mathcal M}_0{\bf c}_{\bf x}
	+\left(\sum_{{\bf x},j=x,y}{\bf c}_{\bf x}^{\dagger}{\mathcal T}_{{\bf x},j}{\bf c}_{{\bf x}+{\bf \hat{e}}_j}+\text{H.c.}\right)
	+
	\left(
	\sum_{{\bf x},j=z}
	{\bf c}_{\bf x}^{\dagger}{\mathcal T}_{{\bf x},j}{\bf c}_{{\bf x}+{\bf \hat{e}}_j}e^{ik_za}
	+
	\text{H.c.}
	\right) + H_{g1}+H_{g2}
\end{eqnarray}
where ${\mathcal M}_0=\left(M_0-6B/a^2\right)\sigma_z{\otimes}s_0$, and ${\mathcal T}_{{\bf x},j}=(B/a^2\sigma_z{\otimes}s_0-iA/(2a)\sigma_x{\otimes}s_j)e^{i(e/\hbar)a A(x,y)\delta_{z,j}}$.
The magnetic field ${\bf B}$ is included by doing the Peierls
substitution \cite{Hofstadter1976} as a phase of the hopping term.
$H_{g1}=\sum_{{\bf x} \in \text{left and front surfaces}}{\bf c}_{\bf x}^{\dagger}{\mathcal M}_{g1}{\bf c}_{\bf x}$ with ${\mathcal M}_{g1}=eV_{g1}\sigma_0{\otimes}s_0$ and $H_{g2}=\sum_{{\bf x} \in \text{right and back surfaces}}{\bf c}_{\bf x}^{\dagger}{\mathcal M}_{g2}{\bf c}_{\bf x}$ with ${\mathcal M}_{g2}=eV_{g2}\sigma_0{\otimes}s_0$.
Here, $a=1$ is the cubic lattice constant.
${\bf c}_{\bf x}$ (${\bf c}_{\bf x}^{\dagger}$) is the annihilation (creation) operator at site ${\bf x}=(x,y)$, and ${\bf \hat{e}}_j$ is a unit vector in the $j$ direction for $j=x,y,z$.
$\delta_{z,j}=1$ for $j=z$ and is zero for $j=x,y$.
By exactly diagonalizing $H(k_z)$, we obtain the band structure as showed in Fig.~\ref{figs5}(c).
This demonstrates that the side surfaces are insulating and the top and bottom surfaces are spatially separated by the insulating side surfaces when the Fermi energy satisfies the relation that $t_{d_1}=-t_{d_2}$. 




\bibliographystyle{apsrev4-1}
\bibliography{suppref.bib}
\end{document}

