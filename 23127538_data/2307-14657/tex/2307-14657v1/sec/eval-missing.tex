%% Figure environment removed

\subsection{Impact of Missing Dynamic Feature Values}
\label{sec:eval-missing}

Some possible explanations for the worse results of dynamic features 
compared to static features %in Section~\ref{sec:classification-results} 
are that a sandbox may fail to stimulate samples adequately 
to cause them to `detonate`, or that samples may not work properly due 
to missing local or remote components. 
As a result, the classifier might need to take a decision based on a 
partial view of the malware runtime behavior. 

% In this section, we examines this issue
% by analyzing the impact of missing dynamic feature values for 
% both binary and family classification. 
%
% In Section~\ref{} we show that the families with the lowest binary and 
% family classification accuracy using dynamic features tend to have 
% large feature missing rate. 
%To capture this aspect we define the \emph{feature missing rate} (FMR) of a malware family 
%as the fraction of family samples that have missing values in the file, registry, service, and process features
% In the dataset of the study, these 4 feature classes are those the most frequently containing missing feature values.  
%(which, among the seven dynamic features classes we consider, are the most relevant for classification according to Table~\ref{tbl:binary_and_multiclass_FeatImportance}). 
%Missing values over \text{all} these four feature classes considerably
%degrades both the amount and quality of useful information available to the classifier.

% Figure~\ref{fig:f1vsfmr} in the appendix compares the family-wise F1 scores in the family classification 
% for each malware family with its FMR. 
% The figure shows that a lower FMR tends to produce higher F1 scores and 
% vice versa. 

We computed the Pearson correlation coefficient between the family-wise recall
of binary classification and the FMR to study the link between the two. Interestingly, 
the correlation is not statistically significant for the binary
classification task (pearson -0.1 and p-value 0.11). However, there is a
clear negative correlation (-0.43, p-value of $7.61*10^{-16}$) for the
family classification task. In this case, as the fraction of samples with
missing feature values for a family increases, its classification accuracy
decreases. 
This is also confirmed by looking at the malware families that are the most difficult
to classify with dynamic features, i.e., those for which the classifier
has the lower accuracy (see Tables~\ref{tbl:binary_dyn_bestAndWorst} and~\ref{tbl:multiclass_dyn_bestAndWorst} in Section.\ref{sec:best_and_worst}). Among the top-10 all have an FMR $>$ 65\%.

This outcome demonstrates that the ML classifier
might still be able to identify signs of malicious behavior in incomplete
dynamic analysis reports, but more feature values are needed to precisely
distinguish among different families (in particular for those, like
downloaders, that might have similar behavioral profiles).
In addition, binary classification is also affected by the quality of the 
behaviors collected from benign samples, while
family classification accuracy is solely 
associated with the feature completeness of malware samples in each family.

% Therefore, in the family classification task, missing feature values in
% malware samples shows more direct impact over the family-wise
% classification performance. 
% This confirms our initial guess: \textit{increasingly more miss feature
% values obtained from dynamic analysis brings less reliable classification
% output, which in turn causes the loss of classification accuracy}. 

% %
% Missing feature values is thus a likely explanation for the 
% lower family classification accuracy of dynamic features 
% compared to static features since static features rarely have such an issue. 

%For the binary classification task, 
%the correlation score between the family F1 score and its FMR is 
%-0.10 with a p value of 0.11. 
%Thus, there is still an inverse correlation between both variables, 
%but it is not statistically significant in this case.

\summary{2}{
Globally, a statistically significant inverse correlation in
the family classification task between the family-wise classification
accuracy using dynamic features and the amount of missing dynamic feature
values exist. 
The correlation is instead not significant for the binary classification task.
}

%\paragraph{The impact of missing dynamic analysis-based features}
%\paragraph{Correlation between dynamic and static accuracy}
%\sav{Not sure it can be useful. Pearson correlation of the static accuracy vs
%dynamic accuracy computed on the malware classes (and overall). The number of
%families for the overall case is 666 (and not 670) because some families with a
%lot of missing values for dynamic were excluded}
%All classes Pearson CC 0.65 - families 666\\
%grayware Pearson CC 0.68 - families 243\\
%clicker Pearson CC 0.8 - families 5\\
%worm Pearson CC 0.43 - families 87\\
%downloader Pearson CC 0.62 - families 118\\
%miner Pearson CC 0.73 - families 9\\
%dialer Pearson CC -1.0 - families 2\\
%virus Pearson CC 0.72 - families 40\\
%ransomware Pearson CC 0.42 - families 24\\
%tool Pearson CC 0.99 - families 5\\
%adware Pearson CC 0.5 - families 36\\
%rogueware Pearson CC 1.0 - families 2\\
%spyware Pearson CC 0.84 - families 17\\
%backdoor Pearson CC 0.8 - families 78\\

