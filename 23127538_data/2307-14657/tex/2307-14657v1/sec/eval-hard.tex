\subsection{Hard-to-Detect Malware}
\label{sec:best_and_worst} 

\begin{table}[t]
\setlength\tabcolsep{3pt}
\footnotesize
\centering
	\caption{Classification accuracy for malware classes.}
\label{tbl:overallResultsGrouped}
\begin{tabular}{l|ccc|ccc}
\toprule
	\multirow{2}{*}{\textbf{Class}} &  \multicolumn{3}{c|}{\textbf{Binary
	class. Recall}} & \multicolumn{3}{c}{\textbf{Family class. F1 score}} \\
	&  \textbf{Static} & \textbf{Dyn.} & \textbf{Comb.} & \textbf{Static}
	& \textbf{Dyn.} & \textbf{Com.} \\
\midrule
Adware     &	0.905	& 0.915	& 0.981 &   0.926 &	0.761	&	0.925	\\
Backdoor   &	0.966	& 0.943	& 0.996 &   0.830 &	0.730	&	0.838	\\
Clicker    &	0.971	& 0.929 & 1.000 &   0.817 &	0.692	&	0.821	\\
Dialer     &	0.994	& 0.875	& 1.000 &   0.988 &	0.888	&	0.984	\\
Downloader &	0.974	& 0.899	& 0.996 &   0.864 &	0.695	&	0.874	\\
Grayware   &	0.932	& 0.895	& 0.986 &   0.832 &	0.675	&	0.852	\\
Miner      &	0.989	& 0.972	& 0.999 &   0.927 &	0.807	&	0.962	\\
Ransomware &	0.967	& 0.945	& 0.997 &   0.839 &	0.580	&	0.853	\\
Rogueware  &	0.984	& 1.000	& 0.992 &   0.616 &	0.401	&	0.663	\\
Spyware    &	0.972	& 0.829	& 0.998 &   0.869 &	0.704	&	0.879	\\
Tool       &	0.992	& 0.929	& 1.000 &   0.864 &	0.778	&	0.830	\\
Virus      &	0.885	& 0.939	& 0.971 &   0.819 &	0.719	&	0.809	\\
Worm       &	0.978	& 0.899	& 0.996 &   0.922 &	0.721	&	0.921	\\

\midrule
\textbf{Average}       &	\textbf{0.967} &	 \textbf{0.920}  &	\textbf{0.9907} &      \textbf{0.848} & \textbf{0.704} &	\textbf{0.865} \\
\bottomrule
\end{tabular}
\end{table}

\captionsetup[subtable]{justification=centering}
\begin{table*}[!htb]
	\centering
	\setlength{\tabcolsep}{1.4pt}
    \caption{Top-10 malware families with the lowest classification accuracy}
	\label{tbl:bestAndWorst_altogether}
    \begin{subtable}{.21\textwidth}
      \centering
		\footnotesize
		\caption{Binary - static}
		\label{tbl:binary_static_bestAndWorst}
		\begin{tabular}{llcc}
		\toprule
		%\multicolumn{4}{c}{\textbf{Static binary classification}}\\
			\multirow{2}{*}{\textbf{Family}} & \multirow{2}{*}{\textbf{Class}} &  \textbf{Avg} &  \textbf{\%} \\
			&&  \textbf{Recall} &  \textbf{packed} \\
		\midrule
		pioneer             &       virus &         0.40 &       6\% \\
		asparnet            &    grayware &         0.41 &       5\% \\
		systweak            &    grayware &         0.46 &      19\% \\
		shopper             &    grayware &         0.50 &       1\% \\
		sality              &       virus &         0.52 &       4\% \\
		vitro               &       virus &         0.55 &       3\% \\
		installcore         &    grayware &         0.60 &      10\% \\
		slugin              &       virus &         0.60 &       4\% \\
		elex                &      adware &         0.60 &       9\% \\
		passview            &    grayware &         0.62 &      35\% \\
		\bottomrule
		\end{tabular}
    \end{subtable}
    \begin{subtable}{.21\textwidth}
      \centering
		\footnotesize
		\caption{Family - static}
		\label{tbl:multiclass_static_bestAndWorst}
		\begin{tabular}{llcc}
		\toprule
		%\multicolumn{4}{c}{\textbf{Static family classification}}\\
			\multirow{2}{*}{\textbf{Family}} & \multirow{2}{*}{\textbf{Class}} &  \textbf{Avg} &  \textbf{\%} \\
			&&  \textbf{F1} &  \textbf{packed} \\
		\midrule
		zpevdo              &    grayware &         0.15 &      15\%  \\
		vitro               &       virus &         0.24 &      3\% \\
		uwamson             &    grayware &         0.25 &      15\%  \\
		gendal              &    grayware &         0.28 &      62\%  \\
		dumpex              &    grayware &         0.29 &      40\%  \\
		alman               &       virus &         0.29 &      11\%  \\
		sality              &       virus &         0.33 &      4\% \\
		pasta               &    grayware &         0.34 &      28\%  \\
		cobra               &    grayware &         0.38 &      60\%  \\
		copidmbe            &       virus &         0.39 &      9\% \\
		\bottomrule
		\end{tabular}
    \end{subtable} 
    \begin{subtable}{.27\textwidth}
		\centering
		\footnotesize
		\caption{Binary - dynamic}
		\label{tbl:binary_dyn_bestAndWorst}
		\begin{tabular}{llccc}
		\toprule
		%\multicolumn{5}{c}{\textbf{Dynamic binary classification}}\\
			\multirow{2}{*}{\textbf{Family}} & \multirow{2}{*}{\textbf{Class}} & \textbf{Avg} &  \textbf{\%} & \multirow{2}{*}{\textbf{FMR}} \\
			&&  \textbf{Recall} &  \textbf{packed} & \\
		\midrule
		tasker   		&       grayware	   	&  	0.0 	& 11\%  & 0.77	\\
		malex           &       downloader		&   0.0		& 1\%	& 0.77 	\\
		rostpay      	&       grayware    	&   0.0		& 96\%	& 0.76	\\
		constructor     &       grayware      	&   0.0		& 13\%	& 0.78	\\
		atcpa           &       virus     		&   0.0		& 0\%	& 0.78	\\
		mocrt          	&       spyware  		&  	0.0		& 73\%	& 0.80	\\
		mokes           &       backdoor  		&   0.0		& 1\%	& 0.65	\\
		bingoml         &       grayware  		&   0.0		& 22\%	& 0.72  \\
		safebytes       &       grayware  		&   0.0		& 99\%	& 0.81	\\
		trymedia        &       adware  		&   0.0		& 73\%	& 0.70	\\
		\bottomrule
		\end{tabular}
    \end{subtable}
    \begin{subtable}{.27\textwidth}
		\centering
		\footnotesize
		\caption{Family - dynamic}
		\label{tbl:multiclass_dyn_bestAndWorst}
		\begin{tabular}{llccc}
		\toprule
		%\multicolumn{5}{c}{\textbf{Dynamic family classification}}\\
			\multirow{2}{*}{\textbf{Family}} & \multirow{2}{*}{\textbf{Class}} & \textbf{Avg} &  \textbf{\%} & \multirow{2}{*}{\textbf{FMR}} \\
			&&  \textbf{F1} &  \textbf{packed} & \\

		\midrule
		bancos      	&  spyware      &   0.0  	& 44\%	& 0.76	\\
		kovter        	&  grayware     &   0.0 	& 0\%	& 0.78	\\
		safebytes     	&  grayware     &   0.0 	& 99\%	& 0.80	\\
		winner      	&  grayware     &   0.0 	& 0\%	& 0.80	\\
		umbra         	&  downloader   &   0.0 	& 0\%	& 0.80	\\
		ulise        	&  grayware     &   0.0 	& 2\%	& 0.80	\\
		contenedor      &  virus    	&   0.0  	& 0\% 	& 0.80  \\
		cobra     		&  grayware 	&   0.0 	& 60\% 	& 0.79	\\
		kuaizip         &  adware   	&   0.0 	& 1\%	& 0.80	\\
		zpevdo          &  grayware 	&   0.0 	& 15\%	& 0.77	\\	
		\bottomrule
		\end{tabular}
    \end{subtable} 
\end{table*}


This section analyzes which malware classes and families pose
a greater challenge for classifiers based on static and dynamic features.
\revision{Note that our multi-class classification models are for families. 
We only use here the coarser malware class 
(e.g., virus, worm) to draw conclusions on similar families.}

Table~\ref{tbl:overallResultsGrouped} shows 
{Recall} and F1-scores for each malware class in binary and family classification respectively. 
In binary classification, the recall value is defined as the number of correctly classified samples in the class
over the total number of samples in the class.
% \footnote{
% Classes are identified by following the tags provided by AVClass2~\cite{avclass2}, which
% categorises malware into 13 classes: adware, backdoor, clicker, dialer,
% downloader, grayware, miner, ransomware, rogueware, spyware, tool, virus, and
% worm.}
The numbers differ from those in Table~\ref{tbl:overallResults} because 
Table~\ref{tbl:overallResultsGrouped} only considers the 
classification results of malware samples, 
while Table~\ref{tbl:overallResults} covers the classification of both 
goodware and malware samples (thus taking also false positives into account).

As we can see, the recall and F1 score are not uniform across all classes 
and can widely vary depending on the task and the features used. 
Static features are considerably better at detecting
downloaders, dialers, and worms. In contrast, dynamic features perform better on rogueware, miner, and ransomware. 
%While dynamic features performs better on viruses.

These results are confirmed also if we look at individual families. We show
in Table~\ref{tbl:bestAndWorst_altogether} the 10 families with the lowest accuracy in both classification tasks using static and dynamic features. 
For instance, among the 10 malware families for which the static classifier
makes more mistakes, we count four 
\emph{viruses} (i.e., file infectors) and six \emph{grayware}.
This is even more remarkable if we consider the fact that there are only
40 families of Viruses in our entire dataset.
The fact that \emph{viruses} typically append their code to benign files 
results in a wide variation in terms of static features among samples of the same family, 
and this can explain why it is hard for a 
static classifier to differentiate them from \emph{goodware} and from other
families. 
Similarly, \emph{grayware} is defined as undesirable code, which
is not outright malicious per se, therefore making it difficult to find a 
clear boundary to isolate these families. In the worst 10 families using dynamic features,  
we can observe a similar pattern: grayware and viruses dominate the list. 
Besides, adware and spyware are also among the worst families. Malware samples in each of the classes have similar behaviors. 



% For example, for the binary static model 
% (where family labels are not used in the training) 
% the lowest classification recall happens for 
% viruses (0.885), adware (0.905), and grayware (0.932), 
% where adware is a subclass of grayware.
% In addition, of the 10 famillies with lowest detection accuracy 
% using the binary static model, 6 are grayware and 4 are viruses
% (even if only 5\% of the families in $M_B$ are viruses).

% \begin{table}[h]
\centering
\small
\caption{Top-10 malware families with the lowest binary classification accuracy using the
	static features (i.e., highest mispredictions as goodware).}
\label{tbl:binary_static_bestAndWorst}
\begin{tabular}{llcr}
\toprule
\multicolumn{4}{c}{\textbf{Static binary classification}}\\
	\textbf{Family} & \textbf{Class} &  \textbf{Avg Recall} &  \textbf{\% packed} \\
\midrule
pioneer             &       virus &         0.401 &       6\% \\
asparnet            &    grayware &         0.410 &       5\% \\
systweak            &    grayware &         0.458 &      19\% \\
shopper             &    grayware &         0.500 &       1\% \\
sality              &       virus &         0.516 &       4\% \\
vitro               &       virus &         0.553 &       3\% \\
installcore         &    grayware &         0.596 &      10\% \\
slugin              &       virus &         0.598 &       4\% \\
elex                &      adware &         0.603 &       9\% \\
passview            &    grayware &         0.617 &      35\% \\
%conduit             &    grayware &         0.627 &      35\% \\
%zpevdo              &    grayware &         0.636 &      15\% \\
%copidmbe            &       virus &         0.646 &       9\% \\
%wajam               &      adware &         0.650 &       2\% \\
%iobit               &    grayware &         0.654 &      38\% \\
%mywebsearch         &    grayware &         0.656 &       1\% \\
%wlksm               &       virus &         0.667 &       1\% \\
%toptools            &    grayware &         0.703 &       0\% \\
%kraddare            &      adware &         0.706 &      36\% \\
%tenga               &       virus &         0.716 &       6\% \\
%expiro              &       virus &         0.722 &      10\% \\
%opensupdater        &      adware &         0.724 &      14\% \\
%chir                &        worm &         0.737 &      12\% \\
%driverpack          &    grayware &         0.748 &      13\% \\
%parite              &       virus &         0.752 &       7\% \\
%kuaiba              &      adware &         0.753 &       5\% \\
%softcnapp           &      adware &         0.758 &       0\% \\
%uwamson             &    grayware &         0.758 &      15\% \\
%citeary             &        worm &         0.759 &      16\% \\
%browsefox           &    grayware &         0.759 &       6\% \\
%presenoker          &    grayware &         0.761 &      16\% \\
%deceptpcclean       &    grayware &         0.766 &       4\% \\
%diztakun            &    grayware &         0.770 &      27\% \\
%diskwriter          &    grayware &         0.781 &      29\% \\
%petya               &  ransomware &         0.782 &      16\% \\
%gamevance           &      adware &         0.786 &       5\% \\
%onescan             &    grayware &         0.787 &      38\% \\
%dupzom              &  downloader &         0.795 &      14\% \\
%pcclient            &    backdoor &         0.796 &      30\% \\
%vrbrothers          &      adware &         0.797 &      88\% \\
%rozena              &    backdoor &         0.802 &       3\% \\
%expressdownloader   &    grayware &         0.802 &      62\% \\
%schoolboy           &    grayware &         0.807 &       7\% \\
%rbot                &        worm &         0.812 &      19\% \\
%alman               &       virus &         0.814 &      11\% \\
%geral               &  downloader &         0.815 &      23\% \\
%tibia               &    grayware &         0.816 &      31\% \\
%mediaget            &    grayware &         0.821 &      33\% \\
%polypatch           &    grayware &         0.821 &      99\% \\
%bulz                &    grayware &         0.821 &      15\% \\
%buzy                &        worm &         0.828 &      35\% \\
%burden              &      adware &         0.832 &       2\% \\
%zurgop              &  downloader &         0.832 &       3\% \\
%filefinder          &      adware &         0.833 &       3\% \\
%speedingupmypc      &    grayware &         0.836 &      11\% \\
%zvuzona             &    grayware &         0.840 &      79\% \\
%sogou               &    grayware &         0.840 &      17\% \\
%relevantknowledge   &      adware &         0.841 &       0\% \\
%downloadadmin       &    grayware &         0.842 &       3\% \\
%loadmoney           &    grayware &         0.846 &       8\% \\
%induc               &       virus &         0.851 &      56\% \\
%bestafera           &    grayware &         0.851 &      57\% \\
%mudrop              &  downloader &         0.852 &      37\% \\
%virut               &       virus &         0.852 &       3\% \\
%bandoo              &    grayware &         0.854 &       0\% \\
%kingsoft            &    grayware &         0.854 &      37\% \\
%ymacco              &    grayware &         0.854 &      68\% \\
%mokes               &    backdoor &         0.855 &       1\% \\
%ibryte              &      adware &         0.857 &       9\% \\
%ipamor              &    grayware &         0.859 &      39\% \\
%emotet              &  downloader &         0.860 &       1\% \\
%installbrain        &    grayware &         0.860 &       2\% \\
%techsnab            &    grayware &         0.862 &       2\% \\
%alien               &  downloader &         0.862 &      20\% \\
%qihoo               &    grayware &         0.863 &       3\% \\
%pcacceleratepro     &    grayware &         0.864 &       1\% \\
%auslogics           &    grayware &         0.865 &       3\% \\
%convagent           &    grayware &         0.868 &      23\% \\
%triusor             &       virus &         0.869 &       0\% \\
%siscos              &  downloader &         0.871 &      48\% \\
%ekstak              &    grayware &         0.872 &      31\% \\
%clipbanker          &    grayware &         0.873 &      11\% \\
%adgazelle           &    grayware &         0.873 &       0\% \\
%crossrider          &    grayware &         0.874 &       1\% \\
%lotok               &    backdoor &         0.875 &      15\% \\
%menti               &        worm &         0.876 &      11\% \\
%skillis             &    grayware &         0.877 &      19\% \\
%fakefire            &        worm &         0.878 &       3\% \\
%bancos              &     spyware &         0.879 &      44\% \\
%blackie             &    grayware &         0.879 &       0\% \\
%kiser               &    grayware &         0.880 &      55\% \\
%amonetize           &    grayware &         0.887 &       7\% \\
%killmbr             &    grayware &         0.887 &       9\% \\
%constructor         &    grayware &         0.887 &      13\% \\
%netwiredrc          &    backdoor &         0.888 &      24\% \\
%cobra               &    grayware &         0.888 &      60\% \\
%mabezat             &        worm &         0.890 &       0\% \\
%remoteutilities     &    grayware &         0.893 &      90\% \\
%reline              &     spyware &         0.893 &      57\% \\
%smartpcsolutions    &    grayware &         0.894 &      46\% \\
%nanocore            &    backdoor &         0.897 &      22\% \\
%crysis              &  ransomware &         0.897 &      27\% \\
%jatif               &    grayware &         0.898 &       2\% \\
%anmalpro            &    grayware &         0.898 &       0\% \\
%bladabindi          &    backdoor &         0.898 &      19\% \\
%neshta              &       virus &         0.900 &       2\% \\
%rising              &    grayware &         0.901 &      34\% \\
%buterat             &     clicker &         0.901 &      17\% \\
%waldek              &    grayware &         0.902 &       0\% \\
%ramnit              &       virus &         0.903 &      50\% \\
%fujacks             &       virus &         0.904 &      35\% \\
%cryptowall          &  ransomware &         0.904 &       2\% \\
%lazy                &    grayware &         0.904 &       6\% \\
%chifrax             &    grayware &         0.905 &      41\% \\
%airinstaller        &    grayware &         0.905 &      87\% \\
%buzus               &  downloader &         0.906 &       3\% \\
%zlob                &  downloader &         0.907 &      21\% \\
%pterodo             &  downloader &         0.908 &       0\% \\
%shutdowner          &    grayware &         0.908 &       6\% \\
%speedbit            &    grayware &         0.909 &       8\% \\
%usteal              &    grayware &         0.911 &      31\% \\
%sdum                &    grayware &         0.911 &      24\% \\
%xyligan             &    backdoor &         0.912 &      55\% \\
%installiq           &    grayware &         0.914 &       2\% \\
%bingoml             &    grayware &         0.914 &      22\% \\
%utorrent            &    grayware &         0.914 &      72\% \\
%bifrose             &    backdoor &         0.916 &      33\% \\
%cjishu              &      adware &         0.917 &       1\% \\
%lulusoftware        &    grayware &         0.918 &      13\% \\
%sabsik              &    grayware &         0.918 &      30\% \\
%gamini              &    grayware &         0.918 &       0\% \\
%dorgam              &  downloader &         0.919 &      19\% \\
%imali               &      adware &         0.919 &       0\% \\
%bsymem              &    grayware &         0.920 &      25\% \\
%fugrafa             &    grayware &         0.921 &       8\% \\
%mucc                &    grayware &         0.923 &       0\% \\
%pornoblocker        &  ransomware &         0.924 &      14\% \\
%winner              &    grayware &         0.924 &       0\% \\
%spigot              &    grayware &         0.924 &      20\% \\
%binder              &  downloader &         0.925 &      16\% \\
%gorillaprice        &      adware &         0.925 &       6\% \\
%scrop               &  downloader &         0.925 &      12\% \\
%babar               &    grayware &         0.926 &       5\% \\
%coins               &    grayware &         0.926 &      18\% \\
%llac                &        worm &         0.929 &      30\% \\
%plugx               &    backdoor &         0.929 &       5\% \\
%macri               &    backdoor &         0.929 &      58\% \\
%dealply             &    grayware &         0.930 &      19\% \\
%agentb              &    grayware &         0.931 &      13\% \\
%sodinokibi          &    grayware &         0.932 &       2\% \\
%veil                &    grayware &         0.932 &       1\% \\
%slimware            &    grayware &         0.933 &       0\% \\
%pirminay            &  downloader &         0.933 &       9\% \\
%hpdefender          &    grayware &         0.935 &       0\% \\
%winwrapper          &    grayware &         0.936 &      44\% \\
%shade               &  ransomware &         0.936 &       5\% \\
%neutrinopos         &  ransomware &         0.937 &      12\% \\
%adpack              &      adware &         0.937 &       2\% \\
%noon                &     spyware &         0.937 &      19\% \\
%firseria            &    grayware &         0.937 &       8\% \\
%banbra              &    backdoor &         0.937 &      50\% \\
%jacard              &    grayware &         0.937 &      34\% \\
%dumpex              &    grayware &         0.937 &      40\% \\
%titirez             &    grayware &         0.938 &       5\% \\
%poison              &    backdoor &         0.938 &      24\% \\
%taranis             &    grayware &         0.938 &       4\% \\
%xorer               &       virus &         0.938 &      11\% \\
%oficla              &    backdoor &         0.938 &      36\% \\
%palevo              &    backdoor &         0.938 &      22\% \\
%emotetu             &  downloader &         0.938 &       5\% \\
%openinstall         &    grayware &         0.939 &      22\% \\
%trickbot            &       miner &         0.939 &       0\% \\
%webalta             &    grayware &         0.942 &       5\% \\
%kelihos             &    backdoor &         0.942 &      12\% \\
%cossta              &    grayware &         0.943 &      13\% \\
%bayrob              &     spyware &         0.943 &       1\% \\
%linkury             &    grayware &         0.943 &       0\% \\
%cryptinject         &  downloader &         0.944 &      34\% \\
%quasar              &    backdoor &         0.944 &      48\% \\
%zbot                &  downloader &         0.945 &      24\% \\
%zard                &    grayware &         0.945 &       3\% \\
%swisyn              &        worm &         0.946 &       7\% \\
%pasta               &    grayware &         0.948 &      28\% \\
%cerbu               &    grayware &         0.948 &       4\% \\
%gendal              &    grayware &         0.949 &      62\% \\
%brresmon            &    grayware &         0.949 &       1\% \\
%downloadassistant   &    grayware &         0.949 &      18\% \\
%rostpay             &    grayware &         0.949 &      96\% \\
%youxun              &    grayware &         0.950 &      11\% \\
%nymaim              &    backdoor &         0.950 &       1\% \\
%zegost              &    backdoor &         0.950 &      28\% \\
%ldpinch             &        worm &         0.950 &      47\% \\
%orcus               &    backdoor &         0.950 &      42\% \\
%wonton              &  downloader &         0.950 &       2\% \\
%tisandr             &        worm &         0.951 &       0\% \\
%farfli              &    backdoor &         0.951 &      38\% \\
%fragtor             &    grayware &         0.951 &      10\% \\
%flystudio           &    grayware &         0.951 &      34\% \\
%sysn                &  downloader &         0.951 &      19\% \\
%soulclose           &        worm &         0.951 &       0\% \\
%fosniw              &  downloader &         0.951 &      42\% \\
%onlinegames         &    grayware &         0.951 &      14\% \\
%midie               &    grayware &         0.952 &      33\% \\
%dstudio             &    grayware &         0.952 &       2\% \\
%prorat              &    backdoor &         0.955 &      69\% \\
%vkontaktedj         &    grayware &         0.955 &       4\% \\
%webprefix           &  downloader &         0.955 &      12\% \\
%zapchast            &  downloader &         0.955 &       4\% \\
%lovgate             &        worm &         0.955 &      29\% \\
%bandit              &  downloader &         0.956 &      61\% \\
%ksdler              &      adware &         0.956 &      69\% \\
%tnega               &  downloader &         0.956 &      33\% \\
%hotbar              &    grayware &         0.956 &      45\% \\
%mint                &    grayware &         0.956 &       3\% \\
%dynamer             &  downloader &         0.956 &      20\% \\
%carberp             &  downloader &         0.956 &       6\% \\
%gamarue             &  downloader &         0.956 &      10\% \\
%viking              &       virus &         0.956 &      98\% \\
%occamy              &    grayware &         0.956 &      26\% \\
%remcos              &    backdoor &         0.957 &       4\% \\
%ardamax             &    grayware &         0.957 &       1\% \\
%gimemo              &  ransomware &         0.957 &      54\% \\
%getnow              &    grayware &         0.957 &      88\% \\
%silentall           &    grayware &         0.957 &      77\% \\
%ursu                &    grayware &         0.957 &      23\% \\
%jaik                &    grayware &         0.957 &       5\% \\
%yoddos              &    backdoor &         0.961 &      19\% \\
%icloader            &    grayware &         0.961 &       2\% \\
%koutodoor           &    backdoor &         0.962 &      16\% \\
%installerex         &    grayware &         0.962 &       0\% \\
%avemaria            &     spyware &         0.962 &      12\% \\
%ursnif              &       virus &         0.962 &      25\% \\
%redcap              &  downloader &         0.962 &      18\% \\
%yandex              &    grayware &         0.962 &       0\% \\
%cosmicduke          &    backdoor &         0.962 &       4\% \\
%primecasino         &    grayware &         0.962 &       0\% \\
%klone               &  downloader &         0.962 &      70\% \\
%kuaizip             &      adware &         0.962 &       1\% \\
%sinau               &       virus &         0.962 &       0\% \\
%softonic            &      adware &         0.963 &      79\% \\
%emogen              &  downloader &         0.963 &      31\% \\
%ruskill             &    backdoor &         0.963 &       7\% \\
%subseven            &    backdoor &         0.963 &      69\% \\
%aenjaris            &    grayware &         0.963 &       0\% \\
%explorerhijack      &  downloader &         0.963 &       1\% \\
%bitmin              &       miner &         0.963 &      25\% \\
%shandaadd           &      adware &         0.968 &      88\% \\
%offergenerator      &    grayware &         0.968 &       0\% \\
%jeefo               &       virus &         0.968 &      99\% \\
%refroso             &    backdoor &         0.968 &      41\% \\
%stelega             &    grayware &         0.968 &      50\% \\
%autinject           &        tool &         0.968 &       4\% \\
%convertad           &      adware &         0.968 &       1\% \\
%woool               &    grayware &         0.968 &      94\% \\
%schoolgirl          &    grayware &         0.968 &       2\% \\
%horst               &    backdoor &         0.969 &      30\% \\
%zusy                &    grayware &         0.969 &       7\% \\
%hupigon             &    backdoor &         0.969 &      59\% \\
%tremp               &    grayware &         0.969 &       0\% \\
%cridex              &    grayware &         0.969 &      15\% \\
%lethic              &  downloader &         0.969 &       1\% \\
%scarsi              &  downloader &         0.969 &       8\% \\
%lockbit             &  ransomware &         0.969 &       0\% \\
%gogo                &       virus &         0.969 &       0\% \\
%installmonster      &    grayware &         0.969 &      83\% \\
%khalesi             &    grayware &         0.969 &       2\% \\
%staser              &  downloader &         0.969 &      61\% \\
%frosparf            &     clicker &         0.969 &      73\% \\
%4shared             &    grayware &         0.969 &       1\% \\
%webcompanion        &    grayware &         0.970 &       0\% \\
%cobaltstrike        &  downloader &         0.970 &       3\% \\
%trickster           &  downloader &         0.970 &       1\% \\
%fraudrop            &  downloader &         0.974 &      31\% \\
%tasker              &    grayware &         0.974 &      11\% \\
%istartsurf          &    grayware &         0.975 &       1\% \\
%alyak               &  downloader &         0.975 &      55\% \\
%blamon              &    grayware &         0.975 &      33\% \\
%fsysna              &        worm &         0.975 &      66\% \\
%inbox               &    grayware &         0.975 &       0\% \\
%aauto               &    grayware &         0.975 &      56\% \\
%zeroaccess          &  downloader &         0.975 &      10\% \\
%atraps              &    grayware &         0.975 &      20\% \\
%dodiw               &    backdoor &         0.975 &      83\% \\
%autoitinject        &    backdoor &         0.975 &      16\% \\
%bloored             &        worm &         0.975 &       0\% \\
%tibs                &  downloader &         0.975 &      29\% \\
%detrahere           &    grayware &         0.975 &       4\% \\
%updane              &    grayware &         0.975 &       0\% \\
%darkkomet           &    backdoor &         0.975 &      14\% \\
%screenmate          &    grayware &         0.975 &       1\% \\
%predator            &    grayware &         0.975 &      13\% \\
%dlhelper            &    grayware &         0.975 &      13\% \\
%sohanad             &        worm &         0.976 &      66\% \\
%diss                &    grayware &         0.976 &       3\% \\
%fareit              &    grayware &         0.976 &      14\% \\
%hacdef              &    backdoor &         0.976 &      21\% \\
%teslacrypt          &  ransomware &         0.976 &       1\% \\
%shylock             &    backdoor &         0.981 &       3\% \\
%antavmu             &        worm &         0.981 &       0\% \\
%vbkryjetor          &  downloader &         0.981 &       4\% \\
%nitol               &  downloader &         0.981 &       7\% \\
%fesber              &        worm &         0.981 &      33\% \\
%cycbot              &    backdoor &         0.981 &       0\% \\
%gippers             &  ransomware &         0.981 &       0\% \\
%wapomi              &       virus &         0.981 &       8\% \\
%ficker              &    grayware &         0.981 &      15\% \\
%winloadsda          &    grayware &         0.981 &      86\% \\
%qjwmonkey           &    grayware &         0.981 &      47\% \\
%cheatengine         &    grayware &         0.981 &       0\% \\
%opencandy           &    grayware &         0.981 &      92\% \\
%msposer             &  downloader &         0.981 &       4\% \\
%megasearch          &      adware &         0.981 &       0\% \\
%gozi                &     spyware &         0.981 &       1\% \\
%racealer            &    grayware &         0.981 &      32\% \\
%esfury              &        worm &         0.981 &       5\% \\
%plurox              &    backdoor &         0.981 &       0\% \\
%mytob               &        worm &         0.981 &      30\% \\
%lamer               &       virus &         0.981 &      43\% \\
%xtrat               &    backdoor &         0.981 &      58\% \\
%tinba               &  downloader &         0.981 &      15\% \\
%hematite            &       virus &         0.981 &       1\% \\
%fraudpack           &   rogueware &         0.981 &       3\% \\
%omaneat             &     spyware &         0.981 &       7\% \\
%grenam              &       virus &         0.981 &       0\% \\
%turkojan            &    backdoor &         0.981 &      16\% \\
%paneidix            &    backdoor &         0.981 &       0\% \\
%lightstone          &    backdoor &         0.981 &      17\% \\
%killwin             &    grayware &         0.981 &      58\% \\
%sillyfdc            &        worm &         0.981 &      55\% \\
%safebytes           &    grayware &         0.981 &      99\% \\
%hafen               &    grayware &         0.981 &       0\% \\
%bezigate            &    backdoor &         0.981 &      15\% \\
%benjamin            &        worm &         0.982 &      94\% \\
%agobot              &    backdoor &         0.982 &      32\% \\
%vebzenpak           &    grayware &         0.982 &       0\% \\
%swizzor             &    grayware &         0.982 &       6\% \\
%gamania             &  downloader &         0.982 &       5\% \\
%glupteba            &    grayware &         0.987 &       1\% \\
%spybot              &        worm &         0.987 &      15\% \\
%domaiq              &    grayware &         0.987 &      53\% \\
%strab               &  ransomware &         0.987 &      38\% \\
%tiggre              &    grayware &         0.987 &      88\% \\
%malex               &  downloader &         0.987 &       1\% \\
%softpulse           &    grayware &         0.987 &      34\% \\
%tufik               &       virus &         0.987 &       0\% \\
%tdss                &  downloader &         0.987 &       6\% \\
%dumpy               &        worm &         0.987 &       0\% \\
%vemply              &    grayware &         0.987 &      61\% \\
%azorult             &    grayware &         0.987 &      57\% \\
%arkeistealer        &    grayware &         0.987 &       6\% \\
%qbot                &  downloader &         0.987 &       0\% \\
%neconyd             &     clicker &         0.987 &       0\% \\
%solmyr              &     spyware &         0.988 &      58\% \\
%regsup              &  downloader &         0.988 &       1\% \\
%vilsel              &  downloader &         0.988 &      41\% \\
%fakefolder          &        worm &         0.988 &      71\% \\
%deyma               &  downloader &         0.988 &      21\% \\
%zylom               &  downloader &         0.988 &       0\% \\
%vittalia            &    grayware &         0.988 &       3\% \\
%nymeria             &    grayware &         0.988 &      32\% \\
%lydra               &     spyware &         0.988 &       5\% \\
%stop                &  ransomware &         0.988 &       0\% \\
%sdbot               &    backdoor &         0.988 &      92\% \\
%growtopia           &    grayware &         0.988 &      69\% \\
%spesr               &        worm &         0.988 &       0\% \\
%kolab               &    grayware &         0.988 &       1\% \\
%spyrix              &    grayware &         0.988 &      56\% \\
%miancha             &  downloader &         0.988 &      11\% \\
%snojan              &  downloader &         0.988 &       1\% \\
%sillyp2p            &        worm &         0.988 &       1\% \\
%duote               &      adware &         0.988 &      98\% \\
%rebhip              &        worm &         0.988 &      16\% \\
%dorifel             &  downloader &         0.988 &       5\% \\
%blackmoon           &    grayware &         0.988 &      74\% \\
%gatak               &    grayware &         0.988 &      11\% \\
%startsurf           &    grayware &         0.988 &       3\% \\
%lokibot             &    backdoor &         0.988 &      88\% \\
%webdialer           &      dialer &         0.988 &      86\% \\
%cmy3u               &    grayware &         0.988 &      98\% \\
%zenlod              &  downloader &         0.988 &      22\% \\
%winwebsec           &   rogueware &         0.988 &      21\% \\
%outbyte             &    grayware &         0.988 &       4\% \\
%schwarzesonne       &    backdoor &         0.988 &      48\% \\
%ranumbot            &    grayware &         0.988 &      28\% \\
%clipspy             &    grayware &         0.988 &       2\% \\
%spyeye              &     spyware &         0.988 &       9\% \\
%bredolab            &  downloader &         0.994 &      19\% \\
%mira                &        worm &         0.994 &      23\% \\
%elemental           &    grayware &         0.994 &       0\% \\
%guloader            &  downloader &         0.994 &       0\% \\
%memery              &       virus &         0.994 &       9\% \\
%macoute             &        worm &         0.994 &       0\% \\
%formbook            &     spyware &         0.994 &      37\% \\
%bundlore            &      adware &         0.994 &       2\% \\
%shyape              &  downloader &         0.994 &      39\% \\
%urelas              &    backdoor &         0.994 &      38\% \\
%wannacry            &  ransomware &         0.994 &       1\% \\
%stopcrypt           &  ransomware &         0.994 &       0\% \\
%betload             &    grayware &         0.994 &       3\% \\
%babonock            &        worm &         0.994 &       0\% \\
%apost               &  downloader &         0.994 &       1\% \\
%pincav              &    backdoor &         0.994 &      18\% \\
%cerber              &  ransomware &         0.994 &      16\% \\
%klez                &        worm &         0.994 &       1\% \\
%esaprof             &  downloader &         0.994 &      84\% \\
%dorv                &  downloader &         0.994 &       8\% \\
%mepaow              &       virus &         0.994 &      97\% \\
%esecn               &      adware &         0.994 &     100\% \\
%dalexis             &  downloader &         0.994 &       0\% \\
%ulise               &    grayware &         0.994 &       2\% \\
%bublik              &  downloader &         0.994 &      79\% \\
%cosmu               &       virus &         0.994 &       0\% \\
%daws                &  downloader &         0.994 &       2\% \\
%kuluoz              &  downloader &         0.994 &       4\% \\
%upatre              &  downloader &         0.994 &       4\% \\
%shodi               &       virus &         0.994 &       0\% \\
%blackhole           &    backdoor &         0.994 &      32\% \\
%pidgeon             &       virus &         0.994 &      16\% \\
%renos               &  downloader &         0.994 &      17\% \\
%speedcat            &    grayware &         0.994 &       0\% \\
%cutwail             &  downloader &         0.994 &      10\% \\
%winactivator        &        tool &         0.994 &       6\% \\
%simda               &    backdoor &         0.994 &      11\% \\
%ngrbot              &    backdoor &         0.994 &       0\% \\
%vybab               &        worm &         0.994 &       0\% \\
%necurs              &  downloader &         0.994 &       8\% \\
%garrun              &    grayware &         0.994 &       0\% \\
%bebloh              &    backdoor &         0.994 &       1\% \\
%coroxy              &    backdoor &         1.000 &       0\% \\
%lassorm             &        worm &         1.000 &       0\% \\
%gofot               &  downloader &         1.000 &      88\% \\
%kovter              &    grayware &         1.000 &       0\% \\
%goabeny             &    grayware &         1.000 &      75\% \\
%krol                &        worm &         1.000 &       0\% \\
%kronosbot           &    backdoor &         1.000 &       0\% \\
%laqma               &    grayware &         1.000 &      12\% \\
%gigex               &        worm &         1.000 &       0\% \\
%obit                &  downloader &         1.000 &       0\% \\
%gobot               &    backdoor &         1.000 &       4\% \\
%offend              &    grayware &         1.000 &      98\% \\
%doina               &    grayware &         1.000 &      50\% \\
%disttrack           &    grayware &         1.000 &       0\% \\
%dluca               &  downloader &         1.000 &      84\% \\
%icedid              &  downloader &         1.000 &       4\% \\
%refinka             &    grayware &         1.000 &       1\% \\
%dofoil              &  downloader &         1.000 &       6\% \\
%reconyc             &    grayware &         1.000 &      36\% \\
%recodrop            &  downloader &         1.000 &       0\% \\
%recex               &       miner &         1.000 &      11\% \\
%konus               &    backdoor &         1.000 &       0\% \\
%raccrypt            &  ransomware &         1.000 &       0\% \\
%qzonit              &     spyware &         1.000 &      87\% \\
%qshell              &    grayware &         1.000 &       0\% \\
%qqware              &    grayware &         1.000 &      50\% \\
%reptile             &    grayware &         1.000 &       0\% \\
%gify                &    grayware &         1.000 &       0\% \\
%houndhack           &    backdoor &         1.000 &     100\% \\
%lecna               &    backdoor &         1.000 &      24\% \\
%qqpass              &  downloader &         1.000 &      94\% \\
%lolbot              &        worm &         1.000 &       0\% \\
%locky               &  downloader &         1.000 &       5\% \\
%gator               &      adware &         1.000 &      47\% \\
%goldrv              &        worm &         1.000 &       0\% \\
%keydoor             &     spyware &         1.000 &       0\% \\
%griptolo            &        worm &         1.000 &       7\% \\
%kasidet             &    grayware &         1.000 &      19\% \\
%jpgiframe           &    grayware &         1.000 &       0\% \\
%johnnie             &    grayware &         1.000 &      65\% \\
%jimmy               &  ransomware &         1.000 &       0\% \\
%kolovorot           &    grayware &         1.000 &       3\% \\
%jevafus             &  downloader &         1.000 &      45\% \\
%havex               &    grayware &         1.000 &       0\% \\
%hebchengjiu         &      adware &         1.000 &       7\% \\
%hebogo              &    grayware &         1.000 &       0\% \\
%hider               &    grayware &         1.000 &      59\% \\
%inlog               &  downloader &         1.000 &       0\% \\
%injuke              &    grayware &         1.000 &      70\% \\
%itorrent            &    grayware &         1.000 &       0\% \\
%oberal              &  downloader &         1.000 &       0\% \\
%kolweb              &    grayware &         1.000 &      96\% \\
%koobface            &        worm &         1.000 &      77\% \\
%lmir                &       virus &         1.000 &       2\% \\
%geegly              &       virus &         1.000 &     100\% \\
%lipler              &  downloader &         1.000 &       0\% \\
%gepys               &  downloader &         1.000 &       0\% \\
%lebreat             &    grayware &         1.000 &     100\% \\
%rifdoor             &    backdoor &         1.000 &       0\% \\
%privateexeprotector &    grayware &         1.000 &      19\% \\
%dinwod              &  downloader &         1.000 &      98\% \\
%eyestye             &  downloader &         1.000 &      27\% \\
%eyoorun             &    grayware &         1.000 &       0\% \\
%ezsoftwareupdater   &    grayware &         1.000 &       0\% \\
%facido              &  downloader &         1.000 &       0\% \\
%fakedoc             &        worm &         1.000 &       0\% \\
%offercore           &    grayware &         1.000 &       0\% \\
%pahooka             &        worm &         1.000 &     100\% \\
%fakens              &      adware &         1.000 &       0\% \\
%drstwex             &  downloader &         1.000 &       0\% \\
%plemood             &        worm &         1.000 &       0\% \\
%protux              &    backdoor &         1.000 &       4\% \\
%dostre              &  downloader &         1.000 &       0\% \\
%pacex               &       miner &         1.000 &       0\% \\
%multiplier          &    grayware &         1.000 &       0\% \\
%downloadguide       &    grayware &         1.000 &       0\% \\
%prepscram           &    grayware &         1.000 &     100\% \\
%powerspider         &    backdoor &         1.000 &       0\% \\
%pistolar            &  downloader &         1.000 &      76\% \\
%powerless           &  downloader &         1.000 &      10\% \\
%pondfull            &        worm &         1.000 &       5\% \\
%dpsk                &       miner &         1.000 &     100\% \\
%drolnux             &        worm &         1.000 &       0\% \\
%pochi               &        worm &         1.000 &       0\% \\
%pluto               &        worm &         1.000 &      34\% \\
%dropware            &      adware &         1.000 &       1\% \\
%plingky             &  downloader &         1.000 &      93\% \\
%downer              &    grayware &         1.000 &      96\% \\
%pajetbin            &        worm &         1.000 &      91\% \\
%picsys              &        worm &         1.000 &     100\% \\
%enosch              &        worm &         1.000 &       0\% \\
%digs                &       miner &         1.000 &       0\% \\
%shiz                &    backdoor &         1.000 &       0\% \\
%shipup              &  downloader &         1.000 &      30\% \\
%cyfin               &    grayware &         1.000 &       5\% \\
%dadobra             &  downloader &         1.000 &      63\% \\
%sfone               &        worm &         1.000 &      65\% \\
%seven               &  ransomware &         1.000 &       0\% \\
%seraph              &  downloader &         1.000 &       0\% \\
%qqhack              &        tool &         1.000 &      98\% \\
%scrarev             &    grayware &         1.000 &       5\% \\
%salgorea            &  downloader &         1.000 &       0\% \\
%dexel               &  downloader &         1.000 &       0\% \\
%rungbu              &        worm &         1.000 &      18\% \\
%rums                &  downloader &         1.000 &       0\% \\
%ruledor             &    backdoor &         1.000 &       0\% \\
%diamin              &      dialer &         1.000 &      80\% \\
%roxer               &    grayware &         1.000 &      97\% \\
%detroie             &       virus &         1.000 &       2\% \\
%dorkbot             &        worm &         1.000 &       7\% \\
%qaccel              &     clicker &         1.000 &       0\% \\
%pykspa              &        worm &         1.000 &       0\% \\
%phorpiex            &        worm &         1.000 &      12\% \\
%phishbank           &        worm &         1.000 &       0\% \\
%perion              &    grayware &         1.000 &       0\% \\
%pedex               &    backdoor &         1.000 &     100\% \\
%eggnog              &        worm &         1.000 &       0\% \\
%egroupdial          &    grayware &         1.000 &      87\% \\
%rikihaki            &        worm &         1.000 &       0\% \\
%miniduke            &    backdoor &         1.000 &       0\% \\
%bcryptinject        &  downloader &         1.000 &       0\% \\
%mocrt               &     spyware &         1.000 &      73\% \\
%connectwise         &        tool &         1.000 &       0\% \\
%amigo               &    grayware &         1.000 &       0\% \\
%yahlover            &        worm &         1.000 &      65\% \\
%xorist              &  downloader &         1.000 &      24\% \\
%socelars            &     spyware &         1.000 &       1\% \\
%socks               &        worm &         1.000 &      18\% \\
%xolxo               &        worm &         1.000 &      99\% \\
%allaple             &        worm &         1.000 &       0\% \\
%xihet               &  downloader &         1.000 &       8\% \\
%wacatac             &    grayware &         1.000 &      14\% \\
%wabot               &    backdoor &         1.000 &      19\% \\
%berbew              &    backdoor &         1.000 &       0\% \\
%skintrim            &      adware &         1.000 &       2\% \\
%upantix             &  downloader &         1.000 &     100\% \\
%syncopate           &    grayware &         1.000 &      20\% \\
%unruy               &  downloader &         1.000 &      13\% \\
%umbra               &  downloader &         1.000 &       0\% \\
%tupym               &        worm &         1.000 &       0\% \\
%sytro               &        worm &         1.000 &      45\% \\
%bunitu              &    grayware &         1.000 &       0\% \\
%broskod             &  downloader &         1.000 &       1\% \\
%taskloader          &      adware &         1.000 &       0\% \\
%blackshades         &        worm &         1.000 &       2\% \\
%tougle              &    grayware &         1.000 &       0\% \\
%brook               &  ransomware &         1.000 &       0\% \\
%tempedreve          &       virus &         1.000 &       0\% \\
%swjoy               &    grayware &         1.000 &       0\% \\
%aitinject           &    backdoor &         1.000 &       1\% \\
%xiaobaminer         &       miner &         1.000 &      74\% \\
%agentino            &  downloader &         1.000 &       0\% \\
%wsgame              &    grayware &         1.000 &      49\% \\
%woozlist            &    grayware &         1.000 &      90\% \\
%ascentive           &    grayware &         1.000 &       0\% \\
%chrop               &      adware &         1.000 &      63\% \\
%atcpa               &       virus &         1.000 &       0\% \\
%autog               &  downloader &         1.000 &       0\% \\
%windigo             &    grayware &         1.000 &       3\% \\
%wenper              &        worm &         1.000 &      42\% \\
%sohana              &  downloader &         1.000 &      18\% \\
%wecod               &     spyware &         1.000 &      94\% \\
%axespec             &    grayware &         1.000 &       2\% \\
%ceatrg              &  downloader &         1.000 &      66\% \\
%catalina            &    grayware &         1.000 &       0\% \\
%wavipeg             &    backdoor &         1.000 &       0\% \\
%cardspy             &     spyware &         1.000 &     100\% \\
%wakme               &  downloader &         1.000 &       0\% \\
%xanfpezes           &    grayware &         1.000 &     100\% \\
%xcnfe               &    grayware &         1.000 &       0\% \\
%blihan              &    backdoor &         1.000 &       0\% \\
%cambot              &        worm &         1.000 &       0\% \\
%adaebook            &    grayware &         1.000 &     100\% \\
%contenedor          &       virus &         1.000 &       0\% \\
%addlyrics           &      adware &         1.000 &       0\% \\
%adposhel            &    grayware &         1.000 &       0\% \\
%zepfod              &        worm &         1.000 &       0\% \\
%agentc              &    grayware &         1.000 &       0\% \\
%agentcrypt          &    grayware &         1.000 &       0\% \\
%brontok             &        worm &         1.000 &      95\% \\
%moarider            &        worm &         1.000 &      98\% \\
%tonmye              &    grayware &         1.000 &      84\% \\
%trymedia            &      adware &         1.000 &      73\% \\
%zygug               &    grayware &         1.000 &      52\% \\
%gandcrab            &  ransomware &         1.000 &       3\% \\
%lyposit             &  ransomware &         1.000 &       0\% \\
%lunastorm           &        worm &         1.000 &      97\% \\
%lunam               &        worm &         1.000 &      99\% \\
%gandcrypt           &  ransomware &         1.000 &       2\% \\
%ludbaruma           &        worm &         1.000 &       0\% \\
%loring              &        worm &         1.000 &     100\% \\
%lolojan             &    grayware &         1.000 &      95\% \\
%gamemodding         &    grayware &         1.000 &       6\% \\
%mewsspy             &    backdoor &         1.000 &       6\% \\
%mimail              &        worm &         1.000 &       0\% \\
%maener              &       miner &         1.000 &       0\% \\
%mimdau              &  downloader &         1.000 &     100\% \\
%nevereg             &        worm &         1.000 &      51\% \\
%netstream           &    grayware &         1.000 &       0\% \\
%filetour            &    grayware &         1.000 &       4\% \\
%neoreklami          &      adware &         1.000 &       0\% \\
%neojit              &  downloader &         1.000 &       0\% \\
%finfish             &    backdoor &         1.000 &       0\% \\
%lollipop            &      adware &         1.000 &       0\% \\
%mydoom              &        worm &         1.000 &      83\% \\
%multibar            &    grayware &         1.000 &      23\% \\
%mulinex             &  downloader &         1.000 &      65\% \\
%mooqkel             &  downloader &         1.000 &     100\% \\
%moonlight           &        worm &         1.000 &       8\% \\
%fasong              &        worm &         1.000 &      78\% \\
%mailru              &    grayware &         1.000 &       0\% \\
%manbat              &        worm &         1.000 &      44\% \\
%mapsgory            &  downloader &         1.000 &       0\% \\
%bobax               &        worm &         1.000 &       3\% \\
%burn                &    backdoor &         1.000 &       0\% \\
%stormattack         &  downloader &         1.000 &       0\% \\
%vundo               &        worm &         1.000 &      89\% \\
%vtflooder           &  downloader &         1.000 &      55\% \\
%bactera             &        worm &         1.000 &       0\% \\
%voltar              &     clicker &         1.000 &       0\% \\
%vobfus              &        worm &         1.000 &       0\% \\
%stampado            &  ransomware &         1.000 &      58\% \\
%balrok              &        worm &         1.000 &       0\% \\
%banload             &       miner &         1.000 &      49\% \\
%vawtrak             &    backdoor &         1.000 &       0\% \\
%bzub                &  downloader &         1.000 &       2\% \\
%downloadsponsor     &    grayware &         1.000 &      87\% \\
%stone               &       virus &         1.000 &      97\% \\
%virlock             &       virus &         1.000 &       0\% \\
%virfire             &        worm &         1.000 &       0\% \\
%virbox              &    grayware &         1.000 &       1\% \\
%beastdoor           &    backdoor &         1.000 &      71\% \\
%vbinder             &        tool &         1.000 &       1\% \\
%vbclone             &  downloader &         1.000 &       0\% \\
%stihat              &        worm &         1.000 &     100\% \\
%sivis               &       virus &         1.000 &      72\% \\
%idsohtu             &    grayware &         1.000 &       6\% \\
%simbot              &    backdoor &         1.000 &       1\% \\
%gametea             &  downloader &         1.000 &     100\% \\
%mbrlock             &  downloader &         1.000 &       2\% \\
%tofsee              &    backdoor &         1.000 &       9\% \\
%regrun              &  ransomware &         1.000 &     100\% \\
\bottomrule
\end{tabular}
\end{table}

% \begin{table}[h]
\centering
\small
\caption{Top-10 families with the lowest family classification accuracy using
	static features (i.e., highest mispredictions to other families)}
\label{tbl:multiclass_static_bestAndWorst}
\begin{tabular}{llrr}
\toprule
\multicolumn{4}{c}{\textbf{Static family classification}}\\
\textbf{Family} & \textbf{Class} &  \textbf{Avg F1} & \textbf{\% packed} \\
\midrule
zpevdo              &    grayware &         0.150 &      15\%  \\
vitro               &       virus &         0.240 &      3\% \\
uwamson             &    grayware &         0.252 &      15\%  \\
gendal              &    grayware &         0.280 &      62\%  \\
dumpex              &    grayware &         0.290 &      40\%  \\
alman               &       virus &         0.293 &      11\%  \\
sality              &       virus &         0.328 &      4\% \\
pasta               &    grayware &         0.346 &      28\%  \\
cobra               &    grayware &         0.381 &      60\%  \\
copidmbe            &       virus &         0.387 &      9\% \\
%refroso             &    backdoor &         0.393 &      41\%  \\
%llac                &        worm &         0.402 &      30\%  \\
%bifrose             &    backdoor &         0.402 &      33\%  \\
%atraps              &    grayware &         0.409 &      0.20 \\
%pioneer             &       virus &         0.419 &      0.06 \\
%coins               &    grayware &         0.435 &      0.18 \\
%bancos              &     spyware &         0.449 &      0.44 \\
%convagent           &    grayware &         0.451 &      0.23 \\
%kiser               &    grayware &         0.453 &      0.55 \\
%buzy                &        worm &         0.459 &      0.35 \\
%bestafera           &    grayware &         0.483 &      0.57 \\
%ldpinch             &        worm &         0.489 &      0.47 \\
%menti               &        worm &         0.490 &      0.11 \\
%midie               &    grayware &         0.491 &      0.33 \\
%alien               &  downloader &         0.493 &      0.20 \\
%occamy              &    grayware &         0.498 &      0.26 \\
%bulz                &    grayware &         0.504 &      0.15 \\
%wsgame              &    grayware &         0.509 &      0.49 \\
%sabsik              &    grayware &         0.514 &      0.30 \\
%diskwriter          &    grayware &         0.515 &      0.29 \\
%tibia               &    grayware &         0.522 &      0.31 \\
%oficla              &    backdoor &         0.523 &      0.36 \\
%klone               &  downloader &         0.528 &      0.70 \\
%sdum                &    grayware &         0.529 &      0.24 \\
%ramnit              &       virus &         0.532 &      0.50 \\
%presenoker          &    grayware &         0.540 &      0.16 \\
%ruskill             &    backdoor &         0.544 &      0.07 \\
%diztakun            &    grayware &         0.545 &      0.27 \\
%bredolab            &  downloader &         0.545 &      0.19 \\
%jatif               &    grayware &         0.558 &      0.02 \\
%jacard              &    grayware &         0.562 &      0.34 \\
%noon                &     spyware &         0.570 &      0.19 \\
%scarsi              &  downloader &         0.579 &      0.08 \\
%clipbanker          &    grayware &         0.580 &      0.11 \\
%poison              &    backdoor &         0.581 &      0.24 \\
%flystudio           &    grayware &         0.588 &      0.34 \\
%ulise               &    grayware &         0.591 &      0.02 \\
%pcclient            &    backdoor &         0.592 &      0.30 \\
%parite              &       virus &         0.594 &      0.07 \\
%dorkbot             &        worm &         0.595 &      0.07 \\
%gify                &    grayware &         0.597 &      0.00 \\
%kasidet             &    grayware &         0.598 &      0.19 \\
%emogen              &  downloader &         0.599 &      0.31 \\
%winwebsec           &   rogueware &         0.599 &      0.21 \\
%zard                &    grayware &         0.599 &      0.03 \\
%geral               &  downloader &         0.599 &      0.23 \\
%dynamer             &  downloader &         0.600 &      0.20 \\
%tdss                &  downloader &         0.606 &      0.06 \\
%lethic              &  downloader &         0.606 &      0.01 \\
%mucc                &    grayware &         0.607 &      0.00 \\
%fareit              &    grayware &         0.609 &      0.14 \\
%virut               &       virus &         0.609 &      0.03 \\
%gimemo              &  ransomware &         0.609 &      0.54 \\
%farfli              &    backdoor &         0.611 &      0.38 \\
%constructor         &    grayware &         0.614 &      0.13 \\
%mudrop              &  downloader &         0.617 &      0.37 \\
%netwiredrc          &    backdoor &         0.618 &      0.24 \\
%polypatch           &    grayware &         0.618 &      0.99 \\
%vemply              &    grayware &         0.619 &      0.61 \\
%cossta              &    grayware &         0.622 &      0.13 \\
%carberp             &  downloader &         0.624 &      0.06 \\
%usteal              &    grayware &         0.625 &      0.31 \\
%bingoml             &    grayware &         0.630 &      0.22 \\
%predator            &    grayware &         0.631 &      0.13 \\
%cridex              &    grayware &         0.632 &      0.15 \\
%ekstak              &    grayware &         0.633 &      0.31 \\
%rozena              &    backdoor &         0.634 &      0.03 \\
%fraudpack           &   rogueware &         0.634 &      0.03 \\
%buterat             &     clicker &         0.634 &      0.17 \\
%siscos              &  downloader &         0.637 &      0.48 \\
%crysis              &  ransomware &         0.638 &      0.27 \\
%palevo              &    backdoor &         0.638 &      0.22 \\
%ymacco              &    grayware &         0.643 &      0.68 \\
%emotet              &  downloader &         0.645 &      0.01 \\
%zlob                &  downloader &         0.647 &      0.21 \\
%quasar              &    backdoor &         0.650 &      0.48 \\
%tenga               &       virus &         0.654 &      0.06 \\
%macri               &    backdoor &         0.654 &      0.58 \\
%simda               &    backdoor &         0.654 &      0.11 \\
%locky               &  downloader &         0.656 &      0.05 \\
%kraddare            &      adware &         0.660 &      0.36 \\
%babar               &    grayware &         0.665 &      0.05 \\
%induc               &       virus &         0.668 &      0.56 \\
%banbra              &    backdoor &         0.669 &      0.50 \\
%dorgam              &  downloader &         0.670 &      0.19 \\
%gamarue             &  downloader &         0.671 &      0.10 \\
%fraudrop            &  downloader &         0.671 &      0.31 \\
%zeroaccess          &  downloader &         0.671 &      0.10 \\
%lotok               &    backdoor &         0.672 &      0.15 \\
%blackhole           &    backdoor &         0.673 &      0.32 \\
%tnega               &  downloader &         0.676 &      0.33 \\
%rebhip              &        worm &         0.677 &      0.16 \\
%titirez             &    grayware &         0.677 &      0.05 \\
%vbinder             &        tool &         0.680 &      0.01 \\
%spyeye              &     spyware &         0.681 &      0.09 \\
%ursnif              &       virus &         0.682 &      0.25 \\
%nanocore            &    backdoor &         0.683 &      0.22 \\
%mokes               &    backdoor &         0.685 &      0.01 \\
%buzus               &  downloader &         0.686 &      0.03 \\
%hupigon             &    backdoor &         0.694 &      0.59 \\
%fugrafa             &    grayware &         0.695 &      0.08 \\
%sysn                &  downloader &         0.696 &      0.19 \\
%slugin              &       virus &         0.700 &      0.04 \\
%onlinegames         &    grayware &         0.704 &      0.14 \\
%cerbu               &    grayware &         0.707 &      0.04 \\
%expiro              &       virus &         0.708 &      0.10 \\
%vittalia            &    grayware &         0.708 &      0.03 \\
%yoddos              &    backdoor &         0.711 &      0.19 \\
%injuke              &    grayware &         0.712 &      0.70 \\
%ursu                &    grayware &         0.714 &      0.23 \\
%elex                &      adware &         0.715 &      0.09 \\
%rbot                &        worm &         0.716 &      0.19 \\
%deyma               &  downloader &         0.716 &      0.21 \\
%emotetu             &  downloader &         0.716 &      0.05 \\
%gamini              &    grayware &         0.716 &      0.00 \\
%pornoblocker        &  ransomware &         0.717 &      0.14 \\
%waldek              &    grayware &         0.717 &      0.00 \\
%racealer            &    grayware &         0.718 &      0.32 \\
%wacatac             &    grayware &         0.719 &      0.14 \\
%killmbr             &    grayware &         0.721 &      0.09 \\
%xtrat               &    backdoor &         0.727 &      0.58 \\
%neconyd             &     clicker &         0.727 &      0.00 \\
%brresmon            &    grayware &         0.729 &      0.01 \\
%qqware              &    grayware &         0.729 &      0.50 \\
%stopcrypt           &  ransomware &         0.730 &      0.00 \\
%zegost              &    backdoor &         0.730 &      0.28 \\
%autinject           &        tool &         0.730 &      0.04 \\
%tonmye              &    grayware &         0.731 &      0.84 \\
%startsurf           &    grayware &         0.732 &      0.03 \\
%schoolboy           &    grayware &         0.733 &      0.07 \\
%tasker              &    grayware &         0.733 &      0.11 \\
%bsymem              &    grayware &         0.736 &      0.25 \\
%wlksm               &       virus &         0.736 &      0.01 \\
%garrun              &    grayware &         0.741 &      0.00 \\
%gamevance           &      adware &         0.743 &      0.05 \\
%sillyfdc            &        worm &         0.746 &      0.55 \\
%zapchast            &  downloader &         0.746 &      0.04 \\
%kuluoz              &  downloader &         0.747 &      0.04 \\
%agentb              &    grayware &         0.747 &      0.13 \\
%ipamor              &    grayware &         0.748 &      0.39 \\
%blamon              &    grayware &         0.749 &      0.33 \\
%orcus               &    backdoor &         0.749 &      0.42 \\
%dorifel             &  downloader &         0.752 &      0.05 \\
%stop                &  ransomware &         0.754 &      0.00 \\
%wajam               &      adware &         0.754 &      0.02 \\
%fragtor             &    grayware &         0.756 &      0.10 \\
%fsysna              &        worm &         0.757 &      0.66 \\
%conduit             &    grayware &         0.759 &      0.35 \\
%kelihos             &    backdoor &         0.759 &      0.12 \\
%bandit              &  downloader &         0.760 &      0.61 \\
%cutwail             &  downloader &         0.761 &      0.10 \\
%bladabindi          &    backdoor &         0.762 &      0.19 \\
%dadobra             &  downloader &         0.764 &      0.63 \\
%speedingupmypc      &    grayware &         0.766 &      0.11 \\
%passview            &    grayware &         0.766 &      0.35 \\
%zurgop              &  downloader &         0.768 &      0.03 \\
%shopper             &    grayware &         0.768 &      0.01 \\
%pincav              &    backdoor &         0.769 &      0.18 \\
%glupteba            &    grayware &         0.770 &      0.01 \\
%ngrbot              &    backdoor &         0.770 &      0.00 \\
%stelega             &    grayware &         0.770 &      0.50 \\
%spesr               &        worm &         0.772 &      0.00 \\
%jaik                &    grayware &         0.772 &      0.05 \\
%voltar              &     clicker &         0.772 &      0.00 \\
%swizzor             &    grayware &         0.773 &      0.06 \\
%nymeria             &    grayware &         0.774 &      0.32 \\
%vebzenpak           &    grayware &         0.776 &      0.00 \\
%windigo             &    grayware &         0.776 &      0.03 \\
%cryptowall          &  ransomware &         0.776 &      0.02 \\
%shutdowner          &    grayware &         0.777 &      0.06 \\
%tibs                &  downloader &         0.777 &      0.29 \\
%dupzom              &  downloader &         0.777 &      0.14 \\
%blackmoon           &    grayware &         0.782 &      0.74 \\
%teslacrypt          &  ransomware &         0.782 &      0.01 \\
%amonetize           &    grayware &         0.783 &      0.07 \\
%lazy                &    grayware &         0.783 &      0.06 \\
%wonton              &  downloader &         0.785 &      0.02 \\
%cerber              &  ransomware &         0.786 &      0.16 \\
%woozlist            &    grayware &         0.786 &      0.90 \\
%renos               &  downloader &         0.787 &      0.17 \\
%spigot              &    grayware &         0.789 &      0.20 \\
%sdbot               &    backdoor &         0.795 &      0.92 \\
%asparnet            &    grayware &         0.796 &      0.05 \\
%vbkryjetor          &  downloader &         0.798 &      0.04 \\
%iobit               &    grayware &         0.799 &      0.38 \\
%bitmin              &       miner &         0.800 &      0.25 \\
%dofoil              &  downloader &         0.801 &      0.06 \\
%lightstone          &    backdoor &         0.802 &      0.17 \\
%killwin             &    grayware &         0.803 &      0.58 \\
%fujacks             &       virus &         0.803 &      0.35 \\
%petya               &  ransomware &         0.803 &      0.16 \\
%ascentive           &    grayware &         0.803 &      0.00 \\
%remcos              &    backdoor &         0.804 &      0.04 \\
%doina               &    grayware &         0.807 &      0.50 \\
%mint                &    grayware &         0.808 &      0.03 \\
%nymaim              &    backdoor &         0.809 &      0.01 \\
%chifrax             &    grayware &         0.809 &      0.41 \\
%citeary             &        worm &         0.809 &      0.16 \\
%mytob               &        worm &         0.810 &      0.30 \\
%softcnapp           &      adware &         0.814 &      0.00 \\
%zbot                &  downloader &         0.814 &      0.24 \\
%plugx               &    backdoor &         0.818 &      0.05 \\
%speedbit            &    grayware &         0.818 &      0.08 \\
%binder              &  downloader &         0.821 &      0.16 \\
%cryptinject         &  downloader &         0.822 &      0.34 \\
%multiplier          &    grayware &         0.823 &      0.00 \\
%gatak               &    grayware &         0.823 &      0.11 \\
%xyligan             &    backdoor &         0.824 &      0.55 \\
%azorult             &    grayware &         0.824 &      0.57 \\
%dorv                &  downloader &         0.825 &      0.08 \\
%skillis             &    grayware &         0.827 &      0.19 \\
%arkeistealer        &    grayware &         0.828 &      0.06 \\
%scrop               &  downloader &         0.828 &      0.12 \\
%hacdef              &    backdoor &         0.829 &      0.21 \\
%brook               &  ransomware &         0.830 &      0.00 \\
%redcap              &  downloader &         0.830 &      0.18 \\
%tofsee              &    backdoor &         0.830 &      0.09 \\
%qbot                &  downloader &         0.834 &      0.00 \\
%darkkomet           &    backdoor &         0.835 &      0.14 \\
%staser              &  downloader &         0.835 &      0.61 \\
%systweak            &    grayware &         0.838 &      0.19 \\
%zusy                &    grayware &         0.839 &      0.07 \\
%taranis             &    grayware &         0.839 &      0.04 \\
%cyfin               &    grayware &         0.839 &      0.05 \\
%balrok              &        worm &         0.841 &      0.00 \\
%guloader            &  downloader &         0.841 &      0.00 \\
%johnnie             &    grayware &         0.841 &      0.65 \\
%diss                &    grayware &         0.841 &      0.03 \\
%prorat              &    backdoor &         0.844 &      0.69 \\
%tiggre              &    grayware &         0.844 &      0.88 \\
%autoitinject        &    backdoor &         0.845 &      0.16 \\
%mabezat             &        worm &         0.845 &      0.00 \\
%schoolgirl          &    grayware &         0.845 &      0.02 \\
%avemaria            &     spyware &         0.845 &      0.12 \\
%trickster           &  downloader &         0.846 &      0.01 \\
%beastdoor           &    backdoor &         0.847 &      0.71 \\
%hematite            &       virus &         0.847 &      0.01 \\
%reline              &     spyware &         0.847 &      0.57 \\
%shade               &  ransomware &         0.848 &      0.05 \\
%deceptpcclean       &    grayware &         0.849 &      0.04 \\
%gandcrypt           &  ransomware &         0.850 &      0.02 \\
%zenlod              &  downloader &         0.850 &      0.22 \\
%spybot              &        worm &         0.851 &      0.15 \\
%xanfpezes           &    grayware &         0.853 &      1.00 \\
%yandex              &    grayware &         0.853 &      0.00 \\
%gandcrab            &  ransomware &         0.853 &      0.03 \\
%stampado            &  ransomware &         0.854 &      0.58 \\
%bactera             &        worm &         0.854 &      0.00 \\
%khalesi             &    grayware &         0.855 &      0.02 \\
%omaneat             &     spyware &         0.855 &      0.07 \\
%jimmy               &  ransomware &         0.855 &      0.00 \\
%raccrypt            &  ransomware &         0.857 &      0.00 \\
%dalexis             &  downloader &         0.858 &      0.00 \\
%xorist              &  downloader &         0.858 &      0.24 \\
%webprefix           &  downloader &         0.858 &      0.12 \\
%installcore         &    grayware &         0.859 &      0.10 \\
%driverpack          &    grayware &         0.860 &      0.13 \\
%trickbot            &       miner &         0.861 &      0.00 \\
%sodinokibi          &    grayware &         0.863 &      0.02 \\
%manbat              &        worm &         0.864 &      0.44 \\
%utorrent            &    grayware &         0.866 &      0.72 \\
%offend              &    grayware &         0.866 &      0.98 \\
%burden              &      adware &         0.868 &      0.02 \\
%skintrim            &      adware &         0.871 &      0.02 \\
%lokibot             &    backdoor &         0.871 &      0.88 \\
%esfury              &        worm &         0.871 &      0.05 \\
%neutrinopos         &  ransomware &         0.872 &      0.12 \\
%recex               &       miner &         0.873 &      0.11 \\
%onescan             &    grayware &         0.873 &      0.38 \\
%hotbar              &    grayware &         0.877 &      0.45 \\
%smartpcsolutions    &    grayware &         0.878 &      0.46 \\
%solmyr              &     spyware &         0.878 &      0.58 \\
%shylock             &    backdoor &         0.879 &      0.03 \\
%subseven            &    backdoor &         0.880 &      0.69 \\
%bunitu              &    grayware &         0.881 &      0.00 \\
%lovgate             &        worm &         0.883 &      0.29 \\
%ficker              &    grayware &         0.884 &      0.15 \\
%turkojan            &    backdoor &         0.884 &      0.16 \\
%yahlover            &        worm &         0.884 &      0.65 \\
%mulinex             &  downloader &         0.885 &      0.65 \\
%crossrider          &    grayware &         0.885 &      0.01 \\
%stihat              &        worm &         0.886 &      1.00 \\
%sogou               &    grayware &         0.886 &      0.17 \\
%bezigate            &    backdoor &         0.888 &      0.15 \\
%opensupdater        &      adware &         0.889 &      0.14 \\
%strab               &  ransomware &         0.889 &      0.38 \\
%istartsurf          &    grayware &         0.890 &      0.01 \\
%xorer               &       virus &         0.890 &      0.11 \\
%dlhelper            &    grayware &         0.892 &      0.13 \\
%nitol               &  downloader &         0.892 &      0.07 \\
%pterodo             &  downloader &         0.893 &      0.00 \\
%mepaow              &       virus &         0.895 &      0.97 \\
%xiaobaminer         &       miner &         0.896 &      0.74 \\
%winwrapper          &    grayware &         0.896 &      0.44 \\
%necurs              &  downloader &         0.897 &      0.08 \\
%privateexeprotector &    grayware &         0.897 &      0.19 \\
%techsnab            &    grayware &         0.898 &      0.02 \\
%mbrlock             &  downloader &         0.901 &      0.02 \\
%hpdefender          &    grayware &         0.901 &      0.00 \\
%vawtrak             &    backdoor &         0.902 &      0.00 \\
%lockbit             &  ransomware &         0.902 &      0.00 \\
%miancha             &  downloader &         0.903 &      0.11 \\
%convertad           &      adware &         0.904 &      0.01 \\
%mywebsearch         &    grayware &         0.905 &      0.01 \\
%ardamax             &    grayware &         0.906 &      0.01 \\
%bobax               &        worm &         0.906 &      0.03 \\
%agobot              &    backdoor &         0.908 &      0.32 \\
%filefinder          &      adware &         0.909 &      0.03 \\
%gamania             &  downloader &         0.910 &      0.05 \\
%lamer               &       virus &         0.910 &      0.43 \\
%woool               &    grayware &         0.910 &      0.94 \\
%bebloh              &    backdoor &         0.911 &      0.01 \\
%pirminay            &  downloader &         0.911 &      0.09 \\
%bandoo              &    grayware &         0.911 &      0.00 \\
%relevantknowledge   &      adware &         0.912 &      0.00 \\
%cobaltstrike        &  downloader &         0.912 &      0.03 \\
%klez                &        worm &         0.913 &      0.01 \\
%kolovorot           &    grayware &         0.913 &      0.03 \\
%schwarzesonne       &    backdoor &         0.914 &      0.48 \\
%kovter              &    grayware &         0.915 &      0.00 \\
%sohanad             &        worm &         0.915 &      0.66 \\
%shandaadd           &      adware &         0.916 &      0.88 \\
%ranumbot            &    grayware &         0.917 &      0.28 \\
%dealply             &    grayware &         0.917 &      0.19 \\
%loadmoney           &    grayware &         0.918 &      0.08 \\
%linkury             &    grayware &         0.919 &      0.00 \\
%netstream           &    grayware &         0.920 &      0.00 \\
%hider               &    grayware &         0.921 &      0.59 \\
%koobface            &        worm &         0.921 &      0.77 \\
%lollipop            &      adware &         0.923 &      0.00 \\
%vilsel              &  downloader &         0.923 &      0.41 \\
%pidgeon             &       virus &         0.924 &      0.16 \\
%blackie             &    grayware &         0.924 &      0.00 \\
%icedid              &  downloader &         0.924 &      0.04 \\
%malex               &  downloader &         0.924 &      0.01 \\
%msposer             &  downloader &         0.925 &      0.04 \\
%scrarev             &    grayware &         0.925 &      0.05 \\
%fakefolder          &        worm &         0.926 &      0.71 \\
%toptools            &    grayware &         0.928 &      0.00 \\
%gozi                &     spyware &         0.928 &      0.01 \\
%cycbot              &    backdoor &         0.929 &      0.00 \\
%benjamin            &        worm &         0.931 &      0.94 \\
%browsefox           &    grayware &         0.931 &      0.06 \\
%eyestye             &  downloader &         0.933 &      0.27 \\
%urelas              &    backdoor &         0.933 &      0.38 \\
%fosniw              &  downloader &         0.934 &      0.42 \\
%auslogics           &    grayware &         0.934 &      0.03 \\
%agentc              &    grayware &         0.934 &      0.00 \\
%virbox              &    grayware &         0.936 &      0.01 \\
%upatre              &  downloader &         0.937 &      0.04 \\
%fakefire            &        worm &         0.938 &      0.03 \\
%opencandy           &    grayware &         0.938 &      0.92 \\
%adpack              &      adware &         0.939 &      0.02 \\
%pcacceleratepro     &    grayware &         0.939 &      0.01 \\
%aitinject           &    backdoor &         0.939 &      0.01 \\
%dodiw               &    backdoor &         0.940 &      0.83 \\
%remoteutilities     &    grayware &         0.941 &      0.90 \\
%horst               &    backdoor &         0.942 &      0.30 \\
%adgazelle           &    grayware &         0.942 &      0.00 \\
%silentall           &    grayware &         0.942 &      0.77 \\
%seraph              &  downloader &         0.942 &      0.00 \\
%refinka             &    grayware &         0.942 &      0.01 \\
%softpulse           &    grayware &         0.943 &      0.34 \\
%neoreklami          &      adware &         0.943 &      0.00 \\
%formbook            &     spyware &         0.943 &      0.37 \\
%imali               &      adware &         0.944 &      0.00 \\
%winactivator        &        tool &         0.944 &      0.06 \\
%qzonit              &     spyware &         0.944 &      0.87 \\
%explorerhijack      &  downloader &         0.944 &      0.01 \\
%multibar            &    grayware &         0.947 &      0.23 \\
%perion              &    grayware &         0.947 &      0.00 \\
%qshell              &    grayware &         0.949 &      0.00 \\
%tremp               &    grayware &         0.949 &      0.00 \\
%installerex         &    grayware &         0.950 &      0.00 \\
%offergenerator      &    grayware &         0.951 &      0.00 \\
%firseria            &    grayware &         0.951 &      0.08 \\
%screenmate          &    grayware &         0.951 &      0.01 \\
%tufik               &       virus &         0.951 &      0.00 \\
%chir                &        worm &         0.951 &      0.12 \\
%speedcat            &    grayware &         0.953 &      0.00 \\
%shiz                &    backdoor &         0.953 &      0.00 \\
%frosparf            &     clicker &         0.954 &      0.73 \\
%qihoo               &    grayware &         0.954 &      0.03 \\
%megasearch          &      adware &         0.955 &      0.00 \\
%duote               &      adware &         0.955 &      0.98 \\
%ibryte              &      adware &         0.955 &      0.09 \\
%domaiq              &    grayware &         0.956 &      0.53 \\
%zygug               &    grayware &         0.956 &      0.52 \\
%swisyn              &        worm &         0.956 &      0.07 \\
%daws                &  downloader &         0.956 &      0.02 \\
%installbrain        &    grayware &         0.956 &      0.02 \\
%vundo               &        worm &         0.957 &      0.89 \\
%xcnfe               &    grayware &         0.957 &      0.00 \\
%tinba               &  downloader &         0.958 &      0.15 \\
%downloadsponsor     &    grayware &         0.958 &      0.87 \\
%autog               &  downloader &         0.959 &      0.00 \\
%icloader            &    grayware &         0.959 &      0.02 \\
%veil                &    grayware &         0.960 &      0.01 \\
%lecna               &    backdoor &         0.960 &      0.24 \\
%downloadassistant   &    grayware &         0.960 &      0.18 \\
%mewsspy             &    backdoor &         0.960 &      0.06 \\
%regsup              &  downloader &         0.960 &      0.01 \\
%kuaizip             &      adware &         0.960 &      0.01 \\
%bublik              &  downloader &         0.961 &      0.79 \\
%simbot              &    backdoor &         0.961 &      0.01 \\
%vobfus              &        worm &         0.961 &      0.00 \\
%lunastorm           &        worm &         0.962 &      0.97 \\
%sohana              &  downloader &         0.962 &      0.18 \\
%downloadadmin       &    grayware &         0.962 &      0.03 \\
%filetour            &    grayware &         0.962 &      0.04 \\
%reptile             &    grayware &         0.963 &      0.00 \\
%kolweb              &    grayware &         0.963 &      0.96 \\
%pacex               &       miner &         0.964 &      0.00 \\
%xolxo               &        worm &         0.964 &      0.99 \\
%axespec             &    grayware &         0.964 &      0.02 \\
%betload             &    grayware &         0.965 &      0.03 \\
%kolab               &    grayware &         0.965 &      0.01 \\
%lydra               &     spyware &         0.965 &      0.05 \\
%idsohtu             &    grayware &         0.965 &      0.06 \\
%wapomi              &       virus &         0.966 &      0.08 \\
%miniduke            &    backdoor &         0.966 &      0.00 \\
%wannacry            &  ransomware &         0.966 &      0.01 \\
%wakme               &  downloader &         0.966 &      0.00 \\
%plurox              &    backdoor &         0.967 &      0.00 \\
%phorpiex            &        worm &         0.967 &      0.12 \\
%koutodoor           &    backdoor &         0.967 &      0.16 \\
%webcompanion        &    grayware &         0.967 &      0.00 \\
%installmonster      &    grayware &         0.967 &      0.83 \\
%protux              &    backdoor &         0.967 &      0.04 \\
%blackshades         &        worm &         0.967 &      0.02 \\
%detrahere           &    grayware &         0.968 &      0.04 \\
%rostpay             &    grayware &         0.969 &      0.96 \\
%moarider            &        worm &         0.969 &      0.98 \\
%gofot               &  downloader &         0.970 &      0.88 \\
%growtopia           &    grayware &         0.970 &      0.69 \\
%fesber              &        worm &         0.970 &      0.33 \\
%sytro               &        worm &         0.971 &      0.45 \\
%bayrob              &     spyware &         0.971 &      0.01 \\
%spyrix              &    grayware &         0.971 &      0.56 \\
%cosmu               &       virus &         0.972 &      0.00 \\
%shipup              &  downloader &         0.973 &      0.30 \\
%lmir                &       virus &         0.973 &      0.02 \\
%4shared             &    grayware &         0.973 &      0.01 \\
%snojan              &  downloader &         0.973 &      0.01 \\
%broskod             &  downloader &         0.974 &      0.01 \\
%socelars            &     spyware &         0.974 &      0.01 \\
%egroupdial          &    grayware &         0.974 &      0.87 \\
%dinwod              &  downloader &         0.974 &      0.98 \\
%havex               &    grayware &         0.974 &      0.00 \\
%dstudio             &    grayware &         0.974 &      0.02 \\
%memery              &       virus &         0.974 &      0.09 \\
%vrbrothers          &      adware &         0.975 &      0.88 \\
%alyak               &  downloader &         0.975 &      0.55 \\
%mooqkel             &  downloader &         0.975 &      1.00 \\
%mocrt               &     spyware &         0.975 &      0.73 \\
%bundlore            &      adware &         0.976 &      0.02 \\
%anmalpro            &    grayware &         0.976 &      0.00 \\
%getnow              &    grayware &         0.976 &      0.88 \\
%qqhack              &        tool &         0.976 &      0.98 \\
%webdialer           &      dialer &         0.976 &      0.86 \\
%maener              &       miner &         0.976 &      0.00 \\
%wecod               &     spyware &         0.976 &      0.94 \\
%expressdownloader   &    grayware &         0.976 &      0.62 \\
%unruy               &  downloader &         0.976 &      0.13 \\
%dexel               &  downloader &         0.977 &      0.00 \\
%reconyc             &    grayware &         0.977 &      0.36 \\
%installiq           &    grayware &         0.977 &      0.02 \\
%bcryptinject        &  downloader &         0.977 &      0.00 \\
%ceatrg              &  downloader &         0.978 &      0.66 \\
%tougle              &    grayware &         0.978 &      0.00 \\
%pajetbin            &        worm &         0.978 &      0.91 \\
%neshta              &       virus &         0.978 &      0.02 \\
%webalta             &    grayware &         0.979 &      0.05 \\
%lipler              &  downloader &         0.979 &      0.00 \\
%mira                &        worm &         0.979 &      0.23 \\
%aauto               &    grayware &         0.979 &      0.56 \\
%plingky             &  downloader &         0.980 &      0.93 \\
%cardspy             &     spyware &         0.980 &      1.00 \\
%banload             &       miner &         0.980 &      0.49 \\
%macoute             &        worm &         0.980 &      0.00 \\
%cjishu              &      adware &         0.980 &      0.01 \\
%kuaiba              &      adware &         0.980 &      0.05 \\
%slimware            &    grayware &         0.981 &      0.00 \\
%inbox               &    grayware &         0.981 &      0.00 \\
%airinstaller        &    grayware &         0.981 &      0.87 \\
%bloored             &        worm &         0.981 &      0.00 \\
%kingsoft            &    grayware &         0.981 &      0.37 \\
%youxun              &    grayware &         0.982 &      0.11 \\
%mediaget            &    grayware &         0.982 &      0.33 \\
%qjwmonkey           &    grayware &         0.982 &      0.47 \\
%disttrack           &    grayware &         0.983 &      0.00 \\
%softonic            &      adware &         0.983 &      0.79 \\
%fasong              &        worm &         0.983 &      0.78 \\
%safebytes           &    grayware &         0.984 &      0.99 \\
%babonock            &        worm &         0.984 &      0.00 \\
%goldrv              &        worm &         0.984 &      0.00 \\
%detroie             &       virus &         0.984 &      0.02 \\
%gobot               &    backdoor &         0.985 &      0.04 \\
%cosmicduke          &    backdoor &         0.985 &      0.04 \\
%openinstall         &    grayware &         0.985 &      0.22 \\
%paneidix            &    backdoor &         0.985 &      0.00 \\
%rising              &    grayware &         0.985 &      0.34 \\
%gepys               &  downloader &         0.986 &      0.00 \\
%ksdler              &      adware &         0.986 &      0.69 \\
%hebchengjiu         &      adware &         0.986 &      0.07 \\
%lolbot              &        worm &         0.986 &      0.00 \\
%cheatengine         &    grayware &         0.986 &      0.00 \\
%zvuzona             &    grayware &         0.986 &      0.79 \\
%outbyte             &    grayware &         0.987 &      0.04 \\
%laqma               &    grayware &         0.987 &      0.12 \\
%winloadsda          &    grayware &         0.987 &      0.86 \\
%powerless           &  downloader &         0.987 &      0.10 \\
%sillyp2p            &        worm &         0.987 &      0.01 \\
%agentino            &  downloader &         0.987 &      0.00 \\
%tupym               &        worm &         0.987 &      0.00 \\
%agentcrypt          &    grayware &         0.987 &      0.00 \\
%seven               &  ransomware &         0.987 &      0.00 \\
%kronosbot           &    backdoor &         0.987 &      0.00 \\
%dumpy               &        worm &         0.988 &      0.00 \\
%cmy3u               &    grayware &         0.988 &      0.98 \\
%elemental           &    grayware &         0.988 &      0.00 \\
%aenjaris            &    grayware &         0.988 &      0.00 \\
%dluca               &  downloader &         0.988 &      0.84 \\
%oberal              &  downloader &         0.988 &      0.00 \\
%antavmu             &        worm &         0.988 &      0.00 \\
%konus               &    backdoor &         0.989 &      0.00 \\
%apost               &  downloader &         0.989 &      0.01 \\
%virlock             &       virus &         0.989 &      0.00 \\
%addlyrics           &      adware &         0.989 &      0.00 \\
%viking              &       virus &         0.989 &      0.98 \\
%gippers             &  ransomware &         0.990 &      0.00 \\
%pluto               &        worm &         0.990 &      0.34 \\
%virfire             &        worm &         0.990 &      0.00 \\
%stone               &       virus &         0.990 &      0.97 \\
%connectwise         &        tool &         0.991 &      0.00 \\
%downer              &    grayware &         0.991 &      0.96 \\
%mailru              &    grayware &         0.991 &      0.00 \\
%lassorm             &        worm &         0.991 &      0.00 \\
%bzub                &  downloader &         0.991 &      0.02 \\
%pedex               &    backdoor &         0.991 &      1.00 \\
%lulusoftware        &    grayware &         0.992 &      0.13 \\
%pondfull            &        worm &         0.992 &      0.05 \\
%pistolar            &  downloader &         0.992 &      0.76 \\
%gorillaprice        &      adware &         0.992 &      0.06 \\
%taskloader          &      adware &         0.993 &      0.00 \\
%brontok             &        worm &         0.993 &      0.95 \\
%winner              &    grayware &         0.993 &      0.00 \\
%vkontaktedj         &    grayware &         0.994 &      0.04 \\
%hafen               &    grayware &         0.994 &      0.00 \\
%gigex               &        worm &         0.994 &      0.00 \\
%clipspy             &    grayware &         0.994 &      0.02 \\
%eyoorun             &    grayware &         0.995 &      0.00 \\
%lunam               &        worm &         0.995 &      0.99 \\
%tisandr             &        worm &         0.995 &      0.00 \\
%fakens              &      adware &         0.995 &      0.00 \\
%roxer               &    grayware &         0.995 &      0.97 \\
%gamemodding         &    grayware &         0.996 &      0.06 \\
%goabeny             &    grayware &         0.996 &      0.75 \\
%jevafus             &  downloader &         0.996 &      0.45 \\
%offercore           &    grayware &         0.996 &      0.00 \\
%pahooka             &        worm &         0.996 &      1.00 \\
%salgorea            &  downloader &         0.996 &      0.00 \\
%zepfod              &        worm &         0.997 &      0.00 \\
%qqpass              &  downloader &         0.997 &      0.94 \\
%lolojan             &    grayware &         0.998 &      0.95 \\
%pochi               &        worm &         0.998 &      0.00 \\
%mydoom              &        worm &         0.998 &      0.83 \\
%allaple             &        worm &         0.999 &      0.00 \\
%socks               &        worm &         0.999 &      0.18 \\
%xihet               &  downloader &         0.999 &      0.08 \\
%tempedreve          &       virus &         0.999 &      0.00 \\
%griptolo            &        worm &         0.999 &      0.07 \\
%shyape              &  downloader &         0.999 &      0.39 \\
%sivis               &       virus &         0.999 &      0.72 \\
%wabot               &    backdoor &         0.999 &      0.19 \\
%moonlight           &        worm &         0.999 &      0.08 \\
%prepscram           &    grayware &         1.000 &      1.00 \\
%grenam              &       virus &         1.000 &      0.00 \\
%phishbank           &        worm &         1.000 &      0.00 \\
%coroxy              &    backdoor &         1.000 &      0.00 \\
%jeefo               &       virus &         1.000 &      0.99 \\
%zylom               &  downloader &         1.000 &      0.00 \\
%hebogo              &    grayware &         1.000 &      0.00 \\
%loring              &        worm &         1.000 &      1.00 \\
%rums                &  downloader &         1.000 &      0.00 \\
%pykspa              &        worm &         1.000 &      0.00 \\
%fakedoc             &        worm &         1.000 &      0.00 \\
%houndhack           &    backdoor &         1.000 &      1.00 \\
%diamin              &      dialer &         1.000 &      0.80 \\
%syncopate           &    grayware &         1.000 &      0.20 \\
%drstwex             &  downloader &         1.000 &      0.00 \\
%enosch              &        worm &         1.000 &      0.00 \\
%neojit              &  downloader &         1.000 &      0.00 \\
%blihan              &    backdoor &         1.000 &      0.00 \\
%sfone               &        worm &         1.000 &      0.65 \\
%cambot              &        worm &         1.000 &      0.00 \\
%qaccel              &     clicker &         1.000 &      0.00 \\
%shodi               &       virus &         1.000 &      0.00 \\
%mimdau              &  downloader &         1.000 &      1.00 \\
%burn                &    backdoor &         1.000 &      0.00 \\
%mimail              &        worm &         1.000 &      0.00 \\
%mapsgory            &  downloader &         1.000 &      0.00 \\
%catalina            &    grayware &         1.000 &      0.00 \\
%adaebook            &    grayware &         1.000 &      1.00 \\
%nevereg             &        worm &         1.000 &      0.51 \\
%recodrop            &  downloader &         1.000 &      0.00 \\
%adposhel            &    grayware &         1.000 &      0.00 \\
%primecasino         &    grayware &         1.000 &      0.00 \\
%powerspider         &    backdoor &         1.000 &      0.00 \\
%regrun              &  ransomware &         1.000 &      1.00 \\
%plemood             &        worm &         1.000 &      0.00 \\
%berbew              &    backdoor &         1.000 &      0.00 \\
%rifdoor             &    backdoor &         1.000 &      0.00 \\
%picsys              &        worm &         1.000 &      1.00 \\
%amigo               &    grayware &         1.000 &      0.00 \\
%ruledor             &    backdoor &         1.000 &      0.00 \\
%atcpa               &       virus &         1.000 &      0.00 \\
%rungbu              &        worm &         1.000 &      0.18 \\
%sinau               &       virus &         1.000 &      0.00 \\
%rikihaki            &        worm &         1.000 &      0.00 \\
%obit                &  downloader &         1.000 &      0.00 \\
%triusor             &       virus &         1.000 &      0.00 \\
%wenper              &        worm &         1.000 &      0.42 \\
%dpsk                &       miner &         1.000 &      1.00 \\
%drolnux             &        worm &         1.000 &      0.00 \\
%dropware            &      adware &         1.000 &      0.01 \\
%vbclone             &  downloader &         1.000 &      0.00 \\
%gator               &      adware &         1.000 &      0.47 \\
%gametea             &  downloader &         1.000 &      1.00 \\
%eggnog              &        worm &         1.000 &      0.00 \\
%finfish             &    backdoor &         1.000 &      0.00 \\
%updane              &    grayware &         1.000 &      0.00 \\
%esaprof             &  downloader &         1.000 &      0.84 \\
%upantix             &  downloader &         1.000 &      1.00 \\
%facido              &  downloader &         1.000 &      0.00 \\
%umbra               &  downloader &         1.000 &      0.00 \\
%esecn               &      adware &         1.000 &      1.00 \\
%ezsoftwareupdater   &    grayware &         1.000 &      0.00 \\
%downloadguide       &    grayware &         1.000 &      0.00 \\
%geegly              &       virus &         1.000 &      1.00 \\
%gogo                &       virus &         1.000 &      0.00 \\
%dostre              &  downloader &         1.000 &      0.00 \\
%ludbaruma           &        worm &         1.000 &      0.00 \\
%chrop               &      adware &         1.000 &      0.63 \\
%lebreat             &    grayware &         1.000 &      1.00 \\
%trymedia            &      adware &         1.000 &      0.73 \\
%wavipeg             &    backdoor &         1.000 &      0.00 \\
%krol                &        worm &         1.000 &      0.00 \\
%keydoor             &     spyware &         1.000 &      0.00 \\
%lyposit             &  ransomware &         1.000 &      0.00 \\
%jpgiframe           &    grayware &         1.000 &      0.00 \\
%vybab               &        worm &         1.000 &      0.00 \\
%itorrent            &    grayware &         1.000 &      0.00 \\
%vtflooder           &  downloader &         1.000 &      0.55 \\
%inlog               &  downloader &         1.000 &      0.00 \\
%stormattack         &  downloader &         1.000 &      0.00 \\
%digs                &       miner &         1.000 &      0.00 \\
%swjoy               &    grayware &         1.000 &      0.00 \\
%contenedor          &       virus &         1.000 &      0.00 \\
%soulclose           &        worm &         1.000 &      0.00 \\
\bottomrule
\end{tabular}
\end{table}

% \begin{table}[h]
\setlength\tabcolsep{4pt}
\centering
\small
\caption{Top-10 malware families with the lowest binary classification accuracy using
	dynamic features (i.e., highest mispredictions as goodware).}
\label{tbl:binary_dyn_bestAndWorst}
\begin{tabular}{llccc}
\toprule
\multicolumn{5}{c}{\textbf{Dynamic binary classification}}\\
\textbf{Family} & \textbf{Class} & \textbf{Avg Recall} & \textbf{Packed} &  \textbf{FMR}\\
\midrule
tasker   		&       grayware	   	&  	0.0 	& 11\%  & 0.77	\\
malex           &       downloader		&   0.0		& 1\%	& 0.77 	\\
rostpay      	&       grayware    	&   0.0		& 96\%	& 0.76	\\
constructor     &       grayware      	&   0.0		& 13\%	& 0.78	\\
atcpa           &       virus     		&   0.0		& 0\%	& 0.78	\\
mocrt          	&       spyware  		&  	0.0		& 73\%	& 0.80	\\
mokes           &       backdoor  		&   0.0		& 1\%	& 0.65	\\
bingoml         &       grayware  		&   0.0		& 22\%	& 0.72  \\
safebytes       &       grayware  		&   0.0		& 99\%	& 0.81	\\
trymedia        &       adware  		&   0.0		& 73\%	& 0.70	\\
\bottomrule
\end{tabular}
\end{table}

% \begin{table}[h]
\centering
\small
\caption{Top-10 malware families with the lowest detection accuracy using
	combined features
	(i.e., highest mispredictions as goodware).}
\label{tbl:binary_combined_bestAndWorst}
\begin{tabular}{llcr}
\toprule
\multicolumn{4}{c}{\textbf{Combined binary classification}}\\
	\textbf{Family} & \textbf{Class} &  \textbf{Avg F1} &  \textbf{\% packed} \\
\midrule
sality              &       virus &         0.755 &       4\% \\
slugin              &       virus &         0.770 &       4\% \\
asparnet            &    grayware &         0.775 &       5\% \\
pioneer             &       virus &         0.790 &       6\% \\
expiro              &       virus &         0.825 &      10\% \\
iobit               &    grayware &         0.874 &      38\% \\
speedingupmypc      &    grayware &         0.875 &      11\% \\
conduit             &    grayware &         0.875 &      35\% \\
softcnapp           &      adware &         0.880 &       0\% \\
installcore         &    grayware &         0.895 &      10\% \\
%systweak            &    grayware &         0.900 &      19 \\
%copidmbe            &       virus &         0.900 &       9 \\
%passview            &    grayware &         0.901 &      35 \\
%wajam               &      adware &         0.915 &       2 \\
%ekstak              &    grayware &         0.935 &      31 \\
%zpevdo              &    grayware &         0.935 &      15 \\
%getnow              &    grayware &         0.935 &      88 \\
%driverpack          &    grayware &         0.935 &      13 \\
%mywebsearch         &    grayware &         0.935 &       1 \\
%kuaiba              &      adware &         0.935 &       5 \\
%rozena              &    backdoor &         0.940 &       3 \\
%elex                &      adware &         0.940 &       9 \\
%shopper             &    grayware &         0.940 &       1 \\
%bulz                &    grayware &         0.940 &      15 \\
%citeary             &        worm &         0.947 &      16 \\
%sogou               &    grayware &         0.950 &      17 \\
%virut               &       virus &         0.950 &       3 \\
%burden              &      adware &         0.950 &       2 \\
%geral               &  downloader &         0.950 &      23 \\
%tisandr             &        worm &         0.950 &       0 \\
%gamevance           &      adware &         0.955 &       5 \\
%yandex              &    grayware &         0.955 &       0 \\
%toptools            &    grayware &         0.955 &       0 \\
%ursu                &    grayware &         0.955 &      23 \\
%sodinokibi          &    grayware &         0.955 &       2 \\
%uwamson             &    grayware &         0.955 &      15 \\
%relevantknowledge   &      adware &         0.955 &       0 \\
%cobra               &    grayware &         0.955 &      60 \\
%bandoo              &    grayware &         0.955 &       0 \\
%diztakun            &    grayware &         0.960 &      27 \\
%skillis             &    grayware &         0.960 &      19 \\
%smartpcsolutions    &    grayware &         0.960 &      46 \\
%downloadassistant   &    grayware &         0.960 &      18 \\
%techsnab            &    grayware &         0.960 &       2 \\
%alien               &  downloader &         0.960 &      20 \\
%presenoker          &    grayware &         0.960 &      16 \\
%schoolboy           &    grayware &         0.960 &       7 \\
%parite              &       virus &         0.960 &       7 \\
%diskwriter          &    grayware &         0.963 &      29 \\
%staser              &  downloader &         0.965 &      61 \\
%cjishu              &      adware &         0.965 &       1 \\
%chir                &        worm &         0.965 &      12 \\
%jatif               &    grayware &         0.965 &       2 \\
%sabsik              &    grayware &         0.965 &      30 \\
%adgazelle           &    grayware &         0.965 &       0 \\
%emotet              &  downloader &         0.965 &       1 \\
%vrbrothers          &      adware &         0.970 &      88 \\
%gamini              &    grayware &         0.970 &       0 \\
%bestafera           &    grayware &         0.970 &      57 \\
%opensupdater        &      adware &         0.970 &      14 \\
%vitro               &       virus &         0.970 &       3 \\
%cridex              &    grayware &         0.970 &      15 \\
%polypatch           &    grayware &         0.970 &      99 \\
%rbot                &        worm &         0.970 &      19 \\
%fugrafa             &    grayware &         0.970 &       8 \\
%kraddare            &      adware &         0.970 &      36 \\
%wakme               &  downloader &         0.975 &       0 \\
%zard                &    grayware &         0.975 &       3 \\
%downloadadmin       &    grayware &         0.975 &       3 \\
%lotok               &    backdoor &         0.975 &      15 \\
%midie               &    grayware &         0.975 &      33 \\
%utorrent            &    grayware &         0.975 &      72 \\
%taranis             &    grayware &         0.975 &       4 \\
%petya               &  ransomware &         0.975 &      16 \\
%mediaget            &    grayware &         0.975 &      33 \\
%constructor         &    grayware &         0.975 &      13 \\
%expressdownloader   &    grayware &         0.975 &      62 \\
%remoteutilities     &    grayware &         0.975 &      90 \\
%sdum                &    grayware &         0.975 &      24 \\
%clipbanker          &    grayware &         0.975 &      11 \\
%scrop               &  downloader &         0.975 &      12 \\
%menti               &        worm &         0.975 &      11 \\
%lamer               &       virus &         0.978 &      43 \\
%installbrain        &    grayware &         0.980 &       2 \\
%filetour            &    grayware &         0.980 &       4 \\
%anmalpro            &    grayware &         0.980 &       0 \\
%vkontaktedj         &    grayware &         0.980 &       4 \\
%buzus               &  downloader &         0.980 &       3 \\
%buzy                &        worm &         0.980 &      35 \\
%pcacceleratepro     &    grayware &         0.980 &       1 \\
%binder              &  downloader &         0.980 &      16 \\
%farfli              &    backdoor &         0.980 &      38 \\
%speedbit            &    grayware &         0.980 &       8 \\
%inlog               &  downloader &         0.980 &       0 \\
%cerbu               &    grayware &         0.980 &       4 \\
%deceptpcclean       &    grayware &         0.980 &       4 \\
%dupzom              &  downloader &         0.980 &      14 \\
%glupteba            &    grayware &         0.980 &       1 \\
%convagent           &    grayware &         0.980 &      23 \\
%cobaltstrike        &  downloader &         0.980 &       3 \\
%ipamor              &    grayware &         0.980 &      39 \\
%offergenerator      &    grayware &         0.985 &       0 \\
%slimware            &    grayware &         0.985 &       0 \\
%kuaizip             &      adware &         0.985 &       1 \\
%pasta               &    grayware &         0.985 &      28 \\
%oficla              &    backdoor &         0.985 &      36 \\
%macoute             &        worm &         0.985 &       0 \\
%predator            &    grayware &         0.985 &      13 \\
%bladabindi          &    backdoor &         0.985 &      19 \\
%pcclient            &    backdoor &         0.985 &      30 \\
%ibryte              &      adware &         0.985 &       9 \\
%mucc                &    grayware &         0.985 &       0 \\
%mudrop              &  downloader &         0.985 &      37 \\
%imali               &      adware &         0.985 &       0 \\
%chifrax             &    grayware &         0.985 &      41 \\
%youxun              &    grayware &         0.985 &      11 \\
%refroso             &    backdoor &         0.985 &      41 \\
%trickster           &  downloader &         0.985 &       1 \\
%hotbar              &    grayware &         0.985 &      45 \\
%dealply             &    grayware &         0.985 &      19 \\
%fujacks             &       virus &         0.985 &      35 \\
%dorkbot             &        worm &         0.985 &       7 \\
%fraudpack           &   rogueware &         0.985 &       3 \\
%spigot              &    grayware &         0.985 &      20 \\
%plurox              &    backdoor &         0.985 &       0 \\
%tibia               &    grayware &         0.985 &      31 \\
%bifrose             &    backdoor &         0.985 &      33 \\
%alman               &       virus &         0.985 &      11 \\
%ulise               &    grayware &         0.985 &       2 \\
%qjwmonkey           &    grayware &         0.985 &      47 \\
%rising              &    grayware &         0.985 &      34 \\
%tenga               &       virus &         0.985 &       6 \\
%tdss                &  downloader &         0.985 &       6 \\
%crossrider          &    grayware &         0.985 &       1 \\
%shutdowner          &    grayware &         0.989 &       6 \\
%turkojan            &    backdoor &         0.990 &      16 \\
%auslogics           &    grayware &         0.990 &       3 \\
%perion              &    grayware &         0.990 &       0 \\
%jeefo               &       virus &         0.990 &      99 \\
%occamy              &    grayware &         0.990 &      26 \\
%amonetize           &    grayware &         0.990 &       7 \\
%omaneat             &     spyware &         0.990 &       7 \\
%jaik                &    grayware &         0.990 &       5 \\
%onescan             &    grayware &         0.990 &      38 \\
%kingsoft            &    grayware &         0.990 &      37 \\
%vbkryjetor          &  downloader &         0.990 &       4 \\
%usteal              &    grayware &         0.990 &      31 \\
%kelihos             &    backdoor &         0.990 &      12 \\
%kiser               &    grayware &         0.990 &      55 \\
%detrahere           &    grayware &         0.990 &       4 \\
%opencandy           &    grayware &         0.990 &      92 \\
%jacard              &    grayware &         0.990 &      34 \\
%upatre              &  downloader &         0.990 &       4 \\
%bancos              &     spyware &         0.990 &      44 \\
%tasker              &    grayware &         0.990 &      11 \\
%filefinder          &      adware &         0.990 &       3 \\
%strab               &  ransomware &         0.990 &      38 \\
%babar               &    grayware &         0.990 &       5 \\
%lockbit             &  ransomware &         0.990 &       0 \\
%lightstone          &    backdoor &         0.990 &      17 \\
%remcos              &    backdoor &         0.990 &       4 \\
%elemental           &    grayware &         0.990 &       0 \\
%tremp               &    grayware &         0.990 &       0 \\
%browsefox           &    grayware &         0.990 &       6 \\
%bredolab            &  downloader &         0.990 &      19 \\
%blackhole           &    backdoor &         0.990 &      32 \\
%installerex         &    grayware &         0.990 &       0 \\
%khalesi             &    grayware &         0.990 &       2 \\
%installmonster      &    grayware &         0.990 &      83 \\
%sillyfdc            &        worm &         0.990 &      55 \\
%cheatengine         &    grayware &         0.995 &       0 \\
%veil                &    grayware &         0.995 &       1 \\
%shandaadd           &      adware &         0.995 &      88 \\
%coins               &    grayware &         0.995 &      18 \\
%dynamer             &  downloader &         0.995 &      20 \\
%megasearch          &      adware &         0.995 &       0 \\
%ursnif              &       virus &         0.995 &      25 \\
%silentall           &    grayware &         0.995 &      77 \\
%emotetu             &  downloader &         0.995 &       5 \\
%dumpy               &        worm &         0.995 &       0 \\
%icloader            &    grayware &         0.995 &       2 \\
%mokes               &    backdoor &         0.995 &       1 \\
%idsohtu             &    grayware &         0.995 &       6 \\
%primecasino         &    grayware &         0.995 &       0 \\
%plugx               &    backdoor &         0.995 &       5 \\
%doina               &    grayware &         0.995 &      50 \\
%swisyn              &        worm &         0.995 &       7 \\
%loadmoney           &    grayware &         0.995 &       8 \\
%mytob               &        worm &         0.995 &      30 \\
%dstudio             &    grayware &         0.995 &       2 \\
%poison              &    backdoor &         0.995 &      24 \\
%explorerhijack      &  downloader &         0.995 &       1 \\
%guloader            &  downloader &         0.995 &       0 \\
%urelas              &    backdoor &         0.995 &      38 \\
%induc               &       virus &         0.995 &      56 \\
%zurgop              &  downloader &         0.995 &       3 \\
%lazy                &    grayware &         0.995 &       6 \\
%hacdef              &    backdoor &         0.995 &      21 \\
%crysis              &  ransomware &         0.995 &      27 \\
%spyrix              &    grayware &         0.995 &      56 \\
%palevo              &    backdoor &         0.995 &      22 \\
%windigo             &    grayware &         0.995 &       3 \\
%hpdefender          &    grayware &         0.995 &       0 \\
%necurs              &  downloader &         0.995 &       8 \\
%woool               &    grayware &         0.995 &      94 \\
%netwiredrc          &    backdoor &         0.995 &      24 \\
%tibs                &  downloader &         0.995 &      29 \\
%fragtor             &    grayware &         0.995 &      10 \\
%nanocore            &    backdoor &         0.995 &      22 \\
%adpack              &      adware &         0.995 &       2 \\
%avemaria            &     spyware &         0.995 &      12 \\
%quasar              &    backdoor &         0.995 &      48 \\
%gendal              &    grayware &         0.995 &      62 \\
%trickbot            &       miner &         0.995 &       0 \\
%triusor             &       virus &         0.995 &       0 \\
%dlhelper            &    grayware &         0.995 &      13 \\
%racealer            &    grayware &         0.995 &      32 \\
%atraps              &    grayware &         0.995 &      20 \\
%tufik               &       virus &         0.995 &       0 \\
%xyligan             &    backdoor &         0.995 &      55 \\
%bezigate            &    backdoor &         1.000 &      15 \\
%brontok             &        worm &         1.000 &      95 \\
%bingoml             &    grayware &         1.000 &      22 \\
%bitmin              &       miner &         1.000 &      25 \\
%blackie             &    grayware &         1.000 &       0 \\
%blackmoon           &    grayware &         1.000 &      74 \\
%blackshades         &        worm &         1.000 &       2 \\
%blamon              &    grayware &         1.000 &      33 \\
%blihan              &    backdoor &         1.000 &       0 \\
%bloored             &        worm &         1.000 &       0 \\
%bayrob              &     spyware &         1.000 &       1 \\
%carberp             &  downloader &         1.000 &       6 \\
%broskod             &  downloader &         1.000 &       1 \\
%brresmon            &    grayware &         1.000 &       1 \\
%bsymem              &    grayware &         1.000 &      25 \\
%bublik              &  downloader &         1.000 &      79 \\
%bundlore            &      adware &         1.000 &       2 \\
%bunitu              &    grayware &         1.000 &       0 \\
%burn                &    backdoor &         1.000 &       0 \\
%buterat             &     clicker &         1.000 &      17 \\
%bzub                &  downloader &         1.000 &       2 \\
%cambot              &        worm &         1.000 &       0 \\
%betload             &    grayware &         1.000 &       3 \\
%cardspy             &     spyware &         1.000 &     100 \\
%brook               &  ransomware &         1.000 &       0 \\
%berbew              &    backdoor &         1.000 &       0 \\
%memery              &       virus &         1.000 &       9 \\
%bebloh              &    backdoor &         1.000 &       1 \\
%catalina            &    grayware &         1.000 &       0 \\
%ludbaruma           &        worm &         1.000 &       0 \\
%lulusoftware        &    grayware &         1.000 &      13 \\
%lunastorm           &        worm &         1.000 &      97 \\
%mimail              &        worm &         1.000 &       0 \\
%lydra               &     spyware &         1.000 &       5 \\
%lyposit             &  ransomware &         1.000 &       0 \\
%mabezat             &        worm &         1.000 &       0 \\
%zygug               &    grayware &         1.000 &      52 \\
%maener              &       miner &         1.000 &       0 \\
%mailru              &    grayware &         1.000 &       0 \\
%benjamin            &        worm &         1.000 &      94 \\
%malex               &  downloader &         1.000 &       1 \\
%mapsgory            &  downloader &         1.000 &       0 \\
%mbrlock             &  downloader &         1.000 &       2 \\
%mepaow              &       virus &         1.000 &      97 \\
%mewsspy             &    backdoor &         1.000 &       6 \\
%miancha             &  downloader &         1.000 &      11 \\
%fraudrop            &  downloader &         1.000 &      31 \\
%fosniw              &  downloader &         1.000 &      42 \\
%formbook            &     spyware &         1.000 &      37 \\
%lovgate             &        worm &         1.000 &      29 \\
%bobax               &        worm &         1.000 &       3 \\
%beastdoor           &    backdoor &         1.000 &      71 \\
%manbat              &        worm &         1.000 &      44 \\
%bcryptinject        &  downloader &         1.000 &       0 \\
%ceatrg              &  downloader &         1.000 &      66 \\
%flystudio           &    grayware &         1.000 &      34 \\
%balrok              &        worm &         1.000 &       0 \\
%banbra              &    backdoor &         1.000 &      50 \\
%cerber              &  ransomware &         1.000 &      16 \\
%chrop               &      adware &         1.000 &      63 \\
%esecn               &      adware &         1.000 &     100 \\
%downer              &    grayware &         1.000 &      96 \\
%downloadguide       &    grayware &         1.000 &       0 \\
%downloadsponsor     &    grayware &         1.000 &      87 \\
%dpsk                &       miner &         1.000 &     100 \\
%drolnux             &        worm &         1.000 &       0 \\
%dropware            &      adware &         1.000 &       1 \\
%bactera             &        worm &         1.000 &       0 \\
%drstwex             &  downloader &         1.000 &       0 \\
%duote               &      adware &         1.000 &      98 \\
%eggnog              &        worm &         1.000 &       0 \\
%egroupdial          &    grayware &         1.000 &      87 \\
%emogen              &  downloader &         1.000 &      31 \\
%enosch              &        worm &         1.000 &       0 \\
%esaprof             &  downloader &         1.000 &      84 \\
%esfury              &        worm &         1.000 &       5 \\
%clipspy             &    grayware &         1.000 &       2 \\
%eyestye             &  downloader &         1.000 &      27 \\
%eyoorun             &    grayware &         1.000 &       0 \\
%loring              &        worm &         1.000 &     100 \\
%dumpex              &    grayware &         1.000 &      40 \\
%babonock            &        worm &         1.000 &       0 \\
%azorult             &    grayware &         1.000 &      57 \\
%axespec             &    grayware &         1.000 &       2 \\
%amigo               &    grayware &         1.000 &       0 \\
%aauto               &    grayware &         1.000 &      56 \\
%adaebook            &    grayware &         1.000 &     100 \\
%addlyrics           &      adware &         1.000 &       0 \\
%adposhel            &    grayware &         1.000 &       0 \\
%aenjaris            &    grayware &         1.000 &       0 \\
%agentb              &    grayware &         1.000 &      13 \\
%agentc              &    grayware &         1.000 &       0 \\
%agentcrypt          &    grayware &         1.000 &       0 \\
%agentino            &  downloader &         1.000 &       0 \\
%agobot              &    backdoor &         1.000 &      32 \\
%airinstaller        &    grayware &         1.000 &      87 \\
%aitinject           &    backdoor &         1.000 &       1 \\
%allaple             &        worm &         1.000 &       0 \\
%alyak               &  downloader &         1.000 &      55 \\
%antavmu             &        worm &         1.000 &       0 \\
%bandit              &  downloader &         1.000 &      61 \\
%apost               &  downloader &         1.000 &       1 \\
%ardamax             &    grayware &         1.000 &       1 \\
%arkeistealer        &    grayware &         1.000 &       6 \\
%ascentive           &    grayware &         1.000 &       0 \\
%atcpa               &       virus &         1.000 &       0 \\
%autinject           &        tool &         1.000 &       4 \\
%autog               &  downloader &         1.000 &       0 \\
%autoitinject        &    backdoor &         1.000 &      16 \\
%banload             &       miner &         1.000 &      49 \\
%lolojan             &    grayware &         1.000 &      95 \\
%hupigon             &    backdoor &         1.000 &      59 \\
%lolbot              &        worm &         1.000 &       0 \\
%kuluoz              &  downloader &         1.000 &       4 \\
%laqma               &    grayware &         1.000 &      12 \\
%horst               &    backdoor &         1.000 &      30 \\
%hematite            &       virus &         1.000 &       1 \\
%ldpinch             &        worm &         1.000 &      47 \\
%gigex               &        worm &         1.000 &       0 \\
%frosparf            &     clicker &         1.000 &      73 \\
%fsysna              &        worm &         1.000 &      66 \\
%gamania             &  downloader &         1.000 &       5 \\
%gamarue             &  downloader &         1.000 &      10 \\
%gamemodding         &    grayware &         1.000 &       6 \\
%ksdler              &      adware &         1.000 &      69 \\
%gametea             &  downloader &         1.000 &     100 \\
%gandcrypt           &  ransomware &         1.000 &       2 \\
%garrun              &    grayware &         1.000 &       0 \\
%gatak               &    grayware &         1.000 &      11 \\
%gator               &      adware &         1.000 &      47 \\
%geegly              &       virus &         1.000 &     100 \\
%gepys               &  downloader &         1.000 &       0 \\
%gify                &    grayware &         1.000 &       0 \\
%gimemo              &  ransomware &         1.000 &      54 \\
%hebogo              &    grayware &         1.000 &       0 \\
%gippers             &  ransomware &         1.000 &       0 \\
%goabeny             &    grayware &         1.000 &      75 \\
%gandcrab            &  ransomware &         1.000 &       3 \\
%gobot               &    backdoor &         1.000 &       4 \\
%kronosbot           &    backdoor &         1.000 &       0 \\
%kovter              &    grayware &         1.000 &       0 \\
%houndhack           &    backdoor &         1.000 &     100 \\
%ezsoftwareupdater   &    grayware &         1.000 &       0 \\
%icedid              &  downloader &         1.000 &       4 \\
%inbox               &    grayware &         1.000 &       0 \\
%injuke              &    grayware &         1.000 &      70 \\
%installiq           &    grayware &         1.000 &       2 \\
%istartsurf          &    grayware &         1.000 &       1 \\
%itorrent            &    grayware &         1.000 &       0 \\
%jevafus             &  downloader &         1.000 &      45 \\
%jimmy               &  ransomware &         1.000 &       0 \\
%johnnie             &    grayware &         1.000 &      65 \\
%krol                &        worm &         1.000 &       0 \\
%jpgiframe           &    grayware &         1.000 &       0 \\
%keydoor             &     spyware &         1.000 &       0 \\
%killwin             &    grayware &         1.000 &      58 \\
%hider               &    grayware &         1.000 &      59 \\
%klez                &        worm &         1.000 &       1 \\
%klone               &  downloader &         1.000 &      70 \\
%kolab               &    grayware &         1.000 &       1 \\
%kolovorot           &    grayware &         1.000 &       3 \\
%kolweb              &    grayware &         1.000 &      96 \\
%konus               &    backdoor &         1.000 &       0 \\
%koobface            &        worm &         1.000 &      77 \\
%koutodoor           &    backdoor &         1.000 &      16 \\
%kasidet             &    grayware &         1.000 &      19 \\
%gofot               &  downloader &         1.000 &      88 \\
%gogo                &       virus &         1.000 &       0 \\
%goldrv              &        worm &         1.000 &       0 \\
%ngrbot              &    backdoor &         1.000 &       0 \\
%nitol               &  downloader &         1.000 &       7 \\
%noon                &     spyware &         1.000 &      19 \\
%nymaim              &    backdoor &         1.000 &       1 \\
%nymeria             &    grayware &         1.000 &      32 \\
%oberal              &  downloader &         1.000 &       0 \\
%obit                &  downloader &         1.000 &       0 \\
%offend              &    grayware &         1.000 &      98 \\
%offercore           &    grayware &         1.000 &       0 \\
%onlinegames         &    grayware &         1.000 &      14 \\
%openinstall         &    grayware &         1.000 &      22 \\
%nevereg             &        worm &         1.000 &      51 \\
%orcus               &    backdoor &         1.000 &      42 \\
%mint                &    grayware &         1.000 &       3 \\
%mimdau              &  downloader &         1.000 &     100 \\
%lecna               &    backdoor &         1.000 &      24 \\
%lunam               &        worm &         1.000 &      99 \\
%lethic              &  downloader &         1.000 &       1 \\
%linkury             &    grayware &         1.000 &       0 \\
%lipler              &  downloader &         1.000 &       0 \\
%llac                &        worm &         1.000 &      30 \\
%lmir                &       virus &         1.000 &       2 \\
%locky               &  downloader &         1.000 &       5 \\
%lokibot             &    backdoor &         1.000 &      88 \\
%outbyte             &    grayware &         1.000 &       4 \\
%miniduke            &    backdoor &         1.000 &       0 \\
%neutrinopos         &  ransomware &         1.000 &      12 \\
%neshta              &       virus &         1.000 &       2 \\
%gorillaprice        &      adware &         1.000 &       6 \\
%gozi                &     spyware &         1.000 &       1 \\
%grenam              &       virus &         1.000 &       0 \\
%griptolo            &        worm &         1.000 &       7 \\
%growtopia           &    grayware &         1.000 &      69 \\
%hafen               &    grayware &         1.000 &       0 \\
%havex               &    grayware &         1.000 &       0 \\
%hebchengjiu         &      adware &         1.000 &       7 \\
%lassorm             &        worm &         1.000 &       0 \\
%lebreat             &    grayware &         1.000 &     100 \\
%pahooka             &        worm &         1.000 &     100 \\
%netstream           &    grayware &         1.000 &       0 \\
%mira                &        worm &         1.000 &      23 \\
%moarider            &        worm &         1.000 &      98 \\
%mocrt               &     spyware &         1.000 &      73 \\
%moonlight           &        worm &         1.000 &       8 \\
%mooqkel             &  downloader &         1.000 &     100 \\
%msposer             &  downloader &         1.000 &       4 \\
%mulinex             &  downloader &         1.000 &      65 \\
%multibar            &    grayware &         1.000 &      23 \\
%multiplier          &    grayware &         1.000 &       0 \\
%mydoom              &        worm &         1.000 &      83 \\
%neconyd             &     clicker &         1.000 &       0 \\
%neojit              &  downloader &         1.000 &       0 \\
%neoreklami          &      adware &         1.000 &       0 \\
%lollipop            &      adware &         1.000 &       0 \\
%facido              &  downloader &         1.000 &       0 \\
%winwrapper          &    grayware &         1.000 &      44 \\
%fakefire            &        worm &         1.000 &       3 \\
%vilsel              &  downloader &         1.000 &      41 \\
%virbox              &    grayware &         1.000 &       1 \\
%virfire             &        worm &         1.000 &       0 \\
%virlock             &       virus &         1.000 &       0 \\
%vittalia            &    grayware &         1.000 &       3 \\
%vobfus              &        worm &         1.000 &       0 \\
%voltar              &     clicker &         1.000 &       0 \\
%vundo               &        worm &         1.000 &      89 \\
%wenper              &        worm &         1.000 &      42 \\
%vybab               &        worm &         1.000 &       0 \\
%wabot               &    backdoor &         1.000 &      19 \\
%viking              &       virus &         1.000 &      98 \\
%wacatac             &    grayware &         1.000 &      14 \\
%wannacry            &  ransomware &         1.000 &       1 \\
%wapomi              &       virus &         1.000 &       8 \\
%wavipeg             &    backdoor &         1.000 &       0 \\
%webalta             &    grayware &         1.000 &       5 \\
%webcompanion        &    grayware &         1.000 &       0 \\
%webdialer           &      dialer &         1.000 &      86 \\
%webprefix           &  downloader &         1.000 &      12 \\
%wecod               &     spyware &         1.000 &      94 \\
%scrarev             &    grayware &         1.000 &       5 \\
%spyeye              &     spyware &         1.000 &       9 \\
%schoolgirl          &    grayware &         1.000 &       2 \\
%waldek              &    grayware &         1.000 &       0 \\
%rikihaki            &        worm &         1.000 &       0 \\
%vemply              &    grayware &         1.000 &      61 \\
%vbinder             &        tool &         1.000 &       1 \\
%xcnfe               &    grayware &         1.000 &       0 \\
%xiaobaminer         &       miner &         1.000 &      74 \\
%xihet               &  downloader &         1.000 &       8 \\
%xolxo               &        worm &         1.000 &      99 \\
%xorer               &       virus &         1.000 &      11 \\
%xtrat               &    backdoor &         1.000 &      58 \\
%winloadsda          &    grayware &         1.000 &      86 \\
%yahlover            &        worm &         1.000 &      65 \\
%ymacco              &    grayware &         1.000 &      68 \\
%yoddos              &    backdoor &         1.000 &      19 \\
%zapchast            &  downloader &         1.000 &       4 \\
%vebzenpak           &    grayware &         1.000 &       0 \\
%zbot                &  downloader &         1.000 &      24 \\
%zenlod              &  downloader &         1.000 &      22 \\
%zepfod              &        worm &         1.000 &       0 \\
%zeroaccess          &  downloader &         1.000 &      10 \\
%zlob                &  downloader &         1.000 &      21 \\
%zusy                &    grayware &         1.000 &       7 \\
%zvuzona             &    grayware &         1.000 &      79 \\
%winner              &    grayware &         1.000 &       0 \\
%winactivator        &        tool &         1.000 &       6 \\
%vawtrak             &    backdoor &         1.000 &       0 \\
%vtflooder           &  downloader &         1.000 &      55 \\
%vbclone             &  downloader &         1.000 &       0 \\
%zegost              &    backdoor &         1.000 &      28 \\
%qbot                &  downloader &         1.000 &       0 \\
%qihoo               &    grayware &         1.000 &       3 \\
%qqhack              &        tool &         1.000 &      98 \\
%pirminay            &  downloader &         1.000 &       9 \\
%pistolar            &  downloader &         1.000 &      76 \\
%plingky             &  downloader &         1.000 &      93 \\
%protux              &    backdoor &         1.000 &       4 \\
%pluto               &        worm &         1.000 &      34 \\
%pochi               &        worm &         1.000 &       0 \\
%pondfull            &        worm &         1.000 &       5 \\
%pornoblocker        &  ransomware &         1.000 &      14 \\
%powerless           &  downloader &         1.000 &      10 \\
%powerspider         &    backdoor &         1.000 &       0 \\
%prepscram           &    grayware &         1.000 &     100 \\
%pincav              &    backdoor &         1.000 &      18 \\
%privateexeprotector &    grayware &         1.000 &      19 \\
%rifdoor             &    backdoor &         1.000 &       0 \\
%4shared             &    grayware &         1.000 &       1 \\
%rostpay             &    grayware &         1.000 &      96 \\
%roxer               &    grayware &         1.000 &      97 \\
%ruledor             &    backdoor &         1.000 &       0 \\
%safebytes           &    grayware &         1.000 &      99 \\
%ruskill             &    backdoor &         1.000 &       7 \\
%scarsi              &  downloader &         1.000 &       8 \\
%salgorea            &  downloader &         1.000 &       0 \\
%rums                &  downloader &         1.000 &       0 \\
%rungbu              &        worm &         1.000 &      18 \\
%prorat              &    backdoor &         1.000 &      69 \\
%pidgeon             &       virus &         1.000 &      16 \\
%picsys              &        worm &         1.000 &     100 \\
%phorpiex            &        worm &         1.000 &      12 \\
%qqpass              &  downloader &         1.000 &      94 \\
%qqware              &    grayware &         1.000 &      50 \\
%qshell              &    grayware &         1.000 &       0 \\
%qzonit              &     spyware &         1.000 &      87 \\
%raccrypt            &  ransomware &         1.000 &       0 \\
%ramnit              &       virus &         1.000 &      50 \\
%ranumbot            &    grayware &         1.000 &      28 \\
%rebhip              &        worm &         1.000 &      16 \\
%recex               &       miner &         1.000 &      11 \\
%recodrop            &  downloader &         1.000 &       0 \\
%reconyc             &    grayware &         1.000 &      36 \\
%redcap              &  downloader &         1.000 &      18 \\
%refinka             &    grayware &         1.000 &       1 \\
%regrun              &  ransomware &         1.000 &     100 \\
%regsup              &  downloader &         1.000 &       1 \\
%reline              &     spyware &         1.000 &      57 \\
%renos               &  downloader &         1.000 &      17 \\
%reptile             &    grayware &         1.000 &       0 \\
%qaccel              &     clicker &         1.000 &       0 \\
%pykspa              &        worm &         1.000 &       0 \\
%pterodo             &  downloader &         1.000 &       0 \\
%plemood             &        worm &         1.000 &       0 \\
%paneidix            &    backdoor &         1.000 &       0 \\
%pedex               &    backdoor &         1.000 &     100 \\
%phishbank           &        worm &         1.000 &       0 \\
%xanfpezes           &    grayware &         1.000 &     100 \\
%fakedoc             &        worm &         1.000 &       0 \\
%wsgame              &    grayware &         1.000 &      49 \\
%wonton              &  downloader &         1.000 &       2 \\
%daws                &  downloader &         1.000 &       2 \\
%detroie             &       virus &         1.000 &       2 \\
%dexel               &  downloader &         1.000 &       0 \\
%deyma               &  downloader &         1.000 &      21 \\
%diamin              &      dialer &         1.000 &      80 \\
%digs                &       miner &         1.000 &       0 \\
%dinwod              &  downloader &         1.000 &      98 \\
%diss                &    grayware &         1.000 &       3 \\
%disttrack           &    grayware &         1.000 &       0 \\
%dluca               &  downloader &         1.000 &      84 \\
%dodiw               &    backdoor &         1.000 &      83 \\
%darkkomet           &    backdoor &         1.000 &      14 \\
%dofoil              &  downloader &         1.000 &       6 \\
%pacex               &       miner &         1.000 &       0 \\
%macri               &    backdoor &         1.000 &      58 \\
%pajetbin            &        worm &         1.000 &      91 \\
%schwarzesonne       &    backdoor &         1.000 &      48 \\
%stampado            &  ransomware &         1.000 &      58 \\
%startsurf           &    grayware &         1.000 &       3 \\
%stelega             &    grayware &         1.000 &      50 \\
%stihat              &        worm &         1.000 &     100 \\
%stone               &       virus &         1.000 &      97 \\
%stop                &  ransomware &         1.000 &       0 \\
%stopcrypt           &  ransomware &         1.000 &       0 \\
%domaiq              &    grayware &         1.000 &      53 \\
%stormattack         &  downloader &         1.000 &       0 \\
%dalexis             &  downloader &         1.000 &       0 \\
%cyfin               &    grayware &         1.000 &       5 \\
%fakefolder          &        worm &         1.000 &      71 \\
%fakens              &      adware &         1.000 &       0 \\
%fareit              &    grayware &         1.000 &      14 \\
%fasong              &        worm &         1.000 &      78 \\
%fesber              &        worm &         1.000 &      33 \\
%ficker              &    grayware &         1.000 &      15 \\
%finfish             &    backdoor &         1.000 &       0 \\
%firseria            &    grayware &         1.000 &       8 \\
%dostre              &  downloader &         1.000 &       0 \\
%dorv                &  downloader &         1.000 &       8 \\
%dorifel             &  downloader &         1.000 &       5 \\
%dadobra             &  downloader &         1.000 &      63 \\
%dorgam              &  downloader &         1.000 &      19 \\
%connectwise         &        tool &         1.000 &       0 \\
%contenedor          &       virus &         1.000 &       0 \\
%convertad           &      adware &         1.000 &       1 \\
%coroxy              &    backdoor &         1.000 &       0 \\
%cosmicduke          &    backdoor &         1.000 &       4 \\
%cosmu               &       virus &         1.000 &       0 \\
%cossta              &    grayware &         1.000 &      13 \\
%cryptinject         &  downloader &         1.000 &      34 \\
%cryptowall          &  ransomware &         1.000 &       2 \\
%cutwail             &  downloader &         1.000 &      10 \\
%cycbot              &    backdoor &         1.000 &       0 \\
%cmy3u               &    grayware &         1.000 &      98 \\
%subseven            &    backdoor &         1.000 &      69 \\
%swizzor             &    grayware &         1.000 &       6 \\
%swjoy               &    grayware &         1.000 &       0 \\
%shyape              &  downloader &         1.000 &      39 \\
%shylock             &    backdoor &         1.000 &       3 \\
%sillyp2p            &        worm &         1.000 &       1 \\
%simda               &    backdoor &         1.000 &      11 \\
%soulclose           &        worm &         1.000 &       0 \\
%sinau               &       virus &         1.000 &       0 \\
%siscos              &  downloader &         1.000 &      48 \\
%sivis               &       virus &         1.000 &      72 \\
%skintrim            &      adware &         1.000 &       2 \\
%snojan              &  downloader &         1.000 &       1 \\
%socelars            &     spyware &         1.000 &       1 \\
%shodi               &       virus &         1.000 &       0 \\
%socks               &        worm &         1.000 &      18 \\
%softpulse           &    grayware &         1.000 &      34 \\
%sohana              &  downloader &         1.000 &      18 \\
%sohanad             &        worm &         1.000 &      66 \\
%solmyr              &     spyware &         1.000 &      58 \\
%unruy               &  downloader &         1.000 &      13 \\
%upantix             &  downloader &         1.000 &     100 \\
%updane              &    grayware &         1.000 &       0 \\
%xorist              &  downloader &         1.000 &      24 \\
%winwebsec           &   rogueware &         1.000 &      21 \\
%killmbr             &    grayware &         1.000 &       9 \\
%wlksm               &       virus &         1.000 &       1 \\
%softonic            &      adware &         1.000 &      79 \\
%shiz                &    backdoor &         1.000 &       0 \\
%shipup              &  downloader &         1.000 &      30 \\
%shade               &  ransomware &         1.000 &       5 \\
%syncopate           &    grayware &         1.000 &      20 \\
%sysn                &  downloader &         1.000 &      19 \\
%sytro               &        worm &         1.000 &      45 \\
%taskloader          &      adware &         1.000 &       0 \\
%tempedreve          &       virus &         1.000 &       0 \\
%teslacrypt          &  ransomware &         1.000 &       1 \\
%tiggre              &    grayware &         1.000 &      88 \\
%tinba               &  downloader &         1.000 &      15 \\
%titirez             &    grayware &         1.000 &       5 \\
%tnega               &  downloader &         1.000 &      33 \\
%tofsee              &    backdoor &         1.000 &       9 \\
%tonmye              &    grayware &         1.000 &      84 \\
%tougle              &    grayware &         1.000 &       0 \\
%trymedia            &      adware &         1.000 &      73 \\
%tupym               &        worm &         1.000 &       0 \\
%umbra               &  downloader &         1.000 &       0 \\
%spybot              &        worm &         1.000 &      15 \\
%spesr               &        worm &         1.000 &       0 \\
%speedcat            &    grayware &         1.000 &       0 \\
%simbot              &    backdoor &         1.000 &       1 \\
%screenmate          &    grayware &         1.000 &       1 \\
%sdbot               &    backdoor &         1.000 &      92 \\
%seraph              &  downloader &         1.000 &       0 \\
%seven               &  ransomware &         1.000 &       0 \\
%sfone               &        worm &         1.000 &      65 \\
%woozlist            &    grayware &         1.000 &      90 \\
%zylom               &  downloader &         1.000 &       0 \\
\bottomrule
\end{tabular}
\end{table}

% \begin{table}[h]
\centering
\small
\caption{Top-10 families with the lowest family classification accuracy using
	dynamic features
(i.e., highest mispredictions to other families)}
\label{tbl:multiclass_dyn_bestAndWorst}
\begin{tabular}{llrrr}
\toprule
\multicolumn{5}{c}{\textbf{Dynamic family classification}}\\
\textbf{Family} & \textbf{Class} &  \textbf{Avg F1} & \textbf{\% packed} & \textbf{FMR} \\
\midrule
bancos      	&  spyware      &   0.0  	& 44\%	& 0.76	\\
kovter        	&  grayware     &   0.0 	& 0\%	& 0.78	\\
safebytes     	&  grayware     &   0.0 	& 99\%	& 0.80	\\
winner      	&  grayware     &   0.0 	& 0\%	& 0.80	\\
umbra         	&  downloader   &   0.0 	& 0\%	& 0.80	\\
ulise        	&  grayware     &   0.0 	& 2\%	& 0.80	\\
contenedor      &  virus    	&   0.0  	& 0\% 	& 0.80  \\
cobra     		&  grayware 	&   0.0 	& 60\% 	& 0.79	\\
kuaizip         &  adware   	&   0.0 	& 1\%	& 0.80	\\
zpevdo          &  grayware 	&   0.0 	& 15\%	& 0.77	\\	
\bottomrule
\end{tabular}
\end{table}


% \begin{table}[h]
\centering
\small
\caption{Top-10 families with the lowest family classification accuracy 
(i.e., highest mispredictions to other families)}
\label{tbl:multiclass_combined_bestAndWorst}
\begin{tabular}{llrr}
\toprule
\multicolumn{4}{c}{\textbf{Combined family classification}}\\
\textbf{Family} & \textbf{Class} &  \textbf{Avg F1} & \textbf{\% packed} \\
\midrule
zpevdo              &    grayware &         0.031 &      15\% \\
sdbot               &    backdoor &         0.160 &      92\% \\
alman               &       virus &         0.175 &      11\% \\
dumpex              &    grayware &         0.259 &      40\% \\
vitro               &       virus &         0.267 &       3\% \\
uwamson             &    grayware &         0.275 &      15\% \\
copidmbe            &       virus &         0.312 &       9\% \\
fugrafa             &    grayware &         0.314 &       8\% \\
sality              &       virus &         0.326 &       4\% \\
refroso             &    backdoor &         0.350 &      41\% \\
%bulz                &    grayware &         0.364 &      15 \\
%bifrose             &    backdoor &         0.376 &      33 \\
%spybot              &        worm &         0.400 &      15 \\
%ldpinch             &        worm &         0.400 &      47 \\
%tenga               &       virus &         0.412 &       6 \\
%pioneer             &       virus &         0.424 &       6 \\
%vbinder             &        tool &         0.430 &       1 \\
%wsgame              &    grayware &         0.432 &      49 \\
%pasta               &    grayware &         0.432 &      28 \\
%cobra               &    grayware &         0.442 &      60 \\
%gendal              &    grayware &         0.442 &      62 \\
%presenoker          &    grayware &         0.446 &      16 \\
%bancos              &     spyware &         0.447 &      44 \\
%poison              &    backdoor &         0.456 &      24 \\
%coins               &    grayware &         0.463 &      18 \\
%bestafera           &    grayware &         0.467 &      57 \\
%parite              &       virus &         0.471 &       7 \\
%sdum                &    grayware &         0.473 &      24 \\
%shutdowner          &    grayware &         0.475 &       6 \\
%llac                &        worm &         0.478 &      30 \\
%atraps              &    grayware &         0.484 &      20 \\
%geral               &  downloader &         0.486 &      23 \\
%menti               &        worm &         0.488 &      11 \\
%klone               &  downloader &         0.500 &      70 \\
%diztakun            &    grayware &         0.518 &      27 \\
%cerber              &  ransomware &         0.525 &      16 \\
%buzy                &        worm &         0.533 &      35 \\
%alien               &  downloader &         0.540 &      20 \\
%kiser               &    grayware &         0.547 &      55 \\
%neconyd             &     clicker &         0.560 &       0 \\
%pcclient            &    backdoor &         0.565 &      30 \\
%bredolab            &  downloader &         0.567 &      19 \\
%tibia               &    grayware &         0.568 &      31 \\
%noon                &     spyware &         0.570 &      19 \\
%dynamer             &  downloader &         0.576 &      20 \\
%emogen              &  downloader &         0.579 &      31 \\
%siscos              &  downloader &         0.588 &      48 \\
%ekstak              &    grayware &         0.589 &      31 \\
%speedingupmypc      &    grayware &         0.600 &      11 \\
%lethic              &  downloader &         0.611 &       1 \\
%ramnit              &       virus &         0.611 &      50 \\
%carberp             &  downloader &         0.612 &       6 \\
%diskwriter          &    grayware &         0.612 &      29 \\
%tdss                &  downloader &         0.613 &       6 \\
%dorgam              &  downloader &         0.613 &      19 \\
%virut               &       virus &         0.621 &       3 \\
%flystudio           &    grayware &         0.621 &      34 \\
%occamy              &    grayware &         0.624 &      26 \\
%midie               &    grayware &         0.625 &      33 \\
%convagent           &    grayware &         0.630 &      23 \\
%jacard              &    grayware &         0.632 &      34 \\
%kasidet             &    grayware &         0.632 &      19 \\
%bingoml             &    grayware &         0.632 &      22 \\
%farfli              &    backdoor &         0.633 &      38 \\
%gimemo              &  ransomware &         0.635 &      54 \\
%kraddare            &      adware &         0.638 &      36 \\
%zegost              &    backdoor &         0.638 &      28 \\
%vemply              &    grayware &         0.640 &      61 \\
%locky               &  downloader &         0.640 &       5 \\
%constructor         &    grayware &         0.643 &      13 \\
%quasar              &    backdoor &         0.644 &      48 \\
%fraudpack           &   rogueware &         0.644 &       3 \\
%scarsi              &  downloader &         0.650 &       8 \\
%predator            &    grayware &         0.653 &      13 \\
%injuke              &    grayware &         0.653 &      70 \\
%ruskill             &    backdoor &         0.653 &       7 \\
%mucc                &    grayware &         0.660 &       0 \\
%zard                &    grayware &         0.662 &       3 \\
%slugin              &       virus &         0.663 &       4 \\
%macri               &    backdoor &         0.663 &      58 \\
%oficla              &    backdoor &         0.667 &      36 \\
%tnega               &  downloader &         0.667 &      33 \\
%dorkbot             &        worm &         0.667 &       7 \\
%sabsik              &    grayware &         0.671 &      30 \\
%induc               &       virus &         0.671 &      56 \\
%titirez             &    grayware &         0.671 &       5 \\
%netwiredrc          &    backdoor &         0.674 &      24 \\
%mudrop              &  downloader &         0.674 &      37 \\
%clipbanker          &    grayware &         0.674 &      11 \\
%fraudrop            &  downloader &         0.680 &      31 \\
%emotet              &  downloader &         0.682 &       1 \\
%winwebsec           &   rogueware &         0.682 &      21 \\
%rbot                &        worm &         0.683 &      19 \\
%cridex              &    grayware &         0.684 &      15 \\
%jaik                &    grayware &         0.684 &       5 \\
%babar               &    grayware &         0.693 &       5 \\
%cossta              &    grayware &         0.695 &      13 \\
%hupigon             &    backdoor &         0.695 &      59 \\
%prorat              &    backdoor &         0.700 &      69 \\
%duote               &      adware &         0.700 &      98 \\
%expiro              &       virus &         0.705 &      10 \\
%ursnif              &       virus &         0.705 &      25 \\
%ymacco              &    grayware &         0.705 &      68 \\
%brresmon            &    grayware &         0.709 &       1 \\
%onlinegames         &    grayware &         0.710 &      14 \\
%waldek              &    grayware &         0.710 &       0 \\
%usteal              &    grayware &         0.710 &      31 \\
%cerbu               &    grayware &         0.712 &       4 \\
%nanocore            &    backdoor &         0.716 &      22 \\
%wacatac             &    grayware &         0.716 &      14 \\
%nymaim              &    backdoor &         0.716 &       1 \\
%wlksm               &       virus &         0.718 &       1 \\
%pornoblocker        &  ransomware &         0.722 &      14 \\
%polypatch           &    grayware &         0.722 &      99 \\
%citeary             &        worm &         0.723 &      16 \\
%sillyfdc            &        worm &         0.725 &      55 \\
%rozena              &    backdoor &         0.725 &       3 \\
%gamarue             &  downloader &         0.726 &      10 \\
%crysis              &  ransomware &         0.729 &      27 \\
%xtrat               &    backdoor &         0.729 &      58 \\
%stelega             &    grayware &         0.730 &      50 \\
%sysn                &  downloader &         0.730 &      19 \\
%zusy                &    grayware &         0.730 &       7 \\
%racealer            &    grayware &         0.733 &      32 \\
%kelihos             &    backdoor &         0.733 &      12 \\
%bebloh              &    backdoor &         0.733 &       1 \\
%ulise               &    grayware &         0.733 &       2 \\
%zurgop              &  downloader &         0.740 &       3 \\
%lotok               &    backdoor &         0.740 &      15 \\
%simda               &    backdoor &         0.741 &      11 \\
%stop                &  ransomware &         0.741 &       0 \\
%ursu                &    grayware &         0.743 &      23 \\
%buterat             &     clicker &         0.747 &      17 \\
%banbra              &    backdoor &         0.750 &      50 \\
%spyeye              &     spyware &         0.750 &       9 \\
%rebhip              &        worm &         0.750 &      16 \\
%spesr               &        worm &         0.750 &       0 \\
%zeroaccess          &  downloader &         0.756 &      10 \\
%garrun              &    grayware &         0.758 &       0 \\
%buzus               &  downloader &         0.758 &       3 \\
%deyma               &  downloader &         0.760 &      21 \\
%cryptinject         &  downloader &         0.760 &      34 \\
%bladabindi          &    backdoor &         0.765 &      19 \\
%zlob                &  downloader &         0.767 &      21 \\
%emotetu             &  downloader &         0.768 &       5 \\
%blackhole           &    backdoor &         0.768 &      32 \\
%killwin             &    grayware &         0.770 &      58 \\
%tibs                &  downloader &         0.773 &      29 \\
%elex                &      adware &         0.775 &       9 \\
%fareit              &    grayware &         0.778 &      14 \\
%gamevance           &      adware &         0.778 &       5 \\
%palevo              &    backdoor &         0.778 &      22 \\
%renos               &  downloader &         0.779 &      17 \\
%tonmye              &    grayware &         0.779 &      84 \\
%dadobra             &  downloader &         0.779 &      63 \\
%pincav              &    backdoor &         0.779 &      18 \\
%yoddos              &    backdoor &         0.779 &      19 \\
%bsymem              &    grayware &         0.780 &      25 \\
%dorifel             &  downloader &         0.780 &       5 \\
%xorist              &  downloader &         0.783 &      24 \\
%blamon              &    grayware &         0.789 &      33 \\
%orcus               &    backdoor &         0.790 &      42 \\
%qqware              &    grayware &         0.800 &      50 \\
%autinject           &        tool &         0.800 &       4 \\
%kuluoz              &  downloader &         0.800 &       4 \\
%startsurf           &    grayware &         0.800 &       3 \\
%asparnet            &    grayware &         0.800 &       5 \\
%kuaizip             &      adware &         0.800 &       1 \\
%ipamor              &    grayware &         0.800 &      39 \\
%dupzom              &  downloader &         0.800 &      14 \\
%mokes               &    backdoor &         0.800 &       1 \\
%burden              &      adware &         0.800 &       2 \\
%cutwail             &  downloader &         0.800 &      10 \\
%jatif               &    grayware &         0.800 &       2 \\
%gamini              &    grayware &         0.810 &       0 \\
%shopper             &    grayware &         0.811 &       1 \\
%fsysna              &        worm &         0.811 &      66 \\
%fujacks             &       virus &         0.811 &      35 \\
%ngrbot              &    backdoor &         0.811 &       0 \\
%vittalia            &    grayware &         0.811 &       3 \\
%iobit               &    grayware &         0.812 &      38 \\
%killmbr             &    grayware &         0.813 &       9 \\
%stopcrypt           &  ransomware &         0.814 &       0 \\
%vebzenpak           &    grayware &         0.820 &       0 \\
%doina               &    grayware &         0.820 &      50 \\
%staser              &  downloader &         0.821 &      61 \\
%nymeria             &    grayware &         0.821 &      32 \\
%deceptpcclean       &    grayware &         0.821 &       4 \\
%conduit             &    grayware &         0.821 &      35 \\
%utorrent            &    grayware &         0.821 &      72 \\
%passview            &    grayware &         0.821 &      35 \\
%gatak               &    grayware &         0.821 &      11 \\
%blackmoon           &    grayware &         0.827 &      74 \\
%dorv                &  downloader &         0.827 &       8 \\
%hacdef              &    backdoor &         0.829 &      21 \\
%dlhelper            &    grayware &         0.830 &      13 \\
%teslacrypt          &  ransomware &         0.832 &       1 \\
%windigo             &    grayware &         0.832 &       3 \\
%qbot                &  downloader &         0.832 &       0 \\
%remcos              &    backdoor &         0.832 &       4 \\
%agentb              &    grayware &         0.832 &      13 \\
%fakefire            &        worm &         0.833 &       3 \\
%skillis             &    grayware &         0.833 &      19 \\
%amonetize           &    grayware &         0.835 &       7 \\
%vbkryjetor          &  downloader &         0.840 &       4 \\
%lazy                &    grayware &         0.840 &       6 \\
%voltar              &     clicker &         0.840 &       0 \\
%tofsee              &    backdoor &         0.840 &       9 \\
%redcap              &  downloader &         0.840 &      18 \\
%avemaria            &     spyware &         0.842 &      12 \\
%glupteba            &    grayware &         0.842 &       1 \\
%xyligan             &    backdoor &         0.842 &      55 \\
%fragtor             &    grayware &         0.842 &      10 \\
%swizzor             &    grayware &         0.842 &       6 \\
%installmonster      &    grayware &         0.844 &      83 \\
%skintrim            &      adware &         0.847 &       2 \\
%strab               &  ransomware &         0.850 &      38 \\
%installcore         &    grayware &         0.850 &      10 \\
%spigot              &    grayware &         0.850 &      20 \\
%autoitinject        &    backdoor &         0.850 &      16 \\
%diss                &    grayware &         0.850 &       3 \\
%ficker              &    grayware &         0.850 &      15 \\
%balrok              &        worm &         0.850 &       0 \\
%dofoil              &  downloader &         0.853 &       6 \\
%trickster           &  downloader &         0.853 &       1 \\
%tiggre              &    grayware &         0.853 &      88 \\
%petya               &  ransomware &         0.853 &      16 \\
%bandit              &  downloader &         0.853 &      61 \\
%pterodo             &  downloader &         0.855 &       0 \\
%shade               &  ransomware &         0.856 &       5 \\
%lovgate             &        worm &         0.856 &      29 \\
%gandcrab            &  ransomware &         0.856 &       3 \\
%lollipop            &      adware &         0.857 &       0 \\
%brook               &  ransomware &         0.860 &       0 \\
%zapchast            &  downloader &         0.860 &       4 \\
%bunitu              &    grayware &         0.860 &       0 \\
%miancha             &  downloader &         0.860 &      11 \\
%tasker              &    grayware &         0.860 &      11 \\
%stihat              &        worm &         0.863 &     100 \\
%plugx               &    backdoor &         0.863 &       5 \\
%dalexis             &  downloader &         0.863 &       0 \\
%woozlist            &    grayware &         0.863 &      90 \\
%schoolboy           &    grayware &         0.863 &       7 \\
%cryptowall          &  ransomware &         0.863 &       2 \\
%kovter              &    grayware &         0.867 &       0 \\
%hpdefender          &    grayware &         0.870 &       0 \\
%manbat              &        worm &         0.870 &      44 \\
%cyfin               &    grayware &         0.870 &       5 \\
%yandex              &    grayware &         0.871 &       0 \\
%mytob               &        worm &         0.874 &      30 \\
%driverpack          &    grayware &         0.878 &      13 \\
%sodinokibi          &    grayware &         0.878 &       2 \\
%raccrypt            &  ransomware &         0.878 &       0 \\
%omaneat             &     spyware &         0.880 &       7 \\
%onescan             &    grayware &         0.880 &      38 \\
%vundo               &        worm &         0.880 &      89 \\
%malex               &  downloader &         0.880 &       1 \\
%wonton              &  downloader &         0.880 &       2 \\
%bandoo              &    grayware &         0.884 &       0 \\
%chifrax             &    grayware &         0.884 &      41 \\
%neutrinopos         &  ransomware &         0.884 &      12 \\
%gamania             &  downloader &         0.884 &       5 \\
%schoolgirl          &    grayware &         0.884 &       2 \\
%darkkomet           &    backdoor &         0.884 &      14 \\
%mabezat             &        worm &         0.888 &       0 \\
%johnnie             &    grayware &         0.889 &      65 \\
%binder              &  downloader &         0.889 &      16 \\
%multiplier          &    grayware &         0.890 &       0 \\
%reline              &     spyware &         0.890 &      57 \\
%bactera             &        worm &         0.890 &       0 \\
%turkojan            &    backdoor &         0.890 &      16 \\
%zbot                &  downloader &         0.890 &      24 \\
%bobax               &        worm &         0.890 &       3 \\
%guloader            &  downloader &         0.890 &       0 \\
%softcnapp           &      adware &         0.893 &       0 \\
%kolovorot           &    grayware &         0.895 &       3 \\
%zenlod              &  downloader &         0.895 &      22 \\
%azorult             &    grayware &         0.895 &      57 \\
%shylock             &    backdoor &         0.895 &       3 \\
%netstream           &    grayware &         0.900 &       0 \\
%subseven            &    backdoor &         0.900 &      69 \\
%techsnab            &    grayware &         0.900 &       2 \\
%istartsurf          &    grayware &         0.900 &       1 \\
%msposer             &  downloader &         0.900 &       4 \\
%jimmy               &  ransomware &         0.900 &       0 \\
%formbook            &     spyware &         0.900 &      37 \\
%opensupdater        &      adware &         0.900 &      14 \\
%lightstone          &    backdoor &         0.900 &      17 \\
%bitmin              &       miner &         0.900 &      25 \\
%lamer               &       virus &         0.900 &      43 \\
%ranumbot            &    grayware &         0.900 &      28 \\
%solmyr              &     spyware &         0.900 &      58 \\
%yahlover            &        worm &         0.905 &      65 \\
%mepaow              &       virus &         0.910 &      97 \\
%bezigate            &    backdoor &         0.910 &      15 \\
%urelas              &    backdoor &         0.910 &      38 \\
%speedbit            &    grayware &         0.910 &       8 \\
%cobaltstrike        &  downloader &         0.910 &       3 \\
%stampado            &  ransomware &         0.910 &      58 \\
%trickbot            &       miner &         0.910 &       0 \\
%opencandy           &    grayware &         0.910 &      92 \\
%agobot              &    backdoor &         0.911 &      32 \\
%beastdoor           &    backdoor &         0.911 &      71 \\
%chir                &        worm &         0.912 &      12 \\
%relevantknowledge   &      adware &         0.914 &       0 \\
%lockbit             &  ransomware &         0.916 &       0 \\
%esfury              &        worm &         0.916 &       5 \\
%khalesi             &    grayware &         0.920 &       2 \\
%scrop               &  downloader &         0.920 &      12 \\
%refinka             &    grayware &         0.920 &       1 \\
%blackie             &    grayware &         0.920 &       0 \\
%vawtrak             &    backdoor &         0.920 &       0 \\
%wajam               &      adware &         0.920 &       2 \\
%rostpay             &    grayware &         0.920 &      96 \\
%sohanad             &        worm &         0.920 &      66 \\
%webprefix           &  downloader &         0.922 &      12 \\
%mbrlock             &  downloader &         0.922 &       2 \\
%privateexeprotector &    grayware &         0.922 &      19 \\
%systweak            &    grayware &         0.923 &      19 \\
%growtopia           &    grayware &         0.925 &      69 \\
%gandcrypt           &  ransomware &         0.926 &       2 \\
%pirminay            &  downloader &         0.926 &       9 \\
%hotbar              &    grayware &         0.926 &      45 \\
%winwrapper          &    grayware &         0.926 &      44 \\
%xorer               &       virus &         0.926 &      11 \\
%taranis             &    grayware &         0.929 &       4 \\
%hematite            &       virus &         0.929 &       1 \\
%hider               &    grayware &         0.929 &      59 \\
%wapomi              &       virus &         0.929 &       8 \\
%shandaadd           &      adware &         0.929 &      88 \\
%mulinex             &  downloader &         0.930 &      65 \\
%schwarzesonne       &    backdoor &         0.930 &      48 \\
%winactivator        &        tool &         0.933 &       6 \\
%qjwmonkey           &    grayware &         0.933 &      47 \\
%ardamax             &    grayware &         0.933 &       1 \\
%nitol               &  downloader &         0.937 &       7 \\
%xiaobaminer         &       miner &         0.937 &      74 \\
%dodiw               &    backdoor &         0.940 &      83 \\
%disttrack           &    grayware &         0.940 &       0 \\
%speedcat            &    grayware &         0.940 &       0 \\
%bublik              &  downloader &         0.940 &      79 \\
%koobface            &        worm &         0.940 &      77 \\
%kuaiba              &      adware &         0.940 &       5 \\
%silentall           &    grayware &         0.940 &      77 \\
%explorerhijack      &  downloader &         0.940 &       1 \\
%auslogics           &    grayware &         0.941 &       3 \\
%installerex         &    grayware &         0.944 &       0 \\
%icedid              &  downloader &         0.947 &       4 \\
%sogou               &    grayware &         0.947 &      17 \\
%crossrider          &    grayware &         0.947 &       1 \\
%recex               &       miner &         0.950 &      11 \\
%seraph              &  downloader &         0.950 &       0 \\
%firseria            &    grayware &         0.950 &       8 \\
%adpack              &      adware &         0.950 &       2 \\
%snojan              &  downloader &         0.950 &       1 \\
%plingky             &  downloader &         0.950 &      93 \\
%upatre              &  downloader &         0.950 &       4 \\
%adgazelle           &    grayware &         0.950 &       0 \\
%necurs              &  downloader &         0.950 &       8 \\
%filefinder          &      adware &         0.950 &       3 \\
%scrarev             &    grayware &         0.950 &       5 \\
%wannacry            &  ransomware &         0.953 &       1 \\
%gozi                &     spyware &         0.958 &       1 \\
%gofot               &  downloader &         0.958 &      88 \\
%cjishu              &      adware &         0.958 &       1 \\
%klez                &        worm &         0.958 &       1 \\
%filetour            &    grayware &         0.958 &       4 \\
%socelars            &     spyware &         0.958 &       1 \\
%toptools            &    grayware &         0.958 &       0 \\
%browsefox           &    grayware &         0.958 &       6 \\
%fakefolder          &        worm &         0.958 &      71 \\
%multibar            &    grayware &         0.958 &      23 \\
%betload             &    grayware &         0.958 &       3 \\
%frosparf            &     clicker &         0.960 &      73 \\
%daws                &  downloader &         0.960 &       2 \\
%virbox              &    grayware &         0.960 &       1 \\
%vilsel              &  downloader &         0.960 &      41 \\
%miniduke            &    backdoor &         0.960 &       0 \\
%horst               &    backdoor &         0.960 &      30 \\
%eyestye             &  downloader &         0.960 &      27 \\
%arkeistealer        &    grayware &         0.960 &       6 \\
%tougle              &    grayware &         0.960 &       0 \\
%loadmoney           &    grayware &         0.960 &       8 \\
%agentc              &    grayware &         0.960 &       0 \\
%pcacceleratepro     &    grayware &         0.960 &       1 \\
%woool               &    grayware &         0.960 &      94 \\
%downloadsponsor     &    grayware &         0.960 &      87 \\
%shiz                &    backdoor &         0.960 &       0 \\
%veil                &    grayware &         0.960 &       1 \\
%wecod               &     spyware &         0.960 &      94 \\
%xanfpezes           &    grayware &         0.960 &     100 \\
%koutodoor           &    backdoor &         0.960 &      16 \\
%cheatengine         &    grayware &         0.964 &       0 \\
%fesber              &        worm &         0.964 &      33 \\
%winner              &    grayware &         0.967 &       0 \\
%imali               &      adware &         0.968 &       0 \\
%webdialer           &      dialer &         0.968 &      86 \\
%xolxo               &        worm &         0.968 &      99 \\
%webalta             &    grayware &         0.968 &       5 \\
%4shared             &    grayware &         0.968 &       1 \\
%lokibot             &    backdoor &         0.968 &      88 \\
%downloadadmin       &    grayware &         0.968 &       3 \\
%dealply             &    grayware &         0.968 &      19 \\
%detrahere           &    grayware &         0.968 &       4 \\
%paneidix            &    backdoor &         0.970 &       0 \\
%vrbrothers          &      adware &         0.970 &      88 \\
%swisyn              &        worm &         0.970 &       7 \\
%vobfus              &        worm &         0.970 &       0 \\
%cmy3u               &    grayware &         0.970 &      98 \\
%dropware            &      adware &         0.970 &       1 \\
%convertad           &      adware &         0.970 &       1 \\
%spyrix              &    grayware &         0.970 &      56 \\
%winloadsda          &    grayware &         0.970 &      86 \\
%smartpcsolutions    &    grayware &         0.970 &      46 \\
%tremp               &    grayware &         0.970 &       0 \\
%qzonit              &     spyware &         0.970 &      87 \\
%cosmu               &       virus &         0.970 &       0 \\
%sohana              &  downloader &         0.970 &      18 \\
%tupym               &        worm &         0.970 &       0 \\
%qihoo               &    grayware &         0.970 &       3 \\
%lmir                &       virus &         0.970 &       2 \\
%mocrt               &     spyware &         0.970 &      73 \\
%prepscram           &    grayware &         0.970 &     100 \\
%kolweb              &    grayware &         0.970 &      96 \\
%benjamin            &        worm &         0.970 &      94 \\
%regsup              &  downloader &         0.970 &       1 \\
%linkury             &    grayware &         0.971 &       0 \\
%pajetbin            &        worm &         0.971 &      91 \\
%bundlore            &      adware &         0.973 &       2 \\
%remoteutilities     &    grayware &         0.975 &      90 \\
%mywebsearch         &    grayware &         0.976 &       1 \\
%cycbot              &    backdoor &         0.976 &       0 \\
%screenmate          &    grayware &         0.976 &       1 \\
%kolab               &    grayware &         0.978 &       1 \\
%mint                &    grayware &         0.978 &       3 \\
%sillyp2p            &        worm &         0.979 &       1 \\
%wakme               &  downloader &         0.979 &       0 \\
%zygug               &    grayware &         0.979 &      52 \\
%tinba               &  downloader &         0.979 &      15 \\
%pidgeon             &       virus &         0.980 &      16 \\
%egroupdial          &    grayware &         0.980 &      87 \\
%shipup              &  downloader &         0.980 &      30 \\
%pistolar            &  downloader &         0.980 &      76 \\
%agentino            &  downloader &         0.980 &       0 \\
%neoreklami          &      adware &         0.980 &       0 \\
%simbot              &    backdoor &         0.980 &       1 \\
%phorpiex            &        worm &         0.980 &      12 \\
%gorillaprice        &      adware &         0.980 &       6 \\
%alyak               &  downloader &         0.980 &      55 \\
%seven               &  ransomware &         0.980 &       0 \\
%moarider            &        worm &         0.980 &      98 \\
%agentcrypt          &    grayware &         0.980 &       0 \\
%offend              &    grayware &         0.980 &      98 \\
%qshell              &    grayware &         0.980 &       0 \\
%offergenerator      &    grayware &         0.980 &       0 \\
%banload             &       miner &         0.980 &      49 \\
%bloored             &        worm &         0.980 &       0 \\
%bayrob              &     spyware &         0.980 &       1 \\
%dumpy               &        worm &         0.980 &       0 \\
%cardspy             &     spyware &         0.980 &     100 \\
%mediaget            &    grayware &         0.980 &      33 \\
%webcompanion        &    grayware &         0.980 &       0 \\
%idsohtu             &    grayware &         0.980 &       6 \\
%ceatrg              &  downloader &         0.980 &      66 \\
%gepys               &  downloader &         0.980 &       0 \\
%aauto               &    grayware &         0.980 &      56 \\
%reconyc             &    grayware &         0.980 &      36 \\
%sytro               &        worm &         0.980 &      45 \\
%allaple             &        worm &         0.987 &       0 \\
%ibryte              &      adware &         0.988 &       9 \\
%laqma               &    grayware &         0.988 &      12 \\
%unruy               &  downloader &         0.988 &      13 \\
%dexel               &  downloader &         0.989 &       0 \\
%macoute             &        worm &         0.989 &       0 \\
%fosniw              &  downloader &         0.989 &      42 \\
%jevafus             &  downloader &         0.989 &      45 \\
%axespec             &    grayware &         0.989 &       2 \\
%mira                &        worm &         0.989 &      23 \\
%fasong              &        worm &         0.989 &      78 \\
%antavmu             &        worm &         0.989 &       0 \\
%softpulse           &    grayware &         0.989 &      34 \\
%zvuzona             &    grayware &         0.989 &      79 \\
%virlock             &       virus &         0.989 &       0 \\
%lecna               &    backdoor &         0.989 &      24 \\
%installiq           &    grayware &         0.989 &       2 \\
%lipler              &  downloader &         0.989 &       0 \\
%pykspa              &        worm &         0.989 &       0 \\
%openinstall         &    grayware &         0.989 &      22 \\
%virfire             &        worm &         0.989 &       0 \\
%installbrain        &    grayware &         0.989 &       2 \\
%viking              &       virus &         0.989 &      98 \\
%reptile             &    grayware &         0.990 &       0 \\
%brontok             &        worm &         0.990 &      95 \\
%bcryptinject        &  downloader &         0.990 &       0 \\
%lydra               &     spyware &         0.990 &       5 \\
%aenjaris            &    grayware &         0.990 &       0 \\
%gify                &    grayware &         0.990 &       0 \\
%adaebook            &    grayware &         0.990 &     100 \\
%megasearch          &      adware &         0.990 &       0 \\
%lolbot              &        worm &         0.990 &       0 \\
%downloadassistant   &    grayware &         0.990 &      18 \\
%outbyte             &    grayware &         0.990 &       4 \\
%kronosbot           &    backdoor &         0.990 &       0 \\
%chrop               &      adware &         0.990 &      63 \\
%pacex               &       miner &         0.990 &       0 \\
%blackshades         &        worm &         0.990 &       2 \\
%mydoom              &        worm &         0.990 &      83 \\
%perion              &    grayware &         0.990 &       0 \\
%xcnfe               &    grayware &         0.990 &       0 \\
%dinwod              &  downloader &         0.990 &      98 \\
%plurox              &    backdoor &         0.990 &       0 \\
%connectwise         &        tool &         0.990 &       0 \\
%gator               &      adware &         0.990 &      47 \\
%hafen               &    grayware &         0.990 &       0 \\
%detroie             &       virus &         0.990 &       2 \\
%slimware            &    grayware &         0.990 &       0 \\
%dluca               &  downloader &         0.990 &      84 \\
%inbox               &    grayware &         0.990 &       0 \\
%hebchengjiu         &      adware &         0.990 &       7 \\
%vkontaktedj         &    grayware &         0.990 &       4 \\
%dstudio             &    grayware &         0.990 &       2 \\
%broskod             &  downloader &         0.990 &       1 \\
%anmalpro            &    grayware &         0.990 &       0 \\
%apost               &  downloader &         0.990 &       1 \\
%protux              &    backdoor &         0.990 &       4 \\
%gippers             &  ransomware &         0.990 &       0 \\
%autog               &  downloader &         0.990 &       0 \\
%pochi               &        worm &         1.000 &       0 \\
%drstwex             &  downloader &         1.000 &       0 \\
%ksdler              &      adware &         1.000 &      69 \\
%konus               &    backdoor &         1.000 &       0 \\
%getnow              &    grayware &         1.000 &      88 \\
%geegly              &       virus &         1.000 &     100 \\
%swjoy               &    grayware &         1.000 &       0 \\
%gamemodding         &    grayware &         1.000 &       6 \\
%syncopate           &    grayware &         1.000 &      20 \\
%finfish             &    backdoor &         1.000 &       0 \\
%fakens              &      adware &         1.000 &       0 \\
%fakedoc             &        worm &         1.000 &       0 \\
%mimdau              &  downloader &         1.000 &     100 \\
%stormattack         &  downloader &         1.000 &       0 \\
%qaccel              &     clicker &         1.000 &       0 \\
%pluto               &        worm &         1.000 &      34 \\
%moonlight           &        worm &         1.000 &       8 \\
%phishbank           &        worm &         1.000 &       0 \\
%picsys              &        worm &         1.000 &     100 \\
%obit                &  downloader &         1.000 &       0 \\
%oberal              &  downloader &         1.000 &       0 \\
%nevereg             &        worm &         1.000 &      51 \\
%plemood             &        worm &         1.000 &       0 \\
%primecasino         &    grayware &         1.000 &       0 \\
%neshta              &       virus &         1.000 &       2 \\
%gigex               &        worm &         1.000 &       0 \\
%neojit              &  downloader &         1.000 &       0 \\
%pondfull            &        worm &         1.000 &       5 \\
%powerless           &  downloader &         1.000 &      10 \\
%powerspider         &    backdoor &         1.000 &       0 \\
%mooqkel             &  downloader &         1.000 &     100 \\
%qqpass              &  downloader &         1.000 &      94 \\
%taskloader          &      adware &         1.000 &       0 \\
%tisandr             &        worm &         1.000 &       0 \\
%tempedreve          &       virus &         1.000 &       0 \\
%pedex               &    backdoor &         1.000 &     100 \\
%grenam              &       virus &         1.000 &       0 \\
%griptolo            &        worm &         1.000 &       7 \\
%soulclose           &        worm &         1.000 &       0 \\
%havex               &    grayware &         1.000 &       0 \\
%hebogo              &    grayware &         1.000 &       0 \\
%softonic            &      adware &         1.000 &      79 \\
%houndhack           &    backdoor &         1.000 &     100 \\
%socks               &        worm &         1.000 &      18 \\
%icloader            &    grayware &         1.000 &       2 \\
%sivis               &       virus &         1.000 &      72 \\
%gobot               &    backdoor &         1.000 &       4 \\
%sinau               &       virus &         1.000 &       0 \\
%itorrent            &    grayware &         1.000 &       0 \\
%jeefo               &       virus &         1.000 &      99 \\
%shyape              &  downloader &         1.000 &      39 \\
%jpgiframe           &    grayware &         1.000 &       0 \\
%eyoorun             &    grayware &         1.000 &       0 \\
%expressdownloader   &    grayware &         1.000 &      62 \\
%esecn               &      adware &         1.000 &     100 \\
%esaprof             &  downloader &         1.000 &      84 \\
%enosch              &        worm &         1.000 &       0 \\
%elemental           &    grayware &         1.000 &       0 \\
%ezsoftwareupdater   &    grayware &         1.000 &       0 \\
%facido              &  downloader &         1.000 &       0 \\
%goabeny             &    grayware &         1.000 &      75 \\
%goldrv              &        worm &         1.000 &       0 \\
%sfone               &        worm &         1.000 &      65 \\
%kingsoft            &    grayware &         1.000 &      37 \\
%shodi               &       virus &         1.000 &       0 \\
%keydoor             &     spyware &         1.000 &       0 \\
%krol                &        worm &         1.000 &       0 \\
%offercore           &    grayware &         1.000 &       0 \\
%inlog               &  downloader &         1.000 &       0 \\
%mimail              &        worm &         1.000 &       0 \\
%cambot              &        worm &         1.000 &       0 \\
%bzub                &  downloader &         1.000 &       2 \\
%burn                &    backdoor &         1.000 &       0 \\
%blihan              &    backdoor &         1.000 &       0 \\
%digs                &       miner &         1.000 &       0 \\
%berbew              &    backdoor &         1.000 &       0 \\
%babonock            &        worm &         1.000 &       0 \\
%xihet               &  downloader &         1.000 &       8 \\
%atcpa               &       virus &         1.000 &       0 \\
%ascentive           &    grayware &         1.000 &       0 \\
%amigo               &    grayware &         1.000 &       0 \\
%youxun              &    grayware &         1.000 &      11 \\
%aitinject           &    backdoor &         1.000 &       1 \\
%airinstaller        &    grayware &         1.000 &      87 \\
%zepfod              &        worm &         1.000 &       0 \\
%adposhel            &    grayware &         1.000 &       0 \\
%addlyrics           &      adware &         1.000 &       0 \\
%triusor             &       virus &         1.000 &       0 \\
%drolnux             &        worm &         1.000 &       0 \\
%dostre              &  downloader &         1.000 &       0 \\
%tufik               &       virus &         1.000 &       0 \\
%dpsk                &       miner &         1.000 &     100 \\
%umbra               &  downloader &         1.000 &       0 \\
%wavipeg             &    backdoor &         1.000 &       0 \\
%domaiq              &    grayware &         1.000 &      53 \\
%updane              &    grayware &         1.000 &       0 \\
%downer              &    grayware &         1.000 &      96 \\
%trymedia            &      adware &         1.000 &      73 \\
%downloadguide       &    grayware &         1.000 &       0 \\
%pahooka             &        worm &         1.000 &     100 \\
%vbclone             &  downloader &         1.000 &       0 \\
%upantix             &  downloader &         1.000 &     100 \\
%qqhack              &        tool &         1.000 &      98 \\
%clipspy             &    grayware &         1.000 &       2 \\
%vybab               &        worm &         1.000 &       0 \\
%lassorm             &        worm &         1.000 &       0 \\
%ludbaruma           &        worm &         1.000 &       0 \\
%salgorea            &  downloader &         1.000 &       0 \\
%rungbu              &        worm &         1.000 &      18 \\
%ruledor             &    backdoor &         1.000 &       0 \\
%lolojan             &    grayware &         1.000 &      95 \\
%roxer               &    grayware &         1.000 &      97 \\
%loring              &        worm &         1.000 &     100 \\
%rising              &    grayware &         1.000 &      34 \\
%rikihaki            &        worm &         1.000 &       0 \\
%rifdoor             &    backdoor &         1.000 &       0 \\
%lulusoftware        &    grayware &         1.000 &      13 \\
%mewsspy             &    backdoor &         1.000 &       6 \\
%lunam               &        worm &         1.000 &      99 \\
%lunastorm           &        worm &         1.000 &      97 \\
%regrun              &  ransomware &         1.000 &     100 \\
%lyposit             &  ransomware &         1.000 &       0 \\
%vtflooder           &  downloader &         1.000 &      55 \\
%contenedor          &       virus &         1.000 &       0 \\
%coroxy              &    backdoor &         1.000 &       0 \\
%cosmicduke          &    backdoor &         1.000 &       4 \\
%diamin              &      dialer &         1.000 &      80 \\
%wenper              &        worm &         1.000 &      42 \\
%wabot               &    backdoor &         1.000 &      19 \\
%catalina            &    grayware &         1.000 &       0 \\
%eggnog              &        worm &         1.000 &       0 \\
%memery              &       virus &         1.000 &       9 \\
%mapsgory            &  downloader &         1.000 &       0 \\
%mailru              &    grayware &         1.000 &       0 \\
%maener              &       miner &         1.000 &       0 \\
%recodrop            &  downloader &         1.000 &       0 \\
%gogo                &       virus &         1.000 &       0 \\
%zylom               &  downloader &         1.000 &       0 \\
\bottomrule
\end{tabular}
\end{table}


%Let's first talk about static
%Static is difficult for grayware and virus (family and binary)
% Tables~\ref{tbl:binary_static_bestAndWorst} and~\ref{tbl:multiclass_static_bestAndWorst} 
% list the 10 malware families that
% have the lowest binary and family classification accuracy when only
% static features are used.
% All families belong to the

%Let's introduce dynamic

% The most common families mis-classified by dynamic features also include
% downloaders 

% and adware also appear in the worst
% classified list of dynamic features. Downloaders largely dominate the
% families that are most often confused with one another. This is
% understandable, as different downloaders and adwares may largely overlap in
% term of runtime behavior. 

% It is instead interesting to note that most of
% the malware families with the lowest detection accuracy have more than half
% of the dynamic features containing missing observations, as reported by the
% Feature Missing Rate (\textbf{FMR}) column In our study, \textbf{FMR}
% captures the fraction of malware samples with at least four types of
% dynamic analysis features containing missing feature values. This clearly
% underlines the negative effect of missing dynamic features over the
% precision of the classification -- which explains why, as mentioned in the
% previous Section, dynamic features are very effective in some cases but
% perform poorly on others. This is also confirmed by the difference in F1
% scores among static and dynamic features. For instance, for the worse
% families the binary classification ranges between 0.4 and 0.62 with static
% features, but remaining consistently at 0 with dynamic features .

% The only exception is \emph{gamania}, characterized by a FMR of 20\%, and yet producing 
% a low accuracy in binary classification.
% \fix{'gamania' is a trojan which steal the xxxx}
% \fix{We can add some discussion to explain why this downloader app
% is difficult to detect.} 
%On the other hand, the static binary classifier has
%perfect accuracy on one third of the families (231).

%Table~\ref{tbl:multiclass_static_bestAndWorst} shows the 10 malware families
%that are the most difficult to classify with the random forest-based family
%classifier using {the static analysis-based features}.  These are the families
%with the highest family misclassification rates.  Similar to the binary
%classification, viruses and grayware still dominate in the most
%hard-to-classify malware families. As discussed above, most virus and grayware
%samples are composed by a mix of benign and malicious codes. Therefore, it is
%difficult to find a clear boundary separating these families from other malware
%families.  

%Dynamic performance
% When focusing on the performance scores, we observe different behaviors compared 
% to the static results. Worst predicted families have a much lower accuracy when
% considering dynamic features for both binary and family classification. In the
% latter case, all of the samples of the listed families are misclassified to other
% families. Each of these families has missing feature values, marked as NULL or
% NA, in at least 4 of the 7 dynamic features of over 80\% of the malware
% samples. More specifically, the top 10 families listed in
% Table.\ref{tbl:multiclass_dyn_bestAndWorst} are the families \emph{salgorea},
% \emph{kuluoz}, \emph{winwebsec}, \emph{kraddare}, \emph{umbra}, \emph{unruy},
% \emph{bandit}, \emph{convagent}, \emph{coins} and \emph{lazy}. 
% In each of the families, there are 60\%, 99\%, 80\%, 68\%,
% 100\%, 97\%, 98\%, 100\%, 99\% and 98\% of malware samples with at least 4
% dynamic feature classes containing NULL / NA feature values. The results confirm
% that high missing feature rate brings significant drop to malware classification
% accuracy. The empty-valued feature vectors lead to the failure of classification
% over malware samples of these families. These 10 families are dominated by
% \emph{trojans}, which typically need to connect external C\&C.  Most probably the
% server was unreachable during our analysis resulting in missing features,

% \fix{We need to discuss why there is no overlap between dynamic and static}

%A note of the combined models
% Moreover, Tables~\ref{tbl:binary_combined_bestAndWorst} and~\ref{tbl:multiclass_combined_bestAndWorst} 
% report the top-10 mispredicted families when combining static and dynamic
% features. We observe such a list mainly overlapping with the worst-predicted
% families of the static case (Tables~\ref{tbl:binary_static_bestAndWorst} 
% and~\ref{tbl:multiclass_static_bestAndWorst}), suggesting that static features
% might be dominant when combined with dynamic ones. We will come to this aspect
% later in the manuscript when analysing the feature contribution in
% \S\ref{sec:feature_classes}.

% Finally, we study to which extent the presence of off-the-shelf packers and 
% protectors harms the classification accuracy.

% In each of the previous tables the column \emph{Packed} provides the fraction 
% of packed malware samples in each family. 
% While one might expect a high correlation 
% between packing and misclassification rate in static-analysis-based
% models, we find that most of the 10 families have low packing percentages, i.e.
% on average 20\%. For models that employ static features, the category of a family
% (e.g., \emph{virus}, \emph{graywares} ) seems to have more significant impact over the
% classification performance.  A possible explanation could be that in the static-analysis
% based feature space, the intra-class difference between malware samples
% belonging to the same family is much less than the inter-class difference
% between malware and goodware, and among different families. As a result, even if some 
% of samples may be packed, the gap between intra-class and inter-class samples 
% is still large enough to provide accurate classification. 

% On the other hand, packing presence is significantly
% higher when looking at the worst classified families with dynamic-analysis-based
% models. In this respect, a higher fraction of packed files together with the missing feature values 
% have a not-negligible negative impact over the classification accuracy. 
% Malware may hide its payload or remain inactive in dynamic analysis due to
% evasive techniques, failure of connection to C\&C servers, or lack of actions triggering malicious payloads.
% The missing feature values in the derived run-time behaviour feature
% representation inject artefacts into training / testing of the binary and family
% classifier, which eventually increases the misclassification rate. 

%\yufei{Shall we also explain why salgorea and kraddare have 0 classification
%accuracy? They don't have that high feature missing rates as the other 8
%families, but still suffering from very bad classification performances.}

% \sav{Need explanation here and a clarification. Can it be that there are other
% families that are not listed with a lower percentage of packing but still 0
% accuracy?}

\summary{6}{Models employing static features find it more difficult to classify
\emph{grayware} and \emph{viruses}. 
Dynamic features can identify ransomware, spyware, and adware as malware, but they
have great difficulty in properly identifying their families, probably due to 
very similar runtime behaviors of different families in these classes.}

