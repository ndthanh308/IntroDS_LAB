% mnras_template.tex 
%
% LaTeX template for creating an MNRAS paper
%
% v3.0 released 14 May 2015
% (version numbers match those of mnras.cls)
%
% Copyright (C) Royal Astronomical Society 2015
% Authors:
% Keith T. Smith (Royal Astronomical Society)

% Change loga
%
% v3.0 May 2022
%    Renamed to match the new package name
%    Version number matches mnras.cls
%    A few minor tweaks to wording
% v1.0 September 2013
%    Beta testing only - never publicly released
%    First version: a simple (ish) template for creating an MNRAS paper

%%%%%%%%%%%%%%%%%%%%%%%%%%%%%%%%%%%%%%%%%%%%%%%%%%
% Basic setup. Most papers should leave these options alone.
\documentclass[letters, fleqn,usenatbib]{mnras}

% MNRAS is set in Times font. If you don't have this installed (most LaTeX
% installations will be fine) or prefer the old Computer Modern fonts, comment
% out the following line
\usepackage{verbdef}% http://ctan.org/pkg/verbdef
\usepackage{siunitx}
\usepackage{gensymb}
\usepackage{comment}
\usepackage[normalem]{ulem}
\usepackage[svgnames]{xcolor}
\verbdef{\vtext}{verb text}
%
%
\usepackage{xcolor}
% Depending on your LaTeX fonts installation, you might get better results with one of these:
%\usepackage{mathptmx}
%\usepackage{txfonts}

% Use vector fonts, so it zooms properly in on-screen viewing software
% Don't change these lines unless you know what you are doing
\usepackage[T1]{fontenc}

% Allow "Thomas van Noord" and "Simon de Laguarde" and alike to be sorted by "N" and "L" etc. in the bibliography.
% Write the name in the bibliography as "\VAN{Noord}{Van}{van} Noord, Thomas"
\DeclareRobustCommand{\VAN}[3]{#2}
\let\VANthebibliography\thebibliography
\def\thebibliography{\DeclareRobustCommand{\VAN}[3]{##3}\VANthebibliography}


%%%%% AUTHORS - PLACE YOUR OWN PACKAGES HERE %%%%%

% Only include extra packages if you really need them. Common packages are:
\usepackage{graphicx}	% Including figure files
\usepackage{amsmath}	% Advanced maths commands
\usepackage{amssymb}	% Extra maths symbols
\usepackage{newtxtext,newtxmath}
\newcommand{\mage}[1]{\textcolor{magenta}{#1}}
%%%%%%%%%%%%%%%%%%%%%%%%%%%%%%%%%%%%%%%%%%%%%%%%%%

%%%%% AUTHORS - PLACE YOUR OWN COMMANDS HERE %%%%%

% Please keep new commands to a minimum, and use \newcommand not \def to avoid
% overwriting existing commands. Example:
%\newcommand{\pcm}{\,cm$^{-2}$}	% per cm-squared

%%%%%%%%%%%%%%%%%%%%%%%%%%%%%%%%%%%%%%%%%%%%%%%%%%

%%%%%%%%%%%%%%%%%%% TITLE PAGE %%%%%%%%%%%%%%%%%%%

% Title of the paper, and the short title which is used in the headers.
% Keep the title short and informative.
\title[The Impact of Cluster Mergers in the Red Sequence.]{Clash of Titans: The Impact of Cluster Mergers in the Galaxy Cluster Red Sequence.}
%\title[The impact of mergers in the cluster galaxy red sequence]{The impact of mergers in the cluster galaxy red sequence}

% The list of authors, and the short list which is used in the headers.
% If you need two or more lines of authors, add an extra line using \newauthor
\author[Franklin Aldás]{Franklin Aldás,$^{1}$\thanks{E-mail: franklin.aldas@userena.cl}
Alfredo Zenteno,$^{2}$
Facundo G\'omez,$^{1,3}$
Daniel Hernandez-Lang,$^{4,5}$\newauthor
%Joseph Mohr,$^{4}$
%Koshy George,$^{4}$
Eleazar R. Carrasco,$^{6}$
Cristian A. Vega-Martínez,$^{1,3}$ \&
J. L. Nilo Castellón,$^{1}$
\\
% List of institutions
$^{1}$Departamento de Astronomía, Universidad de La Serena, Avenida Juan Cisternas 1200, La Serena, Chile\\
$^{2}$Cerro Tololo Inter-American Observatory, NSF's NOIRLab, Casilla 603, La Serena 1700000, Chile\\
$^{3}$Instituto Multidisciplinario de Investigación y Postgrado, Universidad de La Serena, Raúl Bitrán 1305, La Serena, Chile \\
$^{4}$Faculty of Physics, Ludwig-Maximilians-Universit\"{a}t, Scheinerstr.\ 1, 81679 Munich, Germany \\
$^{5}$Excellence Cluster Origins, Boltzmannstr.\ 2, 85748 Garching, Germany\\
$^{6}$Gemini Observatory, NSF's NOIRLab, Casilla 603, La Serena 1700000, Chile.
}

% These dates will be filled out by the publisher
\date{Accepted XXX. Received YYY; in original form ZZZ}

% Enter the current year, for the copyright statements etc.
\pubyear{2023}

% Don't change these lines
\begin{document}
\label{firstpage}
\pagerange{\pageref{firstpage}--\pageref{lastpage}}
\maketitle

% Abstract of the paper
\begin{abstract}
%\textcolor{blue}{The environment where the galaxies are located plays an important role in the evolution and morphological changes of galaxies. The highly density environments like galaxy clusters can produce a increase of star formation rate or can quench the galaxies.} % The cluster merging process is one of the most energetic events that can enhance the star formation. In this work, we used the cluster red sequence study the differences in the galaxy population between relaxed and disturbed clusters.}
Merging of galaxy clusters are some of the most energetic events in the Universe, and they provide a unique environment to study galaxy evolution. We use a sample of 84 merging and relaxed  SPT galaxy clusters candidates, observed with the Dark Energy Camera in the $0.11<z<0.88$ redshift range, to build colour-magnitude diagrams to characterize the impact of cluster mergers on the galaxy population.
%We use a sample of 84 merging and relaxed galaxy SPT clusters, in the $0.11<z<0.88$ redshift range,  observed using the Dark Energy Camera.  We build color-magnitude diagrams to characterize the impact of cluster mergers on the galaxy population. 
%
% Describe our findings in fig 4.
We divided the sample between relaxed and disturbed, and in two redshifts bin at $z = 0.55$.
%
%\textcolor{red}{In the low-z bin we find a complete agreement between the relaxed and disturbed clusters.}
%
When comparing the high-z to low-z clusters we find the  high-z sample is  richer in blue galaxies, independently of the cluster dynamical state.
%
In the high-z bin we find that disturbed clusters exhibit a larger scatter in the Red Sequence, with wider distribution and an excess of bluer galaxies compared to relaxed clusters, while in the low-z bin we find a complete agreement between the relaxed and disturbed clusters.
%
%
%We find that disturbed clusters exhibit a larger scatter in the Red Sequence, particularly at $z>0.55$, with wider distribution and an excess of bluer galaxies compared to relaxed clusters. This difference support the scenario in which, as clusters merge star formation is triggered in cluster galaxies.
%
%\textcolor{red}{Our results} support the scenario in which, as clusters merge star formation is triggered in cluster galaxies... \textcolor{red}{in the high-z sample.}
%
Our results support the scenario in which massive cluster halos at $z<0.55$ galaxies are quenched as satellites of another structure, i.e. outside the cluster, while at $z \geq 0.55$ the quenching is dominated by in-situ processes.

%in a ex-situ groups or clusters, while at $z>0.55$ the quenching is dominated by in-situ processes.

%It should be a single paragraph not more than 250 words (200 words for Letters).
\end{abstract}

% Select between one and six entries from the list of approved keywords.
% Don't make up new ones.
\begin{keywords}
galaxies:clusters:general -- galaxies:clusters:evolution -- galaxies:evolution
\end{keywords}

%%%%%%%%%%%%%%%%%%%%%%%%%%%%%%%%%%%%%%%%%%%%%%%%%%

%%%%%%%%%%%%%%%%% BODY OF PAPER %%%%%%%%%%%%%%%%%%

\section{Introduction}

% Large scale results
%Galaxy evolution as a function of galaxy mass and density is a well known/established ... (many papers... Peng 2010).
%  Going into detail -- What these papers say a

% Turn to our science - What have been done comparing the population in clusters with different dynamical states

% \textcolor{red}{AZ: Puedo trabajar en esto en Febrero. Si JL pudiese escribirlo antes de salir de vacaciones yo podria trabajar en esto despues de el.}

%in context, cites relevant earlier studies in the field by \citet{Fournier1901},
%and describes the problem the authors aim to solve \citep[e.g.][]{vanDijk1902}.
%Multiple citations can be joined in a simple way like \citet{deLaguarde1903, delaGuarde1904}.

% GENERAL OVERVIEW ABOUT THE STUDIES OF THE RS EVOLUTION
% Summary of what we know about the RS
%\textcolor{red}{Observational studies as well as simulations seems to show that the red sequence (RS) in galaxy clusters is in place at least since z $\sim1$ (e.g., Chan et al. 2019; Trayford et al. 2016).}

% Why is the building of the RS interesting?

% DESCRIBE WHAT ARE WE GOING TO TAKLK ABOUT HERE.  HOW THE MERGER OF GALAXIES, AND THE SUBSEQUENT RELAXATION, CHANGE THE RED SEQUENCE
%
%Galaxy clusters are the largest gravitationally bounded structures of the Universe and typically contains  houndreds or thousands of luminous galaxies. They also contains hot gas that can radiate energy covering the whole electromagnetic spectrum from radio to X-rays emission~\citep{Brownstein2006}. The importance of groups and clusters is  because at least a half of all galaxies in the universe are found in galaxy groups or clusters~\citep{Munoz2012}. \\
%The formation of the structures in the Universe is hierarchical. Larger structures are formed by merging smaller halos~\citep{Muldrew2015}.   
Galaxy clusters are the largest collapsed gravitationally bound structures of the Universe and typically contain  hundreds or thousands of %luminous 
galaxies~\citep{Voit2005, Abell1954}. These clusters have grown to their present-day mass through the merger and accretion of neighbouring substructures. Galaxy clusters can be found in different dynamical states. In  a relaxed cluster, the Intra-Cluster Medium (ICM) is generally close to thermodynamic equilibrium. However, in clusters colliding with other massive structures, the released binding energy of the systems results in heating of their ICM \citep{Sarazin2002}. Cluster mergers are the most energetic events after the Big Bang and  can emit energies up to  $10^{64}$ erg \citep[][]{sarazin04, Kravtsov2012}. Indeed, observations in different wavelengths suggest that a significant fraction of clusters, between 30\% and 70\%,   are not fully virialized \citep{Dressler1988, Hou2012, Zhang2009,  Andrade-Santos2017}.   Such energetic cluster merging events can trigger and/or quench star formation in their galaxy members, thus affecting their observable properties such as colours.

 The dynamical state of galaxy clusters is described using different observational proxies. Among them we find the central cooling time~\citep{Bauer2005}, the concentration index which is the excess of the  surface brightness in the central part of the cluster measured in X-rays~\citep{Santos2008, Yuan2022}, the displacement between the the cluster X-ray surface brightness peak from its surface brightness centroid~\citep{Cassano2010, Yuan2022}, and, combinations of different observations such as the offset between the position of the brightest cluster galaxy (BCG) and the Sunyaev-Zeldovich (SZ) centroid~\citep{ Zenteno2020a}. Additionally, some quantities can be used  to study the dynamical state using cosmological simulations, for example, the virial ratio $\eta$ based on the virial theorem that measures the degree of deviation that a cluster has from an equilibrium state, the fraction of mass of the cluster contained in subhalos with respect to its total mass, and the offset between the centre of mass of cluster with the cluster centre, or a weighted combination of those parameters~\citep{Zang2022}.

%affecting  the observable properties of galaxies such as colors.


%As cosmic structures form hierarchically, they collapse in the Dark Matter (DM) potential wells and undergo merging events during their history \citep{White1991}
%, and whose effects can be observed from radio waves to X-rays~\citep{Feretti2002}. 
%

%Energetic cluster merging events can affect the physical properties of the galaxies inside those structures.  %Evidence of galaxy transformation, due to these massive merging events, has been mostly studied in detail on individual clusters or by using small samples at low redshift.

% Although there are some studies about the differences in the star formation rate between relaxed and disturbed clusters, there is no an agreement about if the star formation rate is stimulated or on the contrary, it is suppressed. Some authors  found that the star formation rate is a bit higher in interacting clusters compared to the relaxed ones~\citep{Yoon2020, Sobral2015}. However ~\cite{Shim2011} studied the interacting cluster A255 concluding that the merger tend to supress star formation rate.   On the other hand, ~\cite{Hou2012} showed that galaxies in groups with substructures have a significantly higher blue galaxies compared to groups with no detected  substructures. 

%  AZ: Other studies:
%
% Stroe 2017 "A large Hα survey of star formation in relaxed and merging galaxy cluster environments at z ∼ 0.15–0.3"
% Stroe 2020: "The First Integral Field Unit Spectroscopic View of Shocked Cluster Galaxies"
%
% Cohen 2017: "STAR FORMATION AND SUPERCLUSTER ENVIRONMENT OF 107 NEARBY GALAXY CLUSTERS"
% 107 clusters

% AZ: i) Now discuss studies with larger samples at low-z, then at ii) high-z where we are stronger. End discussin Zenteno to finish with red sequence at high-z
%
%
%The impact of mergers  in the galaxy population has been studied  using large \textcolor{red}{low-z}  samples, \textcolor{red}{providing} evidence that the luminosity  function between relaxed and disturbed clusters is different \citep[e.g.][]{Barrena2012,WenHan2015}. \cite{WenHan2015}, using 2092 rich clusters with SDSS data at redshift $a<0.42$, found that the characteristic magnitude of the luminosity function in relaxed clusters is 0.27 magnitude fainter compared to the disturbed clusters showing there are more bright member galaxies in more unrelaxed clusters. On the other hand ~\cite{Zenteno2020a} and ~\cite{DePropis2013} found that the luminosity function doesn't depend on the dynamical state of the clusters at  $z < 0.55$, and $z<0.6$, respectively. However, \cite{Zenteno2020a} \textcolor{red}{did find} a brighter $m^*$ in disturbed clusters  than relaxed clusters at $z \gtrsim 0.6$.  
%
%
%. Also,  term of the faint end slope $\alpha$ ~\cite{Zenteno2020a} finds that the luminosity functions differed between relaxed and disturbed clusters for redshifts $z\gtrsim 0.6$ and that the disturbed clusters have a steeper slope than relaxed clusters. 

% AZ: Transition need to be smoother
% AZ: This is about star formation.  Do we have a tool to measure it?  IS this part of the low-z or high-z discussion?

%
Today, a significant fraction of the members inhabiting these clusters consist of elliptical and lenticular galaxies \citep{Hubble1931, Oemler1974,  Dressler1980}. %Most of galaxies belonging to a cluster show elliptical and lenticular morphology. 
These early-type galaxies feature a well defined linear relation between their colour and magnitude \citep{Visvanathan1977, Bower1992, Kodama1997}, the so called red sequence (RS). Such galaxies in the cluster RS  have null or little ongoing star formation \citep[][]{Gladders2004,Gladders2007,DePropris2016}, and their colour evolution can be remarkably well described by simple evolutionary models \citep{Stanford1998}. In fact, such models are so successful that they have been used to identify clusters \citep{Gladders2000,Murphy2011,Bleem2015,rykoff16, Vakili2020} and to provide robust photometric redshifts up to $\sim$ 1.5  \citep[e.g.][]{Song2012, Bleem2015, bleem20}, with a precision better than $\sim 0.01 \times (1 + z)$ up to z $\sim$ 1.0 \citep[e.g.][]{rykoff16, klein18, klein19}. 
%
%While the evolution of the colors of the RS can be very well described simply as the aging of galaxies’ stars formed at $z\sim$2 - 3 (at least up to z $\sim$1.3, e.g., Mancone et al. 2010), the width or scatter provides information about the diversity of their stellar population ages. 
%
While the slope and zero-point evolution of the RS  describes the aging of galaxies’ stars formed at $z\sim$2 - 3 \citep[at least up to z $\sim$1.3, e.g.,][]{mancone10}, the  scatter provides information about its stellar population age diversity~\citep{Connor_2019}. 
%

Previous studies have shown that the scatter remains relatively constant up to z $\sim$1  \citep[e.g.][]{jaffe11, Hennig2017}. However, at $z\gtrsim 1.3$,  the RS is found to be wider and bluer, indicating that at such $z$ clusters are approaching their star formation epoch \citep{hilton09,papovich10,snyder12}.  Furthermore, \citet{brodwin13} suggest that $z \sim 1.4$ could be where clusters star formation activity ends and the era of passive evolution begins.

Previous works have studied the differences in the star formation rates of galaxies in disturbed clusters concerning their relaxed counterpart. Yet, there is no agreement about whether a galaxy's star formation rate is stimulated or suppressed  during large merger events.
%Although there are some studies about the differences in the star formation rate between relaxed and disturbed clusters , there is no agreement about if the star formation rate is stimulated or on the contrary, it is suppressed.
%
% AZ:  Below text need to be expanded. sample size, redshift range, cluster mass...
For example, \citet{Pranger2014} through an analysis of the galaxy population
in Abell 3921, and \cite{Kleiner2014} in A1750, found that  mergers quench star formation. \citet{Shim2011} also studied the interacting cluster A2255 finding that the merging process suppresses star formation and transforms galaxies into quiescent galaxies.
%
However, other authors such as \citet{Ferrari2003}, by an spectroscopic analysis of Abell 521, and \cite{Owers2012} by studying the cluster Abell 2744 (which is currently undergoing a major merger) found that the high-pressure merger environment triggers star formation. Supporting this,  ~\cite{Sobral2015} and ~\cite{Yoon2020} found that the star formation rate in interacting clusters is around ~20\% higher than the observed in  relaxed structures.  Such findings are confirmed using H$\alpha$ observations of disturbed and relaxed clusters at $z<0.4$ %\sout{by studies of clusters at $z<0.4$ using H$\alpha$ observations of disturbed and relaxed clusters,} 
finding a higher prevalence of H$\alpha$ emitter galaxies in disturbed clusters, within 2 Mpc from the cluster's center, than in relaxed clusters \citep{Stroe2017,stroe21}. In the same way, \citet{Hou2012} studying groups of galaxies %with 
at intermediate redshifts ($z\sim 0.4$)  show that galaxies in groups with substructures present a significantly higher blue galaxies population %with respect 
compared to galaxy groups with no detected substructures.

The goal of this work is to study the impact that the merging of clusters has on their red galaxy population. %We focus on a} \textcolor{red}{large} redshift range  dominated by passive evolution.
%
%another aspect of galaxy evolution in clusters; the impact of cluster mergers in triggering star formation.  
% This needs work
We compare the cluster galaxies RS of a sample of 84 clusters, both relaxed and disturbed, all within a wide redshift range  between 0.11$\leq z \leq $0.88.  The cluster sample was first introduced in \citet{Zenteno2020a}. % The data used for this work includes dedicated deep observations in $i-$ and $z-$ bands, in addition to Dark Energy Camera (DECam) archival data from the Dark Energy Survey (DES). 
The paper is organized as follow: in \S~2 we provide details of the observations and data reduction. In \S~3 we show the data calibration between DES and Munich pipeline reductions, while in \S~4 we report our findings. Finally in \S~5 are our conclusion. Throughout the paper we assume a flat Universe, with a $\Lambda$CDM cosmology, $h$ = 0.7, $\Omega_M$= 0.27 \citep{komatsu11}. %Within this cosmology, 1 arcsec corresponds to $\sim$6.66 kpc.

\begin{comment}
\newpage
Galaxy clusters are the largest gravitationally bounded structures of the Universe and typically contains  hundreds or thousands of luminous galaxies~\citep{Voit2005, Abell1954}.
%
As they form hierarchically they suffer minor and major mergers during their history. Clusters mergers are the most energetic events after the Big Bang and  can emit energies up to  $10^{64}$ erg \citep[][]{sarazin04, Kravtsov2012}. %, and whose effects can be observed from radio waves to X-rays~\citep{Feretti2002}. 
%
Those energetic merging events can affect the physical properties of the galaxies inside those structures. 
%
% AZ: Add studies in individual systems.  Look at Daniels paper and the references the referee provided.
\textcolor{red}{Evidence of galaxy transformation, due to the merging event, have been found in individual clusters as well as in ensembles of clusters.  For example, Poggianti et al. 2004 found that the merger quench star formation while  Ferrari et al. 2003 and Owers et al. 2012 found triggering star formation. Cohen et al (2017), Stroe et al (2017, 2020), Stroe \& Sobral (2021)}

% Stroe 2017 "A large Hα survey of star formation in relaxed and merging galaxy cluster environments at z ∼ 0.15–0.3"
% Stroe 2020: "The First Integral Field Unit Spectroscopic View of Shocked Cluster Galaxies"
%
% Cohen 2017: "STAR FORMATION AND SUPERCLUSTER ENVIRONMENT OF 107 NEARBY GALAXY CLUSTERS"
% 107 clusters

% AZ: i) Now discuss studies with larger samples at low-z, then at ii) high-z where we are stronger. End discussin Zenteno to finish with red sequence at high-z
\textcolor{red}{The impact in the galaxy population has been studied also using large samples, including} evidence that the luminosity  function between relaxed and disturbed clusters is different. ~\cite{WenHan2015}\textcolor{red}{, using 2092 rich clusters with SDSS data at redshift $a<0.42$,} found that the characteristic magnitude of the luminosity function in relaxed clusters is \st{one} \textcolor{red}{0.27} magnitude fainter compared to the disturbed clusters. On the other hand ~\cite{Zenteno2020a} found \textcolor{red}{no differences at $z < 0.55$, but a brighter $m^*$ in disturbed clusters at higher redshifts.  In term of the faint end slope $\alpha$ ~\cite{Zenteno2020a} finds} that the luminosity functions differed between relaxed and disturbed clusters for redshifts $z\gtrsim 0.6$ and that the disturbed clusters have a steeper slope than relaxed clusters. \\
% AZ: Transition need to be smoother
Although there are some studies about the differences in the star formation rate between relaxed and disturbed clusters, there is no an agreement about if the star formation rate is stimulated or on the contrary, it is suppressed. Some authors  found that the star formation rate is a bit higher in interacting clusters compared to the relaxed ones~\citep{Yoon2020, Sobral2015}. However ~\cite{Shim2011} studied the interacting cluster A255 concluding that the merger tend to supress star formation rate.   On the other hand, ~\cite{Hou2012} showed that galaxies in groups with substructures have a significantly higher blue galaxies compared to groups with no detected  substructures. \\
% 
\textcolor{red}{Most of galaxies belonging to a cluster are elliptical and lenticular population. Early-type galaxies have a well defined linear relation between their color and magnitude~\citep{Bower1992,Kodama1997}, the so called red sequence. The cluster red sequence evolution can be remarkably well described by simple evolutionary models (Stanford, Eisenhardt, & Dickinson 1998).  In fact, such models are so successful that they have been used to find clusters \citep{Gladders2000,rykoff16,Vakili2020,Murphy2011} and to provide robust photometric redshift up to  $\sim 1.5$ \citep[e.g.][]{Gladders2000, Koester2007,zenteno11, Song2012, Bleem2015,zenteno16,Hennig2017,bleem20}, with a precision better than $\delta$$z$ $\sim0.01 \times (1+z)$ up to $z\sim1.0$ \citep[e.g.][]{rykoff16,klein18,klein19}.}  
%
\textcolor{red}{While the evolution of the colors of the red sequence can be described simply as the aging of galaxies' stars formed at $z\sim 2-3$  well \citep[at least  up to $z\sim1.3$, e.g.,][]{mancone10}, the width or scatter provides information about the diversity of their stellar population ages. Previous studies have shown that this scatter remain relatively constant up to $z\sim1$  \citep[e.g. ][]{jaffe11,Hennig2017}.}
%
\textcolor{red}{At $z\gtrsim1.3$ the red sequence is found to be wider and bluer, an indication of approaching the cluster star formation epoch \citep{hilton09,papovich10,snyder12}.  Furthermore, \cite{brodwin13} suggest that at $z \sim 1.4$ may be the redshift at which the cluster formation activity ends and where the era of passive evolution begins.} 
%
%\textcolor{red}{} 
%
\textcolor{red}{In this letter we explore the impact of mergers in triggering star formation by analyzing the red sequence evolution and scatter.} 
%\textcolor{red}{In this letter we explore the scatter of the red sequence as a function of the cluster dynamical state, highlighting the role of mergers in the transformation of the galaxy cluster population.} 
%
%\textcolor{red}{}

%\textcolor{red}{AZ: We could follow \citep{hilton09} to explain when the star formation occurred.  We could assume a merger age and then see if that fits.} 
Red sequence is populated by early-type galaxies formed before $z>2$ \citep{Bower1992, Blakeslee2003, Mei2006}. The red sequence could be characterized by the zero point, slope and red scatter. Each of those parameters differs from one to the other cluster depending on age and mass-matalicity relation \citep{Bernardi2005, DeLucia2007, Faber2007}

Most of galaxies belonging to a cluster are elliptical and lenticular population. Early-type galaxies have a well defined linear relation between their color and magnitude~\citep{Kodama1996,Bower1992}. This relation has been used as a method to detect galaxy clusters searching for over-densities in the color-magnitude diagram~\citep{Gladders2000}\\
% Our work
In this paper we use optical imaging from DECam to explore the impact of cluster mergers in the galaxies they host. We studied the difference in the red sequence and the color density distribution between relaxed and disturbed clusters.
\newpage
\end{comment}

\section{Data}
The observations used in this paper  were carried out with the Dark Energy Camera \citep[DECam;][]{Flaugher2015}, a 570 Mega-pixels CCD, installed
% the primary focus on the 4m Victor M. Blanco  telescope at the  Cerro Tololo Inter-American Observatory. 
at the prime focus of the V. M. Blanco 4-meter Telescope at Cerro Tololo Inter-American Observatory (CTIO). The DECam 
%and the Blanco telescope have 
has a  field-of-view  of $2.2\degree$. %\textcolor{red}{The data consists in archival  (mainly DES) and proprietary data. We use archival data for }
We use two sets of data; %for the 65 clusters at $z\lesssim0.65$ we use
65 clusters ($0.11 \lesssim z\lesssim0.65$) comes from  the Dark Energy Survey public second data release  \citep[DES; ][]{DES2021}, while for  19 clusters at $z\gtrsim0.65$ we use archival data as well as data obtained using Director Discretionary Time (DDT). DDT was pooled with the DECam eROSITA Survey (DeROSITAS;  PI Zenteno) allocation,  taking advantage of the flexibility a large pool of nights provide to programs with different needs.  %\textcolor{red}{The observations of clusters at $z\gtrsim0.65$ were taken when the seeing was $\lesssim1$''.} 



%for 19 clusters with proprietary  for the 84  galaxy clusters studied. A}nother set of deeper imaging \textcolor{red}{and catalogs} for 19 clusters taken during the DECam eROSITA Survey (DeROSITAS,  PI Zenteno) observing nights. \textcolor{red}{The DeROSITAS imaging is then stacked including all archival data, which include DES images, creating deeper catalogs for 19 of the 84 clusters.}
%

%\textcolor{red}{Deeper observations are needed for the clusters in the highest redshift bin. } 
%$r\sim23.7$, $i\sim23.0$, and $z\sim22.27$. Meanwhile the DeROSITAS observations reaches magnitudes around:  $i\sim 23.9$, and $z\sim 23.6$. The final DES cluster sample is presented in the Table~\ref{DES}, and the DeROSITA sample is presented in the Table~\ref{DeROSITA}. 

DES is a 5000 square degree optical survey using the DECam and 5 filters \verb|g,r,i,z,Y|, covering a wavelength range from $400$ nm to $1065$ nm.  The Data Release 2 (DR2) of DES is the result of six years of observations (2013-2019) collecting information of around 700 million galactic and extragalactic sources \citep{DES2021}.  The image reduction  and processing for the DES sample set were done by the DESDM system. This process performs flat-fielding, bad-pixel masking, overscan removal, masking of cosmic rays and artificial satellites, and other image corrections \citep{Morganson2018}. Once the images are fully reduced, the pipeline performs a fitting with PSFEx and source detection using Source Extractor generating the %COADD
co-added images and its associated source catalogues that are ready for science analysis \citep{DES2021, Bertin2011, Bertin1996}.  %We obtained the sources catalogues 
We retrieved the source catalogues from the DES DR2 repository. 
The  downloaded %fields were 
parameters are: \verb|MAG_AUTO, MAGERR_AUTO, FLAGS, IMAFLAGS_ISO|, and \verb|SPREAD_MODEL| for each  $g$, $r$, $i$, $z$ bands. %Those parameters  were derived using SExtractor's automatic aperture photometry.
\verb|MAG_AUTO| are the magnitude estimations using an elliptical model considering the Kron radius, with \verb|MAGERR_AUTO| their uncertainties~\citep{Kron1980}. \verb|FLAGS| store additive flags indicating potential problems in the source extraction process, and \verb|IMAFLAGS_ISO| are flags where the sources have missing/flagged pixels in their single epoch images. Finally, the \verb|SPREAD_MODEL| is a parameter to identify extended sources comparing the fit quality between the local point-spread function (PSF) and an extended circular exponential disk \citep{desai12, DES2021}. %and the \verb|MAG_AUTO| were obtained using the Kron's algorithm,  which is the best for extended sources such as galaxies \citep{Kron1980}. %The DES counts extends down to a $10 \sigma$ magnitude limit of around $r\sim23.7$, $i\sim23.0$, and $z\sim22.27$.
The limiting magnitudes of the DES (DR2) for the selected clusters, at $10 \sigma$ are $g\sim 23.7$, $r\sim23.7$, $i\sim23.0$, and $z\sim22.27$ mag.
%

DeROSITAS is a %south and equatorial sky 
survey designed to complement the German sky of the eROSITA survey \citep[][]{Merloni12} in the optical wave range. DeROSITAS was performed using DECam in filters $g$, $r$, $i$, and $z$, reaching minimum
%\sout{DeROSITAS is an optical survey using DECam in 4 filters \texttt{g,r,i,z}. DeROSITAS was designed to cover all the extragalactic sky in the German side of the eROSITA survey \citep[][]{Merloni12}, to obtain cluster red sequence photometric redshifts  up to $z\sim1$ \citep[e.g.,][]{Song2012, klein19} by reaching a minimum}
depth of 22.7 (23.5), 23.2 (24.0), 23.3 (24.0), 22.5 (23.2) AB magnitudes at 10(5)$\sigma$.  DeROSITAS observing strategy consisted in filling the sky avoiding archival data when at sufficient depth, and carry out observations in coordination with other current surveys, such as DELVE \citep[][]{Drlica-wagner21}, to avoid duplication.  %During  DeROSITAS observations with low seeing (typically 0.9'' or less),  several high-z cluster (0.65$\leq z \leq$ 0.88) from the \cite{Zenteno2020a} sample were observed in the  $i-$ and $z-$ bands.
During DeROSITAS nights, high-z clusters  observations were triggered when the seeing and the effective time t$_{\rm eff}$ \citep[][]{bernstein16} were better than average (seeing better than $\sim$1.0'' and t$_{\rm eff} > 0.4$). %The $t_{{\rm eff}}$ is defined as the  time used by the instrument and subtracting the time lost by weather and operational reasons. 
The t$_{{\rm eff}}$ is a scale factor to be applied to the open shutter time to reflect the quality of the observations compared to good canonical conditions. These good  conditions are defined as observations with a FWHM of 0.9" and sky brightness obtained when pointing the telescope to the zenith under dark conditions.     %The DES data consists in catalogs for 84 clusters in this work, while the  \textcolor{red}{{\it high-z cluster}} data consists in catalogs for 19 clusters at the higher redshift end. % The 19 high-z cluster final images are constructed by stacking DeROSITAS and archival data (including DES images) at the cluster position, creating 1 degree by 1 degree stacks.} 

The DeROSITAS observations used here reach magnitudes $i\sim 23.9$, and $z\sim 23.6$, which is between 0.9 and 1.3 magnitudes deeper than DES (we used just those bands because those observations are focused in high-redshift clusters). The data reduction was done using a pipeline similar to DESDM \citep{desai12}, where the steps are done by first building single epoch (SE) images and then using a co-adding pipeline. The single epoch pipeline groups the observations according to the observation night and then, for each DECam observation that contributes to the cluster area (within $1\times1$ deg$^2$ from the SPT position), it constructs $\sim$62 (one for each CCD) photometrically flattened, astrometrically calibrated single SE images, together with position variable PSF models and PSF corrected model fitting catalogs. The processing includes overscan and bias correction, flat-fielding, initial astrometric calibration and PSF corrected model fitting photometry using PSFex. Final astrometric and photometric calibrations for each SE image are done using Gaia DR2 \citep{Evans2018} photometry data \citep[for details refer to][]{George2020a}. Finally, the coadd pipeline works similar to the DES processing, generating PSF homogenised COADD images and catalogues by using a combination of SourceExtractor and PSFex softwares.



The final {\it high-z clusters} sample is presented in the Table~\ref{DeROSITA}, and the DES cluster sample is presented in the Table~\ref{DES}. 

%The DES counts extends down to a $10 \sigma$ magnitude limit of around %$g\sim 23.8$,


\begin{comment}
%The selected 84  galaxy clusters were detected by the South Pole Telescope SPT using the Sunayev-Zeldovich (SZ) effect, which is the interaction between the Cosmic Microwave Radiation (CMB) with the hot intra-cluster gas. The clusters were also  optically confirmed by the Blanco Cosmology Survey \citep{Bleem2015}. 
The 19 clusters available on our observations are also included in the DES sample to have overlapped observations that allows us to calibrate our  data sets to get accuracy and consistency in our results. The selected sample of clusters covers a redshift range from $z\approx 0.11 $ up to $z\approx 0.88$, and has a mass threshold of $M_{200} > 4\times 10^{14} {\text M}_{\odot}$, where $M_{200}$ (SZE-based) is defined as the mass enclosed in a sphere whose mean density is 200 times the critical density of the Universe. 
%to see the error in the position ERRTHETA_IMAGE
%%%[CRIS] I brought this text from Sec. Results, and slightly modified. 
The dynamical state of this sample were determined by \citet{Zenteno2020a}
using SZ, optical (DES survey) and X-rays (Chandra and XMM-Newton) observations. The clusters were classified into relaxed and disturbed, by studying the offset between the position of their brighter cluster galaxy (BCG) and the position of the center of the SZ effect ($r_{\text BCG-SZ}$). A cluster was consider  disturbed if $r_{\text BCG-SZ} > 0.4\times R_{200}$, where $R_{200}$ is the radius enclosing $M_{200}$, and it  was derived using the~\cite{Duffy2008} mass-concentration relation and the SPT-SZ data. On the other hand, a cluster was considered  relaxed if at least one of those three conditions were meet: i) $A_{\text phot}<0.1$, where $A_{\text phot}$ is the photon asymmetry index \citep{Nurgaliev2013}; ii) the cluster have a cool core \citep[based on the radial entropy,][]{McDonald2013}; or  iii) the distance between the BCG and the X-rays peak is less than 42 kpc.\\Thereby, our sample includes 43 merging clusters in the redshift range  0.11 $< z < $ 0.75, and 41 relaxed clusters in the redshift range of  0.14 $< z < $ 0.88.
\end{comment}


%\section{Data Calibration}
\section{Catalogs}
\label{sec:catalogs}
% Figure environment removed
As we mentioned, we have two sets of data. The first comes from the DES DR2 database public repository, and the second, with deeper photometry, comes from our own reductions. Following the catalogues calibration described in the preceding section, hereafter we will create the final sample joining both catalogues in the following way: 65 clusters from DES described in Table \ref{DES} at $0.1<z<0.65$ (henceforth the {\it DES clusters} sample), and the 19 clusters from DeROSITAS at redshift higher than 0.65 (henceforth the {\it high-z clusters} sample) detailed in Table \ref{DeROSITA}.   %\sout{The first one is composed of 84 clusters at $0.1<z<0.65$. The second includes 19 clusters at redshifts higher than 0.65 (henceforth the {\it high-z clusters} sample).} 
%\sout{As previously mentioned, we have two different sets of observational data. We use 84 clusters from the DES DR2 database public repository, 19 of which we also have deeper {\it high-z cluster} observations ($z \gtrsim 0.65)$.  }
%
\begin{comment}% AZ: This text looks weird to me. I wil write it agian
As those two catalogues were obtained from two different pipelines bring both photometry to the same photometric %system.  We do this by first correcting
systems, we correct the color of both catalogs using the Stellar-Locus-Regression (SLR) technique and then by using the DES zeropoint to adjust the \textcolor{red}{{\it high-z cluster}} catalog zeropoint. The process is outlined below:. %hose two catalogues have overlapping observations which can allow us to compare the photometry  making a cross-match process following the next steps:
\end{comment}
As those two catalogues were obtained from two different pipelines, we expect slight differences.  To reduce such photometric differences we correct the colour of both catalogs using the Stellar-Locus-Regression (SLR) technique \cite{High2009}.  Once the colour is corrected we adjust the zeropoint by adopting the DES zeropoint for the {\it high-z clusters} catalogs. The process is outlined below:


First, we calibrated the colours $g-r$, $g-i$, $r-i$, and $i-z$ for both sets of data using the SLR code. This technique uses a region in the colour-colour diagram populated by stars~\citep{Covey2007, Ivezic2007}. The SLR code accurately calibrates  the colours for stars and galaxies using catalogued flat-fielded images  without having to measure standard stars or  determining the zero-points for each pass-band.  % showed that 
The SLR technique also corrects for differences in instrumental response,  atmospheric response, and for  galactic extinction. 

As input, the SLR code needs magnitudes, magnitudes errors, and extinction value for every source in each passband used ($g$, $r$, $i$, $z$). %The magnitudes and their error measurements are obtained from the photometry on homogenized images using SExtractors. 
The dust extinction were obtained using the Schlafly \& Finkbeiner Dust \citep[]{Schlafly2011, Schlegel1998}.  %This is a dust map based on the far-infrared dust emission calibrated by Sloan Extension for Galactic Understanding from Sloan Digital Sky Survey (SDSS). This map was obtained analyzing the difference between the measured and the predicted colors of stars \citep{Schlafly2011, Schlegel1998}. This model allows us to estimate the dust extinction for each DECam passband.  \\
Colour correction was made considering objects classified as stars. %\sout{Finally, we provide SLR with %for the SLR  code it is necessary to provide the
%objects of the catalogue identified as stars to use them for the colour corrections.} \\
The photometric catalogues include the \verb |SPREAD_MODEL| parameter, which is a star-galaxy separator.  %Stars tend to have an \verb|spread_model| parameter close to zero, whereas extended sources, such as galaxies, have larger values for this parameter.  
Following the same criteria used by ~\citet{Hennig2017}, we consider as stars the sources with  \verb |SPREAD_MODEL < 0.002|. We clean the final sample by excluding sources with \verb|IMAFLAGS_ISO > 0| in all bands to avoid saturated objects and objects with missing data ~\citep{Morganson2018} and \verb|FLAGS| $\geq$ \verb|4|, to include deblended sources but excluding sources flagged with warnings during the extraction process ~\citep{DES2021}. %\sout{Additionally, we selected the clean objects from the sample excluding  the flagged sources. The criteria to flag objects is:   sources  with \verb|IMAFLAGS_ISO > 0| in all bands to avoid saturated objects~\citep{Morganson2018}, and \verb|FLAGS| $\geq$ \verb| 4 | which allows blended objects but prevents other Source Extractor problems~\citep{DES2021}.  }

	%included in the gdpyc python package which includes several HI and dust surveys with nH and E(B-V) estimations. In this work, we used the low resolution map, but there is available one high resolution map. To enable the high resolution map it is necessary to add hires=1 in the python code. The used dust map is the SFD which is an all sky-map  . 

As a result of the  SLR  calibration, we obtained %color-corrected colors
corrected colours for stars and galaxies for the DES and {\it high-z clusters} catalogs.  Next, we correct the absolute magnitude of the {\it high-z clusters} catalogs by comparing its star's magnitudes to the DES star magnitudes for the $i-$band.  We used a 0.25 arcseconds to match stars and a 0.5 magnitude difference to avoid variable stars.  Once such correction is found (which has an average value of $\sim$ 0.02 magnitude), we applied the correction to all objects in the {\it high-z clusters} catalog.  
%corrected \verb|gr, gi, ri|, and \verb|iz| colors for each source including stars and galaxies. 

\begin{comment}

Then, the corrected magnitudes were computing making a cross-match for the overlapped data in both sets of data. This process consists of finding stars and galaxies located at the same on-the-sky position in both catalogues using their coordinates. To consider one source as the same in both catalogues, we imposed the conditions that  the differences between  their positions  must be lower than 0.25 arc seconds and the difference in their measured magnitudes in all bands must be lower than a half magnitude. 

Using the cross-match process, we obtained the zero point between the DES and DeROSITA magnitudes in the \verb|i| band for each cluster. We assumed the DES magnitude as the standard and corrected the DeROSITA ones  subtracting the zero point obtained before. We selected the \verb|i| band as a pivot to compute the rest of the bands because each of them can be obtained  by adding or subtracting only one calibrated color.

\end{comment}



As an example of the results of the calibration process,
in Figure \ref{fig:calibration_final} we present four diagrams, one for each band ($g$, $r$, $i$, $z$)  for the catalogues corresponding to the cluster SPTCLJ0310-4647. In those diagrams, each point corresponds to a Milky-Way star detected in the field of view of the cluster,  present in both DES and DeROSITAS catalogues. Stars were selected by  \verb|SPREAD_MODEL < 0.02|). %showing the results of the calibration process. 
In the x-axis, we show the magnitude (MAG$\_$ AUTO) corresponding to the DES catalogue (taken as a base to calibrate the high-redshift data set),meanwhile in the y-axis we have the $\Delta (magnitude) $ defined as the difference between DES and {\it high-z clusters} MAG$\_$AUTO.  Similar plots were obtained for the 19 clusters in common. %overlapped clusters. 
We can see that the differences in magnitude for DES and {\it high-z clusters} observations  are close to zero for the brightest end. The maximum mean difference for the brightest stars (mag < 20.5), between the two catalogues, computed as the average difference between the DES and DeROSITAS magnitudes, is around 0.04 mag in all four bands.  This result suggests that the fluxes obtained using both pipelines are very close  to each other and, thus, we can consider them as equivalents.
% Figure environment removed
\section{RESULTS}
The goal of this paper is to study the differences in the cluster RS galaxy population as a function of the cluster dynamical state. %The RS is a distinctive characteristic identified  in each detected cluster at $z \lesssim 1.5$, formed by galaxies with  null or little ongoing star formation \citep{Gladders2004,  Propris2016}.
A galaxy spectrum is mostly flat and is mainly composed  of a combination of blackbody emitters, but there is a noticeable break at 4,000 \AA\ in the rest frame where there is an absorption of high energy radiation in the stellar atmospheres in metal poor populations \citep{Mihalas1966, Mihalas1967, Poggianti1997}. This break allows to  separate  blue from red galaxies~\citep{Poggianti1997}. For this reason, we used two photometric bands that contain the  Balmer break in the rest frame. In those filters, the elliptical galaxies tend at a given redshift range, tend to be  redder than normal  galaxies at any lower redshift, thus becoming easily noticeable from the background \citep{Gladders2000}. %The bands used for each  CMD are selected as the photometric band that contains the 4000 $A$ Balmer break and the next red band.
%The selected bands, depend on the redshift of the cluster. In this sense, we present in the Table~\ref{table:bands_redshift} the  selected bands for the color- magnitude diagram as function of redshift. \\
Then, the selected bands depend on the redshift of each cluster, and are presented in the Table~\ref{table:bands_redshift}.

\begin{table}
\centering
\caption{Bands used for the colour- magnitude diagram depending on the cluster redshift to capture the 4000 {\AA} Balmer break. }
\label{table:bands_redshift}
\begin{tabular}{ccc}
\hline
Redshift & color  bands & magnitude band \\
\hline
$0<z\leq0.33$ & (g-r) & r\\
$0.33< z \leq 0.74$ & (r-i) &i \\
$0.74< z \leq 0.9$ & (i-z) &z\\
\hline
\end{tabular}
\end{table}

% Figure environment removed 
\begin{table}
\centering
\caption{Number of clusters in the used sample in each analyzed group. We divided the sample between low and high redshift samples and relaxed and disturbed clusters.}
\label{table:number_clusters}
\begin{tabular}{ccc}
\hline
Redshift range & Relaxed clusters & Disturbed clusters \\
\hline
$0.1<z<0.55$ & 16 & 27 \\
$0.55\leq z<0.9$ &  25 & 16\\
\hline
\end{tabular}
\end{table}
%\sout{To construct the colour-magnitude diagram (CMD), we consider only those galaxies within their $R_{200}$. Objects located in the annulus between  $1.5\times R_{200}< r < 3\times R_{200}$ are considered that not belongs to clusters i.e. background, and foreground objects.} 
To construct a cluster colour-magnitude diagram (CMD), we consider  galaxies within $R_{200}$  as cluster galaxies and galaxies in the annulus between $1.5\times R_{200}< r < 3\times R_{200}$ as background. Where $R_{200}$ is defined as the radius where the cluster density is 200 times the critical density of the Universe at a given redshift. The cluster centers corresponds to the SPT-SZ centers \citep{Bleem2015}, and the $R_{200}$ were estimated by \citet{Zenteno2020a} using the estimated $M_{500}$, and the Duffy mass-concentration relation \citep{Duffy2008}. We bin both the cluster and background CMDs, using bins of 0.6 in magnitude and 0.06 in colour, and perform a statistical background subtraction to correct for contamination due to the projection effects. 
%To correct by the background and foreground contributions, we subtract the density of the objects in the annulus between $1.5\times R_{200}< r < 3\times R_{200}$ from the density of objects inside $R_{200}$, weighted by their areas.  
  
%We bin both the cluster and background  CMDs, using bins of 0.6 in magnitude and 0.06 in colour. The binned CMDs allows us to perform a statistical background subtraction to correct for contamination due to the projection effects.
%
%
% 
\begin{comment}
We use our background galaxies to perform an area-normalized statistical background subtraction to correct for contamination due to the projection effects. %using the galaxies located in the annulus between  $1.5\times R_{200}< r < 3\times R_{200}$ in projected distance
 % We also normalize the number of objects of the cluster and the background using ratio between the cluster and background areas. %We considered that  the cluster RS is composed by the galaxies inside $R_{200}$ after the background correction. 
\end{comment}


%The cluster galaxies population is mainly formed by elliptical galaxies. \\

\begin{comment}
The location of the red sequence in the color-magnitude space evolves with the cosmic time. Early type galaxies in the redshifts between $0.9 <z<0.1$ are dominated by late type stellar population  formed in a single burst between $1.5<z<2$ and evolving passively. %Early-type galaxies have a relation between their color and magnitude. 

The elliptical galaxies in a given cluster form a straight line with a well-defined slope (tilt) in the color magnitude space~\citep{Gladders2000}. However,  the tilt of the slope of the RS is also redshift dependent \citep{Stott2009}.  To stack the RS for multiple clusters, we corrected this slope using the same model development by \citet{Hennig2017} that uses the stellar population synthesis with and exponential starburst decay and a Chabrier IMF  \cite{Bruzual2003}. This model also includes the response of the Blanco telescope, CCD and filters  using the transition curves to create a modeled galaxy magnitude in the griz bands for $0.3-0.4,0.5,1,2,3 L^*~$.
\end{comment}




Once the background-subtracted CMD is built for each cluster, we use a stellar population synthesis models to stack them. We do this by using models with an exponential starburst decay and a Chabrier IMF  \citep{Bruzual2003} for six metallicities described in \citet[e.g.,][]{Song2012}, and \citet{zenteno16}. 
%
Using the cluster's redshift, and following the procedure 
as described by \citet{Hennig2017},  we obtain the model (expected) cluster RS slope as a function of the magnitude and then subtract it from the observed slope, bringing the cluster's colour slope to zero. The model also provides the characteristic magnitude defining the %\st{break in} 
knee of   the luminosity function, $m^*$, for each filter and redshift. The cluster redshift used in this paper corresponds to the same as \citet{Zenteno2020a}, which are photometric and spectroscopic redshifts collected from several literature sources.  \\
In Figure \ref{fig:Red_sequence_cluster}, we show the process of the background correction for the SPT-CLJ0253-6046 galaxy cluster, located at $z= 0.45$. For this cluster we used the $r$ and $i$ bands. In the horizontal axis we have the magnitude ($m_i-m^*$), with $m^* $ obtained from the model. The three vertical axes show the $r-i$ colour.     In the top panel, we show the density of objects within the virial radius, plotted in gray scale. The middle panel focuses on objects located between $1.5 $ to $3$ viral radius. Finally, in the bottom panel we show the resulting red sequence, obtained after subtracting the number object in the background from the number of objects within $R_{200}$,  normalized by their areas.

%Finally, in the bottom panel we show the resulting red sequence, obtained after subtracting the density of object in the background from the density of objects within the virial radius,  weighted by their areas. 
%
%
%It should be noticed the RS can be recognized easily around zero in the color axis.  We repeat the same process to obtain the RS in each cluster.
%We use the tilt of the red sequence model to bring all the cluster's red sequences to zero color. 
%\textcolor{red}{The CMDs for each cluster have bins of 0.6 in magnitude and 0.06 in color (See Fig. \ref{fig:Red_sequence_cluster}). 
We then combined the cluster CMDs by adding all background-corrected numbers of galaxies, normalised by the total number of clusters used in each bin.
%To explore redshift evolution of the slope and width of the RS,  following \cite{Zenteno2020a}, we separate the cluster sample in \textcolor{blue}{four subgroups depending on their redshift and dynamical state:  Low-redshift relaxed clusters $(z<0.55)$, High-redshift relaxed clusters $(z\geq 0.55)$, Low-redshift disturbed clusters $(z<0.55)$, and, High-redshift disturbed clusters $(z \geq 0.55)$}. The number of clusters in each subgroup is presented in the Table~\ref{table:number_clusters}. 
\citet{DePropris2016} studied the evolution of galaxies as they experience gravitational infall into cluster cores during merging processes, resulting in a transformation of the cluster RS (red sequence) morphology, as previously observed by~\citet{DeLucia2007, Stott2007}. The RS evolution below $z \sim 0.5 - 0.6$  shows that there are little or no morphological changes in the galaxy population, unlike earlier in time. For this reason, we separated the cluster sample in two redshift bins: the first one includes clusters between $0.1<z<0.55$ (low redshift bin), and the second one includes clusters with redshifts between  $0.55\leq z < 0.9$. Additionally, we subdivided our cluster sample according to their dynamic state. As a result, we have four subgroups of clusters: disturbed and relaxed clusters at low and high redshifts. The number of clusters in each subgroup is presented in Table~\ref{table:number_clusters}. It should be noticed that we have at least 16 clusters in each subgroup. 

The dynamical state of the clusters used in this paper was estimated by~\citep{Zenteno2020a}. They classified between relaxed and disturbed clusters using four different proxies: The offset between the position of the BCG and the SZ centroid ($D_{BCG-SZ}$), the core temperature, the morphological parameter($A_{Phot}$), which measures the asymmetry of the X-ray emission, and the offset between the BCG and the peak of X-ray emission. In this analysis, a cluster is defined as relaxed if it met any of the following three conditions: i) the cluster has $A_{phot} < 0.1$, ii) the cluster has a cool core $(K_0<30~\text{keV cm}^{2})$~\citep{McDonald2013}, or iii) the offset between BCG and X-ray peak is less than 42 kpc~\citep{Mann2012}. Meanwhile, the clusters were classified as disturbed if the offset between the position of the BCG and the position of the SZ centroid $D_{BCG-SZ}$ is greater than 0.4 $R_{200}$. The last criterion was chosen given that the distribution of  $D_{BCG-SZ}$  looks flat after this value. Clusters that don’t meet any of the four previously mentioned criteria are considered in an intermediate evolutionary state and are excluded from this analysis. The BCG position is used as a proxy of the collisionless component since it is expected to quickly fall to the lowest region of the potential well~\citep{Tremaine1990}. This position was derived from optical observations. Meanwhile, the centroid of the SZ was used as a proxy for the collisional component. When a cluster interacts with other clusters or groups, the gas, dark matter and galaxy components act differently depending on their nature. Then, the offset between the collisional and collisionless matter components can be used as a proxy to quantify the relaxation state of clusters~\citep{Zenteno2020a}.   


%%%% [CRIS] I moved this paragraph to the end of Sec. 2 (before Data Calibration), as it is part of the sample description.
%The dynamical state of this sample were determined by \citet{Zenteno2020a} using Sunayev-Zeldovich (South Pole Telescope), optical (DES survey) and X-rays (Chandra and XMM-Newton) observations. The clusters were classified in relaxed and disturbed studying the offset between the position of the brighter cluster galaxy (BCG ) and the position of the center of the Sunayev-Zeldovich effect ($r_{BCG-SZ}$). A cluster was consider  disturbed if $r_{BCG-SZ}>0.4\times R_{200}$, where $R_{200}$ is the virial radius which is defined as the radius where the mean density is 200 times the critical density of the Universe. The virial radius was derived using the~\cite{Duffy2008} mass-concentration relation and the SPT-SZ data.   On the other hand, a cluster was consider relaxed if at least one of those three conditions were meet: 1) $A_{phot}<0.1$ where $A_{phot}$ is the photon asymmetry index~\citet{Nurgaliev2013} , ii) the cluster have a cool core (based on the radial entropy ~\citet{McDonald2013}), or iii) the distance between the BCG and the X-rays peak is less than 42 kpc. Our sample includes 43 relaxed and 41 disturbed clusters. \\



%After the redshift corrections and the selection of the appropriate bands for the color- magnitude diagram, we stacked the clusters in each group described in the Table~\ref{table:number_clusters}. 
In Figure~\ref{Fig:Red_Sequence}, we present the stacked CMD for the four subgroups: low and high redshift %for 
relaxed clusters and low and high redshift %bins for 
disturbed clusters. The gray scale covers the same numeric range in the four panels. Darker (lighter) colours represents higher (lower) density regions in  colour-magnitude space. In all panels we see a dominant RS with an associated bluer galaxy population.  For the low redshift sample, we find that disturbed and relaxed clusters red sequence show  similar colour distributions. %samples. 
This is, the relative contribution from  early- and late-type galaxy populations to the CMD is comparable in both cases, indicating that the current dynamical state of a galaxy cluster has little impact on its CMD at the present-day. 
%,there is  a predominant early-type galaxy populaiton, with a smaller contribution of late (blue) type galaxies. The cluster RS extends from $\sim m^* -2$ to $\sim m^*+3$, for both cases. % It can be seen that the dynamical state makes no differences in the color-magnitude diagram at low-z.
%
% both cases, but there is a little increase in the density in the bright end $(m-m^*<0)$ for the disturbed sample compared to the relaxed clusters.\\
On the other hand, on the bottom panels of Figure \ref{Fig:Red_Sequence} we show the RS for the high redshift sample. 
 %First, it can be observed that, specially for the disturbed clusters, the
These RS distributions are generally wider with respect to the low-redshift counterpart. %, specially for the disturbed clusters. 
More interestingly, we observe a wider colour distribution in the disturbed cluster population when compared with both the high redshift relaxed distribution and the overall low redshift sample.

%the disturbed cluster populaiton shows 
%Furthermore, There are noticeable differences between the relaxed and disturbed sample. 
%the color distribution in the disturbed sample is wider compared to the relaxed one. 
%
% AZ: From here Franklin starts to draw conclusions of his work.  WORK UNTIL HERE. FROM HERE ON I REQUIRE MORE TIME.
%
\begin{comment}
, this means that the stellar population of the galaxies are more mixed in the disturbed sample founding evidence of different episodes of star formation during more recent time. Comparing the higher and the lower redshifts bins, we can conclude that the galaxy population at low redshift bins is highly dominated by early-type galaxies, however at high redshift bins we can still see a residual bluer galaxy population. These results agree with the RS evolution scenario proposed by \citet{Stott2009} suggesting that those residual blue galaxies quenches rapidly showing a uniform flat RS at low redshifts.   
\end{comment}

%Figures are referred to as e.g. Fig.~\ref{fig:example_figure}, and tables as
%e.g. Table~\ref{tab:example_table}.

% Example figure
%% Figure environment removed
% Figure environment removed

Figure~\ref{Fig:Density} shows histograms from collapsing Figure~\ref{Fig:Red_Sequence}  in the magnitude ($m-m^*$) axis for the four studied subgroups. Those histograms are normalized by their euclidean norm.  In the upper panel we have the relaxed and disturbed clusters for low redshift clusters ($z<0.55$). Similarly, in the bottom panel we have the galaxy color distribution for high redshift clusters ($z>0.55$),  relaxed in blue, and disturbed in red.
%Figure~\ref{Fig:Density} presents the color density distribution stacked for the four studied subgroups. In the upper panel we have the relaxed and disturbed clusters for low redshift clusters ( $z<0.55$). Similarly, in the bottom panel we have the galaxy color distribution for high redshift clusters ($z>0.55$),  relaxed in blue, and disturbed in red.
%
%Those e
Error bars were computed  as  Poisson noise.  In this plot, we also present in vertical dashed lines the medians of the color distribution, we can see that for the low-redshift sample, the color density distribution is very similar for the relaxed and disturbed clusters, and the medians are also nearly identical. However, in the bottom panel,  we can see an excess of blue galaxies in the disturbed sample compared to the relaxed clusters. The differences showed in the bottom panel are above the observational uncertainties, and the values of the medians are significantly different, -0.09 for relaxed,   and -0.15 for disturbed clusters. This result  shows that the galaxy population of disturbed clusters  is bluer and possibly less quenched compared to the relaxed clusters.%\sout{no possible yet-quenched galaxy population in comparison with the relaxed clusters.} 
\begin{comment}
In Fig. \ref{Fig:Red_Sequence} and Fig. \ref{Fig:Density}, we stacked the RS for the four subgroups presented in Table \ref{table:number_clusters}. The RS for low redshift where construction stacking observation using $g-r$ and $r-i$ colors, and for the high redshift bin, the we construct the RS using $r-i$ and $i-z$ colors. %Mixing photometric bands is no a problem because the main differences observed in the RS at high redshift range arises from the clusters between $0.5 \leq z \leq 0.75$ measured by using the $r- $ and $i-$ bands. 
In the Appendix \ref{Appendix1} we repeat the analysis considering just the clusters measured using the $r-i$ bands. Figure \ref{Fig:Red_Sequence_RI} and Fig. \ref{Fig:Density_RI}  are the corresponding color magnitude diagram and color density, respectively. The conclusions in differences between relaxed and disturbed clusters are the same showing that mixing photometric bands doesn't have any effect in our conclusions. 
\end{comment}


\section{Discussion and Conclusions}

We used optical data for 84 galaxy clusters detected by SPT-SZ and optically confirmed by the DES survey. The data set is composed by 65 galaxy clusters from DES survey DR2 and 19 high-z clusters from our dedicated observations. Both data sets were observed using the Blanco telescope and the Dark Energy Camera. 
In order to have an homogeneous data set, we performed a calibration using the Stellar Locus Regression code.  This allowed us to correct systematic differences between both data sets.
%to make the observations equivalent. 
We divided the final sample in two redshift bins at $z = 0.55$. 
% We \textcolor{red}{found} differences between the cluster red sequence between relaxed and disturbed clusters dividing them in \textcolor{red}{two redshift bins}, each subset with at least \textcolor{red}{YY} clusters. To stack the cluster red sequence, we \textcolor{red}{corrected the color-magnitude diagram using an SSP model.} % tilt correction give the dependence of the position of the Red Sequence in the color magnitude space with the redshift. This correction was performed using the~\citet{Bruzual2003} model convoluted with a transmission model for the Blanco telescope. 
%
The results summarized in Fig.~\ref{Fig:Density}, show that, as expected,  high redshift clusters have a  more significant blue galaxy population with respect to the low redshift subsample  \citep[see e.g.][]{butcher84}. At $z \lesssim 0.55$ the relaxed and disturbed show almost identical CMDs.  However, at $z \gtrsim 0.55$ we see significant differences in their cluster galaxy colour distribution.  Specifically,  the disturbed sample shows an excess of blue galaxies with respect to the relaxed sample.

%behave in an almost identical fashion, while at $z \gtrsim 0.55$ we see differences.  Specifically,  the disturbed sample shows an excess of blue galaxies. 
% AZ: It actually could be the one I commented, but then the question is how we do the distinction?
%brings blue galaxies from the cluster outskirt to the cluster center (R$_{200}$), or the event 
  This result  provides more evidence that  the galaxy colour distribution not only depends strongly on the global environment \citep[e.g.][]{Peng2010,Iovino2010, Muzzin2012}, but also depends on the dynamical state of the clusters \citep[e.g.,][]{Stroe2017,Zenteno2020a,stroe21}, at least at $z\gtrsim0.55$. % (AZ: Why discussing a single paper?) This result agrees with \citet{Hou2012}. They studied the difference of color distribution between groups with structures and without substructures founding that there is an excess of blue galaxies in unrelaxed clusters. 
Our results are consistent with the anti-correlation between the relaxed state of the cluster and their star formation  activity found by \citet{Cohen2015} and \citet{Hou2012}.  The  two possible explanations for the excess in the blue population are the same than the ones proposed by \cite{Cohen2015} for the enhancement of the SFR: {\it i)} the merging process triggers star formation in the galaxies inside the merging clusters generating a more mixed and bluer galaxy population or {\it ii)}  disturbed clusters correspond to a less evolved state than relaxed clusters.  Its worth noting that our results are  consistent with \citet{Pallero2019}.  Using a suite of fully cosmological hydrodynamical simulations,  they showed that most quenched galaxies in massive halos ($M_{200}>10^{14}M_{\odot}$) at $z<0.5$  were quenched in  ex-situ groups or clusters. However for $z>0.5$ the in-situ quenched galaxies dominates their population, suggesting that the galaxies quenched inside the first cluster they fall in. In this scenario, at $z>0.5$ relaxed clusters would have a redder population than disturbed clusters, as they have had more time to evolve their population in-situ.  At $z<0.5$ that difference disappears;  the disturbed sample and relaxed clusters have the same galaxy population since they are being assembled from structures whose members have been already preprocessed within massive substructures. 
%has the same galaxy population as the relaxed one as they are being assembled with clusters with galaxies as evolved as galaxies in relaxed clusters.
%
%According to  \citet{Stott2009} the galaxies falling into a cluster are quenched due to the physical process such as ram pressure stripping \citep{Gunn1972}, and strangulation \citep{Larson1980}. 
%
Indeed, this scenario could explain why the galaxy colour distribution at low redshift is the same in both, relaxed and disturbed sample. 
%\textcolor{red}{Those results can also be understood by the scenario proposed by  \citet{Stott2009} where the galaxies  at  high redshifts  must suffer rapid quenching as they fall into the cluster due to the physical process acting over such as ram pressure stripping \citep{Gunn1972}, and strangulation \citep{Larson1980}. For this reason at low redshift, we can see a more homogeneous and early-type dominated RS.}
%Our results  do not change if no SLR correction is applied (\S\ref{sec:catalogs}).
It is worth to mention that our results do not change if SLR correction is not applied to the data in \S\ref{sec:catalogs}.
%\sout{We will study in a deeper way the physical mechanism responsible for the differences in the observed color distribution  in a following article using Cosmological Simulations.}

 Based on the results obtained in this paper, it is not possible to conclude what is the physical mechanism driving the excess of blue galaxies in disturbed high-z clusters. In a follow-up article, we will compare our results against cosmological simulations to characterise the mechanisms responsible for such differences. 
\section*{Acknowledgements}

FA was supported by the doctoral thesis scholarship of Agencia Nacional de Investigaci\'on y Desarrollo (ANID)-Chile, grant 21211648. FAG acknowledges financial support from FONDECYT Regular 1211370. FAG and FA acknowledge funding from the Max Planck Society through a Partner Group grant. FA and FAG gratefully
acknowledges support by the ANID BASAL project FB210003.
ERC is supported by the international Gemini Observatory, a program of NSF’s NOIRLab, which is managed by the Association of Universities for Research in Astronomy (AURA) under a cooperative agreement with the National Science Foundation, on behalf of the Gemini partnership of Argentina, Brazil, Canada, Chile, the Republic of Korea, and the United States of America.
CVM acknowledges support from ANID/FONDECYT through grant 3200918. DHL acknowledges financial support from the MPG Faculty Fellowship program, the new ORIGINS cluster funded by the Deutsche Forschungsgemeinschaft (DFG, German Research Foundation) under Germany's Excellence Strategy - EXC-2094 - 390783311, and the Ludwig-Maximilians-Universit\"at Munich.
%%%%%%%%%%%%%%%%%%%%%%%%%%%%%%%%%%%%%%%%%%%%%%%%%%
\section*{Data Availability}

 The Dark Energy Survey data underlying this article are available at \url{https://www.darkenergysurvey.org/the-des-project/data-access/}
%The inclusion of a Data Availability Statement is a requirement for articles published in MNRAS. Data Availability Statements provide a standardised format for readers to understand the availability of data underlying the research results described in the article. The statement may refer to original data generated in the course of the study or to third-party data analysed in the article. The statement should describe and provide means of access, where possible, by linking to the data or providing the required accession numbers for the relevant databases or DOIs.




%%%%%%%%%%%%%%%%%%%% REFERENCES %%%%%%%%%%%%%%%%%%

% The best way to enter references is to use BibTeX:

\bibliographystyle{mnras}
\bibliography{library} % if your bibtex file is called example.bib


% Alternatively you could enter them by hand, like this:
% This method is tedious and prone to error if you have lots of references
%\begin{thebibliography}{99}
%\bibitem[\protect\citeauthoryear{Author}{2012}]{Author2012}
%Author A.~N., 2013, Journal of Improbable Astronomy, 1, 1
%\bibitem[\protect\citeauthoryear{Others}{2013}]{Others2013}
%Others S., 2012, Journal of Interesting Stuff, 17, 198
%\end{thebibliography}

%%%%%%%%%%%%%%%%%%%%%%%%%%%%%%%%%%%%%%%%%%%%%%%%%%

%%%%%%%%%%%%%%%%% APPENDICES %%%%%%%%%%%%%%%%%%%%%

\appendix
\section{Bands Mixture Analysis}
\label{Appendix1}
In this section, we analyze the effect of mixing photometric bands during the construction of the cluster RS. As showed in Table \ref{table:number_clustersri}, we join in the low redshift bin clusters observed with $g-r$ $(z\leq0.33)$ filters and $r-i$ $(0.33<z\leq0.55)$ filters and for the high redshift bin, we join $r-i$ $(0.55<z\leq0.74)$ and $i-z$ $(0.74<z\leq0.9)$ observations. Those bands capture the 4000 {\AA} Balmer break that allows us to detect a prominent cluster Red Sequence. The RS position in the CMD was corrected using the \citet{Bruzual2003} stellar evolution model before stacking to avoid introducing a redshift bias in our sample. However, in this section we repeated the analysis using just the clusters observed between $0.33<z\leq0.74$ using $r-i $ bands to confirm that the differences observed between relaxed and disturbed clusters for high redshift bin is not a consequence of  mixing photometric bands. In Table \ref{table:number_clustersri}, we have the number of stacked clusters for each considered subgroups, we can see that in each subgroup we have at least 12 clusters for low redshift bins and 22 clusters for high redshift bins. 
\begin{table}
\centering
\caption{Number of cluster in low and redshift bins considering the clusters observed using just the $r$ and $i$ bands. }
\label{table:number_clustersri}
\begin{tabular}{ccc}
\hline
Redshift range & Relaxed clusters & Disturbed clusters \\
\hline
$0.33<z<0.55$ & 12 & 22 \\
$0.55\leq z<0.74$ &  19 & 15\\
\hline
\end{tabular}
\end{table}
Similar to the previous analysis, we stacked the CMD in each subgroup after the redshift correction using the \citet{Bruzual2003} stellar evolution model. The morphology of the RS is the same that the showed in the Fig. \ref{Fig:Red_Sequence} and the conclusions are the same: at low redshift bins, the galaxy population is mainly dominated by early-type galaxies, however at high redshift bins, there is a remnant blue (no yet-quenched) galaxy population. 



%\begin{comment}
% Figure environment removed 
We also repeated a colour density diagram presented in the Fig. \ref{Fig:Density} for cluster observed with $r-i$  bands presenting in the Fig. \ref{Fig:Density_RI}. We can see that those diagrams are very similar and the differences in colour are maintained between relaxed and disturbed clusters at high redshift bins. We also can conclude that the galaxy population at high redshift bins is more mixed in disturbed clusters compared to relaxed ones. With those two plots, we showed that the morphological differences in the RS come from the intrinsic properties of clusters and not from the bands mixing process. 
% Figure environment removed

%\end{comment}

\section{List of clusters}
In Table \ref{DeROSITA}, we present the list of the 19 high-redshift clusters observed in the frame of the DeROSITAS survey. Alongside their names, we also present the redshift, virial mass (SZ-based), and virial radius obtained using spectroscopic and photometric observations as specified in~\citep{Zenteno2020a}. In Table \ref{DES}, we present the list of the clusters observed by DES including their redshifts, masses and radius. Finally, in Table \ref{DES_obs}, we present the exposure time, FWHM and the number of observations made for one of those 19 high-z galaxy clusters in bands ($r$, $i$, and, $z$).   .  

\begin{table}
    \centering
    \begin{tabular}{c c c c c c}
    \hline
        Name & SPT RA & SPT Dec. & $z$ &  $M_{200}$ & $R_{200}$ \\
SPT-CL & J2000 & J2000 & & $10^{14} h^{-1}_{70}\; M_{\odot}$ & $\arcmin$  \\
\hline
J0014-4952 & 3.69 & -49.87 & 0.752 & 3.27 & 8.10 \\
J0058-6145 & 14.58 & -61.76 & 0.83 & 2.87 & 6.65 \\
J0131-5604 & 22.93 & -56.08 & 0.69 & 3.17 & 6.17 \\
J0230-6028 & 37.64 & -60.46 & 0.68 & 3.06 & 5.41 \\
J0310-4647 & 47.62 & -46.78 & 0.709 & 3.17 & 6.53 \\
J0313-5645 & 48.26 & -56.75 & 0.66 & 2.82 & 3.96 \\
J0324-6236 & 51.05 & -62.60 & 0.75 & 3.21 & 7.57 \\
J0406-4805 & 61.72 & -48.08 & 0.737 & 3.16 & 7.01 \\
J0406-5455 & 61.69 & -54.92 & 0.74 & 2.86 & 5.23 \\
J0422-4608 & 65.74 & -46.14 & 0.7 & 2.79 & 4.36 \\
J0441-4855 & 70.45 & -48.91 & 0.79 & 3.05 & 7.23 \\
J0528-5300 & 82.02 & -53.00 & 0.768 & 2.84 & 5.53 \\
J0533-5005 & 83.40 & -50.09 & 0.881 & 2.64 & 5.78 \\
J2043-5035 & 310.82 & -50.59 & 0.723 & 3.18 & 6.88 \\
J2222-4834 & 335.71 & -48.57 & 0.652 & 3.62 & 8.22 \\
J2228-5828 & 337.21 & -58.46 & 0.71 & 2.85 & 4.77 \\
J2242-4435 & 340.51 & -44.58 & 0.73 & 2.66 & 4.10 \\
J2259-6057 & 344.75 & -60.95 & 0.75 & 3.34 & 8.57 \\  
J2352-4657 & 358.06 & -46.95 & 0.73 & 3.14 & 6.71 \\
\hline
    \end{tabular}
    \caption{19 {\it high-z clusters}  used in this analysis. The redshift, virial mass and virial radius were obtained from \citet{Zenteno2020a} }.
    \label{DeROSITA}
\end{table}

\begin{table}
    \centering
    \begin{tabular}{c c c c c c}
    \hline
Name & SPT RA & SPT Dec. & $z$ & $M_{200}$ & $R_{200}$ \\
SPT-CL & J2000 & J2000 & & $10^{14} h^{-1}_{70}\; M_{\odot}$ & $\arcmin$ \\
\hline
J0000-5748 & 0.24 & -57.80 & 0.702 & 3.25 & 6.91 \\
J0033-6326 & 8.47 & -63.44 & 0.597 & 3.67 & 7.12 \\
J0038-5244 & 9.72 & -52.74 & 0.42 & 4.16 & 4.79 \\
J0107-4855 & 16.88 & -48.91 & 0.6 & 3.08 & 4.24 \\
J0111-5518 & 17.84 & -55.31 & 0.56 & 3.23 & 4.23 \\
J0123-4821 & 20.79 & -48.35 & 0.655 & 3.38 & 6.73 \\
J0135-5904 & 23.97 & -59.08 & 0.49 & 3.57 & 4.28 \\
J0144-4807 & 26.17 & -48.12 & 0.31 & 5.27 & 4.8 \\
J0145-5301 & 26.26 & -53.02 & 0.117 & 14.25 & 7.73 \\
J0147-5622 & 26.96 & -56.37 & 0.64 & 3.01 & 4.54 \\
J0151-5654 & 27.78 & -56.91 & 0.29 & 5.54 & 4.75 \\
J0152-5303 & 28.23 & -53.05 & 0.55 & 3.79 & 6.59 \\
J0200-4852 & 30.14 & -48.87 & 0.498 & 4.18 & 7.13 \\
J0212-4657 & 33.10 & -46.95 & 0.655 & 3.71 & 8.93 \\
J0217-4310 & 34.41 & -43.18 & 0.52 & 3.89 & 6.3 \\
J0231-5403 & 37.77 & -54.05 & 0.59 & 3.26 & 4.87 \\     
J0232-5257 & 38.18 & -52.95 & 0.556 & 4.03 & 8.08 \\
J0243-5930 & 40.86 & -59.51 & 0.635 & 3.48 & 6.92 \\
J0253-6046 & 43.46 & -60.77 & 0.45 & 3.9 & 4.6 \\
J0256-5617 & 44.09 & -56.29 & 0.58 & 3.7 & 6.83 \\
J0257-4817 & 44.44 & -48.29 & 0.46 & 3.93 & 4.95 \\
J0257-5732 & 44.35 & -57.54 & 0.434 & 4.04 & 4.73 \\
J0257-5842 & 44.39 & -58.71 & 0.44 & 4.09 & 5.05 \\
J0304-4748 & 46.15 & -47.81 & 0.51 & 3.93 & 6.2 \\
J0307-5042 & 46.95 & -50.70 & 0.55 & 4.03 & 7.92 \\
J0307-6225 & 46.83 & -62.43 & 0.579 & 3.84 & 7.63 \\
J0317-5935 & 49.32 & -59.58 & 0.469 & 4.12 & 5.96 \\
J0334-4659 & 53.54 & -46.99 & 0.485 & 4.48 & 8.29 \\
J0337-4928 & 54.45 & -49.47 & 0.53 & 3.59 & 5.14 \\
J0337-6300 & 54.46 & -63.01 & 0.48 & 3.77 & 4.81 \\
J0342-5354 & 55.52 & -53.91 & 0.53 & 3.58 & 5.11 \\
J0343-5518 & 55.76 & -55.30 & 0.55 & 3.59 & 5.61 \\
J0352-5647 & 58.23 & -56.76 & 0.649 & 3.34 & 6.41 \\
J0354-5904 & 58.56 & -59.07 & 0.41 & 4.61 & 6.19 \\
J0403-5719 & 60.96 & -57.32 & 0.466 & 4.05 & 5.58 \\
J0429-5233 & 67.43 & -52.56 & 0.53 & 3.32 & 4.08 \\
J0439-4600 & 69.80 & -46.01 & 0.34 & 5.77 & 7.86 \\
J0439-5330 & 69.92 & -53.50 & 0.43 & 4.23 & 5.32 \\
J0451-4952 & 72.96 & -49.87 & 0.39 & 4.34 & 4.6 \\
J0509-5342 & 77.33 & -53.70 & 0.461 & 4.52 & 7.57 \\
J0522-5026 & 80.51 & -50.43 & 0.52 & 3.51 & 4.62 \\
J0526-5018 & 81.50 & -50.31 & 0.58 & 3.15 & 4.25 \\
J0542-4100 & 85.71 & -41.00 & 0.642 & 3.6 & 7.82 \\
J0550-5019 & 87.55 & -50.32 & 0.65 & 2.9 & 4.17 \\
J0551-5709 & 87.90 & -57.15 & 0.423 & 4.79 & 7.42 \\
J0559-5249 & 89.925 & -52.82 & 0.609 & 3.88 & 8.76 \\
J0600-4353 & 90.06 & -43.88 & 0.36 & 5.4 & 7.35 \\
J0611-4724 & 92.92 & -47.41 & 0.49 & 3.9 & 5.56 \\
J0612-4317 & 93.02 & -43.29 & 0.54 & 3.73 & 6.02 \\
J2011-5725 & 302.85 & -57.42 & 0.279 & 5.88 & 5.16 \\
J2022-6323 & 305.52 & -63.39 & 0.383 & 4.85 & 6.17 \\
J2040-5342 & 310.21 & -53.71 & 0.55 & 3.66 & 5.94 \\
J2055-5456 & 313.99 & -54.93 & 0.139 & 11.8 & 6.99 \\
J2130-6458 & 322.72 & -64.97 & 0.316 & 5.96 & 7.25 \\
J2134-4238 & 323.50 & -42.64 & 0.196 & 9.45 & 8.85 \\
J2140-5331 & 325.03 & -53.51 & 0.56 & 3.31 & 4.56 \\
J2146-5736 & 326.69 & -57.61 & 0.602 & 3.36 & 5.57 \\
J2148-6116 & 327.18 & -61.27 & 0.571 & 3.71 & 6.7 \\
J2232-5959 & 338.14 & -59.99 & 0.594 & 3.89 & 8.39 \\
J2233-5339 & 338.32 & -53.65 & 0.44 & 4.81 & 8.23 \\
J2254-5805 & 343.58 & -58.08 & 0.153 & 9.75 & 5.08 \\
J2331-5051 & 352.96 & -50.86 & 0.576 & 3.99 & 8.46 \\
J2332-5358 & 353.10 & -53.96 & 0.402 & 5.08 & 7.89 \\
J2344-4224 & 356.14 & -42.41 & 0.29 & 5.44 & 4.49 \\
J2358-6129 & 359.70 & -61.48 & 0.37 & 4.92 & 5.92 \\
\hline
    \end{tabular}
    \caption{65 DES clusters used in this analysis. The redshift, virial mass and virial radius were obtained from \citet{Zenteno2020a} }.
    \label{DES}
\end{table}



\begin{table}
    \centering
    \begin{tabular}{cccccccccc}
    \hline
Name & \multicolumn{3}{c}{Exposure Times$^a$ (s)} &  \multicolumn{3}{c}{FWHM ('')} & \multicolumn{3}{c}{N$_{obs}$}\\
&$r$&$i$&$z$&$r$&$i$&$z$&$r$&$i$&$z$\\
%Name & ExpT$_r^a$ & ExpT$_i$ & ExpT$_z$ & FWHM$_r$ & FWHM$_i$  & FWHM$_z$ & N$_{\rm r}$ & N$_{\rm i}$ & N$_{\rm z}$\\
%&s&s&s&''&''&''&&&\\
\hline
%J0000-5748 & 2140 & 1600 & 1800 & 1.28 & 0.99 & 0.81 & 16 & 8  & 9 \\\
J0014-4952 &  --  & 1840 & 2090 &  --  & 0.95 & 0.78 & -- & 10 & 11\\
J0058-6145 &  --  & 3150 & 4746 &  --  & 1.06 & 1.16 & -- & 15 & 21 \\
J0131-5604 &  --  & 1200 & 1800 &  --  & 0.97 & 0.82 & -- & 8  & 10 \\
J0230-6028 &  --  & 1200 & 2938 &  --  & 0.94 & 1.15 & -- & 8  & 13 \\
J0310-4647 &  --  & 3750 & 1800 &  --  & 1.03 & 0.90 & -- & 27 & 10 \\
J0313-5645 &  --  & 1080 & 3080 &  --  & 1.10 & 1.17 & -- & 9  & 14 \\
J0324-6236 &  --  & 2000 & 2640 &  --  & 1.00 & 0.87 & -- & 10 & 15 \\
%J0352-5647 &  --  & 900  &  --  &  --  & 1.01 &  --  & -- & 10 & -- \\
J0406-4805 &  --  & 1575 & 2460 &  --  & 0.96 & 0.94 & -- & 8  & 12 \\
J0406-5455 &  --  & 1600 & 3750 &  --  & 1.01 & 0.88 & -- & 8  & 15 \\
J0422-4608 &  --  & 2638 & 2152 &  --  & 0.91 & 1.03 & -- & 18 & 13 \\
J0441-4855 &  --  & 1400 & 4200 &  --  & 0.84 & 0.99 & -- & 7  & 14 \\
J0528-5300 &  --  & 3000 & 3300 &  --  & 1.11 & 1.03 & -- & 15 & 11 \\
J0533-5005 &  --  & 5700 & 4815 &  --  & 0.98 & 0.86 & -- & 19 & 15 \\
%J0550-5019 &  --  &  --  & 1830 &  --  &  --  & 0.75 & -- & -- & 11 \\
J2043-5035 & 2045 & 1664 & 1400 & 1.30 & 1.10 & 0.84 & 14 & 8  & 7 \\
%J2218-5532 & 2844 & 2670 &  --  & 1.31 & 1.58 &  --  & 18 & 10 & -- \\
J2222-4834 & 1050 & 1080 &  --  & 1.39 & 0.81 &  --  & 9  & 9  & -- \\
J2228-5828 & 600  &  --  &  --  & 1.51 &  --  &  --  & 4  & -- & -- \\
J2242-4435 & 2085 & 1335 & 2000 & 1.22 & 1.52 & 0.81 & 15 & 10 & 10 \\
J2259-6057 &  --  & 3579 & 3600 &  --  & 1.13 & 1.01 & -- & 18 & 18 \\
J2352-4657 &  --  & 1582 & 5350 &  --  & 0.97 & 0.89 & -- & 9  & 22 \\

\hline
    \end{tabular}
    \caption{Observations of the {\it high-z clusters} sample. $^a$The exposure times shown  are in addition to the observations carried out by  DES \citep{abbott21}, which were included in the final  $griz$ stacks.}
    \label{DES_obs}
\end{table}
%\newpage
%\textcolor{yellow}{This paragraph is to try to explain the correlation between the dynamical state of clusters and their merger history. }
%\textcolor{red}{
%Galaxy clusters can be in different dynamical states. Some of them are relaxed, in those clusters the Intra Cluster Medium (ICM) is generally close to thermodynamic equilibrium. However,   other clusters are colliding with structures with comparable masses releasing  binding energy and  heating their ICM \cite{Sarazin2002}. This dynamical state is described using different observational proxies, among them: computing  the central cooling time, the concentration index, the centroid shift, the power ratio or the morphology index~\citep{Yuan2022}. Additionally, there are also proposed some proxies to study the dynamical state using cosmological simulations, for example:  the virial ratio $\eta$ (derived from the virial theorem), the subhalo mass fraction, which is the fraction of mass of the cluster contained in subhalos, and the offset between the center of mass of cluster with the cluster center, or a weighted combination of those parameters~\cite{Zang2022}.}
%%%%%%%%%%%%%%%%%%%%%%%%%%%%%%%%%%%%%%%%%%%%%%%%%%


% Don't change these lines
%\bsp	% typesetting comment
\label{lastpage}
\end{document}

% End of mnras_template.tex
%De Propis 2013 studied the luminosity function of and color magnitude diagrams of collisional galaxy clusters at redshift $0.2<z<0.6$ and to compare these to the luminsity functions of normal no-collifing clusters: Regardless of how advanced is the merger ot collision, there appears to be no difference between collisional and normal clusters in term of their overall or red/blue separated luminosity functions 
%We divided our sample in 4 bins covering 8 2D space of (z,Relaxed)
% A relaxed, dynam- ically old group or cluster should be characterized by a central galaxy which is the brightest (most massive) member by a significant margin (e.g. Khosroshahi, Ponman, and Jones, 2007; Dariush et al., 2010; Smith et al., 2010) and is located near the minimum of the potential well (e.g. George et al. 2012; Zitrin et al. 2012, however also see Skibba et al. 2011), satellite galaxies which are distributed in velocity space according to a Gaussian profile (e.g. Yahil and Vidal, 1977; Bird and Beers, 1993; Hou et al., 2009; Martínez and Zandivarez, 2012), and diffuse X-ray emission which is symmetric about the group/cluster centre (e.g. Rasia, Meneghetti, and Ettori, 2013; Weißmann et al., 2013; Parekh et al., 2015).
%The dynamical state of clusters is related to the age of the halo and the time since infall for member galaxies, which simulations have shown is an important quantity in determining the degree to which galaxy properties are affected by environment (e.g. Wetzel et al., 2013; Oman and Hudson, 2016; Joshi, Wadsley, and Parker, 2017). 
%The longer a galaxy has been part of the harsh cluster environment, the higher the likelihood that that galaxy will be quenched. On average, unrelaxed clusters host galaxies with shorter times-since-infall (see Chap- ter 3), therefore it could be expected that unrelaxed clusters host more active, star-forming galaxies than more relaxed systems. In Chapter 4 we show that this is the case, and enhanced star-forming fractions are observed in NG clusters relative to G counterparts. This is true even after controlling for the known de- pendences of star formation on stellar mass and group/cluster mass. There also may be differences in the efficiency of quenching processes in relaxed and unre- laxed clusters. Unrelaxed clusters tend to host underdense ICMs (Popesso et al., 2007; Roberts and Parker, 2017), particularly in the cluster interior, which may
%Errors are Poissonian and are computed using the formulae of Gehrels (1986).