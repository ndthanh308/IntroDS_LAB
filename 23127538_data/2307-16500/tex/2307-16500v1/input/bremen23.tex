%\documentclass[11pt]{article}
%
%\topmargin=0cm
%\headheight=0cm
%\headsep=0cm
%\oddsidemargin=0cm
%\textheight=24cm
%\textwidth=16cm
%\evensidemargin=0cm
%
%\usepackage{amsmath,amsthm,mathabx}
%\usepackage{cleveref}
%\usepackage{stmaryrd}
%\usepackage{amsfonts}
%\usepackage{cite}
%%\binoppenalty=-1000\relax
%%\relpenalty=-8000\relax
%
%\newcommand{\N}{\mathbb{N}}
%\newcommand{\wleq}{\preccurlyeq}
%\newcommand{\wless}{\prec}
%\DeclareMathOperator{\mdepth}{mdepth}
%
%%\theoremstyle{plain}
%\theoremstyle{definition}
%\newtheorem{dfn}{Definition}[section]
%\newtheorem{thm}[dfn]{Theorem}
%\newtheorem{lem}[dfn]{Lemma}
%
%\newtheoremstyle{claim}{}{}{\rmfamily}{}{\itshape}{:}{ }{\underline{\thmname{#1}\thmnumber{ #2}\thmnote{ #3}}}
%\theoremstyle{claim}
%%\newtheorem{clm}[dfn]{Claim}
%\newtheorem{clm}{Claim}
%
%\crefname{thm}{Theorem}{Theorems}
%\crefname{lem}{Lemma}{Lemmata}
%
%\newcommand\ab{\allowbreak}
%
%\newcommand\lcop[1]{\left\lfloor{#1}\right\rfloor}
%\newcommand\paras[1]{\mathsf{ps}(#1)}
%\newcommand\nop{\$}
%\newcommand\pout{\mathit{pOut}}
%\newcommand\rhsnop[1]{\left\lceil{#1}\right\rceil}
%\newcommand\ot{\leftarrow}
%\newcommand\eps{\varepsilon}
%\newcommand\rhs{\mathsf{rhs}}
%\def\<#1>{\left\langle #1\right\rangle}
%\newcommand\stpair[1]{\<\!\<#1>\!>}
%%\newcommand\sosubst[2]{\left[\!\left[#1\ot #2\right]\!\right]}
%%\newcommand\sosubst[2]{[\![#1\ot #2]\!]}
%%\newcommand\sosubst[2]{\left\llbracket #1\ot #2\right\rrbracket}
%%\newcommand\sosubst[2]{{\renewcommand\mmid{\mid}\llbracket #1\ot #2\rrbracket}}
%\newcommand\sosubst[2]{\llbracket #1\ot #2\rrbracket}
%\newcommand\sosub[1]{\llbracket #1\rrbracket}
%%\newcommand\subst[2]{\left[#1\ot #2\right]}
%\newcommand\subst[2]{[#1\ot #2]}
%%\newcommand\mmid{\;\middle|\;}
%\newcommand\mmid{\mid}
%\DeclareMathOperator{\rank}{rank}
%
%\begin{document}
%
%\section{Depth proper normal form}

\section{Depth proper normal form}

Because the LSHI and LHI properties of MTTs pertain to the growth of the height of the output, those properties are linked to the nesting of state calls in the output. For example a MTT that is finite-nesting is trivially LSHI, the converse is however not true because some state calls do not always increase the height of the output tree. The goal of this normal form is to remove those states that have a bound on the amount of height they contribute to the output, i.e.\ the depth at which they use their parameters. This normal form will allow us to characterize LSHI and LHI by finite-nesting constraints. 

To obtain this \emph{Depth-proper normal form} we first show in section \ref{ssec:remove_improper_states} how to remove those state calls which have a bound on the depth at which a parameter occurs in their output (we call them \emph{improper state calls}). This process involves adding new look-ahead states which predict the paths (of bounded depth) at which these parameters will appear in the output. This allows us to compute early the branches of the output leading to the parameter, we then add helper states which will compute other branches of the output. 

In order to remove \emph{improper state calls}, we have altered the transducer in a way that may introduce new \emph{improper state calls} (similarly to the normal form in \cite{DBLP:journals/siamcomp/EngelfrietM03}). In subsection \ref{ssec:normal_form_termination} we show that by iterating the process of removing improper state calls we eventually obtain a MTT without any improper state calls, i.e.\ a MTT in \emph{depth-proper normal form}. 

% Charles' version:
%Finite-nesting of state calls constrain how the height of the output tree grows with respect to the input. As such, this family of properties make for good candidate characterizations of LSHI and LHI MTTs. Sadly, the equivalence doesn't hold when there are states which only output their parameters at a bounded depth : those can be nested arbitrarily many times without breaking the LSHI / LHI bounds. We thus seek to find a normal form for MTTs that doesn't use any such states, allowing us to directly relate the LSHI and LHI properties to different forms of finite-nesting.

%In this section, we first show how to remove these problematic state calls, and then how this can be iterated to yield the normal form we want.

\subsection{Removing improper state calls}\label{ssec:remove_improper_states}

First, we need to recognise that for a given state call may never appear in
any derivation if the rules of the MTT don't allow it. We use the general notion of
"reachability" defined in \cite{DBLP:journals/iandc/EngelfrietM99} to formalise which improper state calls actually appear in
practice, and recall the following definition : 

\begin{dfn}
Let $M = (Q, P, \Sigma, \Delta, q_0, R, h)$ be a $\MTTR$. The extension of $M$,
denoted by $\hat{M}$, is the $\MTTR (Q, P, \hat{\Sigma}, \hat{\Delta}, q_0,
\hat{R}, \hat{h})$, where $\hat{\Sigma} = \Sigma \cup \{p^{(0)} \mmid p \in P\},
\hat{\Delta} = \Delta \cup \stpair{Q, P}, \hat{R} = R \cup \{\apa{q,p}(y_1, \dots,
y_m) \rightarrow \stpair{q, p}(y_1, \dots,y_m) \mmid \<q,p> \in \<Q, P>^{(m)}\},
\hat{h}_p() = p$ for $p \in P$, and $\hat{h}_\sigma(p_1, \dots, p_k) =
h_\sigma(p_1, \dots, p_k)$ for $\sigma \in \Sigma^{(k)}, k \geq 0,$ and $p_1,
\dots, p_k \in P$.
\end{dfn}

$\hat{M}$ extends the behavior of $M$ to process symbols of the look-ahead.
These symbols $p \in P$ take on the same role as subtrees $t \in L_p$, and
intuitively, if $\stpair{q, p}$ appears in $\hat{M}(s)$ with $s \in T_\Sigma$,
then replacing $p$ with $t_p \in L_p$ in $s$ yiels a tree $s'$ such that $\<q, t_p>$
appears in the derivation of $M(s')$. We say that a state call $\stpair{q, p}$
is \emph{reachable} if and only if there exists a tree $ s \in T_\Sigma(P)$ such that
$\stpair{q, p}$ appears in $\hat{M}(s)$.

For any given state where a parameter appears only at bounded depth in the output, there are only finitely many output tree branches leading to this parameter. With the help of our look-ahead, we can guess from the input tree which of these branches will be produced, and forgo the state call by immediately producing them.

The \emph{least common output form with parameters} 
\(\lcop{s}_{Y'} \in T_{\Delta\cup\<Q, X_k>\cup\{\nop^{(0)}\}}(Y)\) expresses this idea of output tree branches.
For each \(s\in T_{\Delta\cup\<Q, X_k>}(Y)\), it is obtained from \(s\) by replacing all occurrences of subtrees that do not
contain any parameter \(y\in Y'\) with \(\nop\).
Let us write \(\paras{s}\subseteq{Y}\) for the set of parameters occurring in
\(s\in T_{\Delta\cup\<Q, X>}(Y)\).
Then we can define \(\lcop{s}_{Y'}\) inductively as follows:
\begin{align*}
	\lcop{s}_{Y'} &= 
	\begin{cases}
		y_i &
		\text{if \(s = y_i \in Y'\)}
		\\
		\delta(\lcop{s_1}_{Y'},\dots,\lcop{s_n}_{Y'}) &
		\text{if \(\paras{s} \ne \emptyset\) and \(s = \delta(s_1,\dots,s_n)\)}
		\\
		\nop & \text{if \(\paras{s} = \emptyset\)}
	\end{cases}
\end{align*}
For convenience,
we define \(\pout_M((q,Y'),p)\) to be the set \(\left\{\lcop{M_q(s)}_{Y'} \mmid s\in
L_p\right\}\). For simplicity, we also write $\pout_M((q, y), p)$ instead of
$\pout_M((q,\{y\}), p)$ and $\lcop{s}_y$ instead of $\lcop{s}_{\{y\}}$.
If a state call $\<q, x_i>$ appears in a right-hand side
$\rhs_M(q,\sigma,\<p_1,\dots,p_k>)$ and $y \in Y$ is such that $\pout((q,y),
p_i)$ is finite, we call the state call $\<q, x_i>$ "improper" and the
parameter $y$ one of its "depth-bounded" parameter.

If parameters could appear at arbitrary depth during the derivation of the
ouput, then by the non-deletion property, they would be able to also appear at
arbitrary depth in the ouput. For this reason, improperness of state calls
propagates through rule right-hand sides to the other state calls making use of
their depth-bounded parameters. The following lemma formalises this fact.

\begin{lem}\label{thm:finpout}
	Let \(M=(Q,P,\Sigma,\Delta,q_0,R,h)\) be a nondeleting mtt with look-ahead.
	Let \(q\in Q\), \(\sigma\in\Sigma^{(k)}\), \(k\ge1\), $y \in Y$,and \(p,p_1,\dots,p_k\in P\) such that
	\(p = h_\sigma(p_1,\dots,p_k)\) and \(L_{p_j}\ne\emptyset\) for every \(j\in [k]\).
	If \(\<r,x_i>\in \<Q,X_k>\) occurs in
	\(\lcop{\rhs_M(q,\sigma,\<p_1,\dots,p_k>)}_y\) at position $u$, with $r \in
	Q^{(m)}$ and \(\pout((q,y),p)\) is finite,
	then \(\forall l \in [m], \lcop{\rhs_M(q,\sigma,\<p_1,\dots,p_k>)}_y[ul] \neq \$
	\Rightarrow \pout((r,y_l),p_i)\) is finite.
\end{lem}
%
\begin{proof}
	For $j \in [k]-\{i\}$ fix trees $s_j \in T_\Sigma$ with $h(s_j)=p_j$. Let
	$\xi=\zeta\sosub{\ldots}$ with $\zeta=\rhs_M(q,\sigma,\< p_1,\ldots,p_k>)$ and
	$\sosub{\ldots}=\sosubst{\<q',x_j>}{M_{q'}(s_j) \mmid q'\in Q,j\in
		[k]-\{i\}}$. By the definition of $\pout_M((q,y),p)$, Lemma 3.5 from
	\cite{DBLP:journals/siamcomp/EngelfrietM03}, and
	associativity of second-order tree
	substitutions,$O=\{\lcop{M_q(\sigma(s_1,\ldots,s_k))}_y\mmid s_i\in L_{p_i}\}=
	\{\lcop{\xi\sosub{s_i}}_y \mmid s_i\in L_{p_i}\}$, where $\sosub{s_i}$ denotes the
	subsitution $\sosubst{\<q',x_i>}{M_{q'}(s_i) \mmid q'\in Q}$ is a subset of
	$\pout_M((q,y),p)$ and hence finite. Since $M$ is nondeleting, both
	$\sosub{\ldots}$ and $\sosub{s_i}$ are nondeleting by Lemma 3.11 from 
	\cite{DBLP:journals/siamcomp/EngelfrietM03}. Hence, by Lemma 2.1 from \cite{DBLP:journals/siamcomp/EngelfrietM03}, $\xi$ has a subtree $\<r,
	x_i>(\xi_1,\ldots,\xi_m)$, where $m = \rank_Q(r)$. Again by the same Lemma 2.1,
	$\xi\sosub{s_i}$ has a substree $\<r,x_i>(\xi_1,\ldots,\xi_m)\sosub{s_i} =
	M_r(s_i)[y_j \ot \xi_j\sosub{s_i} \mmid j\in [m]]$. Letting Y' be the set
$\{y_l\in Y \mmid \lcop{\rhs_M(q,\sigma,\<p_1,\dots,p_k>)}_y[ul] \neq \$ \}$, we thus
have, for every $t\in
	\pout_M((r,Y'),p_i)$ (i.e., $t=\lcop{M_r(s_i)}_{Y'}$ for some $s_i\in L_{p_i}$) the tree
	$t[y_j \ot \xi_j\sosub{s_i} \mmid j\in [m]]$ is a subtree of
	$\lcop{\xi\sosub{s_i}}_y$, i.e., it is a subtree of a tree in the finite set $O$. This
	implies finiteness of $\pout_M((r,Y'),p_i)$, which is equivalent to finiteness
of $\pout_M((r, y_l), p_i)$ for all $y_l \in Y'$.
\end{proof}

Recall the definition of $\pout((q,y), p)$ we gave, which is the set
$\{\lcop{M_q(s)}_y \mmid s \in L_p\}$. We can now define what is means for a
given $\MTTR$ to be depth-proper.
\begin{dfn}
Let $M = (Q, P, \Sigma, \Delta, R, h)$ be a $\MTTR$. $M$ is depth-proper if and
only if, for all reachable state call $\stpair{q,p}$, the parameters of $q$ can appear
at arbitrary depth in the output $M_q(L_p)$. Formally, we write $\forall q \in
Q, \forall p \in P, \stpair{q,p} $ is reachable $ \Rightarrow \forall y \in Y,
\pout((q, y), p) $ is infinite.
\end{dfn}
For each \(p\in P\),
we define \(F_p = \left\{ (q, y)\in Q*Y \mmid \text{\(\pout_M((q,y),p)\) is
	finite} \right\}\), and $F_p^1 = \left\{q \in Q \mmid \exists y \in Y, (q,y) \in
F_p\right\}$. For convenience, for all $q\in Q$, we write $F_p(q)$ for the set
$\{y \in Y \mmid (q,y) \in F_p\}$.

We introduce the notation $\pout^f((q,Y), p)$ which is the same as $\pout$
except that all signs $\nop$ have been replaced by sets of (indexes of)
parameters of $q$. Formally, it is defined as $\{s \in
T_{\Sigma \cup \mathcal{P}([rank_M(q)])^{(0)}} \mmid s[u \leftarrow \nop \mmid
s/u \in \mathcal{P}([rank_M(q)])] \in \pout((q,Y), p)\}$.
Let \(\Phi_p\) be the set of mappings \(\varphi\) from \(F^1_p\)
to \(T_{\Delta\cup\mathcal{P}(\N)^{(0)}}(Y)\)
such that \(\varphi(r)\in\pout^f_M((r, F_p(r)),p)\).
%so that 
%\{ \varphi: F_p\to T_{\Delta\cup\{\nop^{(0)}\}}(Y) \mid
\(\Phi_p\) is finite for every \(p\in P\) because the different $\pout$ are by
definition finite if we take $q \in F^1_p$, and the cardinality of $\pout^f$ is
a finite factor away from that of $\pout$.

We will now construct a mtt \(\pi(M)\) equivalent to \(M\)
where there is no occurrence of any improper state call in a right-hand side.
Like stated previously, the look-ahead is enriched with new information, so we
can guess the branches leading to depth-bounded parameters and skip improper
state calls. But we still need to reconstruct the rest of the output tree
afterwards. This is achieved by introducing helper states $\hat{Q}$.
Every call to a helper states produces a missing branch, and uses only some part
of the non depth-bounded parameters. We need to not give them the ones they
don't use if we want to keep the non-deleting property. Thus, the lookahead is
augmented with information about which non depth-bounded is needed in which
branch.

\begin{dfn}
Let $M = (Q,P,\Sigma,\Delta,R,h)$. Then the mtt
\(\pi(M)=(Q',P',\Sigma,\Delta,R',h')\) is given as follows.  Let \(Q'\) be \(Q
\cup \hat{Q}\)
where \\
\(\hat{Q}=\left\{(p,r,t/v,u)^{(n)}\mmid p\in P, r\in F^1_p,
t\in\pout((r,F_p(r)),p), t/vu =
\nop, n\in [rank_M(q)]\right\}\).
Let \(P'\) be 
\(\left\{(p,\varphi) \mmid \varphi \in \Phi_p\right\}\).
For every \(q\in Q\), \(\sigma\in\Sigma\), and \((p_1,\varphi_1),\dots,(p_k,\varphi_k)\in P'\),
let the rule
\begin{align*}
	\<q, \sigma(x_1,\dots,x_k)>(y_1,\dots,y_m) &\to \zeta_q\Theta
	&\<(p_1,\varphi_1),\dots,(p_k,\varphi_k)>
\end{align*}
be in \(R'\), 
where 
\(\zeta_q=\rhs_M(q, \sigma, \<p_1,\dots,p_k>)\)
%a rule \((\<q, \sigma(x_1,\dots,x_k)>(y_1,\dots,y_m) \to \zeta_q ~
%\<p_1,\dots,p_k>)\) is in \(R\),
and
\(\Theta\) is a second-order tree substitution given as
%\(\sosubst{\<r,x_i>}%
%{\varphi_i(r) \left[u \ot \<(p_i,r,\varphi_i(r),u),x_i> \mid \varphi_i(r)/u = \nop\right] \mid r\in F_{p_i}, i\in[k]}\).
\(\sosubst{\<r,x_i>}%
{t \left[u \ot \<(p_i,r,t,u),x_i>(y_i, i \in t/u) \mmid t/u \in \mathcal{P}(\N)\right] \mmid r\in F^1_{p_i}, t=\varphi_i(r), i\in[k]}\).
%%\(\sosubst{\<r,x_i>}{\rhsnop{\varphi_i(r)} \mid r\in F_{p_i}, i\in[k]}\)
%%with an auxiliary function \(\rhsnop{t}\) for \(t\in T_{\Delta\cup\{\$^{(0)}\}}\)
%where every occurrence of \(\nop\) at path \(u\) of \(\varphi_i(r)\) is replaced
%with \(\<(p_i,r,t,u), x_i>\)
%
%%Formally the function \(\rhsnop{t}\) is defined 
%%as \(\rhsnop{t}=\rhsnop{t}_t^{\eps}\) where
%%\begin{align*}
%%\rhsnop{\nop}_t^{u} &= \<(p_i,r,t,u),x_i>
%%\\
%%\rhsnop{y_j}_t^{u} &= y_j
%%\\
%%\rhsnop{\delta(t_1,\dots,t_n)}_t^{u} &=
%%\delta(\rhsnop{t_1}_t^{u1},\dots,\rhsnop{t_n}_t^{un})
%%\text.
%%\end{align*}
For every \((p,r,t,u)\in\hat{Q}\) and \(\sigma\in\Sigma\),
let the rule 
\begin{align*}
	\<(p,r,t,u)(y_1, \dots, y_l), \sigma(x_1,\dots,x_k)> &\to \phi(\zeta_r, t, u)
	&\<(p_1,\varphi_1),\dots,(p_k,\varphi_k)>
\end{align*}
be in \(R'\),
where 
\(\zeta_r=\rhs_M(r,\sigma,\<p_1,\dots,p_k>)\)
%\(\<r, \sigma(x_1,\dots,x_k)>(y_1,\dots,y_m) \to \zeta_r\) is in \(R\)
and the function \(\phi(\zeta, t, u)\) is defined by 
\begin{align*}
	%\phi(y_j, t', u') &= y_j 
	%\\
	\phi(\zeta, \nop, \eps) &= \zeta
	\\
	\phi(\<q,x_j>, t, u) &= \<(p_j,q,t,u),x_j>(y_1,\dots,y_l) &&\text{where \(t\ne\nop\)}
	\\
	\phi(\delta(\zeta_1,\dots,\zeta_n), \delta(t_1,\dots,t_n), iu)
	&= \phi(\zeta_i, t_i, u)\text.
\end{align*}
\end{dfn}
Note that \(\phi(\zeta, t, u)\) may be partial. No rule is constructed if its right-hand side is not defined.
In addition, let \(h'_\sigma((p_1,\varphi_1),\dots,(p_k,\varphi_k)) = (p, \varphi)\),
where \(p = h_\sigma(p_1,\dots,p_k)\),
\(\varphi = \{ q\mapsto\lcop{\zeta_q\Theta'}_{F_p(q)} \mmid q\in F^1_p, 
\zeta_q=\rhs_M(q,\sigma,\<p_1,\dots,p_k>)\}\),
and 
\(\Theta'\) is a second-order tree substitution given as
\(\sosubst{\<r,x_i>}{\varphi_i(r) \mmid r\in F^1_{p_i}, i\in[k]}\).

The following lemma shows that $\varphi$
is indeed one of the finitely many functions in $\Phi_p$, which is needed
because we need to be sure the look-ahead has only finitely many states, and also that
$\varphi$ indeed contains the branches leading to the depth-bounded parameters
for every $q \in Q$ such that its call on $x_i$ would be improper.

\begin{clm}\label{clm:pi-ra}
	%\begin{lem}
	%Let \(M=(Q,P,\Sigma,\Delta,R,h)\) be a nondeleting mtt with look-ahead,
	%\(\pi(M)=(Q',P',\Sigma,\Delta,R',h')\), and \(s\in T_\Sigma\).
	If \(h'(s) = (p, \varphi)\), 
	then \(p = h(s)\) and \(\varphi(q) = \lcop{M_q(s)}_{F_p(q)}\) for every \(q\in
F^1_p\).\\
\end{clm}
\begin{proof}
	The claim is shown by induction on \(s\). 
	Let \(s=\sigma(s_1,\dots,s_k)\) with \(\sigma\in\Sigma^{(k)}\)
	and \(s_1,\dots,s_k\in T_\Sigma\),
	and \(h'(s_i)=(p_i,\varphi_i)\) for every \(i\in[k]\).
	By the definition of \(h'\) and the induction hypothesis, 
	for every \(q\in F_p\),
	we have \(p = h(s)\) and
	\begin{align*}
		\varphi(q) &=
		\lcop{\zeta_q\Theta'}
		\\&=
		\lcop{\zeta_q\sosubst{\<r,x_i>}{\varphi_i(r)\mid r\in F_p, i\in[k]}}
		\\&=
		\lcop{\zeta_q\sosubst{\<r,x_i>}{\lcop{M_r(s_i)}\mid r\in F_p, i\in[k]}}
		\\&= 
		\lcop{M_q(s)}
	\end{align*}
	where \(\zeta_q=\rhs_M(q,\sigma,\<p_1,\dots,p_k>)\).
\end{proof}
%
%\begin{align*}
%&\pi(M)_{(p,r,t/v,u)}(s)
%\\&=\rhs_{\pi(M)}((p,r,t/v,u),\sigma,\<(p_1,\varphi_1),\dots,(p_k,\varphi_k)>)
%\sosubst{\<q',x_i>}{\pi(M)_{q'}(s_i)\mid q'\in Q, i\in[k]]}
%\\&=
%\phi(\rhs_{M}(r,\sigma,\<p_1,\dots,p_k>),t/v,u)
%\sosubst{\<q',x_i>}{\pi(M)_{q'}(s_i)\mid q'\in Q, i\in[k]]}
%\\&=
%\end{align*}
%(We need the lemma about \(\phi\) here.)
%
%\begin{align*}
%&(t/v)\subst{u}{\pi(M)_{(p,r,t/v,u)}(s) \mmid t/vu=\nop}
%\\&=
%(\lcop{M_r(s)}/v)\subst{u}{\pi(M)_{(p,r,\lcop{M_r(s)}/v,u)}(s) \mmid \lcop{M_r(s)}/vu=\nop}
%\\&=
%(\lcop{M_r(s)}\subst{vu}{\pi(M)_{(p,r,\lcop{M_r(s)}/v,u)}(s) \mmid \lcop{M_r(s)}/vu=\nop})/v
%\end{align*}

We now verify that the definition for the right-hand sides of helper state rules
is sound. We show that all recursive calls are defined by assuming that we can't
apply the last rule, and proving that then one of the other two must necessarily
apply.

\begin{clm}
	For all \( (p,r,t,u) \in \hat{Q}, (p_1,\dots, p_k) \in P^k \) and $\sigma \in
	\Sigma$ such that $p=h_\sigma(p_1, \dots, p_k)$, and defining $
	\zeta_r = \rhs_M(r, \sigma, \<p_1,\dots,p_k>)$, \( \phi(\zeta_r, t, u) \)
	is well-defined.
\end{clm}

\begin{proof}
	First, let us quickly verify that the computation of the value of $ \phi $ eventually
	terminates. Initially, $t$ and $u$ come from a state $(p, r, t, u)
\in \hat{Q}$, so $t/u = \nop$ by definition. This holds true after every
recursive call, and because every recursive call also stricly decreases the
length of the third parameter $u$, the recursion must eventually end, thanks to
the existence of the ground case $\phi(\zeta,\nop,\epsilon)=\zeta$.

Now we just need to verify that all cases are covered. The computation of
$\varphi$ can be understood as synchronous exploration of the first and second
arguments $\zeta$ and $t$ following the path given by the third argument $u$. We
show that at any point were the two trees differ (that is, we cannot apply the
third rule to continue exploring if we ever get there), then the
computation should either stop by application of the first or second rule, or
should've stopped for the same reason earlier.

This is formalised as $
\forall u \in \N^*, \zeta[u] \neq t[u]
	\Rightarrow \exists v \wleq u, \zeta[v] = \<q, x> $ or $ t[v] = \$ $, and we
will actually prove its contrapositive.
	Let then $ u \in \N^* $. If $u$ is absent in either $ pos(\zeta) $ or $ pos(t) $,
	take $u'$ to be the longest strict prefix of $u$ such that $u' \in pos(\zeta) \cap
	pos(t) $, and $ i \in \N $ such that $ u'i \wleq u $. Then $ u'i $ is in only
	one set of the two sets $ pos(\zeta) $ and $ pos(t) $, we must have by necessity
	that $ \zeta[u'] \neq t[u'] $, because their degrees are different. We have thus
	reduced our problem to the case where $ u \in pos(t) \cap pos(\zeta) $.
	Because $(p, r, t, u) \in \hat{Q}$, we know $ \exists Y' \subseteq Y,
	s_1,\dots,s_k \in L_{p_1},\dots,L_{p_k}, t = \lcop{M_r(\delta(s_1,
		\dots,s_k)}_{Y'} $. Note $ s = \delta(s_1,\dots,s_k)$. To prove the
	contrapositive, assume $ \forall v \wleq u, \zeta[v] \neq \<q, x> $ and $ t[v]
	\neq \$ $. By definition of $ \lcop{.} $, this means that $ \paras{t/u} \neq
	\emptyset $, and as such $ \forall v \wleq u, t[v] = M_r(s)[v] $. Applying lemma
	3.5 from \cite{DBLP:journals/siamcomp/EngelfrietM03} to $M_r(s)$, we get $M_r(s) = \zeta\sosubst{\<q,
		x_i>}{M_q(s_i)}$. We conclude by repeated applications of the definition of
	second-order substitutions : $ \delta(t_1,\dots,t_n)\sosubst{\delta'}{t'} =
	\delta(t_1\sosubst{\delta'}{t'},\dots,t_n\sosubst{\delta'}{t'}$ if $ \delta \neq
	\delta' $ along the path to node $u$, proving $M_r(s)[u] =
	\zeta\sosubst{\<q,x_i>}{M_q(s_i)}[u] = \zeta[u]$, and thus $t[u]=\zeta[u]$,
	proving the contrapositive.
	We can now show that $\phi$ is well-defined : by our observation, any call on
	$\phi$ such that $\zeta[u] \neq t[u]$ is either such that one of
	the two other rules apply (if $v=u$), or impossible (if $v \wless u$ then the
	recursive calls should have stopped sooner at $v$).
	
\end{proof}

\begin{lem}\label{lem:pi-eq}
	Let \(M=(Q,P,\Sigma,\Delta,R,h)\) be a nondeleting mtt with look-ahead.
	For \(q\in Q\) and \(s\in T_\Sigma\), 
	we have \(\pi(M)_q(s) = M_q(s)\).
\end{lem}
%
\begin{proof}
	We prove the statement by induction on \(s\in T_\Sigma\).
	Let \(t=\lcop{M_r(s)}\) and \(r\in F_p\) with \(p=h(s)\).
	Then \(t\subst{u}{\pi(M)_{(p,r,t,u)}(s) \mmid t/u=\nop} = M_r(s) \).
	More generally, we have 
	\((t/v)\subst{u}{\pi(M)_{(p,r,t/v,u)}(s) \mmid t/vu=\nop} = M_r(s)/v \)
	for every path \(v\) of \(t\).
	
	Let \(s=\sigma(s_1,\dots,s_k)\) with \(\sigma\in\Sigma^{(k)}\) and \(s_1,\dots,s_k\in T_\Sigma\),
	and \(h'(s_i) = (p_i, \varphi_i)\) for every \(i\in[k]\).
	From the induction hypothesis, we have \(\pi(M)_{q}(s_i) = M_{q}(s_i)\) for every \(q\in Q\) and \(i\in[k]\).
	In addition, \(p_i = h(s_i)\) and \(\varphi_i(r) = \lcop{M_{r}(s_i)}\) holds for \(r\in F_{p_i}\) and \(i\in[k]\)
	because of \cref{clm:pi-ra}.
	By the definition of \(\pi(M)\),
	we have
	\begin{align*}
		&\pi(M)_q(s) 
		\\&=
		\rhs_{\pi(M)}(q,\sigma,\<(p_1,\varphi_1),\dots,(p_k,\varphi_k)>)
		\sosubst{\<r,x_i>}{\pi(M)_r(s_i)\mmid r\in Q', i\in [k]}
		\\&=
		\rhs_{M}(q,\sigma,\<p_1,\dots,p_k>)
		\hspace{-.4em}\begin{array}[t]{l}
			\sosubst{\<r,x_i>}{t\left[u \ot \<(p_i,r,t,u),x_i> \mmid t/u = \nop\right] 
				\mmid r\in F_{p_i}, t=\varphi_i(r), i\in [k]}\\
			\sosubst{\<q',x_i>}{\pi(M)_r(s_i)\mmid q'\in Q', i\in [k]}
		\end{array}
		\\&=
		\rhs_{M}(q,\sigma,\<p_1,\dots,p_k>)
		\hspace{-.4em}\begin{array}[t]{l}
			\sosubst{\<r,x_i>}{t\left[u \ot \<(p_i,r,t,u),x_i> \mmid t/u = \nop\right] 
				\mmid r\in F_{p_i}, t=\lcop{M_r(s_i)}, i\in [k]}\\
			\sosubst{\<q',x_i>}{\pi(M)_r(s_i)\mmid q'\in Q', i\in [k]}
		\end{array}
		\\&=
		\rhs_{M}(q,\sigma,\<p_1,\dots,p_k>)
		\hspace{-.4em}\begin{array}[t]{l}
			\sosubst{\<q',x_i>}{\pi(M)_r(s_i)\mmid q'\in Q\setminus F_{p_i}, i\in [k]}\\
			\sosubst{\<r,x_i>}{t\left[u \ot \<(p_i,r,t,u),x_i> \mmid t/u = \nop\right]
				\mmid r\in F_{p_i}, t=\lcop{M_r(s_i)}, i\in [k]}\\
			\sosubst{\<\hat{q},x_i>}{\pi(M)_{\hat{q}}(s_i)\mmid \hat{q}\in \hat{Q}, i\in [k]}
		\end{array}
		\\&=
		\rhs_{M}(q,\sigma,\<p_1,\dots,p_k>)
		\hspace{-.4em}\begin{array}[t]{l}
			\sosubst{\<q',x_i>}{M_r(s_i)\mmid q'\in Q\setminus F_{p_i}, i\in [k]}\\
			\sosubst{\<r,x_i>}{t\left[u \ot \pi(M)_{(p_i,r,t,u)}(s_i) \mmid t/u = \nop\right]
				\mmid r\in F_{p_i}, t=\lcop{M_r(s_i)}, i\in [k]}
		\end{array}
		\\&=
		\rhs_{M}(q,\sigma,\<p_1,\dots,p_k>)
		\hspace{-.4em}\begin{array}[t]{l}
			\sosubst{\<q',x_i>}{M_r(s_i)\mmid q'\in Q\setminus F_{p_i}, i\in [k]}\\
			\sosubst{\<r,x_i>}{M_r(s_i)\mmid r\in F_{p_i}, i\in [k]}
		\end{array}
		\\&=
		\rhs_{M}(q,\sigma,\<p_1,\dots,p_k>)
		\sosubst{\<q',x_i>}{M_r(s_i)\mmid q'\in Q, i\in [k]}
		\\&=
		M_q(s)\text.
	\end{align*}
	
\end{proof}

\subsection{Iterated removal of improper states terminates}\label{ssec:normal_form_termination}

Having proved that our application preserves the semantics of MTTs, we still
need to show that it yields a depth-proper MTT as output. This isn't actually
true, and we will need to iterate our transformation to actually obtain a
depth-proper MTT. To understand why, remember that our transformation splits our
initial look-ahead states into multiple look-ahead states that recognise a
partition of the same language (a look-ahead state $p$ is split into $(p,
\phi)$). A state that was depth-proper in $M$ with respect to look-ahead state
$p$ may then have only finitely many least common output with parameters with
respect to any look-ahead state $(p, \phi)$ in $\pi(M)$. Another problem is that
helper states from $\hat{Q}$ may not be depth-proper when they are created.

What we will show instead is that iteration of $\pi$ yields a fixed point of
$\pi$ after only finitely many steps, and that any fixed point of $\pi$ is
indeed depth-proper.
The proof of termination is done by showing with downwards induction that for
any stricly positive arity, helper states with this arity eventually stop being
introduced by iteration of $\pi$. When we eventually introduce only
parameter-less helper states, which are by definition already depth-proper, an
adaptation of the argument from \cite{DBLP:journals/siamcomp/EngelfrietM03} can be used.

Let then $M = M_0$ be a MTT, and define $(M_n)_{n \in \N}$ by the relation
$M_{n+1} = \pi(M_n)$. Also let $M_n = (Q_n, P_n, \Sigma, \Delta, R_n, h_n) $ and
$d \geq 2$. Assuming that $\forall e \geq d, \forall n \in \N, Q_n^{(e)} \subseteq
Q_0^{(e)} $, we will show that $\exists n_0 \in \N, \forall e \geq d-1, \forall n
\geq n_0, Q_n^{e} \subseteq Q_0^{e}$.  First, observe that $ \forall n \in \N,
\forall p \in P_n, \forall (p, \phi) \in P_{n+1}, F_p \cap Q_0 \subseteq F_{(p,
	\phi)} \cap Q_0 \subseteq Q_0 $. This is because $ \forall r \in Q_0, \pout(r,
(p, \phi)) \subseteq \pout(r, p) $. Since $Q_0$ is a finite set, then every
infinite sequence $ (F_{p_n}) $ with $ p_n \in P_n $ and $p_n \in p_{n+1}$ must
become stationary at one point. Let $n_0' \in \N$ such that $\forall n \geq n_0',
\forall p \in P_n, \forall (p, \phi) \in P_{n+1}, F_p \cap Q_0 = F_{(p, \phi)}
\cap Q_0 $. By definition of $\Theta$, there is no rule $\<q, \sigma(x_1, \dots,
x_k)> \rightarrow \zeta \<p_1, \dots, p_k>$ in $R_{n_0'}$ such that $\exists q \in
Q_{n_0'}, \exists i, \exists y, \<q, x_i> \in \zeta$ and $(q,y) \in F_{p_i}$,
i.e. $\exists q \in Q_{n_0' +1}, \exists i, \exists y, \<q, x_i> \in \zeta$ and
$(q,y) \in F_{p_i}\cap Q_{n_0'}$. Furthermore, by the definition of $\pi$, any call
to a state of $Q_{n_0'}$ appearing in the right hand side of a rule of $R_n$ for
$n > n_0'+1$ must appear in the right hand side of a rule of $R_{n_0'+1}$. We now
know by the fact that $F_p = F_{(p, \phi)}$ for $n \geq n_0'$ that there is no
call to a state $q \in Q_0 \subseteq Q_{n_0}$ that is rewritten by $\Theta$ in a
rule from $R_n, n\geq n_0+1$, and because $Q_0$ contains by hypothesis all
states of arity $\geq d$ and helper states come from states of strictly higher
arity, then no helper states of arity $\geq d-1$ are introduced in $Q_n, n \geq
n_0$, i.e $\forall e \geq d-1, \forall n \in \N, Q^{(e)}_n \subseteq Q^{(e)}_n$

To conclude, first observe for any arbitrary MTT $M$ that if $d$ is the maximal
arity of any state of $Q$, then it is already a given that $\forall e \geq d,
\forall n \in \N, Q^{(e)}_n \subseteq Q^{(e)}_n$. Applying the previous result
in a induction yields d values $n_1, \dots, n_{d}$ such that $\forall e\geq
1, \forall n \geq \sum_i n_i, Q^{(e)}_n \subseteq Q^{(e)}_{\sum_i n_i}$.
Define $\sigma = \sum_i n_i$. We can partially apply our previous proof to
$M_\sigma$, yielding the existence of a $n_0 \in \N$ such that $R_{\sigma+n_0}$
contains no rule $\<q, \sigma(x_1, \dots, x_k)> \rightarrow \zeta \<p_1, \dots
p_k>$ such that $\exists q \in Q_\sigma, \exists i, \exists y, \<q, x_i> \in
\zeta$ and $(q,y) \in F_{p_i}$. Because by construction of $\sigma$,
$Q_{\sigma+n_0} \setminus Q_{\sigma} \subseteq Q_{\sigma +n_0}^{(0)}$ are two sets
containing only depth-proper states, then we finally conclude that $M_{\sigma
	+n_0}$ is depth-proper.






%\bibliography{bibli.bib}{}
%\bibliographystyle{plain}

%\end{document}


% the syntax for setting root file depends on your editor (first one is for Texstudio)
% !TeX root = ../main.tex
%%% Local Variables:
%%% mode: latex
%%% TeX-master: "../main"
%%% End:
