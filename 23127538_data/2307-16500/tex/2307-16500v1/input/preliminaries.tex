\section{Preliminaries}

The set $\{0,1,\dots \}$ of natural numbers is denoted by $\mathbb{N}$.
For $k\in\mathbb{N}$ we denote by $[k]$ the set
$\{1,\dots,k\}$; thus $[0]=\emptyset$.
A ranked alphabet (set) consists of an alphabet (set) $\Sigma$ together
with a mapping $\text{rank}_{\Sigma}: \Sigma\to\mathbb{N}$
that assigns each symbol $\sigma\in\Sigma$ a natural number called its ``rank''.
We will write $\sigma^{(k)}\in\Sigma$ to denote that $\sigma\in\Sigma$ and
$\text{rank}_{\Sigma}(\sigma)=k$.
By $\Sigma^{(k)}$ we denote the symbols of $\Sigma$ that have rank $k$.

The set $T_\Sigma$ of (finite, ranked, ordered) trees over $\Sigma$ is the smallest
set $S$ such that if $\sigma\in\Sigma^{(k)}$, $k\geq 0$, and
$s_1,\dots,s_k\in S$, then also $\sigma(s_1,\dots,s_k)\in S$.
We will write $\sigma$ instead of $\sigma()$.
For a tree $s=\sigma(s_1,\dots,s_k)$ with $\sigma\in\Sigma^{(k)}$, $k\geq 0$,
and $s_1,\dots,s_k\in T_\Sigma$, we define 
the set $V(s)\subseteq\mathbb{N}^*$ of nodes of $s$ as
$\{\varepsilon\}\cup \{ iu\mid i\in[k], u\in V(s_i)\}$.
Thus, $\varepsilon$ denotes the root node of $s$, and
for a node $u$, $ui$ denotes the $i$-th child of $u$.
For $u\in V(s)$ we denote by $s[u]$ the label of $u$ in $s$ and
by $s/u$ the subtree rooted at $u$.
Formally, let $s=\sigma(s_1,\dots,s_k)$ and define 
$s[\epsilon]=\sigma$,
$s[iu]=s_i[u]$, 
$s/\varepsilon=s$, and
$s/iu=s_i/u$ for $\sigma\in\Sigma^{(k)}$, $k\geq 0$,
$s_1,\dots,s_k\in T_\Sigma$, $i\geq 1$ and $u\in\mathbb{N}$ such that $iu\in V(s)$.

We fix two special sets of symbols: the set $X=\{x_1,x_2,\dots \}$ of
variables and the set $Y=\{y_1,y_2,\dots\}$ of parameters.
For $k\geq 1$ let $X_k=\{x_1,\dots,x_k\}$ and $Y_k=\{y_1,\dots,y_k\}$.
Let $A$ be a set that is disjoint from $\Sigma$. Then the set $T_\Sigma(A)$ of
trees over $\Sigma$ indexed by $A$ is defined as $T_{\Sigma'}$ where
$\Sigma'=\Sigma\cup A$ and $\text{rank}_{\Sigma'}(a)=0$ for $a\in A$ and 
$\text{rank}_{\Sigma'}(\sigma)=\text{rank}_{\Sigma}$ for $\sigma\in\Sigma$.

For a ranked alphabet $\Sigma$ and a set $A$ the ranked set
$\<\Sigma,A>$  consists of all symbols $\<\sigma,a>$ with $\sigma\in\Sigma$ and
$a\in A$; the rank of $\<\sigma,a>$ is defined as $\text{rank}_\Sigma(\sigma)$.

\subsection{Tree Substitution}

Let $\Sigma$ be a ranked alphabet and let $s,t\in T_\Sigma$.
For $u\in V(s)$ we define the tree $s[u\leftarrow t]$ that is obtained
from $s$ by replacing the subtree rooted at node $u$ by the tree $t$.
Let $\sigma_1,\dots,\sigma_n\in\Sigma^{(0)}$, $n\geq 1$ be pairwise distinct symbols
and let $t_1,\dots,t_n\in T_\Sigma$. Then $t[\sigma_i\leftarrow t_i\mid i\in[n]]$
is the tree obtained from $t$ by replacing each occurence of $\sigma_i$ by
the tree $t_i$. We have defined trees as particular stings, and this is just
ordinary string substitution (because we only replace symbols of rank zero).
We refer to this as ``first-order tree substitution''.

In ``second-order tree substitution'' it is possible to replace internal nodes $u$, i.e.,
subtrees of $s$ by new trees. These new trees use parameters to indicate where
the ``dangling'' subtrees $s/ui$ of the node $u$ are to be placed.
Let $\sigma_1\in\Sigma^{(k_1)},\dots, \sigma_n\in\Sigma^{(k_n)}$ be pairwise distinct symbols 
with $n\geq 1$ and
$k_1,\dots,k_n\in\mathbb{N}$ and let $t_i\in T_\Sigma[Y_{k_i}]$ for $i\in[n]$.
Let $s\in T_\Sigma$.
Then $s[\![\sigma_i\leftarrow t_i\mid i\in[n]]\!]$ denotes the tree
that is inductively defined as (abbreviating $[\![\sigma_i\leftarrow t_i\mid i\in[n]]\!]$ by
$[\![\dots ]\!]$) follows:
for $s=\sigma(s_1,\dots,s_k)$,
if $\sigma\not\in\{\sigma_1,\dots,\sigma_n\}$ then $s[\![\dots]\!]=\sigma(s_1[\![\dots]\!],
\dots,s_k[\![\dots]\!])$ and if $\sigma=\sigma_j$ for some $\nu\in[n]$ then
$s[\![\dots]\!]=t_\nu[y_j\leftarrow s_j[\![\dots]\!]\mid j\in[k]]$.

\subsection{Macro Tree Transducers}

A \emph{tree automaton} $A$ is given by a tuple
$(P,\Sigma,h)$ where $P$ is a finite set of states, $\Sigma$ is a ranked alphabet,
and $h$ is a collection of mappings $h_\sigma:P^k\to P$ where $\sigma\in\Sigma^{(k)}$
and $k\geq 0$. The extension of $h$ to a mapping $\hat{h}:T_\Sigma\to P$
is defined recursively as $h_\sigma(\hat{h}(s_1),\dots,\hat{h}(s_k))$ for every $\sigma\in\Sigma^{(k)}$,
$k\geq 0$, and $s_1,\dots,s_k$. For every $p\in P$ we define the set $L_p$ of trees
in $T_\Sigma$ as $\{ s\in T_\Sigma\mid \hat{h}(s)=p\}$.

A (total, deterministic) \emph{macro tree transducer with look-ahead} (``mttr'') $M$ is given by a tuple
$(Q,P,\Sigma,\Delta,q_0,R,h)$, where
\begin{itemize}
\item $Q$ is a ranked alphabet of \emph{states},
\item $\Sigma$ and $\Delta$ are ranked alphabet of \emph{input} and \emph{output symbols},
\item $(P,\Sigma,h)$ is a tree automaton (called the \emph{look-ahead automaton} of $M$),
\item $q_0\in Q^{(0)}$ is the \emph{initial state},
\item and $R$ is the \emph{set of rules}, where for each $q\in Q^{(m)}$, $m\geq 0$,
$\sigma\in\Sigma^{(k)}, k\geq 0$, and $p_1,\dots, p_k\in P$ there is exactly one rule of the form 
\[
\<q,\sigma(x_1:p_1,\dots,x_k:p_k)>(y_1,\dots,y_m) \to t
\]
with $t\in T_{\Delta\cup \< Q,X_k>}[Y_m]$.
The right-hand side $t$ of such a rule is denoted by
$\text{rhs}_M(q,\sigma,\<p_1, \dots,p_k>)$
\end{itemize}

The semantics of an mttr $M$ (as above) is defined as follows.
We define the derivation relation $\Rightarrow_M$ as follows.
For two trees $\xi_1,\xi_2\in T_{\Delta\cup\< Q,T_\Sigma>}(Y)$,
$\xi_1\Rightarrow\xi_2$ if there exists a node $u$ in $\xi_1$
with $\xi_1/u=\<q,s>(t_1,\dots,t_m)$, $q\in Q$, $s\in T_\Sigma$, and
$t_1,\dots,t_m\in T_{\Delta\cup\<Q,T_\Sigma>}$ and 
$\xi_2=\xi_1[u\leftarrow\zeta]$ where $\zeta$ equals
\[
\text{rhs}_M(q,\sigma,\<\hat{h}(s_1),\dots,\hat{h}(s_k)>)[\![\<q',x_i>\leftarrow \<q',s_i>\mid q'\in Q,i\in[k]]\!]
[y_j\leftarrow t_j\mid j\in[m]].
\]

Since $\Rightarrow_M$ is confluent and terminating, there is for every $\xi_1$ a unique
tree $\xi_2$ such that $\xi_1\Rightarrow_M^* \xi_2$.
For every $q\in Q^{(m)}$, $m\geq 0$ and $s\in T_\Sigma$ we define the
\emph{$q$-translation of $s$}, denoted by $M_q(s)$, as the unique tree $t$ in $T_\Delta(Y_m)$ such that
$\<q,s>(y_1,\dots, y_m)\Rightarrow_M^* t$.

We denote the translation realize by $M$ also by $M$, i.e., 
$M=M_{q_0}$ and for every $s\in T_\Sigma$,
$M(s)=M_{q_0}(s)$ is the unique tree $t\in T_\Delta$
such that $\< q_0,s>\Rightarrow_M^* t$.
