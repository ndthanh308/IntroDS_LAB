\section{Conclusions}

We have proven that for a given macro tree transducer (with look-ahead) it
is decidable whether or not it has linear height increase (LHI) and,
whether or not it has linear size-to-height increase (LSHI).
Both decision procedures rely on a novel normal form that is called
``depth-proper'' normal form. Roughly speaking the normal form requires
that each parameter of every state of the transducer appears at
arbitrary depth in output trees generate by that state (and for a given
look-ahead state). Our constrution
of the normal form removes parameters that only appear at a bounded number
of depths. Once in depth-normal form, we can reduced the check for LHI and LSHI
to the finiteness of ranges of mtts.

Both LHI and LSHI are natural properties and have several useful applications.
For instance, if an mtt is not of LHI, then we know that it cannot be realized
by a top-down or a bottom-up tree transducers. And, if an mtt is not of LSHI, then
we know that it cannot be realized by any attributed tree transducers.

The most prevailing open problem is to solved the conjecture (see Introduction) that
the translation of an mtt can be realized by an attributed tree transducer if and only
if the translation is of LSOI (linear size-to-number-of-distinct-output-subtrees).
In this paper we merely shows that \emph{deciding} if an mtt is of LSOI is as
hard as deciding equivalence of attributed transducers (a long standing difficult open problem).
We believe that the depth-proper normal form will be helpful in solving the conjecture.
Intuitively, loops need to be considered which produced arbitary number of copies of
states with parameters. In these loops we can exclude certain state nestings, due to
the normal form.

Another interesting open problem is the question whether or not the mtt hierarchy
(generated by sequential compositions of mtts) collapses for LSHI or for LHI.
By this we mean (in the case of LSHI), whether or not there exists some number $n$ such that
$\cup \{ \text{MTT}^k \mid k\geq 1 \} \cap \text{LSHI} \subseteq \text{MTT}^n$.
Note that for linear size increase, the hierarchy collapses~\cite{DBLP:journals/acta/EngelfrietIM21}
to level one, i.e., $n=1$.
Another question is, whether $(\text{MTT}\cap\text{LSHI})^{k+1}\supsetneq (\text{MTT}\cap\text{LSHI})^k$.
To see that this indeed holds, consider compositions of the translation that takes a binary
tree $s$ as input and outputs a full binary tree of height $|s|$.
The corresponding question for LHI is open. 
We do have a candidate translation that would show (for level two) both, that a two-fold composition of
mtts of LHI can do strictly more than a single mtt, and, that two-fold compositions
that are of LHI are strictly more than a single mtt:
input trees are monadic trees of the form $a^n(e)$ and output trees are full binary
trees of height $n$, at the leaves of which are monadic trees that represent the
the Dewey paths of those ``leaves'' of the binary tree.
E.g.
$a(a(e))$ is translated to
\[
f(f(1(1(e)), 1(2(e))), f(2(1(e)), 2(2(e)))).
\]
Note that if we output \emph{reverse} such Dewey paths, then the translation
indeed can be realized by a single mtt (see, e.g., Fig.~6.2 in~\cite{DBLP:series/eatcs/FulopV98}).
Thus, the second transducer of the composition reverses the reverse Dewey paths.
