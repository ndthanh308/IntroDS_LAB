
%%%%%%%%%%%%%%%%%%%% copied from template at https://cs.overleaf.com/latex/templates/modele-pour-theses-en-litterature-a-luniversite-de-montreal/jpzqpxmrpdwb
\usepackage[utf8]{inputenc}  % put latin1 instead of utf8 if you use latin1
\usepackage[T1]{fontenc}
\usepackage{setspace}
%\renewcommand{\baselinestretch}{2} % uncomment to add space between lines
%\usepackage[english]{babel}   %langue française et anglais. Si vous utilisez d'autres langues, ajoutez-les ici. Quand vous avez du texte en anglais, il faut ajouter la commande \selectlanguage{english}. Après, rappelez-vous de revenir au français avec \selectlanguage{french}
\usepackage[maxlevel=3]{csquotes}
%\usepackage[backend=biber]{biblatex} %,isbn=false,style=verbose-trad1
%\DefineBibliographyStrings{english}{in={dans},inseries={dans}}
\usepackage[cyr]{aeguill}
%\usepackage{geometry}
\usepackage{xspace}
%\geometry{centering, verbose, tmargin=30mm,bmargin=10mm,lmargin=35mm,rmargin=0mm, letterpaper }
\usepackage{graphicx}
\usepackage{epigraph}
\setlength\epigraphwidth{13cm}
%\usepackage[center,up,labelfont=bf]{caption}
\usepackage{float}
\usepackage{url}
%\usepackage{tcolorbox}
%\newcommand{\guil}[1]{«~{#1}~»}    %guillemets 
%\newcommand{\guill}[1]{``{#1}''}     %guillements dans les guillemets
%\addbibresource{biblio.bib}
%%%%%%%%%%%%%%%%%%%% end of part copied from template

%Packages
%\usepackage{logic9}
\usepackage{bbm}
\usepackage{tikz}
\usetikzlibrary{graphs,quotes,positioning}
%\usegdlibrary{trees}
%\setlength {\marginparwidth }{2cm} % ça résoud un problème avec todonotes
%\usepackage{todonotes}
\usepackage{amssymb}
\usepackage{cmll}


% from this point on: merged header from keisuke and Charles
\usepackage{amsmath,mathabx} %,amsthm
%\usepackage{cleveref}
\usepackage{stmaryrd}
\usepackage{amsfonts}
\usepackage{cite}
\usepackage{subcaption}
%\usepackage[dvipdfmx]{graphicx}

%%% tikz libraries:
\usetikzlibrary{shapes.geometric}
\usetikzlibrary{decorations.pathreplacing}


\newcommand{\N}{\mathbb{N}}
\newcommand{\wleq}{\preccurlyeq}
\newcommand{\wless}{\prec}
\DeclareMathOperator{\mdepth}{mdepth}

%\theoremstyle{plain}
\theoremstyle{definition}
\newtheorem{dfn}{Definition}[section]
\newtheorem{thm}[dfn]{Theorem}
\newtheorem{lem}[dfn]{Lemma}
\newtheorem{prp}[dfn]{Proposition}

%\newtheoremstyle{claim}{}{}{\rmfamily}{}{\itshape}{:}{ }{\underline{\thmname{#1}\thmnumber{ #2}\thmnote{ #3}}}
%\theoremstyle{claim}
%\newtheorem{clm}[dfn]{Claim}
\newtheorem{clm}{Claim}

\crefname{thm}{Theorem}{Theorems}
\crefname{lem}{Lemma}{Lemmata}

%\newcommand\Nat{\mathbb{N}}
\newcommand\Size[1]{\mathrm{size}(#1)}
\newcommand\Rank[1]{\mathrm{rank}(#1)}
\newcommand\Maxr[1]{\mathrm{maxrank}(#1)}
\newcommand\Root[1]{\mathrm{root}(#1)}
\newcommand\Sub[1]{\mathrm{sub}(#1)}

\DeclareSymbolFont{bbsymbol}{U}{bbold}{m}{n} 
\DeclareMathSymbol\comp{\mathbin}{bbsymbol}{"3B}
\newcommand\hole{\scalebox{0.7}{\(\square\)}}

\newcommand\ab{\allowbreak}

\newcommand\lcop[1]{\left\lfloor{#1}\right\rfloor}
\newcommand\paras[1]{\mathsf{ps}(#1)}
\newcommand\nop{\$}
\newcommand\pout{\text{pOut}}
\newcommand\rhsnop[1]{\left\lceil{#1}\right\rceil}
\newcommand\ot{\leftarrow}
\newcommand\eps{\varepsilon}
\newcommand\rhs{\mathsf{rhs}}
%\def\<#1>{\left\langle #1\right\rangle}
\newcommand\apa[1]{\left\langle #1\right\rangle}
\newcommand\stpair[1]{\left\langle\!\left\langle#1\right\rangle\!\right\rangle}
%\newcommand\sosubst[2]{\left[\!\left[#1\ot #2\right]\!\right]}
%\newcommand\sosubst[2]{[\![#1\ot #2]\!]}
%\newcommand\sosubst[2]{\left\llbracket #1\ot #2\right\rrbracket}
%\newcommand\sosubst[2]{{\renewcommand\mmid{\mid}\llbracket #1\ot #2\rrbracket}}
\newcommand\sosubst[2]{\llbracket #1\ot #2\rrbracket}
\newcommand\sosub[1]{\llbracket #1\rrbracket}
%\newcommand\subst[2]{\left[#1\ot #2\right]}
\newcommand\subst[2]{[#1\ot #2]}
%\newcommand\mmid{\;\middle|\;}
\newcommand\mmid{\mid}
\DeclareMathOperator{\rank}{rank}

\newcommand\MTTR{\text{MTT}^\text{R}}
\newcommand\LMTT{\text{LMTT}}
\newcommand\LSI{\text{LSI}}
\newcommand\LSOI{\text{LSOI}}
\newcommand\LBIS{\text{LBIS}}
\newcommand\MSOTT{\text{MSOTT}}
\newcommand\SC{\mathfrak{S\!C}}

%MACROS from Paul (end of header from Keisuke and Charles
%\newcommand{\paul}[1]{\todo[color=red!40,inline]{\textbf{P.} -- #1}} % note added by paul
\newcommand{\dom}{\mathrm{dom}} % domain of a function
\renewcommand{\max}{\mathrm{max}} % maximum
%\newtheorem{claim}{Claim}
%\newcommand{\N}{\mathbb{N}}
\renewcommand{\epsilon}{\varepsilon}
%\newcommand{\llb}{[\![} % opens semantics brackets
%\newcommand{\rrb}{]\!]} % closes semantics brackets
%\newcommand{\lb}{\llbracket} %opens double brackets
%\newcommand{\rb}{\rrbracket} %closes double brackets

\def \<#1>{{\langle {#1} \rangle}}


%% Redefine the \author command
%\makeatletter
%\renewcommand*\author[2][]{%
%	\def\@tempa{#1}%
%	\ifx\@tempa\@empty%
%	\def\author@arg{#2}%
%	\else%
%	\def\author@arg{#2\thanks{#1}}%
%	\fi%
%	\gdef\author@list{\author@arg}%
%}
%\makeatother






% At the end of a tex file, you have information on how to compile the file. Here we tell your computer which file should be compiled first. 
% In this case it is the file named "main.tex" in the parent directory, so "../main.tex" is the relative path from the directory of the current file to the file which should be compiled. 
% TeXstudio and Emacs/orgmod (Sylvain's setup) use different syntaxes to specify the main file, the TeX root line is for TeXstudio, the TeX master line is for Emacs:
% !TeX root = ../main.tex
%%% Local Variables:
%%% mode: latex
%%% TeX-master: "../main"
%%% End:
