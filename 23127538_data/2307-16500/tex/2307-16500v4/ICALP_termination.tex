%!TEX root = ICALP_main.tex

% macros (will be moved later)

%\section{Termination of the procedure}

We show that the iteration of the construction $\pi(M)$ will
terminate with a transducer that is depth proper.
%For an arbitrary transducer $M$ and a look-ahead state $p$ of $M$ we define
%the set $U_p$ of states $q$ such that $(q,p)$ us unreachable as follows''
%\[
%U_p = \{ q\in Q\mid (q,p)\text{ is unreachable in }M\}.
%\]
%Note that $M$ is depth proper if $F_p\subseteq U_p$ for every look-ahead state $p$ of $M$.
%
First, let us discuss what property a single iteration of $\pi$ ensures.
Let $p\in P$.
Note that the set $F_p$ is defined independently of reachability, i.e.,
$F_p$ may contain states $q$ such that $(q,p)$ is \emph{not} reachable.
Let $\varphi$ such that $(p,\varphi)\in P'$.
Then
%\[
$F_p\subseteq F_{(p,\varphi)}$.
%\]
This inclusion follows from Lemma~\ref{lm:corr} as follows:
let $s\in L_{(p,\varphi)}$ and let $q\in F_p$ be of rank $m$.
The latter means that there exists a $j\in[m]$ and a number $n$ 
such that every occurrence of $y_j$ in $M_q(s')$ is
at depth $\leq n$ for every $s'\in L_p$.
By Lemma~\ref{lm:corr}~(1), $s\in L_p$ and by Lemma~\ref{lm:corr}~(3),
$N_q(s)=M_q(s)$. Thus, every occurrence of $y_j$ in $N_q(s)$ also
occurs at depth $\leq n$. Hence, $q\in F_{(p,\varphi)}$.

We now consider reachability.
We say that a state $q$ is \emph{depth proper}, if for all $p\in P$ such that
$(q,p)$ is reachable, $q\not\in F_p$.
If $q\in F_p$, then for all $\varphi$ such that $(p,\varphi)\in P'$ it holds that
$(q,(p,\varphi))$ is not reachable.
This property follows immediately from the definition of look-ahead and the rules of $\pi(M)$:
the substitution $[\![.]\!]$ replaces each state call $\<q',x_i>$ with $q\in F_{p_i}$ by
a tree that does not contain states of $Q$.
So, if $(q,(p,\varphi))$ is reachable, then $q\not\in F_p$;
however, it may be that $q\in F_{(p,\varphi)}$, which means that $q$ is
not depth proper. It means that if $F_{(p,\varphi)}=F_p$ for
all $(p,\varphi)\in P'$, then all states $q\in Q$ are depth proper.
Let $Q_0=Q$ and consider now the iterated application of $\pi$.
Clearly, after some iterations of $\pi$, it will hold that
$F_{(p,\varphi)}=F_p$ for all $(p,\varphi)\in P'$.
To see this, consider the chain of inclusions
\[
F_p\cap Q_0 \subseteq F_{(p,\varphi_1)}\cap Q_0 \subseteq \dots \subseteq
   F_{(p,\varphi_1,\dots,\varphi_k)}\cap Q_0 \subseteq \dots
\]
for any maps \(\varphi_i\) introduced in the look-ahead of \(\pi^i(M)\).
Since \(Q_0\) is finite, the chain contains only finitely many strict inclusions.
Hence there is a minimal \(n\) such that 
\( F_{(p,\varphi_1,\dots,\varphi_n)}\cap Q_0 = F_{(p,\varphi_1,\dots,\varphi_{n'})} \cap Q_0 \)
for any \(n'>n\).

Consider a tree with an artificial root node which contains all such chains, i.e.,
for each $p\in P$ there is exactly one child of the root node labeled $F_p$, and 
a node labeled $F_p$ has children labeled $F_{(p,\varphi)}$ for each $(p,\varphi)\in P'$, etc. 
Moreover, a node labeled $F_{(p,\varphi_1,\dots,\varphi_n)}$ as in the chain above
is a leaf of this tree.
Since each node of this tree is finitely branching (because $P'$ is finite) and each
path has finite length, we know by K{\"o}nig's lemma that tree is finite.
Thus, if $d$ is the depth of this tree, for the 
mttr $M'=\pi^d(M)$, all states in $Q_0$ are depth proper.

Let $m$ be the maximal rank of the states in $Q_0$.
Since all helper states are of rank $<m$, we know that
$M'$ contains no improper states of rank $\geq m$.
We now proceed in the same fashion and construct a transducer $M''=\pi^{n'}(M')$
which contains no improper states of rank $\geq (m-1)$.
In a similar way we 
eventually obtain an mttr for which \emph{all states} are depth proper
(and which is equivalent to $M$).
Thus, even though we do not constructively derive a precise bound, we know that
after \emph{some} number of applications of $\pi$ we are sure to obtain
a depth proper mttr.

%\begin{lemma}\label{lm:term}
%Let \(M\) be a nondeleting mttr with improper states of rank at most \(m\).
%There exists \(N\) such that, for all $n \geq N$, 
%any reachable improper state call of \(\pi^n(M)\) has rank at most \((m-1)\).
%old version: (I only added the ", for all $n \geq N$, " and " call ")
%Let \(M\) be a nondeleting mttr with improper states of rank at most \(m\).
%There exists \(n\) such that 
%any improper state of \(\pi^n(M)\) has rank at most \((m-1)\).
%\end{lemma}
%
%\begin{proof}
%From the construction, 
%every look-ahead of \(\pi^k(M)\) has the form: \((p,\varphi_1,\dots,\varphi_k)\)
%with a look-ahead state \(p\) of \(M\).
%Let \(Q_0\) be the set of states of \(M\). 
%BLAH
%Note that the construction $\pi$ may introduce new states, called 
%\emph{helper states}, but those states have rank at most $(m-1)$. 
%So we only need to prove that no state in $Q_0$ is called improperly after enough 
%iterations of $\pi$, i.e.\ there exists $N \in \N$ such that for all $n \geq N$ and 
%look-ahead state $(p,\varphi_1, \dots, \varphi_n)$ of $\pi^n(M)$: $Q_0 \cap F_{(p, \varphi_1, \dots, \varphi_{n})} = Q_0 \cap U_{(p, \varphi_1, \dots, \varphi_{n})}$. 
%
%For each $p$, we consider the unranked tree whose nodes of depth $n$ are the 
%look-ahead states of $\pi^n(M)$ of the form $(p,\varphi_1, \dots, \varphi_n)$, 
%rooted in look-ahead state $p$, and such that the parent node of a node 
%$(p,\varphi_1, \dots, \varphi_n)$ is $(p,\varphi_1, \dots, \varphi_{n-1})$. 
%This tree has no node of infinite rank. 
%From this tree we remove all branches without look-ahead states for which there 
%exists a reachable improper state in $Q_0$ (i.e.\ such that 
%$Q_0 \cap F_{(p,\varphi_1, \dots, \varphi_n)} \setminus U_{(p,\varphi_1, \dots, \varphi_n)} \neq \emptyset$). 
%$\pi$ can introduce helper states of rank at most $(m-1)$, so we can conclude 
%by using K{\"o}nig's Lemma on this tree. We now only need to prove that the tree 
%contains no infinite branch. 
%
%BLAH
%
%
%\qed
%\end{proof}

Before we state the main theorem of this section, we need the following
lemma.
%(the proof is a straightforward reduction to Proposition~\ref{prop:finite} and
%can be found in the Appendix).

\begin{lemma}\label{lm:dec}
Let $M=(Q,P,\Sigma,\Delta,q_0,R,h)$ be an mttr and let
$q\in Q^{(m)}$, $m\geq 1$, $j\in[m]$, and $p\in P$.
It is decidable whether or not 
$\lcop{M_q(L_p)}_{\{y_j\}}$ is finite.
In case of finiteness, 
$\lcop{M_q(L_p)}_{\{y_j\}}$ can be constructed.
\end{lemma}

Since for a pair $(q,p)$ it is decidable whether or not it is reachable
(see Section~\ref{sect:mtt}), Lemma~\ref{lm:dec} implies that
it is decidable whether or not a given mttr is depth proper.

\begin{theorem}
  \label{th:proper}
For every mttr \(M\), we can construct an equivalent mttr $M'$ such that
$M'$ is depth proper.
\end{theorem}
%
\begin{proof}
There is a nondeleting mttr \(M_0\) equivalent to \(M\) 
(\cite{DBLP:journals/iandc/EngelfrietM99} or Proposition~\ref{prop:nondeleting}).
We repeatedly construct equivalent transducers $\pi(M)$, $\pi(\pi(M))$, etc.
until a proper mttr is obtained (which is decidable by
Lemma~\ref{lm:dec}). The repetition terminates 
(first eliminating all reachable calls of improper states of the highest rank $m$, then
those or rank $m-1$, etc.).
%\qed
\end{proof}

