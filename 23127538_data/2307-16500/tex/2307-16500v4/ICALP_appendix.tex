%!TEX root = ICALP_main.tex
\section*{Appendix}

\subsection*{Proof of Lemma~\ref{lm:nd}}

Statement of the lemma:
Let $\Delta$ be a ranked alphabet, $m\geq 1$, $Z\subseteq Y_m$, and
$t\in T_\Delta(Y_m)$.
(1)~$t=\lcop{t}_Z[u\leftarrow t/u\mid u\in\mathcal{U}(\lcop{t}_Z)]$.
(2)~$\paras{\lcop{t}_Z} = \paras{t}$. 

\smallskip

\begin{proof}
The proof is by induction on the structure of $\lcop{t}_Z$.
Let us denote the substitution $[u\leftarrow t/u\mid u\in\mathcal{U}(\lcop{t}_Z)]$
by $[t]$.
%
There are three cases.

Case 1: $t = y\in Z$. Then $\mathcal{U}(\lcop{t}_Z)=\emptyset$.
Hence $\lcop{t}_Z[t]=\lcop{y}_Z$. The latter equals $y=t$ by 
the definition of $\lcop{.}_Z$. Thus~(1)~holds. Also 
$\paras{\lcop{t}_Z}=\{y\}=\paras{t}$ and so~(2)~holds. 
%the definition of $\lcop{.}_Z$. Thus~(1)~holds and~(2)~holds for $u=\epsilon$.

Case 2: $\paras{t} \cap Z = \emptyset$. Then $\lcop{t}_Z = \paras{t} = Z' \subseteq Y_m$ 
by the definition of $\lcop{.}_Z$, and $\mathcal{U}(\lcop{t}_Z)=\{ \epsilon \}$. Hence
$\lcop{t}_Z[t] = \lcop{t}_Z[\epsilon\leftarrow t/\epsilon = t] = t$ which proves~(1). 
Again~(2)~holds 
because $\paras{\lcop{t}_Z}=\paras{\paras{t}}=\paras{t}$. 
%for $u=\varepsilon$ for each $y \in \paras{t}$ i.e.\ $y$ occurring in $t$.

Case 3: \(\paras{t}\cap Z\ne\emptyset\). Then $t=\delta(t_1,\dots,t_n)$, 
$\delta\in\Delta^{(n)}$, $n\geq 0$,
and $t_1,\dots, t_n\in T_\Delta(Y_m)$.
To show~(1) we obtain from the definition of $\lcop{t}_Z$ that
\[
\lcop{t}_Z[t]=\delta(\lcop{t_1}_Z,\dots,\lcop{t_n}_Z)[t]
= \delta(\lcop{t_1}_Z[t_1],\dots,\lcop{t_n}_Z[t_n]),
\]
where for $i\in[n]$, $[t_i]$ denotes the substitution
$[u\leftarrow t_i/u\mid u\in\mathcal{U}(\lcop{t_i}_Z)]$.
By induction the latter equals $\delta(t_1,\dots,t_n)=t$.
Finally~(2)~is implied by the induction hypothesis:
$\paras{\lcop{t}_Z}=\bigcup_{j\in[n]} \paras{\lcop{t_j}_Z}
=\bigcup_{j\in[n]} \paras{t_j}= \paras{t}$
%To show~(2), assume $y\in Y_m$ occurs in $t$.
%Thus, $n\geq 1$ and there exists an $i\in[n]$
%such that $y$ occurs in $t_i$.
%By induction of Statement~(2), there exists a leaf $u$ of $t_i$ such that 
%Statement~(2) holds (for $u$ and $t_i$); therefore Statement~(2) holds
%for $iu$ and $t$.
%\qed
\end{proof}

\subsection*{Proof of Lemma~\ref{lm:lcop_finite}}

Statement of the lemma:
Let $M$ be an mtt, $q$ a state of $M$, and $p$ a look-ahead state of $M$
such that $Y(q,p)\not=\emptyset$.
Then $\lcop{M_q(L_p)}_{Y(q,p)}$ is finite.

\smallskip

\begin{proof}
Assume that $Y_{q,p}=\{y_{j_1},\dots,y_{j_n}\}$ where $n\geq 1$.
It follows from the definition of $Y(q,p)$ that for each $i\in[n]$ there exists
a number $d_i$ such that $y_{j_i}$ occurs at depth $\leq d_i$ in 
any tree in $M_q(L_p)$. Let $d$ be the maximum of all numbers in
$\{ d_1,\dots,d_n\}$. Then every parameter in $Y(q,p)$ occurs at depth $\leq d$ in
any tree in $M_q(L_p)$. It follows from the definition of $U=\lcop{M_q(L_p)}_{Y(q,p)}$
that every node of a tree in $U$ has depth $\leq d$. Thus, $U$ is finite.
%\qed
\end{proof}


\subsection*{Example of Constructing a Depth Proper MTTR}

%\newcommand\qid{q_{\mathrm{id}}}
Let \(M=(\{q_0,q_1,q_2,q_3,q_4,\qid\},\{p\},\Sigma,$ $\Delta,q_0,R,h_0)\)
with \(\Sigma=\{a^{(1)},e^{(0)}\}\) and 
%\(\Delta=\{a_1^{(1)},a_2^{(2)},a_3^{(2)},a_4^{(1)},f^{(2)},g^{(3)},e^{(0)}\}\)
\(\Delta=\{a_1^{(1)},a_2^{(1)},\ab f_1^{(2)},f_2^{(2)},f_3^{(2)},g^{(3)},e^{(0)}\}\)
be an mttr 
where \((\Sigma,\{p\},h_0)\) is the trivial look-ahead with
$L_p=T_\Sigma$,
Suppose that \(R\) contains the following rules for \(a\): 
\begin{align*}
\<q_0, a(x)> &\to \<q_1,x>(e)
\\
\<q_1, a(x)>(y) &\to f_1(y, \<q_2,x>(a_1(y)))
\\
\<q_2, a(x)>(y) &\to g(y, \<q_3,x>(f_2(y, \<\qid, x>)), \<\qid, x>)
\\
\<q_3, a(x)>(y) &\to f_1(y, \<q_4,x>(f_3(y, \<\qid, x>)))
\\
\<q_4, a(x)>(y) &\to a_2(y)
\end{align*}
Then
because of \(L_p=\tree\Sigma\)
we have \(q_i\in F_p\) and \(Y(q_i,p)=\{y\}\) for any \(i=1,2,3,4\), and
the corresponding sets of \(\{y\}\)-skeleta are
\begin{align*}
\lcop{M_{q_1}(L_p)}_{\{y\}} &= 
\{f_1(y, g(a_1(y), f_1(f_2(a_1(y), \emp), a_2(f_3(f_2(a_1(y), \emp), \emp))), \emp))\}
\\
\lcop{M_{q_2}(L_p)}_{\{y\}} &= 
\{g(y, f_1(f_2(y, \emp), a_2(f_3(f_2(y, \emp), \emp))), \emp)\}
\\
\lcop{M_{q_3}(L_p)}_{\{y\}} &= \{f_1(y, a_2(f_3(y, \emp)))\}
\\
\lcop{M_{q_4}(L_p)}_{\{y\}} &= \{a_2(y)\}
\end{align*}
when the rules for \(e\) are appropriately given
so as for each of the sets above to be a singleton for simplicity.
Let \(t_i\) be the skeleta such that \(\lcop{M_{q_i}(L_p)}_{\{y\}} = \{t_i\}\).
When the skeleton contains no sequence node such as \(t_4\),
the \(q_4\)-calls in the right-hand side of every rule are easily removed
by substituting (in a second-order fashion) \(t_4\) for \(\<q_4,x>\).
For example,  the \(q_3\)-rule above will be 
\[ 
\<q_3, a(x)>(y) \to f_1(y, a_2(f_3(y, \<\qid, x>)))\text. 
\]
When the skeleton contains sequence nodes such as $\emptyset$,
our construction replaces them by \emph{helper states}; such states
traverse the input until further output needs to be generated.
A helper state has the form \([q,p,t,u]\) where
\(\<[q,p,t,u],s>\) for \(s\in L_p\) computes 
the subtree at node \(u\) of $M_q(s)$.
For example, the \(q_2\)-rule above which contains a \(q_3\)-call will be
\begin{align*}
&\<q_2, a(x)>(y) \to\\&\qquad
     g(y, 
       f_1(f_2(y, \<\qid, x>), a_2(f_3(f_2(y,\<\qid, x>), 
                                       \<[q_3,p,t_3,212],x>))),
       \<\qid, x>)
\end{align*}
by replacing the call by \(t_3\) whose sequence node is replaced by a helper state.
Since the sequence node \(t_3/212\) is $\emptyset$, the helper state has no parameter.
We will see later the case of skeleta containing non-empty sequence nodes.
Similarly the \(q_0\)-rule will be 
\begin{align*}
&\<q_0, a(x)>\to\\&\qquad %\<q_1,x>(e)
f_1(e, g(\begin{array}[t]{l}
         a_1(e), \\
         f_1(\begin{array}[t]{l}
             f_2(a_1(e), \<[q_1,p,t_1,2212],x>), \\
             a_2(f_3(f_2(a_1(e), \<[q_1,p,t_1,222112],x>), 
                     \<[q_1,p,t_1,22212,x>))), 
             \end{array} \\
         \<[q_1,p,t_1,23],x>))
         \end{array}
\end{align*}
%
A rule of a helper state \([q,p,t,u]\) is constructed from the \(q\)-rule
by first replacing each call of an \emph{improper} state
(at least one of whose parameter is improper) in the right-hand side
by the corresponding skeleton
and then taking the subtree at \(u\).
Thereby, from the obtained \(q_1\)-, \(q_2\)-, and \(q_3\)-rules above,
our construction gives
\begin{align*}
\<[q_1,p,t_1,2212],a(x)>   &\to \<[q_2,p,t_2,212],x>   \\
\<[q_1,p,t_1,222112],a(x)> &\to \<[q_2,p,t_2,22112],x> \\
\<[q_1,p,t_1,22212],a(x)>  &\to \<[q_2,p,t_2,2212],x>  \\
\<[q_1,p,t_1,23],a(x)>     &\to \<[q_2,p,t_2,3],x>     \\
\<[q_2,p,t_2,212],a(x)>    &\to \<\qid,x>              \\
\<[q_2,p,t_2,22112],a(x)>  &\to \<\qid,x>              \\
\<[q_2,p,t_2,2212],a(x)>   &\to \<[q_3,p,t_3,212],x>   \\
\<[q_2,p,t_2,3],a(x)>      &\to \<\qid,x>              \\
\<[q_3,p,t_3,212],a(x)>    &\to \<\qid,x> \text.
\end{align*}
For example, 
the right-hand side of the first rule of \(\<[q_1,p,t_1,2212],a(x)>\)
is constructed from the right-hand side of the rule of \(\<q_1,a(x)>\)
after substituted as shown above
by extracting its subtree at \(2212\), that is, \(\<[q_2,p,t_2,212],x>\).

Consider an input tree $s=a^5(s')$ with $s'\in T_\Sigma$
consisting of at least five top-most $a$-symbols
and apply the new $q_0$-rule shown above.
All the helper states in the right-hand side of that rule are now traversing
the node $1$ of $s$.
Now consider the derivation of the helper state $[q_1,p,t_1,22212]$:
it becomes $[q_2,p,t_2,2212]$ on node $11$ of $s$.
Then it become $[q_3,p,t_3,212]$ on node $111$ of $s$.
And finally it becomes $q_{\text{id}}$ on node $1111$ of $s$.
This is exactly right and corresponds to the computation of the original transducer.
All other helper states correctly become $q_{\text{id}}$ on node $111$ of $s$.


\subsection*{Proof of Lemma~\ref{lm:corr}}

Before we repeat the statement of Lemma~\ref{lm:corr} and present
its proof we need a small lemma showing that
the skeleton of the output of an mttr \(M\) can be directly computed
from given a input tree 
by modifying the rules of \(M\).
%given a 
%right-hand-side of rule $t$, the skeleton of $t$ can be computed from the 
%skeleta of state calls appearing in $t$. 
For this lemma we first need to define how to compute second-order 
substitutions of skeleta,
which will be used for the modification of the right-hand sides of rules.
We do so on a \emph{nondeleting} mttr $M$, 
i.e.\ such that states always use all their parameters. 

\begin{definition}\label{def:meta-skeleta}
%Let $M$ be a nondeleting mttr as before. 
  %
Let $\Gamma$ be a ranked alphabet
 and
let $t_1,\dots,t_n\in T_\Gamma(Y)$. 
Let $s\in T_\Gamma(Y_n\cup\mathcal{P}(Y_n))$. 
%
%\begin{itemize}
%\item
%\item Let \(k_i = 0\) for all \(i\in[n]\).
The \emph{special first-order substitution} 
$[y_i \leftarrow t_i \mid i \in [n]]\su$ (for short $[.]\su$) applied to $s$ is
inductively defined as:
\begin{align*}
	s[.]\su &= 
	\begin{cases}
		t_i &
%		\text{if \(s=\gamma_i\) for \(i\in [n]\)}
		\text{if \(s=y_i\) for \(i\in [n]\)}
		\\
		\gamma(s_1[.]\su,\dots,s_k[.]\su) &
%		\text{if \(s = \gamma(s_1,\dots,s_k)\) and }\gamma\not\in\{\gamma_1,\dots,\gamma_n\}
		\text{if \(s = \gamma(s_1,\dots,s_k)\)}
		\\
		\bigcup_{i\in U} \paras{t_i} & \text{if \(s= \{y_i \mid i\in U\} \subseteq Y_n\)
                  for some \(U \subseteq [n]\).}
%		\bigcup_{y_i \in t'} \paras{t_i} & \text{if \(t' \subseteq Y_{m'}\) is a sequence node}.
	\end{cases}
\end{align*}

%For all pairwise distinct symbols $\sigma_1\in\Delta^{(k_1)},\dots, 
%\sigma_n\in\Delta^{(k_n)}$ with $n\geq 1$ and
%$k_1,\dots,k_n\in\mathbb{N}$ and let $t_i$ for $i\in[n]$.
%Let $s\in T_\Delta$.
%Then $s[\![\sigma_i\leftarrow t_i\mid i\in[n]]\!]$ denotes the tree
%that is inductively defined as (abbreviating $[\![\sigma_i\leftarrow t_i\mid i\in[n]]\!]$ by
%$[\![\dots ]\!]$) follows:
%for $s=\sigma(s_1,\dots,s_k)$,
%if $\sigma\not\in\{\sigma_1,\dots,\sigma_n\}$ then $s[\![\dots]\!]=\sigma(s_1[\![\dots]\!],
%\dots,s_k[\![\dots]\!])$ and if $\sigma=\sigma_j$ for some $j\in[n]$ then
%$s[\![\dots]\!]=t_j[y_i\leftarrow s_i[\![\dots]\!]\mid i\in[k]]$.
%\item
Let $\gamma_1^{(k_1)},\dots,\gamma_n^{(k_n)}\in\Gamma$, $n\geq 1$ be pairwise different symbols
%Let $k_i=rank_\Gamma(\gamma_i)$ for $i\in[n]$
and assume now that 
$t_i\in T_\Gamma(Y_{k_i}\cup\mathcal{P}(Y_{k_i}))$ for $i\in[n]$
and that $s\in T_\Gamma(Y_n)$. 
%
The \emph{special second-order substitution} 
$[\![\gamma_i \leftarrow t_i\mid i\in[n]]\!]\sp{}$
(for short $[\![.]\!]\su$) applied to $s$ is
%inductively defined as 
%if $s=\gamma(s_1, \dots, s_k)$
%with $\gamma\not\in\{\gamma_1,\dots,\gamma_n\}$
%then $s[\![.]\!]\su = 
%\gamma(s_1[\![.]\!]\su, \dots, s_k[\![.]\!]\su)$, 
%if $s=y_j$ for $j\in [n]$ then $s[\![.]\!]\su = s$, 
%and if $s=\gamma_i$ for $i\in[n]$ then
%$s[\![.]\!]\su =  t_i
%[y_j \leftarrow t_j[\![.]\!]\su \mid j \in [n]]\su$. 
inductively defined as:
\begin{align*}
s[\![.]\!]\su &=
\begin{cases}
t_i
[y_j \leftarrow s_j[\![.]\!]\su \mid j \in [k_i]]\su
&
\text{if $s=\gamma_i(s_1, \dots, s_{k_i})$ for $i\in[n]$}
\\
\gamma(s_1[\![.]\!]\su, \dots, s_k[\![.]\!]\su)
& \text{if $s=\gamma(s_1, \dots, s_k)$
with $\gamma\not\in\{\gamma_1,\dots,\gamma_n\}$}
\\
s
&
\text{if $s=y_j$ for $j\in [n]$.}
\end{cases}
\end{align*}
%if $s=\gamma(s_1, \dots, s_k)$
%with $\gamma\not\in\{\gamma_1,\dots,\gamma_n\}$
%then $s[\![.]\!]\su = 
%\gamma(s_1[\![.]\!]\su, \dots, s_k[\![.]\!]\su)$, 
%if $s=y_j$ for $j\in [n]$ then $s[\![.]\!]\su = s$, 
%and if $s=\gamma_i$ for $i\in[n]$ then
%$s[\![.]\!]\su =  t_i
%[y_j \leftarrow t_j[\![.]\!]\su \mid j \in [n]]\su$. 

For all sets $Z \subseteq Y_m$ such that no $Y$-node in 
$t[\![.]\!]\su$ intersects $Z$, we define the $Z$-skeleton 
$\lcop{t[\![.]\!]\su}_Z$ of $t[\![.]\!]\su$ inductively as 
before, with a special case for $Y$-nodes: for all $Y$-nodes $S$ we 
have $\lcop{S}_Z = S \subseteq Y_m \setminus Z$. 
%\end{itemize}
\end{definition}

%Old version of the end of the definition of compatibility of rhs
%To define this notion of compatibility, we consider the second-order 
%substitution of a state call (in the right-hand-side of a rule) with the 
%state's skeleton, which induces a 
%first-order substitution of the parameters in the skeleton. We take such 
%first-order substitutions to apply \emph{within} sequence nodes of the skeleton 
%so that, given a sequence node $S=\{y_1,y_3\}$: 
%$S[y_i \leftarrow t_i]_{i\in [3]} = \{t_1,t_3\}$. 
%We assume given an mttr $M$ as before, with the condition that 
%$M$ is \emph{nondeleting}, which 
%means that all parameters of a state are used to build the state's output.
%
%\begin{definition}\label{def:meta-skeleta}
%For all states $q \in Q^{(m)}, q'\in Q^{(m')}$, $\sigma\in\Sigma^{(k)}$, 
%and $p_1,\dots,p_k\in P$, and noting $p=h(\sigma(p_1, \dots, p_k))$ and  
%$t=\text{rhs}_M(q,\sigma,\<p_1,\dots, p_k>)$, any state call $\<q',x_i>$ 
%in $t$ is \emph{compatible} with a set 
%$Z \subseteq Y_m$ if, for all $s_i \in L_{p_i}$, the sequence nodes in 
%$t[\![\<q',x_i>\leftarrow \lcop{M_{q'}(s_i)}_{Y(q',p_i)}]\!]$ contain no 
%parameters from $Z$. 
%\end{definition}

The special first-order substitution is
the same as the normal one
except that it gives special treatment to \(Y\)-nodes
which is replaced by \(Y\)-nodes contains all parameters occurring 
in trees to be substituted for the parameters in the original \(Y\)-nodes.
%
The special second-order substitution is
the same as the normal one
except that the special first-order substitution is applied
for each involved first-order substitution.


\begin{lemma}\label{lm:rhs}
Let $M$ be a nondeleting mttr as before.
Let $q\in Q$, $\sigma\in\Sigma^{(k)}$, and
$p_1,\dots,p_k\in P$. Let $p=h(\sigma(p_1, \dots, p_k))$ and 
$t=\text{rhs}_M(q,\sigma,\<p_1,\dots, p_k>)$. 
Let $s_1 \in L_{p_1}, \dots, s_k \in L_{p_k}$. By $[\![.]\!]^{\$}$ we
denote the substitution $[\![\<q',x_i>\leftarrow \lcop{M_{q'}(s_i)}_{Y(q',p_i)} \mid q'\in Q, 
i \in [k]]\!]^{\$}$ and by $[\![M]\!]$ we denote 
$[\![\<q',x_i>\leftarrow M_{q'}(s_i) \mid q'\in Q, i \in [k]]\!]$.
\begin{enumerate}
\item[(1)] If $y\in Y(q,p)$ and $y$ occurs in $t$ in the $j$-th argument 
of a node $\<q',x_i>$ for $q' \in Q$ and $i\in [k]$, then $y_j \in Y(q',p_i)$. 
%If $y\in Y(q,p)$ and $y$ occurs in  $t_j$ ($j\in[m]$), then $y_j\in Y(q',p_i)$.
\item[(2)] No $Y$-node in $t[\![.]\!]\su$ intersects $Y(q,p)$. 
%Any state call $\<q',x_i>$ in $t$ is compatible with $Y(q,p)$. 
\item[(3)] $\lcop{t[\![.]\!]\su}_{Y(q,p)} = \lcop{t[\![M]\!]}_{Y(q,p)}$
%For all input trees $s_i\in L_{p_i}$, we have: ~~~~
%$\lcop{t[\![\dots]\!]}_{Y(q,p)} = \lcop{t[\![\lcop{\dots}]\!]}_{Y(q,p)}$ \\
%where $[\![\dots]\!]$ denotes $[\![\<q',x_i>\leftarrow M_{q'}(s_i)]\!]$ \\
%and $[\![\lcop{\dots}]\!]$ denotes $[\![\<q',x_i>\leftarrow \lcop{M_{q'}(s_i)}_{Y(q',p_i)} ]\!]$. 
\end{enumerate}
\end{lemma}
\begin{proof}
If some $y_j\notin Y(q',p_i)$ then $\lcop{M_{q'}(L_{p_i})}_{y_j}$ is 
infinite and, if $y$ occurs in $t_j$ ($j\in[m]$), then 
$\lcop{M_{q}(L_{p})}_{y}$ is also infinite and $y\notin Y(q,p)$. 
So~(1)~holds. 

%(2)~is a consequence of~(1).
%Alternative proof of (2):
If $y \in Y(q,p)$ occurs in a $Y$-node of $t[\![.]\!]\su$, 
then it occurs in $t$ in the $j$-th argument of a node $\<q',x_i>$ with 
$y_j \notin Y(q',p_i)$, which contradicts~(1). 
So~(2)~holds. 

%As a consequence of~(1), parameter nodes and inner nodes are identical in 
%$\lcop{t[\![\dots]\!]}_{Y(q,p)}$ and 
%$\lcop{t[\![\lcop{\dots}]\!]}_{Y(q,p)}$. 
%Sequence nodes are also identical as a consequence of Lemma~\ref{lm:nd}(2). 
%Therefore~(3)~holds. 
%Alternatice proof of (3)
%%% Original proof FROM HERE %%%%%%%%%%%%%%%%%%%%%%%%%%%%%%%%%
%Both $\lcop{t[\![M]\!]}_{Y(q,p)}$ and 
%$\lcop{t[\![.]\!]\su}_{Y(q,p)}$ contain three types of nodes: 
%parameter nodes of the form $y\in Y(q,p)$, inner nodes and $Y$-nodes. 
%As a consequence of~(1), paths to parameters nodes are identical in 
%$\lcop{t[\![M]\!]}_{Y(q,p)}$ and $\lcop{t[\![.]\!]\su}_{Y(q,p)}$, 
%and the same is true of the inner nodes along such paths. 
%$Y$-nodes are also identical as a consequence of Lemma~\ref{lm:nd}(2) and 
%of our definition of special second-order substitutions. 
%So~(3)~holds. 
%%% Original proof TO HERE %%%%%%%%%%%%%%%%%%%%%%%%%%%%%%%%%

The statement~(3) is proved by induction on \(t\).
The cases of \(t=y_j\) and \(t=\gamma(t_1,\dots,t_n)\) are easy.
In the case of \(t=\<q',x_i>(t_1,\dots,t_m)\), we have
\begin{align*}
\lcop{t[\![.]\!]\su}_{Y(q,p)}
&=
\lcop{
\lcop{M_{q'}(s_i)}_{Y(q',p_i)}[y_j\leftarrow t_j[\![.]\!]\su\mid j\in[m]]\su
}_{Y(q,p)}
\\&=
\lcop{
\lcop{M_{q'}(s_i)}_{Y(q',p_i)}[y_j\leftarrow t_j[\![M]\!]\mid j\in[m]]\su
}_{Y(q,p)}
\\&=
\lcop{
M_{q'}(s_i)[y_j\leftarrow t_j[\![M]\!]\mid j\in[m]]
}_{Y(q,p)}
\\&=
\lcop{t[\![M]\!]}_{Y(q,p)}\text.
\end{align*}
%where the induction hypothesis and Lemma~\ref{lm:nd}(2) are used.
%
%Another alternatice proof of (3)
%To prove~(3)~we look at the $Y(q,p)$-skeleta of $t[\![\dots]\!]$ and 
%$t[\![\lcop{\dots}]\!]$. Those contain three types of nodes: 
%parameter nodes of the form $y\in Y(q,p)$, inner nodes (which are along 
%the paths to parameter nodes), and sequence nodes. Paths to parameters nodes 
%are identical in $t[\![\dots]\!]$ as in $t[\![\lcop{\dots}]\!]$ and so 
%are the inner nodes along such paths. Sequence nodes are also identical as a 
%consequence of Lemma~\ref{lm:nd}(2). 
%\qed
\end{proof}


\noindent
Statement of the Lemma~\ref{lm:corr}:
%
Let $M$ be a nondeleting mttr and $N=\pi(M)$ be the mttr of
Definition~\ref{df:pi}, both with the tuples as in that definition. 
Let $s\in T_\Sigma$ with $\hat{h'}(s)=(p,\varphi)$.
Then
\begin{enumerate}
\item[(1)] $p=\hat{h}(s)$,
\item[(2)] $\forall q\in F_p$: $\varphi(q)=\lcop{M_q(s)}_{Y(q,p)}$, 
\item[(3)] $\forall q\in Q$: $N_q(s)=M_q(s)$,
\item[(4)] $\forall q\in F_p$ and $u\in V(t)$ with $t=\varphi(q)$ and
$t/u=\{y_{j_1},\dots,y_{j_n}\}$ with \\
$j_1<\cdots <j_n$:
$N_{[q,p,t,u]}(s)=M_q(s)/u[y_{j_\nu}\leftarrow y_\nu\mid\nu\in[n]]$, and
\item[(5)] the mttr $N$ is nondeleting.
\end{enumerate}

\begin{proof}
All the statements are proven by induction on the structure of $s$.
Let $s=\sigma(s_1,\dots,s_k)$ with $\sigma\in\Sigma^{(k)}$,
$k\geq 0$, and $s_1,\dots,s_k\in T_\Sigma$.
For $i\in[k]$ let $\hat{h'}(s_i)=(p_i,\varphi_i)$.
By the definition of $h'$, $p=h_\sigma(p_1,\dots,p_k)$, which 
is equal to $\hat{h}(s)$.
Thus, Statement~(1) holds.
For Statement~(2) let $q\in F_p$: 
Then $\varphi(q)$ is defined as 
$\lcop{\zeta[\![\varphi_i]\!]\su}_{Y(q,p)}$ where 
$\zeta=\text{rhs}_M(q,\sigma,\<p_1,\dots,p_k>)$
and
$[\![\varphi_i]\!]\su$ denotes the special substitution
$[\![ \< q',x_i>\leftarrow \varphi_i(q')\mid q'\in F_{p_i}, i\in[k]]\!]\su$.
By induction, $\lcop{\zeta[\![\varphi_i]\!]\su}_{Y(q,p)}$ equals
$\lcop{\zeta[\![ \< q',x_i>\leftarrow \lcop{M_{q'}(s_i)}_{Y(q',p_i)}
\mid q'\in F_{p_i}, i\in[k]]\!]\su}_{Y(q,p)}$.
By Lemma~\ref{lm:rhs}(3) the latter equals
$\lcop{\zeta[\![ \< q',x_i>\leftarrow M_{q'}(s_i)
\mid q'\in F_{p_i}, i\in[k]]\!]}_{Y(q,p)}
=\lcop{M(s)}_{Y(q,p)}$.

We now prove Statement~(3).
Let $q\in Q$.
Then $N_q(s)=\zeta[\![ . ]\!][\![N]\!]$, 
where 
$\zeta=\text{rhs}_M(q,\sigma,\ab\<p_1,\dots,p_k>)$,
$[\![ . ]\!]$ is the substitution as in the construction, and
$[\![N]\!]=
[\![ \<r,x_i>\leftarrow N_r(s_i)\mid r\in Q',i\in[k] ]\!]$.
%
By the induction hypothesis of Statement~(2), we can replace
$\varphi_i(q')$ by $\lcop{M_{q'}(s_i)}_{Y(q',p_i)}$ in the
substitution $[\![ . ]\!]$. This gives
\begin{multline*}
\zeta
[\![ \<q',x_i>\leftarrow \lcop{M_{q'}(s_i)}_{Y(q',p_i)}
[
u'\leftarrow [q',p_i,\varphi_i(q'),u'](y_{j_1},\dots,y_{j_n})\mid \\
\varphi_i(q')/u'=\{y_{j_1},\dots,y_{j_n}\},
j_1<\cdots <j_n]
\mid q'\in F_{p_i},
i\in[k] ]\!]
[\![ N ]\!].
\end{multline*}
This can be written as 
$\zeta[\![.]\!][\![ H ]\!] [\![ Q ]\!]$,
where
$[\![ H ]\!]=[\![\< q',x_i>\leftarrow N_{q'}(s_i)\mid q'\in H,i\in[k] ]\!]$
and
$[\![ Q ]\!]  = 
[\![\< q',x_i>\leftarrow N_{q'}(s_i)\mid q'\in (Q \setminus F_{p_i}),i\in[k] ]\!]$.
%
By induction of Statement~(4) the substitution $[\![ H]\!]$ replaces the 
subtree $[q',p_i,\varphi_i(q'),u'](y_{j_1},\dots,y_{j_n})$ by the tree 
$M_{q'}(s_i)/u[y_{j_\nu}\ab\leftarrow y_\nu\mid\nu\in[n]]
[y_\nu\leftarrow y_{j_\nu}\mid\nu\in[n]]=M_{q'}(s_i)/u$. 
Thus we obtain:
\begin{multline*}
\zeta
[\![ \<q',x_i>\leftarrow \lcop{M_{q'}(s_i)}_{Y(q',p_i)}
[
u'\leftarrow M_{q'}(s_i)/u'\mid 
u'\in\mathcal{U}(\lcop{M_{q'}(s_i)}_{Y(q',p_i)})]\\
\mid q'\in F_{p_i},
i\in[k] ]\!] 
[\![ Q ]\!]
\end{multline*}
By Lemma~\ref{lm:nd} (for $Z=Y(q',p_i)$ and $t=M_{q'}(s_i)$)
the tree on the right of the arrow
in the leftmost second-order substitution equals $M_{q'}(s_i)$.
We have:
\[
\zeta
[\![ \<q',x_i>\leftarrow M_{q'}(s_i)\mid q'\in F_{p_i},i\in[k] ]\!] 
[\![\< q',x_i>\leftarrow N_{q'}(s_i)\mid q'\in Q \setminus F_{p_i},i\in[k] ]\!].
\]
By induction of Statement~(3), $N_{q'}(s_i)=M_{q'}(s_i)$ for
$q\in Q \setminus F_{p_i}$. This gives us exactly $M_q(s)$, by the definition
of the semantics of mttrs. 
Thus,
\begin{equation}\label{eq:MN}
N_q(s)=\zeta[\![ . ]\!][\![N]\!]=M_q(s).
\end{equation}
This concludes the proof of Statement~(3).


We now prove Statement~(4).
Let $q\in F_p$ and $u\in V(t)$ with $t=\varphi(q)$ 
and $t/u\subseteq Y$.
By the definition of the rules for the helper states, 
$N_{[q,p,t,u]}(s)=(\zeta[\![ . ]\!])/u[\![ N ]\!][y]$
where $t/u=\{y_{j_1},\dots,y_{j_n}\}$, $j_1<\cdots <j_n$, and 
$[y]=[y_{j_\nu}\leftarrow y_\nu\mid\nu\in[n]]$.
It follows from Lemma~\ref{lm:rhs}(1) that if $\<q',x_i>$ occurs in 
$\zeta=\text{rhs}_M(q,\sigma,\langle p_1,$ $\dots,p_k\rangle)$
and $q\in F_p$, then $q'\not\in Q \setminus F_{p_i}$.
Hence, every proper ancestor
$v$ of $u$ is labeled by a symbol in $\Delta$, i.e.,
$(\zeta[\![.]\!][y])[v]\in\Delta$. 
This implies that we can move the ``$/u$'' operation of 
taking the subtree at node $u$
to the right (after the application of the substitution $[\![ N ]\!]$)
in the above displayed formula.
We obtain 
$\zeta[\![ . ]\!][\![ N ]\!]/u[y]$.
By the right equation in Formula~\ref{eq:MN}, this equals
$M_q(s)/u[y]$.

To prove Statement~(5), let $q\in Q^{(m)}$, $m\geq 0$.
Then 
\[
\zeta'=\text{rhs}_N(q,\sigma,\< (p_1,\varphi_1),\dots,(p_k,\varphi_k)>)=
\zeta[\![.]\!],
\]
where 
$\zeta=\text{rhs}_M(q,\sigma,\<p_1,\dots,p_k>)$ and
$[\![.]\!]$ is as before. 
By Statement~(2), $[\![.]\!]$ substitutes occurrences of 
$\<q',x_i>$ with $i\in[k]$ and $q'\in F_{p_i}$ by 
the tree $\lcop{M_{q'}(s_i)}_{Y(q',p_i)}$ in which leaves
labeled by $Z\subseteq Y_m$ are replaced by $\<q_H,x_i>(y_{j_1},\dots,
y_{j_n})$ with $Z=\{y_{j_1},\dots,y_{j_n}\}$.
By Lemma~\ref{lm:nd}(2) this implies that 
$y_j$ occurs in $\zeta'$ for each $j\in[m]$.
%\qed
\end{proof}

\iffalse
\subsection*{Proof of Lemma~\ref{lm:dec}}

Statement of the lemma:
Let $M=(Q,P,\Sigma,\Delta,q_0,R,h)$ be an mttr and let
$q\in Q^{(m)}$, $m\geq 1$, $j\in[m]$, and $p\in P$.
It is decidable whether or not 
$\lcop{M_q(L_p)}_{\{y_j\}}$ is finite.
In case of finiteness, 
$\lcop{M_q(L_p)}_{\{y_j\}}$ can be constructed.

\smallskip

\begin{proof}
Let $Z=\{y_j\}$.
We now consider symbols in $Y_m$ as rank zero symbols.
It is straightforward to construct a  top-down tree transducer with look-ahead $M_Z$ which outputs
$\lcop{t}_Z$ for input trees $t\in T_{\Delta\cup Y_m}$; note that
$T_{\Delta\cup Y_m}=T_\Delta(Y_m)$ because symbols in $Y_m$ are now considered
as symbols of rank zero.
The transducer $M_Z$ computes in its look-ahead $h'$ the set of parameters of
the input tree, i.e.\ $h'(t)=\paras{t}$ for every $t\in T_{\Delta\cup Y_m}$. 
The transducer $M_Z$ (which consists of a single state only) outputs $\paras{t}$
as soon as $\paras{t}\cap Z=\emptyset$.
%The details can be found in the Appendix.

Formally, $M_Z = (\{q_1^{(0)}\},P',\Delta\cup Y_m,\Delta',q_1,R',h')$ where
$P'={\cal P}(Y_m)$ and
$\Delta'=\Delta\cup \{ S^{(0)}\mid S\in P'\}\cup Y_m$.
For $y\in Y_m$ let $h'_y()=\{y\}$ and for
$a\in\Delta^{(0)}$ let $h'_a()=\emptyset$.
Further, for $\delta\in\Delta^{(k)}$, $k\geq 1$, and 
$S_1,\dots,S_k\in P'$ let 
$h'_\delta(S_1,\dots,S_k)=\bigcup_{i\in[k]}S_i$.
For $a\in\Delta^{(0)}$ let the rule
$\< q_1,a>\to\emptyset$ be in $R$.
Let $y\in Y_m$.
If $y\in Z$ then let the rule
$\< q_1,y>\to y$ be in $R$ and otherwise let the rule
$\< q_1,y>\to \{y\}$ be in $R$.
For $\delta\in\Delta^{(k)}$ with $k\geq 1$ 
and $S_1,\dots,S_k\in P'$
we define the rule
$\< q_1,\delta(x_1:S_1,\dots,x_k:S_k)>\to\zeta$ where $\zeta$ is defined as:
\[
\begin{array}{lcll}
  \zeta & = &
\left\{ 
  \begin{array}{ll}
  S & \text{if }S= (\bigcup_{i\in[k]} S_i)\cap Z=\emptyset\\
  \delta(\< q_1,x_1>,\dots,\<q_1,x_k>) & \text{otherwise.}
  \end{array}
\right. 
\end{array}
\]

\noindent
\textbf{Claim.}\quad
For every $t\in T_{\Delta\cup Y_m}$, $h'(t)=\paras{t}$ and
$M_Z(t)=\lcop{t}_Z$.

\medskip

It is straightforward to prove this claim by induction on the structure of $t$.
By the Claim,
$M_{\{y_j\}}(M_q(L_p)) = \lcop{M_q(L_p)}_{\{y_j\}}$.
The tree language $L_p$ can be represented by a partial nondeterministic
top-down tree transducer that realizes the identity on trees in $L_p$
(and is undefined otherwise; its rules are obtained by reading the
definitions of $h$ from right to left). 
In this way $M_{\{y_j\}}(M_q(L_p))$ is represented
so that its finiteness is decidable by Proposition~\ref{prop:finite}
(and in case of finiteness the set can be constructed).
%\qed
\end{proof}
\fi

\subsection*{Proof of Lemma~\ref{lm:dec}}

Statement of the lemma:
Let $M=(Q,P,\Sigma,\Delta,q_0,R,h)$ be an mttr and let
$q\in Q^{(m)}$, $m\geq 1$, $j\in[m]$, and $p\in P$.
It is decidable whether or not 
$\lcop{M_q(L_p)}_{\{y_j\}}$ is finite.
In case of finiteness, 
$\lcop{M_q(L_p)}_{\{y_j\}}$ can be constructed.

%
\begin{proof}
Let $Z=\{y_j\}$.
We now consider symbols in $Y_m$ as rank zero symbols.
It is straightforward to construct a  top-down tree transducer with look-ahead $M_Z$ which outputs
$\lcop{t}_Z$ for input trees $t\in T_{\Delta\cup Y_m}$; note that
$T_{\Delta\cup Y_m}=T_\Delta(Y_m)$ because symbols in $Y_m$ are now considered
as symbols of rank zero.
The transducer $M_Z$ computes in its look-ahead $h'$ the set of parameters of
the input tree, i.e.\ $h'(t)=\paras{t}$ for every $t\in T_{\Delta\cup Y_m}$. 
The transducer $M_Z$ (which consists of a single state only) outputs $\paras{t}$
as soon as $\paras{t}\cap Z=\emptyset$.
%The details can be found in the Appendix.

Formally, $M_Z = (\{q_1^{(0)}\},P',\Delta\cup Y_m,\Delta',q_1,R',h')$ where
$P'={\cal P}(Y_m)$ and
$\Delta'=\Delta\cup \{ S^{(0)}\mid S\in P'\}\cup Y_m$.
For $y\in Y_m$ let $h'_y()=\{y\}$ and for
$a\in\Delta^{(0)}$ let $h'_a()=\emptyset$.
Further, for $\delta\in\Delta^{(k)}$, $k\geq 1$, and 
$S_1,\dots,S_k\in P'$ let 
$h'_\delta(S_1,\dots,S_k)=\bigcup_{i\in[k]}S_i$.
For $a\in\Delta^{(0)}$ let the rule
$\< q_1,a>\to\emptyset$ be in $R$.
Let $y\in Y_m$.
If $y\in Z$ then let the rule
$\< q_1,y>\to y$ be in $R$ and otherwise let the rule
$\< q_1,y>\to \{y\}$ be in $R$.
For $\delta\in\Delta^{(k)}$ with $k\geq 1$ 
and $S_1,\dots,S_k\in P'$
we define the rule
$\< q_1,\delta(x_1:S_1,\dots,x_k:S_k)>\to\zeta$ where $\zeta$ is defined as:
\[
\begin{array}{lcll}
  \zeta & = &
\left\{ 
  \begin{array}{ll}
  S & \text{if }S= (\bigcup_{i\in[k]} S_i)\cap Z=\emptyset\\
  \delta(\< q_1,x_1>,\dots,\<q_1,x_k>) & \text{otherwise.}
  \end{array}
\right. 
\end{array}
\]

\noindent
\textbf{Claim.}\quad
For every $t\in T_{\Delta\cup Y_m}$, $h'(t)=\paras{t}$ and
$M_Z(t)=\lcop{t}_Z$.

\medskip

It is straightforward to prove this claim by induction on the structure of $t$.
By the Claim,
$M_{\{y_j\}}(M_q(L_p)) = \lcop{M_q(L_p)}_{\{y_j\}}$.
The tree language $L_p$ can be represented by a partial nondeterministic
top-down tree transducer that realizes the identity on trees in $L_p$
(and is undefined otherwise; its rules are obtained by reading the
definitions of $h$ from right to left). 
In this way $M_{\{y_j\}}(M_q(L_p))$ is represented
so that its finiteness is decidable by Proposition~\ref{prop:finite}
(and in case of finiteness the set can be constructed).
%\qed
\end{proof}



\subsection*{Proof of Lemma~\ref{lm:decidable}}

Statement of the lemma:
Let $M$ be an mttr.
Then 
(1)~it is decidable whether or not $M$ is finite nesting and
(2)~it is decidable whether or not $M$ is finite yield nesting.

\smallskip

\begin{proof}
Let $M=(Q,P,\Sigma,\Delta,q_0,R,h)$.
We use the extension 
$\widehat{M}=(\hat{Q},P,\hat{\Sigma},\hat{\Delta},q_0,$ $\hat{R},h)$
of $M$ with input trees in $s\in T_\Sigma(P)$ which contain
(1)~exactly one or
(2)~arbitrarily many occurrences of elements of $P$.
We then use a nondeterministic top-down tree transducer $N$ which chooses
any path in the tree $\widehat{M}(s)$ and outputs only the elements from
$\< Q,P>$ on that path, now seen as unary symbols.
The resulting output language $N(\widehat{M}(T_\Sigma))$ is finite if and only
if $M$ is (1)~fnest or (2)~fynest.

Formally, 
$N=(\{q_1^{(0)}\},\hat{\Delta},\Gamma,q_1,R')$ where
$\Gamma=\< Q,P>\cup\{e^{(0)}\}$.
For every $\delta\in\Delta^{(k)}$, $k\geq 1$, and $i\in k$
we let the rule
$\<q_1,\delta(x_1,\dots,x_k)>\to\<q_1,x_i>$ be in $R'$.
For every $\delta\in\Delta^{(0)}$ we let the rule
$\<q_1,\delta>\to e$ be in $R'$.
For every $\<q,p>\in \<Q,P>^{(m)}$, $m\geq 1$, and $i\in[m]$
we let the rule
$\<q_1,\langle q,p\rangle(x_1,\dots,x_m)>\to \<q,p>(\<q_1,x_i>)$ be in $R'$.
For every $\<q,p>\in\<Q,P>^{(0)}$ we let the rule
$\<q_1,\langle q,p\rangle>\to \<q,p>$ be in $R'$.
It is straightforward to show (by induction on the structure of $s$), that
$N(\widehat{M}(T_\Sigma(P)))$ is finite if and only if $M$ is fynest.
Let $L$ be the set of trees in $T_\Sigma(P)$ which contain exactly one occurrence
of an element of $P$. 
It is straightforward to show (by induction on the structure of $s$), that
$N(\widehat{M}(L))$ is finite if and only if $M$ is fnest.
%\qed
\end{proof}

\subsection*{Proof of Lemma~\ref{lm:easy}}

Statement of the lemma:
Let $M$ be an mttr.
(1)~If $M$ is finite nesting, then it is of linear size-to-height increase.
(2)~If $M$ is finite yield nesting, then it is of linear height increase. 

\begin{proof}
Informally, we can understand this lemma by looking at a given path $O$ in an output tree and, using origin semantics, at how many nodes along this path have their origin in different parts of the input tree. 

For~(1), the finite nesting property gives a bound $c$ on the number of state calls to a single input node, nested along path $O$. Intuitively, noting $\text{mhr}$ the maximum height of the right-hand side of a rule, $c.\text{mhr}$ is a bound on the number of output nodes along path $O$ with their origin in a single input node. This bound clearly implies that the height of the output (maximum number of nodes on a path) is linearly bound by the size of the input. 

For~(2), instead of looking at a single input node, we look at all the input nodes at a given depth $d$ in the input. The finite yield nesting property implies a bound $c$ on the nesting (along a path $O$) of state calls to input nodes of depth $d$. Each such call may produce at most $\text{mhr}$ nodes along path $O$ with their origin in a node of depth $d$. So $c.\text{mhr}$ is a bound for the number of nodes along path $O$ with their origin in a node of depth $d$. 
%Note that these bounds only work because we look at state calls along an output path, because states of MTTs can copy other state calls when they appear in their parameters, but cannot copy them vertically (i.e.\ two such copied state calls cannot appear nested along a same output path). 

Formally, we apply $\widehat{M}$ to a tree $t\in T_\Sigma(P)$. We modify $t$ by substituting nodes in $P$, and we bound the growth of the height of $\widehat{M}(t)$ for each substitution. We will conclude by stating that any input tree $s \in T_\Sigma$ can be built by successive substitutions, and so the height of the output is linearly bound by the number of substitutions (which will be the size of $s$ for~(1), and the height of $s$ for~(2)). Let $M=(Q,P,\Sigma,\Delta,q_0,R,h)$ and let $\text{mhr}$ be the maximum height of the right-hand side of any rule in $R$. Let $s$ be a fixed tree in $T_\Sigma$.

To prove~(1), consider $U$ an arbitrary set of pairwise independent (i.e.\ not being descendants of each other) nodes of a fixed input tree $s \in T_\Sigma$. Let $s'=s[u\leftarrow h(s/u)\mid u\in U]$, let $u \in U$, 
%$u\in V(s')$ with $s'/u\in P$, 
and $\sigma=s[u]\in\Sigma^{(k)}$ with $k\geq 0$. Let $c$ be a nesting bound for $M$, then, along any output path in $\widehat{M}(s')$, there are at most $c$ state calls $\<q,s'/u>$ with origin $u$ in $s'$. Then $\widehat{M}(s'[u\leftarrow\sigma(h(s/u1),\dots,h(s/uk))])$ is obtained by replacing such state calls with the corresponding right-hand side of rules, which implies that: $\he{\widehat{M}(s'[u\leftarrow\sigma(h(s/u1),\dots,h(s/uk))])} \leq c\cdot\text{mhr} + \he{\widehat{M}(s')}$. 
The tree $s \in T_\Sigma$ can be obtained from the tree $h(s) \in T_\Sigma(P)$ by $|s|$ such substitutions. 
The height of $\widehat{M}(h(s)) = \<q_0,h(s)>$ is $0$. So $\he{M(s)} \leq c\cdot\text{mhr} \cdot |s|$, so $M$ of linear size-to-height increase. 

To prove~(2), for $i\in[\he{s}]$, let $U_i$ be the set of nodes at depth $i$ in $s$, and consider $s_i$ be the tree obtained from $s$ by replacing all nodes $u \in U_i$ by $h(s/u)$. Let $c$ be a yield nesting bound for $M$, then, along any output path $O$ in $\widehat{M}(s_i)$, there are at most $c$ state calls $\<q,s_i/u>$ with $u \in U_i$. 
%So we can substitute all the state calls on input subtrees of depth $i$ at the same time, and increase the height of the output by less than $c\cdot \text{mhr}$.
Then $\widehat{M}(s_{i+1}) = \widehat{M}(s_i[u\leftarrow \sigma_u(h(s/u1),\dots,h(s/uk)) \mid u \in U_i])$ is obtained by replacing these state calls in $\widehat{M}(s_i)$ with the corresponding right-hand side of rules, so $\he{\widehat{M}(s_{i+1})} \leq c\cdot\text{mhr} + \he{\widehat{M}(s_i)}$.
%Therefore: $\he{\widehat{M}(s_{i+1})} \leq c\cdot\text{mhr} + \he{\widehat{M}(s_i)}$. 
By applying this $\he{s}+1$ times, starting from $s_0$, we obtain that the height of $M(s)$ is $\leq c\cdot\text{mhr} \cdot(\he{s}+1)$. So $M$ is of linear-height-increase. 

%Let $M=(Q,P,\Sigma,\Delta,q_0,R,h)$ and let $\text{mhr}$ the maximum
%height of the right-hand side of any rule in $R$.
%Let $s$ be a fixed tree in $T_\Sigma$.
%
%To prove~(2), let $c$ be a yield nesting bound of $M$.
%For $i\in[\he{s}]$ let $s_i$ be the tree obtained from $s$ by replacing
%all nodes $u$ at depth $i$ by $h(s/u)$.
%We show that 
%$\he{M(s_{i+1}[u\leftarrow\sigma(h(s/u1),\dots,h(s/uk))\mid u\in V_P(s_i)])}
%\leq c\cdot\text{mhr} + \he{M(s_i)}$.
%By applying this $\he{s}$ times, starting from $s_1$, we obtain that
%the height of $M(s)$ is 
%$\leq \he{s}\cdot c\cdot\text{mhr}$
%
%Let $U$ be an arbitrary set of pairwise independent (i.e., not being
%descendants of each other) nodes of $s$.
%Let $s'=s[u\leftarrow h(s/u)\mid u\in U]$, let 
%$u\in V(s')$ with $s'/u\in P$, and $\sigma=s[u]\in\Sigma^{(k)}$ with
%$k\geq 0$.
%To prove~(1), let $c$ be a nesting bound of $M$.
%We now use a slight variation $\bar{M}$ of the extension of $M$:
%each output symbol $\<q,p>$ has rank $m+1$, where $m$ is the rank of $q$.
%In the first subtree of the output symbol $\<q,p>$ the mttr $\bar{M}$
%outputs a monadic tree that is the reverse Dewey path $\text{rev}(u)$ of the input
%node $u$ that created the symbol (this can easilty be achieved).
%We show that 
%$\he{\hat{M}(s'[u\leftarrow\sigma(h(s/u1),\dots,h(s/uk))])}
%\leq c\cdot\text{mhr} + \he{M(s')}$.
%By the definition of the nesting bound,
%there are at most $c$-many occurrences in any path of $\bar{M}(s')$
%of symbols from $\<Q,P>$ which all have $\text{rev}(u)$ as
%their first subtree. By the definition of the semantics of an mttr,
%all of these occurrences (and no other occurrence of that path)
%are replaced by the right-hand side of a rule. Thus,
%the height increases by at most $c\cdot\text{mhr}$.
%If we start with the tree $h(s')$ and successively replace a node $u$
%labeled by a symbol from $P$ by $\sigma(h(s/u1),\dots,h(s/uk))$,
%where $\sigma=s[u]$, then after $|s|$-many applications of rules
%as above we obtain the tree $M(s)$ thus having a height that
%is $\leq |s|\cdot c\cdot \text{mhr}$.
%\qed
\end{proof} 
%\vspace{-1mm}
