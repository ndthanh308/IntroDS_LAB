%!TEX root = ICALP_main.tex
\section{Linear Height and Linear Size-to-Height Increase}
\label{sec:decision_LSHI}

Let $\Gamma$ be a ranked alphabet and $t$ a tree over $\Gamma$.
We define the size $|t|$ of a tree as its number of nodes $|V(t)|$.
The height $\he{t}$ of $t$ is defined as
$\he{t}=0$ if $t\in\Gamma^{(0)}$ and 
$\he{t} = 1 + \text{max}\{\he{t_i}\mid i\in[k]\}$ if
$t=\gamma(t_1,\dots,t_k)$ for $\gamma\in\Gamma^{(k)}$, $k\geq 1$,
and $t_1,\dots,t_k\in T_\Gamma$.

Let $M$ be an mttr (with input ranked alphabet $\Sigma$).
Then $M$ has \emph{linear size-to-height increase} (for short LSHI) if
there exists a number $c$
such that for every input tree
$s\in T_\Sigma$: $\he{M(s)}\leq c\cdot |s|$.
The mttr $M$ has \emph{linear height increase} (for short LHI) if
there exists a number $c$ such that for every input tree
$s\in T_\Sigma$: $\he{M(s)}\leq c\cdot \he{s}$.

We now introduce two additional properties for mttrs which will allow
us to decide whether a given mttr has LSHI or LHI.
Recall that $\widehat{M}$ denotes the extension of $M$: this transducer
can translate input trees that may contain leaves that are labeled
by elements from $P$ (the set of look-ahead states of $M$).
Whenever the state $q$ of $M$, of rank $m$, encounters an input node $u$
labeled by an element $p$ of $P$, the transducer $\widehat{M}$ outputs 
$\< q,p>(y_1,\dots,y_m)$.
%, where 
%$\underline{q}\in\underline{Q}$ is a new output symbol of rank $m$
%and $t=\text{rev}(u)$ is a monadic tree that represents the reverse Dewey notation
%of the node $u$.
We call a tree in $s\in T_\Sigma(P)$ a \emph{$\Sigma$-context} if it contains
exactly one occurrence $u$ of an element of $P$.

We say that the mttr $M$ is \emph{finite nesting} (for short fnest), 
if there exists a number $c$ such that for every $\Sigma$-context $s$
there are at most $c$-many occurrences of symbols
$\<q,p>$ with $q\in Q$ on any path of the tree $\widehat{M}(s)$;
in this case, we say that $c$ is a \emph{nesting bound} of $M$. 
%We say that $M$ is \emph{infinite nesting} when it is not finite nesting. 
We say that $M$ is \emph{finite yield nesting} (for short fynest), 
if there exists
a number $c$ such that for every input tree $s\in T_\Sigma(P)$ 
there are at most $c$-many occurrences of symbols 
from $\< q,p>$ with $q\in Q$ on any path of the 
tree $\widehat{M}(s)$;
in this case, we say that $c$ is a \emph{yield nesting bound} of $M$. 
%We say that $M$ is \emph{infinite yield nesting} when it is not 
%finite yield nesting. 
The proof of the next lemma is straightforward (by reduction to Proposition~\ref{prop:finite})
and can be found in the Appendix.

\begin{lemma}\label{lm:decidable}
Let $M$ be an mttr.
Then 
(1)~it is decidable whether or not $M$ is finite nesting and
(2)~it is decidable whether or not $M$ is finite yield nesting.
\end{lemma}

Informally the next lemma is easy to understand, e.g., for Statement~(1),
if $M$ is finite nesting with bound $c$, then a single node of an input tree
can only ``contribute'' at most $c\cdot\text{mhr}$ nodes to the height of
the output tree, where mhr denotes the maximum height of the right-hand side of
any rule of the mttr. A formal proof can be found in the Appendix.

\begin{lemma}\label{lm:easy}
Let $M$ be an mttr.
(1)~If $M$ is finite nesting, then it is of linear size-to-height increase.
(2)~If $M$ is finite yield nesting, then it is of linear height increase. 
\end{lemma}

For another tree $t$ and a $\Sigma$-context $s$, $s[t]$ denotes the tree
$s[u\leftarrow t]$.

\begin{lemma}\label{lm:nest}
Let $M$ be an mttr that is depth proper.
(1)~If $M$ is not finite nesting, 
then $M$ does not have linear size-to-height increase.
(2)~If $M$ is not finite yield nesting, 
then $M$ does not have linear height increase.
\end{lemma}
\begin{proof} 
Let $M$ be given by a tuple as usual.
To prove~(1), assume that $M$ is not fnest.
We will show that this implies that $M$ does not have LSHI.
Since $M$ is not fnest (and has only finitely many states)
there must be some state $q\in Q^{(m)}$ with $m\geq 1$
that occurs arbitrarily often on paths of output trees of $\widehat{M}$.
More precisely, there are infinite sequences of contexts $c_0,c_1,\dots$
and numbers $n_0<n_1<\cdots$ such that
$q$ occurs $\geq n_0$ times on a path in $\widehat{M}(c_0)$ and
$q$ occurs $\geq n_1$ times on a path in $\widehat{M}(c_0[c_1])$, etc.
From this we can deduce (by considering sufficiently many numbers $n_i$),
similarly to the proof of Lemma~6.5 of~\cite{DBLP:journals/siamcomp/EngelfrietM03},
that $M$ is ``(nested) input pumpable'', i.e.,
there exist $q_1,q_2,j,s_0,s_1,u_0,u_1,p$ such that \vspace{-1mm}

\begin{enumerate}
\item $\< q_1,p>$ occurs in $\widehat{M}(s_0[u_0\leftarrow p])$,
\item $\widehat{M}_{q_1}(s_1[u_1\leftarrow p])$ has either: a subtree 
$\< q_1,p>(t_1,\dots,t_m)$ such that some $t_{j'}$ contains a 
subtree $\< q_2,p>(\xi_1,\dots,\xi_l)$
where $\xi_{j}$ contains $y_{j'}$ for some $j'\in[m]$, 
or a subtree $\< q_2,p>(t_1,\dots,t_l)$ such that $t_j$ contains a subtree 
$\< q_1,p>(\xi_1,$ $\dots,\xi_m)$, 
\item $\widehat{M}_{q_2}(s_1[u_1\leftarrow p])$ has a subtree 
$\< q_2,p>(t_1,\dots,t_l)$ such that $t_j$ contains $y_{j}$, and
\item $p=h(s_1/u_1)=h(s_1[u_1 \leftarrow p])$.
\end{enumerate}%\vspace{-1mm}

By ``pumping'', i.e., considering 
$s_n=s_0[u_0\leftarrow s_1[u_1\leftarrow s_1[u_1\leftarrow \dots ]]]$
with $n$ replacements of the node $u_1$, we obtain that
$\widehat{M}_{q_1}(s_n)$ contains a path with $\geq n$ nested occurrences of $\< q_2,p>$. 
Note that this proof is simpler than that of 
Lemma~6.5 of~\cite{DBLP:journals/siamcomp/EngelfrietM03} because we only look here 
at the height of outputs instead of the size of outputs. This is simpler because, 
in a mttr, a state call can copy a parameter containing large outputs of other state 
calls, creating a size growth of the output that is difficult to track, but these 
copies cannot be copied vertically on top of each other, so the output height is 
easier to track. 

Assume now by contradiction that $M$ has LSHI, i.e., there exists a $c$ such that
for every input tree $s\in T_\Sigma$: $\he{M(s)}\leq c\cdot |s|$.
Since $M$ is depth proper, we may choose $s\in L_p$ such that
$M_{q_2}(s)$ contains an occurrence of $y_{j}$ at depth
$\geq c c_1 +1$, where $c_1=|s_1[u_1\leftarrow p]|-1$.
We know that $\widehat{M}_{q_1}(s_n)$ contains $\geq n$ nested occurrences of $q_2$
(where always the $j$-th subtree of $q_2$ contains further nested occurrences of $q_2$).
Now let $t_n=s_0[u_0\leftarrow s_n[u_1^n\leftarrow s]]$
and take $n>c(c_0+c_2)$, where $c_0=|s_0[u_0\leftarrow p]|-1$ and $c_2=|s|$.
Since $|t_n|=c_0+nc_1+c_2$, we obtain that $\he{M(t_n)}>c\cdot |t_n|$ because
$\he{M(t_n)}\geq n(cc_1+1) > ncc_1 + c(c_0+c_2)=c\cdot |t_n|$ by the choice of $n$. 
So \emph{nested input pumpability} implies that $M$ is not of LSHI. 

%long version proof:
%(\emph{Detailed proof}.)
We now prove that if $M$ is not finite nesting, then it must be \emph{nested input pumpable}. 
In order to do so, we first introduce a few notations and characterize 
\emph{nested input pumpability} and the \emph{finite nesting} property using these notations. 
%The idea of the proof, for~(1) and~(2), is to show that infinite nesting (resp.\ yield nesting) implies the existence of a type of loop which nests a number of state calls that is linear in the size (resp.\ height) of the input. The depth-proper normal form then allows us to conclude by conjointly pumping the loop to linearly increase the number of state calls and increasing the height contributed by one state call, and so building outputs whose height increases more than linearly in the input's size (resp.\ height). To define those types of loops we first introduce a few necessary definitions. 

%new version of definition of \to_s:
%(New version of definition of $\to_s$.) Given a look-ahead state $p$, a $\Sigma$-context $s$ containing a leaf $p$, a path $u$ in $M_{q_0}(s)$, two states $q_0, q \in Q$ and a parameter $y_k$ of $q$, a \emph{call} to state $q$ \emph{nested on parameter} $y_k$ along path $u$ in $M_{q_0}(s)$ is a node $\<q,p>$ at a path $u'$ such that $u= u'\, k \, u''$ for some path $u''$. 
%%($u'$ is a strict prefix of $u$), and path $u$ leads into the $k$-th subtree of this node, i.e., 
%When there are $n$ calls to states $q_1, \dots, q_n$ nested on parameters $y_{k_1}, \dots, y_{k_n}$ respectively, along a path $u$ in $M_{q_0}(s)$ such that $M_{q_0}(s)/u = y_{k_0}$ (where $y_{k_0}$ is a parameter of $q_0$), we write: 
%\[(q_0,h(s),y_{k_0}) \to_s (q_1,p,y_{k_1}) \dots (q_n,p,y_{k_n})\]
%When there are $n$ calls to states $q_1, \dots, q_n$ nested on parameters $y_{k_1}, \dots, y_{k_n}$ respectively, along a path $u$ in $M_{q_0}(s)$ such that $M_{q_0}(s)/u = \< q_{n+1},p>$, we write: 
%\[(q_0,h(s),\bot) \to_s (q_1,p,y_{k_1}) \dots (q_n,p,y_{k_n}) (q_{n+1},p,\bot)\]
%We call \emph{nesting configurations} elements of the set $\QY = \{ (q,p,y_k) \mid q \in Q^{(m)}, p \in P, y_k \in Y_m \cup \{\bot\}\}$. 
%We have defined the relation $\to_s$ over $\QY \times \QY^*$, we now take its natural extension by concatenation over $\QY^* \times \QY^*$, i.e., the smallest extension of $\to_s$ such that, for all $w_1, w'_1, w_2, w'_2 \in \QY^*$, if $w_1 \to_s w'_1$ and $w_2 \to_s w'_2$ then $w_1 w_2 \to_s w'_1 w'_2$. 
%
%\begin{claim}
%The relations $\to_s$ for all $\Sigma$-contexts $s$ have the following properties:
%\begin{itemize}
%\item 
%
%\end{itemize}
%
%\end{claim}

%new new version of the definition of \to_c
%(New new version of definition of $\to_c$.)
Let $c$ be a $\Sigma$-context and $q \in Q$ a state of $M$. 
To talk about the nesting of states in $\widehat{M}_q(c)$, we first give a notation for paths: 
\begin{enumerate}
\item For any node $u$ at depth $n$ in $\widehat{M}_q(c)$, we note the path to node $u$ as the sequence of pairs: 
\[(\ell_1,i_1) \, (\ell_2,i_2) \dots (\ell_{n},i_{n}) \, (\ell_{n+1},\bot)\] 
where $i_1, \dots, i_n$ are indexes such that $u = i_1\, i_2\dots i_n$ and, for all $j \leq n+1$, $\ell_j$ is the label of node $i_1\dots i_{j-1}$ or, if node $i_1\dots i_{j-1}$ is labeled by a state call $\< q',p>$, then $\ell_j = q'$, 
\item Since we are only interested in the nesting of states, we remove from such paths all pairs $(\ell_j,i_j)$ where $\ell \in \Delta$. We obtain nesting sequences of the form: 
\[(q_1,k_1)\, (q_2, k_2) \dots (q_n,k_n)\, (\ell_{n+1},k_{n+1})\]
where $\ell_{n+1}$ is either a state in $Q$ or a parameter, $k_{n+1} \in \{\bot\} \cup \N$, 
and for all $j \leq n$, $k_j \in [m_j]$ where $m_j$ is the arity of state $q_j$. 
\item For each such sequence, if $\ell_{n+1} = q_{n+1} \in Q$ then we write: 
\[(q,\bot) \to_c (q_1,k_1)\, (q_2, k_2) \dots (q_n,k_n)\, (\ell_{n+1},k_{n+1})\]
Otherwise $\ell_{n+1} = y_k$ is a parameter of $q$, $k_{n+1} = \bot$ and we write: 
\[(q,k) \to_c (q_1,k_1)\, (q_2, k_2) \dots (q_n,k_n)\]
\end{enumerate}
This defines a relation $\to_c \subseteq \QY \times \QY^*$ where $\QY = \{ (q,k) \mid q \in Q^{(m)}, k \in [m] \cup \{\bot\}\}$ and $\QY^*$ denotes the set of (possibly empty) sequences of elements of $\QY$. 
Note that if $(q,\bot) \to_c w$, then in the nesting sequence $w\in \QY^*$ only the last pair may contain a $\bot$. 
A \emph{nesting loop} is given by a $\Sigma$-context $c$ with a leaf labeled $p$ such that $h(c) = p$, and two pairs $(q_1,k_1), (q_2,k_2) \in \QY$ such that:
\begin{itemize}
\item $(q_1,k_1) \to_c w_1 \, (q_1,k_1) \, w_2 \, (q_2,k_2) \, w_3$ or $(q_1,k_1) \to_c w_1 \, (q_2,k_2) \, w_2 \, (q_1,k_1) \, w_3$, 
\item $(q_2,k_2) \to_c w_4 \, (q_2,k_2) \, w_5$
\item $(q_1,p)$ is reachable, i.e., there exists $\Sigma$-context $c_0$ with a leaf labeled $p$ such that $\< q_1,p>$ appears in $\widehat{M}(c_0)$. 
\end{itemize}
for some nesting sequences $w_1, w_2, w_3, w_4, w_5 \in \QY^*$. This allows us to rephrase 
the \emph{nested input pumpability} property as the existence of a \emph{nesting loop}. 
We want to prove that if $M$ is not finite nesting then it has a nesting loop. 

We extend the relation $\to_c$ to sequences of pairs on the left so that, for pairs $(q_1,k_1),\ab (q_2,k_2) \in \QY$ and sequences $w_1, w_2 \in \QY^*$, if $(q_1,k_1) \to_c w_1$ and $(q_2,k_2) \to_c w_2$ then $(q_1,k_1)\, (q_2,k_2) \to_c w_1 w_2$. More generally, for all sequences $w_1, w'_1, w_2, w'_2 \in \QY^*$, if $w_1 \to_c w'_1$ and $w_2 \to_c w'_2$ then $w_1 \, w_2 \to_c w'_1 \, w'_2$. We can now show the following claim:

\begin{claim}\label{cla:nesting_decomposition}
For all $\Sigma$-contexts $c$ and $c'$ with leafs labeled resp.\ $p$ and $p'$ such that $p=h(c')$, we can define the $\Sigma$-context $c \cdot c' = c[p \leftarrow c']$ and, for all sequences $w, w'' \in \QY^*$, if $w \to_{c\cdot c'} w''$ then there exists a sequence $w' \in \QY^*$ such that $w \to_c w' \to_{c'} w''$. 
\end{claim}

\begin{proof}
%(Sketch.) 
We only need to show this for $w = (q_0,k_0) \in \QY$ because of the definition of $\to_c$ on sequences of pairs. 
%We do so by looking at the path in $\widehat{M}_{q_1}(c \cdot c')$ reducing to $w''$. Then the corresponding path in $\widehat{M}_{q_0}(c)$ reduces to $w'$. 
%There are two cases depending on $k_0 \in [m] \cup \{\bot\}$. 
%Let us first assume that $k_0 = \bot$. 
%We note $w'' = (q_1,k_1)\, (q_2, k_2) \dots (q_n,k_n) \, (q_{n+1},\bot)$. 
%Because $(q_0,k_0) \to_{c \cdot c'} (q_1,k_1) \dots (q_n,k_n) \, (q_{n+1},\bot)$, 
Because $(q_0,k_0) \to_{c \cdot c'} w''$, there must be a path $\Pi$ in $\widehat{M}_{q_0}(c \cdot c')$ reducing 
to $w''$ (by removing pairs in $\Delta \times \N$ and removing $(y_{k_0},\bot)$ if $k_0 \neq \bot$). 
Because $\widehat{M}_{q_0}(c \cdot c') = \widehat{M}_{q_0}(c)[\< q,p> \leftarrow \widehat{M}_{q}(c')]$, 
path $\Pi$ can be similarly obtained from a path $\Pi'$ in $\widehat{M}_{q_0}(c)$ by substituting 
each $(q,k)$ with a path in $\widehat{M}_q(c')$. 
More specifically, noting $(q_1,k_1), \dots, (q_n,k_n)$ the pairs in path $\Pi'$ that are in $\QY$ 
(in order of apparition in $\Pi'$), we substitute in $\Pi'$:
\begin{itemize}
\item each occurrence of a pair $(q_i,k_i) \in \QY$ by a path $\Pi'_i$ such that $\Pi' (y_{k_0,\bot})$ is a path in $\widehat{M}_{q_i}(c')$ (for $i \leq n$), 
\item each occurrence of a pair $(q_n,\bot)$ by a path $\Pi'_n$ in $\widehat{M}_{q_n}(c')$. 
\end{itemize}
We get: $\Pi = \Pi'[(q_i,k_i) \leftarrow \Pi'_i]$ and, by removing pairs in $(\Delta \times \N) \cup (Y^m \times \{\bot\})$: 
\[w'' = w'_1\, w'_2 \, \dots \, w'_n\]
where for all $i \leq n$, $w'_i$ is obtained from $\Pi'_i$ by removing pairs in $(\Delta \times \N) \cup (Y^m \times \{\bot\})$. 
Then, for all $i \leq n$ and by definition of $\Pi'_i$, we have $(q_i,k_i) \to_{c'} w'_i$. 
So $(q_1,k_1) \dots (q_n,k_n) \to_{c'} w'_1\, w'_2 \, \dots \, w'_n = w''$. 

We note $w'$ the sequence obtained from $\Pi'$ by removing pairs in 
$(\Delta \times \N) \cup (Y^m \times \{\bot\})$.
Then $w' = (q_1,k_1) \dots (q_n,k_n)$ and so $(q_0,k_0) \to_{c} w' \to_{c'} w''$. 
%If $k_0 = \bot$ then we have $w' = (q_1,k_1) \dots (q_n,k_n)$ with $k_n = \bot$, 
%and so $(q_0,k_0) \to_{c} w' \to_{c'} w''$. 
%If $k_0 \neq \bot$ then we have $w' = (q_1,k_1) \dots (q_n,k_n)\, (y_{k_0},\bot)$, 
%and so $(q_0,k_0) \to_{c} (q_1,k_1) \dots (q_n,k_n) \to_{c'} w''$. 
\end{proof}
We could also prove that $\to_{c\cdot c'} \,=\, \to_{c'} \circ \to_{c}$, but it is not necessary for this proof. 

To prove that $M$ has a nesting loop (i.e.\ $M$ is nested input pumpable), we assume that $M$ is not finite nesting. 
Then, for all $n \in \N$, there exists a $\Sigma$-context $c_n$ such that: $(q_0,\bot) \to_{c_n} w$ for some $w \in \QY^*$ with $|w| \geq n$. We can decompose any such $c_n$ into a concatenation $c_{n,1} \cdot c_{n,2} \cdot \dots c_{n,r}$ and use the claim to obtain: 
\[(q_0,\bot) \to_{c_{n,1}} w_1 \to_{c_{n,2}} w_2 \dots \to_{c_{n,r}} w_r \]
where $w_1, w_2, \dots w_r \in \QY^*$ and $|w_r| = |w| \geq n$. By choosing a big enough $n$, we will show how to find a \emph{nesting loop}. To do that, we decompose $c_n$ into several contexts and use the claim. 

A $\Sigma$-context $c$ is \emph{atomic} if its leaf labeled in $P$ is a child node of its root. 
Let $c = \sigma(t_1, \dots, t_{i-1}, p_i, t_{i+1}, \dots, t_k)$ be an atomic $\Sigma$-context and $q \in Q$ a state of $M$. Noting $p_j = h(t_j)$ for $j \neq i$, there is in $M$ a rule $\< q, \sigma(x_1:p_1, \dots, x_k:p_k)> (y_1, \dots, y_m) \to t$. 
Then $\widehat{M}_q(c) = t[\< q', x_j> \leftarrow M_{q'}(c/j) \mid j \neq i]$. Because $M_{q'}(c/j) \in T_\Delta$ for all $q' \in Q$ and $j \neq i$, the nesting of state calls in $\widehat{M}_q(c)$ is the nesting of state calls of the form $\< q',x_i>$ in $t$. 
So, for $(q,k) \in \QY$, the length of nesting sequences $w$ such that $(q,k) \to_c w$ is bounded by the height of $t$. 
There is a finite number of rules for $M$, so there is a finite number of such $t$ and the length of sequences $w \in \QY$ such that $(q,k) \to_c w$ has an upper bound $B$ that does not depend on $q,k$ or $c$. 
In other words, for all $(q,k) \in \QY$, $w \in \QY^*$ and atomic $\Sigma$-context $c$ we have:
\[ (q,k) \to_c w ~~~~ \Rightarrow ~~~~ |w| \leq B \]
Moreover, for all $w_1, w_2 \in \QY^*$: $w_1 \to_c w_2 ~~~ \Rightarrow ~~~ |w_2| \leq B\, |w_1|$. 

We decompose the $\Sigma$-context $c_n$ into atomic $\Sigma$-contexts $c_{n,1}, c_{n,2}, \dots, c_{n,r}$. 
Since $(q_0,\bot)\ab \to_{c_{n,1}} w_1 \to_{c_{n,2}} w_2 \, \dots \to_{c_{n,r}} w_r$, 
we have $|w_r| \leq B^r$ and, since $|w_r| \geq n$: $n \leq B^r$. 
So, by taking $n$ big enough, we can also make $r$ as big as we want. 
In order to find a nesting loop, we require more structure on the nesting sequences 
$c_{n,1}, \dots, c_{n,r}$. The precise structure we need is described in the following claim: 
\begin{claim}
For all for all $r \in \N$, if there exists a $\Sigma$-context $c$, a pair $\theta \in \QY$ and a sequence $w \in \QY^*$ with $\theta \to_{c} w$ and $|w| \geq B^r$, then there exists $\Sigma$-contexts $c_1, \dots, c_{r-1}$, look-ahead states $p_1, \dots, p_r$ and pairs $\theta_{i,j} \in \QY$ for all $i,j \in [r]$ with $j \leq i$ such that:
\begin{itemize}
\item for all $i \in [r-1]$, $h(c_i) = p_i$ and $c_i$ has a leaf labeled $p_{i+1}$, 
\item $\theta_{1,1} = \theta$, 
\item for all $i,j \in [r-1]$ with $j < i$, there exists $w_{i,j}, w'_{i,j} \in \QY^*$ such that: 
$\theta_{i,j}\ab \to_{c_i} w_{i,j} \, \theta_{i+1,j} \, w'_{i,j}$, 
\item for all $i \in [r-1]$, there exists $w_i, w'_i, w''_i \in \QY^*$ such that either 
$\theta_{i,i} \to_{c_i} w_{i} \, \theta_{i+1,i}\ab \, w'_{i} \, \theta_{i+1,i+1} \, w''_i$ or 
$\theta_{i,i} \to_{c_i} w_{i} \, \theta_{i+1,i+1} \, w'_{i} \, \theta_{i+1,i} \, w''_i$. 
\end{itemize}
\end{claim}

These conditions can be summed up graphically. To simplify the picture, we replace all sequences 
$w_{i,j}, w'_{i,j}, w_i, w'_i, w''_i$ for $i,j \in [r]$ with the symbol $\thicksim$. 
\begin{center}
	\begin{tikzpicture}
		\newcommand{\halfblob}[5]{ % #1: radius, #2: half-length, #3: x-coordinate, #4: y-coordinate, #5: name
			\draw (#3 - #2,#4 + #1) arc [start angle=90, end angle=180, radius=#1];
			\draw (#3 + #2 + #1,#4) arc [start angle=0, end angle=90, radius=#1];
			\draw (#3 - #2,#4 + #1) -- (#3 + #2,#4 + #1);
			%\draw (#3 - #2,#4 - #1) -- (#3 + #2,#4 - #1);
			\node at (#3,#4) {#5};
		}
		
		\node at (0,0) {$\theta_{1,1}$};
		\draw[->] (0,-0.3) -> (0,-0.8);
		\halfblob{0.3}{1}{0}{-1.1}{$\thicksim \theta_{2,1} \thicksim \theta_{2,2} \thicksim$}
		
		\draw[->] (-0.6,-1.3) -> (-1.4,-1.9);
		\halfblob{0.3}{0.5}{-1.4}{-2.2}{$\thicksim \theta_{3,1} \thicksim$}
		\draw[->] (0.6,-1.3) -> (1.4,-1.9);
		\halfblob{0.3}{1}{1.4}{-2.2}{$\thicksim \theta_{3,2} \thicksim \theta_{3,3} \thicksim$}
		\node at (-0.23,-2.2) {$\dots$};
		
		\draw[->] (-1.45,-2.4) -> (-2.15,-3);
		\node[rotate=43] at (-2.34,-3.2) {$\dots$};
		\draw[->] (-2.55,-3.4) -> (-3.25,-4);
		\halfblob{0.3}{0.5}{-3.45}{-4.3}{$\thicksim \theta_{r,1} \thicksim$}
		
		\draw[->] (0.75,-2.4) -> (0.05,-3);
		\node[rotate=43] at (-0.13,-3.2) {$\dots$};
		\draw[->] (-0.35,-3.4) -> (-1.05,-4);
		\halfblob{0.3}{0.5}{-1.25}{-4.3}{$\thicksim \theta_{r,2} \thicksim$}
		
		\draw[->] (2,-2.4) -> (2.8,-3);
		\node[rotate=-40] at (3.03,-3.2) {$\dots$};
		\draw[->] (2.7,-3.3) -> (1.8,-4);
		\halfblob{0.3}{0.7}{1.55}{-4.3}{$\thicksim \theta_{r,r-2} \thicksim$}
		\draw[->] (3.22,-3.34) -> (4.1,-4);
		\halfblob{0.3}{1.2}{4.7}{-4.3}{$\thicksim \theta_{r,r-1} \thicksim \theta_{r,r} \thicksim$}
		
		\node at (-2.33,-4.3) {$\dots$};
		\node at (0.08,-4.3) {$\dots$};
		\node at (2.9,-4.3) {$\dots$};
		
		%vertical arrows on the side
		\draw[->] (-5,-0.1) -> (-5,-1);
		\node at (-4.67,-0.55) {$c_{n,1}$};
		\draw[->] (-5,-1.2) -> (-5,-2.1);
		\node at (-4.67,-1.65) {$c_{n,2}$};
		\draw[->] (-5,-2.3) -> (-5,-2.9);
		\node at (-4.67,-2.7) {$c_{n,3}$};
		\node at (-5,-3.1) {$\vdots$};
		\draw[->] (-5,-3.5) -> (-5,-4.2);
		\node at (-4.5,-3.8) {$c_{n,r-1}$};
		
		%horizontal gray lines
		\draw[gray!20] (-5,0) -> (-0.6,0);
		\draw[gray!20] (-5,-1.1) -> (-1.45,-1.1);
		\draw[gray!20] (-5,-2.2) -> (-2.35,-2.2);
		\draw[gray!20] (-5,-4.3) -> (-4.4,-4.3);
		
		%look-ahead states
		\node at (-5.25,0) {$p_1$};
		\node at (-5.25,-1.1) {$p_2$};
		\node at (-5.25,-2.2) {$p_3$};
		\node at (-5.25,-4.3) {$p_r$};		
		
	\end{tikzpicture}
\end{center}
Note that, in this representation, we chose to represent 
$\theta_{i,i} \to_{c_i} w_{i} \, \theta_{i+1,i} \, w'_{i} \, \theta_{i+1,i+1} \, w''_i$ instead of 
$\theta_{i,i} \to_{c_i} w_{i} \, \theta_{i+1,i+1} \, w'_{i} \, \theta_{i+1,i} \, w''_i$ for all $i \in [r-1]$. 
But this distinction does not matter to the proof of the claim. 
%but because the order of nesting does not matter for nesting loops, 
%we can assume that branching is happening on the right without loss of generality (in fact 
%branching on the left makes the proof easier by excluding the case where $\bot$ occurs on the branching side). 
From now on we use $\thicksim$ to denote arbitrary sequences in $\QY^*$ which we will not use to find a nesting loop. 

\begin{proof}
We prove this by induction on $r$. 
Let $c$ be a $\Sigma$-context, $\theta$ a pair in $\QY$ and $w$ a sequence in $\QY^*$ with $\theta \to_{c} w$ and $|w| \geq B^{r+1}$. 
We split $c$ into atomic $\Sigma$-contexts $c'_1, \dots, c'_n$, then we have sequences $w_1, \dots, w_{n-1} \in \QY^*$ 
such that $\theta \to_{c'_1} w_1 \dots \to_{c'_{n-1}} w_{n-1} \to_{c'_n} w$. 
Let $i$ be the largest $i$ such that there is a pair $\theta'$ in sequence $w_i$ with $\theta' \to_{c'_{i+1} \cdots c'_n} w'$ and $|w'| \geq B^r$. 
If we had $|w'| \geq B^{r+1}$ then, because $c'_{i+1}$ is atomic, we would have a $\theta''$ in sequence $c'_{i+1}$ with $\theta'' \to_{c'_{i+2} \cdots c'_n} w''$ and $|w''| \geq B^r$. 
So $B^r \leq |w'| < B^{r+1} \leq |w|$. 
Therefore there is another pair $\theta_{2,1}$ in $w_i$ (other than $\theta'$) with $\theta_{2,1} \to_{c'_{i+1} \cdots c'_n} w''$ and $|w''| \geq 1$. 

Since $\theta' \to_{c'_{i+1} \cdots c'_n} w'$ and $|w'| \geq B^r$, we use the induction hypothesis on $theta'$ and $c'_{i+1} \cdots c'_n$. In order to prove the induction for $r+1$, we rename the $\Sigma$-contexts $c_1, \dots c_{r-1}$, look-ahead states $p_1, \dots, p_r$ and pairs $\theta_{i,j}$ (for $j \leq i \leq r$) into $\Sigma$-contexts $c_2, \dots c_{r}$, look-ahead states $p_2, \dots, p_{r+1}$ and pairs $\theta_{i+1,j+1}$ (for $j \leq i \leq r$). 
Then $c'_{i+1} \cdots c'_n = c_2 \cdots c_{r}$. 

Since $\theta_{2,1} \to_{c_2 \cdots c_r} w''$ with $|w''| \geq 1$, there are pairs $\theta_{3,1}, \dots, \theta_{n,1}$ such that $\theta_{n,1}$ appears in sequence $w''$ and, for all $i \in [r]$ with $i \geq 2$, $\theta_{i,1} \to_{c_i} w'_i \, \theta_{i+1,1} \, w''_i$ with $w'_i, w''_i \in \QY^*$. 
To conclude, we choose $c_1 = c'_1 \dots c'_i$, $p_1 = h(c_1)$ and $\theta_{1,1} = \theta$. 
\end{proof}

In order to find a nesting loop, we need two indexes $i$ and $j$ with: 
\begin{center}
	\begin{tikzpicture}
		\newcommand{\halfblob}[5]{ % #1: radius, #2: half-length, #3: x-coordinate, #4: y-coordinate, #5: name
			\draw (#3 - #2,#4 + #1) arc [start angle=90, end angle=180, radius=#1];
			\draw (#3 + #2 + #1,#4) arc [start angle=0, end angle=90, radius=#1];
			\draw (#3 - #2,#4 + #1) -- (#3 + #2,#4 + #1);
			%\draw (#3 - #2,#4 - #1) -- (#3 + #2,#4 - #1);
			\node at (#3,#4) {#5};
		}
		
		\node at (0,0) {$(q_0,\bot)$};
		\draw[->] (0,-0.3) -> (0,-1.7);
		\halfblob{0.3}{1.7}{0}{-2}{$\thicksim \theta_{i,1} \thicksim \theta_{i,2} \dots \thicksim \theta_{i,i} \thicksim$}
		
		\draw[->] (-1.45,-2.2) -> (-3,-3.7);
		\halfblob{0.3}{0.5}{-3.2}{-4}{$\thicksim \theta_{j,1} \thicksim$}
		
		\draw[->] (-0.4,-2.2) -> (-1.4,-3.7);
		\halfblob{0.3}{0.5}{-1.4}{-4}{$\thicksim \theta_{j,2} \thicksim$}
		
		\draw[->] (1.4,-2.2) -> (1.9,-3.7);
		\halfblob{0.3}{1.2}{2.2}{-4}{$\thicksim \theta_{j,i} \dots \thicksim \theta_{j,j} \thicksim$}
		\node at (0.1,-4) {$\dots$};
		
		%vertical arrows on the side
		\draw[->] (-5,-0.1) -> (-5,-0.7);
		\node at (-4.67,-0.43) {$c_{n,1}$};
		\node at (-5,-0.9) {$\vdots$};
		\draw[->] (-5,-1.3) -> (-5,-1.9);
		\node at (-4.5,-1.65) {$c_{n,i-1}$};

		\draw[->] (-5,-2.1) -> (-5,-2.7);
		\node at (-4.67,-2.45) {$c_{n,i}$};
		\node at (-5,-2.9) {$\vdots$};
		\draw[->] (-5,-3.3) -> (-5,-3.9);
		\node at (-4.5,-3.6) {$c_{n,j-1}$};
		
		%horizontal gray lines
		\draw[gray!20] (-5,0) -> (-0.6,0);
		\draw[gray!20] (-5,-2) -> (-2.1,-2);
		\draw[gray!20] (-5,-4) -> (-4.1,-4);
		
		%look-ahead states
		\node at (-5.25,0) {$p_1$};
		\node at (-5.25,-2) {$p_i$};
		\node at (-5.25,-4) {$p_j$};		
		
	\end{tikzpicture}
\end{center}
Formally, we require two indexes $i,j$ with $i < j < r$ which share the same:
\begin{itemize}
\item look-ahead state $h(c_{n,i}) = h(c_{n,j})$, 
\item pair $\theta_{i,i} = \theta_{j,j} \in \QY$, 
\item set of pairs $\{\theta_{i,\ell}\}_{0\leq \ell \leq i} = \{\theta_{j,\ell}\}_{0\leq \ell \leq j}$.  
\end{itemize}
We ensure the existence of such $i,j$ by taking $r \geq |P|\,|Q|\,(m+1) \, 2^{|Q| (m+1)} +1$ 
%(so $n = B^{|P|\,|Q|\,(m+1)\, 2^{|Q| (m+1)} +1}$) 
where $m$ is the maximum arity of states. 
%We ensure the existence of such $i,j$ by taking $r \geq |P|\,|\QY| \, 2^{|\QY|} +1$ 
%(so $n = B^{|P|\,|\QY|\, 2^{|\QY|} +1}$) with $|\QY| \leq |Q| \, (m+1)$ where $m$ is the maximum arity of states. 
We now show how to build the nesting loop from indexes $i,j$. 
We note $p = h(c_{n,i}) = h(c_{n,j})$, $(q_1,k_1) = \theta_{i,i} = \theta_{j,j}$ and 
$S = \{\theta_{i,\ell}\}_{0\leq \ell \leq i} = \{\theta_{j,\ell}\}_{0\leq \ell \leq j}$. 
We note $c' = c_{n,i}.c_{n,i+1}. \dots. c_{n,j-1}$. 
Note that $c'$ has a leaf labeled $p$ and $h(c')=p$. 
%drawing maybe

We need the sets $\{\theta_{i,\ell}\}_{0\leq \ell \leq i}$ and $\{\theta_{j,\ell}\}_{0\leq \ell \leq j}$ 
to be equal so that the pairs $\theta_{i,k}$ for $k \leq i$ loop on each other through the loop $c'$. 
Formally, noting $\theta'_0 = \theta_{j,i}$, for all $\theta'_k \in S$ for $k \in \N$, there exists 
$\theta'_{k+1} \in S$ such that $\theta'_k \to_c' \,\thicksim \theta'_{k+1} \thicksim$. 
Since $S \subseteq \QY$ is finite, there must be $n, m \in \N$ such that 
$\theta'_{n} = \theta'_{n+m}$ (with $m \geq 1$), so $\theta'_n \to_{c'^m} \,\thicksim \theta'_n \thicksim$. 
Also $(q_1,k_1) \to_{c'} \,\thicksim \theta'_0 \thicksim (q_1,k_1) \thicksim\,$ and $\theta'_0 \to_{c'^n} x_n$, 
so $(q_1,k_1) \to_{c'^{n+1}} \, \thicksim \theta'_n \thicksim (q_1,k_1) \thicksim\,$ and, for all $m' \in \N$: 
$(q_1,k_1) \to_{c'^{m'}} \, \thicksim (q_1,k_1) \thicksim \, \to_{c'^{n+1}} \, \thicksim \theta'_n \thicksim (q_1,k_1) \thicksim$. 
Finally, for $c = c'^{m(n+1)}$, we have 
$(q_1,k_1) \to_{c} \, \thicksim \theta'_n \thicksim (q_1,k_1)\thicksim$ and $\theta'_n \to_{c} \theta'_n$. 
So we have a \emph{nesting loop}. 

In conclusion, if $M$ is not finite nesting, then it is \emph{nested input pumpable}, and so it does not have linear size-to-height increase. 
\vspace{2mm}

The proof of Statement~(2) can be given in a very similar way as for~(1), here we only outline the changes to the notations which allow to adapt the proof of~(1) to~(2). 
We replace $\Sigma$-contexts with elements of the set $T_\Sigma(P)$ containing possibly several leafs labeled in $P$. The rest of the notational changes are consequences of this change. 
Given a $s \in T_\Sigma(P)$, we now consider the nesting of state calls called on distinct subtrees of $s$ with potentially distinct look-ahead states. We augment pairs in $\QY$ so as to include the look-ahead, so $\QY = \{ (q,k,p) \mid q \in Q^{(m)}, k \in [m] \cup \{\bot\}, p \in P\}$. The notation $(q,k,p) \to_s (q_1,k_1,p_1) \dots (q_n,k_n,p_n)$ means that $h(s)=p$ and calls to states $q_1, \dots, q_n$ on leafs of $s$ labeled $p_1, \dots, p_n$ resp.\ are nested on parameters $y_{k_1}, \dots, y_{k_n}$ along a path in $\widehat{M}_q(s)$. This means that, when concatenating contexts, we write $s(s_1, \dots, s_m)$ instead of $s\cdot s'$. 

For~(2), similarly to nesting loops for~(1), we define a \emph{yield nesting loop} as given by contexts $s_1, s_2 \in T_\Sigma(P)$, look-ahead states $p_1, p_2 \in P$ and triplets $(q_1,k_1,p_1), (q_2,k_2,p_2) \in \QY$ such that:
\begin{itemize}
\item $h(s_1) = p_1$, $h(s_2) = p_2$, $s_1$ has two leafs labeled $p_1$ and $p_2$, $s_2$ has one leaf labeled $p_2$, 
\item $\<q_1,p_1>$ is reachable, 
\item either $(q_1,k_1,p_1) \to_{s_1} \thicksim\, (q_2,k_2,p_2) \,\thicksim\, (q_1,k_1,p_1) \,\thicksim$ \\
\phantom{.} \hspace{2.5mm} or $(q_1,k_1,p_1) \to_{s_1} \thicksim\, (q_1,k_1,p_1) \,\thicksim\, (q_2,k_2,p_2) \,\thicksim$,
\item $(q_2,k_2,p_2) \to_{s_2} \thicksim\, (q_2,k_2,p_2) \,\thicksim$. 
\end{itemize}
  
We say that $M$ is \emph{yield nested input pumpable} when it has either a \emph{yield nesting loop} or a \emph{nesting loop}. 
Note that the existence of either of these loops falsifies the linear height increase property. 
To prove~(2) we prove that infinite yield nesting implies the existence of either a yield nesting loop or a nesting loop. That proof works similarly to~(1): $M$ is not fynest so we can find large enough nesting sequences (but with the new definition of $\to_s$), find a repeating triplet $(q_1,k_1,p_1)$, pump the loop enough times that a triplet $(q_2,k_2,p_2)$ loops onto itself. Note that if the nested calls to $(q_1,k_1,p_1)$ and $(q_2,k_2,p_2)$ in $(q_1,k_1,p_1) \to_{s_1} \,\thicksim (q_2,k_2,p_2) \thicksim (q_1,k_1,p_1) \thicksim$ are on the same leaf in $s_1$ (with $p_1 = p_2$), then we get a nesting loop (otherwise we get a yield nesting loop). 



%%old version of definition of \to_s
%(Old version.) When considering the nesting of states, we want to specify the look-ahead the state is called upon, and in which parameter the nesting occurs. We define the set $\QY = \{ (q,p,y_k) \mid q \in Q^{(m)}, p \in P, y_k \in Y_m \cup \{\bot\}\}$ of nesting state configurations. We use configuration $(q,p,\bot)$ to denote the state called at the bottom of a nesting chain, with no state calls in any of its parameters. 
%
%For a $\Sigma$-context $s$ containing a leaf $p$, we write $(q,h(s),\bot) \to_s (q_1,p,y_{k_1}) \dots (q_{n-1},p, y_{k_{n-1}}) (q_n,p,\bot)$ when state calls $\<q_1,p>, \<q_2,p>, \dots$ appear nested in $M_{q}(s)$, with $\<q_2,p>$ appearing in parameter $y_{k_1}$ of $\<q_1,p>$, $\<q_3,p>$ appearing in parameter $y_{k_2}$ of $\<q_2,p>$ and so on. We write $(q,h(s),y_k) \to_s (q_1,p,y_{k_1}) \dots (q_n,p,y_{k_n})$ when $(q,h(s),\bot) \to_s (q_1,p,y_{k_1}) \dots (q_{n-1},p, y_{k_{n-1}}) (q_n,\bot)$ and parameter $y_k$ of $q$ appears in parameter $y_{k_n}$ of $\<q_n,p>$. 
%%For a $\Sigma$-context $s$ containing look-ahead state $p$, we write $(q_0,p_0,y_{k_0}) \to_s (q_1,p,y_{k_1}) \dots (q_n,p,y_{k_n})$ when $p_0 = h(s)$ and, in $M_{q_0}(s)$, state calls $\<q_1,p>, \<q_2,p>, \dots$ appear nested on parameters $y_{k_1}, y_{k_2}, \dots$, along a path leading to parameter $y$. 
%%We write $(q_0,p_0,\bot) \to_s (q_1,p,x_1) \dots (q_n,p,x_n)$ when $p_0 = h(s)$ and there exists exactly one $x_i$ with $x_i = \bot$ such that, in $M_{q_0}(s)$, state calls $\<q_1,p>, \dots$ appear nested on parameters $y_{k_1}, y_{k_2}, \dots$, along a path leading to parameter $y$. 
%Note that $\to_s$ is closed under subsequence on the right (i.e.\ we can "forget" part of the nested configurations). 
%We extend the relation $\to_s$ so that it is commutative and compatible with concatenation of sequences in $\QY^*$, i.e.\ so that $w \to_s w_2 w_1$ when $w \to_s w_1 w_2$, and $w_1 w_2 \to_s w'_1 w'_2$ when both $w_1 \to_s w'_1$ and $w_2 \to_s w'_2$, for all $w, w_1, w'_1, w_2, w'_2 \in \QY^*$. 
%Given two $\Sigma$-contexts $s$ and $s'$ containing look-ahead states $p$ and $p'$, if $p = h(s')$, then $s\cdot s'$ denotes the $\Sigma$-context obtained from $s$ by replacing $p$ with $s'$. Note that, for all $a \in \QY$ and $w \in \QY^*$, $a \to_{s\cdot s'} w$ if and only if there exists $w' \in \QY^*$ such that $a \to_s w'$ and $w' \to_{s'} w$. 
%%We sometimes write $\to$ instead of $\to_s$ when $s$ is obvious from context or does not matter. 
%
%
%%definition of nesting loops
%(Definition of nesting loops.) We can now define \emph{nesting loops}, whose existence characterizes \emph{nested input pumpability}. 
%A \emph{nesting loop} is given by a $\Sigma$-context $s$ and two configurations $(q_1,p,y_i)$ and $(q_2,p,y_j) \in \QY$ such that: $\<q_1,p>$ is reachable, $(q_1,p,y_i) \to_s (q_2,p,y_j) (q_1,p,y_i)$ and $(q_2,p,y_j) \to_s (q_2,p,y_j)$. Note that $y_i$ could be either $\bot$ or a parameter of $q_1$ (these two cases represent the two cases in point 2. of the definition of nested input pumpability), but in either case $y_j \neq \bot$. 
%
%
%%\begin{claim}
%%If $M$ has a nesting loop then $M$ is not of linear size-to-height increase. 
%%\end{claim}
%%
%%\begin{proof}
%%We prove this by pumping the loop: pumping $n$ times gives either $(q_1,y_i) \to_{s^n} (q_1,y_i) (q_2,y_j)^n$ or $(q_1,y_i) \to_{s^n} (q_2,y_j)^n (q_1,y_i)$. Since $\<q_1,p>$ is reachable, there exists a $\Sigma$-context $s_0$ which leads to the state call $\<q_1,p>$. For all input tree $t \in L_p$, the output of $M$ on input $s_0(s^n(t))$ nests $n$ calls to state $q_2$ on tree $t$ (nested along parameter $y_j$). The input size is $|s_0| + n.|s| + |t|$ and the output height is at least $n.d$ where $d$ is the maximum depth of parameter $y_j$ in $q_2(t)$. 
%%If $M$ was of linear size-to-height increase we would have a bound $B$ such that $B.(|s_0| + n.|s| + |t|)\geq n.d$ for all $n$, so $B.|s| \geq d$. 
%%Since $M$ is depth proper, there is no upper bound for the depth $d$ where $y_j$ occurs in $q_2(t)$ for $t \in L_p$, so $M$ can not be of linear size-to-height increase. 
%%\end{proof}
%
%We now only need to prove that if $M$ is not finite nesting then it has a nesting loop. We assume that $M$ is not finite nesting, so for all $n \in \N$ there exists a $Sigma$-context $s$ and a configuration sequence $w \in \QY^n$ such that $(q_0,h(s),\bot) \to_s w$. We decompose $\Sigma$-context $s$ into a concatenation $s_1\cdot s_2\cdot \dots \cdot s_k$ of atomic $\Sigma$-contexts (i.e.\ context whose hole is at depth $1$) and we get: 
%$(q_0,h(s),\bot) \to_{s_1} x_{1,1} \dots x_{1,m_1} \to_{s_2} \dots \to_{s_m} x_{k,1} \dots x_{k,m_k}$ with $x_{i,j} \in \QY$ for all $i,j$ and $m_k = n$. 
%We can represent this structure of nested state calls as a \emph{configuration tree}:
%%tikz picture: configuration tree: $(q_0,h(s),\bot) \to_{s_1} x_{1,1} \dots x_{1,m_1} \to_{s_2} \dots \to_{s_m} x_{k,1} \dots x_{k,m_k}$
%%% Figure environment removed
%
%We say that a configuration tree is a $k$-comb if it is of the form:
%%tikz picture: $k$-comb
%\begin{center}
%\begin{tikzpicture}
%\node {$(q_0,p_0,\bot)$} [sibling distance = 12mm, level distance = 9mm]
%child {node {$x_{1,1}$}
%	child {node {$x_{2,1}$}
%		child {node {$x_{3,1}$}
%			child {node[rotate=57] {$\dots$}
%				child {node {$x_{k,1}$}
%				}
%				child[missing]
%			}
%			child[missing]
%		}
%		child[missing]
%	}
%	child {node {$x_{2,2}$}
%		child {node {$x_{3,2}$}
%			child {node[rotate=57] {$\dots$}
%				child {node {$x_{k,2}$}
%				}
%				child[missing]
%			}
%			child[missing]
%		}
%		child {node {$x_{3,3}$}
%			child {node[rotate=57] {$\dots$}
%			}
%			child {node[rotate=-57] {$\dots$}
%				child {node {$x_{k,k-1}$}
%				}
%				child {node {$x_{k,k}$}
%				}
%			}
%		}
%	}
%}
%;
%\node at (0,-4.5) {$\dots$};
%
%%other version:
%%\node {$(q_0,p_0,\bot)$} [sibling distance = 12mm, level distance = 10mm]
%%child {node {$x_{1,1}$}
%%	child {node {$x_{2,1}$}
%%		child {node[rotate=58] {$\dots$}
%%			child {node {$x_{k-1,1}$}
%%				child {node {$x_{k,1}$}
%%				}
%%				child[missing]
%%			}
%%			child[missing]
%%		}
%%		child[missing]
%%	}
%%	child {node {$x_{2,2}$}
%%		child {node[rotate=58] {$\dots$}
%%			child {node {$x_{k-1,2}$}
%%				child {node {$x_{k,2}$}
%%				}
%%				child[missing]
%%			}
%%			child[missing]
%%		}
%%		child {node[rotate=-58] {$\dots$}
%%			child[missing]
%%			child {node {$x_{k-1,k-1}$}
%%				child {node {$x_{k,k-1}$}
%%				}
%%				child {node {$x_{k,k}$}
%%				}
%%			}
%%		}
%%	}
%%}
%%;
%%\node at (0,-5) {$\dots$};
%\end{tikzpicture}
%\end{center}
%\vspace{-2mm}
%
%In configuration trees obtained by decomposing a $\Sigma$-context into atomic contexts, nodes have their arity bound by the nesting on the right-hand-side of rules of $M$. Noting this bound $B$, and for all $k \in \N$, any configuration tree with $B^{k+1}$ leafs and obtained by atomic decomposition can be pruned into a $k$-comb. 
%
%Formally, we prove this by induction on $k$. The allowed pruning operations are: $(a)$ remove a node and all its descendants, $(b)$ remove all nodes at a given depth by merging them with their parent node. Note that the nesting statements represented by the trees remain true when pruned. Also, pruned trees preserve the property: for all depth $i<k$, noting $x_{i,1}, \dots, x_{i,m_i}$ the nodes at depth $i$ and $x_{i+1,1}, \dots, x_{i+1,m_{i+1}}$ the nodes at depth $i+1$, there is a $\Sigma$-context $s_i$ such that:
%\vspace{-2mm}
%
%\[x_{i,1} \dots x_{i,m_i} \to_{s_i} x_{i+1,1} \dots x_{i+1,m_{i+1}}\]
%\vspace{-3mm}
%
%We now argue that by taking a big enough $k$-comb we ensure the existence of a nesting loop. We take $k= |\QY|.2^{|\QY|} +1$. Such a $k$-comb must have two depths $i,j$ with $i<j\leq k$ which share the same: 
%\begin{enumerate}
%\item configuration $(q_1,p,y_i) = x_{i,i} = x_{j,j}$ (the configuration on the spine of the comb, i.e.\ the right-most node at depth $i$ and $j$), 
%\item set of configurations $S= \{x_{i,\ell}\}_{0\leq \ell \leq i} = \{x_{j,\ell}\}_{0\leq \ell \leq j}$ (the sets of all configurations appearing at depth $i$ and $j$). 
%\end{enumerate}
%%Note that $k= |\QY|.2^{|\QY|/ |P|} +1$, where $|P|$ is the number of look-ahead states, would be enough because nodes of a same depth have the same $p$, but when adapting this proof to point~(2) we would need $k= |\QY|.2^{|\QY|} +1$. 
%So we have a $\Sigma$-context $s$ such that $(q_1,p,y_i) \to_s w_3 (q_1,p,y_i)$ and $w_1 \to_s w_2$ with $w_1, w_2, w_3 \in S^*$ and all configuration $x \in S \setminus \{(q_1,p,y_i)\}$ appears at least once in $w_1$ and once in $w_2 w_3$: 
%% tikzpicture of the loop s
%\begin{center}
%\begin{tikzpicture}
%\node at (-2,0) {$\phantom{B}$} [sibling distance = 12mm, level distance = 9mm]
%child {node[rotate=55] {$\dots$}
%	child {node {$\phantom{B}$}}
%	child[missing]
%}
%child[missing]
%;
%
%\node {$\phantom{B}$} [sibling distance = 12mm, level distance = 9mm]
%child {node[rotate=55] {$\dots$}
%	child {node {$\phantom{B}$}}
%	child[missing]
%}
%child[missing]
%;
%
%\node at (1,0) {$(q_1,p,y_i)$} [sibling distance = 12mm, level distance = 9mm]
%child {node[rotate=55] {$\dots$}
%	child {node {$\phantom{B}$}}
%	child[missing]
%}
%child {node[rotate=-55] {$\dots$}
%	child {node {$\phantom{B}$}}
%	child {node {$(q_1,p,y_i)$}}
%}
%;
%
%\newcommand{\blobby}[5]{ % #1: radius, #2: x1-coordinate, #3: x2-coordinate, #4: y-coordinate, #5: name
%	\draw (#2,#4 + #1) arc [start angle=90, end angle=270, radius=#1];
%	\draw (#3,#4 - #1) arc [start angle=-90, end angle=90, radius=#1];
%	\draw (#2,#4 + #1) -- (#3,#4 + #1);
%	\draw (#2,#4 - #1) -- (#3,#4 - #1);
%	\node at ($(#2,#4)!0.5!(#3,#4)$) {#5};
%}
%
%\blobby{0.23}{-2}{0}{0}{$w_1$}
%\blobby{0.23}{-3.2}{-1.2}{-1.8}{$w_2$}
%\blobby{0.23}{-0.2}{1}{-1.8}{$w_3$}
%
%\node at (-1.55,-0.9) {$\dots$};
%\end{tikzpicture}
%\end{center}
%
%Let $x_0 \in S$ be a configuration in $w_3$ (we can do so because $|w_3| = j-i \geq 1$). 
%Because $w_1 \to_s w_2$, we know that each configuration $x_k \in S$ has a $x_{k+1} \in S$ such that $x_k \to_s x_{k+1}$. Because $S \subseteq \QY$ is finite, there must be $n, m \in \N$ such that $x_{n} = x_{n+m}$ (with $m \geq 1$), so $x_n \to_{s^m} x_n$. Also $(q_1,p,y_i) \to_s (q_1,p,y_i)\, x_0$ and $x_0 \to_{s^n} x_n$, so $(q_1,p,y_i) \to_{s^{n+1}} (q_1,p,y_i)\, x_n$ and, for all $m' \in \N$: $(q_1,p,y_i) \to_{s^{m'}} (q_1,p,y_i) \to_{s^{n+1}} (q_1,p,y_i)\, x_n$. Finally, for $s_\ell = s^{m(n+1)}$, we have $(q_1,p,y_i) \to_{s_\ell} (q_1,p,y_i)\, x_n$ and $x_n \to_{s_\ell} x_n$. So we have a \emph{nesting loop}. 
%
%In conclusion, if $M$ is not finite nesting, then it is \emph{nested input pumpable}, and so it does not have linear size-to-height increase. 
%\vspace{2mm}

%The proof of Statement~(2) can be given in a very similar way as for~(1), here we only outline the changes to the notations which allow to adapt the proof of~(1) to~(2). 
%First we replace $\Sigma$-contexts with regular contexts (i.e.\ elements of the set $T_\Sigma(P)$) containing possibly several leafs in $P$. The rest of the notational changes are consequences of this change. 
%For $s \in T_\Sigma(P)$, the notation $(q,p,y) \to_s (q_1,p_1,y_{k_1}) \dots (q_n,p_n,y_{k_n})$ means that calls to states $q_1, \dots, q_n$ are nested along a path in $M_q(s)$, but now these calls can be on different leafs of $s$. This means that, when concatenating contexts, we use $s(s_1, \dots, s_m)$ instead of $s\cdot s'$. 
%
%For~(2), similarly to nesting loops for~(1), we define a \emph{yield nesting loop} as given by a pair of contexts $s_1, s_2 \in T_\Sigma(P)$, and configurations $(q_1,p_1,y_i), (q_2,p_2,y_j) \in \QY$ such that: $\<q_1,p_1>$ is reachable, $(q_1,p_1,y_i) \to_{s_1} (q_2,p_2,y_j) (q_1,p_1,y_i)$, and $(q_2,p_2,y_j) \to_{s_2} (q_2,p_2,y_j)$. 
%We say that $M$ is \emph{yield nested input pumpable} when it has either a \emph{yield nesting loop} or a \emph{nesting loop}. 
%Note that the existence of either of these loops falsifies the linear height increase property. 
%To prove~(2) we prove that infinite yield nesting implies the existence of either a yield nesting loop or a nesting loop. That proof works similarly to~(1): existence of $k$-combs for any $k$ (but with the new definition of $\to_s$), find a repeating configuration $(q_1,p_1,y_i)$ on the spine of the comb, pump the loop enough times that a configuration $(q_2,p_2,y_j)$ comes onto itself. Note that if the nested calls to $(q_1,p_1,y_i)$ and $(q_2,p_2,y_j)$ in $(q_1,p_1,y_i) \to_{s_1} (q_2,p_2,y_j)\, (q_1,p_1,y_i)$ are on the same leaf in $s_1$, then we get a nesting loop (otherwise we get a yield nesting loop). 

%
%%Paul's version
%Because $M$ is depth-proper, for each $q \in Q^{(m)}, i \in [m], p \in P$ and 
%$n \in \N$, there is $t \in L_p$ such that $\he{\lcop{M_q(t)}_{\{y_i\}}} > n$. 
%We can ensure the existence of a productive loop in $t$ by taking 
%$n = 1+ \text{mhr} . k^{|Q| . |P|}$ where $k$ is the maximum arity of input trees. 
%Noting $t_j$ the tree where the loop is pumped $j$ times, we get 
%\[\he{\lcop{M_q(t_j)}_{\{y_i\}}} > c_s.|t_j| ~~~~\text{ and }~~~~ 
%\he{\lcop{M_q(t_j)}_{\{y_i\}}} > c_h.\he{t_j}\] 
%for all $j \geq 1$, 
%where $c_s$ and $c_h$ are strictly positive constants that depend on $M, q, i$ and $p$. 
%We can make $c_s$ and $c_h$ only dependent on $M$ by taking their maximum over 
%the possible $q \in Q^{(m)}, i\in [m]$ and $p \in P$. 
%
%To prove~(1), assume that $M$ is not fnest.
%We will show that this implies that $M$ does not have LSHI.
%For all $n \in \N$, there is $s \in T_\Sigma(P)$ with one occurrence of $P$ 
%such that there are at least $n$ nested occurrences of $P$ in $\he{\hat{M}(s)}$. 
%By taking $n$ big enough, we can get any number of nested calls of 
%\emph{the same state} $q$, nested in \emph{the same parameter} $y_i$ of $q$. 
%For all $n \in \N$, we note $s_{n} \in T_\Sigma(P)$ the tree with one occurrence of $P$ 
%such that there are at least $n$ occurrences of 
%$\langle q,p \rangle$ nested along parameter $y_i$ in $\he{\hat{M}(s)}$, 
%for some $q \in Q^{(m)}, i \in [m]$ and $p \in P$. We note $u_n$ such that $s_n/u_n \in P$, 
%and $t_{n,j} = s_n[u_n \leftarrow t_j]$ for all $n,j \in \N$. 
%For each $n$, there is $j_n$ such that $|t_{j_n}| > |s_n|$, so $|t_{j_n}| > |t_{n,j_n}|/2$. 
%Then, for all $n \in \N$:
%\[\he{M(t_{n,j_n})} \geq n.\he{\lcop{M_q(t_{j_n})}_{\{y_i\}}} > n.c_s.|t_{j_n}| > n.c_s.|t_{n,j_n}|/2 \]
%If $M$ had LSHI then there would be a constant $c$ such that \\
%$c > \he{M(t_{n,j_n})}/|t_{n,j_n}| > n.c_s/2$ for all $n \in \N$. 
%This is a contradiction, so $M$ does not have LSHI. 
%
%To prove~(2), assume that $M$ is not fynest.
%We will show that this implies that $M$ does not have LHI.
%For all $n \in \N$, there is $s \in T_\Sigma(P)$ 
%such that there are at least $n$ nested occurrences of $P$ in $\he{\hat{M}(s)}$. 
%By taking $n$ big enough, we can get any number of nested calls of 
%\emph{the same state} $q$, nested in \emph{the same parameter} $y_i$ of $q$, 
%applied to \emph{the same look-ahead state} $p$. 
%For all $n \in \N$, we note $s_{n} \in T_\Sigma(P)$ the tree 
%such that there are at least $n$ occurrences of 
%$\langle q,p \rangle$ nested along parameter $y_i$ in $\he{\hat{M}(s)}$, 
%for some $q \in Q^{(m)}, i \in [m]$ and $p \in P$. 
%We note $t_{n,j} = s_n[p \leftarrow t_j]$ for all $n,j \in \N$. 
%For each $n$, there is $j_n$ such that $\he{t_{j_n}} > \he{s_n}$, so $\he{t_{j_n}} > \he{t_{n,j_n}}/2$. 
%Then, for all $n \in \N$:
%\[\he{M(t_{n,j_n})} \geq n.\he{\lcop{M_q(t_{j_n})}_{\{y_i\}}} > n.c_h.\he{t_{j_n}} > n.c_h.\he{t_{n,j_n}}/2 \]
%If $M$ had LHI then there would be a constant $c$ such that \\
%$c > \he{M(t_{n,j_n})}/\he{t_{n,j_n}} > n.c_h/2$, for all $n \in \N$.
%This is a contradiction, so $M$ does not have LHI. 
%
%\qed
\end{proof}

From Theorem~\ref{th:proper} and Lemmas~\ref{lm:decidable},~\ref{lm:easy}, and~\ref{lm:nest} we obtain
our following main theorem.

\begin{theorem}
Let $M$ be an mttr. 
Then 
(1)~it is decidable whether or not $M$ has linear size-to-height increase
(2)~it is decidable whether or not $M$ is linear height-increase.
\end{theorem}
