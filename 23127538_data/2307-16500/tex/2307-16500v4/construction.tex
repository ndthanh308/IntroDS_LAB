%!TEX root = main.tex
\section{Depth Proper Normal Form}

The depth proper normal form requires that each parameter of
each state $q$ occurs at unbounded depth in the output trees of that state 
(for each given look-ahead state $p$ such that $(q,p)$ is reachable).
Formally, let $q$ be a state of rank $m\geq 1$, $j\in[m]$, and
$p\in P$. 
If $(q,p)$ is reachable, then 
for every natural number $n$ there must exist an input tree $s_n \in L_p$ such that
$y_j$ occurs at depth $>n$ in the tree $M_q(s_n)$.
Conversely, we say that parameter $y_j$ is \emph{depth-bounded} for $q$ and $p$ 
if there exists an $n$ for which no such input tree $s_n \in L_p$ exists;
more generally, we say that $Z\subseteq Y_m$ is
\emph{depth-bounded} for $q$ and $p$, if each $y\in Z$ is depth-bounded for $q$ and $p$.

If $Z$ is depth-bounded for $q$ and $p$, then there are only finitely many output paths
in the trees in $M_q(L_p)$ under which the parameters from $Z$ occur.
The \emph{$Z$-skeleton} of an arbitrary tree $t$ is obtained from $t$
by replacing each top-most node $u$ such that $t/u$ does not contain any
occurrence of a parameter from $Z$ by some symbol. 
Clearly, $Z$ is depth-bounded for $q$ and $p$ if and only if the $Z$-skeleta of
all trees in $M_q(L_p)$ form a finite set.

Let $\Delta$ be an arbitrary ranked alphabet, $m\geq 1$, 
$t\in T_\Delta(Y_m)$, and $Z\subseteq Y_m$.
Let us write \(\paras{t}\subseteq{Y_m}\) for the set of parameters 
occurring in $t$.
Let us now be more specific as to which symbols replace the top-most nodes $u$
of $t$ such that $\paras{t/u}\cap Z=\emptyset$. Since in our construction later
we will want to obtain a transducer that is nondeleting, it will be helpful
to know which parameters appear in a given deleted tree. Therefore we replace
such nodes $u$ by the set $\paras{t/u}$. 
We denote by $\lcop{t}_Z$ the $Z$-skeleton of $t$ and define it inductively as follows
(where $\delta\in\Delta$):
\begin{align*}
\lcop{t}_Z &= 
\begin{cases}
t &
\text{if \(t\in Z\)}
\\
\delta(\lcop{t_1}_Z,\dots,\lcop{t_n}_Z) &
\text{if \(\paras{t}\cap Z\ne\emptyset\) and \(t = \delta(t_1,\dots,t_n)\)}
\\
\paras{t} & \text{if \(\paras{t}\cap Z = \emptyset\)}.
\end{cases}
\end{align*}

The definition of $\lcop{t}_Z$ is extended to sets $L$ of trees
as $\lcop{L}_Z=\{\lcop{t}_Z\mid t\in L\}$.
We call \emph{$Y$-nodes} the nodes $u$ in $V(\lcop{t}_Z)$ such 
that $\lcop{t}_Z/u=Z'\subseteq Y_m$. We denote by 
$\mathcal{U}(\lcop{t}_Z)$ the set of $Y$-nodes on $\lcop{t}_Z$. 
The notion of parameters in a tree naturally extends to $Z$-skeleta with, 
for a $Y$-node labeled $Z'$: $\paras{Z'} = Z'$. 

\begin{lemma}\label{lm:nd}
Let $\Delta$ be a ranked alphabet, $m\geq 1$, $Z\subseteq Y_m$, and
$t\in T_\Delta(Y_m)$.
(1)~$t=\lcop{t}_Z[u\leftarrow t/u\mid u\in\mathcal{U}(\lcop{t}_Z)]$.
(2)~$\paras{\lcop{t}_Z} = \paras{t}$. 
%(3)~If $y\in Y_m$ occurs in $t$, then 
%there exists a leaf $u\in V(\lcop{t}_Z)$
%such that either $\lcop{t}_Z/u=y$ or 
%$\lcop{t}_Z/u=Z'\subseteq Y_m$ and $y\in Z'$.
\end{lemma}
\begin{proof}
The proof is by induction on the structure of $\lcop{t}_Z$.
Let us denote the substitution $[u\leftarrow t/u\mid u\in\mathcal{U}(\lcop{t}_Z)]$
by $[t]$.
%
There are three cases.

Case 1: $t = y\in Z$. Then $\mathcal{U}(\lcop{t}_Z)=\emptyset$.
Hence $\lcop{t}_Z[t]=\lcop{y}_Z$. The latter equals $y=t$ by 
the definition of $\lcop{.}_Z$. Thus~(1)~holds. Also 
$\paras{\lcop{t}_Z}=\{y\}=\paras{t}$ and so~(2)~holds. 
%the definition of $\lcop{.}_Z$. Thus~(1)~holds and~(2)~holds for $u=\epsilon$.

Case 2: $\paras{t} \cap Z = \emptyset$. Then $\lcop{t}_Z = \paras{t} = Z' \subseteq Y_m$ 
by the definition of $\lcop{.}_Z$, and $\mathcal{U}(\lcop{t}_Z)=\{ \epsilon \}$. Hence
$\lcop{t}_Z[t] = \lcop{t}_Z[\epsilon\leftarrow t/\epsilon = t] = t$ which proves~(1). 
Again~(2)~holds 
because $\paras{\lcop{t}_Z}=\paras{\paras{t}}=\paras{t}$. 
%for $u=\varepsilon$ for each $y \in \paras{t}$ i.e.\ $y$ occurring in $t$.

Case 3: \(\paras{t}\cap Z\ne\emptyset\). Then $t=\delta(t_1,\dots,t_n)$, 
$\delta\in\Delta^{(n)}$, $n\geq 0$,
and $t_1,\dots, t_n\in T_\Delta(Y_m)$.
To show~(1) we obtain from the definition of $\lcop{t}_Z$ that
\[
\lcop{t}_Z[t]=\delta(\lcop{t_1}_Z,\dots,\lcop{t_n}_Z)[t]
= \delta(\lcop{t_1}_Z[t_1],\dots,\lcop{t_n}_Z[t_n]),
\]
where for $i\in[n]$, $[t_i]$ denotes the substitution
$[u\leftarrow t_i/u\mid u\in\mathcal{U}(\lcop{t_i}_Z)]$.
By induction the latter equals $\delta(t_1,\dots,t_n)=t$.
Finally~(2)~is implied by the induction hypothesis:
$\paras{\lcop{t}_Z}=\bigcup_{j\in[n]} \paras{\lcop{t_j}_Z}
=\bigcup_{j\in[n]} \paras{t_j}= \paras{t}$
%To show~(2), assume $y\in Y_m$ occurs in $t$.
%Thus, $n\geq 1$ and there exists an $i\in[n]$
%such that $y$ occurs in $t_i$.
%By induction of Statement~(2), there exists a leaf $u$ of $t_i$ such that 
%Statement~(2) holds (for $u$ and $t_i$); therefore Statement~(2) holds
%for $iu$ and $t$.
\qed


%%Sebastian's version:
%The proof is by induction on the structure of $t$.
%Let us denote the substitution $[u\leftarrow t/u\mid u\in\mathcal{U}(\lcop{t}_Z)]$
%by $[t]$.
%%
%We first consider $t=y\in Y$. There are two cases.
%
%Case 1: $y\in Z$. Then $\mathcal{U}(\lcop{t}_Z)=\emptyset$.
%Hence $\lcop{t}_Z[t]=\lcop{y}_Z$. The latter equals $y=t$ by 
%the definition of $\lcop{.}_Z$. Thus~(1)~holds and~(2)~holds for $u=\epsilon$.
%
%Case 2: $y\not\in Z$. Then $\mathcal{U}(\lcop{t}_Z)=\{ \epsilon \}$ and hence
%$\lcop{t}_Z[t] = \lcop{t}_Z[\epsilon\leftarrow t/\epsilon = t] = t$ which proves~(1).
%Moreover, $\lcop{t}_Z=\{y_j\}$ by the definition of $\lcop{.}_Z$,
%i.e., again~(2)~holds for $u=\varepsilon$.
%
%Now consider $t=\delta(t_1,\dots,t_n)$, $\delta\in\Delta^{(n)}$, $n\geq 0$,
%and $t_1,\dots, t_n\in T_\Delta(Y_m)$.
%To show~(1) we obtain from the definition $\lcop{t}_Z$ that
%\[
%\lcop{t}_Z[t]=\delta(\lcop{t_1}_Z,\dots,\lcop{t_n}_Z)[t]
%= \delta(\lcop{t_1}_Z[t_1],\dots,\lcop{t_n}_Z[t_n]),
%\]
%where for $i\in[n]$, $[t_i]$ denotes the substitution
%$[u\leftarrow t_i/u\mid u\in\mathcal{U}(\lcop{t_i}_Z)]$.
%By induction the latter equals $\delta(t_1,\dots,t_n)=t$.
%To show~(2) we assume that $y_j$ occurs in $t$ for some $j\in[m]$.
%Thus, $n\geq 1$ and there exists an $i\in[n]$
%such that $y_j$ occurs in $t_i$.
%By the induction for Statement~(2), there exists a leaf $u$ of $t_i$ such that 
%the Statement~(2) holds (for $u$ and $t_i$), and therefore Statement~(2) holds
%for $iu$ and $t$.
\end{proof}

Finally, we define depth properness for mtts with look-ahead. 

\begin{definition} \label{df_depth_proper}
The mttr $M=(Q,P,\Sigma,\Delta,q_0,R,h)$ is in \emph{depth proper normal form}
(or, synonymously, $M$ is \emph{depth proper})
if for every $q\in Q^{(m)}$, $m\geq 1$, and $p\in P$ it holds that
if $(q,p)$ is reachable, then  
$\lcop{M_q(L_p)}_{\{y_j\}}$ is infinite for all $j\in[m]$.
\end{definition}

From now on we will want to make use of the following definitions:
\[
\begin{array}{lcl}
F_p &=& \{ q\in Q^{(m)}\mid \exists j\in[m], \lcop{M_q(L_p)}_{\{y_j\}}
\text{ is finite}\}\\
Y(q,p) &=& \{ y_j\mid j\in[\text{rank}_Q(q)]\text{ such that }
\lcop{M_q(L_p)}_{\{y_j\}}\text{ is finite}\}
\end{array}
\]

It should be clear that $\lcop{M_q(L_p)}_{Y(q,p)}$ is finite for every $q$ and $p$,
as shown in the next lemma.

\begin{lemma}\label{lm:lcop_finite}
Let $M$ be an mtt, $q$ a state of $M$, and $p$ a look-ahead state of $M$
such that $Y(q,p)\not=\emptyset$.
Then $\lcop{M_q(L_p)}_{Y(q,p)}$ is finite.
\end{lemma}
\begin{proof}
Assume that $Y_{q,p}=\{y_{j_1},\dots,y_{j_n}\}$ where $n\geq 1$.
It follows from the definition of $Y(q,p)$ that for each $i\in[n]$ there exists
a number $d_i$ such that $y_{j_i}$ occurs at depth $\leq d_i$ in 
any tree in $M_q(L_p)$. Let $d$ be the maximum of all numbers in
$\{ d_1,\dots,d_n\}$. Then every parameter in $Y(q,p)$ occurs at depth $\leq d$ in
any tree in $M_q(L_p)$. It follows from the definition of $U=\lcop{M_q(L_p)}_{Y(q,p)}$
that every node of a tree in $U$ has depth $\leq d$. Thus, $U$ is finite.
\qed
\end{proof}

Before we proceed with the construction of the depth proper normal form,
we want to be able to decide whether or not a given mttr is depth proper.

%\newpage

\subsection{Deciding whether or not a given Mttr is Depth Proper}

In this section we prove that for a given mttr it is decidable whether or
not it is depth proper.
Moreover, we present a technical lemma giving properties of parameters and 
skeleta of state-calls appearing in right-hand-sides of rules. 
%states that if a certain
%parameter is depth proper and appears nested in the right-hand side of a rule,
%then a parameter of the nested state is depth proper as well. 

\begin{lemma}\label{lm:dec}
Let $M=(Q,P,\Sigma,\Delta,q_0,R,h)$ be an mttr and let
$q\in Q^{(m)}$, $m\geq 1$, $j\in[m]$, and $p\in P$.
It is decidable whether or not 
$\lcop{M_q(L_p)}_{\{y_j\}}$ is finite.
In case of finiteness, 
$\lcop{M_q(L_p)}_{\{y_j\}}$ can be constructed.
\end{lemma}
\begin{proof}
Let $Z=\{y_j\}$.
We now consider symbols in $Y_m$ as rank zero symbols.
It is straightforward to construct a  top-down tree transducer with look-ahead $M_Z$ which outputs
$\lcop{t}_Z$ for input trees $t\in T_{\Delta\cup Y_m}$; note that
$T_{\Delta\cup Y_m}=T_\Delta(Y_m)$ because symbols in $Y_m$ are now considered
as symbols of rank zero.
The transducer $M_Z$ computes in its look-ahead $h'$ the set of parameters of
the input tree, i.e.\ $h'(t)=\paras{t}$ for every $t\in T_{\Delta\cup Y_m}$. 
The transducer $M_Z$ (which consists of a single state only) outputs $\paras{t}$
as soon as $\paras{t}\cap Z=\emptyset$.
%The details can be found in the Appendix.

Formally, $M_Z = (\{q_1^{(0)}\},P',\Delta\cup Y_m,\Delta',q_1,R',h')$ where
$P'={\cal P}(Y_m)$ and
$\Delta'=\Delta\cup \{ S^{(0)}\mid S\in P'\}\cup Y_m$.
For $y\in Y_m$ let $h'_y()=\{y\}$ and for
$a\in\Delta^{(0)}$ let $h'_a()=\emptyset$.
Further, for $\delta\in\Delta^{(k)}$, $k\geq 1$, and 
$S_1,\dots,S_k\in P'$ let 
$h'_\delta(S_1,\dots,S_k)=\bigcup_{i\in[k]}S_i$.
For $a\in\Delta^{(0)}$ let the rule
$\< q_1,a>\to\emptyset$ be in $R$.
Let $y\in Y_m$.
If $y\in Z$ then let the rule
$\< q_1,y>\to y$ be in $R$ and otherwise let the rule
$\< q_1,y>\to \{y\}$ be in $R$.
For $\delta\in\Delta^{(k)}$ with $k\geq 1$ 
and $S_1,\dots,S_k\in P'$
we define the rule
$\< q_1,\delta(x_1:S_1,\dots,x_k:S_k)>\to\zeta$ where $\zeta$ is defined as:
\[
\begin{array}{lcll}
  \zeta & = &
\left\{ 
  \begin{array}{ll}
  S & \text{if }S= (\bigcup_{i\in[k]} S_i)\cap Z=\emptyset\\
  \delta(\< q_1,x_1>,\dots,\<q_1,x_k>) & \text{otherwise.}
  \end{array}
\right. 
\end{array}
\]

\noindent
\textbf{Claim.}\quad
For every $t\in T_{\Delta\cup Y_m}$, $h'(t)=\paras{t}$ and
$M_Z(t)=\lcop{t}_Z$.

\medskip

It is straightforward to prove this claim by induction on the structure of $t$.
By the Claim,
$M_{\{y_j\}}(M_q(L_p)) = \lcop{M_q(L_p)}_{\{y_j\}}$.
The tree language $L_p$ can be represented by a partial nondeterministic
top-down tree transducer that realizes the identity on trees in $L_p$
(and is undefined otherwise; its rules are obtained by reading the
definitions of $h$ from right to left). 
In this way $M_{\{y_j\}}(M_q(L_p))$ is represented
so that its finiteness is decidable by Proposition~\ref{prop:finite}
(and in case of finiteness the set can be constructed).
\qed
\end{proof}

Since for a pair $(q,p)$ it is decidable whether or not it is reachable
(see Section~\ref{sect:mtt}), Lemma~\ref{lm:dec} implies that
it is decidable whether or not a given mttr is depth proper.
We conclude this section with a small lemma showing that
the skeleton of the output of an mttr \(M\) can be directly computed
from given a input tree 
by modifying the rules of \(M\).
%given a 
%right-hand-side of rule $t$, the skeleton of $t$ can be computed from the 
%skeleta of state calls appearing in $t$. 
For this lemma we first need to define how to compute second-order 
substitutions of skeleta,
which will be used for the modification of the right-hand sides of rules.
We do so on a \emph{nondeleting} mttr $M$, 
i.e.\ such that states always use all their parameters. 

\begin{definition}\label{def:meta-skeleta}
%Let $M$ be a nondeleting mttr as before. 
  %
Let $\Gamma$ be a ranked alphabet
 and
let $t_1,\dots,t_n\in T_\Gamma(Y)$. 
Let $s\in T_\Gamma(Y_n\cup\mathcal{P}(Y_n))$. 
%
%\begin{itemize}
%\item
%\item Let \(k_i = 0\) for all \(i\in[n]\).
The \emph{special first-order substitution} 
$[y_i \leftarrow t_i \mid i \in [n]]\su$ (for short $[.]\su$) applied to $s$ is
inductively defined as:
\begin{align*}
	s[.]\su &= 
	\begin{cases}
		t_i &
%		\text{if \(s=\gamma_i\) for \(i\in [n]\)}
		\text{if \(s=y_i\) for \(i\in [n]\)}
		\\
		\gamma(s_1[.]\su,\dots,s_k[.]\su) &
%		\text{if \(s = \gamma(s_1,\dots,s_k)\) and }\gamma\not\in\{\gamma_1,\dots,\gamma_n\}
		\text{if \(s = \gamma(s_1,\dots,s_k)\)}
		\\
		\bigcup_{i\in U} \paras{t_i} & \text{if \(s= \{y_i \mid i\in U\} \subseteq Y_n\)
                  for some \(U \subseteq [n]\).}
%		\bigcup_{y_i \in t'} \paras{t_i} & \text{if \(t' \subseteq Y_{m'}\) is a sequence node}.
	\end{cases}
\end{align*}

%For all pairwise distinct symbols $\sigma_1\in\Delta^{(k_1)},\dots, 
%\sigma_n\in\Delta^{(k_n)}$ with $n\geq 1$ and
%$k_1,\dots,k_n\in\mathbb{N}$ and let $t_i$ for $i\in[n]$.
%Let $s\in T_\Delta$.
%Then $s[\![\sigma_i\leftarrow t_i\mid i\in[n]]\!]$ denotes the tree
%that is inductively defined as (abbreviating $[\![\sigma_i\leftarrow t_i\mid i\in[n]]\!]$ by
%$[\![\dots ]\!]$) follows:
%for $s=\sigma(s_1,\dots,s_k)$,
%if $\sigma\not\in\{\sigma_1,\dots,\sigma_n\}$ then $s[\![\dots]\!]=\sigma(s_1[\![\dots]\!],
%\dots,s_k[\![\dots]\!])$ and if $\sigma=\sigma_j$ for some $j\in[n]$ then
%$s[\![\dots]\!]=t_j[y_i\leftarrow s_i[\![\dots]\!]\mid i\in[k]]$.
%\item
Let $\gamma_1^{(k_1)},\dots,\gamma_n^{(k_n)}\in\Gamma$, $n\geq 1$ be pairwise different symbols
%Let $k_i=rank_\Gamma(\gamma_i)$ for $i\in[n]$
and assume now that 
$t_i\in T_\Gamma(Y_{k_i}\cup\mathcal{P}(Y_{k_i}))$ for $i\in[n]$
and that $s\in T_\Gamma(Y_n)$. 
%
The \emph{special second-order substitution} 
$[\![\gamma_i \leftarrow t_i\mid i\in[n]]\!]\sp{}$
(for short $[\![.]\!]\su$) applied to $s$ is
%inductively defined as 
%if $s=\gamma(s_1, \dots, s_k)$
%with $\gamma\not\in\{\gamma_1,\dots,\gamma_n\}$
%then $s[\![.]\!]\su = 
%\gamma(s_1[\![.]\!]\su, \dots, s_k[\![.]\!]\su)$, 
%if $s=y_j$ for $j\in [n]$ then $s[\![.]\!]\su = s$, 
%and if $s=\gamma_i$ for $i\in[n]$ then
%$s[\![.]\!]\su =  t_i
%[y_j \leftarrow t_j[\![.]\!]\su \mid j \in [n]]\su$. 
inductively defined as:
\begin{align*}
s[\![.]\!]\su &=
\begin{cases}
t_i
[y_j \leftarrow s_j[\![.]\!]\su \mid j \in [k_i]]\su
&
\text{if $s=\gamma_i(s_1, \dots, s_{k_i})$ for $i\in[n]$}
\\
\gamma(s_1[\![.]\!]\su, \dots, s_k[\![.]\!]\su)
& \text{if $s=\gamma(s_1, \dots, s_k)$
with $\gamma\not\in\{\gamma_1,\dots,\gamma_n\}$}
\\
s
&
\text{if $s=y_j$ for $j\in [n]$.}
\end{cases}
\end{align*}
%if $s=\gamma(s_1, \dots, s_k)$
%with $\gamma\not\in\{\gamma_1,\dots,\gamma_n\}$
%then $s[\![.]\!]\su = 
%\gamma(s_1[\![.]\!]\su, \dots, s_k[\![.]\!]\su)$, 
%if $s=y_j$ for $j\in [n]$ then $s[\![.]\!]\su = s$, 
%and if $s=\gamma_i$ for $i\in[n]$ then
%$s[\![.]\!]\su =  t_i
%[y_j \leftarrow t_j[\![.]\!]\su \mid j \in [n]]\su$. 

For all sets $Z \subseteq Y_m$ such that no $Y$-node in 
$t[\![.]\!]\su$ intersects $Z$, we define the $Z$-skeleton 
$\lcop{t[\![.]\!]\su}_Z$ of $t[\![.]\!]\su$ inductively as 
before, with a special case for $Y$-nodes: for all $Y$-nodes $S$ we 
have $\lcop{S}_Z = S \subseteq Y_m \setminus Z$. 
%\end{itemize}
\end{definition}

%Old version of the end of the definition of compatibility of rhs
%To define this notion of compatibility, we consider the second-order 
%substitution of a state call (in the right-hand-side of a rule) with the 
%state's skeleton, which induces a 
%first-order substitution of the parameters in the skeleton. We take such 
%first-order substitutions to apply \emph{within} sequence nodes of the skeleton 
%so that, given a sequence node $S=\{y_1,y_3\}$: 
%$S[y_i \leftarrow t_i]_{i\in [3]} = \{t_1,t_3\}$. 
%We assume given an mttr $M$ as before, with the condition that 
%$M$ is \emph{nondeleting}, which 
%means that all parameters of a state are used to build the state's output.
%
%\begin{definition}\label{def:meta-skeleta}
%For all states $q \in Q^{(m)}, q'\in Q^{(m')}$, $\sigma\in\Sigma^{(k)}$, 
%and $p_1,\dots,p_k\in P$, and noting $p=h(\sigma(p_1, \dots, p_k))$ and  
%$t=\text{rhs}_M(q,\sigma,\<p_1,\dots, p_k>)$, any state call $\<q',x_i>$ 
%in $t$ is \emph{compatible} with a set 
%$Z \subseteq Y_m$ if, for all $s_i \in L_{p_i}$, the sequence nodes in 
%$t[\![\<q',x_i>\leftarrow \lcop{M_{q'}(s_i)}_{Y(q',p_i)}]\!]$ contain no 
%parameters from $Z$. 
%\end{definition}

The special first-order substitution is
the same as the normal one
except that it gives special treatment to \(Y\)-nodes
which is replaced by \(Y\)-nodes contains all parameters occurring 
in trees to be substituted for the parameters in the original \(Y\)-nodes.
%
The special second-order substitution is
the same as the normal one
except that the special first-order substitution is applied
for each involved first-order substitution.


\begin{lemma}\label{lm:rhs}
Let $M$ be a nondeleting mttr as before.
Let $q\in Q$, $\sigma\in\Sigma^{(k)}$, and
$p_1,\dots,p_k\in P$. Let $p=h(\sigma(p_1, \dots, p_k))$ and 
$t=\text{rhs}_M(q,\sigma,\<p_1,\dots, p_k>)$. 
Let $s_1 \in L_{p_1}, \dots, s_k \in L_{p_k}$. By $[\![.]\!]^{\$}$ we
denote the substitution $[\![\<q',x_i>\leftarrow \lcop{M_{q'}(s_i)}_{Y(q',p_i)} \mid q'\in Q, 
i \in [k]]\!]^{\$}$ and by $[\![M]\!]$ we denote 
$[\![\<q',x_i>\leftarrow M_{q'}(s_i) \mid q'\in Q, i \in [k]]\!]$.
\begin{enumerate}
\item[(1)] If $y\in Y(q,p)$ and $y$ occurs in $t$ in the $j$-th argument 
of a node $\<q',x_i>$ for $q' \in Q$ and $i\in [k]$, then $y_j \in Y(q',p_i)$. 
%If $y\in Y(q,p)$ and $y$ occurs in  $t_j$ ($j\in[m]$), then $y_j\in Y(q',p_i)$.
\item[(2)] No $Y$-node in $t[\![.]\!]\su$ intersects $Y(q,p)$. 
%Any state call $\<q',x_i>$ in $t$ is compatible with $Y(q,p)$. 
\item[(3)] $\lcop{t[\![.]\!]\su}_{Y(q,p)} = \lcop{t[\![M]\!]}_{Y(q,p)}$
%For all input trees $s_i\in L_{p_i}$, we have: ~~~~
%$\lcop{t[\![\dots]\!]}_{Y(q,p)} = \lcop{t[\![\lcop{\dots}]\!]}_{Y(q,p)}$ \\
%where $[\![\dots]\!]$ denotes $[\![\<q',x_i>\leftarrow M_{q'}(s_i)]\!]$ \\
%and $[\![\lcop{\dots}]\!]$ denotes $[\![\<q',x_i>\leftarrow \lcop{M_{q'}(s_i)}_{Y(q',p_i)} ]\!]$. 
\end{enumerate}
\end{lemma}
\begin{proof}
If some $y_j\notin Y(q',p_i)$ then $\lcop{M_{q'}(L_{p_i})}_{y_j}$ is 
infinite and, if $y$ occurs in $t_j$ ($j\in[m]$), then 
$\lcop{M_{q}(L_{p})}_{y}$ is also infinite and $y\notin Y(q,p)$. 
So~(1)~holds. 

%(2)~is a consequence of~(1).
%Alternative proof of (2):
If $y \in Y(q,p)$ occurs in a $Y$-node of $t[\![.]\!]\su$, 
then it occurs in $t$ in the $j$-th argument of a node $\<q',x_i>$ with 
$y_j \notin Y(q',p_i)$, which contradicts~(1). 
So~(2)~holds. 

%As a consequence of~(1), parameter nodes and inner nodes are identical in 
%$\lcop{t[\![\dots]\!]}_{Y(q,p)}$ and 
%$\lcop{t[\![\lcop{\dots}]\!]}_{Y(q,p)}$. 
%Sequence nodes are also identical as a consequence of Lemma~\ref{lm:nd}(2). 
%Therefore~(3)~holds. 
%Alternatice proof of (3)
%%% Original proof FROM HERE %%%%%%%%%%%%%%%%%%%%%%%%%%%%%%%%%
%Both $\lcop{t[\![M]\!]}_{Y(q,p)}$ and 
%$\lcop{t[\![.]\!]\su}_{Y(q,p)}$ contain three types of nodes: 
%parameter nodes of the form $y\in Y(q,p)$, inner nodes and $Y$-nodes. 
%As a consequence of~(1), paths to parameters nodes are identical in 
%$\lcop{t[\![M]\!]}_{Y(q,p)}$ and $\lcop{t[\![.]\!]\su}_{Y(q,p)}$, 
%and the same is true of the inner nodes along such paths. 
%$Y$-nodes are also identical as a consequence of Lemma~\ref{lm:nd}(2) and 
%of our definition of special second-order substitutions. 
%So~(3)~holds. 
%%% Original proof TO HERE %%%%%%%%%%%%%%%%%%%%%%%%%%%%%%%%%

The statement~(3) is proved by induction on \(t\).
The cases of \(t=y_j\) and \(t=\gamma(t_1,\dots,t_n)\) are easy.
In the case of \(t=\<q',x_i>(t_1,\dots,t_m)\), we have
\begin{align*}
\lcop{t[\![.]\!]\su}_{Y(q,p)}
&=
\lcop{
\lcop{M_{q'}(s_i)}_{Y(q',p_i)}[y_j\leftarrow t_j[\![.]\!]\su\mid j\in[m]]\su
}_{Y(q,p)}
\\&=
\lcop{
\lcop{M_{q'}(s_i)}_{Y(q',p_i)}[y_j\leftarrow t_j[\![M]\!]\mid j\in[m]]\su
}_{Y(q,p)}
\\&=
\lcop{
M_{q'}(s_i)[y_j\leftarrow t_j[\![M]\!]\mid j\in[m]]
}_{Y(q,p)}
\\&=
\lcop{t[\![M]\!]}_{Y(q,p)}\text.
\end{align*}
%where the induction hypothesis and Lemma~\ref{lm:nd}(2) are used.
%
%Another alternatice proof of (3)
%To prove~(3)~we look at the $Y(q,p)$-skeleta of $t[\![\dots]\!]$ and 
%$t[\![\lcop{\dots}]\!]$. Those contain three types of nodes: 
%parameter nodes of the form $y\in Y(q,p)$, inner nodes (which are along 
%the paths to parameter nodes), and sequence nodes. Paths to parameters nodes 
%are identical in $t[\![\dots]\!]$ as in $t[\![\lcop{\dots}]\!]$ and so 
%are the inner nodes along such paths. Sequence nodes are also identical as a 
%consequence of Lemma~\ref{lm:nd}(2). 
\qed
\end{proof}

\input{constr_plus_examples.tex}

\subsection{Correctness Proof and Termination of Iteration}
\label{sect:corr}

Here we prove the correctness of transducer $\pi(M)$ that was defined
in Definition~\ref{df_depth_proper}. Lemma~\ref{lm:corr} establishes
the correctness of the look-ahead, relates the states of $\pi(M)$ to those
of $M$, and shows that the transducer $\pi(M)$ is nondeleting.
The latter is needed, so that the construction of $\pi$ can be carried
out iteratively (recall from Definition~\ref{df_depth_proper} that $M$ is
required to be nondeleting, in order to construct $\pi(M)$).


\begin{lemma}
\label{lm:corr}
Let $M$ be a nondeleting mttr and $N=\pi(M)$ be the mttr of
Definition~\ref{df:pi}, both with the tuples as in that definition. 
Let $s\in T_\Sigma$ with $\hat{h'}(s)=(p,\varphi)$.
Then
\begin{enumerate}
\item[(1)] $p=\hat{h}(s)$,
\item[(2)] $\forall q\in F_p$: $\varphi(q)=\lcop{M_q(s)}_{Y(q,p)}$, 
\item[(3)] $\forall q\in Q$: $N_q(s)=M_q(s)$,
\item[(4)] $\forall q\in F_p$ and $u\in V(t)$ with $t=\varphi(q)$ and
$t/u=\{y_{j_1},\dots,y_{j_n}\}$ with \\
$j_1<\cdots <j_n$:
$N_{[q,p,t,u]}(s)=M_q(s)/u[y_{j_\nu}\leftarrow y_\nu\mid\nu\in[n]]$, and
\item[(5)] the mttr $N$ is nondeleting.
\end{enumerate}
\end{lemma}
\begin{proof}
All the statements are proven by induction on the structure of $s$.
Let $s=\sigma(s_1,\dots,s_k)$ with $\sigma\in\Sigma^{(k)}$,
$k\geq 0$, and $s_1,\dots,s_k\in T_\Sigma$.
For $i\in[k]$ let $\hat{h'}(s_i)=(p_i,\varphi_i)$.
By the definition of $h'$, $p=h_\sigma(p_1,\dots,p_k)$, which 
is equal to $\hat{h}(s)$.
Thus, Statement~(1) holds.
For Statement~(2) let $q\in F_p$: 
Then $\varphi(q)$ is defined as 
$\lcop{\zeta[\![\varphi_i]\!]\su}_{Y(q,p)}$ where 
$\zeta=\text{rhs}_M(q,\sigma,\<p_1,\dots,p_k>)$
and
$[\![\varphi_i]\!]\su$ denotes the special substitution
$[\![ \< q',x_i>\leftarrow \varphi_i(q')\mid q'\in F_{p_i}, i\in[k]]\!]\su$.
By induction, $\lcop{\zeta[\![\varphi_i]\!]\su}_{Y(q,p)}$ equals
$\lcop{\zeta[\![ \< q',x_i>\leftarrow \lcop{M_{q'}(s_i)}_{Y(q',p_i)}
\mid q'\in F_{p_i}, i\in[k]]\!]\su}_{Y(q,p)}$.
By Lemma~\ref{lm:rhs}(3) the latter equals
$\lcop{\zeta[\![ \< q',x_i>\leftarrow M_{q'}(s_i)
\mid q'\in F_{p_i}, i\in[k]]\!]}_{Y(q,p)}
=\lcop{M(s)}_{Y(q,p)}$.


We now prove Statement~(3).
Let $q\in Q$.
Then $N_q(s)=\zeta[\![ . ]\!][\![N]\!]$, 
where 
$\zeta=\text{rhs}_M(q,\sigma,\<p_1,\dots,p_k>)$,
$[\![ . ]\!]$ is the substitution as in the construction, and
$[\![N]\!]=
[\![ \<r,x_i>\leftarrow N_r(s_i)\mid r\in Q',i\in[k] ]\!]$.
%
By the induction hypothesis of Statement~(2), we can replace
$\varphi_i(q')$ by $\lcop{M_{q'}(s_i)}_{Y(q',p_i)}$ in the
substitution $[\![ . ]\!]$. This gives
\begin{multline*}
\zeta
[\![ \<q',x_i>\leftarrow \lcop{M_{q'}(s_i)}_{Y(q',p_i)}
[
u'\leftarrow [q',p_i,\varphi_i(q'),u'](y_{j_1},\dots,y_{j_n})\mid \\
\varphi_i(q')/u'=\{y_{j_1},\dots,y_{j_n}\},
j_1<\cdots <j_n]
\mid q'\in F_{p_i},
i\in[k] ]\!]
[\![ N ]\!].
\end{multline*}
This can be written as 
$\zeta[\![.]\!][\![ H ]\!] [\![ Q ]\!]$,
where
$[\![ H ]\!]=[\![\< q',x_i>\leftarrow N_{q'}(s_i)\mid q'\in H,i\in[k] ]\!]$
and
$[\![ Q ]\!]  = 
[\![\< q',x_i>\leftarrow N_{q'}(s_i)\mid q'\in (Q \setminus F_{p_i}),i\in[k] ]\!]$.
%
By induction of Statement~(4) the substitution $[\![ H]\!]$ replaces the 
subtree $[q',p_i,\varphi_i(q'),u'](y_{j_1},\dots,y_{j_n})$ by the tree 
$M_{q'}(s_i)/u[y_{j_\nu}\leftarrow y_\nu\mid\nu\in[n]]
[y_\nu\leftarrow y_{j_\nu}\mid\nu\in[n]]=M_{q'}(s_i)/u$. 
Thus we obtain:
\begin{multline*}
\zeta
[\![ \<q',x_i>\leftarrow \lcop{M_{q'}(s_i)}_{Y(q',p_i)}
[
u'\leftarrow M_{q'}(s_i)/u'\mid 
u'\in\mathcal{U}(\lcop{M_{q'}(s_i)}_{Y(q',p_i)})]\\
\mid q'\in F_{p_i},
i\in[k] ]\!] 
[\![ Q ]\!]
\end{multline*}
By Lemma~\ref{lm:nd} (for $Z=Y(q',p_i)$ and $t=M_{q'}(s_i)$)
the tree on the right of the arrow
in the leftmost second-order substitution equals $M_{q'}(s_i)$.
We have:
\[
\zeta
[\![ \<q',x_i>\leftarrow M_{q'}(s_i)\mid q'\in F_{p_i},i\in[k] ]\!] 
[\![\< q',x_i>\leftarrow N_{q'}(s_i)\mid q'\in Q \setminus F_{p_i},i\in[k] ]\!].
\]
By induction of Statement~(3), $N_{q'}(s_i)=M_{q'}(s_i)$ for
$q\in Q \setminus F_{p_i}$. This gives us exactly $M_q(s)$, by the definition
of the semantics of mttrs. 
Thus,
\begin{equation}\label{eq:MN}
N_q(s)=\zeta[\![ . ]\!][\![N]\!]=M_q(s).
\end{equation}
This concludes the proof of Statement~(3).


We now prove Statement~(4).
Let $q\in F_p$ and $u\in V(t)$ with $t=\varphi(q)$ 
and $t/u\subseteq Y$.
By the definition of the rules for the helper states, 
$N_{[q,p,t,u]}(s)=(\zeta[\![ . ]\!])/u[\![ N ]\!][y]$
where $t/u=\{y_{j_1},\dots,y_{j_n}\}$, $j_1<\cdots <j_n$, and 
$[y]=[y_{j_\nu}\leftarrow y_\nu\mid\nu\in[n]]$.
It follows from Lemma~\ref{lm:rhs}(1) that if $\<q',x_i>$ occurs in 
$\zeta=\text{rhs}_M(q,\sigma,\langle p_1,$ $\dots,p_k\rangle)$
and $q\in F_p$, then $q'\not\in Q \setminus F_{p_i}$.
Hence, every proper ancestor
$v$ of $u$ is labeled by a symbol in $\Delta$, i.e.,
$(\zeta[\![.]\!][y])[v]\in\Delta$. 
This implies that we can move the ``$/u$'' operation of 
taking the subtree at node $u$
to the right (after the application of the substitution $[\![ N ]\!]$)
in the above displayed formula.
We obtain 
$\zeta[\![ . ]\!][\![ N ]\!]/u[y]$.
By the right equation in Formula~\ref{eq:MN}, this equals
$M_q(s)/u[y]$.

To prove Statement~(5), let $q\in Q^{(m)}$, $m\geq 0$.
Then 
\[
\zeta'=\text{rhs}_N(q,\sigma,\< (p_1,\varphi_1),\dots,(p_k,\varphi_k)>)=
\zeta[\![.]\!],
\]
where 
$\zeta=\text{rhs}_M(q,\sigma,\<p_1,\dots,p_k>)$ and
$[\![.]\!]$ is as before. 
By Statement~(2), $[\![.]\!]$ substitutes occurrences of 
$\<q',x_i>$ with $i\in[k]$ and $q'\in F_{p_i}$ by 
the tree $\lcop{M_{q'}(s_i)}_{Y(q',p_i)}$ in which leaves
labeled by $Z\subseteq Y_m$ are replaced by $\<q_H,x_i>(y_{j_1},\dots,
y_{j_n})$ with $Z=\{y_{j_1},\dots,y_{j_n}\}$.
By Lemma~\ref{lm:nd}(2) this implies that 
$y_j$ occurs in $\zeta'$ for each $j\in[m]$.
\qed
\end{proof}

%!TEX root = main.tex

% macros (will be moved later)

%\section{Termination of the procedure}

We are now ready to show that the iteration of the construction $\pi(M)$ will
terminate with a transducer that is depth-proper.
We do not present a firm bound but use a K{\"o}nigs Lemma-like argument
for termination.

\begin{lemma}\label{lm:term}
Let \(M\) be a nondeleting mttr with improper states of rank at most \(m\).
There exists \(n\) such that 
any reachable improper state of \(\pi^n(M)\) has rank at most \((m-1)\).
\end{lemma}
%
\begin{proof}
Let \(Q_0\) be the set of states of \(M\).
From the construction, 
every look-ahead of \(\pi^k(M)\) has the form: \((p,\varphi_1,\dots,\varphi_k)\)
with a look-ahead state \(p\) of \(M\).
Since each application of \(\pi\) makes no improper states proper,
we may have an infinite sequence
$ F_p\cap Q_0 \subseteq F_{(p,\varphi_1)}\cap Q_0 \subseteq \dots \subseteq
   F_{(p,\varphi_1,\dots,\varphi_k)}\cap Q_0 \subseteq \dots $
for any maps \(\varphi_i\) introduced in the look-ahead of \(\pi^i(M)\).
Since \(Q_0\) is finite, the chain contains only finitely many strict inclusions.
Hence there is \(n\) such that 
%the set of improper states in \(Q_0\) becomes stable, that is,
\( F_{(p,\varphi_1,\dots,\varphi_n)}\cap Q_0 = F_{(p,\varphi_1,\dots,\varphi_{n'})} \cap Q_0 \)
for any \(n'>n\) and any maps \(\varphi_i\) with \(i\in[n']\).
%Let \(Q_t\) be a set of improper states of \(\pi^n(M)\).
Since no more improper states in \(Q_0\) will be added by another application of \(\pi\),
every reachable improper states of \(\pi^n(M)\) is a helper state, 
which has rank at most \((m-1)\).
\qed
\end{proof}

\begin{theorem}
  \label{th:proper}
For every mttr \(M\), there is a proper mttr \(M'\) equivalent to \(M\).
\end{theorem}
%
\begin{proof}
There is a nondeleting mttr \(M_0\) equivalent to \(M\) 
(\cite{DBLP:journals/iandc/EngelfrietM99} or Proposition~\ref{prop:nondeleting}).
We repeatedly construct equivalent trunsducers $\pi(M)$, $\pi(\pi(M))$, etc.
until a proper mttr is obtained (which is decidable by
Lemma~\ref{lm:dec}). The repetition terminates by Lemma~\ref{lm:term}
(first eleminating all reachable calls of improper states of the highest rank $m$, then
those or rank $m-1$, etc.).
\qed
\end{proof}

