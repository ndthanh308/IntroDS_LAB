\section{Linear Height and Linear Size-to-Height Increase}
\label{sec:decision_LSHI}

Let $\Gamma$ be a ranked alphabet and $t$ a tree over $\Gamma$.
We define the size $|t|$ of a tree as its number of nodes $|V(t)|$.
The height $\he{t}$ of $t$ is defined as
$\he{t}=0$ if $t\in\Gamma^{(0)}$ and 
$\he{t} = 1 + \text{max}\{\he{t_i}\mid i\in[k]\}$ if
$t=\gamma(t_1,\dots,t_k)$ for $\gamma\in\Gamma^{(k)}$, $k\geq 1$,
and $t_1,\dots,t_k\in T_\Gamma$.

Let $M$ be an mttr (with input ranked alphabet $\Sigma$).
Then $M$ has \emph{linear size-to-height increase} (for short LSHI) if
there exists a number $c$
such that for every input tree
$s\in T_\Sigma$: $\he{M(s)}\leq c\cdot |s|$.
The mttr $M$ has \emph{linear height increase} (for short LHI) if
there exists a number $c$ such that for every input tree
$s\in T_\Sigma$: $\he{M(s)}\leq c\cdot \he{s}$.

We now introduce two additional properties for mttrs which will allow
us to decide whether a given mttr has LSHI or LHI.
Recall that $\hat{M}$ denotes the extension of $M$: this transducer
can translate input trees that may contain leaves that are labeled
by elements from $P$ (the set of look-ahead states of $M$).
Whenever the state $q$ of $M$, or rank $m$, encounters an input node $u$
labeled by an element $p$ of $P$, the transducer $\hat{M}$ outputs 
$\< q,p>(y_1,\dots,y_m)$.
%, where 
%$\underline{q}\in\underline{Q}$ is a new output symbol of rank $m$
%and $t=\text{rev}(u)$ is a monadic tree that represents the reverse Dewey notation
%of the node $u$.

We say that the mttr $M$ is \emph{finite nesting} (for short fnest), 
if there exists a number $c$ such that for every input tree $s\in T_\Sigma[P]$
which contains exactly one occurrence of an element $p$ of $P$
there are at most $c$-many occurrences of symbols
$\<q,p>$ with $q\in Q$ on any path of the tree $\hat{M}(s)$;
in this case, we say that $c$ is a \emph{nesting bound} of $M$.
We say that $M$ is \emph{finite yield nesting} (for short fynest), 
if there exists
a number $c$ such that for every input tree $s\in T_\Sigma[P]$ 
there are at most $c$-many occurrences of symbols 
from $\< q,p>$ with $q\in Q$ on any path of the 
tree $\hat{M}(s)$;
in this case, we say that $c$ is a \emph{yield nesting bound} of $M$.

\begin{lemma}\label{lm:decidable}
  
Let $M$ be an mttr.
Then 
(1)~it is decidable whether or not $M$ is finite nesting and
(2)~it is decidable whether or not $M$ is finite yield nesting.
\end{lemma}
\begin{proof}
We use the extension $\hat{M}$ on input trees that contain
(1)~exactly one or
(2)~arbitrarily many occurrences of elements of $P$.
We feed the output into a nondeterministic top-down tree transducer $T$
that outputs the elements from $\< Q,P>$ on any input path. 
The resulting output language $T(\hat{M}(T_\Sigma))$ is finite if and only
if $M$ is (1)~fnest or (2)~fynest. The details can be found in the Appendix.
\qed
\qed
\end{proof}  

\begin{lemma}\label{lm:easy}
Let $M$ be an mttr.
(1)~If $M$ is finite nesting, then it is of linear size-to-height increase.
(2)~If $M$ is finite yield nesting, then it is of linear height increase. 
\end{lemma}
\begin{proof}
Let $M=(Q,P,\Sigma,\Delta,q_0,R,h)$ and let $\text{mhr}$ the maximum
height of the right-hand side of any rule in $R$.
Let $s$ be a fixed tree in $T_\Sigma$.

To prove~(2), let $c$ be a yield nesting bound of $M$.
For $i\in[\he{s}]$ let $s_i$ be the tree obtained from $s$ by replacing
all nodes $u$ at depth $i$ by $h(s/u)$.
We show that 
$\he{M(s_{i+1}[u\leftarrow\sigma(h(s/u1),\dots,h(s/uk))\mid u\in V_P(s_i)])}
\leq c\cdot\text{mhr} + \he{M(s_i)}$.
By applying this $\he{s}$ times, starting from $s_1$, we obtain that
the height of $M(s)$ is 
$\leq \he{s}\cdot c\cdot\text{mhr}$

Let $U$ be an arbitrary set of pairwise independent (i.e., not being
descendants of each other) nodes of $s$.
Let $s'=s[u\leftarrow h(s/u)\mid u\in U]$, let 
$u\in V(s')$ with $s'/u\in P$, and $\sigma=s[u]\in\Sigma^{(k)}$ with
$k\geq 0$.
To prove~(1), let $c$ be a nesting bound of $M$.
We now use a slight variation $\bar{M}$ of the extension of $M$:
each output symbol $\<q,p>$ has rank $m+1$, where $m$ is the rank of $q$.
In the first subtree of the output symbol $\<q,p>$ the mttr $\bar{M}$
outputs a monadic tree that is the reverse Dewey path $\text{rev}(u)$ of the input
node $u$ that created the symbol (this can easilty be achieved).
We show that 
$\he{\hat{M}(s'[u\leftarrow\sigma(h(s/u1),\dots,h(s/uk))])}
\leq c\cdot\text{mhr} + \he{M(s')}$.
By the definition of the nesting bound,
there are at most $c$-many occurrences in any path of $\bar{M}(s')$
of symbols from $\<Q,P>$ which all have $\text{rev}(u)$ as
their first subtree. By the definition of the semantics of an mttr,
all of these occurrences (and no other occurrence of that path)
are replaced by the right-hand side of a rule. Thus,
the height increases by at most $c\cdot\text{mhr}$.
If we start with the tree $h(s')$ and successively replace a node $u$
labeled by a symbol from $P$ by $\sigma(h(s/u1),\dots,h(s/uk))$,
where $\sigma=s[u]$, then after $|s|$-many applications of rules
as above we obtain the tree $M(s)$ thus having a height that
is $\leq |s|\cdot c\cdot \text{mhr}$.
\qed
\end{proof}

We call a tree in $s\in T_\Sigma[P]$ a \emph{$\Sigma$-context} if it contains
exactly one occurrence $u$ of an element of $P$.
For another tree $t$ and such a context, $s[t]$ denotes the tree
$s[u\leftarrow t]$.

\begin{lemma}\label{lm:nest}
Let $M$ be an mttr that is depth proper.
(1)~If $M$ is not finite nesting, 
then $M$ does not have linear size-to-height increase.
(2)~If $M$ is not finite yield nesting, 
then $M$ does not have linear height increase.
\end{lemma}
\begin{proof} (\emph{Sketch}.)
Let $M$ be given by a tuple as usual.
To prove~(1), assume that $M$ is not fnest.
We will show that this implies that $M$ does not have LSHI.
Since $M$ is not fnest (and has only finitely many states)
there must be some state $q\in Q^{(m)}$ with $m\geq 1$
that occurs arbitrarily often on paths of output trees of $M$.
More precisely, there are infinite sequences of contexts $c_0,c_1,\dots$
and numbers $n_0<n_1<\cdots$ such that
$q$ occurs $\geq n_0$ times on a path in $M(s_0)$ and
$q$ occurs $\geq n_1$ times on a path in $M(s_0[s_1])$, etc.
From this we can derive (by considering sufficiently many numbers $n_i$),
similar to the proof of Lemma~6.5 of~\cite{DBLP:journals/siamcomp/EngelfrietM03},
that $M$ is ``(nested)  input pumpable'', i.e.,
there exist $q_1,q_2,j,s_0,u_0,u_1,p$ such that
\begin{enumerate}
\item $\< q_1,p>$ occurs in $M(s_0[u_0\leftarrow p])$,
\item $M_{q_1}(s_0/u_0[u_1\leftarrow p])$ has either: a subtree 
$\< q_1,p>(t_1,\dots,t_m)$ such that some $t_{j'}$ contains a 
subtree $\< q_2,p>(\xi_1,\dots,\xi_l)$
where $\xi_{j}$ contains $y_{j'}$ for some $j'\in[m]$, 
or a subtree $\< q_2,p>(t_1,\dots,t_l)$ such that $t_j$ contains a subtree 
$\< q_1,p>(\xi_1,$ $\dots,\xi_m)$ with $j\in[l]$, 
\item $M_{q_2}(s_0/u_0[u_1\leftarrow p])$ has a subtree 
$\< q_2,p>(t_1,\dots,t_l)$ such that $t_j$ contains $y_j$, and
\item $p=h(s_0/u_0)=h(s_0/u_0u_1)$.
\end{enumerate}
By ``pumping'', i.e., considering 
$s_n=s_0/u_0[u_1\leftarrow s_0/u_0[u_1\leftarrow s_0/u_0[u_1\leftarrow \dots ]]]$
with $n$ replacements of the node $u_1$, we obtain that
$M_{q_1}(s_n)$ contains a path with $\geq n$ nested occurrences of $\< q_2,p>$.

Assume now by contradiction that $M$ has LSHI, i.e., there exists a $c$ such that
for every input tree $s\in T_\Sigma$: $\he{M(s)}\leq c\cdot |s|$.
Since $M$ is depth proper, we may choose $s\in L_p$ such that
$M_{q_2}(s)$ contains an occurrence of $y_{j}$ at depth
$\geq c .c_1 +1$, where $c_1=|s_0/u_0[u_1\leftarrow p]|-1$.
We know that $M_{q_1}(s_n)$ contains $\geq n$ nested occurrences of $q_2$
(where always the $j$-th subtree of $q_2$ contains further nested occurrences of $q_2$).
Now let $t_n=s_0[u_0\leftarrow s_n][u_0u_1^n\leftarrow s]$
and take $n>c(c_0+c_2)$, where $c_0=|s_0[u_0\leftarrow p]|-1$ and $c_2=|s|$.
Since $|t_n|=c_0+nc_1+c_2$, we obtain that $\he{M(t_n)}>c\cdot |t_n|$ because
$\he{M(t_n)}\geq n(cc_1+1) > ncc_1 + c(c_0+c_2)=c\cdot |t_n|$ by the choice of $n$.

The proof of Statement~(2) can be given in a very
similar way as for (1) and is omitted here.
%
%%Paul's version
%Because $M$ is depth-proper, for each $q \in Q^{(m)}, i \in [m], p \in P$ and 
%$n \in \N$, there is $t \in L_p$ such that $\he{\lcop{M_q(t)}_{\{y_i\}}} > n$. 
%We can ensure the existence of a productive loop in $t$ by taking 
%$n = 1+ \text{mhr} . k^{|Q| . |P|}$ where $k$ is the maximum arity of input trees. 
%Noting $t_j$ the tree where the loop is pumped $j$ times, we get 
%\[\he{\lcop{M_q(t_j)}_{\{y_i\}}} > c_s.|t_j| ~~~~\text{ and }~~~~ 
%\he{\lcop{M_q(t_j)}_{\{y_i\}}} > c_h.\he{t_j}\] 
%for all $j \geq 1$, 
%where $c_s$ and $c_h$ are strictly positive constants that depend on $M, q, i$ and $p$. 
%We can make $c_s$ and $c_h$ only dependent on $M$ by taking their maximum over 
%the possible $q \in Q^{(m)}, i\in [m]$ and $p \in P$. 
%
%To prove~(1), assume that $M$ is not fnest.
%We will show that this implies that $M$ does not have LSHI.
%For all $n \in \N$, there is $s \in T_\Sigma[P]$ with one occurrence of $P$ 
%such that there are at least $n$ nested occurrences of $P$ in $\he{\hat{M}(s)}$. 
%By taking $n$ big enough, we can get any number of nested calls of 
%\emph{the same state} $q$, nested in \emph{the same parameter} $y_i$ of $q$. 
%For all $n \in \N$, we note $s_{n} \in T_\Sigma[P]$ the tree with one occurrence of $P$ 
%such that there are at least $n$ occurrences of 
%$\langle q,p \rangle$ nested along parameter $y_i$ in $\he{\hat{M}(s)}$, 
%for some $q \in Q^{(m)}, i \in [m]$ and $p \in P$. We note $u_n$ such that $s_n/u_n \in P$, 
%and $t_{n,j} = s_n[u_n \leftarrow t_j]$ for all $n,j \in \N$. 
%For each $n$, there is $j_n$ such that $|t_{j_n}| > |s_n|$, so $|t_{j_n}| > |t_{n,j_n}|/2$. 
%Then, for all $n \in \N$:
%\[\he{M(t_{n,j_n})} \geq n.\he{\lcop{M_q(t_{j_n})}_{\{y_i\}}} > n.c_s.|t_{j_n}| > n.c_s.|t_{n,j_n}|/2 \]
%If $M$ had LSHI then there would be a constant $c$ such that \\
%$c > \he{M(t_{n,j_n})}/|t_{n,j_n}| > n.c_s/2$ for all $n \in \N$. 
%This is a contradiction, so $M$ does not have LSHI. 
%
%To prove~(2), assume that $M$ is not fynest.
%We will show that this implies that $M$ does not have LHI.
%For all $n \in \N$, there is $s \in T_\Sigma[P]$ 
%such that there are at least $n$ nested occurrences of $P$ in $\he{\hat{M}(s)}$. 
%By taking $n$ big enough, we can get any number of nested calls of 
%\emph{the same state} $q$, nested in \emph{the same parameter} $y_i$ of $q$, 
%applied to \emph{the same look-ahead state} $p$. 
%For all $n \in \N$, we note $s_{n} \in T_\Sigma[P]$ the tree 
%such that there are at least $n$ occurrences of 
%$\langle q,p \rangle$ nested along parameter $y_i$ in $\he{\hat{M}(s)}$, 
%for some $q \in Q^{(m)}, i \in [m]$ and $p \in P$. 
%We note $t_{n,j} = s_n[p \leftarrow t_j]$ for all $n,j \in \N$. 
%For each $n$, there is $j_n$ such that $\he{t_{j_n}} > \he{s_n}$, so $\he{t_{j_n}} > \he{t_{n,j_n}}/2$. 
%Then, for all $n \in \N$:
%\[\he{M(t_{n,j_n})} \geq n.\he{\lcop{M_q(t_{j_n})}_{\{y_i\}}} > n.c_h.\he{t_{j_n}} > n.c_h.\he{t_{n,j_n}}/2 \]
%If $M$ had LHI then there would be a constant $c$ such that \\
%$c > \he{M(t_{n,j_n})}/\he{t_{n,j_n}} > n.c_h/2$, for all $n \in \N$.
%This is a contradiction, so $M$ does not have LHI. 
%
\qed
\end{proof}

From Theorem~\ref{th:proper} and Lemmas~\ref{lm:decidable},~\ref{lm:easy}, and~\ref{lm:nest} we obtain
our following main theorem.

\begin{theorem}
Let $M$ be an mttr. 
Then 
(1)~it is decidable whether or not $M$ has linear size-to-height increase
(2)~it is decidable whether or not $M$ is linear height-increase.
\end{theorem}
