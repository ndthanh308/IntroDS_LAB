




%This version is 
%arXiv: paper password for this article is: 
%%%%%%%
%%%%%%%%%%%%%This version is for
%apart from minor stylistic points it fits with the 
%published version
%\documentstyle[11pt]{article}


%\documentstyle[11pt]{article}  % DEFINIÇÃO DO  FORMATO  COMO  ESTAVA.

\documentclass[12pt]{article}  % DEFINIÇÃO QUE RESOLVEU  O  PROBLEMA COM A  FIGURA

\usepackage{graphicx,color,epsf} % PACOTE PARA INSERIR FIGURAS
%\usepackage{epstopdf}  %PROBLEMA DE  RECONHECIMENTO  DA  IMAGEM

%\usepackage[superscript,biblabel]{cite} %NUMERAÇÃO DA BIBLIOGRAFICA COMO SOBRESCRITO

\usepackage{amsmath,latexsym}

%\documentstyle[11pt]{article}
%%%%%%%%%%%%%%%%%%%%%%%%%%%%%%%%%%%%%%%%%%%%%%%%%%%%%%%%%%%%%%%%%%%%%%%%%%%%%%%%%%%%%%%
\newcommand{\gsim}{\raisebox{-0.13cm}{~\shortstack{$>$ \\[-0.07cm]
      $\sim$}}~}
\newcommand{\lsim}{\raisebox{-0.13cm}{~\shortstack{$<$ \\[-0.07cm]
      $\sim$}}~}
%%%%%%%%%%%%%%%%%%%%%%%%%%%%%%%%%%%%%%%%%%%%%%%%%%%%%%%%%%%%%%%%%%%%%%%%%%%%%%%%%%%%%%%
\oddsidemargin -0.1in
%\topmargin -0.25 in
\topmargin -0.40 in
%\textwidth 6.375 true in
\textwidth 7.0 true in
%\textheight 22cm
\textheight 22.8cm

\begin{document}



%\begin{flushright}
%QMW/95-5
%\\hep-th/9502016
%\end{flushright}
%\vspace{0.5cm}
\begin{center}
{\LARGE \bf Boundary effects on the thermal stability of a black hole  
	at the centre of a conducting \newline spherical shell
}
\\ 
\vspace{1cm} 
{\large E. S. Moreira Jr.}
\footnote{E-mail: moreira@unifei.edu.br}   
\\ 
\vspace{0.3cm} 
{\em Instituto de Matem\'{a}tica e Computa\c{c}\~{a}o,}  
{\em Universidade Federal de Itajub\'{a},}   \\
{\em Itajub\'a, Minas Gerais 37500-903, Brazil}

\vspace{0.3cm}
{\large July, 2023}
\end{center}
\vspace{0.5cm}



%\begin{abstract}


\abstract{ 
This paper reports calculations and analysis of the effects of a 
perfect conducting wall of a very large spherical shell on the stable thermodynamic equilibrium of a black hole sitting at the centre of the shell which is filled with Electromagnetic blackbody radiation.
% is in thermodynamic equilibrium with the wall and the black hole.
A parallel is drawn with the case where electromagnetic is replaced by scalar blackbody radiation with Dirichlet or Neumann boundary conditions on the wall. It is found that the value of the shell radius
%, $R_{c}$,
 above which only blackbody radiation remains in stable
thermodynamic equilibrium 
%in the cavity 
can be considerably affected by vacuum polarization due to the presence of the wall.
%as well as by its nature. 
%A physical explanation is offered.  
}

%\vspace{0.5cm}
%\hspace{-0.5cm}\emph{Keywords:} hot scalar radiation, one-%dimensional boxes, ambiguities,
%zero modes, curvature coupling parameter, thermodynamic %equilibrium
%PACS number(s): 


\section{Introduction}
\label{introduction}
Five decades after Bekenstein, Bardeen, Carter, Hawking, Hartle,   
Gibbons, Perry \cite{bek73,bar73,haw75,gib76}
and others discovered that a black hole is a thermal object, its associated
quantum degrees of freedom remain unknown. Much understanding has been
gained by considering semi-classical gravity, which consists of treating matter quantum mechanically whereas the background geometry is taken to be classical. Indeed, there are also candidates to  a ``quantum theory of gravity'' (see, for example, ref. \cite{sus05} and references therein) which successfully present explanations for thermal properties of certain special black holes. In spite of a lot of progress, it seems fair to say that there is no consensus
on this matter.

It remains promising to explore the issue in setups where the black hole is in thermodynamic  equilibrium with well known thermal systems.  
One of them is a cavity  with 
the black hole at the centre continuously evaporating by emitting  Hawking radiation and simultaneously accreting blackbody radiation \cite{haw76} (see also textbooks \cite{nov10,tho17} and their references).
A Schwarzschild black hole of mass $M$
and  electromagnetic blackbody radiation can be 
in stable thermodynamic equilibrium only if the temperature $T$ of the system is the 
Hawking temperature \footnote{Fundamental constants have the usual meaning.},
\begin{equation}
T_{\tt H}:=	\frac{\hbar c^3}{8\pi G M\, k},
\label{htemperature}	
\end{equation}
and the cavity's volume $V$ is smaller than a critical value \cite{haw76,nov10,tho17}, 
\begin{equation}
V_{h}=	\frac{{2}^{20}}{5^{4}}\, 3\pi^{2}
\left(\frac{E}{E_{\tt P}}\right)^{5}\ell_{\tt P}^{3},
\label{cvolume}
\end{equation}
where $E$ is the total energy in the cavity, which is given by,
\begin{equation}
	E=Mc^{2}+\frac{\pi^{2}}{15}\frac{(kT)^{4}}{(\hbar c)^{3}}V,
	\label{total-energy}
\end{equation}
and, 
\begin{equation}
	\ell_{{\tt P}}:=\sqrt{\hbar G/c^{3}},\hspace{1.0cm}E_{{\tt P}}:=\sqrt{\hbar c^{5}/G}.	
	\label{planck}	
\end{equation}
In deriving eq. 
%(\ref{htemperature}) and 
(\ref{cvolume}), $E$ 
along with $V$ are taken to be fixed,
whereas $M$ and $T$ are let to evolve such that the entropy,
\begin{equation}
S=\frac{4\pi k}{\hbar c} GM^{2}+\frac{4\pi^{2}}{45}\left(\frac{kT}{\hbar c}\right)^{3}kV,
\label{entropy}
\end{equation}
has a local maximum.


By examining eq. (\ref{total-energy}), one sees that vacuum polarization caused by the non trivial background or the presence of the cavity's wall 
(see, e.g., ref. \cite{dav82})
has been ignored. It can be argued that $V$ is large enough such that vacuum polarization effects would simply yield small corrections 
%of the Planckian terms 
in eqs. (\ref{total-energy}) and
(\ref{entropy}), and so they perhaps could be neglected.
However, still remains the question concerned 
the way they could modify $V_{h}$ in eq. (\ref{cvolume}).
The purpose of this paper is to address this issue
\footnote{This paper will consider
vacuum polarization caused by the cavity's wall only.}.
%Nevertheless, certain aspects of vacuum polarization due to non %trivial black hole geometry  will be addressed later on, in the %context of an analysis of the main conclusion in ref. \cite{pav84}.
%}.
The rest of this section consists of a short outline of
calculations and analysis in the following sections.
%\footnote{This paper will address
%vacuum polarization caused by the cavity's wall only.}

The literature is rather vague on the nature of the wall that keeps $E$ and $V$ constant in the equations above. It is simply assumed that it is rigid and that no flux of either energy or momentum takes place through it, i.e.,
the cavity's wall is perfectly reflecting.
Nevertheless, any wall is a physical object, even when it is idealized. Thus the proper way of dealing with the problem in the context of quantum fields at finite temperature is to use suitable boundary conditions (see, e.g., the classic paper ref. \cite{bro69}, and a related thermodynamic account in ref. \cite{mor23}).

Consider a perfect conducting spherical shell of radius $R$,
%in flat spacetime,  
carrying electromagnetic radiation at temperature $T$. As Balian and Duplantier have shown  in ref. \cite{bal78}, at the regime of high temperatures and/or large shells,
i.e., 
\begin{equation}
\frac{kTR}{\hbar c}\gg 1,	
	\label{ht}	
\end{equation}
the Helmholtz free energy in the shell is given by,
%\footnote{Only leading-order corrections to the Planckian expressions will be considered along the text.}
\begin{equation}
{\cal F}=-\frac{\pi^{2}}{45}\frac{(kT)^{4}}{(\hbar c)^{3}}V
-\frac{kT}{4}\left[
{\rm ln}\left(\frac{kTR}{\hbar c}\right)
%\hspace{-0.15cm}+\hspace{-0.15cm} 
+
%\cdots
0.769
\right]+\cdots,	
\label{fenergy}	
\end{equation}
where 
\begin{equation}
V=\frac{4}{3}\pi R^{3}.	
\label{volume}	
\end{equation}
%$V=4\pi R^{3}/3$.
It follows from eq. (\ref{fenergy}) the internal energy and entropy,
\begin{equation}
	{\cal U}=\frac{\pi^{2}}{15}\frac{(kT)^{4}}{(\hbar c)^{3}}V
	+\frac{kT}{4}+\cdots,
	\hspace{0.8cm}
	{\cal S}=\frac{4\pi^{2}}{45}\left(\frac{kT}{\hbar c}\right)^{3}kV
	+\frac{k}{4}\left[
	{\rm ln}\left(\frac{kTR}{\hbar c}\right)
	%\hspace{-0.15cm}+\hspace{-0.15cm} 
	+
	%\cdots
	1.769
	\right]+\cdots,	
	\label{venergy}	
\end{equation}
each containing corrections to the familiar Planckian expressions
due to the presence of the spherical shell's perfect conducting wall. A couple of remarks are in order regarding 
%$U$ in 
eq. (\ref{venergy}) as they will be used later on.
The term $kT/4$ in ${\cal U}$
%of $U$ in eq. (\ref{venergy}) 
is often  identified as been a ``classical'' correction, since it does not carry $\hbar$. However, when one sets $\hbar\rightarrow 0$, the Planckian internal energy becomes ``infinite'' and thus $kT/4$ ``vanishes'':
$kT/4$ is simply the $\hbar^{0}$-term of an expansion in powers of $\hbar$. 
%Another remark is in order regarding $U$ in eq. (\ref{venergy}) as %it will be used later on. 
Another remark is that the presence of the perfect conducting wall increases the heat capacity of the hot radiation in the shell. Consequently, the blackbody radiation temperature becomes less ``agile'' to change when ${\cal U}$ changes.

Now a black hole 
%Schwarzschild black hole of mass $M$ 
is set at the shell's centre and let to get in stable thermodynamic equilibrium with the hot radiation surrounding  it. Then the vacuum polarization corrections in eq. (\ref{venergy}) are taken into account in eqs. (\ref{total-energy}) and (\ref{entropy}), and the problem of finding a local maximum for $S$ is redone, leading to a new $V_{h}$ [see eq. (\ref{cvolume})]. Such a program is implemented in Sec. \ref{vector}. 


In Sec. \ref{scalar}, the problem is reconsidered for spherical shells containing hot scalar radiation, with Dirichlet and Neumann walls. Sec. \ref{conclusion} presents a summary and addresses further issues considered in previous sections, such as  comparison 
between vacuum polarization modifications of eq. (\ref{cvolume})
regarding the nature of the radiation in the spherical shell (electromagnetic or scalar) as well as the type of boundary conditions on its wall (Dirichlet or Neumann).
The paper closes after  pointing out an extension of the work.




%but such a designation should not be misinterpreted






\section{Electromagnetic radiation}
\label{vector}
As has been sketched in Sec. \ref{introduction},
eqs. (\ref{total-energy}) and (\ref{entropy}) are modified by considering eq. (\ref{venergy}), i.e.,
\begin{eqnarray}
&&E=Mc^{2}+\frac{\pi^{2}}{15}\frac{(kT)^{4}}{(\hbar c)^{3}}V
+\frac{kT}{4}+\cdots,	
\label{E}	
\\	
&&S=\frac{4\pi k}{\hbar c} GM^{2}+
\frac{4\pi^{2}}{45}\left(\frac{kT}{\hbar c}\right)^{3}kV
+\frac{k}{4}\left[
{\rm ln}\left(\frac{kTR}{\hbar c}\right)
%\hspace{-0.15cm}+\hspace{-0.15cm} 
+
%\cdots
1.769
\right]+\cdots.
\label{S}	
\end{eqnarray}
Ellipses, denoting smaller corrections
[note eq. (\ref{ht})], will be omitted from now on.
It is convenient to defined the dimensionless quantities,
\begin{equation}
m:=\frac{Mc^{2}}{E}, \hspace{1.0cm}
t:=\frac{kT}{4E},
\label{parameters}
\end{equation}
and to recast eq. (\ref{E}) as,
\begin{equation}
m=1-\frac{2^{10}\pi^{3}}{45}\left(\frac{ER}{\hbar c}\right)^{3}t^{4}-t,	
\label{m}	
\end{equation}
where eq. (\ref{volume}) has been used.
In the literature (see, e.g., ref. \cite{tho17}), where only Planckian contributions are considered [cf. eqs. (\ref{total-energy}) and (\ref{entropy})], it is $m$ in eq. (\ref{parameters})
that plays the role of
%is taken as 
independent variable. In the present case, where corrections to the Planckian contributions are also taken into account, $t$ in eq. (\ref{parameters})
is the suitable independent variable in the  problem. It follows that $m$ in eq. (\ref{m}) is a function of $t$, for given $E$ and $R$.
By defining another dimensionless quantity, namely
\begin{equation}
s:=\frac{\hbar c^{5}S}{4\pi kGE^{2}},	
\label{ps}
\end{equation}
eq. (\ref{S})  leads to,
\begin{equation}
s=m^{2}+
\frac{2^{8}\pi^{2}}{135}\frac{ER^{3}c^{2}}{\hbar^{2}G}t^{3}
+\frac{\hbar c^{5}}{16\pi GE^{2}}\left[
{\rm ln}\left(\frac{4ER}{\hbar c}t\right)
%\hspace{-0.15cm}+\hspace{-0.15cm} 
+
%\cdots
1.769
\right],
\label{s}
\end{equation}
which is also a function of $t$. At this point, it should be remarked that, since $ER\gg kTR \gg \hbar c$ [see eqs. (\ref{ht}) and (\ref{E})], $t$ in eq. (\ref{parameters}) is small, 
but not that much: 
\begin{equation}
\frac{\hbar c}{ER} \ll t \ll 1.	
\label{tbounds}
\end{equation}	
%$\hbar c/ER \ll t \ll 1$.

A comment on the shell's radius is also worth making at this point. As shall be seen soon, a necessary condition for thermodynamic equilibrium is still that $T=T_{\tt H}$ [see eq. (\ref{htemperature})], as expected. Now, noticing eq. (\ref{ht}),
it follows that,
\begin{equation}
R\gg R_{\tt S},	
\label{lradius}
\end{equation}
with $R_{\tt S}$ denoting the Schwarzschild radius of the black hole in the shell:
\begin{equation}
 R_{{\tt S}}:=\frac{2GM}{c^{2}}.	
\label{sradius}
\end{equation}
That is, the size of the shell must be large compared with that of  the hole. If $R$  were such that the inequality in eq. (\ref{ht})
were reversed, then the regime would be that of low temperatures and/or small shells, resulting that  eq. (\ref{venergy}) would no longer hold \cite{bal78}. In particular, the dominant contribution in the expression for ${\cal U}$ would be a Casimir energy, instead of the familiar Planckian internal energy. 
%$\propto VT^{4}$.
%Moreover,  the sizes of the shell and the hole would be comparable, and  nontrivial features of the background could not be ignored, in this case.
%cannot coexist together in thermodynamic equilibrium.


Noticing eq. (\ref{m}), derivative of eq. (\ref{s}) with respect to $t$ yields,
\begin{equation}
\frac{ds}{dt}=
\frac{\hbar c^{5}}{16\pi GE^{2}t}\left[
\frac{4^{6}\pi^{3}}{45}\left(\frac{ERt}{\hbar c}\right)^{3}+1\right]
(1-\tau),
\label{ds/dt}
\end{equation}
where,
\begin{equation}
\tau:=32\pi \frac{GE^{2}}{\hbar c^{5}} mt.
\label{tau}
\end{equation}
It follows immediately from eqs. (\ref{htemperature}), (\ref{parameters}) and (\ref{tau}) that,
\begin{equation}
	\tau=\frac{T}{T_{\tt H}}.
\label{tau2}	
\end{equation}
Clearly, blackbody radiation and the black hole cannot coexist together in thermodynamic equilibrium if the maximum value of $\tau$ is less than unity. Otherwise, according to eq. (\ref{ds/dt}),
$s$ would grow for all $t$ and the black hole would evaporate leaving only blackbody radiation behind.





The maximum value of $\tau$ happens when $d\tau/dt=0$, i.e.[see eqs. (\ref{m}) and (\ref{tau})],
\begin{equation}
1-\frac{2^{10}\pi^{3}}{9}\left(\frac{ER}{\hbar c}\right)^{3}t^{4}-2t=0.
\label{ttau}
\end{equation}
By recalling eq. (\ref{tbounds}), the solution of eq. (\ref{ttau}) is given by \footnote{As mentioned previously, keeping main contributions only.
%in the expression of $t_{\tau}$
}:
\begin{equation}
t_{\tau}=\left(\frac{9}{2^{10}\pi^{3}}\right)^{1/4}
\left(\frac{\hbar c}{ER}\right)^{3/4}
\left[1-\frac{1}{4}\left(\frac{9}{(4\pi)^{3}}\right)^{1/4}
\left(\frac{\hbar c}{ER}\right)^{3/4}\right],	
	\label{ttau2}
\end{equation}
corresponding to,
\begin{equation}
m(t_{\tau})=\frac{4}{5}	
-\frac{3}{10}
\left(\frac{9}{(4\pi)^{3}}\right)^{1/4}
\left(\frac{\hbar c}{ER}\right)^{3/4},
\label{mtau}
\end{equation}
where eq. (\ref{m}) has been used.
Now, eqs. (\ref{tau}), (\ref{ttau2}) and (\ref{mtau}) yield,
\begin{equation}
\tau(t_{\tau})=\frac{2^{6}\pi}{5}\frac{GE^{2}}{\hbar c^{5}}
\left(\frac{9}{(4\pi)^{3}}\right)^{1/4}
\left(\frac{\hbar c}{ER}\right)^{3/4}
\left[1-\frac{5}{8}\left(\frac{9}{(4\pi)^{3}}\right)^{1/4}
\left(\frac{\hbar c}{ER}\right)^{3/4}\right],	
\label{maxtau}
\end{equation}
as the maximum value of $\tau$.


By taking into account the comments just after eq. (\ref{tau2})
and noting that $\tau(t_{\tau})$ 
in eq. (\ref{maxtau}) diminishes as $R$ increases with $E$ fixed,
one can determine the shell's critical volume $V_{h}$ by solving 
$\tau(t_{\tau})=1$ for $R$ [see eqs. (\ref{planck}) and (\ref{volume})], obtaining: 
\begin{equation}
V_{h}=\frac{{2}^{20}}{5^{4}}3\pi^{2}
\left(\frac{E}{E_{\tt P}}\right)^{5}\ell_{\tt P}^{3}-
\frac{{2}^{13}}{5^{2}}3\pi
\left(\frac{E}{E_{\tt P}}\right)^{3}\ell_{\tt P}^{3},	
\label{vvolume}
\end{equation}
which should be compared with eq. (\ref{cvolume}).
An interesting aspect of eq. (\ref{vvolume}) is that the second term on its rhs is ``classical'' as it is $kT/4$ in eq. (\ref{E}) [see remark in the text just after eq. (\ref{venergy})].
Indeed, 
$$
2\frac{E}{E_{\tt P}}\ell_{\tt P}
=\frac{2GE}{c^{4}}.
$$
Thus, the vacuum polarization effect of a perfect conducting wall is to reduce the value in eq. (\ref{cvolume}) by nearly a  hundred of 
%92.16
times the volume of a Schwarzschild black hole of mass $E/c^{2}$ [see eq. (\ref{vvolume})]. 
It should be pointed out that,
since $R_{{\tt S}}>l_{{\tt P}}$,
eqs. (\ref{lradius}) and (\ref{vvolume}) imply that,
\begin{equation}
	\frac{E}{E_{{\tt P}}}\gg 1.	
	\label{lenergy}	
\end{equation}
In order to have a crude idea of the size of each term in eq. (\ref{vvolume}), one sets $E/E_{\tt P}\approx 10$ and $E/E_{\tt P}\approx 1$
%in 
%eq. (\ref{vvolume}).
to see that the reduction is about $0.06\%$ and $6\%$
of the main contribution, respectively. As it will be shown in the next section,
the corresponding correction for scalar radiation is much larger.






\section{A parallel: scalar radiation}
%Scalar blackbody radiation. }
\label{scalar}
At this section, electromagnetic is replaced by scalar 
blackbody radiation coexisting with the Schwarzschild  black hole in a spherical shell of radius $R$.
Dirichlet and Neumann boundary conditions on the shell's wall will be addressed. The results will be compared with those in the previous section.


Still considering the regime of high temperatures and/or large shells [see eq. (\ref{ht})],
the Helmholtz free energy for the scalar radiation 
can be obtained from Dowker's general formula in ref.
\cite{dow84}, and it is given by [cf. eq. (\ref{fenergy})]
\footnote{Recall that ellipses are being omitted.}
\begin{equation}
{\cal F}=-\frac{\pi^{2}}{90}\frac{(kT)^{4}}{(\hbar c)^{3}}V
\pm\frac{\zeta(3)}{8\pi}\frac{(kT)^{3}}{(\hbar c)^{2}}A,
\label{sfenergy}	
\end{equation}
upper sign for Dirichlet, lower sign for Neumann
\footnote{As in the whole text.}, and where $A=4\pi R^{2}$
[note also eq. (\ref{volume})]. Then eq. (\ref{venergy})
gives place to,
\begin{eqnarray}
	{\cal U}=\frac{\pi^{2}}{30}\frac{(kT)^{4}}{(\hbar c)^{3}}V
	\mp\frac{\zeta(3)}{4\pi}\frac{(kT)^{3}}{(\hbar c)^{2}}A,
&&	%\hspace{0.8cm}
	{\cal S}=\frac{2\pi^{2}}{45}\left(\frac{kT}{\hbar c}\right)^{3}kV
	\mp\frac{3\zeta(3)}{8\pi}\left(\frac{kT}{\hbar c}\right)^{2}kA.
	\label{senergy}	
\end{eqnarray}
Unlike the correction to the Planckian internal energy in eq. (\ref{venergy}), ${\cal U}$ in eq. (\ref{senergy}) carries  ``quantum'' corrections. Notice also that the Dirichlet wall makes  the scalar radiation temperature more ``agile'' to change when 
${\cal U}$ varies, whereas the effect caused by the Neumann wall is the opposite one \footnote{Like that for electromagnetic radiation. See Sec. \ref{vector}.}.



Correspondingly, eqs. (\ref{E}) and (\ref{S}) are replaced by their scalar counterparts:
\begin{eqnarray}
	&&E=Mc^{2}+\frac{\pi^{2}}{30}\frac{(kT)^{4}}{(\hbar c)^{3}}V
	\mp\frac{\zeta(3)}{4\pi}\frac{(kT)^{3}}{(\hbar c)^{2}}A,
	\nonumber	
	%\label{sE}	
	\\	
	&&S=\frac{4\pi k}{\hbar c} GM^{2}+\frac{2\pi^{2}}{45}\left(\frac{kT}{\hbar c}\right)^{3}kV
	\mp\frac{3\zeta(3)}{8\pi}\left(\frac{kT}{\hbar c}\right)^{2}kA,
	\nonumber
	%\label{sS}	
\end{eqnarray}
leading to [cf. eqs. (\ref{m}) and (\ref{s})],
\begin{eqnarray}
&&	
m=1-\frac{2^{9}\pi^{3}}{45}\left(\frac{ER}{\hbar c}\right)^{3}t^{4}
\pm \zeta(3)2^{6}\left(\frac{ER}{\hbar c}\right)^{2}t^{3},
\label{sm}		
\\
&&	
s=m^{2}+
\frac{2^{7}\pi^{2}}{135}\frac{ER^{3}c^{2}}{\hbar^{2}G}t^{3}
\mp \frac{\zeta(3)6}{\pi}\frac{R^{2}c^{3}}{\hbar G}t^{2},	
\label{ss}		
\end{eqnarray}
where eqs. (\ref{parameters}) and (\ref{ps}) have been used again.
It should be remarked that eqs. (\ref{tbounds}) and (\ref{lradius})
still hold. Now, proceeding as in the previous section, eqs. (\ref{sm})
and  (\ref{ss}) lead to [cf. eq. (\ref{ds/dt})]:
\begin{equation}
	\frac{ds}{dt}=
	\frac{\hbar c^{5}}{32\pi GE^{2}t}\left[
	\frac{4^{6}\pi^{3}}{45}\left(\frac{ERt}{\hbar c}\right)^{3}\mp\zeta(3)2^{7}3\left(\frac{ERt}{\hbar c}\right)^{2}\right]
	(1-\tau),
	\label{sds/dt}
\end{equation}
where $\tau$ is defined in eq. (\ref{tau}) and it satisfies eq. (\ref{tau2}).
Again, by examining eq. (\ref{sds/dt}), one sees that 
for a black hole and blackbody radiation to coexist in the shell,
the maximum value of $\tau$ must not be less than
unity. Then, noticing eqs. (\ref{sm}) and (\ref{tau}), 
the maximum value of $\tau$ happens when $d\tau/dt=0$ as before,
resulting that eq. (\ref{ttau}) gives place to,
%\begin{equation}
$$
	1-\frac{2^{9}\pi^{3}}{9}\left(\frac{ER}{\hbar c}\right)^{3}t^{4}\pm\zeta(3)2^{8}\left(\frac{ER}{\hbar c}\right)^{2}t^{3}=0,
%	\label{stau}
%\end{equation}
$$
whose solution is given by [cf. eq. (\ref{ttau2})]:
%$$
\begin{equation}
	t_{\tau}=\left(\frac{9}{2^{9}\pi^{3}}\right)^{1/4}
	\left(\frac{\hbar c}{ER}\right)^{3/4}
	\left[1\pm\zeta(3)\left(\frac{9}{2\pi^{3}}\right)^{3/4}
	\left(\frac{\hbar c}{ER}\right)^{1/4}\right].	
	\label{stau2}
\end{equation}
%$$
Now, by setting $t=t_{\tau}$ in eq. (\ref{sm}), it results that,
  \begin{equation}
  	m(t_{\tau})=\frac{4}{5}	
  	\pm \frac{\zeta(3)}{5}
  	\left(\frac{9}{2\pi^{3}}\right)^{3/4}
  	\left(\frac{\hbar c}{ER}\right)^{1/4},
  	\label{smtau}
  \end{equation}
which corresponds to eq. (\ref{mtau}).
Then, the maximum value of $\tau$ is given by,
$$
%\begin{equation}
	\tau(t_{\tau})=\frac{2^{7}\pi}{5}\frac{GE^{2}}{\hbar c^{5}}
	\left(\frac{9}{2^{9}\pi^{3}}\right)^{1/4}
	\left(\frac{\hbar c}{ER}\right)^{3/4}
	\left[1\pm\zeta(3)\, 5\, 2^{4}\left(\frac{9}{2^{9}\pi^{3}}\right)^{3/4}
	\left(\frac{\hbar c}{ER}\right)^{1/4}\right],	
%	\label{maxtau}
%\end{equation}
$$
where eqs. (\ref{tau}), (\ref{stau2}) and (\ref{smtau}) 
have been used.
Finally, considering the arguments just before eq. (\ref{vvolume}), one ends up with,
\begin{equation}
	V_{h}=\frac{{2}^{21}}{5^{4}}3\pi^{2}
	\left(\frac{E}{E_{\tt P}}\right)^{5}\ell_{\tt P}^{3}
	\pm\zeta(3)\left[
	\frac{{2}^{56}}{5^{8}}\frac{3^{7}}{\pi}
	\left(\frac{E}{E_{\tt P}}\right)^{13}\right]^{1/3}
	\ell_{\tt P}^{3},	
	\label{svolume}
\end{equation}
which is the scalar version of eq. (\ref{vvolume}), for Dirichlet and Neumann boundary conditions on the shell's wall. Unlike the correction in eq. (\ref{vvolume}) due to the presence of a conducting wall,
those corrections in eq. (\ref{svolume}) carry $\hbar$ 
\footnote{See comments just after eq. (\ref{vvolume}).}.

Considering  eq. (\ref{lenergy}), which also holds for scalar radiation, it should be remarked that 
by setting $E/E_{\tt P}\approx 10$ and $E/E_{\tt P}\approx 1$
%in 
in eq. (\ref{svolume}),
the corrections are about $13\%$ and $60\%$
of the main contribution, respectively. 
As has been previously mentioned
\footnote{See text closing Sec. \ref{vector}.}, 
these are far larger than 
the corresponding correction in eq. (\ref{vvolume}).



	

\section{Final remarks}
\label{conclusion}

A Black hole is a thermal object, and it can be in stable thermodynamic equilibrium with blackbody radiation, if the box
where they coexist is not that large. This paper reported a study of the effects on thermal stability of acknowledging vacuum polarization caused by the wall of a spherical shell which contains a Schwarzschild black hole at its centre and it is filled with either electromagnetic or scalar hot radiations.
The critical volume $V_{h}\propto (E/E_{\tt P})^{5}\ell_{\tt P}^{3}$ 
above which the two systems cannot coexist in stable thermodynamic equilibrium is well known. It was shown here that this value is reduced by a term proportional to $(E/E_{\tt P})^{3}\ell_{\tt P}^{3}$  
in the case of electromagnetic radiation in a perfect conducting shell, and it is increased
by a term proportional to $\pm(E/E_{\tt P})^{13/3}\ell_{\tt P}^{3}$ 
in the case of scalar radiation in a shell with Dirichlet ($+$ sign) or Neumann ($-$ sign) walls.

As is well known, a Schwarzschild black hole has a negative heat capacity, $C<0$. It can be in stable thermodynamic equilibrium with blackbody radiation, with heat capacity $C_{V}>0$, only if, 
\begin{equation}
C_{V}<|C|,	
	\label{hc}
\end{equation}
as a basic calculation in thermodynamics may show
\footnote{The inequality in eq. (\ref{hc}) is equivalent to the requirement that total entropy has a local maximum at the equilibrium temperature.}. That is, eq. (\ref{hc}) says that the blackbody radiation must be more ``agile'' than the black hole to catch up the temperature of the latter when it rushes away from equilibrium. by replacing ``$<$'' in eq. (\ref{hc})
by ``$=$'', one obtains $V_{h}$ in eqs. (\ref{vvolume}) and (\ref{svolume}).








A natural extension of the present work would be an investigation when  the inequality in eq. (\ref{ht}) is less stringent, 
such that low temperatures and/or small shells could be considered.
%i.e., when:
%\begin{equation}
%\frac{kTR}{\hbar c}\ll 1,
%\label{lt}
%\end{equation}
%which is appropriate to consider low temperatures and/or small %shells.
In an early work, Pav\'{o}n and Israel have speculated that the first term in the rhs of eq. (\ref{svolume}) is a good approximation for $V_{h}$ ``even for Planck-mass black holes and for box radii comparable with the size of the black hole'' \cite{pav84}.
%small boxes and Planckian mass black holes.
Recalling  that by setting $E/E_{\tt P}\approx 1$ 
%in 
in eq. (\ref{svolume})
the corrections reach $60\%$
of the main contribution, it seems that the conclusion in Ref. \cite{pav84} needs to be re-examined. 
At such a regime, %in eq. (\ref{lt}),
vacuum polarization due to non trivial black hole geometry will most likely have to be taken into account, as has indeed been done in Ref. \cite{pav84}. Nevertheless, vacuum polarization due to the wall, which was the focus here,
certainly cannot be neglected.
Perhaps, the Casimir energy mentioned early in the text may play a relevant role. 










\vspace{1cm}
\noindent{\bf Acknowledgements} -- 
Work partially supported by
``Funda\c{c}\~{a}o de Amparo \`{a} Pesquisa do Estado de Minas Gerais'' (FAPEMIG)
and by ``Coordena\c{c}\~{a}o de Aperfei\c{c}oamento de Pessoal de N\'{\i}vel Superior'' (CAPES).








\begin{thebibliography}{88}

\bibitem{bek73}
J. D. Bekenstein, 
Black holes and entropy,
Phys. Rev.
{\bf D 7}, 2333
(1973)







\bibitem{bar73}
J. M. Bardeen, B. Carter, S. W. Hawking, 
The four laws of black hole mechanics,
Commun. Math. Phys.
{\bf 31}, 161
(1973)







\bibitem{haw75}
S. W. Hawking, 
Particle creation by black holes,
Commun. Math. Phys. 
{\bf 43}, 199 
(1975)


\bibitem{gib76}
G. W. Gibbons, M. J. Perry,
Black holes in thermal equilibrium,
Phys. Rev. Lett. {\bf 36}, 985 (1976)





\bibitem{sus05} 
L. Susskind, J. Lindesay, 
\emph{An Introduction to Black Holes, Information and The 
String Theory Revolution: The Holographic Universe}
(World Scientific, Singapore, 2005)


\bibitem{haw76}
S. W. Hawking, 
Black holes and thermodynamics,
Phys. Rev.
{\bf D 13}, 191
(1976) 

\bibitem{nov10}
I. D. Novikov, V. P. Frolov,
\emph{Physics of Black Holes}
(Kluwer Academic Publishers, Dordrecht, 2010)

\bibitem{tho17}
K. S. Thorne, R. D. Blandford,
\emph{Modern Classical Physics}
(Princeton University Press, Princeton, 2017)

\bibitem{dav82} 
N. D. Birrel, P. C. W. Davies,
\emph{Quantum Fields in Curved Space}
(Cambridge University Press, Cambridge UK, 1982)


\bibitem{pav84}
D. Pav\'{o}n, W. Israel,
Stability of thermal equilibrium for a radiating black hole in a box,
Gen. Relativ. Gravit. {\bf 16}, 563 
(1984)





\bibitem{bro69}
L. S. Brown, G. J. Maclay, 
Vacuum stress between conducting plates: An image solution, 
Phys. Rev.
{\bf 184}, 1272 
(1969) 


\bibitem{mor23}
E. S. Moreira Jr., H. da Silva,
Blackbody thermodynamics in the presence of Casimir's effect,
 J. Stat. Mech. 
 063102 
(2023) 


\bibitem{bal78}
R. Balian, B. Duplantier, 
Electromagnetic waves near perfect conductors. II. Casimir effect,
Ann. Phys.  {\bf 112}, 165
(1978)



\bibitem{dow84}
J. S. Dowker, 
Finite temperature and vacuum effects in higher dimensions, 
Class. Quant. Grav. {\bf 1}, 359 (1984)






\end{thebibliography}



\end{document}

