\documentclass[11pt, letterpaper]{article} %
\usepackage{times}
\usepackage{url}            %
\usepackage{booktabs}       %
\usepackage{amsfonts}       %
\usepackage{nicefrac}       %
\usepackage{microtype}      %
\usepackage{xspace}
\usepackage{multirow}
\usepackage{amsmath}
\usepackage{algorithm}
\usepackage{algorithmic}
\usepackage{color, colortbl}
\usepackage{enumitem}
\usepackage{comment}
\usepackage{bm}
\usepackage{fullpage}
\usepackage{caption}
\usepackage{subcaption}
\usepackage{amssymb}
\usepackage{mathtools}
\usepackage{amsthm}
\usepackage[dvipsnames]{xcolor}
\usepackage[colorlinks=true, linkcolor=red, citecolor=blue, urlcolor=WildStrawberry]{hyperref}
\usepackage[capitalize,noabbrev]{cleveref}
\usepackage{natbib}
\usepackage{mathrsfs}
\usepackage{parskip}

\usepackage{mkolar_definitions}

\RequirePackage{bm}
\RequirePackage{nicefrac}
\RequirePackage{amsfonts}
\RequirePackage{dsfont}
\RequirePackage{setspace}
\RequirePackage{amsthm}
\RequirePackage{mathrsfs}
\RequirePackage{hyperref}


\def\*#1{\bm{#1}} 
\def\+#1{\mathcal{#1}}
\def\=#1{\mathbb{#1}}
\def\##1{\mathfrak{#1}}
\def\^#1{\hat{#1}} 
\def\-#1{\bar{#1}}
\def\~#1{\tilde{#1}}
\def\@#1{{\rm{#1}}}
\def\w~#1{\widetilde{#1}}

\def\indic{\mathds{1}}
\def\d{\mathop{}\!\mathrm{d}}
\def\E{\mathop{\mathbb{E}}}
\def\KL{D_{\rm{KL}}}
\def\argmax{\mathop{\arg\max}}
\def\argmin{\mathop{\arg\min}}
\def\given{\,|\,}

\newcommand\red[1]{\textcolor{red}{#1}}
\newcommand\blue[1]{\textcolor{blue}{#1}}
\newcommand\purple[1]{\textcolor{purple}{#1}}
\newcommand\orange[1]{\textcolor{orange}{#1}}
\newcommand\cyan[1]{\textcolor{cyan}{#1}}

\newcommand{\edit}[1]{\cyan{#1}}

\newcommand\numberthis{\addtocounter{equation}{1}\tag{\theequation}}



\newcommand{\alglinelabel}{
  \addtocounter{ALC@line}{-1}
  \refstepcounter{ALC@line}
  \label
}


\Crefname{assumption}{Assumption}{Assumptions}

\author{%
Ayush Sekhari$^{1}$ \quad Karthik Sridharan$^{2}$ \quad Wen Sun$^{2}$ \quad Runzhe Wu$^{2}$
\vspace{5pt}
\\
\normalsize{$^1$MIT \quad $^2$Cornell University}
\vspace{5pt}
\\
\small{\texttt{sekhari@mit.edu \quad \{ks999,ws455,rw646\}@cornell.edu}
}}


\date{}


\title{Contextual Bandits and Imitation Learning via Preference-Based Active Queries\thanks{Authors are listed in alphabetical order of their last names.}} 

\bibpunct{(}{)}{;}{a}{,}{,}

\usepackage[suppress]{color-edits} 
\addauthor{as}{red}
\addauthor{rw}{orange}

\begin{document}

\sloppy 

\maketitle 

\begin{abstract}
We consider the problem of contextual bandits and imitation learning, where the learner lacks direct knowledge of the executed action's reward. Instead, the learner can actively query an expert at each round to compare two actions and receive noisy preference feedback. The learner's objective is two-fold: to minimize the regret associated with the executed actions, while simultaneously, minimizing the number of comparison queries made to the expert. In this paper, we assume that the learner has access to a function class that can represent the expert's preference model under appropriate link functions, and provide an algorithm that leverages an online regression oracle with respect to this function class \asedit{for choosing its actions and deciding when to query}. For the contextual bandit setting, our algorithm achieves a regret bound that combines the best of both worlds, scaling as \(O(\min\{\sqrt{T}, d/\Delta\})\), where \(T\) represents the number of interactions, \(d\) represents the eluder dimension of the function class, and \(\Delta\) represents the minimum preference of the optimal action over any suboptimal action under all contexts. Our algorithm does not require the knowledge of \(\Delta\), and the obtained regret bound is comparable to what can be achieved in the standard contextual bandits setting where the learner observes reward signals at each round. Additionally, our algorithm makes only \(O(\min\{T, d^2/\Delta^2\})\) queries to the expert. 

We then extend our algorithm to the imitation learning setting, where the learning agent engages with an unknown environment in episodes of length \(H\) each, and provide similar guarantees for regret and query complexity. \asedit{The regret bound for our imitation learning algorithm, which relies on preference-based feedback, matches the prior results in interactive imitation learning \citep{ross2014reinforcement} that require access to the expert's actions as well as reward signals. Furthermore, we show that our algorithm enjoys improved query complexity bounds. Interestingly, in some cases, our algorithm for imitation learning via preference-feedback can even learn to outperform the underlying expert thus highlighting a practical benefit of considering preference-based feedback in imitation learning.}  
 
\end{abstract} 

% Figure environment removed

\section{Introduction}
Automatic 3D reconstruction of clothed humans using image inputs has gained increasing significance due to its potential applications in a wide array of AR/VR scenarios. High-fidelity reconstructions typically depend on sophisticated capture systems, which are developed with dense camera arrays~\cite{collet2015high,joo2015panoptic,joo2018total}, programmable light-stages~\cite{Vlasic2009, guo2019relightables}, and depth sensors~\cite{newcombe2011kinectfusion,DoubleFusion,BodyFusion,dou2016fusion4d,newcombe2015dynamicfusion}. However, stringent capture environments equipped with complex hardware pose significant challenges for consumer-level applications.


In this context, considerable research effort has been dedicated to developing methods that allow for more flexible capture configurations, such as utilizing a few RGB inputs. Among these works, learning implicit functions \cite{iccv2020PIFu, saito2020pifuhd, hong2021stereopifu} has proven effective in achieving highly detailed reconstructions by integrating the advancements of deep neural networks. These methods employ large multi-layer perceptrons (MLPs) to predict the occupancy probability or truncated signed distance function (TSDF) value of every queried 3D point based on its associated local feature, which is extracted from images. They can recover a continuous surface at arbitrary resolutions without topology restrictions.


However, in typical MLP-based implicit networks, the occupancy or TSDF value at each location is solved independently with planar image features, rendering them less capable of addressing challenging cases such as occlusions. Consequently, these methods suffer from generalization and robustness issues, particularly when tackling strong occlusions caused by large motion or multiple interacting humans. 
Some follow-up studies  \cite{zheng2021deepmulticap,zheng2021pamir,huang2020arch} utilize an extra geometric model, SMPL~\cite{Loper2015}, to improve robustness by introducing strong shape priors. 
Their success typically relies on the assumption of geometrical similarity \cite{huang2020arch} between the shape prior and target reconstruction, making them intractable for handling complex cases with loose clothes and sensitive to errors in SMPL model fitting.



%\ping{this paragraph sounds like `TSDF is better than MLP/SMPL, and we use TSDF to solve the problem'. But in Sec 3, we are telling a different story, saying `MLP needs a 3D convolutional encoder'. We need to make these two sections consistent.}\sicong{I think in this paragraph we claim that the TSDF}


%We opt for Trucated Signed Distance Funtion (TSDF) volumetric representations as they are naturally suitable for convolution operations, which have shown remarkable performance for learning hierarchical features on 2D visual perception tasks \cite{SunXLW19}. 
%Meanwhile, TSDF also describes the gradual geometry change around shape surface, which is not reflected by occupancy volume. 

We instead revisit the 3D volumetric representation and resort to 3D convolutional neural networks (CNNs) for feature learning, due to their impressive performance in feature learning and the ability to incorporate spatial context. However, volumetric methods and 3D convolution involve discretization, which might raise concerns regarding whether a discretized volume can preserve subtle geometric details as continuous representations learned in implicit functions. We investigate the relationship between volume resolution and quantization error on synthetic data by converting target mesh objects to TSDF volumes, as shown in Figure~\ref{fig:quantization_error}. We observe that the quantization errors are significantly reduced by increasing volume resolution and become nearly negligible when reaching a relatively high resolution (e.g., 512 or higher). In other words, achieving fine-detailed reconstruction is not supposed to be restricted by the use of volume representations as long as a proper volume resolution is utilized. Therefore, we present a method with high-resolution feature volumes, e.g., 256 and 512, while traditional volumetric methods \cite{varol18_bodynet,gilbert2018volumetric} are often limited to much lower resolutions, such as 32 or 128.



On the other hand, an increase in volume resolution may lead to a cubic growth of memory overhead \cite{8100085}. Reducing memory costs while guaranteeing the granularity of volumetric representations is necessary for pursuing high-quality reconstruction. Thus, we adopt a coarse-to-fine approach and cull away irrelevant voxels to build a sparse high-resolution feature volume. At the coarse level, the network computes an initial TSDF by applying a U-Net with sparse 3D CNN \cite{3DSemanticSegmentationWithSubmanifoldSparseConvNet} on the sparse feature volume, which is carved by a visual hull. Through our experiments, it turns out that more than 95\% of the volume grids are discarded by the visual hull culling, making the sparse 3D CNN efficient. At the fine level, the network focuses on a narrow band near the zero-level set of the initial TSDF and discretizes the narrow band with smaller voxels. By employing this narrow-band culling, we further shrink the sampling space, resulting in a relatively small range of grid numbers (usually 300K--500K in our experiments) even with a high volume resolution of 512. The remaining voxels in the narrow band are associated with features that fuse high-frequency information from the computed normal maps upon the low-frequency shape from the coarse level to compute the TSDF at high resolution. The final mesh is then extracted from the TSDF using the Marching-Cube algorithm ~\cite{Lorensen87marchingcubes}.
% Different from the u-net sturcture to preserve global topology context, we then apply a shallow 3dcnn to compute the final TSDF $D_{final}$ which contain more local geometry detail.




% \ping{this paragraph can be expanded. It is an important contribution and often ignored by other works. stress on the novel idea of regressing blending weights instead of colors}

In addition to geometry, high-quality mesh texture is also a crucial factor contributing to visual appearance. Directly computing a color field in 3D space, as in \cite{iccv2020PIFu}, struggles to capture high-frequency texture details, while the neural radiance field (NeRF) \cite{yu2020pixelnerf} or the DoubleField~\cite{shao2022doublefield} require expensive per-instance optimization and are often unstable for sparse input images. In contrast, we adopt an image-based rendering approach to compute a texture atlas map, which is efficient and widely supported in existing computer graphics tools. 
Specifically, we compute a blending weight at each 3D point on the mesh surface to determine its color as a weighted average of the colors at its image projections. The blending weights can be computed at a relatively coarse resolution, e.g., 512 volume resolution in our case, and leave texture details to the high-resolution images, such as 1K or 2K. Unlike previous methods that generate blurry texturing results under sparse input, our method generalizes well on both synthetic and real data with just a few input views. 
Figure~\ref{fig:teaser} shows two examples reconstructed by our method. Despite the challenging garment, pose, and occlusion, our method recovers faithful shape, normal, and texture on the right.

%with a wide variety of poses and clothing styles, and it is also adaptive to handle input image with arbitrary resolutions.
%\sicong{For this concern we claim that when the resolution of dicretized volume meets certain threshold (which is 256 in our experiment), the quantization error can be neglected.} 



In summary, the main contributions of this paper are as follows:
\begin{itemize}
\vspace{-0.1in}
  \item 
  We revisit the 3D volumetric representation and demonstrate that it can support clothed human reconstruction with equal or even better performance compared to implicit representation. 
  \item 
  We develop a memory and computation-efficient method for high-resolution volumetric reconstruction using sophisticated sparse 3D CNN, coarse-to-fine estimation, and voxel culling by visual hull and narrow bands. 
  \item 
  We introduce a novel method to compute a texture atlas map, which captures rich appearance details from high-resolution input images.
  \item 
  We achieve impressive results on standard benchmark datasets Twindom and MultiHuman, significantly reducing the point-2-surface (P2S) precision to approximately 0.2cm from just six input views, with more than $50\%$ error reduction compared to the state-of-the-art methods, including DoubleField~\cite{shao2022doublefield} and PIFuHD~\cite{saito2020pifuhd}.
\end{itemize}
\section{Related Work}
%\subsection{Cost Volume based Deep Stereo Matching}
%Stereo matching is a typical problem that has been studied for decades and a well-known four-step pipeline \cite{scharstein2002taxonomy} has been established, where cost volume construction is an indispensable step. Current state-of-the-art stereo matching methods are all cost volume based methods and they can be categorized into two types. Typically, a cost volume is a 4D tensor of height, width, disparity, and features. The first category just uses a full correlation to generate a single-feature cost volume. Such methods are usually efficient but lose much information because of the decimation of feature channels. Many previous work, including Dispnet \cite{dispnet}, MADNet \cite{madnet}, IResNet \cite{iresnet} and AANet \cite{aanet}, belong to this category. The second category usually uses concatenation \cite{gcnet} or group-wise correlation \cite{gwcnet} to generate a multi-feature 4D cost volume. Such a method can achieve better performance while requiring higher computational complexity and memory consumption. Actually, a majority of the top-performing networks in public leaderboards belong to this category, such as GANet \cite{ganet}, CSPN \cite{cspn} and ACFNet \cite{acfnet}. These methods generally employ multiple 3D convolution layers to constantly regularize the 4D cost volume and then apply softmax over the disparity dimension to produce a discrete disparity probability distribution. The final predicted disparity is obtained by softly weighting indices according to their probability, which is also called soft argmin in GCNet \cite{gcnet}. However, soft argmin leaves the output susceptible to multi-modal disparity probability distributions. ACFNet \cite{acfnet} observes this problem and proposes to directly supervise the cost volume with unimodal ground truth distributions. In contrast, we define an uncertainty estimation to quantify the degree to which the cost volume tends to be multi-modal distribution, higher implies the higher possibility of estimation error.

\subsection{Multi-scale Cost Volume based Stereo Matching}
Cost volume construction is an indispensable step in the well-known four-step pipeline for stereo matching \cite{scharstein2002taxonomy, pamisurvey1, pamisurvey2}. Typically, current state-of-the-art stereo matching methods can be categorized into two types of cost volume-based methods, where the cost volume is a 4D tensor of height, width, disparity, and features. The first category usually uses the single-feature 3D cost volume generated by full correlation, which is efficient while losing much information due to the decimation of feature channels. Many real-time methods, such as Dispnet \cite{dispnet}, MADNet \cite{madnet, madnet_pami} and AANet \cite{aanet}, belongs to the category. Moreover, two-stage refinement \cite{mcvmfc} and pyramidal towers \cite{madnet} are commonly applied in the single-feature cost volume based network to construct multi-scale cost volume. The second category usually uses the multi-feature 4D cost volume generated by concatenation \cite{gcnet} or group-wise correlation \cite{gwcnet}, which can achieve better performance with higher computational complexity and memory consumption. Most top-performing networks, including GANet \cite{ganet}, CSPN \cite{cspn} and ACFNet \cite{acfnet} belong to this category. 
% In these methods, the 4D cost volume is constantly regularized by multiple 3D convolution layers and then a discrete disparity probability distribution can be produced by softmax. Next, the final predicted disparity can be obtained by softly weighting indices according to their probability \cite{gcnet}. However, such output is susceptible to multimodal disparity probability distributions and ACFNet \cite{acfnet} gives a solution by directly supervising the cost volume with unimodal ground truth distributions to alleviate this problem. 
Recently, to alleviate the high computational complexity and memory consumption when employing multi-feature 4D cost volumes, \cite{cvpmvsnet, cascade, uscnet} propose to use cascade cost volume representation in multi-view stereo. These methods usually first predict an initial disparity at the coarsest resolution of the image and then gradually refine the disparity by narrowing down the disparity search space. More closely related to our approach is Casstereo \cite{cascade}, which first extended such representation to stereo matching. It selected to uniform sample a pre-defined range to generate the next stage’s disparity search range. Instead, we employ pixel-level uncertainty estimation to adaptively adjust the next stage disparity searching range and generate pseudo-labels for subsequent domain adaptation. Our method also shares similarities with UCSNet \cite{uscnet}, which constructs uncertainty-aware cost volume in multi-view stereo while it doesn’t employ uncertainty estimation to generate pseudo-labels.

%\subsection{Multi-scale Cost Volume based Deep Stereo Matching} 
% \subsection{Multi-scale Cost Volume based Stereo Matching} 
%Multi-scale cost volume firstly was applied in the single-feature cost volume based network with the form of two-stage refinement \cite{mcvmfc} and pyramidal towers \cite{madnet}. Recently, cascade cost volume representation \cite{cvpmvsnet, cascade, uscnet} was proposed in multi-view stereo to alleviate the high computational complexity and memory consumption when employing multi-feature 4D cost volumes. These methods generally predict an initial disparity at the coarsest resolution of the image. Then, they will narrow down the disparity search space and gradually refine the disparity. More closely related to our approach is Casstereo \cite{cascade}, which first extended such representation to stereo matching. It selected to uniform sample a pre-defined range to generate the next stage’s disparity search range. Instead, we employ uncertainty estimation to adaptively adjust the next stage pixel-level disparity searching range and push the next stage's cost volume to be predominantly unimodal.

% The single-feature cost volume based network with the form of two-stage refinement \cite{mcvmfc} and pyramidal towers \cite{madnet} first employ multi-scale cost volume for stereo matching. Recently, to alleviate the high computational complexity and memory consumption when employing multi-feature 4D cost volumes, \cite{cvpmvsnet, cascade, uscnet} propose to use cascade cost volume representation in multi-view stereo, which generally predict an initial disparity at the coarsest resolution of the image. Then, the disparity search space is narrowed down and the disparity is gradually refined. More closely related to our approach is Casstereo \cite{cascade}, which first extended such representation to stereo matching. It selected to uniform sample a pre-defined range to generate the next stage’s disparity search range. Instead, we employ uncertainty estimation to adaptively adjust the next stage pixel-level disparity searching range and push the next stage's cost volume to be predominantly unimodal.

% Figure environment removed

\subsection{Robust Stereo Matching} 
There exist three categories of generalization definitions for robust stereo matching. 1) Cross-domain Generalization: the network’s ability to perform well on unseen scenes (cannot see the image pairs of the target domain in advance). Towards this end, Jia et al \cite{sungeneralizaiton} propose to incorporate scene geometry priors into an end-to-end network. Zhang et al \cite{dsmnet} introduce a domain normalization and a trainable non-local graph-based filter to construct a domain-invariant stereo matching network. 2) Adapt Generalization: the network’s ability to adapt pre-trained models to the new domain with unlabeled target data. Previous work usually pre-trains the models on synthetic data and then adapts it to new target domains with Graph Laplacian regularization \cite{zoom}, non-adversarial progressive color transfer \cite{adastereo}, and Knowledge Reverse Distillation \cite{aohnet}. More closely related to our approach are \cite{aohnet, unsuperviseddomainadaptation} in stereo matching and Monoresmatch \cite{monoresmatch} in monocular depth estimation, which also proposes to generate a pseudo-label for domain adaptation. However, these methods all select to employ classical stereo matching methods \cite{sgm} alongside with confidence estimators, e.g., left-right consistency check to generate pseudo-labels. That is all these methods need an independent method to generate corresponding pseudo-labels. Instead, the proposed method is an end-to-end network that can generate the predicted disparity map, corresponding uncertainty map and pseudo-labels jointly, which is a more simple, yet efficient way. 
% Instead, our proposed method can employ pixel-level and area-level uncertainty estimation to self-distill the predicted disparity maps of our pre-training model and generate sparse while reliable pseudo-labels to align the domain gap, which is a more simple, yet efficient way. 
3) Joint Generalization: the network’s ability to perform well on a variety of datasets with the same model parameters. MCV-MFC \cite{mcvmfc} introduces a two-stage finetuning scheme to achieve a good trade-off between generalization and fitting capability on multiple datasets. However, it doesn’t touch the inner difference between diverse datasets, e.g, the unbalanced disparity distribution. To further address this problem, we propose a cascade cost volume to adaptively the next stage disparity searching space, where the pixel-level uncertainty estimation is at the core.

% \subsection{Monocular Depth Estimation}
% Monocular depth estimation aims to estimate depth values from a single image, instead of stereo images or multiple frames in a video. This problem is ill-posed because of the ambiguity of object sizes. However, humans could estimate the depth from a single image with prior knowledge of the scenes. Recently, learning based methods were explored to learn depth values by supervised or unsupervised learning. Eigen et al. first employed Convolutional Neural Networks (CNN) to predict depth in a coarse-to-fine manner and further improved its performance by multi-task learning. Liu et al. presented deep convolutional neural fields model by combining deep model with continuous CRF. Li et al. [22] refined deep CNN outputs with a hierarchical CRF. Multi-scale continuous CRF was formulated into a deep sequential network by Xu et al. [45] to refine depth estimation. Unsupervised methods tried to train monocular depth estimation with stereo
% image pairs or image sequences and test on single images. Garg et al. [9] used novel image view synthesis loss to train a depth estimation network in an unsupervised way. Godard et al. [11] introduced left-right consistency regularization to improve the performance of view synthesis loss. Recently, some work also propose to use the stereo matching network as a proxy to learn depth from synthetic data or directly employ traditional stereo matching methods to distill proxies labels from the target domain, which proves the feasibility of distilling stereo matching networks to learn monocular depth estimation.



We first review some basic concepts from probability theory (see standard textbooks such as \cite{pollard2002user,williams1991probability} for a detailed treatment), 
%the background of Bayesian inference, and finally 
%We first review some basic concepts from probability theory, 
and then present the Bayesian probabilistic programming language and the normalised posterior distribution (NPD) problem.
%we consider in this work. 
Throughout the paper,
we denote by $\Nset$, $\Zset$ and $\Rset$ the sets of all natural numbers (including zero), integers, and real numbers, respectively.

\vspace{-1.5ex}
\subsection{Basics of Probability Theory}
%We assume familiarity with basic probability theory (see \cref{app:prelim} for details). 

A \emph{measurable space} is a pair $(U,\Sigma_U)$, where $U$ is a nonempty set and $\Sigma_U$ is a $\sigma$-algebra on $U$, i.e., a family of subsets of $U$ such that $\Sigma_U\subseteq \mathcal{P}(U)$ contains $\emptyset$ and is closed under complementation and countable union. Elements of $\Sigma_U$ are called \emph{measurable} sets. A function $f$ from a measurable space $(U_1,\Sigma_{U_1})$ to another measurable space $(U_2,\Sigma_{U_2})$ is \emph{measurable} if $f^{-1}(A)\in\Sigma_{U_1}$ for all $A\in\Sigma_{U_2}$.

A \emph{measure} $\mu$ on a measurable space $(U,\Sigma_U)$ is a mapping from $\Sigma_U$ to $[0,\infty]$ such that (i) $\mu(\emptyset)=0$ and (ii) $\mu$ 
%satisfies the
is countably additive:
%condition: 
for every pairwise-disjoint set sequence $\{A_n\}_{n\in\Nset}$ in $\Sigma_U$, it holds that $\mu(\bigcup_{n\in\Nset}A_n)=\sum_{n\in\Nset}\mu(A_n)$. We call the triple $(U,\Sigma_U,\mu)$ a \emph{measure space}. 
%If $\mu(U)\le 1$, we call $\mu$ a \emph{subprobability measure}. 
If $\mu(U)=1$, we call $\mu$ a \emph{probability measure}, and $(U,\Sigma_U,\mu)$ a \emph{probability space}.
The Lebesgue measure $\lambda$ is the unique measure on $(\Rset,\Sigma_{\Rset})$ satisfying $\lambda([a,b))=b-a$ for all valid intervals $[a,b)$ in $\Sigma_{\Rset}$. For each $n\in\Nset$, we have a measurable space $(\Rset^n,\Sigma_{\Rset^n})$ 
%such that there exists 
and
a unique product measure $\lambda_n$ on $\Rset^n$ satisfying $\lambda_n(\prod_{i=1}^n A_i)=\prod_{i=1}^n \lambda(A_i)$ for all $A_i\in\Sigma_{\Rset}$.


The \emph{Lebesgue} integral operator $\int$ is a partial operator that maps a measure $\mu$ on $(U,\Sigma_U)$ and a real-valued function $f$ on the same space $(U,\Sigma_U)$ to a real number or infinity, which is denoted by $\int f \mathrm{d}\mu$ or $\int f(x)\mu(\mathrm{d}x)$. 
The detailed definition of Lebesgue integral is somewhat technical, see \cite{rankin1968real,rudin1976principles} for more details. 
Given a measurable set $A\in\Sigma_U$, the integral of $f$ over $A$ is defined by $\int_A f(x)\mu(\mathrm{d} x):=\int f(x) \cdot [x\in A] \mu(\mathrm{d}x)$
%\begin{align*}
%\textstyle\int_A f(x)\mu(\mathrm{d} x):=\int f(x) \cdot [x\in A] \mu(\mathrm{d}x)
%\end{align*} 
where $[-]$ is the Iverson bracket such that $[\phi]=1$ if 
%the predicate 
$\phi$ is true, and $0$ otherwise. If $\mu$ is a probability measure, then we call the integral as the \emph{expectation} of $f$, denoted by $\expectdist{x\sim\mu;A}{f}$, or $\expv[f]$ when the scope is clear from the context.

For a measure $v$ on $(U,\Sigma_U)$, a measurable function $f:U\to \Rset_{\ge 0}$ is the \emph{density} of $v$ with respect to $\mu$ if $v(A)=\int f(x)\cdot [x\in A] \mu(\mathrm{d} x)$ for all measurable $A\in\Sigma_U$, and $\mu$ is called the \emph{reference measure} (most often $\mu$ is the Lebesgue measure). Common families of probability distributions on the reals, e.g., uniform, normal distributions, are measures on $(\Rset,\Sigma_{\Rset})$. Most often these are defined in terms of probability density functions with respect to the Lebesgue measure. That is, for each $\mu_D$ there is a measurable function $\text{pdf}_D:\Rset\to\Rset_{\ge 0}$ that determines it: $\mu_D(A):=\int_A \text{pdf}_D (\mathrm{d}\lambda) $. As we will see, density functions such as $\text{pdf}_D$ play an important role in Bayesian inference.

Given a probability space $\pspace$, a \emph{random variable} is an $\mathcal{F}$-measurable function $X: \Omega \rightarrow \Rset \cup \{+\infty,-\infty\}$. The expectation of a random variable $X$, denoted by $\expv(X)$, is the Lebesgue integral of $X$ w.r.t. $\probm$, i.e., $\int X\,\mathrm{d}\probm$. A \emph{filtration} of $\pspace$ is an infinite sequence $\{ \mathcal{F}_n \}_{n=0}^{\infty}$ such that for every $n\ge 0$, the triple $(\Omega, \mathcal{F}_n, \probm)$ is a probability space and $\mathcal{F}_n \subseteq \mathcal{F}_{n+1} \subseteq \mathcal{F}$. A \emph{stopping time} w.r.t. $\{ \mathcal{F}_n \}_{n=0}^{\infty}$ is a random variable $T: \Omega \rightarrow \Nset \cup \{0, \infty\}$ such that for every $n \geq 0$, the event \{$T \leq n$\} is in $\mathcal{F}_n$. 

A \emph{discrete-time stochastic process} is a sequence $\Gamma = \{X_n\}_{n=0}^\infty$ of random variables in $\pspace$. The process $\Gamma$ is \emph{adapted} to a filtration $\{ \mathcal{F}_n \}_{n=0}^{\infty}$, if for all $n \geq 0$, $X_n$ is a random variable in $(\Omega, \mathcal{F}_n, \probm)$. A discrete-time stochastic process $\Gamma=\{X_n\}_{n=0}^\infty$ adapted to a filtration $\{\mathcal{F}_n\}_{n=0}^\infty$ is a \emph{martingale} (resp. \emph{supermartingale}, \emph{submartingale})
if for all $n \geq 0$, $\expv(|X_n|)<\infty$ and it holds almost surely (i.e.,~with probability $1$) that
$\condexpv{X_{n+1}}{\mathcal{F}_n}=X_n$ (\mbox{resp. } $\condexpv{X_{n+1}}{\mathcal{F}_n}\le X_n$, $\condexpv{X_{n+1}}{\mathcal{F}_n}\ge X_n$).
See~\cite{williams1991probability} for details.
%Intuitively, a martingale is a discrete-time stochastic process, in which at any time $n$, the expected value $\condexpv{X_{n+1}}{\mathcal{F}_n}$ in the next step, given all previous values, is equal to the current value $X_n$. In a supermartingale, this expected value is less than or equal to the current value and a submartingale is defined conversely.
Applying martingales to qualitative and quantitative analysis of probabilistic programs is a well-studied technique~\cite{SriramCAV,ChatterjeeFG16,ChatterjeeNZ2017}.


\subsection{Bayesian Probabilistic Programming Language}

%We consider an imperative arithmetic probabilistic programming language. 
The syntax of our probabilistic programming language (PPL) is given in \cref{fig:syntax}, where the metavariables $S$, $B$ and $E$ stand for statements, boolean expressions and arithmetic expressions, respectively.   
Our PPL is imperative with the usual conditional and loop structures (i.e.,~\textbf{if} and \textbf{while}), as well as the following new structures: (a)~sample constructs of the form ``$\textbf{sample}\  D$'' that sample a value from a prescribed distribution $D$ over $\mathbb{R}$ and then assign this value to a sampling variable $r$; (b)~score statements of the form ``\textbf{score}($EW$)'' that weight the current execution with a value expressed by $EW$ (note that $\textit{pdf}(D,x)$ means the value of a probability density function w.r.t. $D$ at $x$);
%\footnote{Instead of the hard conditioning that refutes the execution when the observation mismatches the value of the sampling variable, we use the more general soft conditioning and assume the existence of a global weight variable initialized  to $1$.}
%for each program
(c)~probabilistic branching statements of the form
``$\textbf{if}\ \textbf{prob}(p)\dots$'' that lead to the then part with probability
$p\in (0,1]$ and to the else part with probability $1-p$. We also have sequential compositions (i.e., ";") and support return statements (i.e., \textbf{return}) that 
return the value of the program variable of interest. %The set of all statements is denoted by $Stmt$.
Note that $c,c_1,c_2\in\Rset$ are constants, and our language supports any distributions with continuous density functions and infinite supports, 
including but not limited to uniform and normal distributions. 



% Figure environment removed





Given a probabilistic program in our language, we distinguish two disjoint sets of variables in the program: (i) the set $\pvars$ of \emph{program variables} whose values are determined by assignments in the program (i.e., the expressions at the LHS of ``:="); (ii)~the set $\rvars$ of \emph{sampling variables} whose values are independently sampled from prescribed probability distributions each time they are accessed (i.e., each ``$\textbf{sample}\ D$" can be regarded as a sampling variable $r$). 




\begin{example}\label{ex:pedestrian-program}

%Consider the pedestrian random walk example~\cite{DBLP:conf/esop/MakOPW21}, a pedestrian is lost on a road, and she only knows that she is away from her house at most $3$ km. Thus, she starts to repeatedly walk a uniformly random distance of at most $1$ km in either direction, until reaching her house. Upon she arrives, an  odometer tells that she has walked $1.1$ km totally. However, this odometer was once broken and the measured distance is normally distributed around the true distance with standard deviation $0.1$ km. 
\cref{fig:pedestrian-program} shows a Bayesian probabilistic program written in our PPL language. In this program, the set of program variables is $\pvars=\{start,pos,dis,step\}$, and the set of sampling variables is $\rvars=\{ \textbf{sample uniform}(0,1)\}$. Each time $\textbf{sample uniform}(0,1)$ is executed, it samples a value uniformly from $[0,1]$ and then assigns the value to the variable $step$. 
%Thus, $step$ is associated with the probability distribution $\textbf{uniform}(0,1)$.
\qed


	
% Figure environment removed
\end{example}

\subsection{The Semantics of Our Programming Language}

%To relate variables with their values, we introduce the notion of valuations. 
Let $V$ be a finite set of variables with an implicit linear order over its elements. A \emph{valuation} on $V$ is a function $\pv: V \rightarrow \Rset$ that assigns a real value to each variable in $V$. We denote the set of all valuations on $V$ by $\val{V}$. For each $1\le i\le |V|$, we denote the value of the $i$-th variable (in the implicit linear order) in $\pv$ by $\pv[i]$, so that we can view each valuation as a real vector on $V$. A \emph{program} (resp. \emph{sampling}) valuation is a valuation on $\pvars$ (resp. $\rvars$), respectively. 
For the sake of convenience, we fix the notations in the following way, i.e., we always use $\pv\in\val{\pvars}$ to denote a program valuation, and $\rv\in\val{\rvars}$ to denote a sampling valuation; we also write $\pv[\mathit{ret}]$ for the value of the return variable in $\pv$. 



Below we present the semantics for our programming language. Existing semantics in the literature are either measure-\cite{DBLP:conf/lics/StatonYWHK16,LeeYRY20} or sampling-based  \cite{DBLP:conf/esop/MakOPW21,Beutner2022b}. To facilitate the development of our algorithm, we consider the \emph{transition-based} semantics~\cite{DBLP:conf/cav/ChakarovS13,DBLP:conf/popl/ChatterjeeFNH16} to our language and 
%To apply template-based algorithmic approaches to NPD problems, we consider  that 
treat each probabilistic program as a \emph{weighted probabilistic transition system} (WPTS). A WPTS extends a PTS  ~\cite{DBLP:conf/cav/ChakarovS13,DBLP:conf/popl/ChatterjeeFNH16} with weights and an initial probability distribution. 





%Below we present a variant of probabilistic transition systems \cite{DBLP:conf/cav/ChakarovS13}.
\begin{definition}
%[Weighted Probabilistic Transition Systems]
[WPTS]\label{def:wpts}
	A \emph{weighted probabilistic transition system} (WPTS) $\Pi$
	is a tuple
\begin{equation}\label{eq:wpts} 
\tag{\dag}
\Pi = (\pvars, \rvars,  L,\lin,\lout,\mu_{\mathrm{init}}, \rdvarjdis,\transset)%\win)
\end{equation}
for which:
	\begin{itemize}
		\item
		$\pvars$ and $\rvars$ are finite disjoint sets of \emph{program} and resp. \emph{sampling} variables.
%  (variables}) 
%  such that $\pvars\cap \rvars=\emptyset$.
    \item $\locs$ is a finite set of \emph{locations} 
  %or \emph{program counters} 
  with special locations $\lin,\lout\in \locs$. Informally, $\lin$ is the initial location and $\lout$ represents program termination. 
		\item
		$\mu_{\mathrm{init}}$ is the \emph{initial probability distribution} over $\mathbb{R}^{\pvars}$ with a finite support (denoted by $\supp{\mu_{\mathrm{init}}}$), 
  %from which the initial program valuation %$\valin$ is sampled, 
  while $\rdvarjdis$ is a function that assigns a probability distribution $\rdvarjdis(r)$ to each 
  %sampling variable 
  $r \in \rvars$. We call each $\pv\in\supp{\mu_{\mathrm{init}}}$ an \emph{initial program valuation}, and abuse the notation so that $\rdvarjdis$ also denotes the independent joint distribution of all $\rdvarjdis(r)$'s ($r\in \rvars$).
		\item 
		$\transset$ is a finite set of \emph{transitions} where
		each transition $\tau \in \transset$ is a tuple $\langle \loc, \phi, F_1,\dots,F_k \rangle$ such that 
(i) $\loc\in L$ is the \emph{source location} of the transition, 
%\item 
(ii) $\phi$ is the \emph{guard condition} which is a predicate over variables $\pvars$, %which serves as the \emph{guard condition}, 
and (iii) each $F_j:=\langle \loc'_j, p_j, \upd_j,\wet_j \rangle$ is called a \emph{weighted fork} for which (a) $\loc'_j\in L$ is the \emph{destination location} of the fork, (b) $p_j\in (0,1]$ is the probability of this fork, (c) $\upd_j:\Rset^{|\pvars|} \times \Rset^{|\rvars|} \rightarrow \Rset^{|\pvars|}$ is an {\em update function} that takes as inputs the current program and sampling valuations  and returns an updated program valuation in the next step, and (d) $\wet_j:\Rset^{|\pvars|} \times \Rset^{|\rvars|}\to [0,\infty)$ is a \emph{score function} that gives the likelihood weight of this fork depending on the current program and sampling valuations.	
\end{itemize}
\end{definition}


In a WPTS, we use update and score functions to model the update on the program variables and resp. the likelihood weight when running a basic block of statements in a program, respectively.  
%and use score functions to model  caused by the execution of the score statements (if exists) in this block. 
If there is no score statement in the block, then the score function is constantly $1$. 
We always assume that a WPTS $\Pi$ is \emph{deterministic} and \emph{total}, i.e., (i) there is no program valuation that simultaneously satisfies the guard conditions of two distinct transitions from the same source location, and (ii) the disjunction of the guard conditions of all the transitions from any source location is a tautology. 
The transformation from a probabilistic program into its WPTS can be done in a straightforward way (see e.g.~\cite{DBLP:journals/toplas/ChatterjeeFNH18,DBLP:conf/cav/ChakarovS13}). 

\begin{example}\label{ex:pedestrian-semantics} 
\cref{fig:pedestrian-wpts} shows the WPTS of the program in \cref{fig:pedestrian-program} which has two locations $\lin,\lout$. 
 %In the WPTS, 
The circle nodes represent locations and square nodes model the forking behavior of transitions. An edge entering a square node is labeled with the condition of its respective transition, while an edge entering a circle node stands for a fork, which is associated with its probability, update functions and score functions that marked by $w$.\footnote{Here we omit the update functions if the values of program variables are unchanged.} The value of $step$ is initialised to $0$. An the initial probability distribution $\mu_{\mathrm{init}}$ is determined by the joint distribution of $(start,pos,dis,step)$ where $start\sim uniform(0,3)$ and $pos,dis,step$ observe the Dirac measures $Dirac(\{start\})$, $Dirac(\{0\})$ and $Dirac(\{0\})$, respectively, e.g., the probability of the event ``$dis\in\{0\}$'' equals $1$. As $step$ simply receives values from a sampling variable, we neglect it in the WPTS.\qed
\end{example}

%\paragraph{Score-recursive WPTS.} 

We say that a WPTS is \emph{non-score-recursive} if for all transitions $\tau=\langle \loc, \phi, F_1,  F_2,\dots,F_k \rangle$ in the WPTS with each fork $F_j=\langle \loc'_j, p_j, \upd_j,\wet_j \rangle$ ($1\le j\le k$), we have that each score function $\wet_j$ is constantly $1$ (i.e., the multiplicative weight does not change) for every $\loc'_j\ne \lout$. Otherwise, the WPTS is \emph{score-recursive}.
Informally, a non-score-recursive WPTS has non-trivial score functions only on the transitions to the termination of a program, while a score-recursive WPTS has {\tt score} statements in the execution of the program. 
For example, the WPTS of the program in~\cref{sec3:pedestrian} is non-score-recursive as the nontrivial (i.e., score values not equal to $1$) {\tt score} statement only appears to the termination, while the WPTS of the program in \cref{sec3:phylogenetic} is score recursive since it has {\tt score} statements inside the loop body.
In the case of a non-score-recursive WPTS, we say that the WPTS is \emph{score-bounded} by a positive real $M>0$ if for every $\tau=\langle \loc, \phi, F_1, F_2,\dots,F_k \rangle$ in the WPTS with $F_j=\langle \loc'_j, p_j, \upd_j,\wet_j \rangle$ ($1\le j\le k$), we have that 
$|\wet_j|\le M$ whenever $\loc'_j=\lout$.


Given a program valuation $\mathbf{v}$ and a predicate $\phi$ over variables $\pvars$, we say that $\mathbf{v}$ \emph{satisfies} $\phi$ (written as $\mathbf{v}\models\phi$) if $\phi$ holds when the variables in $\phi$ are substituted by their values in $\mathbf{v}$. 
A \emph{state} 
%of the WPTS $\Pi$ 
is a pair $\Xi=(\loc, \pv)$ where $\loc \in L$ (resp. $\pv \in \Rset^{|\pvars|}$) represents the current location (resp. program valuation), respectively, while a \emph{weighted state} is a triple 
%$\Xi^w:=(\loc, \pv,w)$ 
$\Theta=(\loc, \pv, w)$ 
where $(\loc, \pv)$ is a state and $w\in [0,\infty)$ represents the multiplicative likelihood weight accumulated so far. 


 
%\paragraph{Semantics.} 
Below we specify the semantics of a WPTS. Consider a WPTS $\Pi$ in the form of \eqref{eq:wpts}. The semantics of $\Pi$ is formalized by the infinite sequence $\Gamma=\{\widehat{\Theta}_n=(\widehat{\loc}_n,\widehat{\pv}_n,\widehat{w}_n)\}_{n\ge 0}$ 
%of \emph{random weighted states} 
where each $(\widehat{\loc}_n,\widehat{\pv}_n,\widehat{w}_n)$ is the random weighted state at the $n$th execution step of the WPTS such that $\widehat{\loc}_n$ (resp. $\widehat{\pv}_n$, $\widehat{w}_n$) is the random variable for the location (resp. the random vector 
%of random variables 
for the program valuation, the random variable for the multiplicative likelihood weight) at the $n$th step, respectively. %The initial random state $\widehat{\Theta}_0$ is constant and equals $(\lin,\valin,\win)$. 
%its corresponding stochastic process $\Gamma:=\{\hat{\Xi}_n\}_{n\ge 0}$ on states.
The sequence $\Gamma$ starts with the initial random weighted state 
$\widehat{\Theta}_0=(\widehat{\loc}_0,\widehat{\pv}_0,\widehat{w}_0)$ such that $\widehat{\loc}_0$ is constantly $\lin$, $\widehat{\pv}_0\in \supp{\mu_\mathrm{init}}$ is sampled from the initial distribution $\mu_\mathrm{init}$ and the initial weight $\widehat{w}_0$ is constantly set to $1$\footnote{This follows the traditional setting in e.g.~\cite{Beutner2022b}.}. 
Then, given the current random weighted state $\widehat{\Theta}_n=(\widehat{\loc}_n,\widehat{\pv}_n,\widehat{w}_n)$ at the $n$th step, the next random weighted state $\widehat{\Theta}_{n+1}=(\widehat{\loc}_{n+1},\widehat{\pv}_{n+1},\widehat{w}_{n+1})$ is determined by:
(a) If $\widehat{\loc}_n=\lout$, then $(\widehat{\loc}_{n+1}, \widehat{\pv}_{n+1},\widehat{w}_{n+1})$ takes the same weighted state as $(\widehat{\loc}_n,\widehat{\pv}_n,\widehat{w}_n)$ (i.e., the next weighted state stays at the termination location $\lout$);
(b) Otherwise, $\widehat{\Theta}_{n+1}$ is determined by the following procedure:
\begin{itemize}
\item First, since the WPTS $\Pi$ is deterministic and total, we take the unique transition $\tau=\langle \hat{\loc}_n,\phi,F_1,\dots, F_k \rangle$ such that $\hat{\pv}_n\models\phi$. 
\item Second, we choose a fork $F_j=\langle \loc_j, p_j,\upd_j,\wet_j\rangle$ with probability $p_j$.
\item 
Third, we obtain a sampling valuation $\rv\in \supp{\rdvarjdis}$ 
%over the sampling variables $\rvars$ 
by sampling each $r \in \rvars$ independently w.r.t the probability distribution $\rdvarjdis(r).$
\item Finally, the value of the next random weighted state $(\widehat{\loc}_{n+1}, \widehat{\pv}_{n+1},\widehat{w}_{n+1})$ is determined as that of 
$(\loc'_j, \upd_j(\hat{\pv}_n,\rv),\widehat{w}_n\cdot \wet_j(\widehat{\pv}_n,\rv))$. 
\end{itemize}


Given the semantics, a \emph{program run} of the WPTS $\Pi$ is a concrete instance of $\Gamma$, i.e., an infinite sequence $\omega=\{\Theta_n\}_{n\ge 0}$ of weighted states where each $\Theta_n=(\loc_n,\pv_n,w_n)$ is the concrete weighted state at the $n$th step in this program run with location $\loc_n$, program valuation $\pv_n$ and multiplicative likelihood weight $w_n$. A state $(\loc,\pv)$ is called \emph{reachable} if there exists a program run $\omega=\{\Theta_n\}_{n\ge 0}$ such that $\Theta_n=(\loc,\pv,w_n)$ for some $n$. 


 
\begin{example}\label{ex:pedestrian-run}
Consider the WPTS in \cref{ex:pedestrian-semantics}. Consider an initial program valuation $(1,1,0)$ which means that the initial values of $start,pos,dis$ are $1,1,0$, respectively. Then starting from the initial weighted state $(\lin,(1,1,0),1)$, a program run w.r.t the WPTS semantics above could be 
\[
(\lin,(1,1,0),1)\to (\lin,(1,0.5,0.5),1)\to (\lin,(1,-0.1,1.1),1)\to (\lout,(1,-0.1,1.1),3.9894).\qed
\]
\end{example}

Given an initial program valuation $\valin$ of a WPTS, one could construct a probability space over the program runs by their probabilistic evolution described above and standard constructions such as general state space Markov chains~\cite{meyn2012markov}. We denote the probability measure in the probability space by $\probm_{\valin}(-)$ and the expectation operator by $\expectdist{\valin}{-}$.  



\subsection{Normalised Posterior Distribution}\label{sec2:NPD}


Before presenting the central problem of Bayesian probabilistic programming, i.e., analyzing normalised posterior distribution with our WPTS models, we introduce some technical concepts.

%\paragraph{Termination.}
\begin{definition}[Termination]
The \emph{termination time} of a WPTS
%The \emph{termination time} of the WPTS 
$\Pi$ 
%is a random variable $T$ defined on programs runs given 
is the random variable $T$ given by
%a program run  $\omega=\{\Xi_n=(\loc_n,\pv_n,w_n)\}_{n\in\Nset}$,
%\begin{align*}	
$T(\omega):=\text{min}\{n\in\Nset\mid \loc_n=\lout\}$ for every program run  $\omega=\{(\loc_n,\pv_n,w_n)\}_{n\ge 0}$
%\end{align*}
where $\text{min}\,\emptyset:=\infty$. That is, $T(\omega)$ is the number of steps a program run $\omega$ takes to reach the termination location $\lout$. A WPTS $\Pi$ is \emph{almost-surely terminating} (AST) if $\probm_{\valin}(T<\infty)=1$ for all initial program valuations $\valin\in \supp{\mu_{\mathrm{init}}}$.  
%in the case that the program run never terminates. 
\end{definition}




\begin{definition}[Expected Weights]\label{def:exp-wt}
 Given a WPTS $\Pi$ in the form of \eqref{eq:wpts}, a designated initial program valuation $\valin$ and a measurable subset $\calU\in\Sigma_{\Rset^{|\pvars|}}$, the \emph{expected weight} $\measureSem{\Pi}_{\valin}(\calU)$ 
%$\measureSem{\Pi}(\valin)$ 
%of $\Pi$ w.r.t $\pv$ 
is defined as
%$\measureSem{\Pi}_\calU(\valin):=\expectdist{\valin}{\widehat{w}_T}$. 
$\measureSem{\Pi}_{\valin}(\calU):=\expectdist{\valin}{[\widehat{\pv}_T\in \calU]\cdot\widehat{w}_T}$. 
\end{definition}

By definition, we have that $\widehat{\pv}_T$ (resp. $\widehat{w}_T$) is the random vector (resp. variable) of the program valuation (resp. the multiplicative likelihood weight) at termination, respectively. Thus, $\measureSem{\Pi}_{\valin}(\calU)$ is the expectation of $\widehat{w}_T$ 
%over all program runs 
that start from the state $(\lin,\valin,1)$ and end with $\widehat{\pv}_T\in\calU$. If $\calU=\Rset^{|\pvars|}$, the restriction of $\widehat{\pv}_T\in\calU$ can be removed.

Below we define the normalised posterior distribution (NPD) problem. %under our WPTS semantics. 

 
\begin{definition}[Normalised Posterior Distribution]\label{def:npd}
Given a WPTS $\Pi$ in the form of \eqref{eq:wpts},
%We write $\measureSem{\Pi}(\valin)$ iff $\calU=\Rset^{|\pvars|}$.)
%Then given a probability distribution $\mu$ over initial program valuations, 
the \emph{normalised posterior distribution} (NPD) $\posterior_\Pi$ of $\Pi$ 
%over $U$ 
is defined by:
\begin{align*}
\posterior_{\Pi}(\calU):=\measureSem{\Pi}(\calU)/Z_\Pi\mbox{ for all measurable subsets } \calU\in \Sigma_{\Rset^{|\pvars|}},   
\end{align*}	
where 
$\measureSem{\Pi}(\calU):=\int_{\calV} \measureSem{\Pi}_{\pv}(\calU)\cdot \mu_{\mathrm{init}}(\mathrm{d} \pv)$ is the \emph{unnormalised posterior distribution} w.r.t. $\calU$, $\calV:=\supp{\mu_{\mathrm{init}}}$, %is the support of $\mu_{\mathrm{init}}$
%is the integral of all expected weights with an initial program valuation $\pv\in U$, 
and $Z_\Pi:=\measureSem{\Pi}(\Rset^{|\pvars|})$ is the \emph{normalising constant}.  
The WPTS $\Pi$ is called \emph{integrable} 
%w.r.t a probability distribution (for initial program valuations) 
if we have $0<Z_{\Pi}<\infty$. 
%\pw{Shall we mention that $\measureSem{\Pi}_{\pv}(\calU)$ is an integrable function here?}
\end{definition}

%We call a WPTS $\Pi$ \emph{integrable} 
%w.r.t a probability distribution (for initial program valuations) 
%if the normalising constant is finite, i.e., ~$0<Z_{\Pi}<\infty$. %for any $\pv\in\val{\pvars}$. 
%Given an integrable program, we are interested in deriving lower and upper bounds on the normalised posterior distribution over some measurable set $U\in \Sigma_\Rset$.
\paragraph{Interval Bounds for NPD.} In this work, we consider the automated interval-bound analysis for NPD of a WPTS. Formally, we aim to derive an interval $[l,u]\subseteq [0,\infty)$ 
for an integrable WPTS $\Pi$ and any measurable sets $\calU\in\Sigma_{\Rset^{|\pvars|}}$ as tight as possible such that $l\le \posterior_{\Pi}(\calU) \le u$. 
%$l,u$ are called \emph{interval bounds} for the NPD $\posterior_{\Pi}(\calU)$. 
%To achieve this, in the following (\cref{sec:math}) we develop approaches to obtain interval bounds for expected weights as $\measureSem{\Pi}(\calU)$ and $Z_\Pi$ are integrations of expected weights over $\calV$. 
 



To achieve interval bounds for NPD, below we introduce the construction of a new WPTS $\Pi_\calU$ based on the original WPTS $\Pi$ and a measurable set $\calU\in \Sigma_{\Rset^{|\pvars|}}$.  

\paragraph{Construction of $\Pi_\calU$.} Consider a probabilistic program $P$ and its WPTS $\Pi$, given a measurable set $\calU\in\Sigma_{\Rset^{|\pvars|}}$, we construct a new program $P_\calU$ by adding a conditional branch of the form ``\textbf{if} $\pv_T\notin\calU$ \textbf{then} \textbf{score}($0$) \textbf{fi}'' immediately after the termination of $P$ and obtain the WPTS $\Pi_\calU$ of $P_\calU$. Therefore, $\Pi$ and $\Pi_\calU$ have the same initial probability distribution $\mu_{\mathrm{init}}$ and the same finite support $\calV=\supp{\mu_{\mathrm{init}}}$. The following proposition shows that interval-bound analysis for NPD can be reduced to interval-bound analysis for expected weights in the form $\llbracket \Pi\rrbracket_{\pv}(\Rset^{|\pvars|})$. 

\begin{proposition}\label{prop:unnorm-norm}
   Given a WPTS $\Pi$ in the form of \eqref{eq:wpts}, a measurable set $\calU\in\Sigma_{\Rset^{|\pvars|}}$ and the WPTS $\Pi_\calU$ constructed as above, we have that $\llbracket \Pi \rrbracket_{\pv}(\calU)=\llbracket \Pi_\calU\rrbracket_{\pv}(\Rset^{|\pvars|})$ for any $\pv\in\calV=\supp{\mu_{\mathrm{init}}}$. Furthermore,
   if there exist intervals $[l_1,u_1],[l_2,u_2]\subseteq [0,\infty)$ such that $\llbracket \Pi_\calU\rrbracket_{\pv}(\Rset^{|\pvars|})\in [l_1,u_1]$ and $\llbracket \Pi\rrbracket_{\pv}(\Rset^{|\pvars|})\in [l_2,u_2 ]$ for any $\pv\in\calV$, then we have two intervals $[l_\calU,u_\calU],[l_Z,u_Z]\subseteq [0,\infty)$ such that the unnormalised posterior distribution $\llbracket \Pi\rrbracket (\calU)\in [l_\calU,u_\calU]$ and the normalising constant $Z_\Pi\in [l_Z,u_Z]$. Moreover, if $\Pi$ is integrable, i.e., $[l_Z,u_Z]\subseteq (0,\infty)$, then we can obtain the NPD $\posterior_{\Pi}(\calU)\in [\frac{l_\calU}{u_Z},\frac{u_\calU}{l_Z}]$.\footnote{The interval bounds derived in this manner may be loose, but they are definitely correct.}  Note that by \cref{def:npd}, $l_\calU=\int_\calV l_1 \cdot\mu_{\mathrm{init}}(\mathrm{d} \pv)$, $u_\calU=\int_\calV u_1 \cdot\mu_{\mathrm{init}}(\mathrm{d} \pv)$, $l_Z=\int_\calV l_2 \cdot\mu_{\mathrm{init}}(\mathrm{d} \pv)$ and $u_Z=\int_\calV u_1 \cdot\mu_{\mathrm{init}}(\mathrm{d} \pv)$.

\end{proposition}

The proof of \cref{prop:unnorm-norm} is relegated to \cref{app:sec2-prop}. In the following, we will develop approaches to obtain interval bounds for expected weights.
%in the form $\llbracket \Pi \rrbracket_{\pv}(\Rset^{|\pvars|})$ where $\pv$ is an initial program valuation.











\section{Contextual Bandits with Preference-Based Active Queries} 



\begin{algorithm}[t]
\begin{algorithmic}[1]
   \caption{Active preference qUeRy fOR contextual bAndits (AURORA)}
   \label{alg:cb}
	\REQUIRE Function class $\+F$, confidence parameter $\beta=\frac{4\Upsilon}{\alpha}+\frac{16+24\alpha}{\alpha^2}\log\big(4\delta^{-1}\log(T)\big)$.
	\STATE Online regression oracle produces $f_1$.
   	\FOR{$t=1,2,\dots,T$}
   		\STATE Learner receives context $x_t$, and computes the version space 
   		\begin{align*}
   			\+F_t\gets\Bigg\{f\in\+F:\sum_{s=1}^{t-1} Z_s\Big(f(x_s,a_s,b_s)-f_s(x_s,a_s,b_s)\Big)^2\leq \beta\Bigg\}.
   		\end{align*}
   		and the candidate arm set $\+A_t\gets\{\pi_f(x_t):\forall f\in\+F_t\}$.
   		\STATE Learner decides whether to query $Z_t\gets\indic\{|\+A_t|>1\}$.
   		\IF{$Z_t=1$}  
   			\STATE Set $w_t\gets\sup_{a,b\in\+A_t}\sup_{f,f'\in\+F_t} f(x_t,a,b)-f'
   			(x_t,a,b)$, and $\lambda_t\gets\indic\{\sum_{s=1}^{t-1}Z_s w_s\geq\sqrt{AT/\beta}\}$.
   			\IF{$\lambda_t=0$}
   				\STATE $p_t\gets\@{Uniform}(\+A_t)$.
   			\ELSE
   				\STATE $\gamma_t\gets\sqrt{AT/\beta}$.
   				\STATE Let $p_t$ be a solution of $\max_{a\in\+A_t}\sum_b f_t(x_t,a,b)p_t(b)+\frac{2}{\gamma_t p_t(a)}\leq\frac{5A}{\gamma_t}$.\alglinelabel{line:compute-p}
   			\ENDIF
   			\STATE Learner samples $a_t,b_t\sim p_t$ independently and receives the feedback $y_t$. 
			\STATE Learner feeds $((x_{t},a_{t},b_{t}),y_{t})$ to the online regression oracle which returns $f_{t+1}$.
   		\ELSE
   			\STATE Learner sets $a_t$ and $b_t$ to be the only action in $\+A_t$, and plays them. \alglinelabel{line:no-query-play-only-arm}
                \STATE $f_{t+1}\gets f_t$.
   		\ENDIF	
   	\ENDFOR 
\end{algorithmic} 
\end{algorithm}


We first present our algorithm, named \textsc{AURORA}, for contextual dueling bandits, as shown in \Cref{alg:cb}. At each round $t\in[T]$, the online regression oracle outputs a predictor $f_t$, using which the learner constructs a version space $\+F_t$ containing all functions close to past predictors on observed data. Here, the threshold 
$	
\beta$ set to $4\Upsilon/\alpha+(16+24\alpha)\log\big(4\delta^{-1}\log(T)\big)/\alpha^2
$ ensures that $f^\star\in\+F_t$ for any $t\in[T]$ with probability at least $1-\delta$ (\Cref{lem:pointwise-bound}). Thus, $\+A_t$ is non-empty for all \(t \in [T]\) and correspondingly  Line~\ref{line:no-query-play-only-arm} is well defined. The learner then forms a candidate arm set $\+A_t$ consisting of greedy arms induced by all functions in the version space. When $|\+A_t|=1$, the only arm in the set is the optimal arm since $f^\star\in\+F_t$, and thus no query is needed ($Z_t=0$). However, when $|\+A_t|>1$, any arm in $\+A_t$ could potentially be the optimal arm, and thus the learner needs to make a comparison query to obtain more information. 

Next, we explain the strategy used by the learner for making a query. Firstly, the learner computes $w_t$, which represents the ``width'' of the version space. Specifically, $w_t$ overestimates the instantaneous regret for playing any arm in $\+A_t$ (\Cref{lem:regret-bounded-by-w}). Then, the learner defines $\lambda_t$ that indicates if the estimated cumulative regret  $\sum_{s=1}^{t-1}Z_w w_s$ has exceeded $\sqrt{AT/\beta}$. Note that $Z_t$ is multiplied to $w_t$ since no regret is incurred when $Z_t=0$. The strategy to choose the actions (to be queried) for different values of $\lambda_t$ are as follows: 

\begin{itemize}[leftmargin=*]
	\item If $\lambda_t=0$, the cumulative reward has not yet exceeded $\sqrt{AT/\beta}=O(\sqrt{T})$, so the learner will explore as much as possible by uniform sampling from $\+A_t$.
	\item If $\lambda_t=1$, the regret may have reached $O(\sqrt{T})$, and therefore the learner uses a technique similar to inverse gap weighting (IGW), as inspired by \cite{saha2022efficient}, to achieve a better balance between exploration and exploitation. Specifically, the learner solves the  convex program\footnote{It is convex as it can be written as $|\+A_t|$ convex constraints: $\sum_b f_t(x_t,a,b)p_t(b)+\frac{2}{\gamma_t p_t(a)}\leq\frac{5A}{\gamma_t},\forall a\in\+A_t$.} in Line~\ref{line:compute-p}, which is feasible and whose solution $p_t$ satisfies (see \Cref{lem:igw})
	\begin{align*}
		\E_{a\sim p_t}\Big[f^\star(x_t,\pi_{f^\star}(x),a)\Big]
		=
		O\left(
		\gamma_t\E_{a,b\sim p_t}\Big[\big(f_t(x_t,a,b)-f^\star(x_t,a,b)\big)^2\Big]+\frac{A}{\gamma_t}\numberthis\label{eq:informal-igw}
		\right).
	\end{align*}
As a result of the above relation, we note that one can convert the instantaneous regret to the point-wise error between the predictor $f_t$ and the truth $f^\star$ plus an additive $A/\gamma_t$. This allows us to bound the cumulative point-wise error by the regret of the online regression oracle. In the special case, when there exists a ``reward function'' $r:\+X\times\+A\rightarrow[0,1]$ for each $f\in\+F$ such that $f(x,a,b)=r(x,a)-r(x,b)$ (\Cref{lem:ex-r-r}), the solution $p_t$ can be directly written as 
	\begin{align*}
		p_t(a)=
		\begin{cases} 
			\frac{1}{A+\gamma_t\big(r_t(x_t,\pi_{f_t}(x_t))-r_t(x_t,a)\big)} & a\neq\pi_{f_t}(x_t)\\
			1-\sum_{a'\neq\pi_{f_t}(x_t)}p_t(a') & a=\pi_{f_t}(x_t)
		\end{cases},
	\end{align*}
	where $r_t$ is the reward function associated with $f_t$, i.e., $f_t(x,a,b)=r_t(x,a)-r_t(x,b)$.
	This is the standard IGW exploration strategy \citep{foster2020beyond} and leads to the same guarantee as \eqref{eq:informal-igw} (see \Cref{lem:igw-r-version}).
\end{itemize}
\subsection{Theoretical Analysis}

Towards the theoretical guarantees of \Cref{alg:cb}, we employ two quantities to characterize a contextual bandit instance: the uniform gap and the eluder dimension, which are introduced below.
\begin{assumption}[Uniform gap]\label{asm:uniform-gap}
We assume the optimal arm $\pi_{f^\star}(x)$ induced by $f^\star$ under any context $x\in\+X$ is unique. Further, we assume a uniform gap $\Delta:=\inf_x \inf_{a\neq\pi_{f^\star}(x)} f^\star(x,\pi_{f^\star}(x),a)>0$.
\end{assumption}

We note that the existence of a uniform gap is a standard assumption in the literature of contextual bandits \citep{dani2008stochastic,abbasi2011improved,audibert2010best,garivier2019explore,foster2020beyond,foster2020instance}. Next, we introduce the eluder dimension \citep{russo2013eluder} and begin by defining ``$\epsilon$-dependence''.
\begin{definition}[$\epsilon$-dependence]
Let $\+G\subseteq\+X\rightarrow\=R$ be any function class. We say an element $x\in\+X$ is $\epsilon$-dependent on $\{x_1,x_2,\dots,x_n\}\subseteq\+X$ with respect to $\+G$ if any pair of functions $g,g'\in\+G$ satisfying $\sum_{i=1}^n(g(x_i)-g'(x_i))\leq\epsilon^2$ also satisfies $g(x)-g'(x)\leq\epsilon$. Otherwise, we say $x$ is $\epsilon$-independent of $\{x_1,x_2,\dots,x_n\}$.
\end{definition}

\begin{definition}[Eluder dimension]
	The $\epsilon$-eluder dimension of a function class $\+G\subseteq\+X\rightarrow\=R$, denoted by $\@{dim}_E(\+G,\epsilon)$, is the length $d$ of the longest sequence of elements in $\+X$ satisfying that there exists some $\epsilon'\geq\epsilon$ such that every element in the sequence is $\epsilon'$-independent of its predecessors.
\end{definition}
Eluder dimension is a standard complexity measure for function classes and has been used in the literature of bandits and RL extensively \citep{chen2022human,osband2014model,wang2020reinforcement,foster2020instance,wen2013efficient,jain2015learning,ayoub2020model,ishfaq2021randomized,huang2021towards}. Examples where the eluder dimension is small include linear functions, generalized linear models, and functions in Reproducing Kernel Hilbert Space (RKHS).  

Given these quantities, we are ready to state our main results. The proofs are provided in \Cref{sec:missing-pf}.

\begin{theorem}\label{thm:cb-regret}
Under \Cref{asm:properties,asm:oracle-regret,asm:uniform-gap}, \Cref{alg:cb} guarantees the following upper bounds of the regret and the number of queries:
\begin{align*}
	&\@{Regret}_T^{\@{CB}}=
	\widetilde{O}\left(
	\min\left\{
	\sqrt{AT\beta}
	,\;
	\frac{A^2\beta^2\@{dim}_E\left(\+F,\Delta\right)}{\Delta}
	\right\}
	\right),\\
	&\@{Queries}_T^{\@{CB}}=
	\widetilde{O}\left(
	\min\left\{T
	,\;
	\frac{A^3\beta^3 \@{dim}^2_E\left(\+F,\Delta\right)}{\Delta^2}
	\right\}
	\right)
\end{align*}
with probability at least $1-\delta$. We recall that $\beta=O(\alpha^{-1}\Upsilon+\alpha^{-2}\log(\delta^{-1}\log(T)))$, and $\alpha$ denotes the coefficient of strong convexity of $\Phi$. We have hidden logarithmic terms in the upper bounds for brevity. 
\end{theorem}
When the loss $\ell_\phi$ is either square loss or logistic loss (\Cref{ex:sq-loss,ex:log-loss}), the parameter $\beta$ is logarithmic in $T$. In such cases, the regret is $\~O(\min\{\sqrt{T},\@{dim}_E\left(\+F,\Delta\right)/\Delta\})$ and the number of queries is $\~O(\min\{T,\@{dim}^2_E(\+F,\Delta)/\Delta^2\})$, ignoring $A$ and logarithmic terms. Both consist of two components: the worst-case and the instance-dependent upper bounds. The worst-case bound provides a guarantee under all circumstances, while the instance-dependent one may significantly improve the upper bound when the underlying problem is well-behaved (i.e., has a small eluder dimension and a large gap).


\paragraph{Intuition of proofs.} We next provide intuition for why our algorithm has the aforementioned theoretical guarantees. First, we observe that from the definition of $\lambda_t$, the left term inside the indicator is non-decreasing, which allows us to divide rounds into two phases. In the first phase, $\lambda_t$ is always 0, and then at some point, it changes to 1 and remains 1 for the rest rounds. After realizing this, we first explain the intuition of the worst-case regret. In the first phase, as $w_t$ is an overestimate of the instantaneous regret (see \Cref{lem:regret-bounded-by-w}), the accumulated regret in this phase cannot exceed $O(\sqrt{T})$. In the second phase, we adapt the analysis of IGW to this scenario to obtain an $O(\sqrt{T})$ upper bound. A similar technique has been used in ~\cite{saha2022efficient,foster2020instance}. As the regret in both phases is at most $O(\sqrt{T})$, the total regret cannot exceed $O(\sqrt{T})$. Next, we explain the intuition of instance-dependent regret. Due to the existence of a uniform gap $\Delta$, we can first prove that as long as $|\+A_t|>1$, we must have $w_t\geq\Delta$ (see \Cref{lem:width-lower-bound}). This means that for all rounds that may incur regret, the corresponding width is at least $\Delta$. However, this cannot happen too many times as this frequency is bounded by the eluder dimension, which leads to an instance-dependent regret upper bound. Leveraging a similar technique, we can also obtain an upper bound on the number of queries. \looseness=-1

\paragraph{Comparion to \textsc{MinMaxDB} \citep{saha2022efficient}.}
In this prior work, the authors assume that $\Pr(y=1\given x,a,b)=(f^\star(x,a,b)+1)/2$, which is a specification of our feedback model (\Cref{ex:sq-loss}). 
While our worst-case regret bound matches their regret bound, our paper improves upon their results by having an additional instance-dependent regret bound that depends on the eluder dimension and gap. Furthermore, we also provide bounds on the query complexity which could be small for benign instances while \textsc{MinMaxDB} simply queries on every round. 

\paragraph{Comparion to \textsc{AdaCB} \citep{foster2020instance}.}
Our method shares some similarities with \citet{foster2020instance}, especially in terms of theoretical results, but differs in two aspects: (1) they assume regular contextual bandits where the learner observes the reward directly, while we assume preference feedback, and (2) they assume a stochastic setting where contexts are drawn i.i.d., but we assume that the context is adversarially chosen. While these two settings may not be directly comparable, it should be noted that \citep{foster2020instance} do not aim to minimize query complexity. 

\paragraph{Lower bounds.}
To understand whether our algorithm attains tight upper bounds, we provide the following lower bound which follows from a reduction from regular multi-armed bandits to contextual dueling bandits.  
\begin{theorem}[Lower bounds]\label{thm:lower-bound}
The following two claims hold:
\begin{enumerate}
\item[(1)] For any algorithm, there exists an instance that leads to $\@{Regret}^{\@{CB}}_T=\Omega(\sqrt{AT})$;
\item[(2)] For any algorithm achieving a worse-case expected regret upper bound in the form of $\E[\@{Regret}^{\@{CB}}_T]= O(\sqrt{AT})$, there exists an instance with gap $\Delta=\sqrt{A/T}$ that results in $\E[\@{Regret}^{\@{CB}}_T]=\Omega(A/\Delta)$ and $\E[\@{Queries}^{\@{CB}}_T]=\Omega(A/\Delta^2)=\Omega(T)$.
\end{enumerate}
\end{theorem}

By relating these lower bounds to \Cref{thm:cb-regret}, we conclude that our algorithm achieves a tight dependence on the gap $\Delta$ and $T$, up to logarithmic factors, in both the regret and query complexity upper bounds. Furthermore, as an additional contribution, we establish an alternative lower bound in Section \ref{sec:lb-2} by conditioning on the limit of regret, rather than the worst-case regret as assumed in Theorem \ref{thm:lower-bound}. %



\paragraph{Results without the uniform gap assumption.}
We highlight that \Cref{thm:cb-regret} can naturally extend to scenarios where a uniform gap does not exist (i.e., when \Cref{asm:uniform-gap} is not satisfied) without any modifications to the algorithm. The result is stated below, which is analogous to \Cref{thm:cb-regret}.
\begin{theorem}\label{thm:cb-general-regret}
Under \Cref{asm:properties,asm:oracle-regret}, \Cref{alg:cb} guarantees the following upper bounds of the regret and the number of queries:
\begin{align*}
	&\@{Regret}_T^{\@{CB}}=
	\widetilde{O}\left(
	\min\left\{
	\sqrt{AT\beta}
	,\,
	\min_{\epsilon>0}\left\{
	T_\epsilon\beta+\frac{A^2\beta^2\@{dim}_E\left(\+F,\epsilon\right)}{\epsilon}
	\right\}\right\}
	\right),\\
	&\@{Queries}_T^{\@{CB}}=
	\widetilde{O}\left(
	\min\left\{
	T
	,\,
	\min_{\epsilon>0}\left\{
	T_\epsilon^2\beta/A+\frac{A^3\beta^3 \@{dim}^2_E\left(\+F,\epsilon\right)}{\epsilon^2}
	\right\}	\right\}
	\right)
\end{align*}
with probability at least $1-\delta$. Here we define the gap of context $x$ as $\@{Gap}(x)\coloneqq\min_{a\neq\pi_{f^\star}(x)} f^\star(x,\pi_{f^\star}(x), a)$ and the number of rounds where contexts have small gap as $T_\epsilon\coloneqq\sum_{t=1}^T \indic\{\@{Gap}(x_t)\leq\epsilon\}$. We also recall that $\beta=O(\alpha^{-1}\Upsilon+\alpha^{-2}\log(\delta^{-1}\log(T)))$, and $\alpha$ denotes the coefficient of strong convexity of $\Phi$. %
\end{theorem}
Compared to \Cref{thm:cb-regret}, the above result has an extra gap-dependent term defined as $T_\epsilon$. Here $\epsilon$ denotes a gap threshold, and $T_\epsilon$ measures how many times the context falls into a small-gap region. We highlight that $T_\epsilon$ is small under certain conditions such as the Tsybakov noise condition \citep{tsybakov2004optimal}. It is also worth mentioning that our algorithm is agnostic to $\epsilon$, thus allowing us to take the minimum over all $\epsilon>0$. 

\paragraph{Comparion to \textsc{SAGE-Bandit} \citep{sekhari2023selective}.}
\Cref{thm:cb-general-regret} bears similarity to Theorem 4 in \citet{sekhari2023selective}, which examines active queries in contextual bandits with standard reward signal (\(0-1\) reward). It is worth noting that although our result looks slightly worse in terms of the factor $A$ (the number of actions), we believe that this inferiority is reasonable since our approach requires two actions to form a query, thus analytically expanding the action space to $\+A^2$. Whether this dependency can be improved remains a question for future investigation. 




























\section{Imitation Learning with Preference-Based Active Queries}

\begin{algorithm}[t]
\begin{algorithmic}[1]
\caption{Active preference qUeRy fOR imitAtion lEarning (AURORAE)}\label{alg:il}
\REQUIRE Function class  $\+F_0,\+F_1,\dots,\+F_{H-1}$, confidence parameter $\beta$.
\STATE Learner creates $H$ instances of \Cref{alg:cb}: $\textsc{AURORA}_h(\+F_h,\beta)$ for $h=0,1,\dots,H-1$.
   	\FOR{$t=1,2,\dots,T$}
   		\STATE Learner receive initial state $x_{t,0}$.
		\FOR{$h=0,1,\dots,H-1$}
			\STATE Learner feeds $x_{t,h}$ to $\textsc{AURORA}_h(\+F_h,\beta)$, and receives back 
		 $a_{t,h}$, $b_{t,h}$, $Z_{t,h}$.
			\IF{$Z_{t,h}=1$}
				\STATE Learner receives feedback $y_{t,h}$.
				\STATE Learner feeds $((x_{t,h},a_{t,h},b_{t,h}), y_{t,h})$
            to $\textsc{AURORA}_h(\+F_h,\beta)$ \asedit{to update its online regression oracle and local variables.} 
			\ENDIF
			\STATE Learner executes $a\sim \text{Uniform}(\{a_{t,h},b_{t,h}\})$ and transits to $x_{t,h+1}$. 
		\ENDFOR 
   	\ENDFOR 
\end{algorithmic}
\end{algorithm}

 In this section, we introduce our second algorithm, which is presented in \Cref{alg:il} for imitation learning. In essence, the learner treats the MDP as a concatenation of $H$ contextual bandits and runs an instance of \textsc{AURORA} (\Cref{alg:cb}) for each time step. Specifically, the learner first creates $H$ instances of \textsc{AURORA}, denoted by $\textsc{AURORA}_h$ (for $h=0,\dots,H-1$). Here, \textsc{AURORA}\(_h\) should be thought of as an interactive program that takes the context $x$ as input and outputs $a$, $b$, and $Z$. At each episode $t$, and each step $h$ therein, the learner first feeds the current state $x_{t,h}$ to $\textsc{AURORA}_h$ as the context; then, $\textsc{AURORA}_h$ decides whether to query (i.e.~\(Z_{t, h}\)) and returns the actions \(a_{t,h}\) and \(b_{t,h}\). If it decides to make a query, the learner will ask for the feedback $y_{t,h}$ on the proposed actions $a_{t,h}, b_{t,h}$, and provide the information $((x_{t,h},a_{t,h},b_{t,h}), y_{t,h})$ back to $\textsc{AURORA}_h$ to update its online regression oracle (and other local variables). We recall that the noisy binary feedback $y_{t,h}$ is sampled as  $y_{t,h}\sim \phi(Q^{\pi_e}_h(x_{t,h},a_{t,h})- Q^{\pi_e}_h(x_{t,h},b_{t,h}))$, and also emphasize that the learner neither has access to  $a\sim \pi_e(x_{t,h})$ like in \textsc{DAgger} \citep{ross2011reduction}  nor reward-to-go like in  \textsc{AggreVaTe(D)} \citep{ross2014reinforcement,sun2017deeply}. Finally, the learner chooses one of the two actions uniformly at random, executes it in the underlying MDP, and transits to the next state $x_{t,h+1}$ in the episode. The above process is then repeated with $\textsc{AURORA}_{h+1}$ till the episode ends. We name this algorithm \textsc{AURORAE}, the plural form of \textsc{AURORA}, which signifies that the algorithm is essentially a stack of multiple \textsc{AURORA} instances.

\subsection{Theoretical Analysis}

As \Cref{alg:il} is essentially a stack of \Cref{alg:cb}, we can inherit many of the theoretical guarantees from the previous section. To state the results, we first extend \Cref{asm:uniform-gap} into imitation learning. 
\begin{assumption}[Uniform Gap]\label{asm:uniform-gap-il} \asedit{Let \(f_h^\star\) be defined such that for any \(x \in \+X\), \(a, b \in \+A^2\), \(f_h^\star(x, a, b) = Q_h^{\pi_e}(x, a) -  Q_h^{\pi_e}(x, b)\).}
For all $h$, we assume the optimal action for $f_h^\star$ under any state $x\in\+X$ is unique. Further, we assume a uniform gap $\Delta:=\inf_h \inf_x \inf_{a\neq\pi_{f_h^\star}(x)} f_h^\star(x,\pi_{f_h^\star}(x),a)>0$.
\end{assumption}

This assumption essentially says that $Q^{\pi_e}_h$ has a gap in actions. 
We remark that, just as \Cref{asm:uniform-gap} is a common condition in the bandit literature, \Cref{asm:uniform-gap-il} is also common in MDPs \citep{du2019provably,foster2020instance,simchowitz2019non,jin2020simultaneously,lykouris2021corruption, he2021logarithmic}. The theoretical guarantee for \Cref{alg:il} is presented in \Cref{thm:il-regret}. We note a technical difference between this result and \Cref{thm:cb-regret}: although we treat the MDP as a concatenation of $H$ contextual bandits, the instantaneous regret of imitation learning is defined as the performance gap between the combined policy $\pi_t$ derived from the $H$ instances as a cohesive unit and the expert policy. This necessitates the use of performance difference lemma (\Cref{lem:pdl}) to get a unified result.

\begin{theorem}\label{thm:il-regret}
Under \Cref{asm:properties,asm:uniform-gap-il,asm:oracle-regret}, \Cref{alg:il} guarantees the following upper bounds of the regret and the number of queries:
\begin{align*}
	&\@{Regret}_T^{\@{IL}}\leq
	\widetilde{O}\left(
	H\cdot\min\left\{
	\sqrt{AT\beta}
	,\;
	\frac{A^2\beta^2\@{dim}_E\left(\+F,\Delta\right)}{\Delta}
	\right\}
	\right)-\@{Adv}_T,\\
	&\@{Queries}_T^{\@{IL}}\leq
	\widetilde{O}\left(
	H\cdot\min\left\{T
	,\;
	\frac{A^3\beta^3 \@{dim}^2_E\left(\+F,\Delta\right)}{\Delta^2}
	\right\}
	\right)
\end{align*}
with probability at least $1-\delta$. Here
$
\@{Adv}_T\coloneqq\sum_{t=1}^T \sum_{h=0}^{H-1} \E_{x_{t,h}\sim d^{\pi_t}_{x_{t,0},h}}[\max_a A^{\pi_e}_h(x_{t,h},a)]
$ is non-negative,
and $d^{\pi_t}_{x_{t,0},h}(x)$ denotes the probability of $\pi_t$ \footnote{Policy $\pi_t$ consists of $H$ time-dependent policies $\pi_{t,1},\dots, \pi_{t,H}$, where each $\pi_{t,h}$ is defined implicitly via $\textsc{AURORA}_h$, i.e., $\pi_{t,h}$ generates action as follows: given $x_{t,h}$, $\textsc{AURORA}_h$ recommends $a_{t,h},b_{t,h}$,  followed by uniformly sampling an action from $\{a_{t,h},b_{t,h}\}$. } reaching the state $x$ at time step $h$ starting from inital state $x_{t,0}$. In the above,  $\beta=O(\alpha^{-1}\Upsilon+\alpha^{-2}\log(H\delta^{-1}\log(T)))$ and $\alpha$ denotes the coefficient of strong convexity of $\Phi$.
\end{theorem}


Compared to \Cref{thm:cb-regret}, the main terms of the upper bounds for imitation learning are precisely the bounds in \Cref{thm:cb-regret} multiplied by $H$. In the proof presented in \Cref{sec:pf-thm-il-regret}, we use the performance difference lemma to reduce the regret of imitation learning to the sum of the regret of $H$ contextual dueling bandits, which explains this additional factor of $H$. 

Another interesting point is that the main term of the regret upper bound is subtracted by a non-negative term $\@{Adv}_T$, which measures the degree to which we can \textit{outperform} the expert policy. This means that our algorithm not only competes with the expert policy but can also surpass it to some extent. This guarantee is stronger than that of \textsc{DAgger} \citep{ross2011reduction} in that \textsc{DAgger} cannot ensure the learned policy is better than the expert policy regardless of how suboptimal the expert may be. While this may look surprising at first glance since we are operating under a somewhat weaker query mode than that of \textsc{DAgger}, we note that by querying experts for comparisons on pairs of actions with feedback sampling as $y\sim \phi( Q^{\pi_e}(x,a) - Q^{\pi_e}(x,b))$, it is possible to identify the action that maximizes $Q^{\pi_e}(x,a)$ (even if we cannot identify the value $Q^{\pi_e}(x,a)$). 
Finally, we remark that our worst-case regret bound is similar to that of \citet{ross2014reinforcement,sun2017deeply}, which can also outperform a suboptimal expert but require access to both expert's actions and reward signals---a much stronger query model than ours. \looseness=-1 






\section{Discussion}
\label{sec: discussion}
\kmsdelete{In this work} We study \kmsreplace{Fairness-Aware PAC learning}{Fair-ERM} in the malicious noise model, and  in some cases allow 
the learner to maintain optimal overall accuracy despite the signal in Group $B$ being almost entirely washed out.
%when we allow learners to use the
%$\PQ$ randomized expansion of the hypothesis class $\mathcal{H}$
In particular we show that different fairness constraints have fundamentally different behavior in the presence of Malicious Noise, in terms of the amount of accuracy loss that a given level of Malicious Noise could cause a fairness-constrained learner to incur. 
The key to achieving our results, which are more optimistic than those in \cite{lampert}, is allowing for improper learners using the (P,Q)-randomized expansions of the given class $\mathcal{H}$.
%We \kmsreplace{present a picture of the}{prove upper and lower bounds on}
%accuracy loss for a range of fairness notions, given \kmsreplace{this simple randomization step.}{learning over $\PQ$.
%In general our results indicate Fair-ERM (given learning over $\PQ$) is more robust than claimed in \cite{lampert}.
The type of smoothness we create by using $\PQ$ seems to be a natural property that is likely shared by many natural hypothesis classes.

Fairness notions are motivated as a response to learned disparities when there is \kmsdelete{data corruption or} systemic error affecting \kmsdelete{the data for}
one group. 
Fairness notions are supposed to mitigate this by ruling out classifiers that have worse performance on a sub-group. 
This can peg both classifiers at a lower level of performance \kmsdelete{(e.g that the lower subgroup)} in order to \emph{motivate} \cite{hardt16} improving the data collection or labelling process to obtain more reliable performance. 
%So in \kmsreplace{some}{a} sense, sensitivity of the fairness notion to poor sub-group performance caused by malicious noise is the \textit{point} of fairness constraints! 
However, it also desirable that fairness constraints perform gracefully when subject to Malicious Noise because fairness constraints will be used in contexts where the data is unreliable and noisy and this might not be known to the learner.
This tension, exposed by our work, motivates 
%a revisiting of fairness notions from first principles approach and trying to axiomatize the 
%desired properties of a fairness intervention a la cryptography and privacy. \footnote{Work in multi-calibration \cite{multicalib} is a viable direction for this problem but it is unclear how 
%that and related notions behave with unreliable data. }
on going work studying the sensitivity level of fairness constraints. 
%If we we are to take a view, if a classifier is deployed 






\bibliography{references.bib}
\bibliographystyle{plainnat}


\newpage
\appendix


\section{Preliminaries}

\begin{lemma}[{\citet[Lemma 3]{kakade2008generalization}}]\label{lem:free2}
Suppose $X_1, \ldots, X_T$ is a martingale difference sequence with $\left|X_t\right| \leq b$. Let
$$
\operatorname{Var}_t X_t=\operatorname{Var}\left(X_t \mid X_1, \ldots, X_{t-1}\right)
$$
Let $V=\sum_{t=1}^T \operatorname{Var}_t X_t$ be the sum of conditional variances of $X_t$ 's. Further, let $\sigma=\sqrt{V}$. Then we have, for any $\delta<1 / e$ and $T \geq 3$,
$$
\Pr\left(\sum_{t=1}^T X_t>\max \{2 \sigma, 3 b \sqrt{\ln (1 / \delta)}\} \sqrt{\ln (1 / \delta)}\right) \leq 4 \ln (T) \delta.
$$	
\end{lemma}


\begin{lemma}[{\citet[Lemma 3]{foster2020beyond}}]\label{lem:igw-general}
	For any vector $\^y\in[0,1]^A$, if we define $p$ to be
	\begin{align*}
		p(a)
=		\begin{cases}
			\frac{1}{A+\gamma \big(\^y(\^a)-\^y(a)\big)} & \text{if } a\not=\^a,\\
			1-\sum_{a\not=\^a}p(a) & \text{if } a=\^a
		\end{cases}
	\end{align*}
	where $\^a=\argmax_a \^y(a)$, then for any $y^\star\in[0,1]^A$ and $\gamma>0$, we have
	$$
		\E_{a\sim p}\left[\Big(y^\star(a^\star)-y^\star(a)\Big)-\gamma\Big(\^y(a)-y^\star(a)\Big)^2\right]\leq \frac{A}{\gamma}.
	$$
\end{lemma}

\begin{lemma}[{\citet[Lemma 2]{zhu2022efficient}}]\label{lem:freedman}
Let $(Z_t)_{t \leq T}$ to be real-valued sequence of positive random variables adapted to a filtration $\mathfrak{F}_t$. If $\left|Z_t\right| \leq B$ almost surely, then with probability at least $1-\delta$,
$$
\sum_{t=1}^T Z_t \leq \frac{3}{2} \sum_{t=1}^T \mathbb{E}_t\left[Z_t\right]+4 B \log \left(2 \delta^{-1}\right),
$$
and
$$
\sum_{t=1}^T \mathbb{E}_t\left[Z_t\right] \leq 2 \sum_{t=1}^T Z_t+8 B \log \left(2 \delta^{-1}\right).
$$
\end{lemma}














\begin{lemma}[Performance difference lemma \citep{agarwal2019reinforcement}]\label{lem:pdl}
	For any two policies $\pi$ and $\pi'$ and any state $x_0\in\+X$, we have
	\begin{align*}
		V^\pi_0(x_0)-V^{\pi'}_0(x_0)
		=
		\sum_{h=0}^{H-1} \E_{x_h,a_h\sim d^\pi_{x_0,h}} \big[A^{\pi'}_h(x_h,a_h)\big]
	\end{align*}
	where $A^\pi_h(x,a)=Q^\pi_h(x,a)-V^\pi_h(x,a)$ and $d^\pi_{x_0,h}(x,a)$ is the probability of $\pi$ reaching the state-action pair $(x,a)$ at time step $h$ starting from initial state $x_0$.
\end{lemma}






\begin{lemma}\label{lem:kl-bern}
	For any two Bernoulli distributions $\@{Bern}(x)$ and $\@{Bern}(y)$ with $x,y\in[b,1-b]$ for some $0<b\leq 1/2$, the KL divergence is bounded as
	\begin{align*}
		\@{KL}\Big(\@{Bern}(x),\@{Bern}(y)\Big)\leq \frac{2(x-y)^2}{b}.
	\end{align*}
\end{lemma}
\begin{proof}[Proof of \Cref{lem:kl-bern}]
Denote $\Delta=x-y$. Then, by definition, we have
	\begin{align*}
		\@{KL}\Big(\@{Bern}(x),\@{Bern}(y)\Big)
		= & x\ln \frac{x}{y} + (1-x)\ln \frac{1-x}{1-y}\\
		= & x\ln \frac{x}{x-\Delta} + (1-x)\ln \frac{1-x}{1-x+\Delta}\\
		= & x\ln \left(1+\frac{\Delta}{x-\Delta}\right) + (1-x)\ln\left(1- \frac{\Delta}{1-x+\Delta}\right)
	\end{align*}
	Since $\ln(1+x)\leq x$ for all $x>-1$, we have
	\begin{align*}
		\@{KL}\Big(\@{Bern}(x),\@{Bern}(y)\Big)
		\leq & x\cdot \frac{\Delta}{x-\Delta} - (1-x)\cdot \frac{\Delta}{1-x+\Delta}\\
		= & \Delta\cdot\left( \frac{x}{x-\Delta} - \frac{1-x}{1-x+\Delta}\right)\\
		= & \Delta\cdot\left(\frac{\Delta}{x-\Delta} + \frac{\Delta}{1-x+\Delta}\right)\\
		\leq & \Delta^2\cdot\left(\frac{1}{y} + \frac{1}{1-y}\right)
		\leq \frac{2\Delta^2}{b}.
	\end{align*}
\end{proof}























%!TEX root = ../main.tex

\newcommand{\Paughproj}{\lawP^{\mathrm{proj}}_{\mathrm{aug},h}}
\newcommand{\seqz}{\mathsf{z}}
\newcommand{\zst}{\seqz^\star}
\newcommand{\zstil}{\tilde{\seqz}^\star}

\newcommand{\phiZ}{\phi_{\cZ}}
\newcommand{\phiV}{\phi_{\cV}}
\newcommand{\seqv}{\mathsf{v}}





\section{Imitation in the Composite MDP}\label{sec:imit_composite}
In this section, we prove our imitation guarantees in the composite MDP under the full generality of data augmentation.  The majority of this section is devoted to proving  a more general version of \Cref{thm:smooth_cor} that applies to vectorized notions of distance and helps tighten our bounds when instantiated in the control setting.  In Appendix \ref{app:generalizationsmooth}, we introduce some notation and state our most general result, \Cref{thm:smooth_cor_general}.  We then proceed to show that \Cref{thm:smooth_cor} follows from \Cref{thm:smooth_cor_general} and in Appendix \ref{app:smoothcor_general_proof}, we provide a detailed and rigorous proof of the main result.  In Appendix \ref{app:smoothcor_proof}, we show that the more general \Cref{thm:smooth_cor_general} impiles \Cref{thm:smooth_cor} from the text.

Throughout, we  also assume $\cS$ admits a direct decomposition. This is useful to capture the fact that we only apply smoothing on the $\pathm$ coordinates (memory chunk), not the full trajectory chunk $\pathc$.  
\begin{definition}[Direct Decomposition]\label{defn:direct_decomp} Let $\cS = \cZ \oplus \cV$ is a direct decomposition. We let $\phiZ$ and $\phiV$ denote projections onto the $\cZ$ and $\cV$ components, respectively.  We say that the $\cS = \cZ \oplus \cV$ is \emph{compatible} with the dynamics if  $F_h((\seqz,\seqv),\seqa) = F_h((\seqz,\seqv'),\seqa)$ for all $\seqv, \seqv' \in \cV$ and $\seqz \in \cZ$, and \emph{compatible} with policy $\pi$ if $\pi_h((\seqz,\seqv),\seqa) = \pi_h((\seqz,\seqv'),\seqa)$.; we define compatibility of a kernel $\lawW$ and of a pseudometric $\dist(\cdot,\cdot): \cS \times \cS \to \R_{\ge 0}$ with $\cS = \cZ \oplus \cV$ similarly.
\end{definition}
We emphasize that compatibility of dynamics with a direct decomposition does not make $\seqv$ irrelevant because $\dists$ still depends on $\seqv$.  For the purposes of the instantiation for control in the following appendix, we wish to control the imitation gaps on distances that do depend on $\seqv_h$, even though $\seqv_h$ does not figure directly into the dynamics.  Note that as defined, $\seqv_h$ does depend on the dynamics up until time $h-1$ and thus it is necessary to deal with this component in order to provide guarantees in $\dists$.

\subsection{A generalization of Theorem \ref{thm:smooth_cor}}\label{app:generalizationsmooth}
\newcommand{\epsvec}{\vec{\epsilon}}
\newcommand{\distsvec}{\vec{\dist}_{\cS}}
\newcommand{\distsi}[1][i]{{\dist}_{\cS,#1}}
\newcommand{\distsone}{\distsi[1]}

\newcommand{\distai}[1][i]{{\dist}_{\cA,#1}}
\newcommand{\distavec}{\vec{\dist}_{\cA}}
\newcommand{\gapjointvec}{\vec\Gamma_{\mathrm{joint},\epsvec}}
\newcommand{\gapmargvec}[1][\epsvec]{\vec\Gamma_{\mathrm{marg},#1}}
\newcommand{\drobvec}[1][\epsvec]{\vec{\dist}_{\mathrm{os},#1}}

We now state a generalization of \Cref{thm:smooth_cor}, which replaces a single distance by a vector of distances of dimension $K$; this will be useful for our instantiation of the composite MDP as a chunked control system in our final application (in particular, for deriving a bound on $\Imitfin$). It also showcases the most general structure accomodated by our proof technique. 

We begin by defining some notation:
\begin{itemize}
\item Let $K \in \N$ denote a dimension
\item Let $\epsvec \in \R_{\ge 0}^K$ denote a vector of tolerances
\item Let $\distsvec(\cdot,\cdot)$ denote a vector of pseudometrics $\distsi$ on $\cS$
\item Let $\distavec$ denote a vector of non-negative functions $\distai:\cA^2 \to \R_{\ge 0}$, not necessarily pseuometrics.
\item Let $\preceq$ denote vector wise inequality, and let the symbols $\wedge$ and $\vee$ be generalized to denote entrywise minima and maxima.  Similarly, addition of vectors is coordinate wise with scalars assumed to be broadcast appropriately.
\item We let $\distsi[1] = \disttvc$ denote the metric we consider for evaluating total variation distance. 
\end{itemize} 
We generalize We assume the following measure-theoretic regularity conditions, generalizing \Cref{ass:polishspaces} as follows.
\begin{assumption} \label{ass:polish_spaces_general}
    We assume that $\cS$ and $\cA$ are Polish spaces. This means they are metrizable, but we do not annotate their metrics because, e.g. the metric on $\cS$ may be other than $\dists$. We further assume that 
\begin{itemize}
\item $\distsi$ is a pseudometric and Borel measurable function from $\cS \times \cS \to \R_{\ge 0}$. 
\item For any $\epsilon \ge 0$, the set $\{(\seqa,\seqa') \in \cA \times \cA : \distai(\seqa,\seqa') > \epsilon\}$ is an open subset of $\cA\times \cA$; i.e. $\distai(\cdot,\cdot)$ is lower semicontinuous. In particular, this means $\distai$ is a Borel measurable function. Note that this implies that the 
\begin{align}\{(\seqa,\seqa') \in \cA \times \cA : \distavec(\seqa,\seqa') \not \preceq \epsvec\}.
\end{align}
is closed and thus measurable.
\end{itemize}
\end{assumption}
Note that the above assumption is the natural vectorized generalization of \Cref{ass:polishspaces}.  Next, we define vector versions of our imitation errors.
\begin{definition}[Imitation Errors, vector version]\label{defn:imit_gaps_vec} Given error parameter $\epsvec \in \R_{\ge 0}^K$, define 
\begin{itemize}
\item The \bfemph{vector joint-error} 
\begin{align}
\gapjointvec(\polhat \parallel \pist) := \inf_{\coup_1}\Pr_{\coup_1}\left[\exists h \in [H]: \distsvec(\shat_{h+1},\sstar_{h+1}) \vee \distavec(\seqast_h,\seqahat_h)   \not \preceq \epsvec\right],
\end{align} 
where the infimum is over trajectory couplings $((\shat_{1:H+1},\seqahat_{1:H}),(\sstar_{1:H+1},\seqa^\star_{1:H})) \sim \coup_1 \in \couple(\Dist_{\polhat},\Dist_{\polst})$ satisfying $\Pr_{\coup_1}[\shat_{1} = \sstar_1] = 1$.   
\item The \bfemph{vector marginal error} 
\begin{align}
\gapmargvec(\polhat \parallel \pist) := \max_{h \in [H]}\max\left\{\inf_{\coup_1}\Pr_{\coup_1}\left[\distsvec(\shat_{h+1},\sstar_{h+1})\not \preceq \epsvec\right],\, \inf_{\coup_1}\Pr_{\coup_1}\left[\distavec (\seqast_h,\seqahat_h)\not \preceq \epsvec\right]\right\}
\end{align} the same as the to joint-gap, with the ``$\max$'' outside the probability and infimum over couplings. 
\item The \bfemph{vector-wise one-step error}  
\begin{align}
\drobvec(\polhat_h(\seqs) \parallel \polst_h(\seqs)) := \inf_{\coup_2}\Pr_{\coup_2}\left[\distavec(\seqahat_h,\seqast_h) \not  \preceq \epsvec \right],
\end{align} where the infimum is over $(\seqast_h, \hat \seqa_h) \sim \coup_2 \in \couple( \bpolhat_h(\seqs),\bpol_h^\star(\seqs))$.
\end{itemize} 
\end{definition}

We now describe input stability. 
\begin{definition}[Input-Stability, vector version] \label{defn:fis_vector} A trajectory $(\seqs_{1:H+1},\seqa_{1:H})$ is \bfemph{input-stable} w.r.t. $(\distsvec,\distavec)$ if  all sequences $\seqs_1' = \seqs_1$ and $\seqs_{h+1}' = F_h(\seqs_h',\seqa_h')$ satisfy  
\begin{align}\distsi(\seqs_{h+1}',\seqs_{h+1}) \le  \max_{1 \le j \le h}\distai\left(\seqa_{j}',\seqa_j\right) ,\quad \forall h \in [H], i \in [K]
\end{align}
\end{definition} 


Finally, define input process stability. A slight technicality is that, in our instantiation, $\pist$ is taken to be a suitable regular condition probability of the joint distribution $\Dexp$ of expert trajectories. This means that $\pist$ can only really satisfy desired regularity conditions on  states visited with positive probabiliy by $\Dexp$. We address this subtlety by considering the following definition generalizing \Cref{defn:ips_body} in the body. We also restrict the kernels under consideration to those which produce distributions \emph{absolutely continuous} (\Cref{defn:abs_cont}) with respect to $\Psth$, and denoted with the $\ll$ comparator. More specifically, we only care about absolute continuity under the projections onto the $\cZ$ component of $\cS$. 
\begin{definition}[Input \& Process Stability, vector version]\label{defn:ips_vec}
Let $\pips \in (0,1)$, $\gamipsvec = (\gamipsi)_{1\le i \le K}$ be a collection non-decreasing maps $\gamipsi:\R_{\ge 0} \to  \R_{\ge 0}$, let   $\distips:\cS \times \cS \to \R$ be a pseudometric (possibly other than any of the $\distsi$), and $\rips > 0$.  We say a policy $\pist$ is \emph{$(\gamipsvec,\distips,\rips,\pips)$-(vectorwise-input-\&-process stable (vIPS)} if the following holds for any $r \in [0,\rips]$: 

Consider any sequence of kernels $\lawW_h:\cS \to \laws(\cS)$, $1\le h \le H$, satisfying 
\begin{align}
\forall h, \seqs \in \cS: \quad \Pr_{\tilde \seqs\sim \lawW_h(\seqs)}[\distips(\tilde \seqs,\seqs) \le r] = 1, \quad \phiZ \circ \lawW_h(\seqs) \ll \phiZ \circ \Psth. \label{eq:supp_contained}
\end{align}
Define a process $\seqs_1 \sim \Dinit$, $\tilde\seqs_h \sim \lawW_h(\seqs_h),\seqa_h \sim \pi_h(\tilde \seqs_h)$, and $\seqs_{h+1} := F_h(\seqs_h,\seqa_h)$. Then, with probability at least $1- \pips$,
\begin{itemize}
\item[(a)] the sequence $(\seqs_{1:H+1},\seqa_{1:H})$ is input-stable w.r.t $(\distsvec,\distavec)$ (as defined by \Cref{defn:fis_vector}).
\item[(b)]$\max_{h \in [H]} \distsi(F_h(\tilde\seqs_h,\seqa_h),\seqs_{h+1}) \le \gamipsi(r)$. 
\end{itemize}
\end{definition}
\newcommand{\epsvecmarg}{\epsvec_{\mathrm{marg}}}




We can now state our desired generalization. 



\begin{theorem}\label{thm:smooth_cor_general}   Suppose that there 
\begin{itemize}
\item[(a)]$\pist$ is $(\gamipsvec,\distips,\rips,\pips)$-vector IPS in the sense of \Cref{defn:ips_vec}.
\item[(b)] There is a direct decomposition of $\cS = \cZ \oplus \cV$, which associated projection maps $\phiZ$ and $\phiV$, and which is compatible with the dynamics, and policies $\pist$, $\pihat$, and smoothing kernel $\Wsig$, and $\distips$.
\item[(c)]  $\phiZ \circ \Wsig$ is $\gamma_{\sigma}$-TVC with respect to the pseudometric $\disttvc = \distsone$. 
\end{itemize} 
Let $\pihatsig$ be any policy which is $\gamhat$-TVC, also w.r.t. $\disttvc = \distsone$. Finally, let $\epsvec \in \R_{\ge 0}^K$, $r \in (0,\frac{1}{2}\rips]$, and define 
\begin{align}
p_r &:= \sup_{\seqs}\Pr_{\seqs' \sim \Wsig(\seqs)}[\distips(\seqs',\seqs) >  r], \quad \epsvecmarg := \epsvec + \gamipsvec(2r).
\end{align} Then, 
\begin{itemize}
\item For any policy $\pihat$,  both  $\gapjointvec (\pihatsig  \parallel \pistrep)$ and  $\gapmargvec[\epsvecmarg] (\pihatsig \parallel \pist)$ are upper bounded by%$\gapmarg[\epsilon + 2r](\pihat \circ \Wsig \parallel \pist)$ are both at most
\begin{align}
%\inf_{r > 0}  
\pips + H(2p_r + \gamhat(\epsvec_1) + (\gamhat + \gamtvcsig) \circ \gamipsone(2r))  + \sum_{h=1}^H\Exp_{\sstar_h \sim \Psth}\drobvec\,( \pihatsigh(\stel_h) \parallel \pistreph(\stel_h)) \label{eq:smooth_ub_app_one}
\end{align}
\item In the special case where $\pihatsig = \pihat \circ \Wsig$, we can take $\gamhat = \gamsig$, and obtain that $\gapjointvec(\pihatsig \parallel \pistrep)$ and $\gapmargvec[\epsvecmarg](\pihatsig \parallel \pist)$ are upper bounded by
\begin{align}
\pips + H\left(2p_r +  3\gamma_{\sigma}(\max\{\epsilon,\gamipsone(2r)\}\right)  + \textstyle \sum_{h=1}^H\Exp_{\sstar_h \sim \Psth}\Exp_{\sstartil_h \sim \Wsig(\sstar_h) } \drobvec( \pihat_{h}(\sstartil_h) \parallel \pidec(\sstartil_h)) . \label{eq:smooth_ub_app_two}
\end{align}
\end{itemize}
\end{theorem}
We note that \Cref{thm:smooth_cor} is  a special case of \Cref{thm:smooth_cor_general} and prove the former assuming the latter here at the end of the section.

\subsection{Proof of Theorem \ref{thm:smooth_cor_general} }\label{app:smoothcor_general_proof}



\subsubsection{Proof Overview and Coupling Construction}\label{sec:proof_construction}
We begin with an intuitive overview of the proof and partially construct the relevant intermediate trajectories used to define our coupling, after which we sketch the organization of the rest of Appendix \ref{app:smoothcor_general_proof}.

The proof proceeds by constucting a sophisticated coupling between the law of a trajectory evolving according to $\pihat$ and a trajectory evolving according to $\pistrep$ by introducing several intermediate sequences of composite states and composite actions.  

We partially specify this coupling below and formally construct it in Appendix \ref{app:proof_smooth_cor_general}.  Our construction is recursive and relies on the input and process stability as well as total variation continuity to show that if the trajectories generated by $\pistrep$ and $\pihat$ are close in $\drobvec[\epsvec]$ evaluated on states at step $h$, then they will remain close at step $h+1$.  There are a number of technical subtelties involved, especially those of a measure-theoretic nature, but much of the inuition can be gleaned from the following partial specification of the coupling $\coup$ over composite-state 
$(\shat_{1:H},\srep_{1:H},\stel_{1:H},\ssq_{1:H}) \subset \cS$, composite-actions  $(\arep_{1:H},\seqahat_{1:h},\atel_{1:H}) \subset \cK$ and interpolating composite-actions, $(\arepinter_{1:H},\atelinter_{1:H}) \subset \cA$. 

To define the construction, we define the probability kernels corresponding to the replica and deconvolution policies.  Note that these are slightly different from the definitions in the body due to the use of the direct decomposition; the intuition is the same, however.

\newcommand{\QdechZ}[1][h]{\lawW^{\star}_{\mathrm{dec},\cZ,h}}

\begin{definition}[Replica and Deconvolution Kernels]\label{defn:all_kernels} Let $\Paughproj$denote the joint distribution over $(\zst_h,\sstar_h,\zstil_h,\astar_h)$ under the generative process
\begin{align}
\sstar_h \sim \Psth, \quad \astar_h \sim \pist_h(\sstar_h), \quad 
\zst_h = \phiZ(\sstar_h), \quad \zstil_h \sim \phiZ\circ \Wsig(\sstar_h)
\end{align}
For $\seqz \in \cZ$, let $\QdechZ(\seqz)$ denote the distribution of $\zst_h$ conditioned on $\zstil_h = \seqz$, under $\Paughproj$. Given $\seqs = (\seqz,\seqv)$, define 
\begin{align}
&\Qdech(\seqs) = \QdechZ(\phiZ(\seqs)) \otimes \dirac_{\phiV(\seqs)}, \quad \\
&\Qreph(\seqs) = \Qdech \circ ( \Wsig(\phiZ(\seqs))\otimes \dirac_{\phiV(\seqs)}) =   (\QdechZ \circ \Wsig(\phiZ(\seqs)))\otimes \dirac_{\phiV(\seqs)}.
\end{align}
where we recall the dirac-delta $\dirac$. Equivalently, $\Qdech(\seqs)$ denotes the conditional sequence of $(\tilde \seqz,\seqv)$, where $\seqv = \phiV(\seqs)$, and $\tilde \seqz \sim \QdechZ(\seqs)$; $\Qreph$ can be expressed similarly. 
\end{definition}
We remark that $\Qdech$ and $\Qreph$ are both kernels and by \Cref{thm:durrett}, we may assume that the joint distribution over $(\sstar_h, \ssq_h)$ admits a regular conditional probability and thus these constructions are well-defined. 
\begin{remark}Note that the kernels $\Qdech$ and $\Qreph$ are  compatible with the decomposition $\cS = \cZ \oplus \cV$ by construction. Moreover, note that if $\seqs = (\seqz,\seqv)$, $\phiV \circ \Qdech(\seqs) = \phiV \circ \Qreph(\seqs)$ is the dirac-delta distribution supported on $\seqv$.
\end{remark}
\begin{lemma} Under our the assumption that $\pist$ and $\Wsig$ are compatible with the direct decomposition,  
\begin{align}
\pidech(\seqs) = \pist \circ \Qdech , \quad \pistreph(\seqs) = \pist \circ \Qreph 
\end{align}
\end{lemma}
\begin{proof} This follows imediately because $\pist$ and $\Wsig$ are  compatile with the direct decomposition, and by the definition of \Cref{defn:body_replica}.
\end{proof}


% Figure environment removed
\paragraph{A template for the coupling.} Our couplings are partially specified by the following generative process, and what remains unspecified are couplings between random variables at each each step $h$. In what follows, let $\cF_0$ denote the $\upsigma$-algebra generatived by $\shat_1 = \srep_1 = \stel_1 $. Let $\cF_h$ denote the sigma-algebra generated by  $(\shat_{1:h},\srep_{1:h},\stel_{1:h})$, $(\arep_{1:h},\sreptil_{1:h},\ssq_{1:h},\atel_{1:h},\seqahat_{1:h})$, and $(\arepinter_{1:h},\atelinter_{1:h})$.
\begin{itemize}
    \item The initial states are drawn as
    \begin{align}
    \shat_1 = \srep_1 = \stel_1 \sim \Dinit. 
    \end{align}
    \item The dynamics satisfy
    \begin{align}
    \shat_{h+1} = F_h(\shat_h,\seqahat_h), \quad \srep_{h+1} = F_h(\srep_h,\arep_h), \quad \stel_{h+1} = F_h(\ssq_h,\atel_h)
    \end{align}
    Note that determinism of the dynamics implies that $\stel_{h+1}$, $\srep_{h+1}$ and $\shat_{h+1}$ are $\cF_{h}$-measurable. 
    \item We generate
    \begin{align}
    &\sreptil_h \mid \cF_{h-1} \sim \Qreph(\srep_h), \quad \arep_h \mid \cF_{h-1},\sreptil_h \sim \pisth(\sreptil_h), \qquad \label{eq:trajevolve1} \\
    &\ssq_h \mid \cF_{h-1} \sim \Qreph(\stel_h), \quad \atel_h \mid \cF_{h-1},\ssq_h \sim \pisth(\ssq_h).\label{eq:trajevolve2}\\
    &\seqahat_h \mid \cF_{h-1} \sim \pihatsigh(\shat_h) \label{eq:trajevolve_ahat}
\end{align}
Importantly, we note that, marginalizing over $\ssq_h$ and $\sreptil_h$, respectively, $\atel_h \mid \cF_{h-1} \sim \pistreph(\stel)$ and $\arep_h \mid \cF_{h-1} \sim \pistreph(\srep_h)$.  
\item Lastly, we select interpolating actions via
\begin{align}
    &\arepinter_h \mid \cF_{h-1} \sim \pihatsigh(\srep_h), \qquad \atelinter_h \mid \cF_{h-1} \sim \pihatsigh(\stel_h)\label{eq:trajevolve3}
\end{align}
\end{itemize}
We will say $\coup$ is ``respects the construction'' as shorthand to mean that $\coup$ obeys the above equations.  The coupling is illustrated graphically in \Cref{fig:coupling_illustration}.  We now establish several key properties of the above constructions, separated into a subsection for the sake of clarity.


\paragraph{Organization of the remaining parts of Appendix \ref{app:smoothcor_general_proof}.}   In Appendix \ref{app:prop_of_deconv_replica}, we prove several prerequisite properties of the construction given above, including concentration of the smoothing kernel, and key properties of the replica distribution. Next, Appendix \ref{app:marg_imit_gap} shows that, due to these properties of the replica distribution, we can bound the marginal imitation gap by controlling the tracking of the teleporting sequence constructed above. Finally, in Appendix \ref{app:proof_smooth_cor_general} we formally construct the coupling and rigorously prove \Cref{thm:smooth_cor_general}.
\begin{comment}
\begin{observation} Let $\trajhat = (\shat_{1:H},\seqahat_{1:H})$, $\trajrep_{1:H} = (\sstar_{1:H},\seqast_{1:H})$, $\trajtel = (\stel_{1:H},\atel_{1:H})$.
\begin{itemize}
\item $\coup$ is an interpolating construction for $(\trajrep,\trajhat,\arepinter_{1:H})$ with respect to $(\pistrep,\pihat,(\cF_{h})_{h \ge 0})$.
\item $\coup$ is a teleporting construction for $(\trajtel,\trajrep,\ssq_{1:H})$ with respect to $(\pist,\Wsig,(\cF_h)_{h \ge 0})$. 
\end{itemize}
\end{observation}
\end{comment}




\newcommand{\Ctelh}[1][h]{\cC_{\mathrm{tel} ,#1}}
\newcommand{\Crephath}[1][h]{\cC_{ \hat{\seqs},#1}}
\newcommand{\Binterh}[1][h]{\cB_{ \mathrm{inter},#1}}
\newcommand{\Bhath}[1][h]{\cB_{\hat{\seqa},#1}}
\newcommand{\Btelh}[1][h]{\cB_{\mathrm{tel},#1}}
\newcommand{\Bfsh}[1][h]{\cB_{\mathrm{est},#1}}
\newcommand{\Callh}[1][h]{\cC_{\mathrm{all},#1}}
\newcommand{\Callbarh}[1][h]{\bar\cC_{\mathrm{all},#1}}
\subsubsection{Properties of smoothing, deconvolution, and replicas.}\label{app:prop_of_deconv_replica}

In this section, we establish several useful properties of smoothed and replica policies.  We begin by showing that smoothed policies are TVC.
\begin{lemma}\label{lem:pistrep_tvc}
The following hold
\begin{itemize}
    \item For any $h$, $\phiZ \circ \Qreph$ and $\pistreph$ are $\gamma_{\sigma}$ TVC.
    \item If $\pi$ is any policy compatible with the direct decomposition $\cS = \cZ \oplus \cV$ (in the sense of \Cref{defn:direct_decomp}), then $\pi\circ \Wsig$ is $\gamma_{\sigma}$-TVC.
\end{itemize}
\end{lemma}
\begin{proof} We observe that $\phiZ \circ \Qreph = \phiZ \circ \Qdech \circ \Wsig(\seqs)$. Moreover, we observe $\Qdech$ satisfies  $\phiZ \circ \Qdech(\seqs) =  \QdechZ \circ \phiZ$, so that $\phiZ \circ \Qreph = \QdechZ \circ \phiZ \circ \Wsig(\seqs)$. As $\phiZ \circ \Wsig$ is TVC, the first claim is a consequence of the data-processing inequality \Cref{cor:tv_two}. The second uses the fact that all listed objects involve composition of kernels with $\Wsig$.
\end{proof}
Next, we show that the replica construction preserves marginals. 
\begin{lemma}[Marginal-Preservation]\label{lem:replica_property} 
 There exists a coupling $\Pr$ of $\seqz_h \sim \phiZ \circ \Psth$, $\seqz_h' \sim \phiZ\circ\Wsig(\seqz_h,\cdot)$ (where ($\cdot$) denotes an irrelevant argument due to compatibility of $\Wsig$ with the direct decomposition), and $\tilde \seqz_h \sim \phiZ \circ \Qreph(\seqz_h,\cdot)$ (again, ($\cdot$) denotes an irrelevant argument) such that 
 \begin{align}
 (\seqz_h,\seqz_h') \overset{\mathrm{d}}{=}  (\tilde\seqz_h,\seqz_h').
 \end{align}
 In particular, for $\stel_h$ and $\ssq_h$ as in our construction, the marginal distributions of $\phiZ(\stel_h)$ and $\phiZ(\ssq_h)$ are the same, where $\stel_h \sim \Psth$ and  $\ssq_h \mid \stel_h \sim \Qreph(\stel_h)$.
\end{lemma}
\begin{proof}
    By \Cref{ass:polishspaces} and \Cref{thm:durrett}, we may assume that all joint distributions' conditional probabilities are regular conditional probabilities and thus almost surely equal to a kernel.  Moreover, since all kernels are compatible with the direct decomposition, it suffices to prove the special case of the trivial direct-decomposition where $\cZ = \cS$.  Fix a common measure $\pp$ over which $\stel_h, \ssq_h$, and $\mathsf{s}_h'$ are defined such that $\stel_h \sim \Psth$, $\mathsf{s}_h' \sim \Wsig(\stel_h)$, and $\ssq_h \sim \Wdeconvh(\mathsf{s}_h')$. Then for any measurable sets $A, B$, we have
    \begin{align}
        \pp(\stel_h \in A,\, \seqs_h' \in B) &= \pp(\seqs_h' \in B) \cdot \ee_{\seqs_h'}\left[\I[\seqs_h' \in B] \cdot \pp(\stel_h \in A | \seqs_h' ) \right] \\
        &= \pp(\seqs_h' \in B) \cdot \ee_{\seqs_h'}\left[\I[\seqs_h' \in B] \cdot \pp(\ssq_h \in A | \seqs_h' ) \right]\\
        &= \pp\left( \ssq_h \in A, \, \seqs_h' \in B \right),
    \end{align}
    where the first equality holds by the fact that we are working with regular conditional probabilities and Bayes' rule, the second equality holds by the definition of the deconvolution kernel above, and the last equality holds again by Bayes' rule and the tower rule for conditional expectations.

    To prove the second statement, we apply induction, again assuming that $\cZ = \cS$ as in the proof of the first statement.  Note that $\stel_1 \sim \Psth[1] = \Dinit$, and $\ssq_1 \sim \Qreph[1] \circ \Psth[1]$. Thus, from the first part of the lemma, $\phiZ (\stel_1) \sim \phiZ \circ \Psth[1]$. Now, suppose the induction holds up to step $h$. Then, $\ssq_h \sim \Psth$, as $\atel_h \sim \pist_h(\atel_h)$, then $\stel_{h+1} = F_{h}(\ssq_h,\atel_h) \sim \Psth[h+1]$. Again  $\ssq_{h+1} \sim \Qreph[h+1](\stel_{h+1})$, so that $\ssq_{h+1}$ has marginal $\Qreph[h+1]\circ \Psth[h+1] = \Psth[h+1]$, as needed.  
\end{proof}
We further show that $\Wreph$ can be defined to be absolutely continuous with respect to $\Psth$.
\begin{lemma}\label{lem:absolute_continuity}
    The kernel $\Wreph$ satisfies that $\phiZ \circ \Wreph \ll \phiZ \circ \Psth$ as laws, validating the second condition in \eqref{eq:supp_contained}.  It further holds that $\phiZ \circ \Wdeconvh \ll \phiZ \circ \Psth$.
\end{lemma}
\begin{proof}
    The first statement follows immediately from \Cref{lem:replica_property} because these distributions are the same.  The second statement follows immediately from the tower law of conditional expectation and the definition of $\Wdeconvh$.
\end{proof}

Lastly, we establish that the replica kernel inherits all concentration properties from the smoothing kernel.
\begin{lemma}[Replica Concentration]\label{lem:rep_conc} Recall that 
\begin{align}
p_r := \sup_{\seqs}\Pr_{\seqs' \sim \Wsig(\seqs) }[\distips(\seqs',\seqs) >  r].
\end{align} We then have
\begin{align}
\Pr_{\seqs_h \sim \Psth,\stil_h \sim \Qreph(\seqs_h)}[\distips(\stil_h,\seqs_h) > 2\rsmooth] \le 2p_r \label{eq:concentration_conv_two}
\end{align}
\end{lemma}
\begin{proof} %We fix a common measure $\Pr[\cdot]$ over $\seqs_h \sim \Psth,\stil_h \sim \Qreph(\seqs_h)$
Again, all terms -- $\Wsig,\Qreph,\Qdech$ and $\distips$ -- are compatible with the direct decomposition, it suffices to consider the case of the trivial direct decomposition under whcih $\cZ = \cS$.

Let $\Pr$ denote a distribution over $\seqs_h \sim \Psth$, $\seqs_h' \sim \Wsig(\seqs_h)$, and $\stil_h \sim \Qdech(\seqs_h')$. In this special case,  we see that $\stil_h \mid \seqs_h \sim \Qreph(\seqs_h)$\footnote{Notice that, for general $\cS = \cZ \oplus \cV$, this condition would become $\phiZ(\stil_h) \mid \phiZ(\seqs_h) \sim \phiZ \circ \Qreph(\phiZ(\seqs_h),\cdot)$, where the $\cdot$ argument is irrelevant.}. By a union bound,
\begin{align}\label{eq:dists_conv_bound_two}
\Pr_{\seqs_h \sim \Psth,\stil_h \sim \Qreph(\seqs_h)}[\distips(\seqs_h,\stil_h) > 2\rsmooth] &\le \Pr[\distips(\stil_h,\seqs'_h) > \rsmooth]  + \Pr[\distips(\seqs_h,\seqs'_h) > \rsmooth] \\
&= 2 \Pr[\distips(\seqs_h,\seqs'_h) > \rsmooth] \le 2p_r,
\end{align}
where the equality follows from the first statment of \Cref{lem:replica_property}.
\end{proof}
\begin{remark}Note that, in the previous lemma, it suffices that the following weaker condition holds: $\Pr_{\seqs \sim \Psth,\seqs' \sim \Wsig(\seqs)}[\distips(\seqs',\seqs) >  \rsmooth] \le p_r$, i.e. for concentration to hold only in distribution over $\seqs \sim \Psth$, instead of \emph{uniformly} over states.
\end{remark}
\newcommand{\Qtilreph}[1][h]{\tilde{\lawW}_{\repsymbol,#1}}


\subsubsection{Bounding the marginal imitation gaps in terms of the teleporting sequence error}\label{app:marg_imit_gap}
Before turning to the proof of \Cref{thm:smooth_cor_general}, we verify that closeness to the \emph{teleporting sequences} suffices to control error in marginal gap to $\pist$. The key property here is that the teleporting sequence, as shown in \Cref{lem:replica_property}, has the same marginal distribution over states as does $\pist$.

\begin{lemma}\label{lem:marg_imit_gap_tel} Let $\coup$ be any coupling obeying the construction of the couplings above. Then, 
\begin{align}
\gapmargvec(\pihatsig \parallel \pist) \le 
\Pr_{\coup}\left[\exists h \in [H]: \left\{\distsvec(\stel_{h+1},\shat_{h+1}) \not \preceq \epsvecmarg\right\} \cup \left\{ \distavec(\atel_h,\ahat_h) \not \preceq \epsvecmarg\right\}\right] 
\end{align}
\end{lemma}
\begin{proof}
 We begin with a (reverse) union bound.
\begin{align}
&\Pr_{\coup}\left[\exists h \in [H]: \left\{\distsvec(\stel_{h+1},\shat_{h+1}) \not \preceq \epsvecmarg\right\} \cup \left\{ \distavec(\atel_h,\ahat_h) \not \preceq \epsvecmarg\right\}\right] \\
&\ge\max_h\max\left\{\Pr_{\coup}\left[\distsvec(\stel_{h+1},\shat_{h+1}) \not \preceq \epsvecmarg\right],\, \Pr_{\coup}\left[\distavec(\atel_h,\ahat_h) \not \preceq \epsvecmarg\right]\right\}.
\end{align}
 By \Cref{lem:replica_property} implies that $\stel_h$ has the marginal distribution of $\sstar_h \sim \Psth$. Moreover, by construction, for each $h$, $\atel_h \mid \cF_h \sim \pistreph(\stel_h)$, Thus, for each $h$, $\stel_{h+1}$ and $\atel_h$ have the same \emph{marginals} as the marginals as $\sstar_{h+1}$ and $\astar_h$ under the distribution $\Dist_{\pist}$ induced by $\pist$. Hence, 
 \begin{align}
 \Pr_{\coup}\left[\distsvec(\stel_{h+1},\shat_{h+1}) \not \preceq \epsvecmarg\right] &\ge \inf_{\coup_1} \Pr\left[\distsvec(\sstar_{h+1},\shat_{h+1}) \not \preceq \epsvecmarg\right] \\
 \Pr_{\coup}\left[\distavec(\atel_{h},\ahat_{h}) \not \preceq \epsvecmarg\right] &\ge \inf_{\coup_1} \Pr\left[\distsvec(\astar_{h},\ahat_{h}) \not \preceq \epsvecmarg\right],
 \end{align}
 where the $\inf_{\coup_1}$ is, as  in \Cref{defn:imit_gaps,defn:imit_gaps_vec}, the infinum over couplings between $\Dist_{\pist}$ and $\Dist_{\pihat}$. Thus, 
 \begin{align}
&\Pr_{\coup}\left[\exists h \in [H]: \left\{\distsvec(\stel_{h+1},\shat_{h+1}) \not \preceq \epsvecmarg\right\} \cup \left\{ \distavec(\atel_h,\ahat_h) \not \preceq \epsvecmarg\right\}\right] \\
&\ge\max_h\max\left\{\inf \Pr_{\coup_1}\left[\distsvec(\sstar_{h+1},\shat_{h+1}) \not \preceq \epsvecmarg\right],\, \inf_{\coup}\Pr_{\coup_1}\left[\distavec(\astar_h,\ahat_h) \not \preceq \epsvecmarg\right]\right\}\\
&:= \gapmargvec(\pihatsig \parallel \pist).
\end{align}

\end{proof}

\subsubsection{Formal proof of Theorem \ref{thm:smooth_cor_general}}\label{app:proof_smooth_cor_general}
We now proceed to formally prove \Cref{thm:smooth_cor_general}

\paragraph{Key Events. } For the random variables defined above, we define three groups of events. 
\begin{itemize}
\item The \emph{coupling events}, denoted by $\cB$, which are controlled by carefully selecting a coupling.
\item The \emph{inductive events}, denoted by $\cC$, which we condition on when bounding the probability of the coupling events.
\item The \emph{stability events}, denoted by $\cQ$, which take advantage of the stability properties of the imitation policy. 
\end{itemize}
\newcommand{\Ballbarh}[1][h]{\bar\cB_{\mathrm{all},#1}}
\newcommand{\Ballh}[1][h]{\cB_{\mathrm{all},#1}}
\newcommand{\Qis}{\cQ_{\textsc{is}}}
\newcommand{\Qips}{\cQ_{\textsc{ips}}}
\newcommand{\Qclose}{\cQ_{\mathrm{close}}}
\newcommand{\Qall}{\cQ_{\mathrm{all}}}

\begin{definition}[Coupling Events]\label{defn:all_key_eents} Define the events
\begin{align}
    \Btelh &=  \left\{ \arep_h = \atel_h, ~\phiZ(\sreptil_h) = \phiZ(\ssq_h) \right\}\\
     \Bfsh &= \left\{ \distavec( \atelinter_h,\atel_h) \not \preceq \epsvec \right\} \\
    \Binterh &= \left\{ \atelinter_h = \arepinter_h  \right\} \\
    \Bhath &= \left\{  \arepinter_h = \seqahat_h \right\} \\
     \Ballh &= \Binterh \cap \Btelh \cap \Bfsh \cap \Bhath\\
    \Ballbarh &=  \bigcap_{j=1}^h \Ballh[h]
\end{align}
Notice that each of the events above are $\cF_{h}$-measurable. Moreover, note that on $\Ballbarh$, $\max_{1\le j \le h}\phiis(\seqahat_j,\arep_j) \le \epsilon$.
\end{definition}
\begin{definition}[Inductive Event]\label{def:inductive_event}

Define the events
\begin{align}
\Crephath &= \left\{  \distsvec(\srep_h, \shat_h) \preceq \epsvec \right\}, \\
\Ctelh &= \left\{  \distsvec(\srep_h, \stel_h) \preceq \gamipsvec(2r) \right\} \\
\Callh &:= \Crephath \cap \Ctelh\\
 \Callbarh &=  \bigcap_{j=1}^h \Callh[j]
\end{align}
Notice that all the above events are $\cF_{h-1}$-measurable, due to determinism of the dynamics. Note that also $\Pr_{\coup}[\Callbarh[1]] = 1$ for any $\coup$ that respects the construction (as $\srep_1 = \stel_1 = \shat_1$).
\end{definition} 
\begin{definition}[Stability Events] Define the events 
\begin{align} 
\Qclose &:= \left\{\forall h \in [H]: \distips(\srep_h,\sreptil_h) \le 2r \right\}\\
\Qis &:= \left\{(\srep_{1:H+1},\arep_{1:H}) \text{ is input-stable w.r.t. } (\distsvec,\distavec)\right\}\\
\Qips &:= \left\{\distsvec(F_h(\sreptil_{h},\arep_h),\srep_{h+1}) \le  \gamipsvec\circ \distips\left(\sreptil_h,\srep_{h}\right), \quad 1 \le j \le H\right\} \\
\Qall &:=  \Qips \cap \Qclose .
\end{align}
In words, $\Qclose$ the event on which $\srep_h$ and $\sreptil_h \sim \Qreph(\stel_h)$ are close, and $\Qis$and  $\Qips$ ensure consequencs of  (vector) input-stability and (vector) input process stability holds.
\end{definition}

\paragraph{Steps of the proof.}
First, we use stability to reduce the event $\Callbarh[h+1]$ to $\Callbarh \cap \Ballbarh$:
\begin{claim}[Stability Claim]\label{claim:stability_claim} By construction, 
\begin{align}\Callbarh[h+1] \subset \Qall \cap \Callbarh \cap \Ballbarh.
\end{align}
\end{claim} 
\begin{proof} It suffices to show that on $\Qall \cap \Callbarh \cap \Ballbarh$, $ \distsvec(\srep_{h+1},\shat_{h+1}) \preceq \epsvec$ and $\distsvec(\srep_{h+1},\stel_{h+1}) \preceq \gamipsvec(2r)$. By applying the event $\Qis$ to the sequence $\seqa'_h = \seqahat_h$ and $\seqs'_h = \shat_h$, we have that on $\Qall \subset \Qis$ that
\begin{align}
 \forall h \in [H], i \in [K], \quad \distsi(\srep_{h+1},\shat_{h+1}) \le  \max_{1 \le j \le h}\distai\left(\arep_j,\seqahat_{j}\right) \label{eq:Qis_consequence}
\end{align}


For the next point, note that the compatibility of the dynamics with the direct decomposition $\cS = \cZ \oplus \cV$ implies that there exists a dynamics map $F_h^{\cZ}$  for which 
\begin{align}
F_h(\seqs,\seqa) = F_h^{\cZ}(\phiZ(\seqs),\seqa).
\end{align}
Similarly, by applying $\Qips$ and $\Qclose$ and the event $\{\phiZ(\sreptil_h) = \phiZ(\ssq_h),\atel_h = \arep_h\}$ on $\Btelh$, it holds that on $\Qall \cap \Callbarh \cap \Ballbarh$ that, for all $h \in [H]$,
\begin{align}
\distsvec(\srep_{h+1},F_h(\sreptil_{h},\arep_h))  &= \distsvec(\srep_{h+1},F_h^\cZ(\phiZ(\sreptil_{h}),\arep_h)) \\
&= \distsvec(\srep_{h+1},F_h^\cZ(\phiZ(\ssq_{h}),\atel_h)) \tag{$\Btelh$}\\
&= \distsvec(\srep_{h+1},F_h(\ssq_{h},\atel_h)) \\
&= \distsvec(\srep_{h+1},\stel_{h+1})\\ 
&\le \gamipsvec\circ\distips\left(\stel_j,\ssq_{j}\right) \tag{$\Qips$}\\
&\le \gamipsvec\circ\distips\left(2r\right) \tag{$\Qclose$}.
\end{align}
\end{proof}
From \Cref{claim:stability_claim}, we decompose our error probability as follows:
\begin{lemma}[Key Error Decomposition] \label{lem:putting_couplings_together} Let $\coup$ respect the construction (in the sense of \Cref{sec:proof_construction}). Then, for any coupling $\coup$ which respects the construction,
\begin{align}
&\gapjointvec(\pihatsig \parallel \pirep) \vee \gapmargvec(\pihatsig \parallel \pist) \le \Pr_{\coup}[\Qall^c] + \sum_{h=1}^H\Pr_{\coup}[ \Ballbarh^c \cap \Callbarh \cap \Ballbarh[h-1]]\label{eq:Gamimit_decomp}
\end{align}
\end{lemma}
\begin{proof} In what follows, we use $\vec{v} \vee \vec{w}$ to denote the entrywise maximum of two vectors of the same dimension. Define the events $\cE_h := \Callbarh[h+1] \cap \Ballbarh$. Observe that the events are nested: $\cE_{h} \supset \cE_{h+1}$, and that on $\cE_H$, we have that for all $h \in [H]$
\begin{align}
\distsvec(\srep_{h+1},\shat_{h+1}) \vee \distavec(\arep_h,\ahat_h) &\preceq \epsvec \vee \distavec(\arep_h,\ahat_h) \tag{$\Crephath[h+1] \supset \Callbarh[h+1] \supset \cE_h$}\\
&\preceq \epsvec \tag{$\Ballbarh \supset \cE_h$}.
\end{align}
On $\Qall \cap \cE_H$, we have that
\begin{align}
\max_h \distsvec(\srep_h,\stel_h) \le \gamipsvec(2r), \quad \text{and}\quad \atel_h = \arep_h
\end{align}
Thus, by the triangle inequality and $\epsvecmarg = \epsvec + \gamipsvec(2r)$, on $\Qall \cap \cE_H$,
\begin{align}
\max_h \distsvec(\srep_h,\stel_h) \le \epsvecmarg, \quad \text{and}\quad  \distavec(\atel_h,\ahat_h)  =\distavec(\arep_h,\ahat_h)  \le \epsvec \le \epsvecmarg.
\end{align}
Thus, 
\begin{align}
&\Pr_{\coup}\left[\exists h \in [H]: \left\{\distsvec(\srep_{h+1},\shat_{h+1}) \vee \distavec(\arep_h,\ahat_h) \not \preceq \epsvec\right\} \cup \left\{\distsvec(\stel_{h+1},\shat_{h+1}) \vee \distavec(\atel_h,\ahat_h) \not \preceq \epsvecmarg\right\}\right]\\
&\le \Pr_{\coup}[(\Qall \cap \cE_H)^c] \label{eq:first_pr_coup_big}
\end{align}
In particular, this shows that
\begin{align}
\gapjointvec(\pihatsig \parallel \pirep)  \le \Pr_{\coup}[(\Qall \cap \cE_H)^c], 
\end{align}
and similarly, by \Cref{lem:marg_imit_gap_tel},
\begin{align}
\gapmargvec(\pihatsig \parallel \pist) \le \Pr_{\coup}[(\Qall \cap \cE_H)^c]
\end{align}
As $(\srep_{1:H+1},\arep_{1:H}) \sim \Dist_{\pistrep}$, \eqref{eq:first_pr_coup_big} shows that
\begin{align}\gapjointvec(\pihatsig \parallel \pirep) \vee \gapmargvec(\pihatsig \parallel \pirep)\le \Pr_{\coup}[(\Qall \cap \cE_H)^c].
\end{align} 
Let us conclude by bounding $\Pr_{\coup}[(\Qall \cap \cE_H)^c]$. Using the nesting structure $\cE_h = \bigcap_{j=1}^h \cE_j$, the peeling lemma, \Cref{lem:peeling_lem}, and a union bound, it holds that
\begin{align}
\Pr_{\coup}\left[(\Qall \cap \cE_H)^c\right] &\le \Pr_{\coup}[\Qall^c] + \Pr\left[ \exists h \in [H] \text{ s.t. } \left(\Qall \cap \cE_{h-1} \cap \cE_h^c\right) \text{ holds } \right ]\\
&\le \Pr_{\coup}[\Qall^c] + \sum_{h=1}^H\Pr_{\coup}\left[ \Qall \cap \cE_{h-1} \cap \cE_h^c \text{ holds } \right ]\\
&= \Pr_{\coup}[\Qall^c] + \sum_{h=1}^H\Pr_{\coup}\left[ \Qall \cap \Ballbarh[h-1] \cap \Callbarh[h]  \cap (\Ballbarh \cap \Callbarh[h+1])^c \text{ holds } \right ]\\
&= \Pr_{\coup}[\Qall^c] + \sum_{h=1}^H\Pr_{\coup}\left[ \Qall \cap \Ballbarh[h-1] \cap \Callbarh[h]  \cap \Ballbarh^c\right ]\\
&= \Pr_{\coup}[\Qall^c] + \sum_{h=1}^H\Pr_{\coup}\left[ \Qall \cap \Ballbarh[h-1] \cap \Callbarh[h]  \cap \Ballh^c\right ],
\end{align}
where the last step invokes \Cref{claim:stability_claim}.
\end{proof}
Next, we bound the contribution of $\Pr_{\coup}[\Qall^c]$ in \eqref{eq:Gamimit_decomp}, uniformly over all couplings.
\begin{lemma}\label{lem:Qall_bound} For all $\coup$ which respect the construnction, 
\begin{align}
\Pr_{\coup}[\Qall^c] \le \pips + 2Hp_r.
\end{align}
\end{lemma}
\begin{proof} $\Pr_{\coup}[\Qclose^c] = \Pr_{\coup}[\exists h: \distips(\stel_h,\ssq_h) > 2r] \le 2Hp_r$ by \Cref{lem:rep_conc} and a union bound. 


Let us now bound $\Pr_{\coup}[\Qclose \cap \Qips^c] \le \Pr_{\coup}[\Qips^c \mid \Qclose ]$. Define the kernels $\lawW_h(\seqs)$ to be equal to the kernel $\Wreph(\seqs)$ conditioned on the event $\seqs' \sim \Wreph(\seqs)$ satisfies $\distips(\seqs',\seqs) \le 2r$. Then, conditional on $\Qclose$, we see that the sequence $(\srep_{1:H+1},\sreptil_{1:H},\arep_{1:H})$ obeys the generative process
\begin{align}
\sreptil_{h} \mid \sreptil_{1:h-1},\srep_{1:h},\arep_{1:h-1} \sim \lawW_h(\seqs), \quad \arep_{h} \mid \sreptil_{1:h},\srep_{1:h},\arep_{1:h-1} \sim \pisth(\sreptil_h), \quad \srep_{h+1} = F_h(\srep_h,\arep_h).
\end{align} 
By construction, for each $h$, $\Pr_{\seqs' \sim \Wreph(\seqs)}[\distips(\seqs',\seqs) > 2r] = 0$. Thus, the definition of (vector) input process stability (\Cref{defn:ips_vec}) and assumption $r \le \frac{1}{2}\rips$ implies that $\Pr_{\coup}[\Qips^c \mid \Qclose ] \le \pips$.
\end{proof}
The remaining step of the proof is therefore to bound the second term in \eqref{eq:Gamimit_decomp}.
\begin{lemma}\label{lem:make_coupling}There exists a coupling $\coup$ which respects the construction and satisfies the following for any $h \in [H]$
\begin{align}
&\Pr_{\coup}[\Ballh^c \mid \cF_{h-1}] \\
&\le \gamhat \circ \disttvc(\srep_h,\shat_h) + (\gamhat + \gamtvcsig) \circ \disttvc(\srep_h,\stel_h)  + \drobvec\,( \pihatsigh(\stel_h) \parallel \pistreph(\stel_h)),~ \text{$\coup$-almost surely }
    %&\pp_{\coup}\left[ \Ballbarh^c \cap \Callbarh \cap \Ballbarh[h-1] \right] \leq \\
   % &\quad\delta + \gamtvc(\epsilon) + (\gamtvc + \gamsig)\circ\gamips(2r) + \Exp_{\stel_h \sim \coup}\drobvec[\epsilon](\pihat \parallel \pistrep \mid \stel_{h}, h)
    %2 \cdot p_r + 2 \cdot \gamtvc\left( \gamips(3r) \right) + \drob[\epsilon]\left( \pistrep, \pihat | \stel_{h+1}, h+1 \right) + \gamtvc(\epsilon).
\end{align}
Consequently, for all $h \in [H]$,
\begin{align}
&\Pr_{\coup}[ \Ballh^c \cap \Callbarh \cap \Ballbarh[h-1]] \\
&\le \gamhat(\epsvec_1) + (\gamhat + \gamtvcsig) \circ \gamipsone(2r) + 
\Exp_{\coup}[\drobvec\,( \pihatsigh(\stel_h) \parallel \pistreph(\stel_h))]
\end{align}
Moreover, $\seqs \mapsto \drobvec\,( \pihatsigh(\seqs) \parallel \pistreph(\seqs))$ is measurable. 
%as well as $\pp_{\coup}[\Callbarh[0]] = 1$.
\end{lemma}

\begin{proof}[Proof Sketch]
We begin by giving a high level overview of the construction, which is done recursively.  The key technical tool is \Cref{lem:couplinggluing} above, which allows us to transform any coupling $\coup$ between random variables $(X, Y)$ into a probability kernel $\coup(\cdot| X)$ mapping instances of $X$ to probability distributions on $Y$ such that $(X, Y) \sim \coup$ has the same law as $(X, Y \sim \coup(\cdot | X))$.  For each $h$, we then show that, assuming the coupling has kept the states and controls close together until time $h-1$, this will imply the following chain:
\begin{align}
    \underbrace{(\arep \leftrightarrow \atel)}_{\gamtvc \text{ and induction}} \to \underbrace{(\atel \leftrightarrow \atelinter)}_{\text{learning and sampling}} \to \underbrace{(\atelinter \leftrightarrow \arepinter)}_{\gamtvc \text{ and induction}} \to \underbrace{(\arepinter \leftrightarrow \seqahat)}_{\gamtvc \text{ and induction}}, \label{eq:mainproofoutline}
\end{align}
where the bidirectional arrows indicate individual couplings between the laws of the random variables that are constructed by the method outlined in text below and the single directional arrows denote the probability kernels described above. The full proof of the lemma is given in \Cref{sec:couplingconstruction}.
\end{proof}

\paragraph{Concluding the proof.}  Here, we finish the proof of \Cref{thm:smooth_cor_general}.  Recall that we wish to bound $\gapjointvec\,(\pihatsig \parallel \pirep) \vee \gapmargvec[\epsvecmarg](\pihatsig \parallel \pist)$. We begin by bounding $\gapjointvec\,(\pihatsig \parallel \pirep) \vee \gapmargvec[\epsvecmarg](\pihatsig \parallel \pistrep)$.  In light of \Cref{lem:putting_couplings_together}, it suffices to bound
 \begin{align}
 \Pr_{\coup}[\Qall^c] + \sum_{h=1}^H\Pr_{\coup}[ \Ballbarh^c \cap \Callbarh \cap \Ballbarh[h-1]], 
 \end{align}
 where $\coup$ is the coupling in \Cref{lem:make_coupling}.
Applying \Cref{lem:Qall_bound} and \Cref{lem:make_coupling},
\begin{align}
&\Pr_{\coup}[\Qall^c] + \sum_{h=1}^H\Pr_{\coup}[ \Ballbarh^c \cap \Callbarh \cap \Ballbarh[h-1]] \\
&\le \pips + 2Hp_r +   \sum_{h=1}^H\Pr_{\coup}[ \Ballbarh^c \cap \Callbarh \cap \Ballbarh[h-1]] \\
&\le   \pips + H(2p_r + \gamhat(\epsvec_1) + (\gamhat + \gamtvcsig) \circ \gamipsone(2r))  + \sum_{h=1}^H\Exp_{\stel_h \sim \coup}\drobvec\,( \pihatsigh(\stel_h) \parallel \pistreph(\stel_h))
\end{align}
To conclude, we note that for any $\coup$ which respects the construction, \Cref{lem:replica_property} ensures that $\stel_h$ as the marginal distribution of $\sstar_h \sim \pisth$. Thus, the above is at most
\begin{align}
\pips + H(2p_r + \gamhat(\epsvec_1) + (\gamhat + \gamtvcsig) \circ \gamipsone(2r))  + \sum_{h=1}^H\Exp_{\sstar_h \sim \Psth}\drobvec\,( \pihatsigh(\sstar_h) \parallel \pistreph(\sstar_h)) \label{eq:first_eq_i_showed}
\end{align}
which concludes the proof of \eqref{eq:smooth_ub_app_one} for $\gapjointvec(\pihat \parallel \pistrep)$. 

To prove \eqref{eq:smooth_ub_app_two} for $\gapjointvec(\pihat \parallel \pistrep)$, we consider the special case that $\pihatsig = \pihat \circ \Wsig$. By definition, $\pihatsigh =\pihat \circ \Wsig$. Thus, the  data-processing inequality for optimal transport (\Cref{cor:opt_trans}) 
\begin{align}\drobvec\,( \pihatsigh(\sstar_h) \parallel \pistreph(\sstar_h))  \le \Exp_{\seqs_h' \sim \Wsig(\sstar_h)}\drobvec\,( \pihat(\seqs_h') \parallel \pidech(\seqs_h')),
\end{align}
for all $\sstar_h$. Substituting this into \eqref{eq:first_eq_i_showed}, and setting $\gamhat = \gamsig$ (in view of \Cref{lem:pistrep_tvc}), finishes the argument.













\subsubsection{Proof of Lemma \ref{lem:make_coupling}}\label{sec:couplingconstruction}

Recall that \Cref{ass:polish_spaces_general} ensures all of the general measure-theoretic guarantees of Appendix \ref{app:prob_theory} hold true in our setting. Notably we need the gluing lemma (\Cref{lem:couplinggluing}) and the commuting of optimal transport metrics and conditional probabilities (\Cref{prop:MK_RCP}).

\paragraph{Proof strategy.} Our proof follows along similar lines as that of \Cref{prop:IS_general_body}, although with the added complication of including the smoothing.  We will inductively construct $\coup$.  A useful schematic for the construction at each step is the following diagram:
\begin{align}
    \underbrace{(\sreptil \leftrightarrow \ssq),(\arep \leftrightarrow \atel)}_{\Btelh} \to \underbrace{(\atel \leftrightarrow \atelinter)}_{\Bfsh} \to \underbrace{(\atelinter \leftrightarrow \arepinter)}_{\Binterh}\to \underbrace{(\arepinter \leftrightarrow \seqahat)}_{\Bhath}, \label{eq:mainproofoutline2}
\end{align}
where the events under each bidirectional arrow refer to the event such ensuring that there exists a coupling such that the objects are close.  We then will apply \Cref{lem:couplinggluing} to glue the individual couplings together.  We will then use \Cref{lem:peeling_lem} and a union bound to control the probability under our constructed coupling that any of the relevant events fail to hold, concluding the proof.

\newcommand{\coupfs}{\coup_{\mathrm{est}}}
\newcommand{\Ebarh}{\bar{\cE}_{h}}

\newcommand{\coupstel}{\coup_{\seqs,\mathrm{tel}}}  
\newcommand{\coupinterr}{\coup_{\mathrm{inter}}}  

\newcommand{\couptel}{\coup_{\mathrm{tel}}}  
\newcommand{\coupahat}{\coup_{\seqahat}}  
\paragraph{Recursive construction of $\coup$.} Let $h \ge 1$, and suppose that we have constructed the coupling $\coup^{(1:h-1)}$ for steps $1,\dots,h-1$ which respects the construction. Recall that $\cF_h$ denotes the sigma-algebra generated by  $(\shat_{1:h},\srep_{1:h},\stel_{1:h})$, $(\arep_{1:h},\sreptil_{1:h},\ssq_{1:h},\atel_{1:h},\seqahat_{1:h})$, and $(\arepinter_{1:h},\atelinter_{1:h})$. Notice that $\stel_{h+1},\srep_{h+1},\shat_{h+1}$ are determined by $\cF_h$ as well. Similarly, it can be seen from \Cref{defn:all_kernels} that $\phiV(\ssq_{h+1})$ and $\phiV(\sreptil_{h+1})$ are also determined by $\cF_{h}$ (since the replica kernel preserves the $\cV$-components). We summarize all these aforementioned variables in a random variable $Y_h$. Let $\cF_0$ denote the filtration generated by $\srep_1 = \stel_1 = \shat_1$. We let $Y_0 = (\srep_1,\stel_1,\shat_1)$. 

Correspondingly, let $Z_h$ denote the random variables $(\arep_{h},\phiZ(\sreptil_{h}),\phiZ(\ssq_{h}),\atel_{h},\seqahat_{h})$, and $(\arepinter_{h},\atelinter_{h})$ such that the joint law of these random variables respects the construction.  Our goal is then to specify, for each $h \in [H]$, a joint distribution of $(Y_{h-1},Z_{h})$.
Note that $Z_h,Y_{h-1}$ determines $Y_{h}$, and we call this induced law $\coup^{(h)}$.






We begin by specifying joint distributions conditional on $Y_{h-1}$ and subsets of $Z_h$, then glue them together by the gluing lemma. Below, we use use information-theoretic notation. 
\begin{itemize}
    \item By total variation continuity of $\phiZ \circ \Qreph$ (\Cref{lem:pistrep_tvc}),
    \begin{align}
    \TV(\pp_{\phiZ(\sreptil_{h}) \mid  Y_{h-1}},\pp_{\phiZ(\ssq_{h}) \mid  Y_{h-1}}) \le \gamtvcsig \circ \disttvc(\srep_h,\stel_h). 
    \end{align}
    Because $\arep_{h} \sim \pisth(\sreptil_{h+1})$ and $\atel_{h} \sim \pisth(\ssq_{h})$,  and $\pist$ is compatible with the decomposition $\cS = \cZ \oplus \cV$ (i.e. $\pisth(\seqs)$ is a function of $\phiZ(\seqs)$)
    \Cref{cor:tv_two} implies that (almost surely)
    \begin{align}
    \TV(\pp_{(\arep_h,\phiZ(\sreptil_{h}) \mid Y_{h-1}},\pp_{(\atel_h,\phiZ(\ssq_{h}) \mid  Y_{h-1}}) \le \gamtvcsig \circ \disttvc(\srep_h,\stel_h). 
    \end{align}
    Hence, \Cref{cor:first_TV} implies that there exists a coupling $\couptel^{(h)}$ over $Y_{h-1},(\phiZ(\sreptil_{h}),\arep_h),(\phiZ(\ssq_{h}),\atel_{h})$ respecting the construction such that $Y_h \sim \coup^{(h-1)}$ and such that (almost surely)
    \begin{align}\label{eq:couptel}
    \Exp_{\couptel^{(h)}}[\Btelh \mid Y_{h-1}] = \Pr_{\couptel^{(h)}}[(\phiZ(\sreptil_{h}),\arep_h) \ne (\phiZ(\ssq_{h}),\atel_{h})\mid Y_{h-1}] &\le \disttvc(\srep_h,\stel_h)].
    \end{align}
    \item In our construction, $\atel_h \mid Y_{h-1} \sim \pistreph(\stel_h)$, and $\atelinter_h \mid Y_{h-1} \sim \pihatsigh(\stel_h)$. 
    Thus, by definition of $\drobvec$, and the assumption $\I\{\distavec(\cdot,\cdot) \not\preceq \epsvec\}$ is lower semicontinuous, \Cref{prop:MK_RCP} implies that we may find a coupling $\coupfs^{(h)}$ of $(\atel_{h},\atelinter_{h},Y_{h-1})$ respecting the construction such that, almost surely,
    \begin{align}\label{eq:coupfs}
    \pp_{\coupfs^{(h)}}\left[\Bfsh^c \mid Y_{h-1} \right] &=  \pp_{\coupfs^{(h)}}\left[ \distavec(\atelinter_{h},\atel_{h}) \not \preceq \epsvec \mid Y_{h-1} \right] \\
    &= \drobvec\,( \pihatsigh(\stel_h) \parallel \pistreph(\stel_h))].
\end{align}
Moreover, that same proposition ensures measurability of $\seqs \to \drobvec\,( \pihatsigh(\seqs) \parallel \pistreph(\seqs))$.
\item Since $\atelinter_{h} \mid \cF_h \sim \pihatsigh(\stel_h)$ and $\arepinter_{h+1} \mid \cF_h \sim \pihatsigh(\srep_h)$, and since  $\pihatsigh(\cdot)$ is $\gamhat$-TVC by assumption,  
\begin{align}
\TV(\pp_{\atelinter_{h} \mid  Y_{h-1}},\pp_{ \arepinter_{h} \mid  Y_{h-1}}) \le \gamhat \circ \disttvc(\srep_h,\stel_h). 
\end{align}

\Cref{cor:first_TV}  implies that there is a coupling $\coupinterr^{(h)}$ between $(\atelinter_{h},\arepinter_{h},Y_{h-1})$ such that
\begin{align}\label{eq:coupinterr}
\pp_{\coupinterr^{(h)}}[\Binterh^c \mid Y_{h-1}] = \pp_{\coupinterr^{(h)}}\left[\atelinter_{h} \ne \arepinter_{h}  \mid Y_{h-1}\right] &\le  \gamhat \circ  \disttvc(\stel_h,\srep_h)
\end{align}
\item  Similarly, since $\arepinter_{h} \mid \cF_{h-1} \sim \pihat_h(\srep_h)$ and $\seqahat_{h+1} \mid \cF_{h-1}\sim \pihat_h(\shat_h)$, $\pihat_h(\cdot)$ is $\gamhat$-TVC,  \Cref{cor:first_TV} implies that there is a coupling $\coupahat^{(h)}$ between $(\arepinter_{h},\seqahat_{h},Y_{h-1})$ such that
\begin{align}\label{eq:coupahat}
\pp_{\coupahat^{(h)}}[\Bhath^c \mid Y_{h-1}] = \pp_{\coupahat^{(h)}}\left[\seqahat_{h} \ne \arepinter_{h} \mid Y_{h-1}  \right] \le \gamhat \circ \disttvc(\srep_h,\shat_h)
\end{align}
\end{itemize}

We can then apply the gluing lemma (\Cref{lem:couplinggluing}) to 
\begin{align}
X_{h,1} &= (\phiZ(\ssq_h),\atel_h,Y_{h-1}) \\ 
X_{h,2} &= (\phiZ(\sreptil_h),\arep_h,Y_{h-1}) \\
 X_{h,3} &= (\atel_h,\atelinter_h,Y_{h-1}) \\
  X_{h,4} &= (\atelinter_h,\arepinter_h,Y_{h-1}) \\
   X_{h,5} &= (\arepinter_h,\ahat_h,Y_{h-1})  
\end{align}
with 
\begin{align}
(X_{h,1},X_{h,2}) \sim \couptel^{(h)},\quad (X_{h,2},X_{h,3}) \sim  \coupfs^{(h)}, \quad (X_{h,3},X_{h,4})\sim \coupinterr^{(h)}, \quad (X_{h,4},X_{h,5})\sim\coupahat^{(h)}.
\end{align}
\Cref{lem:couplinggluing} guarantees the existence of a coupling $\mu^{(h)}$ consident with all sub-couplings $\couptel^{(h)}$, $\coupfs^{(h)},\coupinter^{(h)},\coupahat^{(h)}$. Then, $\coup^{(h)}$-almost surely (and using that $\cF_{h-1}$ is precisely the $\upsigma$-algebra generated by $Y_{h-1}$)
\begin{align}
&\Pr_{\coup^{(h)}}[\Ballh^c \mid \cF_{h-1}] \\
&\le \Pr_{\coup^{(h)}}[\Btelh^c \mid \cF_{h-1}] + \Pr_{\coup^{(h)}}[\Bfsh^c \cF_{h-1}] +  \Pr_{\coup^{(h)}}[\Binterh^c \cF_{h-1}]+\Pr_{\coup^{(h)}}[\Bhath^c \cF_{h-1}]\\
&\le \gamhat \circ \disttvc(\srep_h,\shat_h) + (\gamhat + \gamtvcsig) \circ \disttvc(\srep_h,\stel_h)  + \drobvec\,( \pihatsigh(\stel_h) \parallel \pistreph(\stel_h))\\
&= \gamhat \circ \disttvc(\srep_h,\shat_h) + (\gamhat + \gamtvcsig) \circ \disttvc(\srep_h,\stel_h)  + \drobvec\,( \pihatsigh(\stel_h) \parallel \pistreph(\stel_h))
\end{align}
This concludes the inductive construction.


For the second statement, notice that the events $\Callbarh \cap \Ballbarh[h-1]$ are $\cF_h$ measurable (thus determined by $\coup^{(h-1)}$) and, when they hold, $\distsvec(\srep_h,\stel_h) \preceq \gamipsvec(2r)$ and $\dists(\srep_h,\shat_h)  \preceq \epsvec$. For our purposes, we use $\disttvc = \distsi[1](\srep_h,\stel_h) \preceq \gamipsone(2r)$ and $\dists(\srep_h,\shat_h)  \preceq \epsvec_1$. Hence, 
\begin{align}
\max_{h\in [H]}\Pr_{\coup}[ \Ballh^c \cap \Callbarh \cap \Ballbarh[h-1]] &\le \gamhat(\epsvec_1) + (\gamhat + \gamtvcsig) \circ \gamipsone(2r) \\
&\quad+ \drobvec\,( \pihatsigh(\stel_h) \parallel \pistreph(\stel_h)).
\end{align}
The result follows.


\subsection{Proof of Theorem \ref{thm:smooth_cor}, and generalization to direct decompositions}\label{app:smoothcor_proof}

In this subsection, we consider the special case dealt with in \Cref{thm:smooth_cor}.  Note that there always exists a trivial direct decomposition that is compatible with all policies and dynamics simply by letting $\cV = \emptyset$ and $\cS = \cZ$.  We prove here the version of the result that involves a possibly nontrivial direct decomposition, as we will instantiate this in our control setting by letting $\cZ = \left\{ \pathm[h] \right\}$ and $\cS = \left\{ \pathc[h] \right\}$, i.e., projecting $\pathc[h]$ onto the last $\taum$ coordinates gives $\seqz_h$. We further consider a restriction of IPS to consider kernels absolutely continuous with respect to $\Psth$ in their $\cZ$ component. 
\begin{definition}[Restricted IPS]\label{defn:ips_restricted}
For a non-decreasing maps $\gamipsone,\gamipstwo:\R_{\ge 0} \to  \R_{\ge 0}$ a  pseudometric $\distips:\cS \times \cS \to \R$ (possibly other than $\dists$ or $\disttvc$), and $\rips > 0$, we say a policy $\pi$ is \emph{$(\gamipsone,\gamipstwo,\distips,\rips)$-restricted IPS} if the following holds for any $r \in [0,\rips]$. Consider any sequence of kernels $\lawW_1,\dots,\lawW_H:\cS \to \laws(\cS)$ satisfying 
\begin{align}
\max_{h,\seqs \in \cS}\Pr_{\tilde \seqs\sim \lawW_h(\seqs)}[\distips(\tilde \seqs,\seqs) \le r] = 1, \quad \forall s, \quad \phiZ \circ \lawW_h(\seqs_h) \ll \phiZ \circ \Psth. 
\end{align}
 and define a process $\seqs_1 \sim \Dinit$, $\tilde\seqs_h \sim \lawW_h(\seqs_h),\seqa_h \sim \pi_h(\tilde \seqs_h)$, and $\seqs_{h+1} := F_h(\seqs_h,\seqa_h)$. Then, almost surely, (a) the sequence $(\seqs_{1:H+1},\seqa_{1:H})$ is input-stable w.r.t $(\dists,\dista)$ (b) $\max_{h \in [H]} \disttvc(F_h(\tilde\seqs_h,\seqa_h),\seqs_{h+1}) \le \gamipsone(r)$ and (c) $\max_{h \in [H]} \dists(F_h(\tilde\seqs_h,\seqa_h),\seqs_{h+1}) \le \gamipstwo(r)$.
\end{definition}

Note that the above is a  slightly weaker condition than the one in \Cref{defn:ips_body} in the main text and consequently, the following theorem which uses it as an assumption implies \Cref{thm:smooth_cor} in the body.
\begin{theorem}\label{thm:smooth_cor_decomp} Suppose $\cS = \cZ \oplus \cV$ as in \Cref{defn:direct_decomp} and projections $\phiZ,\phiV$, which is compatible with the dynamics and with given policies $\pihat,\pist$, smoothing kernel $\Wsig$, and pseudometric $\distips$.
Suppose $\pist$ satisfies $(\gamipsone,\gamipstwo,\distips,\rips)$-restricted IPS (\Cref{defn:ips_restricted}) and $\phiZ \circ \Wsig$ is $\gamma_{\sigma}$-TVC. Let $\epsilon > 0$, $r \in (0,\frac{1}{2}\rips]$; define $p_r := \sup_{\seqs}\Pr_{\seqs' \sim \Wsig(\seqs)}[\distips(\seqs',\seqs) >  r]$ and $\epsilon' := \epsilon+\gamipstwo(2r)$. Then, for any policy $\pihat$,  both  $\gapjoint (\pihat \circ \Wsig \parallel \pistrep)$ and  $\gapmarg[\epsilon'] (\pihat \circ \Wsig \parallel \pist)$ are upper bounded by
\begin{align}
%\inf_{r > 0}  
H\left(2p_r +  3\gamma_{\sigma}(\max\{\epsilon,\gamipsone(2r)\})\right)  + \textstyle \sum_{h=1}^H\Exp_{\sstar_h \sim \Psth}\Exp_{\sstartil_h \sim \Wsig(\sstar_h) } \drob( \pihat_{h}(\sstartil_h) \parallel \pidec(\sstartil_h))  . \label{eq:smooth_ub}
\end{align}
\end{theorem}


Consider the special case $K = 2$ with $\distsi[1] = \disttvc$, $\distsi[2] = \dists$, $\distai[1] = \distai[2] = \dista$ and $\epsvec = (\epsilon, \epsilon)$.  In this case, applying \eqref{eq:smooth_ub_app_two}, we see that
\begin{align}
    &\gapjointvec(\pihatsig \parallel \pistrep) \vee \gapmargvec[\epsvecmarg](\pihatsig \parallel \pistrep) \\
    &\leq \pips + H\left(2p_r +  3\gamma_{\sigma}(\max\{\epsilon,\gamipsone(2r)\}\right)  + \textstyle \sum_{h=1}^H\Exp_{\sstar_h \sim \Psth}\Exp_{\sstartil_h \sim \Wsig(\sstar_h) } \drobvec( \pihat_{h}(\sstartil_h) \parallel \pidec(\sstartil_h))
\end{align}
We now observe that under this convention,
\begin{align}
    \gapjoint(\pihatsig \parallel \pistrep) &= \inf_{\coup_1} \pp_{\coup_1}[\max_{h \in [H]} \dists(\shat_{h+1}, \sstar_{h+1}) \vee \dista(\ahat_h, \astar_h) > \epsilon] \\
    &\leq \inf_{\coup_1} \pp_{\coup_1}\left[ \max_{h \in [H]} \left( \disttvc(\shat_{h+1}, \sstar_{h+1}), \dists(\shat_{h+1}, \sstar_{h+1}) \right) \vee \left( \dista(\ahat_h, \astar_h), \dista(\ahat_h, \astar_h) \right) \not \preceq \epsvec\right] \\
    &= \gapjointvec(\pihatsig \parallel \pistrep)
\end{align}
and similarly $\gapmarg[\epsilon'](\pihatsig \parallel \pist) \leq \gapmargvec[\epsvec + \gamips(2r)](\pihatsig \parallel \pist)$.  From the construction of $\distavec$, however, we see that $\left\{ \distavec(\seqa, \seqa') \not \preceq \epsvec \right\} = \left\{ \dista(\seqa, \seqa') > \epsilon \right\}$ for all $\seqa, \seqa'$ and thus for all $h \in [H]$,
\begin{align}
    \drobvec(\pihat_{h}(\sstartil_h) \parallel \pist_h(\sstartil_h)) &= \inf_{\coup_2} \pp_{\coup_2}\left[ \distavec(\seqahat_h, \seqast_h) \not \preceq \epsvec \right] \\
    &= \inf_{\coup_2} \pp_{\coup_2}\left[ \dista(\seqahat_h, \seqast_h) \geq \epsilon \right] \\
    &= \drob(\pihat_h(\sstartil_h) \parallel \pist_h(\sstartil_h)).
\end{align}
Plugging in to \eqref{eq:smooth_ub_app_two} concludes the proof.











\end{document}
