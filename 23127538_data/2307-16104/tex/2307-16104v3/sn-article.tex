%%%%%%%%%%%%%%%%%%%%%%%%%%%%%%%%%%%%%%%%%%%%%%%%%%%%%%%%%%%%%%%%%%%%%
%%                                                                 %%
%% Please do not use \input{...} to include other tex files.       %%
%% Submit your LaTeX manuscript as one .tex document.              %%
%%                                                                 %%
%% All additional figures and files should be attached             %%
%% separately and not embedded in the \TeX\ document itself.       %%
%%                                                                 %%
%%%%%%%%%%%%%%%%%%%%%%%%%%%%%%%%%%%%%%%%%%%%%%%%%%%%%%%%%%%%%%%%%%%%%

%%\documentclass[referee,sn-basic]{sn-jnl}% referee option is meant for double line spacing

%%=======================================================%%
%% to print line numbers in the margin use lineno option %%
%%=======================================================%%

%%\documentclass[lineno,sn-basic]{sn-jnl}% Basic Springer Nature Reference Style/Chemistry Reference Style

%%======================================================%%
%% to compile with pdflatex/xelatex use pdflatex option %%
%%======================================================%%

%%\documentclass[pdflatex,sn-basic]{sn-jnl}% Basic Springer Nature Reference Style/Chemistryhttps://www.overleaf.com/project/62d5975e2b0388593333a0f4 Reference Style

%%\documentclass[sn-basic]{sn-jnl}% Basic Springer Nature Reference Style/Chemistry Reference Style
\documentclass[pdflatex,sn-mathphys]{sn-jnl}% Math and Physical Sciences Reference Style
%%\documentclass[sn-aps]{sn-jnl}% American Physical Society (APS) Reference Style
%%\documentclass[sn-vancouver]{sn-jnl}% Vancouver Reference Style
%%\documentclass[sn-apa]{sn-jnl}% APA Reference Style
%%\documentclass[sn-chicago]{sn-jnl}% Chicago-based Humanities Reference Style
%%\documentclass[sn-standardnature]{sn-jnl}% Standard Nature Portfolio Reference Stylehttps://www.overleaf.com/project/62d5975e2b0388593333a0f4
%%\documentclass[default]{sn-jnl}% Defaulthttps://www.overleaf.com/project/62d5975e2b0388593333a0f4
%%\documentclass[default,iicol]{sn-jnl}% Default with double column layouthttps://www.overleaf.com/project/62d5975e2b0388593333a0f4

%%%% Standard Packages
%%<additional latex packages if required can be included here>
%%%%

%%%%%=============================================================================%%%%
%%%%  Remarks: This template is provided to aid authors with the preparation
%%%%  of original research articles intended for submission to journals published 
%%%%  by Springer Nature. The guidance has been prepared in partnership with 
%%%%  production teams to conform to Springer Nature technical requirements. 
%%%%  Editorial and presentation requirements differ among journal portfolios and 
%%%%  research disciplines. You may find sections in this template are irrelevant 
%%%%  to your work and are empowered to omit any such section if allowed by the 
%%%%  journal you intend to submit to. The submission guidelines and policies 
%%%%  of the journal take precedence. A detailed User Manual is available in the 
%%%%  template package for technical guidance.https://braintex.goog/project/647c76f65ee25900821ad7cd
%%%%%=============================================================================%%%%

\jyear{2021}%

%% as per the requirement new theorem styles can be included as shown below
\theoremstyle{thmstyleone}%
\newtheorem{theorem}{Theorem}%  meant for continuous numbers
%%\newtheorem{theorem}{Theorem}[section]% meant for sectionwise numbers
%% optional argument [theorem] produces theorem numbering sequence instead of independent numbers for Proposition
\newtheorem{proposition}[theorem]{Proposition}% 
%%\newtheorem{proposition}{Proposition}% to get separate numbers for theorem and proposition etc.

\theoremstyle{thmstyletwo}%
\newtheorem{example}{Example}%
\newtheorem{remark}{Remark}%

\theoremstyle{thmstylethree}%
\newtheorem{definition}{Definition}%

\usepackage{natbib}   % omit 'round' option if you prefer square brackets
% \bibliographystyle{plainnat}

\raggedbottom
%%\unnumbered% uncomment this for unnumbered level heads

\begin{document}

\title[Global AI Flood Forecasting]{AI Increases Global Access to Reliable Flood Forecasts}

%%=============================================================%%
%% Prefix	-> \pfx{Dr}
%% GivenName	-> \fnm{Joergen W.}
%% Particle	-> \spfx{van der} -> surname prefix
%% FamilyName	-> \sur{Ploeg}
%% Suffix	-> \sfx{IV}
%% NatureName	-> \tanm{Poet Laureate} -> Title after name
%% Degrees	-> \dgr{MSc, PhD}
%% \author*[1,2]{\pfx{Dr} \fnm{Joergen W.} \spfx{van der} \sur{Ploeg} \sfx{IV} \tanm{Poet Laureate} 
%%                 \dgr{MSc, PhD}}\email{iauthor@gmail.com}
%%=============================================================%%

\author[1]{\fnm{Grey} \sur{Nearing}}\email{gsnearing@google.com}
\author[1]{\fnm{Deborah} \sur{Cohen}}
\author[1]{\fnm{Vusumuzi} \sur{Dube}}
\author[1]{\fnm{Martin} \sur{Gauch}}
\author[1]{\fnm{Oren} \sur{Gilon}}
\author[2]{\fnm{Shaun} \sur{Harrigan}}
\author[1]{\fnm{Avinatan} \sur{Hassidim}}
\author[1]{\fnm{Frederik} \sur{Kratzert}}
\author[1]{\fnm{Asher} \sur{Metzger}}
\author[3]{\fnm{Sella} \sur{Nevo}}
\author[2]{\fnm{Florian} \sur{Pappenberger}}
\author[2]{\fnm{Christel} \sur{Prudhomme}}
\author[1]{\fnm{Guy} \sur{Shalev}}
\author[1]{\fnm{Shlomo} \sur{Shenzis}}
\author[1]{\fnm{Tadele} \sur{Tekalign}}
\author[1]{\fnm{Dana} \sur{Weitzner}}
\author[1]{\fnm{Yossi} \sur{Matias}}

\affil[1]{\orgdiv{Google}}
\affil[2]{\orgdiv{European Center for Medium Range Weather Forecasting}}
\affil[3]{\orgdiv{RAND Corporation, work was done while at Google}}

\abstract{
Floods are one of the most common and impactful natural disasters, with a disproportionate impact in developing countries that often lack dense streamflow monitoring networks. Accurate and timely warnings are critical for mitigating flood risks, but accurate hydrological simulation models typically must be calibrated to long data records in each watershed where they are applied. We developed an Artificial Intelligence (AI) model to predict extreme hydrological events at timescales up to 7 days in advance. This model significantly outperforms current state of the art global hydrology models (the Copernicus Emergency Management Service Global Flood Awareness System) across all continents, lead times, and return periods. AI is especially effective at forecasting in ungauged basins, which is important because only a few percent of the world's watersheds have stream gauges, with a disproportionate number of ungauged basins in developing countries that are especially vulnerable to the human impacts of flooding. We produce forecasts of extreme events in South America and Africa that achieve reliability approaching the current state of the art in Europe and North America, and we achieve reliability at between 4 and 6-day lead times that are similar to current state of the art nowcasts (0-day lead time). Additionally, we achieve accuracies over 10-year return period events that are similar to current accuracies over 2-year return period events, meaning that AI can provide warnings earlier and over larger and more impactful events. The model that we develop in this paper has been incorporated into an operational early warning system that produces publicly available (free and open) forecasts in real time in over 80 countries. This work using AI and open data highlights a need for increasing the availability of hydrological data to continue to improve global access to reliable flood warnings.
}

\keywords{flood forecasting, artificial intelligence, machine learning}

\maketitle

\section{Flood forecasting is limited by data} \label{sec:introduction}

Floods are the most common type of natural disaster \cite{unisdr2015human}, and flood-related disasters have more than doubled since the year 2000 \cite{wmo20212021}. This increase in flood-related disasters is driven by an accelerating hydrological cycle caused by anthropogenic climate change \cite{milly2002increasing, tabari2020climate}. Early warning systems are an effective way to mitigate flood risks -- such systems have been shown to reduce flood-related fatalities by up to 43\% \cite{world2014global,wmo2013global} and economic costs by between 35-50\% \cite{pilon2002guidelines,rogers2011costs}. Populations in low- and middle-income countries make up almost 90\% of the 1.8 billion people vulnerable to flood risks \cite{rentschler2022flood}. The World Bank estimated that upgrading flood early warning systems in developing countries to the standards of developed countries would save an average of 23,000 lives per year \cite{hallegatte2012cost}. 

In this paper, we evaluate the extent to which artificial intelligence (AI) trained on open, public data sets can be used to improve global access to forecasts of extreme riverine events. Based on the model and experiments that are described in this paper, we developed an operational system that produces short term (7-day) flood forecasts in over 80 countries -- these forecasts are available in real time without barriers to access (i.e., monetary charge or website registration). This system has been operational since October, 2022.

A major challenge to riverine forecasting is that hydrological prediction models must be calibrated to individual watersheds using long data records \cite{razavi2013anefficient,LI2010378}. Watersheds that lack stream gauges to supply data for calibration are called \textit{ungauged basins}, and the problem of `Prediction in Ungauged Basins' (PUB) was the decadal problem of the International Association of Hydrological Sciences (IAHS) from 2003--2012 \cite{sivapalan2003iahs}. At the end of the PUB decade, the IAHS reported that little progress had been made against the problem, stating that \textit{``much of the success so far has been in gauged rather than in ungauged basins, which has negative effects in particular for developing countries"} \cite{hrachowitz2013decade}. 

Only a few percent of the world's watersheds are gauged%\footnote{At the finest level of delineation in the HydroSHEDS map \citep{wickel2007hydrosheds}, there are slightly more than one million watersheds globally. Open data records are available for a few tens of thousands of streamflow gauges globally, with the majority of these in Europe and the United States.}. Furthermore, 
, and stream gauges are not distributed uniformly across the world. There is a strong correlation between national GDP and the total publicly available streamflow observation data record in a given country (Figure \ref{fig:gauge-gdp-correlation}), which means that high quality forecasts are especially challenging in areas that are most vulnerable to the human impacts of flooding.

% Figure environment removed

The current state of the art for real time global scale hydrological prediction is the Global Flood Awareness System (GloFAS) \cite{alfieri2013glofas, harrigan2023daily}, which is the global flood forecasting system of Copernicus Emergency Management Service (CEMS), delivered under the responsibility of the European Commission’s Joint Research Centre (JRC) and operated by the European Centre for Medium-Range Weather Forecasts (ECMWF) in its role of CEMS hydrological Forecast Computation Centre. We use GloFAS version 3, which is the operational version at the time of writing this paper. Other forecasting systems exist for different parts of the world \citep[e.g.,][]{arheimer2020global, souffront2019hydrologic, sheffield2014drought}, and many countries have national agencies responsible for producing early warnings. Given the severity of impacts that floods have on communities around the world, we consider it critical that forecasting agencies evaluate and benchmark their predictions, warnings, and approaches. As such, we will benchmark AI forecasts with GloFAS to understand the extent to which AI helps to improve global forecasting.  %To that end, as a supplement to this paper, we are releasing an archive of global historical reforecasts from our AI model over the period 2014 - present, and a reanalysis archive dating back to 1984 (See Data Availability). 


\section{AI improves reliability and lead times of forecasts of extreme events in rivers globally} \label{sec:results}

In previous work \cite{kratzert2019toward}, we showed that machine learning (ML) can be used to develop hydrological simulation models that are transferable to ungauged basins. Here we develop that into a global scale forecasting system with the goal of understanding scalability and reliability. We use the term \textit{simulation} to refer to making predictions up to the present time, and we use the term \textit{forecasting} to refer to making short-term future predictions. This paper addresses the following question: \textbf{Given the publicly available global streamflow data record, is it possible to provide accurate river forecasts across large scales, especially of extreme events, and how does this compare to current state of the art?} 

The AI model developed for this study uses Long Short-Term Memory (LSTM) networks \citep{hochreiter1997long} to predict daily streamflow through a 7-day forecast horizon. The model is described in detail in Section \ref{sec:methods:model} and a version of the model suitable for research is implemented in the open source NeuralHydrology repository \cite{kratzert2022neuralhydrology}. Input, target, and evaluation data are described in Section \ref{sec:methods}. No past streamflow data were used as inputs to the model. An operational version of the model is run daily to produce publicly available forecasts (\url{https://g.co/floodhub}) for locations that have been quality controlled by a human-over-the-loop process, meaning that a hydrologist approves forecasts to be released at each individual stream reach based on a combination of quantitative and qualitative assessment of long-term predicted hydrographs (our operational launch approval process is complex and out of scope for this paper).

Our AI forecast model was trained and tested out-of-sample using k-fold cross validation splits across 5,680 streamflow gauges (see Section \ref{sec:methods:experiments} for a description of cross validation experiments). Other types of cross validation experiments are reported in Supplementary Material (e.g., by withholding entire climate zones or entire continents). All metrics reported in this paper for our AI forecasting model were calculated with streamflow gauge data from watersheds that were not present in training, meaning that we are effectively quantifying expected skill in ungauged basins. 

During evaluation, not all results reported in this paper use all 5,680 gauges -- sample size is noted for each result. This is due to two reasons. First, because some gauges do not have enough data to calculate precision and recall scores over certain return period events (i.e., either there are no predicted or observed events of a given magnitude during the period where observation data is available). Second, because of issues extracting GloFAS predictions for some basins. GloFAS predictions for each gauge were extracted by finding the GloFAS pixel (~10 km horizontal resolution) within a 2 pixel radius of the reported gauge location that has the closest GloFAS modelled upstream drainage area to that which is reported by the Global Runoff Data Center (GRDC). Any gauge that does not have a pixel within the radius with a drainage area that matches to within 10\% was discarded from the evaluation gauge set. 

There are a large number of metrics that hydrologists use to assess hydrograph simulations \cite{https://doi.org/10.1029/2022WR033918}, and extreme events in particular \cite{wwrp2016forecast}. Our objective is to understand the reliability of forecasts of extreme events, so we report precision, recall, and F1 scores calculated on different return period events. These reliability metrics are described in Section \ref{sec:methods:metrics}. Other standard hydrological metrics, like bias, correlation, Nash-Sutcliffe Efficiencies, and Kling-Gupta Efficiencies, are reported in Supplementary Material, and results are qualitatively similar to the extreme event precision and recall metrics that we report in the main body of this paper. All statistical significance values reported in this paper were assessed using Wilcoxon (paired) ranked sum tests. CEMS releases a full historical reanalysis (without lead times) but only releases long-term historical reforecasts every three days (real time forecasts are archived, but the record is too short to derive return period statistics). Given that reliability metrics must consider the timing of event peaks, this means that it is only possible to benchmark GloFAS at a 0-day lead time.

Figure \ref{fig:global-f1-scores-map} shows the global distribution of F1 scores for 5-year return period events at a 0-day lead time over the time period 1984--2021 (N=3,236). Lead time is expressed as the number of days from the time of prediction, such that a 0-day lead time means that streamflow predictions are for the current day. %Precision and recall scores are highly correlated (GloFAS $r^{2}=0.77$, AI model $r^{2}=0.74$), meaning that F1 scores are a reasonable summary statistic for understanding general trends from global maps.
The AI model improved (was at least equivalent to) over GloFAS in 75\% (76\%), 78\% (81\%), 66\% (78\%), and 54\% (79\%) of gauges for return period events of 1-year ($N=3,497$, $p=4e-260$, $d=0.52$), 2-years ($N=3,495$, $p=0$, $d=0.67$), 5-years ($N=3,236$, $p=1e-198$; $d=0.59$), and 10-years ($N=2,856$, $p=2e-111$, $d=0.49$). 

% Figure environment removed

More extreme hydrological events (i.e., events with larger return periods) are both more important and (usually) more difficult to predict. A common concern \cite[e.g.,][]{sellars2018grand, todini2007hydrological, herath2021hydrologically, reichstein2019deep} about using AI or other types of data-driven approaches for predicting extreme events is that reliability might degrade over events that are rare in the training data. There is evidence that this concern might not be valid for streamflow modeling \cite{hess-26-3377-2022}. Figure \ref{fig:global-reliability-with-return-period} shows distributions over precision and recall for different return period events. Precision and recall scores are higher for the AI model at all return periods ($N>=3,000$, $p<1e-90$, $d>0.37$). Differences between precision scores from the AI model over 10-year return period events and from GloFAS over 2-year return period events are not significant ($N=3,165$, $p>0.4$), and recall scores from the AI model for 10-year events are better than GloFAS recall scores for 2-year events ($N=3,183$, $p<1e-10$).

% Figure environment removed

Figure \ref{fig:lead-time-reliability-distributions} shows F1 scores over lead times through the 7-day forecast horizon for return periods between 1 and 10 years. For 2-year events, AI forecasts have better reliability up to a 5-day lead time than GloFAS has at a 0-day lead time ($N=1,955$, $p=4e-10$), and no statistical difference was found between GloFAS at a 0-day lead time and the AI model at a 6-day lead time  ($N=1,960$, $p=0.45$). For 5-year events, the AI model was better at a 4-day lead time than GloFAS at a 0-day lead time ($N=972$, $p=1e-5$) and no statistical differences were found between the AI model at 5-day ($N=962$, $p=0.60$) and 6-day ($N=959$, $p=0.17$) lead times vs. GloFAS at a 0-day lead time.

% Figure environment removed

There are differences in the reliability of forecasts in different areas of the world. Figure \ref{fig:continent-reliability-scores-distributions} shows distributions of F1 scores for different continents and return periods. Over 5-year return period events, GloFAS has a 41\% difference between mean F1 scores in the lowest scoring continent (South America: $f1=0.16$) and the highest scoring continent (Europe: $f1=0.28$), meaning that, on average, true positive predictions are almost twice as likely (at a proportional rate) in certain continents. The AI model has higher average precision, recall, and F1 scores in all continents over all return periods, and the differences are significant ($p<1e-2$, $d>0.15$) except in Africa over 1-year return period events ($p=0.28$). Additionally, the AI model has reliability (F1 scores) in all continents that either match or exceed GloFAS reliability in Europe for the same return period ($p<1e-2$) -- with exceptions in Africa and South America for 1 and 2-year return period events (in the case of comparing results across dissimilar continents, we used an unpaired ranked sum test). 

% Figure environment removed

\subsection{Is forecast reliability predictable?} 
\label{sec:results:predictability}
A challenge to forecasting in ungauged basins is that there is often no way to evaluate reliability in locations without ground-truth data. A desirable quality of a model is that forecast skill should be predictable from other observable variables, like mapped or remotely sensed geographical and/or geophysical data. Additionally, while AI-based forecasting offers better reliability in most places, this is not the case everywhere. It would be beneficial to be able to predict where different models can be expected to be more or less reliable.

We have generally found that it is difficult to use catchment attributes to predict where one model might perform better than another. Figure \ref{fig:which-model-where} shows the confusion matrix from a random forest classifier (trained with stratified k-fold cross validation and balanced sampling) that predicts whether the AI model or GloFAS performed better (or similar) in each individual watershed. Input features to this classifier are a subset of HydroATLAS basin attributes \cite{linke2019global}. 

% Figure environment removed

Although it is difficult to predict where one model will perform better than another, it \textit{is} possible to predict, with some skill, where an individual model will perform well vs. poorly. As an example, Figure \ref{fig:prediction-confusion-matrices} shows confusion matrices from random forest classifiers that predict whether F1 scores for out-of-sample gauges (effectively ungauged locations) will be above or below the mean over all evaluation gauges. Both models (the AI model and GloFAS) have similar overall predictability (GloFAS: 67\%, AI model: 71\%, measured as micro-averaged precision and recall). 

% Figure environment removed

Feature importances from these reliability classifiers are an indication about which geophysical attributes determine high vs. low reliability (i.e., what kind of watersheds do these models simulate well vs. poorly). The most important features for the AI model are: drainage area, mean annual actual evapotranspiration (AET), elevation, and potential evapotranspiration, while the most important features for GloFAS are: slope, nighttime lights index (anthropogenic development in a watershed), elevation, and AET. Full feature importance rankings are in Supplementary Material. Correlations between attributes and reliability scores are generally low, indicating a high degree of parameter interaction.

AET is an (inverse) indicator of aridity, and hydrology models usually perform better in humid basins because peaky hydrographs that occur in arid watersheds are difficult to model. This effect is present for both models. The AI model is more sensitive to basin size (drainage area) and generally performs better in smaller basins. This indicates a way that ML streamflow modeling might be improved, for example by focusing training or fine tuning on larger basins, or by implementing an explicit routing or graph model to allow for direct modeling of subwatersheds or smaller hydrological response units - for example as outlined by \cite{kratzert2021large}.

A global map of the predicted skill from a regression (rather than classifier) version of this random forest skill predictor are shown in Figure \ref{fig:global_predicted_skill_map} for all 1.03 million level 12 HydroBASIN watersheds \cite{lehner2013global}. This gives some indication about where a global version of the ungauged AI forecast model is expected to perform well.

% Figure environment removed


% \subsection{The value of increasing access to data} 
% \label{sec:results:value-of-new-data}

% % Figure environment removed

\section{Methods} \label{sec:methods}
\subsection{Model} \label{sec:methods:model}

Our streamflow forecasting model extends work by \cite{kratzert2019towards}, who developed a hydrologic \textit{simulation} model using a Long Short-Term Memory (LSTM) network that simulates sequences of streamflow data from sequences of meteorological input data. Building on that prior work, we use an encoder/decoder approach that treats a historical sequence of meteorological and geophysical input data with one LSTM and uses a separate LSTM over the forecast horizon that uses meteorological forecasts as the input sequence. The model architecture is illustrated in Figure \ref{fig:model-architecture-diagram}.

% The challenge extending this to a forecast model is that meteorological data have different dimensions and different error distributions in hindcast vs. forecast. As an example, satellite retrievals of precipitation exist only up to the forecast issue time (the present) so any satellite data inputs will be missing from forecast time steps. Additionally, precipitation estimates from global circulation models generally degrade in quality with increasing time since the most recent data assimilation. Extending an LSTM sequence predictor through a forecast horizon requires explicitly accounting for these differences. 

% Our group and others \citep{nevo2021flood, franken2022operational} have previously dealt with this using an encoder-decoder model structure, where one sequence model (e.g., an LSTM) runs up to the forecast issue time, and then the cell and hidden states of that encoder model are used to initialize the cell and hidden states of a forecast (decoder) LSTM, which is run over the forecast horizon. The gates of the encoder (hindcast) and decoder (forecast) LSTMs have different weights and biases to handle different inputs with different distributions. We found that this encoder-decoder approach can cause artifacts in forecast trajectories (we don't discuss this problem in detail in this paper), and therefore the model that we use operationally and in the experiments reported in this paper uses a stacked LSTM where one LSTM receives a hindcast meteorological sequence and the time-shifted outputs from that model are concatenated with a meteorological forecast sequence as inputs into a second LSTM. This is illustrated in Figure \ref{fig:model-architecture-diagram}. Through extensive benchmarking, we found that this gives predictions that are as accurate, on average, as the encoder-decoder approach but are also self-consistent over a 0-7 day forecast horizon.

% Figure environment removed

The model uses a hindcast sequence length of 365 days, meaning that every forecast sequence (0-7 days) saw meteorological input data from the preceding 365 days and meteorological forecast data over the 0-7 day forecast horizon. We used a hidden size of 256 cell states for both the encoder and decoder LSTMs, a linear cell state transfer network and a nonlinear hidden state transfer network (fully connected layer with hyperbolic tangent activation functions). The model predicts a mixture probability distribution (of three asymmetric Laplacians) over area-normalized streamflow discharge, as described by \cite{klotz2022uncertainty}, and the loss function was the negative log-likelihood of that mixture density function. 

\subsection{Data} \label{sec:methods:data}
\subsubsection{Input Data} \label{sec:methods:data:input-data}
Input data came from the following sources:
\begin{itemize}
    \item Daily-aggregated single-level forecasts from the ECMWF IFS (Integrated Forecast System) HRES (High Resolution) atmospheric model. Variables include: total precipitation (TP), 2-meter temperature (T2M), surface net solar radiation (SSR), surface net thermal radiation (STR), snowfall (SF), and surface pressure (SP).
    \item The same six variables from the ECMWF ERA5-Land reanalysis. 
    \item Precipitation estimates from the NOAA CPC Global Unified Gauge-Based Analysis of Daily Precipitation.
    \item Precipitation estimates from the NASA IMERG (Integrated Multi-satellitE Retrievals for GPM) early run.
    \item Geological, geophysical, and anthropogenic basin attributes from the HydroATLAS database \cite{linke2019global}.
\end{itemize}
No streamflow data were used as inputs into the model. All input data were area-weighted averaged over basin polygons over the total upstream area of each gauge or prediction point. The total upstream area for the 5,680 evaluation gauges used in this study ranged from 2.1 to 4,690,998 square kilometers.

Figure \ref{fig:data-timeline} shows the time periods of available data from each source. During training, missing data was imputed either by using a similar variable from another data source (e.g., HRES data was imputed with ERA5-Land data), or by imputing with a mean value and then adding a binary flag to indicate an imputed value, as described by \cite{nearing2022data}. 

% Figure environment removed

\subsubsection{Target Data} \label{sec:methods:data:target-data}
Training and test targets came from the Global Runoff Data Center (GRDC), which is partially supported by the World Meteorological Organization (WMO) \citep{grdc}. We removed watersheds from this data set where the drainage area reported by GRDC differed by more than 20\% from the drainage area that we calculated using watershed polygons from the HydroBASINS repository (this was necessary to ensure that poor quality data, due to imperfect catchment delineation, was not used for training). This left us with 5,680 gauges. The locations and lengths of data record for each streamflow gauge are shown in Supplementary Material. 

% % Figure environment removed

\subsection{Experiments} \label{sec:methods:experiments}
We assessed model performance using a set of cross validation experiments. Data from 5,680 gauges were split in two ways. First, split in time, as described in Section \ref{sec:methods:model}, so that no training data from any gauge was used from within 1 year of any test data from any gauge. Second, split in space using k-fold cross validation with $k=10$. The gauges were divided into 10 groups randomly without replacement, and models were trained on nine groups and tested on the tenth. This pair of cross validation processes were repeated so that all data (1984--2021) from all gauges were predicted in a way that was out-of-sample in both time and space. Other cross validation experiments are reported in Supplementary Material.

\subsection{Metrics} \label{sec:methods:metrics}
In Section \ref{sec:results} we reported precision and recall metrics over predictions of events with magnitudes defined by return periods. Other standard hydrologic metrics are reported in Supplementary Material.

To calculate precision and recall metrics, return periods were calculated separately for each of the 5,680 gauges on both modeled and observed time series (return periods were calculated for observed time series and for modeled time series separately) using the methodology described by the USGS Bulletin 17b \cite{subcommittee1986interagency}. We considered the model to correctly predict an event with a given return period if the modeled hydrograph and the observed hydrograph both crossed their respective return period threshold flow values within two days of each other. Precision, recall, and F1 scores were calculated in the standard way separately for each gauge.


\section{Conclusion and Discussion}\label{sec:discussion}

Although hydrological modeling is a relatively mature area of study, areas of the world that are most vulnerable to flood risks often lack reliable forecasts and reliable early warning systems. Using AI and open data sets, we are able to significantly improve the expected precision, recall, and lead time of short term (0-7 days) forecasts of extreme riverine events. We extended, on average, the reliability of currently-available global nowcasts (lead time 0) to a lead time of 4 or more days, and we were able to use an AI forecasting approach to improve the skill of forecasts in Africa to be similar to what are currently available in Europe.  

Apart from producing accurate forecasts, another aspect of the challenge of providing actionable flood warnings is dissemination of those warnings to individuals and organizations in a timely manner. We support the latter by releasing forecasts publicly in real-time, without cost or barriers to access. We provide open access real-time forecasts to support notifications -- e.g., through CAP (Common Alerting Protocol) and push alerts to personal smartphones, -- and through an open online portal \url{https://g.co/floodhub}. All of the reanalysis and re-forecasts used for this study are included in an open source repository, and the machine learning model used for this study is available as part of the open source NeuralHydrology repository on Github \cite{kratzert2022neuralhydrology}.

There is still a lot of room to improve global flood predictions and early warning systems. Doing so is critical for the wellbeing of millions of people worldwide who's lives (and property) could benefit from timely, actionable flood warnings. We believe that the best way to improve flood forecasts from both data-driven and conceptual modeling approaches is to increase access to data. Hydrologic data is required for training or calibrating accurate hydrology models, and for updating those models in real time \cite[e.g., through data assimilation][]{nearing2022data}. We encourage researchers and organizations with access to streamflow data to contribute to the open source Caravan project \url{https://github.com/kratzert/Caravan} \cite{kratzert2023caravan}.

\backmatter

\bmhead{Supplementary Information}
Please see the accompanying PDF

% \bmhead{Acknowledgments}

\section*{Ethics Declarations}
The authors declare no competing interests.

\section*{Code Availability}
The forecasting model developed for this project was integrated into the NeuralHydrology code base \cite{kratzert2022neuralhydrology} that is available at \url{https://neuralhydrology.github.io}. This research code base differs from the operational model that was used in this article primarily in that it can be run on standard compute systems in Linux, iOS, and Windows environments. Training the models reported in this paper with the NeuralHydrology codebase is not plug-and-play.

Code for reproducing the figures and analyses reported in this paper is available at \url{https://github.com/google-research-datasets/global_streamflow_model_paper}. This repository calculates metrics for the AI model and GloFAS outputs, as reported in this paper, and requires the Zenodo data set linked in the Data Availability section below.

\section*{Data Availability}
Reanalysis (1984--2021) and reforecast (2012--2021) data produced by the AI model for this study, as well as corresponding GloFAS benchmark data are available at \url{https://doi.org/10.5281/zenodo.8139380}. 

Daily river discharge simulations from GloFAS version 3 (operational from 26 May 2021 until 26 July 2023: \url{https://confluence.ecmwf.int/display/CEMS/GloFAS+versioning+system}) and openly available through the Climate Data Store \cite{zsoter2021cds}.


% \bibliography{sn-bibliography}
% \end{document}

\begin{thebibliography}{40}
% BibTex style file: bmc-mathphys.bst (version 2.1), 2014-07-24
\ifx \bisbn   \undefined \def \bisbn  #1{ISBN #1}\fi
\ifx \binits  \undefined \def \binits#1{#1}\fi
\ifx \bauthor  \undefined \def \bauthor#1{#1}\fi
\ifx \batitle  \undefined \def \batitle#1{#1}\fi
\ifx \bjtitle  \undefined \def \bjtitle#1{#1}\fi
\ifx \bvolume  \undefined \def \bvolume#1{\textbf{#1}}\fi
\ifx \byear  \undefined \def \byear#1{#1}\fi
\ifx \bissue  \undefined \def \bissue#1{#1}\fi
\ifx \bfpage  \undefined \def \bfpage#1{#1}\fi
\ifx \blpage  \undefined \def \blpage #1{#1}\fi
\ifx \burl  \undefined \def \burl#1{\textsf{#1}}\fi
\ifx \doiurl  \undefined \def \doiurl#1{\url{https://doi.org/#1}}\fi
\ifx \betal  \undefined \def \betal{\textit{et al.}}\fi
\ifx \binstitute  \undefined \def \binstitute#1{#1}\fi
\ifx \binstitutionaled  \undefined \def \binstitutionaled#1{#1}\fi
\ifx \bctitle  \undefined \def \bctitle#1{#1}\fi
\ifx \beditor  \undefined \def \beditor#1{#1}\fi
\ifx \bpublisher  \undefined \def \bpublisher#1{#1}\fi
\ifx \bbtitle  \undefined \def \bbtitle#1{#1}\fi
\ifx \bedition  \undefined \def \bedition#1{#1}\fi
\ifx \bseriesno  \undefined \def \bseriesno#1{#1}\fi
\ifx \blocation  \undefined \def \blocation#1{#1}\fi
\ifx \bsertitle  \undefined \def \bsertitle#1{#1}\fi
\ifx \bsnm \undefined \def \bsnm#1{#1}\fi
\ifx \bsuffix \undefined \def \bsuffix#1{#1}\fi
\ifx \bparticle \undefined \def \bparticle#1{#1}\fi
\ifx \barticle \undefined \def \barticle#1{#1}\fi
\bibcommenthead
\ifx \bconfdate \undefined \def \bconfdate #1{#1}\fi
\ifx \botherref \undefined \def \botherref #1{#1}\fi
\ifx \url \undefined \def \url#1{\textsf{#1}}\fi
\ifx \bchapter \undefined \def \bchapter#1{#1}\fi
\ifx \bbook \undefined \def \bbook#1{#1}\fi
\ifx \bcomment \undefined \def \bcomment#1{#1}\fi
\ifx \oauthor \undefined \def \oauthor#1{#1}\fi
\ifx \citeauthoryear \undefined \def \citeauthoryear#1{#1}\fi
\ifx \endbibitem  \undefined \def \endbibitem {}\fi
\ifx \bconflocation  \undefined \def \bconflocation#1{#1}\fi
\ifx \arxivurl  \undefined \def \arxivurl#1{\textsf{#1}}\fi
\csname PreBibitemsHook\endcsname

%%% 1
\bibitem{unisdr2015human}
\begin{botherref}
\oauthor{\bsnm{{United Nations International Strategy for Disaster Reduction
  (UNISDR)}}}:
The human cost of natural disasters: A global perspective
(2015)
\end{botherref}
\endbibitem

%%% 2
\bibitem{wmo20212021}
\begin{botherref}
\oauthor{\bsnm{{World Meteorological Organization}}}:
2021 State of Climate Services (WMO-No. 1278).
World Meteorological Organization Geneva, Switzerland
(2021)
\end{botherref}
\endbibitem

%%% 3
\bibitem{milly2002increasing}
\begin{barticle}
\bauthor{\bsnm{Milly}, \binits{P.C.D.}},
\bauthor{\bsnm{Wetherald}, \binits{R.T.}},
\bauthor{\bsnm{Dunne}, \binits{K.}},
\bauthor{\bsnm{Delworth}, \binits{T.L.}}:
\batitle{Increasing risk of great floods in a changing climate}.
\bjtitle{Nature}
\bvolume{415}(\bissue{6871}),
\bfpage{514}--\blpage{517}
(\byear{2002})
\end{barticle}
\endbibitem

%%% 4
\bibitem{tabari2020climate}
\begin{barticle}
\bauthor{\bsnm{Tabari}, \binits{H.}}:
\batitle{Climate change impact on flood and extreme precipitation increases
  with water availability}.
\bjtitle{Scientific reports}
\bvolume{10}(\bissue{1}),
\bfpage{1}--\blpage{10}
(\byear{2020})
\end{barticle}
\endbibitem

%%% 5
\bibitem{world2014global}
\begin{botherref}
\oauthor{\bsnm{{World Health Organization and others}}}:
Global report on drowning: preventing a leading killer
(2014)
\end{botherref}
\endbibitem

%%% 6
\bibitem{wmo2013global}
\begin{botherref}
\oauthor{\bsnm{{World Meteorological Organization}}}:
The global climate 2001-2010: a decade of climate extremes.
Technical report,
World Meteorological Organization
(2013)
\end{botherref}
\endbibitem

%%% 7
\bibitem{pilon2002guidelines}
\begin{botherref}
\oauthor{\bsnm{Pilon}, \binits{P.J.}}:
Guidelines for reducing flood losses.
Technical report,
United Nations International Strategy for Disaster Reduction (UNISDR)
(2002)
\end{botherref}
\endbibitem

%%% 8
\bibitem{rogers2011costs}
\begin{botherref}
\oauthor{\bsnm{Rogers}, \binits{D.}},
\oauthor{\bsnm{Tsirkunov}, \binits{V.}}:
Costs and benefits of early warning systems: Global assessment report on
  disaster risk reduction
(2011)
\end{botherref}
\endbibitem

%%% 9
\bibitem{rentschler2022flood}
\begin{barticle}
\bauthor{\bsnm{Rentschler}, \binits{J.}},
\bauthor{\bsnm{Salhab}, \binits{M.}},
\bauthor{\bsnm{Jafino}, \binits{B.A.}}:
\batitle{Flood exposure and poverty in 188 countries}.
\bjtitle{Nature communications}
\bvolume{13}(\bissue{1}),
\bfpage{3527}
(\byear{2022})
\end{barticle}
\endbibitem

%%% 10
\bibitem{hallegatte2012cost}
\begin{botherref}
\oauthor{\bsnm{Hallegatte}, \binits{S.}}:
A cost effective solution to reduce disaster losses in developing countries:
  hydro-meteorological services, early warning, and evacuation.
World Bank policy research working paper
(6058)
(2012)
\end{botherref}
\endbibitem

%%% 11
\bibitem{razavi2013anefficient}
\begin{barticle}
\bauthor{\bsnm{Razavi}, \binits{S.}},
\bauthor{\bsnm{Tolson}, \binits{B.A.}}:
\batitle{An efficient framework for hydrologic model calibration on long data
  periods}.
\bjtitle{Water Resources Research}
\bvolume{49}(\bissue{12}),
\bfpage{8418}--\blpage{8431}
(\byear{2013})
\end{barticle}
\endbibitem

%%% 12
\bibitem{LI2010378}
\begin{barticle}
\bauthor{\bsnm{Li}, \binits{C.-z.}},
\bauthor{\bsnm{Wang}, \binits{H.}},
\bauthor{\bsnm{Liu}, \binits{J.}},
\bauthor{\bsnm{Yan}, \binits{D.-h.}},
\bauthor{\bsnm{Yu}, \binits{F.-l.}},
\bauthor{\bsnm{Zhang}, \binits{L.}}:
\batitle{Effect of calibration data series length on performance and optimal
  parameters of hydrological model}.
\bjtitle{Water Science and Engineering}
\bvolume{3}(\bissue{4}),
\bfpage{378}--\blpage{393}
(\byear{2010})
\end{barticle}
\endbibitem

%%% 13
\bibitem{sivapalan2003iahs}
\begin{barticle}
\bauthor{\bsnm{Sivapalan}, \binits{M.}},
\bauthor{\bsnm{Takeuchi}, \binits{K.}},
\bauthor{\bsnm{Franks}, \binits{S.}},
\bauthor{\bsnm{Gupta}, \binits{V.}},
\bauthor{\bsnm{Karambiri}, \binits{H.}},
\bauthor{\bsnm{Lakshmi}, \binits{V.}},
\bauthor{\bsnm{Liang}, \binits{X.}},
\bauthor{\bsnm{McDonnell}, \binits{J.}},
\bauthor{\bsnm{Mendiondo}, \binits{E.}},
\bauthor{\bsnm{O'connell}, \binits{P.}}, \betal:
\batitle{Iahs decade on predictions in ungauged basins (pub), 2003--2012:
  Shaping an exciting future for the hydrological sciences}.
\bjtitle{Hydrological sciences journal}
\bvolume{48}(\bissue{6}),
\bfpage{857}--\blpage{880}
(\byear{2003})
\end{barticle}
\endbibitem

%%% 14
\bibitem{hrachowitz2013decade}
\begin{barticle}
\bauthor{\bsnm{Hrachowitz}, \binits{M.}},
\bauthor{\bsnm{Savenije}, \binits{H.}},
\bauthor{\bsnm{Bl{\"o}schl}, \binits{G.}},
\bauthor{\bsnm{McDonnell}, \binits{J.}},
\bauthor{\bsnm{Sivapalan}, \binits{M.}},
\bauthor{\bsnm{Pomeroy}, \binits{J.}},
\bauthor{\bsnm{Arheimer}, \binits{B.}},
\bauthor{\bsnm{Blume}, \binits{T.}},
\bauthor{\bsnm{Clark}, \binits{M.}},
\bauthor{\bsnm{Ehret}, \binits{U.}}, \betal:
\batitle{A decade of predictions in ungauged basins (pub)—a review}.
\bjtitle{Hydrological sciences journal}
\bvolume{58}(\bissue{6}),
\bfpage{1198}--\blpage{1255}
(\byear{2013})
\end{barticle}
\endbibitem

%%% 15
\bibitem{twb2023gdp}
\begin{botherref}
\oauthor{\bsnm{IBRD-IDA}}:
The World Bank Data: Current US\$
(2023).
\url{https://data.worldbank.org/indicator/NY.GDP.MKTP.CD}
Accessed 2023-06-04
\end{botherref}
\endbibitem

%%% 16
\bibitem{alfieri2013glofas}
\begin{barticle}
\bauthor{\bsnm{Alfieri}, \binits{L.}},
\bauthor{\bsnm{Burek}, \binits{P.}},
\bauthor{\bsnm{Dutra}, \binits{E.}},
\bauthor{\bsnm{Krzeminski}, \binits{B.}},
\bauthor{\bsnm{Muraro}, \binits{D.}},
\bauthor{\bsnm{Thielen}, \binits{J.}},
\bauthor{\bsnm{Pappenberger}, \binits{F.}}:
\batitle{Glofas--global ensemble streamflow forecasting and flood early
  warning}.
\bjtitle{Hydrology and Earth System Sciences}
\bvolume{17}(\bissue{3}),
\bfpage{1161}--\blpage{1175}
(\byear{2013})
\end{barticle}
\endbibitem

%%% 17
\bibitem{harrigan2023daily}
\begin{barticle}
\bauthor{\bsnm{Harrigan}, \binits{S.}},
\bauthor{\bsnm{Zsoter}, \binits{E.}},
\bauthor{\bsnm{Cloke}, \binits{H.}},
\bauthor{\bsnm{Salamon}, \binits{P.}},
\bauthor{\bsnm{Prudhomme}, \binits{C.}}:
\batitle{Daily ensemble river discharge reforecasts and real-time forecasts
  from the operational global flood awareness system}.
\bjtitle{Hydrology and Earth System Sciences}
\bvolume{27}(\bissue{1}),
\bfpage{1}--\blpage{19}
(\byear{2023})
\end{barticle}
\endbibitem

%%% 18
\bibitem{arheimer2020global}
\begin{barticle}
\bauthor{\bsnm{Arheimer}, \binits{B.}},
\bauthor{\bsnm{Pimentel}, \binits{R.}},
\bauthor{\bsnm{Isberg}, \binits{K.}},
\bauthor{\bsnm{Crochemore}, \binits{L.}},
\bauthor{\bsnm{Andersson}, \binits{J.C.}},
\bauthor{\bsnm{Hasan}, \binits{A.}},
\bauthor{\bsnm{Pineda}, \binits{L.}}:
\batitle{Global catchment modelling using world-wide hype (wwh), open data, and
  stepwise parameter estimation}.
\bjtitle{Hydrology and Earth System Sciences}
\bvolume{24}(\bissue{2}),
\bfpage{535}--\blpage{559}
(\byear{2020})
\end{barticle}
\endbibitem

%%% 19
\bibitem{souffront2019hydrologic}
\begin{barticle}
\bauthor{\bsnm{Souffront~Alcantara}, \binits{M.A.}},
\bauthor{\bsnm{Nelson}, \binits{E.J.}},
\bauthor{\bsnm{Shakya}, \binits{K.}},
\bauthor{\bsnm{Edwards}, \binits{C.}},
\bauthor{\bsnm{Roberts}, \binits{W.}},
\bauthor{\bsnm{Krewson}, \binits{C.}},
\bauthor{\bsnm{Ames}, \binits{D.P.}},
\bauthor{\bsnm{Jones}, \binits{N.L.}},
\bauthor{\bsnm{Gutierrez}, \binits{A.}}:
\batitle{Hydrologic modeling as a service (hmaas): a new approach to address
  hydroinformatic challenges in developing countries}.
\bjtitle{Frontiers in Environmental Science}
\bvolume{7},
\bfpage{158}
(\byear{2019})
\end{barticle}
\endbibitem

%%% 20
\bibitem{sheffield2014drought}
\begin{barticle}
\bauthor{\bsnm{Sheffield}, \binits{J.}},
\bauthor{\bsnm{Wood}, \binits{E.F.}},
\bauthor{\bsnm{Chaney}, \binits{N.}},
\bauthor{\bsnm{Guan}, \binits{K.}},
\bauthor{\bsnm{Sadri}, \binits{S.}},
\bauthor{\bsnm{Yuan}, \binits{X.}},
\bauthor{\bsnm{Olang}, \binits{L.}},
\bauthor{\bsnm{Amani}, \binits{A.}},
\bauthor{\bsnm{Ali}, \binits{A.}},
\bauthor{\bsnm{Demuth}, \binits{S.}}, \betal:
\batitle{A drought monitoring and forecasting system for sub-sahara african
  water resources and food security}.
\bjtitle{Bulletin of the American Meteorological Society}
\bvolume{95}(\bissue{6}),
\bfpage{861}--\blpage{882}
(\byear{2014})
\end{barticle}
\endbibitem

%%% 21
\bibitem{kratzert2019toward}
\begin{barticle}
\bauthor{\bsnm{Kratzert}, \binits{F.}},
\bauthor{\bsnm{Klotz}, \binits{D.}},
\bauthor{\bsnm{Herrnegger}, \binits{M.}},
\bauthor{\bsnm{Sampson}, \binits{A.K.}},
\bauthor{\bsnm{Hochreiter}, \binits{S.}},
\bauthor{\bsnm{Nearing}, \binits{G.S.}}:
\batitle{Toward improved predictions in ungauged basins: Exploiting the power
  of machine learning}.
\bjtitle{Water Resources Research}
\bvolume{55}(\bissue{12}),
\bfpage{11344}--\blpage{11354}
(\byear{2019})
\end{barticle}
\endbibitem

%%% 22
\bibitem{hochreiter1997long}
\begin{barticle}
\bauthor{\bsnm{Hochreiter}, \binits{S.}},
\bauthor{\bsnm{Schmidhuber}, \binits{J.}}:
\batitle{Long short-term memory}.
\bjtitle{Neural computation}
\bvolume{9}(\bissue{8}),
\bfpage{1735}--\blpage{1780}
(\byear{1997})
\end{barticle}
\endbibitem

%%% 23
\bibitem{kratzert2022neuralhydrology}
\begin{barticle}
\bauthor{\bsnm{Kratzert}, \binits{F.}},
\bauthor{\bsnm{Gauch}, \binits{M.}},
\bauthor{\bsnm{Nearing}, \binits{G.}},
\bauthor{\bsnm{Klotz}, \binits{D.}}:
\batitle{Neuralhydrology---a python library for deep learning research in
  hydrology}.
\bjtitle{Journal of Open Source Software}
\bvolume{7}(\bissue{71}),
\bfpage{4050}
(\byear{2022})
\end{barticle}
\endbibitem

%%% 24
\bibitem{https://doi.org/10.1029/2022WR033918}
\begin{barticle}
\bauthor{\bsnm{Gauch}, \binits{M.}},
\bauthor{\bsnm{Kratzert}, \binits{F.}},
\bauthor{\bsnm{Gilon}, \binits{O.}},
\bauthor{\bsnm{Gupta}, \binits{H.}},
\bauthor{\bsnm{Mai}, \binits{J.}},
\bauthor{\bsnm{Nearing}, \binits{G.}},
\bauthor{\bsnm{Tolson}, \binits{B.}},
\bauthor{\bsnm{Hochreiter}, \binits{S.}},
\bauthor{\bsnm{Klotz}, \binits{D.}}:
\batitle{In defense of metrics: Metrics sufficiently encode typical human
  preferences regarding hydrological model performance}.
\bjtitle{Water Resources Research}
\bvolume{59}(\bissue{6}),
\bfpage{2022}--\blpage{033918}
(\byear{2023})
\end{barticle}
\endbibitem

%%% 25
\bibitem{wwrp2016forecast}
\begin{botherref}
\oauthor{\bsnm{{World Weather Research Programme}}}:
Forecast Verification methods Across Time and Space Scales
(2016).
\url{https://data.worldbank.org/indicator/NY.GDP.MKTP.CD}
Accessed 2023-06-04
\end{botherref}
\endbibitem

%%% 26
\bibitem{sellars2018grand}
\begin{barticle}
\bauthor{\bsnm{Sellars}, \binits{S.}}:
\batitle{“grand challenges” in big data and the earth sciences}.
\bjtitle{Bulletin of the American Meteorological Society}
\bvolume{99}(\bissue{6}),
\bfpage{95}--\blpage{98}
(\byear{2018})
\end{barticle}
\endbibitem

%%% 27
\bibitem{todini2007hydrological}
\begin{barticle}
\bauthor{\bsnm{Todini}, \binits{E.}}:
\batitle{Hydrological catchment modelling: past, present and future}.
\bjtitle{Hydrology and Earth System Sciences}
\bvolume{11}(\bissue{1}),
\bfpage{468}--\blpage{482}
(\byear{2007})
\end{barticle}
\endbibitem

%%% 28
\bibitem{herath2021hydrologically}
\begin{barticle}
\bauthor{\bsnm{Herath}, \binits{H.M.V.V.}},
\bauthor{\bsnm{Chadalawada}, \binits{J.}},
\bauthor{\bsnm{Babovic}, \binits{V.}}:
\batitle{Hydrologically informed machine learning for rainfall--runoff
  modelling: towards distributed modelling}.
\bjtitle{Hydrology and Earth System Sciences}
\bvolume{25}(\bissue{8}),
\bfpage{4373}--\blpage{4401}
(\byear{2021})
\end{barticle}
\endbibitem

%%% 29
\bibitem{reichstein2019deep}
\begin{barticle}
\bauthor{\bsnm{Reichstein}, \binits{M.}},
\bauthor{\bsnm{Camps-Valls}, \binits{G.}},
\bauthor{\bsnm{Stevens}, \binits{B.}},
\bauthor{\bsnm{Jung}, \binits{M.}},
\bauthor{\bsnm{Denzler}, \binits{J.}},
\bauthor{\bsnm{Carvalhais}, \binits{N.}},
\bauthor{\bsnm{Prabhat}, \binits{f.}}:
\batitle{Deep learning and process understanding for data-driven earth system
  science}.
\bjtitle{Nature}
\bvolume{566}(\bissue{7743}),
\bfpage{195}--\blpage{204}
(\byear{2019})
\end{barticle}
\endbibitem

%%% 30
\bibitem{hess-26-3377-2022}
\begin{barticle}
\bauthor{\bsnm{Frame}, \binits{J.M.}},
\bauthor{\bsnm{Kratzert}, \binits{F.}},
\bauthor{\bsnm{Klotz}, \binits{D.}},
\bauthor{\bsnm{Gauch}, \binits{M.}},
\bauthor{\bsnm{Shalev}, \binits{G.}},
\bauthor{\bsnm{Gilon}, \binits{O.}},
\bauthor{\bsnm{Qualls}, \binits{L.M.}},
\bauthor{\bsnm{Gupta}, \binits{H.V.}},
\bauthor{\bsnm{Nearing}, \binits{G.S.}}:
\batitle{Deep learning rainfall--runoff predictions of extreme events}.
\bjtitle{Hydrology and Earth System Sciences}
\bvolume{26}(\bissue{13}),
\bfpage{3377}--\blpage{3392}
(\byear{2022})
\end{barticle}
\endbibitem

%%% 31
\bibitem{linke2019global}
\begin{barticle}
\bauthor{\bsnm{Linke}, \binits{S.}},
\bauthor{\bsnm{Lehner}, \binits{B.}},
\bauthor{\bsnm{Ouellet~Dallaire}, \binits{C.}},
\bauthor{\bsnm{Ariwi}, \binits{J.}},
\bauthor{\bsnm{Grill}, \binits{G.}},
\bauthor{\bsnm{Anand}, \binits{M.}},
\bauthor{\bsnm{Beames}, \binits{P.}},
\bauthor{\bsnm{Burchard-Levine}, \binits{V.}},
\bauthor{\bsnm{Maxwell}, \binits{S.}},
\bauthor{\bsnm{Moidu}, \binits{H.}}, \betal:
\batitle{Global hydro-environmental sub-basin and river reach characteristics
  at high spatial resolution}.
\bjtitle{Scientific data}
\bvolume{6}(\bissue{1}),
\bfpage{283}
(\byear{2019})
\end{barticle}
\endbibitem

%%% 32
\bibitem{kratzert2021large}
\begin{bchapter}
\bauthor{\bsnm{Kratzert}, \binits{F.}},
\bauthor{\bsnm{Klotz}, \binits{D.}},
\bauthor{\bsnm{Gauch}, \binits{M.}},
\bauthor{\bsnm{Klingler}, \binits{C.}},
\bauthor{\bsnm{Nearing}, \binits{G.}},
\bauthor{\bsnm{Hochreiter}, \binits{S.}}:
\bctitle{Large-scale river network modeling using graph neural networks}.
In: \bbtitle{EGU General Assembly Conference Abstracts},
pp. \bfpage{21}--\blpage{13375}
(\byear{2021})
\end{bchapter}
\endbibitem

%%% 33
\bibitem{lehner2013global}
\begin{barticle}
\bauthor{\bsnm{Lehner}, \binits{B.}},
\bauthor{\bsnm{Grill}, \binits{G.}}:
\batitle{Global river hydrography and network routing: baseline data and new
  approaches to study the world's large river systems}.
\bjtitle{Hydrological Processes}
\bvolume{27}(\bissue{15}),
\bfpage{2171}--\blpage{2186}
(\byear{2013})
\end{barticle}
\endbibitem

%%% 34
\bibitem{kratzert2019towards}
\begin{barticle}
\bauthor{\bsnm{Kratzert}, \binits{F.}},
\bauthor{\bsnm{Klotz}, \binits{D.}},
\bauthor{\bsnm{Shalev}, \binits{G.}},
\bauthor{\bsnm{Klambauer}, \binits{G.}},
\bauthor{\bsnm{Hochreiter}, \binits{S.}},
\bauthor{\bsnm{Nearing}, \binits{G.}}:
\batitle{Towards learning universal, regional, and local hydrological behaviors
  via machine learning applied to large-sample datasets}.
\bjtitle{Hydrology and Earth System Sciences}
\bvolume{23}(\bissue{12}),
\bfpage{5089}--\blpage{5110}
(\byear{2019})
\end{barticle}
\endbibitem

%%% 35
\bibitem{klotz2022uncertainty}
\begin{barticle}
\bauthor{\bsnm{Klotz}, \binits{D.}},
\bauthor{\bsnm{Kratzert}, \binits{F.}},
\bauthor{\bsnm{Gauch}, \binits{M.}},
\bauthor{\bsnm{Keefe~Sampson}, \binits{A.}},
\bauthor{\bsnm{Brandstetter}, \binits{J.}},
\bauthor{\bsnm{Klambauer}, \binits{G.}},
\bauthor{\bsnm{Hochreiter}, \binits{S.}},
\bauthor{\bsnm{Nearing}, \binits{G.}}:
\batitle{Uncertainty estimation with deep learning for rainfall--runoff
  modeling}.
\bjtitle{Hydrology and Earth System Sciences}
\bvolume{26}(\bissue{6}),
\bfpage{1673}--\blpage{1693}
(\byear{2022})
\end{barticle}
\endbibitem

%%% 36
\bibitem{nearing2022data}
\begin{barticle}
\bauthor{\bsnm{Nearing}, \binits{G.S.}},
\bauthor{\bsnm{Klotz}, \binits{D.}},
\bauthor{\bsnm{Frame}, \binits{J.M.}},
\bauthor{\bsnm{Gauch}, \binits{M.}},
\bauthor{\bsnm{Gilon}, \binits{O.}},
\bauthor{\bsnm{Kratzert}, \binits{F.}},
\bauthor{\bsnm{Sampson}, \binits{A.K.}},
\bauthor{\bsnm{Shalev}, \binits{G.}},
\bauthor{\bsnm{Nevo}, \binits{S.}}:
\batitle{Data assimilation and autoregression<? xmltex$\backslash$break?> for
  using near-real-time streamflow observations<? xmltex$\backslash$break?> in
  long short-term memory networks}.
\bjtitle{Hydrology and Earth System Sciences}
\bvolume{26}(\bissue{21}),
\bfpage{5493}--\blpage{5513}
(\byear{2022})
\end{barticle}
\endbibitem

%%% 37
\bibitem{grdc}
\begin{botherref}
\oauthor{\bsnm{{Global Runoff Data Center}}}:
Global Composite Runoff Fields (CSRC-UNH and GRDC, 2002)
(2022).
\url{https://www.bafg.de/GRDC/EN/03_dtprdcts/33_CmpR/unh_grdc_node.html}
Accessed 05-01-2023
\end{botherref}
\endbibitem

%%% 38
\bibitem{subcommittee1986interagency}
\begin{botherref}
\oauthor{\bsnm{{Hydrology Subcommittee}}}:
Interagency advisory committee on water data.
Office of Water Data Coordination, US Geologic Survey, Reston, VA
(1986)
\end{botherref}
\endbibitem

%%% 39
\bibitem{kratzert2023caravan}
\begin{barticle}
\bauthor{\bsnm{Kratzert}, \binits{F.}},
\bauthor{\bsnm{Nearing}, \binits{G.}},
\bauthor{\bsnm{Addor}, \binits{N.}},
\bauthor{\bsnm{Erickson}, \binits{T.}},
\bauthor{\bsnm{Gauch}, \binits{M.}},
\bauthor{\bsnm{Gilon}, \binits{O.}},
\bauthor{\bsnm{Gudmundsson}, \binits{L.}},
\bauthor{\bsnm{Hassidim}, \binits{A.}},
\bauthor{\bsnm{Klotz}, \binits{D.}},
\bauthor{\bsnm{Nevo}, \binits{S.}}, \betal:
\batitle{Caravan-a global community dataset for large-sample hydrology}.
\bjtitle{Scientific Data}
\bvolume{10}(\bissue{1}),
\bfpage{61}
(\byear{2023})
\end{barticle}
\endbibitem

%%% 40
\bibitem{zsoter2021cds}
\begin{botherref}
\oauthor{\bsnm{Zsoter}, \binits{E.}},
\oauthor{\bsnm{Harrigan}, \binits{S.}},
\oauthor{\bsnm{Barnard}, \binits{C.}},
\oauthor{\bsnm{Wetterhall}, \binits{F.}},
\oauthor{\bsnm{Ferrario}, \binits{I.}},
\oauthor{\bsnm{Mazzetti}, \binits{C.}},
\oauthor{\bsnm{Alfieri}, \binits{L.}},
\oauthor{\bsnm{Salamon}, \binits{P.}},
\oauthor{\bsnm{Prudhomme}, \binits{C.}}:
River discharge and related historical data from the Global Flood Awareness
  System. v3.1. Copernicus Climate Change Service (C3S) Climate Data Store
  (CDS)[data set]
(2021)
\end{botherref}
\endbibitem

\end{thebibliography}
\end{document}
