
\section{Introduction} \label{sec:introduction}
\Gls*{BC} orbits are low-energy transfers that allow temporary capture about a planet exploiting the natural dynamics, thus without requiring maneuvers \cite{topputo2015earth}. Compared to Keplerian solutions, they are cheaper, safer, and more versatile from the operational perspective at the expense of longer transfer times. \gls*{BC} orbits are bounded by the \gls*{WSB} \cite{belbruno1993sun,belbruno2000calculation,circi2001dynamics,topputo2015earth}. After being initially conceived as a fuzzy boundary region in the Sun--Earth--Moon system \citep{belbruno1987lunar,belbruno1990ballistic}, the \gls*{WSB} was algorithmically defined in \cite{belbruno2004capture}. The definition was later extended in \cite{garcia2007note,topputo2009computation,silva2012applicability}. A formal definition and a technique for its derivation were proposed in \cite{hyeraci2010method}.

Currently, two approaches are known for designing \gls*{BC} orbits: the technique stemmed from invariant manifolds \cite{topputo2005low,belbruno2010weak}, and the method based on stable sets manipulation \cite{hyeraci2010method,luo2014constructing}. The former gives insights into the dynamics but it is only applicable to autonomous systems (\eg the \acrlong*{CR3BP}), while the latter can be applied to more representative, non-autonomous models, although being computationally expensive \cite{topputo2009computation,luo2015analysis}. Lately, the variational theory for \glspl*{LCS} \cite{haller2011variational,haller2015lagrangian}, and the Taylor differential algebra \cite{wittig2015propagation} were applied to derive \gls*{BC} orbits and the \gls*{WSB} more efficiently \cite{manzi2021flow,caleb0000stable}. Alternatively, \glspl*{LD} can be exploited. They reveal separatrices, so providing a qualitative description of the dynamics and highlighting the geometrical template of phase space structures even for systems with generic time dependence \cite{jimenez2009distinguished,mancho2013lagrangian,lopesino2017theoretical}.

The goal of the paper is to study to what extent \glspl*{LD} inform about the \gls*{BC} mechanism and aid in the design of \gls*{BC} orbits. We provide a characterization of the dynamics in the Mars proximity modeled under the planar \gls*{ER3BP}. The geometrical structures featured by \gls*{LD} scalar fields are extracted through an edge detection algorithm based on the Roberts' method \cite{davis1975survey,roberts1963machine}. The separatrices are inspected against the \gls*{WSB} derived on similar integration intervals. For a coherent comparison, the particle stability definition is modified to relax the geometrical constraint on the number of completed revolutions \cite{hyeraci2010method,luo2014constructing}. Results show a strong correlation between extracted separatrices and the \gls*{WSB}, particularly when the geometrical structures governing the transport mechanisms emerge. Eventually, capture sets at Mars are identified in the intricate plot of separatrices.

The remainder of the paper is organized as follows. In \Sec{sec:eqofmotion}, the dynamical model is described. The methodology is discussed in \Sec{sec:methodology}. Results are shown in \Sec{sec:results}. Eventually, conclusions are drawn in \Sec{sec:conclusion}.
