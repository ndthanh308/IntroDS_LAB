
\section{Equations of motion} \label{sec:eqofmotion}
The planar \gls*{ER3BP} describes the motion of a massless particle moving under the gravitational attraction of two primary bodies $P_{1}$ (the Sun) and $P_{2}$ (Mars) without influencing their motion. The two primaries revolve on ellipses about their common barycenter, influenced only by their mutual attraction. The model is expressed in the synodic reference frame centered at the primaries barycenter. The synodic frame non-uniformly rotates and pulsates to keep their distance equal to one \cite{hyeraci2010method}. Let the mass parameter $\mu = m_{2}/(m_{1}+m_{2})$, where $m_{1}$ and $m_{2}$ are the masses of $P_{1}$ and $P_{2}$, respectively. The positions of $P_{1}$ and $P_{2}$ are (-$\mu$, 0) and (1-$\mu$, 0), respectively. The \gls*{EOM} are scaled such that the sum of $P_{1}$ and $P_{2}$ masses is set to one as well as their distance, and their period is scaled to $2\pi$ \cite{hyeraci2010method}. The true anomaly $f$ is designated as the independent variable of the system. The \gls*{EOM} read \cite{hyeraci2010method}
\begin{linenomath*}
\begin{align}
    \begin{split}
        {x}''-2{y}' &= \omega_{x} \\
        {y}''+2{x}' &= \omega_{y} 
    \end{split}
    \label{eq:er3bp}
\end{align}
\end{linenomath*}
where primes represent differentiation with respect to the true anomaly $ f $ that depends on the scaled time as \cite{hyeraci2010method}
\begin{linenomath*}
\begin{equation}
	\frac{\di f}{\di t} = \frac{(1+e_{p}\cos{f})^{2}}{(1-e_{p}^{2})^{3/2}}.
\end{equation}
\end{linenomath*}
In \Eq{eq:er3bp}, subscripts $(\cdot)_{x}$ and $(\cdot)_{y}$ denote the partial derivatives of the potential function $\omega$ defined as \cite{hyeraci2010method}
\begin{linenomath*}
\begin{equation}
    \omega \left(x,y,f\right)=\frac{1}{1+e_{p}\cos{f}} \left[ \frac{1}{2} \left(x^2+y^2\right) +\frac{1-\mu}{r_1} +\frac{\mu}{r_2} +\frac{1}{2}\mu(1-\mu) \right],
\end{equation}
\end{linenomath*}
with $r_{1} = \sqrt{(x+\mu)^2+y^2}$ and $ r_{2} = \sqrt{(x+\mu-1)^2+y^2}$ the distances of the particle from $P_1$ and $P_2$, respectively, while $e_{p}$ is the common eccentricity of the primaries. The Sun--Mars physical parameters used in this study are reported in \Tab{tab:parameters}. The \gls*{EOM} are integrated with a $8^{\rm th}$-order Runge--Kutta scheme with a $7^{\rm th}$-order embedded step-size control. The integration relative tolerance is set to $10^{-9}$ \cite{montenbruck2000satellite,verner2010numerically}.

\begin{table}[pos=tbp,width=0.70\textwidth,align=\centering]
    \caption{Sun--Mars physical parameters.} 
    \begin{tabular*}{0.70\textwidth}{@{}LLLLC@{}}
    \toprule
    \textbf{Parameter} & \textbf{Unit} & \textbf{Value} & \textbf{Description} & \textbf{Reference} \\
    \midrule
    $ \mu $ & - & \SI{3.226201e-7}{} & Mass parameter & \multirow{4}{*}{\cite{hyeraci2010method}} \\
    $ a_{p} $ & \si{\AU} & \SI{1.523688}{} & Primaries semi-major axis & \\ 
    $ e_{p} $ & - & \SI{0.093418}{} & Primaries eccentricity & \\
    $ R $ & \si{\kilo\meter} & \si{3397} & Mars mean equatorial radius & \\
    \cline{5-5}
    $ R_{\mathrm{SOI}} $ \rule{0pt}{10pt} & \si{\kilo\meter} & $ \SI{170}{} R $ & Mars \gls*{SOI} radius & \multirow{1}{*}{\cite{luo2014constructing}} \\
    \bottomrule
    \end{tabular*}
    \label{tab:parameters}
\end{table}
