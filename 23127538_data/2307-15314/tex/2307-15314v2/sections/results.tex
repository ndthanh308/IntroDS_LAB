
\section{Results} \label{sec:results}

% Figure environment removed

% Figure environment removed

Without loss of generality, the \gls*{LD}-based approach is applied to the Sun--Mars system. The correlation between the extracted separatrices and the \gls*{WSB} is tested for several integration intervals, for both forward and backward propagations. In \Fig{fig:LD}, the \gls*{LD} scalar field computed for two distinct final anomalies is shown (see \Figsand{fig:LD1}{fig:LD3}), together with the extracted patterns overlapped to the subsets $ \mathcal{W} $, $ \mathcal{X} $, and $ \mathcal{K} $ derived for the same $f_{f}$ (see \Figsand{fig:LD2}{fig:LD4}). In both cases a good match between separatrices and boundaries of the classified regions is observed. The central green disk identifies \glspl*{IC} located inside the surface of Mars, which immediately generate crash orbits. \tr{The scalar field in \Fig{fig:LD3} is more accurate in revealing the geometrical structures that characterize the \gls*{ER3BP} dynamics, when compared to that in \Fig{fig:LD1}. The longer the finite horizon over which \glspl*{IC} are propagated, the more separatrices the field is able to reveal \cite{jimenez2009distinguished,mancho2013lagrangian,lopesino2017theoretical}. Since initial states in $\mathcal{M}(0,-\pi,0)$ are integrated over a shorter finite horizon, the field cannot reveal structures with the same level of detail as compared with $\mathcal{M}(0,0,2\pi)$, in which initial states are propagated over a longer finite horizon.} \tr{Results for the case $\mathcal{M}(0,-2\pi,0)$, here not included, reach the same accuracy of the ones for the $\mathcal{M}(0,0,2\pi)$ field. Indeed, the two \gls*{LD} fields are symmetric with respect to the $x$-axis \cite{gawlik2009lagrangian}.}

For small values of $ f_{f} $, the matching presents some inconsistencies that are intrinsic to the \gls*{LD} definition. In fact, \gls*{LD} reveals patterns if \glspl*{IC} are integrated long enough for dynamical divergences between orbits to be manifested \cite{mancho2013lagrangian}. Consequently, the classification of the phase space according to the definition of particle stability provided in \Sec{sec:stability-def} may be inconsistent with some regions featured by the \gls*{LD} scalar field if the trajectories are not sufficiently divergent to feature singular structures in the field \cite{mancho2013lagrangian}. The latter is particularly true for short integration intervals as observed in \Fig{fig:LD2}. For instance, \glspl*{IC} `c' and `d' in \Fig{fig:LD2} are classified into two different sets, still their dynamical behavior is very similar as shown by their orbits in \Figsand{fig:TrajC}{fig:TrajD}. Indeed, for a slightly larger integration interval both orbits escape from Mars.

Remarkably, \glspl*{LD} detect divergence (forward propagation) and attraction (backward propagation) in the dynamical behavior even in areas classified in the same way according to our particle stability definition. To illustrate this concept, two grid points can evolve both in crash orbits, nonetheless their trajectories could be strongly different, as well as their impact epochs. For example, samples `e' and `h' in \Fig{fig:LD4} belong to two distinct regions of the same crash set $ \mathcal{K}(2\pi) $, therefore they both impact with Mars. However, they exhibit dissimilar trajectories (see \Figsand{fig:TrajE}{fig:TrajH}). They impact from different directions, and orbit `h' reverses its angular momentum with respect to Mars much earlier than orbit `e'. 

Patterns ruling particles transport in both true anomaly directions are revealed combining the \gls*{LD} structures propagated forwards and backwards \cite{mancho2013lagrangian}. The correlation of the two capture sets $ \mathcal{C}(-\pi,3\pi/2) $ and $\mathcal{C}(-\pi,3\pi)$ with the separatrices extracted from $\mathcal{M}(0,-\pi,3\pi/2)$ and $\mathcal{M}(0,-\pi,3\pi)$ fields, respectively, is presented in \Fig{fig:CaptureSets}. Results show that some of the areas in the phase space enclosed by \gls*{LD} separatrices appear to be capture sets. Based on the outcome of the validation procedure, the devised methodology of computing the \gls*{LD} field and extracting the dynamics separatrices has been proven successful. 

Referring to \Fig{fig:CaptureSets}, the \gls*{LD} approach omits the dynamical behavior featured by the highlighted numerous regions, therefore a classification technique is still required to discern which areas are actually capture sets. A viable strategy to overcome the aforementioned limitation is proposed for practical design of \gls*{BC} orbits. By sampling an individual \gls*{IC} for each identified region and classifying its orbit, all areas in the phase space can be easily categorized either as $ \mathcal{W} $, $ \mathcal{X} $, or $ \mathcal{K} $ subsets according to the particle stability definition given in \Sec{sec:stability-def}.

The exact \glspl*{IC} sampled from \Figsand{fig:LD}{fig:CaptureSets} are collected in \Tab{tab:ics}. Their orbits expressed in the Mars-centered, non-rotating frame, oriented as the synodic frame at $f_{0}$, and with Cartesian coordinates $X$ and $Y$ are plotted in \Fig{fig:Trajectories}. Compared to similar \gls*{BC} orbits found in the literature \cite{hyeraci2010method,luo2014constructing}, the weakly stable trajectories shown in \Fig{fig:Trajectories} do not fully complete the last revolution about Mars due to the dropping of the usual geometrical constraint on the revolutions number. Nevertheless, they grant temporary capture at least over the finite horizon specified by the integration interval.

% Figure environment removed
 
\begin{table}[pos=tbp,width=0.9\textwidth,align=\centering]
	\caption{Initial conditions of sample orbits.}
	\label{tab:ics}
	\begin{tabular*}{0.9\textwidth}{@{}CRRRRL@{}}
		\toprule
		\multirow{2}{*}{\textbf{Orbit}} & \multicolumn{4}{c}{\textbf{Initial condition at} $ f_{0} = 0 $} & \multirow{2}{*}{\textbf{Set}} \\
		\cmidrule(lr){2-5}
		& \multicolumn{1}{c}{$ X_{0} = x_{0}-1+\mu $} & \multicolumn{1}{c}{$ Y_{0} = y_{0} $} & \multicolumn{1}{c}{$ {{x}'}_{0} $} & \multicolumn{1}{c}{$ {{y}'}_{0} $} & \\
		\midrule
		a & \num{-5.170000e-5} & \num{-1.000000e-4} & \num{6.258637e-02} &   \num{-3.235715e-02} & $ \mathcal{W}(-\pi) $ \\
		b & \num{-7.575000e-05} & \num{1.695000e-04} & \num{-4.999940e-02} & \num{-2.234486e-02} & $ \mathcal{K}(-\pi) $ \\
		c & \num{4.509000e-04} & \num{3.621000e-04} & \num{-1.913330e-02} & \num{2.382548e-02} & $ \mathcal{W}(-\pi) $ \\
		d & \num{4.533000e-04} & \num{3.475000e-04} & \num{-1.871302e-02} & \num{2.441039e-02} & $ \mathcal{X}(-\pi) $ \\
		e & \num{-7.094000e-05} & \num{1.960000e-04} & \num{-4.856863e-02} & \num{-1.757887e-02} & $ \mathcal{K}(2\pi) $ \\
		f & \num{-1.719000e-04} & \num{-9.739000e-05} & \num{2.616042e-02} & \num{-4.617492e-02} & $ \mathcal{X}(2\pi) $ \\
		g & \num{-1.551000e-04} & \num{-9.239000e-05} & \num{2.842583e-02} & \num{-4.771994e-02} & $ \mathcal{W}(2\pi) $ \\
		h & \num{1.094000e-04} & \num{-3.258000e-04} & \num{3.796152e-02} & \num{1.274705e-02} & $ \mathcal{K}(2\pi) $ \\
		i & \num{-1.286000e-04} & \num{3.018000e-04} & \num{-3.772815e-02} & \num{-1.607634e-02} & $ \mathcal{C}(-\pi,3\pi/2) $ \\
		j & \num{-6.373000e-05} & \num{2.585000e-04} & \num{-4.429485e-02} & \num{-1.092035e-02} & $ \mathcal{C}(-\pi,3\pi/2) $ \\
		k & \num{-4.990000e-04} & \num{4.317000e-04} & \num{-1.863920e-02} & \num{-2.154496e-02} & $ \mathcal{C}(-\pi,3\pi) $ \\
		l & \num{-1.719000e-04} & \num{7.575000e-05} & \num{-2.195327e-02} & \num{-4.981872e-02} & $ \mathcal{C}(-\pi,3\pi) $ \\
		\bottomrule
	\end{tabular*}
\end{table}


