
\section{Methodology} \label{sec:methodology}

{\color{red}
\subsection{Low-energy regime}
\gls*{BC} orbits can be identified as those solutions allowing for material transfer between interior and exterior realms \cite{conley1968low}. In the \gls*{CR3BP}, they could be quantitatively identified as trajectories with Jacobi constant just below that of the collinear Lagrangian points $L_1$ and $L_2$. In the Sun--Mars system under study, this means trajectories with $ C_J < C_{J1} = \num{3.000203} $ (\ie transfers between primary and secondary interior realms) and $ C_J < C_{J2} = \num{3.000202} $ (\ie transfers between interior and exterior realms) \cite{conley1968low}. On the other hand, when the Jacobi constant falls below that of the collinear Lagrangian point $L_3$ ($ C_J < C_{J3} = \num{3.000001} $), then high-energy transfers are found \cite{campagnola2010endgame,restrepo2017patched}. However, qualitative statement on allowed and forbidden regions are no longer possible in the elliptic problem since the Jacobi value becomes anomaly-dependent \cite{hyeraci2010method}.}

\subsection{Particle stability definition} \label{sec:stability-def}
Particle stability is inferred using an alternative formulation of the stable sets defined in \cite{luo2014constructing}. While propagating \glspl*{IC} in the non-dimensional, synodic reference frame, the particle non-dimensional distance $ r(f) $ and Kepler energy $ H(f) $ with respect to the target body $P_2$ are computed \cite{hyeraci2010method}. The following indications are used to classify stability: A) a particle escapes at $f=f_e$ if $ H(f_e)>0 \ \wedge \ r(f_e)>R_{\mathrm{SOI}} $; B) a particle impacts  the surface of the target at $f=f_i$ if $ r(f_i)<R $. Based on its dynamical behavior over the integration interval $ [f_{0}, f_{f}] $, a propagated trajectory is said to be: i) \textit{weakly stable} if the particle neither escape nor impact with the target, so belonging to the subset $ \mathcal{W}(f_{f}) $; ii) \textit{unstable} if the particle escapes from the target before $ f_{f} $, then condition A) is verified for $ f_{e} \in [f_{0}, f_{f}] $, so belonging to the subset $ \mathcal{X}(f_{f}) $; iii) \textit{crash} if the particle impacts with the target before $ f_{f} $, then condition B) is verified for $ f_{i} \in [f_{0}, f_{f}] $, so belonging to the subset $ \mathcal{K}(f_{f}) $. A capture set is defined as $ \mathcal{C}(f_{B},f_{F}) \coloneqq \mathcal{X}(f_{B}) \cap \mathcal{W}(f_{F}) $ where $ f_{B} < f_{0} $ (backward leg), and $ f_{F} > f_{0} $ (forward leg).

\tr{If neither the escape criteria nor the impact criteria are matched, then the orbit is considered weakly stable in the interval $ [f_0, f_f] $, independently of the type of orbit considered. The definition adopted in this study is the same used in \cite{hyeraci2010method,luo2014constructing}. The aforementioned alternative particle stability formulation only regards the count of revolutions about the target, which in this work is neglected.}

\subsection{Lagrangian descriptors}
By manipulating the definition given in \cite{mancho2013lagrangian}, we define the \gls*{LD} as
\begin{linenomath*}
\begin{equation}
	M(\mathbf{x}_0,f_0,f_{B},f_{F}) = \int_{f_{0}+f_{B}}^{f_{0}} |\mathcal{F}(\mathbf{x}(f))|^{\gamma} \di f + \int_{f_{0}}^{f_{0}+f_{F}} |\mathcal{F}(\mathbf{x}(f))|^{\gamma} \di f,
	\label{eq:LDdef}	
\end{equation}
\end{linenomath*}
where $ \mathbf{x} = [x,y,x',y'] $ is the state vector obtained by rearranging \Eq{eq:er3bp} as a four-dimensional, first-order system of ordinary differential equations $ \mathbf{x}' = \mathbf{f}(\mathbf{x},f) $. The integrand $ |\mathcal{F}(\mathbf{x}(f))|^\gamma$ in \Eq{eq:LDdef} is a bounded, positive quantity, while $\gamma$ is the exponent defining the norm \cite{mancho2013lagrangian}. In this study, we select $ \mathcal{F} \coloneqq \sqrt{(x')^2+(y')^2} $ and $\gamma = 1/2 $ because they highlight the geometrical structures of the phase space better than the other integrands and norms as in \cite{mancho2013lagrangian}. The \gls*{LD} field is then defined as $ \mathcal{M}(f_{0},f_{B},f_{F}) \coloneqq \{ M(\mathbf{x}_0,f_{0},f_{B},f_{F}) \mid \mathbf{x}_{0} \in \Omega\} $, where $ \Omega $ is the set containing the \glspl*{IC}. In practise, the \gls*{LD} is computed appending its integrand to the space state equations with a zero initial value, and propagating the extended dynamics. The integration of the extended dynamics is stopped at $ f_{i} $ if the particle impacts with the target body.

An abrupt change in the \gls*{LD} field yelds discontinuous derivatives along the direction transverse to the change. Such singularities coincide with phase space structures separating trajectories with different dynamics, so abrupt changes correspond to dynamics separatrices \cite{mancho2013lagrangian}. In \Eq{eq:LDdef}, $ M(\mathbf{x}_0,f_{0},0,f_{F}) $ isolates dynamics separatrices obtained propagating \glspl*{IC} forwards, thus they are linked to repelling \glspl*{LCS}. Conversely, $ M(\mathbf{x}_0,f_{0},f_{B},0)$ reveals separatrices backwards, so highlighting the attracting \glspl*{LCS} \cite{lopesino2017theoretical}.

\subsection{Extraction of separatrices}
The structures revealed by the \gls*{LD} field are extracted with an edge detection algorithm. Edge detection is an image processing technique usually exploited for finding boundaries of objects within images. An edge is defined as the locus of points where an abrupt change in intensity of the image occurs. Several edge detection algorithms are available (\eg Sobel, Prewitt, Roberts, Canny, and zero-cross methods) \cite{davis1975survey}. \tr{Roberts' operator appears to be the most effective in extracting edges from the \gls*{LD} field of the problem at hand \cite{roberts1963machine}.} 

Given the image of the scalar field, the algorithm finds edges at those points where the gradient magnitude of the image is larger than a sensitivity threshold $ \sigma $ provided as input. The gradient of the image is approximated by computing the sum of the squares of the differences between diagonal neighbors pixels \cite{davis1975survey,roberts1963machine}. The threshold value is tuned\footnote{Values too small ($ < 10^{-3}$) could generate false positives in the output binary image. The larger the final true anomaly, the larger is the threshold suggested to use. The trend is justified because changes in the \gls*{LD} value at the separatrices are stronger for longer propagations.} to show as many structures as possible associated to abrupt changes in the \gls*{LD} field. 

\subsection{Validation of separatrices}

% Figure environment removed

The dynamics separatrices extracted from the \gls*{LD} field are expected to match the \gls*{WSB} computed on the same integration interval. The validation procedure devised to verify the correlation is outlined in \Fig{fig:Method}. Firstly, a uniform computational grid $ \Omega $ having $ 500 \times 500 $ points and centered at the target body is built over the square domain $ [-\varepsilon,\varepsilon] \times [-\varepsilon,\varepsilon] $, with $ \varepsilon = \num{6e-4} $. At $f_{0}$, the particle is assumed at the periapsis of an osculating prograde elliptic orbit about the target body with given eccentricity $ e_{0} = 0.9 $ (see \cite{hyeraci2010method} for more details). Secondly, \glspl*{IC} are propagated in the $ [f_{0}, f_{f}] $ interval. The \gls*{LD} values are computed and the \glspl*{IC} are allocated into the sets $\mathcal{W}$, $\mathcal{X}$, or $\mathcal{K}$ according to the stability definition discussed in \Sec{sec:stability-def}. Then, the separatrices are extracted with the edge detection algorithm. Finally, the patterns are inspected against the \gls*{WSB}. 

