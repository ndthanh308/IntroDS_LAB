\section{Introduction} \label{intro_hybridaugment}

The last decade witnessed machine learning (ML) elevating many methods to new heights in various fields. Despite surpassing human performance in multiple tasks, the \textit{generalization} of these models are hampered by distribution shifts, such as adversarial examples \cite{szegedy2013intriguing}, common image corruptions \cite{hendrycks2019augmix} and out-of-distribution samples \cite{yang2021generalized}. Addressing these issues are of paramount importance to facilitate the wide-spread adoption of ML models in practical deployment, especially in safety-critical ones \cite{rosenberg2021adversarial,deng2020analysis}, where such distribution shifts are simply inevitable.

Distribution shift-induced performance drops signal a gap between how ML models and us humans perform perception. Several studies attempted to bridge, or at least understand, this gap from architecture \cite{Benz_2021_WACV,Yeo_2021_ICCV,Saikia_2021_ICCV} and training data \cite{hendrycks2019augmix,wang2021augmax,Lee_2020_CVPR_Workshops,chen2021amplitude,prime_aug,calian2022defending,Hendrycks_2022_CVPR} centric perspectives. An interesting perspective is built on the frequency spectra of the training data; convolutional neural networks (CNN) are shown to leverage high-frequency components that are invisible to humans \cite{wang2020high} and also shown to be reliant on the amplitude component, as opposed to the phase component humans favour \cite{chen2021amplitude}. Several studies leveraged frequency spectra insights to improve model robustness. These methods, however, either leverage cumbersome ensemble models \cite{Saikia_2021_ICCV}, formulate complex augmentation regimes \cite{frequencyaug_competitor_method,eccv2022_freqpaper} or focus on a single robustness venue \cite{long2022frequency,frequencyaug_competitor_method,eccv2022_freqpaper} rather than improving the broader robustness to various distribution shifts. Furthermore, it is imperative to preserve, if not improve, the clean accuracy levels of the model while improving its robustness.


% Figure environment removed

Our work aims to improve the robustness of CNNs to various distribution shifts. Inspired by the frequency spectra based data augmentations,  we propose \textit{HybridAugment}, inspired from the well-known hybrid images \cite{oliva2006hybrid}. Based on the observation that the label information of images are predominantly related to the low-frequency components \cite{wang2020towards,li2022robust}, \textit{HybridAugment} simply swaps high-frequency and low-frequency components of randomly selected images in a batch, regardless of their class labels. This forces the network to focus on the low-frequency information of images and makes the models less reliant on the high-frequency information, which are often shown to be the root cause of robustness issues~\cite{wang2020towards}.  With virtually no training overhead, \textit{HybridAugment} 
improves the corruption robustness while preserving or improving the clean accuracy, and additionally induces adversarial robustness. 

Additionally, we set our eyes on jointly exploiting the contributions of frequency spectra augmentation methods while unifying them into a simpler, single augmentation regime. We then propose \textit{HybridAugment++}, which performs hierarchical perturbations in the frequency spectra. Exploiting the fact that the phase component carries most of the information in an image \cite{chen2021amplitude}, \textit{HybridAugment++} first decomposes images into high and low-frequency components, swaps the amplitude and phase of the low frequency component with another image, and then combines this augmented low-frequency information with the high-frequency component of a random image. Essentially, \textit{HybridAugment++} forces the models to rely on the phase and the low-frequency information. As a result, \textit{HybridAugment++} further improves adversarial and corruption robustness, while further improving the clean accuracy against several alternatives. See Figure \ref{fig:figure1_hybridaugment} for a diagram of our methods.

Our main contributions can be summarized as follows.

\begin{itemize}
\vspace{-2.5mm}
\item We propose \textit{HybridAugment}, a simple data augmentation method that helps models rely on low-frequency components of data samples. It is implemented in just three lines of code and has virtually no overhead.
\vspace{-2.5mm}
\item We extend \textit{HybridAugment} and propose \textit{HybridAugment++}, which performs hierarchical augmentations in frequency spectra to help models rely on low-frequency and phase components of images. 
\vspace{-2.5mm}
\item We show that \textit{HybridAugment} improves corruption robustness of multiple CNN models, while preserving (or improving) the clean accuracy. We additionally observe clear improvements in adversarial robustness over strong baselines via \textit{HybridAugment}.
\vspace{-2.5mm}
\item \textit{HybridAugment++} similarly outperforms many alternatives by further improving corruption and clean accuracies on multiple benchmark datasets, with additional gains in adversarial robustness.
\end{itemize}

