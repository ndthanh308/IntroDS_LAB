% mnras_template.tex
%
% LaTeX template for creating an MNRAS paper
%
% v3.0 released 14 May 2015
% (version numbers match those of mnras.cls)
%
% Copyright (C) Royal Astronomical Society 2015
% Authors:
% Keith T. Smith (Royal Astronomical Society)

% Change log
%
% v3.0 May 2015
%    Renamed to match the new package name
%    Version number matches mnras.cls
%    A few minor tweaks to wording
% v1.0 September 2013
%    Beta testing only - never publicly released
%    First version: a simple (ish) template for creating an MNRAS paper

%%%%%%%%%%%%%%%%%%%%%%%%%%%%%%%%%%%%%%%%%%%%%%%%%%
% Basic setup. Most papers should leave these options alone.
\documentclass[fleqn,usenatbib]{mnras}

% MNRAS is set in Times font. If you don't have this installed (most LaTeX
% installations will be fine) or prefer the old Computer Modern fonts, comment
% out the following line
\usepackage{newtxtext,newtxmath}
% Depending on your LaTeX fonts installation, you might get better results with one of these:
%\usepackage{mathptmx}
%\usepackage{txfonts}

% Use vector fonts, so it zooms properly in on-screen viewing software
% Don't change these lines unless you know what you are doing
\usepackage[T1]{fontenc}
\usepackage{ae,aecompl}


%%%%% AUTHORS - PLACE YOUR OWN PACKAGES HERE %%%%%

% Only include extra packages if you really need them. Common packages are:
\usepackage{multicol}
\usepackage{graphicx}    % Including figure files
\usepackage{amsmath}    % Advanced maths commands
%\usepackage{amssymb}    % Extra maths symbols
\usepackage{xcolor}

\usepackage{etoolbox}
\makeatletter
%\patchcmd\@combinedblfloats{\box\@outputbox}{\unvbox\@outputbox}{}{%
%   \errmessage{\noexpand\@combinedblfloats could not be patched}%
%}%
 \makeatother

%%%%%%%%%%%%%%%%%%%%%%%%%%%%%%%%%%%%%%%%%%%%%%%%%%

%%%%% AUTHORS - PLACE YOUR OWN COMMANDS HERE %%%%%

% Please keep new commands to a minimum, and use \newcommand not \def to avoid
% overwriting existing commands. Example:
%\newcommand{\pcm}{\,cm$^{-2}$}    % per cm-squared

%\newcommand{\marco}[1]{{\textcolor{magenta}{\bf #1}}} % MP comments
%\newcommand{\chris}[1]{{\textcolor{blue}{\bf #1}}} % CS comments
%\newcommand{\diego}[1]{{\textcolor{green}{\bf #1}}} % DT comments
%\newcommand{\benoit}[1]{{\textcolor{purple}{\bf #1}}} % DT comments
%\newcommand{\TL}[1]{{\textcolor{olive}{\textbf TL: #1}}}  % TL comments

%%%%%%%%%%%%%%%%%%%%%%%%%%%%%%%%%%%%%%%%%%%%%%%%%%

%%%%%%%%%%%%%%%%%%% TITLE PAGE %%%%%%%%%%%%%%%%%%%

% Title of the paper, and the short title which is used in the headers.
% Keep the title short and informative.
%\title[Chemical evolution of the solar neighbourhood]{Planet-hosting stars and the chemical evolution of the solar neighbourhood}
\title[Chemical evolution of the solar neighbourhood]{The chemical evolution of the solar neighbourhood for planet-hosting stars}

% The list of authors, and the short list which is used in the headers.
% If you need two or more lines of authors, add an extra line using \newauthor
\author[M. Pignatari et al.]{Marco Pignatari$^{1,2,3,8,9}$,
Thomas C.~L. Trueman$^{1,3,8}$,
Kate A. Womack$^{3}$,
Brad K. Gibson$^{3,9}$,\newauthor
Benoit C\^ot\'e$^{1,8,9,10}$,
Diego Turrini$^{4,5,6}$,
Christopher Sneden$^{7}$,
Stephen J. Mojzsis$^{1,11}$,\newauthor
Richard J. Stancliffe$^{8,12}$,
Paul Fong$^{3,8}$,
Thomas V. Lawson$^{3,8,13}$,
James D. Keegans$^{14,8}$,\newauthor
%Samuel Jones$^{7,9}$,
Kate Pilkington$^{15}$,
Jean-Claude Passy$^{16}$,
Timothy C. Beers$^{17,9}$,
Maria Lugaro$^{1,2,18,19}$
\\
% List of institutions
$^{1}$Konkoly Observatory, Research Centre for Astronomy and Earth Sciences (CSFK), ELKH, H-1121, Budapest, Konkoly Thege M. \'ut 15–17., Hungary\\
$^{2}$CSFK, MTA Centre of Excellence, Budapest, Konkoly Thege Miklós út 15-17., H-1121, Hungary\\
$^{3}$E.~A.~Milne Centre for Astrophysics, Department of Physics and Mathematics, University of Hull, HU6 7RX, United Kingdom\\
$^{4}$ INAF - Turin Astrophysical Observatory, Via Osservatorio 20, 10025, Pino Torinese, Italy\\
$^{5}$ INAF - Institute of Space Astrophysics and Planetology, Via Fosso del Cavaliere 100, 00133, Rome, Italy\\
$^{6}$ ICSC National Research Centre for High Performance Computing, Big Data and Quantum Computing, Via Magnanelli 2, 40033, Casalecchio di Reno, Italy\\
$^{7}$Department of Astronomy and McDonald Observatory, The University of Texas, Austin, TX 78712, USA\\
%$^{7}$X Computational Physics (XCP) Division, Los Alamos National Laboratory, New Mexico 87544, USA\\
$^{8}$NuGrid Collaboration, \url{http://nugridstars.org}\\
$^{9}$Joint Institute for Nuclear Astrophysics - Center for the Evolution of the Elements, USA\\
$^{10}$Department of Physics and Astronomy, University of Victoria, Victoria, BC V8P 5C2, Canada\\
$^{11}$Department of Petrology and Geochemistry, Eötvös Loránd University (ELTE), Pázmány Péter sétány, 1/c, Budapest H-1117, Hungary\\
$^{12}$H.H. Wills Physics Laboratory, University of Bristol, Tyndall Avenue, Bristol, BS8 1TL, United Kingdom\\
$^{13}$Centre of Excellence for Data Science, Artificial Intelligence and Modelling, University of Hull, HU6 7RX, United Kingdom\\
$^{14}$Astrophysics Group, Lennard-Jones Laboratories, Keele University, Keele ST5 5BG, United Kingdom\\
$^{15}$School of Physical Sciences, The Open University, Milton Keynes, MK7~6AA, United Kingdom\\
$^{16}$Max Planck Institute for Intelligent Systems, Tübingen, Germany \\
$^{17}$Department of Physics and Astronomy, University of Notre Dame, Notre Dame, IN, 46556, USA\\
$^{18}$ELTE Eötvös Loránd University, Institute of Physics, Budapest 1117, Pázmány Péter sétány 1/A, Hungary\\
$^{19}$School of Physics and Astronomy, Monash University, VIC 3800, Australia
}
% These dates will be filled out by the publisher
\date{Accepted XXX. Received YYY; in original form ZZZ}

% Enter the current year, for the copyright statements etc.
\pubyear{2023}

% Don't change these lines
\begin{document}

\label{firstpage}
\pagerange{\pageref{firstpage}--\pageref{lastpage}}
\maketitle
% Abstract of the paper
\begin{abstract}
%The chemical composition of the Sun and of stars in the solar neighbourhood and the pristine composition of stellar systems in general, are crucial inputs for any theoretical models of the formation of their planetary systems. 
Theoretical physical-chemical models for the formation of planetary systems depend on data quality for the Sun's composition, that of stars in the solar neighbourhood, and %crucially 
of the estimated "pristine" compositions %of 
for stellar systems. % in general.
%As a case of point, t
The effective scatter and the observational uncertainties of %elemental ratios 
elements within a few hundred parsecs from the Sun,
%larger than reported observational errors, 
even for the most abundant metals like carbon, oxygen and silicon, are still controversial. %still a matter of debate. 
%The increasing amount of stellar spectroscopic data within few hundreds parsecs from the Sun revealed a scatter in elemental ratios larger than reported observational errors, even for the most abundant metals like carbon, oxygen and silicon. 
%The clear understanding of the chemical enrichment history of the solar neighbourhood is becoming paramount in the forthcoming years, where new space missions like ARIEL will provide a direct observation of the compositions of giant and super-hearts planets orbiting around other stars. 
Here we analyse the stellar production and the chemical evolution of %several 
key elements that underpin the formation of rocky (C, O, Mg, Si) and gas/ice giant planets (C, N, O, S). %To achieve this, w
We calculate 198 galactic chemical evolution (GCE) models of the solar neighbourhood to analyse the impact of different sets of stellar yields, of the upper mass limit for massive stars contributing to GCE ($M_{\rm up}$) and of %faint supernovae, i.e., 
supernovae from massive-star progenitors which do not eject the bulk of the %Fe-rich ejecta 
iron-peak elements (faint supernovae).
Even considering %taking into account 
the GCE variation %in GCE
produced via different sets of stellar yields, %we do not reproduce 
the observed dispersion of %elemental ratios 
elements reported for stars in %the literature within typical evolution timescales of 
the Milky Way disk is not reproduced. Among others, the observed range of super-solar [Mg/Si] ratios, sub-solar [S/N], and the dispersion of %about 
up to 0.5 dex for [S/Si] challenge our models. %to reproduce. 
The impact of varying $M_{\rm up}$ depends on the adopted %set of 
%massive star 
supernova yields. Thus, %we find that 
observations do not provide a %clear 
constraint on the M$_{\rm up}$ parametrization. When including the impact of %representative 
faint supernova models in GCE calculations, elemental ratios vary by up to 0.1-0.2 dex in the Milky Way disk; this modification better reproduces observations. 
%In the future, improved stellar observations and solar abundances will help to reduce the uncertainties still associated to the chemical evolution of the solar neighbourhood. 
%We discuss the impact of the uncertainties associated to stellar observations and to the solar abundances on the analysis presented. 
%We discuss these results considering the different elements relevant for planet formation. 
%New extended sets of GCE simulations including faint supernovae can be used to generate maps of complete abundance sets with respect time for the solar neighbourhood, to use for the next generation of models for planet formation.    
\end{abstract}

% Select between one and six entries from the list of approved keywords.
% Don't make up new ones.
\begin{keywords}
Galaxy: abundances, disc, evolution; (Galaxy:) solar neighbourhood; (stars:) planetary systems ; stars: abundances. 
\end{keywords}

%%%%%%%%%%%%%%%%%%%%%%%%%%%%%%%%%%%%%%%%%%%%%%%%%%

%%%%%%%%%%%%%%%%% BODY OF PAPER %%%%%%%%%%%%%%%%%%

\section{Introduction}
\label{sec: intro}

The chemical enrichment history of the elements observed in the Sun and in other stars in the solar neighbourhood 
%provides a fundamental source of 
serves as the basis for our information about the formation and the chemical evolution of the Milky Way (MW) disk \citep[e.g.,][]{truran:71, tinsley:78, timmes:95, goswami:00, matteucci:21}. Galactic chemical evolution (GCE) simulations attempt to model the change with time of the chemical elements by taking into account the formation of the MW disk and including theoretical stellar yields from different generations of stars. The GCE models are then compared to stellar abundance trends with metallicity or age of the MW disk, and to the solar abundance pattern \citep[e.g.,][]{matteucci:86,gibson:03, kobayashi:11, molla:15, mishenina:17,prantzos:18, kobayashi:20, prantzos:23}. The composition for all the elements can be measured in the Sun and in meteorites \citep[][and references therein]{lodders:19}, whereas a more limited number of elements are available for other stars. Nevertheless, elemental abundances are preserved with limited modification over time at the stellar surfaces, and are therefore taken to be indicative of the pristine stellar abundances \citep[e.g.,][]{piersanti:07}.

Analyses become rather more complex for discussions of likely compositions of the planets that may have formed around these stars. Stars typically represent $>$98$\%$ of the mass of a star+planet(s) system. So, the original abundances of stellar systems are recapitulated in the composition of the host star, which in turn mirrors the composition at the start of the planetary-formation process. The bulk composition of the stellar system is the ultimate arbiter for the properties of the planets that will form. This holds not only for stable elements, but may also be true for the short-lived radioactive isotopes relevant for the
heating of planetesimals \citep[mostly $^{26}$Al in the case of the early Solar System, e.g.,][]{kleine:05,lichtenberg:16,lugaro:18} and the radiogenic heating of planet interiors via the long-lived radionuclides $^{40}$K, $^{232}$Th, $^{235}$U and $^{238}$U \citep[e.g., ][]{frank:14,unterborn:15,wang:20}.

%Nevertheless, depending on where and how the planets formed, their migration history, the global dynamical history of their system, and the different behaviour of volatile and refractory elements at different distances from the host stars, the planetary formation process can modify or even erase the signatures of the initial chemical abundances of the system 
%\citep[see e.g.,][and references therein]{madhusudhan:16,madhusudhan:19,cridland:19,cridland:20,turrini:21a,turrini:22,adibekyan:21,drazkowska:22,pacetti:22}. 
Nevertheless, depending on where and how the planets formed, their migration history, the global dynamical history of their system, the planetary formation process can modify or even erase the signatures of the initial chemical abundances of the system for ultra-volatile elements like H, C, N, O and S at different distances from the host stars  
\citep[see later in the Introduction, and e.g.,][and references therein]{madhusudhan:16,madhusudhan:19,cridland:19,cridland:20,turrini:21a,turrini:22,adibekyan:21,drazkowska:22,pacetti:22}. 

Iron is also problematic, since it is by definition siderophile and along with other such elements (Ni), tends to be sequestered into the metallic\footnote{Note that in this context metallic cores are made mostly by Fe and Ni. In the rest of the paper, we refer as "metals" all the nuclides heavier than H and He.} cores of rocky planets
during their differentiation process. On the other hand, refractory lithophile elements like Ca and the Rare Earth Elements are unaffected by these processes. Moderately refractory lithophile elements such as Li, Mg and Si, and some other moderately volatile lithophile elements such as K and Na, follow a devolatilization trend based on %the ionization potential 
the different condensation temperatures
of the elements \citep[][]{yoshizaki:20,wang:22,spaargaren:23}. 
%\textbf{[Steve here, 1 sentence to  clarify how this devolatilization trend can be based on the ionization potential.]} 
%, moderately refractories like Mg and Si, and some other moderately volatile elements such as K and Na should be less affected from these processes. Indeed, observations for these elements from Earth, Venus, Mercury and Mars and the Moon, Vesta and all meteorite groups including the achondrites (e.g. Angrites) match the solar composition.
%Two similar stars that formed at the same time and with similar composition may therefore develop two different planetary systems with different types of planets, or with planets belonging to the same class but with different abundances in the mantle and in the atmosphere. Therefore, simulations of planetary formation and evolution are necessary to understand the connection between the GCE of the solar neighbourhood and planetary observations \citep[e.g.,][]{santos:17,turrini:21a,turrini:21b,adibekyan:21,khorshid:21,reggiani:22,jorge:22,pacetti:22}. 

%Therefore, simulations of planetary formation and evolution become necessary to understand the connection between the all GCE of the solar neighbourhood and present planetary observations \citep[e.g.,][]{santos:17,turrini:21a,turrini:21b,adibekyan:21,khorshid:21,reggiani:22,jorge:22,pacetti:22}. 

A criterion often invoked in arguments for the geodynamic predisposition of a planet to host life (so-called "habitability") is the metal enrichment \citep[][]{lineweaver:04,spitoni:14,spitoni:17}. Consequently, initial major elemental ratios such as C/O and Mg/Si are regarded as especially crucial in modulating the chemistry of early condensates and the mineralogy of rocky planets that are conducive for biological activity to take hold \citep[][]{Mojzsis22}. 

Based on observational results, it has been proposed that these ratios also modulate the types of planet formed. For instance, \cite{adibekyan:15} found that low-mass planets are more prevalent around stars with Mg/Si higher than solar, and in general for stellar hosts with high [Mg/Si] ratios after removing the GCE trend of the two elements. From the theoretical point of view this is expected \citep[e.g.,][]{frank:14}, but a broader analysis of Mg/Si with respect to exoplanet populations \citep[][]{spaargaren:23} is warranted.


%%%%%%%%%%%%%%%%%%%%%%%%%%%%%%%%%%%%%%%%%%%%%%%%%%%%%%%%%%%%%%%%%%%%%%
% Steve's comment: I redacted this text, because arguably, little of it is correct.  The observations of Bond and Co. turned out to be strange and incorrect, because C/O above 0.8 is EXCEEDINGLY rare.  Spaargaren+ (submitted) analysed about 7000 FGK stars in the solar neighbourhood and found almost none with this characteristic.  The H and O arguments are probably best left out (they are not developed well enough to make a particular case). The paper by Raymond+07 is wrong; the work is out-of-date and cannot explain the composition of the terrestrial planets as we now understand them.  Best to just leave this stuff out.
%%%%%%%%%%%%%%%%%%%%%%%%%%%%%%%%%%%%%%%%%%%%%%%%%%%%%%%%%%%%%%%%%%%%%
%Additionally, according to e.g., \cite{bond:10, delgadomena:10} and \cite{carter:12}, habitable planets like Earth are more favoured to form with solar-like initial compositions within so called galactic terrestrial zones. 
%For C/O greater than 0.8, more C is available to form graphites and other compounds to contribute to planetary early aggregates such as as silicon carbides. Instead, for lower concentrations of C with respect to O, Si contributes to form silicates, and the detailed chemistry is dominated by the Mg/Si ratio \citep[][]{bond:10,jorge:22}. On the other hand, results by suggest instead that H and O are so widely abundant that C almost always goes into gas. In this case, we should not expect anyway to find exotic planets (e.g., diamond planets) in the terrestrial planet region, independently from the pristine C/O ratio.
%Finally, the initial Mg/Si ratio can also be important to define the mass distribution of the forming population of planets \citep[][]{adibekyan:15, cabral:19, bitsch:20}. Although Earth-like planet formation models do not typically depend on composition \citep[see e.g.,][]{raymond:07}, the chemical composition strongly affects the following evolution of the planet \citep[e.g.,][]{frank:14, alibert:17}.
Given the above criteria, simulations of planetary formation and evolution demand a better understanding of the connection between GCE models for the composition of stars in the solar neighbourhood, and the particular compositional characteristics observed for the planet-hosting stars and for their planets \citep[e.g.,][]{santos:17,turrini:21a,turrini:22,adibekyan:21,khorshid:21,reggiani:22,jorge:22,pacetti:22,fonte:23}. 
Consequently, GCE models can be used then as a theoretical source for the initial abundances at planet formation for all elements (observed with different uncertainties or not available in the stellar spectra) at different times and locations in the Galaxy, and as a benchmark for the results of planet formation obtained from simulations \citep[][]{frank:14,mojzsis:22}. 

%Concerning giant planets
With respect to the origin of the gas and ice giant planets in our Solar System and beyond, their present C/O-ratio has been proposed as a diagnostic to distinguish between different formation processes where gas accretion or capture of planetary material may dominate, with following modifications of the initial C/O-ratio \citep[e.g.,][]{oberg:11,madhusudhan:16,madhusudhan:19}. In particular, \cite{turrini:21a} show that when the capture of planetary material is the dominant source of planetary metallicity, the C/O-ratio of giant planets is close to the stellar C/O-ratio \citep{turrini:22}. 
%for giant planets forming outside the CO$_{2}$ snowline the stellar C/O ratio is preserved. 
For giant planets where the accretion of disc gas is the dominant source of the planetary metallicity, the C/O-ratio can be both super-stellar and sub-stellar depending on the chemical structure of the circumstellar disc where the giant planet was born \citep[][]{pacetti:22}. The precise determination of the stellar C/O-ratio therefore may provide information on the planet-formation history and the native circumstellar disc if the C and O abundances of giant planets can be determined \citep[][]{turrini:22,pacetti:22}. For observational validations see also \cite{carleo:22}, \cite{guilluy:22} and \cite{biazzo:22}.  
%ARIEL capabilities will allow to access N and S molecules, where these elements 
Recent studies further expanded the range of elements that can be used to investigate the formation history of gas giant planets to N \citep[][]{oberg:19,bosman:19,cridland:20,turrini:21a,turrini:22,pacetti:22} and S \citep[][]{turrini:21a,turrini:22,pacetti:22}. In particular, \cite{turrini:21a} and \cite{turrini:22} argue 
%that N and S can be used together with C and O to have more clear diagnostic of the planet formation process and 
that the combined use of the abundance ratios of elements with different volatility like C, O, N, and S provides more unequivocal constraints on the planet-formation history than C/O alone. As an example, the C/N ratio will monotonically grow with migration for solid-enriched giant planets and decrease for gas-dominated giant planets, also in those cases where the C/O-ratio remains close to stellar \citep{turrini:21a,turrini:22}. Furthermore, \cite{turrini:21a} and \cite{turrini:22} showed that the information provided by these elemental ratios becomes immediately accessible once the planetary abundances are normalised to the stellar abundances and that the use of this normalised scale allows for the straightforward comparison between giant planets formed around different stars, as later supported by observational studies \citep[][]{kolecki:21,biazzo:22}. For a recent application with C and S on JWST data, see \cite{crossfield:23}.


While the observed and/or inferred elemental ratios from other nearby planetary systems constrains our knowledge about them, the Solar System will still remain a fundamental benchmark for theoretical planetary models. In this case, physical properties and isotopic anomalies found in meteoritic material provide the data to constrain the main features and structures in the earliest stages of the proto-solar disk \citep[e.g.,][]{burkhardt:19, brasser:20}, and/or the following core formation and accretion timescale of giant planets, in particular of Jupiter \citep[e.g.,][]{kruijer:17, nanne:19}.
The Solar System also shows us the importance of using the information on the stellar composition to validate exoplanetary observations. The comparison between Jupiter's elemental abundances and the Solar ones, in particular, may allow us to infer the Jovian Mg/O, Fe/O e Si/O ratios and quantity how the formation of refractory oxides alters the atmospheric C/O ratio of the giant planet \citep[][]{fonte:23}.
%the C/N and the S/N elemental ratios for giant planets formed inside the CO$_{2}$ snowline are preserving the stellar values, while significant departures from the pristine values are expected otherwise \citep[][]{turrini:21a}.

Whereas the collection and improvement of abundance data for stars within a few hundred parsecs has long been a priority (e.g., the GAIA-ESO and the GALAH surveys, see \citealt{gilmore:12} and \citealt{desilva:15}, respectively), such a capability for planets is in its infancy. In the next two decades, observatories like JWST \citep[James Webb Space Telescope,][]{beichman:14} and ARIEL \citep[Atmospheric Remote-sensing Infrared Exoplanet Large-survey,][]{tinetti:18,turrini:18,edwards:19} will expand and deepen the amount of planetary abundance data mostly from retrieved spectra from (hot) planetary atmospheres, gathering data in alignment with other existing and future facilities like e.g., TESS \citep[Transiting Exoplanet Survey Satellite,][]{ricker:15}, CHEOPS \citep[Characterizing ExoPlanets Satellite,][]{broeg:13} and PLATO \citep[Planetary Transits and Oscillations of stars,][]{rauer:14}. For a comprehensive list of present and future facilities and observatories, we refer to \cite{tinetti:18}. This will be the framework in the coming years where GCE, planet formation, nuclear astrophysics and stellar and planetary observations will be parallelized to provide a new comprehensive picture of stellar and planet systems formation and evolution.   
 
This work begins with an analysis of the stellar production (\S~\ref{sec: stars}) and the galactic chemical evolution (\S~\ref{sec: tk_plots/models}) of the elements that are crucial for the formation of planets, as discussed above: C, N, O, Mg, Si, and S. We present the main uncertainties associated with stellar observations and solar abundances in \S~\ref{sec: solar_wasp_nest}, followed by the main results in \S~\ref{sec: results}, and conclusions in \S~\ref{sec: summary}.


\section{Production of elements in stars}
\label{sec: stars}

It is well established that multiple stellar sources contributed to the chemical enrichment of the Milky Way disk. In particular, the main source of metals are Core-Collapse Supernovae \citep[CCSNe, see e.g.,][]{woosley:02,nomoto:13}, Asymptotic Giant Branch stars \citep[AGB stars e.g.,][]{herwig:05,karakas:14} and Thermonuclear Supernovae \citep[SNIa, e.g.,][]{hillebrandt:13}. We discuss each of these stellar metal sources that ultimately go into planet formation and present some examples below. 

For the metallicity range typical of stars in the solar neighbourhood, 
%the bulk inventories of C and N can be explained by production from low-mass and intermediate-mass AGB stars (M$\lesssim$8M$_{\odot}$). 
the bulk inventory of N can be explained by production from low-mass and intermediate-mass AGB stars (M$\lesssim$8M$_{\odot}$). AGB stars also make an amount of C comparable to the CCSNe contribution \citep[][]{kobayashi:20} or larger \citep[e.g., this work and][]{goswami:00, chiappini:03}, with the relative relevance of the two sources that is still matter of debate \citep[e.g.,][]{prantzos:94, romano:17}. 
The AGB phase is the last evolutionary stage before these low-mass and intermediate-mass stars eject their entire envelope into the Interstellar Medium (ISM) and evolve as planetary nebula, then develop into white dwarfs. With respect to the production of the elements, the AGB phase is crucial since this is when the bulk of new metals under consideration here are made and ejected. For recent studies of the nucleosynthesis in AGB stars, we refer to e.g., \cite{cristallo:15}, \cite{karakas:16},  \cite{jones:16}, \cite{bisterzo:17}, \cite{battino:19} and \cite{denhartogh:19}.
In Figure~\ref{fig: agb_elements}, the abundance profiles for 3M$_{\odot}$ and 5M$_{\odot}$ AGB stellar models by \cite{ritter:18} are shown close to the end of the evolution of these stars. The stellar envelope (to the right in the plot, where H is present) contributes to the enrichment of the ISM, while the interior part is the remnant that will form the future White Dwarf. 
%The boundary of the AGB envelope is at about M=0.64M$_{\odot}$ and M=0.876M$_{\odot}$ for the 3M$_{\odot}$ and 5M$_{\odot}$ models, respectively. 
In the 3M$_{\odot}$ model (top panel, Figure~\ref{fig: agb_elements}), the largest enhancement appears for the species C and N, with much smaller increase for O, Mg and He. In particular, the envelope becomes C-rich, with C/O $>$ 1. Conversely, the initial Si, S, and Fe show no modification. 
%in the envelope by nucleosynthesis processes in the stellar interior. 
Only small variations are triggered by the activation of shell He-burning and shell H-burning in the He-intershell just below the envelope, which are not sufficient to modify the pristine elemental abundances. 
In the bottom panel of Figure~\ref{fig: agb_elements}, the AGB envelope in the 5M$_{\odot}$ model shows a relevant enrichment in N and C, with smaller increases of Mg, He, and O. 
%As for the 3M$_{\odot}$ star, there is no relevant modification for Si, S and Fe.

% Figure environment removed


Moreover, massive stars and CCSNe produced the bulk of O and Mg in stars in the MW disk, as well as relevant quantities %a significant  fraction 
of C, N, Si, S, and Fe. For these stellar objects, a substantial amount of the initial mass is lost by stellar winds, but most of the metals are ejected in the final CCSN explosion \citep[e.g.,][]{woosley:02}. A contemporary view, however, is that not all massive stars will successfully explode as CCSN. Depending on the progenitor structure and on details associated with the as yet poorly understood explosion mechanism, not all the material ejected by the SN explosion manages to escape and fall back to the compact central object \citep[e.g.,][]{fryer:12, ugliano:12, ott:18, fryer:18}. In the most extreme cases, all the ejecta fall back and only the winds contribute to the ISM enrichment. Present uncertainties of the CCSN mechanism, however, continue to undermine the efficacy of theoretical simulations \citep[e.g.,][]{fryer:18,mueller:16,janka:16,burrows:21}. Direct observations of recent CCSN remnants appear to confirm that a large variety of ejecta and energetic configurations are indeed possible \citep[e.g.,][]{nomoto:13,smartt:15,martinez:22}. It is unlikely that these different possibilities observed in recent CCSNe are mostly due to physics mechanisms seldom included in stellar model sets like rotation \citep[e.g.,][]{heger:00, hirschi:05, limongi:18} and/or more exotic types of supernova explosion like Pair-Instability Supernovae \citep[e.g.,][]{heger:02, kozyreva:14, takahashi:18, goswami:22}, which are expected to be more relevant for stars at low metallicity. The start of a successful or weaker CCSN explosion shown by the observations is instead most likely affected by the details of the stellar progenitor structure and by its interaction with the forming SN shock \citep[][]{wongwathanarat:13, burrows:21, varma:23}. For recent studies of the nucleosynthesis in massive stars and CCSNe, we refer to e.g., \cite{pignatari:16}, \cite{sukhbold:16}, \cite{ritter:18}, \cite{limongi:18}, \cite{curtis:19}, \cite{ebinger:20}, \cite{andrews:20}.

Figure~\ref{fig: ccsn_elements} shows the abundance profile of the CCSN ejecta of 15M$_{\odot}$, 20M$_{\odot}$ and 25M$_{\odot}$ models by \cite{ritter:18}. 
%The deepest ejecta are shown to the left, while the H-rich envelope is located to the right of the plots. Material which forms the compact central neutron star directly, or that will afterwards fall back onto it, is not considered.  
The 15M$_{\odot}$ model shows the most classical onion-layer structure (top panel, Figure~\ref{fig: ccsn_elements}). The explosive Si-burning and explosive O-burning ejecta are squeezed within the inner mass range 0.2-0.3 M$_{\odot}$, producing a peak of Fe (originally made as $^{56}$Ni, which will decay to $^{56}$Fe powering the peak of the CCSN lightcurve) and a peak of Si and S, respectively. 
Moving outward (toward the right in the plot), we find the O-rich and Mg-rich extended ashes of pre-supernova C fusion. The next large He-rich region represents the remains of the He-shell, with enrichment in C and O and the signature of explosive He-burning at the bottom (i.e., the Mg peak and a small Si peak at M=3.1M$_{\odot}$). At the top of the He-ashes are the remains of the pre-supernova H-shell, where C and O are consumed to make N. Finally, at the right edge of the plot is the H-rich envelope of the star. Such a structure is common and is shared across several sets of one-dimensional CCSN models \citep[e.g.,][]{woosley:95,thielemann:96}. 

The 20M$_{\odot}$ and 25M$_{\odot}$ models show a fundamental difference when compared to the 15M$_{\odot}$ model. In the 20M$_{\odot}$ model case the explosive Si-burning layers are not ejected, and only a fraction of the explosive O-burning escapes the gravitational bounds of the central compact object. Such a model represents what is dubbed a faint supernova \citep[e.g.,][]{heger:03, nomoto:13}, where the same nucleosynthesis as that in the 15M$_{\odot}$ star produces very different Si/Fe or Mg/Si ratios in the ejecta, due to the different outcome of the CCSN explosion. 

Faint CCSNe are considered potential sources of the peculiar stellar abundances observed in a number of old metal-poor stars, such as the so-called CEMP-no stars, i,e. carbon-enhanced, metal-poor stars with no enrichment of heavy elements \citep[e.g.,][and references therein]{beers:05,ishigaki:14, bonifacio:15, maeder:15, lee:19, zepeda:23}.
In fact, the most Fe-poor star known, SMSS J031300.36-670839.3, was proposed to carry the unique abundance signature of an Fe-poor faint CCSNe \citep[][]{keller:14, bessell:15, nordlander:17}. 
\cite{wehmeyer:19} discussed the contribution of faint CCSNe to explain the observed scatter of heavy element r-process enrichments with respect to iron in the early Galaxy. Those GCE simulations assume as main r-process sources neutron-star mergers and neutron star- black hole mergers, where black holes are considered as the remnants of faint CCSNe. Nevertheless, the contribution of faint CCSNe to GCE is not well defined. This applies also to the MW disk, 
%of the MW disk is unknown, 
although SN lightcurves and remnants of recent faint CCSNe explosions are observed \citep[e.g.,][]{nomoto:13}. 


In Figure~\ref{fig: ccsn_elements}, the 25M$_{\odot}$ model exhibits an even more extreme case of faint supernovae, where the whole explosive O-burning layers are not ejected.
Even so, depending on the CCSN explosion energy and the progenitor structure, a more or less effective explosive He-burning can produce Mg, Si, and even S in relevant quantities (see mass coordinates M = 5M$_{\odot}$ and M = 7-7.5M$_{\odot}$ for the 20M$_{\odot}$ and 25M$_{\odot}$ models, in the central and bottom panels in Figure~\ref{fig: ccsn_elements}, respectively). 
Analysis of the presence of a C/Si zone at the bottom of the He shell during the CCSN explosion explains the anomalous abundance signature measured in C-rich presolar grains made in CCSNe \citep[][]{pignatari:13}. In terms of contribution to the total CCSN yields, the explosive He-burning contribution to Mg, Si and S is typically small compared to that of the explosive Si- and O-burning ejecta. Yet, in the case of faint CCSNe, the contribution of the external layers to the total ejecta of these elements can be relevant. The relative contribution between standard CCSNe and faint CCSNe to the GCE of the Milky Way disk and of the solar neighbourhood is not known. We will return to this point in the next section. 

There are additional major uncertainties that need to be considered. According to basic stellar evolution principles, and considering the nuclear reactions involved, O and Mg are produced and ejected in the same CCSN layers and therefore ought to scale nearly perfectly to each other in their GCE history. Pre-supernova He-burning makes O via the $^{12}$C($\alpha$,$\gamma$)$^{16}$O reaction, together with a small amount of Mg. During C-fusion, O is left mostly unchanged whereas Mg is made efficiently in the form of $^{24}$Mg via the $^{20}$Ne($\alpha$,$\gamma$)$^{24}$Mg reaction. In the following evolutionary stage, O-fusion destroys both O and Mg. Therefore, the Mg/O ratio should be quite similar in the C-burning ashes of all CCSNe \citep[e.g., ][]{arnett:85,thielemann:85,chieffi:98}. Then again, explosive He-burning decouples O and Mg, where O feeds the production of Mg (and eventually Si and S) via a sequence of $\alpha$-capture reactions \citep[][]{pignatari:13}.  

In a similar way, the production of Si and S is generally expected to be connected, since these elements are produced together by the two main O-burning fusion channels in the form of their stable isotopes $^{28}$Si and $^{32}$S, respectively \citep[e.g.,][]{thielemann:85}. It is apparent that abundance profiles of Si and S, however, change significantly in the models shown in Figure~\ref{fig: ccsn_elements}. In the C-burning ashes of the 20 M$_{\odot}$ and 25 M$_{\odot}$ models, nuclear reactions have already started to make Si, while S is only marginally modified. In these CCSN explosions, it is evident that the C-ashes and eventually explosive He-burning products shape the S/Si ratio in the yields. In the 15M$_{\odot}$ model, we find instead that the ratio S/Si$<$1 typical of O-burning is shown only in the small region shaped by the explosive O-burning (M$\sim$1.6M$_{\odot}$). The C-ashes instead show a ratio S/Si$>$1, with both Si and S being more than 10$\%$ in mass fraction. This is the signature of the C-O shell merger, which occurs during the pre-SN evolution of the star and allows for the pollution of the C shell with O-burning products, with a signature quite different compared to pure O-burning material. 

The study of the interaction between the convective C-shell and O shell up to a complete C-O shell mergers was considered in previous nucleosynthesis studies \citep[][]{rauscher:02,ritter:18a,clarkson:18}. In these conditions, the predictive power of one-dimensional models is limited and multi-dimensional hydrodynamics simulations are required \citep[e.g.,][]{meakin:06, cristini:17, andrassy:20, clarkson:20}. The potential relevance of these events in triggering the asymmetries in the progenitor structure favouring successful CCSN explosions is also a matter of debate \citep[e.g.,][]{janka:17, ott:18}. 
For the purpose of our analysis, this implies that some scatter can be expected for the S/Si ratio in CCSN ejecta. 
The same scatter could be possibly visible in stellar observations, if observational uncertainties are small enough \citep[][]{chen:02,reddy:03,reddy:06}. Therefore, the common assumption made in forward simulations of planetary formation, where the initial Si and S abundances scale together with respect to the solar abundances \citep[e.g.,][]{bitsch:20} needs to be carefully checked against the spectroscopic data from the stellar host, or with GCE simulations, when S observations are not available. 
 
% Figure environment removed



The last stellar source we consider here are Supernovae Type-Ia (SNIa) which are responsible for the synthesis of the bulk of Fe measured in stars in the Milky Way disk, as well as a significant fraction of Si and S. Uncertainties surround the relative importance of the single-degenerate scenario (where the SNIa explosion is triggered by accretion of material on a CO-WD reaching the Chandrasekhar mass) with respect to the double-degenerate scenario (where the SNIa explosion is triggered by the merger of two CO-WDs) in the SNIa population of galaxies at present. \citep[][]{hillebrandt:13}. A complementary matter of discussion is the relative contributions of SNIa explosions from Chandrasekhar-mass progenitors and sub-Chandrasekhar-mass explosions, where also in the single-degenerate scenario a SNIa may explode before reaching 1.44M$_{\odot}$. To explain the GCE of the ratio [Mn/Fe]\footnote{With spectroscopic square-bracket notation we refer to the ratio of two elements X and Y, represented as the logarithm of the ratio relative to the same ratio in the Sun: [X/Y] = log$_{10}$(X/Y)$_{\rm star}$ $-$ log$_{10}$(X/Y)$_{\odot}$.}
%\footnote{With square-bracket notation, we refer to the Logarithm of the observed ratio normalized to solar.} 
in the MW disk, \cite{kobayashi:20a}, \cite{seitenzahl:13} and \cite{eitner:20} concluded that only 75$\%$, 50$\%$ and 25$\%$ of all the SNIa population should be from Chandrasekhar-mass progenitors, respectively. On the other hand, based on observational surveys of early-type galaxies \cite{woods:13} and \cite{johansson:16} determined that only a few per cent of all SNIa should be from Chandrasekhar-mass progenitors. These two conclusions are obviously at odds with one another and warrant further study. Finally, it is also important to note that from a study of 407 SNIa in older massive Red-Sequence galaxies and younger less massive Blue-Cloud galaxies, \cite{hakobyan:20} showed that about one-third of all SNIa events are peculiar, possibly related to the contribution from double-degenerate WD mergers, and that the diversity of SNIa progenitors may also be due to the age of the progenitor. Such a phenomenological diversity is difficult to capture within GCE simulations, but it will need to be considered in the future. 

Figure~\ref{fig: snia_prodfac} shows the abundances normalized to their solar values for the elements between Mg and Fe for SNIa models computed with the same initial $^{22}$Ne abundance equivalent to a metallicity of Z=0.014 \citep[][]{keegans:23}. The three models correspond to explosions of different masses of WD: 1.37 M$_{\odot}$ \citep[][]{townsley:16}, 1 M$_{\odot}$ \citep{shen:18} and 0.8 M$_{\odot}$ \citep{miles:19}. For the elements around Fe, the 1.37 M$_{\odot}$ and 1 M$_{\odot}$ progenitors produce very similar distributions, while the lowest mass progenitor produces far less of these in absolute abundances. For Fe itself, the production factor in the low-mass case is almost an order of magnitude lower than in the other two models. On the other hand, for the elements of interest in our discussion here, Si and S production factors are similar in the figure, and show only minor variations between the different models. This is because Si and S are typically produced in the same explosive conditions, and they are ejected together in the SNIa ejecta. This makes their production less sensitive to the relevant stellar uncertainties. %relative to Solar at the level of a factor of a few. 
Note that the contribution timescale (or delay-time) to GCE from different types of SNIa explosions may change depending on how the explosion was triggered in the stellar progenitors. Standard Chandrasekhar-mass SNIa accreting H will have a long delay-time \citep[in the order of 1 Gyr, see e.g.,][]{ruiter:09}. This would be the case for the models by \cite{townsley:16}, shown in Figure~\ref{fig: snia_prodfac}. Sub-Chandrasekhar-mass SNIa accreting He from a WD companion have a comparably long delay-time to the standard Chandrasekhar-mass SNIa \citep[e.g.,][]{gronow:21}. This would be the scenario compatible with the \cite{shen:18} and the \cite{miles:19} models shown in Figure~\ref{fig: snia_prodfac}. On the other hand, in case He would have been accreted by a He-burning star, the delay-time of Sub-Chandrasekhar-mass SNIa would be much shorter \citep[in the order of few hundred million years, see e.g.,][]{ruiter:14}. However, in the present context where different types of SNIa produce yields with similar S/Si abundance ratios, the GCE impact of varying the delay-time for different SNIa progenitors would be marginal.
%, and comparable with the timescale of a WD merger in the double-degenerate scenario (Maoz & Mannucci 2012).


%For the elements around Fe, the 1.37 M$_{\odot}$ and 1 M$_{\odot}$ progenitors produce very similar distributions, while the lowest mass progenitor produces far less of these in absolute abundances. For Fe itself, the production factor in the low-mass case is almost an order of magnitude lower than in the other two models.

%In Figure~\ref{fig: snia_el_ratio}, we show the variation in [Si/Fe]\footnote{Spectroscopic notation: the ratio of two elements X and Y is represented as the logarithm of the ratio relative to the same ratio in the Sun: [X/Y] = log$_{10}$(X/Y)$_{\rm star}$ $-$ log$_{10}$(X/Y)$_{\odot}$.} as a function of [S/Si] for the three progenitor masses, as the progenitor metallicity is increased from Z=0.002 to Z=0.05 (equivalent to $-0.84<$ [Fe/H] $<0.54$). 
%The [Si/Fe] ratio remains relatively constant for each of the three progenitor masses (to the level of about 5$\%$), though this ratio varies between 1.1 and -0.4 between the lowest and highest mass progenitors. Similarly, [S/Si] varies by only around 5$\%$ for each progenitor mass for the metallicity range considered. Also the variation between [S/Si] between the different progenitors is small: the low mass progenitor has [S/Si]$\approx$0.01, while for the two higher mass progenitors [S/Si]$\approx$0.07. For comparison, Figure~\ref{fig: snia_el_ratio} also shows the range in [Si/S] and [Si/Fe] for calculations from the set of models by \cite{seitenzahl:13a}, available in the Heidelberg Supernova Model Archive (HESMA). These show the variation in nucleosynthesis for a 1.4M$_{\odot}$ progenitor at Z=0.02 (plus one model computed at Z=0.01), computed under various configurations for the ignition geometry. Overall, the models by \cite{seitenzahl:13a} show a [S/Si] lower than solar, with their recommended configurations consistent with the  \cite{townsley:16} and \cite{shen:18} models. Nucleosynthesis calculations from these last models show instead up to 30\% larger S/Si ratio compared to the HESMA models.  


% Figure environment removed

%% Figure environment removed




\section{GCE Models and simulations}
\label{sec: tk_plots/models}

To explore the GCE of the solar neighbourhood we produced a set of 198 models; 
we use the \texttt{OMEGA} GCE code \citep[One-zone Model for the Evolution of GAlaxies][]{cote17} to calculate elemental abundance ratios in the ISM for several choices of stellar-yield sets and of the contribution from faint CCSNe yields. %, and of the mass-cut for CCSNe (the mass of the remnant of the massive star models). Particularly relevant for our case study,
\texttt{OMEGA} has also an option for including any number of extra enrichment sources, in addition to the mass- and metallicity-dependent yields from AGB stars, massive stars, CCSNe, and SNe Ia described above which are included by default in the code. The \texttt{OMEGA} code, the simple stellar population (SSP) model \texttt{SYGMA} \citep[Stellar Yields for Galactic Modeling Applications][]{rit18}, and the \texttt{STELLAB} (STELLar ABundances) module used to plot the observational data are all part of the publicly available NuGrid chemical evolution framework\footnote{\url{http://nugrid.github.io/NuPyCEE}}. Here we give a brief description of the \texttt{OMEGA} code. 

The evolution of the mass of the gas-reservoir in the Galaxy as a function of time can be expressed in terms of the gas inflow rate $\dot{M}_{\text{inflow}}(t)$, the rate at which gas is ejected from stars $\dot{M}_{\text{ej}}(t)$, the star-formation rate (SFR; $\dot{M}_{\star}(t)$), and the outflow rate of gas from the galaxy $\dot{M}_{\text{outflow}}(t)$ (e.g., \citealt{tinsley:80,gibson:03,matteucci:21}), where 

\begin{equation}
    \dot{M}_{\text{gas}}(t) = \dot{M}_{\text{inflow}}(t) + \dot{M}_{\text{ej}}(t) - \dot{M}_{\star}(t) - \dot{M}_{\text{outflow}}(t).
\end{equation}
%
The gas ejected by stars is assumed to mix instantaneously with the ISM, so that the metallicity of the gas-reservoir is always homogeneously distributed, as is the case for all one-zone GCE models. 
\texttt{OMEGA} also includes a simple treatment of galactic inflows and outflows. Since stellar feedback is assumed to drive the gas outflows, then

%In essence, outflows represent the removal of gas from the galaxy as a result of stellar feedback, whereas inflows introduce new metal-poor gas into the galaxy. 

\begin{equation}
    \dot{M}_{\text{outflow}}(t)=\eta \dot{M}_{\star}(t),
\end{equation}
%
where the mass-loading factor $\eta$ is a free parameter controlling the magnitude of the gas outflow (e.g., \citealt{murray:05,muratov:15}). The inflow of primordial matter is a catalyst for star-formation in the Galaxy, however \texttt{OMEGA} uses the star-formation history to determine the SFR rather than the inflow rate. For further details regarding the treatment of galactic inflows in \texttt{OMEGA} we refer the reader to \cite{cote17}.  

At each timestep, \texttt{OMEGA} creates a single stellar population (SSP) with a mass proportional to the SFR at that time, which is proportional to the total mass of gas in the Galaxy following the Kennicutt-Schmidt law \citep{schmidt59, kennicutt}:

\begin{equation}
    \dot{M}_{\star}(t)=f_\star \dot{M}_{\text{gas}}(t),
\end{equation}
%
where $f_{\star}=\epsilon_{\star}/\tau_{\star}$ is a combination of the dimensionless star-formation efficiency $\epsilon_{\star}$ and the star-formation timescale $\tau_{\star}$. All stars in a given SSP have the same initial metallicity - that of the gas-reservoir - since they are all assumed to have formed from the same parent gas cloud. At each timestep, the \texttt{SYGMA} code calculates the combined integrated yields from all the SSPs that are currently in the Galaxy. However, each star in an SSP will eject material at different times according to the delay-time distribution (DTD) function of the specific progenitor. For a galaxy with $j$ SSPs at time $t$, the combined integrated yield returned by \texttt{SYGMA} is given by

\begin{equation}
\dot{M}_{\mathrm{ej}}(t)=\sum_{j} \dot{M}_{\mathrm{ej}}\left(M_{j}, Z_{j}, t-t_{j}\right),
\end{equation}
%
where $Z_j$ and $M_j$ are the initial mass and metallicity of the $j^{\text{th}}$ SSP that was born at time $t_j$. Significantly for the present study, if an extra source (e.g., faint CCSNe) is added to \texttt{SYGMA}, then a DTD must also be assigned to it, in addition to specifying the yields, the number of events per M$_{\odot}$, and the mass ejected per event. 

Here, we assume that a fraction of massive stars $f_{\text{faint}}$ with initial mass $M$ above a given mass threshold $M_{\text{min}}$ will explode as faint CCSNe, rather than wholly as regular CCSNe. 
At the moment, it is still unclear if faint CCSNe have a significant impact on the GCE. Indeed, their presence may be hidden in the variations due to the uncertainties affecting the yields of CCSNe. Although previous GCE studies adopted various CCSN yields from the literature, they might have mitigated faint CCSN uncertainties by varying other parameters such as the slope of the IMF, the range of stellar masses contributing to the nucleosynthesis, the star-formation efficiency, and the strength of large-scale gas flows \citep[e.g.,][]{gibson:97,romano:10,molla:15,cote:17,philcox:18}. 
Given the lack of constraints for the value of $f_{\text{faint}}$, we consider values between 0 and 1 in order to fully explore the potential impact of faint CCSNe.



%In a one-zone GCE model such as \texttt{OMEGA}, 
Since both faint and regular CCSNe share the same type of progenitors, the only defining factor between the two types of explosion mechanisms are their yields. Therefore, for M>M$_{\text{min}}$ we make the simplification that faint and regular CCSNe occur at the same frequency in the Galaxy, but we apply a $f_{\text{faint}}$ correction factor to the yields of faint CCSNe, and a $1-f_{\text{faint}}$ correction factor to yields of the latter. We make no modifications to the yields of CCSNe that result from massive-star progenitors with M<$M_{\text{min}}$. The models considered for this work are summarized in Table~\ref{tab: list_tk_plots/models}. The adopted name scheme is o<yield$\_$identifier><faintSN$\_$model><faintSN$\_$weight>. The term $M_{\rm up}$ represents the mass of the most massive stars that can contribute to the GCE. All stars with an initial mass above $M_{\rm up}$ are assumed to directly collapse into a black hole without any ejecta.

With regards to stellar yields, for AGB stars the oK06, oR18, oR18d and oR18h sets use \cite{ritter:18}, while the oK10 and oL18 models use the yields by \cite{karakas:10}. 
The CCSN yields by \cite{kobayashi:06} are adopted in the oK06 and oK10 models, and non-rotating massive-star yields by \cite{limongi:18} are adopted in the oL18 models. The remaining sets of models use different yield setups by \cite{ritter:18}. In particular, oR18d, oR18 and oR18h use the same AGB stellar yields, but oR18d use the CCSN models with a delayed explosion setup \citep[][]{fryer:12}, oR18h is the same as oR18d but the 12M$_{\odot}$ yields are not included, while for oR18 CCSN models adopt a classical mass cut defined by the electron fraction ($Y_e$) jump in the progenitor structure \citep[][]{cote:17}. 
For all GCE models, the \texttt{OMEGA} default W7 SNIa yields by \cite{iwamoto:99} are used for all metallicities. Notice that within our GCE platform there are multiple sets of SNIa yields available \citep[e.g.,][]{lach:20, grunov:21}. However, since for the elements considered in this study the impact of using different SNIa yields is much smaller compared to CCSN yields, we decided to not modify the default setup for the models discussed here.  

For each GCE model setup described above, 
%with no yields correction considered, ten %twelve 
a number of models are generated with five %six 
faint CCSN weighting factors of 10$\%$, 25$\%$, 50$\%$, 75$\%$ %, 90$\%$ 
and 100$\%$ respectively, and two types of faint CCSNe: the stellar model of 20 M$_{\odot}$ by \cite{ritter:18} (model m20, Figure~\ref{fig: ccsn_elements}, central panel), and 25 M$_{\odot}$ (model m25, Figure~\ref{fig: ccsn_elements}, lower panel). The additional contribution of faint CCSNe is considered for stellar progenitor masses larger than $M_{\rm min}$ = 15M$_{\odot}$. The weighting factor mentioned above provides the relative contribution of faint CCSNe compared to default SNe yields for the mass range M > $M_{\rm min}$. 

Analogous GCE model sets are generated considering three different %upper limits of stellar masses contributing to the nucleosynthesis 
$M_{\rm up}$: 20 M$_{\odot}$, 40 M$_{\odot}$ and 100 M$_{\odot}$ as the value of M$_{\rm up}$ is uncertain and it is still a matter of debate. While $M_{\rm up}$ = 40 M$_{\odot}$ and 100 M$_{\odot}$ are more typical choices, we also considered in our calculations the lower value at 20M$_{\odot}$, which would be more consistent with observations from CCSN remnants and their progenitors \citep[e.g.,][]{smartt:15,davies:18}.
% MP: To be added, probably better to use 20Msun ad upper mass... 
Such a lack of CCSNe from massive stars with initial mass M $\gtrsim$ 20 M$_{\odot}$ also seems to be plausible for stellar simulations, where a relevant population of massive stars with initial mass larger than 20 M$_{\odot}$ may fail to explode \citep[e.g.,][]{sukhbold:16,fryer:18}, and it is indepently confirmed by direct element observations of late-time supernova spectra \citep[e.g.,][and references therein]{jerkstrand:14, jerkstrand:15, silverman:17}.

% for a total of 198 models: 3*(6+2*5*6) = 198 [Check if math is right]


% where the contribution from stars up to 100M$_{\odot}$ is considered. The frequency factor in Table~\ref{tab: list_tk_plots/models} provided the relative relevance of faint SNe compared to default SNe yields for the same mass range. 

%Faint SNe have been considered as potential sources of stellar observations from old metal-poor stars, like for the CEMP-no stars, i,e. Carbon Enhanced Metal Poor stars with no enrichment of heavy elements \citep[e.g.,][and references therein]{ishigaki:14, bonifacio:15, maeder:15, lee:19}.
%The most Fe-poor star known, SMSS J031300.36-670839.3, was proposed to carry the unique abundance signature of an Fe-poor faint SNe \citep[][]{keller:14, bessell:15, nordlander:17}. Recently, \cite{wehmeyer:19} discussed the contribution of faint SNe to explain the observed scatter of r-process enrichment with respect to iron in the early Galaxy. In their GCE simulations they assume as main r-process source neutron-star mergers and neutron star- black hole mergers, where black holes are considered as the remnants of faint SNe. On the other hand, as we mentioned before the contribution of faint SNe to GCE is not well defined. This applies also to the MW disk, 
%%of the MW disk is unknown, 
%although remnants of recent faint SNe explosions are observed \citep[e.g.,][]{nomoto:13}. 


%Finally, oK06cut and oR18cut models are based on the extreme assumption that all stars with mass larger than 20M$_{\odot}$ will not have successful explosions, with all the ejecta falling back to the central compact object and not contributing to GCE. 
 


\begin{table}
	\centering
	\caption{Summary of the GCE models used in this work with their properties: name (used in the text), stellar yields set identifier (see the details in the text), faint CCSN model if included, and its frequency with respect to the default yields. The extra sources "m20" and "m25" correspond to the M=20 M$_{\odot}$ and the M=25 M$_{\odot}$ faint CCSN models by \protect\cite{ritter:18}, respectively (Figure~\ref{fig: ccsn_elements}, central panel and lower panel). The last column provides the upper limit of stellar masses contributing to the GCE (in solar masses). The models name are designed as follows: o<yield$\_$identifier><faintSN$\_$model><faintSN$\_$weight>, where the initial o stands for \texttt{OMEGA}. 
}
	\label{tab: list_tk_plots/models}
	\begin{tabular}{lccc} 
		\hline
		yield$\_$identifier & faintSN$\_$model & faintSN$\_$weight  & M$_{\rm up}$ (M$_{\odot}$) \\
		\hline
		K06    &    -     &  no,                  & 20, 40, 100 \\
                       & m20, m25 &  f0p10, f0p25, f0p50, & 20, 40, 100 \\
                       & m20, m25 &  f0p75, f1p00  & 20, 40, 100 \\
		\hline
		K10    &    -     &  no,                  & 20, 40, 100 \\
                       & m20, m25 &  f0p10, f0p25, f0p50, & 20, 40, 100 \\
                       & m20, m25 &  f0p75, f1p00  & 20, 40, 100 \\
		\hline
		R18    &    -     &  no,                  & 20, 40, 100 \\
                       & m20, m25 &  f0p10, f0p25, f0p50, & 20, 40, 100 \\
                       & m20, m25 &  f0p75, f1p00  & 20, 40, 100 \\
		\hline
		R18d   &    -     &  no,                  & 20, 40, 100 \\
                       & m20, m25 &  f0p10, f0p25, f0p50, & 20, 40, 100 \\
                       & m20, m25 &  f0p75, f1p00  & 20, 40, 100 \\
		\hline
		R18h   &    -     &  no,                  & 20, 40, 100 \\
                       & m20, m25 &  f0p10, f0p25, f0p50, & 20, 40, 100 \\
                       & m20, m25 &  f0p75, f1p00  & 20, 40, 100 \\
		\hline  
		L18   &    -     &  no,                  & 20, 40, 100 \\
                       & m20, m25 &  f0p10, f0p25, f0p50, & 20, 40, 100 \\
                       & m20, m25 &  f0p75, f1p00  & 20, 40, 100 \\
        \hline
	\end{tabular}
\end{table}


That the chemical evolution of the solar neighbourhood is complex and a challenging task for GCE, is an understatement \citep[e.g.,][]{goswami:00, kobayashi:11, molla:15, prantzos:18, kobayashi:20, prantzos:23}. This is because stars that are observed within a few hundred parsecs from the Sun may have formed from material with radically different chemical evolution histories from one another and from our star. In fact, the observed [Fe/H] varies by about an order of magnitude and some stars may have formed after the Sun, or billions of years earlier and shortly after the formation of the Galaxy \citep[e.g., HD140283][]{siqueira-mello:15}.
This variety should be taken into account, 
because the relevance of different stellar sources varies during the galactic evolution timescale \citep[e.g.,][]{matteucci:86}. Stellar ages of nearby stars can be derived with a precision of about 1 billion years \citep[e.g.,][]{nissen:20}. We emphasize that the age of the star needs to be considered together with the stellar abundances in order to fully understand the elemental composition directly observed using spectroscopic data, and that can only be inferred \citep[e.g.,][]{spina:16}. 

As an example, Figure~\ref{fig: tk_plots/ok06no_xh} shows the evolution with [Fe/H] of the elements in model oK06no along with some reference evolution timescales. Model oK06no provides a good match to the solar abundances for the elements considered at the time when the Sun formed (8.7 Gyr). The predicted [Fe/H] is about 10$\%$ higher than solar. Carbon (mostly made by AGB stars) and O (mostly made by massive stars) are both about 60$\%$ too low. The elements N, Mg, and S are reproduced within 10$\%$ for solar material, while Si (made by both massive stars and SNIa) is about 20$\%$ higher. However, this same model, when considering both the chemical enrichment and evolution timescale, may not be appropriate to use for another star, even one with solar metallicity. 
We will use the same four reference GCE timescales shown in Figure~\ref{fig: tk_plots/ok06no_xh} (1.0, 3.1, 8.7, and 12.0 Gyr) in the next section, to also analyze the evolution curves of elemental ratios with respect to time. 

%[THIS CASE CLEARLY DEMONSTRATE THAT BY NORMALIZING RATIOS TO FIT THE SUN WE DO NOT LEARN ANYTHING FROM GCE. SINCE WE CARE ABOUT THE SUN HERE ONLY INDIRECTLY, I WOULD NOT PUT HERE THIS PHILOSOFICAL/METHODOLOGICAL DISCUSSION. ]


% Figure environment removed


\section{Stellar observation samples and solar abundances}
\label{sec: solar_wasp_nest}

Figure~\ref{fig: solar_ratios} shows the abundances from two different stellar samples from the galactic disk by \cite{reddy:03} and \cite{reddy:06} (R03, R06) and \cite{suarez-andres:18} (S18), together with their reference solar ratios. No scaling or normalization has been applied to the C/O and Mg/Si ratios. 
The two observed distributions exhibit clear discrepancies, where the S18 data have on average both higher C/O and Mg/Si ratios. 
This difference also appears in the solar abundances used in these surveys, with the S18 C/O and Mg/Si being factors of 1.38 and 1.29 higher than those of R03 and R06, respectively.
%We also plot in 
Figure~\ref{fig: solar_ratios} also plots C/O and Mg/Si ratios from several other solar chemical composition studies. 
The C/O values range by a factor of two, from 0.83 \citep[][D10 HARPS]{delgadomena:10} to 0.43 \citep[][AG89]{anders:89}. 
The Mg/Si ratios are also scattered, varying between 0.83 \citep[][based on their own solar analysis]{reddy:03}, and 1.23 \citep[][A09]{asplund:09}.
These variations indicate that solar abundance differences are not limited to the two stellar surveys employed in our investigation.
%This paper's primary purpose is to compute GCE models and to compare them with observed abundances.  
The details of these abundance determinations can be found in the survey papers. We report below some general remarks about observational uncertainties.


\subsection{Comparing Results from the Abundance Surveys Considered Here}
\label{sec: observed_abundance_details}


% Figure environment removed


Some of the apparent clashes in Figure~\ref{fig: solar_ratios} between the C/O and Mg/Si ratios of R03, R06 and S18 can be alleviated by a normalisation to the solar abundances derived with the same analysis setup.
In this way, it is possible to remove at least some of systematic uncertainties. 
In Figure~\ref{fig: ratios_s18_r03}, we report the same stellar ratios shown in Figure~\ref{fig: solar_ratios}, but in logarithmic notation and normalized to their respective (and different) solar reference ratios. 
Because of these, the two stellar samples show a much better overlap compared to Figure~\ref{fig: solar_ratios}. 
The [C/O] ranges are similar, and the [Mg/Si] is also consistent (albeit with significantly more scatter), when excluding outliers with [C/O] $\lesssim$ $-$0.5 and [Mg/Si]$\lesssim$ $-$0.2. 
The set by S18 is concentrated around the solar values or slightly higher, while R03 and R06 data are more scattered toward larger Mg/Si values, up to about 0.2 dex.
The larger ranges of both C/O and Mg/Si in the R03 and R06 sample combined reveals that the two surveys may not draw their targets from the same Galactic metallicity/kinematic samples. Indeed, from Figure~\ref{fig: ratios_s18_r03} we can see that the high Mg/Si-signature is mostly given by R06 stars, which are mostly thick-disk stars. The R03 stars are instead in better agreement with the S18 scatter.

% Figure environment removed

The abundance surveys considered here were conducted using similar methods.
They both used model atmospheres from the ATLAS grid \citep{kurucz:11,kurucz:18} and performed equivalent width and synthetic spectrum analyses with the current versions of the same LTE plane-parallel code \citep{sneden:73}. On the other hand, R03, R06 and S18 surveys have different selection functions. The HARPS S18 sample is a subset of $\sim$500 stars from the HARPS \cite{adibekyan:12} sample, who chose their objects based on suitability for radial velocity surveys (slowly rotating FGK stars
without chromospheric activity); the R06 sample was primarily selected from kinematically thick Galactic disk but within a given distance and R03 mostly includes selected stars from the kinematically thin Galactic disk.
Different software was employed to measure equivalent widths, but the basic procedure is straightforward and accurate for unblended spectral lines.
The two surveys differ either in the choice of species for study or in the selection of the individual spectral features.  
Here we briefly comment on each element.
The cited papers describe the details of the analyses and discuss the uncertainties.

\textit{Carbon:}
R03 and R06 employed six high-excitation C~{\sc i} lines, using transition probabilities in agreement with those currently recommended by the curated NIST Atomic Spectra Database\footnote{https://www.nist.gov/pml/atomic-spectra-database}.
S18 adopted C abundances derived from the CH molecular G-band by \cite{suarez-andres:17}.
Both C~{\sc i} and CH can yield reliable C abundances in solar-type stars, but the response of these two species to variations in atmospheric temperature and gravity parameters is dissimilar. This could justify the existence of relevant differences in the derived abundances when using these two observational sources.
For additional comments on transitions and references to solar C studies see \cite{asplund:21}. Note that \cite{delgado-mena:21} re-derived the C abundance for the HARPS sample (S18) using atomic lines C~{\sc i} (like R03 and R06): they derived typically larger C abundances compared to S18, especially for cool and metal-poor stars. Therefore, the same [C/O] variation seen in e.g., Figure~\ref{fig: solar_ratios} for S18 with respect to R03 and R06 would have been comparable or larger if we would have used \cite{delgado-mena:21} data instead of S18.

\textit{Oxygen:} No molecular species are available in the optical spectral regime, and there are very few detectable O~{\sc{i}} lines.
R03 and R06 derived O abundances exclusively from the high-excitation O~I 7700~\AA\ triplet lines.
S18 adopted instead their abundances from \cite{bertrandelis:15}, who employed a unique combination of the [O~{\sc i}] 6300~\AA\ ground-state line and a high-excitation O~{\sc i} line at 6158~\AA.
The high-excitation lines of this species have long been known to exhibit departures from local thermodynamic equilibrium \citep[][]{caffau:08,asplund:21}.  
The 6300~\AA [O~{\sc i}] line is very weak in solar-type stars, and is significantly blended with a Ni~{\sc i} transition \citep{allendeprieto:01}.
Such concerns, along with different transition choices in the two surveys, serve as cautionary notes.

\textit{Magnesium:}
Mg~{\sc i} lines are the only reliable Mg abundance sources.
MgH lines are detectable near 5000~\AA, but they are weak and very blended with strong atomic lines and C$_2$ molecular features.
There are relatively few available Mg~{\sc i} transitions, and those well known ones are often very strong.  
Many of the usually employed lines (4730.00~\AA, 4730.30~\AA, the Fraunhofer "b" triplet, 5528~\AA, and 5711.10~\AA) are saturated in the solar spectrum:  log($EW/\lambda$) > -4.8 \citep[][]{moore:66}.
Therefore, the derived Mg abundances depend on atomic damping parameters and on the adopted atmosphere conditions in the outer photospheric model.
%knowledge of outer-photospheric model atmosphere conditions.  
R03 and R06 selected three Mg~{\sc i} lines, two of which are weak enough to be relatively sensitive to Mg abundances.
S18 adopted the abundances from \cite{adibekyan:12}, who in turn used the line lists of \cite{neves:09} for their study of three Mg~{\sc i} lines, with just one of them being in common with R03 and R06.
The log($gf$) values are generally in accord with the values recommended by NIST, however we note that the NIST laboratory sources are decades old and would benefit from modern re-analysis.  
Finally, a carefully developed line list from 4750-8950~\AA~has been constructed by the Gaia-ESO consortium \citep{heiter:21}.  
Their transition probabilities for Mg~{\sc i} are in accord with those used the the two surveys of interest here.

\textit{Silicon:}
A rich Si~{\sc i} spectrum is available in the optical spectra of solar-type stars, but transition probabilities have not been subjected to comprehensive laboratory analyses in recent decades.  
The R03, R06 and S18 (again, based on the earlier papers by \cite{adibekyan:12} and \cite{neves:09}) used 7 and 18 lines; their log($gf$) scales agree reasonably well within %<Reddy0306$-$Neves12> = 
0.0 $\pm$ 0.07~dex, for the 5 lines in common. However, the line-to-line scatter between \cite{neves:09} and NIST (0.21~dex) and between \cite{neves:09} and \cite{heiter:21} (0.14 dex) would be eminently more useful if it was not so large, and deserves to be re-investigated.



Our brief summary of line list issues in the two surveys should be viewed as illustrative; such questions need to be kept in mind for all abundance data sets.
Another fundamental problem that needs to be addressed in the near future is the lack of recent comprehensive investigations by the atomic physics community of transition probabilities.  
With current data it is reasonable to hope for survey-to-survey agreement to the $\simeq$0.05~dex level.
Deriving abundance uncertainties to $\simeq$0.05~dex remains a future goal.


\subsection{Brief Comments on other Surveys}
\label{sec: observed_other}

We have concentrated on the R03, R06 and S18 studies because they have extensive abundance data on all four elements of interest for understanding gross planetary characteristics, and used similar analytical procedures.
Other groups have made significant contributions to Galactic disk abundance surveys.
The $\alpha$ elements as well as C and O have been studied in various surveys at different spectral resolutions in different spectral regions, such as by GALAH (e.g,, \citealt{clark:22}, \citealt{sharma:22}), and Gaia-ESO \citep[e.g.,][]{kordopatis:15}.
%(e.g, \citealt{kordopatis:15}.
Here we call attention to a noteworthy contribution by T. Bensby and collaborators.
Figure~\ref{fig:mgsi2} shows the [Mg/Si] ratios versus [Fe/H] for our main surveys and the 714 star sample of \cite{bensby:14}. These authors used extensive line lists of Mg~{\sc i} and Si~{\sc i}, and transition probabilities from laboratory work and reverse solar analyses, as discussed in \cite{bensby:03}.
%Different symbols in the figure panels are used to denote thin disk, thick disk, and halo populations for stars as defined in the individual papers.
Inspection of Figure~\ref{fig:mgsi2} reveals a drift to larger [Mg/Si] values with decreasing [Fe/H]. %in each survey.
The addition of the \cite{bensby:14} sample highlights this trend, which is weaker in R03, R06 and S18.
Note that it appears to be independent of Galactic thin-disk, thick-disk, and halo-population memberships.

% Figure environment removed

\cite{bensby:14} reported O but not C abundances, so their data could not be added into Figure~\ref{fig: solar_ratios}.
It is worth noting that \cite{bensby:06} derived [C/O] ratios from a unique forbidden-line combination:  [C~{\sc i}] at 8727~\AA\ and [O~{\sc i}] 6300~\AA\ transitions.
In general, their [C/O] ratios are $\simeq$0.15~dex higher than those of R03, R06 and S18.  
The \citeauthor{bensby:06} sample size is relatively small compared to the other samples discussed here, $\simeq$50 stars, and it overlaps with the \cite{bensby:14} list only for a few stars. %, so few general insights can be gained from this study.
In general, a large sample investigation of [C/Fe] ratios from the combination of CH, C$_2$, C~{\sc i}, [C~{\sc i}], and possibly CO in unevolved disk stars would be an important anchor for %on 
all future C abundance studies.



\subsection{Astrophysical Implications of C/O and Mg/Si in Local Samples}
\label{sec: other_surveys}

The stellar samples considered in Figure \ref{fig: solar_ratios} and Figure~\ref{fig:mgsi2} show significant star-to-star scatter in [C/O] and [Mg/Si], $\sigma$~$\simeq$~0.2~dex.
As there are systematic dependencies on [Fe/H] and stellar populations that lie beyond observational uncertainties, the origin of the internal dispersion is a matter of debate.
Some of the scatter must be observational, as outlined in \ref{sec: observed_abundance_details}  
but some of this effect may be an intrinsic property of stars in the solar neighbourhood, which is difficult to quantify with the present observational errors. 
While, e.g., R03, R06, \cite{bond:10} and S18 find a significant dispersion of stellar abundances, \cite{bedell:18} obtain a consistency within 10\% for Sun-like stars (the stars in their sample are solar twins with very similar metallicities) within 100 pc. 
It will be paramount in future work to definitively solve these discrepancies and clarify the diversity in composition of the solar neighbourhood.

In section \ref{sec: results}, we will compare GCE simulations with solar-scaled observations, so that the systematic uncertainties discussed here will not directly affect our analysis. However, since the absolute elemental abundances are needed for the simulations of planetary systems, we are highlighting the uncertainties discussed in this section as an issue that will need to be addressed. The variations seen in Figure \ref{fig: solar_ratios} are a clear source of degeneracy for future planet formation and evolution studies. 

For the comparison with GCE simulations, we will only use stellar data by R03 and R06, where all the elemental abundances are provided for %all 
the elements discussed in the following sections (i.e., C, N, O, Mg, Si and S). Additionally, this allows us to compare GCE simulations with the abundance dispersion observed by R03 and R06, which is larger and more conservative compared to the results by, e.g., \cite{bedell:18}. We have seen that R06 includes indeed a large number of thick-disk stars.
While we can expect that the planet-host stellar samples from e.g., TESS and ARIEL will be biased towards more metal-rich, thin-disk stars, the first observational efforts to homogeneously characterize the physical parameters of planet-host stars in the Ariel Reference Sample \citep[][]{edwards:19} show the presence of both thick-disk stars and stars that could dynamically belong to either the thin and thick disks \citep[][]{magrini:22}.  Furthermore, it is expected that a number of thick-disk stars will be discovered hosting planetary systems (e.g., from TESS Kepler-444, \citealt{campante:15}, and TOI-561, \citealt{weiss:21}). The study of these stars and their planetary systems will be a fundamental benchmark for planet formation and the evolution of planetary systems.


Finally, we note that among the elements considered in this work N is not available in the HARPS data used by S18 and \cite{bedell:18}. However, for a sample of 74 stars \cite{suarez-andres:16} also combined HARPS data for relevant elements to derive the N abundance from the NH band in UVES spectra.
 %Unfortunately, S is not available in the other set \citep[][and references therein]{suarez-andres:18}. 
 
%\subsection{Additional elements relevant for this work: N and S and their observational uncertainties}
%\label{sec: obs_N_and_S}
% 
%As mentioned in the Introduction, in addition to C, O, Mg and Si, the observation of N and S abundances provide crucial diagnostics for planet formation and evolution. In particular, we have seen that N and S are relevant to study the formation of giant planets. Both N and S are measured by \cite{reddy:03} and \cite{reddy:06}. Their main uncertainties are summarized here below. 


\section{Results of the GCE simulations}
\label{sec: results}

We now summarize our analysis for the elements of interest discussed in Section \S~\ref{sec: stars}, using the GCE models introduced in the previous section. All the plots showing the comparison between GCE model predictions and observations are provided in this section or in Appendix \S~\ref{app_1: plots extra}. 


In Figure~\ref{fig: tk_plots/ratios_mup40}, selected element ratios are plotted against each other and compared to abundances observed in stars within 150 parsecs by R03 (which comprises mostly thin-disk stars) and R06 (which instead comprises mostly thick-disk stars). 
This is a similar range of distance to consider for stellar hosts of TESS and ARIEL planetary targets \citep[][]{edwards:19,magrini:22}. Therefore, we may assume that
this observational sample is consistent with the abundance variations that merit exploration.   

The upper-left panel of Figure~\ref{fig: tk_plots/ratios_mup40} is the same diagram shown by \cite{bond:10} but in logarithmic notation, and including GCE models. In general, only the theoretical GCE curves between the red and the magenta stars (which correspond to a GCE evolution age between 1 Gyr and today, Figure~\ref{fig: tk_plots/ok06no_xh}), should be considered as representative of the evolution over time of the Milky Way disk. Indeed, the observed properties of the old stellar population of the Milky Way halo are consistent with the first Gyr of active star formation, while to reproduce the age and metallicity distribution of the stars in the Milky Way disk much longer times are required (see e.g, \citealt{fenner:03}). 

The observational scatter of about a factor of 2.5 is obtained for both [C/O] and [Mg/Si]. At time of Sun formation, the models oK06no and oK10no produce [C/O] ratios 0.2 dex and 0.3 dex lower than the solar abundances, respectively. However, they reproduce the ratio observed in the majority of stars, with a [C/O] ratio increasing over time (or with metallicity) until about 5 Gyr ago \citep[][]{bitsch:20}. After Sun formation, the calculated [C/O] is almost constant. The oR18dno model produces a final solar [C/O] ratio. Between the three GCE models using \cite{ritter:18} CCSN sets, the different CCSN explosion parametrizations affect the O production with a variation of the [C/O] ratio by about 0.2 dex. At evolution timescales representative of the MW disk, oL18 models show a solar [C/O] ratio with only marginal variation. %dispersion. 

The [Mg/Si] is about 30$\%$ lower than the Sun in oK06no and oK10no, between a factor of 1.8 and 2.2 lower for the R18 models, and sub-solar by a factor of $\sim1.8$ for oL18no. While the bulk of the stars in the solar neighbourhood have a solar-like or super-solar ratio up to [Mg/Si]$\sim$0.2, the GCE models predict a ratio lower than solar for all the combinations of stellar yields considered. In particular, the largest departure seen in oR18no is mostly due to the contribution of energetic CCSN explosions for the 12M$_{\odot}$ and 15M$_{\odot}$ models by \cite{ritter:18}. Note that these results would not have changed by considering a different observational stellar sample, e.g. R03 only without the R06 or the S18 sample (see Figure~\ref{fig: ratios_s18_r03}).
We also cannot expect one-zone GCE simulations to reproduce the observed [Mg/Si] scatter, since at a given evolution time of the model the result is given by a single data point, not by some statistical distribution. However, the predictions should still be compatible with the bulk of the observed stars. Even by varying the stellar yields - one of the main uncertainty sources of GCE - we do not achieve this result. It is true that GCE uncertainties could play a relevant role in the abundance analysis, and in our single-zone GCE simulations we do not take into account relevant processes like stellar migration and past infall of fresh material in the galactic disk \citep[][and reference therein]{matteucci:21,prantzos:23}. However, the impact of such processes should be significantly reduced by studying the evolution of primary elements sharing a similar stellar production. For instance, in the specific case of Mg and Si they are both mostly produced by short-lived massive stars, and these do not have sufficient time to migrate significantly \citep[e.g.,][]{sanchez-blazquez:09, minchev:14}. 

Such a result where Mg stellar yields seem to be too low compared to observations (and, to a much lesser extent, Si) are not surprising, and they have been highlighted before in the literature. The artificial Mg and/or Si boosting is a well-known requirement of using e.g., the \cite{woosley:95} CCSN yields \citep[][]{gibson:97}. More recently, the same approach is implemented by \cite{spitoni:21} with \cite{woosley:95} yields, where Mg from CCSNe are boosted up to a factor of seven.

The distribution of the CNO elements is shown in the upper-right panel of Figure~\ref{fig: tk_plots/ratios_mup40}. Models oR18no and OR18dno show an [N/O] ratio increasing with the evolution time and with [Fe/H]. Yet, we find that they both reach the solar [N/O] ratio too early, more than 4 Gyr before the formation of the Sun. The other models instead reproduce the solar ratio to within 0.1 dex. For oL18no, the [N/O] ratio changes little with galactic time, remaining $40-80\%$ super-solar for the duration of the GCE. %oK10no shows an [N/O] similar to the other models at the end of the simulations (about 60$\%$ higher than solar), but the same values do not change significantly in the part of the simulations relevant for the MW disk. Most common [N/O] ratios are obtained too early, at much lower metallicities. 
Several challenges need to be considered for the GCE of CNO elements and their stable isotopes \citep[e.g.,][]{kobayashi:20}. The contingent relevance of fast rotating stars (not included in our models) was highlighted by several previous works to reproduce the abundance patterns in the Milky Way \citep[e.g.,][]{chiappini:06, chiappini:08, prantzos:18, romano:19}. Additionally, \cite{pignatari:15} discussed the contribution of H-ingestion events in massive stars for the GCE of N (and in particular of the N isotopic ratio), using stellar models consistent with abundance measurements in presolar grains made by CCSNe just before the formation of the Sun. 

The predictions for N evolution from our GCE models have problems in reproducing the ratios in the Figure~\ref{fig: tk_plots/ratios_mup40}, lower-left panel. Contrary to most of the observations, all the GCE models produce super-solar [S/N] ratios, except for the oL18no model, which produces a [S/N] ratio significantly lower than solar, but still does not cover the full observational range down to [S/N]$\sim$-0.5 reached by a large number of stars. Most of the stars indeed exhibit a subsolar elemental ratio, with a scatter of the order of a factor of three. 
The observed [C/N] scatter is instead at least partially reproduced by most of GCE models, where this ratio decreases with evolution time. Note that this does not have to be the right physical reason for the observed [C/N] scatter. As we mentioned earlier, the N evolution is a well-known challenge for GCE, where standard CCSN models underproduce N compared to observations. 
Finally, model oK10no shows a smaller variation than the other models within sub-solar [C/N] values around $-$0.3 dex.

In the lower-right panel of Figure~\ref{fig: tk_plots/ratios_mup40}, stars show an observational scatter larger than a factor of two for both [Mg/O] and [S/Si] ratios. Such a variation is only marginally captured by the GCE models: for both ratios, variations at relevant timescale are in the order of 20$\%$ or less. The oK06no, oK10no and oL18no models reproduce the solar ratios within 0.1 dex, the other models are more consistent with the bulk of stars that are mildly S-rich compared to the Sun, up to [S/Si]$\sim$+0.3. If we consider all the GCE models, it may seem that the observed range of [S/Si] is reproduced. However, if we factor in single GCE models we notice that the [S/Si] variation within the acceptable evolution time frame is less than 0.1 dex. The only exception is oR18hno, where between 1Gyr %(orange star) 
and today the [S/Si] ratio increases by about 0.2 dex. Still, such an increase is much smaller than the scatter observed in the Milky Way disk. Such a dispersion may, in part, be due to observational uncertainties \citep[][]{bedell:18}. Chemo-dynamical simulations of the Milky Way disk would be needed to provide a realistic prediction for the expected [S/Si] dispersion \citep[e.g.,][]{thompson:18}, which is beyond the goal of this paper.   

Alternatively, as we discussed in Section \S~\ref{sec: stars}, this dispersion may instead be an indication that CCSN ejecta are not always dominated by an explosive O-burning signature, but that they vary between different supernova events. We have seen that some variations are obtained between different SNIa explosions (Figure~\ref{fig: snia_prodfac}), although the progenitor mass does not affect the S/Si ratio in the ejecta very much, and the same can be said for the initial metallicity \citep[e.g.,][]{keegans:23}. Note that according to the nuclear sensitivity study by \cite{parikh:13}, there should be no relevant impact of nuclear uncertainties on the SNIa yields of Si and S.



% Figure environment removed

% Figure environment removed


% Figure environment removed

\subsection{The effect of changing M$_{\rm up}$ in GCE simulations}
\label{subsec: mup_change}


The models shown in Figure \ref{fig: tk_plots/ratios_mup40} used as mass upper limit M$_{\rm up}$ = 40M$_{\odot}$ (Table~\ref{tab: list_tk_plots/models}).
In Figure~\ref{fig: tk_plots/ratios_mup20} and Figure~\ref{fig: tk_plots/ratios_mup100} we have explored the impact of the M$_{\rm up}$ parameter space on the results, by considering M$_{\rm up}$ = 20 M$_{\odot}$ and M$_{\rm up}$ = 100 M$_{\odot}$, respectively.
The upper-left panels of the two figures show that by increasing M$_{\rm up}$, the predicted [Mg/Si] increases by up to 0.15 dex, but [C/O] decreases by up to 0.1 dex. For instance, with M$_{\rm up}$= 100 M$_{\odot}$, the models oK06no and oK10no approach the solar [Mg/Si] ratio (the super-solar ratios observed are still not reproduced). Rather, the predicted [C/O] is about 0.3-0.4 dex lower than solar. 
On the other hand, with the extreme case M$_{\rm up}$= 20 M$_{\odot}$, oK06no and oK10no predict a [Mg/Si]$\sim$-0.2 for the Sun, and for oL18no and all the oR18 models the same ratio is reduced down to about [Mg/Si]$\sim$-0.4. % (oR18dno).
If we compare the upper-right panels, for models oK06no and oK10no the [N/O] typically decreases by 0.6 dex with increasing M$_{\rm up}$, while there is mostly no effect in all the R18 models. 
This is because in our GCE models the total mass ejected for the higher masses is extrapolated from the list of available models, and the yields abundance pattern is kept identical to that of the highest mass model, which is the M=25M$_{\odot}$ progenitor for \citeauthor{ritter:18} and the M=40M$_{\odot}$ progenitor for \citeauthor{kobayashi:06} yields. The 25M$_{\odot}$ models by \citeauthor{ritter:18} are all weak explosions, leaving large remnants. Thus, stars between 25 and 100 M$_{\odot}$ only eject limited amounts of elements such as O and Mg. Notice that since L18 yields have large remnants for stars above 30 M$_{\odot}$, the M$_{\rm up}$ impact will be limited in these cases too. On the other hand, the yields by \cite{kobayashi:06} are all made of successful CCSN explosions with low remnant masses, and increasing M$_{\rm up}$ of 20 to 100 M$_{\odot}$ makes a huge difference since more massive progenitors do contribute significantly to the chemical evolution. 

The lower-left panel of Figure~\ref{fig: tk_plots/ratios_mup20} shows that with M$_{\rm up}$ =20 M$_{\odot}$ the model oK10no can reach a sub-solar [S/N]$\sim$$-$0.2 dex, which would be consistent with the bulk of local stars, but with a [C/N]$\sim$-0.3 dex. The model oL18no can also reach a sub-solar [S/N]$\sim$$-$0.3 dex, with a [C/N]$\sim$-0.2 dex. All the other models exhibit a [S/N] range between solar and 2.5 times solar (oR18hno). With M$_{\rm up}$ = 100M$_{\odot}$ (Figure~\ref{fig: tk_plots/ratios_mup100}), the models closest to the observations are oR18no and oL18no with predicted [S/N] and [C/N] ratios in the range of $-$0.1 $-$ $-$0.2 dex, since the time of the formation of the Sun. Finally, if we compare the bottom right panels of Figure~\ref{fig: tk_plots/ratios_mup20} and Figure~\ref{fig: tk_plots/ratios_mup100}, the only significant variation is a decrease of the [Mg/O] ratio of about 0.1 dex or less for all models with increasing M$_{\rm up}$. Such a small variation is not surprising. Indeed, both O and Mg are mainly products of massive stars, they are made during the pre-SN stage and their pre-SN ratio is not significantly modified by the CCSN explosion \citep[e.g.,][]{thielemann:96, pignatari:16}. 

In summary, from exploring the impact of the M$_{\rm up}$ parameter space we do not see a clear effect of using a value different from the default M$_{\rm up}$ =40 M$_{\odot}$. While a higher M$_{\rm up}$ slightly increases the [Mg/Si] ratio, it would still not cover the Sun and most of the stars, %full observed range, 
with the [C/O] ratio too low as compared to observations. The impact on the [N/O] and [S/N] ratios is model dependent. Therefore, in the following part of the section we will discuss only the models with M$_{\rm up}$ = 40M$_{\odot}$. Results for the same models but with different M$_{\rm up}$ are available in Appendix~\S~\ref{app_1: plots extra}.  
  

\subsection{The impact of Faint Supernovae}
\label{subsec: fainSN}

To study the impact of faint CCSNe we focus our analysis to the two set of models using the yields K10 and R18 (Table~\ref{tab: list_tk_plots/models}). Results of other models are consistent with the sample of simulations considered here and are available in Appendix~\S~\ref{app_1: plots extra}. 



% Figure environment removed


% Figure environment removed

Figure~\ref{fig: tk_plots/ratios_gce_oK10m40} reports the impact of faint CCSNe for the oK10 models. The full parameter space is considered, with the frequency of faint CCSNe from 0\% (which would correspond to the oK10no model shown in Figure~\ref{fig: tk_plots/ratios_mup40}) to 100\% (models oK10<faintSN$\_$model>f1p00). As representative of faint CCSNe models, we used the 20 M$_{\odot}$ (m20) and 25 M$_{\odot}$ (m25) CCSN models by \cite{ritter:18} shown in Figure~\ref{fig: ccsn_elements} (models oK10m20<faintSN$\_$weight> and oK10m25<faintSN$\_$weight>, respectively). As seen in Section~\S~\ref{sec: stars}, the m20 model still ejects some material carrying the signature of O-burning, while there is no Fe-rich Si-burning ejecta. The m25 model does not eject products of either Si-burning or O-burning. Note that considering present uncertainties in CCSN explosion and the wide zoo of CCSN remnants presently observed, we may expect the real fraction of faint CCSNe to be somewhere in between the two extreme cases, oK10no and oK10<faintSN$\_$model>f1p00. In the left panels of Figure~\ref{fig: tk_plots/ratios_gce_oK10m40}, no substantial effect is observed from considering faint CCSNe. The first reason is that CNO elements are not substantially affected by the CCSN explosion as they are mostly produced during stellar evolution before core collapse. Therefore, their relative abundances do not change significantly in faint CCSNe, as compared to standard CCSNe. The second reason is that although Si and S are O-burning products, the m20 faint CCSNe model used in these tests still eject some Si-rich and S-rich material, without affecting the [S/N] and the [S/Si] ratios much. 

The impact of faint CCSNe becomes more relevant in the GCE models where m25 is used. In the upper-right panel of Figure~\ref{fig: tk_plots/ratios_gce_oK10m40}, [C/O] and [Mg/Si] increase by about 0.2 dex and 0.1 dex respectively, when comparing oK10m25f0p50 to oK10no. Models with higher faint CCSNe frequency oK10m25f0p75, oK10m25f0p90 and oK10m25f1p00 have even larger increases, up to solar ratios. We would, however, consider these last models of the parametric study as less realistic. The reason for the impact on [C/O] is that a significant part of the former O-rich C shell is not ejected by m25 (Figure~\ref{fig: ccsn_elements}). Therefore, the overall galactic enrichment is driven to higher C/O-ratios. The impact on [Mg/Si] is instead smaller, since, as with O, Mg-rich material is not fully ejected by this model. A faint CCSN model with a lower masscut allows for more O and Mg ejection and still no O-burning products like Si. This is still realistic to consider within the uncertainties \citep[e.g.,][]{fryer:18} and would, in principle, achieve larger [Mg/Si]. The impact on the [C/O] would be small. 
In the second from top right panel of Figure~\ref{fig: tk_plots/ratios_gce_oK10m40}, [N/O] increases with increasing the faint CCSNe frequency, up to +0.35. The effect on the [N/O] ratio is the same as discussed for the [C/O] ratio, where N is mostly made in the most external layers of CCSNe. 
The model oK10m25f0p50 shows a reduction of the [S/N] ratio down to $-$0.15, providing a possible explanation of the observation range. As we mentioned in the previous section, however, the nucleosynthesis of N in CCSNe may be affected by physics not considered in this work, like stellar rotation or H-ingestion events, which would both increase the N yields as compared to S. Finally, also in this case there are only minor effects on the [Mg/O] and [S/Si] ratios.
 
Figure~\ref{fig: tk_plots/ratios_oR18_m40} presents the equivalent oR18 models. Qualitatively, there are similar effects using the m20 and m25 faint CCSNe as debated above for the oK10 models, with overall a more significant impact in the oR18m25<faintSN$\_$weight> models. The top panels confirm the increasing trend of [Mg/Si] with the faint SN frequency, with a ratio higher than about 0.1 dex in the oR18m20f0p50 and oR18m25f0p50 with respect to oR18no.
Instead, the [C/O] increase is more limited, as compared to the oK10 models, and more dependent on the evolution time of the model. The evolution of the [N/O] ratio in the oR18 models is also different compared to the effect seen in the oK10 models. While the final abundances vary by less than 0.1 dex between oR18m20no and oR18m20f1p00 and are mostly unaffected when using the m25 faint CCSN model, the ratio starts to increase at much earlier times, with the [N/O] ratio higher up to 0.4 dex. The impact on the final [S/N] in the oR18 models including faint CCSNe is less than 0.1 dex, while it was more significant in oK10 models using an m25 faint CCSN. Finally, like for the oK10 models, there is no effect on the [S/Si] ratio. There is instead an increase of the [Mg/O] ratio, by less than 0.1 dex, between the oR18m20f0p50 and oR18m25f0p50 with respect to oR18no, with an increase up to 0.2 dex reaching the solar ratio for oR18m20f1p00 and oR18m25f1p00. 
%o<yield$\_$identifier><faintSN$\_$model><faintSN$\_$weight>
%no, f0p10, f0p25, f0p50, f0p75, f0p90, f1p0
%\section{Results -- OLD VERSION --}
%\label{sec: results}

%By using the GCE models introduced in the previous section, we can now focus our analysis on the elements of interest discussed in \S~\ref{sec: stars}. 

%In Figure~\ref{fig: tk_plots/ratios_no}, relevant element ratios are shown compared to abundances observed in stars within 150 parsecs by \cite{reddy:03} and \cite{reddy:06} (see discussion in the previous section) [150 PC FOR REDDY2006. cHECK FOR REDDY2003: THERE ARE PARALLAXES SO IT MUST BE LOCAL, BUT I COULD NOT FIND DISTANCE RANGE CLEARLY MENTIONED. TO BE CHECKED!]. We may consider this observational sample as indicative of the variations that will be found in the stellar hosts of TESS and ARIEL planetary targets \citep[][]{edwards:19}.   
%The upper left panel of Figure~\ref{fig: tk_plots/ratios_no} is the same diagram shown by \cite{bond:10} in logarithmic notation, where GCE models are also reported. Only the theoretical curves between the orange and the magenta stars should be considered as generically representative of the evolution of the MW disk. The observational scatter of about a factor of 2.5 is obtained for both [C/O] and [Mg/Si]. The models oK06no and oK10no well reproduce the solar abundances and the observed range, with a [C/O] ratio increasing over time (or with metallicity) until about 5 Gy ago \citep[][]{bitsch:20}. After the Sun formation, the calculated [C/O] is almost constant.
%On the other hand, the [Mg/Si] is about 30$\%$ lower than the Sun in oK06no and oK10no, and 80$\%$ lower in oR18no. While the bulk of the stars in the solar neighbourhood show a [Mg/Si] solar or super-solar, GCE models tend to predict a ratio lower than solar. In particular, the largest departure seen in oR18no is mostly due to the contribution of energetic SN explosions for the 12M$_{\odot}$ models by \cite{ritter:18}. 
%The distribution of the CNO elements is shown in the upper right panel of Figure~\ref{fig: tk_plots/ratios_no}. Models oR18no and oK06no have an [N/O] ratio strongly increasing with the evolution time (and with [Fe/H]). They both reached the solar [N/O] ratio too early, more than 4 Gyr before the formation of the Sun. On the other hand, oK10no shows an [N/O] similar to the other models at the end of the simulations (about 60$\%$ higher than solar), but the same values do not change significantly in the part of the simulations relevant for the MW disk. Most common [N/O] ratios are obtained too early, at much lower metallicities. 
%Several challenges need to be considered for the GCE of CNO elements and isotopes. The possible relevance of fast rotating stars (not included in these models) was highlighted by several previous works in order to reproduce the abundance patterns in the Milky Way \citep[e.g.,][]{chiappini:06, chiappini:08, prantzos:18, romano:19}. Additionally, \cite{pignatari:15} discussed the contribution of H ingestion events in massive star for the GCE of N, using stellar models consistent with abundance measurements in presolar grains made by CCSNe just before the formation of the Sun. 
%The hint that the N predictions from our GCE models have problems is more evident in Figure~\ref{fig: tk_plots/ratios_no}, lower left panel. All GCE profiles show a [S/N] ratio that is too high compared to observations. Most of the stars have indeed a subsolar elemental ratio, with a scatter in the order of a factor of 3.
%Finally, in \S~\ref{sec: stars} we discussed about the stellar production of O and Mg, and of Si and S. Real stars show an observational scatter larger than a factor of two for both [Mg/O] and [S/Si] ratios (lower right panel, Figure~\ref{fig: tk_plots/ratios_no}). Such a variation is only marginally captured by the GCE models shown: for both the two ratios, variations at relevant timescale are in the order of 20$\%$ or less. 




%% Figure environment removed



%In Figure~\ref{fig: tk_plots/ratios_K06}, the models oK06m20 and oK06m25 are used in comparison with the default model without including faint CCSNe. As described in \S \ref{sec: stars}, the faint CCSNe component used in oK06m20 includes C ashes and part of the explosive O-burning ejecta, while C ashes are the deepest part of the ejecta for the faint CCSNe component used for oK06m25. None of these stellar models carry a significant signature of shell mergers. Both oK06m20 and oK06m25 are built with the assumption that most of the massive stars in the mass range 20-25M$_{\odot}$ will mostly produce faint SNe (ratio 10:1, \S~\ref{sec: tk_plots/models}). In Figure~\ref{fig: tk_plots/ratios_K06}, upper left panel, model oK06m20 shows no relevant differences compared to the default GCE model. Indeed, C, O and Mg are all ejected in the m1 component, together with a relevant quantity of Si. On the other hand, the m2 component only carry the C-burning and He burning minor contribution to Si. The oK06m25 model shows a lower scatter of [C/O] at the relevant evolution timescales, with a final solar-like ratio. But the [Mg/Si] pattern is increased by about 60$\%$, reproducing ratios representative of most of the stars in the solar neighbourhood. 
%The [N/O] evolution during MW disk timescales is more compact in the two test cases, with -0.2 $<$ [N/O] $<$ -0.05 and -0.1 $<$ [N/O] $<$ +0.1 for oK06m20 and oK06m25, respectively (Figure~\ref{fig: tk_plots/ratios_K06}, upper right panel). Both the two test models in the figure show a 30$-$50$\%$ lower [C/O] pattern for the same [N/O] values, which qualitatively might fit better more stars than oK06no. In general, the m1 and m2 components show a limited impact on the GCE of CNO elements. 
%In the lower left panel, oK06m20 shows a constant [C/N] ratio, about 50$\%$ lower than solar. However, the extra S in the m1 component generates a final [S/N] that is about 60$\%$ higher than solar, and a factor of 2-3 larger than the bulk of stars in the solar neighbourhood.
%The oK06m25 model instead show a [S/N] ratio evolving from solar to 20$\%$ lower, more consistent with observations. Finally, the [Mg/O] and [S/Si] ratios seem to be only marginally affected by the m1 and m2 components, possibly with 10-20$\%$ higher [Mg/O], while we should aim to a lower [Mg/O] and higher [S/Si] to better reproduce most of the stars (lower right panel). 




%% Figure environment removed


%In Figure~\ref{fig: tk_plots/ratios_R18}, the impact of the additional m2 component is tested on the oR18no setup, with different frequencies. As discussed in \S~\ref{sec: tk_plots/models}, the m1 component is present in the \cite{ritter:18} yields set, hidden with other Si-rich and Fe-rich components generated by shell mergers in the massive star progenitors. Therefore, oR18m25d5, oR18m25d2 and oR18m25 are showing the impact of the m2 component extended in mass range and with increasing relevance (with ratio 2:1, 5:1 and 10:1 with respect to the default CCSN yields, respectively). In general, the increase of the extra m2 component shows a linear impact on the elemental ratios explored in Figure~\ref{fig: tk_plots/ratios_R18}. 
%The [Mg/Si] ratio in models oR18m25d5 and oR18m25d2 seems to be still too low compared to most of the observations, although oR18m25d5 provides the best fit of the solar ratios within this GCE setup (upper left panel). The model oR18m25 is providing the best fit of the solar neighbourhood, similarly to oK06m25 with respect to oK06no (Figure~\ref{fig: tk_plots/ratios_K06}). 
%In Figure~\ref{fig: tk_plots/ratios_R18}, upper right panel, we see again the reduction up to 40$\%$ of the [C/O] ratio for the same [N/O] values. The relevant [N/O] spread goes from a factor of 4 (oR18no) down to a factor of two (oR18m25). The all models tend to reach too early the solar ratio, overestimating the [N/O] by about 60$\%$. 
%All models show the same pattern in the diagram made by the [C/N] and [S/N] (lower left panel). On the other hand, both the final ratios decrease with increasing the relevance of the m2 component, and the scatter of the [S/N] ratio after 1 Gyr from the start of the simulations is becoming consistent with the range of values observed in the MW disk. The model oR18m25d5 shows the best reprocution of the solar ratios at the time of the formation of the Sun, while oR18m25d2 and oR18m25 better reproduce the observed scatter in the solar neighbourhood.  
%In Figure~\ref{fig: tk_plots/ratios_R18}, lower right panel, the extra m2 source increases by up to a factor of two the [Mg/O] ratio, and decreases the [S/Si] by about the same factor. While models oR18m25d5 and oR18m25d2 have a better reproduction of the solar abundances for these elemental ratios, overall the m2 component is x the GCE models to predict too high [Mg/O] (up to about a factor of 2-3) and too low [S/Si] (up to about 60$\%$) with respect to the bulk of the stars in the solar neighbourhood. 




%% Figure environment removed


%In summary, we have shown that an abundant population of faint SNe would allow to better reproduce the observation scatter in the local MW disk. Starting from GCE models using two different sets of CCSN yields, we have obtained that most of faint CCSNe should eject C-burning products and the other external layers, but they should not carry the O-burning products. Otherwise, the observed [Mg/Si] range cannot be reproduced. Considering a population of faint CCSNe in the mass range 20-25M$_{\odot}$, the best results are obtained with faint SNe an order of magnitude more frequent than standard CCSNe. For the other abundance ratios considered, the GCE models including faint SNe provide similar or better results, with the exception of the diagram with [Mg/O] and [S/Si].

%In this case, the large scatter observed is not reproduced by GCE simulations, in particular for large [S/Si] ratios. We have seen in \S~\ref{sec: stars} that SNIa yields from different authors show some variation in the production of Si and S, which will need to be explored more in term of stellar uncertainties and nuclear reaction uncertainties. Notice that according to the sensitivity study by \cite{parikh:13}, there is no relevant impact of nuclear uncertainties on the Si and S yields. Instead, we have seen that the production of a [S/Si] higher than solar may be a direct product of CO shell mergers. We can argue that this component was not present in all faint SNe, since the Si made would not allow to increase the [Mg/Si]. On the other hand, it cannot be only present in standard CCSNe ejecta, since the Si and S peaks made in explosive O-burning conditions would hide the CO-shell merger preSN component. In our set of stellar models, we do not have a faint SNe with the S-rich component seen in the C ashes of the 15M$_{\odot}$ model (Figure~\ref{fig: ccsn_elements}). Therefore, we cannot produce a solid GCE model to better explore this scenario. 



\section{Summary and Conclusions}
\label{sec: summary}


We presented 198 new GCE simulations of the solar neighbourhood, with the main goal to study the production and evolution of the important planet-building nuclides C, N, O, Mg, Si, and S, in comparison to stellar observations. We chose these elements because results from simulations of planet formation and evolution depend on their initial abundances \citep[e.g.,][]{frank:14}. One of the fundamental purposes of this work is also to provide an accessible (but comprehensive) picture about the challenges and the uncertainties in stellar simulations, observations and GCE, and at the same time we also want to make clear what are the needs of planet-formation and evolution studies from observations and from GCE, in the light of present and future opportunities to unfold thanks to the new generation of observation facilities. 
The new data coming from TESS \citep[][]{ricker:15} and from JWST \citep{beichman:14}, as well as from future facilities like ARIEL \citep[][]{tinetti:18,turrini:18,turrini:21a}, will complement those being provided by ongoing ground-based efforts \citep[e.g.][]{giacobbe:21,carleo:22, guilluy:22} and we can anticipate a greatly enhanced window with which to study these processes in more detail.
%will provide a unique observational window to study these processes in greater details.
%In the analysis, we did not apply any normalization to solar to the results from GCE simulations. The chemical enrichment history of these elements is of great interest to understand the GCE of the Milky Way, since all the contributions from CCSNe, SNIa and AGB stars must be taken into account with their different timescales in order to study the evolution of these elements \citep[e.g.,][]{molla:15,mishenina:17,prantzos:18,kobayashi:20}. 
%Results from simulations of planet formation and evolution depend on the initial abundance of these elements. The new data coming from TESS \citep[][]{ricker:15} and, soon, from JWST \citep{beichman:14}, as well as from future facilities like ARIEL \citep[][]{tinetti:18,turrini:18,turrini:21a}, will complement those being provided by ongoing ground-based efforts \citep[e.g.][]{giacobbe:21} and will provide a unique observational window to study these processes in greater details.
In this context, GCE models provide the initial composition of %planets and stars 
stars and of their proto-planetary disks where planets are formed at different times and locations in the Galaxy for all the elements. Based on theoretical simulations and observations, we also expect that planet-formation processes will drastically affect some of the planet abundances measured today with respect to the pristine abundances of the proto-planetary disk, while others will only be marginally affected. The results of GCE models provide therefore a crucial additional benchmark for simulations of planet formation, in particular when elemental abundances from the stellar host are uncertain or not available.

The GCE simulation framework presented here is made of five sets of models, where the impacts of stellar yields, of the stellar mass upper limit contributing to the chemical evolution (M$_{\rm up}$) and of faint CCSNe were explored. 
With our models, classical stellar sources used to reproduce the evolution of the elements C, N, O, Mg, Si and S are not able to fully reproduce the solar abundances, and/or the observed range in the solar neighbourhood, in particular for the [C/O] and [Mg/Si] diagram.
In our analysis, we did not apply any corrections to force the results from GCE simulations to match the solar abundances. 
The chemical enrichment history of these elements in the Milky Way is complicated, since all the contributions from CCSNe, SNIa and AGB stars must be taken into account, along with their different timescales \citep[e.g.,][]{molla:15,mishenina:17,prantzos:18,kobayashi:20}. 

We did not find a specific set of yields that is able to solve all the ratios considered in our analysis within the correct GCE evolution timescale. We also show that the impact of M$_{\rm up}$ is in general limited for these elements considered, and it is model dependent. 

By considering realistic frequencies of faint CCSNe, we obtain instead variations of elemental ratios in the order of 0.1-0.2 dex. In particular, we find that the increase of [C/O] and [Mg/Si] with increasing faint CCSN frequency may help to better reproduce the abundances %dispersion 
observed in stars in the solar neighbourhood. The potential reduction of [S/N] in the order of 0.2 dex can also help match the range of observations, with its impact depending on the set of CCSN yields adopted.

The reduction of the observational uncertainties for the elements considered will be a crucial step towards solving present discrepancies between GCE simulations and observations. The more limited abundance dispersion in the stellar sample by \cite{bedell:18} compared to other analogous works requires independent verification. 
\cite{ramirez:14} discussed the star-to-star scatter for different elements, showing that while several elements (including O and Si discussed in this paper) present a variation compatible with the measurement errors, other elements not discussed here (e.g., Na, Al, V, Y, and Ba) may have larger discrepancies. From a similar analysis \cite{adibekyan:15a} instead found that the star-to-star scatter may simply increase with the decrease of the number of spectral lines used in the derivation of the abundances. This would indicate that a good fraction of the observed scatter is not astrophysical.
In more general terms, the comparison between data from different observational surveys obtained using different spectral lines and stellar parameters should be undertaken with caution. New comprehensive atomic physics investigations of transition probabilities for relevant spectral lines are further needed in order to improve the present results. 
%We have seen that significant observational uncertainties are still affecting the elements. .  

We have highlighted how a different definition of solar references provide a major additional source of uncertainty. We have shown that spectroscopic observations vary significantly between different works once absolute abundances are compared instead of those normalised to solar. Planet-formation simulations, however, use absolute pristine stellar abundances as a starting point, and therefore they are directly affected \citep[][]{spaargaren:23}. Alongside C, N, O, Mg, Si, and S discussed in this work, elements of interest for planet-formation studies include lithophile elements such as Cl, Cr, K, Na, V, P, Ti, Al and Ca 
%whose abundances 
the abundances of which
can potentially be better constrained by future facilities such as ARIEL \citep{tinetti:18,turrini:18}. As discussed by \cite{turrini:21a} and \cite{turrini:22} each of these elements, being more refractory than O, can be used in place of S to study the planet-formation history together with C, N and O. As the accuracy of atmospheric retrieval methods for exoplanetary observations is currently capped to about 10-20\% \citep[see][for discussion]{barstow:20,turrini:22}, the characterisation of stellar abundances with the precision of 0.1 dex would provide a solid base for the next generation of planet-formation studies to compare with atmospheric data. % \textbf{for de-gassed secondary atmospheres}.
Note that Fe is another element essential for mineralogy and planet formation, but it is not included in the analysis presented here. Undeniably, CCSNe yields for Fe are quite uncertain \citep[e.g.,][]{pignatari:16,sukhbold:16,curtis:19}, and stellar-yields uncertainties are then propagated to GCE, where additional uncertainties include for instance assigning the correct populations of SNIa contributing to GCE \citep[e.g.,][]{lach:20, grunov:21}. We therefore report the study of the GCE of Fe and of the Fe-group elements (for a consistent analysis they cannot be treated separately) in the Milky Way disk as a separate work (Trueman et al., submitted).

Once the uncertainties in spectroscopic observations and in the solar composition are sufficiently reduced, GCE simulations hold the potential to generate a %yield 
robust %better 
fit to the compositional catalogue of stars in the solar neighbourhood. Notwithstanding, more powerful constraints need to be derived on the role of faint CCSNe required to cover the full range of observations. 

%We designed GCE simulations to include the contribution from faint SNe. Two different components were used, based on two faint SNe stellar models by \cite{ritter:18}.
%The first component includes part of the explosive O-burning yields, C-ashes and all the other external ejecta. The second component only includes C-ashes and all the external ejecta.
%The GCE models using this second type of faint SNe are able to reproduce the observed range of the [Mg/Si] ratio. Other ratios are not particularly affected, or show moderate improvements compared to GCE models without faint SNe, with the exception of the [S/Si] ratio. 
%On the other hand, the first type of faint SNe does not show improvement concerning the [Mg/Si] ratio, since it includes a significant amount of Si made by the O-burning. 

%In principle, the [Mg/O] and [S/Si] diagram should be the easiest to match because of their stellar production, and the limited dependence expected from GCE uncertainties. However, all GCE models considered in this work fail to reproduce the bulk of the observations, with or without considering faint SNe. We discuss possible solutions of this issue. In particular, the underproduction of the observed [S/Si] ratio could be due to a much stronger contribution of CO shell merger to be taken into account. We did a qualitative exploration of the production of Si and S in these conditions, but a comprehensive study will need the use of multi-dimensional hydrodynamics simulations. 

%[FINAL REMARKS HERE...]




\section*{Acknowledgements}
We acknowledge support from STFC (through the University of Hull's Consolidated Grant ST/R000840/1), and access to {\sc viper}, the University of Hull HPC Facility. MP acknowledges the support to NuGrid from the National Science Foundation (NSF, USA) under grant No. PHY-1430152 (JINA Center for the Evolution of the Elements), the "Lendulet-2014" Program of the Hungarian Academy of Sciences (Hungary), the ERC Consolidator Grant funding scheme (Project RADIOSTAR, G.A. n. 724560, Hungary), the ChETEC COST Action (CA16117), supported by the European Cooperation in Science and Technology, and the IReNA network supported by NSF AccelNet. We acknowledges support from the ChETEC-INFRA project funded by the European Union’s Horizon 2020 Research and Innovation programme (Grant Agreement No 101008324). TCB also acknowledges partial support from PHY 14-30152; Physics Frontier Center/JINA Center for the Evolution of the Elements (JINA-CEE), and OISE-1927130: The International Research Network for Nuclear Astrophysics (IReNA), awarded by the US National Science Foundation. TCB and BKG thank the Leverhulme Trust for the award of a Visiting Professorship for TCB to Hull University. DT acknowledges the support of the Italian National Institute of Astrophysics (INAF) through the INAF Main Stream project ``Ariel and the astrochemical link between circumstellar discs and planets'' (CUP: C54I19000700005), and of the Italian Space Agency (ASI) through the ASI-INAF contract no. 2021-5-HH.0. SJM thanks the Alexander von Humboldt Foundation for support during significant phases of this writing. SJM is supported by the Research Centre for Astronomy and Earth Sciences, a Centre for Excellence of the Hungarian Academy of Sciences.

\section*{Data Availability}

The data generated for this article will be shared on reasonable request to the corresponding author.


%%%%%%%%%%%%%%%%%%%%%%%%%%%%%%%%%%%%%%%%%%%%%%%%%%

%%%%%%%%%%%%%%%%%%%% REFERENCES %%%%%%%%%%%%%%%%%%

% The best way to enter references is to use BibTeX:

\bibliographystyle{mnras}
%\bibliography{references} % if your bibtex file is called example.bib
\bibliography{main.bbl} % if your bibtex file is called example.bib

%%%%%%%%%%%%%%%%%%%%%%%%%%%%%%%%%%%%%%%%%%%%%%%%%%

%%%%%%%%%%%%%%%%% APPENDICES %%%%%%%%%%%%%%%%%%%%%

\appendix
%
\section{Complete list of figures for GCE simulations}
\label{app_1: plots extra}

In this section the full list of figures exploring the impact of both M$_{\rm up}$ and faint supernovae parameter spaces in GCE simulations are provided. 

As in Figure \ref{fig: tk_plots/ratios_gce_oK10m40} for the oK10 set, the results from oK06 models with M$_{\rm up}$=40M$_{\odot}$ are shown using faint supernova models m20 and m25 in Figure \ref{fig: tk_plots/ratios_oK06_m40}. The same is done for M$_{\rm up}$=20M$_{\odot}$ and M$_{\rm up}$=100M$_{\odot}$ in Figures \ref{fig: tk_plots/ratios_oK06_m20} and \ref{fig: tk_plots/ratios_oK06_m100}, respectively.

For oK10, the results using M$_{\rm up}$=20M$_{\odot}$ and 100M$_{\odot}$ for different faint supernovae are given in Figures \ref{fig: tk_plots/ratios_gce_oK10m20} and \ref{fig: tk_plots/ratios_gce_oK10m100}. The same results for M$_{\rm up}$=40M$_{\odot}$ are discussed in section \S~\ref{sec: results} (Figure \ref{fig: tk_plots/ratios_gce_oK10m40}).

For the oR18 set, the results using M$_{\rm up}$=20M$_{\odot}$ and 100M$_{\odot}$ for different faint supernovae are given in Figures \ref{fig: tk_plots/ratios_oR18_m20} and \ref{fig: tk_plots/ratios_oR18_m100}.
The same results for M$_{\rm up}$=40M$_{\odot}$ are discussed in section \S~\ref{sec: results} (Figure \ref{fig: tk_plots/ratios_oR18_m40}).

For models oR18d, the results using M$_{\rm up}$=40M$_{\odot}$, 20M$_{\odot}$ and 100M$_{\odot}$ for different faint supernovae are given in Figures \ref{fig: tk_plots/ratios_oR18d_m40}, \ref{fig: tk_plots/ratios_oR18d_m20} and \ref{fig: tk_plots/ratios_oR18d_m100}, respectively. For models oR18h the results results using M$_{\rm up}$=40M$_{\odot}$, 20M$_{\odot}$ and 100M$_{\odot}$ for different faint supernovae are given in Figures \ref{fig: tk_plots/ratios_oR18h_m40}, \ref{fig: tk_plots/ratios_oR18h_m20} and \ref{fig: tk_plots/ratios_oR18h_m100}, respectively. Finally, the same is reported for models oL18 in Figures \ref{fig: tk_plots/ratios_oL18_m40}, \ref{fig: tk_plots/ratios_oL18_m20} and \ref{fig: tk_plots/ratios_oL18_m100}, respectively.



%
%If you want to present additional material which would interrupt the flow of the main paper,
%it can be placed in an Appendix which appears after the list of references.


%%%%%%%%%%%%%%%%%%%%%%%%%%%%%%%%%%

% Figure environment removed


% Figure environment removed

% Figure environment removed


%%%%%%%%%%%%%%%%%%%%%%%%%%%%%%%%%%%%%%%%%%%%%%%%%%%%




% Figure environment removed

% Figure environment removed


%%%%%%%%%%%%%%%%%%%%%%%%%%%%%%%%%%%%%%%%%%%%%%%%%%%%




% Figure environment removed

% Figure environment removed

%%%%%%%%%%%%%%%%%%%%%%%%%%%%%%%%%%%%%%%%%%%%%%%%%%%%%%


% Figure environment removed

% Figure environment removed

% Figure environment removed

%%%%%%%%%%%%%%%%%%%%%%%%%%%%%%%%%%%%%%%%%%%%%%%%%%%%%%%%%%%%%%%%%%

% Figure environment removed

% Figure environment removed

% Figure environment removed

%%%%%%%%%%%%%%%%%%%%%%%%%%%%%%%%%%%%%%%%%%%%%%%%%%

% Figure environment removed

% Figure environment removed

% Figure environment removed

% Don't change these lines
\bsp    % typesetting comment
\label{lastpage}
\end{document}
% End of mnras_template.tex
