%%%%%%%% ICML 2022 EXAMPLE LATEX SUBMISSION FILE %%%%%%%%%%%%%%%%%

\documentclass[nohyperref]{article}

% Recommended, but optional, packages for figures and better typesetting:
\usepackage{microtype}
\usepackage{graphicx}
% \usepackage{subfigure}
\usepackage{subcaption}
\usepackage{booktabs} % for professional tables

% hyperref makes hyperlinks in the resulting PDF.
% If your build breaks (sometimes temporarily if a hyperlink spans a page)
% please comment out the following usepackage line and replace
% \usepackage{icml2022} with \usepackage[nohyperref]{icml2022} above.
\usepackage{hyperref}


% Attempt to make hyperref and algorithmic work together better:
\newcommand{\theHalgorithm}{\arabic{algorithm}}

% Use the following line for the initial blind version submitted for review:
% \usepackage{icml2022_pods}

% If accepted, instead use the following line for the camera-ready submission:
\usepackage[accepted]{icml2022_pods}

\newcommand{\compresslist}{ % Define a command to reduce spacing within itemize/enumerate environments, this is used right after \begin{itemize} or \begin{enumerate}
\setlength{\itemsep}{1pt}
\setlength{\parskip}{0pt}
\setlength{\parsep}{0pt}
}


% For theorems and such
\usepackage{amsmath}
\usepackage{amssymb}
\usepackage{mathtools}
\usepackage{amsthm}

% if you use cleveref..
\usepackage[capitalize,noabbrev]{cleveref}

%%%%%%%%%%%%%%%%%%%%%%%%%%%%%%%%
% THEOREMS
%%%%%%%%%%%%%%%%%%%%%%%%%%%%%%%%
\theoremstyle{plain}
\newtheorem{theorem}{Theorem}[section]
\newtheorem{proposition}[theorem]{Proposition}
\newtheorem{lemma}[theorem]{Lemma}
\newtheorem{corollary}[theorem]{Corollary}
\theoremstyle{definition}
\newtheorem{definition}[theorem]{Definition}
\newtheorem{assumption}[theorem]{Assumption}
\theoremstyle{remark}
\newtheorem{remark}[theorem]{Remark}

% Todonotes is useful during development; simply uncomment the next line
%    and comment out the line below the next line to turn off comments
%\usepackage[disable,textsize=tiny]{todonotes}
\usepackage[textsize=tiny]{todonotes}


% The \icmltitle you define below is probably too long as a header.
% Therefore, a short form for the running title is supplied here:
\icmltitlerunning{Towards Practicable Sequential Shift Detectors}

\begin{document}

\twocolumn[
\icmltitle{Towards Practicable Sequential Shift Detectors}

% It is OKAY to include author information, even for blind
% submissions: the style file will automatically remove it for you
% unless you've provided the [accepted] option to the icml2022
% package.

% List of affiliations: The first argument should be a (short)
% identifier you will use later to specify author affiliations
% Academic affiliations should list Department, University, City, Region, Country
% Industry affiliations should list Company, City, Region, Country

% You can specify symbols, otherwise they are numbered in order.
% Ideally, you should not use this facility. Affiliations will be numbered
% in order of appearance and this is the preferred way.
% \icmlsetsymbol{equal}{*}

\begin{icmlauthorlist}
\icmlauthor{Oliver Cobb}{yyy}
\icmlauthor{Arnaud Van Looveren}{yyy}
%\icmlauthor{}{sch}
%\icmlauthor{}{sch}
\end{icmlauthorlist}

\icmlaffiliation{yyy}{Seldon Technologies}

\icmlcorrespondingauthor{Oliver Cobb}{oc@seldon.io}

% You may provide any keywords that you
% find helpful for describing your paper; these are used to populate
% the "keywords" metadata in the PDF but will not be shown in the document
\icmlkeywords{Machine Learning, ICML}

\vskip 0.3in
]

% this must go after the closing bracket ] following \twocolumn[ ...

% This command actually creates the footnote in the first column
% listing the affiliations and the copyright notice.
% The command takes one argument, which is text to display at the start of the footnote.
% The \icmlEqualContribution command is standard text for equal contribution.
% Remove it (just {}) if you do not need this facility.

\printAffiliationsAndNotice{}  % leave blank if no need to mention equal contribution
% \printAffiliationsAndNotice{\icmlEqualContribution} % otherwise use the standard text.


\maketitle
Non-line-of-sight (NLOS) imaging methods are capable of reconstructing complex scenes that are not visible to an observer using indirect illumination. 
%
However, they assume only third-bounce illumination, so they are currently limited to single-corner configurations, and present limited visibility when imaging surfaces at certain orientations.
%
\new{To reason about and tackle these limitations, we make the key observation that planar diffuse surfaces behave specularly at wavelengths used in the computational wave-based NLOS imaging domain. We call such surfaces \emph{virtual mirrors}.
%
We leverage this observation to expand the capabilities of NLOS imaging using illumination beyond the third bounce, addressing two problems: imaging single-corner objects at limited visibility angles, and imaging objects hidden behind two corners.}
%
%
\new{To image objects at limited visibility angles,} we first analyze the reflections of the known illuminated point on surfaces of the scene as \NEWW{an} estimator of \NEWW{the position and orientation of objects with limited visibility}.
%
We then image those \new{limited} visibility objects by computationally building secondary apertures at other surfaces that observe the target object from a \new{direct} visibility perspective. 
%
Beyond single-corner NLOS imaging, \new{we exploit the specular behavior of {virtual mirrors}} to image objects hidden behind a second corner \new{by imaging the space behind such virtual mirrors}, where the mirror image of objects hidden around two corners is formed.  
%
No specular surfaces were involved in the making of this paper.  
\section{Introduction}
\label{sec:introduction}

With data and models becoming progressively larger
\citep{DBLP:conf/nips/ChenKSNH20,DBLP:journals/corr/abs-2001-08361,DBLP:journals/corr/abs-2108-07258,DBLP:journals/corr/abs-2206-04615}, 
the ability to reduce training wall--clock time is a requirement for practical \gls{ml} at scale.
Optimizer scaling rules allow us to find faster learning procedures that produce similar results.
For example, the \emph{linear scaling rule} for \gls{sgd}
\citep{DBLP:journals/corr/Krizhevsky14,DBLP:journals/corr/GoyalDGNWKTJH17},
states that the learning rate should be scaled linearly with the batch size.
This optimizer scaling works \emph{both ways}.
Access to larger computational resources means one can train equivalent models in reduced wall-clock time.
Alternatively, with access to limited computational resources, larger distributed computations can be replicated at increased wall-clock time.

Many \gls{ml} algorithms rely on a \emph{model EMA},
a functional copy of a \emph{target model}\footnote{The target model usually undergoes gradient-based optimization, but this does not have to be the case.}, whose 
parameters move towards those of its target model according to an \gls{ema} (\Cref{def:emaUpdateDefinition}) at a rate parameterized by a momentum hyperparameter $\rho$.  
\begin{restatable}[EMA Update]{definition}{firstema}
\label{def:emaUpdateDefinition}
     The \gls{ema} update for the model \gls{ema} parameters $\rvzeta_t$ following target model parameters  $\rvtheta_t$ at iteration $t$ with momentum $\rho\equiv1-\beta_\rho$ is
    \label{def:ema}
    \begin{equation}
    \rvzeta_{t+1}
    =
    \rho \,\rvzeta_t + (1-\rho)\,\rvtheta_t
    \equiv
    (1-\beta_\rho) \,\rvzeta_t +\beta_\rho\,\rvtheta_t.
    \end{equation}
\end{restatable}
The \emph{model \gls{ema}} has a number of desirable properties:
i) 
the model \gls{ema} inhabits wider minima than the target model, reducing overfitting and improving generalization
\citep{Ruppert1988EfficientEF,Polyak92,DBLP:conf/iclr/HuangLP0HW17,DBLP:conf/uai/IzmailovPGVW18,DBLP:conf/cvpr/HeCXLDG22};
ii)
compared to the target model, the model \gls{ema} moves slowly, 
making it useful as a stabilizer for networks governing Bellman updates in 
reinforcement learning,
\citep{DBLP:journals/corr/LillicrapHPHETS15};
and iii)
the model \gls{ema} is relatively cheap to compute, whilst providing a valid model but \emph{different} to the target model.
This third property has made the model \gls{ema} a common choice for the \emph{teacher} in many distillation setups, 
from semi-supervised learning \citep{DBLP:conf/nips/TarvainenV17,sohn2020fixmatch,manohar2021kaizen,higuchi2022momentum},
to \gls{ssl} methods like 
\gls{byol} \citep{DBLP:conf/nips/GrillSATRBDPGAP20},
DINO \citep{DBLP:journals/corr/abs-2104-14294},
and data2vec \citep{baevski2022data2vec,DBLP:journals/corr/abs-2212-07525}.

Despite its significant role in optimization, a recipe for adapting the \gls{ema} Update (\Cref{def:emaUpdateDefinition}) when changing batch size has,
to the best of our knowledge, been absent.
To address this, we derive an \gls{ema} Scaling Rule (\Cref{def:ema-sr})
which states how the \gls{ema} momentum $\rho$ hyperparameter \emph{should} be modified\footnote{We stress that the study of momentum in gradient-based optimizers is not the focus of this work.
We refer to \citet{DBLP:conf/iclr/SmithL18,li2019stochastic} for a discussion on scaling rules for these methods.
}.
\begin{restatable}[\gls{ema} Scaling Rule]{definition}{firstemascaling}
\label{def:emaScalingRuleExponentialVersion}
    When computing the \gls{ema} update (\Cref{def:ema}) of a model undergoing stochastic optimization with batch size $\hat B=\kappa B$,
    use a momentum $\hat\rho=\rho^\kappa$ and scale other optimizers according to their own scaling rules.
    \label{def:ema-sr}
\end{restatable}
In \Cref{def:ema-sr}, the momentum $\rho$, which is defined at batch size $B$, typically corresponds to a ``good hyperparameter choice'', although this does not need to be the case in general.
In this paper, we make the following contributions.
\begin{enumerate}[leftmargin=0.75cm]
    \item With the assumptions of \citet{DBLP:journals/corr/GoyalDGNWKTJH17}, we derive an \gls{ema} Scaling Rule: the \gls{ema} update \emph{momentum} should be scaled \emph{exponentially} with the batch size (\Cref{def:ema-sr}).
    \item To validate this EMA Scaling Rule theoretically, we propose \gls{sde} approximations of optimization in the presence of a model \gls{ema} (\Cref{subsec:ema-sdes}). 
    This model \gls{ema} contributes to the loss, covering semi-supervised learning and \gls{ssl}.
    We prove that these approximations are first order weak approximations, and that our \gls{ema} Scaling Rule is correct in the \gls{sde} limit under realistic gradient assumptions (\Cref{cor:validity-scaling-rule}).
    \item We empirically validate the \gls{ema} Scaling Rule in synthetic settings (\Cref{subsec:toy-experiment})
    and real-world settings where the model \gls{ema} plays an increasingly significant role in optimization: 
    i) 
    where the model \gls{ema} is used during inference instead of the target model (\Cref{subsec:supervised-polyakking}); 
    ii) 
    pseudo-labeling, 
    where the model \gls{ema} (\emph{teacher}) follows the target model (\emph{student}), and the \emph{student} is optimized 
    on a mixture of a) labeled data and b) data without labels, whose pseudo-labels are produced by the \emph{teacher} (\Cref{subsec:semi-supervised}); 
    and iii)
    self-supervised learning, which is the same as the semi-supervised case, except there is no labeled data (\Cref{subsec:self-supervised}).
    \item We observe that pseudo-labeling and \gls{ssl} training dynamics during optimizer warm-up are not always able to be replicated at large batch sizes using \emph{only} the \gls{ema} Scaling Rule.
    We propose and verify practical methods to overcome this limitation, enabling us to scale to a batch size of 24,576 with BYOL \glspl{vit}, reducing wall-clock training by 6$\times$ under idealized hardware scenarios while maintaining performance of the batch size 4096 baseline.
\end{enumerate}
Finally, to aid practitioners looking to scale, in \Cref{app:scaling-toolbox} we provide a \emph{Scaling Toolbox}, which gives practical advice on how to scale systematically, collecting known scaling rules, and explaining how to think about the \gls{sde} perspective of optimization.

\section{The EMA Scaling Rule}
\label{sec:a-momentum-scaling-rule}

We begin with an informal discussion of scaling rules and motivate the existence of an exponential scaling rule for the momentum parameter controlling the update of the model \gls{ema}.

\subsection{Background and an informal discussion of scaling rules}
\label{sec:a-momentum-scaling-rule-background}

Consider a model with parameters $\rvtheta_t$ at iteration $t$
updated with \gls{sgd} (\Cref{def:sgd}).
\begin{definition}[SGD Update]
     The \gls{sgd} update for a model with parameters $\rvtheta_t$ at iteration $t$ given a minibatch $\sB=\{x^{(b)}\sim P_{\rvx}:b=1,2,\ldots,B\}$ of $B=|\sB|$ samples with learning rate $\eta$ is
    \begin{equation}
    \rvtheta_{t+1}
    =
    \rvtheta_t - \eta \times \frac1B
    \sum_{x\in\sB} \nabla_{\rvtheta} \Ls(x;\rvtheta_{t}),
    \end{equation}
    where $\Ls$ is the loss function, 
    $\nabla_\theta \Ls(x;\theta_t)$ is the parameter gradient for the sample $x$ at iteration $t$, and the $x\in \sB$ are \gls{iid} from $P_{\rvx}$.
    \label{def:sgd}
\end{definition}
Iterating over a sequence of independent minibatches 
$\sB_0, \sB_1, \ldots, \sB_{\kappa-1}$
produces model parameters
\begin{align}
    \rvtheta_{t+\kappa}
    =
    \rvtheta_t - \eta \times \frac1B
    \sum_{j=0}^{\kappa-1}
    \sum_{x\in\sB_j} \nabla_{\rvtheta} \Ls(x;\rvtheta_{t+j}).
    \label{eq:sgd}
\end{align}
If gradients vary slowly 
$\nabla_{\rvtheta}\Ls(x;\theta_{t+j})\approx \nabla_{\rvtheta}\Ls(x;\theta_{t})$, $j=0,\ldots,\kappa-1$,
\emph{one} \gls{sgd} step with
$\hat\eta=\kappa\eta$ on a batch $\widehat\sB=\cup_i\sB_i$ of size $\hat B=\kappa B$
results in $\hat\rvtheta_{t+1}\approx\rvtheta_{t+k}$,
yielding the \gls{sgd} Scaling Rule (\Cref{def:lsr}).
\begin{definition}[\gls{sgd} Scaling Rule]
    When running \gls{sgd} (\Cref{def:sgd}) with batch size $\hat B=\kappa B$,
    use a learning rate $\hat\eta=\kappa\eta$ \citep{DBLP:journals/corr/Krizhevsky14,DBLP:journals/corr/GoyalDGNWKTJH17}.
    \label{def:lsr}
\end{definition}
For clarity in this work, we adopt the naming convention \emph{[Algorithm Name] Scaling Rule},
which means all parameters of those algorithms are appropriately scaled from batch size $B$ to $\kappa B$.

As discussed in \citet{DBLP:journals/corr/GoyalDGNWKTJH17},
although the assumption of slowly changing gradients is strong, if it is true, then
$\rvtheta_{t+k}\approx \hat\rvtheta_{t+1}$
\emph{only} if $\hat\eta=\kappa\eta$.
The validity of the \gls{sgd} Scaling Rule has been formally studied.
In particular, there was ambiguity regarding whether the scaling should be a square-root or linear \citep{DBLP:journals/corr/Krizhevsky14}.
\gls{sde} approaches have resolved this ambiguity, and have been used to estimate the scaling $\kappa$ when the \gls{sgd} Scaling Rule is no longer guaranteed to hold \citep{DBLP:conf/nips/LiMA21}.

To address model parameter \glspl{ema}, we first restate the \gls{ema} Update (\Cref{def:ema}).
\firstema*
The model \gls{ema} parameters $\rvzeta$ do not typically receive gradient information, we take the convention that $\rho$ is close to one, and the $\beta_\rho$ subscript will be omitted where it is clear from the context.

Assuming again that gradients change slowly $\nabla_{\rvtheta}\Ls(x;\rvtheta_{t+j},\rvzeta_{t+j})\approx \nabla_{\rvtheta}\Ls(x;\rvtheta_{t},\rvzeta_{t})\approx \rvg$, for gradient $\rvg$,
iterating over $\kappa$ independent minibatches produces model states (see \Cref{app:matrix-calculations} for derivation)
\begin{align}
\label{eq:scalingRuleSummaryEquation}
\begin{bmatrix}
\rvtheta_{t+\kappa}
\\
\rvzeta_{t+\kappa}
\\
\rvg
\end{bmatrix}
=
\begin{bmatrix}
1 & 0 & -\eta \\
(1-\rho) & \rho & 0\\
0 & 0 & 1
\end{bmatrix}^\kappa
\cdot 
\begin{bmatrix}
\rvtheta_{t}
\\
\rvzeta_{t}
\\
\rvg
\end{bmatrix}
=
\begin{bmatrix}
\rvtheta_{t}-\eta\,\kappa \,\rvg
\\
\rho^\kappa \, \rvzeta_{t}
+(1-\rho^\kappa) \, \rvtheta_t
+\mathcal O\left(\eta\times \beta_\rho\right)
\\
\rvg
\end{bmatrix}.
\end{align}
The first row is the \gls{sgd} Scaling Rule (\Cref{def:lsr}). The third row 
implements the \emph{slowly changing gradients} assumption for the first row.
The second row is equivalent to a single \gls{ema} update (\Cref{def:ema}) with momentum $\hat\rho=\rho^\kappa$; we can take a \emph{single} \gls{sgd} update with batch size $\hat B=\kappa B$ and learning rate $\hat\eta=\kappa\eta$, and a 
\emph{single} \gls{ema} update with momentum $\hat\rho=\rho^\kappa$, and we get $(\hat\rvtheta_{t+1},\hat\rvzeta_{t+1})\approx(\rvtheta_{t+\kappa},\rvzeta_{t+\kappa})$ up to terms $\mathcal O(\eta \times \mathcal \beta_\rho)$.
This yields the \gls{ema} Scaling Rule (\Cref{def:ema-sr}).
\firstemascaling*

The \gls{ema} Scaling Rule was derived for \gls{sgd}, and is extended to other optimizers in the following way. 
An optimizer scaling rule ensures $\hat\rvtheta_{t+1}=\rvtheta_{t+\kappa}$,
satisfying identification for the first row.
Next, the zeroth order term in $\eta\times\beta_{\rho}$ in the second row in \Cref{eq:scalingRuleSummaryEquation} is optimizer-independent, and therefore unchanged.
Finally, the first order terms in $\eta\times\beta_{\rho}$ in the second row, corresponding to the scaling rule error, are an \gls{ema} accumulation of target model $\rvtheta$ updates under optimization, which is still 
$\mathcal{O}(\eta \times \mathcal  \beta_\rho)$, although its functional form may be different for different optimizers.

The above discussion is intended to give an intuition for why the \gls{ema} momentum should be scaled exponentially.
As we have used the same slow-moving gradient assumption as the original \gls{sgd} Scaling Rule,
this may cast doubt on whether our rule is correct.
To remove this ambiguity, we will follow 
\citet{DBLP:conf/iclr/SmithL18,DBLP:conf/nips/LiMA21,DBLP:conf/nips/MalladiLPA22}, and show that the \gls{ema} Scaling Rule (\Cref{def:ema-sr}) is correct in the \gls{sde} limit under more realistic gradient assumptions. 

\subsection{The EMA Scaling Rule through the lens of stochastic differential equations}
\label{subsec:ema-sdes}

\glspl{sde} are a tool typically used to obtain scaling rules from first principles~\citep{DBLP:conf/nips/LiMA21,DBLP:conf/nips/MalladiLPA22}.
In the following, we use \glspl{sde} to obtain strong theoretical guarantees for the 
\gls{ema} Scaling Rule found in \Cref{sec:a-momentum-scaling-rule-background}.
We consider the following discrete dynamics for \gls{ema}:
\begin{align}
    \label{eq:iterations}
    \begin{split}
    \rvtheta_{k+1} &= \rvtheta_{k} - \eta\, \rvg_k,
    \enspace \text{with }
    \rvg_k=\nabla f(\rvtheta_k, \rvzeta_k) + \sigma \, \rvepsilon_k, 
    \text{ and }
    \rvepsilon_k \sim \mathcal{E}_\sigma(\rvtheta_k, \rvzeta_k),\\
    \rvzeta_{k+1} &= \rho \, \rvzeta_k + (1-\rho) \,\rvtheta_k,
    \end{split}
\end{align}
where $\sigma>0$ is the noise scale, 
$\mathcal{E}_\sigma(\rvtheta_k,  \rvzeta_k)$
is the gradient noise distribution, assumed to be zero-mean and variance 
$\mSigma(\rvtheta_k,  \rvzeta_k)$ 
independent of $\sigma$, and 
$\nabla f(\rvtheta_k, \rvzeta_k)\equiv\nabla_{\rvtheta} f(\rvtheta_k, \rvzeta_k)$.
We posit a dependency of the loss $f$ on the EMA $\rvzeta$ in order to cover semi-supervised (\Cref{subsec:semi-supervised}) and \gls{ssl} (\Cref{subsec:self-supervised}).
The case of Polyak-Ruppert averaging (\Cref{subsec:supervised-polyakking}), is covered by letting $f$ be independent of $\rvzeta$.

We aim to obtain an \gls{sde} approximation of \Cref{eq:iterations} as $\eta$ goes to zero.
The scaling rule for iterations of $\rvtheta$ is well known~\citep{DBLP:conf/nips/LiMA21}: we let $\sigma_0 = \sigma \sqrt{\eta}$.
The analysis of \Cref{sec:a-momentum-scaling-rule-background}
gives the scaling rule $\hat{\eta} = \eta \kappa$ and $\hat{\rho} = \rho^{\kappa}$.
Linearizing this rule near 
$\eta = 0$ 
gives 
$\hat{\rho} = 1 - \kappa\times(1 - \rho)$, which is a linear relationship between $1 -\rho$ and $\eta$. 
We therefore let $\beta_0=(1 - \rho) / \eta$ and consider the SDE 
\begin{align}
    \label{eq:sde-sgd}
    \begin{split}
        d\Theta_t &= - \nabla f(\Theta_t, Z_t)\,dt 
        +
        \sigma_0\,\mSigma(\Theta_t, Z_t)^{\frac12}\,dW_t,
        \enspace \text{with }
        W_t \text{ a Wiener process},\\
        dZ_t &= \beta_0(\Theta_t - Z_t)dt,
    \end{split}
\end{align}
where $\Theta_t$ and $Z_t$ are \gls{sde} variables relating to model and \gls{ema} parameters respectively.
The SDE in \Cref{eq:sde-sgd} approximates the discrete iterations of \Cref{eq:iterations} when the learning rate $\eta$ goes to zero.
One way to see this is that an Euler-Maruyama discretization of the SDE with learning rate $\eta$ exactly recovers the discrete iterations.
More formally, we have \Cref{thm:sde-for-sgd-ema}, which is in the same spirit as those found in~\cite{DBLP:conf/nips/LiMA21,DBLP:conf/nips/MalladiLPA22}. In the theorem, $G^\alpha$ is the set of functions with derivatives up to order $\alpha$ that have at most polynomial growth (see~\Cref{def:polynomial-growth}).
\begin{theorem}[SDE for SGD + EMA; informal see~\Cref{thm:app:sde}]
     Assume that $f$ is continuously differentiable, with $f\in G^3$.
     Let 
     $\Theta_t,Z_t$ 
     be solutions of  \Cref{eq:sde-sgd},
     and $\rvtheta_k,\rvzeta_k$ iterations of \Cref{eq:iterations}
     with
     $\mSigma^{\frac12}\in G^2$. 
     Then, for any time horizon $T >0$ and function $g\in G^2$, there exists a constant $c>0$ independent of $\eta$ such that 
    \begin{equation}
        \max_{k=0,\,\dots\,,\,\lfloor T /\eta \rfloor} |\mathbb{E}[g(\Theta_{\eta k}, Z_{\eta k})] - \mathbb{E}[g(\rvtheta_k, \rvzeta_k)]| \leq c\times  \eta .
    \end{equation}
    \label{thm:sde-for-sgd-ema}
    \vspace{-0.5cm}
\end{theorem}
\Cref{thm:sde-for-sgd-ema} formalizes the intuition that the SDE is an accurate approximation of the discrete iterations. In turn, it allows validating the scaling rule in the same spirit as in~\citet{DBLP:conf/nips/MalladiLPA22}.
\begin{corollary}[Validity of the EMA Scaling Rule]
     Assume that $f$ is continuously differentiable, with $f\in G^3$ and $\mSigma^{\frac12}\in G^2$. 
     Let $\rvtheta_k^{(B)}, \rvzeta_k^{(B)}$ be iterations of \Cref{eq:iterations} with batch size $B$ and hyperparameters $\eta, \rho$. 
     Let $\rvtheta_k^{(\kappa B)}, \rvzeta_k^{(\kappa B)}$ be iterates with batch size $\kappa B$, and $\hat{\eta}$ determined by the \gls{sgd} Scaling Rule (\Cref{def:lsr}) and $\hat{\rho}$ determined by the \gls{ema} Scaling Rule (\Cref{def:ema-sr}). 
     Then, for any time horizon $T >0$ and function $g\in G^2$, there exists a constant $c>0$ independent of $\eta$ such that 
    \begin{equation}
        \max_{k=0,\,\dots\,,\, \lfloor T /\eta \rfloor} |\mathbb{E}[g(\rvtheta_{\lfloor k / \kappa \rfloor}^{(\kappa B)}, \rvzeta_{\lfloor k / \kappa \rfloor}^{(\kappa B)})] - \mathbb{E}[g(\rvtheta_k^{(B)}, \rvzeta_k^{(B)})]| \leq c\times  \eta .
    \end{equation}
    \label{cor:validity-scaling-rule}
    \vspace{-0.5cm}
\end{corollary}
\Cref{cor:validity-scaling-rule} shows that two trajectories with different batch sizes are close in the limit of small learning rate, demonstrating the validity of \Cref{def:ema-sr}.
A natural follow-up question is 
\emph{what happens when an adaptive optimizer is used instead of SGD?}
\citet{DBLP:conf/nips/MalladiLPA22} study this without an \gls{ema} and characterize how hyperparameters change with the noise scale.
In particular, they show that under a high gradient noise hypothesis, there exists a limiting SDE. 
In \Cref{app:ema-approximation-theorem}, we derive the limiting SDEs for RMSProp and Adam with an EMA.
Although a formal proof of closeness between the iterations and these SDEs is beyond the scope of this work, these \glspl{sde} indicate that the EMA Scaling Rule holds for adaptive algorithms. 
We demonstrate this empirically in \Cref{sec:experiments}.


\section{Experiments}
\label{sec:experiments}

\begin{table}[t!]
  \caption{The role of the model \gls{ema} $\rvzeta$ in the optimization of $(\rvtheta, \rvzeta)$ given a target model $\rvtheta$ for different techniques, ordered by increasing influence of the \gls{ema} model.
  All statements assume a momentum $0\leq\rho<1$ and that the target model $\rvtheta$ is subject to stochastic optimization at a batch size $B$. }
  \label{tab:different-ema}
  \centering
  \small
  \begin{tabular}{p{0.22\textwidth}p{0.71\textwidth}}
    \toprule
    \textsc{Technique} & \textsc{Role of Model \gls{ema}} \\
    \midrule
    \textsc{Polyak-Ruppert averaging, Sec. \ref{subsec:supervised-polyakking}} & 
    $\rvtheta$ undergoes optimization and is tracked by
    $\rvzeta$, which does not affect $\rvtheta$. 
    $\rvzeta$ is an estimate of $\rvtheta$ with a time horizon and variance determined by $B$ and $\rho$.
     \\ \midrule
    \textsc{Continuous pseudo-labeling, Sec.~\ref{subsec:semi-supervised}} &
    \emph{Pre-Training} is as above in Polyak-Ruppert Averaging.
    \emph{After Pre-Training}, $\rvzeta$ (\emph{teacher}) produces targets for $\rvtheta$ (\emph{student}) from unlabeled data, which is combined with labeled data.
    The optimization endpoint is dependent on $B$ and $\rho$.\\ \midrule
    \textsc{Self-supervised learning, Sec.~\ref{subsec:self-supervised}} & 
    As above in \emph{After Pre-Training}, except there is no labeled data.
    The optimization endpoint is dependent on $B$ and $\rho$.
    \\
    \bottomrule
  \end{tabular}
  \vspace{-0.3cm}
\end{table}


Now that we have derived and shown the validity of the \gls{ema} Scaling Rule, 
we verify it empirically.
The experiments validate the \gls{ema} Scaling Rule for a variety of uses of \gls{ema} and are ordered by increasing influence of the role of \gls{ema} on the optimization procedure (see \Cref{tab:different-ema}).
The baseline in all of our experiments is \emph{without the EMA Scaling Rule}, which applies all known relevant scaling rules \emph{except} the EMA Scaling Rule, and represents previous best practice.


\subsection{Polyak-Ruppert averaging in a simple setting}
\label{subsec:toy-experiment}

At inference, it is typical to use a model \gls{ema}, known as Polyak-Ruppert Averaging (\Cref{def:polyak-ruppert-average}).

\begin{definition}[Polyak-Ruppert Average]
    When optimizing model parameters $\rvtheta$, compute their \gls{ema} $\rvzeta$ (\Cref{def:ema}).
    Use $\rvzeta$ instead of $\rvtheta$ at inference \citep{Polyak92,Ruppert1988EfficientEF}.
    \label{def:polyak-ruppert-average}
\end{definition}

We begin by showing the \gls{ema} Scaling Rule is \emph{required} to match parameter trajectories in a simple setting.
Consider the optimization of $\rtheta$ in a \emph{noisy parabola}  whose loss $\Ls(\rtheta)$ is parameterized by coefficients for curvature $a>0$,
scaled additive noise $b\geq0$,
and additive noise $c\geq0$:
\begin{align}
    \Ls(\rtheta)
    &=\frac a2\,\rtheta^2,
    &\rtheta_{k+1} &= \rtheta_{k} - \eta \,\rg_k, & \rg_k&=a\,\rtheta_k + \repsilon_k,
    & \repsilon_k\sim \mathcal{N}\left(0, \tfrac{b \,\rg_k^2 + c}\kappa \right).
\end{align}
The scaling factor $\kappa$ in the covariance denominator implements gradient noise reduction as scaling (i.e. batch size) increases \citep{DBLP:journals/corr/abs-1711-04623}.
Let $\rtheta\in\R$ be optimized with \gls{sgd} (\Cref{def:sgd}) and $\rzeta\in\R$ be a Polyak-Ruppert average (\Cref{def:polyak-ruppert-average}) for $\rtheta$ with momentum $\rho=1-\beta$ .
At scaling $\kappa=1$, we use $\beta_B=\eta_B=10^{-4}$ 
and $I_B=10^4$ iterations, to yield a total time $T=I_B\times \eta_B=1$.
To keep gradients $\mathcal O(1)$ and gradient noise non-negligible, we take
$a=1$, $b=0.5$, and $c=0$.
% Figure environment removed

First, we observe the effect of scaling on a single run (\Cref{fig:parabola-single-runs}) by tracking the position of the model \gls{ema}.
We see that at scaling $\kappa=8$ or $\kappa=256$, the runs using the \gls{ema} Scaling Rule match the baseline trajectory,
whereas the runs using the baseline momentum do not, with a greater deviation induced by greater scaling $\kappa$.
Even at $\kappa=8$, there is a significant difference between scaled and unscaled trajectories, despite the seemingly small numerical difference of their momenta\footnote{Momentum enters optimization exponentially; small changes can lead to very different updates.}.

Second, we consider whether the \gls{ema} Scaling Rule is optimal.
To do this, inspired by the \gls{sde} analysis (\Cref{subsec:ema-sdes}), 
we define the approximation error, $\text{Err}(\rho,\kappa,g)$, of a test function $g$ for a given scaling $\kappa$ using momentum $\rho$, and the value of the momentum $\rho^*(\kappa,g)$ that minimizes this error:
\begin{align}
    \rho^*(\kappa,g)=
    &
    \argmin_\rho
    \text{Err}(\rho,\kappa,g),
    &
    \text{Err}(\rho,\kappa,g)
    &\equiv
    \max_{k=0,\ldots,T/\eta}
    \left|
    \E \,g(\rvzeta_k)
    -
    \E \,g(\rvzeta^{(\kappa,\rho)}_{k/\kappa})
    \right|.
    \label{eq:optimal-momentum}
\end{align}
For scalings $\kappa\in\{1,2,4,\ldots,1024\}$, we determine the optimal momentum $\rho^*$ and compare it to the \gls{ema} Scaling Rule (\Cref{fig:curve-approximation-error}, left).
The scaling rule tracks the $\rho^*$ until $\kappa=256$, when
the $\rho^*$ become systematically higher.
We see target model error increase at $\kappa=256$ (\Cref{fig:curve-approximation-error}, right). 
As the target model error is \gls{ema}-independent, this indicates that the \gls{sgd} Scaling Rule is breaking.
At the lower scaling $\kappa=64$, there is an inflection point in the \gls{ema} Scaling Rule approximation error, before the model error grows.
This difference indicates the $\mathcal{O}(\eta\times \beta_\rho)$ terms of \Cref{eq:scalingRuleSummaryEquation} are beginning to influence the \gls{ema} update.
Finally, 
these observations are true in $D=100$ dimensions, (\Cref{app:noisy-parabola}), and
we stress that \emph{not} changing the momentum at every scaling $\kappa$ induces large approximation error, indicating there is merit to using the \gls{ema} Scaling Rule.

\subsection{Supervised learning on real data with Polyak-Ruppert averaging}
\label{subsec:supervised-polyakking}

We now turn to real-world classification where
the target model $\rvtheta$ optimizes a parametric log-likelihood
$\max_{\rvtheta} \log p(\rvy | \rvx; \rvtheta)$ 
with inputs and labels $(\vx,\vy)$ drawn from a joint distribution $p(\rvy, \rvx)$.

{\bf Image Classification}~~~
We consider a variant of the original \gls{sgd} Scaling Rule result \citep{DBLP:journals/corr/GoyalDGNWKTJH17} and train a ResNetv2 \citep{DBLP:conf/eccv/HeZRS16}
on ImageNet1k \citep{DBLP:journals/corr/RussakovskyDSKSMHKKBBF14} (\Cref{fig:r50-polyak}) using a three step learning rate schedule. 
The base momentum $\rho_B=0.9999$ at batch size 1024 was found by hyperparameter optimizing for \gls{ema} test performance, and we seek to achieve this optimized performance at different batch sizes.
% Figure environment removed
We \emph{do not} apply the \gls{ema} Scaling Rule on the Batch Normalization \citep{DBLP:conf/icml/IoffeS15} statistics\footnote{Since Batch Normalization statistics use an \gls{ema} update, it is reasonable to ask whether the \gls{ema} Scaling Rule should be applied.
We investigate this in \Cref{subsec:polyak-bn}.
We find one \emph{should} apply the scaling rule, however, the effect is less significant than the application of the \gls{ema} Scaling Rule to model parameters.}. 
We observe that \emph{without} the EMA Scaling Rule, there is a significant drop in model \gls{ema} test performance, whereas \emph{with} the \gls{ema} Scaling Rule, we can approximate the baseline model \gls{ema} test top-1 performance across all batch sizes.
We match baseline \gls{ema} statistics across the full trajectory batch size 2048, where the test \gls{ema} performance diverges.
This is due to non-\gls{ema} test performance dropping for high $\kappa$ (see \Cref{app:subsec:polyak-image-classification}).
We observe that model \gls{ema} top-1 is approximately 0.2\% to 0.3\% higher than the target model.



{\bf \gls{asr}}~~~ 
We train a transformer \citep{DBLP:conf/nips/VaswaniSPUJGKP17}
using the \gls{ctc} loss~\citep{graves2006connectionist} and Adam optimizer on the \tco{} subset (100h) of LibriSpeech~\citep{panayotov2015librispeech} (for details see \Cref{app:speech}).
We apply the Adam Scaling Rule (\citet{DBLP:conf/nips/MalladiLPA22}, \Cref{def:adam-sr}) and use dynamic batching (minibatch size $\times$ sequence length $=\text{const}=290s$,
and $s$ indicates audio duration in seconds).

\emph{Without} the EMA Scaling Rule, there is a significant difference in model \gls{ema} test \gls{wer} trajectories compared to the baseline, whereas \emph{with} the \gls{ema} Scaling Rule, trajectories match, as is shown in \Cref{fig:speech-polyak}. 
We note that compared to image classification, in \gls{asr}, the model \gls{ema} converges to similar final performance irrespective of use of the scaling rule. 
This convergence is due to the longer training time compared to the \gls{ema} horizon as discussed in \Cref{tab:different-ema} (see \Cref{app:asymptoticAnalysis} for a proof sketch).
Although in this specific case one can achieve similar \emph{final performance} without the \gls{ema} Scaling Rule, it is \emph{necessary} to use the \gls{ema} Scaling Rule in order to replicate the full training trajectory, which gives \emph{guarantees} on properties like final performance (see \Cref{cor:validity-scaling-rule}).
We also observe
a growing gap between the baseline and \gls{ema}-scaled trajectories as we increase~$\kappa$. 
Inspecting the train loss and non-EMA test WER, which \emph{do not} depend on the \gls{ema} update (see \Cref{fig:app-speech-polyak}, \Cref{subsec:speech-detailed}), indicates this is due to a breakdown of the Adam Scaling Rule.
\emph{In summary, evaluation on ASR shows that the EMA Scaling Rule holds in practice for sequential data with dynamic batch sizes, as well as when using adaptive optimization.}

% Figure environment removed


\subsection{Semi-supervised speech recognition via pseudo-labeling}
\label{subsec:semi-supervised}

We continue using the same \gls{asr} model and training pipeline of~\Cref{subsec:supervised-polyakking}.
However, we consider semi-supervised learning via continuous pseudo-labeling where labeled (\tco{}, 100h) and unlabeled (the rest of LibriSpeech, 860h) data are given during training,
and the model \gls{ema} is involved in the overall optimization~\citep{likhomanenko2020slimipl,likhomanenko2022continuous, manohar2021kaizen,higuchi2022momentum}. 
We first pre-train a target model (\emph{student}) on a limited labeled set for a short period (e.g. 20k steps of $B=8\times 290s$\footnote{Note that number of steps is batch size dependent and should be scaled by $1/\kappa$ (see \Cref{app:scaling-toolbox}).}).
Concurrently, the student updates a model \gls{ema} (\emph{teacher}).
After pre-training,
we continue training the student with both labeled and unlabeled data,
with the teacher first transcribing unlabeled data from the batch producing
\glspl{pl}.
These \glspl{pl} are treated by the student as ground-truth transcriptions, and standard supervised optimization is performed.

Compared to Polyak-Ruppert Averaging (\Cref{subsec:supervised-polyakking}), where the model \gls{ema} plays no role in the joint optimization, 
we observe that in \gls{pl} it is \emph{essential} to employ the \gls{ema} Scaling Rule in order to match the model trajectories at scaled batch sizes.
When the \gls{ema} Scaling Rule is not used, \Cref{fig:speech-pl-9999} reveals a significant difference in  \gls{pl} quality trajectory, leading to a higher test \gls{wer}.

For $\kappa > 2$, we found the Adam Scaling Rule does not match perfectly the reference trajectory in the pre-training phase.
This results in a significantly different \gls{pl} quality
at the start of pseudo-labeling (20k steps of $B=8\times 290s$), which affects the training dynamics~\citep{berrebbi2023continuous}.
To alleviate the Adam Scaling Rule mismatch effect for $\kappa > 2$, we postpone the pseudo-labeling until pre-training on labeled data gives similar validation \gls{wer}, see Appendix~\ref{app:speech}. 
With this heuristic, we can match the baseline trajectory with the \gls{ema} Scaling Rule up to $\kappa=8$ (\Cref{fig:speech-pl-9999}).


\emph{In summary, (a) model EMA affects the optimization process of pseudo-labeling in ASR resulting in
the necessity of EMA Scaling Rule to be applied while scaling the batch size; (b) an optimizer scaling rule breakdown results in the EMA Scaling Rule breakdown but this effect can be alleviated by longer pre-training on labeled data having similar PLs quality at the start across different scalings.}

% Figure environment removed

% Figure environment removed

\subsection{Self-supervised image representation learning}
\label{subsec:self-supervised}

Finally, we turn our attention to  distillation based
\acrfull{ssl}.
where the model \gls{ema} is the \emph{teacher}
\citep{DBLP:conf/nips/GrillSATRBDPGAP20,DBLP:journals/taslp/NiizumiTOHK23,DBLP:journals/corr/abs-2104-14294,DBLP:journals/corr/abs-2304-07193}.

We will use \gls{byol}
(\cite{DBLP:conf/nips/GrillSATRBDPGAP20}, \Cref{def:emaUpdateDefinition})\footnote{
The \gls{byol} \gls{ema} update (\Cref{eq:byol-ema-update}) uses $\rvtheta_{t+1}$ instead of our analyzed $\rvtheta_{t}$ (\Cref{eq:scalingRuleSummaryEquation}).
The effect upon the overall \gls{ema} update is $\mathcal{O}(\eta\times\beta_\rho)$ and so is captured by the \gls{ema} Scaling Rule (\Cref{def:ema-sr}).
}
for our investigation into scaling as it is well-studied \citep{DBLP:conf/icml/TianCG21,DBLP:journals/corr/abs-2302-04817}, relatively simple to implement due to minimal hyper-parameters, and obtains competitive results \citep{DBLP:conf/nips/GrillSATRBDPGAP20,DBLP:journals/corr/abs-2209-15589}.
Since \gls{byol} learns through self-referential distillation, momentum plays a significant role in optimization. 
We analyze: i) a ResNet-18 \citep{DBLP:conf/cvpr/HeZRS16} on CIFAR10 \citep{CIFAR10} (\Cref{fig:r18-byol}) using SGD (\Cref{def:sgd}); and ii) a \gls{vit}-B/16 \citep{DBLP:conf/iclr/DosovitskiyB0WZ21} on ImageNet1k using AdamW \citep{DBLP:conf/iclr/LoshchilovH19}.
A recipe for \gls{byol} using \glspl{vit} is provided in \Cref{app:byol-vit}. 


% Figure environment removed

% Figure environment removed

{\bf ResNet-18 on CIFAR-10}~
We begin with a ResNet-18 model and short training duration to enable quick iteration,
and an \gls{sgd} optimizer as it has as \emph{known} scaling rule. 
This allows us to probe the \gls{ema} Scaling Rule without potential confounders like poor gradient-based optimizer scaling\footnote{For competitive performance with the reference \gls{byol}
\citep{DBLP:conf/nips/GrillSATRBDPGAP20}
using a ResNet-50, adaptive optimization, and longer training duration, see 
\Cref{subsec:byol-additional}
and
\Cref{fig:r50-byol}.}. 

We observe that \emph{without} the \gls{ema} Scaling Rule, there is a  drop in test top-1 linear probe (\Cref{def:linear-probe}) performance compared to the baseline, whereas \emph{with} the \gls{ema} Scaling Rule, we closely match the baseline model until batch size 4096.
We show that this result is consistent for a range of base learning rates $\eta_B$ and momenta $\rho_B$ in \Cref{subsec:byol-sensitivity-analysis}.
At batch size 8192, we see a performance gap between the scaled model using the \gls{ema} Scaling Rule and the baseline.
We speculate that this is due to dynamics early in the \gls{byol} training process that are challenging to replicate at larger batch sizes.
To test, and potentially circumvent this, we introduce \emph{Progressive Scaling} (\Cref{def:progressive-scaling}).
\begin{definition}[Progressive Scaling, informal; see \Cref{subsec:dynamic-batch-scaling}] 
    Given batch size $B$ and hyperparameters at $B$, 
    slowly increase the batch size to the desired largest batch size during training.
    At any intermediate batch size $\hat B=\kappa B$, all hyperparameters are scaled according to their scaling rules.
    \label{def:progressive-scaling}
\end{definition}
We see that transitioning to the higher batch size \emph{during} the warmup period results in a model optimization trajectory that diverges from the baseline, whereas transitioning \emph{after} warmup results in matching final trajectories of the scaled and baseline models.
In summary, \emph{progressive scaling} allows us to match \gls{byol} dynamics at large batch sizes, provided we transition after the warmup period.
This observation is consistent with our hypothesis regarding \gls{byol} dynamics during warmup.


{\bf Vision Transformers on ImageNet1k}~
\label{subsec:vit_byol}
Progressive Scaling coupled with the \gls{ema} Scaling Rule is required when scaling \gls{byol} \glspl{vit} (\Cref{fig:vitb-byol}),
enabling baseline loss tracking to a batch size of 24,576. 
Perfect scaling fails at batch size 32,768,
 consistent with observations in supervised learning 
\citep{DBLP:journals/corr/GoyalDGNWKTJH17,DBLP:conf/aaai/HuoGH21}.
Despite the breakdown, there is only a small drop in 1.6\% probe performance when using the \gls{ema} Scaling Rule, compared to as 44.56\% drop \emph{without} it.
We also observe that it is sometimes possible to match test model performance using \emph{only} Progressive Scaling and \emph{not} the \gls{ema} Scaling Rule, although this still induces a training loss mismatch.
We stress that such an approach is \emph{not} guaranteed to work and discuss when this approach succeeds and fails in \Cref{app:byol-progressive-scaling-regimes} and \Cref{fig:robustness-cartoon}.


At the transition point between batch sizes, 
an impulse perturbation\footnote{Instead of a single large batch transition as in \Cref{fig:vitb-byol} we perform a sequential transition in \Cref{app:byol-waterfall}. We find that a slow increase in batch size minimizes the magnitude of the perturbation and leads to a final model with higher effective linear probe top-1 than the reference by approximately $1.17\%$.} is measured at the student, visible from the training loss. 
This 
is recovered from by the learning process, 
and the new model matches the reference batch size. 
This perturbation happens in both the AdamW and \gls{sgd} settings, 
leading us to 
suspect
this is due to the \gls{byol} learning process, rather than an artifact of optimizer or momentum scaling. However, since this is not directly related to the EMA Scaling Rule proposed in this work, we defer this analysis to future investigation.




















\section{Related work}
\label{sec:related-work}


{\bf Optimizer scaling rules from \glspl{sde}}~
The \gls{sde} perspective has uncovered optimizer scaling rules and allowed an understanding of their limitations.
\citet{DBLP:conf/iclr/SmithL18} used \glspl{sde} to uncover the \gls{sgd} Scaling Rule, while 
\citep{DBLP:conf/nips/LiMA21} used \glspl{sde} to explain that rule's breakdown in terms of discretization error.
The \gls{sde} analysis was extended to adaptive optimization by
\citep{DBLP:conf/nips/MalladiLPA22}, producing an Adam Scaling Rule (\Cref{def:adam-sr}),
indicating that along with the learning rate, the $\beta_{1,2}$ and $\epsilon$ parameters transform.
The $\beta_{1,2}$ transformation is consistent with the \gls{ema} Scaling Rule in the \gls{sde} limit.
Our work differs as it considers a model EMA that alters the objective.

{\bf Varying the batch size during training}~
\citet{DBLP:conf/iclr/SmithKYL18}
investigated the benefits of scheduling the batch size at a fixed learning rate as an alternative to scheduling the learning rate at a fixed batch size.
These two are equivalent through the \gls{sgd} Scaling Rule.
The authors \emph{do not} scale the optimizer hyperparameters during this procedure, as they are intentionally replicating the training dynamics of a learning rate schedule.
This is in contrast with \emph{Progressive Scaling} (\Cref{def:progressive-scaling}) which scales the hyperparameters to \emph{maintain} the optimization process at different levels of discretization.

{\bf Large batch training of SSL distillation methods} 
\gls{ssl} methods learn representations without labels, meaning they can take advantage of web-scale data.
Large batch optimization is required to make use of this data in a reasonable amount of time.
\citet{DBLP:conf/nips/GrillSATRBDPGAP20} demonstrated algorithmic robustness when \emph{reducing} the batch size through gradient accumulation and EMA update skipping, which implements an approximation of our \gls{ema} Scaling Rule for $\kappa<1$.
Our work provides a recipe to scale down \emph{and up} in $\kappa$. 
MoCo-v3 \citep{DBLP:conf/iccv/ChenXH21} enables contrastively distilled \gls{vit}s up to a batch size of 6144, where the model drops in performance. 
More recently, methods like DINO \citep{DBLP:conf/nips/CaronMMGBJ20} present a worse scenario, and are unable to scale beyond batch size 1024 \citep{DBLP:journals/corr/abs-2209-15589}.
In contrast, our work presents practical tools to scale to large batch sizes in the presence of an \gls{ema}, enabling practical training of these \gls{ssl} methods on large scale data.




\section{Conclusion}
\label{sec:conclusion}






We provide an \gls{ema} Scaling Rule: when changing the batch size by a factor of $\kappa$,  exponentiate the momentum of the \gls{ema} update to the power of $\kappa$.
This scaling rule should be applied in addition to optimizer scaling rules
(for example, linearly scaling the SGD learning rate),
and
enables the scaling of methods which rely on \gls{ema} and are sensitive to the choice of \gls{ema} momentum.

We prove the validity of the \gls{ema} Scaling Rule by deriving 
first-order \gls{sde} approximations of discrete model optimization when a model \gls{ema} is present and can contribute to the model objective.
We demonstrate empirical support 
for a variety of uses of \gls{ema}, ordered by increasing influence of the role of \gls{ema} on the optimization procedure: supervised model tracking (i.e. Polyak-Ruppert averaging) in speech and vision domains, pseudo-labeling in speech, and self-supervised image representation learning.
In almost all scenarios, using the \gls{ema} Scaling Rule
enables matching of training dynamics under batch size modification, whereas not using it results in significant differences in optimization trajectories. 
For example, we can scale the \gls{byol} self-supervised method to a batch size of 24,576 without any performance loss \emph{only} when using the \gls{ema} Scaling Rule.

While learning rate scaling rules are relatively commonplace in \gls{ml}, 
the role of \gls{ema} has been overlooked.
With this work, 
we highlight the importance of scaling the \gls{ema} momentum,
and hope that future works will use the \gls{ema} Scaling Rule to scale the \gls{ema} momentum correctly, in the same way that learning rates and other optimizer hyperparameters are scaled.

\ifthenelse{\equal{\anonymous}{0}}{\section{Acknowledgements}
\label{sec:acknowledgements}

We thank
Miguel Sarabia del Castillo,
Adam Golinski,
Pau Rodriguez Lopez,
Skyler Seto,
Amitis Shidani,
Barry Theobald,
Floris Weers,
Luca Zappella, and
Shaungfei Zhai
for their helpful feedback and critical discussions throughout the process of writing this paper;
Okan Akalin,
Hassan Babaie, 
Mubarak Seyed Ibrahim, 
Li Li, 
Cindy Liu, 
Rajat Phull,
Evan Samanas, 
Guillaume Seguin, 
and the wider Apple infrastructure team for assistance with developing scalable, fault tolerant code; 
and 
Kaifeng Lyu and
Abhishek Panigrahi
for discussion and details regarding scaling rules for adaptive optimizers.
Names are in alphabetical order by last name within group.
}{}

%\bibliography{libraries/autobib,libraries/manualbib}
\bibliography{main}
\bibliographystyle{templates/iclr2021/iclr2021_conference}



\bibliography{main}
\bibliographystyle{icml2022}


%%%%%%%%%%%%%%%%%%%%%%%%%%%%%%%%%%%%%%%%%%%%%%%%%%%%%%%%%%%%%%%%%%%%%%%%%%%%%%%
%%%%%%%%%%%%%%%%%%%%%%%%%%%%%%%%%%%%%%%%%%%%%%%%%%%%%%%%%%%%%%%%%%%%%%%%%%%%%%%
% APPENDIX
%%%%%%%%%%%%%%%%%%%%%%%%%%%%%%%%%%%%%%%%%%%%%%%%%%%%%%%%%%%%%%%%%%%%%%%%%%%%%%%
%%%%%%%%%%%%%%%%%%%%%%%%%%%%%%%%%%%%%%%%%%%%%%%%%%%%%%%%%%%%%%%%%%%%%%%%%%%%%%%
\newpage
\appendix
\onecolumn
\begin{comment}
\section{System Architecture}
\label{appendix:architecture}
\system has a novel modularized system architecture with three key components: 
\emph{StreamManager}, 
\emph{TxnManager} and \emph{TxnScheduler}. 
These components are instantiated in each thread locally.
The execution outline of \system is presented in Algorithm~\ref{alg:algo}.
Transactional stream processing is continuous and potentially never ends (Line 1$\sim$8).
The dependency resolution and execution of state transactions are separated into two non-overlapping phases by punctuations~\cite{Tucker:2003:EPS:776752.776780} (Line 2 and 5), which guarantees that no subsequent input event will have a smaller timestamp. 
Effectively, a batch of state transactions is collected during the first phase, and processed during the second phase.

In the first phase (i.e., stream processing phase), 
the \emph{StreamManager} conducts preprocessing for every input event ($e$). Similar to some prior works~\cite{tstream}, state transactions may be issued but not immediately processed during preprocessing (Line 3).
The \emph{pre\_processing} and \emph{post\_processing} functions are exposed as APIs to users.
The \emph{TxnManager} handles dependency resolution (Line 4) among state transactions and insert decomposed operations to construct a \tpg. We discuss the detailed two-phase \tpg construction process in Section~\ref{subsec:construction}.

In the second phase  (i.e., transaction processing phase), 
the \emph{TxnManager} is first involved again to refine (Line 6) the constructed \tpg with further dependency resolution.
The \emph{TxnScheduler} 
schedules operations for concurrent execution based on the constructed \tpg according to the three dimensions of scheduling decisions (Line 7). 
In particular, a scheduling decision model $M$ is instantiated based on the constructed \tpg (Line 14).
\textbf{\circled{1}} Guided by $M$, execution threads adopt an exploration strategy (Section~\ref{subsec:explore}) to explore the constructed \tpg for operations available to be scheduled constrained by dependencies. 
\textbf{\circled{2}} 
During exploration, one or multiple operations may be treated as the 
% basic 
unit of scheduling (Section~\ref{subsec:granularity}). 
Subsequently, \textbf{\circled{3}} every thread executes operation(s) in the unit of scheduling with various abort handling mechanisms (Section~\ref{subsec:abort_handling}).
Only when state transactions are processed (i.e., committed or aborted) can the associated input events be postprocessed (Line 8) by the \emph{StreamManager} based on transaction processing results.
\end{comment}

\begin{comment}
\begin{algorithm}
\footnotesize
    \KwData{$e$ \tcp{Input event}}
    \KwData{$txn_{ts}$ \tcp{State transaction}}
    \KwData{$G$ \tcp{The currently constructed TPG}}
    \While{!finish processing of input streams}{
        \eIf(\tcp*[h]{Phase 1}){\text{$e$ is not a $punctuation$}}{
                $txn_{ts}$ $\gets$ PRE\_Processing($e$)\;
                \textbf{TPG\_Construction}($G$, $txn_{ts}$)\; 
          }(\tcp*[h]{Phase 2}){
                \textbf{TPG\_Refinement}($G$)\; 
                \textbf{TXN\_Scheduling}($G$)\; 
                POST\_Processing()\;
          }
    }
    
    \SetKwFunction{FMain}{TPG\_Construction}
    \SetKwProg{Fn}{Function}{:}{}
    \Fn{\FMain{$G$, $txn_{ts}$}}{
        $O_{1..k}$ $\gets$ \textbf{Partition} $txn_{ts}$\;
        \ForEach{\text{operation $O_{i}$ $\in$ $O_{1..k}$}}{
            \textbf{Identify} its \ld\;
            $G$ $\gets$ $G$ + $O_{i}$ \;
        }
    }
    \SetKwFunction{FMain}{TPG\_Refinement}
    \SetKwProg{Fn}{Function}{:}{}
    \Fn{\FMain{$G$}}{
        \ForEach{\text{vertex $e_{i}$ $\in$ $G$}}{
            \textbf{Identify} its \td, \pd\;
        }
    }
    
    \SetKwFunction{FMain}{TXN\_Scheduling}
    \SetKwProg{Fn}{Function}{:}{}
    \Fn{\FMain{$G$}}{
        $M$ $\gets$ Instantiated with $G$;\tcp{A decision model}
        \While{!finish scheduling of $G$
        }{
          \textbf{\circled{2}} $Scheduling Unit$ $\gets$ \textbf{\circled{1}} \emph{Explore}($G$, $M$)\; 
            \textbf{\circled{3}} \emph{Execute with Abort Handling} ($Scheduling Unit$)\; 
        }
    }
  \caption{Execution Outline of \system}
  \label{alg:algo}
\end{algorithm}
\end{comment}
%%%%%%%%%%%%%%%%%%%%%%%%%%%%%%%%%%%%%%%%%%%%%%%%%%%%%%%%%%%%%%%%%%%%%%%%%%%%%%%
%%%%%%%%%%%%%%%%%%%%%%%%%%%%%%%%%%%%%%%%%%%%%%%%%%%%%%%%%%%%%%%%%%%%%%%%%%%%%%%


\end{document}


% This document was modified from the file originally made available by
% Pat Langley and Andrea Danyluk for ICML-2K. This version was created
% by Iain Murray in 2018, and modified by Alexandre Bouchard in
% 2019 and 2021 and by Csaba Szepesvari, Gang Niu and Sivan Sabato in 2022. 
% Previous contributors include Dan Roy, Lise Getoor and Tobias
% Scheffer, which was slightly modified from the 2010 version by
% Thorsten Joachims & Johannes Fuernkranz, slightly modified from the
% 2009 version by Kiri Wagstaff and Sam Roweis's 2008 version, which is
% slightly modified from Prasad Tadepalli's 2007 version which is a
% lightly changed version of the previous year's version by Andrew
% Moore, which was in turn edited from those of Kristian Kersting and
% Codrina Lauth. Alex Smola contributed to the algorithmic style files.
