\documentclass[conference]{IEEEtran}
\IEEEoverridecommandlockouts
% The preceding line is only needed to identify funding in the first footnote. If that is unneeded, please comment it out.

\usepackage{cite}
\usepackage{amsmath,amssymb,amsfonts}
\usepackage{subcaption}
\usepackage{algorithmic}
\usepackage{algorithm2e}
\RestyleAlgo{ruled} 
\usepackage{graphicx}
\usepackage{textcomp}
\usepackage{xcolor}
\newtheorem{definition}{Definition}
\newtheorem{theorem}{Theorem}[section]
\newtheorem{lemma}[theorem]{Lemma}
\def\BibTeX{{\rm B\kern-.05em{\sc i\kern-.025em b}\kern-.08em
    T\kern-.1667em\lower.7ex\hbox{E}\kern-.125emX}}
\newcommand{\yehuda}[1]{{ #1}}
%\newcommand{\remove}[1]{ \textcolor{red}{ #1} }
\newcommand{\remove}[1]{{}}

\begin{document}

\title{Explainable Disparity Compensation for Efficient Fair Ranking

\thanks{© 2024 IEEE. Personal use of this material is permitted. Permission from IEEE must be
obtained for all other uses, in any current or future media, including
reprinting/republishing this material for advertising or promotional purposes, creating new
collective works, for resale or redistribution to servers or lists, or reuse of any copyrighted
component of this work in other works}
}

\author{\IEEEauthorblockN{Abraham Gale}
\IEEEauthorblockA{\textit{Department of Computer Science} \\
\textit{Rutgers, the State University of New Jersey}\\
Piscataway, USA \\
abraham.gale@rutgers.edu}
\and
\IEEEauthorblockN{Am\'elie Marian}
\IEEEauthorblockA{\textit{Department of Computer Science} \\
\textit{Rutgers, the State University of New Jersey}\\
Piscataway, USA \\
amelie.marian@rutgers.edu}
}
\maketitle
\IEEEpubidadjcol
\begin{abstract}
Ranking functions that are used in decision systems often produce disparate results for different populations because of bias in the underlying data. Addressing, and compensating for, these disparate outcomes is a critical problem for fair decision-making. Recent compensatory measures have mostly focused on opaque transformations of the ranking functions to satisfy fairness guarantees or on the use of quotas or set-asides to guarantee a minimum number of positive outcomes to members of underrepresented groups. 
In this paper we propose easily explainable data-driven compensatory measures for ranking functions. Our measures rely on the generation of bonus points given to members of underrepresented groups to address disparity in the ranking function. The bonus points can be set in advance, and can be combined, allowing for considering the intersections of representations and giving better transparency to stakeholders. We propose efficient sampling-based algorithms to calculate the number of bonus points to minimize disparity. We validate our algorithms using real-world school admissions and recidivism datasets, and compare our results with that of existing fair ranking algorithms.
\end{abstract}

\begin{IEEEkeywords}
Fair rankings, disparity compensation, explainability
\end{IEEEkeywords}
\section{Introduction}
Current quantum hardware is unable to carry out universal quantum computations due to the buildup of errors that occur during the computation. 
The magnitude of the individual error is currently above the value that the Threshold Theorem requires in order to kick-start quantum error correction and fault-tolerant quantum computation~\cite[Section 10.6]{nielsen_chuang_2010}. 
Although the experimentally achieved fidelity rates are promising and the error bounds are inching closer to the required threshold, we will have to work for the foreseeable future with quantum hardware with errors that build-up during the computation.  This implies that we can only do a limited number of steps before the output of the computation has become completely uncorrelated with the intended one.

For fault-tolerant quantum computing, we repeat four steps: 
1) We apply a number of single and two-qubit quantum gates, in parallel whenever possible; 
2) We perform a syndrome measurement on a subset of the qubits; 
3) We perform fast classical computations to determine which errors have occurred and how to correct them; 
and, 4) We apply correction terms based on the classical computations.
We then repeat these four steps with a next sequence of gates. 
These four steps are essential to fault-tolerant quantum computing. 


The starting point of this work is to use the four steps outlined above, not to carry out error correction and fault-tolerant computation, but to enhance short, constant-depth, {\em uncorrected} quantum circuits that perform single qubit gates and {\em nearest-neighbor} two qubit gates. 
Since in the long run we will have to implement error-correction and fault-tolerant computation anyhow, and this is done by such a four-step process, why not make other use of this architecture? Moreover, on some of the quantum hardware platforms, these operations are already in place.
Embracing this idea we naturally arrive at the question: what is the computational power of \textit{low-depth} quantum-classical circuits organized as in the four steps outlined above? 
We thus investigate circuits that execute a small, ideally constant, number of stages, where at each stage we may apply, in parallel, single qubit gates and {\em nearest-neighbor} two qubit gates, followed by measurements, followed by low-depth classical computations of which the outcome can control quantum gates in later stages. 
It is not clear, at first, whether such circuits, especially with constant depth, can do anything remotely useful. 
But we will see that this is indeed the case: many quantum computations can be done by such circuits in constant depth. 
By parallelizing quantum computations in this way, we improve the overall computational capabilities of these circuits, as we do not incur errors on qubits that are idle, simply because qubits are not idle for a very long time. 
Furthermore, reducing the depth of quantum circuits, at the cost of increasing width, allows the circuit to be run faster even if errors occur.

The first usage of such a four-step layout, not to do error correction, but to perform computations, can be found in the paradigm of measurement-based quantum computing~\cite{gottesman1999demonstrating,raussendorf2001one,jozsa2006introduction,clark2007generalised}: 
A universal form of quantum computing where a quantum state is prepared and operations are performed by measuring qubits in different bases, depending on previous measurements and intermediate measurements.

\citeauthor{PhamSvore2013} were the first to formalize the four-step protocol for performing computations~\cite{PhamSvore2013}. They included specific hardware topologies by considering two-dimensional graphs for imposing constraints on qubit interactions. In their model, they develop circuits for particularly useful multi-qubit gates, including specifying costs in the width, number of qubits, depth, number of concurrent time steps, size, and total number of non-Identity operations.
As a result, they find an algorithm that factors integers in polylogarithmic depth.
\citeauthor{Browne:2011} showed that the main tool in the work by \citeauthor{PhamSvore2013}, the fan-out gate, can also be replaced by additional log-depth classical computations in the measurement-based quantum computing setting~\cite{Browne:2011}.

More recently, \citeauthor{Cirac:2021} introduced a scheme to implement unitary operations involving quantum circuits combined with Local Operations and Classical Communication ($\mathsf{LOCC}$) channels: $\mathsf{LOCC}$-assisted quantum circuits~\cite{Cirac:2021}. Similarly to the four-step scheme we just described, they allow for a short depth circuit to be run on the qubits, followed by one round of $\mathsf{LOCC}$, in which ancilla qubits are measured and local unitaries are applied based on the measurement outcomes. They show that in this model any 1D transitionally invariant matrix-product state (MPS) with fixed bond dimension is in the same phase of matter as the trivial state. Similar ideas can be found in~\cite{TVV_NonAbelianTopologicalOrder_2022, tantivasadakarn2021long}.

In this work, we introduce a new model, called \textit{Local Alternating Quantum-Classical Computations} ($\LAQCC$). In this model we alternate between running quantum circuits (constrained by locality), ending in the measurement of a subset of qubits, and fast classical computations based on the measurement results. The outcome of the classical computations are then used to control future quantum circuits. We allow for flexibility in this model, by giving different constraints to the power of both the quantum circuits and the classical circuits as well as the number of alternations between them. 
Most attention will be given to $\LAQCC$ containing quantum circuits of constant depth, classical circuits of logarithmic depth and at most a constant number of alternations between them. 
Any circuit constructed in this model is considered to be of constant depth. 
We restrict ourselves to logarithmic depth classical computations, as this is the first natural and non-trivial extension beyond constant-depth classical computations. 
Constant-depth classical computations do however also have an equivalent constant-depth quantum implementation.

The definition of $\LAQCC$ sharpens the original definition of \citeauthor{PhamSvore2013} by adding constraints to the intermediate classical computations. This allows us to bound the power of $\LAQCC$ from above. 

The main result of \citeauthor{Cirac:2021}, that 1D translational invariant MPS with fixed bond dimension can be prepared by $\mathsf{LOCC}$-assisted circuits, relies on local symmetries of the MPS. These symmetries allow them to prepare local states (on a constant number of qubits) and glue them together by doing one round of the appropriate entangling measurement and corrections, after which they run a round of local unitaries to get the desired result. This general scheme for preparing states that exhibit an MPS description with the appropriate local symmetries requires only geometrically local unitaries and one round of measurement and corrections an therefore is accessible in $\LAQCC$. Studying different local symmetries, known as Symmetry Protected Topological (SPT) phases of matter, to find measurement-based constant depth circuits for states is a broad ongoing field of research~\cite{TVV_NonAbelianTopologicalOrder_2022, tantivasadakarn2021long, smith2023deterministic}. 
All these schemes have a $\LAQCC$ implementation.

%$\LAQCC$-circuits also exist for general schemes of preparing local states, based on the local tensors, and gluing them together using one round of entangled measurement and corrections, based on the local symmetry. 
%The main result of \citeauthor{Cirac:2021}, that 1D translational invariant MPS with fixed bond dimension can be prepared by $\mathsf{LOCC}$-assisted circuits, relies heavily on local symmetries of the MPS and as a result also has an equivalent $\LAQCC$ implementation. 
%The corrections applied after the measurement round are local unitaries depending on the local symmetries of the MPS. 

 

%This general scheme of preparing local states, based on the local tensors, and gluing it together by doing one round of entangled measurement and corrections, based on the local symmetry, is accessible in $\LAQCC$.
Note however that \citeauthor{Cirac:2021} also suggest a circuit for the $W$-state.
This circuit uses sequentially and dependent measurement-based corrections of the ancilla qubits. 
These dependent measurements translate to sequential alternations between the quantum and classical circuits and therefore increase the total depth to linear depth, exceeding the constant-depth constraints imposed by $\LAQCC$-circuits. 

We study the power of the $\LAQCC$ model with respect to state preparation, showing that even with only constant quantum-depth and logarithmic classical depth it remains possible to prepare states with long-range entanglement.
Another surprising result is that it is unlikely that $\LAQCC$ circuits are classically simulatable. We show that any instantaneous quantum polynomial-time (IQP) circuit~\cite{Bremner2010,Shepherd2009} has an $\LAQCC$ implementation.
Classical simulation of IQP circuits implies the collapse of the polynomial hierarchy to the third level, which is not believed to be true~\cite{Bremner2017}. Therefore, we expect that $\LAQCC$ circuits are unlikely to be classically simulatable. We bound the power of $\LAQCC$ by showing that it is contained in $\QNC^1$, the class of polynomial-size, log-depth circuits.

Next, we also study the power that intermediate classical calculations can add to quantum computations, by considering a new model that alternates between polynomially many polynomial-depth quantum circuits and unbounded classical computations
We study this model by doing a complexity theoretical analysis, where we draw inspiration from the notions of complexity given by \citeauthor{RosenthalYuen:2022}, \citeauthor{MetgerYuen:2023}, and \citeauthor{Aaronson:2004}.
All three complexity notions are based on the notion of state preparation, instead of more traditional definition of complexity such as the decidability of a computational problem. 
The first two consider classes based on sequences of quantum states preparable by a polynomial-sized quantum circuit, where the circuits are uniformly generated by a computational class, for instance, the class $\mathsf{PSPACE}$, which results in the complexity class $\mathsf{StatePSPACE}$~\cite{RosenthalYuen:2022,MetgerYuen:2023}.
The third notion considers a relative complexity, where the complexity is measured between two given states, and is measured by the number of gates, from a given gate-set, required to transform one state in another state~\cite{Aaronson:2004}. 
For our definition of state preparation complexity, we drop the uniformity constraint from~\cite{RosenthalYuen:2022,MetgerYuen:2023} and define a class as $\mathsf{StateX}$, which refers to states preparable by circuits of type $\mathsf{X}$. 
As an example, if $\mathsf{X} = \QNC^0$, this results in the class $\mathsf{StateQNC^0}$, which is the set of states preparable from the $\ket{0}^n$ state by poly-size constant-depth circuits. 
This notion is similar to the relative complexity from~\cite{Aaronson:2004}, where one state is the  $\ket{0}^n$ state and instead of counting the number of gates we consider the set of states preparable by a fixed number of gates. Using this notion of complexity we show that any state preparable by an $\LAQCC^*$ circuit is also preparable by a $\mathsf{PostQPoly}$ circuit, the class of circuits of polynomial depth with an additional post-selection gate. 

All Clifford circuits have a constant-depth $\LAQCC$ implementation, implying that any stabilizer state can be implemented by a constant-depth $\LAQCC$ circuit, see Section~\ref{sec:clifford_circuits} for a proof of this statement. 
Efficient circuits for stabilizer states have been known already through measurement-based quantum computing. Therefore this paper focuses on the preparation of non-stabilizer states, and as a surprising result we find novel constant-depth protocols for four very natural classes of non-stabilizer states.
Despite the extensive research into these four classes of non-stabilizer states and the many applications of them, no efficient constant- or low-depth state preparation protocols are known yet. We specifically consider these four classes as they are all often used as initial states in other algorithms.

The first state is a uniform superposition over an arbitrary number of states. 
This state finds applications in many quantum algorithms, as they often start with a uniform superposition over multiple states. 
This superposition is often achieved by applying Hadamard gates to every qubit due to its simplicity to prepare. 
Yet, the analysis of many algorithms, such as Shor's algorithm~\cite{Shor:1997}, would benefit from a different initial superposition. 
The circuit to prepare the uniform superposition over an arbitrary number of states uses an exact version of Grover search as a subroutine, that turns a probabilistic circuit, with a known constant probability of success, into a deterministic circuit. 
We use the circuit for preparing a uniform superposition over an arbitrary number of states as a subroutine in the next two quantum state preparation protocols. 

The second state is the $W$-state, the uniform superposition over all computational basis states of Hamming-weight~$1$, a natural long-ranged entangled state that displays a fundamentally nonequivalent type of entanglement from the Greenberger–Horne–Zeilinger state~\cite{WState:2000}, for which $\LAQCC$-type constant-depth circuits were previously known~\cite{PhamSvore2013, Cirac:2021}. 
The $W$-state is often used as benchmark for new quantum hardware~\cite{Haffner2005,Neeley2010,GarciaPerez:2021}. 
A novel way to prepare the $W$-state therefore gives a new way to benchmark different quantum devices with each other. 
A circuit for preparing the $W$-state was given in~\cite{Cirac:2021}, but this implementation requires sequentially alternating measurements followed by local unitaries, which in the $\LAQCC$ model is not considered to be of constant depth. 
We improve this protocol by giving an $\LAQCC$ implementation of the $W$-state, based on a compress-uncompress method that links the one-hot and binary encoding of integers.

The third state considered is the Dicke state, a generalization of the $W$-state, a superposition over all computational basis states with Hamming-weight $k$~\cite{Dicke:1954}. 
Dicke states have relevance in various practical settings.
For instance, for quantum game theory~\cite{zdemir2007}, quantum storage~\cite{Bacon_Compress:2006,Plesch:2010}, quantum error correction~\cite{ouyang2014permutation}, quantum metrology~\cite{toth2012multipartite}, and quantum networking~\cite{prevedel2009experimental}. 
Dicke states have been used as a starting state for variational optimization algorithms, most notably Quantum Alternating Operator Ansatz (QAOA)~\cite{Hadfield2019}, to find solutions to problems such as Maximum k-vertex Cover~\cite{Brandhofer2022,cook2020quantum}.
The ground states of physical Hamiltonians describing one-dimensional chains tend to show a resemblance to Dicke states such as states resulting from the Bethe ansatz, making them an ideal starting state when investigating the ground state behavior of these Hamiltonians~\cite{TDL_BetheAnsatzDerivation:2010,B_ExcitedStateQuantumPhaseTransitions:2013,DickeTransitions:2021}. 
For instance, the algorithm by \citeauthor{van2021preparing}, who give an algorithm to prepare the Bethe ansatz eigenstates of the spin-1/2 XXZ spin chain, starts by first preparing a Dicke state~\cite{van2021preparing}. 
A Dicke-state preparation protocol based on the compress-uncompress methodology used in the $W$-state furthermore finds applications in entanglement distillation, where the entanglement of a large state is concentrated on only a few qubits. 
Efficient deterministic circuits for preparing Dicke states have been proposed by \citeauthor{bartschi2019deterministic}~\cite{bartschi2019deterministic, bartschi2022deterministic_short_depth}. 
They provide a quantum circuit of depth $\mathO(k \log(\frac{n}{k}))$, allowing arbitrary connectivity, to prepare a Dicke state, which they conjecture to be optimal when $k$ is constant. 
In this work, we provide a constant-depth $\LAQCC$ circuit below their conjectured bound already for constant $k$. 
However, this does not directly disprove their conjecture, as we allow for intermediate measurements and classical computations. 
More significantly, we even construct constant-depth $\LAQCC$ circuits for $k = \mathO(\sqrt{n})$ greatly improving their bound.
This construction extends the compress-uncompress method for the $W$-state combined with additional subroutines. 

We continue with a log-depth state preparation protocol for the Dicke-state for arbitrary $k$. 
This protocol implements an efficient transformation between the factoradic number representation and the combinatorial number representation of a positive integer. 
The combinatorial number representation relates directly to the Dicke state. 
The provided efficient transformation between number representation systems might be of independent interest. 

We conclude by modifying our protocol for preparing a Dicke-state to a protocol that prepares quantum many-body scar states in constant-depth. 
These states have low entanglement and longer coherence times than states with similar energy density.
These characteristics make many-body scar states interesting to analyze and relevant within physics.
Many-body scar states appear for instance in the AKLT model~\cite{AKLT:1987,MRBAR:2018,MRB:2018} and different spin models~\cite{SI:2019,MOBFR:2020}.
Known methods for preparing these states have polynomial-depth~\cite{Gustafson:2023}, whereas our circuit has constant depth. 

% We conclude by studying the power that intermediate classical calculations can add to quantum computations. 
% In this study, we define a new model that relaxes constant-depth quantum circuits to polynomial depth quantum circuits, log-depth classical calculations to unbounded classical computations and a constant number of alternations to a polynomial number of alternations. 
% We call this model $\LAQCC^*$. 
% We study this model by doing a complexity theoretical analysis, where we draw inspiration from the notions of complexity given by \citeauthor{RosenthalYuen:2022}, \citeauthor{MetgerYuen:2023}, and \citeauthor{Aaronson:2004}.
% All three complexity notions are based on the notion of state preparation, instead of more traditional definition of complexity such as the decidability of a computational problem. 
% The first two consider classes based on sequences of quantum states preparable by a polynomial-sized quantum circuit, where the circuits are uniformly generated by a computational class, for instance, the class $\mathsf{PSPACE}$, which results in the complexity class $\mathsf{StatePSPACE}$~\cite{RosenthalYuen:2022,MetgerYuen:2023}.
% The third notion considers a relative complexity, where the complexity is measured between two given states, and is measured by the number of gates, from a given gate-set, required to transform one state in another state~\cite{Aaronson:2004}. 
% For our definition of state preparation complexity, we drop the uniformity constraint from~\cite{RosenthalYuen:2022,MetgerYuen:2023} and define a class as $\mathsf{StateX}$, which refers to states preparable by circuits of type $\mathsf{X}$. 
% As an example, if $\mathsf{X} = \QNC^0$, this results in the class $\mathsf{StateQNC^0}$, which is the set of states preparable from the $\ket{0}^n$ state by poly-size constant-depth circuits. 
% This notion is similar to the relative complexity from~\cite{Aaronson:2004}, where one state is the  $\ket{0}^n$ state and instead of counting the number of gates we consider the set of states preparable by a fixed number of gates. Using this notion of complexity we show that any state preparable by an $\LAQCC^*$ circuit is also preparable by a $\mathsf{PostQPoly}$ circuit, the class of circuits of polynomial depth with an additional post-selection gate. 

\paragraph{Summary of results}
\begin{itemize}
    \item We give a new definition of a computational model that captures the power of the four step process: applying a constant number of layers of one- and two-qubit gates; performing a syndrome measurement; perform a fast classical computation determining corrections; apply corrections. We call this model \emph{Local Alternating Quantum Classical Computations}, or $\LAQCC$ for short. In this model we bound the allowed quantum operations, intermediate classical calculations, and number of rounds separately. In Section~\ref{sec:LAQCC_model} we define this model and give a list of operations based on results from literature contained in this computational model. In some of these operations we explicitly use that we allow for multiple, but at most constant, rounds  of corrections.
    \item  We show show that there exist $\LAQCC$ circuits that can not be weakly simulated in Section~\ref{sec:IQP_in_LAQCC}. We further show that for every $\LAQCC$ circuit there exists a $\QNC^1$ circuit simulating it perfectly, in Section~\ref{sec:LAQCC_in_QNC1}.
    \item We introduce a new type computational complexity for preparing states and show that the extension of $\LAQCC$ where we allow a polynomial number of rounds and unbounded classical computation, is contained in $\mathsf{PostQPoly}$, the class of polynomial circuits with post-selection, in Section~\ref{sec:Complexity results}.
    \item We show a protocol to prepare the uniform superposition state of size $q$ in $\LAQCC$ using $\mathO(\ceil{\log_2(q)}^2)$ qubits in Section~\ref{sec:superposition_modulo_q}. 
    \item We show a protocol to prepare the $W_n$ state in $\LAQCC$ using $\mathO(n\log(n))$ qubits in Section~\ref{sec:W_state_in_LAQCC}.
    \item We show two ways of preparing the Dicke-$(n,k)$ state. The first method is in $\LAQCC$, works up to $k = \mathO(\sqrt{n})$, uses $\mathO(n^2\log(n))$ qubits, and is found in Section~\ref{sec:dicke:small_k}. The second method is in $\LAQCC\text{-}\mathsf{LOG}$ (an extension of $\LAQCC$ allowing for logarithmic number of alterations instead of constant), works for any $k$, uses $\mathO(\text{poly}(n))$ qubits, and is found in Section~\ref{sec:Dicke_in_LAQCC_LOG}. 
    \item We extend on our $\LAQCC$ method of generating Dicke-$(n,k)$ states for $k = \mathO(\sqrt{n})$ and show a protocol to generate many-body scar states for a particular Hamiltonian in $\LAQCC$ (Section~\ref{sec:many_body_scar}). 
\end{itemize}
Summarized in a table, we provide the following state generation protocols:
\begin{table}[htb]
\centering
\begin{tabular}{l|l|l|l}
\textbf{State description} & \textbf{Width} & \textbf{Depth} & \textbf{Implementation}\\
\hline 
Uniform superposition mod $q$: $\frac{1}{\sqrt{q}} \sum_{i = 0}^{q-1}\ket{i}$ & $\mathO(\ceil{\log^2 q})$ & $\mathO(1)$ & Section~\ref{sec:superposition_modulo_q}\\

$W$-state: $\frac{1}{\sqrt{n}}\sum_{i = 0}^{n-1}\ket{e_i}$ & $\mathO(n \log n)$ & $\mathO(1)$ & Section~\ref{sec:W_state_in_LAQCC}\\

Dicke-$(n,k)$, $k = \mathO(\sqrt{n})$: $\binom{n}{k}^{-1/2}\sum_{x \in \{0,1\}^n: |x| = k} \ket{x}$ &  $\mathO(n^2\log n)$ & $\mathO(1)$ 
&Section~\ref{sec:dicke:small_k}\\

Dicke-$(n,k)$: $\binom{n}{k}^{-1/2}\sum_{x \in \{0,1\}^n: |x| = k} \ket{x}$ & $\mathO(\text{poly}(n))$ & $\mathO(\log n)$ &Section~\ref{sec:Dicke_in_LAQCC_LOG}\\

QMBS: $\ket{S_k} = \frac{1}{k! \sqrt{\mathcal N(n,k)}}(Q^\dagger)^k \ket{\Omega}$ &  $\mathO(n^2\log n)$ & $\mathO(1)$  &  Section~\ref{sec:many_body_scar}
\end{tabular}
\caption{Summary of state preparation protocols given in this paper.}
\label{tab:sate_prep}
\end{table}
In the entry for the quantum many-body scar state $Q$ denotes the raising operator and $\mathcal N(n,k)=\binom{n-k-1}{k}$. 
Section~\ref{sec:many_body_scar} will provide more details on the variables and the implementation. 

\paragraph{Organization of the paper}
\noindent We first introduce relevant preliminaries in Section~\ref{sec:preliminaries}. 
In Section~\ref{sec:LAQCC_model} we formally define the class of Local Alternating Quantum-Classical Computations ($\LAQCC$). We also show that any Clifford circuit can be implemented in constant depth $\LAQCC$ (a result based on a result from measurement-based quantum computing~\cite{jozsa2006introduction}). 
This result allows us to give many useful multi-qubit gates and routines in Section~\ref{sec:gates_created_in_LAQCC}. 
Beyond that we show that constant depth $\LAQCC$ circuits are contained in $\QNC^1$ and that any $\mathsf{IQP}$ circuit has an $\LAQCC$ implementation.
We conclude this section with an analysis of a more powerful instantiation of $\LAQCC$ and show an inclusion with respect to the class $\mathsf{PostQPoly}$, which is the class of circuits of polynomial depth with one additional post-selection gate. 
In Section~\ref{sec:state_prep_in_LAQCC} we give $\LAQCC$ circuit implementations for preparing the uniform superposition over an arbitrary number of states, the $W$-state and the Dicke state up to $k = \mathO(\sqrt{n})$. We furthermore give a log-depth circuit implementation for preparing the Dicke state for any $k$. We conclude by showing a $\LAQCC$ circuit for generating many body scar states of a particular type of Hamiltonian.


\section{Related Work}
%\subsection{Cost Volume based Deep Stereo Matching}
%Stereo matching is a typical problem that has been studied for decades and a well-known four-step pipeline \cite{scharstein2002taxonomy} has been established, where cost volume construction is an indispensable step. Current state-of-the-art stereo matching methods are all cost volume based methods and they can be categorized into two types. Typically, a cost volume is a 4D tensor of height, width, disparity, and features. The first category just uses a full correlation to generate a single-feature cost volume. Such methods are usually efficient but lose much information because of the decimation of feature channels. Many previous work, including Dispnet \cite{dispnet}, MADNet \cite{madnet}, IResNet \cite{iresnet} and AANet \cite{aanet}, belong to this category. The second category usually uses concatenation \cite{gcnet} or group-wise correlation \cite{gwcnet} to generate a multi-feature 4D cost volume. Such a method can achieve better performance while requiring higher computational complexity and memory consumption. Actually, a majority of the top-performing networks in public leaderboards belong to this category, such as GANet \cite{ganet}, CSPN \cite{cspn} and ACFNet \cite{acfnet}. These methods generally employ multiple 3D convolution layers to constantly regularize the 4D cost volume and then apply softmax over the disparity dimension to produce a discrete disparity probability distribution. The final predicted disparity is obtained by softly weighting indices according to their probability, which is also called soft argmin in GCNet \cite{gcnet}. However, soft argmin leaves the output susceptible to multi-modal disparity probability distributions. ACFNet \cite{acfnet} observes this problem and proposes to directly supervise the cost volume with unimodal ground truth distributions. In contrast, we define an uncertainty estimation to quantify the degree to which the cost volume tends to be multi-modal distribution, higher implies the higher possibility of estimation error.

\subsection{Multi-scale Cost Volume based Stereo Matching}
Cost volume construction is an indispensable step in the well-known four-step pipeline for stereo matching \cite{scharstein2002taxonomy, pamisurvey1, pamisurvey2}. Typically, current state-of-the-art stereo matching methods can be categorized into two types of cost volume-based methods, where the cost volume is a 4D tensor of height, width, disparity, and features. The first category usually uses the single-feature 3D cost volume generated by full correlation, which is efficient while losing much information due to the decimation of feature channels. Many real-time methods, such as Dispnet \cite{dispnet}, MADNet \cite{madnet, madnet_pami} and AANet \cite{aanet}, belongs to the category. Moreover, two-stage refinement \cite{mcvmfc} and pyramidal towers \cite{madnet} are commonly applied in the single-feature cost volume based network to construct multi-scale cost volume. The second category usually uses the multi-feature 4D cost volume generated by concatenation \cite{gcnet} or group-wise correlation \cite{gwcnet}, which can achieve better performance with higher computational complexity and memory consumption. Most top-performing networks, including GANet \cite{ganet}, CSPN \cite{cspn} and ACFNet \cite{acfnet} belong to this category. 
% In these methods, the 4D cost volume is constantly regularized by multiple 3D convolution layers and then a discrete disparity probability distribution can be produced by softmax. Next, the final predicted disparity can be obtained by softly weighting indices according to their probability \cite{gcnet}. However, such output is susceptible to multimodal disparity probability distributions and ACFNet \cite{acfnet} gives a solution by directly supervising the cost volume with unimodal ground truth distributions to alleviate this problem. 
Recently, to alleviate the high computational complexity and memory consumption when employing multi-feature 4D cost volumes, \cite{cvpmvsnet, cascade, uscnet} propose to use cascade cost volume representation in multi-view stereo. These methods usually first predict an initial disparity at the coarsest resolution of the image and then gradually refine the disparity by narrowing down the disparity search space. More closely related to our approach is Casstereo \cite{cascade}, which first extended such representation to stereo matching. It selected to uniform sample a pre-defined range to generate the next stage’s disparity search range. Instead, we employ pixel-level uncertainty estimation to adaptively adjust the next stage disparity searching range and generate pseudo-labels for subsequent domain adaptation. Our method also shares similarities with UCSNet \cite{uscnet}, which constructs uncertainty-aware cost volume in multi-view stereo while it doesn’t employ uncertainty estimation to generate pseudo-labels.

%\subsection{Multi-scale Cost Volume based Deep Stereo Matching} 
% \subsection{Multi-scale Cost Volume based Stereo Matching} 
%Multi-scale cost volume firstly was applied in the single-feature cost volume based network with the form of two-stage refinement \cite{mcvmfc} and pyramidal towers \cite{madnet}. Recently, cascade cost volume representation \cite{cvpmvsnet, cascade, uscnet} was proposed in multi-view stereo to alleviate the high computational complexity and memory consumption when employing multi-feature 4D cost volumes. These methods generally predict an initial disparity at the coarsest resolution of the image. Then, they will narrow down the disparity search space and gradually refine the disparity. More closely related to our approach is Casstereo \cite{cascade}, which first extended such representation to stereo matching. It selected to uniform sample a pre-defined range to generate the next stage’s disparity search range. Instead, we employ uncertainty estimation to adaptively adjust the next stage pixel-level disparity searching range and push the next stage's cost volume to be predominantly unimodal.

% The single-feature cost volume based network with the form of two-stage refinement \cite{mcvmfc} and pyramidal towers \cite{madnet} first employ multi-scale cost volume for stereo matching. Recently, to alleviate the high computational complexity and memory consumption when employing multi-feature 4D cost volumes, \cite{cvpmvsnet, cascade, uscnet} propose to use cascade cost volume representation in multi-view stereo, which generally predict an initial disparity at the coarsest resolution of the image. Then, the disparity search space is narrowed down and the disparity is gradually refined. More closely related to our approach is Casstereo \cite{cascade}, which first extended such representation to stereo matching. It selected to uniform sample a pre-defined range to generate the next stage’s disparity search range. Instead, we employ uncertainty estimation to adaptively adjust the next stage pixel-level disparity searching range and push the next stage's cost volume to be predominantly unimodal.

% Figure environment removed

\subsection{Robust Stereo Matching} 
There exist three categories of generalization definitions for robust stereo matching. 1) Cross-domain Generalization: the network’s ability to perform well on unseen scenes (cannot see the image pairs of the target domain in advance). Towards this end, Jia et al \cite{sungeneralizaiton} propose to incorporate scene geometry priors into an end-to-end network. Zhang et al \cite{dsmnet} introduce a domain normalization and a trainable non-local graph-based filter to construct a domain-invariant stereo matching network. 2) Adapt Generalization: the network’s ability to adapt pre-trained models to the new domain with unlabeled target data. Previous work usually pre-trains the models on synthetic data and then adapts it to new target domains with Graph Laplacian regularization \cite{zoom}, non-adversarial progressive color transfer \cite{adastereo}, and Knowledge Reverse Distillation \cite{aohnet}. More closely related to our approach are \cite{aohnet, unsuperviseddomainadaptation} in stereo matching and Monoresmatch \cite{monoresmatch} in monocular depth estimation, which also proposes to generate a pseudo-label for domain adaptation. However, these methods all select to employ classical stereo matching methods \cite{sgm} alongside with confidence estimators, e.g., left-right consistency check to generate pseudo-labels. That is all these methods need an independent method to generate corresponding pseudo-labels. Instead, the proposed method is an end-to-end network that can generate the predicted disparity map, corresponding uncertainty map and pseudo-labels jointly, which is a more simple, yet efficient way. 
% Instead, our proposed method can employ pixel-level and area-level uncertainty estimation to self-distill the predicted disparity maps of our pre-training model and generate sparse while reliable pseudo-labels to align the domain gap, which is a more simple, yet efficient way. 
3) Joint Generalization: the network’s ability to perform well on a variety of datasets with the same model parameters. MCV-MFC \cite{mcvmfc} introduces a two-stage finetuning scheme to achieve a good trade-off between generalization and fitting capability on multiple datasets. However, it doesn’t touch the inner difference between diverse datasets, e.g, the unbalanced disparity distribution. To further address this problem, we propose a cascade cost volume to adaptively the next stage disparity searching space, where the pixel-level uncertainty estimation is at the core.

% \subsection{Monocular Depth Estimation}
% Monocular depth estimation aims to estimate depth values from a single image, instead of stereo images or multiple frames in a video. This problem is ill-posed because of the ambiguity of object sizes. However, humans could estimate the depth from a single image with prior knowledge of the scenes. Recently, learning based methods were explored to learn depth values by supervised or unsupervised learning. Eigen et al. first employed Convolutional Neural Networks (CNN) to predict depth in a coarse-to-fine manner and further improved its performance by multi-task learning. Liu et al. presented deep convolutional neural fields model by combining deep model with continuous CRF. Li et al. [22] refined deep CNN outputs with a hierarchical CRF. Multi-scale continuous CRF was formulated into a deep sequential network by Xu et al. [45] to refine depth estimation. Unsupervised methods tried to train monocular depth estimation with stereo
% image pairs or image sequences and test on single images. Garg et al. [9] used novel image view synthesis loss to train a depth estimation network in an unsupervised way. Godard et al. [11] introduced left-right consistency regularization to improve the performance of view synthesis loss. Recently, some work also propose to use the stereo matching network as a proxy to learn depth from synthetic data or directly employ traditional stereo matching methods to distill proxies labels from the target domain, which proves the feasibility of distilling stereo matching networks to learn monocular depth estimation.



\section{Background}
\label{sec:background}
Ranking functions are used in a wide variety of decision systems with high societal impacts: job recruiting tools, school admissions, allocation of resources (e.g., vaccines, treatments, public housing), or risk assessment (e.g., fraud, recidivism). Ensuring that these mechanisms are fair is critical. To this end, we propose a model of disparity compensation measures based on the allocation of targeted bonus points.

We introduce two motivating examples for our work in Section~\ref{sec:motivations}, and set the definitions and parameters of our problem in Section~\ref{sec:definitions}.


\subsection{Motivating Examples}
\label{sec:motivations}

\paragraph{\textbf{NYC High School Admissions}} NYC high school admissions use a deferred acceptance (DA) matching
algorithm~\cite{NYCmatching} similar to the stable marriage algorithm designed by Gale-Shapley~\cite{galeshapley}. The  algorithm  matches students to schools based on students'
preferences and the schools' admission-ranked lists (rubrics). Schools set their own ranking rubrics using metrics such as grades, test scores,  
absences, auditions, or interviews.  Such screens have become a topic of controversy, being targeted as discriminatory because
the underlying metrics often exhibit a high level of
disproportionality in the ranked lists they produce, as students from disadvantaged groups often score lower in some of the metrics used in the rubrics.  

Currently, the NYC Department of Education mostly relies on set-asides (soft quotas) for low-income students to address disparity and produce a diverse group of students at each school. These measures have had mixed results depending on the demographics of the geographical area of the school and on the patterns of student choices. These set-asides are mostly limited to low-income students, although other dimensions of disadvantage have been considered (current school, English language learner, student in temporary housing). However, as mentioned in Section~\ref{sec:related}, combining several quotas can be cumbersome and computationally expensive.

In this paper, we explore the use of compensatory “bonus points" assigned to students who exhibit one or more
dimensions of disadvantage. Our goal is to ensure adequate representation of the underlying population in the students selected by the school
admission rubrics. Because NYC uses a matching algorithm, it is not known in advance how far down its list a school
will accept students; our techniques can adjust to unknown values of the number of selected objects $k$ by minimizing the disparity over all values of $k$, logarithmically discounted to favor smaller $k$ values (Section~\ref{sec:log_discount}). Our experimental evaluation, using NYC high school admission data, shows that our compensatory metrics adapt well to multiple selection percentages (Section~\ref{sec:resschools}).

Some school systems have considered the use of a point-based scheme to diversify schools. In particular, Paris, France has shown good results in improving
socioeconomic diversity~\cite{affelnet} through the use of ``bonus" points for disadvantaged students. However, the system was based on ad hoc bonus points decided somewhat arbitrarily by policymakers, which created some undesirable outcomes in some schools when the points were not calibrated correctly: in one case a high school was assigned a large majority (83\%) of low-income students (instead of the statistical parity goal of 40\%), defeating the diversity purpose. In contrast, we propose a data-driven assignment of bonus points that best reflects the data distribution and its impact on the rankings.  



\paragraph{\textbf{COMPAS Recidivism Data}}
Recidivism algorithms, such as COMPAS, are used to predict the likelihood that a person interacting with the criminal justice system will re-offend, and are used by U.S. Courts to  assist in bail and sentencing decisions. The impact of these decision algorithms  is unquestionable, yet the opacity of the decisions makes it hard to verify that the process is fair.
 
 In 2016 a ProPublica investigation~\cite{angwin_larson_2016} argued that the COMPAS algorithm was unfair to a number of disadvantaged groups, particularly Black Americans. The internal COMPAS ranking algorithm has not been made public; ProPublica based its investigation on  Broward County, Florida data acquired through a public records request, which they made public. Subsequently, this COMPAS data has become a popular dataset for evaluating fairness mechanisms; according to the fairness survey in~\cite{zehlike2021fairness}, it is the most popular large dataset for fairness analysis.

While COMPAS is used for classifying subjects into categories, the categories (deciles) are based on an underlying ranking of subjects. The deciles scores are then often (mistakenly) used as absolute, and not relative, scores of recidivism. Because the scores are based on comparative data, they exacerbate underlying discriminatory practices.
The internal COMPAS algorithm is proprietary, its inner workings, and potential disparate treatments, have been the subject of dispute and speculation~\cite{Rudin2020Age,Jackson2020Setting}. Yet the disparate impact of the COMPAS decile scores
, as they are used in practice, is undeniable. In addition, the process is opaque and not easily understandable. We explore using our disparity compensation techniques in conjunction with the COMPAS scores to address the disparate impacts of the COMPAS tool and report on our results in Section~\ref{sec:rescompas}.

The use of the COMPAS dataset has been the subject of multiple criticisms regarding the ethical use of such data~\cite{COMPASmessy21}.
Our inclusion of COMPAS as a case study is by no means an endorsement of its use for real-life decisions, but rather an illustration of how our compensatory-based approach can help significantly reduce disproportionality on various types of ranking –and classification– functions, even when those are hidden behind black-box proprietary systems. 




\subsection{Definitions}
\label{sec:definitions}

The driving force behind our choices is a focus on explainability so that the fairness compensation choices are simple, transparent, and clearly understandable for stakeholders. This notion of
explainability is especially important to gain support from stakeholders~\cite{goeljustice}. We now define the format of our ranking functions (Section~\ref{sec:ranking functions}) and compensatory bonus points (Section~\ref{sec:bonuspoints}) and our choice of fairness metric (Section~\ref{sec:disparity}).


\subsubsection{Ranking Functions}
\label{sec:ranking functions}



We focus our explanation on score-based ranking functions. Our bonus points can also be adapted to other ranking functions by simulating an underlying score based on rank (see Section~\ref{sec:rescompas}).

\begin{definition}\textit{Score-Based ranking function}
\label{def:WeightedFunction}
We define a score-based ranking function $f$ over a set of $A$ attributes   $a_1,
...., a_A$, over an object $o$ as $f(o)=f(a_1,
...., a_A)$. A ranking process $R$  selects the $k$\% best objects with the highest $f(o)$ values as its answer $R_k$.

\end{definition}



Each object has a set of attributes $A$ that defines its properties and assigns values to them. For the purpose of this work, we recognize a special subset of attributes {\em fairness attributes} (also called protected attributes in the literature), which represent the dimensions on which we want to control for bias and disparate impact. {\em Fairness attributes} may be used by the ranking function $f$ to score the objects, or may not be involved in the ranking but still of interest for assessing the fairness of the outcome.

For example, a school may rank applicants using a 100-point scoring function based on a weighted sum of students' GPA and test scores (attributes). Fairness attributes may include low-income or disability status of the student.


\subsection{Bonus Points}
\label{sec:bonuspoints}

Our approach centers around bonus points to compensate for various dimensions of disparity. Bonus points are multiplied  with the corresponding fairness attribute value and added to the final ranking function score $f(o)$. When the fairness attribute value is binary, this is equivalent to adding the bonus to the final score when that value is equal to 1. For instance, if the low-income status of a school applicant is encoded as a \{0,1\} binary, a bonus of 2 points would add 2 to each low-income applicant's final score; if the low-income status is encoded as a continuous value in [0,1], then the bonus of 2 will be  multiplied by the value of the attribute to give a more precise disparity compensation tool.

We define bonus points as:
\begin{definition}\textit{Bonus Points}
\label{def:Bonus_points}
Given a vector of fairness attributes $\Vec{A_f}$ and a identically shaped vector of bonus points $\Vec{B}$ let the score of an object $o$ be defined as $f_b(o) = f(o) + \Vec{A_f} \cdot \Vec{B} $
\end{definition}

We require bonus points to be positive (negative for scenarios where a lower score is desirable). Negative bonus points would be perceived as a penalty and may not be easily accepted by stakeholders.

In addition to the flexibility of the mechanism, the advantages of using bonus points include: \textbf{intersectionality}, bonus points can be combined and compounded to account for multiple dimensions of bias; \textbf{transparency}, the extent and impact of the fairness intervention are clear to stakeholders; \textbf{comparability}, the score of objects can be easily adjusted and objects compared, increasing transparency and trust; \textbf{predictability}, combined with information on how the selection is done (e.g., historical threshold values), applicants can easily assess their chances and be provided with  \textbf{clarity} as to which actions or interventions are required for selection.

For instance, in our school admission scenario, bonus points could be used to capture the \textit{intersectionality} of students with disability and low-income students: students with both characteristics would receive more bonus points than students with one, or none. This information can \textit{transparently} be published before applications are due, giving clear and \textit{predictable}, and \textit{comparable}, information to families. Admission decisions are \textit{clarified}, with clear thresholds published and the participation of each ranking attribute and fairness compensatory bonus points identified for each applicant.




\subsection{Disparity}
\label{sec:disparity}

We focus on the explainable disparity  from~\cite{Gale2020ExplainingMR} as our target fairness metric as it aims at satisfying statistical parity~\cite{lahoti2019ifair}. Furthermore, it is easily interpretable by humans, behaves well even when the number of dimensions increases and can deal with dimensions with either continuous or discrete data.


Disparity is defined as the vector difference between the average selected object and the average unselected object. Formally it is defined as follows:
\begin{definition}\textit{Disparity}
\label{def:Disparity}
Given a set of $O$ objects and a selection $K$ of $k$ percent of objects in $O$,
Let $\vec{D}^F_O$ be the centroid of $O$ over a set of fairness attributes $F$, and let $\vec{D}^F_k$ be the centroid of the $K$ selected objects  over the same set of attributes. We define the disparity $\vec{D}^F$ as the $|F|$ dimensional disparity vector where $ \vec{D}^F \equiv \vec{D}^F_k - \vec{D}^F_O$.
\end{definition}
When the set of fairness attributes is understood, we omit $^F$ for simplicity of notation: $ \vec{D} \equiv \vec{D}_k - \vec{D}_O$

Intuitively, disparity measures the difference between the average selected object and the average object overall. For example, if the population is 30\% low income and the selected set is 20\% low income that would lead to a 10\% disparity or 0.1. For continuous fairness attributes, disparity is normalized based on the range of values. For instance, in a population with income in [\$0;\$200,000], if the average income of the population is \$40,000 (normalized to 0.2) and the average income of the selected set is \$100,000  (normalized to 0.5) that would lead to a disparity of \$60,000  (normalized to 0.3). Each fairness attribute is one dimension of the disparity vector. Assuming all values are normalized between 0 and 1, a disparity magnitude of -1 or 1 means that the protected attribute is either present only in the population, or only in the protected set respectively. A disparity of zero indicates statistical parity. 

\section{Disparity Compensation Methods}
\label{sec:algo}

We now present our disparity compensation approach. We first highlight several challenges we aim to address in Section~\ref{sec:challenges}. In Section~\ref{sec:DCA_algo} we describe the Disparity Compensation Algorithm (DCA), our primary contribution; we discuss the accuracy of DCA in Section~\ref{sec:accuracy}, and provide its time complexity  in Section~\ref{sec:timecomplexity}. 


\subsection{Challenges}
\label{sec:challenges}
Our goal is to find the optimum number of points to allocate to protected groups in order to minimize disparity. This task can be thought of as an optimization task, in which the goal is to pick a bonus vector $\vec{B}$ such that the $L^2$ norm of the disparity vector $\vec{D}$ is minimized.  Formally, we want to minimize the disparity:
\begin{equation*}
\begin{aligned}
& \underset{B}{\text{minimize}}
& & ||\vec{D}(\vec{B})||_2 \\
& \text{subject to}
& & b_i \geq 0
\end{aligned}
\end{equation*}


Where $\Vec{D}(\vec{B})$ is the disparity on a given population as a function of the bonus vector as defined above, containing the number of bonus points given to each fairness attribute.
This minimization is complicated by the following challenges.


\begin{enumerate}
    \item There are a large number of possible solutions. This means that most traditional algorithmic solutions are very slow. For instance, recent fairness algorithms are super-polynomial~\cite{zehlike2022fair}, and not easily scalable.
    \item In a set selection task, such as identifying the $k$-objects, measures to assess the fairness of the rankings are step functions, as their value change with every new candidate selected, or each change in  the ranking order. A small change in the bonus point vector can therefore lead to an arbitrarily large change in the disparity, which  means the optimization functions are not smooth or continuous. As such, they are non-differentiable, and standard derivative-based optimization methods are inapplicable.  
    \item Our minimization function does not exhibit convexity, or even quasi-convexity, which precludes us from using convex optimization techniques.
    \item Evaluating each possible solution is expensive as it requires a re-ranking of the dataset. Non-differential (derivative-free) optimization solutions  are therefore inefficient because they typically re-rank the data hundreds of times ~\cite{belkhir2017per}.
    \item For practical purposes, it is desirable for our methods to be fast enough so that function designers (e.g. school administrators) can iterate over several options to assess the impacts of fairness adjustments. 
.
    
\end{enumerate}  


To address these challenges, we propose using a novel descent-based method to compute the correct number of bonus points to eliminate disparity

\subsection{Disparity Compensation Algorithm}
\label{sec:DCA_algo}
 


\begin{algorithm}
\caption{Core DCA}
\SetAlgoLined
\KwResult{$\vec{B}$}
 O $\leftarrow$ the entire set of available objects\;
 k $\leftarrow$ size of the selection\;
 lr $\leftarrow$ list of learning rates sorted in decreasing order\;
 t $\leftarrow$ number of iterations\;
 $\vec{B}$ $\leftarrow$ weight vector of dimensionality equal to number of fairness attributes initialized randomly\;
\For{L in lr}{
  \For{x in t}{
  S $\leftarrow$ A random sample of $sample\_size$   from O\;
  $\vec{D}_k$ $\leftarrow$ Disparity of the $k$ selection over S after applying $\vec{B}$ bonus points\;
  $\vec{B}$ $\leftarrow$ $\vec{B} - L \times \vec{D}_k$\;
  \For{D in $\vec{B}$}{$D \leftarrow max(D, 0)$}
  
 }
}
\label{algo_SGD}
\end{algorithm}


\begin{algorithm}
\KwResult{$\vec{B}$}
 A $\leftarrow$ An array for computing the average\;
 O $\leftarrow$ the entire set of available objects\;
 $\vec{B}$ $\leftarrow$ The output vector of DCA\;
 k $\leftarrow$ size of the selection\;
 t $\leftarrow$ number of iterations for refinement\;
 \For{x in t}{
  S $\leftarrow$ The next sample in O\;
  $\vec{D}_k$ $\leftarrow$ Disparity with $\vec{B}$ bonus points over S\;
  $\vec{B}$ $\leftarrow$ Adam.step($\vec{B},\vec{D}_k$)\;
  A $\leftarrow$ $A + \vec{B}$\;
 }
 \Return ROUND(AVERAGE(A))\;
 \caption{DCA Refinement}
\label{refine_function}
\end{algorithm}



Traditional descent-based methods cannot be applied to our setting as the presence of  large plateaus and steps renders the function non-differentiable; standard gradient descent methods are derivative-based and cannot be used on non-differentiable optimization functions. We circumvent this issue by using the disparity vector directly, instead of its gradient.

Our Disparity Compensation Algorithm (DCA) (Algorithm~\ref{algo_SGD})  is based on the observation that any descent movement in a dimension that aims at compensating the disparity in that dimension will results in a better outcome as long as it does not flip the sign of the Disparity $\vec{D}$ (i.e., create a reverse disparate impact). 

We consider cases where we want to prevent disparate outcomes in future decisions, not only on a known dataset. Therefore, it is not enough to perfectly address the disparity of the training data, our solution has to be applicable to any similar dataset. The training data can then be seen as a sample drawn from an underlying distribution, and our goal is to minimize disparity for that distribution. Therefore we can use the Central Limit Theorem and the Quantile Central Limit Theorem~\cite{ruppert2011statistics} to estimate the selectivity of the ranking function on the underlying distributions (Section~\ref{sec:accuracy}). Our methods can be applied similarly in the absence of training data if the expected distribution of the dataset is known. 


The algorithm is shown in Algorithm~\ref{algo_SGD}. DCA works by keeping a bonus vector, which is incrementally adjusted in the opposite direction of disparity. DCA loops though decreasing learning rates (step sizes) to reduce the disparity vector (noted $\vec{D}$) as close as possible to zero. For each learning rate, the algorithm adjusts the bonus vector over a fixed number of steps $t$ to get as close to zero as possible using that learning rate; it then goes down to the next learning rate.  At each step the Disparity is only computed on a small sample drawn uniformly at random from the overall distribution (or a representative training set). The entire set of objects $O$ is never looked at directly. For each learning rate, we take a fixed number of samples. In each step, DCA uses the Disparity on the sample to predict the Disparity on the distribution as a whole, using the current best guess for bonus points $\vec{B}$. DCA then adjusts the current best guess in the opposite direction of the disparity. For example, in the case above, if the population is 30\% low income and the selected set is 20\% low income that would lead to a 10\% disparity or 0.1. With a learning rate of 0.2, each Low-Income member of the population would receive $0.1 \times 0.2 = 0.02$ extra points in their scoring function for the next iteration. Then, a new sample would be taken and ranked and the disparity would be calculated again.

We propose a refinement step in Algorithm~\ref{refine_function}. The refinement consists of a for loop that uses an adaptive learning rate (using the Adam method~\cite{kingma2017adam}) to find the best estimate. Instead of using a fixed learning rate for all the parameters, the Adam method uses an individual learning rate for each parameter which is individually optimized based on the change in the gradient, or in our case the disparity. The Adam method is especially useful and popular to deal with the noise created by samples. 
Next, the average of the guesses is taken to further reduce the noise created by the random samples and get a more consistent result.  As we will see experimentally in Section~\ref{sec:refine}, this refinement step results in smoother Disparity compensation results.  

The values for $lr$ and $t$ provide a tradeoff between time and accuracy. We set them empirically for our experiments (Section~\ref{sec:settings}).



\subsection{Accuracy of DCA}
\label{sec:accuracy}
To show the accuracy of DCA, we first consider a variation of the DCA, called {\em Full DCA} that considers the entire dataset, not a sample. Consider two objects $p$ and $q$, $q$ in the top-$k$ and $p$ outside the top-$k$. If switching their positions will reduce the disparity, {\em Full DCA} will always reduce the difference in score between them. Formally:
\begin{theorem}
At every step of Full DCA, if removing object $q$ from the top-k and replacing it with object $p$ would reduce the overall disparity, Full DCA will allocate more bonus points at that step to $p$ than to $q$.
\end{theorem}
Mathematically this means that:
$$
(\vec{B} - L \times \vec{D}) \cdot \vec{F_q} - \vec{B} \cdot \vec{F_q} < (\vec{B} - L \times \vec{D}) \cdot \vec{F_p} - \vec{B} \cdot \vec{F_p})
$$
Or
$$0 > \vec{D}\cdot (\vec{F_p} - \vec{F_q})$$
Where $\vec{F_p}$ and $\vec{F_q}$ are the fairness attribute vectors, $\vec{D}$ is the disparity, $L$ is the step size, and $\vec{B}$ is the bonus vector. When using DCA without sampling this will always be true. This can be shown from the definition of disparity and the given assumption that switching these two objects will reduce disparity:
$$
|\vec{D}|_2 > |\frac{1}{s} \sum_{i \in \textbf{S}}F_i + \frac{1}{s}\times \vec{F_p} - \frac{1}{s}\times \vec{F_q} - \vec{Q}|_2
$$
Where Q is the centroid of the entire distribution (which is constant during the running of DCA) and s is the number of selected objects. Using the definition of the $L_2$ norm we see that the above inequality only holds if:
$$
\vec{D} \cdot \vec{D} > \vec{D} \cdot \vec{D} + 
\frac{2}{s}\times (\vec{F_p} - \vec{F_q}) \cdot \vec{D} + \frac{1}{s^2}\times (\vec{F_p} - \vec{F_q})\cdot(\vec{F_p} - \vec{F_q}))
$$
Which simplifies to:
$$
-\frac{1}{2s}\times (\vec{F_p} - \vec{F_q})\cdot(\vec{F_p} - \vec{F_q}) > \vec{D} \cdot (\vec{F_p} - \vec{F_q})  
$$
Since the left side is always negative, we have shown $$0 > \vec{D}\cdot (\vec{F_p} - \vec{F_q})$$ and that $p$ will always receive more additional bonus points than $q$. 


Unlike {\em Full DCA}, DCA relies on samples of the distribution to efficiently identify the best bonus point vector $\vec{B}$ to apply on the set of fairness attributes $F$ to minimize disparity. The accuracy of DCA then depends on the accuracy of the computation of the Disparity metric $\vec{D}$ over the samples as  estimators of the Disparity over the whole dataset.

The Disparity $\vec{D}$ is computed as the distance between the centroid over the set of all objects $O$, $\vec{D}_O$, and the centroid of the $K$ selected objects, $\vec{D}_k$ (Section~\ref{sec:definitions}). In this section, we will show that computing $\vec{D}_O$ and $\vec{D}_k$ over a sample of the dataset gives a good estimation of their value over the whole dataset.


\begin{lemma}
The centroid $\vec{D_s}$ of a sample $s$ over a set of objects $O$ is an unbiased, low-error, estimate of the centroid $\vec{D}_O$ over the entire set of objects $O$.
\end{lemma}

\begin{proof}
This results directly follows from the Central Limit Theorem which states that the mean of a sample will approximate that of the original distribution as long as the sample size is sufficiently large (at least 30). 
\end{proof}

Next we show that the score of the object at the  $k^{th}$ percentile in the sample $s$ is a good estimator of the score of the object at the $k^{th}$ percentile over the whole distribution. 

\begin{lemma}
The score at quantile value $k$ of a sample $s$ over a set of objects $O$ is an unbiased, low-error, estimate of score at quantile value $k$ over the entire set of objects $O$.
\label{lemma-quantile}
\end{lemma}

\begin{proof}
This lemma is a direct result of the Quantile Central Limit Theorem. \cite{ruppert2011statistics}, which says that the quantile $k$ of a sample approximate the corresponding quantile of the original distribution, as long as the density function of the sample at $1-k$ is positive. This qualification is met as long as the sample size is at least $\frac{1}{k}$, which gives us a lower bound on the sample size used in DCA. 
\end{proof}

The variance of the sample quantile as an estimator can be large for values of $k$ close to 0 or 1, and  may lead to low-quality estimations in those cases. This is reflected in real-world experiments, as shown in section \ref{sec:varying}; however for  reasonable values of $k$, the $k^{th}$ percentile of the sample is an accurate estimator of the $k^{th}$ percentile of the distribution.


\begin{lemma}
The centroid $\vec{D_s}_k$ of the  $k$ percent selected objects over a sample $s$ over a set of objects $O$ is an unbiased, low-error, estimate of the centroid $\vec{D}_k$ of the  $k$ percent of selected objects over the entire set of objects $O$.
\end{lemma}

\begin{proof}
From Lemma~\ref{lemma-quantile} we know that the $k^{th}$ percentile value of the sample $s$ is an estimator of the $k^{th}$ percentile value of the distribution of all objects $O$. The top $k$ percent objects from $s$ are therefore taken from the same distribution a the top $k$ percent objects from $O$: the distribution of $O$ truncated at the $k^{th}$ percentile value. 

From the Central Limit Theorem, we know that the mean of a sample will approximate that of the original distribution as long as the sample size is sufficiently large (at least 30). Since both top $k$ percent selections over the sample $s$ and $O$ are taken from the same distribution, the Central Limit Theorem applies, and $\vec{D_s}_k$  is an unbiased, low-error, estimate of $\vec{D}_k$.
\end{proof} 

From the above results, it follows that:

\begin{theorem}
The sample Disparity $\vec{D_s} \equiv \vec{D_s}_k - \vec{D_s}_O$ is an unbiased, low-error, estimate of $ \vec{D} \equiv \vec{D}_k - \vec{D}_O$ of the Disparity over the whole dataset.
\end{theorem}


 \subsection{Time Complexity of DCA}
 \label{sec:timecomplexity}
 The time complexity of DCA does not depend on the size of the dataset but on the sample size and characteristics of the distribution, which allows for fast performance in practice. This is because DCA focuses on correcting the disparity in the underlying distribution rather than on a specific dataset. A training dataset represents a larger sample over the hidden distribution. This allows DCA's execution time to depend on the (smaller) samples' size but not on the dataset size. The algorithm takes a constant multiple of the time taken to compute the disparity on one sample. This time is $$O({sample\_size \times log(sample\_size)})$$ if the sample is fully sorted. 
 The size of the sample needs to be large enough so that the  Central Limit Theorem can be applied, this is generally recognized to be around 30. This means that $30 = sample\_size * k$ and the sample size is $O(\frac{1}{k})$. 
 
 In addition, each subgroup of interest needs to appear in the sample in a reasonable number, for the same reason, so the Central limit theorem can apply. This leads to a final sample size of $O(max(\frac{1}{k},\frac{1}{r}))$ where $k$ is the proportion of elements selected by the ranking process and $r$ is the frequency of the least common group in the dataset. Given this, assuming that $k$ is large enough that the entire sample must be sorted, the time complexity of DCA does not depend on the size of the dataset and is: $$O(max(\frac{1}{k},\frac{1}{r})\times log(max(\frac{1}{k},\frac{1}{r}))$$ 

\subsection{Adjusting the Optimization Goal of DCA for multiple values of $k$}
\label{sec:log_discount}

We have presented DCA for the case where the size of the selection $k$ is known in advance. However, it is often useful to optimize an entire ranking, either when the $k$ is unknown in advance (such as ranked lists in school matching applications), or when the ranking over the entire population is used. 

We propose a modification of DCA that updates the definition of disparity to use the whole ranking along with the logarithmic discounting techniques described in~\cite{yang2017measuring}, to assign more importance to objects selected first than to those selected last. Logarithmic discounting replaces the disparity at k with in our minimization goal of Section~\ref{sec:disparity} with:  $$\frac{1}{Z}\sum_{i\in10,20,30..}^{i=k}\frac{\vec{D}_i}{log_2(i + 1)}$$ 
Where Z is defined as$$\sum_{i\in10,20,30..}^{i=k}\frac{1}{log_2(i + 1)}$$

The computation of $\vec{D}_k$ in Algorithms~\ref{algo_SGD} and~\ref{refine_function} is  replaced with this new logarithmically discounted disparity. This new metric retains the useful characteristics of the previous one: it is a vector with each dimension representing an individual fairness attribute and calculated independently, it ranges between -1 for completely unfair in one direction to 1 for completely unfair in the other direction, is equal to 0 for fair representation, and it can  be summarized by its norm. 

 When using logarithmically-discounted disparity, the minimization problem of Section~\ref{sec:challenges} is then changed to:
 \begin{equation*}
\begin{aligned}
& \underset{B}{\text{minimize}}
& & \sum_{j\in10,20,30..}^{j=k}\frac{||\vec{D}_j(\vec{B})||_2}{log_2(j + 1)} \\
& \text{subject to}
& & b_i \geq 0
\end{aligned}
\end{equation*}
Where $\vec{D}_j$ is the disparity with $k=j$.
 

In terms of time complexity, using the logarithmically-discounted disparity version of DCA takes longer by an additional factor of the size of the sample, as we need to evaluate disparity at every point in the sample, leading to an overall time of:
 $$O((max(\frac{1}{k},\frac{1}{r})\times log(max(\frac{1}{k},\frac{1}{r}))\times max(\frac{1}{k},\frac{1}{r}))$$


Often, only part of the ranking is interesting to the user. Logarithmic-discounted disparity can be adjusted to various ranking needs. For example, users might only be interested in the top half of the ranking. In this case, the disparity outside that section of the ranking can be ignored, and the discounted disparity can still be computed straightforwardly only for values of $k<\frac{N}{2}$.


\section{Experimental Setting}
\label{sec:settings}

We now describe the datasets used in our evaluation, the parameters we set for the implementation of our DCA algorithm, and our experimental environment.

\subsection{Datasets}


\paragraph{NYC school dataset}
We evaluate our algorithms using real student data from NYC, which we received through a NYC Data Request~\cite{nycdata}, and for which we have secured IRB approval.  

The data used in this paper consists of the grades, test scores, absences, and demographics of around 80,000 7$^{\text{th}}$ graders each for both the 2016-2017 and 2017-2018 academic years. NYC high schools use the admission matching system described in Section~\ref{sec:motivations} when students are in the 8$^{\text{th}}$ grade; the various attributes used for ranking students therefore are from their 7$^{\text{th}}$ grade report cards. We used data from the 2016-2017 academic year as our training data, and data from the 2017-2018 academic year as our test data. 

We selected our ranking function to model the admission function that several real NYC high schools used for admission in the years 2017 and 2018: a weighted-sum function $f=0.55* GPA + 0.45* TestScores$, where $GPA$ is the normalized average of the students' math, ELA, science, and social studies grades, and $TestScores$ is the normalized average of the math and ELA state test scores. When not otherwise stated, we consider that 5\% of students are selected.

The dataset includes demographics, as well as information about the student's current school. We consider the following dimensions of fairness:
\begin{itemize}
    \item {\em Low-income: } in the NYC public school system, $70\%$ of students qualify as low income.
    \item {\em ELL: }students who are English Learners. These students are obviously disadvantaged by an admission method that takes into account ELA (English Language Arts) grades and test scores.
    \item {\em ENI: } the Economic Need Index, a measure of the overall economic need of students attending the same school as the student. ENI is calculated as the percentage of students in the school who have an economic need. A school is defined as high-poverty if it has an Economic Need Index (ENI) of at least $60\%$. 
    \item {\em Special Ed: }students who are receiving special education services.
\end{itemize}



\paragraph{COMPAS}
The COMPAS dataset consists of recidivism data from Broward County Florida as a result of the 2016 ProPublica investigation. The dataset   contains individual demographic information, criminal history, the COMPAS recidivism risk score, arrest records within a 2-year period, for 7214 defendants. COMPAS decile scores, which represent the decile rank of the defendant compared to a target comparison population of defendants, range from 1 to 10. 

We consider the decile score as the ranking function (the lower the better, see discussion in Section~\ref{sec:rescompas}), and compute compensatory bonus points using race as the fairness attribute. 





\subsection{Evaluation Parameters}
\label{sec:evaluation_parameters}
\paragraph{Bonus Points.}  The bonus points can be understood as a multiplier over the attribute value. When the attribute is binary, the bonus point value is added to the score of objects with that attribute (e.g., the bonus points for ELL are added to the score of students who are marked as English Learners).  For continuous attributes,  the value of the bonus points is multiplied by the value of the attribute (e.g., the ENI value of the school a student is attending will be multiplied by the ENI bonus points, the resulting product will be added to the score of the student). 

In the final step of the algorithm, we round to the desired bonus point granularity, as decided by stakeholders. For simplicity and efficiency, we restricted bonus points to values with a granularity of 0.5 points in both evaluation scenarios. 


\paragraph{DCA vs. Core DCA} In Section~\ref{sec:refine} we evaluate the impact of the refinement step of Algorithm~\ref{refine_function} over the Core DCA algorithm of Algorithm~\ref{algo_SGD}. In the rest of the paper, we use the name DCA to refer to the algorithm {\em with the refinement steps applied}.

\paragraph{Algorithm DCA - Sample Size.} Our rarest fairness category has a frequency of ~10\%, so we picked a sample size of 500 elements to ensure a representation of 50 elements (for our defaults selection percentage of 5\%), enough to show most of the correlation between attributes. 

\paragraph{Algorithm DCA - Learning Rate.} We experimented with different learning rates and settled on 3 sets of DCA with 100 rounds for each learning rate. In the first pass we use a learning rate of 1. This gives us the right general area to search. We use a learning rate of 0.1 to further hone in on the correct location. Then, we take a second pass through the data using a  modern weight updating algorithm, Adam, to find the best bonus point values~\cite{kingma2017adam}. Finally, we take the rolling average of the last 100 points to increase stability and avoid too many random effects of unusual samples near the end. 


\subsection{Experimental Environment}
The experiments were all preformed on an Optiplex7060 with 30GB of RAM. The machine has a Intel(R) Core(TM) i7-8700 CPU @ 3.20GHz. Our proposed algorithms were implemented using Python 3.8 and Pandas. The comparison algorithm (Multinomial FA**IR) was implemented in Java using the implementation by the authors of \cite{zehlike2022fair}. 
\section{Experimental Results}\label{sec:results}
    \subsection{General Results}
        The basic ResSAN model is used to determine reference results which our expanded model can be compared to as it is structurally similar to ResLAN but does not possess the Lidar adaptive components of it. Further, we compare with the full-size PackNet-SAN and the unmodified NLSPN architecture. 
        As it can be seen from Tab.\,\ref{tab:sota-results}, our LiDAR-adaptive ResLAN achieves competitive performance compared to state-of-the-art standard depth completion methods, which are specialized to the unfiltered 64-beam-LiDAR. The performance differences are in the range of a few centimetres in terms of MAE, which is acceptable given the practical advantage that ResLAN can generalize to different beam patterns as will be shown below.

        Furthermore, we compared the architectures for a set of three different input types that contained 64, 32 or 16 LiDAR channels using both filter types on the metrics from the KITTI benchmark. The NLSPN model was trained for the standard depth completion task and then evaluated with different input data. As for the ResSAN models, we trained one model for each input type and tested it for the corresponding one which serve serve as the \emph{Baseline} in Tab.\,\ref{tab:overall-results}. Our ResLAN model was jointly trained for all three settings. As listed in Tab.\,\ref{tab:overall-results}, the ResLAN models outperform the challenging baseline in all metrics for FOV filtering and all but one for sparse filtering. This implies that our LiDAR adaptive model is able to outperform dedicated models in case of very sparse input depth. Fig.\,\ref{fig:comp-plot} shows this is indeed the case for 32 and even more for 16 channels. For FOV-filtered inputs with 16 channels, the ResLAN exhibits approx. $10\%$ smaller MAE than the baseline. As for the NLSPN, it becomes apparent that it is not capable of generalizing to other input types since it shows clearly worse results. The difference is especially pronounced for the FOV filtering where on average more than every fourth predicted pixel is more than $25 \%$ deviating from the ground truth\,($\delta_{1.25}$). Therefore, using a weight-adapting network in combination with differently filtered input depths allows us to train models that outperform their non-adaptive counterparts.

        \begin{table}[]
            \centering
    	    \small
            \vspace{0.4cm}
            \caption{\textbf{Depth estimation result for standard depth completion} when the ResSAN model was only trained for 64 channels and the ResLAN model for multiple tasks. The PackNet-SAN and NLSPN models were trained with the setup that was also used for our model architecture.}
            \footnotesize
            \setlength{\tabcolsep}{5pt}
            \begin{tabular}{@{}lrrrrl@{}}
            \toprule
            \multicolumn{6}{c}{\textbf{Standard LiDAR Depth Completion}}                                                                                                                         \\ \midrule
            \multicolumn{1}{l|}{Method}          & RMSE $\downarrow$            & MAE  $\downarrow$            & iRMSE $\downarrow$             & iMAE $\downarrow$ & $\delta_{1.25}$ $\uparrow$ \\
            \multicolumn{1}{l|}{}                & \multicolumn{1}{l}{{[}mm{]}} & \multicolumn{1}{l}{{[}mm{]}} & \multicolumn{1}{l}{{[}1/km{]}} & {[}1/km{]}        &                            \\ \midrule
            \multicolumn{1}{l|}{PackNet-SAN}     &  914                            &  298                            &  2.78                              &  1.4                 &  99.65 \%                          \\
            \multicolumn{1}{l|}{NLSPN}           &  \textbf{889}                            &   \textbf{263}                           &  \textbf{2.62}                              &   \textbf{1.3}                &   \textbf{99.61} \%                         \\ \midrule
            \multicolumn{1}{l|}{ResSAN (Ours)}   & 948                             &  275                            &  2.75                              &    1.4               &   99.58 \%                         \\
            \multicolumn{1}{l|}{ResLAN (Ours)} &   969                           &  283                            &   2.83                             &   1.4                &  99.56 \%                          \\ \bottomrule
            \end{tabular}
            \vspace{0.2cm}
            \label{tab:sota-results}
        \end{table}

        \begin{table}[]
    	    \centering
    	    \small
    	    \caption{\textbf{Depth estimation results of the two baseline setups and the explicit and implicit ResSAN} when evaluated on a combination of 16, 32 and 64 channel depth inputs. Please note that Specialist Methods need to train three specialized networks, one for each of the three types of inputs while our method only uses one network.}
            \footnotesize
            \setlength{\tabcolsep}{4.8pt}
            \begin{tabular}{@{}lrrrrl@{}}
                \toprule
                \multicolumn{6}{c}{\textbf{Sparse Channel Filter}}                                                                                                                                  \\ \midrule
                \multicolumn{1}{l|}{Method}        & RMSE $\downarrow$            & MAE  $\downarrow$            & iRMSE $\downarrow$             & iMAE $\downarrow$ & $\delta_{1.25}$ $\uparrow$  \\
                \multicolumn{1}{l|}{}              & \multicolumn{1}{l}{{[}mm{]}} & \multicolumn{1}{l}{{[}mm{]}} & \multicolumn{1}{l}{{[}1/km{]}} & {[}1/km{]}        &                             \\ \midrule
                \multicolumn{1}{l|}{NLSPN}         &  1396                            &  437                            & 5.54                               &  2.2                 &  98.82 \%                           \\
                \multicolumn{1}{l|}{Baseline}      & \textbf{1207}                             &  381                            & 4.41                               &  1.8                 &  \textbf{99.37} \%                           \\
                \multicolumn{1}{l|}{ResLAN (Ours)} &  1215                            &  \textbf{378}                            &  \textbf{4.27}                              &  \textbf{1.7}                 &  99.31 \%                           \\ \toprule
                \multicolumn{6}{c}{\textbf{Field-of-View Filter}}                                                                                                                                   \\ \midrule
                \multicolumn{1}{l|}{Method}        & RMSE $\downarrow$            & MAE  $\downarrow$            & iRMSE $\downarrow$             & iMAE $\downarrow$ & $\delta_{1.25}$ $\uparrow$ \\
                \multicolumn{1}{l|}{}              & \multicolumn{1}{l}{{[}mm{]}} & \multicolumn{1}{l}{{[}mm{]}} & \multicolumn{1}{l}{{[}1/km{]}} & {[}1/km{]}        &                             \\ \midrule
                \multicolumn{1}{l|}{NLSPN}         &  2738                            &  1702                            & 12.3                              &  4.3                 &  74.69 \%                           \\
                \multicolumn{1}{l|}{Baseline}      &  1556                            &  525                            &  6.8                              &  3.0                 & 98.14 \%                            \\
                \multicolumn{1}{l|}{ResLAN (Ours)} &  \textbf{1548}                            &  \textbf{519}                            &  \textbf{6.44}                              &  \textbf{2.8}                 & \textbf{98.52 \%}                            \\ \bottomrule
            \end{tabular}
            \label{tab:overall-results}
        \end{table}

        
        
        % Figure environment removed
        
        % Figure environment removed

    \subsection{Filter Effects}
        Comparing the effect of the two different types of depth input filters on the model performance, it becomes apparent that FOV filtering is the more challenging task. In that setting, reducing LiDAR channels is more detrimental to the performance than sparse filtering as it creates regions where no depth information is available. Effectively, the model is forced to perform depth prediction in these regions. These effects are highlighted in the depth images in Fig.\,\ref{fig:dense-maps} where the effect of a 16-channel sparse depth filter and a 16-channel FOV can be compared.

    \subsection{Generalization Capabilities}
        We trained three models for both filter types eaach, so the combinations and number of filtered depth inputs they receive are different. This serves the purpose of testing the generalization capabilities of the ResLAN architecture as well as the robustness to different filter settings. After training, the models were evaluated for the depth input settings they were trained for, as well as for ones they weren't exposed to. Overall, ResLAN shows good generalization capabilities. As one can gather from Fig.\,\ref{fig:explicit-comp} and Fig.\,\ref{fig:implicit-comp}, the consequences of slightly varying sets of input depth settings are limited. The most considerable deviations can be seen when the model is tasked to extrapolate. For instance, the model $\{64, 32, 16\}$ shows a noticeably higher MAE for eight-channel depth inputs than the model that was trained for it. Similar behaviour can be seen for the FOV filtering case as well for the model $\{64, 48, 32\}$ when tasked to generalize for a 16-channel input. There is no such pronounced effect for generalization tasks that lie between two filter settings the model was trained for. At most, it can be observed that models that were trained for a smaller range of filter values perform slightly better than ones that have to cover a wider range. The number of filter settings used in a fixed range does not relevantly influence the model performance, as can be seen, when comparing the two models in Fig.\,\ref{fig:implicit-comp}, which are both trained for a range of 64 to 32 channels but one with three filter settings and the other one with five.
    
    % Figure environment removed
    
    
    % Figure environment removed
\section{Conclusion}
\label{sec:conclusion}


We presented DCA, an algorithm to address disparity in outcomes of ranking processes using compensatory bonus points. We showed that DCA, by relying on a sampling-based approach, successfully reduces disparity in a wide range of settings, while being significantly more efficient than state-of-the-art approaches, running in sub-linear time. This makes DCA a good candidate for iterative processes that would allow users to identify the ranking function that best fits their needs while checking for its fairness impacts and the required compensatory bonus points.  


Our approach relies on the use of compensatory bonus points, a departure from previous work, which has mostly focused on modifying the ranking function directly, or on the use of quotas. A significant advantage of compensatory bonus points is that they are transparent, interpretable, and easily explainable to all stakeholders.


\bibliographystyle{IEEEtran}
\bibliography{bonus, Ranking}

\end{document}
