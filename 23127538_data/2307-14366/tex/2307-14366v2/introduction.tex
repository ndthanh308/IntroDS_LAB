\section{Introduction}


Ranking functions are used in a wide range of decision-making systems, such as resource allocation, candidate selection, or risk assessment. These ranking mechanisms often produce disparate outcomes because of bias in the underlying data. Compensating for disparities is therefore a key task for modern decision-making systems to ensure fairer, more equitable outcomes. Understanding the source of bias, which is sometimes not obvious but  hidden in correlations, is critical to address the disparity in the results. In this paper, we propose data-driven disparity compensation measures to transparently adjust ranking mechanisms based on the underlying data. Our measures are designed to be easily explainable to stakeholders in order to ensure  accountable and trustworthy decision-making systems.

Disparity compensation mechanisms for ranking functions used in real-world systems have mostly relied on the use of quotas, soft or hard, to ensure a minimum representation of members of protected groups. Quotas (or set-asides) have the advantage of being simple to implement and explain when only one dimension of inequity is present (e.g., set-asides for low-income students for school admissions). However, once several dimensions of disparity need to be accounted for (e.g, English language learners, low-income students, students with disabilities), the use of quotas becomes cumbersome and difficult to implement: Do members of two or more protected classes count towards one or more quotas? Does every combination of protected classes get a separate quota?~\cite{sonmez2019affirmative} As more dimensions are added, setting accurate set-aside thresholds can seem arbitrary and capricious to stakeholders. 


We propose a mechanism based on the use of compensatory bonus points to address disparity, as defined ~\cite{Gale2020ExplainingMR} (see Section~ \ref{sec:disparity}), in ranking applications. Disparity represents the lack of statistical parity and is computed by measuring the distance between the selected (high-rank) and unselected (low-rank) objects in the fairness attribute space. Our bonus points approach is simple and easily understandable, as it links bonus points directly to sources of bias in the data. It allows for composing bonus points to model the intersectionality of bias and the compounding effect of different sources of disparity on ranking decision outcomes. It can be quickly and easily adjusted to new data and scenarios. 

We present algorithms for identifying the bonus points values that will minimize disparate outcomes on a given data distribution. Our work is based on a sample-based approach, which considers the underlying data distribution and draws samples to calculate the optimal number of bonus points to allocate to each disparity factor. This sample-based approach has three main benefits over the state of the art. First, it can be used to identify a set of compensatory bonus points before all the data is gathered, as long as the underlying distribution is known, by generating samples over the expected distribution. This can be beneficial in applications where providing transparent information to stakeholders is critical (e.g., letting students know in advance on which criteria they will be ranked, which equity-focused adjustments are in place, and how that would affect them). Second, by processing small samples of the data rather than the whole data set, we are able to identify high-quality compensatory bonus points in sub-linear time, compared to existing techniques which run in as high as exponential runtimes and are unpractical for large datasets. This makes our approach usable in many real-world use cases. Third, using bonus points is flexible enough to account for multiple sources of bias and disparity by allowing for the compounding effects of points to compensate for multiple disparate impacts.   






We make the following contributions:
\begin{itemize}
    \item A model of disparity compensation for ranking functions based on the attribution of bonus points to members of protected classes. (Section~\ref{sec:background})
    \item A  disparity compensation algorithm (DCA) to identify the optimal value of compensatory bonus points to minimize disparity for a top-$k$ selection set. DCA runs in sub-linear time (Section~\ref{sec:algo}). We propose a modification of DCA that takes into account the whole ranking, using logarithmic discounting techniques~\cite{yang2017measuring}, to adapt to different selection sizes (Section~\ref{sec:log_discount}).

    \item An extensive evaluation of the impact of our compensatory mechanism over two real-world data sources: NYC public school student records used for high school admissions, and a dataset used to estimate recidivism risk in bail and sentencing decisions. 
    (Sections~\ref{sec:settings} and~\ref{sec:experiments})
    \item A comparison with state-of-the-art fair-ranking techniques that show that our proposed DCA algorithm results in comparable or better disparity reduction outcomes while being significantly more efficient. (Section~\ref{sec:Comparison}) 
\end{itemize}
We present related work in Section~\ref{sec:related}, present motivating examples in Section~\ref{sec:background}, and conclude in Section~\ref{sec:conclusion}.




