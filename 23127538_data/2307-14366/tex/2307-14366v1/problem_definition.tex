\section{Proposed Solution}
% \subsection{Quotas}
% The traditional way to do this described in \cite{zehlike2017fa} would be by creating quotas for various discriminated against categories and then selecting either the best student with the characteristic with probability p or the best without the characteristic with probability 1-p. Our main issue with this method is that it is not very explainable. It also does not allow for competition, we never compare a student with the characteristics to a student without it. Another problem is that as we increase the number of characteristics, quotas become quickly untenable. If we have 10 correlated characteristics, which is common in this kind of scenario, it is not clear how a quota system would deal with this. Lastly, continuous characteristics are not dealt with at all in this type of system.
% \subsection{Disparity}
% Disparity is the vector difference between the average selected object and the average unselected object. It can be used to measure the difference between discrete and continuous objects in an explainable way. It has several charicteristics that make it useful for a non-expert audience. \begin{enumerate}
%     \item The math is simple enough to be understandable even with only a high school math education
%     \item It is resistant to the curse of dimensionality. Many metrics would have trouble with something such as race which might have many different buckets. Disparity will not be skewed by too much as long as each object only falls into one bucket.
%     \item The data is used transparently. Disparity does not require understanding the causal graph behind the data, it just takes the data you have and outputs an answer.
% \end{enumerate}
% \subsection{Bonus points}
% In addition to the explainability advantages described above, bonus points have a number of technical advantages. Firstly, they satisfy in group monotonicity. This constraint described by \cite{zehlike2022fair} says that within groups, objects are sorted by score. It also guarantees that we will not accidentally disadvantage the students that are given bonuses, the score increases and so their ranking must be strictly better. Lastly, it allows decision makers to publish thresholds that affected parties can can easily compare their own scores to and verify why they missed the cutoff and by how much.
% Ideally, a function should be as self-explanatory as possible. Since most of the real-world examples of these functions require understanding weighted sum ranking functions anyway, the most explainable way to correct disparity is using a weighted sum method. If we simply add points to the students with characteristics that are disadvantages to correct the disadvantage, even someone who is not trained in statistics can easily understand what we are doing. In order to further simplify the function for untrained users, we only allow half point weights, which is the finest granularity that is common in real-world point functions. The question becomes how to calculate the number of points to add to minimize the disparity. As we will see bellow, this method allows us to minimize disparity with minimal added complication to the function.
