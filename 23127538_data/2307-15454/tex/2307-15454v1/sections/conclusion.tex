The Recoil Directionality (\ReD) experiment was designed within the 
GADMC to explore the possible directional sensitivity of an Ar dual-phase \TPC\ to 
nuclear recoils in the energy range of interest for WIMP DM searches.
%, reported 
%by the SCENE Collaboration~\cite{Cao:2015ks}. 
The \ReD\ \TPC\ was 
irradiated with neutrons of known energy and direction at the INFN, 
Laboratori Nazionali 
del Sud, in order to produce Ar recoils of about 70~keV kinetic energy. Nuclear 
recoils traveling in five different directions with respect to the drift field
\edrift\ of the \TPC\ were selected using a neutron spectrometer made by 
liquid scintillation detectors. A statistical analysis based on the 
Cataudella et al. model~\cite{Cataudella:2017kcf} was performed to 
assess the  \TPC\ response for those samples of \NR\ 
events. 

The data from this work do not show any statistically-significant 
dependence of either S1 or S2 on the direction for \NRs\ of $\sim 70$~keV.
The best-fit for the parameter of interest $R$, which measures the aspect ratio 
between the long and short axes of the initial electron cloud, is
$R = (1.037 \pm 0.027)$, or $R < 1.072$ at 90\% CL. The absence of 
significant deviations from the spherical symmetry of the electron cloud
indicates that the electron thermalization process likely plays a 
significant role in weakening any initial direction-induced anisotropy 
of the charge cloud. 
%The best-fit value $\delta R = 0.037$ could potentially allow for a 
%statistical evidence of daily directional modulation in a future-generation 
%multi-tonne argon dual-phase \TPC\ detector. However, due to the smallness 
%of the effect, this would require a large number of candidate WIMP events.



