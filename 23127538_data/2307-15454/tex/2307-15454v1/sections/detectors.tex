\subsection{The $\Delta E$/$E$ telescope} \label{sec:deltaee}
%% Si detectors
Neutrons directed towards the \TPC\  %, at $\theta_{tpc} = 22.3$\textdegree, 
are produced in association with \Be\ nuclei of energy 
$E_{Be} = 19.0$~MeV and emitted at angle $\theta_{Be} = 5.1$\textdegree. $^{7}$Be is detected by a dedicated $\Delta E$/$E$ 
telescope placed in the scattering chamber at a distance of \SI{46}{cm} from the CH$_2$ target. The telescope is made of two Si detectors manufactured by ORTEC, having thickness of 20~\textmu m and 1000~\textmu m, respectively; the $^{7}$Be loses about 7.6~MeV crossing the thinner stage and it is stopped in the thicker one. The detectors have a 100\% efficiency for light charged particles detection and energy resolution of  about 1\%. The telescope is collimated using an Al shield with a hole of 2~mm diameter.  For the fine tuning of the position, the telescope holder is mounted on a two axis remotely-controlled stepper motor which can operate in vacuum.
%stage is a 20~$\textmu$m thick Si detector 
%manufacture by ORTEC; it is followed by the $E$ stage, which is a 1000~$\textmu$m-thick Si detector by ORTEC 
The detectors are 
%biased at 12~V and 150~V respectively,
%according to the specifications from the manufacturer and are 
readout from a standard spectroscopic chain made by a pre-amplifier %(ORTEC 142~A/B) 
and a charge-sensitive amplifier, %(CAEN N568E), 
with 1~\textmu s  shaping time. 
%
%The $\Delta E$ detector is thin enough to be 
%crossed by \Be\ nuclei of kinetic energy in the range of our interest, while the thick $E$ detector can stop the %nuclei completely.

The combined measurement of $\Delta E$ and $E$ provides the discrimination in $Z$, which is necessary to distinguish the
interesting Be from the far more abundant elastically-scattered Li. 
%The two Si detector have a diameter of 7~mm and they are
%assembled on a common grounded holder inside the scattering chamber. The holder is mounted on a stepper %motor (precision of 0.15~\textmu m), 
%which can operate in vacuum and is remotely controlled. The stepper motor allows for the fine tuning of the position
%of the Si telescope. % such to match the position $(\theta_{Be}, \phi_{Be})$ of the telescope according to the actual position
%of the \TPC, $(\theta_{TPC}, \phi_{TPC})$ and also to accommodate for small variations in the beam energy.
%
%Due to the inverse kinematic configuration, there are two different solutions at the same angle
%$\theta_{Be} = 5.1$\textdegree, in which \Be\ have energy of 19.0~MeV (``low energy'') and 20.4~MeV %(``high energy''), respectively. 
%Neutrons in association with the ``low energy'' \Be\ are those traveling towards the \TPC\ ($\theta_{n} = %22.3$\textdegree, $E_n = 7.3$~MeV). The
%``high energy'' \Be\ is associated with neutrons of $E_n = 2.7$~MeV emitted at $\theta_n = 44$\textdegree: 
%these neutrons do not hit directly the \TPC, but can produce accidental coincidences due to scattering on the %floor or on the walls. 
%
%Direct neutrons
%can be effectively identified by selecting only the \Be\ events in the $\Delta E$/$E$ telescope compatible with the ``low energy''
%solution.
%
%
% Figure environment removed
% 
Fig.~\ref{fig:banana1} shows the $\Delta E$ vs. $E$ scatter plot, upon the irradiation of the CH$_2$ target with the $^7$Li beam.
The central, and most intense, band is created by Li ($Z=3$), mostly by elastic scattering on H and C. The uppermost band is due to Be ($Z=4$). 
As the reaction p($^{7}$Li,$^{7}$Be)n occurs in inverse kinematics, two different solutions at the same angle
$\theta_{Be} = 5.1$\textdegree\ are allowed, with \Be\ having energy of 19.0~MeV (``low energy'') and 20.4~MeV (``high energy''), respectively. 
Neutrons in association with the ``low energy'' \Be\ are those travelling towards the \TPC\ ($\theta_{n} = 22.3$\textdegree), with $E_n = 7.3$~MeV
kinetic energy. The
``high energy'' \Be\ is associated with neutrons of $E_n = 2.7$~MeV emitted at $\theta_n = 44$\textdegree: 
these neutrons do not hit directly the \TPC, but can contribute to accidental coincidences due to scattering on the floor or on the walls. 
In  Fig.~\ref{fig:banana1}  the \emph{loci} from the two $^7$Be solutions are visible and clearly separated; the population between them is due to the inelastic interaction p($^{7}$Li,$^{7}$Be*)n', which also emits a neutron. Because of the finite extension of the beam spot 
and of the beam angular divergence, neutrons associated with the $^{7}$Be* detected at $\theta_{Be}$ can still travel inside 
the \TPC\ and produce an interaction; they also contribute to the diffuse 
background, e.g. upon scattering on the walls or on the floor of the 
experimental area.

 In order to suppress the dominant contribution from \Li\ elastic scattering, the thresholds for the $\Delta E$ and $E$  detectors, shown in Fig.~\ref{fig:banana1} as dashed lines, are used during the data acquisition. Fig.~\ref{fig:banana2} displays the $\Delta E$ vs. $E$ scatter plot, acquired with the thresholds 
of Fig.~\ref{fig:banana1}, without (color) and with (dots) the requirement of coincidence with an event in the \TPC\ compatible with 
a neutron interaction and within a 200~ns gate. As expected, neutron events in the \TPC\ are mostly associated with a ``low-energy'' \Be\
nucleus detected by the Si telescope. The dashed red box represents the \Be\ selection cut used in the 
following analysis and described in Sect.~\ref{sec:EventSelection}. 
%with a few accidentals associated with \Li\ or with the other \Be\ \emph{loci}. 
%The inset of Fig.~\ref{fig:banana2} displays the distribution of the time difference $\Delta t$ between \TPC\ and Si telescope, for the events in 
%coincidence, which clearly shows the correlation.
%
% Figure environment removed
% 

\subsection{The Time Projection Chamber} \label{sec:2a}
The heart of the ReD system is the dual-phase Ar \TPC, whose detailed description and 
performance are reported in~\cite{Agnes:2021zyq}. It is a cubic volume of $5 \times 5 \times 6$ 
cm$^3$, delimited on the side walls by acrylic plates interleaved with 
specular reflector foils, and the top and bottom by two transparent acrylic windows. 
The top and bottom windows are coated with a thin transparent conductive layer (indium-tin oxide, ITO), so 
they can be given an electric potential and be operated as anode and cathode, respectively. The extraction grid is a stainless 
steel mesh, having 95\% optical transparency; it is located 10~mm below the anode window and it is kept electrically grounded.
All internal surfaces are coated with a wavelength shifter (tetraphenyl-butadiene, TPB): it converts the UV light emitted by Ar scintillation (128~nm) into visible
light, which better matches the sensitivity of typical photosensors. The lower part of the \TPC\ contains LAr:  
the liquid fills the entire volume between the cathode and the extraction grid, plus 3~mm above the grid. The gas pocket 
is produced by means of a heater and it occupies the 7-mm thick region between the liquid surface and the anode. 
%The emission of the UV scintillation light takes place through two different excited states of the Ar dimer, which exhibit largely different time constant (6 ns and 1.6~$\textmu$s, respectively). Since the relative population between the two excited dimers depends on the ionization density, the time profile of the S1 signal can be used to perform a very efficient discrimination between nuclear and electron recoils~\cite{Amaudruz:2016dq}. 

The \TPC\ electric fields which are set for this work are: drift field (\edrift) of 152~V/cm; extraction field (\eex) of 3.9~kV/cm; and electroluminescence field (\eel) of 5.9~kV/cm.
The maximum drift time is about 66 \textmu s: this is the time required for an electron produced at the
cathode to travel until the liquid surface. Due to a continuous recirculation loop of the liquid through a SAES getter,
the purity of argon is such that the electron life time before 
capture by electronegative impurities is $> 1$~ms, i.e. much longer than the 66-\textmu s maximum drift time~\cite{Agnes:2021zyq}. 
The extraction field is strong enough to give a 100\% extraction efficiency of the electrons from the liquid to the 
gas phase~\cite{Chepel:2012sj}.

After the UV photons from scintillation and electroluminescence of Ar are shifted to the visible range by the TPB coating, they 
can be detected by customised NUV-HD-Cryo Silicon PhotoMultipliers (\SiPMs) from Fondazione Bruno Kessler, which can be operated at cryogenic 
temperature~\cite{Gola:2019idb}. The \SiPMs\ are assembled in two $5 \times 5$~cm$^2$ tiles, each containing 24 devices of 
dimensions 11.7~mm$\times$7.0~mm and arranged in a $4 \times 6$ array.
The tiles are placed behind the top and bottom acrylic windows of the \TPC, providing a 30\% 
total optical coverage. As the position of the S2 event in the gas phase can be used to estimate the \xy\ coordinate of the 
original interaction point in the \TPC, the \SiPMs\ of top tile are readout in 22 channels for improved resolution: 20 \SiPM\
are readout individually, while 4 lateral \SiPMs\ are summed in pairs and grouped into two readout channels. 
%the 24 \SiPMs\ of the top tile are read out individually for improved resolution.
The \SiPMs\ of the bottom tiles are summed in groups of twelve, hence giving two readout channels.
Two custom-made Front-End Boards (FEB) 
supply power to the \SiPMs\ and amplify the output signals at cryogenic temperature. The \SiPMs\ are operated at \SI{+7}{V} of overvoltage 
with respect to the breakdown voltage. Due to the presence of resistors in the bias chain, the effective 
overvoltage of the \SiPMs\ gets smaller than the nominal \SI{+7}{V} when the bias current of the devices is high. 
This typically happens when the \SiPMs\ are exposed to a significant amount of light, e.g. 
due to the high interaction rate under beam irradiation, and causes a change in the \SiPM\ response 
(see Sect.~\ref{sec:Calibration}). 

More details about the cryogenic setup, the \TPC, the photosensors and the readout system can be found in~\cite{Agnes:2021zyq}.

\subsection{The neutron spectrometer} \label{sec:spectrometer}
The neutron spectrometer used in \ReD\ is made of seven 3-inch liquid scintillator (LSci) cells, individually read-out by
photomultipliers (PMTs). The assembly includes the liquid scintillator cell, a ETL-9821B PMT and the front-end electronics with the 
amplifier. The cells are filled with the EJ-309 liquid scintillator by Eljen Technologies, which features a very powerful
neutron-$\gamma$ discrimination based on the time pattern of the scintillation pulse. %Each PMT is biased individually 
%(with typical voltage of 1700--1900~V) in order to equalize the light response of the cells.

The neutron detection efficiency of the detectors was measured individually by using a $^{252}$Cf 
source~\cite{Stevanato2014,simophdthesis} and found to be about 28\% for the 7-MeV neutrons of interest 
for this work. %, assuming an energy threshold of E$_{th}$ = 100 keV$_{ee}$ (electron equivalent).
The calibration of the energy scale was performed with $\gamma$-ray sources 
($^{241}$Am, $^{137}$Cs and $^{22}$Na). Dedicated measurements taken with the annihilation $\gamma$-rays from the 
$^{22}$Na source confirmed the time resolution to be better than 1~ns (rms).

%The individual LScis are mounted on a
%custom holder system, which allows to place them over a cone of opening angle $\theta_{lsci} = %36.8$\textdegree, whose axis is the
%target-TPC line. In this way all detectors identify Ar recoils of the same energy, but at 
%different angles $\theta_r$ with respect to the \TPC\ drift field. 
%The placement of the scintillators   
The scintillators  identify Ar recoils of the same energy but different angles $\theta_r$ with respect to the \TPC\ drift field \edrift:  
$\theta_r$=180\textdegree 
(one LSci), 90\textdegree (two LScis, read out individually and labeled as
``90\textdegree $l$'' and ``90\textdegree $r$''), 40\textdegree (two LScis, summed) and 
20\textdegree (two LScis, summed). 



\subsection{Data acquisition and control infrastructure} 
The output signals from all of the detectors are sent to CAEN V1730 Flash ADC Waveform Digitizers and 
digitized with 14-bit resolution at a sampling rate of 500 MHz. In total a signal of 100~\textmu s (50k samples) is 
acquired at each trigger: this is sufficiently long to contain the S1 and S2 signals of the \TPC, given the maximum 
drift time of 66~\textmu s for events occurring close to the cathode. About 10\% of the digitization window is reserved 
for the pre-trigger. %, in order to allow for a precise estimate of the baseline. 
Two 16-channel CAEN V1730 boards were used for the measurement, synchronized with a daisy chain. 
%Due to the availability of only 32 readout channel, two pairs of top \SiPM\ channels and each half of the bottom 24 \SiPM\ channels are analogly summed, respectively. 
%it was not possible to readout all 24 \SiPM\ of the top tile individually. The signals from two pairs of corner \SiPMs\ of the top tile were hence summed, so to reduce the number of readout channels from the \TPC\ to 26 (22 top and 4 bottom), instead of the customary 28. 

The data acquisition (DAQ) software was built upon a package 
developed for the PADME experiment~\cite{Leonardi_2017} and based on the CAEN Digitizer Libraries. 
%Each digitizer has its own 
%read out process, controlled by a central Linux server, and writes asynchronously its own data files on disk. The 
%event building, i.e. matching the information from the same event split on the different data files, is performed 
%offline. 
The trigger logic is implemented by means of an external NIM logic module 
%(CO4020 by ORTEC) coupled with a LeCroy 428F digital fan-in/fan-out module 
as:
\begin{equation}
\texttt{SiTel} \land (\texttt{TPC} \lor \texttt{LScis})
\end{equation} 
where: \texttt{SiTel}, \texttt{TPC} and \texttt{LScis} are the trigger signals from the Si telescope, the \TPC\ and 
the neutron spectrometer, respectively.
The Si telescope trigger (\texttt{SiTel}) is built as the coincidence of the $\Delta E$ and $E$ 
detectors, with the thresholds displayed in Fig.~\ref{fig:banana1}.
The \TPC\ trigger (\texttt{TPC}) consists in the logical \texttt{AND} between the
two readout channels of the bottom tile within a coincidence gate of 200~ns, in order to suppress the dark rate~\cite{Agnes:2021zyq}.
The individual thresholds are set to approximately 2~PE. The \TPC\ is expected to trigger with 100\% efficiency on
S1 signals from the $E_r = 72$~keV NR events ($\textrm{S1} \sim 190$~PE) which are of interest for this work, although trigger
inefficiencies can possibly come from pile-up. 
Finally, the neutron spectrometer trigger (\texttt{LScis}) is produced by the logical \texttt{OR} of the five readout channels 
of the seven scintillators. The energy threshold of each cell is set to approximately 20~keV$_{ee}$ (electron equivalent), 
which corresponds to about 200~keV for a proton recoil~\cite{Stevanato2014}.
This is sufficient to have a nearly-100\% trigger efficiency for the neutron events of interest, as their
elastic scattering on the scintillator produces protons of average energy $\sim 3.6$~MeV, giving a 1.1~MeV$_{ee}$ 
signal~\cite{Stevanato2014}.


%all seven individual cells. 

All detectors and sensors of the setup can be operated and read out remotely by means of a slow control system made of a suite 
of LabVIEW-based~\cite{LabView} applications. All parameters under control (e.g. temperatures, bias voltages, leakage currents) 
are monitored continuously, and readings are stored in a database every 10~s. 
