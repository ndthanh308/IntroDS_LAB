%Run time, data selection. \\
%Determination of coincidence window, background fraction.\\

%Run time...
The events of interest are triple coincidences between a \Be\ nucleus 
detected in the $\Delta E$/$E$ telescope, and the two subsequent neutron scatterings
in the \TPC\ and in the neutron spectrometer.

A clean sample of signal events with the proper topology was selected through
a sequence of cuts.
%by requiring the time difference between them to be compatible with 
%the velocity of a few-MeV neutron, i.e. a few cm/ns. The \TPC\ and the LSci's both feature 
%neutron/$\gamma$ discrimination based on the time profile of the scintillation signals 
%(see sec.~ \ref{sec:Calibration}) so background can be further suppressed
%by requesting that the events are compatible with neutron-induced recoils. 
Firstly, unambiguous \TPC\ events were selected
according to same criteria of Sect.~\ref{sec:Calibration}:
events with only one S1 and only one S2, separated by a $t_\mathrm{drift}$ within
the range $[6,66]~\si{\us}$. An additional S2 ``echo'' 
signal, namely a secondary event due to photo-ionization of the 
cathode from the main S2 electroluminescence, is allowed in the time 
window $[67.5, 72]~\si{\us}$ after the primary S2.

Afterwards, events in the \TPC\ were selected by requesting that S1 is in time coincindence 
within a gate of 200~ns with the $\Delta E$/$E$ telescope and with one 
single LSci detector of the neutron spectrometer. In addition, 
neutron-induced (n,n') events in the neutron spectrometer were efficiently 
selected by PSD against the dominant $\gamma$-ray background.
The PSD based on the S1 signal of the \TPC\ was not applied. This was 
meant to avoid an undesirable S1-dependent selection efficiency, given 
the fact that the discrimination based on \fprompt\ gets progressively worse 
for S1 signals below \SI{100}{PE}.

%Events passing the basic quality cut are shown in fig.~\ref{fig:LAr:ReD_Cut_TPCquality}.

%% Figure environment removed

The \Be\ ion which accompanies the neutron traveling towards the \TPC\ was selected by a combined 
cut on $\Delta E$ and $E$, which is shown in Fig.~\ref{fig:banana2} (red 
dashed contour). The selection is not sensitive enough to resolve between 
the \Be\  emitted at the ground state in the
p($^{7}$Li,$^{7}$Be)n reaction and $^{7}$Be* in the first excited state coming from the 
p($^{7}$Li,$^{7}$Be*)n' reaction. Therefore, the neutron energy distribution consisted of two different mono-energetic 
components.

%The NR scattering cuts include cuts on the SiTel and the LScis. Recoiling \ISO{Be}{7} ion is selected as $\Delta E_\mathrm{SiTel}\in [600, 900] \si{AU}$ and $E_\mathrm{SiTel} < \SI{2900}{AU}$. The selection is not sensitive enough to distinguish between $\mathrm{Li} + \mathrm{H} \to \mathrm{n} + \mathrm{Be}$ and $\mathrm{Li} + \mathrm{H} \to \mathrm{n} + \mathrm{Be}^{*}$. Therefore the neutron energy distribution consists of two components.  
%Fig.~\ref{fig:LAr:ReD_Cut_banana_all} shows the signal from the SiTel and the accepted region of the cut. Fig.~\ref{fig:LAr:ReD_Cut_banana_coin} shows the comparison between the cut and the selected neutron events with TPC coincidence. The neutrons associated with $\mathrm{Be}^*$ arrive in the late TPC coincidence window. See the following sections for the definition of the coincidence windows.

%% Figure environment removed
%% Figure environment removed
%
The data sample was further selected by using the time-of-flight (ToF) of the \TPC\ with respect to the 
$\Delta E$/$E$ telescope, namely by keeping the events in which the 
delay between the telescope and the \TPC\ (see inset in Fig.~\ref{fig:banana2}) is consistent with the 
flight time of the neutrons.
%A further selection is performed requiring that the time-of-flight between the \TPC\ and the Si
%telescope  is consistent with the flight time of neutrons coupled with both \Be\ in ground state and first %excited state (see inset in Fig. \ref{fig:banana2}).
%Fig.~\ref{fig:LAr:ReD_TPCtiming} shows the dependence of the ToF for 
%\TPC\ events on $S1$. 
The coincidence window in ToF was set to be S1-dependent, in order 
to ensure a S1-independent selection efficiency. 
The boundaries of the coincidence window were defined as the 1\% 
and 99\% quantiles in each S1 slice of 10 PE, after the subtraction of 
the constant background due to random neutrons and $\gamma$-rays. 
%\todo{[LP I removed here the discussion about the middle band, as the Be* 
%and Be discrimination is mentioned later on the 1D distribution.]} 
%Another middle point is defined as the $+2\sigma$ position of the Gaussian fit to the center peak in each $S1$ slice. The region between the lower boundary and the middle point is the prompt coincidence window, which mainly consists of neutrons associated with $\mathrm{Be}$. The region from the middle point to the upper boundary is the late coincidence window, which consists of neutrons associated with $\mathrm{Be}$ and $\mathrm{Be}^*$. 
%The signal distributions in SiTel associated with the prompt and late coincidence windows are shown in fig.~\ref{fig:LAr:ReD_Cut_banana_coin}. 
%Events in both coincidence windows are accepted as signals. 
The random background contributes to about \SI{1}{\percent} of the events 
in the coincidence windows. 

%Fig.~\ref{fig:LAr:ReD_Cut_Coinci} shows the 
%S1 distribution of the \TPC\ events in coincidence with the telescope. 
%The ER band is visible above the main NR band.
%
The coincidence windows for the delay $\Delta t(\mathrm{LSci}-\mathrm{SiTel})$ 
between the LSci and the telescope in triple-coincidence events were set 
with very stringent cuts, so 
to guarantee the selection of pure single-scattering neutron interactions.  
The timing of the individual LScis was calibrated by using as a reference
the $\gamma$-rays produced in the \TPC\ by inelastic interactions (n,n'$\gamma$) and
then detected in the LScis: all $\gamma$ peaks were aligned 
to $\Delta t(\mathrm{LSci}-\mathrm{SiTel}) = 0$, as displayed in 
Fig.~\ref{fig:LAr:ReD_Cut_Eff}, where the effect of used cuts applied sequentially is shown. 
The single-scattered neutron events of 
interest form the peak around \SI{20}{ns}. The low-statistics peak at about 
\SI{25}{ns} comes from the lower-energy neutrons produced in the 
p($^{7}$Li,$^{7}$Be*)n' interactions, while the tails at longer times are 
mostly due to multi-scattered neutron background. Monte Carlo simulations
indicate that the hump around \SI{60}{\nano\second} is originated by
the neutrons associated with the ``high energy'' \Be, which reach the \TPC\ after 
scattering on the floor or other passive structure. The peaks around 
\SI{-35}{\nano\second} and \SI{-20}{\nano\second} are 
$\gamma$-rays emitted by p($^{7}$Li,$^{7}$Li*)p inelastic scattering. 
Gaussian fits to the peak around \SI{20}{ns} determined the position and 
width of the window, individually for each scintillator. As mentioned 
in Sect.\,\ref{sec:spectrometer}, the LSci channels 
which selected \NR\ events at $\theta_r=\SI{20}{\degree}$ and 
$\SI{40}{\degree}$ were each made from the analogue sum of the signals 
of two different detectors. Since the cable lengths for the two detectors 
at $\SI{20}{\degree}$ were not properly matched, this introduced a split 
in the timing: the  $\Delta t(\mathrm{LSci}-\mathrm{SiTel})$ distribution 
for the channel at \SI{20}{\degree} was hence fitted with a double Gaussian. 
The coincidence windows were defined according to the position $\mu$ and width $\sigma$
of the peaks from the Gaussian fits, as summarized in Table~\ref{tab:LSci_Timing} and they are used to select the triple coincidence events. The
coincidence windows were further extended by \SI{5}{ns} in order to include the slower
neutrons from p($^{7}$Li,$^{7}$Be*)n'. Side-bands were also defined to estimate the 
random coincidence rate in each channel, see Tab.~\ref{tab:LSci_Timing}. 
%% Figure environment removed
%The alignment between channel mapping \#3 and \#4 is good, no relative offset is applied to the change of mapping, and the same set of coincidence windows in each channel are used. 
 
%No cut on LSci signal charge is applied in the final selection.
\begin{table*}
    \centering
     \caption[Definition of the coincidence windows in the ToF $\Delta t(\mathrm{LSci}-\mathrm{SiTel})$ for each LSci channel.]{Coincidence and side-band windows in the ToF $\Delta t(\mathrm{LSci}-\mathrm{SiTel})$ for each LSci channel. $d$ is the total width of the coincidence window, $d=6 \sigma + 5$~ns.}
    \label{tab:LSci_Timing}
\begin{tabular}{c|c|c|c|c|c}
\toprule
Angle $\theta_r$ of the \TPC\ \NR  & 90\textdegree $l$ & 40\textdegree & 0\textdegree & 90\textdegree $r$ & 20\textdegree \\
\hline
%    \makecell{$\gamma$ peak \\ offset [\si{ns}]} & -3.0    & -4.5    & -6.5     & -2.5     & -2.0     & -6.5        \\
    Neutron peak  $\mu$ [\si{ns}]   & 19.75  & 19.44   & 19.51   & 20.09   & \makecell{$\mu_1=17.18$, $\mu_2=20.44$} \\
    Timing resolution  $\sigma$ [\si{ns}] & 1.12 & 1.12  & 1.50    & 1.25    & 1.17      \\
    \hline
    Coincidence window        & \multicolumn{4}{c|}{$[\mu-3\sigma,\mu+3\sigma+\SI{5}{ns}]$} & \makecell{$[\mu_1-3\sigma$, $\mu_2+3\sigma+\SI{5}{ns}]$} \\ 
    \hline
    Side-band window          & \multicolumn{5}{c}{$[-\SI{20}{ns}-20d,-\SI{20}{ns}]\cup[\SI{70}{ns},\SI{70}{ns}+20d]$}\\
\bottomrule
\end{tabular}
\end{table*}
%
The triple coincidence events eventually considered for the statistical analysis of 
Sect.~\ref{sec:StatisticalAnalysis} are those which pass the sequence of cuts displayed 
in Fig.~\ref{fig:LAr:ReD_Cut_Eff} and the additional selection in the $\Delta t(\mathrm{LSci}-\mathrm{SiTel})$ 
ToF from Table~\ref{tab:LSci_Timing}.

%%%MOVE THIS BLOCK TO SECTION 7 %%%
%Figure~\ref{fig:LAr:ReD_Cut_Coinci} shows  the S2 vs. S1 distribution of the \NR\ events in the \TPC\ which pass the selection procedure, namely
%the sequence of cuts displayed in Fig.~\ref{fig:LAr:ReD_Cut_Eff} and the selection in the $\Delta t(\mathrm{LSci}-\mathrm{SiTel})$ ToF from
%Table~\ref{tab:LSci_Timing}: the pink dots represent the events selected requiring the triple coincidence (\TPC, Si telescope and spectrometer)
%(pink dots); the colour-coded distribution includes the events in double coincidence, \TPC\ and telescope. 

%The double coincidence events 
%constitute a large sample of \TPC\ \NR\ events is all directions: they were hence
%used to constrain the nuisance model parameters in the global analysis of Sect.~\ref{sec:StatisticalAnalysis}.
%Since the triple coincidence events are a very small fraction of the double coincidences, the large sample of double coincidence 
%events was also used as the template for random coincidence background. 
%The white contour in Fig. \ref{fig:LAr:ReD_Cut_Coinci} shows the 
%range in the (S1,S2) plane used for the statistical analysis described in 
%Sect.~\ref{sec:StatisticalAnalysis}.
% Figure environment removed

%% Figure environment removed


