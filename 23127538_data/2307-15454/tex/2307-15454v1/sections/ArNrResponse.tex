WIMPs deposit energy in LAr through elastic scattering on Ar nuclei. 
The subsequent energy loss of the \NR\ involves nuclear stopping, ionization, charge recombination, and scintillation. 
Through the series of physical processes, the total energy deposited in the \TPC\ is eventually divided into the detectable photons (S1) and electrons (S2), and the undetectable phonons (heat). 
%The charge recombination process determines the ratio between S1 and S2 signals: this could possibly depend on the angle between the recoil momentum and \edrift, thus making argon dual-phase \TPC\ potentially sensitive to the direction of the incident WIMPs through S1-S2 ratio. 

Directional modulation of charge recombination is expected when the spatial charge distribution of ionization is anisotropic. Conventional \NR\ charge recombination models often assume an isotropic charge distribution. For example, the commonly-used Thomas-Imel model~\cite{Thomas:1987ek,Szydagis:2011tk} assumes that charges are uniformly distributed in a cubic box of size $a$; the only free parameter for a given detector material is the initial charge $Q_0$. 
The probability of charge surviving recombination under the electric drift field \edrift\ is 
\begin{equation}
    p(a,Q_0) = \frac{\mathcal{E}_d}{\xi(a,Q_0)}\ln\left(1+\frac{\xi(a,Q_0)}{\mathcal{E}_d}\right),
\end{equation}
where
\begin{equation} 
\xi(a,Q_0)= \frac{\alpha Q_0}{4a^2\mu_-}; \label{eq:TIxi}
\end{equation}
$\alpha$ is the Langevin
recombination coefficient~\cite{Langevin1903,Bubon:2016hc}, which depends on the carrier  
mobilities ($\mu_-$ and $\mu_+$ for electrons and ions, respectively) and on the dielectric constant as
\begin{equation}
\alpha = \frac{(\mu_-+\mu_+)}{\epsilon_0 \epsilon_r}. \label{eq:alpha}
\end{equation}

In order to introduce the directionality, the electron distribution after thermalization needs to be included in the model. One approach is to use the Jaff\'e  model~\cite{Jaffe:1913gs,Birks:1951boa}, commonly referred to as the columnar recombination model, where the charge distribution is modeled by a column with radius $b$, length $l$, and angle $\theta$ between its axis $\hat{r}_0$ and the drift field \edrift. The Jaff\'e model is commonly adopted for the straight tracks from minimum ionizing particles. Since \NR\ tracks are more localized, a more general and flexible parameterization
of the charge distribution $q_0(\vec{r})$ has been proposed by Cataudella et al.~\cite{Cataudella:2017kcf}, which consists of 
a three dimensional Gaussian with an elliptical profile
\begin{equation}
    q_0(\vec{r}) = \frac{Q_0}{(2\pi)^{3/2}R\sigma^3}\exp{\left(-\left(\frac{\vec{r}\cdot\hat{r}_0}{R\sigma}\right)^2 - \left(\frac{\vec{r}\times\hat{r}_0}{\sigma}\right)^2 \right) }, \label{eq:clouddist}
\end{equation}
where $\sigma$ characterizes the size of the distribution, $\hat{r}_0$ is the direction vector of the long axis, and $R$ is the aspect ratio between the long and short axes. 
 The  probability of charge surviving recombination is calculated in Ref.~\cite{Cataudella:2017kcf} as
\begin{equation}
\label{eq:rec_direction}
    p(R,\theta,Q_0) = -\frac{\mathcal{E}_d f(R,\theta)}{\xi_m}\mathrm{Li}_2\left(-\frac{\xi_m}{\mathcal{E}_df(R,\theta)}\right),
\end{equation}
where
\begin{equation}
\xi_m=\frac{\alpha Q_0}{2\pi \sigma^2\mu_-} \label{eq:ximCataudella}
\end{equation}
is the generalization of the Thomas-Imel parameter $\xi$ of
Eq.~\ref{eq:TIxi} and $\mathrm{Li}_2$ is the second order polylogarithm function.
%, and $\theta$ is the angle between $\hat{r_0}$ and
%the drift field.
%Since the ion mobility is negligible compared to electrons,
%By replacing $\alpha$ from Eq.~\ref{eq:alpha}, Eq.~\ref{eq:ximCataudella} can be recast as 
%\begin{equation}
%\xi_m = \frac{Q_0}{2\pi\varepsilon_r\varepsilon_0\sigma^2}.
%\end{equation}
The term $f(R,\theta)$ captures the directionality dependence and it has 
the functional form
\begin{equation}\label{eq:rec_modified_TI_ftheta}
    f(R,\theta) = \sqrt{R^2\sin^2\theta+\cos^2\theta},
\end{equation}
being $\theta$  the angle between $\hat{r}_0$ and \edrift.
When $R=1$, $f(R,\theta)=1$, so directionality vanishes and 
Eq.~\ref{eq:rec_direction} reduces to the Thomas-Imel model.

Since directionality effects do not occur before recombination, well-established models are used here to describe the S1 and S2 yields, that for \NRs\ also depend on nuclear and electronic quenching. Following the Lindhard model~\cite{Lindhard:1963vo,Bezrukov:2010qa}, the nuclear quenching factor, i.e. the ratio of the visible energy in the excitation and ionization channel to the total recoil energy, is described by 
\begin{equation}
    f_{n}(\varepsilon) = \frac{kg(\varepsilon)}{1+kg(\varepsilon)},
\end{equation}
where $k=0.133\, Z^{2/3}A^{-1/2}$ is a dimensionless factor depending on the Ar target nucleus ($A=40$, $Z=18$); the function $g(\varepsilon)$ is numerically approximated by 
Lindhard~\cite{Lindhard:1963vo} and it has the form
\begin{equation}
g(\varepsilon) = 3\,\varepsilon^{0.15}+0.7\,\varepsilon^{0.6}+\varepsilon;
\end{equation}
finally  $\varepsilon$ is the dimensionless reduced energy
\begin{equation}
    \varepsilon = \frac{4\pi \epsilon_0 a}{2e^2Z^2}E_{r} 
    = 11.5\, Z^{-7/3}\frac{E_r}{[\si{\kilo\electronvolt}]},
\end{equation}
being  $E_r$ the recoil energy  and $a$ the Thomas-Fermi screening length, calculated from
the Bohr radius $a_0$ as $a = 0.626 \cdot a_0 \cdot Z^{-1/3}$~\cite{Bezrukov:2010qa}.

The measurable energy is further reduced by electronic quenching, following 
the Mei model~\cite{Mei:2008ca}
\begin{equation}
    f_{l} = \frac{1}{1+k_e s_e}, 
\end{equation}
where $s_e=k\varepsilon^{1/2}$ is the dimensionless electronic stopping power 
and $k_e$ is associated to the original parameter $K_e$ of 
Ref.~\cite{Mei:2008ca} as $k_e = K_e (\mathrm{d}E/\mathrm{d}x)_e / (s_e \rho_m)$, with $\rho_m$ being the mass density of LAr. 
%K_e=\SI{7.4e-4}{\mega\electron\volt^{-1}\gram\centimeter^{-2}} from Mei, equal to k_e=4.1, SCENE adopts k_e=2.8\pm0.1. Here we adopt the same value as SCENE.

%The 
%ratio of energy flowing to the excitation channel and the ionization channel is expressed by the parameter %\NexNi. 
Summing up the components, the expectation of total quanta $\langle N_0 \rangle$, ionization $\langle N_\mathrm{i} \rangle$ and excitation $\langle N_\mathrm{ex} \rangle$ from a \NR\ of energy $E_r$ in LAr before recombination is
\begin{eqnarray}
    \langle N_0 \rangle &=& \frac{E_r f_n f_l}{W_\mathrm{ph}} \\
    \label{eq:NexNiYield}
    \langle N_\mathrm{i} \rangle &=& \langle N_0 \rangle \frac{1}{1+N_\mathrm{ex}/N_\mathrm{i}} \\
    \langle N_\mathrm{ex} \rangle &=& \langle N_0 \rangle - \langle N_\mathrm{i} \rangle %\left(1-\frac{1}{1+N_\mathrm{ex}/N_\mathrm{i}}\right) .
\end{eqnarray}
where $W_{ph}$ is the average energy required to produce one scintillation photon in LAr and \NexNi\ is 
the excitation-to-ionization ratio directly induced by the fast ion and by its secondaries.
As a first approximation, \NexNi\ is usually treated as an energy independent constant~\cite{Doke:2002oab,Hitachi:2021hac}, which is related to the atomic levels in argon. However, the distribution of momentum transfer to electrons in the electronic stopping power is energy-dependent, which motivates the introduction of a variable \NexNi\, vs. energy. This is corroborated by the SCENE data~\cite{Cao:2015ks}, which also indicate an increase in \NexNi\ with respect to the \NR\ energy. The \NexNi\ values adopted for this work are taken from Table VIII of Ref.~\cite{Cao:2015ks}, with a
linear interpolation between the energy points. The \NexNi\ at zero energy is set to the commonly-adopted
value of $0.2$. 

The detectable electron and photon yields after recombination are
%\begin{eqnarray}
 %   \langle N_\mathrm{e^-} \rangle &=& \langle N_0 \rangle \frac{1}{1+N_\mathrm{ex}/N_\mathrm{i}}\, %p(R,\theta,\langle N_\mathrm{i}\rangle) \label{eq:ne} \\ 
 %   \langle N_\mathrm{ph} \rangle &=& \langle N_0 \rangle \left(1 - %\frac{1}{1+N_\mathrm{ex}/N_\mathrm{i}}\, p(R,\theta,\langle N_\mathrm{i}\rangle)\right). \label{eq:nph} 
%\end{eqnarray}
\begin{eqnarray}
\langle N_\mathrm{e^-} \rangle &=& \langle N_\mathrm{i} \rangle p(R,\theta,Q_0) = \langle N_\mathrm{0} \rangle \frac{p(R,\theta,Q_0)}{1+N_\mathrm{ex}/N_\mathrm{i}} \label{eq:ne} \\ 
\langle N_\mathrm{ph} \rangle &=& \langle N_0 \rangle -  \langle N_\mathrm{e^-}\rangle. \label{eq:nph} 
\end{eqnarray}


%% Figure environment removed


%Now consider the $S1$ and $S2$ gain of the \TPC. 
%The \TPC\ signals S1 and S2 are measured in units of photo-electrons (PE) in the photosensor. The scintillation light collection efficiency and the photosensor quantum efficiency are less than unity, so that S1 has a gain $g_1 = \mathrm{S1}/N_\mathrm{ph} [\si{PE/ph}]$ less than one. Charges are amplified and converted to photons through the gas pocket electroluminescence, so that S2 has a gain $g_2 = \mathrm{S2}/N_\mathrm{e^-} [\si{PE/e^-}]$ greater than one. 

%Fluctuations in signals are particularly important for the directionality study, as a directional modulated signal could be washed out by random fluctuations: fluctuations hence reduce the resolution of potential signal directionality. The intrinsic fluctuations during signal generation in LAr and the detector resolution rising from signal propagation, collection, and amplification contribute to the S1-S2 two-dimensional spectrum profile of a mono-energetic \NR. 
The capability to measure the \NR\ direction can be hidden by random fluctuations in S1 and S2. Indeed, the intrinsic fluctuations during signal generation in LAr and the detector resolution arising from signal propagation, collection, and amplification contribute to smear out the S1-S2 two-dimensional spectrum of a mono-energetic \NR. 
The intrinsic fluctuations are present for both the charge and light channels. The fluctuation in the total number of visible quanta $N_0$ is assumed here to be Gaussian distributed with a Fano factor $F=0.107$ \cite{szydagis2021review}: 
\begin{equation}
    N_0 \sim \mathrm{Gaussian}(\langle N_0 \rangle, \sqrt{F\langle N_0\rangle}).
\end{equation}
%Since there is a significant and degenerated contribution from the \NR\ energy spread in our setup, the result is not sensitive to the different choices of nuclear quenching fluctuation. For simplicity, we stick to the Poisson distribution.
The partition of $N_0$ between $N_\mathrm{e^-}$ and $N_\mathrm{ph}$ follows a 
binomial distribution governed by \NexNi\ and by the recombination probability (see Eq.~\ref{eq:ne}):
%\begin{eqnarray}
%N_\mathrm{e^-} \sim \mathrm{Binomial}(\langle N_0 \rangle, \langle N_\mathrm{e^-} \rangle / \langle  N_0 \rangle)  \\ 
%N_\mathrm{ph} =N_0-N_\mathrm{e^-}
%\end{eqnarray}
\begin{equation}
    N_\mathrm{e^-} \sim \mathrm{Binomial}(N_0, \langle N_\mathrm{e^-} \rangle / \langle  N_0 \rangle)
\end{equation}
and $ N_\mathrm{ph} =N_0-N_\mathrm{e^-}$.
%Traditionally, the Fano factor is defined to characterize the charge yield fluctuation in ionization detectors. 
%The binomial description is a good approximation when the difference in energy levels of the signal channels is ignored and charge recombination is significant, which applies to \NR\ in LAr. 
%Note that the concept of Fano factor only applies to cases where the stochastic process of charge or light generation and propagation does not depend on external parameters. For example, if the directional modulation of recombination is considered as signal fluctuation, the fluctuation would exceed the width described by the Fano factor. Here we write out the signal yield dependence on $\theta$ explicitly.

The \TPC\ signals S1 and S2 are measured in units of photo-electrons (PE) in the photosensor. The scintillation light collection efficiency and the photosensor quantum efficiency are less than unity, so that S1 has a gain $g_1 = \mathrm{S1}/N_\mathrm{ph} [\si{PE/ph}]$ less than one. Charges are amplified and converted to photons through the gas pocket electroluminescence, so that S2 has a gain $g_2 = \mathrm{S2}/N_\mathrm{e^-} [\si{PE/e^-}]$ greater than one. 

The stochastic processes of collection of the scintillation light can be described by a binomial distribution, using the gain $g_1$. For S2, the electroluminescence process is described by a Poisson distribution. The detector response also includes a position-dependent non-uniformity which could in principle be corrected in analysis. Practically, a small residual error will be present, which can be modeled by an additional Gaussian smearing of standard deviation 
$\sigma_ \mathrm{S1}^{*}$ and $\sigma_ \mathrm{S2}^{*}$ for S1 and S2, respectively. 
Approximating the S1 and S2 distributions with Gaussians, the total contribution from detector response is 
\begin{eqnarray}
    \mathrm{S1} &\sim& \mathrm{Gaussian}\left( \langle N_\mathrm{ph} \rangle g_1, \sqrt{\langle N_\mathrm{ph} \rangle g_1 (1-g_1) + \sigma_ \mathrm{S1}^{*2}}\right) \\
    \mathrm{S2} &\sim& \mathrm{Gaussian}\left( \langle N_\mathrm{e^-} \rangle g_2, \sqrt{\langle N_\mathrm{e^-} \rangle g_2 + \sigma_\mathrm{S2}^{*2}} \right).
\end{eqnarray}

In conclusion, the argon dual-phase \TPC\ response to a mono-energetic \NR\ follows the probability density function coming from the convolution of the detector and physical terms:
\begin{eqnarray}
    P(\mathrm{S1},\mathrm{S2}) & = &
    P_\mathrm{detector}(\mathrm{S1}/g_1,\mathrm{S2}/g_2;N_\mathrm{ph},N_{e^-}) \nonumber\\
    & & \otimes 
    P_\mathrm{NR}(N_\mathrm{ph},N_{e^-};E_r,R,\theta)  \nonumber\\
    & = & \frac{1}{2\pi\sigma_\mathrm{S1}\sigma_\mathrm{S2}/g_1 g_2} \nonumber\\
    & &
    e^{-\frac{(\mathrm{S1}/g_1 - N_\mathrm{ph})^2}{2(\sigma_\mathrm{S1}/g_1)^2}
    -\frac{(\mathrm{S2}/g_2 - N_{e^-})^2}{2(\sigma_\mathrm{S2}/g_2)^2}} \nonumber\\
    & & \otimes\frac{1}{2\pi\sqrt{F\langle N_\mathrm{ph}\rangle\langle N_{e^-}\rangle}} \nonumber\\
    & & e^{ -\frac{(N_{e^-}+N_\mathrm{ph}-\langle N_0\rangle)^2}{2F\langle N_0 \rangle}
    -\frac{(N_{e^-}\langle N_\mathrm{ph} \rangle - N_\mathrm{ph}\langle N_{e^-}\rangle)^2}
    {2\langle N_{e^-} \rangle\langle N_\mathrm{ph} \rangle\langle N_0\rangle}}. \nonumber\\
    \label{eq:TPCResponse}
\end{eqnarray}
Later in Sect.~\ref{sec:StatisticalAnalysis}, a likelihood function is evaluated from the \TPC\ data using this probability density function. An unbinned profile likelihood study is then performed to determine the confidence interval of the directionality parameter $R$.
