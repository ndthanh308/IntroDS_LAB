The directionality analysis presented in this work depends on 
the specific model by Cataudella et al.~\cite{Cataudella:2017kcf} which 
was adopted to describe the phenomenon. One possibility to release such a 
model depencence and to validate the result is to employ a data-driven 
approach based on Machine Learning (ML) techniques. ML techniques are in 
fact very effective in revealing possible correlations between quantities 
in the study of phenomena for which large data samples are available, 
even if a model for their description is lacking.

Supervised learning algorithms were used to try to highlight the signature for possible directionality effects in the electron-ion recombination in the \ReD\, data~\cite{noemipthesis}. 
%The chosen learning approach was the supervised one: an ML algorithm is trained using data samples (suitably described by a set of numerical information called \emph{features}) for which the output is originally known. 
%Luciano. Cut sentence
%The variables measured in the \TPC\ could be used as input features and the corresponding $\theta_r$ information, reconstructed through the triple coincidence in the same event, as the target value of the non-linear regression problem. This approach would allow probing \emph{directly} the possible existence of correlations between the information made available by the \TPC\ and the angle formed by the recoil nucleus (which produced the ionization track) with the direction of the electric field. 
Due to the limited size of the triple coincidence event samples, an indirect approach was adopted, which makes use of all \TPC\ calibration events. The data set of the double coincidence events provides a two-order-of-magnitude larger amount of data for the training of the model, and this is a desirable condition when working with ML algorithms.

In an ideal \LArTPC, the S2 signal is expected to be related to S1 through some functional form S2$=f$(S1). The basic 
assumption of this strategy is that the function $f$ does not depend on the direction of the Ar recoil, namely that 
the angle $\theta_r$ between the recoil and the drift field does not affect the balance between S1 and S2. 
Deviations from this trend would highlight a possible effect of the recoil directionality.

The model was derived by using the calibration data set, which is made of NR events characterized by a
wide distribution of angles $\theta_r$.  The data set contained about 72000 events and it was randomly split (70:30) in a training 
and testing set, on which the model was trained and tested, respectively. During the training phase, the algorithm built the function $f$ used to predict S2, event-by-event, based on the patterns which are learnt from the training set. 
 %More precisely, during the so-called \textit{training phase}, the algorithm builds the prediction function used to predict $S2$, event-by-event, exploiting \emph{patterns} contained in the training set. 
Each pattern consists of a vector of features: S1 signal [PE], \xy\, position [cm], and \tdrift [\textmu s] as the $z$ coordinate of the event, within the appropriate ranges, and the measured S2 value as a target.
The derived model aims to predict the value of the ionization signal S2, for each of the events, from the knowledge of S1 and of the reconstructed interaction point within the \TPC.
%Target S2 values were chosen to be proportional to the corresponding S1. In this way, the model was trained to suitably mitigate factors affecting the S1-S2 ratio other than that of the recoil directionality.

The Extreme Gradient Boosting algorithm (\texttt{xgboost}) was used to derive the model~\cite{XGB}. 
%This is an algorithm well-known in the literature for its good performance and multi-field-possible application to various problems, including regression ones. 
To evaluate the accuracy of the model, the metric of the relative prediction error was adopted. This 
is defined, for the $i$-th pattern (i.e. the $i$-th event in the \TPC), as

\begin{equation}
\epsilon_ {pred}^i =\frac{\mathrm{S2}_{measured}^i - \mathrm{S2}_{predicted}^i}{\mathrm{S2}_{measured}^i}.
\label{par}
\end{equation}

The trend of $\epsilon_{pred}^i$ was investigated for each event, and also against each feature describing the patterns, to verify that there were no regions in the feature space in which the model has a worse response that could introduce any bias in the predictions. 
%A fraction (about $20\%$) of the testing dataset, was used as the \emph{cross-validation} dataset to tune the \emph{hyperparameters} of the algorithms. 
%At the end of the training phase, the derived model with the optimal set of hyperparameters was able to predict the experimental $S2$ with a mean absolute percentage error lower than $8\%$. 
%Once the model was derived and tested, it was used to make predictions on the triple coincidence dataset. 
At the end of the training phase, the model was able to provide a satisfactory prediction of the experimental S2
of the events in the testing set: the relative errors $\epsilon_ {pred}^i$ resulted to be approximately Gaussian-distributed
with mean 0.0043(6) and standard deviation 0.09.
 
Subsequently, the model was used to make predictions on the triple coincidence data set.
%This set of events included patterns with the same structure as the double coincidence dataset used in the testing phase of the model. 
For these data, %(455 events in total), 
the known $\theta_r$ values are used to check for possible directional-dependent deviations of the predicted S2 values compared to those measured experimentally. 
%If this deviation was found with a statistic such as not to attribute it to random fluctuations, but rather to a certain different behavior of the recombination as the recorded $\theta_r$ angles vary, it would be possible to trace it back to the denied dependence of the angle. 
$\epsilon_ {pred}^i$ was initially calculated for each event in the triple coincidence data set and the corresponding values were subdivided into four subsets, according to the angle $\theta_r$ determined by the coincident neutron detection. The mean value of $\epsilon_ {pred}$ in each data set and the corresponding uncertainty are displayed in Fig.~\ref{fig:points} as a function of the recoil direction $\theta_r$.
%
% Figure environment removed
The point at $\theta_r$ = \SI{0}{\degree} is lower than the others,
%which, instead, are compatible with zero deviation from the experimental value. This shows that the derived model tends to overestimate S2
%for events with traces parallel to \edrift. Such a result is
as expected in the case of directionality effects, since traces parallel to \edrift\ would result in enhanced S1 signals and reduced S2.
Nevertheless, experimental data are compatible with the null hypothesis of no directionality effect: the $p$-value calculated from the
$\chi^2$ test is 23\%. Therefore, the data-driven analysis carried out using ML techniques on the data collected in the \ReD\ \TPC\ is
compatible with the absence of any directional effect\footnote{A dedicated Monte Carlo based sensitivity study confirmed that, despite the small
size of the triple-coincidence sample, a directional effect as hinted by SCENE (7\% difference in S1 between parallel
and perpendicular recoils) would have been detected by this analysis at 3.2 $\sigma$ level.}, in agreement with the analysis based on
the model by Cataudella et al.~\cite{Cataudella:2017kcf}. 

%A higher-statistics data sample will allow to conclusively test the 
%hint for a difference in the behaviour of the system for NRs parallel to the \edrift. 


%The exploiting of a higher statistics dataset will allow to further test the robustness of the result.
%A simple geometrical model, in which the ionization trace is considered with almost zero thickness, can justify in a semi-quantitative way the trend of Fig.~\ref{fig:points}.
%The exploiting of a higher statistics dataset will allow to further test the robustness of the result, together with the development of new possible models adopting a direct approach to data.
