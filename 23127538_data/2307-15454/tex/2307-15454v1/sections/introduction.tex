A range of evidences from astronomy and cosmology~\cite{Ade:2016bk,Rubin:1999hb,Clowe:2006hr,Covone:2009ji,Malaney:1993ep} indicates that a substantial fraction of the Universe is made of non-baryonic dark matter, whose nature is still unknown. 
Weakly interacting massive particles (WIMPs), a common candidate, are actively searched for by many experiments worldwide using different technologies~\cite{Schumann:2019eaa,Cebrian:2022brv,Roszkowski:2017nbc}.  
The expected signal of those direct dark matter experiments is the nuclear recoil (NR) induced by the WIMP elastic scattering, having energy up to a few tens of \si{\kilo\electronvolt}. To improve sensitivity, it is crucial
to strongly suppress any NR generating contamination, until the ``neutrino fog'' is reached, i.e. the irreducible background from coherent elastic neutrino-nucleus scattering.

While evidence of WIMPs could be claimed based on an excess of NRs with respect to the expected background, a convincing discovery requires the observation of the effect in many target materials and the consistency with strong and unmistakable dark matter signatures. The motion of the Solar system relative to the galactic dark matter halo creates an apparent WIMP flux through terrestrial detectors coming from the direction opposite to the Earth's velocity vector, i.e. approximately from the Cygnus constellation. Furthermore, the motion of the Earth around the Sun generates annual modulations in the flux strength and direction. These signatures can be detected in the angular distribution of NRs produced by WIMP elastic scattering in a terrestrial detector, which would be an unmistakable ``smoking gun'' for dark matter, as none of the known backgrounds, including coherent neutrino scattering, can mimic it. 

Directional sensitivity would hence be a crucial asset for future direct dark matter search experiments, especially when claiming a signal. Even in a detector with moderate angular resolution ($\sim \SI{20}{\degree}$) for NRs, a few hundreds of events would be sufficient to reject the hypothesis of isotropic incident flux at $3 \sigma$ level~\cite{Cadeddu:2017ebu}. A number of R\&D programs is currently
in progress for directional direct dark matter search~\cite{Ahlen:2009ev,Hochberg:2016ntt,Battat:2016pap,Vahsen:2020pzb,Belli:2022jqq}.

One of the most promising current approaches for the direct search of WIMPs is based
on argon dual-phase Time Projection Chambers (\TPC).
%Due to the low mass number of Ar, the elastic scattering
%of WIMPs of mass $\sim 100$~GeV/c$^2$ is expected to produce NRs of kinetic energy up to
%100~keV. The dual-phase \TPC\ technology
It offers extremely low background thanks to the efficient rejection of
the electron recoil (ER) background provided 
by pulse shape discrimination (PSD)~\cite{Amaudruz:2016dq} and the use of low-radioactivity argon from underground 
sources~\cite{Galbiati:2007xz,Aalseth:2020nwt}. Based on the successful experience of the
DarkSide-50 experiment~\cite{Agnes:2015gu,Agnes:2015ftt,Agnes:2018ves} at the Gran Sasso Laboratory (LNGS) of
INFN and of the DEAP-3600 experiment~\cite{DEAP-3600:2017uua,DEAP:2019yzn} at SNOLAB, the Global Argon Dark Matter Collaboration (GADMC) is pursuing a multi-staged experimental program aiming to improve the sensitivity down to the ``neutrino fog''~\cite{Aalseth:2017fik}.  
Currently, GADMC is preparing for the DarkSide-20k experiment~\cite{Aalseth:2017fik} which features a 50 tonne underground argon dual-phase \TPC\ with Silicon Photomultiplier readout.

% Figure environment removed
The working principle of an argon dual-phase \TPC\ is depicted in Fig.~\ref{fig:tpcworking}. The \TPC\ contains a volume of liquid argon (LAr) with a thin layer of gaseous argon, the gas pocket, on the top. The elastic scattering of a hypothetical WIMP particle 
with an Ar nucleus in the \TPC\ would originate a NR of kinetic energy ranging from $\sim 20$ to $\sim 100$~keV,
which ionizes the medium along 
its trajectory. The energy deposition of an ionizing particle which travels in the liquid volume produces excitation and ionization, giving rise to excited argon dimers (Ar$_2^*$) and to electron-ion pairs. The de-excitation of Ar$_2^*$ dimers, some of which are produced by electron-ion recombination, emits scintillation light, which produces the S1 signal. 
The residual unrecombined ionization electrons are swept away from the interaction site and drifted towards the liquid-gas interface by an appropriate electric field, the drift field \edrift. 
They are extracted to the gas phase and accelerated by intense fields, the extraction field \eex\!  and the electrolumiscence field \eel, respectively. 
Accelerated electrons in the gas phase emit light by electroluminescence~\cite{Aalseth:2020zdm,Buzulutskov:2020xhd}, which is the S2 signal. 
The S1 and S2 signals are separated by the time interval corresponding to the electron drift time from the interaction site to the gas phase.  The S2 signal intensity is proportional to the number of extracted electrons.
The recombination of electrons with ions produces Ar dimers at the expense of free charge and therefore affects the balance between the intensity of S1 and S2 signals. 

A dual-phase \TPC\ could potentially offer a directional sensitivity for the events featuring long straight ionization tracks, thanks to the mechanism of 
columnar recombination~\cite{Jaffe:1913gs,Birks:1951boa,Cataudella:2017kcf}. 
When the track is nearly parallel to \edrift, electrons pass through the electron-ion column from the track itself and have 
a higher probability to meet an Ar ion and recombine, compared to a perpendicular track. 
Events with tracks parallel to \edrift\ are therefore expected to have an enhanced S1 and a reduced S2.
%As discussed in more detail in Sect.~\ref{sec:ArResponse}, the detector response in $S1$ and $S2$ to a ionising particle 
%depositing a given energy $E$
According to Refs.~\cite{Nygren:2013fy,Cao:2015ks}, the directional dependence occurs only if
 the charge cloud around the ionization track is anisotropic, namely
when the ionizing track is longer than the Onsager radius $r_O$, the distance between an ion and a free electron for
which the electrostatic potential energy equals the thermal kinetic 
energy of the electron.
As $r_O = e^2/(6 \pi \epsilon_0 \epsilon_r k_B T)$ is about \SI{80}{\nano\meter} in LAr ($T = 87$~K, $\epsilon_r = 1.5$), argon ions 
with kinetic energy above $\sim \SI{40}{\kilo\electronvolt}$ have a range longer than $r_O$~\cite{srim}. 
Therefore, an argon dual-phase \TPC\ could potentially be direction-sensitive in the energy range 
of interest for WIMP searches. 
%However, $r_O$ is not the only relevant length scale in recombination. 
However, calculations and simulations~\cite{Wojcik:2003ja,Wojcik:2016gy} show 
that the mean thermalization distance of electrons in LAr is about 
\SI{2.6}{\micro\meter}, which is much longer than the Onsager radius 
and of the range of WIMP-induced recoils. As recombination mostly takes 
place when electrons are fully thermalized, the directional 
sensitivity could hence be diluted by electron diffusion during thermalization. 

The SCENE Collaboration has provided a hint of directional sensitivity in the S1 signal for NRs of about 
\SI{60}{\kilo\electronvolt}~\cite{Cao:2015ks}, and specifically a difference of about 7\% on S1 for NRs parallel
and perpendicular to the drift field, at \edrift=193~V/cm. 

The breakthrough that the directional sensitivity of an argon \TPC\ would offer in 
the framework of direct dark matter searches motivated the Recoil Directionality (\ReD) experiment, 
as a part of the program of the GADMC. \ReD~\cite{Agnes:2021zyq} has been designed and performed with 
the goals to scrutinize the hint by SCENE, by testing a directional effect with a size as reported by 
SCENE, and additionally to provide new experimental data to improve the understanding of recombination and 
thermalization of electrons in LAr.

To this aim, a miniaturized argon dual-phase \TPC\ was irradiated with neutrons at INFN, Laboratori Nazionali del Sud, to produce NRs at a variety of angles with respect to the \TPC\ drift field. 
The kinetic energy of NRs is around \SI{70}{\kilo\electronvolt}, which falls in the range of interest of WIMP search in Ar and corresponds to an ion range larger than the Onsager radius.
This work is organized as follows: Sect.~\ref{sec:ArResponse} discusses the models to
describe the response of an argon dual-phase \TPC\ to NRs of the energy relevant for
dark matter searches, including the potential directional dependence. The experimental layout of \ReD\
and the description of the individual detectors are given in Sect.~\ref{sec:DetectorLayout} and \ref{sec:Detectors}, 
respectively. 
The data treatment, including reconstruction calibration, event selection, and the subsequent statistical analysis for
directional sensitivity are presented in Sect.~\ref{sec:DataAndResult} and~\ref{sec:StatisticalAnalysis}. 
The results and their potential impact are discussed in Sect.~\ref{sec:discussion}, followed by the
summary of conclusions in Sect.~\ref{sec:conclusions}.

