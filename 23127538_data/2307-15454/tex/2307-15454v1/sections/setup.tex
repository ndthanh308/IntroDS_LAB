The experimental layout is conceived in order to produce and detect Ar nuclear recoils of known energy and direction, by neutron elastic scattering. 
%The individual tagging is achieved by making use of two-body reactions, which have closed kinematics. 
Neutrons are produced by the primary reaction p($^{7}$Li,$^{7}$Be)n, by shooting a $^7$Li beam  on a polyethylene (CH$_2$) target. %The neutron is produced in association with a $^7$Be nucleus. 
The neutron energy $E_n$ and its direction are 
kinematically determined by measuring the energy and direction of the accompanying $^7$Be nucleus.
The neutron can undergo elastic scattering $(n,n')$ with an Ar nucleus inside the \TPC, thus producing a \NR\ 
and a secondary neutron whose energies and momenta are again correlated by two-body kinematics. The scattered neutron is
eventually detected by a neutron spectrometer made by an array of liquid scintillator (LSci) detectors; the detection of
the neutron by a specific LSci determines the energy and the direction of the Ar recoil.

% Figure environment removed

The conceptual layout of ReD is sketched in Fig.\,\ref{fig:schema}. The experiment deploys three detector
systems: (1) a $\Delta E$/$E$ telescope made by Si detectors, to identify \Be\ nuclei associated with 
neutrons; (2) the \TPC\ to detect the Ar NRs; (3) a neutron spectrometer made by 7~LSci detectors to detect the
neutrons scattered off Ar. The detectors of the neutron spectrometer are placed along the base circumference of a cone with axis corresponding to the target-TPC line (i.e. the direction of the incoming neutron), vertex on the TPC center and  opening angle $\theta_{lsci}$. Therefore, all LScis detect neutrons which undergo elastic scattering on Ar at the same angle and hence produce NRs of the same energy \Er. 
While the NRs tagged by the seven individual LScis all have the same energy $E_r$, 
%\Er\ and the same scattering angle with respect to the incoming neutron, 
their momenta $\vec{p}_r$ form a different angle $\theta_r$ with respect to the \TPC\ electric field ($z$ axis in Fig.\,\ref{fig:schema}), as
required to test the directional effect. As it is important for this work 
to test the response to NRs also at $\theta_r =180$\textdegree, the \TPC\ 
is placed at a different level with respect to the target, such to provide 
the incoming neutron with a momentum component along the field direction.

Once  the angle $\theta_{tpc}$ between the primary \Li\ beam direction and the target-TPC direction and the angle 
$\theta_{lsci}$ are fixed by the setup geometry, \ReD\ is tuned to select mono-energetic Ar recoils of energy \Er\ by the triple 
coincidence between the Si telescope, the \TPC\ and the neutron spectrometer. 
%version2
%\footnote{Since it is important to
%have the possibility to tag NRs with momentum parallel to the drift field, the incoming neutron 
%cannot be exactly perpendicular to the drift field, i.e. the \TPC\ cannot be located 
%at the same level as the target}.
%Some LSci's are specifically placed to select the notable angles 
%$\theta_r = 90$\textdegree (recoil perpendicular to \edrift) and 180\textdegree (recoil parallel/antiparallel to 
%\edrift). 
%Since it is critical to
%select a population of NRs with momentum (anti)parallel to the drift field, the incoming neutron cannot
%be exactly perpendicular to the drift field, i.e. the \TPC\ cannot be located at the same level as the target: the target-TPC 
%vector must hence have a component along the z-axis. 
The operational parameters chosen for ReD are
$\theta_{tpc} = 22.3$\textdegree\ and $\theta_{lsci} = 36.8$\textdegree. The target-TPC distance and the TPC-LSci distance are 150 and
100~cm respectively, as a reasonable compromise between angular resolution and solid angle coverage: in both cases
the uncertainty on the neutron direction is driven by the dimensions of the \TPC\ and of the LSci, i.e. by the uncertainty
on the interaction point within them. Keeping the geometry fixed, the energy \Er\ of the NR can be changed by varying the primary beam energy.
The \ReD\ experimental layout was designed to allow for the measurements of NRs in the range of interest for dark matter direct searches,
between 20 and 100~keV: this can be achieved by varying the energy of the primary \Li\ beam between 20 and 34~MeV.


\subsection{\Li\ beam and target} \label{sec:nbeam}
 The primary \Li\ beam  is produced by the 15~MV TANDEM accelerator of the INFN LNS~\cite{CIAVOLA199364} at an energy of  28~MeV. The TANDEM offers an excellent resolution in the delivered energy, which is about 1\% FWHM in our case. The data reported in this work were collected 
between January 31$^{st}$ and February 14$^{th}$, 2020. The current of the \Li\ beam ranged between 5 and 15~nA,
 corresponding to $1-3 \cdot 10^{10}$~(\Li/s). 
The beam is driven to a vacuum scattering chamber, which hosts the CH$_2$ target and the $\Delta E$/$E$ telescope. Upstream the target,  the \Li\ beam is collimated to obtain a spot of 2~mm diameter at the target position. Neutrons are produced via the p($^{7}$Li,$^{7}$Be)n reaction. 
 The $\Delta E$/$E$ telescope detects the $^7$Be accompanying the neutrons that travel towards the TPC. As the accelerator 
does not allow the production of a pulsed beam,
the direct detection of \Be\ represents the best solution for event-by-event neutron tagging. The requirement to
detect \Be\ drives the choice of inverse kinematics (i.e. \Li\ beam on a hydrogenous target)~\cite{Drosg1981,Dave1982}, instead of the
direct kinematics approach (proton beam on a \Li\ target) employed by other experiments, as SCENE~\cite{Cao:2015ks}. 
 
 %The chosen to select NRs of energy 
 %$E_r = 72$~keV, whose range in LAr is about 180~nm~\cite{srim}, substantially longer than the Onsager %radius ($\sim 80$~nm).   In this configuration the neutrons traveling towards the \TPC\ which are tagged by 
%by using a two-holes collimator (hole diameters are 2 and 3~mm and they are placed 6~mm apart) placed upstream the target. 
The targets of CH$_2$ have thickness ranging between 150 and 350~\textmu g/cm$^2$, which is thin enough to allow for the escape of \Be. Due to aging effects, each target was used for about 12~hours of data taking, before being replaced by means of a 
12-target holder system placed inside the vacuum scattering chamber.

After the target, the \Li\ beam travels straight forward towards a Ta beam dump placed 3~m downstream (see 
Fig.~\ref{fig:schema}). Such a long distance is functional to minimize the background on the ReD setup due to the 
beam interaction on the beam dump.
The beam intensity was precisely measured every few hours of operation by a 
Faraday Cup deployed about 30~cm downstream the target. However, the Faraday Cup was removed during the data taking, in order
to reduce the background radiation close to the \TPC. The continuous monitoring of the beam intensity
was performed by measuring the rate of the \Li\ elastic scattering on a dedicated Si detector (not shown in Fig. \ref{fig:schema}) placed at
$\theta = 7$\textdegree\ with respect to the beam line, where no \Be\ is allowed by kinematics. 
%The detection of three 
%characteristic elastic scattering peaks, two from p($^7$Li,$^7$Li) and one from $^{12}$C($^7$Li,$^7$Li), also allows to check for 
%the aging of the target: upon irradiation, the CH$_2$ target tends to lose hydrogen by evaporation, thus causing a depletion in the H/C ratio. 
%chamber can accommodate up to 12 targets. %Once the target-holder is loaded and the the scattering chamber is evacuated, the
%target can be replaced without the need to re-open the chamber. %The replacement of the entire 12-target holder, which
%requires the opening and the evacuation of the scattering chamber, was performed every 5 or 6 days: this procedure was performed
%three times in total, during the entire 14-day data taking period. 


