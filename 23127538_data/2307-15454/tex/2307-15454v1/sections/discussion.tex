The results of this work suggest that the charge recombination in \NR s in the energy 
range of interest for WIMP DM searches has a limited directional dependence.
A possible explanation is that the directional effect is washed out in 
the isotropic thermalization process of the electrons: the range of 
\SI{70}{\kilo\electronvolt} argon ions in LAr, 
\SI{0.18}{\micro\meter} \cite{srim}, is much shorter than the 
electron thermalization radius 
$r_t \sim $\SI{2.5}{\micro\meter}~\cite{Wojcik:2003ja,Wojcik:2016gy}. 
%v1
%After thermalization, electrons that diffuse and drift within  
%the Onsager radius $r_O$ from the ions will recombine. Electrons that
%diffuse outside the Onsager cylinder of radius $r_O$ around the
%ionization track can still recombine; however, since $r_t \gg r_O$,
%the recombination probability is driven by the shape of the electron cloud,
%which is nearly spherical and carries little memory of the track direction.
If all electrons were confined within the Onsager radius, the 
recombination probability $A$ would be $8/e^-$, namely, 
two orders of magnitude higher than measured in this work. This
%suggests that the extension of the electron cloud is much bigger
indicates that the extension of the thermalized electron cloud is much bigger
than the Onsager radius, thus weakening the initial directional
effect. Other non-local processes at the length scale of a few \textmu m 
can also contribute to the size of the electron cloud, including the emission of
Auger electrons and fluorescence X-rays from excited Ar atoms~\cite{estar,xcom}.
%end-of-v1

The strongest constraint on $\delta R$ from the fit comes from the position 
of the S2 peak, since $g_2\gg g_1$. In fact, the SCENE hint for directional 
sensitivity was primarily given by the 7\% variation in S1 for \NRs\ parallel and perpendicular 
to \edrift: no variation of S2 vs. direction was 
observed in SCENE. % for S2 at different directions of the \NR.
%The variation
%on S1 observed by SCENE between the parallel and perpendicular direction is
%about 7\% at 57.3~keV and \edrift=193~V/cm (Fig. 21 of Ref.~\cite{Cao:2015ks}).
While the SCENE data were never analyzed according to the directional
model of \cite{Cataudella:2017kcf}, an asymmetry $\delta R \approx 2$ would be required
to generate a 7\%-effect on S1. However, given the anti-correlations of Eqs.~\ref{eq:ne}
and \ref{eq:nph}, such a large $\delta R$ would produce a much more significant variation in S2
($\sim 80\%$ between parallel and perpendicular directions), which is not observed
in SCENE. The lack 
of a variation in the S2 signal, which is further confirmed in this work, 
rules out the directional modulation in charge recombination as the 
explanation of the effect and sets an upper limit on $\delta R$. 
Furthermore, the \ReD\ data, with an improved signal yield and resolution 
in S1, do not confirm the variation in S1 at different directions 
which was reported by SCENE. As for S2, no statistically-significant 
variation was found for S1.

The LAr signal model adopted in this work has two major upgrades
comparing to the models commonly used in the literature. 
The first modification is about charge recombination, by the introduction 
of the directional term of Eq.~\ref{eq:rec_direction} from 
Ref.~\cite{Cataudella:2017kcf}. The second modification is the use of an 
energy-dependent ratio \NexNi, which allows for 
a better fit of the \NR\ band shape and improves significantly the 
performance of the likelihood fit.
%, as displayed in Fig.~\ref{fig:NexNiFitCheck_NR}. 
If \NexNi\ is kept constant to the value 1, which is commonly-adopted for NRs~\cite{Doke:2002oab,Hitachi:2021hac},
the fit still returns a value of $\delta R$ compatible with zero, but the model fails to reproduce the shape 
of the S2 vs. S1 band and the S2 distribution for \NR s; furthermore, the best-fit for $g_2$ in this case 
is $29.9\pm0.1$~\si{PE/e^-}, which is in tension with the prior measurement of Table~\ref{tab:parameters}.
%
%% Figure environment removed
%
While the SCENE data also support the energy dependence of \NexNi, the physical motivation of it requires further 
study. One possibility is that this is the
apparent effect of energy dependences in the nuclear quenching, 
electron quenching and recombination processes, which are unaccounted by
the Lindhard, Mei and Thomas-Imel models used in this work. Specifically:
the Thomas-Fermi screening function used in the Lindhard model is known to
have a bias in the $O(10)\si{keV}$ range~\cite{Bezrukov:2010qa,Sigmund2004}; the Mei model simplifies the
average electronic stopping power by taking the value at the initial electron
kinetic energy; the charge recombination model does not consider the dependence
on the charge cloud size on the recoil energy. All these energy-dependent
factors are not accounted in the models and they could eventually show up
as an effective energy dependence in \NexNi.
%Nonetheless, the ultimate origin of the energy dependence in \NexNi\ is outside the scope
%of this work. 
It has anyhow a small effect on the systematic uncertainty on $\delta R$, 
due to the weak correlation reported in Tab.~\ref{tab:Fit_result}. 

%The small directional anisotropy $\delta R$ found in this work $-$ assuming 
%that it is true $-$ could be in principle exploited by a multi-tonne argon 
%dual-phase \TPC\ DM detector.
%Since the direction dependence arises before the electron drift, it is 
%expected to be preserved also in large detectors. The elastic interactions 
%of WIMPs of $O(100)$~GeV mass would generate smooth and broad spectra 
%in a LAr detector, so the most sensitive probe for the daily modulation 
%in the incident direction would be to look for shifts in S2.
%Yet, t
Taking the central value $\delta R=0.037$ from the fit, the amplitude of the S2 daily 
modulation curve, which results by folding the directional dependence 
S2$(\theta)$ of this work into the WIMP recoil direction distribution 
from Ref.~\cite{Cadeddu:2017ebu}, is only \SI{0.5}{\percent} peak-to-peak. 
Given the very large number of candidate events required to 
statistically identify it, this effect could hardly be of experimental 
use in a future $O(100)$~ton argon dual-phase \TPC\ DM detector. 

%V0
%The S2 daily modulation curve, which results from folding 
%the directional dependence S2$(\theta)$ of this work into the WIMP 
%recoil direction distribution from Ref.~\cite{Cadeddu:2017ebu}, is 
%displayed in Fig.~\ref{fig:DMdailymodulation}. The band displayed 
%in Fig.~\ref{fig:DMdailymodulation} represents the $\pm1\sigma$ uncertainty 
%derived from the observation of 150 total WIMP candidates, assuming 
%(optimistically) the resolution in S2 achieved with the \ReD\ \TPC\ in 
%this work.
%The amplitude of the modulation is only \SI{0.5}{\percent} 
%peak-to-peak. Therefore, many more candidate events would be required to 
%statistically identify the daily modulation.
%% Figure environment removed

