%Data analysis begins with event reconstruction. At each level of reconstruction, associated calibrations are applied. We start the description on \TPC\ events and mention the $\Delta E/E$ telescope and neutron spectrometer events in the end.

%Pulse finding
The raw data from the \TPC\ are the digitized waveforms of each of the \SiPM\ channels, from which the event type, time, and 3D 
position were reconstructed following the procedure described in~\cite{Agnes:2021zyq}. The waveforms of all \SiPM\ channels
were baseline-subtracted, equalized according to the individual gains and summed. The summed waveform was filtered
with a \SI{20}{\nano\second} moving average window and then scanned by a dedicated pulse-finder algorithm, to search
for possible S1 and S2 signals. Each pulse was classified as
either S1 or S2 by using the pulse shape parameter \fprompt, defined as the ratio of the charge in the first
\SI{700}{\nano\second} over the total charge: pulses with $f_\mathrm{p}<0.2$ are classified as S2. The total charge was
then normalized according to the Single Electron Response (SER) of each \SiPM\ channel, so to provide the S1 and S2
in units of PE. The pulse-finder algorithm is fully efficient for S1 signals above a few keV.  The time delay between the S1 and S2 pulses, i.e. the electron drift
time \tdrift, was used to estimate the $z$ coordinate of the interaction below the liquid-gas interface. 
Events with a single S1 pulse and a S2 pulse with $t_\mathrm{drift}$ between $\SI{6}{\micro\second}$ and 
the maximum drift time were kept for the subsequent analysis.
The cut $t_\mathrm{drift} > \SI{6}{\micro\second}$
removes the events produced just below the extraction grid of the \TPC, in which the S1 and S2 pulses are 
piled-up.
% is $Z_\mathrm{LAr}t_\mathrm{drift}/t_\mathrm{drift max}$.
The approximate \xy\ position of the event was evaluated as the charge-weighted center of the S2 signal 
in the top \SiPM\ array. The parameter \fprompt\ defined above was also used to 
perform the NR/ER discrimination: S1 pulses with $f_\mathrm{p} > 0.4$ are selected as from \NR. This simple
cut was shown to allow for a NR/ER separation better than $2~\sigma$ for S1 above 50~PE~\cite{Agnes:2021zyq}. 
%it further improves in the range $[120,400]\si{PE}$ which is used for the analysis of
%Sect.~\ref{sec:StatisticalAnalysis}. 

%SER and laser calibrations
The SER and the duplication factor $K_\mathrm{dup}$ were studied by irradiating the \SiPMs\ with a 403-nm laser source and by modeling the 
photon counting statistics according to the Vinogradov distribution~\cite{vinogradov2009}. 
The calibration was performed channel by channel, as described in \cite{Agnes:2021zyq}.
Typical values of $K_\mathrm{dup}$, which is the average number of secondary PEs produced by cross-talk and afterpulsing 
by each genuine primary photon on the \SiPM, are between 0.31 and 0.37. The PE gain is corrected to remove cross-talk and afterpulsing according to the ratio $\frac{1}{1+K_\mathrm{dup}}$. Dedicated laser calibrations were taken every
12~hours throughout the beam time to monitor the stability of the \SiPMs.

%Bias current correction
As mentioned in Sect.~\ref{sec:2a}, the voltage drop in the bias resistor chain causes a reduction 
of the bias voltage of the \SiPMs, which is proportional to the bias current and must be properly accounted 
for in the data analysis. The bias current registered during the 
laser calibrations by the slow control system was $< \SI{0.5}{\micro\ampere}$.  During the beam irradiation,
because of the much higher interaction 
rate and the much higher amount of light hitting the \SiPMs, the bias current ranged up to 
90 (150)~\textmu A for the bottom (top) \SiPMs, depending on the intensity of the 
primary $^7$Li beam, which was not constant in time. To derive the corrections to the SER for 
each individual \SiPM, three dedicated laser runs in which the \TPC\ was 
simultaneously irradiated with high-activity radioactive sources were performed. The typical correction is of the order of $0.5\% \cdot I$, where $I$ is the bias current 
in \textmu A. 
For this reason, the SER and $K_\mathrm{dup}$ correction was 
time-dependent and calculated using the closest reading of the bias current registered by
the slow control. 
Besides the SER and $K_{dup}$, the photon detection efficiency also changes with bias voltage. A set of runs 
with \ISO{Am}{241} \SI{60}{\kilo\electronvolt} $\gamma$ and the \ISO{Li}{7} beam irradiation was performed 
to calibrate the additional bias current dependency in PE yield after the SER 
and $K_\mathrm{dup}$ correction.

%Light yield
The performance of the \TPC\ was characterized prior to the irradiation, 
in a dedicated campaign~\cite{Agnes:2021zyq}. Specifically, the 
scintillation gain and ionization amplification of the \TPC\ were measured 
to be $g_1 = (0.194 \pm 0.013)$~PE/photon and 
$g_2 = (20.0 \pm 0.9)$~PE/electron, respectively.  
Additional calibrations with \Am\ were taken daily during the campaign 
at LNS. These measurements confirm a light yield of $(8.53 \pm 0.19)$~PE/keV 
at 60~keV and at \edrift=150~V; this is very well consistent with the expectation of
8.6~PE/keV based on the parametrization obtained in the pre-irradiation campaign~\cite{Agnes:2021zyq}.

%Calibrations with Am241
The daily calibration runs with \Am\ were used to evaluate the dependence of the \TPC\
response on the interaction position, and to determine the correction
factors for S1 and S2, to be later applied to the physics runs. The events featuring one single S1 and one single S2 and
having S1 compatible with the full energy deposition of the 60~keV $\gamma$-ray from \Am\ were grouped 
in a $22 \times 11$ mesh, according to the interaction position in the \TPC. The mesh has 22 entries in \xy, based on the
top SiPM channel detecting the largest fraction of the S2 signal, and 11 bins in $z$, equally spaced between
$t_\mathrm{drift}=\SI{6}{\micro\second}$ and $\SI{72}{\micro\second}$.
%\footnote{A range longer than $t_\mathrm{draftmax}$ is included to cover the S2 timing uncertainty.}.

%\begin{table*}[]
%    \centering
%    \label{tab:lifetime}
%    \caption{Electron lifetime in LAr over time. The beam data set is from run 1592 to run 1750.}
%    \begin{tabular}{c|c|c|c|c|c}
%        \hline
%        Run ID & $<=1432$ & $[1433,1525]$ & $[1526,1627]$ & $[1628,1641]$ & $>=1642$ \\
%        \hline
%        Electron lifetime $[\si{\us}]$ & 274 & 313 & 614 & 976 & 1095 \\ 
%        \hline
%    \end{tabular}
%\end{table*}

Firstly, S2 was corrected to account for the presence of impurities in LAr, which can cause the absorption of
electrons during their drift path. The electron life time was
estimated with an exponential fit of the S2 vs. \tdrift\ profile, restricted to the events in the  
central eight \xy\ bins. 
%Results are collected in Tab.~\ref{tab:lifetime}:
%Thanks to the continuous gas recirculation system through a SAES getter, the electron life time 
%improved from the $\sim 600$~\textmu s of the initial runs used for analysis up to the 
%level of 1~ms, i.e. significantly longer than the maximum drift time in the \TPC, within four
%days of data taking. 
The $z$ dependency of S1 and S2 was further corrected by using a set of 5th-order polynomials
S1$_{i}(t_\mathrm{drift})$ and S2$_{i}(t_\mathrm{drift})$:
they are calculated by 
%by taking the mean signals $S1_{ij}$ and $S2_{ij}$ in each voxel of the 3D mesh and
interpolating over the $z$-points within each bin $i$ in \xy. Three examples are shown in
Fig.~\ref{fig:S1S2CorrExample}: the correction vs. $z$ is within 10-15\%, for both S1 and S2.
%A similar
%interpolation was not applied in \xy, due to the limited \xy\ resolution for small-S2 \NR\ events in
%our energy region of interest.
No significant variation in the position correction was found throughout the sequence calibration
runs. Position dependencies mostly result from non-uniformity in the light collection efficiency within
the \TPC: %\footnote{Contributions due to the non-uniformity of the \edrift\ are also possible, but they were evaluated to
%be a second-order effect.}
as a consequence, the same corrections for S1 and S2 derived from \Am\ (ER) events were also applied to \NR\ events.

% Figure environment removed

A simpler processing was performed for the digitized waveforms from the liquid scintillators and from the Si detectors
of the telescope. The signal in the LSci detectors was processed by calculating the total charge, integrated within
a gate of 600~ns. The ratio between the charge in the first 80~ns and the total was used as
the discrimination parameter, resulting in a neutron-$\gamma$ discrimination better than $3 \sigma$ above 
200~keV$_{ee}$~\cite{simophdthesis}. The signals from the $E$ and $\Delta E$ detectors of the telescope were
evaluated by taking the maximum of the digitized shaped waveforms from the charge-sensitive amplifier. 

%timing resolution.
The time signal of all three kinds of detectors in the setup is critical for the coincidence event selection. The time
stamp of a \TPC\ event was defined as the zero-crossing time of the pulse obtained by passing the S1 pulse through a constant 
differential filter (CDF). 
The $\Delta E/E$ telescope generates two time stamps, one for the $\Delta E$ detector and one for the $E$ detector, which were
both evaluated with CDFs. The reference time for the $\Delta E/E$ telescope used for the coincidence was taken as 
the average of the two time stamps. Finally, the time stamp for the neutron spectrometer was defined as the zero-crossing CDF
time of the digitized waveforms. %Due to non-linearity of the PMT readout, the timing from the neutron spectrometer showed a
%slight energy dependence, which was calibrated using the time-of-flight of $\gamma$s from a \ISO{Na}{22} source and
%corrected.
