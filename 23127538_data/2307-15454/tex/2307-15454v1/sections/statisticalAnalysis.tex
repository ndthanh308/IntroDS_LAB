The S2 vs. S1 distribution of the \NR\ events in the \TPC\ which pass the 
selection procedure of  Sect.~\ref{sec:EventSelection} is displayed in Figure~\ref{fig:LAr:ReD_Cut_Coinci}:  
the pink dots represent the events selected requiring the triple coincidence (\TPC, Si telescope and neutron 
spectrometer); 
the colour-coded distribution includes the events in double coincidence (\TPC\ and telescope). 
The triple coincidence sample contains about 650 \NR\ events with S1 above \SI{20}{PE}, which were collected 
during 10.7~live days of beam run. The double coincidence events constitute a large sample of about 
70000 \TPC\ \NR\ events in all directions: they were hence
used as a calibration data set to constrain the nuisance model parameters in the global fit below. Since the triple coincidence events are a 
small fraction of the double coincidences, the large sample of double coincidence events was also used 
as the template for random coincidence background. 

%In fact, the rate and the S1 distribution of background 
%events were evaluated either from data, by using the side bands in the time coincidence, or from 
%\texttt{Geant4}-based~\cite{Agostinelli:2003fg,Allison:2006cd,Allison:2016lfl} Monte Carlo simulations.


% Figure environment removed


The data samples were statistically
analysed in order to evaluate the best estimate of the directionality
parameter $\delta R = R-1$, which measures how much the shape of the
initial ionization charge cloud differs from a sphere. As the number
of events is relatively modest, an unbinned profile likelihood was applied. The
global likelihood $\mathcal{L}$ is written as a product of three likelihood terms:
\begin{eqnarray}\label{eq:LAr:ReD_likelihood_total}
    \mathcal{L}(\vect{X}\,|\,\delta R, \vect{\nu}) & = & \prod_{i=1}^{5} \mathcal{L}_{i}(\vect{X}_i\,|\,\delta R,\theta_{r}^{(i)},\vect{\nu}) \nonumber\\
    & &  \times \mathcal{L}_\mathrm{cali}(\vect{X}_\mathrm{cali}\,|\,\vect{\nu})
    \times \mathcal{L}_\mathrm{constraint}(\vect{\nu}),
\end{eqnarray}
%It is a product of three terms, each of which is a function of the observed 
%array $\vect{X}$ of events $X = (\mathrm{S1},\mathrm{S2})$. The product over $i$ refers to the five
%samples taken at the five angles $\theta_r^{(i)} = {0^\circ, 20^\circ, 40^\circ,
%90^\circ l, 90^\circ r}$ of Tab.~\ref{tab:LSci_Timing}, each containing the
%events $\vect{X}_i$; $\delta R$ is the parameter of interest (POI) and $\vect{\nu}$ the array of nuisance %parameters, listed in Tab. \ref{tab:parameters} and discussed below.
where the product over $i$ refers to the five
samples taken at the five angles $\theta_r^{(i)} = {0^\circ, 20^\circ, 40^\circ, 90^\circ l, 90^\circ r}$ of Tab.~\ref{tab:LSci_Timing}, each containing the observed
array of events $\vect{X}_i = (\mathrm{S1},\mathrm{S2})$; $\delta R$ is the parameter of interest (POI); $\vect{\nu}$ is the array of nuisance parameters; $\vect{X}_\mathrm{cali}$ is the array of calibration data set.
The POI is constrained in this work to $\delta R \ge 0$, as negative values of $\delta R$ are not
physically allowed by the recombination model~\cite{Cataudella:2017kcf}. 
% Notice: if there is only thermal diffusion, the initial cloud shape 
% will be spherical, delta_R=0. If the primary Ar ion track has a finite 
% size, it contributes to a positive delta_R. It is not physical to have a 
% track with a negative length.
The three  likelihood terms of Eq.~\ref{eq:LAr:ReD_likelihood_total} are described in detail below.

$\mathcal{L}_i$ is the extended likelihood of each sample of \NR\ events at the recoil angle $\theta_{r}^{(i)}$:
\begin{equation}
    \mathcal{L}_i = \mathrm{Poisson}(n_i|\hat{n_i})\prod_{X_j\in\vect{X}_i} \mathcal{P}_i(\mathrm{S1}_j,\mathrm{S2}_j; \delta R, \theta_{r}^{(i)}, \vect{\nu}) 
\end{equation}
where $n_i$ and $\hat{n_i}$ are the size of $\vect{X}_i$ and its mean, respectively, and $\mathcal{P}_i$ is the 
joint probability density function (PDF) of the events (S1,S2). 
%The leading term extends the likelihood, $n_i$ is the size of $\vect{X}_i$ and $\hat{n_i}$ is its mean. 
The PDF is made as the combination of three components, one for
signal and two from backgrounds:
\begin{eqnarray} \label{eq:eachpdf}
    \mathcal{P}_i(S1,S2)&=& (1-\lambda_{1i})(1-\lambda_2)F_\mathrm{sig}(E_r) \nonumber\\
    & &  \otimes P(\mathrm{S1},\mathrm{S2}; \delta R, \theta_{r}^{(i)}, \vect{\nu}, E_r) \nonumber\\             
    & & + [\lambda_{1i} F_\mathrm{bkg1}(E_r) + (1-\lambda_{1i})\lambda_2 F_\mathrm{bkg2}(E_r) ]\nonumber \\
    & & \otimes P(\mathrm{S1},\mathrm{S2}; \delta R, \bar{\theta}_r, \vect{\nu}, E_r).
\end{eqnarray}
%ER background are not included explicitly with eq.~\ref{eq:LAr:S1S2PDF_ER}, as the ER fraction is small and at low energies the ER energy response approaches NR. 
%Note that in this study, $f_l$ takes the form of the Lindhard model (eq.~\ref{eq:intro:f_n_Lindhard}), and $f_l$ takes the form of Birks (eq.~\ref{eq:intro:f_l}) with $k_e$ estimated from SCENE \cite{Cao:2014gns}. 
The first component is the energy spectrum for the signal $F_\mathrm{sig}(E_r)$, which
depends on the recoil energy $E_r$, convolved with the response function $P$ of the
\TPC\ to mono-energetic \NR\ events, as defined in Eq.~\ref{eq:TPCResponse}. The parameters $\lambda_{1i}$ are the fractions of random coincidences within 
each data sample: they were estimated from the data, using the counting rate
in the side-band in ToF and are listed in Tab.~\ref{tab:LAr:lambda_1}.
Similarly, $\lambda_2$ is the scaling factor for multi-scattering background, namely
the fraction of those events with respect to all \NR\ events in the coincidence
window.  The other
two components are the energy distributions of the backgrounds due to random
coincidences, $F_\mathrm{bkg1}(E_r)$, and to multiple neutron scattering, 
$F_\mathrm{bkg2}(E_r)$. They are also convolved with the response function $P$ of
the \TPC. 

In order to speed-up the computation of the response function $P$, the
Poisson and binomial distributions are approximated by Gaussian distributions, such
that the convolutions over $N_\mathrm{ph}$ and $N_{e^-}$ in Eq.~\ref{eq:TPCResponse} can
be evaluated analytically.
As the angular distribution for background events is approximately random, the
$\theta_r$ dependence of $f(\theta_r, R)$ is averaged out by using the equivalent
angle $\bar{\theta}_r$ calculated analytically for an isotropic distribution and
the functional dependence on the angle is approximated as
$\langle f(\theta_r, R)\rangle \sim f(\bar{\theta}_r, R)$.
%The only effect of this approximation is a
%global scaling factor: the systematic uncertainty induced on the parameter of interest $R$ is
%estimated to be well below 1\%. 
%In this case, the functional dependence on the angle is approximated
%as $\langle f(\theta_r, R)\rangle \sim f(\bar{\theta_r}, R)$. 
The factor $\lambda_2$ and the three energy spectra ($F_\mathrm{sig}$,
$F_\mathrm{bkg1}$, and $F_\mathrm{bkg2}$) were evaluated by means of a dedicated 
Monte Carlo simulation using the \texttt{Geant4}-based framework \texttt{g4ds}~\cite{Agostinelli:2003fg,Allison:2006cd,Allison:2016lfl,Agnes:2017grb}.
The events from the simulations
underwent the same sequence of selection cuts used for the
real data. The energy distributions derived by the Monte Carlo are displayed
in Fig.~\ref{fig:simspectra}. The three energy distributions were then analytically
parametrized in order to optimize the calculation of the CPU-intensive
PDF $\mathcal{P}_i$. $F_\mathrm{sig}$ consists of two Gaussian peaks corresponding
to the \NR\ induced by neutrons from p($^{7}$Li,$^{7}$Be)n and 
p($^{7}$Li,$^{7}$Be*)n'.
$F_\mathrm{bkg1}$ and $F_\mathrm{bkg2}$ were approximated by a double-exponential and a
single exponential, respectively, whose parameters were calculated by fits to
the Monte Carlo distributions.
%
%Mapping: LSci\_1 = 90l, LSci\_8 = 90r, LSci\_4-6= 20 deg, LSci\_3-7=40 deg
%LSci\_5 = 0 deg

\begin{table}
    \centering
    \caption{Fraction of random coincidence events, $\lambda_{1i}$, in the range
    S1$\,\in[120,400]\si{PE}$ and S2$\,\in[800,2800]\si{PE}$ in the five samples of
    triple-coincidence events at different $\theta_r$. Uncertainty is about 2\% for all 
    samples.}
    \label{tab:LAr:lambda_1}
    \begin{tabular}{c|c|c|c|c}
    \toprule
    \thead{0\textdegree} & \thead{20\textdegree} & \thead{40\textdegree} &  \thead{90\textdegree $l$} & \thead{90\textdegree $r$} \\
    \hline
    0.045 & 0.048 & 0.047 & 0.026 & 0.041 \\
%         \thead{90\textdegree $l$} & \thead{40\textdegree} & \thead{20\textdegree} & \thead{0\textdegree} & \thead{90\textdegree $r$} \\
%        \hline
%         0.026 & 0.047 & 0.048 & 0.045 & 0.041  \\
        \bottomrule
    \end{tabular}
\end{table}

% Figure environment removed


The factor $\mathcal{L}_\mathrm{cali}$ of the global likelihood of Eq.~\ref{eq:LAr:ReD_likelihood_total} is the constraint
term on the nuisance parameters and it depends on the events $\vect{X}_\mathrm{cali}$ in the calibration set (i.e. colour-coded histogram in Fig.~\ref{fig:LAr:ReD_Cut_Coinci}) 
%Calibration
%data consists of the large \NR\ sample of events in double coincidence between the \TPC\ and the telescope, displayed
%in Fig.~\ref{fig:LAr:ReD_Cut_Coinci} as a color-coded histogram. 
While the energy spectrum of the calibration events is 
a broad and featureless distribution, the joint distribution of the \NR\ band in the (S1,S2) plane can set a
strong constraint on the nuisance parameters. Since the fraction of signal events in the calibration sample is negligible,
the energy distribution is well approximated by the random background $F_\mathrm{bkg1}$. The calibration term is hence
written as:
\begin{equation}
\label{eq:L_cal}
    \mathcal{L}_\mathrm{cali} = \prod_{X_j\in\vect{X}_\mathrm{cali}} P(\mathrm{S1}_j,\mathrm{S2}_j; \delta R, \bar{\theta}_r, \vect{\nu}, E_r)\otimes F_\mathrm{bkg1}(E_r).
\end{equation}

In order to avoid any analysis bias, $\delta R$ should be decoupled from the nuisance parameters as much as possible. The
explicit occurrence of the POI $\delta R$ in Eq.~\ref{eq:L_cal} is due to the fact that the parameter $\xi_m$ in
the modified Thomas-Imel model in Eq.~\ref{eq:rec_direction} is dependent on $\delta R$ because of the track length. To
remove such undesirable degeneracy, the angular dependence term and the Thomas-Imel parameter were re-defined as 
\begin{equation}
f'(\theta_r,R) = f(\theta_r,R)/f(\bar{\theta}_r,R)
\end{equation}
and
\begin{equation}
\xi_m' = \xi_m/f(\bar{\theta}_r,R),
\end{equation}
respectively. In this way the angle-averaged position of the \NR\ band in calibration data does not depend on
$\delta R$ and the POI $\delta R$ is left as a pure representation of directionality. Furthermore, the degenerate
nuisance parameters were re-cast into a unique nuisance parameter $A=\xi_m'/(\mathcal{E}_d \cdot \langle N_\mathrm{i} \rangle)$, which represents the
recombination probability of one electron-ion pair. 

The last factor of the global likelihood,
$\mathcal{L}_\mathrm{contraint}(\vect{\nu})$, is the pull term for the nuisance parameters which were known 
by prior independent measurements. Those parameters are constrained by Gaussian terms
\begin{equation}
    \mathcal{L}_\mathrm{constraint}(\vect{\nu}) = \prod_i\frac{1}{\sqrt{2\pi}\sigma_{\nu_i}}\exp{-\frac{(\nu_i-\nu_i^0)^2}{2\sigma_{\nu_i}^2}}
\end{equation}
based on the previously-measured values $\nu_i^0$ of the parameters $\nu_i$ and on their corresponding uncertainties.

%The assigned errors in the prior term are large relative to $\mathcal{L}_\mathrm{cali}$ such that this term only rectifies the fitting. 
%The column ``Reference" in tab.~\ref{tab:LAr:parameters} shows the values of nuisance parameters as $\nu_i^0\pm\sigma_{\nu_i}$. Additional fixed parameters are listed with only the estimated value. 

\begin{table}
    \centering
    \caption[Parameter values in the signal model in ReD directionality analysis.]{List of the parameters used in the model. $\delta R$ is the parameter of interest, while all others are nuisance parameters, constrained by the calibration data and/or by a Gaussian pull term. The error bars are the standard deviation which is taken in the Gaussian pull terms. The parameters reported without uncertainties are fixed. The gains $g_1$ and $g_2$ come from the previous \TPC\ performance study~\cite{Agnes:2021zyq}. The S1 resolution of the \TPC\ of Eq.~\ref{eq:TPCResponse} is parametrized as $\sigma_{\mathrm{S1}}^2 = \mathrm{S1}/[\si{PE}] + {\sigma^*_{\mathrm{S1}}}^{2} $, namely by the combination of the statistical term and of an extra contribution. The same is done for the S2 resolution.}
    \label{tab:parameters}
    \begin{tabular}{c|c|c}
    \toprule
         & \thead{Constraint}   & \thead{Comment} \\
        \hline
        $\delta R$  & -        & Parameter of interest \\
        \hline
       % $\mu_-$     &  \SI{500(10)}{cm^2/V/s}  & -   & Electron mobility at \SI{150}{V/m}\\
       % $\alpha$    &                   & -   & Recombination coefficient\\
       % $\sigma(E_r)$    &                   & -    & Ionization site size   \\
        %$A$    &   \SI{3.95(200)e-2}{1/e^-}      & No    & $A = \alpha/(2\pi E\sigma^2\mu_- f(\langle\phi\rangle,R))$\\
        $A$    &   \makecell{$0.04\pm0.01$\\ $[\si{1/e^-}]$ }    
%        & $A = \alpha/[2\pi\mathcal{E}_d\sigma^2 \mu_- f(\bar{\theta}_r,R)]$\\
        & $A = e/[2\pi\epsilon_r\epsilon_0\mathcal{E}_d\sigma^2 \mu_- f(\bar{\theta}_r,R)]$\\
        \hline
               $k_e$       & $2.8$          
        & Electronic quenching coefficient~\cite{Mei:2008ca}\\
\hline
$W_{ph}$         & \makecell{$19.5$\\ $[\si{eV}]$}       & \makecell{Energy  for scintillation \\photon production~\cite{Doke:2002oab}}\\
\hline
        $N_\mathrm{ex}/N_\mathrm{i}$ & $0.2\sim2$  
        & \makecell{Excitation to ionization ratio.\\ Energy dependence as in \cite{Cao:2015ks}} \\
        \hline
        $g_1$       & \makecell{$0.196\pm0.020$\\ $[\si{PE/ph}]$}      & S1 signal yield \\
        \hline
        $g_2$       & \makecell{$20.5\pm2.5$\\ $[\si{PE/e^-}]$}       & S2 signal yield \\
        \hline
        $\sigma^*_{\mathrm{S1}}/\mathrm{S1}$ &  $0.003\pm0.05$    
        & \makecell{S1 detector resolution\\ in addition to $\sqrt{\mathrm{S1}}$} \\
        \hline
        $\sigma^*_{\mathrm{S2}}/\mathrm{S2}$ &  $0.001\pm0.05$    
        & \makecell{S2 detector resolution\\ in addition to $\sqrt{\mathrm{S2}}$} \\
        \hline
        $\lambda_1$ &  Table~\ref{tab:LAr:lambda_1}      & Fraction of random coincidence \\
        \hline
        $\lambda_2$ &   0.16        & \makecell{Ratio of multi-scattering to \\all NR in coincidence windows}\\
        \bottomrule
    \end{tabular}
\end{table}


As a summary, the parameters and their reference values are summarized in Tab.~\ref{tab:parameters}.
%$A$ is estimated by manually tuning its value to align the NR band to data, an arbitrarily large uncertainty is assigned so that the prior does not restrict the fitting result of $A$. 
The recombination probability $A$ depends on $\sigma$, the size of the ionization cluster of 
Eq.~\ref{eq:clouddist}, which is dominated by
the electron diffusion during thermalization. Due to their high mobility and long thermalization time,
electrons diffuse for a few \textmu m in LAr \cite{Wojcik:2003ja,mozumder1995free}. It is found
that $A=0.04/e^-$, which corresponds to $\sigma=\SI{1.8}{\micro\meter}$, was an appropriate initialization parameter
for the likelihood fit. The ratio $N_\mathrm{ex}/N_\mathrm{i}$ was treated as a function of recoil energy, according to
the indications by SCENE~\cite{Cao:2015ks}. The \TPC\ gains $g_1$ and $g_2$ were estimated according to the \TPC\
characterization in~\cite{Agnes:2021zyq}, and were treated as nuisance parameters in order to accommodate for
possible variations in the \TPC\ performance. The parameters $W_{ph}$, $k_e$, $\lambda_1$ and $\lambda_2$ were fixed
in order to limit the degeneracies in the fit: their effect on the POI is minor and is accounted
below as a systematic uncertainty. %Similarly to \NexNi, $k_e$ is a parameter which affects the nuclear recoil
%band shape, but not directly the directionality. 

%
%\subsection{Simulation and the energy spectra of signal and backgrounds}
%\label{ch:LAr:ReD_simulation}
%$F_\mathrm{sig}(E_r)$, $F_\mathrm{bkg1}(E_r)$, and $F_\mathrm{bkg2}(E_r)$ should be determined from the simulation result. Three configurations are studied in the simulations, the reaction to $\mathrm{Be}$ ground state (gs) with a \SI{3.5}{mm} diameter collimator before the target, the reaction to $\mathrm{Be}$ ground state without the collimator, and the reaction to the first excited state $\mathrm{Be}^*$ without the collimator. They are shown accordingly from left to right in fig.~\ref{fig:LAr:ReD_Sim_Direc}. The top row shows the energy versus direction distribution of the recoiled argon nuclei of triple coincidence events. The blobs around \SI{70}{keV} are single-scatter events, while the broad distribution over all angles consists of multi-scatter events that also generate triple coincidences by chance. The bottom row shows the timing spectra of the single-scatter and multi-scatter events. Similar to the data, the $\Delta T(\mathrm{LSci}-\mathrm{SiTel})$ is offset such that the $\gamma$ peak in the $\mathrm{Be}$ configuration centers at $t=0$. The LSci coincidence windows are slightly different in each channel, roughly speaking, they are \SI{-3}{ns} to \SI{9}{ns} around the peaks at \SI{20}{ns}. 
%The relative rates of single-scatter (black) to multi-scatter (red) events in the coincidence windows of the three configurations are the same. Multi-scatter events make up for \SI{39}{\percent} of the triple coincidence events. Note that an additional simulation of detector responses and $S1$ range selection is required to determine $\lambda_2$. 
%
%%% Figure environment removed
%%
%%% Figure environment removed
%%
%%% Figure environment removed
%
%Fig.~\ref{fig:LAr:ReD_Sim_Spec} shows the energy spectra from the simulation. The peaks consist of single-scatter events. The solid black line and dotted black line correspond to single-scatter events from neutrons associated with $\mathrm{Be}$ and $\mathrm{Be}^*$, respectively. The two spectra are fitted with two Gaussian distributions. The ratio of $\mathrm{Be}^*$ events to $\mathrm{Be}$ is determined from the timing spectrum of events in LSci\_0 at $4.2^\circ$ without the TPC coincidence requirement. As shown in fig.~\ref{fig:LAr:ReD_LSci_0_BeRatio}, the fraction of $\mathrm{Be}^*$ events is \SI{19.6}{\percent}. Approximately, the interaction probabilities of neutrons at the two energies in the TPC are the same, and the TPC detection efficiency to the two types of events are the same. Then we have the estimation of the signal energy spectrum 
%\begin{eqnarray}
%    F_\mathrm{sig}(E_r) &=& m \frac{1}{\sqrt{2\pi}\sigma_{E1}}\exp\left(-\frac{(E_r-E_1)^2}{2\sigma_{E1}^2}\right)\nonumber \\
%    & & + (1-m)\frac{1}{\sqrt{2\pi}\sigma_{E2}}\exp\left(-\frac{(E_r-E_2)^2}{2\sigma_{E2}^2}\right)
%\end{eqnarray}
%where $m=0.196$, $E_1=\SI{63.5}{keV}$, $\sigma_{E1}=\SI{6.8}{keV}$, $E_2=\SI{72.5}{keV}$ and $\sigma_{E2}=\SI{7.8}{keV}$.
%The backgrounds sitting at the bottom of fig.~\ref{fig:LAr:ReD_Sim_Spec} correspond to $F_\mathrm{bkg1}(E_r)$ and $F_\mathrm{bkg2}(E_r)$. The solid red line consists of events with TPC-SiTel coincidence. It is the same distribution as the random coincidence background, $F_\mathrm{bkg1}(E_r)$. A two-components exponential could describe the shape,
%\begin{equation}
%    F_\mathrm{bkg1}(E_r) = l\frac{1}{\tau_1}\exp(-E_r/\tau_1) + (1-l)\frac{1}{\tau_2}\exp(-E_r/\tau_2)
%\end{equation}
%where $\tau_1=\SI{25.0}{keV}$, $\tau_2=\SI{114}{keV}$, and the fraction $l=0.17$. 
%%check the difference from the 2 sim settings, check the sideband shape vs double coincidenc shape.
%The dashed red line consists of multi-scatter events inside the triple coincidence window. They are selected according to the top row of fig.~\ref{fig:LAr:ReD_Sim_Direc}. The shape is described with an exponential
%\begin{equation}
%     F_\mathrm{bkg2}(E_r) = \frac{1}{\tau_3}\exp(-E_r/\tau_3) 
%\end{equation}
%where $\tau_3=\SI{42.7}{keV}$. The multi-scatter background spectrum in terms of $S1$ can be estimated by convolving the point response function in eq.~\ref{eq:LAr:S1S2PDF_NR} with $F_\mathrm{bkg2}(E_r)$. The fraction of $S1\in[100,350]\si{PE}$ is \SI{36}{\percent}. Therefore, we have $\lambda_2=0.39\times0.36/(1-0.39+0.39\times0.36)=0.19$. Now, all the ingredients are ready, and we shall perform the likelihood fittings. 
%

%\subsection{Likelihood fit and limit calculation}
%\label{ch:LAr:ReD_limit}

Finally, experimental data of Fig.~\ref{fig:LAr:ReD_Cut_Coinci} (calibration 
and five triple-coincidence samples) were fitted against the model of
Eq.~\ref{eq:LAr:ReD_likelihood_total}. In order to make the fit stable the 
fit region in the (S1,S2) plane was selected in order to include only the 
\NR\ band, with S1$\,\in[120,400]\si{PE}$, as represented by the white contour in 
Fig.~\ref{fig:LAr:ReD_Cut_Coinci}. The S1 range corresponds to NR energies between
approximately 35 and 150~keV, and hence comfortably includes the expected NR signal at
$\sim 72$~keV. The low-S1 edge $S1 > 120~\si{PE}$ was set in order to avoid any inefficiencies in
the event reconstruction and selection. The center of the \NR\ band
was empirically parametrized with the function S2$\,/[\si{PE}] = 455\ln(\mathrm{S1}/[\si{PE}])-535$
and the cut was set as $\pm500~\si{PE}$ in S2. The fit region globally contains 
529 triple coincidence and 42340 calibration events.

%
% Figure environment removed

\begin{table}
    \centering
    \caption{Best-fit of the parameters and correlation coefficients between the nuisance parameters 
and the POI $\delta R$.}
    \label{tab:Fit_result}
    \begin{tabular}{c|c|c}
    \toprule
 \thead{Parameter} & \thead{Value}  & \thead{Correlation with $\delta R$} \\
 \hline
 $\delta R$ & $0.037\pm0.027$ & -  \\
 $A\,[\si{1/e^-}]$ & $(4.01\pm0.06)\times10^{-2}$ &-0.014  \\
 $g_1\,[\si{PE/ph}]$ & $0.204\pm0.002$ & 0.013  \\
 $g_2\,[\si{PE/e^-}]$& $20.1\pm0.2$ & -0.009  \\
 $\sigma^*_{\mathrm{S1}}$/S1  & $0.017\pm0.003$ & -0.012  \\
 $\sigma^*_{\mathrm{S2}}$/S2  & $0.0002\pm0.0060$ & 0.026  \\
 \bottomrule
    \end{tabular}
\end{table}


%As the limit is a single sided upper limit, the appropriate likelihood ratio $\lambda$ shall be defined as~\cite{Cowan2011}
%\begin{equation}
%    \lambda(\delta R) = 
%        \begin{cases}
%            \frac{\mathcal{L}(\delta R, \hat{\hat{\vect{\nu}}}(\delta R))} 
%            {\mathcal{L}(\hat{\delta R}, \hat{\vect{\nu}})} & 
%            \hat{\delta R} \geq 0, \\
%            \frac{\mathcal{L}(\delta R, \hat{\hat{\vect{\nu}}})} {\mathcal{L}(0,\hat{\hat{\vect{\nu}}}(0))} & 
%            \hat{\delta R} < 0
%        \end{cases}
%\end{equation}
%and the test statistic $q$ is
%\begin{equation}
%    q (\delta R) = 
%    \begin{cases}
%        -2\ln \frac{\mathcal{L}(\delta R, \hat{\hat{\vect{\nu}}}(\delta R))} 
%            {\mathcal{L}(0, \hat{\hat{\vect{\nu}}})} &
%            \hat{\delta R} < 0, \\     
%        -2\ln \frac{\mathcal{L}(\delta R, \hat{\hat{\vect{\nu}}}(\delta R))} 
%            {\mathcal{L}(\hat{\delta R}, \hat{\vect{\nu}})} &
%            0 \leq \hat{\delta R} \leq \delta R, \\
%        0 & \delta R < \hat{\delta R}
%    \end{cases}
%\end{equation}
%where the $\hat{\hat{\vect{\nu}}}(\delta R)$ is the estimation of nuisance parameters with the POI is fixed to $\delta R$. Toy MCs are generated to estimate the distribution of $q$, $f(q(\delta R)|H)$, under the null hypothesis $H_{s+b} : \delta R = \delta R_\mathrm{test} > 0$ and the alternate hypothesis $H_b : \delta R =0$. The p-value is defined as $CL_s = CL_{s+b} / CL_b$, where $CL_{s+b}$ is the integral of $f(q(\delta R)|H_{s+b})$ from the observed $q$ from data to infinite and $CL_{b}$ is the integral of $f(q(\delta R)|H_{b})$ from the observed $q$ from data to infinite. The $\delta R$ scan for p-value is done with the \texttt{StandardHypoTestInvDemo.h} script from \texttt{RooStats}. Results are shown in fig.~\ref{fig:LAr:ReD_cdf_p}. The test statistic is sampled with toy MC 500 times for $H_{s+b}$ and 250 times for $H_{b}$. The accumulative distributions to the right side are shown. Solid lines indicate the distribution of the null hypothesis, dotted lines indicate the alternate hypothesis. Solid and open circles indicate the $CL_{s+b}$ and $CL_b$ of the data. A limit of \SI{90}{\percent} confidence level can be drawn at $\delta R = 0.058$. 
%
%% Figure environment removed


The fit result is shown in Fig.~\ref{fig:ReD_Fit_result} and reported in Tab.~\ref{tab:Fit_result}.
The positions of the signal peak in both S1 and S2 (middle and bottom rows of Fig.~\ref{fig:ReD_Fit_result}) are mutually 
consistent among the five samples at different $\theta_r$. The best-fit for the POI is 
$\delta R = 0.037\pm0.027$, which is less than $2\sigma$ away from a null result; 
the uncertainty on $\delta R$ is largely driven by statistics. 
%, as indicated by the weak correlation with the other nuisance parameters. 
The upper limit of 
$\delta R$ is calculated by a toy Monte Carlo approach, in order to guarantee the correct coverage: it results to be
$\delta R < 0.072$ at 90\% CL. 
The best-fits of the nuisance parameters are in good agreement with the central values of their 
estimates used for the constraints.
%indicating that the \NR\ response model is able to provide a good description of the data. 
In particular, the smallness of the best-fit for the parameters 
$\sigma^*_{\mathrm{S1}}$/S1 and $\sigma^*_{\mathrm{S2}}$/S2, which are the extra (non-statistical) contributions to the 
experimental resolution in S1 and S2, demonstrates that the spatial inhomogeneities of the detector response 
were properly corrected. Furthermore, the proper convergence and the absence of a significant bias for all
fit parameters, notably including the POI $\delta R$, were checked by running a dedicated set of toy Monte Carlo
simulations.

The uncertainties on $\delta R$ related to the nuisance parameters are automatically accounted in 
the fit. All other systematic uncertainties on $\delta R$, e.g. those related to the values of $W_{ph}$, $k_e$,
$\lambda_1$ and $\lambda_2$, to the spectral shapes $F_\mathrm{sig}$, $F_\mathrm{bkg1}$ and $F_\mathrm{bkg2}$, and to
the approximation of  $\bar{\theta}_r$ from isotropic distribution, are globally evaluated to be an order
of magnitude smaller than the statistical term and are hence neglected in this work. 



