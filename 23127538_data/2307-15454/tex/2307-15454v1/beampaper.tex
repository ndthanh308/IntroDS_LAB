%%%%%%%%%%%%%%%%%%%%%%% file template.tex %%%%%%%%%%%%%%%%%%%%%%%%%
%
% This is a  template file for the LaTeX package SVJour3 width change file svepjc3.clo
% for Springer journal:
% The European Physical Journal C
%
% Copy it to a new file with a new name and use it as the basis
% for your article. Delete % signs as needed.
%
% This template includes a few options for different layouts and
% content for various journals. Please consult a previous issue of
% your journal as needed.
%
%%%%%%%%%%%%%%%%%%%%%%%%%%%%%%%%%%%%%%%%%%%%%%%%%%%%%%%%%%%%%%%%%%%
%
% First comes an example EPS file -- just ignore it and
% proceed on the \documentclass line
% your LaTeX will extract the file if required
\begin{filecontents*}{example.eps}
%!PS-Adobe-3.0 EPSF-3.0
%%BoundingBox: 19 19 221 221
%%CreationDate: Mon Sep 29 1997
%%Creator: programmed by hand (JK)
%%EndComments
gsave
newpath
  20 20 moveto
  20 220 lineto
  220 220 lineto
  220 20 lineto
closepath
2 setlinewidth
gsave
  .4 setgray fill
grestore
stroke
grestore
\end{filecontents*}
% 
\RequirePackage{fix-cm}
%
\documentclass[twocolumn,epjc3]{svjour3}  
%
\RequirePackage[T1]{fontenc}
\smartqed  % flush right qed marks, e.g. at end of proof
%
\usepackage[utf8]{inputenc}
\RequirePackage{graphicx}
\RequirePackage{mathptmx}      % use Times fonts if available on your TeX system
\RequirePackage{flushend}
%\RequirePackage[numbers,sort&compress]{natbib}
\RequirePackage[colorlinks,citecolor=blue,urlcolor=blue,linkcolor=blue]{hyperref}
%\usepackage[leftcaption]{sidecap}
%
%
% insert here the call for the packages your document requires
\input{my_ds_macro.tex}
\usepackage{siunitx}
\usepackage{lineno}
\usepackage{textcomp}
\usepackage[dvipsnames]{xcolor}
\usepackage{booktabs}
\usepackage{makecell}
\usepackage{amsmath}
\usepackage{amssymb}
%
% please place your own definitions here and don't use \def but
%\newcommand{\LAr}{\mbox{LAr}}
\renewcommand{\NR}{\mbox{NR}}
\renewcommand{\ER}{\mbox{ER}}
\renewcommand{\DS}{\mbox{DarkSide}}
\renewcommand{\SCENE}{\mbox{SCENE}}
\renewcommand{\ReD}{\mbox{ReD}}
\renewcommand{\LArTPC}{\mbox{LAr~TPC}}
\renewcommand{\TPC}{\mbox{TPC}}
\renewcommand{\DSf}{\mbox{DarkSide-50}}
\renewcommand{\DSp}{\mbox{DarkSide-Proto}}
\renewcommand{\DSk}{\mbox{DarkSide-20k}}
\renewcommand{\SiPM}{\mbox{SiPM}}
\renewcommand{\SiPMs}{\mbox{SiPMs}}
\newcommand{\Am}{$^{241}$Am}
\renewcommand{\Kr}{$^{83m}$Kr}
\newcommand{\Li}{$^{7}$Li}
\newcommand{\Be}{$^{7}$Be}
\newcommand{\Er}{$E_r$}
\renewcommand{\Qy}{$Q_{y}$}
\newcommand{\Nex}{$N_\mathrm{ex}$}
\newcommand{\Ni}{$N_\mathrm{i}$}
\newcommand{\NexNi}{$N_\mathrm{ex}/N_\mathrm{i}$}
\newcommand{\fninety}{$f_{90}$}
\newcommand{\fprompt}{$f_\mathrm{p}$}
\newcommand{\tdrift}{$t_\mathrm{drift}$}
\newcommand{\ISO}[2]{$^{#2}\mathrm{#1}$}
\newcommand{\edrift}{$\mathcal{E}_d$}
\newcommand{\eex}{$\mathcal{E}_{ex}$}
\newcommand{\eel}{$\mathcal{E}_{el}$}
\newcommand{\Wphmax}{$W_{ph}$(max)}
\newcommand{\xy}{$x-y$}
\newcommand{\vect}[1]{\boldsymbol{#1}}
\newcommand{\todo}[1]{\textcolor{red}{#1}}

%
\journalname{Eur. Phys. J. C}
\makeatletter
\renewcommand{\thanksref}[1]{\nolinebreak\textsuperscript{\ref{#1}}\nolinebreak\checknextarg}
\newcommand{\checknextarg}{\@ifnextchar\bgroup{\nolinebreak\gobblenextarg}{}}
\newcommand{\gobblenextarg}[1]{ \textsuperscript{\nolinebreak\hspace{-4pt}\mbox{\nolinebreak$^,$
\nolinebreak\ref{#1}\nolinebreak}\nolinebreak} \@ifnextchar\bgroup{\gobblenextarg}{}}

%
\begin{document}

\title{Directionality of nuclear recoils in a liquid argon time projection chamber
}
%\subtitle{Do you have a subtitle?\\ If so, write it here}

%\titlerunning{Short form of title}        % if too long for running head

%\author{DarkSide\thanksref{e1,addr1}
%        \and
%        Collaboration\thanksref{addr2,addr3} %etc.
%}

\author{The DarkSide-20k Collaboration$^\text{\normalfont a,1}$}
%\author{The Global Argon Dark Matter Collaboration$^\text{\normalfont a,1}$}
%\collaboration{The DarkSide-20k and Aria Collaborations\thanksref{e1}}

\thankstext{e1}{e-mail: ds-ed@lngs.infn.it}

% Set authors
\institute{See back for author list \label{addr1}}


%\thankstext{t1}{Grants or other notes
%about the article that should go on the front page should be
%placed here. General acknowledgments should be placed at the end of the article.
%\thankstext{e1}{e-mail: fauthor@example.com}

%\authorrunning{The DarkSide Collaboration} % if too long for running head


\date{Received: date / Accepted: date}
% The correct dates will be entered by the editor


\maketitle

\begin{abstract}
The direct search for dark matter in the form of weakly interacting massive particles (WIMP) is performed by detecting nuclear recoils (NR) produced in a target material from the WIMP elastic scattering. 
A promising experimental strategy for direct dark matter search employs argon dual-phase time projection chambers (\TPC). One of the advantages of the \TPC\ is the capability to detect both the scintillation and charge signals produced by NRs. 
Furthermore, the existence of a drift electric field in the \TPC\ breaks the rotational symmetry: the angle between the drift field and the momentum of the recoiling nucleus can potentially affect the charge recombination probability in liquid argon and then the relative balance between the two signal channels.
This fact could make the detector sensitive to the directionality of the WIMP-induced signal, enabling unmistakable annual and daily modulation signatures for future searches aiming for discovery. The Recoil Directionality (\ReD) experiment was designed to probe for such directional 
sensitivity. The \TPC\ of \ReD\ was irradiated with neutrons at the INFN Laboratori Nazionali del Sud, and data were taken with \SI{72}{\kilo\electronvolt} NRs of known recoil directions. The direction-dependent liquid argon charge recombination model by Cataudella et al. was adopted and a likelihood statistical analysis was performed, which gave no indications of significant dependence of the detector response to the recoil direction. The aspect ratio $R$ of the initial ionization cloud is estimated to be $1.037\pm0.027$ and the upper limit is $R < 1.072$ with \SI{90}{\percent} confidence level. 
\keywords{Time Projection Chamber \and Dark Matter \and Noble liquid detectors \and Directional response}
\end{abstract}

%\linenumbers
\section{Introduction}\label{intro}
\section{Introduction}
Current quantum hardware is unable to carry out universal quantum computations due to the buildup of errors that occur during the computation. 
The magnitude of the individual error is currently above the value that the Threshold Theorem requires in order to kick-start quantum error correction and fault-tolerant quantum computation~\cite[Section 10.6]{nielsen_chuang_2010}. 
Although the experimentally achieved fidelity rates are promising and the error bounds are inching closer to the required threshold, we will have to work for the foreseeable future with quantum hardware with errors that build-up during the computation.  This implies that we can only do a limited number of steps before the output of the computation has become completely uncorrelated with the intended one.

For fault-tolerant quantum computing, we repeat four steps: 
1) We apply a number of single and two-qubit quantum gates, in parallel whenever possible; 
2) We perform a syndrome measurement on a subset of the qubits; 
3) We perform fast classical computations to determine which errors have occurred and how to correct them; 
and, 4) We apply correction terms based on the classical computations.
We then repeat these four steps with a next sequence of gates. 
These four steps are essential to fault-tolerant quantum computing. 


The starting point of this work is to use the four steps outlined above, not to carry out error correction and fault-tolerant computation, but to enhance short, constant-depth, {\em uncorrected} quantum circuits that perform single qubit gates and {\em nearest-neighbor} two qubit gates. 
Since in the long run we will have to implement error-correction and fault-tolerant computation anyhow, and this is done by such a four-step process, why not make other use of this architecture? Moreover, on some of the quantum hardware platforms, these operations are already in place.
Embracing this idea we naturally arrive at the question: what is the computational power of \textit{low-depth} quantum-classical circuits organized as in the four steps outlined above? 
We thus investigate circuits that execute a small, ideally constant, number of stages, where at each stage we may apply, in parallel, single qubit gates and {\em nearest-neighbor} two qubit gates, followed by measurements, followed by low-depth classical computations of which the outcome can control quantum gates in later stages. 
It is not clear, at first, whether such circuits, especially with constant depth, can do anything remotely useful. 
But we will see that this is indeed the case: many quantum computations can be done by such circuits in constant depth. 
By parallelizing quantum computations in this way, we improve the overall computational capabilities of these circuits, as we do not incur errors on qubits that are idle, simply because qubits are not idle for a very long time. 
Furthermore, reducing the depth of quantum circuits, at the cost of increasing width, allows the circuit to be run faster even if errors occur.

The first usage of such a four-step layout, not to do error correction, but to perform computations, can be found in the paradigm of measurement-based quantum computing~\cite{gottesman1999demonstrating,raussendorf2001one,jozsa2006introduction,clark2007generalised}: 
A universal form of quantum computing where a quantum state is prepared and operations are performed by measuring qubits in different bases, depending on previous measurements and intermediate measurements.

\citeauthor{PhamSvore2013} were the first to formalize the four-step protocol for performing computations~\cite{PhamSvore2013}. They included specific hardware topologies by considering two-dimensional graphs for imposing constraints on qubit interactions. In their model, they develop circuits for particularly useful multi-qubit gates, including specifying costs in the width, number of qubits, depth, number of concurrent time steps, size, and total number of non-Identity operations.
As a result, they find an algorithm that factors integers in polylogarithmic depth.
\citeauthor{Browne:2011} showed that the main tool in the work by \citeauthor{PhamSvore2013}, the fan-out gate, can also be replaced by additional log-depth classical computations in the measurement-based quantum computing setting~\cite{Browne:2011}.

More recently, \citeauthor{Cirac:2021} introduced a scheme to implement unitary operations involving quantum circuits combined with Local Operations and Classical Communication ($\mathsf{LOCC}$) channels: $\mathsf{LOCC}$-assisted quantum circuits~\cite{Cirac:2021}. Similarly to the four-step scheme we just described, they allow for a short depth circuit to be run on the qubits, followed by one round of $\mathsf{LOCC}$, in which ancilla qubits are measured and local unitaries are applied based on the measurement outcomes. They show that in this model any 1D transitionally invariant matrix-product state (MPS) with fixed bond dimension is in the same phase of matter as the trivial state. Similar ideas can be found in~\cite{TVV_NonAbelianTopologicalOrder_2022, tantivasadakarn2021long}.

In this work, we introduce a new model, called \textit{Local Alternating Quantum-Classical Computations} ($\LAQCC$). In this model we alternate between running quantum circuits (constrained by locality), ending in the measurement of a subset of qubits, and fast classical computations based on the measurement results. The outcome of the classical computations are then used to control future quantum circuits. We allow for flexibility in this model, by giving different constraints to the power of both the quantum circuits and the classical circuits as well as the number of alternations between them. 
Most attention will be given to $\LAQCC$ containing quantum circuits of constant depth, classical circuits of logarithmic depth and at most a constant number of alternations between them. 
Any circuit constructed in this model is considered to be of constant depth. 
We restrict ourselves to logarithmic depth classical computations, as this is the first natural and non-trivial extension beyond constant-depth classical computations. 
Constant-depth classical computations do however also have an equivalent constant-depth quantum implementation.

The definition of $\LAQCC$ sharpens the original definition of \citeauthor{PhamSvore2013} by adding constraints to the intermediate classical computations. This allows us to bound the power of $\LAQCC$ from above. 

The main result of \citeauthor{Cirac:2021}, that 1D translational invariant MPS with fixed bond dimension can be prepared by $\mathsf{LOCC}$-assisted circuits, relies on local symmetries of the MPS. These symmetries allow them to prepare local states (on a constant number of qubits) and glue them together by doing one round of the appropriate entangling measurement and corrections, after which they run a round of local unitaries to get the desired result. This general scheme for preparing states that exhibit an MPS description with the appropriate local symmetries requires only geometrically local unitaries and one round of measurement and corrections an therefore is accessible in $\LAQCC$. Studying different local symmetries, known as Symmetry Protected Topological (SPT) phases of matter, to find measurement-based constant depth circuits for states is a broad ongoing field of research~\cite{TVV_NonAbelianTopologicalOrder_2022, tantivasadakarn2021long, smith2023deterministic}. 
All these schemes have a $\LAQCC$ implementation.

%$\LAQCC$-circuits also exist for general schemes of preparing local states, based on the local tensors, and gluing them together using one round of entangled measurement and corrections, based on the local symmetry. 
%The main result of \citeauthor{Cirac:2021}, that 1D translational invariant MPS with fixed bond dimension can be prepared by $\mathsf{LOCC}$-assisted circuits, relies heavily on local symmetries of the MPS and as a result also has an equivalent $\LAQCC$ implementation. 
%The corrections applied after the measurement round are local unitaries depending on the local symmetries of the MPS. 

 

%This general scheme of preparing local states, based on the local tensors, and gluing it together by doing one round of entangled measurement and corrections, based on the local symmetry, is accessible in $\LAQCC$.
Note however that \citeauthor{Cirac:2021} also suggest a circuit for the $W$-state.
This circuit uses sequentially and dependent measurement-based corrections of the ancilla qubits. 
These dependent measurements translate to sequential alternations between the quantum and classical circuits and therefore increase the total depth to linear depth, exceeding the constant-depth constraints imposed by $\LAQCC$-circuits. 

We study the power of the $\LAQCC$ model with respect to state preparation, showing that even with only constant quantum-depth and logarithmic classical depth it remains possible to prepare states with long-range entanglement.
Another surprising result is that it is unlikely that $\LAQCC$ circuits are classically simulatable. We show that any instantaneous quantum polynomial-time (IQP) circuit~\cite{Bremner2010,Shepherd2009} has an $\LAQCC$ implementation.
Classical simulation of IQP circuits implies the collapse of the polynomial hierarchy to the third level, which is not believed to be true~\cite{Bremner2017}. Therefore, we expect that $\LAQCC$ circuits are unlikely to be classically simulatable. We bound the power of $\LAQCC$ by showing that it is contained in $\QNC^1$, the class of polynomial-size, log-depth circuits.

Next, we also study the power that intermediate classical calculations can add to quantum computations, by considering a new model that alternates between polynomially many polynomial-depth quantum circuits and unbounded classical computations
We study this model by doing a complexity theoretical analysis, where we draw inspiration from the notions of complexity given by \citeauthor{RosenthalYuen:2022}, \citeauthor{MetgerYuen:2023}, and \citeauthor{Aaronson:2004}.
All three complexity notions are based on the notion of state preparation, instead of more traditional definition of complexity such as the decidability of a computational problem. 
The first two consider classes based on sequences of quantum states preparable by a polynomial-sized quantum circuit, where the circuits are uniformly generated by a computational class, for instance, the class $\mathsf{PSPACE}$, which results in the complexity class $\mathsf{StatePSPACE}$~\cite{RosenthalYuen:2022,MetgerYuen:2023}.
The third notion considers a relative complexity, where the complexity is measured between two given states, and is measured by the number of gates, from a given gate-set, required to transform one state in another state~\cite{Aaronson:2004}. 
For our definition of state preparation complexity, we drop the uniformity constraint from~\cite{RosenthalYuen:2022,MetgerYuen:2023} and define a class as $\mathsf{StateX}$, which refers to states preparable by circuits of type $\mathsf{X}$. 
As an example, if $\mathsf{X} = \QNC^0$, this results in the class $\mathsf{StateQNC^0}$, which is the set of states preparable from the $\ket{0}^n$ state by poly-size constant-depth circuits. 
This notion is similar to the relative complexity from~\cite{Aaronson:2004}, where one state is the  $\ket{0}^n$ state and instead of counting the number of gates we consider the set of states preparable by a fixed number of gates. Using this notion of complexity we show that any state preparable by an $\LAQCC^*$ circuit is also preparable by a $\mathsf{PostQPoly}$ circuit, the class of circuits of polynomial depth with an additional post-selection gate. 

All Clifford circuits have a constant-depth $\LAQCC$ implementation, implying that any stabilizer state can be implemented by a constant-depth $\LAQCC$ circuit, see Section~\ref{sec:clifford_circuits} for a proof of this statement. 
Efficient circuits for stabilizer states have been known already through measurement-based quantum computing. Therefore this paper focuses on the preparation of non-stabilizer states, and as a surprising result we find novel constant-depth protocols for four very natural classes of non-stabilizer states.
Despite the extensive research into these four classes of non-stabilizer states and the many applications of them, no efficient constant- or low-depth state preparation protocols are known yet. We specifically consider these four classes as they are all often used as initial states in other algorithms.

The first state is a uniform superposition over an arbitrary number of states. 
This state finds applications in many quantum algorithms, as they often start with a uniform superposition over multiple states. 
This superposition is often achieved by applying Hadamard gates to every qubit due to its simplicity to prepare. 
Yet, the analysis of many algorithms, such as Shor's algorithm~\cite{Shor:1997}, would benefit from a different initial superposition. 
The circuit to prepare the uniform superposition over an arbitrary number of states uses an exact version of Grover search as a subroutine, that turns a probabilistic circuit, with a known constant probability of success, into a deterministic circuit. 
We use the circuit for preparing a uniform superposition over an arbitrary number of states as a subroutine in the next two quantum state preparation protocols. 

The second state is the $W$-state, the uniform superposition over all computational basis states of Hamming-weight~$1$, a natural long-ranged entangled state that displays a fundamentally nonequivalent type of entanglement from the Greenberger–Horne–Zeilinger state~\cite{WState:2000}, for which $\LAQCC$-type constant-depth circuits were previously known~\cite{PhamSvore2013, Cirac:2021}. 
The $W$-state is often used as benchmark for new quantum hardware~\cite{Haffner2005,Neeley2010,GarciaPerez:2021}. 
A novel way to prepare the $W$-state therefore gives a new way to benchmark different quantum devices with each other. 
A circuit for preparing the $W$-state was given in~\cite{Cirac:2021}, but this implementation requires sequentially alternating measurements followed by local unitaries, which in the $\LAQCC$ model is not considered to be of constant depth. 
We improve this protocol by giving an $\LAQCC$ implementation of the $W$-state, based on a compress-uncompress method that links the one-hot and binary encoding of integers.

The third state considered is the Dicke state, a generalization of the $W$-state, a superposition over all computational basis states with Hamming-weight $k$~\cite{Dicke:1954}. 
Dicke states have relevance in various practical settings.
For instance, for quantum game theory~\cite{zdemir2007}, quantum storage~\cite{Bacon_Compress:2006,Plesch:2010}, quantum error correction~\cite{ouyang2014permutation}, quantum metrology~\cite{toth2012multipartite}, and quantum networking~\cite{prevedel2009experimental}. 
Dicke states have been used as a starting state for variational optimization algorithms, most notably Quantum Alternating Operator Ansatz (QAOA)~\cite{Hadfield2019}, to find solutions to problems such as Maximum k-vertex Cover~\cite{Brandhofer2022,cook2020quantum}.
The ground states of physical Hamiltonians describing one-dimensional chains tend to show a resemblance to Dicke states such as states resulting from the Bethe ansatz, making them an ideal starting state when investigating the ground state behavior of these Hamiltonians~\cite{TDL_BetheAnsatzDerivation:2010,B_ExcitedStateQuantumPhaseTransitions:2013,DickeTransitions:2021}. 
For instance, the algorithm by \citeauthor{van2021preparing}, who give an algorithm to prepare the Bethe ansatz eigenstates of the spin-1/2 XXZ spin chain, starts by first preparing a Dicke state~\cite{van2021preparing}. 
A Dicke-state preparation protocol based on the compress-uncompress methodology used in the $W$-state furthermore finds applications in entanglement distillation, where the entanglement of a large state is concentrated on only a few qubits. 
Efficient deterministic circuits for preparing Dicke states have been proposed by \citeauthor{bartschi2019deterministic}~\cite{bartschi2019deterministic, bartschi2022deterministic_short_depth}. 
They provide a quantum circuit of depth $\mathO(k \log(\frac{n}{k}))$, allowing arbitrary connectivity, to prepare a Dicke state, which they conjecture to be optimal when $k$ is constant. 
In this work, we provide a constant-depth $\LAQCC$ circuit below their conjectured bound already for constant $k$. 
However, this does not directly disprove their conjecture, as we allow for intermediate measurements and classical computations. 
More significantly, we even construct constant-depth $\LAQCC$ circuits for $k = \mathO(\sqrt{n})$ greatly improving their bound.
This construction extends the compress-uncompress method for the $W$-state combined with additional subroutines. 

We continue with a log-depth state preparation protocol for the Dicke-state for arbitrary $k$. 
This protocol implements an efficient transformation between the factoradic number representation and the combinatorial number representation of a positive integer. 
The combinatorial number representation relates directly to the Dicke state. 
The provided efficient transformation between number representation systems might be of independent interest. 

We conclude by modifying our protocol for preparing a Dicke-state to a protocol that prepares quantum many-body scar states in constant-depth. 
These states have low entanglement and longer coherence times than states with similar energy density.
These characteristics make many-body scar states interesting to analyze and relevant within physics.
Many-body scar states appear for instance in the AKLT model~\cite{AKLT:1987,MRBAR:2018,MRB:2018} and different spin models~\cite{SI:2019,MOBFR:2020}.
Known methods for preparing these states have polynomial-depth~\cite{Gustafson:2023}, whereas our circuit has constant depth. 

% We conclude by studying the power that intermediate classical calculations can add to quantum computations. 
% In this study, we define a new model that relaxes constant-depth quantum circuits to polynomial depth quantum circuits, log-depth classical calculations to unbounded classical computations and a constant number of alternations to a polynomial number of alternations. 
% We call this model $\LAQCC^*$. 
% We study this model by doing a complexity theoretical analysis, where we draw inspiration from the notions of complexity given by \citeauthor{RosenthalYuen:2022}, \citeauthor{MetgerYuen:2023}, and \citeauthor{Aaronson:2004}.
% All three complexity notions are based on the notion of state preparation, instead of more traditional definition of complexity such as the decidability of a computational problem. 
% The first two consider classes based on sequences of quantum states preparable by a polynomial-sized quantum circuit, where the circuits are uniformly generated by a computational class, for instance, the class $\mathsf{PSPACE}$, which results in the complexity class $\mathsf{StatePSPACE}$~\cite{RosenthalYuen:2022,MetgerYuen:2023}.
% The third notion considers a relative complexity, where the complexity is measured between two given states, and is measured by the number of gates, from a given gate-set, required to transform one state in another state~\cite{Aaronson:2004}. 
% For our definition of state preparation complexity, we drop the uniformity constraint from~\cite{RosenthalYuen:2022,MetgerYuen:2023} and define a class as $\mathsf{StateX}$, which refers to states preparable by circuits of type $\mathsf{X}$. 
% As an example, if $\mathsf{X} = \QNC^0$, this results in the class $\mathsf{StateQNC^0}$, which is the set of states preparable from the $\ket{0}^n$ state by poly-size constant-depth circuits. 
% This notion is similar to the relative complexity from~\cite{Aaronson:2004}, where one state is the  $\ket{0}^n$ state and instead of counting the number of gates we consider the set of states preparable by a fixed number of gates. Using this notion of complexity we show that any state preparable by an $\LAQCC^*$ circuit is also preparable by a $\mathsf{PostQPoly}$ circuit, the class of circuits of polynomial depth with an additional post-selection gate. 

\paragraph{Summary of results}
\begin{itemize}
    \item We give a new definition of a computational model that captures the power of the four step process: applying a constant number of layers of one- and two-qubit gates; performing a syndrome measurement; perform a fast classical computation determining corrections; apply corrections. We call this model \emph{Local Alternating Quantum Classical Computations}, or $\LAQCC$ for short. In this model we bound the allowed quantum operations, intermediate classical calculations, and number of rounds separately. In Section~\ref{sec:LAQCC_model} we define this model and give a list of operations based on results from literature contained in this computational model. In some of these operations we explicitly use that we allow for multiple, but at most constant, rounds  of corrections.
    \item  We show show that there exist $\LAQCC$ circuits that can not be weakly simulated in Section~\ref{sec:IQP_in_LAQCC}. We further show that for every $\LAQCC$ circuit there exists a $\QNC^1$ circuit simulating it perfectly, in Section~\ref{sec:LAQCC_in_QNC1}.
    \item We introduce a new type computational complexity for preparing states and show that the extension of $\LAQCC$ where we allow a polynomial number of rounds and unbounded classical computation, is contained in $\mathsf{PostQPoly}$, the class of polynomial circuits with post-selection, in Section~\ref{sec:Complexity results}.
    \item We show a protocol to prepare the uniform superposition state of size $q$ in $\LAQCC$ using $\mathO(\ceil{\log_2(q)}^2)$ qubits in Section~\ref{sec:superposition_modulo_q}. 
    \item We show a protocol to prepare the $W_n$ state in $\LAQCC$ using $\mathO(n\log(n))$ qubits in Section~\ref{sec:W_state_in_LAQCC}.
    \item We show two ways of preparing the Dicke-$(n,k)$ state. The first method is in $\LAQCC$, works up to $k = \mathO(\sqrt{n})$, uses $\mathO(n^2\log(n))$ qubits, and is found in Section~\ref{sec:dicke:small_k}. The second method is in $\LAQCC\text{-}\mathsf{LOG}$ (an extension of $\LAQCC$ allowing for logarithmic number of alterations instead of constant), works for any $k$, uses $\mathO(\text{poly}(n))$ qubits, and is found in Section~\ref{sec:Dicke_in_LAQCC_LOG}. 
    \item We extend on our $\LAQCC$ method of generating Dicke-$(n,k)$ states for $k = \mathO(\sqrt{n})$ and show a protocol to generate many-body scar states for a particular Hamiltonian in $\LAQCC$ (Section~\ref{sec:many_body_scar}). 
\end{itemize}
Summarized in a table, we provide the following state generation protocols:
\begin{table}[htb]
\centering
\begin{tabular}{l|l|l|l}
\textbf{State description} & \textbf{Width} & \textbf{Depth} & \textbf{Implementation}\\
\hline 
Uniform superposition mod $q$: $\frac{1}{\sqrt{q}} \sum_{i = 0}^{q-1}\ket{i}$ & $\mathO(\ceil{\log^2 q})$ & $\mathO(1)$ & Section~\ref{sec:superposition_modulo_q}\\

$W$-state: $\frac{1}{\sqrt{n}}\sum_{i = 0}^{n-1}\ket{e_i}$ & $\mathO(n \log n)$ & $\mathO(1)$ & Section~\ref{sec:W_state_in_LAQCC}\\

Dicke-$(n,k)$, $k = \mathO(\sqrt{n})$: $\binom{n}{k}^{-1/2}\sum_{x \in \{0,1\}^n: |x| = k} \ket{x}$ &  $\mathO(n^2\log n)$ & $\mathO(1)$ 
&Section~\ref{sec:dicke:small_k}\\

Dicke-$(n,k)$: $\binom{n}{k}^{-1/2}\sum_{x \in \{0,1\}^n: |x| = k} \ket{x}$ & $\mathO(\text{poly}(n))$ & $\mathO(\log n)$ &Section~\ref{sec:Dicke_in_LAQCC_LOG}\\

QMBS: $\ket{S_k} = \frac{1}{k! \sqrt{\mathcal N(n,k)}}(Q^\dagger)^k \ket{\Omega}$ &  $\mathO(n^2\log n)$ & $\mathO(1)$  &  Section~\ref{sec:many_body_scar}
\end{tabular}
\caption{Summary of state preparation protocols given in this paper.}
\label{tab:sate_prep}
\end{table}
In the entry for the quantum many-body scar state $Q$ denotes the raising operator and $\mathcal N(n,k)=\binom{n-k-1}{k}$. 
Section~\ref{sec:many_body_scar} will provide more details on the variables and the implementation. 

\paragraph{Organization of the paper}
\noindent We first introduce relevant preliminaries in Section~\ref{sec:preliminaries}. 
In Section~\ref{sec:LAQCC_model} we formally define the class of Local Alternating Quantum-Classical Computations ($\LAQCC$). We also show that any Clifford circuit can be implemented in constant depth $\LAQCC$ (a result based on a result from measurement-based quantum computing~\cite{jozsa2006introduction}). 
This result allows us to give many useful multi-qubit gates and routines in Section~\ref{sec:gates_created_in_LAQCC}. 
Beyond that we show that constant depth $\LAQCC$ circuits are contained in $\QNC^1$ and that any $\mathsf{IQP}$ circuit has an $\LAQCC$ implementation.
We conclude this section with an analysis of a more powerful instantiation of $\LAQCC$ and show an inclusion with respect to the class $\mathsf{PostQPoly}$, which is the class of circuits of polynomial depth with one additional post-selection gate. 
In Section~\ref{sec:state_prep_in_LAQCC} we give $\LAQCC$ circuit implementations for preparing the uniform superposition over an arbitrary number of states, the $W$-state and the Dicke state up to $k = \mathO(\sqrt{n})$. We furthermore give a log-depth circuit implementation for preparing the Dicke state for any $k$. We conclude by showing a $\LAQCC$ circuit for generating many body scar states of a particular type of Hamiltonian.




\section{The response of Ar to nuclear recoils} \label{sec:ArResponse}
WIMPs deposit energy in LAr through elastic scattering on Ar nuclei. 
The subsequent energy loss of the \NR\ involves nuclear stopping, ionization, charge recombination, and scintillation. 
Through the series of physical processes, the total energy deposited in the \TPC\ is eventually divided into the detectable photons (S1) and electrons (S2), and the undetectable phonons (heat). 
%The charge recombination process determines the ratio between S1 and S2 signals: this could possibly depend on the angle between the recoil momentum and \edrift, thus making argon dual-phase \TPC\ potentially sensitive to the direction of the incident WIMPs through S1-S2 ratio. 

Directional modulation of charge recombination is expected when the spatial charge distribution of ionization is anisotropic. Conventional \NR\ charge recombination models often assume an isotropic charge distribution. For example, the commonly-used Thomas-Imel model~\cite{Thomas:1987ek,Szydagis:2011tk} assumes that charges are uniformly distributed in a cubic box of size $a$; the only free parameter for a given detector material is the initial charge $Q_0$. 
The probability of charge surviving recombination under the electric drift field \edrift\ is 
\begin{equation}
    p(a,Q_0) = \frac{\mathcal{E}_d}{\xi(a,Q_0)}\ln\left(1+\frac{\xi(a,Q_0)}{\mathcal{E}_d}\right),
\end{equation}
where
\begin{equation} 
\xi(a,Q_0)= \frac{\alpha Q_0}{4a^2\mu_-}; \label{eq:TIxi}
\end{equation}
$\alpha$ is the Langevin
recombination coefficient~\cite{Langevin1903,Bubon:2016hc}, which depends on the carrier  
mobilities ($\mu_-$ and $\mu_+$ for electrons and ions, respectively) and on the dielectric constant as
\begin{equation}
\alpha = \frac{(\mu_-+\mu_+)}{\epsilon_0 \epsilon_r}. \label{eq:alpha}
\end{equation}

In order to introduce the directionality, the electron distribution after thermalization needs to be included in the model. One approach is to use the Jaff\'e  model~\cite{Jaffe:1913gs,Birks:1951boa}, commonly referred to as the columnar recombination model, where the charge distribution is modeled by a column with radius $b$, length $l$, and angle $\theta$ between its axis $\hat{r}_0$ and the drift field \edrift. The Jaff\'e model is commonly adopted for the straight tracks from minimum ionizing particles. Since \NR\ tracks are more localized, a more general and flexible parameterization
of the charge distribution $q_0(\vec{r})$ has been proposed by Cataudella et al.~\cite{Cataudella:2017kcf}, which consists of 
a three dimensional Gaussian with an elliptical profile
\begin{equation}
    q_0(\vec{r}) = \frac{Q_0}{(2\pi)^{3/2}R\sigma^3}\exp{\left(-\left(\frac{\vec{r}\cdot\hat{r}_0}{R\sigma}\right)^2 - \left(\frac{\vec{r}\times\hat{r}_0}{\sigma}\right)^2 \right) }, \label{eq:clouddist}
\end{equation}
where $\sigma$ characterizes the size of the distribution, $\hat{r}_0$ is the direction vector of the long axis, and $R$ is the aspect ratio between the long and short axes. 
 The  probability of charge surviving recombination is calculated in Ref.~\cite{Cataudella:2017kcf} as
\begin{equation}
\label{eq:rec_direction}
    p(R,\theta,Q_0) = -\frac{\mathcal{E}_d f(R,\theta)}{\xi_m}\mathrm{Li}_2\left(-\frac{\xi_m}{\mathcal{E}_df(R,\theta)}\right),
\end{equation}
where
\begin{equation}
\xi_m=\frac{\alpha Q_0}{2\pi \sigma^2\mu_-} \label{eq:ximCataudella}
\end{equation}
is the generalization of the Thomas-Imel parameter $\xi$ of
Eq.~\ref{eq:TIxi} and $\mathrm{Li}_2$ is the second order polylogarithm function.
%, and $\theta$ is the angle between $\hat{r_0}$ and
%the drift field.
%Since the ion mobility is negligible compared to electrons,
%By replacing $\alpha$ from Eq.~\ref{eq:alpha}, Eq.~\ref{eq:ximCataudella} can be recast as 
%\begin{equation}
%\xi_m = \frac{Q_0}{2\pi\varepsilon_r\varepsilon_0\sigma^2}.
%\end{equation}
The term $f(R,\theta)$ captures the directionality dependence and it has 
the functional form
\begin{equation}\label{eq:rec_modified_TI_ftheta}
    f(R,\theta) = \sqrt{R^2\sin^2\theta+\cos^2\theta},
\end{equation}
being $\theta$  the angle between $\hat{r}_0$ and \edrift.
When $R=1$, $f(R,\theta)=1$, so directionality vanishes and 
Eq.~\ref{eq:rec_direction} reduces to the Thomas-Imel model.

Since directionality effects do not occur before recombination, well-established models are used here to describe the S1 and S2 yields, that for \NRs\ also depend on nuclear and electronic quenching. Following the Lindhard model~\cite{Lindhard:1963vo,Bezrukov:2010qa}, the nuclear quenching factor, i.e. the ratio of the visible energy in the excitation and ionization channel to the total recoil energy, is described by 
\begin{equation}
    f_{n}(\varepsilon) = \frac{kg(\varepsilon)}{1+kg(\varepsilon)},
\end{equation}
where $k=0.133\, Z^{2/3}A^{-1/2}$ is a dimensionless factor depending on the Ar target nucleus ($A=40$, $Z=18$); the function $g(\varepsilon)$ is numerically approximated by 
Lindhard~\cite{Lindhard:1963vo} and it has the form
\begin{equation}
g(\varepsilon) = 3\,\varepsilon^{0.15}+0.7\,\varepsilon^{0.6}+\varepsilon;
\end{equation}
finally  $\varepsilon$ is the dimensionless reduced energy
\begin{equation}
    \varepsilon = \frac{4\pi \epsilon_0 a}{2e^2Z^2}E_{r} 
    = 11.5\, Z^{-7/3}\frac{E_r}{[\si{\kilo\electronvolt}]},
\end{equation}
being  $E_r$ the recoil energy  and $a$ the Thomas-Fermi screening length, calculated from
the Bohr radius $a_0$ as $a = 0.626 \cdot a_0 \cdot Z^{-1/3}$~\cite{Bezrukov:2010qa}.

The measurable energy is further reduced by electronic quenching, following 
the Mei model~\cite{Mei:2008ca}
\begin{equation}
    f_{l} = \frac{1}{1+k_e s_e}, 
\end{equation}
where $s_e=k\varepsilon^{1/2}$ is the dimensionless electronic stopping power 
and $k_e$ is associated to the original parameter $K_e$ of 
Ref.~\cite{Mei:2008ca} as $k_e = K_e (\mathrm{d}E/\mathrm{d}x)_e / (s_e \rho_m)$, with $\rho_m$ being the mass density of LAr. 
%K_e=\SI{7.4e-4}{\mega\electron\volt^{-1}\gram\centimeter^{-2}} from Mei, equal to k_e=4.1, SCENE adopts k_e=2.8\pm0.1. Here we adopt the same value as SCENE.

%The 
%ratio of energy flowing to the excitation channel and the ionization channel is expressed by the parameter %\NexNi. 
Summing up the components, the expectation of total quanta $\langle N_0 \rangle$, ionization $\langle N_\mathrm{i} \rangle$ and excitation $\langle N_\mathrm{ex} \rangle$ from a \NR\ of energy $E_r$ in LAr before recombination is
\begin{eqnarray}
    \langle N_0 \rangle &=& \frac{E_r f_n f_l}{W_\mathrm{ph}} \\
    \label{eq:NexNiYield}
    \langle N_\mathrm{i} \rangle &=& \langle N_0 \rangle \frac{1}{1+N_\mathrm{ex}/N_\mathrm{i}} \\
    \langle N_\mathrm{ex} \rangle &=& \langle N_0 \rangle - \langle N_\mathrm{i} \rangle %\left(1-\frac{1}{1+N_\mathrm{ex}/N_\mathrm{i}}\right) .
\end{eqnarray}
where $W_{ph}$ is the average energy required to produce one scintillation photon in LAr and \NexNi\ is 
the excitation-to-ionization ratio directly induced by the fast ion and by its secondaries.
As a first approximation, \NexNi\ is usually treated as an energy independent constant~\cite{Doke:2002oab,Hitachi:2021hac}, which is related to the atomic levels in argon. However, the distribution of momentum transfer to electrons in the electronic stopping power is energy-dependent, which motivates the introduction of a variable \NexNi\, vs. energy. This is corroborated by the SCENE data~\cite{Cao:2015ks}, which also indicate an increase in \NexNi\ with respect to the \NR\ energy. The \NexNi\ values adopted for this work are taken from Table VIII of Ref.~\cite{Cao:2015ks}, with a
linear interpolation between the energy points. The \NexNi\ at zero energy is set to the commonly-adopted
value of $0.2$. 

The detectable electron and photon yields after recombination are
%\begin{eqnarray}
 %   \langle N_\mathrm{e^-} \rangle &=& \langle N_0 \rangle \frac{1}{1+N_\mathrm{ex}/N_\mathrm{i}}\, %p(R,\theta,\langle N_\mathrm{i}\rangle) \label{eq:ne} \\ 
 %   \langle N_\mathrm{ph} \rangle &=& \langle N_0 \rangle \left(1 - %\frac{1}{1+N_\mathrm{ex}/N_\mathrm{i}}\, p(R,\theta,\langle N_\mathrm{i}\rangle)\right). \label{eq:nph} 
%\end{eqnarray}
\begin{eqnarray}
\langle N_\mathrm{e^-} \rangle &=& \langle N_\mathrm{i} \rangle p(R,\theta,Q_0) = \langle N_\mathrm{0} \rangle \frac{p(R,\theta,Q_0)}{1+N_\mathrm{ex}/N_\mathrm{i}} \label{eq:ne} \\ 
\langle N_\mathrm{ph} \rangle &=& \langle N_0 \rangle -  \langle N_\mathrm{e^-}\rangle. \label{eq:nph} 
\end{eqnarray}


%% Figure environment removed


%Now consider the $S1$ and $S2$ gain of the \TPC. 
%The \TPC\ signals S1 and S2 are measured in units of photo-electrons (PE) in the photosensor. The scintillation light collection efficiency and the photosensor quantum efficiency are less than unity, so that S1 has a gain $g_1 = \mathrm{S1}/N_\mathrm{ph} [\si{PE/ph}]$ less than one. Charges are amplified and converted to photons through the gas pocket electroluminescence, so that S2 has a gain $g_2 = \mathrm{S2}/N_\mathrm{e^-} [\si{PE/e^-}]$ greater than one. 

%Fluctuations in signals are particularly important for the directionality study, as a directional modulated signal could be washed out by random fluctuations: fluctuations hence reduce the resolution of potential signal directionality. The intrinsic fluctuations during signal generation in LAr and the detector resolution rising from signal propagation, collection, and amplification contribute to the S1-S2 two-dimensional spectrum profile of a mono-energetic \NR. 
The capability to measure the \NR\ direction can be hidden by random fluctuations in S1 and S2. Indeed, the intrinsic fluctuations during signal generation in LAr and the detector resolution arising from signal propagation, collection, and amplification contribute to smear out the S1-S2 two-dimensional spectrum of a mono-energetic \NR. 
The intrinsic fluctuations are present for both the charge and light channels. The fluctuation in the total number of visible quanta $N_0$ is assumed here to be Gaussian distributed with a Fano factor $F=0.107$ \cite{szydagis2021review}: 
\begin{equation}
    N_0 \sim \mathrm{Gaussian}(\langle N_0 \rangle, \sqrt{F\langle N_0\rangle}).
\end{equation}
%Since there is a significant and degenerated contribution from the \NR\ energy spread in our setup, the result is not sensitive to the different choices of nuclear quenching fluctuation. For simplicity, we stick to the Poisson distribution.
The partition of $N_0$ between $N_\mathrm{e^-}$ and $N_\mathrm{ph}$ follows a 
binomial distribution governed by \NexNi\ and by the recombination probability (see Eq.~\ref{eq:ne}):
%\begin{eqnarray}
%N_\mathrm{e^-} \sim \mathrm{Binomial}(\langle N_0 \rangle, \langle N_\mathrm{e^-} \rangle / \langle  N_0 \rangle)  \\ 
%N_\mathrm{ph} =N_0-N_\mathrm{e^-}
%\end{eqnarray}
\begin{equation}
    N_\mathrm{e^-} \sim \mathrm{Binomial}(N_0, \langle N_\mathrm{e^-} \rangle / \langle  N_0 \rangle)
\end{equation}
and $ N_\mathrm{ph} =N_0-N_\mathrm{e^-}$.
%Traditionally, the Fano factor is defined to characterize the charge yield fluctuation in ionization detectors. 
%The binomial description is a good approximation when the difference in energy levels of the signal channels is ignored and charge recombination is significant, which applies to \NR\ in LAr. 
%Note that the concept of Fano factor only applies to cases where the stochastic process of charge or light generation and propagation does not depend on external parameters. For example, if the directional modulation of recombination is considered as signal fluctuation, the fluctuation would exceed the width described by the Fano factor. Here we write out the signal yield dependence on $\theta$ explicitly.

The \TPC\ signals S1 and S2 are measured in units of photo-electrons (PE) in the photosensor. The scintillation light collection efficiency and the photosensor quantum efficiency are less than unity, so that S1 has a gain $g_1 = \mathrm{S1}/N_\mathrm{ph} [\si{PE/ph}]$ less than one. Charges are amplified and converted to photons through the gas pocket electroluminescence, so that S2 has a gain $g_2 = \mathrm{S2}/N_\mathrm{e^-} [\si{PE/e^-}]$ greater than one. 

The stochastic processes of collection of the scintillation light can be described by a binomial distribution, using the gain $g_1$. For S2, the electroluminescence process is described by a Poisson distribution. The detector response also includes a position-dependent non-uniformity which could in principle be corrected in analysis. Practically, a small residual error will be present, which can be modeled by an additional Gaussian smearing of standard deviation 
$\sigma_ \mathrm{S1}^{*}$ and $\sigma_ \mathrm{S2}^{*}$ for S1 and S2, respectively. 
Approximating the S1 and S2 distributions with Gaussians, the total contribution from detector response is 
\begin{eqnarray}
    \mathrm{S1} &\sim& \mathrm{Gaussian}\left( \langle N_\mathrm{ph} \rangle g_1, \sqrt{\langle N_\mathrm{ph} \rangle g_1 (1-g_1) + \sigma_ \mathrm{S1}^{*2}}\right) \\
    \mathrm{S2} &\sim& \mathrm{Gaussian}\left( \langle N_\mathrm{e^-} \rangle g_2, \sqrt{\langle N_\mathrm{e^-} \rangle g_2 + \sigma_\mathrm{S2}^{*2}} \right).
\end{eqnarray}

In conclusion, the argon dual-phase \TPC\ response to a mono-energetic \NR\ follows the probability density function coming from the convolution of the detector and physical terms:
\begin{eqnarray}
    P(\mathrm{S1},\mathrm{S2}) & = &
    P_\mathrm{detector}(\mathrm{S1}/g_1,\mathrm{S2}/g_2;N_\mathrm{ph},N_{e^-}) \nonumber\\
    & & \otimes 
    P_\mathrm{NR}(N_\mathrm{ph},N_{e^-};E_r,R,\theta)  \nonumber\\
    & = & \frac{1}{2\pi\sigma_\mathrm{S1}\sigma_\mathrm{S2}/g_1 g_2} \nonumber\\
    & &
    e^{-\frac{(\mathrm{S1}/g_1 - N_\mathrm{ph})^2}{2(\sigma_\mathrm{S1}/g_1)^2}
    -\frac{(\mathrm{S2}/g_2 - N_{e^-})^2}{2(\sigma_\mathrm{S2}/g_2)^2}} \nonumber\\
    & & \otimes\frac{1}{2\pi\sqrt{F\langle N_\mathrm{ph}\rangle\langle N_{e^-}\rangle}} \nonumber\\
    & & e^{ -\frac{(N_{e^-}+N_\mathrm{ph}-\langle N_0\rangle)^2}{2F\langle N_0 \rangle}
    -\frac{(N_{e^-}\langle N_\mathrm{ph} \rangle - N_\mathrm{ph}\langle N_{e^-}\rangle)^2}
    {2\langle N_{e^-} \rangle\langle N_\mathrm{ph} \rangle\langle N_0\rangle}}. \nonumber\\
    \label{eq:TPCResponse}
\end{eqnarray}
Later in Sect.~\ref{sec:StatisticalAnalysis}, a likelihood function is evaluated from the \TPC\ data using this probability density function. An unbinned profile likelihood study is then performed to determine the confidence interval of the directionality parameter $R$.


\section{Experimental setup} \label{sec:DetectorLayout}
\section{Background and Problem Statement}
\label{sec:setup}
We consider the problem of an agent interacting with an SCM for $T$ rounds in order to maximize the value of a reward variable. We start by introducing SCMs, the soft intervention model used in this work, and then define the adversarial sequential decision-making problem we study. In the following, we denote with $[m]$ the set of integers $\{0, \dots, m\}$. \looseness-1

\paragraph{Structural Causal Models}
Our SCM is described by a tuple $\langle \G,  Y, \bX, \fs, \snoiserv \rangle$ of the following elements: $\G$ is a \emph{known} DAG; $Y$ is the reward variable; $\bX = {\{X_i\}_{i=0}^{m-1}}$ is a set of observed scalar random variables; the set $\fs = \{\fofi\}_{i=0}^m$ defines the \emph{unknown} functional relations between these variables; and $\snoiserv = \{\snoiserv_i \}_{i=0}^{m}$ is a set of independent noise variables with zero-mean and known distribution. % \looseness-1
 We use the notation $Y$ and $X_m$ interchangeably and assume the elements of $\bX$ are topologically ordered, i.e., $X_0$ is a root and $X_m$ is a leaf.  We denote with $\pa_i \subset \{0, \dots, m\}$ the indices of the parents of the $i$th node, and use the notation $\bZi = \{ X_j\}_{j \in \pa_i}$ for the parents this node. We sometimes use $X_i$ to refer to both the $i$th node and the $i$th random variable. \looseness-1\looseness-1

Each $X_i$ is generated according to the function $\fofi: \calZ_i \rightarrow \calX_i$, taking the parent nodes $\bZi$ of $X_i$ as input: $\si =\fofi(\zi) + \noisei$, where lowercase denotes a realization of the corresponding random variable. The reward is a scalar $x_m \in [0,1]$ while observation $X_i$ is defined over a compact set $\si \in \calX_i \subset \R$, and its parents are defined over $\calZ_i = \prod_{j \in pa_i} \calX_j$ for $i\in [m-1]$.\footnote{Here we consider scalar observations for ease of presentation, but we note that the methodology and analysis can be easily extended to vector observations as in \citet{sussex2022model}}  \looseness-1

\paragraph{Interventions}

\looseness -1 In our setup, an agent and an adversary both perform \emph{interventions} on the SCM~\footnote{Our framework allows for there to be potentially multiple adversaries, but since we consider everything from a single player's perspective, it is sufficient to combine all the other agents into a single adversary.}. 
We consider a soft intervention model \citep{eberhardt2007interventions} where interventions are parameterized by controllable \emph{action variables}. A simple example of a soft intervention is a shift intervention, where actions affect their outputs additively \citep{zhang2021matching}.

First, consider the agent and its action variables $\bm a = {\{ \ai\}_{i=0}^{m}}$. Each action $a_i$ is a real number chosen from some finite set. That is, the space $\calA_i $  of action $a_i$ is   $\calA_i \subset \R_{[0, 1]}$ where $\abs{\calA_i} = K_i$  for some $K_i \in \nN$. Let $\calA$ be the space of all actions $\bm a = {\{ \ai\}_{i=0}^{m}}$. 
% Let $\calA$ be the space of all actions $\bm a = {\{ \ai\}_{i=0}^{m}}$.
We represent the actions as additional nodes in $\G$ (see \cref{fig:overview}): $\ai$ is a parent of only $X_i$, and hence an additional input to $\fofi$. Since $\fofi$ is unknown, the agent does not know apriori the functional effect of $\ai$ on $X_i$. Not intervening on a node $X_i$ can be considered equivalent to selecting $\ai = 0$. For nodes that cannot be intervened on by our agent, we set $K_i = 1$ and do not include the action in diagrams, meaning that without loss of generality we consider the number of action variables to be equal to the number of nodes $m$.
\footnote{There may be constraints on the actions our agent can take. We refer the reader to \citet{sussex2022model} for how our setup can be extended to handle constraints.}

For the adversary we consider the same intervention model but denote their actions by $\a'$ with each $\ai'$ defined over $\calA_i' \subset \R_{[0, 1]}$ where $\abs{\calA_i'} = K_i'$ and $K_i'$ is not necessarily equal to $K_i$. 

According to the causal graph, actions $\a, \a'$ induce a realization of the graph nodes: 
\begin{align}
\label{eq:groud_truth}
& \si = \fofi(\zi, \ai, \ai') + \noisei, \ \ \forall i \in [m].
\end{align}
 
If an index $i$ corresponds to a root node, the parent vector $\zi$ denotes an empty vector, and the output of $\fofi$ only depends on the actions.

\looseness-1

\paragraph{Problem statement}
Over multiple rounds, the agent and adversary intervene simultaneously on the SCM, with known DAG $\calG$ and fixed but unknown functions $\fs = \{\fofi\}_{i=1}^m$ with $\fofi: \calZ_i \times \A_i \times \A_i' \rightarrow \calX_i$. \looseness-1
At round $t$ the agent selects actions $\at = \{\ait\}_{i=0}^m$ and obtains observations $\st = \{\sit\}_{i=0}^m$, where we add an additional subscript to denote the round of interaction. When obtaining observations, the agent also observes what actions the adversary chose $\at' = \{\ait'\}_{i=0}^m$.  We assume the adversary does not have the power to know $\at$ when selecting $\at'$, but only has access to the history of interactions until round $t$. The agent obtains a reward given by \looseness-1
\begin{align}
\label{eq:groud_truth_target}
& y_t = f_m(\bm z_{m, t}, a_{m, t}, a_{m, t}') + \noise_{m, t},
\end{align}
which implicitly depends on the whole action vector $\at$ and adversary actions $\at'$. 

The agent's goal is to select a sequence of actions that maximizes their cumulative expected reward $\sum_{t=1}^T 
r(\at, \at')$ where $r(\at, \at') = \E{y_t\mid \at, \at'}$ and expectations are taken over $\snoise$ unless otherwise stated. The challenge for the agent lies in not knowing a-priori neither the causal model (i.e., the functions $\fs = \{\fofi\}_{i=1}^m$), nor the sequence of adversarial actions $\{\at'\}_{t=1}^{\cdots}$.

\paragraph{Performance metric} 

After $T$ timesteps, we can measure the performance of the agent via the notion of regret:
\begin{align}
    R(T) = \max_{\a \in \A} \sum_{t=1}^T r(\a, \at') - \sum_{t=1}^T r(\at, \at'),
    \label{eq:regret}
\end{align}
\ie, the difference between the best cumulative expected reward obtainable by playing a single fixed action if the adversary's action sequence and $\fs$ were known in hindsight, and the agent's cumulative expected reward. We seek to design algorithms for the agent that are \emph{no-regret}, meaning that $R(T)/T \rightarrow 0$ as $T\rightarrow \infty$, for any sequence $\at'$. We emphasize that while we use the term `adversary', our regret notion encompasses all strategies that the adversary could use to select actions. This might include cooperative agents or mechanism non-stationarities. \looseness -1


 For simplicity, we consider only adversary actions observed after the agent chooses actions. Our methods can be extended to also consider adversary actions observed \emph{before} the agent chooses actions, i.e., a \textit{context}. This results in learning a policy that returns actions depending on the context, rather than just learning a fixed action. This extension is straightforward and we briefly discuss it in~\Cref{app:contextual}. \looseness-1

\textbf{Regularity assumptions} We consider standard smoothness assumptions for the unknown functions $\fofi:\mathcal{S} \rightarrow \X_i$ defined over a compact domain $\mathcal{S}$ \citep{srinivas10}. In particular, for each node $i \in [m]$, we assume that $\fofi(\cdot)$ belongs to a reproducing kernel Hilbert space (RKHS) $\mathcal{H}_{k_i}$, a space of smooth functions defined on $\calS = \calZ_i \times \calA_i \times \calA_i'$.
This means that $\fofil \in \mathcal{H}_{k_i}$ is induced by a kernel function $k_i: \calS \times  \calS \rightarrow \mathbb{R}$. 
We also assume that $k_i(s,s') \leq 1$ for every $s, s' \in \calS$\footnote{This is known as the bounded variance property, and it holds for many common kernels.}. Moreover, the RKHS norm of $\fofi(\cdot)$ is assumed to be bounded $\|\fofi\|_{k_i} \leq \mathcal{B}_i$ for some fixed constant $\mathcal{B}_i>0$.  Finally, to ensure the compactness of the domains $\Z_i$, we assume that the noise $\snoise$ is bounded, i.e., $\noisei \in \left[-1,1\right]^{d}$. \looseness-1


\section{The detectors} \label{sec:Detectors}
\subsection{The $\Delta E$/$E$ telescope} \label{sec:deltaee}
%% Si detectors
Neutrons directed towards the \TPC\  %, at $\theta_{tpc} = 22.3$\textdegree, 
are produced in association with \Be\ nuclei of energy 
$E_{Be} = 19.0$~MeV and emitted at angle $\theta_{Be} = 5.1$\textdegree. $^{7}$Be is detected by a dedicated $\Delta E$/$E$ 
telescope placed in the scattering chamber at a distance of \SI{46}{cm} from the CH$_2$ target. The telescope is made of two Si detectors manufactured by ORTEC, having thickness of 20~\textmu m and 1000~\textmu m, respectively; the $^{7}$Be loses about 7.6~MeV crossing the thinner stage and it is stopped in the thicker one. The detectors have a 100\% efficiency for light charged particles detection and energy resolution of  about 1\%. The telescope is collimated using an Al shield with a hole of 2~mm diameter.  For the fine tuning of the position, the telescope holder is mounted on a two axis remotely-controlled stepper motor which can operate in vacuum.
%stage is a 20~$\textmu$m thick Si detector 
%manufacture by ORTEC; it is followed by the $E$ stage, which is a 1000~$\textmu$m-thick Si detector by ORTEC 
The detectors are 
%biased at 12~V and 150~V respectively,
%according to the specifications from the manufacturer and are 
readout from a standard spectroscopic chain made by a pre-amplifier %(ORTEC 142~A/B) 
and a charge-sensitive amplifier, %(CAEN N568E), 
with 1~\textmu s  shaping time. 
%
%The $\Delta E$ detector is thin enough to be 
%crossed by \Be\ nuclei of kinetic energy in the range of our interest, while the thick $E$ detector can stop the %nuclei completely.

The combined measurement of $\Delta E$ and $E$ provides the discrimination in $Z$, which is necessary to distinguish the
interesting Be from the far more abundant elastically-scattered Li. 
%The two Si detector have a diameter of 7~mm and they are
%assembled on a common grounded holder inside the scattering chamber. The holder is mounted on a stepper %motor (precision of 0.15~\textmu m), 
%which can operate in vacuum and is remotely controlled. The stepper motor allows for the fine tuning of the position
%of the Si telescope. % such to match the position $(\theta_{Be}, \phi_{Be})$ of the telescope according to the actual position
%of the \TPC, $(\theta_{TPC}, \phi_{TPC})$ and also to accommodate for small variations in the beam energy.
%
%Due to the inverse kinematic configuration, there are two different solutions at the same angle
%$\theta_{Be} = 5.1$\textdegree, in which \Be\ have energy of 19.0~MeV (``low energy'') and 20.4~MeV %(``high energy''), respectively. 
%Neutrons in association with the ``low energy'' \Be\ are those traveling towards the \TPC\ ($\theta_{n} = %22.3$\textdegree, $E_n = 7.3$~MeV). The
%``high energy'' \Be\ is associated with neutrons of $E_n = 2.7$~MeV emitted at $\theta_n = 44$\textdegree: 
%these neutrons do not hit directly the \TPC, but can produce accidental coincidences due to scattering on the %floor or on the walls. 
%
%Direct neutrons
%can be effectively identified by selecting only the \Be\ events in the $\Delta E$/$E$ telescope compatible with the ``low energy''
%solution.
%
%
% Figure environment removed
% 
Fig.~\ref{fig:banana1} shows the $\Delta E$ vs. $E$ scatter plot, upon the irradiation of the CH$_2$ target with the $^7$Li beam.
The central, and most intense, band is created by Li ($Z=3$), mostly by elastic scattering on H and C. The uppermost band is due to Be ($Z=4$). 
As the reaction p($^{7}$Li,$^{7}$Be)n occurs in inverse kinematics, two different solutions at the same angle
$\theta_{Be} = 5.1$\textdegree\ are allowed, with \Be\ having energy of 19.0~MeV (``low energy'') and 20.4~MeV (``high energy''), respectively. 
Neutrons in association with the ``low energy'' \Be\ are those travelling towards the \TPC\ ($\theta_{n} = 22.3$\textdegree), with $E_n = 7.3$~MeV
kinetic energy. The
``high energy'' \Be\ is associated with neutrons of $E_n = 2.7$~MeV emitted at $\theta_n = 44$\textdegree: 
these neutrons do not hit directly the \TPC, but can contribute to accidental coincidences due to scattering on the floor or on the walls. 
In  Fig.~\ref{fig:banana1}  the \emph{loci} from the two $^7$Be solutions are visible and clearly separated; the population between them is due to the inelastic interaction p($^{7}$Li,$^{7}$Be*)n', which also emits a neutron. Because of the finite extension of the beam spot 
and of the beam angular divergence, neutrons associated with the $^{7}$Be* detected at $\theta_{Be}$ can still travel inside 
the \TPC\ and produce an interaction; they also contribute to the diffuse 
background, e.g. upon scattering on the walls or on the floor of the 
experimental area.

 In order to suppress the dominant contribution from \Li\ elastic scattering, the thresholds for the $\Delta E$ and $E$  detectors, shown in Fig.~\ref{fig:banana1} as dashed lines, are used during the data acquisition. Fig.~\ref{fig:banana2} displays the $\Delta E$ vs. $E$ scatter plot, acquired with the thresholds 
of Fig.~\ref{fig:banana1}, without (color) and with (dots) the requirement of coincidence with an event in the \TPC\ compatible with 
a neutron interaction and within a 200~ns gate. As expected, neutron events in the \TPC\ are mostly associated with a ``low-energy'' \Be\
nucleus detected by the Si telescope. The dashed red box represents the \Be\ selection cut used in the 
following analysis and described in Sect.~\ref{sec:EventSelection}. 
%with a few accidentals associated with \Li\ or with the other \Be\ \emph{loci}. 
%The inset of Fig.~\ref{fig:banana2} displays the distribution of the time difference $\Delta t$ between \TPC\ and Si telescope, for the events in 
%coincidence, which clearly shows the correlation.
%
% Figure environment removed
% 

\subsection{The Time Projection Chamber} \label{sec:2a}
The heart of the ReD system is the dual-phase Ar \TPC, whose detailed description and 
performance are reported in~\cite{Agnes:2021zyq}. It is a cubic volume of $5 \times 5 \times 6$ 
cm$^3$, delimited on the side walls by acrylic plates interleaved with 
specular reflector foils, and the top and bottom by two transparent acrylic windows. 
The top and bottom windows are coated with a thin transparent conductive layer (indium-tin oxide, ITO), so 
they can be given an electric potential and be operated as anode and cathode, respectively. The extraction grid is a stainless 
steel mesh, having 95\% optical transparency; it is located 10~mm below the anode window and it is kept electrically grounded.
All internal surfaces are coated with a wavelength shifter (tetraphenyl-butadiene, TPB): it converts the UV light emitted by Ar scintillation (128~nm) into visible
light, which better matches the sensitivity of typical photosensors. The lower part of the \TPC\ contains LAr:  
the liquid fills the entire volume between the cathode and the extraction grid, plus 3~mm above the grid. The gas pocket 
is produced by means of a heater and it occupies the 7-mm thick region between the liquid surface and the anode. 
%The emission of the UV scintillation light takes place through two different excited states of the Ar dimer, which exhibit largely different time constant (6 ns and 1.6~$\textmu$s, respectively). Since the relative population between the two excited dimers depends on the ionization density, the time profile of the S1 signal can be used to perform a very efficient discrimination between nuclear and electron recoils~\cite{Amaudruz:2016dq}. 

The \TPC\ electric fields which are set for this work are: drift field (\edrift) of 152~V/cm; extraction field (\eex) of 3.9~kV/cm; and electroluminescence field (\eel) of 5.9~kV/cm.
The maximum drift time is about 66 \textmu s: this is the time required for an electron produced at the
cathode to travel until the liquid surface. Due to a continuous recirculation loop of the liquid through a SAES getter,
the purity of argon is such that the electron life time before 
capture by electronegative impurities is $> 1$~ms, i.e. much longer than the 66-\textmu s maximum drift time~\cite{Agnes:2021zyq}. 
The extraction field is strong enough to give a 100\% extraction efficiency of the electrons from the liquid to the 
gas phase~\cite{Chepel:2012sj}.

After the UV photons from scintillation and electroluminescence of Ar are shifted to the visible range by the TPB coating, they 
can be detected by customised NUV-HD-Cryo Silicon PhotoMultipliers (\SiPMs) from Fondazione Bruno Kessler, which can be operated at cryogenic 
temperature~\cite{Gola:2019idb}. The \SiPMs\ are assembled in two $5 \times 5$~cm$^2$ tiles, each containing 24 devices of 
dimensions 11.7~mm$\times$7.0~mm and arranged in a $4 \times 6$ array.
The tiles are placed behind the top and bottom acrylic windows of the \TPC, providing a 30\% 
total optical coverage. As the position of the S2 event in the gas phase can be used to estimate the \xy\ coordinate of the 
original interaction point in the \TPC, the \SiPMs\ of top tile are readout in 22 channels for improved resolution: 20 \SiPM\
are readout individually, while 4 lateral \SiPMs\ are summed in pairs and grouped into two readout channels. 
%the 24 \SiPMs\ of the top tile are read out individually for improved resolution.
The \SiPMs\ of the bottom tiles are summed in groups of twelve, hence giving two readout channels.
Two custom-made Front-End Boards (FEB) 
supply power to the \SiPMs\ and amplify the output signals at cryogenic temperature. The \SiPMs\ are operated at \SI{+7}{V} of overvoltage 
with respect to the breakdown voltage. Due to the presence of resistors in the bias chain, the effective 
overvoltage of the \SiPMs\ gets smaller than the nominal \SI{+7}{V} when the bias current of the devices is high. 
This typically happens when the \SiPMs\ are exposed to a significant amount of light, e.g. 
due to the high interaction rate under beam irradiation, and causes a change in the \SiPM\ response 
(see Sect.~\ref{sec:Calibration}). 

More details about the cryogenic setup, the \TPC, the photosensors and the readout system can be found in~\cite{Agnes:2021zyq}.

\subsection{The neutron spectrometer} \label{sec:spectrometer}
The neutron spectrometer used in \ReD\ is made of seven 3-inch liquid scintillator (LSci) cells, individually read-out by
photomultipliers (PMTs). The assembly includes the liquid scintillator cell, a ETL-9821B PMT and the front-end electronics with the 
amplifier. The cells are filled with the EJ-309 liquid scintillator by Eljen Technologies, which features a very powerful
neutron-$\gamma$ discrimination based on the time pattern of the scintillation pulse. %Each PMT is biased individually 
%(with typical voltage of 1700--1900~V) in order to equalize the light response of the cells.

The neutron detection efficiency of the detectors was measured individually by using a $^{252}$Cf 
source~\cite{Stevanato2014,simophdthesis} and found to be about 28\% for the 7-MeV neutrons of interest 
for this work. %, assuming an energy threshold of E$_{th}$ = 100 keV$_{ee}$ (electron equivalent).
The calibration of the energy scale was performed with $\gamma$-ray sources 
($^{241}$Am, $^{137}$Cs and $^{22}$Na). Dedicated measurements taken with the annihilation $\gamma$-rays from the 
$^{22}$Na source confirmed the time resolution to be better than 1~ns (rms).

%The individual LScis are mounted on a
%custom holder system, which allows to place them over a cone of opening angle $\theta_{lsci} = %36.8$\textdegree, whose axis is the
%target-TPC line. In this way all detectors identify Ar recoils of the same energy, but at 
%different angles $\theta_r$ with respect to the \TPC\ drift field. 
%The placement of the scintillators   
The scintillators  identify Ar recoils of the same energy but different angles $\theta_r$ with respect to the \TPC\ drift field \edrift:  
$\theta_r$=180\textdegree 
(one LSci), 90\textdegree (two LScis, read out individually and labeled as
``90\textdegree $l$'' and ``90\textdegree $r$''), 40\textdegree (two LScis, summed) and 
20\textdegree (two LScis, summed). 



\subsection{Data acquisition and control infrastructure} 
The output signals from all of the detectors are sent to CAEN V1730 Flash ADC Waveform Digitizers and 
digitized with 14-bit resolution at a sampling rate of 500 MHz. In total a signal of 100~\textmu s (50k samples) is 
acquired at each trigger: this is sufficiently long to contain the S1 and S2 signals of the \TPC, given the maximum 
drift time of 66~\textmu s for events occurring close to the cathode. About 10\% of the digitization window is reserved 
for the pre-trigger. %, in order to allow for a precise estimate of the baseline. 
Two 16-channel CAEN V1730 boards were used for the measurement, synchronized with a daisy chain. 
%Due to the availability of only 32 readout channel, two pairs of top \SiPM\ channels and each half of the bottom 24 \SiPM\ channels are analogly summed, respectively. 
%it was not possible to readout all 24 \SiPM\ of the top tile individually. The signals from two pairs of corner \SiPMs\ of the top tile were hence summed, so to reduce the number of readout channels from the \TPC\ to 26 (22 top and 4 bottom), instead of the customary 28. 

The data acquisition (DAQ) software was built upon a package 
developed for the PADME experiment~\cite{Leonardi_2017} and based on the CAEN Digitizer Libraries. 
%Each digitizer has its own 
%read out process, controlled by a central Linux server, and writes asynchronously its own data files on disk. The 
%event building, i.e. matching the information from the same event split on the different data files, is performed 
%offline. 
The trigger logic is implemented by means of an external NIM logic module 
%(CO4020 by ORTEC) coupled with a LeCroy 428F digital fan-in/fan-out module 
as:
\begin{equation}
\texttt{SiTel} \land (\texttt{TPC} \lor \texttt{LScis})
\end{equation} 
where: \texttt{SiTel}, \texttt{TPC} and \texttt{LScis} are the trigger signals from the Si telescope, the \TPC\ and 
the neutron spectrometer, respectively.
The Si telescope trigger (\texttt{SiTel}) is built as the coincidence of the $\Delta E$ and $E$ 
detectors, with the thresholds displayed in Fig.~\ref{fig:banana1}.
The \TPC\ trigger (\texttt{TPC}) consists in the logical \texttt{AND} between the
two readout channels of the bottom tile within a coincidence gate of 200~ns, in order to suppress the dark rate~\cite{Agnes:2021zyq}.
The individual thresholds are set to approximately 2~PE. The \TPC\ is expected to trigger with 100\% efficiency on
S1 signals from the $E_r = 72$~keV NR events ($\textrm{S1} \sim 190$~PE) which are of interest for this work, although trigger
inefficiencies can possibly come from pile-up. 
Finally, the neutron spectrometer trigger (\texttt{LScis}) is produced by the logical \texttt{OR} of the five readout channels 
of the seven scintillators. The energy threshold of each cell is set to approximately 20~keV$_{ee}$ (electron equivalent), 
which corresponds to about 200~keV for a proton recoil~\cite{Stevanato2014}.
This is sufficient to have a nearly-100\% trigger efficiency for the neutron events of interest, as their
elastic scattering on the scintillator produces protons of average energy $\sim 3.6$~MeV, giving a 1.1~MeV$_{ee}$ 
signal~\cite{Stevanato2014}.


%all seven individual cells. 

All detectors and sensors of the setup can be operated and read out remotely by means of a slow control system made of a suite 
of LabVIEW-based~\cite{LabView} applications. All parameters under control (e.g. temperatures, bias voltages, leakage currents) 
are monitored continuously, and readings are stored in a database every 10~s. 


\section{Event processing and selection} \label{sec:DataAndResult}
\subsection{Event reconstruction and calibrations} \label{sec:Calibration}
%Move block to a separate file
\label{sec:calib}
In this section, we explore various notions of calibration \cite{dawid} for our model.
Calibration is a desirable property typically considered for classifiers, where predicted label probabilities should correspond to observed frequencies in the long run. For example, in weather forecasting, a well-calibrated predictor should have approximately 60\% of days with rain when it forecasts a 60\% chance of rain. This calibration requirement should hold for every predicted probability value output by the model.

Calibration has important fairness implications \cite{flores2016false, chouldechova2017fair, faircalib,multicalib} because a mis-calibrated predictor can lead to harmful actions in high-stakes settings, such as over-incarceration \cite{compassgender}. 
We show that varying the exact calibration requirements can substantially impact the model's accuracy loss when malicious noise is present in the training data.
%These results may be of independent interest to the calibration literature.

In this section, we align closely  with \cite{faircalib}, where the learner seeks to maximize accuracy while ensuring the classifier is perfectly calibrated.
Throughout this paper, we have focused on binary classifiers, so in Section \ref{subsec: predparity} we consider a related notion called Predictive Parity \cite{chouldechova2017fair, flores2016false}, before considering calibration notions for hypotheses with output in $[0,1]$.


\begin{comment}
To recall, as before the learning problem is 

\begin{align}
\min_{h\in\mathcal{H}} & ~~\mathbb{E}_{(X,Y,Z)\sim\mathcal{D}} \left[\mathbbm{1}(h(X) \neq Y)\right] ~~~~\ \\
\text{subject to} & |K(z)-K(z')| \leq \delta \qquad \forall z,z'\in \mathcal{Z}. \label{FairnessConstraint}
\end{align}
where $K: h \rightarrow \mathbbm{R}$ is some notion of calibration error for $h$ for group $z$ given the true labels $y$.

Typically calibration requirements are most natural for regression problems where predictor $h$ provides 
\emph{fine-grained} scores that corresponds to the underlying 
probability of some outcome. 
\end{comment}

%Throughout these sections we consider a property titled \emph{shared range}.
%Namely that even though the hypotheses are fine tuned for each group, these calibrated classifiers
%share the same range. 

\subsection{ Predictive Parity Lower Bound}
\label{subsec: predparity}
\begin{definition}[Predictive Parity \cite{chouldechova2017fair}]
A binary classifier $h: \mathcal{X} \rightarrow \{0,1\}$ satisfies predictive parity if for groups A and B, $P_{x \sim \DA}[h(x)=1]>0$, $P_{x \sim \DB}[h(x)=1]>0$ and
% \footnote{This mild technical remark is explained a in the Appendix} 
\[P_{(x,y) \sim \DA} [y=1 | h(x)=1] = P_{(x,y) \sim \DB} [y=1 | h(x)=1] \]
\end{definition}
In later sections we consider other calibration notions.
Here we consider an adversary who is attacking a learner constrained by equal predictive parity when group sizes are \emph{imbalanced}.

\begin{comment}
\begin{theorem}
With probability $1-(1-n)^{\alpha}$, there exists a FAIR ERM learner constrained learner with $O(\alpha)$
excess error. 
\end{theorem}
\end{comment}


\begin{theorem}
\label{thm:predparity}
    For a malicious adversary with corruption fraction $\alpha$, for Fair-ERM constrained to satisfy Predictive Parity, then there is no $h \in \PQ$ with less than $\Omega(1)$ error. 
\end{theorem}

The intuition for this statement is that imbalanced group size will allow the adversary to change the conditional mean substantially.
%In expectation, 
Below, we have an informal proof:
\begin{proof}[Proof Sketch:]
Suppose $P(x \in A)=1-\alpha$ and $P(x \in B)= \alpha$.
Observe that whatever the initial value of $P_{(x,y) \sim \DB} [y=1 | h(x)=1]$, the adversary can drive this value $P_{(x,y) \sim \mathcal\DBC} [y=1 | h(x)=1]$ to $50\%$ or below
by adding a duplicate copy of every natural example in group $B$ with the opposite label.

Since all of these points are information-theoretically indistinguishable, any hypothesis for group $B$ that makes any positive predictions incurs at least $50\%$ error and $1/2=P_{(x,y) \sim \mathcal\DBC} [y=1 | h(x)=1]$ calibration error.
%will have to do the same for
Any classifier for group $A$ satisfying Predictive Parity will have to do the same, yielding our $\Omega(1)$ error.
%The full proof in Section \ref{proof:predparity}.
%\begin{align*}
 %   blehp
%\end{align*}
%Observe that this attack 
\end{proof}

\subsection{Extension to Finer Grained Hypothesis Classes}
\label{subsec:calib}
A criticism of this lower bound might be that these calibration notions are very coarse and calibration is intended for fine-grained predictors, meaning those that have a finer grained discretization of the probabilities in $[0,1]$.
%and inappropriate for a binary classifier that in effect has two bins. 
%While a diversion from the rest of the paper where we tend to focus on binary classifiers, 
We now provide extensions for these lower bounds to real valued $\mathcal{H}$. 
Interestingly, we show if the learner can modify their `binning strategy', the learner can `decouple' the classifiers for the groups in the population and 
thus only suffer $O(\alpha)$ accuracy loss.
%Rather than being an algorithmic trick, this attack is fundamental as it seems to occur in the wild
%organically 
%as a type of red-lining. 
%This is because absent further constraints, calibration is a weak notion of mere self-consistency.
%Attacks of this type motivate more constrained notions of calibration like 
We adopt the version of calibration from \cite{faircalib}.
\begin{definition}[Calibration] \label{def:calib}
A classifier $h: \mathcal{X} \rightarrow [0,1]$ is Calibrated with respect to distribution $\mathcal{D}$ if 
\[\forall r \in [0,1], r= \mathbb{E}_{(x,y) \sim \mathcal{D} }[y=1| h(x)=r]\]
We will primarily focus on the discretized version of this definition where the classifier assigns every data point to one of $R$ bins, each with a corresponding label $r$, that partition $[0,1]$ dis-jointly. 
We will refer to this partition as $[R]$ with $r \in [R]$ corresponding to the prediction of a bin. 
\[ \forall r \in [R], r= \mathbb{E}_{(x,y) \sim \mathcal{D} }[y=1| h(x)=r] \]
\end{definition}
%Observe that nothing in this initial definition references groups. 
%The natural generalization to the above definition with 
Calibration as a fairness requirements with demographic groups requires that the classifier $h$ is calibrated with 
respect to the group distributions $\DA$ and $\DB$ simultaneously. 
In the sections that follow when we say `calibrated' this always refers to calibration with respect to $\DA$ and $\DB$. 

\begin{theorem}
\label{thm:calib}
    The learner wants to maximize accuracy subject to using a calibrated classifier, $h: \mathcal{X} \rightarrow [R]$ where $[R]$ is a partition of $[0,1]$ into bins.%^labelled bins with each label.
    
    The learner may modify the binning strategy after the adversary commits to a corruption strategy.
    Then an adversary with corruption fraction $\alpha$ can force at most $O(\alpha)$ excess accuracy loss over the non-corrupted optimal
    classifier. 
\end{theorem}

\newpage


\subsection{Parity Calibration}
Motivated by Theorem \ref{thm:calib}, we introduce a \emph{novel} fairness notion we call \emph{Parity Calibration}\footnote{We would note that this is initial discussion of a novel fairness constraint that arose naturally from considering Theorem \ref{thm:calib}. The idea is in some cases it might be more desirable to have a more sensitive calibration notion, hence we define Parity Calibration. This notion requires further study and analysis before deployment in sensitive contexts.}
% \footnote{This is a strong fairness constraint and should be thought of as a strong prior that while conditional label distribution $\mathcal{D}_{y|x}$ can be different among groups, how much of each group falls in each risk category is the same.}.
Informally, this notion is a generalization of Statistical/Demographic parity \cite{dwork2012fairness} for the case of classifier with 
$R$ bins partitioning $[0,1]$.
\begin{definition}[Parity Calibration]
\label{def:paritycalib}
Classifier $h: \mathcal{X} \rightarrow [R]$, where $[R]$ is a partition of $[0,1]$ into labelled bins, satisfies
\emph{Parity Calibration} if the classifier is Calibrated (Definition \ref{def:calib}) \emph{and}
\begin{align*}
\forall r \in [R], P_{(x,y) \sim \DA} [h(x)=r] =  P_{(x,y) \sim \DB} [h(x)=r]
\end{align*} 
\end{definition}


%These lower bounds still hold for stronger notions of calibration error, namely $K_1(h, \mathcal{D})$ and $K_2(h, \mathcal{D})$ 
%which are average calibration error for the $l_1$ and $l_2$ norms respectively.
\begin{theorem}
\label{thm:paritycalib} 
Consider a learner maximizing accuracy subject to satisfying Parity Calibration.
%$h: \mathcal{X} \rightarrow [R]$ where $[R]$ is a partition of $[0,1]$ into labelled bins with each label.
    The learner may modify the binning strategy after the adversary commits to a corruption strategy.
    Then an adversary with corruption fraction $\alpha$ can force $\Omega(1)$ excess accuracy loss over the non-corrupted optimal
    classifier. 
\end{theorem}

%We defer the proof of this statement to the appendix, but the intuition is a follows.

If the size of Group $B$ is $O(\alpha)$, then following a similar duplication strategy for Predictive Parity Theorem \ref{thm:predparity},
then the adversary can force Group $B$ to have an expected label of $50\%$, i.e.
$\forall x \in B, \mathbb{E}_{x \sim \DB}[y|x]=50\%$.
Thus, any classifier that is calibrated must assign all of Group $B$ to a $50\%$ bucket.
In order to satisfy \emph{Parity Calibration}, the classifier must do the same to Group $A$, yielding $50\%$ error on Group $A$.

% \kmsdelete{\subsection{Discussion}
% In general these results are consistent with the observed behavior of Calibration in other parts of theoretical computer science.
% If the learner/society really only cares about accuracy, then the insensitivity in Section \ref{subsec:calib} is somewhat of a feature, not a bug, 
% especially if the unreliability of data in Group $B$ optimistically could be transient?
% %However, advocates for stronger notions calibration would instead note that in \ref{thm: calib} 
% In general, when thinking about accuracy loss and malicious in the context of fair ERM; what is the appropriate amount of sensitivity in
% the learning process? We shall discuss this somewhat more in Section \ref{sec: discussion}.}
% %We would observe that are substantial 


\begin{comment}
\begin{definition}[Average Calibration Error]
The avergae calibration error of a predictor $h$ (with $h: \mathcal{X} \rightarrow [0,1]$) on distribution $\mathcal{D}$ is:
\[ K_1(f, \mathcal{D}) = \sum_{v \in R(h) } P_{(x,y) \sim D} [h(x)=v]|v-\mathbbm{E}_{(x,y) \sim \mathcal{D}}[y|h(x)=v] |\]
where $R(h)$ is the range of $h$. 

Similarly, the average squared calibration error is 
    \[ K_1(f, \mathcal{D}) = \sum_{v \in R(h) } P_{(x,y) \sim D} [h(x)=v]|(v-\mathbbm{E}_{(x,y) \sim \mathcal{D}}[y|h(x)=v])^2\]\end{definition}

\begin{theorem}
    
\end{theorem}
\end{comment}




\subsection{Selection of signal events} \label{sec:EventSelection}
%Run time, data selection. \\
%Determination of coincidence window, background fraction.\\

%Run time...
The events of interest are triple coincidences between a \Be\ nucleus 
detected in the $\Delta E$/$E$ telescope, and the two subsequent neutron scatterings
in the \TPC\ and in the neutron spectrometer.

A clean sample of signal events with the proper topology was selected through
a sequence of cuts.
%by requiring the time difference between them to be compatible with 
%the velocity of a few-MeV neutron, i.e. a few cm/ns. The \TPC\ and the LSci's both feature 
%neutron/$\gamma$ discrimination based on the time profile of the scintillation signals 
%(see sec.~ \ref{sec:Calibration}) so background can be further suppressed
%by requesting that the events are compatible with neutron-induced recoils. 
Firstly, unambiguous \TPC\ events were selected
according to same criteria of Sect.~\ref{sec:Calibration}:
events with only one S1 and only one S2, separated by a $t_\mathrm{drift}$ within
the range $[6,66]~\si{\us}$. An additional S2 ``echo'' 
signal, namely a secondary event due to photo-ionization of the 
cathode from the main S2 electroluminescence, is allowed in the time 
window $[67.5, 72]~\si{\us}$ after the primary S2.

Afterwards, events in the \TPC\ were selected by requesting that S1 is in time coincindence 
within a gate of 200~ns with the $\Delta E$/$E$ telescope and with one 
single LSci detector of the neutron spectrometer. In addition, 
neutron-induced (n,n') events in the neutron spectrometer were efficiently 
selected by PSD against the dominant $\gamma$-ray background.
The PSD based on the S1 signal of the \TPC\ was not applied. This was 
meant to avoid an undesirable S1-dependent selection efficiency, given 
the fact that the discrimination based on \fprompt\ gets progressively worse 
for S1 signals below \SI{100}{PE}.

%Events passing the basic quality cut are shown in fig.~\ref{fig:LAr:ReD_Cut_TPCquality}.

%% Figure environment removed

The \Be\ ion which accompanies the neutron traveling towards the \TPC\ was selected by a combined 
cut on $\Delta E$ and $E$, which is shown in Fig.~\ref{fig:banana2} (red 
dashed contour). The selection is not sensitive enough to resolve between 
the \Be\  emitted at the ground state in the
p($^{7}$Li,$^{7}$Be)n reaction and $^{7}$Be* in the first excited state coming from the 
p($^{7}$Li,$^{7}$Be*)n' reaction. Therefore, the neutron energy distribution consisted of two different mono-energetic 
components.

%The NR scattering cuts include cuts on the SiTel and the LScis. Recoiling \ISO{Be}{7} ion is selected as $\Delta E_\mathrm{SiTel}\in [600, 900] \si{AU}$ and $E_\mathrm{SiTel} < \SI{2900}{AU}$. The selection is not sensitive enough to distinguish between $\mathrm{Li} + \mathrm{H} \to \mathrm{n} + \mathrm{Be}$ and $\mathrm{Li} + \mathrm{H} \to \mathrm{n} + \mathrm{Be}^{*}$. Therefore the neutron energy distribution consists of two components.  
%Fig.~\ref{fig:LAr:ReD_Cut_banana_all} shows the signal from the SiTel and the accepted region of the cut. Fig.~\ref{fig:LAr:ReD_Cut_banana_coin} shows the comparison between the cut and the selected neutron events with TPC coincidence. The neutrons associated with $\mathrm{Be}^*$ arrive in the late TPC coincidence window. See the following sections for the definition of the coincidence windows.

%% Figure environment removed
%% Figure environment removed
%
The data sample was further selected by using the time-of-flight (ToF) of the \TPC\ with respect to the 
$\Delta E$/$E$ telescope, namely by keeping the events in which the 
delay between the telescope and the \TPC\ (see inset in Fig.~\ref{fig:banana2}) is consistent with the 
flight time of the neutrons.
%A further selection is performed requiring that the time-of-flight between the \TPC\ and the Si
%telescope  is consistent with the flight time of neutrons coupled with both \Be\ in ground state and first %excited state (see inset in Fig. \ref{fig:banana2}).
%Fig.~\ref{fig:LAr:ReD_TPCtiming} shows the dependence of the ToF for 
%\TPC\ events on $S1$. 
The coincidence window in ToF was set to be S1-dependent, in order 
to ensure a S1-independent selection efficiency. 
The boundaries of the coincidence window were defined as the 1\% 
and 99\% quantiles in each S1 slice of 10 PE, after the subtraction of 
the constant background due to random neutrons and $\gamma$-rays. 
%\todo{[LP I removed here the discussion about the middle band, as the Be* 
%and Be discrimination is mentioned later on the 1D distribution.]} 
%Another middle point is defined as the $+2\sigma$ position of the Gaussian fit to the center peak in each $S1$ slice. The region between the lower boundary and the middle point is the prompt coincidence window, which mainly consists of neutrons associated with $\mathrm{Be}$. The region from the middle point to the upper boundary is the late coincidence window, which consists of neutrons associated with $\mathrm{Be}$ and $\mathrm{Be}^*$. 
%The signal distributions in SiTel associated with the prompt and late coincidence windows are shown in fig.~\ref{fig:LAr:ReD_Cut_banana_coin}. 
%Events in both coincidence windows are accepted as signals. 
The random background contributes to about \SI{1}{\percent} of the events 
in the coincidence windows. 

%Fig.~\ref{fig:LAr:ReD_Cut_Coinci} shows the 
%S1 distribution of the \TPC\ events in coincidence with the telescope. 
%The ER band is visible above the main NR band.
%
The coincidence windows for the delay $\Delta t(\mathrm{LSci}-\mathrm{SiTel})$ 
between the LSci and the telescope in triple-coincidence events were set 
with very stringent cuts, so 
to guarantee the selection of pure single-scattering neutron interactions.  
The timing of the individual LScis was calibrated by using as a reference
the $\gamma$-rays produced in the \TPC\ by inelastic interactions (n,n'$\gamma$) and
then detected in the LScis: all $\gamma$ peaks were aligned 
to $\Delta t(\mathrm{LSci}-\mathrm{SiTel}) = 0$, as displayed in 
Fig.~\ref{fig:LAr:ReD_Cut_Eff}, where the effect of used cuts applied sequentially is shown. 
The single-scattered neutron events of 
interest form the peak around \SI{20}{ns}. The low-statistics peak at about 
\SI{25}{ns} comes from the lower-energy neutrons produced in the 
p($^{7}$Li,$^{7}$Be*)n' interactions, while the tails at longer times are 
mostly due to multi-scattered neutron background. Monte Carlo simulations
indicate that the hump around \SI{60}{\nano\second} is originated by
the neutrons associated with the ``high energy'' \Be, which reach the \TPC\ after 
scattering on the floor or other passive structure. The peaks around 
\SI{-35}{\nano\second} and \SI{-20}{\nano\second} are 
$\gamma$-rays emitted by p($^{7}$Li,$^{7}$Li*)p inelastic scattering. 
Gaussian fits to the peak around \SI{20}{ns} determined the position and 
width of the window, individually for each scintillator. As mentioned 
in Sect.\,\ref{sec:spectrometer}, the LSci channels 
which selected \NR\ events at $\theta_r=\SI{20}{\degree}$ and 
$\SI{40}{\degree}$ were each made from the analogue sum of the signals 
of two different detectors. Since the cable lengths for the two detectors 
at $\SI{20}{\degree}$ were not properly matched, this introduced a split 
in the timing: the  $\Delta t(\mathrm{LSci}-\mathrm{SiTel})$ distribution 
for the channel at \SI{20}{\degree} was hence fitted with a double Gaussian. 
The coincidence windows were defined according to the position $\mu$ and width $\sigma$
of the peaks from the Gaussian fits, as summarized in Table~\ref{tab:LSci_Timing} and they are used to select the triple coincidence events. The
coincidence windows were further extended by \SI{5}{ns} in order to include the slower
neutrons from p($^{7}$Li,$^{7}$Be*)n'. Side-bands were also defined to estimate the 
random coincidence rate in each channel, see Tab.~\ref{tab:LSci_Timing}. 
%% Figure environment removed
%The alignment between channel mapping \#3 and \#4 is good, no relative offset is applied to the change of mapping, and the same set of coincidence windows in each channel are used. 
 
%No cut on LSci signal charge is applied in the final selection.
\begin{table*}
    \centering
     \caption[Definition of the coincidence windows in the ToF $\Delta t(\mathrm{LSci}-\mathrm{SiTel})$ for each LSci channel.]{Coincidence and side-band windows in the ToF $\Delta t(\mathrm{LSci}-\mathrm{SiTel})$ for each LSci channel. $d$ is the total width of the coincidence window, $d=6 \sigma + 5$~ns.}
    \label{tab:LSci_Timing}
\begin{tabular}{c|c|c|c|c|c}
\toprule
Angle $\theta_r$ of the \TPC\ \NR  & 90\textdegree $l$ & 40\textdegree & 0\textdegree & 90\textdegree $r$ & 20\textdegree \\
\hline
%    \makecell{$\gamma$ peak \\ offset [\si{ns}]} & -3.0    & -4.5    & -6.5     & -2.5     & -2.0     & -6.5        \\
    Neutron peak  $\mu$ [\si{ns}]   & 19.75  & 19.44   & 19.51   & 20.09   & \makecell{$\mu_1=17.18$, $\mu_2=20.44$} \\
    Timing resolution  $\sigma$ [\si{ns}] & 1.12 & 1.12  & 1.50    & 1.25    & 1.17      \\
    \hline
    Coincidence window        & \multicolumn{4}{c|}{$[\mu-3\sigma,\mu+3\sigma+\SI{5}{ns}]$} & \makecell{$[\mu_1-3\sigma$, $\mu_2+3\sigma+\SI{5}{ns}]$} \\ 
    \hline
    Side-band window          & \multicolumn{5}{c}{$[-\SI{20}{ns}-20d,-\SI{20}{ns}]\cup[\SI{70}{ns},\SI{70}{ns}+20d]$}\\
\bottomrule
\end{tabular}
\end{table*}
%
The triple coincidence events eventually considered for the statistical analysis of 
Sect.~\ref{sec:StatisticalAnalysis} are those which pass the sequence of cuts displayed 
in Fig.~\ref{fig:LAr:ReD_Cut_Eff} and the additional selection in the $\Delta t(\mathrm{LSci}-\mathrm{SiTel})$ 
ToF from Table~\ref{tab:LSci_Timing}.

%%%MOVE THIS BLOCK TO SECTION 7 %%%
%Figure~\ref{fig:LAr:ReD_Cut_Coinci} shows  the S2 vs. S1 distribution of the \NR\ events in the \TPC\ which pass the selection procedure, namely
%the sequence of cuts displayed in Fig.~\ref{fig:LAr:ReD_Cut_Eff} and the selection in the $\Delta t(\mathrm{LSci}-\mathrm{SiTel})$ ToF from
%Table~\ref{tab:LSci_Timing}: the pink dots represent the events selected requiring the triple coincidence (\TPC, Si telescope and spectrometer)
%(pink dots); the colour-coded distribution includes the events in double coincidence, \TPC\ and telescope. 

%The double coincidence events 
%constitute a large sample of \TPC\ \NR\ events is all directions: they were hence
%used to constrain the nuisance model parameters in the global analysis of Sect.~\ref{sec:StatisticalAnalysis}.
%Since the triple coincidence events are a very small fraction of the double coincidences, the large sample of double coincidence 
%events was also used as the template for random coincidence background. 
%The white contour in Fig. \ref{fig:LAr:ReD_Cut_Coinci} shows the 
%range in the (S1,S2) plane used for the statistical analysis described in 
%Sect.~\ref{sec:StatisticalAnalysis}.
% Figure environment removed

%% Figure environment removed








%as required. Don't forget to give each section
%and subsection a unique label (see Sect.~\ref{sec:1}).
%\paragraph{Paragraph headings} Use paragraph headings as needed.

\section{Statistical analysis} \label{sec:StatisticalAnalysis}
The S2 vs. S1 distribution of the \NR\ events in the \TPC\ which pass the 
selection procedure of  Sect.~\ref{sec:EventSelection} is displayed in Figure~\ref{fig:LAr:ReD_Cut_Coinci}:  
the pink dots represent the events selected requiring the triple coincidence (\TPC, Si telescope and neutron 
spectrometer); 
the colour-coded distribution includes the events in double coincidence (\TPC\ and telescope). 
The triple coincidence sample contains about 650 \NR\ events with S1 above \SI{20}{PE}, which were collected 
during 10.7~live days of beam run. The double coincidence events constitute a large sample of about 
70000 \TPC\ \NR\ events in all directions: they were hence
used as a calibration data set to constrain the nuisance model parameters in the global fit below. Since the triple coincidence events are a 
small fraction of the double coincidences, the large sample of double coincidence events was also used 
as the template for random coincidence background. 

%In fact, the rate and the S1 distribution of background 
%events were evaluated either from data, by using the side bands in the time coincidence, or from 
%\texttt{Geant4}-based~\cite{Agostinelli:2003fg,Allison:2006cd,Allison:2016lfl} Monte Carlo simulations.


% Figure environment removed


The data samples were statistically
analysed in order to evaluate the best estimate of the directionality
parameter $\delta R = R-1$, which measures how much the shape of the
initial ionization charge cloud differs from a sphere. As the number
of events is relatively modest, an unbinned profile likelihood was applied. The
global likelihood $\mathcal{L}$ is written as a product of three likelihood terms:
\begin{eqnarray}\label{eq:LAr:ReD_likelihood_total}
    \mathcal{L}(\vect{X}\,|\,\delta R, \vect{\nu}) & = & \prod_{i=1}^{5} \mathcal{L}_{i}(\vect{X}_i\,|\,\delta R,\theta_{r}^{(i)},\vect{\nu}) \nonumber\\
    & &  \times \mathcal{L}_\mathrm{cali}(\vect{X}_\mathrm{cali}\,|\,\vect{\nu})
    \times \mathcal{L}_\mathrm{constraint}(\vect{\nu}),
\end{eqnarray}
%It is a product of three terms, each of which is a function of the observed 
%array $\vect{X}$ of events $X = (\mathrm{S1},\mathrm{S2})$. The product over $i$ refers to the five
%samples taken at the five angles $\theta_r^{(i)} = {0^\circ, 20^\circ, 40^\circ,
%90^\circ l, 90^\circ r}$ of Tab.~\ref{tab:LSci_Timing}, each containing the
%events $\vect{X}_i$; $\delta R$ is the parameter of interest (POI) and $\vect{\nu}$ the array of nuisance %parameters, listed in Tab. \ref{tab:parameters} and discussed below.
where the product over $i$ refers to the five
samples taken at the five angles $\theta_r^{(i)} = {0^\circ, 20^\circ, 40^\circ, 90^\circ l, 90^\circ r}$ of Tab.~\ref{tab:LSci_Timing}, each containing the observed
array of events $\vect{X}_i = (\mathrm{S1},\mathrm{S2})$; $\delta R$ is the parameter of interest (POI); $\vect{\nu}$ is the array of nuisance parameters; $\vect{X}_\mathrm{cali}$ is the array of calibration data set.
The POI is constrained in this work to $\delta R \ge 0$, as negative values of $\delta R$ are not
physically allowed by the recombination model~\cite{Cataudella:2017kcf}. 
% Notice: if there is only thermal diffusion, the initial cloud shape 
% will be spherical, delta_R=0. If the primary Ar ion track has a finite 
% size, it contributes to a positive delta_R. It is not physical to have a 
% track with a negative length.
The three  likelihood terms of Eq.~\ref{eq:LAr:ReD_likelihood_total} are described in detail below.

$\mathcal{L}_i$ is the extended likelihood of each sample of \NR\ events at the recoil angle $\theta_{r}^{(i)}$:
\begin{equation}
    \mathcal{L}_i = \mathrm{Poisson}(n_i|\hat{n_i})\prod_{X_j\in\vect{X}_i} \mathcal{P}_i(\mathrm{S1}_j,\mathrm{S2}_j; \delta R, \theta_{r}^{(i)}, \vect{\nu}) 
\end{equation}
where $n_i$ and $\hat{n_i}$ are the size of $\vect{X}_i$ and its mean, respectively, and $\mathcal{P}_i$ is the 
joint probability density function (PDF) of the events (S1,S2). 
%The leading term extends the likelihood, $n_i$ is the size of $\vect{X}_i$ and $\hat{n_i}$ is its mean. 
The PDF is made as the combination of three components, one for
signal and two from backgrounds:
\begin{eqnarray} \label{eq:eachpdf}
    \mathcal{P}_i(S1,S2)&=& (1-\lambda_{1i})(1-\lambda_2)F_\mathrm{sig}(E_r) \nonumber\\
    & &  \otimes P(\mathrm{S1},\mathrm{S2}; \delta R, \theta_{r}^{(i)}, \vect{\nu}, E_r) \nonumber\\             
    & & + [\lambda_{1i} F_\mathrm{bkg1}(E_r) + (1-\lambda_{1i})\lambda_2 F_\mathrm{bkg2}(E_r) ]\nonumber \\
    & & \otimes P(\mathrm{S1},\mathrm{S2}; \delta R, \bar{\theta}_r, \vect{\nu}, E_r).
\end{eqnarray}
%ER background are not included explicitly with eq.~\ref{eq:LAr:S1S2PDF_ER}, as the ER fraction is small and at low energies the ER energy response approaches NR. 
%Note that in this study, $f_l$ takes the form of the Lindhard model (eq.~\ref{eq:intro:f_n_Lindhard}), and $f_l$ takes the form of Birks (eq.~\ref{eq:intro:f_l}) with $k_e$ estimated from SCENE \cite{Cao:2014gns}. 
The first component is the energy spectrum for the signal $F_\mathrm{sig}(E_r)$, which
depends on the recoil energy $E_r$, convolved with the response function $P$ of the
\TPC\ to mono-energetic \NR\ events, as defined in Eq.~\ref{eq:TPCResponse}. The parameters $\lambda_{1i}$ are the fractions of random coincidences within 
each data sample: they were estimated from the data, using the counting rate
in the side-band in ToF and are listed in Tab.~\ref{tab:LAr:lambda_1}.
Similarly, $\lambda_2$ is the scaling factor for multi-scattering background, namely
the fraction of those events with respect to all \NR\ events in the coincidence
window.  The other
two components are the energy distributions of the backgrounds due to random
coincidences, $F_\mathrm{bkg1}(E_r)$, and to multiple neutron scattering, 
$F_\mathrm{bkg2}(E_r)$. They are also convolved with the response function $P$ of
the \TPC. 

In order to speed-up the computation of the response function $P$, the
Poisson and binomial distributions are approximated by Gaussian distributions, such
that the convolutions over $N_\mathrm{ph}$ and $N_{e^-}$ in Eq.~\ref{eq:TPCResponse} can
be evaluated analytically.
As the angular distribution for background events is approximately random, the
$\theta_r$ dependence of $f(\theta_r, R)$ is averaged out by using the equivalent
angle $\bar{\theta}_r$ calculated analytically for an isotropic distribution and
the functional dependence on the angle is approximated as
$\langle f(\theta_r, R)\rangle \sim f(\bar{\theta}_r, R)$.
%The only effect of this approximation is a
%global scaling factor: the systematic uncertainty induced on the parameter of interest $R$ is
%estimated to be well below 1\%. 
%In this case, the functional dependence on the angle is approximated
%as $\langle f(\theta_r, R)\rangle \sim f(\bar{\theta_r}, R)$. 
The factor $\lambda_2$ and the three energy spectra ($F_\mathrm{sig}$,
$F_\mathrm{bkg1}$, and $F_\mathrm{bkg2}$) were evaluated by means of a dedicated 
Monte Carlo simulation using the \texttt{Geant4}-based framework \texttt{g4ds}~\cite{Agostinelli:2003fg,Allison:2006cd,Allison:2016lfl,Agnes:2017grb}.
The events from the simulations
underwent the same sequence of selection cuts used for the
real data. The energy distributions derived by the Monte Carlo are displayed
in Fig.~\ref{fig:simspectra}. The three energy distributions were then analytically
parametrized in order to optimize the calculation of the CPU-intensive
PDF $\mathcal{P}_i$. $F_\mathrm{sig}$ consists of two Gaussian peaks corresponding
to the \NR\ induced by neutrons from p($^{7}$Li,$^{7}$Be)n and 
p($^{7}$Li,$^{7}$Be*)n'.
$F_\mathrm{bkg1}$ and $F_\mathrm{bkg2}$ were approximated by a double-exponential and a
single exponential, respectively, whose parameters were calculated by fits to
the Monte Carlo distributions.
%
%Mapping: LSci\_1 = 90l, LSci\_8 = 90r, LSci\_4-6= 20 deg, LSci\_3-7=40 deg
%LSci\_5 = 0 deg

\begin{table}
    \centering
    \caption{Fraction of random coincidence events, $\lambda_{1i}$, in the range
    S1$\,\in[120,400]\si{PE}$ and S2$\,\in[800,2800]\si{PE}$ in the five samples of
    triple-coincidence events at different $\theta_r$. Uncertainty is about 2\% for all 
    samples.}
    \label{tab:LAr:lambda_1}
    \begin{tabular}{c|c|c|c|c}
    \toprule
    \thead{0\textdegree} & \thead{20\textdegree} & \thead{40\textdegree} &  \thead{90\textdegree $l$} & \thead{90\textdegree $r$} \\
    \hline
    0.045 & 0.048 & 0.047 & 0.026 & 0.041 \\
%         \thead{90\textdegree $l$} & \thead{40\textdegree} & \thead{20\textdegree} & \thead{0\textdegree} & \thead{90\textdegree $r$} \\
%        \hline
%         0.026 & 0.047 & 0.048 & 0.045 & 0.041  \\
        \bottomrule
    \end{tabular}
\end{table}

% Figure environment removed


The factor $\mathcal{L}_\mathrm{cali}$ of the global likelihood of Eq.~\ref{eq:LAr:ReD_likelihood_total} is the constraint
term on the nuisance parameters and it depends on the events $\vect{X}_\mathrm{cali}$ in the calibration set (i.e. colour-coded histogram in Fig.~\ref{fig:LAr:ReD_Cut_Coinci}) 
%Calibration
%data consists of the large \NR\ sample of events in double coincidence between the \TPC\ and the telescope, displayed
%in Fig.~\ref{fig:LAr:ReD_Cut_Coinci} as a color-coded histogram. 
While the energy spectrum of the calibration events is 
a broad and featureless distribution, the joint distribution of the \NR\ band in the (S1,S2) plane can set a
strong constraint on the nuisance parameters. Since the fraction of signal events in the calibration sample is negligible,
the energy distribution is well approximated by the random background $F_\mathrm{bkg1}$. The calibration term is hence
written as:
\begin{equation}
\label{eq:L_cal}
    \mathcal{L}_\mathrm{cali} = \prod_{X_j\in\vect{X}_\mathrm{cali}} P(\mathrm{S1}_j,\mathrm{S2}_j; \delta R, \bar{\theta}_r, \vect{\nu}, E_r)\otimes F_\mathrm{bkg1}(E_r).
\end{equation}

In order to avoid any analysis bias, $\delta R$ should be decoupled from the nuisance parameters as much as possible. The
explicit occurrence of the POI $\delta R$ in Eq.~\ref{eq:L_cal} is due to the fact that the parameter $\xi_m$ in
the modified Thomas-Imel model in Eq.~\ref{eq:rec_direction} is dependent on $\delta R$ because of the track length. To
remove such undesirable degeneracy, the angular dependence term and the Thomas-Imel parameter were re-defined as 
\begin{equation}
f'(\theta_r,R) = f(\theta_r,R)/f(\bar{\theta}_r,R)
\end{equation}
and
\begin{equation}
\xi_m' = \xi_m/f(\bar{\theta}_r,R),
\end{equation}
respectively. In this way the angle-averaged position of the \NR\ band in calibration data does not depend on
$\delta R$ and the POI $\delta R$ is left as a pure representation of directionality. Furthermore, the degenerate
nuisance parameters were re-cast into a unique nuisance parameter $A=\xi_m'/(\mathcal{E}_d \cdot \langle N_\mathrm{i} \rangle)$, which represents the
recombination probability of one electron-ion pair. 

The last factor of the global likelihood,
$\mathcal{L}_\mathrm{contraint}(\vect{\nu})$, is the pull term for the nuisance parameters which were known 
by prior independent measurements. Those parameters are constrained by Gaussian terms
\begin{equation}
    \mathcal{L}_\mathrm{constraint}(\vect{\nu}) = \prod_i\frac{1}{\sqrt{2\pi}\sigma_{\nu_i}}\exp{-\frac{(\nu_i-\nu_i^0)^2}{2\sigma_{\nu_i}^2}}
\end{equation}
based on the previously-measured values $\nu_i^0$ of the parameters $\nu_i$ and on their corresponding uncertainties.

%The assigned errors in the prior term are large relative to $\mathcal{L}_\mathrm{cali}$ such that this term only rectifies the fitting. 
%The column ``Reference" in tab.~\ref{tab:LAr:parameters} shows the values of nuisance parameters as $\nu_i^0\pm\sigma_{\nu_i}$. Additional fixed parameters are listed with only the estimated value. 

\begin{table}
    \centering
    \caption[Parameter values in the signal model in ReD directionality analysis.]{List of the parameters used in the model. $\delta R$ is the parameter of interest, while all others are nuisance parameters, constrained by the calibration data and/or by a Gaussian pull term. The error bars are the standard deviation which is taken in the Gaussian pull terms. The parameters reported without uncertainties are fixed. The gains $g_1$ and $g_2$ come from the previous \TPC\ performance study~\cite{Agnes:2021zyq}. The S1 resolution of the \TPC\ of Eq.~\ref{eq:TPCResponse} is parametrized as $\sigma_{\mathrm{S1}}^2 = \mathrm{S1}/[\si{PE}] + {\sigma^*_{\mathrm{S1}}}^{2} $, namely by the combination of the statistical term and of an extra contribution. The same is done for the S2 resolution.}
    \label{tab:parameters}
    \begin{tabular}{c|c|c}
    \toprule
         & \thead{Constraint}   & \thead{Comment} \\
        \hline
        $\delta R$  & -        & Parameter of interest \\
        \hline
       % $\mu_-$     &  \SI{500(10)}{cm^2/V/s}  & -   & Electron mobility at \SI{150}{V/m}\\
       % $\alpha$    &                   & -   & Recombination coefficient\\
       % $\sigma(E_r)$    &                   & -    & Ionization site size   \\
        %$A$    &   \SI{3.95(200)e-2}{1/e^-}      & No    & $A = \alpha/(2\pi E\sigma^2\mu_- f(\langle\phi\rangle,R))$\\
        $A$    &   \makecell{$0.04\pm0.01$\\ $[\si{1/e^-}]$ }    
%        & $A = \alpha/[2\pi\mathcal{E}_d\sigma^2 \mu_- f(\bar{\theta}_r,R)]$\\
        & $A = e/[2\pi\epsilon_r\epsilon_0\mathcal{E}_d\sigma^2 \mu_- f(\bar{\theta}_r,R)]$\\
        \hline
               $k_e$       & $2.8$          
        & Electronic quenching coefficient~\cite{Mei:2008ca}\\
\hline
$W_{ph}$         & \makecell{$19.5$\\ $[\si{eV}]$}       & \makecell{Energy  for scintillation \\photon production~\cite{Doke:2002oab}}\\
\hline
        $N_\mathrm{ex}/N_\mathrm{i}$ & $0.2\sim2$  
        & \makecell{Excitation to ionization ratio.\\ Energy dependence as in \cite{Cao:2015ks}} \\
        \hline
        $g_1$       & \makecell{$0.196\pm0.020$\\ $[\si{PE/ph}]$}      & S1 signal yield \\
        \hline
        $g_2$       & \makecell{$20.5\pm2.5$\\ $[\si{PE/e^-}]$}       & S2 signal yield \\
        \hline
        $\sigma^*_{\mathrm{S1}}/\mathrm{S1}$ &  $0.003\pm0.05$    
        & \makecell{S1 detector resolution\\ in addition to $\sqrt{\mathrm{S1}}$} \\
        \hline
        $\sigma^*_{\mathrm{S2}}/\mathrm{S2}$ &  $0.001\pm0.05$    
        & \makecell{S2 detector resolution\\ in addition to $\sqrt{\mathrm{S2}}$} \\
        \hline
        $\lambda_1$ &  Table~\ref{tab:LAr:lambda_1}      & Fraction of random coincidence \\
        \hline
        $\lambda_2$ &   0.16        & \makecell{Ratio of multi-scattering to \\all NR in coincidence windows}\\
        \bottomrule
    \end{tabular}
\end{table}


As a summary, the parameters and their reference values are summarized in Tab.~\ref{tab:parameters}.
%$A$ is estimated by manually tuning its value to align the NR band to data, an arbitrarily large uncertainty is assigned so that the prior does not restrict the fitting result of $A$. 
The recombination probability $A$ depends on $\sigma$, the size of the ionization cluster of 
Eq.~\ref{eq:clouddist}, which is dominated by
the electron diffusion during thermalization. Due to their high mobility and long thermalization time,
electrons diffuse for a few \textmu m in LAr \cite{Wojcik:2003ja,mozumder1995free}. It is found
that $A=0.04/e^-$, which corresponds to $\sigma=\SI{1.8}{\micro\meter}$, was an appropriate initialization parameter
for the likelihood fit. The ratio $N_\mathrm{ex}/N_\mathrm{i}$ was treated as a function of recoil energy, according to
the indications by SCENE~\cite{Cao:2015ks}. The \TPC\ gains $g_1$ and $g_2$ were estimated according to the \TPC\
characterization in~\cite{Agnes:2021zyq}, and were treated as nuisance parameters in order to accommodate for
possible variations in the \TPC\ performance. The parameters $W_{ph}$, $k_e$, $\lambda_1$ and $\lambda_2$ were fixed
in order to limit the degeneracies in the fit: their effect on the POI is minor and is accounted
below as a systematic uncertainty. %Similarly to \NexNi, $k_e$ is a parameter which affects the nuclear recoil
%band shape, but not directly the directionality. 

%
%\subsection{Simulation and the energy spectra of signal and backgrounds}
%\label{ch:LAr:ReD_simulation}
%$F_\mathrm{sig}(E_r)$, $F_\mathrm{bkg1}(E_r)$, and $F_\mathrm{bkg2}(E_r)$ should be determined from the simulation result. Three configurations are studied in the simulations, the reaction to $\mathrm{Be}$ ground state (gs) with a \SI{3.5}{mm} diameter collimator before the target, the reaction to $\mathrm{Be}$ ground state without the collimator, and the reaction to the first excited state $\mathrm{Be}^*$ without the collimator. They are shown accordingly from left to right in fig.~\ref{fig:LAr:ReD_Sim_Direc}. The top row shows the energy versus direction distribution of the recoiled argon nuclei of triple coincidence events. The blobs around \SI{70}{keV} are single-scatter events, while the broad distribution over all angles consists of multi-scatter events that also generate triple coincidences by chance. The bottom row shows the timing spectra of the single-scatter and multi-scatter events. Similar to the data, the $\Delta T(\mathrm{LSci}-\mathrm{SiTel})$ is offset such that the $\gamma$ peak in the $\mathrm{Be}$ configuration centers at $t=0$. The LSci coincidence windows are slightly different in each channel, roughly speaking, they are \SI{-3}{ns} to \SI{9}{ns} around the peaks at \SI{20}{ns}. 
%The relative rates of single-scatter (black) to multi-scatter (red) events in the coincidence windows of the three configurations are the same. Multi-scatter events make up for \SI{39}{\percent} of the triple coincidence events. Note that an additional simulation of detector responses and $S1$ range selection is required to determine $\lambda_2$. 
%
%%% Figure environment removed
%%
%%% Figure environment removed
%%
%%% Figure environment removed
%
%Fig.~\ref{fig:LAr:ReD_Sim_Spec} shows the energy spectra from the simulation. The peaks consist of single-scatter events. The solid black line and dotted black line correspond to single-scatter events from neutrons associated with $\mathrm{Be}$ and $\mathrm{Be}^*$, respectively. The two spectra are fitted with two Gaussian distributions. The ratio of $\mathrm{Be}^*$ events to $\mathrm{Be}$ is determined from the timing spectrum of events in LSci\_0 at $4.2^\circ$ without the TPC coincidence requirement. As shown in fig.~\ref{fig:LAr:ReD_LSci_0_BeRatio}, the fraction of $\mathrm{Be}^*$ events is \SI{19.6}{\percent}. Approximately, the interaction probabilities of neutrons at the two energies in the TPC are the same, and the TPC detection efficiency to the two types of events are the same. Then we have the estimation of the signal energy spectrum 
%\begin{eqnarray}
%    F_\mathrm{sig}(E_r) &=& m \frac{1}{\sqrt{2\pi}\sigma_{E1}}\exp\left(-\frac{(E_r-E_1)^2}{2\sigma_{E1}^2}\right)\nonumber \\
%    & & + (1-m)\frac{1}{\sqrt{2\pi}\sigma_{E2}}\exp\left(-\frac{(E_r-E_2)^2}{2\sigma_{E2}^2}\right)
%\end{eqnarray}
%where $m=0.196$, $E_1=\SI{63.5}{keV}$, $\sigma_{E1}=\SI{6.8}{keV}$, $E_2=\SI{72.5}{keV}$ and $\sigma_{E2}=\SI{7.8}{keV}$.
%The backgrounds sitting at the bottom of fig.~\ref{fig:LAr:ReD_Sim_Spec} correspond to $F_\mathrm{bkg1}(E_r)$ and $F_\mathrm{bkg2}(E_r)$. The solid red line consists of events with TPC-SiTel coincidence. It is the same distribution as the random coincidence background, $F_\mathrm{bkg1}(E_r)$. A two-components exponential could describe the shape,
%\begin{equation}
%    F_\mathrm{bkg1}(E_r) = l\frac{1}{\tau_1}\exp(-E_r/\tau_1) + (1-l)\frac{1}{\tau_2}\exp(-E_r/\tau_2)
%\end{equation}
%where $\tau_1=\SI{25.0}{keV}$, $\tau_2=\SI{114}{keV}$, and the fraction $l=0.17$. 
%%check the difference from the 2 sim settings, check the sideband shape vs double coincidenc shape.
%The dashed red line consists of multi-scatter events inside the triple coincidence window. They are selected according to the top row of fig.~\ref{fig:LAr:ReD_Sim_Direc}. The shape is described with an exponential
%\begin{equation}
%     F_\mathrm{bkg2}(E_r) = \frac{1}{\tau_3}\exp(-E_r/\tau_3) 
%\end{equation}
%where $\tau_3=\SI{42.7}{keV}$. The multi-scatter background spectrum in terms of $S1$ can be estimated by convolving the point response function in eq.~\ref{eq:LAr:S1S2PDF_NR} with $F_\mathrm{bkg2}(E_r)$. The fraction of $S1\in[100,350]\si{PE}$ is \SI{36}{\percent}. Therefore, we have $\lambda_2=0.39\times0.36/(1-0.39+0.39\times0.36)=0.19$. Now, all the ingredients are ready, and we shall perform the likelihood fittings. 
%

%\subsection{Likelihood fit and limit calculation}
%\label{ch:LAr:ReD_limit}

Finally, experimental data of Fig.~\ref{fig:LAr:ReD_Cut_Coinci} (calibration 
and five triple-coincidence samples) were fitted against the model of
Eq.~\ref{eq:LAr:ReD_likelihood_total}. In order to make the fit stable the 
fit region in the (S1,S2) plane was selected in order to include only the 
\NR\ band, with S1$\,\in[120,400]\si{PE}$, as represented by the white contour in 
Fig.~\ref{fig:LAr:ReD_Cut_Coinci}. The S1 range corresponds to NR energies between
approximately 35 and 150~keV, and hence comfortably includes the expected NR signal at
$\sim 72$~keV. The low-S1 edge $S1 > 120~\si{PE}$ was set in order to avoid any inefficiencies in
the event reconstruction and selection. The center of the \NR\ band
was empirically parametrized with the function S2$\,/[\si{PE}] = 455\ln(\mathrm{S1}/[\si{PE}])-535$
and the cut was set as $\pm500~\si{PE}$ in S2. The fit region globally contains 
529 triple coincidence and 42340 calibration events.

%
% Figure environment removed

\begin{table}
    \centering
    \caption{Best-fit of the parameters and correlation coefficients between the nuisance parameters 
and the POI $\delta R$.}
    \label{tab:Fit_result}
    \begin{tabular}{c|c|c}
    \toprule
 \thead{Parameter} & \thead{Value}  & \thead{Correlation with $\delta R$} \\
 \hline
 $\delta R$ & $0.037\pm0.027$ & -  \\
 $A\,[\si{1/e^-}]$ & $(4.01\pm0.06)\times10^{-2}$ &-0.014  \\
 $g_1\,[\si{PE/ph}]$ & $0.204\pm0.002$ & 0.013  \\
 $g_2\,[\si{PE/e^-}]$& $20.1\pm0.2$ & -0.009  \\
 $\sigma^*_{\mathrm{S1}}$/S1  & $0.017\pm0.003$ & -0.012  \\
 $\sigma^*_{\mathrm{S2}}$/S2  & $0.0002\pm0.0060$ & 0.026  \\
 \bottomrule
    \end{tabular}
\end{table}


%As the limit is a single sided upper limit, the appropriate likelihood ratio $\lambda$ shall be defined as~\cite{Cowan2011}
%\begin{equation}
%    \lambda(\delta R) = 
%        \begin{cases}
%            \frac{\mathcal{L}(\delta R, \hat{\hat{\vect{\nu}}}(\delta R))} 
%            {\mathcal{L}(\hat{\delta R}, \hat{\vect{\nu}})} & 
%            \hat{\delta R} \geq 0, \\
%            \frac{\mathcal{L}(\delta R, \hat{\hat{\vect{\nu}}})} {\mathcal{L}(0,\hat{\hat{\vect{\nu}}}(0))} & 
%            \hat{\delta R} < 0
%        \end{cases}
%\end{equation}
%and the test statistic $q$ is
%\begin{equation}
%    q (\delta R) = 
%    \begin{cases}
%        -2\ln \frac{\mathcal{L}(\delta R, \hat{\hat{\vect{\nu}}}(\delta R))} 
%            {\mathcal{L}(0, \hat{\hat{\vect{\nu}}})} &
%            \hat{\delta R} < 0, \\     
%        -2\ln \frac{\mathcal{L}(\delta R, \hat{\hat{\vect{\nu}}}(\delta R))} 
%            {\mathcal{L}(\hat{\delta R}, \hat{\vect{\nu}})} &
%            0 \leq \hat{\delta R} \leq \delta R, \\
%        0 & \delta R < \hat{\delta R}
%    \end{cases}
%\end{equation}
%where the $\hat{\hat{\vect{\nu}}}(\delta R)$ is the estimation of nuisance parameters with the POI is fixed to $\delta R$. Toy MCs are generated to estimate the distribution of $q$, $f(q(\delta R)|H)$, under the null hypothesis $H_{s+b} : \delta R = \delta R_\mathrm{test} > 0$ and the alternate hypothesis $H_b : \delta R =0$. The p-value is defined as $CL_s = CL_{s+b} / CL_b$, where $CL_{s+b}$ is the integral of $f(q(\delta R)|H_{s+b})$ from the observed $q$ from data to infinite and $CL_{b}$ is the integral of $f(q(\delta R)|H_{b})$ from the observed $q$ from data to infinite. The $\delta R$ scan for p-value is done with the \texttt{StandardHypoTestInvDemo.h} script from \texttt{RooStats}. Results are shown in fig.~\ref{fig:LAr:ReD_cdf_p}. The test statistic is sampled with toy MC 500 times for $H_{s+b}$ and 250 times for $H_{b}$. The accumulative distributions to the right side are shown. Solid lines indicate the distribution of the null hypothesis, dotted lines indicate the alternate hypothesis. Solid and open circles indicate the $CL_{s+b}$ and $CL_b$ of the data. A limit of \SI{90}{\percent} confidence level can be drawn at $\delta R = 0.058$. 
%
%% Figure environment removed


The fit result is shown in Fig.~\ref{fig:ReD_Fit_result} and reported in Tab.~\ref{tab:Fit_result}.
The positions of the signal peak in both S1 and S2 (middle and bottom rows of Fig.~\ref{fig:ReD_Fit_result}) are mutually 
consistent among the five samples at different $\theta_r$. The best-fit for the POI is 
$\delta R = 0.037\pm0.027$, which is less than $2\sigma$ away from a null result; 
the uncertainty on $\delta R$ is largely driven by statistics. 
%, as indicated by the weak correlation with the other nuisance parameters. 
The upper limit of 
$\delta R$ is calculated by a toy Monte Carlo approach, in order to guarantee the correct coverage: it results to be
$\delta R < 0.072$ at 90\% CL. 
The best-fits of the nuisance parameters are in good agreement with the central values of their 
estimates used for the constraints.
%indicating that the \NR\ response model is able to provide a good description of the data. 
In particular, the smallness of the best-fit for the parameters 
$\sigma^*_{\mathrm{S1}}$/S1 and $\sigma^*_{\mathrm{S2}}$/S2, which are the extra (non-statistical) contributions to the 
experimental resolution in S1 and S2, demonstrates that the spatial inhomogeneities of the detector response 
were properly corrected. Furthermore, the proper convergence and the absence of a significant bias for all
fit parameters, notably including the POI $\delta R$, were checked by running a dedicated set of toy Monte Carlo
simulations.

The uncertainties on $\delta R$ related to the nuisance parameters are automatically accounted in 
the fit. All other systematic uncertainties on $\delta R$, e.g. those related to the values of $W_{ph}$, $k_e$,
$\lambda_1$ and $\lambda_2$, to the spectral shapes $F_\mathrm{sig}$, $F_\mathrm{bkg1}$ and $F_\mathrm{bkg2}$, and to
the approximation of  $\bar{\theta}_r$ from isotropic distribution, are globally evaluated to be an order
of magnitude smaller than the statistical term and are hence neglected in this work. 





\section{Discussion} \label{sec:discussion}
\section{Discussion}
\label{sec: discussion}
\kmsdelete{In this work} We study \kmsreplace{Fairness-Aware PAC learning}{Fair-ERM} in the malicious noise model, and  in some cases allow 
the learner to maintain optimal overall accuracy despite the signal in Group $B$ being almost entirely washed out.
%when we allow learners to use the
%$\PQ$ randomized expansion of the hypothesis class $\mathcal{H}$
In particular we show that different fairness constraints have fundamentally different behavior in the presence of Malicious Noise, in terms of the amount of accuracy loss that a given level of Malicious Noise could cause a fairness-constrained learner to incur. 
The key to achieving our results, which are more optimistic than those in \cite{lampert}, is allowing for improper learners using the (P,Q)-randomized expansions of the given class $\mathcal{H}$.
%We \kmsreplace{present a picture of the}{prove upper and lower bounds on}
%accuracy loss for a range of fairness notions, given \kmsreplace{this simple randomization step.}{learning over $\PQ$.
%In general our results indicate Fair-ERM (given learning over $\PQ$) is more robust than claimed in \cite{lampert}.
The type of smoothness we create by using $\PQ$ seems to be a natural property that is likely shared by many natural hypothesis classes.

Fairness notions are motivated as a response to learned disparities when there is \kmsdelete{data corruption or} systemic error affecting \kmsdelete{the data for}
one group. 
Fairness notions are supposed to mitigate this by ruling out classifiers that have worse performance on a sub-group. 
This can peg both classifiers at a lower level of performance \kmsdelete{(e.g that the lower subgroup)} in order to \emph{motivate} \cite{hardt16} improving the data collection or labelling process to obtain more reliable performance. 
%So in \kmsreplace{some}{a} sense, sensitivity of the fairness notion to poor sub-group performance caused by malicious noise is the \textit{point} of fairness constraints! 
However, it also desirable that fairness constraints perform gracefully when subject to Malicious Noise because fairness constraints will be used in contexts where the data is unreliable and noisy and this might not be known to the learner.
This tension, exposed by our work, motivates 
%a revisiting of fairness notions from first principles approach and trying to axiomatize the 
%desired properties of a fairness intervention a la cryptography and privacy. \footnote{Work in multi-calibration \cite{multicalib} is a viable direction for this problem but it is unclear how 
%that and related notions behave with unreliable data. }
on going work studying the sensitivity level of fairness constraints. 
%If we we are to take a view, if a classifier is deployed 


\section{Conclusions} \label{sec:conclusions}
\section{Conclusion and Future Work}
In this work, I design corruption-robust algorithms for the Lipschitz contextual search problem. I present the \emph{agnostic checking} technique and demonstrate its effectiveness in designing corruption-robust algorithms. There are several open problems for future research. First, in the algorithm I propose for pricing loss, the schedule for agnostic checks is fixed upfront. Can the learner design an adaptive checking schedule for the pricing loss? Second, this work assumes the learner has knowledge of the Lipschitz constant $L$. Can the learner design efficient no-regret algorithms without knowledge of $L$? 


\appendix
%\section{Likelihood}
%Details of the likelihood statistical analysis

\section{A data-driven analysis approach}

\section*{Appendix B: Unique Monotone Equilibrium}
In this appendix, we show that the game has a unique equilibrium when we restrict attention to monotone strategies. Although the result is standard in the literature, it is included for completeness.

Let $s_L: \Theta \to \Action_L$ be the leader's strategy. 
The leader is said to follow a \emph{monotone strategy} if her strategy takes the form:
\begin{equation*}
    s_L(\theta) = \begin{cases}
        \invest & \text{if $\theta > \thetazerohat$} \\
        \notinvest & \text{if $\theta \leq \thetazerohat$}
    \end{cases}.
\end{equation*}
A strategy for any follower $j$ is a mapping $s_j: X_j \times \Action_L \to \Action_j$. Follower $j$'s strategy is monotone if
\begin{equation*}
    s_j(x_j, h) = \begin{cases}
        \invest & \text{if $x_j > \xhhat$} \\
        \notinvest & \text{if $x_j \leq \xhhat$}
    \end{cases}.
\end{equation*}
A \emph{monotone equilibrium} is a symmetric perfect Bayesian equilibrium in monotone strategies with thresholds $(\theta_L^*, x_\invest^*, x_\notinvest^*)$.




\begin{lemma} \label{lemma_monotone_b}
There exists a monotone equilibrium with thresholds $\thetazerostar = 0$, $\xistar = -\infty$, and $\xnstar = \infty$.
\end{lemma}

\begin{proof}
    Fix a follower type $x$. Suppose that the leader uses threshold $\thetazerostar = 0$ and other followers use thresholds $\xistar = -\infty$ and $\xnstar = \infty$. If the leader exerts effort, then type $x$'s payoff yields
\begin{equation*}
    \pi_F^\invest(x; \thetazerostar, \xistar) = \EE_{\theta \sim \Psi^\invest(\cdot; \, x, 0)} \left[\theta\right] > 0.
\end{equation*}
This means that all types $x$ will exert effort under history $h = \invest$. Thus, follower $j$'s best response is a monotone strategy with threshold $\xistar = -\infty$. In contrast, if the leader does not exert effort, then the payoff for type $x$ is
\begin{equation*}
    \pi_F^\notinvest(x; \thetazerostar, \xnstar) = \EE_{\theta \sim \Psi^\notinvest(\cdot; \,x, 0)} [\theta] - 1< 0.
\end{equation*}
Thus, under history, $h = \notinvest$, follower $j$ will best respond by using a monotone strategy with threshold $\xnstar = \infty$.

Consider now type $\theta$ of the leader. Since all followers will invest if they see the leader invests, investing generates a payoff of $\theta$ for type $\theta$. Therefore, type $\theta$ invests if and only if $\theta > 0$. In other words, the leader will best respond by choosing threshold $\thetazerostar = 0$. The proof is complete.
\end{proof}


\begin{proposition}
There is no monotone equilibrium other than the one given in Lemma \ref{lemma_monotone_b}.

\end{proposition}
\begin{proof}
By way of contradiction, suppose that $\thetazerostar$ and $\xistar$ are the equilibrium thresholds. Then they must solve the indifference conditions
\begin{equation} \label{app_b_leader_eq}
    \pi_L(\thetazerostar; \xistar) = \thetazerostar - \Phi\left(\frac{\xistar - \thetazerostar}{\sigma_F}\right) = 0 \tag{B.1}
\end{equation}
and
\begin{equation} \label{app_b_follower_eq}
    \pi_F^\invest(x_\invest^*; \theta_L^*, x_\invest^*) =  \EE_{\theta \sim \Psi^\invest(\cdot; \,\xistar, \thetazerostar)} \left[ \theta - \frac{n-1}{n}\Phi\left(\frac{\xistar - \thetazerostar}{\sigma_F}\right) \right] = 0. \tag{B.2}
\end{equation}
By Equations (\ref{rank_belief}) and (\ref{eq_eqm_follower}) we can write (\ref{app_b_follower_eq}) as
\begin{equation} \label{app_b_follower_eq_2}
    \xistar + \sigma_F \lambda\left(\frac{\xistar - \thetazerostar}{\sigma_F}\right) = \frac{n-1}{2n}\Phi\left(\frac{\xistar - \thetazerostar}{\sigma_F}\right). \tag{B.3}
\end{equation}
Subtracting Equation (\ref{app_b_leader_eq}) from Equation (\ref{app_b_follower_eq_2}) yields
\begin{equation} \label{app_eq_diff}
    \xistar - \thetazerostar + \sigma_F \lambda\left( \frac{\xistar - \thetazerostar}{\sigma_F} \right) = - \frac{n+1}{2n} \Phi\left(\frac{\xistar - \thetazerostar}{\sigma_F}\right). \tag{B.4}
\end{equation}
Note that $x + \lambda(x)$ is increasing in $x$ with $\lim_{x \to -\infty} x + \lambda(x) = 0$, and hence $x + \lambda(x) > 0$ for all $x$. This implies that the left-hand side of (\ref{app_eq_diff}) is positive. But since the right-hand side of (\ref{app_eq_diff}) is negative, this leads to a contradiction.
\end{proof}


\begin{acknowledgements}
\subsection*{Acknowledgements}

\noindent
USA {\textendash} U.S. National Science Foundation-Office of Polar Programs,
U.S. National Science Foundation-Physics Division,
U.S. National Science Foundation-EPSCoR,
Wisconsin Alumni Research Foundation,
Center for High Throughput Computing (CHTC) at the University of Wisconsin{\textendash}Madison,
Open Science Grid (OSG),
Extreme Science and Engineering Discovery Environment (XSEDE),
Frontera computing project at the Texas Advanced Computing Center,
U.S. Department of Energy-National Energy Research Scientific Computing Center,
Particle astrophysics research computing center at the University of Maryland,
Institute for Cyber-Enabled Research at Michigan State University,
and Astroparticle physics computational facility at Marquette University;
Belgium {\textendash} Funds for Scientific Research (FRS-FNRS and FWO),
FWO Odysseus and Big Science programmes,
and Belgian Federal Science Policy Office (Belspo);
Germany {\textendash} Bundesministerium f{\"u}r Bildung und Forschung (BMBF),
Deutsche Forschungsgemeinschaft (DFG),
Helmholtz Alliance for Astroparticle Physics (HAP),
Initiative and Networking Fund of the Helmholtz Association,
Deutsches Elektronen Synchrotron (DESY),
and High Performance Computing cluster of the RWTH Aachen;
Sweden {\textendash} Swedish Research Council,
Swedish Polar Research Secretariat,
Swedish National Infrastructure for Computing (SNIC),
and Knut and Alice Wallenberg Foundation;
Australia {\textendash} Australian Research Council;
Canada {\textendash} Natural Sciences and Engineering Research Council of Canada,
Calcul Qu{\'e}bec, Compute Ontario, Canada Foundation for Innovation, WestGrid, and Compute Canada;
Denmark {\textendash} Villum Fonden and Carlsberg Foundation;
New Zealand {\textendash} Marsden Fund;
Japan {\textendash} Japan Society for Promotion of Science (JSPS)
and Institute for Global Prominent Research (IGPR) of Chiba University;
Korea {\textendash} National Research Foundation of Korea (NRF);
Switzerland {\textendash} Swiss National Science Foundation (SNSF);
United Kingdom {\textendash} Department of Physics, University of Oxford.
\end{acknowledgements}

% BibTeX users please use one of
%\bibliographystyle{spbasic}      % basic style, author-year citations
%\bibliographystyle{spmpsci}      % mathematics and physical sciences
\bibliographystyle{spphys}       % APS-like style for physics
\bibliography{beampaper}   % name your BibTeX data base

%% Non-BibTeX users please use
%\begin{thebibliography}{}
%%
%% and use \bibitem to create references. Consult the Instructions
%% for authors for reference list style.
%%
%\bibitem{RefJ}
%% Format for Journal Reference
%Author, Article title, Journal, Volume, page numbers (year)
%% Format for books
%\bibitem{RefB}
%Author, Book title, page numbers. Publisher, place (year)
%% etc
%\end{thebibliography}

\author{
  G.~Angloher\thanksref{addrMPI}\and
  S.~Banik\thanksref{addrHEPHY,addrAI}\and
  G.~Benato\thanksref{addrLNGS}\and
  A.~Bento\thanksref{addrMPI,addrCoimbra}\and 
  A.~Bertolini\thanksref{addrMPI}\and 
  R.~Breier\thanksref{addrBratislava}\and
  C.~Bucci\thanksref{addrLNGS}\and 
  J.~Burkhart\thanksref{e1,addrHEPHY}\and
  L.~Canonica\thanksref{addrMPI}\and 
  A.~D'Addabbo\thanksref{addrLNGS}\and
  S.~Di~Lorenzo\thanksref{addrLNGS}\and
  L.~Einfalt\thanksref{addrHEPHY,addrAI}\and
  A.~Erb\thanksref{addrTUM,addrWMI}\and
  F.~v.~Feilitzsch\thanksref{addrTUM}\and 
  S.~Fichtinger\thanksref{addrHEPHY}\and
  D.~Fuchs\thanksref{addrMPI}\and 
  A.~Garai\thanksref{addrMPI}\and 
  V.M.~Ghete\thanksref{addrHEPHY}\and
  P.~Gorla\thanksref{addrLNGS}\and
  P.V.~Guillaumon\thanksref{addrLNGS}\and
  S.~Gupta\thanksref{addrHEPHY}\and 
  D.~Hauff\thanksref{addrMPI}\and 
  M.~Ješkovsk\'y\thanksref{addrBratislava}\and
  J.~Jochum\thanksref{addrTUE}\and
  M.~Kaznacheeva\thanksref{addrTUM}\and
  A.~Kinast\thanksref{addrTUM}\and
  H.~Kluck\thanksref{e2,addrHEPHY}\and
  H.~Kraus\thanksref{addrOxford}\and 
  S.~Kuckuk\thanksref{addrTUE}\and
  A.~Langenk\"amper\thanksref{addrMPI}\and 
  M.~Mancuso\thanksref{addrMPI}\and
  L.~Marini\thanksref{addrLNGS,addrGSSI}\and
  L.~Meyer\thanksref{addrTUE}\and
  V.~Mokina\thanksref{e3,addrHEPHY}\and
  A.~Nilima\thanksref{addrMPI}\and 
  M.~Olmi\thanksref{addrLNGS}\and
  T.~Ortmann\thanksref{addrTUM}\and
  C.~Pagliarone\thanksref{addrLNGS,addrCASS}\and
  L.~Pattavina\thanksref{addrLNGS,addrTUM}\and
  F.~Petricca\thanksref{addrMPI}\and 
  W.~Potzel\thanksref{addrTUM}\and 
  P.~Povinec\thanksref{addrBratislava}\and
  F.~Pr\"obst\thanksref{addrMPI}\and
  F.~Pucci\thanksref{addrMPI}\and 
  F.~Reindl\thanksref{addrHEPHY,addrAI} \and
  J.~Rothe\thanksref{addrTUM}\and 
  K.~Sch\"affner\thanksref{addrMPI}\and 
  J.~Schieck\thanksref{addrHEPHY,addrAI}\and 
  D.~Schmiedmayer\thanksref{addrHEPHY,addrAI}\and
  S.~Sch\"onert\thanksref{addrTUM}\and 
  C.~Schwertner\thanksref{addrHEPHY,addrAI}\and
  M.~Stahlberg\thanksref{addrMPI}\and 
  L.~Stodolsky\thanksref{addrMPI}\and 
  C.~Strandhagen\thanksref{addrTUE}\and
  R.~Strauss\thanksref{addrTUM}\and
  I.~Usherov\thanksref{addrTUE}\and
  F.~Wagner\thanksref{addrHEPHY}\and 
  M.~Willers\thanksref{addrTUM}\and 
  V.~Zema\thanksref{addrMPI}
(CRESST Collaboration),
\\
    F.~Ferella\thanksref{addrLNGS,addrCop}\and
    M.~Laubenstein\thanksref{addrLNGS}\and
%    V.B.~Mikhailik\thanksref{addrDiam}\and
 %   T.~Mroz\thanksref{addrPoland}\and 
    S.~Nisi\thanksref{addrLNGS} %\and 
 %   G.~Zuzel\thanksref{addrPoland}% etc
}

\institute
{Max-Planck-Institut f\"ur Physik, D-80805 M\"unchen, Germany \label{addrMPI} \and
Institut f\"ur Hochenergiephysik der \"Osterreichischen Akademie der Wissenschaften, A-1050 Wien, Austria\label{addrHEPHY} \and
Atominstitut, Technische Universit\"at Wien, A-1020 Wien, Austria \label{addrAI} \and
INFN, Laboratori Nazionali del Gran Sasso, I-67100 Assergi, Italy \label{addrLNGS} \and
Comenius University, Faculty of Mathematics, Physics and Informatics, 84248 Bratislava, Slovakia \label{addrBratislava} \and
Physik-Department, Technische Universit\"at M\"unchen, D-85747 Garching, Germany \label{addrTUM} \and
Eberhard-Karls-Universit\"at T\"ubingen, D-72076 T\"ubingen, Germany \label{addrTUE} \and
Department of Physics, University of Oxford, Oxford OX1 3RH, United Kingdom \label{addrOxford} \and
%
%Jagiellonian University, M. Smoluchowski Institute Of Physics, 30-348 Cracow, Poland\label{addrPoland} \and
%
also at: LIBPhys-UC, Departamento de Fisica, Universidade de Coimbra, P3004 516 Coimbra, Portugal \label{addrCoimbra} \and
also at: Walther-Mei\ss ner-Institut f\"ur Tieftemperaturforschung, D-85748 Garching, Germany \label{addrWMI} \and
%also at: Excellence Cluster Origins, D-85748 Garching, Germany \label{addrClu} \and
also at: GSSI-Gran Sasso Science Institute, I-67100 L'Aquila, Italy \label{addrGSSI} \and
also at: Dipartimento di Ingegneria Civile e Meccanica, Universitá degli Studi di Cassino e del Lazio Meridionale, I-03043 Cassino, Italy\label{addrCASS} \and
also at: Department of Physical and Chemical Sciences, University of l'Aquila, via Vetoio (COPPITO 1-2), I-67100 L'Aquila, Italy\label{addrCop} 
}
\thankstext{e1}{Corresponding author: jens.burkhart@oeaw.ac.at}
\thankstext{e2}{Corresponding author: holger.kluck@oeaw.ac.at}
\thankstext{e3}{Corresponding author: valentyna.mokina@oeaw.ac.at}


\end{document}
% end of file template.tex

