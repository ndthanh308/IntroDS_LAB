\documentclass{article}

\PassOptionsToPackage{numbers, compress}{natbib}
% if you need to pass options to natbib, use, e.g.:
%     \PassOptionsToPackage{numbers, compress}{natbib}
% before loading neurips_2023


% ready for submission
% \usepackage{neurips_2023}


% to compile a preprint version, e.g., for submission to arXiv, add add the
% [preprint] option:
\usepackage[preprint]{neurips_2023}


% to compile a camera-ready version, add the [final] option, e.g.:
%     \usepackage[final]{neurips_2023}


% to avoid loading the natbib package, add option nonatbib:
%    \usepackage[nonatbib]{neurips_2023}


\usepackage[utf8]{inputenc} % allow utf-8 input
\usepackage[T1]{fontenc}    % use 8-bit T1 fonts
\usepackage{hyperref}       % hyperlinks
\usepackage{url}            % simple URL typesetting
\usepackage{booktabs}       % professional-quality tables
\usepackage{amsfonts}       % blackboard math symbols
\usepackage{nicefrac}       % compact symbols for 1/2, etc.
\usepackage{microtype}      % microtypography
\usepackage{xcolor}         % colors

\usepackage{graphicx}
\usepackage{caption}
\usepackage{subcaption}
\usepackage{float}
\usepackage{multirow}
\usepackage{multicol}
\usepackage{sidecap}

\usepackage{amsmath}
\usepackage{amssymb}
\usepackage{amsthm}


\newcommand{\probust}{\texorpdfstring{$p^\mathrm{robust}_{\sigma}$}{probust}}

\newcommand{\pmc}{\texorpdfstring{$p^\mathrm{mc}_{\sigma}$}{pmc}}

\newcommand{\ptaylor}{\texorpdfstring{$p^\mathrm{taylor}_{\sigma}$}{ptaylor}}

\newcommand{\ptaylormvs}{\texorpdfstring{$p^\mathrm{taylor\_mvs}_{\sigma}$}{ptaylormvs}}

\newcommand{\pmmse}{\texorpdfstring{$p^\mathrm{mmse}_{\sigma}$}{pmmse}}

\newcommand{\pmmsemvs}{\texorpdfstring{$p^\mathrm{mmse\_mvs}_{\sigma}$}{pmmsemvs}}

\newcommand{\psoftmax}{\texorpdfstring{$p^\mathrm{softmax}_{T}$}{psoftmax}}

\newcommand{\probustwsigma}[1]{\texorpdfstring{$p^\mathrm{robust}_{\sigma= {#1}}$}{probust}}

\newcommand{\pmmsewsigma}[1]{\texorpdfstring{$p^\mathrm{mmse}_{\sigma = {#1}}$}{pmmse}}

\newcommand{\suraj}[1]{{\color{cyan} Suraj: #1}}


\title{Efficient Estimation of Local Robustness of \\Machine Learning Models}


% The \author macro works with any number of authors. There are two commands
% used to separate the names and addresses of multiple authors: \And and \AND.
%
% Using \And between authors leaves it to LaTeX to determine where to break the
% lines. Using \AND forces a line break at that point. So, if LaTeX puts 3 of 4
% authors names on the first line, and the last on the second line, try using
% \AND instead of \And before the third author name.



\author{%
  Tessa Han \\
  Harvard University\\
  Cambridge, MA \\
  \texttt{than@g.harvard.edu}
  \And
  Suraj Srinivas \\
  Harvard University\\
  Cambridge, MA \\
  \texttt{ssrinivas@seas.harvard.edu}
  \And
  Himabindu Lakkaraju \\
  Harvard University\\
  Cambridge, MA \\
  \texttt{hlakkaraju@hbs.edu}
}


\begin{document}


\maketitle


\begin{abstract}
    Machine learning models often need to be robust to noisy input data. Real-world noise (such as measurement noise) is often random and the effect of such noise on model predictions is captured by a model’s local robustness, i.e., the consistency of model predictions in a local region around an input. Local robustness is therefore an important characterization of real-world model behavior and can be useful for debugging models and establishing user trust. However, the naïve approach to computing local robustness based on Monte-Carlo sampling is statistically inefficient, especially for high-dimensional data, leading to prohibitive computational costs for large-scale applications. In this work, we develop the first analytical estimators to efficiently compute local robustness of multi-class discriminative models. These estimators linearize models in the local region around an input and compute the model’s local robustness using the multivariate Normal cumulative distribution function. Through the derivation of these estimators, we show how local robustness is connected to such concepts as randomized smoothing and softmax probability. In addition, we show empirically that these estimators efficiently compute the local robustness of standard deep learning models and demonstrate these estimators’ usefulness for various tasks involving local robustness, such as measuring robustness bias and identifying examples that are vulnerable to noise perturbation in a dataset. To our knowledge, this work is the first to investigate local robustness in a multi-class setting and develop efficient analytical estimators for local robustness. In doing so, this work not only advances the conceptual understanding of local robustness, but also makes its computation practical, enabling the use of local robustness in critical downstream applications.
    % Local robustness is therefore an important characterization of real-world model behavior and can be useful in model debugging and establishing user trust.
    % The best estimator works by linearizing a randomized smoothed model, which arises naturally as the optimal minimum-mean-squared-error linear estimate.
\end{abstract}


% \begin{abstract}
%   Machine learning models are often required to make predictions for noisy input data. While adversarial robustness indicates the existence (or not) of an adversary in a local region around the input, real world noise is rarely adversarial and is often random, for example from measurement noise. Average local robustness, i.e., the fraction of inputs in a local regions in which the model provides correct prediction, is thus a more comprehensive characterization of real-world model behaviour, and can be useful in debugging machine learning models and establishing user trust.

%   %Therefore, measuring the local robustness of a model, i.e., the fraction of inputs in a local volume on which a model provides correct predictions, is critical to characterizing and debugging model behavior for establishing user trust.
  
%   However, the naive approach of computing local robustness via explicitly sampling noisy inputs in a local region is statistically inefficient especially for high-dimensional data, leading to prohibitive computational costs for large-scale applications. In this paper, we present analytical approaches to efficiently compute local robustness of multi-class discriminative models without costly sampling. Our estimators work by linearizing non-linear models, and computing the local robustness of the resulting linear models via the multivariate normal CDF. Our best estimator  works by linearizing a randomized smoothed model, which arises naturally as the optimal minimum-mean-squared-error linear estimate.    
  
%   We show empirically that our estimators are accurate for computing local robustness of standard deep learning models and demonstrate its usefulness for various tasks involving average local robustness, such as measuring per-class robustness bias and capturing ambiguous and noisy dataset examples.
% \end{abstract}


The meteoric advancement of machine learning and artificial intelligence technologies has enabled the construction of neural networks that effectively emulate the complex computations of the human brain. These deep learning models have found utility in a wide range of applications, such as computer vision, natural language processing, autonomous driving, and more. With the growing complexity and sophistication of these neural network models, the computational requirements, particularly for 32-bit operations, have exponentially increased. This heightened computational demand necessitates the exploration of more efficient alternatives, such as 16-bit operations.

However, the shift to 16-bit operations is riddled with challenges. A common standpoint within the research community argues that 16-bit operations are not ideally suited for neural network computations. This belief is mainly attributable to concerns related to numerical instability during the backpropagation phase, especially when popular optimizers like Adam are employed. This instability, more pronounced during the optimizer-mediated backpropagation process rather than forward propagation, can negatively impact the performance of 16-bit operations and compromise the functioning of the neural network model. Current optimizers predominantly operate on 32-bit precision. If these are deployed in a 16-bit environment without appropriate hyperparameter fine-tuning, the neural network models encounter difficulties during learning. This issue is particularly evident in backward propagation, which heavily relies on the optimizer. Confronted with these challenges, the objective of this research is to conduct an exhaustive investigation into the feasibility and implementation of 16-bit operations for neural network training. We propose and evaluate innovative strategies aimed at reducing the numerical instability encountered during the backpropagation phase under 16-bit environments. A significant focus of this paper is also dedicated to exploring the future possibilities of developing 16-bit based optimizers. One of the fundamental aims of this research is to adapt key optimizers such as Adam to prevent numerical instability, thereby facilitating efficient 16-bit computations. These newly enhanced optimizers are designed to not only address the issue of numerical instability but also leverage the computational advantages offered by 16-bit operations, all without compromising the overall performance of the neural network models. Through this research, our intention goes beyond improving the efficiency of neural network training; we also strive to validate the use of 16-bit operations as a dependable and efficient computational methodology in the domain of deep learning. We anticipate that our research will contribute to a shift in the prevalent perceptions about 16-bit operations and will foster further innovation in the field. Ultimately, we hope our findings will pave the way for a new era in deep learning research characterized by efficient, high-performance neural network models.
\section{Related Work}

\subsection{Remote Collaboration}
\subsubsection{Mixed Reality Remote Collaboration}
Recent advances in mixed reality technologies have enabled immersive remote collaboration that was not possible with traditional desktop interfaces. 
Prior research has explored various approaches for immersive telepresence, such as holographic teleportation (e.g., \textit{Holoportation}~\cite{orts2016holoportation}, \textit{Virtual Makerspaces}~\cite{radu2021virtual}, \textit{Loki}~\cite{thoravi2019loki}), virtual avatars (e.g., \textit{CollaboVR}~\cite{he2020collabovr}, \textit{Mini-Me}~\cite{piumsomboon2018mini}, \textit{Shoulder of Giant}~\cite{piumsomboon2019shoulder}, \textit{ARTEMIS}~\cite{gasques2021artemis}), and projected video stream (e.g., \textit{Room2Room}~\cite{pejsa2016room2room}, \textit{3D-Board}~\cite{zillner20143d}).
These systems allow remote users to be spatially co-located in the same shared space, which greatly enhances collaborative experiences~\cite{bai2020user, cao2020exploratory}. 
For example, by showing virtual hands and bodies in 3D space, the local users can more easily understand the intention of the remote users for various physical tasks such as block assembly~\cite{zhang2022real}, origami~\cite{kim2019evaluating, kim2020combination}, mechanical tasks~\cite{oyama2021augmented, oyama2021integrating}, and physiotherapy education~\cite{faridan2023chameleoncontrol}.
However, current holographic telepresence lacks the physical embodiment of the remote user, which significantly reduces the sense of co-presence~\cite{lee2018physical}. This limitation also constraints rich physical affordances which we naturally employ in co-located physical collaboration~\cite{leithinger2014physical, siu2018investigating}.

\subsubsection{Robotic Telepresence}
To address this limitation, past research has explored robotic telepresence that aims to physically embody remote users by adding a robotic body to a 2D video screen (e.g., \textit{MeBot}~\cite{adalgeirsson2010mebot},  \textit{RemoteCode}~\cite{sakashita2022remotecode}) or by replicating the remote user with a humanoid or non-humanoid robot (e.g., \textit{TELESAR V}~\cite{fernando2012design}, \textit{Telenoid}~\cite{ogawa2011exploring}, \textit{You as a Puppet}~\cite{sakashita2017you}, \textit{GestureMan}~\cite{kuzuoka2000gestureman}, \textit{Geminoid}~\cite{sakamoto2007android}).
The robotic telepresence can greatly enhance user engagement by enabling physical interactions such as  gestures~\cite{adalgeirsson2010mebot} and body movement~\cite{nakanishi2011zoom, rae2014bodies, lee2011now}.
For example, mobile robots allow remote users to move freely around a table to interact with local users and objects for remote education (e.g., \textit{RobotAR}~\cite{villanueva2021robotar}, \textit{ASTEROIDS}~\cite{li2022asteroids}).
Beyond a screen-based representation, \textit{VROOM}~\cite{jones2020vroom, jones2021belonging} overlays a holographic avatar on a telepresence robot that enriches non-verbal communication such as gestures or eye-contact. 
% However, robotic telepresence embodies remote users themselves, while our focus is on embodying remote users through synchronized actuated environments, as we discuss next.

\subsubsection{Physical Telepresence}
An alternative approach to adding physical embodiment to remote users is using \textit{synchronized distributed physical objects}~\cite{brave1998tangible}, rather than embodying users themselves with robotic telepresence. 
Such an approach was originally explored through \textit{InTouch}~\cite{brave1997intouch}, \textit{ComTouch}~\cite{chang2002comtouch}, and \textit{PsyBench}~\cite{brave1998tangible}, in which synchronized tangible tokens embody the remote user's motion and behavior.
This idea has evolved into a concept of \textit{physical telepresence}~\cite{leithinger2014physical}, which synchronizes physical shape rendering with the remote users' visual appearance.
For instance, Leithinger et al.~\cite{leithinger2014physical} uses a shape-changing display~\cite{follmer2013inform} to physically render a remote user's hand and surrounding objects with screen-based visual feedback. 
Recent works have also expanded this concept by combining a virtual avatar with a motorized X-Y plotter to actuate a single token (e.g., \textit{Physical-Virtual Table}~\cite{lee2018physical}).
However, the existing approach using shape displays lacks deployability due to the dedicated hardware requirement, and X-Y plotters lack scalability and generalizability due to a single point actuation and limited interaction area.
More closely related to our work, a few researchers have explored the use of mobile robots for tangible remote collaboration in VR (e.g., \textit{PhyShare}~\cite{he2017physhare}) and mixed reality environments (e.g., \textit{Siu et al.}~\cite{siu2018investigating}).
However, this approach of using multiple mobile robots has not been fully explored yet, as these prior works do not present the comprehensive design space and have not conducted any user evaluation to understand the benefits and limitations of this approach.
Beyond these prior works, we contribute to 1) an exploration of the broader design space with a demonstration of comprehensive applications, and 2) a holistic user evaluation through condition experiments.
% Beyond these prior works, we contribute to 1) an exploration of the broader design space, 2) a demonstration of comprehensive applications, and 3) a holistic user evaluation through condition experiments.

\subsection{Bi-Directional Virtual-Physical Interaction}
Outside the context of remote collaboration, past HCI research has also explored bi-directional virtual-physical interaction by leveraging augmented reality and actuated environments~\cite{suzuki2022augmented}.
For example, systems like \textit{Kobito}~\cite{aoki2005kobito}, \textit{Augmented Coliseum}~\cite{kojima2006augmented}, and \textit{IncreTable}~\cite{leitner2008incretable} explore the synchronous coupling between AR and actuated physical objects, which can enrich visual feedback and affordances of robots and  actuated tangible interfaces.
These interfaces typically employ robot motion (e.g., \textit{exTouch}~\cite{kasahara2013extouch}), actuated tangible tokens (e.g., \textit{PICO}~\cite{patten2007mechanical}, 
\textit{Reactile}~\cite{suzuki2018reactile}, \textit{Actuated Workbench}~\cite{pangaro2002actuated}), IoT devices (e.g., \textit{MechARSpace}~\cite{zhu2022mecharspace}, \textit{WIKA}~\cite{jeong2020wika}, \textit{Kim et al.}~\cite{kim2018does}) to synchronize between virtual and physical outputs in a bi-directional manner.
Similar to our work, \textit{Sketched Reality}~\cite{kaimoto2022sketched} and \textit{Physica}~\cite{li2023physica} explores bi-directional interaction between embedded virtual objects and tabletop robots. 
Our system extends their work in the context of holographic tangible remote collaboration in mixed reality environments. 

\subsection{Actuated Tangible User Interfaces}
Actuated tangible user interfaces were originally developed to address the challenge of digital-physical discrepancies in conventional tangible interfaces~\cite{poupyrev2007actuation}.
Towards this goal, HCI researchers have explored a variety of actuated tangible user interfaces~\cite{poupyrev2007actuation} and shape-changing user interfaces~\cite{rasmussen2012shape, coelho2011shape, alexander2018grand}, using magnetic actuation~\cite{patten2007mechanical}, ultrasonic waves~\cite{marshall2012ultra}, magnetic levitation~\cite{lee2011zeron}, and wheeled and vibrating robots~\cite{nowacka2013touchbugs}.
Rosenfeld et al.~\cite{rosenfeld2004physical} introduced the concept of using physical mobile robots as an actuated tangible user interface.
% Recently, \textit{Zooids}~\cite{le2016zooids} introduces the swarm user interface, which employs a swarm of tabletop robots as actuated tangible objects. 
This concept has been expanded through various systems such as 
\textit{Zooids}~\cite{le2016zooids},
\textit{ShapeBots}~\cite{suzuki2019shapebots}, \textit{HERMITS}~\cite{nakagaki2020hermits}, \textit{Rolling Pixels}~\cite{lee2020rolling}, and \textit{(Dis) Appearables}~\cite{nakagaki2022dis}. 
Swarm user interfaces can also provide haptic sensations~\cite{kim2019swarmhaptics, suzuki2017fluxmarker, suzuki2021hapticbots, zhao2017robotic} and actuate everyday objects~\cite{kim2020user, farajian2022swarm}. 
Inspired by these works, we also leverage multiple tabletop robots for our actuated interfaces.


% % Figure environment removed

\section{3D-to-2D Generative Pre-training}
\subsection{Preliminary: Generative Pre-training}

Generative pre-training is a fundamental branch of pre-training methods that aims at reconstructing integral and complete data given partial or disrupted input. Mathematically, suppose $x$ is a sample from raw data with no annotation. The pre-processing step $T(\cdot)$ either erases part of $x$ randomly or splits $x$ into pieces and intermingles them to get $\tilde{x}=T(x)$. The generative pre-training model $M$ is designed to restore from those broken input $\hat{x}=M(T(x))$ and the training loss function is designed to measure the reconstruction distance $\mathcal{L}=D(\hat{x}, x)$.
In point cloud object analysis, earlier generative pre-training methods propose various pretext tasks as $T$, including deformation~\cite{achituve2021self}, jigsaw puzzles~\cite{Jigsaw3D} and depth projection~\cite{occo} to produce disarrayed or partial point clouds. Recently, inspired by MAE~\cite{mae} in the image domain, generative pre-training in 3D domain mainly focuses on implementing random masking as $T$ and utilizing Transformers model as $M$ for reconstruction~\cite{yu2022point, pang2022masked, liu2022masked, pointm2ae}. The reconstruction distance $D$ is usually measured by the classical $l_2$ Chamfer Distance:
\begin{equation}
    D(\hat{x},x)=\frac{1}{\lvert \hat{x}\rvert}\sum_{a\in \hat{x}}\min_{b\in x}\lVert a-b \rVert_2^2 + \frac{1}{\lvert x\rvert}\sum_{b\in x}\min_{a\in \hat{x}}\lVert a-b \rVert_2^2
\label{eq:chamfer}
\end{equation}
Besides Chamfer Distance between point clouds, some methods also exploit feature distance between latents~\cite{yu2022point} or occupancy value distance~\cite{liu2022masked} as the loss function. 

The exact reason why generative pre-training would help enhance the representation ability of backbone models still remains an open question. However, abundant experimental results have conveyed that predicting missing parts according to known parts demands high reasoning ability and global comprehension capacity of the model. What's more, generative pre-training is more efficient and suitable for point cloud object analysis than contrastive pre-training, given that contrastive pre-training typically requires a large amount of training data to avoid trivial overfitting solutions but there has always been a data-starvation problem in point cloud object research field. 

\subsection{Overall Pipeline}

Different from the aforementioned generative pre-training methods that focus on uni-modal point cloud reconstruction, we propose a novel cross-modal pre-training approach of generating view images from instructed camera poses. 

The overall architecture of our proposed TAP pre-training model is depicted in Figure~\ref{fig:pipeline}. Our model takes as an input point cloud $P\in \mathbb{R}^{N\times 3}$, where $N$ is the number of points in the input point cloud. The basic building block of TAP mainly consists of: 1) a \textit{3D Backbone} that extracts 3D geometric features $F_\textrm{3d}\in \mathbb{R}^{n\times C_\textrm{3d}}$, where $n$ is the number of downsampled center points and $C_\textrm{3d}$ is the geometric feature dimension; 2) a \textit{pose-dependent Photograph Module} that takes as inputs $F_\textrm{3d}$ and pose matrix $R\in \mathbb{R}^{3\times 3}$, and predicts view image features $F_\textrm{2d}^R\in \mathbb{R}^{h\times w\times C_\textrm{2d}}$ conditioned on $R$, where $h, w$ are height and width of predicted view image feature map; 3) an \textit{2D Generator} that decodes $F_\textrm{2d}^R$ into an RGB image $I^R_\textrm{gen}\in \mathbb{R}^{H\times W\times 3}$, where $H, W$ are height and width of the output view image. 

As we place no restriction on $F_\textrm{3d}$, the \textit{3D Backbone} can be arbitrarily chosen and adopted. Therefore, our TAP is more flexible and compatible than existing generative pre-training methods that are limited to Transformer-based architecture. Experimental results in Section~\ref{sec:exp} will later verify that TAP brings consistent improvement to all kinds of point cloud models. The technical designs of the \textit{pose-dependent Photograph Module} will be thoroughly discussed in Section~\ref{sec:photo_module}. The \textit{2D Generator} consists of four Transpose Convolution layers to progressively upsample image resolution and decode RGB colors of each pixel.

\subsection{Photograph Module}
\label{sec:photo_module}

\noindent\textbf{Architectural Design.} As illustrated in Figure~\ref{fig:pipeline}, we leverage cross-attention mechanism from Transformers~\cite{vaswani2017attention} to build our \textit{pose-dependent Photograph Module}.
\begin{equation}
    \textrm{Attention}(Q,K,V) = \textrm{softmax}\left(\frac{QK^T}{\sqrt{d_k}}\right)V
\end{equation}
where $d_k$ is the scaling factor, and $Q,K,V$ are quries, keys and values matrix. More specifically, we design a Query Generator $\Phi$ to encode camera pose conditions into query tokens: $Q=\Phi(R)\in \mathbb{R}^{hw\times C_\textrm{2d}}$. We also design a Memory Builder $\Theta$ to construct $K$ and $V$ from 3D geometric features: $K=V=\Theta(F_\textrm{3d})\in \mathbb{R}^{m\times C_\textrm{2d}}$, where $m$ is the number of memory tokens. The output sequence of the cross attention layers will be rearranged from $hw \times C_\textrm{2d}$ to $h\times w \times C_\textrm{2d}$, forming the predicted view image features $F_\textrm{2d}^R$.

During the cross-attention calculation process, we do not explicitly provide any projection clues of which 3D points would project to which 2D pixel. Instead, the Photograph Module learns by itself how to arrange unordered 3D feature points to ordered 2D pixel grids, purely based on semantic similarities between 3D geometric features and our delicately-designed queries that reveal pose information. Since one sample will only have one set of memory tokens in 3D space but its view images from different poses are quite distinct from each other, learning to predict precise view images from instructed poses in a data-driven manner is not a trivial task. Therefore, during the end-to-end optimization process, the 3D backbone is trained to have a stronger perception of the object's overall geometric structure and gain a higher representative ability of the stereoscopic relations. In this way, our proposed 3D-to-2D generative pre-training would help exploit the potential and enhance the strength of 3D backbone models.

\vspace{6pt}
\noindent\textbf{Query Generator.} The query generator $\Phi$ is designed to encode pose condition $R$ into 2D grid of shape $h\times w$. In object analysis, common practice is leveraging parallel light shading to project 3D objects onto 2D grids, and pose matrix $R$ here is used to rotate objects into desired angles before projection. Therefore, each 2D grid actually represents an optical line that starts from infinity, passes through 3D objects and ends at the 2D plane. As a consequence, we choose the direction and the origin points that the optical line goes through as the delegate of the query grid. 

Before deriving formulations of optical lines for each grid, let us first revisit the parallel light shading process for better comprehension. Given 3D coordinates $\mathbf{x}=(x,y,z)$ of a point cloud $P$ and pose matrix $R$, rotation is first performed to align the object to the ideal pose position:
\begin{equation}
    \mathbf{x'}=(x',y',z')=R\mathbf{x}
\label{eq:rotate}
\end{equation}
Then we just omit the final dimension $z'$ and evenly split the first two dimensions $(x',y')$ into 2D grids $(u,v)$:
\begin{equation}
\begin{aligned}
    u = \frac{x'-x_0}{g_h} + o_h, \quad
    v = \frac{y'-y_0}{g_w} + o_w
\label{eq:proj}
\end{aligned}
\end{equation}
where $(x_0, y_0)$ is the minimum value of $(x',y')$, $(g_h, g_w)$ is the grid size, $(o_h, o_w)$ is the offset value to place the projected object at the center of the image. $0\leq u \leq h-1, 0\leq v \leq w-1$ and $(u,v)$ is a sampled pixel coordinate from the 2D grid.

Now let us begin to derive formulations of the optical line that passes through the query grid. We only know $(u,v)$ for each grid and we want to reversely trace which 3D points $(x,y,z)$ are on the same optical line during parallel light projection. According to Eq.~\ref{eq:proj}:
\begin{equation}
\begin{aligned}
    x' &= g_h u + x_0 - o_h = \Psi_h(u) \\
    y' &= g_w v + y_0 - o_w = \Psi_w(v)
\end{aligned}
\end{equation}
If we denote $A=R^{-1}$ and $A_{ij}$ as the element at $i^{th}$ row and $j^{th}$ column, then according to Eq.~\ref{eq:rotate}:
\begin{equation}
\begin{aligned}
    x &= A_{11}\Psi_h(u) + A_{12}\Psi_w(v) + A_{13}z' = \Omega_x(u,v) + A_{13}z' \\
    y &= A_{21}\Psi_h(u) + A_{22}\Psi_w(v) + A_{23}z' = \Omega_y(u,v) + A_{23}z' \\
    z &= A_{31}\Psi_h(u) + A_{32}\Psi_w(v) + A_{33}z' = \Omega_z(u,v) + A_{33}z' 
\label{eq:line}
\end{aligned}   
\end{equation}
According to the definition of line's parametric equation, Eq.~\ref{eq:line} represents a line passing through the origin point $O:(\Omega_x(u,v), \Omega_y(u,v), \Omega_z(u,v))$ with optical line direction $\mathbf{d}=(A_{13}, A_{23}, A_{33})$, where $\Omega_x, \Omega_y, \Omega_z$ are $xyz$ coordinates of $O$ and their formulations are conditioned on $u,v$. Therefore, we concatenate the coordinate of origin point $O$, normalized direction $\mathbf{d}^\dagger = \mathbf{d} / \lVert \mathbf{d} \rVert_2$ and normalized position $(u/h,v/w)$ as positional embedding together to be the initial state of our query. A multi-layer-perceptron (MLP) module is later leveraged to map the 8-dim initial query to higher dimensional space.

\vspace{6pt}
\noindent\textbf{Memory Builder.} The memory builder takes $F_\textrm{3d}$ as input to prepare for initial state of $K, V$ in cross-attention layers. We first concatenate aligned 3D coordinate $P_\textrm{3d}$ with 3D features to enhance the geometric knowledge of $F_\textrm{3d}$:
\begin{equation}
    \hat{F}_\textrm{3d} = \mathrm{MLP}(\mathrm{cat}(F_\textrm{3d}, P_\textrm{3d}))
\end{equation}
Additionally, we initialize a learnable memory token $T_\textrm{pad}$ as the pad token and concatenate it with $\hat{F}_\textrm{3d}$ to obtain the initial state of $K, V$. The reason for concatenating a learnable pad token $T_\textrm{pad}$ is that there are white background areas on the projected image (as shown in Figure~\ref{fig:pipeline}). As $F_\textrm{3d}$ only encodes foreground objects, we further need a learnable pad token to represent background regions. Otherwise, the cross-attention layers will be confused to learn how to combine foreground tokens into background features and this will inevitably diminish the pre-training effectiveness.

\subsection{Objective Function}

We perform per-pixel supervision with Mean Squared Error (MSE) loss between generated view image $I^R_\textrm{gen}$ and ground truth image $I^R_\textrm{gt}$, aligned by camera pose $R$. For simplicity, we will omit $R$ in later formulations. As the background of the rendered ground truth images is all white and reveals little information, we further design a compound loss to balance the weight between foreground regions and background regions:
\begin{equation}
    \mathcal{L}(I_\textrm{gen}, I_\textrm{gt}) = w^\textrm{fg} \mathcal{D}^\textrm{fg} + w^\textrm{bg} \mathcal{D}^\textrm{bg}
\end{equation}
\begin{equation}
    \mathcal{D}^{k}(I^{k}_\textrm{gen}, I^{k}_\textrm{gt}) = \frac{1}{HW}\sum_{h,w}(I^{k}_{\textrm{gen}}(h,w) - I^{k}_{\textrm{gt}}(h,w))^2
\end{equation}
where $k=\textrm{fg (foreground)}, \textrm{bg (background)}$ and $w^\textrm{fg}, w^\textrm{bg}$ are loss weights for foreground and background, respectively. Such per-pixel supervision is more precise than the ambiguous set-to-set Chamfer Distance introduced in Eq.~\ref{eq:chamfer}. 

\section{Empirical Evaluation}
\label{sec:exp}

In this section, we first evaluate the accuracy and efficiency of the analytical estimators. Then, we analyze the relationship between local robustness and softmax probability. Lastly, we demonstrate the usefulness of local robustness and its analytical estimators in real-world applications. Key results are discussed in this section and full results are in Appendix~\ref{app:experiments}.

%datasets and models
%\subsection{Datasets and Models}
\textbf{Datasets and Models.}
We evaluate the estimators on four datasets: MNIST \citep{deng2012mnist}, FashionMNIST \citep{xiao2017fashion}, CIFAR10 \citep{krizhevsky2009learning}, and CIFAR100 \citep{krizhevsky2009learning}. For MNIST and FashionMNIST, we train linear models and CNNs to perform classification. For CIFAR10 and CIFAR100, we train ResNet18 \citep{he2016deep} models to perform classification. We train the ResNet18 models using varying levels of gradient norm regularization ($\lambda$) to obtain models with varying levels of robustness. The experiments below use each dataset's full test set, each consisting of 10,000 points. Additional details about the datasets and models are described in Appendix~\ref{app:datasets} and \ref{app:models}.


\subsection{Evaluation of the accuracy of analytical estimators}
\label{sec:exp_correctness}

\textbf{The analytical estimators accurately compute local robustness.}
To confirm that the analytical estimators accurately compute \probust{}, we calculate \probust{} for each model and test set using \pmc{}, \ptaylor{}, \pmmse{}, \ptaylormvs{}, \pmmsemvs{}, and \psoftmax{} for different $\sigma$'s. For \pmc{}, \pmmse{}, and \pmmsemvs{}, we use a sample size at which these estimators have converged ($n=10000, 500, \text{and } 500$, respectively). (Convergence analyses are in Appendix~\ref{app:experiments}.) Then, we measure the absolute and relative difference between \pmc{} and the other estimators. The smaller these differences, the more accurately the estimator computes \probust{}. 

%pmmse family = best estimator
The performance of the estimators for the FashionMNIST CNN model is shown in Figure~\ref{fig1a:method-works-over-sigma}. The results indicate that \pmmsemvs{} and \pmmse{} are the best estimators of \probust{}, followed closely by \ptaylormvs{} and \ptaylor{}, trailed by \psoftmax{}. Consistent with the theory in Section~\ref{sec:methods}, the MMSE estimators outperform the Taylor ones because the former obtains better estimates of $\grad g_i(\X)$, and \psoftmax{} performs poorly in general settings because of its multiple levels of approximation.

%smaller noise neighborhood, better approximation
The results also confirm that the smaller the noise neighborhood $\sigma$, the more accurately the estimators compute \probust{}. For the MMSE and Taylor estimators, this is because their linear approximation of the model around the input is more faithful for smaller $\sigma$'s. As expected, when the model is linear, \ptaylor{} and \pmmse{} accurately compute \probust{} for all $\sigma$'s (Appendix~\ref{app:experiments}). For the softmax estimator, \psoftmax{} values are constant over $\sigma$'s and this particular model has high \psoftmax{} values for most points. Thus, for small $\sigma$'s where \probust{} is near one, \psoftmax{} happens to approximate \probust{} for this model. Examples of images with varying levels of noise ($\sigma$) are in Appendix~\ref{app:experiments}.

\textbf{For robust models, the analytical estimators compute local robustness more accurately over a larger noise neighborhood.} 
The performance of \pmmse{} for CIFAR10 ResNet18 models of varying levels of robustness is shown in Figure~\ref{fig1b:method-works-robust}. The results indicate that for more robust models (larger $\lambda$), the estimator is more accurate over a larger $\sigma$. This is because gradient norm regularization leads to models that are more locally linear, making the estimator's linear approximation of the model around the input more accurate over a larger $\sigma$, making its \probust{} values more accurate.


\textbf{The mv-sigmoid function approximates the multivariate Normal CDF well in practice.} To examine \emph{mv-sigmoid}'s approximation of \emph{mvn-cdf}, we compute both functions using the same inputs ($z~=~\left[  \frac{g_1(\X)}{\sigma \|\grad g_1(\X)\|_2}, ..., \frac{g_C(\X)}{\sigma \|\grad g_C(\X) \|_2} \right]$, as described in Proposition~\ref{eqn:taylor-estimator}) for the CIFAR10 ResNet18 model for different $\sigma$'s. The plot of \emph{mv-sigmoid(z)} against \emph{mvn-cdf(z)} for $\sigma=0.05$ is shown in Figure~\ref{fig2:mvsig-mvncdf}. The results indicate that the two functions are strongly positively correlated with low approximation error, suggesting that \emph{mv-sigmoid} approximates the \emph{mvn-cdf} well in practice.

% \clearpage

%fig1 -- method properly approximates p_empirical
%fig1a: method works
%02d_pemp_vs_pothers_over_sigma/rel/fmnist_cnn.png
%fig1b: method works better for robust models
%02e_pemp_vs_pmmse_over_sigma_robust_models/rel/cifar10_resnet18.png
% Figure environment removed


%fig2: mvsigmoid is a good approximator for mvncdf
%correlation for sigma=0.1 
% 03_mvncdf_vs_mvsigmoid/cifar10_resnet18/cifar10_resnet18_gnorm0.0_sigma0.05.png

%fig3: p_mc takes many samples to converge
% 02a_p_emp_convergence_n50000_baseline/rel/cifar10_resnet18_sigma0.1.png
% Figure environment removed


%table: naive method is inefficient, analytical method is efficient
\begin{table}[ht!]
\centering
\begin{tabular}{l|l|l|l|l|l}
    \multicolumn{2}{c}{}   & \multicolumn{2}{|c|}{CPU: Intel x86\_64}   & \multicolumn{2}{|c}{GPU: Tesla V100-PCIE-32GB} \\
    \midrule
    Estimator   & \# samples ($n$)   & Serial   & Batched   & Serial   & Batched \\
    \midrule
    \pmc{}   & \begin{tabular}[c]{@{}l@{}}  $n=10000$\end{tabular}               
             & \begin{tabular}[c]{@{}l@{}}  1:41:11\end{tabular}                                               
             & \begin{tabular}[c]{@{}l@{}}  1:14:38\end{tabular}                                                
             & \begin{tabular}[c]{@{}l@{}}  0:19:56\end{tabular}                                                
             & \begin{tabular}[c]{@{}l@{}}  0:00:35\end{tabular} \\
    \ptaylor{}   & N/A
                 & 0:00:08                                                                                                                     
                 & 0:00:07                                                                                                                      
                 & 0:00:02                                                                                                                      
                 & $<$ 0:00:01 \\
    \pmmse{}   & \begin{tabular}[c]{@{}l@{}} $n=5$\end{tabular} 
               & \begin{tabular}[c]{@{}l@{}} 0:00:41\end{tabular} 
               & \begin{tabular}[c]{@{}l@{}} 0:00:31\end{tabular} 
               & \begin{tabular}[c]{@{}l@{}} 0:00:06\end{tabular} 
               & \begin{tabular}[c]{@{}l@{}} 0:00:02\end{tabular} \\              
\end{tabular}
\vspace{0.2cm}
\caption{Runtimes of \probust{} estimators. Each estimator computes \probustwsigma{0.1} for the CIFAR10 ResNet18 model for 50 data points. Estimators that use sampling use the minimum number of samples necessary for convergence. Runtimes are in the format of hour:minute:second. The analytical estimators (\ptaylor{} and \pmmse{}) are more efficient than the naïve estimator (\pmc{}).}
\vspace{-0.5cm}
\label{table:runtimes}
\end{table}



%fig4 -- p_robust and p_softmax
%fig4a: scatterplot, non-robust model
%02i_pemp_vs_pmmse_corr_robust_models_scatterplots/cifar10_resnet18_sigma0.1_gnormreg0.png
%fig4b: the more robust the model, the more the two are related
%02h_pemp_vs_pmmse_corr_robust_models_lineplots/cifar10_resnet18_cifar100_resnet18_sigma0.1.png
%fig4c: scatterplot, robust model
%02i_pemp_vs_pmmse_corr_robust_models_scatterplots/cifar10_resnet18_sigma0.1_gnormreg0.01.png
% Figure environment removed

%fig5: local robustness bias
% 02f_p_distr_vs_classes/p_all_over_classes/cifar10_resnet18_sigma0.09.png
\begin{SCfigure}
  \vspace{1cm}
  \centering
  % Figure removed
  \vspace{-2.5cm}
  \caption{Local robustness bias among classes for the ResNet18 CIFAR10 model. \probust{} reveals that the model is less locally robust for some classes than for others. The analytical estimator \pmmse{} properly captures this model bias.}
  \label{fig5:robustness-bias}
\end{SCfigure}


%fig4 -- 2x4 images
%top-k and bottom-k images
% 02g_topk_bottomk_images/cifar10_resnet18/p_mmse/
% - cifar10_resnet18_p_mmse_sigma0.1_class9_bottomk.png
% - cifar10_resnet18_p_mmse_sigma0.1_class9_topk.png
% - ... class0 x 2
% 02g_topk_bottomk_images/cifar10_resnet18/p_sm
% - cifar10_resnet18_p_sm_sigma0.1_class9_bottomk.png
% - cifar10_resnet18_p_sm_sigma0.1_class9_topk.png
% - ... class0 x 2
% Figure environment removed



% %fig4 -- 3x4 images
% %top-k and bottom-k images
% % 02g_topk_bottomk_p/resnet18_cifar10/p_mmse/
% % - resnet18_cifar10_p_mmse_sigma0.1_class9_topk.png
% % - resnet18_cifar10_p_mmse_sigma0.1_class9_bottomk.png
% % - ...class8... x 2
% % - ...class0... x 2
% % 02g_topk_bottomk_p/resnet18_cifar10/p_sm/
% % - resnet18_cifar10_p_sm_sigma0.1_class9_topk.png
% % - resnet18_cifar10_p_sm_sigma0.1_class9_bottomk.png
% % Figure environment removed







\subsection{Evaluation of the efficiency of analytical estimators}

\textbf{The naïve estimator is statistically inefficient.} To examine the efficiency of \pmc{}, we calculate \pmc{} for each model and test set using different sample sizes ($n$) over different $\sigma$'s, and measure the absolute and relative difference between \pmc{} at a given $n$ and \pmc{} at $n=50,000$. Results for the CIFAR10 ResNet18 model are shown in Figure~\ref{fig3:pmc-convergence}. The results indicate that \pmc{} requires around 10,000 samples per point to converge, which is impractical.

\textbf{The analytical estimators are more efficient than the naïve estimator.}
Next, we examine the efficiency of the estimators by measuring their runtimes when calculating \probustwsigma{0.1} for the CIFAR10 ResNet18 model for 50 points. Runtimes are displayed in Table~\ref{table:runtimes}. They indicate that \ptaylor{} and \pmmse{} perform 35x and 17x faster than \pmc{}, respectively. Additional runtimes are in Appendix~\ref{app:experiments}.
% Thus, the analytical estimators are more efficient than the naïve estimator.

\subsection{Comparison of local robustness and softmax probability}

\textbf{Local robustness and softmax probability are two distinct measures.} To examine the relationship between \probust{} and \psoftmax{}, we calculate \pmmse{} and \psoftmax{} for CIFAR10 and CIFAR100 models of varying levels of robustness, and measure the correlation of their values and ranks using Pearson and Spearman correlations. Results are in Figure~\ref{fig4:probust-and-psoftmax}. For a non-robust model, \probust{} and \psoftmax{} are not strongly correlated (Figure~\ref{fig4a:ps-nonrob-model}). As model robustness increases, the two quantities become more correlated (Figures~\ref{fig4b:ps-rob-models-lineplot} and~\ref{fig4c:ps-rob-model}). However, even for robust models, the relationship between the two quantities is mild (Figure~\ref{fig4c:ps-rob-model}). That \probust{} and \psoftmax{} are not strongly correlated is consistent with the theory in Section~\ref{sec:methods}: in general settings, \psoftmax{} is not a good estimator for \probust{}.

% that the two probabilities are conceptually different: \probust{} measures the uncertainty of a model’s prediction with respect to input noise (i.e., the probability that the prediction will change upon adding noise to the input) while \psoftmax{} is an uncalibrated uncertainty of the model's prediction being correct with respect to a calibration set. \textcolor{red}{[check interp of raw \psoftmax]}.

\subsection{Applications of local robustness}

\textbf{\boldmath \probust{} detects local robustness bias.} We demonstrate that \probust{} can detect bias in local robustness by examining its distribution for each class for each model and test set over different $\sigma$'s. Results for the CIFAR10 ResNet18 model are in plotted in Figure~\ref{fig5:robustness-bias}. The results show that different classes have different \probust{} distributions, i.e., the model is more locally robust for some classes (e.g., frog) than for others (e.g., airplane). The results also show that \pmc{} and \pmmse{} have very similar distributions, further indicating that the latter well-approximates the former. Thus, \probust{} can be applied to detect local robustness bias, which is critical when models are deployed in high-stakes, real-world settings.

\textbf{\boldmath \probust{} identifies images that are robust to and images that are vulnerable to random noise.} We demonstrate that \probust{} can also distinguish between images that are robust to and images that are vulnerable to random noise in a way that is superior to \psoftmax{}. For each dataset, we train a simple CNN to distinguish between images with high and low \pmmse{} and the same CNN to also distinguish between images with high and low \psoftmax{} (additional setup details described in Appendix~\ref{app:experiments}). Then, we compare the performance of the two models. For CIFAR10, the test set accuracy for the \pmmse{} CNN is 0.92 while that for the \psoftmax{} CNN is 0.58. These results indicate that \probust{} better identifies images that are robust to and vulnerable to random noise than \psoftmax{}.

We also visualize images with the highest and lowest \pmmse{} in each class for each model. For comparison, we do the same with \psoftmax{}. Example CIFAR10 images are displayed in Figure~\ref{fig6:topk-vs-bottomk}. Images with low \probust{} tend to have neutral colors, with the object being a similar color as the background (making the prediction likely to change when the image is slightly perturbed), while images with high \probust{} tend to be brightly-colored, with the object strongly contrasting with the background (making the prediction likely to stay constant when the image is slightly perturbed). These differences are not as evident for images with the highest and lowest \psoftmax{}. Thus, in addition to detecting local robustness bias, \probust{} can also be applied to identify images that are robust to and images that are vulnerable to random noise.



% For all the experiments above, we observe consistent results across datasets and models (Appendix).

% --- OLD STUFF BELOW ---


%\ptaylormvs{} and \pmmsemvs{} perform better than \ptaylor{} and \pmmse{}, respectively, because... \textcolor{blue}{[How to explain why \pmmsemvs and \ptaylormvs perform better than \pmmse and \ptaylor?]} 


% %exp1
% \textbf{Experiment 1. As the noise neighborhood increases, local robustness deteriorates.}
% First, we examine the behavior of \probust{} as the noise neighborhood increases. For a given model, we calculate \pmc{} for different values of $\sigma$ for 1,000 randomly-selected points from the test set (hereafter referred to as the “test set”). To calculate \pmc{} for a given image, we use 10,000 noisy samples (a value at which \pmc{} converged; convergence analyses are described in Appendix~\ref{app:exp-convergence-pmc-pmmse}).

% Results for the linear model and CNN trained on FashionMNIST are shown in Figure~\ref{fig1:pmc-vs-noise}. In the figure, the distribution of \pmc{} is concentrated at one for small values of $\sigma$ and increasingly shifts towards zero as $\sigma$ increases. Thus, as expected, the results indicate that the models are locally robust for small noise neighborhoods, and as the noise neighborhood increases, local robustness deteriorates. This is expected because as more noise is added to the original image, it becomes more likely for the prediction of the noisy image to differ from that of the original image, causing local robustness to deteriorate (i.e., causing \probust{} to decrease). We observe similar results across datasets and models (Appendix~\ref{app:exp-pmc-vs-noise}).


% %exp3
% \textbf{Experiment 3. \boldmath \probust{} and \boldmath \psoftmax{} measure different types of uncertainty.}
% Next, we show that \probust{} and \psoftmax{} measure different types of uncertainty. For a given model, test set, and noise neighborhood ($\sigma$), we calculate \pmmse{} (a close estimate of \probust{}) and \psoftmax{} and examine their relationship. 

% Results for the ResNet18 model trained on CIFAR10 at $\sigma=0.1$ are shown in Figure~\ref{fig3:correlation-pmmse-psm}. The results indicate that \probust{} and \psoftmax{} are not strongly correlated: points which have high \psoftmax{} values have high or low \probust{} values, and points which have low \probust values have high or low \psoftmax{} values. This finding is corroborated by the low Pearson correlation and Spearman correlation coefficients, indicating that the values and ranks, respectively, of \probust{} and \psoftmax{} are not strongly correlated. We observe similar results across datasets and models (Appendix~\ref{app:exp-correlation-pmmse-psm}).

% That \probust{} and \psoftmax{} are not strongly correlated is consistent with the understanding that the two probabilities measure different types of uncertainty: \probust{} measures the uncertainty of a model’s prediction with respect to input noise (i.e., the probability that the prediction will change upon adding noise to the input) while \psoftmax{}... \textcolor{blue}{[what is the interpretation of raw \psoftmax, if any?]}.


% %exp4

% \textbf{Experiment 4. \boldmath \probust{} can be used to detect differences in local robustness among classes.}
% Next, we demonstrate that \probust{} can be used to detect bias in local robustness among classes. We examine two setups. First, for a given model, test dataset, and noise neighborhood ($\sigma$), we calculate \probust{} using \pmmse{} and examine the distribution of \probust{} for each class. This setup examines local robustness based on confidence level for fixed $\sigma$. Second, for a given model, test dataset, and confidence level, we calculate the noise neighborhood that meets the specified confidence level. To do so, we optimize $\sigma$ such that \ptaylor{} equals the specified confidence level (we use \ptaylor{} because it is easier to optimize than \pmmse{}; however, the same idea applies to \pmmse{}). This setup examines local robustness based on $\sigma$ for a fixed confidence level.

% Results for both setups for the ResNet18 model trained on CIFAR10 are shown in Figure~\ref{fig4:robustness-bias}. Different classes have different confidence level distributions when $\sigma$ is fixed (Figure~\ref{fig4a:sigma-fixed}) and different $\sigma$ distributions when confidence level is fixed (Figure~\ref{fig4b:confidence-fixed}), indicating that the model is more locally robust for some classes than for others. We observe similar results across datasets and models (Appendix~\ref{app:exp-robustness-bias}).

% ADD LATER: We also examine local robustness bias among classes by fixing the same confidence level for each point and calculating the $\sigma$ that yields that confidence level, and we obtain consistent results indicating that models are not equally robust for all classes.

% %exp5
% \textbf{Experiment 5. \probust{} can distinguish between clear and ambiguous images.}
% Lastly, we demonstrate that \probust{} can distinguish between clear and ambiguous images. For a given model, test set, and $\sigma$, we calculate \probust{} (using \pmmse{}) and \psoftmax{}. Then, for each class, we measure the difference between images with the highest and lowest \probust{} and the difference between images with the highest and lowest \psoftmax{} by… \textcolor{blue}{[need to do this experiment, see if we get desired result]}.

% Results for the ResNet18 model trained on CIFAR10 are shown in Figure~\ref{fig5:topk-vs-bottomk}. As seen in Figure~\ref{fig5a:differences}, differences between the top and bottom images based on \probust{} are larger than those based on \psoftmax{} \textcolor{blue}{[currently wrote desired results, fill in real results later]}. Visual inspection of the images, such as in Figures~\ref{fig5b:bottomk-probust}-\ref{fig5e:topk-psoftmax}, suggests that top and bottom images based on \probust{} tend to be clear and ambiguous images for each class, respectively, while this distinction is not evident for top and bottom images based on \psoftmax{}. We observe similar results across datasets and models (Appendix~\ref{app:exp-topk-vs-bottomk}). Taken together, these results indicate that \probust{} better distinguishes between clear and ambiguous images than \psoftmax{}, suggesting that \probust{} better reflects image differences in the latent feature space \textcolor{blue}{[check statement after the comma]}. 

%!TEX root = ecai-main.tex



\section{Discussion and Future Work}
\label{sec:con}


% % %   Conclusion
TKB Alignment is a new variant of the alignment problem that admits richer state and property descriptions. Our setting uses \alc-TKBs, CQs with \ltl operators, and a cost function for the edit operations. 
%
We have shown that TKB Alignment \wrt temporal CQs is solvable, by developing computation methods for both TKB and KB Alignment.

% % % %   Discussion
The TKB-alignment problem is closely related to abduction and to computing repairs of KBs, as these tasks also change a KB to either gain a desired consequence or remove an unwanted one. However, although being active research topics, neither of the two has yet been investigated for the temporalized setting and entailment of TCQs. Furthermore, TKB Alignment requires a cost-optimal solution, which is not very common in the context of abduction or repairs.

Interestingly, TKB Alignment can also be used for relaxing temporal CQ answering. Given a tuple of individuals $\bar{a}$ which is not a certain answer of a TCQ $\phi$, solve TKB Alignment for the Boolean TCQ obtained from $\phi'$ by assigning $\bar{a}$ to the answer variables of $\phi$. The costs computed during TKB Alignment for $\phi'$ then measure the \enquote{distance} to a certain answer of the query. 


% % %   Future Work
Our initial investigation on TKB Alignment uses a unitary cost measure for the edit operations mostly to ease presentation, as our approach can handle other cost measures easily. In this work, we did not regard rigid symbols, which are left for future work.
% \todo[inline]{Do we want to mention metric time  or other extensions here?} 







% \section*{Acknowledgements}



\bibliographystyle{unsrtnat}
\bibliography{references.bib}


%\newpage
%\appendix
%\appendix

\section{Ablation Studies}
In this section, we conduct more ablation studies on hyperparameter choices of the proposed 3D-to-2D generative pre-training, discussing more thoroughly the insights into architectural design and objective function design. We implement PointMLP~\cite{pointmlp} as the 3D backbone model and conduct these ablation experiments on the hardest PB-T50-RS variant of the ScanObjectNN~\cite{uy2019revisiting} dataset. We report the classification accuracy of the fine-tuning results.

\subsection{Cross-Attention Hyperparameters}

In Table~\ref{tab:abl_attn}, we display the results of ablation studies on the number of layers and feature channels of the cross-attention layers in our proposed Photograph module. From the quantitative results, we can conclude that 2 layers with 128 channels is the best hyperparameter group for cross-attention layers. When we implement a shallow layer setting (2 layers in line 1 and 4 layers in line 2), lower feature channels (128 dims) achieves better performance. On the contrary, when we implement a deeper layer setting (6 layers in line 3 and 8 layers in line 4), relatively higher feature channels (256 dims) is the best choice. Additionally, if we use 1024 dims as the feature channels in cross-attention layers, which is the same as the channels of output features from the 3D backbone model, the pre-training stage totally collapses and the fine-tuning results are much lower than models of 128 dims and 256 dims, no matter how much layers are implemented. This result indicates that a bottleneck design in our proposed photograph module is essential for the successful pre-training of the proposed 3D-to-2D generation.

The overall trend is that a lightweight architectural design of the cross-attention layers is better than a heavy module design. This may be because we completely drop the photograph module and only keep the 3D backbone in the fine-tuning stage. Therefore, a lightweight photograph module in the pre-training stage will encourage the 3D backbone to exploit more representation ability and avoid information loss in the fine-tuning stage to the best extent. On the contrary, if we implement a heavy photograph module with deep cross-attention layers and high feature dimensions, the photograph module will dominate the generation process and the importance of the 3D backbone will be neglected. What's worse, in the fine-tuning stage, the rich geometry information in the heavy photograph module is totally dropped out and no longer helpful for downstream tasks.

\begin{table}[!t]
\label{tab:ablation_supp}
\caption{\textbf{Ablation Studies on Hyperparameters.} We implement PoinMLP~\cite{pointmlp} as the 3D backbone model and conduct experiments on the hardest PB-T50-RS variant of ScanObjectNN~\cite{uy2019revisiting} dataset.}
\vspace{-6pt}
\centering
\begin{subtable}{0.49\textwidth}
    \setlength\tabcolsep{3pt}
    \centering
    \caption{Cross-Attention Hyperparameters.}
    \vspace{-4pt}
    \label{tab:abl_attn}
    \adjustbox{width=0.9\textwidth}{
    \begin{tabular}{c|ccc}
    \toprule
    LayerNum $\backslash$ Channels & 128 Dims  & 256 Dims &  1024 Dims \\
    \midrule
    2 Layers    & \textbf{89.1} & 87.9 & 86.3 \\
    4 Layers    & 88.7 & 88.0 & 85.9 \\
    6 Layers    & 88.3 & 88.5 & 85.3 \\
    8 Layers    & 87.7 & 88.1 & 85.8 \\
    \bottomrule
    \end{tabular}}
\end{subtable}
\hfill
\vspace{5pt}
\begin{subtable}{0.49\textwidth}
    \centering
    \setlength\tabcolsep{3pt}
    \caption{Loss Weight Hyperparameters.}
    \vspace{-4pt}
    \label{tab:abl_lossw}
    \newcolumntype{g}{>{\columncolor{Gray}}c}
    \adjustbox{width=0.9\textwidth}{
    \begin{tabular}{@{\hskip 3pt}>{\columncolor{white}[3pt][\tabcolsep]}c|cccccc|cc}
    \toprule
        Model   & G$_1$ & G$_2$ & G$_3$ & G$_4$ & G$_5$ & G$_6$ & H$_1$ & H$_2$ \\
    \midrule
        $w^\textrm{fg}$ & 2 & 5 & 10 & 20 & 30 & 50 & 0 & 20\\
        $w^\textrm{bg}$ & 1 & 1 & 1 & 1 & 1 & 1 & 0 & 1\\
        $w^\textrm{feat}$ & 0 & 0 & 0 & 0 & 0 & 0 & 2 & 2\\
    \midrule
        Acc. (\%) & 87.1 & 87.2 & 88.0 & \textbf{88.5} & 88.0 & 86.8 & 86.3 & 87.8 \\
    \bottomrule
    \end{tabular}}
\end{subtable}
\end{table}

% Figure environment removed


\subsection{Objective Function}
In this subsection, we discuss the objective function design of our proposed 3D-to-2D generative pre-training. In our main paper, we implement pixel-level supervision with MSE loss between generative view images $I_\textrm{gen}$ and ground truth images $I_\textrm{gt}$:
\begin{equation}
    \mathcal{L}_\textrm{pix}(I_\textrm{gen}, I_\textrm{gt}) = w^\textrm{fg} \mathcal{D}(I_\textrm{gen}^\textrm{fg}, I_\textrm{gt}^\textrm{fg}) + w^\textrm{bg} \mathcal{D}(I_\textrm{gen}^\textrm{bg}, I_\textrm{gt}^\textrm{bg})
\end{equation}
where fg denotes foreground region, bg denotes background region and $\mathcal{D}$ is the MSE distance. However, in 2D generation, perceptual loss~\cite{johnson2016perceptual} is of equal importance with pixel-wise loss. While pixel-wise MSE loss focuses on low-level similarities, perceptual loss measures high-level semantic differences between feature representations of the images computed by the pre-trained loss network. Technically, perceptual loss makes use of a loss network $\phi$ pre-trained for image classification, which is typically a 16-layer VGG~\cite{simonyan2014vgg} network pre-trained on the ImageNet~\cite{russakovsky2015imagenet} dataset. If we denote $\phi_j(x)$ as the feature map with size $c_j\times h_j\times w_j$ of the $j$th layer of the network $\phi$, then the perceptual loss is defined as the Euclidean distance:
\begin{equation}
\small
    \mathcal{L}_\textrm{feat}(I_\textrm{gen}, I_\textrm{gt}) = \frac{1}{N}\sum_j\frac{1}{c_j h_j w_j}\lVert\phi_j(I_\textrm{gen})- \phi_j(I_\textrm{gt})\rVert_2^2
\end{equation}
where $N$ is the number of total layers of the VGG network and $1\leq j\leq N$. If we combine the pixel-wise loss $\mathcal{L}_\textrm{pix}$ with the perceptual loss $\mathcal{L}_\textrm{feat}$, then the final objective function of the proposed 3D-to-2D generation is:
\begin{equation}
\small
    \mathcal{L} = \mathcal{L}_\textrm{pix} + w^\textrm{feat} \mathcal{L}_\textrm{feat}
\end{equation}

In Table~\ref{tab:abl_lossw}, we conduct ablations on loss weight of foreground pixel-wise loss $w^\textrm{fg}$, background pixel-wise loss $w^\textrm{bg}$ and perceptual loss $w^\textrm{feat}$. In Model $G_1$ to $G_6$, we only implement pixel-wise loss. In Model $H_1$, we only implement perceptual loss. In Model $H_2$, we combine pixel-wise loss with perceptual loss. From the ablation results, we can conclude that $w^\textrm{fg}:w^\textrm{bg}=20:1$ is the best hyperparameter choice for pixel-wise loss. However, the perceptual loss is not effective when we compare Model $G_4$, Model $H_1$ and Model $H_2$. This is mainly due to the reason that the rendered view image of synthetic ShapeNet~\cite{shapenet} dataset is out of the distribution of the realistic ImageNet~\cite{russakovsky2015imagenet} that the loss model $\phi$ is pre-trained on. Therefore, the high-level semantic representation ability of $\phi$ on view images is relatively poor and cannot guide the optimization of the 3D-to-2D generation process. If the rendered images are more realistic with colors and background, then the perceptual loss is expected to help 3D-to-2D generative pre-training.

% Figure environment removed




\section{Visualization Results}

\subsection{Generated View Images}

Figure~\ref{fig:examples_supp} displays more visualization results of our generated view images from the 3D-to-2D generative pre-training process. We take ShapeNet~\cite{shapenet} as the pre-training dataset and implement PointMLP~\cite{pointmlp} as the 3D backbone model. The first line shows the generated results from our model while the second line shows ground truth images for reference. The visualization results convey that our 3D-to-2D generative pre-training can successfully predict the shape and colors of the objects from specific projection views. There are also some unsatisfactory cases in the last three columns, where there are some vague details in our generated images. This is mainly due to the large downsample ratio ($\times32$) in our model design.

\subsection{Feature Distributions}

Figure~\ref{fig:tsne} shows feature distributions of ModelNet40~\cite{modelnet} and ScanObjectNN~\cite{uy2019revisiting} datasets in t-SNE visualization. We choose PointMLP~\cite{pointmlp} as the 3D backbone and pre-train on ShapeNet~\cite{shapenet} dataset. We can conclude that with our proposed 3D-to-2D pre-training, the 3D backbone model can extract discriminative features after fine-tuning on downstream classification datasets.

% Figure environment removed


\subsection{Part Segmentation Visualizations}

Figure~\ref{fig:partseg} presents visualizations of part segmentation results on samples from the ShapeNetPart dataset. Each part is represented by a distinct color for clarity. These qualitative results serve as compelling visual evidence and provide a vivid illustration of the efficacy of our fine-tune model in achieving accurate part segmentation.


\end{document}