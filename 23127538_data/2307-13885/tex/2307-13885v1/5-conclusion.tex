\section{Conclusion}

In this work, we take the first steps towards estimating local model robustness. We show that the naïve approach is inefficient and develop efficient analytical estimators. We empirically confirm the estimators' accuracy and efficiency. Then, we demonstrate the usefulness of these estimators in performing various real-world tasks.
% Then, using these estimators, we demonstrate the usefulness of local robustness for model debugging and establishing user trust through such tasks as measuring robustness bias and identifying clear and ambiguous examples in a dataset. 

To our knowledge, this work is the first to investigate local robustness in a multi-class setting and develop efficient analytical estimators. The analytical aspect of these estimators not only advances conceptual understanding of local robustness, connecting it to randomized smoothing and softmax probability, but also enables local robustness to be used in applications that require differentiability. In addition, the efficiency of these estimators makes the computation of local robustness practical. 
% enabling tasks that help with model debugging and establishing user trust.


% One alternative is to emphasize the analytical aspect of our estimators for differentiability and conceptual understanding. While we don't use the differentiability in this work, we can say that in this work we just develop the estimators and measure their goodness, but it can be used for applications involving differentiability in the future. With regards to conceptual understanding, the estimators somehow link to concepts like randomised smoothing / linearization and softmax so it's interesting from that perspective.

One limitation of this work is its focus on classification. Defining local robustness and developing efficient analytical estimators for regression represent future research directions. Other directions include exploring additional applications of local robustness, such as uncertainty calibration and training locally robust models.

% Other future research directions include exploring additional applications of local robustness, such as using local robustness as an uncertainty measure to perform uncertainty calibration and using local robustness as part of the objective function to train locally robust models.