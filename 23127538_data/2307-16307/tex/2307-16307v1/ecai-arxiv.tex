%%%% ijcai23.tex
\documentclass{ecai}  % use option [doubleblind] for double blind submission and hiding the authors section
\usepackage{tikz}
\usetikzlibrary{positioning, 
                quotes}
\usepackage{graphicx}
\usepackage{latexsym}
\usepackage{amsmath}
%\ecaisubmission      % inserts page numbers. Use only for submission of paper.
                      % Do NOT use for camera-ready version of paper.

%\paperid{123}        % paper id for double blind submission
%\usepackage[subtle,title=tight]{savetrees} 
%\usepackage[small,compact]{titlesec}
% \usepackage{amsthm}
% \usepackage[english]{babel}
\usepackage{ifthen}
\usepackage{xcolor}
% \usepackage{blindtext}
% \usepackage{algorithm}
% %\usepackage{subfigure}
% \usepackage{graphicx}
\usepackage{amsmath}
% \usepackage[noend]{algpseudocode}
% \usepackage{subfig}
\usepackage{xspace}
\usepackage{amssymb} 
\usepackage{multirow}
\usepackage{url}
\usepackage{hyperref}
% \usepackage{balance}
% \newtheorem{remark}{Remark}
% \usepackage{graphicx}
% \usepackage{footnote}
% \usepackage{comment}
\usepackage{array}
\newcolumntype{P}[1]{>{\centering\arraybackslash}p{#1}}

\usepackage{url}
\def\UrlBreaks{\do\/\do-}

\newcommand{\exclude}[1]{}
\newcommand{\showComments}{yes}
\newcommand{\note}[2]{
    \ifthenelse{\equal{\showComments}{yes}}{\textcolor{#1}{#2}}{}
}
\newcommand{\TODO}[1]{%
  \addcontentsline{tdo}{todo}{\protect{#1}}%
  \note{red}{TODO: #1}
}

\newcommand\numberthis{\addtocounter{equation}{1}\tag{\theequation}}

\newcommand{\frank}[1]{\note{brown}{[WW: #1]}}
\newcommand{\manya}[1]{\note{red}{[MG: #1]}}
\newcommand{\naader}[1]{\note{brown}{[NH: #1]}}
\newcommand{\kayvon}[1]{\note{violet}{[KS: #1]}}
\newcommand{\ying}[1]{\note{green}{[YZ: #1]}}


\newcommand{\psass}{\ensuremath{\mathbin{{=}}\ }}
\newcommand{\addeq}{\ensuremath{\mathbin{{+}{=}}\ }}
\newcommand{\subeq}{\ensuremath{\mathbin{{-}{=}}\ }}
\newcommand{\muleq}{\ensuremath{\mathbin{{\times}{=}}\ }}
\newcommand{\diveq}{\ensuremath{\mathbin{{\divides}{=}}\ }}
\newcommand{\eqeq}{\ensuremath{\mathbin{{=}{=}}\ }}
\newcommand{\todo}[1]{{\color{red} #1}}
\newcommand{\name}{{\sc{OEAINet}}\xspace}
\newcommand{\cc}{{\sc{c$^2$}}\xspace}

\newcommand{\fattree}{{Clos}\xspace}
\newcommand{\fattrees}{{Clos}\xspace}
\newcommand{\LBE}{{\fattree}\xspace}
\newcommand{\SBE}{{Ideal Switch}\xspace}
\newcommand{\OBE}{{Oversub. \fattree}\xspace }
\newcommand{\para}[1]{{\textbf{{#1}}}}
\newcommand{\net}{{{Big-Net}}\xspace}
\newcommand{\MP}{{MP}\xspace}
\newcommand{\allreduce}{{AllReduce}\xspace}
\newcommand{\ata}{{All-to-All}\xspace}
\newcommand{\allgather}{{AllGather}\xspace}
\newcommand{\redsca}{{Reduce-Scatter}\xspace}
\newcommand{\fbd}{{Full-Bisection Domain}\xspace}
\newcommand{\dcd}{{Direct-Connected Domain}\xspace}



\newcommand{\captionvspace}{0em}
\pagestyle{plain}


\newenvironment{CompactItemize}
  {\def\usecounter{\compactify\latexusecounter}
   \begin{itemize}}
  {\end{itemize}\let\usecounter=\latexusecounter}
\date{}
\usepackage{lipsum} % for dummy text
\usepackage{enumitem}
\setlist{nosep} % or \setlist{noitemsep} to leave space around whole list

\begin{document}

\begin{frontmatter}

\title{Representing and Reasoning with Multi-Stakeholder Qualitative Preference Queries}

\author[A]{\fnms{Samik}~\snm{Basu}
\thanks{Corresponding Author. Email: sbasu@cs.iastate.edu.}}
\author[B]{\fnms{Vasant}~\snm{Honavar}\orcid{0000-0001-5399-3489}} % use of \orcid{} is optional
\author[C]{\fnms{Ganesh Ram}~\snm{Santhanam}}
\author[D]{\fnms{Jia}~\snm{Tao}\orcid{0000-0002-2342-3271}}


\address[A]{Department of Computer Science, Iowa State University, Ames, IA, USA}
\address[B]{College of Information Sciences and Technology, Pennsylvania State University, University Park, PA, USA}
\address[C]{Department of Electrical and Computer Engineering, Iowa State University, Ames, IA, USA}
\address[D]{Department of Computer Science, Lafayette College, Easton, PA, USA}


\begin{abstract}
Many decision-making scenarios, e.g., public policy, healthcare, business, and disaster response, require accommodating the preferences of multiple stakeholders. We offer the first formal treatment of reasoning with multi-stakeholder qualitative preferences in a setting where stakeholders express their preferences in a qualitative preference language, e.g., CP-net, CI-net, TCP-net, CP-Theory. We introduce a query language for expressing queries against such preferences over sets of outcomes that satisfy  specified criteria, e.g., $\mlangpref{\psi_1}{\psi_2}{A}$ (read loosely as the set of outcomes satisfying $\psi_1$ that are preferred over outcomes satisfying $\psi_2$ by a set of stakeholders $A$).  
Motivated by practical application scenarios, we introduce and analyze several alternative semantics for such queries, and examine their interrelationships.  We provide a provably correct algorithm for answering multi-stakeholder qualitative preference queries using model checking in alternation-free $\mu$-calculus. We present experimental results that demonstrate the feasibility of our approach.
\end{abstract}

\end{frontmatter}

%% Figure environment removed

\section{Introduction}
Automatic 3D reconstruction of clothed humans using image inputs has gained increasing significance due to its potential applications in a wide array of AR/VR scenarios. High-fidelity reconstructions typically depend on sophisticated capture systems, which are developed with dense camera arrays~\cite{collet2015high,joo2015panoptic,joo2018total}, programmable light-stages~\cite{Vlasic2009, guo2019relightables}, and depth sensors~\cite{newcombe2011kinectfusion,DoubleFusion,BodyFusion,dou2016fusion4d,newcombe2015dynamicfusion}. However, stringent capture environments equipped with complex hardware pose significant challenges for consumer-level applications.


In this context, considerable research effort has been dedicated to developing methods that allow for more flexible capture configurations, such as utilizing a few RGB inputs. Among these works, learning implicit functions \cite{iccv2020PIFu, saito2020pifuhd, hong2021stereopifu} has proven effective in achieving highly detailed reconstructions by integrating the advancements of deep neural networks. These methods employ large multi-layer perceptrons (MLPs) to predict the occupancy probability or truncated signed distance function (TSDF) value of every queried 3D point based on its associated local feature, which is extracted from images. They can recover a continuous surface at arbitrary resolutions without topology restrictions.


However, in typical MLP-based implicit networks, the occupancy or TSDF value at each location is solved independently with planar image features, rendering them less capable of addressing challenging cases such as occlusions. Consequently, these methods suffer from generalization and robustness issues, particularly when tackling strong occlusions caused by large motion or multiple interacting humans. 
Some follow-up studies  \cite{zheng2021deepmulticap,zheng2021pamir,huang2020arch} utilize an extra geometric model, SMPL~\cite{Loper2015}, to improve robustness by introducing strong shape priors. 
Their success typically relies on the assumption of geometrical similarity \cite{huang2020arch} between the shape prior and target reconstruction, making them intractable for handling complex cases with loose clothes and sensitive to errors in SMPL model fitting.



%\ping{this paragraph sounds like `TSDF is better than MLP/SMPL, and we use TSDF to solve the problem'. But in Sec 3, we are telling a different story, saying `MLP needs a 3D convolutional encoder'. We need to make these two sections consistent.}\sicong{I think in this paragraph we claim that the TSDF}


%We opt for Trucated Signed Distance Funtion (TSDF) volumetric representations as they are naturally suitable for convolution operations, which have shown remarkable performance for learning hierarchical features on 2D visual perception tasks \cite{SunXLW19}. 
%Meanwhile, TSDF also describes the gradual geometry change around shape surface, which is not reflected by occupancy volume. 

We instead revisit the 3D volumetric representation and resort to 3D convolutional neural networks (CNNs) for feature learning, due to their impressive performance in feature learning and the ability to incorporate spatial context. However, volumetric methods and 3D convolution involve discretization, which might raise concerns regarding whether a discretized volume can preserve subtle geometric details as continuous representations learned in implicit functions. We investigate the relationship between volume resolution and quantization error on synthetic data by converting target mesh objects to TSDF volumes, as shown in Figure~\ref{fig:quantization_error}. We observe that the quantization errors are significantly reduced by increasing volume resolution and become nearly negligible when reaching a relatively high resolution (e.g., 512 or higher). In other words, achieving fine-detailed reconstruction is not supposed to be restricted by the use of volume representations as long as a proper volume resolution is utilized. Therefore, we present a method with high-resolution feature volumes, e.g., 256 and 512, while traditional volumetric methods \cite{varol18_bodynet,gilbert2018volumetric} are often limited to much lower resolutions, such as 32 or 128.



On the other hand, an increase in volume resolution may lead to a cubic growth of memory overhead \cite{8100085}. Reducing memory costs while guaranteeing the granularity of volumetric representations is necessary for pursuing high-quality reconstruction. Thus, we adopt a coarse-to-fine approach and cull away irrelevant voxels to build a sparse high-resolution feature volume. At the coarse level, the network computes an initial TSDF by applying a U-Net with sparse 3D CNN \cite{3DSemanticSegmentationWithSubmanifoldSparseConvNet} on the sparse feature volume, which is carved by a visual hull. Through our experiments, it turns out that more than 95\% of the volume grids are discarded by the visual hull culling, making the sparse 3D CNN efficient. At the fine level, the network focuses on a narrow band near the zero-level set of the initial TSDF and discretizes the narrow band with smaller voxels. By employing this narrow-band culling, we further shrink the sampling space, resulting in a relatively small range of grid numbers (usually 300K--500K in our experiments) even with a high volume resolution of 512. The remaining voxels in the narrow band are associated with features that fuse high-frequency information from the computed normal maps upon the low-frequency shape from the coarse level to compute the TSDF at high resolution. The final mesh is then extracted from the TSDF using the Marching-Cube algorithm ~\cite{Lorensen87marchingcubes}.
% Different from the u-net sturcture to preserve global topology context, we then apply a shallow 3dcnn to compute the final TSDF $D_{final}$ which contain more local geometry detail.




% \ping{this paragraph can be expanded. It is an important contribution and often ignored by other works. stress on the novel idea of regressing blending weights instead of colors}

In addition to geometry, high-quality mesh texture is also a crucial factor contributing to visual appearance. Directly computing a color field in 3D space, as in \cite{iccv2020PIFu}, struggles to capture high-frequency texture details, while the neural radiance field (NeRF) \cite{yu2020pixelnerf} or the DoubleField~\cite{shao2022doublefield} require expensive per-instance optimization and are often unstable for sparse input images. In contrast, we adopt an image-based rendering approach to compute a texture atlas map, which is efficient and widely supported in existing computer graphics tools. 
Specifically, we compute a blending weight at each 3D point on the mesh surface to determine its color as a weighted average of the colors at its image projections. The blending weights can be computed at a relatively coarse resolution, e.g., 512 volume resolution in our case, and leave texture details to the high-resolution images, such as 1K or 2K. Unlike previous methods that generate blurry texturing results under sparse input, our method generalizes well on both synthetic and real data with just a few input views. 
Figure~\ref{fig:teaser} shows two examples reconstructed by our method. Despite the challenging garment, pose, and occlusion, our method recovers faithful shape, normal, and texture on the right.

%with a wide variety of poses and clothing styles, and it is also adaptive to handle input image with arbitrary resolutions.
%\sicong{For this concern we claim that when the resolution of dicretized volume meets certain threshold (which is 256 in our experiment), the quantization error can be neglected.} 



In summary, the main contributions of this paper are as follows:
\begin{itemize}
\vspace{-0.1in}
  \item 
  We revisit the 3D volumetric representation and demonstrate that it can support clothed human reconstruction with equal or even better performance compared to implicit representation. 
  \item 
  We develop a memory and computation-efficient method for high-resolution volumetric reconstruction using sophisticated sparse 3D CNN, coarse-to-fine estimation, and voxel culling by visual hull and narrow bands. 
  \item 
  We introduce a novel method to compute a texture atlas map, which captures rich appearance details from high-resolution input images.
  \item 
  We achieve impressive results on standard benchmark datasets Twindom and MultiHuman, significantly reducing the point-2-surface (P2S) precision to approximately 0.2cm from just six input views, with more than $50\%$ error reduction compared to the state-of-the-art methods, including DoubleField~\cite{shao2022doublefield} and PIFuHD~\cite{saito2020pifuhd}.
\end{itemize}
\section{Introduction}
\label{sec:intro}

The ability to express and reason about preferences over a set of alternatives is central to rational decision-making in a broad range of applications, including software design~\cite{vanLamsweerde:GORE,Liaskos:RE10,Sohrabi:AAAI2011,Sohrabi:ISWC10,fattah2021cpnet,abdulaziz:21},  public policy, e.g., city planning~\cite{CityPlanning:2016,DBLP:conf/birthday/SonPB14}, healthcare \cite{HealthCare:1998},  security \cite{Bistarelli:07,Gunasekharan:CISRC17}, privacy \cite{Oster:FACS2012}, among others.  In general, the preferences can be quantitative~\cite{Keeney:Camb97,French:EH86} or qualitative~\cite{Brafman:AIMag09,Doyle:AIMag99}. But 
stakeholders often find it natural to
express their preferences in qualitative terms \cite{Santhanam:book2016}, e.g., that a cheaper car is preferred to a more expensive car. Hence, there has been a growing interest in languages and tools for representing and reasoning with qualitative preferences\cite{Domshlak:AI2011,Santhanam:book2016,pcpnet:Cornelio:JAIR21}. %Qualitative preferences are expressed in terms of valuations of attributes of the alternatives under consideration. For instance, high reliability is preferred to low reliability. 
For example, ~\cite{Santhanam:AAAI2010} leverage advances in model checking ~\cite{Clarke:MIT2000,Queille:1982,Cimatti:CAV2002} to provide efficient and hence practically useful tools for reasoning with the qualitative preferences of single stakeholders \cite{CRISNER,Santhanam:Arxiv2015}. 

However, decision-making in real-world settings often needs to accommodate the preferences of \textit{multiple} stakeholders. Consider, for example, the task of choosing a care plan for a critically ill patient. The stakeholders, in this case, may include the patient concerned with their health outcome and the cost of care, the physician committed to ensuring that the patient receives the best care available, the family members with an interest in the patient's well-being, the hospital system seeking to maximize its profits, and the insurance provider seeking to minimize the reimbursements. 
A key challenge in extending the preference representation languages and reasoning tools from the single stakeholder setting to the multi-stakeholder setting has to do with maintaining, and reasoning with the (possibly conflicting) preferences of stakeholders. Furthermore, the preferences of some stakeholders in some settings may override those of others, e.g., due to their relative roles in an organization, or due to differences in their expertise as it relates to specific aspects of the application domain, etc. Ensuring transparency and accountability of decision-making requires that the system be able to explain {\em how} the stakeholders' preferences impact the outcomes. 

%\paragraph{Related Work. }
%Existing approaches to multi-stakeholder decision-making leverage voting-based social choice mechanisms~\cite{mcpnet,rossi:synthesis2011,mpcpnet}. This is useful when the preferences are expressed over outcomes (rather than attributes of outcomes); or when they are expressed over attributes of an outcome, they are rather simple (e.g., expressible using CP-nets). Much of this work has focused on voting strategies that are resistant to manipulation and yield fair outcomes \cite{mcpnet,rossi:synthesis2011,mpcpnet}.

%In contrast, we focus on settings that call for  multiple stakeholders to be able to express, explore and understand the implications of their preferences in settings where (i) the individual stakeholder preferences are  naturally expressed over attributes of outcomes (as opposed to outcomes themselves), and are  sufficiently nuanced to require more expressive preference languages e.g., TCP-nets \cite{Brafman:JAIR06} (which involve tradeoffs between conditional preferences), CI-nets \cite{Bouveret:IJCAI2009} (which can express preferences between sets of objects), or their generalizations \cite{Santhanam:book2016}; and (ii)  demand  explanations of the role played by the various stakeholder preferences in determining the outcomes of multi-stakeholder deliberations. 

\smallskip
\noindent\textbf{Contributions.\ } The key contributions of the paper are as follows:
(i) We provide the first formal treatment of reasoning with multi-stakeholder qualitative preferences. We consider the setting where the stakeholders express their preferences in a qualitative preference language, e.g., CP-net, CI-net, TCP-net, CP-Theory.
(ii) We introduce a query language for expressing queries with respect to the preferences of multiple stakeholders over outcomes that satisfy a set of specified criteria. 
(iii) We generalize the {\em induced preference graphs} that encode the qualitative preferences of a single stakeholder to {\em multi-stakeholder induced preference graphs} that encode the preferences of multiple stakeholders.  
(iv) We introduce and analyze several alternative semantics for such queries, motivated by the needs of different application scenarios, and examine their inter-relationships.  
(v)
We provide a provably correct algorithm for answering multi-stakeholder preference queries using model checking in $\mu$-calculus; and
(vi) We present results of experiments that demonstrate the feasibility of our approach.


\section{Qualitative Preference Languages}
\label{sec:preflanguage}
We consider settings in which stakeholders express preferences over a set of alternatives or outcomes, where each alternative is described by a set of attributes or (preference) variables. Stakeholders may directly express their preference between a pair of alternatives, by asserting that one valuation of the variables is preferred to another. In addition, preferences over sets of alternatives may be succinctly stated over (a) the possible valuations of each variable, i.e., \textit{intra-variable preference}; or (b) the variables themselves indicating their \textit{relative importance}.  Several qualitative preference languages with varying expressive power have been studied in the literature. For instance, CP-nets \cite{Boutilier:JAIR04} allow the expression of preferences over the valuations of each variable as a strict partial order, possibly conditioned on specific valuation(s) of one or more other variables. TCP-nets \cite{Brafman:JAIR06} extend CP-nets by additionally allowing expression of the relative importance of one variable over another. CP-theories \cite{Wilson:AAAI2004} further extend TCP-nets by allowing the expression of the relative importance of one variable over a set of variables. 

Formally, let $X = \{X_i \,|\, 0 < i \leq n\}$ be a set of preference variables, $D_i$ be the domain of $X_i$, and $v_i$ be the assignment of $X_i$ to a particular valuation in $D_i$. Let $O = \Pi_{X_i\in X} D_i$ be the set of alternatives or outcomes, and $O^P = \Pi_{X_i \in Y \subseteq X} D_i$ be the set of partial alternatives or outcomes. Each outcome $o \in O$ is represented as a tuple of valuations of each variable, i.e., $o = \langle v_1, v_2, \dots v_n \rangle$. We use the following notation to represent a preference statement 
\[
P:\ [c]\ (X_i = v_i) \succ (X_i = {v_i}')\  [Y]
\] 
where $c \in O^P$ is the condition under which this preference over $X_i$'s valuation holds, and $Y \subseteq X\setminus {X_i}$ is the set of variables less important than $X_i$. 
For brevity, we drop $[c]$ when $c = true$ and $[Y]$ when $Y = \emptyset$. A preference statement $P$ specifies that when $c$ holds, the valuation $v_i$ is preferred to ${v_i}'$ for variable $X_i$, regardless of the valuations and intra-variable preferences of the variables in $Y$.


\begin{example}
\label{ex:running}
Consider the preferences of a set of stakeholders  tasked with prioritizing vulnerabilities to be mitigated as part of protecting a critical network. Each vulnerability may be described by three variables describing the threats it poses, namely (a) attack complexity (A) with values Simple or Complex (indicating whether the complexity of the attack required to exploit the vulnerability is low or high); (b) exploit availability (E) with values Code or No-Code (indicating whether code to exploit the vulnerability is available); and (c) fix availability (F) for the vulnerability with values Fix or No-Fix (indicating whether a fix can be applied or not). Figure \ref{fig:pref-statements} shows some preferences with respect to these variables. Note that $P_5$ is a direct preference between two alternatives, $P_7$ is a relative importance preference, and the rest specify intra-variable preferences. Now consider three stakeholders, say, $1$, $2$, and $3$. Suppose stakeholder $1$ holds the preferences $P_1$ and $P_2$ of the \textit{incident-response} team whose overall goal is to prioritize readily exploitable vulnerabilities with no available fixes when initiating an immediate response, e.g., disconnecting critical systems from the network. Suppose stakeholder $2$ holds the preferences $P_3$, $P_4$ and $P_5$ of the \textit{patch-adaptation} team responsible for adapting existing fixes to address the vulnerability (hence has preferences conditioned on the fixed availability). Finally, suppose stakeholder $3$ holds the preferences $P_1$, $P_6$ and $P_7$ of the \emph{severity-assessment} team that aims to prioritize exploitable vulnerabilities based on their severity for action by the incident-response team.
\end{example}

%\vspace{-1em}
% Figure environment removed



% Figure environment removed
%\vspace*{-0.5in}


\noindent\textbf{Semantics of Preferences. }
\label{sec:ipg}
%\subsubsection{Preference Semantics}
The semantics of CP-nets, TCP-nets, and CP-theories is based on and extends the \textit{ceteris-paribus} principle \cite{Hansson:Springer1995}. The preference statements induce a strict partial order over the alternatives. For instance, for $o, o' \in O$, a preference statement $P$: $[c]$ $(X_i=v_i) \succ (X_i={v_i}')$ induces a preference from $o'$ to $o$ (denoted $o' \prefr o$) if both satisfy $c$; their valuations for $X_i$ are $v_i$ and ${v_i}'$ respectively; and their valuations for all other variables are identical.

\begin{definition}[Induced Preference Graph]
  Given a set of outcomes $O$ described by a set $AP$ of propositional variables, an induced preference graph $\ipg = (O\ \cup\ \{\bot\}, E, L)$ is defined over $O\cup\{\bot\}$ with an edge relation $E\subseteq (O\cup\{\bot\}) \times (O\cup\{\bot\})$ and a labeling function that maps each element in $O\cup\{\bot\}$ to a subset of propositional variables $L:(O\cup\{\bot\})\rightarrow \mathcal{P}(AP)$.  An edge
  $e=(o_1, o_2) \in E$ captures the fact that $o_1\prefr o_2$ and
  there exists a flip in the valuation of exactly one variable that
  contributes to this preference. For each $o\in O$, there exists 
  an edge ($\bot$, o), indicating that every outcome is preferred to $\bot$.
  Furthermore, $L(\bot) = \emptyset$ indicates that the $\bot$ does not satisfy
  any atomic proposition. 
  %QUESTION: why exactly one? -JT
  \label{def:ipg}
\end{definition}

\begin{definition}[Multi-Stakeholder Induced Preference Graph] A multi-stakeholder
  induced preference graph is an induced preference graph where each
  edge in the graph is annotated by the set of stakeholders whose preferences
  induce that edge. That is, $\ipg = (O\cup\{\bot\}, E, L, \mathcal{A})$ where the edge
  relation $E \subseteq (O\cup\{\bot\})\times \mathcal{P}(\mathcal{A}) \times (O\cup\{\bot\})$. An edge
  $e=(o_1, A, o_2) \in E$ captures the fact that $o_1\prefr o_2$ for
  every agent in $A$. We note that $e=(\bot, \mathcal{A}, o)$ for every $o\in O$.
  \end{definition}
  
\begin{example}
  \label{ex:ipg}
  The  (partial view of) induced preference graph of the preferences stated in Figure~\ref{fig:pref-statements} is given in Figure~\ref{fig:induced-pref-graph}. The edges correspond to flips from the less preferred to the more preferred alternative and are labeled with the preferences induced by the corresponding stakeholders. For instance, the edge from $o_4$ to $o_5$ is induced by the preference statement $P_2$ of stakeholder $1$. Similarly, the edge from $o_5$ to $o_6$ is induced by $P_5$ of stakeholder $2$ and the edge from $o_8$ to $o_2$ is induced by $P_7$ of stakeholder $3$. Note that some edges induced by stakeholder $3$'s preferences and the edges from $\bot$ to all of the outcomes are omitted for the sake of readability.
\end{example}
  
We will denote the edges in $\ipg$ as $o_1\trans{A} o_2$, where
$A$ is the set of agents whose preferences have induced the edge from
$o_1$ to $o_2$. 

\begin{definition}[$\prefr_A$ and $\prefr_A^+$]\label{def:path}
We write $o\prefr_A o'$ if there exists an edge
$o\trans{A'}o'$ and $A\cap A' \neq \emptyset$. Similarly, 
$o\prefr_A^+ o'$ if there exists a path
$o=o_1\trans{A_1}o_2\trans{A_2}\ldots \trans{A_k}o_{k+1}=o'$
where $\forall i\in [1,k]. (A\cap A_i)\neq \emptyset$.

\end{definition}
When $A$ is singleton ($A= \{a\}$), we will write
$o\prefr_a o'$.

%\input{lang}
\section{Single Stakeholder Preference Queries}
\label{sec:lang}

We first introduce a language for expressing queries with respect to single stakeholder qualitative preferences  before proceeding to consider multi-stakeholder preferences queries. A key feature of this language is that it allows expressing queries against preferences over properties of outcomes, rather than the outcomes themselves. Thus, it can readily accommodate preferences expressed in existing  qualitative preference languages such as CP-nets \cite{Boutilier:JAIR04}, TCP-nets \cite{Brafman:JAIR06}, and CP-theories \cite{Wilson:AAAI2004}. This allows us, for example, to query for outcomes with properties that are more preferred to all other outcomes. The resulting single stakeholder preference query language can express a range of preference queries (e.g., find the set of non-dominated outcomes) of common 
interest.\\

%\subsection{Syntax}
%\label{sec:syn}
\noindent\textbf{Syntax. }
The syntax of the query language is described over atomic
propositions, propositional constants, boolean connectives
and a (new) operator \texttt{P}: preference operator over properties. 
The language $\Psi$ is defined by the grammar: 
\[
\psi \rightarrow \true~|~\false~|~\texttt{AP}~|~\neg\psi~|~\psi\land\psi~|~\psi\lor\psi~|~\mlangpref{\psi}{\psi}{a}
\]
The answer to a query corresponds to the set of outcomes that belongs to the semantics of the query. 
For instance, all outcomes are returned for a query $\true$, while no outcome is returned for the query $\false$. A query involving an atomic proposition simply returns the outcomes that satisfy the proposition. Answers to queries involving Boolean connectives conform to the natural meaning of the connectives. The query $\mlangpref{\psi_1}{\psi_2}{a}$ returns the outcomes satisfying $\psi_1$ that are more preferred than outcomes satisfying $\psi_2$ based on the preferences of the stakeholder $a$.\\


%\subsection{Semantics}
%\label{sec:sem}
\noindent\textbf{Semantics. }
The semantics of the query language is defined over the set of outcomes (states) in the preference graph $\ipg$ induced by the given preferences. Let $\mathcal{I}$ be the set of all preference graphs that can be induced by single stakeholder preferences with respect to which single stakeholder queries can be expressed given the syntax described above.
%Recall that the states in the preference graphs encode 
%outcomes, whereas the edges denote the fact that the outcome at the
%destination of the edge is preferred to the outcome at the source of
%the edge as per the (flip) semantics of the preferences. 
We use the (semantic) function $\lsem{}{}:\Psi \times\ \mathcal{\ipg} \rightarrow
\mathcal{P}(O)$, to define the semantics of $\psi \in \Psi$ in the
context of an induced preference graph $\ipg \in \mathcal{\ipg}$.
That is, $\lsem{\psi}{\ipg}$ denotes the set of outcomes in
$\ipg$ that satisfy the query expressed using the formula $\psi$.
We will omit $\ipg$ from the definition unless it is explicitly
necessary to distinguish between semantics in the context of two different induced preference graphs.
\[
\begin{array}{rcl}
\lsem{\true}{} & = & O \\

\lsem{\false}{} & = & \emptyset \\

\lsem{p}{} & = & \{o~|~p \in L(o)\} \\

\lsem{\neg\psi}{} & = & O - \lsem{\psi}{} \\

\lsem{\psi_1\land\psi_2}{}  & = & \lsem{\psi_1}{} \ \cap\ \lsem{\psi_2}{} \\

\lsem{\mlangpref{\psi_1}{\psi_2}{a}}{} & = & 
\lsem{\psi_1}{}\ \cap \
\{o~|~\exists o'.o' \in \lsem{\psi_2}\ \land\ o \prefl^+_a o'\} \\
& & \ \ \ \ \ \ \ \ \cap \
 \{o~|~\forall o'.o' \in \lsem{\psi_2}\ \Rightarrow\ o \not\prefr^+_a o'\} \\
\end{array}
\]
Propositional constants $\true$ and $\false$ are satisfied by all and
no outcomes, respectively. The proposition $p$ is satisfied by any
outcome that satisfies $p$. The formulas over Boolean connectives
(negation, conjunction, disjunction) conform to the standard set-based
semantics (complement, intersection, union). The formula
$\mlangpref{\psi_1}{\psi_2}{a}$ is satisfied by outcomes that (i)
satisfy $\psi_1$, (ii) are preferred to at least one outcome that
satisfies $\psi_2$, and (iii) are not less preferred to any outcome that
satisfies $\psi_2$ by the stakeholder $a$. In short,  $\mlangpref{\psi_1}{\psi_2}{a}$ is the set of outcomes satisfying $\psi_1$ that are more preferred to
outcomes satisfying $\psi_2$ by the stakeholder $a$.\\

%\subsubsection{Single Stakeholder Preference Query Patterns}
%\label{sec:pracpattern}
\noindent The resulting query language can be used to express queries such as:

\begin{itemize}
\item
  What is the set of outcomes that are preferred  by the stakeholder $a$ to outcomes that satisfy $\psi$? The is expressed as $\mlangpref{\true}{\psi}{a}$.
  
\item What is the non-dominated set of outcomes relative to stakeholder $a$'s preferences? The query can be expressed as $\mlangpref{\true}{\true}{a}$. What is the non-dominated set of outcomes for stakeholder $a$ that satisfies $\psi$? This can be  expressed as $\mlangpref{\psi}{\true}{a}$.

\item With respect to stakeholder $a$'s preferences, what are the best improvements to outcomes satisfying $\psi$?  The query can be expressed as $(\mlangpref{\true}{\true}{a})\ \land\ (\mlangpref{\true}{\psi}{a})$.
  \end{itemize}

\begin{example}
  If $\psi = \texttt{Code}$, then the semantics
  of $\mlangpref{\true}{\psi}{1}$  (for stakeholder $1$)
  is the set of outcomes $\{o_1, o_5\}$. This is because, 
  while both $o_1$ and $o_5$ dominate some outcome satisfying
  $\texttt{Code}$  with respect to stakeholder $1$'s preferences, they are not dominated by any outcome that satisfies \texttt{Code}. On the other hand,
  the query $\mlangpref{\true}{\psi}{2}$  (for stakeholder $2$)
  yields the set $\{o_2, o_6\}$.
  \label{ex:basicquery}
\end{example}

\begin{example}
For stakeholder $1$, the non-dominated
set of outcomes is $\{o_1, o_5\}$ (result of the query: $\mlangpref{\true}{\true}{1}$), while for stakeholder $2$, the non-dominated set is $\{o_1, o_2, o_6, o_8\}$.
Note that the outcome $o_8$ neither dominates nor is  dominated by any outcome, However, it dominates $\bot$ and hence is included as part of the non-dominated set. 
  \label{ex:singlenondom}
\end{example}
\noindent\textbf{Cycles in Induced Preference Graphs.\ }
Cycles in an induced preference graph are indicative of inconsistencies
in the underlying preferences, the result being some outcome $o$ both more and less preferred to an outcome $o'$.
Does this pose any inconsistencies in the semantic interpretation of
$\mlangpref{\psi_1}{\psi_2}{a}$, when $o$ satisfies $\psi_1$ and $o'$ satisfies $\psi_2$?
The answer is no. This is because semantics of $\mlangpref{\psi_1}{\psi_2}{a}$ excludes all outcomes that are less preferred to outcomes satisfying $\psi_2$. Hence, the outcome $o$ will not be included in the set of outcomes returned by $\mlangpref{\psi_1}{\psi_2}{a}$ as it is less preferred to $o'$.

%\input{multis}
\section{Multi-Stakeholder Preference Queries}
\label{sec:multis}


We proceed to extend the preceding language for expressing preference queries  to allow preference queries with respect to the preferences of a set of stakeholders, as opposed to just a single stakeholder. Specifically, we add a new query construct $\mlangpref{\psi_1}{\psi_2}{A}$ where $A \subseteq \mathcal{A}$, where $\mathcal{A}$ is
the set of all stakeholders. When $A$ is a singleton $a$, we
use $\mlangpref{\psi_1}{\psi_2}{a}$ to denote the query about the preferences of a single stakeholder $a$ (as described in Section~\ref{sec:lang}).  In what follows, we  describe the semantics of multi-stakeholder preference queries  under several alternative interpretations of multi-stakeholder preferences.\\

\noindent\textbf{Consensus Semantics.}
%\label{sec:consensus}
Consensus semantics, as the name suggests, is defined as the set of
outcomes, whose preference over another set of outcomes, is decided by agreement among the set of stakeholders in question. Formally,
\[
\lsem{\mlangpref{\psi_1}{\psi_2}{A}}{}^{cs} = \displaystyle\bigcap_{a\in A} \lsem{\mlangpref{\psi_1}{\psi_2}{a}}{}
%\label{eq:consensus}
\]
%Recall that (from Section~\ref{sec:sem}), 
%\[
%\begin{array}{rcl}
%\lsem{\mlangpref{\psi_1}{\psi_2}{a}}{} & = & 
%\lsem{\psi_1}{}\ \cap \
%\{o~|~\exists o'.o' \in \lsem{\psi_2}\ \land\ o \prefl_a^+ o'\} \\
%& & \ \ \ \ \ \ \ \ \cap \
% \{o~|~\forall o'.o' \in \lsem{\psi_2}\ \Rightarrow\ o \not\prefr_a^+ o'\} \\
%\end{array}
%\]
\begin{example}
In Example~\ref{ex:basicquery}, as per the consensus semantics
the result of the query $(\mlangpref{\true}{\texttt{Code}}{\{1,2\}})$ is the
empty set as the stakeholders $1$ and $2$ do not agree on the outcomes
that are more desirable than outcomes satisfying \texttt{Code}. On the other hand,
stakeholders $1$ and $2$ agree on the non-dominated set $\{o_1\}$ computed
as the semantics of $\mlangpref{\true}{\true}{\{1,2\}}$ (see Example
\ref{ex:singlenondom}).

\label{ex:consensus}
% In Figure~\ref{induced-pref-graph}, node D represents an outcome where code is available, the attack is complex, and a fix is available. What is the set of outcomes that are more desirable than this outcome? Let $\psi$ denote the statement ``code is available, the attack is complex, and a fix is available" and the set of agents is $A=\{1,2\}$. Then the query can be formalized as $\mlangpref{\true}{\psi}{A}$. Based on the consensus semantics, the set of outcomes that satisfy the query includes only node B. 
\end{example}

\noindent\textbf{Collaborative Semantics. }
%\label{sec:collab}
Unlike consensus semantics, which requires a complete agreement
among the stakeholders, a collaborative semantics allows
the stakeholders to arrive at a compromise that is not disagreeable to any stakeholder. There are several ways to realize such a compromise that correspond to different interpretations of the semantics of $\mlangpref{\psi_1}{\psi_2}{A}$. Recall that   $\mlangpref{\psi_1}{\psi_2}{A}$ must return the set of outcomes that (i) satisfy $\psi_1$, (ii) are preferred to at least one outcome that satisfies $\psi_2$, and (iii) are not less preferred to any outcome that satisfies $\psi_2$.
We will refer to the last two conditions (ii and iii) as follows:
\begin{enumerate}
\item \emph{Witness Condition ($\mathtt{W}$)\ }  for determining the set of outcomes that are preferred to at least one outcome satisfying
$\psi_2$.
\item \emph{Agreement Condition ($\mathtt{A}$)\ } for determining
the set of outcomes that are not less preferred to any outcome satisfying
$\psi_2$.
\end{enumerate}
Each of these conditions can be collaboratively decided in two ways: 
 
\begin{enumerate}
\item \emph{\collaba\ Collaboration.} 
%The set of outcomes that are preferred to at least one outcome satisfying $\psi$ is computed  as follows:
The set of outcomes that are preferred to at least one outcome satisfying $\psi$ is chosen to be the \textbf{union} of outcomes preferred by each of the stakeholders
to {\em at least one outcome} satisfying $\psi$. 

\item \emph{\collabb\ Collaboration.\ }  
%The set of outcomes that are preferred to at least one outcome satisfying $\psi$ is computed as follows:
An outcome $o'$ is considered to be preferred to outcome $o$ when there exists a path
in the induced preference graph from $o$ to $o'$ where each edge along the path may be induced by the preferences of one or more stakeholders. Thus, there is no requirement that all of the edges along the path be induced by the preferences of the same stakeholder. Hence, the stakeholders collaboratively construct the path from $o$ to $o'$ by contributing one or more edges to the path based on their individual preferences. This can be viewed as \textbf{chaining}
induced preference edges of different
stakeholders to arrive at the result.
\end{enumerate}
%It is immediate that any conclusion resulting from
%\collaba\ collaboration is also realizable from the
%\collabb\ collaboration, and not vice versa. Hence, we can state that
%\collaba\ collaboration is stricter or stronger condition for
%preference decision than \collabb\ collaboration.  
\collabb\ Collaboration is useful in situations
where each stakeholder may not have complete information or expertise to determine a dominance relation between a pair of outcomes but they may be able to collaborate to arrive at a conclusion. For instance, healthcare providers (doctors, nurses) and hospital administrators may collaborate to develop an optimal placement strategy for hand sanitizers in the hospital. The healthcare providers present their preferences based on their knowledge of the usage of hand sanitizers at different times and locations, whereas the hospital administrators present their preferences based on the cost of procuring hand sanitizers. 

Now we have two different choices for the witness ($\mathtt{W}$) condition and agreement ($\mathtt{A}$) condition: 
\begin{itemize}
    \item[$\mathtt{W}_1$.\ ] \collaba\ collaboration for deciding witness condition for $\psi_2$
    in $\mlangpref{\psi_1}{\psi_2}{A}$:
    \[
    \displaystyle\bigcup_{a\in A}\{o~|~\exists o'.o' \in \lsem{\psi_2}\ \land\ o \prefl_a^+ o'\} 
    \]
    \item[$\mathtt{W}_2$.\ ] \collabb\ collaboration for deciding witness condition for $\psi_2$
    in $\mlangpref{\psi_1}{\psi_2}{A}$:
    \[
    \{o~|~\exists o'.o' \in \lsem{\psi_2}\ \land\ o \prefl_A^+ o'\} 
    \]
    \item[$\mathtt{A}_1$.\ ] \collaba\ collaboration for deciding agreement condition for $\psi_2$
    in $\mlangpref{\psi_1}{\psi_2}{A}$:
    \[
    \begin{array}{ll}
    & O\setminus \displaystyle\bigcup_{a\in A}\{o~|~\exists o'.o' \in \lsem{\psi_2}\ \land\ o \prefr_a^+ o'\} \\[1.25em]
    = & \displaystyle\bigcap_{a\in A}\{o~|~\forall o'.o' \in \lsem{\psi_2}\ \Rightarrow\ o \not\prefr_a^+ o'\} 
    \end{array}
    \]
    
    \item[$\mathtt{A}_2$.\ ] \collabb\ collaboration for deciding agreement condition for $\psi_2$:
    in $\mlangpref{\psi_1}{\psi_2}{A}$
    \[
    \begin{array}{ll}
    & O\setminus \{o~|~\exists o'.o' \in \lsem{\psi_2}\ \land\ o \prefr_A^+ o'\}  \\[0.5em]
    = &
    \{o~|~\forall o'.o' \in \lsem{\psi_2}\ \Rightarrow\ o \not\prefr_A^+ o'\}
    \end{array}
    \]
\end{itemize}

\begin{example}
Consider the induced preference graph in Figure~\ref{fig:induced-pref-graph}.
For stakeholder $1$, the set of outcomes that dominate the outcomes
satisfying \texttt{No-Code} is $\{o_1, o_2, o_4, o_5\}$. This is because
$o_4\prefl_1 o_7$, $o_5\prefl_1 o_4$, $o_2\prefl_1 o_3$ and $o_1\prefl_1 o_2$.
On the other hand, for stakeholder $2$, the set of outcomes that
dominate the outcomes satisfying \texttt{No-Code} is $\{o_2, o_3, o_4\}$. 

Therefore, for $\psi_2 = \texttt{No-Code}$,  we have:
\[
\mathtt{W}_1: \displaystyle\bigcup_{a\in \{1,2\}}\{o~|~\exists o'.o' \in \lsem{\psi_2}\ \land\ o \prefl_a^+ o'\}  = \{o_1, o_2, o_3, o_4, o_5\}
\]
On the other hand, %for $\psi_2 = \texttt{No-Code}$,  
\[
\mathtt{W}_2: \{o~|~\exists o'.o' \in \lsem{\psi_2}\ \land\ o \prefl_{\{1,2\}}^+ o'\} = \{o_1, o_2, o_3, o_4, o_5, o_6\}.
\]
Note that the set includes
all outcomes whose inclusion is decided by stakeholders $1$ and $2$
on their own. Additionally, outcome $o_6$ is included because
$o_4\prefl_1 o_7$, $o_5\prefl_1 o_4$ and $o_6\prefl_2 o_5$. 
\label{ex:w1w2}
\end{example}
\begin{example}
For the induced preference graph in Figure~\ref{fig:induced-pref-graph}, 
consider evaluating the agreement condition. The set of outcomes that
are dominated by outcomes satisfying \texttt{No-Code} as per the stakeholder
$1$ is $\emptyset$. On the other hand, for stakeholder $2$, the set is
$\{o_5, o_7\}$ because $o_6\prefl_2 o_5$ and $o_3\prefl_2 o_7$. 

Therefore, for $\psi_2 = \texttt{No-Code}$,  
\[
\mathtt{A}_1: O\setminus \displaystyle\bigcup_{a\in \{1,2\}}\{o~|~\exists o'.o' \in \lsem{\psi_2}\ \land\ o \prefr_a^+ o'\} = O\setminus\{o_5, o_7\}.
\]
On the other hand, %for $\psi_2 = \texttt{No-Code}$,  
\[
\mathtt{A}_2: O\setminus \{o~|~\exists o'.o' \in \lsem{\psi_2}\ \land\ o \prefr_{\{1,2\}}^+ o'\}  = 
O\setminus \{o_4, o_5, o_7\}.
\]
The membership of
$o_4$ is decided from the relations: $o_6\prefl_2 o_5$  and $o_5\prefl_1 o_4$.
\label{ex:a1a2}
\end{example}


The combinations of $\mathtt{W}_1$ and $\mathtt{W}_2$ with $\mathtt{A}_1$ and $\mathtt{A}_2$  yield  four different
semantics for $\mlangpref{\psi_1}{\psi_2}{A}$. We will denote them by
$\lsem{\mlangpref{\psi_1}{\psi_2}{A}}{}^{\mathtt{W}_i\mathtt{A}_j}$ where
$i, j \in \{1,2\}$.


\begin{example}
Using the Examples~\ref{ex:w1w2} and \ref{ex:a1a2}, we have the following
when $\psi_1=\true$ and $\psi_2=\texttt{No-Code}$: 
\[
\begin{array}{llll}
\lsem{\mlangpref{\psi_1}{\psi_2}{\{1,2\}}}{}^{\mathtt{W}_1\mathtt{A}_2} & = &  \{o_1, o_2, o_3\}, \\%[1em]

\lsem{\mlangpref{\psi_1}{\psi_2}{\{1,2\}}}{}^{\mathtt{W}_1\mathtt{A}_1} & = &  \{o_1, o_2, o_3, o_4\}, \\%[1em]

\lsem{\mlangpref{\psi_1}{\psi_2}{\{1,2\}}}{}^{\mathtt{W}_2\mathtt{A}_2} & = & \{o_1, o_2, o_3, o_6\}, \\%[1em]

\lsem{\mlangpref{\psi_1}{\psi_2}{\{1,2\}}}{}^{\mathtt{W}_2\mathtt{A}_1} & = &  \{o_1, o_2, o_3, o_4, o_6\}. 
\end{array}
\]
\label{ex:collab}
\end{example}

\noindent\textbf{Relationships Between Alternative Collaborative Semantics.} The following Theorem shows the relationship between the two witness conditions and the relationship between the two agreement conditions. 
\begin{theorem}\label{lemma:witness}
$\mathtt{W}_1 \subseteq \mathtt{W}_2$ and $\mathtt{A}_2 \subseteq \mathtt{A}_1$. 
\end{theorem}

%\begin{proof}
\noindent\emph{Proof.\ }
(i) $\mathtt{W}_1 \subseteq \mathtt{W}_2$. \\
Consider any 
$o_1\in \mathtt{W}_1$. Then, $o_1 \in \bigcup_{a\in A}\{o~|~\exists o'.o' \in \lsem{\psi_2}\ \land\ o \prefl_a^+ o'\}$ by the definition of $\mathtt{W}_1$, 
Thus, there is an agent $a_1\in A$ and an outcome $o_2\in \lsem{\psi_2}{}$ such that $o_1\prefl_a^+ o_2$. Since $a\in A$, it then follows from the Definition~\ref{def:path} that 
$o_1 \prefl_A^+ o_2$. 
Therefore, $o_1\in \mathtt{W}_2$, by the definition of $\mathtt{W}_2$. 

(ii) $\mathtt{A}_2 \subseteq \mathtt{A}_1$.
We first show that 
\begin{equation}\label{2023-07-29}
\begin{split}
& \bigcup_{a\in A} \{o~|~\exists o'.o' \in \lsem{\psi_2}\ \wedge\ o \prefr_a^+ o'\}\\
\subseteq \, & \, 
\{o~|~\exists o'.o' \in \lsem{\psi_2}\ \wedge\ o \prefr_A^+ o'\}.
\end{split}
\end{equation}
For any $o_1\in \bigcup_{a\in A} \{o~|~\exists o'.o' \in \lsem{\psi_2}\ \wedge\ o \prefr_a^+ o'\}$,  there is an agent $a_1\in A$ and an outcome $o_2\in \lsem{\psi_2}{}$ such that $o_1\prefr_a^+ o_2$. Then, $o_1\prefr_A^+ o_2$ by Definition~\ref{def:path} because $a_1\in A$. Hence, statement~\eqref{2023-07-29} is true. 
Thus, it follow from the definitions of $\mathtt{A}_1$ and $\mathtt{A}_2$ that
\begin{equation*}
\begin{split}
\mathtt{A}_2 =\, & \{o~|~\forall o'.o' \in \lsem{\psi_2}\ \Rightarrow\ o \not\prefr_A^+ o'\}\\
= \, & \, O \setminus 
\{o~|~\exists o'.o' \in \lsem{\psi_2}\ \wedge\ o \prefr_A^+ o'\}\\
\subseteq \, & \, O \setminus 
\bigcup_{a\in A} \{o~|~\exists o'.o' \in \lsem{\psi_2}\ \wedge\ o \prefr_a^+ o'\}\\
= \, & \bigcap_{a\in A} \{o~|~\forall o'.o' \in \lsem{\psi_2}\ \Rightarrow\ o \not\prefr_a^+ o'\} = \mathtt{A}_1.  \phantom{AAAAA}\Box
\end{split}
\end{equation*}
%\hfill$\Box$
%\end{proof}

 The above theorem leads to the  relationship between different semantics of the query
 as illustrated in the Figure~\ref{fig:relative}.



% Figure environment removed


\begin{comment}
The following theorem follows directly from Lemma ~\ref{lemma:witness}.
%\noindent The proof is presented in the appendix. 

\begin{theorem}
The result of \collaba\ Collaboration is also realizable by
\collabb\ Collaboration, but not the other way around.   
\begin{itemize}
\item $\lsem{\mlangpref{\psi_1}{\psi_2}{A}}{}^{\mathtt{W}_1\mathtt{A}_2}\subseteq \lsem{\mlangpref{\psi_1}{\psi_2}{A}}{}^{\mathtt{W}_1\mathtt{A}_1}$.
\item $\lsem{\mlangpref{\psi_1}{\psi_2}{A}}{}^{\mathtt{W}_1\mathtt{A}_2}\subseteq \lsem{\mlangpref{\psi_1}{\psi_2}{A}}{}^{\mathtt{W}_2\mathtt{A}_2}$.
\item $\lsem{\mlangpref{\psi_1}{\psi_2}{A}}{}^{\mathtt{W}_1\mathtt{A}_1}\subseteq \lsem{\mlangpref{\psi_1}{\psi_2}{A}}{}^{\mathtt{W}_2\mathtt{A}_1}$.
\item $\lsem{\mlangpref{\psi_1}{\psi_2}{A}}{}^{\mathtt{W}_2\mathtt{A}_2}\subseteq \lsem{\mlangpref{\psi_1}{\psi_2}{A}}{}^{\mathtt{W}_2\mathtt{A}_1}$.
\end{itemize}
\end{theorem}
\end{comment}


%\input{mu-multi}
\section{Answering Preference Queries}
\label{sec:mu-multi}
We now proceed to show how to answer multi-stakeholder preference queries.  Specifically, we show that  multi-stakeholder preference queries can be reduced to evaluating a corresponding alternation-free modal $\mu$-calculus expression. This allows us to take advantage of the state-of-the-art tools for $\mu$-calculus model-checking to efficiently answer multi-stakeholder preference queries.\\
%\subsection{Alternation-free Modal $\mu$-Calculus}
%\label{sec:mucalc}

\noindent\textbf{Modal $\mu$-calculus.} Modal $\mu$-calculus \cite{Koz83,EJS01}, ${\mathcal L}_{\mu}$, extends propositional modal logic by adding the least and the greatest fixed point operators.  $L_\mu$  uses explicit fixed point and modal operators to express temporal properties over events and states in a labeled transition system.  {\em Labeled transition systems} consist of a set of states,  a transition relation over state-pairs parameterized with events (transition annotations) and a labeling function that maps each state to a set of propositions that hold in that state. 
%In the domain of verification, such a transition system is used to describe the behavioral specification of  dynamic systems, where the states correspond to configurations of the system, the transitions correspond to the evolution of the system from one configuration to another and the annotations of transitions correspond to the actions or events that lead to the evolution. 
It is easy to see that an induced preference graph can be viewed as a labeled transition system over $O\cup\{\bot\}$, an annotated transition relation (edges being annotated with the set of stakeholders), and a labeling function mapping each outcome to the set of propositions satisfied by the outcome. The primary difference is that the edge-annotation is a set (in an induced preference graph) rather than a symbol (in a labeled transition systems). Note, however, that such a difference is purely syntactical as we can replace an edge annotated with a set by a set of edges, where each edge in the set is annotated by a distinct member of the set. 
We will use `states' and `outcomes' interchangeably in refering to an induced preference graph interpreted as a labeled transition system. \\

\noindent\textbf{Syntax of Modal $\mu$-calculus. } The syntax of $\mu$-calculus involves propositional constants, atomic propositions, modalities, fixed
point variables and expressions and Boolean connectives: 
\[
\small
\begin{array}{lr}
\phi \rightarrow & \true~|\false~|~\texttt{AP}~|~\neg\phi~|~\phi\land\phi~|~\phi\lor\phi~|~\moddiam{A}\phi%\\[0.5em]
%& \hfill 
~|~Z~|~\mu Z.\phi
\end{array}
\]

In the above, the parameter $A$ of the modal operator ($\moddiam{A}$) is associated with the edge annotation of the labeled transition system.  In an induced preference graph, each edge is annotated with a subset of all stakeholders. In our context, in the modal operators, $A$ will represent a set of stakeholders. When $A$ is singleton such as $A=\{a\}$, we will denote the modal condition as $\moddiam{a}$.\\

\noindent\textbf{Semantics of Modal $\mu$-calculus. } The semantics of $\mu$-calculus formula is given in terms of a set of
states in a labeled transition system that satisfy the formula.
The semantics is specified by the function  $\msem{\ }{}:\Phi\times
\mathcal{E}\times \mathcal{\ipg}\rightarrow \mathcal{P}(O)$ where
$\mathcal{E}$ is the power set of mappings of fixed point variables to
outcomes in $O$. This mapping is referred to as the environment:
$e:\mathcal{Z} \rightarrow \mathcal{P}(O)$; $\mathcal{Z}$ being the
set of fixed point variables in the formula whose semantics is being
evaluated. We will use the notation $e[Z\mapsto O']$ to denote the
environment where the mapping of fixed point variable $Z$ in $e$ is
updated to $O'\subseteq O$. We omit $\ipg$ when it is not necessary to distinguish between different induced preference graphs.

Figure \ref{fig:musem} shows the semantics of $\mu$-calculus. The propositional constants $\true$ and $\false$ are satisfied by all states and no states, respectively. The atomic proposition $p$ is satisfied in all states whose labeling includes $p$. The formula
$\varphi_1\land\varphi_2$ is satisfied by all states that satisfy both $\varphi_1$ and $\varphi_2$. The formula $\moddiam{A}\varphi$ is satisfied by any state which has at least one next state (reachable via an edge annotated with a set that has a non-empty intersection with $A$) that satisfies $\varphi$. 

% Figure environment removed

   
%On the other hand, the box-modality formula $\modbox{A}\varphi$ is satisfied by any state
%whose all next states (reachable via edges annotated with sets that have non-empty intersection with %$A$) 
%satisfy $\varphi$; note that a state with no outgoing
%transition that is annotated with set containing elements from $A$ 
%also satisfies $\modbox{A}\varphi$. 
%Note that $\neg\moddiam{A}\varphi$ is equivalent to $\modbox{A}\neg\varphi$
%and vice versa. 
The semantics of fixed
point variable $Z$ is given by the environment mapping $e$.  The
semantics of least fixed point formula $\mu Z.\varphi$ is computed by the $|O|$ applications of function $f_{Z, \varphi, e}$ on $\emptyset$
(Tarski-Knaster fixed point theorem~\cite{Tarski}). We omit the greatest fixed point construct as its semantics can be realized using the least fixed point and negation. 

Model checking a labeled transition system against
a given $\mu$-calculus formula amounts to identifying
the set of states in the transition system that
belong to the semantics of the $\mu$-calculus formula.
\\

\noindent\textbf{Alternation-Free Modal $\mu$-calculus. }For our purposes, it turns out that we only need the  alternation-free  fragment ${\mathcal L}^{af}_{\mu}$ \cite{emerson1986efficient} of ${\mathcal L}_\mu$. An attractive property of ${\mathcal L}^{af}_{\mu}$ is that in it there is no real interaction between least and greatest fixpoint operators \cite{marti2021focus}, which, at the expense of reduced expressive power relative to ${\mathcal L}_\mu$, yields more efficient reasoning  \cite{marti2021focus,EJS01}.\\

\noindent\textbf{Translating Query Language to $\mu$-calculus. }
We present a strategy to evaluate the proposed preference queries using model checking. We will augment the induced preference graph which encodes a labeled transition system with additional reverse edges; this will help in explaining the answers to multi-stakeholder preference queries in relation to the stakeholder preferences and the chosen semantics; however, in the implementation, such reverse edges can be handled implicitly.  For every edge from $o_j$ to $o_i$ due to preference $o_i\prefl_a o_j$ of stakeholder $a$,  we will add a reverse edge from $o_i$ to $o_j$. 

% Figure environment removed



Therefore, the set $\{o~|~\exists o'.o' \in \lsem{\psi}\ \land\ o \prefl^+_a o'\}$
can be expressed in $\mu$-calculus as:
$
\mu Z.(\moddiam{a}_r\psi \ \lor \ \moddiam{a}_r Z)
$.
The semantics captures the set of states which can reach some state
satisfying $\psi$ via one or more reverse edges; the modal requirement
$\moddiam{a}_r$ is satisfied using reverse edges annotated with $a$.

\begin{example}
Consider the formula 
$\mu Z. (\moddiam{1}_r\texttt{Code} \ \lor \ \moddiam{1}_r Z)$ representing the set of all states that have a path to a state satisfying \texttt{Code} via one or more edges annotated with $1$. 
%To illustrate fixed point computation, 
We evaluate this expression using the induced preference graph shown in Figure~\ref{fig:induced-pref-graph}. 

Let $\varphi$ denotes $(\moddiam{1}_r\texttt{Code} \ \lor \ \moddiam{1}_r Z)$. 
Therefore,  $\msem{\mu Z.\varphi}{e} = f^{8}_{Z,\varphi,e}(\emptyset)$ where $f_{Z,\varphi,e}(O') = \msem{\varphi}{e[Z\mapsto O']}$.
\[
\begin{array}{l}
f_{Z,\varphi,e}(\emptyset) = \msem{\moddiam{1}_r\texttt{Code} \ \lor \ \moddiam{1}_r Z}{e[Z\mapsto \emptyset]} \\ [1em]

= \msem{\moddiam{1}_r\texttt{Code}}{e[Z\mapsto\emptyset]} \ \cup\ \msem{\moddiam{1}_r Z}{e[Z\mapsto \emptyset]} \\ [1em]

= \{o~|~\exists o'.(o\trans{A}_r o'\ \land\ A \cap \{1\} \neq \emptyset)\ \land\ o' \in \msem{\texttt{Code}}{e[Z\mapsto \emptyset]}\} \\
\ \ \cup \\ 
\ \ \ \ \{o~|~\exists o'.(o\trans{A}_r o'\ \land\ A \cap \{1\} \neq \emptyset)\ \land\ o' \in \msem{Z}{e[Z\mapsto \emptyset]}\} \\
\mbox{where} \trans{A}_r \mbox{ denotes reverse edge relations} \\ [1em]


= \{(o\trans{A}_r o'\ \land\ A \cap \{1\} \neq \emptyset)\ \land\ o'\in \{o_1,o_2,o_4,o_5\}\} \\
\ \ \cup \\
\ \ \ \ \{o~|~\exists o'.(o\trans{A}_r o'\ \land\ A \cap \{1\} \neq \emptyset)\ \land\ o' \in \emptyset\} \\[1em] 

= \{o_1,o_5\} \ \cup\ \emptyset = \{o_1, o_5\}
\end{array}
\]
Proceeding further
\[
\begin{array}{l}
f^2_{Z,\varphi,e}(\emptyset) = f_{Z, \varphi, e}(f_{Z, \varphi, e}(\emptyset) = f_{Z,\varphi,e}(o_1, o_5) \\[1em]

= \msem{\moddiam{1}_r\texttt{Code}}{e[Z\mapsto\{o_1,o_5\}]} \ \cup\ \msem{\moddiam{1}_r Z}{e[Z\mapsto \{o_1,o_5\}]} \\[1em]

= \{o~|~\exists o'.(o\trans{A}_r o'\ \land\ A \cap \{1\} \neq \emptyset)\ \land\ o' \in \msem{\texttt{Code}}{e[Z\mapsto \{o_1,o5\}]}\} \\
\ \ \cup \\ 
\ \ \ \ \{o~|~\exists o'.(o\trans{A}_r o'\ \land\ A \cap \{1\} \neq \emptyset)\ \land\ o' \in \msem{Z}{e[Z\mapsto \{o_1,o_5\}]}\} \\ 
\mbox{where} \trans{A}_r \mbox{ denotes reverse edge relations} \\ [1em]
\end{array}
\]

\[
\begin{array}{l}
= \{(o\trans{A}_r o'\ \land\ A \cap \{1\} \neq \emptyset)\ \land\ o'\in \{o_1,o_2,o_4,o_5\}\} \\
\ \ \cup \\
\ \ \ \ \{o~|~\exists o'.(o\trans{A}_r o'\ \land\ A \cap \{1\} \neq \emptyset)\ \land\ o' \in \{o_1,o_5\}\} \\[1em]

= \{o_1,o_5\} \ \cup \emptyset = \{o_1,o_5\} 
\end{array}
\]
The (least) fixed point is reached as further application of $f$ onto itself will not alter the
result. 


\label{ex:mu-single}
\end{example}

Similarly, the set $\{o~|~\forall o'.o' \in \lsem{\psi}\ \Rightarrow\ o \not\prefr^+_a o'\}$
is equal to $O\setminus \{o~|~\exists o'.o' \in \lsem{\psi}\ \land\ o \prefr^+_a o'\}$,
which  can be expressed in $\mu$-calculus as:
$
\neg\mu Z.(\moddiam{a}\psi\ \lor\ \moddiam{a}Z)
$.
This semantics yields the set of states which have no path to any state that satisfies $\psi$.
Hence, a query of the form $\mlangpref{p}{q}{a}$, where $p$ and $q$ are atomic
propositions, can
be expressed in $\mu$-calculus as:
\[
p \ \land \ \left[\mu Z.(\moddiam{a}_r q \ \lor \ \moddiam{a}_rZ)\right] \ \land\
\left[\neg\mu Z.(\moddiam{a}q\ \lor\ \moddiam{a}Z) \right]
\]



Now, in \textit{\collaba\ Collaboration},
the set of outcomes that dominate at least one outcome satisfying $\psi$
for a set $A$ of stakeholders is given by
$
\bigcup_{a\in A} \{o~|~\exists o'.\ o' \in \psi\ \land \ o\prefl_a^+ o'\}
$
which in turn is reflected by the semantics of the $\mu$-calculus formula:
$
\bigvee_{a\in A} \left(\mu Z.(\moddiam{a}_r\psi\ \lor\ \moddiam{a}_rZ)\right)
$.
The preceding formula identifies the set of outcomes that have path(s) to some outcome satisfying $\psi$ in the transpose-induced preference graph (i.e., using reversed edges)
$\ipg$. Along each path that decides reachability, each of the edges must be 
annotated by the same $a$.

On the other hand, in the \textit{\collabb\ Collaboration}, the domination of outcomes over at least one outcome satisfying
$\psi$ for a set $A$ of stakeholders is decided by
$
\{o~|~\exists o'.\ o' \in \psi\ \land \ o\prefl_A^+ o'\}
$
which in turn corresponds to the semantics of the $\mu$-calculus formula
$
\mu Z.(\moddiam{A}_r\psi\ \lor\ \moddiam{A}_rZ)
$.
This denotes the set of outcomes that have path(s) to some outcome satisfying $\psi$ in the transpose-induced preference graph $\ipg$; the reachability
is determined by the edges annotated by at least one
element from $A$.

Thus, the Witness and Agreement Conditions can be
expressed in $\mu$-calculus as follows:
\[
\begin{array}{rl}
\mathtt{W}_1: & \mbox{semantics of } \displaystyle\bigvee_{a\in A} \left(\mu Z.(\moddiam{a}_r\psi\ \lor\ \moddiam{a}_rZ)\right) \\[1em]
\mathtt{W}_2: & \mbox{semantics of }  \mu Z.(\moddiam{A}_r\psi\ \lor\ \moddiam{A}_rZ) \\[1em]
\mathtt{A}_1: & \mbox{semantics of }  \displaystyle\bigwedge_{a\in A} \left(\neg\mu Z.(\moddiam{a}\psi\ \lor\ \moddiam{a}Z)\right) \\[1em]
\mathtt{A}_2: & \mbox{semantics of }  \neg\mu Z.(\moddiam{A}\psi\ \lor\ \moddiam{A}Z) \\
\end{array}
\label{eq:WitAgreemucalc}
\]

Figure~\ref{fig:translation} shows the translation function that, given an expression in the multi-stakeholder preference query language and the chosen multi-stakeholder preference semantics as arguments, outputs the corresponding ${\mathcal L}^{af}_{\mu}$  expression. The run-time for translation is linear in the size ($n$) of the number of operators ($\land,\lor,\neg,\mathtt{P}$) in the query. The size of the translation is of the order $O(|\mathcal{A}|^k\times n^k)$, where
$|\mathcal{A}|$ is the number of stakeholders, $k$ the \emph{nesting depth} of the queries of the form $\mlangpref{\psi_1}{\psi_2}{A}$ and $n$ the size of the query. For instance, for a query of the form $(\mlangpref{p}{(\mlangpref{q}{r}{B})}{A})$, $n$ and $k$ are both equal to $2$.  The run-time for model checking ${\mathcal L}^{af}_{\mu}$ formula is linear in the size of the formula and the state space of the labeled transition
system (induced preference graph). We expect  the nesting depth of the query to be reasonably small and the run-time will be determined largely by the number of stakeholders in the query and the
size of the number of outcomes (size of the induced preference graph). Note, however, that the number of outcomes is exponential in the number of attributes describing the outcomes, as in the case of reasoning with qualitative preferences \cite{Goldsmith:JAIR08}. 

The following theorem establishes the correctness of reduction of multi-stakeholder preference queries to ${\mathcal L}^{af}_{\mu}$ expressions.
\begin{theorem}
For a multi-stakeholder preference query $\psi$ 
(as described in Section \ref{sec:multis}), $o\in\lsem{\psi}{\ipg}^{i}$
if and only if $o\in\msem{Tr^i(\psi)}{\ipg}$, where $\ipg$
is the preference graph induced by the stakeholder preferences and $i$ denotes
the type (consensus or variants of collaborative) of semantics used to answer
$\psi$. 
\label{thm:singlecorrectness}
\end{theorem}

The proof of Theorem \ref{thm:singlecorrectness} proceeds by induction over the structure of the mult-stakeholder preference query. 

%\section{Implementation} \label{sec:impl}

In this section, we describe our prototype implementation of
a refinement type checking and inference system, \textsc{RCaml}\footnote{available at \url{https://github.com/hiroshi-unno/coar}}.
It takes a program written in a subset of the OCaml 5 language
(including algebraic data types, pattern matching, recursive functions, exceptions, references,
let-polymorphism, and effect handlers)
and a specification of its main function represented as a refinement type.
It first (1) obtains an ML-typed AST of the program
using OCaml's compiler library,
(2) infers refinement-free operation signatures and control effects,
(3) generates refinement constraints for the program and its specification as Constrained Horn Clauses (CHCs) (see e.g., \cite{Bjorner2015a}),
and finally (4) solves these constraints
to verify whether the program satisfies the specifications.
The steps (3) and (4), where the refinement type checking is reduced to CHC solving,
follow existing standard approach such as \cite{Unno09,Rondon08}.
The inference of (refinement-free) operation signatures is similar to
that of record types using row variables,
and is mutually recursive with the inference of control effects.
It is based on the type inference system of control effects for shift0/reset0
proposed by \citet{Materzok11}.
%
As we split the steps of CHC generation and CHC solving,
we can use different kinds of solvers as the backend CHC solver depending on benchmarks.
In this experiment, we used two kinds of CHC solvers:
\textsc{Spacer}~\cite{Komuravelli13} that is based on Property Directed Reachability (PDR)\cite{Bradley11,Een11} and CEGAR\cite{Clarke00},
and \textsc{PCSat}~\cite{Unno2021} that is based on template-based CEGIS\cite{Solar-Lezama06,Unno2021} with Z3~\cite{Moura2008} as an SMT solver.

Because inputs to the implementation are OCaml programs
that are type-checked by OCaml's type checker which does not allow ATM,
the underlying OCaml types of the answer types cannot be modified.
However, as remarked before in Section~\ref{sec:intro}, our aim is to verify \emph{existing} programs with algebraic effects and handlers,
and, as remarked before, our ARM, that is only modification in the refinements, is useful for that purpose.

Our implementation supports several kinds of polymorphism.
In addition to the standard let-polymorphism on types,
it supports refinement predicate polymorphism.
The implementation extends the formal system by allowing {\em bounded} predicate polymorphism in which abstracted predicates can be bounded by constraints on them, and further allows predicate-polymorphic types to be assigned to let-bound terms.
However, because the implementation can infer predicate-polymorphic types only at let-bindings,
we used a different approach, which we will discuss in Section~\ref{sec:impl/eval},
to simulate predicate polymorphism in operation signatures.

%
Another notable point is that our implementation deals with operations and exceptions uniformly.
That is, exception raising is treated as an operation invocation
and it can be handled by a certain kind of effect handlers which have clauses for exceptions
(the exception clauses are included in the effect handlers of OCaml by default).

\subsection{Evaluation} \label{sec:impl/eval}

We performed a preliminary experiments to evaluate our method
on some benchmark programs that use algebraic effect handlers.
The benchmarks are based on example programs
from \citet{Bauer15} and the repository of the Eff language~\cite{Effrepo}.
We gathered the effect handlers in those examples
and created benchmark programs each of which uses one of the effect handlers.
We also added a refinement type specification of the main function to each benchmark.
(Other functions are not given such extra information,
and so their types are \emph{inferred automatically} even for recursive functions.)
Also, we added small amount of  annotations to the benchmarks.
Most benchmarks could be solved automatically without the annotations,
but some needs them as hints.
We discuss the details at the end of this section.
We refer to the supplementary material
for the concrete source codes and the specifications of our benchmarks.
Particularly, an interesting one (\texttt{queue-2-SAT.ml}) is explained in detail in Appendix~\ref{sec:benchmark-details}.
All the experiments were conducted on
Intel Xeon Platinum8360Y, 256GB RAM.

\begin{table}
    \caption{Evaluation results}
    \label{tab:eval}
    \footnotesize
    \begin{tabular}{lcrcr}
        \toprule
        \multirow[c]{2}{*}{file name} & \multicolumn{2}{c}{\textsc{Spacer}} & \multicolumn{2}{c}{\textsc{PCSat}} \\
        & result correct? & time (sec.) & result correct? & time (sec.) \\
        \midrule
        \texttt{amb-1-SAT.ml} & Yes & 0.55 & Yes & 15.30 \\
        \texttt{amb-1-UNSAT.ml} & Yes & 0.72 & Yes & 63.62 \\
        \texttt{amb-2-SAT.ml} & Yes & 2.31 & Yes & 31.48 \\
        \texttt{amb-2-UNSAT.ml} & Yes & 2.26 & - & timeout$^\dagger$ \\
        \texttt{amb-3-SAT.ml} & Yes & 3.20 & Yes & 182.41 \\
        \texttt{amb-3-simpl-SAT.ml} & Yes & 1.71 & Yes & 16.79 \\
        \texttt{bfs-SAT.ml} & No$^{*1}$ & 1.67 & - & timeout$^{*1}$ \\
        \texttt{bfs-UNSAT.ml} & Yes & 2.00 & - & timeout$^\dagger$ \\
        \texttt{bfs-simpl-SAT.ml} & No$^{*1}$ & 2.22 & - & timeout$^{*1}$ \\
        \texttt{choose-all-SAT.ml} & Yes & 16.23 & - & timeout$^\dagger$ \\
        \texttt{choose-all-UNSAT.ml} & Yes & 12.56 & - & timeout$^\dagger$ \\
        \texttt{choose-max-SAT.ml} & Yes & 23.08 & - & timeout$^\dagger$ \\
        \texttt{choose-max-UNSAT.ml} & Yes & 15.97 & - & timeout$^\dagger$ \\
        \texttt{choose-sum-SAT.ml} & Yes & 1.54 & - & timeout$^\dagger$ \\
        \texttt{choose-sum-UNSAT.ml} & Yes & 7.99 & Yes & 15.00 \\
        \texttt{deferred-1-SAT.ml} & Yes & 0.46 & Yes & 4.49 \\
        \texttt{deferred-1-UNSAT.ml} & Yes & 0.27 & Yes & 4.09 \\
        \texttt{deferred-2-SAT.ml} & Yes & 0.43 & Yes & 4.38 \\
        \texttt{distribution-SAT.ml} & Abort$^\div$ & - & - & timeout$^{*2}$ \\
        \texttt{distribution-UNSAT.ml} & Abort$^\div$ & - & - & timeout$^{*2}$ \\
        \texttt{expectation-SAT.ml} & Yes & 0.51 & Yes & 7.25 \\
        \texttt{expectation-UNSAT.ml} & Yes & 1.45 & Yes & 7.33 \\
        \texttt{io-read-1-SAT.ml} & Yes & 0.43 & Yes & 13.90 \\
        \texttt{io-read-1-UNSAT.ml} & Yes & 0.41 & Yes & 12.21 \\
        \texttt{io-read-2-SAT.ml} & Yes & 0.56 & Yes & 21.10 \\
        \texttt{io-read-3-SAT.ml} & Yes & 0.54 & Yes & 14.88 \\
        \texttt{io-write-1-SAT.ml} & Yes & 0.32 & Yes & 8.48 \\
        \texttt{io-write-1-UNSAT.ml} & Yes & 0.32 & Yes & 8.76 \\
        \texttt{io-write-2-SAT.ml} & Yes & 0.46 & Yes & 11.33 \\
        \texttt{io-write-2-UNSAT.ml} & Yes & 0.68 & Yes & 11.65 \\
        \texttt{modulus-SAT.ml} & Yes & 14.23 & Yes & 11.89 \\
        \texttt{modulus-UNSAT.ml} & Yes & 26.56 & Yes & 11.91 \\
        \texttt{queue-1-SAT.ml} & Yes & 0.78 & Yes & 19.22 \\
        \texttt{queue-1-UNSAT.ml} & Yes & 0.52 & Yes & 16.93 \\
        \texttt{queue-2-SAT.ml} & Yes & 0.89 & Yes & 22.63 \\
        \texttt{round-robin-SAT.ml} & Yes & 0.96 & - & timeout$^\dagger$ \\
        \texttt{round-robin-UNSAT.ml} & Yes & 0.73 & - & timeout$^\dagger$ \\
        \texttt{safe-div-1-SAT.ml} & Abort$^\div$ & - & Yes & 2.71 \\
        \texttt{safe-div-1-UNSAT.ml} & Abort$^\div$ & - & Yes & 2.73 \\
        \texttt{safe-div-2-SAT.ml} & Abort$^\div$ & - & Yes & 2.55 \\
        \texttt{safe-div-2-UNSAT.ml} & Abort$^\div$ & - & Yes & 3.58 \\
        \texttt{select-SAT.ml} & - & timeout$^{\ddagger}$ & Yes & 13.28 \\
        \texttt{select-UNSAT.ml} & - & timeout$^{\ddagger}$ & Yes & 13.26 \\
        \texttt{shift-SAT.ml} & Yes & 0.28 & Yes & 2.92 \\
        \texttt{shift-UNSAT.ml} & Yes & 1.25 & Yes & 3.93 \\
        \texttt{state-SAT.ml} & - & timeout$^{\ddagger}$ & Yes & 33.69 \\
        \texttt{state-UNSAT.ml} & Yes & 0.63 & Yes & 13.56 \\
        \texttt{state-easy-SAT.ml} & Yes & 0.90 & Yes & 35.54 \\
        \texttt{transaction-SAT.ml} & - & timeout$^{\ddagger}$ & Yes & 15.36 \\
        \texttt{transaction-UNSAT.ml} & - & timeout$^{\ddagger}$ & Yes & 15.77 \\
        \texttt{yield-SAT.ml} & Yes & 1.51 & Yes & 17.57 \\
        \texttt{yield-UNSAT.ml} & Yes & 1.52 & - & timeout$^\dagger$ \\
        \bottomrule
    \end{tabular}
\end{table}

Table~\ref{tab:eval} shows the results of the evaluation.
The files that are suffixed with \texttt{-SAT} are expected to result in ``SAT'',
that is, the programs are expected to be typed
with the refinement types given as their specification.
The other files (suffixed with \texttt{-UNSAT}) are expected to result in ``UNSAT'',
that is, the programs are expected not to be typed
with the refinement types given as their specification.
For each program, we conducted verification in two configurations
((1) \textsc{Spacer}, and (2) \textsc{PCSat}).
The field ``time'' indicates the time spent in the whole process of the verification.
We set the timeout to 600 seconds.
%
Our implementation successfully answered correct result for most programs.
The ones that could not be verified in both configurations are
\texttt{bfs(-simpl)-SAT.ml} (marked with $*1$)
and \texttt{distribution-(UN)SAT.ml} (marked with $*2$).
They need some specific features which the implementation does not support.
The formers need
an invariant which states that there exists an element of a list
that satisfies a certain property.
The latter needs recursive predicates
in the type of an integer list, which states a property about the sum of the elements of the list.
These issues are orthogonal to the main contributions of this paper;
they are about the expressiveness of the background theory used for refinement predicates, to which our novel refinement type system is agnostic.


We discuss pros and cons between the two configurations.
First, \textsc{Spacer} does not support division operator,
and so it cannot verify some programs that use division (marked with $\div$,
aborting with the message ``\texttt{Z3 Error: Uninterpreted 'div' in <null>}'').
Also, there are some programs which can be solved by \textsc{Spacer}
but not solved by \textsc{PCSat} in time, and vice versa.
The formers (marked with $\dagger$) seem due to the huge size of generated constraints,
which can be solved by \textsc{Spacer} but not by \textsc{PCSat}.
The latters (marked with $\ddagger$) can be solved
by \textsc{PCSat} but not by \textsc{Spacer}.

It is worth noting that
our benchmark programs do not rely on refinement type annotation in most places,
even for recursive functions and recursive ADTs.
However, a few kinds of annotations are still needed.
First, as mentioned in Section~\ref{sec:language/discussions},
our type system does not support effect polymorphism.
Therefore, we added effect annotations to function-type arguments
which may perform operations when executed.
These annotations are written in the underlying OCaml types,
that is, we did not specify concrete refinements in the annotations.
Second,
we provided refinement type annotations for two small parts of \texttt{state-SAT.ml},
because otherwise it could not be verified within the timeout period in both configurations.
Third, because our implementation
infers predicate-polymorphic types only at let-bindings,
we added \emph{ghost parameters} to some operations and functions
to
infer precise refinement types of them which are not let-bound
but need some abstraction of refinements.
Ghost parameters are parameters which are used to express dependencies in dependent type checking,
but have no impact on the dynamic execution of the program so they can be removed at runtime.
In automated verification, completely inferring predicate variables requires
higher-order predicate constraints, which are not expressible with CHC.
Therefore, we provided ghost parameter annotations
to make it possible to reduce the verification to CHC constraint solving.
For example, the following code is a part of \texttt{state-SAT.ml}:
\begin{verbatim}
let rec counter c =
    let i = perform (Lookup c) in
    if i = 0 then c else (perform (Update (c, i - 1)); counter (c + 1))
in counter 0
\end{verbatim}
which is handled by a handler that simulates a mutable reference
similar to that of Example 2 in Section~\ref{sec:language/examples/state}.
Here, we pass the variable $c$ to the operation \texttt{Lookup} and \texttt{Update}
as the ghost parameter.
In the formal system presented in Section~\ref{sec:language/type-system}
where predicate polymorphism is available in operation signatures,
we can give \texttt{Update} the type
\begin{align}
    \forall X: (\tyint, \tyint).\,
        (x: \tyint) \rarr (\tyunit &\rarr ((s: \tyint) \rarr \tyrfn{z}{\tyint}{X(z, s)})) \\
    &\rarr ((s: \tyint) \rarr \tyrfn{z}{\tyint}{X(z, x)})
\end{align}
in the same way as Example 2 in Section~\ref{sec:language/examples/state},
and instantiate the predicate variable $X$ with $\lambda (z, s). z = c + 1 + s$
to correctly verify \texttt{state-SAT.ml}.
On the other hand, in the implementation, since predicate polymorphism is not available in operation signatures,
the handler needs to know the concrete predicate which replaces $X$.
However, the predicate contains $c$, which the handler cannot know
without receiving some additional information.
Therefore, we need to add the ghost parameter $c$ to \texttt{Update}
(and the same for \texttt{Lookup}).
This time we added them manually,
but one possible approach for automating insertion of ghost parameters is
to adopt the technique proposed by \citet{Unno13}.
We conjecture that a similar technique can be used
for our purpose.

\section{Implementation}
\label{sec:impl}

We have implemented a multi-stakeholder preference reasoner in XSB tabled logic programming environment~\cite{swift2012xsb} to demonstrate the viability of
our approach. The logical encoding of ${\mathcal L}^{af}_{\mu}$ used allows for \emph{on-the-fly} evaluation of logical queries, circumventing the need for constructing the complete multi-stakeholder induced preference graph. In other words, only the portion of the induced preference graph relevant for answering the query is constructed, resulting in significant savings in computational and memory savings relative to a naive implementation. 

\subsection{Input: Preferences as Logical Relations and Facts}
\label{sec:input}

The implementation takes as input a XSB Prolog file containing (a) preference
specifications described in terms of "flips" relation and (b) 
logical fact specifying the different valuations of the attributes
that describe each outcome. 

In the following, we describe the representation of 
preferences of each agents as logical relations in XSB Prolog.
Consider that there are $n$ attributes $x_1, x_2, \ldots, x_n$
that describe the outcomes, where valuations of $x_i$ 
are $v_{i_1}, v_{i_2}, \ldots, v_{i_k}$. This is 
captured by a Prolog fact:

{\footnotesize
\begin{verbatim}
properties([v11, v12, v13, ..., vik],
            [v21, v22, v23, ..., v2k],
            ...
            [vn1, vn2, vn3, ..., vnk]).
\end{verbatim}
}
Note that, the each argument of the term is a Prolog list
is the domain of the corresponding attribute; $i^{th}$ argument
being the domain of the valuations of $x_i$.

Next, let agent $a$ has the preference
\[
[x_1=v_1]\ x_2=v_2 \prec x_2=v'_2\ [x_3, x_4]
\]
capturing the fact that when $x_1=v_1$, any outcome $o'$
where $x_2=v'_2$ is preferred to outcomes $o$ where
$x_j=v_2$ regardless of the valuations of
$x_3$ and $x_4$ (all other attribute valuations in
$o$ and $o'$ being same). 

This is represented by the logical
relation

{\footnotesize
\begin{verbatim}
trans([v_1, v_2, _, _, _V5, _V6, ..., _Vn],
       a,
       [v_1, v_2', _, _, _V5, _V6, ..., _Vn]).
\end{verbatim}
}

In the above, the first, second and third arguments of the \texttt{trans} relation captures the outcomes $o$, agent
$a$ and outcomes $o'$, respectively, Note that, the
valuation of $x_1$ in both $o$ and $o'$ are $v_1$;
the valuations of $x_3$ and $x_4$ are captured by
"don't care" logical variables ("\texttt{\_}") indicating they
can be any valuation;  the valuation of $x_5$ till
$x_n$ are are any values that are same in both $o$
and $o'$.


\begin{example}
In Figure~\ref{fig:pref-statements}, the preference for agent $1$
\[
P_1\ \ E = Code \succ_E E = No\_Code
\]
is captured by the logical relation 

{\footnotesize
\begin{verbatim}
trans([noCode, _X, _Y], 1, [code, _X, _Y]).
\end{verbatim}
}

The relation captures the fact that all else being equal, any
outcome with first attribute valuation \texttt{code} is preferred
to any outcome with first attribute valuation \texttt{noCode}. 

This input file also contains the logical fact:

{\footnotesize
\begin{verbatim}
properties([code, noCode], 
            [simple, complex], 
            [fix, noFix]).
\end{verbatim}
}

which states that the outcomes being considered
contain three attributes and the valuation of three
attributes are given in the form of XSB Prolog list.
\end{example}

The listing for the preferences in Figure~{\ref{fig:pref-statements}}
in (XSB) Prolog is as follows:

{\footnotesize
\begin{verbatim}
trans([noCode, _X, _Y], S, [code, _X, _Y]) 
     :- S = 1; S = 3.

trans([code, _X, fix], 1, [code, _X, noFix]).

trans([noCode, _X, fix], 2, [code, _X, fix]).

trans([_X, complex, fix], 2, [_X, simple, fix]).

trans([code, complex, noFix], 2, 
    [noCode, simple, noFix]).

trans([_X, complex, _Y], 3, [_X, simple, _Y]).

trans([noCode, _, _], 3, [code, _, _]).
\end{verbatim}
}

\subsection{Modules of Prototype Implementation}
\label{sec:modules}

There are three primary modules in the implementation:
a module for $\mu$-calculus model checker, a module for
for appropriately translating the preference queries
to $\mu$-calculus formula, which is used in the
module to evaluate the semantics of preference query. 

The model checker is written using tabling in
XSB Prolog (that allows for efficient least fixed point
computation) in less than 100 lines of Prolog code. 
The relation
{\footnotesize
\begin{verbatim}
models(S, Phi)
\end{verbatim}
}
returns true when the instantiation of variable \texttt{S}
to some state of a Kripke structure satisfies
the $\mu$-calculus formula captured by the variable \texttt{Phi}.
This is a local, on-the-fly realization of the semantics
of $\mu$-calculus formula, where the state-space of
Kripke structure is explored only if it is necessary to
prove the satisfiability of \texttt{Phi} at state \texttt{S}.

To illustrate the connection between the \texttt{models}
relation and the \texttt{trans} relation describing
the preferences, we present below the definition of
\texttt{models} for $\moddiam{.}$-modal formulas. 
{\footnotesize
\begin{verbatim}
models(S, diam(A, Phi)) :-
    trans(S, A1, S1),
    member(A1, A), 
    models(S1, Phi).
\end{verbatim}
}
The above definition states that if there exists
a \texttt{trans} relation over \texttt{S}, \texttt{A1}
and \texttt{S1} indicating some stackholder \texttt{A1} prefers
outcome \texttt{S1} over outcome \texttt{S}; if
\texttt{A1} is a member of \texttt{A}; and if 
\texttt{S1} satisfies the formula \texttt{Phi}, then
we conclude that \texttt{S} satisfies \texttt{diam(A, Phi)}
($\moddiam{A}\varphi$).

The module \texttt{translate} ($<100$ lines of XSB Prolog) contains the definition of
{\footnotesize
\begin{verbatim}
translate(F1, Type, F2)
\end{verbatim}
}
where \texttt{F1} is the formula in preference query
language, \texttt{F2} is the corresponding formula
in the $\mu$-calculus and \texttt{Type} captures
the different combinations of witness and agreement
conditions used in the translation. 

Finally, the module for computing the semantics of 
the query language includes the definition of
{\footnotesize
\begin{verbatim}
sem(F, Type, R)
\end{verbatim}
}
where \texttt{F} is the query, \texttt{Type}
is the combination of witness and agreement conditions
to be used to evaluate the query and \texttt{R} is
the result of the query. For instance,
{\footnotesize
\begin{verbatim}
sem(p(Psi1, Psi2, A), Type, R) :-
    sem(Psi1, Type, L),
    translate(p(true, Psi2, A), Type, MuForm),
    models_list(L, MuForm, R).
\end{verbatim}
}
presents the following computation for the 
formula \texttt{p(Psi1, Psi2, A)} (representing
the query $\mlangpref{\psi_1}{\psi_2}{A}$) . First, we
compute the semantics of \texttt{Psi1}, the result
of which is captured in \texttt{L}. That is,
\texttt{L} is the list of outcomes that
satisfy \texttt{Psi1}. Next, we translate
\texttt{p(true, Psi2, A)} to the corresponding
$\mu$-calculus formula \texttt{MuForm}. Finally, we identify
the outcomes in \texttt{L} that satisfiy
\texttt{MuForm} and include them in \texttt{R}. 
This is performed by the \texttt{models\_list} predicate
which calls the \texttt{models} predicate (see
$\mu$-calculus model checker module above) on
each element of \texttt{L}.



% Figure environment removed

\subsection{On-the-fly Evaluation of Logical Statements}

It is worth noting that logical encoding allows
for on-the-fly evaluation. Intuitively, this
means the a query of the form:
{\footnotesize
\begin{verbatim}
    sem(p(true, prop(noCode), [1,2]), w1a2, R)
\end{verbatim}
}
is resolved only by considering the
trans-predicates that are related to 
stakeholders 1 and 2, and by considering
only those trans-predicates that are necessary
for the resolution. For instance, when the
above query eventually requires the resolution
of {\footnotesize \texttt{
models([code,simple,fix],rdiam([1],prop(noCode)))}}
our implementation will try to resolve the
predicate {\footnotesize\texttt{trans(X,1,[code,simple,fix])}},
by finding the valuation of  \texttt{X} for which the above predicate is true. 
Note that, we are not exploring all the transition relations for all stakeholders and for all outcomes. There may be multiple valuations for \texttt{X}; the logical encoding will find one of them and try to answer {\footnotesize\texttt{models([code,simple,fix],rdiam([1],prop(noCode)))}}. If the answer is false, then the encoding will consider another valuation for \texttt{X}; otherwise, it will not explore any other solutions for \texttt{X}. 

In short, the entire induced preference graph for all stakeholders is never constructed and the exploration proceeds by considering only the edges that are necessary for answering a query.  

\subsection{Evaluating Queries}

The listing of queries from Example~\ref{ex:collab} is presented in Figure~\ref{fig:mu-collab}.

\section{Preliminary Experiments}

\begin{comment}
Our experiments with synthetic stakeholder preferences  
with varying numbers of attributes, number of stakeholders,  
different choices of collaborative semantics and nesting depths of queries
show that all of the queries returned results in less than
2 seconds; For instance,
a query of the form 
\[
\mlangpref{\true}{(\mlangpref{\true}{(\mlangpref{\true}{\true}{L_1})}{L_2})}{L_3}
\]
with $8$ attributes where $L_1 = \{2, 3, 4\}$, $L_2=\{5,6,7\}$ and $L_3 = \{8,9,10\}$ (where $\forall i, L_i \subseteq \mathcal{A}$ where $\mathcal{A}=\{1, \cdots N\}$ is the set of all stakeholders) can be evaluated in less than 2 seconds for all possible choices of collaborative semantics. 
Additional preliminary experiments show that queries on graphs with $\le 200$ vertices and with at most 400 edges per stakeholder  (with nesting depth 3) can be answered within 20 seconds. Additional details of the experiments are given in \cite{}. These results are indicative of the practical feasibility of our approach.
\end{comment}


%\begin{comment}
To stress-test our implementation, we conducted two
types of experiments. For the first type of experiments, we generated for each stakeholder, random preference statements over $n$ binary preference variables, while ensuring to disallow inconsistent preferences. The resulting preferences statements include direct preferences between outcomes
described by the attribute values, conditional preferences
between attribute values, and relative importance between attributes.  The results of this set of experiments are summarized in Table~\ref{tbl:randompref}.
Table entries show the run-time (in seconds) for answering some representative multi-stakeholder preference queries based on different choices of semantics, for several choices of the number of attributes.  The numbers
in parenthesis indicate the size of the solution set for the corresponding query. 

The results in Table~\ref{tbl:randompref} shows the viability of
our approach; in each of the 84 cases, the corresponding query is answered in 
at most 2 seconds. Recall that the run-time for answering a query depends on the
nesting depth of the query, the number of stakeholders that appear in the query, and
the size of the  preference graph induced by their preferences. We observe that
the run-time for answering queries for semantic type $\texttt{W}_1\texttt{A}_1$ is the smallest, whereas that for semantic type
$\texttt{W}_2\texttt{A}_2$ is the
largest. This is explained by the fact that both the
witness and agreement conditions in the case of the former are evaluated using disjunctive constraints, whereas in the case of the latter, they are evaluated using chaining constraints. This implies the state space explored for the latter is at least large as the state space explored for the former. 




\begin{table}[t]
\scriptsize
\begin{tabular}{c|c|lll}
\multirow{2}{*}{\bf Query} & \multirow{2}{*}{\bf Type} & \multicolumn{3}{c}{\bf Number of Attributes} \\ \cline{3-5} 
& & 5 & 6 &  8 \\ \hline\hline
\multirow{2}{*}{$\mlangpref{tt}{(\mlangpref{tt}{tt}{L1})}{L2}$} 
       & $\texttt{W}_1\texttt{A}_2$ & 0.02 (2) & 0.23 (10) & 0.27 (0)\\
       & $\texttt{W}_1\texttt{A}_1$ & 0.02 (2) & 0.11 (10) & 0.12 (0) \\
\multirow{2}{*}{$L1=\{1,2\}, L2=\{3,4\}$} 
       & $\texttt{W}_2\texttt{A}_2$ & 0.02 (2) & 0.41 (12) & 0.49 (0) \\
       & $\texttt{W}_2\texttt{A}_1$ & 0.02 (2) & 0.31 (12)  & 0.29 (0) \\ \hline

\multirow{2}{*}{$\mlangpref{tt}{(\mlangpref{tt}{tt}{L1})}{L2}$} 
       & $\texttt{W}_1\texttt{A}_2$ & 0.02 (2) & 0.28 (0) & 0.25 (32)\\
       & $\texttt{W}_1\texttt{A}_1$ & 0.02 (5) & 0.14 (0) & 0.16 (32)\\
\multirow{2}{*}{$L1=\{2,3\}, L2=\{4, 5\}$} 
       & $\texttt{W}_2\texttt{A}_2$ & 0.02 (4) & 0.39 (0) &  0.35 (32)\\
       & $\texttt{W}_2\texttt{A}_1$ & 0.02 (5) & 0.27 (0) & 0.27 (32) \\ \hline

\multirow{2}{*}{$\mlangpref{tt}{(\mlangpref{tt}{tt}{L1})}{L2}$} 
       & $\texttt{W}_1\texttt{A}_2$ & 0.01 (0) & 0.13 (6) & 0.02 (0) \\
       & $\texttt{W}_1\texttt{A}_1$ & 0.01 (0) & 0.07 (12) & 0.02 (0) \\
\multirow{2}{*}{$L1=\{5,6\}, L2=\{9, 10\}$} 
       & $\texttt{W}_2\texttt{A}_2$ & 0.01 (0) & 0.24 (6) & 0.02 (0) \\
       & $\texttt{W}_2\texttt{A}_1$ & 0.01 (0) & 0.17 (12) & 0.02 (0) \\ \hline

\multirow{2}{*}{$\mlangpref{tt}{(\mlangpref{tt}{tt}{L1})}{L2}$} 
       & $\texttt{W}_1\texttt{A}_2$ & 0.08 (1) & 0.54 (0) & 0.91 (23) \\
       & $\texttt{W}_1\texttt{A}_1$ & 0.05 (1) & 0.23 (0)  & 0.34 (27)\\
\multirow{2}{*}{$L1=\{1,2,3\}, L2=\{4,5,6\}$} 
       & $\texttt{W}_2\texttt{A}_2$ & 0.29 (2) & 1.03 (0) &  1.76 (36) \\
       & $\texttt{W}_2\texttt{A}_1$ & 0.21 (3) & 0.63 (0) & 0.95 (42) \\ \hline

\multirow{2}{*}{$\mlangpref{tt}{(\mlangpref{tt}{tt}{L1})}{L2}$} 
       & $\texttt{W}_1\texttt{A}_2$ & 0.04 (2) & 0.48 (0) &  0.46 (28)\\
       & $\texttt{W}_1\texttt{A}_1$ & 0.03 (2) & 0.23 (0) & 0.23 (29)\\
\multirow{2}{*}{$L1=\{2,3, 4\}, L2=\{5,6,7\}$} 
       & $\texttt{W}_2\texttt{A}_2$ & 0.09 (2) & 0.82 (0) &  0.82 (33)  \\
       & $\texttt{W}_2\texttt{A}_1$ & 0.07 (2) & 0.45 (0) &  0.52 (36)\\ \hline


\multirow{2}{*}{$\mlangpref{tt}{(\mlangpref{tt}{(\mlangpref{tt}{tt}{L1})}{L2})}{L3}$} 
       & $\texttt{W}_1\texttt{A}_2$ & 0.23 (0) & 1.16 (0) & 0.91 (0)\\
       & $\texttt{W}_1\texttt{A}_1$ & 0.19 (0) & 0.76 (0) & 0.65 (0)\\
$L1=\{1,2,3\}, L2=\{4,5,6\},$
       & $\texttt{W}_2\texttt{A}_2$ & 0.70 (2) & 1.73 (0) & 1.58 (0)  \\
$ L3=\{7,8,9\}$        & $\texttt{W}_2\texttt{A}_1$ & 0.54 (2)  & 1.27 (0)  & 1.02 (0)\\ \hline


\multirow{2}{*}{$\mlangpref{true}{(\mlangpref{tt}{(\mlangpref{tt}{tt}{L1})}{L2})}{L3}$} 
       & $\texttt{W}_1\texttt{A}_2$ & 0.26 (0) & 1.11 (0) & 0,76 (0)\\
       & $\texttt{W}_1\texttt{A}_1$ & 0.24 (0) & 0.72 (0) & 0.66 (0) \\
$L1=\{2,3,4\}, L2=\{5,6,7\},$
       & $\texttt{W}_2\texttt{A}_2$ & 0.65 (2)  &  1.54 (0) & 1.69 (0) \\
$ L3=\{8,9,10\}$        & $\texttt{W}_2\texttt{A}_1$ 
& 0.58 (2) &  1.14 (0) & 1.29 (0) \\ \hline


\end{tabular}
%\vspace{-1em}
\caption{Experiments with Randomly Generated Preference Statements}
%\vspace{-2em}
\label{tbl:randompref}
\end{table}
%\end{comment}



For the second set of experiments, we randomly generated
graphs in which the vertices correspond to  outcomes, and the edges denote preference between pairs of outcomes.
Note that in this case, because the graphs are randomly generated, and not induced by the stakeholder preferences, it is possible for the preferences reflected in the graph to be inconsistent, i.e., individual stakeholder's preference graph may be inconsistent. Each edge is annotated with a random subset of stakeholders (simulating
the setting where the stakeholder preferences induce edges in the induced preference graph).  
Table~\ref{tbl:timing} presents the timing results of our experiments
with random graphs. 

The column "configuration"
includes three numbers describing the randomly
generated induced preference graph: the first
number is the number of stakeholders, the
second number is the number of outcomes and
the third number indicates that the maximum
number of edges per stakeholder in
the induced preference graph.  We generate 
$25$ graphs per configuration. For each
configuration, we compute the result
of the three types of queries presented
in the first row of the table. For
each query, we consider four different
collaborative semantics and report the
time in seconds needed for the computation.

Typically, as the induced preference graph
and/or the query size become larger, the
time for computing the semantics increases.
However, it is worth noting that semantics
of the query depends on the structure of the
graph and, hence, in certain cases, it may
be possible that the semantic computation in 
a larger graph or for a larger query
takes less time than the computation in
a smaller graph or for a smaller query. For instance,
we observe that computation of query 
$\mlangpref{\true}{(\mlangpref{\true}{true}{\{1,2,3\}})}{\{4,5,6\}}$ takes less time in most
cases than the computation of query
$\mlangpref{\true}{(\mlangpref{\true}{\true}{\{1,2\}})}{\{2,3\}}$ (even if the former involves
$6$ stakeholders). This can be attributed
to situations where the nested query $\mlangpref{\true}{\true}{\{1,2,3\}}$ in 
$\mlangpref{\true}{(\mlangpref{\true}{\true}{\{1,2,3\}\}})}{\{4,5,6\}}$ returns
a small set (or even an empty set), which
makes the evaluation of overall query
computationally less expensive. 

Recall that the result is an average of timing results
obtained from $25$ randomly generated induced
preference graphs for each configuration. It is
worth noting that the maximum time recorded
in all sample runs is $55$ seconds, which corresponds
to a sample for configuration $30,\ 400,\ 400$
for evaluation of query with nesting depth $3$.


%\begin{sidewaystable*}
\begin{table*}[bth]
\scriptsize
\begin{tabular}{|l||l|l|l|l||l|l|l|l||l|l|l|l|}
\hline
Configuration &
\multicolumn{4}{|c|}{$\mlangpref{\true}{(\mlangpref{\true}{\true}{1,2})}{3,4}$} &
\multicolumn{4}{|c|}{$\mlangpref{\true}{(\mlangpref{\true}{\true}{1,2,3})}{4,5,6}$} &
\multicolumn{4}{|c|}{$\mlangpref{\true}{(\mlangpref{\true}{(\mlangpref{\true}{\true}{1,2,3})}{4,5,6})}{7,8,9}$} \\ \hline 
&
$\texttt{W}_1\texttt{A}_2$ &
$\texttt{W}_1\texttt{A}_1$ &
$\texttt{W}_2\texttt{A}_2$ &
$\texttt{W}_2\texttt{A}_1$ &
$\texttt{W}_1\texttt{A}_2$ &
$\texttt{W}_1\texttt{A}_1$ &
$\texttt{W}_2\texttt{A}_2$ &
$\texttt{W}_2\texttt{A}_1$ &
$\texttt{W}_1\texttt{A}_2$ &
$\texttt{W}_1\texttt{A}_1$ &
$\texttt{W}_2\texttt{A}_2$ &
$\texttt{W}_2\texttt{A}_1$ 
\\ \hline \hline
$10,\ 100,\ 200$ &
0.1589 &
0.1656 &
0.1684 &
0.1753 &

0.1811 &
0.194 &
0.1913 &
0.2049 &

0.7936 &
0.7702 &
0.8762 &
0.8997
\\ \hline

$20,\ 100,\ 200$ &
0.349 &
0.3607 &
0.4912 &
0.5018 &

0.0827 &
0.0873 &
0.0671 &
0.0698 &

1.1415 & 
1.3145 &
1.6448 &
1.8579  \\ \hline 

$30,\ 100, \ 200$ & 
0.6637 &
0.6284 &
0.5505 &
0.5023 &

0.1592 &
0.1807 &
0.1365 & 
0.137 &

1.7729 &
1.9391 &
3.612 &
3.837 \\ \hline \hline \hline

$10,\ 200, \ 200$ &
0.1594  &
0.1667 &
0.1469 &
0.158 &

0.1637 &
0.177 &
0.1433 &
0.1518 &


0.8713  &
0.9955 &
0.8776 &
0.9767
\\ \hline


$20,\ 200, \ 200$ &
0.3364 & 
0.3366 &
0.5669 &
0.5571 & 

0.311 &
0.311 &
0.2698 &
0.2792 &

1.9861 &
1.8095 &
6.674 &
6.2761 \\ \hline

$30,\ 200, \ 200$ & 

0.3097 &
0.306 &
0.2612 &
0.2654 &

0.3357 &
0.3544 &
0.2858 &
0.2913 &

1.5188 &
1.5963 &
1.5699 &
1.4971 

\\  \hline \hline \hline

$20,\ 200, \ 400$ &
0.4341 &
0.4514 &
0.4767 &
0.493 &

0.3863 &
0.4027 &
0.3632 &
0.3709 &

5.53 &
6.0667 &
5.2822 &
5.5943

\\ \hline

$30,\ 200, \ 400$ &
1.4318 &
1.1996 &
3.6899 &
3.4727 &

3.5117 &
3.603 &
4.371 &
4.3949 &

16.0171 &
17.0694 &
20.4699 &
20.7602

\\ \hline

\end{tabular}
\caption{Timing Results}
\label{tbl:timing}
\end{table*}
%\end{sidewaystable*}

%%% No need for this. 
\begin{comment}
\begin{sidewaystable}

\begin{tabular}{|c||l|l|l|l||l|l|l|l||l|l|l|l|}
\hline
Configuration &
\multicolumn{4}{|c|}{$\mlangpref{true}{(\mlangpref{true}{true}{L1})}{L2}$}  &
\multicolumn{4}{|c|}{$\mlangpref{true}{(\mlangpref{true}{true}{L1})}{L2}$} &
\multicolumn{4}{|c|}{$\mlangpref{true}{(\mlangpref{true}{(\mlangpref{true}{true}{L1})}{L2})}{L3}$} \\ \hline 
&
$\texttt{W}_1\texttt{A}_2$ &
$\texttt{W}_1\texttt{A}_1$ &
$\texttt{W}_2\texttt{A}_2$ &
$\texttt{W}_2\texttt{A}_1$ &
$\texttt{W}_1\texttt{A}_2$ &
$\texttt{W}_1\texttt{A}_1$ &
$\texttt{W}_2\texttt{A}_2$ &
$\texttt{W}_2\texttt{A}_1$ &
$\texttt{W}_1\texttt{A}_2$ &
$\texttt{W}_1\texttt{A}_1$ &
$\texttt{W}_2\texttt{A}_2$ &
$\texttt{W}_2\texttt{A}_1$  \\ \hline \hline
&
\multicolumn{4}{c||}{\bf L1=\{1,2\}, L2=\{3,4\}} &
\multicolumn{4}{c||}{\bf L1=\{1,2,3\}, L2=\{4,5,6\}} & 
\multicolumn{4}{c|}{\bf L1=\{1,2,3\}, L2=\{4,5,6\}, L3=\{7,8,9\}} \\ \hline  

5, 6 &

0.022 (2) & 
0.015 (2) & 
0.024 (2) & 
0.018 (2) &

0.079 (1) & 
0.048 (1) & 
0.286 (2) & 
0.211 (3) &

0.228 (0) &
0.189 (0) &
0.696 (2) &
0.536 (2) 
\\
\hline

 &
\multicolumn{4}{c||}{\bf L1=\{2,3\}, L2=\{4,5\}} &
\multicolumn{4}{c||}{\bf L1=\{2,3,4\}, L2=\{5,6,7\}} & 
\multicolumn{4}{c|}{\bf L1=\{2,3, 4\}, L2=\{5,6, 7\}, L3=\{8,9, 10\}} \\ \hline  

5, 6 &

0.019 (2) & 
0.018 (5) & 
0.018 (4) & 
0.019 (5) &

0.039 (2) & 
0.034 (2) & 
0.088 (2) & 
0.071 (2) &

0.264 (0) &
0.243 (0) &
0.648 (2) &
0.579 (2) 
\\
\hline

&
\multicolumn{4}{c||}{\bf L1=\{1,2\}, L2=\{3,4\}} &
\multicolumn{4}{c||}{\bf L1=\{1,2,3\}, L2=\{4,5,6\}} & 
\multicolumn{4}{c|}{\bf L1=\{1,2,3\}, L2=\{4,5,6\}, L3=\{7,8,9\}} \\ \hline  

6, 6 &

0.227 (10) & 
0.112 (10) & 
0.404 (12) & 
0.305 (12) &

0.535 (0) & 
0.229 (0) & 
1.025 (0) & 
0.631 (0) &

1.158 (0) &
0.759 (0) &
1.726 (0) &
1.270 (0) 
\\
\hline

\end{tabular}
\caption{Experiments with Randomly Generated Preference Statements}
\label{tbl:randompref}
\end{sidewaystable}
\end{comment}

\begin{comment}
yes
| ?- testing1([4,5],[8,9], w1a2, R).
% 0.104 CPU in 0.108 seconds (96% CPU)

R = [[a_0,b_0,c_0,d_0,e_0,f_0],[a_0,b_0,c_0,d_0,e_0,f_1],[a_0,b_0,c_0,d_0,e_1,f_0],[a_0,b_0,c_0,d_0,e_1,f_1],[a_0,b_0,c_1,d_0,e_0,f_0],[a_0,b_0,c_1,d_0,e_0,f_1],[a_0,b_0,c_1,d_0,e_1,f_0],[a_0,b_0,c_1,d_0,e_1,f_1],[a_1,b_0,c_0,d_0,e_0,f_0],[a_1,b_0,c_0,d_0,e_0,f_1],[a_1,b_0,c_0,d_0,e_1,f_0],[a_1,b_0,c_0,d_1,e_0,f_0],[a_1,b_0,c_0,d_1,e_1,f_0],[a_1,b_0,c_1,d_0,e_0,f_0],[a_1,b_0,c_1,d_0,e_1,f_0],[a_1,b_1,c_0,d_0,e_0,f_0],[a_1,b_1,c_0,d_0,e_1,f_0],[a_1,b_1,c_1,d_0,e_0,f_0],[a_1,b_1,c_1,d_0,e_1,f_0]];

no
| ?- testing1([4,5],[8,9],w1a1, R).
% 0.10000000000000001 CPU in 0.10199999999999999 seconds (98% CPU)

R = [[a_0,b_0,c_0,d_0,e_0,f_0],[a_0,b_0,c_0,d_0,e_0,f_1],[a_0,b_0,c_0,d_0,e_1,f_0],[a_0,b_0,c_0,d_0,e_1,f_1],[a_0,b_0,c_1,d_0,e_0,f_0],[a_0,b_0,c_1,d_0,e_0,f_1],[a_0,b_0,c_1,d_0,e_1,f_0],[a_0,b_0,c_1,d_0,e_1,f_1],[a_1,b_0,c_0,d_0,e_0,f_0],[a_1,b_0,c_0,d_0,e_0,f_1],[a_1,b_0,c_0,d_0,e_1,f_0],[a_1,b_0,c_0,d_1,e_0,f_0],[a_1,b_0,c_0,d_1,e_1,f_0],[a_1,b_0,c_1,d_0,e_0,f_0],[a_1,b_0,c_1,d_0,e_1,f_0],[a_1,b_1,c_0,d_0,e_0,f_0],[a_1,b_1,c_0,d_0,e_1,f_0],[a_1,b_1,c_1,d_0,e_0,f_0],[a_1,b_1,c_1,d_0,e_1,f_0]];

no
| ?- testing1([4,5],[8,9],w2a2, R).
% 0.11899999999999999 CPU in 0.12 seconds (99% CPU)

R = [[a_0,b_0,c_0,d_0,e_0,f_0],[a_0,b_0,c_0,d_0,e_0,f_1],[a_0,b_0,c_0,d_0,e_1,f_0],[a_0,b_0,c_0,d_0,e_1,f_1],[a_0,b_0,c_1,d_0,e_0,f_0],[a_0,b_0,c_1,d_0,e_0,f_1],[a_0,b_0,c_1,d_0,e_1,f_0],[a_0,b_0,c_1,d_0,e_1,f_1],[a_1,b_0,c_0,d_0,e_0,f_0],[a_1,b_0,c_0,d_0,e_0,f_1],[a_1,b_0,c_0,d_0,e_1,f_0],[a_1,b_0,c_0,d_1,e_0,f_0],[a_1,b_0,c_0,d_1,e_1,f_0],[a_1,b_0,c_1,d_0,e_0,f_0],[a_1,b_0,c_1,d_0,e_1,f_0],[a_1,b_1,c_0,d_0,e_0,f_0],[a_1,b_1,c_0,d_0,e_1,f_0],[a_1,b_1,c_1,d_0,e_0,f_0],[a_1,b_1,c_1,d_0,e_1,f_0]];

no
| ?- testing1([4,5],[8,9],w2a1, R).
% 0.106 CPU in 0.109 seconds (97% CPU)

R = [[a_0,b_0,c_0,d_0,e_0,f_0],[a_0,b_0,c_0,d_0,e_0,f_1],[a_0,b_0,c_0,d_0,e_1,f_0],[a_0,b_0,c_0,d_0,e_1,f_1],[a_0,b_0,c_1,d_0,e_0,f_0],[a_0,b_0,c_1,d_0,e_0,f_1],[a_0,b_0,c_1,d_0,e_1,f_0],[a_0,b_0,c_1,d_0,e_1,f_1],[a_1,b_0,c_0,d_0,e_0,f_0],[a_1,b_0,c_0,d_0,e_0,f_1],[a_1,b_0,c_0,d_0,e_1,f_0],[a_1,b_0,c_0,d_1,e_0,f_0],[a_1,b_0,c_0,d_1,e_1,f_0],[a_1,b_0,c_1,d_0,e_0,f_0],[a_1,b_0,c_1,d_0,e_1,f_0],[a_1,b_1,c_0,d_0,e_0,f_0],[a_1,b_1,c_0,d_0,e_1,f_0],[a_1,b_1,c_1,d_0,e_0,f_0],[a_1,b_1,c_1,d_0,e_1,f_0]];
\end{comment}

%\section{Related Work}
\label{appsec: related work}
Bayesian causal discovery literature has primarily focused on inference in linear models with closed-form posteriors or marginalized parameters. Early works considered sampling directed acyclic graphs (DAGs) for discrete~\cite{cooper1992bayesian, madigan1995bayesian, heckerman2006bayesian} and Gaussian random variables~\cite{friedman2003being, tong2001active} using Markov chain Monte Carlo (MCMC) in the DAG space. However, these approaches exhibit slow mixing and convergence~\cite{eaton2012bayesian,grzegorczyk2008improving}, often requiring restrictions on number of parents~\cite{kuipers2017partition}. %Alternative exact dynamic programming methods are limited to small settings~\cite{koivisto2012advances}. 

Recent advances in variational inference~\cite{zhang2018advances} have facilitated graph inference in DAG space, with gradient-based methods employing the NOTEARS DAG penalty \cite{zheng2018dags}.\cite{annadani2021variational} samples DAGs from autoregressive adjacency matrix distributions, while \cite{lorch2021dibs} utilizes Stein variational approach \cite{liu2016stein} for DAGs and causal model parameters. \cite{cundy2021bcd} proposed a variational inference framework on node orderings using the gumbel-sinkhorn gradient estimator \cite{mena2018learning}. \cite{deleu2022bayesian,nishikawa2022bayesian} employ the GFlowNet framework \cite{bengio2021gflownet} for inferring the DAG posterior. Most methods, except\cite{lorch2021dibs} are restricted to linear models, while \cite{lorch2021dibs} has high computational costs and lacks DAG generation guarantees compared to our method.
% at least quadratic scaling complexity, both with respect to the number of nodes (due to the DAG penalty) as well as number of posterior samples. Our proposed approach instead has linear complexity with respect to number of posterior samples and does not require any additional DAG penalty.     

In contrast, \emph{quasi-Bayesian} methods, such as DAG bootstrap \cite{friedman2013data}, demonstrate competitive performance. DAG bootstrap resamples data and estimates a single DAG using PC \cite{spirtes2000causation}, GES \cite{chickering2002optimal}, or similar algorithms, weighting the obtained DAGs by their unnormalized posterior probabilities. Recent neural network-based works employ variational inference to learn DAG distributions and point estimates for nonlinear model parameters \cite{charpentier2022differentiable,geffner2022deep}.
%\section{Related Work}
%\label{sec:related}




%\section{Conclusion}
\label{sec:conc}
In this paper, we propose a fast leakage and variability-aware thermal simulation method that also captures the temperature dependence of conductivity. We derive a closed-form of the Green's function considering all these effects using novel insights and algebraic techniques. Our approach provides fast and accurate solutions for both the steady-state and the transient thermal profile and has been validated with a wide variety of test cases. As device dimensions continue to shrink, process variation has become a serious problem. The methods proposed in this work can equip designers to tackle this problem and may spawn further research in this area.

\section{Summary and Discussion}
\label{sec:conc}
\noindent\textbf{Summary. }We provided the first formal treatment of reasoning with multi-stakeholder preferences in a setting where each stakeholder expresses their preferences in a qualitative preference language. We introduced a query language for expressing queries with respect to the preferences of a given set of stakeholders over sets of outcomes. Motivated by the needs of application scenarios, we introduced and analyzed several alternative semantics for such queries and examined their inter-relationships.  We provided a provably correct algorithm for answering multi-stakeholder preference queries using model checking in alternation-free $\mu$-calculus. Results of preliminary experiments demonstrate the feasibility of the approach.\\

\noindent\textbf{Related Work. }  Existing approaches to reasoning about qualitative preferences of multi-stakeholders leverage voting-based social choice mechanisms~\cite{mcpnet,rossi:synthesis2011,mpcpnet}, starting with the seminal work of Rossi et al. \cite{mcpnet}.  The applicability of such approaches is limited to settings where the stakeholder preferences are expressed over outcomes (rather than attributes of outcomes); or when they are expressed over attributes of an outcome, they are rather simple (e.g., expressible using CP-nets). A major focus of the social choice based approaches to multi-stakeholder preference reasoning is on voting strategies that are resistant to manipulation by some of the stakeholders and guarantee {\em fair} outcomes. 
The key aspects of our work that distinguish from social choice model such as mCP-net \cite{mcpnet} are as follows: We seek to answer queries of the form 
$\mlangpref{\psi_1}{\psi_2}{A}$, i.e., identify outcomes that satisfy $\psi_1$ and are more preferred to outcomes satisfying $\psi_2$, and are not less preferred to any outcome that satisfies $\psi_2$ by the set $A$ of stakeholders, whereas mCP-net queries are about whether one outcome is preferred to another by the given set of stakeholders. The precise conditions for deciding the answer to $\mlangpref{\psi_1}{\psi_2}{A}$ depends on the type of semantics. In the special case where the set of outcomes satisfying $\psi_2$  is a singleton set, then our semantics is similar to Pareto semantics defined in \cite{mcpnet}. This raises the possibility of extending voting-based semantics where the set of outcomes satisfying $\psi_2$  is not a singleton set.  In such as setting, one may use voting to identify an outcome (say $o$) that is preferred by a majority of the stakeholders, and include it in the solution set if it is preferred to one of the outcomes satisfying $\psi_2$ (similar to the witness condition in the paper), and all of them are not preferred to $o$ (similar to the agreement condition in the paper), with the pair-wise outcome preferences decided using a voting mechanism.  \\

\noindent\textbf{Discussion. } 
%It would be interesting to explore frameworks that combine aspects of mCPnet and our approach, e.g., one in which the preferential dominance of one outcome over another is decided by voting and whether such dominance implies that the dominating outcome from that comparison indeed is preferred over a set of outcomes using logical conditions (similar to the ones discussed in this paper). 
%It would be interesting to formulate a voting-based semantics when the set of outcomes satisfying $\psi_2$ is not a singleton. In particular, consider the question: \emph{What are the outcomes that are preferred to a set of outcomes (say, $\{o_1, o_2\}$) by multiple  stakeholders?} Suppose there is an outcome $o$ that is preferred to $o_1$ by the majority of stakeholders (which could be decided using voting), and it may be less preferred to $o_2$ by the majority of stakeholders. In this case, one may not want to include $o$ among the outcomes that are preferred to  the set of outcomes $\{o_1, o_2\}$. Thus, one may want to include an outcome in the solution set if it is preferred to one of $o_1$ or $o_2$ (similar to the witness condition in the paper), and both $o_1$ and $o_2$ are not preferred to $o$ (similar to the agreement condition in the paper), with the pair-wise outcome preferences decided using a voting mechanism.\\  
The framework introduced in this paper is especially useful in applications where it is necessary for multiple stakeholders to be able to express, explore and understand the implications of their preferences in  settings where (i) the individual stakeholder preferences are naturally expressed over attributes of outcomes (as opposed to outcomes themselves), and are sufficiently nuanced to require more expressive preference languages e.g., TCP-nets \cite{Brafman:JAIR06} (which involve tradeoffs between conditional preferences), CI-nets \cite{Bouveret:IJCAI2009} (which can express preferences between sets of objects), or their generalizations \cite{Santhanam:book2016}; and (ii) there is a need for explanations of the role played by the preferences of different stakeholders in determining the outcomes of multi-stakeholder deliberations. One can envision extending this approach to allow individual stakeholders, once they understand the impact of their respective preferences, to minimally revise their preferences to arrive at a consensus that might otherwise have eluded them. \\

\noindent\textbf{Work in progress.} Work in Progress aims to (i) consider organizational structures that further constrain how preferences of multiple stakeholders influence outcomes (e.g., preferences of superiors overriding those of subordinates) (ii) generate targeted explanations of the answers to multi-stakeholder preference queries, (iii) support interactive revision of preferences by stakeholders in the search for consensus or compromise, (iv) further optimize the implementation of the multi-stakeholder preference reasoner, and rigorously assess its scalability as a function of the relevant factors, and (v) apply the resulting tools to support multi-stakeholder decision-making in public policy, healthcare, etc.

\ack This work is supported in part by the National Science Foundation through grants IIS 2225823 and IIS 2225824. 

%\printbibliography
%\newpage
\typeout{}
%\bibliographystyle{named}
\bibliography{references}


%\clearpage
%\onecolumn
\section{Proofs for Subsection ~\ref{subsec:PGD}}

\begin{lemma}\label{lem:1appen}
If $s \leq \min \{\frac{2\lambda}{G^2}, \frac{1}{L}\}$, then 
\begin{equation}
    \sp (\sigma(\xm^{t+1})) \subseteq \sp (\sigma(\xm^{t})), \textrm{for } t \geq 0,
\end{equation}
and
\begin{equation}
   \rk(\xm^{t+1}) \leq  \rk(\xm^{t}), \textrm{for } t \geq 0,
\end{equation}
which means the support of the singular value vectors of the sequence $\{\xm^t\}_t$ shrinks, also the rank of the sequence $\{\xm^t\}_t$ decreases.
\end{lemma}

\begin{proof}
Let $\bar{\xm}^{t+1} = \xm^t - s\nabla g(\xm^t), \qm^t = -s\nabla g(\xm^t),$ thus we have $\bar{\xm}^{t+1} = \xm^t + \qm^t$. With Weyl's inequality 
\begin{equation}
    \sigma_{i+j-1} (\am + \bm) \leq \sigma_i(\am) + \sigma_j(\bm),
\end{equation}
we get 
\begin{equation}
    \sigma_i(\bar{\xm}^{t+1}) \leq \sigma_i(\xm^t) + \sigma_1(\qm^t) = \sigma_1(\qm^t) \leq sG,
    \textrm{for all } i \textrm{ where } \sigma_i(\xm^t) = 0.  
\end{equation}

With $s \leq \min \{\frac{2\lambda}{G^2}, \frac{1}{L}\}$, we have $\sigma_i(\bar{\xm}^{t+1})  \leq \sqrt{2\lambda s}$, therefore $\sigma_i(\xm^{t+1})  = 0$. So the zero elements of $\sigma(\xm^t)$ remain unchanged in $\sigma(\xm^{t+1})$, and  $\sp (\sigma(\xm^{t+1})) \subseteq \sp (\sigma(\xm^{t}))$, $\rk(\xm^{t+1}) \leq  \rk(\xm^{t})$. 
\end{proof}



\begin{lemma}\label{lem:2appen}
The sequence of the objective $\{F(\xm^t)\}_t$ is nonincreasing, and the following inequality holds for all $t \geq 0$:
\begin{equation}
    F(\xm^{t+1}) \leq F(\xm^t) - (\frac{1}{2s} - \frac{L}{2})\|\xm^{t+1} - \xm^t\|^2_F.
\end{equation}
\end{lemma}

\begin{proof}
    Let $\bar{\xm}^{t+1} = \xm^t - s\nabla g(\xm^t), \qm^t = -s\nabla g(\xm^t),$ thus we have $\bar{\xm}^{t+1} = \xm^t + \qm^t$,
    and 
    \begin{equation}
        \xm^{t+1} = \argmin_{\vm} \frac{1}{2s} \|\vm - \bar{\xm}^{t+1} \|^2_F + h(\vm).
    \end{equation}
    Let $\vm = \xm^t$, we get
    \begin{equation}\label{eq:ap1}
        \langle \nabla g(\xm^t), \xm^{t+1} - \xm^t \rangle + \frac{1}{2s} \|\xm^{t+1} - \xm^t\|^2_F + h(\xm^{t+1}) \leq h(\xm^t).
    \end{equation}
    In addition, we have
    \begin{equation}\label{eq:ap2}
        g(\xm^{t+1}) \leq g(\xm^t) + \langle \nabla g(\xm^t), \xm^{t+1} - \xm^t \rangle + \frac{L}{2} \|\xm^{t+1} - \xm^t\|^2_F.
    \end{equation}
Combine Eq.~(\ref{eq:ap1}) and Eq.~(\ref{eq:ap2}) together, we get
\begin{equation}
    F(\xm^{t+1}) \leq F(\xm^t) - (\frac{1}{2s} - \frac{L}{2}) \|\xm^{t+1} - \xm^t \|^2_F,
\end{equation}
    since $s \leq \frac{1}{L}$, we have $\frac{1}{2s} \geq \frac{L}{2}$, so the sequence $\{ F(\xm^t)\}_t$ is decreasing with lower bound 0. 
\end{proof}




\begin{lemma}~\cite{laurent2000adaptive}\label{lem:aappen}
    Let $Y_1, Y_2, \dots, Y_D$ be i.i.d. Gaussian random variables with 0 mean and unit variance, and $a_1, a_2, \dots, a_D$ be D positive numbers. Define $Z = \sum a_i(Y_i^2 - 1)$ and $\av = [a_1, a_2, \dots, a_D]^T$, then for any $t > 0$, 
    \begin{equation}
        \textrm{Probability} (Z \geq 2\|\av\|_2 \sqrt{t} + 2\|\av\|_{\infty} t) \leq e^{-t}.
    \end{equation}
\end{lemma}


\begin{lemma}~\cite{davidson2001local}\label{lem:bappen}
    Suppose $\am \in \mathbb{R}^{m \times n} ( m \geq n)$ is a random matrix whose entries are i.i.d. sampled from the standard Gaussian distribution \textit{N}$(0,\frac{1}{m})$, then 
    \begin{equation}
        1-\sqrt{\frac{n}{m}} \leq E(\sigma_n (\am)) \leq E(\sigma_1(\am)) \leq 1+ \sqrt{\frac{n}{m}}.
    \end{equation}
    And for any $t>0$,
    \begin{equation}
         \textrm{Probability}(\sigma_n(\am) \leq 1-\sqrt{\frac{n}{m}} - t) < e^{-\frac{mt^2}{2}},
    \end{equation}
    \begin{equation}
        \textrm{Probability}(\sigma_1(\am) \geq 1+\sqrt{\frac{n}{m}} + t) < e^{-\frac{mt^2}{2}}.
    \end{equation}
\end{lemma}


\begin{theorem}\label{thm:1appen}
Suppose $\dm \in \mathbb{R}^{d \times n} (n \geq d)$ is a random matrix with elements i.i.d. sampled from the standard Gaussian distribution \textit{N}(0,1), then 
\begin{equation}
    \textrm{Probability}(\frac{1}{L} \leq \frac{2\lambda}{G^2}) \geq 1-e^{-\frac{a^2}{2}} - ne^{-a},
\end{equation}
if 
\begin{equation}\label{eq:un}
    n \geq (\sqrt{d} + a + \sqrt{\frac{(d+2\sqrt{da}+2a)(x_0 + \lambda |S|))}{\lambda}})^2,
\end{equation}
where $x_0 = \|\ym - \dm \xm^0\|^2_F$, $S = \sp(\sigma(\xm^0))$, and $a$ can be chosen as $a_0 \textrm{log}n$ for $a_0 > 0$ to ensure that Eq.~(\ref{eq:un}) holds with high probability. 
\end{theorem}

\begin{proof}
    Based on Lemma~\ref{lem:bappen}, for any $a > 0$, with probability $\geq 1- e^{-\frac{a^2}{2}}$,
    \begin{equation}
        \sigma_{max}(\dm) > \sqrt{n} - \sqrt{d} - a,
    \end{equation}
    and by Lemma~\ref{lem:aappen}, for any $1 \leq i \leq n$ and $a > 0$, with probability $\geq 1- e^{-a}$, 
    \begin{equation}
        \|\dm_i\|_2 \leq \sqrt{d+2\sqrt{da} + 2a},
    \end{equation}
    where $\dm_i$  denotes $i$-th column of $\dm$. Then, it can be verified with the union bound that with probability $\geq 1-e^{-\frac{a^2}{2}} - ne^{-a}$, 
    \begin{equation}
        \frac{2D^2(x_0+\lambda|S|)}{\lambda} \leq 2\sigma_{max}^2(\dm),
    \end{equation}
    where $D = \max_i \|i_{th} \textrm{ column of }\dm\|_2$,
    if 
    \begin{equation}\label{eq:unres}
    n \geq (\sqrt{d} + a + \sqrt{\frac{(d+2\sqrt{da}+2a)(x_0 + \lambda |S|))}{\lambda}})^2.
\end{equation}
\end{proof}


\begin{lemma}\label{lem:3appen}
(a) All the elements of each subsequence $\xt^k$ ($k = 1, \dots, K)$ in the subsequences with shrinking support have the same support. In addition, for any $1 \leq k_1 < k_2 \leq K$ and any $\xm^{t_1} \in \xt^{k_1}$, and $\xm^{t_2} \in \xt^{k_2}$, we have $t_1 < t_2$ and $\sp(\sigma(\xm^{t_2})) \subset \sp(\sigma(\xm^{t_1}))$. 

(b) All the subsequences except for the last one, $\xt^k$ ($k = 1, \dots, K-1)$ have finite size, and $\xt^K$ have an infinite number of elements, and there exists some $t_0 \geq 0$ such that $\{\xm^t\}_{t=t_0}^\infty \subseteq \xt^K$.
\end{lemma}

\begin{proof}
    (a)
    For any $1 \leq k < K$, let $\xm^{t_1}, \xm^{t_2} \in \xt^k$ and $t_1 \neq t_2$. 
    If $t_1 < t_2$, then $\sp(\sigma(\xm^{t_2})) \subseteq \sp(\sigma(\xm^{t_1}))$ according to the support shrinkage property in Lemma~\ref{lem:1appen}. If $\sp(\sigma(\xm^{t_2})) \subset \sp(\sigma(\xm^{t_1}))$ then $|\sp(\sigma(\xm^{t_2}))| < |\sp(\sigma(\xm^{t_1}))|$, which contradicts with the definition of $\xt^k$ whose elements have the same support size. A similar argument holds for $t_2 < t_1$. Therefore, all the elements of each subsequence $\xt^k (1 \leq k \leq K)$ have the same support.

    For any $1 \leq k_1 \leq k_2 \leq K$ and any $\xm^{t_1} \in \xt^{k_1}$ and $\xm^{t_2} \in \xt^{k_2}$, note that $t_1 \neq t_2$ and $\sp(\sigma(\xm^{t_1})) \neq \sp(\sigma(\xm^{t_2}))$ since $\xt^{k_1}$ and $\xt^{k_2}$ have different support size. Suppose $t_1 > t_2$, we have $\sp(\sigma(\xm^{t_1})) \subset \sp(\sigma(\xm^{t_2}))$ and it follows that $|\sp(\sigma(\xm^{t_1}))| < 
|\sp(\sigma(\xm^{t_2}))|$, again it contradicts with the Definition~\ref{def:1}. Thus, we must have $t_1 < t_2$, and it follows that   $\sp(\sigma(\xm^{t_2})) \subset \sp(\sigma(\xm^{t_1}))$.


(b)
Suppose $\xt^k$ is an infinite sequence for some $1 \leq k \leq K-1$. We can get an infinite sequence from $\xt^k$ as follows:

We have some $\xm^{t_0} \in \xt^k$ for some $t_0 > 0$ since $\xt^k$ is not empty. Suppose we get $\{\xm^{t'_j}\}_{j' = 0}^j$ in the first $j \geq 0$ steps with increasing indices $\{t'_j\}$. Since $\xt^k$ is an infinite sequence, $\xt^k \backslash \{\xm^{t'_j}\}_{j' = 0}^j $ is still an infinite sequence. At the $(j+1)$-th step, we can find $\xm^{t_{j+1}} \in \xt^k \backslash \{\xm^{t'_j}\}_{j' = 0}^j $ with $t_{j+1} > t_j$. Therefore, we are able to get an infinite sequence $\{\xm^{t_j}\}_{j' = 0}^{\infty} \subseteq \xt^k$ with increasing indices $\{t_j\}$. With the fact that the indices $\{t_j\}$ is increasing, we can see that $\lim_{j \rightarrow \infty} t_j = \infty$.

For any element $\xm^q \in \xt^{k+1}$, there must exist some $j > 0$ such that $q \leq t_j$, according to the support shrinkage property we must have $\sp(\sigma(\xm^{t_j})) \subseteq \sp(\sigma(\xm^{q}))$, and $|\sp(\sigma(\xm^{t_j}))| \leq |\sp(\sigma(\xm^{q}))|$. 
On the other hand, since $\xm^{t_j} \in \xt^k$, we have $|\sp(\sigma(\xm^{q}))| < |\sp(\sigma(\xm^{t_j}))|$.
This contradiction shows that each $\xt^k (1\leq k \leq K-1)$ must have a finite size. 
Also, $\{\xm^t\}_{t=0}^{\infty}$ is an infinite sequence and $\{\xt^k\} _{k=1}^{K}$ form a disjoint cover of it, thus $\xt^K$ must contain infinite number of elements.  

According to the proof of (a), there exists an infinite sequence $\{\xm^{t_j}\}_{j=0}^{\infty} \subseteq \xt^K$, and $\lim_{j \rightarrow \infty} t_j = \infty$. For any $t > t_0$, there must be some $t'_j$ with $j' \geq 1$ such that $t_{j'-1} \leq t \leq t_{j'}$. Then we have 
\begin{equation}
    \sp(\sigma(\xm^{t_{j'}})) = S^* \subseteq \sp(\sigma(\xm^t)) \subseteq \sp(\sigma(\xm^{t_{j'-1}})) = S^*,
\end{equation}
therefore we have $|\sp(\sigma(\xm^t))| = |S^*|$ and $\xm^t \in \xt^K$ for any $t \geq t_0$, which is $\{\xm^{t}\}_{t=t_0}^{\infty} \subseteq \xt^K$.
\end{proof}


\begin{theorem}\label{thm:2appen}
    Suppose $s \leq \min \{\frac{2\lambda}{G^2}, \frac{1}{L}\}$. and $\xm^*$ is a limit point of $\{\xm^{t}\}_{t=0}^{\infty}$, 
    and $\sigma(\xm^*)$ is a limit point of $\{\sigma(\xm^{t})\}_{t=0}^{\infty}$, 
    then the sequence $\{\xm^{t}\}_{t=0}^{\infty}$ generated by Algorithm~\ref{alg:1} converges to $\xm^*$,
    % and the sequence $\{\sigma(\xm^{t})\}_{t=0}^{\infty}$ converges to $\sigma(\xm^*)$, 
    $\xm^*$ is a critical point of F($\cdot$), and $\sp(\sigma(\xm^*)) = S^*$, where $S^*$ is the support of any element in $\xt^K$.
    Moreover, there exists $t_0 \geq 0$ such that for all $m \geq t_0$, we have
    \begin{equation}
        F(\xm^{m+1}) - F(\xm^*) \leq \frac{1}{2s(m-t_0+1)}\|\xm^{t_0} - \xm^*\|_F^2.
    \end{equation}
\end{theorem}

\begin{proof}
    Let $S^*$ denote the support of any element in $\xt^K$. First we have $\sp(\sigma(\xm^*)) \subseteq S^*$, otherwise, pick an arbitrary $i \in \sp(\sigma(\xm^*)) \backslash S^*$, then $\|\sigma(\xm^{t_j}) - \sigma(\xm^*)\|_2 \geq  |\sigma_i (\xm^*)|$ contradicts with the fact that $\sigma(\xm^{t_j}) \rightarrow \sigma(\xm^*)$.

    Moreover, suppose $\sp(\sigma(\xm^*)) \subset S^*$, we can pick an arbitrary $i \in S^* \backslash \sp(\sigma(\xm^*))$. And it can be shown that $\sigma_i(\xm^{t_j}) \rightarrow 0$. Otherwise there exists $\epsilon > 0$, for any $j$, there exists $j' \geq j$ such that $|\sigma_i(\xm^{t_{j'}})| \geq \epsilon$. It follows that $\|\sigma(\xm^{t_{j'}}) - \sigma(\xm^*) \|_2 \geq |\sigma_i(\xm^{t_{j'}}| \geq \epsilon$, contradicting with the fact that $\sigma(\xm^{t_j}) \rightarrow \sigma(\xm^*)$.

Let $\epsilon > 0$ be a sufficiently small positive number such that $sG + \epsilon < \sqrt{2\lambda s}$. Since $\sigma_i (\xm^{t_j}) \rightarrow 0$, there exists sufficiently large $j$ such that $|\sigma_i(\xm^{t_j})| < \epsilon$. Let $\bar{\xm}^{t_j+1} = \xm^{t_j} - s\nabla g(\xm^{t_j})$, then
\begin{equation}
    |\sigma_i (\bar{\xm}^{t_j+1})| \leq |\sigma_i (\xm^{t_j})| + sG < \epsilon + sG \leq \sqrt{2\lambda s}.
\end{equation}
Then according to the update rule we have $\sigma_i(\xm^{t_j+1}) = 0$, so $\sp(\sigma(\xm^{t_j+1})) \subseteq \sp(\sigma(\xm^{t_j})) \backslash \{i\}$. On the other hand, $\xm^{t_j+1} \in \xt^k$, so we have $\sp(\sigma(\xm^{t_j+1})) = \sp(\sigma(\xm^{t_j}))$. Such contradict shows that $\sp(\sigma(\xm^*)) \subset S^*$ cannot be true. So $\sp(\sigma(\xm^*)) = S^*$.

Now we will show that $\{\xm^{t}\}_{t=t_0}^{\infty}$ converges to $\xm^*$. 

For any $\vm, \um$, we have
\begin{equation}
    g(\vm) \leq g(\um) + \langle \nabla g(\um), \vm - \um \rangle + \frac{L}{2}\|\vm-\um\|_F^2,
\end{equation}
also since $g(\cdot)$ is convex, for any $\vm$ and $t \geq 0$:
\begin{equation}
    g(\xm^{t+1}) + \langle \nabla g(\xm^{t+1}), \vm - \xm^{t+1} \rangle \leq g(\vm).
\end{equation}
    Also, since
    \begin{equation}
        \xm^{t+1} = \argmin_{\vm} \frac{1}{2s} \|\vm - (\xm^t - s\nabla g(\xm^t))\|^2_F + h(\vm),
    \end{equation}
    we have
    \begin{equation}
        -\nabla g(\xm^t) - \frac{1}{s} (\xm^{t+1} -\xm^t) \in \partial h(\xm^{t+1}),
    \end{equation}
    and 
    \begin{equation}\begin{split}
        &\frac{1}{s}(\xm^{t+1} - (\xm^t - s\nabla g(\xm^t))) + \partial h(\xm^{t+1}) = 0,\\
        & \frac{1}{s}(\sum_{i \in \sp(\sigma(\xm^{t+1}))} \sigma_i \uv_i \vv_i^T - \sum_{i} \sigma_i \uv_i \vv_i^T) + \partial h(\xm^{t+1}) = 0,
    \end{split}\end{equation}
    where $\xm^t - s\nabla g(\xm^t) = \sum_i \sigma_i \uv_i \vv_i^T$ is the singular value decomposition,
    then it follows
    \begin{equation}
        \partial h(\xm^{t+1}) = \frac{1}{s} \sum_{i \notin \sp(\sigma(\xm^{t+1}))} \sigma_i \uv_i \vv_i^T,
    \end{equation}
therefore 
\begin{equation}
    \langle \partial h(\xm^{t+1}), \xm^{t+1} \rangle = 0.
\end{equation}
    For any matrix $\vm$ such that $\sp(\sigma(\vm)) = \sp(\sigma(\xm^{t+1}))$, we have $h(\vm) = h(\xm^{t+1}) + \langle \partial h(\xm^{t+1}), \xm^{t+1} \rangle$. 

For $t \geq t_0$, we have 
\begin{equation}\begin{split}\label{eq:33}
    F(\xm^{t+1}) & \leq g(\xm^t) + \langle \nabla g(\xm^t), \xm^{t+1} -\xm^t \rangle + \frac{L}{2} \|\xm^{t+1}-\xm^t\|^2_F + h(\xm^{t+1}) \\
    & \leq g(\vm) + \langle \nabla g(\xm^t), \xm^{t+1}-\vm \rangle + \langle \nabla g(\xm^t), \xm^{t+1} -\xm^t \rangle + \frac{L}{2} \|\xm^{t+1}-\xm^t\|^2_F + h(\xm^{t+1}) \\
    & = g(\vm) + \langle \nabla g(\xm^t), \xm^{t+1} - \vm \rangle + \frac{L}{2} \|\xm^{t+1}-\xm^t\|^2_F + h(\xm^{t+1}) \\
& = g(\vm) + \langle \nabla g(\xm^t), \xm^{t+1} - \vm \rangle + \frac{L}{2} \|\xm^{t+1}-\xm^t\|^2_F + h(\vm) + \langle \nabla g(\xm^t) + \frac{1}{s} (\xm^{t+1} - \xm^t), \vm-\xm^{t+1} \rangle \\
& = F(\vm) + \frac{1}{s} \langle \xm^{t+1} -\xm^t, \vm-\xm^{t+1} \rangle + \frac{L}{2} \|\xm^{t+1} - \xm^t\|^2_F \\    
& = F(\vm) + \frac{1}{s} \langle \xm^{t+1} -\xm^t, \vm-\xm^{t} \rangle -\frac{1}{s} \|\xm^{t+1} -\xm^t \|^2_F + \frac{L}{2} \|\xm^{t+1} - \xm^t\|^2_F \\  
& = F(\vm) + \frac{1}{s}  \langle \xm^{t+1} -\xm^t, \vm-\xm^{t} \rangle -(\frac{1}{s} - \frac{L}{2}) \|\xm^{t+1} - \xm^t\|^2_F  \\
    & \leq F(\vm) + \frac{1}{s} \langle \xm^{t+1} - \xm^t, \vm - \xm^t \rangle - \frac{1}{2s} \|\xm^{t+1} - \xm^t\|^2_F.
\end{split}\end{equation}
Now suppose $\sp(\sigma(\xm^*)) = \sp(\sigma(\xm^{t+1})) = S^*$, let $\vm = \xm^*$, we have
\begin{equation}
    F(\xm^{t+1}) - F(\xm^*) \leq \frac{1}{s} \langle \xm^{t+1} - \xm^t, \xm^* - \xm^t \rangle  - \frac{1}{2s} \|\xm^{t+1} - \xm^t\|^2_F = \frac{1}{2s} (\|\xm^{t} - \xm^*\|^2_F - \|\xm^{t+1} - \xm^*\|^2_F).
\end{equation}
    Now, sum the above equation over $t = t_0,\dots,m$ with $m \geq t_0$, we get
\begin{equation}
    \sum_{t = t_0}^m F(\xm^{t+1}) - F(\xm^*) \leq
    \sum_{t = t_0}^m \frac{1}{2s} (\|\xm^{t} - \xm^*\|^2_F - \|\xm^{t+1} - \xm^*\|^2_F) 
    = \frac{1}{2s} (\|\xm^{t_0} - \xm^*\|^2_F - \|\xm^{m+1} - \xm^*\|^2_F).
\end{equation}
Since $\{F(\xm^t)\}_t$ is non-increasing,
$\sum_{t = t_0}^m F(\xm^{t+1}) - F(\xm^*) > (m - t_0 +1) F(\xm^{m+1}) - F(\xm^*) $, therefore,
\begin{equation}
    F(\xm^{m+1}) - F(\xm^*) \leq \frac{1}{2s(m - t_0 +1)} (\|\xm^{t_0} - \xm^*\|^2_F - \|\xm^{m+1} - \xm^*\|^2_F) \leq 
     \frac{1}{2s(m - t_0 +1)} (\|\xm^{t_0} - \xm^*\|^2_F). 
\end{equation}

Again, since $\xm^{t+1} = \argmin_{\vm} \langle \nabla g(\xm^t), \vm - \xm^t \rangle + \frac{1}{2s}\|\vm - \xm^t\|^2_F + h(\vm)$, then 
\begin{equation}
    \langle \nabla g(\xm^t), \xm^{t+1}-\xm^t \rangle + \frac{1}{2s}\|\xm^{t+1} - \xm^t \|^2_F + h(\xm^{t+1}) 
    \leq  \langle \nabla g(\xm^t), \xm^{t}-\xm^t \rangle + \frac{1}{2s}\|\xm^{t} - \xm^t \|^2_F + h(\xm^{t}) = h(\xm^t).
\end{equation}
Therefore, 
\begin{equation}\begin{split}
    &F(\xm^{t+1}) \leq g(\vm) + \langle \nabla g(\xm^t), \xm^{t+1}-\vm \rangle  + \frac{L}{2}\|\xm^{t+1} -\xm^t\|^2_F + h(\xm^{t+1}) \\
    & \leq g(\vm)  + \langle \nabla g(\xm^t), \xm^{t+1}-\vm \rangle + \frac{L}{2}\|\xm^{t+1} -\xm^t\|^2_F + h(\xm^t) - \langle \nabla g(\xm^t), \xm^{t+1}-\xm^t \rangle - \frac{1}{2s}\|\xm^{t+1} - \xm^t \|^2_F.
\end{split}\end{equation}
Let $\vm = \xm^t$, we get $F(\xm^{t+1}) \leq F(\xm^t) - (\frac{1}{2s} - \frac{L}{2}) \|\xm^{t+1} - \xm^t\|^2_F.$
Thus, we have
\begin{equation}
    (\frac{1}{2s} - \frac{L}{2}) \sum_{t=0}^{\infty} \|\xm^{t+1} - \xm^t\|^2_F
    \leq F(\xm^0) - F(\xm^*) < \infty,
\end{equation}
then
\begin{equation}
    \|\xm^{t+1} - \xm^t\|^2_F \rightarrow 0, \textrm{ as } t \rightarrow \infty.
\end{equation}

Now we show $\xm^*$ is a critical point of $F(\cdot)$. For $t_j \geq 1$, we have 
\begin{equation}
    \nabla g(\xm^{t_j}) - \nabla g(\xm^{t_j-1}) - \frac{1}{s}(\xm^{t_j} - \xm^{t_j-1}) \in \partial F(\xm^{t_j}),
\end{equation}
when $j \rightarrow \infty$ we have 
\begin{equation}\begin{split}
   \|\partial F(\xm^{t_j})\|_F & =  \|\nabla g(\xm^{t_j}) - \nabla g(\xm^{t_j-1}) - \frac{1}{s}(\xm^{t_j} - \xm^{t_j-1}) \|_F \\
   & \leq L\| \xm^{t_j} - \xm^{t_j-1}\|_F +\frac{1}{s}\|\xm^{t_j} - \xm^{t_j-1}\|_F \rightarrow 0.
\end{split}\end{equation}
Also, when $j \rightarrow \infty$,
\begin{equation}
    F(\xm^{t_j}) = g(\xm^{t_j}) + h(\xm^{t_j}) = g(\xm^{t_j}) + \lambda|S^*| 
    \rightarrow g(\xm^*) + \lambda|S^*|  = F(\xm^*).
\end{equation}
Therefore, $0 \in \partial F(\xm^*) $ and $\xm^*$ is a critical point. 
    \end{proof}










\section{Proofs for Subsection \ref{subsec:acc}}


\begin{lemma}\label{lem:4appen}
    The sequence $\{\xm^t\}_{t=1}^{\infty}$ generated by Algorithm~\ref{alg:2} satisfies
    \begin{equation}
        \sp(\sigma(\xm^{t+1})) \subseteq \sp(\sigma(\xm^{t})), t \geq 1.
    \end{equation}
\end{lemma}
\begin{proof}
    We will prove the above lemma using mathematical induction. 

    When $t=1$, we have $\um^1 = \xm^1, \vm^1 = \xm^1$, $\xm^2 = T_{\sqrt{2\lambda s}} (\xm^1 - s\nabla g(\xm^1))$,
    using the argument in the proof of Lemma~\ref{lem:1appen}, we have 
    \begin{equation}
        \sp(\sigma(\xm^2)) \subseteq \sp(\sigma(\xm^1)). 
    \end{equation}

    Suppose  $\sp(\sigma(\xm^{t+1})) \subseteq \sp(\sigma(\xm^{t}))$ holds for all $t \leq t'$ with $t' \geq 1$, now consider the case that $t = t'+1$. Based on the update rule for $\vm^t$, we have 
    \begin{equation}
        \sp(\sigma(\vm^{t'+1})) \subseteq \sp(\sigma(\xm^{t'+1})). 
    \end{equation}
    Let $\bar{\xm}^{t'+2} = \vm^{t'+1} - s\nabla g(\vm^{t'+1})$, then $\sigma_i(\xm^{t'+2}) = 0$ for any $i \notin \sp(\sigma(\vm^{t'+1}))$ since $\sigma_i(\bar{\xm}^{t'+2}) \leq \sqrt{2\lambda s}$ for such $i$. So the zero elements in $\sigma(\vm^{t'+1})$ remain unchanged in $\sigma(\xm^{t'+2})$, and it follows that 
    \begin{equation}
        \sp(\sigma(\xm^{t'+2})) \subseteq \sp(\sigma(\vm^{t'+1})) \subseteq \sp(\sigma(\xm^{t'+1})),
    \end{equation}
    therefore $\sp(\sigma(\xm^{t+1})) \subseteq \sp(\sigma(\xm^{t}))$ holds for $t = t'+1$. 
    Based on mathematical induction it holds for all $t \geq 1$.
\end{proof}



\begin{theorem}\label{thm:3appen}
      Suppose $s \leq \min \{\frac{2\lambda}{G^2}, \frac{1}{L}\}$. and $\xm^*$ is a limit point of $\{\xm^{t}\}_{t=0}^{\infty}$ generated by Algorithm~\ref{alg:2}, 
  then there exists $t_0 \geq 1$ such that for all $m \geq t_0$, we have
    \begin{equation}
        F(\xm^{m+1}) - F(\xm^*) \leq \frac{4}{(m+1)^2}V^{t_0},
    \end{equation}
    where $V^{t_0}$ is a value defined as
    \begin{equation}\begin{split}
        V^{t_0} =  \frac{1}{2s} \|(\alpha^{t_0-1}-1)\xm^{t_0-1} - \alpha^{t_0-1}\xm^{t_0} +\xm^*\|_F^2 + (\alpha^{t_0-1})^2(F(\xm^{t_0})-F(\xm^*)).
    \end{split}\end{equation}
\end{theorem}

\begin{proof}
    According to Lemma~\ref{lem:4appen}, there exists $t_0 \geq 0$ such that $\{\xm^t\}_{t=t_0}^\infty \subseteq \xt^K$. It follows that $\sp(\sigma(\xm^*)) = S^*$. 

    When $\sp(\sigma(\wm)) = \sp(\sigma(\xm^{t+1}))$ for $t \geq t_0$, with the similar process in Eq.~(\ref{eq:33}), we get
    \begin{equation}
        F(\xm^{t+1}) \leq F(\wm) + \frac{1}{s} \langle \xm^{t+1} - \vm^t, \wm - \vm^t \rangle - (\frac{1}{s} - \frac{L}{2}) \|\xm^{t+1} - \vm^t\|^2_F,
    \end{equation}
    Let $\wm = \xm^t$ and $\wm = \xm^*$, we get
    \begin{equation}\label{eq:47}
        F(\xm^{t+1}) \leq F(\xm^t) + \frac{1}{s} \langle \xm^{t+1} - \vm^t, \xm^t - \vm^t \rangle - (\frac{1}{s} - \frac{L}{2}) \|\xm^{t+1} - \vm^t\|^2_F,
    \end{equation}
    and
    \begin{equation}\label{eq:48}
        F(\xm^{t+1}) \leq F(\xm^*) + \frac{1}{s} \langle \xm^{t+1} - \vm^t, \xm^* - \vm^t \rangle - (\frac{1}{s} - \frac{L}{2}) \|\xm^{t+1} - \vm^t\|^2_F,
    \end{equation}
    $(\alpha^t-1) \times $Eq.~(\ref{eq:47}) + Eq.~(\ref{eq:48}), we obtain
    \begin{equation}\begin{split}\label{eq:49}
        &\alpha^t F(\xm^{t+1}) - (\alpha^t -1)F(\xm^t) - F(\xm^*) \\
        \leq & \frac{1}{s} \langle \xm^{t+1} -\vm^t, (\alpha^t-1)(\xm^t-\vm^t) + \xm^* - \vm^t \rangle - \alpha^t(\frac{1}{s} - \frac{L}{2}) \|\xm^{t+1} - \vm^t\|^2_F.
    \end{split}\end{equation}
    Multiply both sides of Eq.~(\ref{eq:49}) by $\alpha^t$, and use the fact that $(\alpha^t)^2 - \alpha^t = (\alpha^{t-1})^2$, we have
    \begin{equation}\begin{split}
        & (\alpha^t)^2(F(\xm^{t+1}) - F(\xm^*)) - (\alpha^{t-1})^2(F(\xm^{t}) - F(\xm^*))  \\
        \leq  &
        \frac{1}{2s}(\|(\alpha^t-1)\xm^t - \alpha^t \vm^t + \xm^* \|^2_F - \|(\alpha^{t}-1)\xm^t - \alpha^t \xm^{t+1} + \xm^* \|^2_F).
    \end{split}\end{equation}
    Since for any matrix $\am, \bm, \cm$, when $\sp(\sigma(\am)) \subseteq \sp(\sigma(\cm))$, we have $\|\am - P_{\sp(\sigma(\cm))} (\bm)\|_F \leq \|\am-\bm\|_F$, and $\vm^t = P_{\sp(\sigma(\xm^t))} (\um^t)$, it follows that
    \begin{equation}\begin{split}\label{eq:51}
        &(\alpha^t)^2(F(\xm^{t+1}) - F(\xm^*)) - (\alpha^{t-1})^2(F(\xm^{t}) - F(\xm^*)) \\
        \leq &
        \frac{1}{2s}(\|(\alpha^t-1)\xm^t - \alpha^t \um^t + \xm^* \|^2_F - \|(\alpha^{t}-1)\xm^t - \alpha^t \xm^{t+1} + \xm^* \|^2_F).
    \end{split}\end{equation}
    For simplicity, we define $\am^{t+1} = (\alpha^t-1)\xm^t - \alpha^t \xm^{t+1} + \xm^*, \am^{t} = (\alpha^{t-1}-1)\xm^{t-1} - \alpha^{t-1} \xm^t + \xm^*$, according to the update rule for $\um^t$, we can get $\am^t = (\alpha^t-1)\xm^t - \alpha^t \um^t + \xm^*$, then based on Eq.~(\ref{eq:51}), we obtain
    \begin{equation}\label{eq:52}
        (\alpha^t)^2(F(\xm^{t+1}) - F(\xm^*)) - (\alpha^{t-1})^2(F(\xm^{t}) - F(\xm^*)) 
        \leq 
        \frac{1}{2s}(\|\am^t\|^2_F - \| \am^{t+1}\|^2_F).
    \end{equation}
    Sum Eq.~(\ref{eq:52}) over $t = t_0, \dots, m$ for $m \geq t_0$, we have
    \begin{equation}
         (\alpha^m)^2(F(\xm^{m+1}) - F(\xm^*)) - (\alpha^{t_0-1})^2(F(\xm^{t_0}) - F(\xm^*)) 
        \leq 
        \frac{1}{2s}(\|\am^{t_0}\|^2_F - \| \am^{m+1}\|^2_F)
        \leq 
        \frac{1}{2s}\|\am^{t_0}\|^2_F,
    \end{equation}
    therefore, with $\alpha^t \geq \frac{t+1}{2}$, we get
    \begin{equation}\begin{split}
        F(\xm^{m+1}) - F(\xm^*) 
        \leq &
        \frac{1}{2s(\alpha^m)^2} \|\am^{t_0}\|^2_F + \frac{(\alpha^{t_0-1})^2}{(\alpha^m)^2} (F(\xm^{t_0}) - F(\xm^*)) \\
        \leq  &
        \frac{4}{(m+1)^2}(\frac{1}{2s} \|\am^{t_0}\|^2_F + (\alpha^{t_0-1})^2(F(\xm^{t_0}) - F(\xm^*))),
    \end{split}\end{equation}
    where $\am^{t_0} = (\alpha^{t_0-1}-1)\xm^{t_0-1} - \alpha^{t_0-1}\xm^{t_0} +\xm^*$.
\end{proof}



\section{Proofs for Subsection \ref{subsec:mon}}


\begin{lemma}\label{lem:5appen}
    The sequence $\{\zm^t\}_{t=1}^{\infty}$ and $\{\xm^t\}_{t=1}^{\infty}$ generated by Algorithm~\ref{alg:3} satisfies
     \begin{equation}
        \sp(\sigma(\zm^{t+1})) \subseteq \sp(\sigma(\zm^{t})),
        \sp(\sigma(\xm^{t+1})) \subseteq \sp(\sigma(\xm^{t})), t \geq 1.
    \end{equation}
\end{lemma}

\begin{proof}
    We will prove the above lemma using mathematical induction. 

    It can be easily verified that $\sp(\sigma(\zm^2)) \subseteq \sp(\sigma(\zm^1))$. 
    
     Suppose  $\sp(\sigma(\zm^{t+1})) \subseteq \sp(\sigma(\zm^{t}))$ holds for all $t \leq t'$ with $t' \geq 1$, now consider the case that $t = t'+1$. With the similar thought process in the proof for Lemma~\ref{lem:4appen}, based on the update rule for $\wm^t$, the zero elements in $\sigma(\vm^{t'+1})$ remain unchanged in $\sigma(\zm^{t'+2})$, thus we have $\sp(\sigma(\zm^{t'+2})) \subseteq \sp(\sigma(\vm^{t'+1})) \subseteq \sp(\sigma(\zm^{t'+1}))$.

     Therefore, $\sp(\sigma(\zm^{t+1})) \subseteq \sp(\sigma(\zm^{t}))$ holds for all $t \geq 1$. 

We already show that for all $t \geq 1$, $\sp(\sigma(\xm^{t})) = \sp(\sigma(\zm^{\bar{t}}))$ for some $\bar{t} \leq t$. And based on the update rule for $\xm$, we have $\xm^{t+1} = \zm^{t+1}$ or $\xm^{t+1} = \xm^t$. 
If $\xm^{t+1} = \zm^{t+1}$, $\sp(\sigma(\xm^{t+1})) = \sp(\sigma(\zm^{t+1})) \subseteq \sp(\sigma(\zm^{\bar{t}})) = \sp(\sigma(\xm^{t}))$ since $\bar{t} \leq t < t+1$.
If $\xm^{t+1} = \xm^t$, it's easy to see  $\sp(\sigma(\xm^{t+1})) = \sp(\sigma(\xm^{t}))$.
Therefore, $\sp(\sigma(\xm^{t+1})) \subseteq \sp(\sigma(\xm^{t}))$ holds for all $t \geq 1.$
     
\end{proof}


\begin{theorem}\label{thm:4appen}
      Suppose $s \leq \min \{\frac{2\lambda}{G^2}, \frac{1}{L}\}$. and $\xm^*$ is a limit point of $\{\xm^{t}\}_{t=0}^{\infty}$ generated by Algorithm~\ref{alg:3}, 
  then there exists $t_0 \geq 1$ such that for all $m \geq t_0$, we have
    \begin{equation}
        F(\xm^{m+1}) - F(\xm^*) \leq \frac{4}{(m+1)^2}W^{t_0},
    \end{equation}
    where $W^{t_0}$ is a value defined as
    \begin{equation}\begin{split}
        W^{t_0} =  \frac{1}{2s} \|(\alpha^{t_0-1}-1)\xm^{t_0-1} - \alpha^{t_0-1}\zm^{t_0} +\xm^*\|_F^2 
         + (\alpha^{t_0-1})^2(F(\xm^{t_0})-F(\xm^*)).
    \end{split}\end{equation}
\end{theorem}
\begin{proof}
 Based on Lemma~\ref{lem:5appen},  $\{\xm^t\}_{t=0}^\infty$ forms at most $K_1 \leq |S|+1$ subsequences with shrinking support $\{\xt^k\}_{k=1}^{K_1}$, and  $\{\zm^t\}_{t=0}^\infty$ forms at most $K_2 \leq |S|+1$ subsequences with shrinking support $\{\zt^k\}_{k=1}^{K_2}$. Based on Lemma~\ref{lem:3appen}, there exists $t_1 \geq 0$ such that $\{\xm^t\}_{t=t_1}^\infty \subseteq \xt^{K_1}$, and there exists $t_2 \geq 0$ such that $\{\zm^t\}_{t=t_2}^\infty \subseteq \zt^{K_2}$. Let all the elements of $\sigma(\xt^{K_1})$ have support $S_1$, let all the elements of $\sigma(\zt^{K_2})$ have support $S_2$, we show that $S_1 = S_2$: let $t_0 = \max \{t_1, t_2\}$, then there exists $t' \geq t_0$ such that $\xm^{t'} = \zm^{t'}$, and due to the fact that $\{\xm^t\}_{t=t_1}^\infty \subseteq \xt^{K_1}$ and $\{\zm^t\}_{t=t_2}^\infty \subseteq \zt^{K_2}$, we have $S_1 = \sp(\sigma(\xm^{t'})) = \sp(\sigma(\zm^{t'})) = S_2$. 

 Let $S_1 = S_2 = S^*$, then the singular value vectors of all the elements of  $\{\xm^t\}_{t=t_0}^\infty$ and $\{\zm^t\}_{t=t_0}^\infty$ have the same support $S^*$, and $\sp(\sigma(\xm^*)) = S^*$.

Following the same process in the proof for Theorem~\ref{thm:3appen}, we get
    \begin{equation}\label{eq:62}
        F(\zm^{t+1}) \leq F(\xm^t) + \frac{1}{s} \langle \zm^{t+1} - \vm^t, \xm^t - \vm^t \rangle - (\frac{1}{s} - \frac{L}{2}) \|\zm^{t+1} - \vm^t\|^2_F,
    \end{equation}
    and
    \begin{equation}\label{eq:63}
        F(\xm^{t+1}) \leq F(\xm^*) + \frac{1}{s} \langle \zm^{t+1} - \vm^t, \xm^* - \vm^t \rangle - (\frac{1}{s} - \frac{L}{2}) \|\zm^{t+1} - \vm^t\|^2_F,
    \end{equation}
    and  $(\alpha^t-1) \times $Eq.~(\ref{eq:62}) + Eq.~(\ref{eq:63}), multiply both sides by $\alpha^t$, and use the fact that $(\alpha^t)^2 - \alpha^t = (\alpha^{t-1})^2$, we have
    \begin{equation}\begin{split}
        &(\alpha^t)^2(F(\xm^{t+1}) - F(\xm^*)) - (\alpha^{t-1})^2(F(\xm^{t}) - F(\xm^*))  \\
        \leq  &
        \frac{1}{2s}(\|(\alpha^t-1)\xm^t - \alpha^t \vm^t + \xm^* \|^2_F - \|(\alpha^{t}-1)\xm^t - \alpha^t \zm^{t+1} + \xm^* \|^2_F).
    \end{split}\end{equation}
    Based on the update rule for $\vm^t$, we have 
    \begin{equation}\begin{split}\label{eq:67}
       & (\alpha^t)^2(F(\zm^{t+1}) - F(\xm^*)) - (\alpha^{t-1})^2(F(\xm^{t}) - F(\xm^*)) \\
        \leq  &
        \frac{1}{2s}(\|(\alpha^t-1)\xm^t - \alpha^t \um^t + \xm^* \|^2_F - \|(\alpha^{t}-1)\xm^t - \alpha^t \zm^{t+1} + \xm^* \|^2_F).
    \end{split}\end{equation}

    Define $\am^{t+1} = (\alpha^t-1)\xm^t - \alpha^t \zm^{t+1} + \xm^*, \am^{t} = (\alpha^{t-1}-1)\xm^{t-1} - \alpha^{t-1} \zm^t + \xm^*$, we can get $\am^t = (\alpha^t-1)\xm^t - \alpha^t \um^t + \xm^*$, therefore,
    \begin{equation}\label{eq:68}
        (\alpha^t)^2(F(\zm^{t+1}) - F(\xm^*)) - (\alpha^{t-1})^2(F(\xm^{t}) - F(\xm^*)) 
        \leq 
        \frac{1}{2s}(\|\am^t\|^2_F - \| \am^{t+1}\|^2_F).
    \end{equation}
    Sum Eq.~(\ref{eq:68}) over $t = t_0, \dots, m$ for $m \geq t_0$, we have
    \begin{equation}
         (\alpha^m)^2(F(\zm^{m+1}) - F(\xm^*)) - (\alpha^{t_0-1})^2(F(\xm^{t_0}) - F(\xm^*)) 
        \leq 
        \frac{1}{2s}(\|\am^{t_0}\|^2_F - \| \am^{m+1}\|^2_F)
        \leq 
        \frac{1}{2s}\|\am^{t_0}\|^2_F,
    \end{equation}
    therefore, with $\alpha^t \geq \frac{t+1}{2}$, we get
    \begin{equation}
        F(\zm^{m+1}) - F(\xm^*) 
        \leq 
        \frac{4}{(m+1)^2}(\frac{1}{2s} \|\am^{t_0}\|^2_F + (\alpha^{t_0-1})^2(F(\xm^{t_0}) - F(\xm^*))),
    \end{equation}
    where $\am^{t_0} = (\alpha^{t_0-1}-1)\xm^{t_0-1} - \alpha^{t_0-1}\zm^{t_0} +\xm^*$.
\end{proof}



\section{Experiment}

\subsection{Datasets and metrics}

% \noindent\textbf{Dataset.}
\subsubsection{Dataset}
% We adopt three datasets in our experiments, i.e., ClearGrasp \cite{sajjan2020clear}, TransCG \cite{fang2022transcg} and ClearPose \cite{chen2022clearpose}. The ClearGrasp dataset is the pioneering large-scale synthetic dataset that specifically focused on transparent objects. It provids a large-scale synthetic dataset as well as a real-world benchmark. The TransCG dataset comprises 57K RGB-D images from 130 different real-world scenes. 
% ClearPose dataset contains 350K RGB-D images of 63 household objects in real-world settings. Depth completion experiments and generalization verification (reported respectively in Section \ref{sec:depth} and \ref{sec:generalization}) are conducted on ClearGrasp, TransCG and ClearPose. Ablation study (reported in Section \ref{sec:ablation}) is performed on TransCG.
We use three datasets including ClearGrasp \cite{sajjan2020clear}, TransCG \cite{fang2022transcg}, and ClearPose \cite{chen2022clearpose}. The ClearGrasp dataset is a pioneering large-scale synthetic dataset that specifically focuses on transparent objects. It provides a large-scale synthetic dataset as well as a real-world benchmark. The TransCG dataset comprises 57K RGB-D images from 130 different real-world scenes. The ClearPose dataset contains 350K RGB-D images of 63 household objects in real-world settings. 
% We conducted depth completion experiments and generalization verification on ClearGrasp, TransCG, and ClearPose, reported respectively in Section \ref{sec:depth} and \ref{sec:generalization}. We performed an ablation study on TransCG, which is reported in Section \ref{sec:ablation}.

% ClearGrasp\cite{sajjan2020clear} is the first large-scale synthetic dataset as well as a real-world test benchmark focusing on transparent objects. TransCG\cite{fang2022transcg} is a large-scale real-world dataset, which contains 57K RGB-D images from 130 different scenes. ClearPose\cite{chen2022clearpose} is a recentily proposed real-world dataset, containing 350K RGB-D images covering 63 household objects.

% \newgeometry{letterpaper,top=60pt,bottom=43pt,left=48pt,right=48pt}
% \begin{table*}[!t]
% \caption{Ablation study. We show the impact of progressively substituting the components of the DFNet with ours. \label{tab:table1}
% }
% \centering
% \resizebox{\linewidth}{!}{%
% \begin{tabular}{cccccccccc}
% \toprule
% Model/Metric    & RMSE  & REL   & MAE   & $\delta$1.05 & $\delta$1.10 & $\delta$1.25          & Inference time (s)& Parameters & Size (MB)   \\ \midrule
% DFNet\cite{fang2022transcg}          & 0.018 & 0.027 & 0.012 & 83.76 & 95.67 & 99.71          & 0.0244s        & 1.25M & 4.819MB \\ \midrule
% New Loss        & 0.017 & 0.026 & 0.012 & 84.42 & 96.30 & \textbf{99.81} & 0.0244s        & 1.25M & 4.819MB \\ \midrule
% Shortcut Fusion & 0.017 & 0.024 & 0.011 & 86.18 & 96.67 & 99.79          & 0.0218s        & 1.02M & 3.919MB \\ \midrule
% Ours(slim) & 0.016          & 0.024          & 0.011          & 86.22          & 96.64          & \textbf{99.81} & \textbf{0.0143s} & \textbf{0.39M} & \textbf{1.518MB} \\ \midrule
% Ours       & \textbf{0.015} & \textbf{0.022} & \textbf{0.010} & \textbf{88.18} & \textbf{97.15} & \textbf{99.81} & 0.0153s          & 1.25M          & 4.803MB          \\
% \bottomrule
% \end{tabular}%
% }
% \end{table*}
\begin{table}[!t]
\renewcommand{\arraystretch}{1.05}
\setlength{\tabcolsep}{5pt}
\caption{Ablation study. We show the impact of progressively substituting the components of the DFNet with ours. \label{tab:table1}
}
\centering
\resizebox{\linewidth}{!}{%
\begin{threeparttable}
\begin{tabular}{cccccccccc}
\toprule
Model   & RMSE  & REL   & MAE   & $\delta$1.05 & $\delta$1.10 & $\delta$1.25          & Time(s)& Para(M) & Size (MB)   \\ \midrule
DFNet\cite{fang2022transcg}          & 0.018 & 0.027 & 0.012 & 83.76 & 95.67 & 99.71          & 0.0244        & 1.25 & 4.819 \\ \midrule
Huber Loss &0.017   &0.027  &0.012  &84.10  &95.82  &99.74 &0.0244  &1.25   &4.819  \\ \midrule
New Loss        & 0.017 & 0.026 & 0.012 & 84.42 & 96.30 & \textbf{99.81} & 0.0244        & 1.25 & 4.819 \\ \midrule
SF* & 0.017 & 0.024 & 0.011 & 86.18 & 96.67 & 99.79          & 0.0218        & 1.02 & 3.919 \\ \midrule
Ours(s)* & 0.016          & 0.024          & 0.011          & 86.22          & 96.64          & \textbf{99.81} & \textbf{0.0143} & \textbf{0.39} & \textbf{1.518} \\ \midrule
Ours       & \textbf{0.015} & \textbf{0.022} & \textbf{0.010} & \textbf{88.18} & \textbf{97.15} & \textbf{99.81} & 0.0153          & 1.25          & 4.803          \\
\bottomrule
\end{tabular}%
% \multicolumn{10}{l}{Note: NL* represents New Loss, SF* represents Shortcut Fusion and Ours(s)* represents Ours(slim).}
\begin{tablenotes}
\footnotesize
\item Note: SF* represents Shortcut Fusion and Ours(s)* represents Ours(slim).
\end{tablenotes}

\end{threeparttable}
}


\end{table}
% \vspace{-0.5cm}
\subsubsection{Metrics}
For evaluating the performance of our depth completion model, we employ four common metrics: RMSE, REL, MAE and Threshold $\delta$ (where $\delta$ is set to 1.05, 1.10, and 1.25). These metrics are calculated only on the transparent areas, as determined by transparent masks.
% Me use common metrics RMSE, REL, MAE and Threshold $\delta$ ($\delta$ is set to 1.05, 1.10 and 1.25) to evaluate our model. All metrics are calculated on the transparent areas according to transparent masks.


% We use three metrics to evaluate performance on pose estimation task. The average closest point distance (ADD-S)\cite{xiang2017posecnn} calculates the mean distance from each 3D model point to its closest neighbor on the target model. Followed DenseFusion\cite{wang2019densefusion} we report the area under the ADD-S curve (AUC) and the percentage of ADD-S smaller than 2cm ($<$2cm).

\subsection{Implementation Details}
% \noindent
% \textbf{Network configuration.}
\subsubsection{\bf Network Configuration}
% \textcolor{blue}{
In the network architecture, the number of hidden channels, \textbf{$C$}, is set to 64. Each FFEB/DFCB contains a single OSA module. Each OSA module is composed of 5 layers with stage channels of 20. The SFM module maintains \textbf{$C$} channels throughout the pipeline, while cross-layer shortcuts have only 1 channel. Residual connections between the encoder and decoder retain only \textbf{$C$} channels. The input head module and output head module use $3\times3$ convolution to adjust the number of channels and resolution (with resolution changes only occurring in the input head module). For the slim version, \textbf{$C$} is set to 32, and the OSA block contains 4 layers with stage channels of 16.
% }
% The hidden channels \textbf{$C$} in the network is set to 64. Each FFEB/DFCB contains one OSA module, in which, we use 5 layers per block and set stage channels \textbf{$C'$} to 20. SFM keeps \textbf{$C$} channels throughout the pipeline while cross-layer shortcuts take 1 channel only. Residual connections between encoder and decoder just keep channel \textbf{$C$}. $3\times3$ convolution is used in the input head module and the output head module to modify channels and resolution (resolution modified in the input head module only). For slim version, \textbf{$C$} is set to 32, \textbf{$C'$} is set to 16 and uses 4 layers per OSA block.

\subsubsection{\bf Training Details}
% \noindent
% \textbf{Training details.}
All experiments are carried out using the AdamW optimizer with an initial learning rate of $10^{-3}$. The learning rate is reduced by half after 5, 15, 25, and 35 epochs, and training continues for a total of 40 epochs with a batch size of 32. The threshold $\delta$ is kept constant at 0.1 during the training process. The weights $\alpha$ and $\beta$ for the loss function are set to 0.1 and 0.001, respectively. The images are resized to $320\times240$ for both training and testing. The experiments were conducted using an NVIDIA GeForce RTX 3090 GPU.
% We use AdamW optimizer with initial learning rate of $10^{-3}$ and multi-step learning rate scheduler which decays the learning rate by half after 5, 15, 25, 35 epochs. We train the model for 40 epochs with the batch size of 32. Threshold $\delta$ keeps 0.1 during training. Considering loss, we set $\alpha=0.1$, $\beta=0.001$. For all methods, we scale the images to $320\times240$ during training and testing. We use NVIDIA GeForce RTX 3090 for training and testing. 

 % Depth completion task and generalization ability are tested on ClearGrasp, TransCG and ClearPose. Pose estimation task is carried out on the set1 of ClearPose, since Clearpose has an accurate pose annotation without sticker. We use typical network DenseFusion\cite{wang2019densefusion} as pose estimation network. Following the learning strategy of DenseFusion, we train the network on 12G NVIDIA TITAN Xp GPU for 5 epochs with batch size of 128. The margin of refinement is set to 0.03. For fair comparison, we evaluate others works using their released source codes and optimal hyper-parameters or statistics reported in their paper.

\subsection{Ablation study} \label{sec:ablation}
We conduct an ablation study to investigate the effectiveness of our proposed components, including  new loss function, fusion branch, cross-layer shortcut and backbone structure. We take DFNet as baseline method since it is constructed following UNet structure. We  gradually replace its original components by our proposed ones and show the influence of using our proposed components. All the experiments of the ablation study are conducted on TransCG dataset.

% In view that DFNet is also constructed based on UNet, We here gradually replace its original components by our proposed. This study is conducted on TransCG dataset.
% To study the impact of each component in our proposed method, we perform experiments with different configurations of loss functions, network architecture, and backbones. Our method is compared against the recent transparent object depth completion work DFNet, which serves as our baseline. The ablation study experiments are all performed on the TransCG dataset.
% To verify the effectiveness of each component in our method, we evaluate the performance w.r.t. different configurations of loss functions, network architecture, and backbones. We use recently proposed transparent objects depth completion work DFNet as baseline. Ablation study is carried out on TransCG.




\subsubsection{\bf Loss Function}
The training of DFNet employs the mean squared error (MSE) and smooth loss as its loss function. However, these simple loss functions can lead to overfitting to local features, which makes the model more sensitive to the noise from low-level features such as edges and positions, negatively impacting its accuracy. To validate our proposed loss function, we first replaced the MSE loss with Huber loss in DFNet and termed it as Huber Loss. And then we replaced the loss function of DFNet with ours, leaving all other aspects unchanged and termed it as New Loss in Table \ref{tab:table1}. It can be observed by comparing New Loss with DFNet that all metrics showed improvement without requiring any additional parameters. 

% Qualitatively, the use of our proposed loss function can let the network to concentrate on the global structure rather than local details. By comparing the rows 3 and 4 of Figure \ref{fig:figure5}, the boundaries become smoother and even less distinct.
% The training of DFNET uses MSE and cosine distance. The simple loss function may lead to overfit to local features during training. This makes the model more sensitive to the noise of low-level features such as edge and position, which in turn affects its accuracy. So we propose a loss function consisting of Huber loss, SSIM loss and Smooth loss to suppress it. To verify its validity, we replaced the loss function of DFNet with ours and remain its other parts unchanged, then compared the results output by the mixed model (New Loss in Table \ref{tab:table1}) with the original one.
% All metrics are improved without extra parameters. Furthermore, we manually designed a feature to describe those pixels by computing the gradient of depth image and doing Gaussian blur to form an 'edge mask'. As their wights drop, the performance of the model is improved (Edge weight modified in Table \ref{tab:table2}), suggesting that it is necessary to treat pixels differently.
%and lower their weight during training. Specifically, we compute the gradient of depth image and do gaussian blur to form an 'edge mask'. Result (Edge weight modified in Table \ref{tab:table2}) supports our idea and shows it is necessary to treat pixels differently. 

\subsubsection{\bf Fusion Branch and Cross-layer Shortcuts}
In order to evaluate the impact of our proposed fusion branch and cross-layer shortcuts, we make changes to DFNet's architecture. First, we remove the redundant CDC blocks in DFNet from its skip connections, in line with our insight of preserving low-level features and the purpose of light weighting. Then, we added cross-layer shortcuts and a fusion branch to the modified network. It can be seen in Table \ref{tab:table1} that adopting this new architecture (referred to as Shortcut Fusion), almost all metrics show improvement with fewer parameters. 

\subsubsection{\bf Backbone}
We finally replace the denseblock in DFNet with our OSA module and utilized max pooling as the downsampling method. This final modification has transformed DFNet into our network. As shown in Table \ref{tab:table1}, our network outperforms the previous state-of-the-art (SOTA) by at least 16\% on difference-based metrics and improves ratio-based metrics by up to 4.42\%, resulting in a new SOTA performance. To make it practical for low-power robots, we created a slim version to balance speed and accuracy. 


% Qualitatively, figure \ref{fig:figure5} shows our method predicts clearer edges and is better handling crowded area.

% The fusion branch in our proposed network introduces a rich collection of low-level features, while the OSA module promotes feature reuse. Additionally, raw depth information is provided throughout the network, which enhances the representation of low-level features but may also hinder the learning of high-level semantic information. Our hypothesis is that the use of max pooling as a less aggressive downsampling method can mitigate these side effects while also reducing the number of parameters. The results in Table \ref{tab:table2} support our viewpoint.
% We fianlly relace the denseblock in DFNet by our used OSA module, and use max pooling as downsampling method. After this final modification, DFNet is tranformed to our proposed network. We thus show the performance by :Our"  in Table \ref{tab:table1}. It can be observed that ours outperforms previous SOTA by at least 16\% on difference-based metrics and improves ratio-based metrics by 0.1\% to 4.42\%, achieving the new state-of-the-art performance. In order to be capable in real applications, we also construct a slim version for speed/accuracy trade-off. 
% As we mentioned above, fusion branch introduces abundant low-level features and OSA encourages feature reuse. Furthermore, Raw depth is provided throughout the network. They enrich the representation of low-level features but may also harm to the learning of high-level semantic information. We suppose that using maxpooling to loosely downsampling may reduce their side effects as well as parameters saving. Result in Table \ref{tab:table2} proved our point of view.

% For summary, with our loss function, network tend to learn high-level features, with fusion branch, raw depth image and shortcuts, network can take advantage of low-level features. These components working together gives the network ability to take into account both local details and global structures. OSA module and max-pooling downsampling accelerate inference speed and reduce side effects.




% To intuitively show the impact of the proposed components, we visualize the predicted depth on TransCG and CleargGrasp dataset in Figure \ref{fig:figure5}. All networks are trained on TransCG dataset. Qualitatively, with our loss function, network is likely to focus on global structure rather than local detail. Red rectangle in row 3 and 4 show that with our loss function, boundaries become smoothy and even ambiguous, and outliers in the bottom right corner of the second column are suppressed. 



% FDCT performs domain adaption to the concatenation of raw depth and deep features and adopts maxpooling to lossly downsampling. It is supposed to reduce the disadvantage of the inaccuracy of raw depth. Our method predicts more accuracy and smooth edge as shown by the red circle on the left and the black square on the right. And even correct the ground truth as depicted in black circle on the right. The light spot reflected on the apple significantly affects the performance in row 2,3,5, but has little impact on row 4,6. Our methods successfully overcome the side effect of the raw depth information.

\subsection{Depth Completion Experiments} \label{sec:depth}

We compare our method with others on synthetic dataset ClearGrasp and real-world dataset TransCG. The quantitative results are respectively reported in Table \ref{tab:table2} and Table \ref{tab:table3}. Our proposed network surpasses others in almost every metric on these datasets which contain  synthetic and real-world scenes. Our method achieves a new state-of-the-art performance with a smaller model size and faster inference time, making it a highly competitive solution in this field.
%except on ClearGrasp synthetic validation set. It may be result of that the local implicit depth function which is environment-dependent, as well as the extra training data. 

% {\color{blue}
Specifically, our method outperforms the other methods by a larger margin in terms of REL and $\delta1.05$ metrics. This indicates its robustness to noise in the raw depth information, as these metrics are computed based on relative values and are sensitive to noise. Additionally, the gap between our method and others is larger in tests involving novel objects in ClearGrasp (CG Syn-novel in Table \ref{tab:table4} and the ClearGrasp column in Figure \ref{fig:figure5}), indicating that our method has a better ability to generalize to unseen objects. The qualitative results is reported in Figure \ref{fig:figure5}. The prediction of our method exhibits a clearer boundary and finer details than DFNet.
% }
% Specifically, our method has a bigger gap in REL and $\delta1.05$ to others most of the time. It demonstrates that our method is more stable to the noise in raw depth information of pixels, because these metrics are computed by relative value and significantly affected by noise. Noteworthy, the gap between our method and others getting bigger in the test of novel objects in most cases, indicates our method is able to generalize better to unseen objects.

\begin{table}[!t]
\caption{Depth Completion Result on TransCG dataset.}
\label{tab:table2}

\centering
\resizebox{\linewidth}{!}{%
\begin{tabular}{ccccccccc}
\toprule
Model & RMSE  & REL   & MAE   & $\delta1.05$ & $\delta1.10$ & $\delta1.25$ & Time ($\second$)   & Size ($\mega$B)    \\ \midrule
ClearGrasp\cite{sajjan2020clear}   & 0.054 & 0.083 & 0.037 & 50.48 & 68.68 & 95.28 & 2.281          & 934          \\
LIDF-Refine\cite{zhou2021pr}  & 0.019 & 0.034 & 0.015 & 78.22 & 94.26 & 99.80 & 0.018          & 251          \\
DFNet\cite{fang2022transcg}        & 0.018 & 0.027 & 0.012 & 83.76 & 95.67 & 99.71 & 0.024          & 4.8          \\
Ours (slim)   & 0.017 & 0.025 & 0.011 & 85.53 & 96.46 & 99.79 & \textbf{0.014} & \textbf{1.6} \\
Ours & \textbf{0.015} & \textbf{0.022} & \textbf{0.010} & \textbf{88.18} & \textbf{97.15} & \textbf{99.81} & 0.015 & 4.8 \\ \bottomrule
\end{tabular}}
% \vspace{-0.5cm}
\end{table}


\begin{table}[!t]
\renewcommand{\arraystretch}{0.9}
\caption{Depth Completion Results on ClearGrasp dataset\label{tab:table3}}
\centering
\resizebox{\linewidth}{!}{%
\begin{tabular}{ccccccc}
\toprule
\multicolumn{1}{c}{Model/Metric} &
  \multicolumn{1}{c}{RMSE} &
  \multicolumn{1}{c}{REL} &
  \multicolumn{1}{c}{MAE} &
  \multicolumn{1}{c}{$\delta$1.05} &
  \multicolumn{1}{c}{$\delta$1.10} &
  $\delta$1.25 \\ \midrule
\multicolumn{7}{c}{Train CG Test CG Syn-novel} \\ \midrule
\multicolumn{1}{c}{ClearGrasp} &
  \multicolumn{1}{c}{0.040} &
  \multicolumn{1}{c}{0.071} &
  \multicolumn{1}{c}{0.035} &
  \multicolumn{1}{c}{42.95} &
  \multicolumn{1}{c}{80.04} &
  98.10 \\ 
\multicolumn{1}{c}{Local Implicit} &
  \multicolumn{1}{c}{\underline{0.028}} &
  \multicolumn{1}{c}{\underline{0.045}} &
  \multicolumn{1}{c}{\underline{0.023}} &
  \multicolumn{1}{c}{\underline{68.62}} &
  \multicolumn{1}{c}{\underline{89.10}} &
  \underline{99.20} \\ 
\multicolumn{1}{c}{DFNet} &
  \multicolumn{1}{c}{0.032} &
  \multicolumn{1}{c}{0.051} &
  \multicolumn{1}{c}{0.027} &
  \multicolumn{1}{c}{62.59} &
  \multicolumn{1}{c}{84.37} &
  98.39 \\ 
\multicolumn{1}{c}{FDCT (Ours)} &
  \multicolumn{1}{c}{\textbf{0.025}} &
  \multicolumn{1}{c}{\textbf{0.040}} &
  \multicolumn{1}{c}{\textbf{0.021}} &
  \multicolumn{1}{c}{\textbf{71.66}} &
  \multicolumn{1}{c}{\textbf{92.95}} &
  \textbf{99.64} \\ \midrule
\multicolumn{7}{c}{Train CG Test CG Syn-known} \\ \midrule
\multicolumn{1}{c}{Local Implicit} &
  \multicolumn{1}{c}{\textbf{0.012}} &
  \multicolumn{1}{c}{\textbf{0.017}} &
  \multicolumn{1}{c}{\textbf{0.009}} &
  \multicolumn{1}{c}{\textbf{94.79}} &
  \multicolumn{1}{c}{\textbf{98.52}} &
  99.67 \\ 
\multicolumn{1}{c}{ClearGrasp} &
  \multicolumn{1}{c}{0.044} &
  \multicolumn{1}{c}{0.047} &
  \multicolumn{1}{c}{0.033} &
  \multicolumn{1}{c}{71.23} &
  \multicolumn{1}{c}{92.60} &
  98.24 \\ 
\multicolumn{1}{c}{DFNet} &
  \multicolumn{1}{c}{0.018} &
  \multicolumn{1}{c}{0.023} &
  \multicolumn{1}{c}{0.013} &
  \multicolumn{1}{c}{88.85} &
  \multicolumn{1}{c}{97.57} &
  \underline{99.92} \\ 
\multicolumn{1}{c}{FDCT (Ours)} &
  \multicolumn{1}{c}{\underline{0.015}} &
  \multicolumn{1}{c}{\underline{0.020}} &
  \multicolumn{1}{c}{\underline{0.012}} &
  \multicolumn{1}{c}{\underline{90.53}} &
  \multicolumn{1}{c}{\underline{98.21}} &
  \textbf{99.99} \\ \bottomrule

\end{tabular}%
% \tablen}
}
\end{table}



\subsection{Generalization Experiment} \label{sec:generalization}
% The generalization capability of a network is essential for practical applications. We evaluated the generalization ability of our proposed method from two perspectives: from synthetic images to real-world images and from one real-world dataset to another. The results of our experiments, shown in Table \ref{tab:table6}, indicate that our method (FDCT) has a comparable generalization capability to the state-of-the-art methods in cross-dataset evaluations, and it outperforms similar works in the synthetic-to-real test. However, it lags behind methods that focus solely on sim-to-real (noted as "local implicit*").
% The generalization ability of a network is critical for real-world application. The proposed method has a generalization ability that can be trained on synthetic data and aply to real world scene (syn-to-real) or trained on one real world dataset TransCG and adap to ClearGrasp (real-to-real). Comparison result is reported in Table \ref{tab:table4}. It shows that although there is still a certain gap compared with the method Local Implicit designed for syn-to-real; compared with the similar method DFNet, our method achieves a better result in the syn-to-real setting, and a competitive result in the syn-to-syn setting.
The generalization ability of a network is critical for real-world application. Our proposed method exhibits a high degree of generalization, being able to be trained on synthetic data and applied to real-world scenes (syn-to-real), or trained on one real-world dataset TransCG and adapted to the other real-world dataset (real-to-real), such as ClearGrasp. Comparison results are reported in Table \ref{tab:table4}, which show that while there is still a certain gap compared to the syn-to-real method (Local Implicit \cite{zhu2021rgb}), our method achieves better results in the syn-to-real setting when compared to the similar method DFNet, and competitive results in the real-to-real setting.

% We inspect the generalization ability of our proposed method from two aspects, from synthetic image to real-world image and from one real-world dataset to another. Experiment results in Table \ref{tab:table5} show that FDCT has a similar generalization ability to previous SOTA in cross-dataset and get better result in synthetic-to-real test compared to similar work, but is far below to methods focusing on sim-to-real.

% Since both datasets comprise real-world image, we train models on TransCG and test it on ClearGrasp real-world set for cross-dataset test. DFNet outperformed other method with a huge gap in generalization test and is chosen to be compared with ours. Comparison result is reported in Table \ref{tab:table5}. Our method outperforms the closest work in all metrics both for known and novel objects in synthetic-to-real test. There is a bigger gap between DFNet and ours in terms of novel objects. It might owe to a better utilization of RGB cues. Our method gets similar results to DFNet in cross dataset test, showing that our method has the ability to generalize from real-world dataset to another. With a series of real-world transparent objects datasets being proposed, we believe that the generalization ability in real-world is more important than sim-to-real.



% {\color{blue}
% Figure environment removed

\begin{table}[!t]
\caption{
% Result of Synthetic to Real and Cross Dataset Generalization Experiment
Generalization test on syn-to-real and real-to-real.}
\label{tab:table4}
\renewcommand{\arraystretch}{0.95}
\centering
\resizebox{\linewidth}{!}{%
% \begin{threeparttable}
\begin{tabular}{ccclclclclcl}
\toprule
\multicolumn{1}{c}{Model/Metric} &
  \multicolumn{1}{c}{RMSE} &
  \multicolumn{2}{c}{REL} &
  \multicolumn{2}{c}{MAE} &
  \multicolumn{2}{c}{$\delta$1.05} &
  \multicolumn{2}{c}{$\delta$1.10} &
  \multicolumn{2}{c}{$\delta$1.25} \\ \midrule
\multicolumn{12}{c}{Train CG Test CG Real-known (syn-to-real)} \\ \midrule
\multicolumn{1}{c}{Local Implicit\cite{zhu2021rgb}} &
  \multicolumn{1}{c}{\textbf{0.028}} &
  \multicolumn{2}{c}{\textbf{0.033}} &
  \multicolumn{2}{c}{\textbf{0.020}} &
  \multicolumn{2}{c}{\textbf{82.37}} &
  \multicolumn{2}{c}{\textbf{92.98}} &
  \multicolumn{2}{c}{\textbf{98.63}} \\ 
\multicolumn{1}{c}{DFNet} &
  \multicolumn{1}{c}{0.068} &
  \multicolumn{2}{c}{0.107} &
  \multicolumn{2}{c}{0.059} &
  \multicolumn{2}{c}{32.42} &
  \multicolumn{2}{c}{56.88} &
  \multicolumn{2}{c}{91.47} \\ 
\multicolumn{1}{c}{FDCT (Ours)} &
  \multicolumn{1}{c}{\underline{0.065}} &
  \multicolumn{2}{c}{\underline{0.103}} &
  \multicolumn{2}{c}{\underline{0.057}} &
  \multicolumn{2}{c}{\underline{33.08}} &
  \multicolumn{2}{c}{\underline{59.81}} &
  \multicolumn{2}{c}{\underline{91.70}} \\ \midrule
\multicolumn{12}{c}{Train CG Test CG Real-novel (syn-to-real)} \\ \midrule
\multicolumn{1}{c}{Local Implicit\cite{zhu2021rgb}} &
  \multicolumn{1}{c}{\textbf{0.025}} &
  \multicolumn{2}{c}{\textbf{0.036}} &
  \multicolumn{2}{c}{\textbf{0.020}} &
  \multicolumn{2}{c}{\textbf{76.21}} &
  \multicolumn{2}{c}{\textbf{94.01}} &
  \multicolumn{2}{c}{\textbf{99.35}} \\ 
\multicolumn{1}{c}{DFNet} &
  \multicolumn{1}{c}{0.051} &
  \multicolumn{2}{c}{0.088} &
  \multicolumn{2}{c}{0.046} &
  \multicolumn{2}{c}{31.23} &
  \multicolumn{2}{c}{64.66} &
  \multicolumn{2}{c}{97.77} \\ 
\multicolumn{1}{c}{FDCT (Ours)} &
  \multicolumn{1}{c}{\underline{0.043}} &
  \multicolumn{2}{c}{\underline{0.073}} &
  \multicolumn{2}{c}{\underline{0.038}} &
  \multicolumn{2}{c}{\underline{39.42}} &
  \multicolumn{2}{c}{\underline{75.54}} &
  \multicolumn{2}{c}{\underline{99.09}} \\ \midrule
\multicolumn{12}{c}{Train TCG Test CG Real-novel (real-to-real)} \\ \midrule
\multicolumn{1}{c}{Local Implicit\cite{zhu2021rgb}} &
  \multicolumn{1}{c}{0.152} &
  \multicolumn{2}{c}{0.225} &
  \multicolumn{2}{c}{0.139} &
  \multicolumn{2}{c}{9.86} &
  \multicolumn{2}{c}{20.63} &
  \multicolumn{2}{c}{46.02} \\ 
\multicolumn{1}{c}{DFNet} &
  \multicolumn{1}{c}{\textbf{0.041}} &
  \multicolumn{2}{c}{\textbf{0.054}} &
  \multicolumn{2}{c}{\textbf{0.031}} &
  \multicolumn{2}{c}{\textbf{62.74}} &
  \multicolumn{2}{c}{\textbf{83.31}} &
  \multicolumn{2}{c}{\textbf{97.33}} \\ 
\multicolumn{1}{c}{FDCT (Ours)} &
  \multicolumn{1}{c}{\textbf{0.041}} &
  \multicolumn{2}{c}{\underline{0.055}} &
  \multicolumn{2}{c}{\underline{0.032}} &
  \multicolumn{2}{c}{\underline{61.23}} &
  \multicolumn{2}{c}{\underline{82.84}} &
  \multicolumn{2}{c}{\underline{97.28}} \\ \bottomrule
\end{tabular}
%     \begin{tablenote}
%         \footnotesize
%         \item [*]Local Implicit is method aiming at sim-to-real.
%     \end{tablenote}
% \end{threeparttable}
}
%\vspace{-0.5cm}
\end{table}
% Figure environment removed
\subsection{Analysis} \label{sec:analysis}
In our proposed method, the loss function plays a crucial role in enabling the network to focus on structural information and alleviate the effects of unstable pixels. However, this focus on structural information may come at the expense of some details. On the other hand, the fusion branch and shortcuts draw attention to the details, which can introduce extra redundancy. Nonetheless, the use of maxpooling facilitates lossy and aggressive downsampling, which can reduce redundancy and improve robustness. The convolution based fusion method make better use of the raw depth image. All components work together and complement each other to achieve the best possible balance between structural information and details. In this section, we analyze the four critical components of our method and demonstrate their effectiveness.

\subsubsection{Influence of loss term}
% As we mentioned above, some unstable pixels can unwantedly make big penalty to the loss. By computing the gradient of the depth image and applying Gaussian blur, we manually created a feature to represent these pixels. As the weights of these pixels were reduced, the model's performance improved (as seen in Experiment of weight in Table \ref{tab:table5}), indicating the importance of treating pixels differently and pointing out the necessity of the so designed loss function. However, the side effect of such loss function is that the network pays too much attention to the structure and ignores some details. The highlighted area of the feature map changes from dotted to regional in the Loss column in Figure \ref{fig:figure6}.
As mentioned in \ref{section:Loss}, unstable pixels can have a significant negative influence on the calculation of the training loss. To illustrate this issue, we manually created a feature to represent these pixels by computing the gradient of the depth image and applying a Gaussian blur. By reducing the weights of these pixels, we observed an improvement in the model's performance (as seen in the Experiment of weight in Table \ref{tab:table5}), highlighting the importance of treating pixels differently and emphasizing the necessity of the used loss functions (especially the Huber Loss). Qualitatively, as shown in Figure \ref{fig:figure6}, the New Loss model places greater emphasis on the overall structure of transparent objects, as compared to DFNet, which primarily focuses on local information. The downside of such a loss function is that the network may ignore some details.
% Figure environment removed

\subsubsection{Low-level feature preservation}
% Fusion branch and cross-layer shortcuts alleviate the indistinct boundaries and perceptual details by taking more low-level cues into consideration. The highlighted area of the feature map changes from regional to scattered in the Fusion column in Figure \ref{fig:figure6}. Loss function and low-level feature awareness components together make a good trade-off between detail and structure information.
The fusion branch and cross-layer shortcuts help alleviate the issue of blurry boundaries and low perceptual details by incorporating more low-level cues. As a result, more low-level features such as object edges and holes are preserved in the feature map of Fusion model in Figure \ref{fig:figure6}. The combination of the loss function and low-level feature awareness components strikes a good balance between detail and structural information.

\subsubsection{Influence of downsampling}
Our hypothesis is that the use of max pooling as a lossy downsampling method can mitigate the side effects of the low-level awareness components while reducing the number of parameters. The results in Table \ref{tab:table5} that are noted as ``Experiment of downsampling'' support our viewpoint. It can be observed that the performance of using convolutional downsampling and average pooling is slightly worse than that of using max pooling.

% The loss function makes the network focus on structural information and alleviating the affects of unstable pixels, but may harming to the details. The fusion branch and shortcuts draws the attention to details, but may introduce extra redundancy. Maxpooling is used to lossy and aggressively downsampling. It can reduce redundancy and improve robustness. These components work together and complement each other.
% }

\subsubsection{Fusion method of depth image}
To demonstrate that fusing the raw depth image with feature map via convolution is better than directly concatenation. We removed the convolution layers used for fusion in the model Ours and named it Ours(concat). The result labeled Table ``Experiment on fusion method'' in Table \ref{tab:table5} support our viewpoint.

\begin{table}[!ht]
\centering
\caption{Experiment Result on Weight Modification, Downsampling Implementation and Fusion Method\label{tab:table5}}

\resizebox{\linewidth}{!}{%
\begin{tabular}{ccccccc}
\toprule
\multicolumn{1}{c}{Model/Metric} &
  \multicolumn{1}{c}{RMSE} &
  \multicolumn{1}{c}{REL} &
  \multicolumn{1}{c}{MAE} &
  \multicolumn{1}{c}{$\delta$1.05} &
  \multicolumn{1}{c}{$\delta$1.10} &
  $\delta$1.25 \\ \midrule
\multicolumn{7}{c}{Experiment on weight} \\ \midrule
\multicolumn{1}{c}{Baseline} &
  \multicolumn{1}{c}{0.018} &
  \multicolumn{1}{c}{0.027} &
  \multicolumn{1}{c}{0.012} &
  \multicolumn{1}{c}{83.76} &
  \multicolumn{1}{c}{95.67} &
  99.71 \\ 
\multicolumn{1}{c}{Edge Weight Modified} &
  \multicolumn{1}{c}{\textbf{0.017}} &
  \multicolumn{1}{c}{\textbf{0.025}} &
  \multicolumn{1}{c}{\textbf{0.011}} &
  \multicolumn{1}{c}{\textbf{85.34}} &
  \multicolumn{1}{c}{\textbf{96.26}} &
  \textbf{99.75} \\ \midrule
\multicolumn{7}{c}{Experiment on downsampling} \\ \midrule
\multicolumn{1}{c}{Conv Down} &
  \multicolumn{1}{c}{0.016} &
  \multicolumn{1}{c}{0.023} &
  \multicolumn{1}{c}{0.011} &
  \multicolumn{1}{c}{87.16} &
  \multicolumn{1}{c}{96.83} &
  99.80 \\ 
\multicolumn{1}{c}{AvgPooling Down} &
  \multicolumn{1}{c}{0.016} &
  \multicolumn{1}{c}{0.024} &
  \multicolumn{1}{c}{0.011} &
  \multicolumn{1}{c}{87.16} &
  \multicolumn{1}{c}{96.93} &
  99.80 \\ 
\multicolumn{1}{c}{MaxPooling Down} &
  \multicolumn{1}{c}{\textbf{0.015}} &
  \multicolumn{1}{c}{\textbf{0.022}} &
  \multicolumn{1}{c}{\textbf{0.010}} &
  \multicolumn{1}{c}{\textbf{88.18}} &
  \multicolumn{1}{c}{\textbf{97.15}} &
  \textbf{99.81} \\ \midrule
  \multicolumn{7}{c}{Experiment on fusion method} \\ \midrule
  \multicolumn{1}{c}{Ours(concat)} &
  \multicolumn{1}{c}{\textbf{0.015}} &
  \multicolumn{1}{c}{0.023} &
  \multicolumn{1}{c}{0.011} &
  \multicolumn{1}{c}{87.90} &
  \multicolumn{1}{c}{96.68} &
  99.80 \\ 
\multicolumn{1}{c}{Ours} &
  \multicolumn{1}{c}{\textbf{0.015}} &
  \multicolumn{1}{c}{\textbf{0.022}} &
  \multicolumn{1}{c}{\textbf{0.010}} &
  \multicolumn{1}{c}{\textbf{88.18}} &
  \multicolumn{1}{c}{\textbf{97.15}} &
  \textbf{99.81} \\ 
\bottomrule
\end{tabular}%
}
%\vspace{-0.5cm}
\end{table}
\vspace{-0.2cm}


\subsection{Pose Estimation Experiment}
In this experiment, we aim to demonstrate the applicability of our network for downstream tasks and to show that it can improve the accuracy of pose estimate.
To evaluate the performance of pose estimation, we use three evaluation metrics, i.e, the average closest point distance (ADD-S), the area under the ADD-S curve (AUC), and the percentage of ADD-S values that are smaller than 2 \centi\meter.
%\cite{xiang2017posecnn}
% The higher the metrics the stronger the performance.

% This experiment is carried out on the set1 of ClearPose, since Clearpose has an accurate pose annotation without sticker. We use typical network DenseFusion \cite{wang2019densefusion} as pose estimation network. Following the learning strategy of DenseFusion, we train the network on 12G NVIDIA TITAN Xp GPU for 5 epochs with batch size of 128. The margin of refinement is set to 0.03. For fair comparison, we evaluate others works using their released source codes and optimal hyper-parameters or statistics reported in their paper.
Both our method and DFNet are trained on the ClearPose Set 1 and are used to predict the depth of Set 1-Scene 5 for pose estimation purposes. The depth completion result is reported in Table \ref{tab:table6} and a screenshot of the live demonstration is reported in Figure \ref{fig:figure7}. In our experiments, we use DenseFusion \cite{wang2019densefusion}  as the pose estimation method. We trained DenseFusion with the restored depth and tested it on 3,000 randomly selected images. Ideally, a more accurate depth prediction can lead to improved performance in pose estimation. The results of our evaluations, presented in Table \ref{tab:table7}, indicate that the depth restored by our method outperforms DFNet in almost every object in the pose estimation task. This results validate that the depth map given by our method is more appropriate for addressing the downstream task, i.e., pose estimation.
% Depth completion models are trained on ClearPose set 1 and predict the depth of set 1-scene 5 for pose estimation. We train DenseFusion with the restored depth and test on 3k randomly chosen images. Metrics for each object are reported in Table \ref{tab:table7}. Result shows that the depth restored by FDCT outperforms DFNet's in almost every object in pose estimation task.
% \todo{format of tablehead!!}
\begin{table}[!t]
\caption{Depth Completion Results on ClearPose dataset.}
\label{tab:table6}
\centering
\begin{tabular}{ccccccc}
\toprule
Model & RMSE           & REL            & MAE            & $\delta$1.05          & $\delta$1.10          & $\delta$1.25          \\ \midrule
DFNet        & 0.048          & 0.038          & 0.033          & 76.36          & 94.22          & \textbf{99.40} \\
Ours         & \textbf{0.045} & \textbf{0.033} & \textbf{0.028} & \textbf{82.15} & \textbf{94.43} & 99.25          \\
\bottomrule
\end{tabular}%
\end{table}



\begin{table}[!t]
\caption{Pose Estimation Results on ClearPose dataset\label{tab:table7}}
\centering
\resizebox{\linewidth}{!}{%
\begin{tabular}{ccccccc}
\toprule
Models &
  \multicolumn{3}{c}{DFNet} &
  \multicolumn{3}{c}{Ours} \\ \midrule
Object/Metirc &
  \multicolumn{1}{c}{AUC} &
  \multicolumn{1}{c}{\textless{}2cm} &
  ADD-S(10\%) &
  \multicolumn{1}{c}{AUC} &
  \multicolumn{1}{c}{\textless{}2cm} &
  ADD-S(10\%) \\ 
beaker\_1 &
  \multicolumn{1}{c}{79.07} &
  \multicolumn{1}{c}{\textbf{0.00}} &
  0.68 &
  \multicolumn{1}{c}{\textbf{80.44}} &
  \multicolumn{1}{c}{\textbf{0.00}} &
  \textbf{7.53} \\ 
dropper\_1 &
  \multicolumn{1}{c}{\textbf{67.76}} &
  \multicolumn{1}{c}{61.00} &
  \textbf{48.00} &
  \multicolumn{1}{c}{31.70} &
  \multicolumn{1}{c}{\textbf{65.33}} &
  0.00 \\ 
dropper\_2 &
  \multicolumn{1}{c}{81.09} &
  \multicolumn{1}{c}{\textbf{33.10}} &
  1.78 &
  \multicolumn{1}{c}{\textbf{84.24}} &
  \multicolumn{1}{c}{0.00} &
  \textbf{9.61} \\ 
flask\_1 &
  \multicolumn{1}{c}{84.96} &
  \multicolumn{1}{c}{60.33} &
  42.33 &
  \multicolumn{1}{c}{\textbf{86.71}} &
  \multicolumn{1}{c}{\textbf{68.33}} &
  \textbf{68.00} \\ 
funnel\_1 &
  \multicolumn{1}{c}{78.85} &
  \multicolumn{1}{c}{91.33} &
  0.00 &
  \multicolumn{1}{c}{\textbf{82.91}} &
  \multicolumn{1}{c}{\textbf{98.33}} &
  \textbf{12.33} \\ 
cylinder\_1 &
  \multicolumn{1}{c}{78.77} &
  \multicolumn{1}{c}{48.33} &
  28.67 &
  \multicolumn{1}{c}{\textbf{79.83}} &
  \multicolumn{1}{c}{\textbf{77.00}} &
  \textbf{33.33} \\ 
cylinder\_2 &
  \multicolumn{1}{c}{62.75} &
  \multicolumn{1}{c}{54.67} &
  3.33 &
  \multicolumn{1}{c}{\textbf{75.68}} &
  \multicolumn{1}{c}{\textbf{58.67}} &
  \textbf{29.33} \\ 
pan\_1 &
  \multicolumn{1}{c}{86.76} &
  \multicolumn{1}{c}{13.67} &
  33.33 &
  \multicolumn{1}{c}{\textbf{89.37}} &
  \multicolumn{1}{c}{\textbf{53.67}} &
  \textbf{50.00} \\ 
pan\_2 &
  \multicolumn{1}{c}{88.71} &
  \multicolumn{1}{c}{84.67} &
  44.00 &
  \multicolumn{1}{c}{\textbf{89.73}} &
  \multicolumn{1}{c}{\textbf{90.33}} &
  \textbf{56.00} \\ 
pan\_3 &
  \multicolumn{1}{c}{\textbf{88.90}} &
  \multicolumn{1}{c}{87.67} &
  \textbf{53.33} &
  \multicolumn{1}{c}{88.10} &
  \multicolumn{1}{c}{\textbf{91.00}} &
  48.00 \\ 
bottle\_1 &
  \multicolumn{1}{c}{86.05} &
  \multicolumn{1}{c}{91.53} &
  24.41 &
  \multicolumn{1}{c}{\textbf{88.71}} &
  \multicolumn{1}{c}{\textbf{93.22}} &
  \textbf{31.53} \\ 
bottle\_2 &
  \multicolumn{1}{c}{71.81} &
  \multicolumn{1}{c}{83.16} &
  4.04 &
  \multicolumn{1}{c}{\textbf{77.01}} &
  \multicolumn{1}{c}{\textbf{88.22}} &
  \textbf{13.47} \\ 
stick\_1 &
  \multicolumn{1}{c}{69.53} &
  \multicolumn{1}{c}{32.32} &
  32.66 &
  \multicolumn{1}{c}{\textbf{79.60}} &
  \multicolumn{1}{c}{\textbf{57.58}} &
  \textbf{58.92} \\ 
syringe\_1 &
  \multicolumn{1}{c}{73.03} &
  \multicolumn{1}{c}{31.67} &
  25.67 &
  \multicolumn{1}{c}{\textbf{80.15}} &
  \multicolumn{1}{c}{\textbf{57.00}} &
  \textbf{47.00} \\ 
MEAN &
  \multicolumn{1}{c}{78.43} &
  \multicolumn{1}{c}{55.25} &
  24.45 &
  \multicolumn{1}{c}{\textbf{79.58}} &
  \multicolumn{1}{c}{\textbf{64.19}} &
  \textbf{33.22} \\


  \bottomrule
  \end{tabular}%
}
\vspace{-0.5cm}
\end{table}
%\clearpage

%\input{changes-due-to-ijcai}

%\newpage
%\verbatiminput{aamas-reviews.txt}
\end{document}
