\section{\texorpdfstring{Proof of \Cref{thm:stability}}{Proof of Theorem 3.1}}
\label{sec:proof_stability}

% To begin the proof of \cref{thm:stability}, we first provide the maximum principle for operator $L$ which has the following form
% \[Lu=\frac{\partial u}{\partial t}-\sigma^2\Delta u-v(\vx,t)\nabla u\]

% \begin{theorem}[Theorem 6.10 in \cite{protter2012maximum}]
% 	\label{thm:maximum}
% 	Suppose $u\in C^2(R^n\times(0,T])\cap C(R^n\times[0,T])$ satisfies $Lu\leq 0$ in $R^n\times(0,T]$, where $L$ is a uniformly parabolic operator with bounded coefficients. If for any $\vx \in \mathbb{R}^n, 0 \leq t \leq T$, $u$ satisfies the growth condition
% 	\[u(\vx, t) \leq e^{c|\vx|^2}\]
% 	for some positive constant $c$, then
%     \[\max_{R^n\times[0,T]} u(\vx, t)=\max_{R^n} u(\vx,0)\]
% \end{theorem}

% Obviously the isotropic model which contains Laplacian is uniformly parabolic. Following the techniques in \cite{wang2020enresnet}, we provide the proof of \cref{thm:stability}.

% \begin{proof}
% First of all, we consider the binary classification case, where $u(\vx,0)=\mathrm{Sigmoid}(\vw_{\text{fc}}^\top x), \vw_{\text{fc}}\in \mathbb{R}^{d}$. Let $w(\vx,t)=(\mu u^2(\vx,t)+\|\nabla u(\vx,t)\|^2)e^{-2\lambda t}$, where $\mu$ and $\lambda$ are constants defined later. Note that $u^2$ satisfies
% \[\frac{\partial(u^2)}{\partial t}-v\cdot\nabla(u^2)=\sigma^2\Delta(u^2)-2\sigma^2\|\nabla u\|^2\]
% and $\|\nabla u\|^2$ satisfies
% \[\frac{\partial\|\nabla u\|^2}{\partial t}-v\cdot\nabla\|\nabla u\|^2=2\nabla u\cdot \nabla v\cdot\nabla u+\sigma^2\Delta\|\nabla u\|^2-2\sigma^2\|H\|^2_F\]
% where $H=\left(\frac{\partial^2 u}{\partial x_i\partial x_j}\right)\in \mathbb{R}^{d\times d}$ is the Hessian matrix. Therefore,
% \begin{eqnarray*}
%     Lw&=&\frac{\partial w}{\partial t}-v\cdot \nabla w-\sigma^2\Delta w\\
%     &=&e^{-2\lambda t}[-2\lambda(\mu u^2+\|\nabla u\|^2)-2\mu\sigma^2\|\nabla u\|^2+2\nabla u\nabla v\nabla u- 2\sigma^2\|H\|^2_F]
% \end{eqnarray*}
% Let
% \[\gamma(\vx,t)=\max_{\|\xi\|=1}\xi^T\nabla v(\vx,t)\xi\]
% and $\gamma=\max_{\vx,t}\gamma(\vx,t)$. The maximum exists because we assume $v$ to be Lipschitz continuous. Then
% \[Lw\leq -2e^{-2\lambda t}[\lambda\mu u^2+(\lambda+\mu\sigma^2-\gamma)\|\nabla u\|^2]\]
% If $\lambda\geq 0$, then $w$ satisfy the growth condition in \cref{thm:maximum}. If we choose $\lambda$ and $\mu$ large enough, such that $\lambda\mu\geq 0$, $\lambda+\mu\sigma^2-\gamma\geq 0$, then $Lw\leq 0$. From \cref{thm:maximum}, we have $\max_{\vx}w(\vx,T)\leq\max_{\vx}w(\vx,0)$, namely,
% \begin{equation}
% 	\label{eq:a2_core}
% 	e^{-2\lambda T}\max_{\vx}(\mu u^2(\vx,T)+\|\nabla u(\vx,T)\|^2)\leq\max_{\vx}(\mu u^2(\vx,0)+\|\nabla u(\vx,0)\|^2)
% \end{equation}

% thus
% \[\max_{\vx}\|\nabla u(\vx,T)\|^2\leq e^{2\lambda T}\max_{\vx}(\mu u^2(\vx,0)+\|\nabla u(\vx,0)\|^2)\]
% Let $\lambda=0$, $\mu = \frac{\gamma}{\sigma^2}$. Since $u(\vx,0)=\mathrm{Sigmoid}(\vw_{\text{fc}}^\top x)$, we have
% \begin{eqnarray*}
%     \max_{\vx}\|\nabla u(\vx,T)\| & \leq & \max_{\vx}(\sqrt{\gamma/\sigma^2}u(\vx,0)+\|\nabla u(\vx,0)\|)\\
%     & \leq & \sqrt{\gamma/\sigma^2}+\|\vw_{\text{fc}}\|_2
% \end{eqnarray*}

% First of all, we consider the binary classification case, where $u(\vx,0)=f(\vx)=\mathrm{Sigmoid}(\vw_{\text{fc}}^\top x), \vw_{\text{fc}}\in \mathbb{R}^{d}$. 

\begin{proof}
First of all, consider the transport equation
\[\frac{\partial u(\boldsymbol{x},t)}{\partial t}= v(\boldsymbol{x},t) \cdot \nabla u(\boldsymbol{x},t), \quad \vx \in \mathbb R^{d}, \quad  t\in [0,T-1]\]
Using the method of characteristics, $u(\vx,t)$ is constant along the characteristics curve $(X(s),s)$ satisfying the following ordinary differential equation,
\begin{equation}
\label{eq:a2_characteristics}
    \begin{cases}
    X'(s) = -v(X(s), s)\\
    X(t) = \vx
\end{cases}
\end{equation}
Since $v(\vx,t)$ is Lipschitz continuous on $\mathbb R^d \times [0,T])$, using Cauchy-Lipschitz-Picard theorem, we have \cref{eq:a2_characteristics} exists a unique solution $X(s)$. Along the characteristics curve, we have that $u(\vx,t)=f(X(0))$. Thus,
\[\|u(\vx,T-1)\|_{L^\infty}=\|f\|_{L^\infty}\leq 1\]
Then, consider the diffusion equation
\[\frac{\partial u(\boldsymbol{x},t)}{\partial t} = \sigma^2 \Delta u, \quad \vx \in \mathbb R^{d},\quad t\in[T-1,T]\]
Since we assume that $u(\vx,t)$ is a bounded solution, we can express $u(\vx,T)$ in the form of fundamental solution
\[u(\vx,T)= \frac{1}{(4\pi \sigma^2 )^{d/2}}\int_{\mathbb{R}^d} u(\vy,T-1)\exp\left(-\frac{\|\vx-\vy\|^2}{4\sigma^2 }\right)\,\mathrm d\vy\]
Then, its gradient can be bounded by
\begin{eqnarray*}
\left\|\nabla u(\vx,T)\right\| & = &\left\|\frac{1}{(4\pi \sigma^2 )^{d/2}}\int_{\mathbb{R}^d} u(\vy,T-1)\frac{\vx-\vy}{2\sigma^2}\exp\left(-\frac{\|\vx-\vy\|^2}{4\sigma^2 }\right)\,\mathrm d\vy \right\| \\
& \leq & \frac{\|u(\vy,T-1)\|_{L^\infty}}{(4\pi \sigma^2 )^{d/2}}\int_{\mathbb{R}^d} \frac{\|\vx-\vy\|}{2\sigma^2}\exp\left(-\frac{\|\vx-\vy\|^2}{4\sigma^2 }\right)\,\mathrm d\vy \\
& \leq &\frac{1}{(4\pi \sigma^2 )^{d/2}} \frac{1}{\sigma} (2\sigma)^d \int_{\mathbb{R}^d} \|s\| e^{-\|s\|^2} \mathrm d s\\
& = &\frac{1}{\pi^{d/2}\sigma} \int_{\mathbb{R}^d} \|s\| e^{-\|s\|^2} \mathrm d s
\end{eqnarray*}
where we use change of variables $s=\frac{\vx-\vy}{2\sigma}$. The leftover integral can be computed using polar coordinates and expressed in the form of Gamma function
\[\int_{\mathbb{R}^d} \|s\| e^{-\|s\|^2} \mathrm d s=\frac{\Gamma(\frac{d+1}{2})}{\Gamma(\frac{d}{2})}\cdot \pi^{d/2}\]

The ratio of Gamma function can be controlled using series expansion,
\[\frac{\Gamma(\frac{d+1}{2})}{\Gamma(\frac{d}{2})}=\sqrt{\frac{d}{2}}-\sqrt{\frac{1}{32d}}+O(d^{-3/2})<\sqrt{\frac{d}{2}}\]
and finally we get for any $\vx\in \mathbb{R}^d$
\[\left\|\nabla u(\vx,T)\right\| \leq \frac{1}{\pi^{d/2}\sigma} \frac{\Gamma(\frac{d+1}{2})}{\Gamma(\frac{d}{2})}\cdot \pi^{d/2} < \sqrt{d/2\sigma^2}\]

According to the Taylor formula, we can get that for any $\bm{\delta}\in \mathbb{R}^d$,
\[|u(\vx,T)-u(\vx+\bm{\delta},T)|\leq \max_{\vx}\|\nabla u(\vx,T)\| \|\bm{\delta}\|_2\]

% Let $\mu=1$ and $\lambda=\gamma-\sigma^2$, we can easily get
% \begin{align}\label{regularity}
%     \max_{\vx}\|\nabla u(\vx,1)\|\leq e^{\gamma-\sigma^2}\|f(\vx)\|_{C^1}
% \end{align}
% \begin{align}\label{holder1}
% |f_{\vw}(\vx)-f_{\vw}(\vx+\bm{\delta})|\leq \|\nabla f_{\vw}(\vx)\|_{q}\delta\leq d^{1/q}\| f_{\vw}(\vx)\|_{C^{1}}\delta,
% \end{align}
% Then we extend the above results to multi-class classification problem. Using the Frobenius norm of matrix to control the $\ell_2$ norm of each row, we have,
Thus,
\begin{eqnarray*}
    u^{c_A}(\vx+\bm{\delta},T)&\geq& p_A-\|\bm{\delta}\|_2 \sqrt{d/2\sigma^2}\\
    \max_{i\neq c_A}u^{i}(\vx+\bm{\delta},T)&\leq &p_B+\|\bm{\delta}\|_2 \sqrt{d/2\sigma^2}
\end{eqnarray*}
which implies that when
\[\|\bm{\delta}\|_2 \leq \frac{\sigma}{\sqrt{2d}}(p_A-p_B)\]
we have
\[u^{c_A}(\vx+\bm{\delta},T) \geq \max_{i\neq c_A}u^{i}(\vx+\bm{\delta},T)\]
The theorem is proved.
\end{proof}

Using H\"{o}lder inequality, we have $\|\bm{\delta}\|_2\leq d^{1/2-1/p} \|\bm{\delta}\|_p$ when $p>2$. When $p\leq 2$, $\|\bm{\delta}\|_2\leq \|\bm{\delta}\|_p$. Thus, our results can be extended to $\ell^p$ case by introducing a constant which depends on the dimension when $p>2$.

% Especially, if we assume that the base classifier $f_{\vw}(\vx)$ is ResNet, i.e. $f_{\vw}(\vx)\in \mathcal F$, where
%     \begin{equation*}
% \begin{aligned}
% \mathcal F:&=\Big\{f(\vx)=\phi(\vw^{\rm T}\vx^{M})|\vx^{i+1}=\rho\left(\Ub^{i}\vx^{i}\right),\mbox{where}\ \vx^0=\mbox{input data},\ \vw\in\mathbb R^{d}, \\
% & \mbox{and},\ \|\Ub^{i}\|_F\leq u_F,\ \Ub^{i}\in\mathbb R^{d\times d}\ \mbox{for}\  i=0,\cdots,M-1,\ \|\vw\|_2\leq w_2 \Big\}.
% \end{aligned}
% \end{equation*}
% Notice that 
% \begin{align*}
%     \sup_{f\in \mathcal F}\|f\|_{C^1}&=\|f\|_{\infty}+\|\nabla f\|_{\infty}\\
%     &\leq 1+\|\frac{\partial f}{\partial \vx^M}\|_2\prod_{i=0}^{M-1}\|\frac{\partial \vx^{i+1}}{\partial \vx^i}\|_2\\
%     &\leq 1+\|\vw\phi'(\vw^T\vx^M)\|_2\prod_{i=0}^{M-1}\|\Ub^i\rho'(\Ub^i\vx^i)\|_2\\
%     &\leq 1+L_1w_2(L_2u_F)^M.
% \end{align*}
% Combining (\ref{regularity}) and (\ref{CR}), we can obtain $u^{c_A}(\vx+\bm{\delta},T)\geq \max_{i\neq c_A}u^{i}(\vx+\bm{\delta},T)$ for all 
% $$\delta\leq \frac{(p_A-p_B)e^{\sigma^2T}e^{-\gamma T}}{2d^{1/q}(1+L_1w_2(L_2u_F)^M)}.$$
