%% LyX 2.3.6.1 created this file.  For more info, see http://www.lyx.org/.
%% Do not edit unless you really know what you are doing.
\documentclass[11pt,english]{article}
\usepackage{lmodern}
\renewcommand{\ttdefault}{cmtt}
% store roman font
\let\origrmdefault\rmdefault
\usepackage[math]{iwona}
% reset stored roman font
\renewcommand{\rmdefault}{\origrmdefault}
\usepackage[T1]{fontenc}
\usepackage[latin9]{inputenc}
\usepackage{geometry}
\geometry{verbose,tmargin=0.75in,bmargin=0.75in,lmargin=0.75in,rmargin=0.75in}
\setlength{\parskip}{\medskipamount}
\setlength{\parindent}{0pt}
\usepackage{array}
\usepackage{multirow}
\usepackage{amsmath}
\usepackage{amssymb}
\usepackage{graphicx}
\usepackage{setspace}
\usepackage[authoryear]{natbib}
\onehalfspacing

\makeatletter

%%%%%%%%%%%%%%%%%%%%%%%%%%%%%% LyX specific LaTeX commands.
%% Because html converters don't know tabularnewline
\providecommand{\tabularnewline}{\\}

%%%%%%%%%%%%%%%%%%%%%%%%%%%%%% Textclass specific LaTeX commands.
\newcommand{\lyxaddress}[1]{
	\par {\raggedright #1
	\vspace{1.4em}
	\noindent\par}
}

\@ifundefined{date}{}{\date{}}
%%%%%%%%%%%%%%%%%%%%%%%%%%%%%% User specified LaTeX commands.
\usepackage{pdflscape}

\makeatother

\usepackage{babel}
\begin{document}
\title{Minimal chaotic models from the Volterra gyrostat}
\author{Ashwin K Seshadri\textsuperscript{1} and S Lakshmivarahan\textsuperscript{2}}
\maketitle

\lyxaddress{\textsuperscript{1}Centre for Atmospheric and Oceanic Sciences and
Divecha Centre for Climate Change, Indian Institute of Science, Bangalore
560012, India. Email: ashwins@iisc.ac.in. }

\lyxaddress{\textsuperscript{2}Emeritus faculty at the School of Computer Science,
University of Oklahoma, Norman, OK 73012, USA. Email: varahan@ou.edu.}

\subsection*{Declarations of interest: none}

\pagebreak{}
\begin{abstract}
Low-order models obtained through Galerkin projection of several physically
important systems (e.g., Rayleigh-B$\acute{e}$nard convection, mid-latitude
quasi-geostrophic dynamics, and vorticity dynamics) appear in the
form of coupled gyrostats. Forced dissipative chaos is an important
phenomenon in these models, and this paper considers the minimal chaotic
models, in the sense of having the fewest external forcing and linear
dissipation terms, arising from an underlying gyrostat core. It is
shown here that a critical distinction is whether the gyrostat core
(without forcing or dissipation) conserves energy, depending on whether
the sum of the quadratic coefficients is zero. The paper demonstrates
that, for the energy-conserving case of the gyrostat core, the requirement
of a characteristic pair of fixed points that repel the chaotic flow
dictates placement of forcing and dissipation in the minimal chaotic
models. In contrast, if the core does not conserve energy, the forcing
can be arranged in additional ways for chaos to appear, especially
for the cases where linear feedbacks render fewer invariants in the
gyrostat core. In all cases, the linear mode must experience dissipation
for chaos to arise. Thus, the Volterra gyrostat presents a clear example
where the arrangement of fixed points circumscribes more complex dynamics.
\end{abstract}

\section{Introduction}

Forced dissipative chaos appears in many climatic and geophysical
flows (\citet{Howard1986,Swart1988,Tong2009}), with many well-known
dynamical systems combining effects of forcing as well as dissipation
(\citet{Lorenz1960,Hide1994}). A prominent example involves the special
projection of Rayleigh-B$\acute{e}$nard convection onto $3$ modes,
with one momentum and two thermal components (\citet{Lorenz1963}).
One of the earliest examples of chaos in $3$-dimensional flows, the
forcing in this model comes from an external thermal gradient driving
dynamics away from equilibrium, while dissipation appears in both
momentum and temperature dynamics. This model simplified an earlier
low-order model derived by Saltzman (\citet{Saltzman1962}), and is
derived from governing equations of convection in a fluid of uniform
depth forced by an external thermal gradient. Since the model approximates
incompressible flow in two dimensions, the equations describe streamfunction
evolution (in a single mode), in addition to two temperature modes
evolving nonlinearly (\citet{Lorenz1963}). The route to chaos in
this forced-dissipative model has been studied extensively (\citet{Sparrow1982})
and involves a sequence of bifurcations that are initiated by destabilization
of a pair of fixed points, eventually giving rise to a strange attractor. 

Many other examples of nonlinear flows in geophysics, such as wave-mean
flow interactions in mid-latitudes in the context of quasi-geostrophic
dynamics (\citet{Swart1988}), vorticity dynamics (\citet{Lorenz1960,Charney1979}),
convection in shear flows (\citet{Howard1986,Thiffeault1996,Gluhovsky1999}),
as well as flows in electrically conducting fluids (\citet{Kennett1976,Hide1994}),
have yielded low-order models admitting complex evolution. The governing
equations in such systems have generally been discretized using the
Galerkin projection method (\citet{Holmes2012}). A general difficulty
with such model reductions has been that the resulting equations do
not necessarily retain the invariants of the governing equations in
the limit without any external forcing or dissipation (\citet{Gluhovsky2002,Gluhovsky2006,Thiffeault1996}).
Several authors have considered the difficulties that can arise if
the invariants are not held by truncated equations, and maintaining
such invariants is important to avoid nonphysical numerical dissipation,
preserve analogous energy flows in the truncated equations, and avoid
spurious divergent solutions (\citet{Thiffeault1996}). 

It has been shown that the Volterra gyrostat can naturally form the
building block of Galerkin projections of the governing partial-differential
equations, while maintaining energy conservation in the unforced and
dissipationless limit (\citet{Oboukhov1975,Thiffeault1996,Gluhovsky2002}).
Motivated by early studies pointing to the importance of modular approaches
to constructing low order models (LOMs) from the governing equations
(\citet{Oboukhov1975}), as well as the role of systematic approaches
for ensuring the maintenance of invariants within the conservative
core (\citet{Thiffeault1996,Gluhovsky1997,Gluhovsky1999}), more recent
studies have not only expanded on the earlier approaches but also
exemplified the ideas (\citet{Gluhovsky2006,Lakshmivarahan2006,Lakshmivarahan2008,Tong2008}).
Many examples from these domains have been identified that can be
described in terms of systems of coupled gyrostats (\citet{Gluhovsky1997,Gluhovsky1999,Gluhovsky2002,Gluhovsky2006,Tong2008}).
Furthermore, where there exist quadratic invariants such as kinetic
energy or the squared angular momentum, these are maintained in the
resulting truncated equations as well. Owing to such properties, it
is of widespread importance to study dynamics of models arising from
the Volterra gyrostat. An important generalization of the Volterra
gyrostat involves the inclusion of nonlinear feedback between modes
(\citet{Lakshmivarahan2008a}).

This paper considers chaos in models with forcing and dissipation
added to the equations of the Volterra gyrostat having linear feedback
terms. The Volterra gyrostat is a three-dimensional volume-conserving
flow with a skew-symmetric structure of linear feedbacks and nonlinear
interactions between modes. To form the building blocks of LOMs, it
has been convenient to transform the original equations written by
Volterra through a smooth change of variables (\citet{Gluhovsky1999}).
As a result, in general, the building blocks of these LOMs have two
invariants, analogous to the conservation of kinetic energy and angular
momentum in the physical gyrostat. Each invariant confines the dynamics
to a two-dimensional surface, and their intersection gives rise to
oscillatory dynamics for the gyrostat core of these models. With inclusion
of forcing and dissipation (F\&D), there are no longer any quadratic
invariants, and thus F\&D can generate higher dimensional dynamics,
including chaos, when two invariants exist in the gyrostatic core. 

In the original gyrostat equations, the sum of the quadratic coefficients
is zero, a property rooted in the kinetic energy conservation of the
physical system (\citet{Gluhovsky1999}). It is remarkable that this
constraint leads to not one, but two, quadratic invariants in the
gyrostat core (\citet{Seshadri2023}). As a result, three-dimensional
systems where this constraint holds require forcing and dissipation
to be present for chaos to appear. More generally, a simple modification
to the gyrostat\textquoteright s quadratic coefficients (even without
F\&D being present) can reduce the number of quadratic invariants
(\citet{Seshadri2023}). In particular, it was previously shown that
if the sum of quadratic coefficients is nonzero, the gyrostat does
not conserve energy and then the number of invariants depends on the
number of linear feedbacks (\citet{Seshadri2023}). For example, if
there are three distinct linear feedbacks, then there are no quadratic
invariants in the gyrostat without a zero-sum of quadratic coefficients,
and such models can admit chaotic dynamics even without F\&D (\citet{Seshadri2023}). 

While maintaining other features of the gyrostatic models such as
conservation of volumes in phase space, such cores can also naturally
appear in Galerkin projections. Depending on the number of quadratic
and linear terms present, \citet{Gluhovsky1999} have shown that the
Volterra gyrostat can be specialized into nine different subclasses
by setting various combinations of parameters to zero. Different subclasses
constitute different linear and nonlinear interactions between modes
and together describe the different types of gyrostatic cores. The
number of quadratic invariants in the gyrostatic core can be expected
to influence the ways in which chaos can be produced due to F\&D effects.

Considering the combined effects of forcing and dissipation, we identify
minimal chaotic models derived from the Volterra gyrostat. Prior studies
of chaos with F\&D do not take energy conservation into account, but
we show that this property makes an important difference. We first
consider the possibility of chaos due to F\&D without the energy conservation
constraint, where we must distinguish different cases by merely identifying
those equations where nonzero F\&D must arise. Given the three components
of the vector field, we obtain a possible $2^{6}=64$ cases for placement
of F\&D, out of which $2\times2^{3}=16$ cases have either no forcing
or no dissipation (or neither) in any of the equations. We need only
consider the remaining $48$ cases for the presence of chaos, and
simulate ensembles to sample the parameter space for each of these
cases. Upon listing all the chaotic cases, we note that there often
exist cases that are proper subsets, containing some (but not all)
of the forcing and dissipation terms. These proper subsets are defined
as \textquotedblleft minimal chaotic models (MCMs)\textquotedblright .
We identify all the MCMs for the gyrostat having two nonlinear terms.
There could be more than one MCM, corresponding to distinct proper
subsets. The significance of these MCMs lies not only in their specific
arrangements of forcing and dissipation, but also in common features
across different subclasses of the gyrostat. Following this, we consider
the effects of whether energy is conserved in the gyrostat's core,
which influences where forcing must be placed for chaos to appear
in the equations.

\section{Models and methods}

Volterra's equations for the gyrostat
\begin{align}
K_{1}^{2}\dot{y}_{1} & =\left(K_{2}^{2}-K_{3}^{2}\right)y_{2}y_{3}+h_{2}y_{3}-h_{3}y_{2}\nonumber \\
K_{2}^{2}\dot{y}_{2} & =\left(K_{3}^{2}-K_{1}^{2}\right)y_{3}y_{1}+h_{3}y_{1}-h_{1}y_{3}\nonumber \\
K_{3}^{2}\dot{y}_{3} & =\left(K_{1}^{2}-K_{2}^{2}\right)y_{1}y_{2}+h_{1}y_{2}-h_{2}y_{1}\label{eq:p1}
\end{align}
with $y_{i}$, $i=1,2,3$, being the angular velocity of the carrier
body, $K_{i}^{2}=I_{i}$ the principal moments of inertia of the gyrostat,
and $h_{i}$ the fixed angular momenta of the rotor relative to the
carrier, are transformed smoothly (\citet{Gluhovsky1999,Seshadri2023})
using $K_{i}y_{i}=x_{i}$ and upon defining new parameters
\begin{align}
p & =K_{2}^{2}-K_{3}^{2},q=K_{3}^{2}-K_{1}^{2},\textrm{and }r=K_{1}^{2}-K_{2}^{2}\nonumber \\
a & =K_{1}h_{1},b=K_{2}h_{2},\textrm{and }c=K_{3}h_{3}\label{eq:p2}
\end{align}
into the system
\begin{align}
x_{1}' & =px_{2}x_{3}+bx_{3}-cx_{2}\nonumber \\
x_{2}' & =qx_{3}x_{1}+cx_{1}-ax_{3}\nonumber \\
x_{3}' & =rx_{1}x_{2}+ax_{2}-bx_{1},\label{eq:p3}
\end{align}
that naturally appears in modular form in many LOMs. Here $'$ denotes
$d/ds$, while $\dot{}$ is $d/dt$, where $t=K_{1}K_{2}K_{3}s$.
Henceforth we shall work exclusively with the model in Eq. (\ref{eq:p3})
and denote time as appearing there as $t$. We shall refer to Eq.
(\ref{eq:p3}) as the gyrostat core. The resulting flow preserves
volumes in phase space, as the trace of the Jacobian is zero. Broadly
we must distinguish two types of conditions for the gyrostat core:
\begin{itemize}
\item With $p+q+r=0$ following Eq. (\ref{eq:p2}), the model in Eq. (\ref{eq:p3})
conserves kinetic energy as well as squared angular momentum and the
solutions are oscillatory for all initial conditions (\citet{Seshadri2023}). 
\item In contrast for $p+q+r\neq0$, where the model does not have a direct
analogue to the physical gyrostat, the number of invariants depends
on the number of nonzero linear coefficients $\left(a,b,c\right)$,
with zero, one, and two invariants for three, two, or fewer nonzero
coefficients respectively. Only with all of $a,b,c\neq0$ in the absence
of energy conservation does the model in Eq. (\ref{eq:p3}) admit
chaos as a result of no invariants being present (\citet{Seshadri2023}). 
\end{itemize}
Despite the necessity of non-conservation of energy from $p+q+r\neq0$
for chaos in the gyrostat core, such a distinction is not usually
made for chaos in the presence of F\&D. As is well known, counterparts
of Eq. (\ref{eq:p3}) with F\&D can present chaos even when energy
conservation is present and the gyrostat core has periodic dynamics
(e.g., \citet{Lorenz1963}). 

In this paper we consider models with F\&D
\begin{align}
x_{1}' & =px_{2}x_{3}+bx_{3}-cx_{2}-\epsilon_{1}x_{1}+F_{1}\nonumber \\
x_{2}' & =qx_{3}x_{1}+cx_{1}-ax_{3}-\epsilon_{2}x_{2}+F_{2}\nonumber \\
x_{3}' & =rx_{1}x_{2}+ax_{2}-bx_{1}-\epsilon_{3}x_{3}+F_{3}\label{eq:p4}
\end{align}
with $\epsilon_{i}\geq0$, $i=1,2,3$ and $F_{j}\in\mathbb{R}$, $j=1,2,3$.
First, we shall sample from the parameter space without imposing the
constraint $p+q+r=0$, wherein our goal will be to identify chaotic
dynamics with sparse inclusion of forcing and dissipation, where as
few of the $\epsilon_{1},\epsilon_{2},\epsilon_{3},F_{1},F_{2},F_{3}$
are nonzero as possible. 

\citet{Gluhovsky1999} have identified special cases (``subclasses'')
of Eq. (\ref{eq:p3}), by specializing the quadratic and linear coefficients.
With the energy conservation constraint that is present throughout
their analysis, there must be at least two nonzero $p,q,r$ otherwise
the model is linear, so without loss of generality they assume that
$p,q\neq0$. Further restricting various combinations of linear coefficients
to be zero gives nine subclasses in addition to the general case with
nonzero parameters. We examine these subclasses for the role of F\&D
in producing chaos, considering only those subclasses with two (but
not three) quadratic terms (subclasses $1-4$ of \citet{Gluhovsky1999},
with $r=0$ in Eq. (\ref{eq:p4})).\footnote{We have also omitted the degenerate cases in subclasses $8-9$, as
defined by \citet{Gluhovsky1999}, for which the dynamics is two-dimensional. } This restriction gives a cubic equation for the steady states, as
shown below, making these subclasses amenable to a common approach.

For each of the subclasses $1-4$ in \citet{Gluhovsky1999}, we distinguish
different cases according to where forcing and dissipation arise.
Thus for each subclass, we can have either $\epsilon_{i}=0$ or $\epsilon_{i}>0$
and likewise $F_{j}=0$ or $F_{j}\neq0$, for $i=1,2,3$ and $j=1,2,3$
, for a total of $4^{3}=64$ cases that differ in whether each equation
has dissipation, forcing, both or neither. Eight of these cases have
no dissipation in the model and eight have no forcing, and we are
left with $48$ cases having nonzero forcing as well as dissipation,
which must be evaluated for the possibility of chaos. Additionally,
each case has parameters $\left(p,q,a,b,c\right)$ as well as values
of $\left(\epsilon_{i},F_{j}\right)$ that must be varied $\left(r=0\right)$.
For each of these $48$ cases of each of the $4$ subclasses with
two nonlinear terms, we generate a large (as much as $10$-dimensional)
Latin hypercube sample to vary model parameters $\left(p,q,a,b,c\right)$,
forcing and dissipation $\left(\epsilon_{i},F_{j}\right)$, and initial
conditions $\left(x_{10},x_{20},x_{30}\right)$. Where dissipation
appears in more than one mode, i.e. $\epsilon_{i},\epsilon_{j}>0$
for $i\neq j$ it is assigned the same value $\epsilon_{i}=\epsilon_{j}=\epsilon$
in the ensemble, and likewise with the value of the forcing terms,
since the main goal is to identify MCMs. 

For each member of the sample, we integrate from the corresponding
initial conditions $\left(x_{10},x_{20},x_{30}\right)$ a $12$-dimensional
system describing the state-variables in Eq. (\ref{eq:p3}) as well
as the evolution of the $3\times3$ initial condition sensitivity
matrix $\mathrm{D_{x_{0}}}\left(t\right)$ that allows us to evaluate
the $3$ Lyapunov exponents of the model. Lyapunov exponents are computed
using a standard approach, based on singular value decomposition (SVD)
of the matrix $\mathrm{M}\left(t\right)$ defined as $\mathrm{M}\left(t\right)=\mathrm{D_{x_{0}}}\left(t\right)^{T}\mathrm{D_{x_{0}}}\left(t\right)$.
The matrix $\mathrm{D_{x_{0}}\left(\mathit{t}\right)\in\mathbb{R}^{\mathit{3\times3}}}$
consists of elements $\mathrm{D_{x_{0}}^{\mathit{i,j}}\left(\mathit{t}\right)=}\partial x_{i}\left(t\right)/\partial x_{j0}$
describing forward sensitivities to perturbation in the initial condition
(\citet{Pikovsky2016}). This matrix is initialized to the $3\times3$
identity matrix at $t=0$. We integrate in time for the state $\left(x_{1}\left(t\right),x_{2}\left(t\right),x_{3}\left(t\right)\right)$
as well as the nine elements of $\mathrm{D_{x_{0}}}\left(t\right)$,
from $t=0$ to $t=5000$. 

For a given case, once the $3$ Lyapunov exponents are estimated from
the SVD of matrix $\mathrm{M}\left(t=5000\right)$, and the largest
Lyapunov exponent (LLE) is found for each of the samples, we identify
the most unstable sample among those whose LLE is positive. In doing
so, we consider only those samples whose evolution is bounded and
for which the LLE calculation does not diverge. Many combinations
of forcing and dissipation can give rise to unbounded dynamics even
for these volume-contracting flows with dissipation. In case there
exists a most unstable sample with bounded evolution and positive
but finite LLE, that case of the subclass is examined further for
the presence of chaos. Specifically, we plot the $3$-dimensional
orbits, time-series of $x_{1}$, Poincar$\acute{\textrm{e}}$ sections,
power spectra of $x_{1}$, and the evolution of LLE with time $\lambda\left(t\right)$.
Together these multiple lines of evidence allow us to distinguish
chaotic orbits from non-chaotic ones among those where LLE is estimated
to be positive. The detailed results are plotted in the Supplementary
Information (SI).

This procedure is repeated for each of the $48$ cases for all $4$
subclasses. A case is chaotic if it presents a sample (i.e., choice
of parameters and initial conditions) with bounded evolution and concurrent
lines of evidence for chaos: broadband power spectrum, Poincar$\acute{\textrm{e}}$
sections with fractal structure, and evolution of LLE to a positive
value with time, in addition to highly irregular time-series of $x_{1}$.
For a given subclass, after chaotic cases (among the possible $48$)
have been identified, minimal chaotic cases are found by inspection.
A minimal chaotic case has a proper subset of all the forcing and
dissipation terms. 

The above calculations are repeated for each of the $4$ subclasses
with the energy conservation constraint for the gyrostatic core being
present, i.e. $p+q+r=0$, by setting $q=-p$, since $r=0$ in each
of these subclasses. 

\section{Identification of minimal chaotic models}

Chaotic cases are identified from among each of the $48$ possible
arrangements of nonzero forcing and dissipation (``cases''), with
each case consisting of $2000$ samples across model parameters, initial
conditions, and forcing and dissipation coefficients, arranged in
a Latin hypercube. While such an empirical approach cannot ensure
that all cases admitting chaos are identified, each minimal chaotic
case appears to have been identified explicitly, as shown below. Multiple
lines of evidence have been used to identify chaotic cases. As described
in the previous section, we identify simulations having a positive
value of the largest Lyapunov exponent (LLE) to shortlist potential
cases admitting chaos, and deploy further lines of evidence to confirm
the appearance of chaos: inspection of the orbits and time-series
of $x_{1}$, examining the Poincar$\acute{\textrm{e}}$ sections,
considering whether the simulated time-series (once transients have
diminished) have a broadband power spectrum, and the evolution of
the LLE towards positive values. We recall that our present analysis
takes $p+q+r\neq0$. 

For subclass 1 there are many cases with positive LLE ($17$ have
been found explicitly; see Supplementary Information (SI) Figures
1-5), but only some of these are chaotic by the above measures. The
chaotic cases have nonzero $\left(F_{1},\epsilon_{3}\right)$, $\left(F_{1},\epsilon_{1},\epsilon_{3}\right)$,
$\left(F_{1},\epsilon_{2},\epsilon_{3}\right)$, $\left(F_{1},F_{2},\epsilon_{3}\right)$,
$\left(F_{1},F_{3},\epsilon_{3}\right)$, $\left(F_{1},F_{2},F_{3},\epsilon_{3}\right)$
and $\left(F_{1},F_{2},F_{3},\epsilon_{2},\epsilon_{3}\right)$. Despite
the diversity of chaotic orbits (Figure 1), inspection identifies
a unique MCM with nonzero $\left(F_{1},\epsilon_{3}\right)$, since
each of these cases has nonzero $F_{1}$ and $\epsilon_{3}$. In each
case the most chaotic version of the $2000$-member ensemble, having
maximum positive value of LLE, has been plotted to illustrate the
appearance of chaos. 

Similarly, for subclass $2$, of the $23$ cases with positive LLE
(SI Figures 6-10), there are $12$ cases that are chaotic by these
above measures. These cases (orbits in Figure 2), have nonzero $\left(F_{1},\epsilon_{3}\right)$,$\left(F_{1},\epsilon_{1},\epsilon_{3}\right)$,$\left(F_{1},\epsilon_{2},\epsilon_{3}\right)$,
$\left(F_{2},\epsilon_{3}\right)$,$\left(F_{2},\epsilon_{1},\epsilon_{3}\right)$,$\left(F_{1},F_{2},\epsilon_{3}\right)$,$\left(F_{1},F_{2},\epsilon_{1},\epsilon_{3}\right)$
,$\left(F_{1},F_{2},\epsilon_{2},\epsilon_{3}\right),\left(F_{3},\epsilon_{2},\epsilon_{3}\right),\left(F_{1},F_{3},\epsilon_{3}\right)$,$\left(F_{2},F_{3},\epsilon_{3}\right)$
and $\left(F_{1},F_{2},F_{3},\epsilon_{3}\right)$. The MCMs as defined
above include all proper subsets, and not only those that have the
fewest number of terms. The simplest such models, with one forcing
and one dissipation term, involve nonzero $\left(F_{1},\epsilon_{3}\right)$
and $\left(F_{2},\epsilon_{3}\right)$. In addition, there is the
case with nonzero $\left(F_{3},\epsilon_{2},\epsilon_{3}\right)$,
which is irreducible to the other two cases owing to different placement
of forcing. Thus there are $3$ MCMs for subclass $2$: $\left(F_{1},\epsilon_{3}\right)$,
$\left(F_{2},\epsilon_{3}\right)$, and $\left(F_{3},\epsilon_{2},\epsilon_{3}\right)$,
with $p+q+r\neq0$. 

Similar analysis for subclasses $3$ and $4$ identifies $11$ and
$8$ chaotic cases respectively, whose orbits are shown in Figures
3--4. For both of these subclasses, there are $3$ MCMs, with nonzero
$\left(F_{1},\epsilon_{3}\right)$, $\left(F_{2},\epsilon_{3}\right)$,
and $\left(F_{3},\epsilon_{3}\right)$. The corresponding time-series
of $x_{1}$, Poincar$\acute{\textrm{e}}$ sections, power spectra
of the stationary orbits, and evolution of LLE, are shown in SI Figures
11-14, and Figures 15-18, respectively for the two subclasses. Table
1 below summarizes the chaotic cases and MCMs, for each of the subclasses.
While siting of forcing can vary, dissipation is circumscribed and
all chaotic cases must involve dissipation in the third (i.e., linear)
equation through nonzero $\epsilon_{3}$.

The table also lists the MCMs when the energy conservation constraint
is present, based on simulations with $q=-p$, since $r=0$. The cases
with positive LLEs are listed in the SI (Figures 22-24, 25-27, 28-30,
and 31-33 for subclasses $1-4$ respectively), leading to fewer MCMs
when energy is conserved. For subclass $1,$there is no effect of
the energy conservation constraint, with MCM $\left(F_{1},\epsilon_{3}\right)$
in either case. In subclass $2$, only $\left(F_{1},\epsilon_{3}\right)$
and $\left(F_{2},\epsilon_{3}\right)$ are MCMs. For subclass $3$,
$\left(F_{1},\epsilon_{3}\right)$ is an MCM while $\left(F_{2},\epsilon_{3}\right)$
and $\left(F_{3},\epsilon_{3}\right)$ are not. Similarly, in subclass
$4$, $\left(F_{1},\epsilon_{3}\right)$ and $\left(F_{2},\epsilon_{3}\right)$
are MCMs, while $\left(F_{3},\epsilon_{3}\right)$ is not. If the
gyrostatic core conserves energy, forcing can be placed in fewer ways
for the model to admit chaos. These effects of the energy conservation
constraint are accounted for in the following section. 

\pagebreak{}

\begin{landscape}

Table 1: Chaotic cases, minimal chaotic models (MCMs), and coefficients
of cubic describing $x_{3}^{*}$ for subclasses $1-4$. 

\begin{tabular}{|c|c|c|c|c|}
\hline 
{\footnotesize{}Subclass} & {\footnotesize{}Chaotic cases $\left(p+q+r\neq0\right)$} & {\footnotesize{}MCM $\left(p+q+r\neq0\right)$} & {\footnotesize{}MCM $\left(p+q+r=0\right)$} & {\footnotesize{}Coefficients of cubic equation for $x_{3}^{*}$}\tabularnewline
\hline 
\hline 
\multirow{4}{*}{{\footnotesize{}$1$}} & \multirow{1}{*}{{\footnotesize{}$\left(F_{1},\epsilon_{3}\right)$, $\left(F_{1},\epsilon_{1},\epsilon_{3}\right)$,
$\left(F_{1},\epsilon_{2},\epsilon_{3}\right)$,}} & \multirow{4}{*}{{\footnotesize{}$\left(F_{1},\epsilon_{3}\right)$}} & \multirow{4}{*}{{\footnotesize{}$\left(F_{1},\epsilon_{3}\right)$}} & {\footnotesize{}$\gamma_{3}=\epsilon_{3}\frac{p}{a}$}\tabularnewline
\cline{2-2} \cline{5-5} 
 & {\footnotesize{}$\left(F_{1},F_{2},\epsilon_{3}\right)$, $\left(F_{1},F_{3},\epsilon_{3}\right)$, } &  &  & {\footnotesize{}$\gamma_{2}=-F_{3}\frac{p}{a}$,}\tabularnewline
\cline{2-2} \cline{5-5} 
 & {\footnotesize{}$\left(F_{1},F_{2},F_{3},\epsilon_{3}\right)$,} &  &  & {\footnotesize{}$\gamma_{1}=F_{1}-\epsilon_{1}\frac{a}{q}-\epsilon_{1}\epsilon_{2}\epsilon_{3}\frac{1}{aq}$,}\tabularnewline
\cline{2-2} \cline{5-5} 
 & {\footnotesize{}$\left(F_{1},F_{2},F_{3},\epsilon_{2},\epsilon_{3}\right)$.} &  &  & {\footnotesize{}$\gamma_{0}=\frac{F_{2}\epsilon_{1}}{q}+\frac{F_{3}\epsilon_{1}\epsilon_{2}}{aq}$.}\tabularnewline
\hline 
\multirow{4}{*}{{\footnotesize{}$2$}} & {\footnotesize{}$\left(F_{1},\epsilon_{3}\right)$, $\left(F_{1},\epsilon_{1},\epsilon_{3}\right)$,
$\left(F_{1},\epsilon_{2},\epsilon_{3}\right)$,} & \multirow{4}{*}{{\footnotesize{}$\left(F_{1},\epsilon_{3}\right),$$\left(F_{2},\epsilon_{3}\right)$,$\left(F_{3},\epsilon_{2},\epsilon_{3}\right)$}} & \multirow{4}{*}{{\footnotesize{}$\left(F_{1},\epsilon_{3}\right),$$\left(F_{2},\epsilon_{3}\right)$}} & {\footnotesize{}$\gamma_{3}=\epsilon_{3}\frac{p}{a}$,}\tabularnewline
\cline{2-2} \cline{5-5} 
 & {\footnotesize{}$\left(F_{2},\epsilon_{3}\right)$, $\left(F_{2},\epsilon_{1},\epsilon_{3}\right)$,$\left(F_{1},F_{2},\epsilon_{3}\right)$,} &  &  & {\footnotesize{}$\gamma_{2}=-F_{3}\frac{p}{a}+b\left(1+\frac{p}{q}\right)$,}\tabularnewline
\cline{2-2} \cline{5-5} 
 & {\footnotesize{}$\left(F_{1},F_{2},\epsilon_{1},\epsilon_{3}\right)$,$\left(F_{1},F_{2},\epsilon_{2},\epsilon_{3}\right),\left(F_{3},\epsilon_{2},\epsilon_{3}\right),$} &  &  & {\footnotesize{}$\gamma_{1}=F_{1}-F_{2}b\frac{p}{aq}-\epsilon_{1}\frac{a}{q}-\epsilon_{1}\epsilon_{2}\epsilon_{3}\frac{1}{aq}-\epsilon_{2}\frac{b^{2}}{aq}$,}\tabularnewline
\cline{2-2} \cline{5-5} 
 & {\footnotesize{}$\left(F_{1},F_{3},\epsilon_{3}\right)$, $\left(F_{2},F_{3},\epsilon_{3}\right)$,
$\left(F_{1},F_{2},F_{3},\epsilon_{3}\right)$.} &  &  & {\footnotesize{}$\gamma_{0}=\frac{F_{2}\epsilon_{1}}{q}+\frac{F_{3}\epsilon_{1}\epsilon_{2}}{aq}-F_{1}\epsilon_{2}\frac{b}{aq}$.}\tabularnewline
\hline 
\multirow{4}{*}{{\footnotesize{}$3$}} & {\footnotesize{}$\left(F_{1},\epsilon_{3}\right)$, $\left(F_{1},\epsilon_{2},\epsilon_{3}\right)$,
$\left(F_{2},\epsilon_{3}\right)$,} & \multirow{4}{*}{{\footnotesize{}$\left(F_{1},\epsilon_{3}\right)$, $\left(F_{2},\epsilon_{3}\right)$,
$\left(F_{3},\epsilon_{3}\right)$}} & \multirow{4}{*}{{\footnotesize{}$\left(F_{1},\epsilon_{3}\right)$}} & {\footnotesize{}$\gamma_{3}=\epsilon_{3}\frac{p}{a}$,}\tabularnewline
\cline{2-2} \cline{5-5} 
 & {\footnotesize{}$\left(F_{2},\epsilon_{2},\epsilon_{3}\right)$,$\left(F_{1},F_{2},\epsilon_{3}\right)$,} &  &  & {\footnotesize{}$\gamma_{2}=-F_{3}\frac{p}{a}+\epsilon_{3}\frac{c}{a}\left(\frac{p}{q}-1\right)$,}\tabularnewline
\cline{2-2} \cline{5-5} 
 & {\footnotesize{}$\left(F_{1},F_{2},\epsilon_{1},\epsilon_{3}\right)$,$\left(F_{3},\epsilon_{3}\right),\left(F_{1},F_{3},\epsilon_{3}\right)$,} &  &  & {\footnotesize{}$\gamma_{1}=F_{1}-F_{3}\frac{c}{a}\left(\frac{p}{q}-1\right)-\epsilon_{1}\frac{a}{q}-\epsilon_{1}\epsilon_{2}\epsilon_{3}\frac{1}{aq}-\epsilon_{3}\frac{c^{2}}{aq}$,}\tabularnewline
\cline{2-2} \cline{5-5} 
 & {\footnotesize{}$\left(F_{2},F_{3},\epsilon_{3}\right)$, $\left(F_{2},F_{3},\epsilon_{2},\epsilon_{3}\right)$,
$\left(F_{1},F_{2},F_{3},\epsilon_{3}\right)$.} &  &  & {\footnotesize{}$\gamma_{0}=F_{1}\frac{c}{q}+\frac{F_{2}\epsilon_{1}}{q}+F_{3}\left(\frac{c^{2}}{aq}+\frac{\epsilon_{1}\epsilon_{2}}{aq}\right)$.}\tabularnewline
\cline{2-5} \cline{3-5} \cline{4-5} \cline{5-5} 
\multicolumn{1}{c|}{\multirow{4}{*}{{\footnotesize{}$4$}}} & {\footnotesize{}$\left(F_{1},\epsilon_{3}\right)$, $\left(F_{2},\epsilon_{3}\right)$,
$\left(F_{1},F_{2},\epsilon_{3}\right)$,} & \multirow{4}{*}{{\footnotesize{}$\left(F_{1},\epsilon_{3}\right)$, $\left(F_{2},\epsilon_{3}\right)$,
$\left(F_{3},\epsilon_{3}\right)$}} & \multirow{4}{*}{{\footnotesize{}$\left(F_{1},\epsilon_{3}\right)$, $\left(F_{2},\epsilon_{3}\right)$}} & {\footnotesize{}$\gamma_{3}=\epsilon_{3}\frac{p}{a}$}\tabularnewline
\cline{2-2} \cline{5-5} 
 & {\footnotesize{}$\left(F_{1},F_{2},\epsilon_{2},\epsilon_{3}\right)$,$\left(F_{3},\epsilon_{3}\right),\left(F_{1},F_{3},\epsilon_{3}\right)$,} &  &  & {\footnotesize{}$\gamma_{2}=-F_{3}\frac{p}{a}+\epsilon_{3}\frac{c}{a}\left(\frac{p}{q}-1\right)+b\left(1+\frac{px_{1}^{*}}{a}\right)$,}\tabularnewline
\cline{2-2} \cline{5-5} 
 & {\footnotesize{}$\left(F_{2},F_{3},\epsilon_{3}\right)$, $\left(F_{1},F_{2},F_{3},\epsilon_{3}\right)$.} &  &  & {\footnotesize{}$\gamma_{1}=F_{1}-F_{3}\frac{c}{a}\left(\frac{p}{q}-1\right)-\epsilon_{1}\frac{a}{q}-\epsilon_{1}\epsilon_{2}\epsilon_{3}\frac{1}{aq}-\epsilon_{3}\frac{c^{2}}{aq}+\frac{bc}{q}+\frac{bcx_{1}^{*}}{a}\left(\frac{p}{q}-1\right)$,}\tabularnewline
\cline{2-2} \cline{5-5} 
 &  &  &  & {\footnotesize{}$\gamma_{0}=F_{1}\frac{c}{q}+\frac{F_{2}\epsilon_{1}}{q}+F_{3}\left(\frac{c^{2}}{aq}+\frac{\epsilon_{1}\epsilon_{2}}{aq}\right)-\frac{bx_{1}^{*}\left(c^{2}+\epsilon_{1}\epsilon_{2}\right)}{aq}$.}\tabularnewline
\cline{2-5} \cline{3-5} \cline{4-5} \cline{5-5} 
\end{tabular}

\end{landscape}

\pagebreak{}

\section{Accounting for minimal chaotic models}

\subsection{Subclass 1 with $r=0,b=c=0$}

This subclass of the Volterra gyrostat has evolution
\begin{align}
x_{1}' & =px_{2}x_{3}\nonumber \\
x_{2}' & =qx_{3}x_{1}-ax_{3}.\nonumber \\
x_{3}' & =ax_{2}\label{eq:p5}
\end{align}
and serves as the conservative core of many important LOMs (\citet{Gluhovsky1999}),
including that of \citet{Lorenz1963}. It has two constants of motion
irrespective of whether $p+q=0$ (\citet{Seshadri2023}), and rotational
symmetry $R_{x_{1}}\left(\pi\right)$ about the axis of $x_{1}$,
i.e. the equations are preserved under the transformation $\left(x_{1},x_{2},x_{3}\right)\rightarrow\left(x_{1},-x_{2},-x_{3}\right)$.
Since all chaotic cases necessarily have nonzero $\left(F_{1},\epsilon_{3}\right)$,
the MCM
\begin{align}
x_{1}' & =px_{2}x_{3}+F_{1}\nonumber \\
x_{2}' & =qx_{3}x_{1}-ax_{3}.\nonumber \\
x_{3}' & =ax_{2}-\epsilon_{3}x_{3}\label{eq:p6}
\end{align}
also maintains $R_{x_{1}}\left(\pi\right)$. For this last system
there are two distinct fixed points given by $\left(a/q,x_{2}^{*},x_{3}^{*}\right)$
and $\left(a/q,-x_{2}^{*},-x_{3}^{*}\right)$ with $x_{2}^{*2}=-\frac{\epsilon_{3}F_{1}}{ap}$
and $x_{3}^{*2}=-\frac{aF_{1}}{p\epsilon_{3}}$, which are real if
$F_{1}$ has opposite sign from $ap$ $\left(\epsilon_{3}>0\right)$.
The Jacobian evaluated at $\left(a/q,x_{2}^{*},x_{3}^{*}\right)$
becomes 
\begin{equation}
\mathrm{D}_{\mathrm{f}}=\left[\begin{array}{ccc}
0 & px_{3}^{*} & px_{2}^{*}\\
qx_{3}^{*} & 0 & 0\\
0 & a & -\epsilon_{3}
\end{array}\right]\label{eq:p7}
\end{equation}
having characteristic polynomial 
\begin{equation}
\lambda^{3}+\epsilon_{3}\lambda^{2}-pqx_{3}^{*2}\lambda-2\epsilon_{3}pqx_{3}^{*2}=0,\label{eq:p8}
\end{equation}
where we have used, from the last component of Eq. (\ref{eq:p6}),
$x_{2}^{*}=\left(\epsilon_{3}/a\right)x_{3}^{*}$. From the symmetry
of the equations, the Jacobian evaluated at $\left(a/q,-x_{2}^{*},-x_{3}^{*}\right)$
also has the same characteristic polynomial, and thus the above fixed
points make a pair with identical stability. Moreover the discriminant
of the characteristic polynomial in Eq. (\ref{eq:p8})\footnote{For a general cubic given by $\beta_{3}x^{3}+\beta_{2}x^{2}+\beta_{1}x+\beta_{0}=0,$the
discriminant is defined as $\Delta=18\beta_{3}\beta_{2}\beta_{1}\beta_{0}-4\beta_{2}^{3}\beta_{0}+\beta_{2}^{2}\beta_{1}^{2}-4\beta_{3}\beta_{1}^{3}-27\beta_{3}^{2}\beta_{0}^{2}$
and its sign determines the number of real and complex roots.} 
\begin{equation}
\Delta=pqx_{3}^{*2}\left(8\epsilon_{3}^{4}-71\epsilon_{3}^{2}pqx_{3}^{*2}+4p^{2}q^{2}x_{3}^{*4}\right)<0\label{eq:p9}
\end{equation}
in case $pq<0$, since $x_{3}^{*2}>0$. The MCM plotted in Figure
1 has parameters $a=-0.94$, $p=0.19$, and $q=-0.39$. Therefore,
$pq$ is indeed negative, and there is one zero eigenvalue and two
complex eigenvalues for the underlying fixed point. A Hopf bifurcation
is expected to occur where the real part of the complex roots crosses
the imaginary axis. With product of eigenvalues $2\epsilon_{3}pqx_{3}^{*2}<0$
there is one stable direction, but when $F_{1}$ is sufficient large
the real part of the complex conjugate pair becomes positive and both
these fixed points repel the nearby flows.

These properties are not maintained in case a single forcing and dissipation
term occur elsewhere. There are $8$ alternate ways in which a single
F\&D term can be placed and these give different arrangements of fixed
points from the above (Table 2): ranging from no fixed points to an
infinite number. Where a pair of fixed points exists, they do not
possess the above symmetry. The MCM $\left(F_{1},\epsilon_{3}\right)$
is not alone in maintaining the symmetry of the gyrostatic core, and
other choices in Table 2, such as $\left(F_{1},\epsilon_{1}\right)$,
$\left(F_{1},\epsilon_{2}\right)$ maintain it as well. The model
$\left(F_{1},\epsilon_{3}\right)$ has the unique attribute of maintaining
symmetry of the equations while giving rise to a pair of distinct
and symmetrically placed fixed points. Other chaotic cases listed
in Table 1 do not always maintain this symmetry. As we shall see below,
what matters in general for the appearance of chaos is a pair of fixed
points arranged on opposite sides of the $x_{1}-x_{2}$ plane. The
MCM $\left(F_{1},\epsilon_{3}\right)$ also appears when energy is
conserved in the gyrostat core (SI Figs. 22-24; Table 1).

\pagebreak{}

\begin{landscape}

Table 2: Fixed points for various other placements of one F\&D term
(besides the MCM) in Subclass 1. 

\begin{tabular}{|c|c|c|c|}
\hline 
S. No & Model & Equations & Fixed Points\tabularnewline
\hline 
$1$ & $F_{1},\epsilon_{1}$ & $x_{1}'=px_{2}x_{3}+F_{1}-\epsilon_{1}x_{1}\textrm{, }x_{2}'=qx_{3}x_{1}-ax_{3}\textrm{, }x_{3}'=ax_{2}$ & $\left(\frac{F_{1}}{\epsilon_{1}},0,0\right)$\tabularnewline
\hline 
$2$ & $F_{1},\epsilon_{2}$ & $x_{1}'=px_{2}x_{3}+F_{1}\textrm{, }x_{2}'=qx_{3}x_{1}-ax_{3}-\epsilon_{2}x_{2}\textrm{, }x_{3}'=ax_{2}$ & None\tabularnewline
\hline 
$3$ & $F_{2},\epsilon_{1}$ & $x_{1}'=px_{2}x_{3}-\epsilon_{1}x_{1}\textrm{, }x_{2}'=qx_{3}x_{1}-ax_{3}+F_{2}\textrm{, }x_{3}'=ax_{2}$ & $\left(0,0,\frac{F_{2}}{a}\right)$\tabularnewline
\hline 
$4$ & $F_{2},\epsilon_{2}$ & $x_{1}'=px_{2}x_{3}\textrm{, }x_{2}'=qx_{3}x_{1}-ax_{3}+F_{2}-\epsilon_{2}x_{2}\textrm{, }x_{3}'=ax_{2}$ & $\left(x_{1}^{*},0,-\frac{F_{2}}{qx_{1}^{*}-a}\right)$\tabularnewline
\hline 
$5$ & $F_{2},\epsilon_{3}$ & $x_{1}'=px_{2}x_{3}\textrm{, }x_{2}'=qx_{3}x_{1}-ax_{3}+F_{2}\textrm{, }x_{3}'=ax_{2}-\epsilon_{3}x_{3}$ & None\tabularnewline
\hline 
$6$ & $F_{3},\epsilon_{1}$ & $x_{1}'=px_{2}x_{3}-\epsilon_{1}x_{1}\textrm{, }x_{2}'=qx_{3}x_{1}-ax_{3}\textrm{, }x_{3}'=ax_{2}+F_{3}$ & $\left(0,-\frac{F_{3}}{a},0\right)$ and $\left(\frac{a}{q},-\frac{F_{3}}{a},-\frac{\epsilon_{1}a^{2}}{pqF_{3}}\right)$\tabularnewline
\hline 
$7$ & $F_{3},\epsilon_{2}$ & $x_{1}'=px_{2}x_{3}\textrm{, }x_{2}'=qx_{3}x_{1}-ax_{3}-\epsilon_{2}x_{2}\textrm{, }x_{3}'=ax_{2}+F_{3}$ & None\tabularnewline
\hline 
$8$ & $F_{3},\epsilon_{3}$ & $x_{1}'=px_{2}x_{3}\textrm{, }x_{2}'=qx_{3}x_{1}-ax_{3}\textrm{, }x_{3}'=ax_{2}+F_{3}-\epsilon_{3}x_{3}$ & $\left(\frac{a}{q},0,\frac{F_{3}}{\epsilon_{3}}\right)$ and $\left(x_{1}^{*},-\frac{F_{3}}{a},0\right)$\tabularnewline
\hline 
\end{tabular}

\end{landscape}

\pagebreak{}

More generally, this subclass with F\&D
\begin{align}
x_{1}' & =px_{2}x_{3}-\epsilon_{1}x_{1}+F_{1}\nonumber \\
x_{2}' & =qx_{3}x_{1}-ax_{3}-\epsilon_{2}x_{2}+F_{2}\nonumber \\
x_{3}' & =ax_{2}-\epsilon_{3}x_{3}+F_{3}\label{eq:p10}
\end{align}
has fixed points $\left(x_{1}^{*},x_{2}^{*},x_{3}^{*}\right)$ as
follows: from the third equation, $x_{2}^{*}=-\left(F_{3}-\epsilon_{3}x_{3}^{*}\right)/a$
and from the second equation $qx_{1}^{*}x_{3}^{*}=ax_{3}^{*}+\epsilon_{2}x_{2}^{*}-F_{2}$.
Together these equations must obey consistency 
\begin{equation}
\left(qx_{1}^{*}-\frac{\epsilon_{2}\epsilon_{3}}{a}-a\right)x_{3}^{*}=-\frac{F_{3}\epsilon_{2}}{a}-F_{2}\label{eq:p11}
\end{equation}
so that when $x_{3}^{*}=0$ the right hand side of Eq. (\ref{eq:p11})
must also vanish. A second consistency condition is furnished by the
first equation 
\begin{equation}
px_{2}^{*}x_{3}^{*}=-F_{1}+\epsilon_{1}x_{1}^{*}\label{eq:p12}
\end{equation}
so that zero $x_{3}^{*}$ in conjunction with nonzero $F_{1}$ also
entails nonzero $\epsilon_{1}$. These consistency conditions are
met by the cases illustrated in Table 1. Using these relations to
eliminate $x_{1}^{*},x_{2}^{*}$ from the first equation we obtain
a cubic in $x_{3}^{*}$ 
\begin{align}
\gamma_{3}x_{3}^{*3}+\gamma_{2}x_{3}^{*2}+\gamma_{1}x_{3}^{*}+\gamma_{0} & =0\label{eq:p13}
\end{align}
where 
\begin{align}
\gamma_{3} & =\epsilon_{3}\frac{p}{a},\nonumber \\
\gamma_{2} & =-F_{3}\frac{p}{a},\nonumber \\
\gamma_{1} & =F_{1}-\epsilon_{1}\frac{a}{q}-\epsilon_{1}\epsilon_{2}\epsilon_{3}\frac{1}{aq},\nonumber \\
\gamma_{0} & =\frac{F_{2}\epsilon_{1}}{q}+\frac{F_{3}\epsilon_{1}\epsilon_{2}}{aq}.
\end{align}
The number and arrangement of fixed points depends on the number of
real roots of Eq. (\ref{eq:p13}), subject to the two consistency
conditions. Recall that the MCM has nonzero $\left(F_{1},\epsilon_{3}\right)$.
The resulting expression becomes
\begin{equation}
\epsilon_{3}\frac{p}{a}x_{3}^{*3}+F_{1}x_{3}^{*}=0
\end{equation}
yielding two roots satisfying $x_{3}^{*2}=-\frac{aF_{1}}{p\epsilon_{3}}$
as found above, since the third root $x_{3}^{*}=0$ does not meet
consistency condition in Eq. (\ref{eq:p12}). This is a simple model
with F\&D yielding two fixed points with positive/negative $x_{3}^{*}$.

Another model with two fixed points has nonzero $\left(F_{3},\epsilon_{3}\right)$:
we obtain $\epsilon_{3}\frac{p}{a}x_{3}^{*3}-F_{3}\frac{p}{a}x_{3}^{*2}=0$,
giving a fixed point $x_{3}^{*}=0$ and $x_{3}^{*}=F_{3}/\epsilon_{3}$.
What precludes it becoming chaotic? Owing to the same placement of
dissipation the model has the same underlying Jacobian and resulting
characteristic polynomial as Eq. (\ref{eq:p8}), and consequently
the same discriminant as Eq. (\ref{eq:p9}). At $x_{3}^{*}=0$ clearly
$\Delta=0$ and the equation has a double eigenvalue, precluding complex
roots. Thus a Hopf bifurcation cannot arise in this case. In summary,
the MCM $\left(F_{1},\epsilon_{3}\right)$ provides the simplest model
with a pair of fixed points, with nonzero $x_{3}^{*}$, which can
undergo a Hopf bifurcation. 

Such fixed points play an important role in the Lorenz model (\citet{Sparrow1982}).
On the route to chaos, as the forcing is increased, trajectories spiral
around these fixed points at an increasing distance (\citet{Kaplan1979}).
There is also an important difference with the Lorenz model, which
will be examined later: the absence of a third fixed point at the
origin for the MCM precludes homoclinic orbits that are important
in the Lorenz model. Such a comparison of subclass $1$ that closely
parallels the Lorenz model (\citet{Gluhovsky1999}) identifies the
critical condition for chaos in these models: the appearance of a
pair of fixed points, with positive/negative $x_{3}^{*}$, yielding
opposing spirals when projected onto the $x_{1}-x_{2}$ plane.\footnote{The role of the pair of fixed points with opposite signs of $x_{3}^{*}$
in creating opposing spirals when projected onto $x_{1}-x_{2}$ is
clear from the Jacobian in Eq. (\ref{eq:p7}). } 

\subsection{Subclass 2 with $r=0,c=0$}

Here the conservative core 
\begin{align}
x_{1}' & =px_{2}x_{3}+bx_{3}\nonumber \\
x_{2}' & =qx_{3}x_{1}-ax_{3}\nonumber \\
x_{3}' & =ax_{2}-bx_{1}\label{eq:p16}
\end{align}
has a single constant of motion (\citet{Seshadri2023}) in the general
situation with $p+q\neq0$, and it is nonzero $b$ that breaks symmetry
$R_{x_{1}}\left(\pi\right)$. The MCMs $\left(F_{1},\epsilon_{3}\right)$
and $\left(F_{2},\epsilon_{3}\right)$, which both appear even with
$p+q+r=0$, are analogous, so we consider $\left(F_{1},\epsilon_{3}\right)$
\begin{align}
x_{1}' & =px_{2}x_{3}+bx_{3}+F_{1}\nonumber \\
x_{2}' & =qx_{3}x_{1}-ax_{3}\nonumber \\
x_{3}' & =ax_{2}-bx_{1}-\epsilon_{3}x_{3}\label{eq:p17}
\end{align}
whose fixed points are given by $\left(\frac{a}{q},\frac{b}{q}+\frac{\epsilon_{3}}{a}x_{3}^{*},x_{3}^{*}\right)$
where $x_{3}^{*}$ solves the quadratic equation $\epsilon_{3}pqx_{3}^{*2}+ab\left(p+q\right)x_{3}^{*}+aqF_{1}=0$.
The Jacobian evaluated at the fixed point
\begin{equation}
\mathrm{D}_{\mathrm{f}}=\left[\begin{array}{ccc}
0 & px_{3}^{*} & px_{2}^{*}+b\\
qx_{3}^{*} & 0 & 0\\
-b & a & -\epsilon_{3}
\end{array}\right]\label{eq:p18}
\end{equation}
has characteristic polynomial
\begin{equation}
\lambda^{3}+\epsilon_{3}\lambda^{2}-\left(pqx_{3}^{*2}-b\right)\lambda-\epsilon_{3}pqx_{3}^{*2}+aqF_{1}=0,\label{eq:p22}
\end{equation}

which depends on $x_{3}^{*2}$. For $p+q\neq0$ the two fixed points
do not have identical stability. As with subclass $1$, for $F_{1}$
sufficiently large the complex roots have positive real part so this
pair of fixed points repel the flow in the neighbourhood. The other
MCM $\left(F_{3},\epsilon_{2},\epsilon_{3}\right)$
\begin{align}
x_{1}' & =px_{2}x_{3}+bx_{3}\nonumber \\
x_{2}' & =qx_{3}x_{1}-ax_{3}-\epsilon_{2}x_{2}\nonumber \\
x_{3}' & =ax_{2}-bx_{1}-\epsilon_{3}x_{3}+F_{3}\label{eq:p20}
\end{align}
does not appear in the presence of energy conservation, and has origins
that are not rooted in the aforementioned property of fixed points
(as shown below). With the energy conservation constraint, let us
consider alternate placements of forcing and dissipation terms (Table
3). Most of the alternate placements yield only a single fixed point,
whereas those with two fixed points have one of them lying on the
$x_{1}-x_{2}$ plane. This does not favor the pair of spirals described
in the context of subclass $1$. Only the MCMs $\left(F_{1},\epsilon_{3}\right)$
and $\left(F_{2},\epsilon_{3}\right)$ lead to a pair of fixed points
with opposite signs of $x_{3}^{*}$, and these are the ones leading
to the pair of opposing spirals (see Eq. (\ref{eq:p18})). It is hardly
surprising that these are the MCMs also observed with $p+q+r=0$ (SI
Figs. 25-27; Table 1). 

\pagebreak{}

\begin{landscape}

Table 3: Fixed points for various other placements of one forcing
and dissipation term in Subclass 2. Here we have taken $p+q+r=0$. 

\begin{tabular}{|c|c|c|c|}
\hline 
S. No & Model & Equations & Fixed Points\tabularnewline
\hline 
$1$ & $F_{1},\epsilon_{1}$ & $x_{1}'=px_{2}x_{3}+bx_{3}+F_{1}-\epsilon_{1}x_{1}\textrm{, }x_{2}'=qx_{3}x_{1}-ax_{3}\textrm{, }x_{3}'=ax_{2}-bx_{1}$ & $\left(\frac{F_{1}}{\epsilon_{1}},\frac{bF_{1}}{a\epsilon_{1}},0\right)$\tabularnewline
\hline 
$2$ & $F_{1},\epsilon_{2}$ & $x_{1}'=px_{2}x_{3}+bx_{3}+F_{1}\textrm{, }x_{2}'=qx_{3}x_{1}-ax_{3}-\epsilon_{2}x_{2}\textrm{, }x_{3}'=ax_{2}-bx_{1}$ & $\left(\frac{a}{b}x_{2}^{*},x_{2}^{*}=-\frac{bx_{3}^{*}+F_{1}}{px_{3}^{*}},x_{3}^{*}=\frac{F_{1}\epsilon_{2}b}{F_{1}aq-\epsilon_{2}b}\right)$\tabularnewline
\hline 
$3$ & $F_{2},\epsilon_{1}$ & $x_{1}'=px_{2}x_{3}+bx_{3}-\epsilon_{1}x_{1}\textrm{, }x_{2}'=qx_{3}x_{1}-ax_{3}+F_{2}\textrm{, }x_{3}'=ax_{2}-bx_{1}$ & $\left(\frac{a}{b}x_{2}^{*},x_{2}^{*}=-\frac{bx_{3}^{*}}{px_{3}^{*}-\epsilon_{1}\frac{a}{b}},x_{3}^{*}=\frac{F_{2}\epsilon_{1}a}{F_{2}b-\epsilon_{1}a^{2}}\right)$\tabularnewline
\hline 
$4$ & $F_{2},\epsilon_{2}$ & $x_{1}'=px_{2}x_{3}+bx_{3}\textrm{, }x_{2}'=qx_{3}x_{1}-ax_{3}+F_{2}-\epsilon_{2}x_{2}\textrm{, }x_{3}'=ax_{2}-bx_{1}$ & $\left(\frac{a}{b}\frac{F_{2}}{\epsilon_{2}},\frac{F_{2}}{\epsilon_{2}},0\right)$\tabularnewline
\hline 
$5$ & $F_{3},\epsilon_{1}$ & $x_{1}'=px_{2}x_{3}+bx_{3}-\epsilon_{1}x_{1}\textrm{, }x_{2}'=qx_{3}x_{1}-ax_{3}\textrm{, }x_{3}'=ax_{2}-bx_{1}+F_{3}$ & $\left(0,-\frac{F_{3}}{a},0\right)$ and $\left(\frac{a}{q},-\frac{F_{3}-\frac{ba}{q}}{a},-\frac{\epsilon_{1}a^{2}}{pqF_{3}}\right)$\tabularnewline
\hline 
$6$ & $F_{3},\epsilon_{2}$ & $x_{1}'=px_{2}x_{3}+bx_{3}\textrm{, }x_{2}'=qx_{3}x_{1}-ax_{3}-\epsilon_{2}x_{2}\textrm{, }x_{3}'=ax_{2}-bx_{1}+F_{3}$ & $\left(\frac{F_{3}}{b},0,0\right)$ and $\left(\frac{-\frac{ab}{p}+F_{3}}{b},-\frac{b}{p},-\frac{\epsilon_{2}b^{2}}{pqF_{3}}\right)$\tabularnewline
\hline 
$7$ & $F_{3},\epsilon_{3}$ & $x_{1}'=px_{2}x_{3}+bx_{3}\textrm{, }x_{2}'=qx_{3}x_{1}-ax_{3}\textrm{, }x_{3}'=ax_{2}-bx_{1}+F_{3}-\epsilon_{3}x_{3}$ & $\left(x_{1}^{*},-\frac{F_{3}-bx_{1}^{*}}{a},0\right)$ and $\left(\frac{a}{q},-\frac{b}{p},\frac{F_{3}}{\epsilon_{3}}\right)$ \tabularnewline
\hline 
\end{tabular}

\end{landscape}

\pagebreak{}

As in subclass $1$, the possible solutions for $x_{3}^{*}$ for general
placement of forcing and dissipation follows cubic
\begin{align}
\gamma_{3}x_{3}^{*3}+\gamma_{2}x_{3}^{*2}+\gamma_{1}x_{3}^{*}+\gamma_{0} & =0\label{eq:p21}
\end{align}
where 
\begin{align}
\gamma_{3} & =\epsilon_{3}\frac{p}{a},\nonumber \\
\gamma_{2} & =-F_{3}\frac{p}{a}+b\left(1+\frac{p}{q}\right),\nonumber \\
\gamma_{1} & =F_{1}-F_{2}b\frac{p}{aq}-\epsilon_{1}\frac{a}{q}-\epsilon_{1}\epsilon_{2}\epsilon_{3}\frac{1}{aq}-\epsilon_{2}\frac{b^{2}}{aq},\nonumber \\
\gamma_{0} & =\frac{F_{2}\epsilon_{1}}{q}+\frac{F_{3}\epsilon_{1}\epsilon_{2}}{aq}-F_{1}\epsilon_{2}\frac{b}{aq},
\end{align}
which reduces to Eq. (\ref{eq:p13}) for $b=0$. 

The MCMs generally describe the simplest ways to obtain a pair of
fixed points with opposite signs of $x_{3}^{*}$, and nonzero $\epsilon_{3}$
along with suitable placement of forcing supports this. Nonzero $\left(F_{1},\epsilon_{3}\right)$
and $\left(F_{2},\epsilon_{3}\right)$ both yield $\gamma_{3}x_{3}^{*3}+\gamma_{1}x_{3}^{*}=0$
for the condition $p+q+r=0$, giving a pair of fixed points having
the aforementioned property. The other MCM $\left(F_{3},\epsilon_{2},\epsilon_{3}\right)$
gives $\gamma_{3}x_{3}^{*3}+\gamma_{2}x_{3}^{*2}+\gamma_{1}x_{3}^{*}=0$:
it does not fall into the same pattern as the above cases,\footnote{Here, it is easily seen that $\gamma_{1}$ and $\gamma_{3}$ are of
the same sign, precluding roots of opposite sign.} as nonzero $F_{3}$ precludes a second fixed point across the plane,
and this has consequences described later. It is notable that this
case does not occur with $p+q+r=0$ (SI Figs. 25-27; Table 1). 

If only two successive coefficients are nonzero, no matter their degree,
this would give at most one non-trivial $x_{3}^{*}$. This occurs
with $\left(F_{3},\epsilon_{3}\right)$, giving $\gamma_{3}x_{3}^{*3}+\gamma_{2}x_{2}^{*2}=0$.
How about nonzero $\text{\ensuremath{\left(F_{3},\epsilon_{1},\epsilon_{2}\right)}}$?
This gives $-F_{3}\frac{p}{a}x_{3}^{*2}-\left(\epsilon_{1}\frac{a}{q}+\epsilon_{2}\frac{b^{2}}{aq}\right)+\frac{F_{3}\epsilon_{1}\epsilon_{2}}{aq}=0$.
In order for the roots to have opposite sign, the constant and quadratic
coefficients must be of opposite sign, i.e. $\left(-F_{3}\frac{p}{a}x_{3}^{*2}\right)\left(\frac{F_{3}\epsilon_{1}\epsilon_{2}}{aq}\right)<0$.
However this is not possible since $pq<0$. The case $\text{\ensuremath{\left(F_{3},F_{2},\epsilon_{1}\right)}}$
leads to the same difficulty, as the roots are given by $-F_{3}\frac{p}{a}x_{3}^{*2}-\left(F_{2}b\frac{p}{aq}+\epsilon_{1}\frac{a}{q}\right)+\frac{F_{2}\epsilon_{1}}{q}=0$,
with quadratic and constant coefficients of the same sign (from our
premise of $F_{3}=F_{2}$). Since the second and third equation are
interchangeable, we can also rule out chaos in $\text{\ensuremath{\left(F_{3},F_{1},\epsilon_{2}\right)}}$. 

In summary, we have shown that the pair of fixed points with opposite
signs of $x_{3}^{*}$ cannot be achieved with a quadratic equation
for the roots, and nonzero $\gamma_{3}=\epsilon_{3}\frac{p}{a}$ is
required. This accounts for the presence of dissipation in the linear
equation, in all chaotic models. The cases $\left(F_{1},\epsilon_{3}\right)$
and $\left(F_{2},\epsilon_{3}\right)$ each yield pairs of fixed points
on opposite sides of this plane, and it is no coincidence that these
MCMs appear even with $p+q+r=0$. In contrast, the MCM $\left(F_{3},\epsilon_{2},\epsilon_{3}\right)$,
which appears only when the gyrostat core does not conserve energy,
is not tied to spiraling orbits around this fixed point pair. 

\subsection{Subclass 3 with $r=0,b=0$ and Subclass 4 with $r=0$}

For subclass $3$ the equations 
\begin{align}
x_{1}' & =px_{2}x_{3}-cx_{2}-\epsilon_{1}x_{1}+F_{1}\nonumber \\
x_{2}' & =qx_{3}x_{1}-ax_{3}+cx_{1}-\epsilon_{2}x_{2}+F_{2}\nonumber \\
x_{3}' & =ax_{2}-\epsilon_{3}x_{3}+F_{3}\label{eq:p23}
\end{align}
give fixed points $x_{2}^{*}=-\left(F_{3}-\epsilon_{3}x_{3}^{*}\right)/a$,
and $x_{1}^{*}=\left(ax_{3}^{*}+\epsilon_{2}x_{2}^{*}-F_{2}\right)/\left(qx_{3}^{*}+c\right)$,
resulting in a cubic equation $\gamma_{3}x_{3}^{*3}+\gamma_{2}x_{3}^{*2}+\gamma_{1}x_{3}^{*}+\gamma_{0}=0$
as before, with coefficients
\begin{align}
\gamma_{3} & =\epsilon_{3}\frac{p}{a},\nonumber \\
\gamma_{2} & =-F_{3}\frac{p}{a}+\epsilon_{3}\frac{c}{a}\left(\frac{p}{q}-1\right),\nonumber \\
\gamma_{1} & =F_{1}-F_{3}\frac{c}{a}\left(\frac{p}{q}-1\right)-\epsilon_{1}\frac{a}{q}-\epsilon_{1}\epsilon_{2}\epsilon_{3}\frac{1}{aq}-\epsilon_{3}\frac{c^{2}}{aq},\nonumber \\
\gamma_{0} & =F_{1}\frac{c}{q}+\frac{F_{2}\epsilon_{1}}{q}+F_{3}\left(\frac{c^{2}}{aq}+\frac{\epsilon_{1}\epsilon_{2}}{aq}\right).
\end{align}
As in the previous case, chaos requires nonzero $\gamma_{3}=\epsilon_{3}\frac{p}{a}$,
because the pair with positive/negative $x_{3}^{*}$ cannot be realized
with a quadratic equation alone. For e.g., nonzero $\left(F_{3},\epsilon_{1}\right)$,
$\left(F_{3},\epsilon_{2}\right),\left(F_{3},\epsilon_{1},\epsilon_{2}\right)$,
etc. give quadratic and constant coefficients of the same sign.\footnote{With nonzero $\epsilon_{3}$ we obtain nonzero $\gamma_{3},\gamma_{2},\gamma_{1}$;
however the unforced equation alone is not up to the task, since $\epsilon_{3}\frac{p}{a}$
and $-\epsilon_{3}\frac{c^{2}}{aq}$ have the same sign. Moreover,
there are simpler grounds for precluding unforced chaos, as discussed
in Section 4.4. } 

In the presence of forcing, the minimal chaotic case $\left(F_{1},\epsilon_{3}\right)$
gives a cubic equation with all coefficients being nonzero, with many
settings of the parameters allowing the desired configuration of the
fixed-point pair. 

In case of $\left(F_{2},\epsilon_{3}\right)$ we obtain $\gamma_{3}x_{3}^{*3}+\gamma_{2}x_{3}^{*2}+\gamma_{1}x_{3}^{*}=0$,
with $\gamma_{1},\gamma_{3}$ of same sign. Thus the necessary configuration
of nontrivial fixed points is not available, indicating other processes
at work. It is thus hardly surprising that this case is not a MCM
in case $p+q+r=0$ (SI Figs. 28-30; Table 1). The same goes for $\left(F_{3},\epsilon_{3}\right)$,
which requires nonconservation of energy in the gyrostat core to give
rise to F\&D chaos.

Finally we consider Subclass $4$, for which the F\&D equations
\begin{align}
x_{1}' & =px_{2}x_{3}+bx_{3}-cx_{2}-\epsilon_{1}x_{1}+F_{1}\nonumber \\
x_{2}' & =qx_{3}x_{1}-ax_{3}+cx_{1}-\epsilon_{2}x_{2}+F_{2}\nonumber \\
x_{3}' & =ax_{2}-bx_{1}-\epsilon_{3}x_{3}+F_{3}\label{eq:p25}
\end{align}
have fixed points $x_{2}^{*}=-\left(F_{3}-bx_{1}^{*}-\epsilon_{3}x_{3}^{*}\right)/a$,
and $x_{1}^{*}=\left(ax_{3}^{*}+\epsilon_{2}x_{2}^{*}-F_{2}\right)/\left(qx_{3}^{*}+c\right)$,
resulting in cubic $\gamma_{3}x_{3}^{*3}+\gamma_{2}x_{3}^{*2}+\gamma_{1}x_{3}^{*}+\gamma_{0}=0$
having coefficients
\begin{align}
\gamma_{3} & =\epsilon_{3}\frac{p}{a},\nonumber \\
\gamma_{2} & =-F_{3}\frac{p}{a}+\epsilon_{3}\frac{c}{a}\left(\frac{p}{q}-1\right)+b\left(1+\frac{px_{1}^{*}}{a}\right),\nonumber \\
\gamma_{1} & =F_{1}-F_{3}\frac{c}{a}\left(\frac{p}{q}-1\right)-\epsilon_{1}\frac{a}{q}-\epsilon_{1}\epsilon_{2}\epsilon_{3}\frac{1}{aq}-\epsilon_{3}\frac{c^{2}}{aq}+\frac{bc}{q}+\frac{bcx_{1}^{*}}{a}\left(\frac{p}{q}-1\right)\nonumber \\
\gamma_{0} & =F_{1}\frac{c}{q}+\frac{F_{2}\epsilon_{1}}{q}+F_{3}\left(\frac{c^{2}}{aq}+\frac{\epsilon_{1}\epsilon_{2}}{aq}\right)-\frac{bx_{1}^{*}\left(c^{2}+\epsilon_{1}\epsilon_{2}\right)}{aq},\label{eq:p26}
\end{align}
which closely parallels Subclass 3, with additional contributions
from nonzero $b$. Even though this is cubic the structure is really
different, since it must be solved simultaneously with the other equations
for $x_{1}^{*}$ and $x_{2}^{*}$, and yet the close resemblance points
to how the same MCMs are attained (Table 1), once $\epsilon_{3}$
is nonzero. There are subtle differences, however, which make not
only $\left(F_{1},\epsilon_{3}\right)$ , but also $\left(F_{2},\epsilon_{3}\right)$
as MCMs, in the presence of an energy conserving core (SI Figs. 31-33;
Table 1). The presence of $\left(F_{2},\epsilon_{3}\right)$ can be
expected from the similarity of the equations for $dx_{1}/dt$ and
$dx_{2}/dt$.

\subsection{Role of fixed points}

For subclass $1$, Figures 5-6 (and SI Figs. 19-20) illustrate the
orbits for the models $\left(F_{1},\epsilon_{3}\right)$, and those
for $\left(F_{1},\epsilon_{2},\epsilon_{3}\right)$. The latter closely
resembles the model of \citet{Lorenz1963}. Chaos in this model requires
spiraling orbits surrounding the pair of fixed points (\citet{Sparrow1982}).
An important difference is the absence of the third fixed point at
the origin, whose saddle structure permits stable homoclinic orbits
for intermediate values of forcing. The MCM cannot have such homoclinic
orbits, as it possesses only the two fixed points. It is evident from
the calculations here that chaos only requires this pair with opposite
signs of $x_{3}^{*}$, making the discussion of chaos with constant
forcing terms clearly germane to such models despite the differences.
Moreover, this MCM appears even in the absence of energy conservation
in the gyrostat core. The progression of the three-dimensional orbits
with increasing forcing (Fig. 5) illustrates the dynamics that are
relevant, with this pair of fixed points. The $x_{1}$ axis is invariant
in this model, and the effects are seen in the long stretches of time
for which orbits can remain nearby. 

Similar results occur for the model with $\left(F_{1},\epsilon_{3}\right)$
in subclass 2 (SI Fig. 21), subclass 3 (Figure 7), as well as subclass
$4$. Here, there is a pair of fixed points with opposite $x_{3}^{*}$
as before. Analogous plots can be made for $\left(F_{2},\epsilon_{3}\right)$
in subclasses $2$ and $4$. As can be seen from inspecting Table
1, these are the cases where the cubic equations naturally yield at
least a pair of fixed points with opposite signs of $x_{3}^{*}$.
The Jacobian of the vector fields acquires a skew-symmetric contribution
from nonzero $x_{3}^{*}$ (this pressupposes opposite signs of $p$
and $q)$, with opposite orientations in each hemisphere, influencing
the characteristic spirals surrounding the fixed-point pair that alternate
between the hemispheres. This occurs even with energy conservation
of the gyrostatic core, and it is therefore not surprising that these
cases also constitute MCMs for the case with $p+q+r=0$ (Table 1).
The effects of increasing the external forcing for these cases are
shown in SI Figs. 34-35 (subclass $2$), SI Fig. 36 (subclass $3)$,
and SI Fig. 39 (subclass $4$). 

The other MCMs found for the case of $p+q+r\neq0$ do not fit this
pattern. In particular, $\left(F_{3},\epsilon_{2},\epsilon_{3}\right)$
in subclass 2, $\left(F_{2},\epsilon_{3}\right)$ and $\left(F_{3},\epsilon_{3}\right)$
in subclass $3$ (SI Figs. 37-38), and $\left(F_{3},\epsilon_{3}\right)$
in subclass $4$ do not have a pair of fixed points with opposite
$x_{3}^{*}$. In fact, where the forcing occurs on the $x_{3}$ mode,
one cannot expect steady states of both signs and the resulting attractor
looks quite different (SI Fig. 41). This is not limited to coincident
forcing and dissipation, with a similar phenomenon occurring also
with $\left(F_{2},\epsilon_{3}\right)$ of subclass $3$ (SI Fig.
37). The attractors here (e.g., SI Figs. 37-38) are very different
from those of the Lorenz model, which has in its core $p+q=0$ ($r=0$
by the choice of streamfunction), but can resemble the attractors
found in the gyrostatic core when there are no constants of motion
(\citet{Seshadri2023}). Sometimes in these cases, for example $\left(F_{3},\epsilon_{2},\epsilon_{3}\right)$
of subclass 2, and $\left(F_{3},\epsilon_{3}\right)$ in subclasses
$3-4$, forcing and dissipation appears in the same equation, but
this is not always the case, for example $\left(F_{2},\epsilon_{3}\right)$
in subclass $3$. Despite their differences, they all depend on the
lack of energy conservation in the gyrostat core, and the absence
of the aforementioned pair of fixed points. It is likely that chaos
in these cases is tied to the loss of invariants in the gyrostat core
when $p+q+r\neq0$. With $p+q+r\neq0$, subclasses 2 and 3 have $1$
invariant owing to the presence of two linear feedbacks (\citet{Seshadri2023}),
and the inclusion of forcing might play a role analogous to the third
linear feedback that creates conditions for Hamiltonian chaos in the
gyrostatic core. Without energy conservation, subclass $4$ has no
invariants in the gyrostatic core and its MCM $\left(F_{1},\epsilon_{3}\right)$
can encounter very different dynamics on the route to forced and dissipative
chaos as compared to the correspondng MCM of subclass $1$ (SI Fig.
39). 

Furthermore, where they are important as in cases $\left(F_{1},\epsilon_{3}\right)$
and $\left(F_{1},\epsilon_{2},\epsilon_{3}\right)$ of subclass $1$,
fixed points can play various roles in pathways to chaos, with not
all of the resulting fixed points being germane. The Lorenz model
has three fixed points, whereas the variant of subclass $1$ closest
to it, i.e. $\left(F_{1},\epsilon_{2},\epsilon_{3}\right)$, has only
two. Yet the dynamics resemble each other and crucially involve the
pair of fixed points with nonzero $x_{3}^{*}$. Once these become
unstable, they lead to spiraling orbits eventually to the Lorenz attractor. 

To take up the comparison further, we consider the case $\left(F_{1},\epsilon_{2},\epsilon_{3}\right)$
of subclass $1$, without the fixed point at the origin but having
an invariant set in the line $x_{1}$. The equations for $\dot{x}_{2}$
and $\dot{x}_{3}$, being linear in $x_{2}$ and $x_{3}$, remain
unchanged under the transformation $\left(x_{1},x_{2},x_{3}\right)\rightarrow\left(x_{1},-x_{2},-x_{3}\right)$,
as does the equation for $\dot{x}_{1}$ because it depends only on
the product $x_{2}x_{3}$. When forcing is applied to the first equation,
and dissipation to the second and third, as in $\left(F_{1},\epsilon_{3}\right)$
and $\left(F_{1},\epsilon_{2},\epsilon_{3}\right)$, the line with
$x_{2}=0,x_{3}=0$ remains an invariant set. However there is no fixed
point the origin. Near this invariant set, the coordinate $x_{1}$
evolves as $\dot{x_{1}}\approx F_{1}$, growing if $F_{1}>0$. Let
us consider the resulting dynamics near $x_{2}=0,x_{3}=0$ by examining
the transverse stability (transverse to this invariant set), given
by the linearized equations
\begin{equation}
\left\{ \begin{array}{c}
\delta\dot{x}_{2}\\
\delta\dot{x}_{3}
\end{array}\right\} =\left[\begin{array}{cc}
-\epsilon_{2} & qx_{1}-a\\
a & -\epsilon_{3}
\end{array}\right]\left\{ \begin{array}{c}
\delta x_{2}\\
\delta x_{3}
\end{array}\right\} 
\end{equation}
with the above matrix having characteristic equation $\lambda^{2}+\left(\epsilon_{2}+\epsilon_{3}\right)\lambda+\epsilon_{2}\epsilon_{3}-a\left(qx_{1}-a\right)=0$,
with eigenvalues $\lambda=-\epsilon\pm\sqrt{a\left(qx_{1}-a\right)}$,
where we have used $\epsilon_{2}=\epsilon_{3}=\epsilon$. When $a\left(qx_{1}-a\right)<0$,
the eigenvalues are complex conjugate leading to a stable spiral towards
the invariant set. For $0\leq a\left(qx_{1}-a\right)<\epsilon^{2}$,
both eigenvalues are real and negative with local dynamics resembling
a sink. However for $\epsilon^{2}<a\left(qx_{1}-a\right)$ points
on the invariant set behave as a saddle, and nearby trajectories are
repelled. 

Similar results are obtained for the MCM $\left(F_{1},\epsilon_{3}\right)$,
with characteristic polynomial $\lambda^{2}+\epsilon_{3}\lambda-a\left(qx_{1}-a\right)=0$
and eigenvalues
\begin{equation}
\lambda=-\frac{\epsilon_{3}}{2}\pm\sqrt{\frac{\epsilon_{3}^{2}}{4}+a\left(qx_{1}-a\right)}.\label{eq:p28}
\end{equation}
Here points are repelled from the invariant set once $0<a\left(qx_{1}-a\right)$,
and the transition in the neighbourhood of the invariant set remains
the same. These dynamics are shown in Figure 8, for initial condition
$\left(1,0.2,0.2\right)$, fixed $F_{1}=0.021$ so that $x_{1}$ increases
near the invariant set, and $aq>0$ so that the invariant set eventually
becomes unstable. Arrows indicate the direction of the vector field
along the orbit. Near the invariant set the increase of $x_{1}$ is
roughly linear in time, and as it grows the changing stability can
be observed. Initially the invariant set is transverse stable, with
oscillatory dynamics evident from the gaps in the time-series of $x_{2}$
and $x_{3}$ (where these are negative) when plotted on the logarithmic
scale. The logarithmic plots show that $x_{2}$ and $x_{3}$ approach
the invariant set but never reach it, before the set becomes unstable
as $x_{1}$ grows. There is no fixed point along this invariant set
and thus no homoclinic orbits.

Such dynamics has similarities with the Lorenz model
\begin{align}
X' & =-\sigma X+\sigma Y\nonumber \\
Y' & =-XZ+rX-Y\nonumber \\
Z' & =XY-bZ\label{eq:p29}
\end{align}
where $X$ is related to the amplitude of the first mode of the streamfunction,
and $Y,Z$ are related to modes of temperature evolution. Analogous
to the above cases there is an invariant set given by $X=0,Y=0$,
with flow in the neighbourhood contracting towards the origin (a fixed
point) following $Z'=-bZ$. The corresponding transverse stability
is described by
\begin{equation}
\left\{ \begin{array}{c}
\delta\dot{X}\\
\delta\dot{Y}
\end{array}\right\} =\left[\begin{array}{cc}
-\sigma & \sigma\\
-Z+r & -1
\end{array}\right]\left\{ \begin{array}{c}
\delta X\\
\delta Y
\end{array}\right\} 
\end{equation}
with the above matrix having characteristic equation $\lambda^{2}+\left(1+\sigma\right)\lambda+\sigma\left(1-r+Z\right)=0$,
with eigenvalues $\lambda=-\left(1+\sigma\right)\pm\sqrt{\left(1+\sigma\right)^{2}-4\sigma\left(1-r+Z\right)}$,
which behaves as a saddle whenever $\left(1-r+Z\right)<0$, or $r>1$
which is the well known condition for instability of the fixed point
at the origin. Thus, there are close parallels with subclass $1$,
despite the additional fixed point at the origin. In each case, it
is the pair of repelling fixed points away from the origin that circumscribes
the possibility for chaos.

\section{Discussion}

This paper is based on large ensemble simulations, to identify the
simplest chaotic models derived from the Volterra gyrostat. This revealed
that minimal chaotic models exist, involving proper subsets of the
forcing and dissipation terms present in each of the chaotic cases
that are found. The existence of such MCMs is explicable through common
conditions for chaos in these models. 

Our analysis showed that the forcing and dissipation in these MCMs
play very specific roles. Dissipation induces a stable direction in
the flow. The skew-symmetric property of the nonlinear coupling between
$x_{1}$ and $x_{2}$ in each of the subclasses investigated here
(owing to opposite signs of the quadratic terms) makes the linear
mode $x_{3}$ a natural candidate for governing the stable direction,
and hence where dissipation appears. Instead, placing dissipation
in $x_{1}$ or $x_{2}$ alone alters the arrangement of fixed points.
With dissipation but in the absence of forcing, the energy is decreasing:
defining $E=\frac{1}{2}\left(x_{1}^{2}+x_{2}^{2}+x_{3}^{3}\right)$
the model of subclass $1$
\begin{align}
x_{1}' & =px_{2}x_{3}-cx_{2}+F_{1}\nonumber \\
x_{2}' & =qx_{3}x_{1}-ax_{3}+cx_{1}+F_{2}\nonumber \\
x_{3}' & =ax_{2}-\epsilon_{3}x_{3}+F_{3}\label{eq:p31}
\end{align}
has $E'=-\epsilon_{3}x_{3}^{2}\leq0$ in the absence of any forcing,
and chaos cannot ensue. Forcing shifts the attractor along the corresponding
axis, for example nonzero $F_{3}$ makes $E'=-\epsilon_{3}x_{3}^{2}+F_{3}x_{3}$
and for chaos to appear $F_{3}x_{3}$ must be positive most of the
time. Such an attractor cannot be distributed on either side of the
$x_{1}-x_{2}$ plane, and the coincidence of forcing and dissipation
precludes the fixed point pair associated with the more characteristic
MCMs. In contrast, nonzero $F_{1}$ makes $E'=-\epsilon_{3}x_{3}^{2}+F_{1}x_{1}$
and fixed points on either side of the $x_{1}-x_{2}$ plane are readily
obtained, giving rise to the more typical structures resembling the
Lorenz attractor. Inspection of the Jacobian of the two fixed points
also indicates how these have opposite orientations in their surrounding
flows, which is crucial for the chaotic set that follows. We caution
that this discussion presents an oversimplified intuition, and consideration
of the case $\left(F_{2},\epsilon_{3}\right)$ of subclass $3$ refutes
overgeneralization prior to analysis. 

In summary, when the gyrostat equations have two nonlinear terms,
chaos requires dissipation of the linear mode. As for forcing, the
main factor is whether the gyrostat core conserves energy. If it does,
then there are fewer ways in which forcing can appear for chaos to
be present. In these circumstances, chaos requires fixed points with
opposite signs of $x_{3}^{*}$ (the linear mode) and this circumscribes
where forcing can appear. The precise results for each subclass are
easily found through the corresponding expression for $x_{3}^{*}$
which takes the form $\gamma_{3}x_{3}^{*2}+\gamma_{1}=0$ for subclasses
$1-2$, and with all cubic terms nonzero for subclasses $3-4$. Previous
studies have pointed to the importance of investigating how the arrangement
of fixed points can sometimes circumscribe more complex dynamics (\citet{Eschenazi1989,Gilmore1998}),
and the gyrostat equations present a clear example. 

Such analyses can also shed light on the origins of chaos in the model
of \citet{Lorenz1963}. Although nonlinear momentum advection is present
in the model, symmetry in the assumed basis function of streamfunction
renders nonlinear advection's effects absent. Therefore, with two
nonlinear terms, and one linearly evolving mode $X$ it is not surprising
that chaos in this model requires dissipation to be present through
nonzero kinematic viscosity, which is related to the parameter $\sigma$.
The model also includes additional dissipation terms, through effects
of thermal diffusivity, and it is hardly surprising that the case
of subclass $1$ with nonzero $\left(F_{1},\epsilon_{2},\epsilon_{3}\right)$
resembles the Lorenz attractor. One difference between the Lorenz
model and subclass $1$ is that the former has a third fixed point
at the origin, whereas the models of subclass $1$ have only the pair
with nonzero $x_{3}^{*}$. This is because the forcing in the Lorenz
model appears through the term $rX$, with Rayleigh number $r$ being
the bifurcation parameter, which couples momentum to temperature.
In contrast the case $\left(F_{1},\epsilon_{2},\epsilon_{3}\right)$
of subclass $1$ has a constant forcing term, leading to only the
pair of fixed points. Nevertheless, the resulting dynamics are quite
similar, showing that only this pair of fixed points is essential
to the appearance of chaos. This also shows how linear coupling and
external forcing can have similar effects in such models. An implication
is that momentum diffusion would be necessary and sufficient for chaos
in the model of \citet{Lorenz1963}, and the further presence of thermal
diffusion only influences the shape of the attractor, analogous to
$\left(F_{1},\epsilon_{3}\right)$ and $\left(F_{1},\epsilon_{2},\epsilon_{3}\right)$
of subclass $1$. Furthermore, the Lorenz model has been obtained
from a wide variety of physical processes (\citet{Brindley1980,Gibbon1982,Matson2007}),
and such inquiries can inform the understanding of irregular dynamics
in a variety of systems. Since chaos in these cases does not depend
on the status of $p+q+r$, irregular dynamics can be experienced regardless
of whether energy is conserved.

In contrast, if the gyrostat core does not conserve energy, there
are additional possibilities for forcing to appear. These possibilities
do not require two fixed points with opposite $x_{3}^{*}$. Moreover,
such cases do not admit chaos when the energy conservation constraint
is present in gyrostatic core. Thus, the appearance of chaos in these
cases is closely tied to the presence of fewer invariants in the gyrostatic
core. 

Broadly, our findings about forced-dissipative chaos in the Volterra
gyrostat can be summarized as follows. When there is one linear mode
(let us call it $x_{3}$), it sets the direction where points in phase
space experience contraction, and dissipation must necessarily be
present in $x_{3}$. If the placement of external forcing allows two
fixed points with opposite signs of $x_{3}^{*}$, then attractors
that resemble the Lorenz attractor can appear. This condition is necessary
if the gyrostat core is energy-conserving, with these fixed points
acting as repellors. 

If the gyrostat core does not conserve energy, then there are further
ways for chaos to arise. These further arrangements are closely tied
to the loss of invariants in subclasses of the gyrostat having two
or more linear feedback terms. That is the reason this possibility
is absent from Subclass $1$, whose gyrostat core maintains two invariants
even with $p+q+r\neq0$. In Subclass 3, with two linear feedbacks
and thus only one invariant in the gyrostat core, if forcing is applied
to $x_{3}$ and thus coincides with dissipation, the attractor is
shifted along the $x_{3}$ axis and the aforementioned pair of fixed
points cannot exist. Although the resulting chaos collapses volumes
in phase space, it is possible that it is closely tied to the chaos
in volume-conserving flows found earlier (\citet{Seshadri2023}). 

Chaos in general need not depend on the arrangement of fixed points,
and many simple chaotic models have previously been found that do
not contain any fixed points (\citet{Sprott1994}). Such minimal chaotic
cases that only appear when the gyrostatic core does not conserve
energy merit further inquiry. Of course, with forcing and dissipation
both present, the model no longer has any quadratic invariants, and
it remains an open question as to whether such chaotic cases with
coincident forcing and dissipation arise from similar pathways as
in the conservative case without invariants. Much of this seems to
turn on whether adding forcing has effects that parallel the linear
feedbacks that limit the number of invariants.

There are other subclasses of the Volterra gyrostat, but we have not
considered those with three nonlinear terms, whose fixed-point equations
contain higher degrees. Further generalization of our present results
to such subclasses, as well as to systems of coupled gyrostats, calls
for explicit analysis of these models. Prior studies have indicated
the appearance of Lorenz-like attractors in low-order models of higher
dimension that discretize Rayleigh-B$\acute{e}$nard convection with
additional modes (\citet{Musielak2009,Reiterer1998}). Similar attractors
are especially prevalent when such discretization maintains the conservation
properties of these original models. Since such models must contain
systems of coupled gyrostats, similar constraints on fixed points
(and concomitant routes to chaos) might be present in higher dimensions.
Nonlinear feedback is also an important extension of the gyrostat
model (\citet{Lakshmivarahan2008a}). Studying chaos in models involving
coupled gyrostats, investigating the relationships with the number
of invariants, and the possibility of a wide range of chaotic attractors
circumscribed by fixed points in these models, with and without the
presence of nonlinear feedbacks, is a rich area of study.

\section*{Declarations of interest}

The authors have no competing interests to declare. 

\section*{Acknowledgments}

The authors are grateful to S Krishna Kumar and Rajat Masiwal for
suggesting improvements to the manuscript.

\pagebreak{} 

\bibliographystyle{agufull08}
\bibliography{VG_constants}

\pagebreak{}

\begin{landscape}

% Figure environment removed

\pagebreak{}

% Figure environment removed

\pagebreak{}

% Figure environment removed

\pagebreak{}

% Figure environment removed

\pagebreak{}

% Figure environment removed

\pagebreak{}

% Figure environment removed

\pagebreak{}

% Figure environment removed

\pagebreak{}

% Figure environment removed

\pagebreak{}

\end{landscape}
\end{document}
