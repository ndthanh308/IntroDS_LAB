\documentclass{amsart}[12pt.]

\usepackage[linktoc=none]{hyperref}

\usepackage{amsmath}
\usepackage{amssymb}
\usepackage{mathrsfs}
\usepackage[shortalphabetic]{amsrefs}

%\usepackage[foot]{amsaddr}

%\usepackage[numbers,square]{natbib}



\newtheorem{thm}{Theorem}[section]
\newtheorem{cor}[thm]{Corollary}
\newtheorem{conj}[thm]{Conjecture}
\newtheorem{lem}[thm]{Lemma}
\newtheorem{prop}[thm]{Proposition}
\newtheorem{definition}[thm]{Definition}

\newtheorem{rmk}[thm]{Remark}
\newtheorem{example}[thm]{Example}




%\usepackage[a4paper, total={6in, 8in}]{geometry}

\setcounter{section}{-1}





\begin{document}

\title{Infinitude of palindromic almost-prime numbers}






\author{Aleksandr Tuxanidy $^{\dagger}$}
\thanks{$^{\dagger}$ \em{Corresponding author}}

\author{Daniel Panario}




\address{School of Mathematics and Statistics, Carleton University, 1125 Colonel By Drive, Ottawa, Ontario, K1S 5B6, Canada}
\email{AleksandrTuxanidyTor@cmail.carleton.ca, daniel@math.carleton.ca}


	\begin{abstract}
	It is proven that, in any given base, there are infinitely many palindromic numbers having at most six prime divisors, each relatively large. The work involves equidistribution estimates for the palindromes in residue classes to large moduli, offering upper bounds for moments and averages of certain products closely related to exponential sums over palindromes. 
\end{abstract}	






	\maketitle
	
	\setcounter{tocdepth}{1}
	
	
\tableofcontents













%%%%%%%%%%%%%%%%%%%%%%%%%%%%%%%%%%%%%%%%%%%%%%%%%%%%%%%%%%%%%%%%%%%%%%%%%%%%%%%%%%%%%%%%%%%%%%%%%%%%%%%%%%%%%%%%%%%%%%%%%%%%%%%%%%%%%%%%%%%%%%%%%%%%%%%%%%%%%%%%%%%%%%%%%%%%%%%%%%%%%%%%%%%%%%%%%%%%

\section{Notation}\label{sec: notation}



\subsection{Literary notation}


We use the asymptotic notations $X \ll Y$, $Y \gg X$, $X = O(Y)$, all signifying here that $|X| \leq C|Y|$ for some unspecified constant $C > 0$. The notation $X \asymp Y$ is used to assert that both $X \ll Y$ and $Y \ll X$, simultaneously. Dependence of the implied constants on parameters will be denoted by a subscript. The symbol $o(X)$ denotes a quantity satisfying $o(X)/X \to 0$ as $X \to \infty$.  

The letters $\mu, \tau, \varphi$ denote the M\"{o}bius, divisor and Euler's totient functions, respectively. For 
an integer $n \geq 1$, the symbol $\Omega(n)$ denotes the total number of prime factors of $n$, counting their multiplicity. The symbol $P^-(n)$ denotes the smallest prime factor of $n$. The letter $p$ is reserved to denote a prime number and $b$ denotes an integer larger than $1$.  




The notation $n \equiv a (q)$ denotes the assertion that $n$ is congruent to $a$ modulo $q$. We use $d \mid n$ to assert that $d$ divides $n$, and $(m,n)$ to represent the greatest common divisor of $m$ and $n$. If $(m,n)=1$, we use the notation $\text{ord}_m(n)$ for the multiplicative order of $n$ modulo $m$.  

The symbol 
$$\sum _{a (q)}$$
denotes a summation over all the residue classes $a$ modulo $q$, whereas 
$$\sum_{a(q)}^*$$ 
denotes a sum over the invertible residue classes $a$ modulo $q$. 

Given a function $f : \mathbb{R} \to \mathbb{C}$, we denote by $\|f\|_1$ its $L^1$-norm and by $\|f\|_\infty$ its $L^{\infty}$-norm (whenever these exist).

For a real number $x$ and an integer $q \geq 1$, we use the notation $e(x) := e^{2\pi i x}$ for the complex exponential, $e_q(x) := e(x/q)$ for the additive character modulo $q$ and $\exp(x) := e^x$ for the exponential. If $x > 0$, its natural logarithm is denoted by $\log x$ and its base-$b$ logarithm by $\log_b x$. 


We use $\|x\| $ to symbolize the distance from $x$ to its nearest integer; that is, $\|x\| := \min_{k \in \mathbb{Z}}|x-k|$. Moreover $\lfloor x \rfloor$ denotes the floor of $x$ and $\{x\} := x - \lfloor x \rfloor$ denotes its fractional part. 



Given a statement $E$, we use $\mathbf{1}_E$ to symbolize the indicator function of $E$; that is, $\mathbf{1}_E = 1$ if $E$ is true and $\mathbf{1}_E = 0$ if $E$ is false. For example, $\mathbf{1}_{n \equiv a (q)}$ equals $1$ if $n \equiv a (q)$ and is zero otherwise. 

Given real numbers $\alpha_1, \ldots, \alpha_N$, their {\em discrepancy} $D_N(\alpha_1, \ldots, \alpha_N)$ is defined by
\begin{equation}\label{eqn: discrepancy}
D_N(\alpha_1, \ldots, \alpha_N) := \sup_{0 \leq c \leq d \leq 1}\left|\dfrac{ \# \left\{1 \leq n \leq N \ : \ c \leq \{\alpha_n\} < d  \right\}}{N} - (d-c)\right|
\end{equation}
and their {\em star-discrepancy} $D_N^*(\alpha_1, \ldots, \alpha_N)$ is defined as
\begin{equation}\label{eqn: star discrepancy}
D_N^*(\alpha_1, \ldots, \alpha_N) := \sup_{0 \leq d \leq 1} \left|\dfrac{ \# \left\{1 \leq n \leq N \ : \ \{\alpha_n\} < d \right\}}{N} - d\right|.
\end{equation}








\subsection{Non-standard notation}

Let $b \geq 2$ be an integer. The notation $\mathscr{P}_b$ symbolizes the set of all positive $b$-palindromic integers. We define the sets $\mathscr{P}_b^0, \mathscr{P}_b^*$, as follows: 
\begin{align}
\mathscr{P}_b^0 &:= \left\{ n \in \mathscr{P}_b  \ : \ \lfloor \log_b(n) \rfloor \equiv 0 (2) \right\},
\label{def of P0}
\\
\mathscr{P}_b^* &:= \left\{ n \in \mathscr{P}_b \ : \ (n,b^3-b)=1 \right\}. \label{def of P*}
\end{align}
From the definition above one observes that $\mathscr{P}_b^0$ is comprised exactly of all positive $b$-palindromes with an odd number of digits (in their $b$-adic expansion). 
Since every $b$-palindrome with an even number of digits is divisible by $b + 1$, and $b + 1$ is a divisor of $b^3-b$, it follows $\mathscr{P}_b^* \subset \mathscr{P}_b^0$. 
In fact
\begin{align*}
\mathscr{P}_b^0 &= \bigcup_{N=0}^\infty \Pi_b(2N),\\
\mathscr{P}_b^* &= \bigcup_{N=0}^\infty \Pi_b^*(2N),
\end{align*}
where for an integer $N \geq 0$, 
\begin{align}
\Pi_b(N) &:= \mathscr{P}_b \cap \left[b^N, b^{N+1}\right), \label{eqn: def of pi} \\
\Pi_b^*(N) &:= \left\{ n \in \Pi_b(N) \ : \ (n,b^3-b)=1  \right\}. \label{eqn: def of pi star}
\end{align}

Given a set $\mathscr{A}$ of integers, a real number $x$ and integers $a,q$, we define the sets
\begin{align*}
\mathscr{A}(x) &:= \mathscr{A} \cap \left\{ n \in \mathbb{Z} \ : \ n \leq x\right\},\\
\mathscr{A}(x,a,q) &:= \left\{n \in \mathscr{A}(x) \ : \ n \equiv a (q)  \right\}.
\end{align*} 
Thus for instance 
\begin{align}
\mathscr{P}_b^*(x) &:= \mathscr{P}_b^* \cap [1,x]\label{def: P star},\\
\mathscr{P}_b^*(x,a,q) &:= \left\{ n \in \mathscr{P}_b^*(x) \ :\ n \equiv a (q)   \right\},\label{def: P star cong}
\end{align}
with our notation. 

Given a real number $\alpha$ and an integer $N\geq 0$, we define
\begin{equation}\label{eqn: phi def}
\phi_b(\alpha) := \left|\sum_{0 \leq m < b}e(\alpha m)\right|
\end{equation}
and
\begin{equation}\label{eqn: def of Phi}
\Phi_N(\alpha) := \prod_{1 \leq n < N} \phi_b\left(\alpha\left(b^n + b^{2N-n}\right)\right)
\end{equation}
with the convention $\Phi_N(\alpha) = 1$ for $N \leq 1$. The definition of $\Phi_N$ also depends on $b$ but for the sake of brevity we omit it on the left hand side above. 

Finally given integers $n,K,b$ with $K,b \geq 2$, we let 
\begin{equation}\label{eqn: num of compo}
r(n;K,b) := \left\{ (m_1, \ldots, m_K) \in \left([0,b) \cap \mathbb{Z}\right)^K \ : \ m_1 + \cdots + m_K = n\right\} 
\end{equation}
be the number of additive compositions of $n$ into $K$ integers $m_j$, each satisfying $0 \leq m_j < b$. 





\section{Introduction}

\subsection{Tattarrattat}

Generally speaking, a {\em palindrome} is a finite sequence of objects which reads the same forwards as backwards. Any such sequence is said to be {\em palindromic}. For example the sequence of the (decimal) digits of the prime number
$$9609457639843489367549069$$ 
is a palindrome. According to \cite{SigmaGeek} (see also \cite{Google}) it is the largest known palindromic prime appearing in the decimal expansion of the digits of $\pi$. 

One of the earliest historical palindromic objects, a SATOR square, was found in the remains of the ancient city of Pompeii (see for instance \cite{Sheldon}) destroyed in the year 79 AD after the eruption of Mount Vesuvius. Palindromes appear in several facets of human endeavor, for instance in music (e.g. Joseph Haydn's Symphony No. 47, ``The Palindrome") in literature (e.g. Georges Perec's ``Le Grand Palindrome''). In recent decades, palindromes are studied intensively in various areas of research \cite{Banks, Banks-Hart-Sakata, Banks-Shparlinski, Cilleruelo, Col, Fici, Gao, Garefalakis, Gusfield, Porto, Rajasekaran}.
They also appear in nature; for example, large portions of the human X and Y chromosomes are palindromic (in a slightly different sense; see \cite{Larionov}). 

\subsection{Conjecture and previous work} 

On the topic of palindromic numbers, perhaps one of the most difficult unsolved problems is the following conjecture (see for instance the concluding remarks in \cite{Banks-Hart-Sakata, Banks-Shparlinski}). First from here on out, $b$ denotes an integer larger than $1$. One says that a natural number is {\em $b$-palindromic}, or is a {\em $b$-palindrome}, if the sequence of its $b$-adic digits is palindromic. For a real number $x > 0$, we let $\mathscr{P}_b(x)$ be the set of all positive $b$-palindromic integers at most $x$ in size.

\begin{conj}\label{conj: pal primes}
	Let $b \geq 2$ be an integer. Then there are infinitely many prime numbers that are $b$-palindromic. In fact
	$$
	\#\left\{p \in \mathscr{P}_b(x) \right\} \asymp_b  \dfrac{\#\mathscr{P}_b(x)}{\log x}
	$$ 
	for $x$ large.
\end{conj}



Conjecture \ref{conj: pal primes} seems far out of reach with current methods. As pointed out by Banks-Shparlinski \cite{Banks-Shparlinski}, one of the main difficulties stems from the large sparsity of $b$-palindromic numbers, hindering the efficacy of analytic methods. The sparsity of these is, in essence, as large as that of numbers with form $n^2+1$. However Landau's problem (posed in the year 1912) of whether or not there are infinitely many such primes, remains unsolved. On the other hand, Iwaniec \cite{Iwaniec} proved there are infinitely many integers $n$ such that $n^2+1$ is the product of at most two primes. In view of recent results (see the discussion below and the following subsection) we believe that proving the analogous result, for $b$-palindromes, may be feasible over the upcoming years. 


Some progress was made in the direction of Conjecture \ref{conj: pal primes}. In the year 2004, Banks-Hart-Sakata \cite{Banks-Hart-Sakata} showed that for $x$ large enough, 
\begin{equation}\label{eqn: BHS density}
\#\left\{p \in \mathscr{P}_b(x)\right\}\ll_b \dfrac{\log \log \log x}{\log \log x}\#\mathscr{P}_b(x).
\end{equation}
In 2009, Col \cite{Col} improved this to the expected
\begin{equation}\label{eqn: Col density}
\#\left\{p \in \mathscr{P}_b(x)\right\} \ll_b \dfrac{\#\mathscr{P}_b(x)}{\log x}.
\end{equation}
Col also obtained (see Theorem \ref{thm: Col} below) a lower bound for the amount of $b$-palindromic almost-primes, when the number of prime divisors is bounded above by a certain value depending only on $b$. This was the first, and up to now unique, result of its kind. 

\begin{thm}[Col \citelist{\cite{Col},  Corollary 2}]\label{thm: Col} 
	We have
	$$
	\#\left\{n \in \mathscr{P}_b(x) \ : \ \Omega(n) \leq k_b \right\} \gg_b \dfrac{\#\mathscr{P}_b(x)}{\log x}
	$$
	for some value $k_b > 0$ depending only on $b$ and satisfying $k_b \sim 24\pi b$ as $b \to \infty$. In particular $k_2=60$ and $k_{10} = 372$.
\end{thm}



Theorem \ref{thm: Col} was a consequence of Col's equidistribution estimate (see Theorem \ref{thm: Col equi estimate} below) for $b$-palindromes in residue classes to large moduli, combined with well-established facts from sieve theory. First recall the notation 
$$\mathscr{P}_b(x,a,q) := \left\{n \in \mathscr{P}_b(x) \ : \ n \equiv a (q)\right\}.$$ 

\begin{thm}[Col \citelist{\cite{Col},  Theorem 2}]\label{thm: Col equi estimate}
	There exists some value $\beta > 0$, depending only on $b$, such that for any $\epsilon,A > 0$,
	$$
	\sum_{\substack{q \leq x^{\beta - \epsilon} \\ (q,b^3-b)=1}} \sup_{y \leq x} \max_{a \in \mathbb{Z}} \left|\#\mathscr{P}_b(y,a,q) - \dfrac{\#\mathscr{P}_b(x)}{q}\right| \ll_{\epsilon,A,b} \dfrac{\#\mathscr{P}_b(x)}{\log^A x}. 
	$$
	For $b$ large, we may take $\beta \sim 1/6\pi b $. 
	Setting $\beta_b = \beta - \epsilon$, the inequality above holds for the following particular values.
	\begin{center} 
	\begin{tabular}{|c|c|c|c|c|c|c|c|c|c|}
	\hline	
	$b$ & $2$ & $3$ & $4$ & $5$ & $6$ & $7$ & $8$ & $9$ & $10$\\
	\hline
	$\beta_b$ &  $1/30$& $1/94$ &  $1/74$& $1/122$ & $1/114$ &  $1/158$&  $1/150$&  $1/194$&$1/186$\\
	\hline
	\end{tabular}
	\end{center}
\end{thm}	

Theorem \ref{thm: Col equi estimate} (see also Proposition \ref{prop: Linfty bound} here) considerably improved over previous estimates of Banks-Hart-Sakata \cite{Banks-Hart-Sakata} valid for $\beta \ll (\log \log x)/\log x$ with the implied constant small enough (see Corollary 4.5 in \cite{Banks-Hart-Sakata}). 

\subsection{Statement of results}


In the present work we obtain the following improved versions of Theorems \ref{thm: Col} and \ref{thm: Col equi estimate}.

\begin{thm}\label{thm: at most 6 primes}
	Let $b\geq 2$ be an integer. Then for $x \gg_b 1$ large enough, 
	$$
	\#\left\{n \in \mathscr{P}_b(x) \ : \ \Omega(n) \leq 6, \ P^{-}(n) \geq x^{1/21}\right\} \asymp_{b} \dfrac{\#\mathscr{P}_b(x)}{\log x}.
	$$
\end{thm} 

Theorem \ref{thm: at most 6 primes} is a consequence of Theorem \ref{thm: equidistribution} below and machinery from sieve theory (see Lemma \ref{lem: sieve lemma} for the latter). Here $\mathscr{P}_b^*(x)$ and $\mathscr{P}_b^*(x,a,q)$ are defined as in (\ref{def: P star}), (\ref{def: P star cong}), respectively.


\begin{thm}\label{thm: equidistribution}
	Let $b\geq 2$ be an integer and let $x \geq 1$. Then for any $\epsilon > 0$,
	\begin{equation}\label{eqn: equi 1}
	\sum_{\substack{q \leq x^{1/5-\epsilon} \\ (q,b^3-b)=1}} \sup_{y \leq x }\max_{a \in \mathbb{Z}} \left|\#\mathscr{P}^*_b(y,a,q) - \dfrac{\#\mathscr{P}^*_b(y)}{q}\right| \ll_{b,\epsilon} \dfrac{\#\mathscr{P}_b^*(x)}{e^{\sigma(b, \epsilon) \sqrt{\log x}}} ,
	\end{equation}
	where $  \sigma(b,\epsilon) > 0$ is some value depending only on $b$ and $\epsilon$. 
\end{thm}

Let us now discuss some of the main ingredients in the proof of Theorem \ref{thm: equidistribution}.



\subsection{Setup}

For the sake of brevity we denote by $E$ the left hand side of the inequality in Theorem \ref{thm: equidistribution}. Let $\mathscr{P}_b^0$, $\mathscr{P}_b^*$ be defined as in (\ref{def of P0}), (\ref{def of P*}), respectively. Set $\mathscr{P}_b^0(x) := \mathscr{P}_b^0 \cap [1,x]$. 
From the observation 
$
\mathscr{P}_b^* = \left\{n \in \mathscr{P}_b^0 \ : \ (n,b^3-b)=1\right\}  
$, 
the M\"{o}bius inversion formula $$\mathbf{1}_{(n,b^3-b)=1} = \sum_{r \mid (n, b^3-b)} \mu(r),$$ 
the Fourier expansions 
$$
\mathbf{1}_{r \mid n} = \dfrac{1}{r}\sum_{k(r)}e_r(nk)
$$ and 
\begin{equation}\label{eqn: Fourier type expansion}
\mathbf{1}_{n\equiv a (q)} = \dfrac{1}{q}\sum_{h(q)} e_q(-ah)e_q(n h),
\end{equation}
one derives
$$
E \ll \tau(b^3-b)(\log x) \max_{k\in \mathbb{Z}}\sum_{\substack{2 \leq q \leq x^{1/5 - \epsilon} \\ (q,b^3-b)=1}} \dfrac{1}{q} \sum_{h(q)}^*\sup_{\substack{y \leq x }} \left|\sum_{n \in \mathscr{P}_b^0(y)} e\left(\dfrac{hn}{q} + \dfrac{kn}{b^3-b} \right)\right| .
$$
The use of the decomposition in (\ref{eqn: Fourier type expansion}) is rather typical in the literature and it will shortly lead to qualitative gains in multiplicative algebraic structures, with the appearance of the products $\Phi_N$ defined in (\ref{eqn: def of Phi}). Unfortunately we pay the price of a quantitative loss of roughly $x^{1/5-\epsilon}$ in the trivial bound. Thus in order for such a Fourier-type approach to succeed, we must not only be able to recover the lost $x^{1/5-\epsilon}$, but gain much more (in fact gain an extra $\exp(\sigma\sqrt{\log x})$). 

From the works of Banks-Hart-Sakata \cite{Banks-Hart-Sakata} and Col \cite{Col} (see Lemma \ref{lem: bound for exp sum of pals} here) the inner sum admits a decomposition into linear combinations of the products $\Phi_N$ with $0 \leq N \leq \frac{1}{2}\log_b x$. After a dyadic subdivision of the interval $2 \leq q \leq x^{1/5 - \epsilon}$, the problem of bounding $E$ then boils down to bounding averages of the form
$$
S = \sum_{\substack{2 \leq q \leq Q \\ (q,b^3-b)=1}} \sum_{h(q)}^* \Phi_N\left(\dfrac{h}{q} + \dfrac{k}{b^3-b}\right) 
$$ 
for $Q \ll x^{1/5-\epsilon}$. The trivial bound gives $S \leq Q^2 b^N \leq Q^2 \sqrt{x}$ and our job is to beat this by much more than a factor of $Q$.

Thanks to the work of Col \cite{Col} (see Proposition \ref{prop: Linfty bound} here) we can readily dispose of the cases with $Q$ relatively small, say $Q \ll \exp(c\sqrt{\log x})$. We are thus left with considering $S$, or more strongly the average
\begin{equation}\label{eqn: def of T}
T = \sum_{\substack{q \leq Q \\ (q,b)=1}} \sum_{h(q)}^* \Phi_N\left(\dfrac{h}{q} + \beta \right),
\end{equation}
for $\exp(c\sqrt{\log x}) \ll Q \ll x^{1/5-\epsilon}$ and $\beta \in \mathbb{R}$. 

As a first rough thought, one could envision splitting the product $\Phi_N$ into two products, one over $1 \leq n \leq M$ and the other over $M < n < N$, applying Cauchy-Schwarz and end up considering averages of squares (just as Bombieri-Vinogradov's treatment of the Type II sums). These could be attacked say via direct arguments (see for instance Lemma \ref{lem: L2 bound}) or the large sieve inequality (see Lemma \ref{lem: large sieve} below). The direct arguments (namely that of expanding the square, switching orders of summation, using the orthogonality of the additive characters and bounding the resulting sums trivially) can at best aid in recovering a $Q$ from the trivial bound, and are thus insufficient. This brings us to the large sieve inequality.

\subsection{Large sieve}

First given a real number $0 < \delta \leq 1/2$, one says that a set of points $\alpha_1, \ldots, \alpha_R \in \mathbb{R}/\mathbb{Z}$ is $\delta${\em -spaced} if $\|\alpha_r - \alpha_s\| \geq \delta$ whenever $r \neq s$. 

\begin{lem}[Large sieve inequality]\label{lem: large sieve}
	For any set of $\delta$-spaced points $\alpha_1, \ldots, \alpha_R \in \mathbb{R}/\mathbb{Z}$ and any complex numbers $a_n$ with $M < n \leq M + N$, where $0 < \delta \leq 1/2$ and $N \geq 1$ is an integer, we have
	\begin{equation}\label{ineq: large sieve}
	\sum_{r=1}^R \left|\sum_{ M < n \leq M+N} a_n e(\alpha_r n)\right|^2 \leq \left(\delta^{-1} + N - 1\right) \sum_{M< n \leq M+N}|a_n|^2. 
	\end{equation}
\end{lem}

\begin{proof}
	See for instance Theorem 7.7 in \cite{Iwaniec-Kowalski}. 	
\end{proof}

The large sieve inequality is a powerful and useful tool for tackling averages of exponential sums and has enjoyed several applications in the literature (see for instance \cite{Friedlander-Iwaniec, Iwaniec-Kowalski, maud, Maynard, Tenenbaum} to name a few). Unfortunately the quality of the bounds it offers highly depends on the density $\eta = \#\mathcal{S}/N$ of the set $\mathcal{S} \subseteq (M, M+N]$ in which the sequence $(a_n)$ is supported on, with better bounds for denser $\mathcal{S}$ and worse bounds for sparser $\mathcal{S}$. For example in the case when each $a_n \ll 1$ and $R \asymp Q^2 \asymp \delta^{-1}  \asymp N$, the large sieve inequality beats the trivial bound by a factor of $\#\mathcal{S} = \eta N \asymp Q \eta \sqrt{N}$. In particular in order for the large sieve to beat the trivial bound by much more than $Q$ (which is what we seek) one needs $\eta$ to be much larger than $1/\sqrt{N}$. Unfortunately sparse sets such as, say, the $b$-palindromes or close relatives, have density $ \ll 1/\sqrt{N}$ on $(M, M+N]$. For sequences supported on such sets, direct use of the large sieve proves insufficient. 


An important idea for the case of the sparse sequences is to consider instead larger moments such as
$$
\sum_{r=1}^R \left|\sum_{ M < n \leq M+N} a_n e(\alpha_r n)\right|^{2K}
$$
for some integer $K \geq 2$; see for example \cite{Iwaniec-Kowalski}. One may arrive at these from smaller moments in a variety of ways, say via H\"{o}lder's inequality. To explain the concept, note we may write 
$$
\left(\sum_{ M < n \leq M+N} a_n e(\alpha_r n)\right)^K = \sum_{KM < n \leq K(M+N)} b_n e(\alpha_r n),
$$
where the coefficients $b_n$ are now supported on the set $K\mathcal{S}$ comprised of the additions of any $K$ elements from $\mathcal{S}$. If the set $\mathcal{S}$ is sufficiently dissociated additively, then $K\mathcal{S}$ is much denser in $(KM, K(M+N)]$. In this case we can make a more effective use of the large sieve. 


\subsection{Approach}

Let us now return to (\ref{eqn: def of T}). 
Motivated by the discussion above, we split the product $\Phi_N$ into three pieces as $\Phi_N = P_1P_2P_3$, where $P_1$ is the product over $1 \leq n \leq M$, $P_2$ is the product over $M < n \leq 2M$ and $P_3$ is the product over $2M < n < N$ (actually this is our approach for the case when $Q \gg b^{(\frac{2}{5} - \epsilon_0)N}$; the case $Q \ll b^{(\frac{2}{5} - \epsilon_0)N}$ is treated in a slightly different way). The choice of $M$ is optimized later. Applying H\"{o}lder's inequality with the triple $(\ell,\ell,2K)$ (where $K\geq 2$ is an integer to be chosen later and $\ell > 2$ is such that $2/\ell + 1/2K = 1$) one obtains
$$
T \leq \left(\sum_{\substack{q \leq Q \\ (q,b)=1}} \sum_{h(q)}^* P_1^\ell \right)^{1/\ell} \left(\sum_{\substack{q \leq Q \\ (q,b)=1}} \sum_{h(q)}^* P_2^\ell \right)^{1/\ell} \left(\sum_{\substack{q \leq Q \\ (q,b)=1} } \sum_{h(q)}^* P_3^{2K}\right)^{1/2K}. 
$$
For $K$ somewhat large (although bounded in terms of some parameters for our purposes) $\ell$ is close to $2$. The first two sums corresponding to $P_1, P_2$ are then treated by us in direct fashion (as in Lemma \ref{lem: L2 bound}). Here we avoid using the large sieve which, given the sparsity of the coefficient sequences corresponding to $P_1, P_2$, gives no significant gains over the direct approach.

There is a reason why we have chosen to raise $P_3$ to the $2K$-th power and not $P_1$ nor $P_2$, namely, the following algebraic identity
$$
\sum_{\substack{q \leq Q \\ (q,b)=1}} \sum_{h(q)}^*P_3^{2K} = \sum_{\substack{q \leq Q \\ (q,b)=1}} \sum_{h(q)}^* \Phi_{N-2M}^{2K}\left(\dfrac{h}{q} + \beta b^{2M}\right). 
$$
The trivial bound shows the above is at most $ Q^2 b^{2K(N-2M)} \leq Q^2 x^{K(1 - \frac{2M}{N})}$ and our main goal then boils down to beating this by much more than $Q$ for certain suitable choices of $M,K$. This is accomplished via Proposition \ref{prop:2K-th moment2} giving upper bounds for moments of $\Phi_N$ for any $N\geq 1$. Its proof relies partly on the large sieve. 






















\section{Acknowledgments}

We would like to express our gratitude to Qiang Wang for useful advices and to Igor E. Shparlinski for notifying us of Col's work \cite{Col}. D. Panario was funded by the Natural Science and Engineering Research Council of Canada (NSERC), reference number RPGIN-2018-05328.

\section{Paper structure}

The rest of the work goes as follows. In Section \ref{sec: preparatory lemmas} we compile several technical tools needed in the following two sections. In Section \ref{sec: exp sums} we discuss the current (but limited) state of knowledge on exponential sums over palindromes, as well as slightly add to the literature via Proposition \ref{prop: product bound for modulus p} and Corollary \ref{cor: bound using product of Bourgain sequence} there. These are applications of Bourgain's bound in \cite{Bourg}. In Section \ref{sec: moments} we prove Proposition \ref{prop:2K-th moment2} giving an upper bound for moments of $\Phi_N$. In Section \ref{sec: average bound} we prove Proposition \ref{prop: average bound} bounding an average of $\Phi_N$. In Section \ref{sec: equi estimate} we prove Theorem \ref{thm: equidistribution} giving average equidistribution estimates for palindromes in residue classes. We then conclude the work in Section \ref{sec: final proof} with a proof of Theorem \ref{thm: at most 6 primes}. 






 





%%%%%%%%%%%%%%%%%%%%%%%%%%%%%%%%%%%%%%%%%%%%%%%%%%%%%%%%%%%%%%%%%%%%%%%%%%%%%%%%%%%%%%%%%%%%%%%%%%%%%%%%%%%%%%%%%%%%%%%%%%%%%%%%%%%%%%%%%%%%

\section{Technical lemmas}\label{sec: preparatory lemmas}

In this section we go over several technical tools needed in the following two sections. The first two are classical in the theory of uniform distribution of sequences modulo $1$.





\begin{lem}[Koksma-Hlawka]\label{lem: Koksma-Hlawka}
Suppose $f : [0,1] \to \mathbb{R}$ is a bounded integrable function of bounded variation $V(f)$. Then for any real numbers $\alpha_1, \ldots, \alpha_N$ with $N \geq 1$ an integer,
$$
\left|\dfrac{1}{N} \sum_{n=1}^N f(\{\alpha_n\}) - \int_0^1 f(x)dx  \right| \leq V(f)D_N^*(\alpha_1, \ldots, \alpha_N), 
$$
where $D_N^*(\alpha_1, \ldots, \alpha_N)$ is their star-discrepancy as defined in (\ref{eqn: star discrepancy}).
\end{lem}

\begin{proof}
See \cite{Kuipers}. 
\end{proof}








\begin{lem}[Erd\H{o}s-Tur\'{a}n]\label{lem: erdos-turan}
Let $\alpha_1, \ldots, \alpha_N$ be real numbers with $N \geq 1$ an integer. Then
$$
D_N(\alpha_1, \ldots, \alpha_N) \ll \dfrac{1}{H} + \sum_{1 \leq h \leq H} \dfrac{1}{h} \left|\dfrac{1}{N}\sum_{n=1}^N e(h\alpha_n)\right|
$$
for any $H\geq 1$,
where $D_N(\alpha_1, \ldots, \alpha_N)$ is the discrepancy defined in (\ref{eqn: discrepancy}).
\end{lem}

\begin{proof}
	See \cite{Kuipers}. 
\end{proof}



	



\begin{lem}[Smooth exponential sum bounds]\label{lem: smooth exp sums}
	Let $f : \mathbb{R} \to \mathbb{C}$ be a smooth compactly supported function and let $\alpha$ be a real number. Then 	
	\begin{equation}\label{eqn: trivial bound}
		\left|\sum_{n \in \mathbb{Z}}f(n)e(\alpha n) \right| \leq \|f\|_1 + \dfrac{\|f'\|_1}{2}
	\end{equation}
	and
	\begin{equation}\label{eqn: sum by parts bound}
		\left|\sum_{n \in \mathbb{Z}}f(n)e(\alpha n) \right| \leq \dfrac{\|f^{(k)}\|_1}{|2 \sin(\pi \alpha)|^k}
	\end{equation}
	for all natural numbers $k\geq 1$.
\end{lem}	

\begin{proof}
	See for instance Lemma 3.1 in \cite{Tao}. 
\end{proof}	

The following lemma appears, either implicitly or explicitly, on several works on digital functions. See for instance \cite{Bourgain, Col, maud, Maynard, Morgenbesser}. We give a proof for the convenience of the reader.


\begin{lem}[Ergodic-type integral bound]\label{lem: integral of product}
	Let $f : \mathbb{R} \to \mathbb{R}_0^+$ be a bounded integrable $1$-periodic function. Then for any integer $N \geq 1$,
	$$
	\int_{0}^1 \prod_{0 \leq n < N} f(\alpha b^n)d\alpha \leq \left(\sup_{0 \leq\theta \leq 1}\dfrac{1}{b}\sum_{n(b)} f\left(\dfrac{n + \theta}{b} \right)\right)^N. 
	$$	
\end{lem}

\begin{proof}
	By induction on $N \geq 1$. For $N=1$, we have
	$$
	\int_0^1 f(\alpha)d\alpha = \dfrac{1}{b}\sum_{0 \leq h < b} \int_0^{1} f\left(\dfrac{h + \theta}{b} \right)d\theta
	$$
	and the claim follows. Suppose now that the claim holds for some $N \geq 1$. We have
	\begin{align*}
		\int_0^1 \prod_{0 \leq n < N+1} f(\alpha b^n)d\alpha 
		&=\dfrac{1}{b}\sum_{0 \leq h < b} \int_0^{1} \prod_{0 \leq n < N+1} f\left(\left(\dfrac{h + \theta}{b}\right) b^n\right)d\theta
		\\
		&=
		\int_0^{1} \dfrac{1}{b}\sum_{0 \leq h < b}f\left(\dfrac{h + \theta}{b} \right) \prod_{0 \leq n < N}f\left(\theta b^n\right)d\theta.
	\end{align*}
	The last equality holds since $f$ is $1$-periodic by assumption. The above is
	$$
	\leq \sup_{0 \leq \theta \leq 1} \dfrac{1}{b}\sum_{0 \leq h < b}f\left(\dfrac{h + \theta}{b} \right)	\int_0^{1}\prod_{0 \leq n < N}f\left(\alpha b^n\right)d\alpha
	$$
	and the claim for $N+1$ now follows from the inductive hypothesis.	
\end{proof}	


The following lemma gives asymptotics for the number of compositions with some restrictions. We were unable to obtain a reference in the literature and we thus give a proof. If one allows the error term to depend on $n$, one should be able to considerably improve it with a more detailed analysis of the integrals involved. Nevertheless the error term here suffices for our purposes.

\begin{lem}[Compositions with restrictions]\label{lem:comp2}
	Let $n,K,b$ be integers with $K,b \geq 2$ and let $r(n;K,b)$ be as defined in (\ref{eqn: num of compo}). We have 
	$$
	\dfrac{r(n;K,b)}{b^K} =  \sqrt{\dfrac{6}{\pi(b^2-1)K}}\exp\left(-\dfrac{6}{(b^2-1)K}\left(n - \dfrac{(b-1)K}{2}\right)^2\right) + O\left(\dfrac{1}{bK^{3/2}}\right).
	$$
\end{lem}

\begin{proof}
	An application of the identity
	\begin{equation}\label{eqn: orth identity}
	\mathbf{1}_{\ell =0} = \int_{-1/2}^{1/2} e(\ell \alpha)d\alpha
	\end{equation}
	valid for integers $\ell$, yields
	$$
	r(n;K,b) = \int_{-1/2}^{1/2} e(-\alpha n) \left(\sum_{0 \leq m < b}e(\alpha m)\right)^Kd\alpha.
	$$
	In what follows $\delta$ denotes a fixed real number satisfying $1/2 < \delta < 1$. We allow implied constants to depend on $\delta$. To evaluate the integral above we split it into two pieces, one over $|\alpha| \leq \delta/b$ and the other over $\delta/b < |\alpha| \leq 1/2$. We treat each separately. 
	
	When $\alpha \not\in \mathbb{Z}$, the sum inside the brackets above equals
	$$
	\dfrac{e(\alpha b)-1}{e(\alpha)-1} = e\left(\dfrac{b-1}{2}\alpha\right) \dfrac{\sin(\pi \alpha b)}{\sin(\pi \alpha)}.
	$$
	By Euler's product formula,
	$$
	\dfrac{\sin(\pi \alpha b)}{\sin(\pi \alpha)} = b \prod_{m=1}^\infty \left(1 - \dfrac{\alpha^2 b^2}{m^2}\right)\left(1 - \dfrac{\alpha^2}{m^2} \right)^{-1}.
	$$
	If $|\alpha| < 1/b$, the logarithm of the product over $m\geq 1$ equals
	\begin{equation}\label{eqn: log expansion in terms of zeta}
	-\sum_{m\geq 1} \sum_{k\geq 1} \dfrac{(b^{2k}-1)\alpha^{2k}}{m^{2k}k} = - \sum_{k\geq 1} \dfrac{\zeta(2k)}{k}\left(b^{2k}-1\right)\alpha^{2k},
	\end{equation}
	where $\zeta$ is the Riemann zeta function. For $|\alpha| \leq \delta/ b$, the terms with $k\geq 2$ contribute $\ll \alpha^4 b^4$ to the sum. Then it follows 
	$$
	\left(\dfrac{\sin(\pi \alpha b)}{\sin(\pi \alpha)}\right)^K = b^Ke^{-\zeta(2)(b^2-1)K\alpha^2}\left(1 + O\left(\alpha^4  b^4 K\right)\right)
	$$
	for $0 < |\alpha| \leq \delta/b$. 
	Thus, after dividing by $b^K$,
	\begin{align*}
	&\dfrac{1}{b^K}\int_{-\delta/b}^{\delta/b} e(-\alpha n) \left(\sum_{0 \leq m < b} e(\alpha m)\right)^Kd\alpha\\
	 &= \int_{-\delta/b}^{\delta/b} e\left(\dfrac{b-1}{2}\alpha K - \alpha n\right)e^{-\zeta(2)(b^2-1)K\alpha^2}\left(1 + O\left(\alpha^4  b^4 K\right)\right)d\alpha\\
	&=
	\int_{\mathbb{R}} e\left(\dfrac{b-1}{2}\alpha K - \alpha n\right)e^{-\zeta(2)(b^2-1)K\alpha^2} d\alpha \\
	&\qquad 
	+ O\left(b^4 K\int_{\mathbb{R}}e^{-\zeta(2)(b^2-1)K\alpha^2}\alpha^4 d\alpha + \int_{\delta/b}^\infty e^{-\zeta(2)(b^2-1)K\alpha^2}d\alpha\right).
	\end{align*}
	The expression inside of the $O$-brackets can be shown to be $\ll b^{-1}K^{-3/2}$ after substitutions of variables. By the well-known formula
	$$
	\int_{\mathbb{R}}e^{-at^2}e(-\theta t)dt = \sqrt{\dfrac{\pi}{a}} e^{-\pi^2\theta^2/a} \hspace{2em} (\theta \in \mathbb{R}, \ a > 0) 
	$$
	for the Fourier transform of the Gaussian $e^{-at^2}$ and the fact $\zeta(2) = \pi^2/6$, the integral outside of the $O$-brackets equals
	$$
	\sqrt{\dfrac{6}{\pi (b^2-1)K}} \exp\left(-\dfrac{6}{(b^2-1)K}\left(n - \dfrac{(b-1)K}{2}\right)^2\right).
	$$
	
	
	To conclude it suffices to show that
	$$
	\int_{\delta/b < |\alpha| \leq 1/2} e(-\alpha n)\left(\sum_{0 \leq m < b}e(\alpha m)\right)^Kd\alpha \ll \dfrac{b^{K-1}}{K^{3/2}}.
	$$
	The sum inside the brackets is at most $|\csc(\pi \alpha)| \leq 1/2|\alpha|$ (for $|\alpha| \leq 1/2$) in absolute value. Then the absolute value of the integral above is 
	$$\leq 2^{1-K}\int_{\delta/b }^{1/2}\alpha^{-K}d\alpha \leq \dfrac{b^{K-1}}{(2\delta)^{K-1}(K-1)} \ll \dfrac{b^{K-1}}{K^{3/2}}
	$$
	for $\delta > 1/2$ fixed. 
\end{proof}	


We also need some trigonometric facts involving the function $\phi_b$ defined as in (\ref{eqn: phi def}). It will be useful to note that
$$
\phi_b(\alpha) = \begin{cases}
	b &\mbox{ if } \alpha \in \mathbb{Z},\\
	|\sin(\pi \alpha b)/\sin(\pi \alpha)| &\mbox{ otherwise.}
\end{cases}
$$
Clearly $\phi_b$ is $1$-periodic and even; thus $\phi_b(\alpha) = \phi_b(\{\alpha\}) = \phi_b(\|\alpha\|)$ for any real $\alpha$.

The following fact, but with $\|\alpha\|$ in the range $\|\alpha\|^2 \leq 6/\pi^2(b^2-1)$, appears in \cite{MR} (see their Lemma 3).

\begin{lem}\label{lem: exponential bound}
	If $\|\alpha\| \leq 1/b$, then
	$$
	\phi_b(\alpha) \leq b \exp\left(-\dfrac{\pi^2 }{6}(b^2-1)\|\alpha\|^2\right).
	$$	
\end{lem}

\begin{proof}
	We have $\phi_b(0) = b$ and $\phi_b(1/b)=0$, whence the result holds for $\|\alpha\| \in \{0, 1/b \}$. For $0 < \|\alpha\| < 1/b$, the claim follows from (\ref{eqn: log expansion in terms of zeta}). 
\end{proof}

\begin{lem}\label{lem: bound for phi}
	If $\alpha$ and $0\leq \delta \leq 2/3b$ are real numbers with $\|\alpha\| \geq \delta$, then $\phi_b(\alpha) \leq \phi_b(\delta)$.
\end{lem}

\begin{proof}
	See Lemma 5 in \cite{MR}.
\end{proof}	

%%%%



\begin{lem}\label{lem: bound for product of pair}
	Let $\alpha,\beta, \gamma$ be real numbers. Then either 
	$$
	\left\|\alpha\left(b^{\beta} + b^{\gamma + 1}\right) \right\| \geq \dfrac{\|\alpha(b^2-1)b^{\gamma}\|}{b+1} 
	$$
	or
	$$
	\left\| \alpha \left(b^{\beta + 1} + b^{\gamma}\right)\right\|\geq \dfrac{\|\alpha(b^2-1)b^{\gamma}\|}{b+1} 
	$$
	(or both).
	Moreover
	$$
	\phi_b\left(\alpha\left(b^{\beta} + b^{\gamma + 1}\right)\right)\phi_b\left(\alpha \left(b^{\beta + 1} + b^{\gamma}\right)\right) \leq b \phi_b \left(\dfrac{\|\alpha(b^2-1)b^\gamma\|}{b+1}\right).
	$$	
\end{lem}

\begin{proof}
	We closely follow the argument in Lemma 6 of Mauduit-Rivat \cite{MR}.
	Making the substitutions $u = \alpha(b^\beta + b^{\gamma+1})$ and $\delta = \|\alpha(b^2-1)b^\gamma\|/(b+1)$, we need to show that either $\|u\| \geq \delta$ or $\|bu - \alpha(b^2-1)b^{\gamma}\| \geq \delta$, and that 
	\begin{equation}\label{eqn: product ineq}
		\phi_b(u)\phi_b(bu - \alpha (b^2-1)b^\gamma) \leq b\phi_b(\delta).
	\end{equation}
	Suppose $\|bu - \alpha(b^2-1)b^\gamma\| < \delta$. The triangle inequality yields
	\begin{align*}
		b\|u\| \geq \|b u\| &= \left\| bu - \alpha(b^2-1)b^\gamma + \alpha(b^2-1)b^\gamma\right\| \\
		&\geq  \left\| \alpha(b^2-1)b^\gamma\right\| - \left\| bu - \alpha(b^2-1)b^\gamma\right\| \\
		&\geq (b+1)\delta - \delta = b\delta.
	\end{align*}
	Thus $\|u\| \geq \delta$. Consequently either $\|u\| \geq \delta$ or $\|bu -\alpha(b^2-1)b^\gamma\|\geq \delta$, while (\ref{eqn: product ineq}) now follows from Lemma \ref{lem: bound for phi}.
\end{proof}	

\begin{lem}[Weyl product bound]\label{lem: bound for product using KH and ET}
	There exists an absolute constant $A > 0$ such that, for any real numbers $\alpha_1, \ldots, \alpha_N$ with $N \geq 1$ an integer, 
	$$
	\prod_{1 \leq n \leq N} \phi_b(\alpha_n) \leq \left(\dfrac{2}{D_N^*} \right)^{Ab N D_N^* },
	$$
	where $D_N^* = D_N^*(\alpha_1, \ldots, \alpha_N)$ is their star-discrepancy as defined in (\ref{eqn: star discrepancy}).
\end{lem}

\begin{proof}
	We adapt an argument of Aistleitner et al  \cite{Aistleitner} (see the proof of Theorem 1 there).  
	Let $0 < \epsilon \leq 1/2b$ to be chosen later. For a real number $x$, we define 
	$$
	\phi_{b,\epsilon}(x) := \begin{cases}
	\sin(\pi b \epsilon)/|\sin(\pi x)| &\mbox{ if } |x - \frac{k}{b}| < \epsilon \text{ for some integer $k \not\equiv 0 (b)$}\\
	\phi_b(x) &\mbox{ otherwise.}
	\end{cases}
	$$
	Clearly $\phi_{b,\epsilon}$ is $1$-periodic and continuous satisfying $\phi_{b,\epsilon}(x) \asymp_{b,\epsilon} 1$ and $\phi_{b,\epsilon}(x) \geq \phi_{b}(x)$ for every $x$. In particular
	$$
	\prod_{1 \leq n \leq N} \phi_b(\alpha_n) \leq 
	\prod_{1 \leq n \leq N} \phi_{b,\epsilon}(\alpha_n).
	$$
	Denoting by $P_N$ the product on the right hand side above, taking logarithms and applying the Koksma-Hlawka inequality (Lemma \ref{lem: Koksma-Hlawka}) one has
	$$
	\dfrac{\log P_N}{N} = \dfrac{1}{N}\sum_{1 \leq n \leq N} \log \phi_{b,\epsilon}(\alpha_n) \leq \int_0^1 \log\phi_{b,\epsilon}(x)dx +  V(\log \circ \phi_{b,\epsilon}) D^*_N,
	$$
	where $V(\log \circ \phi_{b,\epsilon})$ is the total variation of $\log \circ \phi_{b,\epsilon} $ on $[0,1]$. 
	It satisfies
	\begin{align*}
	V(\log \circ \phi_{b,\epsilon}) &\leq 
	2\pi \int_{0}^\epsilon \left|b \cot(\pi b x) - \cot(\pi x)\right| dx+
	2\pi \int_{\epsilon} ^{1/2} \cot(\pi x)dx\\
	&\qquad  + 
	\pi b\sum_{0 \leq k < b}\int_{\frac{k}{b} + \epsilon}^{\frac{k+1}{b} - \epsilon} |\cot(\pi b x)|dx\\
	&\ll b \log (1/b\epsilon). 
	\end{align*}
	Note we used $|\cot(\pi x)| = \cot(\pi \|x\|)  $ and $\cot(x) = x^{-1} + O(x)$ for $0 < |x| \leq \pi/2$.   
	
	Splitting the range of integration according to the definition of $\phi_{b,\epsilon}$, using $\phi_b(x) \leq b$, the additivity of the logarithm, and ignoring the terms $\log |\sin(\pi b \epsilon)|$ (these are non-positive) one derives 
	\begin{align*}
	\int_0^1 \log \phi_{b,\epsilon}(x)dx &\leq 2\epsilon \log b +  2\int_{\epsilon}^{1/2} \log \csc(\pi x)dx + \sum_{0 \leq k < b} \int_{\frac{k}{b} + \epsilon}^{\frac{k+1}{b} - \epsilon} \log |\sin(\pi b x)|dx\\
	&=2\epsilon \log b +  2\int_{\epsilon}^{b\epsilon} \log \csc(\pi x)dx\\
	&\ll b\epsilon \log(1/b \epsilon). 
	\end{align*}
	The result then follows if we let $\epsilon = D_N^*/2b $. 
\end{proof}




Our final tool is the following version of Vinogradov's lemma. 

\begin{lem}[Vinogradov-type lemma]\label{lem: vinogradov}
	Let $A,B, \theta$ be real numbers, $A,B > 0$, and let $q\geq 2$ be an integer. Then
	$$
	\sum_{n(q)} \min\left(A, B\csc^2\left(\pi \dfrac{n + \theta}{q}\right)\right) \leq \min\left(A, B\csc^2\left(\dfrac{\pi \|\theta\|}{q}\right)\right) + \left(1 - \dfrac{4}{\pi^2}\right)Bq^2.
	$$
\end{lem}

\begin{proof}
	We can write $\theta = m + r$ for some integer $m$ and real number $-1/2 < r \leq 1/2$. In fact $|r| = \|\theta\|$. Since $m + n$ uniquely covers each residue class modulo $q$ when so does $n$, the sum equals
	$$
	S := \sum_{0 \leq n < q} \min\left(A, B\csc^2\left(\pi \dfrac{n + r}{q}\right)\right).
	$$
	If $r < 0$, we may replace $n$ with $-n$ in the summand. Thus without loss of generality we may assume that $r \geq 0$. Now note
	$$
	S \leq \min\left(A, B\csc^2\left( \dfrac{\pi r}{q}\right)\right) + B H_q(r), 
	$$
	where for $0 \leq r \leq 1/2$, 
	$$
	H_q(r) :=  \sum_{1 \leq n < q} \csc^2\left(\pi \dfrac{n + r}{q} \right). 
	$$
	Since $x \mapsto \csc^2(\pi x)$ is convex in $0 < x < 1$, so is $H_q(r)$ in $0 \leq r \leq 1/2$. From the well-known identity
	$$
	\pi^2\csc^2(\pi x) = \sum_{n \in \mathbb{Z}} \dfrac{1}{(n + x)^2}
	$$
	for $x \not\in \mathbb{Z}$,
	one derives (see for instance \cite{Hofbauer})
	$$
	H_q(r) = \begin{cases}
	(q^2-1)/3 &\mbox{ if } r=0, \\
	q^2 \csc^2(\pi r) - \csc^2(\pi r/q) &\mbox{ if } 0 < r \leq 1/2.
	\end{cases}
	$$
	One can show that $H_q(0)\leq H_q(1/2)$ (say by using the inequality $\sin(\pi x) \geq 2^{3/2} x$ valid for $0 \leq x \leq 1/4$) and thus $H_q(r) \leq H_q(1/2)$ for $0 \leq r \leq 1/2$ by the convexity of $H_q(r)$ in $[0,1/2]$. We also have $H_q(1/2) = q^2 - \csc^2(\pi/2q) \leq q^2(1 - 4/\pi^2)$ since $\sin(x) \leq x$ for $x \geq 0$. The result now follows.
\end{proof}

\section{Exponential sums over $\mathscr{P}_b^0(x)$}\label{sec: exp sums}

The current state of knowledge on exponential sums over palindromes, specifically on bounds for these, is rather limited. To our understanding, the literature on this topic is thus far comprised of the works of Banks-Hart-Sakata \cite{Banks-Hart-Sakata}, Banks-Shparlinski \cite{Banks-Shparlinski} and Col \cite{Col}.
Here we give a brief exposition of what is known, as well as add slightly to the literature via the results in Proposition \ref{prop: product bound for modulus p} and Corollary \ref{cor: bound using product of Bourgain sequence} for special prime moduli. These are consequences of a result of Bourgain \cite{Bourg} giving upper bounds for certain types of exponential sums over finite fields with prime order. 

For the sake of simplicity, and indeed, for our purposes, we focus exclusively on exponential sums over $\mathscr{P}_b^0(x)$; that is, over $b$-palindromes, at most $x$ in size, with an odd number of digits in their $b$-adic expansion. Sums over palindromes with an even number of digits are rather similar in nature and all results here extend naturally, albeit with minor differences, to these.

The starting point, just as that in the works \cite{Banks-Hart-Sakata}, \cite{Col}, is that such sums can essentially be decomposed into linear combinations of objects enjoying rather useful multiplicative and algebraic structures.
These properties are crucial in the upcoming sections. In this regard we have the following lemma, implicit in the works \cite{Banks-Hart-Sakata}, \cite{Col}. We give a proof for the convenience of the reader.




\begin{lem}\label{lem: bound for exp sum of pals}
	For any $\alpha,x \in \mathbb{R}$, $x \geq 1$,
	$$
	\left| \sum_{n \in \mathscr{P}_b^0(x)} e(\alpha n)\right| \leq b^2 \sum_{0 \leq N \leq \frac{1}{2}\log_b x} \sum_{0 \leq M \leq N} \Phi_{M}\left(\alpha b^{N-M}\right),
	$$
	where $\Phi_M$ is as defined in (\ref{eqn: def of Phi}).
\end{lem}

\begin{proof}
	From the definition of $\mathscr{P}_b^0(x)$ and the triangle inequality, we have
	$$
	\left| \sum_{n \in \mathscr{P}^0_b(x)} e(\alpha n)\right| \leq \sum_{0 \leq N \leq \frac{1}{2} \log_b x} \left|\sum_{\substack{n \in \Pi_b(2N) \\ n\leq x}}e(\alpha n)\right|, 
	$$
	where
	\begin{equation}\label{eqn: def of Pi}
		\Pi_b(2N) := \left\{n \in \mathscr{P}_b \ : \ \lfloor \log_b n \rfloor = 2N\right\} = \mathscr{P}_b \cap [b^{2N}, b^{2N + 1}).
	\end{equation}
	The sum inside the vertical brackets equals
	$$
	\sum_{\substack{n \in \Pi_b(2N) \\ n\leq y}}e(\alpha n),
	$$
	where $y$ is the largest palindrome in $\Pi_b(2N)$ satisfying $y \leq x$. If no such palindrome $y$ exists, the sum is trivially zero. Otherwise we may write 
	$$
	y = 	 y_N b^N + \sum_{0 \leq j < N} y_j \left(b^j + b^{2N-j}\right)
	$$ for some digits $0 \leq y_j < b$ with $y_0 > 0$. We now seek to establish the inequality 
	\begin{equation}\label{eqn: palind exp sum}
		\left|\sum_{\substack{n \in \Pi_b(2N) \\ n\leq y } } e(\alpha n)\right| \leq b^2 \sum_{0 \leq M \leq N} \Phi_{M}\left(\alpha b^{N-M}\right).
	\end{equation}  
	To this end, we first define for an integer $\lambda \geq 0$,	
	$$
	\Psi_\lambda(\alpha) := \sum_{0 \leq c_N < b} e\left(\alpha c_N b^N\right) \prod_{\lambda < n < N} \sum_{0 \leq c < b} e\left(\alpha c \left(b^n + b^{2N-n}\right)\right).
	$$ 
	We have
	\begin{align*}
		&\sum_{\substack{n \in \Pi_b(2N) \\ n\leq y } } e(\alpha n) \\
		&= \Psi_0(\alpha) \sum_{1 \leq c < y_0} e\left(\alpha c\left(1 + b^{2N}\right)\right) \\
		&
		+ \Psi_1(\alpha) e\left(\alpha y_0\left(1 + b^{2N}\right)\right) 
		\sum_{0 \leq c < y_1}e\left(\alpha c \left(b + b^{2N-1}\right)\right) \\
		&+ 
		\Psi_2(\alpha) e\left(\alpha y_0\left(1 + b^{2N}\right) + \alpha y_1 \left(b + b^{2N-1}\right)\right) \sum_{0 \leq c < y_2} e\left(\alpha c\left(b^2 + b^{2N-2}\right)\right) \\
		&+\cdots\\
		&+
		\Psi_{N-1}(\alpha) e\left(\sum_{0 \leq j < N-1} y_j \left( b^j + b^{2N-j}\right)\right)\sum_{0 \leq c < y_{N-1}} e\left(\alpha c\left(b^{N-1} + b^{2N-(N-1)}\right)\right)\\
		&+
		e\left(\sum_{0 \leq j < N} y_j \left( b^j + b^{2N-j}\right)\right) \sum_{0 \leq c \leq y_N } e\left(\alpha c b^N\right). 
	\end{align*}
	Now the result follows when we note that 
	$$
	\prod_{\lambda < n < N} \sum_{0 \leq c < b} e\left(\alpha c \left(b^n + b^{2N-n}\right)\right) = \prod_{1 \leq n < N - \lambda} \sum_{0 \leq c < b} e\left(\alpha b^{\lambda}  c \left(b^n + b^{2(N-\lambda)-n}\right)\right) 
	$$	
	and take absolute values. 
\end{proof}	

From the lemma above it is clear that to study upper bounds for sums such as $\sum_{n \in \mathscr{P}_b^0(x)}e(\alpha n)$, it is sufficient to study the products $\Phi_N$ or, more generally, products such as
\begin{equation}\label{def: P_M}
P_M(\alpha) := \prod_{1 \leq n \leq M} \phi_b\left(\alpha\left(b^n + b^{2N-n}\right)\right)
\end{equation}
with $M \leq 2N$.
Of course the product's definition depends also on $b,N$, but for the sake of brevity we omit these on the left hand side above. 

Let us now state Col's result in our context. This will be needed later on. We give a proof somewhat different to that of Col \cite{Col} in its initial steps, although it ultimately boils to considering lower bounds for sums such as $\sum_{n \leq M}\|ab^n/q\|^2$, just as in Col's argument.  





\begin{prop}[Col \citelist{\cite{Col}, Corollary 4}]\label{prop: Linfty bound}
	Let $h,q,k,M,N$ be integers with $q \geq 2$, $(q,h(b^3-b))=1$ and $0 \leq M \leq 2N$. Then for $P_M$ defined as in (\ref{def: P_M}),
	$$
	P_M\left(\dfrac{h}{q} + \dfrac{k}{b^3-b}\right) \ll_b b^{M} \exp\left(-\sigma_\infty(b) \dfrac{M}{\log q}\right)
	$$
	uniformly in $h,k,N$, where $\sigma_\infty(b) > 0$ is some value depending only on $b$.
\end{prop}

\begin{proof}
	We may assume that $M \geq \log_b q$ as otherwise the statement is trivial. 
	Now for the sake of brevity let $\alpha := \frac{h}{q} + \frac{k}{b^3-b}$. 
Grouping the factors in the product by pairs of adjacent factors, we have
	\begin{align*}
	P_M(\alpha) \leq b \prod_{1 \leq n <M} \sqrt{\phi_b\left(\alpha \left(b^n + b^{2N-n}\right)\right)\phi_b\left(\alpha\left(b^{n+1} + b^{2N - n-1}\right)\right)}.
	\end{align*}
	By Lemma \ref{lem: bound for product of pair} with $\beta = 2N - n - 1$ and $\gamma = n$, 
	$$
	\phi_b\left(\alpha \left(b^n + b^{2N-n}\right)\right)\phi_b\left(\alpha\left(b^{n+1} + b^{2N - n-1}\right)\right) \leq b\phi_b \left( \dfrac{\|\alpha(b^2-1)b^{n}\|}{b+1}\right).
	$$
	Thus
	$$
	P_M(\alpha) \leq b^{\frac{M-1}{2} + 1} \sqrt{\prod_{1 \leq n < M} \phi_b \left(\dfrac{\|\alpha (b^2-1)b^n\|}{b+1}\right)}.
	$$
	By Lemma \ref{lem: exponential bound} and the $1$-periodicity of $\|\cdot\|$, the above is
	\begin{align*}
		\ll b^M \exp\left(- \dfrac{\pi^2(b-1)}{12(b+1)}\sum_{1 \leq n \leq M} \left\| \dfrac{h(b^2-1)b^n}{q}\right\|^2\right) 
	\end{align*}
	uniformly in $k,N$. Since $q \geq 2$ and $(q,h(b^3-b))=1$ by assumption, each term in the sum above is $\geq 1/q^2$. In particular the result holds if $q < b$. Consider now the case when $q \geq b$. Here we argue as done by  Maynard \cite{Maynard} in Lemma 10.1 there. 
	Dividing the sum above into sums over segments of length $\lfloor \log_b q\rfloor$, we observe it is
	$$
	\geq \left\lfloor \dfrac{M}{\log_b 	q}\right\rfloor \min_{(a,q)=1}\sum_{1 \leq n \leq \log_b q} \left\| \dfrac{a b^n}{q}\right\|^2.
	$$
	Note we used the assumption $(q,h(b^3-b))=1$. 
	Since $q \geq 2$ and $(q,ab)=1$, we have $\|ab^n/q\| \geq 1/q$ for each $n$. Moreover if $\|ab^n/q\| \leq 1/2b$, then $\|ab^{n+1}/q\| = b\|ab^n/q\|$. Consequently there exists an integer $1 \leq n \leq \log_b q$ for which $\|ab^n/q\| \geq 1/2b^2$. Hence the sum is $\geq 1/4b^4$ and the result follows. 
\end{proof}


The bound due to Col \cite{Col} in Proposition \ref{prop: Linfty bound} is non-trivial in the range $q \ll e^{o(M)}$ and is rather general in the sense of the lack of restrictions on the arithmetic shape of $q$. As such, it considerably improved over the corresponding general bound due to Banks-Hart-Sakata \cite{Banks-Hart-Sakata} (see Lemma 3.2 there) non-trivial only in the range $q = o(\sqrt{M})$ (strictly speaking, the bounds of Banks-Hart-Sakata were essentially given for products such as $\Phi_N$ and not the product over $1 \leq n \leq M$ in Proposition \ref{prop: Linfty bound}, but their work easily extends to this).

In the range $q \ll M^2$ with $(q,b)=1$ satisfying $\tau(q)\sqrt{q} \ll \text{ord}_q(b) \ll M$ (with suitable implied constants) and $P^-(q) > b$, the arguments of Banks-Hart-Sakata \cite{Banks-Hart-Sakata} (see Lemma 3.1 there) yield power-saving bounds of the form $P_M(h/q)\ll b^{\delta M}$ with $0 < \delta < 1$ fixed or depending only on $b$. Although applicable only on the much smaller range $q \ll M^2$, this is superior in strength over that in Proposition \ref{prop: Linfty bound} for such $q$. Indeed, the latter attains (at best) power-savings only for $q \ll 1$. In few words, their argument can be explained as follows:

Assume $(h,q) = 1$. Squaring $P_M(h/q)$, applying the inequality of the arithmetic and geometric means, expanding the square and switching orders of summation, all in turn, one has
\begin{align}
P_M^{2/M}(h/q) &=  \sum_{0 \leq c_1, c_2 < b}\dfrac{1}{M}\sum_{1 \leq n \leq M} e_q\left(h(c_1-c_2)\left(b^n + b^{2N-n}\right)\right) \label{eqn: AGM bound}\\
&= 
 b + \sum_{0 \leq c_1 \neq c_2 < b}\dfrac{1}{M}\sum_{1 \leq n \leq M} e_q\left(h(c_1-c_2)\left(b^n + b^{2N-n}\right)\right).\nonumber
\end{align}
If we impose the restriction $P^-(q) > b$, we have $(c_1 - c_2, q)=1$ for $0 \leq c_1 \neq c_2 < b$. In this case
\begin{equation}\label{eqn: bound for P_M}
P_M^{2/M}(h/q) \leq b + \dfrac{b^2}{M} \max_{a,k \in (\mathbb{Z}/q\mathbb{Z})^\times} S_q(M,a,k), 
\end{equation}
where
\begin{equation}\label{def: S}
S_q(M,a,k) := \left|\sum_{1 \leq n \leq M} e_q\left(ab^n + k\bar{b}^{n}\right)\right|.
\end{equation}
Here $\overline{b}$ denotes the multiplicative inverse of $b$ modulo $q$. 
Using well-known bounds for twisted Kloosterman sums, one derives (see Lemma 2.1 in \cite{Banks-Hart-Sakata})
$$
S_q(M,a,k) \leq \dfrac{M \tau(q) \sqrt{q} }{\text{ord}_q(b)} + \text{ord}_q(b).
$$
For this to be non-trivial, it is necessary that $\tau(q)\sqrt{q} \ll \text{ord}_q(b) \ll M$; hence their assumptions on $q$. One thus obtains
$$
P_M^{2/M}(h/q) \leq b + b^2 \left(\dfrac{\tau(q) \sqrt{q}}{\text{ord}_q(b)} + \dfrac{\text{ord}_q(b)}{M}\right).
$$

Banks-Shparlinski \cite{Banks-Shparlinski} also studied the sum $S_q(M,a,k)$ after the work of Banks-Hart-Sakata \cite{Banks-Hart-Sakata}
and obtained a bound for $P_M(h/p)$, with $p$ prime, essentially comparable to Col's result in Proposition \ref{prop: Linfty bound} for $q=p \ll M^2/\log^4 M$ with no restrictions on $\text{ord}_p(b)$; see their Theorem 6.

From the argument above, it is clear that improving bounds for $S_q(M,a,k)$ and/or enlarging the range of applicability of such bounds, may yield stronger results. However due to the presence of the $b$ term in (\ref{eqn: bound for P_M}) coming from the diagonal cases $c_1=c_2$ in (\ref{eqn: AGM bound}), one can never do better than $P_M \ll b^{M/2}$ via this argument, regardless of the quality of the bounds available for $S_q(M,a,k)$.  

Soon after the works of Banks-Hart-Sakata \cite{Banks-Hart-Sakata} and Banks-Shparlinski \cite{Banks-Shparlinski}, Bourgain \cite{Bourg} obtained the following result. First given a prime $p$, let $\mathbb{F}_p$ be the finite field with $p$ elements and let $\mathbb{F}_p^*$ be its subset of non-zero elements, endowed with the usual properties of $\mathbb{F}_p$. Given $\theta \in \mathbb{F}_p^*$, we use the notation $\text{ord}(\theta)$ to denote its multiplicative order. 

\begin{thm}[Bourgain \citelist{\cite{Bourg}, Theorem 2}]\label{thm: Bourgain}
	Let $\epsilon > 0$, let $p$ be a prime number and let $\theta_1, \ldots, \theta_r \in \mathbb{F}_p^*$ satisfying $\text{ord}(\theta_j) > p^\epsilon$ and $\text{ord}(\theta_i \theta_j^{-1}) > p^{\epsilon}$ for each $1 \leq i\neq j\leq r$. Then for any integer $N > p^\epsilon$ and any $a_1, \ldots, a_r \in \mathbb{F}_p^*$, 
	$$
	\left|\sum_{n=1}^N e_p\left(\sum_{j=1}^r a_j \theta_j^n \right)\right| < \dfrac{N}{p^\delta},
	$$
	where $0 < \delta = \delta(\epsilon) < 1$ is some value depending only on $\epsilon$. 
\end{thm}




We recognize that the sum, in Theorem \ref{thm: Bourgain}, with $r=2, a_1=a, a_2=k, \theta_1 = b, \theta_2 = \bar{b}$, $N=M$, is precisely $S_p(M,a,k)$. Thus we have the following. 

\begin{cor}\label{cor: cor of Bourgain}
	Let $M $ be an integer. Then for any prime $p$ satisfying $(p,b)=1$ and $p^{\epsilon} < \min(M, \text{ord}_p(b^2))$ for some $\epsilon > 0$, we have
$$
\max_{a,k \in (\mathbb{Z}/p\mathbb{Z})^\times} S_p(M,a,k) < \dfrac{M}{p^{\delta(\epsilon)}},
$$
where $\delta(\epsilon) > 0$ depends only on $\epsilon$, and $S_p(M,a,k)$ is defined as in (\ref{def: S}). 
\end{cor}

\begin{proof}
	Suffices to note $\text{ord}_p(b^{\pm 1}) \geq \text{ord}_p(b^{\pm 2}) > p^\epsilon$, the last holding by assumption.
\end{proof}

From this and (\ref{eqn: bound for P_M}) we may then obtain, under the same assumptions of Corollary \ref{cor: cor of Bourgain} with $p > b$, 
$$
P_M\left(\dfrac{h}{p}\right) \leq b^M \left(\dfrac{1}{b} + \dfrac{1}{p^{\delta}}\right)^{M/2}.
$$
One may improve this considerably, in Corollary \ref{cor: bound using product of Bourgain sequence} below, by combining use of Lemma \ref{lem: bound for product using KH and ET}, the Erd\H{o}s-T\'{u}ran inequality and Bourgain's bound. More generally we have the following. 





\begin{prop}\label{prop: product bound for modulus p}
With the same assumptions of Theorem \ref{thm: Bourgain},
$$
\prod_{1 \leq n \leq N} \phi_b\left(\dfrac{1}{p} \sum_{j=1}^r a_j \theta_j^n \right) \leq \exp\left( A(\epsilon) b\dfrac{ N  }{p^{\delta(\epsilon)}}\right),
$$
where $A(\epsilon), \delta(\epsilon) > 0$ are some values depending only on $\epsilon$.
\end{prop}

\begin{proof}
	For each integer $1 \leq n \leq N$, let 
	$$
	\alpha_n = \dfrac{1}{p} \sum_{j=1}^r a_j \theta_j^n.
	$$
	By Lemma \ref{lem: bound for product using KH and ET}, 
	$$
	\prod_{1 \leq n \leq N} \phi_b(\alpha_n) \leq \exp\left(AbND_N^* \log\left(\dfrac{2}{D_N^*}\right)\right) 
	$$
	for some absolute constant $A > 0$, where $D_N^*$ is the star-discrepancy of $(\alpha_n)_{n=1}^N$. By the Erd\H{o}s-Tur\'{a}n inequality (Lemma \ref{lem: erdos-turan}) and the fact $D_N^* \leq D_N$, we have
	$$
	D_N^* \ll \dfrac{1}{H} + \sum_{1\leq h \leq H} \dfrac{1}{h} \left|\dfrac{1}{N} \sum_{1 \leq n \leq N}e(h\alpha_n)\right| 
	$$
	for any $H \geq 2$. The contribution of the terms with $p \mid h$ to the series above is trivially $\ll \log(H)/p$. By Bourgain's bound in Theorem \ref{thm: Bourgain}, the terms with $p \nmid h$ contribute $\ll \log(H)/p^{\delta}$ to the sum, where $0 < \delta = \delta(\epsilon) < 1 $ is some value depending only on $\epsilon$. Thus
	$$
	D_N^* \ll \dfrac{1}{H} + \dfrac{\log H}{p^\delta} \ll \dfrac{\log p}{p^{\delta}} 
	$$
	after letting $H = p$. Hence
	$$
	D_N^* \log(2/D_N^*) \ll \dfrac{\log^2 p}{p^{\delta}} \ll_\delta p^{-\delta/2}
	$$
	and the result follows.
\end{proof}

Specializing to the palindromes we have the following immediate consequence.

\begin{cor}\label{cor: bound using product of Bourgain sequence}
	With the same assumptions of Corollary \ref{cor: cor of Bourgain} and $M\leq 2N$ there, we have
	$$
	\max_{(h,p)=1}\prod_{1 \leq n \leq M} \phi_b\left(\dfrac{h}{p} \left(b^n + b^{2N-n}\right)\right) \leq \exp\left( A(\epsilon)b\dfrac{ M  }{p^{\delta(\epsilon)}}\right),
	$$
	where $A(\epsilon), \delta(\epsilon) > 0$ are some values depending only on $\epsilon$.
	\end{cor}
  





  






\section{Bounding moments of $\Phi_N$}\label{sec: moments}

In this section we prove Proposition \ref{prop:2K-th moment2}. 


\begin{prop}[$2K$-th moment]\label{prop:2K-th moment2}
	For any integers $N,K,b \geq 2$, we have
\begin{equation}\label{eqn: 2K integral}
\int_{0}^1 \Phi_N^{2K}(\alpha) d\alpha \leq b^{2(K-1)N + 2} \left(1 + O\left(\dfrac{1}{\sqrt{K}} + \dfrac{b^2}{K}\right)\right)^{2N}.
\end{equation}
	
If $\alpha_1, \ldots, \alpha_R \in \mathbb{R}/\mathbb{Z}$ are $\delta$-spaced for some $0 < \delta \leq 1/2$, then
\begin{equation}\label{eqn: 2K sieve sum}
\sum_{r=1}^R \Phi_N^{2K}\left(\alpha_r\right) \leq \left(\delta^{-1} + Kb^{2N}\right)b^{2(K-1)N + 2}\left(1 + O\left(\dfrac{1}{\sqrt{K}} + \dfrac{b^2}{K}\right)\right)^{2N}.
\end{equation}
In particular for any $Q \geq 1$ and uniformly in $\beta \in \mathbb{R/\mathbb{Z}}$,
	\begin{equation}\label{eqn: sieve fractions}
	\sum_{q \leq Q} \sum_{h(q)}^* \Phi_N^{2K}\left(\dfrac{h}{q} + \beta\right) \leq \left(Q^2 + Kb^{2N}\right)b^{2(K-1)N + 2} \left(1 + O\left(\dfrac{1}{\sqrt{K}} + \dfrac{b^2}{K}\right)\right)^{2N}.
	\end{equation}
\end{prop}

\begin{proof}
	We first note that (\ref{eqn: sieve fractions}) follows from (\ref{eqn: 2K sieve sum}) since the points $h/q + \beta$ with $1 \leq h \leq q \leq Q$ and $(h,q)=1$ are $Q^{-2}$-spaced modulo $1$. Indeed, for any two distinct such fractions $h_1/q_1, h_2/q_2$, we have
	$$
	\left\| \dfrac{h_1}{q_1} - \dfrac{h_2}{q_2}\right\| \geq \dfrac{1}{q_1q_2} \geq \dfrac{1}{Q^2}.
	$$
	
	
	Let us now set $\psi_b(\alpha) := \sum_{0 \leq m < b}e(\alpha m)$. By definition,
	$$
	\Phi_N^{2K}(\alpha) = \left|\prod_{1 \leq n < N}\psi_b^K\left(\alpha \left(b^n + b^{2N-n}\right)\right)\right|^2.
	$$
	We have
	$$
	\prod_{1 \leq n < N} \psi_b\left(\alpha \left(b^n + b^{2N-n}\right)\right) = \sum_{0 \leq c_1, \ldots,c_{N-1} < b }e\left(\alpha  \sum_{1 \leq n < N}c_n \left(b^n + b^{2N-n}\right)\right)
	$$
	and one observes
	\begin{align*}
	&\prod_{1 \leq n < N} \psi_b^K\left(\alpha \left(b^n + b^{2N-n}\right)\right)\\
	&= \sum_{0 \leq v_1, \ldots, v_{N-1} \leq (b-1)K} \left(\prod_{1 \leq n < N} r(v_n; K,b) \right)e\left(\alpha \sum_{1 \leq m < N} v_m \left(b^m + b^{2N-m}\right)\right),
	\end{align*}
	where $r(v;K,b)$ is defined as in (\ref{eqn: num of compo}). We can rewrite this as 
	$$
	\prod_{1 \leq n < N} \psi_b^K\left(\alpha \left(b^n + b^{2N-n}\right)\right) = \sum_{0 \leq \ell \leq Kb^{2N}} a_\ell e(\alpha \ell),
	$$
	where
	$$
	a_\ell := \sum_{\substack{0 \leq v_1, \ldots, v_{N-1} \leq (b-1)K \\ \sum_{1 \leq m < N} v_m(b^m + b^{2N-m}) = \ell}}\prod_{1 \leq n <N} r(v_n;K,b).
	$$
	Thus 
	$$
	\Phi_N^{2K}\left(\alpha \right) = \left|\sum_{0\leq \ell \leq Kb^{2N}}a_\ell e(\alpha \ell)\right|^2.
	$$
	Parseval's identity then gives
	$$
	\int_0^1 \Phi_N^{2K}\left(\alpha \right)d\alpha = \sum_{0\leq \ell \leq Kb^{2N}} a_\ell^2
	$$
	while the large sieve inequality (Lemma \ref{lem: large sieve}) yields
	$$
	\sum_{r=1}^R \Phi_N^{2K}\left(\alpha_r \right)\leq \left(\delta^{-1} + Kb^{2N}\right)\sum_{0\leq \ell \leq Kb^{2N}} a_\ell^2
	$$
	for any $\delta$-spaced points $\alpha_1, \ldots, \alpha_R \in \mathbb{R}/\mathbb{Z}$. Thus to prove the proposition we must show that
	\begin{equation}\label{eqn: bound for sum of squares}
	\sum_{0\leq \ell \leq Kb^{2N}} a_\ell^2 \leq b^{2(K-1)N + 2} \left(1 + O\left(\dfrac{1}{\sqrt{K}} + \dfrac{b^2}{K}\right)\right)^{2N}.
	\end{equation}
	
	
	Expanding the square and switching orders of summation, we have
	$$
	\sum_{0 \leq \ell \leq Kb^{2N}}a_\ell^2 = \sum_{\substack{0\leq u_1, \ldots, u_{N-1} \leq (b-1)K \\ 0 \leq v_1, \ldots, v_{N-1} \leq (b-1)K \\ \sum_{1 \leq m < N}(u_m - v_m)(b^m + b^{2N-m}) = 0}} \prod_{1 \leq n < N} r(u_n; K,b)r(v_n;K,b).
	$$
	It will be convenient later on to introduce extra variables $u_N,v_N$ to the sum. To this end, we claim that 
	\begin{align*}
	&\sum_{\substack{0\leq u_1, \ldots, u_{N-1} \leq (b-1)K \\ 0 \leq v_1, \ldots, v_{N-1} \leq (b-1)K \\ \sum_{1 \leq m < N}(u_m - v_m)(b^m + b^{2N-m}) = 0}} \prod_{1 \leq n < N} r(u_n; K,b)r(v_n;K,b)\\
	& 
	\leq \sum_{\substack{0\leq u_1, \ldots, u_{N} \leq (b-1)K \\ 0 \leq v_1, \ldots, v_{N} \leq (b-1)K \\ \sum_{1 \leq m \leq N}(u_m - v_m)(b^m + b^{2N-m}) = 0}} \prod_{1 \leq n \leq N} r(u_n; K,b)r(v_n;K,b).
	\end{align*}
	Indeed, any solution $(u_1', \ldots, u_{N-1}') \times (v_1', \ldots, v_{N-1}')$ to the equation 
	$$\sum_{1 \leq m < N} (u_m - v_m)(b^m + b^{2N-m}) = 0$$ yields the solution $(u_1', \ldots, u_{N-1}', 0) \times (v_1', \ldots, v_{N-1}', 0)$ to the equation $$\sum_{1 \leq m \leq N}(u_m-v_m)(b^m + b^{2N - m})=0.$$ Since $r(0;K,b)=1$ additionally, the claim follows. 
	
	
	
	
	
	For any integer $n$, Lemma \ref{lem:comp2} implies
	$$
	r(n;K,b) \leq  \dfrac{b^K\mathbf{1}_{0 \leq n \leq(b-1)K}}{\sqrt{(b^2-1)K}} \left(\sqrt{\dfrac{6}{\pi}}\exp\left(-\dfrac{6}{(b^2-1)K}\left(n - \dfrac{(b-1)K}{2}\right)^2 \right) + \dfrac{c}{K}\right) 
	$$
	for some absolute constant $c > 0$. We fix a smooth function $\nu : \mathbb{R} \to [0,1]$ compactly supported on $[-2,2]$ satisfying $\nu(t) = 1$ for $0 \leq t \leq 1$ and $\|\nu^{(j)}\|_\infty \ll_j 1$ for each $j\geq 0$. In particular $\mathbf{1}_{0 \leq n \leq (b-1)K} \leq \nu(n/(b-1)K)$ and
	$$
	r(n;K,b) \leq  \dfrac{b^K  \nu(\frac{n}{(b-1)K})}{\sqrt{(b^2-1)K}} \left(\sqrt{\dfrac{6}{\pi}}\exp\left(-\dfrac{6}{(b^2-1)K}\left(n - \dfrac{(b-1)K}{2}\right)^2 \right) + \dfrac{c}{K}\right).
	$$ 
	We can rewrite this as
	$$
	r(n;K,b) \leq  \dfrac{b^K}{\sqrt{(b^2-1)K}} \eta_{K,b} \left(\dfrac{n}{\sqrt{(b^2-1)K}}\right),
	$$
	where $\eta_{K,b} : \mathbb{R} \to \mathbb{R}^+_0$ is the smooth compactly supported function defined by 
	\begin{equation}\label{eqn: def2 of eta}
	\eta_{K,b}(t) = \left(\sqrt{\dfrac{6}{\pi}} \exp\left(-6\left(t - \dfrac{1}{2} \sqrt{\dfrac{(b-1)K}{b+1}}\right)^2\right) + \dfrac{c}{K} \right)\nu\left(t \sqrt{\dfrac{b+1}{(b-1)K}}\right).
	\end{equation}
	Then we have 
	$$
	\sum_{\substack{0\leq u_1, \ldots, u_{N} \leq (b-1)K \\ 0 \leq v_1, \ldots, v_{N} \leq (b-1)K \\ \sum_{1 \leq m \leq N}(u_m - v_m)(b^m + b^{2N-m}) = 0}} \prod_{1 \leq n \leq N} r(u_n; K,b)r(v_n;K,b) \leq \dfrac{b^{2KN}}{(b^2-1)^NK^N}J,
	$$
	where $J$ is the sum
	$$
	\sum_{\substack{ u_1, \ldots, u_{N} \in \mathbb{Z} \\ v_1, \ldots, v_{N} \in \mathbb{Z} \\ \sum_{1 \leq m \leq N}(u_m - v_m)(b^m + b^{2N-m}) = 0}} \prod_{1 \leq n \leq N} \eta_{K,b}\left(\dfrac{u_n}{\sqrt{(b^2-1)K}}\right) \eta_{K,b}\left(\dfrac{v_n}{\sqrt{(b^2-1)K}}\right).
	$$
	In view of (\ref{eqn: bound for sum of squares}) we complete the proof if we show that 
	\begin{equation}\label{eqn: desired bound}
	J \leq b^2 \dfrac{(b^2-1)^N K^N}{b^{2N}} \left(1 + O\left(\dfrac{1}{\sqrt{K}} + \dfrac{b^2}{K}\right)\right)^{2N}. 
	\end{equation}
	To this end we proceed as follows.
	
	An application of the identity 
	$$
	\mathbf{1}_{k=0} = \int_0^1 e(k\alpha) d\alpha
	$$
	valid for integers $k$, gives
	$$
	J = \int_{0}^1 \left|\prod_{1 \leq n \leq N}\sum_{v \in \mathbb{Z}} \eta_{K,b}\left(\dfrac{v}{\sqrt{(b^2-1)K}}\right)e\left(\alpha v \left(b^n + b^{2N-n}\right)\right)\right|^2d\alpha.
	$$
	Let us set
	\begin{equation}\label{def: A}
	A := \sqrt{(b^2-1)K}\|\eta_{K,b}\|_1 + \|\eta_{K,b}'\|_1.
	\end{equation}
	By the symmetry of $n \mapsto b^n + b^{2N-n}$ around $n = N$ and the bound 
	$$
	\left|\sum_{v \in \mathbb{Z}} \eta_{K,b}\left( \dfrac{v}{\sqrt{(b^2-1)K}} \right) e\left(2\alpha v b^N \right) \right| \leq A
	$$
	coming from (\ref{eqn: trivial bound}) in Lemma \ref{lem: smooth exp sums}, note
	$$
	J \leq A \int_0^1 \prod_{1 \leq n < 2N}\left|\sum_{v \in \mathbb{Z}}\eta_{K,b}\left(\dfrac{v}{\sqrt{(b^2-1)K}}\right)e\left(\alpha v\left(b^n + b^{2N-n}\right)\right)\right|d\alpha.
	$$
	Grouping by pairs of adjacent factors and bounding square roots of two endpoint factors, we have 
	\begin{align*}
	&\prod_{1 \leq n < 2N}\left|\sum_{v \in \mathbb{Z}}\eta_{K,b}\left(\dfrac{v}{\sqrt{(b^2-1)K}}\right)e\left(\alpha v\left(b^n + b^{2N-n}\right)\right)\right|\\
	&\leq
	A
	\prod_{1 \leq n \leq 2(N-1)}
	\sqrt{\left|
	\sum_{u \in \mathbb{Z}}\eta_{K,b}\left(\dfrac{u}{\sqrt{(b^2-1)K}}\right)e\left(\alpha u\left(b^n + b^{2N-n}\right)\right)\right|}\\
	&\qquad
	\times
	\sqrt{\left|\sum_{v \in \mathbb{Z}}\eta_{K,b}\left(\dfrac{v}{\sqrt{(b^2-1)K}}\right)e\left(\alpha v\left(b^{n+1} + b^{2N-n-1}\right)\right)
	\right|}.
	\end{align*}
	By Lemma \ref{lem: smooth exp sums} with $k=4$, the product over $1 \leq n \leq 2(N-1)$ is at most
	\begin{align*}
	&\prod_{1 \leq n \leq 2(N-1)} \sqrt{\min\left(A, \dfrac{\|\eta_{K,b}^{(4)}\|_1}{(b^2-1)^{3/2}K^{3/2}\sin^4(\pi \alpha(b^n + b^{2N-n}))}\right)}\\
	&\qquad 
	\times
	\sqrt{\min\left(A, \dfrac{\|\eta_{K,b}^{(4)}\|_1}{(b^2-1)^{3/2}K^{3/2}\sin^4(\pi \alpha(b^{n+1} + b^{2N-n-1}))}\right)}.
	\end{align*}
	By Lemma \ref{lem: bound for product of pair} with $\beta = 2N - n - 1$ and $\gamma = n$, the fact that $t \mapsto \sin(\pi t)$ is increasing on $[0,1/2]$ and $|\sin(\pi t)| = \sin(\pi \|t\|)$, we have either
	$$
	\left|\sin\left(\pi \alpha \left(b^n + b^{2N-n}\right)\right)\right| \geq \sin\left(\pi \dfrac{\|\alpha(b^2-1)b^n\|}{b+1}\right)
	$$
	or
	$$
	\left|\sin\left(\pi \alpha \left(b^{n+1} + b^{2N-n-1}\right)\right)\right| \geq \sin\left(\pi \dfrac{\|\alpha(b^2-1)b^n\|}{b+1}\right).
	$$
	Then the product above is
	\begin{align*}
	&\leq A^{N-1}
	\prod_{1 \leq n \leq 2(N-1)} \sqrt{\min\left(A, \dfrac{\|\eta_{K,b}^{(4)}\|_1}{(b^2-1)^{3/2}K^{3/2}\sin^4(\pi\|\alpha(b^2-1)b^n\|/(b+1))}\right)}\\
	&\leq
	A^{N-1}
	\prod_{1 \leq n \leq 2(N-1)} \sqrt{\min\left(A, \dfrac{(b+1)^4\|\eta_{K,b}^{(4)}\|_1}{(b^2-1)^{3/2}K^{3/2}\sin^4(\pi \alpha(b^2-1)b^n)}\right)}.
	\end{align*}
	Note in the last line we used the inequality $r \sin(\pi t/r) \geq \sin(\pi t)$ valid for $r \geq 1$ and $0 \leq t < 1$ (which may be shown to hold, say, via Euler's product formula). Thus
	\begin{align*}
	&\prod_{1 \leq n < 2N}\left|\sum_{v \in \mathbb{Z}}\eta_{K,b}\left(\dfrac{v}{\sqrt{(b^2-1)K}}\right)e\left(\alpha v\left(b^n + b^{2N-n}\right)\right)\right|\\
	&
	\leq
	A^{N}
	\prod_{1 \leq n \leq 2(N-1)} \sqrt{\min\left(A, \dfrac{(b+1)^4\|\eta_{K,b}^{(4)}\|_1}{(b^2-1)^{3/2}K^{3/2}\sin^4(\pi \alpha(b^2-1)b^n)}\right)}.
	\end{align*}
	Inserting this in the integral, substituting $\alpha b(b^2-1)$ with $\alpha$ and using the $1$-periodicity of the integrand, we obtain
	\begin{align*}
	J &\leq A^{N+1}
	\int_0^1 \prod_{0 \leq n < 2(N-1)}\sqrt{\min\left(A, \dfrac{(b+1)^4\|\eta_{K,b}^{(4)}\|_1}{(b^2-1)^{3/2}K^{3/2}\sin^4(\pi \alpha b^n)}\right)}d\alpha.
	\end{align*}
	By Lemma \ref{lem: integral of product} and the fact $\sqrt{\min(A,B)} = \min(\sqrt{A}, \sqrt{B})$ for $A,B \geq 0$, this integral is
	$$
	\leq \left(\sup_{\theta \in \mathbb{R}} \dfrac{1}{b} \sum_{n (b)} \min\left(A^{1/2},
	\dfrac{(b+1)^2\|\eta_{K,b}^{(4)}\|_1^{1/2}}{(b^2-1)^{3/4}K^{3/4}\sin^2(\pi (n+\theta)/b)}
	\right)\right)^{2(N-1)}
	$$  
	while Lemma \ref{lem: vinogradov} implies this is
	$$
	\leq \left(\dfrac{A^{1/2}}{b} + \dfrac{b(b+1)^2\|\eta_{K,b}^{(4)}\|_1^{1/2}}{(b^2-1)^{3/4}K^{3/4}} \right)^{2(N-1)}.
	$$
	Factoring $\frac{1}{b}(b^2-1)^{1/4}K^{1/4}$ out in the estimate, it follows
	$$
	J \leq b^2\dfrac{(b^2-1)^N K^N}{b^{2N}}\mathcal{E}(N,K,b),
	$$
	where
	\begin{align*}
	&\mathcal{E}(N,K,b)\\
	&:= \left(\dfrac{A}{\sqrt{(b^2-1)K}}\right)^{N+1}\left(\dfrac{A^{1/2}}{(b^2-1)^{1/4}K^{1/4}} + 
	\dfrac{b^2(b+1)^2\|\eta_{K,b}^{(4)}\|_1^{1/2}}{(b^2-1)K}
	\right)^{2(N-1)}\\
	&\leq
	\left(\dfrac{A}{\sqrt{(b^2-1)K}}\right)^{N}\left(\dfrac{A^{1/2}}{(b^2-1)^{1/4}K^{1/4}} + 
	\dfrac{3b^2\|\eta_{K,b}^{(4)}\|_1^{1/2}}{K}
	\right)^{2N}\\
	&=
	\left(\|\eta_{K,b}\|_1 + \dfrac{\|\eta_{K,b}'\|_1}{\sqrt{(b^2-1)K}}
	+
	\dfrac{3b^2\|\eta_{K,b}^{(4)}\|_1^{1/2}}{K}\sqrt{\|\eta_{K,b}\|_1 + \dfrac{\|\eta_{K,b}'\|_1}{\sqrt{(b^2-1)K}}}
	\right)^{2N}
	\end{align*}
	for $b\geq 2$.
	The last holds by the definition of $A$ in (\ref{def: A}). From the definition of $\eta_{K,b}$ in (\ref{eqn: def2 of eta}) and the assumptions on $\nu$, we have $\|\eta_{K,b}'\|_1, \|\eta_{K,b}^{(4)}\|_1 \ll 1$ uniformly in $K,b$. Moreover the fact $\int_{\mathbb{R}}e^{-t^2}dt = \sqrt{\pi}$ and the assumption $\nu \leq 1$ imply
	$$
	\|\eta_{K,b}\|_1 \leq 1 +  \dfrac{c\|\nu\|_1}{\sqrt{K}} = 1 + O\left(\dfrac{1}{\sqrt{K}}\right). 
	$$
	Hence
	$$
	\mathcal{E}(N,K,b) \leq \left(1 + O\left(\dfrac{1}{\sqrt{K}} + \dfrac{b^2}{K}\right)\right)^{2N}
	$$
	and
	$$
	J \leq b^2 \dfrac{(b^2-1)^N K^N}{b^{2N}} \left(1 + O\left(\dfrac{1}{\sqrt{K}} + \dfrac{b^2}{K}\right)\right)^{2N}.
	$$
	We have shown that (\ref{eqn: desired bound}) holds and the proof is thus complete.
\end{proof}	

\section{Bounding the average}\label{sec: average bound}


In this section we prove Proposition \ref{prop: average bound} combining results from the previous two sections. 

\begin{prop}
	[$L^1/L^2/L^{2K}/L^\infty$ hybrid bound]\label{prop: average bound}
	Let $N\geq 0$ be an integer and let $Q\geq 1$. For any $0 < \epsilon \leq \frac{1}{15}$ and $\frac{1}{3} \leq \delta \leq \frac{2}{5} - \epsilon$, 
	\begin{equation}\label{eqn: general L1 bound}
		\sup_{\beta \in \mathbb{R}}\sum_{\substack{q \leq Q \\ (q,b)=1}} \sum_{h(q)}^* \Phi_{N}\left(\dfrac{h}{q} + \beta\right) \ll_{b,\epsilon} Q^2b^{N(1 - \delta - \sigma_1(b,\epsilon))} + Q^{1 - \frac{\sigma_1(b,\epsilon)}{\delta}}b^N,
	\end{equation}
	where $\sigma_1(b,\epsilon) > 0$ is some value depending only on $b$ and $\epsilon$. 
	Moreover 
	\begin{equation}\label{eqn: L1 bound for r dividing b(b^2-1)}
		\sum_{\substack{2 \leq q \leq Q \\ (q,b^3-b)=1}} \sum_{h(q)}^* \Phi_{N}\left(\dfrac{h}{q} + \dfrac{k}{b^3-b}\right) \ll_{b,\epsilon} \left(Q^2b^{N(1 - \delta - \sigma_1(b,\epsilon))} + Q^{1 - \frac{\sigma_1(b,\epsilon)}{\delta}}b^N\right)e^{-\frac{\sigma_\infty(b)N}{\log Q}}
	\end{equation}
uniformly in $k \in \mathbb{Z}$, 
	where $\sigma_\infty(b) > 0$ is some value depending only on $b$.
\end{prop}

We will need the following bound. 

\begin{lem}[$L^2$-bound]\label{lem: L2 bound}
	Let $0 \leq L < M < N$ be integers and let $Q \geq 1$. Then
	$$
	\sup_{\beta \in \mathbb{R}}\sum_{q\leq Q} \sum_{h(q)} \prod_{L < n \leq M} \phi_b^2\left(\left(\dfrac{h}{q} + \beta\right)\left(b^n + b^{2N-n}\right)\right) \ll_\epsilon \left(Q + b^{M - L + \epsilon N}\right)Qb^{M-L}
	$$	
	for any $\epsilon > 0$.
\end{lem}	

\begin{proof}
	Expanding the square, switching orders of summation, using the orthogonality of the additive characters modulo $q$ and taking absolute values, we have that the left hand side above is
	$$
	\leq Q\sum_{\substack{0 \leq u_{L+1}, \ldots, u_{M} < b \\ 0\leq v_{L+1}, \ldots, v_{M} < b}} \sum_{\substack{q \leq Q \\ q \mid S(\mathbf{u}, \mathbf{v})}}1,
	$$
	where
	$$
	S(\mathbf{u}, \mathbf{v}) := \sum_{L < n \leq M} \left(u_n - v_n\right)\left(b^n + b^{2N-n}\right). 
	$$
	By the uniqueness of the representation of integers in base $b$, we have $S(\mathbf{u}, \mathbf{v}) =0$ if and only if $u_n = v_n$ for each $L < n\leq M$. The contribution to the overall sum of such (diagonal) terms is $\leq Q^2b^{M-L}$. For the off-diagonal terms with $u_n \neq v_n$ for some $n$, the inner sum is $\leq \tau(S(\mathbf{u}, \mathbf{v})) \ll_\epsilon b^{\epsilon N}$ for any $\epsilon > 0$, where $\tau(m) := \sum_{d \mid m }1$ is the divisor function. Thus these contribute $\ll_\epsilon Q b^{2(M-L) + \epsilon N}$ for any $\epsilon > 0$. 
\end{proof}


\begin{proof}[{\bf Proof of Proposition \ref{prop: average bound}}]
	We may assume that $N\geq 100$, say, as otherwise the statement is trivial. 
	Let us treat the cases when $Q \ll_{b,\epsilon} b^{\delta N}$ and $Q \gg_{b,\epsilon} b^{\delta N}$ separately. 
	
	We begin by considering the case when $Q \gg_{b,\epsilon} b^{\delta N}$. Here one can check that both right hand sides of (\ref{eqn: general L1 bound}) and (\ref{eqn: L1 bound for r dividing b(b^2-1)}) are $\gg_{b,\epsilon} Q^2b^{N(1-\delta - \sigma_1(b,\epsilon))}$ (in particular the exponential factor in (\ref{eqn: L1 bound for r dividing b(b^2-1)}) is $\asymp_{b,\epsilon} 1$ since $\delta \geq 1/3$ by assumption). Thus to prove (\ref{eqn: general L1 bound}) and (\ref{eqn: L1 bound for r dividing b(b^2-1)}) when $Q \gg_{b,\epsilon} b^{\delta N}$ it suffices to show that the left hand side of (\ref{eqn: general L1 bound}) is $\ll_{b,\epsilon} Q^2b^{N(1-\delta - \sigma_1(b,\epsilon))}$ in this case. To this end we proceed as follows.
	
	Let $M = \lfloor \delta N \rfloor$ and split the product as  
	$$
	\Phi_N\left(\dfrac{h}{q} + \beta\right) = P_1\left(\dfrac{h}{q}\right)P_2\left(\dfrac{h}{q}\right)P_3\left(\dfrac{h}{q}\right),
	$$	
	where
	\begin{align*}
		P_1\left(\dfrac{h}{q}\right) 
		&=
		\prod_{1 \leq n \leq M} \phi_b\left(\left(\dfrac{h}{q} + \beta\right)\left(b^n + b^{2N-n}\right)\right)
		,\\
		P_2\left(\dfrac{h}{q}\right) 
		&=
		\prod_{M < n \leq 2M}\phi_b\left(\left(\dfrac{h}{q} + \beta\right)\left(b^n + b^{2N-n}\right)\right)
		,\\
		P_3\left(\dfrac{h}{q}\right) 
		&=
		\prod_{2M < n < N}
		\phi_b\left(\left(\dfrac{h}{q} + \beta\right)\left(b^n + b^{2N-n}\right)\right)
		.
	\end{align*}
	Let $K \geq 2$ be an integer to be specified later and 
	set $\ell := 4K/(2K-1) > 2$. Note that $2/\ell + 1/2K= 1$. Then by  H\"{o}lder's inequality with the triple $(\ell,\ell,2K)$, we have
	\begin{align*}
	&\sum_{\substack{q \leq Q \\ (q,b)=1}}\sum_{h(q)}^* \Phi_N\left(\dfrac{h}{q} + \beta\right)\\ 
		& \leq 
		\left(\sum_{\substack{q \leq Q }}\sum_{h(q)}P_1^\ell\left(\dfrac{h}{q}\right) \right)^{1/\ell}\left(\sum_{\substack{q \leq Q }}\sum_{h(q)}P_2^\ell\left(\dfrac{h}{q}\right)\right)^{1/\ell}\left(\sum_{\substack{q \leq Q \\ (q,b)=1}}\sum_{h(q)}^* P_3^{2K}\left(\dfrac{h}{q}\right) \right)^{1/2K}.
	\end{align*}
	Note $P_1^\ell(h/q) = P_1^{\ell-2}(h/q) P_1^{2}(h/q) \leq b^{(\ell-2)M} P_1^{2}(h/q)$ while Lemma \ref{lem: L2 bound} gives
	$$
	\sum_{q \leq Q}\sum_{h(q)} P_1^2\left(\dfrac{h}{q}\right) \ll_\gamma \left(Q + b^{M  }\right)Qb^{M + \gamma N}
	$$
	for any $\gamma > 0$.
	Thus
	$$
	\left(\sum_{\substack{q \leq Q }}\sum_{h(q)}P_1^\ell\left(\dfrac{h}{q}\right) \right)^{1/\ell} \ll_\gamma \left(Q + b^{M }\right)^{1/\ell}Q^{1/\ell}b^{M(1 - 1/\ell) + \gamma N}
	$$
	for any $\gamma > 0$. 
	Similarly we can show that 
	$$
	\left(\sum_{\substack{q \leq Q }}\sum_{h(q)}P_2^\ell\left(\dfrac{h}{q}\right) \right)^{1/\ell} \ll_\gamma \left(Q + b^{M }\right)^{1/\ell}Q^{1/\ell}b^{M(1 - 1/\ell) + \gamma N}.
	$$
	
	Consider now the third sum with $P_3$. We have
	\begin{align*}
		\sum_{h(q)}^*P_3^{2K}\left(\dfrac{h}{q}\right) &= \sum_{h(q)}^* \prod_{1 \leq n < N-2M} \phi_b^{2K} \left( \left(\dfrac{h}{q} + \beta \right)\left(b^{2M + n} + b^{2N - 2M - n}\right)\right)\\
		&=
		\sum_{h(q)}^* \prod_{1 \leq n < N-2M} \phi_b^{2K} \left( \left(\dfrac{hb^{2M}}{q} + \beta b^{2M}\right) \left(b^{ n} + b^{2(N - 2M) - n}\right)\right)\\
		&= 
		\sum_{h(q)}^*\Phi_{N-2M}^{2K}\left(\dfrac{hb^{2M}}{q} + \beta b^{2M}\right) .
	\end{align*}
	If $(q,b)=1$, we may substitute $hb^{2M}$ with $h$ above. Thus
	$$
	\sum_{\substack{q \leq Q \\ (q,b)=1}}\sum_{h(q)}^*P_3^{2K}\left(\dfrac{h}{q}\right)
	= 
	\sum_{\substack{q \leq Q \\ (q,b)=1}}\sum_{h(q)}^*\Phi_{N-2M}^{2K}\left(\dfrac{h}{q} + \beta b^{2M}\right).
	$$	
	Removing now the constraint $(q,b)=1$ via positivity, Proposition \ref{prop:2K-th moment2} implies the above is
	$$
	\leq \left(Q + \sqrt{K}b^{N - 2M}\right)^2b^{2(K-1)(N - 2M) + 2} \left(1 + \dfrac{c}{\sqrt{K}} + \dfrac{cb^2}{K}\right)^{2(N-2M)}
	$$
	for some absolute constant $c > 0$.
	Multiplying the three bounds together and recalling the assumptions $M = \lfloor \delta N \rfloor$ and $2/\ell + 1/2K=1$ gives
	\begin{align}
		&\sum_{\substack{q \leq Q \\ (q,b)=1}}\sum_{h(q)}^* \Phi_N\left(\dfrac{h}{q} + \beta\right)\nonumber\\
		& 
		\ll_{b, \gamma} \left(Q + b^{M }\right)^{\frac{2}{\ell}}\left(Q + b^{N- 2M}\right)^{\frac{1}{K}}Q^{\frac{2}{\ell}}b^{N - \frac{N}{K} - \frac{2M}{\ell} + \frac{2M}{K} + \gamma N}\left(1 + \dfrac{cb^2}{K} + \dfrac{c}{\sqrt{K}}\right)^{\frac{N-2M}{K}}\nonumber\\
		&\ll_b \left(Q + b^{\delta N}\right)^{1-1/2K}\left(Q + b^{N(1-2\delta)}\right)^{1/K}Q^{1-1/2K} b^{N(1 - \delta - \frac{1}{K}L(\delta, b,K,\gamma))}\label{eqn: last line},
	\end{align}
	where
	\begin{align*}
		L(\delta,b,K,\gamma) &:= 1-\dfrac{5\delta}{2} - \gamma K - (1-2\delta)\log_b\left(1 + \dfrac{cb^2}{K} + \dfrac{c}{\sqrt{K}}\right)\\
		&\geq \dfrac{5\epsilon}{2}
		- \gamma K - \dfrac{1}{3}\log_b\left(1 + \dfrac{cb^2}{K} + \dfrac{c}{\sqrt{K}}\right).
	\end{align*}
	The last line holds by the assumption $\frac{1}{3} \leq \delta \leq \frac{2}{5} - \epsilon$. We may choose $K \asymp_{b,\epsilon} 1$ large enough and $\gamma \ll_{b,\epsilon} 1$ small enough so that 
	$$
	\gamma K + \dfrac{1}{3}\log_b\left(1 + \dfrac{cb^2}{K} + \dfrac{c}{\sqrt{K}}\right) \leq \dfrac{3\epsilon}{2},
	$$
	say. For any such choice, $L(\delta,b,K,\gamma) \geq \epsilon$. This and the  assumption $Q \gg_{b,\epsilon} b^{\delta N} \geq b^{(1-2\delta)N}$ for $\delta \geq 1/3$ implies (\ref{eqn: last line}) is
	$
	\ll_{b,\epsilon} Q^2 b^{N(1 - \delta - \epsilon/K)}.
	$
	Thus if we take $\sigma_1(b,\epsilon) = \epsilon/K$ we see that (\ref{eqn: general L1 bound}) and hence (\ref{eqn: L1 bound for r dividing b(b^2-1)}) hold in our case of $Q \gg_{b,\epsilon} b^{\delta N}$.
	
	Next we treat the case when $Q\ll_{b,\epsilon} b^{\delta N}$. Here one observes that the right hand sides of (\ref{eqn: general L1 bound}) and (\ref{eqn: L1 bound for r dividing b(b^2-1)}) are $\gg_{b,\epsilon} b^N Q^{1 - \frac{\sigma_1(b,\epsilon)}{\delta}}$ and $\gg_{b,\epsilon} b^N Q^{1 - \frac{\sigma_1(b,\epsilon)}{\delta}}e^{-\sigma_\infty(b)N/\log Q}$, respectively. It thus suffices to show that the left hand sides of (\ref{eqn: general L1 bound}) and (\ref{eqn: L1 bound for r dividing b(b^2-1)}) are $\ll_{b,\epsilon} b^N Q^{1 - \frac{\sigma_1(b,\epsilon)}{\delta}}$ and $\ll_{b,\epsilon} b^N Q^{1 - \frac{\sigma_1(b,\epsilon)}{\delta}}e^{-\sigma_\infty(b)N/\log Q}$, respectively. We treat these two in similar fashion (but with a small difference) as follows.
	
	Since $Q \ll_{b,\epsilon} b^{\delta N}$ by assumption, we can find an integer $1 \leq M < N-1$ such that $b^{\delta(N-M)} \asymp_{b,\epsilon} Q$. For any such choice of $M$, we split the products according to whether $n \leq M$ or $M < n< N$. This yields that the left hand sides of (\ref{eqn: general L1 bound})
	and (\ref{eqn: L1 bound for r dividing b(b^2-1)}) are
	$
	\leq b^{M}
	\mathscr{Z}
	$ (with $\mathscr{Z}$ defined below in (\ref{def: definition of Z})) and
	$$
	\leq \mathscr{Z}\max_{\substack{h \in \mathbb{Z} \\ 2 \leq q \leq Q \\ (q,h(b^3-b))=1}}\prod_{1 \leq n \leq M} \phi_b\left(\left(\dfrac{h}{q} + \dfrac{k}{b^3-b}\right)\left(b^n + b^{2N-n}\right)\right)
	$$
	respectively, 
	where
	\begin{equation}\label{def: definition of Z}
	\mathscr{Z} := \sup_{\beta \in \mathbb{R}}\sum_{\substack{q \leq Q \\ (q,b)=1}} \sum_{h(q)}^* \prod_{M < n < N} \phi_b\left(\left(\dfrac{h}{q} + \beta \right)\left(b^n + b^{2N-n}\right)\right).
	\end{equation}
	With regards to the product over $1 \leq n \leq M$ above, Proposition \ref{prop: Linfty bound} implies it is
	$$
	\ll_b b^M \exp\left(-\sigma_\infty(b) \dfrac{M}{\log Q}\right) \ll_{b,\epsilon} b^M \exp\left(-\sigma_\infty(b) \dfrac{N}{\log Q}\right).
	$$ 
	Note we used the assumption $b^{\delta(N-M)} \asymp_{b,\epsilon} Q$ implying $M = N - \frac{1}{\delta} \log_b Q + O_{b,\epsilon}(1)$. 
	Similarly as done previously, one can show that 
	$$
	\sum_{\substack{q \leq Q \\ (q,b)=1}} \sum_{h(q)}^* \prod_{M < n < N} \phi_b\left(\left(\dfrac{h}{q} + \beta \right)\left(b^n + b^{2N-n}\right)\right) = \sum_{\substack{q \leq Q \\ (q,b)=1}} \sum_{h(q)}^* \Phi_{N-M}\left(\dfrac{h}{q} + b^M \beta \right).
	$$
	Recall that $ b^{\delta(N-M)}\asymp_{b,\epsilon} Q $ by assumption. Since we already showed above that (\ref{eqn: general L1 bound}) holds for arbitrary $N \geq 1$ when $Q \gg_{b,\epsilon } b^{\delta N}$ (and in particular when $Q \asymp_{b,\epsilon} b^{\delta N}$) we may apply this to our case of $Q \asymp_{b,\epsilon} b^{\delta(N-M)}$ (substituting $N$ there with $N-M$ here). This shows 
	$$
	\mathscr{Z} \ll_{b,\epsilon} Q^2 b^{( N -M)(1 - \delta - \sigma_1(b,\epsilon))} + Q^{1-\frac{\sigma_1(b,\epsilon)}{\delta}} b^{N-M} \asymp_{b,\epsilon} Q^{1-\frac{\sigma_1(b,\epsilon)}{\delta}} b^{N-M}.
	$$
	It follows that (\ref{eqn: general L1 bound}) and (\ref{eqn: L1 bound for r dividing b(b^2-1)}) also hold when $Q \ll_{b,\epsilon} b^{\delta N}$ and we thus conclude the proof. 
\end{proof}	

\section{Equidistribution estimate}\label{sec: equi estimate}

In this section we prove Theorem \ref{thm: equidistribution}. First we need the following fact. 

\begin{lem}\label{lem: number of coprimes}
	For an integer $N\geq 0$, let $\Pi_b(2N)$ be as defined in (\ref{eqn: def of Pi}) and let 
	$$
	\Pi_b^*(2N) := \left\{ n \in \Pi_b(2N) \ : \ (n,b^3-b)=1\right\}. 
	$$
	We have $\#\Pi_b(2N) = (b-1)b^N$ and 
	\begin{equation}\label{bound: size of Pi star}
	\#\Pi_b^*(2N) = \gamma_2(b) \dfrac{\varphi(b^3-b)}{b^3-b} \#\Pi_b(2N) + O\left(b^2 \tau(b^2-1)\right), 
	\end{equation}
	where
	$$
	\gamma_2(b) := \begin{cases}
		b/(b-1) &\mbox{ if $b$ is even,} \\
		1 &\mbox{ otherwise.}
	\end{cases}
	$$
	Moreover $\#\mathscr{P}_b^*(x) \asymp_b \sqrt{x}$ for $x \geq 1$. 
\end{lem}

\begin{proof}
	Every integer $n \in \Pi_b(2N)$ can be written uniquely as 
	\begin{align*}
		n &= n_N b^N + \sum_{0 \leq j < N} n_j \left(b^j + b^{2N-j}\right)\\
		&=
		n_N b^N + 2\sum_{0 \leq j < N} n_j b^j
		+ \sum_{0 \leq j < N} n_j \left( b^{2N-j} - b^j\right)
	\end{align*}	
	for some unique digits $0 \leq n_j < b$ with $n_0 > 0$. Thus $\#\Pi_b(2N) = (b-1)b^{N}$. We note that every term of the right-most sum above is divisible by $b^2-1$. Then 
	$$(n,b^2-1) = \left(n_N b^N + 2\sum_{0 \leq j < N} n_j b^j, \ b^2-1\right).$$ 
	Since $(n,b)=(n_0,b)$ and $b^3-b = b(b^2-1)$, it follows
	\begin{align*}
		\#\Pi_b^*(2N) &= \sum_{\substack{0 \leq n_0, \ldots, n_N < b \\ (n_0,b) = (n_Nb^N + 2\sum_{0 \leq j < N} n_j b^j, b^2-1)=1}} 1\\ 
		&= \sum_{\substack{0 \leq n_0 < b \\ (n_0,b)=1}} \sum_{0 \leq n_N < b} \sum_{\substack{0 \leq n < b^{N-1} \\ (n_N b^N + 2n_0 + 2bn, b^2-1) = 1}}1 \\ 
		&= 
		\sum_{\substack{0 \leq n_0 < b \\ (n_0,b)=1}} \sum_{\substack{0 \leq n_N < b \\  (n_N,2)=1 \text{ if } (b,2)=1}} \sum_{\substack{0 \leq n < b^{N-1} \\ ( n_N b^N + 2n_0 +2bn, b^2-1) = 1}}1.
	\end{align*}
	The equality before the last follows from the unique representation of integers in base $b$.
	The last equality follows from the fact that if $b$ is odd, whence $b^2-1$ is even, then $$\left(n_N b^N + 2n_0 + 2bn, \ b^2-1\right) = 1$$ implies $n_N$ is odd. 
	
	The inner sum equals $S(n_Nb^N + 2n_0)$, where for an integer $a$, 
	$$
	S(a) := \sum_{\substack{0 \leq n < b^{N-1} \\ (a + 2bn, b^2-1) = 1}}1.
	$$
	If $b$ odd and $a$ is even, $S(a) = 0$. Thus we assume $a$ is odd if so is $b$. By the M\"{o}bius inversion formula $\mathbf{1}_{(m,n)=1} = \sum_{d \mid (m,n)} \mu(d)$, we have
	$$
	S(a) = \sum_{d \mid (b^2-1)} \mu(d)  \sum_{\substack{0 \leq n < b^{N-1} \\ 2bn \equiv -a (d)}}1.
	$$ 
	If $b$ is odd and $2bn \equiv -a(d)$, then $d$ is odd as so is $a$ by assumption. In the case when $b$ is even, $b^2-1$ is odd and so is any divisor $d$ of $b^2-1$. Thus $d$ runs over odd divisors of $b^2-1$ regardless. These are also coprime to $b$ as $(b^2-1,b)=1$. Letting $\overline{2b}$ denote the inverse of $2b$ modulo $d$, the above is 
	\begin{align*}
		S(a) &= \sum_{\substack{d \mid (b^2-1) \\ (d,2)=1}} \mu(d)  \sum_{\substack{0 \leq n < b^{N-1} \\ n \equiv -\overline{2b}a (d)}}1
		= \sum_{\substack{d \mid (b^2-1) \\ (d,2)=1}} \mu(d) \left(\dfrac{b^{N-1}}{d} + O(1)\right)\\
		&= \gamma(b) \dfrac{\varphi(b^2-1)}{b^2-1} b^{N-1}  + O(\tau(b^2-1)) 
	\end{align*}
	uniformly in $a$,
	where
	$$
	\gamma(b) := \begin{cases}
		1 &\mbox{ if $b$ is even}\\
		2 &\mbox{ if $b$ is odd}.
	\end{cases}
	$$ 
	Now (\ref{bound: size of Pi star}) follows after we insert this into the expression for $\#\Pi_b^*(2N)$ above, compute the resulting sum and recall that $\#\Pi_b(2N) = (b-1)b^N$.
	
	
	
	We now establish $\sqrt{x} \ll_b \#\mathscr{P}_b^*(x) \ll_b \sqrt{x}$ for $x \geq 1$. Since $\mathscr{P}_b^*(x) \subseteq \mathscr{P}_b(x) $ and $\#\mathscr{P}_b(x) \asymp_b \sqrt{x}$, the upper bound holds. To show $\#\mathscr{P}_b^*(x) \gg_b \sqrt{x}$, we first note that since $1 \in \mathscr{P}_b^*(x)$, the lower bound holds for $\log_b x$ bounded and we may assume then that $\log_b x 
	$ is arbitrarily large. In this case, let $N$ be the largest integer such that $\Pi_b(2N) \subseteq \mathscr{P}_b(x)$. 
	Clearly $b^N \asymp_b \sqrt{x}$ and $\#\mathscr{P}_b^*(x) \geq \#\Pi_b^*(2N) \asymp_b b^N$ for $N$ large.   
\end{proof}	


\begin{proof}[{\bf Proof of Theorem \ref{thm: equidistribution}}]
	Let $y \leq x$. 
	Since every $b$-palindromic integer $n\geq 1$ with $\lfloor \log_b n\rfloor$ odd is divisible by $b+1$, and $b+1 > 1$ is a divisor of $ b^3-b$, it follows
	$$
	\mathscr{P}^*_b(y) = \left\{n \in \mathscr{P}_b^0(y) \ : \ (n,b^3-b)=1\right\}. 
	$$
	Then by this and the M\"{o}bius inversion formula $\mathbf{1}_{(b^3-b,n)=1} = \sum_{r \mid (b^3-b,n)}\mu(r)$, we have
	$$
	\#\mathscr{P}_b^*(y,a,q) - \dfrac{\#\mathscr{P}_b^*(y)}{q} =\sum_{r \mid (b^3-b)} \mu(r) \sum_{\substack{n \in \mathscr{P}_b^0(y) \\ r\mid n}} \left(\mathbf{1}_{n \equiv a (q)} - \dfrac{1}{q}\right). 
	$$	
	Note 
	$$\mathbf{1}_{r \mid n} = \dfrac{1}{r} \sum_{k (r)} e_r(nk)$$ and 
	$$\mathbf{1}_{n \equiv a (q)} - \dfrac{1}{q} = \dfrac{1}{q} \sum_{1 \leq h < q} e_q(-ah)e_q(nh).$$
	Inserting these expressions above, switching orders of summation and taking absolute values, we obtain
	\begin{align*}
		&\left|\#\mathscr{P}_b^*(y,a,q) - \dfrac{\#\mathscr{P}_b^*(y)}{q}\right| \leq \sum_{r \mid (b^3-b)} \dfrac{1}{r}\sum_{k(r)} \dfrac{1}{q}\sum_{1 \leq h < q} \left| \sum_{n \in \mathscr{P}_b^0(y)}e\left(\left(\dfrac{h}{q} + \dfrac{k}{r}\right)n\right)\right|\\
		&\leq b^2 \sum_{0 \leq N \leq \frac{1}{2}\log_b x} \sum_{0 \leq M \leq N} \sum_{r \mid (b^3-b)} \dfrac{1}{r}\sum_{k(r)} \dfrac{1}{q}\sum_{1 \leq h < q}\Phi_M \left(\dfrac{b^{N-M}h}{q} + \dfrac{ b^{N-M}k}{r}\right)
	\end{align*}
	uniformly in $a \in \mathbb{Z}$ and $y \leq x$, by Lemma \ref{lem: bound for exp sum of pals} and a switch in the order of summation. If $(q,b)=1$, we may substitute $b^{N-M}h$ with $h$ above. Thus for any $Q\geq 1$, 
	\begin{align}\label{eqn: average discrepancy}
		&\sum_{\substack{q \leq Q \\ (q,b^3-b)=1}} \sup_{ y \leq x}\max_{a \in \mathbb{Z}} \left|\#\mathscr{P}_b^*(y,a,q) - \dfrac{\#\mathscr{P}_b^*(y)}{q}\right|\nonumber\\
		&\leq 
		b^2 \sum_{0 \leq N \leq \frac{1}{2}\log_b x} \sum_{0 \leq M \leq N} \sum_{r \mid (b^3-b)} \dfrac{1}{r}\sum_{k(r)} \sum_{\substack{q \leq Q \\ (q,b^3-b)=1}}\dfrac{1}{q}\sum_{1 \leq h < q}\Phi_M \left(\dfrac{h}{q} + \dfrac{ b^{N-M}k}{r}\right)\nonumber\\
		&\leq
		b^2 \tau(b^3-b)\sum_{0 \leq N \leq \frac{1}{2}\log_b x} \sum_{0 \leq M \leq N} \max_{\substack{r \mid (b^3-b) \\ k \in \mathbb{Z}}} \sum_{\substack{q \leq Q \\ (q,b^3-b)=1}}\dfrac{1}{q}\sum_{1 \leq h < q}\Phi_M \left(\dfrac{h}{q} + \dfrac{ k}{r}\right)\nonumber\\
		&=
		b^2 \tau(b^3-b)\sum_{0 \leq N \leq \frac{1}{2}\log_b x} \sum_{0 \leq M \leq N} \max_{\substack{ k \in \mathbb{Z}}}S(M,Q,k),
	\end{align}
	where
	$$
	S(M,Q, k) := \sum_{\substack{q \leq Q \\ (q,b^3-b)=1}} \dfrac{1}{q}\sum_{1 \leq h < q} \Phi_M\left(\dfrac{h}{q} + \dfrac{k}{b^3-b}\right). 
	$$
	Splitting the sum according to the GCD of $h,q$ and substituting variables, we have
	$$
	S(M,Q, k) \leq \sum_{d \leq Q/2} \dfrac{1}{d} \sum_{\substack{2 \leq q \leq Q/d \\ (q,b^3-b)=1}} \dfrac{1}{q} \sum_{h(q)}^*\Phi_M\left(\dfrac{h}{q} + \dfrac{k}{b^3-b}\right). 
	$$
	We now split the sum over $2 \leq q \leq Q/d$ into $\ll \log Q/d$ sums over dyadic segments $(R/2, R]$ with $2 \leq R \leq Q/d$. Let $0 < \epsilon \leq 1/15$ and consider the contribution of those $q$ in any one such interval $(R/2, R]$.
	Bounding $1/q$ and applying Proposition \ref{prop: average bound} with $\delta = 2/5 - \epsilon$, we get
	\begin{align*}
		&\sum_{\substack{R/2 < q \leq R \\ (q,b^3-b)=1} }\dfrac{1}{q} \sum_{h(q)}^*\Phi_M\left(\dfrac{h}{q} + \dfrac{k}{b^3-b}\right) \leq \dfrac{2}{R} \sum_{\substack{2 \leq q \leq R \\ (q,b^3-b)=1}} \sum_{h(q)}^*\Phi_M\left(\dfrac{h}{q} + \dfrac{k}{b^3-b}\right)\\
		&\ll_{b,\epsilon} 
		\left(Rb^{(\frac{3}{5} + \epsilon - \sigma_1(b,\epsilon))M} + \dfrac{b^M}{R^{\sigma_1(b,\epsilon)}}\right)\exp\left(-\sigma_\infty(b)\dfrac{M}{\log R}\right)
		\\
		&\leq
		Rb^{(\frac{3}{5} + \epsilon)M} + b^M \exp\left(-c(b,\epsilon)\sqrt{M}\right)
	\end{align*}
	for some $\sigma_1(b,\epsilon),  c(b,\epsilon), \sigma_\infty(b) > 0$ depending at most on $b,\epsilon$. Note we used $e^{-\sigma_1(b,\epsilon)\log R - \sigma_\infty(b)M/\log R} \leq e^{-c(b,\epsilon)\sqrt{M}}$ for $R \geq 2$ and $M\geq 0$. Then the combined contribution of the $\ll \log (Q/d)$ dyadic pieces to the sum over $2 \leq q \leq Q/d$ is 
	$$
	\ll_{b,\epsilon} \dfrac{Q}{d}b^{(\frac{3}{5} + \epsilon )M} + b^M \exp\left(-c(b,\epsilon)\sqrt{M}\right) \log Q.
	$$
	Dividing this by $d$ and summing over $1 \leq d \leq Q/2$ we obtain
	$$
	S(M,Q, k) \ll_{b,\epsilon} Qb^{(\frac{3}{5} + \epsilon )M} + b^M \exp\left(-c(b,\epsilon)\sqrt{M}\right) \log^2 Q
	$$ 
	uniformly in $k \in \mathbb{Z}$. Performing the summations over $M,N$ on the right hand side of (\ref{eqn: average discrepancy}) yields, for any $1 \leq Q \leq x^{1/5}$ and any $0 < \epsilon \leq 1/30$, 
	$$
	\sum_{\substack{q \leq Q \\ (q,b^3-b)=1}} \sup_{ y \leq x} \max_{a \in \mathbb{Z}} \left|\#\mathscr{P}_b^*(y,a,q) - \dfrac{\#\mathscr{P}_b^*(y)}{q}\right| \ll_{b,\epsilon} Qx^{\frac{3}{10} + \epsilon} + \sqrt{x} \exp\left(-k_{b,\epsilon} \sqrt{\log x}\right),
	$$
	where $k_{b,\epsilon} > 0$ is some value depending only on $b,\epsilon$. Since the above holds for $0 < \epsilon \leq 1/30$, it also holds for $\epsilon > 1/30$. Now the result follows when we let $Q = x^{1/5 - \epsilon_0}$ with $\epsilon_0 > 0$ arbitrary, fix $\epsilon = \epsilon_0/2$, say, and use the fact $\sqrt{x} \ll_b \#\mathscr{P}_b^*(x)$ from Lemma \ref{lem: number of coprimes}. 
\end{proof}	 

\section{Palindromic almost-primes}\label{sec: final proof}

We conclude the work with a proof of Theorem \ref{thm: at most 6 primes}.
To this end we use the following version of the almost-prime sieve. 

\begin{lem}[Almost-prime linear sieve with Richert's weights]\label{lem: sieve lemma}
	Let $ (a_n)$ be a sequence of non-negative real numbers and let $x > 2$. For each integer $d \geq 1$, set
	$$
	A_d := \sum_{\substack{n \leq x \\ d \mid n}}a_n.
	$$
	Suppose for each such $d$ that 
	$$
	A_d = Xg(d) + r_d
	$$
	for some $r_d$, real number $X$ and some multiplicative function $g$ satisfying $0 \leq g(p) < 1$ for each prime $p$ and
	$$
	\prod_{u \leq p < v} (1 - g(p))^{-1} \leq K\dfrac{\log v}{\log u}
	$$
	for any $2 \leq u < v \leq x$, where $K > 1$ is some constant. 
	
	Let $r \geq 2$ be an integer and set
	$$
	\Delta_r := r + \dfrac{1}{\log 3}\log\left(\dfrac{3}{4}\left(1 + 3^{-r}\right)\right).
	$$
	Suppose that 
	$$
	\sum_{d \leq D} |r_d| \ll \dfrac{X}{\log^3 x}
	$$ 
	for some $D$ satisfying  $D \geq x^{1/\Delta_r + \epsilon}$ for some $\epsilon > 0$. Then for $z = D^{1/4}$, we have
	$$
	\sum_{\substack{n \leq x \\ P^-(n) \geq z \\ \Omega(n) \leq r}}a_n \asymp_{r,\epsilon} X \prod_{p < x}(1 - g(p)).
	$$
\end{lem}

\begin{proof}
	See Chapter 25 in \cite{Friedlander-Iwaniec}. In particular see Theorem 25.1 and Equations 25.23, 25.25, page 479. 
\end{proof}	

\begin{proof}[{\bf Proof of Theorem \ref{thm: at most 6 primes}}]
	We may assume that $x \gg_b 1$ is large enough so that $z := x^{1/21}$ is much larger than $b^3-b$. 
	For each integer $n \geq 1$, define $a_n := \mathbf{1}_{n \in \mathscr{P}^*_b}$ and note 
	$$
	S := \sum_{\substack{n \in \mathscr{P}_b(x) \\ P^-(n) \geq z\\ \Omega(n)\leq 6}} 1=
	\sum_{\substack{n \in \mathscr{P}_b^*(x) \\ P^-(n) \geq z \\ \Omega(n)\leq 6}} 1= \sum_{\substack{n\leq x \\ P^-(n) \geq z \\ \Omega(n) \leq 6}} a_n.
	$$	
	For any integer $d \geq 1$, set
	$$
	A_d := \sum_{\substack{n \leq x \\ d \mid n }} a_n = \#\mathscr{P}_b^*(x,0,d).  
	$$ 
	We can write this as 
	$$
	A_d = g(d)\#\mathscr{P}_b^*(x) +r_d, 
	$$
	where
	$$
	g(d) = \dfrac{\mathbf{1}_{(d,b^3-b)=1}}{d}
	$$
	and
	$$
	r_d =  \#\mathscr{P}_b^*(x,0,d) - g(d) \#\mathscr{P}_b^*(x).
	$$
	Clearly $g$ is multiplicative, satisfies $0 \leq g(p) < 1$ for each prime $p$ and 
	$$
	\prod_{u \leq p < v} (1 - g(p))^{-1} \leq \prod_{u \leq p < v} \left(1 - \dfrac{1}{p}\right)^{-1} = \dfrac{\log v}{\log u}\left(1 + O\left(\dfrac{1}{\log u}\right)\right)
	$$
	for any $2 \leq u < v \leq x$,
	by the Mertens' theorems. 
	
	Set $D := z^4 = x^{4/21} = x^{1/5 - 1/105}$. Since every number in $\mathscr{P}_b^*$ is coprime to $b^3-b$, 
	$$
	\sum_{d \leq D}|r_d| = \sum_{\substack{d \leq D \\ (d,b^3-b)=1}} \left|\#\mathscr{P}_b^*(x,0,d) - \dfrac{ \#\mathscr{P}_b^*(x)}{d}\right| \ll_{b}  \dfrac{\#\mathscr{P}_b^*(x)}{\log^{10}x} 
	$$ 
	by Theorem \ref{thm: equidistribution}. For $\Delta_6$ defined as in Lemma \ref{lem: sieve lemma}, one can check that $x^{1/\Delta_6 + 1/100} \leq D$. Then all the assumptions of Lemma \ref{lem: sieve lemma} are satisfied. It gives
	$$
	S \asymp_{b} \#\mathscr{P}_b^*(x) \prod_{p < x} \left(1 - g(p)\right) \asymp_b \dfrac{ \#\mathscr{P}_b(x)}{\log x}
	$$
	as required. 
\end{proof}	















%%%%%%%%%%%%%%%%%%%%%%%%%%%%%%%%%%%%%%%%%%%%%%%%%%%%%%%%%%%%%%%%%%%%%%%%%%%%%%%%%%%%%%%%%%%%%%%%%%%%%%%%%%%%%%%%%%%%%%%%%%%%%%%%%%%%%%%%%%%%%%%%%%





	


	
	
	




\begin{thebibliography}{99}
	\bibitem{Aistleitner}
	C. Aistleitner, G. Larcher, F. Pillichshammer, S. S. Eddin, R. F. Tichy, {\em On Weyl products and uniform distribution modulo one}, Monatshefte f\"{u}r Mathematik (2018), 185, 365--395
	
	\bibitem{Banks}
	 W.D. Banks, {\em Every natural number is the sum of forty-nine palindromes}, Preprint, 2015, \href{https://arxiv.org/abs/1508.04721}{arXiv
	1508.04721v1.pdf}
	
	\bibitem{Banks-Hart-Sakata}
	W.D. Banks, D. Hart and M. Sakata,
	{\em Almost all palindromes are composite}, 
	Mathematical Research Letters 11 (2004) nos.5-6, pp.853--868
	
	\bibitem{Banks-Shparlinski}
	W.D. Banks, I.E. Shparlinski, {\em Prime divisors of palindromes}, Period. Math. Hungar. 51 (2005), 1--10.
	
	\bibitem{Bourg}
	J. Bourgain, {\em Mordell's exponential sum estimate revisited}, Journal of AMS (2005) v.18, n.2, 477--499
	
	\bibitem{Bourgain}
	J. Bourgain, {\em Prescribing the binary digits of primes, II}, 
	Israel Journal of Mathematics, v. 206, p. 165--182 (2015)
	
	\bibitem{Cilleruelo}
	J. Cilleruelo, F. Luca, L. Baxter, {\em Every positive integer is a sum of three palindromes}, Math. of Comput. 87(314), (2018)
	
	\bibitem{Col}
	S. Col, {\em Palindromes dans les progressions arithm\'{e}tiques}, Acta Arith. 137, n.1, (2009), 1--41 
	
	\bibitem{Fici}
	G. Fici, L.Q. Zamboni, {\em On the least number of palindromes contained in an infinite word}, Theoretical Computer Science v.481 (2013), 1--8
	
	\bibitem{Ford}
	K. Ford, {\em Sieve methods lecture notes}, 2020 \url{https://faculty.math.illinois.edu/~ford/sieve2020.pdf}
	
	\bibitem{Friedlander-Iwaniec}
	J. Friedlander and H. Iwaniec, {\em Opera de Cribro}, Colloquium Publications, v.57, Providence, RI, 2010
	
	\bibitem{Gao}
	Z. Gao, {\em Enumeration of self-reciprocal irreducible monic polynomials with prescribed leading coefficients over a finite field}, Finite Fields and their Application, v.83, (2022), 102083
	
	\bibitem{Garefalakis}
	T. Garefalakis and G. Kapetanakis, {\em On the Hansen-Mullen conjecture
	for self-reciprocal irreducible polynomials}, Finite Fields Appl. 69 (2012),
	832--841
	
	
	\bibitem{Gelfond}
	A. O. Gelfond, {\em Sur les nombres qui ont des propri\'{e}t\'{e}s additives et multiplicatives donn\'{e}es}, Acta Arithmetica, 13, (1968), p.259--265
	
	\bibitem{Google}
	Google Cloud Blog, \href{https://cloud.google.com/blog/products/compute/calculating-100-trillion-digits-of-pi-on-google-cloud}{\em Even more pi in the sky: Calculating 100 trillion digits of pi on Google Cloud}, 2022 (last visited on July 29, 2023)
	
	\bibitem{Gusfield}
	D. Gusfield, {\em Algorithms on Strings, Trees, and Sequences},
	Cambridge University Press, New York, 1997
	
	\bibitem{Hofbauer}
	J. Hofbauer, {\em A simple proof of $1 + 1/2^2 + 1/3^2 + \cdots = \pi^2/6$ and related identities}, Am. Math. Mon., v.109, n.2 (2002), p.196--200.
	
	\bibitem{Iwaniec}
	H. Iwaniec, {\em Almost-primes represented by quadratic polynomials},  Inventiones Mathematicae. 47 (2), (1978), 178--188
	
	\bibitem{Iwaniec-Kowalski}
	H. Iwaniec and E. Kowalski, {\em Analytic number theory}, Colloquium Publications, v.53, American Mathematical Society, Providence, RI, 2004
	
	\bibitem{Kuipers}
	L. Kuipers and H. Niederreiter, {\em Uniform distribution of sequences}, Wiley, New York, 1974
	
	\bibitem{Larionov}
	S. Larionov, A. Loskutov, E. Ryadchenko,  {\em Chromosome evolution with naked eye: palindromic context of the life origin}, Chaos. 18, n.1, 013105 (2008)
	
	\bibitem{MR}
	C. Mauduit and J. Rivat, {\em La somme des chiffres des carr\'{e}s}, Acta Math., 203 (2009), 107--148
	
	
	
	
	\bibitem{maud}
	C. Mauduit and J. Rivat, {\em Sur un probl\`eme de Gelfond : la somme des chiffres des nombres premiers}, Annals of Math., v.171 no.3 (2010), \url{http://annals.math.princeton.edu/wp-content/uploads/annals-v171-n3-p04-p.pdf}
	
	\bibitem{Maynard}
	J. Maynard, {\em Primes with restricted digits}, Inventiones Mathematicae, v.217, p. 127--218 (2019) 
	
	\bibitem{Morgenbesser}
	J. Morgenbesser, {\em Gelfond's sum of digits problems}, Vienna University of Tech., Diploma thesis,  \url{http://dmg.tuwien.ac.at/drmota/morgenbesserda.pdf}
	
	
	\bibitem{Porto}
	A.H.L. Porto, V.C. Barbosa, {\em Finding approximate palindromes in strings}, Pattern recognition, v.35 n.11 (2002), 2581--2591
	
	\bibitem{Rajasekaran}
	A. Rajasekaran, J. Shallit, T. Smith, {\em Additive number theory via automata theory}, Theory of Computing Systems 64 (2020), 542--567
	
	\bibitem{Sheldon}
	R.M. Sheldon, {\em The SATOR rebus: An unsolved cryptogram?}, Cryptologia, v.27 (2003), 233--287
	
	\bibitem{SigmaGeek}
	SigmaGeek, \href{https://sigmageek.com/challenge_results/1656603146901x235034290182684670}{\em Find large palindromic prime numbers in the decimal expansion of $\pi$ ($3,1415\ldots$)}, 2022 (last visited on July 29, 2023)
	
	
	
	
	\bibitem{Tao} T. Tao, {\em Every odd number greater than $1$ is the sum of at most five primes}, Math. of Comput., v.83, n.286, (2014), p. 997--1038
	
	\bibitem{Tenenbaum} G. Tenenbaum, {\em Introduction to analytic and probabilistic number theory}, Cambridge studies in advanced mathematics: 46, Cambridge University press, 1995
	
\end{thebibliography}



\end{document}	


