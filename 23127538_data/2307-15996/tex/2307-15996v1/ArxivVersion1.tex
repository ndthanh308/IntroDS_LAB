\documentclass{article}

\setlength{\oddsidemargin}{.0in}
\setlength{\evensidemargin}{.0in}
\setlength{\textwidth}{6.5in}
\setlength{\topmargin}{-.3in}
\setlength{\headsep}{.20in}
\setlength{\textheight}{9.in}
\raggedbottom

% PACKAGES

\usepackage{enumitem}
\usepackage{url}
\usepackage{amsmath}
\usepackage{amssymb}
\usepackage{graphicx}
\usepackage{float}
\usepackage{caption}
\usepackage{amsthm}
\usepackage{color}
\definecolor{green}{rgb}{0,0.5977,0}
\usepackage[linesnumbered,algoruled,boxed,lined]{algorithm2e}

% GENERAL SHORTCUTS

% Common symbols
\newcommand{\rr}{\mathbb R}
\newcommand{\cc}{\mathbb C}
\newcommand{\zz}{\mathbb Z}
\newcommand{\nn}{\mathbb N}
\newcommand{\qq}{\mathbb Q}
\newcommand{\ee}{\mathbb E}
\newcommand{\ff}{\mathbb F}
\newcommand{\zzn}[1]{\zz/{#1}\zz}
\newcommand{\id}{\text{id}}

% Common operations
\newcommand{\inv}{^{-1}}
\newcommand{\abs}[1]{\left|{#1}\right|}
\newcommand{\man}[1]{\left|\left|{#1}\right|\right|_1}
\newcommand{\suchthat}{\ | \ }
\newcommand{\tand}{\txt{ and }}
\newcommand{\normalize}[1]{\frac{#1}{\abs{#1}}}
\newcommand{\normallize}[1]{\frac{#1}{\abs{{\abs#1}}}}
\newcommand{\tth}{^\text{th}}
\newcommand{\genseq}[3]{{#1}_1 {#3} {#1}_2 {#3} \dots {#3} {#1}_{#2}}
\newcommand{\seq}[2]{\genseq{#1}{#2}{,}}
\DeclareMathOperator*{\argmin}{argmin}
\DeclareMathOperator*{\argmax}{argmax}
\newcommand{\pp}[1]{\mathbb{P}\left\{#1\right\}}

% General tools
\newcommand{\twocases}[4]{\begin{cases} #2 & #1 \\ #4 & #3 \end{cases}}
\newcommand{\threecases}[6]{\begin{cases} #2 & #1 \\ #4 & #3 \\ #6 & #5 \end{cases}}
\newcommand{\triad}[5]{\threecases{#1 < #2}{#3}{#1 = #2}{#4}{#1 > #2}{#5}}
\newcommand{\emptybox}[2][\textwidth]{\begingroup\setlength{\fboxsep}{-\fboxrule}\noindent\mbox{\rule{0pt}{#2}}\endgroup}
\newcommand{\leavespace}{{\centering\emptybox[15cm]{10cm}}}
\newcommand{\txt}[1]{\text{#1}}
\newcommand{\fcite}[1]{\texttt{#1.pdf}}
\newcommand{\red}[1]{{\color{red} #1}}

% Align mode tools
\newcommand{\stext}[1]{\ \ \ \ \ \text{(#1)}}
\newcommand{\stextn}[1]{\\&\ \ \ \ \ \ \stext{#1}}
\newcommand{\bcause}[1]{\stext{because ${#1}$}}
\newcommand{\snc}[1]{\stext{since ${#1}$}}
\newcommand{\indhyp}{\stext{by the inductive hypothesis}}
\newcommand{\triineq}{\stext{by the triangle inequality}}
\newcommand{\ub}{\stext{by the union bound}}
\newcommand{\bydef}{\stext{by definition}}
\newcommand{\push}{\\ & \ \ \ \ \ \ \ \ \ \ }
\newcommand{\topindent}{&\hspace{1.335em}}

% Graphics
\newcommand{\ig}[2]{% Figure environment removed}
\newcommand{\igc}[3]{% Figure environment removed}
\newcommand{\ignc}[3]{% Figure environment removed}
\newcommand{\igm}[2]{% Figure environment removed}
\newcommand{\igcm}[3]{% Figure environment removed}
\newcommand{\igncm}[3]{% Figure environment removed}
\newcommand{\ignsm}[4]{% Figure environment removed}
\newcommand{\ip}[2]{% Figure environment removed}
\newcommand{\ipc}[3]{% Figure environment removed}
\newcommand{\ipnc}[3]{% Figure environment removed}
\newcommand{\ipm}[2]{% Figure environment removed}
\newcommand{\ipcm}[3]{% Figure environment removed}
\newcommand{\ipncm}[3]{% Figure environment removed}
\newcommand{\ipnsm}[4]{% Figure environment removed}

% Algorithms
\makeatletter 
\g@addto@macro{\@algocf@init}{\SetKwInOut{Parameter}{Parameters}} 
\makeatother

% SUBJECT SPECIFIC SHORTCUTS

% Algebra
\newcommand{\pres}[2]{\langle {#1} \ | \ {#2} \rangle}

% Analysis
\newcommand{\infsum}{\sum_{n = 0}^\infty}
\newcommand{\finsum}{\sum_{n = 0}^N}
\newcommand{\limh}{\lim_{h \to 0}}

% Complex analysis
\newcommand{\real}{\text{Re}}
\newcommand{\imag}{\text{Im}}
\newcommand{\nz}{|z|^2}
\newcommand{\Arg}{\text{Arg}}
\newcommand{\Log}{\text{Log}}
\newcommand{\ddz}{\frac{d}{dz}}

% Complexity theory
\newcommand{\PP}{\mathsf{P}}
\newcommand{\NP}{\mathsf{NP}}
\newcommand{\NPh}{$\NP$-hard }
\newcommand{\NPc}{$\NP$-complete }
\newcommand{\BQP}{\mathsf{BQP}}

% Optimization
\newcommand{\succinctlp}[3]{\begin{tabular}{l l}\textbf{#1} & \begin{tabular}{l}$#2$\end{tabular}\\\textbf{subject to} & \begin{tabular}{l}$#3$\end{tabular}\end{tabular}}
\newcommand{\spmax}[2]{\succinctlp{maximize}{#1}{#2}}
\newcommand{\spmin}[2]{\succinctlp{minimize}{#1}{#2}}
\newcommand{\lp}[3]{\begin{tabular}{l l}\textbf{#1} & \begin{tabular}{l}$#2$\end{tabular}\\\textbf{subject to} & \begin{tabular}{l l}#3\end{tabular}\end{tabular}}
\newcommand{\lpmax}[2]{\lp{maximize}{#1}{#2}}
\newcommand{\lpmin}[2]{\lp{minimize}{#1}{#2}}

% Partition theory
\newcommand{\pofn}[1]{p(n \ | \ \text{#1})}

% p-adic numbers
\newcommand{\pabs}[1]{\left|{#1}\right|_p}
\newcommand{\qp}{\qq_p}
\newcommand{\zp}{\zz_p}

% Quantum mechanics
\newcommand{\bra}[1]{\langle {#1} |}
\newcommand{\ket}[1]{| {#1} \rangle}
\newcommand{\braket}[2]{\langle {#1} | {#2} \rangle}

% Topology
\newcommand{\rp}[1]{\rr\text{P}^{#1}}
\newcommand{\cp}[1]{\cc\text{P}^{#1}}
\newcommand{\Ht}{\widetilde{H}}

% Linear algebra
\newcommand{\mTwoTwo}[4]{\begin{bmatrix}{#1}&{#2}\\{#3}&{#4}\end{bmatrix}}
\newcommand{\mTwoThree}[6]{\begin{bmatrix}{#1}&{#2}&{#3}\\{#4}&{#5}&{#6}\end{bmatrix}}
\newcommand{\mTwoFour}[8]{\begin{bmatrix}{#1}&{#2}&{#3}&{#4}\\{#5}&{#6}&{#7}&{#8}\end{bmatrix}}
\newcommand{\mThreeTwo}[6]{\begin{bmatrix}{#1}&{#2}\\{#3}&{#4}\\{#5}&{#6}\end{bmatrix}}
\newcommand{\mThreeThree}[9]{\begin{bmatrix}{#1}&{#2}&{#3}\\{#4}&{#5}&{#6}\\{#7}&{#8}&{#9}\end{bmatrix}}
\newcommand{\mFourTwo}[8]{\begin{bmatrix}{#1}&{#2}\\{#3}&{#4}\\{#5}&{#6}\\{#7}&{#8}\end{bmatrix}}
\newcommand{\vectTwo}[2]{\begin{bmatrix}{#1}\\{#2}\end{bmatrix}}
\newcommand{\rowTwo}[2]{\begin{bmatrix}{#1}&{#2}\end{bmatrix}}
\newcommand{\vectThree}[3]{\begin{bmatrix}{#1}\\{#2}\\{#3}\end{bmatrix}}
\newcommand{\rowThree}[3]{\begin{bmatrix}{#1}&{#2}&{#3}\end{bmatrix}}
\newcommand{\vectFour}[4]{\begin{bmatrix}{#1}\\{#2}\\{#3}\\{#4}\end{bmatrix}}
\newcommand{\rowFour}[4]{\begin{bmatrix}{#1}&{#2}&{#3}&{#4}\end{bmatrix}}

% Category theory
\newcommand{\inl}{\txt{inl}}
\newcommand{\inr}{\txt{inr}}
\newcommand{\obj}{\txt{obj }}

\theoremstyle{plain}
\newtheorem{theorem}{Theorem}
\newtheorem{lemma}[theorem]{Lemma} 
\newtheorem{proposition}[theorem]{Proposition}
\newtheorem{claim}[theorem]{Claim}
\newtheorem{corollary}[theorem]{Corollary}
\newtheorem{conjecture}[theorem]{Conjecture}
\newtheorem{question}[theorem]{Question}
\newtheorem{globalTheorem}{Theorem}
\theoremstyle{definition}
\newtheorem{definition}[theorem]{Definition}
\newtheorem{example}[theorem]{Example}
\newtheorem{exercise}[theorem]{Exercise}

\numberwithin{theorem}{section}

\begin{document}
	\title{Locked Polyomino Tilings}
	\author{Jamie Tucker-Foltz\footnote{\texttt{jtuckerfoltz@gmail.com}}\\Harvard University}
	\maketitle
	
	\begin{abstract}
		A locked $t$-omino tiling is a tiling of a finite grid or torus by $t$-ominoes such that, if you remove any pair of tiles, the only way to fill in the remaining space with $t$-ominoes is to use the same two tiles in the exact same configuration as before. We exclude degenerate cases where there is only one tiling due to the grid/torus dimensions. Locked $t$-omino tilings arise as obstructions to popular political redistricting algorithms. It is a classic (and straightforward) result that grids do not admit locked 2-omino tilings. In this paper, we construct explicit locked 3-, 4-, and 5-omino tilings of grids of various sizes. While 3-omino tilings are plentiful, we find that 4- and 5-omino tilings are remarkably elusive. Using an exhaustive computational search, we find that, up to symmetries, the $10 \times 10$ grid admits a locked 4-omino tiling, the $20 \times 20$ grid admits a locked 5-omino tiling, and there are no others for any other grid size attempted. Finally, we construct an infinite family of locked $t$-omino tilings on tori with unbounded $t$.
	\end{abstract}
	
	\section{Introduction}\label{secIntro}
	
	In the United States, \emph{redistricting} is the process by which states regularly  redraw electoral districts in accordance with new census data. To mathematicians and computer scientists, political redistricting is most commonly thought of as a graph partitioning problem. The vertices represent census blocks, precincts, or counties, and the edges represent geographic adjacency. Given such a vertex-weighted graph $G$ and a target number of districts $k$, redistricting is the act of partitioning the vertex set of $G$ into $k$ sets of equal population, each inducing connected subgraphs. One of the most successful algorithmic paradigms for exploring the space of redistricting plans (graph partitions) is to run a Markov chain whereby adjacent districts are merged into a double-district and then instantly split again, hopefully in a different way \cite{ReCom}. Such an operation is called a \emph{recombination move}. Certain variants of these Markov chains are observed to be rapidly mixing in practice -- but are they in theory as well?
	
	Arguably the most simple model of this problem is to take $G$ to be an $m \times n$ grid graph, which we notate $G(m, n)$. We assume vertices have equal population. The \emph{metagraph} $\mathcal{M}(G, k)$ is the graph whose vertices are partitions of $G$ into $k$ connected subgraphs, each containing an equal number of vertices, with two partitions connected by an edge whenever they differ only on two parts. A Markov chain on redistrictings of $G$ is thus a walk on $\mathcal{M}(G, k)$, so, given a probabilistic transition function, we may inquire about mixing properties. But first, it is important to settle of whether or not $\mathcal{M}(G, k)$ is even connected. Unfortunately, even for small $k$, very little is known.
	
	\begin{conjecture}\label{cnjGridKEquals3}
		For any positive integer $n$ that it divisible by 3, $\mathcal{M}(G(n, n), 3)$ is connected.
	\end{conjecture}

	It is obvious to anyone who has spent a few hours thinking about this problem that the conjecture is true. Every step, you are allowed to repartition two-thirds of the entire grid graph any way you wish -- surely it is possible to eventually get from one partition to any other? To get a sense of the massive gap between what is known and what the conjecture requires, note that it was only recently shown that $\mathcal{M}(G, k)$ is connected whenever $G$ is \emph{if we drop the requirement that parts have equal numbers of vertices} \cite{AnySizeDistrictsNew}. More relevant is the groundbreaking triangular lattice metagraph connectivity result for $k = 3$ by Sarah Cannon \cite{TriangularLattice}. Compared to Conjecture \ref{cnjGridKEquals3}, this result is weaker in the following three ways.
	\begin{itemize}
		\item The graph $G$ is a triangular lattice rather than a square lattice. This makes things much easier since the neighbors of any vertex are connected by a path/cycle, a fact that is leveraged by the proof in a crucial way.
		\item The districts may vary in size $\pm 1$ from the target size. This makes it easier to transition from one partition to another, as only one vertex need be reassigned in a given step.
		\item Districts must always be simply-connected.
	\end{itemize}
	
	In this paper, we approach the metagraph connectivity question from the opposite end of the parameter space: instead of taking $k$ to be small, we assume the district size $t := \frac{mn}{k}$ to be small. In other words, we study the space of $t$-omino tilings of $m \times n$ grids. Here we find new obstacles, exemplified by the 3-omino tiling in Figure \ref{fig3Omino6}.
	
	\ipnc{.2}{Triomino6}{\label{fig3Omino6}A locked 3-omino tiling of $G(6, 6)$.}
	
	Observe that there are no two adjacent tiles that can be merged and split into 3-ominoes in a different way than they are currently arranged. Since there are no recombination moves, this tiling is an isolated vertex in $\mathcal{M}(G(6, 6), 12)$, proving that the graph is disconnected. Thus, Conjecture \ref{cnjGridKEquals3} is false if we replace $k = 3$ with $k = 12$. Further, observe that we may tile any $6m \times 6n$ grid with copies of this tiling arranged in an $m \times n$ grid of $6 \times 6$ blocks as in Figure \ref{fig3Omino12}. By flipping every other block, we ensure that there are no recombination moves available on the boundaries between blocks either. This shows that metagraphs can be disconnected for arbitrarily large $k$, even fixing $t = 3$.
	
	\ipnc{.2}{Triomino12}{\label{fig3Omino12}A locked 3-omino tiling of $G(12, 12)$ obtained from copying and flipping the tiling from Figure \ref{fig3Omino6}.}
	
	Formally, we define a \emph{locked $t$-omino tiling} of an $n$-vertex graph $G$ to be an isolated vertex in $\mathcal{M}(G, \frac{n}{t})$ that is not the only vertex in $\mathcal{M}(G, \frac{n}{t})$. This final condition outlaws degenerate cases: a tiling cannot be considered locked merely because there are no other tilings (as this is not a connectivity counterexample anyway). The fundamental question we ask in this paper is,
	
	\begin{center}
		\emph{For which $t$ do there exist grids with locked $t$-omino tilings?}
	\end{center}
	
	For $t = 2$, no such tilings exist. In the special case of the $8 \times 8$ grid, this is a classic mathematical puzzle: is it possible to cover a chessboard with $1 \times 2$ dominoes without creating a $2 \times 2$ square? Not only is this not possible, but it is well-known that the metagraph is connected on any rectangular grid, and moreover, the \emph{Glauber dynamics} recombination chain is rapidly mixing \cite{GlauberConnected2, GlauberConnected1}. For completeness, in Section \ref{sec2Omino} we give an elementary proof of connectivity of 2-omino metagraphs on grids.

	In Section \ref{sec3Omino} we then turn to 3-ominoes, where locked tilings are abundant. The example from Figures \ref{fig3Omino6} and \ref{fig3Omino12} show that grids with side lengths divisible by 6 admit locked tilings; we prove a stronger result, that all sufficiently large grids admit locked tilings. On the other hand, if one of the side lengths is at most 3, we show that the metagraph is connected, so no locked tilings exist.
	
	In the author's opinion, the most surprising and beautiful result of this paper concerns 4- and 5-ominoes. Locked tilings have proved extremely difficult to construct by hand, so an exhaustive computational search was conducted to find them. One would expect to find none, as is the case with 2-ominoes, or many, as is the case with 3-ominoes. Instead, the program discovered only a single tiling of each type (up to symmetry): a 4-omino tiling of the $10 \times 10$ grid and a 5-omino tiling of the $20 \times 20$ grid. In Section \ref{sec45Omino} we describe the algorithm, present these tilings, and explain the extents to which they are unique.

	While the focus of this paper is grid graphs, in Section \ref{secTori} we briefly consider torus graphs, in which the left edge of the grid is understood to be connected to the right edge and the top edge connected to the bottom edge. These are easier to reason about since there is no boundary, so all vertices look the same. While we conjecture that the set of integers $t$ for which there exist locked $t$-omino partitions of grids is bounded (possibly even by 5), it turns out that this is not the case for tori. We construct an infinite family of locked $t$-omino torus tilings for increasing $t$.
	
	\section{Domino Tilings}\label{sec2Omino}
	
	In this section we present an elementary proof of the following result.
	
	\begin{theorem}\label{thmDominoConnected}
		For any positive integers $m$ and $n$, $\mathcal{M}(G(m, n), \frac{mn}{2})$ is connected.
	\end{theorem}

	This theorem is originally credited to William Thurston \cite{GlauberConnected1}, who constructs a ``height function'' on the dual graph to the tiling and shows that the highest point can always be decreased by a recombination move. The following proof, while less elegant, is perhaps easier to understand.
	
	\begin{proof}
		If either $m$ or $n$ is 1, or if $mn$ is odd, then there is at most one tiling, so the metagraph is obviously connected. So suppose both side lengths are at least 2, and without loss of generality assume the horizontal dimension has even length. We show that any pair of tilings are connected to each other via a sequence of recombination moves by showing that any given tiling is connected to a ``central'' tiling in which all dominoes are horizontally oriented. Starting from the top-left corner, reading left-to-right, top-to-bottom, let $D_1$ be the first domino that is vertically oriented. In Figure \ref{figDominoConnectivityProof}, $D_1$ is the black-outlined domino.
		
		\ipnc{.2}{DominoConnectivityProof}{\label{figDominoConnectivityProof}A generic step of the algorithm to transition to the all-horizontal domino tiling.}
		
		If the domino $D_2$ to the right of $D_1$ is also vertically oriented, then we can recombine $D_1$ and $D_2$ so that $D_1$ is horizontal. Otherwise we check if we can recombine $D_2$ with the domino $D_3$ below it, and so on. Iteratively supposing we cannot recombine, we find a ladder of dominoes extending downward and to the right, which must eventually terminate in a $2 \times 2$ square (the gold vertical dominoes in the bottom of Figure \ref{figDominoConnectivityProof}) because it cannot go on forever without hitting the boundary of the rectangular grid. It is straightforward to see that a sequence of recombination moves can push this square up the ladder until $D_1$ is horizontal. Note that these moves do not touch any of the horizontal dominoes preceding $D_1$. Thus, we have strictly increased the number of consecutive horizontal dominoes reading from the top-left corner. We can continue iteratively applying this algorithm until all dominoes are horizontal.
	\end{proof}
	
	\section{Triomino Tilings}\label{sec3Omino}
	
	We begin our discussion of 3-ominoes by considering the easy cases where the metagraph is connected. The proof proceeds along similar lines as the proof of Theorem \ref{thmDominoConnected}, though there are a few more cases to consider.
	
	\begin{theorem}\label{thmTriominoConnected}
		For any positive integers $m$ and $n$ such that $m \leq 3$, $\mathcal{M}(G(m, n), \frac{mn}{3})$ is connected.
	\end{theorem}

	\begin{proof}
		As before, we may assume without loss of generality that $mn$ is divisible by 3 and both $m$ and $n$ are at least 2.
		
		\ipnc{.2}{TriominoConnectivityProofEasy}{\label{figTriominoConnectivityProof2} Steps required to make all 3-ominoes horizontal in a $2 \times n$ grid.}
		
		If $m = 2$, then we show that we can transition from any partition to a partition in which all 3-ominoes are horizontal. Indeed, the first column that is not configured this way must be configured according to one of the four cases in Figure \ref{figTriominoConnectivityProof2}. As indicated by the arrows, we can reconfigure these 3-ominoes to be horizontal in at most 2 moves. Continuing inductively, we can eventually make all 3-ominoes horizontal.
		
		\ipnc{.2}{TriominoConnectivityProof}{\label{figTriominoConnectivityProof3} All possible cases of what the frontier could look like in transitioning to an all-vertical tiling of a $3 \times n$ grid.}
		
		If $m = 3$, we analogously transition to a partition in which 3-ominoes are \emph{vertical}. in Figure \ref{figTriominoConnectivityProof3} exhaustively lists all possible cases of what the neighborhood around the first column that is not a 3-omino could look like. We leave it to the reader to verify that it is always possible to make this column into a vertical 3-omino in at most 3 recombination moves.
	\end{proof}

	The smallest nontrivial case that is not covered by this theorem is $m = 6$ and $n = 4$. Already here there are locked 3-omino tilings. We conjecture that locked tilings exist whenever Theorem \ref{thmTriominoConnected} does not apply, i.e., whenever $m$ is divisible by 3 (which is without loss of generality), $m \geq 6$, and $n \geq 4$. It is tedious to prove a statement like this as, even though 3-omino tilings are plentiful, they are largely unstructured and difficult to find, especially when $m$ or $n$ is odd. We can, however, obtain a weaker result, replacing the number 4 with 100.
	
	\begin{theorem}\label{thm3OminoLocked}
		For any integers $m$ and $n$ such that $m$ is divisible by 3, $m \geq 6$, and $n \geq 100$, the $m \times n$ grid admits a locked 3-omino tiling.
	\end{theorem}
	
	\begin{proof}
		Given the assumptions on $m$ and $n$, one can verify that it is always possible to find nonnegative integers $x_1, x_2, x_3, x_4, x_5$ such that $m = 6x_1 + 9x_2$ and $n = 22x_3 + 19x_4 + 8x_5$, with one of $x_1$ or $x_2$ greater than zero and $x_3$ also greater than zero. We may thus draw straight horizontal and vertical lines across the entire grid to partition the grid into rectangular blocks of size $6 \times 22$, $9 \times 22$, $6 \times (22 + 19x_4 + 8x_5)$, and $9 \times (22 + 19x_4 + 8x_5)$. We may fill these blocks in with the tilings depicted in Figure \ref{figTriominoLocked}.
		
		\ipnc{.165}{Triomino100Colorful}{\label{figTriominoLocked} Locked 3-omino tilings that can tile with themselves and each other.}
		
		Each tiling consists of four parts: two green parts on the end of total length 22, a red part in the middle of length 19, and a blue part of length 8. To make a rectangle of length $22 + 19x_4 + 8x_5$ we repeat the red part $x_4$ times and the blue part $x_5$ times. One can check that there are no recombination moves within the interiors of any of these tilings, nor between neighboring blocks.
	\end{proof}

	
	
	\section{Tetromino and Quintomino Tilings}\label{sec45Omino}
	
	While locked 3-omino tilings can be found by hand with some effort, locked $t$-omino tilings are extremely difficult to construct for $t \geq 4$. To date, only two tilings have been discovered, both by the means of computational search.
	
	The algorithm, which has been made publicly available online,\footnote{\url{https://github.com/jtuckerfoltz/LockedPolyominoTilings}} works as follows. We enumerate all possible locations within all possible $t$-ominoes that a given cell could be contained in, which we call the \emph{type} of the cell. For example, when $t = 2$, there are 4 types, depending on whether the other cell in the domino is above, below, to the left, or to the right. For $t = 3$, there are 18 types, for $t = 4$, there are 76 types, and for $t = 5$, there are 315 types.\footnote{\url{https://oeis.org/A048664}} Next, we enumerate, for every $(x, y)$ offset, all the incompatible pairs of types, i.e., the set of pairs of types $(t_1, t_2)$ such that it is not possible for a cell at location $(x_0, y_0)$ to have type $t_1$ and a cell at location $(x_0 + x, y_0 + y)$ to have type $t_2$ because the tiles either intersect or admit a recombination move. We store the incompatible pairs in a massive lookup table, representing types as arbitrary integers to save space. We then augment the lookup table by iteratively applying the following rule until no more additions are made: it is not possible for cell $a$ to have type $t_1$ and cell $b$ to have type $t_2$ if that would mean that there is no compatible type for some other cell $c$. All of these steps are slow, but only need to be computed only once for each polyomino size.
	
	For any given grid size, we initialize each grid cell with the set of types it is allowed to have. For most interior cells, this is all types (this is true only for $t \leq 5$), but cells near the grid boundary have additional restrictions. We then iteratively pick a cell with few possible types and recursively try all of them, eliminating incompatible types for other cells as we go. This enables us to exhaustively search for all locked polyomino tilings. The reason this algorithm is so successful at handling large grid sizes is probably that the locking constraint is quite strong, which severely restricts the branching factor.
	
	\ipnc{.2}{TetrominoLocked10}{\label{figUniqueTilings10}The only locked 4-omino grid tiling known to exist.}
	
	\begin{theorem}\label{thm4OminoLocked}
		Up to isomorphism, there is only one locked 4-omino tiling of any grid of up to 400 vertices (shown in Figure \ref{figUniqueTilings10}). Among tilings that are additionally 4-fold symmetric, it is unique for grids up to size $30 \times 30$.
	\end{theorem}

	\ipnc{.2}{QuintominoLocked20Final}{\label{figUniqueTilings20}The only locked 5-omino grid tiling known to exist.}
	
	\begin{theorem}\label{thm5OminoLocked}
		There are no locked 5-omino tilings on an $m \times n$ grid for any $m, n \leq 15$. There is a locked 5-omino tiling of a $20 \times 20$ grid (shown in Figure \ref{figUniqueTilings20}), which is unique (up to isomorphism) among tilings that are additionally 4-fold symmetric for grids up to size $25 \times 25$.
	\end{theorem}
	
	\section{Torus Tilings}\label{secTori}
	
	The most difficult part of constructing locked polyomino partitions seems to be making straight borders. In this section, we remove this obstacle by considering torus graphs. One can alternatively think of torus tilings as infinite periodic tilings of the plane. Already, we can construct several different kinds of tilings that are not possible for grids, as illustrated in Figure \ref{figTorusTilings}.
	
	\ipnc{.2}{TorusTetrominoLocked12}{\label{figTorusTilings} Some locked 2-, 3-, and 4-omino tilings on tori.}
	
	Surprisingly, we find that it is possible to construct locked $t$-omino tilings for \emph{arbitrarily large} $t$.
	
	\begin{theorem}\label{thmTorusFamily}
		For any positive integer $n$, the square $((2n + 1)^2 + 1) \times ((2n + 1)^2 + 1)$ torus admits a locked $(2n(n + 1) + 1)$-omino tiling.
	\end{theorem}

	\begin{proof}
		We direct the reader to Figure \ref{figBigTorusTiling}, which shows the tiling for $n = 4$. In general, each tile is a double spiral with $n$ turns on each side. Every pair of adjacent tiles intersects in the exact same way, so it suffices to verify that they cannot be recombined. This follows from the observation that the lengths of the two long paths trailing off of the region obtained by removing two adjacent tiles sum up $t + 1$, so they cannot be in the same $t$-omino. This forces them to partition up the rest of the region in a unique way.
	\end{proof}
	
	\section{Open Problems}\label{secOpenProblems}
	
	This paper is the first foray into the strange and unstructured world of metagraph connectivity with small districts. It leaves open numerous directions for future work. As previously mentioned, it would be very satisfying to prove Conjecture \ref{cnjGridKEquals3} and to close the gap between Theorems \ref{thmTriominoConnected} and \ref{thm3OminoLocked}. Additionally, the following questions seem particularly interesting.
	
	\begin{enumerate}[label=(\arabic*)]
		\item Are there \emph{any} other locked $t$-omino tilings of any grids with $t \geq 4$, aside from the two presented in this paper? Do they exist only for bounded $t$? Is the bound 5?
		\item Is there some structure to locked 3-omino partitions, and more generally, 3-omino metagraphs? As it so happens, 3-omino metagraphs suffer from more connectivity issues than just isolated points. The metagraph for $6 \times 6$ grids has the following components:
		\begin{center}
			\begin{tabular}{ c|c }
				Number of vertices in component & Number of components\\
				\hline
				73738 & 1\\
				384 & 4\\
				235 & 8\\
				199 & 8\\
				68 & 8\\
				20 & 16\\
				19 & 16\\
				16 & 2\\
				8 & 8\\
				2 & 16\\
				1 & 50\\
			\end{tabular}
		\end{center}
		It is not surprising that there are as many as 50 locked tilings (components of size 1) or that most of the vertices lie in one giant component. What \emph{is} surprising is that there are many other large components. Strangest of all is the fact that, for all components of size 68 and above, there is no pair of grid cells that are always together in the same 3-omino throughout the component. In other words, some components do not correspond to local regions of the $6 \times 6$ grid, but rather global invariants that are somehow preserved through recombination moves.
	\end{enumerate}

	\section*{Acknowledgments}
	
	I am grateful to Parker Rule and Moon Duchin for multiple thought-provoking discussions on this topic. And also to Peep Mouse Kumar, the kind and fluffy cat that sat on my lap (and sometimes arms) as I wrote most of this paper.
	
	\ipnc{.098}{Torus40OminoLockedCropped}{\label{figBigTorusTiling} A locked 41-omino tiling of the $82 \times 82$ torus.}
	
	\bibliographystyle{plain}
	\bibliography{bibliography}
\end{document}