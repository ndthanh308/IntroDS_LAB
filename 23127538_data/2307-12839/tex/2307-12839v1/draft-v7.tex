\documentclass[%
 reprint,
%superscriptaddress,
%groupedaddress,
%unsortedaddress,
%runinaddress,
%frontmatterverbose, 
%preprint,
%showpacs,preprintnumbers,
nofootinbib,
%nobibnotes,
%bibnotes,
 amsmath,amssymb,
 aps
%pra,
%rmp,
%prstab,
%prstper,
%floatfix,
]{revtex4-1}
\pdfoutput=1
\usepackage{graphicx}% Include figure files
\usepackage[utf8]{inputenc}
\usepackage{flushend}
\usepackage{xcolor}
%\usepackage{caption}
%\captionsetup{justification=raggedright,singlelinecheck=false}
\usepackage{tabularray}
\UseTblrLibrary{booktabs}
%\usepackage[T1]{fontenc}
\usepackage{dcolumn}% Align table columns on decimal point
\usepackage{bm}% bold math
\usepackage{balance}
%\usepackage{hyperref}% add hypertext capabilities
%\usepackage[mathlines]{lineno}% Enable numbering of text and display math
%\linenumbers\relax % Commence numbering lines
%\usepackage[colorinlistoftodos]{todonotes}

%\usepackage[showframe,%Uncomment any one of the following lines to test 
%%scale=0.7, marginratio={1:1, 2:3}, ignoreall,% default settings
%%text={7in,10in},centering,
%%margin=1.5in,
%%total={6.5in,8.75in}, top=1.2in, left=0.9in, includefoot,
%%height=10in,a5paper,hmargin={3cm,0.8in},
%]{geometry}

\usepackage[normalem]{ulem}

\usepackage[colorlinks = true,
            linkcolor = blue,
            urlcolor  = blue,
            citecolor = blue,
            anchorcolor = blue]{hyperref}
\usepackage{verbatim}
\usepackage{color,ulem}
\usepackage[english]{babel}
%\usepackage{MnSymbol,wasysym}
\newcommand\MB[1]{\textcolor{red}{\bf [MB:\,#1]}}
\newcommand\YL[1]{\textcolor{magenta}{ [YL:\,#1]}}
\newcommand\ZC[1]{\textcolor{blue}{\bf [ZC:\,#1]}}
%\newcommand{\YL}[1]{{\bf {\textcolor{blue}{ [#1]}}}}



\usepackage[utf8]{inputenc}
\input Starburst.fd
\newcommand*\initfamily{\usefont{U}{Starburst}{xl}{n}}\initfamily 

\DeclareMathOperator{\arcsec}{arcsec}
\newcommand{\beq}{\begin{eqnarray}}
\newcommand{\eeq}{\end{eqnarray}}
\usepackage{amsmath}
\usepackage{tikz}
\usetikzlibrary{decorations.pathmorphing}
\usetikzlibrary{shapes.misc}
\tikzset{cross/.style={cross out, draw=black, minimum size=8*(#1-\pgflinewidth), inner sep=0pt, outer sep=0pt},
%default radius will be 1pt. 
cross/.default={1pt}}
\usetikzlibrary{patterns,math}

\def\be{\begin{equation}}
\def\ee{\end{equation}}
\def\bea{\begin{eqnarray}}
\def\eea{\end{eqnarray}}
\newcommand{\dd}{\mathrm{d}}



\begin{document}
\title{\Large A quantitative theoretical model\\ of the boson peak based on stringlet excitations}
\author{Cunyuan Jiang$^{1,2}$}
\author{Matteo Baggioli$^{1,2}$}\email{b.matteo@sjtu.edu.cn}
\author{Jack F. Douglas$^{3,4}$}\email{jack.douglas@nist.gov}

\affiliation{$^{1}$Wilczek Quantum Center, School of Physics and Astronomy, Shanghai Jiao Tong University, Shanghai 200240, China}
\affiliation{$^{2}$Shanghai Research Center for Quantum Sciences, Shanghai 201315, China}
\affiliation{$^{3}$Materials Science and Engineering Division, National Institute of Standards and
Technology, Gaithersburg, Maryland 20899, United States}

\begin{abstract}  % Science journal - 125 word or less; current 119.
The boson peak (BP), a low-energy excess in the vibrational density of states over the phonon Debye contribution, is usually identified as one of the distinguishing features between ordered crystals and amorphous solid materials. Despite decades of efforts, its microscopic origin still remains a mystery and a consensus on its theoretical derivation has not yet been achieved. Recently, it has been proposed, and corroborated with simulations, that the BP might stem from intrinsic localized modes which involve string-like excitations (``stringlets") having a one-dimensional (1D) nature. In this work, we build on a theoretical framework originally proposed by Lund that describes the localized modes as 1D vibrating strings, but we specify the stringlet size distribution to be exponential, as observed in independent simulation studies. We show that a generalization of this framework provides an analytically prediction for the BP frequency $\omega_{BP}$ in the temperature regime well below the glass transition temperature in both 2D and 3D amorphous systems. The final result involves no free parameters and is in quantitative agreement with prior simulation observations. Additionally, this stringlet theory of the BP naturally reproduces the softening of the BP frequency upon heating and offers an analytical explanation for the experimentally observed scaling with the shear modulus in the glass state and changes in this scaling in cooled liquids. Finally, the theoretical analysis highlights the existence of a strong damping for the stringlet modes at finite temperature which leads to a large low-frequency contribution to the 3D vibrational density of states, as observed in both experiments and simulations.
\end{abstract}
\maketitle
\color{blue}\textbf{Introduction} \color{black} -- Amorphous solids exhibit a variety of characteristic anomalies in their vibrational, thermodynamic and transport properties, when compared to the more familiar case of ordered crystalline matter \cite{PhysRevB.4.2029,doi:10.1142/q0371}. Among the different, and possibly universal \cite{ramos2020universal}, anomalous properties of glassy materials, the boson peak (BP) is probably the most commonly discussed and controversial. This peak refers to an excess in the vibrational density of states (VDOS) normalized by the phonon density of states of Debye theory, $g(\omega)/\omega^{d-1}$, where $d$ is the number of spatial dimensions and $\omega$ the frequency. The BP is ubiquitously observed in amorphous systems \cite{ALEXANDER199865} and recently this feature has also been observed in crystalline systems at finite temperature (\textit{e.g.}, \cite{RevModPhys.86.669,PhysRevB.99.024301}), so that it remains unclear whether it is due exclusively to structural disorder \cite{PhysRevLett.122.145501}.

Several theoretical models have been proposed to rationalize the BP anomaly in glasses \cite{Elliott_1992,Schirmacher_2006,PhysRevLett.98.025501,PhysRevLett.115.015901,PhysRevLett.97.055501,PhysRevB.43.5039,PhysRevB.67.094203,PhysRevB.76.064206,PhysRevE.61.587,PhysRevLett.122.145501,PhysRevLett.86.1255,PhysRevLett.106.225501,doi:10.1080/00018738900101162,KLINGER2002311,Grigera2003}, but a final verdict has not yet been reached. A common idea is that the BP represents a signature for the emergence of additional vibrational modes coexisting, but distinct from phonons. These ``excess modes" have been suggested to arise from both structural disorder or from the anharmonicity in intermolecular interactions inherent to the liquid state and heated crystals. These commonly observed vibrational modes are expected to be effectively localized as they involve a rather limited number of atoms or molecular segments in complex molecules \cite{PhysRevB.67.094203,PhysRevB.76.064206,PhysRevB.53.11469,2022arXiv221010326L,10.1063/5.0147889,2023arXiv230403661M,10.21468/SciPostPhysCore.4.2.008,VURAL20113528,SCHOBER1993965}. At large enough wave-vector, they also hybridize with the collective phonons, making them overdamped, \textit{i.e.}, reaching the Ioffe-Regel (IR) limit \cite{Shintani2008,PhysRevB.87.134203,PhysRevLett.96.045502}, and hindering thermal transport because of the consequent strong scattering.

Despite numerous experimental and simulation studies, a fully predictive theoretical model explaining the existence of these modes
and how they exactly give rise to the observed boson peak remains elusive. Recently it has been recognized that, on atomic scales, localized string-like excitations that involve reversible particle exchange motion in the form of linear polymeric structures arise in glass-forming liquids. These dynamic structures have been termed as ``\textit{stringlets}'' \cite{10.1063/5.0039162} (see Fig.\ref{fig:0} for a visual representation from simulations). Moreover, it has been shown \cite{10.1063/5.0039162} that these stringlets are the dominant contribution to the boson peak and grow upon heating, leading to the material softening upon heating\footnote{A similar phenomenon is observed in the glassy interfacial dynamics of crystalline Ni nanoparticles in their ``premelted" state, where anharmonic interactions in the crystalline material are prevalent \cite{C2SM26789F}}. These excitation structures and their significance for the boson peak have been confirmed in both 2D and 3D amorphous systems by Hu and Tanaka \cite{Hu2022,PhysRevResearch.5.023055}. Moreover, Betancourt et al. \cite{10.1063/1.5009442,doi:10.1073/pnas.1418654112} have shown a direct relation between the stringlets and string-like collective motion, observed in connection with activated transport occurring on a much longer timescale. A general relation between the stringlet size to the mean square displacement on the fast $\beta$-relaxation time, a quantity that has been related to the long time $\alpha$-relaxation time, has been observed as well \cite{Douglas_2016}. Hu and Tanaka confirmed and extended these relations between the fast dynamics and $\alpha$-relaxation \cite{Hu2022,PhysRevResearch.5.023055}. Additionally, it has been demonstrated that the excess modes are mainly of transverse nature, and it has also been noted that such 1D vibrational modes are consistent with the linear frequency dependence of the light-vibrational coupling parameter above the BP frequency, an important relation for density of states estimations from low-frequency Raman scattering data \cite{PhysRevB.59.38} (see also \cite{PhysRevB.48.12539}). These observations and their interpretation are broadly in accord with the idea that topological ``defects'' can give rise to the BP in glasses \cite{Angell_2004,BHAT20064517}, and also with the related suggestion that these excitations are important for the plasticity and yield strength of glassy materials \cite{PhysRevLett.127.015501,Baggioli2023,PhysRevE.105.024602,Wu2023}.

% Figure environment removed
We note that previous models \cite{MALINOVSKY199163,Hu2022,PhysRevResearch.5.023055} have heuristically suggested that the excess modes responsible for the boson peak involve structures whose size distribution was taken to be a log-normal function. Lund \cite{PhysRevB.91.094102}\footnote{See also an even earlier model by Novikov \cite{novikov1990spectrum}, and a similar construction by Granato and L\"{u}cke \cite{granato1956theory} introduced for the description of plasticity in crystals, where the strings being modeled in this instance correspond to pinned segments of dislocation lines} considered the somewhat similar problem of describing excess normal modes within a continuum solid having a random distribution of line 1D ``defects'' modeled as elastic strings. The theory was able to rationalize many of the anomalous features of glasses, including the presence of a BP anomaly, Rayleigh sound attenuation constant, a negative sound dispersion for acoustic phonons with a minimum near the BP, and the saturation of the IR limit for transverse phonons. In Ref.\cite{PhysRevB.101.174311}, this model was used to estimate the length distribution $p(l)$ of these hypothetical string-like structures using experimental observations of the VDOS in glycerol and silica. Unfortunately, the string length distribution in Ref.\cite{PhysRevB.101.174311} is not consistent with simulation observations \cite{10.1063/5.0039162,Hu2022,PhysRevResearch.5.023055} for $p(l)$, which appears to universally follow an approximate exponential form, $p(l)\sim \exp(- l/\lambda)$.

In the present work, we revisit Lund model of the boson peak \cite{PhysRevB.91.094102} and extend it to make a parameter-free prediction of $\omega_{BP}$ in both 2D and 3D for amorphous solids. In contrast to the work of Lund, however,
we assume an exponential form for the string length distribution $p(l)$, as observed in simulations \cite{10.1063/5.0039162,Hu2022,PhysRevResearch.5.023055}, and that the average stringlet vibration speed equals the transverse sound velocity. This model leads an extremely simple analytic expression for the BP frequency involving no fitting parameters, which is reliable in the the low temperature glass state where many boson peak measurements are performed. In particular, our predictions are in good agreement with the experimental results for the BP frequency at $T=0$, with an uncertainty between $1\%$ and $5\%$. The remarkably simple model also captures the correct qualitative trend at high temperatures where these vibrational modes begin to become overdamped. Interestingly, we are also able to reproduce the softening of the BP frequency with temperature and to explain the experimentally observed scaling of $\omega_{BP}$ with the shear modulus $G$ \cite{Tomoshige2019}. Above the glass transition temperature $T_g$, our theory suggests that a strong damping of the stringlet modes is required to describe simulation observations. As a by-product, the introduction of a friction term for the string vibrations produces a low-frequency divergence in the VDOS normalized to the Debye law, $g(\omega)/\omega^{d-1}$, which is ubiquitously observed in experimental and simulation data at finite temperature \cite{PhysRevE.83.061508,BUCHENAU1993275} (resembling similar features in anharmonic crystals and incommensurate structures), and usually not discussed in most of the traditional theories for the BP (see \cite{PhysRevLett.93.245902,Baggioli_2020,PhysRevResearch.1.012010,doi:10.1142/S0217979221300024,jiang2023glassy} for some exceptions).\\

\color{blue}\textbf{The stringlet model} \color{black} -- We initially directly follow Lund et al. \cite{PhysRevB.91.094102,PhysRevB.101.174311}, and model the stringlets as 1D vibrating strings that are anchored at both ends\footnote{Same results follow by assuming that the strings are free at both ends, since, in both cases, the vibrations would be characterized by a fundamental wave-vector $k=\pi/l$, with $l$ the length of the string.}. As noted before, 1D vibrational modes being excited by light are consistent with the low-frequency Raman scattering observations on glasses, where a linear scaling for the light to vibration coupling coefficient $C(\omega)$ has been found \cite{PhysRevB.59.38}. According to this simple continuum based model, which is agnostic to the actual physical origin of these vibrational modes, each stringlet of length $l$ has a corresponding frequency of vibration given by:
\begin{equation}
    \omega_l=\sqrt{\left(\frac{\pi v}{l}\right)^2-\gamma^2}\equiv \sqrt{\omega_0^2-\gamma^2}\label{ab}
\end{equation}
where $v$ is the speed of propagation of the stringlet wave, and $\gamma$ represents a ``friction parameter'' (see \cite{PhysRevB.91.094102,PhysRevB.101.174311} for details). Eq.\eqref{ab} is based on the assumption that only the fundamental frequency of vibration of the stringlet is relevant, \textit{i.e.}, higher harmonics are neglected. We consider a set of stringlets whose length follows a distribution denoted as $p(l)$. Based on this assumption, the stringlet density of states $g_s(\omega)$ can readily be shown to equal,
\begin{equation}
    g_s(\omega_l) \,d\omega_l=(d-1)\, p(l)\, dl \label{ide}
\end{equation}
where $d$ is the spatial dimension. In three dimensions ($d=3$), the factor of $d-1=2$ in the r.h.s. of Eq.\eqref{ide} reflects the ability of the stringlets to oscillate along two linearly independent directions that are perpendicular to its equilibrium orientation. Formally, this framework is equivalent to considering a distribution of damped oscillators with frequency $\omega_0$ and damping rate $\gamma$. We can then define the quality factor $Q=\omega_0/\gamma$ according to which $Q\ll 1$ corresponds to underdamped vibrations and $Q$ of $\mathcal{O}(1)$ to overdamped ones. Notice that a priori $Q$ depends on the stringlet length $l$ since $\omega_0$ exhibits this property.

We next introduce the new assumption (with respect to \cite{PhysRevB.91.094102,PhysRevB.101.174311}) into this type of quasi-continuum framework which is an exponential distribution for the stringlet length,
\begin{equation}
    p(l)=p_0 \,e^{-l/\lambda}\label{di}
\end{equation}
which is motivated directly by the simulation results \cite{10.1063/5.0039162,Hu2022,PhysRevResearch.5.023055}. Here, $\lambda$ is the average stringlet length. Notice that this distribution is different from what assumed in \cite{PhysRevB.101.174311}, where a localized Gaussian-like distribution of lengths is considered (see Fig.7 therein). As shown in \cite{10.1063/5.0039162,Hu2022,PhysRevResearch.5.023055}, this assumption is inconsistent with the simulation data.
At this point, using Eq.\eqref{ab} together with Eq.\eqref{ide}, we can obtain analytically the stringlet VDOS,
\begin{equation}
    g_s(\omega)=p_0\,\pi (d-1) \, v\,\frac{ \omega \,e^{-\frac{\pi  v}{\lambda\,\sqrt{\gamma^2
   +\omega^2}}}}{\left(\gamma^2 +\omega^2\right)^{3/2}}\,.\label{final}
\end{equation}
In the limit of zero damping, well below $T_g$, this expression becomes particularly simple,
\begin{equation}
    \lim_{\gamma \rightarrow 0}g_s(\omega)=p_0\,\pi (d-1)\,  v\,\frac{  \,e^{-\frac{\pi   v}{\lambda\,\omega}}}{\omega^2}\,.\label{final2}
\end{equation}
From Eq.\eqref{final}, we can see that in two spatial dimensions the Debye normalized stringlet VDOS $g_s(\omega)/\omega$ has a maximum $\omega_{max}$,
\begin{equation}\label{twotwo}
    \omega_{max}^{2D}=\frac{1}{3} \sqrt{\pi ^2 v^2/\lambda^2-9 \gamma^2 },
\end{equation}
which in the limit of zero damping becomes,
\begin{equation}\label{w1}
    \omega_{max}^{2D}(\gamma=0)=\frac{\pi}{3}\,\frac{v}{\lambda}.
\end{equation}
In 3D, there is also an analytical maximum in $g_s(\omega)/\omega^2$ but the expression is complicated. The effect of the friction $\gamma$ is always that of softening the bare prediction, as shown explicitly for the 2D case in Eq.\eqref{twotwo}. The 3D solution simplifies in the zero damping limit, for which we obtain the concise result,
\begin{equation}\label{w2}
    \omega_{max}^{3D}(\gamma=0)=\frac{\pi}{4}\,\frac{v}{\lambda}.
\end{equation}
Finally, we can consider the full vibrational density of states $g_{tot}(\omega)$, including the phononic Debye contribution. We then arrive at the simple relation,
\begin{equation}
    g_{\text{tot}}(\omega)=g_{\text{Debye}}(\omega)+g_s(\omega),
\end{equation}
where, in 3D for example,
\begin{equation}
    g_{\text{Debye}}(\omega)=\frac{\omega^2}{2\pi^2 \bar{v}^3},\quad \text{and}\quad \frac{1}{\bar{v}^3}=\frac{1}{v_L^3}+\frac{2}{v_T^3},
\end{equation}
where $v_L,v_T$ are respectively the speed of longitudinal and tranverse acoustic phonons. Thus, the arguments above indicate that the boson peak frequency $\omega_{BP}$ coincides with the location of the maximum in the stringlet VDOS $g_s(\omega)$.

We notice that a similar expression for the BP frequency as in Eqs.\eqref{w1}-\eqref{w2} was suggested before by Granato \cite{PhysRevLett.68.974}, Hong et al. \cite{PhysRevE.83.061508,HONG2011351} and Kalampounias et al. \cite{10.1063/1.2136878} on a more heuristic basis (see also \cite{MALINOVSKY1988111,MALINOVSKY1986757}). The length-scale appearing in Eqs.\eqref{w1}-\eqref{w2} has been interpreted in recent work as a ``cooperativity length scale'' $\xi$ \cite{PhysRevE.83.061508,HONG2011351} and that there is some evidence that this characteristic length defined from an equation of the assumed form of Eqs.\eqref{w1}-\eqref{w2} increases with temperature, as observed for the stringlet length $\lambda$ in simulations \cite{10.1063/5.0039162,C2SM26789F}. We note that the numerical prefactor in Eqs.\eqref{w1}-\eqref{w2}, defining the characteristic length associated with the boson peak, is completely unspecified in previous studies, while it is specified exactly in our theory.\\

In summary, this simple model exhibits three striking predictions. (I) It gives an analytical and simple estimate for the BP frequency in both 2D and 3D systems, Eqs.\eqref{w1}-\eqref{w2}. These results imply that the BP in a 3D system should appear at lower energies compared to a 2D system with the same length distribution and sound speed. (II) If we reasonably assume that the speed of the stringlet excitation is the same of that of transverse phonons, this model immediately predicts that $\omega_{BP}\sim G^{1/2}$ in the low temperature glass state where the stringlet length $\lambda$ is independent of temperature. (III) At finite temperature, however, the theory predicts a linear frequency dependence of the VDOS, and a $1/\omega$ divergence of the reduced density of states, the density of states normalized by the Debye density of states appropriate to a glass-forming material in its low temperature solid state.\\




% Figure environment removed

\color{blue}\textbf{Prediction of the boson peak deep in the glass state} \color{black} -- Given the numerous theoretical models proposed for the BP anomaly, it is important to make a step forward and verify the predictions of the stringlet model from a more quantitative point of view. In order to do so, we resort to the following strategy. First, we obtain the value of the parameter $\lambda$ in Eq.\eqref{di} by fitting the distribution of stringlet length obtained from simulations. Then, we take the value for the velocity of transverse phonons $v_T$ which is obtained by fitting the dispersion relation obtained from simulations. Once these two parameters are known, and fixed by the simulations, our theory provides a parameter-free prediction for the BP frequency in both 2D and 3D amorphous solids, Eqs.\eqref{w1}-\eqref{w2}. We can then directly compare the predicted value for $\omega_{BP}$, which we will denote as $\omega_{BP}^{th}$, with the experimental numbers obtained from the simulation data, which we will label as $\omega_{BP}^{exp}$.\\

We start by considering the simulation data for 2D and 3D  zero temperature glasses presented in \cite{Hu2022,PhysRevResearch.5.023055}. In Fig.\ref{fig:1}, we present an explicit example performed using the data for the 2DPL system available in \cite{Hu2022}. By plausibly assuming that the average velocity of the stringlet propagation coincides with the speed of transverse sound, we obtain a prediction for the BP of the 3D glass given by $\omega_{BP}^{th} \approx 1.344$ which has to be compared with the reported simulation value $\omega_{BP}^{exp}=1.41$. The difference between the two is less than $5\%$. 

To support these arguments, we performed the same analysis for all the data available for 2D and 3D systems in \cite{Hu2022,PhysRevResearch.5.023055} (see Fig.\ref{fig:s1} in the Supplementary Information, \color{blue}SI\color{black}). The predictions of the theory and the compared experimental values are reported in Table \ref{tab1}.
\begin{table}
  \centering
  \begin{tblr}{
      colspec={lllll},
      row{1}={font=\bfseries},
      column{1}={font=\itshape},
      row{even}={bg=gray!10},
    }
              & $\lambda$  & $v_T$  & $\omega_{BP}^{th}$  & $\omega_{BP}^{exp}$ & $\Upsilon$ \\
    \toprule
    3DIPL \cite{PhysRevResearch.5.023055} & $1.335$ & $4.23$ & $2.487$ & $2.46$ & $1\%$\\
    2DPL \cite{Hu2022} & $3.717$ & $4.78$ & $1.344$ & $1.41$ & $4.7\%$\\
    \bottomrule
  \end{tblr}
   \caption{The theoretical prediction versus the experimental value for the BP frequency using the exponential fit, Eq.\eqref{di}. The uncertainty $\Upsilon$ is defined as $\Upsilon= 2 |\omega_{BP}^{th}-\omega_{BP}^{exp}|/(\omega_{BP}^{th}+\omega_{BP}^{exp})$. All quantities presented here are in reduced Lennard-Jones units.}
  \label{tab1}
\end{table}
The estimates from the theoretical model are in remarkable agreement, \textit{i.e.} agreement to within $5\%$ uncertainty, with the value for the BP frequency obtained from simulations. It is emphasized that, once the distribution of length is extracted from the experimental data and the value of the speed of transverse sound is taken from the simulation dispersion relation, the theory has no free parameters.

A more careful analysis reveals, however, that the fit to an exponential becomes rather uncertain for longer strings which are relatively rare in the simulation sampling, \textit{i.e.}, $l>15$ in reduced LJ units (see inset in Fig.\ref{fig:1}). This was already observed in \cite{10.1063/1.4878502}, where a combination of a power-law with an exponential was considered and shown to fit the data better. In order to investigate this point further, we have performed a second analysis in which the stringlet length distribution $p(l)$ is not fitted with the exponential form, Eq.\eqref{di}, but rather taken as an experimental input for the stringlet theory of the boson peak. The stringlet VDOS can be derived using Eq.\eqref{ide}, even if not analytically available this time. The theoretical predictions without using the exponential fitting function are presented in Fig.\ref{fig:2} for the 3DIPL system in \cite{PhysRevResearch.5.023055}. A similar analysis for the 2DPL system \cite{Hu2022} can be found in the \color{blue}SI \color{black}(see Fig.\ref{fig:s1}). Since the simulation data for the stringlet length distribution are quite scattered for $l>15$ (see inset in Fig.\ref{fig:2}), the corresponding theoretical predictions for $g_s(\omega)$ present large fluctuations at low frequency, below the BP scale $\omega_{BP}$. In order to obtain a better estimate from theory for $\omega_{BP}$, we fit the theoretical data with a log-normal function as suggested in \cite{MALINOVSKY199163}. Suggestively, this commonly utilized phenomenological fitting function capturing the ``shape" of the boson peak turns out to be quite reasonable, supporting the consistency of the stringlet theory in comparison to experimental observations. The maximum of the fitted log-normal function, which serves to specify the boson peak frequency following experimental practice to the value $\omega \approx 1.94$, accords with the simulation value $\omega_{BP}^{exp}=2.46$ to within a rather large uncertainty of $23 \%$. Following the same procedure for the 2DPL system \cite{Hu2022}, we find $\omega \approx 1.04$ and a corresponding $\approx 30\%$ apparent deviation from from the simulation value $\omega_{BP}^{exp}=1.41$. 

% Figure environment removed

In summary, we find that using the stringlet simulation data, with no exponential fit, leads to appreciable uncertainty in the estimation of the boson peak because of large uncertainties in extracting $p(l)$ for large $l$. Thus, the theoretical predictions of the boson peak based on the stringlet model using the exponential distribution and Eqs.\eqref{w1}-\eqref{w2} are advised with respect to quantitatively estimating the boson peak from the simulation data for $p(l)$.\\


\color{blue}\textbf{Damped dynamics in cooled liquids above the glass transition} \color{black} -- Inspired by these positive outcomes in the comparisons of the stringlet model to simulations deep in the glass state, we next consider simulations for a simulated Al-Sm metallic glass-forming material in a temperature range above $T_g$ discussed previously in \cite{10.1063/5.0039162} (see also \cite{Zhang2021}) for which there is data for the stringlet length distribution (see Fig.\ref{fig:3}), the speed of transverse sound\footnote{We obtained the values for the density $\rho$ and the shear modulus $G$ from \cite{10.1063/5.0039162,Zhang2021} and derived the transverse speed of sound using:
\begin{equation}
    v_T^2=\frac{G}{\rho}.
\end{equation}}, and the BP frequency at different temperatures between $450$K and $650$K. The data from these works required for our purposes are summarized in Table \ref{tab2}. By taking the experimental values for these quantities, we are again able to provide a parameter-free theoretical prediction for the BP frequency which is shown, in the limit of zero damping, with orange symbols in Fig.\ref{fig:3}. This result has to be compared to the experimental values for the BP frequency, which are shown with black symbols in the same figure.


\begin{table}[hb]
  \centering
  \begin{tblr}{
      colspec={lllll},
      row{1}={font=\bfseries},
      column{1}={font=\itshape},
      row{even}={bg=gray!10},
    }
           Al-Sm \cite{10.1063/5.0039162}  & $\lambda$  & $v_T$   & $\omega_{BP}^{exp}$   \\
    \toprule
    $T=450$K & $0.930$ & $2151$ & $2.681$ & \\
    $T=500$K  & $1.018$ & $2110$ &  $2.558$ \\
    $T=550$K  & $1.132$ & $2043$ &  $2.348$ \\
    $T=600$K  & $1.269$ & $1968$ &  $2.329$ \\
    $T=650$K & $1.460$ & $1851$ &  $2.226$ \\
    \bottomrule
  \end{tblr}
   \caption{The parameters for the Al-Sm system at finite temperature \cite{10.1063/5.0039162}. The parameter $\lambda$ is reported in nanometers, the velocity in m/s and the frequencies in meV.}
  \label{tab2}
\end{table}

One qualitative finding is evident from Fig.\ref{fig:3}. Simulation observations cannot be described by a theory that neglects stringlet damping. The theoretical predictions neglecting damping overestimate the observed values of $\omega_{BP}$ by more than a factor of two. The reason for this deviation is easy to understand. As opposed to the simulations of Hu and Tanaka \cite{Hu2022,PhysRevResearch.5.023055}, which were performed in the limit of $T=0$ where stringlet damping can be reasonably neglected, the simulations for the Al-Sm metallic glass forming material \cite{10.1063/5.0039162} were performed at relatively high temperatures corresponding to a highly cooled liquid. At such temperatures, it is unreasonable to assume $\gamma=0$ even as rough approximation. 

Some insight in the damping parameter can be obtained by adopting idealized assumptions about its temperature dependence and comparing to the observed behavior of how the boson peak varies with temperature. As a first crude approximation, we take $\gamma$ to be a constant and not excessively large in magnitude, \textit{i.e.}, $\gamma$ smaller than $\omega_0$ in Eq.\eqref{ab}. We see that this option does not allow for a reasonable estimate of the BP frequency; see the pink curve in Fig.\ref{fig:3}. As it might have been anticipated, damping must be temperature dependent and relatively large in magnitude at temperatures well above $T_g$. Next, we consider for the sake of argument a temperature dependent $\gamma$ that is larger in amplitude but that decreases with temperature (see Fig.\ref{fig:s1} in \color{blue}SI\color{black}), leads to a better accord between the stringlet model and simulation estimates of the boson peak, see red line in Fig.\ref{fig:3}, but the agreement is still inadequate. Moreover, it is hard to envision a microscopic reason for why the damping should decrease with temperature. A damping that is larger in amplitude and that grows with temperature would then appear to be the choice for the variation of $\gamma$. Heuristically, this problem has a simple origin. A large damping is necessary to soften the stringlet frequency and consequently the predicted value for $\omega_{BP}$, which, in absence of damping, is too large compared to the simulation observations. Nevertheless, a very large damping in Eq.\eqref{ab} clearly makes the stringlet frequency uncertain when $\gamma$, along with the stringlet length and the transverse sound velocity, all depend on temperature. We leave the task of more quantitatively understanding the temperature dependence of the boson peak for future work.

As a general expectation, however, we infer from our model that a large variation of $\gamma$ in comparison to $\omega_0$, and possibly a situation in which $\gamma>\omega_0$ and $\omega_l$ in Eq.\eqref{ab} becomes complex, are required to describe the simulated boson peak observations of the metallic glass forming liquid at large temperature. This indicates that the stringlets dynamics ultimately become completely overdamped in the high temperature liquid regime and the present formalism requires generalization. In particular, we have assumed that the effects of damping enter only into the renormalization of the stringlet energy $\omega_l$ in Eq.\eqref{ab}. This is perhaps highly questionable at high temperatures, since the lifetime of the stringlets vibrations is short so that the imaginary part of their frequency $\omega_l$ cannot then be neglected. As already explained, this assumption is clearly not valid if $\gamma$ is comparable or larger than $\omega_0$ in Eq.\eqref{ab}. Among the other possible imitations of the stringlet model in this high temperature regime, we have also made the bold assumption that the stringlet damping is independent of the stringlet length $l$. Again, we emphasize that these and other issues with the stringlet model will require a thorough examination in the future.

% Figure environment removed

For 3D systems, the stringlet model with finite $\gamma$ formally predicts another interesting feature. In particular, using the stringlet reduced VDOS presented in Eq.\eqref{final}, one can derive the low-frequency behavior of the stringlet VDOS as
\begin{equation}\label{div}
    \frac{g_s(\omega)}{\omega^2}=\frac{p(0)\,\pi\,(d-1)\,v}{\gamma^2}\frac{1}{\omega}+\dots
\end{equation}
The expression above indicates a low-frequency divergence in the reduced VDOS, when the stringlet dynamics are strongly damped, \textit{i.e.}, $\gamma \gg 0$. In Fig.\ref{fig:4}, we show the reduced stringlet VDOS as deduced from Eq.\eqref{final} by increasing $\gamma$ while keeping all the other parameters held constant. The result is evident. As expected, the BP frequency is softened by the introduction of damping and the boson peak becomes broader. Additionally, a $1/\omega$ divergence starts to become prevalent in the low frequency range, a trend that becomes progressively stronger with increasing $\gamma$. When the damping becomes sufficiently strong, the BP feature completely disappears and the shape of the VDOS at intermediate temperatures is often found to be similar to the curves in Fig.\ref{fig:4} \cite{PhysRevLett.92.245508,doi:10.1021/jp4104905,KHODADADI201015,Takahashi2012}. 

% Figure environment removed

Note also that the result in Eq.\eqref{div} implies a VDOS which is linear in frequency at low energy. This behavior is consistent with the experimental observations in classical liquids \cite{PhysRevLett.63.2381,dehong2022,jin2023dissecting} where it has been rationalized based on the concept of unstable modes \cite{Keyes1997,doi:10.1073/pnas.2022303118}. Interestingly, string-like excitations have been also discussed in glass-forming liquids at low temperatures \cite{PhysRevLett.80.2338,10.1063/1.453836,10.1063/1.4878502}, so that it would be important to study further the BP in relation to stringlet dynamics by heating such cooled liquids to watch how the boson peak evolves in shape and eventually disappears. Notably, this disappearance of the boson peak upon heating is well-established experimentally \cite{PhysRevE.83.061508,PhysRevLett.104.067402,PhysRevLett.92.245508,Zanatta2011-ZANTEO-3,10.1063/1.2136878,González-Jiménez2023}. Similarities to the occurrence of glassy anomalies in materials undergoing commensurate-incommensurate transitions \cite{PhysRevLett.93.245902,PhysRevLett.114.195502,PhysRevLett.76.2334,jiang2023glassy,BILJAKOVIC20121741} and we suggest that the pursuit of this ``analogy" might offer potential insights into the boson peak in glass-forming liquids. \\

\color{blue}\textbf{Outlook} \color{black} -- In this work, we showed that a simple theoretical model based on a distribution of 1D vibrating strings \cite{PhysRevB.91.094102} -- the stringlet theory of the boson peak -- provides an excellent description of the BP frequency $\omega_{BP}$ for 2D and 3D amorphous systems in the glass state, where the damping of these excitations is reasonably be assumed to be negligible. In particular, this theoretical framework gives an analytical, and parameter-free, prediction for $\omega_{BP}$ that is in quantitative agreement with recent simulation data at $T=0$ \cite{Hu2022,PhysRevResearch.5.023055}. Moreover, the observed scaling of the BP frequency with the shear modulus in this type of zero temperature simulation, $\omega_{BP}\sim G^{1/2}$, follows as an analytic prediction from the stringlet model. 

We tentatively explored the stringlet model in the high temperature ``cooled liquid'' regime above $T_g$ where some damping of the stringlet modes can naturally be expected. In this temperature regime, we addressed the qualitative temperature dependence of the boson peak frequency and its disappearance at high temperatures, in connection with the dynamic correlation length introduced phenomenologically in previous work investigating boson peak measurements \cite{PhysRevE.83.061508,HONG2011351}. The qualitative softening of $\omega_{BP}$ upon heating is reproduced by the stringlet model and attributed mainly to the change of the stringlet length and transverse sound velocity with temperature. Our qualitative analysis suggests that the stringlet modes become completely overdamped in the liquid regime at sufficiently high temperatures. In association with this phenomenon, we observe the emergence of a density of states that changes to a linear variation with frequency rather than the quadratic law expected for the Debye theory of solids, a transition noted in recent studies attempting to define a density of states appropriate to materials in their liquid state \cite{jin2023dissecting,Keyes1997,doi:10.1073/pnas.2022303118,dehong2022}. We finally mention that recent studies \cite{10.1063/5.0039162,C2SM26789F,10.1063/1.5009442,Hu2022} of the boson peak frequency as a function of temperature show a nearly linear behavior of $\omega_{BP}$ with $G$, rather than $\propto G^{1/2}$, an observation that offers some clues about the variation of the boson peak in the temperature regime where the average stringlet length, the sound velocity and $\gamma$ are all varying in their own individual ways with temperature. Recent measurements of the pressure dependence of the boson peak \cite{PhysRevLett.99.055502,ahart2017pressure} indicating a deviation from the simple scaling of the boson peak with the transverse sound velocity also offer some clues about this more complicated regime for temperatures near or above $T_g$.

Our analysis supports the idea that the stringlets correspond to the localized modes suggested \cite{10.1063/5.0039162,Hu2022,PhysRevResearch.5.023055} to be responsible for the BP anomaly in amorphous systems. This identification raises questions about how these modes might be related to the multipolar four-leaf modes \cite{PhysRevLett.117.035501,doi:10.1073/pnas.1709015114} previously discussed in the literature in connection with the emergence of the boson peak. 

In order to augment the predictive nature of the stringlet theory, and to develop it into a full-fledged theory, clearly more work needs to be done. First, a theoretical model is required for the first-principle prediction of the stringlet length distribution $p(l)$, without resorting to simulation data. A promising avenue would be to combine Lund model \cite{PhysRevB.91.094102} with the the well-developed thermodynamic theory of equilibrium polymers discussed in Refs.\cite{10.1063/1.4878502,10.1063/1.2909195,10.1063/1.2356863}. Second, it would be interesting to directly estimate of the average string length $\lambda$ as a function of temperature, and compare it with independent estimates of the dynamic correlation length based on Eqs.\eqref{w1}-\eqref{w2}, to verify that that this length can be interrelated as cooperativity length, as proposed heuristically in some experimental studies \cite{PhysRevE.83.061508,HONG2011351}. Specifically, Eqs.\eqref{w1}-\eqref{w2} are equivalent to the ``standard'' relation taken to define the dynamic correlation length by Hong et al. \cite{PhysRevE.83.061508,HONG2011351} and others \cite{10.1063/1.2136878}, when the parameter $S$ of these experimental studies is taken to have the specific predicted value, $S=\pi/(d+1)$.

Finally, the stringlet dynamics share many similarities with the behavior of defects and anharmonic modes in crystals \cite{PhysRevLett.68.974,TAKENO19881023} (see \cite{PhysRevB.105.014204} for a direct proof of the stringlet anharmonicity in relation to the BP in glasses) and the role of weakly-dispersing soft optical modes, which has been identified in several instances as the origin for the BP anomaly in crystalline materials with no structural disorder \cite{PhysRevB.23.3886,PhysRevLett.76.2334,PhysRevLett.114.195502,PhysRevLett.93.245902,PhysRevB.99.024301,RevModPhys.86.669,Schliesser_2015,doi:10.1021/acs.jpclett.2c01224,Baggioli_2020}. A unifying picture based on defect dynamics would be highly desirable and possible related also to the roton-like excitations in liquids and glasses \cite{PhysRevLett.49.1271,PhysRevLett.50.49,DESCHEPPER198429}, as reviewed in \cite{ASTRATH20063368}. Part of the dispersion curve with ``softer'' phonons (like rotons in liquid helium) have been suggested to be related to the BP anomaly \cite{KOVALENKO1990115,KRASNY199892} (see also \cite{PhysRevResearch.4.029001}), and might be connected with the dynamics of stringlets discussed in this work.

Our work revives the long-standing question of whether the structures responsible for the BP are structural defects or dynamical excitations deriving from the anharmonicity of intermolecular interactions. The evidence would appear to favor the latter interpretation, explaining why the boson peak should also be expected in crystalline materials at elevated temperatures where anharmonicity in intermolecular interactions becomes prevalent as in cooled liquids. Recent simulations \cite{10.1063/1.5091042} have indeed confirmed that the boson peak also arises in crystalline materials at elevated temperatures, but below the melting temperature, where anharmonic intermolecular interactions start to become prevalent.\\

\color{blue}\textbf{Acknowledgments} \color{black} -- We would like to thank A.~Zaccone, J.~Zhang and M.~A.~Ramos for many discussions about the BP and the vibrational dynamics of amorphous solids. We are grateful to Y.~Wang. and W.~Xu for useful comments on a preliminary version of this draft. CJ and MB acknowledge the support of the Shanghai Municipal Science and Technology Major Project (Grant No.2019SHZDZX01). MB acknowledges the sponsorship from the Yangyang Development Fund.
%


\bibliographystyle{apsrev4-1}
\bibliography{ref}
\onecolumngrid
\clearpage
\appendix 
\renewcommand\thefigure{S\arabic{figure}}    
\setcounter{figure}{0} 
\renewcommand{\theequation}{S\arabic{equation}}
\setcounter{equation}{0}
\renewcommand{\thesubsection}{SM\arabic{subsection}}
\section*{Supplementary Information}
We report some additional figures regarding the analysis presented in the main text. 
% Figure environment removed
\end{document}
