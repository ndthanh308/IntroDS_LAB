\section{Open BPS invariants, integrality, and finiteness}\label{sect:OpenBPS}
In this section, we define the open BPS invariants of $(X,L,\vf)$ via resummation of the open Gromov-Witten invariants and prove their integrality and finiteness properties (Theorem \ref{thm:OpenBPS}).


\subsection{Definitions and statements}
Given effective class $(\beta, \vmu) \in \Eff(X,L, \vf)$, we consider the following resummation formula of Labastida-Mari\~no-Ooguri-Vafa (LMOV) \cite{OV00,LM00,LMV00,MV02}:
\begin{equation}\label{eqn:LMOVResumGs}
    \begin{aligned}
        F^{X, L, \vf}_{\beta, \vmu}(g_s) & = \sum_{g \in \bZ_{\ge 0}} N^{X, L, \vf}_{g, \beta, \vmu} g_s^{2g-2+\ell(\vmu)}\\
        &= \frac{1}{\prod_{i = 1}^s z_{\mu^i}} \sum_{k \mid \beta, \vmu}  \sum_{g \in \bZ_{\ge 0}} (-1)^{\ell(\vmu)+g}k^{\ell(\vmu)-1}n^{X, L, \vf}_{g,\frac{\beta}{k}, \frac{\vmu}{k}}\left(2\sin \frac{kg_s}{2} \right)^{2g-2} \prod_{i=1}^s \prod_{j=1}^{\ell(\mu^i)} 2\sin \frac{\mu^i_jg_s}{2}\\
        &= \sum_{k \mid \beta, \vmu} \sum_{g \in \bZ_{\ge 0}} \frac{(-1)^{\ell(\vmu)+g}n^{X, L, \vf}_{g,\frac{\beta}{k}, \frac{\vmu}{k}}}{k \prod_{i = 1}^s z_{\frac{\mu^i}{k}}} \left(2\sin \frac{kg_s}{2} \right)^{2g-2} \prod_{i=1}^s \prod_{j=1}^{\ell(\mu^i)} 2\sin \frac{\mu^i_jg_s}{2}.
    \end{aligned}
\end{equation}
Here for $k \in \bZ_{\ge 1}$, we say that $k \mid \beta$ if $\frac{\beta}{k} \in H_2(X,L;\bZ)$, and $k \mid \vmu$ if $k \mid \mu^i$ for each $i$ (or equivalently $\frac{\vmu}{k}$ exists). We refer to the coefficient
$$
    n^{X, L, \vf}_{g, \beta, \vmu} \in \bQ
$$
determined by \eqref{eqn:LMOVResumGs} as the genus-$g$, degree-$(\beta, \vmu)$ \emph{open BPS invariant} of $(X, L, \vf)$. Our main results of this section are the following properties of these invariants.

\begin{theorem}\label{thm:OpenBPS}
For any $(\beta, \vmu) \in \Eff(X, L, \vf)$, we have that
$$
    n^{X, L, \vf}_{g, \beta, \vmu} \in \bZ
$$
for all $g \in \bZ_{\ge 0}$ and is zero for $g \gg 0$.
\end{theorem}


To prove Theorem \ref{thm:OpenBPS}, we first rewrite the resummation formula \eqref{eqn:LMOVResumGs} using the change of variables $q = e^{\sqrt{-1}g_s}$. For $a \in \bZ_{\ge 1}$, we write
$$
    [a]:= q^{\frac{a}{2}} - q^{-\frac{a}{2}},
$$
and for any partition $\lambda \in \cP$, we write
$$
    [\lambda]:= \prod_{j=1}^{\ell(\lambda)} [\lambda_j].
$$
Note that
$$
    2\sin \frac{ag_s}{2} = \frac{[a]}{\sqrt{-1}}.
$$
Then \eqref{eqn:LMOVResumGs} gives
\begin{equation}\label{eqn:LMOVResumQ}
    F^{X, L, \vf}_{\beta, \vmu}(q) = \sum_{k \mid \beta, \vmu} \sum_{g \in \bZ_{\ge 0}} \frac{-\sqrt{-1}^{\ell(\frac{\vmu}{k})}n^{X, L, \vf}_{g,\frac{\beta}{k}, \frac{\vmu}{k}}}{k \prod_{i = 1}^s z_{\frac{\mu^i}{k}}} [k]^{2g-2} \prod_{i=1}^s \prod_{j=1}^{\ell(\frac{\mu^i}{k})} [k \cdot \frac{\mu^i_j}{k}].
\end{equation}
We define an auxiliary function
$$
    H^{X, L, \vf}_{\beta, \vmu}(q) := \sum_{g \in \bZ_{\ge 0}} \frac{-\sqrt{-1}^{\ell(\vmu)}n^{X, L, \vf}_{g, \beta, \vmu}}{ \prod_{i = 1}^s z_{\mu^i}} [1]^{2g-2} \prod_{i=1}^s [\mu^i].
$$
Then \eqref{eqn:LMOVResumQ} can be rewritten as
$$
    F^{X, L, \vf}_{\beta, \vmu}(q) = \sum_{k \mid \beta, \vmu} \frac{1}{k} H^{X, L, \vf}_{\frac{\beta}{k}, \frac{\vmu}{k}}(q^k)
$$
which is equivalent to
\begin{equation}\label{eqn:HtoF}
    H^{X, L, \vf}_{\beta, \vmu}(q) = \sum_{k \mid \beta, \vmu} \frac{\mu(k)}{k} F^{X, L, \vf}_{\frac{\beta}{k}, \frac{\vmu}{k}}(q^k).
\end{equation}
Here, $\mu(k)$ is the \emph{M\"obius function} and satisfies that
$$
    \sum_{k' \mid k} \mu(k') = \begin{cases}
        1 & \text{if } k=1,\\
        0 & \text{for } k \in \bZ_{\ge 2}.
    \end{cases}
$$

Moreover, we set up a new formal variable
$$
    t := [1]^2 = q + q^{-1} -2
$$
and define a generating function
\begin{equation}\label{eqn:GDef}
    G^{X, L, \vf}_{\beta, \vmu}(t) := \sum_{g \in \bZ_{\ge 0}} -n^{X, L, \vf}_{g, \beta, \vmu} t^{g-1}.
\end{equation}
The definitions imply that
\begin{equation}\label{eqn:GtoH}
    G^{X, L, \vf}_{\beta, \vmu}(t) = H^{X, L, \vf}_{\beta, \vmu}(q) \prod_{i=1}^s \frac{z_{\mu^i}}{\sqrt{-1}^{\ell(\mu^i)}[\mu^i]}. 
\end{equation}

The proof of Theorem \ref{thm:OpenBPS} is based on the following two technical lemmas, which are the counterparts of \cite[Propositions 3.2, 3.3]{Konishi06b} respectively.

\begin{lemma}\label{lem:GIntegral}
For any $(\beta, \vmu) \in \Eff(X, L, \vf)$, we have that
$$
    G^{X, L, \vf}_{\beta, \vmu}(t) \in \cL[t].
$$
\end{lemma}

Here,
$$
    \cL[t] := \left\{ \frac{a(t)}{b(t)} \biggm| a(t) \in \bZ[t], b(t) \in \bZ_0[t], b(t) \neq 0  \right\}
$$
where $\bZ_0[t] \subset \bZ[t]$ is the subring of monic polynomials in $t$.


\begin{lemma}\label{lem:GFinite}
For any $(\beta, \vmu) \in \Eff(X, L, \vf)$, we have that
$$
    tG^{X, L, \vf}_{\beta, \vmu}(t) \in \bQ[t]. 
$$
\end{lemma}

The proofs of the lemmas will be given in Sections \ref{sect:ProofIntegral}, \ref{sect:ProofFinite} below respectively.

\begin{proof}[Proof of Theorem \ref{thm:OpenBPS}]
As observed by \cite{Konishi06a,Konishi06b}, Lemmas \ref{lem:GIntegral} and \ref{lem:GFinite} together imply that
$$
    tG^{X, L, \vf}_{\beta, \vmu}(t) \in \bZ[t]
$$
which is equivalent to the theorem.
\end{proof}



\begin{comment}

We first show that Lemma \ref{lem:GIntegral} implies Theorem \ref{thm:OpenBPS} using an argument similar to one in \cite[Section 5.2]{Peng07}.

\begin{proof}[Proof of Theorem \ref{thm:OpenBPS} via Lemma \ref{lem:GIntegral}]
We use Lemma \ref{lem:GIntegral} to write
$$
    G^{X, L, \vf}_{\beta, \vmu}(t) = \frac{a(t)}{b(t)}
$$
where $a(t), b(t) \in \bZ[t]$ and $b(t) \neq 0$ is monic. Viewing $a$ and $b$ as Laurent polynomials in $q$, we see that $G^{X, L, \vf}_{\beta, \vmu}$ is a rational function in $q$ and in particular, $q=0$ is a pole of $G^{X, L, \vf}_{\beta, \vmu}$. This implies that $n^{X, L, \vf}_{g, \beta, \vmu} = 0$ for $g \gg 0$, since otherwise $q=0$ would be an essential singularity of $G^{X, L, \vf}_{\beta, \vmu}$. Thus
the definition \eqref{eqn:GDef} implies that
$$
    tG^{X, L, \vf}_{\beta, \vmu}(t) \in \bQ[t].
$$
Since $b(t)$ is monic, we have in fact that
$$
    tG^{X, L, \vf}_{\beta, \vmu}(t) \in \bZ[t],
$$
which implies that $n^{X, L, \vf}_{g, \beta, \vmu} \in \bZ$ for all $g \in \bZ_{\ge 0}$.
\end{proof}
\end{comment}


\begin{comment}
In view of \eqref{eqn:OpenRelFunction}, given effective class $(\vd, \vmu)$ of $\Gamma$, we consider the following resummation formula:
\begin{equation}\label{eqn:LMOVResumRel}
    \begin{aligned}
        F^{\hY, \hD}_{\vd, \vmu}(g_s) &= \sum_{g \in \bZ_{\ge 0}} N^{\hY, \hD}_{g, \vd, \vmu} g_s^{2g-2+\ell(\vmu)}\\
        &= \sum_{g \in \bZ_{\ge 0}} \sum_{k \mid \beta, \vmu} \frac{(-1)^{|\vmu|+g}n^{\hY, \hD}_{g, \frac{\vd}{k}, \frac{\vmu}{k}}}{k \prod_{i = 1}^s z_{\frac{\mu^i}{k}}} \left(2\sin \frac{kg_s}{2} \right)^{2g-2} \prod_{i=1}^s \prod_{j=1}^{\ell(\mu^i)} 2\sin \frac{\mu^i_jg_s}{2}.
    \end{aligned}
\end{equation}
Here for $k \in \bZ_{\ge 1}$, we say that $k \mid \vd$ if $\frac{\vd}{k}$ is a $\bZ$-valued function on $E_c(\Gamma)$. We refer to the coefficient
$$
    n^{\hY, \hD}_{g, \vd, \vmu} \in \bQ
$$
determined by \eqref{eqn:LMOVResumRel} as the genus-$g$, degree-$(\vd, \vmu)$ \emph{formal relative BPS invariants} of $(\hY, \hD)$. It follows from \eqref{eqn:OpenRelFunction} that
\begin{equation}\label{eqn:OpenRelBPS}
    n^{X, L, \vf}_{g, \beta, \vmu} = \sum_{\vd: \pi(\vd) = \beta} n^{\hY, \hD}_{g, \vd, \vmu}.
\end{equation}


Our main results of this section are the integrality and finiteness properties of the BPS invariants. We start with the following statement for formal relative BPS invariants:

\begin{theorem}\label{thm:RelBPS}
Let $(\vd, \vmu)$ be an effective class of $\Gamma$. Then $n^{\hY, \hD}_{g, \vd, \vmu}$ is an integer for all $g \in \bZ_{\ge 0}$ and is zero for $g \gg 0$.
\end{theorem}

Theorem \ref{thm:RelBPS} combined with \eqref{eqn:OpenRelBPS} immediately implies the following statement for open BPS invariants:

\begin{theorem}\label{thm:OpenBPS}
Let $(\beta, \vmu)$ be an effective class of $(X, L, \vf)$. Then $n^{X, L, \vf}_{g, \beta, \vmu}$ is an integer for all $g \in \bZ_{\ge 0}$ and is zero for $g \gg 0$.
\end{theorem}
\end{comment}


\subsection{Proof of Lemma \ref{lem:GIntegral}}\label{sect:ProofIntegral}
Now we prove Lemma \ref{lem:GIntegral}. We start by using \eqref{eqn:GtoH}, \eqref{eqn:HtoF}, \eqref{eqn:OpenRelFunction} to rewrite the generating function $G^{X, L, \vf}_{\beta, \vmu}(t)$ as follows:
\begin{equation}\label{eqn:GRewrite1}
\begin{aligned}
    G^{X, L, \vf}_{\beta, \vmu}(t) & = H^{X, L, \vf}_{\beta, \vmu}(q) \prod_{i=1}^s \frac{z_{\mu^i}}{\sqrt{-1}^{\ell(\mu^i)}[\mu^i]}\\
    & = \prod_{i=1}^s \frac{z_{\mu^i}}{\sqrt{-1}^{\ell(\mu^i)}[\mu^i]}  \sum_{k \mid \beta, \vmu} \frac{\mu(k)}{k} F^{X, L, \vf}_{\frac{\beta}{k}, \frac{\vmu}{k}}(q^k)\\
    & = \prod_{i=1}^s \frac{\sqrt{-1}^{\ell(\mu^i)}z_{\mu^i}}{[\mu^i]} \sum_{k \mid \beta, \vmu} \frac{\mu(k)}{k} (-1)^{|\frac{\vmu}{k}|}\sum_{\vd: \pi(\vd) = \frac{\beta}{k}} F^{\hY, \hD}_{g, \vd, \frac{\vmu}{k}}(q^k).
\end{aligned}
\end{equation}

Now we bring in the topological vertex to continue rewriting \eqref{eqn:GRewrite1}, using techniques in \cite[Section 5]{Konishi06b}. Note that $F^{\hY, \hD}_{g, \vd, \frac{\vmu}{k}}(q^k)$ is the coefficient of $Q^{\vd}P_{\frac{\vmu}{k}}$ in
$$
    \ln \left(Z^{\hY, \hD}(q^k, Q, P)\right).
$$
To extract this coefficient, we introduce some notations. For any effective class $(\vd, \vmu) \in \Eff(\Gamma)$, we denote
$$
    \cD(\vd, \vmu) := \{ (\vdelta, \vnu) \in \Eff(\Gamma) \mid \vdelta \le \vd, \vnu \subseteq \vmu\},
$$
and
$$
    \cA(\vd, \vmu) := \left\{ \va: \cD(\vd, \vmu) \to \bZ_{\ge 0} \biggm| \sum_{(\vdelta, \vnu) \in \cD(\vd, \vmu)} \va(\vdelta, \vnu) \vdelta = \vd, \bigsqcup_{(\vdelta, \vnu) \in \cD(\vd, \vmu)} \vnu^{(\va(\vdelta, \vnu))} = \vmu \ \right\}
$$
which is the set of $\va$ such that
$$
    \prod_{(\vdelta, \vnu) \in \cD(\vd, \vmu)} (Q^{\vdelta}P_{\vnu})^{\va(\vdelta, \vnu)} = Q^{\vd}P_{\vmu}.
$$
For $\va \in \cA(\vd, \vmu)$, we write
$$
    |\va| := \sum_{(\vdelta, \vnu) \in \cD(\vd, \vmu)} \va(\vdelta, \vnu), \qquad \gcd(\va) := \gcd(\{\va(\vdelta, \vnu) \mid (\vdelta, \vnu) \in \cD(\vd, \vmu) \}).
$$
Moreover, we denote
$$
    \cA_1(\vd, \vmu) := \{\va \in \cA(\vd, \vmu) \mid \gcd(\va)=1\}.
$$

Returning to \eqref{eqn:GRewrite1}, for any $\vd$ such that $\pi(\vd) = \frac{\beta}{k}$, we have
\begin{equation}\label{eqn:FRewrite}
\begin{aligned}
    F^{\hY, \hD}_{g, \vd, \frac{\vmu}{k}}(q^k) = & \sum_{\va \in \cA(\vd, \frac{\vmu}{k})} \frac{|\va|!}{\prod_{(\vdelta, \vnu) \in \cD(\vd, \frac{\vmu}{k})} \va(\vdelta, \vnu)!} \frac{(-1)^{|\va|-1}}{|\va|} \prod_{(\vdelta, \vnu) \in \cD(\vd, \frac{\vmu}{k})} \left(Z^{\hY, \hD}_{\vdelta, \vnu}(q^k)\right)^{\va(\vdelta, \vnu)}\\
    = & \sum_{\substack{k' \mid \vd \\ (\frac{\vmu}{k})^{(\frac{1}{k'})} \text{ exists}}} \sum_{\va \in \cA_1\left(\frac{\vd}{k'}, (\frac{\vmu}{k})^{(\frac{1}{k'})}\right)} \frac{(k'|\va|)!}{\displaystyle\prod_{(\vdelta, \vnu) \in \cD\left(\frac{\vd}{k'}, (\frac{\vmu}{k})^{(\frac{1}{k'})}\right)} (k'\va(\vdelta, \vnu))!} \frac{(-1)^{k'|\va|-1}}{k'|\va|} \\
    &  \cdot \prod_{(\vdelta, \vnu) \in \cD\left(\frac{\vd}{k'}, (\frac{\vmu}{k})^{(\frac{1}{k'})}\right)} \left(Z^{\hY, \hD}_{\vdelta, \vnu}(q^k)\right)^{k'\va(\vdelta, \vnu)}.
\end{aligned}
\end{equation}
By Theorem \ref{thm:TopVertex}, for $(\vdelta, \vnu) \in \cD\left(\frac{\vd}{k'}, (\frac{\vmu}{k})^{(\frac{1}{k'})}\right)$ as above,
$$
    Z^{\hY, \hD}_{\vdelta, \vnu}(q^k) = \sum_{\vlambda \in T_{\vdelta}} \prod_{\bar{e} \in E(\Gamma)} (-1)^{(n^e + 1)\vdelta(\bar{e})} q^{\frac{k\kappa_{\vlambda(e)}n^e}{2}} \prod_{v \in V^3(\Gamma)} \cW_{\vlambda^v}(q^k) \prod_{i = 1}^s \frac{\chi_{\vlambda(-e_i)}(\nu^i)}{z_{\nu^i}}\sqrt{-1}^{\ell(\nu^i)}(-1)^{|\nu^i|}.
$$
This combined with \eqref{eqn:GRewrite1} gives
\begin{equation}\label{eqn:GRewrite2}
    \begin{aligned}
        & G^{X, L, \vf}_{\beta, \vmu}(t) = \prod_{i=1}^s \frac{\sqrt{-1}^{\ell(\mu^i)}z_{\mu^i}}{[\mu^i]} \sum_{k \mid \beta, \vmu} \frac{\mu(k)}{k} (-1)^{|\frac{\vmu}{k}|}  \sum_{\vd: \pi(\vd) = \frac{\beta}{k}} \sum_{\substack{k' \mid \vd \\ (\frac{\vmu}{k})^{(\frac{1}{k'})} \text{ exists}}}\\
        \phantom{aaaa} &  \sum_{\va \in \cA_1\left(\frac{\vd}{k'}, (\frac{\vmu}{k})^{(\frac{1}{k'})}\right)} \frac{(k'|\va|)!}{\displaystyle\prod_{(\vdelta, \vnu) \in \cD\left(\frac{\vd}{k'}, (\frac{\vmu}{k})^{(\frac{1}{k'})}\right)} (k'\va(\vdelta, \vnu))!} \frac{(-1)^{k'|\va|-1}}{k'|\va|}  \prod_{(\vdelta, \vnu) \in \cD\left(\frac{\vd}{k'}, (\frac{\vmu}{k})^{(\frac{1}{k'})}\right)}\\
        \phantom{aaaa} & \left(\sum_{\vlambda \in T_{\vdelta}} \prod_{\bar{e} \in E(\Gamma)} (-1)^{(n^e + 1)\vdelta(\bar{e})} q^{\frac{k\kappa_{\vlambda(e)}n^e}{2}} \prod_{v \in V^3(\Gamma)} \cW_{\vlambda^v}(q^k) \prod_{i = 1}^s \frac{\chi_{\vlambda(-e_i)}(\nu^i)}{z_{\nu^i}}\sqrt{-1}^{\ell(\nu^i)}(-1)^{|\nu^i|}\right)^{k'\va(\vdelta, \vnu)}.
    \end{aligned}
\end{equation}
Here, we have by definition that
$$
    \frac{\vmu}{k} = \bigsqcup_{(\vdelta, \vnu) \in \cD\left(\frac{\vd}{k'}, (\frac{\vmu}{k})^{(\frac{1}{k'})}\right)} \vnu^{(k'\va(\vdelta, \vnu))}
$$
and in particular
$$
    \ell(\vmu) = \ell\left(\frac{\vmu}{k}\right) = \sum_{(\vdelta, \vnu) \in \cD\left(\frac{\vd}{k'}, (\frac{\vmu}{k})^{(\frac{1}{k'})}\right)} k'\va(\vdelta, \vnu) \ell(\vnu), \qquad 
    \left|\frac{\vmu}{k}\right| =  \sum_{(\vdelta, \vnu) \in \cD\left(\frac{\vd}{k'}, (\frac{\vmu}{k})^{(\frac{1}{k'})}\right)} k'\va(\vdelta, \vnu) |\vnu|.
$$
Then \eqref{eqn:GRewrite2} simplifies to
\begin{equation}\label{eqn:GRewrite3}
    \begin{aligned}
        G^{X, L, \vf}_{\beta, \vmu}(t) = & (-1)^{\ell(\vmu)}\prod_{i=1}^s z_{\mu^i} \sum_{k \mid \beta, \vmu} \frac{\mu(k)}{k} \sum_{\vd: \pi(\vd) = \frac{\beta}{k}} \sum_{\substack{k' \mid \vd \\ (\frac{\vmu}{k})^{(\frac{1}{k'})} \text{ exists}}} \\
        &  \sum_{\va \in \cA_1\left(\frac{\vd}{k'}, (\frac{\vmu}{k})^{(\frac{1}{k'})}\right)} \frac{(k'|\va|)!}{\displaystyle\prod_{(\vdelta, \vnu) \in \cD\left(\frac{\vd}{k'}, (\frac{\vmu}{k})^{(\frac{1}{k'})}\right)} (k'\va(\vdelta, \vnu))!} \frac{(-1)^{k'|\va|-1}}{k'|\va|}  \prod_{(\vdelta, \vnu) \in \cD\left(\frac{\vd}{k'}, (\frac{\vmu}{k})^{(\frac{1}{k'})}\right)}\\
        & \left(\sum_{\vlambda \in T_{\vdelta}} \prod_{\bar{e} \in E(\Gamma)} (-1)^{(n^e + 1)\vdelta(\bar{e})} q^{\frac{k\kappa_{\vlambda(e)}n^e}{2}} \prod_{v \in V^3(\Gamma)} \cW_{\vlambda^v}(q^k) \prod_{i = 1}^s \frac{\chi_{\vlambda(-e_i)}(\nu^i)}{z_{\nu^i} [k\nu^i]}\right)^{k'\va(\vdelta, \vnu)}.
    \end{aligned}
\end{equation}
Moreover, we have
$$
    k^{\ell(\vmu)} (k'!)^s \biggm|  \frac{\displaystyle\prod_{i=1}^s z_{\mu^i}}{\displaystyle \prod_{(\vdelta, \vnu) \in \cD\left(\frac{\vd}{k'}, (\frac{\vmu}{k})^{(\frac{1}{k'})}\right)} \prod_{i=1}^s z_{\nu^i}^{k'\va(\vdelta, \vnu)}}
$$
where the factor $(k'!)^s$ comes from permutations the $k'$ copies of the $\nu^i$'s in the automorphism group, which implies that
$$
    \frac{1}{kk'} \frac{\displaystyle\prod_{i=1}^s z_{\mu^i}}{\displaystyle \prod_{(\vdelta, \vnu) \in \cD\left(\frac{\vd}{k'}, (\frac{\vmu}{k})^{(\frac{1}{k'})}\right)} \prod_{i=1}^s z_{\nu^i}^{k'\va(\vdelta, \vnu)}} \in \bZ
$$
since $\ell(\vmu), s \ge 1$. By \cite[Lemma A.2]{Konishi06b}, we also have
$$
    \frac{(k'|\va|)!}{\displaystyle |\va| \prod_{(\vdelta, \vnu) \in \cD\left(\frac{\vd}{k'}, (\frac{\vmu}{k})^{(\frac{1}{k'})}\right)} (k'\va(\vdelta, \vnu))!} \in \bZ.
$$
We may therefore rewrite \eqref{eqn:GRewrite3} as
\begin{align*}
    G^{X, L, \vf}_{\beta, \vmu}(t) = & \sum_{k \mid \beta, \vmu} \sum_{\vd: \pi(\vd) = \frac{\beta}{k}} \sum_{\substack{k' \mid \vd \\ (\frac{\vmu}{k})^{(\frac{1}{k'})} \text{ exists}}} \sum_{\va \in \cA_1\left(\frac{\vd}{k'}, (\frac{\vmu}{k})^{(\frac{1}{k'})}\right)}  \prod_{(\vdelta, \vnu) \in \cD\left(\frac{\vd}{k'}, (\frac{\vmu}{k})^{(\frac{1}{k'})}\right)} c_{k,\vd,k',\va,\vnu}\\
    & \left(\sum_{\vlambda \in T_{\vdelta}} \prod_{\bar{e} \in E(\Gamma)} (-1)^{(n^e + 1)\vdelta(\bar{e})} q^{\frac{k\kappa_{\vlambda(e)}n^e}{2}} \prod_{v \in V^3(\Gamma)} \cW_{\vlambda^v}(q^k) \prod_{i = 1}^s \frac{\chi_{\vlambda(-e_i)}(\nu^i)}{[k\nu^i]}\right)^{k'\va(\vdelta, \vnu)}.
\end{align*}
for some $c_{k,\vd,k',\va,\vnu} \in \bZ$.

To prove Lemma \ref{lem:GIntegral}, it suffices to show that for any $k,\vd,k',\va, (\vdelta,\vnu)$ as above,
$$
    \prod_{i=1}^s \frac{1}{[k\nu^i]^2} \sum_{\vlambda \in T_{\vdelta}} Y_{\vlambda}(q) \in \cL[t],
$$
where for $\vlambda \in T_{\vdelta}$ we set
$$
    Y_{\vlambda}(q) := \prod_{\bar{e} \in E(\Gamma)} (-1)^{(n^e + 1)\vdelta(\bar{e})} q^{\frac{k\kappa_{\vlambda(e)}n^e}{2}} \prod_{v \in V^3(\Gamma)} \cW_{\vlambda^v}(q^k) \prod_{i = 1}^s \chi_{\vlambda(-e_i)}(\nu^i)[k\nu^i].
$$
By \cite[Lemma P1]{BP01}, $[a]^2 \in \bZ_0[t]$ for any $a \in \bZ_{\ge 1}$, and thus it is further reduced to showing that
\begin{equation}\label{eqn:SumYq}
    \sum_{\vlambda \in T_{\vdelta}} Y_{\vlambda}(q) \in \cL[t].
\end{equation}
By \cite[Lemma 5.3(vii)]{Konishi06b},
$$
    \cW_{\vlambda^v}(q^k) \in q^{\frac{k|\vlambda^v|}{2}}\frac{\bZ[q,q^{-1}]}{\bZ_0[t]}.
$$
Note that for each edge $e_i$, since $|\vlambda(e_i)| = |\nu^i|$, we have
$$
    q^{\frac{k|\vlambda(e_i)|}{2}}[k\nu^i] \in \bZ[q].
$$
Therefore,
$$
    Y_{\vlambda}(q) \in \frac{\bZ[q,q^{-1}]}{\bZ_0[t]}.
$$
Now for $\vlambda \in T_{\vdelta}$, let $\vlambda^t \in T_{\vdelta}$ denote the element such that $\vlambda^t(e) = (\vlambda(e))^t$. By \cite[Lemma 5.3(viii)]{Konishi06b},
$$
    \cW_{(\vlambda^t)^v}(q^k) = (-1)^{|\vlambda^v|}\cW_{\vlambda^v}(q^{-k}).
$$
For each edge $e_i$, we have
$$
    (-1)^{|\vlambda^t(e_i)|}\chi_{\vlambda^t(-e_i)}(\nu^i)[k\nu^i] = \chi_{\vlambda(-e_i)}(\nu^i)[k\nu^i] \bigg|_{q \to q^{-1}}.
$$
Therefore,
$$
    Y_{\vlambda^t}(q) = Y_{\vlambda}(q^{-1}).
$$
For $\vlambda \in T_{\vdelta}$ such that $\vlambda^t = \vlambda$, $Y_{\vlambda}(q)$ is an element in $\frac{\bZ[q,q^{-1}]}{\bZ_0[t]}$ that is symmetric in $q$ and $q^{-1}$, and is thus in $\cL[t]$ by \cite[Lemma 6.2]{Konishi06a}. If otherwise $\vlambda^t \neq \vlambda$, similarly we have $Y_{\vlambda}(q) + Y_{\vlambda^t}(q) \in \cL[t]$. Summarizing the two cases gives \eqref{eqn:SumYq} and completes the proof. \qed



\subsection{An auxiliary FTCY graph}\label{sect:AuxGraph}
In this section, we prepare for the proof of Lemma \ref{lem:GFinite} by collecting additional notations and results. The main idea of the proof is to relate $G^{X, L, \vf}_{\beta, \vmu}(t)$ to generating functions defined by a trivalent FTCY graph constructed from $\Gamma$ and apply the analysis of this auxiliary graph given by \cite[Section 6]{Konishi06b}. Let $\Gamma'$ be the FTCY graph obtained from $\Gamma$ by adding two directed edges to each univalent vertex $v_i$ to make it a trivalent vertex. The position vectors of the new edges are chosen such that the degree $n^{e_i}$ is preserved for all $i = 1, \dots, s$. See Figure \ref{fig:AuxGraph} for an illustration. We have
$$
    V(\Gamma') = V^3(\Gamma') = V(\Gamma), \qquad E_c(\Gamma') = E_c^3(\Gamma') = E_c(\Gamma).
$$


% Figure environment removed



Let $\hY'$ be the FTCY 3-fold defined by $\Gamma'$. Then any effective class $\vd$ of $\hY$ can also be viewed as an effective class of $\Gamma'$ or $\hY'$, and we set
$$
    \Eff(\Gamma') := \{\vd: E_c(\Gamma') \to \bZ_{\ge 0} \mid \vd \neq 0\}.
$$
Following Theorem \ref{thm:TopVertex}, we define
$$ 
    Z^{\hY'}_{\vd}(q) = \sum_{\vlambda \in T_{\vd}} \prod_{\bar{e} \in E(\Gamma')} (-1)^{(n^e + 1)\vd(\bar{e})} q^{\frac{\kappa_{\vlambda(e)}n^e}{2}} \prod_{v \in V^3(\Gamma')} \cW_{\vlambda^v}(q).
$$
for each $\vd \in \Eff(\Gamma')$ and
$$
    Z'(q, Q) := 1 + \sum_{\vd \in \Eff(\Gamma')}Z^{\hY'}_{\vd}(q) Q^{\vd}.
$$
Moreover, we set
$$
    F^{\hY'}(q, Q): = \ln \left( Z^{\hY'}(q, Q)\right) = \sum_{\vd \in \Eff(\Gamma')}F^{\hY'}_{\vd}(q) Q^{\vd}.
$$
Then $Z^{\hY'}$ and $F^{\hY'}$ can be interpreted as generating functions of formal \emph{closed} Gromov-Witten invariants of $\hY'$.

The analysis of \cite[Section 6]{Konishi06b} expresses $Z^{\hY'}$ and $F^{\hY'}$ as sums of contributions indexed by certain sets of labeled graphs. We now briefly summarize the relevant results. Let $F(\Gamma')$ denote the set of flags in $\Gamma'$ and $F^3(\Gamma')$ denote the subset of flags $f = (v,e)$ such that $\bar{e} \in E_c(\Gamma')$. For each vertex $v \in V(\Gamma')$, we fix a counterclockwise ordering of the three flags at $v$ by $1, 2, 3$. We $f_1(v), f_2(v), f_3(v) \in F(\Gamma')$ to denote the flag at a vertex $v \in V(\Gamma')$ ordered by $1, 2, 3$ respectively, and $\iota(f) \in \{1, 2, 3\}$ to denote the order of a flag $f \in F(\Gamma')$. Moreover, we require that the ordering at each $v_i$ is chosen such that $\iota(v_i, -e_i) = 2$. 

Let $\vd \in \Eff(\Gamma')$. Let
$$
    D_{\vd}
$$
denote the set of \emph{$\Gamma'$-sets of degree $\vd$}, which are pairs
$$
    \left(\vnu_V = (\nu^v)_{v \in V(\Gamma')}, \vnu_F = (\nu^f)_{f \in F^3(\Gamma')} \right)
$$
of tuples of partitions satisfying the following conditions:
\begin{itemize}
    \item For $f = (v,e) \in F^3(\Gamma')$, $|\nu^v| + |\nu^f| = \vd(\bar{e})$ if $\iota(f) \in \{1,3\}$.
    \item For $f = (v,e) \in F^3(\Gamma')$, $|\nu^f| = \vd(\bar{e})$ if $\iota(f) = 2$.
    \item For $v \in V(\Gamma')$, $\nu^v = \emptyset$ if $f_1(v) \not \in F^3(\Gamma')$ or $f_3(v) \not \in F^3(\Gamma')$.
\end{itemize}
In particular, by our condition at the vertices $v_i$'s, $\nu^{v_i} = \emptyset$ and $|\nu^{(v_i, -e_i)}| = \vd(\bar{e}_i)$ for each $i = 1, \dots, s$. For any $\vmu = (\mu^1, \dots, \mu^s) \in \cP^s$ such that $|\mu^i| = \vd(\bar{e}_i)$ for each $i$, we set
$$
    D_{\vd, \vmu} := \{\left(\vnu_V , \vnu_F  \right) \in D_{\vd} \mid \nu^{(v_i, -e_i)} = \mu^i \text{ for } i = 1, \dots, s\}.
$$

Given any $\vlambda \in T_{\vd}$, at each vertex $v \in V(\Gamma')$, the three point function $\cW_{\vlambda^v}$ can be written as a sum of contributions indexed by the choice of partitions $\nu^v, \nu^{f_1(v)}, \nu^{f_2(v)}, \nu^{f_3(v)}$ (\cite[Lemma 6.2]{Konishi06b}). As a result, \cite{Konishi06b} constructed a set of (possibly disconnected) labeled graphs $\Comb_{\Gamma'}^\bullet(\vnu_V, \vnu_F)$ for each $(\vnu_V, \vnu_F) \in D_{\vd}$ and assigned an amplitude $\cH(W)(q)$ for each labeled graph $W \in \Comb_{\Gamma'}^\bullet(\vnu_V, \vnu_F)$ such that
\begin{equation}\label{eqn:KonishiZ}
    Z^{\hY'}_{\vd}(q) = \sum_{(\vnu_V, \vnu_F) \in D_{\vd}} \frac{1}{z_{\vnu_V}z_{\vnu_F}} \sum_{W \in \Comb_{\Gamma'}^\bullet(\vnu_V, \vnu_F)} \cH(W)(q)
\end{equation}
(\cite[Proposition 6.11]{Konishi06b}). Moreover, if $\Comb_{\Gamma'}^\circ(\vnu_V, \vnu_F) \subseteq \Comb_{\Gamma'}^\bullet(\vnu_V, \vnu_F)$ denotes the subset of connected graphs, then
\begin{equation}\label{eqn:KonishiF}
    F^{\hY'}_{\vd}(q) = \sum_{(\vnu_V, \vnu_F) \in D_{\vd}} \frac{1}{z_{\vnu_V}z_{\vnu_F}} \sum_{W \in \Comb_{\Gamma'}^\circ(\vnu_V, \vnu_F)} \cH(W)(q)
\end{equation}
(\cite[Proposition 6.13]{Konishi06b}).


\subsection{Proof of Lemma \ref{lem:GFinite}}\label{sect:ProofFinite}
We start by considering the following expression of $G^{X, L, \vf}_{\beta, \vmu}(t)$ which follows from the first line of \eqref{eqn:FRewrite} in the same way as how \eqref{eqn:GRewrite3} is derived:
\begin{equation}\label{eqn:GRewrite4}
    \begin{aligned}
        G^{X, L, \vf}_{\beta, \vmu}(t) = & (-1)^{\ell(\vmu)}z_{\vmu} \sum_{k \mid \beta, \vmu} \frac{\mu(k)}{k} \sum_{\vd: \pi(\vd) = \frac{\beta}{k}} \sum_{\va \in \cA\left(\vd, \frac{\vmu}{k}\right)} \frac{|\va|!}{\displaystyle\prod_{(\vdelta, \vnu) \in \cD\left(\vd, \frac{\vmu}{k}\right)} \va(\vdelta, \vnu)!} \frac{(-1)^{|\va|-1}}{|\va|}  \\
        & \prod_{(\vdelta, \vnu) \in \cD\left(\vd, \frac{\vmu}{k}\right)} \left(\sum_{\vlambda \in T_{\vdelta}} \prod_{\bar{e} \in E(\Gamma)} (-1)^{(n^e + 1)\vdelta(\bar{e})} q^{\frac{k\kappa_{\vlambda(e)}n^e}{2}} \prod_{v \in V^3(\Gamma)} \cW_{\vlambda^v}(q^k) \prod_{i = 1}^s \frac{\chi_{\vlambda(-e_i)}(\nu^i)}{z_{\nu^i} [k\nu^i]}\right)^{\va(\vdelta, \vnu)}.
    \end{aligned}
\end{equation}
Motivated by this expression, we define
$$
    Z'_{\vd,\vmu}(q) := \sum_{\vlambda \in T_{\vd}} \prod_{\bar{e} \in E(\Gamma)} (-1)^{(n^e + 1)\vd(\bar{e})} q^{\frac{\kappa_{\vlambda(e)}n^e}{2}} \prod_{v \in V^3(\Gamma)} \cW_{\vlambda^v}(q) \prod_{i = 1}^s \frac{\chi_{\vlambda(-e_i)}(\mu^i)}{z_{\mu^i} [\mu^i]}
$$
for each $(\vd, \vmu) \in \Eff(\Gamma)$ and
$$
    Z'(q, Q, P) := 1 + \sum_{(\vd, \vmu) \in \Eff(\Gamma)}Z'_{\vd, \vmu}(q) Q^{\vd}P_{\vmu}.
$$
Moreover, we set
$$
    F'(q, Q, P): = \ln \left( Z'(q, Q, P)\right) = \sum_{(\vd, \vmu) \in \Eff(\Gamma)}F'_{\vd, \vmu}(q) Q^{\vd}P_{\vmu}.
$$
Then \eqref{eqn:GRewrite4} can be rewritten as
\begin{equation}\label{eqn:GRewrite5}
    G^{X, L, \vf}_{\beta, \vmu}(t) = (-1)^{\ell(\vmu)} z_{\vmu} \sum_{k \mid \beta, \vmu} \frac{\mu(k)}{k} \sum_{\vd: \pi(\vd) = \frac{\beta}{k}} F'_{\vd, \frac{\vmu}{k}}(q^k).
\end{equation}

Now we relate $Z'$ and $F'$ to the generating functions $Z^{\hY'}$ and $F^{\hY'}$ defined by the FTCY graph $\Gamma'$ introduced in Section \ref{sect:AuxGraph}. Let $\vd \in \Eff(\Gamma')$ and $\vlambda \in T_{\vd}$. At each $v_i$ viewed as a trivalent vertex in $\Gamma'$, we have $\vlambda^{v_i} = (\vlambda(-e_i), \emptyset, \emptyset)$. By the Frobenius character formula,
$$
    \cW_{\vlambda^{v_i}}(q) = \cW_{(\vlambda(-e_i), \emptyset, \emptyset)} = s_{\vlambda(-e_i)}(q^\rho) = \sum_{\mu^i \vdash \vd(\bar{e}_i)}\frac{\chi_{\vlambda(-e_i)}(\mu^i)}{z_\mu[\mu^i]},
$$
and this summation coincides with the expression of $\cW_{\vlambda^{v_i}}$ as a sum of contributions indexed by $\nu^{f_2(v_i)} = \mu^i, \nu^{v_i} = \nu^{f_1(v_i)} = \nu^{f_3(v_i)} = \emptyset$ described in Section \ref{sect:AuxGraph} (\cite[Lemma 6.2]{Konishi06b}). It follows that 
\begin{align*}
    Z^{\hY'}_{\vd}(q) & = \sum_{\mu^i \vdash \vd(\bar{e}_i)} \sum_{\vlambda \in T_{\vd}} \prod_{\bar{e} \in E(\Gamma)} (-1)^{(n^e + 1)\vd(\bar{e})} q^{\frac{\kappa_{\vlambda(e)}n^e}{2}} \prod_{v \in V^3(\Gamma)} \cW_{\vlambda^v}(q) \prod_{i = 1}^s \frac{\chi_{\vlambda(-e_i)}(\mu^i)}{z_{\mu^i} [\mu^i]}\\
    & = \sum_{\mu^i \vdash \vd(\bar{e}_i)} Z'_{\vd, \vmu}(q),
\end{align*}
and \eqref{eqn:KonishiZ} further implies that
$$
    Z'_{\vd, \vmu}(q) = \sum_{(\vnu_V, \vnu_F) \in D_{\vd, \vmu}} \frac{1}{z_{\vnu_V}z_{\vnu_F}} \sum_{W \in \Comb_{\Gamma'}^\bullet(\vnu_V, \vnu_F)} \cH(W)(q).
$$
The proof of \eqref{eqn:KonishiF} from \eqref{eqn:KonishiZ} in \cite[Proposition 6.13]{Konishi06b} in fact shows that\footnote{In more detail, and in the context of the proof of \cite[Proposition 6.13]{Konishi06b}, it suffices to introduce the formal variables $P$ to the definition of the map $\Psi$:
$$
    \frac{1}{|\Aut(G)|}\Psi(\bar{G}) = \sum_{\substack{W \in \Comb_{\Gamma'}^\bullet(\vnu_V, \vnu_F) \\ \bar{W} = G}} \frac{1}{z_{\vnu_V}z_{\vnu_F}}\cH(W)Q^{\vd}P_{(\nu^{(v_1, -e_1)}, \dots, \nu^{(v_s, -e_s)})}.
$$
The resulting map $\Psi$ is still grade-preserving and multiplicative. Our desired result follows from identifying the coefficients of $Q^{\vd}P_{\vmu}$ on the two sides of the exponential formula.}
$$
    F'_{\vd, \vmu}(q) = \sum_{(\vnu_V, \vnu_F) \in D_{\vd, \vmu}} \frac{1}{z_{\vnu_V}z_{\vnu_F}} \sum_{W \in \Comb_{\Gamma'}^\circ(\vnu_V, \vnu_F)} \cH(W)(q).
$$

We now return to \eqref{eqn:GRewrite5} and continue rewriting $G^{X, L, \vf}_{\beta, \vmu}(t)$ as
\begin{align*}
    G^{X, L, \vf}_{\beta, \vmu}(t) & = (-1)^{\ell(\vmu)} z_{\vmu} \sum_{k \mid \beta, \vmu} \sum_{k' \mid k} \sum_{\substack{\vd: \pi(\vd) = \frac{\beta}{k} \\ \gcd(\vd, \frac{\mu}{k})=1}} \frac{k'\mu(\frac{k}{k'})}{k} F'_{k'\vd, \frac{k'\vmu}{k}}(q^{\frac{k}{k'}})\\
    & = (-1)^{\ell(\vmu)} z_{\vmu} \sum_{k \mid \beta, \vmu} \sum_{\substack{\vd: \pi(\vd) = \frac{\beta}{k} \\ \gcd(\vd, \frac{\vmu}{k})=1}} \sum_{(\vnu_V, \vnu_F) \in D_{\vd, \frac{\vmu}{k}}} \sum_{k' \mid k}  \frac{k'\mu(\frac{k}{k'})}{kz_{k'\vnu_V}z_{k'\vnu_F}} \sum_{W \in \Comb_{\Gamma'}^\circ(\vnu_V, \vnu_F)} \cH(W_{(k')})(q^{\frac{k}{k'}}).
\end{align*}
Here, as in \cite[Section 6.7]{Konishi06b}, $W_{(k')} \in \Comb_{\Gamma'}^\circ(k'\vnu_V, k'\vnu_F)$ denotes the labeled graph obtained from $W$ by multiplying all vertex labels by $k'$. Moreover, $\gcd(\vd, \frac{\mu}{k})$ denotes the greatest common divisor of the set of all values $\vd(\bar{e})$ for $e \in E_c(\Gamma)$ and all parts of partitions in $\frac{\vmu}{k}$. The condition $\gcd(\vd, \frac{\mu}{k}) = 1$ implies that for any $(\vnu_V, \vnu_F) \in D_{\vd, \frac{\vmu}{k}}$, we have $\gcd(\vnu_V, \vnu_F) = 1$, which is the greatest common divisor of the set of all parts of partitions in $\vnu_V, \vnu_F$. Thus we may apply \cite[Proposition 6.14]{Konishi06b} to obtain that for any $k, \vd, (\vnu_V, \vnu_F)$,
$$
    t\sum_{k' \mid k}  \frac{k'\mu(\frac{k}{k'})}{kz_{k'\vnu_V}z_{k'\vnu_F}} \sum_{W \in \Comb_{\Gamma'}^\circ(\vnu_V, \vnu_F)} \cH(W_{(k')})(q^{\frac{k}{k'}}) \in \bQ[t].
$$
It then follows that $tG^{X, L, \vf}_{\beta, \vmu}(t) \in \bQ[t]$, proving the lemma. \qed

