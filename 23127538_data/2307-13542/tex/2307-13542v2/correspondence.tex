\section{Open/closed BPS correspondence}\label{sect:OpenClosed}
In this section, we specialize to the case of disk invariants of a single outer brane and extend the open/closed correspondence \cite{LY21,LY22} of Gromov-Witten invariants to a correspondence of BPS invariants (Theorem \ref{thm:BPSCorrespondence}). Using the integrality of the open BPS invariants (Theorem \ref{thm:OpenBPS}), we obtain the integrality of the closed BPS invariants (Corollary \ref{cor:ClosedBPS}).

\subsection{Disk invariants and Gromov-Witten correspondence}\label{sect:OpenClosedGW}
We first consider the specialization of results in Section \ref{sect:OpenBPS} to the case of disk invariants of a single outer brane. Let $s=1$ and we write $(L,f) = (L^1, f_1)$, $d = d_1 \in \bZ_{\ge 1}$, $B = B_1$, and $C = C_1$. We further consider the case $\mu = \mu^1 = (d)$. The open Gromov-Witten invariant
$$
    N^{X, L, f}_{0, \beta, (d)} \in \bQ
$$
is a virtual count of open stable maps from domains of arithmetic genus zero and with a single boundary component, and we refer to it as the degree-$(\beta, (d))$ \emph{disk invariant} of $(X, L, f)$.

In the above setting, the genus-zero limit $g_s \to 0$ (or equivalently $q \to 1$) of the LMOV resummation formula \eqref{eqn:LMOVResumGs} gives
\begin{equation}\label{eqn:LMOVOpen}
    N^{X, L, f}_{0, \beta, (d)} = - \sum_{k \mid \beta, d} \frac{n^{X, L, f}_{0,\frac{\beta}{k}, \left(\frac{d}{k}\right)}}{k^2}.
\end{equation}
By Theorem \ref{thm:OpenBPS}, the genus-zero, degree-$(\beta, (d))$ open BPS invariant of $(X, L, f)$ defined by the above is an integer for any $\beta, d$:
$$
    n^{X, L, f}_{0, \beta, (d)} \in \bZ.
$$


%\subsection{Closed geometry }\label{sect:OpenClosedGW}
In \cite{LY21}, the author and Liu identified the disk invariants of the open geometry $(X,L,f)$ with the genus-zero Gromov-Witten invariants of a closed geometry $\tX$.

\begin{theorem}[\cite{LY21}]\label{thm:GWCorrespondence}
There is a smooth toric Calabi-Yau 4-fold $\tX$ such that:
\begin{itemize}
    \item There is an isomorphism $\iota: H_2(X,L;\bZ) \to H_2(\tX;\bZ)$.
    \item There is a cohomology class $\tgamma \in H^4(\tX;\bZ)$ such that for the effective class $\tbeta:= \iota(\beta = \beta'+d[B])$,
    $$
        N^{X, L, f}_{0, \beta, (d)} = N^{\tX}_{0, \tbeta}(\tgamma)
    $$
    where $N^{\tX}_{0, \tbeta}(\tgamma)$ is the genus-zero, degree-$\tbeta$ \emph{closed Gromov-Witten invariant} of $\tX$ with 1-pointed insertion $\tgamma$.
\end{itemize}
\end{theorem}

We note that $N^{\tX}_{0, \tbeta}(\tgamma)$ is defined by localization with respect to the Calabi-Yau 3-torus $\tT'$ of $\tX$. The construction of $\tX$ can be described as follows: In the case $s=1$, the FTCY graph $\Gamma$ defines a relative Calabi-Yau 3-fold $(Y,D)$ such that $Y = X \sqcup D$ and $(\hY, \hD)$ is the formal completion of $(Y,D)$ along the toric 1-skeleton $Y^1$. Then $\tX = \Tot(\cO_Y(-D))$. The normal bundle of the new $\tT'$-invariant projective line $C$ is
$$
    N_{C/\tX} \cong \cO_{\bP^1}(f) \oplus \cO_{\bP^1}(-f-1) \oplus \cO_{\bP^1}(-1).
$$
We refer to \cite[Section 2, 3]{LY21} for additional details of the construction of $\tX$ and the definition of the closed Gromov-Witten invariants.



\subsection{BPS correspondence and integrality}\label{sect:OpenClosedBPS}
For Calabi-Yau 4-folds, Klemm-Pandharipande \cite{KP08} expressed integral structures of genus-zero Gromov-Witten invariants by a generalization of the Aspinwall-Morrison multiple covering formula \cite{AM93}. For invariants with 1-pointed insertions, the resummation of \cite{KP08} is
\begin{equation}\label{eqn:KPClosed}
    N^{\tX}_{0, \tbeta}(\tgamma) = \sum_{k \mid \tbeta} \frac{n^{\tX}_{0,\frac{\tbeta}{k}}(\tgamma)}{k^2}.
\end{equation}
Here for $k \in \bZ_{\ge 1}$, we say that $k \mid \tbeta$ if $\frac{\tbeta}{k} \in H_2(\tX;\bZ)$. Note that if $\tbeta = \iota(\beta = \beta' + d)$, then $k \mid \tbeta$ if and only if $k \mid \beta, d$. We refer to the coefficient
$$
    n^{\tX}_{0, \tbeta}(\tgamma) \in \bQ
$$
as the genus-zero, degree-$\tbeta$ \emph{closed BPS invariant} of $\tX$.

By a direct comparison of the resummations \eqref{eqn:LMOVOpen}, \eqref{eqn:KPClosed}, we see that Theorem \ref{thm:GWCorrespondence} immediately implies the following open/closed correspondence of BPS invariants:

\begin{theorem}\label{thm:BPSCorrespondence}
Under the setup of Theorem \ref{thm:GWCorrespondence}, for $\tbeta = \iota(\beta = \beta'+d[B])$, we have
$$
    n^{X, L, f}_{0, \beta, (d)} = -n^{\tX}_{0, \tbeta}(\tgamma).
$$
\end{theorem}

As a result of the integrality of open BPS invariants (Theorem \ref{thm:OpenBPS}), Theorem \ref{thm:BPSCorrespondence} implies the integrality of closed BPS invariants, verifying \cite[Conjecture 0]{KP08}:

\begin{corollary}\label{cor:ClosedBPS}
For $\tbeta = \iota(\beta = \beta'+d[B])$, we have
$$
    n^{\tX}_{0, \tbeta}(\tgamma) \in \bZ.
$$
\end{corollary}

\begin{remark}\rm{
We note that when $X$ is semi-projective, the open/closed correspondence (Theorem \ref{thm:GWCorrespondence}) can be further established for a semi-projective partial compactification of the 4-fold $\tX$ \cite{LY22}, where $\tgamma$ may need to be replaced by a $\tT'$-equivariant class. Theorem \ref{thm:BPSCorrespondence} and Corollary \ref{cor:ClosedBPS} hold in this extended context as well and give additional examples for \cite[Conjecture 0]{KP08}.
}\end{remark}