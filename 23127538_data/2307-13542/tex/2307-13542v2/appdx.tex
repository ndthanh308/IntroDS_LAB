\appendix

\section{Topological vertex formulas}\label{appdx:TopVertex}
In this section, we supply the details of the expression \eqref{eqn:TopVertex} by clarifying the signs in \cite{LLLZ09} from equation (7-9) to Proposition 7.4. We adopt the definitions and notations in \cite{LLLZ09} without repeating them here and we fix a choice of $\sqrt{-1}$ as before. Our formulas are based on the statements and equations up to (7-8). Note that (7-8) directly implies (7-10).

We first correct (7-9) as follows:
\begin{equation}\label{eqn:7-9}
\begin{aligned}
    & \sqrt{-1}^{\ell(\vmu)} G^\bullet_{\vmu}(\lambda; \fp(e_1), \fp(e_2), \fp(e_3))\\
    & = \sqrt{-1}^{\ell(\vmu)} \sum_{|\nu^i| = |\mu^i|} \tF^\bullet_{\vnu}(\lambda; 0) \prod_{i=1}^3 z_{\nu^i} \Phi^\bullet_{\nu^i, \mu^i} \left(\sqrt{-1}\frac{\fl_0(e_i)}{\fp(e_i)}\lambda \right)\\
    & = \sqrt{-1}^{\ell(\vmu)} \sum_{|\nu^i| = |\mu^i|} (-1)^{|\vnu|}\sqrt{-1}^{\ell(\vnu)}F^\bullet_{\vnu}(\lambda; 0) \prod_{i=1}^3 z_{\nu^i} \Phi^\bullet_{\nu^i, \mu^i} \left(\sqrt{-1}\frac{\fl_0(e_i)}{\fp(e_i)}\lambda \right) && \qquad (\text{by (6-6)})\\
    & = (-1)^{\sum_{i=1}^3 \vd(\bar{e}_i)}  \sum_{|\nu^i| = |\mu^i|} F^\bullet_{\vnu}(\lambda; 0) \prod_{i=1}^3 \sqrt{-1}^{\ell(\mu^i) + \ell(\nu^i)}z_{\nu^i} \Phi^\bullet_{\nu^i, \mu^i} \left(\sqrt{-1}\frac{\fl_0(e_i)}{\fp(e_i)}\lambda \right)\\
    & = (-1)^{\sum_{i=1}^3 \vd(\bar{e}_i)} \sum_{|\nu^i| = |\mu^i|} F^\bullet_{\vnu}(\lambda; 0) \prod_{i=1}^3 \sqrt{-1}^{(-\ell(\mu^i)-\ell(\nu^i))}z_{\nu^i} \Phi^\bullet_{\nu^i, \mu^i} \left(-\sqrt{-1}\frac{\fl_0(e_i)}{\fp(e_i)}\lambda \right).
\end{aligned}
\end{equation}

Now we deduce (7-12) from (7-10) using the correction \eqref{eqn:7-9} of (7-9) above:
\begin{equation}\label{eqn:7-12}
    \begin{aligned}
        F^{\bullet \Gamma}_{\vd, \vmu}(\lambda;u_1, u_2)
        = & \sum_{|\rho^{\bar{e}}| = d^{\bar{e}}} \prod_{\bar{e} \in E(\Gamma)} (-1)^{n^ed^{\bar{e}}}z_{\rho^{\bar{e}}} \prod_{v \in V_3(\Gamma)} \sqrt{-1}^{\ell(\vrho^v)}G^\bullet_{\vrho^v}(\lambda;\bw_v) \\
        & \cdot \prod_{v \in V_1(\Gamma), v_1(e) = v} (-1)^{d^{\bar{e}}}\sqrt{-1}^{\ell(\rho^{\bar{e}}) + \ell(\mu^v)}\Phi^\bullet_{\rho^{\bar{e}}, \mu^v} \left(\sqrt{-1}\frac{\ff(e)}{\fp(e)}\lambda\right)\\
        = & \sum_{|\rho^{\bar{e}}| = d^{\bar{e}}} \prod_{\bar{e} \in E(\Gamma)} (-1)^{n^ed^{\bar{e}}}z_{\rho^{\bar{e}}} \prod_{v \in V_3(\Gamma)} (-1)^{|\vrho^v|}  \sqrt{-1}^{(-\ell(\vrho^v) -\ell(\vnu^v))} \\
        &  \cdot \sum_{|\nu^{v,i}| = |\rho^{v,i}|}F^\bullet_{\vnu^v}(\lambda; 0) \prod_{i=1}^3 z_{\nu^{v,i}} \Phi^\bullet_{\nu^{v,i}, \rho^{v,i}} \left(-\sqrt{-1}\frac{\fl_0(e_i)}{\fp(e_i)}\lambda \right)\\
        & \cdot \prod_{v \in V_1(\Gamma), v_1(e) = v} (-1)^{d^{\bar{e}}}\sqrt{-1}^{\ell(\rho^{\bar{e}}) + \ell(\mu^v)}\Phi^\bullet_{\rho^{\bar{e}}, \mu^v} \left(\sqrt{-1}\frac{\ff(e)}{\fp(e)}\lambda\right) \qquad \qquad \qquad (\text{by \eqref{eqn:7-9}})\\
        = & \sum_{\vnu \in P_{\vd, \vmu}} \prod_{v \in V_3(\Gamma)} F^\bullet_{\vnu^v}(\lambda; 0) z_{\vnu^v} \prod_{\bar{e} \in E(\Gamma)} \cE_{\bar{e}}
    \end{aligned}
\end{equation}
where the term $ \prod_{\bar{e} \in E(\Gamma)} \cE_{\bar{e}}$ above is a product of contributions from edges $\bar{e} \in E(\Gamma)$. Here, for an edge $\bar{e}$ between two trivalent vertices $v = v_0(e), v' = v_1(e) \in V_3(\Gamma)$, the contribution is
\begin{align*}
    \cE_{\bar{e}} & = (-1)^{n^ed^{\bar{e}}}\sum_{\rho \vdash d^{\bar{e}}} \sqrt{-1}^{(-\ell(\nu^e) - \ell(\nu^{-e}) - 2\ell(\rho))}\Phi^\bullet_{\nu^e, \rho}\left(-\sqrt{-1}\frac{\fl_0(e)}{\fp(e)}\lambda\right) z_\rho \Phi^\bullet_{\nu^{-e}, \rho}\left(-\sqrt{-1}\frac{\fl_0(-e)}{\fp(-e)}\lambda\right)\\
    & = (-1)^{n^ed^{\bar{e}}}\sum_{\rho \vdash d^{\bar{e}}} \sqrt{-1}^{(-\ell(\nu^e) - \ell(\nu^{-e}) - 2\ell(\rho))}(-1)^{\ell(\nu^e) + \ell(\rho)}\Phi^\bullet_{\nu^e, \rho}\left(\sqrt{-1}\frac{\fl_0(e)}{\fp(e)}\lambda\right) z_\rho \Phi^\bullet_{\nu^{-e}, \rho}\left(\sqrt{-1}\frac{\fl_0(-e)}{\fp(e)}\lambda\right)\\
    & = (-1)^{n^ed^{\bar{e}}}\sum_{\rho \vdash d^{\bar{e}}} \sqrt{-1}^{\ell(\nu^e) - \ell(\nu^{-e})}\Phi^\bullet_{\nu^e, \rho}\left(\sqrt{-1}\frac{\fl_0(e)}{\fp(e)}\lambda\right) z_\rho \Phi^\bullet_{\nu^{-e}, \rho}\left(\sqrt{-1}\frac{\fl_0(-e)}{\fp(e)}\lambda\right)\\
    & = (-1)^{n^ed^{\bar{e}}} \sqrt{-1}^{\ell(\nu^e) - \ell(\nu^{-e})}\Phi^\bullet_{\nu^e, \nu^{-e}}\left(\sqrt{-1}n^e\lambda\right)
\end{align*}
where the last equality follows from (2-9) applied with
$$
    \fl_0(e) + \fl_0(-e) = \fl_0(e) - \fl_1(e) + \fp(e) = n^e\fp(e).
$$
For an edge $\bar{e}$ between a trivalent vertex $v' = v_0(e) \in V_3(\Gamma)$ and a univalent vertex $v = v_1(e) \in V_1(\Gamma)$, the contribution is
\begin{align*}
    \cE_{\bar{e}} & = (-1)^{n^ed^{\bar{e}}}\sum_{\rho \vdash d^{\bar{e}}} \sqrt{-1}^{(-\ell(\nu^e) + \ell(\mu^v))}\Phi^\bullet_{\nu^e, \rho}\left(-\sqrt{-1}\frac{\fl_0(e)}{\fp(e)}\lambda\right) z_\rho \Phi^\bullet_{\mu^v, \rho}\left(\sqrt{-1}\frac{\ff(e)}{\fp(e)}\lambda\right)\\
    & = (-1)^{n^ed^{\bar{e}}}\sum_{\rho \vdash d^{\bar{e}}} \sqrt{-1}^{(-\ell(\nu^e) + \ell(\mu^v))}(-1)^{\ell(\nu^e) + \ell(\mu^v)}\Phi^\bullet_{\nu^e, \rho}\left(\sqrt{-1}\frac{\fl_0(e)}{\fp(e)}\lambda\right) z_\rho \Phi^\bullet_{\mu^v, \rho}\left(-\sqrt{-1}\frac{\ff(e)}{\fp(e)}\lambda\right)\\
    & = (-1)^{n^ed^{\bar{e}}} \sqrt{-1}^{\ell(\nu^e) - \ell(\mu^v)}\Phi^\bullet_{\nu^e, \mu^v}\left(\sqrt{-1}n^e\lambda\right)\\
    & = (-1)^{n^ed^{\bar{e}}} \sqrt{-1}^{\ell(\nu^e) - \ell(\nu^{-e})}\Phi^\bullet_{\nu^e, \nu^{-e}}\left(\sqrt{-1}n^e\lambda\right)
\end{align*}
where the second-to-last equality follows from (2-9) applied with
$$
    \fl_0(e) - \ff(e) = n^e\fp(e)
$$
and the last equality follows from $\nu^{-e} = \mu^v$. Therefore, \eqref{eqn:7-12} is identical to (7-12).

Lastly, we use (2-8), (7-12), and (7-13) to derive the following correction of Proposition 7.4:
\begin{equation}\label{eqn:Prop7-4}
\begin{aligned}
    F^{\bullet\Gamma}_{\vd, \vmu} = &  \sum_{\vnu \in P_{\vd, \vmu}} \prod_{v \in V_3(\Gamma)} F^\bullet_{\vnu^v}(\lambda; 0) z_{\vnu^v} \prod_{\bar{e} \in E(\Gamma)} (-1)^{n^ed^{\bar{e}}} \sqrt{-1}^{\ell(\nu^e) - \ell(\nu^{-e})}\Phi^\bullet_{\nu^e, \nu^{-e}}\left(\sqrt{-1}n^e\lambda\right) \qquad (\text{by (7-12)})\\
    = & \sum_{\vnu \in P_{\vd, \vmu}} \prod_{v \in V_3(\Gamma)} \frac{(-1)^{|\vnu^v|}}{\sqrt{-1}^{\ell(\vnu^v)}}\sum_{|\xi^{v,i}| = |\nu^{v,i}|} \tC_{\vxi^v}(\lambda)\prod_{i=1}^3 \chi_{\xi^{v,i}}(\nu^{v,i})\\
     & \cdot \prod_{\bar{e} \in E(\Gamma)} (-1)^{n^ed^{\bar{e}}} \sqrt{-1}^{\ell(\nu^e) - \ell(\nu^{-e})} \sum_{\rho^{e} \vdash d^{\bar{e}}} e^{\sqrt{-1}n^e\kappa_{\rho^e}\lambda/2} \frac{\chi_{\rho^e}(\nu^e)}{z_{\nu^e}}\frac{\chi_{\rho^e}(\nu^{-e})}{z_{\nu^{-e}}} \quad (\text{by (2-8) and (7-13)})\\
     = & \sum_{\vnu \in P_{\vd, \vmu}} \prod_{v \in V_3(\Gamma)} \frac{(-1)^{|\vnu^v|}}{\sqrt{-1}^{\ell(\vnu^v)}}\sum_{|\xi^{v,i}| = |\nu^{v,i}|} \tC_{\vxi^v}(\lambda)\prod_{i=1}^3 \chi_{\xi^{v,i}}(\nu^{v,i})\\
     & \cdot \prod_{\bar{e} \in E(\Gamma)} (-1)^{(n^e+1)d^{\bar{e}}} \sqrt{-1}^{\ell(\nu^e) + \ell(\nu^{-e})} \sum_{\rho^{e} \vdash d^{\bar{e}}} e^{\sqrt{-1}n^e\kappa_{\rho^e}\lambda/2} \frac{\chi_{\rho^e}(\nu^e)}{z_{\nu^e}}\frac{\chi_{(\rho^e)^t}(\nu^{-e})}{z_{\nu^{-e}}} \\
     = & \sum_{\vxi, \vrho} \prod_{\bar{e} \in E(\Gamma)} (-1)^{(n^e+1)d^{\bar{e}}} e^{\sqrt{-1}n^e\kappa_{\rho^e}\lambda/2} \prod_{v \in V_3(\Gamma)} (-1)^{|\vxi^v|} \tC_{\vxi^v}(\lambda) \\
     & \cdot \prod_{v = v_0(e) \in V_3(\Gamma)} \sum_{\nu^e \vdash d^{\bar{e}}} \frac{\chi_{\xi^e}(\nu^e)\chi_{\rho^e}(\nu^e)}{z_{\nu^e}} \prod_{v = v_0(e) \in V_1(\Gamma)} \frac{\chi_{\rho^e}(\mu^v)}{z_{\mu^v}}\sqrt{-1}^{\ell(\mu^v)} \qquad (\text{where $\rho^{-e} := (\rho^e)^t$})\\
     = & \sum_{\vxi, \vrho} \prod_{\bar{e} \in E(\Gamma)} (-1)^{(n^e+1)d^{\bar{e}}} e^{\sqrt{-1}n^e\kappa_{\rho^e}\lambda/2} \prod_{v \in V_3(\Gamma)} (-1)^{|\vxi^v|} \tC_{\vxi^v}(\lambda) \\
     & \cdot \prod_{v = v_0(e) \in V_3(\Gamma)} \delta_{\xi^e, \rho^e} \prod_{v = v_0(e) \in V_1(\Gamma)} \frac{\chi_{\rho^e}(\mu^v)}{z_{\mu^v}}\sqrt{-1}^{\ell(\mu^v)} \qquad \qquad  (\text{by orthogonality of characters})\\
     = &  \sum_{\vrho \in T_{\vd, \vmu}} \prod_{\bar{e} \in E(\Gamma)} (-1)^{(n^e+1)d^{\bar{e}}} e^{\sqrt{-1}n^e\kappa_{\rho^e}\lambda/2} \prod_{v \in V_3(\Gamma)} \tC_{\vrho^v}(\lambda) \prod_{v = v_0(e) \in V_1(\Gamma)} \frac{\chi_{\rho^e}(\mu^v)}{z_{\mu^v}}\sqrt{-1}^{\ell(\mu^v)}(-1)^{|\mu^v|}.
\end{aligned}
\end{equation}
Equation \eqref{eqn:Prop7-4} differs from Proposition 7.4 in two ways:
\begin{itemize}
    \item For each $v \in V_1(\Gamma)$, there is an additional sign of $(-1)^{|\mu^v|+\ell(\mu^v)}$.
    
    \item For each $\bar{e} \in E(\Gamma)$, the sign in the power $e^{\sqrt{-1}n^e\kappa_{\rho^e}\lambda/2}$ is positive rather than negative. 
    \begin{comment}
    This seems to be consistent with the gluing formula (3.15) of [AKMV]: In [AKMV], the three legs of a vertex are ordered clockwise instead of counterclockwise. Using the notation $C^T$ for the $C$ of [AKMV], we write (3.15) as
    $$
        \sum_{R_i}C^T_{R_jR_kR_i} (-1)^{(n_i+1)|R_i|}q^{-n_i\kappa_{R_i}/2}C^T_{R_i^tR_j'R_k'}
    $$
    (where the ``Novikov variable'' $e^{-|R_i|t_i}$ is omitted). Here, $n_i = n^e$ where $e$ is the oriented edge from vertex $v$ to $v'$. Now, ordering the legs in counterclockwise order, changing all partitions to their conjugates, and applying (3.12), we get
    $$
        \sum_{R_i}C_{R_j^tR_k^tR_i^t} (-1)^{(n_i+1)|R_i|}q^{n^e\kappa_{R_i^t}/2}C_{R_i(R_j')^t(R_k')^t}q^{(\kappa_{R_j}+\kappa_{R_k}+\kappa_{R_j'}+\kappa_{R_k'})/2}
    $$
    which is consistent with \eqref{eqn:Prop7-4}. (For e.g. an entirely trivalent graph, the extra powers of $q^{1/2}$ cancel out.)

    \end{comment}
\end{itemize}
Moreover, \eqref{eqn:TopVertex} directly follows from \eqref{eqn:Prop7-4}.



\begin{comment}

We use (2-8) to rewrite the double-Hurwitz term in (7-12) as
$$
    \Phi^{\bullet}_{\nu^e, \nu^{-e}}(\sqrt{-1}n^e\lambda) = \sum_{\nu \vdash d^{\bar{e}}} e^{-\sqrt{-1}n^e\kappa_{\nu}/2} \frac{\chi_{\nu}(\nu^e)}{z_{\nu^e}}\frac{\chi_{\nu}(\nu^{-e})}{z_{\nu^{-e}}}.
$$
In order to cancel the vertex terms using the orthogonality of characters (the equation after (2-8)), we need to change the last term in the equation above into
$$
    \frac{\chi_{\nu^t}(\nu^{-e})}{z_{\nu^{-e}}}
$$
so that after summing over all possible $\nu^e$ and $\nu^{-e}$, the only contributing terms satisfy
$$
    \nu^e = \nu, \quad \nu^{-e} = \nu^t.
$$
Making this change incurs a sign of
$$
    (-1)^{\sgn(\nu^{-e})} = (-1)^{|\nu^{-e}| + \ell(\nu^{-e})}
$$
which is provided by adding a sign of $(-1)^{d^{\bar{e}}}$ in the edge contribution and making the change
$$
    \sqrt{-1}^{\ell(\nu^e) - \ell(\nu^{-e})} \Rightarrow \sqrt{-1}^{\ell(\nu^e) + \ell(\nu^{-e})}.
$$
If $e$ connects two trivalent vertices, then the term $\sqrt{-1}^{\ell(\nu^e) + \ell(\nu^{-e})}$ is cancelled by the pre-factors of the vertex contribution in (7-13), and the signs in the pre-factors also cancel. However, if $e$ connects a trivalent vertex to a univalent vertex, then we don't have a similar pre-factor for the univalent vertex. If we want to use the final term in the formula of Proposition 7.4, then it seems that we need to use the opposite orientation on $e$ since we want $\mu^v = \nu^{-e}$.

\end{comment}