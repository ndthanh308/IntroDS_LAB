\section{Open Gromov-Witten invariants and the topological vertex}\label{sect:OpenGW}
In this section, we briefly review the open geometry set-up, open Gromov-Witten invariants, and the topological vertex formalism, introducing notations along the way. We refer to \cite{LLLZ09,FL13} for full definitions and additional details. We work over $\bC$.


\subsection{Preliminaries}\label{sect:Partition}
We start with the preliminaries on partitions. 
\subsubsection{Notations on a single partition}
A \emph{partition} is a non-increasing sequence
$$
    \lambda = (\lambda_1, \lambda_2, \dots)
$$
of non-negative integers where only finitely many are non-zero. We set up the following notations on $\lambda$:

\begin{itemize}
    \item The \emph{length} of $\lambda$ is the number of non-zero terms, called \emph{parts} of $\lambda$:
    $$
        \ell(\lambda) := |\{j \mid \lambda_j \neq 0\}|.
    $$
    We will interchangeably write $\lambda$ as the finite sequence consisting of its parts:
        $$
        \lambda = (\lambda_1, \dots, \lambda_{\ell(\lambda)})
        $$
    
    \item The \emph{degree} of $\lambda$ is the sum of all parts:
        $$
            |\lambda| := \sum_{j=1}^{\ell(\lambda)} \lambda_j.
        $$
    If $d = |\lambda|$, we say that $\lambda$ is a partition of $d$ and write $\lambda \vdash d$.

    \item For $a \in \bZ_{\ge 1}$, the \emph{multiplicity} of $a$ in $\lambda$ is the number of occurrences of $a$ in $\lambda$:
        $$
            m_a(\lambda) := |\{j \mid \lambda_j = a\}|.
        $$
    
    \item The \emph{automorphism group} of $\lambda$, denoted $\Aut(\lambda)$, is the group of permutations of parts of $\lambda$ that preserve $\lambda$. We have
        $$
            \Aut(\lambda) \cong \prod_{a \in \bZ_{\ge 1}} S_{m_a(\lambda)}, \qquad |\Aut(\lambda)| = \prod_{a \in \bZ_{\ge 1}} m_a(\lambda)!
        $$
    where $S_m$ denotes the symmetric group on $m$ letters.

    \item We set
        $$
            z_\lambda := |\Aut(\lambda)| \prod_{j=1}^{\ell(\lambda)} \lambda_j.
        $$

    \item The \emph{conjugate partition} of $\lambda$ is the sequence $\lambda^t = (\lambda^t_1, \lambda^t_2, \dots)$ where
        $$
            \lambda^t_i := \sum_{a \in \bZ_{\ge i}} m_a(\lambda).
        $$

    \item We set
        $$
            \kappa_\lambda := \sum_{j=1}^{\ell(\lambda)} \lambda_j(\lambda_j - 2j + 1),
        $$
    which is even and satisfies
        $$
            \kappa_{\lambda^t} = - \kappa_\lambda.
        $$

    \item For $k \in \bZ_{\ge 1}$, we set
        $$
            k\lambda := (k\lambda_1, \dots, k\lambda_{\ell(\lambda)}).
        $$
    We have
        $$
            \ell(k\lambda) = \ell(\lambda), \quad |k\lambda| = k |\lambda|, \quad \Aut(k\lambda) \cong \Aut(\lambda), \quad z_{k\lambda} = k^{\ell(\lambda)}z_{\lambda}.
        $$
    On the other hand, if there exists a partition $\mu$ such that $\lambda = k\mu$, we write $k \mid \lambda$ and $\mu = \frac{\lambda}{k}$.
    
    \item For $k \in \bZ_{\ge 1}$, let $\lambda^{(k)}$ denote the partition where
        $$
            m_a(\lambda^{(k)}) = k m_a(\lambda)
        $$
    for all $a \in \bZ_{\ge 1}$. On the other hand, if there exists a partition $\mu$ such that $\lambda = \mu^{(k)}$, we write $\mu = \lambda^{(\frac{1}{k})}$.
\end{itemize}


\subsubsection{Notations on multiple partitions}
Let $\cP$ denote the set of all partitions. Given $\lambda^1, \dots, \lambda^t \in \cP$, we define
$$
    \bigsqcup_{i = 1}^t \lambda^i
$$
to be the partition whose set of parts is the disjoint union of all sets of parts of $\lambda^1, \dots, \lambda^t$. In particular, when $\lambda^1 = \cdots = \lambda^t = \lambda$, $\bigsqcup_{i = 1}^t \lambda^i = \lambda^{(t)}$. We have
$$
    \ell\left( \bigsqcup_{i = 1}^t \lambda^i\right) = \sum_{i=1}^t \ell(\lambda^i), \qquad \left|\bigsqcup_{i = 1}^t \lambda^i\right| = \sum_{i=1}^t |\lambda^i|.
$$
Moreover, $\prod_{i=1}^t \Aut(\lambda^i)$ is a subgroup of $\Aut \left(\bigsqcup_{i = 1}^t \lambda^i \right)$ and
$$
    \prod_{i=1}^t z_{\lambda^i} \biggm| z_{\bigsqcup_{i = 1}^t \lambda^i}.
$$
For partitions $\mu, \nu \in \cP$, we write
$$
    \nu \subseteq \mu
$$
if there exists $\nu' \in \cP$ such that $\mu = \nu \sqcup \nu'$, or equivalently $m_a(\nu) \le m_a(\mu)$ for all $a \in \bZ_{\ge 1}$.

Now given a sequence of partitions $\vmu = (\mu^1, \dots, \mu^s) \in \cP^s$, we denote
$$
    \ell(\vmu) := \sum_{i = 1}^s \ell(\mu^i), \qquad |\vmu| := \sum_{i=1}^s |\mu^i|, \qquad z_{\vmu}:= \prod_{i=1}^s z_{\mu^i}.
$$
For $k \in \bZ_{\ge 1}$, we define
$$
    k\vmu := (k\mu^1, \dots, k\mu^s), \qquad \vmu^{(k)} := ((\mu^1)^{(k)}, \dots, (\mu^s)^{(k)}),
$$
and $\frac{\vmu}{k}$ (resp. $\vmu^{(\frac{1}{k})}$) similarly if $\frac{\mu^i}{k}$ (resp. $(\mu^i)^{(\frac{1}{k})}$) exists for all $i$. Now for another sequence $\vnu = (\nu^1, \dots, \nu^s) \in \cP^s$, we define
$$
    \vmu \sqcup \vnu := (\mu^1 \sqcup \nu^1, \dots, \mu^s \sqcup \nu^s),
$$
and we write
$$
    \vnu \subseteq \vmu
$$
if $\nu^i \subseteq \mu^i$ for all $i$.

\subsubsection{Three point function}
Finally, we define a function associated to triples of partitions following \cite{AKMV03, Zhou03, LLLZ09, Konishi06b}. Let $q$ be a formal variable and write
$$
    q^{\rho} = \left(q^{-\frac{1}{2}}, q^{-\frac{3}{2}}, \cdots \right), \qquad
    q^{\lambda + \rho} =  \left(q^{\lambda_1-\frac{1}{2}}, q^{\lambda_2-\frac{3}{2}}, \cdots \right) \quad \text{for $\lambda \in \cP$.}
$$
For $\lambda, \eta \in \cP$, let $s_{\lambda}, s_{\lambda/\eta}$ denote the Schur function and skew Schur function respectively.

\begin{definition}\label{def:ThreePoint}
\rm{
For a triple $(\lambda^1, \lambda^2, \lambda^3) \in \cP^3$ of partitions, define
$$
    \cW_{\lambda^1, \lambda^2, \lambda^3}(q) := q^{\frac{\kappa_{\lambda^3}}{2}}s_{\lambda^2}(q^\rho) \sum_{\eta \in \cP} s_{\lambda^1/\eta}(q^{(\lambda^2)^t+\rho})s_{(\lambda^3)^t/\eta}(q^{\lambda^2+\rho}).
$$
}
\end{definition}

Our expression is the same as \cite[Definition 2.2]{Konishi06b} and is equivalent to that of \cite{AKMV03,Zhou03,LLLZ09} (see \cite[Definition 2.1]{LLLZ09}) by \cite[Proposition 4.4]{Zhou03}. $\cW_{\lambda^1, \lambda^2, \lambda^3}(q)$ is a rational function in $q^{\frac{1}{2}}$ and has cyclic symmetry \cite{ORV03}
$$
    \cW_{\lambda^1, \lambda^2, \lambda^3} = \cW_{\lambda^2, \lambda^3, \lambda^1} = \cW_{\lambda^3, \lambda^1, \lambda^2}.
$$
We refer to \cite{Zhou03} for additional properties.


\subsection{Open geometry}\label{sect:OpenGeometry}
Let $N \cong \bZ^3$ and $X$ be a smooth toric Calabi-Yau 3-fold defined by a finite fan $\Sigma$ in $N \otimes \bR \cong \bR^3$. We assume that $\Sigma$ contains at least one 3-cone and that every cone in $\Sigma$ is a face of a 3-cone. The Calabi-Yau condition, i.e. the triviality of the canonical bundle $K_X$ of $X$, is equivalent to the existence of $\su_3 \in M := \Hom(N, \bZ)$ such that
$$
    \inner{\su_3, v} = 1
$$
for all primitive generators $v \in N$ of rays of $\Sigma$, where $\inner{-,-}: M \times N \to \bZ$ is the natural pairing. Let $N' := \ker(\su_3) \cong \bZ^2 \subset N$. The algebraic 3-torus of the toric 3-fold $X$ is $T := N \otimes \bC^* \cong (\bC^*)^3$, which contains the Calabi-Yau 2-subtorus $T' := N' \otimes \bC^* \cong (\bC^*)^2$. We complete $\su_3$ into a $\bZ$-basis $\{\su_1, \su_2, \su_3\}$ of $M$. Then
$$
    H^*_{T'}(\pt;\bZ) \cong \bZ[\su_1, \su_2].
$$

We set up some notations. For $d = 0,1,2,3$, let $\Sigma(d)$ to denote the set of $d$-cones in $\Sigma$. For a cone $\sigma \in \Sigma(d)$, let $V(\sigma)$ denote the codimension-$d$ $T$-orbit closure in $X$. In particular, the set $\Sigma(3)$ of maximal cones corresponds to the set of $T$-fixed points, and $\Sigma(2)$ corresponds to the set of $T$-invariant lines, which are isomorphic to either $\bC$ or $\bP^1$. We note that the action of the subtorus $T'$ has the same sets of fixed points and invariant lines. Let
$$
    \Sigma(2)_c := \{\tau \in \Sigma(2) \mid V(\tau) \cong \bP^1\}.
$$
We define the 1-skeleta
$$
    X^1 := \bigcup_{\tau \in \Sigma(2)} V(\tau), \qquad X^1_c := \bigcup_{\tau \in \Sigma(2)_c} V(\tau) \subset X^1.
$$
The inclusion $X^1_c \to X^1 \to X$ induces a surjective group homomorphism
\begin{equation}\label{eqn:CurveClass}
    H_2(X^1_c;\bZ) = H_2(X^1;\bZ) = \bigoplus_{\tau \in \Sigma(2)_c} \bZ[V(\tau)] \to H_2(X;\bZ).
\end{equation}


Let $s \in \bZ_{\ge 1}$ and 
$$
    L = L_1 \sqcup \cdots \sqcup L_s
$$
be a disjoint union of $s$ \emph{Aganagic-Vafa outer branes} in $X$ (see \cite[Section 2]{FL13}). Each $L_i$ is a Lagrangian submanifold of $X$ diffeomorphic to $S^1 \times \bC$ and invariant under the action of the maximal compact subtorus $T'_{\bR} \cong U(1)^2$ of $T'$. Moreover, each $L_i$ intersects a unique 1-dimensional $T$-orbit closure $V(\tau_i)$ for some $\tau_i \in \Sigma(2) \setminus \Sigma(2)_c$. The intersection $L_i \cap V(\tau_i)$ is isomorphic to $S^1$ and bounds a holomorphic disk $B_i$ in $V(\tau_i) \cong \bC$ centered around the $T$-fixed point $V(\sigma_i)$, where $\sigma_i \in \Sigma(3)$ is the unique 3-cone that contains $\tau_i$. We orient the disks $B_i$'s by the holomorphic structure of $X$. We assume that $\tau_1, \dots, \tau_s$ are distinct. Then
\begin{equation}\label{eqn:RelHomology}
    H_2(X,L;\bZ) = H_2(X; \bZ) \oplus \bigoplus_{i=1}^s \bZ[B_i], \qquad H_1(L; \bZ) =  \bigoplus_{i=1}^s \bZ[\partial B_i].
\end{equation}
Moreover, we choose additional parameters
$$
    \vf = (f_1, \dots, f_s)
$$
where for $i = 1, \dots, s$, $f_i \in \bZ$ is called the \emph{framing} of the brane $L_i$.



\subsection{FTCY graphs}\label{sect:FTCY}
The toric Calabi-Yau 3-fold $X$ can be equivalently described by its associated \emph{formal toric Calabi-Yau (FTCY) graph} $\Gamma_X$ as in \cite[Section 3]{LLLZ09}. The underlying graph of $\Gamma_X$ is a trivalent graph where:
\begin{itemize}
    \item The set $V(\Gamma_X)$ of vertices corresponds to the set $\Sigma(3)$ of 3-cones, or equivalently the set of $T$-fixed points. For $v \in V(\Gamma_X)$ we write $\sigma^v \in \Sigma(3)$ for the corresponding $3$-cone.
    
    \item The set $E(\Gamma_X)$ of (unoriented) edges corresponds to the set $\Sigma(2)$ of 2-cones, or equivalently the set of $T$-invariant lines. The subset $E_c(\Gamma_X)$ of compact edges corresponds to $\Sigma(2)_c$.\footnote{As in \cite[Section 3.1]{FL13}, we do not replace the non-compact edges by compact ones ending at univalent vertices, which is done in \cite{LLLZ09}.} For $\bar{e} \in E(\Gamma_X)$ we write $\tau^{\bar{e}} \in \Sigma(2)$ for the corresponding $2$-cone. 
    
    \item An edge $\bar{e} \in E(\Gamma_X)$ is incident to a vertex $v \in V(\Gamma_X)$ if and only if $\tau^{\bar{e}}$ is a facet of $\sigma^v$.
\end{itemize}
The data of $\Gamma_X$ also consists of a \emph{position map}
$$
    \fp: E^o(\Gamma_X) \to \bZ^2 \setminus \{0\},
$$
where $E^o(\Gamma_X)$ is the set of \emph{oriented} edges. To describe $\fp$, we may draw $\Gamma_X$ on the hyperplane $(N' \otimes \bR) \times \{1\}$ in $N \otimes \bR$ as a planar graph dual to the triangulated polytope given by the cross-section of the fan $\Sigma$. Under the coordinates on $M$ specified by the choice of basis $\{\su_1, \su_2, \su_3\}$, and thus the dual coordinates on $N$, $\fp(e)$ is given by the primitive integral vector in the direction of the oriented edge $e \in E^o(\Gamma_X)$ in the planar drawing.

As in \cite[Section 3.1]{FL13}, the framed Aganagic-Vafa branes $(L_1, f_1), \dots, (L_s,f_s)$ determine a FTCY graph
$$
    \Gamma = \Gamma_{X, L, \vf}
$$
obtained by replacing the non-compact edge in $\Gamma_X$ corresponding to $\tau_i$ by a compact edge $\bar{e}_i$ ending at a univalent vertex $v_i$ and with \emph{framing vector} $\ff_i \in \bZ^2 \setminus \{0\}$ depending on $f_i$. The set of vertices in $\Gamma$ is decomposed as
$$
    V(\Gamma) = V^3(\Gamma) \sqcup V^1(\Gamma)
$$
where $V^3(\Gamma) := V(\Gamma_X)$ is the set of trivalent vertices in $\Gamma$ and $V^1(\Gamma):= \{v_1, \dots, v_s\}$ is the set of univalent vertices. The set of (unoriented) compact edges in $\Gamma$ is decomposed as
$$
    E_c(\Gamma) = E_c^3(\Gamma) \sqcup E_c^1(\Gamma)
$$
where $E_c^3(\Gamma) := E_c(\Gamma_X)$ and $E_c^1(\Gamma) := \{\bar{e}_1, \dots, \bar{e}_s\}$.

As in \cite[Definition 3.4]{LLLZ09}, the position map $\fp$ and the framing vectors $\ff_i$'s together define a map
$$
    \vn: E^o(\Gamma) \to \bZ,
$$
where $E^o(\Gamma)$ is the set of oriented edges in $\Gamma$. We write $n^e := \vn(e)$. Let $E^o_c(\Gamma)$ denote the subset of oriented compact edges and for $e \in E^o_c(\Gamma)$, let $-e \in E^o_c(\Gamma)$ denote the edge with opposite orientation. Then $\vn$ satisfies that
$$
    n^{-e} = -n^e.
$$
For oriented edges $\pm e \in E^o_c(\Gamma)$ whose corresponding unoriented edge $\bar{e}$ belongs to $E_c^3(\Gamma)$ and corresponds to $\tau \in \Sigma(2)$, $n^{\pm e}$ is determined by the degree of the normal bundle of $V(\tau)$ in $X$:
$$
    N_{V(\tau)/X} \cong \cO_{\bP^1}(n^e-1) \oplus \cO_{\bP^1}(n^{-e}-1).
$$
For the oriented edge $e_i$ given by orienting $\bar{e}_i$ as terminating at $v_i$, we have
$$
    n^{e_i} = f_i.
$$

Let $\hX$ denote the FTCY 3-fold associated to the FTCY graph $\Gamma_X$, which is the formal completion of $X$ along the 1-skeleton $X^1$. Let $(\hY, \hD)$ denote the relative FTCY 3-fold associated to $\Gamma$, which satisfies $\hY \setminus \hD = \hX$ and admits an action of the Calabi-Yau 2-torus $T'$. The divisor $\hD$ is a disjoint union of its $s$ connected components $\hD^s, \dots, \hD^s$ corresponding to the $s$ univalent vertices $v_1, \dots, v_s \in V^1(\Gamma)$. Each edge $\bar{e}_i \in E^1_c(\Gamma)$ corresponds to a $T'$-invariant projective line $C_i$ in $\hY$ which is the compactification of $V(\tau_i) \subset \hX$ by the $T'$-fixed point contained in $\hD^i$. The homomorphism \eqref{eqn:CurveClass} extends to a surjective group homomorphism
\begin{equation}\label{eqn:RelCurveClass}
    \pi: H_2(\hY;\bZ) = H_2(X^1;\bZ) \oplus \bigoplus_{i=1}^s \bZ[C_i] \to H_2(X,L;\bZ) = H_2(X; \bZ) \oplus \bigoplus_{i=1}^s \bZ[B_i]
\end{equation}
that maps each $[C_i]$ to $[B_i]$ (see \eqref{eqn:RelHomology}).



\subsection{Gromov-Witten invariants}\label{sect:GW}
Let $g \in \bZ_{\ge 0}$. Let
$$
    \beta = \beta' + \sum_{i=1}^s d_i[B_i] \qquad \in H_2(X,L; \bZ)
$$
where $\beta' \in H_2(X;\bZ)$ is an effective class\footnote{Our notations $\beta, \beta'$ are interchanged from those in \cite{FL13}.} and $(d_1, \dots, d_s) \in \bZ_{\ge 0}^s \setminus \{(0, \dots, 0)\}$. Let
$$
    \vmu = (\mu^1, \dots, \mu^s) \in \cP^s \setminus \{(\emptyset, \dots, \emptyset)\}
$$
be partitions with $\mu^i \vdash d_i$ for $i = 1, \dots, s$, where $\emptyset$ denotes the empty partition.

\begin{definition}\label{def:OpenClass}
\rm{
The pair $(\beta, \vmu)$ as above is called an \emph{effective class} of $(X, L, \vf)$. Let
$$
    \Eff(X, L, \vf)
$$
denote the set of all effective classes.
}
\end{definition}

The genus-$g$, degree-$(\beta, \vmu)$ \emph{open Gromov-Witten invariant}
$$
    N^{X, L, \vf}_{g, \beta, \vmu} \in \bQ
$$
is a virtual count of open stable maps
$$
    u: \left(C, \partial C = \bigsqcup_{i = 1}^s \bigsqcup_{j = 1}^{\ell(\mu^i)} R^i_j \right) \to (X,L)
$$
where:
\begin{itemize}
    \item $C$ is a connected prestable bordered Riemann surface of arithmetic genus $g$ and whose boundary $\partial C$ consists of $\sum_{i=1}^s \ell(\mu^i)$ components $R^i_j \cong S^1$.
    
    \item $u_*[C] = \beta \in H_2(X,L;\bZ)$ and $u_*[R^i_j] = \mu^i_j[\partial B_i] \in H_1(L^i; \bZ)$.
\end{itemize}

Following \cite[Section 3.4]{FL13}, we define $N^{X, L, \vf}_{g, \beta, \vmu}$ by \emph{formal relative Gromov-Witten invariants} of the relative FTCY 3-fold $(\hY, \hD)$.\footnote{In the case $s=1$ of a single brane, see \cite[Section 3.5]{FL13} for an equivalent definition by virtual integration on the moduli spaces of open stable maps; see also \cite{KL01}.} Let
$$
    \vd: E_c(\Gamma) \to \bZ_{\ge 0}
$$
such that the image of the effective class
$$
    \sum_{\bar{e} \in E^3_c(\Gamma)} \vd(\bar{e})[V(\tau^{\bar{e}})] + \sum_{i = 1}^s \vd(\bar{e}_i)[C_i] \qquad \in H_2(\hY;\bZ),
$$
under the map $\pi$ (see \eqref{eqn:RelCurveClass}) is $\beta$.\footnote{Our notation $\vd$ corresponds to $\vd'$ in \cite{FL13}.} In particular, $\vd(\bar{e}_i) = d_i$ for $i = 1, \dots, s$. We denote this effective class of $\hY$ also by $\vd$ by an abuse of notation. 

\begin{definition}\label{def:RelClass}
\rm{
The pair $(\vd, \vmu)$ as above is called an \emph{effective class} of the FTCY graph $\Gamma$. Let
$$
    \Eff(\Gamma)
$$
denote the set of all effective classes.
}
\end{definition}

The genus-$g$, degree-$(\vd, \vmu)$ formal relative Gromov-Witten invariant
$$
    N^{\hY, \hD}_{g, \vd, \vmu} \in \bQ
$$
is a virtual count of relative stable maps
$$
    u: \left(C, \{q^i_j \mid i = 1, \dots, s, j = 1, \dots, \ell(\mu^i) \} \right) \to (\hY,\hD)
$$
where:
\begin{itemize}
    \item $C$ is a connected prestable (borderless) Riemann surface of arithmetic genus $g$ and $\{q^i_j\}$ is a set of distinct smooth points on $C$.
    
    \item $u_*[C] = \vd \in H_2(\hY;\bZ)$ and $u^{-1}(\hD^i) = \sum_{j=1}^{\ell(\mu^i)} \mu^i_j q^i_j$ as Cartier divisors.
\end{itemize}
The invariant $F^{\hY, \hD}_{g, \vd, \vmu}$ is defined by $T'$-equivariant localization on the corresponding moduli space of relative stable maps; see \cite[Section 4]{LLLZ09}, \cite[Section 3.2]{FL13}. A priori, it is a rational function in the equivariant parameters $\su_1, \su_2$ that is homogeneous of degree zero, but it is in fact a rational number independent of $\su_1, \su_2$ \cite{LLLZ09}. Then, the open Gromov-Witten invariant is defined as
$$
    N^{X, L, \vf}_{g, \beta, \vmu} := (-1)^{|\vmu|-\ell(\vmu)}\sum_{\vd: \pi(\vd) = \beta} N^{\hY, \hD}_{g, \vd, \vmu}.
$$

We consider the generating functions
$$
    F^{X, L, \vf}_{\beta, \vmu}(g_s) := \sum_{g \in \bZ_{\ge 0}} N^{X, L, \vf}_{g, \beta, \vmu} (g_s)^{2g-2+\ell(\vmu)}, \qquad F^{\hY, \hD}_{\vd, \vmu}(g_s) := \sum_{g \in \bZ_{\ge 0}} N^{\hY, \hD}_{g, \vd, \vmu} (g_s)^{2g-2+\ell(\vmu)}
$$
where $g_s$ is a formal variable. Then
\begin{equation}\label{eqn:OpenRelFunction}
    F^{X, L, \vf}_{\beta, \vmu}(g_s) = (-1)^{|\vmu|-\ell(\vmu)}\sum_{\vd: \pi(\vd) = \beta}F^{\hY, \hD}_{\vd, \vmu}(g_s).
\end{equation}
Moreover, let
$$
    \{Q^\beta \mid \beta \in H_2(X,L; \bZ) \text{ effective}\}, \qquad \{Q^{\vd} \mid \vd : E_c(\Gamma) \to \bZ_{\ge 0}\}, \qquad \{P_{\vmu} \mid \vmu \in \cP^s\}
$$
be formal variables with multiplication rules
$$
    Q^{\beta_1}Q^{\beta_2} = Q^{\beta_1 + \beta_2}, \qquad Q^{\vd_1}Q^{\vd_2} = Q^{\vd_1 + \vd_2}, \qquad  P_{\vmu} P_{\vnu} = P_{\vmu \sqcup \vnu},
$$
where for $\vmu = (\mu^1, \dots, \mu^s), \vnu = (\nu^1, \dots, \nu^s) \in \cP^s$ we write $\vmu \sqcup \vnu = (\mu^1 \sqcup \nu^1, \dots, \mu^s \sqcup \nu^s)$. We define
\begin{align*}
    & F^{X, L, \vf}(g_s, Q, P) := \sum_{(\beta, \vmu) \in \Eff(X, L, \vf)} F^{X, L, \vf}_{\beta, \vmu}(g_s)Q^\beta P_{\vmu}, \\
    & F^{\hY, \hD}(g_s, Q, P) := \sum_{(\vd, \vmu) \in \Eff(\Gamma)}F^{\hY, \hD}_{\vd, \vmu}(g_s) Q^{\vd}P_{\vmu}.
\end{align*}

\subsection{The topological vertex}\label{sect:TopVertex}
We now state the computation of $F^{\hY, \hD}(g_s, Q, P)$ using the \emph{topological vertex} formalism \cite{AKMV03, LLLZ09}. We define
$$
    Z^{\hY, \hD}(g_s, Q, P) := \exp \left( F^{\hY, \hD}(g_s, Q, P) \right)
$$
and write
$$
    Z^{\hY, \hD}(g_s, Q, P) = 1 + \sum_{(\vd, \vmu) \in \Eff(\Gamma)}Z^{\hY, \hD}_{\vd, \vmu}(g_s) Q^{\vd}P_{\vmu}.
$$
Geometrically, the coefficient $Z^{\hY, \hD}_{\vd, \vmu}(g_s)$ is the generating function of degree-$(\vd, \vmu)$ formal relative Gromov-Witten invariants of $(\hY, \hD)$ that account for stable maps with possibly disconnected domains (see \cite[Section 4]{LLLZ09}). 

Let $(\vd, \vmu) \in \Eff(\Gamma)$. We set
$$
    T_{\vd} := \{ \vlambda: E^o(\Gamma) \to \cP \mid \vlambda(e) \vdash \vd(\bar{e}), \vlambda(-e) = \vlambda(e)^t\}.
$$
Here $\bar{e} \in E(\Gamma)$ denotes the unoriented edge corresponding to $e \in E^o(\Gamma)$ and we set $\vd(\bar{e}) = 0$ if $\bar{e} \not \in E_c(\Gamma)$. Moreover, for any trivalent vertex $v \in V^3(\Gamma)$, if $e^1, e^2, e^3 \in E^o(\Gamma)$ are the three edges emanating from $v$ and appearing in counterclockwise order in $\Gamma$, we set
$$
    \vlambda^v := (\vlambda(e^1), \vlambda(e^2), \vlambda(e^3))
$$
which is unique up to cyclic symmetry.

\begin{theorem}[\cite{LLLZ09}] \label{thm:TopVertex}
For any $(\vd, \vmu) \in \Eff(\Gamma)$, we have
\begin{equation}\label{eqn:TopVertex}
    Z^{\hY, \hD}_{\vd, \vmu}(q) = \sum_{\vlambda \in T_{\vd}} \prod_{\bar{e} \in E(\Gamma)} (-1)^{(n^e + 1)\vd(\bar{e})} q^{\frac{\kappa_{\vlambda(e)}n^e}{2}} \prod_{v \in V^3(\Gamma)} \cW_{\vlambda^v}(q) \prod_{i = 1}^s \frac{\chi_{\vlambda(-e_i)}(\mu^i)}{z_{\mu^i}}\sqrt{-1}^{\ell(\mu^i)}(-1)^{|\mu^i|}
\end{equation}
under the change of variables
$$
    q = e^{\sqrt{-1}g_s}.
$$
\end{theorem}

Here and throughout the paper, we fix a choice of $\sqrt{-1}$. We further explain the notations in \eqref{eqn:TopVertex}. For each $\bar{e} \in E(\Gamma)$, $e \in E^o(\Gamma)$ is a choice of an orientation of $\bar{e}$ and the choice does not affect \eqref{eqn:TopVertex} since
$$
    n^{-e} = -n^e, \qquad \kappa_{\vlambda(-e)} = \kappa_{\vlambda(e)^t} = -\kappa_{\vlambda(e)}.
$$
For each $v \in V^3(\Gamma)$, $\cW_{\vlambda^v}(q)$ is the function defined in Definition \ref{def:ThreePoint} associated to the triple $\vlambda^v \in \cP^3$. Finally, for any $\lambda, \mu \in \cP$ with $|\lambda| = |\mu| = d$, $\chi_\lambda$ denotes the irreducible character of $S_d$ indexed by $\lambda$ and $\chi_\lambda(\mu) \in \bZ$ is the value on the conjugacy class indexed by $\mu$. Recall that $e_i$ is the edge $\bar{e}_i$ oriented to terminate at the univalent vertex $v_i \in V_1(\Gamma)$.

Theorem \ref{thm:TopVertex} is obtained from \cite[Proposition 7.4, Corollary 7.6]{LLLZ09} by using the combinatorial expression $\cW_{\vlambda^v}$ to substitute the expression $\tC_{\vlambda^v}$ defined in \cite[Section 6.4]{LLLZ09} via Gromov-Witten invariants. The substitution is valid by results in \cite[Section 8]{LLLZ09} and \cite{MOOP11}. We note that there are sign discrepancies between \eqref{eqn:TopVertex} and \cite[Proposition 7.4]{LLLZ09}, which we address in detail in Appendix \ref{appdx:TopVertex}.



