\section{Introduction}\label{sect:Intro}

\subsection{Open BPS invariants of Calabi-Yau 3-folds}\label{sect:IntroOpen}
Let $X$ be a smooth Calabi-Yau 3-fold. The famous conjecture of Gopakumar and Vafa \cite{GV98a,GV98b} concerns the relation between two enumerative theories of $X$:
\begin{itemize}
    \item \emph{Gromov-Witten invariants} $N^X_{g,\beta}$, which are virtual counts of curves in $X$ and are rational numbers in general;
    \item \emph{BPS invariants} $n^X_{g,\beta}$, which are counts of BPS states supported on curves in $X$ and are integers according to physical interpretations and predictions. 
\end{itemize}
Here, the invariants are parameterized by the genus $g \in \bZ_{\ge 0}$ and class $\beta \in H_2(X; \bZ)$ of the curve. Gopakumar and Vafa conjectured that the two sets of invariants are related by the resummation formula
\begin{equation}\label{eqn:IntroGVResum}
    \sum_{g \in \bZ_{\ge 0}} N^{X}_{g, \beta} g_s^{2g-2}\\
    = \sum_{k \mid \beta} \sum_{g \in \bZ_{\ge 0}} \frac{n^{X}_{g,\frac{\beta}{k}}}{k} \left(2\sin \frac{kg_s}{2} \right)^{2g-2}
\end{equation}
where $g_s$ is a formal variable, $\beta \in H_2(X; \bZ)$ is a non-zero effective curve class, and for $k \in \bZ_{\ge 1}$ we say that $k \mid \beta$ if $\frac{\beta}{k}\in H_2(X; \bZ)$.

While the mathematical foundations of Gromov-Witten theory are relatively well-established, a rigorous mathematical definition of the BPS invariants is yet to be given and there have been rich developments in the literature; see e.g. \cite{HST01,Katz08,PT10,MT18}. If one takes \eqref{eqn:IntroGVResum} to be the definition, the Gopakumar-Vafa conjecture can be phrased as \emph{integrality} and \emph{finiteness} properties of the BPS invariants thereby defined \cite{GV98a,GV98b,BP01}. Namely, for any fixed $\beta \in H_2(X; \bZ)$, $n^X_{g,\beta}$ is an integer for all $g \in \bZ_{\ge 0}$ and is zero for $g \gg 0$. When the Calabi-Yau 3-fold $X$ is compact, the integrality and finiteness parts were proved by Ionel-Parker \cite{IP18} and Doan-Ionel-Walpuski \cite{DIW21} respectively using symplectic methods. When $X$ is toric Calabi-Yau (in which case it is non-compact), a proof was given in the earlier works of Peng \cite{Peng07} and Konishi \cite{Konishi06a,Konishi06b} based on the computation of all-genus Gromov-Witten invariants by the topological vertex \cite{AKMV03,LLLZ09} and the Gromov-Witten/Donaldson-Thomas correspondence \cite{MNOP06,MOOP11}. The conjecture has also been studied in the Fano case by Zinger \cite{Zinger11} and Doan-Walpuski \cite{DW19}.

Analogously, for open topological strings, the open Gromov-Witten invariants of $X$ with prescribed Lagrangian boundary conditions are also expected to carry integrality properties encoded by open BPS counts. We consider the case where $X$ is toric and the boundary condition is given by a disjoint union $L = L_1 \sqcup \cdots \sqcup L_s$ of $s$ special Lagrangian submanifolds called \emph{Aganagic-Vafa outer branes} \cite{AV00,AKV02,FL13} framed by integers $\vf = (f_1, \dots, f_s)$. Each $L_i$ is diffeomorphic to $S^1 \times \bC$ and invariant under a $U(1)^2$-action on $X$. Given genus $g \in \bZ_{\ge 0}$, effective class $\beta \in H_2(X,L;\bZ)$, and a sequence of partitions $\vmu = (\mu^1, \dots, \mu^s)$, the \emph{open Gromov-Witten invariant}
$$
    N^{X, L, \vf}_{g, \beta, \vmu}
$$
is a virtual count of genus-$g$, degree-$\beta$ bordered Riemann surfaces in $X$ whose winding profile at $L_i$ is determined by the partition $\mu^i$. These invariants are also defined and computed by the topological vertex. A particularly well-studied example of such open geometries is the resolved conifold $X = \Tot(\cO_{\bP^1}(-1) \oplus \cO_{\bP^1}(-1))$ relative to a single brane, whose open Gromov-Witten invariants conjecturally correspond to knot and link invariants originating from Chern-Simons gauge theory on $S^3$; see e.g. \cite{Witten89,Witten95,GV99,OV00,LM00,LMV00,RS01,MV02}. Labastida, Mari\~no, Ooguri, and Vafa (LMOV) \cite{OV00,LM00,LMV00,MV02} discovered a resummation formula, similar to \eqref{eqn:IntroGVResum}, that exhibit the integrality of these invariants. For a general open geometry $(X, L, \vf)$ and winding profile $\vmu$, the formula can be written as
\begin{equation}\label{eqn:IntroLMOVResum}
    \begin{aligned}
        \sum_{g \in \bZ_{\ge 0}} & N^{X, L, \vf}_{g, \beta, \vmu} g_s^{2g-2+\ell(\vmu)} \\
        & = \frac{1}{\prod_{i = 1}^s z_{\mu^i}} \sum_{k \mid \beta, \vmu}  \sum_{g \in \bZ_{\ge 0}} (-1)^{\ell(\vmu)+g}k^{\ell(\vmu)-1}n^{X, L, \vf}_{g,\frac{\beta}{k}, \frac{\vmu}{k}}\left(2\sin \frac{kg_s}{2} \right)^{2g-2} \prod_{i=1}^s \prod_{j=1}^{\ell(\mu^i)} 2\sin \frac{\mu^i_jg_s}{2}.
    \end{aligned}
\end{equation}
Here, for each partition $\mu^i$, $\ell(\mu^i)$ is its length, the $\mu^i_j$'s are its parts, and
$$
    z_{\mu^i} := |\Aut(\mu^i)| \prod_{j=1}^{\ell(\mu^i)} \mu^i_j
$$
where $\Aut(\mu^i)$ is the automorphism group of $\mu^i$. We set $\ell(\vmu) := \sum_{i=1}^s \ell(\mu^i)$. Moreover, for $k \in \bZ_{\ge 1}$ we say that $k \mid \vmu$ if $k \mid \mu^i_j$ for all $i, j$, in which case $\frac{\vmu}{k}$ denotes the sequence of partitions whose parts are $\frac{\mu^i_j}{k}$.

We refer to the invariants
$$
    n^{X, L, \vf}_{g, \beta, \vmu}
$$
arising from \eqref{eqn:IntroLMOVResum} as the \emph{open BPS invariants} of $(X, L, \vf)$. Similar to the Gopakumar-Vafa conjecture in the closed sector, the open BPS invariants are also predicted to satisfy integrality and finiteness properties. As the first main result of the present paper, we verify this prediction.

\begin{theorem}[See Theorem \ref{thm:OpenBPS}]\label{thm:IntroOpenBPS}
For any effective class $\beta \in H_2(X,L;\bZ)$ and winding profile $\vmu$, we have that
$$
    n^{X, L, \vf}_{g, \beta, \vmu} \in \bZ
$$
for all $g \in \bZ_{\ge 0}$ and is zero for $g \gg 0$.
\end{theorem}

Theorem \ref{thm:IntroOpenBPS} generalizes several interesting special cases previously known in the literature. Luo and Zhu \cite{LZ16,LZ19} considered the resolved conifold relative to a single brane and studied its genus-zero invariants with a general winding profile and higher-genus invariants ``with one hole'', i.e. the length of the partition is 1. It is worth noting that their method involves the computation of the Gromov-Witten invariants by the Bouchard-Klemm-Mari\~{n}o-Pasquetti \emph{Remodeling conjecture} \cite{BKMP09,BKMP10,EO15,FLZ20}. Zhu \cite{Zhu19} further studied higher-genus one-hole invariants of $\bC^3$ relative to a single brane. More recently, Panfil and Su\l{}kowski \cite{PS19} considered \emph{``strip geometries''}, which is a class of toric Calabi-Yau 3-folds without compact 4-cycles, and their \emph{disk invariants}, or genus-zero one-hole invariants, with boundary at a certain choice of a brane. Using the \emph{knots-quivers correspondence} \cite{KRSS17,KRSS19,EKL20a,EKL20b}, they related such invariants to the Donaldson-Thomas invariants of symmetric quivers, which are known to be integers \cite{Efimov12}. This perspective was taken by Bousseau, Brini, van Garrel \cite{BBvG20}, and Zhu \cite{Zhu22}. In particular, \cite{BBvG20} considered strip geometries that arise from \emph{Looijenga pairs}, or log Calabi-Yau surfaces with maximal boundary, and established an all-genus correspondence between the open and log Gromov-Witten invariants. They further proved the integrality and finiteness of the higher-genus open BPS invariants, which then carry over to the log side. Integrality properties of log invariants in connection with open invariants have also been studied in the recent works \cite{GRZ22,BS23,Schuler24}.

Compared to the previous works, Theorem \ref{thm:IntroOpenBPS} takes a very general form in the sense that it applies to a general toric Calabi-Yau 3-fold, any number of branes in an arbitrary configuration, and any winding profiles. In terms of techniques, the proof is closest to that of \cite{Peng07,Konishi06a,Konishi06b} in the closed sector which is based on the topological vertex.



\subsection{Closed BPS invariants of Calabi-Yau 4-folds and the open/closed correspondence}\label{sect:IntroClosed}
Back to the closed sector and in the genus zero case, the Gopakumar-Vafa formula \eqref{eqn:IntroGVResum} specializes to the Aspinwall-Morrison multiple covering formula \cite{AM93}
\begin{equation}\label{eqn:IntroAMResum}
    N^{X}_{0, \beta} \\
    = \sum_{k \mid \beta} \frac{n^{X}_{0,\frac{\beta}{k}}}{k^3}.
\end{equation}
This formula is also expected to exhibit the integrality of Gromov-Witten invariants of Calabi-Yau manifolds of higher dimensions. Klemm and Pandharipande \cite[Conjecture 0]{KP08} conjectured that for a Calabi-Yau 4-fold $Z$ and insertions $\gamma_1, \dots, \gamma_n \in H^*(Z;\bZ)$, the BPS invariants $n^{Z}_{0, \beta}(\gamma_1, \dots, \gamma_n)$ defined from the Gromov-Witten invariants $N^{Z}_{0, \beta}(\gamma_1, \dots, \gamma_n)$ by the formula
\begin{equation}\label{eqn:IntroKPResum}
    N^{Z}_{0, \beta}(\gamma_1, \dots, \gamma_n) \\
    = \sum_{k \mid \beta} \frac{n^{Z}_{0,\frac{\beta}{k}}(\gamma_1, \dots, \gamma_n)}{k^{3-n}}
\end{equation}
are integers. The conjecture has been verified in examples in \cite{KP08,CMT18,Cao20,CMT22,CKM22,COT22,COT24,BBvG20,BS23}. For a compact semi-positive symplectic manifold $Z$ of real dimension at least 6, the conjecture was proved by Ionel-Parker \cite{IP18}. We note that the approach of \cite{CMT18,Cao20,CMT22,CKM22,COT22,COT24} is based on the proposal that the BPS invariants of Calabi-Yau 4-folds admit sheaf-counting interpretations in terms of e.g. Donaldson-Thomas invariants and Pandharipande-Thomas invariants. We do not pursue this perspective in the present paper.

On the other hand, we take the approach of relating closed invariants of 4-folds to open invariants of 3-folds via the \emph{open/closed correspondence}, developed by Liu and the author \cite{LY21,LY22} based on the original proposal of Mayr \cite{Mayr01} and recent studies of correspondences among different types (open, log/relative, local) of enumerative invariants \cite{LLLZ09,FL13,vGGR19,BBvG20}. To be more specific, given an open geometry on a toric Calabi-Yau 3-fold $X$ relative to a single framed outer brane $(L,f)$, we consider its disk invariants $N^{X, L, f}_{0, \beta, (d)}$ where for an effective class $\beta \in H_2(X,L;\bZ)$, $d \in \bZ_{\ge 1}$ denotes the winding number at the unique boundary component of the curve and is determined by $\beta$. Assuming that $L$ remains an outer brane in a toric Calabi-Yau semi-projective partial compactification of $X$ (see Assumption \ref{assump:Outer}), on the level of Gromov-Witten theory, the open/closed correspondence produces a toric Calabi-Yau 4-fold $\tX$ together with an isomorphism $\iota: H_2(X, L;\bZ) \to H_2(\tX;\bZ)$ and a class $\tgamma \in H^4(\tX;\bZ)$ such that
$$
    N^{X, L, f}_{0, \beta, (d)} = N^{\tX}_{0, \iota(\beta)}(\tgamma)
$$
for any $\beta$.

Here we give a high-level summary of the construction of $\tX$, deferring a more detailed description to Section \ref{sect:OpenClosedGW}. Starting from the open geometry $(X,L,f)$, one first partially compactifies $X$ by adding an additional toric divisor $D$ at the location of the brane $L$, obtaining a toric 3-fold $Y = X \sqcup D$ such that the pair $(Y,D)$ is log Calabi-Yau. This step is the construction of \cite{LLLZ09,FL13} who identified the open Gromov-Witten invariants of $(X,L,f)$ with the relative invariants of $(Y,D)$. Then, $\tX$ is taken to be the total space of the canonical bundle $\cO_Y(-D)$, which is a toric Calabi-Yau 4-fold. This may be viewed as an instantiation of the \emph{log-local principle} \cite{vGGR19} which identifies the log/relative invariants of $(Y,D)$ with the closed invariants of the local geometry $\tX$. We note that when $X$ is semi-projective, \cite{LY22} further established the open/closed correspondence with $\tX$ replaced by its toric Calabi-Yau semi-projective partial compactification. 

Now in the case of disk invariants, the LMOV formula \eqref{eqn:IntroLMOVResum} specializes to
%\begin{equation}\label{eqn:IntroLMOVOpen}
$$
    N^{X, L, f}_{0, \beta, (d)} = - \sum_{k \mid \beta, d} \frac{n^{X, L, f}_{0,\frac{\beta}{k}, \left(\frac{d}{k}\right)}}{k^2},
$$
%\end{equation}
which coincides up to sign with the Klemm-Pandharipande resummation \eqref{eqn:IntroKPResum} for invariants with 1-pointed insertions. Indeed, this was already observed by Bousseau, Brini, and van Garrel \cite{BBvG20} who obtained the integrality of BPS invariants of closed geometries arising from Looijenga pairs. We establish the open/closed correspondence for the BPS invariants of more general geometries.

\begin{theorem}[See Theorem \ref{thm:BPSCorrespondence}]\label{thm:IntroBPSCorrespondence}
For the open geometry $(X,L,f)$ and closed geometry $\tX$ under the open/closed correspondence above, given any effective class $\beta \in H_2(X,L;\bZ)$, we have
$$
    n^{X, L, f}_{0, \beta, (d)} = -n^{\tX}_{0, \iota(\beta)}(\tgamma).
$$
\end{theorem}

Combining this result with Theorem \ref{thm:IntroOpenBPS}, we obtain the integrality of the closed BPS invariants of $\tX$.

\begin{corollary}[See Corollary \ref{cor:ClosedBPS}]\label{cor:IntroClosedBPS}
Under the setup of Theorem \ref{thm:IntroBPSCorrespondence}, we have
$$
    n^{\tX}_{0, \iota(\beta)}(\tgamma) \in \bZ.
$$
\end{corollary}

Therefore, we obtain a general class of non-compact examples for the conjecture of Klemm-Pandharipande \cite[Conjecture 0]{KP08} which covers certain previously known examples. We give a more detailed account in Section \ref{sect:ClosedBPS}.



%\subsection{Our main results}\label{sect:IntroResults}


\subsection{Conjectures on open BPS invariants of general geometries}
With the toric case as evidence, it would be interesting to study the integrality of open BPS invariants of more general geometries. Let $(X,\omega,J)$ be a Calabi-Yau symplectic 6-manifold together with an almost complex structure $J$ compatible with the symplectic form $\omega$, and $L \subset X$ be a Lagrangian submanifold that is orientable and has zero Maslov class. For simplicity, suppose $L$ is connected. In this case, the moduli space of $J$-holomorphic open stable maps to $(X,L)$ has virtual dimension 0 regardless of the genus or number of boundary components of the domain \cite{KL01}, and one may attempt to define all-genus open Gromov-Witten invariants of $(X,L)$, e.g. when $X$ is compact or $(X,L)$ admits a suitable $U(1)$-action. The open BPS invariants of $(X,L)$ may then be defined by the LMOV formula \eqref{eqn:IntroLMOVResum} and its generalizations.

We first make the following proposal for disk invariants.

\begin{conjecture}
Assume that $H_2(X,L;\bZ)$ is torsion free and the disk invariant $N^{X,L}_{0,\beta}$ is defined for any effective class $\beta \in H_2(X,L;\bZ)$. Then the open BPS invariants $n^{X,L}_{0,\beta}$ defined by the formula
$$
    N^{X, L}_{0, \beta} = - \sum_{k \mid \beta} \frac{n^{X, L}_{0,\frac{\beta}{k}}}{k^2}
$$
are integers for all $\beta$.
\end{conjecture}

We refer to \cite{PSW08} and the references therein for discussions on the case where $X$ is the quintic 3-fold and $L \subset X$ is the real quintic. The open/closed correspondence has also been extended to this geometry \cite{AL23}.

In addition, in the case where a notion of winding numbers can be defined, we make the following proposal.

\begin{conjecture}\label{conj:General}
Assume that $H_2(X,L;\bZ)$ is torsion free, $H_1(L;\bZ) \cong \bZ$, and the open Gromov-Witten invariant $N^{X,L}_{0,\beta,\mu}$ is defined for any genus $g \in \bZ_{\ge 0}$, effective class $\beta \in H_2(X,L;\bZ)$, and partition $\mu$ specifying the winding profile. Let the open BPS invariants $n^{X,L}_{g,\beta, \mu}$ be defined by the formula
$$    
    \sum_{g \in \bZ_{\ge 0}} N^{X, L}_{g, \beta, \mu} g_s^{2g-2+\ell(\mu)} 
    = \frac{1}{z_{\mu}} \sum_{k \mid \beta, \mu}  \sum_{g \in \bZ_{\ge 0}} (-1)^{\ell(\mu)+g}k^{\ell(\mu)-1}n^{X, L}_{g,\frac{\beta}{k}, \frac{\mu}{k}}\left(2\sin \frac{kg_s}{2} \right)^{2g-2} \prod_{j=1}^{\ell(\mu)} 2\sin \frac{\mu_jg_s}{2}.
$$
Then for any $\beta, \mu$, we have that $n^{X, L}_{g, \beta, \mu} \in \bZ$ for all $g \in \bZ_{\ge 0}$ and is zero for $g \gg 0$.
\end{conjecture}

It would be interesting to generalize the LMOV formula \eqref{eqn:IntroLMOVResum} to the case where $H_1(L;\bZ)$ has higher rank and study the integrality and finiteness of open BPS invariants in the general setup. In any case, for a general geometry $(X,L)$, the LMOV formula can be employed to study the \emph{local} contributions of embedded (bordered) curves and their multiple covers, following the approach of \cite{IP18,DIW21}. To be more specific, let $(C, \partial C) \subset (X, L)$ be a bordered embedded curve and consider the pair $(E, E_{\bR})$, where $E := N_{C/X}$ and $E_{\bR} := N_{\partial C/L}$ which is a Lagrangian (real) subbundle of $N_{C/X} \big|_{\partial C}$ (c.f. \cite{KL01}). As in Bryan-Pandharipande \cite{BP08} and Ionel-Parker \cite{IP18}, one may consider the local Calabi-Yau geometry on the total space of $(E, E_{\bR})$, define a local (or residue) version of open Gromov-Witten invariants using a suitable fiberwise $U(1)$-action, define a local version of open BPS invariants using the LMOV formula, and study their integrality and finiteness as in Conjecture \ref{conj:General}. This would account for the multiple-cover contributions of $(C, \partial C)$. Under the philosophy of \cite{IP18,DIW21}, the global statement for a relative curve class $\beta \in H_2(X,L;\bZ)$ would follow from collecting contributions from a finite set of embedded curves in this class.


\begin{comment}
Future work
\begin{itemize}
    \item A general conjecture
    \item Castelnuovo's bound
    \item Orbifolds (via remodeling)
    \item Multi-brane open/closed 
\end{itemize}
\end{comment}

%\subsection{Organization of the paper}

\subsection{Acknowledgments}
This work originates from the author's joint project with Chiu-Chu Melissa Liu \cite{LY21,LY22} and I would like to thank Melissa for many valuable discussions and suggestions. I would like to thank Eleny Ionel for enlightening discussions on the study of open BPS invariants of general geometries. I would like to thank Bohan Fang, Yannik Schuler, and Zhengyu Zong for helpful discussions. I would also like to thank the authors of \cite{BBvG20} for their inspiring results on the open and closed BPS invariants. Finally, I would like to thank the anonymous referees for the valuable comments which greatly helped improve the paper.