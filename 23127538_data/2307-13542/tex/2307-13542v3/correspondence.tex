\section{Open/closed BPS correspondence and integrality}\label{sect:OpenClosed}
In this section, we specialize to the case of disk invariants of a single outer brane and extend the open/closed correspondence \cite{LY21,LY22} of Gromov-Witten invariants to a correspondence of BPS invariants (Theorem \ref{thm:BPSCorrespondence}). Using the integrality of the open BPS invariants (Theorem \ref{thm:OpenBPS}), we obtain the integrality of the closed BPS invariants (Corollary \ref{cor:ClosedBPS}).

\subsection{Specializing to disk invariants}\label{sect:Disk}
We first consider the specialization of results in Section \ref{sect:OpenBPS} to the case of disk invariants of a single outer brane. Let $s=1$ and we write $(L,f) = (L^1, f_1)$, $d = d_1 \in \bZ_{\ge 1}$, $B = B_1$, and $C = C_1$. We further consider the case $\mu = \mu^1 = (d)$. The open Gromov-Witten invariant
$$
    N^{X, L, f}_{0, \beta, (d)} \in \bQ
$$
is a virtual count of open stable maps from domains of arithmetic genus zero and with a single boundary component, and we refer to it as the degree-$(\beta, (d))$ \emph{disk invariant} of $(X, L, f)$.

In the above setting, the genus-zero limit $g_s \to 0$ (or equivalently $q \to 1$) of the LMOV resummation formula \eqref{eqn:LMOVResumGs} gives
\begin{equation}\label{eqn:LMOVOpen}
    N^{X, L, f}_{0, \beta, (d)} = - \sum_{k \mid \beta, d} \frac{n^{X, L, f}_{0,\frac{\beta}{k}, \left(\frac{d}{k}\right)}}{k^2}.
\end{equation}
By Theorem \ref{thm:OpenBPS}, the genus-zero, degree-$(\beta, (d))$ open BPS invariant of $(X, L, f)$ defined by the above is an integer for any $\beta, d$:
$$
    n^{X, L, f}_{0, \beta, (d)} \in \bZ.
$$


\subsection{Gromov-Witten correspondence}\label{sect:OpenClosedGW}
In \cite{LY21}, Liu and the author identified the disk invariants of the open geometry $(X,L,f)$ with the genus-zero Gromov-Witten invariants of a closed geometry on a smooth toric Calabi-Yau 4-fold $\tX$. As in \cite[Assumption 2.3]{LY21}, we make the following assumption.

\begin{assumption}\label{assump:Outer} \rm{
We assume that $L$ remains an outer brane in a (and thus any) toric Calabi-Yau semi-projective partial compactification of $X$.
}
\end{assumption}

In the case of a single outer brane ($s=1$), the FTCY graph $\Gamma$ constructed in Section \ref{sect:FTCY} in fact defines a relative Calabi-Yau 3-fold $(Y,D)$ such that $Y = X \sqcup D$ and $(\hY, \hD)$ is the formal completion of $(Y,D)$ along the toric 1-skeleton $Y^1$. Then $\tX = \Tot(\cO_Y(-D))$. The normal bundle of the new $\tT'$-invariant projective line $C$ in $\tX$ is
$$
    N_{C/\tX} \cong \cO_{\bP^1}(f) \oplus \cO_{\bP^1}(-f-1) \oplus \cO_{\bP^1}(-1).
$$

To provide more detail of the construction, we now summarize the description of $\tX$ in \cite[Section 2]{LY21} in terms of the toric data. Recall from Section \ref{sect:OpenGeometry} that the outer brane $L$ determines cones $\tau_1 \in \Sigma(2) \setminus \Sigma(2)_c$ and $\sigma_1 \in \Sigma(3)$. Now, let
$$
    b_1, \dots, b_R \in \{v \in N \mid \inner{\su_3, v} = 1\} \subset N
$$
be a listing of the primitive generators of the rays in $\Sigma$, labelled in such a way that:
\begin{itemize}
    \item $\tau_1$ is spanned by $\{b_2, b_3\}$;
    \item $\sigma_1$ is spanned by $\{b_1, b_2, b_3\}$;
    \item the orientation on $N$ determined by the basis $\{b_1, b_2, b_3\}$ agrees with the standard orientation.
\end{itemize}
Then, to construct $Y$ from $X$, we add an additional ray spanned by
$$
    b_{R+1} = -b_1 - fb_2 + (f+1)b_3 \quad \in \ker(\su_3) \subset N.
$$
Moreover, we add an additional 3-cone that is spanned by $\{b_2, b_3, b_{R+1}\}$, as well as all the faces of the cone. The resulting fan $\hSi$ is the toric fan of $Y$, and the divisor corresponding to the ray spanned by $b_{R+1}$ is $D = Y \setminus X$. The curve $C \cong \bP^1$ is the orbit $V(\tau_1)$ in $Y$, with
$$
    N_{C/Y} \cong \cO_{\bP^1}(f) \oplus \cO_{\bP^1}(-f-1).
$$
Finally, let $\tN := N \oplus \bZ \tv \cong \bZ^4$. We describe the toric fan $\tSi$ of $\tX$ in $\tN \otimes \bR \cong \bR^4$ by listing all the rays and 4-cones. There are $R+2$ rays in $\tSi$, spanned by $b_1, \dots, b_R$, and
$$
    \tb_{R+1} = b_{R+1} + b_3 + \tv, \qquad \tb_{R+2} = b_3 + \tv
$$
respectively. Extending $\su_3: N \to \bZ$ to $\tN$ by setting $\inner{\su_3,\tv} = 0$, we see that $\su_3$ pairs to 1 with all the primitive generators above, which is equivalent to that $\tX$ is Calabi-Yau. Each 3-cone $\sigma \in \Sigma(3)$ gives rise to a 4-cone in $\tSi$ that is spanned by $\sigma$ and $\tb_{R+2}$. Moreover, there is an additional $4$-cone spanned by $\{b_2, b_3, \tb_{R+1}, \tb_{R+2}\}$. The fan $\tSi$ consists of the 4-cones above and all their faces.


It follows from the construction that there is an isomorphism
$$
    \iota: H_2(X,L; \bZ) \to H_2(Y;\bZ) \to H_2(\tX;\bZ).
$$
Moreover, we consider the integral cohomology class
$$
    \tgamma := [\tD][\tD_2] \in H^4(\tX;\bZ)
$$
where $\tD$ (resp. $\tD_2$) is the toric divisor in $\tX$ corresponding to the ray spanned by $\tb_{R+1}$ (resp. $b_2$).



\begin{theorem}[\cite{LY21}]\label{thm:GWCorrespondence}
For the effective class $\tbeta:= \iota(\beta = \beta'+d[B])$, we have
$$
    N^{X, L, f}_{0, \beta, (d)} = N^{\tX}_{0, \tbeta}(\tgamma)
$$
where $N^{\tX}_{0, \tbeta}(\tgamma)$ is the genus-zero, degree-$\tbeta$ \emph{closed Gromov-Witten invariant} of $\tX$ with 1-pointed insertion $\tgamma$.
\end{theorem}

We note that since $\tX$ is non-compact, the invariant $N^{\tX}_{0, \tbeta}(\tgamma)$ is defined by localization with respect to the Calabi-Yau 3-torus $\tT'$ of $\tX$. We refer to \cite[Section 3]{LY21} for additional details.

%The construction of $\tX$ can be described as follows: In the case $s=1$, the FTCY graph $\Gamma$ defines a relative Calabi-Yau 3-fold $(Y,D)$ such that $Y = X \sqcup D$ and $(\hY, \hD)$ is the formal completion of $(Y,D)$ along the toric 1-skeleton $Y^1$. Then $\tX = \Tot(\cO_Y(-D))$. The normal bundle of the new $\tT'$-invariant projective line $C$ is
%$$
%    N_{C/\tX} \cong \cO_{\bP^1}(f) \oplus \cO_{\bP^1}(-f-1) \oplus \cO_{\bP^1}(-1).
%$$

\begin{remark}\label{rem:sp} \rm{
When $X$ is semi-projective, in \cite{LY22}, Liu and the author further showed that the disk invariants of $(X, L, f)$ can be identified with the genus-zero Gromov-Witten invariants of the toric Calabi-Yau semi-projective partial compactification of $\tX$, which we denote by $\tX^{\mathrm{sp}}$ here. Depending on the choice of framing $f$, $\tX^{\mathrm{sp}}$ is in general a toric Calabi-Yau 4-\emph{orbifold}. For our present purpose, consider the case where $\tX^{\mathrm{sp}}$ is still smooth. In the process of the partial compactification, the two toric divisors $\tD$ and $\tD_2$ may acquire an additional common torus fixed point in $\tX^{\mathrm{sp}}$. If this does not happen, we can still use the class $[\tD][\tD_2]$ in the open/closed correspondence as above. It this happens, we need to replace the insertion by an \emph{equivariant} cohomology class that restrict trivially to the new fixed point. In fact, the new fixed point corresponds to an Aganagic-Vafa outer brane in $X$ neighboring $L$, with framing depending on $f$. If we still insert the non-equivariant class $[\tD][\tD_2]$ for $\tX^{\mathrm{sp}}$, in localization we obtain contributions from both fixed points in $\tD \cap \tD_2$ which correspond to disk invariants of the two framed outer branes in $X$ respectively with the same boundary winding number.
}\end{remark}

\begin{example}\label{ex:C3} \rm{
Consider the basic example $X = \bC^3$ as in \cite[Section 1.2]{LY21}, \cite[Section 2.6.1]{LY22}. For an outer brane $L$ with framing $f$, we have
$$
    Y = \Tot(\cO_{\bP^1}(f) \oplus \cO_{\bP^1}(-f-1)), \qquad \tX = \Tot(\cO_{\bP^1}(f) \oplus \cO_{\bP^1}(-f-1) \oplus \cO_{\bP^1}(-1)).
$$
Consider the case $f<-1$, where $\tX$ is not semi-projective. The partial compactification produces
$$
    \tX^{\mathrm{sp}} = \Tot(\cO_{\bP(1,1,-f-1)}(f) \oplus \cO_{\bP(1,1,-f-1)}(-1)).
$$
Now we specialize to $f = -2$, where $\tX^{\mathrm{sp}} = \Tot(\cO_{\bP^2}(-2) \oplus \cO_{\bP^2}(-1))$. The construction is illustrated in Figure \ref{fig:ExC3}, where the fan of $X$ is the cone over the triangle on the left and the fan of $\tX^{\mathrm{sp}}$ is the cone over the triangulated 3-dimensional polytope on the right. In this case, the same 4-fold $\tX^{\mathrm{sp}}$ is reached if we start the construction instead with the neighboring outer brane in $X$ corresponding to the 2-cone spanned by $\{b_1, b_2\}$ with framing $1$. The closed invariant of $\tX^{\mathrm{sp}}$ with non-equivariant insertion $[\tD][\tD_2]$ has contribution from the disk invariants of the two framed outer branes. 
}
\end{example}


% Figure environment removed





\subsection{BPS correspondence}\label{sect:OpenClosedBPS}
For Calabi-Yau 4-folds, Klemm-Pandharipande \cite{KP08} expressed integral structures of genus-zero Gromov-Witten invariants by a generalization of the Aspinwall-Morrison multiple covering formula \cite{AM93}. For invariants with 1-pointed insertions, the resummation of \cite{KP08} is
\begin{equation}\label{eqn:KPClosed}
    N^{\tX}_{0, \tbeta}(\tgamma) = \sum_{k \mid \tbeta} \frac{n^{\tX}_{0,\frac{\tbeta}{k}}(\tgamma)}{k^2}.
\end{equation}
Here for $k \in \bZ_{\ge 1}$, we say that $k \mid \tbeta$ if $\frac{\tbeta}{k} \in H_2(\tX;\bZ)$. Note that if $\tbeta = \iota(\beta = \beta' + d[B])$, then $k \mid \tbeta$ if and only if $k \mid \beta, d$. We refer to the coefficient
$$
    n^{\tX}_{0, \tbeta}(\tgamma) \in \bQ
$$
as the genus-zero, degree-$\tbeta$ \emph{closed BPS invariant} of $\tX$.

By a direct comparison of the resummations \eqref{eqn:LMOVOpen}, \eqref{eqn:KPClosed}, we see that Theorem \ref{thm:GWCorrespondence} immediately implies the following open/closed correspondence of BPS invariants.

\begin{theorem}\label{thm:BPSCorrespondence}
For $\tbeta = \iota(\beta = \beta'+d[B])$, we have
$$
    n^{X, L, f}_{0, \beta, (d)} = -n^{\tX}_{0, \tbeta}(\tgamma).
$$
\end{theorem}



\subsection{Integrality of closed BPS invariants}\label{sect:ClosedBPS}

As a result of the integrality of open BPS invariants (Theorem \ref{thm:OpenBPS}), Theorem \ref{thm:BPSCorrespondence} implies the integrality of closed BPS invariants of $\tX$ with 1-pointed insertion $\tgamma$, verifying the conjecture of Klemm-Pandharipande \cite[Conjecture 0]{KP08}.

\begin{corollary}\label{cor:ClosedBPS}
For $\tbeta = \iota(\beta = \beta'+d[B])$, we have
$$
    n^{\tX}_{0, \tbeta}(\tgamma) \in \bZ.
$$
\end{corollary}

As discussed in Section \ref{sect:Intro}, \cite[Conjecture 0]{KP08} has been extensively studied in the literature. For compact Calabi-Yau 4-folds, it has been verified in examples in \cite{KP08,CMT18,Cao20,CMT22,COT22,COT24} and in general by \cite{IP18}. In the non-compact setting, a general proof is yet to be given, and to the best of the author's knowledge the known examples are the following:
\begin{itemize}
    \item Local curve $\Tot(\cL_1 \oplus \cL_2 \oplus \cL_3 \to C)$, where $\cL_1, \cL_2, \cL_3$ are line bundles on a smooth projective curve $C$ with $K_C \cong \cL_1 \otimes \cL_2 \otimes \cL_3$, in the cases
    
    \begin{itemize}
        \item[$\circ$] (\cite{CMT18}) $C$ has genus at least 1;
        
        \item[$\circ$] (\cite{CMT18}) $C = \bP^1$ and the $\cL_i$'s have low degree.  
    \end{itemize}

    \item Local surface $\Tot(\cL_1 \oplus \cL_2 \to S)$, where $\cL_1, \cL_2$ are line bundles on a smooth projective surface $S$ with $K_S \cong \cL_1 \otimes \cL_2$, in the cases
    
    \begin{itemize}
        \item[$\circ$] (\cite{KP08}) $S = \bP^2$, $\cL_1 = \cO_{\bP^2}(-1)$, $\cL_2 = \cO_{\bP^2}(-2)$;
        
        \item[$\circ$] (\cite{KP08,CMT22,CKM22}) $S$ is the Hirzebruch surface $\bP^1 \times \bP^1$, $F_1$, or $F_2$, and $\cL_1, \cL_2$ have certain low degrees;
        
        \item[$\circ$] (\cite{CMT18,CMT22}) $S$ is a toric del Pezzo surface, $\cL_1 = \cO_S$, $\cL_2 = K_S$;
        
        \item[$\circ$] (\cite{CMT18,CMT22}) $S$ is a rational elliptic surface, $\cL_1 = \cO_S$, $\cL_2 = K_S$, and the curve class is primitive;
        
        \item[$\circ$] (\cite{BBvG20,BS23,Schuler24}) $(S \mid D_1 + D_2)$ is a two-component Looijenga pair (see e.g. \cite[Section 2]{BBvG20}), $\cL_1 = \cO_S(-D_1)$, $\cL_2 = \cO_S(-D_2)$.\footnote{This covers certain toric cases above as well as some non-toric cases. We refer to \cite[Section 2]{BBvG20} and the references therein for a classification of the deformation types of Looijenga pairs.}
        
    \end{itemize}

    \item Local 3-fold $\Tot(K_W)$, where $W$ is a smooth projective 3-fold, in the cases
    
    \begin{itemize}
        \item[$\circ$] (\cite{KP08}) $W = \bP^3$;
        
        \item[$\circ$] (\cite{Cao20}) $W$ is a Fano hypersurface in $\bP^4$, and the curve class is the line class;
        
        \item[$\circ$] (\cite{Cao20}) $W = S \times \bP^1$ for a toric del Pezzo surface $S$.
    \end{itemize}

    \item (\cite{COT22,COT24}) $\Tot(T^*\bP^2)$, as an example of a holomorphic symplectic variety.

\end{itemize}

By Corollary \ref{cor:ClosedBPS}, the toric Calabi-Yau 4-folds $\tX$ arising from the open/closed correspondence can be added to the above list of non-compact examples. From the constructions, $\tX = \Tot(K_Y)$ where the (non-compact) toric 3-fold $Y$ may arise from an \emph{arbitrary} toric Calabi-Yau 3-fold $X$ and is not necessarily a local curve or local surface.

\begin{remark} \rm{
Following Remark \ref{rem:sp}, when $X$ is semi-projective and the resulting partial compactification $\tX^{\mathrm{sp}}$ is smooth, Corollary \ref{cor:ClosedBPS} can be directly generalized to $\tX^{\mathrm{sp}}$ with the non-equivariant insertion $[\tD][\tD_2]$, therefore providing yet another collection of examples. As observed in \cite[Section 0.5]{Schuler24}, if $(X, L, f)$ is the open geometry arising from a two-component Looijenga pair $(S \mid D_1 + D_2)$, as constructed in \cite{BBvG20,BS23,vGNS23,Schuler24}, then $\tX^{\mathrm{sp}}$ is deformation equivalent to the local surface $\Tot(\cO_S(-D_1) \oplus \cO_S(-D_2))$ mentioned in the list above.
}    
\end{remark}



