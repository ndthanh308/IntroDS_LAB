\subsection{Meta-market}\label{sec: meta}

% Figure environment removed 

% model mechanism
% The multi-agent simulation can only reflect reality with careful calibration.
% Thus, we build a meta-market model to calibrate the multi-agent system.
% Given the target order stream and 
% market state, we want to build a meta-market that calibrates the multi-agent system 
% and generalizes well to unseen days.

Figure~\ref{fig: model_mech} shows the mechanism that market simulation tries to model:
various factors impact the market participants' behavior which dominates the real market to generate order stream.
In the process,
calibration methods model the impact mechanism to produce behavior parameters;
and executing the multi-agent simulation system based on the behavior parameters embodies the later part.
Our goal is to build a calibration model, named meta-market, to produce behavior parameters.

As mentioned in Section~\ref{sec: ham_sys}, 
many types of agents are designed to equip multi-agent simulation systems with different model assumptions.
Here, we denote $M_h$ as the multi-agent model with an assumption $h$
\begin{equation}
M_h(b)=s,
\label{fig: model_assump}
\end{equation}
where the model $M_h$ generate order stream $s$ based on \emph{behavior parameters} $b$.
Thus, the meta-market should provide the multi-agent system $M_h$ with behavior parameters 
so that the multi-agent model $M_h$ mimics well to reality, 
even upon further interventional studies, i.e., high fidelity. 

In real markets, there are many factors that impact market behaviors,
% which dominate the participants to trade and interact in the real markets, resulting in distinct order streams.
Among the factors, some of them are explicitly identified in economics and characterized by macro indices (i.e., market state indicators), such as CPI and PPI~\cite{AKShareZhaiQuanShuJuAKShare}.
Besides, more implicit factors that directly impact market behaviors exist without being named.

% Challenge and design for implicit factors
To enable search-free calibration,
the meta-market needs to model the impact mechanism shown in Figure~\ref{fig: model_mech}.
The explicit factors can be manually selected, though,
the implicit factors are innumerable, and thus, hard to model.
% The implicit factors are commonly considered to be inexhaustive; it is intractable to handcraft implicit factors.
Here, we observe that implicit factors determine market behaviors that dominate the  target order streams.
% we argue that the order stream contains useful information about the implicit factors.
Thus, we extract features from the target order stream to represent the implicit factors.

The implicit features are extracted from the target order stream in binary ways:
\begin{itemize}
    \item Market factors dominate a price estimation upon a market, which leads to a price trend in order stream. 
    Thus, we extract mid-price series from the order stream to facilitate agents with value assessments,
    by serving as the fundamental price~\cite{lespagnol2018trading} for agents who possess value-based trading strategies;
    \item Order stream is much more informative than merely price trend;
    for example, order volume can reflect the liquidity of a market. 
    To sufficiently exploit information from target order streams,
    we build a data-driven implicit feature extractor for implicit factor representation
    \begin{equation}
        u=p_{\theta_1}(s),
    \label{eq: implicit_fea}
    \end{equation}
    where the extractor $p_{\theta_1}$ extracts order stream $s$ to produce implicit feature $u$. % $s$ denotes the order stream and $u$ is the implicit feature.

\end{itemize}
% motivation for meta

% Based on the features, we aim to model an impact mechanism that leverages both implicit and explicit features to produce behavior parameters, that is,
Given the explicit and implicit features,
we can build an impact mechanism model 
\begin{equation}
K\left(u, x\right)=b,
\label{eq: impact_mech}
\end{equation}
where it produces behavior parameter vector $b$ based on implicit and explicit features $u$ and $x$.

Usually, the two factors contribute differently to market behaviors in real markets -- 
explicit factors tend to collectively characterize participant behaviors over a period,
while implicit factors impact the markets more directly and promptly. 
For example, 
although a market tends to be passive during a period of a high unemployment rate, 
some temporal market irrationalities can still lead to a price increase in stock.

% Figure environment removed 

We involve such an observation in building the impact mechanism model: 
the behavior vector is decided by implicit features while the explicit feature determines the way it decides.
Such a design resembles an analyst -- 
it analyzes and speculates the market behaviors based on the more direct (implicit) factors,
while the market states (explicit factors) decide the way they think. 
This leads to our meta-design of the mechanism model
\begin{equation}
K\left(u, x\right)=q_{\theta_2}(x),
\label{eq: k1}
\end{equation}
where we estimate behavior via a behavior estimator $q_{\theta_2}$;
and the model hyperparameter of the behavior estimator is based on a market state analyzer
\begin{equation}
\theta_2=g_{\omega}(u).
\label{eq: k2}
\end{equation}

We finally overview the meta-market in Figure~\ref{fig: meta_design}~(mid-price extraction is omitted for simplicity). 
Meta-market first employs an implicit feature extractor to extract the implicit feature from order stream.
Then, a behavior estimator $q_{\theta_2}$ produces a behavior vector based on the implicit feature,
during which
a market state analyzer $g_\omega$ considers the explicit feature to decide the model assumption, i.e., hyperparameters, of the behavior estimator. 
The market state analyzer and behavior estimator, overall, constitute the impact mechanism as shown in Figure~\ref{fig: model_mech}. 




\iffalse

% what is known and what is not (challenge)
As mentioned in Section~\ref{sec: ham_sys}, 
the heterogeneous multi-agent market simulation has been well-studied --
many types of agents are designed to equip multi-agent simulation systems with different model assumptions.
We denote $M_h$ as a multi-agent model with an assumption $h$. 
Thus, our task is to model the impact mechanism, 
which leverages both explicit and implicit factors to determine agent behaviors,
so that the multi-agent model $M_h$ mimics well to reality, 
even upon further interventional studies, i.e., high fidelity. 

We model the complex impact mechanism $K$ by  a neural network.
However, it would be hard for the model to generalize if we jointly consider the two factors as the inputs.
This is because the two factors usually contribute differently to market behaviors in real markets -- 
explicit factors tend to collectively characterize participant behaviors over a period,
while implicit factors impact the markets more directly and promptly. 
For example, a market tends to be passive during a period of a high unemployment rate, 
though, some temporal market irrationalities can still lead to a price increase in stock.
% Thus, naively treating the two features as inputs may render the deep mechanism model $K$ hard to generalize.

\fi