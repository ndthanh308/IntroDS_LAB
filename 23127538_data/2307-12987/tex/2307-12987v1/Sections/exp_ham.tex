\subsection{Multi-agent System Building}\label{sec: exp_ham}

% Discrete system / platform
We build a discrete multi-agent system to verify the performance of \sysname{}.
In the system, the orders are made and executed collectively by time slot -- 
each agent and the matching algorithm of the orderbook only wake up once in one slot.
During a wake-up, an agent place one order and, if needed, any instructions on order cancellation,
and the orderbook executes all agent requests at the end of each time slot.
Note that we do not consider short-selling in this experiment.
We implement the system in Python.

% System setting
We designate the time slot to be one second, that is,
the orderbook makes 14400 executions on agent requests in a trading day with four exchange hours.
In a time slot, each agent decides to make an order or not.
This models the order-placing interval of a trader. 
As it is not the focus of the paper, 
we set the wake-up of agents to be independent and probabilistic:
the wake-up probability of each agent in each time slot is $1\%$.
To guarantee the heterogeneity of the agents, 
we involve 500 agents in the system.
% Agent commonality
Addtionally, each agent has an investment horizon: the agent will cancel its orders if the orders' lifecycles exceed the agent's horizon~(not being executed by the orderbook).
To unify the system, we use time reference~$\tau$ to serve as the reference horizon of the system.

We build the agents based on the work in~\cite{chiarella2009impact} --
the price and size of the order are correlated by CARA utility function~\cite{babcock1993risk}
% \begin{equation}
% U(w,\alpha) = -e^{-\alpha w},
% \label{eq: cara}
% \end{equation}
% where $w$ is the total assets of the agent and $\alpha$ represents risk aversion.
% The utility function denotes that the consequent high risk makes the high profit less attractive, as the expected profit increases.
which denotes that the consequent high risk makes the high profit less attractive, as the expected profit increases.
Based on the understanding, % the price and size of the order are dependent by 
the size of the order is determined by its price
\begin{equation}
\pi(P)=\frac{\log(\hat{P}_{t+\tau}/P)}{\alpha V_t P},
\label{eq: price_size}
\end{equation}
where $\pi$ is the expected asset holding quantity, $\hat{P}_{t+\tau}$ is the agent's estimated price at time $t+\tau$, and $V_t$ is the variance of historical mid-price.  
% Equation~\ref{eq: price_size} denotes that the order size is decided by price $P$.
We follow the paper and sample the price $P$ from a uniform distribution whose range is bounded by the agent's account~(leverage is not considered in its paper).

Each agent is a composite of three agent types -- fundamentalist, chartist, and noise trader.
Agents mainly differentiate each other by proportions of the agent types, parameterized as $g_f$, $g_c$, and $g_n$, respectively.
The estimated price of each agent is calculated as 
\begin{equation}
P_{t+\tau}=\frac{g_fP_f^{(t)}+g_cP_c^{(t)}+g_nP_n^{(t)}}{g_f+g_c+g_n},
\label{eq: price_est}
\end{equation}
where $P_f^{(t)}$ denotes the estimated price by a fundamentalist at time~$t$ and so do $P_c^{(i)}$ and $P_n^{(i)}$ for the chartist and the noise trader.
The horizon of the $i^{th}$ agent is 
\begin{equation}
\tau^{(i)}=\tau\frac{1+g_f^{(i)}}{1+g_c^{(i)}},
\label{eq: horizon}
\end{equation}
and its risk aversion, similarly, is determined as
\begin{equation}
\alpha^{(i)}=\alpha\frac{1+g_f^{(i)}}{1+g_c^{(i)}},
\label{eq: risk_aversion}
\end{equation}
where the $\alpha$ is the reference risk aversion of the system.

% Specific setting of agents
We initialize each agent's account by sampling from a uniform distribution, 
whose range is related to the asset price~(so that agents can afford to bid and ask).
The different types of agents mainly differentiate by price estimation methods:
\begin{itemize}
    \item The fundamentalist estimates asset price by an exogenous reference denoting a common sense about the true value of the asset, i.e., the fundamental value. 
    In the system, we sample from the mid-price of the simulated order stream as the fundamental value
    and empirically select the sample interval as ten minutes;
    \item The chartist produces price estimation by historical mid-price in the market. 
    In our experiment, we employ the chartist linear regression to estimate asset price, where the chartist looks back upon the 
    historical data length of the agent's horizon;
   \item The noise trader determines asset price by probability. 
   In our setting, we use Gaussian distribution, that is, the estimated price conforms to~$\mathcal{N}(P_t,\sigma^2)$ 
   where~$p_t$ is the current mid-price at time~$t$ and the standard deviation is set to be $1\%$ of the open price of each trading day;
   \item Institutional investor estimates asset prices in the same way as regular one. But 
   its account is abundant with more initial cash and holding of shares;
   and thus, it is likely to place orders in large sizes.
   In deployment, we double the account distribution range against the regular agent to initialize initial holdings of institutional investors.
\end{itemize}

% Our learnt behavior
\emph{Behavior parameters in out experiment:} We sample $g_f$, $g_c$, and $g_n$ from Laplacian distribution, parameterized by $\delta_f$, $\delta_c$ and $\delta_n$.
In the experiments,
we demonstrate \sysname{} by optimizing these parameters and investment horizon unit $\tau$ and the probability of an agent being an institutional investor.
