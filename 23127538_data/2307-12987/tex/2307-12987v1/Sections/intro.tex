
\section{Introduction}\label{sec: intro}

% Explain order steam (task output) 
In order-driven markets, say, the stock market, 
participants interact and compete for maximal profits through order-making, during which 
the order streams are generated. 
% pave the background
Usually, the order stream can be characterized by market states, marked by many crafted economic indicators such as the unemployment rate. 
% what is our task
Market simulation is to mimic the mechanism behind order stream generation, 
which enables many impactful applications in the financial industry, such as trading strategy back-testing~\cite{harvey2015backtesting} and risk management~\cite{gupta2013risk}. 

% multi over single
Literature in market simulation devises agents to generate order streams,
which
can be categorized as single and multi-agent market simulations.
The single-agent market simulation~\cite{coletta2021towards, shi2022state}
use one agent to generate orders, representing the whole market.
%--
%being conceptually simple, though, 
%the approach suffers poor explainability.
The multi-agent simulation emulates real exchange environments 
by employing  heterogeneous agents --
order streams are generated over multi-agent interactions.
Multi-agent simulation provides very details of the trading process,
and thus, is more explainable than its counterpart,
enabling it easy to interpret the simulation results.

% Problem statement
In a multi-agent simulation system, 
each agent represents one~(or a group of) market participants.
Various types of agents have been devised and successfully shown the ability to reflect real market participants, though,
these agents require calibrations on their \emph{behavior parameters}, i.e., the parameters set for the trading strategy, value assessment method, investment horizon, and so on.
Since the real market evolves over time,
one constant set of behavior parameters cannot always faithfully reflect market situations over all periods.
This calls for efficient and careful calibration methods for multi-agent simulation systems.

% Figure environment removed 

% The first challenge: efficiency
One essential goal for the calibration of the multi-agent simulation is to reproduce the target order stream.
Existing works~\cite{bai2022efficient, cao2022dslob} calibrates behavior parameters by heuristic search: 
for each time unit~(such as one trading day), they heuristically search for a set of parameters for the multi-agent system to generate similar order streams to the target.
Although encouraging results have been reported, 
the search process is tedious and time-consuming, making them inefficient for large-scale~(e.g., the year-basis) simulations. 

% calibration redundancy
Additionally, 
due to the randomness and dynamics of the multi-agent system, 
it is possible that several sets of behavior parameters lead to similar order streams.
Failing to consider such calibration redundancy,
the simulations with these parameters, calibrated by current approaches, can perform differently with further interventional studies, 
such as trading strategy back-testing, making their resultant
simulations unstable and arbitrary.

% efficiency
In this paper, 
we target efficient multi-agent system calibration for order stream reproduction in one shot,
based on the observation that the calibration experiences  
share commonalities over time.
Specifically,
since behavior parameters contribute to order stream generation, 
we can capture the behavior information from the target order stream for calibration, retrospectively.

% consistency
To tackle the calibration redundancy,
we argue that the calibration for behavior parameters should observe the patterns in the real market.
In particular, the overall behaviors of the market participants are temporally dependent.
Thus, while targeting order stream reproduction, 
we introduce two additional behavior consistency patterns to reduce the calibration redundancy:
\begin{itemize}
    \item \emph{Temporal consistency:} Without a striking exogenous intervention~(e.g., the Black Swan), 
    the overall behaviors of the market participants should vary smoothly and continuously;
    \item \emph{Market state consistency: } Similar market states usually indicate close trading environments.
    Thus, the overall behaviors of the market participants should be similar under close market states.
\end{itemize}

% system overview
Based on the observations, we propose \sysname{}, an efficient and behavior consistent \emph{\textbf{cali}}bration for multi-agent market \emph{\textbf{sim}}ulation
and overview the system in Figure~\ref{fig: overview}.
% \sysname{} mainly differentiates existing calibration approaches by a data-driven model, named meta-market.
During inference, \sysname{} employs a meta-market that directly generates behavior parameters by analyzing the target order stream with market state indicators,
which avoids the tedious parameter searching process. 
During model training, it leverages a surrogate trading system for order stream reproduction and 
 a consistency loss function to comply with the consistency rules aforementioned.

\sysname{} is an end-to-end system and considers generalization ability in its design. 
The main novelties of our system are listed as follows:
\begin{itemize}
    \item \emph{\textbf{Meta-market design to achieve generalized calibration:}} 
    Although market states characterize participant behaviors, they usually cannot directly determine behaviors due to their macro properties. 
    We consider this by treating the market states as mechanisms, that is, under different market states, we use distinct ways to analyze order streams.
    This leads to our meta-design of the model: 
    the order stream determines behavior parameters, while the market state decides the way it analyzes. 
    Such a meta-design explores the mechanisms behind the input distribution, 
    which enables a generalized calibration;
    
    \item \emph{\textbf{Surrogate trading system for reparameterization:}} 
    The multi-agent system is dynamic, complex, and thus, not differentiable. 
    This poses a big challenge for the end-to-end meta-market to reproduce target order streams.
    We tackle this by re-parameterizing the multi-agent system.
    Specifically,
    we observe that the behavior parameters contribute and correlate to some important features in the generated order stream.
    Thus, we learn a surrogate trading system to capture the correlation between the behavior parameter and the features, 
    constructing a substitution of the multi-agent system to facilitate the meta-market learning.
    
    \item \emph{\textbf{Consistency loss to achieve consistent calibration:}}
    We propose a consistency loss function to achieve the two behavior consistency patterns. 
    For temporal consistency,
    we treat the units as time series and minimize the behavior variations.
    To enforce the market state consistency --
    participants perform similarly under similar market states, given other factors unchanged --
    we devise a contrastive hypothesis scheme: 
    hypothesize real markets with various market states and permutes the behavior parameters contrastively.
\end{itemize}

By accumulating episteme from past calibration experiences, 
\sysname{} enables efficient search-free calibration, 
offering a stable simulation that reflects patterns in real markets.
Additionally, \sysname{} considers and interpolates the market state, 
which not only calibrates but also makes it possible to scenario hypothesis. 
This further extends its usage to more applications, such as stress scenario testing. 

We have implemented and conducted extensive experiments to verify \sysname{} on the orderbook-level data from the A-share market over a year, 
noting that the existing market simulation approaches mostly experiment for up to a few days. 
The experimental results show that \sysname{} can achieve similar reproduction results, 
compared to existing calibration works,
while observing the consistency rules.
We also conduct case studies to show that \sysname{} can capture some patterns in the real market. 

The paper's contributions are summarized as follows:
\begin{itemize}
    \item To the best of our knowledge, \sysname{} is the first searching-free calibration system for multi-agent market simulation;
    \item We first show how to leverage exogenous information, outside the orderbook, to calibrate the multi-agent market simulation system;
    \item We have conducted large-scale (on a year basis) experiments to verify \sysname{} and firstly illustrates scenario hypothesis based on the system.
\end{itemize}

The remainder of this paper is organized as follows.
We first introduce preliminaries in Section~\ref{sec: preliminary}, followed by
system design in Section~\ref{sec: method}, 
where the three novel modules: surrogate trading system (Section~\ref{sec: proxy}), meta-market (Section~\ref{sec: meta}), and consistency loss (Section~\ref{sec: consist_loss}) are included. 
Then, we show the illustrative experimental results in Section~\ref{sec: exp} and 
finally conclude in Section~\ref{sec: conclude}. 
We discuss related works in Section~\ref{sec: related_work}.

\iffalse

In a stock market, participants make either bid or ask orders about an investment object. 
These orders, managed by an orderbook, constitute an order stream over time. 
Usually, the order stream can be characterized by market states, marked by many crafted economic indicators such as the unemployment rate and PMI~\cite{AKShareZhaiQuanShuJuAKShare}. 
Market simulation is to mimic the mechanism behind order stream generation, 
which enables many impactful applications in the financial industry, such as trading strategy back-testing~\cite{harvey2015backtesting} and risk management~\cite{gupta2013risk}. 

The order generation mechanism is the market participants' behaviors 
-- participants may possess different trading strategies, value assessment methods, investment horizons, and so on. 
Literature in market simulation usually leverages proxies, representing participants, to interact with the orderbook, 
which can be categorized as the single and multi-agent market simulation by proxy duties. 
The single-agent market simulation~\cite{coletta2021towards, shi2022state} models an aggregated behavior via one consolidated proxy to make orders,
%which interacts with the orderbook, the so-called ``world agent'';
% being conceptually straightforward, though, the model suffers poor explainability.
which is straightforward but suffers poor explainability.
Multi-agent market simulation~\cite{chiarella2009impact, byrd2019abides} represents each agent as one~(or a group of) market participants with parameterized behaviors; 
each agent strives to maximize its proprietary profits, 
where order streams are generated over their interactions. 
Multi-agent simulation provides very details of trading, such as agent account variations, 
which are essential for users to understand the simulation results.

The model assumptions of multi-agent simulation have been well-studied in economics~\cite{iori2012agent} --
numerous types of agents, such as the fundamentalist and chartist, have shown the ability to reflect parts of the participants and contribute to realistic market dynamics. 
Nevertheless, these agents require calibrations on hyperparameters such as initial accounts and investment horizons;
these hyperparameters, as well as their proportions in markets, are usually referred to as behavior parameters. 

% Figure environment removed 

Since participant behaviors may shift as the real market evolves over time,
one constant parameter set cannot faithfully reflect the real participant behaviors,
and thus, suffers low fidelity. 
Existing calibration works~\cite{bai2022efficient, cao2022dslob} leverage order stream reproduction to evaluate fidelity:
for each unit~(say, one day), they heuristically search for a set of parameters that generate similar order streams to the target.
However, due to the system dynamics and randomness, it is possible that several parameter sets lead to similar order streams -- 
reproduction is insufficient to indicate high fidelity.
This is important in market simulation, because, without careful calibration, the found parameters could deviate from reality, 
which may lead to a false derivation and jeopardize further interventional studies based on the simulation system.

We argue that calibration by considering market regularities can push simulation one step further toward high fidelity. 
% we observe that the real participants' behaviors are dependent over time.
% Thus, we argue that achieving behavior consistency can push simulation one step further toward high fidelity.
In this paper, we calibrate behavior parameters based on the observation that participants' behaviors in the real market
are dependent over time.
% In particular, we consider the following dependency over the behaviors:
We enforce the dependency by considering the two consistencies that calibration should observe
\begin{itemize}
    \item \emph{Temporal consistency:} Usually, the market participants' overall behaviors vary continuously. 
    Thus, while calibrating for behavior parameters that reproduce similar target order streams, 
    we should observe its nature of temporal dependency and achieve minimum behavior variations over time;
    \item \emph{Market state consistency: } Market state indicators reflect market situations and
    characterizes participants' behaviors in all. 
    Thus, we should comply with this rule 
    and consider market states for consistent calibration. 
\end{itemize}

Based on the observations, we propose \sysname{}, a \emph{\textbf{con}}sistent \emph{\textbf{ca}}libration sys\emph{\textbf{t}}em for \emph{\textbf{m}}ulti-\emph{\textbf{a}}gent market \emph{\textbf{s}}imulation.
We overview the system in Figure~\ref{fig: overview}.
\sysname{} mainly differentiates existing calibration approaches by a data-driven model, named meta-market.
During inference, the meta-market directly generates a set of behavior parameters by analyzing the target order stream with market state indicators,
which avoids the tedious parameter searching process. 
During model training, we leverage a surrogate trading system to facilitate calibrating behavior parameters that can reproduce the target order stream.
Additionally, the meta-market observes the two mentioned consistency in calibration,
which is enforced by a proposed consistency loss.
%can determine behavior parameters by considering consistency;
%thus, we propose a consistency loss to achieve consistent calibration.

The main novelties of our system are listed as follows:
\begin{itemize}
    \item \emph{\textbf{Meta-market design to achieve generalized calibration:}} 
    Although market states characterize participant behaviors, they usually cannot directly determine behaviors due to their macro properties. 
    We consider this by treating the market states as mechanisms, that is, under different market states, we use distinct ways to analyze order streams.
    This leads to our meta-design of the model: 
    the order stream determines behavior parameters, 
    while the market state decides how we analyze the order stream. 
    Such a meta-design explores the mechanisms behind the input distribution, 
    which enables a generalized calibration;
    
    \item \emph{\textbf{Surrogate trading system for reparameterization:}} 
    The multi-agent system is dynamic, complex, and thus, not differentiable. 
    This poses a big challenge for us to optimize the meta-market to reproduce target order streams in an end-to-end manner.
    We tackle this by an observation that the behavior parameters contribute to some important features in the generated order stream, 
    which can serve as a substitution to supervise meta-market learning. 
    This enables us to re-parameterize the multi-agent simulation by building a surrogate trading system 
    and using the system to guide the meta-market for order stream reproduction;
    
    \item \emph{\textbf{Consistency loss to achieve consistent calibration:}}
    We propose temporal and market state consistency loss, correspondingly, to achieve our goal aforementioned. 
    The end-to-end training enables easy enforcement of temporal consistency: treating the units as time series and minimizing the behavior variations.
    To enforce the market state as a mechanism and achieve its consistency, we devise a contrastive hypothesis scheme: 
    participants perform similarly under similar market states, given other factors unchanged.
    We propose market state consistency loss based on the scheme, 
    which hypothesizes real markets with various market states and permutes the behavior parameters contrastively.
\end{itemize}

By accumulating episteme from historical data, \sysname{} enables a painless searching-free calibration, 
observing the basic market rule of behavior consistency, which sheds a light on high simulation fidelity.
Additionally, \sysname{} introduces the exogenous information from the market states, 
which not only calibrates but also makes it possible to scenario hypothesis. 
This further extends its usage to more applications, such as stress scenario testing. 
We have implemented and conducted extensive experiments to verify \sysname{} on the orderbook-level data from the A-share market over a year, 
noting that existing market simulation approaches mostly experiment for up to a few days. 
The experimental results show that \sysname{} can achieve similar reproduction results, 
compared to existing calibration works,
while observing the consistency rules.
We also conduct case studies on scenario hypotheses to show \sysname{}'s generalization ability. 

Overall, the paper's contributions are summarized as follows:
\begin{itemize}
    \item To the best of our knowledge, \sysname{} is the first searching-free calibration system for multi-agent market simulation;
    \item We first show how to leverage exogenous information, outside the orderbook, to calibrate the multi-agent market simulation system;
    \item We have conducted large-scale (on a year basis) experiments to verify \sysname{} and firstly illustrates scenario hypothesis based on the system.
\end{itemize}

The remainder of this paper is organized as follows.
We first introduce preliminaries in Section~\ref{sec: preliminary}, followed by
system design in Section~\ref{sec: method}, 
where the three novel modules: surrogate trading system (Section~\ref{sec: proxy}), meta-market (Section~\ref{sec: meta}), and consistency loss (Section~\ref{sec: consist_loss}) are included. 
Then, we show the illustrative experimental results in Section~\ref{sec: exp} and 
finally conclude in Section~\ref{sec: conclude}. 
We discuss related works in Section~\ref{sec: related_work}.

\fi











