\subsection{Experimental Settings}\label{sec: exp_set}

% Figure environment removed

% Overview: two verifications
We verify \sysname{} by two experiments: market replay and case study on the market state.
In the market replay, 
we evaluate \sysname{} by order stream reproduction, measured by the \emph{reconstruction error}, 
and compare \emph{behavior variation} to show that \sysname{} observe the temporal consistency rule.  
%we demonstrate \sysname{} that analyzes the simulated order stream and produce behavior parameters for the multi-agent simulation system~(in Section~\ref{sec: exp_ham}) to reproduce the order stream.
In the case study, % we conduct case studies to show the \sysname{}'s ability to reflect some regularities in the real market. 
we justify that the meta-market, trained by the market state consistency loss function, 
is able to capture some real market patterns. 
% by the correlation coefficients between the market states and behavior  parameters and

% Dataset description
We experiment with orderbook-level data of a stock (WindCode: 000001) from the A-share market in the whole year of 2020. 
We treat a trading day as a unit during which the behavior parameters remain unchanged.
In detail, we use data from the first nine months for model training and the left three months for test. 
To represent the market state, we select five well-studied indicators: 
consumer price index~(CPI), producer price index~(PPI), purchasing manager's index~(PMI)~(available in~\cite{AKShareZhaiQuanShuJuAKShare}),
market trend, and market noise --
among them, CPI, PPI and PMI are monthly updated; and,
we calculate the market trend based on average true range~(ATR) and market noise using efficiency ratio~(ER) over a month. 

% Comparison scheme
We compare \sysname{} with the following schemes:
\begin{itemize}
    \item \emph{Random search~(RandSearch)}~\cite{cao2022dslob} arbitrarily samples behaviors in parameter space  and selects the best parameter combination that reproduces the target order stream.  
    In the implementation, we try ten combinations for each trading day;
    \item \emph{Bayesian optimization~(BayesianOPT)}~\cite{bai2022efficient} searches for behavior parameters in each step
    by referring  to its historical steps. 
    In the implementation, we conduct Bayesian optimization based on the Gaussian process and search in ten steps for each trading day.
    
\end{itemize}

% Training data of proxy model and meta-market input
We train the surrogate model by collecting a dataset using the built multi-agent simulation system --
for each trading day, we randomly select ten sets of behavior parameters to generate order streams.
The selected behavior parameters, as well as the fundamental value of the day, will be the surrogate model input.
%Instead of using the raw order stream as output, 
For the output of the surrogate system,
we extract features~(as the feature extractor $Q$ in Section~\ref{sec: proxy}) from the order stream in four aspects:
% We label training data for the surrogate model and evaluate market replay in four aspects: 
\begin{itemize}
    \item \emph{Mid-price return distribution} reflects price trend in micro view. The return at time $t$ is calculated as 
    \begin{equation}
        r_t=\log\left(\frac{P_t}{P_{t-\Delta t}}\right),
    \label{eq: return_dist}
    \end{equation}
    where use minutely return and extract  from return distribution of these features:
    the gain-loss ratio, kurtosis, and the ratio of zero return;
    \item \emph{Volatility clustering } reflects how intensive mid-price volatility clusters in time. 
    It is usually calculated by autoregression function~\cite{vyetrenko2020get}: $corr(r^2_t, r^2_{t+n\Delta t})$.
    Here, $\Delta t$ is one minute and % evaluate the generated order stream with n from 1 to 10.
    we use $n=1, 2, 3$ and the average of the ten values as the surrogate model output. 
    \item \emph{Order size distribution } is the distribution of the limit order sizes within a trading day.
    In the experiment, we consider order sizes that are less than 100 lots  and use, 
    % We evaluate the approaches by order size distribution and supervise the surrogate model training by using labels as
    the ratio of orders whose sizes are less than 1, 5, 10, and 50, as the surrogate model output;
    \item \emph{Order price from current mid-price} denotes the price gap between the price of a new order 
    and the current mid-price.
    We evaluate its distribution from one to ten tick sizes. 
    and extract 
    the ratio of orders whose price deviates not more than one and five ticks from its current mid-price.
\end{itemize}

% Evaluation method
We compare and evaluate our approach through the following metrics:
\begin{itemize}
    \item \emph{Reconstruction error:} We sum the normalized MSE of the four mentioned aspects 
    to evaluate the reproduction quality between the generated and target order stream,
    where we use the z-score for normalization;
    \item \emph{Behavior variation: } We sum the MSE of all behaviors over two consecutive trading days
    and refer it as behavior variation. 
    We also use the z-score to balance the different behavior parameters.
    
\end{itemize}

% Hyperparameters / extracted features
Since the behavior parameters are temporally correlated, we extract the implicit feature by LSTM from order streams of a month.%.
To facilitate model learning, we extract some features, like the label of the surrogate model, from the order stream of each trading day as the input of the meta-market. 
In model building, we set the LSTM to two layers with two hidden layers. 
For the market state analyzer $g_\omega$, we use two fully connected (FC) layers with neurons of 200 and 100, respectively.
For the behavior estimator, we use one FC layer to connect to transform the analyzed feature to behavior parameters.
To balance the input of the surrogate model, 
we first reduce the dimension of the fundamental value from 24 to 4 by one FC layer with sizes of 50
and employ three fully-connected layers with 50, 100, and 50 to reparameterize the multi-agent system. 
We use ReLU to provide non-linearity and employ Adam Optimizer for training.
Finally, we empirically set $w_t=0.1$ and $w_s=1$ in Equation~\ref{eq: loss_overall}.

% Figure environment removed

% Figure environment removed 

% Figure environment removed 

\begin{table}
    \centering
        \caption{Correlation analysis between market state indicators and behavior parameters }
    \begin{tabular}{cccc}
        \hline
        % ~ & \makebox[0.080\textwidth][c]{RandSearch} & \makebox[0.1\textwidth][c]{BayesianOPT} & \makebox[0.11\textwidth][c]{CONCAT-MAS} & 

         ~     & RandSearch   & BayesianOPT & \textbf{\sysname{}} \\
         \hline
        CPI           &0.0764   & 0.0447    & 0.2555\\
        % \hline
        PPI           & 0.0790  & 0.0513    & 0.2595\\
        % \hline
        PMI           & 0.0494  & 0.0549   & 0.2634 \\
        % \hline
        Market Trend  & 0.0466  & 0.0921  &  0.3266 \\
        % \hline
        Market Noise  &  0.0122 &  0.0601  &  0.3102\\
        \hline
\end{tabular}
    \label{tab: correlation}
\end{table}

\iffalse
\begin{table*}
    \centering
    \begin{tabular}{cllllllll}
        \hline
        ~ & Approach   &   Chartist & Noise trader & Time horizon & Institution & Average (abs)\\
        \hline
        ~ & RandSearch	& -0.074138 &	-0.146789 &	-0.042209 &	-0.042300 &	0.076359\\ 
        CPI & BayesianOPT	& 0.020804 &	0.033529 &	0.063465 &	-0.061150 &	0.044737\\
        ~ & \textbf{CONCAT-MAS}	& -0.412967	&  -0.372014 & 0.196702	  &  0.040466&  \textbf{0.255537} \\
        \hline
        ~ &RandSearch	& 0.144844 	&  0.125062 &	0.040222 & 0.005761 &  0.078972 \\
        PPI &BayesianOPT	& -0.158507 &  	-0.010719& 	0.019172 & -0.016685 &  0.051271 \\
        ~ &\textbf{CONCAT-MAS}	& -0.493080 &  	0.302488 &	0.148873 & -0.093546 &  \textbf{0.259497} \\
        \hline
        ~ &RandSearch	& 0.014812 	&  0.052394 	&0.044525 & 0.085922 &  0.049413 \\
        PMI &BayesianOPT	& -0.012622 &  	0.109719 &	0.004437 & 0.092910 &  0.054922 \\
        ~ &\textbf{CONCAT-MAS}	& 0.722647 	&  0.115076 	&-0.138258& 0.077687 &  \textbf{0.263417} \\
        \hline
        ~ &RandSearch	& -0.109374 &  	-0.037863& 	-0.032681 & -0.006319 &  0.046559 \\
       Market Trend &BayesianOPT	& 0.095705 	&  0.165325 	&0.057766 & 0.049510 &  0.092077 \\
        ~ &\textbf{CONCAT-MAS}	& 0.248611 	&  -0.418734 &	-0.345636 & 0.293292 &  \textbf{0.326568} \\
        \hline
        ~ &RandSearch	& -0.006681 &  	-0.009029& 	-0.010050 & -0.023114 &  0.012219 \\
        Market Noise &BayesianOPT	& -0.063053 &  	0.118438 &	0.033763 & 0.025319 &  0.060143 \\
        ~ &\textbf{CONCAT-MAS}	& 0.226373 	&  -0.455059 &	-0.391578 & 0.168119 &  \textbf{0.310282} \\
        \hline
    \end{tabular}
    \caption{Correlation coefficients between market state indicators and behavior parameters. }
    \label{tab: correlation}
\end{table*}

% Figure environment removed
\fi