\section{Related Work}\label{sec: related_work}

With the advancement in data mining~\cite{dogan2021machine} and storage technologies~\cite{chen2022tvconv},
market analysis using orderbook-level data has flourished in the field of financial technology,
such as return prediction~\cite{hou2021stock, hou2022multi, shah2019stock} and LOB recreation~\cite{shi2021lob, shi2021limit}.
Among them, 
market simulation~\cite{assefa2020generating} leverages the orderbook information to explore its generation mechanism, 
which enables many financial applications such as trading strategy-back testing~\cite{dowd2012back, gort2022deep, vezeris2020optimization}, risk management~\cite{gupta2013risk, leo2019machine}, and stress testing~\cite{reinders2023finance}. 

Market simulation mainly leverages proxies, representing participants, to interact with the orderbook, 
which can be categorized as the single and multi-agent market simulation by proxy duties. 
The single-agent market simulation~\cite{coletta2021towards, shi2022state, li2020generating, coletta2022learning} models an aggregated behavior via one consolidated proxy to make orders,
which interacts with the orderbook, the so-called ``world agent'';
being conceptually straightforward, though, the model suffers poor explainability.
% which is straightforward but suffers poor explainability.
Multi-agent market simulation~\cite{liu2021finrl, byrd2019abides, storchan2021learning, karpe2020multi} represents each agent as one~(or a group of) market participants with parameterized behaviors; 
each agent strives to maximize its proprietary profits, 
where order streams are generated over their interactions. 
Multi-agent simulation provides very details of trading, such as agent account variations, 
which are essential for users to understand the simulation results.
% Still requires calibration
Nevertheless, 
the agents of the multi-agent market simulation require calibration on hyperparameters, or behavior parameters, 
to reproduce target order streams.

% The third cut, multi-agent simulation calibration
Most approaches calibrate the multi-agent simulation system through ``try and see'':
try some behavior parameters with the multi-agent system and compare the output order stream in terms of  reproduction to target.
Works in~\cite{chiarella2009impact, cao2022dslob} empirically set behavior parameters by random search;
the approach in~\cite{storchan2021learning} utilizes grid search to reduce the randomness during the parameter search, 
compromising the accuracy of order stream reproduction.
However, these calibration approaches depend on a vast number of trials for each trading day, 
which is tedious and labor-intensive.
% Heuristic
Work in~\cite{bai2022efficient} leverages Bayesian optimization to expedite the parameter searching process,
where each parameter set is dependent on its historical search results.
Even though, 
the search process for each calibration unit is independent, 
which is still inefficient for large-scale applications.
Furthermore, 
the randomness and dynamics of the multi-agent simulation render calibration redundancy
where the simulations may perform differently under interventional studies, and thus, produces unstable and random results.
In comparison,
\sysname{} explores the dependency between trading days and
tackles the calibration redundancy by patterns from the real markets,
which calibrates in one shot and produces behavior parameters that reflect real market patterns.
% In this way, \sysname{} achieves high reproduction accuracy in an end-to-end manner.
% To tackle the calibration redundancy, 







