%%
%% This is file `sample-acmsmall-conf.tex',
%% generated with the docstrip utility.
%%
%% The original source files were:
%%
%% samples.dtx  (with options: `acmsmall-conf')
%% 
%% IMPORTANT NOTICE:
%% 
%% For the copyright see the source file.
%% 
%% Any modified versions of this file must be renamed
%% with new filenames distinct from sample-acmsmall-conf.tex.
%% 
%% For distribution of the original source see the terms
%% for copying and modification in the file samples.dtx.
%% 
%% This generated file may be distributed as long as the
%% original source files, as listed above, are part of the
%% same distribution. (The sources need not necessarily be
%% in the same archive or directory.)
%%
%%
%% Commands for TeXCount
%TC:macro \cite [option:text,text]
%TC:macro \citep [option:text,text]
%TC:macro \citet [option:text,text]
%TC:envir table 0 1
%TC:envir table* 0 1
%TC:envir tabular [ignore] word
%TC:envir displaymath 0 word
%TC:envir math 0 word
%TC:envir comment 0 0
%%
%%
%% The first command in your LaTeX source must be the \documentclass
%% command.
%%
%% For submission and review of your manuscript please change the
%% command to \documentclass[manuscript, screen, review]{acmart}.
%%
%% When submitting camera ready or to TAPS, please change the command
%% to \documentclass[sigconf]{acmart} or whichever template is required
%% for your publication.
%%
%%
% \documentclass[sigconf, anonymous, review]{acmart}
\documentclass[sigconf]{acmart}

%%
%% \BibTeX command to typeset BibTeX logo in the docs
\AtBeginDocument{%
  \providecommand\BibTeX{{%
    Bib\TeX}}}

%% Rights management information.  This information is sent to you
%% when you complete the rights form.  These commands have SAMPLE
%% values in them; it is your responsibility as an author to replace
%% the commands and values with those provided to you when you
%% complete the rights form.
\iffalse

\setcopyright{acmcopyright}
\copyrightyear{2023}
\acmYear{2023}
\acmDOI{XXXXXXX.XXXXXXX}

%% These commands are for a PROCEEDINGS abstract or paper.
\acmConference[Conference acronym 'XX]{Make sure to enter the correct
  conference title from your rights confirmation emai}{June 03--05,
  2018}{Woodstock, NY}
%%
%%  Uncomment \acmBooktitle if the title of the proceedings is different
%%  from ``Proceedings of ...''!
%%
%%\acmBooktitle{Woodstock '18: ACM Symposium on Neural Gaze Detection,
%%  June 03--05, 2018, Woodstock, NY}
\acmPrice{15.00}
\acmISBN{978-1-4503-XXXX-X/18/06}
\fi

%%
%% Submission ID.
%% Use this when submitting an article to a sponsored event. You'll
%% receive a unique submission ID from the organizers
%% of the event, and this ID should be used as the parameter to this command.
%%\acmSubmissionID{123-A56-BU3}

%%
%% For managing citations, it is recommended to use bibliography
%% files in BibTeX format.
%%
%% You can then either use BibTeX with the ACM-Reference-Format style,
%% or BibLaTeX with the acmnumeric or acmauthoryear sytles, that include
%% support for advanced citation of software artefact from the
%% biblatex-software package, also separately available on CTAN.
%%
%% Look at the sample-*-biblatex.tex files for templates showcasing
%% the biblatex styles.
%%

%%
%% The majority of ACM publications use numbered citations and
%% references.  The command \citestyle{authoryear} switches to the
%% "author year" style.
%%
%% If you are preparing content for an event
%% sponsored by ACM SIGGRAPH, you must use the "author year" style of
%% citations and references.
%% Uncommenting
%% the next command will enable that style.
%%\citestyle{acmauthoryear}


%%
%% end of the preamble, start of the body of the document source.
\begin{document}

%%
%% The "title" command has an optional parameter,
%% allowing the author to define a "short title" to be used in page headers.
\title{Efficient Behavior-consistent Calibration for Multi-agent Market Simulation}

%%
%% The "author" command and its associated commands are used to define
%% the authors and their affiliations.
%% Of note is the shared affiliation of the first two authors, and the
%% "authornote" and "authornotemark" commands
%% used to denote shared contribution to the research.
\author{Tianlang He}
% \authornote{Both authors contributed equally to this research.}
\email{theaf@cse.ust.hk}
\orcid{0000-0002-4939-5993}
\author{Keyan Lu}
% \authornotemark[1]
\email{klual@connect.ust.hk}
\affiliation{%
  \institution{The Hong Kong University of Science and Technology}
  \streetaddress{Clear Water Bay}
%  \city{Sai Kung}
%  \state{Kowloon}
  \country{Hong Kong S.A.R.}
 %  \postcode{}
}

\author{Chang Xu}
% \email{chanx@microsoft.com}
\author{Yang Liu}
%\email{yangliu2@microsoft.com}
\author{Weiqing Liu}
\email{{chanx, yangliu2, weiqing.liu}@microsoft.com}
\affiliation{%
  \institution{Microsoft Research Asia}
%   \streetaddress{Clear Water Bay}
  \city{Beijing}
  \country{China}
 %  \postcode{}
}

\author{S.-H. Gary Chan}
\email{gchan@cse.ust.hk}
\affiliation{%
  \institution{The Hong Kong University of Science and Technology}
%   \streetaddress{Clear Water Bay}
%     \city{Sai Kung}
%  \state{Kowloon}
  \country{Hong Kong S.A.R.}
 %  \postcode{}
}
\author{Jiang Bian}
\email{jiang.bian@microsoft.com}
\affiliation{%
  \institution{Microsoft Research Asia}
%   \streetaddress{Clear Water Bay}
  \city{Beijing}
  \country{China}
 %  \postcode{}
}


%%
%% By default, the full list of authors will be used in the page
%% headers. Often, this list is too long, and will overlap
%% other information printed in the page headers. This command allows
%% the author to define a more concise list
%% of authors' names for this purpose.
%\renewcommand{\shortauthors}{Trovato et al.}
\newcommand*{\sysname}{CaliSim}

%%
%% The abstract is a short summary of the work to be presented in the
%% article.
\begin{abstract}

The Fast Reciprocal Square Root Algorithm is a well-established approximation technique consisting of two stages: first, a coarse approximation is obtained by manipulating the bit pattern of the floating point argument using integer instructions, and second, the coarse result is refined through one or more steps, traditionally using Newtonian iteration but alternatively using improved expressions with carefully chosen numerical constants found by other authors. The algorithm was widely used before microprocessors carried built-in hardware support for computing reciprocal square roots. At the time of writing, however, there is in general no hardware acceleration for computing other fixed fractional powers. This paper generalises the algorithm to cater to all rational powers, and to support any polynomial degree(s) in the refinement step(s), and under the assumption of unlimited floating point precision provides a procedure which automatically constructs provably optimal constants in all of these cases. It is also shown that, under certain assumptions, the use of monic refinement polynomials yields results which are much better placed with respect to the cost/accuracy tradeoff than those obtained using general polynomials. Further extensions are also analysed, and several new best approximations are given.

\end{abstract}


%%
%% The code below is generated by the tool at http://dl.acm.org/ccs.cfm.
%% Please copy and paste the code instead of the example below.
%%

\iffalse
\begin{CCSXML}
<ccs2012>
   <concept>
       <concept_id>10010405</concept_id>
       <concept_desc>Applied computing</concept_desc>
       <concept_significance>500</concept_significance>
       </concept>
   <concept>
       <concept_id>10010405.10010481.10010484</concept_id>
       <concept_desc>Applied computing~Decision analysis</concept_desc>
       <concept_significance>300</concept_significance>
       </concept>
   <concept>
       <concept_id>10010405.10010481.10010484.10011817</concept_id>
       <concept_desc>Applied computing~Multi-criterion optimization and decision-making</concept_desc>
       <concept_significance>300</concept_significance>
       </concept>
 </ccs2012>
\end{CCSXML}

\ccsdesc[500]{Applied computing}
\ccsdesc[300]{Applied computing~Decision analysis}
\ccsdesc[300]{Applied computing~Multi-criterion optimization and decision-making}
\fi
%%
%% Keywords. The author(s) should pick words that accurately describe
%% the work being presented. Separate the keywords with commas.
\keywords{Market simulation, multi-agent system optimization, meta-learning}

%% A "teaser" image appears between the author and affiliation
%% information and the body of the document, and typically spans the
%% page.
% \begin{teaserfigure}
%   % Figure removed
%   \caption{Seattle Mariners at Spring Training, 2010.}
%   \Description{Enjoying the baseball game from the third-base
%   seats. Ichiro Suzuki preparing to bat.}
%   \label{fig:teaser}
% \end{teaserfigure}

% \received{20 February 2007}
% \received[revised]{12 March 2009}
% \received[accepted]{5 June 2009}

%%
%% This command processes the author and affiliation and title
%% information and builds the first part of the formatted document.
\maketitle

% Figure environment removed

\section{Introduction}
Automatic 3D reconstruction of clothed humans using image inputs has gained increasing significance due to its potential applications in a wide array of AR/VR scenarios. High-fidelity reconstructions typically depend on sophisticated capture systems, which are developed with dense camera arrays~\cite{collet2015high,joo2015panoptic,joo2018total}, programmable light-stages~\cite{Vlasic2009, guo2019relightables}, and depth sensors~\cite{newcombe2011kinectfusion,DoubleFusion,BodyFusion,dou2016fusion4d,newcombe2015dynamicfusion}. However, stringent capture environments equipped with complex hardware pose significant challenges for consumer-level applications.


In this context, considerable research effort has been dedicated to developing methods that allow for more flexible capture configurations, such as utilizing a few RGB inputs. Among these works, learning implicit functions \cite{iccv2020PIFu, saito2020pifuhd, hong2021stereopifu} has proven effective in achieving highly detailed reconstructions by integrating the advancements of deep neural networks. These methods employ large multi-layer perceptrons (MLPs) to predict the occupancy probability or truncated signed distance function (TSDF) value of every queried 3D point based on its associated local feature, which is extracted from images. They can recover a continuous surface at arbitrary resolutions without topology restrictions.


However, in typical MLP-based implicit networks, the occupancy or TSDF value at each location is solved independently with planar image features, rendering them less capable of addressing challenging cases such as occlusions. Consequently, these methods suffer from generalization and robustness issues, particularly when tackling strong occlusions caused by large motion or multiple interacting humans. 
Some follow-up studies  \cite{zheng2021deepmulticap,zheng2021pamir,huang2020arch} utilize an extra geometric model, SMPL~\cite{Loper2015}, to improve robustness by introducing strong shape priors. 
Their success typically relies on the assumption of geometrical similarity \cite{huang2020arch} between the shape prior and target reconstruction, making them intractable for handling complex cases with loose clothes and sensitive to errors in SMPL model fitting.



%\ping{this paragraph sounds like `TSDF is better than MLP/SMPL, and we use TSDF to solve the problem'. But in Sec 3, we are telling a different story, saying `MLP needs a 3D convolutional encoder'. We need to make these two sections consistent.}\sicong{I think in this paragraph we claim that the TSDF}


%We opt for Trucated Signed Distance Funtion (TSDF) volumetric representations as they are naturally suitable for convolution operations, which have shown remarkable performance for learning hierarchical features on 2D visual perception tasks \cite{SunXLW19}. 
%Meanwhile, TSDF also describes the gradual geometry change around shape surface, which is not reflected by occupancy volume. 

We instead revisit the 3D volumetric representation and resort to 3D convolutional neural networks (CNNs) for feature learning, due to their impressive performance in feature learning and the ability to incorporate spatial context. However, volumetric methods and 3D convolution involve discretization, which might raise concerns regarding whether a discretized volume can preserve subtle geometric details as continuous representations learned in implicit functions. We investigate the relationship between volume resolution and quantization error on synthetic data by converting target mesh objects to TSDF volumes, as shown in Figure~\ref{fig:quantization_error}. We observe that the quantization errors are significantly reduced by increasing volume resolution and become nearly negligible when reaching a relatively high resolution (e.g., 512 or higher). In other words, achieving fine-detailed reconstruction is not supposed to be restricted by the use of volume representations as long as a proper volume resolution is utilized. Therefore, we present a method with high-resolution feature volumes, e.g., 256 and 512, while traditional volumetric methods \cite{varol18_bodynet,gilbert2018volumetric} are often limited to much lower resolutions, such as 32 or 128.



On the other hand, an increase in volume resolution may lead to a cubic growth of memory overhead \cite{8100085}. Reducing memory costs while guaranteeing the granularity of volumetric representations is necessary for pursuing high-quality reconstruction. Thus, we adopt a coarse-to-fine approach and cull away irrelevant voxels to build a sparse high-resolution feature volume. At the coarse level, the network computes an initial TSDF by applying a U-Net with sparse 3D CNN \cite{3DSemanticSegmentationWithSubmanifoldSparseConvNet} on the sparse feature volume, which is carved by a visual hull. Through our experiments, it turns out that more than 95\% of the volume grids are discarded by the visual hull culling, making the sparse 3D CNN efficient. At the fine level, the network focuses on a narrow band near the zero-level set of the initial TSDF and discretizes the narrow band with smaller voxels. By employing this narrow-band culling, we further shrink the sampling space, resulting in a relatively small range of grid numbers (usually 300K--500K in our experiments) even with a high volume resolution of 512. The remaining voxels in the narrow band are associated with features that fuse high-frequency information from the computed normal maps upon the low-frequency shape from the coarse level to compute the TSDF at high resolution. The final mesh is then extracted from the TSDF using the Marching-Cube algorithm ~\cite{Lorensen87marchingcubes}.
% Different from the u-net sturcture to preserve global topology context, we then apply a shallow 3dcnn to compute the final TSDF $D_{final}$ which contain more local geometry detail.




% \ping{this paragraph can be expanded. It is an important contribution and often ignored by other works. stress on the novel idea of regressing blending weights instead of colors}

In addition to geometry, high-quality mesh texture is also a crucial factor contributing to visual appearance. Directly computing a color field in 3D space, as in \cite{iccv2020PIFu}, struggles to capture high-frequency texture details, while the neural radiance field (NeRF) \cite{yu2020pixelnerf} or the DoubleField~\cite{shao2022doublefield} require expensive per-instance optimization and are often unstable for sparse input images. In contrast, we adopt an image-based rendering approach to compute a texture atlas map, which is efficient and widely supported in existing computer graphics tools. 
Specifically, we compute a blending weight at each 3D point on the mesh surface to determine its color as a weighted average of the colors at its image projections. The blending weights can be computed at a relatively coarse resolution, e.g., 512 volume resolution in our case, and leave texture details to the high-resolution images, such as 1K or 2K. Unlike previous methods that generate blurry texturing results under sparse input, our method generalizes well on both synthetic and real data with just a few input views. 
Figure~\ref{fig:teaser} shows two examples reconstructed by our method. Despite the challenging garment, pose, and occlusion, our method recovers faithful shape, normal, and texture on the right.

%with a wide variety of poses and clothing styles, and it is also adaptive to handle input image with arbitrary resolutions.
%\sicong{For this concern we claim that when the resolution of dicretized volume meets certain threshold (which is 256 in our experiment), the quantization error can be neglected.} 



In summary, the main contributions of this paper are as follows:
\begin{itemize}
\vspace{-0.1in}
  \item 
  We revisit the 3D volumetric representation and demonstrate that it can support clothed human reconstruction with equal or even better performance compared to implicit representation. 
  \item 
  We develop a memory and computation-efficient method for high-resolution volumetric reconstruction using sophisticated sparse 3D CNN, coarse-to-fine estimation, and voxel culling by visual hull and narrow bands. 
  \item 
  We introduce a novel method to compute a texture atlas map, which captures rich appearance details from high-resolution input images.
  \item 
  We achieve impressive results on standard benchmark datasets Twindom and MultiHuman, significantly reducing the point-2-surface (P2S) precision to approximately 0.2cm from just six input views, with more than $50\%$ error reduction compared to the state-of-the-art methods, including DoubleField~\cite{shao2022doublefield} and PIFuHD~\cite{saito2020pifuhd}.
\end{itemize}
\section{Preliminaries}\label{sec: preliminary}
This section introduces the fundamentals of multi-agent market model.  
We first overview the market model in Section~\ref{subsec: ham_sys},
and discuss more details about the order book and trading agent in Section~\ref{subsec: ham_var} and Section~\ref{subsec: agent} respectively. 

\subsection{Multi-agent Market Model}\label{subsec: ham_sys}
Multi-agent market model is a complex and dynamic trading simulation system
that generates order flow by modeling the interaction of multiple elements of stock market. 
The element of stock market has different flavors in economics~\cite{lebaron2006agent}. 
In this paper, we follow one of the mainstream viewpoints that 
a market is dominated by different trading strategies (modeled as agent variable) and the fundamental values of the asset~\cite{chiarella2009impact}. 
In other words, the stock price of an asset jointly depends on its fundamental value as well as the traders' behaviors/strategies. 

As such, a multi-agent market model can be regarded as a parameterized generation model, which is shown as 
\begin{equation}
    Pr(x|a, w), 
\end{equation}
where an order flow, denoted as $x$, is generated from a conditional distribution
that depends on the vector of agent variable, denoted as $a$, and the asset's fundamental value, denoted as $w$. 

Each trading agent has a set of variables to determine its ordering decisions.
The decisions of the agents collectively influence the generated order flow. 
The details of agent variable and their trading strategies will be discussed in Section~\ref{subsec: agent}. 

An asset's fundamental value refers to the common sense of the asset's pricing in market. 
In reality, it is usually from the reports of noted financial experts and institutions. 
Since we do not focus on stock pricing or prediction, 
we use the mid-price (see Equation~(\ref{eq: order_book})) upon the target order flow to calculate the fundamental value. 

The multi-agent model has randomness due to agent irrationality. 
In practice, traders are not completely rational such that
they may not strictly follow their trading strategies~\cite{chiarella2009impact, mathew2012genetic}.  
To model this, the agents' trading decisions are designed to be partially rational, 
making the market model a probabilistic model. 

\subsection{Order Book}\label{subsec: ham_var}

% Figure environment removed 

In a multi-agent market model, 
agents interact and trade through an order book, which is illustrated in Figure~\ref{fig: orderbook}. 
Agents making an order to an order book is analogous to placing an item on shelves, 
where the levels of the shelf stands for the asset prices, 
and weight of the item represents the order size (the number of shares). 
The order book matchmakes a pair of bid and ask orders once they make a deal on the asset price. 

We regard the mean of the highest bid and lowest ask prices on an order book as the asset's current price at time instant~$t$, the so-called mid-price $P_t$. 
Centered at the mid-price, the shape of an order book is represented as a vector of the order sizes at numbers of price levels, 
ranging from~$P-n\delta$ to $P+n\delta$, where~$\delta$ is the minimal variation unit of price, and $n$ is often called the depth of an order book. 
Given the above, the state of the order book at time instant $t$, denoted as $x_t$, can be represented as
\begin{equation}
    x_t=\left( P_t, Q_t \right)
    \label{eq: order_book}
\end{equation}
where $Q$ is the shape of the order book, and the dimension of Q is $2n+1$. 
The state of the order book is the result of order flow. 
Therefore, we use the order book's state to represent the order flow. 
For simple discussion, we do not consider short sale, and  
market order will be considered as a special case of limit order~\cite{gould2013limit}. 


\subsection{Trading Agent}\label{subsec: agent}

The main difference among trading agents' trading strategies lies in the evaluation of asset price, the so-called expected price.  
The expected price of an agent directly determines the agent's order, specifically the ordering price and size.  
Intuitively, a trader should be willing to buy/sell more stock shares when the stock price deviates more from its expected price. 

To introduce how the agents evaluate the price of an asset, 
we exemplify the taxonomy in~\cite{chiarella2006asset},
where the expected price of an agent is determined by the three components of fundamentalist ($f$), chartist ($c$) and irrational trader ($n$). 
Formally, denoting $\alpha$ as the proportion of the components and $\hat{P}$ as the expected price of the component, 
the expected price of an agent, indexed by $i$, is given as 
% which regards each agent as a composition of three components that determines how it evaluates the expected price:
\begin{equation}
    \hat{P}_t^{(i)} = \frac{\alpha_w^{(i)}\hat{P}_t^{(w)}+\alpha_c^{(i)}\hat{P}_t^{(c)}+\alpha_n^{(i)}\hat{P}_t^{(n)}   }{\alpha_w^{(i)}+\alpha_c^{(i)}+\alpha_n^{(i)}}.   
\end{equation}


A pure fundamentalist always agrees with the fundamental value of the asset, whose pricing is shown as 
\begin{equation}
    \hat{P}_t^{(w)}=w_t.
\end{equation}
On the other side, a pure chartist estimates price by looking at the historical prices on the market, which is given as 
\begin{equation}
    \hat{P}_t^{(c)}=g\left(P_{t-1}, P_{t-2}, ..., P_{t-\tau_i}\right), 
\end{equation}
where $g(\cdot)$ is usually a regression function, and $\tau_i$ is the horizon of the $i$th agent. 
The irrational trader characterizes the irrationality of real trader, 
whose expected price is sampled from a certain statistical distribution. 
Take Gaussian distribution as an example, the expected price of a pure irrational trader is given as
\begin{equation}
    \hat{P}_t^{(n)}\sim \mathcal{N}\left(P; P_{t-1}, \sigma_n^2\right), 
\end{equation}
where $\sigma_n$ quantifies the degree of irrationality. 

% Figure environment removed 

Given the expected price, 
an agent would make an order that specifies the price, denoted as $p_t$,  and size, denoted as $v_t$, of an order (we omit the index $i$ for conciseness). 
Commonly in behavioral economics, a trader tends to make a larger size of order 
when the order price deviates more from the expected price, 
because a large deviation means a large profit, the so-called CARA utility~\cite{babcock1993risk}. 
Based on this, the relationship between order price and size can be modeled as
\begin{equation}
    v_t=\frac{1}{\beta\Delta P_tp_t}\log\left(\frac{\hat{P}_t}{p_t} \right),
    \label{eq: risk}
\end{equation}
where $\beta$ is the degree of risk aversion, and $\Delta P_t$ is the variance of historical prices~\cite{vyetrenko2020get}. 
In this example, the agent's ordering depends on the proportion of 
the components~$\alpha_i^{(w)}, \alpha_i^{(c)}, \alpha_i^{(n)}$, irrationality $\sigma_n$, risk aversion $\beta_i$, and horizon $\tau_i$. 
These are the variables of each individual agent. 

Multi-agent market model needs the heterogeneity of agents to ensure the volatility of the market~\cite{iori2012agent, liu2020semiglobal}. 
Conversely, if all trading agents behave the same, no deal would be made on the order book, and the simulation becomes pointless.  
On the other hand, since a market model usually involves hundreds of agents and more, 
it is unrealistic to calibrate the variables of each agent. 
Therefore, literature summarizes the variables of all agents in the market model as multiple statistical distributions, 
upon which the variable of each agent is sampled. 
Accordingly, the set of prior parameters of these distributions is called the agent variable of the market model.  
For example, the risk aversion is sampled from a Gaussian distribution as
\begin{equation}
    \beta_i\sim \mathcal{N}\left(\beta; a_\beta, \sigma^2 \right),
\end{equation}
where $a_\beta$ is an element of the agent variable of the market model. 
% Essentially, the adoption of distribution introduces system randomness, 
% which is necessary for the simulation system to provide agent heterogeneity.

Overall, we summarize the agent variable of the model as a vector, which is written as
\begin{equation}
\begin{aligned}
    a&=\left[a_w, a_c, a_n, \sigma_n, a_\beta, a_\tau \right]\\
    &=\left[a_w, a_c, \sigma_n, a_\beta, a_\tau \right], 
\end{aligned}
\end{equation}
where the equality holds by normalizing the proportions of the three components as one. 
Note that, despite we introducing the above agent as an example,  
\sysname{} is also applicable to other types of rule-based agents.  



\iffalse
\subsection{Fintech Multi-agent System}\label{sec: ham_sys}
% Definition
Fintech multi-agent System, specifically referred to as heterogeneous multi-agent system~\cite{lebaron2006agent} in this paper, is a computational model that simulates the actions and interactions of different types of traders.
Even with simple agents, 
it can exhibit complex behavior patterns and provide valuable information about the dynamics of the real-world system which they emulate~\cite{iori2012agent}. 
Such a simulation system usually consists of an orderbook~\cite{gould2013limit} and a group of basic agents. Heterogeneous agents are created by combining the components of basic agents using linear combinations.


The basic agents mainly differ in how they evaluate the price of the asset. 
There are some common types of basic agents:
\begin{itemize}
    % zero-intelligence agent
    \item \emph{Fundamentalist}  (or value-based agent) estimates asset price according to fundamental price, which is an exogenous information that conveys the value of the asset in a moment:
    \begin{equation}
        \hat{P}_f^{(t)}=P_f^{(t)},
    \label{eq: fundamental}
    \end{equation}
    where $P_f$ is the fundamental price at time $t$.

    
    \item \emph{Chartist} estimates asset prices based on historical data on the orderbook.
    For example, chartist in~\cite{chiarella2006asset} evaluates asset price by
    \begin{equation}
        \hat{P}^{(t+1)}_c=P^{(t)}+\frac{1}{\tau}\sum_{T=t-\tau}^{t} \left( P^{(T)}-P^{(T-1)} \right),
    \label{eq: chartist}
    \end{equation}
    where $\tau$ is the investment horizon of the agent and $P^{(t)}$ is the mid-price at time $t$.

    \item \emph{Noise trader} models the irrationality of the market,
    where the estimated price is sampled from a distribution.
    The noise trader in~\cite{chiarella2006asset} samples prices from Gaussian distribution
    \begin{equation}
        \hat{P}_n^{(t)}\sim \mathcal{N}(P^{(t-1)}, \sigma^2),
        \label{eq: noise_trader}
    \end{equation}
    where $\sigma$ is the pre-determined standard deviation. 
\end{itemize}
More types of agents can be found in~\cite{chiarella2009impact}.


The estimated price of a heterogenous agent is calculated by
\begin{equation}
\hat{P}_{t+\tau}=\frac{g_f\hat{P}_f^{(t)}+g_c\hat{P}_c^{(t)}+g_n\hat{P}_n^{(t)}}{g_f+g_c+g_n},
\label{eq: price_est}
\end{equation}
where $f$, $c$, and $n$ refer to fundamentalist, chartist, and noise trader.
The agents differ by the component of being these three characters, i.e.~$g_f$, $g_c,$ and $g_n$.
In this paper, the component of each agent is separately sampled from Laplacian distributions.

% make orders
Agents make buy/sell orders decision according to their esitimated asset prices.
A buy or sell order consists of price and size (a market order is regarded as a limited order with the current price).
In this paper,
each agent calculates the price and size based on CARA utility function~\cite{babcock1993risk}:
\begin{equation}
\pi(P)=\frac{\log(\hat{P}_{t+\tau^{(i)}}/P)}{\alpha^{(i)} V_t P},
\label{eq: price_size}
\end{equation}
where order size $\pi(P)$ is determined by estimated price $\hat{P}_{t+\tau}$, current price $P$, 
risk aversion of the agent $\alpha^{(i)}$, and historical price variance $V_t$.
The $\tau^{(i)}$ denotes the investment horizon of the agent, 
where its order will be canceled after $\tau$ time slots without being executed.
Basically, 
Equation~\ref{eq: price_size} means to buy more at a lower price, and vice versa.
And the price can be sampled from a range that the agent's account can take.
We differentiate the institutional trader~\cite{} from a regular trader by initial cash and stocks in their accounts.
The investment horizon of agent $i$ is
\begin{equation}
\tau^{(i)}=\tau\frac{1+g_f^{(i)}}{1+g_c^{(i)}},
\label{eq: horizon}
\end{equation}
and its risk aversion is calculated by 
\begin{equation}
\alpha^{(i)}=\alpha\frac{1+g_f^{(i)}}{1+g_c^{(i)}},
\label{eq: risk_aversion}
\end{equation}
where $\tau$ and $\alpha$ are the base parameters. The initial cash and account are sampled from the uniform distribution.
More details can be found in~\cite{chiarella2006asset, chiarella2009impact}.



\subsection{Macroscopic Market Information}
% \chang{introduce CPI, PPI, ..., and why they can represent market state, how they impact the calibration, interpret why we need different strategies for differen market state}

Macroscopic market information usually reflects the running condition of the whole market,
effectively characterizing the overall behaviors of investors~\cite{levy1995microscopic}. 
Therefore, it will be useful to integrate such information in calibrating multi-agent simulation systems. 
Here, we list some indices that are commonly adopted to describe market states:

\begin{itemize}
    \item CPI stands for Consumer Price Index, 
    which is a measure of the average change in prices of goods and services over time. 
    A high CPI indicates severe inflation, where investors may become concerned about the future profitability of companies and may sell their stocks, 
    causing the stock market to decline;
    \item PPI measures the average change in prices that producers receive for their goods and services. 
    It is used to track inflation in the early stages of the production process, which can provide insight into the future direction of inflation. 
    PPI can impact the stock market and the behaviors of stock investors because changes in producer prices can affect the profitability of companies;
    \item PMI stands for Purchasing Managers' Index, which is a measure of the economic activity in the manufacturing sector. 
    % It is based on a survey of purchasing managers who provide information on factors such as new orders, production levels, and employment.    
    When PMI is high, it indicates that the manufacturing sector is expanding, which can lead to increased production and potentially higher profits for companies. This can cause investors to become more optimistic and may lead to a rise in the stock market;
    \item Market trend is a technical indicator of market state. In a market whose price forms a strong trend, investor strategies tend to resort to the technical side. In this paper, we calculate market trend by
    \begin{equation}
        MT=\frac{P_t-P_N}{ATR_t},
    \end{equation}
    where~$P_N$ is the averaged asset price of the recent N days and the average true range (ATE) is calculated by 
    \begin{equation}
        ATE_t=\frac{ATE_{t-1}\times (n-1)+TR_t}{n}.
    \end{equation}
    Here, the true range~$TR=\max\{(P_{high}-P_{low}), abs(P_{high}-P_{close}), abs(P_{low}-P_{close})\}$;
    \item Market noise is a technical indicator of price uncertainty. 
    It is usually represented by efficiency ratio
    \begin{equation}
        ER_t=\frac{P_t-P_{t-n}}{\sum_{i=t-n}^t P_i-P_{i-1}}.
    \end{equation}
    
\end{itemize}
More details of macroscopic market information can be found in~\cite{AKShareZhaiQuanShuJuAKShare}.

To guarantee diversity in market states, 
we select five monthly-updated indicators in the experiments:
consumer price index~(CPI), producer price index~(PPI), purchasing manager's index~(PMI),
market trend, and market noise. 
The historical records of CPI, PPI, and PMI can be found in~\cite{AKShareZhaiQuanShuJuAKShare}, 
and the market trend and noise are two well-known indicators that can be calculated separately based on 
average true range~(ATR) and efficiency ratio~(ER) using daily close price. 

\fi

\section{Method} \label{method_hybridaugment}
In this section, we formally define the problem, motivate our work and then present our proposed techniques.


\subsection{Preliminaries}
Let $\mathcal{F}(x;W)$ be an image classification CNN trained on the training set $\mathcal{T}_\text{train} = (x_{i}, y_{i})^{N}_{i=1}$  with $N$ samples, where $x$ and $y$ correspond to images and labels. The clean accuracy (CA) of $\mathcal{F}(x;W)$ is formally defined as its accuracy over a clean test set $\mathcal{T}_\text{test} = (x_{j}, y_{j})^{M}_{j=1}$. Assume two operators ${A}(\cdot)$ and ${C}(c, s)$ that adversarially attacks or corrupts a given set of images with the corruption category $c$ and severity $s$, respectively.  Let $A\mathcal{T}_\text{test}$ and $C\mathcal{T}_\text{test}$ be the adversarially attacked and corrupted versions of $\mathcal{T}_\text{test}$, and let $\mathcal{F}(x;W)$ have a robust accuracy (RA) on $A\mathcal{T}_\text{test}$ and a corruption accuracy (CRA) on $C\mathcal{T}_\text{test}$. 
The aim is to fit $\mathcal{F}(x;W)$ such that the model gains robustness (\ie. increased RA and CRA compared its the baseline version), while retaining (or improving) the clean accuracy of its baseline version trained without robustness concerns.


\noindent \textbf{What we know.} Our work builds on the following crucial observations: i) CNNs favour high-frequency content \cite{wang2020high}, ii) adversaries and corruptions often reside in high-frequency \cite{wang2020towards}, iii) images are dominated by low-frequency \cite{Saikia_2021_ICCV} and iv) models relying on low-frequency components are more robust \cite{li2022robust,wang2020towards}. The robustness-accuracy trade-off is visible; low-frequency reliant models are more robust, but tend to miss out on clean accuracy brought by the high-frequency components. 

\subsection{HybridAugment}
We hypothesize that a \textit{sweet spot} in the robustness-accuracy trade-off can be found. Unlike the \textit{hard} approaches that completely rule out the reliance on high-frequency components (i.e. low-pass filters), we propose to \textit{reduce} the reliance on them. To this end, we adopt a data augmentation approach that aims to diversify $\mathcal{T}_\text{train}$ by an operation $\mathcal{HA(\cdot)}$. Keeping the strong relation intact between labels and low-frequency content (i.e. labels come from low-frequency-component image), we propose to swap high and low-frequency components of images in a batch on-the-fly. Unlike \cite{mukai2022improving}, we \textit{do not} restrict the images to belong to the same class; this diversifies the training distribution even further while preserving the image semantics. We call this basic version of our approach \textit{HybridAugment}, which corresponds to: 
%
\begin{equation} \label{hybrid_augment_paired}
    \mathcal{HA_{P}}(x_{i}, x_{j}) = \mathcal{LF}(x_{i}) + \mathcal{HF}(x_{j})
\end{equation}
%
where $x_{i}$ is the input image and $x_{j}$ is a randomly sampled image from the whole training set, which we simply sample from the mini batch at each training iteration in practice. $\mathcal{HF}$ and $\mathcal{LF}$ operators select the high and low-frequency components of an input image, for which we use:
%
\begin{equation} \label{eq:cutoff}
\begin{split}
    \mathcal{LF}(x) = GaussBlur(x) \\
    \mathcal{HF}(x) = x - \mathcal{LF}(x)
    \end{split}
\end{equation}
%
where $GaussBlur$ is used as a low-pass filter. Note that a similar outcome is possible by using Discrete Fourier Transforms (DFT), swapping the frequency bands and then applying Inverse DFT (IDFT). We find the gaussian blur operation to be faster and better in practice. 


Inspired from \cite{chen2021amplitude}, in addition to the image-pair scheme in Eq.~\ref{hybrid_augment_paired}, we propose a single image variant of \textit{HybridAugment}. In the single image variant, instead of combining two images, $x_i$ and $x_{j}$ are obtained by applying randomly sampled augmentations to a single image. The single image variant $\mathcal{HA_{S}}$ can therefore be defined as 
%
\begin{equation} \label{hybrid_augment_single}
    \mathcal{HA_{S}}(x_{i}) = \mathcal{LF}(Aug(x_{i})) + \mathcal{HF}(\hat{Aug}(x_{i}))
\end{equation}
%
where $Aug$ and $\hat{Aug}$ correspond to two sets of randomly sampled augmentation operations. Note that paired and single versions can work in tandem ($\mathcal{HA_{PS}}$), and actually outperform single or paired image versions. 


\subsection{HybridAugment++}


The frequency analysis is a vast literature, however, two core aspects often stand out; frequency-band analysis (i.e. low, high) and the decomposition of signals into amplitude and phase. \textit{HybridAugment} covers the former and shows competitive results in various benchmarks (see Section \ref{sec:exp_hybridaugment}). The latter is investigated in $\mathcal{APR}$ \cite{chen2021amplitude}, where phase is shown to be the more relevant component for correct classification, and training models based on their phase labels and swapping amplitude components of images randomly lead to more robust models. Note that frequency-band and phase/amplitude discussions are arguably orthogonal, since frequency, phase and amplitude provide distinct characterizations of a signal: intuitively speaking, frequency, phase and amplitude can be seen as the separation of visual patterns in terms of scale, location and significance. 


We hypothesize these two approaches can be complementary; a model reliant on low-frequency and spatial information (i.e. phase) can further improve robustness. Inspired by the successes of cascaded augmentation methods \cite{hendrycks2019augmix,wang2021augmax,calian2022defending}, we unify these two core aspects into a single, hierarchical augmentation method. We refer to this method as \textit{HybridAugment++} and define its paired version as:
%
\begin{equation}
  \mathcal{HA_{P}}^{++}(x_{i}, x_{j}, x_{z}) = \mathcal{APR_{P}}(\mathcal{LF}(x_{i}), x_{z}) + \mathcal{HF}(x_{j})
\end{equation}
%
where $x_{i}$, $x_{j}$ and $x_{z}$ are images sampled from the same batch. Here, $\mathcal{APR_{P}}$~\cite{chen2021amplitude} is defined as
\begin{equation}
    \mathcal{APR_{P}}(x_{i}, x_{z}) = \mathcal{IDFT}(A_{x_{z}} \otimes e^{i. P_{x_{i}}}) \\
\end{equation}
%
where $\otimes$ is element-wise multiplication, $A$ is the amplitude and $P$ is the phase component. Similar to $\mathcal{HA}$ and $\mathcal{APR}$, we also define a single-image version of \textit{HybridAugment++} as
%
\begin{equation}
 \mathcal{HA_{S}}^{++}(x_{i}) = \mathcal{APR_{S}}(\mathcal{LF}(Aug(x_{i}))) + \mathcal{HF}(\hat{Aug}(x_{i}))
\end{equation}
%
where $\mathcal{APR_{S}}$~\cite{chen2021amplitude} is defined as
%
\begin{equation}
\mathcal{APR_{S}}(x_{i}) = \mathcal{IDFT}\left(A_{\bar{Aug}(x_{i})} \otimes e^{i. P_{\overline{Aug}\left(x_{i}\right)}}\right)    
\end{equation}
%
where $Aug$, $\hat{Aug}$, $\bar{Aug}$ and $\overline{Aug}$ are different sets of randomly sampled augmentation operations. Note that we essentially propose a framework; one can use different single and paired image augmentations, either individually or together, and can still achieve competitive results (see ablations in Section \ref{sec:exp_hybridaugment}). There are also other alternatives, such as swapping phase/amplitude first and then performing $\mathcal{HA}$, but we observe poor performance in practice; dividing the phase component into frequency-bands is not interpretable as frequencies of the phase component are not well defined. The pseudo-code of our methods can be found in the supplementary material.




\begin{table}[t]
	\centering
	\caption{Preallocation strategy results with $3$ machines per tool group and $10$ operations per lot}
	\label{tab:table}
	\figspace\scriptsize
	%	\resizebox{15.5cm}{!}{
		\begin{tabular}{|l%r
				cl||rr|rr|rr|rr|}
			%			\hline
			%			&                    &                      & %        &
			%			 \multicolumn{8}{c}{\textbf{M = 9}} \\
			\hline
			& \multicolumn{1}{@{\hspace{-3mm}}c@{\hspace{-3mm}}}{\textbf{9 Machines}}                   &                      & % &
			\multicolumn{2}{r|}{\textbf{70 Operations}}                 & \multicolumn{2}{r|}{\textbf{80 Operations}}                 & \multicolumn{2}{r|}{\textbf{90 Operations}}                 & \multicolumn{2}{r|}{\textbf{100 Operations}}                 \\
			& Size % \multicolumn{2}{c}{\textbf{Parameters}}            
			&        &
			Lot                         & Step                        & Lot                         & Step                        & Lot          & Step         & Lot          & Step         \\
			%			& size              % & setup % idx
			%			                  &         & 0                           & 1                           & 0                           & 1                           & 0            & 1            & 0            & 1            \\
			%			&                    & setup                &         &                             &                             &                             &                             &              &              &              &              \\
			\hline\hline
			\multirow{3}{*}{\textbf{Fixed}}    & \multirow{3}{*}{1} & % \multirow{3}{*}{0/1} &
			Makespan    & 483                         & 428                         & 489                         & 440                         & 486          & 531          & 592          & 553         \\
			&                    & %                     &
			Setup/Batch & 6/12                        & 2/12                        & 5/14                        & 0/13                        & 5/14         & 3/12         & 3/12         & 0/16         \\
			&                    & %                     &
			1\ts{st}/2\ts{nd} Stage & 2/1                         & TO/27                          & 6/2                        & TO/13                          & 11/13         & TO           & TO/78           & TO           \\
			\midrule
			\multirow{6}{*}{\textbf{Flexible}} & \multirow{3}{*}{2} & % \multirow{6}{*}{0}   &
			Makespan    & 483                         & 475                         & 592                         & 592                         & 592          & 539          & 745          & 698          \\
			&                    & %                     &
			Setup/Batch & 2/8                        & 0/9                        & 1/8                        & 1/8                        & 1/10         & 0/11          & 0/12          & 0/15          \\
			&                    & %                     &
			1\ts{st}/2\ts{nd} Stage & 5/1                         & TO                          & TO/114                          & TO/1                          & TO/130           & TO           & TO           & TO          \\
			\cline{2-11}
			%			& & & & & & & & & & &   \\
			& \multirow{3}{*}{3} & %                     &
			Makespan    & 559                         & --                          & 815                         & --                          & 1357 & -- & 1486 & -- \\ % \multicolumn{4}{c|}{\multirow{3}{*}{Assignment issue}}     \\
			&                    & %                     &
			Setup/Batch & 0/8                         & --                          & 0/8                        & --                          & 0/10 & -- & 10/18 & -- \\ %\multicolumn{4}{c|}{}                                      \\
			&                    & %                     &
			1\ts{st}/2\ts{nd} Stage & TO                       & --                          & TO/140                          & --                          & TO/79 & -- & TO & -- \\ %\multicolumn{4}{c|}{}                                      \\
			\midrule
			\multirow{6}{*}{\textbf{Setup}}    & \multirow{3}{*}{2} & % \multirow{6}{*}{1}   &
			Makespan    & 483                         & 475                         & 592                         & 592                         & 592          & 536          & 745          & 683          \\
			&                    & %                     &
			Setup/Batch & 2/8                        & 0/9                        & 1/8                        & 1/8                        & 1/10         & 0/12          & 0/13          & 0/16          \\
			&                    & %                     &
			1\ts{st}/2\ts{nd} Stage & 2/1                        & TO                          & TO/21                          & TO/25                          & TO/22           & TO           & TO/76           & TO           \\
			%			& & & & & & & & & & &   \\
			\cline{2-11}
			& \multirow{3}{*}{3} & %                     &
			Makespan    & \textbf{334}                         & --                          & \textbf{345}                         & --                          & \textbf{434}          & --           & \textbf{555}          & --           \\
			&                    & %                     &
			Setup/Batch & 0/8                         & --                          & 0/8                         & --                          & 0/11          & --           & 0/12          & --           \\
			&                    & %                     &
			1\ts{st}/2\ts{nd} Stage & TO/20                       & --                          & TO/123                          & --                          & TO           & --           & TO/73           & --           \\
			\hline
		\end{tabular}
		%	}
\end{table}
%
We constructed a scalable set of benchmark instances, focusing on sub-routes of
$10$ production operations for two product types from the SMT2020 simulation scenario~\cite{kopp2020smt2020}.
The $10$ operations in both sub-routes are processed by machines
belonging to three tool groups and do thus involve re-entrant flow,
as a lot visits the same tool group multiple times.
Moreover, the operations incorporate batching and specific setups, and machines undergo periodic maintenance operations.
In the following, we concentrate on instances with $9$ machines, i.e., $3$ per
tool group, and gradually increasing number of lots.
Further smaller- and larger-scale instances along with our implementation are
available online.\footref{foo:online}

We ran our experiments with \clingodl\ (version 1.4.0) on an Intel® Core™i7-8650U CPU Dell Latitude 5590 machine under Windows 10, imposing two time limits per run:
the first stage for makespan minimization is aborted at $450$ seconds, in which case the best schedule found so far % (if any) 
is taken as upper bound on the makespan for proceeding to minimize setup and batch violations with 
another $150$ seconds time limit.

Table~\ref{tab:table} reports the quality of best schedules obtained within the time limits for both optimization stages, split into `Makespan' and `Setup/Batch'
values, while two runtimes or `TO' for a timeout, respectively, are given in the
`1\ts{st}/2\ts{nd} Stage' rows, only listing a single `TO' entry in case both stages timed out.
The `Size' column provides the value taken for the constant \lstinline{sub_size},
limiting the number of machines in subgroups to which the operations are preallocated.
For the latter, the `Lot' columns include results with value \lstinline{0} for the constant \lstinline{lot_step}, where a common subgroup takes all operations for a lot, or for value \lstinline{1} in the `Step' columns, leading to their distribution among subgroups.

The `Size' value 1 necessarily leads to a fixed machine assignment, for which the
quality indicators clearly show that the `Step' strategy yields better schedules,
although it incurs more timeouts and thus fewer certain optima because operations on different lots increase the flexibility of execution sequences and thus search complexity.
While flexibility within subgroups by setting their `Size' to 2 or 3 in principle allows for improved schedules, we observe a deterioration due to sharply increasing instantiation size and search effort, as already observed in \cite{ali2023flexible}.
The setup strategy to differentiate operations and machines within subgroups,
activated by changing the constant \lstinline{by_setup},
aims to cut down the scheduling complexity in line with the optimization objectives by reducing the need for setup changes.
This leads to significantly improved schedules with `Size' 3, where the
`Lot' and `Step' preallocation strategies are indifferent and redundant results for the latter are omitted, up to a critical size reached with $100$~operations.

With our preliminary approach~\cite{ali2023flexible}, using a more naive and less feature-rich encoding of either fixed or fully flexible machine assignments, the
threshold at which problem size and combinatorics get prohibitive was reached at less than $50$ operations already.
Despite gearing up to double that size, our benchmark instances still represent small excerpts of the large-scale semiconductor fabs with more than $100$ tool groups and from $242$ to $543$ production operations per lot modeled by~\cite{kopp2020smt2020}.
%
The elevated complexity in comparison to basic settings like the traditional FJSP is mainly due to sophisticated setup and maintenance operations, requiring a detailed analysis of execution sequences on machines for SMSP.
We conjecture that similar scalability limits would also be encountered with MIP or CP encodings, yet the first-order modeling language of ASP with difference logic facilitates rapid prototyping and experimentation.
In fact, our performance evaluation aims to explore the feasibility of search and optimization, in order to come up with strategies for breaking down large SMSP instances into more manageable portions, e.g., focusing on some bottleneck tool groups or re-entrant flow of operations.

% This section will show the experimental results performed by applying the machine assignment strategies mentioned before, with several instances ranging from $30$ to $130$ steps and $6$ to $12$ machines. All experiments are run using an Intel\textsuperscript{\textregistered} Core\texttrademark{} i7-8650U CPU Dell Latitude 5590 machine under Windows 10. Our timeout limit is $600$ seconds, splitted to $450$ seconds for the makespan and $150$ seconds for the setup and batching. 

% We considered three tool groups for all generated instances in which batch processing, time/counter-based maintenance, and setup are considered. For generating the instances, we started with a small instance containing $30$ steps and $6$ machines where each tool group has $2$ machines and then we generate the next instance by adding one more lot, which has $10$ steps. We kept the tool group size till the fixed machine assignment strategy could not reach the optimum within the time limit. We created $3$ parameters \textit{size, idx} and \textit{setup} to activate a specific machine assignment strategy. The size determines the size of a sub-group in each tool group. The $idx$ defines the Job/Step-based indexing of all steps in the same tool group where all steps of the same lot will have the same index if the $idx = 0$ and Hence, they are assigned to the same sub-group/machine. If $idx = 1$, then each step in the tool group will have an identical index. The last parameter setup is to activate the setup strategy or not. If the $setup = 1$, then the setup strategy is applied; if $setup = 0$ then it's not applied.

% % To continue tomorrow isA :)
% Table \ref{tab:table01} shows the results of the instances with $2$ machines in each toll group. The first column refers to the strategy applied for the machine assignment. The second and third columns show the parameters for selecting a particular strategy. The assignment is fully flexible if the \textit{size} is greater than or equal to the number of machines in a tool group. Otherwise, the assignment is partially flexible. In the fourth column, we list our optimization criteria and the time limit for the makespan and setup/batching represented by 1st/2nd call. Each following two consecutive columns illustrate the results of an instance when the Job/Step-based indexing is selected. From the \ref{tab:table01}, we observed that the best-obtained results were achieved by the full flexible assignment in the first three instances and for the last instance, the setup strategy was the best. The fixed/setup strategies terminated within the time limit except for only one case.

% \begin{table}[h]
% 	\centering
% 	\caption{Comparison between the allocation strategies with 2 machines per tool group}
% 	\label{tab:table01}
% %	\resizebox{15.5cm}{!}{
% 		\begin{tabular}{|l%r
% 			cl||rr|rr|rr|rr|}
% 			\hline
% %			&                    &                      &         & \multicolumn{8}{c}{\textbf{M = 6}} \\
% %			\hline
% 			& \textbf{M = 6}                   & %                     &
% 			  & \multicolumn{2}{r|}{\textbf{Instance 01}}                 & \multicolumn{2}{r|}{\textbf{Instance 02}}                 & \multicolumn{2}{r|}{\textbf{Instance 03}}                 & \multicolumn{2}{r|}{\textbf{Instance 04}}                 \\
% 			& Size % \multicolumn{2}{c}{\textbf{Parameters}}            
% 			 &			         & Job                         & Step                        & Job                         & Step                        & Job          & Step         & Job          & Step         \\
% 			\hline
% %			& size               & setup %idx
% %			                  &         & 0                           & 1                           & 0                           & 1                           & 0            & 1            & 0            & 1            \\
% %			&                    & setup                &         &                              &                             &                             &                             &              &              &              &              \\
% 			\hline
% 			\multirow{3}{*}{\textbf{Fixed}}    & \multirow{3}{*}{1} & % \multirow{3}{*}{0/1} &
% 			 Makespan    & 409                         & 353                         & 409                         & 409                         & 525          & 424          & 525          & 493          \\
% 			&                    & %                     &
% 			 Setup/Batch & 5/6                         & 4/6                         & 4/8                         & 4/8                         & 4/9          & 1/9          & 3/11          & 2/10          \\
% 			&                    & %                     &
% 			 1\ts{st}/2\ts{nd}-Call & \textless{}1/\textless{}1 & \textless{}1/\textless{}1 & \textless{}1/\textless{}1 & \textless{}1/\textless{}1 & 31/1         & 137/6        & 37/11          & TO/53           \\
% 			\midrule
% 			\multirow{3}{*}{\textbf{Flexible}} & \multirow{3}{*}{2} & % \multirow{3}{*}{0}   &
% 			 Makespan   & \textbf{233}                         & --                          & \textbf{281}                         & --                          & \textbf{365}          & --           & 587          & --           \\
% 			&                    & %                     &
% 			 Setup/Batch & 0/5                         & --                          & 0/6                         & --                          & 0/8          & --           & 3/9          & --           \\
% 			&                    & %                     &
% 			 1\ts{st}/2\ts{nd}-Call & 7/0                         & --                          & TO/6                          & --                          & TO/83           & --           & TO           & --           \\
% 			\midrule
% 			\multirow{3}{*}{\textbf{Setup}}    & \multirow{3}{*}{2} & % \multirow{3}{*}{1}   &
% 			 Makespan  & 277                         & --                          & 321                         & --                          & 381          & --           & \textbf{419}          & --           \\
% 			&                    & %                     &
% 			 Setup/Batch & 0/4                         & --                          & 0/6                         & --                          & 0/8          & --           & 0/9          & --           \\
% 			&                    & %                     &
% 			 1\ts{st}/2\ts{nd}-Call & \textless{}1/\textless{}1 & --                          & 25/1                         & --                          & TO/12        & --           & TO/122           & -- \\
% 			 \hline
% 		\end{tabular}
% %	}
% \end{table}

% Table~\ref{tab:table02} summarizes the results of the subsequent $4$ instances where each tool group has $3$ machines. In this instances group, we can split the machines into sub-group by setting the \textit{size} parameter to $2$; in that case, we have two sub-groups in each tool group. The experiments demonstrated that the fixed strategy has the same or better performance than the flexible. In addition, the flexible strategy could not find a feasible solution for instances $7$ and $8$ when all machines were in the same group. On the other hand, the setup strategy performed better than the other two strategies when all machines were in one group, in addition to reaching the optimal value of the setup for all instances. 

% \begin{table}[h]
% 	\centering
% 	\caption{Comparison between the allocation strategies with 3 machines per tool group}
% 	\label{tab:table02}
% %	\resizebox{15.5cm}{!}{
% 		\begin{tabular}{|l%r
% 			cl||rr|rr|rr|rr|}
% %			\hline
% %			&                    &                      & %        &
% %			 \multicolumn{8}{c}{\textbf{M = 9}} \\
% 			\hline
% 			& \textbf{M = 9}                   &                      & % &
% 			 \multicolumn{2}{r|}{\textbf{Instance 05}}                 & \multicolumn{2}{r|}{\textbf{Instance 06}}                 & \multicolumn{2}{r|}{\textbf{Instance 07}}                 & \multicolumn{2}{r|}{\textbf{Instance 08}}                 \\
% 			& Size % \multicolumn{2}{c}{\textbf{Parameters}}            
% 			&        &
% 			 Job                         & Step                        & Job                         & Step                        & Job          & Step         & Job          & Step         \\
% %			& size              % & setup % idx
% %			                  &         & 0                           & 1                           & 0                           & 1                           & 0            & 1            & 0            & 1            \\
% %			&                    & setup                &         &                             &                             &                             &                             &              &              &              &              \\
% 			\hline\hline
% 			\multirow{3}{*}{\textbf{Fixed}}    & \multirow{3}{*}{1} & % \multirow{3}{*}{0/1} &
% 			 Makespan    & 525                         & 433                         & 525                         & 452                         & 525          & 521          & 643          & \textbf{559}          \\
% 			&                    & %                     &
% 			 Setup/Batch & 6/13                        & 1/13                        & 5/15                        & 0/14                        & 5/16         & 6/16         & 6/12         & 3/12         \\
% 			&                    & %                     &
% 			 1\ts{st}/2\ts{nd}-Call & 30/3                         & TO/153                          & 24/8                        & TO/63                          & 231/81         & TO           & TO           & TO           \\
% 			\midrule
% 			\multirow{6}{*}{\textbf{Flexible}} & \multirow{3}{*}{2} & % \multirow{6}{*}{0}   &
% 			 Makespan    & 525                         & 475                         & 650                         & 650                         & 650          & 595          & 745          & 742          \\
% 			&                    & %                     &
% 			 Setup/Batch & 2/11                        & 0/11                        & 1/12                        & 1/12                        & 6/13         & 4/14          & 3/17          & n/a          \\
% 			&                    & %                     &
% 			 1\ts{st}/2\ts{nd}-Call & 26/7                         & TO                          & TO/12                          & TO                          & TO           & TO           & TO           & TO           \\
% 			\cline{2-11}
% %			& & & & & & & & & & &   \\
% 			& \multirow{3}{*}{3} & %                     &
% 			 Makespan    & 744                         & --                          & 1206                         & --                          & 1698 & -- & n/a & -- \\ % \multicolumn{4}{c|}{\multirow{3}{*}{Assignment issue}}     \\
% 			&                    & %                     &
% 			 Setup/Batch & 2/12                         & --                          & n/a                        & --                          & 8/15 & -- & n/a & -- \\ %\multicolumn{4}{c|}{}                                      \\
% 			&                    & %                     &
% 			 1\ts{st}/2\ts{nd}-Call & TO                       & --                          & TO                          & --                          & TO & -- & TO & -- \\ %\multicolumn{4}{c|}{}                                      \\
% 			\midrule
% 			\multirow{6}{*}{\textbf{Setup}}    & \multirow{3}{*}{2} & % \multirow{6}{*}{1}   &
% 			 Makespan    & 525                         & 475                         & 650                         & 650                         & 643          & 553          & 745          & 642          \\
% 			&                    & %                     &
% 			 Setup/Batch & 2/11                        & 0/11                        & 1/12                        & 1/12                        & 1/14         & 0/13          & 1/14          & 1/16          \\
% 			&                    & %                     &
% 			 1\ts{st}/2\ts{nd}-Call & 44/2                        & TO                          & TO/4                          & TO/2                          & TO           & TO/7           & TO           & TO           \\
% %			& & & & & & & & & & &   \\
% 			\cline{2-11}
% 			& \multirow{3}{*}{3} & %                     &
% 			 Makespan    & \textbf{346}                         & --                          & \textbf{373}                         & --                          & \textbf{429}          & --           & 820          & --           \\
% 			&                    & %                     &
% 			 Setup/Batch & n/a                         & --                          & n/a                         & --                          & n/a          & --           & n/a          & --           \\
% 			&                    & %                     &
% 			 1\ts{st}/2\ts{nd}-Call & TO                       & --                          & TO                          & --                          & TO           & --           & TO           & --           \\
% 			\hline
% 		\end{tabular}
% %	}
% \end{table}

% Table~\ref{tab:table03} considers $4$ machines in each tool group and the flexible strategy obtained the best result for the first instance. However, it had the same feasibility issue when all machines were in the same group. For the rest instances, the setup strategy dominated when the machines were equally distributed into sub-groups. 

% From the conducted experiments, we can conclude that 
% \begin{itemize}
% 	\item The flexible assignment performed well on the small-scale.
% 	\item While increasing the scale, the setup strategy dominates in the most cases
% 	\item Assigning the steps of the same lot independently with the fixed assignment leads to better performance
% 	\item The Setup strategy has a significant impact in minimizing the setup objective through all instances
% 	\item The full flexible assignment has an assignment issue while increasing the number of machines
% \end{itemize}

% \begin{table}[h]
% 	\centering
% 	\caption{Comparison between the allocation strategies with 4 machines per tool group}
% 	\label{tab:table03}
% %	\resizebox{15.5cm}{!}{%
% 		\begin{tabular}{|l%r
% 			cl||rr|rr|rr|rr|}
% 			\hline
% %			&                    &                      &  &  \multicolumn{8}{c}{\textbf{M = 12}} 
% %			\\ \hline
% 			& \textbf{M = 12}                   & %                     & 
% 			 & \multicolumn{2}{r|}{\textbf{Instance 09}}                 & \multicolumn{2}{r|}{\textbf{Instance 10}}                 & \multicolumn{2}{r|}{\textbf{Instance 11}}                 & \multicolumn{2}{r|}{\textbf{Instance 12}}                 \\
% 			& Size % \multicolumn{2}{l}{\textbf{Parameters}}            
% 			 &			 &			 Job                    & Step                   & Job                    & Step                   & Job                    & Step                   & Job                    & Step                   \\
% %			& Size               & setup % idx
% %			                  &  & 0                      & 1                      & 0                      & 1                      & 0                      & 1                      & 0                      & 1                      \\
% %			&                    & setup                &  &  &                        &                        &                        &                        &                        &                        &                                               \\
% 			\hline\hline
% 			\multirow{3}{*}{\textbf{Fixed}}    & \multirow{3}{*}{1} & % \multirow{3}{*}{0/1} &
% 			 Makespan                 & 525                    & 453                    & 525                    & 452                    & 525                    & 493                    & 643                    & 561                    \\
% 			&                    & %                     &
% 			 Setup/Batch              & 7/19                   & 3/20                   & 7/20                  & n/a                   & 6/22                   & 4/20                   & 4/22                   & n/a                   \\
% 			&                    & %                     &
% 			 1\ts{st}/2\ts{nd}-Call              & 124/5                 & TO & 25/17                 & TO & 25/53                 & TO/142 & TO & TO \\
% 			\midrule
% 			\multirow{9}{*}{\textbf{Flexible}} & \multirow{3}{*}{2} & % \multirow{9}{*}{0}   &
% 			 Makespan                 & \textbf{373}                    & 503                    & 491                    & 778                    & 569                    & 569                    & 765                    & 1673                   \\
% 			&                    & %                     &
% 			 Setup/Batch              & n/a                    & 6/17                    & n/a                   & n/a                    & n/a                    & n/a                   & n/a                    & 12/24                  \\
% 			&                    & %                     &
% 			 1\ts{st}/2\ts{nd}-Call              & TO & TO & TO & TO & TO & TO & TO & TO \\
% 			\cline{2-11}
% %			& & & & & & & & & & &   \\
% 			& \multirow{3}{*}{3} & %                     &
% 			 Makespan                 & 709                    & 688                    & 800                    & 907                    & 876                    & 876                    & 905                    & 1643                   \\
% 			&                    & %                     &
% 			 Setup/Batch              & 5/17                    & n/a                   & 3/18                   & 5/19                   & n/a                   & n/a                   & n/a                  & 15/24                    \\
% 			&                    & %                     &
% 			 1st/2nd              & TO & TO & TO & TO & TO & TO & TO & TO \\
% 			\cline{2-11}
% %			& & & & & & & & & & &   \\
% 			& \multirow{3}{*}{4} & %                     &
% 			 Makespan                 & n/a & -- & n/a & -- & n/a & -- & n/a & -- \\ %\multicolumn{8}{c|}{\multirow{3}{*}{Assignment issue}}                                                                                                                                                 \\
% 			&                    & %                     &
% 			 Setup/Batch              & n/a & -- & n/a & -- & n/a & -- & n/a & -- \\ %\multicolumn{8}{c|}{}                                                                                                                                                                                  \\
% 			&                    & %                     &
% 			 1\ts{st}/2\ts{nd}-Call              & TO & -- & TO & -- & TO & -- & TO & -- \\ %\multicolumn{8}{c|}{}                                                                                                                                                                                  \\
% 			\midrule
% 			\multirow{9}{*}{\textbf{Setup}}    & \multirow{3}{*}{2} & % \multirow{9}{*}{1}   &
% 			 Makespan                 & 401                    & 396                    & 419                    & \textbf{416}                    & \textbf{419}                    & \textbf{419}                    & \textbf{457}                    & 471                    \\
% 			&                    & %                     &
% 			 Setup/Batch              & 0/15                   & 0/14                   & 0/16                   & 0/16                   & n/a                   & n/a                   & 0/21                    & n/a                    \\
% 			&                    & %                     &
% 			 1\ts{st}/2\ts{nd}-Call              & TO & TO/92 & TO & TO & TO & TO & TO & TO \\
% 			\cline{2-11}
% %			& & & & & & & & & & &   \\
% 			& \multirow{3}{*}{3} & %                     &
% 			 Makespan                 & 706                    & 642                    & 792                    & 753                    & 942                    & 942                    & 939                    & 894                    \\
% 			&                    & %                     &
% 			 Setup/Batch              & 1/14                    & n/a                    & 2/16                    & n/a                   & n/a                   & n/a                    & n/a                    & 1/22                    \\
% 			&                    & %                     &
% 			 1\ts{st}/2\ts{nd}-Call              & TO & TO & TO & TO & TO & TO & TO & TO \\
% 			\cline{2-11}
% %			& & & & & & & & & & &   \\
% 			& \multirow{3}{*}{4} & %                     &
% 			 Makespan                 & 679                    & -- & 1725                    & -- & n/a                    & -- & n/a                    & -- \\
% 			&                    & %                     &
% 			 Setup/Batch              & n/a                   & -- & n/a                    & -- & n/a                   & -- & n/a                   & -- \\
% 			&                    & %                     &
% 			 1st/2nd              & TO & -- & TO & -- & TO & -- & TO & -- \\
% 			\hline
% 		\end{tabular}%
% %	}
% \end{table}
\section{Conclusions}
In this paper we developed WOOD, a general-purpose multi-modal and weakly-supervised OOD detection framework that combines contrastive learning and a binary classifier in a weakly-supervised fashion. To achieve this, we employ  Hinge loss in contrastive learning to maximize the difference in similarity scores between ID and OOD pairs. In addition, we introduced a Feature Sparsity Regularizer to combine important features from the two data modalities in the binary classifier. We also incorporated a new scoring metric to fuse the prediction results from both components. The evaluation results demonstrated that the integration of the binary classifier and contrastive learning can achieve high accuracy in detecting all three anomaly scenarios. 

\section{Related Work}
%\subsection{Cost Volume based Deep Stereo Matching}
%Stereo matching is a typical problem that has been studied for decades and a well-known four-step pipeline \cite{scharstein2002taxonomy} has been established, where cost volume construction is an indispensable step. Current state-of-the-art stereo matching methods are all cost volume based methods and they can be categorized into two types. Typically, a cost volume is a 4D tensor of height, width, disparity, and features. The first category just uses a full correlation to generate a single-feature cost volume. Such methods are usually efficient but lose much information because of the decimation of feature channels. Many previous work, including Dispnet \cite{dispnet}, MADNet \cite{madnet}, IResNet \cite{iresnet} and AANet \cite{aanet}, belong to this category. The second category usually uses concatenation \cite{gcnet} or group-wise correlation \cite{gwcnet} to generate a multi-feature 4D cost volume. Such a method can achieve better performance while requiring higher computational complexity and memory consumption. Actually, a majority of the top-performing networks in public leaderboards belong to this category, such as GANet \cite{ganet}, CSPN \cite{cspn} and ACFNet \cite{acfnet}. These methods generally employ multiple 3D convolution layers to constantly regularize the 4D cost volume and then apply softmax over the disparity dimension to produce a discrete disparity probability distribution. The final predicted disparity is obtained by softly weighting indices according to their probability, which is also called soft argmin in GCNet \cite{gcnet}. However, soft argmin leaves the output susceptible to multi-modal disparity probability distributions. ACFNet \cite{acfnet} observes this problem and proposes to directly supervise the cost volume with unimodal ground truth distributions. In contrast, we define an uncertainty estimation to quantify the degree to which the cost volume tends to be multi-modal distribution, higher implies the higher possibility of estimation error.

\subsection{Multi-scale Cost Volume based Stereo Matching}
Cost volume construction is an indispensable step in the well-known four-step pipeline for stereo matching \cite{scharstein2002taxonomy, pamisurvey1, pamisurvey2}. Typically, current state-of-the-art stereo matching methods can be categorized into two types of cost volume-based methods, where the cost volume is a 4D tensor of height, width, disparity, and features. The first category usually uses the single-feature 3D cost volume generated by full correlation, which is efficient while losing much information due to the decimation of feature channels. Many real-time methods, such as Dispnet \cite{dispnet}, MADNet \cite{madnet, madnet_pami} and AANet \cite{aanet}, belongs to the category. Moreover, two-stage refinement \cite{mcvmfc} and pyramidal towers \cite{madnet} are commonly applied in the single-feature cost volume based network to construct multi-scale cost volume. The second category usually uses the multi-feature 4D cost volume generated by concatenation \cite{gcnet} or group-wise correlation \cite{gwcnet}, which can achieve better performance with higher computational complexity and memory consumption. Most top-performing networks, including GANet \cite{ganet}, CSPN \cite{cspn} and ACFNet \cite{acfnet} belong to this category. 
% In these methods, the 4D cost volume is constantly regularized by multiple 3D convolution layers and then a discrete disparity probability distribution can be produced by softmax. Next, the final predicted disparity can be obtained by softly weighting indices according to their probability \cite{gcnet}. However, such output is susceptible to multimodal disparity probability distributions and ACFNet \cite{acfnet} gives a solution by directly supervising the cost volume with unimodal ground truth distributions to alleviate this problem. 
Recently, to alleviate the high computational complexity and memory consumption when employing multi-feature 4D cost volumes, \cite{cvpmvsnet, cascade, uscnet} propose to use cascade cost volume representation in multi-view stereo. These methods usually first predict an initial disparity at the coarsest resolution of the image and then gradually refine the disparity by narrowing down the disparity search space. More closely related to our approach is Casstereo \cite{cascade}, which first extended such representation to stereo matching. It selected to uniform sample a pre-defined range to generate the next stage’s disparity search range. Instead, we employ pixel-level uncertainty estimation to adaptively adjust the next stage disparity searching range and generate pseudo-labels for subsequent domain adaptation. Our method also shares similarities with UCSNet \cite{uscnet}, which constructs uncertainty-aware cost volume in multi-view stereo while it doesn’t employ uncertainty estimation to generate pseudo-labels.

%\subsection{Multi-scale Cost Volume based Deep Stereo Matching} 
% \subsection{Multi-scale Cost Volume based Stereo Matching} 
%Multi-scale cost volume firstly was applied in the single-feature cost volume based network with the form of two-stage refinement \cite{mcvmfc} and pyramidal towers \cite{madnet}. Recently, cascade cost volume representation \cite{cvpmvsnet, cascade, uscnet} was proposed in multi-view stereo to alleviate the high computational complexity and memory consumption when employing multi-feature 4D cost volumes. These methods generally predict an initial disparity at the coarsest resolution of the image. Then, they will narrow down the disparity search space and gradually refine the disparity. More closely related to our approach is Casstereo \cite{cascade}, which first extended such representation to stereo matching. It selected to uniform sample a pre-defined range to generate the next stage’s disparity search range. Instead, we employ uncertainty estimation to adaptively adjust the next stage pixel-level disparity searching range and push the next stage's cost volume to be predominantly unimodal.

% The single-feature cost volume based network with the form of two-stage refinement \cite{mcvmfc} and pyramidal towers \cite{madnet} first employ multi-scale cost volume for stereo matching. Recently, to alleviate the high computational complexity and memory consumption when employing multi-feature 4D cost volumes, \cite{cvpmvsnet, cascade, uscnet} propose to use cascade cost volume representation in multi-view stereo, which generally predict an initial disparity at the coarsest resolution of the image. Then, the disparity search space is narrowed down and the disparity is gradually refined. More closely related to our approach is Casstereo \cite{cascade}, which first extended such representation to stereo matching. It selected to uniform sample a pre-defined range to generate the next stage’s disparity search range. Instead, we employ uncertainty estimation to adaptively adjust the next stage pixel-level disparity searching range and push the next stage's cost volume to be predominantly unimodal.

% Figure environment removed

\subsection{Robust Stereo Matching} 
There exist three categories of generalization definitions for robust stereo matching. 1) Cross-domain Generalization: the network’s ability to perform well on unseen scenes (cannot see the image pairs of the target domain in advance). Towards this end, Jia et al \cite{sungeneralizaiton} propose to incorporate scene geometry priors into an end-to-end network. Zhang et al \cite{dsmnet} introduce a domain normalization and a trainable non-local graph-based filter to construct a domain-invariant stereo matching network. 2) Adapt Generalization: the network’s ability to adapt pre-trained models to the new domain with unlabeled target data. Previous work usually pre-trains the models on synthetic data and then adapts it to new target domains with Graph Laplacian regularization \cite{zoom}, non-adversarial progressive color transfer \cite{adastereo}, and Knowledge Reverse Distillation \cite{aohnet}. More closely related to our approach are \cite{aohnet, unsuperviseddomainadaptation} in stereo matching and Monoresmatch \cite{monoresmatch} in monocular depth estimation, which also proposes to generate a pseudo-label for domain adaptation. However, these methods all select to employ classical stereo matching methods \cite{sgm} alongside with confidence estimators, e.g., left-right consistency check to generate pseudo-labels. That is all these methods need an independent method to generate corresponding pseudo-labels. Instead, the proposed method is an end-to-end network that can generate the predicted disparity map, corresponding uncertainty map and pseudo-labels jointly, which is a more simple, yet efficient way. 
% Instead, our proposed method can employ pixel-level and area-level uncertainty estimation to self-distill the predicted disparity maps of our pre-training model and generate sparse while reliable pseudo-labels to align the domain gap, which is a more simple, yet efficient way. 
3) Joint Generalization: the network’s ability to perform well on a variety of datasets with the same model parameters. MCV-MFC \cite{mcvmfc} introduces a two-stage finetuning scheme to achieve a good trade-off between generalization and fitting capability on multiple datasets. However, it doesn’t touch the inner difference between diverse datasets, e.g, the unbalanced disparity distribution. To further address this problem, we propose a cascade cost volume to adaptively the next stage disparity searching space, where the pixel-level uncertainty estimation is at the core.

% \subsection{Monocular Depth Estimation}
% Monocular depth estimation aims to estimate depth values from a single image, instead of stereo images or multiple frames in a video. This problem is ill-posed because of the ambiguity of object sizes. However, humans could estimate the depth from a single image with prior knowledge of the scenes. Recently, learning based methods were explored to learn depth values by supervised or unsupervised learning. Eigen et al. first employed Convolutional Neural Networks (CNN) to predict depth in a coarse-to-fine manner and further improved its performance by multi-task learning. Liu et al. presented deep convolutional neural fields model by combining deep model with continuous CRF. Li et al. [22] refined deep CNN outputs with a hierarchical CRF. Multi-scale continuous CRF was formulated into a deep sequential network by Xu et al. [45] to refine depth estimation. Unsupervised methods tried to train monocular depth estimation with stereo
% image pairs or image sequences and test on single images. Garg et al. [9] used novel image view synthesis loss to train a depth estimation network in an unsupervised way. Godard et al. [11] introduced left-right consistency regularization to improve the performance of view synthesis loss. Recently, some work also propose to use the stereo matching network as a proxy to learn depth from synthetic data or directly employ traditional stereo matching methods to distill proxies labels from the target domain, which proves the feasibility of distilling stereo matching networks to learn monocular depth estimation.





\bibliographystyle{acm}
\bibliography{main}

\end{document}
\endinput
%%
%% End of file `sample-acmsmall-conf.tex'.
