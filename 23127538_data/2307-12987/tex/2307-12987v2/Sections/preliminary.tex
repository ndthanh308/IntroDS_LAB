\section{Preliminaries}\label{sec: preliminary}
This section introduces the fundamentals of multi-agent market model.  
We first overview the market model in Section~\ref{subsec: ham_sys},
and discuss more details about the order book and trading agent in Section~\ref{subsec: ham_var} and Section~\ref{subsec: agent} respectively. 

\subsection{Multi-agent Market Model}\label{subsec: ham_sys}
Multi-agent market model is a complex and dynamic trading simulation system
that generates order flow by modeling the interaction of multiple elements of stock market. 
The element of stock market has different flavors in economics~\cite{lebaron2006agent}. 
In this paper, we follow one of the mainstream viewpoints that 
a market is dominated by different trading strategies (modeled as agent variable) and the fundamental values of the asset~\cite{chiarella2009impact}. 
In other words, the stock price of an asset jointly depends on its fundamental value as well as the traders' behaviors/strategies. 

As such, a multi-agent market model can be regarded as a parameterized generation model, which is shown as 
\begin{equation}
    Pr(x|a, w), 
\end{equation}
where an order flow, denoted as $x$, is generated from a conditional distribution
that depends on the vector of agent variable, denoted as $a$, and the asset's fundamental value, denoted as $w$. 

Each trading agent has a set of variables to determine its ordering decisions.
The decisions of the agents collectively influence the generated order flow. 
The details of agent variable and their trading strategies will be discussed in Section~\ref{subsec: agent}. 

An asset's fundamental value refers to the common sense of the asset's pricing in market. 
In reality, it is usually from the reports of noted financial experts and institutions. 
Since we do not focus on stock pricing or prediction, 
we use the mid-price (see Equation~(\ref{eq: order_book})) upon the target order flow to calculate the fundamental value. 

The multi-agent model has randomness due to agent irrationality. 
In practice, traders are not completely rational such that
they may not strictly follow their trading strategies~\cite{chiarella2009impact, mathew2012genetic}.  
To model this, the agents' trading decisions are designed to be partially rational, 
making the market model a probabilistic model. 

\subsection{Order Book}\label{subsec: ham_var}

% Figure environment removed 

In a multi-agent market model, 
agents interact and trade through an order book, which is illustrated in Figure~\ref{fig: orderbook}. 
Agents making an order to an order book is analogous to placing an item on shelves, 
where the levels of the shelf stands for the asset prices, 
and weight of the item represents the order size (the number of shares). 
The order book matchmakes a pair of bid and ask orders once they make a deal on the asset price. 

We regard the mean of the highest bid and lowest ask prices on an order book as the asset's current price at time instant~$t$, the so-called mid-price $P_t$. 
Centered at the mid-price, the shape of an order book is represented as a vector of the order sizes at numbers of price levels, 
ranging from~$P-n\delta$ to $P+n\delta$, where~$\delta$ is the minimal variation unit of price, and $n$ is often called the depth of an order book. 
Given the above, the state of the order book at time instant $t$, denoted as $x_t$, can be represented as
\begin{equation}
    x_t=\left( P_t, Q_t \right)
    \label{eq: order_book}
\end{equation}
where $Q$ is the shape of the order book, and the dimension of Q is $2n+1$. 
The state of the order book is the result of order flow. 
Therefore, we use the order book's state to represent the order flow. 
For simple discussion, we do not consider short sale, and  
market order will be considered as a special case of limit order~\cite{gould2013limit}. 


\subsection{Trading Agent}\label{subsec: agent}

The main difference among trading agents' trading strategies lies in the evaluation of asset price, the so-called expected price.  
The expected price of an agent directly determines the agent's order, specifically the ordering price and size.  
Intuitively, a trader should be willing to buy/sell more stock shares when the stock price deviates more from its expected price. 

To introduce how the agents evaluate the price of an asset, 
we exemplify the taxonomy in~\cite{chiarella2006asset},
where the expected price of an agent is determined by the three components of fundamentalist ($f$), chartist ($c$) and irrational trader ($n$). 
Formally, denoting $\alpha$ as the proportion of the components and $\hat{P}$ as the expected price of the component, 
the expected price of an agent, indexed by $i$, is given as 
% which regards each agent as a composition of three components that determines how it evaluates the expected price:
\begin{equation}
    \hat{P}_t^{(i)} = \frac{\alpha_w^{(i)}\hat{P}_t^{(w)}+\alpha_c^{(i)}\hat{P}_t^{(c)}+\alpha_n^{(i)}\hat{P}_t^{(n)}   }{\alpha_w^{(i)}+\alpha_c^{(i)}+\alpha_n^{(i)}}.   
\end{equation}


A pure fundamentalist always agrees with the fundamental value of the asset, whose pricing is shown as 
\begin{equation}
    \hat{P}_t^{(w)}=w_t.
\end{equation}
On the other side, a pure chartist estimates price by looking at the historical prices on the market, which is given as 
\begin{equation}
    \hat{P}_t^{(c)}=g\left(P_{t-1}, P_{t-2}, ..., P_{t-\tau_i}\right), 
\end{equation}
where $g(\cdot)$ is usually a regression function, and $\tau_i$ is the horizon of the $i$th agent. 
The irrational trader characterizes the irrationality of real trader, 
whose expected price is sampled from a certain statistical distribution. 
Take Gaussian distribution as an example, the expected price of a pure irrational trader is given as
\begin{equation}
    \hat{P}_t^{(n)}\sim \mathcal{N}\left(P; P_{t-1}, \sigma_n^2\right), 
\end{equation}
where $\sigma_n$ quantifies the degree of irrationality. 

% Figure environment removed 

Given the expected price, 
an agent would make an order that specifies the price, denoted as $p_t$,  and size, denoted as $v_t$, of an order (we omit the index $i$ for conciseness). 
Commonly in behavioral economics, a trader tends to make a larger size of order 
when the order price deviates more from the expected price, 
because a large deviation means a large profit, the so-called CARA utility~\cite{babcock1993risk}. 
Based on this, the relationship between order price and size can be modeled as
\begin{equation}
    v_t=\frac{1}{\beta\Delta P_tp_t}\log\left(\frac{\hat{P}_t}{p_t} \right),
    \label{eq: risk}
\end{equation}
where $\beta$ is the degree of risk aversion, and $\Delta P_t$ is the variance of historical prices~\cite{vyetrenko2020get}. 
In this example, the agent's ordering depends on the proportion of 
the components~$\alpha_i^{(w)}, \alpha_i^{(c)}, \alpha_i^{(n)}$, irrationality $\sigma_n$, risk aversion $\beta_i$, and horizon $\tau_i$. 
These are the variables of each individual agent. 

Multi-agent market model needs the heterogeneity of agents to ensure the volatility of the market~\cite{iori2012agent, liu2020semiglobal}. 
Conversely, if all trading agents behave the same, no deal would be made on the order book, and the simulation becomes pointless.  
On the other hand, since a market model usually involves hundreds of agents and more, 
it is unrealistic to calibrate the variables of each agent. 
Therefore, literature summarizes the variables of all agents in the market model as multiple statistical distributions, 
upon which the variable of each agent is sampled. 
Accordingly, the set of prior parameters of these distributions is called the agent variable of the market model.  
For example, the risk aversion is sampled from a Gaussian distribution as
\begin{equation}
    \beta_i\sim \mathcal{N}\left(\beta; a_\beta, \sigma^2 \right),
\end{equation}
where $a_\beta$ is an element of the agent variable of the market model. 
% Essentially, the adoption of distribution introduces system randomness, 
% which is necessary for the simulation system to provide agent heterogeneity.

Overall, we summarize the agent variable of the model as a vector, which is written as
\begin{equation}
\begin{aligned}
    a&=\left[a_w, a_c, a_n, \sigma_n, a_\beta, a_\tau \right]\\
    &=\left[a_w, a_c, \sigma_n, a_\beta, a_\tau \right], 
\end{aligned}
\end{equation}
where the equality holds by normalizing the proportions of the three components as one. 
Note that, despite we introducing the above agent as an example,  
\sysname{} is also applicable to other types of rule-based agents.  



\iffalse
\subsection{Fintech Multi-agent System}\label{sec: ham_sys}
% Definition
Fintech multi-agent System, specifically referred to as heterogeneous multi-agent system~\cite{lebaron2006agent} in this paper, is a computational model that simulates the actions and interactions of different types of traders.
Even with simple agents, 
it can exhibit complex behavior patterns and provide valuable information about the dynamics of the real-world system which they emulate~\cite{iori2012agent}. 
Such a simulation system usually consists of an orderbook~\cite{gould2013limit} and a group of basic agents. Heterogeneous agents are created by combining the components of basic agents using linear combinations.


The basic agents mainly differ in how they evaluate the price of the asset. 
There are some common types of basic agents:
\begin{itemize}
    % zero-intelligence agent
    \item \emph{Fundamentalist}  (or value-based agent) estimates asset price according to fundamental price, which is an exogenous information that conveys the value of the asset in a moment:
    \begin{equation}
        \hat{P}_f^{(t)}=P_f^{(t)},
    \label{eq: fundamental}
    \end{equation}
    where $P_f$ is the fundamental price at time $t$.

    
    \item \emph{Chartist} estimates asset prices based on historical data on the orderbook.
    For example, chartist in~\cite{chiarella2006asset} evaluates asset price by
    \begin{equation}
        \hat{P}^{(t+1)}_c=P^{(t)}+\frac{1}{\tau}\sum_{T=t-\tau}^{t} \left( P^{(T)}-P^{(T-1)} \right),
    \label{eq: chartist}
    \end{equation}
    where $\tau$ is the investment horizon of the agent and $P^{(t)}$ is the mid-price at time $t$.

    \item \emph{Noise trader} models the irrationality of the market,
    where the estimated price is sampled from a distribution.
    The noise trader in~\cite{chiarella2006asset} samples prices from Gaussian distribution
    \begin{equation}
        \hat{P}_n^{(t)}\sim \mathcal{N}(P^{(t-1)}, \sigma^2),
        \label{eq: noise_trader}
    \end{equation}
    where $\sigma$ is the pre-determined standard deviation. 
\end{itemize}
More types of agents can be found in~\cite{chiarella2009impact}.


The estimated price of a heterogenous agent is calculated by
\begin{equation}
\hat{P}_{t+\tau}=\frac{g_f\hat{P}_f^{(t)}+g_c\hat{P}_c^{(t)}+g_n\hat{P}_n^{(t)}}{g_f+g_c+g_n},
\label{eq: price_est}
\end{equation}
where $f$, $c$, and $n$ refer to fundamentalist, chartist, and noise trader.
The agents differ by the component of being these three characters, i.e.~$g_f$, $g_c,$ and $g_n$.
In this paper, the component of each agent is separately sampled from Laplacian distributions.

% make orders
Agents make buy/sell orders decision according to their esitimated asset prices.
A buy or sell order consists of price and size (a market order is regarded as a limited order with the current price).
In this paper,
each agent calculates the price and size based on CARA utility function~\cite{babcock1993risk}:
\begin{equation}
\pi(P)=\frac{\log(\hat{P}_{t+\tau^{(i)}}/P)}{\alpha^{(i)} V_t P},
\label{eq: price_size}
\end{equation}
where order size $\pi(P)$ is determined by estimated price $\hat{P}_{t+\tau}$, current price $P$, 
risk aversion of the agent $\alpha^{(i)}$, and historical price variance $V_t$.
The $\tau^{(i)}$ denotes the investment horizon of the agent, 
where its order will be canceled after $\tau$ time slots without being executed.
Basically, 
Equation~\ref{eq: price_size} means to buy more at a lower price, and vice versa.
And the price can be sampled from a range that the agent's account can take.
We differentiate the institutional trader~\cite{} from a regular trader by initial cash and stocks in their accounts.
The investment horizon of agent $i$ is
\begin{equation}
\tau^{(i)}=\tau\frac{1+g_f^{(i)}}{1+g_c^{(i)}},
\label{eq: horizon}
\end{equation}
and its risk aversion is calculated by 
\begin{equation}
\alpha^{(i)}=\alpha\frac{1+g_f^{(i)}}{1+g_c^{(i)}},
\label{eq: risk_aversion}
\end{equation}
where $\tau$ and $\alpha$ are the base parameters. The initial cash and account are sampled from the uniform distribution.
More details can be found in~\cite{chiarella2006asset, chiarella2009impact}.



\subsection{Macroscopic Market Information}
% \chang{introduce CPI, PPI, ..., and why they can represent market state, how they impact the calibration, interpret why we need different strategies for differen market state}

Macroscopic market information usually reflects the running condition of the whole market,
effectively characterizing the overall behaviors of investors~\cite{levy1995microscopic}. 
Therefore, it will be useful to integrate such information in calibrating multi-agent simulation systems. 
Here, we list some indices that are commonly adopted to describe market states:

\begin{itemize}
    \item CPI stands for Consumer Price Index, 
    which is a measure of the average change in prices of goods and services over time. 
    A high CPI indicates severe inflation, where investors may become concerned about the future profitability of companies and may sell their stocks, 
    causing the stock market to decline;
    \item PPI measures the average change in prices that producers receive for their goods and services. 
    It is used to track inflation in the early stages of the production process, which can provide insight into the future direction of inflation. 
    PPI can impact the stock market and the behaviors of stock investors because changes in producer prices can affect the profitability of companies;
    \item PMI stands for Purchasing Managers' Index, which is a measure of the economic activity in the manufacturing sector. 
    % It is based on a survey of purchasing managers who provide information on factors such as new orders, production levels, and employment.    
    When PMI is high, it indicates that the manufacturing sector is expanding, which can lead to increased production and potentially higher profits for companies. This can cause investors to become more optimistic and may lead to a rise in the stock market;
    \item Market trend is a technical indicator of market state. In a market whose price forms a strong trend, investor strategies tend to resort to the technical side. In this paper, we calculate market trend by
    \begin{equation}
        MT=\frac{P_t-P_N}{ATR_t},
    \end{equation}
    where~$P_N$ is the averaged asset price of the recent N days and the average true range (ATE) is calculated by 
    \begin{equation}
        ATE_t=\frac{ATE_{t-1}\times (n-1)+TR_t}{n}.
    \end{equation}
    Here, the true range~$TR=\max\{(P_{high}-P_{low}), abs(P_{high}-P_{close}), abs(P_{low}-P_{close})\}$;
    \item Market noise is a technical indicator of price uncertainty. 
    It is usually represented by efficiency ratio
    \begin{equation}
        ER_t=\frac{P_t-P_{t-n}}{\sum_{i=t-n}^t P_i-P_{i-1}}.
    \end{equation}
    
\end{itemize}
More details of macroscopic market information can be found in~\cite{AKShareZhaiQuanShuJuAKShare}.

To guarantee diversity in market states, 
we select five monthly-updated indicators in the experiments:
consumer price index~(CPI), producer price index~(PPI), purchasing manager's index~(PMI),
market trend, and market noise. 
The historical records of CPI, PPI, and PMI can be found in~\cite{AKShareZhaiQuanShuJuAKShare}, 
and the market trend and noise are two well-known indicators that can be calculated separately based on 
average true range~(ATR) and efficiency ratio~(ER) using daily close price. 

\fi
