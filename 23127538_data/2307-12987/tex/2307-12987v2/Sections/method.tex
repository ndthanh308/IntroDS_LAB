

\section{\sysname{} Design}\label{sec: method}
This section introduces the technical details of \sysname{}. 
First, we introduce the problem formulation in Section~\ref{subsec: prob}. 
Then, we discuss 
order-flow embedding in Section~\ref{subsec: extractor}, 
surrogate-trading loss function in Section~\ref{subsec: loss}, 
and context-aware variable estimator in Section~\ref{subsec: adapter}. 

\subsection{Problem Formulation}\label{subsec: prob}

% Table
\begin{table}
    \centering
        \caption{Notation summary.}
    \begin{tabular}{c|c}
        \hline
            Notation & Description \\
            \hline
            $P_t$ & Mid-price on the order book at time instant $t$ \\
            $\hat{P}_t^{(w)}$ & Expected mid-price of a pure fundamentalist at time instant $t$ \\
            $\hat{P}_t^{(c)}$ & Expected mid-price of a pure chartist at time instant $t$ \\
            $\hat{P}_t^{(n)}$ & Expected mid-price of a pure noise trader at time instant $t$ \\
            $\hat{P}_t^{(i)}$ & Expected mid-price of the $i$th agent at time instant $t$ \\
            $p_t$ & Price of an order at time instant $t$ \\ 
            $Q_t$ & Shape of the order book at time instant $t$ \\
            $v_t$ & Size of an order at time instant $t$ \\
            $x_t$ & State of the order book at time instant $t$ ($x_t=(P_t, Q_t)$)  \\
            $x$ & Order flow of a trading day ($x=\{x_t\}$) \\
            $\hat{x}$ & Simulated order flow \\
            $\beta_i$ & Degree of risk aversion of the $i$th agent \\
            $\tau_i$ & Horizon of the $i$th agent \\
            $a$ & Agent variable of a trading day  \\
            $w$ & Fundamental value of an asset  \\
            $s$ & Index of market condition  \\
        \hline
    \end{tabular}
    \label{tb: Notation}
\end{table}

In the problem, 
denoting $s$ as the index of market condition and $x=\{x_t\}$ as the order flow of one trading day, 
we aim to capture the distribution of agent variable of the target day, which is given as 
\begin{equation}
    Pr(a|x,s), 
    \label{eq: objective}
\end{equation}
such that the multi-agent market model can generate an order flow similar to $x$. 

In the meantime, to correlate agent variable with market condition, 
we maximize their correlation, denoted as $Corr(\cdot, \cdot)$, while maintaining desirable simulation accuracy, which is summarized as 
\begin{equation}
    \begin{aligned}
        &\max\ \operatorname{Corr}(a, s) \\
        \text{s.t.}\ & \left\lVert \mathbb{E}_{Pr(x|a, w)}(x)-x\right\rVert_n <\epsilon,  
    \end{aligned}
\end{equation}
where $\epsilon$ controls the degree of simulation accuracy. 
The optimization is possible because the multi-agent market model is a probabilistic model, 
where multiple candidate agent variables can possibly lead to a similar order flow. 

To derive the distribution, 
we consider the relationship among the elements for calibrating the multi-agent model as shown in Figure~\ref{fig: sim_graph}. 
As mentioned, the simulation of the market model jointly depends on the fundamental value of the asset and the agent variable. 
Meanwhile, we want to simulate the reality that market condition affects the behaviors of traders, such that the 
market condition (reflected by some indexes) directly affects the agent variable. 
Given the graph, we next disassemble the objective in Equation~(\ref{eq: objective}).  

Recalling in Section~\ref{subsec: ham_sys} that the fundamental value~$w$ can be derived from the order flow,   
we rewrite our objective in Equation~(\ref{eq: objective}) as
\begin{equation}
    Pr(a|x,s)=Pr(a|x,s,w). 
    \label{eq: fundamental}
\end{equation}
Since Equation~(\ref{eq: fundamental}) is intractable to estimate,
we tackle this by applying the Bayesian rule, which is given as 
\begin{equation}
    Pr(a|x,s,w)=\frac{Pr(x|a,s,w)Pr(a|s,w)}{Pr(x|s,w)}.
    \label{eq: bayesian}
\end{equation}

Equation~(\ref{eq: bayesian}) suggests estimating agent variables using previous calibration experiences on the multi-agent market model.  
First, when agent variables are given, 
market condition does not affect order flow because it influences order flow mediated by agent variables (as illustrated in Figure~\ref{fig: sim_graph}), and thus we have 
\begin{equation}
    Pr(x|a, s, w)=Pr(x|a,w). 
\end{equation}
Second, the agent variable is independent of the fundamental value when order flow is unknown based on the D-separation~\cite{pearl2009causal}, and thus we get
\begin{equation}
    Pr(a|s,w)=Pr(a|s). 
\end{equation}

In a nutshell, the calibration of agent variable in Equation~(\ref{eq: objective}) can be estimated by the joint probability from two conditional distributions, which is given as 
\begin{equation}
    Pr(a|x,s)\propto Pr(x|a,w)Pr(a|s). 
    \label{eq: goal}
\end{equation}
On the right-hand side, the first term indicates that the agent variables should be able to simulate the order flow by the multi-agent model, 
and the second term suggests that market conditions are the prior knowledge for estimating the agent variables. 
In the following, we discuss how to estimate the two probabilities independently. 


\subsection{Order-flow Embedding}\label{subsec: extractor}

% Figure environment removed 

We first introduce how to calibrate the agent variable using deep learning model to achieve simulation accuracy. 
As the first step, we discuss embedding order flow as deep feature in this section. 

To calibrate the agent variable using deep learning models, 
we embed the order book information into deep feature space using 
an order-flow embedding module. 
As shown in Figure~\ref{fig: extractor}, 
the module decomposes the information of order book into mid-price and order-book shape based on Equation~(\ref{eq: order_book})
and uses two LSTM models, separately denoted as $k_{\theta_P}(\cdot)$ and $k_{\theta_Q}(\cdot)$, to capture their temporal features individually, which is shown as 
\begin{equation}
    k_{\theta_P}(P_0, P_2, ..., P_{T-1})=h_P,
\end{equation}
and 
\begin{equation}
    k_{\theta_Q}(Q_0, Q_2, ..., Q_{T-1})=h_Q,
\end{equation}
where $T$ is the step number in a simulated trading day. 
Then, the price feature $h_P$ and shape feature $h_Q$ are concatenated to form a {\em flow feature} as
\begin{equation}
    h_x=[h_P, h_Q].  
\end{equation}

\subsection{Surrogate-trading Loss Function}\label{subsec: loss}

% Overview
The agent variable, with the market model, should be able to reproduce the target order flow, as suggested in Equation~(\ref{eq: goal}). 
Considering the irrationality of the market model, 
we build a mapping from the target order flow to the distribution of the agent variable. 
As shown in Figure~\ref{fig: loss_function}, 
we employ a variational auto-encoder (VAE) to learn the mapping (the VAE decoder is used for generating the agent variable). 
The VAE model, denoted as $p_{\omega_1}(\cdot)$, maps the flow feature into a Gaussian distribution in latent feature space
\begin{equation}
    p_{\omega_1}(h_x)=[z_0, \sigma_0],  
\end{equation}
where the feature mean and standard deviation are presented as $z_0$ and $\sigma_0$ respectively. 
Then, we use a decoder to convert the feature distribution to the agent variable by a sampling, which is given as 
\begin{equation}
    q_{\omega_2}(z_0+\epsilon\sigma_0)=a, 
\end{equation}
where $\epsilon\sim \mathcal{N}({\bf 0}, \sigma_0^2)$ implements the reparameterization trick of VAE. 

% Problem
Different from most deep-learning tasks, 
the multi-agent market model is not differentiable. 
As a result, we cannot learn the VAE by evaluating the quality of the simulated order flows. 
One intuitive idea is to pseudo-label the order flows with the agent variable searched by the traditional approaches, 
and the model training would be given as 
\begin{equation}
    \min_{(\omega, \theta_P, \theta_Q)} \sum_{x\in \mathcal{D}} \left\lVert a-a_x \right\rVert_n,
    \label{eq: comp_goal}
\end{equation}
where $a_x$ is the pseudo label of the order flow, and $\mathcal{D}$ is the training dataset.  
However, this could be very costly because acquiring each pseudo label requires conducting multiple simulations. 
% and the search randomness may cause unstable training. 

To tackle this non-differentiable issue, 
we reparameterize the multi-agent market model by a deep learning surrogate.  
Specifically, we learn a surrogate model to mimic the behaviors of the multi-agent model, 
and let the surrogate provide gradients for model training. 
We argue that the surrogate model is sufficient to provide the gradients to supervise the calibration learning, 
as it directly reflects the behavior of the multi-agent market model. 
Another advantage of learning the surrogate model is training data-efficiency. 
In other words, 
we merely need to randomly create sets of agent variables to probe the multi-agent model's behavior, serving as the training data. 
This is much more data-efficient than searching the pseudo-labels. 
% Also, as the surrogate learns to predict the expectation of the model output, it is more stable for learning the calibration. 

% Figure environment removed 

However, it is nontrivial to reparameterize the multi-agent market model due to the system complexity. 
To tackle this, we characterize order flows by a group of representative features and 
train the surrogate model to predict these features based on the agent variable. 
For example, agents tend to make larger orders when they hold a lower risk aversion. Another example is that
the mid-price tends to fluctuate more significantly with more chartists in the market. 
In the field of market simulation, these features are called stylized facts, which are believed to capture the main components of the market, 
serving as good representations of order flow. 
In this paper, we consider the following stylized facts (details can be found in~\cite{vyetrenko2020get}):
\begin{itemize}[leftmargin=*]
    \item \emph{Thickness of return tail } reflects the heaviness of the tail of the return distribution, which is captured by the kurtosis of return distribution; 
    \item \emph{Volatility clustering } reflects the fluctuation of the stock price, which is captured by the parameter of the auto-regression model; 
    \item \emph{Order book volume } refers to the distribution of order volume at the best bid and ask, which is often described by Gamma distribution. In this paper, we measure it by its distribution parameter; 
    \item \emph{Order-size distribution } describes the pattern of order size from various traders, which is often modeled as power-law distribution. In this paper, we measure it by the distribution parameter. 
\end{itemize}

Based on these stylized facts, 
we reparameterize the multi-agent model by learning a surrogate model, denoted as $u_\gamma(\cdot)$, which is given as
\begin{equation}
    \min_{\gamma} \sum_{a\in \mathcal{A}} \left\lVert u_{\gamma}(a)-r(\hat{x}) \right\rVert_2^2,
\end{equation}
where $r(\cdot)$ calculates the stylized facts from the simulated order flow $\hat{x}$. 
Overall, the calibration modules of \sysname{} learns from a surrogate-trading loss function based on the surrogate model, which is shown as 
\begin{equation}
    \mathcal{L}_{sur}(x)= \left\lVert p\circ q\circ u(h_x)-r(x)  \right\rVert_2^2, 
    \label{eq: sur_loss}
\end{equation}
recalling that $q$ is the VAE decoder, $p$ is the surrogate model, $h_x$ is the flow embedding, and 
$x$ is the target order flow. 



\subsection{Condition-aware Variable Estimator}\label{subsec: adapter}

% Figure environment removed 

Besides simulation accuracy, 
we want to correlate the agent variable with the indexes of market conditions. 
On one hand, this makes the calibration, as well as the market simulation, more interpretable~\cite{iori2012agent}. 
On the other hand, if we can use the market condition to fine-tune the agent variable, 
we can hypothesize and create more market scenarios through calibration, 
which is important for many Fintech studies. 

As discussed, the order-flow embedding module maps each order flow to a distribution in deep feature space. 
Suppose we know the distribution of the feature under each market condition, 
we can calculate the joint distribution to fine-tune the order flow feature. 
This is similar to the style transfer in image generation. 

As shown in Figure~\ref{fig: adapter}, 
we build a condition-aware variable estimator to calculate such joint distribution. 
Specifically, we learn a condition adapter to map each market condition to a feature distribution in latent space, which is shown as 
\begin{equation}
    f_\eta(s)=[z_s, \sigma_s].  
\end{equation}
Similarly, the condition adapter is an additional encoder of the VAE, and it outputs the parameters of the Gaussian distribution. 
From another point of view as in Equation~(\ref{eq: goal}), the output of the conditional adapter is the prior knowledge of the flow feature. 

To avoid the market condition degrading the simulation accuracy, 
we transform it as a uniform distribution and constrain it by a range parameter $k$. 
Therefore, the joint distribution is calculated as
\begin{equation}
    z\sim \mathcal{U}[z_s-k\sigma_s, z_s+k\sigma_s]*\mathcal{N}(z_0, \sigma_0^2), 
    \label{eq: convolution}
\end{equation}
where we consider the latent feature within $k$ standard deviation from $z_s$ as the candidate set, and 
$*$ is the operator of convolution. 
After getting the joint distribution, 
we calculate the the expectation as the flow feature, which would be decoded to estimate the agent variable.   

% 
However, the order flow distribution under the market condition is non-trivial to learn, as we do not have the ground truth. 
To tackle this, our observation is that the real trader tends to behave similarly under similar market conditions. 
Therefore, we cluster the deep feature of order flow by their market condition to learn the condition adapter. 
In other words, it is more likely for the agent variables of two trading days to be similar if their market conditions are more similar.   
As such, for each pair of order flow of different trading days, 
we enforce a constraint of consistency based on their market conditions.  
The constraint is stronger if their market conditions are more similar. 
With real trading data denoted as $\mathcal{D}$, 
we learn the condition adapter using a market-condition loss function, which is shown as
\begin{equation}
    \mathcal{L}_{con}(s)= \sum_{i=1}^{|\mathcal{D}|}\sum_{j=1}^{|\mathcal{D}|} \frac{\left\lVert a_i-a_j \right\rVert_n}{\left\lVert s_i-s_j\right\rVert_n+1}. 
\end{equation}
In the loss function, the weight of the constraint is inversely proportional to the distance of market conditions between the pair, 
as the constraint should be stronger if two pairs have more similar market conditions. 

The estimator can be regarded as a fine-tuning step, which is built upon the learned encoder and decoder. 
Therefore, it converges very fast without requiring massive real trading data. 
In the training of the estimator, 
we additionally use the surrogate-trading loss function to ensure the simulation accuracy, and the overall loss function is shown as
\begin{equation}
    \mathcal{L}_{ada}=\mathcal{L}_{sur}(x)+\alpha\mathcal{L}_{con}(s). 
    \label{eq: all_loss}
\end{equation}







\iffalse

\begin{equation}
    \begin{aligned}
            z&\sim \mathcal{N}(z|z_0, \sigma_0^2)\times \mathcal{N}(z|z_s, \sigma_s^2) \\
            &=\mathcal{N}\left(z\bigg|\frac{z_0\sigma_s^2+z_s\sigma_0^2}{\sigma_0^2+\sigma_s^2}, \frac{\sigma_0^2\sigma_s^2}{\sigma_0^2+\sigma_s^2}\right). 
    \end{aligned}
\end{equation}

In reality, market condition characterizes traders' behaviors. 
For example, traders tend to be more optimistic and aggressive under an environment of low interest and inflation, 
while they would become more cautious in an opposite scenario. 
On the other hand, market conditions partially explain traders' behaviors due to the extreme sophistication of stock markets. 
Therefore, 
we use the analogy of images with different styles to describe traders' behaviors in various market conditions.


In a standard pipeline of deep learning, 
we can estimate the agent variables based on the flow feature:
\begin{equation}
    g_\omega(h_x)=a,
\end{equation}
where we name $g_\omega(\cdot)$ as the variable estimator. 

% Figure environment removed 



\section{\sysname{} Design}\label{sec: method}
% \chang{overview}
We overview the design of \sysname{} in Figure~\ref{fig: overview}.
To simulate a target trading period, 
a state-aware calibration module generates behavior variables based on the order flow of the period.  
The behavior variables can be promptly used to configure the multi-agent system for further financial analysis. 
Considering that investors may perform differently under various market states,
a meta-calibration module is designed to fine-tune the calibration module using the market state of the target period.
In the training phase, 
\sysname{} involves a surrogate trading system 
and two consistency constraints, 
with the goal of achieving order-flow reconstruction while maintaining interpretable behavior variables. 

\subsection{Meta-design for State-aware Calibration}\label{sec: meta}

\vspace{3pt}
\noindent
\textbf{Meta-calibration Module} 
% We treat macroscopic data as representations of market states, denoted as $s$. 
To capture the influence of these market states while maintaining reasonable variations among different states, 
we conceptualize the market state as a domain index, as discussed in prior work~\cite{wang2020continuously}. 
This domain index informs the model parameters of a state-specific calibration module, represented as:
\begin{equation}
\theta = g_\omega(s).
\label{eq: adapt}
\end{equation}
Here, $g_\omega(\cdot)$ refers to the meta-calibration module, parameterized by $\omega$.

Essentially, 
this module characterizes the different calibration strategies, parameterized as~$\theta$, under various market states~$s$, for which we name it as a meta-module. 
This differentiates directly using market state to calibrate behavior variables in that market states, which are microscopic data, are usually not considered to have a direct impact on investor behavior. 
In other words, market state may not be useful in reconstructing the order flow of the target trading period, though, it helps to determine a set of behavior variables when multiple candidate variables, caused by the calibration-redundancy issue, are to be chosen. 

% Introduce adaptor
\vspace{3pt}
\noindent
\textbf{State-specific Calibration Module} 
Given the parameters of the state-specific calibration module $\theta$, we define the calibration model $K_\theta(\cdot)$. The behavior indicator is calculated as follows:
\begin{equation}
b = K_{\theta}(y),
\label{eq:adaptor}
\end{equation}
Here, $y$ represents the extracted features of the order stream~$x$, specifically denoted as $y = f(x)$, where $f(\cdot)$ functions as the feature extractor. 
Generally, the features can be extracted by deep models~\cite{coletta2022learning} or manually designed (the so-called stylized facts)~\cite{vyetrenko2020get}. 
The features used in our experiments will be introduced in Section~\ref{sec: exp_set}.

Based on such a model structure, 
\sysname{} effectively integrates the macroscopic~(market state) and microscopic~(order flow) data to calibrate multi-agent system, achieving order-flow reconstruction while maintaining interpretable behavior variables. 
Note that, the model structure provides the possibility to accommodate both kinds of data for calibration. The training objectives will be discussed in the following. 

\subsection{Consistency Constraints} 
% Motivation
To make the simulation results easy to interpret, besides order-flow reconstruction,
calibration should also consider the temporal correlation of the behavior variables. 
In this paper, 
we consider market-state consistency and temporal consistency, both serving as the training objectives of the calibration modules. 

\noindent
{\bf Market-state Consistency}
Generally,
the investor behaviors perform similarly under similar market states. 
To reflect this, 
we target the consistency of behavior variables,
building up a relationship between behavior variables with market states. 
Intuitively, in a bull market, 
investors are generally optimistic about making profits and thus can be bold in making moves. 
While, in a bear market, they tend to be risk-averse and act holding back. 
We characterize this feature as market-state consistency, 
where the behavior variables, in similar markets, should resemble each other. 
This serves as an objective to explore the relationship between behavior variables and market states from historical data
and makes it possible to apply such a relationship back to the calibration of multi-agent systems. 

In detail, the market-state consistency is enforced by a sampled-based approach. 
We sample from some trading days and design a loss function based on the contrastive method. 
For the sampled trading days~$i$, $j$, and $k$,
whose market states conform to
\begin{equation}
\left\lVert s_j-s_i\right\rVert_n< \left\lVert s_k-s_i\right\rVert_n,
\label{eq: state_sim}
\end{equation}
we apply a triplet loss to permutate their behavior variables according to their market states:
\begin{equation}
\mathcal{L}_{stat} =  \max
\left\{
\left\lVert b_i-b_j \right\rVert_n - \left\lVert b_i-b_k \right\rVert_n + \sigma 
, 0\right\},
\label{eq: contrast_loss}
\end{equation}
where $\sigma$ denotes a pre-defined contrastive gap. 
With multiple such kinds of sampling,
the calibration modules will learn to assign each market state with a domain of behavior variable. 
Thus, in the calibration of each period,
the generated behavior variables will conform to the domain of their market states, 
making the simulation results easier to interpret. 

\noindent
{\bf Temporal Consistency}
Without a striking intervention, 
the overall behaviors of the market usually vary smoothly and continuously.
Thus,  
we further consider a temporal consistency to constrain the behavior variables: 
\begin{equation}
\mathcal{L}_{temp}= \mathbbm{1}(\left\lVert s_t-s_{t+1}\right\rVert_n<\tau) \left\lVert b_t-b_{t+1} \right\rVert_n,
\label{eq: temp_loss}
\end{equation}
where $\mathbbm{1}(\cdot)$ denotes the indicator function and~the threshold $\tau$ indicates an exogeneous intervention. 

In a simulation system, 
the exogenous intervention is usually referred to as a drastic change in market states. 
This makes the selection of market-state indicators a crucial assumption of a Fintech market simulation system. 
From a user perspective,
the selected market-state indicator should be the key factor that dominates the simulated market. 
Luckily, 
\sysname{} can capture the correlation between the market states and behavior variables, which provides a way for users to examine if the selected market-state indicators are desirable. 
This will be shown in the case study of our experiment. 

Overall, the proposed two constraints are integrated into one loss function
\begin{equation}
\mathcal{L}_{cons}=\mathcal{L}_{stat}+\mathcal{L}_{temp},
\label{eq: state_loss_func}
\end{equation}
which is used to supervise the training process of the calibration modules in Section~\ref{sec: meta}. 
 
\subsection{Surrogate Trading System}
The proposed calibration model requires end-to-end training to achieve order-flow reconstruction,
while the multi-agent simulation is not differentiable because its order-generation process is based on sampling. 
Specifically,
to model the irrationality of investors, 
the order-making decision of an agent usually involves randomness. 
Therefore,
it is hard to numerically determine the behavior variables for a specific day via end-to-end training. 

To tackle this,
our basic idea is to learn a differentiable evaluation metric to measure the discrepancy between a set of behavior variables and an order flow
\begin{equation}
    \mathcal{L}_{rec}=\left\lVert y-\hat{y_1} \right\rVert_n,
\label{eq: sur1}
\end{equation}
where~$\hat{y_1}$ is generated by an actual multi-agent system. 
Since getting~$\hat{y_1}$ necessitates involving the tedious and inefficient simulation in the training process that is not differentiable, 
we replace it by 
\begin{equation}
    \hat{y_2}=G_\gamma(b),
\label{eq: sur2}
\end{equation}
where~$G$ is a surrogate trading system parameterized by~$\gamma$. 
The surrogate trading system can be learned from past calibration experiences
\begin{equation}
\min_{\gamma \in Y}\ \left\lVert G_{\gamma}(b)-\hat{y}_1\right\rVert_n.
\label{eq: learn_obj}
\end{equation}
As the simulation model is dominated by behavior variables, it is also possible to learn to estimate order-flow features from a set of behavior variables. 
Further, the past calibration experiences not only involve those successful calibrations in terms of order-flow reconstruction, but all others are useful in learning a surrogate trading system. 

Overall, the loss function for training the calibration model is
\begin{equation}
\mathcal{L}=\mathcal{L}_{rec}+\eta\mathcal{L}_{cons},
\label{eq: all_loss}
\end{equation}
where $\eta$ is a balance weight.
The weight is suggested to be smaller than 1 so that the two constraints do not influence the order-flow recontruction. 

\fi