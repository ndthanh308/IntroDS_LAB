



\iffalse

\subsubsection{System setup}

% proxy trading system can be learned
We first show the learning process of the surrogate model in Figure~\ref{fig: eval_proxy_model},
where the loss of the training curve decreases with training epochs.
The model converges at the epoch of 25 and flats off after that.
The plot demonstrates a common pattern of learning curves about training deep learning models.
This shows that 
the idea of learning a surrogate trading system is feasible.
In the following, we select the model at 25-epoch as the surrogate trading system that contributes to the calibration loss function.

% There exist parameters that satisfy reconstruction and consistency 
We demonstrate the training process of the state-aware calibration model by showing the reconstruction error and behavior variation in Figure~\ref{fig: eval_reconstruction_with_temp}.
It shows that both reconstruction error and behavior variation are reduced in the training process,
where both metrics decrease drastically in the first fifteen epochs, and the reduction slows down after that. 
This verifies optimizing both objectives concurrently.
Although there are several sets of behavior variables that result in similar order streams,
the proposed two loss functions can optimize in terms of order-flow reconstruction and behavior consistency in calibration.

\subsubsection{Market replay}

% Overall result on consistency in the training set
We compare our approach with the state-of-the-art calibration approaches and 
show their distribution of behavior variations over the first nine months (training set) in Figure~\ref{fig: pdf_behavior_variation}.
From the histogram, most behavior variations of the comparison schemes reside in the range from two to four. 
Such large variations do not reflect the market pattern that the overall participant behaviors vary continuously. 
Notably, the behavior variation of RandSearch is slightly larger than that of BayesianOPT due to its larger randomness.
In comparison, \sysname{} considers the behavior consistency and employs the state-consistency loss function to train the calibration model.
As a result, most behavior variations of \sysname{}, from the figure, fall into the range from zero to two,
which observes the market regularity in calibration.

% Generality in the test set
To verify the generality of \sysname{}, 
we further compare behavior variations over the last three months~(the testing dataset)
and  show the results in Figure~\ref{fig: pdf_test_variation}.
The figure shows that the behavior variation distribution of both comparison schemes 
remains unchanged compared with their training distribution
because their searching processes for each trading day are independent.
Without a significant increase in behavior variation,
\sysname{} outperforms the random search and Bayesian optimization.
This verifies the generality of \sysname{} to observe behavior consistency in calibration.

% Reconstruction error
Figure~\ref{fig: comp_train} compares the reconstruction error by cumulative distribution function~(CDF) 
over calibration on trading days in the training dataset.
From the figure, \sysname{} performs similarly to Bayesian optimization to achieve low reconstruction error,
while random search shows unsatisfactory results. 
This is because \sysname{} and BayesianOPT have objectives for reducing the error, 
while the parameter searching of RandSearch is pretty random.
Even though, 
both comparison schemes require a tedious search process that takes around $120min$ to calibrate,
while \sysname{} is search-free and takes less than $1s$.
This verifies the efficiency of \sysname{}.

We further show the reconstruction error on the test set in Figure~\ref{fig: comp_test}.
It shows that the performance of RandSearch and BayesianOPT does not change much since they search parameters independent of trading days.
In the figure,
\sysname{} also performs similarly to BayesianOPT in terms of reconstruction error in a search-free manner,
while observing behavior consistency in the meantime~(as demonstrated in Figure~\ref{fig: pdf_test_variation}).
This verifies the generality of \sysname{} in reconstructing historical markets
and shows the superiority of \sysname{} over its comparison schemes. 

% Ablation study in market state loss
We conduct an ablation study on the market state consistency loss function
and compare \sysname{} with or without the loss function
in terms of reconstruction error, as shown in Figure~\ref{fig: abla_study}.
We did not show fundamentalist since the three components are complementary.
The figure shows that considering the market state facilitates calibration to reduce reconstruction error
in both training and test sets.
Although both models get a slight error increase from training to test data, 
considering market states in \sysname{} can reduce the error increase.
This is because the market state, serving as a domain index, 
reduces the domain gap between the training and testing set.
This verifies that the market state is not only useful in maintaining behavior consistency but also for the generality of order-flow reconstruction.

\subsubsection{Case study}

% Correlation in market state
To show that \sysname{} can correlate agent behaviors with market states,
we calculate the Pearson correlation coefficients between the market states and the behavior variables,
which is shown in Table~\ref{tab: correlation}.
From the table, 
the produced behavior variables of \sysname{} are much more correlated to the market states than 
the comparison schemes. 
% This is because \sysname{} considers the market state and employs the market state consistency loss function.
This shows the ability of \sysname{} to explore the market state information
and validates the effectiveness of the state-consistency loss function.
% This shows that \sysname{} successfully calibrates by observing the rule of market state consistency.

We further analyze the behavior relationship with the market trend
and show its correlation coefficients with the calibrated variables in Figure~\ref{fig: market_trend_corr}.
When the market trend becomes significant, from the correlation coefficient,
\sysname{} will increase
the ratio of chartists and institutional investors, and reduces the number of noise traders and the investment horizon of the whole market.
This reflects the real market pattern that the number of chartists will increase 
when a market shows a clear trend that fits their trading strategy.
Concurrently, the number of noise traders will decrease because of the obvious signal in the market.
On the contrary, institutions often choose to enter stock markets when it shows significant trends, or sometimes the impacts are mutual.
Under such a scenario, 
participants are more likely to take profits out of the market~(the liquidation), 
leading to a low investment horizon over the whole market.


\fi




