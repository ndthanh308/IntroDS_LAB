\section{Related Work}\label{sec: related_work}

% First cut: Relationship with other Fintech studies
Many Fintech research topics are related to trading simulation. 
    % Return prediction
Return prediction studies to predict stock prices using indigenous and exogenous information of the market~\cite{bai2020entropic,hou2021stock, hou2022multi, xing2023learning, gulmez2023stock, zhang2023stock}. 
These works can be used for predicting the fundamental value of the trading simulation system for simulating the future markets.   
    % Trading Strategy
The research on trading strategies, particularly those based on machine learning methods, 
has become prevalent~\cite{brogaard2023machine, sahu2023overview, han2023machine, hsu2021fingat, deng2016deep}. 
Before engaging in real trading, these approaches often need backtesting with historical data, 
where trading simulation can provide the opportunity for interactive backtesting. 
% These approaches can play roles as either trading agents or testing targets in market simulation systems. 
    % Sentiment analysis
The sentiment analysis of the stock market on social media is extensively studied~\cite{cao2014adaptive,xiao2023stock, costola2023machine, agarwal2023sentiment}. 
In terms of trading simulation, the output of these works can be considered to assess the fundamental value of the stock or to describe the market conditions. 

% Second cut: Market simulation
Trading simulation mainly studies trader behaviors to simulate real-world stock trading on order books. 
Existing trading simulations can be categorized as single and multiple-agent trading simulations. 
The single-agent market simulation~\cite{coletta2021towards, shi2022state, li2020generating, coletta2022learning} 
models an aggregated behavior via one consolidated agent to make orders, the so-called ``world agent''~\cite{coletta2021towards}.
However, the single-agent simulation cannot reflect the trading details of the agents as well as their interactions, 
thus lacking the explainability that is critical to Fintech applications.
    % Multi-agent
Multi-agent trading simulation~\cite{liu2021finrl, byrd2019abides, storchan2021learning, karpe2020multi} represents each agent as one~(or a group of) traders with parameterized behaviors.  
Each agent strives to maximize its proprietary profits, where the order flows are generated over their interactions and competitions. 
Multi-agent simulation provides the very details of trading for understanding the simulation results, 
such as probing each agent's account, tracking agents' order-making behaviors, understanding agents' reactions to exogenous interventions, etc.
Before conducting a simulation,
multi-agent market simulation requires calibration of the agent variables to account for simulation fidelity.  
% which might be inefficient due to the system's complexity and dynamic nature. 

% Third cut: Calibration
Since multi-agent market model is not differentiable, 
the core issue of the system calibration is black-box optimization. 
Population-based approach searches for variables by sampling over the variable spaces, e.g., 
particle filter~\cite{carpenter1999improved, djuric2003particle, zhang2023diesel}, genetic algorithm~\cite{alhijawi2023genetic, mathew2012genetic} and ant colony algorithm~\cite{dorigo2006ant, blum2005ant}. 
However, the massive sampling is too costly for calibrating the market simulation system. 
Trajectory-based approach, such as Bayesian optimization~\cite{garnett2023bayesian, frazier2018bayesian} and simplex algorithm~\cite{singer2009nelder}, searches variables along a single searching path, which is possible, if not affordable, for the system calibration~\cite{bai2022efficient, cao2022dslob}.
Surrogate-based approach parameterizes the black-box system to facilitate optimization~\cite{han2012surrogate, queipo2005surrogate}. 
However, it is usually difficult to parameterize complex and dynamic systems. 
In this paper, 
we leverage stylized facts to parameterize the multi-agent market model 
and consider market conditions to enhance the explainability of calibration. 

\iffalse
With the advancement in data mining~\cite{dogan2021machine} and storage technologies~\cite{chen2022tvconv},
market analysis using orderbook-level data has flourished in the field of quantitative finance,
such as return prediction~\cite{hou2021stock, hou2022multi, shah2019stock} and LOB recreation~\cite{shi2021lob, shi2021limit}.
Among them, 
market simulation~\cite{assefa2020generating} leverages the orderbook information to simulate trader behaviors, 
which enables many financial applications such as trading-strategy backtesting~\cite{dowd2012back, gort2022deep, vezeris2020optimization}, risk management~\cite{gupta2013risk, leo2019machine}, and stress testing~\cite{reinders2023finance}. 

Market simulation leverages agents, representing market traders, to interact via orderbook. 
The existing multi-agent simulation can be categorized as single and multi-agent market simulation by agent duties. 
The single-agent market simulation~\cite{coletta2021towards, shi2022state, li2020generating, coletta2022learning} 
models an aggregated behavior via one consolidated agent to make orders, the so-called ``world agent''~\cite{coletta2021towards},
which interacts with the orderbook.
Being conceptually straightforward, though, the single-agent model cannot simulate the interactions of traders and suffers poor explainability.
% which is straightforward but suffers poor explainability.
Multi-agent market simulation~\cite{liu2021finrl, byrd2019abides, storchan2021learning, karpe2020multi} represents each agent as one~(or a group of) traders with parameterized behaviors; 
each agent strives to maximize its proprietary profits, 
where order streams are generated over their interactions and competitions. 
Multi-agent simulation provides very details of trading, such as agent account variations, 
which are essential for users to understand the simulation results.
% Still requires calibration
Nevertheless,
to reconstruct the order flows of target trading days, 
the multi-agent market simulation requires calibration of agent behaviors that are represented by behavior variables~\cite{bai2022efficient}.

% The third cut, multi-agent simulation calibration
The previous calibration for multi-agent simulation is search-based:
they try multiple sets of behavior parameters via simulation and select the best setting that reconstructs targets.
For example, works in~\cite{chiarella2009impact, cao2022dslob, storchan2021learning} search the behavior variables by grid search.
However, 
these calibration approaches necessitate vast numbers of simulations for each trading day, 
which is computationally intensive and time-costly.
% Heuristic
To consider the search efficiency,
work in~\cite{bai2022efficient} leverages Bayesian optimization for parameter searching.
Even though, 
it still requires massive simulations for satisfactory reproducing accuracy,
which is not applicable to extensive experiments that the industry usually requires.
Also, 
the previous calibration approaches has not fully considered about the redundancy issue of calibration,
making the resulting simulation unstable, unconvincing, and hard to interpret.
In comparison,
\sysname{} directly utilizes the features from order flows to reconstruct and leverages market states to reduce the calibration redundancy,
which enables the calibration to be search-free and interpretable.
% tackles the calibration redundancy by patterns from the real markets,
% which calibrates in one shot and produces behavior parameters that reflect real market patterns.
% In this way, \sysname{} achieves high reproduction accuracy in an end-to-end manner.
% To tackle the calibration redundancy, 




\fi



