\section{Problem Formulation}
% \chang{what behavior indicators are calibrated? what's the target of calibration? evaluation: reproduce loss}
Given an order flow~$x$ of a period (such as a historical trading day) whose market state is~$s$, 
we want to configure a multi-agent simulation system~$M_h$ with its behavior variables, denoted as a vector~$b$, 
so the order-generation mechanism of the simulation system can reflect the reality of the target period as possible, i.e.~simulation fidelity. 
Note that~$h$ refers to the model assumption of the multi-agent system~$M_h$. 
For example, the system in~\cite{chiarella2009impact} assumes that a market is composed of fundamentalists, chartists, and noise traders. 
Although these assumptions may have various impacts on the simulated order flow~\cite{lebaron2006agent}, 
our proposed calibration approach is general to multi-agent market simulation systems and thus is orthogonal to the model assumption~$h$.

Previous works have considered order-flow reconstruction~\cite{bai2022efficient, cao2022dslob, storchan2021learning} to evaluate how well the calibrated behavior variables reflect reality
\begin{equation}
    V(x,\hat{x})=\left\lVert f(x)-f(\hat{x}) \right\rVert_n,
\label{eq: reconstruction_fidelity}
\end{equation}
where~$\hat{x}=M_h(b)$ and~$f(\cdot)$ extracts order-flow features such as h-index and volatility clustering~\cite{vyetrenko2020get}.
Intuitively, 
the multi-agent system~$M_h$ should be able to generate similar order flow~$\hat{x}$ to the target~$x$ if the configured behavior variables~$b$ reflects reality. 
% However, the previous calibration approaches are search-based.
% The calibration, for each trading period, involves massive rounds of simulations that may take hours of computation for common devices (such as a PC). 
% This makes multi-agent simulation time-consuming, computationally intensive, and unfit for large-scale experiments. 
% Additionally, 
But order-flow reconstruction may not be sufficient to indicate simulation fidelity due to the redundancy issue of calibration~\cite{bai2022efficient}, 
where multiple sets of behavior variables can reconstruct similar order flows; that is, $\exists b\in \mathcal{B}_0\subset \mathcal{B}$ and $|\mathcal{B}_0|>1$, so that
\begin{equation}
    V\left(M_h(b), x\right)< \epsilon.
\end{equation}
Here,~$\mathcal{B}$ refers to the space of behavior variables and~$\epsilon$ is the threshold within which order-flow reconstruction is satisfiable. 
For example, having one agent place ten orders and ten agents each place one order can lead to an identical impact on the order flow, while the trading strategies behind the coincidence can be entirely different. 


To push calibration closer to simulation fidelity, 
we argue to additionally consider the temporal correlation of the calibrated behavior variables:
1) Without a striking exogenous intervention~(such as a Black Swan Event), 
the overall behaviors of investors usually exhibit smooth and continuous variations over time; that is, 
when calibrating for period~$t$, 
\begin{equation}
    \mathop{argmin}_{b^*\in \mathcal{B}_0} \left\lVert b^{(t)}-b^{(t+1)} \right\rVert_n.
\end{equation}
2) Similar market states usually indicate similar trading environments, so the overall behaviors of the investors should be similar under those market states; that is,
\begin{equation}
    \mathop{argmax}_{b^*\in \mathcal{B}_0} Corr\left(b^{(t)}, s^{(t)}\right),
\end{equation}
where~$Corr(\cdot)$ calculates correlation. 

