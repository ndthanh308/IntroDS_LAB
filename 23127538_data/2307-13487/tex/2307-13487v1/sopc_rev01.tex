%% ****** Start of file apsguide4-2.tex ****** %
%%
%%   This file is part of the APS files in the REVTeX 4.2 distribution.
%%   Version 4.2b of REVTeX, December 2018.
%%
%%   Copyright (c) 2019 The American Physical Society.
%%
%%   See the REVTeX 4.2 README file for restrictions and more information.
%%
%\documentclass[twocolumn,secnumarabic,amssymb, nobibnotes, aps, prd]{revtex4-2}
\documentclass[twocolumn,secnumarabic,amssymb, nobibnotes, aps, prd]{revtex4-2}
%\usepackage{acrofont}%NOTE: Comment out this line for the release version!
\newcommand{\revtex}{REV\TeX\ }
\newcommand{\classoption}[1]{\texttt{#1}}
\newcommand{\macro}[1]{\texttt{\textbackslash#1}}
\newcommand{\m}[1]{\macro{#1}}
\newcommand{\env}[1]{\texttt{#1}}
\setlength{\textheight}{9.5in}
%=== Sugimoto added 2022/05/06
\usepackage{graphicx}% Include figure files
\graphicspath{{./picture/}}
\usepackage{amsmath,amssymb}
\usepackage{mathtools}
\usepackage{physics}
\usepackage{color}
\usepackage{upgreek}% non-italic Greek symbols
\usepackage{hyperref}  
\usepackage{bm}
\newcommand{\micron}{{\upmu\mathrm{m}}}   

\newcommand{\gc}[1]{\textcolor{black}{#1}}

\newcommand{\rc}[1]{\textcolor{black}{#1}}
\newcommand{\RC}[1]{\textcolor{black}{#1}}

\begin{document}

%\title{Positron generation with self-organized photon collider driven by ultra intense laser plasma interaction}%
\title{Positron generation and acceleration in a self-organized photon collider\\ enabled by an ultra-intense laser pulse}%

\author{K. Sugimoto$^{1,2}$, Y. He$^{3}$, N. Iwata$^{2,4}$, I-L. Yeh$^{5}$, K. Tangtartharakul$^{3}$, A. Arefiev$^{3}$, and Y. Sentoku$^{2}$}

%\email[Corresponding authors: ]{sugimoto-k@ile.osaka-u.ac.jp}

\affiliation{$^{1}$Department of Physics, Graduate School of Science, Osaka University, 1-1 Machikanecho, Toyonaka, Osaka 560-0043, Japan}
\affiliation{$^{2}$Institute of Laser Engineering, Osaka University, 2-6 Yamadaoka, Suita, Osaka 565-0871, Japan}
\affiliation{$^{3}$Department of Mechanical and Aerospace Engineering, University of California at San Diego, La Jolla, CA 92093, United States of America}
\affiliation{$^{4}$Institute for Advanced Co-Creation Studies, Osaka University, 1-1 Yamadaoka, Suita, Osaka 565-0871, Japan}
\affiliation{$^{5}$Department of Physics, University of California at San Diego, La Jolla, CA 92093, United States of America}

\date{December 2022}%

\begin{abstract}
\RC{We discovered a simple regime where a near-critical plasma irradiated by a laser of experimentally available intensity can self-organize to produce positrons and accelerate them to ultra-relativistic energies. The laser pulse piles up electrons at its leading edge, producing a strong longitudinal plasma electric field. The field creates a moving gamma-ray collider that generates positrons via the linear Breit-Wheeler process -- annihilation of two gamma-rays into an electron-positron pair.  At the same time, the plasma field, rather than the laser, serves as an accelerator for the positrons. The discovery of positron acceleration was enabled by a first-of-its-kind kinetic simulation that generates pairs via photon-photon collisions. Using available laser intensities of $10^{22}$\,$\rm W/cm^2$, the discovered regime can generate a GeV positron beam with divergence angle of $\sim10^{\circ}$ and total charge of 0.1\,pC. The result paves the way to experimental observation of the linear Breit-Wheeler process and to applications requiring positron beams.}

%The linear Breit-Wheeler (BW) process or the annihilation of two photons into an electron-positron pair is one of elementary processes for matter and antimatter generation in the Universe. We have discovered a regime where a single laser pulse traveling through a near-critical plasma creates a self-organized gamma-ray collider and an adjoining accelerator for the generated positrons. The energies of positrons are boosted by a forward-directed electric field that is driven by photon pressure at the leading edge of the laser pulse. The advantage of our regime is that it relies simply on the collective phenomena of relativistic electromagnetic waves in plasmas. The currently available laser intensities are sufficient to achieve an appreciable positron yield and the accelerated positrons are much more energetic than the laser-accelerated electrons, reaching GeV energies at laser intensity of $10^{22}$\,$\rm W/cm^2$ with a narrow divergence angle of $\sim10^{\circ}$ and total charge of about 0.1\,pC. 

\end{abstract}

\maketitle

In astrophysics, creation of matter from light is ubiquitous, playing an important role for various astrophysical objects~(e.g. see \cite{medin.2010, Beloborodov_2008, Philippov_2018, uzdensky.2022, Hakobyan_2023}). The advent of ultra-high-intensity laser facilities~\cite{danson2019petawatt_review, ELI_NP_Tanaka, Gist_multi-PW_10_22} promises to enable, for the first time, creation of electron-positrons pairs from light alone on a macroscopic scale in laboratory. If successfully implemented, this capability will open a new area of QED research~\cite{RevModPhys.84.1177,RevModPhys.94.045001,zhang.pop.2020} and it will enable laboratory studies of astrophysically relevant electron-positron plasmas~\cite{MP3}. The ability to generate positrons by a laser is also likely to impact the research on laser-driven positron acceleration. Currently, positrons are produced by an external source and the focus is on finding augmented configurations that facilitate positron acceleration \cite{Gessner_ncomm_2016,Zhou_prl_2021,Silva_prl_2021,Vieira_prl_2014}. 


In the context of pair production from light alone, it is important to distinguish between the non-linear~\cite{Burke_prl_1997} and linear~\cite{Breit_pr_1934} Breit-Wheeler (BW) processes. The nonlinear BW or the multiphoton process is the decay of a $\gamma$-ray propagating through a laser pulse into a pair. The decay involves multiple coherent optical photons. The linear BW or the two-photon process is the annihilation of two energetic $\gamma$-rays that leads to pair production. The setups that many pairs via the nonlinear BW~\cite{Ridgers_2012, Vranic_sci_2018, Xing-Long_mre_2019, Jian-Xun_ppcf_2019, Zhao_nat_2022,Mercuri-Baron_2021,martinez.prab.2023} require a laser intensity in excess of $10^{23}$\,$\rm W/cm^2$. The two-photon process has no laser intensity requirement, but it does require a dense population of energetic $\gamma$-rays to overcome the smallness of the cross-section, \rc{$\sigma_{\gamma \gamma} \sim 10^{-25}$~cm$^2$}, and the energy threshold. \RC{A laser-irradiated plasma can  efficiently generate a $\gamma$-ray beam~\cite{nakamura.prl.2012, ji.prl.2014, Stark_prl_2016}, so colliding in vacuum two such beams (produced by two different laser) is a possible approach to produce pairs~\cite{Ribeyre_pre_2016,wang.PhysRevApp.2020}. The inherent $\gamma$-ray beam divergence requires the targets generating $\gamma$-rays to be close to each other and makes experimental implementation challenging. A conceptually different approach is to generate and collide $\gamma$-ray beams inside one target~\cite{Yutong_nat_2021}. It not only allows to overcome the divergence and thus boost the pair yield~\cite{Yutong_nat_2021}, but, more importantly, it offers an unexplored opportunity to accelerate the linear BW positrons. If the positrons can be accelerated and collimated, then this would facilitate their detection, making a first laboratory observation of the linear BW process possible, and enable their use for applications like positron annihilation lifetime spectroscopy \cite{audet.prab.2021,krause.1999}.}



%To overcome the divergence, the two beams can be generated and collided inside one target~\cite{Yutong_nat_2021}. However, the real step forward is enabled by recent realization that a colliding $\gamma$-ray population can be created by a single laser beam~\cite{Yutong_njp_2021}.

%Each beam is generated by a separate laser and a separate target. The inherent beam divergence requires the targets to be close to each other and makes experimental implementation challenging. To overcome the divergence, the two beams can be generated and collided inside one target~\cite{Yutong_nat_2021}. However, the real step forward is enabled by recent realization that a colliding $\gamma$-ray population can be created by a single laser beam~\cite{Yutong_njp_2021}.



%%[OLD VERSION] In this Letter, we report a regime where a single laser pulse with an experimentally achievable intensity of $\sim 10^{22}$W/cm$^2$ creates a \emph{self-organized photon collider and an adjoining positron accelerator} while traveling through a near-critical plasma. Using fully kinetic simulations that include radiation dynamics, we examine the mechanisms of electron acceleration that \rc{make} the $\gamma$-ray collider \rc{possible} and \rc{the mechanism of acceleration for generated positrons}. The simulation shows that the $\gamma$-rays emitted in the plasma produce over $10^7$ pairs via the linear BW process while the nonlinear BW process produces no pairs at the considered laser intensity. About 10\% of the positrons experience forward acceleration and form an ultra-relativistic (GeV-level) beam with a divergence angle of $10^{\circ}$. Our regime uses a simple setup and it only requires a single laser pulse with a peak intensity that is already accessible at most flagship laser facilities, e.g. ELI~\cite{ELI},  CoReLS~\cite{CoReLS}, and, ZEUS~\cite{ZEUS}. Our regime is much less demanding than its predecessors. Rather than requiring the user to create optimal conditions for pair production, our regime relies on natural collective phenomena of the propagation dynamics of intense electromagnetic waves in near-critical plasmas. 

\RC{In this Letter, we present a simple but previously unknown regime where a dense plasma irradiated by a laser of experimentally achievable intensity self-organizes to produce positrons from light alone and accelerate them to ultra-relativistic energies. The laser pulse piles up electrons at its leading edge, producing a strong longitudinal plasma electric field that moves with the pulse. The field creates a moving $\gamma$-ray collider that generates positrons via the linear BW process and, at the same time, serves as an accelerator for the produced positrons. The discovery of the new positron acceleration mechanism and the synergistic interplay between the photon collider and the plasma accelerator was enabled by a first-of-its-kind kinetic simulation that generates pairs via photon-photon collisions. This work builds on an important observation based on post-processed photon data that a single laser-pulse can generate a colliding population of $\gamma$-rays in a dense structured plasma~\cite{Yutong_njp_2021}. We find that the linear BW process produces about $10^7$ pairs at $3 \times 10^{22}$\,$\rm W/cm^2$, whereas the nonlinear BW process produces no pairs at all. About 10\% of the positrons experience the forward acceleration and form a GeV beam with a divergence angle of $10^{\circ}$. The advantage of our regime is that it uses a simple setup and requires only a single laser with intensity already accessible at ELI~\cite{ELI} and CoReLS~\cite{CoReLS}. }
%, and, ZEUS~\cite{ZEUS}. }

The laser-plasma interaction is \RC{self-consistently} simulated in 2D-3V with the PIC code PICLS~\textcolor{black}{\cite{Sentoku_jcp_2008}} that includes a radiation transport module~\cite{Sentoku_pre_2014,Royle_pre_2017} for energetic photons emitted via synchrotron radiation~\cite{Pandit_pop_2012} and Bremsstrahlung~\cite{Sentoku_pop_1998}. \RC{We have developed a module for simulating the linear BW process [see Supplemental Materials], making PICLS the first PIC code capable of generating linear BW pairs during the laser-plasma interaction and thus suitable for studies of positron dynamics.} In our setup, a 25~fs, $3 \times 10^{22}$~W/cm$^{2}$ laser pulse irradiates a dense uniform carbon plasma (see Supplemental Materials for simulation parameters). {We normalize all electric fields, $\bm{E}$, and use a dimensionless quantity $\bm{a} = |e| \bm{E}/m_{e} c \omega_0$ instead}, where $e$ and $m_e$ are the electron charge and mass, $c$ is the speed of light, and $\omega_{0}$ is the laser frequency corresponding to vacuum wavelength $\lambda = 0.8~\micron$. The laser amplitude is $a_L = 120$. This laser makes electrons ultra-relativistic and renders a plasma with electron density $n_e$ less than $\gamma_L n_c \sim a_L n_c$ transparent, where $\gamma_L\equiv \sqrt{1+a_L^2/2}$ is the electron Lorentz factor for ponderomotive energy~\cite{Wilks_prl_1992} and $n_c = m_e \omega_0^2 / 4\pi e^2$ is the classical critical density. In our main simulation, the initial electron density is $n_{e0} = 2.8 n_c \ll a_L n_c$, so the laser easily propagates into the plasma.


%The plasma density is $n_{e0} = 2.8 n_c \ll a_L n_c$, so the laser easily propagates into the plasma and interacts with a large population of electrons. %, which is important for efficient generation of gamma-rays. 

% Figure environment removed  

Figure~\ref{fig1} illustrates key aspects of the laser-plasma interaction. All snapshots are taken when the laser pulse reaches $x = 30~\micron$. The corresponding time is $t=117$\,fs, with $t = 0$\,fs being the time when the pulse reaches the target. Figure~\ref{fig1}(a) shows the normalized transverse electric field $a_y$ that is dominated by the field of the laser. %To highlight the effect of the plasma, the dashed curves mark the expected laser beam waist in the absence of the target. 
Due to the relativistic self-focusing, the beam remains tightly focused after having traveled a distance greater than the Rayleigh length ($l_{R}=\pi w_{0}^{2}/\lambda \simeq 25~\micron$ for a focal spot with radius $w_0=2.5~\micron$). The dashed curves mark the expected beam waist in the absence of the target. The self-focusing also increases the laser amplitude to $a_y=150$. The beam becomes fully depleted after propagates $70~\micron$ into the plasma. The profiles of electron density and generated azimuthal magnetic field are shown in Figs.~\ref{fig1}(b)\&(c). Transverse electron expulsion by the ponderomotive force produces a density pileup ($n_e \sim 10 n_c$) at the periphery of the beam that helps guide the laser. The electrons remaining in the beam accelerate forward in the laser field and form longitudinal current. The current generates a strong quasi-static magnetic field $B_z$~\cite{Stark_prl_2016} whose peak strength is 30\% of that for the laser magnetic field. Figure~\ref{fig1}(c) shows the field profile while providing an additional figure of merit $b_{z} = \omega_c / \omega_0$, where $\omega_c = |e| B_{z} / m_e c$ is the cyclotron frequency. 

% Figure environment removed 

The B-field plays a key role in generating forward-directed $\gamma$-rays. It transversely confines the electrons that are accelerated and pushed forwarded by the laser. \RC{The B-field defects electrons forward instead of causing the conventional rotation} and the deflections change the orientation of the transverse velocity $v_{\perp}$ with respect to $E_{\perp}$ of the laser. If their frequency is comparable to the Doppler-shifted frequency of the laser, then $v_{\perp}$ remains antiparallel to $E_{\perp}$ as the laser field and the electron oscillate. This mechanism of direct laser acceleration assisted by the plasma B-field~\cite{gong.PRE.2020} produces $\sim 500$\,MeV electrons with a forward momentum of 1000\,$m_e c$. They are located in Fig.~\ref{fig1}(e) at $22~\micron \leq x \leq 28~\micron$. The deflections of the electrons by the magnetic field has another important effect -- they cause the electrons to emit MeV $\gamma$-rays in the direction of laser propagation~\cite{Stark_prl_2016, Jansen_ppcf_2018, wang.PhysRevApp.2020}. 

Due to the high plasma density, the laser also generates a strong longitudinal \RC{plasma} electric field that is essential for the production of backward-directed $\gamma$-rays. This is a charge-separation field that arises as the leading edge of the laser pulse sweeps up plasma electrons. \RC{Its peak amplitude is 25\% of $a_y$ and it dominates over the oscillating longitudinal field of the laser.} The positive plasma field is clearly visible in Fig.~\ref{fig1}(d) at $x \approx 29.5~\micron$. After initial forward acceleration to $p_x\sim 200\,m_e c$, the electrons swept up by the leading edge of laser pulse slow down under the influence of $a_x$ %the longitudinal field 
and then re-accelerate in the backward direction to $p_x\sim -100\,m_ec$. \rc{These electrons emit backward-directed photons. In contrast to the forward-moving electrons, the emission is induced by the laser field~\cite{Koga_PRE_2004} that is much stronger than the plasma magnetic field. This makes the emission more efficient, causing the electrons to quickly lose a large portion of their energy, as seen in Fig.\ \ref{fig1}(e) at $x > 28~\micron$. The emission process accompanies laser propagation since the population of backward-moving electrons is constantly replenished by $a_x$ that is moving forward with the laser pulse.} 

%%%In addition to the magnetic field, the laser generates a longitudinal electric field that is essential for the production of backward-directed $\gamma$-rays. Figure~\ref{fig1}(d) shows that there is a positive electric field, $a_x$, at the leading edge of the laser pulse ($x \approx 29.5~\micron$) whose amplitude is 25\% of the laser amplitude. This charge-separation field arises because the leading edge sweeps up the electrons it encounters in its path while the ions are left behind. After initial forward acceleration to $p_x\sim 200\,m_e c$, the swept up electrons slow down under the influence of $a_x$ %the longitudinal field and then re-accelerate in the backward direction to $p_x\sim -100\,m_ec$. These features are visible in Fig.\ \ref{fig1}(e) at $x > 28~\micron$. The backward-accelerated electrons are moving in the direction opposite to the laser propagation and efficiently emit backward-directed photons~\cite{Koga_PRE_2004} while experiencing strong radiation friction. The population of the swept up electrons is constantly replenished, so that the field propagates with the laser and continues generating photons. 


%Figure \ref{fig1}(d) shows the distribution of the electric field in the $x$-direction averaged over one laser oscillation period. A positive electrostatic field with peak amplitude $a_{x}\sim$ 30 appeared at the pulse front ($x=29.5$\,$\micron$). Its widths in $x$- and $y$-direction are about 2\,$\micron$ and 4\,$\micron$, respectively. This electrostatic field is originated from the electron accumulation at the pulse front \cite{Yutong_njp_2021}. The $a_{x}$ reaches about 25\,$\%$ of the laser electric field and it can accelerate electrons backward from the pulse front. Fig.\ \ref{fig1}(e) shows the profile of electron phase $x$-$p_x$ and electric field distribution in the region marked in plot (d). The field is obtained by averaging in $|y| \le 0.5~\micron$. In the region $x$=22-28\,$\micron$, the electrons are accelerated by the direct laser acceleration (DLA) \cite{Pukhov_pop_1998} over 1000\,$m_e c$ corresponding to energy $\sim 500$\,MeV. At the pulse front electrons are pushed up to $p_x\sim 200\,m_e c$. These electrons correspond to the accumulated electrons seen in Fig.\ \ref{fig1}(b). Behind the accumulation, electrons are accelerated backward around $x=29$\,$\micron$ with momentum $p_x\sim -100\,m_ec$ by the positive electrostatic field $a_x$. These electrons soon interact with strong laser field in the counter configuration and thus emit photons backward efficiently \cite{Koga_PRE_2004} as discussed in Fig.\,2.

 
The two photon populations form a moving $\gamma$-ray collider. 
%Figure \ref{fig2} summarizes key features of the resulting photon collider. 
Figures~\ref{fig2}(a)\&(b) show photon spectra versus the polar angle $\theta$ %with respect to the positive $x$-direction 
in the region where the energy density of forward-  and backward-moving photons ($|\theta| \leq \pi/2$ and $|\theta| > \pi/2$) overlap ($22~\micron \le x \le 27.5~\micron$; $|y| \le 1~\micron$). The corresponding energy density plots are shown in Figs.~\ref{fig2}(c)\&(d). The Bremsstrahlung that plays a secondary role is included for completeness. The synchrotron emission converts 40\% of the laser energy into photons over the entire simulation (vs. 2\% for Bremsstrahlung). The linear BW process has a threshold of $\epsilon_{\gamma 1} \epsilon_{\gamma 2} > m_e^2 c^4 \approx 0.26~\mbox{MeV}^2$, where $\epsilon_{\gamma 1,2}$ are the energies of colliding photons. Therefore, linear BW pairs are mainly produced by forward-moving photons with $0.5~\mbox{MeV} \lesssim \epsilon_{\gamma} \lesssim 100~\mbox{MeV}$ colliding with backward-moving photons with $10~\mbox{keV} \lesssim \epsilon_{\gamma} \lesssim 1~\mbox{MeV}$. The photon densities in these two groups are comparable, with $n_{\gamma} \sim 10^{22}$\,cm$^{-3}$.
%\rc{Forward-moving photons with energies $0.5~\mbox{MeV} \lesssim E \lesssim 100~\mbox{MeV}$ and backward-moving photons with $10~\mbox{keV} \lesssim E \lesssim 1~\mbox{MeV}$ mainly contribute to the pair creation with exceeding the threshold energy product $m_e^2 c^4 \approx 0.26~\mbox{MeV}^2$. Densities of these photon groups are both observed as $n_{\gamma} \sim 10^{22}$\,cm$^{-3}$. 
The probability for a backward-moving photon to produce a pair is 
$\sigma_{\gamma \gamma} n_{\gamma} l \sim 10^{-6}$, where $l \sim 10~\micron$ is the length of the forward-moving photon cloud. The total number of backward-photons is $n_{\gamma} S L \sim 10^{13}$, where $L \approx 70~\micron$ is the laser depletion length and $S \approx 25~\micron^2$ is the cross-section of the cloud, assuming the length in the third dimension is the laser spot diameter. The predicted pair yield is $10^7$, which matches the yield evaluated using the developed module for the linear BW process~\textcolor{black}{\cite{Hubbell_jpcrfd_1980}}. \RC{A similar module implemented by us into the PIC code EPOCH~\cite{Epoch} that has a different approach for treating emitted photons produced a comparable yield.} A time integrated density of the pair-production events is shown in Fig.\,2(e). 



\RC{The $\gamma$-ray collider is moving with the laser, continuously producing positrons with a mildly relativistic momentum $p \sim m_e c$ within the laser pulse [see Fig.\ \ref{fig3}(a)]. The positron dynamics is strongly influenced by the laser and plasma fields,  with two distinct populations emerging over time: forward-moving positrons whose energies reach 1~GeV and backward-moving positrons whose energies reach 100~MeV. Figures~\ref{fig3}(c)\&(d) show terminal positron distributions in the energy-angle space for the forward- and backward positrons. Figure~\ref{fig3}(e) shows the electron and positron energy spectra, distinguishing the linear BW and Bethe-Heitler [see Supplemental Material] positrons to emphasize the dominant role of the linear BW process. A striking feature of Fig.~\ref{fig3}(e) is that the peak energy of forward positrons exceeds the peak energy of forward electrons by a factor of two. The electrons gain their energy from the laser via the direct laser acceleration assisted by the plasma magnetic field~\cite{gong.PRE.2020}, but the positrons are not able to do that because they are positively charged. The plasma magnetic field deflects positrons backward rather than forward, which causes the formation of the backward positron population.}


%%%The positron dynamics that follows is influenced by the superposition of laser and plasma fields. It depends on the `initial' positron position with respect to the laser pulse and laser wave fronts. Despite the complexity of the process, we observe that two populations emerge over time: forward-moving positrons whose energies reach 1~GeV and backward-moving positrons whose energies reach 100~MeV. 


%\RC{The $\gamma$-ray collider is moving with the laser, continuously producing positrons within the laser pulse [see Fig.\ \ref{fig3}(a)].} They are created with a mildly relativistic momentum $p \sim m_e c$, but it quickly increases as a result of the positron interaction with the laser. The positron dynamics that follows is influenced by the superposition of laser and plasma fields. It depends on the `initial' positron position with respect to the laser pulse and laser wave fronts. Despite the complexity of the process, we observe that two populations emerge over time: forward-moving positrons whose energies reach 1~GeV and backward-moving positrons whose energies reach 100~MeV.

% Figure environment removed  

%%%Particle tracking points to the crucial role of the longitudinal plasma electric field $a_x$ at the leading edge of the laser pulse [see Figs.~\ref{fig3}(a)\&(b)]. The leading edge and the spike in $a_x$ move forward with velocity $u/c \approx 0.8$. Depletion of the laser pulse makes $u$ lower than the group velocity~\cite{Sentoku_fst_2006} estimated as $v_g/c \approx \sqrt{1-\omega_{pe}^2/\gamma_L \omega_0^2} \approx 0.98$, where $\omega_{pe}=\sqrt{4\pi e^2 n_{e0}/m_e}$ is the plasma frequency for the initial electron density. It is instructive to compare $u$ and $v_g$ to $v_x$ of an ultra-relativistic positron moving at an angle $\theta$ to the axis of the laser beam. We have $v_x \approx c \cos(\theta) > u$ for $|\theta| \leq 37^{\circ} $, whereas $v_x \approx c \cos(\theta) > v_g$ for $|\theta| \leq 11^{\circ} $. The 20\% reduction in $u$ compared to $v_g$ increases the range of $\theta$ for the positrons that can move faster than the leading edge by a factor of three. These positrons that are pushed forward by the laser pulse catch up with the spike of $a_x$ and experience additional energy gain. The resulting increase in $p_x$ is seen in Fig.~\ref{fig3}(b) that shows a snapshot of positrons in the $x$-$p_{x}$ space at $t = 217$~fs, i.e. 67~fs after the snapshot in Fig.\ \ref{fig3}(a). \rc{This acceleration mechanism enabled by $a_x$ only works for positrons, whereas the same field pulls electrons backward creating the backward emission.} 

\RC{We tracked the energetic forward-moving positrons and found that they gain most of their energy (80\%) from the strong forward-moving longitudinal plasma electric field, thus discovering a new positron acceleration mechanism. Figure~\ref{fig3}(b) confirms that the energetic positrons are surfing with the spike in $a_x$. The positrons continue accelerating until they overtake the laser pulse or leave the acceleration region in lateral direction. The acceleration by $a_x$ only works for positrons, whereas the same field pulls plasma electrons backward creating the backward emission that contributes to the photon collider.}

\RC{The discovered acceleration mechanism produces $10^6$ or $0.1$~pC of positrons with energies above 100~MeV and  average divergence angle $|\theta| \sim 10^{\circ}$. The high plasma density is not only important for generating strong $a_x$ needed for positron acceleration (no $a_x$ spike is produced at subcritical densities~\cite{martinez.prab.2023}), but it is also crucial for achieving a high number of accelerated positrons. Positrons must catch up with $a_x$ to experience the acceleration, but this is hard to achieve if $a_x$, whose speed is $u$, moves too fast. In a low density plasma, $u$ is close to the group velocity $v_g/c \approx \sqrt{1-n_e/\gamma_L n_c}$~\cite{Sentoku_fst_2006}. In a dense plasma, $u$ is lower than $v_g$ due to laser depletion, which enables more positrons to experience acceleration. In our case, $u/c \approx 0.8$, but $v_g/c \approx 0.98$. Only relativistic positrons with $v_x \approx c \cos \theta > u$ are able to catch up with $a_x$. We have $v_x \approx c \cos \theta > u$ for $|\theta| \leq 37^{\circ} $, whereas $v_x \approx c \cos \theta > v_g$ for $|\theta| \leq 11^{\circ} $. The 20\% reduction in $u$ compared to $v_g$ increases the range of $\theta$ by a factor of three and thus significantly increase the number of positrons that can catch up with $a_x$.}


%%We find that the leading edge of the laser pulse and the spike in $a_x$ move forward with velocity $u/c \approx 0.8$. Due to the laser depletion in the dense plasma, $u$ is noticeably lower than the group velocity~\cite{Sentoku_fst_2006} estimated as $v_g/c \approx \sqrt{1-\omega_{pe}^2/\gamma_L \omega_0^2} \approx 0.98$, where $\omega_{pe}=\sqrt{4\pi e^2 n_{e0}/m_e}$ is the plasma frequency for the initial electron density. 


%%It is instructive to compare $u$ and $v_g$ to $v_x$ of an ultra-relativistic positron moving at an angle $\theta$ to the axis of the laser beam. We have $v_x \approx c \cos(\theta) > u$ for $|\theta| \leq 37^{\circ} $, whereas $v_x \approx c \cos(\theta) > v_g$ for $|\theta| \leq 11^{\circ} $. The 20\% reduction in $u$ compared to $v_g$ increases the range of $\theta$ for the positrons that can move faster than the leading edge by a factor of three. %%%These positrons that are pushed forward by the laser pulse catch up with the spike of $a_x$ and experience additional energy gain. The resulting increase in $p_x$ is seen in Fig.~\ref{fig3}(b) that shows a snapshot of positrons in the $x$-$p_{x}$ space at $t = 217$~fs, i.e. 67~fs after the snapshot in Fig.\ \ref{fig3}(a). 


%%%The discussed longitudinal acceleration generates about $10^6$ or $0.1$~pC of positrons with energies above 100~MeV and averaged divergence angle $|\theta| \sim 10^{\circ}$. Such a narrow divergence angle is comparable to that obtained in Ref.~\cite{Zhao_nat_2022} using a more demanding setup with two colliding lasers one of which has helical wave fronts (Laguerre-Gaussian beam). 

%%%The positrons moving forward continue accelerating, i.e, ``surfing'', until they overtake the laser pulse or leave the acceleration region in lateral direction. Figures~\ref{fig3}(c)\&(d) show terminal positron distributions in the energy-angle space, where we separately plotted the distributions for forward- and backward positrons. Figure~\ref{fig3}(e) summarizes the electron and positron spectra, distinguishing the linear BW and Bethe-Heitler [see Supplemental Material] positrons to emphasize the dominant role of the linear BW process. The discussed longitudinal acceleration generates about $10^6$ or $0.1$~pC of positrons with energies above 100~MeV and averaged divergence angle $|\theta| \sim 10^{\circ}$. Such a narrow divergence angle is comparable to that obtained in Ref.~\cite{Zhao_nat_2022} using a more demanding setup with two colliding lasers one of which has helical wave fronts (Laguerre-Gaussian beam). 

To examine the impact of the plasma density $n_{e0}$ on the strength of $a_x$ and the positron energy gain, we performed extra simulations with $n_{e0} / n_c =$ 0.5, 1.0, 1.75, and 5.6. Figures~\ref{fig5}(a)\&(b) show $a_x$ at the leading edge of the pulse and the energy gain by forward-moving positrons versus $n_{e0}$. We average $a_x$ over $y$ at the time when the laser peak intensity reaches the pulse leading edge to obtain the values in Fig.~\ref{fig5}(a). The energies in Fig.~\ref{fig5}(b) were averaged over the top 5, 10, and 20 percent of the positron spectrum to confirm the trend. The discovered regime is robust and can be achieved over a wide range of plasma densities. For $n_c \le n_{e0} \le 5.6\,n_c$, the number of positron with energies above 100\,MeV and $|\theta| \lesssim 10^\circ$ is consistently about $10^6$. At $n_{e0} / n_c = 0.5$, the speed of $a_x$ is very close to $c$, which makes $a_x$ too fast to effectively accelerate positrons that are originally only mildly relativistic. 

We next use estimates for $a_x$ and the positron energy gain to determine their scaling at high $n_{e0}$. The electron density pileup responsible for $a_x$ is sustained due to force balance, $0=F_p + F_s$, between the laser ponderomotive force $F_p = -m_e c^{2}\nabla_x \gamma_L$ and $F_s= - a_x m_e c \omega_0$. We estimate that $\gamma_L / |\nabla_x \gamma_L | \simeq l_{\rm skin}$, where $l_{\rm skin}=\sqrt{\gamma_L} c/\omega_{pe}$ is the relativistic skin depth. Taking into account that $a_L \gg 1$, we obtain
\begin{equation}
	\label{eq1}
	a_{x} \simeq \sqrt{\gamma_L n_e / n_c},
\end{equation}
where $n_e$ is the density of the electron pileup. The shaded area in Fig.\ \ref{fig5}(a) shows $a_x$ from Eq.\,\eqref{eq1} for $a_L=120$ and $2n_{e0} \le n_e \le 6n_{e0}$. The latter is the entire range of $n_e$ observed in the simulations, with $n_e \approx 2n_{e0}$ for $n_{e0}=5.6 n_c$ and $n_e \approx 6n_{e0}$ for $n_{e0}= 0.5n_c$. The momentum gain, $\Delta p_{e^{+}}$, from $a_x$ can be estimated by integrating the positron equation of motion $dp_{e^{+}}/dt \simeq m_ec \omega_0 {\bar a}_x$ over the acceleration time interval $\Delta t_{\rm acc}$, where ${\bar a}_x=a_x/2$ is the average field amplitude in the acceleration region. The length of the region with positive $a_x$ is the width of the electron pileup, $l_{\rm skin}$, plus the length of the positively-charged electron cavity, $l_{\rm cav}$, formed behind the pulse leading edge. We estimate $l_{\rm cav}$ from the charge conservation: $\left( n_{e0}-n_{c} \right) l_{\rm cav} = \left( n_e-n_{e0} \right) l_{\rm skin}$ for $n_{e0}>n_c$. The acceleration region is moving forward with velocity $u$ while the positron velocity is $v_x$, so that $\Delta t_{\rm acc} \equiv (l_{\rm cav} + l_{\rm skin})/(v_x-u)$. Assuming an ultra-relativistic positron, we set $v_x \sim c$. After taking into account that $\gamma_L n_{c} \gg n_e$ for $a_L \gg 1$, we find that that the positron momentum gain is 
	\begin{equation}
	\label{eq4}
	 \Delta p_{e^{+}} \simeq 
 	\frac{\gamma_L m_ec}{2}\frac{1}{1-u/c} \ 
	 \frac{n_{e} - n_c}{n_{e0} - n_c}.
	\end{equation}
%where $\Delta p_{e^{+}} = p_{e^{+}} - p_{e0^{+}}$ and $p_{e0^{+}}$ is the positron momentum before the acceleration by $a_x$.
Equation~\eqref{eq4} gives $\Delta p_{e^{+}}/m_ec\simeq 1200$ for $n_{e0}=2.8\,n_c$, $a_L = 120$, $u=0.8c$, and $n_e=4n_{e0}$, reproducing the significant positron momentum increase at the pulse leading edge seen in Fig.\,\ref{fig3}(b). The energy gain, $\Delta \epsilon_{e^{+}} = c\Delta p_{e^{+}}$, obtained from Eq.~\eqref{eq4} is shown in Fig.\,\ref{fig5}(b) with a dashed curve. For high densities, $\Delta \epsilon_{e^{+}}$ has a weak dependence on $n_{e0}$, because the increase in $a_x$ is counteracted by the reduction in the acceleration time caused by lower $u$.   


%%%Figure~\ref{fig5}(a) shows the normalized longitudinal electric field $a_x$ at the leading edge of the pulse. We calculate $a_x$ by transverse averaging the field at the time when the laser peak intensity reaches the pulse leading edge. As expected, $a_x$ increases with $n_{e0}$. To find the scaling, we note that the electron density pileup responsible for $a_x$ is sustained due to force balance, $0=F_p + F_s$, between the laser ponderomotive force $F_p = -m_e c^{2}\nabla_x \gamma_L$ and $F_s= - a_x m_e c \omega_0$. We estimate that $\gamma_L / |\nabla_x \gamma_L | \simeq l_{\rm skin}$, where $l_{\rm skin}=\sqrt{\gamma_L} c/\omega_{pe}$ is the relativistic skin depth. Taking into account that $a_L \gg 1$, we obtain
%%%\begin{equation}
%%%	\label{eq1}
%%%	a_{x} \simeq \sqrt{\gamma_L n_e / n_c},
%%%\end{equation}
%%%where $n_e$ is the density of the electron pileup. The shaded area in Fig.\ \ref{fig5}(a) shows $a_x$ from Eq.\,\eqref{eq1} for $a_L=120$ and $2n_{e0} \le n_e \le 6n_{e0}$. The latter is the entire range of $n_e$ observed in the simulations, with $n_e \approx 2n_{e0}$ for $n_{e0}=5.6 n_c$ and $n_e \approx 6n_{e0}$ for $n_{e0}= 0.5n_c$. 

%The shaded area in Fig.\ \ref{fig5}(a) shows Eq.\,\eqref{eq1} for $a_L=120$ and $2n_{e0} \le n_e \le 6n_{e0}$ representing the electron density accumulated at the leading edge in the simulations.  
%Namely, $n_e \approx 2n_{e0}$ is observed in the case of $n_{e0}=5.6n_c$, and $n_e \approx 6n_{e0}$ is observed in the case of $n_{e0}=0.5n_c$. 


% Figure environment removed  




%The momentum gain, $\Delta p_{e^{+}}$, from $a_x$ can be estimated by integrating the positron equation of motion $dp_{e^{+}}/dt \simeq m_ec \omega_0 {\bar a}_x$ over the acceleration time interval $\Delta t_{\rm acc}$, where ${\bar a}_x=a_x/2$ is the average field amplitude in the acceleration region. The length of the region with positive $a_x$ is the width of the electron pileup, $l_{\rm skin}$, plus the length of the positively-charged electron cavity, $l_{\rm cav}$, formed behind the pulse leading edge. We estimate $l_{\rm cav}$ from the charge conservation: $\left( n_{e0}-n_{c} \right) l_{\rm cav} = \left( n_e-n_{e0} \right) l_{\rm skin}$ for $n_{e0}>n_c$. The acceleration region is moving forward with velocity $u$ while the positron velocity is $v_x$, so that $\Delta t_{\rm acc} \equiv (l_{\rm cav} + l_{\rm skin})/(v_x-u)$. Assuming an ultra-relativistic positron, we set $v_x \sim c$. After taking into account that $\gamma_L n_{c} \gg n_e$ for $a_L \gg 1$, we find that that the positron momentum gain is 
%	\begin{equation}
%	\label{eq4}
%	 \Delta p_{e^{+}} \simeq 
% 	\frac{\gamma_L m_ec}{2}\frac{1}{1-u/c} \ 
%	 \frac{n_{e} - n_c}{n_{e0} - n_c}.
%	\end{equation}
%where $\Delta p_{e^{+}} = p_{e^{+}} - p_{e0^{+}}$ and $p_{e0^{+}}$ is the positron momentum before the acceleration by $a_x$.
%Equation~\eqref{eq4} gives $\Delta p_{e^{+}}/m_ec\simeq 1200$ for $n_{e0}=2.8\,n_c$, $a_L = 120$, $u=0.8c$, and $n_e=4n_{e0}$, reproducing the significant positron momentum increase at the pulse leading edge seen in Fig.\,\ref{fig3}(b). 

%%%The energy gain, $\Delta \epsilon_{e^{+}} = c\Delta p_{e^{+}}$, obtained from Eq.~\eqref{eq4} is shown with a dashed line in Fig.\,\ref{fig5}(b). \rc{The circle, square, and triangle markers are the average energies of forward-moving positrons that have top 5, 10, and 20 percent energies at the moment when the maximum energy appeared.} For higher densities, $\Delta \epsilon_{e^{+}}$ has a weak dependence on $n_{e0}$, since the acceleration time decreases due to lower leading edge velocity $u$ while $a_x$ increases.   At $n_{e0}=0.5\,n_c$, $u$ becomes close to $c$, so that the positrons, especially with finite angle $\theta$, could not be fully accelerated by $a_x$, making the energy boost inefficient. The number of positron coming out in the laser direction with energy greater than 100\,MeV and divergence angle $|\theta|$ of $\sim10^\circ$ is about $10^6$, insensitive to the initial target density in the regime $n_c \le n_{e0} \le 5.6\,n_c$. 



%In summary, we demonstrate the positron generation via the linear BW process in a self-organized plasma structure created by an intense laser pulse. The intense laser pulse simultaneously produces energetic electrons moving forward and backward, hence a photon collider is naturally realized during the propagation in near critical density plasmas. 

%\rc{We employ the uniform plasma for simplification only. A simulation with $n_e$ ramping up from 0.5 to $3 n_c$ over $60~\micron$ has a pair yield that is also $10^7$. Additional 3D simulations with PICLS (see Supplemental Materials) and EPOCH~\cite{Epoch} have $n_{\gamma}$ that is similar to that in our 2D simulations, further confirming the robustness of the phenomena discussed in the Letter.} 

\RC{In summary, we discovered a robust regime where a laser-irradiated plasma self-organizes to produce positrons and accelerate them. The GeV-level positron beam can be generated using just a single laser with an experimentally available intensity. The regime requires the use of a dense plasma that can create a strong longitudinal electric field via electron pileup. The field is crucial for creating the $\gamma$-ray collider and for accelerating positrons. The positron acceleration was discovered by a first-of-its-kind simulation code generating pairs via photon-photon collisions. This code has direct relevance to astrophysics research since correct treatment of secondary pairs is one of the main problems facing modern PIC simulations of pulsars~\cite{Philippov_2018,Hakobyan_2023} The uniform density is a simplification and not a requirement. A simulation with $n_e$ ramping up from 0.5 to $3 n_c$ over $60~\micron$ has a similar pair yield of $10^7$. 3D simulations with PICLS (see Supplemental Materials) and EPOCH~\cite{Epoch} have $n_{\gamma}$ that is similar to that in our 2D simulations, confirming the robustness of the discussed phenomena.} Lastly, our regime can be instrumental in gauging the focal intensity of multi-PW lasers. At $10^{21}$\,W/cm$^2$, the positron yield is five orders of magnitude lower than at $10^{22}$\,W/cm$^2$. Therefore, the presence of energetic positrons in the laser direction can be a confirmation of laser intensity exceeding $10^{22}$\,W/cm$^2$.


%%%Lastly, there is an experimental challenge to determine the focal laser intensity of PW lasers, like ELI. Our scheme requires a laser intensity above $10^{22}$\,W/cm$^2$. In a simulation with an intensity of $10^{21}$\,W/cm$^2$ the positron yield is five orders of magnitude lower. Therefore, monitoring positrons with energies above 100\,MeV in the laser direction can be a verification of whether the intensity exceeds $10^{22}$\,W/cm$^2$ or not. 


%\noindent {\fontsize{11pt}{11pt}\selectfont {\bf Acknowledgements}} \\
This study was supported by JSPS KAKENHI Grants No. JP19KK0072, No. JP20K14439, No. JP20H00140, No. JP22J10867, JP23K03354, and JST PRESTO Grant No. JPMJPR21O1. The work by Y.H., I.-L. Y., K. T., and A. A. was supported by AFOSR (Grant No. FA9550-17-1-0382) and by National Science Foundation  – Czech Science Foundation partnership (NSF award PHY-2206777). \\



%\bibliographystyle{ieeetr}
\bibliographystyle{apsrev4-1}
\bibliography{reference}

\end{document}

\begin{thebibliography}{99}

%1
\bibitem{Strickland_opt_1985} Strickland, D. {\&} Mourou, G. Compression of amplifies chirped optical pulses. {\it Opt. Commun.} {\bf 56}, 219 (1985).

%2
\bibitem{Abbott_prl_2016} Abbott, D. et al. Production of highly polarized positrons using polarized electrons as MeV energies. {\it Phys. Rev. Lett.} {\bf 116}, 214801 (2016).

%3
\bibitem{Raichle_1985} Raichle, M. E. Positron emission tomography: Progress in brain imaging, {\it Nature} {\bf 317}, 574 (1985).

%4
\bibitem{Ruffini_pr_2010}  Ruffini, R., Vereshchagin G. {\&} Xue, S. Electron-positron pairs in physics and astrophysics: From heavy nuclei to black holes. {\it Phys. Rep.} {\bf 487}, 1 (2010).

%5
\bibitem{Bethe_prsl_1934} Bethe, H. A. {\&} Heitler, W. On the stopping of fast particles and on the creation of positive electrons. {\it Proc. R. Soc. London A} {\bf 146}, 83 (1934).

%6
\bibitem{Gessner_ncomm_2016} Gessner, S. et al. Demonstration of a positron beam-driven hollow channel plasma wakefield 
accelerator. {\it Nat. Commun.} {\bf 7}, 11785 (2016).

%7
\bibitem{Zhou_prl_2021} Zhou, S. Y. et al. High efficiency uniform wakefield acceleration of a positron beam using stable asymmetric mode in a hollow channel plasma. {\it Phys. Rev. Lett.} {\bf 127}, 174801 (2021).

%8
\bibitem{Silva_prl_2021} T. Silva, T. et al. Stable positron acceleration in thin, warm, hollow plasma channels. {\it Phys. Rev. Lett.} {\bf 127}, 104801 (2021).

%9
\bibitem{Vieira_prl_2014} Vieira, J {\&} Mendonca, J. T. Nonlinear laser driven donut wakefields for positron and electron acceleration. {\it Phys. Rev. Lett.} {\bf 112}, 215001 (2014).

%10
\bibitem{Burke_prl_1997} Burke, D. L. et al. Positron production in multiphoton light-by-light scattering. {\it Phys. Rev. Lett.} {\bf 79}, 1626 (1997).

%11
\bibitem{Breit_pr_1934} Breit, G. {\&} Wheeler, John A. Collision of two light quanta. {\it Phys. Rev.} {\bf 46}, 15 (1934). 

%12
\bibitem{Ridgers_2012} Ridgers, C. P. et al. Dense electron-positron plasmas and ultraintense $\gamma$ rays from laser-irradiated solids. {\it Phys. Rev. Lett.} {\bf 108}, 165006 (2012).

%13
\bibitem{Vranic_sci_2018} Vranic, M., Klimo, O. {\&} Weber, S. Multi-GeV electron-positron beam generation from laser-electron scattering. {\it Sci. Rep.} {\bf 8}, 4702 (2018).

%14
\bibitem{Xing-Long_mre_2019} Xing-Long, Z. et al. Collimated GeV attosecond electron-positron bunches from a plasma channel driven by 10 PW lasers. {\it Matter and radiation at extremes} {\bf 4}, 014401 (2019).

%15
\bibitem{Jian-Xun_ppcf_2019} Jian-Xun, L. et al. Tens GeV positron generation and acceleration in a compact plasma channel. {\it Plasma Phys. Control. Fusion} {\bf 61}, 065014 (2019).

%16
\bibitem{Zhao_nat_2022} Zhao, J. et al. All-optical quasi-monoenergetic GeV positron bunch generation by twisted laser fields. {\it Commun. Phys.} {\bf 5}, 15 (2022).

%17
\bibitem{Stark_prl_2016} Stark, D. J., Toncian, T. {\&} A. V. Arefiev, A. V. Enhanced multi-MeV photon emission by a laser-driven electron beam in a self-generated magnetic field. {\it Phys. Rev. Lett.} {\bf 116}, 185003 (2016).

%18
\bibitem{Ribeyre_ppcf_2017} Ribeyre, X., d'Humi{\`e}res, E., Jansen, O., Jequier, S. {\&} Tikhonchuk, V. T. Electron-positron pairs beaming in the Breit-Wheeler process. {\it Plasma Phys. Control. Fusion} {\bf 59}, 014024 (2017).

%19
\bibitem{Ribeyre_ppcf_2018} Ribeyre, X., d'Humi{\`e}res, E., Jequier, S. {\&} Tikhonchuk, V. T. Effect of differential cross section in Breit-Wheeler pair production. {\it Plasma Phys. Control. Fusion} {\bf 60}, 104001 (2018).

%20
\bibitem{Jansen_ppcf_2018} Jansen, O. et al. Leveraging extreme laser-driven magnetic field for gamma-ray generation and pair production. {\it Plasma Phys. Control. Fusion} {\bf 60}, 054006 (2018).

%21
\bibitem{Ribeyre_pre_2016} Ribeyre, X., d'Humi{\`e}res, E., Jansen, O., Jequier, S. {\&} Tikhonchuk, V. T. Pair creation in collision of $\gamma$-ray beams produced with high-intensity lasers. {\it Phys. Rev. E} {\bf 93}, 013201 (2016).

%22
\bibitem{wang.PhysRevApp.2020} Wang, T. et al. Power scaling for collimated $\gamma$-ray beams generated by structured laser-irradiated targets and its application to two-photon pair production {\it Phys. Rev. Applied} {\bf 13}, 054024 (2020).

%23
\bibitem{Yutong_nat_2021} He, Y., Blackburn, T. G., Toncian, T. {\&} Arefiev, A. V. Dominance of $\gamma$-$\gamma$ electron-positron pair creation in a plasma driven by high-intensity lasers. {\it Commun. Phys.} {\bf 4}, 139, (2021).

%24
\bibitem{Sentoku_pre_2014} Sentoku, Y., Paraschiv, I., Royle, R., Mancini, R. C. {\&} Johzaki, T. Kinetic effects and nonlinear heating in intense x-ray-laser-produced carbon plasmas. {\it Phys. Rev. E} {\bf 90}, 051102 (2014). 

%25
\bibitem{Pandit_pop_2012} Pandit, Rishi R. {\&} Sentoku, Y.  Higher order terms of radiative damping in extreme intense laser-matter interaction. {\it Physics of Plasmas} {\bf 19}, 073304 (2012).

%26
\bibitem{Sentoku_pop_1998} Sentoku, Y., Mima, K., Taguchi, T., Miyamoto, Y. {\&} Kishimoto, Y. Particle simluation on x-ray emissions from ultra-intense laser produces plasma. {\it Physics of Plasmas} {\bf 5}, 4366 (1998).

%27
\bibitem{Iwata_pop_2016} Iwata,  N., Nagatomo, H., Fukuda, Y., Matsui, R. {\&} Kishimoto, Y. Effects of radiation reaction in the interaction between cluster media ans high intesity lasers in the radiation dominant regime. {\it Physics of Plasmas} {\bf 23}, 063115 (2016).

%28
\bibitem{Rinderknecht_njp_2021} Rinderknecht, H. G. et al.  Relativistically transparent magnetic filaments: scaling laws, initial results and prospects for strong-field QED studies. {\it New J. Phys.} {\bf 23}, 095009 (2021).

%29
\bibitem{Yutong_njp_2021} He, Y., Yeh, I-L. Blackburn, T. G. {\&} Arefiev, A. V. A single-laser scheme for observation of liner Breit-Wheeler electron-positron pair creation. {\it New J. Phys.} {\bf 23}, 115005 (2021).

%30
\bibitem{ELI} https://eli-laser.eu

%31
\bibitem{CoReLS} https://corels.ibs.re.kr/html/corels\_en/

%32
\bibitem{ZEUS} https://zeus.engin.umich.edu/laser/

%33
\bibitem{Wilks_prl_1992} Wilks, S. C., Kruer, W. L., Tabak, M. {\&} Langdon, A. B. Absorption of ultraintense laser pulses. Phys. Rev. Lett. {\bf 69}, 1383 (1992).

%34
\bibitem{gong.PRE.2020} Gong, Z. et al. Direct laser acceleration of electrons assisted by strong laser-driven azimuthal plasma magnetic fields {\it Phys. Rev. E} {\bf 102}, 013206 (2020).

%35
\bibitem{Koga_PRE_2004} Koga, J. Integration of the Lorentz-Dirac equation: Interaction of an intense laser pulse with high-energy electrons. {\it Phys. Rev. E} {\bf 70}, 046502 (2004).

%36
\bibitem{Sentoku_fst_2006} Sentoku, Y., Kruer, W., Matsuoka, M.  {\&} Pukhov, A. Laser hole boring and hot electron generation in the fast ignition scheme. {\it Fusion Science and Technology}  {\bf 49:3}, 278-296 (2006).

%37
\bibitem{Hubbell_jpcrfd_1980} Hubbell, J. H., Gimm, H. A. {\&}  {\O}verb{\o}, I. Pair, triplet, and total atomic cross sections (and mass attenuation coefficients) for 1 MeV-100 GeV photons in elements Z=1 to 100.  {\it J. Phys. Chem. Ref. Data.} {\bf 9}, 4 (1980).

%38
\bibitem{Landau_1978} Landau, D. {\&} Lifshitz, E. M. {\it Quantum Mechanics 3rd ed.} (Pergamon,London,1978).

%39
\bibitem{Royle_pre_2017} Royle, R., Sentoku, Y., Mancini, R. C., Paraschiv, I. {\&} Johzaki, T. Kinetic modeling of x-ray laser-driven solid Al plasmas via particle-in-cell simulation. {\it Phys. Rev. E} {\bf 95}, 063203 (2017).

%40
\bibitem{LANL} C. E. Lee, Los Alamos Scientific Laboratory Report No. LA-2595 (1962).

%41
\bibitem{Sentoku_jcp_2008} Sentoku, Y. {\&} Kemp, A. J. Numerical methods for particle simulations at extreme densities and temperatures: Weighted particles, relativistic collisions and reduced currents. {\it J. Comput. Phys.} {\bf 227}, 6846 (2008).

%---
%\bibitem{Chen_prl_2010} Chen, H. et al. Relativistic quasimonoenergetic positron jets from intense laser-solid interactions. {\it Phys. Rev. Lett.} {\bf 105}, 015003 (2010).

%---
%\bibitem{Sarri_nc_2015} G. Sarri et al. Generation of neutral and high-density electron–positron pair plasmas in the laboratory. {\it Nat. Comm.} {\bf 6}, 6747 (2015).

%---
%\bibitem{Qian_prd_2022} Zhao, Q. et al. Signatures of linear Breit-Wheeler pair production in polarized $\gamma \gamma$ collisions. {\it Phys. Rev. D} {\bf 105}, L071902 (2022).

%---
%\bibitem{Pukhov_prl_1996} Pukhov, A. {\&} Meyer-ter-Vehn, J. Relativistic magnetic self-channeling of light in near-critical plasma: three-dimensional particle-in-cell simulation. {\it Phys. Rev. Lett.} {\bf 76}, 3975 (1996).

%---
%\bibitem{Pukhov_pop_1998} Pukhov, A. {\&} Meyer-ter-Vehn, J. Relativistic laser-plasma interaction by multi-dimensional  particle-in-cell simulations. {\it Physics of Plasmas} {\bf 5}, 1880 (1998).

%---
%\bibitem{Pukhov_pop_1999} Pukhov, A., Sheng, Z.-M., {\&}  Meyer-ter-Vehn, J. Particle acceleration in relativistic laser channels. {\it Physics of Plasmas} {\bf 6}, 2847 (1999).

%---
%\bibitem{Hussein_njp_2021} Hussein, A. E. et al. Towards the optimisation of direct laser acceleration. {\it New J. Phys.}  {\bf 23}, 023031 (2021).

%---
%\bibitem{Nishiuchi_2020} Nishiuchi, M. et al. Dynamics of laser-driven heavy-ion acceleration clarified by ion charge states. {\it Phys. Rev. Res.}  {\bf 2}, 033081 (2020).


\end{thebibliography}

%\end{document}

