\section{Website Inspection} 
\label{sec:inspection}
To understand the landscape of modern \captchas and guide the design of the subsequent user study, we manually inspected the 200 most popular websites from the Alexa Top Website list~\cite{alexa}.
Where applicable, we use the terminology from the taxonomy proposed by Guerar et al.~\cite{Guerar}.

Our goal was to imitate a normal user's web experience and trigger \captchas in a natural setting.
Although \captchas can be used to protect any section or action on a website, they are often encountered during user account creation to prevent bots creating accounts.
Thus, for each website, we investigated the process of creating an account (wherever available).
Of the inspected websites, $185$ had some type of account creation process, and we could successfully create accounts on $142$ websites.
Distinct domains operated by the same organization (e.g., \url{amazon.com} and \url{amazon.co.jp}) were counted separately.
We visited each website twice: once with Google Chrome in incognito mode, and once with the Tor browser over the Tor network~\cite{tor}.
We used incognito mode to avoid websites changing their behavior based on cookies presented by our browser.
We used Tor since anecdotal evidence suggests Tor users are asked to solve \captchas more frequently and with greater difficulty than non-Tor users.
If no \captchas were displayed, we searched the page source for the string ``\captcha'' (case insensitive).

\taggedpara{Ethical considerations:} Based on the Guidelines for Internet Measurement Activities~\cite{rfc1262}, we did not engage in malicious behavior which may trigger additional \captchas.
We used only manual analysis to avoid various challenges that arise from automated website crawling.


\subsection{Results and analysis}
\label{subsec:inspection_results}
Figure~\ref{fig:disc_cap} shows the distribution of \captcha types we observed during our inspection.
The most prevalent types were:


\taggedpara{reCAPTCHA}~\cite{reCAPTCHAv2,reCAPTCHA,reCAPTCHAv3} was the most prevalent, appearing on 68 websites (34\% of the inspected websites).
This is a Google-owned and operated service that presents users with ``click'' tasks, which include behavioral analytics and may potentially result in an image challenge.
reCAPTCHA allows website operators to select a difficulty level, ranging from ``easiest for users'' to ``most secure''.

\taggedpara{Slider-based} \captchas appeared on 14 websites (7\%).
These typically ask users to slide a puzzle piece into a corresponding empty spot using a drag interaction.
The timing and accuracy is checked for bot-like behavior.

\taggedpara{Distorted Text} \captchas appeared on 14 websites (7\%).
We observed differences in terms of text type, color, length, masking, spacing, movement, and background.
Text type varied in several ways: 2D or 3D, solid or hollow, font, and level of distortion.
Certain \captchas used masking, i.e., lines or shapes obscured parts of the letters.

\taggedpara{Game-based} \captchas appeared on 9 websites (4.5\%).
These present users with dynamic games and compute a risk profile from the results.
For example, users are asked to rotate an image or select the correctly oriented image.

\taggedpara{hCAPTCHA}~\cite{hCaptcha} appeared on 1 website.
This is a service provided by Intuition Machines, Inc.\ that was recently adopted by Cloudflare~\cite{cloudflare} and is gaining popularity.

\taggedpara{Invisible \captchas} were found on 12 websites (6\%).
These websites did not display any visible \captchas, but contained the string ``\captcha'' in the page source.

\taggedpara{Other \Captcha{s}} found during our inspection included: a \captcha resembling a scratch-off lottery ticket; a \captcha asking 
users to locate Chinese characters within an image; and a proprietary \captcha service called ``NuCaptcha''~\cite{nucaptcha}.

% Figure environment removed



\subsection{Potential limitations}
\label{sec:inspection_limitations}

\taggedpara{Choice of website list:}
There are several lists of \emph{``popular''} websites that could be used for this type of study, including the Alexa Top Website list~\cite{alexa}, Cisco Umbrella~\cite{ciscoUmbrella}, Majestic~\cite{majestic}, TRANCO~\cite{pochat2019tranco}, Cloudflare Radar~\cite{cloudflare_radar}, and SecRank TopDomain~\cite{xie2022secrank}.
These lists vary because of the differences in the methodology used to identify and rank websites.
Following the work of Bursztein et al.~\cite{Bursztein} and the recommendation of Scheitle et al.~\cite{Scheitle2018}, we used the Alexa list.

\taggedpara{Number of inspected websites:}
Since our website inspection was a manual process, we could only inspect the top 200 websites.
This may also introduce a degree of systemic bias towards the types of \captchas used on the most popular websites.
However, we specifically chose these websites because they are visited by large numbers of users.

\taggedpara{Lower bound:}
Since we did not exercise all possible functionality of every website, it is possible that we might not have encountered all \captchas. 
Therefore, our results represent a lower bound, while the actual number of deployed \captchas may be higher.
Nevertheless, we believe that we identified the most prevalent \captcha types across all inspected websites.

\taggedpara{Timing:}
Web page rankings change on the daily basis and \captcha{s} shown by the same service may change.
Given that our inspection was performed at a particular point in time, the precise results will likely change if the inspection were repeated at a different point in time.
However, as explained above, we believe that the identified set of \captcha types is representative of currently-deployed \captchas.

\taggedpara{Other types of \captchas:}
We only inspected mainstream websites (i.e., those that would appear on a top websites list).
This means that there could be \captchas that are prevalent on other types of websites (e.g., on the dark web) but are not included in our study.
However, studying these \emph{special-purpose} \captchas might require recruiting participants who have prior experience solving them, which was beyond the scope of our study.


\taggedpara{Impact of limitations:}
The above limitations could have had an impact on the set of \captcha types we identified and subsequently used in our user study.
However, we have high confidence that the \captcha types we identified are a realistic sample of those a real user would encounter during typical web browsing.
For instance, BuiltWith~\cite{cap_usage} has analyzed a dataset of 673 million websites and identified 15.2 million websites that use \captchas.
reCAPTCHA accounts for 97.3\% and hCAPTCHA for a further 1.4\%.
The \captcha types used in our study therefore account for over 98\% of \captchas in this large-scale dataset.