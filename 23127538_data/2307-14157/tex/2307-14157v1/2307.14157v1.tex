\documentclass[12pt]{amsart}
\usepackage{amsmath, amsthm, amssymb,amscd}
%\usepackage{cite}
\usepackage{float}
\usepackage{graphicx}
%\usepackage{xypic}
\usepackage[arrow, matrix, curve]{xy}
\usepackage[all]{xypic}


%\usepackage{xypic}


%\usepackage{setspace}
%\usepackage[all,cmtip]{xy}
%\usepackage[english]{babel}
%\usepackage{appendix}
%\usepackage{MnSymbol}
%\usepackage{hyperref}
%\usepackage{MA_Titlepage}
%\usepackage[thinlines,thicklines]{easybmat}
\setlength{\topmargin}{0.0in}
\setlength{\textheight}{9.0in}
 \setlength{\evensidemargin}{0.0in}
\setlength{\oddsidemargin}{0.0in}
\setlength{\textwidth}{6.3in}




\newcommand{\proj}{\mathbb{P}}
\newcommand{\PROJ}{\mathbb{P}}
\newcommand{\seq}{\subseteq}
\newcommand{\C}{\mathbb{C}}
\newcommand{\R}{\mathbb{R}}
\newcommand{\CC}{\mathcal{C}}
\newcommand{\HH}{\mathcal{H}}
\newcommand{\LL}{\mathcal{L}}
\newcommand{\F}{\mathcal{F}}
\newcommand{\mcoarse}{\overline{M}_{md+1}}
\newcommand{\msmall}{\overline{M}_{3}}
\newcommand{\mfine}{\overline{M}_{md+1}^{*}}
\newcommand{\hypsur}{\mathcal{H}}
\newcommand{\unihyp}{\mathfrak{H}}
\newcommand{\sheafO}{\mathcal{O}}
\newcommand{\mcoarsebig}{\overline{M}_{(md+1)l}}
\newcommand{\DIAG}{\bigtriangleup}
\newcommand{\ISOM}{\simeq}
\newcommand{\UN}{\underline}
\newcommand{\MAC}{\mathcal}
\newcommand{\rank}{\text{rank }}
\newtheorem{thm}{Theorem}[section]
\newtheorem*{thm-nl}{Theorem}
\newtheorem*{prop-nl}{Proposition}
\newtheorem{definition}[thm]{Definition}
%\newtheorem{mydef}[thm]{Definition}
\newtheorem{lem}[thm]{Lemma}
\def\ZZ{{\mathbb Z}}
\def\GG{{\textbf G}}
\def\QQ{{\mathbb Q}}
\def\PP{{\textbf P}}
\def\OO{\mathcal{O}}
\def\cN{\mathcal{N}}
\def\cD{\mathcal{D}}
\def\cB{\mathcal{B}
\def\H{\mathcal{H}}}
\def\cA{\mathcal{A}}
\def\P{\mathcal{P}}
\def\W{\mathcal{W}}
\def\E{\mathcal{E}}
\def\G{\mathcal{G}}
\def\F{\mathcal{F}}


\def\cS{\mathcal{S}}
\def\K{\mathcal{K}}
\def\L{\mathcal{L}}
\def\I{\mathcal{I}}
\def\J{\mathcal{J}}
\def\cM{\mathcal{M}}
\def\cR{\mathcal{R}}
\def\ra{\mathfrak{Rat}}
\def\rr{\overline{\mathcal{R}}}
\def\cZ{\mathcal{Z}}
\def\cU{\mathcal{U}}
\def\cX{\mathcal{X}}
\def\cC{\mathcal{C}}
\def\H{\mathcal{H}}
\def\Pic0{{\rm Pic}^0(X)}
\def\pxi{P_{\xi}}
\def\ww{\overline{\mathcal{W}}}
\def\ff{\overline{\mathcal{F}}}
\def\mm{\overline{\mathcal{M}}}
\def\ss{\overline{\mathcal{S}}}
\def\kk{\overline{\mathcal{K}}}
\def\dd{\overline{\mathcal{D}}}
\def\pm{\widetilde{\mathcal{M}}}

\newtheorem{cor}[thm]{Corollary}
\newtheorem*{cor-nl}{Corollary}
\newtheorem{conjecture}[thm]{Conjecture}
\newtheorem*{conjecture-nl}{Conjecture}
\newtheorem{quest-nl}{Question}
\newtheorem*{quests-nl}{Questions}
\newtheorem{quest}[thm]{Question}
\newtheorem{prop}[thm]{Proposition}
\newtheorem{lemma}[thm]{Lemma}
\newtheorem*{wassump}{Working Assumption}



\theoremstyle{remark}
\newtheorem*{rem}{Remark}
\newtheorem*{note}{Note}
\newtheorem{eg}{Example}
\newtheorem{remark}[thm]{Remark}

\bibliographystyle{plain}
\title{{Difference varieties and the Green-Lazarsfeld Secant Conjecture}}

\author[G. Farkas]{Gavril Farkas}
\address{Humboldt-Universit\"at zu Berlin, Institut f\"ur Mathematik,  Unter den Linden 6
\hfill \newline\texttt{}
 \indent 10099 Berlin, Germany} \email{{\tt farkas@math.hu-berlin.de}}





\begin{document}
\maketitle

\begin{center}
\emph{To Claire Voisin, with admiration and friendship.}
\end{center}

\begin{abstract}
We establish the Green-Lazarsfeld Secant Conjecture for general curves of genus $g$ in all the divisorial case, that is, when the line bundles that fail to be $(p+1)$-very ample form a divisor in the Jacobian of the curve. 
\end{abstract}



\section{Introduction}

For a smooth curve $C$ of genus $g$ and a line bundle $L$ on $C$, following Green \cite{green-koszul}, the Koszul cohomology group $K_{p,q}(C,L)$ of $p$-syzygies of weight $q$ is obtained from the minimal free graded $S:=\mbox{Sym } H^0(C,L)$-resolution of the coordinate ring
$$\Gamma_C(L):=\bigoplus_{q\in \mathbb Z} H^0\bigl(C, L^{q}\bigr).$$
We write $b_{p,q}(C,L):=\mbox{dim } K_{p,q}(C,L)$ for the number of $p$-syzygies of weight $q$ of the embedded curve $\phi_L\colon C\rightarrow \PP^r$.

\vskip 3pt

Green's Conjecture \cite{green-koszul} characterizing the vanishing of the Betti numbers of a canonical curve $C\subseteq \PP^{g-1}$ in terms of the Clifford index $\mbox{Cliff}(C)$ of the curve has probably been the most influential statement in the theory of syzygies. Even though Green's Conjecture for arbitrary smooth curves remains open, several solutions have been found in the case of generic curves, starting with Voisin's landmark papers \cite{voisin-even}, \cite{voisin-odd} in the early 2000s, and continuing more recently with the approach via Koszul modules in \cite{AFPRW} (which has the benefit of proving the statement in positive characteristic as well, establishing a conjecture of Eisenbud and Schreyer \cite{ES}), and finishing with Kemeny's simpler proof using $K3$ surfaces \cite{K2}. Even more recently, alternative proofs of each of these new approaches to the generic Green's Conjecture have been put forward, see \cite{RS}, \cite{Ra}, \cite{Pa}.

\vskip 3pt

The \emph{Secant Conjecture} proposed by Green and Lazarsfeld \cite{GL1} is a generalization of Green's Conjecture to the case of line bundles of not too low degree. It predicts that if $L\in \mbox{Pic}^d(C)$ is a line bundle on a curve $C$ of genus $g$ such that

\begin{equation}\label{ineq:gl}
d\geq 2g+p+1-2h^1(C,L)-\mbox{Cliff}(C),
\end{equation}
then one has $K_{p,2}(C,L)= 0$ if and only if $L$ is $(p+1)$-very ample\footnote{Recall that a line bundle $L$ on $C$ is $(p+1)$-very ample if every effective divisor of degree $p+2$ on $C$
imposes independent conditions on the linear system $|L|$.}. Disposing quickly of the case $H^1(C,L)\neq 0$, which is well-known to be equivalent to Green's Conjecture for $C$, for non-special line bundles, the Secant Conjecture can be reformulated as the following equivalence:
\begin{equation}\label{conj:GL}
K_{p,2}(C,L)\neq 0\  \Longleftrightarrow \  L-\omega_C\in C_{p+2}-C_{2g-d+p}.
\end{equation}

Here $C_b-C_a\subseteq \mbox{Pic}^{b-a}(C)$ denotes the \emph{difference variety} of line bundles $\OO_C(D_b-D_a)$, where $D_b$ and $D_a$ are effective divisors of degrees $b$ and $a$ on $C$. The Secant Conjecture has been established in all degrees for a general pair $(C,L)$ in
\cite[Theorem 1.3]{FK1}. In the case $\mbox{deg}(L)=2g+p+1-c$, there are complete results for an \emph{arbitrary} smooth curve $C$ when $c=1$, see \cite[Theorem 2]{GL1} and when $c=2$, as long as $C$ is not bielliptic, see  \cite{Ag}. There are furthermore complete results in the case $d=2g$ and when $g$ is odd and $C$ has maximal gonality, see \cite[Theorem 1.4]{FK1}.

\vskip 4pt

In this paper we establish the Secant Conjecture for curves of genus $g$ in the divisorial case $d=g+2p+3$, that is, when the difference variety
$C_{p+2}-C_{2g-d+p}$ on the right hand side of the conjectured equivalence (\ref{conj:GL}) describes a divisor in the corresponding Jacobian variety.

\begin{thm}\label{thm:GL}
Let $C$ be a smooth curve of genus $g$ and fix $p\geq \bigl\lceil \frac{g-3}{2} \bigr\rceil$. Assume
$$\mathrm{dim }\ W^1_{p+2}(C)=2p-g+2.$$ One has the following equivalence for a line bundle $L\in \mathrm{Pic}^{g+2p+3}(C)$:
$$K_{p,2}(C,L)=0 \ \Longleftrightarrow \ L-\omega_C\notin C_{p+2}-C_{g-p-3}.$$
\end{thm}

Note that in Theorem \ref{thm:GL} one can assume $p\leq g-3$, else $d\geq 2g+p+1$, in which case the vanishing $K_{p,2}(C,L)=0$ follows automatically  from Green's result \cite[Theorem 4.a.1]{green-koszul}. Theorem \ref{thm:GL} establishes the  Secant Conjecture in its strongest form for \emph{general} curves in the divisorial case.


\begin{cor}
We fix an integer $\bigl\lceil \frac{g-3}{2} \bigr\rceil \leq p\leq g-3 $. Then the Green-Lazarsfeld Secant Conjecture holds for a general curve $C$ and for arbitrary line bundle $L\in \mathrm{Pic}^{g+2p+3}(C)$.
\end{cor}

\vskip 4pt

An important ingredient in the proof of Theorem \ref{thm:GL} is provided by the syzygetic interpretation of the divisorial difference varieties given in \cite{FMP}; for every smooth curve $C$ of genus $g$ and $p\leq g-3$, one has the following identification of cycles on $\mbox{Pic}^{2p+5-g}(C)$

\begin{equation}\label{eq:diffvar}
C_{p+2}-C_{g-p-3}=\Bigl\{\xi\in \mathrm{Pic}^{2p+5-g}(C): H^0\Bigl(C, \bigwedge^{p+2} M_{\omega_C}\otimes \omega_C\otimes \xi\Bigr)\neq 0\Bigr\},
\end{equation} where $M_{\omega_C}:=\mbox{Ker}\bigl\{H^0(C,\omega_C)\otimes \OO_C \rightarrow \omega_C\bigr\}$  is the syzygy (or kernel) bundle on $C$. Note that the slope of the vector bundle $\bigwedge^{p+2} M_{\omega_C}\otimes \omega_C\otimes \xi$ is equal to $g-1$, therefore  the right hand side of (\ref{eq:diffvar}) is indeed expected to be a divisor on $\mbox{Pic}^{2p+5-g}(C)$. Using (\ref{eq:diffvar}), coupled with Green's Duality Theorem \cite[Theorem 2.c.6]{green-koszul}

$$K_{p,2}\bigl(C,L\bigr)^{\vee}\cong K_{r(L)-p-1,0}\bigl(C, \omega_C, L\bigr),$$
where $r(L)=\mbox{deg}(L)-g=2p+3$,
Theorem \ref{thm:GL} can be reformulated as a form of \emph{strange duality} for mixed Koszul cohomology groups. Using the notation of \cite{green-koszul}, we recall that for line bundles $L, L'$ on a curve $C$ the Koszul cohomology group $K_{p,q}(C,L',L)$ is obtained from the minimal resolution of the $\mbox{Sym } H^0(C,L)$-module $\bigoplus_{q\in \mathbb Z} H^0\bigl(C, L'\otimes L^{q}\bigr)$.

\begin{cor}\label{cor:strange}
Let $C$ be a smooth curve of genus $g$ and fix $\bigl \lceil \frac{g-3}{2}\bigr \rceil \leq p\leq g-3$. Assuming $\mathrm{dim} \ W^1_{p+2}(C)=2p-g+2$, the following equivalence holds for a line bundle $L\in \mathrm{Pic}^{g+2p+3}(C)$:
$$K_{p+2,0}\bigl(C,\omega_C,L\bigr)=0\  \Longleftrightarrow \ K_{p+2,0}\bigl(C, L, \omega_C\bigr)=0.$$
\end{cor}

\vskip 4pt

I do not currently have a direct geometric explanantion of the equivalence provided in Corollary \ref{cor:strange}, that is, one that does not go through the main identification of \cite{FMP}.

\vskip 4pt

The proof of Theorem \ref{thm:GL} relies in an essential way on the case $d=2g$ of the Secant Conjecture handled in \cite[Theorem 1.4]{FK1}. When $d=2g$ and $g=2p+3$, via a comparison of two effective divisors of the universal Picard stack $\mathfrak{Pic}_g^{2g}$ parametrizing pairs $[X, L_X]$, where $X$ is a smooth curve of genus $g$ and $L_X$ is a line bundle on $X$ of degree $\mbox{deg}(L_X)=2g$, for \emph{every} curve $[X]\in \cM_{g}$ having maximal gonality $p+3$, one has the following equivalence:

\begin{equation}\label{eq:equiv}
K_{p,2}(X, L_X)\neq 0 \Longleftrightarrow L_X-\omega_X\in X_{p+2}-X_{p}.
\end{equation}

\vskip 4pt

We explain our strategy to proving Theorem \ref{thm:GL} in odd genus, the case of even genus being similar. Starting with a  curve $C$ of genus $g=2i+1$, where $i\leq p+1$ and with a line bundle $L\in \mbox{Pic}^{2i+2p+4}(C)$ such that $K_{p,2}(C,L)\neq 0$, the challenge is to manufacture a $(p+2)$-secant $p$-plane of the embedded curve
$\phi_L\colon C\rightarrow \PP^{2p+2}$. To that end, we set $\ell:=p+1-i\geq 0$, attach to $C$ general $2$-secant rational curves $R_1, \ldots, R_{2 \ell}$ and consider the following semistable curve
$$X:=C\cup R_1\cup \ldots \cup R_{2 \ell}.$$
Note that $g(X)=2p+3=2i+2\ell+1$. The Brill-Noether assumption on $C$ implies that $X$ has maximal gonality $p+3$.  We consider a line bundle $L_X$ on $X$, whose  restrictions to its components of $X$ are given by
$$L_{X|C}=L \ \ \ \mbox{ and } \ \ L_{X|R_j}\cong \OO_{R_j}(1),\ \ \mbox{ for } j=1, \ldots, 2\ell.$$ In particular $\mbox{deg}(L_X)=4p+6=2p_a(X)$ and the assumption $K_{p,2}(C,L)\neq 0$ quickly implies that $K_{p,2}(X,L_X)\neq 0$ as well. Using a suitable extension of (\ref{eq:equiv}) that covers the case of the nodal curve $X$ (and which has appeared in different guises in \cite{Ap} or \cite{K1}), coupled with the main result of \cite{FMP} which enables one to express the difference variety on the left hand side of (\ref{eq:equiv}) in a way that makes sense on singular curves as well, we obtain that
$$H^0\Bigl(X, \bigwedge^{p} M_{\omega_X}\otimes \omega_X^{2}\otimes L_X^{\vee}\Bigr)=0.$$
Varying the points of intersection of the rational curves $R_j$ with $C$, we obtain via a homological argument that for an integer $0\leq j\leq 2\ell+1$  the following inclusion
\begin{equation}\label{eq:intro_diff1}
L-\omega_C-C_{2(2\ell-j+1)}\subseteq C_{i+j-\ell}-C_{i+\ell-j}
\end{equation}
holds, where the left hand side is regarded as a divisor in the Jacobian $\mbox{Pic}^{2j-2\ell}(C)$. An argument using secant varieties on curves leads us then to conclude that necessarily $L-\omega_C\in C_{i+\ell+1}-C_{i-\ell-1}$, which precisely corresponds to the inclusion (\ref{eq:intro_diff1}) in the extremal case $j=2\ell+1$. This is equivalent to $\phi_L\colon C\rightarrow \PP^{2p+2}$ having a $(p+2)$-secant $p$-plane.

\vskip 8pt






\noindent {\bf Acknowledgment:} I profitted from discussions with M. Aprodu, M. Kemeny and C. Voisin on this circle of ideas. The author was supported by the Berlin Mathematics Research Center MATH+ and by the ERC Advanced Grant SYZYGY.
This project has received funding from the European Research Council (ERC) under the European Union Horizon 2020 research and innovation program (grant agreement No. 834172).



\section{Syzygies of curves}

In this section we discuss a number of basic facts on syzygies of curves that will be used throughout the paper. If $X$ is a projective variety and  $L\in \mbox{Pic}(X)$ is a globally generated line bundle on $X$, the Koszul cohomology group $K_{p,q}(X, L)$ is by definition the cohomology of the complex
\begin{align*}
\bigwedge^{p+1} H^0(X,L)\otimes H^0\bigl(X, L^{q-1}\bigr)\stackrel{d_{p+1,q-1}}\longrightarrow \bigwedge^p H^0(X,L)\otimes H^0\bigl(X, L^q\bigr) \\
\stackrel{d_{p,q}}\longrightarrow \bigwedge^{p-1} H^0(X,L)\otimes H^0(X,  L^{q+1}\bigr),
\end{align*}
where $d_{p,q}$ is the Koszul differential. With the help of the \emph{syzygy bundle} 
$$M_L:=\mbox{Ker}\bigl\{H^0(X,L)\otimes \mathcal{O}_X\rightarrow L\bigr\},$$ 
one has the following  interpretation of the Koszul cohomology group:

\begin{equation}\label{eq:kernel_bundle}
K_{p,q}(X, L) \cong \frac{H^0\bigl(X, \bigwedge^p M_L\otimes L^q \bigr)}{\bigwedge^{p+1} H^0(X,L)\otimes H^0(X,  L^{q-1}\bigr)}.
\end{equation}

We also use the notation $Q_L:=M_L^{\vee}$ for the dual of the syzygy bundle. If $L$ is a non-special line bundle, using (\ref{eq:kernel_bundle}) we have the following equivalence
$$K_{p,2}(X,L)=0\Longleftrightarrow H^1\bigl(X, \bigwedge^{p+1} M_L\otimes L\bigr)=0.$$

\vskip 5pt

We will use the setting when $L$ is a globally generated line bundle on a smooth curve $C$ and $X:=C\cup R_1\cup \ldots \cup R_a$ is a nodal curve obtained from $C$ by attaching $2$-secant mutually disjoint rational curves $R_j$ meeting $C$ at general points $x_j, y_j$. Since the restriction map $\mbox{Pic}(X)\rightarrow \mbox{Pic}(C)$ is surjective, we can choose a line bundle $L_X\in \mbox{Pic}(X)$  such that
$$L_{X| C}\cong L \ \mbox{  and }   L_{X|R_j}\cong \OO_{R_j}(1), \ \ \mbox{  for  }\  j=1, \ldots,a.$$

\begin{prop}\label{prop:degen}
One has the inclusion $K_{p, 2}(C,L)\hookrightarrow K_{p,2}(X,L_X)$.
\end{prop}
\begin{proof}  Set $r:=h^0(C,L)-1$. From the Mayer-Vietoris sequence on $X$
$$0\longrightarrow H^0(X,L_X)\longrightarrow  H^0(C,L)\oplus \Bigl(\bigoplus_{j=1}^a H^0\bigl(R_j, \OO_{R_j}(1)\bigr)\Bigr)\longrightarrow \bigoplus_{j=1}^a \mathbb C^2_{x_j, y_j},$$
we obtain that $H^0(X,L_X)\cong H^0(C,L)$. Next we apply the duality theorem
$$K_{p,2}(X, L_X)\cong K_{r-p-1,0}(X, \omega_X, L_X)^{\vee},$$
together with the following commutative diagram,
\[
\xymatrixcolsep{5pc}
\xymatrix{
  \bigwedge^{r-p-1} H^0(C, L)\otimes H^0(C, \omega_C)\ar[d]_{j}  \ar[r]^{d_{r-p-1,0}^C} & \bigwedge^{r-p-2} H^0(C,L)\otimes H^0(C,\omega_C\otimes L)\ar[d]_{\bar{j}} &\\
\bigwedge^{r-1-p} H^0(X,L_X)\otimes H^0(X,\omega_X)\ar[r]^{d_{r-1-p,0}^X} &  \bigwedge^{r-p-2} H^0(X,L_X)\otimes H^0(X,\omega_X\otimes L_X) &.
}
\]
where $d_{r-p-1,0}^C$ and $d_{r-p-1,0}^X$ are the Koszul  differentials for $C$ and $X$ respectively.
Observe that both  $j$ and $\bar{j}$ are injective, which implies  that $\mbox{ker}\bigl(d_{r-1-p,0}^C\bigr)\hookrightarrow \mbox{ker}\bigl(d_{r-1-p,0}^X)$ is injective, which finishes the proof.
\end{proof}

\subsection{The divisorial difference varieties in Jacobians}
For a smooth curve $C$ and integers $a, b\geq 0$, we denote by $C_b-C_a\subseteq \mbox{Pic}^{b-a}(C)$ the difference variety consisting of all line bundles of the form $\OO_C(D_b-D_a)$, where $D_b$ and $D_a$ are effective divisors of degree $b$ respectively $a$ on $C$. In the case the difference varieties are divisors in the respective Jacobians, that is, when $b=g-a-1$, the main result of \cite{FMP} provides an alternative description of the divisorial difference variety in terms of the (non-abelian) theta divisor of the vector bundle $\bigwedge^a Q_{\omega_C}$, precisely
\begin{equation}\label{eq:fmp}
C_{g-a-1}-C_a=\Theta_{\bigwedge^a Q_{\omega_C}}=\Bigl\{\xi\in \mbox{Pic}^{g-2a-1}(C): h^0\bigl(C, \bigwedge^a Q_{\omega_C}\bigr)\geq 1\Bigr\}.
\end{equation}

\vskip 4pt

We shall also make use of the varieties of secant divisors for a linear system on a curve. Given a line bundle $L$ on $C$ with $h^0(C, L)=r+1$, for integers $e>0$ and $0\leq f <e$, we introduce the variety of $e$-secant $(e-f-1)$-divisors with respect to the complete linear system $|L|$
\begin{equation}\label{eq:secant_var}
V_e^{e-f}(L):=\Bigl\{D\in C_e: \mbox{dim } |L(-D)|\geq r-e+f\Bigr\}.
\end{equation}
The expected dimension of $V_e^{e-f}(L)$ as a determinantal subvariety of $C_e$ is equal to
$$\mbox{exp.dim } V_e^{e-f}(L)=e-f(r+1-e+f).$$ For various results on the structure of the secant loci $V_e^{e-f}(L)$ we refer to \cite[§ 8]{ACGH}, \cite{AS} and \cite{Fa}.

\vskip 5pt

The Green-Lazarsfeld Secant Conjecture (\ref{conj:GL}) offers a characterization of those line bundles $L\in \mbox{Pic}^d(C)$ satisfying the condition $K_{p,2}(C,L)\neq 0$ in terms of secant varieties of linear systems, predicting the following equivalence

$$K_{p,2}(C,L)\neq 0 \ \Longleftrightarrow \ V_{p+2}^{p+1}(L)\neq \emptyset,$$
as long as $H^1(C,L)=0$ and $d\geq 2g+p+1-\mbox{Cliff}(C)$. Note that via a projection argument it is immediate to see that if $V_{p+2}^{p+1}(L)\neq 0$, then $K_{p,2}(C,L)\neq 0$, therefore the Secant Conjecture concerns the reverse implication.
The divisorial case of the Secant Conjecture treated in this paper  corresponds to the situation
$$\mbox{exp.dim } V_{p+2}^{p+1}(L)=g-1\Longleftrightarrow d=2g+2p+3.$$

\vskip 5pt

A solution  to the Secant Conjecture that holds for \emph{every} curve of maximal gonality is provided in \cite{FK1} in the particular divisorial case $d=2g$, when the genus $g=2p+3$ is odd, see the equivalence (\ref{eq:equiv}). The aim of this paper is to establish the Secant Conjecture for \emph{all} divisorial cases and for curves of arbitrary gonality, at the price of allowing the base curve $C$ to satisfy certain (mild) generality assumptions of Brill-Noether nature.

\vskip 5pt

We fix an odd genus $g=2p+3$ and recall that $\pi\colon \overline{\mathfrak{Pic}}^{2g}_{g}\rightarrow \mm_{g}$ denotes Caporaso's \cite{Ca} compactification of the universal Jacobian $\mathfrak{Pic}^{2g}_{g}\rightarrow \cM_{g}$ of degree $2g$ over the moduli space
of curves of genus $g$. A point of $\overline{\mathfrak{Pic}}^{2g}_{g}$ corresponds to a pair $[X, L_X]$, where $X$ is a quasi-stable curve of genus $g$ and $L_X$ is a \emph{balanced} line bundle of total degree $2g$ on $X$, see \cite[§ 3]{Ca}.

\begin{prop}\label{prop:div}
Let $X$ be a quasi-stable curve of genus $g=2p+3$ having no disconnecting nodes and let $L_X$ be a globally generated balanced bundle of degree $2g$ on $X$. Assume $X$ has gonality $p+3$. Then the following equivalence holds:
$$K_{p,2}(X,L_X)=0\ \Longleftrightarrow \ H^0\Bigl(X, \bigwedge^p Q_{\omega_{X}}\otimes L_X\otimes \omega_X^{\vee}\Bigr)=0.$$
\end{prop}
\begin{proof}
We denote by $\widetilde{\mathfrak{Syz}}$ the effective divisor on $\overline{\mathfrak{Pic}}^{2g}_{g}$ consisting of pairs $[X, L_X]$ with $K_{p,2}(X, L_X)\neq 0$. The determinantal structure giving rise to $\widetilde{\mathfrak{Syz}}$ can be read off from \cite[§ 6]{FK1}. We write
$\mathfrak{Syz}:=\widetilde{\mathfrak{Syz}}\cap \mathfrak{Pic}^{2g}_{g}$. Similarly, we consider the \emph{secant divisor}
$$\widetilde{\mathfrak{Sec}}:=\Bigl\{[X,L_X]\in \overline{\mathfrak{Pic}}^{2g}_{g}: H^0\Bigl(\bigwedge^p Q_{\omega_X}\otimes L_X\otimes \omega_X^{\vee}\Bigr)\neq 0\Bigr\},$$
and set $\mathfrak{Sec}:=\widetilde{\mathfrak{Sec}}\cap \mathfrak{Pic}^{2g}_{g}$. Using \cite[Theorem 1.4]{FK1}
and the identification (\ref{eq:diffvar}) of divisorial varieties, we conclude that we have the following set-theoretic equality of divisors on $\overline{\mathfrak{Pic}}^{2g}_g$
$$\overline{\mathfrak{Syz}}=\pi^{-1}\bigl(\mm_{g,p+2}^{1}\bigr) \cup \overline{\mathfrak{Sec}},$$
where $\mm_{g,p+2}^1$ is the Hurwitz divisor consisting of stable curves of gonality at most $p+2$. The closure in both sides of this equality is taken inside $\overline{\mathfrak{Pic}}_g^{2g}$. To conclude,  since $\Delta_{\mathrm{irr}}$ is the only boundary divisor of $\mm_g$ possibly containing the point $[X]\in \mm_g$, it suffices to show that $\widetilde{\mathfrak{Syz}}$ does not contain the boundary divisor $\pi^{-1}\bigl(\Delta_{\mathrm{irr}}\bigr)$, where $\Delta_{\mathrm{irr}}$ denotes the closure in $\mm_g$ of the locus of irreducible curves.
 Since $\pi^{-1}\bigl(\Delta_{\mathrm{irr}}\bigr)$ is irreducible (see \cite[Theorem 3.2]{MV}), it suffices to provide \emph{one example} of a one-nodal curve $Y$ of genus $2p+3$ and a line bundle $L_Y\in \mbox{Pic}^{4p+6}(Y)$ such that $K_{p,2}(Y,L_Y)=0$. This follows from \cite[Theorem 1.8]{FK1}, where such a vanishing is provided for all curves lying in an ample linear system on a $K3$ surface. Since such a linear system contains one-nodal irreducible curves, the conclusion follows.
\end{proof}

\section{The Secant Conjecture and divisorial difference varieties}

In this section we prove Theorem \ref{thm:GL}. We start with a smooth curve $C$ of genus $g$  and a non-special line bundle $L\in \mbox{Pic}^{g+2p+3}(C)$, where $\bigl \lceil \frac{g-3}{2}\bigr \rceil\leq p\leq p-3$. If $c:=\mbox{Cliff}(C)$, since $W^1_{c+2}(C)+C_{p-c}\subseteq W^1_{p+2}(C)$, the assumptions on $C$ ensure that the inequality
(\ref{ineq:gl}) is satisfied.

\vskip 3pt

The proof of Theorem \ref{thm:GL} will depend on the parity of $g$ and we treat first that the case of odd genus, when we write
\begin{equation}\label{eq:nota}
g=2i+1, \ \mbox{ with }  \  \  i\leq p+1.
\end{equation}

We set $\ell:=p+1-i\geq 0$. We pick $2\ell$ pairs of general points $(x_j, y_j)$ on $C$ and we introduce the curve
\begin{equation}\label{def:X}
X:=C\cup_{\{x_1, y_1\}} R_1 \cup \ldots \cup_{\{x_{2\ell}, y_{2\ell}\}} R_{2\ell},
\end{equation}
where each $R_j$ is a smooth rational curve meeting transversally $C$ at the points $x_j$ and $y_j$, for $j=1, \ldots, 2\ell$. The curves $R_{j'}$ and $R_j$ are disjoint for $j\neq j'$.  Note that $X$ is a quasi-stable curve of genus $g(X)=2i+2\ell+1=2p+3$. We further introduce a line bundle $L_X\in \mbox{Pic}^{4p+6}(X)$ such that
$$L_{X|C} \cong L \ \ \mbox{ and } \ \ L_{|R_j}\cong \OO_{R_j}(1), \ \mbox{ for } j=1, \ldots, 2\ell.$$
Note that indeed $\mbox{deg}(L_X)=\mbox{deg}(L)+2\ell=2i+2p+4+2\ell=4p+6$. Observe furthermore, that $L_X$ is a \emph{balanced} line bundle on $X$ in the sense of Caporaso  \cite{Ca}, in particular the pair $[X, L_X]$ can be regarded as a point in the compactification $\overline{\mathfrak{Pic}}^{4p+6}_{2p+3} \rightarrow \mm_{2p+3}$ of the universal Jacobian $\mathfrak{Pic}_{2p+3}^{4p+6}$ constructed in \cite{Ca}. From the generality assumptions on $C$ and on the points $(x_j, y_j)$, we obtain that the stabilization of $[X]\in \mm_{2p+3}$ (obtained by contracting the rational curves $R_j$) is a stable curve of maximal gonality, that is, it does not lie in the Hurwitz divisor $\mm_{2p+3, p+2}^1$ of (stable) curves of gonality at most $p+2$ (see also \cite{Ap}, \cite{FK2} for variations of this argument).

\vskip 6pt


\noindent {\emph{Proof of Theorem \ref{thm:GL}.} We keep the same notation as above and assume $K_{p,2}(C,L)\neq 0$ and we aim to show that $L-\omega_C\in C_{i+\ell+1}-C_{i-\ell-1}$. Applying Proposition \ref{prop:degen}, we obtain that $K_{p,2}(X,L_X)\neq 0$. The assumption $\mbox{dim } W^1_{p+2}(C)=2p-g+2$ implies that when choosing the points $x_j, y_j\in C$ generally, then $[X]\notin \mm_{2p+3,p+2}^1$, that is, $X$ is not a limit of smooth curves of gonality $p+2$. We are therefore in a position to apply Proposition \ref{prop:div}, to obtain that
$$H^0\Bigl(\bigwedge^p Q_{\omega_X}\otimes L_X\otimes \omega_X^{\vee}\Bigr)\neq 0\Longleftrightarrow H^1\Bigl(\bigwedge^p Q_{\omega_X}\otimes L_X\otimes \omega_X^{\vee}\Bigr)\neq 0.$$

By Serre duality, this last condition translates into the non-vanishing

\begin{equation}\label{eq:non_van2}
H^0\Bigl(X, \bigwedge^p M_{\omega_X}\otimes \omega_X^2\otimes L_X^{\vee}\Bigr)\neq 0.
\end{equation}

We denote by $D_{2\ell}:=x_1+y_1+\cdots+x_{2\ell}+y_{2\ell}\in C_{4\ell}$ the divisor of marked points on $C$. In order to make the condition (\ref{eq:non_van2}) explicit, we note that since $\omega_{X|R_j}\cong \OO_{R_j}$, one has $M_{\omega_X|R_j} \cong \OO_{R_j}^{\oplus (2p+2)}$ and
$\bigl(\omega_X^2 \otimes L_X^{\vee}\bigr)_{|R_j} \cong \OO_{R_j}(-1)$. From the Mayer-Vietoris sequence 

\begin{align*}
0\longrightarrow H^0\Bigl(X, \bigwedge^p M_{\omega_X}\otimes \omega_X^2\otimes L_X^{\vee}\Bigr)\longrightarrow 
H^0\Bigl(C, \bigwedge^p M_{\omega_X|C}\otimes \omega_C^2\otimes L^{\vee}(2D_{2\ell})\Bigr)\oplus \\ \oplus  \ \bigoplus_{j=1}^{2\ell} H^0\Bigl(R_j, \bigwedge^p  \bigl(\OO_{R_j}^{\oplus (2p+2)}(-1)\bigr)\Bigr)\longrightarrow 
H^0\Bigl(\bigwedge^p M_{\omega_X}\otimes \omega_X^2\otimes L_X^{\vee} \otimes \OO_{D_{2\ell}}\Bigr)\longrightarrow \cdots,
\end{align*}
we conclude that (\ref{eq:non_van2}) implies that
\begin{equation}\label{eq:non-van3}
H^0\Bigl(C, \bigwedge^p M_{\omega_X|C}\otimes \omega_C^2\otimes L^{\vee}(D_{2\ell})\Bigr)\neq 0\Longleftrightarrow H^1\Bigl(C, \bigwedge^p Q_{\omega_X|C}\otimes L\otimes \omega_C^{\vee}(-D_{2\ell})\Bigr)\neq 0.
\end{equation}

In order to describe  $M_{\omega_X|C}$, we write down the following morphism of short exact sequences

\[
\xymatrixcolsep{4pc}
\xymatrix{
0 \ar[r] & M_{\omega_C} \ar[d]_{} \ar[r] & H^0(C, \omega_C)\otimes \OO_C \ar[d] \ar[r] & \omega_{C} \ar[d]_{} \ar[r] & 0 \\
0 \ar[r]  & M_{\omega_X|C} \ar[r] & H^0(X, \omega_X)\otimes \OO_C \ar[r] & \omega_C(D_{2\ell}) \ar[r] & 0
}
\]
to obtain the following exact sequence of vector bundles on $C$
$$0\longrightarrow M_{\omega_C}\longrightarrow M_{\omega_X|C}\longrightarrow \bigoplus_{j=1}^{2\ell} \OO_C\bigl(-x_j-y_j\bigr) \longrightarrow 0,$$
which, after dualizing, can be rewritten as the exact sequence
\begin{equation}\label{eq:exseq3}
0\longrightarrow \bigoplus_{j=1}^{2\ell} \OO_C\bigl(x_j+y_j\bigr)\longrightarrow Q_{\omega_X|C}\longrightarrow Q_{\omega_C}\longrightarrow 0.
\end{equation}

We compute the $p$-th exterior power of the sequence (\ref{eq:exseq3}). The assumption $p\leq g-3$ translates into $\ell\leq i-1$. There exists a filtration
$$\bigwedge^p Q_{\omega_X|C}=\F^0\supset \F^1\supset \cdots \supset \F^{2\ell}\supset \F^{2\ell+1}=0$$
of vector bundles on $C$, such that the successive quotients are given by
$$\F^{k-1}/\F^{k}\cong \bigwedge^{k-1} \Bigl(\bigoplus_{j=1}^{2\ell} \OO_C(x_j+y_j)\Bigr) \otimes \bigwedge^{p+1-k} Q_{\omega_C}.$$
For each $k=1, \ldots, 2\ell$ we tensor with the line bundle $L\otimes \omega_C^{\vee}(D_{2\ell})$ the exact sequence
$$0\longrightarrow \F^k \longrightarrow \F^{k-1} \longrightarrow \bigwedge^{p+1-k} Q_{\omega_C} \otimes \bigwedge^{k-1} \Bigl(\bigoplus_{j=1}^{2\ell} \OO_C\bigl(x_j+y_j\bigr)\Bigr)\longrightarrow 0,$$
and taking cohomology, we obtain that the condition (\ref{eq:non-van3}) implies that there exists an integer $j\leq 2\ell+1$ such that
\begin{equation}\label{eq:non-van4}
H^1\Bigl(C, \bigwedge^{i+\ell-j} Q_{\omega_C}\otimes L\otimes \omega_C^{\vee}(-D_{2\ell-j+1}')\Bigr)\neq 0,
\end{equation}
where $D_{2\ell-j+1}$ is an effective divisor of degree $2(2\ell-j+1)$ on $C$ supported on some of the points $x_1, y_1, \ldots, x_{2\ell}, y_{2\ell}$. Since the marked points on $C$ are chosen generally, the divisor $D_{2\ell-j+1}\in C_{2(2\ell-j+1)}$ is general as well, therefore we obtain that
(\ref{eq:non-van3}) implies the following inclusion
\begin{equation}\label{eq:incl}
L- \omega_C-C_{2(2\ell-j+1)}\subseteq C_{i+j-\ell}-C_{i+\ell-j}
\end{equation}
for some $0\leq j\leq 2\ell+1$. Note that the desired conclusion, that is, $L-\omega_C\in C_{i+\ell+1}-C_{i-\ell-1}$ corresponds precisely to the statement  (\ref{eq:incl}) in the extremal case $j=2\ell+1$. We now show that the inclusion (\ref{eq:incl}) for some $j\leq 2\ell+1$, implies the statement (\ref{eq:incl}) for $j=2\ell+1$.

\vskip 5pt

We set $\eta:=L\otimes \omega_C^{\vee}\in \mbox{Pic}^{2\ell+2}(C)$. The condition (\ref{eq:incl}) can be translated into stating that the variety of secant divisors
$$V_{i+\ell-j}^{i+\ell-j-1}\bigl(\omega_C\otimes \eta^{\vee}\bigr):=\Bigl\{B\in C_{i+\ell-j}: h^0(C, \eta(B))\geq 1\Bigr\}$$
has dimension at least $2(2\ell-j+1)$, which exceeds by $1$  the expected dimension of the degeneracy locus $V_{i+\ell-j}^{i+\ell-j-1}(\omega_C\otimes \eta)$. We may assume $h^0(C,\eta)=0$, else, since 
$$C_{2\ell+2}\subseteq C_{i+\ell+1}-C_{i-\ell-1},$$
the conclusion $\eta\in C_{i+\ell+1}-C_{i-\ell-1}$ is immediate. Therefore $h^0(C,\eta)=0$, or equivalently, $\omega_C\otimes \eta^{\vee}$ is a non-special line bundle and $h^0(C, \omega_C\otimes \eta^{\vee})=2i-2\ell-2$.

\vskip 4pt

Recalling the definition (\ref{eq:secant_var}) of the varieties of secant divisors, we  apply \cite[page 356]{ACGH} to the non-special linear system $|\omega_C\otimes \eta^{\vee}|$ to conclude that $V_{i+\ell-a}^{i+\ell-a-1}(\omega_C\otimes \eta^{\vee})\neq \emptyset$, as long as the following inequality is satisfied

$$\mbox{exp.dim } V_{i+\ell-a}^{i+\ell-a-1}\bigl(\omega_C\otimes \eta^{\vee}\bigr)=4\ell-2a+1\geq  0\Longleftrightarrow a\leq 2\ell.$$

In particular, setting $a=2\ell$, we find $V_{i-\ell}^{i-\ell-1}\bigl(\omega_C\otimes \eta^{\vee}\bigr)\neq \emptyset$ and applying the result of Aprodu and Sernesi \cite[Theorem 4.1]{AS} on secant varieties of excessive dimension, we obtain that (\ref{eq:incl}) implies the estimate
$$\mbox{dim } V_{i-\ell}^{i-\ell-1}\bigl(\omega_C\otimes \eta^{\vee}\bigr)\geq \mbox{dim } V_{i+\ell-j}^{i+\ell-j-1}\bigl(\omega_C\otimes \eta^{\vee}\bigr)-2(2\ell-j)=2.$$

\vskip 4pt

Thus $V_{i-\ell}^{i-\ell-1}\bigl(\omega_C\otimes \eta^{\vee}\bigr)=\bigl\{D\in C_{i-\ell}: H^0(C, \eta(D))\geq 1\bigr\}$ is at least $2$-dimensional. Having fixed a base point $p_0\in C$, this implies that the locus
$$Y:=\Bigl\{E\in C_{i-\ell-1}: H^0\bigl(C, \eta(E+ p_0)\bigr)\neq 0\Bigr\}$$
has dimension at least $1$. We consider the Abel-Jacobi map $\theta\colon Y\rightarrow \mbox{Pic}^{i+\ell+2}(C)$  given by $\theta(E):=\eta(E+p_0)$.
If $\theta$ is generically finite, it follows that $\mbox{dim } \theta(Y)\geq1$. Since the locus $p_0+C_{i+\ell+1}$ is an \emph{ample} divisor in $C_{i+\ell+2}$, it follows that $\theta(Y)$ intersects $p_0+C_{i+\ell+1}$, therefore there exists $E\in C_{i-\ell-1}$ and $E'\in C_{i+\ell+1}$ such that $\eta(E+p_0)=\OO_C(p_0+E')$, amounting to
\begin{equation}\label{eq:conclusion}
L-\omega_C\in C_{i+\ell+1}-C_{i-\ell-1},
\end{equation}
as desired. If on the other hand, $\theta$ is not generically finite, then $h^0\bigl(C, \eta(E+p_0)\bigr)\geq 2$, for every $E\in Y$, in which case once again, one finds an effective divisor $E'\in C_{i+\ell+1}$ such that $\eta(E+p_0)=\OO_C(p_0+E')$, which once more implies the conclusion (\ref{eq:conclusion}), which finishes the proof of Theorem \ref{thm:GL} in the odd genus case.
\hfill $\Box$

\vskip 7pt

\noindent {\emph{{Proof of Theorem \ref{thm:GL} in the case of even genus}}. The proof follows with minor modification along the lines of the odd genus case. We fix a smooth curve $C$ of genus $g=2i$ satisfying the condition 
$$\mbox{dim } W^1_{p+2}(C)=g-2p+2$$ and a line bundle $L\in \mbox{Pic}^{2i+2p+3}(C)$ such that $K_{p,2}(C,L)\neq 0$. Setting $\ell:=p+1-i\geq 0$ once more, the aim is to show that
$$L-\omega_C\in C_{i+\ell+1}-C_{i-\ell-2}.$$ Note that in this case the assumption $d\leq 2g+p$ translates into $\ell\leq i-2$.

\vskip 4pt


This time we attach $2\ell+1$ rational curves $R_1, \ldots, R_{2\ell+1}$ to $C$, such that  each $R_j$ meets $C$ at two general points $x_j, y_j\in C$. The resulting curve $X$ is a quasi-stable curve of genus $g(X)=2p+3$ and gonality $p+3$. We choose a line bundle $L_X\in \mbox{Pic}^{4p+6}(X)$ whose restriction to $C$ is the line bundle $L$, whereas $L_{X|R_j}\cong \OO_{R_j}(1)$, for $j=1, \ldots, 2\ell+1$. Following \emph{mot \`a mot} the reasoning in the odd genus case, denoting by 
$$D_{2\ell+1}:=(x_1+y_1)+\cdots+ (x_{2\ell+1}+y_{2\ell+1})\in C_{4\ell+2}$$ 
the divisor of marked points, knowing that
$K_{p,2}(X, L_X)\neq 0$, we obtain that
$$H^0\Bigl(X, \bigwedge^p M_{\omega_X|C}\otimes \omega_C^2\otimes L^{\vee}(D_{2\ell+1})\Bigr)\neq 0.$$
Setting $\eta:=L\otimes \omega_C^{\vee}\in \mbox{Pic}^{2\ell+3}(C)$,  the same filtration argument as in the odd genus case, implies there exists $j\leq 2\ell+2$ such that
$$\mbox{dim } V_{i+\ell-j}^{i+\ell-j-1}\bigl(\omega_C\otimes \eta^{\vee}\bigr)\geq 4\ell-2j+4 \ \Bigl(=\mbox{exp.dim } V_{i+\ell-j}^{i+\ell-j-1}\bigl(\omega_C\otimes \eta^{\vee}\bigr)+1\Bigr).$$
The conclusion $\eta\in C_{i+\ell+1}-C_{i-\ell-2}$ follows applying once more \cite{AS}. 
\hfill $\Box$

\vskip 8pt


\subsection{Brill-Noether theory and the Secant Conjecture}
Using standard results in Brill-Noether theory, for large values of $p$ the condition
\begin{equation}\label{eq:BNdim}
\mbox{dim } W^1_{p+2}(C)=2p-g+2
\end{equation}
appearing as in the statement of Theorem \ref{thm:GL} is always satisfied and our results are in these cases complete. In the extremal case $p=g-3$, from Martens' Theorem \cite[5.1]{ACGH}, we obtain that $\mbox{dim } W^1_{g-1}(C)=g-4$ unless $C$ is hyperelliptic. In this case, we recover \cite[Theorem 3.3]{GL2}. In the next case $p=g-4$, applying Mumford's Theorem \cite[5.2]{ACGH}, we obtain that if $C$ is a non bielliptic curve with $\mbox{Cliff}(C)\geq 2$, then the estimate (\ref{eq:BNdim}) is satisfied and Theorem \ref{thm:GL} reduces to Agostini's result \cite{Ag} in the case of line bundles of degree $3g-5$. The next cases are already new results and we record them:

\begin{thm}\label{thm:keem}
1) Let $C$ be a smooth curve of genus $g$ with $\mathrm{Cliff}(C)\geq 3$. Then one has the following equivalence for a line bundle $L\in \mathrm{Pic}^{3g-7}(C)$:
$$K_{g-5,2}(C,L)=0  \ \Longleftrightarrow \  L-\omega_C \notin C_{g-3}-C_2.$$

\vskip 3pt

\noindent

2) Let $C$ be a smooth curve of genus $g\geq 12$ such that $\mathrm{Cliff}(C)\geq 4$. Assume that $C$ is not a triple cover of an elliptic curve.
Then one has the following equivalence for $L\in \mathrm{Pic}^{3g-9}(C)$:
$$K_{g-6,2}(C,L)=0 \ \Longleftrightarrow \  L-\omega_C\notin C_{g-4}-C_3.$$
\end{thm}

\begin{proof} We apply Theorem \ref{thm:GL} observing that the assumption (\ref{eq:BNdim}) is satisfied by applying Keem's results \cite[Theorem 2.1]{Ke} when $p=g-5$, respectively  \cite[Corollary 3.3]{Ke} when $p=g-6$.
\end{proof}

\vskip 5pt

For smaller values of $p$, our understanding of which curve verify (\ref{eq:BNdim}) is less complete. In the extremal case $p=\frac{g-3}{2}$, the condition (\ref{eq:BNdim}) reduces to saying that $C$ has maximal gonality. In general, it turns out that the sufficient condition from Theorem \ref{thm:GL} is related to the study of the curves having infinitely many pencils of minimal degree. The following result makes this connection precise:

\begin{prop}\label{prop:bn}
Let $a\geq 0$ and  $C$ be a smooth curve of genus $g\geq (a+1)(2a+1)$ such that $\mathrm{Cliff}(C)\geq a+1$ and $\mathrm{dim} \ W^1_{a+3}(C)<1$. Then the following equivalence holds for a line bundle $L\in \mathrm{Pic}^{3g-2a-3}(C)$:
$$K_{g-3-a,2}(C,L)=0\ \Longleftrightarrow \  L-\omega_C\notin C_{g-1-a}-C_a.$$
\end{prop}
\begin{proof} It suffices to observe that under these hypotheses, applying \cite[Theorem 15]{Co} if $W^1_{a+3}(C)$ is at most zero-dimensional, then necessarily $\mbox{dim } W^1_{g-1-a}(C)=g-2a-4$ and then the hypothesis (\ref{eq:BNdim}) is satisfied.
\end{proof}

\vskip 6pt

\begin{rem} The first case where Proposition \ref{prop:bn} may not always apply is for $p=g-7$, where we are not aware of a classification of the smooth $7$-gonal curves $C$ with $\mbox{Cliff}(C)=5$ and such that $\mbox{dim } W^1_7(C)=1$. One class of curves where the condition (\ref{eq:BNdim}) generally fails is that of covers of elliptic curves.  For such curves however, Kemeny \cite{K3} introduced recently novel techniques to understand their syzygies.
\end{rem}

\begin{thebibliography}{aaaaaaaa}
\bibitem[Ag]{Ag} D. Agostini, {\em{The Martens-Mumford Theorem and the Green-Lazarsfeld Secant Conjecture}}, arXiv:2110.03561.
\bibitem[Ap]{Ap} M. Aprodu, {\em{Remarks on syzygies of $d$-gonal curves}}, Math. Res. Lett. \textbf{12} (2005), 387--400.
\bibitem[AS]{AS} M. Aprodu and E. Sernesi, {\em{Excess dimension for secant loci in symmetric products of cuvres}}, Collectanea Math. \textbf{68} (2017), 1--7.
\bibitem[AFPRW]{AFPRW} M. Aprodu, G. Farkas, S. Papadima, C. Raicu and J. Weyman, {\em{Koszul modules and Green's Conjecture}}, Inventiones Math. \textbf{218} (2019), 657--720.
\bibitem[ACGH]{ACGH} E. Arbarello, M. Cornalba, P. A. Griffiths and J. Harris, {\emph{Geometry of algebraic curves}}, Volume I, Grundlehren der mathematischen Wissenschaften \textbf{267}, Springer-Verlag (1985).
%\bibitem[B]{B} A. Beauville, {\em{Some stable vector bundles with reducible theta divisors}}, Manuscripta Math. \textbf{110} (2003), 343--349.
\bibitem[Ca]{Ca} L. Caporaso, {\em{A compactification of the Picard variety over the moduli space of stable curves}}, Journal of the American Math. Soc. \textbf{7} (1994), 589--660.
\bibitem[Co]{Co} M. Coppens, {\em{Some remarks on the schemes $W^r_d$}}, Annali di Matematica pura ed appl. \textbf{157}
(1990), 183--197.    
%\bibitem[CEFS]{CEFS} A. Chiodo, D. Eisenbud, G. Farkas and F.-O. Schreyer, {\em{Syzygies of torsion bundles and the geometry of the level $\ell$ modular varieties over $\overline{\mathcal{M}}_g$}}, Inventiones Math. \textbf{194} (2013), 73--118.
%\bibitem[CFVV]{CFVV} E. Colombo, G. Farkas, A. Verra and C. Voisin, {\em{Syzygies of Prym and paracanonical curves of genus $8$}}, \'Epijournal de G\'eom\'etrie Alg\'ebrique \textbf{1} (2017).
%\bibitem[EL]{EL}  L. Ein and R. Lazarsfeld, {\em{The gonality conjecture on syzygies of algebraic curves of large
%degree}}, Publ. Math. Inst. Hautes   \'Etudes Sci. \textbf{122} (2015), 301--313.
\bibitem[ES]{ES} D. Eisenbud and F.-O. Schreyer, {\em{Equations and syzygies of $K3$ carpets and unions of scrolls}}, Acta Math. Vietnam. \textbf{44} (2019), 3--29.
\bibitem[Fa]{Fa} G. Farkas, {\em{Higher ramification and varieties of secant divisors on the generc curve}}, Journal of the London Math. Soc.
\textbf{78} (2008), 418--440.    
\bibitem[FK1]{FK1} G. Farkas and M. Kemeny, {\em{The generic Green-Lazarsfeld Secant Conjecture}}, Inventiones Math. \textbf{203} (2016), 265--301.
\bibitem[FK2]{FK2}  G. Farkas and M. Kemeny, {\em{Linear Syzygies of Curves with prescribed gonality}}, Advances in Math. \textbf{356} (2019), 106810.
%\bibitem[FL]{FL}  G. Farkas and E. Larson, {\em{The Minimal Resolution Conjecture for points on general curves}}, arXiv:2209.11308
\bibitem[FMP]{FMP} G. Farkas, M. Musta\c{t}\u{a} and M. Popa, {\em{Divisors on
$\cM_{g, g+1}$ and the Minimal Resolution Conjecture for points on
canonical curves}}, Annales Sci. de L'\'Ecole  Normale Sup\'erieure \textbf{36} (2003), 553--581.
%\bibitem[GG]{GG} A. Geramita and A. Gimigliano, {\em{Generators for the defining ideal of certain rational surfaces}}, Duke Math. Journal \textbf{62} (1991), 61--83.
\bibitem[G]{green-koszul} M. Green, {\em{Koszul cohomology and the cohomology of projective varieties}}, Journal of Differential Geometry \textbf{19} (1984), 125--171.
\bibitem[GL1]{GL1} M. Green and R. Lazarsfeld, {\em{On the projective normality of complete linear series on an algebraic curve}}, Inventiones Math. \textbf{83} (1986), 73--90.
\bibitem[GL2]{GL2} M. Green and R. Lazarsfeld, {\em{Some results on the syzygies of finite sets and algebraic curves}}, Compositio Mathematica \textbf{67} (1988), 301--314.
\bibitem[Ke]{Ke} C. Keem, {\em{On the variety of special linear systems on an algebraic curve}}, Math. Annalen \textbf{288} (1990)
309--322.
\bibitem[K1]{K1} M. Kemeny, {\em{The extremal Secant Conjecture for curves of arbitrary gonality}}, Compositio Math. \textbf{153} (2017), 347--357.
\bibitem[K2]{K2} M. Kemeny, {\em{Universal Secant Bundles and Syzygies of Canonical Curves}}, Inventiones Math. \textbf{223} (2021), 995--1026.
\bibitem[K3]{K3} M. Kemeny, {\em{Projecting syzygies of curves}}, Algebraic Geometry \textbf{7} (2020),  561--580.
\bibitem[MV]{MV} M. Melo and F. Viviani, {\em{The Picard group of the compactified universal Jacobian}}, Documenta Math. \textbf{19} (2014), 457--507.
\bibitem[Pa]{Pa} J. Park, {\em{Syzygies of tangent developables surfaces and $K3$ carpets via secant varieties}},  arXiv:2212.07584.
\bibitem[RS]{RS} C. Raicu and S. Sam, {\em{Bi-graded Koszul modules, $K3$ carpets, and Green's conjecture}}, Compositio Math. \textbf{158} (2022), 33--56.
\bibitem[Ra]{Ra} J. Rathmann, {\em{Geometric Koszul complexes, syzygies of $K3$ surfaces and the Tango bundle}}, arXiv:2205.00266.
%\bibitem[Sch]{Sch} F.-O. Schreyer, {\em{Syzygies of canonical curves and special linear series}}, Mathematische Annalen \textbf{275} (1986), 105--137.
\bibitem[V1]{voisin-even} C. Voisin, {\em{Green's generic syzygy conjecture
for curves of even genus lying on a $K3$ surface}}, Journal of
European Mathematical Society \textbf{4} (2002), 363--404.
\bibitem[V2]{voisin-odd} C. Voisin, {\em{Green's canonical syzygy conjecture for generic curves of odd genus}},
Compositio Mathematica \textbf{141} (2005), 1163--1190.
\end{thebibliography}


\end{document} 