\documentclass[12pt]{iopart}
%\pdfoutput=1
\usepackage{hyperref}
\usepackage{graphicx}
\usepackage[caption=false]{subfig}
\usepackage{exscale}
\usepackage{latexsym}
\usepackage{epstopdf}
\usepackage[percent]{overpic}
\usepackage{iopams}
\usepackage{float}
%\usepackage{mathtools}
\expandafter\let\csname equation*\endcsname\relax
\expandafter\let\csname endequation*\endcsname\relax
%\usepackage{amsmath,amssymb}
\usepackage{amssymb}
\usepackage{cases}
\usepackage{dsfont}
\usepackage{txfonts}
\usepackage{array}
\bibliographystyle{iopart-num}
%\usepackage{citesort}
\usepackage{cleveref}
%\usepackage{threeparttable}
%\usepackage{bm}
\usepackage{multirow}
\usepackage{booktabs}
\crefname{equation}{equation}{equations}
\crefname{figure}{figure}{figures}
\crefname{section}{section}{sections}
%\crefname{appendix}{appendix}{appendices}
\newcommand{\gguide}{{\it Preparing graphics for IOP Publishing journals}}
\usepackage[numbers,sort&compress]{natbib}
\usepackage{xcolor}
%\usepackage{hyperref}


%\newcommand{\ket}[1]{|#1\rangle}
\newcommand{\ket}[1]{\left|#1\right\rangle}
\newcommand{\bra}[1]{\langle #1|}
\newcommand{\ud}{\mathrm{d}}
\newcommand{\Cr}{\mathcal{C}_{r}}
\newcommand{\Cl}{\mathcal{C}_{l_1}}
\newcommand{\D}{\mathcal{D}}
\newcommand{\I}{\mathcal{I}}
\newcommand{\C}{\mathcal{C}}
\newcommand{\n}{\nonumber\\}
\newtheorem{theorem}{Theorem}
\newcommand{\abs}[1]{\lvert #1\rvert}
\newcommand{\p}{\partial}
\newcommand{\nab}{\nabla}
\newcommand{\pp}{\mathbf{p}}
\newcommand{\Ss}{\mathbf{S}}
\newcommand{\Lambdasp}{\Lambda_{\frac{1}{2}}}
\newcommand{\Up}{\Uparrow}
\newcommand{\up}{\uparrow}
\newcommand{\Down}{\Downarrow}
\newcommand{\dw}{\downarrow}
\newcommand{\ex}[1]{\langle #1\rangle}
\newcommand{\lf}{\left}
\newcommand{\rg}{\right}
\newcommand{\QQ}{\mathcal Q}
\newcommand{\PP}{\mathcal P}
\newcommand{\LL}{\mathcal L}
\newcommand{\GG}{\mathcal G}
\newcommand{\KK}{\mathcal K}
\newcommand{\II}{\mathcal I}




\begin{document}

\title{Quantum metrology enhanced by the $XY$ spin interaction in a generalized Tavis-Cummings model}

\author{Yuguo Su$^{1,*}$, Wangjun Lu$^{2}$, and Hai-Long Shi$^{3,*}$}

\address{$^{1}$ Innovation Academy for Precision Measurement Science and Technology, Chinese Academy of Sciences, Wuhan 430071, China}
\address{$^{2}$ Department of Maths and Physics, Hunan Institute of Engineering, Xiangtan 411104, China}
\address{$^{3}$ INO-CNR, Largo Enrico Fermi 2, 50125 Firenze, Italy}
\address{$^{*}$ Authors to whom any correspondence should be addressed.}
%\ead{hl\_shi@yeah.net}
\eads{\mailto{suyuguo@apm.ac.cn} and \mailto{hl\_shi@yeah.net}}

\begin{abstract}
Quantum metrology is recognized for its capability to offer high-precision estimation by utilizing quantum resources, such as quantum entanglement.
%
Here, we propose a generalized Tavis-Cummings model by introducing the $XY$ spin interaction to explore the impact of the many-body effect on estimation precision, quantified by the quantum Fisher information (QFI).
%
By deriving the effective description of our model, we establish a closed relationship between the QFI and the spin fluctuation induced by the $XY$ spin interaction.
%
Based on this exact relation, we emphasize the indispensable role of the spin anisotropy in achieving the Heisenberg-scaling precision for estimating a weak magnetic field.
%
Furthermore, we observe that the estimation precision can be enhanced by increasing the strength of the spin anisotropy. 
%
We also reveal a clear scaling transition of the QFI in the Tavis-Cummings model with the reduced Ising interaction.
%
Our results contribute to the enrichment of metrology theory by considering many-body effects, and they also present an alternative approach to improving the estimation precision by harnessing the power provided by many-body quantum phases.

\end{abstract}


\noindent{\it Keywords}: quantum metrology, quantum Fisher information, Heisenberg limit, %cavity quantum electrodynamics, 
Tavis-Cummings model,
$XY$ model, quantum sensing, quantum magnetometry



\section{Introduction}\label{Sec.I}
%
Quantum metrology~\cite{PhysRevLett.96.010401,WOS:000288984900012,RevModPhys.89.035002,RevModPhys.90.035005,PhysRevLett.128.160505} aims to achieve enhanced sensitivity in estimating an unknown parameter $\xi$ compared to classical metrology by exploiting diverse quantum resources, such as quantum entanglement~\cite{PhysRevA.54.R4649,PhysRevLett.102.100401,PhysRevLett.106.130506,PhysRevLett.107.080504,PhysRevA.85.022321}  and quantum squeezing~\cite{PhysRevD.23.1693,PhysRevLett.88.231102,MA201189,PhysRevLett.110.163604,PhysRevA.92.023603,PhysRevA.102.052423}.
%
The widespread applications of quantum metrology have emerged in numerous experimental domains, including Ramsey spectroscopy~\cite{WOS:A1980KA25400008,PhysRevLett.86.5870}, atomic clocks~\cite{RevModPhys.87.637,Louchet_Chauvet_2010,PhysRevLett.112.190403}, gravitational-wave detectors~\cite{WALLS1981118,WOS:000256613000015,Abbott_2009}, magnetometry~\cite{budker_optical_2007,PhysRevLett.109.253605,PhysRevLett.120.260503,PhysRevLett.126.010502}, and biophysical  measurements~\cite{doi:10.1073/pnas.1004037107,WOS:000316154700018,TAYLOR20161}.
%
The quantum Cram\'er-Rao bound (QCRB)~\cite{PhysRevLett.72.3439} $\delta \xi\geq1/\sqrt{\mathcal{F}_{\xi}}$ provides a theoretical approach to assess the suitability of a quantum state for a high-precision quantum estimation.
%
In other words, a quantum state with a larger quantum Fisher information (QFI), $\mathcal F_\xi$, can yield a higher precision.
% 
For example, in the quantum phase estimation, when utilizing non-entangled $N$-particle states, the precision is bounded by the standard quantum limit (SQL), namely, $\mathcal F_{\xi}\propto N$.
%
However, when the Greenberger-Horne-Zeilinger (GHZ) entangled state is employed, it becomes feasible to approach the Heisenberg limit (HL), meaning that $\mathcal F_{\xi}\propto N^2$.
%
Preparing GHZ-like states is, however, a challenging endeavor from the experimental perspective.
%
Therefore, our aim is to explore whether certain many-body ground states, achievable through system cooling, suffice to achieve HL-precision metrology.
  


Cavity quantum electrodynamics (Cavity-QED) contributes significantly to our foundational understanding of the interaction between atoms and the electromagnetic field within a cavity~\cite{Dutra-Book:2005,Raimond-RMP:2001,hepp1973superradiant,PhysRevA.8.2517}.
%
Its remarkable capability to manipulate atoms using the electromagnetic field has proven to be highly valuable across a wide spectrum of domains, spanning from quantum simulation~\cite{Buluta2009,Georgescu2014}, quantum key distribution~\cite{Scarani2009,obrien_photonic_2009} to quantum metrology~\cite{Giovannetti2011,Liu_2019,YuguoSu2021,PhysRevLett.130.170801,deng2023,WOS:000368673800032,PhysRevLett.116.093602,Flower_2019,Gietka_2021,Kim1999,WOS:000263818900002,PhysRevA.94.022313,wan2023quantum}.
%
The Tavis-Cummings (TC) model~\cite{PhysRev.170.379,PhysRev.188.692} is the simplest one for describing a multi-atom cavity quantum electrodynamics system.
%
To enhance the many-body effects within the TC model, we extend our consideration to a generalized TC model in which the atoms are organized into a spin chain with the $XY$ interaction.
%
Our primary focus will be on investigating variations in metrological precision when the initial state of the atoms is prepared under different quantum phases of the $XY$ spin chain.
%


In this work, a promising quantum-enhanced protocol is proposed for sensing a weak magnetic field in a cavity-QED system by introducing the many-body spin effect.
%
We establish a direct relationship between the estimation precision and the spin fluctuations.
%
Moreover, we illustrate that the spin anisotropy is pivotal in achieving the HL-precision.
%
Additionally, we utilize the estimation precision to identify quantum phase transitions. 
%
Our work provides an effective attempt at designing high-precision quantum estimation strategies by incorporating the quantum many-body effect.




This paper is organized as follows.
In section~\ref{Sec.II}, we introduce a generalized TC model featuring the $XY$ spin interaction.
%
The effective description of our model is obtained by using the time-averaged method and validated through numerical calculations.
%
In section~\ref{Sec.III}, based on the effective Hamiltonian, we analytically derive the QFI and demonstrate the indispensability of the spin fluctuation for achieving the HL-precision.
% 
By utilizing the correlation function of the $XY$ model, we analyze the scaling of the QFI within different quantum phases. 
%
We show that the TC model, devoid of any spin interactions, achieves only the SQL-precision.
%
Conversely, when incorporating the $XY$ spin interaction and the spin anisotropy into the TC model, the HL-precision becomes attainable for estimating a weak magnetic field.
%
We elucidate a distinct scaling transition of the QFI in the TC model with the Ising interaction.
%
Furthermore, we illustrate that the QFI increases with the increase of the spin anisotropy.
%
Finally, a conclusion is made in section~\ref{Sec.V}.



\section{System and effective Hamiltonian}\label{Sec.II}
%

% Figure environment removed


We consider a generalized TC model, where a collection of $N$ two-level atoms (or spin-1/2 spins) constitutes a $XY$ spin chain and collectively interacts with a single bosonic cavity mode,  as illustrated in figure~\ref{Fig1}.
The system is described by the following Hamiltonian:
%
\begin{eqnarray}\label{Total-H}
&&H=\omega_0 J_z+\omega_{\rm{a}} a^\dagger a+H_0\left(h\right)+H_{\rm I},\label{H-1}\\
&&H_{0}\left(h\right)=-\frac{\lambda}{2}\sum_{i=1}^{N}{\left[\frac{1+\gamma}{2}\sigma^x_i\sigma^x_{i+1}+\frac{1-\gamma}{2}\sigma^y_i\sigma^y_{i+1}\right]}-\frac{h}{2}\sum_{i=1}^{N}\sigma^z_i,\nonumber\\
&&H_{\rm I}=g\left(a^\dagger J_-+aJ_+\right),\nonumber
\end{eqnarray}
where $H_{\rm I}$ represents the interaction Hamiltonian governing the atom-light coupling and the $XY$ Hamiltonian $H_0\left(h\right)$ characterizes interactions between nearest-neighbor spins.
%
Here, $J_{x,y,z}\!=\!\sum_{j=1}^{N}{\sigma_j^{x,y,z}/2}$ and $J_\pm\!=\!J_x\pm \rmi  J_y$ are the collective spin operators.
%
The operators $a$ and $a^\dagger$ correspond to the creation and annihilation of cavity mode photons.
%
The parameters $\omega_0$, $\omega_{\rm{a}}$, and $2g$ denote the spin transition frequency, cavity frequency, and single-photon Rabi frequency, respectively.
%
$\lambda$ describes the strength of the nearest-neighbor interaction.
%
$\gamma$ quantifies the degree of anisotropy.
%
$h$ represents the magnetic field to be estimated.
%
The experimental implementation of this generalized TC model is feasible in a superconducting quantum processor~\cite{PhysRevLett.120.050507,doi:10.1126/science.aay0600}.




However, it is important to note that the Hamiltonian~(\ref{Total-H}) cannot be solved exactly. 
%
To overcome this challenge, we introduce the time-averaged method to derive an effective description of the original system.
%
Moreover, this approach will provide valuable insights into the relationship between metrological scaling and quantum phases.
%
Employing the unitary transformation $U\!=\!\exp\left\{-\rmi \left[\omega_0 J_z+\omega_{\rm{a}} a^\dagger a+H_0\left(h\right)\right]t\right\}$  and the large detuning condition  ($\left|\omega_0-h-\omega_{\rm{a}}\right|\gg\lambda$), the total Hamiltonian (\ref{H-1}) in the interaction picture could be read as 
$H_{\rm int-pic}= g\left[J_-a^\dagger\mathrm{e}^{-\mathrm{i}\left(\Delta+\delta\right) t}+\rm{H.c.}\right]$ with a large detuning $\Delta\!\equiv\!\omega_0-h-\omega_{\rm{a}}$ and a small residue $\delta$.
%
Utilizing the time-averaged method given in the reference~\cite{James2007}, the total Hamiltonian can be approximated as (see more details in  appendix A):
\begin{equation}\label{H2}
H_{\rm eff}^{\rm (s)}\simeq H_0\left(h-\omega_0\right)+\frac{2g^2}{\Delta}J_za^\dagger a+\frac{g^2}{\Delta}J_+J_-,
\end{equation}
which is written in the Schr\"{o}dinger picture.
%
%
The essence of the time-averaged method is to eliminate high-frequency contribution and thus it can be viewed as a natural generalization of the rotating-wave approximation~\cite{TAM}.
%
This approximation holds under the conditions of $\Delta^2\gg g^2N$, $\Delta^2\gg g^2N^2/\bar n$, and $|\Delta|\gg \lambda$.
%
Here, $\bar n\equiv \langle a^\dag a\rangle$ is the average photon number.



Generally, the average photon number is much larger than the number of spins, i.e., $\bar n\gg\langle J_z\rangle\approx N$, then we can further reduce the Hamiltonian~(\ref{H2}) to obtain the following effective Hamiltonian: 
\begin{equation}\label{H-eff}
H_{\rm {eff}}=H_0\left(h-\omega_0\right)+\frac{2g^2}{\Delta}J_za^\dagger a,
\end{equation}
%
which is similar to the Hepp-Coleman model~\cite{Sun06}.
%
In summary, the conditions ensuring the validity of this effective Hamiltonian~(\ref{H-eff}) are $\Delta^2\gg g^2N$,  $|\Delta|\gg \lambda$, and $\bar n\gg N$.



% Figure environment removed

%%
To demonstrate the validity of our effective Hamiltonian approximation, we perform calculations for the dynamics of the expectation value  $\left\langle M\left(t\right)\right\rangle $, where the observable $M$ is chosen as $J_{\varphi}=J_{x}\cos\varphi+J_{y}\sin\varphi$.
%
The initial state is considered as a product state, with the cavity initialized in a coherent state $\left|\alpha\right\rangle $ and the spins in a spin-coherent state $\ket{\theta,\phi}=\otimes_{i=1}^{N}(\cos\frac{\theta}{2}\ket{\uparrow}_i+e^{{\rm i}\phi}\sin\frac{\theta}{2}\ket{\downarrow}_i)$, where $\ket{\uparrow}_i$ and $\ket{\downarrow}_i$ are the eigenstates of $\sigma_i^z$ with the eigenvalues $1$ or $-1$, respectively. 
%
The consistency of the numerical expectations of the original and effective Hamiltonians (blue circles and red dashed line), as shown in figure~\ref{Fig2},  demonstrates that the effective Hamiltonian can faithfully describe the original system.
%
%More numerical verification is available in Appendix B.
%




\section{Quantum Fisher information of the generalized TC model}\label{Sec.III}
%
The metrological scheme constitutes the initial preparation of the spins in the ground state $|\phi_{\rm g}\rangle$ of the $XY$ Hamiltonian $H_0(h)$, followed by the injection of a coherent light $\ket{\alpha}$. 
%
As a consequence, the initial state takes the form of a product state, denoted as $\ket{\psi(0)}=|\phi_{\rm g}\rangle\ket{\alpha}$.
%
Following quantum dynamics, the information about the magnetic field $h$ becomes encoded in the evolved state $\ket{\psi(t)}=\exp(-{\rm i}Ht)\ket{\psi(0)}$.
%
The maximum precision for estimating the parameter $h$ that can be provided by $\ket{\psi(t)}$ is determined by the QCRB: $\delta h\geq 1/\sqrt{\mathcal{F}_{h}}$, where $\mathcal{F}_{h}$ is the QFI~\cite{PhysRevLett.72.3439,helstrom1969quantum,holevo2011probabilistic,PhysRevD.23.357}:
\begin{equation}
\mathcal{F}_{h}=4[\ex{\psi(0)|\mathcal H^2|\psi(0)}-\ex{\psi(0)|\mathcal H|\psi(0)}^2].\label{QFI}
\end{equation}
%
Here, the metrological generator is given by
$
\mathcal{H}=\rmi \left(\partial_{h}U^{\dagger}\right)U
$ with $U=\exp(-{\rm i}Ht)$.
%
Having verified the validity of the effective Hamiltonian, we can replace $H$ with $H_{\rm eff}$~(\ref{H-eff}) to derive the generator as follows:
%
\begin{eqnarray}\label{generator}
\mathcal H=\left(1-\frac{2g^{2}}{\Delta^{2}}a^{\dagger}a\right)J_{z}t.
\end{eqnarray}
%
Substituting the generator~(\ref{generator}) into equation~(\ref{QFI}) we obtain
\begin{eqnarray}\label{QFI-2}
\mathcal F_h
& = & 4t^{2}\left[ \left(1-\frac{2g^{2}}{\Delta^{2}}\bar{n}\right)^{2}{\rm Var}(J_z)
+\frac{4g^{4}}{\Delta^{4}}\bar{n} \ex{J_{z}^{2}} \right],
\end{eqnarray}
where $\bar{n}\equiv\left|\alpha\right|^{2}$ is the average photon number of the coherent light $\ket{\alpha}$, $\ex{\cdot}$ denotes the expectation value over the ground state $|\phi_{\rm g}\rangle$ of the $XY$ model, and ${\rm Var}(J_z)=\ex{J_z^2}-\ex{J_z}^2$ is the variance.
%

Recall that we are exclusively considering the scenario where the average photon number $\bar n$ significantly exceeds the number of spins $N$.
%
As a result, in this context, the HL-precision refers to $\mathcal F_h\propto\bar n^2$.
%
Equation~(\ref{QFI-2}) makes it evident that the presence of nonvanishing spin fluctuations ${\rm Var}(J_z)$ is both necessary and sufficient to achieve the HL-precision in our metrological scheme.
%
Next, we will calculate the spin correlation functions, specifically ${\rm Var}(J_z)$ and $\ex{J_z^2}$, for the subsequent discussion.






\subsection{Correlation functions in the \texorpdfstring{$XY$}{} model}\label{Sec3-1}
%
To calculate the correlation functions, we can express the $XY$ model as a free fermion model by using the standard Jordan-Wigner transformation, Fourier transformation, and Bogoliubov transformation~\cite{LIEB1961407}.
%
Finally, the Hamiltonian of the spin part can be rewritten as 
\begin{equation}
H_{0}\left(h\right)=\sum_{k}\Lambda_{k}b_{k}^{\dagger}b_{k}+{\rm Constant},\label{DiaH}
\end{equation}
%
where the momentum is denoted as $k=2\pi m/N$ with $m=-N/2+1,\dots,N/2$ and $b_{k}$ represents the fermion  operator.
%
The excitation energy is given by
\begin{equation}
\Lambda_{k}=\sqrt{\left(h-\lambda\cos k\right)^{2}+\lambda^{2}\gamma^{2}\sin^{2}k},
\end{equation}
where the Bogoliubov angles are determined by
\begin{eqnarray}\label{angle}
\sin\nu_{k}=\frac{\lambda\gamma\sin k}{\Lambda_{k}},\quad
\cos\nu_{k}=\frac{\lambda\cos k-h}{\Lambda_{k}}.
\end{eqnarray} 
%
In terms of the $b_k$ operator, the total spin operator can be expressed as 
\begin{equation}\label{Jz}
J_z=\!-\frac{N}{2}+\frac{1}{2}\!\sum_{k}\!\left[\!(1\!+\!\cos\nu_k)b_k^\dag b_k\!+\!(1\!-\!\cos\nu_k)b_{-k}b_{-k}^\dag\!+\!{\rm i}\sin\nu_k (b_k^\dag b_{-k}^\dag\!-\! b_{-k}b_k)\!\right]\!.\nonumber
\end{equation}
%\begin{eqnarray}\label{Jz}
%J_z&=&-\frac{N_c}{2}\\ 
%& &+\frac{1}{2}\sum_{k}\left[(1+\cos\nu_k)b_k^\dag b_k+(1-\cos\nu_k)b_{-k}b_{-k}^\dag+{\rm i}\sin\nu_k (b_k^\dag b_{-k}^\dag- b_{-k}b_k)\right].\nonumber
%\end{eqnarray}
%
Based on equation~(\ref{Jz}) and the relation $b_k|\phi_{\rm g}\rangle=0$, we have 
\begin{eqnarray}\label{Cor-Func}
&&\ex{J_z^2}=\frac{1}{2}\sum_k\sin^2\nu_k+\frac{1}{4}\left(\sum_k\cos\nu_k\right)^2,
{\rm Var}(J_z)=\frac{1}{2}\sum_k\sin^2\nu_k.
\end{eqnarray}
%

The phase diagram of the $XY$ model is given in figure~(\ref{Fig3}), where the paramagnetic and the ferromagnetic phases are separated by the critical magnetic field $h_{\rm c}=1$.
%
Our objective is to elucidate the impact of the $XY$ interaction on the scaling behavior of the QFI based on equations~(\ref{QFI-2}, \ref{angle},\ref{Cor-Func}).



\subsection{Quantum Fisher information for the TC model}\label{Sec3-2}
We first consider the case of $\lambda\!=\!0$.
%
In this case, the Hamiltonian~(\ref{Total-H}) reduces to the TC model where the spins do not have direct interactions.
%  
Substituting the condition $\lambda\!=\!0$ into equations~(\ref{angle}) and (\ref{Cor-Func}), we obtain $\ex{J_z^2}=N^2/4$ and ${\rm Var}(J_z)=0$.
%
This is because the spin ground state is now fully polarized, $|\phi_{\rm g}\rangle\!=\!\ket{\uparrow,\uparrow\cdots,\uparrow}$.
%
Consequently,  from equation~(\ref{QFI-2}), we find that the QFI for the TC model is given by
\begin{eqnarray}
 \mathcal{F}_{h}(\lambda\!=\!0)\approx\frac{4g^4}{\Delta^4}t^{2} N^2 \bar n,
\end{eqnarray}
which indicates that only the SQL-precision, i.e., $\mathcal F_h\propto\bar n$, can be achieved.
%
Hence, our attention will shift to the impact of the spin interaction.


   % Figure environment removed

\subsection{Quantum Fisher information for the TC model with the Ising interaction}\label{Sec.3-3}
%
In this subsection, we focus on the TC model with the Ising interaction, i.e., $\gamma=1$.
%
For convenience, we set $\lambda=1$ in what follows.
%
We know that the Ising model will undergo a quantum phase transition from the ferromagnetic phase to the paramagnetic phase phase.
%
The corresponding quantum critical point is located at $h_{\rm c}=1$.
%
Our primary concern is how the QFI behaves in different quantum phases.
 

In the case of $h\ll h_{\rm c}$, we can approximate the Bogoliubov angles~(\ref{angle}) as $\sin\nu_{k}\approx\sin k$ and $\cos\nu_{k}\approx\cos k$.
 %
Substituting this approximation into equation~(\ref{Cor-Func}), we obtain 
 \begin{eqnarray}
 &&\ex{J_z^2}\approx\frac{1}{2}\left(\frac{N}{2\pi}\int_{-\pi}^\pi dk\sin^2k\right)+\frac{1}{4}\left(\frac{N}{2\pi}\int_{-\pi}^\pi dk\cos k\right)^2=\frac{N}{4},\\ 
&&{\rm Var}(J_z)=\frac{1}{2}\left(\frac{N}{2\pi}\int_{-\pi}^\pi dk\sin^2k\right)=\frac{N}{4},
 \end{eqnarray}
where the thermodynamic limit has been considered. 
 %
Compared to the scenario without spin interactions ($\lambda=0$), the Ising interaction induces a non-zero spin fluctuation, ${\rm Var}(J_z)=N/4$, for the weak magnetic field case.
%
As a result, the QFI is given by
\begin{eqnarray}\label{Ising-1}
 \mathcal{F}_{h}(\gamma\!=\!1,h\!\ll\!h_{\rm c})
 &\approx&
 4t^{2}\left[\left(1-\frac{2g^{2}}{\Delta^{2}}\bar{n}\right)^{2}\frac{N}{4}+\frac{4g^{4}}{\Delta^{4}}\bar{n}\left(\frac{N}{4}\right)\right]\n 
 &\approx&
 \frac{4g^{4}}{\Delta^{4}}t^{2}N\bar{n}^{2}
 +\mathcal O(\bar n),
\end{eqnarray}
which implies that the HL-precision for the photon number $\bar n$ can be achieved.
%


 % Figure environment removed
  % Figure environment removed
 
However, in the case of $h\gg h_{\rm c}$, the approximation from equation~(\ref{angle}) $\sin \nu_k\approx0$ and $\cos\nu_k\approx -1$ results in $\ex{J_z^2}\approx N^2/4$ and ${\rm Var}(J_z)\approx 0$ which is consistent with the $\lambda=0$ case, see section~(\ref{Sec3-2}).
%
It should be emphasized that this approximation only requires $h\gg h_{\rm c}$ and also holds for the case when $\gamma\neq 1$.
%
Therefore, we can deduce from equation~(\ref{QFI-2}) that for $h\gg h_{\rm c}$ case, the QFI is the same as the one obtained from the TC model without any interaction, i.e., 
\begin{eqnarray}\label{Ising-2}
 \mathcal{F}_{h}(h\!\gg\!h_{\rm c})\approx\frac{4g^4}{\Delta^4}t^{2} N^2\bar n.
\end{eqnarray}
%

The analytical scalings of the QFIs (\ref{Ising-1}) and (\ref{Ising-2}) are numerically verified in figure~\ref{Fig4}, suggesting that by introducing spin interactions, our scheme can provide the HL-precision for estimating a weak magnetic field.
%
For the case of a strong magnetic field, the vanishing spin fluctuation ${\rm Var}(J_z)$ results in only achieving the SQL-precision.
%
Furthermore, figure~\ref{Fig5} illustrates a clear transition in the scaling of the QFI as we gradually increase the strength of the magnetic field $h$, ultimately crossing the quantum critical point $h_{\rm c}\!=\!1$.

 \subsection{Quantum Fisher information for the TC model with the \texorpdfstring{$XY$}{} interaction}\label{Sec.3-4}
%
As discussed in section~(\ref{Sec.3-3}), we cannot suppress the SQL-precision for estimating a strong magnetic field in our metrological scheme. 
%
However, it is attainable for the weak field case by introducing the Ising interaction among the spins.
%
Therefore, in this subsection, we will mainly focus on the weak field case and illustrate the role of the spin anisotropy $\gamma$ within the $XY$ interaction played in the estimation precision.

For the isotropic $XX$ model ($\gamma\!=\!0$), the Bogoliubov angles~(\ref{angle}) are given by $\sin\nu_k=0$ and $\cos\nu_k={\rm sgn}(\cos k-h)$, where we also assume $\lambda\!=\!1$ for convenience.
%
By equations~(\ref{QFI-2}) and~(\ref{Cor-Func}), we know the spin fluctuation vanishes ${\rm Var}(J_z)=0$  and can conclude that the HL-precision cannot be achieved in this case. 
%It follows from $\sin\nu_k=0$ and equation~(\ref{Cor-Func}) that ${\rm Var}(J_z)=0$.
%%
%Thus, by equation~(\ref{QFI-2}), we can conclude that the HL-precision cannot be achieved in this case. 
%
Explicitly, if $h>h_{\rm c}=1$ then we have $\cos\nu_k=-1$, and  as a result, $\ex{J^2_z}=N^2/4$.
%
Thus, the QFI~(\ref{QFI-2}) is given by
\begin{eqnarray}
\mathcal{F}_{h}(\gamma\!=\!0,h\!>\!h_{\rm c})=\frac{4g^4}{\Delta^4}t^{2} N^2 \bar n.
\end{eqnarray}
%
If $h<h_{\rm c}$, then by substituting $\sin\nu_k=0$ and $\cos\nu_k={\rm sgn}(\cos k-h)$ into equation~(\ref{Cor-Func}) we obtain
\begin{eqnarray}
\ex{J_z^2}&\ \ =&\frac{1}{2}\sum_k\sin^2\nu_k+\frac{1}{4}\left(\sum_k\cos\nu_k\right)^2\n 
&\stackrel{N\to\infty}{=}&\frac{1}{4}\frac{N^2}{4\pi^2}\left(-\int_{-\pi}^{-\arccos(h)}dk+\int_{-\arccos(h)}^{\arccos(h)}dk-\int^{\pi}_{\arccos(h)}dk\right)^2\n 
&\ \ =&\frac{N^2}{4\pi^2}[2\arccos(h)-\pi]^2.
\end{eqnarray}
%\begin{eqnarray}
%\ex{J_z^2}&=&\lim_{N\to\infty}\left[\frac{1}{2}\sum_k\sin^2\nu_k+\frac{1}{4}\left(\sum_k\cos\nu_k\right)^2\right]\n
%&=&\lim_{N\to\infty}\left[\frac{1}{4}\frac{N^2}{4\pi^2}\left(-\int_{-\pi}^{-\arccos(h)}dk+\int_{-\arccos(h)}^{\arccos(h)}dk-\int^{\pi}_{\arccos(h)}dk\right)^2\right]\n 
%&=&\frac{N^2}{4\pi^2}[2\arccos(h)-\pi]^2.\n
%\end{eqnarray}
%
Finally, from equation~(\ref{QFI-2}), the QFI is given by
\begin{eqnarray}\label{g-0-h<hc}
\mathcal{F}_{h}(\gamma\!=\!0,h\!<\!h_{\rm c})=\frac{[2\arccos(h)-\pi]^2}{\pi^2}\frac{4g^4}{\Delta^4}t^{2} N^2 \bar n,
\end{eqnarray}
which is verified in figure~\ref{Fig6}(a). 
%
Both of these two quantum phases only support the SQL-precision when $\gamma\!=\!0$.
%
This QFI scaling is the same as the one obtained in the TC model without spin interactions ($\lambda=0$).
%

 % Figure environment removed

In the region $0<\gamma<1$, no evident approximation is feasible, prompting us to resort to numerical calculations.
%
As depicted in figure~\ref{Fig6}(a), the emergence of the HL-precision is apparent even with a small anisotropy, i.e., $\gamma=0.01$.
%
This phenomenon can be understood from figure~(\ref{Fig6}) and equation~(\ref{QFI-2}) that the presence of nonvanishing spin fluctuation, i.e., ${\rm Var}(J_z)\neq0$ for $\gamma\neq0$, ensures the manifestation of such HL-precision.


For the $\gamma=1$ case (the Ising model), we have discussed in section~(\ref{Sec.3-3}) and the HL-precision can be achieved in the weak field case.
%
If we increase the anisotropy to infinity $\gamma\!\gg\!1$ then equation~(\ref{angle}) can be approximated as $\sin\nu_k\approx{\rm sgn}(\sin k)$ and $\cos\nu_k\approx 0$.
%
Then, by equation~(\ref{Cor-Func}), we have $\ex{J_z^2}\approx N/2$ and ${\rm Var}(J_z)\approx N/2$.
%
Thus, the QFI is given by
\begin{eqnarray}\label{g>>1}
\mathcal{F}_{h} (\gamma\!\gg\!1)
& \approx &  4t^{2}\left[\left(1-\frac{2g^{2}}{\Delta^{2}}\bar{n}\right)^{2}\frac{N}{2}+\frac{4g^{4}}{\Delta^{4}}\bar{n}\frac{N}{2}\right]\n
&\approx&\frac{8g^4}{\Delta^4}t^{2} N \bar n^2+\mathcal O(\bar n),
\end{eqnarray}
%
which has been verified in figure~\ref{Fig6}(a).
%
Figure~\ref{Fig6}(b) demonstrates that the spin fluctuation ${\rm Var}(J_z)$ increase with the increase of the anisotropy $\gamma$ and ultimately converges to the limiting value $N/2$.
%
Consequently, the QFI exhibits a behavior akin to ${\rm Var}(J_z)$ due to equation~(\ref{QFI-2}) and attains its maximum at sufficiently large values of $\gamma$.




\section{Conclusion}\label{Sec.V}
In conclusion, our study systematically delves into the pivotal role played by the $XY$ spin interaction in magnetic field sensing, employing a generalized TC model as the basis. 
%
The estimated precision is quantified by the QFI.
%
The effective description of the model is derived using the time-averaged method and rigorously validated through numerical calculations.
%
Based on the effective Hamiltonian, we establish a closed relationship between the QFI and the spin fluctuation, highlighting the indispensable nature of the spin fluctuation for achieving the HL-precision in estimation. 
%
Notably, in comparison to the TC model without any spin interactions, the introduction of the anisotropic $XY$ spin interaction beats the SQL and attains the HL-precision in estimating a weak magnetic field.
%
Furthermore, our investigation reveals a direct correlation between the QFI and the spin anisotropy, underlining the significant role of the spin anisotropy in realizing high-precision quantum metrology. 
%
Additionally, we observe a scaling transition of the QFI in the TC model with the Ising interaction. 
%
These findings not only contribute to quantum metrology within cavity-QED systems but also provide valuable insights for exploring many-body effect enhanced quantum metrology.



\section*{Acknowledgments}
This work is supported by the National Natural Science Foundation of China Key Grants No.~12134015 and No.~92365202.
Y.G.S. is supported by the National Natural Science Foundation of China (Grant No.~12247158), the ``Wuhan Talent'' (Outstanding Young Talents), and the Postdoctoral Innovative Research Post in Hubei Province.
W.J.L. is supported by the National Natural Science Foundation of China (Grant No.~12205092) and the Hunan Provincial Natural Science Foundation of China (Grant No.~2023JJ40208).

\appendix

%\setcounter{appendix}{1}
\setcounter{section}{1}
\section*{Appendix A.~Effective Hamiltonian of the cavity-QED system}\label{Appendix_A}
%
Here, we employ the time-averaged method~\cite{TAM} to derive an effective description for the original system~(\ref{Total-H}).
%
This method can be regarded as a generalized rotating wave approximation, which eliminates various high-frequency contributions.
%
It states that if we can express our Hamiltonian in the following form:
\begin{eqnarray}\label{H-int-2}
H_{\rm int-pic}=\sum_i f_i\exp(-\mathrm{i} \Omega_i t)+{\rm H.c.},
\end{eqnarray}
then we can approximate it as 
\begin{eqnarray}\label{H-int-3}
H_{\rm int-pic}\approx \sum_{i,j}{\frac{1}{\bar{\Omega}_{ij}}\left[f_i^\dagger,f_j\right]\mathrm{e}^{\mathrm{i}\left(\Omega_i-\Omega_j\right)t}}.
\end{eqnarray}
%
Here, $\bar\Omega_{ij}$ is the harmonic average of frequencies $\Omega_i$ and $\Omega_j$, i.e., 
$1/\bar \Omega_{ij}=\left(1/\Omega_i+1/\Omega_j\right)/2$.


In our case, the total Hamiltonian (\ref{H-1}) in the interaction picture is given by
\begin{eqnarray}
H_{\rm int-pic}\label{H-int-1}
&=&
\mathrm{i}\frac{d U_1^\dag}{dt} U_1 +U_1^\dag HU_1  \nonumber \\ 
&=&g\left[\mathrm{e}^{\mathrm{i}\omega_{\rm{a}}t}a^\dagger\mathrm{e}^{ \mathrm{i}H_0\left(h-\omega_0\right)t}J_-\mathrm{e}^{-\mathrm{i}H_0\left(h-\omega_0\right)t}+\rm{H.c.}\right]\nonumber \\
&=& g\left[J_-a^\dagger\mathrm{e}^{-\mathrm{i}\left(\Delta+\delta\right) t}+\rm{H.c.}\right],
\end{eqnarray}
where $U_1\!=\!\exp\left\{-\mathrm{i}\left[\omega_0J_z+\omega_{\rm{a}} a^\dag a+H_0(h)\right]t\right\}$ is the
unitary transformation operator, $\left.\Delta=\omega_0-h-\omega_{\rm{a}}\right.$ is a large effective detuning, and a small residue $\delta$ comes from the commutators $[H_0(0), J_-]$ and $[H_0(0), J_z]$.
%
Under the large detuning condition ($|\Delta|\gg \lambda$), the Hamiltonian (\ref{H-int-1}) takes the form of the Hamiltonian~(\ref{H-int-2}) with $f_i=g\sigma_i^-a^\dag/2$ and $\Omega_i=\Omega_j\simeq\Delta$.
%
Then, using equation~(\ref{H-int-3}), we derive the effective Hamiltonian in the interaction picture:
\begin{eqnarray}\label{H-int-eff-4}
H_{\rm int-pic}
\approx
\sum_{i,j=1}^{N}\frac{g^2}{4\Delta}\left[\sigma_i^+ a,\sigma^-_j a^\dag\right]
= \frac{g^2}{\Delta}(J_+J_-+2J_za^\dag a).
\end{eqnarray}
%
Rewriting equation~(\ref{H-int-eff-4}) into the Schr\"{o}dinger picture, we obtain the effective Hamiltonian: 
\begin{eqnarray}
H_{\rm eff}^{\rm (s)}
&\approx&
\omega_0J_z+\omega_{\rm{a}} a^\dag a+H_0(h)
+\frac{2g^2}{\Delta}J_za^\dagger a+\frac{g^2}{\Delta}J_+J_-\nonumber \\
&\approx&
 H_0\left(h-\omega_0\right)+\frac{2g^2}{\Delta}J_za^\dagger a+\frac{g^2}{\Delta}J_+J_-,
\end{eqnarray}
where the term $a^\dag a$ has been omitted since the photon number remains conserved.
%


To ensure the validity of this approximation, in addition to the condition $|\Delta|\gg\lambda$, we also require that the timescale of the first term in $H_{\rm eff}$ be greater than the timescale introduced by the corrections (the second and the third terms)~\cite{PhysRevA.102.052615}.
%
To compare the magnitude of these terms, it is convenient to express the effective Hamiltonian in the rotating frame by employing the unitary transformation $U_{2}=\exp\left[{-\mathrm{i}\omega_0\left(J_z+a^\dagger a\right)t}\right]$, resulting in
\begin{eqnarray}
H_{\rm{eff}}^{\left(\rm{r}\right)}&=&\mathrm{i}\frac{dU_{2}^{\dagger}}{dt}U_{2}+U_{2}^{\dagger}H_{\rm{eff}}^{\rm (s)}U_{2}
\approx-\Delta a^{\dagger}a+\frac{2g^{2}}{\Delta}J_{z}a^{\dagger}a+\frac{g^{2}}{\Delta}J_{+}J_{-},
\end{eqnarray}  
which suggests that the first term scales as $|\Delta| \bar n$, the second term scales as $g^2N\bar n/|\Delta|$, and the third term scales as most $g^2N^2/|\Delta|$.
%
Thus, the conditions ensuring the validity of the approximation are $|\Delta|\bar n\gg g^2N\bar n/|\Delta|$, $|\Delta|\bar n\gg  g^2N^2/|\Delta|$, i.e., $\Delta^2\gg g^2N$, $\Delta^2\gg g^2N^2/\bar n$, and the large detuning condition ($|\Delta|\gg \lambda$).
%
Here, $\bar n=\langle a^\dag a\rangle$ is the average photon number. 







%\section*{References}

%\bibliography{jparef}
%\bibliography{JPARef}
%\bibliography{XYModel}

\providecommand{\newblock}{}
\begin{thebibliography}{10}
	\expandafter\ifx\csname url\endcsname\relax
	\def\url#1{{\tt #1}}\fi
	\expandafter\ifx\csname urlprefix\endcsname\relax\def\urlprefix{URL }\fi
	\providecommand{\eprint}[2][]{\url{#2}}
	% Bibliography created with iopart-num v2.1
	% /biblio/bibtex/contrib/iopart-num
	
	\bibitem{PhysRevLett.96.010401}
	Giovannetti V, Lloyd S and Maccone L 2006 {\em Phys. Rev. Lett.\/} {\bf 96}
	010401
	
	\bibitem{WOS:000288984900012}
	Giovannetti V, Lloyd S and Maccone L 2011 {\em Nat. Photon.\/} {\bf 5} 222--229
	
	\bibitem{RevModPhys.89.035002}
	Degen C~L, Reinhard F and Cappellaro P 2017 {\em Rev. Mod. Phys.\/} {\bf 89}
	035002
	
	\bibitem{RevModPhys.90.035005}
	Pezz\`e L, Smerzi A, Oberthaler M~K, Schmied R and Treutlein P 2018 {\em Rev.
		Mod. Phys.\/} {\bf 90} 035005
	
	\bibitem{PhysRevLett.128.160505}
	Yang J, Pang S, Chen Z, Jordan A~N and del Campo A 2022 {\em Phys. Rev.
		Lett.\/} {\bf 128} 160505
	
	\bibitem{PhysRevA.54.R4649}
	Bollinger J~J, Itano W~M, Wineland D~J and Heinzen D~J 1996 {\em Phys. Rev.
		A\/} {\bf 54} R4649--R4652
	
	\bibitem{PhysRevLett.102.100401}
	Pezz\'e L and Smerzi A 2009 {\em Phys. Rev. Lett.\/} {\bf 102} 100401
	
	\bibitem{PhysRevLett.106.130506}
	Monz T, Schindler P, Barreiro J~T, Chwalla M, Nigg D, Coish W~A, Harlander M,
	H\"ansel W, Hennrich M and Blatt R 2011 {\em Phys. Rev. Lett.\/} {\bf 106}
	130506
	
	\bibitem{PhysRevLett.107.080504}
	Krischek R, Schwemmer C, Wieczorek W, Weinfurter H, Hyllus P, Pezz\'e L and
	Smerzi A 2011 {\em Phys. Rev. Lett.\/} {\bf 107} 080504
	
	\bibitem{PhysRevA.85.022321}
	Hyllus P, Laskowski W, Krischek R, Schwemmer C, Wieczorek W, Weinfurter H,
	Pezz\'e L and Smerzi A 2012 {\em Phys. Rev. A\/} {\bf 85} 022321
	
	\bibitem{PhysRevD.23.1693}
	Caves C~M 1981 {\em Phys. Rev. D\/} {\bf 23} 1693--1708
	
	\bibitem{PhysRevLett.88.231102}
	McKenzie K, Shaddock D~A, McClelland D~E, Buchler B~C and Lam P~K 2002 {\em
		Phys. Rev. Lett.\/} {\bf 88} 231102
	
	\bibitem{MA201189}
	Ma J, Wang X, Sun C and Nori F 2011 {\em Phys. Rep.\/} {\bf 509} 89--165
	
	\bibitem{PhysRevLett.110.163604}
	Pezz\'e L and Smerzi A 2013 {\em Phys. Rev. Lett.\/} {\bf 110} 163604
	
	\bibitem{PhysRevA.92.023603}
	Muessel W, Strobel H, Linnemann D, Zibold T, Juli\'a-D\'{\i}az B and Oberthaler
	M~K 2015 {\em Phys. Rev. A\/} {\bf 92} 023603
	
	\bibitem{PhysRevA.102.052423}
	Su Y, Liang H and Wang X 2020 {\em Phys. Rev. A\/} {\bf 102} 052423
	
	\bibitem{WOS:A1980KA25400008}
	Ramsey N 1980 {\em Phys. Today\/} {\bf 33} 25--30
	
	\bibitem{PhysRevLett.86.5870}
	Meyer V, Rowe M~A, Kielpinski D, Sackett C~A, Itano W~M, Monroe C and Wineland
	D~J 2001 {\em Phys. Rev. Lett.\/} {\bf 86} 5870--5873
	
	\bibitem{RevModPhys.87.637}
	Ludlow A~D, Boyd M~M, Ye J, Peik E and Schmidt P~O 2015 {\em Rev. Mod. Phys.\/}
	{\bf 87} 637--701
	
	\bibitem{Louchet_Chauvet_2010}
	Louchet-Chauvet A, Appel J, Renema J~J, Oblak D, Kjaergaard N and Polzik E~S
	2010 {\em New J. Phys.\/} {\bf 12} 065032
	
	\bibitem{PhysRevLett.112.190403}
	Kessler E~M, K\'om\'ar P, Bishof M, Jiang L, S\o{}rensen A~S, Ye J and Lukin
	M~D 2014 {\em Phys. Rev. Lett.\/} {\bf 112} 190403
	
	\bibitem{WALLS1981118}
	Walls D and Zoller P 1981 {\em Phys. Lett. A\/} {\bf 85} 118--120
	
	\bibitem{WOS:000256613000015}
	Goda K, Miyakawa O, Mikhailov E~E, Saraf S, Adhikari R, McKenzie K, Ward R,
	Vass S, Weinstein A~J and Mavalvala N 2008 {\em Nat. Phys.\/} {\bf 4}
	472--476
	
	\bibitem{Abbott_2009}
	Abbott B~P \etal 2009 {\em Rep. Prog. Phys.\/} {\bf 72} 076901
	
	\bibitem{budker_optical_2007}
	Budker D and Romalis M 2007 {\em Nat. Phys.\/} {\bf 3} 227--234
	
	\bibitem{PhysRevLett.109.253605}
	Sewell R~J, Koschorreck M, Napolitano M, Dubost B, Behbood N and Mitchell M~W
	2012 {\em Phys. Rev. Lett.\/} {\bf 109} 253605
	
	\bibitem{PhysRevLett.120.260503}
	Troiani F and Paris M~G~A 2018 {\em Phys. Rev. Lett.\/} {\bf 120} 260503
	
	\bibitem{PhysRevLett.126.010502}
	Chu Y, Zhang S, Yu B and Cai J 2021 {\em Phys. Rev. Lett.\/} {\bf 126} 010502
	
	\bibitem{doi:10.1073/pnas.1004037107}
	Carlton P~M, Boulanger J, Kervrann C, Sibarita J~B, Salamero J, Gordon-Messer
	S, Bressan D, Haber J~E, Haase S, Shao L, Winoto L, Matsuda A, Kner P, Uzawa
	S, Gustafsson M, Kam Z, Agard D~A and Sedat J~W 2010 {\em Proc. Natl. Acad.
		Sci.\/} {\bf 107} 16016--16022
	
	\bibitem{WOS:000316154700018}
	Taylor M~A, Janousek J, Daria V, Knittel J, Hage B, Bachor H~A and Bowen W~P
	2013 {\em Nat. Photon.\/} {\bf 7} 229--233
	
	\bibitem{TAYLOR20161}
	Taylor M~A and Bowen W~P 2016 {\em Phys. Rep.\/} {\bf 615} 1--59
	
	\bibitem{PhysRevLett.72.3439}
	Braunstein S~L and Caves C~M 1994 {\em Phys. Rev. Lett.\/} {\bf 72} 3439--3443
	
	\bibitem{Dutra-Book:2005}
	Dutra S 2005 {\em Cavity quantum electrodynamics\/} (John Wiley and Sons Inc.)
	
	\bibitem{Raimond-RMP:2001}
	Raimond J~M, Brune M and Haroche S 2001 {\em Rev. Mod. Phys.\/} {\bf 73}
	565--582
	
	\bibitem{hepp1973superradiant}
	Hepp K and Lieb E~H 1973 {\em Ann. Phys.\/} {\bf 76} 360--404
	
	\bibitem{PhysRevA.8.2517}
	Hepp K and Lieb E~H 1973 {\em Phys. Rev. A\/} {\bf 8} 2517--2525
	
	\bibitem{Buluta2009}
	Buluta I and Nori F 2009 {\em Science\/} {\bf 326} 108--111
	
	\bibitem{Georgescu2014}
	Georgescu I~M, Ashhab S and Nori F 2014 {\em Rev. Mod. Phys.\/} {\bf 86}
	153--185
	
	\bibitem{Scarani2009}
	Scarani V, Bechmann-Pasquinucci H, Cerf N~J, Du\ifmmode~\check{s}\else
	\v{s}\fi{}ek M, L\"utkenhaus N and Peev M 2009 {\em Rev. Mod. Phys.\/} {\bf
		81} 1301--1350
	
	\bibitem{obrien_photonic_2009}
	O'Brien J~L, Furusawa A and Vu\v{c}kovi\'c J 2009 {\em Nat. Photon.\/} {\bf 3}
	687--695
	
	\bibitem{Giovannetti2011}
	Giovannetti V, Lloyd S and Maccone L 2011 {\em Nat. Photon.\/} {\bf 96}
	222--229
	
	\bibitem{Liu_2019}
	Liu J, Yuan H, Lu X~M and Wang X 2019 {\em J. Phys. A: Math. Theor.\/} {\bf 53}
	023001
	
	\bibitem{YuguoSu2021}
	Su Y and Wang X 2021 {\em Results Phys.\/} {\bf 24} 104159
	
	\bibitem{PhysRevLett.130.170801}
	Chu Y, Li X and Cai J 2023 {\em Phys. Rev. Lett.\/} {\bf 130} 170801
	
	\bibitem{deng2023}
	Deng X, Li S, Chen Z~J, Ni Z, Cai Y, Mai J, Zhang L, Zheng P, Yu H, Zou C~L,
	Liu S, Yan F, Xu Y and Yu D 2023 {\em \rm {arXiv}:2306.16919v1\/}
	
	\bibitem{WOS:000368673800032}
	Hosten O, Engelsen N~J, Krishnakumar R and Kasevich M~A 2016 {\em Nature\/}
	{\bf 529} 505
	
	\bibitem{PhysRevLett.116.093602}
	Cox K~C, Greve G~P, Weiner J~M and Thompson J~K 2016 {\em Phys. Rev. Lett.\/}
	{\bf 116} 093602
	
	\bibitem{Flower_2019}
	Flower G, Goryachev M, Bourhill J and Tobar M~E 2019 {\em New J. Phys.\/} {\bf
		21} 095004
	
	\bibitem{Gietka_2021}
	Gietka K, Mivehvar F and Busch T 2021 {\em New J. Phys.\/} {\bf 23} 043020
	
	\bibitem{Kim1999}
	Kim M~S and Agarwal G~S 1999 {\em Phys. Rev. A\/} {\bf 59} 3044--3048
	
	\bibitem{WOS:000263818900002}
	Scheel S 2009 {\em J. Mod. Opt.\/} {\bf 56} 141--160
	
	\bibitem{PhysRevA.94.022313}
	Penasa M, Gerlich S, Rybarczyk T, M\'etillon V, Brune M, Raimond J~M, Haroche
	S, Davidovich L and Dotsenko I 2016 {\em Phys. Rev. A\/} {\bf 94} 022313
	
	\bibitem{wan2023quantum}
	Wan Q~K, Shi H~L and Guan X~W 2023 {\em \rm {arXiv}:2305.08045\/}
	
	\bibitem{PhysRev.170.379}
	Tavis M and Cummings F~W 1968 {\em Phys. Rev.\/} {\bf 170} 379--384
	
	\bibitem{PhysRev.188.692}
	Tavis M and Cummings F~W 1969 {\em Phys. Rev.\/} {\bf 188} 692--695
	
	\bibitem{PhysRevLett.120.050507}
	Xu K, Chen J~J, Zeng Y, Zhang Y~R, Song C, Liu W, Guo Q, Zhang P, Xu D, Deng H,
	Huang K, Wang H, Zhu X, Zheng D and Fan H 2018 {\em Phys. Rev. Lett.\/} {\bf
		120} 050507
	
	\bibitem{doi:10.1126/science.aay0600}
	Song C, Xu K, Li H, Zhang Y~R, Zhang X, Liu W, Guo Q, Wang Z, Ren W, Hao J,
	Feng H, Fan H, Zheng D, Wang D~W, Wang H and Zhu S~Y 2019 {\em Science\/}
	{\bf 365} 574--577
	
	\bibitem{James2007}
	James D~F and Jerke J 2007 {\em Can. J. Phys.\/} {\bf 85} 625--632
	
	\bibitem{TAM}
	Shore B~W and Knight P~L 1993 {\em J. Mod. Opt.\/} {\bf 40} 1195--1238
	
	\bibitem{Sun06}
	Quan H~T, Song Z, Liu X~F, Zanardi P and Sun C~P 2006 {\em Phys. Rev. Lett.\/}
	{\bf 96} 140604
	
	\bibitem{WOS:000243867300038}
	Schuster D~I, Houck A~A, Schreier J~A, Wallraff A, Gambetta J~M, Blais A,
	Frunzio L, Majer J, Johnson B, Devoret M~H, Girvin S~M and Schoelkopf R~J
	2007 {\em Nature\/} {\bf 445} 515--518
	
	\bibitem{JOHANSSON20121760}
	Johansson J, Nation P and Nori F 2012 {\em Comput. Phys. Commun.\/} {\bf 183}
	1760--1772
	
	\bibitem{helstrom1969quantum}
	Helstrom C~W 1969 {\em J. Stat. Phys.\/} {\bf 1} 231--252
	
	\bibitem{holevo2011probabilistic}
	Holevo A~S 2011 {\em Probabilistic and statistical aspects of quantum theory\/}
	vol~1 (Springer Science \& Business Media)
	
	\bibitem{PhysRevD.23.357}
	Wootters W~K 1981 {\em Phys. Rev. D\/} {\bf 23} 357--362
	
	\bibitem{LIEB1961407}
	Lieb E, Schultz T and Mattis D 1961 {\em Ann. Phys.\/} {\bf 16} 407--466
	
	\bibitem{PhysRevA.102.052615}
	Barberena D, Lewis-Swan R~J, Thompson J~K and Rey A~M 2020 {\em Phys. Rev. A\/}
	{\bf 102}(5) 052615
	
\end{thebibliography}




\end{document}
