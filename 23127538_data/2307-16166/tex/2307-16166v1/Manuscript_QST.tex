%\pdfminorversion=4
\documentclass[12pt]{iopart}
%\usepackage{array}
\usepackage{hyperref}
\usepackage{graphicx}
%\usepackage{subfig}
\usepackage[caption=false]{subfig}
\usepackage{exscale}
\usepackage{latexsym}

\usepackage{epstopdf}
%\usepackage{subfigure}
\usepackage[percent]{overpic}

\usepackage{iopams}
\usepackage{float}
%\usepackage{mathtools}
\expandafter\let\csname equation*\endcsname\relax
\expandafter\let\csname endequation*\endcsname\relax
\usepackage{amsmath,amssymb}
\usepackage{cases}
\usepackage{dsfont}
\usepackage{txfonts}
\usepackage{array}
\bibliographystyle{iopart-num}
%\usepackage{citesort}
\usepackage{cleveref}
%\usepackage{threeparttable}
%\usepackage{bm}
\usepackage{multirow}
\usepackage{booktabs}
\crefname{equation}{equation}{equations}
\crefname{figure}{figure}{figures}
\crefname{section}{section}{sections}
%\crefname{appendix}{appendix}{appendices}
\newcommand{\gguide}{{\it Preparing graphics for IOP Publishing journals}}
%Uncomment next line if AMS fonts required
\begin{document}

\title{Quantum metrology enhanced by an XY model in cavity-QED}

\author{Yuguo Su$^{1,*}$, Wangjun Lu$^{2}$, and Hai-Long Shi$^{1,3,*}$}

\address{$^{1}$ Innovation Academy for Precision Measurement Science and Technology, Chinese Academy of Sciences, Wuhan 430071, China}
\address{$^{2}$ Department of Maths and Physics, Hunan Institute of Engineering, Xiangtan 411104, China}
\address{$^{3}$ University of Chinese Academy of Sciences, Beijing 100049, China}
\address{$^{*}$ Authors to whom any correspondence should be addressed.}
%\ead{hl\_shi@yeah.net}
\eads{\mailto{suyuguo@apm.ac.cn} and \mailto{hl\_shi@yeah.net}}

\begin{abstract}
%Performing a quantum-enhanced estimation without using experiment-friendly states has long been pursued for reducing the experimental complexity.
Quantum metrology employs given quantum resources to provide a route to overcome practical limits of measurements, which inevitably distort the information.
%
%One of the main quests in quantum metrology is to attain the ultimate precision limit with given resources, where the resources are not only of the number of queries, but more importantly of the allowed strategies.
%%
%Quantum metrology employs quantum resources to enhance the measurement sensitivity beyond that can be achieved classically. While multiphoton entangled N00N states can in principle beat the shot-noise limit and reach the Heisenberg limit, high N00N states are difficult to prepare and fragile to photon loss which hinders them from reaching unconditional quantum metrological advantages.
%%
%Quantum metrology is the use of quantum techniques such as entanglement to yield higher statistical precision than purely classical approaches.
%%
%Quantum metrology provides a route to overcome practical limits in sensing devices.
%It holds particular relevance to biology, where sensitivity and resolution constraints restrict applications both in fundamental biophysics and in medicine.
%
Here we propose a promising protocol based on a cavity quantum electrodynamics system where the trapped spins are responsible for sensing a weak field meanwhile the cavity field allows us to perform high-precision measurements.
%
By calculating the quantum Fisher information, we show that the XY interaction among spins is indispensable for enhancing the measurement precision to the Heisenberg limit.
%
%Furthermore, experimentally accessible scheme is also proposed and we show that the theoretically optimal measurement precision quantified by QFI is attainable.
%
We discuss the effect of the anisotropy parameter $\gamma$ and show that Heisenberg limit sensitivity could be approached for a weak magnetic field in the ordered phase region.
%
The metrological gain provided by our protocol could be enhanced beyond the standard quantum limit by 10-20 dB when using 100 photons as a resource. 
%
Superconducting circuit and strong coupling optical cavity systems are ideal platforms to implement our protocol.
\end{abstract}

%
% Uncomment for keywords
%\vspace{2pc}
\noindent{\it Keywords}: quantum metrology, quantum Fisher information, Heisenberg limit, cavity quantum electrodynamics, XY model, quantum sensing, quantum magnetometry
%
% Uncomment for Submitted to journal title message
%\submitto{\JPA}
%
% Uncomment if a separate title page is required
%\maketitle
%
% For two-column output uncomment the next line and choose [10pt] rather than [12pt] in the \documentclass declaration
%\ioptwocol
%


\section{Introduction}\label{Sec.I}
Quantum metrology \cite{WOS:000288984900012,RevModPhys.89.035002}, which deals with the measurement and discrimination procedures of an unknown parameter, aims to achieve supersensitivity over classical metrology by exploiting various quantum resources, such as quantum entanglement  \cite{PhysRevA.54.R4649,PhysRevLett.106.130506}  and quantum squeezing \cite{PhysRevA.92.023603,MA201189,PhysRevA.102.052423}.
%
The widespread applications of quantum metrology have emerged in manifold experimental fields, including Ramsey spectroscopy \cite{WOS:A1980KA25400008,PhysRevLett.86.5870}, as well as atomic clocks \cite{RevModPhys.87.637,Louchet_Chauvet_2010,PhysRevLett.112.190403}, gravitational-wave detectors \cite{WALLS1981118,WOS:000256613000015,Abbott_2009}, magnetometry \cite{budker_optical_2007,PhysRevLett.109.253605,PhysRevLett.120.260503}, and biophysical  measurements \cite{doi:10.1073/pnas.1004037107,WOS:000316154700018,TAYLOR20161}.
%
%A central challenge of quantum metrology is to identify fundamental  limits and to develop strategies for enhancing the precision of parameter estimation.
The estimating limit of an unknown parameter $\xi$ can be determined via  the quantum Cram\'er-Rao bound (QCRB) \cite{PhysRevLett.72.3439} as $\delta \xi\geq1/\sqrt{\nu\mathcal{F}_{\xi}}$, where $\mathcal{F}_{\xi}$ is the quantum Fisher information (QFI) and $\nu$ is the number of independent repetitions.
Phase measurements will be limited in precision due to the inherent quantum nature of states.
In quantum phase estimation, employing nonentangled $N$-particle states, the precision is bound to the standard quantum limit (SQL) or shot noise limit, i.e., $\mathcal F_{\xi}\propto N$.
However, if entangled states are used, it is possible to approach the Heisenberg limit (HL), i.e., $\mathcal F_{\xi}\propto N^2$.

%Quantum metrology aims to provide high-precision measurements of physical parameters by exploiting various quantum resources, such as quantum entanglement  \cite{PhysRevA.54.R4649,PhysRevLett.106.130506}  and quantum squeezing \cite{PhysRevA.92.023603}.
%Significant advances in experimental technology have promoted the widespread applications of quantum metrology in atomic clocks \cite{RevModPhys.87.637,Louchet_Chauvet_2010,PhysRevLett.112.190403}, gravitational wave detectors \cite{Abbott_2009}, magnetometry \cite{budker_optical_2007,PhysRevLett.109.253605,PhysRevLett.120.260503} and quantum imaging \cite{Genovese_2016,PhysRevA.95.063847,PhysRevA.96.062107}.
%%
%A central challenge of quantum metrology is to identify fundamental  limits and to develop strategies for enhancing the precision of parameter estimation.
%The theoretical limit for estimation of an unknown parameter  $\xi$ is given by  the quantum Cram\'er-Rao bound (QCRB) \cite{PhysRevLett.72.3439} $\delta \xi\geq1/\sqrt{\nu\mathcal{F}_{\xi}}$ where $\mathcal{F}_{\xi}$ is the quantum Fisher information (QFI) and $\nu$ is the number of independent repetitions.
%%
%In quantum phase estimation, the highest precision which  we can achieve by using nonentangled $N$-particle states is called the standard quantum limit (SQL) or shot noise limit, i.e., $\mathcal F_{\xi}\propto N$.
%%
%Thus, entangled states are necessary to enhance  this limit  to approach  the Heisenberg limit (HL), i.e., $\mathcal F_{\xi}\propto N^2$.
%Instead of the direct use of entangled states, many schemes have been turned to use the quantum features generated by many-body dynamics to achieve HL, such as optimal adapative controls \cite{pang_optimal_2017,PhysRevA.96.020301,PhysRevLett.128.160505}, quantum nondemolition measurement \cite{Appel2009,Kuzmich_1998,Louchet_Chauvet_2010,RevModPhys.82.1041}, one-axis twisting \cite{PhysRevA.81.021804,Kitagawa1993,Sorensen2001,Haine2014}, two-axis countertwisting \cite{Kitagawa1993,Ma2009}, twist-and-turn squeezing \cite{PhysRevA.92.023603,Law2001}, spin changing collisions \cite{Liicke2011,Duan2000,Pu2000,Nolan2016}, and superradiant phase transition \cite{PhysRevLett.124.120504}.





Cavity quantum electrodynamics (Cavity-QED)  stimulates intense theoretical and experimental investigations in the study of a variety of physics, ranging from the  light-matter interaction \cite{Dutra-Book:2005,Raimond-RMP:2001} to quantum entanglement \cite{doi:10.1063/1.882326,RevModPhys.73.565}, coherence dynamics \cite{doi:10.1126/science.1078446,WOS:000223746000038}, and other  quantum-classical phenomena  \cite{PhysRevLett.76.1800,mckeever_experimental_2003}.
%
Cavity-QED provides an ideal platform to simulate cavity-atom interaction from weak to strong that make for quantum computing \cite{Bennett2000,Knill2001,Buluta_2011}, quantum simulation \cite{Buluta2009,Georgescu2014}, quantum key distribution \cite{Scarani2009,obrien_photonic_2009}, and quantum metrology \cite{Giovannetti2011,Liu_2019,YuguoSu2021}.
In particular, the  cavity-QED also provides  promising  applications in precision measurements \cite{WOS:000368673800032,PhysRevLett.116.093602,Flower_2019,Gietka_2021} and  quantum-enhanced sensing  \cite{Kim1999,WOS:000263818900002,PhysRevA.94.022313}.
%

%Quantum metrology employs given quantum resources to provide a route to overcome practical limits of measurements, which inevitably distort the information.
%Here we propose  a promising protocol based on a cavity-quantum electrodynamics (QED) system where the trapped spins are responsible for sensing a weak field meanwhile the cavity field allows us to perform high precision measurements.
%%
%By calculating the quantum Fisher information (QFI), we show that the XY interaction among spins is indispensable for enhancing  the measurement precision to the Heisenberg limit.
%%
%Furthermore, experimentally accessible scheme is also proposed and we show that the theoretically optimal measurement precision quantified by QFI is attainable.
%%
%The metrological gain provided by our protocol could be enhanced beyond the standard quantum limit by 10-20 dB when using 100 photons as a resource. 
%%
%We discuss the effect of the anisotropy parameter $\gamma$ and show that Heisenberg limit sensitivity could be achieved for a weak magnetic field in the ordered phase region. 

%In this work, we  achieve a quantum enhanced parameter estimation in cavity-QED without preparing entangled initial  states.
%
In this work, we propose  a quantum-enhanced protocol for sensing a weak magnetic field in a cavity-QED system via experiment-friendly states.
%
We demonstrate that the XY interaction among spins enhances the spin-cavity entanglement and leads to reaching the HL.
%
For a weak magnetic field, we discuss the effect of the anisotropy parameter $\gamma$ and show that HL sensitivity could be achieved in the ordered phase region. 
%
Measurement precision could be enhanced beyond the standard quantum limit by 10-20 dB when  100 photons are considered as a resource.
%
Our methods provide a promising  opportunity for experimental implementation of a high-precision quantum measurement with experiment-friendly states and pave the way for future applications in quantum metrology.

This paper is organized as follows.
In ~\cref{Sec.II}, we concretely introduce the cavity-QED system and the Holstein-Primakoff (H-P) transformation.
We derive the effective Hamiltonian obtained by the time-averaged method and demonstrate its validity by comparing the results of the original and effective one.
In~\cref{Sec.III}, we show the reduced density matrix of the system via the H-P transformation, which could describe the time evolution of numerous intriguing quantities.
Furthermore, we gain the QFI of the cavity-QED system in~\cref{Sec.IV}.
%QFIs of the Ising model and anisotropic XY model is presented in~\cref{Sec.IV.1} and~\cref{Sec.IV.2}.
%The HL sensitivity is performed by the Ising case in~\cref{Sec.IV.1}.
We obtain the HL sensitivity with the Ising model in~\cref{Sec.IV.1}.
In~\cref{Sec.IV.2},  the effect of the anisotropy parameter $\gamma$ on the QFI is calculated and discussed.
Finally, a summary is given in ~\cref{Sec.V}.

% Figure environment removed

\section{System and H-P transformation}\label{Sec.II}
We consider an atom-light coupling system describing the coupling of a single bosonic cavity mode to a collection of $N_b$ two-level atoms (or spin-1/2 spins),  as depicted in \cref{Fig1}:
%\begin{eqnarray}
%&&H=\omega_0 J_z+\omega_a a^\dagger a+H_0\left(h\right)+H_{\rm I},\nonumber\\
%&&H_{0\!}\!\left(\!h\!\right)\!=\!-\frac{1}{2}\!\sum_{i=1}^{N_c}\!{\!\left\{\!\frac{\lambda}{2}\!\left[\!\left(\!1\!+\!\gamma\!\right)\!\sigma^x_i\!\sigma^x_{\!i+1\!}\!+\!\left(\!1\!-\!\gamma\!\right)\!\sigma^y_i\!\sigma^y_{\!i+1}\!\right]\!+\!h\sigma^z_i\!\right\}},\nonumber\\
%&&H_{\rm I}=g\left(a^\dagger J_-+aJ_+\right).\label{H-1}
%\end{eqnarray}
\begin{eqnarray}
&&H=\omega_0 J_z+\omega_a a^\dagger a+H_0\left(h\right)+H_{\rm I},\label{H-1}\\
&&H_{0}\left(h\right)=-\frac{1}{2}\sum_{i=1}^{N_c}{\left\{\frac{\lambda}{2}\left[\left(1+\gamma\right)\sigma^x_i\sigma^x_{i+1}+\left(1-\gamma\right)\sigma^y_i\sigma^y_{i+1}\right]+h\sigma^z_i\right\}},\nonumber\\
&&H_{\rm I}=g\left(a^\dagger J_-+aJ_+\right).\nonumber
\end{eqnarray}
Here, $H_{\rm I}$ denotes the interaction Hamiltonian and $J_{x,y,z}=\sum_{j=1}^{N_c}{\sigma_j^{x,y,z}/2}$, $J_\pm=J_x\pm \rmi  J_y$ are the collective spin operators; $a$ and $a^\dagger$ are creation and annihilation operators for the cavity mode; $\omega_0$, $\omega_a$ and $2g$ are the spin transition frequency, the cavity frequency and the single-photon Rabi frequency, respectively.
The XY Hamiltonian $H_0\left(h\right)$ describes the interaction between the spins, where $\lambda$ is the nearest neighbor interaction, $\sigma^{\alpha}_{i}$ denotes the Pauli matrix ($\alpha=x,y,z$) on site $i$, $N_c$ is the number of sites, $\gamma$ the degree of anisotropy, and $h$ a transverse field.

Employing the unitary transformation $U=\exp\left\{-\rmi \left[\omega_0 J_z+\omega_a a^\dagger a+H_0\left(h\right)\right]t\right\}$ 
%$U=\rme^{-\rmi \left[\omega_0 J_z+\omega_a a^\dagger a+H_0\left(h\right)\right]t}$
and the high-frequency approximation for spins ($\left|h-\omega_0\right|\gg\lambda/2$), the total Hamiltonian (\ref{H-1}) in the interaction picture could be read as 
$H= g\left[J_-a^\dagger\rme^{-\rmi \left(\Delta+\delta\right) t}+\text{H.c.}\right]$ with a large effective detuning $\Delta=\omega_0-h-\omega_a$ and a small field-irrelative residue $\delta$ ($\Delta\gg\delta\sim\lambda$\ or\ $\lambda\gamma$).
%
Utilizing the time-averaged method of the reference~\cite{James2007}, the total Hamiltonian can be approximated as (see more details in appendix~A):
\begin{equation}\label{H2}
H\simeq H_0\left(h-\omega_0\right)+\frac{2g^2}{\Delta}J_za^\dagger a+\frac{g^2}{\Delta}J_+J_-,
\end{equation}
which is written in the Schr\"{o}dinger picture and the photon number is conserved.
The valid condition of the approximation is that the timescale of the first term should be greater than the second and third terms, i.e., $\Delta\gg g\sqrt{N}$ and $\Delta\gg g\sqrt{N^{2}/\bar{n}}$ ($\bar{n}=\left\langle a^+a\right\rangle$ is the average photon number).
The core of the time-averaged method is to eliminate high-frequency contributions, and thus it can be regarded as a natural generalization of the rotating-wave approximation \cite{TAM}.


% Figure environment removed

Generally, the average photon number is much larger than the number of up spins, i.e., $a^\dag a\gg J_z= [J_+,J_-]/2, $ then we can further reduce the Hamiltonian (\ref{H2}) to obtain the following effective Hamiltonian: 
\begin{equation}
H_{\text{eff}}=H_0\left(h-\omega_0\right)+\frac{2g^2}{\Delta}J_za^\dagger a.\label{H_eff}
\end{equation}
%
Under the same assumption $a^\dag a\gg 1$, by using the H-P transformation $S_+=\sqrt{N_b}a^\dagger\sqrt{1-a^\dagger a/N_b}$, $S_-=\sqrt{N_b}\sqrt{1-a^\dagger a/N_b}a$ and $S_z=-N_b/2+a^\dagger a$, the cavity-QED system can be mapped onto a central spin system where cavity photons correspond to $N_b$ bath spins.
%
The corresponding Hamiltonian calculated from (\ref{H_eff}) reads
\begin{equation}
H_{\rm CSM}=H_0\left(h-\omega_0+\eta N_b\right)-2\eta J_z S_z
,\label{CentralSpinContext}
\end{equation}
where $\eta=-g^2/\Delta^2$ and $S_{z}$ can be treated as  the  collective spin operator of the bath spins.
%
This equivalent Hamiltonian (\ref{CentralSpinContext}) is also known as the generalized Hepp-Coleman model \cite{Sun06} and we will use it to simplify some calculations.

To demonstrate our effective Hamiltonian is a valid approximation, we calculate the expectation $\left\langle M\left(t\right)\right\rangle $ with respect to the final state of the spins.
Here, the observable is $J_{\varphi}=J_{x}\cos\varphi+J_{y}\sin\varphi$, and the initial state is a product state of the coherent state $\left|\alpha\right\rangle $ and spin-coherent state 
\begin{equation}
	\left|\mu\right\rangle  =  \frac{\exp\left(\mu J_{-}\right)}{\left(1+\left|\mu\right|^{2}\right)^{j}}\left|0\right\rangle =\frac{1}{\left(1+\left|\mu\right|^{2}\right)^{j}}\sum_{p=0}^{2j}\sqrt{\frac{\left(2j\right)!}{p!\left(2j-p\right)!}}\mu^{p}\left|p\right\rangle ,
\end{equation}
where $\left|j,j\right\rangle =\left|0\right\rangle $ is the eigenstate of $J_{z}$ with eigenvalue $j=N/2$, and the phase parameter $\mu$ an be parameterized by angles $\left(\theta,\phi\right)$ via the stereographic projection $\mu=e^{i\phi}\tan\frac{\theta}{2}$ and $\phi\in\left[0,2\pi\right)$.

As shown in \cref{Fig2}, comparing the numerical expectations of the original and effective Hamiltions (blue circles and orange dashed line) , one could know that the result of the effective Hamiltonian exactly corresponds with the result of the original Hamiltonian, which means the effective Hamiltonian can fully describe the system.

% Figure environment removed


Furthermore, we introduce the difference ratio 
\begin{equation}
K=\frac{2\left[\left\langle J_{\varphi}\left(t\right)\right\rangle _{\text{eff}}-\left\langle J_{\varphi}\left(t\right)\right\rangle _{\text{ori}}\right]}{N_{c}}
\end{equation}
to quantify the difference of the expectation between the original and effective Hamiltonians for time $t$ and an arbitrary rotation angle $\varphi$.
As seen in \cref{Fig3}, one could find that the difference ratio ($\left|K_{\text{max}}\right|\approx1.28\%$) is acceptably small in our parameter region, which is mostly caused by the validity of $N_{c}\ll\bar{n}$, $\Delta\gg g\sqrt{N}$ and the finite Hilbert space of the field.

%Although our theoretical starting point will be effective Hamiltonians (\ref{H-eff} and \ref{CentralSpinContext}) without any dissipation, we emphasize that this simplification make sense in the strongly coupled cavity-QED systems.
%%
%The cavity-atom system with   a strong light-atom interaction  and   large detuning will suppress  the atomic
%spontaneous emission $\gamma_e$ and the cavity dissipation $\gamma_d$~\cite{WOS:000243867300038,WOS:000322086100035,PhysRevA.71.013817,PhysRevA.67.033806}.
%%
%In superconducting circuits, the high cooperativity condition $C=2g^2/(\gamma_\text{d}\gamma_\text{e})>10^4$  has been achievable ~\cite{WOS:000257665300034,PhysRevLett.128.123602}.
%%
%Recent experimental work has realized the multi-atom cavity with a strong coupling between deterministic atom arrays and a high finesse miniature optical cavity \cite{PhysRevLett.130.173601}. 

\section{Reduced density matrix of the bath spin system}\label{Sec.III}

Here, we derive the  evolution of the reduced density matrix for the cavity, with which one could obtain the time evolution of observables and other quantities about the cavity (or, equivalently, bath spins).
Taking advantage of the mapping between the cavity-QED and the central spin system (\ref{CentralSpinContext}), we  obtain the evolution of the reduced density matrix for the cavity:
\begin{equation}
\rho_b\left(t\right)
=\!\sum_{i,j}{\!c_i c_j^{*}\!\left|i\right\rangle\!\left\langle j\right|\!\mathrm{Tr}_\text{c}\!\left[\rme^{\rmi N_ctH_j}\rme^{-\rmi N_ctH_i}\frac{\rme^{-\beta H_0\left(h\right)}}{Z\left(\beta,h\right)}\!\right]},\label{rho_b}
\end{equation}
where $H_{m}=H_0\left(h-\omega_0+\eta N_b+2\eta m\right)$ ($m=i,j$) is the effective Hamiltonian,  
$Z\left(\beta,h\right)=\text{Tr}\left[\rme^{-\beta H_0\left(h\right)}\right]=\prod_k 2\cosh \left(\beta\Lambda_k/2\right)$ is the partition function, and $\left|m\right\rangle $ is a Dicke state.
In fact, the XY Hamiltonian can be diagonalized as $H_0\left(h\right)=\sum_k{\Lambda_k\left(b_k^\dagger b_k-1/2\right)}$ by employing the Jordan-Wigner, Fourier and Bogoliubov transformations \cite{WOS:000285157400001}, where 
$\Lambda_{k}=\sqrt{\left(h-\lambda \cos k\right)^2+\lambda^2\gamma^2\sin^2k}$ is the excitation energy, $b_k$ and $b_k^\dagger$ are anticommuting fermion operators (see more details in appendix~B).

We can reconstruct the fermion operators by following the Bogliubov transformation
%\begin{eqnarray}
%d_{m,\pm k}=\cos\left(\theta_{m,k}\right)b_{m,\pm k}\mp\rmi \sin\left(\theta_{m,k}\right)\left(b_{m,\mp k}\right)^{\dagger},\label{d}
%\end{eqnarray}
\begin{equation}
d_{m,\pm k}=\cos\left(\theta_{m,k}\right)b_{m,\pm k}\mp\rmi \sin\left(\theta_{m,k}\right)\left(b_{m,\mp k}\right)^{\dagger},\label{d}
\end{equation}
where angles $\theta_{m,k}=\left(\mu_{m,k}-\nu_k\right)/2$, $\mu_{m,k}$ and $\nu_k$ are determined by the diagonalization conditions of the effective Hamiltonian $H_{m}$ and the original Hamiltonian $H_0\left(h\right)$.
From \cref{d}, in the eigenspace $\left\{\prod_{k>0}{\left|\phi_{l}\right\rangle_k};l=0,\pm,2\right\}$ of the original bath spin chain $H_0\left(h\right)$,
we could gain the reduced density matrix of the central spins (see more details in appendix~B)
%\begin{widetext}
\begin{align}
 \rho_b\left(t\right)
&=\sum_{i,j}\bigg\{\frac{c_i c_j^{*}}{Z\left(\beta,h\right)}\left|i\right\rangle\left\langle j\right|\prod_{k>0}\left[2+2\cosh\left(\beta \Lambda_k\right)C\left(\theta_{i,k},E_{i,k}t\right)C\left(\theta_{j,k},E_{j,k}t\right)\right.\nonumber\\
&\quad \left.+\rme^{\beta \Lambda_k}A\left(\theta_{j,k},E_{j,k}t\right)A^{*}\left(\theta_{i,k},E_{i,k}t\right)+\rme^{-\beta \Lambda_k}B\left(\theta_{j,k},E_{j,k}t\right)B^{*}\left(\theta_{i,k},E_{i,k}t\right)
\right]\bigg\},\label{Final StateContext}
\end{align}
%\end{widetext}
where the coefficients are defined as $A\left(X,Y\right)=\rme^{-\rmi Y}+2\rmi \sin^2X\sin Y$, $B\left(X,Y\right)=\rme^{-\rmi Y}+2\rmi \cos^2X\sin Y$, $C\left(X,Y\right)=\sin2X\sin Y$ and $X$,$Y$ are real variables, and $E_{m,k}=E_{m,-k}=\sqrt{\left(h-\omega_0+\eta N_b+4\eta m/N_b-\lambda\cos k\right)^2+\lambda^2\gamma^2\sin^2 k}$ are  the excitation energy.
%


%As being shown in \cref{CFI&QFI}, the measurement precision $\delta h$ could be divided into three domains: a small-$\bar{n}$ domain ($\bar{n}\sim N_c$), a valid-$\bar{n}$ domain ($N_c\ll \bar{n}\ll N_b$) and a large-$\bar{n}$ domain ($\bar{n}\sim N_b$).
%In the valid-$\bar{n}$ domain, the $\delta h$ approaches the QFI with the  ratio $\left(\delta h\right)^{-2}/\mathcal{F}_h>91\%$, see  the inset.
%The behavior of the $(\delta h)^{-2}/t^2$ is linear in the small-$\bar{n}$ domain due to the violation of the  assumption ($N_c\ll \bar{n}$).
%The measurement precision $\delta h$ does not match the QFI in the large-$\bar{n}$ domain because  the Holstein-Primakoff transformation requires  a small excitation,  i.e. $\bar{n}\ll N_b$. Here the central spin system is just used for the confirmation of the measurement precision approachable via the cavity-QED. 

%It is worth noting that our initial state possesses no spin-cavity entanglement, but entanglement is dynamically induced via the strong coupling between spins and cavity.
%% 
%Such entanglement facilitates the information flow from the spin part to the cavity and enables us to perform high precision measurement on the cavity to obtain the information of the weak magnetic field initially encoded in the spin part.
%% 
%A high degree of consistency between the actual measurement precision $\delta h$ and QFI $\mathcal F_h$ shows that theoretical optimal measurement precision of the HL scaling can be achieved via the cavity-QED.
%

\section{Quantum Fisher information of the cavity-QED system}\label{Sec.IV}
%In this section, we consider that the initial state is the product state of the even and odd coherent superposition state, and the collective ground state (the case-a).

We employ the quantum Fisher information to quantify the sensitivity with respect to the estimated parameter $h$ and it can be expressed as \cite{Liu_2019,YuguoSu2021}
\begin{equation}
\mathcal{F}_{h}=\mathcal{F}_{h}^{c}+\mathcal{F}_{h}^{q}=\sum_{p_{i}\in\mathcal{S}}{\frac{\left(\partial_{h}p_{i}\right)^{2}}{p_{i}}}+\sum_{p_{i}\in\mathcal{S}}{4p_{i}\left\langle \Psi_{i}\left|\mathcal{H}_{h}^{2}\right|\Psi_{i}\right\rangle }-\sum_{p_{i},p_{j}\in\mathcal{S}}{\frac{8p_{i}p_{j}}{p_{i}+p_{j}}\left|\left\langle \Psi_{i}\left|\mathcal{H}_{h}\right|\Psi_{j}\right\rangle \right|^{2}},\label{QFI}
\end{equation}
where the spectral decomposition of the initial density matrix is $\rho\left(0\right)=\sum_{p_{i}\in\mathcal{S}}{p_{i}\left|\Psi_{i}\left\rangle \right\langle \Psi_{i}\right|}$, $p_{i}$ and $\left|\psi_{i}\right\rangle $ are its ith eigenvalue and eigenstate, $\mathcal{S}=\left\{ p_{i}\in\left\{ p_{i}\right\} |p_{i}\neq0\right\} $ is the support of the initial density matrix, $\mathcal{H}=\rmi \left(\partial_{h}U^{\dagger}\right)U=\left(1-2g^{2}a^{\dagger}a/\Delta^{2}\right)tJ_{z}$ and $U=\exp{\left(-\rmi H_{\text{eff}}t\right)}$ are the Hermitian generator and the unitary evolution operator.
From \cref{QFI},
one could find that its first term $\mathcal{F}_{h}^{c}$ is the classic contribution induced by the initial density matrix, its second and third terms $\mathcal{F}_{h}^{q}$ are the quantum contribution. 
Since the non-negativity of the first term, implying $\mathcal{F}_{h}\geq\mathcal{F}_{h}^{q}$, we only consider the quantum contribution $\mathcal{F}_{h}=\mathcal{F}_{h}^{q}$, henceforth.



\subsection{Quantum Fisher information of the Ising model}\label{Sec.IV.1}
In this section, for the Ising model, we study the QFI of our system sensing a weak magnetic field.
In our scheme, our initial density matrix is
\begin{equation}
\rho\left(0\right)=\sum_{p_i\in\mathcal{S}}{p_i\left|\Psi_i\right\rangle\left\langle\Psi_i\right|}=\left|\alpha\right\rangle\left\langle\alpha\right|\otimes\sum_{p_i\in\mathcal{S}}{p_i\left|\psi_i\right\rangle\left\langle\psi_i\right|},
\end{equation}
where $\rho_{\text{opt}}\left(0\right)=\left|\alpha\right\rangle\left\langle\alpha\right|$ and $\rho_{\text{spin}}\left(0\right)=e^{-\beta H_0(h)}/Z(\beta,h)=\sum_{p_i\in\mathcal{S}}{p_i\left|\psi_i\right\rangle\left\langle\psi_i\right|}$ are given by the coherent state of the cavity field and the thermal state of the $N_c$ spins, $p_i$ and $\left|\psi_i\right\rangle$ are the  $i$-th eigenvalue and eigenstate of $H_0(h)$, respectively.

Here, we consider the zero temperture (large $\beta$) case.
After a tedious calculation (see more details in appendix~C), we obtain the QFI
\begin{equation}
\mathcal{F}_{h} 
 =  4t^{2}\left[\left(1-\frac{2g^{2}}{\Delta^{2}}\bar{n}\right)^{2}\left(\left\langle \psi_{0}\right|J_{z}^{2}\left|\psi_{0}\right\rangle -\left\langle \psi_{0}\right|J_{z}\left|\psi_{0}\right\rangle ^{2}\right)+\frac{4g^{4}}{\Delta^{4}}\bar{n}\left\langle \psi_{0}\right|J_{z}^{2}\left|\psi_{0}\right\rangle \right],\label{QFI_h}
\end{equation}
where $\bar{n}=\left|\alpha\right|^{2}$, $\left\langle \alpha\right|\left(1-2g^{2}a^{\dagger}a/\Delta^{2}\right)^{2}\left|\alpha\right\rangle =\left(1-2g^{2}\bar{n}/\Delta^{2}\right)^{2}+4g^{4}\bar{n}/\Delta^{4}$,
\begin{align}
%\left\langle \alpha\right|\left(1-\frac{2g^{2}}{\Delta^{2}}a^{\dagger}a\right)^{2}\left|\alpha\right\rangle &=\left(1-\frac{2g^{2}}{\Delta^{2}}\bar{n}\right)^{2}+4g^{4}\bar{n}/\Delta^{4}\nonumber\\
 \left\langle \psi_{0}\left|J_{z}^{2}\right|\psi_{0}\right\rangle  
 & =  \frac{N_{c}}{2}-\sum_{k>0}\cos^{2}\left(2\theta_{k}\right)+\sum_{k{}^{\prime},k^{\prime\prime}>0}\cos\left(2\theta_{k^{\prime}}\right)\cos\left(2\theta_{k^{\prime\prime}}\right),\nonumber\\
  \left\langle \psi_{0}\left|J_{z}\right|\psi_{0}\right\rangle ^{2} 
  &=  \left|\bigotimes_{k>0}\sideset{_{k}}{_{k}}{\mathop{\left\langle \phi_{0}\left|\sum_{k{}^{\prime}>0}\left(Q_{k^{\prime}}\bigotimes_{k_{1}\neq k^{\prime}}\mathbf{1}_{k_{1}}\right)\right|\phi_{0}\right\rangle}}\right|^{2}
 = \sum_{k{}^{\prime},k^{\prime\prime}>0}\cos\left(2\theta_{k^{\prime}}\right)\cos\left(2\theta_{k^{\prime\prime}}\right),\nonumber
 \end{align}
 and
\begin{equation}
 \text{Var}\left(J_{z}\right)=\left\langle \psi_{0}\left|J_{z}^{2}\right|\psi_{0}\right\rangle -\left\langle \psi_{0}\left|J_{z}\right|\psi_{0}\right\rangle ^{2}=\sum_{k>0}\sin^{2}\left(2\theta_{k}\right).\nonumber
\end{equation}
The collective spin operators could be rewritten as $J_{z}=-\sum_{k}\left(c_{k}^{\dagger}c_{k}-1/2\right)=-\sum_{k>0}\left(c_{k}^{\dagger}c_{k}+c_{-k}^{\dagger}c_{-k}-1\right)=\sum_{k>0}Q_{k}$, where  $\left|\psi_{0}\right\rangle =\bigotimes_{k>0}\left|\phi_{0}\right\rangle _{_{k}}$,  $\ _{k}\left\langle \phi_{0}\left|Q_{k}^{2}\right|\phi_{0}\right\rangle _{k}=1$, and $\ _{k}\left\langle \phi_{0}\left|Q_{k}\right|\phi_{0}\right\rangle _{k}=\cos\left(2\theta_{k}\right)$.
  
 % Figure environment removed
  
 We focus on the weak magnetic field case $h\ll2\pi\lambda/N_{c}$.
 When $\gamma=1$ (quantum Ising model), we know that $\Lambda_{k}=\sqrt{\left(\lambda\cos k-h\right)^{2}+\lambda^{2}\gamma^{2}\sin^{2}k}\approx\lambda$,
 $\sin\nu_{k}=\sin\left(2\theta_{k}\right)=\lambda\gamma\sin k/\Lambda_{k}\approx\sin k$
 and $\cos\nu_{k}=\cos\left(2\theta_{k}\right)=\left(\lambda\cos k-h\right)/\Lambda_{k}\approx\cos k$.
 From \cref{QFI_h}, one could gain the QFI
 \begin{align}
 \mathcal{F}_{h}
 & \approx  4t^{2}\left[\left(1-\frac{2g^{2}}{\Delta^{2}}\bar{n}\right)^{2}\left(\frac{N_{c}}{2}-\sum_{k>0}\cos^{2}k\right)+\frac{4g^{4}}{\Delta^{4}}\bar{n}\left(\frac{N_{c}}{2}-\sum_{k>0}\cos^{2}k+\sum_{k{}^{\prime},k^{\prime\prime}>0}\cos k^{\prime}\cos k^{\prime\prime}\right)\right]\nonumber\\
 & \approx 4t^{2}\left[\left(1-\frac{2g^{2}}{\Delta^{2}}\bar{n}\right)^{2}\frac{N_{c}}{4}+\frac{4g^{4}}{\Delta^{4}}\bar{n}\left(\frac{N_{c}}{4}+1\right)\right],\label{QFI_Ising}
 \end{align}
 where $\sum_{k>0}\cos^{2}k\approx N_{c}/4$, and $\sum_{k{}^{\prime},k^{\prime\prime}>0}\cos k^{\prime}\cos k^{\prime\prime}\approx1$.
 Since the average photon number is large ($\bar{n}\gg N$ and $2g^{2}\bar{n}/\Delta^{2}\gg1$), we gain the QFI
 \begin{equation}
 \mathcal{F}_{h}  \approx \frac{4g^{4}}{\Delta^{4}}t^{2}N_{c}\bar{n}^{2}.
 \end{equation}
 
%\begin{table}[htbp]
%	\scriptsize
%	\centering
%	\caption{QFI with respect to the anisotropy parameter $\gamma$.}
%	\begin{tabular}{cp{2cm}<{\centering}p{3cm}<{\centering}}
%		\toprule[1pt]
%		\specialrule{0em}{2pt}{2pt}
%		Anisotropy parameter $\gamma$ &Parity of $\frac{N_c}{2}$& QFI $\mathcal{F}_{h}$  \\
%		\specialrule{0em}{1pt}{2pt}	
%		\midrule[0.5pt]
%		\specialrule{0em}{2pt}{2pt}
%		\multirow{2}{*}{$\gamma=0$} & Even&$\approx	\frac{64g^{4}}{\Delta^{4}}t^{2}\bar{n}$\\
%		\specialrule{0em}{2pt}{2pt}
%%		 \cmidrule{2-3}
%		% 		\cline{2}
%		&Odd&$\approx\frac{16g^{4}}{\Delta^{4}}t^{2}\bar{n}$\\
%		\specialrule{0em}{2pt}{2pt}
%		\cmidrule{2-3}
%		\multirow{2}{*}{$0\ll \gamma \ll\frac{h}{\lambda}$} & Even &$\approx\frac{16g^{4}}{\Delta^{4}}t^{2}\bar{n}\left(\bar{n}\frac{\lambda^{2}\gamma^{2}}{h^{2}+\lambda^{2}\gamma^{2}}+4\right) $\\
%		\specialrule{0em}{2pt}{2pt}
%		& Odd &$\approx\frac{16g^{4}}{\Delta^{4}}t^{2}\bar{n} $\\
%		\specialrule{0em}{2pt}{2pt}
%		\cmidrule{2-3}
%		\multirow{2}{*}{$\frac{h}{\lambda}\ll \gamma \ll\frac{2\pi}{N_c}$} & Even &$\approx \frac{16g^{4}}{\Delta^{4}}t^{2}\bar{n}^{2}$\\
%		\specialrule{0em}{2pt}{2pt}
%		& Odd &$\approx	\frac{16g^{4}}{\Delta^{4}}t^{2}\bar{n}$\\
%		\specialrule{0em}{2pt}{2pt}
%		\cmidrule{2-3}
%		$\gamma\approx1$ & Even/Odd&$ \approx \frac{4g^{4}}{\Delta^{4}}t^{2}N_{c}\bar{n}^{2}$\\
%		\specialrule{0em}{2pt}{2pt}
%		\cmidrule{2-3}
%		$\gamma\gg\frac{N_c}{2\pi}$ & Even/Odd&$\approx  \frac{8g^{4}}{\Delta^{4}}t^{2}N_{c}\bar{n}^{2}$\\
%		\specialrule{0em}{2pt}{2pt}
%		\bottomrule[1pt]
%	\end{tabular}\label{Table}
%\end{table}



%\begin{table}[htbp]
%	\scriptsize
%	\centering
%	\caption{QFI with respect to the anisotropy parameter $\gamma$.}
%	\begin{tabular}{cp{2cm}<{\centering}p{3cm}<{\centering}}
%		\toprule[1pt]
%		\specialrule{0em}{2pt}{2pt}
%		Anisotropy parameter $\gamma$ &Parity of $\frac{N_c}{2}$& QFI $\mathcal{F}_{h}$  \\
%		\specialrule{0em}{1pt}{2pt}	
%		\midrule[0.5pt]
%		\specialrule{0em}{2pt}{2pt}
%		\multirow{2}{*}{$\gamma=0$} 		&Odd&$\approx\mathcal{F}_{\text{Odd}}^{\text{XX}}\coloneqq\frac{16g^{4}}{\Delta^{4}}t^{2}\bar{n}$\\
%		\specialrule{0em}{2pt}{2pt}
%		% 		\cmidrule{2}
%		% 		\cline{2}
%		& Even&$\approx	4\mathcal{F}_{\text{Odd}}^{\text{XX}}$\\
%		\specialrule{0em}{2pt}{2pt}
%		\multirow{2}{*}{$0\ll \gamma \ll\frac{h}{\lambda}$} & Odd &$\approx\mathcal{F}_{\text{Odd}}^{\text{XX}} $\\
%		\specialrule{0em}{2pt}{2pt}
%		& Even &$\approx\mathcal{F}_{\text{Odd}}^{\text{XX}}\left(\bar{n}\frac{\lambda^{2}\gamma^{2}}{h^{2}+\lambda^{2}\gamma^{2}}+4\right) $\\
%		\specialrule{0em}{2pt}{2pt}
%		\multirow{2}{*}{$\frac{h}{\lambda}\ll \gamma \ll\frac{2\pi}{N_c}$} & Odd &$\approx	\mathcal{F}_{\text{Odd}}^{\text{XX}}$\\
%		\specialrule{0em}{2pt}{2pt}
%		& Even &$\approx \mathcal{F}_{\text{Odd}}^{\text{XX}}\bar{n}$\\
%		\specialrule{0em}{2pt}{2pt}
%		$\gamma\approx1$ & Odd/Even&$ \approx \mathcal{F}_{\text{Odd}}^{\text{XX}}N_{c}\bar{n}/4$\\
%		\specialrule{0em}{2pt}{2pt}
%		$\gamma\gg\frac{N_c}{2\pi}$ & Odd/Even&$\approx  \mathcal{F}_{\text{Odd}}^{\text{XX}}N_{c}\bar{n}/2$\\
%		\specialrule{0em}{2pt}{2pt}
%		\bottomrule[1pt]
%	\end{tabular}\label{Table}
%\end{table}
 
  \begin{table}[b]
 	\scriptsize
 	\centering
 	\caption{QFI with respect to the anisotropy parameter $\gamma$.}
 	\begin{tabular}{cp{2cm}<{\centering}p{3cm}<{\centering}}
 		%		\toprule[1pt]
 		\br
 		%		\specialrule{0em}{2pt}{2pt}
 		Anisotropy parameter $\gamma$ &Parity of $\frac{N_c}{2}$& QFI $\mathcal{F}_{h}$  \\
 		%		\specialrule{0em}{1pt}{2pt}	
 		%		\midrule[0.5pt]
 		\mr
 		%		\specialrule{0em}{2pt}{2pt}
 		\multirow{2}{*}{$\gamma=0$} & Even&$\approx	\frac{64g^{4}}{\Delta^{4}}t^{2}\bar{n}$\\
 		%		\specialrule{0em}{2pt}{2pt}
 		%		 \cmidrule{2-3}
 		% 		\cline{2}
 		&Odd&$\approx\frac{16g^{4}}{\Delta^{4}}t^{2}\bar{n}$\\
 		%		\specialrule{0em}{2pt}{2pt}
 		\cmidrule{2-3}
 		\multirow{2}{*}{$0\ll \gamma \ll\frac{h}{\lambda}$} & Even &$\approx\frac{16g^{4}}{\Delta^{4}}t^{2}\bar{n}\left(\bar{n}\frac{\lambda^{2}\gamma^{2}}{h^{2}+\lambda^{2}\gamma^{2}}+4\right) $\\
 		%		\specialrule{0em}{2pt}{2pt}
 		& Odd &$\approx\frac{16g^{4}}{\Delta^{4}}t^{2}\bar{n} $\\
 		%		\specialrule{0em}{2pt}{2pt}
 		\cmidrule{2-3}
 		\multirow{2}{*}{$\frac{h}{\lambda}\ll \gamma \ll\frac{2\pi}{N_c}$} & Even &$\approx \frac{16g^{4}}{\Delta^{4}}t^{2}\bar{n}^{2}$\\
 		%		\specialrule{0em}{2pt}{2pt}
 		& Odd &$\approx	\frac{16g^{4}}{\Delta^{4}}t^{2}\bar{n}$\\
 		%		\specialrule{0em}{2pt}{2pt}
 		\cmidrule{2-3}
 		$\gamma\approx1$ & Even/Odd&$ \approx \frac{4g^{4}}{\Delta^{4}}t^{2}N_{c}\bar{n}^{2}$\\
 		%		\specialrule{0em}{2pt}{2pt}
 		\cmidrule{2-3}
 		$\gamma\gg\frac{N_c}{2\pi}$ & Even/Odd&$\approx  \frac{8g^{4}}{\Delta^{4}}t^{2}N_{c}\bar{n}^{2}$\\
 		%		\specialrule{0em}{2pt}{2pt}
 		%		\bottomrule[1pt]
 		\br
 	\end{tabular}\label{Table}
 \end{table}
 
Figure~\ref{Fig4} represents the numerical and analytical sensitivity $1/\sqrt{\mathcal{F}_h}$ (units of $1/t$) of the cavity-QED system as a function of the average photon number $\bar{n}$. 
One could find that the analytical sensitivity (red dashed line) gained from the approximate expression \eref{QFI_Ising} is bounded by the numerical sensitivity obtained from the exact expression  \eref{QFI_h} (blue circles), which is corresponded with the QCRB. 
Furthermore, both are beyond the SQL (black dot-dash line) and approach the HL (black line) with a large average photon number.
For $\bar n=100$, our rescaled sensitivity $\delta h=1/\sqrt{\mathcal{F}_h}$ and the SQL are about $0.286$ and $1.603$ , respectively, so metrological gain $\left(\delta h\right)^2_{\text{SQL}}/\left(\delta h\right)^2$ provided by our protocol could be enhanced below the standard quantum limit by 10-20 dB ($20\lg(1.603/0.286)\simeq 15$).  
Our scheme presents a quantum-enhanced metrology by utilizing the easy-to-implement state, i.e., the ground state of the XY model $\rho_{\text{spin}}\left(0\right)=\left|\psi_0\right\rangle\left\langle\psi_0\right|$.
 



   % Figure environment removed


 \subsection{Quantum Fisher information of the anisotropic XY model}\label{Sec.IV.2}
We calculate and analyse the effect of the anisotropy parameter $\gamma$ on the QFI (see more details in appendix~D).
As seen in \cref{Table}, we show the expressions of QFIs with a general anisotropy parameter $\gamma$ and demonstrate the XY interaction is of significant importance in obtaining the HL.
Since the isotropic XX model ($\gamma=0$) is analogous to the classical Ising model with a collective phase factor, there is a SQL sensitivity, which means the noncommutativity of the Hamiltonian $\left[H_{0}\left(h\right),\ H_{\text{eff}}\right]\neq0$ plays a coral role in sensitivity enhancement.
For the non-interaction case ($\lambda=0$), there is little metrological significance that HL sensitivity exists for the spin number $N_{c}$ (not for the average photon number $\bar{n}$) because of the commutativity $\left[H_{0}\left(h\right),\ H_{\text{eff}}\right]=0$.
Therefore, the nonzero atomic fluctuation (i.e., $\text{Var}\left(J_{z}\right)$) is the sufficient and necessary condition for $\bar{n}$-HL sensitivity, which directly implies $\gamma\lambda\neq0$.

   % Figure environment removed



% \begin{table}[htbp]
% 	\scriptsize
% 	\centering
% 	\caption{QFIs and sensitivities with respect to parameters in the Hamiltonian for CSs.}
% 	\begin{tabular}{cp{7cm}<{\centering}p{3.9cm}<{\centering}}
% 		\toprule[1pt]
% 		\specialrule{0em}{2pt}{2pt}
% 		Anisotropy parameter $\gamma$& QFI $\mathcal{F}_{h}$ &Sensitivity $\left(\delta \gamma\right)^{2}$ \\
% 		\specialrule{0em}{1pt}{2pt}	
% 		\midrule[0.5pt]
% 		\specialrule{0em}{2pt}{2pt}
% 		$\gamma=0$&$\approx\begin{cases}
% 		\frac{64g^{4}}{\Delta^{4}}t^{2}\bar{n} &\text{for}\  \frac{N_{c}}{2}=2l\\
% 		\frac{16g^{4}}{\Delta^{4}}t^{2}\bar{n}  &\text{for}\   \frac{N_{c}}{2}=2l+1
% 		\end{cases}$&$\left(\delta \omega\right)^{2} \gtrsim 1/\left(4g^{4}t^{2}N\bar{n}^{2}/\Delta^{4}\right)$\\
% 		\specialrule{0em}{2pt}{2pt}
% 		$0\ll \gamma \ll\frac{h}{\lambda}$&$\approx\begin{cases}
% 		4t^{2}\left[\frac{4g^{4}}{\Delta^{4}}t^{2}\bar{n}^{2}\frac{\lambda^{2}\gamma^{2}}{h^{2}+\lambda^{2}\gamma^{2}}+\frac{16g^{4}}{\Delta^{4}}\bar{n}\right] &\text{for}\   \frac{N_{c}}{2}=2l\\
% 		\frac{16g^{4}}{\Delta^{4}}t^{2}\bar{n} &\text{for}\   \frac{N_{c}}{2}=2l+1
% 		\end{cases}$&$=\left(\delta \omega\right)^{2}$\\
% 		\specialrule{0em}{2pt}{2pt}
% 		$\frac{h}{\lambda}\ll \gamma \ll\frac{2\pi}{N_c}$&$\approx\begin{cases}
% 		\frac{16g^{4}}{\Delta^{4}}t^{2}\bar{n}^{2} &\text{for}\   \frac{N_{c}}{2}=2l\\
% 		\frac{16g^{4}}{\Delta^{4}}t^{2}\bar{n} &\text{for}\   \frac{N_{c}}{2}=2l+1
% 		\end{cases}$&$=\left(\delta \omega\right)^{2}$\\
% 		\specialrule{0em}{2pt}{2pt}
% 		$\gamma\approx1$&$ \mathcal{F}_{h}^{\text{Ising}}  \approx \frac{4g^{4}}{\Delta^{4}}t^{2}N_{c}\bar{n}^{2}$&$\approx\left(\delta \omega\right)^{2}$\\
% 		\specialrule{0em}{2pt}{2pt}
% 		$\gamma\gg\frac{N_c}{2\pi}$&$\approx  \frac{8g^{4}}{\Delta^{4}}t^{2}N_{c}\bar{n}^{2}$&$\approx g^{2}\left(\delta \omega\right)^{2}/\left(4\Delta^{2}\right)$\\
% 		\specialrule{0em}{2pt}{2pt}
% 		\bottomrule[1pt]
% 	\end{tabular}\label{Table}
% \end{table}
 
%  \begin{table}[htbp]
% 	\scriptsize
% 	\centering
% 	\caption{QFIs and sensitivities with respect to parameters in the Hamiltonian for CSs.}
% 	\begin{tabular}{cp{7cm}<{\centering}}
% 		\toprule[1pt]
% 		\specialrule{0em}{2pt}{2pt}
% 		Anisotropy parameter $\gamma$& QFI $\mathcal{F}_{h}$  \\
% 		\specialrule{0em}{1pt}{2pt}	
% 		\midrule[0.5pt]
% 		\specialrule{0em}{2pt}{2pt}
% 		$\gamma=0$&$\approx\begin{cases}
% 		\frac{64g^{4}}{\Delta^{4}}t^{2}\bar{n} &\text{for}\  \frac{N_{c}}{2}=2l\\
% 		\frac{16g^{4}}{\Delta^{4}}t^{2}\bar{n}  &\text{for}\   \frac{N_{c}}{2}=2l+1
% 		\end{cases}$\\
% 		\specialrule{0em}{2pt}{2pt}
% 		$0\ll \gamma \ll\frac{h}{\lambda}$&$\approx\begin{cases}
% 		\frac{16g^{4}}{\Delta^{4}}t^{2}\bar{n}\left(\bar{n}\frac{\lambda^{2}\gamma^{2}}{h^{2}+\lambda^{2}\gamma^{2}}+2\right) &\text{for}\   \frac{N_{c}}{2}=2l\\
% 		\frac{16g^{4}}{\Delta^{4}}t^{2}\bar{n} &\text{for}\   \frac{N_{c}}{2}=2l+1
% 		\end{cases}$\\
% 		\specialrule{0em}{2pt}{2pt}
% 		$\frac{h}{\lambda}\ll \gamma \ll\frac{2\pi}{N_c}$&$\approx\begin{cases}
% 		\frac{16g^{4}}{\Delta^{4}}t^{2}\bar{n}^{2} &\text{for}\   \frac{N_{c}}{2}=2l\\
% 		\frac{16g^{4}}{\Delta^{4}}t^{2}\bar{n} &\text{for}\   \frac{N_{c}}{2}=2l+1
% 		\end{cases}$\\
% 		\specialrule{0em}{2pt}{2pt}
% 		$\gamma\approx1$&$ \mathcal{F}_{h}^{\text{Ising}}  \approx \frac{4g^{4}}{\Delta^{4}}t^{2}N_{c}\bar{n}^{2}$\\
% 		\specialrule{0em}{2pt}{2pt}
% 		$\gamma\gg\frac{N_c}{2\pi}$&$\approx  \frac{8g^{4}}{\Delta^{4}}t^{2}N_{c}\bar{n}^{2}$\\
% 		\specialrule{0em}{2pt}{2pt}
% 		\bottomrule[1pt]
% 	\end{tabular}\label{Table}
% \end{table}

%  \begin{table}[htbp]
%	\scriptsize
%	\centering
%	\caption{QFIs and sensitivities with respect to parameters in the Hamiltonian for CSs.}
%	\begin{tabular}{cp{3.9cm}<{\centering}p{7cm}<{\centering}}
%		\toprule[1pt]
%		\specialrule{0em}{2pt}{2pt}
%		Anisotropy parameter $\gamma$ &Parity of $\frac{N_c}{2}$& QFI $\mathcal{F}_{h}$  \\
%		\specialrule{0em}{1pt}{2pt}	
%		\midrule[0.5pt]
%		\specialrule{0em}{2pt}{2pt}
%		$\gamma=0$ & Even&$\approx	\frac{64g^{4}}{\Delta^{4}}t^{2}\bar{n}$\\
%		\specialrule{0em}{2pt}{2pt}
%		 &Odd&$\approx\frac{16g^{4}}{\Delta^{4}}t^{2}\bar{n}$\\
%		\specialrule{0em}{2pt}{2pt}
%		$0\ll \gamma \ll\frac{h}{\lambda}$ & Even &$\approx\frac{16g^{4}}{\Delta^{4}}t^{2}\bar{n}\left(\bar{n}\frac{\lambda^{2}\gamma^{2}}{h^{2}+\lambda^{2}\gamma^{2}}+2\right) $\\
%		\specialrule{0em}{2pt}{2pt}
%		& Odd &$\approx\frac{16g^{4}}{\Delta^{4}}t^{2}\bar{n} $\\
%		\specialrule{0em}{2pt}{2pt}
%		$\frac{h}{\lambda}\ll \gamma \ll\frac{2\pi}{N_c}$ & Even &$\approx4\mathcal{F}_{h}^{\text{Ising}}/{N_c}$\\
%		\specialrule{0em}{2pt}{2pt}
%		 & Odd &$\approx	4\mathcal{F}_{h}^{\text{Ising}}/\left({N_c}\bar{n}\right) $\\
%		\specialrule{0em}{2pt}{2pt}
%		$\gamma\approx1$ & Even/Odd&$ \mathcal{F}_{h}^{\text{Ising}}  \approx \frac{4g^{4}}{\Delta^{4}}t^{2}N_{c}\bar{n}^{2}$\\
%		\specialrule{0em}{2pt}{2pt}
%		$\gamma\gg\frac{N_c}{2\pi}$ & Even/Odd&$\approx  2\mathcal{F}_{h}^{\text{Ising}}$\\
%		\specialrule{0em}{2pt}{2pt}
%		\bottomrule[1pt]
%	\end{tabular}\label{Table}
%\end{table}
 

 
%   \begin{table}[htbp]
% 	\scriptsize
% 	\centering
% 	\caption{QFIs and sensitivities with respect to parameters in the Hamiltonian for CSs.}
% 	\begin{tabular}{cp{2cm}<{\centering}p{5cm}<{\centering}}
% 		\toprule[1pt]
% 		\specialrule{0em}{2pt}{2pt}
% 		Anisotropy parameter $\gamma$ &Parity of $\frac{N_c}{2}$& QFI $\mathcal{F}_{h}$  \\
% 		\specialrule{0em}{1pt}{2pt}	
% 		\midrule[0.5pt]
% 		\specialrule{0em}{2pt}{2pt}
% 		\multirow{2}{*}{$\gamma=0$} & Even&$\approx	16\mathcal{F}_{h}^{\text{Ising}}/\left(N_{c}\bar{n}\right)$\\
% 		\specialrule{0em}{2pt}{2pt}
%% 		\cmidrule{2}
%% 		\cline{2}
% 		&Odd&$\approx4\mathcal{F}_{h}^{\text{Ising}}/\left(N_{c}\bar{n}\right)$\\
% 		\specialrule{0em}{2pt}{2pt}
% 		\multirow{2}{*}{$0\ll \gamma \ll\frac{h}{\lambda}$} & Even &$\approx4\mathcal{F}_{h}^{\text{Ising}}/\left(N_{c}\bar{n}\right)\left(\bar{n}\frac{\lambda^{2}\gamma^{2}}{h^{2}+\lambda^{2}\gamma^{2}}+2\right) $\\
% 		\specialrule{0em}{2pt}{2pt}
% 		& Odd &$\approx 4\mathcal{F}_{h}^{\text{Ising}}/\left(N_{c}\bar{n}\right)$\\
% 		\specialrule{0em}{2pt}{2pt}
% 		\multirow{2}{*}{$\frac{h}{\lambda}\ll \gamma \ll\frac{2\pi}{N_c}$} & Even &$\approx4\mathcal{F}_{h}^{\text{Ising}}/{N_c}$\\
% 		\specialrule{0em}{2pt}{2pt}
% 		& Odd &$\approx	4\mathcal{F}_{h}^{\text{Ising}}/\left({N_c}\bar{n}\right) $\\
% 		\specialrule{0em}{2pt}{2pt}
% 		$\gamma\approx1$ & Even/Odd&$ \mathcal{F}_{h}^{\text{Ising}}  \approx \frac{4g^{4}}{\Delta^{4}}t^{2}N_{c}\bar{n}^{2}$\\
% 		\specialrule{0em}{2pt}{2pt}
% 		$\gamma\gg\frac{N_c}{2\pi}$ & Even/Odd&$\approx  2\mathcal{F}_{h}^{\text{Ising}}$\\
% 		\specialrule{0em}{2pt}{2pt}
% 		\bottomrule[1pt]
% 	\end{tabular}\label{Table}
% \end{table}
 
% \begin{table}
% 	\caption{\label{label}A simple example produced using the standard table commands and 
% 		to assist in aligning columns on the decimal point. The width of the table and rules is
% 		set automatically by the preamble.}
% 	\begin{indented}
% 		\item[]\begin{tabular}{@{}llll}
% 			\br
% 			Head 1&Head 2&Head 3&Head 4\\
% 			\mr
% 			1.1&1.2&1.3&1.4\\
% 			2.1&2.2&2.3&2.4\\
% 			\br
% 		\end{tabular}
% 	\end{indented}
% \end{table}



Figure~\ref{Fig5} shows the QFI $\mathcal{F}_h$ (units of $t^2$) versus the anisotropy parameter $\gamma$ (units of $10^{-7}\times2\pi/{N_c}$).
There are four regions: region-I (pink) for $0\ll\gamma\ll h/\lambda$; region-II (cyan) for $h/\lambda\ll\gamma\ll2\pi/N_{c}$; region-III (green) for $\gamma\approx1$; region-IV (gray) for $\gamma\gg N_{c}/\left(2\pi\right)$.
%There are two steady stages for the even $N_{c}/2$ case in the region-II and region-IV.
%While, the odd $N_{c}/2$ case only has two steady stages: the longer stage in region-I-II and region-IV.
%(region-I: $0\ll\gamma\ll h/\lambda$; region-II: $h/\lambda\ll\gamma\ll2\pi/N_{c}$;
%region-III: $\gamma\approx1$; region-IV: $\gamma\gg N_{c}/\left(2\pi\right)$.)
Because the approximation $\sum_{k>0}\cos^{2}\nu_{k}\approx N_{c}/2$ is invalid in the region between II and III, the QFI is increasing dramatically for the odd $N_{c}/2$ case.
When the degree of anisotropy is large $\gamma\gtrsim1$, the parity of $N_{c}/2$ does not indeed impact the behavior of the QFI.

In \cref{Fig6}, we draw the schematic phase diagram of the XY model for $\gamma\geq0$ and $h\geq0$ under $\lambda=1$, a weak magnetic field $\left(h\ll2\pi\lambda/N_{c}\right)$ and a large particle number $N_{c}$. 
The XY model is critical for the critical magnetic field $h_{c}=1$ (an Ising transition, which separates a doubly degenerate phase from a non-degenerate one and thus corresponds to the spontaneous breaking of $\mathbb{Z}_{2}$) and for the isotropic line $\gamma=0$ ($h\leq1$).
The phase of the XY model could be divided into the disordered phase ($h>1$) without a net magnetization along the $x$-direction, the ordered phase ($\text{Re}\left[\sqrt{1-\gamma^{2}}\right]<h<1$) with a net magnetization along the $x$-direction, and the oscillatory phase ($\gamma^{2}+h^{2}<1$).

As shown in \cref{Fig6}, crossing the isotropic line $\gamma=0$, the QFIs dramatically increases as the scaling changes from linear-$\bar{n}$ to quadratic-$\bar{n}$ for an even $N_{c}/2$, while it is unsusceptible for an odd $N_{c}/2$.
One could find that region-I (pink) and II (cyan) are in the oscillatory phase region, and region-IV is in the ordered phase region.
However, region-III (green) crosses over the oscillatory phase region and ordered phase region.
In region-III, the behavior of the QFI is analogous to the quantum Ising model case, which means QFI $\mathcal{F}_{h}\propto N_{c}\bar{n}^{2}$.
Compared with region-III, there is only a prefactor enhancement of the QFI in region-IV (gray).
 

 
%Now we show  that the XY interaction is of significant importance in obtaining the HL.
%%
%It can be immediately see from Eq.~(\ref{F1}) that $\mathcal F_h/t^2\simeq4g^4N_c^2\bar n/\Delta^4\propto \bar n$ (the SQL) in the absence of  the XY interaction ($\lambda=0$) while 
%\begin{eqnarray}\label{F2}
%\mathcal{F}_h/t^2
%&\simeq& 4\left(1-\frac{2g^2}{\Delta^2}\bar{n}\right)^2+4\frac{g^4}{\Delta^4}N_c^2\bar{n}\nonumber \\
%&\simeq & 16g^4\bar{n}^2/\Delta^4+\mathcal O(\bar n),
%\end{eqnarray}
%since $\lambda\gg h$  and $\gamma=1$ that leads to $\prod_{k>0}^\pi\cos^2\nu_k\simeq \prod_{k>0}^\pi\cos^2k\simeq 0$. 
%%
%As shown in \cref{Sen}, 
%the analytic sensitivity (obtained from Eq.~(\ref{F2})) fits the numerical one (obtained from Eq.~(\ref{F0})) well and approaches the HL ($\mathcal F_h/t^2\propto \bar n^2$).
%%
%Physically speaking, the spin-cavity entanglement is enhanced by the existence of the XY interaction and the stronger entanglement leads to the HL.
%%
%For $\bar n=100$, the measurement precision determined by the HL and the SQL are $10^{0.05}$ and $10^1$, respectively, so metrological gains provided by our protocol could be enhanced below the standard quantum limit by 10-20 dB ($10\lg(1/0.05)\simeq 13$).  
%%
%Our scheme present a quantum-enhanced metrology by utilizing the easy-to-implement state, i.e., the thermal state $\rho_{\rm spin}=e^{-\beta H_0(h)}/Z(\beta,h)$.


\section{Conclusion}\label{Sec.V}

%It is worth noting that our initial state possesses no spin-cavity entanglement, but entanglement is dynamically induced via the strong coupling between spins and cavity.
%% 
%Such entanglement facilitates the information flow from the spin part to the cavity and enables us to perform high precision measurement on the cavity to obtain the information of the weak magnetic field initially encoded in the spin part.
%% 
%A high degree of consistency between the actual measurement precision $\delta h$ and QFI $\mathcal F_h$ shows that theoretical optimal measurement precision of the HL scaling can be achieved via the cavity-QED.

It is worth noting that our initial state is the ground state of the XY model, which could be spontaneously obtained by reducing the temperature of the system.
Moreover, the noncommutativity of the XY interaction plays a coral role in sensitivity enhancement.
Concretely, the sensitivities will exhibit different scalings of the average photon number $\bar{n}$ and the spin number $N_{c}$ in different phase regions.
We emphasize that it makes sense for the simplification that our theoretical starting point will be the effective Hamiltonian~\eqref{H_eff} without any dissipation in the strongly coupled cavity-QED systems~\cite{PhysRevLett.130.173601} and superconducting circuits~\cite{WOS:000257665300034,PhysRevLett.128.123602}, where the atomic spontaneous emission $\gamma_e$ and the cavity dissipation $\gamma_d$~\cite{WOS:000243867300038,WOS:000322086100035,PhysRevA.71.013817,PhysRevA.67.033806} could be suppressed.
%Although our theoretical starting point will be the effective Hamiltonian~\eqref{H_eff}  without any dissipation, we emphasize that this simplification makes sense in the strongly coupled cavity-QED systems~\cite{PhysRevLett.130.173601}, where the atomic spontaneous emission $\gamma_e$ and the cavity dissipation $\gamma_d$~\cite{WOS:000243867300038,WOS:000322086100035,PhysRevA.71.013817,PhysRevA.67.033806} could be suppressed.
%In the strongly coupled cavity-QED systems~\cite{PhysRevLett.130.173601}, we emphasize that the simplification that our theoretical starting point will be effective the Hamiltonian~\eqref{H_eff} without any dissipation makes sense, where the atomic spontaneous emission $\gamma_e$ and the cavity dissipation $\gamma_d$~\cite{WOS:000243867300038,WOS:000322086100035,PhysRevA.71.013817,PhysRevA.67.033806} could be suppressed.

%
%Recent experimental work has realized the multi-atom cavity with a strong coupling between deterministic atom arrays and a high finesse miniature optical cavity \cite{PhysRevLett.130.173601}. 

%%Performing a quantum-enhanced estimation without using experiment-friendly states has long been pursued for reducing the experimental complexity.
%Quantum metrology employs given quantum resources to provide a route to overcome practical limits of measurements, which inevitably distort the information.
%%
%%One of the main quests in quantum metrology is to attain the ultimate precision limit with given resources, where the resources are not only of the number of queries, but more importantly of the allowed strategies.
%%%
%%Quantum metrology employs quantum resources to enhance the measurement sensitivity beyond that can be achieved classically. While multiphoton entangled N00N states can in principle beat the shot-noise limit and reach the Heisenberg limit, high N00N states are difficult to prepare and fragile to photon loss which hinders them from reaching unconditional quantum metrological advantages.
%%%
%%Quantum metrology is the use of quantum techniques such as entanglement to yield higher statistical precision than purely classical approaches.
%%%
%%Quantum metrology provides a route to overcome practical limits in sensing devices.
%%It holds particular relevance to biology, where sensitivity and resolution constraints restrict applications both in fundamental biophysics and in medicine.
%%
%Here we propose  a promising protocol based on a cavity-quantum electrodynamics (QED) system where the trapped spins are responsible for sensing a weak field meanwhile the cavity field allows us to perform high precision measurements.
%%
%By calculating the quantum Fisher information (QFI), we show that the XY interaction among spins is indispensable for enhancing  the measurement precision to the Heisenberg limit.
%%
%%Furthermore, experimentally accessible scheme is also proposed and we show that the theoretically optimal measurement precision quantified by QFI is attainable.
%%
%We discuss the effect of the anisotropy parameter $\gamma$ and show that Heisenberg limit sensitivity could be approached for a weak magnetic field in the ordered phase region.
%%
%The metrological gain provided by our protocol could be enhanced beyond the standard quantum limit by 10-20 dB when using 100 photons as a resource. 
%%
%Superconducting circuit cavity and strong coupling optical cavity are ideal platforms to implement our protocol.

In summary, we have proposed a quantum-enhanced scheme sensoring a weak magnetic field in a cavity-QED setup to achieve the HL with experiment-friendly states.
%
We show a quantum enhancement of sensitivity in the presence  of the sensor with the XY type of interaction.
%
%Experimentally accessible measurement is given and discuss possible experimental implementations. 
%
%We have shown  that the theoretically optimal measurement precision of the Heisenberg scaling can be experimentally  attainable.
%
We have discussed the effect of the anisotropy parameter $\gamma$ and demonstrated that Heisenberg limit sensitivity could be reached for a weak magnetic field in the ordered phase region.
%
It turns out that the sensitivity can be enhanced beyond the standard quantum limit by 10-20 dB for $100$ average photons.
%
Our methods provide promising opportunities for experimental realization of weak magnetic field sensing via cavity-QED and superconducting circuit systems.

\section*{Acknowledgments}

This work is supported by the National Natural Science Foundation of China (Grant No. \ 12134015 and No. \ 11874393).
%, and No.\ 11935012).
% and the National Key Research and Development  Program of China  No.\ 2017YFA0304500.
Y.S. is partially supported by the National Natural Science Foundation of China (Grant No. \ 12247158), the ``Wuhan Talent'' (Outstanding Young Talents), and Postdoctoral Innovative Research Post in Hubei Province.
W.L. is partially supported by the National Natural Science Foundation of China (Grant No. \ 12205092) and the Hunan Provincial Natural Science Foundation of China (Grant No. \ 2023JJ40208).

%This work is supported by the National Natural Science Foundation of China grant No. \ 12134015 and No. \ 11874393.
%% and the National Key Research and Development  Program of China  No.\ 2017YFA0304500.
%Y.S. is supported by the National Natural Science Foundation of China grant No. \ 12247158, the ``Wuhan Talent'' (Outstanding Young Talents), and Postdoctoral Innovative Research Post in Hubei Province.
%W.L. is supported by the National Natural Science Foundation of China (Grant No. \ 12205092) and the Hunan Provincial Natural Science Foundation of China (Grant No. \ 2023JJ40208).

\appendix

%\setcounter{appendix}{1}
\setcounter{section}{1}
\section*{Appendix A.~Effective Hamiltonian of the cavity-QED system}\label{Appendix_A}
Here, we derive the effective Hamiltonian of the cavity-QED system.
Utilizing the unitary transformation $U=\exp\left\{-\rmi \left[\omega_0 J_z+\omega_a a^\dagger a+H_0\left(h\right)\right]t\right\}$,
the total Hamiltonian in the interaction picture could be written as
\begin{equation}
H=g\left(\rme ^{\rmi \omega_{a}t}a^{\dagger}\rme ^{\rmi \left(\omega_{0}J_{z}+H_{0}\left(h\right)\right)t}J_{-}\rme ^{-\rmi \left(\omega_{0}J_{z}+H_{0}\left(h\right)\right)t}+\text{H.c.}\right).
\end{equation}
Under the high atomic frequency condition ($\left|h-\omega_{0}\right|\gg\lambda/2$), the Hamiltonian can be written as $H=g\left[J_{-}a^{\dagger}\rme ^{-\mathrm{i}\left(\Delta+\delta\right)t}+\text{H.c.}\right]$ with a large effective detuning $\Delta=\omega_{0}-h-\omega_{a}$ and a small field-irrelative residue $\delta$ ($\Delta\gg\delta\sim\lambda$\ or\ $\lambda\gamma$)
. %Utilizing the time-averaged method of the Ref.~\cite{James2007} and the photon energy conservation $\left[\omega_a a^\dagger a,H\right]=0$, we gain the effective Hamiltonian $H_{\text{eff}}=H_0\left(h-\omega_0\right)+2g^2J_za^\dagger a/\Delta+g^2J_+J_-/\Delta$ in the Schr\"{o}dinger picture.
Employing the time-averaged method of the reference~\cite{TAM}, the interaction Hamiltonian $H=\sum_{i=1}^{N_{c}}{\left(f_{i}\rme ^{-\mathrm{i}\Delta_{i}t}+\text{H.c.}\right)}$ could be rewritten as the following compact form: 
\begin{equation}
H\approx\sum_{i,j=1}^{N_{c}}{\frac{1}{\bar{\Delta}_{ij}}\left[f_{i}^{\dagger},f_{j}\right]\rme ^{\mathrm{i}\left(\Delta_{i}-\Delta_{j}\right)t}}=-\frac{g^{2}}{\left(\Delta+\delta\right)}\left[J_{-}a^{\dagger},J_{+}a\right]\approx\frac{g^{2}}{\Delta}\left(J_{+}J_{-}+2J_{z}a^{\dagger}a\right),\label{H_TAM}
\end{equation}
where the operator $f_{i}=g\sigma_{i}^{+}a/2$ and the frequency $\Delta_{i}=\bar{\Delta}_{ij}\equiv\left|\Delta+\delta\right|\approx\left|\Delta\right|\gg1$ are corresponding to our system. 
The fact that the high-frequency contributions disappear from the average and the effective detuning is large, guarantees the validities of the first and second approximations in \cref{H_TAM}.
In the Schr\"{o}dinger picture, we gain the effective Hamiltonian (presented as \cref{H2} in the main text) 
\begin{equation}
H_{\text{eff}}=H_{0}\left(h-\omega_{0}\right)+\frac{2g^{2}}{\Delta}J_{z}a^{\dagger}a+\frac{g^{2}}{\Delta}J_{+}J_{-}
\end{equation}
by considering the initial light state $\left|\alpha\right\rangle $ and the photon energy conservation $\left[\omega_{a}a^{\dagger}a,H_{\text{eff}}\right]=0$.

And we find that the second term rotates the photon distribution in phase space at a rate $g^{2}N/\Delta$, and the third term rotates the light field at a rate $g^{2}N^{2}/\left(\Delta\bar{n}\right)$ ($\bar{n}=\left\langle a^{\dagger}a\right\rangle $ is the average photon number) or alternatively generates a mean-field rotation of the spins at a rate $g^{2}N/\Delta$. 
To make the approximation valid, the timescale of the first term should be greater than the second and third terms, which means $\Delta\gg g\sqrt{N}$ and $\Delta\gg g\sqrt{N^{2}/\bar{n}}$.
The essence of the time-averaged method is to eliminate high-frequency contributions and thus it can be viewed as a natural generalization of the rotating-wave approximation.

\setcounter{section}{2}
\setcounter{equation}{0}
\section*{Appendix B.~Reduced density matrix of the bath spin system}\label{Appendix_B}

The Hamiltonian of the general anisotropic XY model in a transverse field is given by 
\begin{equation}
H_{0}\left(h\right)=-\frac{1}{2}\sum_{i=1}^{N_{c}}{\left\{ \frac{\lambda}{2}\left[\left(1+\gamma\right)\sigma_{i}^{x}\sigma_{i+1}^{x}+\left(1-\gamma\right)\sigma_{i}^{y}\sigma_{i+1}^{y}\right]+h\sigma_{i}^{z}\right\} }.
\end{equation}
Here $\lambda$ is the nearest neighbor interaction, $\sigma_{i}^{\alpha}$ the Pauli matrix ($\alpha=x,y,z$) on site $i$, $N_{c}$ the number of sites, $\gamma$ the degree of anisotropy, and $h$ a transverse field.

By applying the Jordan-Wigner, Fourier and Bogoliubov transformations \cite{WOS:000285157400001}, the Hamiltonian can be rewriten as 
\begin{equation}
H_{0}\left(h\right)=\sum_{k}{\Lambda_{k}\left(b_{k}^{\dagger}b_{k}-\frac{1}{2}\right)}.\label{DiaH}
\end{equation}
The anticommuting fermion operators $b_{k}$ and $b_{k}^{\dagger}$ satisfy the relations $\left\{ b_{i}^{\dagger},b_{j}\right\} =\delta_{i,j}$ and $\left\{ b_{i},b_{j}\right\} =\left\{ b_{i}^{\dagger},b_{j}^{\dagger}\right\}=\left[b_{i}^{\dagger}b_{i},b_{j}^{\dagger}b_{j}\right]=0$
, the excitation energy is 
\begin{equation}
\Lambda_{k}=\sqrt{\left(h-\lambda\cos k\right)^{2}+\lambda^{2}\gamma^{2}\sin^{2}k},
\end{equation}
and the ground state is 
\begin{equation}
\left|\phi\right\rangle =\prod_{k}\left|\phi_{0}\right\rangle _{k}=\prod_{k>0}{\cos\frac{\nu_{k}}{2}\left|0\right\rangle _{k}\left|0\right\rangle _{-k}+\mathrm{i}\sin\frac{\nu_{k}}{2}\left|1\right\rangle _{k}\left|1\right\rangle _{-k}},
\end{equation}
where $k=\frac{2\pi m}{N_{c}}$ ($m=-\frac{N_{c}}{2}+1,\dots,\frac{N_{c}}{2}$), $\left|0\right\rangle _{k}$ ($\left|1\right\rangle _{k}$) is the state for $k$th particle, and the coefficients are determined by $\sin\nu_{k}=\lambda\gamma\sin k/\Lambda_{k},\cos\nu_{k}=\left(\lambda\cos k-h\right)/\Lambda_{k}.$
The partition function is 
\begin{equation}
Z\left(\beta,h\right)=\text{Tr}\left(\rme ^{-\beta H_{0}\left(h\right)}\right)=\prod_{k}{\sum_{n_{k}}{\rme ^{-\beta\Lambda_{k}\left(n_{k}-1/2\right)}}}=\prod_{k}\left(\rme ^{-\beta\Lambda_{k}}+1\right)\rme ^{\beta\Lambda_{k}/2}=\prod_{k}2\cosh\left(\beta\Lambda_{k}/2\right),
\end{equation}
$\beta$ is the inverse temperature and the QPT points are $h_{c}=1$,
$\gamma_{c}=0$ when $h\in\left[-1,1\right]$. 

The total Hamiltonian of the system (shown as \cref{CentralSpinContext} in main text when the magnetic field shifts from $h$ to $\left(h-\omega_{0}+\eta N_{b}\right)$) is 
\begin{equation}
H=H_{0}\left(h\right)\otimes I_{2N_{b}\times2N_{b}}+2\eta H_{1}\otimes S_{z},\label{Orig.Centr.Spin}
\end{equation}
where $H_{0}\left(h\right)$ is the Hamiltonian of the central, $\eta H_{1}\otimes S_{z}$ is the Hamiltonian of the spin-bath interaction, $\eta$ is a coupling constant, $H_{1}=-\sum_{j=1}^{N_{c}}{\sigma_{j}^{z}}/2$ acts as the random field for the bath spin, $S_{z}=\sum_{j=1}^{N_{b}}{\sigma_{j}^{z}/2}=\sum_{j=1}^{N_{b}}{\left(\left|\text{e}\right\rangle _{jj}\!\left\langle \text{e}\right|-\left|\text{g}\right\rangle _{jj}\!\left\langle \text{g}\right|\right)/2}$ is the total bath spin operator, and $I_{2N_{b}\times2N_{b}}$ is the identity operator. The central spin lying in the bath is equally coupled to all the $N_{b}$ bath spins. Employing the identical equations $\left|\text{e}\right\rangle _{jj}\!\left\langle \text{e}\right|=\left(1+\sigma_{j}^{z}\right)/2$ and $\left|\text{g}\right\rangle _{jj}\!\left\langle \text{g}\right|=\left(1-\sigma_{j}^{z}\right)/2$, the total Hamiltonian of the system can be rewritten as 
\begin{eqnarray}
H & = & \left(\frac{H_{0}\left(h\right)}{N_{b}}+\eta H_{1}\right)\otimes\sum_{j}^{N_{b}}{\left|\text{e}\right\rangle _{jj}\!\left\langle \text{e}\right|}+\left(\frac{H_{0}\left(h\right)}{N_{b}}-\eta H_{1}\right)\otimes\sum_{j}^{N_{b}}{\left|\text{g}\right\rangle _{jj}\!\left\langle \text{g}\right|}\nonumber \\
& = & H_{+}\otimes\left(\frac{N_{b}}{2}+S_{z}\right)+H_{-}\otimes\left(\frac{N_{b}}{2}-S_{z}\right),\label{CentralSpin}
\end{eqnarray}
where $H_{\pm}=H_{0}\left(h\right)/N_{b}\pm\eta H_{1}=H_{0}\left(h\pm\eta N_{b}\right)/N_{b}$ denote the effective Hamiltonians of two slightly different evolution
branches. 

We consider the initial states of the central spins and bath spins are a thermal state $\left|\psi\right\rangle _{\text{thermal}}$ and a spin coherent state $\left|\vartheta,\varphi\right\rangle $, respectively.
Utilizing the unitary transformation, we can gain the time evolution density matrix of the system 
\begin{equation}
\rho\left(t\right)=U\left(\left|\vartheta,\varphi\right\rangle \left\langle \vartheta,\varphi\right|\otimes\rho_{c}\left(0\right)\right)U^{\dagger},\label{rho_t}
\end{equation}
where the initial state is $\rho\left(0\right)=\left|\vartheta,\varphi\right\rangle \left\langle \vartheta,\varphi\right|\otimes\rho_{c}\left(0\right)$ and the unitary matrix is $U\equiv\exp{\left(-\rmi Ht\right)}$.
By representing the spin coherent state in terms of the Dicke states, i.e., $\left|\vartheta,\varphi\right\rangle =\sum_{n=-N_{b}/2}^{N_{b}/2}{c_{n}\left|n\right\rangle }$ with $S_{z}\left|n\right\rangle =n\left|n\right\rangle $ and $c_{n}=\left|\cos\left(\vartheta/2\right)\right|^{N_{b}}\tan^{N_{b}+n}\left(\vartheta/2\right)$ $\times\rme ^{-\rmi \left(N_{b}/2+n\right)\varphi}\sqrt{N_{b}!/\left[\left(N_{b}/2+n\right)!\left(N_{b}/2-n\right)!\right]}$, the reduced density matrix of the bath spins (presented as \cref{rho_b} in the main text) is 
\begin{eqnarray}
\rho_{b}\left(t\right) & = & \mathrm{Tr}_{c}\left[U\sum_{i,j}{c_{i}c_{j}^{*}\left|i\right\rangle \left\langle j\right|\rho_{c}\left(0\right)}U^{\dagger}\right]\nonumber \\
& = & \sum_{i,j}{c_{i}c_{j}^{*}\mathrm{Tr}_{c}\left[\rme ^{-\rmi t\left[H_{+}\otimes\left(\frac{N_{b}}{2}+S_{z}\right)+H_{-}\otimes\left(\frac{N_{b}}{2}-S_{z}\right)\right]}\left|i\right\rangle \left\langle j\right|\rho_{c}\left(0\right)\rme ^{\rmi t\left[H_{+}\otimes\left(\frac{N_{b}}{2}+S_{z}\right)+H_{-}\otimes\left(\frac{N_{b}}{2}-S_{z}\right)\right]}\right]}\nonumber \\
& = & \sum_{i,j}{c_{i}c_{j}^{*}\mathrm{Tr}_{c}\left[\rme ^{-\rmi t\left[H_{+}\left(\frac{N_{b}}{2}+i\right)+H_{-}\left(\frac{N_{b}}{2}-i\right)\right]}\left|i\right\rangle \left\langle j\right|\rho_{c}\left(0\right)\rme ^{\rmi t\left[H_{+}\left(\frac{N_{b}}{2}+j\right)+H_{-}\left(\frac{N_{b}}{2}-j\right)\right]}\right]}\nonumber \\
& = & \sum_{i,j}{c_{i}c_{j}^{*}\left|i\right\rangle \left\langle j\right|\mathrm{Tr}_{c}\left[\rme ^{-\rmi t\left[i\left(H_{+}-H_{-}\right)+\frac{N_{b}}{2}\left(H_{+}+H_{-}\right)\right]}\rho_{c}\left(0\right)\rme ^{\rmi t\left[j\left(H_{+}-H_{-}\right)+\frac{N_{b}}{2}\left(H_{+}+H_{-}\right)\right]}\right]}\nonumber \\
& = & \sum_{i,j}{c_{i}c_{j}^{*}\left|i\right\rangle \left\langle j\right|\mathrm{Tr}_{c}\left[\rme ^{-\rmi t\left(2\eta iH_{1}+H_{0}\left(h\right)\right)}\rho_{c}\left(0\right)\rme ^{\rmi t\left(2\eta jH_{1}+H_{0}\left(h\right)\right)}\right]}\nonumber \\
& = & \sum_{i,j}{c_{i}c_{j}^{*}\left|i\right\rangle \left\langle j\right|\mathrm{Tr}_{c}\left[\rme ^{\rmi t\left(H_{0}\left(h\right)+2\eta jH_{1}\right)}\rme ^{-\rmi t\left(H_{0}\left(h\right)+2\eta iH_{1}\right)}\rho_{c}\left(0\right)\right]}\nonumber \\
& = & \sum_{i,j}{c_{i}c_{j}^{*}\left|i\right\rangle \left\langle j\right|\mathrm{Tr}_{c}\left[\rme ^{\rmi tH_{0}\left(h+2\eta j\right)}\rme ^{-\rmi tH_{0}\left(h+2\eta i\right)}\frac{\rme ^{-\beta H_{0}\left(h\right)}}{Z\left(\beta,h\right)}\right]}\nonumber \\
& = & \sum_{i,j}{c_{i}c_{j}^{*}\left|i\right\rangle \left\langle j\right|\tilde{\rho}_{ij}\left(t\right)},
\end{eqnarray}
where the simplified reduced density matrix $\tilde{\rho}_{ij}\left(t^{\prime}\right)=\mathrm{Tr}_{c}\left[\rme ^{\rmi t^{\prime}H_{0}\left(h+2\eta j\right)}\rme ^{-\rmi t^{\prime}H_{0}\left(h+2\eta i\right)}\rme ^{-\beta H_{0}\left(h\right)}/Z\left(\beta,h\right)\right]$.
Here, we obtain the simplified reduced density matrix in the eigenspace $\left\{ \prod_{k>0}{\left|\phi_{l}\right\rangle _{k}};l=0,\pm,2\right\} $ of the original central spin chain $H_{0}\left(h\right)$, where the eigenstates are $\left|\phi_{0}\right\rangle _{k}$, $\left|\phi_{\pm}\right\rangle _{k}=b_{\pm k}^{\dagger}\left|\phi_{0}\right\rangle _{k}$ and $\left|\phi_{2}\right\rangle _{k}=b_{k}^{\dagger}b_{-k}^{\dagger}\left|\phi_{0}\right\rangle _{k}$ and satisfy $b_{\pm k}^{\dagger}b_{\pm k}\left|\phi_{0}\right\rangle _{k}=b_{\pm k}^{\dagger}b_{\pm k}\left|\phi_{\mp}\right\rangle _{k}=0$, $b_{\pm k}^{\dagger}b_{\pm k}\left|\phi_{\pm}\right\rangle _{k}=\left|\phi_{\pm}\right\rangle _{k}$ and $b_{\pm k}^{\dagger}b_{\pm k}\left|\phi_{2}\right\rangle _{k}=\left|\phi_{2}\right\rangle _{k}$.
It follows that the simplified reduced density matrix is given by
\begin{align}
 \!\tilde{\rho}_{ij}\!\left(t\right)\!  &=  \mathrm{Tr}_{c}\left[\rme ^{\rmi tH_{0}\left(h+2\eta j\right)}\rme ^{-\rmi tH_{0}\left(h+2\eta i\right)}\frac{\rme ^{-\beta H_{0}\left(h\right)}}{Z\left(\beta,h\right)}\right]\nonumber \\
&=  \frac{1}{\! Z\!\left(\beta, h\right)\!}\prod_{k^{\prime}>0}{\sum_{l=0,\pm,2}{_{k^{\prime}}\left\langle \phi_{l}\right|\rme ^{\rmi tH_{0}\left(h+2\eta j\right)}\rme ^{-\rmi tH_{0}\left(h+2\eta i\right)}\rme ^{-\beta H_{0}\left(h\right)}\left|\phi_{l}\right\rangle _{k^{\prime}}}}\nonumber \\
& =  \frac{1}{\! Z\!\left(\beta, h\right)\!}\prod_{k^{\prime}>0}{\sum_{l=0,\pm,2}{_{k^{\prime}}\left\langle \phi_{l}\right|\rme ^{\rmi t\sum_{k}{E_{j,k}\left(d_{j,k}^{\dagger}d_{j,k}-\frac{1}{2}\right)}}\rme ^{-\rmi t\sum_{k}{E_{i,k}\left(d_{i,k}^{\dagger}d_{i,k}-\frac{1}{2}\right)}}\rme ^{-\beta\sum_{k}{\Lambda_{k}\left(b_{k}^{\dagger}b_{k}-\frac{1}{2}\right)}}\left|\phi_{l}\right\rangle _{k^{\prime}}}}\nonumber \\
&=  \frac{1}{\! Z\!\left(\beta, h\right)\!}\prod_{\! k>0\!}{\sum_{l=0,\pm,2\!}{\!_{k}\!\left\langle \phi_{l}\right|\rme ^{\rmi tE_{j,k}\left(d_{j,k}^{\dagger}d_{j,k}+d_{j,-k}^{\dagger}d_{j,-k}-\!1\!\right)}\rme ^{\!-\rmi tE_{i,k}\left(d_{i,k}^{\dagger}d_{j,k}+d_{i,-k}^{\dagger}d_{j,-k}-\!1\!\right)}\rme ^{\!-\!\beta\Lambda_{k}\left(b_{k}^{\dagger}b_{k}+b_{-k}^{\dagger}b_{-k}-\!1\right)}\!\left|\phi_{l}\right\rangle _{k}}},
\end{align}
where the effective Hamiltonians $H_{m}=H_{0}\left(h+2\eta m\right)$ ($m=i,j$) can be diagonalized in a similar way as \cref{DiaH}), the angles $\mu_{m,k}=\arctan\left[\lambda\gamma\sin k/\left(h+2\eta m-\lambda\cos k\right)\right]$ and $\theta_{m,k}=\left(\mu_{m,k}-\nu_{k}\right)/2$ are determined by the diagonalization conditions, $d_{m,\pm k}=\cos\left(\theta_{m,k}\right)b_{m,\pm k}\mp\rmi \sin\left(\theta_{m,k}\right)\left(b_{m,\mp k}\right)^{\dagger}$ and $E_{m,k}=E_{m,-k}=\sqrt{\left(h+2\eta m-\lambda\cos k\right)^{2}+\lambda^{2}\gamma^{2}\sin^{2}k}$ are the anticommuting fermion operators and the excitation energy, respectively.
By some calculations, we could gain the useful identity
relations 
\begin{align}
d_{m,\pm k}^{\dagger}d_{m,\pm k}\left|\phi_{\pm}\right\rangle _{k} & =  \left|\phi_{\pm}\right\rangle _{k},\\
d_{m,\pm k}^{\dagger}d_{m,\pm k}\left|\phi_{\mp}\right\rangle _{k} & =  0,\\
\left(d_{m,\pm k}^{\dagger}d_{m,\pm k}\right)^{n}\left|\phi_{0}\right\rangle _{k} & =  d_{m,\pm k}^{\dagger}d_{m,\pm k}\left|\phi_{0}\right\rangle _{k}=-\rmi \sin\theta_{m,k}\left(\cos\theta_{m,k}\left|\phi_{2}\right\rangle _{k}+\rmi \sin\theta_{m,k}\left|\phi_{0}\right\rangle _{k}\right),\\
\left(d_{m,\pm k}^{\dagger}d_{m,\pm k}\right)^{n}\left|\phi_{2}\right\rangle _{k} & = d_{m,\pm k}^{\dagger}d_{m,\pm k}\left|\phi_{2}\right\rangle _{k}=\cos\theta_{m,k}\left(\cos\theta_{m,k}\left|\phi_{2}\right\rangle _{k}+\rmi \sin\theta_{m,k}\left|\phi_{0}\right\rangle _{k}\right),
\end{align}
and the expressions 
%\numparts
\begin{align}
 \rme ^{-\rmi tE_{m,k}\left(d_{m,k}^{\dagger}d_{m,k}-\frac{1}{2}\right)}\left|\phi_{\pm}\right\rangle _{k} & =  \rme ^{\mp\rmi \frac{tE_{m,k}}{2}}\left|\phi_{\pm}\right\rangle _{k},\label{phi_pm1}\\
\rme ^{-\rmi tE_{m,k}\left(d_{m,-k}^{\dagger}d_{m,-k}-\frac{1}{2}\right)}\left|\phi_{\pm}\right\rangle _{k} & = \rme ^{\pm\rmi \frac{tE_{m,k}}{2}}\left|\phi_{\pm}\right\rangle _{k},\label{phi_pm2}\\
\rme ^{\pm\rmi tE_{m,k}\left(d_{m,k}^{\dagger}d_{m,k}-\frac{1}{2}\right)}\left|\phi_{0}\right\rangle _{k} & =  \left(\rme ^{\mp\rmi \frac{tE_{m,k}}{2}}\pm2\rmi \sin^{2}\theta_{m,k}\sin\frac{E_{m,k}t}{2}\right)\left|\phi_{0}\right\rangle _{k}\pm\sin2\theta_{m,k}\sin\frac{E_{m,k}t}{2}\left|\phi_{2}\right\rangle _{k},\label{phi_0}\\
\rme ^{\pm\rmi tE_{m,k}\left(d_{m,k}^{\dagger}d_{m,k}-\frac{1}{2}\right)}\left|\phi_{2}\right\rangle _{k} & =  \mp\sin2\theta_{m,k}\sin\frac{E_{m,k}t}{2}\left|\phi_{0}\right\rangle _{k}+\left(\rme ^{\mp\rmi \frac{tE_{m,k}}{2}}\pm2\rmi \cos^{2}\theta_{m,k}\sin\frac{E_{m,k}t}{2}\right)\left|\phi_{2}\right\rangle _{k}.\label{phi_2}
\end{align}
%\endnumparts
From \cref{phi_pm1,phi_pm2,phi_0,phi_2}, we obtain the inner product results 
\begin{align}
& \   \quad_{k}\left\langle \phi_{\pm}\right|\rme ^{\rmi tE_{j,k}\left(d_{j,k}^{\dagger}d_{k}+d_{j,-k}^{\dagger}d_{-k}-1\right)}\rme ^{-\rmi tE_{i,k}\left(d_{i,k}^{\dagger}d_{k}+d_{i,-k}^{\dagger}d_{-k}-1\right)}\rme ^{-\beta\Lambda_{k}\left(b_{k}^{\dagger}b_{k}+b_{-k}^{\dagger}b_{-k}-1\right)}\left|\phi_{\pm}\right\rangle _{k}\nonumber \\
& =  \,_{k}\left\langle \phi_{\pm}\right|\rme ^{\rmi tE_{j,k}\left(1-1\right)}\rme ^{-\rmi tE_{i,k}\left(1-1\right)}\rme ^{-\beta\Lambda_{k}\left(1-1\right)}\left|\phi_{\pm}\right\rangle _{k}\nonumber \\
& =  1,\label{InnerProdpm}\\
& \ \quad  _{k}\left\langle \phi_{0}\right|\rme ^{\rmi tE_{j,k}\left(d_{j,k}^{\dagger}d_{k}+d_{j,-k}^{\dagger}d_{-k}-1\right)}\rme ^{-\rmi tE_{i,k}\left(d_{i,k}^{\dagger}d_{k}+d_{i,-k}^{\dagger}d_{-k}-1\right)}\rme ^{-\beta\Lambda_{k}\left(b_{k}^{\dagger}b_{k}+b_{-k}^{\dagger}b_{-k}-1\right)}\left|\phi_{0}\right\rangle _{k}\nonumber \\
& =  \rme ^{\beta\Lambda_{k}}\,_{k}\left\langle \phi_{0}\right|\rme ^{\rmi tE_{j,k}\left(d_{j,k}^{\dagger}d_{k}+d_{j,-k}^{\dagger}d_{-k}-1\right)}\rme ^{-\rmi tE_{i,k}\left(d_{i,k}^{\dagger}d_{k}+d_{i,-k}^{\dagger}d_{-k}-1\right)}\left|\phi_{0}\right\rangle _{k}\nonumber \\
& =  \rme ^{\beta\Lambda_{k}}\,_{k}\left\langle \phi_{0}\right|\rme ^{2\rmi tE_{j,k}\left(d_{j,k}^{\dagger}d_{k}-\frac{1}{2}\right)}\rme ^{-2\rmi tE_{i,k}\left(d_{i,k}^{\dagger}d_{k}-\frac{1}{2}\right)}\left|\phi_{0}\right\rangle _{k}\nonumber \\
& =  \rme ^{\beta\Lambda_{k}}\left\{ _{k}\left\langle \phi_{0}\right|\left[\rme ^{-\rmi tE_{j,k}}+2\rmi \sin^{2}\theta_{j,k}\sin\left(tE_{j,k}\right)\right]-_{k}\left\langle \phi_{2}\right|\left[\sin2\theta_{j,k}\sin\left(tE_{j,k}\right)\right]\right\} \nonumber \\
& \quad  \times\left\{ \left[\rme ^{\rmi tE_{i,k}}-2\rmi \sin^{2}\theta_{i,k}\sin\left(tE_{i,k}\right)\right]\left|\phi_{0}\right\rangle _{k}-\sin2\theta_{i,k}\sin\left(tE_{i,k}\right)\left|\phi_{2}\right\rangle _{k}\right\} \nonumber \\
& =  \rme ^{\!\beta\Lambda_{k}}\!\left\{ \left[\rme ^{-\rmi tE_{j,k}\!}\!+\!2\rmi \sin^{2}\!\theta_{j,k}\sin\left(tE_{j,k}\!\right)\right]\left[\rme ^{\rmi tE_{i,k}\!}\!-\!2\rmi \sin^{2}\!\theta_{i,k}\sin\left(tE_{i,k}\!\right)\right]\!+\!\sin2\theta_{i,k}\sin2\theta_{j,k}\sin\left(tE_{i,k}\!\right)\sin\left(tE_{j,k}\!\right)\right\} \!,\label{InnerProd0}\\
& \ \quad  _{k}\left\langle \phi_{2}\right|\rme ^{\rmi tE_{j,k}\left(d_{j,k}^{\dagger}d_{k}+d_{j,-k}^{\dagger}d_{-k}-1\right)}\rme ^{-\rmi tE_{i,k}\left(d_{i,k}^{\dagger}d_{k}+d_{i,-k}^{\dagger}d_{-k}-1\right)}\rme ^{-\beta\Lambda_{k}\left(b_{k}^{\dagger}b_{k}+b_{-k}^{\dagger}b_{-k}-1\right)}\left|\phi_{2}\right\rangle _{k}\nonumber \\
& =  \rme ^{-\beta\Lambda_{k}}\,_{k}\left\langle \phi_{2}\right|\rme ^{\rmi tE_{j,k}\left(d_{j,k}^{\dagger}d_{k}+d_{j,-k}^{\dagger}d_{-k}-1\right)}\rme ^{-\rmi tE_{i,k}\left(d_{i,k}^{\dagger}d_{k}+d_{i,-k}^{\dagger}d_{-k}-1\right)}\left|\phi_{2}\right\rangle _{k}\nonumber \\
& =  \rme ^{-\beta\Lambda_{k}}\,_{k}\left\langle \phi_{2}\right|\rme ^{2\rmi tE_{j,k}\left(d_{j,k}^{\dagger}d_{k}-\frac{1}{2}\right)}\rme ^{-2\rmi tE_{i,k}\left(d_{i,k}^{\dagger}d_{k}-\frac{1}{2}\right)}\left|\phi_{2}\right\rangle _{k}\nonumber \\
& =  \rme ^{-\beta\Lambda_{k}}\left\{ _{k}\left\langle \phi_{0}\right|\left[\sin2\theta_{j,k}\sin\left(tE_{j,k}\right)\right]+_{k}\left\langle \phi_{2}\right|\left[\rme ^{-\rmi tE_{j,k}}+2\rmi \cos^{2}\theta_{j,k}\sin\left(tE_{j,k}\right)\right]\right\} \nonumber \\
& \quad  \times\left\{ \sin2\theta_{i,k}\sin\left(tE_{i,k}\right)\left|\phi_{0}\right\rangle _{k}+\left[\rme ^{\rmi tE_{i,k}}-2\rmi \cos^{2}\theta_{i,k}\sin\left(tE_{i,k}\right)\right]\left|\phi_{2}\right\rangle _{k}\right\} \nonumber \\
& =  \rme ^{\!-\!\beta\Lambda_{k}\!}\!\left\{ \left[\rme ^{-\rmi tE_{j,k}\!}\!+\!2\rmi \cos^{2}\!\theta_{j,k}\sin\left(tE_{j,k}\!\right)\!\right]\!\left[\rme ^{\rmi tE_{i,k}\!}\!-\!2\rmi \cos^{2}\!\theta_{i,k}\sin\left(tE_{i,k}\!\right)\!\right]\!+\!\sin2\theta_{i,k}\sin2\theta_{j,k}\sin\left(tE_{i,k}\!\right)\sin\left(tE_{j,k}\!\right)\right\} \!.\label{InnerProd2}
\end{align}


From \cref{InnerProd0,InnerProdpm,InnerProd2}, we could gain the reduced density matrix of the bath spins (presented as \cref{Final StateContext} in the main text) 
\begin{align}
\rho_{b}\left(t\right) & =  \sum_{i,j}{c_{i}c_{j}^{*}\left|i\right\rangle \left\langle j\right|\tilde{\rho}_{ij}\left(t\right)}\nonumber \\
& =  \sum_{i,j}\bigg\{\frac{c_{i}c_{j}^{*}}{Z\left(\beta,h\right)}\left|i\right\rangle \left\langle j\right|\prod_{k>0}\left[2+2\cosh\left(\beta\Lambda_{k}\right)C\left(\theta_{i,k},E_{i,k}t\right)C\left(\theta_{j,k},E_{j,k}t\right)\right.\nonumber \\
& \quad  \left.+\rme ^{\beta\Lambda_{k}}A\left(\theta_{j,k},E_{j,k}t\right)A^{*}\left(\theta_{i,k},E_{i,k}t\right)+\rme ^{-\beta\Lambda_{k}}B\left(\theta_{j,k},E_{j,k}t\right)B^{*}\left(\theta_{i,k},E_{i,k}t\right)\right]\bigg\},\label{Final State}
\end{align}
where the coefficients are defined as $A\left(X,Y\right)=\rme ^{-\rmi Y}+2\rmi \sin^{2}X\sin Y$, $B\left(X,Y\right)=\rme ^{-\rmi Y}+2\rmi \cos^{2}X\sin Y$, $C\left(X,Y\right)=\sin2X\sin Y$ and $X$,$Y$ are real variables.
With the reduced density matrix, one could calculate the time evolution of numerous intriguing quantities about the bath spins.
 
\setcounter{section}{3}
\setcounter{equation}{0}
\section*{Appendix C.~Calculation of QFI}\label{Appendix_C}

For zero temperture (large $\beta$), from the previous results in appendix~B, we have
\begin{eqnarray}
	\left\langle \psi_{0}\left|J_{z}^{2}\right|\psi_{0}\right\rangle  
	& = & \bigotimes_{k>0} \sideset{_{k}}{_{k}}{\mathop{\left\langle \phi_{0}\left|\sum_{k{}^{\prime},k^{\prime\prime}>0}\left(Q_{k^{\prime}}\bigotimes_{k_{1}\neq k^{\prime}}\mathbf{1}_{k_{1}}\right)\left(Q_{k^{\prime\prime}}\bigotimes_{k_{2}\neq k^{\prime\prime}}\mathbf{1}_{k_{2}}\right)\right|\phi_{0}\right\rangle }}\nonumber\\
	& = & \sum_{k>0}\sideset{_{k}}{_{k}}{\mathop{\left\langle \phi_{0}\left|Q_{k}^{2}\right|\phi_{0}\right\rangle}}\bigotimes_{k_{1}\neq k}\mathbf{1}_{k_{1}}\nonumber\\
	&\quad&+\sum_{k{}^{\prime}\neq k^{\prime\prime}>0}\left({}_{k^{\prime}}\left\langle \phi_{0}\left|Q_{k^{\prime}}\right|\phi_{0}\right\rangle _{k^{\prime}}\bigotimes_{k_{1}\neq k^{\prime}}\mathbf{1}_{k_{1}}\right)\left({}_{k^{\prime\prime}}\left\langle \phi_{0}\left|Q_{k^{\prime\prime}}\right|\phi_{0}\right\rangle _{k^{\prime\prime}}\bigotimes_{k_{2}\neq k^{\prime\prime}}\mathbf{1}_{k_{2}}\right)\nonumber\\
	& = & \frac{N_{c}}{2}-\sum_{k>0}\cos^{2}\left(2\theta_{k}\right)+\sum_{k{}^{\prime},k^{\prime\prime}>0}\cos\left(2\theta_{k^{\prime}}\right)\cos\left(2\theta_{k^{\prime\prime}}\right), \\
	&=& \left|\bigotimes_{k>0} \sideset{_{k}}{_{k}}{\mathop{\left\langle \phi_{0}\left|\sum_{k{}^{\prime}>0}\left(Q_{k^{\prime}}\bigotimes_{k_{1}\neq k^{\prime}}\mathbf{1}_{k_{1}}\right)\right|\phi_{0}\right\rangle}}\right|^{2}
\end{eqnarray}
\begin{equation}
	\left\langle \psi_{0}\left|J_{z}\right|\psi_{0}\right\rangle ^{2} 
	=  \left|\bigotimes_{k>0} \sideset{_{k}}{_{k}}{\mathop{\left\langle \phi_{0}\left|\sum_{k{}^{\prime}>0}\left(Q_{k^{\prime}}\bigotimes_{k_{1}\neq k^{\prime}}\mathbf{1}_{k_{1}}\right)\right|\phi_{0}\right\rangle}}\right|^{2}
	 = \sum_{k{}^{\prime},k^{\prime\prime}>0}\cos\left(2\theta_{k^{\prime}}\right)\cos\left(2\theta_{k^{\prime\prime}}\right),
\end{equation}
and 
\begin{equation}
\text{Var}\left(J_{z}\right)=\left\langle \psi_{0}\left|J_{z}^{2}\right|\psi_{0}\right\rangle -\left\langle \psi_{0}\left|J_{z}\right|\psi_{0}\right\rangle ^{2}=\sum_{k>0}\sin^{2}\left(2\theta_{k}\right),
\end{equation}
where $J_{z}=-\sum_{k}\left(c_{k}^{\dagger}c_{k}-1/2\right)=-\sum_{k>0}\left(c_{k}^{\dagger}c_{k}+c_{-k}^{\dagger}c_{-k}-1\right)=\sum_{k>0}Q_{k}$,
$\left|\psi_{0}\right\rangle =\bigotimes_{k>0}\left|\phi_{0}\right\rangle _{_{k}}$,
$\ _{k}\left\langle \phi_{0}\left|Q_{k}^{2}\right|\phi_{0}\right\rangle _{k}=1$,
$\ _{k}\left\langle \phi_{0}\left|Q_{k}\right|\phi_{0}\right\rangle _{k}=\cos\left(2\theta_{k}\right)$.

Thus, the QFI is 
\begin{eqnarray}
	\mathcal{F}_{h} 
	& = & 4t^{2}\left\{ \left[\left(1-\frac{2g^{2}}{\Delta^{2}}\bar{n}\right)^{2}+\frac{4g^{4}}{\Delta^{4}}\bar{n}\right]\left\langle \psi_{0}\right|J_{z}^{2}\left|\psi_{0}\right\rangle -\left(1-\frac{2g^{2}}{\Delta^{2}}\bar{n}\right)^{2}\left\langle \psi_{0}\right|J_{z}\left|\psi_{0}\right\rangle ^{2}\right\}\nonumber \\
	& = & 4t^{2}\left[\left(1-\frac{2g^{2}}{\Delta^{2}}\bar{n}\right)^{2}\left(\left\langle \psi_{0}\right|J_{z}^{2}\left|\psi_{0}\right\rangle -\left\langle \psi_{0}\right|J_{z}\left|\psi_{0}\right\rangle ^{2}\right)+\frac{4g^{4}}{\Delta^{4}}\bar{n}\left\langle \psi_{0}\right|J_{z}^{2}\left|\psi_{0}\right\rangle \right]\nonumber\\
	& = & 4t^{2}\left\{ \left(1-\frac{2g^{2}}{\Delta^{2}}\bar{n}\right)^{2}\left[\frac{N_{c}}{2}
	-\sum_{k>0}\cos^{2}\left(2\theta_{k}\right)\right]\right.\nonumber\\
	&\quad&\left.+\frac{4g^{4}}{\Delta^{4}}\bar{n}\left[\frac{N_{c}}{2}-\sum_{k>0}\cos^{2}\left(2\theta_{k}\right)+\sum_{k{}^{\prime},k^{\prime\prime}>0}\cos\left(2\theta_{k^{\prime}}\right)\cos\left(2\theta_{k^{\prime\prime}}\right)\right]\right\} 
\end{eqnarray}
where $\bar{n}=\left|\alpha\right|^{2}$, $\left\langle \alpha\right|\left(1-2g^{2}a^{\dagger}a/\Delta^{2}\right)^{2}\left|\alpha\right\rangle =\left(1-2g^{2}\bar{n}/\Delta^{2}\right)^{2}+4g^{4}\text{Var}\left(a^{\dagger}a\right)/\Delta^{4}=\left(1-2g^{2}\bar{n}/\Delta^{2}\right)^{2}+4g^{4}\bar{n}/\Delta^{4}$.

\setcounter{section}{4}
\setcounter{equation}{0}
\section*{Appendix D.~QFI with respect to the anisotropy parameter $\gamma$}\label{Appendix_D}

Here, we consider the weak magnetic field case $h\ll2\pi\lambda/N_{c}$.
When $\gamma\neq1$ (anisotropic XY model), from \cref{QFI_h}, we know that $\Lambda_{k}\approx\lambda\sqrt{\cos^{2}k+\gamma^{2}\sin^{2}k}=\lambda\sqrt{1+\left(\gamma^{2}-1\right)\sin^{2}k}$.

If $\gamma\approx1$, the behavior of the QFI is analogous to the quantum Ising model case ($\gamma=1$), where $\Lambda_{k}\approx\lambda$, $\sin\nu_{k}=\lambda\gamma\sin k/\Lambda_{k}\approx\sin k$ and $\cos\nu_{k}=\left(\lambda\cos k-h\right)/\Lambda_{k}\approx\cos k$.

If $\gamma^{2}\gg\left[N_{c}/\left(2\pi\right)\right]^{2}$, then $\Lambda_{k}\approx\lambda\gamma\sin k$, $\sin\nu_{k}=\lambda\gamma\sin k/\Lambda_{k}\approx1$ and $\cos\nu_{k}=\left(\lambda\cos k-h\right)/\Lambda_{k}\approx1/\left(\gamma\tan k\right)\approx0$ for $k=2\pi/N_{c},\dots,\ \pi-2\pi/N_{c}$.
For $k=\pi$, $\Lambda_{k}=\lambda$, $\sin\nu_{k}=\lambda\gamma\sin k/\Lambda_{k}=0$ and $\cos\nu_{k}=\left(\lambda\cos k-h\right)/\Lambda_{k}=1$.
Thus, the QFI
\begin{align}
\mathcal{F}_{h} 
& \approx  4t^{2}\left\{\left(1-\frac{2g^{2}}{\Delta^{2}}\bar{n}\right)^{2}\left(\frac{N_{c}}{2}-1-\sum_{k>0}^{\pi-2\pi/N_{c}}\cos^{2}\nu_k\right)\right.\nonumber\\
&\quad \left.+\frac{4g^{4}}{\Delta^{4}}\bar{n}\left[\frac{N_{c}}{2}\!-\!1\!-\!\sum_{k>0}^{\pi-2\pi/N_{c}}\!\cos^{2}\!\nu_k\!+\!\sum_{k^{\prime},k^{\prime\prime}>0}^{\pi-2\pi/N_{c}}\!\cos\! \nu_k\!\cos\! \nu_{k{\prime}}\!+\!1\!+\!2\left(\frac{N_{c}}{2}-1\right)\sum_{k{}^{\prime}>0}^{\pi-2\pi/N_{c}}\cos\nu_{\pi}\cos\nu_{k^{\prime}}\right]\right\}\nonumber\\
& \approx 4t^{2}\left[\left(1-\frac{2g^{2}}{\Delta^{2}}\bar{n}\right)^{2}\left(\frac{N_{c}}{2}-1\right)+\frac{2g^{4}}{\Delta^{4}}\bar{n}N_{c}\right],
\end{align}
where $\sum_{k>0}^{\pi-2\pi/N_{c}}\cos\nu_{k}\approx0$, $\sum_{k>0}^{\pi-2\pi/N_{c}}\cos^{2}\nu_{k}\approx0$,
and $\sum_{k{}^{\prime},k^{\prime\prime}>0}^{\pi-2\pi/N_{c}}\cos\nu_{k^{\prime}}\cos\nu_{k^{\prime\prime}}\approx0$.
We gain the QFI
\begin{equation}
\mathcal{F}_{h} \approx  \frac{8g^{4}}{\Delta^{4}}t^{2}N_{c}\bar{n}^{2}
\end{equation}
for a large average photon number.

If $\gamma^{2}\ll\left(2\pi/N_{c}\right)^{2}$ , then $\Lambda_{k}\approx\lambda\left|\cos k\right|$, $\sin\nu_{k}=\lambda\gamma\sin k/\Lambda_{k}\approx\gamma\left|\tan k\right|$ and $\cos\nu_{k}=\left(\lambda\cos k-h\right)/\Lambda_{k}\approx\cos k/\left|\cos k\right|$ for $k\neq\pi/2$.
For  $N_{c}/2=2l$ and $k=\pi/2$, $\Lambda_{k}=\sqrt{h^{2}+\lambda^{2}\gamma^{2}}$, $\sin\nu_{k}\lambda\gamma/\sqrt{h^{2}+\lambda^{2}\gamma^{2}}$ and $\cos\nu_{k}=-h/\sqrt{h^{2}+\lambda^{2}\gamma^{2}}$.
Therefore, the QFI
\begin{eqnarray}
\mathcal{F}_{h} 
& \approx & \begin{cases}
4t^{2}\left[\left(1-\frac{2g^{2}}{\Delta^{2}}\bar{n}\right)^{2}\frac{\lambda^{2}\gamma^{2}}{h^{2}+\lambda^{2}\gamma^{2}}+\frac{4g^{4}}{\Delta^{4}}\bar{n}\left(2+\frac{2h}{\sqrt{h^{2}+\lambda^{2}\gamma^{2}}}\right)\right] &\text{for}\   \frac{N_{c}}{2}=2l\\
\frac{16g^{4}}{\Delta^{4}}t^{2}\bar{n} &\text{for}\   \frac{N_{c}}{2}=2l+1
\end{cases},
\end{eqnarray}
where 
\begin{eqnarray}
\sum_{k>0}\cos\nu_{k}&\approx&\begin{cases}
-1-h/\sqrt{h^{2}+\lambda^{2}\gamma^{2}} &\text{for}\   \frac{N_{c}}{2}=2l\\
-1 &\text{for}\   \frac{N_{c}}{2}=2l+1
\end{cases},\nonumber\\
\sum_{k>0}\cos^{2}\nu_{k}&\approx&\begin{cases}
N_{c}/2-1+h^{2}/\left(h^{2}+\lambda^{2}\gamma^{2}\right) &\text{for}\  \frac{N_{c}}{2}=2l\\
N_{c}/2 &\text{for}\   \frac{N_{c}}{2}=2l+1
\end{cases},\nonumber
\end{eqnarray}
and
\begin{eqnarray}
\sum_{k{}^{\prime},k^{\prime\prime}>0}\cos\nu_{k^{\prime}}\cos\nu_{k^{\prime\prime}}\approx\begin{cases}
\left(1+h/\sqrt{h^{2}+\lambda^{2}\gamma^{2}}\right)^{2} &\text{for}\  \frac{N_{c}}{2}=2l\\
1 &\text{for}\  \frac{N_{c}}{2}=2l+1
\end{cases}.\nonumber
\end{eqnarray}
%where $\sum_{k>0}\cos\nu_{k}\approx\begin{cases}
%-1-h/\sqrt{h^{2}+\lambda^{2}\gamma^{2}}, & \frac{N_{c}}{2}=2l\\
%-1, &\frac{N_{c}}{2}=2l+1
%\end{cases}$, $\sum_{k>0}\cos^{2}\nu_{k}\approx\begin{cases}
%N_{c}/2-1+h^{2}/\left(h^{2}+\lambda^{2}\gamma^{2}\right), &\frac{N_{c}}{2}=2l\\
%N_{c}/2, & \frac{N_{c}}{2}=2l+1
%\end{cases}$, and $\sum_{k{}^{\prime},k^{\prime\prime}>0}\cos\nu_{k^{\prime}}\cos\nu_{k^{\prime\prime}}\approx\begin{cases}
%\left(1+h/\sqrt{h^{2}+\lambda^{2}\gamma^{2}}\right)^{2}, & \frac{N_{c}}{2}=2l\\
%1, &\frac{N_{c}}{2}=2l+1
%\end{cases}$.

%For a large average photon number, we obtain the QFIs
%\begin{eqnarray*}
%	\mathcal{F}_{h} & \approx & \begin{cases}
%		4t^{2}\left[\left(1-\frac{2g^{2}}{\Delta^{2}}\bar{n}\right)^{2}+\frac{8g^{4}}{\Delta^{4}}\bar{n}\right] & ,\ \frac{N_{c}}{2}=2l\\
%		\frac{16g^{4}}{\Delta^{4}}t^{2}\bar{n} & \ \frac{N_{c}}{2}=2l+1
%	\end{cases}
%\end{eqnarray*}
%with $h\ll\lambda\gamma$ and 
%\begin{eqnarray*}
%	\mathcal{F}_{h} & \approx & \begin{cases}
%		4t^{2}\left[\left(1-\frac{2g^{2}}{\Delta^{2}}\bar{n}\right)^{2}\frac{\lambda^{2}\gamma^{2}}{h^{2}+\lambda^{2}\gamma^{2}}+\frac{16g^{4}}{\Delta^{4}}\bar{n}\right] & ,\ \frac{N_{c}}{2}=2l\\
%		\frac{16g^{4}}{\Delta^{4}}t^{2}\bar{n} & \ \frac{N_{c}}{2}=2l+1
%	\end{cases}
%\end{eqnarray*}
%with $h\gg\lambda\gamma$.

Particularly, for $\gamma=0$ (isotropic XX model), we know the QFI 
\begin{align}
\mathcal{F}_{h} 
& =  4t^{2}\left\{ \left(1-\frac{2g^{2}}{\Delta^{2}}\bar{n}\right)^{2}\left(\frac{N_{c}}{2}\!-\!\sum_{k>0}1\right)+\frac{4g^{4}}{\Delta^{4}}\bar{n}\left[\frac{N_{c}}{2}\!-\!\sum_{k>0}\!1\!+\!\sum_{k{}^{\prime},k^{\prime\prime}>0}\frac{\left(\lambda\cos k^{\prime}-h\right)\left(\lambda\cos k^{\prime\prime}-h\right)}{\left|\left(\lambda\cos k^{\prime}-h\right)\left(\lambda\cos k^{\prime\prime}-h\right)\right|}\right]\right\} \nonumber\\
& =  \begin{cases}
\frac{64g^{4}}{\Delta^{4}}t^{2}\bar{n} &\text{for}\  \frac{N_{c}}{2}=2l\\
\frac{16g^{4}}{\Delta^{4}}t^{2}\bar{n}  &\text{for}\   \frac{N_{c}}{2}=2l+1
\end{cases},
\end{align}
where $\Lambda_{k}=\left|\lambda\cos k-h\right|$,
$\sin\nu_{k}=0$,
$\cos\nu_{k}=\left(\lambda\cos k-h\right)/\left|\lambda\cos k-h\right|$,
$\sum_{k>0}\cos^{2}\nu_{k}=N_{c}/2$, and $\sum_{k{}^{\prime},k^{\prime\prime}>0}\cos\nu_{k^{\prime}}\cos\nu_{k^{\prime\prime}}=\begin{cases}
4  &\text{for}\   \frac{N_{c}}{2}=2l\\
1  &\text{for}\   \frac{N_{c}}{2}=2l+1
\end{cases}$.

In fact, for the non-interaction case ($\lambda=0$), we have the QFI
\begin{align}
\mathcal{F}_{h} 
& =4t^{2}\left[ \left(1-\frac{2g^{2}}{\Delta^{2}}\bar{n}\right)^{2}\left(\frac{N_{c}}{2}-\sum_{k>0}1\right)+\frac{4g^{4}}{\Delta^{4}}\bar{n}\left(\frac{N_{c}}{2}-\sum_{k>0}1+\sum_{k{}^{\prime},k^{\prime\prime}>0}1\right)\right]
=  \frac{4g^{4}}{\Delta^{4}}t^{2}N_{c}^{2}\bar{n},
\end{align}
where $\Lambda_{k}=\left|h\right|$, $\sin\nu_{k}=0$, and $\cos\nu_{k}=-1$.

\setcounter{section}{4}
\setcounter{equation}{0}


\section*{References}

%\bibliography{jparef}
%\bibliography{JPARef}
%\bibliography{XYModel}

\providecommand{\newblock}{}
\begin{thebibliography}{10}
	\expandafter\ifx\csname url\endcsname\relax
	\def\url#1{{\tt #1}}\fi
	\expandafter\ifx\csname urlprefix\endcsname\relax\def\urlprefix{URL }\fi
	\providecommand{\eprint}[2][]{\url{#2}}
	% Bibliography created with iopart-num v2.1
	% /biblio/bibtex/contrib/iopart-num
	
	\bibitem{WOS:000288984900012}
	Giovannetti V, Lloyd S and Maccone L 2011 {\em Nat. Photonics\/} {\bf 5}
	222--229
	
	\bibitem{RevModPhys.89.035002}
	Degen C~L, Reinhard F and Cappellaro P 2017 {\em Rev. Mod. Phys.\/} {\bf 89}
	035002
	
	\bibitem{PhysRevA.54.R4649}
	Bollinger J~J, Itano W~M, Wineland D~J and Heinzen D~J 1996 {\em Phys. Rev.
		A\/} {\bf 54} R4649--R4652
	
	\bibitem{PhysRevLett.106.130506}
	Monz T, Schindler P, Barreiro J~T, Chwalla M, Nigg D, Coish W~A, Harlander M,
	H\"ansel W, Hennrich M and Blatt R 2011 {\em Phys. Rev. Lett.\/} {\bf 106}
	130506
	
	\bibitem{PhysRevA.92.023603}
	Muessel W, Strobel H, Linnemann D, Zibold T, Juli\'a-D\'{\i}az B and Oberthaler
	M~K 2015 {\em Phys. Rev. A\/} {\bf 92} 023603
	
	\bibitem{MA201189}
	Ma J, Wang X, Sun C and Nori F 2011 {\em Phys. Rep.\/} {\bf 509} 89--165
	
	\bibitem{PhysRevA.102.052423}
	Su Y, Liang H and Wang X 2020 {\em Phys. Rev. A\/} {\bf 102} 052423
	
	\bibitem{WOS:A1980KA25400008}
	Ramsey N 1980 {\em Phys. Today\/} {\bf 33} 25--30
	
	\bibitem{PhysRevLett.86.5870}
	Meyer V, Rowe M~A, Kielpinski D, Sackett C~A, Itano W~M, Monroe C and Wineland
	D~J 2001 {\em Phys. Rev. Lett.\/} {\bf 86} 5870--5873
	
	\bibitem{RevModPhys.87.637}
	Ludlow A~D, Boyd M~M, Ye J, Peik E and Schmidt P~O 2015 {\em Rev. Mod. Phys.\/}
	{\bf 87} 637--701
	
	\bibitem{Louchet_Chauvet_2010}
	Louchet-Chauvet A, Appel J, Renema J~J, Oblak D, Kjaergaard N and Polzik E~S
	2010 {\em New J. Phys.\/} {\bf 12} 065032
	
	\bibitem{PhysRevLett.112.190403}
	Kessler E~M, K\'om\'ar P, Bishof M, Jiang L, S\o{}rensen A~S, Ye J and Lukin
	M~D 2014 {\em Phys. Rev. Lett.\/} {\bf 112} 190403
	
	\bibitem{WALLS1981118}
	Walls D and Zoller P 1981 {\em Phys. Lett. A\/} {\bf 85} 118--120
	
	\bibitem{WOS:000256613000015}
	Goda K, Miyakawa O, Mikhailov E~E, Saraf S, Adhikari R, McKenzie K, Ward R,
	Vass S, Weinstein A~J and Mavalvala N 2008 {\em Nat. Phys.\/} {\bf 4}
	472--476
	
	\bibitem{Abbott_2009}
	\text{Abbott B P \textit{et al}} 2009 {\em Rep. Prog. Phys.\/} {\bf 72} 076901
	
	\bibitem{budker_optical_2007}
	Budker D and Romalis M 2007 {\em Nat. Phys.\/} {\bf 3} 227--234
	
	\bibitem{PhysRevLett.109.253605}
	Sewell R~J, Koschorreck M, Napolitano M, Dubost B, Behbood N and Mitchell M~W
	2012 {\em Phys. Rev. Lett.\/} {\bf 109} 253605
	
	\bibitem{PhysRevLett.120.260503}
	Troiani F and Paris M~G~A 2018 {\em Phys. Rev. Lett.\/} {\bf 120} 260503
	
	\bibitem{doi:10.1073/pnas.1004037107}
	Carlton P~M, Boulanger J, Kervrann C, Sibarita J~B, Salamero J, Gordon-Messer
	S, Bressan D, Haber J~E, Haase S, Shao L, Winoto L, Matsuda A, Kner P, Uzawa
	S, Gustafsson M, Kam Z, Agard D~A and Sedat J~W 2010 {\em Proc. Natl. Acad.
		Sci.\/} {\bf 107} 16016--16022
	
	\bibitem{WOS:000316154700018}
	Taylor M~A, Janousek J, Daria V, Knittel J, Hage B, Bachor H~A and Bowen W~P
	2013 {\em Nat. Photonics\/} {\bf 7} 229--233
	
	\bibitem{TAYLOR20161}
	Taylor M~A and Bowen W~P 2016 {\em Phys. Rep.\/} {\bf 615} 1--59
	
	\bibitem{PhysRevLett.72.3439}
	Braunstein S~L and Caves C~M 1994 {\em Phys. Rev. Lett.\/} {\bf 72} 3439--3443
	
	\bibitem{Dutra-Book:2005}
	Dutra S 2005 {\em Cavity quantum electrodynamics\/} (John Wiley and Sons Inc.)
	
	\bibitem{Raimond-RMP:2001}
	Raimond J~M, Brune M and Haroche S 2001 {\em Rev. Mod. Phys.\/} {\bf 73}
	565--582
	
	\bibitem{doi:10.1063/1.882326}
	Haroche S 1998 {\em Phys. Today\/} {\bf 51} 36--42
	
	\bibitem{RevModPhys.73.565}
	Raimond J~M, Brune M and Haroche S 2001 {\em Rev. Mod. Phys.\/} {\bf 73}
	565--582
	
	\bibitem{doi:10.1126/science.1078446}
	Mabuchi H and Doherty A~C 2002 {\em Science\/} {\bf 298} 1372--1377
	
	\bibitem{WOS:000223746000038}
	Wallraff A, Schuster D, Blais A, Frunzio L, Huang R, Majer J, Kumar S, Girvin S
	and Schoelkopf R 2004 {\em Nature\/} {\bf 431} 162--167
	
	\bibitem{PhysRevLett.76.1800}
	Brune M, Schmidt-Kaler F, Maali A, Dreyer J, Hagley E, Raimond J~M and Haroche
	S 1996 {\em Phys. Rev. Lett.\/} {\bf 76} 1800--1803
	
	\bibitem{mckeever_experimental_2003}
	McKeever J, Boca A, Boozer A~D, Buck J~R and Kimble H~J 2003 {\em Nature\/}
	{\bf 425} 268--271
	
	\bibitem{Bennett2000}
	Bennett C~H and Divincenzo D~P 2000 {\em Nature\/} {\bf 404} 247--255
	
	\bibitem{Knill2001}
	Knill E, Laflamme R and Milburn G~J 2001 {\em Nature\/} {\bf 409} 46--52
	
	\bibitem{Buluta_2011}
	Buluta I, Ashhab S and Nori F 2011 {\em Rep. Prog. Phys.\/} {\bf 74} 104401
	
	\bibitem{Buluta2009}
	Buluta I and Nori F 2009 {\em Science\/} {\bf 326} 108--111
	
	\bibitem{Georgescu2014}
	Georgescu I~M, Ashhab S and Nori F 2014 {\em Rev. Mod. Phys.\/} {\bf 86}
	153--185
	
	\bibitem{Scarani2009}
	Scarani V, Bechmann-Pasquinucci H, Cerf N~J, Du\ifmmode~\check{s}\else
	\v{s}\fi{}ek M, L\"utkenhaus N and Peev M 2009 {\em Rev. Mod. Phys.\/} {\bf
		81} 1301--1350
	
	\bibitem{obrien_photonic_2009}
	O'Brien J~L, Furusawa A and Vu\v{c}kovi\'c J 2009 {\em Nat. Photonics\/} {\bf
		3} 687--695
	
	\bibitem{Giovannetti2011}
	Giovannetti V, Lloyd S and Maccone L 2011 {\em Nat. Photonics\/} {\bf 96}
	222--229
	
	\bibitem{Liu_2019}
	Liu J, Yuan H, Lu X~M and Wang X 2019 {\em J. Phys. A: Math. Theor.\/} {\bf 53}
	023001
	
	\bibitem{YuguoSu2021}
	{Yuguo Su} and Wang X 2021 {\em Results Phys.\/} {\bf 24} 104159
	
	\bibitem{WOS:000368673800032}
	Hosten O, Engelsen N~J, Krishnakumar R and Kasevich M~A 2016 {\em Nature\/}
	{\bf 529} 505
	
	\bibitem{PhysRevLett.116.093602}
	Cox K~C, Greve G~P, Weiner J~M and Thompson J~K 2016 {\em Phys. Rev. Lett.\/}
	{\bf 116} 093602
	
	\bibitem{Flower_2019}
	Flower G, Goryachev M, Bourhill J and Tobar M~E 2019 {\em New J. Phys.\/} {\bf
		21} 095004
	
	\bibitem{Gietka_2021}
	Gietka K, Mivehvar F and Busch T 2021 {\em New J. Phys.\/} {\bf 23} 043020
	
	\bibitem{Kim1999}
	Kim M~S and Agarwal G~S 1999 {\em Phys. Rev. A\/} {\bf 59} 3044--3048
	
	\bibitem{WOS:000263818900002}
	Scheel S 2009 {\em J. Mod. Opt.\/} {\bf 56} 141--160
	
	\bibitem{PhysRevA.94.022313}
	Penasa M, Gerlich S, Rybarczyk T, M\'etillon V, Brune M, Raimond J~M, Haroche
	S, Davidovich L and Dotsenko I 2016 {\em Phys. Rev. A\/} {\bf 94} 022313
	
	\bibitem{PhysRevLett.120.050507}
	Xu K, Chen J~J, Zeng Y, Zhang Y~R, Song C, Liu W, Guo Q, Zhang P, Xu D, Deng H,
	Huang K, Wang H, Zhu X, Zheng D and Fan H 2018 {\em Phys. Rev. Lett.\/} {\bf
		120} 050507
	
	\bibitem{doi:10.1126/science.aay0600}
	Song C, Xu K, Li H, Zhang Y~R, Zhang X, Liu W, Guo Q, Wang Z, Ren W, Hao J,
	Feng H, Fan H, Zheng D, Wang D~W, Wang H and Zhu S~Y 2019 {\em Science\/}
	{\bf 365} 574--577
	
	\bibitem{James2007}
	James D~F and Jerke J 2007 {\em Can. J. Phys.\/} {\bf 85} 625--632
	
	\bibitem{TAM}
	Shore B~W and Knight P~L 1993 {\em J. Mod. Opt.\/} {\bf 40} 1195--1238
	
	\bibitem{WOS:000243867300038}
	Schuster D~I, Houck A~A, Schreier J~A, Wallraff A, Gambetta J~M, Blais A,
	Frunzio L, Majer J, Johnson B, Devoret M~H, Girvin S~M and Schoelkopf R~J
	2007 {\em Nature\/} {\bf 445} 515--518
	
	\bibitem{JOHANSSON20121760}
	Johansson J, Nation P and Nori F 2012 {\em Comput. Phys. Commun.\/} {\bf 183}
	1760--1772
	
	\bibitem{Sun06}
	Quan H~T, Song Z, Liu X~F, Zanardi P and Sun C~P 2006 {\em Phys. Rev. Lett.\/}
	{\bf 96} 140604
	
	\bibitem{WOS:000285157400001}
	Gu S~J 2010 {\em Int. J. Mod. Phys. B\/} {\bf 24} 4371--4458
	
	\bibitem{PhysRevLett.130.173601}
	Liu Y, Wang Z, Yang P, Wang Q, Fan Q, Guan S, Li G, Zhang P and Zhang T 2023
	{\em Phys. Rev. Lett.\/} {\bf 130} 173601
	
	\bibitem{WOS:000257665300034}
	Fink J~M, Goeppl M, Baur M, Bianchetti R, Leek P~J, Blais A and Wallraff A 2008
	{\em Nature\/} {\bf 454} 315--318
	
	\bibitem{PhysRevLett.128.123602}
	Allcock T, Langbein W and Muljarov E~A 2022 {\em Phys. Rev. Lett.\/} {\bf 128}
	123602
	
	\bibitem{WOS:000322086100035}
	Brennecke F, Mottl R, Baumann K, Landig R, Donner T and Esslinger T 2013 {\em
		Proc. Natl. Acad. Sci. U.S.A.\/} {\bf 110} 11763--11767
	
	\bibitem{PhysRevA.71.013817}
	Spillane S~M, Kippenberg T~J, Vahala K~J, Goh K~W, Wilcut E and Kimble H~J 2005
	{\em Phys. Rev. A\/} {\bf 71} 013817
	
	\bibitem{PhysRevA.67.033806}
	Buck J~R and Kimble H~J 2003 {\em Phys. Rev. A\/} {\bf 67} 033806
	
\end{thebibliography}


\end{document}
