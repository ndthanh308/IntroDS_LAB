The volume imbalance in a limit order book is defined as
\begin{equation*}
    I = \frac{q^b - q^a}{q^b + q^a}
\end{equation*}
where $q^b$ is the quantity of an asset posted on the best bid price, and $q^a$ the quantity posted on the best ask price. Empirical studies indicate that the volume imbalance is a very good predictor of future asset price moves: when it is close to 1, the mid-price is likely to jump upwards, and when it is close to -1, the mid-price is likely to jump downwards \parencite{huang_simulating_2015,lehalle_limit_2017,sfendourakis_lob_2023}. In view of the importance of the volume imbalance to explain price moves, \textcite{stoikov_micro-price_2018} introduced the notion of \enquote{micro-price}, representing a \enquote{fair} price of the asset, which is defined in terms of observable quantities among which the volume imbalance plays an important role.  

In this work, we propose to take the opposite viewpoint. We consider a market maker who is the only liquidity provider of the market and knows an \enquote{efficient} price driving the mid-price and the market order dynamics. The market maker quotes volumes at the best bid and ask prices. The posted volumes do not influence the price moves nor the arrival of market orders. We investigate then, the character of the optimal posted volumes and, in particular, whether these volumes generate a predictive imbalance in the limit order book.

The market-making literature in continuous time is  abundant. In the pioneering work of \textcite{avellaneda_high-frequency_2008}, studied in more mathematical detail in \textcite{gueant_dealing_2013} by introducing inventory bounds, the market maker quotes a fixed quantity of the asset on the bid and the ask side and, contrary to our approach, controls the price of these quotes. Their model, well-suited for over-the-counter markets has been enriched in various ways: allowing to quote different volumes \parencite{bergault_size_2021,barzykin_algorithmic_2023}, or considering principal-agent problems related to the optimal make-take fees of trading platforms \parencite{el_euch_optimal_2021,baldacci2021optimal,baldacci_market_2023}.

Models considering a fixed tick size, which is the smallest allowed increment between two prices, are well-suited to the limit order book framework. This is the point of view taken in \parencite{guilbaud_optimal_2013,guilbaud_optimal_2015,fodra_high_2015} where the limit order book evolves in a Markovian way. The best limits are observed, and the market maker can quote at them, or in the spread. More recently, \textcite{abergel_algorithmic_2020}, propose a model where the whole (up to some depth) order book is modeled, and the market maker can have quotes pending anywhere. In these models, however, there is no efficient price driving the mid-price fluctuations.

In \parencite{baldacci_bid_2020}, two tick grids are considered --one for the bid, and one for the ask. There is an underlying efficient price observed by the market maker. She can only quote at the considered \enquote{fair} bid and ask prices, lying on their respective tick grid, and evolving according to the model with uncertainty zones, introduced by \textcite{robert_new_2011}: when the efficient price hits a defined barrier, the \enquote{fair} bid or ask price jumps.

We consider a framework close to the one of \parencite{baldacci_bid_2020}: there is an underlying efficient price, following a Bachelier model, and a \enquote{fair} mid-price that jumps up if the efficient price hits the upper barrier and jumps down if it hits the lower barrier. A large-tick stock is considered, hence the market maker is only allowed to quote at half-tick distance from the mid-price. She controls the volume that she posts on each side. The market orders are represented by marked Poisson processes, where the mark is their volume. They depend on the distance between the efficient price and the mid-price, and consequently they are not controlled by the market maker.

The market maker aims to maximize the exponential utility of her terminal gains. A Hamilton-Jacobi-Bellman system of PDEs with nonlocal boundary conditions is associated to this control problem. Contrary to the majority of the above mentioned studies on optimal market-making, where viscosity solutions are considered, we prove using PDE techniques, the existence of a smooth solution to this system. Using a verification argument, we show that the said solution is equal to her value function, and we give a characterization of her optimal controls. In addition, and remarkably, we are able to derive the uniqueness of her optimal control policy. Numerical approximations of the HJB equation -- and consequently of the optimal posted volumes -- indicate that for certain choices of the model parameters, the optimal posted volume imbalance is predictive of price moves. More precisely, under certain scenarios, the optimal volume imbalance is a monotone function of the distance between the efficient and mid-prices, which in turn determine the price moves via the uncertainty zones model. 

The paper is organized as follows. In Section \ref{sec:notation}, we introduce the mathematical notation used throughout the paper. In Section \ref{sec:control_problem}, we present the market dynamics and the market maker's control problem. In Section \ref{sec:results}, we state the main existence and uniqueness results of the control problem. Numerical illustrations are given in Section \ref{sec:numerics}. The verification theorem is proved in Section \ref{sec:verification}. In Section \ref{sec:hjb}, we prove the existence of a smooth solution to the HJB equation. Section \ref{sec:uniqueness} contains the proof of the uniqueness of the control policy. Some of the proofs are relegated to the appendices.