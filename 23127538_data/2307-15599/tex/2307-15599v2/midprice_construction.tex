\section{Construction of the mid-price process}
\label{sec:midprice_construction}

\subsection{Setup}

Let $a < b$ be two real numbers and $a_0, b_0 \in (a,b)$. Let $T > 0$ and define $\Delta \defeq \min\{a_0-a, b-a_0, b_0-a, b - b_0\} > 0$.

In this section, we will construct a measurable mapping $Y:[0,T] \times [0, T] \times (a,b) \times \mathcal{C}_0^0([0, T]) \mapsto \mathbb{R}$, valued in $(a,b)$, where $\mathcal{C}_0^0([0,T])$ is the space of real-valued continuous functions $w$ on $[0,T]$ verifying $w(0)=0$, equipped with the sup norm.

Intuitively, $Y(t,\cdot,y,w)$ is a function starting with the value $y$ at $t$ which behaves like $w$ until it attains the boundary $a$ or $b$. If it reaches the boundary at $a$, it jumps to $a_0$, and if it reaches the boundary at $b$, it jumps to $b_0$.

\subsection{Construction}
\label{subsec:midrpice_construction}

For $(t, y, w) \in [0,T] \times (a,b) \times \mathcal{C}_0^0([0, T])$, define for notational convenience $\tau_0(t,y,w) \defeq t$.

Assume that we have constructed measurable mappings $\tau_0 \leqslant \dots \leqslant \tau_n \leqslant T$ and $\epsilon_1, \dots, \epsilon_n$  from $[0,T] \times (a,b) \times \mathcal{C}_0^0([0, T])$ to $[0,T]$ and $\{-1,0,1\}$, respectively, for some $n \in \mathbb{N}$. The variable $\tau_i$ represents the barrier hitting times, $\epsilon_i = 1$ if the boundary is attained at $b$, $\epsilon_i = -1$ if the boundary is attained at $a$. We define $\epsilon'$ to be the jump size, namely $\epsilon_i'(t,y,w) \defeq b_0-b$ if $\epsilon_i(t,y,w)=1$, $\epsilon_i'(t,y,w) \defeq a_0-a$ if $\epsilon_i(t,y,w)=-1$, and $\epsilon_i'(t,y,w) \defeq 0$ otherwise. 

Assume moreover that for every $i \in \integerInterval{n}$ and $(t,y,w)$,
\begin{equation}\label{eq:appA1}
\begin{split}
    &\tau_i(t,y,w) = T \implies \forall j \in \integerInterval[i+1]{n}, \left\{\begin{array}{l}
        \tau_j(t,y,w) = T \\
        \epsilon_j(t,y,w) = 0
    \end{array} \right. \\
    &\tau_i(t,y,w) < T \implies \epsilon_i(t,y,w) \neq 0 \\
    &\tau_i(t,y,w) < T \implies y + (w_{\tau_{i}(t,y,w)} - w_t) + \sum_{j=1}^{i-1} \epsilon_{j}'(t,y,w) \in \left\lbrace a, b \right\rbrace \\
    & i < n,\ \epsilon_{i+1}(t,y,w) \neq 0 \implies \left|w_{\tau_{i+1}(t,y,w)} - w_{\tau_{i}(t,y,w)}\right| \geqslant \Delta \\
    &y + (w_{u} - w_t) +  \sum_{j=1}^{n} \epsilon_{j}'(t,y,w)\mathds{1}_{[\tau_i(t,y,w), T]}(u) \in (a,b),\quad u \in (t,\tau_n(t,y,w)].
\end{split}
\end{equation}

Let $(t, y, w) \in [0,T] \times (a,b) \times \mathcal{C}_0^0([0, T])$. If $\tau_n(t,y,w) = T$, set $\tau_{n+1}(t,y,w) = T$ and $\epsilon_{n+1}(t,y,w) = 0$. In this case the five properties \eqref{eq:appA1} are trivially verified for $\tau_{n+1}(t,y,w)$ and $\epsilon_{n+1}(t,y,w)$.

Suppose now that $\tau_n(t,y,w) < T$. We omit temporarily the dependence on $(t,y,w)$ for concision. We define
\begin{equation}\label{eq:appAtaunepsn}
    \begin{split}
    &\tau_{n+1} \defeq \inf \left\lbrace r \in (\tau_n, T]: y + (w_r - w_t) + \sum_{i=1}^n \epsilon_{i}' \in \left\lbrace a,b \right\rbrace \right\rbrace \wedge T\\
    &\epsilon_{n+1} \defeq \mathds{1}_{\left\{y + (w_{\tau_{n+1}} - w_t) + \sum_{i=1}^n \epsilon_{i}' = b\right\}} - \mathds{1}_{\left\{y + (w_{\tau_{n+1}} - w_t) +\sum_{i=1}^n \epsilon_{i}' = a\right\}}
\end{split}
\end{equation}

The first three properties in \eqref{eq:appA1} are trivially verified for $\tau_{n+1}$ and $\epsilon_{n+1}$.
By assumption, $y + (w_{\tau_n} - w_t) + \sum_{i=1}^n\epsilon_i' \in (a,b)$. Therefore, by the continuity of $w$, the intermediate value theorem and the definition of $\tau_{n+1}$, $y + (w_{u} - w_t) + \sum_{i=1}^n\epsilon_i' \in (a,b)$ for every $u \in (\tau_{n}, \tau_{n+1})$. If $\epsilon_{n+1} = 0$, this is also verified at $u = T$, and the fourth property in \eqref{eq:appA1} holds. Suppose now that $\epsilon_{n+1} \neq 0$. Then, $y + (w_{\tau_n} - w_t) + \sum_{i=1}^{n-1}\epsilon_i' \in \left\lbrace a,b\right\rbrace$ and $y + (w_{\tau_n} - w_t) + \sum_{i=1}^{n}\epsilon_i'  \in \{a_0,b_0\}$. Since $y + (w_{\tau_{n+1}} - w_t) + \sum_{i=1}^{n}\epsilon_i' \in \left\lbrace a,b \right\rbrace$, by taking the difference of these last two expressions, $(w_{\tau_{n+1}} - w_{\tau_n}) \in \left\{a-a_0, b-a_0, a-b_0, b - b_0 \right\}$, hence the fourth property in \eqref{eq:appA1} holds. Similarly, $y + (w_{\tau_{n+1}} - w_t) + \sum_{i=1}^{n+1}\epsilon_i' \in \left\lbrace a_0,b_0\right\rbrace \subset (a,b)$, and the fifth property in \eqref{eq:appA1} is verified.

The following lemma shows that $Y$ can only jump finitely many times.
\begin{lemma}
    \label{lem:finite_activity}
    For all $(t, y, w) \in [0,T] \times \mathcal{Y} \times \mathcal{C}_0^0([0, T])$, there exists $N \in \mathbb{N}$ such that for all $n \geqslant N$, $\tau_n(t,y,w) = T$.
\end{lemma}

\begin{proof}
    Suppose that the conclusion does not hold for some $(t,y,w)$. Then, $(\tau_n(t,y,w))_{n \in \mathbb{N}}$ increases to some limit $t'$. By the left-continuity of $w$, $w_{\tau_n(t,y,w)} \to w_{t'}$, therefore $w_{\tau_{n+1}(t,y,w)}- w_{\tau_n(t,y,w)} \to 0$, contradicting the fourth property in \eqref{eq:appA1}.
\end{proof}

We can now define the function $Y$ for $(t,s,y,w) \in [0,T] \times \mathcal{Y} \times \mathcal{C}_0^0([0, T])$ by
\begin{equation}
    \label{eq:y}
    Y \left(t,s,y,w\right) = y + (w_s - w_t) + \sum_{n = 1}^{\infty}\epsilon'_n(t,y,w)\mathds{1}_{[\tau_{n}(t,y,w),T]}(s).
\end{equation}
It is well-defined thanks to Lemma \ref{lem:finite_activity}.


\subsection{Some useful properties of the barrier reaching times}
\label{subsec:tau_properties}

In this section, we will state some properties of the $\tau_n$'s and the $\epsilon_n$'s that will help us to prove useful properties of $Y$, as stated in Lemmas \ref{lem:finite_activity_expectation} and \ref{lem:y_independence} below.

The following two lemmas are a direct consequence of the recursive construction of the $\tau_n$'s and $\epsilon_n$'s in \eqref{eq:appAtaunepsn}.
\begin{lemma}
    \label{lem:tau_independence_past}
    For all $(t,y,w) \in [0,T] \times (a,b) \times \mathcal{C}_0^0([0,T])$, $n \in \mathbb{N}$ and $t_1 \in [0,t]$,
    $\tau_n(t,y,w) = \tau_n \left(t,y,\left(w_{u \vee t_1} - w_{t_1}\right)_{u \in [0, T]}\right)$ and $\epsilon_n(t,y,w) = \epsilon_n \left(t,y,\left(w_{u \vee t_1} - w_{t_1}\right)_{u \in [0, T]}\right)$.
\end{lemma}

%\begin{proof}
%    This is a direct consequence of the recursive construction of the $\tau_n$'s and $\epsilon_n$'s, see in particular \eqref{eq:appAtaunepsn}, and the fact that, for $r>t$, $w_{r} -w_{t} = w_{r \vee t_1} - w_{t \vee t_1} = (w_{r \vee t_1} - w_{t_1} )- (w_{t \vee t_1} - w_{t_1})$.
    %Let $(t,y,w) \in [0,T] \times (a,b) \times \mathcal{C}_0^0([0,T])$, $n \in \mathbb{N}$ and $t_1 \in [0,t]$. For all $r > \tau_{n}(t,y,w)$, we have . The construction of $\tau_{n+1}$ and $\epsilon_{n+1}$ is thus exactly the same.
%\end{proof}

\begin{lemma}
    \label{lem:tau_independence_future}
    For all $(t,s,y,w) \in [0,T] \times [0,T] \times (a,b) \times \mathcal{C}_0^0([0,T])$, $n \in \mathbb{N}$ and $t_2 \in [s, T]$,
    \begin{equation*}
        (\tau_n(t,y,w), \epsilon_n(t,y,w)) \mathds{1}_{\{\tau_n(t,y,w) \leqslant s\}} =
        (\tau_n(t,y,w_{\cdot \wedge t_2}), \epsilon_n(t,y,w_{\cdot \wedge t_2})) \mathds{1}_{\{\tau_n(t,y,w_{\cdot \wedge t_2}) \leqslant s\}}.
    \end{equation*}
\end{lemma}

\begin{comment}
\begin{proof}
    If $t_2 = T$, $w_{\cdot \wedge t_2} = w$ so there is nothing to prove. If $t_2 < t$, the indicators in the property are always 0 so the result follows immediately. Assume $t \leqslant t_2 < T$. The property is obviously true for $n = 0$ (setting an arbitrary constant value to $\epsilon_0$). Assume it is true for all indices until some $n \in \mathbb{N}$.

    Let $(t,s,y,w) \in [0,T] \times [0,T] \times (a,b) \times \mathcal{C}_0^0([0,T])$.
    
    If $\tau_n(t,y,w) > s$, then $\tau_n(t,y,w_{\cdot \wedge t_2}) > s$ (by assumption), thus $\tau_{n+1}(t,y,w), \tau_{n+1}(t,y,w_{\cdot \wedge t_2}) > s$ and the result follows at rank $n+1$. Suppose $\tau_n(t,y,w) \leqslant s$.

    % If $\tau_{n+1}(t,y,w) > s$, then $\mathds{1}_{\{\tau_{n+1}(t,y,w) \leqslant s\}} = 0$ and the desired equality is trivially verified at rank $n+1$. From now on, assume that $\tau_{n+1}(t,y,w) \leqslant s$ (thus $\leqslant t_2$).

    By assumption, for all $i \leqslant n$, $\tau_i(t,y,w) = \tau_i(t, y, w_{\cdot \wedge t_2})$ and $\epsilon_i(t,y,w) = \epsilon_i(t, y, w_{\cdot \wedge t_2})$ (and same for the $\epsilon'$). Hence,
    \begin{align*}
        &\tau_{n+1}(t,y,w_{\cdot \wedge t_2}) = \inf \left\{
            r \in (\tau_n(t,y,w),T] : y + (w_{r \wedge t_2} - w_t) + \sum_{i=1}^n \epsilon_i'(t,y,w)  \in \{a,b\}
        \right\} \wedge T \\
        &\tau_{n+1}(t,y,w_{\cdot \wedge t_2}) = \inf \left\{
            r \in (\tau_n(t,y,w),t_2] : y + (w_{r} - w_t) + \sum_{i=1}^n \epsilon_i'(t,y,w)  \in \{a,b\}
        \right\} \wedge T \\
        &\tau_{n+1}(t,y,w_{\cdot \wedge t_2}) \geqslant \tau_{n+1}(t,y,w), 
    \end{align*}
    the last inequality holding because the two sides are defined as an $\inf$ of a set, the left-hand side being over a smaller set. Thus, if $\tau_{n+1} (t,y,w) > s$, then $\tau_{n+1}(t,y,w_{\cdot \wedge t_2}) > s$ and the property is verified at rank $n+1$. Suppose $\tau_{n+1} (t,y,w) \leqslant s$.
    \begin{equation*}
        \forall r \in (\tau_n(t,y,w), \tau_{n+1}(t,y,w)),\ 
        y + (w_{r \wedge t_2} - w_t) + \sum_{i=1}^n \epsilon_i'(t,y,w)  \notin \{a,b\}
    \end{equation*}
    and $\tau_{n+1}(t,y,w),\ 
    y + (w_{r \wedge t_2} - w_t) + \sum_{i=1}^n \epsilon_i'(t,y,w)  \in \{a,b\}$ for $r = \tau_{n+1}(t,y,w) \leqslant s \leqslant t_2$. Thus, $\tau_{n+1}(t,y,w) = \tau_{n+1}(t,y,w_{\cdot \wedge t_2})$. By the definition of $\epsilon_{n+1}$, we have also that $\epsilon_{n+1}(t,y,w) = \epsilon_{n+1}(t,y,w_{\cdot \wedge t_2})$ and the property holds at rank $n+1$.
\end{proof}
\end{comment} 

The next lemma states that if the driving function is a Brownian motion with bounden drift, the number of jumps of $Y$ has finite moments. We already know that is is finite thanks to Lemma \ref{lem:finite_activity}.

\begin{lemma}
    \label{lem:finite_activity_expectation}
    Let $(t,y) \in [0,T] \times (a,b)$ and $(\Omega, \mathcal{F}, (\mathcal{F}_t)_{t \in [0,T]}, \mathbb{P})$ be a filtered probability space supporting an $(\mathcal{F}_t)$-Brownian motion $W$ on $[0,T]$ and $\beta_t$ a uniformly bounded stochastic process on this space. Let $\sigma > 0$, and define $\tilde{W}_t \defeq W_t + \int_0^t \beta_s \diff s$ and $N \defeq \inf\{n \in \mathbb{N}: \tau_i(t,y,\sigma\tilde{W}) = T\}$. Then $\mathbb{E}[N^k]<\infty$ for all $k \in \mathbb{N}^*$.
\end{lemma}

\begin{proof}
    \begin{description}
        \item[Case 1: $\beta \equiv 0$.] %Let $(\Omega', \mathcal{F}', (\mathcal{F}'_t)_{t \in [0,2T]}, \mathbb{P}')$ be a probability space supporting a Brownian motion $W'$ on $[0,2T]$. We denote its restriction to $[0,T]$ again by $W'$. 
        Let $n \in \mathbb{N}$. For concision, we write $\tau_n \defeq \tau_n(t,y,\sigma W)$.
        %Then $\mathbb{P}(\tau_n(t,y,\sigma W)<T) = \mathbb{P}'(\tau_n(t,y,\sigma W')<T)$. For concision, we define $\tau_n' \defeq \tau_n(t,y,\sigma W')$. 
        By the fourth property of $\tau_i$ in \eqref{eq:appA1},
        \begin{align*}
            \mathbb{P}(\tau_{n+1} < T)
            & \leqslant \mathbb{E}\left[\mathds{1}_{\{\tau_n < T\}}\mathbb{P}\left(\left.\sup\limits_{s \in[\tau_n,T]} \sigma|W_s - W_{\tau_n}| \geqslant \Delta\right| \mathcal{F}_{\tau_n} \right)\right] \\
           % & \leqslant \mathbb{E}'\left[\mathds{1}_{\{\tau_n' < T\}}\mathbb{P}'\left(\left.\sup\limits_{s \in[0,T]} \sigma|W'_{s + \tau_n'} - W'_{\tau_n'}| \geqslant \Delta\right| \mathcal{F}'_{\tau_n'} \right)\right]\\
            & \leqslant \mathbb{E}\left[\mathds{1}_{\{\tau_n < T\}}\mathbb{P}\left(\sup\limits_{s \in[0,T]} \sigma|W_{s}| \geqslant \Delta \right)\right]\\
            & = a \mathbb{P}(\tau_{n}< T)
        \end{align*}
        where $a \defeq \mathbb{P}\left(\sup\limits_{s \in[0,T]} \sigma|W_{s}| \geqslant \Delta \right) < 1$. The second inequality is a consequence of  the strong independence and invariance property of the Brownian motion. 
        
        By induction, we have for all $n \in \mathbb{N}$,
        $\mathbb{P}(\tau_{n}< T) \leqslant a^{n}$. Let $k \in \mathbb{N}^*$. We conclude
        \begin{equation*}
            \mathbb{E}[N^k] = \sum_{n=0}^{\infty} \mathbb{P}(N^k > n) = \sum_{n=0}^{\infty} \mathbb{P}\left(\tau_{\lfloor n^{\frac{1}{k}} \rfloor}< T\right) \leqslant \sum_{n=0}^{\infty} a^{\lfloor n^{\frac{1}{k}}\rfloor} < \infty.
        \end{equation*}
        \item[Case 2: general $\beta$.] Let $m > 0$ be a constant such that $|\beta| \leqslant m$. Define
        \begin{equation*}
            Z \defeq \exp\left(-\int_0^T \beta_s \diff W_s - \frac{1}{2} \int_0^T \beta_s^2 \diff s\right).
        \end{equation*}
        Since $\beta$ is bounded, by Novikov's criterion $\frac{\diff \mathbb{Q}}{\diff \mathbb{P}} = Z$ defines a change of probability. By Girsanov's theorem, $\tilde{W}$ is a brownian motion under $\mathbb{Q}$. Let $n \in \mathbb{N}$. Using the Cauchy-Schwarz inequality and the results from case 1,
        \begin{align*}
            \mathbb{P}\left(\tau_n(t,y,\sigma \tilde{W}) < T\right)
            &= \E[\mathbb{Q}][Z^{-1} \mathds{1}_{\{\tau_n(t,y,\sigma \tilde{W}) < T\}}]   \\
            &\leqslant  \sqrt{\E[\mathbb{Q}][Z^{-2}]}  \sqrt{\mathbb{Q}(\tau_n(t,y,\sigma \tilde{W}) < T)}\\
           % \mathbb{P}\left(\tau_n(t,y,\sigma \tilde{W}) < T\right)
           % &\leqslant  (\sqrt{a})^n\exp\left(\int_0^T \beta_s^2 \diff s\right)\sqrt{\E[\mathbb{Q}][\exp\left(\int_0^T \beta_s \diff W_s - \frac12\int_0^T \beta_s^2 \diff s\right)]}\\
            &\leqslant  (\sqrt{a})^n\exp\left(\frac12 m^2T\right).
        \end{align*}
        We conclude in the same way as in case 1.
    \end{description}
\end{proof}

\subsection{Properties of the mid-price function}

\begin{lemma}
    \label{lem:y_independence}
    For all $(t,s,y,w) \in [0,T] \times [0,T] \times (a,b) \times \mathcal{C}_0^0([0,T])$, $t_1 \leqslant t$ and $t_2 \geqslant s$,
    \begin{equation*}
        Y(t,s,y,w) = Y \left(t,s,y,\left(w_{(u\wedge t_2) \vee t_1} - w_{t_1}\right)_{u \in [0,T]}\right).
    \end{equation*}
\end{lemma}

\begin{proof}
    This is a direct consequence of Lemmas \ref{lem:tau_independence_past} and \ref{lem:tau_independence_future} and the definition of $Y$ \eqref{eq:y}.
\end{proof}

This lemma implies the following proposition regarding the adaptedness of the process $Y$, which is used throughout our study.
\begin{proposition}\label{prop:Yadapted}
    Let $(t, y) \in [0,T] \times (a,b)$. Let $(\Omega, \mathcal{F})$ a measurable space and $X = (X_r)_{r \in [0,T]}$ a process on this space. Let $\mathcal{F}^{X,t}$ the natural filtration of $(X_{r \vee t} - X_t)_{r \in [0,T]}$. Then, $Y(t,\cdot,y,X)$ is a $\mathcal{F}^{X, t}$-adapted process.
\end{proposition}