In this section, we derive some properties of continuous concave functions, their integrals and the continuous concave enveloppes. We use the notations from Section \ref{sec:uniqueness}.

\subsection{Some lemmas about concave functions}

\begin{lemma}
    \label{lem:basic_max_concave}
    Let $I$ be a compact interval and $\phi$ be a real-valued continuous concave function defined on $I$.\\
    (i) Let $q^* \in I$ be such that $\phi(q^*) = \max \phi$. Then, for all  $z \in \mathbb{R}_+$, $\phi\left(q^* \wedge z\right) = \max_{q \in I} \phi(q \wedge z)$.\\
    (ii) Let $q \in I$ be such that $\phi(q) < \max \phi$. Then, for all $ z \geqslant \max I$, $\phi\left(q \wedge z\right) < \max \phi(\cdot \wedge z)$.
\end{lemma}

\begin{proof}
    Point (ii) is immediate since for $q\in I$ and $z \geqslant \max I$, $q\wedge z = q$. We now show (i). Let $z \in \mathbb{R}_+$. If $z \geqslant q^*$, $\phi(q^* \wedge z) = \phi(q^*) = \max \phi \geqslant \max \phi(\cdot \wedge z)$. Suppose that $z < q^*$. Since $\phi$ is increasing on $[0, q^*]$, which contains $[0,z]$, we have $\max \phi(\cdot \wedge z) = \phi(z) = \phi(q^* \wedge z)$.
\end{proof}

We have the following corollary regarding the maximization of an integral by maximizing the integrand. 
\begin{corollary}
    \label{corol:max_integral}
    Let $Q \in \Q$ and $i \in \{a,b\}$. Define $\varepsilon \defeq -1$ if $i = b$ and $\varepsilon \defeq 1$ if $i = a$.
    Let $\phi: \Qi \cap [0, \varepsilon Q + \Qmax] \to \mathbb{R}$ be a continuous concave function. Let $F:\mathbb{R} \to \mathbb{R}$ be a strictly increasing function.\\
    (i) Let $q^* \in \argmax \phi$. Then $q^* \in \argmax\limits_{q \in \Qi \cap [0, \varepsilon Q +\Qmax]} \int_{\Qi} F(\phi(q \wedge z)) \mu^i(\diff z)$.\\
    (ii) Let $\hat{q} \in [0, \varepsilon Q +\Qmax] \setminus \argmax \phi$. Then $\hat{q} \notin \argmax\limits_{q \in \Qi \cap [0, \varepsilon Q +\Qmax]} \int_{\Qi} F(\phi(q \wedge z)) \mu^i(\diff z)$.
\end{corollary}

\begin{proof}
    Point (i) is an immediate consequence of Lemma \ref{lem:basic_max_concave} (i), since the maximum holds pointwise.
    
    We now prove (ii). By Lemma \ref{lem:basic_max_concave}, since $\phi$ is continuous, there exists $\eta > 0$ such that for all $z \in (\qi - \eta, \qi]$, $\phi\left(\hat{q} \wedge z\right) < \max \phi(\cdot \wedge z) = \phi(q^* \wedge z)$. Thus, since $\qi$ is in the support of $\mu^i$,
    \begin{equation}\label{eq:appF1}
        \int_{\Qi \cap (\qi - \eta, \qi]} F(\phi(\hat{q} \wedge z)) \mu^i(\diff z) < \int_{\Qi \cap (\qi - \eta, \qi]} F(\phi(q^* \wedge z)) \mu^i(\diff z).
    \end{equation}
    In addition, by Lemma \ref{lem:basic_max_concave} (i), and since the inequality holds pointwise,  we have
    \begin{equation}\label{eq:appF2}
        \int_{\Qi \cap [0,\qi - \eta]} F(\phi(\hat{q} \wedge z)) \mu^i(\diff z) \leqslant \int_{\Qi \cap [0,\qi - \eta]} F(\phi(q^* \wedge z)) \mu^i(\diff z).
    \end{equation}
    The inequalities \eqref{eq:appF1} and \eqref{eq:appF2} yield the conclusion.
\end{proof}

Recall the definition of the log-Hamiltonians $h^a,h^b$ in \eqref{eq:deflogHamiltonian}.
\begin{lemma}
    \label{lem:same_maximizer}
    Let $k \in \mathbb{R}$, $Q \in \Q$ and $g:\Q \to \mathbb{R}$ be a continuous concave function.\\
    (i) Define $Q^* \defeq \inf \left\{R \in \Q \setminus \{\Qmax\}: g'_+(R) \leqslant k\right\}\wedge \Qmax$.\\
    If $Q^* > Q$, then $h^a(g,k,Q) = -1$.\\
    If $Q^* \leqslant Q$, then $h^a(g,k,Q) = \int_{\Qa} - e^{-k\left((Q-Q^*) \wedge z\right) - g\left(Q - (Q-Q^*)\wedge z\right) + g(Q)}\mu^a(\diff z)$.\\
    (ii) Define $Q^* \defeq \inf \left\{R \in \Q \setminus \{-\Qmax\}: g'_-(R) \leqslant -k\right\}\vee (-\Qmax)$.\\
    If $Q^* < Q$, then $h^b(g,k,Q) = -1$.\\
    If $Q^* \geqslant Q$, then $h^b(g,k,Q) = \int_{\Qa} - e^{-k\left((Q^*-Q) \wedge z\right) - g\left(Q - (Q^*-Q)\wedge z\right) + g(Q)}\mu^b(\diff z)$.
\end{lemma}

\begin{proof}
    We only prove (i) since (ii) is similar.
    Let $\phi: q \in [Q - \Qmax, Q + \Qmax] \mapsto kq + g(Q- q)$. The function $\phi$ is continuous and concave, and for all $q$, $\phi'_-(q) = k - g'_+(Q - q)$. We have
    \begin{equation*}
        \sup\left\{q : \phi'_{-}(q) \geqslant 0\right\} \wedge (Q + \Qmax) = Q - Q^*
    \end{equation*}
    therefore $Q - Q^*$ maximizes $\phi$.

    If $Q^* > Q$, then $\phi$ is nonincreasing on $[0, Q + \Qmax]$ and $\phi(0) = g(Q) = \max\limits_{\Qa \cap [0,Q + \Qmax]}\phi$. Corollary \ref{corol:max_integral} allows us to conclude.

    If $Q^* \leqslant Q$ and $Q - Q^* \leqslant \qa$, then $\phi(Q- Q^*) = \max\limits_{\Qa \cap [0,Q + \Qmax]}\phi$. The conclusion follows from Corollary \ref{corol:max_integral}

    If $Q^* \leqslant Q$ and $Q - Q^* > \qa$, then $\phi(\qa) = \max\limits_{\Qa \cap [0,Q + \Qmax]}\phi$ and for all $ z \in \Qa$, $(Q-Q^*)\wedge z = z = \qa \wedge z$. By Corollary \ref{corol:max_integral},
    \begin{equation*}
        h^a(g,k,Q) =\int_{\Qa} - e^{-k\left(\qa \wedge z\right) - g\left(Q - \qa\wedge z\right) + g(Q)}\mu^a(\diff z) = \int_{\Qa} - e^{-k\left((Q-Q^*) \wedge z\right) - g\left(Q - (Q-Q^*)\wedge z\right) + g(Q)}\mu^a(\diff z).
    \end{equation*}
\end{proof}

The following lemma shows that over the intervals where $g$ is affine, the log-Hamiltonians $h^a$ and $h^b$ are concave. In addition, it provides sufficient conditions to have a strict concavity inequality.
\begin{lemma}
    \label{lem:bid_ask_max}
    Let $k \in \mathbb{R}$ and $g: \Q \to \mathbb{R}$ be a continuous concave function.\\
    Suppose that for $(Q,Q') \in \Q \times \Q$ and $\lambda \in [0,1]$, $(1-\lambda)g(Q) + \lambda g(Q') = g((1-\lambda)Q + \lambda Q')$. Then,\\
    (ia) $(1-\lambda) h^a(g,k,Q) + \lambda h^a(g,k,Q') \leqslant h^a(g,k,(1-\lambda)Q + \lambda Q')$.\\
    (ib) $(1-\lambda) h^b(g,k,Q) + \lambda h^b(g,k,Q') \leqslant h^b(g,k,(1-\lambda)Q + \lambda Q')$.\\
    (iia) If $\lambda \notin \{0,1\}$, $Q < Q'$, $p \defeq \frac{g(Q')-g(Q)}{Q'-Q} < k$, and
    $g'_+(R) > p$ for all $ R \in \Q \cap (-\infty, Q)$, then\\
    $(1-\lambda) h^a(g,k,Q) + \lambda h^a(g,k,Q') < h^a(g,k,(1-\lambda)Q + \lambda Q')$.\\
    (iib) If $\lambda \notin \{0,1\}$, $Q < Q'$, $p\defeq\frac{g(Q')-g(Q)}{Q'-Q} > -k$, and
    $g'_-(R) < p$ for all $R \in \Q \cap (Q, \infty)$, then\\
    $(1-\lambda) h^b(g,k,Q) + \lambda h^b(g,k,Q') < h^b(g,k,(1-\lambda)Q + \lambda Q')$.
\end{lemma}

\begin{rem}
    The hypothesis in $(iia)$ states that, while $g$ is affine on $[Q,Q']$, it is not affine on $[R,Q']$ for any $R < Q$.
\end{rem}

\begin{proof}
    We only prove parts (ia) and (iia), since parts (ib) and (iib) are similar. Without loss of generality, suppose $Q < Q'$. Let $Q^* \defeq \inf \left\{R \in \Q \setminus \{\Qmax\}: g'_+(R) \leqslant k\right\}\wedge \Qmax$. Since $g'_+$ is constant on $[Q,Q')$ and equal to $p$, then $Q^* \in \Q \setminus (Q,Q')$. For the rest of the proof we write $\tilde{Q} \defeq (1-\lambda)Q + \lambda Q'.$

    We first prove point (ia). If $Q^* \geqslant Q'$, by Lemma \ref{lem:same_maximizer}, $h^a(g,k,Q) = h^a(g,k,Q') = h^a(g,k,\tilde{Q})=-1$ and the result follows immediately.\\
    Suppose that $Q^* \leqslant Q$. Let $z \in \Qa$. Since the function $x \mapsto -e^{-x}$ is strictly increasing and strictly concave, and since $g$ is concave and affine between $Q$ and $Q'$, 
    \begin{equation}
        \label{eq:pointwise_concavity}
        \begin{split}
            -(1-\lambda)&e^{-k((Q-Q^*)\wedge z) - g(Q - (Q-Q^*)\wedge z) + g(Q)}
        - \lambda e^{-k((Q'-Q^*)\wedge z) - g(Q' - (Q'-Q^*)\wedge z) + g(Q')}\\
        &\leqslant -e^{-k\left[(1-\lambda)((Q - Q^*) \wedge z) + \lambda((Q' - Q^*) \wedge z)\right] - (1- \lambda)g(Q - (Q-Q^*)\wedge z) - \lambda g(Q' - (Q'-Q^*)\wedge z) + (1-\lambda)g(Q) + \lambda g(Q')}\\
        &\leqslant
        -e^{-k\left[(1-\lambda)((Q - Q^*) \wedge z) + \lambda((Q' - Q^*) \wedge z)\right] - g\left(\tilde{Q} - (1-\lambda)((Q-Q^*)\wedge z) - \lambda((Q'-Q^*)\wedge z)\right) + g(\tilde{Q})}.
        \end{split}
    \end{equation}
    We define the function $\phi:q \in [0,\tilde{Q} + \Qmax] \mapsto kq + g(\tilde{Q} - q)$. Reasoning as in the proof of Lemma \ref{lem:same_maximizer}, we have that $\phi$ is concave and maximized at $\tilde{Q} - Q^*$ and is therefore nondecreasing on $[0, \tilde{Q} - Q^*]$. Since, by the concavity of $(\cdot \wedge z)$, we have
    \begin{equation*}
        (1-\lambda)((Q-Q^*)\wedge z) + \lambda ((Q' - Q^*) \wedge z) \leqslant \left(\tilde{Q} - Q^*\right) \wedge z \leqslant \tilde{Q}-Q^*,
    \end{equation*}
    the inequality \eqref{eq:pointwise_concavity} becomes
    \begin{equation}
        \label{eq:pointwise_concavity_end}
        \begin{split}
            -(1-\lambda)e^{-k((Q-Q^*)\wedge z) - g(Q - (Q-Q^*)\wedge z) + g(Q)}
        &- \lambda e^{-k((Q'-Q^*)\wedge z) - g(Q' - (Q'-Q^*)\wedge z) + g(Q')}\\
        &\leqslant
        -e^{-k((\tilde{Q} - Q^*) \wedge z) - g\left(\tilde{Q} - (\tilde{Q}-Q^*)\wedge z\right) + g(\tilde{Q})}.
        \end{split}
    \end{equation}
    Thanks to Lemma \ref{lem:same_maximizer}, the result follows by integrating \eqref{eq:pointwise_concavity_end} with respect $\mu^a(\diff z)$.

    We now prove point (iia). Since $g_+'$ is nonincreasing and $g_+'(Q) = p < k$, we have that $Q^* \leqslant Q$. Suppose that
    \begin{equation}
        \label{eq:different_gains}
        \begin{split}
            k\left((Q-Q^*)\wedge \qa\right) +& g\left(Q - (Q-Q^*) \wedge \qa\right) - g(Q)\\ &<
        k\left((Q'-Q^*)\wedge \qa\right) + g\left(Q' - (Q'-Q^*) \wedge \qa\right) - g(Q').
        \end{split}
    \end{equation}
    Then, since $x \mapsto -e^{-x}$ is strictly concave, the first inequality in \eqref{eq:pointwise_concavity} is strict with $z = \qa$, and consequently so is the one in \eqref{eq:pointwise_concavity_end} (with $z = \qa$). By the continuity of all the functions involved, strict inequality in \eqref{eq:pointwise_concavity_end} holds for $z \in (\qa - \eta ,\qa]$ for some $\eta > 0$. We have the desired result because $\mu^a((\qa - \eta, \qa]) > 0$.
    
    It only remains to show that \eqref{eq:different_gains} is true. Considering the function $q \in [0,Q + \Qmax] \mapsto g(Q-q)$ which is concave and has left derivative $-g_+'(Q - \cdot)$, \parencite[][Proposition 1.6.1]{niculescu_convex_2005} yields
    \begin{equation}\label{eq:Niculescu1}
        k\left((Q-Q^*)\wedge \qa\right) + g\left(Q - (Q-Q^*) \wedge \qa\right) - g(Q) = \int_0^{(Q-Q^*)\wedge \qa}\left(-g_+'(Q - q) + k\right) \diff q
    \end{equation}
    and, similarly,
    \begin{equation}\label{eq:Niculescu2}
        k\left((Q'-Q^*)\wedge \qa\right) + g\left(Q' - (Q'-Q^*) \wedge \qa\right) - g(Q') = \int_0^{(Q'-Q^*)\wedge \qa}\left(-g_+'(Q' - q) + k\right) \diff q.
    \end{equation}
    By the definition of $Q^*$, the terms in the integrals above  are strictly positive. If $Q = Q^*$, \eqref{eq:different_gains} follows from the hypothesis $p<k$ and the fact that $g$ is affine between $Q$ and $Q'$. Assume now that $Q^* < Q$. For $q \in [0, (Q-Q^*)\wedge \qa]$, $-g'_+(Q-q) \leqslant -g'_+(Q'-q)$ since $g'_+$ is nonincreasing. Let $\epsilon > 0$ be such that $\epsilon < (Q-Q^*)\wedge \qa$ and $\epsilon < (Q'-Q)\wedge \qa$. We have
    \begin{equation*}
        \begin{split}
            \int_0^{(Q'-Q^*)\wedge \qa}\left(-g_+'(Q' - q) + k\right) \diff q
        = &\int_0^{\epsilon}\left(-g_+'(Q' - q) + k\right) \diff q
        + \int_{\epsilon}^{(Q-Q^*)\wedge \qa}\left(-g_+'(Q' - q) + k\right) \diff q\\
        &+ \int_{(Q-Q^*)\wedge \qa}^{(Q'-Q^*)\wedge \qa}\left(-g_+'(Q' - q) + k\right) \diff q.
        \end{split}
    \end{equation*}
    For $q \in (0,\epsilon]$, by the condition on $Q$, $g'_+(Q-q) > p = g'_+(Q'-q)$. This implies that the first integral is strictly greater that $\int_0^{\epsilon}\left(-g_+'(Q - q) + k\right) \diff q$. The second integral is greater than or equal to $\int_{\epsilon}^{(Q-Q^*)\wedge \qa}\left(-g_+'(Q - q) + k\right) \diff q$ since $g_+'$ is nonincreasing. The third one is nonnegative by the definition of $Q^*$. The result follows from \eqref{eq:Niculescu1} and \eqref{eq:Niculescu2}.
\end{proof}

\subsection{Some lemmas about the continuous concave envelope}

For a continuous function $g:\Q \to \mathbb{R}$, $\tilde{Q} \in \Q$ and $(Q, Q') \in [-\Qmax,\tilde{Q}] \times [\tilde{Q}, \Qmax]$, we define the quantity
\begin{equation*}
    A_{g, \tilde{Q}}(Q,Q') \defeq g\left(\tilde{Q}\right) - \frac{Q'-\tilde{Q}}{Q'-Q} g\left(Q\right) - \frac{\tilde{Q} - Q}{Q' - Q} g\left(Q'\right)
\end{equation*}
if $Q' > Q$ and $A_{g, \tilde{Q}}(Q,Q') = 0$ otherwise. It is clear that $A_{g,\tilde{Q}}$ is continuous and $\min A_{g, \tilde{Q}} \leqslant 0$. Recall also the definition of $C_g$ in \eqref{eq:defCg}.

\begin{lemma}
    \label{lem:envelope_shape}
    Let $g:\Q \to \mathbb{R}$ be a continuous function and $\tilde{Q} \in \Q$.\\
    (i) If $\min A_{g, \tilde{Q}} = 0$, then $\hat{g}\left(\tilde{Q}\right) = g \left(\tilde{Q}\right)$.\\
    (ii) If $\min A_{g, \tilde{Q}} < 0$ and $(Q, Q') \in [-\Qmax,\tilde{Q}] \times [\tilde{Q}, \Qmax]$ minimizes $A_{g, \tilde{Q}}$, then $\hat{g}(Q) = g(Q)$, $\hat{g}(Q') = g(Q')$, $C_{\hat{g}}\left(Q,Q', \frac{\tilde{Q}-Q}{Q'-Q}\right) = 0 = \min C_{\hat{g}}$ and
    \begin{equation*}
        \hat{g}\left(\tilde{Q}\right) - g\left(\tilde{Q}\right) = -g\left(\tilde{Q}\right) + \frac{Q'-\tilde{Q}}{Q'-Q}g(Q) + \frac{\tilde{Q} - Q}{Q'-Q}g(Q') = -C_{g}\left(Q,Q',\frac{\tilde{Q}-Q}{Q'-Q}\right).
    \end{equation*}
\end{lemma}
\begin{proof}
    (i) Suppose $\min A_{g, \tilde{Q}} = 0$. Let $\epsilon > 0$ and define the functions
    \begin{equation*}
        \begin{array}[b]{rccl}
            r:&\mathbb{R} & \to & \mathbb{R}\\
            & a & \mapsto & \min\limits_{R \in [\tilde{Q}, \Qmax]}
            \left(\epsilon + g(\tilde{Q}) + a (R-\tilde{Q}) - g(R) \right)
        \end{array}
    \end{equation*}
    and
    \begin{equation*}
        \begin{array}[b]{rccl}
            l:&\mathbb{R} & \to & \mathbb{R}\\
            & a & \mapsto & \min\limits_{R \in [-\Qmax, \tilde{Q}]}
            \left(\epsilon + g(\tilde{Q}) + a (R-\tilde{Q}) - g(R) \right)
        \end{array}.
    \end{equation*}
    It is sufficient to prove that there exists $a \in \mathbb{R}$ such that $r(a) \geqslant 0$ and $l(a) \geqslant 0$. Indeed, this would imply that the affine function $\phi:R \mapsto \epsilon + g(\tilde{Q}) + a (R-\tilde{Q})$ is greater or equal than $g$ and thus that $g(\tilde{Q}) \leqslant \hat{g}(\tilde{Q}) \leqslant \phi(\tilde{Q}) = g(\tilde{Q}) + \epsilon$.
    
We have that for $a \in \mathbb{R}$, $l(a) \leqslant 0$ implies $r(a) > 0$ and $r(a) \leqslant 0$ implies $l(a)> 0$, otherwise we would have the existence of $Q < \tilde{Q} < Q'$ such that $A_{g,\tilde{Q}}(Q,Q') \leqslant -\epsilon$, contradicting the main assumption. If $r(0) \geqslant 0$ and $l(0) \geqslant 0$, there is nothing to prove. 

Without loss of generality, suppose that $r(0)<0$ (and thus $l(0)>0$). By the continuity of $g$, there exists $\eta > 0$ such that $g < g(\tilde{Q}) + \epsilon$ on $[\tilde{Q},\tilde{Q}+\eta]$. Let $a > 0$, then
\begin{equation*}
    \min\limits_{R \in [\tilde{Q}, \tilde{Q} + \eta]}
            \left(\epsilon + g(\tilde{Q}) + a (R-\tilde{Q}) - g(R) \right) \geqslant 0
\end{equation*}
and
\begin{equation*}
    \min\limits_{R \in [\tilde{Q} + \eta, \Qmax]}
            \left(\epsilon + g(\tilde{Q}) + a (R-\tilde{Q}) - g(R) \right) \geqslant 
            \left(\epsilon + g(\tilde{Q}) + a \eta - \max g \right) \xrightarrow[a \to \infty]{}\infty.
\end{equation*}
Therefore, there exists $a$ such that $r(a)>0$. Since $r$ is continuous, there exists $a > 0$ such that $r(a) = 0$ (and therefore $l(a) > 0$) which is what we wanted to prove.

(ii) Suppose that $\min A_{g, \tilde{Q}} < 0$ and that $(Q, Q') \in [-\Qmax,\tilde{Q}] \times [\tilde{Q}, \Qmax]$ minimizes $A_{g, \tilde{Q}}$. Since $A_{g, \tilde{Q}}(Q,Q') < 0$, $Q < \tilde{Q} < Q'$. Define $\lambda \defeq \frac{\tilde{Q} - Q}{Q' - Q}$ and $\phi:R \in \Q \mapsto \frac{g(Q')-g(Q)}{Q'-Q}(R-Q) + g(Q)$.

Suppose that $\phi \geqslant g$. Then, $\hat{g} \geqslant \hat{g} \wedge \phi \geqslant g$ and, because $\hat{g} \wedge \phi$ is concave, $\hat{g} = \hat{g} \wedge \phi$. Furthermore, since $\phi(Q) = g(Q)$ and $\phi(Q') = g(Q')$, then $g(Q) = \hat{g}(Q)$ and $g(Q') = \hat{g}(Q')$. In addition,
\begin{equation*}
    0 \leqslant C_{\hat{g}}(Q,Q',\lambda) = \hat{g}(\tilde{Q}) - (1-\lambda)\phi(Q) - \lambda \phi(Q') \leqslant \phi(\tilde{Q})- (1-\lambda)\phi(Q) - \lambda \phi(Q') = 0,
\end{equation*}
the last equality coming from the fact that $\phi$ is affine. Hence, all the above inequalities are equalities and, in particular, $\hat{g}(\tilde{Q}) = \phi(\tilde{Q})$. Thus,
\begin{equation*}
    (\hat{g} - g)(\tilde{Q}) = \phi(\tilde{Q}) - g(\tilde{Q}) = \frac{g(Q')-g(Q)}{Q'-Q}(\tilde{Q}-Q) + g(Q) - g(\tilde{Q}) = -g(\tilde{Q}) + (1-\lambda)g(Q) + \lambda g(Q')
\end{equation*}
which is the desired result.

It remains to show that $\phi \geqslant g$. Let $R \in \Q \setminus \{Q,Q'\}$. Suppose that $R \leqslant \tilde{Q}$ (the other case is treated similarly). By the optimality of $(Q,Q')$, we have $A_{g, \tilde{Q}}(Q,Q')\leqslant A_{g, \tilde{Q}}(R,Q')$, which, by straightforward computations, is equivalent to $g(R)\leqslant\phi(R)$. 
\begin{comment}
\begin{align*}
    A_{g, \tilde{Q}}(Q,Q') &\leqslant A_{g, \tilde{Q}}(R,Q') \\
    g(\tilde{Q}) - \frac{Q'-\tilde{Q}}{Q'-Q}g(Q) - \frac{\tilde{Q} - Q}{Q'-Q}g(Q')
    &\leqslant
    g(\tilde{Q}) - \frac{Q'-\tilde{Q}}{Q'-R}g(R) - \frac{\tilde{Q} - R}{Q'-R}g(Q')\\
    \frac{Q' - \tilde{Q}}{Q'-R} g(R)
    &\leqslant
    \frac{Q'-\tilde{Q}}{Q'-Q}g(Q) + \frac{(\tilde{Q} - Q')(Q-R)}{(Q'-Q)(Q'-R)}g(Q')\\
    g(R) &\leqslant
    \frac{Q'-R}{Q'-Q}g(Q) + \frac{R-Q}{Q'-Q}g(Q')\\
    g(R) &\leqslant
    \frac{g(Q')-g(Q)}{Q'-Q}(R-Q) + \frac{Q(g(Q')-g(Q)) + Q'g(Q) -Qg(Q')}{Q'-Q}\\
    g(R) &\leqslant
    \frac{g(Q')-g(Q)}{Q'-Q}(R-Q) + g(Q) = \phi(R).
\end{align*}
\end{comment}
\end{proof}

\begin{corollary}
    \label{corol:envelope_shape}
    Let $g:\Q \to \mathbb{R}$ be a continuous function. Suppose that $(Q,Q', \lambda) \in \Q \times \Q \times [0,1]$ minimizes $C_g$. Then, $g(Q) = \hat{g}(Q)$, $g(Q') = \hat{g}(Q')$, $C_{\hat{g}}(Q,Q',\lambda)=0$ and
    $(\hat{g} - g)((1-\lambda)Q + \lambda Q') = \max (\hat{g} - g)$.
\end{corollary}

\begin{proof}
    Assume that $\min C_g < 0$.
    Define $\tilde{Q} \defeq (1-\lambda)Q + \lambda Q'$ and suppose, without loss of generality, that $Q < Q'$. We have that $(Q,Q')$ minimizes $A_{g,\tilde{Q}}$ and the results follow from Lemma \ref{lem:envelope_shape}.

    If $\min C_g = 0$, then $g$ is concave and, since $\hat{g} = g$, the result is immediate.
\end{proof}

\begin{lemma}
    \label{lem:envelope_inequality}
    Let $g:\Q \to \mathbb{R}$ be a continuous function and $k \in \mathbb{R}$. Suppose that $(Q,Q', \lambda) \in \Q \times \Q \times [0,1]$ minimizes $C_g$. Then,
    \begin{equation*}
        \begin{split}
            (i)\ h^a(g,k,(1-\lambda)Q + \lambda Q') &- (1-\lambda)h^a(g,k,Q) - \lambda h^a(g,k,Q') \\&\geqslant h^a(\hat{g},k,(1-\lambda)Q + \lambda Q') - (1-\lambda)h^a(\hat{g},k,Q) - \lambda h^a(\hat{g},k,Q')
        \end{split}
    \end{equation*}
    and
    \begin{equation*}
        \begin{split}
            (ii)\ h^b(g,k,(1-\lambda)Q + \lambda Q') &- (1-\lambda)h^b(g,k,Q) - \lambda h^b(g,k,Q') \\&\geqslant h^b(\hat{g},k,(1-\lambda)Q + \lambda Q') - (1-\lambda)h^b(\hat{g},k,Q) - \lambda h^b(\hat{g},k,Q').
        \end{split}
    \end{equation*}
\end{lemma}

\begin{proof}
    We only prove (i) since (ii) is similar. Let $q \in \Qa \cap [0, Q + \Qmax]$ and $z \in \Qa$. By Corollary \ref{corol:envelope_shape} $g(Q) = \hat{g}(Q)$. In addition, $g(Q - q \wedge z) \leqslant \hat{g}(Q - q \wedge z)$, hence
    \begin{equation*}
        \int_{\Qa}-e^{-k(q\wedge z) - g(Q - q \wedge z) + g(Q)}\mu^a(\diff z)
        \leqslant
        \int_{\Qa}-e^{-k(q\wedge z) - \hat{g}(Q - q \wedge z) + \hat{q}(Q)}\mu^a(\diff z) \leqslant h^a(\hat{g},k,Q).
    \end{equation*}
    Taking the supremum over $q$, we obtain $h^a(g,k,Q) \leqslant h^a(\hat{g},k,Q)$. Similarly, $h^a(g,k,Q') \leqslant h^a(\hat{g},k,Q')$.
    
    Let $q \in \Qa \cap [0, (1-\lambda)Q + \lambda Q' + \Qmax]$ and $z \in \Qa$. By Corollary \ref{corol:envelope_shape},
    \begin{equation*}
        (\hat{g} - g)((1-\lambda)Q + \lambda Q') \geqslant (\hat{g} - g)((1-\lambda)Q + \lambda Q' - q \wedge z)
    \end{equation*}
    therefore
    \begin{equation*}
        \hat{g}((1-\lambda)Q + \lambda Q') - \hat{g}((1-\lambda)Q + \lambda Q' - q \wedge z) \geqslant
        g((1-\lambda)Q + \lambda Q') - g((1-\lambda)Q + \lambda Q' - q \wedge z). 
    \end{equation*}
    Reasoning as before, we deduce that $h^a(\hat{g}, k, (1-\lambda)Q + \lambda Q') \leqslant h^a(g, k, (1-\lambda)Q + \lambda Q')$, which yields the conclusion.
\end{proof}
