We work on a nonempty interval $(a,b)$. Let $T > 0$ and $a_0, b_0 \in (a,b)$. We define, for this section $\D \defeq C([0,T] \times [a,b]) \cap C^{1,2}([0,T) \times (a,b))$.

We shall use some results from \parencite{friedman_partial_1983}. To better suit our framework, we work with final conditions instead of initial ones. All results remain the same, by using the transformation $t \mapsto T-t$, with the time derivatives $\partial_t$ being replaced by $-\partial_t$.

Let $\rho_1, \rho_2 \in (0,\infty)$. Let $\alpha$, $\beta$, $\gamma$, $f$ be real-valued functions defined on $[0,T) \times (a,b)$ and $u_T$ be a real-valued function defined on $[a,b]$. We are interested in equations, for $u \in \D$, of the form
\begin{equation}
    \label{eq:linear_equation_general}
    \left\{
        \begin{array}{rl}
            \partial_t u(t,y) + \alpha(t,y)\partial^2_{yy}u(t,y)+ \beta(t,y)\partial_y u (t,y) + \gamma(t,y) u(t,y) = f(t,y) & \text{on } [0,T) \times (a,b) \\
            u(T,y) = u_T(y,) & y \in [a,b]\\
            u(t,a) = \rho_1 u(t,a_0), & t \in [0,T]\\
            u(t,b) = \rho_2 u(t,b_0), &  
            t \in [0,T].
        \end{array}
    \right.
\end{equation}

The following lemma follows from straightforward computations.
\begin{lemma}
    \label{lem:equivalence_linear_pde}
    Let $u \in \D$. Let $\psi:[a,b]\to (0, \infty)$ be a function in $C^2$. Let $m \in \mathbb{R}$, and define $\phi(t,y):=e^{m(t-T)}\psi (y)$ and $v \defeq \phi u \in \D$. Then, $u$ solves \eqref{eq:linear_equation_general} if and only if $v$ solves 
    \begin{equation}
        \label{eq:linear_equation_modified}
        \left\{
        \begin{array}{rl}
            \partial_t v(t,y) + \alpha(t,y)\partial^2_{yy}v(t,y)+ \tilde{\beta}(t,y)\partial_y v (t,y) + \tilde{\gamma}(t,y) v(t,y) = \tilde{f}(t,y) & \text{on } [0,T) \times (a,b) \\
            v(T,y) = \tilde{u}_T(y), &
            y \in [a,b]\\
            v(t,a) = \tilde{\rho}_1 v(t,a_0), &  t \in [0,T]\\
            v(t,b) = \tilde{\rho}_2 v(t,b_0), & t \in [0,T]
        \end{array}
        \right.
    \end{equation}
    with
    \begin{align*}
        &\tilde{\beta} = \beta - 2\alpha\frac{\psi'}{\psi},\ \tilde{\gamma} = \gamma - m + 2 \alpha \left(\frac{\psi'}{\psi}\right)^2 - \alpha \frac{\psi''}{\psi} - \beta \frac{\psi'}{\phi},\ \tilde{f} = \phi f,\\
        &\tilde{u}_T = \psi u_T,\ \tilde{\rho}_1 = \rho_1\frac{\psi(a)}{\psi(a_0)},\ \tilde{\rho}_2 = \rho_2\frac{\psi(b)}{\psi(b_0)}.
    \end{align*}
\end{lemma}

\subsection{Uniqueness results}
\label{subsec:uniqueness_results}

In this section we fix three bounded continuous functions $\alpha, \beta, \gamma$ on $[0,T]\times [a,b]$ such that $\alpha \geqslant 0$. The following results are based on the maximum principle.

\begin{lemma}
\label{lem:linear_uniqueness}
Let $\rho_1, \rho_2 \in (0,1)$ and assume $f= u_T \equiv 0$. Suppose that $\gamma < 0$ and $u \in \D$ solves \eqref{eq:linear_equation_general}.
Then $u \equiv 0$.
\end{lemma}

\begin{proof}
Suppose that there exists $(t,y) \in [0,T] \times [a,b]$ such that $u(t,y) > 0$.
Then by continuity of $u$, there exists $(t^*, y^*) \in [0,T] \times [a,b]$ such that $u(t^*,y^*) = \max u > 0$.

Since $u(T,  \cdot) = 0$, necessarily $t^* < T$. Following \parencite{baldacci_bid_2020}, we show that $y^*\in(a,b)$. Indeed, if $y^*=a$, $u(t^*,a_0) = \frac{1}{\rho_1}u(t^*, y^*)> u(t^*,y^*)$, which contradicts the definition of $(t^*,y^*)$. Hence, $y^*\neq a$. 
A similar argument shows that $y^* \neq b$.

Then, we have $\partial_t u(t^*,y^*) \leqslant 0$, $\partial_y u(t^*, y^*)=0$, $\partial^2_{yy}u(t^*, y^*)\leqslant 0$.
Thus, $\gamma(t^*, y^*) u(t^*, y^*) \geqslant 0$. Since $\gamma < 0$, $u(t^*,y^*) \leqslant 0$, which is a contradiction.

Similar arguments can be carried out to find a contradiction if $u(t,y) < 0$ for some $(t,y)$.
\end{proof}

We now remove the hypotheses $\gamma < 0$ and $\rho_1, \rho_2 \in (0,1)$ from the previous lemma.

\begin{proposition}
\label{prop:linear_uniqueness}
Let $\rho_1, \rho_2 \in (0,\infty)$ and assume $f= u_T \equiv 0$. Suppose that $u \in \D$ solves \eqref{eq:linear_equation_general}.
Then $u \equiv 0$.
\end{proposition}

\begin{proof}
Let $N = \max\left(\frac{\rho_1}{(a_0-a)(b-a_0)},\frac{\rho_2}{(b_0-a)(b-b_0)}\right)$. Define $\psi:y \in[a,b] \mapsto 1+N(y-a)(b-y)$.
Consider $m > \max\left(\gamma + 2 \alpha \left(\frac{\psi'}{\psi}\right)^2 - \alpha \frac{\psi''}{\psi} - \beta \frac{\psi'}{\phi}\right) $ and define $\phi(t,y) := e^{m(t-T)}\psi(t,y)$. Let $v \defeq \phi u$ and take $\tilde{\beta}$, $\tilde{\gamma}$, $\tilde{f}$, $\tilde{u}_T$, $\tilde{\rho}_1$ and $\tilde{\rho}_2$ as in Lemma \ref{lem:equivalence_linear_pde}. Then, $\tilde{\rho}_1$, $\tilde{\rho}_2 \in (0,1)$, $\tilde{\gamma} < 0$, and $v$ solves \eqref{eq:linear_equation_modified}. By Lemma \ref{lem:linear_uniqueness}, $v \equiv 0$, and consequently $u \equiv 0$.
\end{proof}

\subsection{Existence results}

For all the PDEs mentioned in this section, uniqueness holds by the results of Section \ref{subsec:uniqueness_results}.

\begin{proposition}
    \label{prop:linear_existence}
    Let $\rho_1, \rho_2 \in (0,1)$ and $\delta \in (0,1)$. Consider $\alpha, \beta, \gamma, f$ four continuous functions on $(0,T)\times (a,b)$ such that $\min \alpha > 0$ and $|\alpha|_{\delta},|\beta|_{\delta},|\gamma|_{\delta}, |f|_{\delta}< \infty$. Let $u_T$ be a bounded continuous function on $[a,b]$ such that $u_T(a) = \rho_1 u_T(a_0)$ and $u_T(b) = \rho_2 u_T(b_0)$. Assume $\gamma \leqslant 0$. Then, there exists $u \in \D$ solving \eqref{eq:linear_equation_general}.
    
    Furthermore, if $|\alpha|_{\delta},|\beta|_{\delta},|\gamma|_{\delta} \leqslant K_1$, for some $K_1>0$. Then, there exists a constant $C > 0$, depending on the domain, $\delta$, $K_1$, $\rho_1$, $\rho_2$ and $\min \alpha$, but not on $f$ and $u_T$, such that
    \begin{equation*}
        |u|_{2+\delta} \leqslant C(|u_T|_{\infty} + |f|_{\delta}).
    \end{equation*}
\end{proposition}

\begin{proof}
    The proof is almost identical to the one of \parencite[Theorem 2.1]{friedman_monotonic_1986}.
    Let $\epsilon > 0$ and extend $\alpha, \beta,\gamma,f$ to $(-\epsilon, T) \times (a,b)$ such that$|\alpha|_{\delta},|\beta|_{\delta},|\gamma|_{\delta}, |f|_{\delta}< \infty$ still holds in $(-\epsilon, T) \times (a,b)$.
    We define $u_1(t,y) \defeq u_T(y)$. Suppose we have built $u_1,\dots,u_n$. By \parencite[Chapter 3, Theorem 9]{friedman_partial_1983}, there exists a unique $C([-\epsilon,T]\times [a,b]) \cap C^{1,2}((-\epsilon,T)\times (a,b))$ solution of
    \begin{equation*}
        \left\{
            \begin{array}{rl}
                \partial_t u(t,y) + \alpha(t,y)\partial^2_{yy}u(t,y)+ \beta(t,y)\partial_y u (t,y) + \gamma(t,y) u(t,y) = f(t,y) & \text{on } (-\epsilon,T) \times (a,b) \\
                u(T,y) = u_T(y) & y \in [a,b]\\
                u(t,a) = \rho_1 u_n(t,a_0) &  t \in (-\epsilon,T]\\
                u(t,b) = \rho_2 u_n(t,b_0) &  t \in (-\epsilon,T]
            \end{array},
        \right.
    \end{equation*}
    that we denote by $u_{n+1}$. Then, for every $n \geqslant 2$, $u_{n+1} - u_n$ solves 
    \begin{equation*}
        \left\{
            \begin{array}{rl}
                \partial_t u(t,y) + \alpha(t,y)\partial^2_{yy}u(t,y)+ \beta(t,y)\partial_y u (t,y) + \gamma(t,y) u(t,y) = 0 & \text{on } (-\epsilon,T) \times (a,b) \\
                u(T,y) = 0 &  y \in [a,b]\\
                u(t,a) = \rho_1 (u_n-u_{n-1})(t,a_0) & t \in (-\epsilon,T]\\
                u(t,b) = \rho_2 (u_n-u_{n-1})(t,b_0) & t \in (-\epsilon,T]
            \end{array},
        \right.
    \end{equation*}
    By the maximum principle \parencite[Chapter 2, equation (3.8)]{friedman_partial_1983}, $|u_{n+1} - u_n|_{\infty} \leqslant \rho |u_n-u_{n-1}|_{\infty}$, where $\rho \defeq \min(\rho_1,\rho_2) \in (0,1)$. By induction, for all $n \geqslant 2$,
    \begin{equation}\label{eq:appB1}
        |u_n-u_{n-1}|_{\infty} \leqslant \rho^{n-2}|u_2-u_1|_{\infty}.
    \end{equation}
    Thus, $(u_n)_n$ is a Cauchy sequence in $C([-\epsilon,T]\times [a,b])$, which converges uniformly to a function $u \in C([-\epsilon,T]\times [a,b])$. It is clear that $u$ verifies the boundary conditions. Additionally, thanks to \eqref{eq:appB1},
    \begin{equation}\label{eq:appB2}
        |u_n|_{\infty} \leqslant \sum_{i=2}^n|u_i-u_{i-1}|_{\infty} + |u_1|_{\infty} \leqslant \frac{1}{1-\rho}|u_2 - u_1|_{\infty} + |u_1|_{\infty},\quad n\geqslant 2.
    \end{equation}
    By the maximum principle \parencite[Chapter 2, equation (3.10)]{friedman_partial_1983}, there exists a constant $C$, depending only on $\alpha$ and $\beta$, such that $|u_2|_{\infty} \leqslant |u_1|_{\infty} + C|f|_{\infty}$. Given that $|u_1|_{\infty} = |u_T|_{\infty}$, we have, by \eqref{eq:appB2}, the existence of a constant $C'$, depending only on $\alpha,\beta,\rho$, such that 
    \begin{equation}\label{eq:appB3}
    |u_n|_{\infty} \leqslant C'(|u_T|_{\infty} + |f|_{\infty}).
    \end{equation}
    
    Let $K$ be a compact included in $(-\epsilon,T) \times (a,b)$. By \parencite[Chapter 3, Theorem 5]{friedman_partial_1983}, there exists a constant $D$, independent of $n$, such that for all $n \geqslant 2$, the Hölder norms on $K$ of $u_n$, $\partial_t u_n$, $\partial_y u_n$ and $\partial^2_{yy} u_n$ are bounded by $D(|u_n|_{\infty} + |d^2 f|_{\delta})$ and thus, thanks to \eqref{eq:appB3}, by $D(C'|u_T|_{\infty} + C' |f|_{\infty} + |d^2f|_{\delta})$.

    Thus, by the Arzelà-Ascoli theorem, there exists a subsequence $(u_{n_k})_k$ such that $(u_{n_k})_k$, $(\partial_t u_{n_k})_k$, $(\partial_y u_{n_k})_k$ and $(\partial^2_{yy} u_{n_k})_k$ converge uniformly on $K$. Finally, we have that $u \in C^{1,2}(K)$ and, uniformly on $K$,
    \begin{equation*}
        u_{n_k}\to u, \quad \partial_t u_{n_k}\to \partial_t u,
        \quad \partial_y u_{n_k}\to \partial_y u,
        \quad \partial^2_{yy} u_{n_k}\to \partial^2_{yy} u.
    \end{equation*}

    We conclude that the restriction of $u$ to $[0,T) \times [a,b]$, verifies the equation \eqref{eq:linear_equation_general}.

    Now, suppose that $|\alpha|_{\delta},|\beta|_{\delta},|\gamma|_{\delta} \leqslant K_1$. By \parencite[Chapter 3, Theorem 5]{friedman_partial_1983} and \eqref{eq:appB3}, there exists a constant $C''$ depending only on the domain, $K_1$, $\delta$, $\rho_1$, $\rho_2$, $\min \alpha$ such that for all $n \geqslant 2$,
    \begin{equation*}
        |u_n|_{2+\delta} \leqslant C''\left(|u_n|_{\infty} + |f|_{\delta}\right)
        \leqslant C''\left(C'|u_T|_{\infty} + C'|f|_{\infty} + |f|_{\delta}\right)
        \leqslant C''(1+C')(|u_T|_{\infty} + |f|_{\delta}).
    \end{equation*}

    Hence, for all $P,Q \in [0,T) \times (a,b)$ and $n \geqslant 2$ such that $P \neq Q$,
    \begin{equation*}
        d^{2+\delta}_{PQ}\frac{|\partial^2_{yy}u_n(P) - \partial^2_{yy}u_n(Q)|}{d(P,Q)^{\delta}} \leqslant |u_n|_{2+\delta} \leqslant C''(1+C')(|u_T|_{\infty} + |f|_{\delta}).
    \end{equation*}
    Since $\partial_{yy}^2 u_{n_k} \to \partial_{yy}^2 u$ pointwise, taking the limit $n \to \infty$ and the supremum over $P,Q$, yields
    \begin{equation*}
        \sup_{P \neq Q} d^{2+\delta}_{PQ}\frac{|\partial^2_{yy}u(P) - \partial^2_{yy}u(Q)|}{d(P,Q)^{\delta}} \leqslant C'''(|u_T|_{\infty} + |f|_{\delta}).
    \end{equation*}
    Similarly, $|d^2\partial_{yy}^2 u|_{\infty} \leqslant  C''(1+C')(|u_T|_{\infty} + |f|_{\delta})$. Analogous arguments are valid for the other derivatives. Hence, $|u|_{2+\delta} <  8C''(1+C')(|u_T|_{\infty} + |f|_{\delta})$.
\end{proof}

Now we remove the assumptions $\gamma \leqslant 0$ and $\rho_1,\rho_2 < 1$ from the previous result.
\begin{corollary}
    \label{corol:linear_existence}
    Let $\rho_1, \rho_2 \in (0,\infty)$ and $\delta \in (0,1)$. Consider $\alpha, \beta, \gamma, f$ four continuous functions on $(0,T)\times (a,b)$ such that $\min \alpha > 0$ and $|\alpha|_{\delta},|\beta|_{\delta},|\gamma|_{\delta}, |f|_{\delta}< \infty$. Let $u_T$ be a bounded continuous function on $[a,b]$ such that $u_T(a) = \rho_1 u_T(a_0)$ and $u_T(b) = \rho_2 u_T(b_0)$. Then, there exists $u \in \D$ solving \eqref{eq:linear_equation_general}.
    
    Furthermore, if $|\alpha|_{\delta},|\beta|_{\delta},|\gamma|_{\delta} \leqslant K_1$, for some $K_1>0$. Then, there exists a constant $C > 0$, depending on the domain, $\delta$, $K_1$, $\rho_1$, $\rho_2$ and $\min \alpha$, but not on $f$ and $u_T$, such that
    \begin{equation*}
        |u|_{2+\delta} \leqslant C(|u_T|_{\infty} + |f|_{\delta}).
    \end{equation*}
\end{corollary}

\begin{proof}
    Consider the functions $\psi$, $m$ and $\phi$ as in the proof of Proposition \ref{prop:linear_uniqueness}.
   Define $\tilde{\beta}$, $\tilde{\gamma}$, $\tilde{f}$, $\tilde{u}_T$, $\tilde{\rho}_1$ and $\tilde{\rho}_2$ as in Lemma \ref{lem:equivalence_linear_pde}. 
    
    Then $\tilde{\rho}_1, \tilde{\rho}_2 \in (0,1)$, $|\tilde{\beta}|_{\delta}, |\tilde{\gamma}|_{\delta}, |\tilde{f}|_{\delta} < \infty$ and $\tilde{\gamma} \leqslant 0$.
    By Proposition \ref{prop:linear_existence}, there exists $v \in \D$ solving \eqref{eq:linear_equation_modified}, and by Lemma \ref{lem:equivalence_linear_pde},  $u\defeq \frac{v}{\phi}$ solves \eqref{eq:linear_equation_general}.

    Now suppose that $|\alpha|_{\delta},|\beta|_{\delta},|\gamma|_{\delta} \leqslant K_1$ and $|f|_{\delta} < \infty$. Let $K_2 \defeq \max \left\{\left|\frac{\partial_y\phi}{\phi}\right|_{\delta},\left|\left(\frac{\partial_y\phi}{\phi}\right)^2\right|_{\delta}, \left|\frac{\partial^2_{yy}\phi}{\phi}\right|_{\delta}\right\}$, which depends only on $a_0$, $b_0$, $\rho_1$, $\rho_2$, and the domain.
    
    We chose $m = K_1\left(1+4K_2\right)+1$, which is still greater than $\max\left(\gamma + 2 \alpha \left(\frac{\psi'}{\psi}\right)^2 - \alpha \frac{\psi''}{\psi} - \beta \frac{\psi'}{\psi}\right)$. We have
    \begin{align*}
        \left|\tilde{\beta}\right|_{\delta} &\leqslant 
        \left|\beta\right|_{\delta} + 2 \left|\alpha\right|_{\delta}\left|\frac{\partial_y \phi}{\phi}\right|_{\infty} + 2 \left|\alpha\right|_{\infty}\left|\frac{\partial_y \phi}{\phi}\right|_{\delta}
        \leqslant K_1(1+4K_2),\\
        \left|\tilde{\gamma}\right|_{\delta} &\leqslant 
        \left|\gamma\right|_{\delta} + m + 2 \left|\alpha\left(\frac{\partial_y\phi}{\phi}\right)^2\right|_{\delta}
        + \left|\alpha\frac{\partial^2_{yy} \phi}{\phi}\right|_{\delta}
        + \left|\beta\frac{\partial_y \phi}{\phi}\right|_{\delta}
        \leqslant K_1(2+12K_2)+1.
    \end{align*}

    By Proposition \ref{prop:linear_existence}, there exists $C > 0$ depending only on the domain, $K_1$, $\rho_1$, $\rho_2$, $\delta$, $\min \alpha$ such that $|v|_{2+\delta} \leqslant C(|\tilde{u}_T|_{\infty} + |\tilde{f}|_{\delta})$ ($C$ is chosen by replacing $K_1$ by $K_1(2+12K_2)+1$ in Proposition \ref{prop:linear_existence}, which only depends on the mentioned constants). Hence,
    \begin{equation*}
        |v|_{2+\delta} \leqslant C(|\psi|_{\infty}|u_T|_{\infty} + |\phi|_{\infty}|f|_{\delta} + |f|_{\infty}|\phi|_{\delta}) \leqslant 3C|\phi|_{\delta}(|u_T|_{\infty} + |f|_{\delta}).
    \end{equation*}
    Furthermore, since $u = \frac{v}{\phi}$, there exists a constant $C' > 0$ depending only on the domain, $|\phi|_{2+\delta}$ and $\left|\frac{1}{\phi}\right|_{\infty}$ (hence only on the domain, $a_0$, $b_0$, $\rho_1$ and $\rho_2$) such that $|u|_{2+\delta} \leqslant C'|v|_{2+\delta}$. We conclude that
    \begin{equation*}
        |u|_{2+\delta} \leqslant 3CC'|\phi|_{\delta}(|u_T|_{\infty} + |f|_{\delta}).
    \end{equation*}
\end{proof}

\subsection{Probabilistic representation}

In this section we derive in Proposition \ref{prop:feynmankac} a Feynman-Kac-type formula which is useful to establish the results in Section \ref{subsec:continuity} and the a priori estimates in Corollary \ref{corol:bound_on_u}. Let $a < b$, $a_0, b_0 \in (a,b)$. Let $(\Omega, \mathcal{F}, (\mathcal{F}_t)_{t\in [0,T]}, \mathbb{P})$ be a filtered probability space supporting a Brownian motion $W$ and let $Y$ be the mapping built in Appendix \ref{sec:midprice_construction}. Let $\sigma > 0$ and $\beta \in \mathbb{R}$.

For $(t,y) \in [0,T] \times (a,b)$, define $\Y \defeq Y\left(t, \cdot, y, (\beta s + \sigma W_s)_{s \in [0,T]}\right)$, which is a right-continuous adapted process, due to Proposition \ref{prop:Yadapted}.

\begin{proposition}
    \label{prop:feynmankac}
    Let $\gamma, f:[0,T) \times (a,b)\to \mathbb{R}$ be two bounded measurable functions. Let $\beta \in \mathbb{R}$ and $u \in \D$. Suppose that $u$ solves
    \begin{equation*}
        \left\{
            \begin{array}{rl}
                \partial_t u(t,y) + \frac{\sigma^2}{2}\partial^2_{yy}u(t,y)+ \beta\partial_y u (t,y) + \gamma(t,y) u(t,y) + f(t,y) = 0 & \text{on } [0,T) \times (a,b) \\
                u(t,a) = u(t,a_0) & t \in [0,T]\\
                u(t,b) = u(t,b_0) & t \in [0,T]
            \end{array}.
        \right.
    \end{equation*}
    Then, for every $(t,y) \in [0,T] \times (a,b)$ and $(\mathcal{F}_t)$-stopping time $\tau$ taking values in $[t,T]$,
    \begin{equation*}
        u(t,y) = \mathbb{E}\left[u(\tau,\Y_{\tau})e^{\int_t^{\tau}\gamma(r, \Y_r)\diff r} + \int_t^{\tau}f(s, \Y_s) e^{\int_t^{s}\gamma(r, \Y_r)\diff r}  \diff s\right].
    \end{equation*}
\end{proposition}

\begin{proof}
    Let  $(t,y) \in [0,T] \times (a,b)$ and $\tau$ be an $(\mathcal{F}_t)$-stopping time taking values in $[t,T]$. For $i \in \mathbb{N}$, we denote by $\tau_i \defeq \tau_{i}\left(t,y,(\beta s + \sigma W_s)_s\right) \wedge \tau$ the barrier reaching time of $\Y$, which is defined in Appendix \ref{sec:midprice_construction} and is a $(\mathcal{F}_t)$-stopping time. Let $k_0 \in \mathbb{N}^*$ such that $\min\{y - a, b - y, a_0-a, b_0-a, b-a_0,b-b_0\} > \frac{1}{k_0}$. For $k \geqslant k_0$, $i \in \mathbb{N}$, define the stopping time
    \begin{equation*}
        \tau_i^{(k)} \defeq \inf\left\{s \in (\tau_i, \tau_{i+1}) : \Y_s \in\left\{a + \frac{1}{k}, b - \frac{1}{k}\right\}\right\}\wedge\left(T - \frac{1}{k}\right) \wedge \tau_{i+1}.
    \end{equation*}
    By the intermediate value theorem, we have that for all $k \geqslant k_0$ and  $i \in \mathbb{N}$, $\tau_i \leqslant \tau_i^{(k)} \leqslant \tau_{i+1}$ and for all $s \in [\tau_i, \tau_{i}^{(k)}]$, $(s,\Y_s) \in \left[0, T - \frac{1}{k}\right] \times \left[a + \frac{1}{k}, b - \frac{1}{k}\right]$. Furthermore, for fixed $i$, $(\tau_i^{(k)})_{k \geqslant k_0}$ is nondecreasing.

    Let $N \defeq \min\{i \in \mathbb{N} : \tau_i  = \tau\}$ which is finite thanks to Lemma \ref{lem:finite_activity}. 
    Fix $k \geqslant k_0$. Then,
    \begin{equation*}
        \begin{split}
            e^{\int_t^{\tau} \gamma(s, \Y_s) \diff s}u(\tau, \Y_{\tau}) &- u(t,y) =
        \sum_{i=0}^{N-1} \left(e^{\int_t^{\tau^{(k)}_{i}} \gamma(s, \Y_s) \diff s}u\left(\tau^{(k)}_{i}, \Y_{\tau^{(k)}_{i}}\right) -  e^{\int_t^{\tau_i} \gamma(s, \Y_s) \diff s}u(\tau_i, \Y_{\tau_i})\right) \\
        &+
        \sum_{i=0}^{N-1} \left( e^{\int_t^{\tau_{i+1}} \gamma(s, \Y_s) \diff s}u\left(\tau_{i+1}, \Y_{\tau_{i+1}}\right) -  e^{\int_t^{\tau^{(k)}_{i}} \gamma(s, \Y_s) \diff s}u\left(\tau^{(k)}_{i}, \Y_{\tau^{(k)}_{i}}\right)\right).
        \end{split}
    \end{equation*}
    Now, using Itô's formula in each interval $[\tau_i, \tau_i^{(k)}]$, where $\Y_s = y + \beta (s-t) + \sigma(W_s - W_t) + \sum_{j=1}^i \epsilon_j'(t,y,(\beta r + W_r)_r)$,
    \begin{equation*}
        \begin{split}
            e^{\int_t^{\tau} \gamma(s, \Y_s) \diff s}&u(\tau, \Y_{\tau}) - u(t,y) \\
            &=
             \sum_{i=0}^{N-1} \left( e^{\int_t^{\tau_{i+1}} \gamma(s, \Y_s) \diff s}u\left(\tau_{i+1}, \Y_{\tau_{i+1}}\right) -  e^{\int_t^{\tau^{(k)}_{i}} \gamma(s, \Y_s) \diff s}u\left(\tau^{(k)}_{i}, \Y_{\tau^{(k)}_{i}}\right)\right) \\
            &+ \sum_{i=0}^{N-1} \left( \int_{\tau_i}^{\tau_i^{(k)}} e^{\int_t^{s} \gamma(r, \Y_r) \diff r}\left( \partial_t u + \beta \partial_y u + \frac{\sigma^2}{2} \partial_{yy}^2 u + \gamma(s,\Y_s)u \right) (s,\Y_s) \diff s \right) \\
            &+\sum_{i=0}^{N-1} \int_{\tau_i}^{\tau_i^{(k)}} e^{\int_t^{s} \gamma(r, \Y_r) \diff r} \partial_y u(s, \Y_s) \diff W_s.
        \end{split}
    \end{equation*}
    Using the fact that $u$ solves the PDE, we get
    \begin{equation}\label{eq:appB4}
        \begin{split}
            e^{\int_t^{\tau} \gamma(s, \Y_s) \diff s}&u(\tau, \Y_{\tau}) - u(t,y) \\
            &=
            -\int_{t}^{\tau}e^{\int_t^{s} \gamma(r, \Y_r) \diff r} f(s, \Y_s) 
            \left(1-\sum_{i=0}^{N-1}\mathds{1}_{[\tau_i^{(k)},\tau_{i+1}]}(s)\right)\diff s\\
             &+\sum_{i=0}^{N-1} \left( e^{\int_t^{\tau_{i+1}} \gamma(s, \Y_s) \diff s}u\left(\tau_{i+1}, \Y_{\tau_{i+1}}\right) -  e^{\int_t^{\tau^{(k)}_{i}} \gamma(s, \Y_s) \diff s}u\left(\tau^{(k)}_{i}, \Y_{\tau^{(k)}_{i}}\right)\right) \\
            &+\sum_{i=0}^{N-1} \int_{\tau_i}^{\tau_i^{(k)}} e^{\int_t^{s} \gamma(r, \Y_r) \diff r} \partial_y u(s, \Y_s) \diff W_s.
        \end{split}
    \end{equation}
    Since $\partial_t u$ is bounded on $\left[0, T - \frac{1}{k}\right] \times \left[a + \frac{1}{k}, b - \frac{1}{k}\right]$, the last term has 0 expectation. Hence, taking the expectation in \eqref{eq:appB4}
    \begin{equation*}
        \begin{split}
            \mathbb{E}\big[ e^{\int_t^{\tau} \gamma(s, \Y_s) \diff s}&u(\tau, \Y_{\tau})\big] - u(t,y) \\
            &=
            -\mathbb{E}\left[\int_{t}^{\tau}e^{\int_t^{s} \gamma(r, \Y_r) \diff r} f(s, \Y_s) 
            \left(1-\sum_{i=0}^{N-1}\mathds{1}_{[\tau_i^{(k)},\tau_{i+1}]}(s)\right)\diff s \right]\\
             &+\mathbb{E}\left[\sum_{i=0}^{N-1} \left( e^{\int_t^{\tau_{i+1}} \gamma(s, \Y_s) \diff s}u\left(\tau_{i+1}, \Y_{\tau_{i+1}}\right) -  e^{\int_t^{\tau^{(k)}_{i}} \gamma(s, \Y_s) \diff s}u\left(\tau^{(k)}_{i}, \Y_{\tau^{(k)}_{i}}\right)\right)\right].
        \end{split}
    \end{equation*}

    The integrand of the first term is bounded by $e^{|\gamma|_{\infty}T}|f|_{\infty}$ and converges to 0 almost everywhere as $k\to \infty$. Hence, by dominated convergence, the first term converges to 0 as $k\to \infty$. To complete the proof, it only remains to show that the second term tends to 0 as $k \to \infty$. 

    The term inside the expectation is bounded by $2N |u|_{\infty} e^{T|\gamma|_{\infty}}$, which is integrable thanks to Lemma \ref{lem:finite_activity_expectation}).

    One can show  that for every $i$ 
    \begin{equation*}
        \tau_i^{(k)}\xrightarrow[k \to \infty]{}\tau_{i+1}\text{ and } u\left(\tau^{(k)}_{i}, \Y_{\tau^{(k)}_{i}}\right)\xrightarrow[k \to \infty]{} u\left(\tau_{i+1}, \Y_{\tau_{i+1}}\right),
    \end{equation*}
    the last limit being a consequence of the boundary conditions. Hence, by dominated convergence, the second term also tends to 0 as $k \to \infty$.
\end{proof}

\begin{corollary}
    \label{corol:bound_on_u}
    Let $\gamma, f:[0,T) \times (a,b)\mapsto \mathbb{R}$ be two bounded measurable functions and $\beta \in \mathbb{R}$. Suppose that $u\in\D$ solves
    \begin{equation*}
        \left\{
            \begin{array}{rl}
                \partial_t u(t,x) + \frac{\sigma^2}{2}\partial^2_{xx}u(t,x)+ \beta\partial_x u (t,x) + \gamma(t,x) u(t,x) + f(t,x) = 0 & \text{on } (0,T] \times (a,b) \\
                u(T,y) = 0 &  y \in [a,b]\\
                u(t,a) = u(t,a_0) & t \in [0,T]\\
                u(t,b) = u(t,b_0) & t \in [0,T]
            \end{array}.
        \right.
    \end{equation*}
    Let $g$ be a measurable function on $[0,T]$ such that for almost every $s \in [0,T]$, $g(s) \geqslant \sup_{y \in (a,b)}|f(s,y)|$. Then, for all $t \in [0,T]$ and $c > \sup \gamma$,
    \begin{equation*}
        e^{c t}\sup_{y \in [a,b]}|u(t,y)| \leqslant
        \int_t^T e^{c s}g(s)\diff s.
    \end{equation*}
\end{corollary}

\subsection{The Krylov-Safonov estimates}

In this section, we recall the Krylov-Safonov estimates, which control the local Hölder norms of $u$ with respect to the $L^2$-norm of $f$ and $|u|_{\infty}$. For $D \subset [0,T] \times [a,b]$, we denote by $W^{1,2}_2(D)$ the space of measurable functions $u$ on $D$ admitting weak derivatives $\partial_t u$, $|\partial_y u|$ and $\partial^2_{yy}u$ such that $\int_D|u|^2$, $\int_D |\partial_t u|^2$, $\int_D |\partial_yu|^2$ and $\int_D |\partial^2_{yy}u|^2$ are all finite.

We recall the Krylov-Safonov estimate \parencite[Theorem 4.2]{krylov_certain_1981} (replacing the variable $t$ by $T-t$ to suit our framework).
\begin{theorem}
    Let $D$ an open subset of $[0,T] \times [a,b]$. Let $K > 0$ and $\alpha$, $\beta$, $\gamma$ be three bounded measurable functions such that $\frac{1}{K} \leqslant \alpha \leqslant K$, $|\beta| \leqslant K$ and $-K \leqslant \gamma \leqslant 0$. Then, there exist two constants $C > 0$, $\delta \in (0,1)$ depending on $K$ (but not on the specific subset $D$) such that for every $P, P' \in D$ with $d(P,P')\leqslant \frac{1}{4} \left(d^D_{PP'} \wedge 1 \right)$, and every $u \in W^{1,2}_2(D)$
    \begin{equation*}
        \left(d^D_{PP'} \wedge 1\right)^{\delta}\left|u(P) - u(P')\right| \leqslant C d(P,P')^{\delta} \left(|u|_{\infty} + \sqrt{\int_D |\partial_t u + \alpha \partial^2_{yy} u + \beta \partial_y u + \gamma u|^2}\right).
    \end{equation*}
\end{theorem}

Proceeding like in the proof of \textcite[Theorem 4.3]{krylov_certain_1981}, we deduce the following corollary, which shows that we can remove the hypotheses $\gamma\leqslant 0$ and $d(P,P')\leqslant \frac{1}{4} \left(d^D_{PP'} \wedge 1 \right)$, and obtain uniform estimates with respect to $\delta'\in(0,\delta]$.
\begin{corollary}
    \label{corol:get_rid_of_one_fourth_condition}
    Let $D$ an open subset of $[0,T] \times [a,b]$. Let $K > 0$ and $\alpha$, $\beta$, $\gamma$ be three bounded measurable functions such that $\frac{1}{K} \leqslant \alpha \leqslant K$, $|\beta| \leqslant K$ and $|\gamma| \leqslant K$. Then, there exist two constants $C > 0$, $\delta \in (0,1)$ depending on $K$, $a$, $b$, $T$ (but not on the specific subset $D$) such that for every $P, P' \in D$, $\delta' \in (0,\delta]$ and $u \in W^{1,2}_2(D)$
    \begin{equation*}
        \left(d^D_{PP'} \right)^{\delta'}\left|u(P) - u(P')\right| \leqslant C d(P,P')^{\delta'} \left(|u|_{\infty} + \sqrt{\int_D |\partial_t u + \alpha \partial^2_{yy} u + \beta \partial_y u + \gamma u|^2}\right).
    \end{equation*}
\end{corollary}

\begin{proof}
    First observe that, if $P,P' \in D$ satisfy $d(P,P')> \frac{1}{4}(d^D_{PP'} \wedge 1)$, then, $\left(d^D_{PP'} \wedge 1\right)^{\delta}\frac{\left|u(P) - u(P')\right|}{d(P,P')^{\delta}} \leqslant 2\cdot 4^{\delta}|u|_{\infty}$. Defining $C' \defeq \max\{C, 2\cdot 4^{\delta}\}\max\left\{1, T \wedge \frac{b-a}{2}\right\}^{\delta}> 2$ ($(C, \delta)$ given by the preceding theorem replacing $K$ by $2K$), thanks to the preceding theorem, and since $d^D_{PP'} \leqslant T \wedge \frac{b-a}{2}$, we have, for any $P,P' \in D$,
    \begin{equation*}
        \left(d^D_{PP'}\right)^{\delta}\left|u(P) - u(P')\right| \leqslant C' d(P,P')^{\delta} \left(|u|_{\infty} + \sqrt{\int_D |\partial_t u + \alpha \partial^2_{yy} u + \beta \partial_y u + (\gamma - K) u|^2}\right).
    \end{equation*}

    Using the Minkowski inequality on the $\sqrt{\int}$ term, we get
    \begin{equation*}
        \left(d^D_{PP'}\right)^{\delta}\left|u(P) - u(P')\right| \leqslant C' d(P,P')^{\delta} \left(|u|_{\infty} + K\sqrt{\int_D |u|^2} +\sqrt{\int_D |\partial_t u + \alpha \partial^2_{yy} u + \beta \partial_y u + \gamma u|^2}\right).
    \end{equation*}
    Then, setting $C'' \defeq C'(1+K\sqrt{T(b-a)}) > 2$ (which only depends on $K$, $b-a$ and $T$)
    \begin{equation*}
        \left(d^D_{PP'} \right)^{\delta}\left|u(P) - u(P')\right| \leqslant C'' d(P,P')^{\delta} \left(|u|_{\infty} + \sqrt{\int_D |\partial_t u + \alpha \partial^2_{yy} u + \beta \partial_y u + \gamma u|^2}\right).
    \end{equation*}
    
    Let $\delta' \in (0,\delta)$ and $P, P' \in D$ be such that $P \neq P'$. We have
    \begin{equation*}
        \frac{\left(d^D_{PP'} \right)^{\delta'}\left|u(P) - u(P')\right|}{d(P,P')^{\delta'}} \leqslant
        C'' \left(|u|_{\infty} + \sqrt{\int_D |\partial_t u + \alpha \partial^2_{yy} u + \beta \partial_y u + \gamma u|^2}\right) \frac{\left(d^D_{PP'} \right)^{\delta- \delta'}}{d(P,P')^{\delta-\delta'}}.
    \end{equation*}
If $d(P,P') > d^D_{PP'}$, then $\frac{\left(d^D_{PP'} \right)^{\delta- \delta'}}{d(P,P')^{\delta-\delta'}} \leqslant 1$. If $d(P,P') \leqslant d^D_{PP'}$, then $\frac{\left(d^D_{PP'} \right)^{\delta'}\left|u(P) - u(P')\right|}{d(P,P')^{\delta'}} \leqslant 2 |u|_{\infty}$. Thus, we have for any $P,P' \in D$,
    \begin{equation*}
        \frac{\left(d^D_{PP'} \right)^{\delta'}\left|u(P) - u(P')\right|}{d(P,P')^{\delta'}} \leqslant
        C'' \left(|u|_{\infty} + \sqrt{\int_D |\partial_t u + \alpha \partial^2_{yy} u + \beta \partial_y u + \gamma u|^2}\right)
    \end{equation*}
as desired.
\end{proof}

Now we state the result when a function $u \in \D$ solves a parabolic PDE. In this case $u$ is not necessarily in $W^{1,2}_{2}$ because its derivatives may be unbounded close to the boundary.

\begin{corollary}
    \label{corol:krylovsafonov}
    Let $K > 0$ and $\alpha$, $\beta$, $\gamma$ be three bounded measurable functions such that $\frac{1}{K} \leqslant \alpha \leqslant K$, $|\beta| \leqslant K$ and $|\gamma| \leqslant K$. Then, there exist two constants $C' > 0$, $\delta \in (0,1)$, depending on $K$, $a$, $b$ and $T$, such that for every $P, P' \in [0,T)\times (a,b)$, $\delta' \in (0,\delta]$ and $u \in \D$,
    \begin{equation*}
        \left(d_{PP'} \right)^{\delta'}\left|u(P) - u(P')\right| \leqslant C' d(P,P')^{\delta'} \left(|u|_{\infty} + \sqrt{\int_{[0,T)\times (a,b)} |\partial_t u + \alpha \partial^2_{yy} u + \beta \partial_y u + \gamma u|^2}\right).
    \end{equation*}
    In particular, there exists a constant $C$ depending only on $K$, $a$, $b$ and $T$ such that for every measurable function $f$ on $[0,T) \times (a,b)$, if $u \in \D$ solves
    \begin{equation*}
        \partial_t u + \alpha \partial^2_{yy} u + \beta \partial_y u + \gamma u = f
    \end{equation*}
    on $[0,T) \times (a,b)$, then for every $\delta' \in (0,\delta]$, $|u|_{\delta'} \leqslant C\left(|u|_{\infty} + |f|_{\infty}\right)$.
\end{corollary}

\begin{proof}
    Let $C$, $\delta$ be the constants given by corollary \ref{corol:get_rid_of_one_fourth_condition}. Let $P,P' \in [0,T) \times(a,b)$. Let $\varepsilon > 0$ such that $P,P' \in D^{\varepsilon} \defeq [0,T-\varepsilon) \times (a + \varepsilon , b - \varepsilon)$. Then,
    \begin{equation*}
        \left(d^{D^{\varepsilon}}_{PP'} \right)^{\delta'}\left|u(P) - u(P')\right| \leqslant C d(P,P')^{\delta'} \left(|u|_{\infty} + \sqrt{\int_{[0,T)\times (a,b)} |\partial_t u + \alpha \partial^2_{yy} u + \beta \partial_y u + \gamma u|^2}\right)
    \end{equation*}
    for all $\delta' \in (0,\delta]$. Since $\lim\limits_{\varepsilon \to 0}d^{D^{\varepsilon}}_{PP'} = d_{PP'}$, we get the desired result.
\end{proof}

\begin{rem}
    Note that in the previous corollary $\sqrt{\int_{(0,T)\times (a,b)} |\partial_t u + \alpha \partial^2_{yy} u + \beta \partial_y u + \gamma u|^2}$ might be infinite.
\end{rem}