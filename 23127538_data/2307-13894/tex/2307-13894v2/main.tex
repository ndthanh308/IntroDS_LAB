\documentclass{article}


% if you need to pass options to natbib, use, e.g.:
\PassOptionsToPackage{numbers, compress}{natbib}
% before loading neurips_2023


% ready for submission
\usepackage[preprint]{neurips_2023}


% to compile a preprint version, e.g., for submission to arXiv, add add the
% [preprint] option:
%     \usepackage[preprint]{neurips_2023}


% to compile a camera-ready version, add the [final] option, e.g.:
%     \usepackage[final]{neurips_2023}


% to avoid loading the natbib package, add option nonatbib:
%\usepackage[nonatbib]{neurips_2023}

\usepackage[utf8]{inputenc} % allow utf-8 input
\usepackage[T1]{fontenc}    % use 8-bit T1 fonts
\usepackage{hyperref}       % hyperlinks
\usepackage{url}            % simple URL typesetting
\usepackage{booktabs}       % professional-quality tables
\usepackage{amsfonts}       % blackboard math symbols
\usepackage{nicefrac}       % compact symbols for 1/2, etc.
\usepackage{microtype}      % microtypography
\usepackage{xcolor}         % colors
\usepackage{graphicx}
\usepackage{textcomp}
\usepackage{subcaption}


%% CUSTOM PACKAGES AND COMMANDS %%
\usepackage{gensymb}

% add algorithm reference to autoref package
\renewcommand{\sectionautorefname}{Section}

% let autoref use section instead of (sub)subsection
\let\subsectionautorefname\sectionautorefname{}
\let\subsubsectionautorefname\sectionautorefname{}


\title{AI4GCC - Team: Below Sea Level \\Critiques and Improvements}


% The \author macro works with any number of authors. There are two commands
% used to separate the names and addresses of multiple authors: \And and \AND.
%
% Using \And between authors leaves it to LaTeX to determine where to break the
% lines. Using \AND forces a line break at that point. So, if LaTeX puts 3 of 4
% authors names on the first line, and the last on the second line, try using
% \AND instead of \And before the third author name.


\author{%
Bram Renting*
\\
Leiden University \\ Delft University of Technology
\And 
Phillip Wozny*
\\
Tilburg University \\ Vrije Universiteit Amsterdam 
\AND 
Robert Loftin \\
Delft University of Technology
\And 
Claudia Wieners\\
Utrecht University
\And
Erman Acar \\
University of Amsterdam 
  % examples of more authors
  % \And
  % Coauthor \\
  % Affiliation \\
  % Address \\
  % \texttt{email} \\
  % \AND
  % Coauthor \\
  % Affiliation \\
  % Address \\
  % \texttt{email} \\
  % \And
  % Coauthor \\
  % Affiliation \\
  % Address \\
  % \texttt{email} \\
  % \And
  % Coauthor \\
  % Affiliation \\
  % Address \\
  % \texttt{email} \\
}



\begin{document}


\maketitle


\begin{abstract}
We present a critical analysis of the simulation framework RICE-N, an integrated assessment model (IAM) for evaluating the impacts of climate change on the economy. We identify key issues with RICE-N, including action masking and irrelevant actions, and suggest improvements such as utilizing tariff revenue and penalizing overproduction. We also critically engage with features of IAMs in general,  namely overly optimistic damage functions and unrealistic abatement cost functions. Our findings contribute to the ongoing efforts to further develop the RICE-N framework in an effort to improve the simulation, making it more useful as an inspiration for policymakers. 

% Our efforts aim to contribute to the ongoing efforts to develop IAMs that leverage multi-agent reinforcement learning to yield actionable climate-economic policies.
\end{abstract}


\section{Introduction}

Over the course of implementing the first two submissions, we engaged with the simulation framework RICE-N. We interrogate both RICE-N as implemented by \citet{zhang2022ai} and the assumptions inherent to the class of models to which it belongs, known as integrated assessment models (IAMs). 

\paragraph{Executive summary}

Our key points of criticism can be categorically summarised as follows:

\begin{itemize}
    \item Action masking inflates model performance
    \item Issues with RICE-N:
    \begin{itemize}
        \item Most actions are irrelevant for key climate and economic indices.
        \item Tariffs do not impact reward of the affected state.
        \item Trade does not impact reward of the affected state as intended.
        \item Damages have little impact on reward.
    \end{itemize}
    \item Suggestions for RICE-N improvement:
    \begin{itemize}
        \item Use tariff revenue.
        \item Penalize overproduction.
        \item Allow for technology sharing and wealth redistribution.
        \item Let abatement cost depend on the previous mitigation level.
        \item Strengthen the damage function.
    \end{itemize}
    \item Issues with IAMs:
    \begin{itemize}
        \item The damage function is overly optimistic.
        \item Abatement costs do not depend on previous mitigation.
    \end{itemize}
\end{itemize}

% In the following sections, we will explain these points in detail.

\section{Problems with Action Masking}\label{sec:masking}

Action masking is used to disable actions that are inconsistent with negotiated agreements. Excessive masking can force protocols into seemingly desirable behavior despite being the result of random behavior. This phenomenon is evident in the Bilateral Negotiator released with the competition. The protocol performs better after training for one episode than after extensive training. This is due to the following well-intended design features. First, agents commit to the maximum accepted proposal. With 27 agents sending and receiving proposals, one of the 54 total proposals likely corresponds to a high mitigation rate. Second, masks are used to enforce the maximum accepted proposal. As such, high levels of mitigation are almost certainly enforced.

Finally, we argue that actions rendered inaccessible through masking are unrealistic as states can never be forced to commit to their agreements. Instead, collectively unfavorable actions should be made less desirable through extrinsic factors imposed by other states.

% To illustrate this effect, we trained a Bilateral Negotiator for 5 episodes. A comparison of the low episode model to the default configuration model can be found in \autoref{fig:gt}. This indicates that there is a period of training, during which a quirk of random behavior is favorably enforced, due to masking. This begs the question of whether masking is necessary and whether the sanction mechanism is sufficient to shape behavior. 

\section{Problems with the RICE-N model}

\subsection{Influence of actions on key measures}\label{sec:interconnectedness}
% We now study the interconnectedness of the actions that are input to the RICE-N model and the important performance measures through analyzing correlations
In order to explore the space of attainable outcomes in RICE-N we now analyze the correlation of key performance measures to a range of possible action inputs. In this context, we restrict the actions such that they are fixed for a full simulation and that all regions perform the same action. We consider RICE-N with 10 discrete action levels. There are 5 different types of actions of which we want to sample every possible combination, resulting in a total of $10^5$ environment rollouts. We collected the climate index, economic index, and reward of the episode and generated the correlation matrix shown in \autoref{fig:correlation}.

% Figure environment removed

We observe that there is practically no correlation between desired import, tariff rate, maximum export, and climate/economic index. Only the mitigation rate and savings rate have an impact on the economic index and the climate index. Realize that this makes all the other actions irrelevant for the first submission track of this competition. We further explore this by plotting the results of every episode in \autoref{fig:climate_vs_economic} and observe that the $10^5$ episodes form $100$ perfectly overlapping dots. The difference between dots along the $x$-axis represents the savings rate and along the $y$-axis the mitigation rate. We also observe that optimizing for reward without any restriction results in a low score both in the economic and climate index. Finally, we would like to note that gross output, and thus the economic index, is an internal affair that the states will not punish each other over. It is unrealistic to expect states to force themselves to increase their gross output as this would only lower their reward. Such issues make high-economic index solutions not learnable.

This is not the case for climate mitigation as states do have the incentive to punish other states that do not mitigate. There are two methods to punish non-compliant states: (1) imposing tariffs on those states, and (2) limiting imports from that state. However, \autoref{fig:correlation} suggests that there is barely an effect of tariffs on the reward. Even more surprising is the negative correlation between desired imports and the reward, which suggests that limiting imports from a state is rewarding that state. We study this in more depth in the following sections.


% Figure environment removed

\subsubsection{Effects of tariffs on reward}\label{sec:tariffeffects}
The first sanctioning mechanism of the RICE-N model  is the tariff. This is informed by literature on climate economics, which uses tariffs and levies to adjust the cost of carbon~\cite{overland2022climate, nordhaus2015climate}. In previous IAMs, there is a static parameter corresponding to the loss of welfare per unit of tariff~\cite{nordhaus2021dynamic}. At each step, agents compare the cost of mitigation to the expected loss of welfare due to tariffs. Once the latter exceeds the former, mitigation becomes the preferred action.

Currently, there is no global authority that can externally sanction defection. As such, it is the responsibility of states to sanction one another. The organizers' implementation of the RICE-N model acknowledges this fact and grounds the sanction in the trade dynamics of the simulation itself~\cite{zhang2022ai}. As a result of that, there is no ``loss of welfare due to tariff'' parameter. It is assumed that if agents apply tariffs to each other it will result in a loss of revenue. However, as we will show empirically in the following analysis, tariffs have negligible to no impact on reward in most cases.

% Figure environment removed

% Figure environment removed


\paragraph{Methods}

To measure the effects of tariffs on reward and trade dynamics, we conducted the following experiment comparing three agent groups: pariahs, controls, and free trade. Pariah agents received a fixed tariff from all other agents, control agents simply followed their trained protocol, and free trade agents received no tariff. We examined three different fixed tariff values for the pariah agents; 5, 7, and 9. Using a model trained without negotiation for $100\,000$ episodes, we ran the simulation 1000 times for each group and each experimental condition. 

During each iteration of the experiment, an agent was chosen uniformly at random as the subject. Between each group and each fixed tariff value, the agent and environment were reset. As such, each experimental condition contains data from all agents. As the reward varies considerably between states and is relatively stable within states, we normalize the reward by each state id. Therefore, we compare the effect of tariffs on rewards relative to each state.

In a follow-up experiment, we sampled the RICE-N model with fixed actions that represent an ideal trade scenario where the mitigation rate is $0.9$, savings rate $0.3$, desired import $0.9$, and maximum export $0.9$. Under such high trade conditions, the effect of tariffs would be expected to be magnified. We compared the rewards of each state with both maximum tariffs and no tariffs at all in order to measure the effect of tariffs on reward.

\paragraph{Results}
As visible in \autoref{fig:anr}, manipulating tariff magnitude does not result in significant differences in relative reward. To ensure that our tariff manipulation is effective, we also gathered the average tariff per subject in each treatment condition (see \autoref{fig:ats3}). Taken together, it indicates neither the maximum tariff nor the absence of tariff from all agents towards any single agent impacts that agent's reward. This suggests a flaw in the trade component of the reward calculation.

The second experiment showed that the reward of a state is untouched by other states' tariff actions. Even more surprising is that the reward of the state that imposed the tariff decreases by $0.02-6.50\%$.

\paragraph{Explanation}
For the RICE-N model to accurately model interstate commerce and climate-related tariff mechanisms, the tariff mechanism  needs to impact reward. The failure of tariffs to impact the reward can be traced back to how trade influences the reward which we explain in the following.

The reward is calculated as aggregate consumption which can be decomposed into two additive terms, foreign and domestic consumption, and trade is the amount of import/export from one country to another. Critically, trade takes two forms, scaled imports, and tariffed imports. The former is the overlap between the desired imports of one state corrected for gross output and the exporting capacity of the other. The latter adjusts the scaled imports by the inverse of the tariff. Foreign consumption utilizes tariffed imports and domestic consumption utilizes scaled imports. Therefore, if state A tariffs state B, it is only factored into the reduction of state A's foreign consumption term. State B's domestic consumption, which depends on its export to State A, remains unchanged by the tariff. Therefore, tariffing a state does not affect the reward of the state being tariffed.

\subsubsection{Effects of desired imports on reward}
We now focus on the desired imports and maximum exports actions, or simply the amount of trade. We sampled the RICE-N model with fixed actions where the mitigation rate is $0.9$, savings rate $0.3$, and tariffs $0$ to maximize the benefit of trade. We varied the desired imports and maximum export actions, observing the reward.

Reward increases for all states when trade is limited. If we compare the extreme cases of maximum trade and no trade, the reward per state increases by a percentage of $1-2322\%$, depending on the state, when no trade is happening. It appears that limiting imports cannot be used to negatively impact the reward of another state, but actually positively impacts it. This is likely the case because domestic consumption is preferred over foreign consumption in the current setup of RICE-N. As export lowers your domestic consumption and the import does not compensate for this, states will prefer not to export.


\subsubsection{State punishment potential}
We can conclude that neither imposing tariffs nor limiting imports are suitable sanction mechanisms in RICE-N. Moreover, limiting imports even increases the reward of the subjected state. As such, states have no leverage with which to negotiate. Any observed form of ``learned'' negotiation is likely a result of random behavior or favorable masking or both. Any optimal learning agent will end up at the point of maximum reward (see \autoref{fig:climate_vs_economic}) at a climate index of 0.33 and an economic index of 0.6. Therefore, the only mechanism to obtain more optimal policies is to exploit the RICE-N framework itself.

\subsection{Our submission one}
Due to these shortcomings, we \emph{strategically} utilized three Bilateral Negotiator variations which make use of the action masking quirk described in \autoref{sec:masking}. That is, the protocols were essentially untrained. One protocol made use of favorable masking for savings, another for mitigation, and the third for both savings and mitigation. The resulting Pareto frontier is visible in \autoref{fig:gt}. We are aware of the fact that such protocols are not in the spirit of the competition. However, it was stated during the first workshop session that tracks one and two submissions may be independent and that the goal of track one is purely metric optimization.

% Figure environment removed

% Any optimal learning agent would end up at the point of maximum reward in \autoref{fig:climate_vs_economic}, which is at a climate index of 0.33 and economic index of 0.6. Any improvement above that can only be caused sub-optimally trained agents with favorable masking. Due to these shortcomings, we \emph{strategically} utilized three BilateralNegotiator variations which make use of the action masking quirk described in \autoref{sec:masking}. One protococol made use of favorable masking for savings, another for mitigation, and the third for both savings and mitigation. The resulting Pareto frontier is visible in \autoref{fig:gt}.

% The combination of a large high-dimensional action space that is difficult to explore efficiently and the action masking that favors sought-after behavior gives a good result.

% \subsubsection{Reward decoupled from output}

% reward is not indicative of economic activity and climate. There sould be an effect of damages. and there should be an effect of doomsday scenarios (depends on the horizon)

% swap economic index with reward, does that change the Pareto frontier

\subsection{Finite horizon}
When no mitigation is performed by states, the damages reduce the gross output of states by $\sim8.5\%$. This is only a minor deduction of the gross output and does not reflect the real-world \emph{sword of Damocles} which is a climate disaster. From the perspective of the agents, the game simply ends and no large damages are achieved. Reducing an infinite horizon game to a finite one changes optimal strategies from a game-theoretical perspective.

We extended the horizon of the simulation to 200 and 300 years to let damages influence the agents more heavily. We trained agents without a protocol with the hope that major future damages would cause the agents to learn to mitigate. However, while damages rose to $\sim 13\%$ and $\sim 22\%$, mitigation rates remained as before. We also saw earlier in \autoref{fig:correlation} and \autoref{fig:climate_vs_economic} that reward negatively correlates with gross output (economic index). We can conclude that there is little effect of temperature rise on state reward.



% climate=global public good, global public goods are sensitive to free-riding
% free-riding is an instance of defection in a repeated prisoners dilemma
% repeated prisoners dilemmas cooperation is prefereable with infinite time horizon
% RICE-N is finite 
% With an extended time horizon you nwould expect: reward loss due damages alone, more cooperation

% show:
% try the longest time horizon possible (if approx infinity not possible, thats another shortcoming of RICE-N we can list)
% Do damages grow exponentially? if not, thats anothe shortcoming
% Are states more willing to cooperate early on? if not, thats another shortcoming, inability to attribute
% Experiment, are damages alone on an extended time horizon sufficient for learning to mitigate in the absence of a negotiation protocol.


\section{Suggestions for improvement}

\subsection{Tariff revenue}

Currently, tariff revenue is calculated during the climate economic simulation step but not used. Tariff revenue could contribute to government balance, the debt ratio, and the subsequent import capacity for the next climate economic simulation step. Literature on climate policy-induced economic inequality suggests that carbon tariff revenue should be reinvested in developing countries to develop climate mitigation infrastructure~\cite{goldthauHowOpenClimate2022}. 

% However, this value remains untouched in the simulation for any reason. The result is that tariffs cause lower consumption while having no benefit whatsoever.



\subsection{Cost of overproduction}

As stated in \autoref{sec:tariffeffects}, tariffs do not impact reward. A possible workaround could make use of scaled and tariffed imports. The differences between the two correspond to the amount of output produced for export but not ultimately consumed by the importer. Currently, the exporting country faces no penalty for overproduction. Adding the difference between tariffed imports and scaled imports to the export term of domestic consumption would lower overall consumption and function as a penalty to the country being tariffed. 

% This is due to scaled imports being used for the exports term of the domestic consumption calculation. As such, the exports are treated as if they were fully taken into account of importing countries' foreign consumption, which uses tariffed imports instead. 

\subsection{Disaster influenced reward}
Future damages should have an impact on the reward of states, which seems to be limited currently. The presence of a horizon in the simulation creates a situation without a doomsday scenario. Simulating an infinite game is of course impossible, but adding an artificial \emph{sword of Damocles} might help. This could be in the form of a high negative reward when a certain temperature rise is passed.

\section{Issues with IAMs}

\subsection{Damage by climate change}

DICE's damage function describes the fraction of GDP lost to climate damage. It is unlikely that catastrophic climate conditions of a 5\degree C temperature rise would only result in a 5\% loss in GDP~\cite{woillez2020economic}. As such, more realistic damage functions are required, such as \citet{weitzman2012ghg} which allows for an ``infinitely bad'' climate, or \citet{burke2015global}, which attempts to construct a damage function from empirical data. 

Damage functions can also be viewed as secondary to the ``guardrail approach'', which emphasizes a target warming threshold. Policymakers then focus on staying under the threshold as opposed to avoiding damages~\cite{stern2022economics}

% According to \citet{stern2022economics}: Climate and Earth science (and political considerations) are used to set a threshold warming (2 or 1.5 degrees) beyond which warming is considered ``too dangerous''. Once a threshold is established, policy-makers figure out how to reach the target. To summarize, while acknowledging that there is no `correct'' damage function, DICE's damage underestimates climate damage. 

\subsection{The dynamics of mitigation costs}

Currently, mitigation costs persistently depend only on the current mitigation level. In reality, mitigation costs are transitional. Constructing a wind farm is costly during construction, but not so once completed. Persistent mitigation costs incentivize the latter investment, once abatement costs are decreased due to technology. In contrast, transitional mitigation costs incentivize early investment, as that reduces the cost of subsequent mitigation.

\citet{grubb2021modeling} proposed a correction to DICE which includes a transitional mitigation cost function. This allows states to negotiate around rates of change as opposed to absolute mitigation levels which is more in line with real-world climate targets (e.g.~halving emissions by 2030). 

% Potentially, one could then also adapt the negotiation protocol -- for example, rather than negotiating about current mitigation levels, negotiations may entail rates of change (a country commits to increasing mitigation by a certain amount over the next 30 years), or net-zero targets (a country commits to eliminating emissions by a certain year, say 2060; this translates to, say, a linear increase in mitigation over time). 
% Future mitigation targets are more in line with actual targets countries set for themselves; many countries have targets such as halving emissions (w.r.t.{} some baseline) by some target date (e.g., 2030) and reaching (net) zero emissions later (e.g., by 2050 or 2060). 

% The mitigation cost dynamics in DICE is flaws in that it only depend on the current mitigation level: 

% \begin{equation}
%     C_{\mu,abs}(t)=C_{0}(t) \cdot E_{0}(t) \cdot \mu(t)^{\beta},
% \end{equation}
% where $\mu$ is the abatement fraction (or fraction of emissions prevented), $E_0$ is the emissions that would occur in absence of mitigation (depend on economic production) and $\beta$. The higher the emissions, $E_{0}$, the higher the cost of their reduction and visa versa. The exponent $\beta>1$ indicates that mitigation becomes cumulatively more costly; the exogenous function $C_{0}(t)$ decreases over time, representing technological progress. 

% This is problematic as in reality, the cost of mitigation depends on previous mitigation. For example, constructing a windfarm is costly during construction, but not so once completed. Furthermore, experienced gained during construction decreases the cost of similar future projects due to knowledge accumulation. Therefore, mitigation costs are transitional and not persistent. Persistent mitigation costs incentivize latter investment, once abatement costs are decreased due to technology. In contrast, transitional mitigation costs incentivize early investment, as that reduces the cost of subsequent mitigation.

% In real life, mitigation is a -- very large, costly and complex -- investment. It is transitional, in the sense that at some stage, the process is completed. Once wind and solar farms are in place, energy storage installed, transport electrified, etc. -- i.e., once the green infrastructure is in place, mitigation costs go down or may even become negative (it is possible that a green world, once established, is cheaper than a fuel-based one, e.g., because it avoids the cost of mining fuel). Conversely, in the DICE model, mitigation costs are persistent, i.e., mitigating in 2100 will cost money, regardless of how much has been mitigated in preceding decades. 

% If mitigation is transition-like, it makes sense to start early -- one has to go through the transition anyway, but the earlier it is done, the more CO2 reduction is achieved. 
% Conversely, in the DICE model, especially since mitigation costs exogeneously reduce in time (due to some mysterious ``technological advance'' -- which, unlike learning-by-doing, is independent of mitigation), it makes sense not to mitigate too much initially: Early abatement is particularly expensive, but it does not give advantages beyond direct CO2 savings in that time step -- it is not an investment into a greener future through building green infrastructure and enabling learning-by-doing. 

% \citet{grubb2021modeling} proposed a correction to DICE, 
% \begin{equation}
% C_{\mu,abs}(t)=(1-p) \cdot C_{0}(t) \cdot E_{0}(t) \cdot \mu(t)^{\beta}+p \cdot \gamma_{plia} \cdot C_{0}(t) \cdot E_{0}(t)\left(\frac{d\mu(t)}{dt}\right)^{\beta}
% \end{equation}
% where the second term represents the transitional costs, i.e., the costs for changing from brown to green technology: The faster the rate of change $d\mu(t)/dt$, the higher the costs, but once the transition is completed ($\mu(t)=1$ constant), no additional costs are made. The first term captures persistent costs which depend on the abatement level. The ``pliability'' $0\le p\le1$ scales the relative importance of the two cost components; for $p=0$, the DICE model is recovered. The constant $\gamma_{plia}$ is a scaling factor. \citet{grubb2021modeling} show that for $p=0$, abatement expenditure is initially low, but increases over time, as the decreasing exogenous cost function $C_{0}$ ensures greater CO2 reduction per dollar spent as time progresses. On the other hand, for $p=1$, abatement expenditure is highest in the first decades: increasing $\mu$ early on (interpreted as early investment in green technology) will have lasting benefits for emission reduction and is hence worthwhile to pursue. If mitigation costs are transitional (i.e., $p\rightarrow1$), strong early efforts are useful. If mitigation costs are persistent but exogenously decreasing, then mitigation efforts should be postponed. 

% It should be possible to insert the Grubb formula into the DICE version used here. At least if $p=1$ is chosen, mitigation action would then not be measured in mitigation levels, but in terms of how much mitigation is increased (or decreased, alghough unintuitive, it would basically involve investing to revert green changes). 
% Potentially, one could then also adapt the negotiation protocol -- for example, rather than negotiating about current mitigation levels, negotiations may entail rates of change (a country commits to increasing mitigation by a certain amount over the next 30 years), or net-zero targets (a country commits to eliminating emissions by a certain year, say 2060; this translates to, say, a linear increase in mitigation over time). 
% Future mitigation targets are more in line with actual targets countries set for themselves; many countries have targets such as halving emissions (w.r.t.{} some baseline) by some target date (e.g., 2030) and reaching (net) zero emissions later (e.g., by 2050 or 2060). 

\begin{ack}
We would like to thank Maikel van der Knaap, Cale Davis, Albert Bomer, Catholijn Jonker, and Holger Hoos for their time spent discussing various topics of this competition.

This research was (partly) funded by the \href{https://hybrid-intelligence-centre.nl}{Hybrid Intelligence Center}, a 10-year programme funded by the Dutch Ministry of Education, Culture and Science through the Netherlands Organisation for Scientific Research, grant number 024.004.022.
\end{ack}


\bibliographystyle{IEEEtranN}
\bibliography{ref}


\end{document}