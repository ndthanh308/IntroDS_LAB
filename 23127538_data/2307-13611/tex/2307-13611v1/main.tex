\documentclass[preprint,floatfix,aps,prd,showpacs,nofootinbib,amsmath,amssymb,amsfonts,superscriptaddress]{revtex4-1}
%\documentclass[reprint,floatfix,aps,prd,showpacs,nofootinbib,amsmath,amssymb,amsfonts,superscriptaddress]{revtex4-1}
\usepackage{bm}
\usepackage{graphicx,color}
\usepackage{physics}
\usepackage{dsfont}
\usepackage[normalem]{ulem}
\usepackage{hyperref,url}
\usepackage{mathrsfs}
\usepackage{calrsfs}
\usepackage{bbold}
\usepackage{mathtools}

%%%%
\newcommand*{\red}{\textcolor{red}}%new macro for text colour
\newcommand*{\blue}{\textcolor{blue}}
%%%%%%
\usepackage{subfig}
%\usepackage{subfigure}


\renewcommand{\vec}[1]{\mathbf{#1}}
%\usepackage[utf8]{inputenc}
\usepackage{varioref}
\usepackage[active]{srcltx}
\newcommand{\BigO}[1]{\ensuremath{\operatorname{O}\bigl(#1\bigr)}}
\expandafter\ifx\csname package@font\endcsname\relax\else
\expandafter\expandafter
\expandafter\usepackage
\expandafter\expandafter
\expandafter{\csname package@font\endcsname}%
\fi
%
\hypersetup{
	%bookmarks=false,         % show bookmarks bar?
	%pdfstartview={FitH},    % fits the width of the page to the window
	colorlinks=true,       % false: boxed links; true: colored links
	linkcolor=blue,          % color of internal links
	citecolor=blue,        % color of links to bibliography
	%filecolor=blue,      % color of file links
	%urlcolor=blue           % color of external links
}

%\usepackage{caption}
%\usepackage{subcaption}
\usepackage[section]{placeins}
\usepackage{enumitem}
\usepackage{fancybox}
%%%%%%%%%%%%%%%%%%%%%
%\labelformat{equation}{Eq.(#1)} 
\labelformat{figure}{Fig.~#1}
%%%%%%%%%%%%%%%%%%%%%
\def\be {\begin{equation}}
\def\ee {\end{equation}}
\def\bea {\begin{eqnarray}}
\def\eea {\end{eqnarray}}
\def\f {\phi}
\def\la {\label}
\def\fr {\frac}
\def\le {\left}
\def\ri {\right}
\def\O  {\Omega}
\def\p  {\pi}
\def\r  {\rho}
\def\pa {\partial}
\def\nn {\nonumber}
\def\th {\theta}
\DeclareMathOperator\ei{Ei}
\newcommand{\Lim}[1]{\raisebox{0.5ex}{\scalebox{0.8}{$\displaystyle \lim_{#1}\;$}}}

\DeclareMathAlphabet{\mathcal}{OMS}{cmsy}{m}{n}

%\usepackage{amsmath,amsfonts,amssymb,hyperref,latexsym,color,comment,inputenc,graphicx}
%\documentclass[aps,prl,preprint,superscriptaddress]{revtex4-1}
%\documentclass[aps,prl,reprint,groupedaddress]{revtex4-1}

% You should use BibTeX and apsrev.bst for references
% Choosing a journal automatically selects the correct APS
% BibTeX style file (bst file), so only uncomment the line
% below if necessary.
\bibliographystyle{apsrev4-1}

\begin{document}
\title{Real-space quantum-to-classical transition of time dependent background fluctuations}

\author{S. Mahesh Chandran} 
\email{maheshchandran@iitb.ac.in}
\affiliation{Department of Physics, Indian Institute of Technology Bombay, Mumbai 400076, India}
%
\author{Karthik Rajeev} 
\email{karthik\_rajeev@iitb.ac.in}
\affiliation{Department of Physics, Indian Institute of Technology Bombay, Mumbai 400076, India}
%
\author{S. Shankaranarayanan}
\email{shanki@iitb.ac.in}
\affiliation{Department of Physics, Indian Institute of Technology Bombay, Mumbai 400076, India}
%%%
\begin{abstract}
Understanding the emergence of classical behavior from a quantum theory is vital to establishing the quantum origin for the temperature fluctuations observed in the Cosmic Microwave Background (CMB). We show that a real-space approach can comprehensively address the quantum-to-classical transition problem in the leading order of curvature perturbations.
%While a momentum-space approach towards understanding this quantum-to-classical transition has been widely addressed in literature, we show that a real-space approach can comprehensively address the problem even in the leading order of curvature perturbations. 
To this end, we test spatial bipartitions of quadratic systems for the interplay between three different signatures of classical behavior — i) decoherence, ii) peaking of the Wigner function about classical trajectories, and iii) relative suppression of non-commutativity in observables. We extract these signatures from the covariance matrix of a multi-mode Gaussian state and address them primarily in terms of entanglement entropy and log-classicality. Through a phase-space stability analysis of spatial sub-regions via their reduced Wigner function, we ascertain that the underlying cause for the dominance of classicality signatures is the occurrence of gapped inverted mode instabilities. While the choice of conjugate variables enhances some of these signatures, decoherence studied via entanglement entropy is the stronger and more reliable condition for classicality to emerge. We demonstrate the absence of decoherence, which preempts a quantum-to-classical transition of scalar fluctuations in an expanding background in $(1+1)$-dimensions using two examples --- i) a Tanh-like expansion and ii) a de-Sitter expansion. We then extend the analysis to leading order fluctuations in $(3+1)-$dimensions to show that a quantum-to-classical transition occurs in the de-Sitter expansion and discuss the relevance of our analysis in distinguishing cosmological models.
\end{abstract}
\pacs{}
%\marginpar{\tiny\textcolor{green}{Configuration-space instead of real-space?}}
\maketitle
\section{Introduction}

The emergence of the Universe's classical behavior from its predominantly quantum mechanical early stage is one of the most intriguing phenomena in cosmology~\cite{1989HalliwellPhys.Rev.D,1989PadmanabhanPhys.Rev.D}.
This fascinating process is believed to be rooted in the dynamics shared by generic quantum systems when they interact with their environments. A crucial effect in this context is the loss of quantum coherence induced by the environment. Quantum coherence is a fundamental property of quantum mechanics that results from the superposition of orthogonal states with regard to a reference basis~\cite{2008VedralNature}. Specifically, it refers to the ability of a quantum system to maintain a well-defined quantum state over time, unaffected by external disturbances or interactions. Quantum coherence is necessary for both entanglement~\cite{2010-Eisert.etal-Rev.Mod.Phys.} and other measures of quantum correlations (like discord, negativity, and circuit complexity). It is also vital for quantum computing because quantum algorithms depend on the ability to manipulate and preserve superposition and entanglement. 

Due to the nature of closed quantum evolution, quantum coherence can never vanish permanently from a closed quantum system. However, realistic physical systems are embedded in an inaccessible or partially accessible environment. A quantum system will typically become entangled with many environmental degrees of freedom when interacting with the environment. This entanglement can in turn non-trivially affect local measurements made in the system. Quantum systems progressively lose coherence to the environment due to interactions with the external environment and can be treated as classical~\cite{1982ZurekPhys.Rev.D,1985Joos.ZehZ.Phys.B}. As a closed system, the origin of the classical world requires explanation.

%Since the Universe is a closed system, this raises the question \marginpar{Reword} of how the classical world came to be. 
%Early Universe cosmology is a physical system where the emergence of classical behavior from quantum mechanics is a particularly intriguing phenomenon~\cite{1989HalliwellPhys.Rev.D,1989PadmanabhanPhys.Rev.D}. 

Returning to the cosmological scenario, the Cosmic Microwave Background (CMB)~\cite{2020Aghanim.othersAstron.Astrophys.,2021Alam.othersPhys.Rev.D} provides essential proof of temperature variations in a relatively homogeneous distribution of matter, radiation, and (potentially) dark energy.
%captures the key observation of temperature fluctuations in a nearly-homogeneous distribution of matter and radiation content in the observable Universe. 
These inhomogeinities can be traced all the way back to the early-Universe, and are understood to be seeded by vacuum quantum fluctuations stretched to cosmological scales during a rapidly expanding \textit{inflationary} phase~\cite{1981Mukhanov.ChibisovSovietJournalofExperimentalandTheoreticalPhysicsLetters,1982Guth.PiPhys.Rev.Lett.,1982HawkingPhysicsLettersB,1982StarobinskyPhysicsLettersB,1983Bardeen.etalPhys.Rev.D}. Interestingly, such inhomogeinities, when treated as classical stochastic fluctuations seeded in the CMB after the end of inflation, provide a compelling explanation for the evolution of large-scale structures in the Universe as observed at late-times~\cite{1991Brandenberger.etalPhysicaScripta,1996Polarski.StarobinskyClassicalandQuantumGravity,1998Kiefer.etalInt.J.Mod.Phys.D,2005Lombardo.NacirPhys.Rev.D,2016Martin.VenninPhys.Rev.D}. Then the questions of how vacuum fluctuations evolved to resemble classical fluctuations, and how non-trivial signatures of such a transition can be observed, become pertinent towards establishing the quantum origin of CMB fluctuations~\cite{1990Grishchuk.SidorovPhys.Rev.D,2006Perez.etalClassicalandQuantumGravity,2011SudarskyInt.J.Mod.Phys.D,2020Ashtekar.etalPhys.Rev.D}. 

While there is no single, unified criterion for the emergence of classicality within a quantum field theoretical framework, it is largely addressed via a collection of phenomenological signatures associated with different facets of classical behavior. For instance, as a result of 
 the mixing of the super-Hubble (system) and sub-Hubble (environment) momentum-modes of fluctuations due to non-linear curvature perturbations, the super-Hubble modes are found to \textit{decohere}. A continuously evolving quantum information toolbox comprising of quantum entanglement~\cite{2012Balasubramanian.etalPhys.Rev.D,2017Kumar.ShankaranarayananPhys.Rev.D,2020Brahma.etalPhys.Rev.D,2022Brahma.etalClassicalandQuantumGravity}, quantum discord~\cite{2016Martin.VenninPhys.Rev.D,2022Martin.etalJCAP,2023Martin.etalEurophysicsLetters}, open-effective field theory (EFT) approaches~\cite{2022Brahma.etalJournalofHighEnergyPhysics,2023Burgess.etalJournalofCosmologyandAstroparticlePhysics}, in the momentum space has lately proved decisive in making robust predictions for the (extremely rapid) decoherence rate and the (highly suppressed) quantum corrections to the power spectrum resulting from this. However, these signatures are reportedly too small to be captured by current observations. Furthermore, the absence of decoherence in the leading (linear) order of curvature perturbations due to mode-decoupling, and various pitfalls associated with the emergence of classical behaviour in squeezed quantum states have been critically addressed in recent works~\cite{2021Berjon.etalPhys.Rev.D,2022Hsiang.HuUniverse,2023Agullo.etal}. 

A real-space approach towards understanding quantum-classical transition is much less explored in this context, in spite of providing a more intuitive picture of field entanglement~\cite{1986-Bombelli.etal-Phys.Rev.D,1993-Srednicki-Phys.Rev.Lett.,1995Mueller.LoustoPhys.Rev.D,2004-Calabrese.Cardy-JournalofStatisticalMechanicsTheoryandExperiment,2009-Calabrese.Cardy-JournalofPhysicsAMathematicalandTheoretical,2010-Eisert.etal-Rev.Mod.Phys.} and its underlying connection with the thermodynamic properties of the background space-time~\cite{1997-Mukohyama.etal-Phys.Rev.D,1998-Mukohyama.etal-Phys.Rev.D,2011-Solodukhin-LivingRev.Rel.,2020Chandran.ShankaranarayananPhys.Rev.D}. While this may have much to do with real-space field-entanglement being plagued by UV-divergences, recent works have proposed ways in which the sensitivity to UV-cutoff can be mitigated through field-smearing in disjoint spatial regions~\cite{2021Martin.VenninPhys.Rev.D,2021Martin.VenninJournalofCosmologyandAstroparticlePhysics,2023Agullo.etal} or scaling symmetry arguments~\cite{2023Chandran.ShankaranarayananPhys.Rev.D}. However, as we will show in this work, the biggest advantage of the real-space picture is that it captures phenomenological signatures of quantum-classical transition even up to the linear order of curvature perturbations. Therefore, the resulting quantum corrections are expected to be significantly less suppressed than in the momentum-space picture.

To identify quantum-classical transition in the real space, we test spatial bipartitions of leading order fluctuations for three different signatures of classical behavior --- i) loss of quantum coherence, which allows the system to be well described by a classical statistical ensemble, ii) peaking of the phase-space distribution of the quantum state about classical trajectories, and iii) relative suppression of non-commutativity. While these signatures may jointly manifest in the momentum-space picture for (higher-order) fluctuations propagating in a (near) de-Sitter background, they are in general inequivalent for the broader class of quantum systems~\cite{2020Ashtekar.etalPhys.Rev.D}. Therefore, the exact interplay between these concepts in real space will be relevant not only for early-Universe fluctuations but also for any quantum system with entangled spatial degrees of freedom. In turn, its applications potentially extend to laboratory simulators for time-dependent backgrounds~\cite{2022Viermann.othersNature,2022SanchezKuntz.etalPhys.Rev.D,2022TolosaSimeon.etalPhys.Rev.A} as well as table-top experiments being proposed for detecting ``quantumness" of gravity in the coming years~\cite{2017Marletto.VedralPhys.Rev.Lett.,2021Rijavec.etalNewJ.Phys.,2022Bose.etalPhys.Rev.D,2023Hanif.etal}.

The paper is organized as follows: In Section \ref{sec:cho}, we develop the tools to extract and measure the aforementioned signatures of classicality in time-dependent quadratic systems, in particular, the CHO system, in detail. Through a phase-space stability analysis of Gaussian states, we identify the presence of gapped inverted modes (in the momentum space) of the entire system as the primary trigger for the quantum-to-classical transition of subsystems (in the real space). In Section \ref{sec:dss}, we study the effects of canonical transformations on classicality criteria. In Section \ref{sec:1dcosmology}, we demonstrate the absence of quantum-to-classical transition of scalar fluctuations in an expanding background in $(1+1)-$dimensions using two examples --- i) a Tanh-like expansion and ii) a de-Sitter expansion. Section \ref{sec:3dcosmology} extends the analysis to $(3+1)-$dimensions to show that the quantum-to-classical transition occurs in the de-Sitter expansion but not in the Tanh-expansion. In Section \ref{sec:conc}, we discuss the physical interpretation of our results and future directions. Throughout this work, we use metric signature $(+,-,-,-)$ and set $\hbar=c=1$ unless otherwise specified.

\section{Quantum-to-classical transition in Time-dependent oscillators}\label{sec:cho}
In this section, we analyse the signatures of quantum-classical transition in the phase-space representation of quantum states. We begin our analysis with the coupled harmonic oscillator (CHO) system, which serves as a fundamental building block for the lattice-regularized approach to field theory that will be extensively studied in the later sections. The Hamiltonian for such a system is characterized by a frequency $\omega(t)$ and a coupling parameter $\chi(t)$, both of which are arbitrary (smooth, bounded) functions of time: 
%allowed to evolve with time:
\begin{equation}\label{cho1}
\mathscr{H}(t) =\frac{p_1^2}{2 }+\frac{p_2^2}{2 }+\frac{1}{2} \omega^2(t)\left(x_1^2+x_2^2\right)+\frac{1}{2}\chi^2(t)\left(x_1-x_2\right)^2
\end{equation}
Under the transformations $x_{\pm}=(x_1\pm x_2)/\sqrt{2}$, the above Hamiltonian reduces to:
\begin{equation}
	\label{eq:CHO-Hamil02}
	\mathscr{H}(t) =\frac{p_+^2}{2}+\frac{p_-^2}{2}+\frac{1}{2}\omega_+^2(t)x_+^2+\frac{1}{2}\omega_-^2(t)x_-^2,
\end{equation}
where the time-dependent normal modes are:
\begin{equation}
	\omega_-(t)=\sqrt{\omega^2(t)+2\chi^2(t)}; \qquad  \omega_+(t)=\omega(t).
\end{equation}
We consider the form-invariant Gaussian state (GS), which takes the form~\cite{2008LoheJournalofPhysicsAMathematicalandTheoretical}:
\begin{equation}\label{GS}
\Psi_{\rm GS}(x_+,x_-,t)=\prod_{j=\{+,-\}}\left(\frac{\omega_j(t_0)}{\pi b_j^2(t)}\right)^{1/4}\exp{-\left(\frac{\omega_j(t_0)}{b_j^2(t)}-i\frac{\dot{b}_j(t)}{b_j(t)}\right)\frac{x_j^2}{2}-\frac{i}{2}\omega_j(t_0)\tau_j(t)},
\end{equation}
where $\tau_j =\int b_j^{-2}(t) dt$. The scaling parameters $b_j$ are solutions of the non-linear Ermakov-Pinney equation \cite{Pinney_1950,1967LewisPhys.Rev.Lett.,1968LewisJournalofMathematicalPhysics,2008LoheJournalofPhysicsAMathematicalandTheoretical} :
%
\begin{equation}\label{ermakov}
	\ddot{b}_j(t)+\omega_j^2(t)b_j(t)=\frac{\omega_j^2(t_0)}{b_j^3(t)}
\end{equation}
The scaling parameters $b_j(t)$ drive the evolution of the Gaussian state as well as its deviation from the initial vacuum state defined at $t=t_0$. While the system evolves to an excited state in the corresponding instantaneous eigenbasis at later time-slices~\cite{2008Mahajan.PadmanabhanGeneralRelativityandGravitation,2018Rajeev.etalGeneralRelativityandGravitation}, its state remains pure $[\Tr\rho^2=1]$ over the course of the evolution. The dynamics of the constituent subsystems ($x_1, x_2$), on the other hand, may exhibit interesting properties by virtue of the entanglement between them. Notably, one subsystem may act as an external environment to the other, causing the latter to ``decohere", or lose some of its quantum features. To illustrate this in the case of CHO, we describe one constituent oscillator (say, $x_2$) with the help of its reduced density matrix (RDM), obtained by tracing out the other oscillator (viz., $x_1$) from the full density matrix of the CHO:

\begin{align}\label{rdm}
	\rho_2(x_2,x_2')&=\int dx_1 \Psi_{\rm GS}^*(x_1,x_2') \Psi_{\rm GS}(x_1,x_2)\nonumber\\
	&=\left(\frac{K_+K_-}{2\pi\Re(A)}\right)^{1/2}\exp{-\frac{\Gamma_1}{2}\left(x_2^2+x_2'^2\right)+\Gamma_2 x_2x_2'+i\frac{\Gamma_3}{2}\left(x_2^2-x_2'^2\right)},
\end{align}
%
where 
%
\begin{align}\label{eq:rdmredef}
	\Gamma_1&=2A_R-\left(\frac{B_R^2-B_I^2}{A_R}\right)\quad;\quad
	\Gamma_2=\frac{\abs{B}^2}{A_R}\quad;\quad
	\Gamma_3=2A_I-\frac{2B_RB_I}{A_R}\nonumber\\
 	A&=\frac{1}{4}\left[(K_++K_-)-i(L_++L_-)\right]=A_R+iA_I\nonumber\\
	B&=\frac{1}{4}\left[-(K_+-K_-)+i(L_+-L_-)\right]=B_R+iB_I\nonumber\\
        K_\pm&=\frac{\omega_\pm(t_0)}{b_\pm^2(t)}\quad;\quad
        L_\pm=\frac{\dot{b}_\pm(t)}{b_\pm(t)}
\end{align}
To identify possible signatures of a quantum-classical transition, it is useful to shift to a phase-space representation of the above reduced density matrix.

%\textcolor{blue}{Usually, in an experiment/observation, it is not possible to obtain complete information about the system. Hence, it is impossible to obtain complete information about the system. However, understanding the subsystem properties can provide sufficient information about the system. This is because the subsystems are entangled, and their time evolution rests on the time evolution of the other subsystem. We need to construct some physical quantities and study their properties to extract this information. In this section, we construct two quantities that can provide features for generic quantum systems.} \textcolor{red}{not sure where to put this para. May in introd?}

\subsection{Classicality criteria from phase-space representation}
A phase-space picture is possible within the framework of quantum mechanics with the help of Wigner-Weyl transform~\cite{1987Simon.etalPhys.Rev.A,1987Simon.etalPhysicsLettersA,2008CaseAmericanJournalofPhysics}, which maps operators to phase-space functions:
\begin{equation}
    \mathscr{W}[\hat{O}]\to O(x,p).
\end{equation}
The Wigner-Weyl transform of the density matrix $\rho(x,x')$, also known as the Wigner function, therefore provides a phase-space distribution pertaining to a quantum state:
\begin{equation}\label{eq:WignerGaussian}
    W(x_c,p)=\mathscr{W}[\hat{\rho}]=\frac{1}{2\pi}\int_{-\infty}^{\infty}dx_{\scriptscriptstyle \Delta} \rho\left(x_c-\frac{x_{\scriptscriptstyle \Delta}}{2},x_c+\frac{x_{\scriptscriptstyle \Delta}}{2}\right)e^{-ipx_{_{\scriptscriptstyle \Delta}} },
\end{equation}
where
\begin{equation}
    x_c=\frac{x+x'}{2}\quad;\quad x_{\scriptscriptstyle \Delta}=x-x'.
\end{equation}
For Gaussian states, the Wigner function takes the following form~\cite{1990MorikawaPhys.Rev.D,2008Mahajan.PadmanabhanGeneralRelativityandGravitation}:
\begin{equation}
    W(x,p)=\frac{\alpha}{2\pi\gamma}\exp{-\frac{(p-\beta x)^2}{4\gamma^2}-\alpha^2x^2}.
\end{equation}
In particular, the parameters characterizing the (reduced) Wigner function for the reduced density matrix given in \eqref{rdm}, which shall be our focus in this section, are equated below:
\begin{equation}
    \alpha^2=\Gamma_1-\Gamma_2\quad;\quad \gamma^2=\frac{\Gamma_1+\Gamma_2}{4}\quad;\quad \beta=\Gamma_3,
\end{equation}
where $\Gamma_{1}, \Gamma_{2}$ and $\Gamma_{3}$ are as defined in \eqref{eq:rdmredef}.

The Wigner function is a distribution in the phase-space that exactly captures the probabilistic nature and non-trivial effects (e.g., interference, entanglement) of quantum states in a system, in contrast to well-defined trajectories pertaining to its classical counterpart. The expectation values for observables can be calculated using averages weighted by the Wigner distribution:
\begin{equation}\label{eq:expectation}
    \langle \hat{O} \rangle = \int dx\int dp W(x,p,t)\mathscr{W}[\hat{O}]
\end{equation}
For Gaussian states, the following averages (two-point correlators), computed in the above manner, encode all information about the system:
\begin{equation}\label{eq:averages}
    \langle \{\hat{x},\hat{x}\}\rangle=\frac{1}{\alpha^2}\quad;\quad \langle \{\hat{p},\hat{p}\}\rangle=\frac{\beta^2}{\alpha^2}+4\gamma^2\quad;\quad \langle \{\hat{x},\hat{p}\}\rangle=\frac{\beta}{\alpha^2}
\end{equation}
%
To better visualize the phase-space features of a Gaussian state, it is convenient to introduce the dimensionless quadratures $P=\frac{p}{\sqrt{2\alpha\gamma}}$ and $X=\sqrt{2\alpha\gamma}x$, in terms of which the Wigner function takes the general form:
\begin{equation} W(X,P)=\frac{\delta_{QD}}{\pi}\exp\left[-\delta_{QD}\left\{\left(P-\frac{1}{\delta_{CC}}X\right)^2+X^2\right\}\right]\quad;\quad 0\leq W\leq \frac{\delta_{QD}}{\pi}
\end{equation}
where $\delta_{QD}$ is referred to as the \textit{degree of quantum decoherence} and 
$\delta_{CC}$ is referred to as the \textit{degree of classical correlations}. The Wigner function is therefore fully characterized by these two dimensionless parameters that capture distinct properties of the quantum state, as outlined below~\cite{1990MorikawaPhys.Rev.D}:
\begin{itemize}
    \item \textbf{Degree of Quantum Decoherence} $\delta_{QD}$ : This measure coincides with the purity of the reduced density matrix $\rho_2$ given in \eqref{rdm}:
    \begin{equation}
        \delta_{QD}\equiv\frac{\alpha}{2\gamma}=\Tr\rho_{2}^2=\sqrt{\frac{4K_+K_-}{(K_++K_-)^2+(L_+-L_-)^2}}
    \end{equation}
Consequently, $\delta_{QD}\in[0,1]$. 
The upper extreme is saturated by the pure states, for which $\delta_{QD}=1$. On the other hand, when the effects of an external environment are significant, the state may undergo decoherence, i.e., the non-diagonal entries drop to zero and the reduced density matrix resembles a classical statistical ensemble. This case corresponds to the limit $\delta_{QD}\to 0$.
    
\item \textbf{Degree of Classical Correlations} $\delta_{CC}$ : This measure is associated with the sharpness of squeezing of the Wigner function:
\begin{eqnarray}\label{eq:dccc}
\delta_{CC}\equiv \abs{\frac{2\alpha\gamma}{\beta}} &=& \frac{\sqrt{K_+K_-\left[(K_++K_-)^2+(L_+-L_-)^2\right]}}{K_+L_-+K_-L_+} \nonumber \\
& = & \frac{2 K_+K_-}{K_+L_-+K_-L_+} \sqrt{\frac{\left[(K_++K_-)^2+(L_+-L_-)^2\right]}{4 K_+K_-}} 
\nonumber \\
&=& \frac{1}{\delta_{QD}} 
\left(\frac{2 K_+K_-}{K_+L_-+K_-L_+}\right)  
\end{eqnarray} 
%
For the CHO, $\delta_{CC}$ is also directly related to the classicality parameter ($\mathscr{C}$) proposed in \cite{2008Mahajan.PadmanabhanGeneralRelativityandGravitation}. Therein, $\mathscr{C}$ was introduced as a more intuitive measure for quantifying classicality, viz., in terms of the width of the Wigner function around the classical phase-space trajectory. Hence,
    \begin{equation}\label{eq:cpara}
        \mathscr{C}\equiv\frac{\langle xp\rangle_W}{\sqrt{\langle p^2\rangle_W\langle x^2\rangle_W}}=\frac{1}{\sqrt{1+\delta_{CC}^2}}
    \end{equation}
%
It follows that the classicality parameter $\mathscr{C}\in [0,1]$ (or $\delta_{CC}\in [0,\infty)$). The lower bound corresponds to the ``quantum" limit ($\delta_{CC}\to\infty$) wherein the Wigner function becomes separable in position and momentum. This follows from the uncertainty principle wherein fixing the value of $x$ can amplify the error in $p$ and vice versa, resulting in probability distributions along $x$ and $p$ that are uncorrelated. On the other hand, the upper bound corresponds to the classical limit ($\delta_{CC}\to 0$) wherein the Wigner function is no longer separable in $x$ and $p$, and its peak coincides with well-defined classical phase-space trajectories.
\end{itemize}

As we remarked earlier, there is a convenient geometrical picture that captures the manner in which the above parameters fully characterize a Gaussian state. To visualize this, consider a particular `slice' of the Wigner function that corresponds to an ellipse in the phase space, referred to as a Wigner ellipse, described below in terms of rotated co-ordinates $\tilde{X}$ and $\tilde{P}$:
\begin{align}
    &\frac{\tilde{X}^2}{a^2}+\frac{\tilde{P}^2}{b^2}=\frac{1}{\delta_{QD}}\log{\frac{\delta_{QD}}{\pi W}}\quad;\quad \begin{bmatrix} \tilde{X}\\\tilde{P}\end{bmatrix}=\begin{bmatrix}\cos\theta &\sin\theta\\-\sin\theta&\cos\theta\end{bmatrix} \begin{bmatrix}X\\P\end{bmatrix}\nonumber\\
    &a^2=1+\frac{1}{2\delta_{CC}^2}\left\{1+\sqrt{1+4\delta_{CC}^2}\right\}\nonumber\\
    &b^2=1+\frac{1}{2\delta_{CC}^2}\left\{1-\sqrt{1+4\delta_{CC}^2}\right\}\nonumber\\
    &\theta=\sin^{-1}\left[\sqrt{\frac{1}{2}\left\{1+\frac{1}{1+4\delta_{CC}^2}\right\}}\right]
\end{align}
where $a$ and $b$ are the length of semi-major/minor axes of the rotated ellipse and $\theta$ is the squeezing angle. A useful such slice of the Wigner function to look at is at the half of its peak, wherein the corresponding ellipse serves as a 2D-generalization of the FWHM (Full-width at Half-maxima) for Gaussian/normal distributions. The equation for the corresponding Wigner ellipse would then take the form:
 \begin{equation}
    \frac{\tilde{X}^2}{a^2}+\frac{\tilde{P}^2}{b^2}=\frac{\log{2}}{\delta_{QD}}
\end{equation}
When $\delta_{CC}\to\infty$ (or $\mathscr{C}\to0$), we see that the Wigner ellipse reduces to a circle ($a=b=1$, $\theta=\pi/4$) corresponding to a state that is time-independent or at the beginning of its evolution $t=t_0$. 

The phase-space picture of the quantum state can therefore be outlined as follows: (i) Wigner function for the Gaussian state is fully characterized by dimensionless parameters $\delta_{QD}$ and $\delta_{CC}$ (or $\mathscr{C}$); (ii) State purity $\delta_{QD}$ determines the amplitude features of the Wigner function. For instance, its peak (maxima) and spread (area of the Wigner ellipse at half-maxima) are given by $\delta_{QD}/\pi$ and $\pi\log{2}/\delta_{QD}$ respectively; iii) Classicality parameter $\mathscr{C}$ determines the extent of squeezing (the ratio of semi-major/semi-minor axes $a$ and $b$) and squeezing angle ($\theta$) of the distribution. From here onwards, we stick to classicality parameter $\mathscr{C}$ as the characteristic measure for squeezing, since it is a fundamental feature of the covariance matrix as we will see in the next subsection, and has a natural extension for large subsystem sizes.

Using this phase-space picture, we may now analyze the conditions required for classicality to emerge in a Gaussian state~\cite{1990MorikawaPhys.Rev.D,1996Polarski.StarobinskyClassicalandQuantumGravity}:

\begin{itemize}
    \item $\mathscr{C}\to 1$ : In this limit, the Wigner function undergoes a runaway squeezing about the classical phase-space trajectory ($p=\beta x$) of the system. %This limit is only possible via an explicit time-dependence of the Hamiltonian wherein modes become inverted (as we will see in the next section).
    \item $\delta_{QD}\to 0$ : In this limit, the subsystem experiences a runaway decoherence due to its interaction with the environment (here, the other oscillator), causing the amplitude of the Wigner function to fall and spread out over the entire phase space. %Here, the system (oscillator $x_1$) decoheres due to its interaction with the environment (the oscillator $x_2$). %This limit can manifest both in time-dependent ($t\to\infty$ in the presence of instabilities) and time-independent cases (zero mode limit $\omega/\alpha\to0$).
\end{itemize}
It is to be noted that this notion of ``classicality" fundamentally differs from taking the formal limit $\hbar\to0$~\cite{2022Hsiang.HuUniverse}. To illustrate this, let us briefly put back in the Planck's constant which was set to $\hbar=1$ and consider the Wigner function as well as the  marginal probability distributions along $X$ and $P$ coordinates separately:
\begin{align}
W(X,P)&=\frac{\delta_{QD}}{\pi\hbar}\exp\left[-\frac{\delta_{QD}}{\hbar}\left\{\frac{\tilde{X}^2}{a^2}+\frac{\tilde{P}^2}{b^2}\right\}\right]\quad;\quad \begin{bmatrix} \tilde{X}\\\tilde{P}\end{bmatrix}=\begin{bmatrix}\cos\theta &\sin\theta\\-\sin\theta&\cos\theta\end{bmatrix} \begin{bmatrix}X\\P\end{bmatrix}\nonumber\\
f(X)&=\int dP W(X,P)=\frac{1}{\sqrt{2\pi}\sigma_X}\exp{-\frac{X^2}{2\sigma_X^2}}\quad;\quad\sigma_X=\sqrt{\frac{\hbar}{2\delta_{QD}}}\nonumber\\
g(P)&=\int dX W(X,P)=\frac{1}{\sqrt{2\pi}\sigma_P}\exp{-\frac{P^2}{2\sigma_P^2}}\quad;\quad\sigma_P=\sqrt{\frac{\hbar}{2\delta_{QD}(1-\mathscr{C}^2)}}
\end{align}
It is interesting to note here that while purity $\delta_{QD}$ affects the variance for distributions in both $X$ and $P$, classicality parameter $\mathscr{C}$ only affects the variance in $P$. The measurement error in the $(x,p)$-coordinates is therefore:
\begin{equation}
    \sigma_x\sigma_p=\sigma_X \sigma_P=\frac{\hbar}{2\delta_{QD}\sqrt{1-\mathscr{C}^2}}\geq\frac{\hbar}{2}
\end{equation}
We see that the uncertainty principle is saturated when $\delta_{QD}=1$ and $\mathscr{C}=0$, corresponding to a pure-state at $t=t_0$. The classical limit $\hbar\to 0$, as seen from above, corresponds to the case where the uncertainty (as well as the commutator of conjugate variables) vanishes, and the phase-space distributions are highly localized  ($\delta-$functions)~\cite{1996RipamontiJournalofPhysicsAMathematicalandGeneral,1996Polarski.StarobinskyClassicalandQuantumGravity}:
\begin{equation}
    W\to \delta\left(\frac{\tilde{X}}{a}\right)\delta\left(\frac{\tilde{P}}{b}\right)\quad;\quad f\to\delta(X)\quad;\quad g\to\delta(P)
\end{equation}
Morikawa's classicality criteria on the other hand points to a divergent uncertainty in both $x$ and $p$ measurements, wherein the phase-space distributions are less and less localized (\ref{fig:Wigfun}). Despite this contrast, Morikawa's criteria leads to a notion of ``quasi-classicality"~\cite{1996Polarski.StarobinskyClassicalandQuantumGravity} \textit{within} the framework of quantum mechanics in the following sense --- i) decoherence essentially leads to a (reduced) density matrix that resembles a classical statistical ensemble, and ii) squeezing further aligns the peaks of (reduced) Wigner function along classical phase-space trajectories. The overall implication is that the features that make a state \textit{distinctly} quantum are greatly suppressed. For instance, let us look at the Wigner-Weyl transform of the following observables whose expectation values are to be calculated via \eqref{eq:expectation}:
\begin{align}\label{eq:weyl}
    &\mathscr{W}\left[\hat{x}\hat{p}+\hat{p}\hat{x}\right]= \mathscr{W}\left[2 \mathcal{S}(\hat{x}\hat{p})\right]=2xp\nonumber\\
    &\mathscr{W}\left[\hat{x}^2\hat{p}^2+\hat{p}^2\hat{x}^2\right]= \mathscr{W}\left[2\mathcal{S}(\hat{x}^2\hat{p}^2)+[\hat{x},\hat{p}]^2\right]=2x^2p^2-\hbar^2\nonumber\\
    &\mathscr{W}[f(\hat{x},\hat{p})]=\mathscr{W}\left[\mathcal{S}\left(f(\hat{x},\hat{p})\right)+g\left([\hat{x},\hat{p}]\right)\right]=f(x,p)+\tilde{g}(\hbar),
\end{align}
where $\mathcal{S}$ is a symmetrizer for combinations of $\hat{x}$ and $\hat{p}$ operators, and satisfies $\mathscr{W}\left[\mathcal{S}(\hat{x}^n\hat{p}^m)\right]=x^np^m$~\cite{2005Braunstein.LoockRev.Mod.Phys.,1995Leonhardt.PaulProgressinQuantumElectronics}. The Weyl-transform of a Hermitian, polynomial combination $f(\hat{x},\hat{p})$ of conjugate variables is therefore real-valued phase-space functions that can be split into a ``classical" contribution (from the symmetrizer) and a ``quantum" contribution (from the commutator)~\cite{2008CaseAmericanJournalofPhysics}. Since all higher-order correlators are polynomial functions of two-point correlators for a Gaussian state, it is sufficient to perform a comparison using expectation values of the commutator and the anti-commutator (i.e., symmetrizer at second order):
\begin{equation}\label{eq:commratio}
R_{(x,p)}\equiv\abs{\frac{\langle[\hat{x},\hat{p}]\rangle}{\langle\{\hat{x},\hat{p}\}\rangle}}=\delta_{QD}\delta_{CC}=\frac{\delta_{QD}}{\mathscr{C}}\sqrt{1-\mathscr{C}^2},
\end{equation}
The above ratio compares the strength of quantum and classical contributions over the course of state evolution~\cite{2020Ashtekar.etalPhys.Rev.D}. We see that Morikawa's classicality criteria ($\delta_{QD}\to 0$ and $\mathscr{C}\to1$) leads to an \textit{extremely rapid} suppression of non-commutativity, in favour of the aforementioned notion of quasi-classicality. This further implies that even if we are able to somehow measure observables that directly capture quantum signatures as in \eqref{eq:weyl}, these signatures will be tremendously suppressed by squeezing and/or decoherence, leaving little room to distinguish between a quantum and classical origin for observations. It is also to be noted that the classicality criteria places \textit{stronger} conditions than $R_{(x,p)}\to 0$, requiring simultaneous decoherence ($\delta_{QD}\to0$) \textit{and} squeezing ($\mathscr{C}\to1$). We again set $\hbar=1$ for the rest of the paper, and in the next subsection we will see how the classicality criteria can be reconciled with a stability analysis of the quantum state in the phase-space.


\subsection{Phase space stability analysis}
The vacuum states are typically well-defined when the Hamiltonian becomes time-independent. Hence in the case of CHO we consider an evolution in $\omega(t)$ and $\chi(t)$ that are asymptotically constant. In Ref. \cite{2023Chandran.ShankaranarayananPhys.Rev.D}, the authors showed that the asymptotic values of the normal modes decided the late-time stability of the system, the signatures of which were obtained from various correlation measures. Similarly, we may consider the stability analysis of the quantum state in the phase-space via the Wigner function. Let us set the values of the two normal modes --- $\omega_+^2(t)=\omega^2(t)$ and $\omega_-^2(t)=\omega^2(t)+2\chi^2(t)$ --- to constant values $u_+^2$ and $u_-^2$ ($u_+^2 \leq u_-^2$), respectively at late-times. In the asymptotic future ($t\to\infty$), the Ermakov equation, therefore, takes the following form: 
%
\begin{equation}
	\ddot{b}_j(t)+u_j^2b_j(t) \sim \frac{\omega_j^2(t_0)}{b_j^3(t)}\quad;\quad j=+,-
\end{equation}
Since the co-efficient in the second term of the above equation is time-independent, we can obtain the following solutions~\cite{2017Ghosh.etalEPLEurophysicsLetters}:
%
\begin{equation}
\label{eq:bjlongtime}
	b_j(t)\sim \sqrt{1+\left(\frac{\omega_j^2(t_0)}{u_j^2}-1\right)\sin^2{u_jt}}\quad;\quad \dot{b}_j(t)\sim \left(\omega_j^2(t_0)-u_j^2\right)\frac{\sin{2u_jt}}{2u_jb_j(t)} \, .
\end{equation}
We now look at various stability regimes of these solutions below and track its features in the phase-space picture (see \ref{fig:Wigfun}) :
\begin{itemize}
    \item Stable Modes $u_{j}^2>0$ : Scaling parameters $\{b_{j}\}$ are oscillatory and bounded.
    \item Zero Modes $u_j^2=0$ :
            \begin{equation}\label{b:zero}
		b_j(t)\sim \omega_j(t_0)t\quad;\quad \dot{b}_j(t)\sim \omega_j(t_0)
	\end{equation}
 Suppose $\omega_+$ is a zero mode and $\omega_-$ is a stable mode. At late-times, we have:
 %
 \begin{equation}
     \delta_{QD}\sim \frac{2}{t}\sqrt{\frac{K_-}{(K_-^2+L_-^2)\omega_+(t_0)}}\quad;\quad \mathscr{C}\sim \frac{1}{\sqrt{1+\frac{K_-^2+L_-^2}{K_-\omega_+(t_0)}}}
 \end{equation}
 We see that the purity falls to zero as $t\to\infty$, whereas classicality parameter retains its oscillatory behaviour about a value between 0 and 1, i.e., there is no runaway squeezing.

%For $\delta_{QD}\to0$, we see that:
%\begin{equation}
    %S\sim \log{\frac{1}{\delta_{QD}}}\sim \log{t}
%\end{equation}

    \item Inverted Modes $u_j^2<0$ : At late times, the solutions \eqref{eq:bjlongtime} further reduce to :
	\begin{equation}\label{b:inverted}
		b_j(t)\sim c_je^{v_jt}\quad;\quad \dot{b}_j(t)\sim c_jv_je^{v_jt} \quad;\quad c_j=\frac{1}{2}\sqrt{1+\frac{\omega_j^2(t_0)}{v_j^2}}
	\end{equation}
where we have defined $u_j=iv_j$. When both modes are inverted, we see that $v_+\geq v_-$ in general, and as a result:
\begin{equation}
    \lim_{t\to\infty}\delta_{QD}\sim \begin{cases} \frac{2\sqrt{\omega_+(t_0)\omega_-(t_0)}}{c_+c_-(v_+-v_-)}e^{-(v_++v_-)t} &v_+>v_-\\\frac{2c_+c_-\sqrt{\omega_+(t_0)\omega_-(t_0)}}{c_+^2\omega_-(t_0)+c_-^2\omega_+(t_0)} & v_+\to v_-    
    \end{cases}
\end{equation}
The result gives us two distinct cases --- if $v_+>v_-$ (gapped), the long-time limit will always result in a purity that exponentially decays to zero, thereby exhibiting rapid decoherence. On the other hand, if the inverted modes converge asymptotically (i.e., ungapped), the subsystem is protected from further decoherence (this conditions for purity saturation is much more general, as worked out in Appendix \ref{app:saturation}). The degree of classical correlation $\delta_{CC}$, on the other hand, has the following late-time behaviour:
\begin{equation}
    \lim_{t\to\infty}\mathscr{C}^2\sim \begin{cases} 1-\frac{\omega_+(t_0)c_-^2}{\omega_-(t_0)c_+^2}\left(1-\frac{v_-}{v_+}\right)^2 e^{-2(v_+-v_-)t} &v_+>v_-\\1-\frac{\omega_+(t_0)\omega_-(t_0)}{c_+^2c_-^2v^2}e^{-4vt} & v_+\to v_-\sim v  
    \end{cases}
\end{equation}
We see that at late times $\mathscr{C}\to1$, with the squeezing being much faster in the ungapped case than in the gapped case. Therefore, we see that the only case that simultaneously results in both rapid decoherence and runaway squeezing, thereby satisfying the classicality criteria, is when the system develops gapped inverted modes.
\end{itemize}
%For $\delta_{QD}\to0$ and $v_+\neq v_-$, we see that:
%\begin{equation}
 %   S\sim \log{\frac{1}{\delta_{QD}}}\sim (v_++v_-)t
%\end{equation}

% Figure environment removed


From \ref{fig:Wigfun} and Table \ref{tab:CHO}, we see that for the CHO the only case that asymptotically satisfies the classicality criteria is when the modes are inverted and gapped. Here, the inverted modes cause an exponentially fast suppression of non-commutativity as per \eqref{eq:commratio}. Interestingly, we see that this is also the only regime where entanglement entropy ($S=-\Tr\rho_{red}\log{\rho_{red}}$) mimics its classical counterpart, the Kolmogorov-Sinai entropy~\cite{2018Hackl.etalPhys.Rev.A,2023Chandran.ShankaranarayananPhys.Rev.D}:
\begin{equation}
    S(t)\sim h_{KS}t\quad;\quad h_{KS}=\sum_i{\lambda_i},
\end{equation}
where growth rate $h_{KS}$ is the sum of all positive Lyapunov exponents. Therefore, we may argue that this is indeed the regime where an asymptotic quantum-classical transition occurs in the case of a CHO. While Morikawa's criteria in CHO has been explored to varying extents in previous works~\cite{1996Polarski.StarobinskyClassicalandQuantumGravity,2014Lochan.etalGeneralRelativityandGravitation,2022AndrzejewskiQuantumInformationProcessing}, our approach reconciles it with a stability analysis of the quantum state in its phase-space representation, in a way that is also scale-able to larger subsystem sizes. To the best of our knowledge, this criterion has not been extended in this way, and we will describe how to calculate it in next subsection.
%Such an extension of this criteria has not been explored before in literature to the best of our knowledge, and we will detail its calculation in the following subsection.


\begin{table}[!htb]
	\centering
	\resizebox{0.7\textwidth}{!}{%
		\begin{tabular}{@{}|c|c|c|@{}}
			\toprule
			Asymptotics & $\delta_{QD}\not\to0$ & $\delta_{QD}\to0$ \\  
			\toprule
			$\mathscr{C}\not\to 1$ & Stable modes ($u_\pm^2>0$) & Zero mode ($u_+^2\to0$)  \\
   &$[R_{(x,p)}\not\to0]$ & $[R_{(x,p)}\to0]$\\[6pt] \hline
		$\mathscr{C}\to1$ & Inverted modes ($u_\pm^2<0$) & Inverted modes ($u_\pm^2<0$)\\& Case 1 : $v_+\to v_-$ & Case 2 : $v_+\neq v_-$\\
  &$[R_{(x,p)}\to0]$ & $[R_{(x,p)}\to0]$
   \\[6pt] 
			\toprule
		\end{tabular}
	}
	\caption{Testing classicality criteria for various stability regimes in CHO.}
	\label{tab:CHO}
\end{table}
\subsection{Classicality criteria for $N$ oscillators}
For a system of $N$ oscillators, the reduced density matrix after integrating out $m$ oscillators will have the following form as derived in Appendix \ref{App:NHO}:
\begin{equation}
	\rho_{out}=\sqrt{\frac{\det{1-{\Gamma_D}}}{\pi^{N-m}}}\exp{-Y^T\left( \frac{1-i\tilde{\Gamma}_3}{2}\right)Y-Y'^T\left( \frac{1+i\tilde{\Gamma}_3}{2}\right)Y'+Y'^T\Gamma_D Y},
\end{equation}
where,
\begin{equation}
\Gamma=W^T \Gamma_DW \, \quad \tilde{\Gamma}_3=W\gamma_D^{-1/2}V\Gamma_3 V^T\gamma_D^{-1/2}W^T \, .   
\end{equation}
and
\begin{align}\label{eq:matrices1}
	\Gamma_1&=C-\frac{1}{2}B^TA^{-1}B+\frac{1}{2}Z_B^T A^{-1}Z_B\nonumber\\
	\Gamma_2&=\frac{1}{2}B^TA^{-1}B+\frac{1}{2}Z_B^TA^{-1}Z_B\nonumber\\
	\Gamma_3&=Z_C-Z_B^TA^{-1}B	
\end{align}
The Wigner function for the $N-m$ reduced oscillator system is given by:
\begin{align}\label{eq:WigneNHO}
    W(X,P)&=\frac{1}{(2\pi)^{N-m}}\int_{-\infty}^{\infty}dx_{\scriptscriptstyle \Delta} \rho\left(X-\frac{x_{\scriptscriptstyle \Delta}}{2},X+\frac{x_{\scriptscriptstyle \Delta}}{2}\right)e^{-iP^Tx_{_{\scriptscriptstyle \Delta}} }\nonumber\\
    &=\frac{\Delta_{QD}^{1/2}}{\pi^{N-m}}\exp\left[-X^T(1-\Gamma_D)X-\left\{P^T-X^T\tilde{\Gamma}_3\right\}\frac{1}{1+\Gamma_D}\left\{P-\tilde{\Gamma}_3X\right\}\right],
\end{align}
where we obtain the purity spectrum $\delta_{QD}^{(j)}$ of the reduced state in the basis that diagonalizes the $\Gamma$ matrix, along with the overall purity $\Delta_{QD}$:
\begin{equation}\label{eq:dqdent}
\delta_{QD}^{(j)}=\sqrt{\frac{1-(\Gamma_D)_{jj}}{1+(\Gamma_D)_{jj}}}\quad;\quad \Delta_{QD}=\prod_j\delta_{QD}^{(j)}
\end{equation}
While the overall purity $\Delta_{QD}$ of the reduced state appears to be a natural extension of the measure $\delta_{QD}$ in CHO, the \textit{entanglement entropy} of the subsystem is a richer measure of decoherence for larger subsystem sizes~\cite{2003Serafini.etalJournalofPhysicsBAtomicMolecularandOpticalPhysics}. The entanglement entropy for the subsystem is obtained as follows~\cite{1993-Srednicki-Phys.Rev.Lett.,2023Chandran.ShankaranarayananPhys.Rev.D}:
\begin{equation}\label{eq:EntNHO}
    S=\sum_{j=m+1}^NS_j\quad;\quad S_j=-\log{[1-\xi_j]}-\frac{\xi_j}{1-\xi_j}\log{\xi_j}\quad;\quad \xi_j=\frac{1-\delta_{QD}^{(j)}}{1+\delta_{QD}^{(j)}}
\end{equation}
The key to reformulating the classicality criteria for large subsystem sizes lies in the \textit{covariance matrix} of the reduced system. This is because for Gaussian states, all information about correlations are captured in the covariance matrix, which can be effectively used to  measure both decoherence as well as squeezing even for large system sizes. In order to see this, let us first write down the covariance matrix corresponding to the reduced Wigner function as follows~\cite{2010-Eisert.etal-Rev.Mod.Phys.}:
\begin{equation}
    \Sigma=\begin{bmatrix}
        \sigma_{XX}&\sigma_{XP}\\\sigma_{XP}^T&\sigma_{PP}    \end{bmatrix}\,;\, (\sigma_{XX})_{ij}=\langle \{x_i,x_j\}\rangle\,;\,(\sigma_{XP})_{ij}=\langle \{x_i,p_j\}\rangle\,;\,(\sigma_{PP})_{ij}=\langle \{p_i,p_j\}\rangle
\end{equation}
Upon evaluating the correlators, we obtain the following matrix elements:
\begin{equation}
    (\sigma_{XX})_{ij}=\frac{\delta_{ij}}{1-\Gamma_j}\quad;\quad (\sigma_{XP})_{ij}=\frac{(\tilde{\Gamma}_3)_{ij}}{1-\Gamma_i}\quad;\quad (\sigma_{PP})_{ij}=(1+\Gamma_j)\delta_{ij}+\sum_k\frac{(\tilde{\Gamma}_3)_{ik}(\tilde{\Gamma}_3)_{jk}}{1-\Gamma_k}
\end{equation}
Writing down the conditional covariance matrix~\cite{2012Adesso.etalPhys.Rev.Lett.} for the reduced system is a useful step towards identifying decoherence from the covariance matrix:
\begin{equation}
    \Sigma_{P|X}=\sigma_{PP}-\sigma_{XP}^T\sigma_{XX}^{-1}\sigma_{XP}\quad;\quad \left(\Sigma_{P|X}\right)_{ij}=(1+\Gamma_j)\delta_{ij}
\end{equation}
It is now easy to see that the $N\times N$ block determinant matrix ($\det\Sigma$) for the $2N\times2N$ covariance matrix gives the purity spectrum:
\begin{equation}   
[\det\Sigma]_{ij}=\left[\sigma_{XX}^{1/2}\Sigma_{P|X}\sigma_{XX}^{1/2}\right]_{ij}=\frac{\delta_{ij}}{\left[\delta_{QD}^{(j)}\right]^2}=\lambda_{j}\delta_{ij}
\end{equation}
The entanglement entropy is therefore related to the eigenvalues $\{\lambda_j\}$ of the block determinant ($\det\Sigma$) as follows, consistent with the expression in \eqref{eq:EntNHO}:
\begin{equation}
    S=\sum_{j=1}^{N-m}S_j\quad;\quad S_j=\left(\frac{\sqrt{\lambda_j}+1}{2}\right)\log\left(\frac{\sqrt{\lambda_j}+1}{2}\right)-\left(\frac{\sqrt{\lambda_j}-1}{2}\right)\log\left(\frac{\sqrt{\lambda_j}-1}{2}\right)
\end{equation}
On the other hand, in order to generalize the classicality parameter (that measures squeezing of the Wigner function) for large subsystem sizes, let us first look at the determinant of the reduced covariance matrix for the CHO using \eqref{eq:averages}:
\begin{equation}
    \frac{1}{4}\left(\det \Sigma\right)_{red}=\langle x^2 \rangle\langle p^2 \rangle-\langle xp\rangle^2=\frac{1}{4\delta_{QD}^2}
\end{equation}
Upon rescaling this by $\langle x^2 \rangle \langle p^2\rangle$ and rearranging using \eqref{eq:cpara}, we get:
\begin{equation}
    \mathscr{C}=\sqrt{1-\frac{1}{4\langle x^2 \rangle\langle p^2\rangle\delta_{QD}^2}}=\sqrt{1-\frac{\det\Sigma_{red}}{\sigma_{XX}\sigma_{PP}}}
\end{equation}
We now generalize this measure in terms of \textit{classicality matrix} $C$ defined as follows, and propose it as a powerful tool towards quantifying the extent of squeezing in a multi-mode covariance matrix:
\begin{align}
    C&=\sqrt{1-[\sigma_{XX}^{1/2}\sigma_{PP}\sigma_{XX}^{1/2}]^{-1/2}(\det\Sigma)[\sigma_{XX}^{1/2}\sigma_{PP}\sigma_{XX}^{1/2}]^{-1/2}}\nonumber\\   &=\sqrt{[\sigma_{XX}^{1/2}\sigma_{PP}\sigma_{XX}^{1/2}]^{-1/2}\sigma_{XX}^{1/2}\sigma_{XP}^T\sigma_{XX}^{-1}\sigma_{XP}\sigma_{XX}^{1/2}[\sigma_{XX}^{1/2}\sigma_{PP}\sigma_{XX}^{1/2}]^{-1/2}}
\end{align}
The above matrix captures the relative contribution of the off-diagonal blocks $\sigma_{XP}$ and $\sigma_{PX}$ with respect to the diagonal blocks $\sigma_{XX}$ and $\sigma_{PP}$ in the covariance matrix, i.e., it measures how sharply the multi-variate reduced Wigner function squeezes. For the case of CHO, the above result reduces to \eqref{eq:cpara}. However, for a larger subsystem size, we construct its determinant (a matrix invariant) using the eigenvalues $\{C_j\}$. In order to have a better comparison with entanglement entropy of the same subsystem, we further rewrite the determinant in terms of what we refer from here on out as ``log classicality" $LC(t)$:
\begin{equation}
    \mathscr{C}=\prod_{j}C_j\quad;\quad LC=-\log\sqrt{1-\mathscr{C}^2},
\end{equation}
The above measure is well-behaved, and is a characteristic feature of a multi-mode covariance matrix. Entanglement entropy and log classicality are therefore insightful single-valued measures that extract the extent of decoherence and squeezing directly from the covariance matrix associated with a given (multi-mode) quantum state. The criteria for asymptotic quantum-classical transition can hence be reformulated for large subsystem sizes as follows:
\begin{equation}\label{eq:criteria}
    \lim_{t\to\infty}S\to\infty \quad;\quad \lim_{t\to\infty} LC\to\infty
\end{equation}
In the above limit, a multi-mode generalization for the ratio defined in \eqref{eq:commratio}
 is also expected to vanish. However, since it is a weaker requirement for classicality than \eqref{eq:criteria}, we do not address such a generalization in this work.
 
 For CHO, we see that the inverted modes lead to the following leading order behaviour at late-times, with only the gapped ($v_+>v_-$) case satisfying the classicality criteria:
\begin{equation}
    \lim_{t\to\infty}S\sim \begin{cases}
        (v_++v_-)t &v_+>v_-\\ const. & v_+\to v_-
    \end{cases}\quad;\quad \lim_{t\to\infty}LC\sim \begin{cases}
        (v_+-v_-)t &v_+>v_-\\2v_{\pm}t & v_+\to v_- 
    \end{cases}
\end{equation}
%
Having successfully generalized the classicality criteria for multi-mode Gaussian states, we may now utilize this to identify quantum-classical transition in physical scenarios modeled by dynamically evolving harmonic lattices. The criteria, however, may have a possible caveat --- even for a fixed bipartition, it depends on the choice of conjugate variables and is subject to change under canonical transformations. We will address this in much detail in the next section before proceeding to early-Universe fluctuations.

% % Figure environment removed

% {% Figure environment removed

\section{Classicality criteria and Canonical transformations}\label{sec:dss}
In general, a scalar field propagating in a background space-time may be quantized in different co-ordinate settings. The respective conjugate variables are related via canonical transformations, and ideally we require a classicality criteria that is independent of the choice of these variables. While a lot of progress has been made in identifying quantum-classical transition particularly in the two-mode squeezed-state representation in the momentum space~\cite{1996Polarski.StarobinskyClassicalandQuantumGravity,2022Martin.etalJCAP,2023Martin.etalEurophysicsLetters}, the choice of conjugate variables is found to play a crucial role, i.e., a system identified as ``classical" can be made ``quantum" with a simple canonical transformation~\cite{2020Grain.VenninJournalofCosmologyandAstroparticlePhysics}. Similarly, while in Section \ref{sec:cho} we were able to successfully extend Morikawa's classicality criteria to multi-mode Gaussian states, we see in this section that it is not completely independent of the choice of conjugate variables either. 

\subsection{Time-dependent Harmonic Oscillator}
To investigate the effects of canonical transformations, let us consider the Hamiltonian of a time-dependent oscillator as follows:
\begin{equation}
    \mathscr{H}^{(I)}(\eta)=\frac{P^2}{2}+\frac{\omega_{_{I}}^2(\eta)X^2}{2}=\frac{P^2}{2}+\frac{a^2(\eta)\Omega^2(\eta)X^2}{2} \,  ,
\end{equation}
%
where we now use $\eta$ as the time coordinate for comparison. The wave-function that describes the system is a solution to the time-dependent Schr\"{o}dinger equation, and unitarily evolves from an initial state defined at $\eta=\eta_0$ as follows~\cite{1967LewisPhys.Rev.Lett.,2009CampbellJ.Phys.A.}:
\begin{equation}
    \Psi(\eta)=\exp{-i\int_{\eta_0}^\eta \mathscr{H}^{(I)}(\eta')d\eta'}\Psi(\eta_0)
\end{equation}
Let us now transform the Hamiltonian $\mathscr{H}^{(I)}(\eta) \to \mathscr{H}^{(II)}(\eta)$ as follows:
%
\begin{equation}\label{eq:hamconnection}
    \mathscr{H}^{(II)}(t)=\frac{\mathscr{H}^{(I)}(\eta(t))}{a(\eta(t))}=\frac{P^2}{2a(\eta(t))}+\frac{a(\eta(t))\Omega^2(\eta(t))X^2}{2}
\end{equation}
With the above rescaling, the time-evolution of a particular state can be preserved by also rescaling the time-coordinate appropriately:
\begin{equation}
    \int_{\eta_0}^\eta \mathscr{H}^{(I)}(\eta)d\eta=\int_{t_0}^t \mathscr{H}^{(II)}(t) dt\quad;\quad t=\int a(\eta)d\eta
\end{equation}
Now, we employ the following canonical transformations with respect to $\mathscr{H}^{(II)}$~\cite{1987PedrosaJournalofMathematicalPhysics}:
\begin{equation}\label{eq:cantrans}
    X=\frac{x}{\sqrt{a(t)}} \quad;\quad P=\sqrt{a(t)}p-\frac{\dot{a}(t)}{2\sqrt{a(t)}}x
\end{equation}
The resultant Hamiltonian is:
\begin{equation}\label{eq:frequencyrelation}
    \mathscr{H}^{(II)}(t)=\frac{p^2}{2}+\frac{\omega_{II}^2(t)x^2}{2}\quad;\quad \omega_{II}^2(t)=\frac{\omega_{I}^2(\eta(t))}{a^2(t)}+\frac{1}{4}\left(\frac{\dot{a}(t)}{a(t)}\right)^2-\frac{\ddot{a}(t)}{2a(t)}
\end{equation}
We now look at how the scaling parameters corresponding to $\mathscr{H}^{(I)}$ and $\mathscr{H}^{(II)}$, namely $B(\eta)$ and $b(t)$ are related. For this, we look at the non-linear Ermakov equation:

\begin{equation}\label{eq:ermakovv}
	B''(\eta)+\omega_{I}^2(\eta)B(\eta)=\frac{\omega_{I}^2(\eta_0)}{B^3(\eta)}
\end{equation}
To arrive at a solution for the Ermakov equation, we first consider solutions to the classical time-dependent oscillator:
\begin{equation}
    Y''(\eta)+\omega_{I}^2(\eta)Y(\eta)=0
\end{equation}
From a set of independent solutions $Y_1(\eta)$ and $Y_2(\eta)$ of the above equation, the scaling parameter $B(\eta)$ can be obtained as follows:
\begin{equation}\label{eq:bsolution}
    B^2(\eta)=\frac{1}{W_{Y}^2}\left[\left\{Y_1(\eta)Y'_2(\eta_0)-Y_1'(\eta_0)Y_2(\eta)\right\}^2+\omega_{I}^2(\eta_0)\left\{Y_1(\eta)Y_2(\eta_0)-Y_2(\eta)Y_1(\eta_0)\right\}^2\right],
\end{equation}
where $W_Y$ is the Wronskian for solutions $Y_1(\eta)$ and $Y_2(\eta)$, which is here time-independent. The above solution automatically satisfies the initial conditions $B(\eta_0)=1$ and $B'(\eta_0)=0$ \textit{regardless} of the initial conditions chosen for $Y_1$ and $Y_2$.

Similarly, for Hamiltonian $\mathscr{H}^{(II)}$, we write down the classical equation of motion and Ermakov equations respectively as follows:
\begin{equation}
    \ddot{y}(t)+\omega_{II}^2(t)y(t)=0\quad;\quad \ddot{b}(t)+\omega_{II}^2(t)b(t)=\frac{\omega_{II}^2(t_0)}{b^3(t)}
\end{equation}
Suppose the independent solutions are $y_1(t)$ and $y_2(t)$, the scaling parameter $b(t)$ are obtained as follows:
\begin{equation}
    b^2(t)=\frac{1}{W_y^2}\left[\left\{y_1(t)\dot{y}_2(t_0)-\dot{y}_1(t_0)y_2(t)\right\}^2+\omega_{II}^2(t_0)\left\{y_1(t)y_2(t_0)-y_2(t)y_1(t_0)\right\}^2\right],
\end{equation}
where $W_y$ is the Wronskian for solutions $y_1(t)$ and $y_2(t)$. The above solution automatically satisfies the initial conditions $b(t_0)=1$ and $\dot{b}(t_0)=0$. Using the equation connecting frequencies $\omega_I^2(t)$ and $\omega_{II}^2(\eta)$ in \eqref{eq:frequencyrelation}, we obtain the following relations connecting $y(t)$ and $Y(\eta)$:
\begin{equation}
    y(t)=\sqrt{a(t)}Y(\eta(t))\quad;\quad \dot{y}(t)=\frac{1}{\sqrt{a(t)}}\left(Y'(\eta(t))+\frac{\dot{a}(t)}{2}Y(\eta(t))\right);\quad W_y=W_Y
\end{equation}
Substituting this back into the solution $b(t)$, we obtain the following relation:
\begin{multline}\label{eq:bconnection2}
    b^2(t)=\frac{a(t)}{a(t_0)}B^2(\eta)+\frac{a(t)a(t_0)}{2W_Y^2}\left\{\frac{\dot{a}^2(t_0)}{a^2(t_0)}-\frac{\ddot{a}(t_0)}{a(t_0)}\right\}\left[Y_1(\eta)Y_2(\eta_0)-Y_1(\eta_0)Y_2(\eta)\right]^2\\+\frac{a(t)\dot{a}(t_0)}{W_Y^2a(t_0)}\left[Y_1(\eta)Y_2'(\eta_0)-Y_1'(\eta_0)Y_2(\eta)\right]\left[Y_1(\eta)Y_2(\eta_0)-Y_1(\eta_0)Y_2(\eta)\right]
\end{multline}
The above expression relates the time-evolution from the respective vacuum states corresponding to $\mathscr{H}^{(I)}$ and $\mathscr{H}^{(II)}$. Alternatively, one may be interested in studying the evolution of, say, the $\eta-$vacuum in the $t$ representation (see, for instance, \cite{2018Rajeev.etalGeneralRelativityandGravitation}). This leads to a simplified relation between the corresponding scaling parameters. To see this, notice that the wave functions, in the two different representations, of a given state of the system are related via:
%However, since different canonical variables select out different vacuum states~\cite{2020Grain.VenninJournalofCosmologyandAstroparticlePhysics}, the resulting time-evolution in both these representations are not equivalent. Such an equivalence can only be ensured by requiring that the initial vacuum state corresponding to $\mathscr{H}^{(I)}$ and $\mathscr{H}^{(II)}$ match exactly. Imposing this will lead to the same time-evolved state in both representations upto a phase-factor~\cite{2018Rajeev.etalGeneralRelativityandGravitation}:
\begin{equation}
    \Psi_{II}(x,t)=\frac{1}{a^{1/4}(t)}\Psi_{I}[X(x),\eta(t)]\exp{i\frac{\dot{a}(t)}{4a(t)}x^2}
\end{equation}
For the special case of a Gaussian state, the above relation translates to the following relation between the corresponding scaling parameters:
\begin{equation}\label{eq:bconnection}
    \frac{\omega_{II}(t_0)}{b^2(t)}=\frac{\omega_{I}(\eta_0)}{a(t)B^2(\eta)}\quad;\quad \frac{\dot{b}(t)}{b(t)}=\frac{1}{a(t)}\left[\frac{B'(\eta(t))}{B(\eta(t))}+\frac{\dot{a}(t)}{2}\right]
\end{equation}
Consequently, the above relation is also valid if one can further specialize to the vacuum state of one of the representations. The relevance of this relation is that its direct extension to the case of harmonic lattices can be used to study the consequences of canonical transformations. Note that in the limit 
\begin{equation}\label{eq:adotinit}
    \dot{a}(t_0)\rightarrow0\quad\text{and}\quad \ddot{a}(t_0)\rightarrow 0
\end{equation} we obtain \eqref{eq:bconnection} from \eqref{eq:bconnection2}. This limit, therefore, translates to the case when the instantaneous vacua of both representations coincide at $t=t_0$.

\subsection{Time-dependent CHO}

In order to observe the effects of canonical transformations on the classicality criteria, we now look at the CHO:
\begin{equation}
\mathscr{H}^{(I)}(\eta) =\frac{P_1^2}{2 }+\frac{P_2^2}{2 }+\frac{1}{2} \omega_I^2(\eta)\left(X_1^2+X_2^2\right)+\frac{1}{2}\chi_I^2(\eta)\left(X_1-X_2\right)^2\quad;\quad \mathscr{H}^{(II)}=\frac{\mathscr{H}^{(I)}[\eta(t)]}{a[\eta(t)]}
\end{equation}
The canonical transformations in \eqref{eq:cantrans} result in the following Hamiltonian:
\begin{align}
    \mathscr{H}^{(II)}(t)&=\frac{p_1^2}{2}+\frac{p_2^2}{2}+\frac{1}{2}\omega_{II}^2(t)(x_1^2+x_2^2)+\frac{1}{2}\chi_{II}^2(t)(x_1^2-x_2^2)\\ \omega_{II}^2(t)&=\frac{\omega_{I}^2[\eta(t)]}{a^2(t)}+\frac{1}{4}\left(\frac{\dot{a}(t)}{a(t)}\right)^2-\frac{\ddot{a}(t)}{2a(t)}\\ \chi_{II}^2(t)&=\frac{\chi_I^2[\eta(t)]}{a^2(t)}
\end{align}
In terms of $K_\pm$ and $L_\pm$ defined with respect to Hamiltonians $\mathscr{H}^{(I)}$ and $\mathscr{H}^{(II)}$ as given in \eqref{eq:rdmredef}, we get:
\begin{equation}
    K_\pm^{(II)}(t)=\frac{K_\pm^{(I)}[\eta(t)]}{a(t)}\quad;\quad L_\pm^{(II)}(t)=\frac{1}{a(t)}\left(L_\pm^{(I)}[\eta(t)]+\frac{\dot{a}(t)}{2}\right)
\end{equation}
We now look at how the characteristic parameters of the Wigner function are affected upon going from $\mathscr{H}^{(I)}$ described in terms of time $\eta$ to $\mathscr{H}^{(II)}$ described in terms of time $t$:
\begin{itemize}
    \item Degree of Quantum Decoherence $\delta_{QD}$ : 

    \begin{align}
        \delta_{QD}^{(II)}(t)&=\sqrt{\frac{4K_+^{(II)}(t)K_-^{(II)}(t)}{(K_+^{(II)}(t)+K_-^{(II)}(t))^2+(L_+^{(II)}(t)-L_-^{(II)}(t))^2}}\nonumber\\
        &=\sqrt{\frac{4K_+^{(I)}[\eta(t)]K_-^{(I)}[\eta(t)]}{(K_+^{(I)}[\eta(t)]+K_-^{(I)}[\eta(t)])^2+(L_+^{(I)}[\eta(t)]-L_-^{(I)}[\eta(t)])^2}}=\delta_{QD}^{(I)}[\eta(t)].
    \end{align}
\item Degree of Classical Correlation $\delta_{CC}$ : 
\begin{align}
        \frac{1}{\delta_{CC}^{(II)}(t)}&=\frac{K_+^{(II)}(t)L_-^{(II)}(t)+K_-^{(II)}(t)L_+^{(II)}(t)}{\sqrt{K_+^{(II)}(t)K_-^{(II)}(t)\left[(K_+^{(II)}(t)+K_-^{(II)}(t))^2+(L_+^{(II)}(t)-L_-^{(II)}(t))^2\right]}}\nonumber\\
        &=\frac{1}{\delta_{CC}^{(I)}(\eta)}\left[1+\left(\frac{\dot{a}(t)}{2}\right)\frac{K_+^{(I)}[\eta(t)]+K_-^{(I)}[\eta(t)]}{K_+^{(I)}[\eta(t)]L_-^{(I)}[\eta(t)]+K_-^{(I)}[\eta(t)]L_+^{(I)}[\eta(t)]}\right]
    \end{align}

\end{itemize}
Upon plugging the above expressions into \eqref{eq:dccc} and \eqref{eq:EntNHO}, we see that entanglement entropy (being a symplectic invariant~\cite{2003Serafini.etalJournalofPhysicsBAtomicMolecularandOpticalPhysics}) stays invariant under the canonical transformation in \eqref{eq:cantrans}, whereas log classicality does not:
\begin{equation}
    S^{(II)}(t)=S^{(I)}(\eta(t))\quad;\quad LC^{(II)}(t)\neq LC^{(I)}(\eta(t))\quad;\quad t=\int a(\eta)d\eta
\end{equation}
For the special case where $a(t)=a_0$ (constant), however, we see that they are both invariant~\cite{2023Chandran.ShankaranarayananPhys.Rev.D}:
\begin{equation}
    S^{(II)}(t)=S^{(I)}(a_0^{-1}t)\quad;\quad LC^{(II)}(t)= LC^{(I)}(a_0^{-1}t)\quad;\quad t= a_0 \eta
\end{equation}
The classicality criteria therefore has an ambiguity --- the condition on classicality parameter $\mathscr{C}$ is subject to change under a canonical transformation, even for the same time-evolved state, and the same subsystem division. Therefore, in order to manage this ambiguity, we make the second condition stronger by claiming that both representations $\mathscr{H}^{(I)}$ and $\mathscr{H}^{(II)}$ must satisfy the classicality criteria in \eqref{eq:criteria}, 
\begin{equation}
    \lim_{t\to\infty}S\to\infty \quad;\quad \lim_{t\to\infty} LC\to\infty,\nonumber
\end{equation}
failing which an asymptotic quantum-classical transition may be ruled out. Having amended the classicality criteria this way, we will now proceed to analyze early-Universe fluctuations in the following sections.

%\begin{equation}
%    \frac{1}{R_{(x,p)}^{(II)}}=\frac{1}{R_{(X,P)}^{(I)}}\left[1+\left(\frac{\dot{a}(t)}{2}\right)\frac{K_+^{(I)}[\eta(t)]+K_-^{(I)}[\eta(t)]}{K_+^{(I)}[\eta(t)]L_-^{(I)}[\eta(t)]+K_-^{(I)}[\eta(t)]L_+^{(I)}[\eta(t)]}\right]
%\end{equation}

\section{Early universe fluctuations in $(1+1)-D$}\label{sec:1dcosmology}
In this section, we apply the classicality criteria developed in Sections \ref{sec:cho} and \ref{sec:dss} for fluctuations propagating in an expanding universe in $(1+1)-$dimensions. Although this does not reflect the physical situation that concerns us, the extensive analytic control we have compared to $(3+1)-$dimensions can provide us with valuable insight on how an expanding background affects the ``quantumness" of such fluctuations. The unperturbed FLRW metric in comoving coordinates clocked by cosmic time ($\tilde{t}$) and conformal time ($\tilde{\eta}$) are respectively given below:
%
\begin{equation}
    ds^2=d\tilde{t}^2-a^2(\tilde{t})d\tilde{x}^2=a^2(\tilde{\eta})\left[d\tilde{\eta}^2-d\tilde{x}^2\right]\quad;\quad d\tilde{t}=a(\tilde{\eta})d\tilde{\eta}
\end{equation}
The action for a massive test scalar field in an arbitrary space-time background is given below:
\begin{equation}
S=\frac{1}{2}\int d\tilde{x}^{\mu} \sqrt{-g}\left[g^{\mu\nu}\partial_{\mu}\tilde{\Phi}\partial_{\nu}\tilde{\Phi} -\tilde{m}_f^2\tilde{\Phi}^2\right]
\end{equation}
%
In $(1+1)-$dimensions, the above action reduces to~\cite{1992Mukhanov.etalPhys.Rept.}:
\begin{equation}
S= \int d\tilde{t} L~~; \quad  
L= \frac{1}{2} \int d\tilde{x}  \left[\tilde{\Phi}'^2-(\partial_{\tilde{x}}\tilde{\Phi})^2-\tilde{m}_f^2\tilde{\Phi}^2\right]
\end{equation}
Upon defining the canonical momentum as $\tilde{\Pi}=\partial_{\tilde{\Phi}'}L$, and discretizing the system as $\tilde{x}=j\tilde{d}$, we get:
\begin{equation}
    \mathscr{H}[\tilde{\eta}]=\frac{1}{2\tilde{d}}\sum_j\left[\tilde{\Pi}_j^2+\left\{\tilde{\Phi}_j-\tilde{\Phi}_{j+1}\right\}^2+\tilde{d}^2\tilde{m}_f^2a^2(\tilde{\eta})\tilde{\Phi}_j^2\right]=\frac{\mathscr{H}^{(I)}}{\tilde{d}}
\end{equation}
We now absorb the UV cutoff $\tilde{d}$ via appropriate canonical transformations~\cite{2023Chandran.ShankaranarayananPhys.Rev.D}:
\begin{equation}
    \mathscr{H}^{(I)}[\eta]=\frac{1}{2}\sum_j\left[\Pi_j^2+\left\{\Phi_j-\Phi_{j+1}\right\}^2+\Lambda a^2(\eta)\Phi_j^2\right]\quad;\quad \eta=\frac{\tilde{\eta}}{\tilde{d}}\quad;\quad \Lambda=\tilde{d}^2\tilde{m}_f^2,
\end{equation}
where we have now shifted to a Hamiltonian that is fully described by dimensionless conformal time $\eta$ and dimensionless field mass $\Lambda$. When we follow a similar procedure to obtain the Hamiltonian in (dimensionless) cosmic time, we obtain:
\begin{equation}\label{eq:cosmicHam1D}
    \mathscr{H}^{(II)}[t]=\frac{1}{2a(t)}\sum_j\left[\Pi_j^2+\left\{\Phi_j-\Phi_{j+1}\right\}^2+\Lambda a^2(t)\Phi_j^2\right]\quad;\quad t=\frac{\tilde{t}}{\tilde{d}}\quad;\quad \Lambda=\tilde{d}^2\tilde{m}_f^2,
\end{equation}
We indeed see that the two Hamiltonians are connected the same way as in \eqref{eq:hamconnection}:
\begin{equation}
    \mathscr{H}^{(II)}[t]=\frac{\mathscr{H}^{(I)}[\eta(t)]}{a(\eta(t))}
\end{equation}
Following the same procedure as in Section \ref{sec:dss}, we finally get:
\begin{equation}
    \mathscr{H}^{(II)}[t]=\frac{1}{2}\sum_j\left[\pi_j^2+\frac{\left(\varphi_j-\varphi_{j+1}\right)^2}{a^2(t)}+\Omega^2(t)\varphi_j^2\right]\quad;\quad \Omega^2(t)=\Lambda+\frac{1}{4}\left(\frac{\dot{a}}{a}\right)^2-\frac{\ddot{a}}{2a}
\end{equation}
It should be noted that on going from conformal-time to cosmic-time Hamiltonian, the regularization that places field amplitudes along the comoving lattice $\tilde{x}=j\tilde{d}$ is preserved. Canonical transformations meanwhile act on the regularized field amplitudes, keeping the lattice structure intact. Therefore any bipartition in the real-space also carries over from $\mathscr{H}^{(I)}(\eta)$ to $\mathscr{H}^{(II)}(t)$, and the spatial entanglement can be directly compared for both representations.  

The normal modes spectrum for $\mathscr{H}^{(II)}(t)$ is given below~\cite{2008-Willms-SIAMJournalonMatrixAnalysisandApplications,2021Jain.etalPhys.Rev.D,2023Chandran.ShankaranarayananPhys.Rev.D}:
\begin{equation}
    \omega_j^2(t)=\Omega^2(t)+\frac{4}{a^2}f_j^2\quad;\quad f_j=\begin{cases}
        \sin\left[\frac{j\pi}{2(N+1)}\right] & \text{Dirichlet} \\
        \sin\left[\frac{(j-1)\pi}{2N}\right] & \text{Neumann}
    \end{cases}
\end{equation}
In the massless limit $\Lambda\to0$, and in terms of dimensionless Hubble paramater $H$, the normal modes become:
\begin{equation}
    \omega_j^2(t)=\frac{4}{a^2}f_j^2-\frac{1}{4}\left(H^2+2\dot{H}\right)\quad;\quad H=\frac{\dot{a}(t)}{a(t)}=\tilde{H}\tilde{d}
\end{equation}
In the thermodynamic limit $N\to\infty$, we may further rewrite the normal mode equation in terms of (dimensionless) co-moving momentum $k_j$ as follows:
\begin{equation}
    4a^2\omega_j^2\sim k_j^2-a^2H^2-2a^2\dot{H}\quad;\quad k_j=\frac{2\pi j}{N}=\tilde{k}_j\tilde{d},
\end{equation}
where we see that the normal mode spectrum maps to Fourier modes (for a lattice this spectrum is just the discrete fourier transform~\cite{2023Choudhury.etalSymmetry}). Shifting from co-moving to physical normal modes ($\bar{\omega}_j=a\omega_j$) and physical momenta ($\bar{k}_j=k_j/a$), we get:
\begin{equation}
    4\bar{\omega}_j^2\sim \bar{k}_j^2-H^2-2\dot{H}
\end{equation}
The normal mode $\bar{\omega}_j$ therefore corresponds to a momentum-mode $\bar{k}_j$ that is either sub-Hubble ($k_j>H$) or super-Hubble ($k_j<H$), whereas its \textit{stability} depends further on $\dot{H}$. We see that the inversion/squeezing of super-Hubble modes in general are amplified by an accelerated expansion ($\dot{H}>0$) and suppressed by a decelerated expansion ($\dot{H}<0$). On the other hand, the stability of sub-Hubble modes is enhanced by a
decelerated expansion ($\dot{H}<0$) and worsened by an accelerated expansion ($\dot{H}>0$).

In the case of $(1+1)$-dimensions, we may easily resolve the problem of quantum-classical transition via the connection formulas developed in Section \ref{sec:dss}. In the massless limit $\Lambda\to0$, we observe that the Hamiltonian $H^{(I)}$ is time($\eta$)-independent. As a result of this, the conformal-time scaling parameters are trivially fixed:
\begin{equation}
    B_j(\eta)=1\quad;\quad B_j'(\eta)=0
\end{equation}
Using \eqref{eq:bconnection}, we see that:
\begin{equation}\label{eq:cosmicb}
     \frac{\omega(t_0)}{b^2(t)}=\frac{\Omega(\eta_0)}{a(t)}\quad;\quad \frac{\dot{b}(t)}{b(t)}=\frac{\dot{a}(t)}{2a(t)}
\end{equation}
Since the entanglement entropy is a symplectic invariant, cosmic-time Hamiltonian must also result in constant entropies. However, log classicality depends on the choice of conjugate variables, i.e., in this case, the time co-ordinate employed. In any case, however, since entanglement entropy remains a constant throughout, massless fluctuations  \textit{never} undergo a quantum-classical transition in $(1+1)-$dimensions, regardless of squeezing or the choice of conjugate variables. We confirm this through numerical simulations of the cosmic-time Hamiltonian, which is explicitly time-dependent even in the massless case. For this demonstrative exercise, we consider two types of time-dependent background: (i) $a(t)\propto 1+A \tanh(Qt)$, where $A$ and $Q$ are constants. This describes a universe that smoothly expands by a finite factor over its entire evolution from asymptotic past to future, and (ii) $a(t)\propto e^{H t}$, which corresponds to a De-Sitter universe. 
\subsection{Tanh Expansion}
We first consider a simple evolution used for studying particle-production in an expanding background, with an asymptotic past and future where the in- and out- vacua are well-defined~\cite{1982-Birrell.Davies-QuantumFieldsCurved}:
\begin{equation}\label{eq:tanh}
    a(t)=\frac{1}{2}\left[\left\{a_1+a_0\right\}+\left\{a_1-a_0\right\}\tanh{\left(Qt\right)}\right]=\frac{a_0+a_1e^{2Qt}}{1+e^{2Qt}}
\end{equation}
where $a_0$ and $a_1$ are the respective initial and final values of the evolving scale factor and $Q^{-1}$ is the time-scale of quench. Upon evolving from $t_0\to-\infty$, the scaling parameter for all the modes can be obtained from \eqref{eq:bconnection} as follows:
\begin{equation}
    b_j(t)=\sqrt{\frac{1+\frac{a_1}{a_0}e^{2Qt}}{1+e^{2Qt}}}
\end{equation}

% Figure environment removed
From \ref{fig:Tanh1A}, we see that some of the modes undergo a brief inversion during the expansion, signified by the period in which $\omega_{k}^2(t)<0$. While it has no effect on entanglement entropy (symplectic invariance ensures that it stays constant regardless of the choice of conjugate variables), it translates to a brief squeezing of the reduced Wigner function and eventual stabilization, clearly captured by the log classicality plot ($LC$ vs $t$) in \ref{fig:Tanh1B}. This short-lived squeezing is a byproduct of choosing conjugate variables in the cosmic-time Hamiltonian, whereas the same is completely absent ($LC=0$) upon considering conformal-time conjugate variables. Both choices, therefore, fail to satisfy the two-fold classicality criteria.
% Figure environment removed


\subsection{De-Sitter Expansion}
The scale factor during de-Sitter expansion takes the following form:
\begin{equation}\label{eq:dsa}
    a(t)=a_0e^{H(t-t_0)},
\end{equation}
where $a_0$ is the initial value of the scale factor at $t=t_0$, and $H$ is the Hubble constant. Substituting this into \eqref{eq:cosmicHam1D}, we obtain a Hamiltonian that describes a chain of Caldirola-Kanai oscillators~\cite{1941CaldirolaIlNuovoCimento19241942,1948KanaiProgressofTheoreticalPhysics} with nearest-neighbour coupling, and therefore the results we outline here are in turn relevant to understanding dissipative systems. The classical solution for each mode in this case can be obtained by solving:
\begin{equation}
     y_j''(t)+\omega_j^2(t)y_j(t)=0
\end{equation}
where the normal-mode spectrum is given by:
\begin{equation}\label{eq:dsmode}
    \omega_j^2(t)=-\frac{H^2}{4}+\frac{4}{a_0^2}f_j^2e^{-2H(t-t_0)}
\end{equation}
We obtain the independent solutions to be $y_j(t)$ and $y_j^*(t)$, where:
\begin{equation}
    y_j(t)=\exp{\frac{1}{2}Ht+i\frac{2f_j}{a_0H}e^{-Ht}}\quad;\quad W[y_j,y_j^*]=4if_j.
\end{equation}
Using \eqref{eq:bsolution}, we obtain the scaling parameters as follows:
\begin{equation}
    b_j^2(t)=\left[e^{-Ht_0}-\frac{a_0H}{4f_j}\sin{\left\{\frac{4f_j(e^{-Ht_0}-e^{-Ht})}{a_0H}\right\}}\right]e^{Ht}
\end{equation}
It should be noted that the above solution for each $j-$mode is only valid if $a_0H<4f_j$, which along with \eqref{eq:dsmode} tells us that no mode can be inverted at the beginning of the evolution $t=t_0$. In the long-time limit, the scaling parameter takes a similar form as \eqref{b:inverted}:
\begin{equation}
    b_j\sim c_j e^{\frac{H(t-t_0)}{2}}\quad;\quad  c_j=\sqrt{1-\frac{a_0He^{Ht_0}}{4f_j}\sin{\left(\frac{4f_je^{-Ht_0}}{a_0H}\right)}}
\end{equation}
The key thing to note here is that all the $k-$modes that cross the horizon will have the exact same exponential growth factor ($\sim H/2$) for their respective scaling parameters $b_j(t)$. This eventually results in the saturation of entropy growth (Appendix \ref{app:saturation}), the time-scale ($t_{sat}$) for which is given by the inversion time for the mode with the largest index, i.e., $j=N$. For large $N$, this time-scale can be obtained from \eqref{eq:dsmode}:
\begin{equation}
    t_{sat}\sim t_0+\frac{1}{H}\log{\frac{4}{a_0H}}
\end{equation}

However, if we consider the beginning of the evolution to be at $t_0=-\infty$, the connections and the conditions in \eqref{eq:cosmicb} are satisfied, thereby matching the vacua in both cosmic-time and conformal-time and greatly simplifying the problem. The entanglement entropy therefore saturates instantly ($t_{sat}\to-\infty$) and it remains time-independent throughout the evolution, consistent with the results for the time-independent form conformal-time Hamiltonian. However, the classicality parameter picks up a non-trivial behaviour upon choosing cosmic-time conjugate variables.


% Figure environment removed

From \ref{fig:dS1A}, we see that all the normal modes for a de-Sitter expansion in cosmic-time conjugate variables eventually get inverted, and furthermore, they converge asymptotically to the same value, i.e., it exhibits an ungapped inverted mode spectrum as $t\to\infty$. In \ref{fig:dS1B}, the runaway squeezing of the reduced Wigner function translates to a linear growth of log classicality once the first mode becomes inverted (i.e., it has crossed the horizon), and its slope is found to saturate once all the modes have become inverted. The entanglement entropy, despite mode inversion, stays constant. The overall behaviour for any subsystem size in cosmic-time conjugate variables can be summarized below:
\begin{equation}
    S(t)=const\quad;\quad \lim_{t\to\infty} LC(t)\propto Ht
\end{equation}


% Figure environment removed

While the Tanh and de-Sitter models have proved useful in understanding the effects of mode-inversion and squeezing of the Wigner function for large subsystem-sizes, the time-independent behaviour of entanglement entropy effectively rules out any occurrence of quantum-classical transition in $(1+1)-$dimensions. This is also a perfect example of --- i) how squeezing by itself does not imply classicality, echoing the ungapped inverted mode scenario in the CHO (Table \ref{tab:CHO}), and ii) how inverted mode instabilities do not always generate a linear growth in entanglement entropy, in contrast with recent works~\cite{2018Hackl.etalPhys.Rev.A,2023Chandran.ShankaranarayananPhys.Rev.D}. We now turn our attention to $(3+1)-$dimensions in the next section, and apply the classicality criteria for fluctuations propagating in an expanding background.

\section{Early-universe fluctuations in $(3+1)-$D}\label{sec:3dcosmology}
In this Section, we apply the classicality criteria for fluctuations propagating in $(3+1)-$dimensions. The unperturbed expanding background in $(3+1)-$dimensions in co-moving coordinates ($\tilde{r}$,$\theta$,$\phi$) clocked by cosmic time ($\tilde{t}$) or conformal time ($\tilde{\eta}$) is described by:
\begin{equation}
    ds^2=d\tilde{t}^2-a^2(\tilde{t})(d\tilde{r}^2+\tilde{r}^2d\Omega^2)=a^2(\tilde{\eta})\left[d\tilde{\eta}^2-(d\tilde{r}^2+\tilde{r}^2d\Omega^2)\right]\quad;\quad d\tilde{t}=a(\tilde{\eta})d\tilde{\eta},
\end{equation}
where $d\Omega^2=d\theta^2+\sin^2{\theta}d\phi^2$. In terms of the conformal-time, the Lagrangian for a massive test scalar field in an expanding background is given by:
\begin{equation}
    L=\frac{a^2(\tilde{\eta})}{2}\int d\tilde{r}d\theta d\phi \tilde{r}^2\sin{\theta}  \left[\tilde{\Phi}'^2-(\partial_{\tilde{r}}\tilde{\Phi})^2-\frac{1}{\tilde{r}^2}(\partial_{\theta}\tilde{\Phi})^2-\frac{1}{\tilde{r}^2\sin^2{\theta}}(\partial_{\phi}\tilde{\Phi})^2-a^2(\tilde{\eta})\tilde{m}_f^2\tilde{\Phi}^2\right]
\end{equation}
In the massless limit, the above system equivalently describes the leading (linear) order scalar perturbations of the background metric. We may employ spherical decomposition to reduce the system to an effective $(1+1)-$dimensional system~\cite{1993-Srednicki-Phys.Rev.Lett.,2020Chandran.ShankaranarayananPhys.Rev.D}:
\begin{equation}
    \tilde{\Pi}=\frac{1}{\tilde{r}}\sum_{lm}\tilde{\Pi}_{lm}(\tilde{r})Z_{lm}(\theta,\phi)\quad;\quad \tilde{\Phi}=\frac{1}{\tilde{r}}\sum_{lm}\tilde{\Phi}_{lm}(\tilde{r})Z_{lm}(\theta,\phi)
\end{equation}
Upon further obtaining the canonical momentum $\tilde{\Pi}_{lm}=\partial_{\tilde{\Phi}_{lm}'}L$, and discretizing the system as $\tilde{r}=j\tilde{d}$, we get:
\begin{equation}
    \mathscr{H}[\tilde{\eta}]=\frac{1}{2\tilde{d}}\sum_{lmj}\left[\frac{\tilde{\Pi}_{lmj}^2}{a^2(\eta)}+a^2(\eta)\left(j+\frac{1}{2}\right)^2\left\{\frac{\tilde{\Phi}_{lmj}}{j}-\frac{\tilde{\Phi}_{lm,j+1}}{j+1}\right\}^2+\tilde{d}^2\tilde{m}_f^2a^4(\tilde{\eta})\tilde{\Phi}_{lmj}^2\right]=\sum_{lm}\frac{\mathscr{H}_{lm}^{(I)}}{\tilde{d}}
\end{equation}
The UV-cutoff $\tilde{d}$ can be absorbed and the Hamiltonian can be rewritten in terms of dimensionless parameters as follows~\cite{2020Chandran.ShankaranarayananPhys.Rev.D}:
\begin{equation}\label{eq:3dheta}
    \mathscr{H}_{lm}^{(I)}[\eta]=\frac{1}{2}\sum_{lmj}\left[\Pi_{lmj}^2+\left(j+\frac{1}{2}\right)^2\left\{\frac{\Phi_{lmj}}{j}-\frac{\Phi_{lm,j+1}}{j+1}\right\}^2+\left(\Lambda a^2(\tilde{\eta})-\frac{a''(\eta)}{a(\eta)}+\frac{l(l+1)}{j^2}\right)\Phi_{lmj}^2\right]
\end{equation}
where we have defined dimensionless conformal time $\eta=\tilde{d}^{-1}\tilde{\eta}$ and dimensionless field mass $\Lambda=\tilde{d}^2\tilde{m}_f^2$. Unlike the $(1+1)$-D case, we see that the massless conformal-time Hamiltonian has an explicit time($\eta$)-dependence in $(3+1)-D$ case, thereby leading to non-trivial dynamics in entanglement entropy and log classicality. When we follow a similar procedure to obtain the Hamiltonian in cosmic time, we see that the following relation holds as laid out in Section \ref{sec:dss}:
\begin{equation}
    \mathscr{H}_{lm}^{(II)}[t]=\frac{\mathscr{H}_{lm}^{(I)}[\eta]}{a(\eta(t))}.
\end{equation}
We therefore obtain the following Hamiltonian in terms of (dimensionless) cosmic time:
\begin{equation}
    \mathscr{H}_{lm}^{(II)}[t]=\frac{1}{2}\sum_{lmj}\left[\Pi_{lmj}^2+\frac{1}{a^2(t)}\left(j+\frac{1}{2}\right)^2\left\{\frac{\Phi_{lmj}}{j}-\frac{\Phi_{lm,j+1}}{j+1}\right\}^2+\Omega_{lmj}^2(t)\Phi_{lmj}^2\right]\quad;\quad t=\frac{\tilde{t}}{\tilde{d}},
\end{equation}
where,
\begin{equation}\label{eq:diag3d}
    \Omega_{lmj}^2(t)=\Lambda+\frac{l(l+1)}{j^2a^2(t)}-\frac{3}{4}\left(\frac{\dot{a}(t)}{a(t)}\right)^2-\frac{3\ddot{a}(t)}{2a(t)}
\end{equation}
Unlike in $(1+1)$-dimensions, the coupling matrix in $(3+1)$-dimensions is not a Toeplitz matrix, as a result of which an exact analytic expression for the normal mode spectrum cannot be obtained~\cite{2020Chandran.ShankaranarayananPhys.Rev.D}. However, using the insights from the normal mode spectrum obtained in Section \ref{sec:1dcosmology} and comparing it with the effective frequencies in \eqref{eq:diag3d}, we can infer that mode inversion is facilitated in cases of accelerated expansion, i.e., $\ddot{a}>0$. The extra input that we get here is that the angular momentum term, whose contribution is maximum at early times, \textit{counters} mode-inversion. The low-$l$ modes are the first to get inverted, whereas large-$l$ modes follow suit at later times. We sum up the angular-momentum contributions, which are expected to converge as $\l\to\infty$ for $(3+1)$-dimensions~\cite{1993-Srednicki-Phys.Rev.Lett.}, as follows:
\begin{equation}
    S(t)=\sum_l (2l+1)S_l(t)\quad;\quad LC(t)=\sum_l (2l+1)LC_l(t)
\end{equation}
For the rest of this section, we rely on numerics to see how various expansion models fare in the classicality test developed in Section \ref{sec:cho}.
\subsection{Tanh Evolution}
For the same quench function used in \eqref{eq:tanh}, we see from \ref{fig:Tanh2} that the entanglement entropy has a stable oscillatory behaviour throughout whereas log classicality temporarily peaks when the expansion kicks in. This is quite similar to the behaviour observed in $(1+1)-$dimensions, except that the entanglement entropy is no longer a constant. As a result, such an expansion does not satisfy the classicality criteria either in $(1+1)-$dimensions or in $(3+1)-$dimensions.

The above results indicate that during a Tanh expansion in $(3+1)-$dimensions, the momentum modes of discretized linear fluctuations briefly become inverted when they cross the Hubble radius, and stabilize when they reenter. Such an evolution causes \textit{spatial} bipartitions of fluctuations in the co-moving frame to maintain its coherence ($S$ has stable oscillatory behaviour) and exhibit a high-degree of classical correlations (sharp peak in $LC$) during this brief inversion. Upon reentering, the fluctuations recover from this brief buildup of classical correlations in the system, which can also be accompanied by an increase in $R_{(x,p)}$, the relative strength \eqref{eq:commratio} of quantum to classical contributions in observables. Therefore, in both $(1+1)-$ and $(3+1)-$dimensions, the real-space bipartitions of fluctuations in the comoving frame regain their quantumness as the scale factor of expansion relaxes to a final value.

% Figure environment removed

\subsection{de-Sitter Expansion}

During a de-Sitter expansion \eqref{eq:dsa} in $(3+1)-$dimensions, we see from \ref{fig:dS2} that both the entanglement entropy and log classicality of all subsystem sizes exhibit unbounded growth in time, thereby fulfilling the classicality criteria at late-times. This is in stark contrast with the behaviour observed in $(1+1)-$dimensions, where the entanglement entropy remained constant, thereby failing the classicality criteria at late-times. The fluctuations therefore undergo a quantum-classical transition in a de-Sitter background in $(3+1)-$dimensions, but not in $(1+1)-$dimensions.


Physically, this implies that during a de-Sitter expansion in $(3+1)-$dimensions, the momentum modes of discretized linear fluctuations become inverted as they cross the Hubble radius. This inversion in turn causes \textit{spatial} bipartitions of fluctuations in the co-moving frame to both quickly decohere ($S\to\infty$) and also exhibit a high-degree of classical correlations ($LC\to\infty$). The Gaussian nature of fluctuations further enables Hermitian observables of the form in \eqref{eq:weyl} to be fully described in terms of two-point functions. However, non-trivial quantum signatures in such observables are rapidly suppressed in the classicality limit as discussed in $\eqref{eq:commratio}$. As a result, at late-times, real-space bipartitions of fluctuations in the co-moving frame are essentially described by classical statistical ensembles, with their phase-space distribution sharply peaking about classical trajectories. This also implies that at late-times, it is nearly impossible to distinguish whether these fluctuations were of quantum or classical origin without high-precision observations.


% Figure environment removed


\section{Conclusions and Discussions}\label{sec:conc}

In this work, our primary focus is to understand the quantum-to-classical transition of entangled quadratic systems with spatial degrees of freedom. Our investigation involved three distinct signatures of classical behavior: i) decoherence as a measure of how well the system can be described by a classical statistical ensemble, ii) runaway squeezing of the Wigner function about classical phase-space trajectories, and iii) rapid suppression of non-commutativity in observables. We developed the necessary tools in Section \ref{sec:cho} to extract and measure these signatures in terms of entanglement entropy $S(t)$, log classicality $LC(t)$, and relative strength $R_{(x,p)}$ from a multi-mode Gaussian state.

We obtained a simple geometric picture of the interplay between these signatures through the stability analysis of the reduced Wigner function of the subsystem, as illustrated in \ref{fig:Wigfun}. The results, summarized in Table \ref{tab:CHO}, reveals that the presence of instabilities arising from a gapped inverted mode spectrum in the system leads to the emergence of all three classicality signatures in the CHO (quadratic system). On the other hand, other stability regimes exhibited only partial or no indications of classical behavior.

%In this work, we have studied the quantum-to-classical transition of quadratic systems with spatial degrees of freedom that are entangled. We tested real-space subsystems for three different signatures of classical behavior and analyzed their interplay --- i) decoherence as a measure of how well the system can be described by a classical statistical ensemble, ii) runaway squeezing of the Wigner function about classical phase-space trajectories, and iii) rapid suppression of non-commutativity in observables. In Section \ref{sec:cho}, we developed the tools required to extract and measure these signatures from a multi-mode Gaussian state in terms of entanglement entropy $S(t)$, log classicality $LC(t)$ and relative strength $R_{(x,p)}$ respectively. A simple geometric picture (\ref{fig:Wigfun}) of the interplay between these signatures is derived from the stability analysis of the reduced Wigner function of the subsystem. As summarized in Table \ref{tab:CHO}, the CHO exhibits all three signatures of classicality in the presence of instabilities arising from a gapped inverted mode spectrum, whereas other stability regimes result in partial or no signatures of classical behaviour.

Furthermore, in Section \ref{sec:dss}, we investigated the dependence of these signatures on conjugate variables by exploring canonical transformations connecting conformal time ($\eta$) to cosmic time ($t$) Hamiltonians. While entanglement entropy is a symplectic invariant, the other two measures showed explicit dependence on the choice of conjugate variables. This observation suggests that although decoherence is the strongest and most reliable condition for classical behavior to manifest in quantum systems~\cite{2020Ashtekar.etalPhys.Rev.D}, certain choices of conjugate variables could lead to a higher degree of classical correlations and faster suppression of quantum signatures during subsystem evolution.

%In Section \ref{sec:dss}, we tested these signatures for dependence on conjugate variables by looking at canonical transformations that connect conformal time ($\eta$) to cosmic time ($t$) Hamiltonians. While entanglement entropy is a symplectic invariant, the other two measures show an explicit dependence on the choice of conjugate variables. This implies that while decoherence is the strongest and most reliable condition of the three for classical behaviour to emerge in quantum systems, certain choices of conjugate variables can result in a higher degree of classical correlations and faster suppression of quantum signatures in physical observables over the course of the evolution. (\textbf{superselection?}) 

Next, in Section \ref{sec:1dcosmology}, we analyzed linear fluctuations of an expanding background in $(1+1)$ dimensions. We found that a quantum-to-classical transition did not occur as the dynamics preempted decoherence. This was demonstrated by considering two different scale factors of expansion: i) a Tanh expansion with fixed values at asymptotic past and future, and ii) an exponentially growing scale factor corresponding to a de-Sitter expansion. 
In Section \ref{sec:3dcosmology}, we extended the analysis to $(3+1)-$dimensions 
and showed that the background fluctuations of a Tanh expansion eventually recovered their quantum nature, whereas a de-Sitter expansion underwent a quantum-to-classical transition.

%In Section \ref{sec:1dcosmology}, we analyzed linear fluctuations of an expanding background in $(1+1)-$dimensions and ruled out a quantum-to-classical transition as the dynamics preempts decoherence. We demonstrate this by considering two different scale factors of expansion --- i) a Tanh expansion with fixed values at asymptotic past and future, ii) an exponentially growing scale factor corresponding to a de-Sitter expansion. In Section \ref{sec:3dcosmology}, we extended the analysis to $(3+1)-$dimensions and showed that the background fluctuations of a Tanh expansion asymptotically recovered its quantumness whereas for a de-Sitter expansion the fluctuations underwent a quantum-to-classical transition.

Throughout, we discovered that the inversion of normal modes in momentum-space acted as a common trigger for the emergence of classical behavior. While this inversion had limited impact on the momentum space of quadratic systems, it significantly affected the entangled degrees of freedom in real-space.  For a flat background in $(1+1)$ dimensions, recent studies have revealed that: i) the entanglement entropy ``classicalizes" i.e., it mimics the statistical entropy of classically chaotic systems via a linear growth, wherein the growth rate is given by the sum of all positive Lyapunov exponents~\cite{2018Bianchi.etalJournalofHighEnergyPhysics,2018Hackl.etalPhys.Rev.A,2023Chandran.ShankaranarayananPhys.Rev.D,2023Boutivas.etalJournalofHighEnergyPhysics}, 
%
ii) the leading order behavior of entanglement entropy asymptotically converges with other correlation measures, such as fidelity, Loschmidt echo, and circuit complexity of the entire system~\cite{2023Chandran.ShankaranarayananPhys.Rev.D}, and iii)  entanglement entropy asymptotically transitions from an area-law to a volume-law with subsystem size, thereby mimicking thermodynamic entropy~\cite{2023Katsinis.etal,2022Bianchi.etalPRXQuantum,2023Chandran.ShankaranarayananPhys.Rev.D}. In addition to our analysis, these effects further signal the emergence of both classical \textit{and} possible thermodynamic behaviour in the real-space from quantum foundations. However, the generalization of these properties to higher dimensions is subject of future work and will be addressed elsewhere using the tools developed here.

Our analysis may further provide the tools necessary to distinguish between cosmological models, such as those with similar observable power spectra, as has been the subject of recent investigations~\cite{2023Raveendran.Chakraborty}. Of particular interest is using these measures to distinguish inflation from bounce, which is currently under investigation. Lastly, a generalization of our real-space approach to account for higher-order curvature perturbations is an outstanding problem we hope to address in future works. Resolving this can filter out the spatial effects exclusively arising from non-Gaussianity in the context of the quantum-to-classical transition problem while also laying out potential new ways of obtaining direct observational evidence for the quantum origin of CMB fluctuations.


%As shown for the case of CHO and linear fluctuations of de-Sitter background, the common trigger for the emergence of classical behaviour is the inversion of normal modes in the momentum-space. While this inversion does not bring about non-trivial effects in the momentum space of quadratic systems, it has distinct consequences on the entangled degrees of freedom in the real-space. For a flat background in $(1+1)-$dimensions, earlier works had shown that --- i) the entanglement entropy ``classicalizes" i.e., it mimics the statistical entropy of classically chaotic systems via a linear growth, wherein the growth rate is given by the sum of all positive Lyapunov exponents, ii) The leading order behaviour of entanglement entropy (a subsystem measure) asymptotically converges with other correlation measures such as fidelity, Loschmidt echo and circuit complexity of the full system, and iii) It triggers an asymptotic transition from an area-law to a volume-law with subsystem size, mimicking thermodynamic entropy. However, the generalization of these properties to higher dimensions is subject of future work and will be commented on elsewhere.

%Things that may need to be added :
%\begin{itemize}
 %   \item Comment on Massive $(1+1)$-D
 %   \item Squeezing parameter and particle production
%\end{itemize}
\begin{acknowledgements}
The authors thank Suddhasattwa Brahma, Ashu Kushwaha, Orlando Luongo, Amaury Micheli, Abhishek Naskar and Vincent Vennin for useful discussions. SMC is supported by Prime Minister's Research Fellowship offered by the Ministry of Education, Govt. of India. The work is supported by the SERB Core Research grant.
\end{acknowledgements}
\appendix

\section{Conditions for purity saturation}\label{app:saturation}
The purity of CHO has the following form:
\begin{equation}
        \delta_{QD}(t)=\sqrt{\frac{4K_+K_-}{(K_++K_-)^2+(L_+-L_-)^2}}\quad;\quad K_\pm=\frac{\omega_\pm(t_0)}{b_\pm^2}\quad;\quad L_\pm=\frac{\dot{b}_\pm}{b_\pm}
    \end{equation}

Let us now rewrite $b_-(t)=f(t)b_+(t)$:
\begin{equation}
    \delta_{QD}(t)=\sqrt{\frac{4\omega_+(t_0)\omega_-(t_0)}{(f\omega_+(t_0)+f^{-1}\omega_-(t_0))^2+\dot{f}b_+^4}}\xRightarrow[]{\dot{f}\to0}\frac{\sqrt{4\omega_+(t_0)\omega_-(t_0)}}{f\omega_+(t_0)+f^{-1}\omega_-(t_0)},
\end{equation}
where we see that the evolution of purity (and thereby entanglement entropy) saturates in regimes where the Ermakov solutions $b_\pm(t)$ have the same time-evolution ($\dot{f}=0$) upto a proportionality constant ($f$).

\section{Entanglement entropy of CHO}
Like in the case of time-independent CHO~\cite{1993-Srednicki-Phys.Rev.Lett.,2006Ahmadi.etalCan.J.Phys.}, to evaluate the entanglement entropy, we must first calculate the eigenvalues of the reduced density matrix (RDM) of the system~\cite{1993-Srednicki-Phys.Rev.Lett.,2017Ghosh.etalEPLEurophysicsLetters} by solving the following integral equation~\cite{1993-Srednicki-Phys.Rev.Lett.,2006Ahmadi.etalCan.J.Phys.}:
\begin{equation}
	\int dx'_2 \, \rho_2(x_2,x_2')f_n(x_2')=p_nf_n(x_2) \, .
\end{equation}
The solution for the above integral equation is~\cite{2017Ghosh.etalEPLEurophysicsLetters}:
\begin{align}
	f_n(x)&=\frac{1}{\sqrt{2^n n!}}\left(\frac{\epsilon}{\pi}\right)^{1/4}H_n(\sqrt{\epsilon}x)\exp{-\left(\epsilon+i\delta\right)\frac{x^2}{2}}\nonumber\\
	\epsilon&=\sqrt{\Gamma_1^2-\Gamma_2^2}\nonumber\\
	p_n&=\left(1-\xi(t)\right)\xi^n(t)\\
	\xi(t)&=\frac{\Gamma_2}{\Gamma_1+\epsilon}=\frac{\sqrt{\left(\frac{\omega_+(t_0)}{b_+^2(t)}+\frac{\omega_-(t_0)}{b_-^2(t)}\right)^2+\left(\frac{\dot{b}_+(t)}{b_+(t)}-\frac{\dot{b}_-(t)}{b_-(t)}\right)^2}-2\sqrt{\frac{\omega_+(t_0)\omega_-(t_0)}{b_+(t)b_-(t)}}}{\sqrt{\left(\frac{\omega_+(t_0)}{b_+^2(t)}+\frac{\omega_-(t_0)}{b_-^2(t)}\right)^2+\left(\frac{\dot{b}_+(t)}{b_+(t)}-\frac{\dot{b}_-(t)}{b_-(t)}\right)^2}+2\sqrt{\frac{\omega_+(t_0)\omega_-(t_0)}{b_+(t)b_-(t)}}} \nonumber
\end{align}
%

The entanglement entropy is calculated as follows: 
%
\begin{equation}
\label{eq:CHO-ent1}
	S(t)=-\sum_n p_n\log{p_n}=-\log{[1-\xi(t)]}-\frac{\xi(t)}{1-\xi(t)}\log{\xi(t)},
\end{equation}

\section{Entanglement entropy of N-HO with time-dependent frequencies}\label{App:NHO}

Consider the following Hamiltonian~\cite{2023Chandran.ShankaranarayananPhys.Rev.D}:
\begin{equation}
	H=\frac{1}{2}\sum_{i=1}^N p_i^2+\frac{1}{2}\sum_{i,j=1}^N K_{ij}x_{i}x_{j},
\end{equation}
%
The coupling matrix ($K_{ij}$) here has time-dependent entries and captures all the necessary information about correlations in the system. The initial values of normal modes, labelled as $\{\omega_i(t_0)\}$, are the eigenvalues of $K^{1/2}(t_0)$. The ground state wave-function here is given by~\cite{2023Chandran.ShankaranarayananPhys.Rev.D}:
\begin{equation}
	\Psi_{\rm GS}(\tilde{X},t)=\left(\prod_{n=1}^N\frac{\omega_n(t_0)}{\pi b_n^2(t)}\right)^{1/4}\exp{-\frac{1}{2}\tilde{X}^T(\Omega_D^{1/2}-iZ_D)\tilde{X}-\frac{i}{2}\sum_{n=1}^N\omega_n(t_0)\tau_n},
\end{equation}
where $\tilde{X}=M X$ is the normal mode co-ordinate system that diagonalizes the Hamiltonian. We also see that $\Omega_D$ and $Z_D$ are diagonal matrices whose entries are given below:
\begin{equation}
	(\Omega_D)_{nn}=\frac{\omega_n(t_0)}{b_n^2(t)}\qquad;\qquad(Z_D)_{nn}=\frac{\dot{b}_n(t)}{b_n(t)}
\end{equation}
In the physical co-ordinates, the wave-function is entangled, taking the following form:
\begin{equation}
	\Psi_{\rm GS}(X,t)=\left(\prod_{n=1}^N\frac{\omega_n(t_0)}{\pi b_n^2(t)}\right)^{1/4}\exp{-\frac{1}{2}\left[X^T(\Omega-iZ)X\right]-\frac{i}{2}\sum_{n=1}^N\omega_n(t_0)\tau_n},
\end{equation}
where $Z=MZ_DM^T$ and $\Omega=M\Omega_DM^T$. In order to trace out some $m$ degrees of freedom from the system, let us first rewrite the following matrices:
\begin{equation}
	\Omega=\begin{bmatrix}
	(A)_{m\times m}&(B)_{m\times N-m}\\
		(B^T)_{N-m\times m}&(C)_{N-m\times N-m}\end{bmatrix} \quad;\quad Z=\begin{bmatrix}
	(Z_A)_{m\times m}&(Z_B)_{m\times N-m}\\
(Z_B^T)_{N-m\times m}&(Z_C)_{N-m\times N-m}\end{bmatrix}
\end{equation}
To calculate entanglement entropy, we first obtain the reduced density matrix as follows:
\begin{equation}\label{nrho}
	\rho_{out}=\sqrt{\frac{\det{\Omega/\pi}}{\det{A/\pi}}}\exp{-\frac{1}{2}X_{out}'^T(\Gamma_1+i\Gamma_3)X_{out}'-\frac{1}{2}X_{out}^T(\Gamma_1-i\Gamma_3)X_{out}+X_{out}^T\Gamma_2 X_{out}'}
\end{equation}
where we see that:
\begin{align}\label{nrho2}
	\Gamma_1&=C-\frac{1}{2}B^TA^{-1}B+\frac{1}{2}Z_B^T A^{-1}Z_B\nonumber\\
	\Gamma_2&=\frac{1}{2}B^TA^{-1}B+\frac{1}{2}Z_B^TA^{-1}Z_B\nonumber\\
	\Gamma_3&=Z_C-Z_B^TA^{-1}B	
\end{align}
Now we perform a series of diagonalizations to simplify the RDM further. Let $V$ be a diagonalizing matrix for $\Gamma_1$ such that $\Gamma_1=V^T \Gamma_{1_{D}} V$ and $\Gamma=\Gamma_{1_{D}}^{-1/2}V\Gamma_2 V^T\Gamma_{1_{D}}^{-1/2}$. Let $W$ diagonalize $\Gamma$ such that
%
\begin{equation}
\label{nrho3}
\Gamma=W^T\Gamma_DW \, \quad \tilde{\Gamma}_3=W\Gamma_{1_{D}}^{-1/2}V\Gamma_3 V^T\Gamma_{1_{D}}^{-1/2}W^T \, .   
\end{equation}
%
 In the new co-ordinates $Y=W\Gamma_{1_{D}}^{1/2}V X_{out}=\{y_j\}$, the RDM can hence be rewritten as:
\begin{equation}
	\rho_{out}=\frac{1}{\pi^{N-m}}\exp{\frac{i}{2}\left[Y^T\tilde{\Gamma}_3Y-Y'^T\tilde{\Gamma}_3Y'\right]}\prod_{j=m+1}^N\sqrt{1-\Gamma_j}\exp{-\frac{1}{2}\left(y_j^2+y_j'^2\right)+\Gamma_j y_jy_j'},
\end{equation}
where $\Gamma_j$ are the eignevalues of $\Gamma$. The integral eigenvalue equation for RDM will therefore have the following solution:
\begin{align}
	f_n(Y,t)&=\left(\prod_{j=m+1}^N H_n(\epsilon^{1/2}y_j)\right)\exp{-Y^T\left(\frac{\epsilon-i\tilde{\Gamma}_3}{2}\right)Y}\nonumber\\
	p_n(t)&=\prod_{j=m+1}^N \left(1-\xi_j(t)\right)\xi_j^n(t)\nonumber\\
	\xi_j(t)&=\frac{\Gamma_j}{1+\sqrt{1-\Gamma_j^2}}
	\end{align}
The entanglement entropy therefore accumulates contributions from each of the remaining degrees of freedom as $S=\sum_{j=m+1}^N S_j(t)$ where $S_j(t)$ has the familiar form~\cite{2023Chandran.ShankaranarayananPhys.Rev.D}:
\begin{equation}
	S_j(t)=-\log{[1-\xi_j(t)]}-\frac{\xi_j(t)}{1-\xi_j(t)}\log{\xi_j(t)}
\end{equation}



%\bibliography{ref}
%\bibliography{ref,refshanki}
\documentclass[a4paper,11pt]{article}
\pdfoutput=1 % if your are submitting a pdflatex (i.e. if you have
             % images in pdf, png or jpg format)

%\usepackage[utf8]{inputenc}
%\usepackage{mathrsfs, amssymb, amsmath}  
%\usepackage{comment}
%\usepackage{dcolumn}
%\usepackage{multirow}
%\usepackage{color}
%\usepackage{amsfonts,amssymb,amsmath, txfonts}
%\usepackage{float}

\usepackage{jcappub} % for details on the use of the package, please
                     % see the JCAP-author-manual

\usepackage[T1]{fontenc} % if needed

\hypersetup{ linktoc=all,
    colorlinks=true, linkcolor={blue},  
       citecolor={red}, urlcolor={darkred}
}
\definecolor{Redgreen}{RGB}{153,76,0}
\definecolor{vividviolet}{rgb}{0.62, 0.0, 1.0}
\definecolor{green}{RGB}{11,98,17}
\definecolor{darkgreen}{RGB}{40,150,65}
\definecolor{darkblue}{rgb}{0,0,0.3}
\definecolor{darkred}{rgb}{0.7,0,0}

\def\blue{\textcolor{blue}}
\def\red{\textcolor{red}}
\def\be{\begin{equation}}
\def\ee{\end{equation}}
\def\bea{\begin{eqnarray}}
\def\eea{\end{eqnarray}}


\title{MCMC Marginalisation Bias and $\Lambda$CDM tensions}
%\title{Overcoming bias in MCMC marginalisation to elucidate $\Lambda$CDM tensions}
%\title{Temp}

%%Markov Chain Monte Carlo

%% %simple case: 2 authors, same institution
%% \author{A. Uthor}
%% \author{and A. Nother Author}
%% \affiliation{Institution,\\Address, Country}

% more complex case: 4 authors, 3 institutions, 2 
\author[a]{Eoin \'O Colg\'ain}
\author[b]{Saeed Pourojaghi}
\author[b, c]{M. M. Sheikh-Jabbari}
\author[a]{Darragh Sherwin}

% The "\note" macro will give a warning: "Ignoring empty anchor..."
% you can safely ignore it.

\affiliation[a]{Atlantic Technological University, Ash Lane, Sligo, Ireland}
\affiliation[b]{School of Physics, Institute for Research in Fundamental Sciences (IPM), P.O.Box 19395-5531, Tehran, Iran}
\affiliation[c]{The Abdus Salam ICTP, Strada Costiera 11, I-34014 Trieste, Italy}

% e-mail addresses: one for each author, in the same order as the authors
\emailAdd{eoin.ocolgain@atu.ie}
\emailAdd{pourojaghi@ipm.ir}
\emailAdd{jabbari@theory.ipm.ac.ir}
\emailAdd{darragh.sherwin@research.atu.ie}




\abstract{Probability distributions become non-Gaussian when the flat $\Lambda$CDM model is fitted to redshift binned data in the late Universe. We explain mathematically why this non-Gaussianity arises and confirm that Markov Chain Monte Carlo (MCMC) marginalisation leads to biased inferences in observational Hubble data (OHD). In particular, in high redshift bins we find that $\chi^2$ minima, as identified from both least squares fitting and the MCMC chain, fall outside of the $1 \sigma$ confidence intervals. We resort to profile distributions to correct this bias. Doing so, we observe that $z \gtrsim 1$ cosmic chronometer (CC) data currently prefers a non-evolving (constant) Hubble parameter over a Planck-$\Lambda$CDM cosmology at $\sim 2 \sigma$. We confirm that both mock simulations and profile distributions agree on this significance. Moreover, on the assumption that the Planck-$\Lambda$CDM cosmological model is correct, using profile distributions we confirm  a $> 2 \sigma$ discrepancy with Planck-$\Lambda$CDM in a combination of  CC and baryon acoustic oscillations (BAO) data beyond $ z \sim 1.5$ that was noted earlier through comparison of least square fits of observed and mock data.}



\begin{document}
\maketitle
\flushbottom

\section{Introduction}
\label{sec:intro}
The flat $\Lambda$CDM model is the minimal model that fits Cosmic Microwave Background (CMB) data. Remarkably, CMB data from the Planck satellite \cite{Planck:2018vyg} constrains the $\Lambda$CDM model to sub-percent errors, thereby not only providing the strongest constraints, but also a concrete prediction for cosmological probes in the late Universe. The unmitigated success of the $\Lambda$CDM model is that CMB, Type Ia supernovae (SN) \cite{Riess:1998cb, Perlmutter:1998np} and baryon acoustic oscillations (BAO) \cite{Eisenstein:2005su} agree on a $\Lambda$CDM Universe that is approximately $30 \%$ matter. Thus, one key prediction of the Planck-$\Lambda$CDM model agrees across early and late Universe cosmological probes. Given this non-trivial agreement, any discrepancies that arise elsewhere constitute challenging puzzles. 

Nevertheless, one cannot define any \textit{model} for a dynamical system, especially a complicated system like the Universe, using data from a cosmic snapshot.\footnote{Here, we mean CMB data with an effective redshift $z \sim 1100$.} At best, one has a \textit{prediction} and not a model. In recent years, key predictions of Planck data have been challenged by late Universe determinations of the Hubble constant $H_0$ \cite{Riess:2021jrx, Freedman:2021ahq, Pesce:2020xfe, Blakeslee:2021rqi, Kourkchi:2020iyz} and the $S_8:= \sigma_8 \sqrt{\Omega_m/0.3}$ parameter \cite{HSC:2018mrq, KiDS:2020suj, DES:2021wwk, Boruah:2019icj, Said:2020epb}. Given the diversity of the late Universe probes (see reviews \cite{Perivolaropoulos:2021jda, Abdalla:2022yfr}), it is highly unlikely that any single systematic can be found to explain the discrepancies. That being said, in astrophysics one can never preclude systematics; 3 decades after Phillips' seminal paper \cite{Phillips:1993ng}, we are still debating an ad hoc correction for the mass of the host galaxy in Type Ia SN \cite{NearbySupernovaFactory:2018qkd, Kang:2019azh, Brout:2020msh, Lee:2021txi}. Bearing in mind that Type Ia SN are one of our best understood cosmological probes, one quickly understands that any systematics debate may be endless. 

Thus, it is far more expedient to assume that the $\Lambda$CDM model is breaking down and to look for tell-tale signatures of model breakdown. If signatures cannot be found, one arrives at a contradiction, and revisits the assumption that the model is breaking down. For physicists, \textit{model breakdown comes about when model fitting parameters return discrepant values at different time slices or epochs}. Translated into astronomy, this equates to discrepant cosmological parameters in different redshift ranges. The usual $H_0, S_8$ tensions  may also be viewed in the same light: a discrepancy between high and low redshift inferences/measurements of the parameters \cite{Perivolaropoulos:2021jda, Abdalla:2022yfr}. Nevertheless, early and late Universe observables are typically not the same, so one is confronted with a rich set of potential systematics. 

Within the context of $\Lambda$CDM tensions, it was recently observed that the integration constant from the Friedmann equations, aka the Hubble constant $H_0$, picks up redshift dependence whenever our model assumption - required to close the Friedmann equations - disagrees with the Hubble parameter $H(z)$ extracted from observations \cite{Krishnan:2020vaf, Krishnan:2022fzz}. %\footnote{One is free to speculate about the nature of the missing physics \cite{Liao:2020zko, Montani:2023xpd}.} 
Similarly, $\rho_{m0}=H_0^2\Omega_m$, an integration constant of the matter continuity equation, implies matter density $\Omega_m$ is a mathematically constant quantity. 
These are irrefutable predictions from mathematics, i. e. a prediction that is \textit{robust to systematics}. However, observationally $H_0$ and $\Omega_{m}$ are model fitting parameters and nothing precludes them picking up redshift dependence (except of course if one assumes they do not!), and providing a signature of model breakdown. If this happens in the late Universe within the $\Lambda$CDM model, $H_0$ is correlated with matter density $\Omega_m$, 
while $\Omega_m$ is correlated with $S_8 \propto \sigma_8 \sqrt{\Omega_m}$. Thus, there is at least one simple scenario, namely redshift evolution of cosmological parameters in the late Universe, where ``$H_0$ tension'' and ``$S_8$ tension'' are not independent and simply symptoms of $\Lambda$CDM model breakdown. 

The next relevant question is, where is the evidence for evolving cosmological parameters in the late Universe? Starting with strong lensing time delay \cite{Wong:2019kwg, Millon:2019slk},\footnote{Systematics are explored in \cite{Millon:2019slk} and the descending trend is not an obvious systematic. The lensed system RXJ1131-1231 \cite{Sluse:2003iy}, which partly drives the trend, has recently been re-analysed using spatially resolved stellar kinematics of the host galaxy \cite{Shajib:2023uig}, and the higher $H_0$ value remains robust, admittedly with inflated errors. As TDCOSMO project to analyse 40 lenses, the prospect of a discovery of a descending $H_0$ trend assuming the $\Lambda$CDM model remain strong.} descending trends of $H_0$ with redshift have been reported in Type Ia SN \cite{Dainotti:2021pqg, Colgain:2022nlb, Colgain:2022rxy,  Malekjani:2023dky, Hu:2022kes, Jia:2022ycc} and combinations of data sets \cite{Krishnan:2020obg, Dainotti:2022bzg}. On the other hand, larger values of $\Omega_m$ have been noted in high redshift observables, primarily quasars (QSOs) \cite{Risaliti:2015zla, Risaliti:2018reu, Lusso:2020pdb, Yang:2019vgk, Khadka:2020vlh, Khadka:2020tlm, Khadka:2021xcc, Pourojaghi:2022zrh},\footnote{Just as with Type Ia SN, the systematics of QSOs are being investigated \cite{Zajacek:2023qjm}.} but also Type Ia SN \cite{Colgain:2022nlb, Colgain:2022rxy, Malekjani:2023dky, Pasten:2023rpc} (see also \cite{Wagner:2022etu, Sakr:2023hrl}). Note, as emphasised earlier, if $H_0$ evolves at the background level, correlated fitting parameters are expected to also evolve. Moreover, mock analysis within the $\Lambda$CDM setting reveals that evolution of best fit $(H_0, \Omega_m)$ parameters cannot be precluded, and conversely possesses a finite likelihood, in either observational Hubble data (OHD) \textit{or} angular diameter distance data \textit{or} luminosity distance data \cite{Colgain:2022tql}. We stress that this result \textit{rests on mock analysis}; it represents a purely mathematical statement about the $\Lambda$CDM model that is independent of systematics. 

Separately, at the perturbative level, redshift evolution of $S_8$ or $\sigma_8$ has been reported in galaxy cluster number counts and Lyman-$\alpha$ spectra \cite{Esposito:2022plo}, $f \sigma_8$ constraints from peculiar velocities and redshift space distortions (RSD) 
 \cite{Adil:2023jtu}, comparison between weak \cite{HSC:2018mrq, KiDS:2020suj, DES:2021wwk} and CMB lensing \cite{ACT:2023dou, ACT:2023kun}. What is important here is that these observations appear to restrict the evolution in $S_8$ to the late Universe. In \cite{ACT:2023ipp} the possibility was raised that \textit{``tracers at higher redshift and probing larger scales prefer higher $S_8$''}.\footnote{There are also conflicting observations of high redshift $\sigma_8$ or $S_8$ values that are lower than Planck in the late Universe \cite{Miyatake:2021qjr, Alonso:2023guh}, so either this trend is not universal, or systematics are at play.} Nevertheless, one can argue against evolution with scale on the grounds that cosmic shear \cite{HSC:2018mrq, KiDS:2020suj, DES:2021wwk}, which is sensitive to smaller scales (larger $k$), and peculiar velocity constraints \cite{Boruah:2019icj, Said:2020epb}, which are sensitive to larger scales (smaller $k$), both prefer lower values of $S_8$. Moreover, both galaxy clusters and Lyman-$\alpha$ spectra are expected to probe similar scales.\footnote{We thank Matteo Viel for correspondence on this point.} Thus, if systematics are not impacting results, then redshift evolution is the only point of agreement in the observations \cite{Esposito:2022plo, Adil:2023jtu, HSC:2018mrq, KiDS:2020suj, DES:2021wwk, ACT:2023dou, ACT:2023kun, ACT:2023ipp}. Note also that redshift is more fundamental than scale in FLRW cosmology; one must solve the Friedmann equations in either time or redshift before one contemplates any discussion of scale.  

 The purpose of this letter is to revisit the analysis presented in \cite{Colgain:2022rxy,Colgain:2022tql}, where the evidence for evolution was quantified on the basis of mock simulations and not Markov Chain Monte Carlo (MCMC), the technique most familiar in cosmology. The fundamental problem is that once one bins low redshift data and studies evolution of cosmological parameters with bin redshift, one quickly encounters projection effects in MCMC analyses. These effects are not just the preserve of exotic models \cite{Herold:2021ksg, Gomez-Valent:2022hkb, Meiers:2023gft}, such as Early Dark Energy (EDE) \cite{Poulin:2018cxd, Niedermann:2019olb}, and happen in the simplest model when one bins data. The most striking demonstration of the resulting bias is that the peaks of MCMC posteriors no longer coincide with the minimum of the likelihood (see \cite{Gomez-Valent:2022hkb}). Ultimately, this bias is expected  because one is working in a regime of the $\Lambda$CDM model with non-Gaussian probability distributions   \cite{Colgain:2022tql}  (see also \cite{Colgain:2022rxy}).

 The structure of this paper is as follows. In section \ref{sec:MCMC_bias} we confirm the bias in MCMC marginalisation. In section \ref{sec:PD} we introduce profile distributions (PDs) \cite{Gomez-Valent:2022hkb} as a means of addressing the bias and confirm that the statistical significance of discrepancies from mock simulations agree well with PD analysis. In section \ref{sec:tension}, we revisit and confirm the high redshift OHD tensions reported in \cite{Colgain:2022rxy}. We end in section \ref{sec:discussion} with concluding remarks. 
 %A short appendix is also added on Fisher matrix for $\Lambda$CDM mdoel. 

\section{A bias in MCMC marginalisation}
\label{sec:MCMC_bias}
In this section we illustrate a bias in MCMC marginalisation that arises in the (flat) $\Lambda$CDM model when data is binned by redshift. This bias can be traced to a regime of the $\Lambda$CDM model with non-Gaussian distributions and is independent of systematics  \cite{Colgain:2022rxy, Colgain:2022tql}. 

\subsection{Mathematical Foundations}
\label{sec:math}
Consider an exercise where one bins OHD and confronts it to the $\Lambda$CDM Hubble Parameter $H(z)$ in the late Universe, a setting where the radiation sector can be safely decoupled. In high redshift bins ($z \gg 0$) in the matter-dominated regime, the Hubble parameter becomes insensitive to the dark energy (DE) sector: 
\be
\label{eq:lcdm}
H(z) = H_0 \sqrt{1-\Omega_m + \Omega_m (1+z)^3} \xrightarrow[z \gg 0]{} H_0 \sqrt{\Omega_m} (1+z)^{\frac{3}{2}}.  
\ee
More concretely, taking $z \rightarrow \infty$ we see that data can only constrain the combination $\rho_{m0}=H_0^2{\Omega_m}$. For \textit{hypothetical} data in a redshift bin with effective redshift $z = \infty$, this means that one can only constrain the combination $\Omega_m h^2$ ($h:= H_0/100)$, but $H_0$ and $\Omega_m$ remain unconstrained. Alternatively put, for any given $\Omega_m h^2$ constraint, there is an infinite number of corresponding $(H_0, \Omega_m)$ pairs. Translated into a probability density function (PDF), this is simply the statement that in a very high redshift bin at $z = \infty$, one expects uniform or flat distributions for $H_0$ and $\Omega_m$ with the model (\ref{eq:lcdm}).  

Of course, observed data resides at finite $z$ and not $z = \infty$. As a result, one does not encounter \textit{exactly} flat PDFs in $H_0$ and $\Omega_m$ at high redshift, but \textit{almost} flat PDFs. More important to us is the observation that these PDFs must flatten in a non-Gaussian manner. To appreciate this fact, we observe that high redshift OHD only constrains $\Omega_m h^2$ well.\footnote{Note that observables like SN or QSO that measure $D_L(z)=c (1+z)\int_0^z \textrm{d} z'/H(z')$ are mainly sensitive to the low redshift part of $H(z)$, i. e. the combination $H_0^2 (1-\Omega_m)$, and in this sense they are complementary to the OHD data which is more sensitive to high redshift part of $H(z)$, $H_0^2\Omega_m$. The complementarity can be demonstrated by combining $H(z)$ and $D_{L}(z)$ constraints and checking that one recovers mock data input parameters in all redshift bins \cite{Colgain:2022tql}. } For this reason, best fit parameters are constrained to a $\Omega_m h^2 = \textrm{constant}$ curve in the $(H_0, \Omega_m)$-plane. The almost flat $H_0$ and $\Omega_m$ PDFs can only arise if this curve stretches in the $(H_0, \Omega_m)$-plane. As a result of this stretching, one ends up with a relatively uniform distribution on a curve. At the extremes of the curve, one finds a distribution of large $H_0$ values, which do not differ greatly in $\Omega_m$, and they get projected to a peak at small values on the $\Omega_m$ axis. Conversely, at the other end of the curve, one finds a distribution of small $\Omega_m$ values, which do not differ greatly in $H_0$, and they get projected onto a peak at large values on the $H_0$ axis.  This is a ``projection effect'' in common cosmology parlance.  It is driven by the irrelevance of the DE sector at high redshift and the constraint $\Omega_m h^2 = \textrm{constant}$ from the $\Lambda$CDM model (\ref{eq:lcdm}). Together these features distort the distribution away from a Gaussian configuration. 

Thus, simply by binning and fitting OHD to the $\Lambda$CDM model one enters a non-Gaussian regime as the effective redshift of the bin increases. This effect, which is expected from the purely mathematical arguments above, has been confirmed in mock data \cite{Colgain:2022rxy, Colgain:2022tql}, and in line with expectations, we demonstrate that it impacts MCMC inferences with observed data in the next subsection.  

% Figure environment removed

\subsection{Cosmic Chronometer (CC) Data}
\label{sec:CCbias}
Here we work with OHD from the cosmic chronometer (CC) program \cite{Jimenez:2001gg}. Concretely, we work with 34 $H(z)$ constraints spanning the redshift range $0.07 \leq z \leq 1.965$ \cite{Stern:2009ep, Moresco:2012jh, Zhang:2012mp, Moresco:2016mzx, Ratsimbazafy:2017vga, Borghi:2021rft, Jiao:2022aep, Tomasetti:2023kek}. We illustrate the data in Fig.~\ref{fig:CC}, where it is consistent with Fig. 9 of \cite{Tomasetti:2023kek} {modulo the fact that we have an additional data point at $z = 0.8$, which is not independent. See Table 1.1 of \cite{Moresco:2023zys}. While CC data may eventually be good enough to arbitrate on Hubble tension \cite{Moresco:2023zys}, the data is not good enough on its own to do cosmology. To put this comment in context, we observe that the errors in Fig.~\ref{fig:CC} do not include systematic errors (see \cite{Moresco:2020fbm} for an account of the systematics). As a result the constraints we get on cosmological parameters will be underestimated. Thus, from our perspective the data in Fig.~\ref{fig:CC} is simply some representative cosmological data in the OHD class.}

\paragraph{Methodology:} We impose a low redshift cut-off on the OHD $z_{\textrm{min}}$, removing all data points with redshifts $z_i < z_{\textrm{min}}$, and then extremising the $\chi^2$ likelihood, 
\be
\label{eq:chi2}
\chi^2 = Q^{T} \cdot C^{-1} \cdot Q, 
\ee
where $C$ is the covariance matrix, which is simply the square of the $H_i$ errors on the diagonal, and $Q$ is the vector, 
\be
\label{eq:Q}
Q_i = H_i - H_{\textrm{model}}(z_i), 
\ee
where $H_i:=H(z_i)$ denotes OHD and $H_{\textrm{model}}(z)$ is the model (\ref{eq:lcdm}) without the high redshift limit. The best fit $(H_0, \Omega_m)$ parameters correspond to the minumum of the $\chi^2$, while on the assumption of Gaussian errors, we estimate the errors from a Fisher matrix (appendix \ref{sec:fisher}). In parallel, we perform MCMC marginalisation through \textit{emcee} \cite{Foreman-Mackey:2012any}. More concretely, subject to the priors $H_0 \in [0, 200 ]$ and $\Omega_m \in [ 0, 1]$, the latter restricting us to a physical regime, we record $16^{\textrm{th}}$, $50^{\textrm{th}}$ and $84^{\textrm{th}}$ percentiles for MCMC posteriors, as is common practice with Gaussian distributions. Thus, both techniques are tailored to Gaussian posteriors, yet non-Gaussianities will be evident in MCMC posteriors. By comparing the output from these two techniques in Table \ref{tab:LCDM_CC} for different values of $z_{\textrm{min}}$ we observe that error estimates from Fisher matrix and MCMC quickly disagree as $z_{\textrm{min}}$ increases. 

From Table \ref{tab:LCDM_CC}, we see that MCMC inferences lead to non-Gaussian $1 \sigma$ confidence intervals, where in line with the expectations from \cite{Colgain:2022tql}, $H_0$ errors are larger for smaller values, and $\Omega_m$ errors are larger for larger values, respectively. This is expected if the $H_0$ and $\Omega_m$ posteriors are peaked at larger and smaller values, respectively, in line with our earlier mathematical argument. Only for the full data set with $z_{\textrm{min}} = 0$  do we find reasonable agreement between the Fisher matrix and MCMC $1 \sigma$ confidence intervals. As can be seen from the lopsided MCMC confidence intervals, the non-Gaussianity becomes more pronounced with increasing $z_{\textrm{min}}$. Interestingly, beyond $z_{\textrm{min}} = 1$, the minimum of the $\chi^2$ falls outside of the MCMC $1 \sigma$ confidence intervals. Nevertheless, by evaluating the MCMC chains on the $\chi^2$ likelihood (\ref{eq:chi2}), we confirm that the parameters corresponding to the minimum $\chi^2$ value are tracking the best fit. Note, the peak of the MCMC posterior is no longer a measure of goodness of fit and inferences have become biased in a regime of model parameter space where distributions are expected to be inherently non-Gaussian. Our analysis here underscores potential problems with a blind MCMC analysis with the traditional $16^{\textrm{th}}$, $50^{\textrm{th}}$ and $84^{\textrm{th}}$ percentiles.       



\begin{table}[htb]
    \centering
    \begin{tabular}{c|c|c|c|c|c}
    \rule{0pt}{3ex} $z_{\textrm{min}}$ & \# CC & \multicolumn{2}{c}{Fisher Matrix}  & \multicolumn{2}{|c}{MCMC} \\
    \hline
    \rule{0pt}{3ex} & & $H_0$ (km/s/Mpc) & $\Omega_m$ & $H_0$ (km/s/Mpc) & $\Omega_m$ \\
    \hline
    \rule{0pt}{3ex} $0$ & $34$ & $68.14 \pm 3.07$ & $0.320 \pm 0.059$ & $67.76^{+3.03}_{-3.09}$  ($68.12$) & $0.328^{+0.065}_{-0.055}$ ($0.321$) \\
    \hline 
    \rule{0pt}{3ex} $0.2$ & $27$ & $65.03 \pm 6.65$ & $0.368 \pm 0.118$ & $63.05^{+6.64}_{-7.23}$ ($64.98$) & $0.405^{+0.170}_{-0.111}$ ($0.369$) \\
    \hline 
    \rule{0pt}{3ex} $0.4$ & $22$ & $62.42 \pm 8.38$ & $0.411 \pm 0.161$ & $59.54^{+8.30}_{-8.22}$ ($62.39$) & $0.470^{+0.229}_{-0.151}$ ($0.411$)\\
    \hline 
    \rule{0pt}{3ex} $0.6$ & $15$ & $59.83 \pm 17.21$ & $0.454 \pm 0.338$ & $56.45^{+13.16}_{-9.33}$ ($59.86$) & $0.526^{+0.288}_{-0.225}$ ($0.453$) \\
    \hline 
    \rule{0pt}{3ex} $0.7$ & $14$ & $79.11 \pm 19.40$ & $0.222 \pm 0.162$ & $67.59^{+19.19}_{-16.57}$ ($79.18$) & $0.344^{+0.344}_{-0.178}$ ($0.222$) \\
    \hline 
    \rule{0pt}{3ex} $0.8$ & $11$ & $103.97 \pm 24.94$ & $0.097 \pm 0.088$ & $82.43^{+28.33}_{-27.03}$ ($104.02$) & $0.206^{+0.357}_{-0.131}$ ($0.096$) \\
    \hline 
    \rule{0pt}{3ex} $1$ & $8$ & $150.37 \pm 31.21$ & $0.010 \pm 0.035$ & $108.92^{+33.94}_{-44.47}$ ($150.38$) & $0.087^{+0.304}_{-0.068}$ ($0.010$) \\
    \hline 
    \rule{0pt}{3ex} $1.2$ & $7$ & $154.35 \pm 42.95$ & $0.006 \pm 0.042$ & $83.07^{+48.52}_{-32.19}$ ($154.47$) & $0.194^{+0.439}_{-0.159}$ ($0.006$) \\
    \hline 
    \rule{0pt}{3ex} $1.4$ & $4$ & $125.41 \pm 79.55$ & $0.039 \pm 0.132$ & $65.32^{+44.88}_{-20.30}$ ($125.44$) & $0.320^{+0.423}_{-0.250}$ ($0.039$) \\
    \hline 
    \rule{0pt}{3ex} $1.5$ & $3$ & $36.12 \pm 72.69$ & $1.000 \pm 4.269$ & $55.19^{+34.64}_{-14.73}$ ($36.16$) & $0.393^{+0.387}_{-0.283}$ ($0.999$)
    \end{tabular}
    \caption{Comparison between Fisher matrix and MCMC analysis for CC data with a low redshift cut-off $z_{\textrm{min}}$. We record the number of data points, the extremum of the $\chi^2$ and $1 \sigma$ confidence interval estimated from the Fisher matrix,  $16^{\textrm{th}}$, $50^{\textrm{th}}$ and $84^{\textrm{th}}$ percentiles from MCMC posteriors corresponding to $1 \sigma$ confidence intervals, and the minimum $\chi^2$ from the MCMC chain in brackets. MCMC marginalisation exhibits non-Gaussian $1 \sigma$ confidence intervals, and for $z_{\textrm{min}} > 1$, the minimum value of the $\chi^2$ from the MCMC chain falls outside of this interval. The latter tracks the best fit up to small numbers in line with expectations. }
    \label{tab:LCDM_CC}
\end{table}

\subsection{Features in CC Data}
\label{sec:features}
Once one accounts for biases, it is clear from Table \ref{tab:LCDM_CC} that there are trends in CC data when it is binned. Starting from $z_{\textrm{min}} = 0$ through to $z_{\textrm{min}} = 0.6$ we see a decreasing trend in best fit values of $H_0$ (also central $H_0$ values from MCMC), which is compensated by a increasing trend in $\Omega_m$ best fit values. From Fig.~\ref{fig:CC} it is difficult to visibly discern any trend from the raw data. From $z_{\textrm{min}} = 0.7$ through to $z_{\textrm{min}} = 1.4$, there is in contrast a preference for larger $H_0$ and smaller $\Omega_m$ values. This trend is evident from the raw data, where at higher redshifts one sees large scatter and large fractional errors in the data. For $z_{\textrm{min}} = 1$, it is clear that the best fit line in magenta corresponding to $(H_0, \Omega_m) = (150.4, 0.01)$ (Table \ref{tab:LCDM_CC}) is closer to horizontal line than the Planck-$\Lambda$CDM cosmology in red. To be more explicit, for $z_{\textrm{min}} = 0$, $\rho_{m0}:=H_0^2\Omega_m\simeq 1500$ which is close to the Planck value, whereas for $z_{\textrm{min}} = 1$, $\rho_{m0}\simeq 225$. The sharp drop in $\rho_{m0}$ means the magenta line should be almost horizontal. For $z_{\textrm{min}} = 1.5$, we switch to an opposite regime of parameter space with unexpectedly low and high values of $H_0$ and $\Omega_m$, respectively, a trend which is evident in the data, but there are only three data points. Despite, the small number of data points, the tendency for smaller $H_0$ and larger $\Omega_m$ inferences within $\Lambda$CDM cosmology at high redshifts has been documented across three independent observables \cite{Colgain:2022rxy}. We will come back to this claim in section \ref{sec:tension}. Finally, it is worth noting that for large $z_{\textrm{min}}$ and samples with few data points, one expects broad MCMC posteriors. These posteriors are severely impacted by the prior on $\Omega_m$, as is evident from Table \ref{tab:LCDM_CC}. 

For the moment we leave physical speculations to the discussion and return to the trend in CC data above $z=1$ favouring less evolution in the Hubble parameter than the Planck-$\Lambda$CDM model. We would like to quantify the significance of this trend, but since we are working in a non-Gaussian regime of the model, we can expect both Fisher matrix and MCMC to give biased results. In Fig.~\ref{fig:CCsplit1} we show MCMC posteriors for $z>1$ CC data in blue alongside posteriors for low redshift ($z < 1$) CC data, which is simply added to aid comparison and also highlight the Gaussianity of the low redshift posteriors. One notes that the peaks of the $z > 1$ distributions are a little displaced from to the values minimising the $\chi^2$. However, the emergence of the lower peak in the $H_0$ posterior at $H_0 \sim 50$ km/s/Mpc has the hallmarks of a projection effect. To appreciate this, note that the configurations in the blue curve in the top left corner of the 2D posterior are projected onto the lower $H_0$ peak. Moreover, if one shifts the $H_0$ peak from $H_0 \sim 150$ to $H_0 \sim 50$ km/s/Mpc while maintaining $\Omega_m \sim 0$, this shifts the magenta curve in Fig. \ref{fig:CC} outside of all the data points, so the lower $H_0$ peak is a phantom artefact unrelated to the goodness of fit. We also observe a shift in the higher $H_0$ peak away from the minimum of the $\chi^2$.

Ignoring these features, one could attempt to interpret the overlap in the 2D posteriors in Fig. \ref{fig:CCsplit1}. Doing so, one may conclude that low and high redshift CC data are consistent within $1 \sigma$. However, since Hubble tension is a 1D problem (local $H_0$ determinations are insensitive to other parameters), to compare with locally observed values of $H_0$ one needs to project onto the $H_0$ axis. Alternatively put, Hubble tension is a problem in 1D posteriors. Projecting onto the $H_0$ axis by determining $16^{\textrm{th}}$, $50^{\textrm{th}}$ and $84^{\textrm{th}}$ percentiles, one sees from Table \ref{tab:LCDM_CC} that the $z_{\textrm{min}} = 1$ MCMC confidence interval encloses the $z_{\textrm{min}} = 0$ central values within $1 \sigma$,\footnote{Note, removing the eight high redshift data points from the $z_{\textrm{min}} = 0$ sample will not shift the central values much.} but not the point in parameter space that best fits the data!


% Figure environment removed



Evidently, given the non-Gaussian posteriors, care is required when interpreting the significance of the trend towards a non-evolving (horizontal) $H(z)$ at higher redshifts in Fig.~\ref{fig:CC}. We cannot use the errors from the Fisher matrix as we are clearly in a non-Gaussian regime, whereas MCMC inferences are impacted by projection effects to the extent that the minimum of the $\chi^2$ (confirmed from the MCMC chain) falls outside of the $1 \sigma$ confidence interval. For this reason, we resort to mock simulations. While this may seem a little redundant if we are going to employ profile distributions in section \ref{sec:PD}, there is motivation for this exercise. In \cite{Colgain:2022rxy} the significance of a descending $H_0$/increasing $\Omega_m$ trend with effective redshift in OHD, Type Ia SN and QSOs was estimated to be a $\sim 3 \sigma$ effect on the basis of combining $\sim 2 \sigma$ effects in each of the \textit{independent} data sets using Fisher's method. Here, working with the same data throughout, we can directly compare the significance of a discrepancy estimated through mock simulations from the significance of a discrepancy estimated through profile distributions. In particular, we will address the question: how significant is a constant $H(z)$ with $z_{\textrm{min}}=1$ (8 data points) against the Planck consistent cosmology favoured by the full data set ($z_{\textrm{min}}=0$ entry in Table \ref{tab:LCDM_CC})? Note, the significance will be overestimated due to missing systematic uncertainties (see \cite{Moresco:2020fbm}), but we can still make comparison between the two techniques.

\paragraph{{Mock simulations:}} To address this question using mock simulations, we begin with the MCMC chains for the full sample. For each entry in the MCMC chain (approximately 15,000 entries in total), we generate a new realisation of the 8 high redshift data points $(z > 1)$ that are by construction statistically consistent with both the best fits from the full sample and also the Planck-$\Lambda$CDM values \cite{Planck:2018vyg}. More concretely, for each $(H_0, \Omega_m)$ entry in our MCMC chain, we displace the data points to the corresponding $\Lambda$CDM Hubble parameter before generating new data points in a normal distribution where the errors serve as standard deviations. We then fit back the $\Lambda$CDM model to each realisation of the mock data and record the best fit $(H_0, \Omega_m)$ values, which give us a distribution of expected $(H_0, \Omega_m)$ best fits. The distributions are presented in Fig.~\ref{fig:CCsims} alongside the best fits from observed data. Throughout, we assume canonical values $(H_0, \Omega_m) = (70, 0.3)$ for the initial guess of the fitting algorithm. Best fits can saturate our bounds, i. e. $\Omega_m = 0$ and $\Omega_m = 1$, and this leads to an unsightly pile up of best fits at $\Omega_m = 0$ and $\Omega_m = 1$ in Fig.~\ref{fig:CCsims} \cite{Colgain:2022rxy}. It is important to retain all the configurations, otherwise one is not accounting for the probability that a best fit falls outside our priors. As a consistency check, we see that the median or 50$^{\textrm{th}}$ percentile, $(H_0, \Omega_m) = (68.32, 0.321)$ agrees well with the mock input parameters, thereby demonstrating that there are an equal number of best fits with values above and below the injected parameters in the mocks. We find that probability of a more extreme (larger) $H_0$ value to be $p = 0.022$, while the probability of a more extreme (smaller) $\Omega_m$ value to be $p = 0.035$, respectively. Converted into a Gaussian statistic, these correspond to $2 \sigma$ and $1.8 \sigma$, respectively, for a one-sided normal distribution. Thus, on the basis of mock simulations, we estimate the non-evolving constant $H(z)$ with $z_{\textrm{min}} = 1$ as a $\sim 2 \sigma$ effect. In the next section we will recover this number more or less from the profile distribution analysis. 

% Figure environment removed


\section{Profile Distributions}
\label{sec:PD}
Having explained the mathematics behind the bias, which gives rise to a projection effect, in subsection \ref{sec:math}, and having illustrated how it affects MCMC inferences in subsection \ref{sec:CCbias} - the minimum of the $\chi^2$ may fall outside of $1 \sigma$ confidence intervals - we turn to profile distributions (PDs) \cite{Gomez-Valent:2022hkb}, an extension of the profile likelihood, e. g. \cite{Trotta:2017wnx}, in order to address the bias. Consider two sets of parameters $\theta_1$ and $\theta_2$ and a normalised distribution $\mathcal{P}(\theta_1, \theta_2)$. The basic idea \cite{Gomez-Valent:2022hkb} is to study the ratio 
\be
\label{R}
R(\theta_1) = \frac{\tilde{\mathcal{P}}(\theta_1)}{\max_{\theta_1} \tilde{\mathcal{P}}(\theta_1) } = \frac{\tilde{\mathcal{P}}(\theta_1)}{\max_{\theta_1, \theta_2} \mathcal{P}(\theta_1, \theta_2) },  
\ee
where $\tilde{\mathcal{P}}(\theta_1)$ is the PD, defined to be the maximum of $\mathcal{P}$ for each $\theta_1$ along the $\theta_2$ direction: 
\be
\label{PD}
\tilde{\mathcal{P}} (\theta_1) = \max_{\theta_2} \mathcal{P}(\theta_1, \theta_2). 
\ee
The advantage of this approach is that $R(\theta_1)$ can serve as a probability distribution function (up to an overall normalization), however we do not need to perform any integration, so $R(\theta_1)$ is not prone to volume or projection effects. At this juncture, given the simplicity of our setup with only two parameters $(H_0, \Omega_m)$, we can be more explicit. Consider the probability distribution,   
\be
\mathcal{P}(\theta_1, \theta_2) = \exp \left( - \frac{1}{2} \chi^2(\theta_1, \theta_2) \right), 
\ee
where $\theta_i \in \{H_0, \Omega_m \}$  and $\chi^2(H_0, \Omega_m)$ is our earlier likelihood (\ref{eq:chi2}). The maximum value of $\mathcal{P}$ occurs for the minimum value of $\chi^2$ from the MCMC chain, $\mathcal{P}_{\textrm{max}} = e^{-\frac{1}{2} \chi^2_{\textrm{min}}}$. In this concrete setting, the PD becomes 
\be
\tilde{\mathcal{P}}(\theta_1) = e^{-\frac{1}{2} \chi^2_{\textrm{min}}(\theta_1)}, 
\ee
where $\chi^2_{\textrm{min}}(\theta_1)$ denotes the minimum value of the $\chi^2$ along the $\theta_2$ direction for a fixed $\theta_1$ value. It should not be confused with the overall minimum $\chi^2_{\textrm{min}}$, which can be extracted easily from the MCMC chain. In practice, one can also determine $\chi^2_{\textrm{min}}(\theta_1)$ from the MCMC chain by breaking the $\theta_1$ direction up into bins and finding the minimum of the $\chi^2$ for each bin. Having done so, we are in a position to define a PDF \cite{Gomez-Valent:2022hkb}: 
\be
\label{eq:w}
w(\theta_1) = \frac{e^{-\frac{1}{2} \chi^2_{\textrm{min}}(\theta_1)}}{\int e^{-\frac{1}{2} \chi^2_{\textrm{min}}(\theta_1)} \, \textrm{d} \theta_1} = \frac{R(\theta_1)}{\int R(\theta_1) \, \textrm{d} \theta_1}, 
\ee
where in the second equality we have divided top and bottom by $\mathcal{P}_{\textrm{max}} = e^{-\frac{1}{2} \chi^2_{\textrm{min}}}$. As a result, $R(\theta_1) = e^{-\frac{1}{2} \Delta \chi_{\textrm{min}}^2}$, where $\Delta \chi^2_{\textrm{min}} := \chi_{\textrm{min}}^2(\theta_1) - \chi^2_{\textrm{min}}$, so that $R(\theta_1)$ peaks at $R(\theta_1) = 1$. Note that $\int_{-\infty}^{+\infty} w(\theta_1) \, \textrm{d} \theta_1 = 1$ by construction, so $w(\theta_1)$ describes a properly normalised PDF. Thus we can identify the $1 \sigma, 2 \sigma$ and $3 \sigma$ confidence intervals corresponding to the 68\%, 95\% and 99.7\% confidence level, respectively, by simply identifying $\theta_1^{(1)}$ and $\theta_1^{(2)}$ such that \cite{Gomez-Valent:2022hkb}
\be
\label{eq:wsigma}
\int_{\theta_1^{(1)}}^{\theta_1^{(2)}} w(\theta_1) \, \textrm{d} \theta_1 = I, \quad w(\theta_1) = w(\theta_2), \quad I \in \{0.68, 0.95, 0.997\}. 
\ee
We will outline how these conditions can most easily be satisfied when we turn to explicit examples. 

Our first port of call is making sure that the PD methodology gives sensible results. This can be best judged by applying it to the CC data with $z_{\textrm{min}} = 0$, since this is where we expect a distribution closest to a Gaussian distribution, as is evident from the agreement between Fisher matrix and MCMC results in Table \ref{tab:LCDM_CC}. In particular, we will be interested in a comparison between $1 \sigma$ confidence intervals to make sure that (\ref{eq:wsigma}) is not underestimating or overestimating the $1 \sigma$ confidence interval. 

% Figure environment removed

We start by running a long MCMC chain (100,000 iterations) in order to ensure bins are well populated, and begin by analysing $\theta_1 = H_0$ with $\theta_2 = \Omega_m$. From the MCMC chain we identify the smallest and largest value of $H_0$ in the chain and break up this range into approximately 200 uniform bins, which we label using the $H_0$ value at the centre of the bin. We omit any empty bins. One can increase the number of bins by simply running a longer MCMC chain. In each $H_0$ bin we identify the minimum value of the $\chi^2$, $\chi^2_{\textrm{min}}(H_0)$, and calculate $R(H_0)$. One then repeats the steps for $\Omega_m$. In Fig.~\ref{fig:R_zmin0} we plot $R(H_0)$ against $H_0$ and $R(\Omega_m)$ against $\Omega_m$, noting that the distributions are Gaussian to first approximation. 

Since the distributions from the MCMC chain are sparse in the tails, empty bins are evident in Fig.~\ref{fig:R_zmin0}. Nevertheless, with 200 bins, modulo any empty bins, we have sufficient density of points to calculate the total area under the $R(H_0)$ and $R(\Omega_m)$ curve using Simpson's rule. Any concern about precision can simply be mitigated by running a longer MCMC chain and increasing the number of bins. 
One may directly use $R(H_0)\leq 1$ and $R(\Omega_m)\leq 1$   to find $68$, $95$ and $99.7$ percentiles,  respectively corresponding to $1 \sigma, 2 \sigma$ and $3 \sigma$ confidence intervals. Consider $F_\kappa:= \int_{R\geq \kappa} R (\theta_1) \, \textrm{d} \theta_1$, where $\kappa \leq 1$. Observe that $F_{\kappa=1}=0$ and $F_{\kappa=0}:=F_0=\int R(\theta_1) \textrm{d} \, \theta_1$. Then move $\kappa$ through and terminate the process when $F_\kappa/F_0$ is equal to $0.68$, $0.95$ and $0.997$. This gives the corresponding range for $\theta_1$ that defines the confidence interval.
Working with the precision afforded to us by approximately 200 bins, the $H_0$ and $\Omega_m$ $1 \sigma$ confidence intervals are presented in Fig.~\ref{fig:R_zmin0} and the first entry in Table \ref{tab:LCDM_CC_PD}. The outcome is in excellent agreement with both Fisher matrix and MCMC analysis. In particular, a mild non-Gaussianity in $\Omega_m$ is evident in both Fig.~\ref{fig:R_zmin0} and the errors. 
Thus, we have succeeded in recovering results in the (almost) Gaussian regime that are consistent with Fisher matrix and MCMC analysis and this provides an important check of the methodology.  

% Figure environment removed

We now apply the same PD methodology to the non-Gaussian regime where MCMC marginalisation leads to biased results. To be concrete, we focus on the eight data points in the range $1 < z < 2$ where a non-evolving $H(z)$ trend is evident in the raw data in Fig.~\ref{fig:CC}. Our goal here is to quantify the disagreement with the full data set, where one infers $H_0 \sim 68$ km/s/Mpc and $\Omega_m \sim 0.32$. A similar exercise was performed in subsection \ref{sec:features} with mock simulations and the disagreement was estimated to be approximately $2 \sigma$. Repeating the steps outlined above for the CC data with $z_{\textrm{min}} = 1$ we find the distributions in Fig.~\ref{fig:R_zmin1}. The first observation is that the distributions are non-Gaussian, but a comparison to the MCMC posteriors from the same data in blue in Fig.~\ref{fig:CCsplit1} reveals that there is no secondary $H_0$ peak at $H_0 \sim 50$ km/s/Mpc. Thus, we confirm the secondary peak to be a projection effect. That being said, the primary $H_0$ peak from Fig.~\ref{fig:CCsplit1} has shifted to the dashed line corresponding to the minimum of the $\chi^2$, since the peak of the distribution and $\chi^2$ minimum agree by construction. Comparing the blue $\Omega_m$ distribution from Fig.~\ref{fig:CCsplit1} to the $R(\Omega_m)$ distribution in Fig.~\ref{fig:R_zmin1}, we see that the peak is close to $\Omega_m = 0$ and that the tails continue to $\Omega_m = 1$. In both plots we see that there is a non-zero probability of inferring $\Omega_m = 1$. In some sense, this is not so surprising, the reason being that one is free to adopt generous priors for $H_0$, so that probability of large and small $H_0$ values is zero, but the priors on $\Omega_m$ in the flat $\Lambda$CDM model are restricted. For this reason, as a distribution spreads one invariably finds that distributions are impacted by the $\Omega_m$ priors.\footnote{Note, this is a problem for the flat $\Lambda$CDM model. In particular, one may easily find that the peak of the $\Omega_m$ distribution is larger than $\Omega_m=1$, as is the case with Hubble Space Telescope SN with redshifts $z > 1$ in the Pantheon+ sample \cite{Malekjani:2023dky}.}

It is evident from Fig.~\ref{fig:R_zmin1} that any tension that exists is confined to the $H_0$ parameter. Moreover, since there may be only one binned value of $\Omega_m$ below the $R(\Omega_m)$ peak, at the precision afforded to us by 200 bins, the $R(\Omega_m)$ distribution in Fig.~\ref{fig:R_zmin1} is essentially one-sided and the $1 \sigma$ confidence interval stretches beyond $\Omega_m \sim 0.32$, so there is no disagreement in the $\Omega_m$ parameter. Nevertheless, in the $H_0$ parameter we see that $H_0 \sim 68$ km/s/Mpc, the value favoured by the full data set is just under $2 \sigma$ removed from the peak. The main point here is that, as is obvious from the raw data, current CC data with $z > 1$ has a preference for a non-evolving Hubble parameter $H(z)$ with a large constant $H_0 \sim 150$ km/s/Mpc. The disagreement is just under $2 \sigma$, more accurately $1.9 \sigma$ from $R(H_0)$, and only $0.9 \sigma$ from $R(\Omega_m)$. Although this may not be a serious discrepancy, essentially because of the poor data quality (8 data points), this disagreement supports the $\sim 2 \sigma$ discrepancy seen in the mock simulations. It should be borne in mind that systematic uncertainties have been omitted and these will reduce this discrepancy once properly propagated. Given the agreement between the PD and mock simulation analysis, there is nothing to suggest that the three independent trends highlighted in \cite{Colgain:2022rxy} across OHD, Type Ia SN and QSOs are not \textit{bona fide} disagreements and that redshift evolution is present in the sample. The task remains to combine them at the level of a $\chi^2$ likelihood instead of combining them using Fisher's method on the basis that they are independent probabilities. We leave this exercise for a forthcoming paper, but revisit the tension in OHD data in the following section.  %\ref{sec:tension}. 
For completeness, in Table \ref{tab:LCDM_CC_PD} we perform a reanalysis of CC data subsets with the PD approach and record the $1 \sigma$ intervals.  

\begin{table}[htb]
    \centering
    \begin{tabular}{c|c|c|c}
    \rule{0pt}{3ex} $z_{\textrm{min}}$ & \# CC & \multicolumn{2}{c}{PD}  \\
    \hline
    \rule{0pt}{3ex} & & $H_0$ (km/s/Mpc) & $\Omega_m$ \\
    \hline
    \rule{0pt}{3ex} $0$ & $34$ & $68.15^{+3.04}_{-3.11}$ & $0.320^{+0.065}_{-0.055}$ \\
    \hline 
    \rule{0pt}{3ex} $0.2$ & $27$ & $65.03^{+6.52}_{-7.03}$ & $0.368^{+0.167}_{-0.110}$ \\
    \hline 
    \rule{0pt}{3ex} $0.4$ & $22$ & $62.42^{+7.78}_{-8.74}$ & $0.411^{+0.236}_{-0.113}$ \\
    \hline
    \rule{0pt}{3ex} $0.6$ & $15$ & $59.75^{+11.73}_{-13.97}$ & $0.455^{+0.355}_{-0.160}$ \\
    \hline
    \rule{0pt}{3ex} $0.7$ & $14$ & $79.10^{+16.42}_{-20.56}$ & $0.222^{+0.386}_{-0.117}$ \\
    \hline
    \rule{0pt}{3ex} $0.8$ & $11$ & $103.94^{+22.88}_{-28.54}$ & $0.097^{+0.378}_{-0.074}$ \\
    \hline
    \rule{0pt}{3ex} $1$ & $8$ & $150.35^{+17.12}_{-35.95}$ & $ < 0.339$ \\
    \hline
    \rule{0pt}{3ex} $1.2$ & $7$ & $154.26^{+14.88}_{-54.82}$ & $ < 0.570$ \\
    \hline
    \rule{0pt}{3ex} $1.4$ & $4$ & $124.81^{+35.38}_{-52.60}$ & $ < 0.661$ \\
    \hline
    \rule{0pt}{3ex} $1.5$ & $3$ & $36.11^{+72.87}_{-2.43}$ & $ > 0.354$
    \end{tabular}
    \caption{Same as Table \ref{tab:LCDM_CC} but with the PD methodology in lieu of Fisher matrix and MCMC analysis. The high redshift $R(\Omega_m)$ distributions are typically one-sided, so one encounters $1 \sigma$ upper and lower bounds.}
    \label{tab:LCDM_CC_PD}
\end{table}




\section{A tension with Planck}
\label{sec:tension}
A $2 \sigma$ ($p = 0.021$) tension with Planck has been reported in OHD through best fits and mock simulations in \cite{Colgain:2022rxy}. In particular, it was noted that a combination of 7 CC and BAO data points above $z = 1.45$ resulted in a $(H_0, \Omega_m) = (37.8, 1)$ best fit, where in line with analysis here, an $\Omega_m \in [0, 1]$ uniform prior was assumed. Based on mock simulations, the probability of such a best fit configuration arising by chance in mocks assuming input parameters consistent with Planck was estimated to be $p = 0.021$ \cite{Colgain:2022rxy}. A similar best fit appears in the last entry of Table \ref{tab:LCDM_CC} and Table \ref{tab:LCDM_CC_PD}, but there is no tension with Planck within the errors, even with our PD analysis, because CC data is inherently of poorer quality than BAO data. One further difference between the analysis is that \cite{Colgain:2022rxy} imposes a Gaussian Planck prior $\Omega_m h^2 = 0.1430 \pm 0.0011$ \cite{Planck:2018vyg} \footnote{This prior essentially prevents high redshift CC data from tracking a non-evolving $H(z)$.} to fix the high redshift behaviour of $H(z)$, whereas our analysis here so far has not introduced a prior. 

% Figure environment removed

Nevertheless, armed with a new PD methodology, we are in a position to revisit the earlier result and see if we can recover the $2 \sigma$ tension with Planck. Since \cite{Colgain:2022rxy} made use of older BAO data, here we replace QSO and Lyman-$\alpha$ BAO with the latest eBOSS results \cite{Hou:2020rse, Neveux:2020voa, duMasdesBourboux:2020pck}. Moreover, we work directly with the $D_{H}/r_d$ constraints and do not invert them. This entails assuming a value for the radius of the sound horizon, which we take to be the Planck value, $r_d = 147.09 \pm 0.26$ Mpc \cite{Planck:2018vyg}. In addition, we reinstate the prior $\Omega_m h^2 = 0.1430 \pm 0.0011$, so that the only difference with \cite{Colgain:2022rxy} is simply to update OHD BAO to the latest constraints. We stress that the priors we introduce are consistent with the Planck cosmology, so \textit{they cannot be driving any disagreement}. Moreover, the $\Omega_m h^2$ prior restricts one to a curve in the $(H_0, \Omega_m)$, but it cannot dictate where one is on the curve, this is done by the remaining 3 CC and 3 BAO data points.  

We again marginalise over the free parameters $(H_0, \Omega_m, r_d)$ with MCMC. In Fig.~\ref{fig:CC_BAO_MCMC} we present the posteriors. While $r_d$ is Gaussian and peaked on our Planck prior, as expected, the $\Omega_m$ posterior is peaked at $\Omega_m \sim 0.6$ and the fact that the fall off in the distribution is gradual beyond the peak leads to a pile up of configurations in the top left corner of the $(H_0, \Omega_m)$-plane. This fall off continues beyond $\Omega_m = 1$ and if the prior is relaxed, the $H_0$ peak shifts to smaller values. So,  once again all the hallmarks of projection effects are present. That being said, given the sharp fall off in the $\Omega_m$ distribution to smaller $\Omega_m$ values, some tension appears to be evident with the Planck values (dashed lines). 

% Figure environment removed

We now run the MCMC chain through our PD methodology. From Fig.~\ref{fig:CC_BAO}, we can see that the $R(H_0)$ and $R(\Omega_m)$ distributions prefer smaller values of $H_0$ and larger values of $\Omega_m$. The peak of the distributions occurs at $H_0 = 42.40$ km/s/Mpc and $\Omega_m = 0.795$.  The lone dot in the $R(H_0)$ distribution at low values of $H_0$ tells us that the distribution falls off sharply below $H_0 = 40$ km/s/Mpc. Note, since we employed generous uniform priors $H_0 \in [0, 200]$, the priors are not impacting the $R(H_0)$ distribution, so it is expected that the distribution falls off to zero on both sides. In contrast, the $R(\Omega_m)$ distribution is one-sided and fails to fall off in the direction of larger values within the uniform priors $\Omega_m \in [0, 1]$. The tension with Planck falls between $2 \sigma$ and $3 \sigma$. By integrating the PDF as far as the black lines corresponding to the Planck values in Fig.~\ref{fig:CC_BAO}, we estimate that the Planck $H_0$ is located at $2.1 \sigma$ from the peak, while the Planck $\Omega_m$ value is $2.5 \sigma$ from the peak.

The main take-away from this section is that OHD data comprising CC and BAO data points beyond $z=1.45$ is inconsistent with the Planck cosmology at in excess of $2 \sigma$. We have employed Planck priors to arrive at this result, but these priors cannot drive the disagreement. Moreover, independent analysis based on least squares fitting and mock simulations presented in \cite{Colgain:2022rxy} also points to a $2 \sigma$ tension, albeit with less up-to-date high redshift BAO data. In summary, different methodologies agree on a $2 \sigma$ discrepancy with Planck, which is robust to interchanging older and newer BAO data. 

\section{Concluding remarks}
\label{sec:discussion}
A $\chi^2$ likelihood is a metric or measure of how well a model fits data. The point in model parameter space that fits the data the best possesses the lowest $\chi^2$. Once one has identified this point, the problem remains to establish $1 \sigma$, $2 \sigma$, etc, confidence intervals in parameter space. In cosmology and astrophysics, MCMC is the prevailing technique for estimating confidence intervals. Its great advantage is that it allows one to i) globally sample the parameter space and ii) arrive at posteriors that serve as an estimate of the errors even with non-Gaussian distributions. In contrast, if one minimises the $\chi^2$ by gradient descent, there is always a risk that one ends up in a local minimum, i. e. the global minimum is missed, while error estimation through Fisher matrix assumes any distribution is Gaussian. The appeal of MCMC marginalisation is that it is widely applicable. However, the point of this paper is that limitations exist, even in the simplest model. 

Indeed, what happens when the MCMC posterior no longer tracks points in parameter space that fit the data better? Traditionally, volume effects are seen as the preserve of higher-dimensional models, e. g. \cite{Herold:2021ksg, Gomez-Valent:2022hkb, Meiers:2023gft}, but projection effects also occur in the minimal $\Lambda$CDM model when one fits the model to data binned by redshift in the late Universe \cite{Colgain:2022tql}. As explained in \cite{Colgain:2022tql}, this ``projection effect'' is driven by OHD, $H(z_i)$, and angular diameter or luminosity distance data, $D_{A}(z_i)$ or $D_{L}(z_i)$, {respectively} only constraining the combinations $\Omega_m h^2$ and $ (1-\Omega_m) h^2$ well, with high redshift data $z_i \gg 0$. In practice, this restricts MCMC configurations to constant $\Omega_m h^2$ and constant $(1-\Omega_m) h^2$ curves in the $(H_0, \Omega_m)$ plane, and as the curves stretch due to DE or matter being less well constrained in high redshift bins, projection effects lead to shifts in the peaks of MCMC posteriors and the emergence of non-Gaussian tails \cite{Colgain:2022tql}. We stress that one sees the same effect in PDFs of best fit $(H_0, \Omega_m)$ parameters in a large number of mock data realisations \cite{Colgain:2022tql}, so the problem is more general than MCMC; there is an inherent bias in the $\Lambda$CDM model when one fits it to redshift binned $H(z)$ \textit{or} $D_{A}(z)$ \textit{or} $D_{L}(z)$ data. Within MCMC, one sees this effect in the errors, but also in the drift of the parameters corresponding to the $\chi^2$ minimum outside of the $1 \sigma$ confidence intervals. Highlighting this (expected) bias in MCMC using OHD is the opening salvo (result) of this paper.     

Why should one care? This is evidently only a problem if one bins data and confronts the $\Lambda$CDM model. First, note that some data sets are inherently binned. For example, effective redshifts are assigned to CC and BAO analysed in a given redshift bin, while each strongly lensed system constitutes its own bin. Working with binned data is unavoidable. Secondly, $\Lambda$CDM tensions point to a problem with the $\Lambda$CDM model once the tensions become widespread and persistent. As explained in \cite{Krishnan:2020vaf}, if the minimal $\Lambda$CDM model is too simple, one expects redshift evolution of $\Lambda$CDM cosmological parameters as it is confronted to redshift binned data. Hints of these trends are now evident in $H_0$ \cite{Wong:2019kwg, Millon:2019slk, Dainotti:2021pqg, Colgain:2022nlb, Colgain:2022rxy, Malekjani:2023dky, Hu:2022kes, Jia:2022ycc, Krishnan:2020obg, Dainotti:2022bzg}, $\Omega_m$ \cite{Risaliti:2015zla, Risaliti:2018reu, Lusso:2020pdb, Yang:2019vgk, Khadka:2020vlh, Khadka:2020tlm, Khadka:2021xcc, Pourojaghi:2022zrh, Colgain:2022nlb, Colgain:2022rxy, Malekjani:2023dky, Pasten:2023rpc, Sakr:2023hrl} and $S_8$/$\sigma_8$ \cite{Esposito:2022plo, Adil:2023jtu, ACT:2023dou, ACT:2023kun} (also \cite{Miyatake:2021qjr, Alonso:2023guh}) across a host of different observables. This evolution is an expected hallmark of model breakdown, which must happen at some redshift if systematics are not universally at play. 

The main problem with redshift dependent $\Lambda$CDM cosmological parameters\footnote{There is a separate interpretation problem as the cosmology literature works with  parameters ``defined today''. In more mathematical language, this is simply the statement that one solves an ordinary differential equation (ODE), namely the Friedmann equation or equivalent, by specifying an integration constant, e.g. $H_0 = H(z=0)$ or $\rho_m(z=0)=\rho_{m0}=H_0^2\Omega_{m}$. However, this is a mathematical statement and it still needs to be confirmed observationally that $H_0$ or $\rho_{m0}$ are \textit{bona fide} constants. This cannot be \textit{a priori} assumed, because it is mathematical prediction of the model. If the model is correct, a constant $H_0$ and $\Omega_m$  will be supported by the data. See \cite{Krishnan:2020vaf} for further discussion.} is one needs to assign a statistical significance to any trend. At a purely practical level, this entails constructing bins centered on different redshifts and identifying discrepancies in $\Lambda$CDM parameters between bins, \textit{ideally in the same observable}, so that the potential systematics are under greatest control. As demonstrated both mathematically and observationally with the CC data in section \ref{sec:MCMC_bias}, MCMC marginalisation leads to biased inferences when one bins the data. In this paper we have resorted to profile distributions \cite{Gomez-Valent:2022hkb} to overcome this bias and have applied the technique to a setting where $\Lambda$CDM distributions are expected to be non-Gaussian for the reasons outlined above and in section \ref{sec:MCMC_bias}. This new technique, provides a complementary perspective that confirms the least square fits of observed and mock data presented in \cite{Colgain:2022nlb, Colgain:2022rxy, Malekjani:2023dky}, where evidence for redshift evolution in $H_0$ and $\Omega_m$ was presented. Regardless of the methodology, the objective is to drill down on the prevailing \textit{assumption} that cosmological parameters are constants. \textit{In the era of tensions in cosmology, nothing can be assumed, especially noting that the tensions are in essence showing an example of evolution of these parameters with redshift.}

More concretely, in this paper with both mock simulations and profile distributions we have shown that high redshift CC data has a preference for a non-evolving $H(z)$ over Planck-$\Lambda$CDM at approximately $\sim 2 \sigma$. This trend, which constitutes the second result of the paper, is unquestionable, as it is visible in the data. Note, we have not propagated systematic uncertainties, so the significance will be less when these are properly propagate. Nevertheless, low and high redshift CC data currently have a preference for different $\Lambda$CDM cosmological parameters. This is important because if the CC program is claiming an 8\% constraint on the Hubble constant, $H_0 = 66.7 \pm 5.5$ km/s/Mpc \cite{Moresco:2023zys}, it is imperative that \textit{all subsets of the data are consistent with this result}. If they are not, then we are staring at either systematics or model breakdown. Admittedly, demanding self-consistency of subsets of a data set confronted to a model is a high bar, but it is important that data sets result in overlapping constraints on $\Lambda$CDM parameters, otherwise this makes cosmological inferences moot. Note, the $\Lambda$CDM model is largely only well tested in the DE dominated regime $z \lesssim 1$ and at very high redshifts $z \sim 1100$, which leaves a wide expanse of redshifts to be explored in order to confirm or refute the model. Given the existing $\Lambda$CDM tensions \cite{Perivolaropoulos:2021jda, Abdalla:2022yfr}, and the hints of evolution in $H_0$, $\Omega_m$ and $S_8$ across assorted probes in the late Universe $z \lesssim 5$, it would be surprising if all discrepancies could be explained away by systematics.\footnote{We are open to the possibility, we just consider it a bad bet at the moment. The odds can of course change as observations improve.}

As an aside, it is intriguing that CC data has a preference for larger best fit values of $H_0$ and smaller best fit values of $\Omega_m$ beyond $z_{\textrm{min}} = 0.7$, as this is traditionally the transition redshift between decelerated and accelerated expansion. % where $\ddot{a} = 0$. 
Moreover, at higher redshifts $z \sim 2.3$, there is not only a longstanding anomaly in Lyman-$\alpha$ BAO \cite{duMasdesBourboux:2020pck}, but QSOs also show a preference for a lower luminosity distance, $D_{L}(z)$, relative to Planck-$\Lambda$CDM \cite{Risaliti:2015zla, Risaliti:2018reu}. Translated into $\Lambda$CDM parameters, this corresponds to conversely larger $\Omega_m$ values, e. g.  \cite{Yang:2019vgk, Khadka:2020vlh, Khadka:2020tlm, Khadka:2021xcc, Pourojaghi:2022zrh}, and consequently smaller $H_0$ values. Thus, the emerging probes CC and QSOs  \cite{Moresco:2022phi} do not appear to be in sync on high redshift $\Lambda$CDM inferences. Nevertheless, neither may be inconsistent with the anomaly in Lyman-$\alpha$ BAO. Relative to Planck-$\Lambda$CDM, Lyman-$\alpha$ BAO prefers \textit{smaller} values of $D_{M}(z) := c \int_{0}^z 1/H(z^{\prime}) \, \textrm{d} z$ and \textit{smaller} values of $H(z)$ (larger values of $D_{H}(z) := c/H(z)$).\footnote{In this statement we assumed the Planck value $r_d \sim 147$ Mpc \cite{Planck:2018vyg} If we reinstate the radius of the sound horizon in these expressions, one recognises that changing the sound horizon, as advocated by early Universe resolutions to Hubble tension, cannot consistently address the Lyman-$\alpha$ BAO anomaly. In general, even for the Planck-$\Lambda$CDM sound horizon, one cannot get both a smaller $D_{M}(z)$ and smaller $H(z)$ from a strictly increasing function, such as the $\Lambda$CDM $H(z)$. As a result, deviations from the Planck-$\Lambda$CDM model that address this anomaly are expected to lead to wiggles in $H(z)$ \cite{Akarsu:2022lhx}, which are unsurprisingly seen in data reconstructions \cite{Zhao:2017cud, Wang:2018fng, Escamilla:2021uoj}. Finally, evolution in $H_0, \Omega_m$ discussed here cannot be explained or accommodated by early resolutions to Hubble tension relying on a change in the $r_d$ at very high $z$.}. If CC data prefer less evolution in $H(z)$ in the matter-dominated regime, then this is consistent with the preference for a smaller $H(z)$ from Lyman-$\alpha$ BAO. Furthermore, QSO data prefers smaller luminosity distances $D_{L}(z)$ relative to Planck, which are consistent with the smaller $D_{M}(z) \propto D_{L}(z)$ values preferred by Lyman-$\alpha$ BAO. Thus, even if CC and QSOs appear to be showing diverging behaviour in the cosmological parameters $(H_0, \Omega_m)$, this may still turn out to be consistent with Lyman-$\alpha$ BAO. We await future DESI \cite{DESI:2023ytc} data releases to ascertain if the non-evolving $H(z)$ trend in high redshift CC data is physical or not. 

Finally, we come to our third and main result outlined in section \ref{sec:tension}. We have revisited a $\sim 2 \sigma$ tension between high redshift CC and BAO data reported in \cite{Colgain:2022rxy}, where the significance was estimated through mock simulations. Here, we have upgraded the BAO data to the latest constraints and again  recover a $>2 \sigma$ discrepancy in $(H_0, \Omega_m)$ with different methodology. This provides a consistency check that there is evolution in OHD between low and high redshifts in the late Universe. Note, this evolution runs contrary to the non-evolving $H(z)$ seen in high redshift CC data because it assumes Planck has accurately constrained the high redshift behaviour of the Hubble parameter in (\ref{eq:lcdm}). Nevertheless, both with and without a Planck prior on $\Omega_m h^2$, evolution at $ \gtrsim 2 \sigma$ is evident in OHD data. It should be stressed that evolution is evident in PDFs of best fit $\Lambda$CDM parameters fitted to a large number of Planck-$\Lambda$CDM mocks \cite{Colgain:2022tql}, so evolution in observed data can be expected. It is imperative to revisit the remaining observations in \cite{Colgain:2022rxy, Malekjani:2023dky} in order to confirm the significance of $\sim 2 \sigma$ hints of evolution found separately in Type Ia SN and QSO data sets. 




\acknowledgments
We would like to thank Adri\`a G\'omez-Valent for discussions and comments on the draft. We thank Gabriela Marques, Mike Hudson and Matteo Viel for related discussions on late Universe evolution in $S_8$. E\'OC thanks Yonsei University and Asia Pacific Center for Theoretical Physics for hospitality. 
This article/publication is based upon work from COST Action CA21136 – “Addressing observational tensions in cosmology with systematics and fundamental physics (CosmoVerse)”, supported by COST (European Cooperation in Science and Technology). SP and MMShJ acknowledge SarAmadan grant No. ISEF/M/401332. MMShJ thanks the support from ICTP associates office (under Senior Associate program) and ICTP HECAP section for hospitality.  


\appendix
\section{Fisher Matrix}
\label{sec:fisher}
Consider the $\chi^2$ (\ref{eq:chi2}). 
Defining $H_{\textrm{model}}(z) = H_0 \sqrt{1-\Omega_m + \Omega_m (1+z)^3}$ and $Q_i$ as in \eqref{eq:Q}, we can now work out the derivatives
\begin{equation}
    \begin{split}
\partial_{H_0} Q_i &= -\sqrt{1-\Omega_m + \Omega_m (1+z_i)^3}, \\  \partial_{\Omega_m} Q_i &= - \frac{1}{2} H_0 (z_i^3 + 3 z_i^2 + 3 z_i)/\sqrt{1-\Omega_m + \Omega_m (1+z_i)^3}, \\
\partial^2_{H_0} Q_i &= 0, \\
\partial_{H_0} \partial_{\Omega_m} Q_i &= - \frac{1}{2} (z_i^3 + 3 z_i^2 + 3 z_i)/\sqrt{1-\Omega_m + \Omega_m (1+z_i)^3}, \\
\partial^2_{\Omega_m} Q_i =& \frac{1}{4} H_0 (z_i^3 + 3 z_i^2 + 3 z_i)^2/(1-\Omega_m + \Omega_m (1+z_i)^3)^{\frac{3}{2}}.      
    \end{split}
\end{equation}
We can then define the Fisher matrix 
\be
F_{ij} = \frac{1}{2} \frac{\partial^2 \chi^2(H_0, \Omega_m)}{\partial p_i \partial p_j}
\ee
where $p_i \in \{ H_0, \Omega_m \}$. Note that the Fisher matrix is evaluated on the best fit parameters. The result is a $2 \times 2$ matrix, which one inverts and the estimated errors are the square root of the diagonal entries. 








\begin{thebibliography}{99}

\bibitem{Planck:2018vyg}
N.~Aghanim \textit{et al.} [Planck],
``Planck 2018 results. VI. Cosmological parameters,''
Astron. Astrophys. \textbf{641} (2020), A6
% doi:10.1051/0004-6361/201833910
%[arXiv:1807.06209 [astro-ph.CO]].

\bibitem{Riess:1998cb}
A.~G.~Riess \textit{et al.} [Supernova Search Team],
``Observational evidence from supernovae for an accelerating universe and a cosmological constant,''
Astron. J. \textbf{116} (1998), 1009-1038
% doi:10.1086/300499
%[arXiv:astro-ph/9805201 [astro-ph]].
%13031 citations counted in INSPIRE as of 02 Feb 2021

\bibitem{Perlmutter:1998np}
S.~Perlmutter \textit{et al.} [Supernova Cosmology Project],
``Measurements of $\Omega$ and $\Lambda$ from 42 high redshift supernovae,''
Astrophys. J. \textbf{517} (1999), 565-586
% doi:10.1086/307221
%[arXiv:astro-ph/9812133 [astro-ph]].
%13057 citations counted in INSPIRE as of 02 Feb 2021

\bibitem{Eisenstein:2005su}
D.~J.~Eisenstein \textit{et al.} [SDSS],
``Detection of the Baryon Acoustic Peak in the Large-Scale Correlation Function of SDSS Luminous Red Galaxies,''
Astrophys. J. \textbf{633} (2005), 560-574
%doi:10.1086/466512
%[arXiv:astro-ph/0501171 [astro-ph]].
%3380 citations counted in INSPIRE as of 08 Oct 2020

\bibitem{Riess:2021jrx}
A.~G.~Riess, W.~Yuan, L.~M.~Macri, D.~Scolnic, D.~Brout, S.~Casertano, D.~O.~Jones, Y.~Murakami, L.~Breuval and T.~G.~Brink, \textit{et al.}
``A Comprehensive Measurement of the Local Value of the Hubble Constant with 1 km s$^{?1}$ Mpc$^{?1}$ Uncertainty from the Hubble Space Telescope and the SH0ES Team,''
Astrophys. J. Lett. \textbf{934} (2022) no.1, L7
%doi:10.3847/2041-8213/ac5c5b
%[arXiv:2112.04510 [astro-ph.CO]].
%370 citations counted in INSPIRE as of 09 Jan 2023

\bibitem{Freedman:2021ahq}
W.~L.~Freedman,
``Measurements of the Hubble Constant: Tensions in Perspective,''
Astrophys. J. \textbf{919} (2021) no.1, 16
%doi:10.3847/1538-4357/ac0e95
%[arXiv:2106.15656 [astro-ph.CO]].
%179 citations counted in INSPIRE as of 09 Jan 2023

\bibitem{Pesce:2020xfe}
D.~W.~Pesce, J.~A.~Braatz, M.~J.~Reid, A.~G.~Riess, D.~Scolnic, J.~J.~Condon, F.~Gao, C.~Henkel, C.~M.~V.~Impellizzeri and C.~Y.~Kuo, \textit{et al.}
%``The Megamaser Cosmology Project. XIII. Combined Hubble constant constraints,''
Astrophys. J. Lett. \textbf{891} (2020) no.1, L1
%doi:10.3847/2041-8213/ab75f0
%[arXiv:2001.09213 [astro-ph.CO]].
%96 citations counted in INSPIRE as of 12 Jul 2021

\bibitem{Blakeslee:2021rqi}
J.~P.~Blakeslee, J.~B.~Jensen, C.~P.~Ma, P.~A.~Milne and J.~E.~Greene,
%``The Hubble Constant from Infrared Surface Brightness Fluctuation Distances,''
Astrophys. J. \textbf{911} (2021) no.1, 65
%doi:10.3847/1538-4357/abe86a
%[arXiv:2101.02221 [astro-ph.CO]].
%11 citations counted in INSPIRE as of 12 Jul 2021

\bibitem{Kourkchi:2020iyz}
E.~Kourkchi, R.~B.~Tully, G.~S.~Anand, H.~M.~Courtois, A.~Dupuy, J.~D.~Neill, L.~Rizzi and M.~Seibert,
%``Cosmicflows-4: The Calibration of Optical and Infrared Tully\textendash{}Fisher Relations,''
Astrophys. J. \textbf{896} (2020) no.1, 3
%doi:10.3847/1538-4357/ab901c
%[arXiv:2004.14499 [astro-ph.GA]].
%15 citations counted in INSPIRE as of 12 Jul 2021

\bibitem{HSC:2018mrq}
C.~Hikage \textit{et al.} [HSC],
``Cosmology from cosmic shear power spectra with Subaru Hyper Suprime-Cam first-year data,''
Publ. Astron. Soc. Jap. \textbf{71}, 43  (2019).
%doi:10.1093/pasj/psz010

\bibitem{KiDS:2020suj}
M.~Asgari \textit{et al.} [KiDS],
``KiDS-1000 Cosmology: Cosmic shear constraints and comparison between two point statistics,''
Astron. Astrophys. \textbf{645} (2021), A104
%doi:10.1051/0004-6361/202039070
%[arXiv:2007.15633 [astro-ph.CO]].
%113 citations counted in INSPIRE as of 18 Aug 2021

\bibitem{DES:2021wwk}
T.~M.~C.~Abbott \textit{et al.} [DES],
``Dark Energy Survey Year 3 results: Cosmological constraints from galaxy clustering and weak lensing,''
Phys. Rev. D \textbf{105} (2022) no.2, 023520
%doi:10.1103/PhysRevD.105.023520
%[arXiv:2105.13549 [astro-ph.CO]].
%519 citations counted in INSPIRE as of 14 Jul 2023

\bibitem{Boruah:2019icj}
S.~S.~Boruah, M.~J.~Hudson and G.~Lavaux,
``Cosmic flows in the nearby Universe: new peculiar velocities from SNe and cosmological constraints,''
Mon. Not. Roy. Astron. Soc. \textbf{498} (2020) no.2, 2703-2718
%doi:10.1093/mnras/staa2485
%[arXiv:1912.09383 [astro-ph.CO]].
%54 citations counted in INSPIRE as of 14 Jul 2023

\bibitem{Said:2020epb}
K.~Said, M.~Colless, C.~Magoulas, J.~R.~Lucey and M.~J.~Hudson,
``Joint analysis of 6dFGS and SDSS peculiar velocities for the growth rate of cosmic structure and tests of gravity,''
Mon. Not. Roy. Astron. Soc. \textbf{497} (2020) no.1, 1275-1293
%doi:10.1093/mnras/staa2032
%[arXiv:2007.04993 [astro-ph.CO]].
%49 citations counted in INSPIRE as of 14 Jul 2023

\bibitem{Perivolaropoulos:2021jda}
L.~Perivolaropoulos and F.~Skara,
``Challenges for \ensuremath{\Lambda}CDM: An update,''
New Astron. Rev. \textbf{95}, 101659  (2022).
%doi:10.1016/j.newar.2022.101659
%\href{https://arxiv.org/abs/2105.05208}{2105.05208}

\bibitem{Abdalla:2022yfr}
E.~Abdalla, G.~Franco Abell\'an, A.~Aboubrahim, A.~Agnello, O.~Akarsu, Y.~Akrami, G.~Alestas, D.~Aloni, L.~Amendola and L.~A.~Anchordoqui, \textit{et al.}
``Cosmology intertwined: A review of the particle physics, astrophysics, and cosmology associated with the cosmological tensions and anomalies,''
JHEAp \textbf{34}, 49  (2022).
%doi:10.1016/j.jheap.2022.04.002
%\href{https://arxiv.org/abs/2203.06142}{2203.06142}

\bibitem{Phillips:1993ng}
M.~M.~Phillips,
``The absolute magnitudes of Type IA supernovae,''
Astrophys. J. Lett. \textbf{413} (1993), L105-L108
%doi:10.1086/186970
%1245 citations counted in INSPIRE as of 24 Aug 2021

\bibitem{NearbySupernovaFactory:2018qkd}
M.~Rigault \textit{et al.} [Nearby Supernova Factory],
``Strong Dependence of Type Ia Supernova Standardization on the Local Specific Star Formation Rate,''
Astron. Astrophys. \textbf{644} (2020), A176
%doi:10.1051/0004-6361/201730404
%[arXiv:1806.03849 [astro-ph.CO]].
%143 citations counted in INSPIRE as of 20 Jul 2023

\bibitem{Kang:2019azh}
Y.~Kang, Y.~W.~Lee, Y.~L.~Kim, C.~Chung and C.~H.~Ree,
``Early-type Host Galaxies of Type Ia Supernovae. II. Evidence for Luminosity Evolution in Supernova Cosmology,''
Astrophys. J. \textbf{889} (2020) no.1, 8
%doi:10.3847/1538-4357/ab5afc
%[arXiv:1912.04903 [astro-ph.GA]].
%56 citations counted in INSPIRE as of 20 Jul 2023

\bibitem{Brout:2020msh}
D.~Brout and D.~Scolnic,
``It\textquoteright{}s Dust: Solving the Mysteries of the Intrinsic Scatter and Host-galaxy Dependence of Standardized Type Ia Supernova Brightnesses,''
Astrophys. J. \textbf{909} (2021) no.1, 26
%doi:10.3847/1538-4357/abd69b
%[arXiv:2004.10206 [astro-ph.CO]].
%82 citations counted in INSPIRE as of 20 Jul 2023

\bibitem{Lee:2021txi}
Y.~W.~Lee, C.~Chung, P.~Demarque, S.~Park, J.~Son and Y.~Kang,
``Evidence for strong progenitor age dependence of type Ia supernova luminosity standardization process,''
Mon. Not. Roy. Astron. Soc. \textbf{517} (2022) no.2, 2697-2708
%doi:10.1093/mnras/stac2840
%[arXiv:2107.06288 [astro-ph.GA]].
%5 citations counted in INSPIRE as of 20 Jul 2023


\bibitem{Krishnan:2020vaf}
C.~Krishnan, E.~\'O~Colg\'ain, M.~M.~Sheikh-Jabbari and T.~Yang,
``Running Hubble Tension and a H0 Diagnostic,''
Phys. Rev. D \textbf{103} (2021) no.10, 103509
%doi:10.1103/PhysRevD.103.103509
%[arXiv:2011.02858 [astro-ph.CO]].
%65 citations counted in INSPIRE as of 14 Jul 2023 

\bibitem{Krishnan:2022fzz}
C.~Krishnan and R.~Mondol,
``$H_0$ as a Universal FLRW Diagnostic,''
[arXiv:2201.13384 [astro-ph.CO]].
%12 citations counted in INSPIRE as of 14 Jul 2023

%\bibitem{Liao:2020zko}
%K.~Liao, A.~Shafieloo, R.~E.~Keeley and E.~V.~Linder,
%``Determining Model-independent H 0 and Consistency Tests,''
%Astrophys. J. Lett. \textbf{895} (2020) no.2, L29
%doi:10.3847/2041-8213/ab8dbb
%[arXiv:2002.10605 [astro-ph.CO]].
%51 citations counted in INSPIRE as of 14 Jul 2023

%\bibitem{Montani:2023xpd}
%G.~Montani, M.~De Angelis, F.~Bombacigno and N.~Carlevaro,
%``Metric $f(R)$ gravity with dynamical dark energy as a paradigm for the Hubble Tension,''
%[arXiv:2306.11101 [gr-qc]].
%1 citations counted in INSPIRE as of 14 Jul 2023

\bibitem{Wong:2019kwg}
K.~C.~Wong, S.~H.~Suyu, G.~C.~F.~Chen, C.~E.~Rusu, M.~Millon, D.~Sluse, V.~Bonvin, C.~D.~Fassnacht, S.~Taubenberger and M.~W.~Auger, \textit{et al.}
``H0LiCOW \textendash{} XIII. A 2.4 per cent measurement of H0 from lensed quasars: 5.3\ensuremath{\sigma} tension between early- and late-Universe probes,''
Mon. Not. Roy. Astron. Soc. \textbf{498} (2020) no.1, 1420-1439
%doi:10.1093/mnras/stz3094
%[arXiv:1907.04869 [astro-ph.CO]].
%804 citations counted in INSPIRE as of 18 May 2023

\bibitem{Millon:2019slk}
M.~Millon, A.~Galan, F.~Courbin, T.~Treu, S.~H.~Suyu, X.~Ding, S.~Birrer, G.~C.~F.~Chen, A.~J.~Shajib and D.~Sluse, \textit{et al.}
``TDCOSMO. I. An exploration of systematic uncertainties in the inference of $H_0$ from time-delay cosmography,''
Astron. Astrophys. \textbf{639} (2020), A101
%doi:10.1051/0004-6361/201937351
%[arXiv:1912.08027 [astro-ph.CO]].
%114 citations counted in INSPIRE as of 18 May 2023

\bibitem{Sluse:2003iy}
D.~Sluse, J.~Surdej, J.~F.~Claeskens, D.~Hutsemekers, C.~Jean, F.~Courbin, T.~Nakos, M.~Billeres and S.~V.~Khmil,
``A Quadruply imaged quasar with an optical Einstein ring candidate: 1RXS J113155.4-123155,''
Astron. Astrophys. \textbf{406} (2003), L43-L46
%doi:10.1051/0004-6361:20030904
%[arXiv:astro-ph/0307345 [astro-ph]].
%83 citations counted in INSPIRE as of 14 Jul 2023

\bibitem{Shajib:2023uig}
A.~J.~Shajib, P.~Mozumdar, G.~C.~F.~Chen, T.~Treu, M.~Cappellari, S.~Knabel, S.~H.~Suyu, V.~N.~Bennert, J.~A.~Frieman and D.~Sluse, \textit{et al.}
``TDCOSMO. XIII. Improved Hubble constant measurement from lensing time delays using spatially resolved stellar kinematics of the lens galaxy,''
Astron. Astrophys. \textbf{673} (2023), A9
%doi:10.1051/0004-6361/202345878
%[arXiv:2301.02656 [astro-ph.CO]].
%3 citations counted in INSPIRE as of 18 May 2023

\bibitem{Dainotti:2021pqg}
M.~G.~Dainotti, B.~De Simone, T.~Schiavone, G.~Montani, E.~Rinaldi and G.~Lambiase,
``On the Hubble constant tension in the SNe Ia Pantheon sample,''
Astrophys. J. \textbf{912}, 150  (2021).
%doi:10.3847/1538-4357/abeb73


\bibitem{Colgain:2022nlb}
E.~\'O~Colg\'ain, M.~M.~Sheikh-Jabbari, R.~Solomon, G.~Bargiacchi, S.~Capozziello, M.~G.~Dainotti and D.~Stojkovic,
``Revealing intrinsic flat \ensuremath{\Lambda}CDM biases with standardizable candles,''
Phys. Rev. D \textbf{106}, L041301  (2022).
%doi:10.1103/PhysRevD.106.L041301

\bibitem{Colgain:2022rxy}
E.~\'O~Colg\'ain, M.~M.~Sheikh-Jabbari, R.~Solomon, M.~G.~Dainotti and D.~Stojkovic,
``Putting Flat $\Lambda$CDM In The (Redshift) Bin,''
[arXiv:2206.11447 [astro-ph.CO]].
%42 citations counted in INSPIRE as of 14 Jul 2023

%\cite{Colgain:2022tql}
%\bibitem{Colgain:2022tql}
%E.~\'O.~Colg\'ain, M.~M.~Sheikh-Jabbari and R.~Solomon,
%``High redshift \ensuremath{\Lambda}CDM cosmology: To bin or not to bin?,''
%Phys. Dark Univ. \textbf{40} (2023), 101216
%doi:10.1016/j.dark.2023.101216
%[arXiv:2211.02129 [astro-ph.CO]].
%10 citations counted in INSPIRE as of 25 Jul 2023


\bibitem{Malekjani:2023dky}
M.~Malekjani, R.~M.~Conville, E.~\'O.~Colg\'ain, S.~Pourojaghi and M.~M.~Sheikh-Jabbari,
``Negative Dark Energy Density from High Redshift Pantheon+ Supernovae,''
[arXiv:2301.12725 [astro-ph.CO]].
%13 citations counted in INSPIRE as of 17 Jul 2023

\bibitem{Hu:2022kes}
J.~P.~Hu and F.~Y.~Wang,
``Revealing the late-time transition of H0: relieve the Hubble crisis,''
Mon. Not. Roy. Astron. Soc. \textbf{517}, 576  (2022).

\bibitem{Jia:2022ycc}
X.~D.~Jia, J.~P.~Hu and F.~Y.~Wang,
``Evidence of a decreasing trend for the Hubble constant,''
Astron. Astrophys. \textbf{674} (2023), A45
%doi:10.1051/0004-6361/202346356
%[arXiv:2212.00238 [astro-ph.CO]].
%10 citations counted in INSPIRE as of 17 Jul 2023

\bibitem{Krishnan:2020obg}
C.~Krishnan, E.~\'O~Colg\'ain, Ruchika, A.~A.~Sen, M.~M.~Sheikh-Jabbari and T.~Yang,
``Is there an early Universe solution to Hubble tension?,''
Phys. Rev. D \textbf{102} (2020) no.10, 103525
%doi:10.1103/PhysRevD.102.103525
%[arXiv:2002.06044 [astro-ph.CO]].
%69 citations counted in INSPIRE as of 17 Jul 2023

\bibitem{Dainotti:2022bzg}
M.~G.~Dainotti, B.~De Simone, T.~Schiavone, G.~Montani, E.~Rinaldi, G.~Lambiase, M.~Bogdan and S.~Ugale,
``On the Evolution of the Hubble Constant with the SNe Ia Pantheon Sample and Baryon Acoustic Oscillations: A Feasibility Study for GRB-Cosmology in 2030,''
Galaxies \textbf{10}, 24  (2022).
%doi:10.3390/galaxies10010024

\bibitem{Risaliti:2015zla}
G.~Risaliti and E.~Lusso,
``A Hubble Diagram for Quasars,''
Astrophys. J. \textbf{815} (2015), 33
%doi:10.1088/0004-637X/815/1/33
%[arXiv:1505.07118 [astro-ph.CO]].
%146 citations counted in INSPIRE as of 16 Jun 2023

\bibitem{Risaliti:2018reu}
G.~Risaliti and E.~Lusso,
``Cosmological constraints from the Hubble diagram of quasars at high redshifts,''
Nature Astron. \textbf{3}, 272  (2019).

\bibitem{Lusso:2020pdb}
E.~Lusso, G.~Risaliti, E.~Nardini, G.~Bargiacchi, M.~Benetti, S.~Bisogni, S.~Capozziello, F.~Civano, L.~Eggleston and M.~Elvis, \textit{et al.}
``Quasars as standard candles III. Validation of a new sample for cosmological studies,''
Astron. Astrophys. \textbf{642}, A150  (2020).


\bibitem{Yang:2019vgk}
T.~Yang, A.~Banerjee and E.~\'O~Colg\'ain,
``Cosmography and flat $\Lambda$CDM tensions at high redshift,''
Phys. Rev. D \textbf{102}, 123532  (2020).

\bibitem{Khadka:2020vlh}
N.~Khadka and B.~Ratra,
``Using quasar X-ray and UV flux measurements to constrain cosmological model parameters,''
Mon. Not. Roy. Astron. Soc. \textbf{497}, 263  (2020).


\bibitem{Khadka:2020tlm}
N.~Khadka and B.~Ratra,
``Determining the range of validity of quasar X-ray and UV flux measurements for constraining cosmological model parameters,''
Mon. Not. Roy. Astron. Soc. \textbf{502}, 6140  (2021).


\bibitem{Khadka:2021xcc}
N.~Khadka and B.~Ratra,
``Do quasar X-ray and UV flux measurements provide a useful test of cosmological models?,''
Mon. Not. Roy. Astron. Soc. \textbf{510}, 2753  (2022).
%doi:10.1093/mnras/stab3678

\bibitem{Pourojaghi:2022zrh}
S.~Pourojaghi, N.~F.~Zabihi and M.~Malekjani,
``Can high-redshift Hubble diagrams rule out the standard model of cosmology in the context of cosmography?,''
Phys. Rev. D \textbf{106}, 123523  (2022).


\bibitem{Zajacek:2023qjm}
M.~Zaja\v{c}ek, B.~Czerny, N.~Khadka, R.~Prince, S.~Panda, M.~L.~Mart\'\i{}nez-Aldama and B.~Ratra,
``Extinction biases quasar luminosity distances determined from quasar UV and X-ray flux measurements,''
[arXiv:2305.08179 [astro-ph.GA]].
%0 citations counted in INSPIRE as of 17 Jul 2023

\bibitem{Pasten:2023rpc}
E.~Past\'en and V.~H.~C\'ardenas,
``Testing \ensuremath{\Lambda}CDM cosmology in a binned universe: Anomalies in the deceleration parameter,''
Phys. Dark Univ. \textbf{40} (2023), 101224
%doi:10.1016/j.dark.2023.101224
%[arXiv:2301.10740 [astro-ph.CO]].

\bibitem{Wagner:2022etu}
J.~Wagner,
``Casting the $H_0$ tension as a fitting problem of cosmologies,''
[arXiv:2203.11219 [astro-ph.CO]].
%5 citations counted in INSPIRE as of 28 Jul 2023

\bibitem{Sakr:2023hrl}
Z.~Sakr,
``One matter density discrepancy to alleviate them all or further trouble for $\Lambda$CDM model,''
[arXiv:2305.02846 [astro-ph.CO]].
%0 citations counted in INSPIRE as of 24 Jul 2023


\bibitem{Colgain:2022tql}
E.~\'O~Colg\'ain, M.~M.~Sheikh-Jabbari and R.~Solomon,
``High redshift \ensuremath{\Lambda}CDM cosmology: To bin or not to bin?,''
Phys. Dark Univ. \textbf{40} (2023), 101216
%doi:10.1016/j.dark.2023.101216
[arXiv:2211.02129 [astro-ph.CO]].
%10 citations counted in INSPIRE as of 28 Jun 2023

\bibitem{Esposito:2022plo}
M.~Esposito, V.~Ir\v{s}i\v{c}, M.~Costanzi, S.~Borgani, A.~Saro and M.~Viel,
``Weighing cosmic structures with clusters of galaxies and the intergalactic medium,''
Mon. Not. Roy. Astron. Soc. \textbf{515}, 857  (2022).
%doi:10.1093/mnras/stac1825
[arXiv:2202.00974 [astro-ph.CO]].

\bibitem{Adil:2023jtu}
S.~A.~Adil, \"O.~Akarsu, M.~Malekjani, E.~\'O~Colg\'ain, S.~Pourojaghi, A.~A.~Sen and M.~M.~Sheikh-Jabbari,
``$S_8$ increases with effective redshift in $\Lambda$CDM cosmology,''
[arXiv:2303.06928 [astro-ph.CO]].
%1 citations counted in INSPIRE as of 14 Jul 2023

\bibitem{ACT:2023dou}
F.~J.~Qu \textit{et al.} [ACT],
``The Atacama Cosmology Telescope: A Measurement of the DR6 CMB Lensing Power Spectrum and its Implications for Structure Growth,''
[arXiv:2304.05202 [astro-ph.CO]].
%10 citations counted in INSPIRE as of 14 Jul 2023

\bibitem{ACT:2023kun}
M.~S.~Madhavacheril \textit{et al.} [ACT],
``The Atacama Cosmology Telescope: DR6 Gravitational Lensing Map and Cosmological Parameters,''
[arXiv:2304.05203 [astro-ph.CO]].
%10 citations counted in INSPIRE as of 14 Jul 2023

\bibitem{ACT:2023ipp}
G.~A.~Marques \textit{et al.} [ACT and DES],
``Cosmological constraints from the tomography of DES-Y3 galaxies with CMB lensing from ACT DR4,''
[arXiv:2306.17268 [astro-ph.CO]].
%0 citations counted in INSPIRE as of 14 Jul 2023

\bibitem{Miyatake:2021qjr}
H.~Miyatake, Y.~Harikane, M.~Ouchi, Y.~Ono, N.~Yamamoto, A.~J.~Nishizawa, N.~Bahcall, S.~Miyazaki and A.~A.~Plazas Malag\'on,
``First Identification of a CMB Lensing Signal Produced by 1.5~Million Galaxies at z\ensuremath{\sim}4: Constraints on Matter Density Fluctuations at High Redshift,''
Phys. Rev. Lett. \textbf{129} (2022) no.6, 061301
%doi:10.1103/PhysRevLett.129.061301
[arXiv:2103.15862 [astro-ph.CO]].
%7 citations counted in INSPIRE as of 25 Jul 2023

\bibitem{Alonso:2023guh}
D.~Alonso, G.~Fabbian, K.~Storey-Fisher, A.~C.~Eilers, C.~Garc\'\i{}a-Garc\'\i{}a, D.~W.~Hogg and H.~W.~Rix,
``Constraining cosmology with the Gaia-unWISE Quasar Catalog and CMB lensing: structure growth,''
[arXiv:2306.17748 [astro-ph.CO]].
%0 citations counted in INSPIRE as of 25 Jul 2023


\bibitem{Herold:2021ksg}
L.~Herold, E.~G.~M.~Ferreira and E.~Komatsu,
``New Constraint on Early Dark Energy from Planck and BOSS Data Using the Profile Likelihood,''
Astrophys. J. Lett. \textbf{929} (2022) no.1, L16
%doi:10.3847/2041-8213/ac63a3
%[arXiv:2112.12140 [astro-ph.CO]].
%43 citations counted in INSPIRE as of 17 Jul 2023

\bibitem{Gomez-Valent:2022hkb}
A.~G\'omez-Valent,
``Fast test to assess the impact of marginalization in Monte~Carlo analyses and its application to cosmology,''
Phys. Rev. D \textbf{106} (2022) no.6, 063506
%doi:10.1103/PhysRevD.106.063506
%[arXiv:2203.16285 [astro-ph.CO]].
%20 citations counted in INSPIRE as of 11 Jul 2023

\bibitem{Meiers:2023gft}
M.~Meiers, L.~Knox and N.~Sch\"oneberg,
``Exploration of the Pre-recombination Universe with a High-Dimensional Model of an Additional Dark Fluid,''
[arXiv:2307.09522 [astro-ph.CO]].
%0 citations counted in INSPIRE as of 22 Jul 2023

\bibitem{Poulin:2018cxd}
V.~Poulin, T.~L.~Smith, T.~Karwal and M.~Kamionkowski,
``Early Dark Energy Can Resolve The Hubble Tension,''
Phys. Rev. Lett. \textbf{122} (2019) no.22, 221301
%doi:10.1103/PhysRevLett.122.221301
%[arXiv:1811.04083 [astro-ph.CO]].
%608 citations counted in INSPIRE as of 17 Jul 2023

\bibitem{Niedermann:2019olb}
F.~Niedermann and M.~S.~Sloth,
``New early dark energy,''
Phys. Rev. D \textbf{103} (2021) no.4, L041303
%doi:10.1103/PhysRevD.103.L041303
[arXiv:1910.10739 [astro-ph.CO]].
%140 citations counted in INSPIRE as of 24 Jul 2023

\bibitem{Jimenez:2001gg}
R.~Jimenez and A.~Loeb,
``Constraining cosmological parameters based on relative galaxy ages,''
Astrophys. J. \textbf{573} (2002), 37-42
%doi:10.1086/340549
%[arXiv:astro-ph/0106145 [astro-ph]].
%598 citations counted in INSPIRE as of 28 Jun 2023

\bibitem{Stern:2009ep}
D.~Stern, R.~Jimenez, L.~Verde, M.~Kamionkowski and S.~A.~Stanford,
``Cosmic Chronometers: Constraining the Equation of State of Dark Energy. I: H(z) Measurements,''
JCAP \textbf{02} (2010), 008
%doi:10.1088/1475-7516/2010/02/008
%[arXiv:0907.3149 [astro-ph.CO]].
%740 citations counted in INSPIRE as of 20 May 2022

\bibitem{Moresco:2012jh}
M.~Moresco, A.~Cimatti, R.~Jimenez, L.~Pozzetti, G.~Zamorani, M.~Bolzonella, J.~Dunlop, F.~Lamareille, M.~Mignoli and H.~Pearce, \textit{et al.}
``Improved constraints on the expansion rate of the Universe up to z\textasciitilde{}1.1 from the spectroscopic evolution of cosmic chronometers,''
JCAP \textbf{08} (2012), 006
%doi:10.1088/1475-7516/2012/08/006
%[arXiv:1201.3609 [astro-ph.CO]].
%508 citations counted in INSPIRE as of 20 May 2022

\bibitem{Zhang:2012mp}
C.~Zhang, H.~Zhang, S.~Yuan, T.~J.~Zhang and Y.~C.~Sun,
``Four new observational $H(z)$ data from luminous red galaxies in the Sloan Digital Sky Survey data release seven,''
Res. Astron. Astrophys. \textbf{14} (2014) no.10, 1221-1233
%doi:10.1088/1674-4527/14/10/002
%[arXiv:1207.4541 [astro-ph.CO]].
%425 citations counted in INSPIRE as of 20 May 2022

\bibitem{Moresco:2016mzx}
M.~Moresco, L.~Pozzetti, A.~Cimatti, R.~Jimenez, C.~Maraston, L.~Verde, D.~Thomas, A.~Citro, R.~Tojeiro and D.~Wilkinson,
``A 6\% measurement of the Hubble parameter at $z\sim0.45$: direct evidence of the epoch of cosmic re-acceleration,''
JCAP \textbf{05} (2016), 014
%doi:10.1088/1475-7516/2016/05/014
%[arXiv:1601.01701 [astro-ph.CO]].
%505 citations counted in INSPIRE as of 17 May 2022

\bibitem{Ratsimbazafy:2017vga}
A.~L.~Ratsimbazafy, S.~I.~Loubser, S.~M.~Crawford, C.~M.~Cress, B.~A.~Bassett, R.~C.~Nichol and P.~V\"ais\"anen,
``Age-dating Luminous Red Galaxies observed with the Southern African Large Telescope,''
Mon. Not. Roy. Astron. Soc. \textbf{467} (2017) no.3, 3239-3254
%doi:10.1093/mnras/stx301
%[arXiv:1702.00418 [astro-ph.CO]].
%162 citations counted in INSPIRE as of 17 May 2022

\bibitem{Borghi:2021rft}
N.~Borghi, M.~Moresco and A.~Cimatti,
``Toward a Better Understanding of Cosmic Chronometers: A New Measurement of H(z) at z \ensuremath{\sim} 0.7,''
Astrophys. J. Lett. \textbf{928} (2022) no.1, L4
%doi:10.3847/2041-8213/ac3fb2
%[arXiv:2110.04304 [astro-ph.CO]].
%10 citations counted in INSPIRE as of 17 May 2022

\bibitem{Jiao:2022aep}
K.~Jiao, N.~Borghi, M.~Moresco and T.~J.~Zhang,
``New Observational H(z) Data from Full-spectrum Fitting of Cosmic Chronometers in the LEGA-C Survey,''
Astrophys. J. Suppl. \textbf{265} (2023) no.2, 48
%doi:10.3847/1538-4365/acbc77
%[arXiv:2205.05701 [astro-ph.CO]].
%14 citations counted in INSPIRE as of 17 Jul 2023

\bibitem{Tomasetti:2023kek}
E.~Tomasetti, M.~Moresco, N.~Borghi, K.~Jiao, A.~Cimatti, L.~Pozzetti, A.~C.~Carnall, R.~J.~McLure and L.~Pentericci,
``A new measurement of the expansion history of the Universe at z=1.26 with cosmic chronometers in VANDELS,''
[arXiv:2305.16387 [astro-ph.CO]].
%1 citations counted in INSPIRE as of 28 Jun 2023

\bibitem{Moresco:2023zys}
M.~Moresco,
``Addressing the Hubble tension with cosmic chronometers,''
[arXiv:2307.09501 [astro-ph.CO]].
%0 citations counted in INSPIRE as of 24 Jul 2023

\bibitem{Moresco:2020fbm}
M.~Moresco, R.~Jimenez, L.~Verde, A.~Cimatti and L.~Pozzetti,
``Setting the Stage for Cosmic Chronometers. II. Impact of Stellar Population Synthesis Models Systematics and Full Covariance Matrix,''
Astrophys. J. \textbf{898} (2020) no.1, 82
%doi:10.3847/1538-4357/ab9eb0
[arXiv:2003.07362 [astro-ph.GA]].
%57 citations counted in INSPIRE as of 28 Jul 2023

\bibitem{Foreman-Mackey:2012any}
D.~Foreman-Mackey, D.~W.~Hogg, D.~Lang and J.~Goodman,
``emcee: The MCMC Hammer,''
Publ. Astron. Soc. Pac. \textbf{125} (2013), 306-312
%doi:10.1086/670067
%[arXiv:1202.3665 [astro-ph.IM]].
%3393 citations counted in INSPIRE as of 17 Jul 2023


\bibitem{Hou:2020rse}
J.~Hou, A.~G.~S\'anchez, A.~J.~Ross, A.~Smith, R.~Neveux, J.~Bautista, E.~Burtin, C.~Zhao, R.~Scoccimarro and K.~S.~Dawson, \textit{et al.}
``The Completed SDSS-IV extended Baryon Oscillation Spectroscopic Survey: BAO and RSD measurements from anisotropic clustering analysis of the Quasar Sample in configuration space between redshift 0.8 and 2.2,''
Mon. Not. Roy. Astron. Soc. \textbf{500} (2020) no.1, 1201-1221
%:10.1093/mnras/staa3234
%[arXiv:2007.08998 [astro-ph.CO]].
%135 citations counted in INSPIRE as of 28 Jun 2023

\bibitem{Neveux:2020voa}
R.~Neveux, E.~Burtin, A.~de Mattia, A.~Smith, A.~J.~Ross, J.~Hou, J.~Bautista, J.~Brinkmann, C.~H.~Chuang and K.~S.~Dawson, \textit{et al.}
``The completed SDSS-IV extended Baryon Oscillation Spectroscopic Survey: BAO and RSD measurements from the anisotropic power spectrum of the quasar sample between redshift 0.8 and 2.2,''
Mon. Not. Roy. Astron. Soc. \textbf{499} (2020) no.1, 210-229
%doi:10.1093/mnras/staa2780
%[arXiv:2007.08999 [astro-ph.CO]].
%133 citations counted in INSPIRE as of 28 Jun 2023

\bibitem{duMasdesBourboux:2020pck}
H.~du Mas des Bourboux, J.~Rich, A.~Font-Ribera, V.~de Sainte Agathe, J.~Farr, T.~Etourneau, J.~M.~Le Goff, A.~Cuceu, C.~Balland and J.~E.~Bautista, \textit{et al.}
``The Completed SDSS-IV Extended Baryon Oscillation Spectroscopic Survey: Baryon Acoustic Oscillations with Ly\ensuremath{\alpha} Forests,''
Astrophys. J. \textbf{901} (2020) no.2, 153
%doi:10.3847/1538-4357/abb085
%[arXiv:2007.08995 [astro-ph.CO]].
%172 citations counted in INSPIRE as of 28 Jun 2023

\bibitem{Trotta:2017wnx}
R.~Trotta,
``Bayesian Methods in Cosmology,''
[arXiv:1701.01467 [astro-ph.CO]].
%96 citations counted in INSPIRE as of 18 Jul 2023

\bibitem{Moresco:2022phi}
M.~Moresco, L.~Amati, L.~Amendola, S.~Birrer, J.~P.~Blakeslee, M.~Cantiello, A.~Cimatti, J.~Darling, M.~Della Valle and M.~Fishbach, \textit{et al.}
``Unveiling the Universe with emerging cosmological probes,''
Living Rev. Rel. \textbf{25} (2022) no.1, 6
%doi:10.1007/s41114-022-00040-z
%[arXiv:2201.07241 [astro-ph.CO]].
%71 citations counted in INSPIRE as of 16 Jun 2023

\bibitem{DESI:2023ytc}
G.~Adame \textit{et al.} [DESI],
``The Early Data Release of the Dark Energy Spectroscopic Instrument,''
%doi:10.5281/zenodo.7964161
[arXiv:2306.06308 [astro-ph.CO]].
%13 citations counted in INSPIRE as of 26 Jul 2023

\bibitem{Akarsu:2022lhx}
O.~Akarsu, E.~\'O~Colg\'ain, E.~\"Ozulker, S.~Thakur and L.~Yin,
``Inevitable manifestation of wiggles in the expansion of the late Universe,''
Phys. Rev. D \textbf{107} (2023) no.12, 123526
%doi:10.1103/PhysRevD.107.123526
%[arXiv:2207.10609 [astro-ph.CO]].
%6 citations counted in INSPIRE as of 17 Jul 2023

\bibitem{Zhao:2017cud}
G.~B.~Zhao, M.~Raveri, L.~Pogosian, Y.~Wang, R.~G.~Crittenden, W.~J.~Handley, W.~J.~Percival, F.~Beutler, J.~Brinkmann and C.~H.~Chuang, \textit{et al.}
``Dynamical dark energy in light of the latest observations,''
Nature Astron. \textbf{1} (2017) no.9, 627-632
%doi:10.1038/s41550-017-0216-z
%[arXiv:1701.08165 [astro-ph.CO]].
%356 citations counted in INSPIRE as of 17 Jul 2023

\bibitem{Wang:2018fng}
Y.~Wang, L.~Pogosian, G.~B.~Zhao and A.~Zucca,
``Evolution of dark energy reconstructed from the latest observations,''
Astrophys. J. Lett. \textbf{869} (2018), L8
%doi:10.3847/2041-8213/aaf238
%[arXiv:1807.03772 [astro-ph.CO]].
%92 citations counted in INSPIRE as of 17 Jul 2023

\bibitem{Escamilla:2021uoj}
L.~A.~Escamilla and J.~A.~Vazquez,
``Model selection applied to reconstructions of the Dark Energy,''
Eur. Phys. J. C \textbf{83} (2023) no.3, 251
%doi:10.1140/epjc/s10052-023-11404-2
%[arXiv:2111.10457 [astro-ph.CO]].
%13 citations counted in INSPIRE as of 17 Jul 2023

\end{thebibliography}
\end{document}

\end{document} 
