\documentclass[10pt]{article}
%\usepackage[T1]{fontenc}
%\usepackage{latexsym}
%\usepackage{amssymb}
%\usepackage{jcappub}
%\usepackage{cite}
\usepackage{amsmath}
\usepackage{lmodern}
\usepackage{amssymb}
\usepackage{accents}
\usepackage{latexsym}
\usepackage{multirow}
\usepackage{color,graphicx}
\usepackage{dcolumn}% Align table columns on decimal point
\usepackage{bm}% bold math
\usepackage{mathrsfs}
%\usepackage[utf8]{inputenc}
\usepackage[T1]{fontenc}
%\bibliographystyle{unsrt}
% \def\beq{\begin{equation}}
% \def\eeq{\end{equation}}
% \def\br{\begin{eqnarray}}
% \def\er{\end{eqnarray}}
\newcommand*{\red}{\textcolor{red}}%new macro for text colour
\newcommand*{\blue}{\textcolor{blue}}
\newcommand*{\magenta}{\textcolor{magenta}}

\def\dual#1{\accentset{\boldsymbol{\neg}\vspace{-0.2ex}}{#1}}

\def\bi{\begin{itemize}}
\def\ei{\end{itemize}}




\newsavebox{\blambox}\savebox{\blambox}[.6em]{$\lambda \!\!\!
$\raisebox{.5em}{$\neg$}}\newcommand{\blambda}{\usebox{\blambox}}

%%%%%%%%%%%%%%%%%%%%%
\def\MPlh{M_{_{\rm Pl}}}

\def\x{{\mathbf x}}
\def\im{{\mathrm i}}
\def\d{{\mathrm d}}
\def\lan{\langle}
\def\ran{\rangle}
%%%%%%%%%%%%%%%%%%%%%%


%%%%%%% NEW  %%%%%%%%%%%%%%
\tolerance 4000           %
\textwidth 17.50cm        %
\topmargin -2.10cm        %
\oddsidemargin -0.1cm    %
\evensidemargin -0.1cm   %
\textheight 23.0cm        %
\parindent 22pt           %
%%%%%%%%%%%%%%%%%%%%%%%%%%%
\begin{document}
%\tighten
\reversemarginpar

\centerline{\bf REPLY TO THE REFEREE'S COMMENTS}
\vskip 8pt
\hrule\hrule
\vskip 8pt\noindent
{\bf Title:}~ 
Real-space quantum-to-classical transition of time dependent background fluctuations
\vskip 2pt\noindent
{\bf Authors:}~S. Mahesh Chandran, Karthik Rajeev and S. Shankaranarayanan
\vskip 2pt\noindent
%{\bf Manuscript number:}~
\vskip 8pt
%\hrule\hrule
\vskip 8pt
\noindent
We thank the referee for their valuable comments on our work. Unless otherwise specified, Page numbers, Equation numbers, and references in the report correspond to the updated manuscript numbers. \blue{The changes made in the manuscript are highlighted in blue.} New Sections and appendices are in black text.
\vskip 8pt

\hrule\hrule
\vspace*{0.4cm}


\centerline{\bf I. Reply to the comments by the referee}
\vspace*{0.5cm}
\noindent  \textit{Referee's Comment} : The authors describe the transit process of the quantum system into
the classical regime. They characterize the classicality measure by
calculating the entanglement entropy and the log classicality. They
analyze a partially coarse-grained, coupled harmonic oscillator model
in tanh-expansion and De-Sitter-expansion in dimensions 1+1 and 1+3.
They conclude that both the classical measures show ever-increasing
classicality for the 1+3-dimensional De-Sitter expansion, and a single
measure or no measure shows classicality in other models and other
dimensions.

The topics and calculations are relevant and interesting. However, the
preparation part of this paper, before sections IV and V, is redundant
and boring. The present referee recommends revising this paper along
with the following comments.
\vspace*{0.3cm}

\noindent  \textbf{Our response} : We appreciate the referee's summarization of the background and aim of our paper. Below we have provided detailed comments to the referee's insightful comments and have made necessary changes in the updated manuscript. We are confident that we have addressed all the questions and comments raised by the referee.
%We thank the referee again and address below their comments. 

\vspace*{0.5cm}

\noindent  \textit{Referee's Comment} : 1. Sections IV and V EARLY UNIVERSE FLUCTUATIONS may be the highlight
of this paper. But the 24 pages-long redundant introductions before
these may distract the readers. Is it possible to sharpen the
argument?

\vspace*{0.3cm}

\noindent  \textbf{Our response} : We thank the referee for the comments. Upon revisiting the manuscript in view of the referee's comments, we realized that Sections IIB and III indeed digressed from the highlights of the article. In order to sharpen the argument, we have moved Section II B (Phase-Space stability analysis of CHO) and Section III (Classicality criteria and canonical transformations) of the earlier version of the manuscript to sections in the appendix. We briefly summarize the results from these portions in Section II, and then proceed directly to the main results. We have also simplified the derivation of log-classicality from the covariance matrix in Section IIC, which helped us further trim the text by removing Appendix C of the earlier version of the manuscript. The sections studying early-universe fluctuations now begin on page 15 instead of page 22.\\

\noindent  \textbf{Changes made}:  We have moved Section IIB of the earlier version of the manuscript to Appendix C (Page 33), and Section III of the earlier version of the manuscript to Appendix E (Page 36). We have added a summary of these results in Section II. See changes on Page 9 and Pages 12 - 15. We have also removed Appendix C of the earlier version of the manuscript.

\vspace*{0.5cm}

\noindent  \textit{Referee's Comment} : 2. Tanh expansion asymptotically approaches the flat region and yields
no Log classicality LC(t). On the other hand, fields in the same tanh expansion generally cause particle creation because of the vacuum change. What is the difference between the classicalize process and the particle creation? The particle creation is probably related to the squeezing of the state or LC(t). Then, why does squeezing disappear in the tanh expansion?


\vspace*{0.3cm}

\noindent  \textbf{Our response} : We thank the referee for the comments. The referee has made a valid point regarding log-classicality behaviour due to Tanh-expansion in $(3+1)$-dimensions. While rechecking the codes, we found that we had inadvertently defined the normal modes for Tanh-expansion in $(3+1)$-dimensions (Eq. 74, Page 24) using the expression for Tanh-expansion in $(1+1)-$dimensions (Eq. 48, Page 17) instead, thereby missing particle-creation effects at late-times in the earlier version of the manuscript. We have fixed this issue, and as the referee had anticipated, log-classicality indeed showed signs of particle production at late-times in $(3+1)$ for Tanh-expansion (Fig 6, Page 25). In the process, we were also able to optimize our code for a larger number of oscillators and angular momentum modes --- this helped us scale up the parameters for de-Sitter expansion (Fig 7, Page 27), thereby confirming our earlier results with more accuracy. We are therefore grateful to the referee for pointing this out.

In order to understand the connection between log-classicality and particle number ($\mathscr{N}_k$), we studied the evolution of each normal mode ($k$) subject to a late-time instability and established the following behaviour:
\begin{equation}
    LC\sim \begin{cases}
        \frac{1}{2}\log{\mathscr{N}_k} & \quad\text{Zero mode}\\
        \log{\mathscr{N}_k} & \quad \text{Inverted mode}
    \end{cases},
\end{equation}
%
where we see that log-classicality increases logarithmically with particle number at late-times (details in Appendix D). 
This confirms the referee's intuition about the relation between log-classicality and particle number. \\
%We expect a similar relation to hold for spatial subsystems, however we keep these calculations for future work in order to explore them in much more detail.
%The above relation supports referee's concerns, which prompted us to recheck and debug the code. \\

\noindent  \textbf{Changes made}: We have replaced the plots in Fig 6 (Page 25) and Fig 7 (Page 27). We have added a new section Appendix D (Page 35). See changes made in Pages 24 - 25. We have added this point in the abstract. 

\vspace*{0.5cm}

\noindent  \textit{Referee's Comment} : 3. Inflation or de-sitter expansion continues within some finite
period and is immediately followed by the power-law expansion stage in
the Universe. Does this realistic process still yield infinite
squeezing and entropy production?

\vspace*{0.3cm}

\noindent  \textbf{Our response} : We thank the referee for this question. To address this, we have included a preliminary analysis in Section IV-C, modeling both hard and soft transitions from inflation to radiation-dominated (power-law) expansion. We show that the growths in entanglement entropy and log-classicality are cut off as inflation ends, after which they transition to an oscillatory behaviour in the radiation-dominated era. Our analysis therefore implies that there is rapid classicalization during the inflationary phase, after which the spatial subregions end up retaining remnant quantum signatures depending on the e-fold values. We hope to obtain further insights from an exact quantitative analysis of entanglement entropy and log-classicality in future work.\\ 

\noindent  \textbf{Changes made}: We have added a new subsection Section IVC (Pages 26-30) addressing this problem. New plots have been added in Fig 8 (Page 28) and Fig 9 (Page 29). We have also added Eqs 72-74 and surrounding text for more context (Pages 23-24).

\vspace*{0.5cm}

\noindent  \textit{Referee's Comment} : 4. What is the meaning of preparing multiple N-CHOs? Do they couple
with each other? Are the entropy and the squeezing simply proportional to the number N?

\vspace*{0.3cm}

\noindent  \textbf{Our response} : We thank the referee for the questions. Our approach models a quantum scalar field $\varphi$ in $(1+1)-$dimensional background using a set of $N$ harmonic oscillators with only nearest-neighbour coupling. In $(3+1)$-dimensions, spherical symmetry can be exploited to reduce the system to $(2l+1)$ such independent $(1+1)$-dimensional lattices. Each $(l,m)$-mode will therefore have its own harmonic lattice of $N$-oscillators with nearest-neighbour coupling, however, the lattices across different $(l,m)$-modes are not coupled to each other in any way. Therefore the entanglement entropy/log-classicality measures are calculated by simply adding the contributions from each of these $(2l+1)$ lattices separately [Ref 39].

%This follows naturally upon lattice-regularizing the field Hamiltonian, with the UV-cutoff being fixed by lattice-spacing ($\tilde{d}$) and IR cutoff fixed by the total size of the lattice ($N\tilde{d}$). 

Since the dependence of entanglement entropy on the total system size $N$ is captured in the sub-leading (IR) terms that are typically suppressed in the energy scales of interest [Refs 81,92], we do not expect it to play a major role in the quantum-to-classical transition problem in the early-universe. However, it will be interesting to see if it plays a role (due to large wavelength modes that reenter) in the matter to dark energy dominated stages of the expansion, which we would like to address in the future. \\

\noindent  \textbf{Changes made}: We have briefly commented on this in Sections IV and V. See text between Eqs 68 - 69 (Page 23), and text before Eq. 75 (Page 24). See the text on Page 31 as well. 

\vspace*{0.5cm}


\noindent  \textit{Referee's Comment} : To what extent is the Gaussian state, used in this paper, universal for discussing the classicalize process?

\vspace*{0.3cm}

\noindent  \textbf{Our response} : We thank the referee for the question. Since the temperature fluctuations in the CMB are predominantly Gaussian as per current observations, the Gaussian state we have considered here is sufficient for addressing the quantum-to-classical transition in this context. However, it is not currently understood whether a non-Gaussian state would accelerate or slow down classicalization, and is therefore an important problem that needs to be studied more generally. In this case, while the covariance matrix cannot capture the full information about quantum correlations in the system, analysis of the Wigner function may still be universal in addressing phenomenological indicators such as decoherence.\\ %For instance, in non-Gaussian states, the negativity of the  Wigner function implies highly non-classical behaviour, typically indicating decoherence when it drops to zero. \\

\noindent  \textbf{Changes made}: We have added a paragraph on this in Section V (Page 31). \\

We thank the referee for their comments and valuable suggestions, which we have incorporated into the revised manuscript. Based on these, we are confident that the current draft will now be suitable for publication in Phys. Rev. D. \\


\vspace*{0.5cm}
\hrule\hrule
\vspace*{0.4cm}

\centerline{\bf II. List of changes made in the updated manuscript}

\begin{enumerate}
\item Moved Section II-B (earlier version) to Appendix C (Page 33).
\item Moved Section III (earlier version) to Appendix E (Page 36).
\item Modified text in subsection II-A (Page 9, Paragraph 3).
\item Modified text in subsection II-B (Page 12 - 15).
\item Removed Appendix C (earlier version).
\item Added text in Section IV (Pages 23 - 24).
\item Replaced plots in Fig 6 (Page 25) and Fig 7 (Page 26).
\item Added new section Appendix D (Page 35).
\item Added text and Eqs 72-74 in Pages 23-24.
\item Added new subsection IV-C (Page 26-30).
\item Added 2 new paragraphs towards the end of Section V (Page 31).
\item Minor typos corrected throughout.
\item Added References [76, 77, 86, 87, 92].
\end{enumerate}
\vspace*{0.5cm}



\hrule\hrule

\end{document}
