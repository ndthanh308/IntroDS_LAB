
\section{Benchmark Evaluation}

We begin our evaluation with two benchmark problems treating the transient impact of linearly elastic bars in one dimension. Importantly, both problems are equipped with analytic solutions. This allows us to compare results with prior methods analyzed by Doyen and colleagues~\cite{doyen2011time} and to also demonstrate convergence of our contact model to known elastodynamic impact solutions. 

Following Doyen et al.~\cite{doyen2011time}, both problems resolve the dynamics of a one-dimen\-sional linearly elastic bar of length $L=\SI{10}{\meter}$, Young's modulus $E=\SI{900}{\newton}$, and density $\rho=\SI{1}{\kilo\gram/\meter}$, initialized (undeformed) at a height of $h_0=\SI{5}{\meter}$ above a rigid ground. Each bar is spatially discretized with a uniform mesh size of $\Delta x$ via linear finite elements. With this common framework there are then two benchmark problems. 

The first, an \emph{impact} problem, resolves a single impact of an elastic bar by initializing the bar's velocity to $v_0 = \SI{-10}{\meter/\second}$ and eliminating gravitational acceleration ($g_0 = 0$). This benchmark has been widely applied in prior analyses and enables comparison of the numerical oscillation artifacts generated by differing contact models~\cite{doyen2011time}.

The second, a \emph{bouncing} problem, resolves a periodic sequence of elastic bar impacts and free-flights. To do so, with the above chosen material parameters, we initialize the bar at rest ($v_0 =0$) with a gravitational acceleration of $g_0 = \SI{-10}{\meter/\second^2}$. This obtains a periodic trajectory of alternating contacts and free-flight for the bar described by an analytic solution. This benchmark, introduced by Doyen et al.~\cite{doyen2011time}, further enables us to analyze the energy evolution and longer-term trajectories generated by contact models over sequential impacts. 

Below we cover the results of our benchmark testing in detail. Here, we first quickly summarize our key takeaways. In brief we note that the following tests demonstrate that \EIPC \emph{qualitatively} follows the the displacement and contact pressure behavior of a penalty-based contact model. However, unlike penalty-based methods, we confirm that \EIPC additionally provides interpenetration-free trajectories independent of choice of discretization and contact-stiffness parameters. We then show convergence of our model under refinement to the benchmarks' analytical solutions. To our knowledge, these are the first results to demonstrate this convergence. 


\subsection{Comparison with Doyen et al.'s benchmark}

\label{sec:1d}
\label{sec:1d_impact}
\label{sec:1d_bounces}



For a direct, side-by-side comparison with Doyen et al.'s evaluation we begin by applying implicit Newmark ($\beta=\tfrac{1}{4}$, $\gamma=\tfrac{1}{2}$) time integration, spatial discretization with $\Delta x = \SI{0.1}{\meter}$, and calculate time steps $h = \nu_C \> \Delta x / \sqrt{E/\rho}$ with a Courant number of $\nu_C = 1.5$. We comparably set our contact model with $\hat{d}=\Delta x$ and $\kappa = 0.1 E$, and so effectively treat the contact barrier as an additional hyperelastic potential. 

We summarize simulation results for the \emph{impact} and \emph{bouncing} problems in \cref{fig:1d_impact,fig:1d_bounces} respectively. Here we observe that \EIPC generates displacement (bottom node), contact pressure, and energy trajectories with closely comparable profiles, and so qualitatively similar error behaviors, to penalty methods in both magnitude and over time. 

However, in these benchmarks we also see key differences between penalty methods and \EIPC. As demonstrated in the top-left plot of \cref{fig:1d_impact} we observe \EIPC preserves a penetration-free (more generally interpenetration-free) trajectory independent of choice of contact stiffness. This is in contrast to penalty methods where penetration errors are uncontrollable for a fixed contact stiffness. Here \EIPC's displacement error is controllable with a curve remaining above the analytic solution during contact by no more than $\hat{d}$. Following this observation, we next evaluate \EIPC's behavior over variations in the contact model's threshold and stiffness, as well as for alternate choices of numerical time integration.



% Figure environment removed


\subsection{Varying barrier stiffness, threshold, and time-integration}

While there is no need to change \EIPC's contact model parameters in order to avoid interpenetration, reducing the threshold parameter $\hat{d}$ (and/or the contact stiffness $\kappa$) will improve the complementarity accuracy (decreased gap at contact) in simulation results. At the same time varying these parameters has direct implications for the contact-pressure oscillations produced, and so on the stability of the solutions obtained. In turn it is then also important to consider choice of the time integration method applied. 

In \cref{fig:1d_impact_diffStiff}, top and middle, we see that varying \EIPC's contact stiffness by $0.1\times$ or $10\times$ introduces significantly smaller variations in contact pressure when compared to the large jumps obtained by varying contact stiffness with the  penalty method (as observed in \citetdoyen{}). However, varying $\hat{d}$ in the \EIPC model similarly reproduces comparably large contact pressure oscillations to varying stiffness in the penalty model. This is because the \EIPC barrier has local support in the distance range of $(0, \hat{d})$, while the sharpness of this potential is more sensitive to changes in $\hat{d}$ than $\kappa$.


% Figure environment removed

This increase in generated contact pressure oscillation, as we decrease $\hat{d}$, implies an important tradeoff. We obtain improved gap accuracy (given by smaller $\hat{d}$) at the cost of a sharper contact potential. In turn, generated pressure oscillations are artifacts from time-integration with these increasingly sharp potentials. Simulations are then significantly improved if we step away from employing marginally stable time integrators like implicit Newmark with $\beta=\tfrac{1}{4}$, $\gamma=\tfrac{1}{2}$. For example, switching to A-stable integrators like BDF-2 and implicit Euler (IE) provides smooth contact pressures with reduced $\hat{d}=0.1\Delta x$ (see \cref{fig:1d_impact_ie_bdf2}) \emph{without} decreasing timestep size. If we then additionally lower the time step size of the IE solution by $0.1\times$ for less numerical dissipation, we see the contact pressure profile (\cref{fig:1d_impact_ie_bdf2} middle) then closely follows many discretizations proposed for improved stability (i.e., see methods 4.1, 4.3, 4.6, 4.7, 6.2, 7.1 in~\cite{doyen2011time}). Similarly, in terms of accuracy, we see BDF-2 generates a solution more than $2\times$ closer to the analytical solution than IE, with significantly less numerical dissipation of the total system's energy.

% Figure environment removed

% Figure environment removed

\subsection{Refinement Analysis} 

As covered above, reduction in $h$ smooths the barrier for decreasing $\hat{d}$, and so improves stability. This is unsurprising as the contact pressure oscillations we observe are artifacts generated by refining the spatial discretization $\hat{d}$ without accompanying temporal refinement. Correspondingly, to improve accuracy, both of \EIPC's spatial parameters, $\Delta x$ and $\hat{d}$, must jointly be refined with $h$. Here we next analyze convergence under refinement for both the \emph{impact} problem and \emph{bouncing} problems, refining by successively halving $\Delta x$, setting the relationship to the threshold as $\hat{d} =  c_B \Delta x$ (with parameter $c_B$) and time step (as above) with, $h = \nu_C\tfrac{\Delta x}{\sqrt{E/\rho}}$.

\paragraph{Impact Problem}

In \cref{fig:1d_impact_conv}, we consider implicit Newmark time integration for the impact problem with $c_B = 4$, and observe that both displacement and contact pressure converge to the analytic solution. Displacement converges linearly while contact pressure converges sublinearly (rate of $\sim 0.5$). Both rates follow reasonable expectations with contact gap error decreasing linearly w.r.t. $\hat{d}$ and contact pressure given by the barrier energy derivative.
Next we consider BDF-2 time integration with $c_B = 4$. To support BDF-2 (wider time stencil) we provide consistent initialization of displacement and velocity history with the analytic solution at time $t=-h$. In \cref{fig:1d_impact_conv_BDF2}, we see BDF-2 provides comparable convergence to IE for both the displacement and contact pressure. We also note that if we decrease our stability criterion for $\hat{d}$ to $c_B = 1$, convergence rates significantly degrade for both Newmark and BDF-2 due to lack of smoothness in the barrier.

\paragraph{Bouncing Problem} 

Next in \cref{fig:1d_bounces_conv}, to look at longer, time-varying behavior with multiple impacts we consider implicit Newmark time integration for the bouncing problem with $c_B = 8$. Here we observe both displacement (bottom node) and total energy converge linearly to analytical solution. To achieve comparable (linear) convergence with BDF-2 (\cref{fig:1d_bounces_conv_BDF2}) time-integration in this problem requires setting $\nu_C = 0.75$ so that respective time step sizes are halved suggesting that Newmark's improved energy conservation helps in capturing the longer-term behavior of repeated elastic bouncing. 

 \paragraph{Comparison to Constraint-based IPC}
 
 In contrast to constraint-based contact model of the original IPC~\cite{Li2020IPC} formulation, \EIPC provides a consistent discretization of the contact potential in the smooth setting.
Here we consider the resulting, improved convergence behavior for {\EIPC} by considering the original IPC's behavior on the \emph{impact} problem benchmark (same settings as \EIPC above). We begin with \citetIPC{} original model which augments the unconstrained incremental potential with an uncalibrated barrier energy. Instead the barrier stiffness is adaptively and automatically updated to gain improved numerical conditioning of the Hessian. For this original formulation we observe no convergence for both displacement (order=$0.2334$) and contact pressure (order=$0.0823$) in the impact problem. Alternately, if we update the original IPC model to keep the contact barrier stiffness fixed (stiffness selected to match \EIPC simulation at coarsest resolution), convergence improves (displacement order=$0.6997$, contact pressure order=$0.2745$) but is still far from satisfactory.



% Figure environment removed

% Figure environment removed



% Figure environment removed

% Figure environment removed

\section{Evaluation in 2D and 3D}

In two and three-dimensions, frictional contact now becomes possible and we must consider the contact-interaction of meshed interfaces. Here we first examine the sliding and bouncing behavior of an elastic square on a fixed analytical ground and show that the maximal energy dissipation and displacement and contact pressure curves all converge under refinement just as in our 1D evaluation above. We then demonstrate the accurate capture of stick and slip behaviors under varying friction coefficients by \EIPC with an analytical slope test and show that with \EIPC's consistent smooth approximation to the max operator, the vertical displacement of a square slipping on a fixed meshed ground can converge to a straight line with only spatial refinement of the mesh boundary. We then further consider frictional benchmark tests and close with challenging geometric collision ``stress-tests'', a large-deformation high-speed dynamic collision problem, and an application to the analysis of compressed microstructure testing.




\subsection{Refinement in 2D}

\label{sec:2d_slides}

\paragraph{Block on ground}
We first consider refinement of a slower-speed contact problem in 2D with a $\SI{2}{\meter}$-wide square, initialized to a height immediately ($\hat{d}$) above a fixed analytical ground without friction. We use a nonlinear (neo-Hookean) material with Young's: $E=\SI{4000}{\newton/\meter}$, Poisson: $\nu=0.2$, and density $\rho=\SI{100}{\kilo\gram/\meter^2}$; gravity is $g=\SI{-5}{\meter/\second^2}$, and time step is set by $\nu_C=1.5$. The square is uniformly and symmetrically tessellated with $\Delta x$. Under gravity, this soft square will compress while its bottom interface slides periodically back and forth along the ground.
Fixing the relations $\hat{d} = 0.5\Delta x$ and $h = \nu_C\frac{\Delta x}{\sqrt{E/\rho}}$, we perform  refinements by half down to $\Delta x = 0.0125m$. We measure system energy, (central top node's) vertical displacement, and (at center bottom node) contact pressure over time. Applying BDF-2 time integration, all above measures converge linearly (to finest solution -- no analytic model is available) as resolution increases despite the nonlinear elasticity applied (\cref{fig:2d_slides}).



% Figure environment removed

\paragraph{Impact and bouncing on ground}
We next consider refinement with higher-speed impacts and repeated bouncing in 2D. We extrude the \emph{bouncing} benchmark problem from 1D (\cref{sec:1d_bounces}), setting the stiffness $2\times$ as large, $\nu_C = 0.75$, and apply BDF-2 time integration. This gives a $\SI{10}{\meter}$-wide square initialized $\SI{5}{\meter}$ above a fixed analytical ground (again no friction). The square is uniformly and symmetrically tessellated by $\Delta x$, with $\rho = \SI{1}{\kilo\gram/\meter^2}$, $E = \SI{1800}{\newton/\meter}$, $\nu=0.2$, applying neo-Hookean elasticity (ensuring no element inversion). Gravity is set to $g=\SI{-10}{\meter/\second^2}$ and $\kappa = 0.1E$.
During simulation, over repeated bounces, gravitational energy progressively transfers to elasticity energy as the highest bouncing point decreases and high-frequency elastic waves become more pronounced.
Repeating the same refinement as for the ``block-on-ground'' above, we now observe that all measurements converge with resolution increase (\cref{fig:2d_bounces}). 
However, as the simulation continues,  high-frequency elastic waves magnify, so that the simulation becomes less stable and the curves at changing resolution diverge increasingly from accumulated errors.

% Figure environment removed

\subsection{Tessellation Error: A Sliding Block on Meshed Boundary}

\subsubsection{2D}

To compare and verify the direct summation approximation~\cite{Li2020IPC} and our consistent approximation to the max operator, we test an example with a $\SI{2}{\meter}$-wide square sliding on a fixed $\SI{16}{\meter}$-wide meshed ground ($\mu=0$).
The square is placed right $\hat{d}=\SI{0.1}{\meter}$ above the ground in the middle with an initial velocity $v_0 = \SI{1}{\meter/\second}$. It has Young's modulus $E = \SI{2e11}{\newton/\meter}$, Poisson's ratio $\nu = 0.3$, and density $\rho = \SI{8000}{\kilo\gram/\meter^2}$, nearly rigid. The gravity is $g=\SI{-5}{\meter/\second^2}$, and we set $\kappa = \SI{1e6}{\newton/\meter}$ and fix the time step size at $h=\SI{0.01}{\second}$. Both the square and the single layer ground are uniformly tessellated with $\Delta x$.

With $\Delta x = \SI{2}{\meter}$, after the initial drop for acquiring contact forces in IPC framework, we clearly see the jumps on the horizontal displacement curve given by direct summation everytime when the square corner is crossing a node on the ground (\cref{fig:2d_slip} top). The arc between the jumps are due to the ground point to square edge contact pair, which hold the square at different location at bottom, forming unbalanced force distributions during slipping. Note that these all only happen within the scale of $\SI{e-3}{\meter}$, nearly 2 orders-of-magnitude smaller than $\hat{d}$.
Our consistent approximation still have jumps but the magnitude is much smaller (\cref{fig:2d_slip} bottom), and the arcs have very similar profile.

% Figure environment removed

This is because with a slipping square, the jumps also come from the activation of the ground point to square edge pairs during slipping in addition to the duplicate square point to ground edge pair when applying direct summation. As we show in \cref{fig:2d_slip} bottom, our approximation with only the first source of jumps generates results converging to a straight line under only $\Delta x$ refinement with $\hat{d}$ and time step size $h$ fixed at $0.1m$ and $0.01s$ respectively. 
But the second source of jumps does not converge as shown in \cref{fig:2d_slip} top with direct summation.
Although by refining $\hat{d}$ at fixed $\Delta x$, the portion of the duplication in direct summation vanishes, once a refinement has a fixed $\hat{d}/\Delta x$ ratio, the portion is then also fixed and thus not vanishing.

\subsubsection{3D} \label{sec:3D-convergence}

In 3D, it becomes more complicated with edge-edge stencils. We extrude the sliding experiment setup in 2D to 3D to check the convergence behavior for a cube sliding on a 3D plane with both point-triangle and edge-edge stencils.

We set $\hat{d} = 10^{-3}m$, orders-of-magnitude smaller than $\Delta x$, so that no extra duplication of the potential field from nearby edges is possible from edge-edge stencils. This enables convergence to a straight line, but not the reference solution obtained by sliding with $\Delta x = 0.25$ and $\hat{d} = 10^{-3}m$ on an analytical plane (\cref{fig:3d_sliding} left). This is because for two meshed planes touching each other, each edge would result in multiple edge-edge stencils, thus over integrating the quantity using our edge weights.
If $\hat{d}$ is also refined starting from $10^{-3}m$ linearly w.r.t. $\Delta x$, our results converge to the analytical solution $y=0$ nearly linearly (\cref{fig:3d_sliding} right).

% Figure environment removed

\subsection{A Frictional Benchmark: Critical Angle on Slope}

To verify the accuracy of our friction model, an experiment with a stiff cube resting or sliding on a fixed analytical slope with a certain friction coefficient is created. 
When a rigid cube is placed on a slope with zero initial velocity, its acceleration has the following analytical form in the slope's tangent space:
\begin{equation}
    \mathbf{a} = \begin{bmatrix} g \min(0, \mu \cos{\theta} - \sin{\theta}) \\ 0 \end{bmatrix}.
\end{equation}
where $\mu$ is the friction coefficient between the cube and the slope, $g$ is the gravity acceleration, $\theta \in [0, \pi/4)$ is the inclined angle of the slope.

The initial configuration of this example is obtained by placing the cube $\hat{d}$ away from the slope, and then simulate under gravity ($g=\SI{5.099}{\meter/\second^2}$) with friction coefficient $\mu=0.5$ for $\SI{1}{\second}$ until the box becomes static. After obtaining the initial configuration, the slope test simulation is performed with different friction coefficients and the dynamic-static friction transition velocities $\epsilon_v$.

Here the cube is $\SI{0.02}{\meter}\times \SI{0.02}{\meter}\times \SI{0.02}{\meter}$, composed of just 8 nodes with density $\rho=\SI{1000}{\kilo\gram/\meter^3}$, Young's modulus $E=\SI{10}{\giga\pascal}$ and Poisson's ratio $\nu=0.4$. Slopes with friction coefficient $0.1$, $0.1999$, and $0.2$ have been tested (\cref{fig:3d_slope} 1st row), all with contact active distance $\hat{d}=10^{-5}m$, contact stiffness $\kappa=\SI{1}{\mega\pascal}$, static friction velocity threshold $\epsilon_v=10^{-5}m/s$, and with the lagged normal forces in friction iteratively updated until converging to a solution with fully-implicit friction. All simulations are using implicit Euler time integration with time step size $h=\SI{0.01}{\second}$, and the Newton tolerance is set to $\epsilon_d=10^{-8}m/s$.

With sliding velocity and acceleration of the cube's center of mass plotted over time (\cref{fig:3d_slope}), they have all been shown to well match analytical solutions with small absolute errors. Even for $\mu=0.1999$ ($99.95\%$ that of the critical coefficient), the sliding behavior can still be accurately captured. The dip is formed as our velocity-time curve started above the analytical one due to the mollification of static-dynamic transition, and then cross and goes below it. For $\mu=0.2$, it is also confirmed that the acceleration vanishes, and the velocity throughout the simulation is around $\epsilon_v$, the static friction velocity threshold in our approximation to provide the static friction force in the same magnitude as dynamic friction.


% Figure environment removed

However, notice that the error of acceleration is always much larger at the first time step than the latter steps after release. This is also an error introduced by our mollification of the dynamic-static friction transition. For $\mu=0.2$, the tangent velocity needs to increase immediately to nearly $\epsilon_v$ to obtain the static friction force for balance, which effectively ends up with a much larger acceleration than the solution ($\SI{0}{\meter/\second^2}$). By only refining $\epsilon_v$, at $\mu=0.2$ our velocity and acceleration errors (including the error in the first time step) both converges nearly linearly to the analytical solution (\cref{fig:3d_slope} 2nd row).



\subsection{Refinement in 3D with Self-Contact and Friction} 


% Figure environment removed

We study the static equilibrium of a stack of three blocks under refinement. We choose material parameters ($E=\SI{2.5e5}{\pascal}$, $\nu=0.4$, and $\rho=\SI{1000}{\kilo\gram/\meter^3}$) which are able to reach stable equilibrium without the stack falling yet produces a visible deformation (see \cref{fig:3D-cube-stack}). Each block is $1\times1\times1~\si{\meter^3}$, and we use five levels of refinement $\Delta x=0.5,0.25,0.125,0.0625,\text{ and } 0.03125~\si{\meter}$. We set $\hat{d}=0.01\Delta x$ and create an initial gap of $0.9\hat{d}$. Additionally, we use friction with a coefficient of $\mu=0.5$ to prevent blocks from sliding off.

To solve for the static equilibrium we perform a series of incremental solves (time-stepping while zeroing out any velocity components at the start of every step). We perform this until convergence of the vertex positions (i.e. $\|x_i - x_{i-1}\|_\infty \leq \epsilon$). We choose a convergence tolerance of $\epsilon = \SI{1e-5}{\meter}$.

\Cref{fig:3D-cube-stack} shows the results of this study where we see convergence of the center of mass, elastic energy, and contact pressure.

\subsection{Dynamic Collision in 3D}

We setup a two spheres colliding experiment to test the ability of our method to resolve elasticity and kinetic energy transfer caused by high-speed collisions.
Two sphere meshes each with 29K nodes, \SI{0.04}{\meter} wide are placed \SI{0.08}{\meter} away from each other (\cref{fig:3d_2spheres_collide_sim} a and b), both with Neo-Hookean elasticity, Young's modulus $E=$\SI{1e7}{\pascal}, Poisson's ratio \SI{0.45}, and density \SI{1150}{\kilo\gram/\meter^3}, exactly the same material with the high-speed golf ball example in \citetIPC{}, except that we do not apply any damping here.
The two spheres are both with \SI{30}{\meter/\second} initial velocity towards each other. We set $\hat{d}$ to \SI{4e-5}{\meter} (0.1\% that of the sphere's diameter) and $\kappa$ to $0.1E$ as usual. For stability and accuracy, we apply BDF-2 time integration at time step size $h=$\SI{2e-5}{\second}.

% Figure environment removed

During the simulation, the total x-direction momentum of the system is perfectly conserved with the the momentum of each sphere symmetrically and smoothly reversed (\cref{fig:3d_2spheres_collide_plot} right).
Since BDF-2 time integration is applied, the energy slightly dissipates around 10\% of the initial total energy (\cref{fig:3d_2spheres_collide_plot} left).
But nice symmetry and coherence on the elasticity and kinetic energy profile of the two spheres are accurately resolved. With the energy data of the left and right spheres plotted as curve and dots respectively, it is clear that the curves are well-aligned.
In \cref{fig:3d_2spheres_collide_sim} c and d, we visualize the Von Mises stress on the spheres at a collision state and right after separation.
Please see our supplemental video for the nice elastic wave propagation captured by our method.

% Figure environment removed

\subsection{3D Unit tests} 

% Figure environment removed

We reproduce the unit tests presented by \citetIPC{} (\cref{fig:3D-unittests}). 

The first unit test tests the ability of our method to handle tight conforming contacts. We drop a unit cube into a C-shaped slot. The slot is only \SI{1e-5}{\meter} wider than the cube. We use a soft material ($E = \SI{1e6}{\pascal}$, $\nu = 0.4$, and $\rho = \SI{1e3}{\meter/\kilo\gram})$) and a $\hat{d}$ of \SI{1e-5}{\meter} with $\kappa=0.1 E$. We also utilize a framerate time step of \SI{0.04}{\second}. Our method passes this test without problem demonstrating the ability to handle small gaping and conforming contact.

Our second test positions two spikes such that they contact at the tips. This degenerate case is often challenging for traditional methods~\cite{erleben2018methodology}. We use the same material parameters as the first unit test and set $\hat{d} = \SI{1e-3}{\meter}$. Again we use a large $\Delta t = \SI{0.025}{\second}$. Our method has no difficulty in handling this contact, resolving the point-point contact into a downward diagonal motion.

\subsection{Application: Microstructures} 

% Figure environment removed

% Figure environment removed

As an illustration of the importance of proper contact handling we apply our method to simulate the compression of a 3D printed gyroid micro-structure (\cref{fig:microstructures}). We perform three simulations: (1) Dirichlet boundary conditions on the gyroid to compress it \emph{without contact resolution}, (2) the same boundary conditions as (1) but now with contact modeled by our formulation, and (3) we apply the boundary conditions to rigid plates, leaving the gyroid's boundaries free (again with contact). For the material parameters we match those of an 3D printed elastomeric polyurethane ($E = \SI{9e6}{\pascal}$, $\nu = 0.48$, and $\rho = \SI{1.1e3}{\meter/\kilo\gram}$). We plot the strain energy versus compression in \cref{fig:microstructures-plot}. 