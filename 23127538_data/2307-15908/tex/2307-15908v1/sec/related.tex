

The robust and accurate modeling of large-deformation frictionally contacting elastodynamics remains a challenging problem in simulation. The recently proposed Incremental Potential Contact (IPC) model\ \cite{Li2020IPC} enables inversion and intersection-free simulation of contacting elastodynamics via the application of mollified barriers, filtered line-search, and optimization-based solvers for time integration. As originally formulated, IPC begins with a discrete model, replacing non-interpenetration constraints with locally-supported $C^2$ barrier potentials, on an already spatially discretized domain. Each barrier potential, in turn, evaluates unsigned distances between boundary mesh-primitive pairs to obtain intersection-free trajectories for complex multibody simulations where domains can have arbitrarily sharp geometries and undergo large deformation. Subsequent work has extended the IPC model to solve problems in rigid and multibody dynamics\ \cite{Ferguson2021Rigid,Lan2022ABD,Chen2022MultibodyIPC}, codimensional simulation of shells and rods\ \cite{li2021codimensional}, subspace modeling \cite{Lan2021Medial}, embedded interfaces\ \cite{choo2021barrier}, viscoelasticity and elastoplasticity\ \cite{Li2022ECI}, and coupled MPM-FEM modeling\ \cite{Li2022BFEMP}.

However, while effective, the original IPC model's purely discrete formulation prohibits convergence under refinement. To enable a convergent IPC model we reformulate IPC potentials in the continuous setting and provide a first, convergent discretization thereof. We focus on providing a consistent frictional contact model that converges under refinement of discretization, while retaining the original IPC model's non-interpenetration and global convergence properties for both highly refined and coarse models --- regardless of problem complexity. To do so we re-derive contact barrier and dissipative friction, beginning from a continuous formulation while addressing short-comings in the original handling of the max operator and so physical forces.















