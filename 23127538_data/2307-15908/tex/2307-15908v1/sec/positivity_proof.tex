\subsection[Positivity of Phi\_c, Phi\_s, and Phi\_w]{Positivity of $\Phi_c$, $\Phi_s$, and $\Phi_w$}
\label{sec:positivity}

Modeling contact via barrier representation requires a sufficiently small $\dhat$. Here, for the purposes of demonstrating positivity of our barrier energies, we  further define an acceptably small scale for $\dhat$:
\begin{definition}
    For any point $x$, we call $\dhat$ \acceptable{} if, every connected component of the intersection of ball $B_{\dhat}^x$ of radius $\dhat$ centered on $x$ with the boundary contains at most one vertex.
\end{definition}
We remark that, if the rest mesh is not in contact, then $\dhat$ is \acceptable. However, $\dhat$ must be \acceptable for every time step, which can be achieved by shrinking.

Before showing the positivity of the contact barriers, we settle on a few simple statements.
\begin{remark}\label{rem:edge-bound}
    Let $x_1$ and $x_2$ be the endpoints of an edge $e$; for any point $x$ in two and three dimension, $d(x,e)\le \min(d(x,x_1), d(x,x_2))$,
\end{remark}
\begin{remark}\label{rem:tri-bound}
    Let $e_1$, $e_2$ and $e_3$ be the edges of a triangle $t$; for any point $x$ in two and three dimension, $d(x,t)\le \min(d(x,e_1),d(x,e_2), d(x,x_3))$,
\end{remark}
and
\begin{remark}\label{rem:eedge-bound}
    Let $e_2$ be an edge and $x$ be one of its endpoints; for any edge $e_1$ in two and three dimensions, $d(e_1,e_2)\le d(x,e_1)$.
\end{remark}
We are now ready to show the positivity of the different barrier potentials. 

\begin{proposition}\label{prop:psic}
If $\dhat$ is \acceptable{}, then $\Psi_c \ge 0$.
\end{proposition}
\begin{proof}
We show that 
\begin{align*}
\Psi_c(x) &= \sum_{e\in E\setminus x} b( d(x, e),\dhat) - \sum_{x_2\in \Vint\setminus x}b( d(x, x_2),\dhat)
\end{align*} 
is positive for every connected component $C$ in the intersection  between $B_{\dhat}^x$ and the boundary. Note that if $C=\emptyset$, then $\Psi_c(x)=0$. We now count the number of possible edges and vertices in $C$:
\begin{enumerate}
    \item $C$ contains only one edge and no vertices.
    \item $C$ contains only two edges and one vertex.
\end{enumerate}
No other cases are possible as the boundary is manifold, $C$ contains only one connected component, and $B_{\dhat}^x$ can contain at most a vertex. For case 1, $\Psi_c(x)\ge 0$ is trivial as it contains only positive terms. For case 2, it follows from \cref{rem:edge-bound} and $b$ being a monotonically decreasing function, that for the vertex $\tilde x$ shared by the two edges $e_1$ and $e_2$
\[
b( d(x, e_1),\dhat) - b( d(x, \tilde x),\dhat)= \Delta b\ge 0,
\]
therefore 
\[
\Psi_c(x) =  b( d(x, e_1),\dhat)+ b( d(x, e_2),\dhat) - b( d(x, \tilde x),\dhat)=
\Delta b+b( d(x, e_2),\dhat)\ge0.
\]
\end{proof}
The proof for $\Psi_s$ follows a similar idea.
\begin{proposition}\label{prop:psis}
If $\dhat$ is \acceptable{}, then $\Psi_s \ge 0$.
\end{proposition}
\begin{proof}
We show that 
\begin{align*}
\Psi_s(x) &= \sum_{t\in T\setminus x} b( d(x, t),\dhat) 
 - \sum_{e\in \Eint\setminus x} b( d(x, e),\dhat)
 + \sum_{x_2\in \Vint\setminus x}b( d(x, x_2),\dhat)
\end{align*} 
is positive for every connected component $C$ in the intersection  between $B_{\dhat}^x$ and the boundary. Note that if $C=\emptyset$, then $\Psi_s(x)=0$. We now count the number of possible triangles, edges, and vertices in $C$:
\begin{enumerate}
    \item $C$ contains only one triangle and no edges or vertices.
    \item $C$ contains only two triangles, one edge, and no vertices.
    \item $C$ contains only $n$ triangles, $m$ edged, and one vertex $\tilde x$.
\end{enumerate}
No other cases are possible as the boundary is manifold, $C$ contains only one connected component, and $B_{\dhat}^x$ can contain at most a vertex. 
For case 1, $\Psi_s(x)\ge 0$ is trivial as it contains only positive terms. 
For case 2, it follows from \cref{rem:tri-bound} and $b$ being a monotonically decreasing function, that for the vertex $\tilde e$ shared by the two triangles $t_1$ and $t_2$
\[
b( d(x, t_1),\dhat) - b( d(x, \tilde e),\dhat)= \Delta b\ge 0,
\]
therefore 
\[
\Psi_s(x) =  b( d(x, t_1),\dhat)+ b( d(x, t_2),\dhat) - b( d(x, \tilde e),\dhat)=
\Delta b+b( d(x, t_2),\dhat)\ge0.
\]
For case 3, we first note that the number of triangles $n$ is always larger or equal to the number of edges $m$. This is the case since the edges need to be in the interior (no boundary edges), and if an edge is included in $C$, then the two adjacent triangles are. Following a similar argument as for case 2, we can bound every edge barrier $b( d(x, e),\dhat)$ with one of the adjacent triangles' barriers $b( d(x, t),\dhat)$. Since $n\ge m$,
\[
\sum_{t\in T\setminus x} b( d(x, t),\dhat) 
 - \sum_{e\in \Eint\setminus x} b( d(x, e),\dhat)= \Delta b'\ge 0,
\]
and
\[
\Psi_s(x) = 
 \Delta b'+b( d(x, \tilde x),\dhat)\ge 0.
\]
\end{proof}
Finally, we show the positivity of the edge-edge energy.
\begin{proposition}\label{prop:psiw}
If $\dhat$ is \acceptable{}, then $\Psi_w \ge 0$.
\end{proposition}
\begin{proof}
We show that 
\begin{align*}
\Psi_w(e) = \sum_{e_2 \in E\setminus e} b(d(e, e_2),\dhat) -
\sum_{x_1\in V \setminus e} (\rho(x_1)-1) b( d(x_1, e),\dhat)
\end{align*} 
is positive for every connected component $C$ in the intersection  between $B_{\dhat}^x$ and the boundary. We now count the number of possible edges and vertices in $C$ (excluding the trivial case $C=\emptyset$):
\begin{enumerate}
    \item $C$ contains only one edge and no vertices.
    \item $C$ contains $n_{e,\tilde x}$ edges and one vertex $\tilde x$.
\end{enumerate}
No other cases are possible as the boundary is manifold, $C$ contains only one connected component, and $B_{\dhat}^x$ can contain at most a vertex. For case 1, $\Psi_w(x)\ge 0$ is trivial as it contains only positive terms. For case 2, it follows from \cref{rem:eedge-bound} that for $\tilde x$ shared by the $n_{e,\tilde x}$  
\[
b( d(e, e_2),\dhat) - b( d(\tilde x, e),\dhat) \ge 0,
\]
for every edge $e_2\in C$. Therefore for an $e^\star \in E\setminus e$
\[
\Psi_w(e) = \sum_{e_2 \in E\setminus (e\cup
e^\star)} (b(d(e, e_2),\dhat) - b( d(\tilde x, e),\dhat))+b(d(e, e^\star),\dhat)\ge 0.
\]
\end{proof}


\subsection{Quality of the smooth approximation}
We just showed that our approximations to the actual maximum distance share the positivity property of the actual non-smooth max. By looking into the previous proofs, we can estimate how and where our approximations break down. In all cases, if the set $C$ contains more than one connected component, the approximation is poor. For instance, this can happen when $\dhat$ is \acceptable but larger than the high-frequency details of the mesh. In the following, we will focus on the case where $C$ contains only one component. If $C$ contains only one primitive (case 1 in the proofs), our approximation is trivially exact as the sum contains only one term, which coincides with the maximum.

\paragraph{Vertex-edge case} In the proof of \cref{prop:psic}, we arbitrary select $e_1$ to bound the point distance $b( d(x, \tilde x),\dhat)$. Let $e^\star=\argmin_{e_i=\{e_1, e_2\}}(d(x, e_i))$, $\bar e^\star$ the other edge and $\Delta b= b( d(x, \bar e^\star),\dhat) - b( d(x, \tilde x),\dhat)$. It is now clear to see that the error in the approximation is exactly $\Delta b$, which happens when $x$ is closer to the edge $\bar e^\star$ than to the vertex $\tilde x$ as shown in \cref{fig:corner}.

\paragraph{Vertex-triangle case} Case 2 in the proof of \cref{prop:psis} follows the same argument as the vertex-edge case. We call $t^\star$ the triangle closest to $x$, $\bar t^\star$ the other one, and $\Delta b= b( d(x, \bar t^\star),\dhat) - b( d(x, \tilde e),\dhat)$. In this case, the error is $\Delta b$, which happens when $x$ is closer to the triangle $\bar t^\star$ than to the edge $\tilde e$ (imagine an extruded version of \cref{fig:corner}). For the sake of simplicity, we exclude the case where the surface has boundaries; therefore, in case 3, we have the same number of edges and faces. Let us call $t^\star$ the triangle closest to $x$ and $\bar T$ the triangles in $C$ excluding $t^\star$. If there exists one edge $e^\star$ such that $d(x, e^\star)=d(x, \tilde x)$ (i.e., $x$ is sufficiently far from concavities), then the error is the sum of the differences between the barriers on the triangles and their adjacent edges. This is the case in concave parts of the mesh.

\paragraph{Edge-edge case} In the proof of \cref{prop:psiw}, we clearly see that our approximation is related to the errors introduced if the distances between the edges and the vertex are non-zero. This happens when more than one edge is closer to the vertex $\tilde x$ which is the case when $e$ is large or next to $\tilde x$.



