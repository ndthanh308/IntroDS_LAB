\documentclass[aps, reprint, prl, twocolumn, superscriptaddress,amsmath,amssymb,noeprint]{revtex4-2}
% \documentclass{article}
\usepackage{times}
\usepackage{amsmath}
\usepackage{amssymb}
\usepackage{xfrac}
\usepackage{dsfont}
\usepackage{graphicx}
\usepackage{float}
\usepackage{braket}
\usepackage{bbold}
\usepackage[normalem]{ulem}
\usepackage{bm}
\usepackage{bbm}
\usepackage{color}
\usepackage{xcolor}

\usepackage{pdfpages}
\usepackage{pgffor}

\makeatletter
\AtBeginDocument{\let\LS@rot\@undefined}
\makeatother


\usepackage[colorlinks=true, urlcolor=blue, linkcolor=blue, citecolor=blue]{hyperref}
\usepackage{siunitx}
\newcommand{\zy}[1]{{\color{red}{(ZY: #1)}}}
\newcommand{\sm}[1]{{\color{orange}{(SJM: #1)}}}
\newcommand{\dmsk}[1]{{\color{blue}{DMSK: #1}}}
\newcommand{\ana}[1]{{\color{blue}{Ana: #1}}}

\newcommand{\berkeleyphy}{Department of Physics, University of California, Berkeley, California 94720}
\newcommand{\CIQC}{Challenge Institute for Quantum Computation, University of California, Berkeley, California 94720}
\newcommand{\LBL}{Materials Sciences Division, Lawrence Berkeley National Laboratory, Berkeley, California 94720}
\newcommand{\columbia}{Department of Physics, Columbia University, New York, NY 10027}

\begin{document}

\title{Super-radiant and Sub-radiant Cavity Scattering by Atom Arrays}


\author{Zhenjie Yan}
\affiliation{\berkeleyphy}
\affiliation{\CIQC}

\author{Jacquelyn Ho}
\affiliation{\berkeleyphy}
\affiliation{\CIQC}


\author{Yue-Hui Lu}
\affiliation{\berkeleyphy}
\affiliation{\CIQC}

\author{Stuart J. Masson}
\affiliation{\columbia}

\author{Ana Asenjo-Garcia}
\affiliation{\columbia}

\author{Dan M. Stamper-Kurn}
\email[]{dmsk@berkeley.edu}
\affiliation{\berkeleyphy}
\affiliation{\CIQC}
\affiliation{\LBL}
\begin{abstract}


We realize collective enhancement and suppression of light scattered by an array of tweezer-trapped $^{87}$Rb atoms positioned precisely within a strongly coupled Fabry-P\'{e}rot optical cavity.  We illuminate the array with light directed transverse to the cavity axis and detect photons scattered by the array into the cavity.  
For an array with integer-optical-wavelength spacing, in the low saturation regime, each atom Rayleigh scatters light into the cavity with nearly identical scattering amplitude, leading to an observed $N^2$ scaling of cavity photon number as the atom number increases stepwise from $N=1$ to $N=8$.
By contrast, in an array with half-integer-wavelength spacing, the scattering amplitude for neighboring atoms is equal in magnitude but alternates in sign. Scattering from such an array yields a non-monotonic, sub-radiant cavity intensity versus $N$.  By analyzing the polarization of light emitted from the cavity, we find that Rayleigh scattering can be collectively enhanced or suppressed with respect to Raman scattering. We observe also that atom-induced shifts and broadenings of the cavity resonance are precisely tuned by varying the atom number and positions. Altogether, deterministically loaded atom tweezer arrays provide exquisite control of atomic cavity QED spanning from the single- to the many-body regime.

\end{abstract}

\maketitle

As highlighted by Dicke's seminal work on super- and sub-radiance \cite{Dicke1954}, the interaction of multiple emitters with a quantum mode of light differs from that of the emitters individually.
Collective ``super-radiant'' (or ``bright'') states, resulting from constructive interference, give rise to an enhanced emission rate per excitation, which grows with the number of emitters.
Conversely, ``sub-radiant'' (or ``dark'') states arise from destructive interference, leading to a suppression or complete cancellation of photon emission.


In the case of extended samples, with emitters distributed over distances longer than the emitted optical wavelength, the observation and control of super- and sub-radiance depends critically on the exact spatial distribution of the emitters.  For example, the precise structure of mesoscopic samples \cite{Clemens2003,Masson2020,Scully2006} or periodic emitter arrays \cite{Masson2022} controls whether their collective emission is enhanced or suppressed, or directed into single or multiple optical modes.

Similarly, in cavity quantum electrodynamics (QED), where each of multiple emitters couples strongly to a single-mode cavity field, the collective optical properties depend strongly on the spatial positions of the emitters within the cavity.
Already for single emitters, controlling the position of atoms in a cavity advanced the field of cavity QED ~\cite{Guthohrlein2001,Nussmann2005,Reiserer2013a,Thompson2013a}. Basic effects of few-body cavity QED have been illustrated by the controlled placement of two atoms~\cite{Reimann2015,Neuzner2016}, ions \cite{Casabone2015,Begley2016}, and  superconducting quantum circuits~\cite{Majer2007,VanLoo2013,Mirhosseini2019,Kannan2020} within resonant cavities or waveguides.
Notably, a recent experiment showcased super-radiant and sub-radiant emission from up to 10 superconducting qubits into a microwave cavity~\cite{Wang2020a}.

In this work, we employ deterministically loaded atom tweezer arrays~\cite{Endres2016,Barredo2016}, a powerful new platform for quantum simulation~\cite{Browaeys2020}, metrology~\cite{Madjarov2019,Young2020}, and information processing~\cite{Kaufman2021}, to advance atomic cavity QED from the few- to the many-body regime while preserving precise control over the cavity interaction of each individual atom.
With such a tweezer array, we place a fixed number of $^{87}$Rb atoms at fully controlled positions along the axis of a strongly coupled Fabry-P\'{e}rot optical resonator~\cite{Liu2022,comment_zhang} [Fig.~\ref{fig:M1}(a)].
This method improves upon the incomplete control of atom number, position, and motion in previous approaches used in trapped-atom and trapped-ion cavity QED experiments~\cite{Reiserer2015,Mivehvar2021}. 
By driving this emitter array with light propagating transverse to the cavity axis while monitoring the cavity emission, we demonstrate that photon cavity number scales as $N^2$ for super-radiant emission and sub-linearly for sub-radiant scattering, both in the low saturation regime




% Figure environment removed

% Figure environment removed



We employ an in-vacuum near-concentric Fabry-P\'{e}rot optical cavity with a TEM$_{00}$ mode whose frequency ($\omega_\mathrm{c}$)  lies near the transition frequency ($\omega_\mathrm{a}$) of the $^{87}$Rb $F=2 \rightarrow F^\prime=3$ $D_2$ transition (wavelength $\lambda = \SI{780}{\nano\meter}$) ~\cite{Deist2022a,Deist2022b}.
The coupling amplitude of a single $^{87}$Rb atom to this mode varies as $g(x) = g_0 \cos kx$ near the cavity center and along the cavity axis; here, $g_0 = 2 \pi \times 3.1$ MHz  (on the cycling transition) and $k = 2 \pi/\lambda$.  Given the atomic and cavity resonance half-linewidths of $\gamma = 2 \pi \times 3.0$ MHz and $\kappa = 2 \pi \times 0.53$ MHz, respectively, the cavity achieves the strong coupling condition with single-atom cooperativity $C = g_0^2/(2 \kappa \gamma) = 3.0$.

We trap atoms within this cavity using a one-dimensional array of optical tweezers, formed by laser beams with a wavelength of \SI{808}{\nano \meter} sent transversely to the cavity through a high numerical-aperture imaging system.
Pre-cooled and optically trapped $^{87}$Rb atoms are loaded into as many as 16 tweezers, detected through fluorescence imaging, and then sorted into regularly spaced arrays of between $N=1$ and $8$ atoms \cite{suppmat}.  The array is aligned to place atoms at the radial center of the cavity.  Piezo-controlled mirrors and an acousto-optical deflector are used to position the array with nanometer-scale precision along the cavity axis.

Our experiments focus on the response of this atom array when it is driven optically.  
For this, we illuminate the array with a standing-wave of monochromatic probe light that also enters the cavity from a transverse direction, see Fig.~\ref{fig:M1}(a).
The probe is linearly polarized along the $z$-axis (chosen as our quantization axis), perpendicular to the cavity, and drives each atom with a spatially dependent Rabi frequency $\Omega(y)=\Omega_0 \cos{k y}$. 
We employ a weak probe amplitude to ensure that the probability of having more than one atom excited simultaneously is negligible.
%
All tweezer traps are centered at $y=0$. 
The probe frequency $\omega_\mathrm{p}$ operates at a small detuning  $\Delta_\mathrm{pc}=\omega_\mathrm{p}-\omega_\mathrm{c}$ from the cavity resonance [Fig.~\ref{fig:M1}(b)].  Prior to probe illumination, the tweezer-trapped atoms are polarization-gradient cooled and prepared in the $F=2$ ground hyperfine manifold, without control of their Zeeman state.  Probe light scattered by the array into the cavity, and thence through one mirror of our one-sided cavity, is detected by a single-photon counting module.

To demonstrate the strong sensitivity of collective cavity scattering on the exact positioning of the scatterers, we first examine the scattering by just two atoms.  As illustrated in Fig.\ \ref{fig:M2}(a), we first position one atom at an antinode of the cavity field.  With the cavity at a large detuning $\Delta_\mathrm{ca} = \omega_\mathrm{c} - \omega_\mathrm{a} = - 2 \pi \times 507$ MHz -- a specific ``magic'' value chosen to suppress fluctuations from internal-state dynamics, as described below -- we denote the steady-state cavity photon number generated by this single atom as $n_1$.  We then add a second atom at a variable distance $d$ from the first, and record the cavity photon number produced by the atom pair as $n_2$.  At integer-wavelength separation ($d = m \lambda$ with $m$ an integer), we observe super-radiant light scattering, where the total cavity emission rate is greater than that of two individual atoms ($n_2 > 2 n_1$).  At half-integer separation ($d = (m+1/2) \lambda$), we observe sub-radiant light scattering ($n_2 < 2 n_1$).

To account for the observed behavior, let us consider that each atom $i$, positioned at location $(x_i, y_i)$, serves as a source for light in the cavity, with scattering amplitude $\eta(x_i, y_i) = g(x_i) \Omega(y_i)/2 \Delta_\mathrm{ca} \equiv \eta_0 \cos kx_i \cos ky_i$, which we obtain by treating the atom as a two-level emitter and adiabatically eliminating its excited state.  The scattering amplitudes from all atoms add coherently.  The steady-state cavity photon number scattered by $N$ atoms is then $n_N = |\bar{a}|^2 $, where the expectation value of the cavity-field amplitude is given, following a semi-classical treatment~\cite{suppmat} in the dispersive coupling regime $|\Delta_\mathrm{ca}| \gg \{\Omega_0,g_0,\gamma\}$, as
\begin{align}
\bar{a}=\frac{\sum_{i}\eta(x_i,y_i)}{[\Delta_\mathrm{pc}-\sum_{i} g(x_i)^2/\Delta_\mathrm{ca} ] + i[\kappa + \sum_{i} \gamma g(x_i)^2/{\Delta_\mathrm{ca}^2}  ]}.
\label{eq:CavityField}
\end{align}
Here, we have accounted for the atom-induced dispersive shift $\sum_i g(x_i)^2/\Delta_\mathrm{ca}$ and absorptive broadening $\sum_i \gamma g(x_i)^2/\Delta_\mathrm{ca}^2$ of the cavity resonance.

At large $|\Delta_\mathrm{ca}|$, where we can neglect atom-induced modifications of the cavity resonance, and with $\Delta_\mathrm{pc}=0$, the cavity photon number varies simply as $n_N = |\sum_i \eta(x_i, y_i)|^2/\kappa^2$.  For two atoms, with the first situated exactly at the antinode and the second at an exact axial distance $d$, one then expects $n_2/n_1 = [1+\cos kd]^2$, with limiting values $n_2/n_1 = 4$ from constructive interference at integer-wavelength separation, and $n_2/n_1 = 0$ from destructive interference at half-integer wavelengths.  In our experiment, uncorrelated fluctuations in $\eta$, deriving from thermal position fluctuations of the two atoms within their tweezer traps, constrain these limiting values to $n_2/n_1 = 2 \left( 1 \pm D  \right)$, where the ratio $D = |\langle \eta\rangle|^2/ \langle |\eta|^2\rangle$ is the Debye-Waller factor and $\langle \rangle$ denotes an average over the position distribution of a single trapped atom.  The data in Fig.\ \ref{fig:M2}(a) are consistent with rms variations of $\sigma = 100(14)$ nm in both the $x$ and $y$ directions of motion, with  $\sigma$ determined independently by measuring light scattering from a single atom \cite{suppmat}.  In future work, $\sigma$ can be reduced further through bursts of dark-state cooling ~\cite{Kaufman2012,AngOngA2022,Brown2019}.  Strong contrast between constructive and destructive interference is retained even at large atomic separations -- as far as  $d\simeq \SI{30}{\micro\meter}$, a distance that is much smaller than the cavity mode Rayleigh range of $\sim\SI{1}{\milli\meter}$. 

The photon scattering rate also depends on the positions of the atoms with respect to the standing-wave cavity mode. 
We illustrate this dependence by measuring light scattering from atom arrays with fixed integer or half-integer spacing while translating these arrays altogether by $\Delta x$ from a cavity antinode [Fig.\ \ref{fig:M2}(b)].  Again, we observe either super- or sub-radiance when the array is aligned onto cavity antinodes.  In contrast, when either type of array is aligned to cavity nodes, we observe scattering that scales linearly with $N$.  Here, position fluctuations cause the scattering amplitude from each atom to vary between positive and negative values with equal probability, producing a cavity field with finite variance but zero average amplitude.


% Figure environment removed

The super-radiant emission by atoms at the constructive interference condition is further enhanced by increasing the atom number [Fig.\ \ref{fig:M3}(a)].  At large $|\Delta_\mathrm{ca}|$ and on cavity resonance, the number of cavity photons emitted by an $N$-atom array is predicted to be
\begin{align}
n_N = \left[ N\left(\braket{|\eta|^2}-\lvert\braket{\eta}\rvert^2\right)+N^2 \lvert\braket{\eta}\rvert^2\right]/\kappa^2,
\label{eq:N_atom_constructive}
\end{align}
This expression consists of an incoherent part that scales linearly with $N$ and that vanishes if atoms are fully localized ($\sigma=0$), and a coherent part that scales quadratically as $N^2$.  Our measurements with $N$ ranging from 1 to 8 match well with this prediction, clearly exhibiting super-radiant scattering.

At the destructive interference condition, i.e.\ an atom array with half-integer wavelength spacing, collective scattering is sub-radiant, falling below the linear scaling with $N$ expected for an incoherent sample.  Here, at similar probe and cavity settings, one expects the cavity photon number to vary as 
\begin{align}
n_N = \left[ N\left(\braket{|\eta^2|}-\lvert\braket{\eta}\rvert^2\right)+\frac{1-(-1)^N}{2} \lvert\braket{\eta}\rvert^2 \right]/\kappa^2.
\label{eq:N_atom_destructive}
\end{align}
While the incoherent scattering rate remains linear in $N$, the coherent photon scattering by pairs of atoms with opposite phase cancel out,  resulting in a total coherent contribution equal to that of either zero emitters (even $N$) or a single emitter (odd $N$).
The non-monotonic behavior observed in our experiment emerges as the coherent scattering rate exceeds the incoherent one: $\lvert\braket{\eta}\rvert^2>\braket{\lvert\eta^2\rvert}-\lvert\braket{\eta}\rvert^2$.

At smaller detuning $|\Delta_\mathrm{ca}|$, collective light scattering is affected by two additional effects.  First,  coherent scattering is degraded by polarization and intensity fluctuations arising from internal state dynamics of the multi-level $^{87}$Rb atom.  In our setup, incoherent Raman scattering causes each atom's magnetic quantum number $m_F$ to be distributed among all possible values.  For small $\Delta_\mathrm{ca} = - 2 \pi \times 38$ MHz, the amplitude $\eta$ for emitting $z$-polarized light, arising primarily from the near-resonant $F=2 \rightarrow F^\prime = 3$ transition, varies strongly with $m_F$.  This random variation further degrades the Debye-Waller factor, reducing the the collective enhancement of light scattering as shown in Fig.\ \ref{fig:M3}(a).

Further, at this detuning, atoms can scatter the $z$-polarized probe light into cavity modes of two orthogonal polarizations.  Scattering into the $z$-polarized cavity mode, i.e.\ Rayleigh scattering, is elastic in that the $m_F$ quantum number of an atom is unchanged by such scattering.  The final state of the atoms after a Rayleigh scattering event is independent of which atom scattered a photon.  Thus, the total Rayleigh scattering rate is determined by the interference of the scattering amplitudes from all atoms, and can either be enhanced or suppressed by collective effects.  By contrast, scattering into the $y$-polarized cavity mode, i.e.\ Raman scattering, is inelastic in that $m_F$ is changed, leading to a final state after a scattering event that differs depending on which atom scattered light.  As such, the intensity of Raman scattered $y$-polarized cavity light is the incoherent sum of emission intensities from each atom.


In support of this description, we examine separately the cavity emission of $z$- and $y$-polarized light [Fig.~\ref{fig:M3}(b)].  At small detuning, and comparing to the polarization of light emitted by a single atom, the emission of $z$-polarized light is enhanced by super-radiant scattering at integer-wavelength atomic spacing and suppressed by sub-radiant scattering at half-integer atomic spacing.


For much of our work, we avoid these internal-state and polarization effects by operating at the aforementioned ``magic'' detuning of $\Delta_\mathrm{ca} = - 2\pi \times 507$ MHz~\cite{suppmat,paper_in_perp}.  Here, when accounting for transitions to all three accessible excited hyperfine states ($F^\prime = 1$, $2$, and $3$), the amplitude for scattering $z$-polarized light is nearly identical for all Zeeman states within the ground $F=2$ manifold.  Simultaneously, the rate for Raman scattering $y$-polarized light is nearly extinguished, as observed in the nearly pure $z$-polarization of light emitted from the cavity [Fig.\ \ref{fig:M3}(c)].
The identification of such a ``magic'' detuning for nearly all alkali species~\cite{suppmat,paper_in_perp}, which allows light scattering by an alkali atom to resemble closely that of just a two-level atom, should be beneficial to other quantum optics experiments using alkali gases.

Additionally, at small detuning, we observe significant modifications of the cavity resonance by the atomic array within. As shown in Fig.\ \ref{fig:M4}(a), we record $n_N$ for various arrays as a function of the detuning $\Delta_\mathrm{pc}$ of the probe from the empty-cavity resonance.  From the observed emission lineshapes, we extract the atom-induced cavity resonance shift and also the total cavity linewidth.  The measured spectral shifts and widths show a near-linear dependence on the atom number $N$ [Fig.~\ref{fig:M4}(c-d)], and agree well with a theoretical calculation accounting for both atomic position and $m_F$ fluctuations ~\cite{suppmat}.    
Owing to the $g^2(x_i)$ dependence of the cavity dispersion and absorption terms, the atom-induced cavity modification is the same for atoms at both integer- and half-integer spacing.  These modifications are reduced for atom arrays centered on cavity field nodes.  By comparison, the atom-induced cavity modifications are all negligible for large $|\Delta_\mathrm{ca}|$.


% Figure environment removed


Our realization of a well-controlled many-body cavity QED system, compatible with single-atom control, offers a wide range of potential applications.
Long-range cavity-photon-induced interactions among atoms~\cite{Mivehvar2021} can be exquisitely studied, controlled, and used for realization of quantum solvers~\cite{Torggler2019}.
Constructing super-radiant atomic ensembles within cavities can enhance cavity cooperativity, allowing one to control cavity optical non-linearity~\cite{Fernandez-Vidal2007,Habibian2011} and to advance quantum networking applications~\cite{Casabone2015,Zhong2017,Holzinger2022,Ramette2022,Huie2021a}.

Moreover, this platform enables the study of many-excitation super- and sub-radiance with unprecedented control. While the current work explores only the bottom of the Dicke ladder (where only one atom is excited), stronger excitation of the tweezer array, up to its full inversion, would allow us to explore the full complexity of Dicke super-radiance~\cite{Dicke1954,Gross1982}. Our setup would provide insight into key features of many-body decay in the presence of multiple decay channels (due to the multimode nature of the cavity), such as the scaling with atom number of the peak intensity and the time of peak emission, as well as large shot-to-shot fluctuations~\cite{Clemens2004,PineiroOrioli2022,Cardenas-Lopez2022} in the light polarization due to self-reinforcing feedback.

% acknowledgment
\textbf{Acknowledgment-} We acknowledge support from the AFOSR (Grant No.\ FA9550-1910328 and Young Investigator Prize Grant No.\ 21RT0751), from ARO through the MURI program (Grant No.\ W911NF-20-1-0136), from DARPA (Grant No.\ W911NF2010090), from the NSF (QLCI program through grant number OMA-2016245, and CAREER Award No.\ 2047380), and from the David and Lucile Packard Foundation.
J.H. acknowledges support from the National Defense Science and Engineering Graduate (NDSEG) fellowship.
We thank Erhan Saglamyurek for helpful discussions.
We also thank Jacopo De Santis, Florian Zacherl, and Nathan Song for their assistance in the lab.

\bibliography{main.bib}


\onecolumngrid
\pagebreak

\foreach \x in {1,...,6}
{%
\clearpage
\includepdf[pages={\x}]{supplement.pdf} 
}


\end{document}
