%% Beginning of file 'sample631.tex'
%%
%% using aastex version 6.3
\documentclass[twocolumn]{aastex631}
%\documentclass[modern]{aastex631}
%\documentclass[preprint]{aastex631}

%% The default is a single spaced, 10 point font, single spaced article.
%% There are 5 other style options available via an optional argument. They
%% can be invoked like this:
%%
%% \documentclass[arguments]{aastex631}
%% 
%% where the layout options are:
%%
%%  twocolumn   : two text columns, 10 point font, single spaced article.
%%                This is the most compact and represent the final published
%%                derived PDF copy of the accepted manuscript from the publisher
%%  manuscript  : one text column, 12 point font, double spaced article.
%%  preprint    : one text column, 12 point font, single spaced article.  
%%  preprint2   : two text columns, 12 point font, single spaced article.
%%  modern      : a stylish, single text column, 12 point font, article with
%% 		  wider left and right margins. This uses the Daniel
%% 		  Foreman-Mackey and David Hogg design.
%%  RNAAS       : Supresses an abstract. Originally for RNAAS manuscripts 
%%                but now that abstracts are required this is obsolete for
%%                AAS Journals. Authors might need it for other reasons. DO NOT
%%                use \begin{abstract} and \end{abstract} with this style.
%%
%% Note that you can submit to the AAS Journals in any of these 6 styles.
%%
%% There are other optional arguments one can invoke to allow other stylistic
%% actions. The available options are:
%%
%%   astrosymb    : Loads Astrosymb font and define \astrocommands. 
%%   tighten      : Makes baselineskip slightly smaller, only works with 
%%                  the twocolumn substyle.
%%   times        : uses times font instead of the default
%%   linenumbers  : turn on lineno package.
%%   trackchanges : required to see the revision mark up and print its output
%%   longauthor   : Do not use the more compressed footnote style (default) for 
%%                  the author/collaboration/affiliations. Instead print all
%%                  affiliation information after each name. Creates a much 
%%                  longer author list but may be desirable for short 
%%                  author papers.
%% twocolappendix : make 2 column appendix.
%%   anonymous    : Do not show the authors, affiliations and acknowledgments 
%%                  for dual anonymous review.
%%
%% these can be used in any combination, e.g.
%%
%% \documentclass[twocolumn,linenumbers,trackchanges]{aastex631}
%%
%% AASTeX v6.* now includes \hyperref support. While we have built in specific
%% defaults into the classfile you can manually override them with the
%% \hypersetup command. For example,
%%
%% \hypersetup{linkcolor=red,citecolor=green,filecolor=cyan,urlcolor=magenta}
%%
%% will change the color of the internal links to red, the links to the
%% bibliography to green, the file links to cyan, and the external links to
%% magenta. Additional information on \hyperref options can be found here:
%% https://www.tug.org/applications/hyperref/manual.html#x1-40003
%%
%% Note that in v6.3 "bookmarks" has been changed to "true" in hyperref
%% to improve the accessibility of the compiled pdf file.
%%
%% If you want to create your own macros, you can do so
%% using \newcommand. Your macros should appear before
%% the \begin{document} command.
%%
\newcommand{\vdag}{(v)^\dagger}
\newcommand\aastex{AAS\TeX}
\newcommand\latex{La\TeX}

%% Reintroduced the \received and \accepted commands from AASTeX v5.2
%\received{March 1, 2021}
%\revised{April 1, 2021}
%\accepted{\today}

%% Command to document which AAS Journal the manuscript was submitted to.
%% Adds "Submitted to " the argument.
%\submitjournal{PSJ}

%% For manuscript that include authors in collaborations, AASTeX v6.31
%% builds on the \collaboration command to allow greater freedom to 
%% keep the traditional author+affiliation information but only show
%% subsets. The \collaboration command now must appear AFTER the group
%% of authors in the collaboration and it takes TWO arguments. The last
%% is still the collaboration identifier. The text given in this
%% argument is what will be shown in the manuscript. The first argument
%% is the number of author above the \collaboration command to show with
%% the collaboration text. If there are authors that are not part of any
%% collaboration the \nocollaboration command is used. This command takes
%% one argument which is also the number of authors above to show. A
%% dashed line is shown to indicate no collaboration. This example manuscript
%% shows how these commands work to display specific set of authors 
%% on the front page.
%%
%% For manuscript without any need to use \collaboration the 
%% \AuthorCollaborationLimit command from v6.2 can still be used to 
%% show a subset of authors.
%
%\AuthorCollaborationLimit=2
%
%% will only show Schwarz & Muench on the front page of the manuscript
%% (assuming the \collaboration and \nocollaboration commands are
%% commented out).
%%
%% Note that all of the author will be shown in the published article.
%% This feature is meant to be used prior to acceptance to make the
%% front end of a long author article more manageable. Please do not use
%% this functionality for manuscripts with less than 20 authors. Conversely,
%% please do use this when the number of authors exceeds 40.
%%
%% Use \allauthors at the manuscript end to show the full author list.
%% This command should only be used with \AuthorCollaborationLimit is used.

%% The following command can be used to set the latex table counters.  It
%% is needed in this document because it uses a mix of latex tabular and
%% AASTeX deluxetables.  In general it should not be needed.
%\setcounter{table}{1}

%%%%%%%%%%%%%%%%%%%%%%%%%%%%%%%%%%%%%%%%%%%%%%%%%%%%%%%%%%%%%%%%%%%%%%%%%%%%%%%%
%%
%% The following section outlines numerous optional output that
%% can be displayed in the front matter or as running meta-data.
%%
%% If you wish, you may supply running head information, although
%% this information may be modified by the editorial offices.
\shorttitle{The central region of NGC 4395} 
\shortauthors{Payel Nandi et al.}
%%
%% You can add a light gray and diagonal water-mark to the first page 
%% with this command:
%% \watermark{text}
%% where "text", e.g. DRAFT, is the text to appear.  If the text is 
%% long you can control the water-mark size with:
%% \setwatermarkfontsize{dimension}
%% where dimension is any recognized LaTeX dimension, e.g. pt, in, etc.
%%
%%%%%%%%%%%%%%%%%%%%%%%%%%%%%%%%%%%%%%%%%%%%%%%%%%%%%%%%%%%%%%%%%%%%%%%%%%%%%%%%
\graphicspath{{./}{figures/}}
%% This is the end of the preamble.  Indicate the beginning of the
%% manuscript itself with \begin{document}.

\begin{document}

\title{Evidence for low power radio jet-ISM interaction at 10 parsec in the dwarf AGN host NGC~4395} 

%% LaTeX will automatically break titles if they run longer than
%% one line. However, you may use \\ to force a line break if
%% you desire. In v6.31 you can include a footnote in the title.

%% A significant change from earlier AASTEX versions is in the structure for 
%% calling author and affiliations. The change was necessary to implement 
%% auto-indexing of affiliations which prior was a manual process that could 
%% easily be tedious in large author manuscripts.
%%
%% The \author command is the same as before except it now takes an optional
%% argument which is the 16 digit ORCID. The syntax is:
%% \author[xxxx-xxxx-xxxx-xxxx]{Author Name}
%%
%% This will hyperlink the author name to the author's ORCID page. Note that
%% during compilation, LaTeX will do some limited checking of the format of
%% the ID to make sure it is valid. If the "orcid-ID.png" image file is 
%% present or in the LaTeX pathway, the OrcID icon will appear next to
%% the authors name.
%%
%% Use \affiliation for affiliation information. The old \affil is now aliased
%% to \affiliation. AASTeX v6.31 will automatically index these in the header.
%% When a duplicate is found its index will be the same as its previous entry.
%%
%% Note that \altaffilmark and \altaffiltext have been removed and thus 
%% can not be used to document secondary affiliations. If they are used latex
%% will issue a specific error message and quit. Please use multiple 
%% \affiliation calls for to document more than one affiliation.
%%
%% The new \altaffiliation can be used to indicate some secondary information
%% such as fellowships. This command produces a non-numeric footnote that is
%% set away from the numeric \affiliation footnotes.  NOTE that if an
%% \altaffiliation command is used it must come BEFORE the \affiliation call,
%% right after the \author command, in order to place the footnotes in
%% the proper location.
%%
%% Use \email to set provide email addresses. Each \email will appear on its
%% own line so you can put multiple email address in one \email call. A new
%% \correspondingauthor command is available in V6.31 to identify the
%% corresponding author of the manuscript. It is the author's responsibility
%% to make sure this name is also in the author list.
%%
%% While authors can be grouped inside the same \author and \affiliation
%% commands it is better to have a single author for each. This allows for
%% one to exploit all the new benefits and should make book-keeping easier.
%%
%% If done correctly the peer review system will be able to
%% automatically put the author and affiliation information from the manuscript
%% and save the corresponding author the trouble of entering it by hand.

%\correspondingauthor{August Muench}
%\email{greg.schwarz@aas.org, gus.muench@aas.org}

\author{Payel Nandi}
\affiliation{Indian Institute of Astrophysics, Block II, Koramangala, Bangalore, India}
\affiliation{Indian Institute of Science, Bangalore, India}
\author{C. S. Stalin}
\affiliation{Indian Institute of Astrophysics, Block II, Koramangala, Bangalore, India}
\author{D. J. Saikia}
\affiliation{Inter-University Centre for Astronomy and Astrophysics, Pune 411007, India}
\author{Rogemar A. Riffel}
\affiliation{Departamento de Física, CCNE, Universidade Federal de Santa Maria, 97105-900, Santa Maria, RS, Brazil}
\author{Arijit Manna}
\affiliation{Midnapore City College, Kuturia, Bhadutala, Paschim Medinipur, West Bengal, 721129, India}
\author{Sabyasachi Pal}
\affiliation{Midnapore City College, Kuturia, Bhadutala, Paschim Medinipur, West Bengal, 721129, India}
\author{O. L. Dors}
\affiliation{UNIVAP - Universidade do Vale do Paraíba. Av. Shishima Hifumi, 2911, CEP: 12244-000 São José dos Campos, SP, Brazil}
\author{Dominika Wylezalek}
\affiliation{Astronomisches Rechen-Institut, Zentrum fur Astronomie der Universitat Heidelberg, Monchhofstr. 12-14, 69120 Heidelberg, Germany}
\author{Vaidehi S. Paliya}
\affiliation{Inter-University Centre for Astronomy and Astrophysics, Pune 411007, India}
\author{ P. Saikia}
\affiliation{Center for Astro, Particle and Planetary Physics, New York University Abu Dhabi, PO Box 129188, Abu Dhabi, UAE}
\author{Pratik Dabhade}
\affiliation{Instituto de Astrofísica de Canarias, Calle Vía Láctea, s/n, E-38205, La Laguna, Tenerife, Spain}
\author{Markus-Kissler Patig}
\affiliation{ESA - ESAC - European Space Agency, Camino Bajo del Castillo s/n, 28692 Villafranca del Castillo, Madrid, Spain}
\author{Ram Sagar}
\affiliation{Indian Institute of Astrophysics, Block II, Koramangala, Bangalore, India}

\begin{abstract}
Black hole driven outflows in galaxies hosting active galactic nuclei (AGN) 
may interact with their interstellar medium (ISM) affecting star formation. 
Such feedback processes, reminiscent of those seen in massive galaxies, have 
been reported recently in some dwarf galaxies. However, such studies have usually 
been on kiloparsec and larger scales and our knowledge on the smallest spatial
scales to which this feedback processes can operate is unclear. 
Here we demonstrate radio jet-ISM interaction on the scale of an 
asymmetric triple radio structure of $\sim$10 parsec size in NGC 4395. This triplet radio structure
is seen in the 15 GHz continuum image and the two asymmetric jet like structures
are situated on either side of the radio core that coincides with the optical
{\it Gaia} position.  The high resolution radio image and 
the extended [OIII]$\lambda$5007 emission, indicative of an outflow, are 
spatially coincident and are consistent with the interpretation of a low
power radio jet interacting with the ISM. Modelling of the spectral lines using 
CLOUDY and MAPPINGS, and estimation of temperature using Gemini and MaNGA integral
field spectroscopic data suggest shock ionization of the gas. The continuum 
emission at 237 GHz, though weak was 
found to spatially coincide with the AGN, however, the CO(2-1) line emission was found 
to be displaced by around 1 arcsec northward of the AGN core.  The spatial 
coincidence of molecular H2$\lambda$2.4085 along the jet direction, the morphology of 
ionised [OIII]$\lambda$5007 and displacement of CO(2-1) emission argues for conditions less 
favourable for star formation at $\sim$5 parsec.
\end{abstract}
\keywords{Dwarf galaxies (416) --- Active galactic nuclei (16) --- radio jets (1347) --- AGN 
host galaxies (2017)}

\section{Introduction} \label{sec:intro}

\noindent
Active galactic nuclei (AGN), powered by accretion of matter onto supermassive black holes at the centre of galaxies \citep{1984ARA&A..22..471R} affect their host galaxies through the feedback process.  They can have an impact on the interstellar medium (ISM) of their hosts via energetic outflows \citep{2021A&A...648A..17V}. 
These outflows driven by radiation pressure, jets or winds from AGN, can occur from accretion disk to galaxy scales \citep{2018MNRAS.479.5544M,2021A&A...656A..55M}, 
and  can inhibit (negative feedback; \citealt{2012MNRAS.425L..66M}) or enhance star formation (SF) (positive feedback; \citealt{2020A&A...639L..13N}, \citealt{2017Natur.544..202M}). Both positive and negative feedback processes are also seen in a single system  \citep{2019ApJ...881..147S,2021A&A...645A..21G}.
Outflows affecting SF through the interaction of radio jets with the ISM in the host galaxies, are known for large massive galaxies \citep{2012ARA&A..50..455F}. Recent observational evidences of dwarf galaxies hosting AGN \citep{2022Natur.601..329S}, challenge theoretical models that generally invoke supernovae feedback in dwarf galaxies \citep{2022MNRAS.516.2112K}. Also, recently outflows have been observed in AGN hosted by dwarf galaxies \citep{2019ApJ...884...54M,2021ApJ...911...70B}.
Available observations of AGN have usually identified the impact of jets on the ISM on kpc or larger scales.  To have a clear understanding of the effect of jets
on the ISM, one needs to study their impact from sub-pc to kpc and larger scales. 
%On small scales, one needs to spatially resolve the site of jet-ISM interaction as well as disentangle its effects from radiation and winds from the AGN. 
There are hardly any observational evidences on jet-ISM interaction and its 
impact on the host galaxies of AGN on parsec scales. 


NGC~4395 is a bulgeless dwarf galaxy at a distance of 4.3$\pm$0.3 Mpc 
\citep{2004AJ....127.2322T} and hosting a radio-quiet  AGN \citep{1989ApJ...342L..11F}. Its 
nucleus has the optical spectrum characteristic of a Seyfert 1 type AGN, hosts a 
black hole \citep{1993ApJ...410L..75F} in the mass range of 
10$^3$$-$10$^5$ M$_{\odot}$ \citep{2005ApJ...632..799P,2019NatAs...3..755W}, is 
point like at X-ray wavelengths and is also variable \citep{2005MNRAS.356..524V}. 
NGC~4395 appeared unresolved with the Very Large Array (VLA) A-configuration at 
1.4 GHz with a flux density of 1.68 mJy \citep{2001ApJS..133...77H}. 
In this paper, we present the first observational
evidence of the radio jet in NGC 4395 interacting with 
the ISM of its host galaxy driving shock and  [OIII]$\lambda$5007 outflow on 
parsec scale. Adopting a cosmology of H$_0$ = 70 km s$^{-1}$ Mpc$^{-1}$, 
$\Omega_M$ = 0.7,  $\Omega_{vac}$ = 0.3 and a distance of 4.3$\pm$0.3 Mpc
\citep{2004AJ....127.2322T}, 1$^{\prime\prime}$ in NGC 4395 corresponds 
to 21 parsec. The 5 $\times$ 5 square arcmin of the central region of 
NGC 4395 in near-UV (NUV), H$\alpha$ and 8 $\mu$m is shown in Fig. \ref{figure-1}.

% Figure environment removed

% Figure environment removed

% Figure environment removed

% Figure environment removed

% Figure environment removed

% Figure environment removed

% Figure environment removed

% Figure environment removed

% Figure environment removed

% Figure environment removed

% Figure environment removed

\section{Data reduction and analysis}
To characterise the jet-ISM interaction at the scale of parsec, we utilised data from
both ground and space based telescopes from low energy radio to high energy X-rays.

\subsection{X-ray}
We used four epochs of observations (OBSID: 402, 882, 5301, 5302) carried out by 
the {\it Chandra} X-ray observatory with the advanced CCD imaging spectrometer (ACIS,
0.5$-$7 keV) for exposures ranging from $\sim$1 ksec to $\sim$31 ksec.
We reduced the data using the Chandra Interactive Analysis
of Observations (CIAO, version 4.14) software and calibration files
(CALDB version 4.9.8). We first downloaded the data and reprocessed them
by running the task {\tt chandra\_repro} to generate the cleaned and
calibrated event files. Next, we  combined all the event files, 
computed the exposure maps and generated 
exposure-corrected image in the default 0.5$-$7 keV energy range for a
total exposure of $\sim$79 kilosec. We adopted the task {\tt merge\_obs} 
for this purpose. We also rebinned the data by one-quarter of the native
0.492 arcsec per pixel giving an effective resolution of
0.123 arcsec per pixel. The image is shown in Fig. \ref{figure-2}.


\subsection{UV}
The  Ultra Violet Imaging Telescope (UVIT; \citealt{2020AJ....159..158T}) on 
board AstroSat \citep{2014SPIE.9144E..1SS}, India's multi-wavelength
Observatory,  observed NGC 4395 in far-UV (FUV; 1300 $–$ 1800 
\AA) and near UV (NUV; 2000 $–$ 3000 \AA) on 27 February 2018 (PI: 
Kshama S Kurian, Observation  ID:A04 176T02 9000001924).  We  used  the  
science  ready  level  2 images available at the Indian Space Science Data 
Center and further processed using locally developed {\it Python} 
scripts (see \citealt{2023arXiv230408986N} for details). 
In this work, we used the image acquired in the 
NUV filter, N263M ($\lambda_{eff}$= 2632 \AA; $\Delta \lambda$ = 27.5 \AA)
for a total exposure time of 1355 seconds (see Fig. \ref{figure-3}). 

\subsection{Optical Imaging}
Observations of NGC~4395 carried out by HST WFC3-UVIS2 using a range of filters are available in the HST archives\footnote{https://archive.stsci.edu/} {(Proposal ID: 12212, PI: D. Michael Crenshaw)}. Of these, we used the data 
in two filters, one F502N, which is centred at 5009.87 \AA ~and the other F547M, which is centred around the 
nearby continuum at 5756.9 \AA.
We converted the observed [OIII]$\lambda$5007 F502N and F547M images to flux scale using the KEYWORD {\tt PHOTFLAM} given in the image headers. We then subtracted the flux-calibrated F547M image from
the flux calibrated F502N image to get the continuum subtracted [OIII]$\lambda$5007 image
following \cite{1990PASP..102.1217W} as
\begin{equation}
f(line) = \frac{[f_{\lambda}(N) - f_{\lambda}(B)]W(N)} {[1-W(N)/W(B)]}.
\end{equation}
Here, $f_{\lambda}(N)$ and $f_{\lambda}(B)$ are the flux densities
in the narrow F502N  and broad F547M filters, while W(N) and W(B)
are the widths of the narrow F502N and broad F547M filters, respectively.
The observed broad band image, the narrow band image and the continuum
subtracted image are shown in Fig. \ref{figure-4}. 


\subsection{Optical/infrared integral field spectroscopy}
We used archival near infrared and optical integral field spectroscopic (IFS) 
observations obtained with the Gemini and SDSS telescopes. 

\subsubsection{Gemini}
For the optical, we used the archival data from the Gemini Multi-Object
Spectrograph (GMOS) under the program ID GN-2015A-DD-6 (PI. Mason Rachel). GMOS 
with a field of view (FoV) of 5.0 $\times$ 3.5 square arcsec covers 
the spectral 
range from 4500$-$7300 \AA.  In the infrared, we used the archival data from the 
adaptive optics assisted K-band observations acquired with the near infrared 
integral field spectrograph (NIFS) under the program ID GN-2010A-Q-38 (PI. Anil 
Seth). The K-band centred at 2.2 $\mu$m covers a FoV of 3.4$\times$3.4 square 
arcsec. In both the optical and infrared integral field unit (IFU) observations, 
each spaxel covers a region of 0.05$\times$0.05 square arcsec on the sky.  
We reduced the GMOS and NIFS data following standard procedures in IRAF (see 
\citealt{2019MNRAS.486..691B} for details).

Fitting of the emission lines involve removal of the underlying continuum. In 
the case of GMOS data, we identified line free regions on either side of our 
region of interest, namely the [OIII]$\lambda$5007 region 
($\lambda\lambda$ = 4990$-$5040 \AA).  We fitted a polynomial to the line 
free regions and then subtracted the function from the observations. After 
continuum subtraction, we fitted the [OIII]$\lambda$5007 emission line with two 
Gaussian components to extract the flux and other properties of the line using
the non-linear least square minimization algorithm within Curvefit module
of Scipy \citep{2020SciPy-NMeth}.
An example of the fit is shown in Fig. \ref{figure-5}. We adopted similar 
procedure to fit the other lines from the optical GMOS spectra such as H$\beta$, 
H$\alpha$, [NII]$\lambda\lambda$6548,6584, [SII]$\lambda\lambda$6716,6732 and 
H2$\lambda$2.4085 line from NIFS.

From the Gaussian fits to the [OIII]$\lambda$5007 line emission in the 
observed spectra (not corrected for instrumental resolution), we estimated 
non-parametric values \citep{2014MNRAS.442..784Z} such as the  velocity ($v50$, 
the velocity where the cumulative flux of the line becomes half of the total flux), 
velocity dispersion $(W90 = v95 - v5$, where $v95$ and $v5$ are the velocities 
at which the flux becomes 95$\%$ and 5$\%$ of total flux) and the asymmetry 
$\left(R= \frac{(v95 - v50)-(v50 - v5)}{v95-v5}\right)$ of the line. Also, from 
fits to the [SII] doublet and using the ratio of the [SII]$\lambda$6716 to 
[SII]$\lambda$6731 lines, we estimated the electron density. This line ratio is 
sensitive to electron densities of the order of $\sim$10$^2$ - 10$^4$ cm$^{-3}$. 
We calculated internal extinction E(B-V) from H$\alpha$ and H$\beta$ line ratio 
using the following formula \citep{1972ApJ...172..593M, 1995ApJS...98..171V}
\begin{equation}
\label{eq:extinction}
%E(B-V)= 1.925 log{\frac{(\frac{I_{H\alpha}}{I_{H\beta}})_{obs}}{(\frac{I_{H\alpha}}{I_{H\beta}})_{int}}}
E(B-V)= 1.925 \times log{\frac{\left(\frac{I_{H\alpha}}{I_{H\beta}}\right)_{obs}}{3.1}}.
\end{equation}
%$(\frac{I_{H\alpha}}{I_{H\beta}})_{int}$ we have consider 3.1 \citep{1995ApJS...98..171V}.
The maps for the velocity and velocity dispersion of the [OIII]$\lambda$5007 line 
emitting gas and for the asymmetry parameter of the line are given in 
Fig.~\ref{figure-6}, whereas the E(B-V) map and the electron density map are 
shown in Fig.~\ref{figure-7}. The molecular H2 image is shown in Fig. 
\ref{figure-8}.
%\\[2pt]


\vspace{3 pt}
\subsubsection{\bf SDSS/MaNGA}
From the Sloan Digital Sky Survey (SDSS) we  used data of NGC 4395 observed
as part of the Mapping Nearby Galaxies at Apache Point Observatory (MaNGA) 
survey.  The pixel scale of MaNGA product is 0.5 $\times$ 0.5 square arcsec. It 
covers the wavelength range of 3600 \AA ~to 10000 \AA ~with a spectral 
resolution ($\lambda/\Delta\lambda$) of $\approx$ 2000. We used the spectrum of the central pixel which 
is an average spectrum of the whole region, we are interested in. By using the 
advantage of this wavelength region, we detected shock sensitive lines 
[OIII]$\lambda$4363, HeII$\lambda$4886 (which are beyond the limit of GMOS), 
[OIII]$\lambda \lambda$4959,5007,  H$\beta$, [NII]$\lambda$5755 and 
[NII]$\lambda \lambda$6548,6584 lines (as shown in Fig \ref{figure-9}) and 
estimated the overall average parameters. We fitted the emission lines in the 
same way as explained in Section 2.4.1  and estimated the emission line fluxes.

% Figure environment removed


\subsection{ALMA}
We used the archival data, observed with the high-resolution 
Atacama Large Millimeter/submillimeter Array (ALMA) with 12-m antennas (Data-ID: 
2017.1.00572.S, PI: Davis, Timothy). 
The 
observation was carried out on March 22, 2018 and January 23, 2019, with ALMA band 6 in the frequency 
range of 227.47$-$246.43 GHz. The on-source integration time was 2037 s and 423 s, respectively. During 
the observations, a total of 46 antennas were used, with a minimum baseline of 
15.1 m and a maximum baseline of 783.5 m. For the observations on both days, the quasar J1221+2813 was observed as 
a phase calibrator, and J1229+0203 was observed as a flux density and bandpass calibrator.

We reduced the data using the Common Astronomy Software Application (CASA v5.4.1) 
with the standard data reduction pipeline of the ALMA observatory. 
We show in Fig.~\ref{figure-10} (upper left) the continuum image of NGC~4395 at 
237.1227 GHz observed on 22 March 2018 having a synthesized beam size of 
0.805$\times$0.469 square arcsec along position angle (PA) of 356 deg.  From two 
dimensional Gaussian fits we found the peak and integrated flux densities to 
be 93.1$\pm$20.0 $\mu$Jy beam$^{-1}$ and 131.2$\pm$45.0 $\mu$Jy respectively. 
Also, the continuum image at 237.1227 GHz, observed on 23 January 2019 is shown 
in Fig.~\ref{figure-10} (upper right). It has a synthesized beam size of 
1.935$\times$1.253 square arcsec along PA of 3 deg. From Gaussian fits to the data, we found the peak and integrated flux densities to be 274$\pm$21 $\mu$Jy beam$^{-1}$ and 287$\pm$41 $\mu$Jy respectively. These results from an independent analysis are also in agreement with those of \cite{2022MNRAS.514.6215Y}. We used the task TCLEAN to generate the spectral data cubes. The CO line maps are shown in Fig. \ref{figure-10} (lower panels). From both the observations, we found the peak of the CO(2-1) emission to be displaced by around 0.9 arcsec (19 parsec) from the nucleus (as determined by \textit{Gaia}) of NGC~4395.  



\subsection{VLA}
The source was observed with the VLA A-configuration at 15GHz (PI: Payaswini Saikia, 
Legacy  ID: AS1409). We reduced the data using  standard  procedures that includes
flagging for bad data  using  
CASA (see \citealt{2018A&A...616A.152S} for  details). The  beam  size  obtained  
is  0.129 $\times$ 0.124 square arcsec  with a PA of $-$18 deg. The 
final image in 15 GHz has a rms flux of 11.2 $\mu$Jy.  

The source was also  observed  at  
4.8  GHz (C band) in  VLA  B-configuration  (PI:J.S.  Ulvestad,  Legacy  ID: 
AU079). We reduced the object using standard procedures in AIPS by using 3C286 
as flux calibrator and 1227+365 as phase calibrator. We achieved a rms flux of 
47.6 $\mu$Jy. The beam size in the final reduced image is is 1.75 $\times$ 1.19 
square arcsec with a position angle of  89 deg. The final images in 15 GHz and
4.8 GHz are shown in Fig. \ref{figure-11}.

% Figure environment removed


\section{Result and Discussion}
\subsection{Radio morphology}
%Building on possible evidences of ultra-violet \citep{2004ApJ...612..152C} and 
%X-ray \citep{2003MNRAS.341..973S} outflows, \cite{2006ApJ...646L..95W} observed 
%it with the High Sensitive Array (HSA) at 1.4 GHz. They found an extension of 
%15 mas (0.3 pc), along a position angle of 28$^\circ$,  and a total flux density 
%of 0.74$\pm$0.05 mJy. They suggested this to be an outflow from the prominent 
%component (E in Figure \ref{figure-1}) then believed to be the core. 
The VLA 
15~GHz image (Fig. \ref{figure-11}, left panel) showed the source to be a 
triple with E being the eastern component 
of the triple \citep{2018A&A...616A.152S}. The weak central component of the 
triple is coincident with the \textit{Gaia} position, while component E is 
displaced from it by about 220 mas, corresponding to a projected distance 
of 4.6 parsec. The western component (W) is separated from the central component by 
4.2 parsec. However, the overall projected extension of the source is 11.0 parsec. The 
source is also highly asymmetric in brightness, the ratio of peak brightness of 
components E to W is 3.8. Component E has a spectral index, $\alpha$, 
of $-0.64\pm0.05$ (S$\propto \nu^{\alpha}$) and a brightness temperature, 
$T_B$ of (2.3$\pm$0.4)$\times$10$^6$ K, showing it to be a non-thermal source. 
For the central feature $\alpha = -0.12\pm0.08$, and non-detection of a 
sub-parsec scale compact component sets $T_B$ $<$ 5.9$\times$ 10$^5$ K 
\citep{2022MNRAS.514.6215Y}. Radio cores being resolved out in low-mass AGN 
when observed with milliarcsec resolution has been reported 
earlier \citep{2017ApJ...845...50N}. Variability or episodic nuclear jet activity 
could also contribute to non-detection of a core. 

The triple structure is 
strongly reminiscent of bipolar jet ejection in radio-loud AGN, and we suggest 
that the outer components (W and E) are formed by weak radio jets from the 
intermediate-mass black hole, and refer to the central component as the radio 
core. We refer to W and E, the end-points of the radio emission as jets in this 
paper to explore jet-ISM interaction. Low power radio jets (P $<$ 10$^{42}$ 
erg s$^{-1}$) can have a significant effect on the ISM of the host galaxy, 
interacting with clouds of gas and heating the gas, entraining ambient gas, 
losing collimation and sometimes forming arc-like fronts \citep{10.1093/mnras/sty390}.  
\subsection{Multi-wavelength structure of NGC~4395}
%We show in Fig. \ref{figure-1}, the UVIT image of NGC 4395 in the 
%filter F263M over a region of $\sim$ 30 $\times$ 30 square arcsec. Also, in the 
%same figure are the expanded views of the central region at different wavelengths. 
The 15 GHz image (Fig.  \ref{figure-11}) shows the highly 
asymmetric triple structure discussed earlier. In luminous radio galaxies, the 
components seen on the side of the jets interacting with a dense cloud in the 
ISM are usually nearer and brighter \citep{2021A&ARv..29....3O,2022JApA...43...97S}, 
as there is greater dissipation of energy on this side and the dense clouds 
inhibit the advancement of the jets. In the case of NGC~4395 the brighter 
component is farther from the nucleus, although its high-resolution radio 
structure and our optical emission line study suggests interaction of the jet 
with the ISM. Therefore a degree of intrinsic asymmetry in the radio jets 
cannot be ruled out. 

% \textbf{similar to that seen in the compact steep radio source B2 0258+35 (\citealt{2019A&A...629A..58M}; \citealt{2022ApJ...938..105F}) and in IC 5063 \citep{1998AJ....115..915M}}.
%We also calculated the radio-loudness of the source as R = F$_{5 GHz}$/F$_{4400}$. 
%Using the flux density of (0.80$\pm$0.09) mJy at 5 GHz \citep{2001ApJS..133...77H} and the optical g-band psf brightness of \textbf{(16.541$\pm$0.015) mag} \citep{2009ApJS..182..543A}, we obtained a value of \textbf{R = 0.76$\pm$0.09}. Therefore, NGC~4395 is a radio-quiet AGN (see also,\citealt{2006ApJ...646L..95W}), with a small scale jet covering a total extent 
%of 11.1 parsec corresponding to an angular size of 0.53 arcsec. 
%In the relativistic beaming model, the observed emission is the sum of two components, namely the beamed core emission and the unbeamed lobe emission. We calculated the core dominance parameter as CD = S$_{core}$/(S$_{total}$ - S$_{core}$). From the 15 GHz image, 
%using the measured core flux density of 0.07 mJy and the total flux density of 0.21 mJy obtained using the task TVSTAT in AIPS, we obtained a value of CD = 0.22. The source is thus not  a core dominated source. 
%Also, it is known that sources having 
%R$_X$ which is the ratio of flux density at 5 GHz to the flux density in the 
%X-ray band, lesser than $-$5.5 can be called as radio-quiet. We found a value 
%of R$_X$ = $-$5.5. 

Fig. \ref{figure-4} shows the [OIII]$\lambda$5007 image 
of the 2$\times$2 square arcsec, from HST We found the [OIII]$\lambda$5007 
emission to be prevalent over the central 1$\times$1 square arcsec. The image 
exhibits a 
convex-shaped structure at the terminal points on either side of the AGN core, 
although less conspicuous on the eastern side. We suggest that this structure 
indicates an outflow. The [OIII]$\lambda$5007 emission also has an asymmetric 
morphology, being more prominent on the western side where the radio component 
is weaker. The line emission peaks at the nucleus identified 
by the {\it Gaia} position.  

In Fig. \ref{figure-8} we show the 1$\times$1 square 
arcsec map of the source in molecular H2 at $2.4085$ $\mu$m, obtained from
NIFS on the Gemini telescope. The molecular H2$\lambda$2.4085 is also extended, 
in the East-West direction and spatially coincident with the 15 GHz radio emission.
The 4.8 GHz emission is also spatially coincident 
with the 15 GHz emission and oriented in the East-West direction. The continuum 
emission at 237 GHz too coincides with the central radio source at 15 GHz and 
the optical {\it Gaia} position. However, the CO(2-1) line emission is 
concentrated at a larger distance ($\sim$0.9 arcsec) from the central nuclear 
emission (see Fig. \ref{figure-10}). The X-ray image (Fig. \ref{figure-2}) 
too shows emission centred around the nuclear 
emission and having extended emission along the East-West direction. 

\subsection{Radio and [OIII]$\lambda$5007 emission}
Fig.~\ref{figure-12} shows the [OIII]$\lambda$5007 map of NGC~4395 over a region 
of 1$\times$1 square arcsec in the total line emission (left panel), the narrow 
line component (middle panel) and the broad outflowing line component (right 
panel). Also, overplotted in these figures are the 15 GHz radio contours in 
green and the [OIII]$\lambda$5007 HST emission in black. The broad outflowing 
component of [OIII]$\lambda$5007 emission is brighter in the eastern side, where 
the radio emission also tends to be brighter. From these figures, it is evident 
that the total [OIII]$\lambda$5007 emission is prevalent over the entire extent 
of the radio emission, with the peak of the [OIII]$\lambda$5007 emission 
coinciding with the peak of the 15 GHz emission. We note here that the 
[OIII]$\lambda$5007 flux from GMOS and 
that from HST are comparable, the pixel scales of the [OIII]$\lambda$5007 images 
from GMOS and HST are 0.05 and 0.04 arcsec, respectively. The HST 
[OIII]$\lambda$5007 emission is asymmetric, has a cone like structure with a 
convex shape near the terminal points of the radio jet, resembling bow shock  
typically seen in the lobes of FRII radio galaxies \citep{1999ASPC..176..377K}. 
The eastern [OIII]$\lambda$5007 cone has a narrower opening angle, while the 
western cone has a wider opening angle. Also, the [OIII]$\lambda$5007 brightness 
is nearly flat at the terminal point of the eastern cone while the western 
[OIII]$\lambda$5007 cone has a more convex shaped morphology. We found an 
anti-correlation between the brightness in the radio and [OIII]$\lambda$5007 
emission.  The component W is fainter in the radio, while  the 
[OIII]$\lambda$5007 emission here is brighter. Similarly, component E is brighter 
in radio, but the [OIII]$\lambda$5007 emission is fainter. The difference in the 
brightness of the [OIII]$\lambda$5007 emission could be due to the 
jets passing through an inhomogeneous medium. The eastern jet is 
brighter in the radio band, travelling in a denser and larger E(B-V) 
medium (see Fig. \ref{figure-7}), could have ionized the gas, leading 
to dimmer [OIII]$\lambda$5007 emission and brighter synchrotron emission because 
of shock-induced compression.  Similarly, the western jet is travelling in a 
less dense medium, with smaller E(B-V) values, and enhanced [OIII]$\lambda$5007 
emission. The observed morphology of the source in radio and [OIII]$\lambda$5007 
is unambiguous evidence for the interaction of the radio jets with the ISM of 
the host of NGC~4395 and is the first structural evidence of jet-ISM interaction 
operating on scales $\sim$10 parsec in a triple radio source. 



% Figure environment removed

\subsection{BPT analysis}
Emission line ratios in the optical are an essential tool to distinguish between
star forming galaxies and AGN. Also, they can be used to disentangle
processes that lead to the line emission from SF, AGN and shock . To measure the emission line fluxes, we fitted line profiles of
H$\alpha$ and [NII]$\lambda\lambda$ 6548,6584, [SII]$\lambda\lambda$ 6717,6731,
[OIII]$\lambda$5007 and H$\beta$ in the spectra of each spaxel, using two Gaussian
components for narrow lines and three Gaussian components for broad Balmer lines (H$\alpha$ and H$\beta$). The extra component in all lines is to represent the contribution from outflowing gas, while other components are for the broad line region (BLR) and narrow line region (NLR). During the fitting of the [NII] and
H$\alpha$ lines, the line widths of the narrow components were tied together, and the peak fluxes were left free. For Balmer lines (H$\alpha$ and H$\beta$) we used the same velocity shift for narrow component and one broad component, which are responsible for the NLR and BLR region, respectively. While fitting the [SII] lines, the width of these two lines were tied together.
%Similarly, while fitting the H$\beta$ [OIII]$\lambda$5007 region, the widths of the narrow component of both H$\beta$ and [OIII]$\lambda$5007 were tied together.
We used the [OIII]$\lambda$5007/H$\beta$ versus [SII]/H$\alpha$ as well 
as [OIII]$\lambda$5007/H$\beta$ versus [NII]/H$\beta$ diagnostic diagrams to 
investigate the physical processes causing the emission lines. These diagnostic 
diagrams are shown in Fig.~\ref{figure-13}. Each point in these diagrams 
represents one spaxel in the 1$\times$1 square arcsec region centered around 
NGC~4395. Here, the red star is the AGN, and the blue and magenta triangles 
represent the spaxels in the eastern and western jet components. Though  all the 
spaxels lie in the AGN region of the Baldwin, Phillips and 
Terlevich (BPT; \citealt{1981PASP...93....5B}) diagram, there is a clear 
segregation between the core, the eastern and the western components.

\subsection{Diagnostics of the emission lines: Photoionization by AGN and/or shock}
\subsubsection{Photoionization modelling }
To characterise the ionization processes that operate in the central 0.8 arsec region
of NGC~4395, we carried out comparison of emission line measurements from the observed
GMOS spectra to photoionization using CLOUDY and shock models from MAPPINGS-III and
implemented in ITERA \citep{2013ascl.soft07012G}.
The emission lines
in the spectra of material photoionized by AGN depend on the ionization parameter U, the slope of
the ionizing continuum, $\beta$ ($\phi_\nu \propto \nu^{\beta}$),  the gas density and its metallicity.
We generated output spectra for a range of input parameters with $\beta$ ranging from $-$2 to $-$1.2 and
log (U) varying from $-$4.0 to 0.0. We assumed solar metallicity and a hydrogen density of
$n_H$ = 1000 cm$^{-3}$.

Similarly, to generate the emission line spectra from shocked material, we used the MAPPINGS-III code
again implemented in ITERA. We considered shock velocities (v) between 100 and 1000 km s$^{-1}$.
The metallicity was assumed solar consistent with the photo-ionization model calculations, and we
considered both pure shock and shock plus precursor models. The magnetic parameter B was allowed to
vary between 0.01 to 1000 $\mu$G. We show in Fig. \ref{figure-14}, the comparison between model
line ratios and observed line ratios in the log([OIII]$\lambda$5007/H$\beta$) and
log([SII]/H$\alpha$) plane for photo-ionization by AGN (left panel) and
photo-ionization by shock  (right panel).  The observed line ratios of the pixels
in the central 1$\times$1 square arcsec tend to lie in the region predicted by
shock models. Thus, the observations analysed in this work show evidence of
shock  contributing to the ionization of the gas in the central region of NGC~4395.
This is possible with the hypothesis that the expanding radio jets from the central
core, on its interaction with the ISM, leads to the shock  in the medium, which
dominates the ionization of the gas over other processes, such as  photoionization by AGN or stars.

\subsubsection{Electron temperature distribution}
Knowledge of the electron temperature ($T_e$) in the central regions of AGN can help one to constrain the contribution of AGN to gas ionization.  Shock from AGN outflows could produce higher values of  $T_e$ \citep{2021MNRAS.501L..54R}. We calculated the integrated $T_e$ using two line intensity ratios namely R$_{O3}$ = ([OIII]$\lambda\lambda$ 4959,5007)/$\lambda$4363 and R$_{N2}$ = ([NII]$\lambda\lambda$6548,6584)/$\lambda$5755 from MaNGA spectra and adopting the following relations (\citealt{2021MNRAS.501L..54R}; \citealt{10.1093/mnras/staa1781}).
\begin{equation}
\frac{T_{e[OIII]}}{10^4 K} = 0.8254 - 0.0002415R_{O3} + \frac{47.77}{R_{O3}}
\end{equation}
\begin{equation}
\frac{T_{e[NII]}}{10^4 K} = 0.537 + 0.000253 \times R_{N2} + \frac{42.13}{R_{N2}}
\end{equation}
We found $T_{e[NII]}$ = 16420 K and $T_{e[OIII]}$ = 16818 K. These values are too large to be produced solely by AGN photo-ionizaion.

To better characterise the spatial nature of $T_e$, we used [NII] lines from GMOS spectra to generate a spatially resolved map of $T_e$. Since [OIII]$\lambda$4363 is not covered by the GOMS spectra we used the line ratio R$_{N2}$ to generate the $T_e$ map. For this we considered only those spaxels where the S/N ratio (ratio of the peak of the [NII]$\lambda$5755 line to the standard deviation of the pixels in the adjacent continuum) is greater than 30. The $T_e$ map is shown in Fig.~\ref{figure-15}. We found $T_e$ to have a range of values, with the value increasing from the center of NGC~4395 outwards, both towards the eastern and western terminal points of the radio jet. This increase of temperature towards the eastern and western side is evident in the temperature difference map shown in the right panel of Fig~\ref{figure-15}. This temperature difference map is generated by subtracting each temperature value from the mean of the temperature calculated over the central 0.05$\times$0.05 square arcsec region. The increase of temperature from the center of NGC~4395 towards the edges coinciding with the radio jet points to the gas being ionised by shock. Shock could be produced by the interaction of radio jet with the ISM, and this increase of $T_e$ from the center towards the edges is a direct evidence of shock ionisation \citep{2021MNRAS.501L..54R}.

\subsection{Nature of radio emission in the central 10 parsec region}
The radio emission observed in the central parcsec-scale region of a dwarf 
galaxy could be from a variety of physical processes, such as low-power jets, 
AGN-driven wind, SF, coronal activity and free-free emission from thermal 
gas \citep{2019NatAs...3..387P}. Radio structure, spectral index, polarization 
characteristics of the radio emission if detectable in future, spatial 
correlation of radio structure and different gasesous components, spectral 
line diagnostics of the different components of the ISM, could provide 
valuable clues in identifying the dominant processes.
In NGC~4395, radio morphology of a triple radio source clearly showing signs of interaction prominently on the eastern side with the emission following the line of least resistance clearly indicated jet-ISM interaction. This was reinforced from a detailed study of the line-emitting gas.
The [OIII]$\lambda$5007 emission appears closely associated with the radio source, ionized by shock as the jets flow outwards. Line-ratio diagnostics, estimation of gas temperatures from line ratios, and comparison of line ratios with theoretical predictions using the CLOUDY and MAPPINGS models, all showed shock to be the dominant process responsible for the ionization. The eastern component which is more prominent at radio wavelengths showed stronger signs of interaction with the ISM than the western component.

The spectral index derived over the 1.4 $-$ 15 GHz range gives a value of $\alpha$ = $-$0.54$\pm$0.34 (S$_\nu$ $\propto$ $\nu^{\alpha}$). However, considering only the eastern jet/component, $\alpha$ = $-$0.64$\pm$0.05 \citep{2022MNRAS.514.6215Y}. This is very close to the theoretical injection spectral index \citep{2000ApJ...542..235K}.
Considering the source to have a spectral index in the range $-$0.54 to $-$0.64, the inverse Compton scattering of the CMB photons and radio photons can give rise to a power law X-ray spectrum whose photon index, $\Gamma$ can be 1.54 to 1.64. This is close to the value of $\Gamma$ = 1.67 found by \cite{2019ApJ...886..145K} from an analysis of XMM data in the 2$-$10 keV band.

The above considerations all show that the radio emission in the central 10 parsec region in NGC 4395 is from a low-power jet launched by an intermediate-mass black hole.

% Figure environment removed
\subsection{Warm ionized gas and shock} The availability of gas reservoir in the few tens of parsec in the central regions of AGN is an important ingredient in the feeding and feedback processes in them. In particular the presence of ionized gas in the central regions of AGN is believed to be a consequence of SF as well as AGN activity. Such ionised emission could also be produced by shock excitation.  The presence of such ionised gas is easily traced in the optical through emission lines and could trace the effect of AGN and the presence of outflows.  From recent IFU observations in the optical and infrared of the central 1$\times$1 square arcsec region, 
\citet{2019MNRAS.486..691B} suggest that these may be ionized by the AGN based on the location of these spatially resolved measurements in the BPT diagram \citep{1981PASP...93....5B}
(in the case of optical) and IR line ratio diagram (in the case of infrared). However, in the zoomed in version of the BPT diagram, the eastern component, the core and the western component nicely gets 
segregated (see Fig. \ref{figure-13}). It is thus likely (similar to that seen in a nearby AGN NGC~1068 by \citealt{2019MNRAS.487.4153D}), the emission in the spaxels within the central 1$\times$1 square arcsec region could have contribution from AGN, as well as shock.
%We show here that shock are playing an important role in NGC~4395, consistent with the jet-ISM interaction.

Outflows can have multiple constituents, such as the hot ionized gas produced at the shock front as well as neutral and molecular gas entrained in the flow. Shocks produced by AGN-driven outflows and/or radio jet-ISM interaction could 
also provide the possibility of energetic feedback altering the SF 
characteristics of the ISM. We consider here the possibility of the shock 
leading to the observed morphology of the ionized [OIII]$\lambda$5007 
emission. Using the observed luminosity at 1.4 GHz, we calcualted 
the  jet power as
\begin{equation}
P_{jet} = 5.8 \times 10^{43} \left(\frac{L_{1.4GHz}}{10^{40} erg s^{-1}} \right)^{0.7}
\end{equation}
Using L$_{1.4 GHz}$ = (2.34$\pm$0.43)$\times$10$^{34}$ erg s$^{-1}$ we 
 estimated the jet power to be 
P$_{jet}$ = 6.63$\times$10$^{39}$ erg s$^{-1}$.
It is thus evident that the jet in NGC~4395 is weak compared to powerful radio 
galaxies \citep{10.1093/mnras/stw3330}. 

We calculated the mass of the outflowing ionised hydrogen from the measured 
luminosity of the H$\alpha$ emission using the following relation 
\citep{2017A&A...604A.101C}.
\begin{equation}
M_{ion}^{out} = 3.2 \times 10^5\left(\frac{L_{broad}(H\alpha)}{10^{40} erg s^{-1}}\right) \left(\frac{n_e}{100 cm^{-3}}\right)^{-1}
\end{equation}
By considering $F_{broad}({H\alpha}$) = 7.40$\times$10$^{-14}$ erg cm$^{-2}$ s$^{-1}$ (integrated flux density over a circular aperture of radius 0.4 arcsec 
on extinction corrected outflowing component of H$\alpha$ line image from 
GMOS), and mean electron density, $n_e$ = 1724 cm$^{-3}$, we obtained 
M$_{ion}$$^{out}$$\sim$ 652M$_{\odot}$. Using a $\sigma$ of 123 km s$^{-1}$ 
(median $\sigma$ of outflowing component of [OIII]$\lambda$5007 line), we 
calculated the kinetic energy of this ionised mass as $E_{KE} = M_{ion}^{out} 
(\sigma^2)$ = 1.97 $\times$ 10$^{50}$ erg. Taking a velocity of 9 km/s (median 
of velocity shift of outflowing component of [OIII]$\lambda$5007 line) and the 
projected distance of the tip of the eastern
jet as 0.3 arcsec (6.3 parsec), the time required to reach the terminal point 
is 2.16$\times$10$^{13}$ s. The power of the outflow is thus 
$P_{out} = E_{KE}/t$ = 9.14$\times$10$^{36}$ erg s$^{-1}$.


We calculated the mass and radius of the NLR using the following relations 
\citep{1997iagn.book.....P}.
\begin{equation}
M_{NLR} = 7 \times 10^5\left(\frac{L(H\beta)}{10^{41} erg s^{-1}}\right) \left(\frac{10^3 cm^{-3}}{n_e}\right) M_{\odot}
\end{equation}


\begin{equation}
R_{NLR} = 19 \left(\frac{L(H\beta)}{\epsilon 10^{41} erg s^{-1}}\right)^{1/3} \left(\frac{10^3 cm^{-3}}{n_e}\right)^{2/3} parsec 
\end{equation}

By considering $n_e$ of 1724 cm$^{-3}$ (obtained from GMOS observations, 
see Fig.~\ref{figure-7}) and assuming a filling factor ($\epsilon$) of 
10$^{-2}$ (typical upper limit;\citealt{1997iagn.book.....P}) we obtained 
mass and radius of the NLR of NGC~4395 as 282 $M_{\odot}$ and 5.35 parsec 
respectively over an circular region of 0.4 arcsec radius.

We calculated the bolometric 
luminosity (L$_{Bol}$) using the observed brightness in soft X-ray, hard X-ray 
and H$\alpha$. In the hard X-ray band (14$-$195 keV), using the logarithm of observed luminosity of 
40.797 \citep{2014ApJ...783..106L}, we obtained  L$_{Bol}$ = 
4.968$\times$10$^{41}$ erg s$^{-1}$ using the following relation \citep{2017ApJ...835...74I}:
\begin{equation}
log (L_{Bol}) = 0.0378 \times (log (L_X))^2 - 2.03 \times log (L_X) + 61.6.
\end{equation}

In the soft X-ray band (2$-$10 keV)  using the logarithm of the observed  luminosity of 
40.3 \citep{2011MNRAS.417.2571N}, we obtained a L$_{Bol}$ of 1.95$\times$10$^{41}$ erg s$^{-1}$ using the
relation given below:
\begin{equation}
log(L_{Bol}) = 0.0378 \times log(L_{2-10})^2 - 2.00 \times log(L_{2-10}) + 60.5.
\end{equation}


Similarly, from H$\alpha$ GMOS observations (considering a circular aperture of 0.31$^{\prime\prime}$) using a H$\alpha$  luminosity of
$5.43\times10^{38} erg s^{-1}$ , we obtain L$_{Bol}$ = 3.64 $\times$ 10$^{41}$ erg s$^{-1}$ using the equation given 
below \citep{2007ApJ...670...92G}.
\begin{equation}
L_{Bol} = 2.34 \times 10^{44} \times (L_{H\alpha}/10^{42} erg s^{-1})^{0.86}
\end{equation}

Thus, from optical and X-ray observations,  we find the source to have a
bolometric luminosity in the range of 
1.95$-$4.97$\times$10$^{41}$  erg s$^{-1}$.

The disk accretion rate
is generally represented by the Eddington ratio ($\lambda_{Edd}$) and is
defined as

\begin{equation}
\lambda_{Edd} = L_{Bol}/L_{Edd}
\end{equation}
Here, L$_{Edd}$ is the Eddington luminosity defined as 


\begin{equation}
L_{Edd} = 1.22 \times 10^{38}\left(\frac{M_{BH}} {M_{\odot}}\right) erg s^{-1}
\end{equation}


Using L$_{Bol}$ of (1.95$-$4.97)$\times$10$^{41}$ erg s$^{-1}$  and M$_{BH}$ 
values of (9.1$\times$10$^3$ - 3.6$\times$10$^5$) M$_{\odot}$ we obtain 
$\lambda_{Edd}$ values of 0.004 to 0.044.

Given the jet power and the bolometric luminosity to be larger than the power 
of the outflowing ionized emission, the outflow seen in this source on the 
scale of the NLR of the source could be because of either jet-mode or 
radiative mode process. The optical spectrum from MaNGA for the central region 
encompassing the compete core-jet structure having an angular size of 0.5 
arcsec, shows the presence of the [OIII]$\lambda$4363 and HeII$\lambda$4686 
lines (see Fig. \ref{figure-9}). The logarithm of the ratio 
between [OIII]$\lambda$4363 and [OIII]$\lambda$5007 lines is $-$1.6; 
HeII$\lambda$4686 and H$\beta$ ratio is $-$0.72 and [OIII]$\lambda$5007 and 
H$\beta$ is 0.86. These line ratios point to the presence of shock 
(\citealt{2017ApJ...847...41C}; \citealt{2002A&A...383...46M}).% in the central region of NGC~4395.

%From the [OIII]$\lambda$4363/[OIII]$\lambda$5007 ration, we obtained a temperature of the emitting gas as 1.6$\times$10$^4$ K, which again indicates the schock ionization \citep{2002A&A...383...46M}.
A comparison of emission line ratios obtained from photoionization and shock 
modelling and observed line ratios also indicate the gas in the central regions 
of NGC~4395 to be ionised by shock  (see Fig.~\ref{figure-14}). Assuming a spectral index ($\alpha$) of $-$0.64 \citep{2022MNRAS.514.6215Y} for the eastern jet component we derived a Mach number $\left(M_s= \sqrt{\frac{2\alpha -3}{2\alpha + 1}}\right)$ \citep{2021ApJ...916..102A} of the shock as $M_s$=3.91. In the line ratios diagnostic diagrams, such as the [OIII]$\lambda$5007/H$\beta$ versus [SII]$\lambda$6717,6731/H$\alpha$ and [OIII]$\lambda$5007/H$\beta$ versus [NII]$\lambda$6584/H$\alpha$ diagrams, though all the spaxels lie in the region occupied by AGN, the structure is clearly delineated (see Fig. \ref{figure-13}). Also, in
the asymmetry of the line versus the velocity dispersion diagram (see Fig. \ref{figure-16}; left panel), the spaxels in the eastern jet, occupy a region of higher line asymmetry and
higher velocity dispersion, while the western jet occupies a region of lower asymmetry index and lower velocity dispersion. High velocity dispersion and high asymmetry of the lines are attributed to shock excitation \citep{2019MNRAS.485L..38D}. The eastern
jet thus seems to occupy a region that is dominated by shock excitation , while the western jet seems to occupy a region of weaker shocks.
%Also, (a) the electron temperature derived from [OIII] and [NII] line intensity ratios from low resolution MaNGA spectra, (b) the position of NGC~4395 in the $T_{e[OIII]}$ v/s $T_{e[NII]}$ plane and (c) the spatial variation of $T_e$ in the spatially resolved $T_e$ map derived from GMOS spectra (see Section 2.3 in Methods) argues for shock playing a role in the ionisation. 
Pixels in the cental 1$\times$1 square arcsec region show a tight correlation between the velocity dispersion and the shock sensitive line ratio [NII]/H$\alpha$ (see Fig. \ref{figure-16}; right panel) (\citealt{10.1093/mnras/stu1653}). Shock models predict an increase in [NII]/H$\alpha$ with an increase in shock velocity \citep{2010A&A...519A..40A}.
%This is evident in the plot of the spaxels in the central 1 $\times$ 1 square arcsec in the W90 versus [NII]/H$\alpha$ plane (Fig \ref{figure-3}; right panel).
We show in Fig. \ref{figure-17} the position of NGC 4395 in the $T_{e[NII]}$ versus $T_{e[OIII]}$ diagram estimated from MaNGA spectrum. In the same diagram there are measurements for few AGN along with predictions from AGN photoionisation from CLOUDY. AGN 4395 lies in a distinct position in this Figure, pointing to such high temperatures being produced by shocks.

Photoinoization modelling by CLOUDY, shock models from MAPPINGS III, the electron temperature distribution and disturb kinematics point to the gas in the central region of NGC 4935, excited by shocks.
From a multitude of arguments, we conclude shocks contributing to the excitation of the gas and such shock could be due to the interaction of the jet with the ISM in the central 10 parsec region of NGC~4395.
%\textbf{ Along with that, the spaxels of eastern side of jet are having high asymmetry and high velocity dispersion along with high shock sensitive line ratios (see Fig. \ref{figure-12}) compared to western region and the pixels are following the positive trend. This is again indicating that the shock at the eastern side is more than the western side of the core.}


%The presence of such outflows on kpc scales are
%known in AGN (reference). However, in the case of AGN hosted in dwarf galaxies, results on  the impact
%AGN have on their host is very limited, however, recent studies point to significant progress.
%In a study of a sample of  dwarf galaxies having AGN,  using long slit spectroscopy, 
%\cite{2019ApJ...884...54M} have found evidence of outflows in few sources. In some of these
%sources with outflows, it is found that these outflows are photo-ionized by AGN 
%\citep{2020ApJ...905..166L,2021ApJ...911...70B}. It has also been found that radio jets in 
%dwarf AGN, have jet efficiecies similar to that of their massive galaxy counterparts
%\citep{2019MNRAS.488..685M}. HST observations (xx) have shown the presence of a star cluster.
%With the new position of the radio core from {\it Gaia}, the star cluster is at a distance
%of xxx from the core. The lack of any strong absorption lines in optical IFU spectrum of the central
%region both from GMOS and MANGA rule out the star cluster playing a role in ionising the 
%gas in the central region.

%In Fig. \ref{figure-2}, we show the radio contour overlaid on the [OIII]$\lambda$5007 emission 
%line image from HST. On the same Figure in the right hand panel we show the 
%radio image in grey with the [OIII]$\lambda$5007 emission in contours.
%In the HST images, the [OIII]$\lambda$5007 emission clearly shows resolved structures such as
%(a) the terminal points of both the eastern and western [OIII]$\lambda$5007 show a convex structure resembling
%bow shock  typically seen in the lobes of FRII radio galaxies and (b) the [OIII]$\lambda$5007 contours are closely
%spaced (more evident in the eastern region) and bent suggestive of the plasma being pushed outward 
%possibly by the advancing low power jet. The radio and [OIII]$\lambda$5007 observations thus seem to provide a clear
%evidence of an interplay between the low power jet and the plasma already photo-ionised by the AGN.

%\subsection{Jet-ISM interaction}

%% Figure environment removed

% Figure environment removed

% Figure environment removed


% Figure environment removed

%\noindent {\bf Warm ionized gas and shock :} 
%\subsection{Feedback at small scales}


\subsection{A radio jet-ISM interaction on 10 parsec scale in NGC~4395}
From an analysis of data in the optical, infrared, radio and sub-mm, we have 
found evidence of a low-luminosity jet interacting with its host on the scale 
of about 10 parsec. The eastern jet component which is brighter in the radio band, 
is resolved in the high-resolution High Sensitive Array (HSA) image into two 
components oriented approximately in the North-South direction, which is nearly 
orthogonal to the source axis \citep{2006ApJ...646L..95W}. This indicates 
interaction of the jet plasma with the ISM, with the plasma following the path 
of least resistance. On the eastern side, the [OIII]$\lambda$5007 line-emitting 
gas has higher velocity, higher velocity dispersion and higher asymmetry (see 
Fig. \ref{figure-16}, left panel), possibly due to shock associated with the interaction of the radio plasma with the [OIII]$\lambda$5007 gas. The weakness of the [OIII]$\lambda$5007 emission can either be due to the gas being more ionized or larger extinction, E(B-V) (see Fig. \ref{figure-7}) or a combination of both. The weaker jet on the western side has a smaller effect on the [OIII]$\lambda$5007 gas with a lower velocity dispersion and asymmetry. This suggests that there may be an intrinsic asymmetry in the oppositely-directed jets. The radio emission is found to exist co-spatially with the emission at other wavelengths such as the hot ionised [OIII]$\lambda$5007 emission in the optical
band, the warm molecular H2$\lambda$2.4085 in the infrared band and the cold 237 GHz emission in the sub-mm band, however, the cold CO(2-1)
emission is displayed by $\approx$1 arcsec from the core. The radio jet has possibly displaced the CO(2-1) emission leading to conditions less favourable
for SF at 10 parsec. A schematic of our proposed coherent picture of the central region of NGC 4395 is shown in Fig. \ref{figure-18}

\section{Summary}
In this work we carried out a systematic investigation of the central
region of NGC 4395 using imaging and spatially resolved spectroscopic
observations. We summarize our main findings below:
\begin{enumerate}
\item From VLA images at 15 GHz, NGC 4395 is found to show a triple radio
structure having a projected size of $\sim$10 parsec. The weak central component
of the triple structure is found to coincide with the optical 
{\it Gaia}
position which we call as the radio core. The source is also highly asymmetric
in brightness with the eastern component more brighter than the western
component.
\item The triple radio structure in NGC 4395 is reminiscent of bipolar
jet ejection in radio-loud AGN and the eastern and western components of
this triple structure are formed by the low power jet 
(P$_{jet}$ = 6.63 $\times$ 10$^{39}$ erg s$^{-1}$) powered by the
intermediate mass black hole in NGC 4395.
\item From HST observations we found the [OIII]$\lambda$5007 emission to be 
prevalent over the entire extent of the radio emission with the peak of the 
[OIII]$\lambda$5007 emission
coinciding with the {\it Gaia} position and the radio core.
The [OIII]$\lambda$5007 emission is asymmetric and shows a convex-shaped structure at the
terminal points on either side of the core of NGC 4395 indicating an outflow. 
This asymmetry in the brightness of [OIII]$\lambda$5007 emission could be due to the low power
radio jets propagating in an inhomogeneous medium.
\item The peak of the X-ray emission in the 0.5$-$7 keV band is found to 
coincide with the radio core and is also extended along the radio jet. 
Similarly, the peak of the continuum emission at 237 GHz is spatially 
coincident with the radio core. Also, the molecular H2$\lambda$2.4085 is found 
to be extended along the radio jet direction and having close correspondence 
with the radio emission.
\item From photoionization modelling by CLOUDY, shock modelling from 
MAPPINGS III and the distribution of the electron temperature distribution, 
we conclude that the gas in the central region of NGC 4395 is excited by 
shocks and such shocks could be due to the interaction of radio jet with the 
ISM in the central parsec region of NGC 4395. This is the first detection of 
radio jet - ISM interaction at such small spatial scales.
\item The cold CO (2-1) emission is found to be displaced northwards of the
radio core by about 1 arcsec. Such displacement of the cold gas by the
radio jet naturally leads to conditions less favourable for star formation
at scales of about 5 parsec in NGC 4395.
\end{enumerate}

The detection of AGN and intermediate mass black holes in a number of dwarf galaxies in recent years have opened the possibility of studying feedback processes in dwarf galaxies. Studies of nearby dwarfs also enable us to probe feedback processes on parsec scales. \citet{2022Natur.601..329S} reported a 150 parsec long ionized filament in the dwarf galaxy Henize 2-10 from HST observations, which connect the black hole region with a site of recent SF. \citet{2017ApJ...845...50N} report possible evidence of shock excitation in the nearby dwarf AGN galaxy NGC~404 with an amorphous radio outflow extending over $\approx$17 parsec. NGC~4395 is the clearest example of a dwarf AGN with a triple radio structure, where there is clear evidence of jet-ISM interaction on the smallest scale of $\approx$4.6 parsec. 
This finding will bolster the prospect of finding more such instances in dwarf AGN host galaxies,  paving the way for a better understanding of the complex interplay between AGN and their hosts on such small scales in these galaxies. 

%\begin{enumerate}
%\item From 15 GHz VLA data, we found the source to be resolved, consisting of a
%radio core that coincides with the optical {\it Gaia} position and two jets
%oriented on the east west direction
%\item The eastern jet is brighter and is at a distance of xxx parsec from the core, 
%The western jet at a distance of xx from the core is relatively fainter than its
%eastern counterpart. The total extent of the radio source is xxx parsec. The 
%asymmetry of the jets both in terms of brightness and
%morphology indicates the interaction of the jet with the ISM 
%\item NGC 4395 is a radio-quiet AGN, however, 15 GHz observations shows the 
%source to have a core dominated structure
%\item Ionised [OIII[ emission is found to be spatially coincident wiht the terminal
%points of the radio jets. The structural alignment and the shape (convex at the terminal point, 
%and bent at the starting point close to the core)  of the ionizing [OIII]$\lambda$5007
%emission indicates the presence of shock . Such signature of shock is also supported from
%the [OIII]$\lambda$5007 line seen in the MANGA spectrum of the central region. shock  are also known 
%to contribute from about 20-40\% to the ionization seen in [OIII]$\lambda$5007
%\item The lack of CO, presence of H$_2$ and [OIII]$\lambda$5007 along the jet direction of NGC 4395
%could favour inhibition of SF and it the first evidence of feeback processes
%operating at very small scales of 10 parsec 
%\end{enumerate}

\vskip12pt
\newpage










%The source was observed with VLA A-configuration at 15 GHz (PI: Payaswini Saikia, Legacy ID: AS1409). The data was reduced using standard procedures using CASA (see \citealt{2018A&A...616A.152S} for details). The synthesized beam size obtained is 0.129$^{\prime\prime}$ $\times$ 0.124$^{\prime\prime}$ and PA -17.89$^\circ$. We achieved rms flux 11.2 $\mu$Jy. The reduced image is shown in lower left panel of Fig. \ref{figure-1}.

%The source was also observed at C band in VLA B-configuration (PI:J.S. Ulvestad, Legacy ID: AU079). We reduced the object using standard procedures in AIPS (version: 31DEC21)  by using 3C286 as flux calibrator and 1227+365 as phase calibrator. We achieved a rms flux of 47.6 $\mu$Jy. The synthesized beam size obtained for this observation is 1.75$^{\prime\prime}$ $\times$ 1.19$^{\prime\prime}$ with PA 89.14$^\circ$.
%% Figure environment removed

%\begin{equation}
%M_{H2} = 5.0776 \times 10^{13} \left(\frac{F(H2)}{erg s^{-1} cm^{-2}}\right) \left(\frac{D}{Mpc}\right)^2
%\end{equation}

%\begin{equation}
%M_{CO} = 
%\end{equation}

%\subsubsection{Electron temperature distribution}
%Knowledge of the electron temperature ($T_e$) in the central regions of AGN can help one to constrain the contribution of AGN to gas ionization.  shock from AGN outflows could produce higher values of  $T_e$ \citep{2021MNRAS.501L..54R}. We calculated the integrated $T_e$ using two line intensity ratios namely R$_{O3}$ = ([OIII]$\lambda\lambda$ 4959,5007)/$\lambda$4363 and R$_{N2}$ = ([NII]$\lambda\lambda$6548,6584)/$\lambda$5755 from MaNGA spectra and adopting the following relations (\citealt{2021MNRAS.501L..54R}; \citealt{10.1093/mnras/staa1781}).
%\begin{equation}
%\frac{T_{e[OIII]}}{10^4 K} = 0.8254 - 0.0002415R_{O3} + \frac{47.77}{R_{O3}}
%\end{equation}
%\begin{equation}
%\frac{T_{e[NII]}}{10^4 K} = 0.537 + 0.000253 \times R_{N2} + \frac{42.13}{R_{N2}}
%\end{equation}
%We found $T_{e[NII]}$ = 16420 K and $T_{e[OIII]}$ = 16818 K. These values are too large to be produced solely by AGN photo-ionizaion.
%We show in the right panel of Fig.~\ref{fig-temperature} the position of NGC~4395 in the $T_{e[NII]}$ v/s $T_{e[OIII]}$ diagram. Also shown in the same diagram are measurements from few AGN along with predictions from AGN photoionization from CLOUDY. NGC~4395 lies in a distinct position in this Figure, pointing to such high temperatures being produced by shock from AGN jet.}

%To better characterise the spatial nature of $T_e$, we used [NII] lines from GMOS spectra to generate a spatially resolved map of $T_e$. Since [OIII]$\lambda$4363 is not covered by the GOMS spectra we used the line ratio R$_{N2}$ to generate the $T_e$ map. For this we considered only those spaxels where the S/N ratio (ratio of the peak of the [NII]$\lambda$5755 line to the standard deviation of the pixels in the adjacent continuum) is greater than 30. The $T_e$ map is shown in Fig.~\ref{fig-temperature}. We found $T_e$ to have a range of values, with the value increasing from the center of NGC~4395 outwards, both towards the eastern and western terminal points of the radio jet. This increase of temperature towards the eastern and western side is evident in the temperature difference map shown in the right panel of Fig~\ref{fig-temperature}. This temperature difference map is generated by subtracting each temperature value from the mean of the temperature calculated over the central 0.05$\times$0.05 square arcsec region. The increase of temperature from the center of NGC~4395 towards the edges coinciding with the radio jet points to the gas being ionised by shock. shock could be produced by the interaction of radio jet with the ISM, and this increase of $T_e$ from the center towards the edges is a direct evidence of shock ionisation \citep{2021MNRAS.501L..54R}.


%\subsection{\textbf{Nature of radio emission in the central 10 pc region}}
%The radio emission observed in the central parcsec-scale region of a dwarf galaxy could be from a variety of physical processes, such as low-power jets, AGN-driven wind, SF, coronal activity and free-free emission from thermal gas \citep{2019NatAs...3..387P}. Radio structure, spectral index, polarization characteristics of the radio emission if detectable in future, spatial correlation of radio structure and different gasesous components, spectral line diagnostics of the different components of the ISM, could provide valuable clues in identifying the dominant processes. 
%In NGC~4395, radio morphology of a triple radio source clearly showing signs of interaction prominently on the eastern side with the emission following the line of least resistance clearly indicated jet-ISM interaction. This was reinforced from a detailed study of the line-emitting gas. 
%The [OIII]$\lambda$5007 emission appears closely associated with the radio source, ionized by shock as the jets flow outwards. Line-ratio diagnostics, estimation of gas temperatures from line ratios, and comparison of line ratios with theoretical predictions using the CLOUDY and MAPPINGS models, all showed shock to be the dominant process responsible for the ionization. The eastern component which is more prominent at radio wavelengths showed stronger signs of interaction with the ISM than the western component.
%
%The spectral index derived over the 1.4 $-$ 15 GHz range gives a value of $\alpha$ = $-$0.54$\pm$0.34 (S$_\nu$ $\propto$ $\nu^{\alpha}$). However, considering only the eastern jet/component, $\alpha$ = $-$0.64$\pm$0.05 \citep{2022MNRAS.514.6215Y}. This is very close to the theoretical injection spectral index \citep{2000ApJ...542..235K}.
%Considering the source to have a spectral index in the range $-$0.54 to $-$0.64, the inverse Compton scattering of the CMB photons and radio photons can give rise to a power law X-ray spectrum whose photon index, $\Gamma$ can be 1.54 to 1.64. This is close to the value of $\Gamma$ = 1.67 found by \cite{2019ApJ...886..145K} from an analysis of XMM data in the 2$-$10 keV band. 

%The above considerations all show that the radio emission in the central 10 parsec region in NGC 4395 is from a low-power jet launched by an intermediate-mass black hole.


%\subsection{Asymmetry analysis}
%We show in Fig. \ref{figure-3} (left panel), the position of the spaxels in the asymmetry indexversus W90 parameter plane. 
%The asymmetry index was derived as \textbf{xxx}, while W90, parameter that corresponds to the velocity dispersion was derived as xxx.
%In this diagram, the nucleus of NGC~4395, the eastern jet, and the western jet are well separated. The spaxels in the eastern jet have high asymmetry and high velocity dispersion, while the spaxels in the western jet have low asymmetry index and low velocity dispersion. High velocity dispersion and high asymmetry of the lines are attributed to shock excitation \citep{2019MNRAS.485L..38D}. In this diagnostic diagram, the eastern jet seems to occupy a region that is dominated by shock excitation process, and  the western jet seems to occupy a region of weaker shock. 



%% Figure environment removed

%% Figure environment removed


%% Figure environment removed


%% For this sample we use BibTeX plus aasjournals.bst to generate the
%% the bibliography. The sample631.bib file was populated from ADS. To
%% get the citations to show in the compiled file do the following:
%%
%% pdflatex sample631.tex
%% bibtext sample631
%% pdflatex sample631.tex
%% pdflatex sample631.tex

\section*{Acknowledgments}
This publication uses the data from the AstroSat mission of the Indian Space
Research Organisation (ISRO), archived at the Indian Space Science Data Centre
(ISSDC). This publication uses UVIT data processed by the payload operations
centre at IIA (Indian Institute of Astrophysics). The UVIT is built in
collaboration between IIA, IUCAA (Inter University Center for Astronomy and 
Astrophysics), TIFR (Tata Institute of Fundamental Research), ISRO and
CSA (Canadian Space Agency).  This work has made use of the NASA Astrophysics 
Data System (ADS)\footnote{https://ui.adsabs.harvard.edu/} and the NASA/IPAC 
extragalactic database (NED)\footnote{https://ned.ipac.caltech.edu}. 
This work has made
use of data from the European Space Agency (ESA)
mission Gaia (https://www.cosmos.esa.int/gaia), processed by the Gaia Data Processing and Analysis
Consortium (DPAC, https://www.cosmos.esa.int/web/
gaia/dpac/consortium). Funding for the DPAC has
been provided by national institutions, in particular,
the institutions participating in the Gaia Multilateral
Agreement. A few of the authors thank the Alexander von Humboldt Foundation, 
Germany, for the award of the Group Linkage long-term research program. 
PN thanks the Council of Scientific and Industrial Research (CSIR), Government 
of India, for supporting her research under the CSIR Junior/Senior research 
fellowship program.
\software{IRAF \citep{1986SPIE..627..733T}, Astropy \citep{2013A&A...558A..33A}, 
Scipy \citep{2020SciPy-NMeth}, Numpy \citep{harris2020array}, Matplotlib \citep{Hunter:2007}, 
AIPS \citep{1985daa..conf..195W}, CASA \citep{2007ASPC..376..127M}, 
Chandra \citep{2006SPIE.6270E..1VF}
}


\bibliography{ref1}{}
\bibliographystyle{aasjournal}
\end{document}

