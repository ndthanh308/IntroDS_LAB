\documentclass{article}
\usepackage[a4paper, total={6in, 8in}]{geometry}
\usepackage[utf8]{inputenc}
\usepackage[T1]{fontenc}
\usepackage{color}
\usepackage{amsmath,bm}
\usepackage{amsthm}
\usepackage{amssymb}
\usepackage{graphicx}
% \usepackage{subfig}
\usepackage{subcaption}
\usepackage{multirow}
\usepackage{xcolor}
\usepackage{comment}
\usepackage{hyperref}
\usepackage{enumerate}
\makeatletter
\usepackage{pifont}% http://ctan.org/pkg/pifont
\newcommand{\cmark}{\ding{51}}%
\newcommand{\xmark}{\ding{55}}
\newtheorem{assumption}{Assumption}
%%%%%%%%%%%%%%%%%%%%%%%%%%%%%% Textclass specific LaTeX commands.
\theoremstyle{plain}
\newtheorem{prop}{\protect\propositionname}
\theoremstyle{plain}
\newtheorem{thm}{\protect\theoremname}
\newtheorem{remark}{Remark}
\newcommand*{\Scale}[2][4]{\scalebox{#1}{$#2$}}%

%%%%%%%%%%%%%%%%%%%%%%%%%%%%%% User specified LaTeX commands.

% for subfigures/subtables

\usepackage{algorithm}
%\usepackage{algpseudocode}
\usepackage{cite}
%\usepackage{times}
%\usepackage{URL}
%\usepackage{hyperref}
%\newcommand{\singlespace}{\addtolength{\baselineskip}{2\baselineskip}}
%\singlespace
\special{papersize=8.5in, 11in}

%\usepackage{setspace}
%\doublespacing
%\pagestyle{empty}
%\newtheorem{assumption}{Assumption}
%\newtheorem{challenge}{Challenge}
\newtheorem{condition}{Condition}
%\usepackage{subfig}
%\def\qed{\hfill \vrule height 6pt width 6pt depth 0pt}
\usepackage{amsmath}
%\usepackage{amsmath}
%\usepackage{boondox-cal}
%\usepackage[cal=boondoxo]{mathalfa}
\usepackage{algorithmic}

\makeatother

\newtheorem{lem}[thm]{\protect\lemmaname}
\newtheorem{corollary}{Corollary}[thm]

\newtheorem{rem}{\protect\remarkname}
%%%%%%%%%%%%%%%%%%%%%%%%%%%%%% User specified LaTeX commands.
\renewcommand{\labelenumi}{\alph{enumi})}



\providecommand{\claimname}{Claim}
\providecommand{\definitionname}{Definition}
\providecommand{\lemmaname}{Lemma}
\providecommand{\propositionname}{Proposition}
\providecommand{\remarkname}{Remark}
\providecommand{\theoremname}{Theorem}

\providecommand{\propositionname}{Proposition}
\providecommand{\remarkname}{Remark}
\providecommand{\theoremname}{Theorem}

\newcommand{\vb}{\mathbf{v}}
\newcommand{\xb}{\mathbf{x}}
%\newcommand{\1b}{\textbf{1}}
%\newcommand{\Ib}{\textbf{I}}
\newcommand{\yb}{\mathbf{y}}
\newcommand{\Q}{\textbf{Q}'}
\newcommand{\I}{\bar{I}}
\newcommand{\ka}{\tau}
\newcommand{\ta}{\tau}
\newcommand{\la}{\bar{\lambda}}
\newcommand{\1}{1}
\newcommand{\n}{\nabla}
\newcommand{\x}{\bar{x}}
\newcommand{\EX}{\mathbb{E}}
\newcommand{\dx}{\Delta x}
\newcommand{\dd}{\Delta x^*}
\newcommand{\E}{\tilde{E}}
\newcommand{\ep}{\epsilon}
\newcommand{\xx}{\mathbf{\bar{x}}}
\newcommand{\N}{\mathbb{N}}
\newcommand{\defeq}{\stackrel{\rm def}{=}}
\newcommand{\II}{\textbf{I}_n}
\newcommand{\III}{\bar{\textbf{I}}_n}
\newcommand{\Eb}{\textbf{E}}
\newcommand{\g}{\textbf{g}}
\newcommand{\ddd}{\textbf{d}}
\newcommand{\ty}{\texttt}
\newcommand{\rb}[1]{{\color{black}#1}}
\newcommand{\rbnote}[1]{{\color{red}#1}}
\newcommand{\rbfix}[1]{{\color{magenta}#1}}
\usepackage{authblk}
\usepackage{hyperref}
%\title{Optimal Convergence Rates for Stochastic Decentralized Optimization with Inexact Communication}
\title{A Stochastic Gradient Tracking Algorithm  for Decentralized Optimization With Inexact Communication}
%\title{A Stochastic Gradient Tracking Algorithm With Optimal Convergence Rate for Decentralized Optimization With Inexact Communication}
\author{Suhail M. Shah$^*$  \qquad \text{  }\qquad\qquad   Raghu Bollapragada\thanks{University of Texas at Austin. \texttt{(\href{mailto:suhail.mohmad@utexas.edu}{suhail.mohmad@utexas.edu}, \href{mailto:raghu.bollapragada@utexas.edu}{raghu.bollapragada@utexas.edu})}}}




\begin{document}
\maketitle
\begin{abstract}
Decentralized optimization is typically studied under the assumption of noise-free transmission. However, real-world scenarios often involve the presence of noise due to factors such as additive white Gaussian noise channels or probabilistic quantization of transmitted data.
%However, in practical scenarios, various factors can introduce noise into the system, such as the utilization of additive white Gaussian noise channels or probabilistic quantization of transmitted data. 
These sources of noise have the potential to degrade the performance of decentralized optimization algorithms if not effectively addressed. In this paper, we focus on the noisy communication setting and propose an algorithm that bridges the performance gap caused by communication noise while also mitigating other challenges like data heterogeneity.
We establish theoretical results of the proposed algorithm that quantify the effect of communication noise and gradient noise on the performance of the algorithm. 
%A detailed theoretical analysis of the proposed algorithm is performed to understand the impact of communication noise and gradient noise on the performance. 
Notably, our algorithm achieves the optimal convergence rate for minimizing strongly convex, smooth functions in the context of inexact communication and stochastic gradients. Finally, we illustrate the superior performance of the proposed algorithm compared to its state-of-the-art counterparts on machine learning problems using MNIST and CIFAR-10 datasets.
%An empirical demonstration of the superior performance of the proposed algorithm compared to its state-of-the-art counterparts is demonstrated on practical datasets such as MNIST and CIFAR-10.

\end{abstract}
\section{Introduction}
The seminal works \cite{tsit1,tsit2}  were one of the earliest works to formally study the problem of decentralized decision making and optimization. These works helped launch the field of decentralized optimization, where a connected network of multi agents collectively optimize an objective function by only exchanging information between neighboring agents in the network. 
%communicating along the neighbors in the network}.
%The seminal works \cite{tsit1,tsit2}  were one of the earliest to formally study the problem of decentralized decision making and optimization which helped launch the field of decentralized optimization \rb{where a connected network of multi agents collectively optimize an objective function}.  
Over the last four decades, this area has intermittently experienced phases of extensive research activity with the current iteration being mainly spurred by machine learning (ML) based optimization on decentralized data among other applications. Adding a distributed component to an optimization algorithm for ML naturally lends itself to several advantages over its centralized counterparts such as data privacy and fault tolerance while improving scalability with problem size. Formally, the problem of decentralized optimization in its most succinct form can be stated as:
\begin{align}\label{mainprob}
\min_{x_i \in \mathbb{R}^d} \,&\textbf{f}(\textbf{x}) \defeq \frac{1}{n} \sum_{i=1}^n f_i(x_i)\nonumber\\
\text{s.t.} \, & x_i =x_j,\,\forall\, i,j \in \{1,2,\cdots,n\}
\end{align}
where $\textbf{x} := (x_1, \cdots,x_n) \in \mathbb{R}^{nd} $ with $x_i$ being the copy of the optimization variable held by the $i$th node (agent) of a network and $f_i:\mathbb{R}^d \to \mathbb{R}$ is the expected value $ f_i(.) = \EX_{\xi_i}\, [  F_i(.,\xi_i)]$ of the stochastic function $F_i(.,\xi_i):\mathbb{R}^d \to \mathbb{R}$ %a function 
private to node $i$. Problems of this nature arise in several applications with a prominent example being machine learning, where the $f_i$ is a function of the data held at node $i$.

A key aspect of decentralized algorithms is the need for communication between nodes to achieve consensus ($x_i =x_j,\,\forall\, i,j \in [n]:=\{1,2,\cdots,n\}$). However, this communication is typically not noise-free, and any form of inexactness in the algorithm can potentially degrade its performance if not addressed properly. Even %basic 
fundamental algorithms like decentralized gradient descent (DGD) do not %provide 
possess convergence guarantees or assured performance in the presence of inexact communication \cite[Theorem III.8]{near} or, Section IV, \textit{ibid}. Therefore, it is essential to develop a framework that incorporates inexact communication to design algorithms that effectively mitigate its adverse effects.
%Therefore, developing a framework that accounts for inexact communication becomes crucial in order to design effective algorithms that mitigate its adverse effects.

Data heterogeneity poses another challenge in decentralized optimization where the training data is decentralized over the nodes or generated on client devices so that each node has only access to $f_i(\cdot)$.  %where each node has access only to its local function $f_i(\cdot)$ due to distributed or client-generated training data.}
%Another challenging aspect of decentralized optimization is known to be data-heterogeneity. This essentially means that the training data is decentralized over the nodes or generated on client devices so that each node has only access to $f_i(\cdot)$. 
Fundamental algorithms such as stochastic decentralized  gradient descent (\texttt{S-DGD}), used to solve (\ref{mainprob}) are adversely affected by data heterogeneity \cite{sgt1}. To overcome these limitations, Gradient Tracking (\texttt{GT}) type methods \cite{EXTRA,DIG} have been developed which communicate an additional vector that tracks the gradient of the global objective function. However, any inexactness in the communication can again severely degrade the overall performance \cite{near, yuan2}. In fact, with quantization, \texttt{GT} can empirically show divergent behaviour \cite[Section IV]{near}. 
%Even the %most basic 
%\rb{fundamental} algorithms, such as stochastic decentralized  gradient descent (\texttt{S-DGD}), used to solve (\ref{mainprob}) can be adversely affected by data heterogenity \cite{sgt1}. Many important techniques such as Gradient Tracking (\texttt{GT}) require communication amongst the nodes with any inexactness again severely desuhgrading their performance \cite{near, yuan2}. In fact, with quantization, \texttt{GT} can empirically show divergent behaviour \cite[Section IV]{near}. 



In this paper we consider the question of whether the inadequacies in performance resulting from inexact communication in decentralized algorithms can be properly
addressed while retaining the benefits such as achieving consensus or removing data heterogeneity dependence. %To put things concisely, 
Specifically, our focus %will be centered 
is on designing and analyzing algorithms based on the \texttt{GT} strategy in the setting where the information, which could be the current iterate or the gradient tracking vector, is corrupted by additive zero-mean noise with finite variance.
%which may be the current iterate or the gradient, being communicated between the nodes is contaminated by additive zero mean, finite variance noise. 



%The rest of the paper is organized as follows: In the next two sub-sections we discuss the related work and the contributions of this paper. Section \ref{sec:mod} describes the network model. In Section \ref{sec:algo}, we present the proposed algorithm and its implementation. Section \ref{sec:conv} provides the convergence analysis while Section \ref{sec:num} presents the numerical evidence in its support. Future directions of research and conclusions are listed in Section 6. 
\begin{table}\label{tbt}
\footnotesize
\caption{\small{Comparison of convergence rates for strongly convex, smooth functions with stochastic gradients/communication noise for related works.} }
\centering
\begin{tabular}{ c  c  c c  }
\hline 
\hline
Reference     & Grad. Noise & Comm. Noise &  No. of iterations to $\epsilon$-acc.\\
\hline 
\hline
\cite{DIG,gt,sgt1} - \texttt{Gradient Tracking (GT)} &\xmark&\xmark& $\mathcal{O}\Big(\frac{L}{\mu\ka} \log \frac{1}{\epsilon}\Big)$\\
\hline
\cite{sgt2} - \texttt{Stochastic  DGD} &\checkmark&\xmark& $\mathcal{O}\Big(\frac{1}{n \mu }\frac{\sigma_g^2}{\epsilon}+\frac{\sqrt{L}}{\mu\ka} \frac{\sigma_g}{\sqrt{\epsilon}}   +\frac{\sqrt{L}}{\mu\ka } \frac{\chi^2}{\sqrt{\epsilon}}+ \frac{L}{\mu\ka} \log \frac{1}{\epsilon} \Big)$\\
  \hline
\cite{pudi,sgt1} -\texttt{ Stochastic GT}  &\checkmark&\xmark& $\mathcal{O}\Big(\frac{1}{n \mu }\frac{\sigma_g^2}{\epsilon}+\frac{\sqrt{L}}{\mu\ka} \frac{\sigma_g}{\sqrt{\epsilon}}  + \frac{L}{\mu\ka} \log \frac{1}{\epsilon} \Big)$\\
  \hline

\cite{com1} - \texttt{QDGD} &\xmark&\checkmark& $\mathcal{O}\Big(\frac{L^2}{\mu^2\ka} \frac{n\chi^2}{\epsilon^{2}}+\frac{L^2}{\mu^2\ka} \frac{n\sigma_c^2}{\epsilon^{2}}\Big)$\\
  \hline
\cite{do4} - \texttt{S-Near DGD}$^t$  &\checkmark&\checkmark & Non convergent. \\

% & \\
  \hline
  \hline
\texttt{This work,} (\texttt{IC-GT}) &\checkmark&\checkmark& $\mathcal{O}\Big(\frac{1}{n \mu }\frac{\sigma_g^2}{\epsilon}+\frac{\sqrt{L}}{\mu\ka} \frac{\sigma_g}{\sqrt{\epsilon}}  +  \frac{1}{\mu\ka}  \frac{\sigma^2_c}{   \epsilon} +\frac{L}{\mu\ka} \log \frac{1}{\epsilon}\Big)$\\
  \hline
  \hline
\end{tabular}
    \caption*{\\\small{ Notation: $\sigma^2_g$: Gradient noise Variance, $\sigma^2_c$: Communication noise variance, $\chi^2$: Data heterogeneity constant satisfying $n^{-1} \sum_{i=1}^n \|\n f_i(x^*)- \n f(x^*)\|^2 \leq \chi^2 $ for optimal point $x^*$. 
    
$L,\mu,\ka,n$: Smoothness constant, strong convexity parameter, constant depending on network topology, total number of nodes. 

For S-Near \texttt{DGD}, $t$ denotes the number of consensus steps during each iteration and convergence is inexact even with $t\to \infty$.  The convergence is to  a neighbourhood of size $\mathcal{O}\big(\ka^{2t}\chi^2 + \frac{L^2}{\mu^2}  \sigma^2_c\big)$.
}}

\end{table}
\subsection{Related Work}
Several works %have 
explored the topic of inexact communication in the context of decentralized optimization, including \cite{doan, com1,do2_,do3, lee,doan2,do4,nsl,suh1,suh2,suh3}. Notably, one of the earliest and significant works in this setting are \cite{ned1,ned2}. The current work extends them in several ways, including the utilization of \texttt{GT} to address data heterogeneity and the assumptions about the underlying functions. These differences allow us to achieve superior theoretical and empirical convergence properties compared to contemporary works, as documented in Table 1 and discussed in Section \ref{sec:algo}.

Another related line of research to our work is that of decentralized optimization with randomized compressed communication \cite{cs1,cs2,cs3,cs4}. These works focus on iterate quantization for smooth and strongly convex deterministic optimization problems using randomized compression operators. However, there are significant distinctions %several important distinctions 
between our work and these prior works, including differences in the underlying assumptions. Specifically, the algorithms proposed in the aforementioned works assume access to the compression error vector, which is transmitted to the receiving node for error compensation over a noiseless channel. Furthermore, the error variance is assumed to be controllable (\cite[Assumption 2]{cs1}) with the convergence performance being intricately linked to it(\cite[Theorem 1]{cs1}). In our setting, neither of these assumptions are applicable as they are violated in many practical scenarios, as discussed in Section \ref{sec:mod}. 


The benefits of using  the \texttt{GT} strategy to address data heterogeneity have been extensively studied in numerous works \cite{DIG,gt,sgt1}.
%\texttt{GT} to tackle data heterogeneity has been studied by a large number of works \cite{?}. 
In the deterministic setting, algorithms such as \texttt{EXTRA} \cite{EXTRA} achieve linear convergence for strongly convex, smooth functions. For the stochastic optimization setting (without communication noise), \cite{pudi,sgt1} demonstrate that \texttt{GT} based \texttt{DGD} is agnostic to the data heterogeneity. Furthermore, variants of \texttt{GT} such as \texttt{NEXT} \cite{NEXT} or the \texttt{D}$^2$ algorithm proposed in \cite{DD} have been shown to mitigate the effects of data heterogeneity. Other works exploring the \texttt{GT} strategy in various contexts include \cite{gt, uribe, DIG,sgt3,sgt4,sgt5,sgt6}.
%Additionally, the \texttt{GT} variants such as \texttt{NEXT} \cite{NEXT} or the \texttt{D}$^2$ algorithm of \cite{DD} have also been shown to mitigate the effects of data heterogeneity. Some other works that study \texttt{GT} in different contexts include \cite{gt, uribe, DIG,sgt3,sgt4,sgt5,sgt6}.




\subsection{Contributions}

The main contributions can be summarized as follows:
\begin{itemize}
    \item[-] %We study decentralized stochastic optimization for smooth, strongly convex functions in the presence of inexact communication between the nodes. To address the challenges posed by communication noise and data heterogeneity, we propose and analyze a novel variant of the %\texttt{DGD} 
    %\rb{Gradient Tracking} algorithm called Inexact Communication based Gradient Tracking (\texttt{IC-GT}). 
    We propose and analyze a novel variant of the %\texttt{DGD} 
    Gradient Tracking algorithm called Inexact Communication based Gradient Tracking (\texttt{IC-GT}) to address the challenges posed by communication noise and data heterogeneity. 
    Unlike previous approaches, our method not only retains the benefits of \texttt{GT} but also effectively eliminates the negative impact of inexact communication on algorithm performance through careful design interventions. 
    
    \item[-] %On the theoretical side, 
    We show \texttt{IC-GT} can recover (upto logarithmic factors) the optimal convergence rate requirements of $\mathcal{O}(1/\epsilon)$ iterations required to achieve $\epsilon$-accuracy for stochastic optimization  while removing the data heterogeneity dependence even in the presence of communication noise. By extending the theory for exact communication based decentralized optimization \cite{st,sgt1}, our results improve upon the existing works which consider communication and gradient noise under similar assumptions and achieve either a worse convergence rate or inexact convergence (cf. Table 1).
    %To the best of our knowledge, this is first algorithm that has these guarantees with previous works which consider communication and/or gradient noise under similar assumptions having either a worse convergence rate or inexact convergence (cf. Table 1).
    
    
    \item[-] To validate our theoretical results, we report experimental results that compare \texttt{IC-GT} with similar methods like \texttt{DGD} \cite{tsit1}, \texttt{DIGGing}\cite{DIG} and  \texttt{EXTRA}\cite{EXTRA}. Our experiments demonstrate the superior performance of \texttt{IC-GT} on logistic regression and image recognition problems on well known datasets. 
    
    
    
\end{itemize}
The paper is organized as follows. We introduce the notation that is used through out the paper in the rest of this section. 
%The rest of the paper is organized as follows: In the next two sub-sections we discuss the related work and the contributions of this paper.
%\rb{We discuss that is used through out the paper in the rest of this section}
In Section \ref{sec:mod}, we describe the problem formulation and in Section \ref{sec:algo}, we present the proposed algorithm and its implementation. Section \ref{sec:conv} provides the convergence analysis while Section \ref{sec:num} presents the numerical evidence in its support. Future directions of research and conclusions are listed in Section 6. 


\textit{Notation: } We use $\mathbb{R}$ to denote the set of real numbers and $\mathbb{N}$ to denote the set of all strictly positive integers. We use $\xb_k \in \mathbb{R}^{ nd}$ to denote the stacked version of $\{x_{i,k}\}_{i\in [n]}$, where $x_{i,k} \in \mathbb{R}^d$ is a column vector which denotes the value of the objective variable held by node $i$ at %time 
iteration $k$, i.e. $\xb_k:= (x_{1,k},\cdots, x_{n,k})$. We define $\bar{\textbf{{x}}}_k := \frac{1}{n}(1_n1_n^T\otimes I_d)\xb_k= \left(\frac{1}{n}\sum_{i=1}^n x_{i,k},\cdots,\frac{1}{n}\sum_{i=1}^n x_{i,k} \right) $, where the column vector $1_n:= (1,\cdots,1) \in \mathbb{R}^{n}$ and $I_d \in \mathbb{R}^{d\times d}$ being the identity matrix. The symbol $\otimes$ is used to denote the Kronecker product between any two matrices while $\|\cdot\|$ is understood to be the $\ell_2 $-norm of a vector or a matrix depending upon the argument. The $\ell_2$ inner product between any two vectors is denoted using $\langle \cdot,\cdot \rangle$. The following notation is used for the gradients,
$$
\n \textbf{f}( \xb_k) := \left(\n f_1 (x_{1,k}), \cdots, \n f_n(x_{n,k})\right) \text{ and }  \n \textbf{f}(\bar{\textbf{x}}_k) :=\left(\n f_1 (\bar{x}_k),\, \cdots\, ,\n f_n(\bar{x}_k)\right).
$$
We also define the matrices,
$$
\textbf{I}_n = I_n \otimes I_d \quad \text{ and } \quad \III := \II - \frac{1_n1_n^T \otimes I_d}{n}.
$$
Finally, for any two real valued functions $f(\cdot)$ and $g(\cdot)$, $f(x) = \mathcal{O}(g(x))$ denotes the standard Big-O notation which implies that there exists a finite constant $C>0$ and $x_0$ such that $|f(x)| \leq Cg(x)$ for all $x\geq x_0$.  We use $\mathcal{\tilde{O}}(\cdot)$ when ignoring logarithmic factors. %The symbol $\mathcal{O}(\cdot)$ is the standard Big-O notation. We use $\mathcal{\tilde{O}}(\cdot)$ when ignoring logarithmic factors.

%\section{Problem Formulation}\label{sec:mod}
\section{Preliminaries}\label{sec:mod}
In this section, we provide preliminaries regarding the network and communication model, and also state the assumptions that are used in the paper.
%In this section, we describe the network model in detail and outline the specific assumptions we make.

The network is represented by a (undirected) graph $\mathcal{G} =\{\mathcal{V},\mathcal{E}\}$, where $\mathcal{V}$ denotes the set of nodes and $\mathcal{E}$ represents the set of edges. 
%The network is assumed to be modelled by a (undirected) graph $\mathcal{G} =\{\mathcal{V},\mathcal{E}\}$, where $\mathcal{V}$ is the node set and $\mathcal{E}$ is the edge set. 
We use the matrix $Q = [q_{ij}]_{i,j\in[n]} \in \mathbb{R}^{n \times n}$ to denote the mixing matrix (or consensus matrix) that captures the connectivity of the network. By this, we mean that the entry $q_{ij}>0$ (assumed to be equal to $q_{ji}$), if there is %a link
an edge between any two nodes $i,\,j\in \mathcal{V}$. We use $\mathcal{N}(i)$ to denote the set of neighbours of $i$, i.e., the set $j\in \mathcal{V}$ with $j\neq i$ for which $q_{ij}>0$. 
%The formal statement of the problem we aim to solve can be written as: 
%\begin{align}\label{mainprob1}
%\min_{x \in \mathbb{R}^{nd}} \,&f(x) := %\sum_{i=1}^n f_i(x^i)\nonumber\\
%\text{s.t.} \, & (Q\otimes I_d)\,x = x.
%\end{align}
We make the following assumption regarding the matrix $Q$. 
\begin{assumption}[Mixing matrix]\label{asmp1}  
The mixing matrix $Q$ is symmetric and doubly stochastic. Furthermore, the eigenvalues $\{\lambda_i\}_{i\in [n]}$ of $Q$ satisfy $1 = \lambda_1 > \lambda_2 \geq \cdots \geq \lambda_n > -1$.
\end{assumption}
\begin{remark}
The symmetric and double stochasticity assumption of $Q$ is standard in decentralized optimization along with $\lambda_2<1$ which implies that the graph is connected. Therefore, it implies that $(Q \otimes I_d )\textbf{x } = \textbf{x }$ if and only if $x_i = x_j$ for all $i,j \in \mathcal{V}$. %In particular, 
Moreover, it also ensures that the spectral gap $\delta(Q):=1- \max\{|\lambda_2|,|\lambda_n|\}$ is greater than zero which in turn ensures that the consensus error decreases linearly after each averaging step, i.e.,
\begin{equation}\label{contract}
\left\| (Q \otimes I_d ) \textbf{x }- \left(\frac{1_n1_n^T}{n}\otimes I_d \right)  \textbf{x}\right\|^2 \leq (1-\delta)^2 \left\| \textbf{x}  - \left(\frac{1_n1_n^T}{n}\otimes I_d \right)  \textbf{x}\right\|^2
\end{equation}
for any $\textbf{x}\in \mathbb{R}^{nd}$. For undirected graphs, this assumption can be guaranteed by using the Metropolis weights(\cite[Section 3]{DIG}).
\end{remark}
We next describe the communication model considered in this work. %we assume.  
We make the assumption that when any node $i \in [n]$ sends a signal vector $x_{i,k} \in \mathbb{R}^d$ to a neighboring node $j$ at iteration $k \in \mathbb{N}$, node $j$ receives the vector $\varphi_c(x_{i,k}) \in \mathbb{R}^d$ instead of the original vector $x_{i,k}$ , where $\varphi_c(\cdot):\mathbb{R}^d \to \mathbb{R}^d$ represents a random transformation given by
$$\varphi_c(x_{i,k}) := x_{i,k} + \epsilon_{i,k,c}\,,$$
where $\epsilon_{i,k,c} \in \mathbb{R}^d$ is a random vector. 
We emphasize that we do not assume access to the values of $\epsilon_{i,k,c}$. We make the following assumption concerning $\epsilon_{i,k,c}$.
\begin{assumption}[Noisy signal transmission]  \label{asmp2}
The random noisy vector $\epsilon_{i,k,c}$ is assumed to be zero mean conditioned on $x_k$ with bounded variance for all $i \in [n]$ and $k\in \mathbb{N}$, i.e., 
$$
\EX \,[\epsilon_{i,k,c}|x_{i,k}]=0 ,\,\,\,\,\,\EX\,[\|\epsilon_{i,k,c}\|^2]\leq \sigma^2_{c},
$$
for some finite $\sigma_c>0$.
\end{assumption}
\noindent We describe two important examples of $\varphi_c(\cdot)$ for which Assumption \ref{asmp2} is satisfied.\\

\noindent \textbf{Additive White Gaussian Noise channel (AWGN)}: The most common approach to modeling an analog based communication channel between two nodes is through an AWGN channel \cite{AWGN}. In this scenario, when a node transmits a signal $y_{\text{tr}} \in \mathbb{R}$ to a neighboring node, the received signal at the receiving node, denoted as $y_{\text{rc}}$, can be represented as
$$
y_{\text{rc}} = h y_{\text{tr}} + \epsilon_{c},
$$
where $h \in \mathbb{R}$ captures channel effects like fading \cite{fading}, and $\epsilon_{c}$ represents zero-mean Gaussian noise with variance $\sigma_c^2$, independent of the transmitted signal $y_{\text{tr}}$. Assuming that the receiving node possesses a prior estimate of $h$ \cite[Chapter 4]{book}, it can construct an estimate of the true signal $\hat{y}_{\text{rc}}$ as, % follows,
$$
\hat{y}_{\text{rc}} =  \frac{1}{h} y_{\text{rc}}= y_{\text{tr}}  + \frac{\epsilon_{c}}{h}.
$$
Hence, in this scenario, we can express $\varphi_c(\cdot)$ as,
$$
\varphi_c(y_{\text{tr}}) = y_{tr} + \frac{\epsilon_{c}}{h},
$$
implying Assumption \ref{asmp2} is satisfied since $\EX \,[\varphi_c(y_{\text{tr}})] = y_{\text{tr}} $ and $\EX \,[\|\varphi_c(y_{\text{tr}})- y_{\text{tr}}\|^2] \leq \sigma_c^2/h^2 $.\\

\noindent \textbf{Probabilistic Quantization:} Another significant example of operator $\varphi_c$ arises in the context of quantization with unbiased compression operators. Specifically, consider a scalar $x \in \mathbb{R}$. The quantized value $\varphi_c(x)$ can be determined based on the following rule:
%To elaborate, for any scalar $x \in \mathbb{R}$, suppose the quantized value $\varphi_c(x)$ is given according to the rule:
\begin{equation}\label{randquant}
\varphi_c(x)= 
    \begin{cases}
    \lfloor x \rfloor_p \text{ with probability } (  \lceil x \rceil_p -x)\Delta_p\\
    \lceil x \rceil_p \text{ with probability } (x-  \lfloor x \rfloor_p )\Delta_p
    \end{cases}
\end{equation}
where $\lfloor x \rfloor_p$ and $\lceil x \rceil_p$ denote the operations of rounding down and up to the nearest integer multiple of $\frac{1}{\Delta_p}$ respectively, and $\Delta_p$ is a positive integer. The operator $\varphi_c$ defined in (\ref{randquant}) satisfies $\EX \,[\varphi_c(x)] =x$ and $\EX\,[| \varphi_c(x) -x|^2] \leq \tfrac{1}{4\Delta_p^2}$ as shown in \cite{pq} implying Assumption \ref{asmp2} is satisfied.\\

\noindent We also make the following assumptions regarding the objective function. %$f_i$: 
 
\begin{assumption}[Regularity and convexity]\label{asmp3}  
Each local function $f_i$ is $L$-smooth and %the global function $f$ is 
$\mu$-strongly convex.
\end{assumption}
\begin{assumption}[Unbiased Gradient Samples]\label{asmp4} 
Each node $i$ has access to conditionally unbiased, finite variance gradient samples $\n F_i(x_{i,k},\xi_k)$ of $\n f_i(x_{i,k})$ for any given $x_{i,k}\in \mathbb{R}^d,$ $k\in \mathbb{N}$. That is,  %i.e.
$$
\EX_{\xi_{i,k}}\, [\n  F_i(x_{i,k},\xi_{i,k})\,|\,x_{i,k}] = \n f_i(x_{i,k}) \text{ and }  \EX_{\xi_{i,k}} \,[\|\n F_i(x_{i,k},\xi_{i,k}) -\n f_i(x_{i,k})\|^2] \leq \sigma^2_{g}
$$
for some finite $\sigma_g>0$ with $\xi_{i,k}$ being assumed to be independent of $\epsilon_{i,k,c}$.
\end{assumption} 
\begin{remark}
The finite variance assumption in Assumption 4 can be relaxed along two possible lines with minor modifications to the convergence analysis. One relaxation would be to allow the noise to grow with the gradient norm (cf. Assumption 3b, \cite{st}). The other possibility is to replace $\sigma^2$ with $\sigma^2_*:= \frac{1}{n} \sum_{i=1}^n \|\n F(x^*,\xi_i)- \n f(x^*)\|^2$, the noise at the optimal point $x^*$, as in \cite{ngu}. 
\end{remark}

\begin{remark}
The convergence analysis can also be extended to a non-convex setting by modifying the measure of stationary to be the $\ell_2$-norm of the gradient. 
\end{remark}

% \textbf{Effects of inexact transmission on DGD:} We next discuss how inexact communication can affect the performance and convergence guarantees of DGD. To keep the discussion clear, we assume a fully connected graph (i.e. $q_{ij}=1/n,\,\forall \,i,j$ ). Let us recall the basic iteration of the DGD algorithm here (with $x_0^i = x_0^j \,\forall i,j$):
% \begin{align*}
%     x^i_{k+1} &= \frac{x_{i,k}}{n}  + \frac{1}{n}\sum_{j \in \mathcal{N}(i)} \big( x^j_k + \epsilon_{k,c}^{j}\big) -\alpha \n F_i(x^i_k,\xi_{i,k}) 
% \end{align*}
% We prove the following bound on the optimal error $\|\bar{x}_k-x^*\|^2$, where $x^*$ is any optimal solution:
% \begin{equation}\label{main2}
%  \EX  [ \|\x_{k+1}  - x^*\|^2] \leq \big( 1 -2\alpha\eta \big)^{k}  \| \x_0 -x^* \|^2 + \frac{1}{2\alpha \eta}\Bigg\{ \frac{\alpha L^2(1- \alpha \eta) \sigma_c^2}{n\eta}    + 2\sigma^2_{c} \Bigg\}  +   \frac{\alpha \sigma_g^2}{\eta}
% \end{equation}
% where $\eta:= \frac{\mu L}{\mu+L}$ and $\alpha< \min\big\{ \frac{2}{\mu+L},\frac{1}{2\eta}\big\}$. The proof is provided in Appendix I. There are a few interesting points worth noting here. If $\sigma_c^2,\,\sigma_g^2 =0$, we recover the linear convergence rate of gradient descent (for a fully connected graph).  The second term in the right hand side pertaining to iterate noise is scaled by factor $\frac{1}{\alpha}$ while the last term which is related to the gradient noise is scaled by $\alpha$. This puts a direct constraint on how small $\alpha$ can be to reduce the effects of gradient noise since a small $\alpha$ would increase the contribution of the iterate noise. 


% We next consider how a GT iteration would be affected by inexact communication. Let us recall the following version of GT algorithm with a fully connected graph (the setting/version of the algorithm is again assumed to best highlight the effects of iterate noise):
% \begin{align*}
% x^i_{k+1} &= \frac{1}{n}\sum_{i=1}^n \big(x_{i,k} -\alpha y_{i,k} \big)\\
% y^i_{k} &= \frac{1}{n}\sum_{j=1}^n y_{k-1}^j + \n F_i(x^i_{k},\xi^i_k) -\n F_i(x^i_{k-1},\xi^i_{k-1})
% \end{align*}
% % If we consider the iteration for $\bar{y}_k := \frac{1}{n}\sum_{j=1}^n y_k^j$:
% % \begin{align*}
% %     \bar{y}_{k} &= \bar{y}_{k-1} + \frac{1}{n}\sum_{i=1}^n  \n f_i(x^i_{k}) -\frac{1}{n}\sum_{i=1}^n \n f_i(x^i_{k-1})\\
% %     &=  \frac{1}{n}\sum_{i=1}^n  \n f_i(x^i_{k}) + \bar{y}_0- \frac{1}{n}\sum_{i=1}^n  \n f_i(x^i_{0}) 
% % \end{align*}
% % Assume for simplicity that $\bar{y}_0 = \frac{1}{n}\sum_{i=1}^n  \n f_i(x_{0}) $ for some $x_0$. Then, we have
% % $$
% %  \bar{y}_{k}  = \frac{1}{n}\sum_{i=1}^n  \n f_i(x^i_{k})= \frac{1}{n}\sum_{i=1}^n  \n f_i(\bar{x}_{k})
% % $$
% % where the last inequality follows from $x^i_0=x^j_0=x_0$ for all $i,\,j$. Thus the algorithm essentially can written as:
% % $$
% % \x_{k+1} =\x_k - \frac{\alpha}{n} \sum_{j=1}^n f_i (\x_k) 
% % $$
% % for $x^i_k=\x_k$ for all $i\in[n].$ 
% Consider the gradient tracking part with iterate noise:
% \begin{align*}
%     \bar{y}_{k} &= \bar{y}_{k-1} + \frac{1}{n}\sum_{i=1}^n  \n F_i(x^i_{k},\xi_{i,k}) -\frac{1}{n}\sum_{i=1}^n \n F_i(x^i_{k-1},\xi_{k-1}^i) + \frac{1}{n}\sum_{i=1}^n \epsilon^i_{k,c}\\
%     &=  \frac{1}{n}\sum_{i=1}^n  \n F_i(x^i_{k},\xi_{i,k})  + \frac{1}{n} \sum_{t=1}^k\sum_{i=1}^n \epsilon^i_{t,c}
% \end{align*}
% This essentially destroys the gradient tracking capability of $y_k$, even in the fully connected case, since now we have
% $$
% \EX \big\|\bar{y}_{k} -\frac{1}{n}\sum_{i=1}^n \n F_i(x^i_{k},\xi_{i,k}) \big\|^2 \leq k \sigma^2_{c,k}
% $$
\section{The \texttt{IC-GT} method}\label{sec:algo}
In this section, we describe the proposed method that accounts for inexact communication, referred to as Inexact Communication based Gradient Tracking \ty{(IC-GT)} designed to solve the problem (\ref{mainprob}). Algorithm 1 presents the pseudo code of \ty{(IC-GT)}.
%Algorithm~\ref{alg:code1} presents the pseudo code for the proposed algorithm, referred to as Inexact Communication based Gradient Tracking \ty{(IC-GT)}, designed to solve the problem (\ref{mainprob}).
\begin{algorithm}[H]
\label{alg:code1}
\footnotesize
   \caption{\small \texttt{INEXACT COMMUNICATION based GRADIENT TRACKING (IC-GT)}}
    \footnotesize
  \begin{algorithmic}[1]
   \footnotesize
     \STATE \textbf{Input} Graph $\mathcal{G}(\mathcal{V},\mathcal{E})$; Matrix $Q= [q_{ij}]_{i,j\in [n]} \in \mathbb{R}^{n \times n}$ ; Operator $\varphi_c(\cdot)$;  Noise attenuation parameter $\gamma > 0$; Step size parameter $\alpha > 0$.
     \STATE \textbf{Initialization} $x_{i,0} \in \mathbb{R}^d, \,\forall i$; $y_{i,0} := \n F_i(x_{i,0} ,\xi_{i,0} ), \,\forall i$. 
    \WHILE{$ k \geq 1$ in parallel:}
     \FOR{ all $i\in [n],$} 
         \STATE 
          $
          v_{i,k} = (1-\gamma)x_{i,k} + \gamma q_{ii} x_{i,k} +\gamma \sum_{j\in \mathcal{N}(i)}q_{ij}\varphi_c(x_{j,k})\
          $
           \STATE 
          $
           x_{i,k+1} = v_{i,k} - \alpha y_{i,k}
          $
           \STATE 
          $
          y_{i,k+1}=  (1-\gamma)y_{i,k} + \gamma q_{ii} y_{i,k}  +\gamma \sum_{j\in \mathcal{N}(i)}q_{ij}\varphi_c(y_{j,k}) + \nabla F_i(x_{i,k+1},\xi_{i,k+1}) - \nabla F_i(x_{i,k}, \xi_{i,k} ) 
          $
       \ENDFOR  
      \STATE $k  \rightarrow k + 1$ 
   
    \ENDWHILE
       % \STATE
  %   \STATE \texttt{METROPOLIS HASTINGS} ($\bar{\mathcal{G}}(\bar{\E},\mathcal{V})$): 
  %   \STATE Set $\bar{q}_{ij}$ according to:
  %      \begin{equation*}
  % \bar{q}_{ij} =
  %   \begin{cases}
  %     1/\big(1+\max\{ |\bar{\E}_i|,|\bar{\E}_j|\}\big) &  \text{if } (i,j) \in \bar{\mathcal{E}}\\
  %     1-\sum_{j\in \mathcal{N}_i}\bar{q}_{ij} & \text{if $i=j$}\\
  %     0 & \text{otherwise}
  %   \end{cases}       
% \end{equation*}
  \end{algorithmic}
\end{algorithm}
 To express \texttt{IC-GT} in matrix form, we introduce the matrices $Q' := [q'_{ij}]_{i,j\in [n]}$ and $\hat{Q} :=[\hat{q}_{ij}]_{i,j\in [n]}$ defined as follows: 
 \begin{equation}\label{q'hat}
 \Q\defeq \left( I_n-Q \right)\otimes I_d \qquad  \qquad   \hat{\textbf{Q}}\defeq\left( Q-\text{diag}(Q)\right)\otimes I_d,
 \end{equation}
 where $\text{diag}(Q)$ denotes the diagonal matrix with entries $q_{ij}$ for $i=j$ and 0 otherwise. Using the communication model, $\varphi_c(x_{j,k}) = x_{j,k}  + \epsilon_{j,k,c},$ we can express the iteration for $v_{i,k}$ as follows:  
 \begin{align*}
      v_{i,k} &= \left(1-\gamma(1-q_{ii})\right)x_{i,k} +\gamma \sum_{j\in \mathcal{N}(i)}q_{ij}\varphi_c(x_{j,k})\\
      &= \left(x_{i,k}-\gamma(1-q_{ii}))x_{i,k} + \gamma \sum_{j\in \mathcal{N}(i)}q_{ij} x_{j,k}\right) +\gamma \sum_{j\in \mathcal{N}(i)}q_{ij} \epsilon_{j,k,c}.
 \end{align*}
 %where we used $\varphi_c(x^i_{k}) = x_{i,k}  + \epsilon_k^i$ to get the last inequality. 
 Performing a similar manipulation for the $y$ update, we can express \texttt{IC-GT} using (\ref{q'hat}) as follows:
\begin{align}
    \vb_{k}&= (\textbf{I}_n-\gamma \Q)\xb_{k} + \gamma\hat{\textbf{Q}}\bm{\epsilon}_{k,c} \label{mat0}\\
    \xb_{k+1} &= \vb_{k} - \alpha \yb_{k} \label{matI}\\
     \yb_{k+1}&=  (\textbf{I}_n-\gamma \Q)  \yb_{k}  + \nabla \textbf{F}(\xb_{k+1},\mathbf{\bm{\xi}}_{k+1} ) - \nabla \mathbf{F}(\xb_{k},\mathbf{\bm{\xi}}_{k} ) + \gamma  \hat{\textbf{Q}}\hat{\bm{\epsilon}}_{k,c}\label{matII}
\end{align}
where $\hat{\epsilon}_{i,k,c} := \varphi_c(y_{i,k}) -y_{i,k}$, $\bm{\epsilon}_{k,c}:= 
    \left(
\epsilon_{1,k,c} ,\cdots, \epsilon_{n,k,c} 
\right)$ and $\nabla \textbf{F}(\xb_{k},\mathbf{\bm{\xi}}_{k} ) := \big( \n F_1(x_{1,k},\xi_{1,k}), \cdots,$ $\n F_n(x_{n,k},\xi_{n,k}) \big)$. \\




\noindent %\textbf{Discussion:} 
We next discuss the main modification made to the standard \texttt{DGD} algorithm \cite{tsit1} utilized in \texttt{IC-GT} to better understand its communicating and computational capabilities.\\

\noindent \textbf{(i) Use of $\textbf{I}_n - \gamma\Q$:}  In the context of \texttt{IC-GT}, the weight matrix $\II-\gamma\Q$ is employed instead of the typical $\textbf{Q}$ used in \texttt{DGD} \cite{tsit1}. To illustrate its effectiveness in mitigating communication noise, let us examine the sequence  $\{\xb_k \}_{k\geq0}$ generated according to the recursion:
\begin{align}\label{mat01}
	\xb_k &= (\textbf{I}_n-\gamma \Q)\xb_{k-1} + \gamma \hat{\textbf{Q}}\bm{\epsilon}_{k-1,c}, 
\end{align}
where the noise term $\bm{\epsilon}_{k-1,c}$ satisfies Assumption \ref{asmp2}. The recursion in (\ref{mat01}) can be interpreted as a distributed averaging algorithm using the weight matrix $\textbf{I}_n - \gamma\Q$. Specifically, when $\gamma=1$ and $\bm{\epsilon}_{k-1,c}=0$, (\ref{mat01}) reduces to the standard distributed averaging algorithm \cite{xiao}. Next, we consider the expression for the averaged iterates $\bar{\xb}_k$ obtained by multiplying (\ref{mat01}) by $\frac{1}{n}\left( 1_n 1_n^T \otimes I_d\right)$:
\begin{align}\label{mat00}
	\xx_{k} = \xx_{k-1} -\gamma \frac{1}{n} \big(1_n1_n^T\otimes I_d\big)\hat{\textbf{Q}} \bm{\epsilon}_{k-1,c},
\end{align}
where we used $\left(1_n^T \otimes I_d \right)\left(\textbf{I}_n-\gamma \Q\right) = (1^T_n\otimes I_d)$ from Assumption \ref{asmp1}. 
Subtracting (\ref{mat00}) from (\ref{mat01}) and defining $\tilde{\textbf{Q}}:= \left(\II -n^{-1} 1_n1^T_n\otimes I_d \right) \hat{\textbf{Q}}$ and recalling $\III:= \II - \frac{1_n1_n^T \otimes I_d}{n}$, we get,
\begin{align*}
	\xb_k - \xx_k &= (\textbf{I}_n-\gamma \Q)(\xb_{k-1} - \xx_{k-1}) + \gamma\tilde{\textbf{Q}}\bm{\epsilon}_{k-1,c} \\
	&=   (\textbf{I}_n-\gamma \Q)(\xb_{k-1} - \xx_{k-1}) + \gamma\tilde{\textbf{Q}}\bm{\epsilon}_{k-1,c}  -\frac{1_n1_n^T \otimes I_d}{n}(\xb_{k-1} - \xx_{k-1} ) \\
	&=  (\III-\gamma \Q )(\xb_{k-1} - \xx_{k-1}) + \gamma\tilde{\textbf{Q}}\bm{\epsilon}_{k-1,c}, 
\end{align*}	
where the second equality is due to $\frac{1_n1_n^T \otimes I_d}{n}(\xb_{k-1} - \xx_{k-1} ) = 0$.  Applying norms and taking squares yields, 

%and taking square norms, we get,
\begin{align}%\label{qaa}
	\|\xb_k - \xx_k\|^2 &\leq  \|\III-\gamma \Q \|^2 \|\xb_{k-1} -\xx_{k-1}\|^2 + \gamma^2  \|\tilde{\textbf{Q}} \bm{\epsilon}_{k-1,c}\|^2\nonumber +2\gamma\left\langle (\III-\gamma \Q)(\xb_{k-1} - \bar{\textbf{x}}_{k-1}), \tilde{\textbf{Q}}\bm{\epsilon}_{k-1,c}\right\rangle. %\nonumber
\end{align}
%where $\tilde{\textbf{Q}}:= \left(\II -n^{-1} 1_n1^T_n\otimes I_d \right) \hat{\textbf{Q}}$. 
Using the conditional zero mean and finite variance  assumption for $\bm{\epsilon}_{k-1,c}$ (Assumption \ref{asmp2}), we get,
\begin{align*}
	\EX [\|\xb_k - \xx_k\|^2]  &\leq  \left( 1- \gamma (1-  \lambda_2)\right)^{2} \EX [\|\xb_{k-1} -\xx_{k-1}\|^2] + 2n\gamma^2 \sigma_{c}^2, 
\end{align*}
where we used $\|\III-\gamma \Q \| \leq 1- \gamma(1-\lambda_2)$ (cf. (\ref{2normbd})) and $\|\tilde{\textbf{Q}}\|^2\leq 2$. Applying the above inequality repeatedly through iteration $k=0$ yields, %Recurring the above inequality till $k=0$ yields:
\begin{align}\label{u19}
	\EX [\|\xb_k - \xx_k\|^2]  &\leq  \left( 1- \gamma (1-  \lambda_2)\right)^{2k} \|\xb_{0} -\xx_{0}\|^2 + \frac{2 n \gamma \sigma_{c}^2}{1-\lambda_2}.
\end{align}
(\ref{u19}) unveils a fundamental trade-off between two crucial aspects: the rate of decay of the consensus error and the mitigation of the influence exerted by the communication noise variance. As the parameter $\gamma$ decreases, %is diminished, 
a smaller final consensus error can be achieved. However, this improvement comes at the expense of a slower convergence rate in reducing the consensus error. In view of this trade-off, the parameter $\gamma$ is referred to as the `\textit{noise attenuation}' parameter. \\ %However, this improvement comes at the cost of a slower convergence rate in reducing the consensus error. In light of this discussion, the parameter $\gamma$ is referred to as the `\textit{noise attenuation}' parameter.  \\
%}
 
\noindent \textbf{(ii) Use of Gradient Tracking:} Another crucial feature of \texttt{IC-GT} is its ability to track gradients while accommodating inexact communication through gradient tracking. The inclusion of gradient tracking offers the advantage of making the algorithm agnostic to data heterogeneity. To elaborate,
%The second important feature of \texttt{IC-GT} is its capability to track gradients while accommodating inexact communication using gradient tracking. The advantage of incorporating gradient tracking is that it makes the algorithm agnostic to data heterogeneity. To explain further, 
 the number of iterations required to achieve $\epsilon$-accuracy using stochastic \texttt{DGD} depends on $\mathcal{O} \left( \frac{\sqrt{L}\chi^2} {\sqrt{\epsilon}}\right)$ \cite{st}, where $\chi$ is a constant satisfies the inequality
$$
\frac{1}{n} \sum_{i=1}^n \|\n f_i(x^*)- \n f(x^*)\|^2 \leq \chi^2, 
$$
with $x^*$ denoting the optimal solution of (\ref{mainprob}). In contrast, \texttt{IC-GT} eliminates the dependence on $\chi$ entirely and, moreover, recovers the linear convergence rate in scenarios where the variances of both the gradient and communication noise are zero.

\section{Convergence Analysis}\label{sec:conv}
In this section, we establish theoretical convergence guarantees for the proposed \texttt{IC-GT} algorithm. We build up to our main result through a series of technical lemmas which we state next. 
\subsection*{Preliminaries}
%Throughout this section, we assume that Assumptions \ref{asmp1}-\ref{asmp4} \rb{hold. }%are satisfied. 
For the sake of brevity, %convenience, 
we assume $\bm{\epsilon}_{k,c} = \hat{\bm{\epsilon}}_{k,c}$ in (\ref{mat0})-(\ref{matII}) for all $k\in \mathbb{N}$ without loss of generality. We begin by expressing the algorithm in terms of the difference between the variables and their corresponding averages, which we refer to as the \textit{consensus} error. To denote this, we adopt the notation  $\Delta \textbf{z }:=\textbf{z} - \Bar{\textbf{z}}$ for any variable $\textbf{z}\in \mathbb{R}^{nd}$, where $\Bar{\textbf{z}}$ denotes the average, i.e. $\Bar{\textbf{z}}:= \Big( \frac{1_n1_n^T}{n}\otimes I_d\Big)z$. We first establish a recursive relation for the consensus error. 
\begin{lem}\label{lem0}
\emph{[\textbf{Recursive relation for consensus errors}]}
%The following recursive relation is satisfied by the consensus errors for \texttt{IC-GT}:
Suppose %Assumptions \ref{asmp1}-\ref{asmp4} hold and 
$\bm{\epsilon}_{k,c} = \hat{\bm{\epsilon}}_{k,c}$ in (\ref{mat0})-(\ref{matII}) for all $k\in \mathbb{N}$. 
%\rb{Suppose Assumptions \ref{asmp1}-\ref{asmp4} hold. 
Then, the iterates generated by \texttt{IC-GT} satisfy the following recursive relation:
\begin{equation}\label{mm1}
\Psi_k = \textbf{J}_\gamma \Psi_{k-1} + \alpha \Eb_{k-1},
\end{equation}
where 

\begin{equation}\label{not1}
	\Psi_k\defeq  \begin{bmatrix}
		\Delta \vb_{k}\\
		\Delta \xb_{k}\\
		\alpha \Delta \yb_{k}
	\end{bmatrix},\qquad 
	\,\,\,
	\textbf{J}_\gamma\defeq
	\begin{bmatrix}
		\III-\gamma \Q &0 &-(\III-\gamma \Q )\\
		0 & \III-\gamma \Q  &-\III\\
		0 & 0&  \III-\gamma \Q 
	\end{bmatrix},
\end{equation}
and 
$$
\Eb_{k-1} \defeq  \frac{\gamma}{\alpha}
\begin{bmatrix}
	\tilde{\textbf{Q}}\bm{\epsilon}_{k,c} \\
	\tilde{\textbf{Q}} \bm{\epsilon}_{k-1,c} \\
	%\alpha  \hat{\textbf{Q}} \bm{\epsilon}_{k-1,c} 
	\alpha\tilde{\textbf{Q}} \bm{\epsilon}_{k-1,c} 
\end{bmatrix} 
+ 
\begin{bmatrix}
	0\\
	0\\
	\III  \left(\nabla \textbf{F}(\xb_{k},\bm{\xi}_{k} ) - \nabla \textbf{F}(\xb_{k-1},\bm{\xi}_{k-1} )\right)
\end{bmatrix}
$$
with $\III:= \left(I_n  - \frac{1_n1_n^T}{n}\right)\otimes I_d$, $\Q\defeq (I_n-Q)\otimes I_d$, $\hat{\textbf{Q}}\defeq(Q - \text{diag}(Q))\otimes I_d$ and $\tilde{\textbf{Q}}  \defeq \III \hat{\textbf{Q}} $.
\end{lem}
The proof of this lemma is provided in Appendix \hyperref[sec:apndxI]{I}. One of the challenges in analyzing \texttt{IC-GT} is that the matrix $\textbf{J}_{\gamma}$ defined in (\ref{not1}) is not necessarily a contractive matrix. In other words, the condition $\|\textbf{J}_{\gamma}\|<1$ is not guaranteed to hold. However, the following result demonstrates that despite this restriction, there exists a positive integer $\tau$ such that $\|\textbf{J}_{\gamma}\|^\tau <1$.
%The proof is provided in Appendix I. One of the complications that arises in analyzing \texttt{IC-GT} is that the matrix $\textbf{J}_{\gamma}$ defined in (\ref{not1}) is not a contractive matrix. By this, we mean that $\|\textbf{J}_{\gamma}\|<1$ is not guaranteed to be satisfied. The next result shows that despite this, there exists a $\tau \in \mathbb{N}$ such that $\|\textbf{J}_{\gamma}\|^\tau <1$.
\begin{lem}\label{lem3}
\emph{[\textbf{Strict contractive property for $\textbf{J}_\gamma$}]}
Suppose Assumption~\ref{asmp1} holds. 
%\rb{Suppose Assumptions \ref{asmp1}-\ref{asmp4} hold.}
For any given %$a\in (0,1)$
$\delta \in (0,1)$, $\gamma \in (0,1/4)$ and $\lambda_2 $ associated with the matrix $Q$, suppose $\ka\in \mathbb{N}$ satisfies
\begin{equation}\label{kappaprop}
	\ka \geq \left\lceil\frac{2}{\gamma(1-\lambda_2)}\max\left\{4\ln  \left(\frac{2}{\gamma(1-\lambda_2)}\right)\,,\, \left( \gamma (1-\lambda_2) -\ln  \frac{\sqrt{\delta}}{4}  \right)\right\}\right\rceil,
\end{equation}
where $\lceil \cdot \rceil$ denotes the ceiling function. Then, $\|\textbf{J}^{\ka}_\gamma\|^2 \leq \delta<1$, where $\textbf{J}^\ka_\gamma := \underbrace{\textbf{J}_\gamma \times \cdots\times \textbf{J}_\gamma}_{\ka \text{ times }} $.
\end{lem}
The proof of this lemma is provided in Appendix \hyperref[sec:apndxII]{II}. The next result establishes a  descent relation for the consensus error $\EX \,[\|\Psi_{t+\ta}\|^2]$ in terms of $\EX [\|\Psi_{t}\|^2]$.
\begin{lem}\label{r2}
\emph{[\textbf{Descent relation for consensus error, $\EX [\|\Psi_{t}\|^2] \,$}]}
Suppose Assumptions \ref{asmp1}-\ref{asmp4} hold and $\bm{\epsilon}_{k,c} = \hat{\bm{\epsilon}}_{k,c}$ in (\ref{mat0})-(\ref{matII}) for all $k\in \mathbb{N}$. If $\gamma $ and $\alpha$ satisfy (\ref{alphacond}), then, for a given $0<\rho' \leq 1/4$, there exists a $\ka\in \mathbb{N }$ such that the following relations are satisfied for $t\geq \ta$:
\begin{align}\label{main_rec}
\EX [\|\Psi_{t}\|^2] &\leq  \rho' \EX [\|\Psi_{t-\tau}\|^2]+ 576 \alpha^2 \ka L^2  \sum_{i=t-\tau}^{t-1} \EX [\|\Psi_{i}\|^2] 
+ 1344  \alpha^2 \ka \sum_{i=t-\tau}^{t-1} \EX \left[\left\|\n \textbf{f}(\bar{\textbf{x}}_{i})-\n \textbf{f}(\textbf{x}^*) \right\|^2\right] \nonumber \\
  &\qquad + 64\gamma^2 \big(2 + \alpha^2 (\ka^2 +1/2) + \alpha^2 t\big)n\sigma_{c}^2 \ka+196n(\ka+1)\alpha^2\sigma^2_{g}
\end{align}
and for any $\ell < \ta$: % we have,
\begin{align}\label{main_rec_2}
\EX [\|\Psi_{\ell}\|^2] &\leq 2(1+\tau^2) \|\Psi_{0}\|^2+ 576 \alpha^2 \ka L^2  \sum_{i=0}^{\ell-1} \EX [\|\Psi_{i}\|^2] 
+ 1344  \alpha^2 \ka \sum_{i=0}^{\ell-1} \EX \left[\left\|\n \textbf{f}(\bar{\textbf{x}}_{i})-\n \textbf{f}(\textbf{x}^*) \right\|^2\right] \nonumber \\
  &\qquad + 64\gamma^2 \big(2 + \alpha^2 (\ka^2 +1/2) + \alpha^2 \ell \big)n\sigma_{c}^2 \ka+196n(\ka+1)\alpha^2\sigma^2_{g}.
\end{align}
\end{lem}
The proof of this lemma is provided in Appendix \hyperref[sec:apndxII]{III}. %The proof is provided in Appendix III. 
We next prove an auxiliary result that will be useful for bounding the consensus error. %which will be of use in bounding the consensus error.

\begin{lem}\label{lem4}
Suppose the non-negative scalar sequences $\{a_t\}_{t\geq0}$ and $\{e_t\}_{t\geq0}$ %$\{a_t,e_t\}_{t>0}$ 
satisfy the following recursive relation for a fixed $\ka\in \mathbb{N}$:
\begin{align}
a_{t} &\leq 
\begin{cases}
    \rho' a_{t-\ka} + \frac{b}{\ka} \sum_{i=t-\ta}^{t-1} a_{i}  + c  \sum_{i=t-\ta}^{t-1} e_{i} + r  & \textit{if }t\geq \ta \label{rel} \vspace{1em} \\
    \rho'' a_{0} + \frac{b}{\ka} \sum_{i=0}^{t -1} a_{i} + c  \sum_{i=0}^{t-1}  e_{i} + r & \textit{if }t < \ta %\label{rel1}
\end{cases}
\end{align}
%for any $t \geq \ka$ for a fixed $\ka\in \mathbb{N}$:
% \begin{align}\label{rel}
% a_{t} &\leq \rho' a_{t-\ka} + \frac{b}{\ka} \sum_{i=t-\ta}^{t-1} a_{i}  + c  \sum_{i=t-\ta}^{t-1} e_{i} + r ,\qquad t\geq \ta
% \end{align}
% and for $t<\ta$:
% \begin{align}\label{rel1}
% a_{t} &\leq \rho'' a_{0} + \frac{b}{\ka} \sum_{i=0}^{t -1} a_{i} + c  \sum_{i=0}^{t-1}  e_{i} + r,\qquad \,t<\ta
% \end{align}
where $b,\,c,\,r,\,\rho''$ are non-negative constants satisfying $b \leq \rho'/4$ and $\rho'\in \left(0,1/4\right]$. Then, for any $t \in \mathbb{N}$,% we have,
%with $\rho := 1-2\rho' $ the following bound:
\begin{align}\label{lem5_main}
a_t &\leq 20\rho''\Big(1- \frac{3\rho}{4\ka}\Big)^t a_0 + 60 c\sum_{i=0}^{t-1}   \Big(1- \frac{3\rho}{4\ka}\Big)^{t-i}e_i +\frac{26 r}{\rho},
\end{align}
where $\rho := 1-2\rho'$.
\end{lem}
The proof of this lemma is provided in Appendix \hyperref[sec:apndxIV]{IV}. We are ready to state and prove the main convergence result. 

\begin{comment}
\begin{lem}\label{lem4}
Suppose the non-negative scalar sequences $\{a_t\}_{t\geq0}$ and $\{e_t\}_{t\geq0}$ %$\{a_t,e_t\}_{t>0}$ 
satisfy the following recursive relation for any $t \geq \ka$ for a fixed $\ka\in \mathbb{N}$:
\begin{align}\label{rel}
a_{t} &\leq \rho' a_{t-\ka} + \frac{b}{\ka} \sum_{i=t-\ta}^{t-1} a_{i}  + c  \sum_{i=t-\ta}^{t-1} e_{i} + r ,\qquad t\geq \ta
\end{align}
and for $t<\ta$:
\begin{align}\label{rel1}
a_{t} &\leq \rho'' a_{0} + \frac{b}{\ka} \sum_{i=0}^{t -1} a_{i} + c  \sum_{i=0}^{t-1}  e_{i} + r,\qquad \,t<\ta
\end{align}
where $b,\,c,\,r$ are non-negative constants satisfying $b \leq \rho'/4$ and $\rho'\in \left(0,1/4\right)$. Then, for any $t\geq2\ta$,% we have,
%with $\rho := 1-2\rho' $ the following bound:
\begin{align}\label{lem5_main}
a_t &\leq 20\rho''\Big(1- \frac{3\rho}{4\ka}\Big)^t a_0 + 60 c\sum_{i=0}^{t-1}   \Big(1- \frac{3\rho}{4\ka}\Big)^{t-i}e_i +\frac{24 r}{\rho},
\end{align}
where $\rho := 1-2\rho'$.
\end{lem}
The proof of this lemma is provided in Appendix~\ref{sec:apndxIV}. We are ready to state and prove the main convergence result. 
\end{comment}


\subsection*{Main Result}
For convenience, we define $\dd_k$ as
\begin{align}
\dd_k \defeq \EX \left[\|\bar{x}_{k} -x^*\|^2\right], \qquad \forall k \in \N. 
\end{align}
where $x^*$ is the optimal solution of (\ref{mainprob}). 

\begin{thm}\label{thm1}
\emph{[\textbf{Convergence rate of IC-GT}]}
%\emph{[\textbf{Computational complexity of IC-GT}]}
Suppose Assumptions \ref{asmp1}-\ref{asmp4} hold and $\bm{\epsilon}_{k,c} = \hat{\bm{\epsilon}}_{k,c}$ in (\ref{mat0})-(\ref{matII}) for all $k\in \mathbb{N}$. %are satisfied. 
If
%Suppose that 
\begin{equation}\label{alphacond}
    \alpha  \leq \min\left\{1, \frac{1}{161280 \ka L}\right\}  \quad \mbox{and} \quad 0<\gamma <1/4,
\end{equation}
where 
\begin{equation}\label{kappabd}
\ka = \left\lceil\tfrac{2}{\gamma(1 - \lambda_2)}\max\left\{4\ln  \left(\tfrac{2}{\gamma (1-\lambda_2)}\right)\,,\, \gamma (1-\lambda_2) + \ln 16 \right\}\right\rceil.
\end{equation}
%is a positive constant satisfying (\ref{kappaprop}) with $a = \frac{4}{5}\rho'$ for any $\rho' \in (0,1/4)$. 
Then, for any $T\in \mathbb{N}$, we have, 
\begin{align}\label{finalresult}
\dd_{T} &\leq  (1-\alpha\mu/4 )^T\left(\dd_0 +  \frac{800(1+\ka^2)L}{n (1-\alpha\mu/4)\mu}\| \Psi_{0}\|^2 \right)  + \left( \frac{4\alpha}{\mu} + \frac{101920\,  L (\ka+1) n\alpha^2}{ \mu}\right)\frac{\sigma_g^2}{n}\nonumber\\
&+ \left( \frac{4(1+2\mu^{-1}T\alpha)}{\mu} \frac{\gamma^2}{\alpha}+ \frac{33280(2 + \alpha^2 (\ka^2+\frac{1}{2}) + \alpha^2T)L}{ \mu}n \ka \gamma^2 \right)\frac{\sigma^2_{c}}{n}.  \end{align}
\end{thm}
We make the following remarks regrading Theorem \ref{thm1}. 
%Several remarks are in order regarding Theorem \ref{thm1}.

\begin{remark}
  \text{(\textbf{Dependence of $\ka$ on network})} 
The parameter $\ka$ depends on the network connectivity $(\lambda_2)$ and the noise attenuation parameter $\gamma$ (cf. \ref{kappabd}) which highlights the role played by $\gamma$ in shaping the consensus properties of \texttt{IC-GT} (cf. Lemma \ref{r2}). From (\ref{kappabd}), we note that a smaller value of $\gamma$ increases $\tau$ but reduces the impact of the communication noise variance $\sigma^2_c$ in (\ref{finalresult}) which is reminiscent of the trade-off discussed in Section \ref{sec:algo}.
\end{remark}



\begin{remark} \textbf{(Iteration complexity of \texttt{IC-GT})} (\ref{alphacond}) and (\ref{kappabd}) suggest that the choices of the step size $\alpha$ and the noise attenuation parameter $\gamma$ are inherently connected. Using (\ref{kappabd}) in (\ref{alphacond}), we have the following relation:
\begin{equation}\label{50}
\frac{\alpha}{\gamma} = %< 
\tilde{\mathcal{O}} \left(\frac{1-\lambda_2}{L}\right)
\end{equation}
To calculate the number of iterations $T$ required to reach $\epsilon$-accuracy, we note that the contribution of the gradient noise terms in (\ref{finalresult}) is given by
 \begin{equation}\label{var_red}
     \mathcal{O}\big(( \alpha +n\alpha^2  )\sigma_g^2/n\big) = \mathcal{O}\big( \alpha\sigma_g^2/n\big)\qquad \text{if }\alpha \leq 1/n
      \end{equation}
while the contribution  of the communication noise terms in (\ref{finalresult}) is given by:
    \begin{align}\label{arrstep}
     \mathcal{O}\Bigg( \left(\frac{(1+T\alpha)\gamma^2}{\alpha}+(\alpha^2 \ka^2 +\alpha^2T)n\ka \gamma^2 \right)\frac{\sigma_c^2}{n}\Bigg)
     &= \mathcal{\tilde{O}}\Bigg(\left( \frac{(1+T\alpha)\gamma^2}{\alpha}+ \frac{n \alpha^2}{\gamma} + n\alpha^2\gamma T\right)\frac{\sigma_c^2}{n}  \Bigg), 
    \end{align}
     where we used $\ka= \mathcal{\tilde{O}}(1/\gamma)$ and ignored the dependency on other problem parameters.  If we set $\gamma= \tilde{\mathcal{O}}(\alpha) $ such that (\ref{50}) is satisfied, the above bound further simplifies to 
    $$
   \mathcal{\tilde{O}}\Bigg(\left( \frac{(1+T\alpha)\gamma^2}{\alpha}+ \frac{n \alpha^2}{\gamma} + n\alpha^2\gamma T\right)\frac{\sigma_c^2}{n}  \Bigg)=  \mathcal{\tilde{O}}\left(\left((1+T\alpha)\alpha + n\alpha+n\alpha^3T\right)\frac{\sigma_c^2}{n}\right)
    $$
   For any given $\epsilon>0$, we can set $\alpha=\epsilon$ implying that $T = \mathcal{\tilde{O}}(\epsilon^{-1})$ iterations are required to achieve the specified $\epsilon$-accuracy.
\end{remark}



\begin{comment}
%\begin{itemize}
    \item The network dependence is captured by $\tau$ in (\ref{finalresult}). \rb{Suppose we set} $\rho=1/4$. Then, we recall from Lemma \ref{lem3} that to ensure $\|J_\gamma^\ka\|^2<1/5$, $\ka$ should satisfy
  \begin{equation*}
\ka = \max\left\{8\ln  \big(2/\gamma (1-\lambda_2)\big)/\gamma (1-\lambda_2)\,,\, 2\right\{ 1 +\frac{\ln n +1.6 }{2\gamma (1-\lambda_2)}  \Big\}\Big\}
\end{equation*}
Thus, $\ka = \tilde{\mathcal{O}}(1/\gamma)$, where $\tilde{\mathcal{O}}(\cdot)$ hides logarithmic factors. This restricts us to $\alpha,\gamma$ which satisfy $\alpha < \gamma$ since Theorem \ref{thm1} would require (with $\rho=1/4$) that $\alpha\ka  < 1/16\sqrt{480} L $.
\end{comment}
% Note that as $\gamma \to 0$, we have $I-\gamma\Q\to I$ implying that we cannot achieve consesnus.

   \begin{remark} \textbf{($\mathbf{\sigma_c^2=0,\,\sigma_g^2=0}$ and $\mathbf{\sigma_c^2=0,\,\sigma_g^2>0}$):} In the absence of communication or gradient approximation errors ($\sigma_c^2=0,\,\sigma_g^2=0$), we can achieve the deterministic linear convergence rate of the gradient tracking algorithm \cite{DIG}. Referring to equation (\ref{finalresult}), we obtain the following inequality:
    $$
    \dd_{T} \leq(1-\alpha\mu/4 )^T\left(\dd_0 +  \frac{800(1+\ka^2) L}{n (1-\alpha\mu/4 )\mu}\| \Psi_{0}\|^2 \right)
    $$
The case $\sigma_c^2=0,\,\sigma_g^2>0$ considers stochastic decentralized optimization with no communication noise. For this scenario, with a constant $\alpha>0$, we have linear convergence to a neighbourhood of size $\mathcal{O}\big((\alpha^2 n + \alpha)\sigma_g^2/n\big)$ \cite{pudi}. A point to be remarked here is that \texttt{IC-GT} not only removes the data heterogeneity terms which arise in the convergence bound for \texttt{DGD} (cf. Table 1) but also makes sure that the variance scales linearly with the number of nodes provided $\alpha \leq 1/n$ (cf. \eqref{var_red}).
    

%\end{comment}
  \end{remark} 
\rb{
  \begin{remark} \textbf{(Consensus Error):} 
  We can establish convergence error bounds for the expected consensus error $\mathbb{E}[\|\Psi_k\|^2]$ by combining the results of Lemma \ref{r2} and Theorem \ref{thm1}. However, for brevity, we omit the explicit presentation of these results as they are of the same order as the results for $\Delta x^*_T$.
  %However, to maintain brevity, the specific convergence results for $\mathbb{E}[\|\Psi_k\|^2]$ are not explicitly presented here since these results are of the same order as the results for $\Delta x^*_T$. 
  %For the sake of brevity, the convergence error bounds on the expected consensus error $\mathbb{E}[\|\Psi_k\|^2]$ are not explicitly mentioned in Theorem \ref{thm1} since they are of the same order as $\Delta x^*_T$. We can derive the required bounds by combining the results of Lemma \ref{r2} and Theorem \ref{thm1}.
%\end{comment}
  \end{remark} 
}

\subsubsection*{Proof of Theorem \ref{thm1}}
% The road map for the proof of Theorem \ref{thm1} is as follows:

% \begin{enumerate}[(i)]
%     \item We first provide Lemma~\ref{r1} that establishes one-step \emph{descent} for the iterates generated by the algorithm with inexact communication using consensus error, and the variance in the gradient estimation and communication noise.
%     \item We then analyze the consensus error, $\Psi_t$, and derive a descent relationship for it by expressing it in terms of $\Psi_{t+i},\,0\leq i \leq \tau-1$. $\ka$ used here is as in Lemma \ref{lem3} and arises since the contractive property, $\|J^{\ka'}_\gamma\|^2<1$, is only satisfied when $\ka' \geq \ka$.
%     \item Finally, we prove Theorem~\ref{thm1} using Lemma~\ref{r1}, Lemma~\ref{r2} and an auxiliary result (Lemma~\ref{lem4}) about sequences proved in Appendix III.  
% \end{enumerate}

%(i) We begin by proving a descent result (Lemma \ref{r1}) for distributed gradient descent on strongly convex, smooth functions with inexact communication which involves the consensus error and the variance of the gradient and the communication noise.
Using (\ref{mat0}) and recalling that $\xx := \frac{(1_n\1_n^T) \otimes I_d}{n}\xb$,  the recursion for $\xx_k$ can be expressed as
\begin{align}\label{1-1}
\xx_{k+1} &= \bar{\textbf{v}}_k - \alpha \bar{\textbf{y}}_k \nonumber\\
&=\xx_{k}+ \gamma\bar{\bm{\epsilon}}_{k,c} -\alpha \bar{\textbf{y}}_k ,
\end{align}
where $\bar{\bm{\epsilon}}_{k,c}:=  \frac{1}{n} \big(1_n 1_n^T\otimes I_d \big)\hat{\textbf{Q}} \bm{\epsilon}_{k,c}$ and the last equality is due to $\bar{\textbf{v}}_k = \bar{\xb}_k+\gamma \bar{\bm{\epsilon}}_{k,c}$. Similarly, the recursion for $\bar{\mathbf{y}}_k:= \frac{1}{n} \big(1_n1_n^T\otimes I_d \big)y_k$ can be given as,
$$
\bar{\mathbf{y}}_{k} = \bar{\mathbf{y}}_{k-1} +   \frac{1}{n} \big(1_n1_n^T\otimes I_d \big) \big(\n \textbf{F} (\xb_{k},\bm{\xi}_{k}) - \n \textbf{F} (\xb_{k-1},\bm{\xi}_{k-1}) \big) + \gamma \bar{\bm{\epsilon}}_{k-1,c}.
$$
Taking telescopic sum from $0$ to $k$ leads to the following recursion: 
%which on iterating gives:
\begin{equation}\label{eq:bary}
\bar{\mathbf{y}}_{k} =  \frac{1}{n} \big(1_n1_n^T\otimes I_d \big)  \n \textbf{F} (\xb_{k},\bm{\xi}_{k}) +\gamma  \sum_{j=1}^{k}\bar{\bm{\epsilon}}_{j-1,c}
\end{equation}
since $\bar{\mathbf{y}}_0 =\frac{1}{n} \big(1_n1_n^T\otimes I_d \big)  \n \textbf{F}(\xb_{0},\bm{\xi}_{0}) $. Plugging \eqref{eq:bary} %the above equation 
in \eqref{1-1}, we get,
\begin{align}\label{ll}
\xx_{k+1} &= \xx_{k}   + \gamma\bar{\bm{\epsilon}}_{k,c} -   \frac{\alpha}{n} \big(1_n1_n^T\otimes I_d \big) \n \textbf{F} (\xb_{k},\bm{\xi}_{k})   - \gamma \alpha \sum_{j=0}^{k-1}\bar{\bm{\epsilon}}_{j,c}\nonumber\\
&= \bar{\textbf{x}}_{k}-   \frac{\alpha}{n} \big(1_n1_n^T\otimes I_d \big) \n \textbf{f} (\xb_{k})   + \gamma\left( \bar{\bm{\epsilon}}_{k,c} - \alpha \sum_{j=0}^{k-1}\bar{\bm{\epsilon}}_{j,c}\right)+ \frac{\alpha}{n} \big(1_n1_n^T\otimes I_d \big)\Big( \n \textbf{f}(\xb_{k}) -  \n\textbf{ F} (\xb_{k},\bm{\xi}_k)\Big)\nonumber\\
&= \bar{\textbf{x}}_{k}-   \frac{\alpha}{n} \big(1_n1_n^T\otimes I_d \big) \n \textbf{f} (\xb_{k})  + \underbrace{\alpha \bm{\epsilon}_{k,g} + \gamma \bar{\bm{\epsilon}}_{k,c}}_{\delta_k}  - \alpha\gamma \sum_{j=0}^{k-1}\bar{\bm{\epsilon}}_{j,c}
\end{align}
where $\bm{\epsilon}_{k,g} $ is defined to be $\bm{\epsilon}_{k,g} := \frac{1}{n} \big(1_n1_n^T\otimes I_d \big)\Big( \n \textbf{f}( \xb_{k}) -  \n \textbf{F} (\xb_{k},\bm{\xi}_k)\Big)$ with $\EX\, [\bm{\epsilon}_{k,g}|\xb_k]  = 0 $ and $\EX [\|\bm{\epsilon}_{k,g} \|^2]\leq \sigma_g^2$ from Assumption \ref{asmp4}. Now, let $\mathcal{F}_k\defeq 
\sigma(\xb_{0}, \bm{\xi}_{0}, \bm{\epsilon}_{0,c}, \cdots, \bm{\xi}_{k-1}, \bm{\epsilon}_{k-1,c})$
%$\mathcal{F}_k\defeq \sigma(v_{i,k'},x_{i,k'},y_{i,k'}\,, i\in[n], k'\leq k)$ 
be the sigma algebra generated by the random variables up to iteration $k$. Then, for any constant $\beta>0$, we have,
\begin{align}\label{altnew}
&\EX [ \|\bar{\mathbf{x}}_{k+1}-\mathbf{x}^*\|^2|\mathcal{F}_k] \nonumber \\
&~~~~~\leq (1+\beta)\EX [\|\bar{\mathbf{x}}_{k}-   \frac{\alpha}{n} \big(1_n1_n^T\otimes I_d \big) \n \textbf{f} (\xb_{k})   -\mathbf{x}^* + \delta_k\|^2 |\mathcal{F}_k]  + (1+\beta^{-1})\alpha^2 \gamma^2\EX \left[ \left\|\sum_{j=0}^{k-1}\bar{\bm{\epsilon}}_{j,c} \right\|^2\Big|\mathcal{F}_k\right]\nonumber\\
 &~~~~~=  (1+\beta) \|\bar{\mathbf{x}}_{k}-   \frac{\alpha}{n} \big(1_n1_n^T\otimes I_d \big)\n \textbf{f} (\xb_{k})   -\mathbf{x}^*\|^2  \nonumber\\
 &~~~~~\qquad + (1+\beta)\EX[ \|\delta_k\|^2 |\mathcal{F}_k]+ (1+\beta^{-1})\alpha^2 \gamma^2\EX \left[ \left\|\sum_{j=0}^{k-1}\bar{\bm{\epsilon}}_{j,c} \right\|^2\Big|\mathcal{F}_k\right],
\end{align}
where the equality is due to $\EX [\delta_k| \mathcal{F}_k] = 0$ from Assumption \ref{asmp3}. %we used $\EX [\delta_k| \mathcal{F}_k] = 0$ in the equality from Assumptions \ref{asmp3} and \ref{asmp4}. 
From Assumptions \ref{asmp2} and \ref{asmp4}, we have, 
\begin{equation}\label{del1}
      \EX [\|\delta_k\|^2] = \EX[\left\|\alpha \bm{\epsilon}_{k,g} + \gamma \bar{\bm{\epsilon}}_{k,c}\right\|^2]\leq\alpha^2 \sigma_g^2 + \gamma^2\sigma^2_{c}, 
\end{equation}
where we have used $\EX [\langle \bm{\epsilon}_{k,g}, \bar{\bm{\epsilon}}_{k,c}\rangle] = 0$. 
Furthermore, we have,
\begin{align}\label{epsbound}
  \EX \left[\left\| \sum_{j=0}^{k-1}\bar{\bm{\epsilon}}_{j,c} \right\|^2\right] &=  \EX \left[ \sum_{j=0}^{k-1}\left\|\bar{\bm{\epsilon}}_{j,c} \right\|^2\right]  + \sum_{1\leq p,p' \leq k-1}\EX \left[\langle \bar{\bm{\epsilon}}_{p,c} ,\bar{\bm{\epsilon}}_{p'
  ,c} \rangle \right]  %\nonumber\\
  \leq \sum_{j=0}^{k-1} \sigma_c^2 =  k\sigma_c^2,
\end{align}
where we use
%we have neglected the cross terms in the second inequality using 
$ \EX [\langle \bar{\bm{\epsilon}}_{p,c}, \bar{\bm{\epsilon}}_{p',c}\rangle] =  \EX [ \EX  \left[\langle \bar{\bm{\epsilon}}_{p,c}, \bar{\bm{\epsilon}}_{p',c}\rangle|\mathcal{F}_{p'}\right]]=0$ for $p<p'$.
Taking full expectations in (\ref{altnew}), it then follows that,
\begin{multline*}
 \EX  [\|\bar{\mathbf{x}}_{k+1}-\mathbf{x}^*\|^2 ]\leq (1+\beta)\EX \left[\left\|\bar{\mathbf{x}}_{k}-   \frac{\alpha}{n} \big(1_n1_n^T\otimes I_d \big)\n \textbf{f} (\xb_{k})  -\mathbf{x}^* \right\|^2\right]\\ + \big( (1+\beta)(\gamma^2\sigma_c^2 + \alpha^2\sigma_g^2) +k(1+\beta^{-1})\alpha^2\gamma^2\sigma^2_{c} \big).
\end{multline*}
where we used (\ref{del1}) to get the inequality.
We note that since $ \|\bar{\mathbf{x}}_{k+1}-\mathbf{x}^*\|^2 = \big\|\frac{(1\1^T) \otimes I_d}{n} (\textbf{x}_{k+1} - \textbf{x}^*)\big\|^2=n\|\bar{x}_{k+1}-x^*\|^2$, the above inequality leads to,
 \begin{multline}\label{ll0}
 \EX \left[\|\bar{x}_{k+1}-x^*\|^2\right] \leq (1+\beta) \EX  \left[ \left\|\bar{x}_{k}- \frac{\alpha }{n} \sum_{i=1}^n \n f_i (x_{i,k}) -x^* \right\|^2 \right]\\ + n^{-1}\big( (1+\beta)(\gamma^2\sigma_c^2 + \alpha^2\sigma_g^2) +k(1+\beta^{-1})\alpha^2\gamma^2\sigma^2_{c} \big).
\end{multline}
%where we used (\ref{del1}) to get the last inequality. \rbnote{(It is a bit unsatisfactory to see the need for each function to be strongly convex. I think it is possible to get the results using just strong convexity requirement on the global function. This is possible, if we try to do the following...please check if you can proceed in this direction...
% \begin{align*}
%    \bar{\textbf{x}}_{k+1} &= \bar{\textbf{x}}_{k}-   \frac{\alpha}{n} \big(11^T\otimes I_d \big) \n \textbf{f} (\bar{\textbf{x}}_{k}) +   \frac{\alpha}{n} \big(11^T\otimes I_d \big) (\n \textbf{f} (\bar{\textbf{x}}_{k}) - \n \textbf{f} (\xb_{k}))  + \underbrace{\alpha \bm{\epsilon}_{k,g} + \gamma \bar{\bm{\epsilon}}_{k,c}}_{\delta_k}  - \alpha\gamma \sum_{j=0}^{k-1}\bar{\bm{\epsilon}}_{j,c}
% \end{align*}
% Using this equation, you can derive...
% \begin{align*}
%   &\EX [ \|\bar{\mathbf{x}}_{k+1}-\mathbf{x}^*\|^2|\mathcal{F}_k] \nonumber \\
% &~~~~~\leq (1+\beta)\EX [\|\bar{\mathbf{x}}_{k}-   \frac{\alpha}{n} \big(11^T\otimes I_d \big) \n \textbf{f} (\bar{\mathbf{x}}_{k})   -\mathbf{x}^* + \delta_k\|^2 |\mathcal{F}_k]  \\
% &~~~~~\quad+ (1+\beta^{-1})\alpha^2 \EX \left[ \gamma^2\left\|\sum_{j=0}^{k-1}\bar{\bm{\epsilon}}_{j,c} \right\|^2 +   \|\frac{1}{n} \big(11^T\otimes I_d \big) (\n \textbf{f} (\bar{\textbf{x}}_{k}) - \n \textbf{f} (\xb_{k}))\|^2\Big|\mathcal{F}_k\right]\nonumber\\
%  &~~~~~=  (1+\beta) \|\bar{\mathbf{x}}_{k}-   \frac{\alpha}{n} \big(11^T\otimes I_d \big)\n \textbf{f} (\bar{\mathbf{x}}_{k})   -\mathbf{x}^*\|^2  \nonumber\\
%  &~~~~~\qquad + (1+\beta)\EX[ \|\delta_k\|^2 |\mathcal{F}_k]+ (1+\beta^{-1})\alpha^2 \gamma^2\EX \left[ \left\|\sum_{j=0}^{k-1}\bar{\bm{\epsilon}}_{j,c} \right\|^2\Big|\mathcal{F}_k\right] \\
%  &~~~~~\leq (1+\beta)n(1 - \alpha \mu)^2\|\bar{x}_k - x^*\|^2 + .....(fill in)
% \end{align*}
% where ..
% )
% }
Considering the first term on the right hand side of \eqref{ll0}, we have,
\begin{align}\label{ll2}
    \Big\| \x_{k} - \frac{\alpha}{n} \sum_{i=1}^n\n f_i (x_{i,k}) -x^* \Big\|^2 &=  \| \x_k -x^* \|^2 -\frac{2\alpha}{n} \left\langle   \sum_{i=1}^n\n f_i (x_{i,k}), \x_{k}-x^* \right\rangle +\alpha^2 \Big\|\frac{1}{n}  \sum_{i=1}^n\n f_i (x_{i,k}) \Big\|^2.
\end{align}
The second term on the right hand side of \eqref{ll2} can be bounded as 
\begin{align}\label{bd2-}
     \langle   \sum_{i=1}^n\n f_i (x_{i,k}), \bar{x}_{k}-x^* \rangle &=  \langle   \sum_{i=1}^n\n f_i (x_{i,k}), \bar{x}_{k}-x_{i,k} \rangle +  \langle   \sum_{i=1}^n\n f_i (x_{i,k}), x_{i,k}-x^* \rangle \nonumber \\
     &\geq\sum_{i=1}^n \Big[ f_i(\bar{x}_k) -f_i(x_{i,k})- \frac{L}{2}\|\bar{x}_k -x_{i,k}\|^2 + f_i(x_{i,k}) - f_i(x^*)+ \frac{\mu}{2}\|x_{i,k}-x^*\|^2\Big] \nonumber\\
     &\geq\sum_{i=1}^n \Big[ f_i(\bar{x}_k) -f_i(x^*) -  \frac{L+\mu}{2} \|\bar{x}_k -x_{i,k}\|^2 +\frac{\mu }{4}\|\bar{x}_k -x^*\|^2\Big],
\end{align}
where the second inequality is due to Assumption~\ref{asmp3} and the last inequality is due to the inequality $ \|\bar{x}_k -x^*\|^2 \leq 2 \|\bar{x}_k-x_{i,k}\|^2 + 2\|x_{i,k} -x^*\|^2$. The last term on the right hand side of \eqref{ll2} can be bounded as
\begin{align}\label{bd1-}
     \Big\|\frac{1}{n}  \sum_{i=1}^n\n f_i (x_{i,k}) \Big\|^2&=  \Big\|\frac{1}{n}  \sum_{i=1}^n\n f_i (x_{i,k})-\frac{1}{n}  \sum_{i=1}^n\n f_i (\bar{x}_{k})+\frac{1}{n}  \sum_{i=1}^n\n f_i (\bar{x}_{k})-\frac{1}{n}  \sum_{i=1}^n\n f_i (x^*) \Big\|^2\nonumber\\
     &\leq  \frac{2L^2}{n} \sum_{i=1}^n \| \bar{x}_k-x_{i,k}\|^2+ \frac{4L}{n} \sum_{i=1}^n (f_i(\bar{x}_k) -f_i(x^*)),
\end{align}
where in the second summation, we used the fact that $\|\n f_i (\bar{x}_{k})  - \n f_i (x^*) \|^2 \leq 2L (f_i(\bar{x}_k) -f_i(x^*))$ by Assumption \ref{asmp3} \cite[Theorem 2.1.5]{yuri}. Using (\ref{bd2-}) and (\ref{bd1-}) in (\ref{ll2}) along with $\alpha<1/4L$, we have, 
\begin{align}\label{ll01}
    \left\| \bar{x}_{k+1} -  x^*\right\|^2 &\leq (1-\alpha\mu/2) \| \bar{x}_k -x^* \|^2  - \frac{\alpha}{n}\big(\sum_{i=1}^n f_i(\bar{x}_k) -f_i(x^*)\big) +\frac{(3L/2+\mu)\alpha}{n} \sum_{i=1}^n \|\bar{x}_k-x_{i,k}\|^2\nonumber\\
    &\leq (1-\alpha\mu/2) \| \bar{x}_k -x^* \|^2  - \alpha \left( \textbf{f}(\bar{\textbf{x}}_k) -\textbf{f}(\textbf{x}^*)\right) +\rb{\frac{5\alpha L}{2n}} \|\Psi_k\|^2,
\end{align}
where the last inequality is due to $\|\Delta \textbf{x}_k\|^2 \leq \|\Psi_k \|^2$.
%we have used $\|\Delta \textbf{x}_k\|^2 \leq \|\Psi_k \|^2$  in the last inequality. 
Using (\ref{ll01}) in (\ref{ll0}), we get,
\begin{align*}
\EX[\|\bar{x}_{k+1}  -x^*\|^2]   &\leq (1+\beta)\Bigg\{(1-\alpha\mu/2) \EX [\| \x_k -x^* \|^2]  -  \alpha \left( \EX[\textbf{f}(\bar{\textbf{x}}_k)] -\textbf{f}(\textbf{x}^*)\right) \\ &\qquad +\rb{\frac{5\alpha L}{2n}} \EX[\|\Psi_k\|^2]\Bigg\} +\frac{1}{n}\big( (1+\beta)(\gamma^2\sigma_c^2 + \alpha^2\sigma_g^2) +k(1+\beta^{-1})\alpha^2\gamma^2\sigma^2_{c} \big).
\end{align*}
Set $\beta\defeq \frac{\alpha \mu}{4(1-\frac{\alpha \mu}{2})}$. We note that $(1+\beta^{-1}) \leq 4/\alpha \mu$ and $(1+\beta)= \frac{(1-\alpha\mu/4)}{(1-\alpha\mu/2)}$ with $1\leq (1+\beta)\leq 2$. Then, we have,
\begin{align}\label{r1}
\EX\left[\|\bar{x}_{k+1}  -x^*\|^2\right]   &\leq \Big\{(1-\alpha\mu/4) \EX \left[\| \x_k -x^* \|^2\right]  - \alpha \left( \EX[ \textbf{f}(\bar{\textbf{x}}_k)] -\textbf{f}(\textbf{x}^*)\right) \nonumber\\ &\qquad +\frac{5\alpha L}{n} \EX [\|\Psi_k\|^2]\Big\} +\frac{1}{n}\big(  (2+4\mu^{-1}k\alpha)\gamma^2 \sigma^2_{c} +2\alpha^2 \sigma_g^2\big). 
\end{align}
Multiplying both sides of \ref{r1} by $w_{k+1}\defeq (1 -\alpha \mu/4)^{-(k+1)}$, we have,
\begin{align}
w_{k+1}\dd_{k+1}&\leq 
(1-\alpha\mu/4) w_{k+1}\dd_{k} +\frac{5\alpha L}{n} w_{k+1}\EX [\|\Psi_k\|^2]\nonumber\\
&\qquad- \alpha w_{k+1}(\EX[ \textbf{f}(\mathbf{\x}_k)]-\textbf{f}(\mathbf{x}^*)) +\frac{w_{k+1}}{n}\big( \gamma^2 (2+4k\mu^{-1}\alpha)\sigma^2_{c} +2\alpha^2 \sigma_g^2\big) \nonumber.
\end{align}
Rearranging the terms, we get,
\begin{align}
 0  &\leq w_{k} \dd_k-w_{k+1}\dd_{k+1}  +\frac{5\alpha L}{n} w_{k+1} \EX [\|\Psi_k\|^2 ]\nonumber\\
&\qquad - \alpha w_{k+1}(\EX\, [\textbf{f}(\mathbf{\x}_k) ]-\textbf{f}(\mathbf{x}^*)) +\frac{w_{k+1}}{n}\big( \gamma^2 (2+4k\mu^{-1}\alpha)\sigma^2_{c} +2\alpha^2 \sigma_g^2\big) \nonumber \nonumber.
\end{align}
Summing the above inequality from $k=0$ to $T-1$, we get,
\begin{align}\label{intm22}
w_T\dd_{T} &\leq w_{0} \dd_0 +\frac{1}{n}\big( \gamma^2 (2+4T\mu^{-1}\alpha)\sigma^2_{c} +2\alpha^2 \sigma_g^2\big) \sum_{k=0}^{T-1}w_{k+1} + \frac{5\alpha L}{n}\sum_{k=0}^{T-1} w_{k+1}\EX[\|\Psi_k\|^2] \nonumber \\
&\qquad - \alpha \sum_{k=0}^{T-1} w_{k+1}(\EX [f(\mathbf{\x}_k)] -f(\mathbf{x}^*)).
\end{align}
We note that we can write the relations (\ref{main_rec})-(\ref{main_rec_2}) in Lemma \ref{r2} in the form of (\ref{rel}) %and (\ref{rel1}) 
with 
\begin{align}
   & b:=576 \alpha^2 L^2  \ka^2\qquad  c := 1344 \alpha^2\ka \nonumber \\
    &r:=   64\gamma^2 \big(2 + \alpha^2 (\ka^2 +1/2) + \alpha^2T\big)n\sigma_{c}^2 \ka+196n(\ka+1)\alpha^2\sigma^2_{g}\label{dd}\\
   & e_k:= \EX \left[\|\n \textbf{f}(\bar{\xb}_{k})-\n \textbf{f}(\xb^*) \|^2 \right]\nonumber
\end{align}
and we have taken $\rho'=1/4$ which fixes $\ka$ in (\ref{kappabd}) according to the bound (\ref{kappaprop}) (cf. (\ref{rho'eq})). Note that since $\alpha <  \frac{\sqrt{\rho'}}{2\sqrt{576} L \ka} $, $b\leq \frac{\rho'}{4}=\frac{1}{16}$. Then, with $a_t \defeq\EX[ \| \Psi_{t}\|^2] $ in Lemma \ref{lem4}, we get,
\begin{align}\label{bss1}
\EX [ \| \Psi_{t}\|^2] &\leq 40(1+\tau^2) \left(1- \frac{3\rho}{4\ka}\right)^t \|  \Psi_{0}\|^2 + 60 c  \sum_{j=0}^{t-1}   \left(1- \frac{3\rho}{4\ka}\right)^{t-j}  \EX \left[ \|\n \textbf{f(}\bar{\xb}_{j})-\n \textbf{f}(\textbf{x}^*) \|^2\right] + 52 r
\end{align}
with $\rho'' = 2(1+\ta^2)\|\Psi_0\|^2$ and $\rho=1-2\rho'=1/2$. We next bound the summation $\sum_{k=0}^{T-1} w_{k+1} \EX \|\Psi_k\|^2$ in (\ref{intm22}). To do this, we multiply both sides  of (\ref{bss1}) by $w_{k+1}$ and sum from $t=0$ to $T-1$: 
\begin{align}\label{intm101}
\sum_{k=0}^{T-1} (1-\alpha\mu/4)^{-(k+1)}  \EX \|\Psi_k\|^2 &\leq   40(1+\ka^2)\|\Psi_0\|^2\sum_{k=0}^{T-1} (1-\alpha\mu/4)^{-(k+1)} \left(1- \frac{3\rho}{4\ka}\right)^k\nonumber \\
&\qquad +   60c\sum_{k=0}^{T-1}(1-\alpha\mu/4)^{-(k+1)} \sum_{j=0}^{k-1} \left(1- \frac{3\rho}{4\ka}\right)^{t-j} e_j + 52 r  W_{T-1},
\end{align}
where $W_{T-1} =  \sum_{k=0}^{T-1} w_{k+1}$.  From (\ref{alphacond}), we have,
 \begin{equation}\label{a-r}
     \alpha \mu/2 \leq 3\rho/4\ka \implies  \alpha \mu/2(1-\alpha\mu/8) \leq 3\rho/4\ka \implies 1-3\rho/4\ka \leq (1-\alpha \mu/4)^2.
 \end{equation}
 We use (\ref{a-r}) to bound the two summations on the right hand side of (\ref{intm101}) as follows:
 \begin{equation}\label{fbd1}
\sum_{k=0}^{T-1} (1-\alpha\mu/4)^{-(k+1)} \Big(1- \frac{3\rho}{4\ka}\Big)^k \leq \sum_{k=0}^{T-1} (1-\alpha\mu/4)^{k-1} \leq \frac{4w_1}{\alpha \mu}, 
\end{equation}
and 
\begin{align}
     \sum_{k=0}^{T-1}(1-\alpha\mu/4)^{-(k+1)} &\sum_{j=0}^{k-1} \Big(1- \frac{3\rho}{4\ka}\Big)^{k-j} e_j= \sum_{k=0}^{T-1}\sum_{j=0}^{k-1} (1-\alpha\mu/4)^{-(k+1)+j+1}  \Big(1- \frac{3\rho}{4\ka}\Big)^{k-j} w_{j+1} e_j \nonumber\\
     &~~~~~= \sum_{k=0}^{T-1}\sum_{j=0}^{k-1}  \Bigg(\frac{1- 3\rho/4\ka}{1-\alpha\mu/4}\Bigg)^{k-j} w_{j+1} e_j %\stackrel{(\ref{a-r})}{\leq}
     \leq
     \sum_{k=0}^{T-1}\sum_{j=0}^{k-1}  (1-\alpha\mu/4)^{k-j} w_{j+1} e_j \nonumber \\
     &~~~~~\leq \sum_{k=0}^{T-1}  (1-\alpha\mu/4)^{k} \sum_{k=0}^{T-1} w_{k+1} e_k \leq \frac{4}{\mu \alpha}\sum_{k=0}^{T-1} w_{k+1} e_k \label{fbd2},
\end{align}
%\begin{multline}\label{fbd2}
%     \sum_{k=0}^{T-1}(1-\alpha\mu/4)^{-(k+1)} \sum_{j=0}^{k-1} \Big(1- \frac{3\rho}{4\ka}\Big)^{k-j} e_j \leq \sum_{k=0}^{T-1}\sum_{j=0}^{k-1} (1-\alpha\mu/4)^{-(k+1)+j+1}  \Big(1- \frac{3\rho}{4\ka}\Big)^{k-j} w_{j+1} e_j \\
%    \leq \sum_{k=0}^{T-1}\sum_{j=0}^{k-1}  \Bigg(\frac{1- 3\rho/4\ka}{1-\alpha\mu/4}\Bigg)^{k-j} w_{j+1} e_j \stackrel{(\ref{a-r})}{\leq}
%     \sum_{k=0}^{T-1}\sum_{j=0}^{k-1}  (1-\alpha\mu/4)^{k-j} w_{j+1} e_j \\
%     \leq \sum_{k=0}^{T-1}  (1-\alpha\mu/4)^{k} \sum_{k=0}^{T-1} w_{k+1} e_k \leq \frac{4}{\mu \alpha}\sum_{k=0}^{T-1} w_{k+1} e_k,
%\end{multline}
where the first equality is due to \eqref{a-r} and the second inequality is obtained using the relation $\sum_{k=0}^{T-1} \sum_{j=0}^{k-1}a_{k-j}b_j \leq\sum_{k=0}^{T-1} a_{k} \sum_{k=0}^{T-1} b_{k}$ for any two non-negative scalar sequences $a_k,\,b_k,\,k\geq 0$. Plugging the previous two bounds in (\ref{intm101}), we get,
\begin{align*}
\sum_{k=0}^{T-1} w_{k+1}  \EX[ \|\Psi_k\|^2] &\leq  \frac{ 160w_1(1+\ka^2) \|\Psi_0\|^2 }{\alpha\mu }+\frac{240 n cL}{\mu \alpha}\sum_{k=0}^{T-1}w_{k+1} \left(\EX[ \textbf{f}(\bar{\xb}_k)] -\textbf{f}(\textbf{x}^*)\right) +  52r W_{T-1},
\end{align*}
where we have additionally used the fact that $\|e_k\|^2 = \EX \left[ \|\n \textbf{f}(\mathbf{\x}_k))-\n \textbf{f}(\textbf{x}^*) \|^2 \right]\leq 2nL( \EX [\textbf{f}(\mathbf{\x}_k))] -\mathbf{f}(\textbf{x}^*)) $ from Assumption \ref{asmp3} \cite[Theorem 2.1.5]{yuri}. Finally, using the above bound in (\ref{intm22}), we get,
\begin{align*}
  w_T\dd_{T} &\leq w_{0} \dd_0 +\frac{1}{n}\big( \gamma^2 (2+4T\mu^{-1}\alpha)\sigma^2_{c} +2\alpha^2 \sigma_g^2\big) W_{T-1} \\
&\quad +\frac{5\alpha L}{n}\Bigg( \frac{160w_1(1+\ka^2)\|\Psi_0\|^2 }{\mu \alpha}+\frac{240ncL}{\mu \alpha}\sum_{k=0}^{T-1}w_{k+1} \left(\EX [\textbf{f}(\mathbf{\x}_k)) ]-\mathbf{f}(\textbf{x}^*)\right) +  52r W_{T-1}\Bigg) \\
 & \quad - \alpha \sum_{k=0}^{T-1}w_{k+1}(\EX [\,\textbf{f}(\mathbf{\x}_k)] -\mathbf{f}(\textbf{x}^*)).
\end{align*}
 Rearranging the terms in the above inequality and recalling that $c=1344 \alpha^2\ka$, we get, % (recalling that $c=1344 \alpha^2\ka$):
\begin{align*}
\dd_{T} &\leq \frac{1}{w_T} \Bigg\{\dd_0 + \frac{800w_1(1+\ka^2) L}{n \mu } \| \Psi_{0}\|^2 \Bigg\}+\frac{2}{\mu\alpha}\Bigg\{  \frac{\gamma^2 (2+4T\mu^{-1}\alpha)\sigma^2_{c}}{n} +\frac{2\alpha^2\sigma_g^2}{n}  \Bigg\}  \\
&\quad+\frac{ 520  L r}{ n\mu }  +\alpha\underbrace{\Bigg\{\frac{1612800\ka L^2}{\mu }\alpha-1\Bigg\}}_{\leq 0} \sum_{k=0}^{T-1}\frac{w_{k+1}}{w_T} \underbrace{(\EX [\mathbf{f}(\mathbf{\x}_k)] -f(\mathbf{x}^*))}_{\geq 0},
\end{align*}
where we used $ W_{T-1}/w_T \leq2/  \mu \alpha$. The last term on the right had side is less than zero due to the condition on $\alpha$ (see \eqref{alphacond}). Plugging the value of $r$ from (\ref{dd}) in the above inequality completes the proof.\qed 

%proves the theorem.

\section{Numerical Experiments}\label{sec:num}
In this section, we present an empirical evaluation of the performance of \texttt{IC-GT} through two sets of numerical experiments. %We present two sets of numerical experiments. 
The first set focuses on logistic regression on the MNIST dataset, while the second set 
%considers 
explores the effect of different %varying 
noise variances in a deep learning setting. All experiments were implemented using PyTorch, with a dedicated CPU core functioning as a node. 

% \subsection*{Effect of $\Q$ and $\gamma$ on distributed averaging}

% The formal statement for distributed averaging can be stated as:
% $$
% \min_{\substack{x^i \in \mathbb{R}^{d}\\i\in [n]}} \frac{1}{n}\sum_{i=1}^n\|x^i - \bar{x}_0\|^2, 
% $$
% where $\bar{x}_0:=\frac{1}{n} \sum_{i=1}^n x_0^i$ with $x_0^i$ being the observation held by node $i$. We consider the averaging scheme (\ref{qda}), referred to as $\Q$-DA, which uses the modified consensus matrix $\Q$ in place of $Q$. The performance and the effect of $\gamma$ on the convergence is studied and compared to the unmodified distributed averaging (DA) algorithm (\ref{bda}). For this experiment, we consider a 5-node network with ring topology.
% % We remark that the unmodified distributed averaging algorithm (\ref{bda}) can be obtained by taking $\gamma= 1$ in (\ref{qda}) making it a sub-case of (\ref{qda}). 

% The results have been plotted in Figure 1 from which the following can be inferred: (i) For the noiseless case ($\sigma^2_c=0$), the bias decay of the unmodified DA scheme is determined by $\lambda_2$ (cf. (\ref{da})) while that of $\Q$-DA is determined by $(1-\gamma(1-\lambda_2))$ (cf. (\ref{qaa})). The latter quantity is strictly smaller than $\lambda_2$ for $\gamma<1$ and one can notice from the figure that the slope of convergence increases as $\gamma$ decreases towards 0. (ii) The situation is reversed for $\sigma_c^2>0$ with $\Q$-DA having a better $\epsilon$-accuracy compared to  unmodified scheme. We further note that as $\gamma$ decreases, the $\epsilon$-accuracy improves which is as predicted by the bound in (\ref{qaa}) where the size of the final convergence neighbourhood is proportional to $\gamma$. For a too small $\gamma$, the bias decay becomes too slow indicating a standard bias-variance trade-off well observed in stochastic algorithms.

\subsection*{Logistic regression}
We first consider $\ell_2$ regularized logistic regression problems of the form,
%In this subsection, we perform a comparative analysis between IC-GT and contemporary methods for decentralized optimization, specifically focusing on regularized logistic regression. The formal statement for logistic regression can be stated as:
  \begin{equation}
       \min_{x\in \mathbb{R}^d} \, \left\{L(x;y,z) :=-\frac{1}{m} \sum_{i=1}^m\Big\{ z^i \log \varphi(x^Ty^i) + (1-z^i)\log (1-\varphi(x^Ty^i))\Big\} + \frac{\lambda}{2}\|x\|^2\right\}, 
    \end{equation}
where  $x \in \mathbb{R}^d$ denote the learnable model parameters, %learning model, 
$\{y^i,z^i\}_{i=1}^m$ denote the set of $m$ data points, $\varphi(\cdot)$ denotes the sigmoid function, and $\lambda>0$  is the regularization parameter. We use the \texttt{MNIST} dataset which consists of 60,000,  28$\times$28 pixel grayscale images of handwritten single digits between 0 and 9. The data is partitioned in a disjoint manner amongst the nodes by assigning each node $10^3$ data samples independently. 
% Figure environment removed
To simulate the inexact communication setting, we incorporate zero-mean Gaussian noise with a variance of $\sigma_c^2$ into the transmitted model estimates independently. We adopt a star topology with $n=10$ for the communication structure. In evaluating the performance, we employ the $\ell_2$ distance between the averaged variable $\x_k$ and the optimal point $x^*$. The optimal point $x^*$ is computed using the L-BFGS algorithm from the SciPy library in Python.
%To simulate the inexact communication setting, we introduce zero-mean Gaussian noise with variance $\sigma_c^2$ to the transmitted model estimates independently. We use a star topology with $n=10$. The performance metric used is the $\ell_2$ distance of the averaged variable $\x_k$ to the optimal point $x^*$, where $x^*$ is computed using the L-BFGS algorithm from the SciPy library in Python.
We also include the average consensus error as a performance metric, which is computed as $\frac{1}{|\mathcal{E}|}\sum_{(i,j)\in \mathcal{E}} \|x_i-x_j\|^2$, where $\mathcal{E}$ represents the edge set. We compare our proposed algorithm (\ty{IC-GT}) with several baselines, including the \ty{NEAR-SGD} algorithm from \cite{do4}, the \ty{EXTRA} algorithm proposed in \cite{EXTRA}, and the gradient tracking method \cite{sgt1}. Additionally, we include the performance of the \ty{DGD} algorithm for comparison purposes.

In our experiments, we set the batch size to $32$ and tune the step size $\alpha$ using a grid-search over the range $\alpha \in [10^{-4}, 1]$ to obtain the best performance for all the algorithms. The total number of communication rounds is set to $T=5\times 10^3$. For \ty{IC-GT}, we set the attenuation noise parameter $\gamma$ to $\gamma = \alpha \times \log T$.
The performance results are reported in Figure \ref{fg1}(a)-(b). From the plots, it is evident that \ty{IC-GT} outperforms all the other algorithms in terms of both the optimality error and the consensus error.

\begin{comment}
We also report the average consensus error, computed as $\frac{1}{|\mathcal{E}|}\sum_{(i,j)\in \mathcal{E}} \|x_i-x_j\|$, where $\mathcal{E}$ represents the edge set. The baselines used are the \ty{NEAR-SGD} algorithm of \cite{do4}, \ty{EXTRA} algorithm proposed in \cite{EXTRA} and the gradient tracking method \cite{sgt1}. The performance of \ty{DGD} is also plotted for comparison purposes. The batch size is kept equal to $32$ and the step size $\alpha$ is tuned using a grid-search over the range $\alpha \in  [10^{-4} , 1]$ to give the best performance for all algorithms. The total number of communication rounds is set to $T=5\times 10^3$. For \ty{IC-GT}, the attenuation noise parameter $\gamma$ is set to $\gamma =\alpha\times \log T$. The performance is reported in Figure \ref{fg1}(a)-(b). From the plots, it is evident that \ty{IC-GT} performs the best among all the algorithm both in terms of the optimality error as well as the consensus error.
\end{comment} 

To assess the scalability of \ty{IC-GT} and examine the impact of graph connectivity on its convergence accuracy, we conducted experiments with varying network sizes, specifically $n \in \{5, 10, 15, 20, 25\}$. We kept the noise variance fixed at $\sigma_c^2=0.01$ for the following graph topologies: (i) Fully connected (f.c.), (ii) Erdős-Rényi graph with an edge probability of 0.5 (rand), (iii) Ring topology, and (iv) Star topology. From Figure \ref{fg1}(c), we observe that as the graph connectivity deteriorates, the final performance of \ty{IC-GT} also deteriorates. In the case of a fully connected graph, there is an improvement in performance with an increasing number of nodes due to a decrease in gradient variance resulting from an increased effective mini-batch size. Finally, we also investigate the effect of varying $\sigma_c^2$ on the performance of \ty{IC-GT}, as depicted in Figure \ref{fg1}(d).

\begin{comment}
To evaluate the scalability of \ty{IC-GT} and the effect of graph connectivity on its convergence accuracy, we consider varying the network sizes as $n \in\{ 5, 10, 15, 20, 25\}$, while keeping $\sigma$ fixed as $\sigma_c^2=0.01$, for the following graph topologies: (i) Fully connected (f.c.) ii) Erd\H{o}s-R\'{e}nyi graph with edge probability 0.5 (rand) (iii) Ring topology (iv) Star topology (path). From Figure \ref{fg1}(c), we note that as the graph connectivity worsens, the final performance also deteriorates. For fully connected graph there is an improvement in performance with the number of nodes due to the decrease in gradient variance resulting from an increased effective mini-batch size. Finally, we also report the effect of varying $\sigma_c^2$ on the performance of \ty{IC-GT} in Figure \ref{fg1}(d).
\end{comment}
\begin{comment}
% Figure environment removed
\end{comment}
\begin{comment}
% Figure environment removed
\end{comment}


% % Figure environment removed  

\subsection*{Neural network based experiments}
In this subsection, we investigate a deep learning scenario that involves random compressed communication using probabilistic quantization (see \eqref{randquant}). We assume a star-based topology with $n=10$ for both the \ty{MNIST} and \ty{CIFAR} datasets. For the \ty{MNIST} dataset, we utilize a learning model with a total of $8.4$K parameters. This model comprises two convolution layers, the first with $250$ parameters and the second with $5$K parameters, followed by a fully connected layer with $3.2$K parameters. For the \ty{CIFAR-10} dataset, we adopt the standard \ty{LENET} architecture, which consists of three convolution layers and two fully connected layers. This architecture has a total of $0.54$M parameters. The configuration of the max-pooling and batch normalization layers follows the standard settings used in \ty{LENET} models.

%In this subsection, we consider a deep learning setting with random compressed communication using the probabilistic quantizer (\ref{randquant}). We assume star based topology wiht $n=10$ for the \ty{MNIST} and \ty{CIFAR} datasets. The learning model used for \ty{MNIST} has a total of $8.4$K parameters with two convolution layers of 250 and 5K parameters respectively followed by a fully connected layer of 3.2K parameters. For \ty{CIFAR-10}, we use the standard \ty{LENET} architecture which has a total of 0.54M parameters with three convolution layers and two fully connected layers. We use the same configuration for max-pooling/batch normalization layers as in the standard \ty{LENET} models.

We compare \ty{IC-GT} with two other strategies commonly employed to address noise in an inexact communication setting. The first strategy involves utilizing a decreasing noise variance policy, where the variance decreases as the number of communication rounds progresses. In this approach, we employ \ty{GT} with quantization and adjust the quantization levels to become finer as the rounds increase. Specifically, in the case of \eqref{randquant}, we uniformly increase the parameter $\Delta_p$ from $\Delta_p=1$ to $\Delta_p=5 \times 10^3$ as the rounds progress. This results in higher levels of noise variance in the initial rounds and lower levels in the final rounds. The second strategy maintains a uniform quantization level of $\Delta_p=10^2$ throughout all communication rounds, leading to a fixed noise variance. We employ the same quantization level of $\Delta_p=10^2$ for \ty{IC-GT}.



%We compare \ty{IC-GT} to two other strategies used to address noise in an inexact communication setting. The first strategy involves employing a decreasing noise variance policy, where the variance decreases with the number of communication rounds. For this purpose, we employ \ty{GT} with quantization, adjusting the quantization levels to become finer as the number of rounds increases. In the case of \eqref{randquant}, the parameter $\Delta_p$ is increased from $\Delta_p=1$ to $\Delta_p=5 \times 10^3$ as the rounds progress. This leads to higher levels of noise variance in the initial rounds and lower levels in the final rounds. The second strategy maintains a uniform quantization level of $\Delta_p=10^2$ for all communication rounds, resulting in a fixed noise variance throughout. We employ the same $\Delta_p$ value for \ty{IC-GT}.


The results of the comparison have been plotted in Figure~\ref{fg2}(a)-(b). In both plots, the baseline represents the highest achievable accuracy that can be obtained in a centralized setting using the models employed. From the plots, we observe that for both the \ty{CIFAR-10} and \ty{MNIST} datasets, the performance of \ty{IC-GT} is the closest to the baseline. The performance difference between \ty{IC-GT} and the baseline appears to be more pronounced in the case of \ty{CIFAR-10} compared to \ty{MNIST}.

% Figure environment removed

%The results have been plotted in Figure \ref{fg1}(e)-(f). The baseline in both the plots represent the best possible accuracy that can be achieved in a centralized setting with the models used.  We note that for both \ty{CIFAR-10} and \ty{MNIST} datasets, \ty{IC-GT}'s performance is closest to the baseline. The performance difference for \ty{CIFAR-10}  seem to be more pronounced than \ty{MNIST}. 




%\section{Conclusions}% and }Future Work}
\section{Final Remarks}
In this paper, we proposed a gradient tracking based algorithm for decentralized optimization in an inexact communication scenario. We established theoretical convergence guarantees and analyzed the impact of communication and gradient noise on performance. Our algorithm effectively mitigates the impact of communication noise and data heterogeneity,  and achieves optimal iteration complexity for strongly convex, stochastic smooth functions. Experimental results on logistic regression and neural networks demonstrated the superiority of the proposed algorithm over existing methods. As future work, the algorithm can be extended to other settings, such as directed graphs and asynchronous updates, and incorporate variance reduction techniques to enhance convergence rate.
%We established theoretical convergence guarantees for the proposed method and quantified the effect of communication noise and gradient noise on the final convergence performance. Moreover, we showed that the proposed algorithm can effectively mitigate the impact of communication noise and data heterogeneity on the convergence performance, and achieves optimal iteration complexity for strongly convex stochastic smooth functions with inexact communications. Our experiments on logistic regression and neural networks illustrate the benefits of the proposed algorithm compared to other existing approaches. Finally, these algorithmic approaches can be extended to other interesting settings such as network setting with directed graphs and asynchronous updates, as well as incorporating variance reduction techniques to enhance the convergence rate of the algorithm.} 
%Future research directions include exploring a more general network setting with directed graphs and asynchronous updates, as well as incorporating variance reduction techniques to enhance the convergence rate of the algorithm.} 
%A detailed theoretical and empirical analysis of the proposed algorithm was provided to show that it can effectively mitigate the impact of communication noise and data heterogeneity on the convergence performance. 
%To conclude, we proposed a gradient tracking based algorithm for decentralized optimization in an inexact communication scenario. A detailed theoretical and empirical analysis of the proposed algorithm was provided to show that it can effectively mitigate the impact of communication noise and data heterogeneity on the convergence performance. Future research directions include exploring a more general network setting with directed graphs and asynchronous updates, as well as incorporating variance reduction techniques to enhance the convergence rate of the algorithm. 

% \section*{Appendix I}
% \noindent \textbf{Proof of (\ref{main2}):}
%  The recursion for $\bar{x}_{k+1}$ can be seen to be equal to:
% $$
% \x_{k+1} = \x_{k} -\frac{\alpha}{n} \sum_{j=1}^n \n F_j(x^j_{k},\xi^j_k) + \frac{1}{n} \sum_{j=1}^n \bar{\epsilon}^j_{k,c} 
% $$
% where $\bar{\epsilon}^j_{k,c}  := \sum_{p=1}^n \epsilon^{p}_{k,c}$. Note that the consensus error for the fully connected case simplifies to:
% $$
% \|\Delta x_{k} \|^2:=  \sum_{j=1}^n \|x_{i,j}-\x_k\|^2= \sum_{j=1}^n \Big\| \frac{1}{n} \sum_{p=1}^n \epsilon_{k,c}^{p}-  \frac{1}{n}\sum_{j=0}^n \bar{\epsilon}^j_{k,c}\Big\|^2
% $$
% so that 
% \begin{equation}\label{consensbd}
%     \EX \|\Delta x_{k} \|^2 \leq  2\sigma^2_c
% \end{equation}
% Let $\epsilon_{k,g}^{j} : = -\n F_j(x^j_k,,\xi^j_k) +\n f_j(x^j_k) $. Then, we have:
% \begin{align*}
%     \|\x_{k+1}  - x^*\|^2 &=  \Big\|\x_{k} - \frac{\alpha}{n}\sum_{j=1}^n \n F_j(x^j_k,\xi^j_k) + \frac{1}{n}\sum_{j=1}^n\bar{\epsilon}_{k,c}^{j} - x^*\Big\|^2\\
%     &=  \Big\|\x_{k} - \frac{\alpha}{n}\sum_{j=1}^n \n f_j(x^j_k) + \frac{1}{n}\sum_{j=1}^n\bar{\epsilon}_{k,c}^{j} + \frac{\alpha}{n}\sum_{j=1}^n \epsilon_{k,g}^{j} - x^*\Big\|^2\\
%     &=  \Big\|\x_{k} - \frac{\alpha}{n}\sum_{j=1}^n \n f_j(x^j_{k})- x^*\Big\|^2 +\langle  \x_{k} - \frac{\alpha}{n}\sum_{j=1}^n \n f_j(x^j_{k})- x^*, \frac{1}{n}\sum_{j=1}^n(\bar{\epsilon}_{k,c}^{j}+\alpha\epsilon_{k,g}^{j}) \rangle\\
%     &\qquad\qquad \qquad  \qquad\qquad \qquad   \qquad\qquad \qquad + \Big\|\frac{1}{n}\sum_{j=1}^n(\bar{\epsilon}_{k,c}^{j} +\alpha\epsilon_{k,g}^{j})\Big\|^2
% \end{align*}
% Taking conditional expectation and using Assumption (iii) we have:
% \begin{align}\label{errorbd1}
%   \EX  [ \|\x_{k+1}  - x^*\|^2|\mathcal{F}_k] &\leq  \Big\|\x_{k} - \frac{\alpha}{n}\sum_{j=1}^n \n f_j(x^j_{k})- x^*\Big\|^2  + 2\sigma^2_{c}+ 2\alpha^2\sigma_g^2\nonumber \\ 
%   \implies \EX  [ \|\x_{k+1}  - x^*\|^2] &\leq  \EX \Big\|\x_{k} - \frac{\alpha}{n}\sum_{j=1}^n \n f_j(x^j_{k})- x^*\Big\|^2  +2 \sigma^2_{c} + 2\alpha^2\sigma_g^2
% \end{align}
% Let us consider the first term in the above inequality. We have for any $\beta>0$:
% \begin{align*}
%  \Big\|\x_{k} - \frac{\alpha}{n}\sum_{j=1}^n \n f_j(x^j_k)- x^*\Big\|^2 &=   \Big\|\x_{k} -  \frac{\alpha}{n}\sum_{j=1}^n \n f_j(\x_k)-x^* -\frac{\alpha}{n}\sum_{j=1}^n \n f_j(x^j_k)+\frac{\alpha}{n}\sum_{j=1}^n \n f_j(\x_k)\Big\|^2  \\
%  &\leq \big( 1 + \beta\big)\Big\|\bar{x}_{k}-x^*- \frac{\alpha }{n} \sum_{i=1}^n \n f_i( \bar{x}_{k})  \Big\|^2 +  \Big( 1 + \frac{1}{\beta}\Big)\frac{\alpha^2}{n^2} \Big\|\sum_{j=1}^n \n f_j(\bar{x}^j_k)-\sum_{j=1}^n \n f_j(x^j_k)\Big\|^2 \nonumber \\
%   &\leq \big( 1 + \beta\big)\Big\|\bar{x}_{k}-x^*- \frac{\alpha }{n} \sum_{i=1}^n \n f_i( \bar{x}_{k})  \Big\|^2 +  \Big( 1 + \frac{1}{\beta}\Big)\frac{2\alpha^2L^2}{n}\|\Delta x_{k} \|^2 , \nonumber 
% \end{align*}
% With $\eta := \frac{\mu L}{\mu+L}$, for $\beta = \frac{\alpha \eta}{1- 2\alpha \eta}$, we have 
% $$
% \big(1+\beta\big)\Big( 1-\frac{2\alpha\mu L}{\mu +L }\Big) \leq 1-\alpha \eta
% $$
% so that the first term in the previous inequality can be bounded as (using (\ref{consensbd}) for the second term in the right hand side):
% \begin{align}\label{alt2}
%  \EX \Big\|\x_{k} - \frac{\alpha}{n}\sum_{j=1}^n \n f_j(x^j_k)- x^*\Big\|^2 
%   &\leq \big( 1 -\alpha\eta \big)\EX \Big\|\bar{x}_{k}-x^*- \frac{\alpha }{n} \sum_{i=1}^n \n f_i( \bar{x}_{k})  \Big\|^2 +  \frac{\alpha L^2(1- \alpha \eta) \sigma_c^2}{n\eta} , 
% \end{align}
% Let $\bar{f}(\bar{x}) =n^{-1} \sum_{i=1}^n f_i(\bar{x})$. The first term in the above can be bounded as:
% \begin{align}\label{alt1}
%      \| \x_{k} - \alpha \n \bar{f}(\x_k)-x^*\|^2 &=  \| \x_k -x^* \|^2 - \alpha \langle  \n \bar{f}(\x_k), \x_{k}-x^* \rangle +\alpha^2 \|\n \bar{f}(\x_k) \|^2 \nonumber \\
%     & \leq \Big( 1-\frac{2\alpha\mu L}{\mu +L }\Big) \| \x_k -x^* \|^2 + \alpha\Big(\alpha -\frac{2}{\mu+L}\Big) \| \n  \bar{f}(\x_k) \|^2 \nonumber \\
%     & \leq \Big( 1- 2\alpha \eta\Big) \| \x_k -x^* \|^2 
% \end{align}
% where we have used the fact that $\alpha< \frac{2}{\mu+L}$ to get the last inequality and the relation (for s.c. smooth functions) to get the second inequality:
% $$
% \langle \n \bar{f}(x) - \n \bar{f}(y) , x-y\rangle \geq \frac{\mu L}{\mu +L }\|x-y\|^2 +\frac{1}{\mu+L} \| \n \bar{f}(x) - \n  \bar{f}(y)\|^2
% $$
% Plugging (\ref{alt2}) in (\ref{alt1}), we have
% $$
% \EX \Big\|\x_{k} - \frac{\alpha}{n}\sum_{j=1}^n \n F_j(x^j_k,\xi_{j,k})- x^*\Big\|^2 
%   \leq \big( 1 -2\alpha\eta \big)\| \x_k -x^* \|^2 +  \frac{\alpha L^2(1- \alpha \eta) \sigma_c^2}{n\eta} , 
% $$
% Using the above in (\ref{errorbd1}), we get the main descent relation:
% $$
%  \EX  [ \|\x_{k+1}  - x^*\|^2] \leq \big( 1 -2\alpha\eta \big) \EX \| \x_k -x^* \|^2 +  \frac{\alpha L^2(1- \alpha \eta) \sigma_c^2}{n\eta}  + 2 \sigma^2_{c} + 2\alpha^2\sigma_g^2
% $$
% Iterating the above proves (\ref{main2}).
\section*{Appendix I: Proof of Lemma \ref{lem0}}\label{sec:apndxI}
\begin{proof}
	From (\ref{mat0}), we have,
	\begin{align}\label{mat}
		\vb_{k}&= (\II-\gamma \Q)\xb_{k} + \gamma \hat{\textbf{Q}}\epsilon_{k,c}.
	\end{align}
	Multiplying both sides of (\ref{mat}) by $\frac{1}{n}1_n1_n^T\otimes I_d$, we get,
	\begin{align}\label{mat1}
		\Bar{\textbf{v}}_k &= \Bar{\textbf{x}}_{k} +  \frac{\gamma}{n} \big(1_n1_n^T\otimes I_d \big)\hat{\textbf{Q}} \epsilon_{k,c} \nonumber\\
		& = (\II-\gamma \Q) \bar{\textbf{x}}_k  + \frac{\gamma}{n} \big(1_n 1_n^T\otimes I_d \big)\hat{\textbf{Q}} \epsilon_{k,c}.
	\end{align}
	where we used $\frac{1}{n}1_n^T\Q =0$ to get the first inequality and $(\II-\gamma \Q)\bar{\textbf{x}}_k  = \bar{\textbf{x}}_k $ to get the last inequality. Subtracting (\ref{mat1}) from  (\ref{mat}) and adding $- \frac{1}{n}1_n1_n^T \Delta \xb_{k}$, we get,
	\begin{equation}\label{mat3}
		\Delta \vb_k = (\III-\gamma \Q ) \Delta \xb_{k} +  \gamma \tilde{\textbf{Q}}\epsilon_{k,c}. 
	\end{equation}
	From (\ref{matI}), the expression for $\Delta \xb_k$ can be written as,
	\begin{equation}\label{mat4}
		\Delta \xb_{k} = \Delta \vb_{k-1}  - \alpha \Delta \yb_{k-1}.
	\end{equation}
	Substituting for $\Delta \xb_k$ in (\ref{mat3}) using (\ref{mat4}) yields the following
	%, we get the 
	recursive relation for $\Delta \vb_k$ in terms of $\Delta \vb_{k-1}$ and $\Delta \yb_{k-1}$:
	\begin{align*}
		\Delta \vb_k =(\III-\gamma \Q ) \Delta \vb_{k-1}  - \alpha (\III-\gamma \Q ) \Delta \yb_{k-1} +  \gamma \tilde{\mathbf{Q}} \epsilon_{k,c}
	\end{align*}
	Next, the recursive relation for $\Delta \xb_k$ in terms of $\Delta \xb_{k-1}$ and $\Delta \yb_{k-1}$ is obtained by substituting  for $\Delta \vb_{k-1}$ in (\ref{mat4}) using \eqref{mat3}. That is,
	%Using (\ref{mat3}) to substitute for $\Delta \vb_{k-1}$ in (\ref{mat4}), we get the \rb{recursive relation }%equation 
	%for $\Delta \xb_k$ as, % follows:
	\begin{align*}
		\Delta \xb_k =(\III-\gamma \Q ) \Delta \xb_{k-1}  +  \gamma \tilde{\textbf{Q}} \epsilon_{k-1,c} - \alpha \Delta \yb_{k-1}. 
	\end{align*}
The recursive expression for $\Delta \yb_k$ can be obtained similarly using the expression for $\bar{\yb}_k$ and subtracting it from (\ref{matII}), concluding the proof. 
	%The expression for $\Delta \yb_k$ is straightforward from (\ref{matII}) and concluding the proof. \rbnote{(write this sentence in a different form)}
\end{proof}


% \subsection*{Appendix II}\label{sec:apndxII}
% \begin{proof}
% \rb{(previous proof...Wrote an alternate version...)Using mathematical induction, we can show that $\textbf{J}^\tau_\gamma$ for any $\ka\in \mathbb{N}$ is given as,} 
% %The expression for $\textbf{J}^\tau_\gamma$ for a given $\ka\in \mathbb{N}$ can be shown by an induction argument to be equal to:
% \begin{equation}\label{matproddef}
% \textbf{J}^\ka_\gamma =
% \begin{bmatrix}
% (\II-\gamma \Q)^\ka &0 &-\ka (\II-\gamma \Q)^\ka \\
% 0 & (\II-\gamma \Q)^{\ka} &-\ka (\II-\gamma \Q)^{\left(\ka-1\right)} \\
% 0 & 0&(\II-\gamma \Q)^{\ka} \
% \end{bmatrix}.
% \end{equation}
% \rb{Clearly, the statement holds for $\tau = 1$ due to \eqref{not1}. Let us suppose that it holds for some $\tau \in \N$.  Consider, 
% \begin{align*}
% 	\textbf{J}^{\ka+1}_\gamma &=
% 	\begin{bmatrix}
% 		(\II-\gamma \Q)^\ka &0 &-\ka (\II-\gamma \Q)^\ka \\
% 		0 & (\II-\gamma \Q)^{\ka} &-\ka (\II-\gamma \Q)^{\left(\ka-1\right)} \\
% 		0 & 0&(\II-\gamma \Q)^{\ka} \
% 	\end{bmatrix}\begin{bmatrix}
% 	\II-\gamma \Q &0 &-(\II-\gamma \Q)\\
% 	0 & \II-\gamma \Q &-\II\\
% 	0 & 0&  \II-\gamma \Q
% \end{bmatrix} \\
% &= \begin{bmatrix}
% 	(\II-\gamma \Q)^{(\ka+1)} &0 &-(\ka+1) (\II-\gamma \Q)^{(\ka+1)} \\
% 	0 & (\II-\gamma \Q)^{(\ka+1)} &-(\ka+1) (\II-\gamma \Q)^{\ka} \\
% 	0 & 0&(\II-\gamma \Q)^{(\ka+1)} \
% \end{bmatrix}
% \end{align*}
% Therefore, by mathematical induction, \eqref{matproddef} holds for any $\ka \in \N$.
% }
% The above expression can be used to \rb{upper }bound $\|\textbf{J}_\gamma\|$. % required norm. 
% We note that
% \begin{equation}\label{basic}
% \|\textbf{J}_\gamma^\ka\|^2 \leq \|\II-\gamma \Q\|^{2\ta} +\ta^2\|\II-\gamma \Q\|^{2(\ta-1)} +\ta^2\|\II-\gamma \Q\|^{2\ta}     
% \end{equation}
% \rbnote{(this statement needs an explanation. It is not immediately clear to me. If there is no square then it is fine if $\ka$ is inside the norm. Why do we need to take squares here? Can't we just derive everything with respect to the norm and just square them in the end.)}
% We will next bound the terms on the right hand side of \eqref{basic}. %the above inequality. 
% Note that the eigenvalues of the matrix $(\II-\gamma \Q)^\ka$ are of the form $(1-\gamma (1-\lambda_i))^\ka$, where $\lambda_i$ are the eigenvalues of $Q$, so that
% \begin{align}\label{2normbd}
% \|(\II-\gamma \Q)^\ka\|^2 &\leq \|\II-\gamma \Q\|^{2\ka}\nonumber\\
% &= \max_{i} (1-\gamma(1-\lambda_i))^{2\ka}\nonumber \\
% &= (1-\gamma(1-\lambda_2))^{2\ka}
% \end{align}
% \rbnote{(I believe there is a mistake in the above statement. $\max_{i} (1-\gamma(1-\lambda_i))^{2\ka} = 1$ with $i=1$. That is the maximum occurs at the first eigen value which is $1$. We might have an issue here.  Consider an example where $n=3, d=1, \gamma = 1/3$ line graph. That is,  
% \begin{align*}	
% 	Q &= \begin{bmatrix}
% 			2/3 &1/3 &0 \\
% 			1/3 & 1/3 & 1/3 \\
% 			0 & 1/3 & 2/3
% 		\end{bmatrix}
% 	\quad Q' = I - Q =  \begin{bmatrix}
% 		1/3 &2/3 &1 \\
% 		2/3 & 2/3 & 2/3 \\
% 		1 & 2/3 & 1/3
% 	\end{bmatrix}
% \quad (I - \gamma Q') =  \begin{bmatrix}
% 	8/9 &2/3 &1 \\
% 	2/3 & 2/3 & 2/3 \\
% 	1 & 2/3 & 1/3
% \end{bmatrix}
% \end{align*}	
% for which $\|(\II-\gamma \Q)\| = 1$.  Wrote an alternate version of this proof.. check that.	
% 	)}
% We select $\ka_1$ such that $\|(\II-\gamma \Q)^{\ka_1} \|^2 \leq \delta/2$. This can be done by adjusting $\tau_1$ as follows:
% \begin{align*}
%     \big(1-\gamma (1-\lambda_2)\big)^{2\ka_1} \leq \exp\big( -2\ka_1\gamma & (1-\lambda_2) \big)\leq \frac{\delta}{2}\\
%     \text{if } \ka_1 \geq -\frac{\ln \delta/2}{ 2\gamma (1-\lambda_2) }
% \end{align*}
% We next bound the second term in (\ref{basic}). For convenience, we define $\textbf{Q}_1\defeq\ka (I-\gamma \Q)^{\ka-1} $. We note that the eigenvalues of $\textbf{Q}_1$ are of the form $\ka (1-\gamma(1-\lambda_i) )^{\ka-1},\,i\in [n]$. Then, we have for $\ka_2\in \mathbb{N}$:
% \begin{align}\label{Amatrix}
% \|\textbf{Q}_1\|^2 &\leq \ka_2^{2} (1-\gamma(1-\lambda_2) )^{2\ka_2-2} \nonumber\\
% &\leq \ka_2^{2} \exp (- (2\ka_2-2)\gamma(1-\lambda_2) ) 
% \end{align}
% so that
% \begin{align}\label{firsteq}
% \ka_2^{2} \exp (- (2\ka_2-2)\gamma(1-\lambda_2) ) &\leq \frac{\delta}{4}\\
%   2 \ln \ka_2 - 2(\ka_2 -1) \gamma (1-\lambda_2)  &\leq \ln \delta/4\nonumber\\
% \Big( 1 -\frac{\ln \ka_2 }{ \gamma (1-\lambda_2)\ka_2}\Big)\ka_2 &\geq  1  - \frac{ \ln \delta/4}{2\gamma (1-\lambda_2)} \nonumber
% \end{align}
% Let $\tilde{\ka}_2 \in \mathbb{N}$ be such that $\ln \tilde{\ka}_2/\gamma (1-\lambda_2)\tilde{\ka}_2 \leq 1/2$. Then, assuming that $\ka'_2 \geq \tilde{\ka}_2$, we have using the above inequality:
% \begin{equation}\label{kadash}
% \ka'_2 \geq 2\Big\{ 1 -\frac{\ln \delta/4}{2\gamma (1-\lambda_2)}  \Big\}
% \end{equation}
% will make sure (\ref{firsteq}) is satisfied. It remains to show that there exists a $\tilde{\ka}_2$ satisfying $\ln \tilde{\ka}_2/\gamma (1-\lambda_2)\tilde{\ka}_2 \leq 1/2$ assumed in proving (\ref{kadash}). For this condition to hold, we can equivalently show that a $\tilde{\ka}_2$ exists such that
% \begin{align*}
%      \ln \tilde{\ka}_2/\tilde{\ka}_2 < \gamma (1-\lambda_2)/2\defeq \ep
% \end{align*}
% For $y = \frac{4\ln 1/\ep}{\ep}$, we have
% \begin{align*}
%      \ln y/y\leq \epsilon \frac{\ln \frac{4}{\epsilon}+ \ln \ln \frac{1}{\ep}}{4\ln 1/\epsilon}\leq \epsilon
% \end{align*}
% Then, $\tilde{\ka}_2 \geq y=8\ln  \big(2/\gamma (1-\lambda_2)\big)/ \gamma (1-\lambda_2)$ will satisfy the required bound. Thus, $\|\textbf{Q}_1\|^2 \leq \delta/4$ if we select $\ka_2 \geq \max\{\tilde{\ka}_2,\ka'_2\}$, where the bound on $\ka'_2$ is given in (\ref{kadash}).
% Note that the same $\ka_2$ will also work for bounding the last term in (\ref{basic}). 
% \noindent Finally, we note from (\ref{basic}) that selecting $\ka = \max\{\ka_1,\ka_2\} $ will ensure that 
% $$\|\textbf{J}^\ka_\gamma\|^2 \leq  \frac{\delta}{2} + \frac{\delta}{4} + \frac{\delta}{4} = \delta
% $$

% \end{proof}

\subsection*{Appendix II: Proof of Lemma \ref{lem3}}\label{sec:apndxII}
\begin{proof}
	Using mathematical induction, we can show that $\textbf{J}^\tau_\gamma$ for any $\ka\in \mathbb{N}$ is given as, 
	%The expression for $\textbf{J}^\tau_\gamma$ for a given $\ka\in \mathbb{N}$ can be shown by an induction argument to be equal to:
	\begin{equation}\label{matproddef}
		\textbf{J}^\ka_\gamma =
		\begin{bmatrix}
			(\III-\gamma \Q )^\ka &0 &-\ka (\III-\gamma \Q )^\ka \\
			0 & (\III-\gamma \Q)^{\ka} &-\ka (\III-\gamma \Q )^{\left(\ka-1\right)} \\
			0 & 0&(\III-\gamma \Q)^{\ka} \
		\end{bmatrix}.
	\end{equation}
	Taking norms in \eqref{matproddef} and using triangular inequality, we get,
	%The above expression can be used to \rb{upper }bound $\|\textbf{J}_\gamma\|$. % required norm. 
	%We note that
	\begin{equation}\label{basic}
		\|\textbf{J}_\gamma^\ka\| \leq \|(\III-\gamma \Q )^{\ta}\| +\ta\|(\III-\gamma \Q)^{(\ta-1)}\| +\ta\|(\III-\gamma \Q)^{\ta}\|.    
	\end{equation}
	%\rbnote{(this statement needs an explanation. It is not immediately clear to me. If there is no square then it is fine if $\ka$ is inside the norm. Why do we need to take squares here? Can't we just derive everything with respect to the norm and just square them in the end.)}
	We will next bound the terms on the right hand side of \eqref{basic}. %the above inequality. 
Note that the smallest eigenvalue of the matrix $(\III-\gamma \Q)^\ka$ is zero and the remaining eigenvalues 
	%Note that the eigenvalues of the matrix $(\II-\gamma \Q\rbfix{- \textbf{1}_n})^\ka$ is either zero or 
	are of the form $(1-\gamma (1-\lambda_i))^\ka$ for $i =2,\dots,n$, where $\lambda_i$ are the eigenvalues of $Q$ defined in Assumption~\ref{asmp1}. Therefore, 
	\begin{align}\label{2normbd}
		\|(\III-\gamma \Q )^\ka\| &= \max_{i=2,\dots,n} (1-\gamma(1-\lambda_i))^{\ka}\nonumber \\
		&= (1-\gamma(1-\lambda_2))^{\ka}.
	\end{align}
 From \eqref{kappaprop}, it follows that $	\ka \geq 2\left(1-\frac{\ln \sqrt{\delta}/4}{ \gamma (1-\lambda_2)}\right) > -\frac{\ln \sqrt{\delta}/2}{ \gamma (1-\lambda_2)}$. Substituting this inequality in \eqref{2normbd}, we get, 
	\begin{align}\label{5-2}
			\big(1-\gamma (1-\lambda_2)\big)^{\ka} &\leq \exp\big( -\ka\gamma  (1-\lambda_2) \big)\leq \frac{\sqrt{\delta}}{2}.
	\end{align}	
We next bound the second term in \eqref{basic}. For convenience, we define $\textbf{Q}_1\defeq\ka (\III-\gamma \Q )^{\ka-1} $. The smallest eigenvalue of $\textbf{Q}_1$ is zero and the remaining eigenvalues are of the form $\ka (1-\gamma(1-\lambda_i) )^{\ka-1},$ for $i=2,\dots,n$. Therefore,  %Then, we have for $\ka_2\in \mathbb{N}$:
	\begin{align}\label{Amatrix_1}
		\|\textbf{Q}_1\| &\leq \ka (1-\gamma(1-\lambda_2) )^{\ka-1} %\nonumber\\
		\leq \ka \exp (- (\ka-1)\gamma(1-\lambda_2) ). 
	\end{align}
	 Taking logarithm on both sides of \eqref{Amatrix_1} yields,
	 \begin{align}\label{Amatrix}
	 			\ln \|\textbf{Q}_1\| &\leq  \ln \ka - (\ka-1)\gamma(1-\lambda_2).
	 \end{align}	
	Now, consider $\frac{\ln \ka}{\ka}$  as a function of $\ka$ and observe that it is monotonically decreasing for any $\ka > \exp(1)$ since its first derivative $\frac{1 - \ln \ka}{\ka^2} < 0$. From \eqref{kappaprop}, we have $\tau\geq 16 \ln 4 > \exp(1)$ since $\gamma < 1/4$ and $\lambda_2 \in (-1,1)$.  For convenience, we define $\epsilon_{\gamma,\lambda_2} = \frac{\gamma (1 - \lambda_2)}{2} \in [0,1/4)$.  Therefore, from\eqref{kappaprop}, it follows that,  
	\begin{align}\label{tmp9}
		\frac{\ln \tau}{\tau} \leq \epsilon_{\gamma,\lambda_2} \frac{\ln \frac{4}{\epsilon_{\gamma,\lambda_2}}+ \ln \ln \frac{1}{\epsilon_{\gamma,\lambda_2}}}{4\ln 1/\epsilon_{\gamma,\lambda_2}}\leq \epsilon_{\gamma,\lambda_2} =  \frac{\gamma (1 - \lambda_2)}{2}.%\frac{\gamma (1 - \lambda_2)}{2}		
	\end{align}	
%Now, taking logarithm on both sides of \eqref{Amatrix}, and 
Using (\ref{tmp9}) and $	\ka \geq 2\left(1-\frac{\ln \sqrt{\delta}/4}{ \gamma (1-\lambda_2)} \right)$ in  \eqref{Amatrix}, we get, 
\begin{align}
	\ln \|\textbf{Q}_1\| &\leq  \ln \ka - (\ka-1)\gamma(1-\lambda_2) \nonumber \\%= \ka\gamma(1-\lambda_2) \left(\frac{\ln \ka}{\ka\gamma(1-\lambda_2)} - 1 + \frac{1}{\tau}\right)
	&\leq \frac{\ka\gamma(1 - \lambda_2)}{2} -  (\ka-1)\gamma(1-\lambda_2) \nonumber \\
	&=   \gamma(1 - \lambda_2)\left(1-\frac{\ka}{2}\right) \nonumber \\ &\leq \ln \sqrt{\delta}/4. \nonumber 
\end{align}
Therefore, 
\begin{align}\label{Bmatrix}
	 \|\textbf{Q}_1\| \leq \sqrt{\delta}/4.
\end{align}	
Finally, we bound the third term in \eqref{basic} as, 
\begin{align}\label{Cmatrix}
	\ta\|(\III-\gamma \Q)^{\ta}\| \leq \|\textbf{Q}_1\| \leq \sqrt{\delta}/4.
\end{align}	
Combining, \eqref{basic}, \eqref{5-2},\eqref{Bmatrix} and \eqref{Cmatrix}, we get, 
	$$\|\textbf{J}^\ka_\gamma\|^2 \leq  \left(\frac{\sqrt{\delta}}{2} + \frac{\sqrt{\delta}}{4} + \frac{\sqrt{\delta}}{4}\right)^2 = \delta.
$$.
 
	%\begin{align}\label{Amatrix}
	%	\|\textbf{Q}_1\|^2 &\leq \ka_2^{2} (1-\gamma(1-\lambda_2) )^{2\ka_2-2} \nonumber\\
	%	&\leq \ka_2^{2} \exp (- (2\ka_2-2)\gamma(1-\lambda_2) ) 
	%\end{align}
	%so that
	%\begin{align}\label{firsteq}
	%	\ka_2^{2} \exp (- (2\ka_2-2)\gamma(1-\lambda_2) ) &\leq \frac{\delta}{4}\\
	%	2 \ln \ka_2 - 2(\ka_2 -1) \gamma (1-\lambda_2)  &\leq \ln \delta/4\nonumber\\
	%	\Big( 1 -\frac{\ln \ka_2 }{ \gamma (1-\lambda_2)\ka_2}\Big)\ka_2 &\geq  1  - \frac{ \ln \delta/4}{2\gamma (1-\lambda_2)} \nonumber
	%\end{align}
	\begin{comment}
	Let $\tilde{\ka}_2 \in \mathbb{N}$ be such that $\ln \tilde{\ka}_2/\gamma (1-\lambda_2)\tilde{\ka}_2 \leq 1/2$. Then, assuming that $\ka'_2 \geq \tilde{\ka}_2$, we have using the above inequality:
	\begin{equation}\label{kadash}
		\ka'_2 \geq 2\Big\{ 1 -\frac{\ln \delta/4}{2\gamma (1-\lambda_2)}  \Big\}
	\end{equation}
	will make sure (\ref{firsteq}) is satisfied. It remains to show that there exists a $\tilde{\ka}_2$ satisfying $\ln \tilde{\ka}_2/\gamma (1-\lambda_2)\tilde{\ka}_2 \leq 1/2$ assumed in proving (\ref{kadash}). For this condition to hold, we can equivalently show that a $\tilde{\ka}_2$ exists such that
	\begin{align*}
		\ln \tilde{\ka}_2/\tilde{\ka}_2 < \gamma (1-\lambda_2)/2\defeq \ep
	\end{align*}
	For $y = \frac{4\ln 1/\ep}{\ep}$, we have
	\begin{align*}
		\ln y/y\leq \epsilon \frac{\ln \frac{4}{\epsilon}+ \ln \ln \frac{1}{\ep}}{4\ln 1/\epsilon}\leq \epsilon
	\end{align*}
	Then, $\tilde{\ka}_2 \geq y=8\ln  \big(2/\gamma (1-\lambda_2)\big)/ \gamma (1-\lambda_2)$ will satisfy the required bound. Thus, $\|\textbf{Q}_1\|^2 \leq \delta/4$ if we select $\ka_2 \geq \max\{\tilde{\ka}_2,\ka'_2\}$, where the bound on $\ka'_2$ is given in (\ref{kadash}).
	Note that the same $\ka_2$ will also work for bounding the last term in (\ref{basic}). 
	\noindent Finally, we note from (\ref{basic}) that selecting $\ka = \max\{\ka_1,\ka_2\} $ will ensure that 
	$$\|\textbf{J}^\ka_\gamma\|^2 \leq  \left(\frac{\sqrt{\delta}}{2} + \frac{\sqrt{\delta}}{4} + \frac{\sqrt{\delta}}{4}\right)^2 = \delta.
	$$
	\end{comment}
\end{proof}

\section*{Appendix III: Proof of Lemma \ref{r2}}\label{sec:apndxIII}
\begin{proof}
We begin by iterating the relation \eqref{mm1} with $k=t+\tau$:
\begin{align}
\Psi_{t+\ka} &=  \textbf{J}_\gamma \Psi_{t+\ka-1} + \alpha \textbf{E}_{t+\ka-1} \nonumber\\
&= \textbf{J}^\ka_\gamma \Psi_t + \alpha \sum_{i=0}^{\ka-1} \textbf{J}^{\ka-i-1}_\gamma  \textbf{E}_{t+i}\label{e0}
\end{align}
We next consider $\textbf{E}_k$ whose definition is recalled here:
$$
\textbf{E}_{k-1} =  \underbrace{\frac{ \gamma}{\alpha}
\begin{bmatrix}
\tilde{\textbf{Q}}\bm{ \epsilon}_{k,c} \\
\tilde{\textbf{Q}} \bm{\epsilon_{k-1,c}} \\
\alpha \tilde{\textbf{Q}} \bm{\epsilon_{k-1,c} }
\end{bmatrix} }_{\textbf{E}^c_{k-1}}
 +
\underbrace{ 
\begin{bmatrix}
0\\
0\\
\III (\nabla \textbf{F}(\xb_{k},\bm{\xi}_{k} ) - \nabla \textbf{F}(\xb_{k-1},\bm{\xi}_{k-1} ))
\end{bmatrix}}_{\textbf{E}_{k-1}^g}, \quad \forall k \in \N.
$$
We note that 
\begin{equation}\label{decomp1}
\EX \left[ \left\| \sum_{i=0}^{\ka-1} \textbf{J}^{\ka-i-1}_\gamma \textbf{E}_{t+i} \right\|^2\right] \leq  2\EX \left[\left\|\sum_{i=0}^{\ka-1} \textbf{J}^{\ka-i-1}_\gamma  \textbf{E}^c_{t+i} \right\|^2\right] + 2\EX \left[\left\| \sum_{i=0}^{\ka-1} \textbf{J}^{\ka-i-1}_\gamma \textbf{E}^g_{t+i}\right\|^2\right]
\end{equation}
We first bound the first term on the right hand side of \eqref{decomp1}.
%consider bounding the first term in the above decomposition. 
%We note that 
Using the expression for the matrix product $\textbf{J}_\gamma^{\tau-i-1}$ for any $0\leq i\leq\ka-1$ (cf. (\ref{matproddef})), we have,
\begin{equation}\label{normmat}
\textbf{J}^{\ka-i-1}_\gamma \textbf{ E}^c_{t+i} = \frac{ \gamma}{\alpha}
\begin{bmatrix}
(\III-\gamma \Q)^{\ka-i-1} \tilde{\textbf{Q}} \bm{\epsilon}_{t+i+1,c} -  \alpha (\ka-i-1)(\III-\gamma \Q)^{\ka-i-1}\tilde{\textbf{Q}} \bm{\epsilon}_{t+i,c} \\
(\III-\gamma \Q)^{\ka-i-1} \tilde{\textbf{Q}}- \alpha (\ka-i-1) (\III-\gamma \Q)^{\ka-i-2} \tilde{\textbf{Q}}\big)\bm{\epsilon}_{t+i,c}  \\
 \alpha (\III-\gamma \Q)^{\ka-i-1} \tilde{\textbf{Q}}\bm{\epsilon}_{t+i,c}
\end{bmatrix}. 
\end{equation}
%Note that
Note that, using $\|Q\|,\,\|\text{diag}(Q)\|\leq 1$, we have,
\begin{equation}\label{bds1}
\|(\III-\gamma \Q)^{\ka-i-1}\|^2\leq 1, \qquad  \|\hat{\textbf{Q}}\| = \|(Q' - \text{diag}(Q'))\otimes I_d\| \leq  2, \quad \mbox{and}\quad \|\tilde{\textbf{Q}} \| \leq%=
\left\|\II - \frac{11^T}{n}\right\|\|\hat{\textbf{Q}}\|  \leq 2. 
\end{equation}
%since $\|Q\|,\,\|\text{diag}(Q)\|\leq 1$. 
Taking norms in (\ref{normmat}) and using the bounds (\ref{bds1}), we get,
\begin{align}\label{tmp111}
\EX\left[ \|\textbf{J}^{\ka-i-1}_\gamma  \textbf{E}^c_{t+i}\|^2\right]  &\leq \frac{4\gamma^2}{\alpha^2}( 2(1+ \alpha^2(\ka-i-1)^2) +\alpha^2) \max\{  \EX\left[ \|\bm{\epsilon}_{t+i+1,c} \|^2\right], \EX \left[\|\bm{\epsilon}_{t+i,c} \|^2 \right]\}\nonumber\\
&\leq \frac{4\gamma^2}{\alpha^2}( 2+ 2\alpha^2\ka^2 +\alpha^2) \max\{  \EX \left[\|\bm{\epsilon}_{t+i+1,c} \|^2\right], \EX\left[\|\bm{\epsilon}_{t+i,c} \|^2 \right]\}\nonumber\\
&\leq \frac{8\gamma^2}{\alpha^2}( 1+ \alpha^2\left( \ka^2 +1/2\right)n\sigma_{c}^2,
\end{align}
where the last inequality is due to Assumption \ref{asmp2}. Hence,  
%we used Assumption \ref{asmp2} in the last inequality. Using (\ref{tmp111}), we can bound the first term in (\ref{decomp1}):
\begin{align}\label{bd1}
   \EX \left[\left\|\sum_{i=0}^{\ka-1} \textbf{J}^{\ka-i-1}_\gamma \textbf{ E}^c_{t+i}\right\|^2\right] &= 
   \sum_{i=0}^{\ka-1}  \EX \left[\left\|\textbf{J}^{\ka-i-1}_\gamma  \textbf{E}^c_{t+i}\right\|^2\right]
   \leq \frac{8\gamma^2}{\alpha^2} \big(1 + \alpha^2 (\ka^2 +1/2)\big)n\sigma_{c}^2 \ka,
\end{align}
where the equality is due to Assumption \ref{asmp2} and the fact that the cross terms of the form $\langle \bm{\epsilon}_{i,c},\bm{\epsilon}_{j,c}\rangle $ are all zero. That is, 
%where we have neglected the cross terms of the form $\langle \bm{\epsilon}_{i,c},\bm{\epsilon}_{j,c}\rangle $ in obtaining the first equality using Assumption \ref{asmp2}: 
if we denote $\mathcal{F}_k\defeq 
\sigma(\xb_{0}, \bm{\xi}_{0}, \bm{\epsilon}_{0,c}, \cdots, \bm{\xi}_{k-1}, \bm{\epsilon}_{k-1,c})$ 
%\sigma(\xb_{0}, \bm{\epsilon}_{0,c}, \bm{\xi}_{0}, \cdots, \bm{\epsilon}_{k-1,c}, \bm{\xi}_{k-1})$} 
%$\mathcal{F}_k\defeq \sigma(v_{i,k'},x_{i,k'},y_{i,k'}, i\in[n], k'\leq k)$ 
to be the sigma algebra generated by the random variables up to iteration $k$, we have for any $i,j$ with $i<j$, $ \EX [\langle\bm{\epsilon}_{i,c}, \bm{\epsilon}_{j,c}\rangle] =  \EX [ \EX  \left[\langle\bm{\epsilon}_{i,c}, \bm{\epsilon}_{j,c}\rangle|\mathcal{F}_{j}\right]]=0$.

Next we consider $\textbf{E}^g_k$ to bound the second summation in (\ref{decomp1}). Let $\g_{k} \defeq  \n \textbf{F}(\xb_{k},\bm{\xi}_{k})-\n \textbf{f}(\xb_k)$ and $\ddd_k = \n \textbf{f}(\textbf{x}_k) - \n \textbf{f}(\textbf{x}^*)$. We note from Assumption \ref{asmp4}, $\g_k$ is a zero mean vector given $\xb_k$ with variance $n \sigma_g^2$. Using $\bar{\mathbf{Q}} %\defeq(\III-\gamma \Q) \III$ 
\defeq(\III-\gamma \Q)$ 
and the expression for the matrix product $\textbf{J}_\gamma^{\ka-i-1}$ (cf. \ref{matproddef}), we have,
\begin{align}\label{sixterms}
 \EX &\left[\left\|\sum_{i=0}^{\ka-1} \textbf{J}^{\ka-i-1}_\gamma  \textbf{E}^g_{t+i}\right\|^2\right] \nonumber \\&= \EX \left[\left\|\sum_{i=0}^{\ka-1} (\ka-i - 1) \bar{\mathbf{Q}} ^{\ka-i-1}\III( \n \textbf{F}(\xb_{t+i+1},\bm{\xi}_{t+i+1})-\n \textbf{F}(\xb_{t+i},\bm{\xi}_{t+i}) ) \right\|^2\right] \nonumber \\
 &~~~~~+\EX\left[ \left\|\sum_{i=0}^{\ka-1}  (\ka-i - 1) \bar{\mathbf{Q}}^{\ka-i-2}\III(\n \textbf{F}(\xb_{t+i+1},\bm{\xi}_{t+i+1})-\n \textbf{F}(\xb_{t+i},\bm{\xi}_{t+i})) \right\|^2 \right] \nonumber \\
 &~~~~~+ \EX \left[\left\|\sum_{i=0}^{\ka-1} \bar{\mathbf{Q}} ^{\ka-i-1} \III\left(\n \textbf{F}(\xb_{t+i+1},\bm{\xi}_{t+i+1})-\n \textbf{F}(\xb_{t+i},\bm{\xi}_{t+i})) \right)  \right\|^2\right]  \nonumber \\
&\leq 2\Bigg\{ \EX \left[\left\|\sum_{i=0}^{\ka-1} (\ka-i-1) \bar{\mathbf{Q}} ^{\ka-i-1}\III (\g_{t+i+1}-\g_{t+i}) \right\|^2 \right]+ \EX \left[\left\|\sum_{i=0}^{\ka-1} (\ka-i-1) \bar{\mathbf{Q}} ^{\ka-i-1}\III (\ddd_{t+i+1}-\ddd_{t+i}) \right\|^2\right] \nonumber \\
  &~~~~~+\EX\left[ \left\|\sum_{i=0}^{\ka-1}  (\ka-i-1) \bar{\mathbf{Q}} ^{\ka-i-2}\III(\g_{t+i+1}-\g_{t+i}) \right\|^2\right] + \EX \left[\left\|\sum_{i=0}^{\ka-1} (\ka-i-1) \bar{\mathbf{Q}} ^{\ka-i-2}\III (\ddd_{t+i+1}-\ddd_{t+i}) \right\|^2 \right] \nonumber \\ 
  &~~~~~+ \EX \left[\left\|\sum_{i=0}^{\ka-1} \bar{\mathbf{Q}} ^{\ka-i-1}\III (\g_{t+i+1}-\g_{t+i}) \right\|^2\right]+ \EX \left[\left\|\sum_{i=0}^{\ka-1}  \bar{\mathbf{Q}} ^{\ka-i-1}\III (\ddd_{t+i+1}-\ddd_{t+i}) \right\|^2 \right]\Bigg\}, 
\end{align}
\begin{comment}
\begin{multline}\label{sixterms}
 \EX \left[\left\|\sum_{i=0}^{\ka-1} \textbf{J}^{\ka-i-1}_\gamma  \textbf{E}^g_{t+i}\right\|^2\right] = \Bigg\{\EX \left[\left\|\sum_{i=0}^{\ka-1} (\ka-i) \bar{\mathbf{Q}} ^{\ka-i-1}( \n \textbf{F}(\xb_{t+i+1},\bm{\xi}_{t+i+1})-\n \textbf{F}(\xb_{t+i},\bm{\xi}_{t+i}) ) \right\|^2\right] + \\
 \EX\left[ \left\|\sum_{i=0}^{\ka-1}  (\ka-i) \bar{\mathbf{Q}}^{\ka-i-2}(\n \textbf{F}(\xb_{t+i+1},\bm{\xi}_{t+i+1})-\n \textbf{F}(\xb_{t+i},\bm{\xi}_{t+i})) \right\|^2 \right] \\
 + \EX \left[\left\|\sum_{i=0}^{\ka-1} \bar{\mathbf{Q}} ^{\ka-i} \left(\n \textbf{F}(\xb_{t+i+1},\bm{\xi}_{t+i+1})-\n \textbf{F}(\xb_{t+i},\bm{\xi}_{t+i})) \right)  \right\|^2\right] \Bigg\} \\
\leq 2\Bigg\{ \EX \left[\left\|\sum_{i=0}^{\ka-1} (\ka-i) \bar{\mathbf{Q}} ^{\ka-i-1} (\g_{t+i+1}-\g_{t+i}) \right\|^2 \right]+ \EX \left[\left\|\sum_{i=0}^{\ka-1} (\ka-i) \bar{\mathbf{Q}} ^{\ka-i-1} (\ddd_{t+i+1}-\ddd_{t+i}) \right\|^2\right] \\
  +\EX\left[ \left\|\sum_{i=0}^{\ka-1}  (\ka-i) \bar{\mathbf{Q}} ^{\ka-i-2}(\g_{t+i+1}-\g_{t+i}) \right\|^2\right] + \EX \left[\left\|\sum_{i=0}^{\ka-1} (\ka-i) \bar{\mathbf{Q}} ^{\ka-i-1} (\ddd_{t+i+1}-\ddd_{t+i}) \right\|^2 \right]\\ 
  + \EX \left[\left\|\sum_{i=0}^{\ka-1} \bar{\mathbf{Q}} ^{\ka-i} (\g_{t+i+1}-\g_{t+i}) \right\|^2\right]+ \EX \left[\left\|\sum_{i=0}^{\ka-1}  \bar{\mathbf{Q}} ^{\ka-i-1} (\ddd_{t+i+1}-\ddd_{t+i}) \right\|^2 \right]\Bigg\}, 
\end{multline}
\end{comment}
where the inequality is obtained by adding and subtracting
%to obtain the second inequality we have added and subtracted 
the terms $\n \textbf{f}(\xb_{t+i+ 1}),\,\n\textbf{f}(\xb_{t+i})$, and $\n \textbf{f}(\xb^*)$ in each of the three terms in the first equality. We bound the first term on the right hand side of \eqref{sixterms} and follow a similar approach to bound the rest of the terms. However, before proceeding, we state the following fact whose proof is provided at the end of this appendix:
%We consider bounding the first term in the above as the ones for the rest of the terms can be obtained along similar lines. Before doing so, we state the following fact whose proof is provided at end of this appendix: 
% Let the eigenvalues of $\tilde{Q}:= I-\gamma\Q$ be denoted by $\la_i$ with  $0\leq \la_i < 1,\,i\in [n]$. 
\begin{equation}\label{proof_app}
   \left\|(i+1)\bar{\textbf{Q}}^{i+1}\III -i \bar{\textbf{Q}}^{i}\III\right\|^2\leq 4, \quad \forall i \in \N.
\end{equation}
%We note the the first term of (\ref{sixterms}) can be written as:
The first term on the right hand side of \eqref{sixterms} is bounded as,
\begin{align}\label{st1}
   \EX  &\left[ \left\|\sum_{i=0}^{\ka-1} (\ka- i-1)\bar{\mathbf{Q}}^{\ka- i-1}\III(\g_{t+i+1}-\g_{t+i}) \right\|^2  \right]  \nonumber \\ 
   &~~~~= \EX \Bigg[\Bigg\|  \sum_{i=1}^{\ka-1} \big((\ka - i) \bar{\mathbf{Q}}^{\ka-i}\III - (\ka-i-1)\bar{\mathbf{Q}}^{\ka-i-1}\big) \g_{t+i} -(\ka-1)  \bar{\mathbf{Q}}^{\ka-1}\III \g_{t}\Bigg\|^2 \Bigg]  \nonumber \\
   &~~~~~\leq  %\nonumber \\
   %&\qquad 
   \sum_{i=1}^{\ka-1} \left\|(\ka - i)  \bar{\mathbf{Q}}^{\ka-1}\III - (\ka-i-1)\bar{\mathbf{Q}}^{\ka-i-1}\III\right\|^2 \EX [\|\g_{t+i}\|^2] 
   + \left\| (\ka-1)\bar{\mathbf{Q}}^{\ka-1}\III \right\|^2 \EX  [\|\g_{t}\|^2]\nonumber \\
    &~~~~~\leq  4\sum_{i=1}^{\ka-1} \EX [\| \g_{t+i}\|^2] + n\sigma_g^2  \leq  4\ka n\sigma_g^2  
\end{align}
where the first inequality is due to Assumption~\ref{asmp4} and the fact that the cross terms of the form $\EX[\langle \g_{p}, \g_{p'} \rangle] = 0$, for any $p<p'$, and the second the inequality is due to Assumption~\ref{asmp4}, \eqref{proof_app}, and the fact that $\ka \| \II-\gamma \Q\|^{\ka-1}\|\III\|\leq  1$(cf. (\ref{Bmatrix})). 
%where to obtain the second inequality, we have used the fact $\ka \| \II-\gamma \Q\|^{\ka-1}\leq  1$ (cf. (\ref{Bmatrix})) and (\ref{proof_app}) while neglecting the cross terms involving $\g_k$ using Assumption \ref{asmp4}.
Following a similar approach, we can bound the rest of the terms involving $\textbf{g}_k$ as:
%We can similarly bound the rest of the terms involving $\textbf{g}_k$ as:
\begin{align}\label{st1-t}
   \EX  \left[ \left\|\sum_{i=0}^{\ka-1} (\ka- i-1)\bar{\mathbf{Q}}^{\ka- i-2}\III(\g_{t+i+1}-\g_{t+i}) \right\|^2  \right]  &\leq 4\ka n\sigma_g^2 \nonumber \\
   \EX  \left[ \left\|\sum_{i=0}^{\ka-1}\bar{\mathbf{Q}}^{\ka- i-1}\III(\g_{t+i+1}-\g_{t+i}) \right\|^2  \right]  &\leq  4(\ka+1) n\sigma_g^2.  
\end{align}
Similarly, considering the second term in (\ref{sixterms}), we have,
\begin{align}\label{intd_t}
   \EX &\left[\left\|\sum_{i=0}^{\ka-1} (\ka- i-1)\bar{\mathbf{Q}}^{\ka- i-1}\III(\ddd_{t+i+1}-\ddd_{t+i}) \right\|^2\right] \nonumber \\
   &~~~~~\leq\ka\left( (\ka-1)  \|\bar{\mathbf{Q}}^{\ka-1}\III\|^2 \EX [\left\|\ddd_{t}\|^2\right] 
    + \sum_{i=1}^{\ka-1} \left\|(\ka - i) \bar{\mathbf{Q}}^{\ka-1}\III - (\ka-i-1)\bar{\mathbf{Q}}^{\ka-i-1}\III\right\|^2 \EX [\|\ddd_{t+i}\|^2]\right) \nonumber\\
    &~~~~~\leq 4\ka\sum_{i=0}^{\ka-1}\EX [ \left\| \ddd_{t+i}\right\|^2 ],
\end{align}
where the first inequality is due to the fact that $\left\|\sum_{i=0}^{\tau-1}a_{i}\right\|^2 \leq \tau \sum_{i=0}^{\tau-1} \|a_i\|^2$ for any $a \in \mathbb{R}^d$.
%\textcolor{orange}{where we have used $\left \|\sum_{i=0}^{\tau-1}a_{i}\right\|^2 \leq \tau \sum_{i=0}^{\tau-1} \|a_i\|^2$ in the first inequality. }
%\rbnote{(I am not sure how this above statement is %true.. I don't understand how $\tau$ appeared outside and why the cross terms are missing in the first inequality?)}
The same bound also holds for the fourth term in (\ref{sixterms}) while for the last term, we have,
\begin{align}\label{intd}
   \EX \left[\left\|\sum_{i=0}^{\ka-1}\bar{\mathbf{Q}}^{\ka- i-1}\III(\ddd_{t+i+1}-\ddd_{t+i}) \right\|^2\right]  &\leq 4\ka\sum_{i=0}^{\ka}\EX [ \left\| \ddd_{t+i}\right\|^2 ].
\end{align}
We next bound the summation $\sum_{i=0}^{\ka}\EX [ \left\| \ddd_{t+i}\right\|^2$. For all $i<\tau$:
\begin{align}\label{ll012}
\|\ddd_{t+i}\|^2 &= \| \n \textbf{f}(\xb_{t+i}) - \n \textbf{f}(\bar{\xb}_{t+i})+\n \textbf{f}(\bar{\xb}_{t+i})-\n \textbf{f}(\textbf{x}^*) \|^2\nonumber\\
& \leq 2L^2 \|\Psi_{t+i} \|^2 +2\|\n \textbf{f}(\bar{\textbf{x}}_{t+i})-\n \textbf{f}(\textbf{x}^*) \|^2,
\end{align}
where the second inequality is due to Assumption \ref{asmp3}. Now, for $i=\tau$, we have,  
%where we have used Assumption \ref{asmp3} to obtain the second inequality. For $i=\tau$, we have:
\begin{align}\label{ttaubd}
  \|\ddd_{t+\tau}\|^2 &\leq  2L^2\|\Psi_{t+\ta} \|^2 +2\|\n \textbf{f}(\bar{\textbf{x}}_{t+\ka})-\n \textbf{f}(\textbf{x}^*) \|^2\nonumber\\
   &\leq  2L^2\|\Psi_{t+\ta} \|^2+4\|\n \textbf{f}(\bar{\textbf{x}}_{t+\ka})-\n \textbf{f}(\bar{\textbf{x}}_{t+\ka-1}) \|^2 +4 \|\n \textbf{f}(\bar{\textbf{x}}_{t+\ka-1})-\n \textbf{f}(\textbf{x}^*) \|^2 \nonumber\\
   &\leq  2L^2\|\Psi_{t+\ta} \|^2+4L^2\|
 \bar{\textbf{x}}_{t+\ka}- \bar{\textbf{x}}_{t+\ka-1} \|^2 +4 \|\n \textbf{f}(\bar{\textbf{x}}_{t+\ka-1})-\n \textbf{f}(\textbf{x}^*) \|^2. 
\end{align}
The expression for $\bar{\textbf{x}}_{t+\ka}$ can be written as (cf. \eqref{ll}),
\begin{align*}
\xx_{t+\ka} &= \bar{\textbf{x}}_{t+\ka-1}-   \frac{\alpha}{n} \big(1_n1_n^T\otimes I_d \big) \n \textbf{f} (\xb_{t+\tau-1})  + \alpha \bm{\epsilon}_{t+\tau-1,g} + \gamma \bar{\bm{\epsilon}}_{t+\tau-1,c} - \alpha\gamma \sum_{j=0}^{t+\tau-2}\bar{\bm{\epsilon}}_{j,c},
\end{align*}
where $\bm{\epsilon}_{k,g} \defeq \frac{1}{n} \big(1_n1_n^T\otimes I_d \big)\Big( \n \textbf{f}( \xb_{k}) -  \n \textbf{F} (\xb_{k},\bm{\xi}_k)\Big)$ and $\bar{\bm{\epsilon}}_{k,c} \defeq \frac{1}{n} \big(1_n1_n^T\otimes I_d \big)\hat{\textbf{Q}} \bm{\epsilon}_{k,c}$ for any $k \in \N$. Taking square norms and expectations, we get,
\begin{align}\label{ll1}
\EX &[\|\bar{\textbf{x}}_{t+\tau} - \bar{\textbf{x}}_{t+\tau-1} \|^2 | \mathcal{F}_{t+\tau-1}] \nonumber \\
&~~~~~~\leq  2\EX \left[\left\|  \frac{\alpha}{n} \big(1_n1_n^T\otimes I_d \big) \left(\n \textbf{f} (\xb_{t+\tau-1})  - \n \textbf{f}(\xb^*) \right)  + \alpha \bm{\epsilon}_{t+\tau-1,g} + \gamma \bar{\bm{\epsilon}}_{t+\tau-1,c}\right\|^2 \Big| \mathcal{F}_{t+\tau-1}\right]\nonumber \\
&\qquad \qquad +  2\alpha^2\gamma^2 \EX\left[\left\| \sum_{j=0}^{t+\tau-2}\bar{\bm{\epsilon}}_{j,c}\right\|^2\Big|  \mathcal{F}_{t+\tau-1}\right],
\end{align}
where we used the fact that $\frac{1}{n}\left(1_n1_n^T\otimes I_d \right) \n \textbf{f}(\textbf{x}^*) = 0$. From Assumptions \ref{asmp2} and \ref{asmp4}, we have for all $k\in \mathbb{N}$,
\begin{equation}\label{del1-t}
 \EX[ \alpha \bm{\epsilon}_{k,g} + \gamma \bar{\bm{\epsilon}}_{k,c} | \mathcal{F}_k] =0,  \qquad \qquad   \EX[\left\|\alpha \bm{\epsilon}_{k,g} + \gamma \bar{\bm{\epsilon}}_{k,c}\right\|^2]\leq\alpha^2 \sigma_g^2 + \gamma^2\sigma^2_{c}, 
\end{equation}
and
\begin{align}\label{epsbound-t}
  \EX \left[\left\| \sum_{j=0}^{k-1}\bar{\bm{\epsilon}}_{j,c} \right\|^2\right] &=  \EX \left[ \sum_{j=0}^{k-1}\left\|\bar{\bm{\epsilon}}_{j,c} \right\|^2\right]  + \sum_{1\leq p,p' \leq k-1}\EX \left[\langle \bar{\bm{\epsilon}}_{p,c} ,\bar{\bm{\epsilon}}_{p'
  ,c} \rangle \right]  %\nonumber\\
  \leq \sum_{j=0}^{k-1} \sigma_c^2 =  k\sigma_c^2,
\end{align}
where the last inequality is due to the fact that
%we have neglected the cross terms in the second inequality using 
$ \EX [\langle \bar{\bm{\epsilon}}_{p,c}, \bar{\bm{\epsilon}}_{p',c}\rangle] =  \EX [ \EX  \left[\langle \bar{\bm{\epsilon}}_{p,c}, \bar{\bm{\epsilon}}_{p',c}\rangle|\mathcal{F}_{p'}\right]]=0$ for any $p<p'$. Combining \eqref{ll1}, \eqref{del1-t} and \eqref{epsbound-t}, we have,
\begin{align}\label{ll1-t}
\EX &[\|\bar{\textbf{x}}_{t+\tau} - \bar{\textbf{x}}_{t+\tau-1} \|^2]\nonumber \\
&~~~~~\leq  2\EX\left[\left\|  \frac{\alpha}{n} \big(1_n1_n^T\otimes I_d \big) \left(\n \textbf{f} (\xb_{t+\tau-1})  - \n \textbf{f}(\xb^*) \right)\right\|^2 \right] +2\alpha^2 \sigma_g^2 +  2\left( 1 +  \alpha^2 (t+\tau)\right)\gamma^2 \sigma^2_{c} \nonumber\\
&~~~~~\leq  2\EX\left[\left\|  \frac{\alpha}{n} \big(1_n1_n^T\otimes I_d \big) \left(\n \textbf{f} (\xb_{t+\tau-1}) - \n \textbf{f}(\bar{\textbf{x}}_{t+\ka-1})+\n \textbf{f}(\bar{\textbf{x}}_{t+\ka-1})  - \n \textbf{f}(\xb^*) \right)\right\|^2 \right]\nonumber\\
&\qquad \qquad \qquad \qquad\qquad \qquad\qquad \qquad \qquad \qquad\qquad \qquad
+2\alpha^2 \sigma_g^2 +  2\left( 1 +  \alpha^2 (t+\tau)\right)\gamma^2 \sigma^2_{c} \nonumber\\
&~~~~~\leq  4 \alpha^2 L^2\EX \left[\|\Psi_{t+\tau-1}\|^2\right] + 4\alpha^2\EX\left[ \|\n \textbf{f}(\bar{\textbf{x}}_{t+\ka-1})-\n \textbf{f}(\textbf{x}^*) \|^2\right]   + 2\alpha^2 \sigma_g^2 +  2\left( 1 +  \alpha^2 (t+\tau)\right)\gamma^2 \sigma^2_{c}. 
\end{align}
%\rbnote{(although the above inequality is correct, we need to be careful, $t + \tau$ should be $\geq 1$ and shouldn't it be $1 +  \alpha^2 (t+\tau -1)$?) \textcolor{blue}{$\tau\geq1 $ and we assume $t\geq \tau$ in the statement of the lemma. We dont loose anything by neglecting $-1$ in the factor $(t+\tau-1)$.} okay}
Taking expectations in (\ref{ttaubd}) and using (\ref{ll1-t}), we get, 
\begin{align}\label{ll123}
\EX  [\|\ddd_{t+\tau}\|^2]  &\leq  2L^2\EX [\|\Psi_{t+\ta} \|^2 ]+4 \EX[\|\n \textbf{f}(\bar{\textbf{x}}_{t+\ka-1})-\n \textbf{f}(\textbf{x}^*) \|^2] +  16L^4\alpha^2\EX[\left\|\Psi_{t+\ta-1} \right\|]^2\nonumber\\ 
& \qquad  + 16\alpha^2 L^2\EX [\left\| \n \textbf{f} (\bar{\xb}_{t+\tau-1})  - \n \textbf{f}(\xb^*) \right\|^2] + 8L^2\left(\alpha^2\sigma^2_{g} + \gamma^2 (1+\alpha^2 (t+\tau)) \sigma_c^2 \right)\nonumber\\
 &\leq 2L^2\EX[ \|\Psi_{t+\ta} \|^2   ] + 5 \EX[\|\n \textbf{f}(\bar{\textbf{x}}_{t+\ka-1})-\n \textbf{f}(\textbf{x}^*) \|^2] + L^2 \EX[\left\|\Psi_{t+\ta-1} \right\|^2 ] \nonumber\\
 &\qquad + 8L^2\left(\alpha^2\sigma^2_{g} + \gamma^2 (1+\alpha^2 (t+\tau)) \sigma_c^2 \right), 
\end{align}
where the last inequality is due to $\alpha^2<1/16L^2$.
%we have used $\alpha^2<1/8L^2$ to obtain the last inequality. 
Using (\ref{ll012}) and (\ref{ll123})  in (\ref{intd}), we have,
\begin{align}\label{st2}
   \EX \left[\left\|\sum_{i=0}^{\ka-1}\bar{\mathbf{Q}}^{\ka- i-1}\III(\ddd_{t+i+1}-\ddd_{t+i}) \right\|^2\right]  &\leq 12 \ka L^2\sum_{i=0}^{\ka-1}\EX [\| \Psi_{t+i}\|^2] +28\ka\sum_{i=0}^{\ka-1}\EX [\left\|\n \textbf{f}(\bar{\textbf{x}}_{t+i})-\n \textbf{f}(\textbf{x}^*) \right\|^2] \nonumber\\
&\qquad +8\tau L^2\EX [\|\Psi_{t+\ta} \|^2]  + 32\tau L^2\left(\alpha^2\sigma^2_{g} + \gamma^2 (1+\alpha^2 (t+\tau)) \sigma_c^2 \right)
\end{align}
%\rbnote{(I get the first terms as $8 \ka L^2\sum_{i=0}^{\ka-1}\EX [\| \Psi_{t+i}\|^2]$:) \textcolor{orange}{I think its correct: For $t+\tau-1$, we have two terms for $\|\Psi_{t+\tau-1}\|^2$, one term in (74) that gets added to another one comming from (68) which add to $12\ka L^2$. For the rest of the terms, the factor is  $8 \ka L^2$.}}
The rest of the terms involving $d_k$ in \eqref{sixterms} can be bounded in the same manner. %so that using (\ref{st1}), (\ref{st1-t}) and (\ref{st2}) in (\ref{sixterms}) we get:
Using \eqref{st1}, \eqref{st1-t} and \eqref{st2} in \eqref{sixterms}, we get,
\begin{align}\label{e2}
 \sum_{i=0}^{\ka-1}\EX \left[ \|\textbf{J}^{\ka-i-1}_\gamma  \textbf{E}^g_{t+i}\|^2\right] &\leq 24n(\ka+1)\sigma^2_{g}  + 72 \ka L^2  \sum_{i=0}^{\ka-1} \EX [\|\Psi_{t+i}\|^2] + 168 \ka \sum_{i=0}^{\ka-1} \EX [\left\|\n \textbf{f}(\bar{\textbf{x}}_{t+i})-\n \textbf{f}(\textbf{x}^*) \right\|^2]\nonumber\\
&\qquad + 48\tau L^2\EX [\|\Psi_{t+\ta} \|^2]  + 192\tau L^2\left(\alpha^2\sigma^2_{g} + \gamma^2 (1+\alpha^2 (t+\tau)) \sigma_c^2 \right).
 \end{align}
Using (\ref{bd1}) and (\ref{e2}) to bound the right hand side in (\ref{decomp1}), we have,
%\begin{multline}\label{finalbd_t}
\begin{align}\label{finalbd_t}
&\EX \left[\Big\| \sum_{i=0}^{\ka-1} \textbf{J}^{\ka-i-1}_\gamma \textbf{E}_{t+i} \Big\|^2 \right] \nonumber \\
&~~~~~\leq \frac{16\gamma^2}{\alpha^2} \big(1 + \alpha^2 (\ka^2 +1/2)\big)n\sigma_{c}^2 \ka+48n(\ka+1)\sigma^2_{g}  + 144 \ka L^2  \sum_{i=0}^{\ka-1} \EX [\|\Psi_{t+i}\|^2] \nonumber \\
&~~~~~\qquad + 336 \ka \sum_{i=0}^{\ka-1} \EX \left[\left\|\n \textbf{f}(\bar{\textbf{x}}_{t+i})-\n \textbf{f}(\textbf{x}^*) \right\|^2\right]+
  96\tau L^2\EX [\|\Psi_{t+\ta} \|^2]  + 384\tau L^2\left(\alpha^2\sigma^2_{g} + \gamma^2 (1+\alpha^2 (t+\tau)) \sigma_c^2 \right) \nonumber \\
  &~~~~~\leq 144 \ka L^2  \sum_{i=0}^{\ka-1} \EX [\|\Psi_{t+i}\|^2] 
+ 336 \ka \sum_{i=0}^{\ka-1} \EX \left[\left\|\n \textbf{f}(\bar{\textbf{x}}_{t+i})-\n \textbf{f}(\textbf{x}^*) \right\|^2\right]+
  \frac{1}{4\alpha^2}\EX [\|\Psi_{t+\ta} \|^2]   \nonumber \\
  &~~~~~\qquad + \frac{16\gamma^2}{\alpha^2} \big(2 + \alpha^2 (\ka^2 +1/2) + \alpha^2(t+\tau)\big)n\sigma_{c}^2 \ka+49n(\ka+1)\sigma^2_{g}
\end{align}
%\end{multline}
where we used $\alpha^2 < 1/384\tau L^2$. Next, taking square norms and expectations in \eqref{e0}, we get, 
\begin{align}\label{finalbd0_t}
\EX \left[ \|\Psi_{t+\ka}\|^2 \right]&= \EX \left[\left\| \textbf{J}^\ka_\gamma \Psi_t + \alpha \sum_{i=0}^{\ka-1} \textbf{J}^{\ka-i-1}_\gamma  \textbf{E}_{t+i}\right\|^2 \right]\nonumber\\
&\leq 2 \EX \left[\left\| \textbf{J}^\ka_\gamma \Psi_t\right\|^2\right] + 2\alpha^2 \EX \left[\left\| \sum_{i=0}^{\ka-1} \textbf{J}^{\ka-i-1}_\gamma  \textbf{E}_{t+i}\right\|^2 \right].
\end{align}
From Lemma~\ref{lem3}, it follows that there there exists a $\ka$ such that $\|\textbf{J}^\ka_\gamma\|^2 \leq \frac{1}{4}\rho'$ for a given $\rho'\in(0,1/4]$. Therefore, 
% Note that from Lemma \ref{lem3}, there exists a $\ka$ with:
\begin{equation}\label{rho'eq}
4\| \textbf{J}^\ka_\gamma \Psi_t\|^2 \leq 4\|\textbf{J}^\ka_\gamma\|^2 \| \Psi_{t} \|^2  \leq \rho'  \| \Psi_{t} \|^2 .
\end{equation}
To conclude, we substitute \eqref{finalbd_t} and \eqref{rho'eq} in \eqref{finalbd0_t} to get the required inequality,
\begin{align*}
\EX \left[\|\Psi_{t+\ka}\|^2 \right] &\leq  2\EX\left[\left\| \textbf{J}^\ka_\gamma \Psi_t\right\|^2\right]+ 288 \alpha^2 \ka L^2  \sum_{i=0}^{\ka-1} \EX \left[\|\Psi_{t+i}\|^2\right] 
+ 672  \alpha^2 \ka \sum_{i=0}^{\ka-1} \EX \left[\left\|\n \textbf{f}(\bar{\textbf{x}}_{t+i})-\n \textbf{f}(\textbf{x}^*) \right\|^2\right]+
 \frac{1}{2}\EX [\|\Psi_{t+\ta} \|^2]   \\
 &\qquad + 32\gamma^2 \big(2 + \alpha^2 (\ka^2 +1/2) + \alpha^2(t+\tau)\big)n\sigma_{c}^2 \ka+98n(\ka+1)\alpha^2\sigma^2_{g}\\
 \EX \left[\|\Psi_{t+\ka}\|^2 \right]  &\leq \rho' \EX \|\Psi_{t}\|^2+ 576 \alpha^2 \ka L^2  \sum_{i=0}^{\ka-1} \EX [\|\Psi_{t+i}\|^2] 
+ 1344  \alpha^2 \ka \sum_{i=0}^{\ka-1} \EX \left[\left\|\n \textbf{f}(\bar{\textbf{x}}_{t+i})-\n \textbf{f}(\textbf{x}^*) \right\|^2\right]  \\
  &\qquad + 64\gamma^2 \big(2 + \alpha^2 (\ka^2 +1/2) + \alpha^2(t+\tau)\big)n\sigma_{c}^2 \ka+196n(\ka+1)\alpha^2\sigma^2_{g},
\end{align*}
which proves the bound \eqref{main_rec}. %we used $\alpha^2< \frac{1}{2 \times 48\tau L^2}$ in the first inequality and \eqref{rho'eq} in the last inequality. 
The bound (\ref{main_rec_2}) for $\ell<\tau$ is proved exactly along the same lines with the only modification being that the first term is scaled by $\|\textbf{J}^\ell_\gamma\|^2,\,\ell<\tau$ instead of $\|\textbf{J}^\tau_\gamma\|^2$. The former can be bounded by using the expression for $\textbf{J}_\gamma^\ell$ (cf. \ref{matproddef}) as follows:
\begin{align*}
    \|\textbf{J}_\gamma^\ell \Psi_0\|^2 &\leq \big\|(\III-\gamma \Q)^\ell -\ell (\III-\gamma \Q)^\ell\big\|^2\|\Delta \textbf{v}_0\|^2 + \big\|(\III-\gamma \Q)^\ell -\ell (\III-\gamma \Q)^{\ell-1}\big\|^2\|\Delta \textbf{x}_0\|^2 \nonumber \\
    &\qquad + \alpha^2 \left\|\left(\III-\gamma \Q\right)^\ell\right\|^{2}\|\Delta \textbf{y}_0\|^2 \nonumber\\
    &\leq  2(1+\ell^2) (\|\Delta \textbf{v}_0\|^2 + \|\Delta \textbf{x}_0\|^2 + \alpha^2\|\Delta \textbf{y}_0\|^2) \nonumber\\
    &\leq   2(1+\ka^2)\|\Psi_0\|^2%\qquad (\ell < 2\ka)
\end{align*}
where the second inequality is due to $ \left\|\left(\III-\gamma \Q\right)^\ell\right\|^{2}\leq 1$ and the last inequality is due to $\ell < \ka$.
%\textcolor{orange}{where we use the expression  for $\textbf{J}_\gamma^\ell$ (cf. \eqref{matproddef}) in the first inequality, $ \left\|\left(\III-\gamma \Q\right)^\ell\right\|^{2}\leq 1$ in the second inequality and  $\ell < 2\ka$ in the the last inequality.}
\end{proof}
\noindent To conclude, we provide the proof of (\ref{proof_app}).\\

\noindent \textit{Claim:} $\|(i+1)\bar{\mathbf{Q}}^{i+1}\III -i \bar{\mathbf{Q}}^{i}\III\|^2\leq4$.
\begin{proof}
We have
\begin{align*}
    \|(i+1)\bar{\mathbf{Q}}^{i+1}\III -i \bar{\textbf{Q}}^{i}\III\|^2 &\leq \|(i+1)\bar{\mathbf{Q}}^{i+1} -i \bar{\textbf{Q}}^{i}\|^2\|\III\|^2 \\
    &\leq \max_{j\in [n]} |(i+1)(1-\gamma(1-\lambda_j)^{i+1} -  i(1-\gamma(1-\lambda_j)^{i} |^2\\
    &= \big|(1+i)(1-\gamma(1-\la))^{i+1} -  i(1-\gamma(1-\la))^{i} \big|^2 \,\,(\text{for some } \la)\\
    &= \big|(1-\gamma(1-\la))^{i+1}  + i (1-\gamma(1-\la))^{i} (1- \gamma(1-\la)-1)\big|^2\\ 
     &= |(1-\gamma(1-\la))^{i+1}  -i \gamma(1-\la) (1-\gamma(1-\la))^{i} |^2\\
     &\leq 2|(1-\gamma(1-\la))|^{2i+2}  +2 \gamma^2(1-\la)^2 \big(\underbrace{i(1-\gamma(1-\la))^{i}}_{\leq \frac{1}{\gamma(1-\la)}}\big)^2 \\
    &\leq 4
\end{align*}
where the second inequality is due to $\|\III\| \leq 1$ and the last inequality is due to $\gamma(1-\la) \in [0,1]$.
\end{proof}

\section*{Appendix IV: Proof of Lemma \ref{lem4}}\label{sec:apndxIV}
\begin{proof}
We begin by defining the following quantities, for any $t\geq\ta$,
\begin{equation}\label{EP}
    A_{t} \defeq\frac{1}{\ka} \sum_{i=t-\ta}^{t-1} a_{i} \quad \text{and} \quad E_{t} \defeq \sum_{i=t-\ta}^{t-1} e_{i}.
\end{equation}
%Note that $A_{k}$ considers the average of the sequence $\{a_{i}\}_{i=k-\ka}^{k-1}$ while $A_{k+j}$ for any $j< \ta$ considers the average of the shifted sequence $\{a_{i}\}_{i=k+j-\ka}^{k+j-1}$. 
For future reference, we note that for the index $t=k+j$ with $j<\ta \leq t$, \eqref{rel} can be expressed in terms of $A_{k+j}$ and $E_{k+j}$
\begin{equation}\label{alt_ineq}
a_{k+j} \leq \rho' a_{k+j-\ka} + bA_{k+j} + c   E_{k+j} +r
\end{equation}
\textbf{Step (i):} We first prove a recursive relation for $A_{k+\ta}$ in terms of $A_k$ for $k\geq \ta$  and  $E_{k+i}$ for $0\leq i\leq \ka-1$. We begin by considering $A_{k+j}$ for any $j<\ta$,
\begin{align*}
A_{k+j} &= \frac{1}{\ka}\left((a_{k+j-\ta} +\cdots a_{k-1})+ (a_{k} +\cdots + a_{k+j-1})\right)  \nonumber \\
&= \frac{1}{\ta} \left(\sum_{i=j}^{\ka-1} a_{k+i-\ta}  + \sum_{i=0}^{j-1} a_{k+i}\right).
\end{align*}
By \eqref{alt_ineq} (with $j=i$), substituting for $a_{k+i}$ in the second summation above, 
\begin{align*}
    A_{k+j}&\leq \frac{1}{\ka}\left( \sum_{i=j}^{\ka-1} a_{k+i-\ta}   + \sum_{i=0}^{j-1} \rho' a_{k+i-\ka}\right) +\frac{b}{\ka}\sum_{i=0}^{j-1} A_{k+i} +\frac{c}{\ka} \sum_{i=0}^{j-1} E_{k+i} +r \\
    &\leq  \frac{1}{\ka}\left(a_{k-1} +\cdots+ a_{k+j-\ta} + a_{k+j-\ta-1} + a_{k-\ta} \right)+ \frac{b}{\ka}\sum_{i=0}^{j-1} A_{k+i} + \frac{c}{\ka} \sum_{i=0}^{j-1} E_{k+i} +r \\
    &=  A_k + \frac{b}{\ka}\sum_{i=0}^{j-1} A_{k+i} + \frac{c}{\ka} \sum_{i=0}^{j-1} E_{k+i} +r,
\end{align*}
where the second inequality holds since $\rho'<1$. Thus, if follows for $j<\ta$,
\begin{equation}\label{tt1}
    A_{k+j} \leq  A_{k} + \frac{b}{\ka} \sum_{i=0}^{j-1}A_{k+i} + \frac{c}{\ka}   \sum_{i=0}^{j-1} E_{k+i} +r.
\end{equation}
By the definition of $A_{k+\ta}$ \eqref{EP} and \eqref{alt_ineq},
\begin{equation}\label{tt}
\begin{aligned}
    A_{k+\ka} =\frac{1}{\ka}\sum_{j=0}^{\ka-1} a_{k+j}&\leq \frac{1}{\ka}\sum_{j=0}^{\ka-1} \left(\rho'a_{k+j-\ka}  + b  A_{k+j} + c E_{k+j}   +r \right)\\
    &= \rho' A_k + \frac{b}{\ka} \sum_{j=0}^{\ka-1} A_{k+j} + \frac{c}{\ka} \sum_{j=0}^{\ka-1} E_{k+j}  +r\\
    &= \rho' A_k +\frac{b}{\ka} A_{k+\ta-1}+ \frac{b}{\ka} \sum_{j=0}^{\ka-2} A_{k+j} + \frac{c}{\ka} \sum_{j=0}^{\ka-2} E_{k+j} +\frac{c}{\ka} E_{k+\ta-1} +r,  
\end{aligned}
\end{equation}
Next, by \eqref{tt1} with $j=\ta-1$ and $b < \rho'/4$, it follows that
\begin{align*}
    A_{k+\ka} &\leq  \rho' A_k +\frac{b}{\ka} A_{k+\ta-1}+ \frac{b}{\ka} \sum_{j=0}^{\ka-2} A_{k+j} + \frac{c}{\ka} \sum_{j=0}^{\ka-2} E_{k+j} +\frac{c}{\ka} E_{k+\ta-1} + r\\
    &\leq \rho' A_k
  + \frac{b}{\ka}\left(A_{k} +\frac{b}{\ka} \sum_{j=0}^{\ka-2}A_{k+j} + \frac{c}{\ka}  \sum_{j=0}^{\ka-2} E_{k+j} + r\right)\\
    &   \qquad \qquad+ \frac{b}{\ka}\sum_{j=0}^{\ka-2} A_{k+j} + \frac{c}{\ka} \sum_{j=0}^{\ka-2} E_{k+j}   + \frac{c}{\ka}E_{k+\ka-1} + r \\
  &\leq   \rho'\left(1+\frac{1}{4\ka}\right) A_k
 + \left(1+\frac{b}{\ka}\right) \left( \frac{b}{\ka}\sum_{j=0}^{\ka-2}A_{k+j} + \frac{c}{\ka} \sum_{j=0}^{\ka-2} E_{k+j}  + r \right) +\frac{c}{\ka} E_{k+\ka-1}.
\end{align*}
Recursive application of the above, over $j$ for $1\leq j \leq \ka -1$ yields the following inequality
\begin{equation}\label{majineq}
\begin{aligned}
  A_{k+\ka} &\leq   \rho'\Bigg(1+\frac{1}{4\ka}\Bigg)^{\ka-1} A_k
 + \frac{c}{\ka}  \sum_{j=0}^{\ka-1} \Big(1+\frac{b}{\ka}\Big)^{\ka-j-1} E_{k+j} +  \Big(1+\frac{b}{\ka}\Big)^{\ka-1} r \\
  &\leq   (1-\rho) A_k
 + \frac{2c}{\ka} \sum_{j=0}^{\ka-1}  E_{k+j} +2r,   
 \end{aligned}
 \end{equation}
where $b\leq 1/4$, 
$\left(1+\frac{b}{\ka}\right)^p \leq \left(1+\frac{1}{4\ka}\right)^p   \leq \exp(1/4) \leq 2
$ for any $p\leq \ka$ and  $2\rho'= 1-\rho $. Next, recall \eqref{alt_ineq} with $j=\ta$:
\begin{equation}\label{eq.1}
\begin{aligned}
    a_{k+\ka} &\leq  \rho' a_k + b A_{k+\ka} + c  \sum_{j=0}^{\ka-1} e_{k+j} + r \\
  & \stackrel{(\ref{majineq})}{ \leq}   \rho' a_k + b(1-\rho) A_k +   \frac{2bc}{\ka} \sum_{j=0}^{\ka-1}  E_{k+j}  + c \sum_{j=0}^{\ka-1}  e_{k+j} +(2b+1)r \\
    &\leq \frac{1}{2}(1-\rho)a_k + \frac{\rho}{4} (1-\rho)A_k + c\sum_{j=0}^{\ka-1}  \Big(\frac{1}{\ka} E_{k+j} + e_{k+j}\Big) +2r, 
\end{aligned}
\end{equation}
where we have used the fact that $\rho' = (1-\rho)/2$, $b\leq\rho'/4<\rho/4$ for $\rho'\in (0,1/4]$ to get the last inequality. Adding, \eqref{majineq} and \eqref{eq.1}, 
\begin{equation}\label{tmp3}
\begin{aligned}
    A_{k+\ka} + a_{k+\ka} &\leq (1-\rho)\left(1+\frac{\rho}{4}\right)(A_k + a_{k}) +  \frac{3c}{\ka}\sum_{j=0}^{\ka-1}  E_{k+j} + c\sum_{j=0}^{\ka-1} e_{k+j} +4r\\
    &\leq       \left(1-\frac{3\rho}{4}\right)(A_k + a_{k}) + c \sum_{j=0}^{\ka-1} \left(\frac{3}{\ka} E_{k+j}+ e_{k+j}\right) +4r. 
\end{aligned}
\end{equation}

\textbf{Step (ii):} In this step, we establish a descent relation for $A_{t}+a_{t}$. With $k=(m-1)\ta$ for any integer $m\geq2$, \eqref{tmp3} can be expressed as
\begin{equation}\label{mn1}
\begin{aligned}
    A_{m\ta} + a_{m\ta} &\leq \left(1-\frac{3\rho}{4} \right) \left(A_{(m-1)\ka} + a_{(m-1)\ka} \right) + c \sum_{i=0}^{\ka-1} \left(\frac{3}{\ka} E_{(m-1)\ka+i}+ e_{(m-1)\ka+i}\right)+4r \\
    &\leq \left(1-\frac{3\rho}{4} \right)^{m-1} \left(A_{\ta} + a_{\ta} \right) + c \sum_{j=1}^{m-1}   \left(1-\frac{3\rho}{4}\right)^{(m-j-1)} \sum_{i=0}^{\ka-1} \left(\frac{3}{\ka} E_{j\ka+i}+ e_{j\ka+i}\right)\\
    & + 4r \sum_{j=0}^{m-1}   \left(1-\frac{3\rho}{4}\right)^{j}
\end{aligned}
\end{equation}
Let $t=m\ta$ and $m\geq 2$. To bound the summation in \eqref{mn1}, we note that
\begin{align*}
    \left(1-\frac{3\rho}{4}\right)^{m-j-1} &= \left(1-\frac{3\rho}{4}\right)^{-1}\left(1-\frac{3\rho}{4}\right)^{\frac{t-j\ta}{\ka}}\leq c'\left(1- \frac{3\rho }{4\ka}\right)^{t-j\ta}\label{bb1}
\end{align*}
where $c':=  \left(1-\frac{3\rho}{4}\right)^{-1}$, $t=m\ta$ and $(1-x)^a\leq 1-ax$ for $a,x\in [0,1]$. Thus, it follows that
\begin{equation}\label{tmp10}
\begin{aligned}
    \sum_{j=1}^{m-1}   \left(1-\frac{3\rho}{4}\right)^{(m-j-1)}\sum_{i=0}^{\ka-1} \left(\frac{3}{\ka}E_{j\ka+i}+e_{j\ka+i}\right)  &\leq   c' \sum_{j=1}^{m-1}   \left(1-\frac{3\rho}{4\ta}\right)^{t-j\ta}\sum_{i=0}^{\ka-1}  \left(\frac{3}{\ka}E_{j\ka+i}+e_{j\ka+i}\right)\\
    &\leq c'\sum_{j=1}^{m-1}  \sum_{i=0}^{\ka-1} \Big(1- \frac{3\rho}{4\ka}\Big)^{t-{j\ka-i}}  \left(\frac{3}{\ka}E_{j\ka+i}+e_{j\ka+i}\right)\\
    &\leq  c'  \sum_{p=\ta}^{t-1} \Big(1- \frac{3\rho}{4\ka}\Big)^{t-p}  \left(\frac{3}{\ka}E_{p}+e_p\right),
\end{aligned}
\end{equation}
where $\left(1-3\rho/4\ka\right)^{-i}\geq 1$ since $\rho<1$ and the index  $p:= j\ka+i$. To bound the term involving $E_p$ in the summation in \eqref{tmp10}, it follows by \eqref{EP} 
\begin{align*}
    \Big(1- \frac{3\rho}{4\ka}\Big)^{t-p}  E_{p}\leq  \Big(1- \frac{3\rho}{4\ka}\Big)^{t-p}\sum_{i=p-\ka}^{p-1}e_i &\leq  \Big(1- \frac{3\rho}{4\ka}\Big)^{-\ta}  \sum_{i=p-\ka}^{p-1}\Big(1- \frac{3\rho}{4\ka}\Big)^{t-i}e_i\\
     &\leq  \Big(1+\frac{3\rho}{4\ka}\Big)^{\ka}  \sum_{i=p-\ka}^{p-1}\Big(1- \frac{3\rho}{4\ka}\Big)^{t-i}e_i  \\
     &\leq  3 \sum_{i=p-\ka}^{p-1}\Big(1- \frac{3\rho}{4\ka}\Big)^{t-i}e_i,
\end{align*}
where the second inequality follows due to $(1-x)^{-1}<1+x$ for $x\in (0,1)$ and the last inequality follows due to  $\left(1+3\rho/4\ka\right)^{\ka} \leq \exp(3\rho/4)\leq 3$ for $\rho<1$. Summing the above for $p=\ta$ to $t-1$
\begin{align}\label{e-ineq}
   \sum_{p=\ta}^{t-1} \Big(1- \frac{3\rho}{4\ka}\Big)^{t-p}  E_{p}
    &\leq 3\sum_{p=\ka}^{t-1}  \sum_{i=p-\ka}^{p-1}\Big(1- \frac{3\rho}{4\ka}\Big)^{t-i}e_i\nonumber\\
    &=3\left( \sum_{i=t-\ka-1}^{t-2}\Big(1- \frac{3\rho}{4\ka}\Big)^{t-i}e_i+\sum_{i=t-\ka-2}^{t-3}\Big(1- \frac{3\rho}{4\ka}\Big)^{t-i}e_i  +\cdots+ \sum_{i=0}^{\ta-1}\Big(1- \frac{3\rho}{4\ka}\Big)^{t-i}e_i \right)\nonumber\\
     &\leq  3\ka \sum_{p=0}^{t-2} \Big(1- \frac{3\rho}{4\ka}\Big)^{t-p}e_p.  
\end{align}
Substituting \eqref{e-ineq} into \eqref{tmp10}, 
\begin{equation}\label{eq2_albert}
\begin{aligned}
    \sum_{j=1}^{m-1}   \left(1-\frac{3\rho}{4}\right)^{(m-j-1)}\sum_{i=0}^{\ka-1}  \left(\frac{3}{\ka}E_{j\ka+i}+e_{j\ka+i}\right) 
    &\leq  10c' \sum_{p=0}^{t-1} \left(1- \frac{3\rho}{4\ka}\right)^{t-p}e_p\\
    &\leq  40 \sum_{p=0}^{t-1} \left(1- \frac{3\rho}{4\ka}\right)^{t-p}e_p,
\end{aligned}
\end{equation}
where for $\rho<1$ and $c':=\left(1-3\rho/4\right)^{-1}$ it follows that $c'\leq 4$. Substituting \eqref{eq2_albert} into \eqref{mn1}, if $t=m\ka$ and $m\geq 2$, 
\begin{equation}\label{rel2}
\begin{aligned}
    a_t \leq A_{t} + a_{t} &\leq \Big(1- \frac{3\rho}{4}\Big)^{m-1}(A_{\ta} + a_{\ta}) + 40 c\sum_{j=0}^{t-1}   \Big(1- \frac{3\rho}{4\ka}\Big)^{t-j}e_j + 4r \sum_{j=0}^{m-1}   \left(1-\frac{3\rho}{4}\right)^{j}\\
    &\leq \Big(1- \frac{3\rho}{4}\Big)^{\frac{t}{\ta}-1}(A_{\ta}+a_{\ta}) + 40 c\sum_{j=0}^{t-1}   \Big(1- \frac{3\rho}{4\ka}\Big)^{t-j}e_j+ \frac{16r}{3\rho}, \\
     &\leq \Big(1- \frac{3\rho}{4\ta}\Big)^{t-\ta}(A_{\ta}+a_{\ta}) + 40 c\sum_{j=0}^{t-1}   \Big(1- \frac{3\rho}{4\ka}\Big)^{t-j}e_j +\frac{16r}{3\rho}, 
\end{aligned}
\end{equation}
where we have used $(1-x)^\frac{1}{\ta} <1-\frac{x}{\ta}$ for $x \in [0,1]$ and $\ta\in \mathbb{N}$ to get the last inequality. For the case where $t=m\ta+\ell$, $m\geq 2$ and $\ell<\ta$,  the above bound is
\begin{equation}\label{lastbound}
\begin{aligned}
    a_t &\leq \Big(1- \frac{3\rho}{4\ka}\Big)^{t-\ta-\ell} (A_{\ta+\ell} + a_{\ta+\ell}) + 40c\sum_{j=\ell}^{t-1}   \Big(1- \frac{3\rho}{4\ka}\Big)^{t-j}e_j +\frac{16r}{3\rho}\\
    &\leq 5\Big(1- \frac{3\rho}{4\ka}\Big)^{t} (A_{\ta+\ell} + a_{\ta+\ell}) + 40c\sum_{j=\ell}^{t-1}   \Big(1- \frac{3\rho}{4\ka}\Big)^{t-j}e_j+\frac{16r}{3\rho}, 
\end{aligned}
\end{equation}
where the last inequality holds due to $\left(1- \frac{3\rho}{4\ka}\right)^{-\ell-\ta} <\exp\left( 3\rho/2\right)< 5$ for $\ell<\ta$.

\textbf{Step (iii):} In this step, we  use \eqref{rel} to bound the term $A_{\ta+\ell} + a_{\ta+\ell}$ in \eqref{lastbound} where $\ell<\ta$. The argument is similar to the one employed in \textbf{Step (i)}  with appropriate modifications. By \eqref{rel}, for $t=j$ and $j<\ka$, 
\begin{equation}\label{tmp}
    a_{j} \leq \rho'' a_{0} + \frac{b}{\ka} \sum_{i=0}^{j-1} a_{i} + c \sum_{i=0}^{j-1} e_{i} + r.  
\end{equation}
Note, by \eqref{rel} and $\rho'<1$, \eqref{tmp} holds for $\ta\leq j<2\ta$ with a larger $r$,
\begin{equation}\label{tmp2}
\begin{aligned}
    a_{j}           &\leq\rho' a_{j-\ta} + \frac{b}{\ka}\sum_{i=j-\ta}^{j-1} a_{i} + c\sum_{i=j-\ta}^{j-1} e_{i}  + r \\ 
    &\leq \rho'\left(\rho''a_0   + \frac{b}{\ka} \sum_{i=0}^{j-\ta-1} a_{i} + c \sum_{i=0}^{j-\ta-1} e_{i} + r \right) + \frac{b}{\ka}\sum_{i=j-\ta}^{j-1} a_{i} + c\sum_{i=j-\ta}^{j-1} e_{i}  +r \\ 
    &\leq \rho'' a_{0} + \frac{b}{\ka} \sum_{i=0}^{j-1} a_{i} + c \sum_{i=0}^{j-1} e_{j} + 2r.
\end{aligned}
\end{equation}
Recursive application of \eqref{tmp2}, over $j$ for $1\leq j < 2\ka $ yields
\begin{align}\label{rel00}
    a_j&\leq\left(1 +\frac{b}{\ka} \right)^{j-1} \rho''a_0+c \sum_{i=0}^{j-1} \left(1 +\frac{b}{\ka} \right)^{j-i-1} e_i + 2 \left(1 +\frac{b}{\ka} \right)^{j-1}r \nonumber\\
    &\leq 2 \rho''a_0+2c \sum_{i=0}^{j-1}  e_i +4r,\qquad j<2\ta
\end{align}
where the inequality holds due to the fact that $\left(1 +b/\ka \right)^{p} \leq \exp(pb/\ka) \leq \exp(2b)< 2$ for $p<2\ta$. Next, by \eqref{rel00} we bound $A_{\ka+\ell}$ for $\ta+\ell<2\ta$
\begin{equation}\label{1bound}
\begin{aligned}
    A_{\ta+\ell} = \frac{1}{\ta} \sum_{j=\ell}^{\ta+\ell-1} a_j &\leq \frac{1}{\ta} \sum_{j=\ell}^{\ta+\ell-1} \left(2\rho'' a_{0} +   2c \sum_{i=0}^{j-1} e_{i}+4r \right) \\
    &\leq  \left(2\rho'' a_{0} +   2c\sum_{i=0}^{\ta+\ell-1} e_{i}  +4 r\right) \left(\frac{1}{\ta}\sum_{j=\ell}^{\ta+\ell-1} 1 \right)\\
     &=  2\rho'' a_{0} +   2c \sum_{i=0}^{\ta+\ell-1}e_{i} +4r. 
\end{aligned}
\end{equation}
Finally, adding \eqref{rel00} with $j = \ka + \ell$ and \eqref{1bound} gives 
\begin{align}\label{ft}
   A_{\ka+\ell} + a_{\ka+\ell}  &\leq  4\rho'' a_{0} + 4c  \sum_{i=0}^{\ta+\ell-1}  e_{i} +4r. 
\end{align}
Substituting for $A_{\ka+\ell} + a_{\ka+\ell}$ in \eqref{lastbound} using the above inequality completes the proof for any $t\geq 2\tau$. For $t<2\tau$, we have from (\ref{rel00}) with $j=t$ and the fact that $\left(1 - \frac{3\rho}{4\tau}\right)^{-t} \leq 2$ for any $t<2\tau$, 
\begin{align*}
  a_{t}  &\leq  2\rho'' a_{0} + 2c  \sum_{i=0}^{t-1}  e_{i} +4r\\
  &\leq 4\rho''\left(1 - \frac{3\rho}{4\tau}\right)^{t} a_{0} + 4c  \sum_{i=0}^{t-1} \left(1 - \frac{3\rho}{4\tau}\right)^{t-i} e_{i} + 8\left(1 - \frac{3\rho}{4\tau}\right)^{t}r,
\end{align*}
implying that \eqref{lem5_main} also holds for any $t<2\tau$.
\end{proof}


\bibliographystyle{IEEEtran}
\bibliography{references}

\end{document}

