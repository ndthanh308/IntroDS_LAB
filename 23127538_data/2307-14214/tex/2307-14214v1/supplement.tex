\documentclass[11pt,a4paper]{article}
\usepackage[margin=1in]{geometry}
\usepackage[sort&compress,square,numbers]{natbib}
\usepackage{lipsum,times,graphicx,hyperref,cleveref}
\usepackage[labelfont=bf]{caption}

\title{\vspace{-2cm} {\large Supplementary Materials for}\\[-2mm] 
\textbf{Nuclear quantum effect on the elasticity of ice VII under pressure: path-integral molecular dynamics study}
}
 
\author{Jun Tsuchiya$^\star$, Motoyuki Shiga, Shinji Tsuneyuki and Elizabeth C. Thompson\\ 
\small $^\star$Corresponding author. Email: junt@ehime-u.ac.jp
}
\date{}

%\renewcommand{\figurename}{Fig.}
\renewcommand{\baselinestretch}{1.33} 
\renewcommand{\thepage}{S\arabic{page}}
\renewcommand{\thetable}{S\arabic{table}}
\renewcommand{\thefigure}{S\arabic{figure}}
\renewcommand{\theequation}{S\arabic{equation}}
%==============================Content=========================
\begin{document}\maketitle

\section*{Elastic constants of static disordered phase of ice VII at 0 K}
To investigate the effect of static hydrogen disorder in the ice VII phase on the elasticity, elastic constants were calculated for various hydrogen configurations in the ice VII supercell. In 2$\times$2$\times$2 supercell (16 H$_2$O) of ice VII. That supercell is made of two interpenetrating Ic lattices, each of which consists of eight H$_2$O molecules, allowing 80 hydrogen configurations to satisfy the Ice rule. Therefore, there are 90$\times$90$=$8100 possible hydrogen configurations. However, some of them have the same structure, which can be eliminated to 52 irreducible configurations (one of which is the same hydrogen configuration as in the ice VIII phase).
% Figure environment removed
We calculated the elastic constants of ice VII with these 52 hydrogen configurations at static 0 K conditions. Figure S1 shows the pressure dependence of the elastic constants of ice phase VII with these 52 different hydrogen configurations (full blue lines). These results are almost identical and overlap each other. This is because disordered states with asymmetric hydrogen bonds do not themselves cause an increase in the elastic constants. 

%\lipsum[1] \Cref{fig:sup}

%\begin{equation}
%mc^2 = \frac{hc}{\lambda}
%\label{eq:sup1}
%\end{equation}

%\begin{equation}
%mg= \frac{GMm}{r^2}
%\label{eq:sup2}
%\end{equation}

\section*{ab initio molecular dynamics (AIMD) calculation of ice VII under pressure}

\begin{table*}[b]
\caption{\label{tab:table3}
The high pressure elastic constants and moduli (Hill average) of ice VII at 300 K calculated by AIMD method.}
%\begin{ruledtabular}
\begin{tabular}{cccccccc}
P (GPa) &V (cm$^3$/mol)& C$_{11}$  &  C$_{12}$  & C$_{44}$ & B  & G  \\
\hline
  3.7	& 11.152 & 46.0 &  15.1 &  11.1 &  25.4	&  12.6 \\
 10.3 &  9.562 & 91.1 &  52.8 &  46.1 &  65.5 &  32.4 \\
 23.6 &	8.130 & 164.1 &  86.2 &  99.4 & 112.2 &  68.3 \\
 32.2 &	7.600 & 190.1 & 127.9 & 123.3 & 148.6 &  71.4 \\
 43.4 &	7.093 & 239.7 & 155.6 & 149.1 & 183.6 &  90.1 \\
 49.5 &	6.849 & 263.0 & 169.7 & 165.8 & 200.8 & 100.1 \\
 56.9 &  6.610 & 290.6 & 188.2 & 191.0 & 222.3 & 113.2 \\
 65.2 &	6.380 & 323.5 & 206.4 & 214.5 & 245.4 & 128.0 \\
 75.4 &  6.150 & 391.9 & 266.6 & 254.0 & 308.3 & 145.9 \\
 88.6 &  5.927 & 473.8 & 342.3 & 297.0 & 386.1 & 164.0 \\
105.2 &  5.710 & 566.5 & 427.3 & 340.9 & 473.7 & 182.8 \\
114.0 &  5.603 & 627.3 & 489.7 & 366.2 & 535.5 & 190.7 \\
124.2 &  5.498 & 670.4 & 520.9 & 376.6 & 570.7 & 199.9 \\
130.6 &  5.436 & 670.8 & 517.7 & 389.6 & 568.7 & 206.1 \\
132.7 &  5.415 & 674.6 & 538.6 & 397.6 & 583.9 & 200.5 \\
134.9 &  5.395 & 659.3 & 491.8 & 397.4 & 547.6 & 215.5 \\
137.1 &  5.374 & 687.8 & 542.0 & 401.2 & 590.6 & 206.5 \\
139.2 &  5.354 & 700.3 & 561.4 & 402.0 & 607.7 & 203.4 \\
146.4 &  5.292 & 726.6 & 585.8 & 415.3 & 632.7 & 208.8 \\
159.0 &  5.191 & 778.4 & 614.0 & 438.1 & 668.8 & 228.0 \\
\hline
\end{tabular}
%\end{ruledtabular}
\end{table*}


%\Cref{eq:sup1,eq:sup2} \lipsum[1-3] \cite{Bangham1964660a}.

%\centering 
%\rule{3cm}{3cm} 
%\caption{Supplementary figure.\label{fig:sup}}

%\bibliography{mybib}
%\bibliographystyle{apalike}

\end{document}