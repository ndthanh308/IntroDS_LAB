\section{Conclusions} \label{sec:conclusions}

In this paper, we expand on the ideas originally proposed in \citet{salo2022} providing multiple methodological enhancements. These enhancements include a novel and more efficient formulation, valid bounds to tighten relaxations, and a heuristic which can be used to find feasible solutions and, consequently, to warm start the MIP solver. % We also introduce \texttt{DecisionProgramming.jl}, a Julia package that allows representing decision problems as MIP models.
Furthermore, we conduct a novel case study based on the study originally proposed by \citet{hynninen2019value}. Our objective is to demonstrate that the proposed models can be used in settings which would normally require resorting to more ad-hoc computational tools, lending themselves to be a general and accessible tool for practitioners. We believe that this will allow for a much wider range of practitioners and researchers to have access to mathematical optimisation-based tools for supporting decision-making. Furthermore, this will create novel inroads for the use of mathematical optimisation in the area of decision analysis at large, potentially unveiling new and promising directions for future developments.

In terms of alternative further developments, we see several directions that deserve further investigation. First, decision programming as a modelling framework is still in its infancy, and, consequently, many obstacles are still to be overcome for its widespread adoption. One of these obstacles is computational requirements. Decision programming models grow large as the number of nodes and/or states increase, and thus it would greatly benefit from alternative ideas that can tackle such large-scale problems. These can be, for example, related to alternative formulations that convert the influence diagram into an intermediate structure and employ ideas from Bayesian inference to yield a more compact MIP model (see \citet{parmentier2020integer}). Another direction worth exploring is the employment of decomposition methods, in particular, those which allow for a delayed generation of structural elements of the model, in our case the paths $s \in S$ (see Section \ref{sec:decision_programming}). Another interesting avenue would be to pursue methods that can reap benefits from employing parallelization, given the increasing availability of high-performance computing clusters.
