\documentclass[a4paper, 12pt, review]{elsarticle}
\usepackage[utf8]{inputenc}
\usepackage[margin=1in]{geometry}
\usepackage{amsmath}
\usepackage{amsfonts}
\usepackage{amssymb}
\usepackage{caption}
\usepackage{subcaption}
\usepackage{algorithm2e}
\usepackage{tikz}
    \usetikzlibrary[shapes.geometric] 
\usepackage[inline]{enumitem}
\usepackage[implicit=false]{hyperref}
\usepackage[theme=grayscale]{jlcode}

% Removes the "Preprint submitted to Elsevier" text
\makeatletter
\def\ps@pprintTitle{%
 % \let\@oddhead\ \textcolor{red}{This preprint is work-in-progress and subject to change.}\vspace{0.5cm}
 \let\@oddhead\@empty
 \let\@evenhead\@empty
 \def\@oddfoot{}%
 \let\@evenfoot\@oddfoot}
\makeatother

% Our own commands
\newcommand{\braces}[1]{\left\{#1\right\}}


% Definitions
\definecolor{blue}{rgb}{0.23,0.58,0.89}

%\title{Efficient mixed-integer programming formulations for modeling decision problems}

\title{Solving influence diagrams via efficient mixed-integer programming formulations and heuristics}

\author{Helmi Hankimaa}
\author{Olli Herrala}
\author{Fabricio Oliveira\corref{cor1}}
\author{Jaan Tollander de Balsch}

\cortext[cor1]{Corresponding author: fabricio.oliveira@aalto.fi}

% \author{Helmi Hankimaa, Olli Herrala, Fabricio Oliveira, Jaan Tollander de Balsch}
\address{Department of Mathematics and Systems Analysis,
Aalto University,
Espoo, Finland}

\begin{document}

\begin{abstract}
     In this paper, we propose novel mixed-integer linear programming (MIP) formulations to model decision problems posed as influence diagrams. We also present a novel heuristic that can be employed to warm start the MIP solver, as well as provide heuristic solutions to more computationally challenging problems. 
    %All of those improvements are implemented in  \texttt{DecisionProgramming.jl}, a new Julia package for modelling decision problems as MIP equivalents using the proposed formulations. 
    We provide computational results showcasing the superior performance of these improved formulations as well as the performance of the proposed heuristic. Lastly, we describe a novel case study showcasing decision programming as an alternative framework for modelling multi-stage stochastic dynamic programming problems. 
\end{abstract}


\begin{keyword}
    decision problems under uncertainty \sep influence diagrams \sep decision analysis \sep mixed-integer programming
\end{keyword}


\maketitle



\section{Introduction}
\label{sec:introduction}

The recent surge of Large Language Models (LLMs), such as GPT-3.5/4~\cite{bubeck_sparks_2023}, PaLM~\cite{chowdhery_palm_2022}, FLAN-T5~\cite{chung_scaling_2022}, and Alpaca~\cite{taori_stanford_2023}, has shown a promising trend of large pre-trained models to do a variety of tasks in a zero-shot setting (\ie without any new training data). Example tasks include question answering~\cite{omar2023chatgpt,robinson2023leveraging}, logic reasoning~\cite{wei_chain--thought_2023,zhou_least--most_2023}, machine translation~\cite{brants2007large,gulcehre2017integrating} \etc\ 
A number of experiments have revealed that, built on hundreds of billions of parameters, these LLMs have started to show the capability to understand the human common sense beneath the natural language and do proper reasoning and inference accordingly~\cite{bubeck_sparks_2023,nori_capabilities_2023}.

Among different applications, one particular question yet to be answered is how well LLMs can understand human mental health states through natural language.
Mental health problems represent a significant burden for individuals and societies worldwide.
A recent report suggested that more than 20\% of adults in the U.S. would experience at least one mental disorder in their lifetime~\cite{mental2022state} and 5.6\% of adults experienced a serious psychotic disorder that significantly impairs functioning~\cite{mental2023stats}. The global economy loses around \$1 trillion annually in productivity due to depression and anxiety alone~\cite{mentalcost2023}.

In the past decade, there has been a plethora of research in natural language processing (NLP) and computational social science on detecting mental health issues via online text data such as social media~(\eg \cite{guntuku_detecting_2017,eichstaedt2018facebook,coppersmith_clpsych_2015,de_choudhury_social_2013,de_choudhury_mental_2014}). However, most of these studies have focused on building domain-specific machine learning (ML) models (\ie one model for one particular task, such as stress detection~\cite{nijhawan2022stress,guntuku2019understanding}, depression prediction~\cite{eichstaedt2018facebook,tadesse2019detection,xu_leveraging_2019}, or suicide risk assessment~\cite{de_choudhury_discovering_2016,coppersmith2018natural}). Even for traditional pre-trained language models such as BERT, it needs to be finetuned for specific downstream tasks~\cite{devlin_bert_2019,liu_roberta_2019}.
Since natural language is a major component of mental health assessment and treatment~\cite{sharma2018mental,gkotsis2016language}, LLMs might be a potentially powerful tool to understand end-users' mental states based on the language users' wrote. These instruction-finetuned and general-purpose models can understand a variety of inputs and obviate the need to train multiple models for different tasks. Thus, we can envision using one LLM for a variety of mental-health-related tasks, such as multiple question-answering, reasoning, and inference.
Such a vision opens up a wide range of opportunities for UbiComp, Human-Computer Interaction (HCI), and mental health communities, such as online public health monitoring systems~\cite{patel2018psyheal,graham2019artificial}, intelligent assistants for mental counselors and supporters~\cite{sharma_towards_2021,sharma_humanai_2023}, mental-health-aware personal chatbots~\cite{abd2021perceptions,denecke2020mental}, to just name a few.
However, there is a lack of investigation into understanding, evaluating, and improving the capability of LLMs for mental health prediction tasks.

There are few very recent studies on the evaluation of LLMs (\eg ChatGPT) on mental-health-related tasks, most of which are in zero-shot settings with simple prompt engineering~\cite{yang_evaluations_2023,amin_will_2023,lamichhane_evaluation_2023}. Researchers have shown preliminary results that LLMs have some initial capability of predicting mental health disorders with natural language with some promising but still limited performance compared to state-of-the-art domain-specific NLP models~\cite{yang_evaluations_2023,lamichhane_evaluation_2023}.
This remaining gap is expected since existing general-purpose LLMs are not specifically trained on mental health tasks.
However, to achieve our vision of leveraging LLMs for mental health support and assistance, we need to answer the research question: \textbf{How to empower LLMs with more mental health domain knowledge and become an expert}?

We conducted a series of experiments with multiple LLMs, including Alpaca~\cite{noauthor_stanford_2023}, Alpaca-LoRA~\cite{hu_lora_2021}, and GPT-3.5~\cite{noauthor_introducing_2022}.
Considering the data availability, we focused on online social media data with high-quality human-generated mental health labels.
Our experiments contained three stages: (1) zero-shot prompting, where we experimented with various prompts related to mental health, (2) few-shot prompting, where we inserted examples into prompt inputs, and (3) instruction-finetuning, where we finetuned LLMs on multiple mental-health datasets with various tasks.

Our results indicate that zero-shot obtained promising but limited performance on multiple mental health prediction tasks across all models. GPT-3.5 had relatively better results since it has a larger scale. But their performance is still far from task-specific models. 
Meanwhile, providing a few shots in the prompt can improve the model performance to some extent ($\overline{\Delta}$ = 4.7\%), but the advantage is limited.
Finally and most importantly, we found that instruction-finetuning can significantly improve the model performance across multiple mental-health-related tasks at the same time. Our finetuned Alpaca, namely \textbf{Mental-Alpaca}, significantly outperforms the original GPT-3.5 ($\times$25 times of model size) by an average of 16.7\% on balance accuracy. 
Meanwhile, Mental-Alpaca can further perform on par with the task-specific state-of-the-art Mental-RoBERTa~\cite{ji_mentalbert_2021}. It is noteworthy that Mental-RoBERTa needs to be trained on each task individually, 
while our Mental-Alpaca can solve different tasks off the shelf. 
% We open-source our training code and model at [github link].
Our experiments present the first comprehensive evaluation of various techniques to enhance LLMs' capability in the mental health domain.

The contribution of our paper can be summarized as follows:
\begin{s_enumerate}
\item We present the first comprehensive evaluation of prompt engineering, few-shot, and finetuning techniques on multiple LLMs in the mental health domain.
\item With online social media data, our results reveal that finetuning on a variety of datasets can significantly improve LLM's capability on multiple mental-health-specific tasks simultaneously.
% We release our model \textbf{Mental-Alpaca} as the first open-source LLM targeted at mental health prediction tasks.
\item We provide a few technical guidelines for future researchers and developers on turning LLMs into experts in specific domains.
\end{s_enumerate}

\section{Decision Programming} \label{sec:decision_programming}

Decision programming relies on influence diagrams, which are graphical representations of decision problems. In influence diagrams, nodes represent chance events, decisions and consequences. Specifically, let $G(N,A)$ be an acyclic graph formed by nodes in $N = C \cup D \cup V$, where $C$ is a subset of chance nodes, $D$ a subset of decision nodes, and $V$ a subset of value nodes. Value nodes represent consequences incurred from decisions made at nodes $D$ and chance events observed at nodes $C$. Each decision and chance node $j \in C \cup D$ can assume a state $s_j$ from a discrete and finite set of states $S_j$. For a decision node $j \in D$, $S_j$ represents the decision alternatives. For a chance node $j \in C$, $S_j$ is the set of possible outcomes. 

In the diagram, arcs represent interdependency among decisions and chance events. Set $A = \{(i,j) \mid i,j \in N\}$ contains the arcs $(i,j)$, which represent the influence between nodes $i$ and $j$. This influence is propagated in the diagram in the form of \emph{information}. That is, an arc $(i,j)$ that points to a decision node $j \in D$ indicates that the decision at $j \in D$ is made \emph{knowing} the realisation (i.e., uncertainty outcome or decision made) of state $s_i \in S_i$, with $i \in C \cup D$. On the other hand, an arc that points to a chance node $j \in C$ indicates that the realisation $s_j \in S_j$ is dependent (or conditional) on realisation $s_i \in S_i$ of node $i \in C \cup D$.

The \emph{information set} $I(j) = \{i \in N \mid (i,j) \in A\}$ comprises all immediate predecessors (or 
%
parents) 
%
of a given node $j \in N$. Despite being a less common terminology, we opt for the term ``information set'' to highlight the role of information in the modelling of the decision process. The decisions $s_j \in S_j$ made in each decision node $j \in D$ depend on their \emph{information state} $s_{I(j)} \in S_{I(j)}$, where $S_{I(j)} = \prod_{i \in I(j)} S_i$ is the set of all possible information states for node $j$. Analogously, the possible realisations $s_j \in S_j$ for each chance node $j \in C$ and their associated probabilities also depend on their information state $s_{I(j)} \in S_{I(j)}$.

Let us define $X_j \in S_j$ as the realised state at a chance node $j \in C$. For a decision node $j \in D$, let $Z_j: S_{I(j)} \to S_j$ be a mapping between each information state $s_{I(j)} \in S_{I(j)}$ and decision $s_j \in S_j$. That is, $Z_j(s_{I(j)})$ defines a local decision strategy, which represents the choice of some $s_j \in S_j $ in $j \in D$, given the information $s_{I(j)}$. Such a mapping can be represented by an indicator function $\mathbb{I}: S_{I(j)} \times S_j \to \{0,1\}$ defined so that
%
\begin{align*}
    \mathbb{I}(s_{I(j)}, s_j) = \begin{cases} 1, &\text{ if } Z_j \text{ maps } s_{I(j)} \text{ to } s_j \text{, i.e., } Z_j(s_{I(j)}) = s_j; \\ 0, &\text{ otherwise.} \end{cases}
\end{align*}
%
A (global) \emph{decision strategy} is the collection of local decision strategies in all decision nodes: $Z = (Z_j)_{j \in D}$, selected from the set of all possible strategies $\mathbb{Z}$.

A \emph{path} is a sequence of states $s = (s_i)_{i=1,\dots,n}$, with $n = |C| + |D|$ and 
\begin{equation}
    S = \{(s_i)_{i=1,\dots,n} \mid s_i \in S_i, i =1, \dots, n\}\label{eq:paths}    
\end{equation} 
is the set of all possible paths. We assume that the nodes $C \cup D$ are numbered from 1 to $n$ such that for each arc $(i,j) \in A$, $i<j$. Moreover, we say that a strategy $Z$ is compatible with a path $s \in S$ if $Z_j(s_{I(j)}) = s_j$ for all $j \in D$. We denote as $S(Z) \subseteq S$ the subset of all paths that are compatible with a strategy $Z$. 

Using the notion of information states, the conditional probability of observing a given state $s_j$ for $j \in C$ is $\mathbb{P}(X_j = s_j \mid X_{I(j)} = s_{I(j)})$. The probability associated with a path $s \in S$ being observed given a strategy $Z$ can then be expressed as
%
\begin{align}
    \mathbb{P}(s \mid Z) = \left(\prod_{j \in C}\mathbb{P}(X_j = s_j \mid X_{I(j)} = s_{I(j)})\right)\left(\prod_{j \in D} \mathbb{I}(s_{I(j)},s_j)\right) \label{eq:path_probability}.
\end{align}
%
Notice that the term $\prod_{j \in D} \mathbb{I}(s_{I(j)},s_j)$ in equation \eqref{eq:path_probability} takes value one if the strategy $Z$ is compatible with the path $s \in S$, being zero otherwise. Furthermore, notice that one can pre-calculate the probability 
%
\begin{equation}
    p(s) = \left(\prod_{j \in C}\mathbb{P}(X_j = s_j \mid X_{I(j)} = s_{I(j)})\right) \label{eq:p-def}
\end{equation}
%
of a path $s \in S$ being observed, in case a compatible strategy is chosen.

At the value node $v \in V$, a real-valued utility function $U_v : S_{I(v)} \to \mathbb{R}$ maps the information state $s_{I(v)}$ to a utility value $U_v(s_{I(v)})$. We usually assume the utility value of a path $s$ to be the sum of individual value nodes' utilities: $U(s) = \sum_{v \in V} U_v(s_{I(v)})$. The default objective is to choose a strategy $Z \in \mathbb{Z}$ maximising the expected utility, which can be expressed as
%
\begin{equation} 
\underset{Z \in \mathbb{Z}}{\text{max }} \sum_{s \in S}  \mathbb{P}(s \mid Z) U(s). \label{eq:orig-obj}
\end{equation}
%
Notice that other objective functions can also be modelled. For example, \citet{salo2022} discuss the use of the conditional value-at-risk.

To formulate this into a mathematical optimisation problem, we start by representing the local strategies $Z_j$ using binary variables $z(s_j \mid s_{I(j)})$ that take value one if $\mathbb{I}(s_{I(j)},s_j) = 1$, and 0 otherwise. We then observe that using \eqref{eq:path_probability} and \eqref{eq:p-def}, the objective function \eqref{eq:orig-obj} becomes 
%
\begin{equation*} 
\underset{z}{\text{max }} \sum_{s \in S}  p(s) U(s) \prod_{j \in D} z(s_j \mid s_{I(j)}). \label{eq:nonlin-obj}
\end{equation*}
%
This function is nonlinear and is used only for illustrating the nature of the formulations. The usefulness of this construction becomes more obvious in Section \ref{sec:formulations}. \citet{salo2022} instead replace the conditional path probability $\mathbb{P}(s \mid Z)$ in \eqref{eq:orig-obj} with a continuous decision variable $\pi(s)$, enforcing the correct behaviour of this variable using affine constraints.

With these building blocks, the problem can be formulated as a mixed-integer linear programming (MILP) model, which allows for employing off-the-shelf mathematical programming solvers. The MILP problem presented in \citet{salo2022} can be stated as \eqref{eq:dp_obj}-\eqref{eq:dp_z_bin}.
%
\begin{align}
    \underset{Z \in \mathbb{Z}}{\text{max }}  &\sum_{s \in S} \pi(s) U(s)\label{eq:dp_obj}\\
    \text{subject to }  &\sum_{s_j \in S_j} z(s_j \mid s_{I(j)}) = 1, &&\forall j \in D, s_{I(j)} \in S_{I(j)}, \label{eq:dp_z_sum}\\
    &0 \le \pi(s) \le p(s), &&\forall s \in S, \label{eq:dp_pi_lim}\\
    &\pi(s) \le z(s_j \mid s_{I(j)}), &&\forall j \in D, s \in S, \label{eq:dp_pi_upper}\\
    &\pi(s) \ge p(s) + \sum_{j \in D} z(s_j \mid s_{I(j)}) \, - | D |, &&\forall s \in S, \label{eq:dp_pi_lower} \\
    &z(s_j \mid s_{I(j)}) \in \{0,1\}, &&\forall j \in D, s_j \in S_j, s_{I(j)} \in S_{I(j)}. \label{eq:dp_z_bin}
\end{align}

Variables $\pi(s)$ are nonnegative continuous variables representing the conditional path probability in equation \eqref{eq:path_probability}. They take the value of the path probability $p(s)$ in case the selected strategy $Z$ is compatible with the path $s \in S$ and zero otherwise. Notice that this compatibility is equivalent to observing $z(s_j \mid s_{I(j)}) = 1$ for all $s_j \in S$ such that $j \in D$. 

The objective function \eqref{eq:dp_obj} defines the expected utility value, which is calculated considering only the paths that are compatible with the strategy. Constraint \eqref{eq:dp_z_sum} enforces the one-to-one nature of the mapping $\mathbb{I}(s_{I(j)}, s_j)$, represented by the $z$-variables. The correct behaviour of variables $\pi(s)$ is guaranteed by constraints \eqref{eq:dp_pi_lim}-\eqref{eq:dp_pi_lower}, which enforce that $\pi(s) = p(s)$ if $z(s_j \mid s_{I(j)}) = 1$ for all $s_j \in S$ such that $j \in D$. The term $| D |$ in \eqref{eq:dp_pi_lower} represents the cardinality of the set $D$, that is, the number of decision nodes in the diagram. Notice that the domain of $\pi(s)$ is defined in \eqref{eq:dp_pi_lim}.


\section{Improved formulations}
\label{sec:formulations}

One key challenge associated with formulation \eqref{eq:dp_obj}--\eqref{eq:dp_z_bin}, and, in fact, any MILP formulation, is that computational performance is strongly tied to the tightness of the formulation. In this context, the tightness of a MILP formulation is related to how close the linear relaxation solution is to the initial primal bound, e.g., the first integer feasible solution value obtained by the solver during the solution process or one obtained using primal heuristics. 

Next, we present reformulations developed to enhance the numerical performance of the decision programming formulation \eqref{eq:dp_obj}--\eqref{eq:dp_z_bin}. For that, let us first define the subset of paths
%
\begin{equation*}
    S_{s_j \mid s_{I(j)}} = \braces{s \in S \mid (s_{I(j)}, s_j) \subseteq s}.    
\end{equation*}
%
Notice that we use the notation $(s_{I(j)}, s_j)$ to represent a portion of a path $s$, formed by the combination of the information state $s_{I(j)}$ (which may itself be a collection of states, if $|I(j)| > 1$) and the state $s_j$. We also utilise the set operator $\subseteq$ to indicate that the states $(s_{I(j)}, s_j)$ are part of the path $s \in S$. Notice that the states $(s_{I(j)}, s_j)$ do not need to be consecutive in the path $s$, although the ordering between $s_{I(j)}$ and $s_j$ is naturally preserved in $s$. 

Considering $j \in D$, the subset $S_{s_j \mid s_{I(j)}}$ allows us to define the notion of \emph{locally compatible paths}, that is, the collection of paths $s$ compatible with local strategies $Z_j$ for which $\mathbb{I}(s_{I(j)}, s_j) = 1$. The definition of the subset $S_{s_j \mid s_{I(j)}}$ allows us to derive the following valid inequality for \eqref{eq:dp_obj}--\eqref{eq:dp_z_bin}. 
%
\begin{equation} \label{eq:locally_compatible_paths}
    \sum_{s \in S_{s_j \mid s_{I(j)}}} \pi(s) \leq z(s_j \mid s_{I(j)}), \quad \forall j \in D, s_j \in S_j, s_{I(j)} \in S_{I(j)}.
\end{equation}

Constraint \eqref{eq:locally_compatible_paths} states that only paths that are compatible with the selected strategy might be allowed to have a probability different than zero. Moreover, since it is enforced on all decision nodes, it means that this constraint guarantees that only the paths that are compatible with the strategy $Z$ are active. Recall that we denote this set of compatible paths as $S(Z) \subseteq S$.

As pointed out in \citet{salo2022}, for expected utility maximization, constraint \eqref{eq:dp_pi_lower}, which prevents variables $\pi(s)$ from wrongly taking value zero, is only required when some of the utility values $U(s)$, $s \in S$, are negative. Notice that this is otherwise prevented by the maximization of the objective function \eqref{eq:dp_obj}, naturally steering these variables to their upper bound values. Another way to guarantee that the variables $\pi(s)$ take their correct value, i.e., $\pi(s) = p(s)$, if $s \in S(Z)$, is to impose the constraint
%
\begin{equation} \label{eq:total_prob}
    \sum_{s \in S} \pi(s) = 1.
\end{equation}

As it will be discussed in Section \ref{sec:computational_experiments}, replacing \eqref{eq:dp_pi_upper} and \eqref{eq:dp_pi_lower} with \eqref{eq:locally_compatible_paths} and \eqref{eq:total_prob} provides considerable gains in terms of linear relaxation strengthening. Furthermore, we observe that the computational performance can be even further improved by employing a simple variable substitution. Recall that in the original formulation \eqref{eq:dp_obj}-\eqref{eq:dp_z_bin}, variables $\pi(s)$ represent the conditional path probability $\mathbb{P}(s \mid Z) = p(s) \prod_{j \in D} z(s_j \mid s_{I(j)})$. If we let $x(s) \in [0,1]$, $s \in S$ represent the product $\prod_{j \in D} z(s_j \mid s_{I(j)})$, then we can reformulate the problem by substituting $\pi(s) = p(s)x(s)$ for all $s \in S$.

Although $x(s)$, $s \in S$, is continuous, it behaves as a binary variable which takes value one whenever the path is compatible with the strategy and zero, otherwise. This is analogous to the behaviour of variable $\pi(s) \in [0,p(s)]$ in \eqref{eq:dp_obj}--\eqref{eq:dp_z_bin}. We highlight that, from a theoretical standpoint, there is no obvious reason for performing such a substitution. On the other hand, we will show that it yields significant practical benefits in terms of computational performance.


Using these $x$-variables, we can reformulate \eqref{eq:locally_compatible_paths} as
%
\begin{equation} \label{eq:dp2_locally_compatible_paths_1}
    \sum_{s \in S_{s_j \mid s_{I(j)}}} x(s) \leq |S_{s_j \mid s_{I(j)}}|z(s_j \mid s_{I(j)}), \quad \forall j \in D, s_j \in S_j, s_{I(j)} \in S_{I(j)}, 
\end{equation}
%
a consequence of $x(s) \in [0,1]$ and the fact that $z(s_j \mid s_{I(j)})$ must be equal to 1 for $x(s)$ to be positive for $s \in S_{s_j \mid s_{I(j)}}$.

% Constraint \eqref{eq:dp2_locally_compatible_paths_1} can be strengthened further. It turns out that we can infer the number of paths from the set of active paths $S(Z)$ that are locally compatible with a given decision node state $s_j$, with $j \in D$, given its information state $s_{I(j)}$. For that, we observe that because we enforce binary values for the $z$-variables representing decision strategies, at every decision node $k \in D \setminus{j}$ only one alternative $s_k \in S_k$ will be selected, in accordance with constraint \eqref{eq:dp_z_sum}. This means that the number of locally compatible paths that will also be active, i.e., $|S_{s_j \mid s_{I(j)}} \cap S(Z)|$ can be defined as
Constraint \eqref{eq:dp2_locally_compatible_paths_1} can be strengthened further. We note that a path must be in the set of compatible paths $S(Z)$ in order for $x(s)$ to be positive with strategy $Z$. Using this information, we can infer a tighter upper bound for the number of paths that can be active ($x(s) >0$) from the set of locally compatible paths. We observe that in a set of compatible paths $S(Z)$, each information state $s_{I(j)}$ maps to exactly one decision alternative $s_j$ for each decision node $j \in D$, in accordance with constraint \eqref{eq:dp_z_sum}. However, the set of locally compatible paths for a given pair of information state and decision node state $(s_{I(j)}, s_j)$ of decision node $j \in D$, includes paths for all combinations $(s_{I(k)}, s_k)$ of information states and decisions for the other decision nodes $k \in D \setminus \braces{j}$. Hence, only a fraction of the locally compatible paths can be active. The fraction is linked to the number of states $| S_k |$ of the other decision nodes $k \in D \setminus \braces{j}$. The number of locally compatible paths that will also be active, i.e., $|S_{s_j \mid s_{I(j)}} \cap S(Z)|$ can be defined as
%
\begin{equation} \label{eq:dp2_locally_active_compatible_paths}
    |S_{s_j \mid s_{I(j)}} \cap S(Z)| = \frac{|S_{s_j | s_I(j)}|} {\Pi_{k \in D \setminus (\braces{j} \cup I(j))} |S_k|}.
\end{equation}
%
Notice that the calculation of the number of active paths must take into account the fact that some decision nodes may be part of the information state $I(j)$ of node $j \in D$, and, as such, will have their states observed (or fixed) in the set $S_{s_j | s_I(j)}$. Therefore, these decision nodes must be excluded from the product in the denominator in equation \eqref{eq:dp2_locally_active_compatible_paths}. Using \eqref{eq:dp2_locally_active_compatible_paths}, we can reformulate \eqref{eq:dp2_locally_compatible_paths_1} into the strengthened form
%
\begin{equation} \label{eq:dp2_locally_compatible_paths_2}
    \sum_{s \in S_{s_j \mid s_{I(j)}}} x(s) \leq  \frac{|S_{s_j | s_I(j)}|} {\Pi_{k \in D \setminus (\braces{j} \cup I(j))} |S_k|}z(s_j \mid s_{I(j)}), \quad \forall j \in D, s_j \in S_j, s_{I(j)} \in S_{I(j)}. 
\end{equation}

One last aspect that can be taken into account is that, depending on the problem structure, some sequence of states $s = (s_i)_{i=1,\dots,n}$ forming a path may never be observed and can be preemptively filtered out from the set of paths $S$. This is the case, for example, in problems where earlier decisions or uncertain events dictate whether alternatives or uncertainties are observed. For instance, an initial decision regarding whether or not to build an industrial plant naturally restricts subsequent decisions regarding capacity expansion. Analogously, it may be that an uncertain production rate is only observed if one decides to build the production facility in the first place. To prevent the assembling of these unnecessary paths, we consider a set of \emph{forbidden} paths, which, once removed, lead to a set $S^* \subseteq S$ of effective paths. Notice that these forbidden paths have probability zero by the structure of the problem, and therefore their removal does not affect the expected utility nor the constraints of the model. Furthermore, their removal allows for significant savings in terms of the scale of the model.

One issue emerges in settings where $S^* \subset S$ regarding the term \eqref{eq:dp2_locally_active_compatible_paths}. Notice that the bound is based on the premise that we can infer the total number of paths by considering the Cartesian product of the state sets $S_j$, $j \in N$. However, as forbidden paths are removed, some of the $x$-variables corresponding to paths $s \in S_{s_j | s_I(j)}$ might be removed, making inequality \eqref{eq:dp2_locally_compatible_paths_2} loose. A simple safeguard for this is to consider  
%
\begin{equation} \label{eq:gamma_value}
    \Gamma(s_j|s_{I(j)}) = \min \braces{|S^*_{s_j \mid s_{I(j)}}|, \frac{|S_{s_j | s_I(j)}|} {\Pi_{k \in D \setminus (\braces{j} \cup I(j))} |S_k|}}
\end{equation}
%
and reformulate \eqref{eq:dp2_locally_compatible_paths_2} as
%
\begin{equation} \label{eq:dp2_locally_compatible_paths_3}
    \sum_{s \in S_{s_j \mid s_{I(j)}}} x(s) \leq \Gamma(s_j|s_{I(j)})z(s_j \mid s_{I(j)}), \quad \forall j \in D, s_j \in S_j, s_{I(j)} \in S_{I(j)}. 
\end{equation}

Combining the above, we can reformulate \eqref{eq:dp_obj}--\eqref{eq:dp_z_bin} as follows.
%
\begin{align}
    \underset{Z \in \mathbb{Z}}{\text{maximize }}  &\sum_{s \in S^*} U(s)p(s)x(s)\label{eq:dp2_obj}\\
    \text{subject to }  
    &\sum_{s_j \in S_j} z(s_j \mid s_{I(j)}) = 1, &&\forall j \in D, s_{I(j)} \in S_{I(j)} \label{eq:dp2_z_sum}\\
    & \sum_{s \in S_{s_j \mid s_{I(j)}}} x(s) \leq \Gamma(s_j|s_{I(j)}) z(s_j \mid s_{I(j)}), &&\forall j \in D, s_j \in S_j, s_{I(j)} \in S_{I(j)} \label{eq:dp2_x_upper} \\
    & \sum_{s \in S^*} p(s)x(s) = 1, && \label{eq:dp2_prob_sum} \\
    & 0 \leq x(s) \leq 1, &&\forall s \in S^* \label{eq:dp2_x_lim} \\ 
    &z(s_j \mid s_{I(j)}) \in \{0,1\}, &&\forall j \in D, s_j \in S_j, s_{I(j)} \in S_{I(j)} \label{eq:dp2_z_bin}.
\end{align}
%
where $\Gamma(s_j|s_{I(j)})$ is defined as in \eqref{eq:gamma_value}. Note that this formulation preserves the (mixed-integer) linear nature of \eqref{eq:dp_obj}--\eqref{eq:dp_z_bin}.

As discussed earlier, one of the main advantages of the decision programming formulation is the ability to incorporate objectives and constraints involving arbitrary utility functions and probability distributions within the model. \citet{salo2022} demonstrate this by proposing a model that considers conditional value-at-risk as one of the utility functions. The same can be achieved with our proposed formulation, by simply substituting $p(s)x(s)$ in place of variables $\pi(s)$. 

The path-based structure of \eqref{eq:dp_obj}--\eqref{eq:dp_z_bin} makes formulating chance and budget constraints straightforward. For modeling chance constraints, we can define $\tilde{S}$ as the set of ``undesirable" paths, the total probability of which must not exceed $\rho$. The corresponding chance constraint is then 
\begin{equation} \label{eq:chance_constraint}
    \sum_{s \in \tilde{S}} x(s)p(s) \le \rho. 
\end{equation}

Likewise, the set $\tilde{S}$ could further be defined as, e.g., the set of paths with a small utility $U(s) \le u_{threshold}$. With $\rho=0$, \eqref{eq:chance_constraint} can be seen as a budget constraint, stating that for all compatible paths $s \in S(Z)$ with $p(s)>0$, the utility $U(s)$ must be at least $u_{threshold}$. 


% \subsection{A tighter bound for the $x$-variables (not yet implemented in the package)}

% A possible modification to formulation \eqref{eq:dp2_obj}-\eqref{eq:dp2_z_bin} is to tighten the upper bound in \eqref{eq:dp2_x_lim}. The upper bound $\frac{\sum_{j \in D} z(s_j \mid s_{I(j)})}{|D|}$ takes value 1 if $z(s_j \mid s_{I(j)})=1$ for all $j \in D$, that is, the path $s$ is compatible with the decision strategy represented by the $z$-variable values. If the path is not compatible with the strategy, the upper bound is in the interval $[0,1)$.

\section{Primal heuristic: single policy update (SPU)} 
\label{sec:heuristics}

While our approach of formulating influence diagrams into mixed-integer linear models does allow us to use powerful off-the-shelf solvers, it is still hindered by the well-known fact that solving such problems is NP-hard \citep{schrijver2003combinatorial}. To make MIP solvers more efficient, \emph{primal heuristics} are used to obtain and improve integer solutions. Obtaining good starting integer solutions can have a significant impact on the performance of branch-and-bound solvers, as it helps in pruning poor-quality solutions early.

Decision programming is based on limited-memory influence diagrams (LIMIDs) and solution approaches presented in previous literature can be used to obtain solutions to these problems. A notable contribution of \citet{lauritzen2001representing} is the single policy update (SPU) heuristic for obtaining ``locally optimal'' strategies in the sense that the corresponding solutions cannot be improved by changing only one of the local strategies $Z_j(s_{I(j)})$.  

Our proposed heuristic is loosely based on the ideas in \citet{lauritzen2001representing}, as described in Algorithm \ref{alg:spu}. The first step of the heuristic is to obtain a random strategy $Z$ (note that this too is a heuristic, albeit a very simple one). Additionally, we initialise the $lastImprovement$ variable that will be used to stop the algorithm after finding a local optimum. The strategy $Z$ is then iteratively improved by examining each information state $s_{I(j)} \in S_{I(j)}$ for each decision node $j \in D$ in order, choosing the local strategy $Z'_j(s_{I(j)})$ maximising the expected utility. We obtain incrementally improving strategies by replacing the local strategy $Z_j(s_{I(j)})$ with $Z'_j(s_{I(j)})$ whenever the change results in an increase in expected utility. Finally, the pair $(j, s_{I(j)})$ is stored in the $lastImprovement$ variable if an improvement has been observed.

\begin{algorithm}
\caption{The single policy update heuristic}\label{alg:spu}
% \KwData{$n \geq 0$}
% \KwResult{$y = x^n$}
$Z \gets randomstrategy()$\;
$lastImprovement \gets (undef, undef)$\;
\While{true }{
    \For{$j \in D$, $s_{I(j)} \in S_{I(j)}$}{
        \eIf{$(j,s_{I(j)}) = lastImprovement$}{
            \KwRet{$Z$}\;
        }{
            $Z'_j(s_{I(j)}) \gets bestLocalStrategy(Z,j,s_{I(j)})$\;
            $Z' \gets modifyStrategy(Z,Z'_j(s_{I(j)}))$\;
            \If{$EU(Z') > EU(Z)$}{
                $Z \gets Z'$\;
                $lastImprovement \gets (j, s_{I(j)})$\;
            }
        }
    }
}
\end{algorithm}

This process of locally improving the strategy is performed repeatedly for all pairs $(j, s_{I(j)})$ until no improvement is made during a whole iteration through the set of such pairs, that is, $(j, s_{I(j)}) = lastImprovement$. The number of possible strategies $Z$ is finite, and the algorithm thus converges in a finite number of iterations. It is also easy to see that at termination, there is no possible local improvement and the strategy $Z$ is thus, in that sense, locally optimal. \citet{lauritzen2001representing} show that for \emph{soluble} LIMIDs, this heuristic results in a globally optimal solution. However, influence diagrams are not generally soluble. The performance of the heuristic is explored in Section \ref{sec:computational_experiments}.


% \section{Julia interface} 
\label{sec:interface}

The ideas and formulations presented in this paper have been implemented as a Julia \citep{Julia-2017} package. The main purpose of the package is to provide an interface through which a user can build an influence diagram and automatically convert it to a MILP model that can be solved using an off-the-shelf solver. This brings us to the main reasons for choosing Julia as the language of the package. Thanks to JuMP \citep{DunningHuchetteLubin2017} and MathOptInterface \citep{legat2022mathoptinterface}, implementing optimization problems in a general form is straightforward, and the models constructed in this way can be passed to many different solvers.

In order to create a user-friendly package that precludes a deeper understanding of MILP modeling by the user, we implemented a structure for influence diagrams and an interface for constructing them, demonstrated in Figure \ref{fig:code-nodes}. In that, we illustrate the implementation of the pig problem, originally proposed in \citet{lauritzen2001representing}. 

The influence diagram structure consists of nodes and their state spaces and information sets, as well as the corresponding probabilities and utility values. The interface includes functions for adding these elements into the influence diagram and type structures that guide the user to include the required information. For example, all node types require the user to define an information set for the node, even if it is empty in the case of a root node. In turn, the function for adding nodes stops the user from accidentally including a node in its own information set, ensures that the names of all nodes are unique, and warns about redundant nodes. 

% Figure environment removed

Furthermore, specialised structures and functions (Figure \ref{fig:code-diagram}) for defining and adding chance nodes' conditional probabilities and value nodes' utility values ensure that these matrices have correct dimensions, that the probabilities sum to one and that utility values are defined for all information states of value nodes. The interface also includes a function for generating the arcs in the diagram and giving the nodes a topological order based on the information sets that the user has defined for them, throwing an error if the diagram has a directed cycle. 

% Figure environment removed

After the user has successfully generated an influence diagram, they can generate the model directly from the diagram structure, as shown in Figure \ref{fig:code-model}. The model is generated using specialised functions for declaring decision variables and constraints which merely require the (empty) JuMP model and the influence diagram structure as parameters. These functions include optional keyword arguments which allow the user to define forbidden and fixed subpaths, probability cuts and probability scaling for better computational performance. A significant advantage of implementing the framework using JuMP is that more advanced users can easily extend these models by adding new variables or constraints as needed. All of the features presented in this paper are implemented in the framework. These include valid inequalities as lazy constraints, conditional Value-at-Risk objective function and the single policy update heuristic. 

% Figure environment removed

Finally, the results of the optimization model can be extracted using a variety of functions. Figure \ref{fig:code-results} showcases arguably the most important of these, namely the functions for showing the optimal strategy. Other result functions include printing of the distribution and statistics for the path utilities, and printing so-called state probabilities, that is, probabilities of given states occurring. For instance, with the optimal strategy, the probability of the pig being healthy in stages 2, 3 and 4 is 73\%, 70.5\% and 69.5\%, respectively. 

% Figure environment removed
\section{Computational experiments} \label{sec:computational_experiments}

\subsection{Problem size}

First, we compare the model sizes of the two formulations presented in Chapters \ref{sec:decision_programming} and \ref{sec:formulations}. In both formulations, the number of variables is the same. There are $\prod_{j \in D} |S_j||S_{I(j)}|$ $z$-variables and $|S|$ path variables, either $\pi$ or $x$, depending on the formulation. As for the number of constraints, the formulation \eqref{eq:dp_obj}-\eqref{eq:dp_z_bin} has $\sum_{j \in D}|S_{I(j)}|$ constraints \eqref{eq:dp_z_sum}, $2|S|$ bounds for $\pi$-variables, $|D||S|$ constraints \eqref{eq:dp_pi_upper} and $|S|$ constraints \eqref{eq:dp_pi_lower}. Arranging the terms, the total number of constraints becomes 
\begin{equation}
    \label{eq:num_paths}
    (3+|D|)|S| + \sum_{j \in D}|S_{I(j)}|.
\end{equation}

The formulation \eqref{eq:dp2_obj}-\eqref{eq:dp2_z_bin} has $\sum_{j \in D}|S_{I(j)}|$ constraints \eqref{eq:dp2_z_sum}, $\sum_{j \in D}|S_j||S_{I(j)}|$ constraints \eqref{eq:dp2_x_upper}, 1 constraint \eqref{eq:dp2_prob_sum} and $2|S|$ bounds for $x$-variables. Arranging the terms, the total number of constraints becomes 
\begin{equation}
    \label{eq:num_paths_2}
    2|S| + \sum_{j \in D}(1+|S_j|)|S_{I(j)}|.
\end{equation}

We note that $|S|=\prod_{j \in C \cup D}|S_j|$ and that especially with a large number of nodes, the first term becomes impractically large in both \eqref{eq:num_paths} and \eqref{eq:num_paths_2}. The increase in the number of path-related constraints is exponential, while the increase in the rest of the constraints is often linear, as shown in the following two example problems.

\subsubsection{Pig farm}

For the pig farm example presented in \citet{lauritzen2001representing}, we observe that the problem consists of $3n+1$ decision and chance nodes, where $n$ is the number of decision stages\footnote{Note that this is slightly different to the original paper where the length of the problem is tied to the number of health nodes. The length of a problem with $n$ decision nodes would be $n+1$.}, and that $|S_j|=2$ for all nodes $j \in C \cup D$. 

With these observations, $|S|=2^{3n+1}$ and $|D|=n$. Thus, the number of constraints in Eq. \eqref{eq:num_paths} becomes $(3+n)2^{3n+1} + 2n$ and the corresponding number in Eq. \eqref{eq:num_paths_2} becomes $2^{3n+2} + 6n$. 


\subsubsection{N-monitoring}

For the N-monitoring example presented in \citet{salo2022}, we can perform a similar analysis. The problem consists of $2n+2$ decision and chance nodes, where $n$ is the number of report-action pairs. As in the pig farm problem, $|S_j|=2$ for all nodes $j \in C \cup D$. 

With these observations, $|S|=2^{2n+2}$ and $|D|=n$. Thus, the number of constraints in Eq. \eqref{eq:num_paths} becomes $(3+n)2^{2n+2} + 2n$ and the corresponding number in Eq. \eqref{eq:num_paths_2} becomes $2^{2n+3} + 6n$.  

\subsection{Solution times}

% Figure environment removed

For our framework to have a practical impact, we need to be able to solve the models in a reasonable time. Fig. \ref{fig:sol_times} shows the increase in average solution times over 50 instances as the number of decision stages increases in the two example problems. For the original formulation \eqref{eq:dp_obj}-\eqref{eq:dp_z_bin}, \citet{salo2022} show that solution times are greatly improved by adding a \emph{probability cut} $\sum_{s \in S} \pi(s) = 1$ as a lazy constraint to the model. A lazy constraint is a constraint that is added to the model formulation when it is violated, instead of adding it in the beginning of the process. This approach is thus used in the computational experiments for the original formulation. For \eqref{eq:dp2_obj}-\eqref{eq:dp2_z_bin}, a similar constraint is included in the formulation by default. In this section, unless stated otherwise, constraint \eqref{eq:dp2_prob_sum} is a regular constraint instead of lazy.

For both problems, it seems that the rate of increase in the solution times quickly renders the original formulation \eqref{eq:dp_obj}-\eqref{eq:dp_z_bin} computationally intractable, as seen in Fig \ref{fig:sol_times}. This was also noted by \citet{salo2022} in their computational results. The solution times for the improved formulation \eqref{eq:dp2_obj}-\eqref{eq:dp2_z_bin} using locally compatible path sets seem to increase slightly slower than for the original formulation. The behavior of both formulations for very small models with one or two decision nodes seems to be different from larger models, especially in the N-monitoring problem. This might indicate, e.g., that solver does not use branching for these very small models. Finally, the lazy probability cut that was found to improve solution times in \citet{salo2022} is detrimental to computational performance in the new formulation \eqref{eq:dp2_obj}-\eqref{eq:dp2_z_bin}. 

\begin{table}[ht]
\centering
\begin{tabular}{l|ll}
                & v0.1 & v1.1 \\ \hline
10th percentile & 15.4 & 1.00 \\
median          & 26.4 & 1.18 \\
90th percentile & 31.1 & 1.73 \\
mean            & 25.0 & 1.31
\end{tabular}
\caption{Statistics of the root relaxation quality relative to the optimal solution for 50 randomly generated pig farm problems with 5 decision stages. The solutions are scaled so that a value of 1 corresponds to the optimal solution.}
\label{tbl:stats}
\end{table}

In Table \ref{tbl:stats} we present statistics on the LP relaxation quality. As discussed before, the hypothesis is that the formulation \eqref{eq:dp2_obj}-\eqref{eq:dp2_z_bin} (implemented in v1.1 of the package) is considerably tighter than \eqref{eq:dp_obj}-\eqref{eq:dp_z_bin} (implemented in v0.1). The results from the pig farm problem strongly support this, as more than half of the LP relaxation solutions for the novel formulation are within 20\% of the optimal solution, while the solutions using \eqref{eq:dp_obj}-\eqref{eq:dp_z_bin} are orders of magnitude further from the optimal solution. 

% Figure environment removed

Figure \ref{fig:spu} shows the process of improving solutions in the single policy update (SPU) heuristic. For the 50 instances in this test set, the last solution is found within one second, and it is the optimal solution, despite \citet{lauritzen2001representing} showing that this version of the pig farm problem is not soluble, and the SPU heuristic is thus not guaranteed to find the optimal solution. While the single policy update heuristic is successful in finding good initial solutions quickly, the effect of providing the solver with these initial solutions is negligible (Figure \ref{fig:sol_times}). Combined with the increased performance of the improved formulation, this strongly suggests that improving the LP relaxation bound is much more relevant for improving solution times.



\section{Case study: optimal preventive healthcare for CHD} \label{sec:case_study}

One of the first frameworks for medical decision-making considering whether to treat, test or not treat was developed by \citet{pauker1980threshold}. This framework provides an analytical basis for optimal testing and treatment strategies. They developed two thresholds, referred to as ``testing'' and ``test-treatment'' thresholds. The thresholds are probability cut-offs and they divide subjects into three groups: if the risk of disease is below the ``testing'' threshold, treatment and testing should be withheld, if it is above the ``test-treatment'' threshold, treatment should be given and if the risk falls in between these thresholds then a diagnostic test should be performed and the treatment decision made based on its results. The thresholds are visualised in Figure \ref{fig:thresholds}. 
%
% Figure environment removed

In this case study we use decision programming to optimize the use of traditional and genetic testing to support the targeting of statin medication treatment for preventing coronary heart disease (CHD). This case study is replicated from \citet{hynninen2019value}, where the authors developed a testing and treatment strategy by optimizing net monetary benefit (NMB), a cost-benefit objective consisting of the health outcomes and testing costs within a 10-year time horizon. 

The decision process stems from the patient's state of health, represented by a chance event $H$ describing whether the patient will or will not have a CHD event in the following 10 years. The probability of a CHD event is assumed to be described by a prior risk estimate $R_0$ based on factors such as the age and sex of the patient. The likelihood of a correct prognosis can be improved by carrying out tests on traditional risk factors (TRS), genetic risk factors (GRS) or both. Based on their prognosis, a decision is made on whether a patient is subjected to preventive treatment with statin medication. 

In \citet{hynninen2019value}, six predefined testing and treatment strategies were evaluated independently. In each of these strategies, the optimal allocation of tests and treatment according to risk estimates was obtained by solving the associated decision tree (via dynamic programming). The six strategies considered in \citet{hynninen2019value} were: 
%
\begin{enumerate*}[label=(\roman*)]
    \item no tests and no treatment (‘No treatment’);
    \item using prior risk to allocate treatment (‘Treatment optimized’);
    \item performing TRS on optimized patient segment and allocating treatment based on updated risk estimates (‘TRS optimized’);
    \item performing GRS on optimized patient segment and allocating treatment based on updated risk estimates (‘GRS optimized’);
    \item performing TRS on optimized patient segment and based on its results performing GRS optimally to allocate treatment (‘TRS \& GRS optimized’);
    \item performing GRS on optimized patient segment and based on its results performing TRS optimally to allocate treatment (‘GRS \& TRS optimized’).
\end{enumerate*}

Essentially, this comprises determining optimal ``testing'' and ``test-treatment'' thresholds (cf. Figure 
\ref{fig:thresholds}) for TRS and GRS from the perspective of net monetary benefit (NMB) for each strategy (i-vi). Interestingly, the threshold values for GRS in \citet{hynninen2019value} were different than the ones found in the study presented in \citet{tikkanen2013genetic}. This is due to the different perspectives – pure patient welfare versus NMB – that the studies were conducted from. For example, the national health care guidelines for allocating treatment were not considered in the optimization in \citet{hynninen2019value}. This showcases that the two thresholds described in \citet{pauker1980threshold} are not unique for a given disease and prognostic test because the perspective of the study affects the threshold values.

Analogously, our decision programming model determines an optimal decision strategy for allocating preventative care for CHD. The data and structure of the problem are the same as those utilised in \citet{hynninen2019value}. However, due to the flexibility of decision programming, the strategies (i-vi) do not need to be explicitly predefined. Instead, we can optimize the design of the strategy itself simultaneously with the threshold values, meaning that all of these strategies are within the feasible solutions of the model. 

The problem setting is such that the patient is assumed to have a prior risk estimate $R_0$. A risk estimate is a prediction of the patient’s chance of having a CHD event in the next ten years. The risk estimates are grouped into risk levels, which range from 0\% to 100\% with a suitable discretization, e.g., $S_{R_0}=\{0\%, 1\%, ..., 99\%, 100 \% \}$. We note that it might be beneficial to consider a less trivial discretization that is finer in the region where most of the probability mass is assumed to lie and coarser elsewhere. Nevertheless, we chose to proceed as such since it requires no information on the probability distributions. The first testing decision $T_1$ is made based on the prior risk estimate. This entails deciding whether to perform TRS or GRS or if no testing is needed. If a test is conducted, the risk estimate is updated ($R_1$) and based on the new information a second testing decision $T_2$ follows. It entails deciding whether further testing should be conducted or not. The second testing decision is constrained so that the same test which was conducted in the first stage cannot be repeated. If a second test is conducted, the risk estimate is updated again ($R_2$). The treatment decision $T_D$ – dictating whether the patient receives preventive statin medicine or not – is made based on the resulting risk estimate of this testing process. Note that if no tests are conducted, the treatment decision is made based on the prior risk estimate. Figure \ref{fig:CHD_influence_diagram} provides an influence diagram for the decision problem. 

% Figure environment removed

Node $H$ represents the uncertainty of whether the patient has a CHD event or remains healthy during the 10-year time frame. Node $H$ has the prior risk level $R_0$ in its information set because a premise of the modeling proposed in \citet{hynninen2019value} is that the prior risk accurately describes the
probability of having a CHD event, i.e., 
%
\begin{equation*}
    P(H = \text{CHD} \mid R_0=\alpha)=\alpha.
\end{equation*}
%
On the other hand, nodes $R_1$ and $R_2$ represent the updated risk level after the first and second test decisions, respectively. If a test is conducted, the risk estimate is updated using the Bayes' rule
%
\begin{equation*}
    P(\text{CHD} \mid \text{test result}) = \frac{ P(\text{test result} \mid \text{CHD}) \times P(\text{CHD}) }{P(\text{test result})},
\end{equation*}
%
where the conditional probabilities $P(\text{test result}) \mid \text{CHD})$ are from \citet{abraham2016genomic} and the probability of having a CHD event, denoted by $P(\text{CHD})$, is the prior risk level $R_0$ or the updated risk level $R_1$, depending on whether it is the first or second test in question. The denominator $P(\text{test result})$ is calculated as a sum of the numerator and $P(\text{test result} \mid \text{no CHD}) \times  P(\text{no CHD})$, where $P(\text{no CHD}) = 1 - P(\text{CHD})$. As the states of nodes $R_i$, $i \in \{0,1,2\}$, represent risk levels, the probability of a state in these nodes is the probability of the given test updating the risk estimate to that level from the previous estimate.

The first and second testing decisions are represented by $T_1$ and $T_2$, respectively. Since conducting the same test twice is forbidden, all paths where the same test is repeated in $T_1$ and $T_2$ are included in the set of forbidden paths (cf. Section \ref{sec:formulations}). Furthermore, the forbidden paths include all paths where the first testing decision $T_1$ is to not perform testing but then the second testing decision $T_2$ is to perform a test. This is because the information yielded from performing only one test is not affected by whether the test is performed in the first or second stage of testing. Therefore, forbidding the paths where no test is performed in $T_1$ and TRS or GRS is performed in $T_2$ reduces redundancy in the model without information loss. The final treatment decision is represented by node $T_D$, where the options are to provide or withhold treatment. The treatment decision is made based on the updated risk estimate represented by node $R_2$.

Since the first node in the influence diagram presented in Figure \ref{fig:CHD_influence_diagram} is the chance node $R_0$, any decision strategy would be conditioned on its realisation. This leads to a natural separability of the problem, meaning that it can be solved for individual risk levels $0\%, 1\%, \dots 100\%$. This has the benefit of allowing the calculations to be parallelised, at the expense of potentially causing inconsistencies related to e.g., multiple solutions in the MIP problem or rounding-induced errors.

An interesting result is that the optimal strategy found by our model is the same strategy that was deemed the best among strategies (i-vi) in \citet{hynninen2019value}. In a way, this provides optimality guarantees to their results, which, in principle, they could not have determined without exhaustively testing all possible (9) testing strategies. In addition, the optimal threshold from our model corresponded closely to those in \citet{hynninen2019value}. Figure \ref{fig:CHD_result} illustrates the strategy obtained by our model, indicating also the thresholds found in \citet{hynninen2019value} for comparison. We are confident that the small differences in the threshold values are simply artefacts related to the way the discretization (i.e., rounding) was performed.

% Figure environment removed


%% -*- mode: LaTeX; fill-column: 78; -*-

\section{Concluding Remarks}
\label{sec:conclusions}

In this paper, we presented a novel SMC algorithm, \EventDPOR, tailored to the
characteristics of event-driven multi-threaded programs running under the SC
semantics. The algorithm was proven correct and optimal for event-driven
programs in which the variable accesses of events do not depend on how their
execution is interleaved with other threads.

We have implemented \EventDPOR in the \Nidhugg tool, and we will open-source
our implementation.
%
With a wide range of event-driven programs, we have shown that \EventDPOR
incurs only a moderate constant overhead over its baseline implementation
(\OptimalDPOR), it is exponentially faster than existing state-of-the-art SMC
algorithms in time and number of traces examined on programs where events'
actions do not conflict, and does not suffer from performance degradation
caused by having to examine
% a significant number of
non-serializable executions.
%
%% \bjcom{Should we include:
%% Moreover, in our benchmarks, also those that are not non-branching,
%% \EventDPOR explores only the optimal number of executions, and never
%% had to resort to a potentially expensive decision procedure.}

\EventDPOR assumes that handlers can process their events in arbitrary order.
Directions for future work include to retarget \EventDPOR for event-driven
programs with other policies (e.g., FIFO), and for specific event-driven
execution models.


\section*{Acknowledgements}

We are enormously grateful for the input from Juho Andelmin, whose initial implementations led to the development of \texttt{DecisionProgramming.jl}. We are also grateful for the contributions of a number of graduate and undergraduate students to the development of the package, as well as to the welcoming and supportive JuMP community. We also gratefully acknowledge the financial support from the Research Council of Finland (decision number 332180). Finally, the computational experiments were performed using computer resources within the Aalto University School of Science “Science-IT” project.


\bibliographystyle{plainnat}
\bibliography{references}

\end{document}
