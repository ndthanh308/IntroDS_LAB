\section{Julia interface} 
\label{sec:interface}

The ideas and formulations presented in this paper have been implemented as a Julia \citep{Julia-2017} package. The main purpose of the package is to provide an interface through which a user can build an influence diagram and automatically convert it to a MILP model that can be solved using an off-the-shelf solver. This brings us to the main reasons for choosing Julia as the language of the package. Thanks to JuMP \citep{DunningHuchetteLubin2017} and MathOptInterface \citep{legat2022mathoptinterface}, implementing optimization problems in a general form is straightforward, and the models constructed in this way can be passed to many different solvers.

In order to create a user-friendly package that precludes a deeper understanding of MILP modeling by the user, we implemented a structure for influence diagrams and an interface for constructing them, demonstrated in Figure \ref{fig:code-nodes}. In that, we illustrate the implementation of the pig problem, originally proposed in \citet{lauritzen2001representing}. 

The influence diagram structure consists of nodes and their state spaces and information sets, as well as the corresponding probabilities and utility values. The interface includes functions for adding these elements into the influence diagram and type structures that guide the user to include the required information. For example, all node types require the user to define an information set for the node, even if it is empty in the case of a root node. In turn, the function for adding nodes stops the user from accidentally including a node in its own information set, ensures that the names of all nodes are unique, and warns about redundant nodes. 

% Figure environment removed

Furthermore, specialised structures and functions (Figure \ref{fig:code-diagram}) for defining and adding chance nodes' conditional probabilities and value nodes' utility values ensure that these matrices have correct dimensions, that the probabilities sum to one and that utility values are defined for all information states of value nodes. The interface also includes a function for generating the arcs in the diagram and giving the nodes a topological order based on the information sets that the user has defined for them, throwing an error if the diagram has a directed cycle. 

% Figure environment removed

After the user has successfully generated an influence diagram, they can generate the model directly from the diagram structure, as shown in Figure \ref{fig:code-model}. The model is generated using specialised functions for declaring decision variables and constraints which merely require the (empty) JuMP model and the influence diagram structure as parameters. These functions include optional keyword arguments which allow the user to define forbidden and fixed subpaths, probability cuts and probability scaling for better computational performance. A significant advantage of implementing the framework using JuMP is that more advanced users can easily extend these models by adding new variables or constraints as needed. All of the features presented in this paper are implemented in the framework. These include valid inequalities as lazy constraints, conditional Value-at-Risk objective function and the single policy update heuristic. 

% Figure environment removed

Finally, the results of the optimization model can be extracted using a variety of functions. Figure \ref{fig:code-results} showcases arguably the most important of these, namely the functions for showing the optimal strategy. Other result functions include printing of the distribution and statistics for the path utilities, and printing so-called state probabilities, that is, probabilities of given states occurring. For instance, with the optimal strategy, the probability of the pig being healthy in stages 2, 3 and 4 is 73\%, 70.5\% and 69.5\%, respectively. 

% Figure environment removed