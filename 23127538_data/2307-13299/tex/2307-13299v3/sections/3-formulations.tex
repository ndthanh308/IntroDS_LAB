\section{Improved formulations}
\label{sec:formulations}

One key challenge associated with formulation \eqref{eq:dp_obj}--\eqref{eq:dp_z_bin}, and, in fact, any MILP formulation, is that computational performance is strongly tied to the tightness of the formulation. In this context, the tightness of a MILP formulation is related to how close the linear relaxation solution is to the initial primal bound, e.g., the first integer feasible solution value obtained by the solver during the solution process or one obtained using primal heuristics. 

Next, we present reformulations developed to enhance the numerical performance of the decision programming formulation \eqref{eq:dp_obj}--\eqref{eq:dp_z_bin}. For that, let us first define the subset of paths
%
\begin{equation*}
    S_{s_j \mid s_{I(j)}} = \braces{s \in S \mid (s_{I(j)}, s_j) \subseteq s}.    
\end{equation*}
%
Notice that we use the notation $(s_{I(j)}, s_j)$ to represent a portion of a path $s$, formed by the combination of the information state $s_{I(j)}$ (which may itself be a collection of states, if $|I(j)| > 1$) and the state $s_j$. We also utilise the set operator $\subseteq$ to indicate that the states $(s_{I(j)}, s_j)$ are part of the path $s \in S$. Notice that the states $(s_{I(j)}, s_j)$ do not need to be consecutive in the path $s$, although the ordering between $s_{I(j)}$ and $s_j$ is naturally preserved in $s$. 

Considering $j \in D$, the subset $S_{s_j \mid s_{I(j)}}$ allows us to define the notion of \emph{locally compatible paths}, that is, the collection of paths $s$ compatible with local strategies $Z_j$ for which $\mathbb{I}(s_{I(j)}, s_j) = 1$. The definition of the subset $S_{s_j \mid s_{I(j)}}$ allows us to derive the following valid inequality for \eqref{eq:dp_obj}--\eqref{eq:dp_z_bin}. 
%
\begin{equation} \label{eq:locally_compatible_paths}
    \sum_{s \in S_{s_j \mid s_{I(j)}}} \pi(s) \leq z(s_j \mid s_{I(j)}), \quad \forall j \in D, s_j \in S_j, s_{I(j)} \in S_{I(j)}.
\end{equation}

Constraint \eqref{eq:locally_compatible_paths} states that only paths that are compatible with the selected strategy might be allowed to have a probability different than zero. Moreover, since it is enforced on all decision nodes, it means that this constraint guarantees that only the paths that are compatible with the strategy $Z$ are active. Recall that we denote this set of compatible paths as $S(Z) \subseteq S$.

As pointed out in \citet{salo2022}, for expected utility maximization, constraint \eqref{eq:dp_pi_lower}, which prevents variables $\pi(s)$ from wrongly taking value zero, is only required when some of the utility values $U(s)$, $s \in S$, are negative. Notice that this is otherwise prevented by the maximization of the objective function \eqref{eq:dp_obj}, naturally steering these variables to their upper bound values. Another way to guarantee that the variables $\pi(s)$ take their correct value, i.e., $\pi(s) = p(s)$, if $s \in S(Z)$, is to impose the constraint
%
\begin{equation} \label{eq:total_prob}
    \sum_{s \in S} \pi(s) = 1.
\end{equation}

As it will be discussed in Section \ref{sec:computational_experiments}, replacing \eqref{eq:dp_pi_upper} and \eqref{eq:dp_pi_lower} with \eqref{eq:locally_compatible_paths} and \eqref{eq:total_prob} provides considerable gains in terms of linear relaxation strengthening. Furthermore, we observe that the computational performance can be even further improved by employing a simple variable substitution. Recall that in the original formulation \eqref{eq:dp_obj}-\eqref{eq:dp_z_bin}, variables $\pi(s)$ represent the conditional path probability $\mathbb{P}(s \mid Z) = p(s) \prod_{j \in D} z(s_j \mid s_{I(j)})$. If we let $x(s) \in [0,1]$, $s \in S$ represent the product $\prod_{j \in D} z(s_j \mid s_{I(j)})$, then we can reformulate the problem by substituting $\pi(s) = p(s)x(s)$ for all $s \in S$.

Although $x(s)$, $s \in S$, is continuous, it behaves as a binary variable which takes value one whenever the path is compatible with the strategy and zero, otherwise. This is analogous to the behaviour of variable $\pi(s) \in [0,p(s)]$ in \eqref{eq:dp_obj}--\eqref{eq:dp_z_bin}. We highlight that, from a theoretical standpoint, there is no obvious reason for performing such a substitution. On the other hand, we will show that it yields significant practical benefits in terms of computational performance.


Using these $x$-variables, we can reformulate \eqref{eq:locally_compatible_paths} as
%
\begin{equation} \label{eq:dp2_locally_compatible_paths_1}
    \sum_{s \in S_{s_j \mid s_{I(j)}}} x(s) \leq |S_{s_j \mid s_{I(j)}}|z(s_j \mid s_{I(j)}), \quad \forall j \in D, s_j \in S_j, s_{I(j)} \in S_{I(j)}, 
\end{equation}
%
a consequence of $x(s) \in [0,1]$ and the fact that $z(s_j \mid s_{I(j)})$ must be equal to 1 for $x(s)$ to be positive for $s \in S_{s_j \mid s_{I(j)}}$.

% Constraint \eqref{eq:dp2_locally_compatible_paths_1} can be strengthened further. It turns out that we can infer the number of paths from the set of active paths $S(Z)$ that are locally compatible with a given decision node state $s_j$, with $j \in D$, given its information state $s_{I(j)}$. For that, we observe that because we enforce binary values for the $z$-variables representing decision strategies, at every decision node $k \in D \setminus{j}$ only one alternative $s_k \in S_k$ will be selected, in accordance with constraint \eqref{eq:dp_z_sum}. This means that the number of locally compatible paths that will also be active, i.e., $|S_{s_j \mid s_{I(j)}} \cap S(Z)|$ can be defined as
Constraint \eqref{eq:dp2_locally_compatible_paths_1} can be strengthened further. We note that a path must be in the set of compatible paths $S(Z)$ in order for $x(s)$ to be positive with strategy $Z$. Using this information, we can infer a tighter upper bound for the number of paths that can be active ($x(s) >0$) from the set of locally compatible paths. We observe that in a set of compatible paths $S(Z)$, each information state $s_{I(j)}$ maps to exactly one decision alternative $s_j$ for each decision node $j \in D$, in accordance with constraint \eqref{eq:dp_z_sum}. However, the set of locally compatible paths for a given pair of information state and decision node state $(s_{I(j)}, s_j)$ of decision node $j \in D$, includes paths for all combinations $(s_{I(k)}, s_k)$ of information states and decisions for the other decision nodes $k \in D \setminus \braces{j}$. Hence, only a fraction of the locally compatible paths can be active. The fraction is linked to the number of states $| S_k |$ of the other decision nodes $k \in D \setminus \braces{j}$. The number of locally compatible paths that will also be active, i.e., $|S_{s_j \mid s_{I(j)}} \cap S(Z)|$ can be defined as
%
\begin{equation} \label{eq:dp2_locally_active_compatible_paths}
    |S_{s_j \mid s_{I(j)}} \cap S(Z)| = \frac{|S_{s_j | s_I(j)}|} {\Pi_{k \in D \setminus (\braces{j} \cup I(j))} |S_k|}.
\end{equation}
%
Notice that the calculation of the number of active paths must take into account the fact that some decision nodes may be part of the information state $I(j)$ of node $j \in D$, and, as such, will have their states observed (or fixed) in the set $S_{s_j | s_I(j)}$. Therefore, these decision nodes must be excluded from the product in the denominator in equation \eqref{eq:dp2_locally_active_compatible_paths}. Using \eqref{eq:dp2_locally_active_compatible_paths}, we can reformulate \eqref{eq:dp2_locally_compatible_paths_1} into the strengthened form
%
\begin{equation} \label{eq:dp2_locally_compatible_paths_2}
    \sum_{s \in S_{s_j \mid s_{I(j)}}} x(s) \leq  \frac{|S_{s_j | s_I(j)}|} {\Pi_{k \in D \setminus (\braces{j} \cup I(j))} |S_k|}z(s_j \mid s_{I(j)}), \quad \forall j \in D, s_j \in S_j, s_{I(j)} \in S_{I(j)}. 
\end{equation}

One last aspect that can be taken into account is that, depending on the problem structure, some sequence of states $s = (s_i)_{i=1,\dots,n}$ forming a path may never be observed and can be preemptively filtered out from the set of paths $S$. This is the case, for example, in problems where earlier decisions or uncertain events dictate whether alternatives or uncertainties are observed. For instance, an initial decision regarding whether or not to build an industrial plant naturally restricts subsequent decisions regarding capacity expansion. Analogously, it may be that an uncertain production rate is only observed if one decides to build the production facility in the first place. To prevent the assembling of these unnecessary paths, we consider a set of \emph{forbidden} paths, which, once removed, lead to a set $S^* \subseteq S$ of effective paths. Notice that these forbidden paths have probability zero by the structure of the problem, and therefore their removal does not affect the expected utility nor the constraints of the model. Furthermore, their removal allows for significant savings in terms of the scale of the model.

One issue emerges in settings where $S^* \subset S$ regarding the term \eqref{eq:dp2_locally_active_compatible_paths}. Notice that the bound is based on the premise that we can infer the total number of paths by considering the Cartesian product of the state sets $S_j$, $j \in N$. However, as forbidden paths are removed, some of the $x$-variables corresponding to paths $s \in S_{s_j | s_I(j)}$ might be removed, making inequality \eqref{eq:dp2_locally_compatible_paths_2} loose. A simple safeguard for this is to consider  
%
\begin{equation} \label{eq:gamma_value}
    \Gamma(s_j|s_{I(j)}) = \min \braces{|S^*_{s_j \mid s_{I(j)}}|, \frac{|S_{s_j | s_I(j)}|} {\Pi_{k \in D \setminus (\braces{j} \cup I(j))} |S_k|}}
\end{equation}
%
and reformulate \eqref{eq:dp2_locally_compatible_paths_2} as
%
\begin{equation} \label{eq:dp2_locally_compatible_paths_3}
    \sum_{s \in S_{s_j \mid s_{I(j)}}} x(s) \leq \Gamma(s_j|s_{I(j)})z(s_j \mid s_{I(j)}), \quad \forall j \in D, s_j \in S_j, s_{I(j)} \in S_{I(j)}. 
\end{equation}

Combining the above, we can reformulate \eqref{eq:dp_obj}--\eqref{eq:dp_z_bin} as follows.
%
\begin{align}
    \underset{Z \in \mathbb{Z}}{\text{maximize }}  &\sum_{s \in S^*} U(s)p(s)x(s)\label{eq:dp2_obj}\\
    \text{subject to }  
    &\sum_{s_j \in S_j} z(s_j \mid s_{I(j)}) = 1, &&\forall j \in D, s_{I(j)} \in S_{I(j)} \label{eq:dp2_z_sum}\\
    & \sum_{s \in S_{s_j \mid s_{I(j)}}} x(s) \leq \Gamma(s_j|s_{I(j)}) z(s_j \mid s_{I(j)}), &&\forall j \in D, s_j \in S_j, s_{I(j)} \in S_{I(j)} \label{eq:dp2_x_upper} \\
    & \sum_{s \in S^*} p(s)x(s) = 1, && \label{eq:dp2_prob_sum} \\
    & 0 \leq x(s) \leq 1, &&\forall s \in S^* \label{eq:dp2_x_lim} \\ 
    &z(s_j \mid s_{I(j)}) \in \{0,1\}, &&\forall j \in D, s_j \in S_j, s_{I(j)} \in S_{I(j)} \label{eq:dp2_z_bin}.
\end{align}
%
where $\Gamma(s_j|s_{I(j)})$ is defined as in \eqref{eq:gamma_value}. Note that this formulation preserves the (mixed-integer) linear nature of \eqref{eq:dp_obj}--\eqref{eq:dp_z_bin}.

As discussed earlier, one of the main advantages of the decision programming formulation is the ability to incorporate objectives and constraints involving arbitrary utility functions and probability distributions within the model. \citet{salo2022} demonstrate this by proposing a model that considers conditional value-at-risk as one of the utility functions. The same can be achieved with our proposed formulation, by simply substituting $p(s)x(s)$ in place of variables $\pi(s)$. 

The path-based structure of \eqref{eq:dp_obj}--\eqref{eq:dp_z_bin} makes formulating chance and budget constraints straightforward. For modeling chance constraints, we can define $\tilde{S}$ as the set of ``undesirable" paths, the total probability of which must not exceed $\rho$. The corresponding chance constraint is then 
\begin{equation} \label{eq:chance_constraint}
    \sum_{s \in \tilde{S}} x(s)p(s) \le \rho. 
\end{equation}

Likewise, the set $\tilde{S}$ could further be defined as, e.g., the set of paths with a small utility $U(s) \le u_{threshold}$. With $\rho=0$, \eqref{eq:chance_constraint} can be seen as a budget constraint, stating that for all compatible paths $s \in S(Z)$ with $p(s)>0$, the utility $U(s)$ must be at least $u_{threshold}$. 


% \subsection{A tighter bound for the $x$-variables (not yet implemented in the package)}

% A possible modification to formulation \eqref{eq:dp2_obj}-\eqref{eq:dp2_z_bin} is to tighten the upper bound in \eqref{eq:dp2_x_lim}. The upper bound $\frac{\sum_{j \in D} z(s_j \mid s_{I(j)})}{|D|}$ takes value 1 if $z(s_j \mid s_{I(j)})=1$ for all $j \in D$, that is, the path $s$ is compatible with the decision strategy represented by the $z$-variable values. If the path is not compatible with the strategy, the upper bound is in the interval $[0,1)$.
