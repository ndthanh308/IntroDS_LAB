\section{Primal heuristic: single policy update (SPU)} 
\label{sec:heuristics}

While our approach of reformulating influence diagrams into mixed-integer linear models does allow us to use powerful off-the-shelf solvers, it is still hindered by the well-known fact that solving such problems is NP-hard \cite{schrijver2003combinatorial}. To make MIP solvers more efficient, \emph{primal heuristics} are used to obtain and improve integer solutions. Obtaining good starting solutions can have a significant impact on the performance of branch-and-bound solvers, as it helps in pruning poor quality solutions early.

Decision Programming is based on the so-called limited-memory influence diagrams (LIMIDs) and solution approaches presented in previous literature can be used to obtain solutions to these problems. A notable contribution of \citet{lauritzen2001representing} is the single policy update (SPU) heuristic for obtaining ``locally optimal'' strategies in the sense that the corresponding solutions cannot be improved by changing only one of the local strategies $Z_j(s_{I(j)})$.  

The heuristic in our implementation of DecisionProgramming.jl is loosely based on the ideas in \citet{lauritzen2001representing} but it is a simplified version. The first step of the heuristic is to obtain a random strategy $Z$ (note that this too is a heuristic, albeit a very simple one). This strategy is then iteratively improved by examining each information state $s_{I(j)} \in S_{I(j)}$ for each decision node $j \in D$ in order, choosing the local strategy $Z'_j(s_{I(j)})$ maximizing the expected utility. We obtain incrementally improving strategies $Z$ by replacing the local strategy $Z_j(s_{I(j)})$ with $Z'_j(s_{I(j)})$ whenever these two strategies differ, that is, we found a local strategy $Z'_j(s_{I(j)})$ improving the expected utility.

This process of locally improving the strategy is performed repeatedly for all pairs $(j, s_{I(j)})$ until no improvement is made during a whole iteration through the set of such pairs. It is easy to see that at this point, there is no possible local improvement and the strategy is thus, in that sense, locally optimal. \citet{lauritzen2001representing} show that for \emph{soluble} LIMIDs, this heuristic results in a global optimal solution. The performance of the heuristic in DecisionProgramming.jl is explored in Section \ref{sec:computational_experiments}.

