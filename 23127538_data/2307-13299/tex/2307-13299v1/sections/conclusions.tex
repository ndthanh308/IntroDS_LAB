\section{Conclusions} \label{sec:conclusions}

In this paper, we have presented improvements to decision programming \cite{salo2022}, and \texttt{DecisionProgramming.jl}, a Julia package that allows representing decision problems as MIP models. We expand on the ideas originally proposed in \cite{salo2022} by providing not only a computational user interface for modelling such problems but also by providing several methodological enhancements. These enhancements include a novel and more efficient formulation, valid bounds to tighten relaxations, and a heuristic which can be used to warm start the MIP solver.

We have also conducted a novel case study based on the study originally proposed by \citet{hynninen2019value}. Our objective is to demonstrate that our package can be used in settings which would normally require resorting to more ad-hoc computational tools, and that it lends itself to be a general and accessible tool for practitioners.

...