\documentclass{amsart}[12pt]

\newcommand{\red}{\textcolor{red}}
\newcommand{\blue}{\textcolor{blue}}
\newcommand{\hn}[1]{\h^{#1}}
\newcommand{\cL}{\mathcal{L}}
\renewcommand{\S}{\Sigma}
\newcommand{\norma}[1]{\Vert #1 \Vert}
\newcommand{\ol}{\overline}
\renewcommand{\O}{\Omega}
\newcommand{\cH}{\mathcal H}
\newcommand{\cS}{\mathcal S}
\newcommand{\es}{\emptyset}
\newcommand{\pai}{\partial_\infty}


\usepackage{enumerate}
\usepackage[hidelinks]{hyperref}

\usepackage{soul}
\usepackage{yfonts}
\usepackage{amssymb}
\usepackage{amsthm}
\usepackage{array}
\usepackage{booktabs}
\usepackage{hhline}
\usepackage{xy}
\usepackage{epsfig}
\usepackage{color}
\usepackage{upgreek}
\usepackage[english]{babel}
\usepackage{epigraph}
\usepackage{fancybox}
\setcounter{totalnumber}{2}
\usepackage{shadow}
\usepackage{afterpage}
\usepackage{mathrsfs}
\usepackage{enumitem}
\usepackage{tabularx}
\usepackage{subcaption}
\usepackage{graphicx}
\usepackage{type1cm}
\usepackage{eso-pic}
\usepackage{color}
\usepackage{upgreek}
\usepackage[foot]{amsaddr}
%\usepackage[pagewise]{lineno}\linenumbers
\usepackage{verbatim}



\newtheorem{theorem}{Theorem}[section]
\newtheorem{proposition}[theorem]{Proposition}
\newtheorem{question}[theorem]{Question}
\newtheorem{conjecture}[theorem]{Conjecture}
\newtheorem{lemma}[theorem]{Lemma}
\newtheorem{claim}[theorem]{Claim}
\theoremstyle{definition}
\newtheorem{definition}[theorem]{Definition}
\newtheorem{remark}[theorem]{Remark}
\newtheorem{example}[theorem]{Example}
\newtheorem*{acknowledgments}{Acknowledgments}
\theoremstyle{plain}
\newtheorem{corollary}[theorem]{Corollary}


\newcommand{\otoprule}{\midrule[\heavyrulewidth]}
\newcommand{\vt}{\vspace{.1cm}}
\newcommand{\vtt}{\vspace{.2cm}}
\newcommand{\ws}{W_{\scriptscriptstyle S}}
\newcommand{\R}{\mathbb{R} }
\newcommand{\q}{\mathbb{Q}_{\epsilon}^n }
\newcommand{\Z}{\mathbb{Z} }
\newcommand{\N}{\mathbb{N} }
\newcommand{\h}{\mathbb{H} }
\newcommand{\ha}{\mathscr{H} }
\newcommand{\hf}{\mathbb{H}_{\mathbb F}^m}
\newcommand{\hfr}{\mathbb{H}_{\mathbb F}^m\times\R}
\newcommand{\g}{{\rm grad}}
\newcommand{\ga}{\gamma}
\newcommand{\n}{\partial_t}
\newcommand{\s}{\mathbb{S}}
\renewcommand{\thefootnote}{\scriptsize{(\roman{footnote})}}
\renewcommand{\rho}{\varrho}
%\renewcommand{\theta}{\varTheta}
\newcommand{\abs}[1]{\vert #1\vert}
\renewcommand{\Theta}{\varTheta}
\renewcommand{\Lambda}{\varLambda}
\renewcommand{\Sigma}{\varSigma}
\renewcommand{\tau}{\uptau}
\captionsetup[subfigure]{labelfont=rm}
\renewcommand{\thesubfigure}{\alph{subfigure}}
\usepackage{amsmath}% http://ctan.org/pkg/amsmath
\newcommand{\myfrac}[3][0pt]{\genfrac{}{}{}{}{\raisebox{#1}{$#2$}}{\raisebox{-#1}{$#3$}}}
\newcommand{\overbar}[1]{\mkern 1.5mu\overline{\mkern-1.5mu#1\mkern-1.5mu}\mkern 1.5mu}
\newcommand{\mr}{\overbar M^n\times\R}
%\renewcommand{\overbar{M}\times_\rho\,\R}{\R\times_\rho\overbar{M}}
\newcommand{\forma}[1]{\langle #1\rangle}
\newcommand{\ssr}{\scriptscriptstyle{R}}
\newcommand{\ssl}{\scriptscriptstyle{L}}
\newcommand{\wt}[1]{\widetilde{#1}}

\newcommand{\transv}{\mathrel{\text{\tpitchfork}}}
\makeatletter
\newcommand{\tpitchfork}{%
 \vbox{
 \baselineskip\z@skip
 \lineskip-.52ex
 \lineskiplimit\maxdimen
 \m@th
 \ialign{##\crcr\hidewidth\smash{$-$}\hidewidth\crcr$\pitchfork$\crcr}
 }%
}
\makeatother



\begin{document}

\title[Solitons to mean curvature flow in $\hn3$]{Solitons to Mean Curvature Flow \\ in the hyperbolic $3$-space}
\author{R. F. de Lima \and A. K. Ramos \and J. P. dos Santos}
\address[A1]{Departamento de Matem\'atica - UFRN}
\email{ronaldo.freire@ufrn.br}
\address[A2]{Departamento de Matemática Pura e Aplicada - UFRGS}
\email{alvaro.ramos@ufrgs.br}
\address[A3]{Departamento de Matem\'atica - UnB}
\email{joaopsantos@unb.br}
\thanks{The second and third authors were partially supported by the National Council for
Scientific and Technological Development – CNPq}

\maketitle

\begin{abstract}
We consider translators to mean curvature flow
in hyperbolic $3$-space $\mathbb H^3$, providing existence
and some classification results.
More specifically, we show the existence and uniqueness
of a one-parameter family of complete rotational
catenoid-type translators, as well as of a
one-parameter family of translators which are
parabolic cylinders.
We establish a tangency principle for translators
in $\h^3$ and apply it
to prove that {properly immersed} translators to mean
curvature flow in $\h^3$ are not cylindrically bounded.
In addition, we classify all translators in $\mathbb H^3$
of constant mean curvature. 
Finally, we construct a one-parameter family of
complete helicoidal rotating
solitons (rotators) to mean curvature flow in $\h^3.$


\vspace{.15cm}
\noindent{\it 2020 Mathematics Subject Classification:} 53E10 (primary), 53E99 (secondary).

\vspace{.1cm}

\noindent{\it Key words and phrases:} soliton -- mean curvature flow -- hyperbolic space -- invariant surfaces.


\end{abstract}

\section{Introduction}
{The last decades flourished with great regard to the theory
of extrinsic geometric flows in Riemannian manifolds, especially to
mean curvature flow
in Euclidean spaces, giving rise to a
vast literature on the subject
(cf.~\cite{andrewsetal} and the references therein).
Extrinsic geometric flows constitute evolution equations that describe hypersurfaces
of a Riemannian manifold evolving in the normal direction with velocity given by the corresponding
extrinsic curvature. A special class of solutions is that of the {\em solitons}, also known as the
{\em self-similar} solutions, which are characterized
for being generated by the Killing field defined by a one-parameter
subgroup of isometries of the ambient manifold.
When these isometries are translations, we call
the corresponding self-similar solutions
{\em translating solitons} to the giving flow, and the initial
hypersurfaces are known as {\em translators}.
{A main feature of translators in Euclidean spaces is that
they naturally appear as type II singularities of certain compact solutions
to mean curvature flow (cf.~\cite[Theorem 4.1]{huisken-sinestrari})}.}

There exist many examples of translators to mean
curvature flow (MCF, for short)
in Euclidean space $\R^3.$
Three of the best known
are the cylinder over the graph of the function $f(t)=-\log(\cos t),$ $t\in(-\pi/2,\pi/2),$ called
the {\em grim reaper}, the rotational entire graph over $\R^2$ obtained by Altschuler and Wu~\cite{altschuler-wu}
known as the {\em translating paraboloid} {or {\em bowl soliton}}, and the one-parameter family of rotational annuli obtained by
Clutterbuck, Schn\"urer and Schulze~\cite{schulzeetal}, the so called {\em translating catenoids}.
On the other hand, little is known about translators in hyperbolic spaces.

In this paper, we consider solitons to MCF in hyperbolic
space $\h^3$, and first we focus on the case of
translators. We classify all such surfaces with
constant mean curvature and also obtain
new families of examples,
which, by some similarities with the translators
in $\R^3$ described above, will be called
the {\em translating catenoid}
(Theorem~\ref{th-translatingcatenoids})
and the {\em grim reaper} (Theorem~\ref{th-grimreapers}). 
These translators are then proven to be unique with respect to 
their fundamental properties (Theorems \ref{th-uniquenesscatenoid} 
and \ref{th-uniquenesscylinder}).
We also establish a tangency principle for
translators (Theorem~\ref{tangencyprin}) which is used,
together with the family of translating catenoids,
to prove that properly immersed translators to MCF in $\hn3$ are
never cylindrically bounded.

We study, as well, 
{\em rotators} to MCF, that is, initial data of solitons
whose associated isometries are rotations.
In \cite{halldorsson}, Halldorsson considered rotators in $\R^3,$ obtaining a one-parameter
family of complete helicoidal rotators in $\R^3$ which are also translators.
Inspired by Halldorsson's work,
we obtain here an analogous result
{(Theorem~\ref{th-MCFh3})},
in which we construct a one-parameter family of helicoidal
surfaces in $\h^3$ that, under mean curvature flow, rotates around its axis
and translates downwards with velocity that equals its pitch.

The paper is organized as follows. In Section \ref{sec-preliminaries}, we set some
notation and formulae. In Section \ref{sec-translators}, we introduce translators
to MCF in $\h^3$ and establish the aforementioned results related to them.
In Section \ref{sec-rotators} we deal with rotators to MCF in $\h^3.$
Finally, we use Section~\ref{seclastproof}
to present the classification of minimal translators in $\hn3$.

\begin{acknowledgments}
We heartily thank Eric Toubiana for pointing
us to the existence result necessary for the final argument in Section 5, and also
Marcus Marrocos, for the enlightening conversations that led to the proofs of
Lemmas \ref{lem-rotationalODE02} and \ref{lem-parabolicODE02}.
\end{acknowledgments}



\section{Preliminaries} \label{sec-preliminaries}

Throughout the paper, we shall consider the upper half-space model of
$\h^3,$ that is, $\h^3:=(\R_+^3,ds^2),$
where $\R^3_+ = \{(x_1,x_2,x_3)\in \R^3\mid x_3>0\}$,
$ds^2:={d\bar s^2}/{x_3^2}$ and
$d\bar s^2$ is the standard Euclidean
metric of $\R_+^3.$ We will also denote $ds^2$ by
$\langle\,,\,\rangle.$

Let $\Sigma$ be an oriented surface
in a Riemannian 3-manifold $\ol M$. Set
$\ol \nabla$ for the Levi-Civita connection of $\ol M,$
$\eta$ for the unit normal field of $\Sigma,$
and $A$ for its shape operator with respect to
$\eta,$ so that
\[
AX=-\ol \nabla_X\eta, \,\, X\in T\Sigma,
\]
where $T\Sigma$ stands for the tangent bundle of $\Sigma$.
The principal curvatures of $\Sigma,$ that is,
the eigenvalues of $A,$ will be denoted by
$k_1, k_2$ and
the mean curvature $H$ of $\Sigma$ is
expressed by
\[
H=\frac{k_1+k_2}2.
\]
The mean curvature vector of $\S$ is
$$\mathbf{H}=H\eta,$$
which is invariant under the choice of orientation
$\eta\to -\eta$ and satisfies $\norma{\mathbf{H}} = \abs{H}$.




Given an oriented surface $\Sigma\subset\R_+^3,$ let
$\bar\eta=(\bar\eta_1,\bar\eta_2,\bar\eta_3)$
be the unit normal of $\Sigma$ with respect to
the induced Euclidean metric $d\bar s^2.$
It is easily checked that
\[
\eta(p)=x_3\bar\eta(p), \,\,\, p=(x_1,x_2,x_3)\in\Sigma
\]
defines a unit normal of $\Sigma$ with respect to the
hyperbolic metric $ds^2.$
With these orientations, if we denote by
$\overbar H$ (resp.~$H$) the mean curvature of $\Sigma$ with respect to the Euclidean metric
(resp. hyperbolic metric) of $\R_+^3,$
we have that $\overbar H$ and $H$ satisfy the following relation (cf. \cite[Lemma 10.1.1]{lopez}):
\begin{equation} \label{eq-MCrelation}
H(p)=x_3\overbar H(p)+\bar\eta_3(p) \,\,\, \forall p=(x_1,x_2,x_3)\in\Sigma.
\end{equation}


\subsection{Mean curvature flow}
We say that a family of oriented
surfaces $\Sigma_t=X_t(M)$ of a Riemannian $3$-manifold $\overbar M$
\emph{evolves under mean curvature flow} if the corresponding one-parameter
family of immersions
\[
X_t\colon M\rightarrow\overbar M, \,\,\, t\in[0,\delta), \,\,\, 0<\delta\le+\infty,
\]
satisfies the following condition:
\begin{equation} \label{eq-Kalphaflow}
\frac{\partial X_t}{\partial t}^\perp(p)=H_t(p)\eta_t(p) \,\,\, \forall p\in M,
\end{equation}
where $\eta_t$ is the unit normal to $X_t,$ $H_t$ is the mean curvature
of $X_t$ with respect to $\eta_t,$ and
$\frac{\partial X_t}{\partial t}^\perp$ denotes the normal component of
$\frac{\partial X_t}{\partial t},$ that is,
\[
\frac{\partial X_t}{\partial t}^\perp=\left\langle\frac{\partial X_t}{\partial t},\eta_t\right\rangle \eta_t\,.
\]
In particular, the equality \eqref{eq-Kalphaflow} is equivalent to
$$
\left\langle\frac{\partial X_t}{\partial t},\eta_t\right\rangle=H_t.
$$

We call such a family $X_t:M\rightarrow\overbar M,$ $t\in[0,\delta)$
a \emph{mean curvature flow} (MCF, for short) in $\overbar M$ with initial data $X_0.$
{In this setting,
we say that $\Sigma_t=X_t(M)$
is a \emph{soliton} or a \emph{self-similar solution}} to MCF if there exists a
one-parameter subgroup $\mathcal G:=\{\Gamma_t\,;\, t\in\R\}$ of the
group of isometries of $\overbar M,$ such that $\Gamma_0$ is the identity map of
$\overbar M$ and
$$
\Sigma_t=\Gamma_t(\Sigma)\,\,\,\forall t\in\R
$$
is a MCF. More specifically, we shall call such a
family $\Sigma_t$ a $\mathcal G$-\emph{soliton}.


Let $\xi$ be the Killing field determined by the subgroup $\mathcal G,$ that is,
for any $p\in\overbar M,$
\[
\xi(p):=\frac{\partial\Gamma_t}{\partial t}(p) \,\,\,\,\, \text{at} \,\,\, t=0.
\]
It can be proved (see, e.g., \cite{hungerbuhler-smoczyk})
that the surface $\Sigma=X_0(M)$ with unit normal
$\eta$ is the initial condition of a $\mathcal G$-\emph{soliton} generated by $\xi$
in $\overbar M$ if and only if the equality
\begin{equation} \label{eq-main}
{H = \forma{\xi,\eta}}
\end{equation}
holds everywhere on $\Sigma.$ So, {in the class of solitons, equation \eqref{eq-Kalphaflow}
is in fact a prescribed mean curvature problem.}



\section{Translators to MCF in $\h^3$} \label{sec-translators}
Consider in hyperbolic space $\h^3$
the group $\mathcal G=\{\Gamma_t\,;\, t\in\R\}\subset{\rm Iso}(\h^3)$
of hyperbolic translations defined by
\[
\Gamma_t(p)=e^tp, \,\,\, p\in\h^3.
\]
In this setting, an initial condition of a $\mathcal G$-soliton will be called
a \emph{translating soliton} or simply a \emph{translator}.
Using the abuse of notation
$$p = (x_1,x_2,x_3)\in \hn3 \leftrightarrow
x_1\partial_{x_1}+ x_2\partial_{x_2}+ x_3\partial_{x_3}
\in T_p\hn3,$$
the Killing field associated to $\mathcal G$ is
$\xi(p)=p,$ $p\in\h^3$.
Thus, it follows from~\eqref{eq-main} that a
surface $\Sigma\subset\h^3$ is a translator to
MCF if and only if
\begin{equation} \label{eq-translatorH301}
H(p)=\langle p,\eta(p)\rangle \,\,\,\forall p\in\Sigma.
\end{equation}


\begin{example}
Let $\Pi$ be a totally geodesic vertical plane of $\h^3$
which contains $(0,0,1)$.
Since $H$ vanishes on $\Pi$, it is clear that
\eqref{eq-translatorH301} holds for $\Sigma=\Pi.$
Thus, $\Pi$ is a stationary translator to MCF in $\h^3.$
\end{example}

In fact, equation~\eqref{eq-main} implies that a
minimal surface $\S\subset \hn3$ is a (stationary)
translator to MCF
if and only if it is invariant under the group $\mathcal G$
of hyperbolic isometries as above.
A complete classification of such surfaces
is given by the following description.



\begin{theorem}\label{thmconj}
{There exists a one-parameter family $\S_\theta$, $\theta\in(0,\pi],$
of properly embedded minimal surfaces in $\h^3$ with the following properties:}
\begin{itemize}[parsep=1ex]
\item[\rm i)] {$\Sigma_\theta$ is invariant under
the one-parameter group $\{\Gamma_t\}_{t\in\R}$ of
hyperbolic translations}
\[
{p\in\h^3\mapsto e^tp\in\h^3,}
\]
{and so it is a stationary translator to MCF in $\h^3.$}

\item[\rm ii)] {$\partial_\infty\Sigma_\theta\cap\R^2$ is the union of two half lines making an angle $\theta.$}

\item[\rm iii)] {$\S_\pi$ is a vertical plane.}
\end{itemize}
{Furthermore, if $\Sigma$ is a properly embedded minimal surface of \,$\h^3$ which is invariant under the group
$\Gamma_t,$ then $\Sigma=\S_\theta$ for some $\theta\in(0,\pi].$}
\end{theorem}

The proof of Theorem~\ref{thmconj}, for convenience,
will be presented separately in Section~\ref{seclastproof}.
Concerning the case of translators with nonzero
constant mean curvature,
we start with the next example.



\begin{example}
Let $\mathscr H_h$ be the horosphere of $\h^3=(\R^3_+\,,\,ds^2)$
at height $h>0,$ i.e.,
\[
\mathscr H_h=\{(x_1,x_2,h)\in\h^3\,;\, x_1,x_2\in\R\}.
\]
At any point $p=(x_1,x_2,h)\in\mathscr H_h,$ we have that
$H(p)=1$ and $\eta(p)=he_3,$ so that
\[
\langle p,\eta(p)\rangle=\frac1{h^2}h^2=1=H(p) \,\,\ \forall p\in\mathscr H_h.
\]
Hence, $\mathscr H_h$ is a translator to MCF in $\h^3.$
\end{example}



{In our next result we show that horospheres are the only translators
to MCF which have nonzero constant mean curvature.
In the proof, we shall
use the following evolution formula for the mean curvature $H_t$ (notation as in Section \ref{sec-preliminaries})
of a mean curvature flow $X_t:M\to\overbar M$:}
\begin{equation} \label{eq-evolutionequation}
\frac{\partial H_t}{\partial t}=\Delta H_t+H_t(\|A_t\|^2+\overbar{\rm Ric}(\eta_t,\eta_t)),
\end{equation}
{where $\overbar{\rm Ric}$ denotes the Ricci tensor of $\overbar M$ (see \cite[Theorem 3.2-(v)]{huisken-polden}).}

\begin{theorem} \label{th-cmctranslator}
Let $\Sigma$ be a connected translator to MCF in $\h^3$ which
has nonzero constant mean curvature.
Then, $\Sigma$ is an open subset of a horosphere.
\end{theorem}
\begin{proof}
{After a change of orientation, we may assume without loss
of generality that the mean curvature $H$ of $\S$ is positive.}
{Let $X_t:M\to\h^3,$ $t>0,$ be the MCF such that $X_0(M)=\Sigma$ and
$$X_t(p)=e^tX_0(p), \,\,\, p\in M.$$
Since
$X_t(M)$ differs from $X_0(M)$ by an ambient isometry,
$H_t=H>0$ is constant in space and time, thus
${\partial H_t}/{\partial t}=\Delta H_t=0.$
Also, in $\h^3,$ $\overbar{\rm Ric}(\eta_t,\eta_t)=-2.$
Then, formula \eqref{eq-evolutionequation} yields $\|A_t\|^2=2$
for all $t\ge 0.$ Taking $t=0,$ we conclude that
the principal curvatures $k_1, k_2$ of $\Sigma$ satisfy:}
\[
{\left\{
\begin{array}{cccccl}
k_1&+&k_2&=&2H,\\[1ex]
k_1^2&+&k_2^2&=&2,
\end{array}
\right.}
\]
{from where it follows that $H\in(0,1]$ and,
after reindexing,}
$${k_1 = H+\sqrt{1-H^2},\quad k_2 = H-\sqrt{1-H^2}.}$$
{Since $H$ is constant, both $k_1$
and $k_2$ are constant, so $\S$ is isoparametric.
The isoparametric surfaces of $\hn3$ are classified
(see~\cite[Theorem 3.14]{cecil-ryan}) and the fact
that $H\in(0,1]$ imply that $\S$ is either an open subset of
a horosphere
or of an equidistant surface to a totally geodesic
plane. However, $k_1^2+k_2^2 = 2$ only
holds when $\S$ is contained in a horosphere, which finishes the proof
of the theorem.}
\end{proof}




\begin{remark} \label{rem-cmcsoliton}
Since \eqref{eq-evolutionequation} holds for
any $\mathcal{G}$-soliton, the proof of Theorem \ref{th-cmctranslator} applies to show that any initial condition of a
$\mathcal{G}$-soliton in $\h^3$ with nonzero constant mean curvature is necessarily an open subset of a horosphere.
\end{remark}




\subsection{Rotational translators.}
{In this section, we focus on translators to MCF in $\h^3$ which are
invariant under rotations about the $x_3$-axis. With this purpose, we
first consider vertical rotational graphs. More precisely, let
$\phi$ be a positive smooth function on an open interval
$I\subset(0,+\infty),$ and assume its graph $\S$ in $\h^3$ is invariant under
rotations about the $x_3$-axis. Then, $\S$ admits a parameterization of the form
$$
X(u,v)=(v\cos u,v\sin u,\phi(v)), \,\,\,\, (u,v)\in U:=\R\times I\subset\R^2.
$$
We shall call $\Sigma=X(U)$ the \emph{rotational vertical graph determined by} $\phi.$}


For a rotational graph $\S$ as above,
a direct computation gives that
\[
\bar\eta:=(\bar\eta_1,\bar\eta_2,\bar\eta_3)=\rho(-\phi'\cos u,-\phi'\sin u,1), \quad \rho:=\frac{1}{\sqrt{1+(\phi'))^2}},
\]
is a unit normal with respect to the induced Euclidean metric,
and that the corresponding Euclidean mean curvature is
\[
\overbar H=\frac{\rho}{2}\left(\frac{\phi''}{1+(\phi')^2}+\frac{\phi'}{v}\right).
\]
Thus, from \eqref{eq-MCrelation}, the mean curvature $H$ of $\Sigma$ in $\h^3$ with respect to $\eta:=\phi\bar\eta$ is
\begin{equation} \label{eq-Hrotationaltranslator}
H=\phi\overbar H+\bar\eta_3=\rho\left(\frac{\phi}{2}\left(\frac{\phi''}{1+(\phi')^2}+\frac{\phi'}{v}\right)+1\right).
\end{equation}
It is also straightforward to see that the equality
\begin{equation} \label{eq-Xeta}
\langle X,\eta\rangle=\frac{\rho}{\phi}(\phi-v\phi')
\end{equation}
holds everywhere on $\Sigma.$


From \eqref{eq-Hrotationaltranslator} and \eqref{eq-Xeta}, we conclude that
equation \eqref{eq-translatorH301} for the vertical graph
$\Sigma$ is equivalent to the second order ODE:
\begin{equation} \label{eq-rotationalODE01}
\phi''=-\phi'(1+(\phi')^2)\left(\frac{2v}{\phi^2}+\frac1v\right).
\end{equation}

Defining $\Omega:= (0,+\infty)\times(0,+\infty)\times\R,$ and
\[
\Psi(x,y,z)=-z(1+z^2)\left(\frac{2x}{y^2}+\frac1x\right), \,\,\, (x,y,z)\in\Omega,
\]
we get from \eqref{eq-rotationalODE01} the following

\begin{lemma} \label{lem-rotationalODE01}
A vertical rotational graph determined by a smooth function $\phi$ is
a translator to MCF in \,$\h^3$
if and only if $\phi$ is a solution to the second order ODE:
\begin{equation} \label{eq-EDOPsi}
y''=\Psi(x,y,y').
\end{equation}
\end{lemma}

Next, we establish some properties of the solutions to \eqref{eq-EDOPsi}.


 \begin{lemma} \label{lem-rotationalODE02}
For any $x_0,y_0>0$ and any $\lambda\in\R,$ the initial value problem
\begin{equation} \label{eq-cauchyproblem}
\left\{
\begin{array}{l}
y''=\Psi(x,y,y')\\[1ex]
y(x_0)=y_0\\[1ex]
y'(x_0)=\lambda
\end{array}
\right.
\end{equation}
has a unique smooth solution $\phi$ on
$[x_0,+\infty)$ which has the following properties:
\begin{itemize}[parsep=1ex]
\item[\rm (i)] $\phi$ is constant if $\lambda=0.$
\item[\rm (ii)] $\phi$ is increasing, concave and bounded above by a positive constant if $\lambda>0.$
\item[\rm (iii)] $\phi$ is decreasing, convex and bounded below by a positive constant if $\lambda<0.$
\end{itemize}
\end{lemma}

\begin{proof}
 Since $\Psi$ is $C^\infty$ in $\Omega,$ the standard results on solutions for
 ODE's ensure the existence and uniqueness of a $C^\infty$ solution $\phi$
 defined in a maximal interval $I_{\rm max}:=[x_0,x_{\rm max}), \,x_{\rm max}\le+\infty,$
in the sense that the equality
 \begin{equation} \label{eq-phi''proof}
 \phi''=\Psi(x,\phi,\phi')=-\phi'(1+(\phi')^2)\left(\frac{2x}{\phi^2}+\frac{1}{x}\right)
 \end{equation}
 holds in $I_{\rm max}.$

 If $\lambda=0,$ it is clear from \eqref{eq-phi''proof} that the solution
 $\phi$ is constant, in which case $x_{\rm max}=+\infty.$ This proves (i).

 Assume now that $\lambda>0.$ Then, $\phi$ is increasing near $x_0.$ Also, from
 property (i) and the uniqueness of solutions,
 $\phi$ has no critical points. Hence, $\phi$ is increasing in $I_{\rm max}.$
 In addition, equality \eqref{eq-phi''proof}
 gives that $\phi$ is concave in $I_{\rm max},$ which yields $x_{\rm max}=+\infty.$

 Let us prove that $\phi$ is bounded above. To do so, set
 \[
 F(x,\phi,\phi'):=-(1+(\phi')^2)\left(\frac{2x}{\phi^2}+\frac1x\right)
 \]
 and observe that \eqref{eq-phi''proof}, together with the equality $(\log(\phi'))'=\phi''/\phi',$ yields
 \begin{equation} \label{eq-marrocos}
 \phi'(x)=\lambda \exp\left(\int_{x_0}^{x}F(t,\phi(t),\phi'(t))dt\right).
 \end{equation}
 Clearly, $F(x,\phi(x),\phi'(x))<-1/x$ for all $x\in I_{\rm max}.$ Thus,
 \[
 0<\phi'(x)\le\lambda\exp\left(\int_{x_0}^{x}-\frac1tdt\right)
=\frac{\lambda x_0}{x}\,,
 \]
 which implies that
 \begin{equation} \label{eq-lim}
 \lim_{x\to+\infty}\phi'(x)=0.
 \end{equation}

 Now, notice that the equality
 \[
 \lim_{x\to+\infty}\frac{x}{\phi(x)}=+\infty
 \]
 holds regardless $\phi$ being bounded or unbounded. Indeed, in the first
 case, the equality is trivial, and in the latter case, it follows from
 \eqref{eq-lim} and the l'Hôpital rule.
 In particular, there exists $x_1>x_0$ such that
 ${x^2}/{\phi^2(x)}>1/2\ \forall x\ge x_1,$ which yields
 \[
 \frac{2x}{\phi^2(x)}+\frac{1}{x}>\frac{2}{x} \quad \forall x\in I_1:=[x_1,+\infty).
 \]

 From this last inequality, we have that $F(x,\phi(x),\phi'(x))<-2/x$ for all
 $x\in I_1.$ Set
$$\log(\Lambda) = \int_{x_0}^{x_1}F(t,\phi(t),\phi'(t))dt.$$
Then, considering \eqref{eq-marrocos} once more, we obtain,
$\forall x\in I_1$
\begin{eqnarray*}
\phi'(x)& = & \lambda
\exp\left(\int_{x_0}^{x_1}F(t,\phi(t),\phi'(t))dt +
\int_{x_1}^{x}F(t,\phi(t),\phi'(t))dt\right)\\
&\le&\lambda\Lambda\exp\left(\int_{x_0}^{x}-\frac2tdt\right)\\
& = & \frac{\lambda\Lambda x_0^2}{x^2} \,.
\end{eqnarray*}
By integrating both sides on $[x_1,x]\subset I_1$,
we finally get
 \[
 \phi(x)-\phi(x_1)\le \lambda\Lambda x_0^2\left(\frac{1}{x_1}-\frac{1}{x}\right)< \lambda \Lambda \frac{x_0^2}{x_1} \quad \forall x\in I_1,
 \]
 which implies that $\phi$ is bounded above. This proves (ii).



 To prove (iii), we can argue as in the proof of (ii) to conclude
 that $\phi$ is decreasing and convex in $I_{\rm max}$ if $\lambda<0.$ We claim that
 $$\lim_{x\to x_{\rm max}}\phi(x)>0,$$
 which, by the definition of $\Psi$, implies that $x_{\rm max}=+\infty$.


 Assume, by contradiction, that $\lim_{x\to x_{\rm max}}\phi(x)=0.$ Then,
 one has
 \begin{equation} \label{eq-phi''infinite}
 \lim_{x\to x_{\rm max}}\phi''(x)=+\infty.
 \end{equation}
 Indeed, equality~\eqref{eq-phi''infinite} follows directly from
 \eqref{eq-phi''proof} if
 $\lim_{x\to x_{\rm max}}\phi'(x)\ne0.$
 If, instead, $\lim_{x\to x_{\rm max}}\phi'(x)=0,$ then
 \[
 \lim_{x\to x_{\rm max}}\frac{\phi'(x)}{\phi^2(x)}=\lim_{x\to x_{\rm max}}\frac{\phi''(x)}{2\phi(x)\phi'(x)}\cdot
 \]
 So, if $\phi''$ were bounded, the above limit would be infinite. But then, from \eqref{eq-phi''proof},
 we would have $\lim_{x\to x_{\rm max}}\phi''(x)=+\infty,$ which would be a contradiction.
 Hence,~\eqref{eq-phi''infinite} holds.

 Now, we compute $\phi'''$ from equality \eqref{eq-phi''proof}, obtaining
 \begin{eqnarray*}
 \phi''' &=& -\phi''(1+(\phi')^2)\left(\frac{2x}{\phi^2}+\frac1x\right)
 -2(\phi')^2\phi''\left(\frac{2x}{\phi^2}+\frac1x\right) \\
 && -\phi'(1+(\phi')^2)\left(\frac{2}{\phi^2}-\frac{4x\phi'}{\phi^3}-\frac{1}{x^2}\right).
\end{eqnarray*}

 Therefore, setting
 \[
 \chi:= -\phi''(1+(\phi')^2)\frac1x-2(\phi')^2\phi''\frac1x+\phi'(1+(\phi')^2)\frac{1}{x^2},
 \]
 we have $\chi<0$ in $I_{\rm max}$ and

 \begin{eqnarray}
 \phi''' &=&\frac{2x(1+(\phi')^2)}{\phi^2}\left(\frac{2(\phi')^2}{\phi}-\phi''\right)-\frac{2\phi'(1+(\phi')^2)}{\phi^2}-\frac{4x(\phi')^2\phi''}{\phi^2}+\chi \nonumber\\
 &=& \frac{2x(\phi')^2(1+(\phi')^2)}{\phi^3}\left(2-\frac{\phi\phi''}{(\phi')^2}\right)-\frac{2\phi'}{\phi^2}(1+(\phi')^2+2x\phi'\phi'')+\chi. \label{eq-phi'''}
 \end{eqnarray}
 However, from \eqref{eq-phi''proof}, one has
 \begin{eqnarray*}
 \lim_{x\to x_{\rm max}}\frac{\phi(x)\phi''(x)}{(\phi'(x))^2}&=&
 \lim_{x\to x_{\rm max}}-\frac{1+(\phi'(x))^2}{\phi'(x)}\left(\frac{2x}{\phi(x)}+\frac{\phi(x)}{x}\right)\\
 &\ge& \lim_{x\to x_{\rm max}}-\frac{2x}{\phi(x)\phi'(x)} =+\infty,
 \end{eqnarray*}
 and
 \begin{eqnarray*}
 \lim_{x\to x_{\rm max}}[\phi'(x)\phi''(x)]&=& \lim_{x\to x_{\rm max}}\left[-(\phi'(x))^2(1+(\phi'(x))^2)\left(\frac{2x}{\phi^2(x)}+\frac{1}{x}\right)\right]\\
 &\le& \lim_{x\to x_{\rm max}}-\frac{2x(\phi'(x))^2}{\phi^2(x)}=-\infty.
 \end{eqnarray*}
 In the last limit, we used the fact that
 $$\lim_{x\to x_{\rm max}}\frac{\phi'(x)}{\phi(x)}=-\infty,$$
 which is immediate if $\lim_{x\to x_{\rm max}}\phi'(x)<0.$ Otherwise, it follows easily from the l'Hôpital rule.



 From the above limits and \eqref{eq-phi'''}, we conclude that
 $\phi'''(x)<0$ for all sufficiently large $x\in I_{\rm max},$ which contradicts
 \eqref{eq-phi''infinite}. This finishes the proof of (iii), and so of the lemma.
 \end{proof}


Lemmas~\ref{lem-rotationalODE01} and~\ref{lem-rotationalODE02}
already imply the existence of rotational translators. However,
to improve the description of these examples, we next
consider rotational surfaces which are also horizontal
graphs. More precisely, given a rotational surface
$\Sigma\subset\hn3$ with axis
$\ell:=\{0\}\times(0,+\infty)$, let us consider
$\gamma = \S\cap \{x_1 = 0\}$ as the profile
curve of $\S$ and assume that
the tangent plane of $\Sigma$ at a given point $p\in \gamma$
is not orthogonal to $\ell$. If we let
$d$ denote the Euclidean distance function from $\gamma$ to
$\ell$ on $\R_+^3$ and let $v$ parameterize $\gamma$, then,
in a neighborhood of $p,$ $\Sigma$ can be parameterized as
$$
X(u,v):=(u,\sqrt{d^2(v)-u^2},v), \,\,\, (u,v)\in U\subset\R\times(0,+\infty).
$$
We shall call $X(U)$ the \emph{horizontal rotational graph determined by $d$.}

\begin{lemma} \label{lem-rotationalODE021}
A horizontal rotational graph determined by a smooth function $d$ is
a translator to MCF in $\h^3$ if and only if the function $d$ is a
solution to the ODE:
\[
y''=\left(\frac{2y^2}{x^2}+1\right)\frac{1+(y')^2}{y}\cdot
\]
In particular, such a solution $d$ is strictly convex.
\end{lemma}

\begin{proof}
Writing $\varphi(u,v):=\sqrt{d^2(v)-u^2},$
we have that a Euclidean unit normal to $\Sigma$ is
\[
\bar\eta:=(\bar\eta_1,\bar\eta_2,\bar\eta_3)=\rho(-\varphi_u,1,-\varphi_v), \quad \rho:=\frac{1}{\sqrt{1+\varphi_u^2+\varphi_v^2}},
\]
and the corresponding Euclidean mean curvature is
\[
\overbar H(X(u,v))=\frac{\rho^3(u,v)}{2}\Lambda(u,v),
\]
where $\Lambda$ is the function
\[
\Lambda:=\varphi_{uu}(1+\varphi_v^2)-2\varphi_{uv}\varphi_u\varphi_v+\varphi_{vv}(1+\varphi_u^2).
\]
Hence, the hyperbolic mean curvature $H$ of $\Sigma$ is
\begin{equation} \label{eq-Hhorizontalgraph}
H=v\overbar H+\bar\eta_3=\rho\left(\frac{v\rho^2}{2}\Lambda-\varphi_v\right),
\end{equation}
and its hyperbolic unit normal is $\eta:=v\bar\eta,$ so that
\begin{equation} \label{eq-Xetahorizontalgraph}
\langle X,\eta\rangle=\frac{\rho}{v}(\varphi-u\varphi_u-v\varphi_v).
\end{equation}


From \eqref{eq-Hhorizontalgraph} and \eqref{eq-Xetahorizontalgraph}, after noticing that
$\varphi_u=\frac{-u}{\varphi}$, we have that
the translating soliton equation $\langle X,\eta\rangle=H$
for $\Sigma$ is equivalent to
\begin{equation} \label{eq-lambda1}
\Lambda=\frac{2d^2}{v^2\varphi\rho^2}\cdot
\end{equation}



After taking all first and second order partial derivatives of $\varphi$ and applying to
$\Lambda,$ we get from a direct and long calculation that
\begin{equation} \label{eq-lambda2}
\Lambda=\frac{d^2}{\varphi^3}(dd''-(d')^2-1).
\end{equation}

Finally, observing that
\[
\frac{\varphi^2}{\rho^2}=\varphi^2(1+\varphi_u^2+\varphi_v^2)=\varphi^2\frac{u^2+(dd')^2+\varphi^2}{\varphi^2}=d^2(1+(d')^2),
\]
it follows from \eqref{eq-lambda1} and \eqref{eq-lambda2} that
\[
d''=\left(\frac{2d^2}{v^2}+1\right)\frac{1+(d')^2}{d}\,,
\]
as we wished to prove.
\end{proof}

Now, we are in position to prove the existence of
properly embedded annular translators to MCF
in $\h^3,$ which we shall
call \emph{translating catenoids}, see
Figure~\ref{fig-translatingcatenoid}.

% Figure environment removed


\begin{theorem} \label{th-translatingcatenoids}
There exists a one-parameter family
$\mathscr F:=\{\Sigma_{r}\,;\, r>0\}$
of noncongruent, properly embedded rotational
annular translators in $\h^3.$ For
each $r>0$, the surface $\S_r\in \mathscr F$ satisfies:
\begin{itemize}[parsep=1ex]
\item[\rm i)] $\Sigma_{r}$ is contained in a slab determined by
two horospheres $\mathscr H_{r^-}$ and $\mathscr H_{r^+}.$
In particular, the asymptotic boundary of $\S_r$ is the
point at infinity of the horosphere $\mathscr H$ at height $1$.
%%%%%%%%%%%
\item[\rm ii)] $\S_r$ is the union of two vertical graphs $\Sigma_{r}^-$ and
$\Sigma_{r}^+$ over the complement of the Euclidean $r$-disk
\,$\mathcal D_r$ centered at the rotation axis
in the horosphere $\mathscr H$.
%%%%%%%%%%%%%%
\item[\rm iii)] The graphs $\Sigma_{r}^-$ and $\Sigma_{r}^+$ lie in
distinct connected components of \,$\h^3-\mathscr H$
with common boundary the $r$-circle that bounds $\mathcal D_r$ in
$\mathscr H,$ being $\Sigma_{r}^-$
asymptotic to $\mathscr H_{r^-}$ and $\Sigma_{r}^+$
asymptotic to $\mathscr H_{r^+}.$
\end{itemize}

{In addition, {the limiting behaviour of $\S_r$ is
as follows}:}

\begin{itemize}[parsep=1ex]
\item[\rm iv)] As $r\to 0,$ $\Sigma_r$ converges{\footnote{{The convergence is
on the $C^{2,\alpha}$-norm, on compact sets
outside $(0,0,1)$.}}} to a double copy of
$\mathscr H.$

\item[\rm v)] As $r\to +\infty,$ $\S_r$ escapes to
infinity, and both $\mathscr H_{r^-}$ and
$\mathscr H_{r^+}$ converge to $\mathscr H$.
\end{itemize}
\end{theorem}

\begin{proof}
Given $r>0,$ let $d_r:(1-\delta,1+\delta)\rightarrow(0,+\infty)$ be the local solution to
the following initial value problem:
\begin{equation} \label{eq-cauchyproblem02}
\left\{
\begin{array}{l}
y''=\left(\frac{2y^2}{x^2}+1\right)\frac{1+(y')^2}{y}\,, \\[1ex]
y(1)=r,\\[1ex]
y'(1)=0.
\end{array}
\right.
\end{equation}

By Lemma \ref{lem-rotationalODE021}, the rotational horizontal graph $\Sigma_{r}$ determined by $d_r$ is a translator
to MCF in $\h^3.$ Since $d_r$ is strictly convex, $x=1$ is a strict local minimum of $d_r$
and $\Sigma_{r}-\mathscr H$ is the union of two disjoint rotational vertical
graphs $\Sigma_{r}^-$ and $\Sigma_{r}^+$
over an open set contained in $\mathscr H-\mathcal D_r.$
Let us index $\S_r^+$ as being the component contained in the
horoball $\{x_3>1\}$. Then, Lemma~\ref{lem-rotationalODE01}
applies to $\S_r^+$, which
corresponds to an increasing solution
of \eqref{eq-cauchyproblem}
(i.e., one for which the initial condition
$\lambda$ is positive).
By Lemma~\ref{lem-rotationalODE02}, such a solution is defined in an
interval $[x_0,+\infty)$ and is bounded above.
Therefore, $\Sigma_r^+$ can be continued indefinitely,
being asymptotic to a horosphere $\mathscr H_{r^+}$ of $\h^3.$
In particular, $\Sigma_{r}^+$ is a graph over
$\mathscr H-\mathcal D_r.$

Analogously, Lemmas \ref{lem-rotationalODE01} and
\ref{lem-rotationalODE02} give that $\Sigma_r^-$
can be continued indefinitely and is asymptotic
to a horosphere $\mathscr H_{r^-}$ of $\h^3.$ Since
$\Sigma_{r}={\rm closure}(\Sigma_{r}^-)\cup{\rm closure}(\Sigma_{r}^+),$
we have that $\Sigma_{r}$ is an annular properly
embedded translator to MCF in $\h^3.$
{This proves assertions (i)--(iii).}

{
To prove assertions (iv) and (v), consider the following parameterization of the
graph $G_r$ of the solution $d_r$ of \eqref{eq-cauchyproblem02}:
$$\alpha_r(s):=(0,d_r(s),s), \,\,\, s\in(r_-,r_+).$$
We get from a direct computation that, with the induced Euclidean metric, the
curvature of $\alpha_r$ at $s=1$ is $k_r(1)=d_r''(1)=2r+1/r.$ So, we have (see Fig.~\ref{fig-sequencecurves}):}
\begin{equation}\label{eqkblowsup}
{\lim_{r\to 0}k_r(1)=\lim_{r\to+\infty}k_r(1)=+\infty.}
\end{equation}

% Figure environment removed



Now, set $r_*$ for either $0$ or $+\infty.$
For each $r>0,$ $G_r$ intersects both
$\{x_3>1\}$ and $\{x_3<1\}$. In particular,~\eqref{eqkblowsup}
allows us to choose points
$p_r^-:=\alpha_r(s_r^-)\in G_r\cap\{x_3<1\}$ and
$p_r^+:=\alpha_r(s_r^+)\in G_r\cap\{x_3>1\}$ such that
\[
{\lim_{r\to r_*} s_r^-=\lim_{r\to r_*}s_r^+=1,}
\]
{with unit tangent vectors (with respect to
the Euclidean metric) satisfying}
\begin{equation} \label{eq-vectors}
\lim_{r\to r_*}\frac{\alpha_r'(s_r^+)}{\|\alpha_r'(s_r^+)\|}
=-\lim_{r\to r_*}\frac{\alpha_r'(s_r^-)}{\|\alpha_r'(s_r^-)\|}
={\partial_{x_2},}
\end{equation}
{or, equivalently,
$\lim_{r\to r_*}d_r'(s_r^+) = - \lim_{r\to r_*}d_r'(s_r^-) =
+\infty$.}
{Consider the surfaces
$S_r^-$ and $S_r^+$ obtained from the solutions
of \eqref{eq-cauchyproblem} with the following initial conditions:}
\[
{\left\{
\begin{array}{l}
y(d_r(s_r^-))=s_r^-,\\[1ex]
y'(d_r(s_r^-)=\frac1{d_r'(s_r^-)}
\end{array}
\right.
\quad\text{and}\qquad
\left\{
\begin{array}{l}
y(d_r(s_r^+))=s_r^+,\\[1ex]
y'(d_r(s_r^+)=\frac1{d_r'(s_r^+)}\cdot
\end{array}
\right.}
\]
{Hence, by uniqueness, $S_r^- = \S_r^-$ and
$S_r^+ = \S_r^+$. But then, in the case $r_* = 0$,
the continuity of the family of solutions
of~\eqref{eq-cauchyproblem} with respect to initial data,
together with~\eqref{eq-vectors}, implies that
both $\S_r^-$ and $\S_r^+$ converge, on compact sets,
to the horosphere $\mathscr H$, proving (iv).}

{Furthermore, when $r_* = +\infty$, (v) follows
from~\eqref{eq-vectors} and
items (ii), (iii) in Lemma~\ref{lem-rotationalODE02}.
This concludes our proof.}
\end{proof}


%%%%%%%%%%%%%%%%%%%%%%%%%%%%%%%%%%%%%%%%%%%%%%%%%%%%%%%5

Let $\S$ be a connected rotational translator in $\h^3$
with (possibly empty) boundary.
Then, Lemmas \ref{lem-rotationalODE01} and \ref{lem-rotationalODE021}, together with  
the uniqueness of solutions of ODE's with given initial conditions, 
imply that the profile curve of $\S$ coincides, up to its boundary, to the profile curve of 
some translating catenoid $\Sigma_r$ obtained in Theorem \ref{th-translatingcatenoids}. 
Therefore, we have the following uniqueness result.


\begin{theorem} \label{th-uniquenesscatenoid}
Any connected rotational translator of $\h^3$
is an open subset of some member of the family 
$\mathscr F$ presented in Theorem~\ref{th-translatingcatenoids}.
\end{theorem}

%%%%%%%%%%%%%%%%%%%%%%%%

A distinguished property of translators to MCF in $\R^3$ is that they are critical
points of a weighted area functional and, therefore, they become minimal surfaces when changing
the ambient metric in a suitable manner \cite{ilmanen}.
In particular, the tangency principle applies to them, which allows one
to use translators as barriers (cf.~\cite{lopez2}). On the other hand,
it is unknown to us if translators to MCF in $\h^3$ can be made minimal in
a similar fashion. Nevertheless, as we establish in
the next result,
the tangency principle holds for translators in $\h^3$,
and this will be applied, together with Theorem~\ref{th-translatingcatenoids},
to prove that complete translators
in $\h^3$ are never cylindrically bounded.


\begin{theorem}[{\bf tangency principle for translators}]
\label{tangencyprin}
Let $\Sigma_1$ and $\Sigma_2$ be two translators to MCF in $\h^3$
which are tangent at a point $p\in{\rm int}\,\Sigma_1\cap{\rm int}\,\Sigma_2.$
If $\Sigma_1$ lies on one side of
$\Sigma_2$ in a neighborhood of $p$ in $\h^3,$ then
$\Sigma_1$ and $\Sigma_2$ coincide in a neighborhood of $p$ in $\Sigma_1\cap\Sigma_2.$
Moreover, if $\Sigma_1$ and $\Sigma_2$ are both complete
and connected, then $\Sigma_1=\Sigma_2.$
\end{theorem}

\begin{proof}
Let $\S_1$ and $\S_2$ be two translators to
MCF in $\hn3$, tangent
at a point $p\in \S_1\cap \S_2$, and such that $\S_1$
stays locally
on one side of $\S_2$.
If $T_p\S_1$ is not vertical,
there exist a domain
$\Omega\subset\R^2$ and positive
functions $u_1,\,u_2\colon \Omega \to \R$
such that neighborhoods
$U \subset \S_1$ and $V\subset \S_2$ containing $p$
are respectively parameterized by
$$U = \{(x,y,u_1(x,y))\mid (x,y)\in \Omega\},\quad
V = \{(x,y,u_2(x,y))\mid (x,y)\in \Omega\}.$$
Furthermore, after reindexing we may assume that
$u_1\geq u_2$ in $\Omega$.

Let
$\S_1$ and $\S_2$ be oriented with respect to vector fields
$\eta_1$ and $\eta_2$ so that $\eta_1(p) = \eta_2(p)$ points
upwards. Thus, if $Q$ is the quasilinear elliptic
operator
\begin{equation}\label{eqopQ}
Q(u) = u_{xx}(1+u_y^2)+u_{yy}(1+u_x^2)-2u_{xy}u_xu_y,
\end{equation}
it follows
from~\eqref{eq-MCrelation} that
the mean curvature functions $H_1,\,H_2$ of $U$ and $V$ satisfy
$$H_i =
u_i\frac{Q(u_i)}{2(1+(u_i)_x^2+(u_i)_y^2)^{\frac32}}+
\frac{1}{(1+(u_i)_x^2+(u_i)_y^2)^{\frac12}},\quad
i\in\{1,2\}.$$

Then,
after setting $B(x,y,u,Du) = 2(1+u_x^2+u_y^2)(xu_x+yu_y)$,
where
$Du$ denotes the (Euclidean) gradient of $u$,
it follows from~\eqref{eq-translatorH301} that
\begin{equation}\label{eq:operator}
(u_i)^2Q(u_i) +B(x,y,u_i,Du_i) = 0,\quad
i\in\{1,2\}.
\end{equation}
But the operator $u^2Q(u)+B(x,y,u,Du)$ in~\eqref{eq:operator}
satisfies the hypothesis of the tangency principle for
quasilinear operators~\cite[Theorem~2.2.2]{pserrin},
thus $U = V$.

The case where $T_p\S_1$ is vertical can be treated
analogously: after a rotation about the $x_3$-axis (which
preserves the property of being a translator to MCF),
locally, both $\S_1$ and $\S_2$
can be parameterized as horizontal graphs
$$
\{(x,u_1(x,z),z)\mid (x,z)\in \widehat{\Omega}\} \,\,\, \text{and} \,\,\,
\{(x,u_2(x,z),z)\mid (x,z)\in \widehat{\Omega}\}
$$
for some domain $\widehat{\Omega}\subset \R^2_+$, and both
$u_1,\,u_2$ satisfy
$$z^2Q(u) +\widehat{B}(x,z,u,Du) = 0$$
for $\widehat{B}(x,z,u,Du) = 2(xu_x-u)(1+u_x^2+u_z^2)$
and $Q$ as in~\eqref{eqopQ}. Once again,
we obtain from~\cite[Theorem~2.22]{pserrin} that
$\S_1$ and $\S_2$ coincide in a neighborhood of $p$.

At this point, we have shown that if $\S_1$ and $\S_2$
are tangent at a point $p$, they must coincide
in neighborhoods which are either horizontal
or vertical graphs for $\S_1$ and $\S_2$. The proof
for the case where $\S_1$ and $\S_2$ are complete and connected
now follows from covering $\S_1$ and $\S_2$ with such
(overlapping) neighborhoods.
\end{proof}

\begin{remark}
Theorem~\ref{tangencyprin} contrasts with the tangency
principle for the constant mean curvature case
(see, for instance,~\cite[Theorem~3.2.4]{lopez}): two
distinct
geodesic spheres in $\R^3$ with the same mean curvature
can be tangent to each other without violating the tangency
principle.
In the setting of translators, the tangency principle
does not require any assumptions on the orientation
of $\S_1$ and $\S_2$ because,
from~\eqref{eq-translatorH301}, if $\S_1$ and $\S_2$
are translators to MCF which are tangent at a point $p$,
then necessarily their mean curvature vectors
$\mathbf{H}_1$ and $\mathbf{H}_2$ must agree at $p$,
which defines a coinciding, {\em standard}
(local) orientation for both $\S_1$ and $\S_2$.
\end{remark}

Recall that a circular cone in
$\R_+^3:=\R^2\times (0,+\infty)$ with vertex at
$p\in\R^2$ and axis $\gamma_p:=\{p\}\times (0,+\infty)$ constitutes a
\emph{cylinder} $\mathscr C$ in $\h^3,$ that is, the set of points
of $\h^3$ at a fixed distance to the vertical geodesic $\gamma_p.$ The convex side of
$\mathscr C$ is the component of $\h^3-\mathscr C$ which contains $\gamma_p.$

\begin{corollary} \label{cor-nocylindricallybounded}
{There is no {properly immersed}
translator to MCF in $\h^3$ which is contained in
the convex side of a cylinder with vertex at $p=(0,0).$
In particular, there is no {closed (i.e., compact
without boundary)}
translator to MCF in $\h^3.$}
\end{corollary}

\begin{proof}
Suppose, by contradiction, that there exists a
properly immersed translator $\Sigma$ to MCF in
$\h^3$ which is contained in the convex side $\Omega$ of a cylinder
$\mathscr C$ with vertex at $p=(0,0).$
Clearly, the property of being a translator is invariant
by the translations $\Gamma_t(p):=e^tp, \,t\in\R.$
Therefore, we can assume without loss of generality that
$\Sigma$ intersects the horosphere $\mathscr H$ of height $1.$


Under the above conditions, we have from item (v) of
Theorem~\ref{th-translatingcatenoids} that
there exists $R>0$ such that, for any $r>R,$ the translating catenoid
$\Sigma_r$ of the family $\mathscr F$
is disjoint from $\mathscr C,$ and
so from $\Sigma.$ On the other hand, for a sufficiently
small $r>0,$ $\Sigma_r$
and $\Sigma$ have nonempty intersection.
Taking into account the asymptotic behavior of $\S_r$,
together with the hypothesis that $\S$ is contained in
$\Omega,$, as $r$ decreases from
$R$ to zero,
a standard argument shows that there will be a first
value $r_*$ such that $\Sigma_{r_*}$ is the element
of $\mathscr F$ that first establishes a contact with
$\Sigma$ at a point $p\in\Sigma\cap\Sigma_{r_*}$,
as in Figure~\ref{fig-translatingcatenoidB}. Then,
$\Sigma$ and $\Sigma_{r_*}$ are tangent at $p$
with $\Sigma$ on one side of
$\Sigma_r$, and the tangency principle
(Theorem~\ref{tangencyprin}) applies to show
that $\Sigma=\Sigma_{r_*},$ which is a contradiction,
since $\Sigma$ is contained in $\Omega$ and
$\Sigma_{r_*}$ is not.
\end{proof}

% Figure environment removed


\subsection{Parabolic translators.}

Having considered rotational translators in the previous
section, we now look at
translators which are invariant by a 1-parameter
group of {\em parabolic} isometries of $\hn3$, i.e.,
isometries of $\hn3$ that fix
parallel families of horospheres.
Horizontal cylinders over curves on
vertical totally geodesic planes
of $\h^3$ (to be called \emph{parabolic cylinders})
are the simplest examples of surfaces which are invariant by
parabolic translations. When these generating curves are graphs on the whole of $\R,$
such a surface can be parameterized by a map
$X\colon\R^2\to\R_+^3$ defined by
$$
X(u,v)=(u,v,\phi(v)), \,\, (u,v)\in\R^2,
$$
where $\phi$ is a smooth positive function on $\R.$
We shall call $\Sigma:=X(\R^2)$ the \emph{parabolic cylinder determined by}
$\phi.$

Defining $\rho(v):=(1+(\phi'(v))^2)^{-1/2},$
we have that
\[
\bar\eta:=\rho(0,-\phi',1)
\]
is a unit normal to $\Sigma$ with respect to the induced Euclidean
metric of $\R_+^3.$ With this orientation, the Euclidean mean curvature
$\overbar H$ of $\Sigma$ is
\[
\overbar H=\frac{\rho^3\phi''}{2}\,\cdot
\]

From this last equality and \eqref{eq-MCrelation}, we have that
the hyperbolic mean curvature $H$ of $\Sigma$ with respect to
the orientation $\eta:=\phi\bar\eta$ is
\[
H=\rho\left(\frac{\rho^2\phi\phi''}{2}+1\right).
\]

Since $\langle\eta,X\rangle=\rho(\phi-v\phi')/\phi,$
we also have that the identity \eqref{eq-translatorH301} for
the parabolic cylinder $\Sigma=X(\R^2)$ is equivalent to the following
second order ODE:
\[
\phi''=-\phi'(1+(\phi')^2)\frac{2v}{\phi^2}\cdot
\]

The above considerations yield

\begin{lemma} \label{lem-parabolicODE01}
A parabolic cylinder determined by a smooth function $\phi$ is
a translator to MCF in \,$\h^3$ if and only if $\phi$ is a solution to the second order ODE:
\begin{equation} \label{eq-EDOparabolic}
y''=-y'(1+(y')^2)\frac{2x}{y^2}\cdot
\end{equation}
\end{lemma}

The solutions of \eqref{eq-EDOparabolic} are all increasing on $\R$ and their graphs
are ``S-shaped'', as attested by the following

\begin{lemma} \label{lem-parabolicODE02}
Given $\lambda\ge 0,$ the initial value problem
\begin{equation} \label{eq-parabolicIVP}
\left\{
\begin{array}{l}
y''=-y'(1+(y')^2)\frac{2x}{y^2}\\[1ex]
y(0)=1\\[1ex]
y'(0)=\lambda
\end{array}
\right.
\end{equation}
has a unique smooth solution
$\phi:\R\to\R$ which has the following properties:
\begin{itemize}[parsep=1ex]
\item[\rm i)] $\phi$ is constant if $\lambda=0.$
\item[\rm ii)] $\phi$ is increasing, convex in $(-\infty,0),$ and concave in $(0,+\infty)$ if $\lambda>0.$
\item[\rm iii)] $\phi$ is bounded above and below by positive constants.
\end{itemize}
\end{lemma}
\begin{proof}
Assertion (i) is immediate. So, assume $\lambda>0.$
Proceeding as in the proof of Lemma \ref{lem-rotationalODE02},
we get from the equality
\begin{equation} \label{eq-phi''proof001}
 \phi''=-\phi'(1+(\phi')^2)\frac{2x}{\phi^2}
 \end{equation}
that $\phi$ necessarily satisfies:
\[
\phi'(x)=\lambda\exp\left(\int_0^xF(t,\phi(t),\phi'(t))\right)dt,
\]
where $F$ is given by
\[
F(x,y,z):=-\frac{2x}{y^2}(1+z^2).
\]
Hence, $\phi$ is increasing. This, together with \eqref{eq-phi''proof001}, implies that
$\phi$ is defined
in a maximal interval $I_{\rm max}:=(x_{\rm min},+\infty)$
with $-\infty\le x_{\rm min}<0.$ It also follows from \eqref{eq-phi''proof001} that
$\phi$ is convex in $(x_{\rm min},0),$ and concave in $(0,+\infty).$


Next, we prove that the solution $\phi$ is bounded above.
Since $\phi$ is concave in $(0,+\infty),$ we have that
$\lambda=\phi'(0)>\phi'(x)$ for all $x> 0.$
Then, integration on both sides of this last inequality yields
\[
\phi(x)\le\lambda x+1 \,\,\, \forall x\ge 0,
\]
which implies that
\begin{equation} \label{eq-limitfrac}
\lim_{x\to+\infty}\frac{x}{\phi(x)}\ge\lim_{x\to+\infty}\frac{x}{\lambda x+1}=\frac{1}{\lambda}>0.
\end{equation}

Now, choose a small $\epsilon>0$ such that $C:=\lambda^{-1}-\epsilon$ is positive.
It follows from
\eqref{eq-limitfrac} that, for a sufficiently large
$x_1>0,$ one has
$x/\phi(x)>C$ for all $x\ge x_1,$ so that
$$
\frac{2x}{\phi^2(x)}\ge\frac{2C^2}{x}\,\,\,\, \forall x\ge x_1,
$$
from which we obtain
\begin{equation}\label{eq-2x/phi2}
F(x,\phi(x),\phi'(x))\le -\frac{2C^2}{x} \,\,\, \forall x\ge x_1.
\end{equation}

Now, for any given
$x_0>0$, consider
the following initial value problem:
\begin{equation} \label{eq-parabolicIVPproof}
\left\{
\begin{array}{l}
y''=-y'(1+(y')^2)\frac{2x}{y^2}\\[1ex]
y(x_0)=\phi(x_0)\\[1ex]
y'(x_0)=\phi'(x_0).
\end{array}
\right.
\end{equation}

By uniqueness, $\phi$ is a solution
to~\eqref{eq-parabolicIVPproof}, and once again
we may write
\begin{equation} \label{eq-phi'proof01}
\phi'(x)=\lambda_0\exp\left(\int_{x_0}^xF(t,\phi(t),\phi'(t))\right)dt \,\,\,\, \forall x\ge x_0,
\end{equation}
where $\lambda_0:=\phi'(x_0)>0.$
Thus, defining $\lambda_1 = \phi'(x_1)$, we obtain
\begin{equation} \label{eq-phi'proof02}
\phi'(x)=\lambda_1\exp\left(\int_{x_1}^xF(t,\phi(t),\phi'(t))\right)dt \,\,\,\, \forall x\ge x_1
\end{equation}
and it follows from~\eqref{eq-2x/phi2}
and~\eqref{eq-phi'proof02} that
\[
\phi'(x)\le\lambda_1\left(\frac{x_1}{x}\right)^{2C^2} \,\,\, \forall x\ge x_1,
\]
so that $\phi'(x)\to 0$ as $x\to+\infty.$ Therefore, we have
\[
\lim_{x\to+\infty}\frac{x}{\phi(x)}=+\infty.
\]
In particular, there exists $x_2\ge x_1$ such that the inequality
\[
\frac{2x}{\phi^2(x)}\ge\frac{2}{x}
\]
 holds for all $x\ge x_2,$ which gives that $F(x,\phi(x),\phi'(x))<-2/x$ for all
 $x\ge x_2.$ Therefore, applying~\eqref{eq-phi'proof01}
for $x_2$,
 \[
 \phi'(x)=\lambda_2\exp\left(\int_{x_2}^xF(t,\phi(t),\phi'(t))\right)dt, \,\,\, \lambda_2=\phi'(x_2)
 \]
and we may proceed just as in the proof of Lemma \ref{lem-rotationalODE02}
 to conclude that
 \[
 \phi(x)-\phi(x_2)\le \lambda_2x_2^2\left(\frac{1}{x_2}-\frac{1}{x}\right)<\lambda_2x_2 \quad \forall x\ge x_2,
 \]
 which implies that $\phi$ is bounded above.

Next, we show that $\phi$ is bounded below by a
positive constant. With this purpose,
assume by contradiction that
\begin{equation} \label{eq-contradiction}
\lim_{x\to x_{\rm min}}\phi(x)=0.
\end{equation}
Reasoning as in the proof of item iii of
Lemma \ref{lem-rotationalODE02}, we obtain from this assumption that
 \begin{equation} \label{eq-phi''infinite001}
 \lim_{x\to x_{\rm min}}\phi''(x)=+\infty,
 \end{equation}
 which, in turn, implies that
 \begin{equation} \label{eq-phi'/phi2001}
 \lim_{x\to x_{\rm min}}\frac{\phi'(x)}{\phi(x)}=+\infty.
 \end{equation}

 Equality \eqref{eq-phi''infinite001} gives that $\phi''$ is
 necessarily decreasing in a neighborhood of any
 sufficiently small $x\in(x_{\rm min}, 0).$ However, by computing
 $\phi'''$ from \eqref{eq-phi''proof001}, we get
\begin{equation} \label{eq-ph'''again}
\phi'''= -\phi''(1+(\phi')^2)\frac{2x}{\phi^2}-4(\phi')^2\phi''\frac{x}{\phi^2}
 -\phi'(1+(\phi')^2)\left(\frac{2}{\phi^2}-\frac{4x\phi'}{\phi^3}\right).
 \end{equation}
 Then, considering the equality $-\phi'(1+(\phi')^2)=\phi^2\phi''/(2x),$
 which we get from \eqref{eq-phi''proof001},
 and applying it to the last summand of \eqref{eq-ph'''again}, we obtain
 \[
 \phi'''=\phi''\left[-\frac{2x}{\phi^2}(1+3(\phi')^2)+\frac1x-\frac{2\phi'}{\phi}\right]=
 \phi''\left[-\frac{2x}{\phi^2}-2\frac{\phi'}{\phi}\left(1+3x\frac{\phi'}{\phi}\right)+\frac1x\right].
 \]

 This last equality, together with equations \eqref{eq-contradiction}--\eqref{eq-phi'/phi2001}, clearly yields
 \[
 \lim_{x\to x_{\rm min}}\phi'''(x)=+\infty,
 \]
 which contradicts \eqref{eq-phi''infinite001}. Therefore, we have
 \[
 \lim_{x\to x_{\rm min}}\phi(x)>0,
 \]
 from which we conclude that $\phi$ is bounded from below by a positive constant. In particular,
 we must have $x_{\min}=-\infty.$ This finishes the proof.
 \end{proof}


Lemmas \ref{lem-parabolicODE01} and \ref{lem-parabolicODE02}
immediately give the following result (see Fig. \ref{fig-grimreaperH3}).


% Figure environment removed


\begin{theorem} \label{th-grimreapers}
There exists a one-parameter family
$\mathscr F:=\{\Sigma_{\lambda}\,;\, \lambda\in[0,+\infty)\}$
of noncongruent,
complete translators in $\h^3$ (to be called \emph{hyperbolic grim reapers})
which are horizontal parabolic cylinders
generated by the solutions of \eqref{eq-parabolicIVP}.
As a consequence, $\Sigma_0$ is the horosphere $\mathscr H\subset\h^3$ at height one, and
for $\lambda>0,$
each $\Sigma_{\lambda}\in\mathscr F$ is an entire graph over $\R^2$ which is
contained in a slab determined by two
horospheres $\mathscr H_-$ and $\mathscr H_+.$
Furthermore, there exist open sets $\Sigma_{\lambda}^-$ and $\Sigma_{\lambda}^+$ of $\Sigma_\lambda$
such that $\Sigma_{\lambda}^-$ is
asymptotic to $\mathscr H_-$, $\Sigma_{\lambda}^+$ is asymptotic to $\mathscr H_+,$
and $\Sigma_\lambda={\rm closure}\,(\Sigma_{\lambda}^-)\cup{\rm closure}\,(\Sigma_{\lambda}^+).$
\end{theorem}

\begin{remark} \label{rem-symmetry}
The symmetry in~\eqref{eq-EDOparabolic} allows us to extend
the family $\mathscr F$ in Theorem~\ref{th-grimreapers}
for values $\lambda<0$ by simply defining
$\widetilde{\phi}(x) = \phi(-x)$ for a given solution $\phi$
to~\eqref{eq-parabolicIVP} with positive initial data for
$\phi'$. However, the respective grim reaper generated
by $\widetilde{\phi}$ correspond to a rotation of $\pi$
around the $x_3$-axis, being therefore congruent to
an element of $\mathscr F$.
\end{remark}

Analogously to the rotational case,
the uniqueness of solutions of ODE’s with given initial conditions yields the
following result.


\begin{theorem} \label{th-uniquenesscylinder}
Any connected rotator  in $\h^3$ which is a parabolic cylinder 
is, up to an ambient isometry (see Remark \ref{rem-symmetry}), 
an open subset of some member of the family $\mathscr F$ presented in Theorem~\ref{th-grimreapers}.
\end{theorem}


{If $\Gamma\subset \R^2$ is the graph of the function
$t\in(-\pi/2,\pi/2)\mapsto-\log(\cos t)$, then the
cylinder $\S= \Gamma\times\R\subset\R^3$ is a translator to MCF
contained in a slab $\mathcal S$ of $\R^3$, known as the
\emph{grim reaper cylinder}.}
{This nomenclature is due to the fact that the curve $\Gamma$ provides a solution to
the curve shortening flow,
called \emph{the grim reaper}, which is
given by the translation of $\Gamma$ in $\mathbb{R}^2$ in the $\vec{e}_2$-direction. By the
avoidance principle, such a solution ``kills'' any other solution in the
region $(-\pi/2,\pi/2)\times\mathbb{R}$ (see \cite[Chapter 2]{andrewsetal}). Similarly,
two surfaces (one of them compact) in $\R^3$ moving under MCF which are initially disjoint
remain so until one of them collapses. Hence, as $\S$ translates under MCF, it ``kills'' all solutions to \eqref{eq-Kalphaflow}
in $\mathcal S$ with compact initial condition. An analogous process occurs in our case: any surface of the family $\mathscr F$ in
Theorem~\ref{th-grimreapers} has this
``killing'' property. Indeed, by \cite[Theorem 4]{delima},
the avoidance principle applies to surfaces moving under MCF
in $\h^3.$ For this reason, we named the elements of $\mathscr F$
hyperbolic grim reapers.}


\begin{remark}
At the completion of this manuscript, we became acquainted with the preprint \cite{mari},
in which the authors consider solitons to MCF generated by conformal fields in $\mathbb{H}^n$, called
\emph{conformal solitons}.
There, they obtained rotational and cylindrical conformal solitons whose initial conditions are
named {winglike} catenoids and grim reaper {cylinder}, respectively.
However, such solitons are not related to the ones considered here, since their generating
fields are not Killing.
\end{remark}

We close this section with the following

\begin{conjecture}
A translator of \,$\h^3$ which is an entire graph over $\R^2$ is, up to an ambient isometry,
one of the members of the family $\mathscr F$ presented in Theorem~\ref{th-grimreapers}.
\end{conjecture}


\section{Rotators to MCF in $\h^3$} \label{sec-rotators}

Let us consider now the one-parameter group $\mathcal G\subset{\rm Iso}(\h^3)$ of
rotations $\Gamma_t$ of $\h^3=(\R_+^3,ds^2)$ about the $x_3$-axis.
Considering the decomposition $\R_+^3=\R^2\times (0,+\infty),$ we have that
\[
\Gamma_t=\left[
 \begin{array}{cc}
 e^{tJ} & \\
 & 1
 \end{array}
\right], \quad J=\begin{bmatrix}
0 & -1\\
1 & \phantom-0
\end{bmatrix}.
\]

In this setting, an initial condition of
a $\mathcal G$-soliton will be called
a \emph{rotating soliton} or simply a \emph{rotator}.
The (horizontal) Killing field associated to $\mathcal G$ is
$\xi(p)=J\pi(p),$ $p\in\h^3,$
where $\pi$ denotes the projection over
$\{(0,0,1)\}^\perp\subset\R^3$, i.e.,
$\pi(x_1,x_2,x_3)=x_1\partial_{x_1}+x_2\partial_{x_2}$.
Hence, a surface $\Sigma$ of hyperbolic space $\h^3$ is a rotator to
MCF if and only if
\begin{equation} \label{eq-translatorH300}
H(p)=\langle J\pi(p),\eta(p)\rangle \,\,\,\forall p\in\Sigma.
\end{equation}

Since no horosphere is a rotator in $\h^3$,
the considerations of Remark \ref{rem-cmcsoliton} yield

\begin{proposition}
There is no rotator of nonzero constant mean curvature in $\h^3$.
\end{proposition}


We shall seek for rotators to MCF in $\h^3$ in the class of \emph{helicoidal surfaces},
which are described as follows.
Choose a smooth curve
with trace contained in the horosphere $\mathscr H:=\R^2\times\{1\}$
of height $1:$
\[
s\in\R\mapsto (\alpha(s),1)\in\mathscr H,
\]
where $\alpha\colon\R\rightarrow\R^2$ is a regular curve parameterized by
arc length. Given a constant $h>0,$ we call a parameterized surface
$\Sigma=X(\R^2)\subset\h^3$ a \emph{helicoidal surface} generated by $\alpha$ with \emph{pitch} $h,$
if the parameterization $X:\R^2\rightarrow\h^3$ writes as
\begin{equation} \label{eq-parametrizationh3}
X(u,v)=e^{hv}(e^{vJ}\alpha(u),1), \,\,\,(u,v)\in\R^2.
\end{equation}

{Considering a parameterization of $\alpha$ by arc length,
$\alpha(s) = (u(s),v(s),0), \, s\in\R,$ and writing
$$T(s) = u'(s)\partial_x+v'(s) \partial_y,\quad
N(s) = -v'(s)\partial_x+u'(s) \partial_y,$$
the curvature of $\alpha$ is given by}
$${k(s) = \langle\alpha''(s),N(s)\rangle_e = -u''(s)v'(s)+v''(s)u'(s),}$$
{where $\langle\,,\,\rangle_e$ stands for the Euclidean metric of $\R^2.$
Furthermore, by the well known Frenet-Serret equations, one has}
$${T'= kN,\quad N'= -kT.}$$


In this setting, if we define the functions
\begin{equation} \label{eq-tau&mu}
\tau:=\langle\alpha,T\rangle_{{e}} \quad\text{and}\quad \mu:=\langle\alpha,N\rangle_{{e}},
\end{equation}
we get from a direct computation that
\begin{equation} \label{eq-euclideanNormal}
\overbar\eta=\rho(e^{vJ}N,-(\tau+h\mu)/h), \,\,\, \rho:=h(h^2+(\tau+h\mu)^2)^{-1/2},
\end{equation}
is an Euclidean unit normal to the helicoidal surface $\Sigma,$
and that its Euclidean mean curvature in this orientation is
\[
\overbar H=e^{-hv}\rho\frac{k((h^2+1)r^2+h^2)-(h\tau-\mu)}{2(h^2+(\tau+h\mu)^2)},
\]
where $r^2:=\tau^2+\mu^2.$ From this equality and \eqref{eq-MCrelation}, we have
that the hyperbolic mean curvature $H$ of $\Sigma$ is
\begin{equation} \label{eq-Hh3}
H=\frac{\rho}{h}\left(h\frac{k((h^2+1)r^2+h^2)-(h\tau-\mu)}{2(h^2+(\tau+h\mu)^2)}-(\tau+h\mu)\right).
\end{equation}

These considerations yield the following existence result,
which brings~\cite[Theorem~3.1]{halldorsson} to $\hn3$.


\begin{theorem} \label{th-prescribedHh3}
For any smooth function $\Psi\colon\R^2\rightarrow\R$ and any
constant $h>0,$ there exists a
one-parameter family of complete
helicoidal surfaces of pitch $h$ in $\h^3$
each of them with mean curvature function $H$ satisfying
$$H(X(u,v))=\Psi(\tau(u),\mu(u)),$$
where $X$ is the parameterization given
in~\eqref{eq-parametrizationh3} and $\tau$ and $\mu$
are as in~\eqref{eq-tau&mu}.
\end{theorem}

\begin{proof}
 Considering equality \eqref{eq-Hh3} for the given function $H=H(\tau,\mu)$ and solving for
 $k,$ we have that $k=k(\tau,\mu)$ is a smooth function of $(\tau,\mu)\in\R^2.$
 However, by \cite[Lemma 3.2]{halldorsson}, there exists a one-parameter family of plane curves
 $\alpha:\R\rightarrow\R^2,$ each of them
 with curvature $k.$ Therefore, for such an $\alpha,$ and for a given $h>0,$
 the helicoidal surface of $\h^3$ with pitch
 $h$ whose generating curve is $\alpha$
 has mean curvature function $H=H(\tau,\mu),$ as we wished to prove.
\end{proof}

Now, we verify the conditions under which a helicoidal surface
$\Sigma=X(\R^2)$ of $\h^3$ is a
rotator to MCF. {By \eqref{eq-parametrizationh3}--\eqref{eq-euclideanNormal},}
\begin{eqnarray*}
\langle J\pi(X),\eta(X)\rangle &=& {\langle J(e^{hv}e^{vJ}\alpha),e^{hv}\overbar\eta(X)\rangle}=
{\langle Je^{vJ}\alpha,\overbar\eta(X)\rangle_e}\\
&=& {\langle e^{vJ}J\alpha,\rho e^{vJ}N\rangle_e=\rho\langle J\alpha,N\rangle_e}\\
&=& {-\rho\langle\alpha,JN\rangle_e=\rho\langle\alpha,T\rangle_e}\\
&=& \rho\tau,
\end{eqnarray*}
which, together with~\eqref{eq-translatorH300},
implies the following result.

\begin{lemma} \label{lem-conditionrotatorH3}
A helicoidal surface $\Sigma=X(\R^2)$ of pitch $h>0$
parameterized as in~\eqref{eq-parametrizationh3}
is a rotator to MCF in $\h^3$ if and only if
its mean curvature function
$H=H(\tau,\mu)$ satisfies
\begin{equation} \label{eq-hyperbolicH02}
H=\frac{h\tau}{\sqrt{h^2+(h\mu+\tau)^2}}.
\end{equation}
\end{lemma}


In what follows, we prove the main result of this section, which provides the existence 
of complete rotators in $\h^3$ by means of helicoidal surfaces, and 
completely describe the topology of the corresponding generating curves (see Figure \ref{fig-generating-rotator}).

% Figure environment removed


\begin{theorem} \label{th-MCFh3}
For any $h>0,$ there exists a one-parameter family of complete
rotators to MCF in $\h^3$ whose elements are all helicoidal surfaces of pitch $h.$
For each such surface, the trace of the generating curve
$\alpha\colon\R\to\R^2$
consists of two unbounded properly embedded arms
centered at the point of $\alpha$ which is closest to the
origin $o\in\R^2,$ with each arm spiraling around $o.$
\end{theorem}

\begin{proof}
The existence part of the statement follows directly from
Lemma~\ref{lem-conditionrotatorH3} and
Theorem~\ref{th-prescribedHh3}.
So, it remains to prove that the generating curve $\alpha$
of any such helicoidal surface
has the asserted geometric properties.

Keeping the above notation, we first observe that,
from equalities \eqref{eq-Hh3} and \eqref{eq-hyperbolicH02},
the curvature $k$ of $\alpha$ satisfies:
\begin{equation} \label{eq-hyperbolick}
k=\frac{2(h^2+(\tau+h\mu)^2)((h+1)\tau+h\mu)+h(h\tau-\mu)}{h((h^2+1)r^2+h^2)}\cdot
\end{equation}
Also, from \eqref{eq-tau&mu} and the Frenet-Serret equations, one has
\begin{equation} \label{eq-frenet}
\tau'=1+k\mu \quad\text{and}\quad \mu'=-k\tau,
\end{equation}
which, together with \eqref{eq-hyperbolick}, yields the ODE system (see Fig. \ref{fig-phaseportraithyp})
\begin{equation}\label{eq-hypODEsystem}
\left\{
\begin{array}{ccl}
\tau' & = & \displaystyle 1+\frac{2(h^2+(\tau+h\mu)^2)((h+1)\tau\mu+h\mu^2)+h^2\tau\mu-h\mu^2}{h((h^2+1)r^2+h^2)},\\[3ex]
\mu' & = & \displaystyle -\frac{2(h^2+(\tau+h\mu)^2)((h+1)\tau^2+h\tau\mu)+h^2\tau^2-h\tau\mu}{h((h^2+1)r^2+h^2)}\cdot
\end{array}
\right.
\end{equation}

% Figure environment removed

Now, we establish the properties of $\alpha=\tau T+\mu N$ through the following
claims.

\begin{claim} \label{claim-noconstantsolutions}
The ODE system \eqref{eq-hypODEsystem} has no constant solutions, and all solutions are defined
on $\R.$
\end{claim}
\begin{proof}[Proof of Claim~\ref{claim-noconstantsolutions}]
Assume, by contradiction, that there exists a constant solution
$\psi(s)=(\tau_0,\mu_0), \, s\in\R.$ Since $\tau'=\mu'=0,$ we have
from \eqref{eq-frenet} that $k_0:=k(\tau_0,\mu_0)$ satisfies
$k_0\mu_0=-1$ and $k_0\tau_0=0,$ which yields $\tau_0=0$
and $\mu_0\ne0.$ However, from the first equation in \eqref{eq-hypODEsystem},
one has
\[
\tau'=1+\frac{2h(1+\mu_0^2)\mu_0^2-\mu_0^2}{(h^2+1)\mu_0^2+h^2}=\frac{h^2(\mu_0^2+1)+2h(1+\mu_0^2)\mu_0^2}{(h^2+1)\mu_0^2+h^2}>0,
\]
which is a contradiction. Therefore, \eqref{eq-hypODEsystem} has no constant solutions.
From this fact, and since $k=k(\tau,\mu)$ is defined on $\R^2,$ we conclude that any solution
of \eqref{eq-hypODEsystem} is defined on $\R.$
\end{proof}

\begin{claim} \label{claim-reverselimit}
Suppose that any integral curve $\psi(s):=(\tau(s),\mu(s))$
of~\eqref{eq-hypODEsystem} satisfies that the limit
$\lim_{s\to+\infty}\tau(s)$ (resp. $\lim_{s\to+\infty}\mu(s)$)
exists. Then, $\lim_{s\to-\infty}\tau(s)$ (resp.
$\lim_{s\to-\infty}\mu(s)$) also exists. Furthermore,
if there exists some $L\in[-\infty,+\infty]$ with the property
that for any integral curve
\[
\lim_{s\to+\infty}\tau(s)=L \,\,\, (\text{resp.}\,\, \lim_{s\to+\infty}\mu(s)=L),
\]
then it also holds that any integral curve satisfies
\[
\lim_{s\to-\infty}\tau(s)=-L \,\,\, (\text{resp.}\,\, \lim_{s\to-\infty}\mu(s)=-L).
\]
\end{claim}
\begin{proof}[Proof of Claim~\ref{claim-reverselimit}]
Let $\psi(s):=(\tau(s),\mu(s))$ be an integral curve
of the system \eqref{eq-hypODEsystem}. Then,
it is easily checked that $\overbar\psi(s):=-\psi(-s)$ is also
an integral curve of that system.
Setting $\overbar\psi=(\overbar\tau,\overbar\mu),$
we have that $\overbar\tau(s)=-\tau(-s)$
and $\overbar\mu(s)=-\mu(-s).$ By hypothesis,
$\lim_{s\to +\infty}\overbar\tau(s)$ exists and the first
part of the claim follows from observing that
$\lim_{s\to -\infty}\mu(s)
= -\lim_{s\to+\infty} \overbar\mu(s)$.
The remainder of proof is argued analogously and will
be omitted.
\end{proof}

\begin{claim} \label{claim-tauhasonezero}
The function $\tau$ has precisely one zero $s_0$ and
$\tau$ is negative in $(-\infty, s_0)$ and positive in
$(s_0,+\infty).$ As a consequence, the function
$r^2=\tau^2+\mu^2$ has a global minimum and satisfies
$\lim_{s\rightarrow\pm\infty}r^2=+\infty.$
\end{claim}

\begin{proof}[Proof of Claim~\ref{claim-tauhasonezero}]
First, observe that the equalities~\eqref{eq-frenet} yield
\[
(r^2)'=2(\tau\tau'+\mu\mu')=2(\tau(1+k\mu)+\mu(-k\tau))=2\tau,
\]
which implies that the zeroes of $\tau$ are the critical points
of $r^2$. Also, as seen in the first part of the proof of Claim \ref{claim-noconstantsolutions},
if $\tau(s_0)=0$ for some
$s_0,$ then $\tau'(s_0)>0,$
which gives that $\tau$ has at most one zero $s_0,$
in which case $\tau$ is
negative in $(-\infty, s_0),$ and positive in
$(s_0,+\infty).$

Next, we argue by contradiction and
assume that $\tau$ has no zeroes.
We will also assume that $\tau>0$ on $\R,$ since
the complementary case $\tau<0$ can be treated analogously.
Under this assumption, the function
$r^2$ is strictly increasing. So, there exists $\delta\ge 0$ such that
$$
\lim_{s\to -\infty}r^2(s)=\delta.
$$
In particular, since $\tau=\frac{(r^2)'}{2}$, we also have that
\begin{equation}\label{eqtaugoestozero}
\lim_{s\to -\infty}\tau(s)=0,
\end{equation}
which implies that $\mu^2\rightarrow\delta$ as $s\rightarrow-\infty.$
However, the first equality in \eqref{eq-hypODEsystem} yields
$\lim_{s\rightarrow-\infty}\tau'(s)>0,$
which contradicts~\eqref{eqtaugoestozero}, proving
that $\tau$ has exactly one
zero and that $r^2$ has only one critical point. Consequently,
both the limits of
$r^2$ as $s\rightarrow\pm\infty$ exist in $[0,+\infty]$.

To finish the proof of the claim, just note that
if either $\lim_{s\to -\infty}r^2=\delta$ or
$\lim_{s\to +\infty}r^2=\delta$ for some
$\delta>0,$ the same arguments as before
lead to a contradiction, thus
$\lim_{s\to \pm\infty}r^2(s)=+\infty$.
\end{proof}

\begin{claim} \label{claim1}
The limits of $\tau$ and $\mu$ as $s\rightarrow\pm\infty$ exist (possibly being infinite).
\end{claim}
\begin{proof}[Proof of Claim~\ref{claim1}]
First, we show that $k$ has at most one zero in $\R.$
Assume that $k(s_0)=0$ for some $s_0\in\R.$ We have from
\eqref{eq-hyperbolick} that, at $s_0,$
\begin{equation} \label{eq-forkzero}
2(h^2+(\tau+h\mu)^2)((h+1)\tau+h\mu)+h(h\tau-\mu)=0.
\end{equation}
Also, by~\eqref{eq-frenet}, $\tau'(s_0)=1$ and $\mu'(s_0)=0.$ This, together with
\eqref{eq-forkzero}, gives that, at $s_0,$
\begin{equation} \label{eq-k'}
k'=\frac{2(h+1)(h^2+(\tau+h\mu)^2)+4((h+1)\tau+h\mu)(\tau+h\mu)+h^2}{h((h^2+1)r^2+h^2)}\cdot
\end{equation}

If $\tau(s_0)\mu(s_0)\ge 0,$ we have from \eqref{eq-k'} that
$k'(s_0)>0.$ Assume then $\tau(s_0)\mu(s_0)<0$ and notice that, by~\eqref{eq-forkzero},
one has
\begin{equation} \label{eq-sign}
{\rm sign}((h+1)\tau(s_0)+h\mu(s_0))={\rm sign}(\mu(s_0)-h\tau(s_0)).
\end{equation}
If $\tau(s_0)<0<\mu(s_0),$ then both signs in~\eqref{eq-sign} are
positive. In addition,
$$\tau(s_0)+h\mu(s_0)=(h+1)\tau(s_0)+h\mu(s_0)-h\tau(s_0)>0,$$
and then \eqref{eq-k'} yields $k'(s_0)>0.$ Analogously,
$\mu(s_0)<0<\tau(s_0)$ implies $k'(s_0)>0.$

It follows from the above that $k$ has at most one zero $s_0\in\R$ and, if so,
$k$ is negative in $(-\infty, s_0)$ and positive in $(s_0,+\infty).$
Since, by Claim \ref{claim-tauhasonezero}, $\tau$ has exactly one zero,
we have that $\mu'=-k\tau$ has at most two zeros,
which implies that $\mu$ has at most two critical points. In particular,
the limits $\lim_{s\to\pm\infty}\mu(s)$ exist.

To finish the proof of the claim,
let us assume, by contradiction, that the limit of
$\tau$ as $s\rightarrow+\infty$ does not exist.
In this case, for some $\tau_0>0,$ there exists a
strictly increasing
sequence $(s_n)_{n\in\N}$ diverging to $+\infty$ and
such that (see Fig. \ref{fig-taugraph})
\[
\tau(s_n)=\tau_0 \quad\text{and}\quad \tau'(s_n)\tau'(s_{n+1})<0 \quad \forall n\in\N.
\]

% Figure environment removed

Claim~\ref{claim-tauhasonezero} implies that
$\lim r^2(s_n) = +\infty$, then we must have
$\lim\mu^2(s_n)= +\infty$.
In this case, our previous
arguments show that either
$\lim\mu(s_n)= +\infty$ or
$\lim\mu(s_n)= -\infty$.
In any case, we have from \eqref{eq-hyperbolick} that
$$\lim_{n\to +\infty} (k(s_n)\mu(s_n))=
\lim_{n\to +\infty} \frac{2h^3\mu(s_n)^4}{h(h^2+1)\mu(s_n)^2}
=+\infty.$$
In particular, for any sufficiently large $n\in\N,$
$\tau'(s_n) =1+k(s_n)\mu(s_n)>0$. This, however,
contradicts the fact that $(\tau'(s_n))_{n\in\N}$
is an alternating sequence.
Therefore, $\lim_{s\to+\infty}\tau(s)$ exists.
Since $(\tau,\mu)$ is an arbitrary integral curve
of~\eqref{eq-hypODEsystem}, Claim~\ref{claim-reverselimit}
implies that $\lim_{s\to-\infty}\tau(s)$ also exists,
thereby finishing the proof of the claim.
\end{proof}


\begin{claim} \label{claim-taumulimitsH3}
$\lim_{s\to \pm\infty}\tau(s)=\pm\infty$ and\, $\lim_{s\to \pm\infty}\mu(s)=\mp\infty.$
\end{claim}
\begin{proof}[Proof of Claim~\ref{claim-taumulimitsH3}]
By Claim~\ref{claim1}, all the limits above exist and,
arguing by contradiction, we first treat the case
when $\lim_{s\to +\infty}\mu(s) = L\in \R$.
Under this assumption, we have from
Claims~\ref{claim-tauhasonezero} and~\ref{claim1} that
$\lim_{s\to +\infty} \tau(s)=+\infty$.
Then, it follows from the second equation
in~\eqref{eq-hypODEsystem} that
$\lim_{s\to +\infty}\mu'(s)= -\infty$, which
contradicts the assumed fact $L\in \R$.

{
Suppose now that $\lim_{s\to+\infty}\mu(s)=+\infty$.
From this assumption and Claim \ref{claim-tauhasonezero}, we have that
$h^2+(\tau(s)+h\mu(s))^2>{1/2}$ for all sufficiently large $s>0.$
Applying this last inequality to \eqref{eq-hyperbolick} yields
\[
k(s)>\frac{(h+1)\tau(s)+h\mu(s)+h(h\tau(s)-\mu(s))}{h((h^2+1)r^2+h^2)}=\frac{(h^2+h+1)\tau(s)}{h((h^2+1)r^2+h^2)}>0.
\]
However, for such values of $s,$
$\mu'(s)=-k(s)\tau(s)<0,$ which is a contradiction.
Therefore, $\lim_{s\to+\infty}\mu(s)=-\infty$.
Since $(\tau,\mu)$ is an arbitrary integral curve,
Claim~\ref{claim-reverselimit} applies to show that
$\lim_{s\to-\infty}\mu(s)=+\infty.$}

\begin{comment}
\begin{equation} \label{eq-hyperbolick}
k=\frac{2(h^2+(\tau+h\mu)^2)((h+1)\tau+h\mu)+h(h\tau-\mu)}{h((h^2+1)r^2+h^2)}\cdot
\end{equation}
Also, from \eqref{eq-tau&mu} and the Frenet equations, one has
\begin{equation} \label{eq-frenet}
\tau'=1+k\mu \quad\text{and}\quad \mu'=-k\tau,
\end{equation}
\end{comment}
{
To finish the proof, assume, by contradiction, that
$\lim_{s\to+\infty}\tau(s)$ is finite. Then, since $\lim_{s\to+\infty}\mu(s)=-\infty,$
we have from \eqref{eq-hyperbolick} that
$\lim_{s\to+\infty}k(s)=-\infty.$ From this, we have
$\lim_{s\to+\infty}\tau'(s)=\lim_{s\to+\infty}(1+k(s)\mu(s))=+\infty,$
which is a contradiction.
Therefore, {Claim~\ref{claim-tauhasonezero} gives that}
$\lim_{s\to+\infty}\tau(s)=+\infty$. Once again,
$\lim_{s\to-\infty}\tau(s)=-\infty$ follows from
Claim~\ref{claim-reverselimit}.}
\end{proof}



\begin{claim} \label{claim-tau/mulimitedH3}
The function $\nu:=-\tau/\mu$ is bounded outside of a compact interval.
\end{claim}

\begin{proof}[Proof of Claim~\ref{claim-tau/mulimitedH3}]
It follows from Claim \ref{claim-taumulimitsH3} that $\nu$ is well defined and positive at all points
outside of a compact interval of $\R.$
Assume by contradiction that there exists a sequence $(s_n)_{n\in\N}$ in $\R$ which
diverges to infinity, and such that $\lim\nu(s_n)=+\infty,$ i.e.,
$\lim(-\mu(s_n)/\tau(s_n))=0.$

From \eqref{eq-hyperbolick},
one has {
\[
k\tau=\frac{2(h^2+(\tau+h\mu)^2)(h+1+h\mu/\tau)+h^2-h\mu/\tau}{h((h^2+1)(1+\mu^2/\tau^2)+h^2/\tau^2)}\,\cdot
\]
}
{Hence, after passing to a subsequence, we can assume
that $k(s_n)\tau(s_n)$ is positive and bounded away from zero for all $n\in\N.$ However,}
{\[
+\infty=\lim\nu(s_n)=\lim\frac{\phantom-\tau(s_n)}{-\mu(s_n)}=\lim\frac{\phantom-\tau'(s_n)}{-\mu'(s_n)}=
\lim\left(\frac{1}{k(s_n)\tau(s_n)}+\frac{\mu(s_n)}{\tau(s_n)}\right)<+\infty,
\]}
{which is a contradiction. }

Analogously, we derive a contradiction by assuming that there exists
$s_n\rightarrow-\infty$ satisfying $\lim\nu(s_n)=+\infty.$
This proves Claim \ref{claim-tau/mulimitedH3}.
\end{proof}



In what follows, we shall denote by $\omega=\omega(s)$ the angle function of
$\alpha,$ that is,
\[
\alpha=r(\cos\omega,\sin\omega).
\]
It then follows from~\eqref{eq-frenet} that the equality
\begin{equation} \label{eq-Tandomega'}
T=\frac{\tau}{r^2}\alpha+\omega'J\alpha
\end{equation}
holds at all points where $r\ne 0.$


\begin{claim} \label{claim-infiniteangleH3}
$\omega(s)\rightarrow+\infty$ as $s\rightarrow\pm\infty.$
\end{claim}
\begin{proof}[Proof of Claim~\ref{claim-infiniteangleH3}]
Considering \eqref{eq-Tandomega'} and the equality $(r^2)'=2\tau,$ we have that
\[
r'=\frac{\tau}{r} \quad\text{and}\quad \omega'=-\frac{\mu}{r^2}\,\cdot
\]
So, given a differentiable function $\varphi=\varphi(r),$ $r\in (0,+\infty),$ one has
\begin{equation} \label{eq-derivativevarphiH3}
\frac{d\varphi}{d\omega}=\frac{d\varphi}{dr}\frac{dr}{ds}\frac{ds}{d\omega}=-r\varphi'(r)\frac{\tau}{\mu}\cdot
\end{equation}

Now, define $\varphi(r)=\log(\log r).$ Then, $\varphi(r)\rightarrow+\infty$ as $r\rightarrow+\infty$ and
\begin{equation} \label{eq-rvarphi'H3}
r\varphi'(r)=\frac1{\log r}\rightarrow 0 \,\,\, \text{as} \,\,\, r\rightarrow+\infty.
\end{equation}
Since, by Claim \ref{claim-tau/mulimitedH3}, $-\tau/\mu$ is bounded outside of a compact interval,
it follows from Claim \ref{claim-tauhasonezero}
and \eqref{eq-derivativevarphiH3}--\eqref{eq-rvarphi'H3} that ${d\varphi}/{d\omega}\rightarrow 0$
as $s\rightarrow\pm\infty.$ Consequently,
${d\omega}/d\varphi\rightarrow+\infty$ as $s\rightarrow\pm\infty,$
which proves Claim \ref{claim-infiniteangleH3}.
\end{proof}


It follows from the above that
the trace of $\alpha$ has one point $p_0$ closest to the origin
$o$ (Claim \ref{claim-tauhasonezero}),
and consists of two properly embedded arms centered at
$p_0$ (Claim~\ref{claim-taumulimitsH3}) which proceed to infinity by spiraling around $o$
(Claim~\ref{claim-infiniteangleH3}).
This finishes our proof.
\end{proof}

Let $\Sigma=X(\R^2)$ be a helicoidal surface of pitch $h$ in $\h^3$
as given in~\eqref{eq-parametrizationh3}. Consider the subgroup
$\mathcal G=\{\Gamma_t\,;\, t\in\R\}\subset{\rm Iso}(\h^3)$ of downward translations of
constant speed $h,$ i.e., $\Gamma_t(p)=e^{-ht}p,$
and notice that the Killing field on $\h^3$ determined by $\mathcal G$ is $\xi(p)=-hp.$
Now, recall that the unit normal to $\Sigma$ is $\eta=e^{hv}\bar\eta,$ with
$\bar\eta$ as in \eqref{eq-euclideanNormal}. From this, we have:
\[
\langle\xi(X),\eta\rangle=-h\rho\left(\mu-\frac{\tau+h\mu}{h}\right)=\rho\tau,
\]
so that $\Sigma$ is a $\mathcal G$-soliton if and only if
its mean curvature function is given by
\[
H=\rho\tau=\frac{h\tau}{\sqrt{h^2+(h\mu+\tau)^2}}\cdot
\]

From this last equality and Lemma \ref{lem-conditionrotatorH3}, we have the following result.

\begin{proposition}
Let $\Sigma=X(\R^2)$ be a helicoidal surface of pitch $h$ in $\h^3.$
Then, the following assertions are equivalent:
\begin{itemize}[parsep=1ex]
\item[\rm i)]$\Sigma$ is a rotator to MCF.
\item[\rm ii)] $\Sigma$ is a translator to MCF with respect to the Killing field $\xi(p)=-hp.$
\end{itemize}

\end{proposition}

\section{The Classification of Minimal Translators}\label{seclastproof}

In this section, we prove Theorem~\ref{thmconj}, which is
a classification of complete, properly immersed minimal
surfaces of $\hn3$ invariant under the 1-parameter
group $\{\Gamma_t\}_{t\in\R}$ of hyperbolic isometries of
$\hn3$ defined (in the half-space model) by
$${(x_1,x_2,x_3)\in \R^3_+\mapsto\Gamma_t(x_1,x_2,x_3) = (e^tx_1,e^tx_2,e^tx_3).}$$

We let $\ol{\hn3}$ be the topological space given by the
compactification of $\hn3$ with respect to the
so-called {\em cone topology} (as defined
in~\cite{eberloneill}) and let
$S^2(\infty)$ denote the asymptotic boundary
of $\h^3$. In the upper half space model of $\h^3,$
$S^2(\infty)$ is identified with
the one point compactification of $\R^2=\{x_3=0\}$:
\[
S^2(\infty)=\R^2\cup\{\infty\}.
\]
Along the proof, given a surface $\Sigma\subset\h^3,$
we will write $\partial_\infty\Sigma$ for the
asymptotic boundary of $\Sigma,$ that is,
$\partial_\infty\Sigma:=\overbar\S\cap S^2(\infty),$
where $\overbar\Sigma$ is the closure of $\Sigma$
in $\overbar{\h^3}$.

\begin{proof}[Proof of Theorem~\ref{thmconj}]
Consider $\alpha$ a curve in the horosphere $\mathscr H:=\{x_3 = 1\}$
and assume that $\S = \{e^t \alpha\mid t\in \R\}$ is a
complete, properly immersed minimal
surface invariant under the action of
$\{\Gamma_t\}_{t\in\R}$ and generated by $\alpha$.
For the remainder of the proof, we will assume that
$\alpha$ is parameterized by arc length over a maximal
interval $I$.

\begin{claim}\label{lem2}
Let $\theta\in [0,\pi)$. If $\alpha$ intersects
the line
$L_\theta = \{(r\cos(\theta),r\sin(\theta),1)\mid r\in \R\}$
in two (or more) points, then $\alpha = L_\theta$.
In particular, $\alpha$ is properly embedded, $I = \R$ and,
if $(0,0,1)\in \alpha$, $\Sigma$ is a vertical
plane.
\end{claim}
\begin{proof}
After a rotation, it suffices to prove
the claim for $\theta = 0$. Assume that
there are two distinct points
$p_1 = (r_1,0,1),\,p_2 = (r_2,0,1)\in \alpha$
and, arguing by contradiction,
assume that $\alpha \neq L_0$.

Consider the compact arc $a$ that $\{p_1,p_2\}$ bounds in
$\alpha$. Since $\alpha\neq L_0$, there exists a point
$\widehat{p}$ in the interior of $a$ where the second coordinate
function $x_2$ has a local maximum or a local minimum, and
we may rotate $\alpha$ once again to assume it is
a local maximum, attained at $(\widehat{x}_1,\widehat{x}_2,1)$
with $\widehat{x}_2>0$.
Let $P$ be the tilted plane of $\hn3$
that contains the line
$\{(r,\widehat{x}_2,1)\mid r\in \R\}$
and whose asymptotic boundary contains $(0,0,0)$.
Then, $P$ is an
equidistant surface to the totally geodesic plane
$\{x_2 = 0\}$ and its mean curvature vector points upwards.
Since $\S$ locally stays in the mean convex side of $P$
and intersects $P$ tangentially
along the line $\{e^t \widehat{p}\mid
t\in\R\}$, we obtain a contradiction with
the mean curvature comparison principle.
\end{proof}

Assuming that $\S$ is not a vertical plane,
we may parameterize $\alpha$ as
\begin{equation}\label{eqAlpha}
\alpha(s)=(r(s)\cos(\theta(s)),r(s)\sin(\theta(s)),1),
\end{equation}
where $s$ is the arc length of $\alpha$ and
$r(s)>0$ for all $s\in \R$.
Claim~\ref{lem2} implies that,
after a rotation in $\hn3$ (and
possibly reparameterizing on the opposite orientation)
the function $\theta$ must satisfy
$\theta'(s)\geq 0$ and
\begin{equation} \label{eq-limitangles}
\lim_{s\to -\infty}\theta(s) = 0,\quad
\lim_{s\to +\infty}\theta(s) = \theta_+ >0.
\end{equation}
In fact, $\theta_+\in(0,\pi]$. Indeed,
arguing by contradiction, assume that $\theta_+>\pi$
and choose $\theta^*\in(\pi,\theta_+)$. Then,
Claim~\ref{lem2} implies that $\alpha$ intersects
$L = L_{\theta^+-\pi}$ at most in one point,
so the fact that $(0,0,1)\not\in \alpha$ implies that
either $\theta(s)\in(0,\theta^*)$ for all $s\in I$ or
$\theta(s)\in(\theta^*-\pi,\theta_+)$ for all $s\in I$,
both situations in contradiction with~\eqref{eq-limitangles}.






\begin{claim}
$\partial_\infty\Sigma\cap\R^2$ is a
$\theta_+$-hinge{\footnote{The union of two half lines in $\R^2$
issuing from a point $p$ and making an oriented angle
$\theta\in (0,2\pi)$ will be called a
$\theta$-\emph{hinge} with \emph{vertex} $p.$}}
with vertex at
the origin $\mathbf 0:=(0,0,0).$
\end{claim}
\begin{proof}
Using the notation of~\eqref{eqAlpha}, we may parameterize
$\S$ as
$$\S = \{(e^tr(s)\cos(\theta(s)),e^tr(s)\sin(\theta(s)),e^t)
\mid t,s\in\R\}.$$
Our next
argument is to show that
$\ell_0\cup \ell_{\theta_+}\cup \{\mathbf 0\}\subset \pai\S$,
where, for $\theta \in [0,2\pi)$,
$\ell_\theta = \{(r\cos(\theta),r\sin(\theta),0)\mid r>0\}.$

Note that $\lim_{s\to \pm\infty}r(s) = +\infty$, as
$\alpha$ is properly embedded and noncompact.
Let $(s_n)_{n\in\N}$ be a sequence in $\R$
such that $s_n\to+\infty$, thus
$\lim_{n\to \infty} r(s_n) = +\infty$. For a given $r>0$,
let $t_n = \log(r/r(s_n))$ and let
$$p_n = e^{t_n}\alpha(s_n) =
\left(r\cos(\theta(s_n)),r\sin(\theta(s_n)),
\frac{r}{r(s_n)}\right)\in \S.$$
Since $r>0$ and $r(s_n)\to +\infty$,
$\lim_{n\to \infty}\frac{r}{r(s_n)} = 0$.
Furthermore,~\eqref{eq-limitangles} implies that
$\lim_{n\to \infty}\theta(s_n) = \theta_+$, and it follows
that
$$\lim_{n\to \infty} p_n =(r\cos(\theta_+),r\sin(\theta_+),0)
\in \pai \S.$$
Since $r$ is arbitrary, this gives $\ell_{\theta_+}\subset
\pai \S$. Analogously, we may prove that
$\{\mathbf 0\} \cup \ell_0\subset \pai \S$.


Next, we prove that
$\ell_0\cup \ell_{\theta_+}\cup \{\mathbf 0\} \supset(\pai\S\cap\R^2).$
Choose $\ol{p}\in\partial_\infty\S\cap\R^2$ and,
assuming that $\ol{p}\neq \mathbf0$,
write $\ol{p} = (r\cos(\theta),r\sin(\theta),0)$ for some
$r >0$ and $\theta \in[0,2\pi)$.
Let $(p_n)_{n\in\N}$ be a sequence in $\S$ such that
$p_n\to\ol{p}$, so there exist uniquely defined
$s_n,t_n\in \R$ such that
$$p_n = e^{t_n}\alpha(s_n) =
\left(
e^{t_n}r(s_n)\cos(\theta(s_n)),
e^{t_n}r(s_n)\sin(\theta(s_n)),e^{t_n}\right).
$$
The fact that $p_n\to \ol{p}$ implies that $e^{t_n}\to 0$.
Moreover,
$\lim_{n\to \infty}e^{t_n}r(s_n) = r$, thus
$r(s_n)\to +\infty$, and it follows
that either $s_n\to +\infty$ (in which case
$\theta(s_n)\to \theta_+$) or $s_n\to -\infty$ (and
$\theta(s_n)\to 0$). In both situations we obtain
$\ol{p}\in \ell_0\cup \ell_{\theta_+}$, which proves the claim.
\end{proof}
At this point, we have shown that a properly immersed minimal
surface $\S\subset\hn3$ which is invariant under the group
$\{\Gamma_t\}_{t\in\R}$ is in fact properly embedded
and its asymptotic boundary $\pai \S\cap \R^2$ is a
$\theta_+$-hinge with vertex at $\mathbf 0$.
The existence
and uniqueness of such surfaces
was proven in~\cite{GRR} (unpublished) and also presented
in~\cite[Proposition~A.1]{ST}, which finishes the proof
of Theorem~\ref{thmconj}.
\end{proof}





\begin{thebibliography}{99}


\bibitem{altschuler-wu} S. J. Altschuler, L. F. Wu: Translating surfaces of the non-parametric
mean curvature flow with prescribed contact angle. Calc. Var. Partial Differ. Equ. 2(1), (1994) 101--111.

\vt

\bibitem{andrewsetal} B. Andrews, B. Chow, C. Guenther, M. Langford: Extrinsic Geometric Flows.
Graduate studies in mathematics {\bf 206}, American Mathematical Society (2020).


\vt

\bibitem{cecil-ryan} T. Cecil, P. Ryan: Geometry of hypersurfaces. Springer Verlag (2015).

\vt


\bibitem{schulzeetal} J. Clutterbuck, O. C. Schnürer, F. Schulze: Stability of translating solutions
to mean curvature flow. Calc. Var. {\bf 29}, (2007) 281--293.

\vt

\bibitem{eberloneill} P. Eberlein and B. O'Neill: Visibility
manifolds. Pacific J. Math. {\bf 46}, (1973) 45--109.

\vt

\bibitem{GRR} M. Gomes, J. Ripoll and
L. Rodriguez:
On surfaces of constant mean curvature in hyperbolic space,
preprint 1985, IMPA.

\vt
\bibitem{halldorsson}
H. P. Halldorsson: Helicoidal surfaces rotating/translating under the mean curvature flow, Geom. Dedicata {\bf 162} (2013), 45--65.
\vt


\bibitem{huisken-polden} G. Huisken, A. Polden: Geometric evolution equations for hypersurfaces,
Calculus of variations and geometric evolution problems (Cetraro,
1996), Lecture Notes in Math., vol. {\bf 1713}, Springer, Berlin (1999), 45--84.

\vt

\bibitem{huisken-sinestrari} G. Huisken, C. Sinestrari: Convexity estimates for mean
curvature flow and singularities of mean
convex surfaces. Acta Math. {\bf 183} (1), 45--70 (1999).

\bibitem{ilmanen} T. Ilmanen: Elliptic regularization and partial regularity for motion by mean curvature. Mem. Amer. Math. Soc. 108 (1994), no. 520, {\rm x}+90 pp.

\vt

\bibitem{hungerbuhler-smoczyk} N. Hungerbühler, K. Smoczyk: Soliton solutions for the mean curvature flow. Differ Integr. Equ. 13,
(2000) 1321--1345.

\vt

\bibitem{delima} R. F. de Lima: Weingarten flows in Riemannian manifolds. To appear in
Illinois journal of mathematics. Available at https://arxiv.org/abs/2205.09566.

\vt

\bibitem{lopez} R. López: Constant mean curvature surfaces with boundary. Springer (2010).

\vt

\bibitem{lopez2} R. López: The Translating Soliton Equation.
In: Hoffmann, T., Kilian, M., Leschke, K., Martin, F. (eds) Minimal Surfaces: Integrable Systems and Visualisation.
187--216 (2021).
\vt

\bibitem{mari} L. Mari, J. D. R. Oliveira, A. Savas-Halilaj, R. S. Sena: Conformal solitons for the mean curvature flow in hyperbolic space. preprint. arXiv:2307.05088 (2023).

\bibitem{pserrin} P. Pucci, J. Serrin: The maximum principle. Progress in Nonlinear Differential Equations and
their Applications, {\bf 73}, Birkhäuser Verlag, Basel (2007). %x+235 pp. ISBN: 978-3-7643-8144-8

\vt

\bibitem{ST} R. Sa Earp and E. Toubiana:
Existence and uniqueness of minimal graphs in hyperbolic
space. Asian J. Math. {\bf 4} (2000), no. 3, 669--693.

\end{thebibliography}

\end{document}




