% This is samplepaper.tex, a sample chapter demonstrating the
% LLNCS macro package for Springer Computer Science proceedings;
% Versi
% If you use the hyperref package, please uncomment the following line
% to display URLs in blue roman font accoron 2.20 of 2017/10/04
%
\documentclass[runningheads]{llncs}
%
\usepackage{caption}
\usepackage{subcaption}
\usepackage{graphicx}
\usepackage{xcolor}
%\usepackage{epstopdf}
% Used for displaying a sample figure. If possible, figure files should
% be included in EPS format.
%ding to Springer's eBook style:
% \renewcommand\UrlFont{\color{blue}\rmfamily}
\usepackage{hyperref}


% *** MATH PACKAGES ***
%
\usepackage{amsmath}
\DeclareMathOperator\supp{supp}
% A popular package from the American Mathematical Society that provides
% many useful and powerful commands for dealing with mathematics.
%
% Note that the amsmath package sets \interdisplaylinepenalty to 10000
% thus preventing page breaks from occurring within multiline equations. Use:
%\interdisplaylinepenalty=2500
% after loading amsmath to restore such page breaks as IEEEtran.cls normally
% does. amsmath.sty is already installed on most LaTeX systems. The latest
% version and documentation can be obtained at:
% http://www.ctan.org/pkg/amsmath





% *** SPECIALIZED LIST PACKAGES ***
%
\usepackage{algorithmic}
% algorithmic.sty was written by Peter Williams and Rogerio Brito.
% This package provides an algorithmic environment fo describing algorithms.
% You can use the algorithmic environment in-text or within a figure
% environment to provide for a floating algorithm. Do NOT use the algorithm
% floating environment provided by algorithm.sty (by the same authors) or
% algorithm2e.sty (by Christophe Fiorio) as the IEEE does not use dedicated
% algorithm float types and packages that provide these will not provide
% correct IEEE style captions. The latest version and documentation of
% algorithmic.sty can be obtained at:
% http://www.ctan.org/pkg/algorithms
% Also of interest may be the (relatively newer and more customizable)
% algorithmicx.sty package by Szasz Janos:
% http://www.ctan.org/pkg/algorithmicx




% *** ALIGNMENT PACKAGES ***
%
\usepackage{array}
% Frank Mittelbach's and David Carlisle's array.sty patches and improves
% the standard LaTeX2e array and tabular environments to provide better
% appearance and additional user controls. As the default LaTeX2e table
% generation code is lacking to the point of almost being broken with
% respect to the quality of the end results, all users are strongly
% advised to use an enhanced (at the very least that provided by array.sty)
% set of table tools. array.sty is already installed on most systems. The
% latest version and documentation can be obtained at:
% http://www.ctan.org/pkg/array


% IEEEtran contains the IEEEeqnarray family of commands that can be used to
% generate multiline equations as well as matrices, tables, etc., of high
% quality.




% *** SUBFIGURE PACKAGES ***
%\ifCLASSOPTIONcompsoc
%  \usepackage[caption=false,font=normalsize,labelfont=sf,textfont=sf]{subfig}
%\else
%  \usepackage[caption=false,font=footnotesize]{subfig}
%\fi
% subfig.sty, written by Steven Douglas Cochran, is the modern replacement
% for subfigure.sty, the latter of which is no longer maintained and is
% incompatible with some LaTeX packages including fixltx2e. However,
% subfig.sty requires and automatically loads Axel Sommerfeldt's caption.sty
% which will override IEEEtran.cls' handling of captions and this will result
% in non-IEEE style figure/table captions. To prevent this problem, be sure
% and invoke subfig.sty's "caption=false" package option (available since
% subfig.sty version 1.3, 2005/06/28) as this is will preserve IEEEtran.cls
% handling of captions.
% Note that the Computer Society format requires a larger sans serif font
% than the serif footnote size font used in traditional IEEE formatting
% and thus the need to invoke different subfig.sty package options depending
% on whether compsoc mode has been enabled.
%
% The latest version and documentation of subfig.sty can be obtained at:
% http://www.ctan.org/pkg/subfig




% *** FLOAT PACKAGES ***
%
%\usepackage{fixltx2e}
% fixltx2e, the successor to the earlier fix2col.sty, was written by
% Frank Mittelbach and David Carlisle. This package corrects a few problems
% in the LaTeX2e kernel, the most notable of which is that in current
% LaTeX2e releases, the ordering of single and double column floats is not
% guaranteed to be preserved. Thus, an unpatched LaTeX2e can allow a
% single column figure to be placed prior to an earlier double column
% figure.
% Be aware that LaTeX2e kernels dated 2015 and later have fixltx2e.sty's
% corrections already built into the system in which case a warning will
% be issued if an attempt is made to load fixltx2e.sty as it is no longer
% needed.
% The latest version and documentation can be found at:
% http://www.ctan.org/pkg/fixltx2e


%\usepackage{stfloats}
% stfloats.sty was written by Sigitas Tolusis. This package gives LaTeX2e
% the ability to do double column floats at the bottom of the page as well
% as the top. (e.g., "% Figure environment removed

\comment{
% Figure environment removed

% Figure environment removed
}

In addition, we also examine attacker's reward against different defender policies to deceive attacker and protect the network. Fig. ~\ref{fig:nhcen} shows how attacker gain decreases as the number of honeypots increases and its dependence on the entry nodes. 

% Figure environment removed

\comment{
% Figure environment removed
}



In Fig. ~\ref{fig:nh} we plot the average attack reward for different defender policies on honeypot budgets. We compare the performance of our optimal allocation with the greedy allocation that always allocates honeypots in the path of highest values nodes and random policies where defender uniformly select one node to protect rather than considering network topology analysis. The analysis of Fig. ~\ref{fig:nh} illustrates that the optimal budget of honeypot in this network is three or more honeypots, as it dramatically reduces the effect of the attack. Also deploying 3 or more honeypots is very costly. 


In our 20-node network, three entry nodes are compromised at the start of the attack, so the attacker can attack using all possible existing paths in the network starting from any of the compromised entry nodes.
We also plot the attacker’s reward for a different number of compromised nodes in the network as shown in Fig. ~\ref{fig:cen} over different defender policies. Here greedy and optimal allocation produces the same magnitude result.  

Both Fig. ~\ref{fig:capesc}  and Fig. ~\ref{fig:nhcen} illustrate that deviating from Nash equilibrium and selecting some naive policies would not be optimal. Developing optimal mitigating strategies against a well-informed attacker critical for the defender to outperform naive deception policies such as random or greedy policies.

\comment{
% Figure environment removed
}


\subsection{Impact of zero-days vulnerability}
In our analysis, we find out high impact locations (zero-day vulnerabilities) for the 20-node network. We measure the impact of zero-day vulnerability. We consider two scenarios, first, when the attacker is certain that the defender does not know zero-day vulnerability. Second, the attacker is not sure whether the defender is aware of these zero-day vulnerabilities. 

We also observe some zero-day vulnerabilities increase attacker reward massively and some remain the same compared to the naive defender. It is worth mentioning that some zero-day vulnerabilities also increase attacker reward in both cases, some increase only one scenario, not both depending on the reward function.

In Table. ~\ref{table:1} we present attacker reward for different high-impact locations against the naive, optimistic, and pessimistic defender. Attacker reward against naive defender is the benchmark, attacker reward against pessimistic and optimistic defender defines how impactful that zero-days is.  

\begin{table}[!htb]
    \captionsetup{justification=centering}
    \caption{Attacker reward against naive defender, optimistic defender, pessimistic defender for top 10 edges}
    \label{table:1}
    \centering
    \small
    \begin{tabular} {|c|c|c|c|} 
        \toprule
\thead {Edge}  
    & {\thead{Naive defender}} & {\thead{Optimistic defender}} & {\thead{Pessimistic defender}} \\
    \midrule
(6, 7) & 153.43 & 401.03 & 398.70 \\
(5, 7) & 153.43 & 374.65 & 370.26 \\
(3, 7) & 153.43 & 344.39 & 345.20 \\
(16, 17) & 153.43 & 326.52 & 326.52 \\
(12, 13) & 153.43 & 325.54 & 325.55 \\
(15, 17) & 153.43 & 323.60 & 323.60 \\
(11, 13) & 153.43 & 322.42 & 322.42 \\
(11, 17) & 153.43 & 315.72 & 315.71 \\
(14, 17) & 153.43 &  313.99 & 313.99\\
(12, 17) & 153.43 &  307.24 & 307.23\\
    \bottomrule
    \end{tabular}
\end{table}



\subsubsection{Attacker reward increases: } 

Based on our study, we highlight several reasons why certain zero-day vulnerabilities cause high damage to the defender compared to others. First, if a zero-day vulnerability creates multiple attack paths to any or all target nodes, that challenges the defender base-deception policy with limited honeypots in place and hence, causes significant damage. Second, zero-day vulnerabilities that are very close to any target nodes on the attack graph empower the attacker through a shortcut and enhance her reward. Also, a combination of the first two features leads to a significant loss for the defender.  



% For some cases the reward of the attacker is slightly higher against the optimistic defender compared to the pessimistic defender. The reason is that the attacker gain due to the advantage that he does not sure whether the optimistic defender knows about zero-day vulnerabilities whereas he is sure that the pessimistic defender is imperfectly informed about zero-day vulnerabilities.



\subsubsection{Attacker reward remain same: } Interestingly, not all potential zero-day vulnerabilities cause significant damage to defender in terms of increasing attacker reward. Such zero-day vulnerabilities do not add useful actions to attacker action spaces that benefits the defender, consequently, the defender does not need to take mitigating measures for these types of vulnerabilities. Therefore, these observations benefit the defender to develop proactive defense focusing on most critical vulnerability locations.





\subsection{Mitigation:}

As detailed in Section 5, we proposed several approaches to develop  mitigating strategies against zero-day attacks. In our approach, the defender goal is to thwart the attacker's progress in the network by observing network information. We present numerical results to show the effectiveness of our mitigating approaches such as measuring proportion under various settings.
% Figure environment removed
% \vspace{-0.5cm}
%\subsubsection{Alpha:}

In Fig. ~\ref{fig:alphalp} we show the proportion of attacker capture both for the optimistic and pessimistic defender with impact and linear programming-based mitigation. Fig. ~\ref{fig:alpha} presents the result of our impact-based mitigation. In our Alpha mitigation, we place honeypot based on the high-impact location whereas random strategies choose a location uniformly to place honeypot. Mitigation effectiveness denotes the percentage of zero-day vulnerabilities defender mitigation (Alpha) prevents among all vulnerabilities. And capture proportion denotes the percentage of time an attacker is captured when exploiting a particular vulnerability.

Fig. ~\ref{fig:alpha} shows optimistic defender Alpha mitigation with one honeypot has higher mitigation effectiveness compared to random mitigation with one honeypot. On the other hand, the same strategies with 2 honeypots show a higher degree of deviation compared to the previous which denotes an increasing number of honeypots is useful but not a feasible solution.  

Fig. ~\ref{fig:lp} denotes attacker capture proportion over no, random, and LP-based honeypot mitigation both for the optimistic and pessimistic defender. No-mitigation and random mitigation are very close to each other meaning that randomly allocating honeypots will not bring any gain. After having the probability of allocating mitigating honeypot at different locations by solving a linear program explained in Section \ref{sec:mitigation}, we place honeypot on the corresponding location(s) and measure the proportion of capture the attacker increases.


%It is worth noting that the capture proportion of Fig. ~\ref{fig:alpha} and Fig. ~\ref{fig:lp} are not comparable as one represents the percentage of vulnerability capture from all zero-day vulnerabilities and another denotes the percentage of time attacker being captured when selects particular vulnerability.



%Fig. ~\ref{fig:honeypot2} actually compares different defender strategies for one and two mitigating honeypots. Increasing number of mitigating honeypot decreases attacker reward. 
\comment{
% Figure environment removed}

%Fig. ~\ref{fig:proportion_alpha} shows capture proportion of attacker over different defender strategies both for optimistic and pessimistic defender. Increased number of mitigating honeypots increased proportion of attacker being captured. 



\comment{
% Figure environment removed}

%\subsubsection{Capture-based mitigation:}
\comment{
% Figure environment removed}
%\subsubsection{Worst-based mitigation:}

%\subsubsection{Critical point analysis:}

% Figure environment removed
% \vspace{-0.5cm}

In critical point mitigation, we modify the base policy without additional honeypot to take into account the criticality of the most impactful vulnerability. Fig. ~\ref{fig:cnhcen} denotes the capture proportion over the increased number of honeypots for the defender.


Fig. ~\ref{fig:cnh} shows capture rate increase for both no-mitigation and critical point mitigation strategies at different deception budgets (numbers of honeypots in the base policy). The difference between no and critical point mitigation reduces over the increased number of honeypots, which denotes that the number of honeypots more than three is not useful for mitigation. Fig. ~\ref{fig:ccen} shows the same result compare to Fig. ~\ref{fig:cnh}. 



% Figure environment removed

Fig. ~\ref{fig:cpoint} demonstrates capture proportion on different defender mitigation strategies. Critical point mitigation and critical point mitigation with added honeypot outperform no mitigation. It is worth noting that adding one honeypot with critical point mitigation is useful as it increases the proportion of capturing the attacker.

% TODO:
% \textcolor{blue}{Interestingly, LP-based mitigation with additional honeypot (shown in Fig. ~\ref{fig:lp}) in terms of capture rate is less than the capture rate of the critical point mitigation strategy shown in Fig. ~\ref{fig:cpoint}. This may seem counter-intuitive as mitigation with additional honeypot should have higher performance compare to the mitigation without honeypot. However, LP-based mitigation as the defender regardless of its type imperfectly informed about the location where to deploy honeypot to thwart zero-day attack which is not the case in critical point mitigation. Additionally, critical point mitigation is different than any other mitigation as it modifies the base policy of the defender, we also observe its reflection as added honeypot with critical point mitigation is not helpful.} 

% % Figure environment removed


\section{Conclusion}\label{sec:conc}
% The conclusion goes here.
% https://www.overleaf.com/project/62150a05665b9f44c586713c
In this paper, we proposed a security resource allocation problem for cyber deception against reconnaissance attacks. We proposed a novel framework to assess the effectiveness of cyber deception against zero-day attacks using an attack graph. We formulated this problem as a two-player game played on an attack graph with asymmetric information assuming that part of the attack graph is unknown to the defender. We identified the critical locations that may impact the defender payoff the most if specific nodes suffer a zero-day vulnerability. The proposed analysis is limited to considering a single vulnerability at a time, and focusing on the node location. Our future work will consider a set of vulnerabilities at a time which will follow the proposed analysis while significantly increasing the action space of the game model.  

% conference papers do not normally have an appendix


% use section* for acknowledgment
% \section*{Acknowledgment}


% The authors would like to thank...



%
% ---- Bibliography ----
%
% BibTeX users should specify bibliography style 'splncs04'.
% References will then be sorted and formatted in the correct style.
%
% \bibliographystyle{splncs04}
% \bibliography{mybibliography}
% \section*{Acknowledgment}

%
% ---- Bibliography ----
%
% BibTeX users should specify bibliography style 'splncs04'.
% References will then be sorted and formatted in the correct style.
%

%\bibliographystyle{splncs04}
\bibliographystyle{unsrt}
\bibliography{reference}
\end{document}
