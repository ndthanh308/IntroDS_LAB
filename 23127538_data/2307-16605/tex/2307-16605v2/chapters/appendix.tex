\newpage
\appendix
\section{Additional Related Work}
3D Representation Learning includes point-based~\citep{PointNet, PointNet++}, voxel-based~\citep{voxelnet15}, and multiview-based methods~\citep{mvcnn15,MVTN}, \etc. Due to the sparse but geometry-informative representation, point-based methods~\citep{PointNext, PointTrans21} have become mainstream approaches in object classification~\citep{ModelNet15, ScanObjectNN19}. Voxel-based CNN methods~\citep{SyncSpecCNN17,voxelrcnn21} provide dense representation and translation invariance, achieving outstanding performance in object detection~\citep{ScanNet17} and segmentation~\citep{ShapeNetPart16,S3DIS16}. Furthermore, due to the vigorous development of attention mechanisms~\citep{AttentionIsAllYouNeed}, 3D Transformers~\citep{PointTrans21,groupfree21,voxeltransformer21} have also brought about effective representations for downstream tasks. Recently, 3D self-supervised representation learning has been widely studied. PointContrast~\citep{PointContrast20} leverages contrastive learning across different views to acquire discriminative 3D scene representations. Point-BERT~\citep{PointBERT} and Point-MAE~\citep{PointMAE} first introduce masked modeling pretraining into 3D. ACT~\citep{ACT23} pioneers cross-modal geometry understanding via 2D/language foundation models. {\scshape ReCon}~\citep{recon23} further proposes to unify generative and contrastive learning.
Facilitated by foundation vision-language models like CLIP~\citep{CLIP}, another line of works are proposed towards open-world 3D representation learning~\citep{ULIP22,OpenScene23,CLIPFO3D23,LAS3D23,PLA23, PointGCC23}.

\section{Implementation Details}\label{app:impl_detail}
\subsection{Experimental Details}
\textbf{Training Details}~
We use ShapeNetCore from ShapeNet~\citep{ShapeNet15} as the training dataset. ShapeNet is a clean set of 3D CAD models with rich annotations, including ~51K unique 3D models from 55 common object categories.
For the acquisition of multi-modal data, we follow ReCon~\citep{recon23} for multi-view rendering and utilize BLIP~\citep{blip} based on rendered images to obtain textual data.
\cref{tab:details} shows the training hyperparameters and model architecture information of each part of \vpp.
\begin{table*}[h!]
\caption{Training recipes for 3D VQGAN, Voxel Generator, Grid Smoother and Point Upsampler.}
\label{tab:details}
\begin{center}
\resizebox{\linewidth}{!}
{
\begin{tabular}{lcccc}
 \toprule
 Config & 3D VQGAN & Voxel Generator & Grid Smoother & Point Upsampler \\
 % \midrule
 \midrule
 \multicolumn{5}{c}{\textbf{Training Parameters}} \\
 \midrule
 Optimizer & Adam & AdamW & AdamW & AdamW \\
 Learning rate & 1e-4 & 1e-3 & 1e-3 & 1e-3 \\
 Weight decay & 1e-4 & 5e-2 & 5e-2 & 5e-2 \\
 Training epochs &100 & 100 & 100 & 300 \\
 Warmup epochs & - & 5 & 5 & 10 \\
 Learning rate scheduler & cosine & cosine & cosine & cosine\\
 Batch size & 32 & 128 & 128 & 128\\
 Drop path rate & - & 0.1 & 0.1 & 0.1 \\
 Input point size & 8192 & 8192 & 8192 & 1024 \\
 \midrule
 \multicolumn{5}{c}{\textbf{Model Architecture}} \\
 \midrule
 Backbone & CNN & Transformer & Transformer & Transformer \\
 Layers & 6 & 12 & 4 & 6\\
 Hidden size & 256 & 256 & 64 & 384 \\
 Heads & - & 6 & 4 & 6 \\
 Voxel resolution & 24/32 & 24/32 & 24/32 & 24/32 \\
 Point patch size & - & - & - & 32 \\
 \midrule
 GPU device & NVIDIA A100 & NVIDIA A100 & NVIDIA A100 & NVIDIA A100 \\
\bottomrule
\end{tabular}
}
\end{center}
\end{table*}


\textbf{Downstream Tasks Details}~
Following Point-E~\citep{pointe22}, we use pre-trained PointNet++ as a classifier in all ACC, FID, and IS evaluations to extract the features and calculate the accuracy of generated point clouds. 
In point cloud generation and editing, we employ 8 or 4 steps for a parallel generation. The generation task utilizes initial voxel codebooks composed entirely of \texttt{[MASK]} tokens. In editing, we extract VQGAN features from the original point cloud to initialize the voxel codebooks.
As for the transfer classification task on ScanObjectNN~\citep{ScanObjectNN19} and ModelNet40~\citep{ModelNet15}, we fully follow the previous work~\citep{PointMAE, ACT23} configuration and trained 300 epochs with the AdamW optimizer, and used the voting strategy in testing.

\subsection{Implementation Details of 3D VQGAN}\label{app:vqgan}
\input{fig/vqgan}
\input{fig/vqgan_vis}
We show the detailed training overview of 3D VQGAN in \cref{fig:vqgan}. Following the training recipe of VQGAN~\citep{vqgan21}, we use L1 loss to supervise the reconstruction of the voxel and feed the reconstructed voxel into the discriminator to ensure the generated authenticity by GAN loss. Besides the L1 loss and GAN loss, we also introduce the occupancy rate loss to make the occupancy rate of the reconstructed voxel grid similar to the ground truth so as to obtain a better reconstruction of the voxel. \cref{fig:vqgan_vis} illustrates the reconstructed results of 3D VQGAN, showcasing its impressive capabilities in the domain of 3D voxel reconstruction. As depicted in the figure, the model exhibits remarkable noise rectification capabilities. It not only reconstructs the voxels faithfully but also manages to rectify certain imperfections and artifacts present in the input data.

\section{Additional Experiments}\label{app:add_exp}
We conduct more experiments to further demonstrate the generation quality and universality of \vpp. Including diversity \& specific text-conditional generation, few-shot transfer classification, and ablation studies.

\input{fig/novel_category}
\subsection{Novel Categories Point Clouds Generation}
Besides the generation of common categories, we also conduct novel categories generation experiments to show the generalization performance of VPP in \cref{fig:novel}. Based on the learning of common categories, VPP can generalize to generate more novel category shapes, which are visually reasonable, such as \texttt{a car boat}.

\subsection{Few-Shot Transfer Classification}
We conduct few-shot experiments on the ModelNet40~\citep{ModelNet15} dataset, and the results are shown in \cref{tab:few-shot}. In the self-supervised benchmark without the use of additional modality data, \vpp\ achieves excellent performance compared to previous works.
\begin{center}
\begin{table}[t!]
    \centering
    % \setlength\tabcolsep{8pt}
    \caption{Downstream few-shot results on ModelNet40. Overall accuracy (\%) w/o voting is reported.}\label{tab:few-shot}
    \vspace{5pt}
    \resizebox{0.7\linewidth}{!}{
    \begin{tabular}{lcccc}
    \toprule[0.95pt]
    \multirow{2}{*}[-0.5ex]{Method}& \multicolumn{2}{c}{5-way} & \multicolumn{2}{c}{10-way}\\
    \cmidrule(lr){2-3}\cmidrule(lr){4-5} & 10-shot & 20-shot & 10-shot & 20-shot\\
    \midrule[0.6pt]
    \multicolumn{5}{c}{\textit{Supervised Learning Only}}\\
    \midrule[0.6pt]
    DGCNN~\citep{DGCNN} &31.6 $\pm$ 2.8 &  40.8 $\pm$ 4.6&  19.9 $\pm$  2.1& 16.9 $\pm$ 1.5\\
    OcCo~\citep{occo} &90.6 $\pm$ 2.8 & 92.5 $\pm$ 1.9 &82.9 $\pm$ 1.3 &86.5 $\pm$ 2.2\\
    \midrule[0.6pt]
    \multicolumn{5}{c}{\textit{with Self-Supervised Representation Learning}}\\
    \midrule[0.6pt]
    Transformer~\citep{AttentionIsAllYouNeed} & 87.8 $\pm$ 5.2& 93.3 $\pm$ 4.3 & 84.6 $\pm$ 5.5 & 89.4 $\pm$ 6.3\\
    OcCo~\citep{occo} & 94.0 $\pm$ 3.6& 95.9 $\pm$ 2.3 & 89.4 $\pm$ 5.1 & 92.4 $\pm$ 4.6\\
    Point-BERT~\citep{PointBERT} & 94.6 $\pm$ 3.1 & 96.3 $\pm$ 2.7 &  91.0 $\pm$ 5.4 & 92.7 $\pm$ 5.1\\
    MaskPoint~\citep{MaskPoint} & 95.0 $\pm$ 3.7 & 97.2 $\pm$ 1.7 & 91.4 $\pm$ 4.0 & 93.4 $\pm$ 3.5\\
    Point-MAE~\citep{PointMAE} & 96.3 $\pm$ 2.5&97.8 $\pm$ 1.8 & 92.6 $\pm$ 4.1 & 95.0 $\pm$ 3.0\\
    Point-M2AE~\citep{PointM2AE22} & 96.8 $\pm$ 1.8&98.3 $\pm$ 1.4 & 92.3 $\pm$ 4.5 & 95.0 $\pm$ 3.0\\
    \rowcolor{linecolor}\vpp\ (ours) & \textbf{96.9 $\pm$ 1.9} & \textbf{98.3 $\pm$ 1.5} & \textbf{93.0 $\pm$ 4.0} & \textbf{95.4 $\pm$ 3.1} \\
    \midrule[0.6pt]
    \multicolumn{5}{c}{\textit{with Pretrained Cross-Modal Teacher Representation Learning}}\\
    \midrule[0.6pt]
    ACT~\citep{ACT23} & 96.8 $\pm$ 2.3&98.0 $\pm$ 1.4 & 93.3 $\pm$ 4.0 & 95.6 $\pm$ 2.8\\
    I2P-MAE~\citep{i2pmae23} & 97.0 $\pm$ 1.8&98.3 $\pm$ 1.3 & 92.6 $\pm$ 5.0 & 95.5 $\pm$ 3.0\\
    ReCon~\citep{recon23} & 97.3 $\pm$ 1.9&98.9 $\pm$ 1.2 & 93.3 $\pm$ 3.9 & 95.8 $\pm$ 3.0\\
    \bottomrule[0.95pt]
    \end{tabular}
}
% \vspace{-10pt}
\end{table}
\end{center}

\subsection{Ablation Study}
\textbf{Training Hyper Parameter}~
\cref{fig:ablation} (a-b) shows the ablation study of the image-text features ratio and Gaussian noise strength. It can be observed that either too large or too small text-image feature ratio and noise are not conducive to the quality and diversity of 3D generation.

\begin{wrapfigure}[12]{r}{0.4\linewidth}
    \vspace{-4pt}
    \centering  
	% Figure removed
        \vspace{-10pt}
	\caption{Ablation study on CLIP backbone choices.}
	\label{fig:ablation_clip}
    \vspace{-12pt}
\end{wrapfigure}
\textbf{Inference Parameters}~
To explore the parameter dependencies of the Mask Voxel Transformer in the inference process, we study the effects of temperature and iteration steps. The results are shown in~\cref{fig:ablation} (c-d). 
It can be seen that moderate temperature achieves optimal generation classification accuracy. Higher temperature promotes model diversity, while with the increase of iteration steps, the model's FID initially decreases and then increases.

\section{Ablation study on YCBV}
\label{sec:ablation_ycbv}

In Tab.~\ref{tab:ablation_ycbv} we report the results of our ablation study on YCBV~\cite{ycbv}.
We choose the Large Marker object and train a single model on it for each modification we applied.
Each model is trained for 20 epochs on the standard training set.
For the computation of the Feature Matching Recall (FMR), we set the distance threshold $\tau_1=10$ voxels and the inlier ratio threshold $\tau_2=5$\%, to account for the different density of the scene point cloud in YCBV.
All the other settings and parameters are the same as those in our ablation study on LMO~\cite{lmo} in the main paper.

We can observe that some changes do not increment performance, but instead cause a slight drop, in particular when adapting the safety threshold to the object dimension (third row, $-0.4$) and when colour augmentation is applied (sixth row, $-$0.3).
These additions do not benefit this particular object, but are instead advantageous when averaging all the object in the dataset.

We can note that, as in the ablation study on the LMO dataset in the main paper, the most significant improvements in ADD-S AUC result from applying the safety threshold ($+$1.5), adding RGB information ($+$5.5), and using the Adam optimiser ($+$12.3).
\renewcommand{\arraystretch}{0.9}
\begin{table}%[t!]
\centering
\tabcolsep 3pt
\caption{
Ablation study on the Large Marker object of YCBV.
Performance are compared in terms of RRE [radiants] and RTE [cm] errors (the lower the better), and FMR and ADD-S AUC (shortened to ADD) scores (the higher the better).
$\Delta$ shows the improvement of each contribution in terms of ADD-S AUC with respect to the previous row.
}
\vspace{-3mm}
\resizebox{\columnwidth}{!}{%
\begin{tabular}{clrrrrr}
\toprule
& Improvements &
RRE{\color{black!50}{$\,\downarrow$}} &
RTE{\color{black!50}{$\,\downarrow$}} & 
FMR{\color{black!50}{$\,\uparrow$}} & 
ADD{\color{black!50}{$\,\uparrow$}} & 
$\Delta$ \\ 
\toprule
& Baseline & 2.0 & 4.6 & 0.00 & 77.2 & -- \\
\midrule
\multirow{2}{*}{\rotatebox{90}{Loss}} & $+$ $\tau_{NS} = 0.1 D_S$ & 2.0 & 4.2 & 0.00 & 78.7 & $+$1.5 \\
& $+$ $\tau_{NS} = 0.1 D_O$ & 2.0 & 4.3 & 0.00 & 78.3 & $-$0.4 \\
\midrule
\multirow{2}{*}{\rotatebox{90}{Arch.}} & $+$ Independent weights & 2.0 & 4.1 & 0.00 & 79.4 & $+$1.1 \\
& $+$ Add RGB information & 1.2 & 3.2 & 49.1 & 84.9 & $+$5.5 \\
\midrule
\multirow{2}{*}{\rotatebox{90}{Aug.}} & $+$ Color augmentation & 1.2 & 3.3 & 50.0 & 84.6 & $-$0.3 \\
& $+$ Random erasing & 1.2 & 3.1 & 53.4 & 85.2 & $+$0.6 \\
\midrule
\multirow{2}{*}{\rotatebox{90}{Optim.}} & $+$ SGD $\to$ Adam & 0.0 & 0.4 & 100 & 97.5 & $+$12.3 \\
& $+$ Adam $\to$ AdamW  & 0.0 & 0.4 & 100 & 97.5 & 0 \\
& $+$ Exp $\to$ Cosine & 0.0 & 0.4 & 100 & 97.4 & $-$0.1 \\\bottomrule
\end{tabular}}
\label{tab:ablation_ycbv}
\end{table}
\renewcommand{\arraystretch}{1}

\section{Additional ablation study on LMO}

We include an ablation study on the $t_\text{scale}$ hyperparameter, which is used to set the radius of the ball volume in which negative mining around a certain point is not allowed. We train on the Can object of LMO using the standard setting, and varying only $t_\text{scale}$. The results are shown in Tab.~\ref{tab:ablation_ycbv}.
We can observe that our choice of $t_\text{scale} = 0.1$ leads to the best result. When $t_\text{scale}$ is increased, many candidate points are forbidden to be used as negatives, therefore decreasing the final performance. On the other hand, a lower $t_\text{scale}$ implies negative pairs composed by points which are near in the 3D space. This reduces the performance, as similar points are forced to have different descriptors. Notably, the worst results is obtained when $t_\text{scale} = 0.1$, i.e. when no negative candidates are excluded.

\begin{table}
\tabcolsep 3pt
\caption{
Ablation study on the Can object of LMO. Performance is shown in terms of ADD-0.1 (the higher the better) in function of the hyperparameter $t_\text{scale}$.}
\centering
\resizebox{.9\columnwidth}{!}{
\begin{tabular}{c|ccccc}
    \toprule
    $t_\text{scale}$ & 0.0 & 0.01 & 0.05 & \textbf{0.1} & 0.5 \\
    ADD-0.1d & 66.55 & 91.80 & 93.79 & \textbf{93.95} & 81.28 \\
    \bottomrule
\end{tabular}
\label{tab:tscale}
}
\end{table}

\textbf{Backbone Choice}~
\cref{fig:ablation_clip} shows the selection of the CLIP model backbones during VPP training. Both ViT-B/32 and ViT/L14 achieve excellent accuracy and diversity.

\section{Limitations \& Future Works}
Despite the substantial advantages of our model in terms of generation speed and applicability, there exists a considerable gap in generation quality compared to NeRF-based models~\citep{dreamfields22,dreamfusion22,sjc22}. To address this limitation, we intend to explore the utilization of larger models and more extensive datasets for training, \eg, Objaverse~\citep{objaverse23,objaversexl23}. Furthermore, multi-modal large language models with combined comprehension and generation abilities~\citep{dreamllm23} have advanced rapidly. Incorporating VPP's 3D generation capabilities into such models represents a potential avenue for future research.

\section*{Broader Impact}
\vpp\ enables the rapid generation of high-quality 3D models in a fraction of a second. This has wide-ranging applications in fields such as gaming, virtual reality, augmented reality, and digital media production. These advances can lead to more immersive and visually stunning user experiences in entertainment and educational content. Furthermore, as a generative model, VPP may produce deceptive and malicious content and potentially impact associated employment opportunities.
