\documentclass[twocolumn]{IEEEtran}

\usepackage{amsmath,amssymb,amsfonts,cuted,amsthm}
\usepackage[final]{graphicx}
\usepackage{psfrag}
\usepackage{epsfig}
\usepackage[numbers,sort&compress]{natbib}
\usepackage{array,booktabs}
\usepackage[nolist,nohyperlinks]{acronym}
\usepackage{ucs} 
\usepackage[utf8x]{inputenc}
\usepackage[usenames,dvipsnames]{xcolor}
\usepackage{tikz}
\usepackage{tkz-tab}
\usepackage{multirow}
\usepackage{latexsym}
\usepackage{mathrsfs}
\usepackage{siunitx}
\usepackage{tikz}
\usepackage{tikz-3dplot}
\usepackage{circuitikz}
\usepackage{subfigure}
%\usepackage{caption}
%\usepackage{subcaption}
\usepackage{float}
\usepackage{arydshln}

\usetikzlibrary{calc}

\renewcommand{\bibfont}{\footnotesize}

\newtheorem{prop}{Proposition}
\newtheorem{remark}{Remark}
\newtheorem{theorem}{Theorem}
\newtheorem{lemma}[theorem]{Lemma}

\makeatletter
\def\blfootnote{\xdef\@thefnmark{}\@footnotetext}
\makeatother


\begin{acronym}
    \acro{5G}{fifth generation}
    \acro{6G}{sixth generation}
    \acro{mmWave}{millimeter wave}
    \acro{3GPP}{3rd Generation Partnership Project}
    \acro{BER}{bit error rate}
    \acro{TWDP}{two-wave with diffuse power}
    \acro{FTR}{fluctuating two-ray}
    \acro{MFTR}{multi-cluster fluctuating two-ray}
    \acro{IFTR}{independent fluctuating two-ray}
    \acro{ML}{machine learning}
    \acro{KPI}{key performance indicator}
    \acrodefplural{KPI}[KPIs]{key performance indicators}
    \acro{LoS}{line-of-sight}
    \acro{MIMO}{multiple-inputs multiple-outputs}
    \acro{GTR-V}{generalized two-ray model with von Mises phase distribution}
    \acro{SWAT}{Smart and Wireless Applications and Techonologies}
    \acro{PDF}{probability density function}
    \acrodefplural{PDF}[PDFs]{probability density functions}
    \acro{GDA}{gradient descent}
    \acro{GA}{genetic algorithms}
    \acro{CDF}{cumulative density function}
    \acrodefplural{CDF}[CDFs]{cumulative density functions}
    \acro{GoF}{goodness of fit}
    \acro{CIR}{channel impulse response}
    
    \acro{nLoS}{non line-of-sight}
    \acro{SNR}{signal-to-noise ratio}
    \acrodefplural{SNR}[SNRs]{signal-to-noise ratios}
    \acro{AWGN}{additive white Gaussian noise}
    \acro{CSI}{channel state information}
    \acro{GMGF}{generalized moment generating function}
    \acro{MGF}{moment generating function}
    \acro{RV}{random variable}
    \acro{MRT}{maximal ratio transmission}
    \acro{MRC}{maximal ratio combining}
    \acro{dR}{double-Rayleigh}
    \acro{CLT}{central limit theorem}
    \acro{i.i.d.}{independent and identically distributed}
    \acro{fdRLoS}{fluctuating double-Rayleigh with line-of-sight}
    \acro{fSOSF}{fluctuating second order scattering fading}
    \acro{dRLoS}{double-Rayleigh with line-of-sight}
    \acro{SOSF}{second order scattering fading}
    \acro{OP}{outage probability}
    \acro{MC}{Monte Carlo}
    \acro{MPC}{multipath component}
    \acrodefplural{MPC}[MPCs]{multipath components}
\end{acronym}

\begin{document}
    \title{Empirical Validation of a Class of \\Ray-Based Fading Models}
    %\author{AUTHOR 1, AUTHOR 2, AUTHOR 3, AUTHOR 4}
    \author{Juan E. Galeote-Cazorla, Alejandro Ramírez-Arroyo, F. Javier Lopez-Martinez and Juan F. Valenzuela-Valdés}
    \maketitle

    \begin{abstract}
        As new wireless standards are developed, the use of higher operation frequencies comes in hand with new use cases and propagation effects that differ from the well-established state of the art. Numerous stochastic fading models have recently emerged under the umbrella of \textit{generalized fading} conditions, to provide a fine-grain characterization of propagation channels in the mmWave and sub-THz bands. For the first time in literature, this work carries out an experimental validation of a class of such ray-based models, in a wide range of propagation conditions (anechoic, reverberation and indoor) at mmWave bands. We show that the independently fluctuating two-ray (IFTR) model has good capabilities to recreate rather dissimilar environments with high accuracy. We also put forth that the key limitations of the IFTR model arise in the presence of reduced diffuse propagation, and also due to a limited phase variability for the dominant specular components. 
        % \textcolor{red}{With the proliferation of wireless devices in recent years, there is a growing need to test the operation and functionality of these various devices in different multipath environments, ranging from line-of-sight environment to a pure Rayleigh environment. In this paper we discuss how a reverberation chamber can be used to simulate a controllable Rician radio environment for the testing of a wireless device. We show that by varying the characteristics of the reverberation chamber and/or the antenna configurations in the chamber, any desired Rician -factor.}
    \end{abstract}
    
    \begin{IEEEkeywords}
        Channel characterization, stochastic fading channels, IFTR, GTR-V, anechoic chamber, reverberation chamber, indoor environments
        %\textcolor{red}{K-factor, multipath, multipath environment, Rayleigh distribution, reverberation chamber, Rician distribution, iftr model, fitting}
    \end{IEEEkeywords}

    \blfootnote{\noindent This work has been supported by grant TED2021-129938B-I00 funded by MCIN/AEI/10.13039/501100011033 and by the European Union NextGenerationEU/PRTR. It has also been supported by Junta de Andaluc\'ia through grant EMERGIA20-00297, and in part by MCIN/AEI/10.13039/501100011033 through grants PID2020-118139RB-I00, PDC2022-133900-I00, PID2020-112545RB-C54 and TED2021-131699B-I00; and in part by the Predoctoral Grant FPU19/01251. (\textit{Corresponding author: Juan E. Galeote-Cazorla}).}
    
    \blfootnote{\noindent The authors are with the Dept. Signal Theory, Networking and Communications, Research Centre for Information and Communication Technologies (CITIC-UGR), University of Granada, 18071, Granada, Spain. F.J. L\'opez-Mart\'inez is also with the Communications and Signal Processing Lab, Telecommunication Research Institute (TELMA), Universidad de M{\'a}laga, M{\'a}laga, 29010, Spain (e-mail: juane@ugr.es; alera@ugr.es;
    fjlm@ugr.es; juanvalenzuela@ugr.es).}
    % \blfootnote{Digital Object Identifier 10.1109/XXX.2023.XXXXXXX}
    
    \section{Introduction}
Current quantum hardware is unable to carry out universal quantum computations due to the buildup of errors that occur during the computation. 
The magnitude of the individual error is currently above the value that the Threshold Theorem requires in order to kick-start quantum error correction and fault-tolerant quantum computation~\cite[Section 10.6]{nielsen_chuang_2010}. 
Although the experimentally achieved fidelity rates are promising and the error bounds are inching closer to the required threshold, we will have to work for the foreseeable future with quantum hardware with errors that build-up during the computation.  This implies that we can only do a limited number of steps before the output of the computation has become completely uncorrelated with the intended one.

For fault-tolerant quantum computing, we repeat four steps: 
1) We apply a number of single and two-qubit quantum gates, in parallel whenever possible; 
2) We perform a syndrome measurement on a subset of the qubits; 
3) We perform fast classical computations to determine which errors have occurred and how to correct them; 
and, 4) We apply correction terms based on the classical computations.
We then repeat these four steps with a next sequence of gates. 
These four steps are essential to fault-tolerant quantum computing. 


The starting point of this work is to use the four steps outlined above, not to carry out error correction and fault-tolerant computation, but to enhance short, constant-depth, {\em uncorrected} quantum circuits that perform single qubit gates and {\em nearest-neighbor} two qubit gates. 
Since in the long run we will have to implement error-correction and fault-tolerant computation anyhow, and this is done by such a four-step process, why not make other use of this architecture? Moreover, on some of the quantum hardware platforms, these operations are already in place.
Embracing this idea we naturally arrive at the question: what is the computational power of \textit{low-depth} quantum-classical circuits organized as in the four steps outlined above? 
We thus investigate circuits that execute a small, ideally constant, number of stages, where at each stage we may apply, in parallel, single qubit gates and {\em nearest-neighbor} two qubit gates, followed by measurements, followed by low-depth classical computations of which the outcome can control quantum gates in later stages. 
It is not clear, at first, whether such circuits, especially with constant depth, can do anything remotely useful. 
But we will see that this is indeed the case: many quantum computations can be done by such circuits in constant depth. 
By parallelizing quantum computations in this way, we improve the overall computational capabilities of these circuits, as we do not incur errors on qubits that are idle, simply because qubits are not idle for a very long time. 
Furthermore, reducing the depth of quantum circuits, at the cost of increasing width, allows the circuit to be run faster even if errors occur.

The first usage of such a four-step layout, not to do error correction, but to perform computations, can be found in the paradigm of measurement-based quantum computing~\cite{gottesman1999demonstrating,raussendorf2001one,jozsa2006introduction,clark2007generalised}: 
A universal form of quantum computing where a quantum state is prepared and operations are performed by measuring qubits in different bases, depending on previous measurements and intermediate measurements.

\citeauthor{PhamSvore2013} were the first to formalize the four-step protocol for performing computations~\cite{PhamSvore2013}. They included specific hardware topologies by considering two-dimensional graphs for imposing constraints on qubit interactions. In their model, they develop circuits for particularly useful multi-qubit gates, including specifying costs in the width, number of qubits, depth, number of concurrent time steps, size, and total number of non-Identity operations.
As a result, they find an algorithm that factors integers in polylogarithmic depth.
\citeauthor{Browne:2011} showed that the main tool in the work by \citeauthor{PhamSvore2013}, the fan-out gate, can also be replaced by additional log-depth classical computations in the measurement-based quantum computing setting~\cite{Browne:2011}.

More recently, \citeauthor{Cirac:2021} introduced a scheme to implement unitary operations involving quantum circuits combined with Local Operations and Classical Communication ($\mathsf{LOCC}$) channels: $\mathsf{LOCC}$-assisted quantum circuits~\cite{Cirac:2021}. Similarly to the four-step scheme we just described, they allow for a short depth circuit to be run on the qubits, followed by one round of $\mathsf{LOCC}$, in which ancilla qubits are measured and local unitaries are applied based on the measurement outcomes. They show that in this model any 1D transitionally invariant matrix-product state (MPS) with fixed bond dimension is in the same phase of matter as the trivial state. Similar ideas can be found in~\cite{TVV_NonAbelianTopologicalOrder_2022, tantivasadakarn2021long}.

In this work, we introduce a new model, called \textit{Local Alternating Quantum-Classical Computations} ($\LAQCC$). In this model we alternate between running quantum circuits (constrained by locality), ending in the measurement of a subset of qubits, and fast classical computations based on the measurement results. The outcome of the classical computations are then used to control future quantum circuits. We allow for flexibility in this model, by giving different constraints to the power of both the quantum circuits and the classical circuits as well as the number of alternations between them. 
Most attention will be given to $\LAQCC$ containing quantum circuits of constant depth, classical circuits of logarithmic depth and at most a constant number of alternations between them. 
Any circuit constructed in this model is considered to be of constant depth. 
We restrict ourselves to logarithmic depth classical computations, as this is the first natural and non-trivial extension beyond constant-depth classical computations. 
Constant-depth classical computations do however also have an equivalent constant-depth quantum implementation.

The definition of $\LAQCC$ sharpens the original definition of \citeauthor{PhamSvore2013} by adding constraints to the intermediate classical computations. This allows us to bound the power of $\LAQCC$ from above. 

The main result of \citeauthor{Cirac:2021}, that 1D translational invariant MPS with fixed bond dimension can be prepared by $\mathsf{LOCC}$-assisted circuits, relies on local symmetries of the MPS. These symmetries allow them to prepare local states (on a constant number of qubits) and glue them together by doing one round of the appropriate entangling measurement and corrections, after which they run a round of local unitaries to get the desired result. This general scheme for preparing states that exhibit an MPS description with the appropriate local symmetries requires only geometrically local unitaries and one round of measurement and corrections an therefore is accessible in $\LAQCC$. Studying different local symmetries, known as Symmetry Protected Topological (SPT) phases of matter, to find measurement-based constant depth circuits for states is a broad ongoing field of research~\cite{TVV_NonAbelianTopologicalOrder_2022, tantivasadakarn2021long, smith2023deterministic}. 
All these schemes have a $\LAQCC$ implementation.

%$\LAQCC$-circuits also exist for general schemes of preparing local states, based on the local tensors, and gluing them together using one round of entangled measurement and corrections, based on the local symmetry. 
%The main result of \citeauthor{Cirac:2021}, that 1D translational invariant MPS with fixed bond dimension can be prepared by $\mathsf{LOCC}$-assisted circuits, relies heavily on local symmetries of the MPS and as a result also has an equivalent $\LAQCC$ implementation. 
%The corrections applied after the measurement round are local unitaries depending on the local symmetries of the MPS. 

 

%This general scheme of preparing local states, based on the local tensors, and gluing it together by doing one round of entangled measurement and corrections, based on the local symmetry, is accessible in $\LAQCC$.
Note however that \citeauthor{Cirac:2021} also suggest a circuit for the $W$-state.
This circuit uses sequentially and dependent measurement-based corrections of the ancilla qubits. 
These dependent measurements translate to sequential alternations between the quantum and classical circuits and therefore increase the total depth to linear depth, exceeding the constant-depth constraints imposed by $\LAQCC$-circuits. 

We study the power of the $\LAQCC$ model with respect to state preparation, showing that even with only constant quantum-depth and logarithmic classical depth it remains possible to prepare states with long-range entanglement.
Another surprising result is that it is unlikely that $\LAQCC$ circuits are classically simulatable. We show that any instantaneous quantum polynomial-time (IQP) circuit~\cite{Bremner2010,Shepherd2009} has an $\LAQCC$ implementation.
Classical simulation of IQP circuits implies the collapse of the polynomial hierarchy to the third level, which is not believed to be true~\cite{Bremner2017}. Therefore, we expect that $\LAQCC$ circuits are unlikely to be classically simulatable. We bound the power of $\LAQCC$ by showing that it is contained in $\QNC^1$, the class of polynomial-size, log-depth circuits.

Next, we also study the power that intermediate classical calculations can add to quantum computations, by considering a new model that alternates between polynomially many polynomial-depth quantum circuits and unbounded classical computations
We study this model by doing a complexity theoretical analysis, where we draw inspiration from the notions of complexity given by \citeauthor{RosenthalYuen:2022}, \citeauthor{MetgerYuen:2023}, and \citeauthor{Aaronson:2004}.
All three complexity notions are based on the notion of state preparation, instead of more traditional definition of complexity such as the decidability of a computational problem. 
The first two consider classes based on sequences of quantum states preparable by a polynomial-sized quantum circuit, where the circuits are uniformly generated by a computational class, for instance, the class $\mathsf{PSPACE}$, which results in the complexity class $\mathsf{StatePSPACE}$~\cite{RosenthalYuen:2022,MetgerYuen:2023}.
The third notion considers a relative complexity, where the complexity is measured between two given states, and is measured by the number of gates, from a given gate-set, required to transform one state in another state~\cite{Aaronson:2004}. 
For our definition of state preparation complexity, we drop the uniformity constraint from~\cite{RosenthalYuen:2022,MetgerYuen:2023} and define a class as $\mathsf{StateX}$, which refers to states preparable by circuits of type $\mathsf{X}$. 
As an example, if $\mathsf{X} = \QNC^0$, this results in the class $\mathsf{StateQNC^0}$, which is the set of states preparable from the $\ket{0}^n$ state by poly-size constant-depth circuits. 
This notion is similar to the relative complexity from~\cite{Aaronson:2004}, where one state is the  $\ket{0}^n$ state and instead of counting the number of gates we consider the set of states preparable by a fixed number of gates. Using this notion of complexity we show that any state preparable by an $\LAQCC^*$ circuit is also preparable by a $\mathsf{PostQPoly}$ circuit, the class of circuits of polynomial depth with an additional post-selection gate. 

All Clifford circuits have a constant-depth $\LAQCC$ implementation, implying that any stabilizer state can be implemented by a constant-depth $\LAQCC$ circuit, see Section~\ref{sec:clifford_circuits} for a proof of this statement. 
Efficient circuits for stabilizer states have been known already through measurement-based quantum computing. Therefore this paper focuses on the preparation of non-stabilizer states, and as a surprising result we find novel constant-depth protocols for four very natural classes of non-stabilizer states.
Despite the extensive research into these four classes of non-stabilizer states and the many applications of them, no efficient constant- or low-depth state preparation protocols are known yet. We specifically consider these four classes as they are all often used as initial states in other algorithms.

The first state is a uniform superposition over an arbitrary number of states. 
This state finds applications in many quantum algorithms, as they often start with a uniform superposition over multiple states. 
This superposition is often achieved by applying Hadamard gates to every qubit due to its simplicity to prepare. 
Yet, the analysis of many algorithms, such as Shor's algorithm~\cite{Shor:1997}, would benefit from a different initial superposition. 
The circuit to prepare the uniform superposition over an arbitrary number of states uses an exact version of Grover search as a subroutine, that turns a probabilistic circuit, with a known constant probability of success, into a deterministic circuit. 
We use the circuit for preparing a uniform superposition over an arbitrary number of states as a subroutine in the next two quantum state preparation protocols. 

The second state is the $W$-state, the uniform superposition over all computational basis states of Hamming-weight~$1$, a natural long-ranged entangled state that displays a fundamentally nonequivalent type of entanglement from the Greenberger–Horne–Zeilinger state~\cite{WState:2000}, for which $\LAQCC$-type constant-depth circuits were previously known~\cite{PhamSvore2013, Cirac:2021}. 
The $W$-state is often used as benchmark for new quantum hardware~\cite{Haffner2005,Neeley2010,GarciaPerez:2021}. 
A novel way to prepare the $W$-state therefore gives a new way to benchmark different quantum devices with each other. 
A circuit for preparing the $W$-state was given in~\cite{Cirac:2021}, but this implementation requires sequentially alternating measurements followed by local unitaries, which in the $\LAQCC$ model is not considered to be of constant depth. 
We improve this protocol by giving an $\LAQCC$ implementation of the $W$-state, based on a compress-uncompress method that links the one-hot and binary encoding of integers.

The third state considered is the Dicke state, a generalization of the $W$-state, a superposition over all computational basis states with Hamming-weight $k$~\cite{Dicke:1954}. 
Dicke states have relevance in various practical settings.
For instance, for quantum game theory~\cite{zdemir2007}, quantum storage~\cite{Bacon_Compress:2006,Plesch:2010}, quantum error correction~\cite{ouyang2014permutation}, quantum metrology~\cite{toth2012multipartite}, and quantum networking~\cite{prevedel2009experimental}. 
Dicke states have been used as a starting state for variational optimization algorithms, most notably Quantum Alternating Operator Ansatz (QAOA)~\cite{Hadfield2019}, to find solutions to problems such as Maximum k-vertex Cover~\cite{Brandhofer2022,cook2020quantum}.
The ground states of physical Hamiltonians describing one-dimensional chains tend to show a resemblance to Dicke states such as states resulting from the Bethe ansatz, making them an ideal starting state when investigating the ground state behavior of these Hamiltonians~\cite{TDL_BetheAnsatzDerivation:2010,B_ExcitedStateQuantumPhaseTransitions:2013,DickeTransitions:2021}. 
For instance, the algorithm by \citeauthor{van2021preparing}, who give an algorithm to prepare the Bethe ansatz eigenstates of the spin-1/2 XXZ spin chain, starts by first preparing a Dicke state~\cite{van2021preparing}. 
A Dicke-state preparation protocol based on the compress-uncompress methodology used in the $W$-state furthermore finds applications in entanglement distillation, where the entanglement of a large state is concentrated on only a few qubits. 
Efficient deterministic circuits for preparing Dicke states have been proposed by \citeauthor{bartschi2019deterministic}~\cite{bartschi2019deterministic, bartschi2022deterministic_short_depth}. 
They provide a quantum circuit of depth $\mathO(k \log(\frac{n}{k}))$, allowing arbitrary connectivity, to prepare a Dicke state, which they conjecture to be optimal when $k$ is constant. 
In this work, we provide a constant-depth $\LAQCC$ circuit below their conjectured bound already for constant $k$. 
However, this does not directly disprove their conjecture, as we allow for intermediate measurements and classical computations. 
More significantly, we even construct constant-depth $\LAQCC$ circuits for $k = \mathO(\sqrt{n})$ greatly improving their bound.
This construction extends the compress-uncompress method for the $W$-state combined with additional subroutines. 

We continue with a log-depth state preparation protocol for the Dicke-state for arbitrary $k$. 
This protocol implements an efficient transformation between the factoradic number representation and the combinatorial number representation of a positive integer. 
The combinatorial number representation relates directly to the Dicke state. 
The provided efficient transformation between number representation systems might be of independent interest. 

We conclude by modifying our protocol for preparing a Dicke-state to a protocol that prepares quantum many-body scar states in constant-depth. 
These states have low entanglement and longer coherence times than states with similar energy density.
These characteristics make many-body scar states interesting to analyze and relevant within physics.
Many-body scar states appear for instance in the AKLT model~\cite{AKLT:1987,MRBAR:2018,MRB:2018} and different spin models~\cite{SI:2019,MOBFR:2020}.
Known methods for preparing these states have polynomial-depth~\cite{Gustafson:2023}, whereas our circuit has constant depth. 

% We conclude by studying the power that intermediate classical calculations can add to quantum computations. 
% In this study, we define a new model that relaxes constant-depth quantum circuits to polynomial depth quantum circuits, log-depth classical calculations to unbounded classical computations and a constant number of alternations to a polynomial number of alternations. 
% We call this model $\LAQCC^*$. 
% We study this model by doing a complexity theoretical analysis, where we draw inspiration from the notions of complexity given by \citeauthor{RosenthalYuen:2022}, \citeauthor{MetgerYuen:2023}, and \citeauthor{Aaronson:2004}.
% All three complexity notions are based on the notion of state preparation, instead of more traditional definition of complexity such as the decidability of a computational problem. 
% The first two consider classes based on sequences of quantum states preparable by a polynomial-sized quantum circuit, where the circuits are uniformly generated by a computational class, for instance, the class $\mathsf{PSPACE}$, which results in the complexity class $\mathsf{StatePSPACE}$~\cite{RosenthalYuen:2022,MetgerYuen:2023}.
% The third notion considers a relative complexity, where the complexity is measured between two given states, and is measured by the number of gates, from a given gate-set, required to transform one state in another state~\cite{Aaronson:2004}. 
% For our definition of state preparation complexity, we drop the uniformity constraint from~\cite{RosenthalYuen:2022,MetgerYuen:2023} and define a class as $\mathsf{StateX}$, which refers to states preparable by circuits of type $\mathsf{X}$. 
% As an example, if $\mathsf{X} = \QNC^0$, this results in the class $\mathsf{StateQNC^0}$, which is the set of states preparable from the $\ket{0}^n$ state by poly-size constant-depth circuits. 
% This notion is similar to the relative complexity from~\cite{Aaronson:2004}, where one state is the  $\ket{0}^n$ state and instead of counting the number of gates we consider the set of states preparable by a fixed number of gates. Using this notion of complexity we show that any state preparable by an $\LAQCC^*$ circuit is also preparable by a $\mathsf{PostQPoly}$ circuit, the class of circuits of polynomial depth with an additional post-selection gate. 

\paragraph{Summary of results}
\begin{itemize}
    \item We give a new definition of a computational model that captures the power of the four step process: applying a constant number of layers of one- and two-qubit gates; performing a syndrome measurement; perform a fast classical computation determining corrections; apply corrections. We call this model \emph{Local Alternating Quantum Classical Computations}, or $\LAQCC$ for short. In this model we bound the allowed quantum operations, intermediate classical calculations, and number of rounds separately. In Section~\ref{sec:LAQCC_model} we define this model and give a list of operations based on results from literature contained in this computational model. In some of these operations we explicitly use that we allow for multiple, but at most constant, rounds  of corrections.
    \item  We show show that there exist $\LAQCC$ circuits that can not be weakly simulated in Section~\ref{sec:IQP_in_LAQCC}. We further show that for every $\LAQCC$ circuit there exists a $\QNC^1$ circuit simulating it perfectly, in Section~\ref{sec:LAQCC_in_QNC1}.
    \item We introduce a new type computational complexity for preparing states and show that the extension of $\LAQCC$ where we allow a polynomial number of rounds and unbounded classical computation, is contained in $\mathsf{PostQPoly}$, the class of polynomial circuits with post-selection, in Section~\ref{sec:Complexity results}.
    \item We show a protocol to prepare the uniform superposition state of size $q$ in $\LAQCC$ using $\mathO(\ceil{\log_2(q)}^2)$ qubits in Section~\ref{sec:superposition_modulo_q}. 
    \item We show a protocol to prepare the $W_n$ state in $\LAQCC$ using $\mathO(n\log(n))$ qubits in Section~\ref{sec:W_state_in_LAQCC}.
    \item We show two ways of preparing the Dicke-$(n,k)$ state. The first method is in $\LAQCC$, works up to $k = \mathO(\sqrt{n})$, uses $\mathO(n^2\log(n))$ qubits, and is found in Section~\ref{sec:dicke:small_k}. The second method is in $\LAQCC\text{-}\mathsf{LOG}$ (an extension of $\LAQCC$ allowing for logarithmic number of alterations instead of constant), works for any $k$, uses $\mathO(\text{poly}(n))$ qubits, and is found in Section~\ref{sec:Dicke_in_LAQCC_LOG}. 
    \item We extend on our $\LAQCC$ method of generating Dicke-$(n,k)$ states for $k = \mathO(\sqrt{n})$ and show a protocol to generate many-body scar states for a particular Hamiltonian in $\LAQCC$ (Section~\ref{sec:many_body_scar}). 
\end{itemize}
Summarized in a table, we provide the following state generation protocols:
\begin{table}[htb]
\centering
\begin{tabular}{l|l|l|l}
\textbf{State description} & \textbf{Width} & \textbf{Depth} & \textbf{Implementation}\\
\hline 
Uniform superposition mod $q$: $\frac{1}{\sqrt{q}} \sum_{i = 0}^{q-1}\ket{i}$ & $\mathO(\ceil{\log^2 q})$ & $\mathO(1)$ & Section~\ref{sec:superposition_modulo_q}\\

$W$-state: $\frac{1}{\sqrt{n}}\sum_{i = 0}^{n-1}\ket{e_i}$ & $\mathO(n \log n)$ & $\mathO(1)$ & Section~\ref{sec:W_state_in_LAQCC}\\

Dicke-$(n,k)$, $k = \mathO(\sqrt{n})$: $\binom{n}{k}^{-1/2}\sum_{x \in \{0,1\}^n: |x| = k} \ket{x}$ &  $\mathO(n^2\log n)$ & $\mathO(1)$ 
&Section~\ref{sec:dicke:small_k}\\

Dicke-$(n,k)$: $\binom{n}{k}^{-1/2}\sum_{x \in \{0,1\}^n: |x| = k} \ket{x}$ & $\mathO(\text{poly}(n))$ & $\mathO(\log n)$ &Section~\ref{sec:Dicke_in_LAQCC_LOG}\\

QMBS: $\ket{S_k} = \frac{1}{k! \sqrt{\mathcal N(n,k)}}(Q^\dagger)^k \ket{\Omega}$ &  $\mathO(n^2\log n)$ & $\mathO(1)$  &  Section~\ref{sec:many_body_scar}
\end{tabular}
\caption{Summary of state preparation protocols given in this paper.}
\label{tab:sate_prep}
\end{table}
In the entry for the quantum many-body scar state $Q$ denotes the raising operator and $\mathcal N(n,k)=\binom{n-k-1}{k}$. 
Section~\ref{sec:many_body_scar} will provide more details on the variables and the implementation. 

\paragraph{Organization of the paper}
\noindent We first introduce relevant preliminaries in Section~\ref{sec:preliminaries}. 
In Section~\ref{sec:LAQCC_model} we formally define the class of Local Alternating Quantum-Classical Computations ($\LAQCC$). We also show that any Clifford circuit can be implemented in constant depth $\LAQCC$ (a result based on a result from measurement-based quantum computing~\cite{jozsa2006introduction}). 
This result allows us to give many useful multi-qubit gates and routines in Section~\ref{sec:gates_created_in_LAQCC}. 
Beyond that we show that constant depth $\LAQCC$ circuits are contained in $\QNC^1$ and that any $\mathsf{IQP}$ circuit has an $\LAQCC$ implementation.
We conclude this section with an analysis of a more powerful instantiation of $\LAQCC$ and show an inclusion with respect to the class $\mathsf{PostQPoly}$, which is the class of circuits of polynomial depth with one additional post-selection gate. 
In Section~\ref{sec:state_prep_in_LAQCC} we give $\LAQCC$ circuit implementations for preparing the uniform superposition over an arbitrary number of states, the $W$-state and the Dicke state up to $k = \mathO(\sqrt{n})$. We furthermore give a log-depth circuit implementation for preparing the Dicke state for any $k$. We conclude by showing a $\LAQCC$ circuit for generating many body scar states of a particular type of Hamiltonian.


    \section{Physical Channel Model}\label{sec:IFTR}

Let us consider a general formulation for the received radio signal over a wireless channel~\cite{durgin2000theory}, given as a superposition of a number of waves
\begin{align}
\label{eq1}  
V_r =\sum_{n=1}^{M} A_n \exp\left({j\varphi_n}\right),
\end{align}
where $A_n$ and $\varphi_n$ denote their corresponding amplitudes and phases. Now, it is possible to reexpress \eqref{eq1} in the following way:
\begin{align}
\label{eq2}  
V_r =\sum_{n=1}^{N} A_n \exp\left({j\varphi_n}\right) + \underbrace{\sum_{m=1}^{P} A_m \exp\left({j\varphi_m}\right)}_{Z},
\end{align}
with $M=N+P$, so that $N$ now represents a group of dominant specular waves, while $P$ indicate a group of numerous and relatively weak diffusely propagating waves. For sufficiently large $P$, the \ac{CLT} holds for the second term in \eqref{eq2}, implying that $Z\triangleq\sum_{m=1}^{P} A_m \exp\left({j\varphi_m}\right)$ can be approximated as a complex Gaussian RV with zero-mean and variance $2\sigma^2$. For $N=2$, constant-amplitude $A_n$ and uniformly distributed phases $\varphi_n$, the \ac{TWDP} model emerges \cite{Durgin2002_TWDP}. In the sequel, and taking the TWDP model as a baseline reference, we consider the general case on which $A_n=V_n\sqrt{\xi_n}$ so that $V_n$ denotes the amplitude of each of these dominant specular components, whereas $\xi_n$ are unit-mean independent Gamma random variables characterizing amplitude fluctuations. This model, referred to as independently-fluctuating two-ray (IFTR) model, was recently formulated in \cite{Olyaee2022_IFTR}, and is fully characterized by the following set of parameters:
\begin{align}
	K &= \frac{V_1^2+V_2^2}{2\sigma^2} \in [0,\infty),\\
	\Delta &= \frac{2V_1V_2}{V_1^2+V_2^2} \in [0,1],\\
	m_n\,&\{n=1,2\} \in (0,\infty),\\
        \Omega &\triangleq {\rm E\{|V_r|^2\}} \in [0,\infty),
\end{align}
where $1/m_n$ denote the fading severity (i.e., the amount of fluctuation) for each of the specular components, $K$ is the Rician factor defined as the ratio between the average powers of the dominant specular waves and the diffuse components, and $\Delta$ captures the amplitude dissimilarity between the two dominant specular waves. Finally, the scale parameter $\Omega=V_1^2+V_2^2+2\sigma^2$ represents the average received power.


    \section{Background and Problem Statement}
\label{sec:setup}
We consider the problem of an agent interacting with an SCM for $T$ rounds in order to maximize the value of a reward variable. We start by introducing SCMs, the soft intervention model used in this work, and then define the adversarial sequential decision-making problem we study. In the following, we denote with $[m]$ the set of integers $\{0, \dots, m\}$. \looseness-1

\paragraph{Structural Causal Models}
Our SCM is described by a tuple $\langle \G,  Y, \bX, \fs, \snoiserv \rangle$ of the following elements: $\G$ is a \emph{known} DAG; $Y$ is the reward variable; $\bX = {\{X_i\}_{i=0}^{m-1}}$ is a set of observed scalar random variables; the set $\fs = \{\fofi\}_{i=0}^m$ defines the \emph{unknown} functional relations between these variables; and $\snoiserv = \{\snoiserv_i \}_{i=0}^{m}$ is a set of independent noise variables with zero-mean and known distribution. % \looseness-1
 We use the notation $Y$ and $X_m$ interchangeably and assume the elements of $\bX$ are topologically ordered, i.e., $X_0$ is a root and $X_m$ is a leaf.  We denote with $\pa_i \subset \{0, \dots, m\}$ the indices of the parents of the $i$th node, and use the notation $\bZi = \{ X_j\}_{j \in \pa_i}$ for the parents this node. We sometimes use $X_i$ to refer to both the $i$th node and the $i$th random variable. \looseness-1\looseness-1

Each $X_i$ is generated according to the function $\fofi: \calZ_i \rightarrow \calX_i$, taking the parent nodes $\bZi$ of $X_i$ as input: $\si =\fofi(\zi) + \noisei$, where lowercase denotes a realization of the corresponding random variable. The reward is a scalar $x_m \in [0,1]$ while observation $X_i$ is defined over a compact set $\si \in \calX_i \subset \R$, and its parents are defined over $\calZ_i = \prod_{j \in pa_i} \calX_j$ for $i\in [m-1]$.\footnote{Here we consider scalar observations for ease of presentation, but we note that the methodology and analysis can be easily extended to vector observations as in \citet{sussex2022model}}  \looseness-1

\paragraph{Interventions}

\looseness -1 In our setup, an agent and an adversary both perform \emph{interventions} on the SCM~\footnote{Our framework allows for there to be potentially multiple adversaries, but since we consider everything from a single player's perspective, it is sufficient to combine all the other agents into a single adversary.}. 
We consider a soft intervention model \citep{eberhardt2007interventions} where interventions are parameterized by controllable \emph{action variables}. A simple example of a soft intervention is a shift intervention, where actions affect their outputs additively \citep{zhang2021matching}.

First, consider the agent and its action variables $\bm a = {\{ \ai\}_{i=0}^{m}}$. Each action $a_i$ is a real number chosen from some finite set. That is, the space $\calA_i $  of action $a_i$ is   $\calA_i \subset \R_{[0, 1]}$ where $\abs{\calA_i} = K_i$  for some $K_i \in \nN$. Let $\calA$ be the space of all actions $\bm a = {\{ \ai\}_{i=0}^{m}}$. 
% Let $\calA$ be the space of all actions $\bm a = {\{ \ai\}_{i=0}^{m}}$.
We represent the actions as additional nodes in $\G$ (see \cref{fig:overview}): $\ai$ is a parent of only $X_i$, and hence an additional input to $\fofi$. Since $\fofi$ is unknown, the agent does not know apriori the functional effect of $\ai$ on $X_i$. Not intervening on a node $X_i$ can be considered equivalent to selecting $\ai = 0$. For nodes that cannot be intervened on by our agent, we set $K_i = 1$ and do not include the action in diagrams, meaning that without loss of generality we consider the number of action variables to be equal to the number of nodes $m$.
\footnote{There may be constraints on the actions our agent can take. We refer the reader to \citet{sussex2022model} for how our setup can be extended to handle constraints.}

For the adversary we consider the same intervention model but denote their actions by $\a'$ with each $\ai'$ defined over $\calA_i' \subset \R_{[0, 1]}$ where $\abs{\calA_i'} = K_i'$ and $K_i'$ is not necessarily equal to $K_i$. 

According to the causal graph, actions $\a, \a'$ induce a realization of the graph nodes: 
\begin{align}
\label{eq:groud_truth}
& \si = \fofi(\zi, \ai, \ai') + \noisei, \ \ \forall i \in [m].
\end{align}
 
If an index $i$ corresponds to a root node, the parent vector $\zi$ denotes an empty vector, and the output of $\fofi$ only depends on the actions.

\looseness-1

\paragraph{Problem statement}
Over multiple rounds, the agent and adversary intervene simultaneously on the SCM, with known DAG $\calG$ and fixed but unknown functions $\fs = \{\fofi\}_{i=1}^m$ with $\fofi: \calZ_i \times \A_i \times \A_i' \rightarrow \calX_i$. \looseness-1
At round $t$ the agent selects actions $\at = \{\ait\}_{i=0}^m$ and obtains observations $\st = \{\sit\}_{i=0}^m$, where we add an additional subscript to denote the round of interaction. When obtaining observations, the agent also observes what actions the adversary chose $\at' = \{\ait'\}_{i=0}^m$.  We assume the adversary does not have the power to know $\at$ when selecting $\at'$, but only has access to the history of interactions until round $t$. The agent obtains a reward given by \looseness-1
\begin{align}
\label{eq:groud_truth_target}
& y_t = f_m(\bm z_{m, t}, a_{m, t}, a_{m, t}') + \noise_{m, t},
\end{align}
which implicitly depends on the whole action vector $\at$ and adversary actions $\at'$. 

The agent's goal is to select a sequence of actions that maximizes their cumulative expected reward $\sum_{t=1}^T 
r(\at, \at')$ where $r(\at, \at') = \E{y_t\mid \at, \at'}$ and expectations are taken over $\snoise$ unless otherwise stated. The challenge for the agent lies in not knowing a-priori neither the causal model (i.e., the functions $\fs = \{\fofi\}_{i=1}^m$), nor the sequence of adversarial actions $\{\at'\}_{t=1}^{\cdots}$.

\paragraph{Performance metric} 

After $T$ timesteps, we can measure the performance of the agent via the notion of regret:
\begin{align}
    R(T) = \max_{\a \in \A} \sum_{t=1}^T r(\a, \at') - \sum_{t=1}^T r(\at, \at'),
    \label{eq:regret}
\end{align}
\ie, the difference between the best cumulative expected reward obtainable by playing a single fixed action if the adversary's action sequence and $\fs$ were known in hindsight, and the agent's cumulative expected reward. We seek to design algorithms for the agent that are \emph{no-regret}, meaning that $R(T)/T \rightarrow 0$ as $T\rightarrow \infty$, for any sequence $\at'$. We emphasize that while we use the term `adversary', our regret notion encompasses all strategies that the adversary could use to select actions. This might include cooperative agents or mechanism non-stationarities. \looseness -1


 For simplicity, we consider only adversary actions observed after the agent chooses actions. Our methods can be extended to also consider adversary actions observed \emph{before} the agent chooses actions, i.e., a \textit{context}. This results in learning a policy that returns actions depending on the context, rather than just learning a fixed action. This extension is straightforward and we briefly discuss it in~\Cref{app:contextual}. \looseness-1

\textbf{Regularity assumptions} We consider standard smoothness assumptions for the unknown functions $\fofi:\mathcal{S} \rightarrow \X_i$ defined over a compact domain $\mathcal{S}$ \citep{srinivas10}. In particular, for each node $i \in [m]$, we assume that $\fofi(\cdot)$ belongs to a reproducing kernel Hilbert space (RKHS) $\mathcal{H}_{k_i}$, a space of smooth functions defined on $\calS = \calZ_i \times \calA_i \times \calA_i'$.
This means that $\fofil \in \mathcal{H}_{k_i}$ is induced by a kernel function $k_i: \calS \times  \calS \rightarrow \mathbb{R}$. 
We also assume that $k_i(s,s') \leq 1$ for every $s, s' \in \calS$\footnote{This is known as the bounded variance property, and it holds for many common kernels.}. Moreover, the RKHS norm of $\fofi(\cdot)$ is assumed to be bounded $\|\fofi\|_{k_i} \leq \mathcal{B}_i$ for some fixed constant $\mathcal{B}_i>0$.  Finally, to ensure the compactness of the domains $\Z_i$, we assume that the noise $\snoise$ is bounded, i.e., $\noisei \in \left[-1,1\right]^{d}$. \looseness-1

    \section{Experimental Results}\label{sec:results}
    \subsection{General Results}
        The basic ResSAN model is used to determine reference results which our expanded model can be compared to as it is structurally similar to ResLAN but does not possess the Lidar adaptive components of it. Further, we compare with the full-size PackNet-SAN and the unmodified NLSPN architecture. 
        As it can be seen from Tab.\,\ref{tab:sota-results}, our LiDAR-adaptive ResLAN achieves competitive performance compared to state-of-the-art standard depth completion methods, which are specialized to the unfiltered 64-beam-LiDAR. The performance differences are in the range of a few centimetres in terms of MAE, which is acceptable given the practical advantage that ResLAN can generalize to different beam patterns as will be shown below.

        Furthermore, we compared the architectures for a set of three different input types that contained 64, 32 or 16 LiDAR channels using both filter types on the metrics from the KITTI benchmark. The NLSPN model was trained for the standard depth completion task and then evaluated with different input data. As for the ResSAN models, we trained one model for each input type and tested it for the corresponding one which serve serve as the \emph{Baseline} in Tab.\,\ref{tab:overall-results}. Our ResLAN model was jointly trained for all three settings. As listed in Tab.\,\ref{tab:overall-results}, the ResLAN models outperform the challenging baseline in all metrics for FOV filtering and all but one for sparse filtering. This implies that our LiDAR adaptive model is able to outperform dedicated models in case of very sparse input depth. Fig.\,\ref{fig:comp-plot} shows this is indeed the case for 32 and even more for 16 channels. For FOV-filtered inputs with 16 channels, the ResLAN exhibits approx. $10\%$ smaller MAE than the baseline. As for the NLSPN, it becomes apparent that it is not capable of generalizing to other input types since it shows clearly worse results. The difference is especially pronounced for the FOV filtering where on average more than every fourth predicted pixel is more than $25 \%$ deviating from the ground truth\,($\delta_{1.25}$). Therefore, using a weight-adapting network in combination with differently filtered input depths allows us to train models that outperform their non-adaptive counterparts.

        \begin{table}[]
            \centering
    	    \small
            \vspace{0.4cm}
            \caption{\textbf{Depth estimation result for standard depth completion} when the ResSAN model was only trained for 64 channels and the ResLAN model for multiple tasks. The PackNet-SAN and NLSPN models were trained with the setup that was also used for our model architecture.}
            \footnotesize
            \setlength{\tabcolsep}{5pt}
            \begin{tabular}{@{}lrrrrl@{}}
            \toprule
            \multicolumn{6}{c}{\textbf{Standard LiDAR Depth Completion}}                                                                                                                         \\ \midrule
            \multicolumn{1}{l|}{Method}          & RMSE $\downarrow$            & MAE  $\downarrow$            & iRMSE $\downarrow$             & iMAE $\downarrow$ & $\delta_{1.25}$ $\uparrow$ \\
            \multicolumn{1}{l|}{}                & \multicolumn{1}{l}{{[}mm{]}} & \multicolumn{1}{l}{{[}mm{]}} & \multicolumn{1}{l}{{[}1/km{]}} & {[}1/km{]}        &                            \\ \midrule
            \multicolumn{1}{l|}{PackNet-SAN}     &  914                            &  298                            &  2.78                              &  1.4                 &  99.65 \%                          \\
            \multicolumn{1}{l|}{NLSPN}           &  \textbf{889}                            &   \textbf{263}                           &  \textbf{2.62}                              &   \textbf{1.3}                &   \textbf{99.61} \%                         \\ \midrule
            \multicolumn{1}{l|}{ResSAN (Ours)}   & 948                             &  275                            &  2.75                              &    1.4               &   99.58 \%                         \\
            \multicolumn{1}{l|}{ResLAN (Ours)} &   969                           &  283                            &   2.83                             &   1.4                &  99.56 \%                          \\ \bottomrule
            \end{tabular}
            \vspace{0.2cm}
            \label{tab:sota-results}
        \end{table}

        \begin{table}[]
    	    \centering
    	    \small
    	    \caption{\textbf{Depth estimation results of the two baseline setups and the explicit and implicit ResSAN} when evaluated on a combination of 16, 32 and 64 channel depth inputs. Please note that Specialist Methods need to train three specialized networks, one for each of the three types of inputs while our method only uses one network.}
            \footnotesize
            \setlength{\tabcolsep}{4.8pt}
            \begin{tabular}{@{}lrrrrl@{}}
                \toprule
                \multicolumn{6}{c}{\textbf{Sparse Channel Filter}}                                                                                                                                  \\ \midrule
                \multicolumn{1}{l|}{Method}        & RMSE $\downarrow$            & MAE  $\downarrow$            & iRMSE $\downarrow$             & iMAE $\downarrow$ & $\delta_{1.25}$ $\uparrow$  \\
                \multicolumn{1}{l|}{}              & \multicolumn{1}{l}{{[}mm{]}} & \multicolumn{1}{l}{{[}mm{]}} & \multicolumn{1}{l}{{[}1/km{]}} & {[}1/km{]}        &                             \\ \midrule
                \multicolumn{1}{l|}{NLSPN}         &  1396                            &  437                            & 5.54                               &  2.2                 &  98.82 \%                           \\
                \multicolumn{1}{l|}{Baseline}      & \textbf{1207}                             &  381                            & 4.41                               &  1.8                 &  \textbf{99.37} \%                           \\
                \multicolumn{1}{l|}{ResLAN (Ours)} &  1215                            &  \textbf{378}                            &  \textbf{4.27}                              &  \textbf{1.7}                 &  99.31 \%                           \\ \toprule
                \multicolumn{6}{c}{\textbf{Field-of-View Filter}}                                                                                                                                   \\ \midrule
                \multicolumn{1}{l|}{Method}        & RMSE $\downarrow$            & MAE  $\downarrow$            & iRMSE $\downarrow$             & iMAE $\downarrow$ & $\delta_{1.25}$ $\uparrow$ \\
                \multicolumn{1}{l|}{}              & \multicolumn{1}{l}{{[}mm{]}} & \multicolumn{1}{l}{{[}mm{]}} & \multicolumn{1}{l}{{[}1/km{]}} & {[}1/km{]}        &                             \\ \midrule
                \multicolumn{1}{l|}{NLSPN}         &  2738                            &  1702                            & 12.3                              &  4.3                 &  74.69 \%                           \\
                \multicolumn{1}{l|}{Baseline}      &  1556                            &  525                            &  6.8                              &  3.0                 & 98.14 \%                            \\
                \multicolumn{1}{l|}{ResLAN (Ours)} &  \textbf{1548}                            &  \textbf{519}                            &  \textbf{6.44}                              &  \textbf{2.8}                 & \textbf{98.52 \%}                            \\ \bottomrule
            \end{tabular}
            \label{tab:overall-results}
        \end{table}

        
        
        % Figure environment removed
        
        % Figure environment removed

    \subsection{Filter Effects}
        Comparing the effect of the two different types of depth input filters on the model performance, it becomes apparent that FOV filtering is the more challenging task. In that setting, reducing LiDAR channels is more detrimental to the performance than sparse filtering as it creates regions where no depth information is available. Effectively, the model is forced to perform depth prediction in these regions. These effects are highlighted in the depth images in Fig.\,\ref{fig:dense-maps} where the effect of a 16-channel sparse depth filter and a 16-channel FOV can be compared.

    \subsection{Generalization Capabilities}
        We trained three models for both filter types eaach, so the combinations and number of filtered depth inputs they receive are different. This serves the purpose of testing the generalization capabilities of the ResLAN architecture as well as the robustness to different filter settings. After training, the models were evaluated for the depth input settings they were trained for, as well as for ones they weren't exposed to. Overall, ResLAN shows good generalization capabilities. As one can gather from Fig.\,\ref{fig:explicit-comp} and Fig.\,\ref{fig:implicit-comp}, the consequences of slightly varying sets of input depth settings are limited. The most considerable deviations can be seen when the model is tasked to extrapolate. For instance, the model $\{64, 32, 16\}$ shows a noticeably higher MAE for eight-channel depth inputs than the model that was trained for it. Similar behaviour can be seen for the FOV filtering case as well for the model $\{64, 48, 32\}$ when tasked to generalize for a 16-channel input. There is no such pronounced effect for generalization tasks that lie between two filter settings the model was trained for. At most, it can be observed that models that were trained for a smaller range of filter values perform slightly better than ones that have to cover a wider range. The number of filter settings used in a fixed range does not relevantly influence the model performance, as can be seen, when comparing the two models in Fig.\,\ref{fig:implicit-comp}, which are both trained for a range of 64 to 32 channels but one with three filter settings and the other one with five.
    
    % Figure environment removed
    
    
    % Figure environment removed
    %% -*- mode: LaTeX; fill-column: 78; -*-

\section{Concluding Remarks}
\label{sec:conclusions}

In this paper, we presented a novel SMC algorithm, \EventDPOR, tailored to the
characteristics of event-driven multi-threaded programs running under the SC
semantics. The algorithm was proven correct and optimal for event-driven
programs in which the variable accesses of events do not depend on how their
execution is interleaved with other threads.

We have implemented \EventDPOR in the \Nidhugg tool, and we will open-source
our implementation.
%
With a wide range of event-driven programs, we have shown that \EventDPOR
incurs only a moderate constant overhead over its baseline implementation
(\OptimalDPOR), it is exponentially faster than existing state-of-the-art SMC
algorithms in time and number of traces examined on programs where events'
actions do not conflict, and does not suffer from performance degradation
caused by having to examine
% a significant number of
non-serializable executions.
%
%% \bjcom{Should we include:
%% Moreover, in our benchmarks, also those that are not non-branching,
%% \EventDPOR explores only the optimal number of executions, and never
%% had to resort to a potentially expensive decision procedure.}

\EventDPOR assumes that handlers can process their events in arbitrary order.
Directions for future work include to retarget \EventDPOR for event-driven
programs with other policies (e.g., FIFO), and for specific event-driven
execution models.

    \begin{thebibliography}{99}
\footnotesize

\bibitem{AbbottDahmaniPnaive2019} \textbf{C. R. Abbott and F. Dahmani}, \emph{Property $P_{naive}$ for acylindrically hyperbolic groups}, Mathematische Zeitschrift 291 (1-2 Feb. 2019), pp. 555–568.

\bibitem{AntolinCumplidoParabolic21} \textbf{Y. Antolín  and M. Cumplido}, \emph{Parabolic subgroups acting on the additional length graph}, Algebraic \& Geometric Topology 21 (4 Aug. 2021), pp. 1791–1816.

\bibitem{arnold} \textbf{V. I. Arnol’d}, \emph{Normal forms of functions near degenerate critical points, the Weyl groups $A_k$, $D_k$, $E_k$ and Lagrangian singularities}, Funkcional. Anal. i Priložen. (no. 4, 1972), pp. 3–25.

\bibitem{arnoldbook} \textbf{V. I. Arnold, S. N. Gusein-Zade and A. N. Varchenko} Singularities of differentiable maps, volume 2: Monodromy and asymptotics of integrals, vol. 2. \emph{Springer} 64 (2012): 65.

\bibitem{BainbridgeSmillieWeissHorocycle2022} \textbf{M. Bainbridge, J. Smillie and B. Weiss}, \emph{Horocycle Dynamics: New Invariants and Eigenform Loci in the Stratum $\mathcal{H}(1,1)$}, Memoirs of the American Mathematical Society 280 (1384 Nov. 2022).

\bibitem{BaumslagResidually1963} \textbf{G. Baumslag}, \emph{Automorphism Groups of Residually Finite Groups}, Journal of the London Mathematical Society s1-38 (1 1963), pp. 117–118.

%\bibitem{bestvina1999non} \textbf{M. Bestvina}, \emph{Non-positively curved aspects of Artin groups of finite type}, Geometry \& Topology 3.1 (1999), pp. 269–302.

\bibitem{BrieskornArtin1972} \textbf{E. Brieskorn and K. Saito}, \emph{Artin-Gruppen und Coxeter-Gruppen}, Inventiones Mathematicae 17 (4 Dec. 1972), pp. 245–271.

%\bibitem{bridson2010cofinitely} \textbf{M. R. Bridson}, \emph{Cofinitely Hopfian groups, open mappings and knot complements}, Groups, Geometry, and Dynamics 4.4, (2010), pp. 693-707.

\bibitem{metricbridson} \textbf{M. R. Bridson and A. Haefliger} \emph{Metric Spaces of Non-Positive Curvature}, Vol. 319. Springer Science \& Business Media, 2013.

\bibitem{CalderonConnected2020} \textbf{A. Calderon}, \emph{Connected components of strata of Abelian differentials over Teichmüller space}, Commentarii Mathematici Helvetici 95 (2 June 2020), pp. 361–420.

\bibitem{CalderonSalterFramed2022} \textbf{A. Calderon and N. Salter}, \emph{Framed mapping class groups and the monodromy of strata of abelian differentials}, Journal of the European Mathematical Society (Nov.
2022).

\bibitem{CalvezWiestAcyArt2017} \textbf{M. Calvez and B. Wiest}, \emph{Acylindrical hyperbolicity and Artin-Tits groups of spherical type}, Geometriae Dedicata 191 (1 Dec. 2017), pp. 199–215.

\bibitem{CalvezWiestCurve2017} \textbf{M. Calvez and B. Wiest}, \emph{Curve graphs and Garside groups}, Geometriae Dedicata 188 (1 June 2017), pp. 195–213.
 
\bibitem{Cohen2002} \textbf{A. M. Cohen and D. B. Wales}, \emph{Linearity of artin groups of finite type}, Israel
Journal of Mathematics 131 (1 Dec. 2002), pp. 101–123.

\bibitem{Costantini2022} \textbf{M. Costantini, M. Möller, and J. Zachhuber}, \emph{The Chern classes and Euler characteristic of the moduli spaces of Abelian differentials}, Forum of Mathematics, Pi 10 (July 2022), e16.

\bibitem{Cuadrado2021} \textbf{P. P. Cuadrado and N. Salter}, \emph{Vanishing cycles, plane curve singularities and framed
mapping class groups}, Geometry \& Topology 25 (6 2021), pp. 3179–3228.
 
\bibitem{Deligne1972} \textbf{P. Deligne}, \emph{Les immeubles des groupes de tresses généralisés}, Inventiones Mathematicae 17 (4 Dec. 1972), pp. 273–302.

\bibitem{farb2011primer} \textbf{B. Farb and D. Margalit}, A primer on mapping class groups. Vol. 41. \emph{Princeton university press}, 2011.

\bibitem{ham} \textbf{U. Hamenstädt}, \emph{On the orbifold fundamental group of the odd component of the stratum $\mathcal{H}(2,\dots,2)$} Preprint, 2020.

\bibitem{HarerZagier} \textbf{J. Harer and D. Zagier}, \emph{The Euler characteristic of the moduli space of curves} Invent. Math. 85.3 (1986), pp. 457–485.

%\bibitem{harris2013algebraic} \textbf{J. Harris}, Algebraic geometry: a first course. Vol. 133. \emph{Springer Science \& Business Media}, 2013.

\bibitem{hartshorne} \textbf{R. Hartshorne}, Algebraic geometry. \emph{Springer-Verlag, New York-Heidelberg}, 1977.

\bibitem{hatcher2002algebraic} \textbf{A. Hatcher}, Algebraic Topology. \emph{Cambridge University Press}, 2002.

%\bibitem{hirshon_1977} \textbf{R. Hirshon}, \emph{Some properties of endomorphisms in residually finite groups}, Journal of the Australian Mathematical Society 24.1 (1977), pp. 117–120.

\bibitem{humphr} \textbf{J. E. Humphreys} Reflection groups and Coxeter groups. \textit{Cambridge university press}, 1992.

\bibitem{Kontsevich1997} \textbf{M. Kontsevich and A. Zorich}, \emph{Lyapunov exponents and Hodge theory}, The mathematical beauty of physics (Saclay, 1996). Vol. 24. Adv. Ser. Math. Phys. World Sci. Publ., River Edge, NJ, 1997, pp. 318–332

\bibitem{Kontsevich2003} \textbf{M. Kontsevich and A. Zorich}, \emph{Connected components of the moduli spaces of Abelian differentials with prescribed singularities}, Inventiones Mathematicae 153 (3 Sept. 2003), pp. 631–678.

\bibitem{lek} \textbf{H. der Lek}, \emph{The homotopy type of complex hyperplane complements}, Katholieke Universiteit te Nijmegen, Ph.D. Thesis, 1983.

%\bibitem{Lonne06} \textbf{Michael Lönne}, \emph{Fundamental Groups of Spaces of Smooth Projective Hypersurfaces}, Duke Mathematical Journal 150 (2 Aug. 2006), pp. 357–405.

\bibitem{Looijenga2014} \textbf{E. Looijenga and G. Mondello}, \emph{The fine structure of the moduli space of abelian differentials in genus 3}, Geometriae Dedicata 169 (1 Apr. 2014), pp. 109–128.

\bibitem{Looijenga93} \textbf{E. Looijenga} \emph{Cohomology of $\mathcal{M}_3$ and $\mathcal{M}_3^1$}, Contemp. Math 150 (1993) pp. 205-228.

\bibitem{Maclachlan} \textbf{C. Maclachlan}, \emph{Modulus space is simply-connected}, Proc. Amer. Math. Soc. 29 (1971), pp. 85–86.

\bibitem{Masur82} \textbf{H. Masur}, \emph{Interval exchange transformations and measured foliations}, Ann. of Math. (2) 115.1 (1982), pp. 169–200.

\bibitem{McCammond2017} \textbf{J. McCammond}, \emph{The mysterious geometry of Artin groups}, Winter Braids Lecture Notes 4 (Feb. 2017), pp. 1–30.

\bibitem{Miranda} \textbf{R. Miranda}, Algebraic curves and Riemann surfaces. Vol. 5. \emph{Graduate Studies in Mathematics}, American Mathematical Society, Providence, RI, 1995.

\bibitem{Mulholland2002} \textbf{J. T. Mulholland}, \emph{Artin groups and local indicability}, arXiv preprint, arXiv:math/0606116 (2002).

\bibitem{discriminant} \textbf{P. Orlik and L. Solomon,}  \emph{Discriminants in the invariant theory of reflection groups.} Nagoya Math. J. 109 (1988): 23-45.

\bibitem{Osin2016} \textbf{D. Osin}, \emph{Acylindrically hyperbolic groups}, Transactions of the American Mathematical Society 368 (2016), pp. 851–888.

\bibitem{Perron1996} \textbf{B. Perron and J. P. Vannier}, \emph{Groupe de monodromie géométrique des singularités simples}, Mathematische Annalen 306 (1 1996), pp. 231–245.

\bibitem{Pinkham} \textbf{H. C. Pinkham}, Deformations of algebraic varieties with $\mathbb{G}_m$ action, \emph{Société Mathématique de France}, Paris, 1974, pp. i+131.

\bibitem{Shioda1993} \textbf{T. Shioda}, \emph{Plane quartics and Mordell-Weil lattices of type E7}, Comment. Math. Univ. St. Paul. 42.1 (1993), pp. 61–79.

%\bibitem{TT} \textbf{Takahashi Tadashi}, \emph{Normal forms of smooth plane quartics and their restrictions}, ScienceAsia 42S (2016), pp. 26–33.

\bibitem{Thurston} \textbf{W. P. Thurston}, \emph{On the geometry and dynamics of diffeomorphisms of surfaces}, Collected works of William P. Thurston with commentary. Vol. I. Foliations, surfaces and differential geometry. Reprint. Amer. Math. Soc., Providence, RI, 2022, pp. 495–509.

\bibitem{Veech} \textbf{W. A. Veech}, \emph{Interval exchange transformations}, J. Analyse Math. 33 (1978), pp. 222–272.

\bibitem{Wajnryb1999} \textbf{B. Wajnryb}, \emph{Artin groups and geometric monodromy}, Inventiones Mathematicae 138 (3 Dec. 1999), pp. 563–571.

\bibitem{Wright2015} \textbf{A. Wright}, \emph{Translation surfaces and their orbit closures: An introduction for a broad audience}, EMS Surveys in Mathematical Sciences 2 (1 2015), pp. 63–108.

%\bibitem{Zariski1937} \textbf{Oscar Zariski}, \emph{A Theorem on the Poincare Group of an Algebraic Hypersurface}, TheAnnals of Mathematics 38 (1 Jan. 1937), p. 131.

%\bibitem{ZorichFlat} \textbf{Anton Zorich}, \emph{Flat surfaces}, Frontiers in number theory, physics, and geometry. I. Springer, Berlin, 2006, pp. 437–583.

\bibitem{zykoski2022isodelaunay} \textbf{B. Zykoski}, \emph{The l-isodelaunay decomposition of strata of abelian differentials}, arXiv preprint, arXiv:2206.04143 (2022).


\end{thebibliography}
    
\end{document}