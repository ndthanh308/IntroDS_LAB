\section{Measurement Setup}\label{sec:Setup}
The measurements consists in the channel response obtained for three different scenarios, namely: (semi)-anechoic and (semi)-reverberation chambers, and indoor. All of them has been analyzed in a great variety of Tx-Rx configurations in the \ac{mmWave} frequency band from \num{24.25} to \SI{27.5}{\giga\hertz}, sampled with \num{651} points (that is, a \SI{5}{\mega\hertz} step). A detailed description of the acquisition and the channel environments can be consulted in \cite{RamirezArroyo2022_tSNE}.

% Figure environment removed

% Figure environment removed

% % Figure environment removed

The first two scenarios were reproduced in a half anechoic-reverberation chamber whose dimensions are $\num{5} \times \num{3.5} \times \num{3.5}$ meters (see Fig. \ref{fig:semianechoic_chamber_scheme}). The semi-anechoic side is completely covered with absorbents to suppress any possible reflection in order to receive exclusively the ray from the \ac{LoS} path. In contrast, the semi-reverberation side is totally composed by metallic walls emerging a mutipath channel behaviour due to the multiple received reflected rays. The measured Tx-Rx configurations have been analogous for both cases, with a \num{160} and \SI{600}{\centi\metre} distance between antennas respectively. On the one hand, the receiver moves in \SI{4}{\centi\metre} steps along the XZ plane, sweeping \num{11} positions in each axis. The complete composition describes a $\num{40}\times\num{40}$ centimeters square centered on the transmitter-aligned position. On the other hand, the transmitter only modifies the pointing angle and the polarization with the azimuth and roll positions. In both cases, the transmitter sweeps three different values: \ang{-30}, \num{0} and \ang{30}, taking the zero as the receiver-aligned in azimuth and roll coordinates (\ac{LoS} path and no-depolarization losses).

The indoor setup is slightly different respect the chamber ones (an illustrative figure can be found in \cite{RamirezArroyo2022_tSNE}). Transmitter was located inside the chamber, pointing to the door, while the receiver was placed outside, in the laboratory room. The movement for the latter has no changes, describing the $\num{40}\times\num{40}$ centimetres square. Nevertheless, the former modifies the azimuth variation adopting the following three angles: \ang{-30}, \ang{-15} and \num{0}; keeping the previous for roll.

All the measurements were taken at the \ac{SWAT} research group facilities, located at the University of Granada (Spain). The acquisition was performed with a Vectorial Network Analyzer (VNA Rohde \& Schwarz ZVA67), which is able to measure the scattering parameters operating up to \SI{67}{\giga\hertz}. The receiver and transmitter antennas were standardized gain horns fed with a WR-34 waveguide-to-coaxial transition (Flann K-band antenna model: \#21240-20). Finally, the transmit power was set to \SI{10}{\decibel m} in the VNA.

The described setup results in \num{1089} point-to-point measurements per environment where, in most of cases, only one ray would be relevant. For the anechoic case, all reflected rays are deeply attenuated by construction. Furthermore, at this frequency band (up to \SI{27.5}{\giga\hertz}) scattering effects have low relevance. In order to recreate a two-ray propagation channels, linear combinations are computed adding the contributions of two original measurements. The main condition to give them a real physical meaning is that receiver or transmitter position has to be locked in the same position. This leads to classify the possible configurations in two groups: "direct" and "reverse" paths respectively (see Fig. \ref{fig:complete_channel_scheme_in_reverse_path}). The former represents channels where the two-rays differs in polarization and pointing, with the same Tx-Rx physical path. For the latter, however, the physical path differs while keeping the transmitter polarization and pointing unchanged. 



The application of the exposed method results in a large amount of possible combinations. Because of that, only a reduced selection of \num{142} configurations for each measured scenario have been selected to work with, searching for significance differences between them. For a fair comparison, all of them correspond to the same Tx-Rx configurations in anechoic, reverberation and indoor scenarios. Conductive to illustrate this fact, two animations have been included in the supplementary material for the reader. Fig. \ref{fig:scenarios_diversity} shows the great diversity of channel \acp{PDF}, where rician-like and bimodal behaviours are found, these latter especially in the anechoic case. The last ones implies there are two common values of received amplitude, one associated to the mode of the distribution and the other to the next local maxima. Additionally, \ac{PDF} shapes with higher complexity such as trimodal and beyond are sometimes observed in the selected set\footnote{While the measurement set-up should correspond to a two-ray case by construction, in situations with a very reduced multipath the \ac{CLT} does not apply; hence, additional rays (e.g., those arising from reflections in the posts in Fig. 1) need to be accounted for individually. In this circumstance, the effect of these rays is translated into some additional modality \cite{Romero2022}.}.

% Figure environment removed

%, where compositions as the shown in Fig. \ref{fig:diversity_animation} are built-in them.

% % Figure environment removed

