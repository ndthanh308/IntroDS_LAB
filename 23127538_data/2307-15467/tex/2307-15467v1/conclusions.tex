\section{Conclusions}\label{sec:Conclusions}
We conducted an empirical validation of the independently-fluctuating two-ray fading model, covering a wide range of scenarios in the frequency band n258 (from \SI{24.25}{\giga\hertz} to \SI{27.5}{\giga\hertz}). We confirmed that the IFTR model is highly versatile, and is able to recreate rather dissimilar propagation conditions including anechoic, reverberation and indoor scenarios. Besides, the four shape parameters that characterize the model have a solid physical meaning, which agrees with the expected properties of each analyzed channel.

We also observed some relevant effects that put forth the limitations of the IFTR model: (\textit{i}) the assumption of uniform phases for the dominant specular components does not always hold, especially when there is a lack of phase richness. This is the case in scenarios with a reduced multipath, and is likely to become a dominant effect as we move up in frequency; (\textit{ii}) the IFTR model is not able to capture extreme bimodal behaviors observed in some of the measurements. This can be caused by the combination of a number of effects, such as the presence of additional rays (due to spurious reflections), the absence of rich multipath propagation, and the interaction between dominant specular components. In this regard, the development of physically-motivated models that capture these behaviours deserves special attention, together with additional empirical validations at higher frequencies.


\section*{Appendix}
The supplementary material is available at the IEEE DataPort repository: https://dx.doi.org/10.21227/hzw9-6q21. It consists in three different animations, where two of them present the channel diversity along the computed combinations and the other one the obtained solutions for the \ac{IFTR} fitting. The former present multitude of configurations with a representative scheme, the associated \ac{PDF} of received amplitude for each scenario and the phase difference \ac{PDF}. The latter shows the obtained solutions for the Pareto front, the distribution of the \ac{IFTR} parameters and the RMSE metric, the phase \ac{PDF} and the associated configuration scheme.