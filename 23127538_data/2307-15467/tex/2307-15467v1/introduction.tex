\section{Introduction}\label{sec:Introduction}
\IEEEPARstart{T}{}he advent and widespread deployment of the \ac{5G} of wireless technology has been key to improve the achievable data rates well-beyond tenths of {Gbps}, and also to reduce latency in one order of magnitude with respect to previous standards -- now approaching to \SI{1}{\milli\second} \cite{Rappaport2017_5G_overview}. One of the key enablers for such achievements is the use of higher frequency bands in the \ac{mmWave} range. This trend is likely to continue as \ac{5G} evolves, and \ac{6G} is often envisioned to provide coverage in the sub-THz range for some use cases \cite{Polese2020}.

To properly modeling and describing propagation channels, different approaches can be taken to capture the true (and often intractable) nature of electromagnetic effects as signals traverse the wireless environment. Classically, channel modeling strategies have been categorized as empirical vs. analytical, and also as deterministic vs. stochastic \cite{Matolak2008}. Today, state-of-the-art channel models combine the key features from these approaches to improve accuracy. For instance, COST \cite{cost1999_cost231} and 3GPP-like \cite{3GPP2022_TR38.901} models combine empirical measurements (e.g. for path-loss exponents) with geometric modeling of the propagation environment, also incorporating some stochastic features to provide randomness inherent to wireless fading effects. These models often have hundreds of parameters that allow to recreate wireless propagation with high versatility, at the expense of a higher complexity and computational burden. Recently, ray-tracing approaches have been considered as an alternative to provide a realistic channel characterization in terms of a number of \acp{MPC} \cite{Lecci2021}, at the price of an overwhelming mathematical complexity. 

To circumvent the intricacy of these approaches, analytical stochastic channel models are of widespread use due to their comparatively reduced complexity, much fewer parameters, and mathematical tractability \cite{Simon2004_rician_rayleigh}. Rayleigh and Rician models are immensely popular in the wireless community, as the de facto standards to model small-scale fading in \ac{nLoS} and \ac{LoS} cases, respectively. As new use cases promote the ubiquity of wireless devices, many new environments in sensor networks and industrial settings require the development of more advanced models\footnote{While any distribution borrowed from statistics may be used to approximate the behavior of wireless channels, only a certain class of distributions that comply with electromagnetic propagation laws will capture the true nature of communications channels.}, often referred to as \textit{generalized}. For instance, the \ac{TWDP} model \cite{Durgin2002_TWDP} proposed by Durgin, Rappaport and de Wolf describes a multipath environment with two dominant constant-amplitude specular rays, plus an aggregate diffuse component. With only one parameter addition compared to the Rician case, the \ac{TWDP} model has an increased capability to recreate not-so-common propagation conditions; these include bimodality and hyper-Rayleigh behaviour \cite{Frolik2008}. Subsequently, different generalizations and alternatives have been proposed in the literature: for instance, the \ac{FTR} model \cite{RomeroJerez2017_FTR_model} allows that the two rays of the \ac{TWDP} model fluctuate \textit{jointly} or \textit{independently} \cite{Olyaee2022_IFTR}. This philosophy is inherited from Abdi's Rician-shadowed model \cite{Abdi2003_Rician_Shadowed}, which also generalizes the Rician one. The $\kappa$-$\mu$ and the $\eta$-$\mu$ models proposed by Yacoub add additional flexibility to the Rician and Rayleigh models through the notion of multipath wave clusters. These were also later generalized by Paris' $\kappa$-$\mu$ shadowed fading model \cite{Paris2014}. 

Recently, it was shown that it is possible to bridge the gap between the models used for industry (e.g., 3GPP TR 38.901) and academia (e.g., ray based), by developing a MIMO channel model that calibrates the few parameters of the latter using the full 3GPP TR 38.901 channel model as a reference \cite{Pagin2023}. Clearly, the key to use any of the aforementioned statistical models for \ac{5G} performance evaluation in practical conditions is their empirical validation over a great variety of scenarios. One way to accomplish this is through measurement campaigns on a specific scenario; for instance, mobile radio channels over the sea at \SI{5.9}{\giga\hertz} or vehicular channels during overtakes at \SI{60}{\giga\hertz} exhibit \ac{TWDP} behaviours \cite{Zochmann2019_vehicular_TWDP,Yang2018_sea_TWDP}. However, such campaigns are rather costly in terms of time and resources, and cannot be replicated in a controlled way. Another alternative is the use of laboratory chambers combining anechoic and reverberation features. These allow to recreate Rician-fading \cite{Holloway2006_Rician_AnechoicChamber} or two-ray \cite{Frolik2009} environments, which can be exploited to emulate \ac{MIMO} channels with such particular behaviour \cite{SanchezHeredia2011_RicianMIMO}.

In this paper, we aim to empirically validate a class of ray-based stochastic fading models in a wide variety of scenarios. Specifically, we used our mixed anechoic/reverberation chamber facilities, together with indoor acquisition, to find a concordance between measurements and different propagation models in the frequency band between \num{24.25} to \SI{27.5}{\giga\hertz} (i.e., 3GPP n258 band), which belongs to the \ac{mmWave} general classification \cite{rangos_5G}. For this purpose, we consider the \ac{IFTR} fading model recently proposed in \cite{Olyaee2022_IFTR}, showing a remarkable goodness of fit in most scenarios. Interestingly, we identify that the widespread assumption of uniform phases for the two dominant specular components \cite{Durgin2002_TWDP} does not hold in some scenarios with reduced multipath. We show that in those scenarios where this behavior is identified, a more general modeling for the phases \cite{Rao2015_GTR-V} improves the fitting performance. 

The remainder of the article is organized as follows. Section \ref{sec:IFTR} summarizes the \ac{IFTR} propagation model with the relevant physical and mathematical basis. Section \ref{sec:Setup} explains how the measures were acquired and how they are used for recreating new scenarios. Section \ref{sec:Results} shows and discuss the obtained results from fitting the scenarios with the propagation models. Finally, section \ref{sec:Conclusions} summarizes the key conclusions and the future lines derived from this work.