\section{Empirical Validation and Analysis}\label{sec:Results}
\subsection{\ac{IFTR} channel fitting}
Once the Tx-Rx configurations have been chosen, a fitting between the empirical \acp{PDF} and the \ac{IFTR} theoretical one is carried out. A common way is the use of optimization numerical algorithms such as \ac{GDA} or \ac{GA} in order to minimize an objective function, corresponding to a given error metric to be minimized. In this work, a multi-objective optimization problem has been established for finding the Pareto front for the following set of error metrics\footnote{The $f_{\rm exp}$ and $f_{\rm mod}$ functions represent empirical and IFTR channel model \acp{PDF} respectively.} as targets:

\begin{itemize}
    \item Mean Squared Error:
    \begin{equation}\label{eq:mse_error}
        \mathrm{MSE} \triangleq \frac{1}{N}\sum_{i=1}^{N}(f_{\rm exp}(x_i)-f_{\rm mod}(x_i))^2.
    \end{equation}
    \item Root Mean Squared Error:
    \begin{equation}\label{eq:rmse_error}
        \mathrm{RMSE} \triangleq \sqrt{\frac{1}{N}\sum_{i=1}^{N}(f_{\rm exp}(x_i)-f_{\rm mod}(x_i))^2}.
    \end{equation}
    \item Mean Absolute Error:
    \begin{equation}\label{eq:mae_error}
        \mathrm{MAE} \triangleq \frac{1}{N}\sum_{i=1}^{N}|f_{\rm exp}(x_i)-f_{\rm mod}(x_i)|.
    \end{equation}
    \item Modified Kolmogorov-Smirnov statistic:
    \begin{equation}\label{eq:ks_error}
        \mathrm{KS} \triangleq \max_{x_i}|f_{\rm exp}(x_i)-f_{\rm mod}(x_i)|.
    \end{equation}
\end{itemize}

\noindent The use of multiple objective functions aims to find differences between the fitting results when minimizing each of the error metrics.


% Figure environment removed

\begin{table}[t]
    \centering
    \renewcommand{\arraystretch}{1.75}
    \caption{\ac{IFTR} parameters for each minimum error metric solution of the Pareto front in anechoic cofiguration A.}
    \begin{tabular}{c|cccc|}
        \multicolumn{1}{c}{} & \multicolumn{4}{c}{\textbf{Min. Error Metric Solutions (Conf. A)}} \\ \cline{2-5}
        \multicolumn{1}{c|}{\textbf{Parameter}} & \textbf{MSE} & \textbf{RMSE} & \textbf{MAE} & \textbf{KS} \\ \hline
        Error & \num{16.56} & \num{4.07} & \num{2.96} & \num{8.18} \\ \hline
        $K$ [\si{\decibel}] & \num{28.6} & \num{28.6} & \num{27.2} & \num{27.3} \\
        $\Delta$ & \num{0.44} & \num{0.44} & \num{0.44} & \num{0.44} \\
        $(m_1,m_2)$ & $(\num{35},\num{16})$ & $(\num{35},\num{16})$ & $(\num{35},\num{16})$ & $(\num{42},\num{23})$ \\ \hline
    \end{tabular}
    \label{tab:pareto_comparison_results}
\end{table}

% Figure environment removed

% Figure environment removed

To solve the optimization problem, the chosen algorithm is the NSGA-II \cite{Deb2001_multiobjective_optim}, an elitist \ac{GA} designed for working with real numbers. A population size of \num{200} and generations number of \num{400} has been set, with \SI{30}{\decibel} and \num{50} as upper limits for both $K$ and $\{m_1,m_2\}$ \ac{IFTR} parameters respectively. The elite count (number of individuals guaranteed to survive to the next generation) is a \num{5}\% of the population. Finally, as stop criterion, a number of \num{100} generations without significance changes in the solutions is taken as a convergence signal. 






After the \ac{GA} execution, a Pareto front set is obtained for each configuration per scenario. Fig. \ref{fig:pareto_comparison} shows a comparison for an arbitrary selected anechoic scenario named A between the particular solutions that minimize each metric (see Table \ref{tab:pareto_comparison_results}). It is clear that solutions produces similar curves shape, even if the Pareto point is different. In fact, a common behaviour is the presence of two well-defined solutions as in Fig. \ref{fig:pareto_front_two_sols}, where the \ac{GA} has found two local minima for $m_1$ and $m_2$. On the other hand, an unique value for $\Delta$ meanwhile the factor $K$ varies over a higher range. 




For the purpose of choosing a particular Pareto point $P$ in the form $(K,\Delta,m_1,m_2)$ as final fitting result, a weighed normalized error metric $\varepsilon_n$ is computed as follows:

\begin{equation}\label{eq:mean_normalized_error_metric}
    \varepsilon_n(P) \triangleq \left.\frac{\mathrm{MSE}_n + \mathrm{RMSE}_n + \mathrm{MAE}_n + \mathrm{KS}_n}{4}\right|_P,
\end{equation}

\noindent where $[\:\cdot\:]_n$ refers to the normalization of each error metric respect its maximum from the complete Pareto front. Then, the chosen solution is the particular point $\hat{P}$ that minimizes this $\varepsilon_n$ function:

\begin{equation}
    \hat{P} = \min_P \varepsilon_n(P).
\end{equation}



In Fig. \ref{fig:iftr_parameters_cdf} the \acp{CDF} of the optimal \ac{IFTR} parameters for each scenario are represented. The most remarkable fact is the contrast between anechoic and multipath\footnote{The term "multipath" refers, henceforth, to both reverberation and indoor propagation scenarios.} cases, with a noticeable different trend in the curves. The Rician-like $K$ factor takes higher values due to the absence of diffuse power, whereas the $\Delta$ parameter concentrates in lower values because of the lack of reflections in \ac{LoS} (the two rays are notably different). The $m_1$ parameter shows that the fluctuation of the first ray is less pronounced in the anechoic case, which is explained by the no presence of strong reflections. However, the second ray fluctuates in a similar way for the three analysed scenarios as the $m_2$ curves are quite close. The values for the multipath cases are larger than in the first, which implies that the fitting model assumes the second ray (with a lower amplitude) to be less fluctuating.

\begin{table*}[t]
    \centering
    \caption{Fitting results for the example configurations in each scenario.}
    \renewcommand{\arraystretch}{1.75}
    \begin{tabular}{c|c|cccccc|}
        \multicolumn{1}{c}{} & \multicolumn{1}{c}{} & \multicolumn{6}{c}{\textbf{Configuration}} \\ \cline{3-8}
        \multicolumn{1}{c}{\textbf{Parameter}} & \multicolumn{1}{c|}{\textbf{Scenario}} & \textbf{A} & \textbf{B} & \textbf{C} & \textbf{D} & \textbf{E} & \textbf{F} \\ \hline
        \multirow{3}{*}{$K$ [\si{\decibel}]} & Anechoic & \num{29.3} & \num{27.8} & \num{29.9} & \num{25.0} & \num{29.9} & \num{30.0} \\
        & Reverberation & \num{25.8} & \num{27.3} & \num{29.4} & \num{29.8} & \num{22.7} & \num{26.3} \\ 
        & Indoor & \num{22.8} & \num{25.6} & \num{29.9} & \num{27.1} & \num{29.9} & \num{20.6} \\ \hline
        \multirow{3}{*}{$\Delta$} & Anechoic & \num{0.44} & \num{0.09} & \num{0.23} & \num{1.00} & \num{0.45} & \num{0.21} \\
        & Reverberation & \num{0.09} & \num{0.80} & \num{0.49} & \num{0.73} & \num{0.33} & \num{0.52} \\
        & Indoor & \num{0.58} & \num{0.24} & \num{0.02} & \num{1.00} & \num{0.54} & \num{0.99} \\ \hline
        \multirow{3}{*}{$(m_1,m_2)$} & Anechoic & $(\num{35},\num{16})$ & $(\num{27},\num{13})$ & $(\num{50},\num{35})$ & $(\num{11},\num{22})$ & $(\num{36},\num{12})$ & $(\num{50},\num{14})$ \\
        & Reverberation & $(\num{12},\num{19})$ & $(\num{1},\num{18})$ & $(\num{1},\num{33})$ & $(\num{0.60},\num{0.39})$ & $(\num{15},\num{27})$ & $(\num{3},\num{32})$ \\
        & Indoor & $(\num{0.69},\num{19})$ & $(\num{0.93},\num{22})$ & $(\num{0.46},\num{15})$ & $(\num{5},\num{38})$ & $(\num{1},\num{25})$ & $(\num{2},\num{25})$ \\ \hline\hline
        \multirow{3}{*}{$\rm RMSE$} & Anechoic & \num{4.10} & \num{16.46} & \num{26.51} & \num{33.48} & \num{9.10} & \num{16.64} \\
        & Reverberation & \num{4.00} & \num{9.71} & \num{4.22} & \num{10.41} & \num{3.10} & \num{2.46} \\
        & Indoor & \num{4.30} & \num{3.52} & \num{8.86} & \num{4.24} & \num{1.10} & \num{3.29} \\ \hline
    \end{tabular}
    \label{tab:fitting_results}
\end{table*}

To compare the fitting accuracy between the analyzed scenarios, the RMSE metric associated to each solution is studied. Fig. \ref{fig:errors_cdf} shows the \acp{CDF} in the three analyzed scenarios. We see that the indoor case is associated to the lowest error, which confirms that the multipath richness is beneficial for fitting to the IFTR model. Conversely, the largest RMSE corresponds to the anechoic case; we will later evaluate the reasons for such behavior, which are linked to some of the underlying assumptions for the IFTR model.



% Figure environment removed

% Figure environment removed

In order to exemplify some particular \ac{PDF} fittings, six sample configurations (A, B, C, D, E, and F; see Fig. \ref{fig:configurations}) results shall be found in Fig. \ref{fig:fitting_solutions}. This figure visually shows the good accuracy of the model for describing the channel over the three scenarios (numerical results can be consulted in Table \ref{tab:fitting_results}). 
In fact, Rician/Rayleigh-like channels usually associated to reverberation scenarios are also improved when using the IFTR model. As supplementary material for the reader, an additional animation showing \ac{PDF} solutions for each configuration with the complete pareto front and the error metrics is included.






\subsection{Anechoic environment}
As mentioned previously, anechoic scenarios are associated to higher fitting error than multipath ones. Although most of the configurations can be described with a high accuracy, there exists a certain percentage of fitted configurations with an unsatisfactory \ac{GoF}. After a thorough analysis, the key aspect that justifies this behavior is the lack of phase diversity, which has two main implications: on the one hand, it causes that the \ac{CLT} assumption for the diffuse component is not met; on the other hand, it affects the assumption of uniformly distributed phases for the dominant specular components, as the travelled path by the two-rays is insufficient to produce all phases between $-\pi$ and $\pi$. 
% Figure environment removed

The implications of a non-uniform phase distribution have been discussed in \cite{Rao2015_GTR-V}, where the \ac{TWDP} model was modified to account for this particular situation. The resulting model, termed as \ac{GTR-V}, has the following \ac{PDF} for the received amplitude:
\begin{equation}\label{eq:GTR-V PDF}
    f_{\rm GTR-V}(r) = \int_{-\pi}^{\pi} f_{\rm rice}(r;K[1 + \Delta\cos(\alpha)])f_{\alpha}(\alpha)\mathrm{d}\alpha,
\end{equation}
%
\noindent where $f_{\rm rice}()$ is the well-known Rician distribution with a factor $K_r = K[1 + \Delta\cos(\alpha)]$, given by
%
\begin{equation}\label{eq:Rician PDF}
    f_{\rm rice}(r;K_r) = \frac{r}{\sigma^2}\exp\left(-\frac{r^2}{2\sigma^2} - K_r\right)I_0\left(\frac{r}{\sigma}\sqrt{2K_r}\right),
\end{equation}
%
\noindent with $I_0()$ being the modified Bessel function of the first class and order zero, whereas $f_{\alpha}()$ represents the phase difference distribution. In the \ac{GTR-V} channel model, this corresponds to a von Mises \ac{PDF} \cite{von_mises_distribution}:
%
\begin{equation}\label{eq:von Mises PDF}
    f_\alpha(\alpha;\kappa,\phi) = \frac{\exp(\kappa\cos(\alpha - \phi))}{2\pi I_0(\kappa)} \:\: \forall \:\: \alpha \in [-\pi,\pi],
\end{equation}
%
\noindent where $\phi \in \mathbb{R}$ represents the mean of the distribution and $\kappa \geq 0$ is inversely related to its variance. 

While it is indeed possible to generalize the IFTR model to incorporate the effect of a non-uniform phase distribution following the rationale in \cite{Rao2015_GTR-V}, this would imply that the resulting model would have two additional parameters, incurring in additional complexity when it comes to fitting. Instead, we will use the \ac{GTR-V} model even though it has no fluctuation on the two dominant specular components; in practice, this is equivalent to assuming a sufficiently large value of $m_1$ and $m_2$ in the \ac{IFTR} model (see Fig. \ref{fig:iftr_parameters_cdf}). Therefore, the use of this alternative model is well-justified for the sake of simplicity. After analyzing the distribution of the phase difference between the dominant specular components in the relevant anechoic configurations, a small sub-set with a von Mises like \ac{PDF} has been selected (named confs. G, H and I).

The fitting process for the \ac{GTR-V} model has been carried out as a two-step optimization problem. First, the empirical phase difference distribution is modelled as a von Mises distribution, finding the two parameters $\phi$ and $\kappa$ that provide the best fit for the phase distribution alone. After finding these parameters, a second optimization is performed in an analogous way as for IFTR finding the best pair $\{K,\Delta\}$ of \ac{GTR-V} model applying \ac{GA} search.

% Figure environment removed

\begin{table}[t]
    \centering
    \renewcommand{\arraystretch}{1.75}
    \caption{Comparison between \ac{IFTR} and \ac{GTR-V} modelling.}
    \begin{tabular}{c|c|ccc|}
        \multicolumn{1}{c}{} & \multicolumn{1}{c}{} & \multicolumn{3}{c}{\textbf{Anechoic configuration}} \\ \cline{3-5}
        \multicolumn{1}{c}{\textbf{Model}} & \multicolumn{1}{c|}{\textbf{Parameter}} & \textbf{G} & \textbf{H} & \textbf{I} \\ \hline
        \multirow{4}{*}{\ac{IFTR}} & $K$ [\si{\decibel}] & \num{30.0} & \num{30.0} & \num{25.1} \\
        & $\Delta$ & \num{2.7e-3} & \num{0.7e-3} & \num{0.40} \\
        & $(m_1,m_2)$ & $(\num{50},\num{32})$ & $(\num{50},\num{5})$ & $(\num{31},\num{13})$ \\ \cdashline{2-5}
        & $\mathrm{RMSE}$ & \num{142.38} & \num{28.17} & \num{17.58} \\ \hline
        \multirow{6}{*}{\ac{GTR-V}} & $K$ [\si{\decibel}] & \num{19.8} & \num{23.8} & \num{10.6} \\
        & $\Delta$ & \num{0.45} & \num{2.6e-4} & \num{0.11} \\
        & $\phi$ & \num{-0.10} & \num{0.08} & \num{-0.10} \\
        & $\kappa$ & \num{12.04} & \num{638.94} & \num{1.23}  \\ \cdashline{2-5}
        & $\mathrm{MSE_{VM}}$ & \num{2.9e-3} & \num{0.19} & \num{6.2e-4} \\
        & $\mathrm{RMSE}$ & \num{37.66} & \num{10.04} & \num{28.66} \\ \hline
    \end{tabular}
    \label{tab:GTR-V_fit_solution_data}
\end{table}



Fig. \ref{fig:non-uniform_distribution} shows the fitting of the phase difference distribution with a von Mises \ac{PDF} for configuration G. After the first step of the optimization, the resulting parameters $\kappa$ and $\phi$ yield an MSE of \num{0.0029}, which implies an excellent \ac{GoF}. The next step is find the optimal parameters for the GTR-V model by the use of \ac{GA}. The obtained results for the configurations G, H and I are summarized in Table \ref{tab:GTR-V_fit_solution_data}, and a visual comparison for the \acp{PDF} can be found in Fig. \ref{fig:GTR-V_fit_solutions}. For configurations G and H, we see that the \ac{GTR-V} model has a better capability to improve the fitting. The effect of a non-uniform phase distribution is translated into an effective reduction of the variance, and it can visually confirmed that the \ac{GTR-V} model achieves this. However, we see that the fitting is not improved when the \ac{PDF} is bimodal -- see configuration I in Fig. \ref{fig:GTR-V_fit_solutions}. We see that the original ability of the TWDP model to capture bimodality is affected when including the effect of non-uniform phases in the \ac{GTR-V} model. 



\subsection{Multipath environments}
% Figure environment removed



Multipath environments, as Fig. \ref{fig:errors_cdf} shows, have lower RMSE compared to the anechoic ones. This implies that most of the configurations has been described correctly with the \ac{IFTR} model, as confirmed in Fig. \ref{fig:fitting_solutions}. However, we have identified some cases on which the fitting to IFTR fails, which are explained next. Specifically, we have identified that in the event of sharp bimodal behaviors, that we refer henceforth as \textit{extreme bimodality}, the ability of the \ac{IFTR} model to capture such bimodality is not enough. This is exemplified in Fig. \ref{fig:bad_fitting_multipath} for configurations J and K. In these situations, the model tries to fit the dominant peak but masking the secondary one. Numerical results can be consulted in Table \ref{tab:multipath_bad}.


% Figure environment removed



\begin{table}[t]
    \centering
    \renewcommand{\arraystretch}{1.75}
    \caption{\ac{IFTR} solutions for multipath configurations with extreme bimodality behaviours.}
    \begin{tabular}{c|cc|}
        \multicolumn{1}{c}{} & \multicolumn{2}{c}{\textbf{Multipath configuration}} \\ \cline{2-3}
        \multicolumn{1}{c|}{\textbf{Parameter}} & \textbf{J (reverberation)} & \textbf{K (indoor)} \\ \hline
        $K$ [\si{\decibel}] & \num{28.9} & \num{28.2} \\
        $\Delta$ & \num{0.92} & \num{0.87} \\
        $(m_1,m_2)$ & $(\num{0.45},\num{30})$ & $(\num{3},\num{31})$ \\ \cdashline{1-3}
        $\mathrm{RMSE}$ & \num{18.27} & \num{6.33} \\\hline
    \end{tabular}
    \label{tab:multipath_bad}
\end{table}

One way to determine the origin of this particular behaviour is to analyze the \ac{CIR} of the path. Fig. \ref{fig:cir_multipath} shows the estimation of the channel time response for configurations J and K. The former reveals that there are more than two rays arriving at the antenna, situation not covered by the \ac{IFTR} model. The presence of several rays may cause a high variation in the received amplitude, which can result in a high imbalance between the modal values. The latter, however, presents two dominant rays but with a high proximity in time. Hence, its interaction may not be described independently as the IFTR model predicts. Even though the consideration of additional rays can be usually encompassed by integrating these into the diffuse component \cite{Romero2022}, this is not the case when the overall number of rays is reduced, multipath propagation is limited, and there is a lack of phase richness in the propagation environment.



