\section{Physical Channel Model}\label{sec:IFTR}

Let us consider a general formulation for the received radio signal over a wireless channel~\cite{durgin2000theory}, given as a superposition of a number of waves
\begin{align}
\label{eq1}  
V_r =\sum_{n=1}^{M} A_n \exp\left({j\varphi_n}\right),
\end{align}
where $A_n$ and $\varphi_n$ denote their corresponding amplitudes and phases. Now, it is possible to reexpress \eqref{eq1} in the following way:
\begin{align}
\label{eq2}  
V_r =\sum_{n=1}^{N} A_n \exp\left({j\varphi_n}\right) + \underbrace{\sum_{m=1}^{P} A_m \exp\left({j\varphi_m}\right)}_{Z},
\end{align}
with $M=N+P$, so that $N$ now represents a group of dominant specular waves, while $P$ indicate a group of numerous and relatively weak diffusely propagating waves. For sufficiently large $P$, the \ac{CLT} holds for the second term in \eqref{eq2}, implying that $Z\triangleq\sum_{m=1}^{P} A_m \exp\left({j\varphi_m}\right)$ can be approximated as a complex Gaussian RV with zero-mean and variance $2\sigma^2$. For $N=2$, constant-amplitude $A_n$ and uniformly distributed phases $\varphi_n$, the \ac{TWDP} model emerges \cite{Durgin2002_TWDP}. In the sequel, and taking the TWDP model as a baseline reference, we consider the general case on which $A_n=V_n\sqrt{\xi_n}$ so that $V_n$ denotes the amplitude of each of these dominant specular components, whereas $\xi_n$ are unit-mean independent Gamma random variables characterizing amplitude fluctuations. This model, referred to as independently-fluctuating two-ray (IFTR) model, was recently formulated in \cite{Olyaee2022_IFTR}, and is fully characterized by the following set of parameters:
\begin{align}
	K &= \frac{V_1^2+V_2^2}{2\sigma^2} \in [0,\infty),\\
	\Delta &= \frac{2V_1V_2}{V_1^2+V_2^2} \in [0,1],\\
	m_n\,&\{n=1,2\} \in (0,\infty),\\
        \Omega &\triangleq {\rm E\{|V_r|^2\}} \in [0,\infty),
\end{align}
where $1/m_n$ denote the fading severity (i.e., the amount of fluctuation) for each of the specular components, $K$ is the Rician factor defined as the ratio between the average powers of the dominant specular waves and the diffuse components, and $\Delta$ captures the amplitude dissimilarity between the two dominant specular waves. Finally, the scale parameter $\Omega=V_1^2+V_2^2+2\sigma^2$ represents the average received power.

