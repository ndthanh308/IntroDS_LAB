\documentclass[12pt,a4paper]{article}
\usepackage{latexsym,amssymb,amsmath,mathrsfs,amsthm}
\usepackage{sectsty}
\usepackage{upgreek}
\usepackage{color}
\usepackage{wasysym}
% \usepackage{fancybox}
\usepackage{bm}
% \usepackage{appendix}
\usepackage[T1]{fontenc}
\usepackage[anchorcolor=blue,%
 bookmarks=true,%
 bookmarksnumbered=true,%
% dvipdfm%
]{hyperref}
%\usepackage{cite}
\usepackage{pst-all}
\usepackage{graphicx}
%\usepackage[color]{showkeys}

%% Layout
\pagestyle{plain}
\addtolength{\topmargin}{-1cm}   %-1
\addtolength{\oddsidemargin}{-1.5cm}%-0.8
\setlength{\textwidth}{16cm}
\setlength{\textheight}{23cm}

%% Title and section
\allsectionsfont{\normalsize\sc\center}

%% Numbering for enumerate environment
\renewcommand{\labelenumi}{\theenumi}
\renewcommand{\theenumi}{{\rm(\roman{enumi})}}
\renewcommand{\labelenumii}{\theenumii}
\renewcommand{\theenumii}{{\rm(\alph{enumii})}}

%% Statement environments
\newtheorem{thm}{Theorem}[section]
\newtheorem{prop}[thm]{Proposition}
\newtheorem{cor}[thm]{Corollary}
\newtheorem{lem}[thm]{Lemma}
\newtheorem{fact}[thm]{Fact}
\newtheorem{defn}[thm]{Definition}
\newtheorem{remark}[thm]{Remark}
\newtheorem{example}[thm]{Example}
\newtheorem{assumption}{Assumption}

%% Proof environment
%\newenvironment{proof} {\par \begin{trivlist}%
%\item[]{\bf Proof.}\ }{\hfill $\square$ \end{trivlist}\par}
%\newenvironment{apf}[1]{\par\begin{trivlist}%
%\item[]{\bf Proof of #1.}\ }{\hfill $\square$ \end{trivlist}\par}

% Counter for statements and equations
\renewcommand{\theequation}{\thesection.\arabic{equation}}
\makeatletter \@addtoreset{equation}{section} \makeatother

%% Math environments
\newcommand{\nn}{\nonumber}
\newcommand{\ds}{\displaystyle}

%%% Symbols
\renewcommand{\P}{\mathbb{P}}
\newcommand{\E}{\mathbb{E}}
\newcommand{\Q}{\mathbb{Q}}
\newcommand{\R}{\mathbb{R}}
% \newcommand{\C}{\mathbb{C}}
\newcommand{\Z}{\mathbb{Z}}
\newcommand{\N}{\mathbb{N}}

\DeclareMathOperator{\Ric}{Ric}
\DeclareMathOperator{\diam}{diam}
\newcommand{\supp}{\mathop{\rm supp}\nolimits}
\newcommand{\lip}{\mathrm{Lip}}
\newcommand{\IR}{\operatorname{InRad}}
\newcommand{\Div}{\operatorname{div}}

\newcommand{\e}{\mathrm{e}}
\newcommand{\ep}{\varepsilon}
\newcommand{\ph}{\varphi}
\renewcommand{\d}{\mathrm{d}}
\newcommand{\m}{\mathfrak{m} }
\newcommand{\rr}{\mathfrak{r}}
\newcommand{\h}{\mathfrak{p}}
\newcommand{\s}{\mathfrak{s}}

\newcommand{\EE}{\mathcal{E}}
\newcommand{\FF}{\mathcal{F}}
\newcommand{\CC}{\mathcal{C}}
\newcommand{\PP}{\mathscr{P}}
\newcommand{\loc}{{\rm loc}}
\newcommand{\1}{{\bf 1}}
\newcommand{\eps}{{\varepsilon}}
\newcommand{\wt}{\widetilde}
\newcommand{\wh}{\widehat}
%\renewcommand{\<}{\langle}
%\renewcommand{\>}{\rangle}
\renewcommand{\wh}{\widehat}
\renewcommand{\wt}{\widetilde}

\newcommand{\cut}{\mathrm{Cut}\,}
\newcommand{\inte}{\mathrm{Int}\,}

\title{\large\bf The Littlewood-Paley-Stein inequality for 
Dirichlet space tamed by 
distributional curvature lower bounds}
\author{
Syota Esaki\thanks{Department of Applied Mathematics, Fukuoka University,
Fukuoka 814-0180, Japan ({\sf sesaki@fukuoka-u.ac.jp}). Supported in part by JSPS Grant-in-Aid for Scientific Research (C) (No. 23K03158) and fund
fund (No.~215001) from the Central Research Institute of Fukuoka University.}\ \ \ \
Zi Jian Xu\thanks{Department of Applied Mathematics, Fukuoka University,
Fukuoka 814-0180, Japan ({\sf a535218668@} {\sf yahoo.co.jp}).}
\ \ and\ \
Kazuhiro Kuwae\thanks{Department of Applied Mathematics, Fukuoka University,
Fukuoka 814-0180, Japan ({\sf kuwae@fukuoka-u.ac.jp}). Supported in part by JSPS Grant-in-Aid for Scientific Research (S) (No. 22H04942) and fund (No.~215001) from the Central Research Institute of Fukuoka University.}
}
\date{}
\begin{document}
\maketitle
\begin{abstract}
The notion of tamed Dirichlet space by distributional lower Ricci curvature bounds was proposed by Erbar, Rigoni, Sturm and Tamanini~\cite{ERST} as the Dirichlet space having a weak form of Bakry-\'Emery curvature lower bounds in distribution sense. 
In this framework,    
we establish the Littlewood-Paley-Stein inequality for $L^p$-functions and $L^p$-boundedness of $q$-Riesz transforms with $q>1$, which partially 
generalizes the result by Kawabi-Miyokawa~\cite{KawabiMiyokawa}. 
\end{abstract}

{\it Keywords}:  Strongly local Dirichlet space, tamed Dirichlet space, Bakry-\'Emery condition, smooth measures of (extended) 
Kato class, smooth measures of Dynkin class, 
Littlewood-Paley-Stein inequality, Riesz transform, Wiener space, path space with Gibbs measure,   RCD-space, 
Riemannian manifold with boundary, configuration space,  
{\it Mathematics Subject Classification (2020)}: Primary 31C25, 60H15, 60J60 ; 
Secondary 30L15, 53C21, 58J35, 35K05, 42B05, 47D08



%%%%%%%%%%%%%%%%%%%%%
%%%%%%%%%%%%%%%%%%%%%
%%%%%%%%%%%%%%%%%%%%%
\section{Statement of Main Theorem}\label{sec:StatementMain}
%%%%%%%%%%%%%%%%%%%%%
\subsection{Framework}\label{subsec:Frame}
Let $(X,\tau)$ be a topological Lusin space, i.e., a continuous injective image of a Polish 
space, endowed with a $\sigma$-finite Borel measure $\m$ on $X$ with full topological support. 
Let $(\mathscr{E},D(\mathscr{E}))$ be a quasi-regular symmetric strongly local Dirichlet space on $L^2(X;\m)$ 
and $(P_t)_{t\geq0}$ the associated symmetric sub-Markovian strongly continuous semigroup on $L^2(X;\m)$ (see \cite[Chapter IV, Definition~3]{MR} for the quasi-regularity and see \cite[Theorem~5.1(i)]{Kw:func} for the strong locality). 
Then there exists an $\m$-symmetric special standard process ${\bf X}=(\Omega, X_t, \P_x)$ 
associated with $(\mathscr{E},D(\mathscr{E}))$, i.e. for $f\in L^2(X;\m)\cap \mathscr{B}(X)$, 
$P_tf(x)=\E_x[f(X_t)]$ $\m$-a.e.~$x\in X$ (see \cite[Chapter IV, Section~3]{MR}).

It is known that for $u,v\in D(\mathscr{E})\cap L^{\infty}(X;\m)$ there exists a unique signed finite Borel 
measure $\mu_{\langle u,v\rangle}$ on $X$ such that 
\begin{align*}
2\int_X\tilde{f}\d\mu_{\langle u,v\rangle}=\mathscr{E}(uf,v)+\mathscr{E}(vf,u)-\mathscr{E}(uv,f)\quad \text{ for }\quad u,v\in D(\mathscr{E})\cap L^{\infty}(X;\m).
\end{align*}
Here $\tilde{f}$ denotes the $\mathscr{E}$-quasi-continuous $\m$-version of $f$ (see \cite[Theorem~2.1.3]{FOT}, \cite[Chapter IV, Proposition~3.3(ii)]{MR}). 
We set $\mu_{\langle f\rangle}:=\mu_{\langle f,f\rangle}$ for $f\in D(\mathscr{E})\cap L^{\infty}(X;\m)$. 
Moreover, 
for $f,g\in D(\mathscr{E})$, there exists a signed finite measure $\mu_{\langle f,g\rangle}$ on $X$ such that 
$\mathscr{E}(f,g)=\mu_{\langle f,g\rangle}(X)$, hence $\mathscr{E}(f,f)=\mu_{\langle f\rangle}(X)$. 
We assume $(\mathscr{E},D(\mathscr{E}))$ admits a carr\'e-du-champ $\Gamma$, i.e. 
$\mu_{\langle f\rangle}\ll\m$ for all $f\in D(\mathscr{E})$ and set $\Gamma(f):=\d\mu_{\langle f\rangle}/\d\m$. 
Then $\mu_{\langle f,g\rangle}\ll\m$ for all $f,g\in D(\mathscr{E})$ and $\Gamma(f,g):=
\d\mu_{\langle f,g\rangle}/{\d\m}\in L^1(X;\m)$ is expressed by $\Gamma(f,g)=\frac14(\Gamma(f+g)-\Gamma(f-g))$ for $f,g\in D(\mathscr{E})$. 

Fix $q\in\{1,2\}$. Let $\kappa$ be a signed smooth measure with its Jordan-Hahn decomposition $\kappa=\kappa^+-\kappa^-$. 
We assume that $\kappa^+$ is of Dynkin class smooth measure ($\kappa^+\in S_D({\bf X})$ in short) 
and $2\kappa^-$ is of extended Kato class smooth measure ($2\kappa^-\in S_{E\!K}({\bf X})$ in short). 
More precisely, $\nu\in S_D({\bf X})$ (resp.~$\nu\in S_{E\!K}({\bf X})$) if and only if $\nu\in S({\bf X})$ and 
$\m\text{-}\sup_{x\in X}\E_x[A_t^{\nu}]<\infty$ for any/some $t>0$ 
(resp.~$\lim_{t\to0}\m\text{-}\sup_{x\in X}\E_x[A_t^{\nu}]<1$) (see \cite{AM:AF}). 
Here $S({\bf X})$ denotes the family of smooth measures with respect to ${\bf X}$ (see \cite[Chapter VI, Definition~2.3]{MR}, \cite[p.~83]{FOT} for the definition of smooth measures) and $\m$-$\sup_{x\in X}f(x)$ denotes the $\m$-essentially supremum for  a function $f$ on $X$. 
Then, for $q\in\{1,2\}$, the quadratic form 
\begin{align*}
\mathscr{E}^{q\kappa}(f):=\mathscr{E}(f)+\langle q\kappa, \tilde{f}^2\rangle
\end{align*}
with finiteness domain $D(\mathscr{E}^{q\kappa})=D(\mathscr{E})$ is closed, lower semi-bounded, moreover, 
there exists $\alpha_0>0$ and $C>0$ such that 
\begin{align*}
C^{-1}\mathscr{E}_1(f,f)\leq \mathscr{E}^{q\kappa}_{\alpha_0}(f,f)\leq C\mathscr{E}_1(f,f)\quad \text{ for all }\quad f\in D(\mathscr{E}^{q\kappa})=D(\mathscr{E})
\end{align*}
(see \cite[(3.3)]{CFKZ:Pert} and \cite[Assumption of Theorem~1.1]{CFKZ:GenPert}).   
The Feynman-Kac semigroup $(p_t^{q\kappa})_{t\geq0}$ defined by 
\begin{align*}
p_t^{q\kappa}f(x)=\E_x[e^{-qA_t^{\kappa}}f(X_t)],\quad f\in \mathscr{B}_b(X)
\end{align*}
is $\m$-symmetric, i.e. 
\begin{align*}
\int_Xp_t^{q\kappa}f(x)g(x)\m(\d x)=\int_Xf(x)p_t^{q\kappa}g(x)\m(\d x)\quad\text{ for all }\quad f,g\in \mathscr{B}_+(X)
\end{align*} 
and coincides with the strongly continuous semigroup $(P_t^{q\kappa})_{t\geq0}$ on $L^2(X;\m)$ associated with 
$(\mathscr{E}^{q\kappa}, D(\mathscr{E}^{q\kappa}))$ (see \cite[Theorem~1.1]{CFKZ:GenPert}). 
Here $A_t^{\kappa}$ is a continuous additive functional (CAF in short) associated with the signed smooth measure $\kappa$ under Revuz correspondence. 
Under $2\kappa^{-}\in S_{E\!K}({\bf X})$ and $p\in[2,+\infty]$, 
$(p_t^{\kappa})_{t\geq0}$ can be extended 
to be a bounded operator on $L^{p}(X;\m)$ denoted by  $P_t^{\kappa}$ such that 
there exist finite constants $C=C(\kappa)>0, C_{\kappa}\geq0$ depending only on $\kappa^-$ such that 
for every $t\geq0$
\begin{align}
\|P_t^{\kappa}\|_{p,p}\leq Ce^{C_{\kappa}t}.\label{eq:KatoContraction}
\end{align}
%Moreover, under $2\kappa^{-}\in S_{E\!K}({\bf X})$, we can deduce
%\begin{align}
%\|P_t^{\kappa}\|_{p,p}\leq Ce^{C_{\kappa}t}\label{eq:KatoContraction}
%\end{align}
%for $p\in[2,+\infty]$. 
%where we use the convention that $\frac{p}{p-1}:=1$ if $p=\infty$.  
Here $C=1$ and 
$C_{\kappa}\geq0$  
can be taken to be $0$ under $\kappa^-=0$ (cf. \cite[Theorem~2.2]{Sznitzman}). When $\kappa=-R\m$ for a constant $R\in \R$, $C_{\kappa}$ (resp.~$C$) can be taken to be $R\lor 0$ (resp.~$1$). Let $\Delta^{q\kappa}$  be the $L^2$-generator associated with $(\mathscr{E}^{q\kappa},D(\mathscr{E}))$ 
defined by   
\begin{align}
\left\{\begin{array}{ll} D(\Delta^{q\kappa})&:=\{u\in D(\mathscr{E})\mid \text{there exists } w\in L^2(X;\m)\text{ such that }\\
&\hspace{3cm}\mathscr{E}^{q\kappa}(u,v)=-\int_Xwv\,\d\m\quad \text{ for any }\quad v\in D(\mathscr{E})\}, \\ \Delta^{q\kappa} u&:=w\quad\text{ for } w\in L^2(X;\m)\quad\text{specified as above,} \end{array}\right.\label{eq:generatorL2}
\end{align}
called the {\it Schr\"odinger operator} with potential $q\kappa$.  
Formally, $\Delta^{q\kappa}$ can be understood as \lq\lq$\Delta^{q\kappa}=\Delta-q\kappa$\rq\rq, 
where $\Delta$ is the $L^2$-generator associated with $(\mathscr{E},D(\mathscr{E}))$.  
\begin{defn}[$q$-Bakry-\'Emery condition]
{\rm 
Suppose that $q\in\{1,2\}$, $\kappa^+\in S_D({\bf X})$, $2\kappa^-\in S_{E\!K}({\bf X})$ and $N\in[1,+\infty]$. We say that $(X,\mathscr{E},\m)$ or simply $X$ satisfies the $q$-Bakry-\'Emery condition, briefly 
${\sf BE}_q(\kappa,N)$, if for every $f\in  D(\Delta)$ with $\Delta f\in D(\mathscr{E})$ 
and every nonnegative $\phi\in D(\Delta^{q\kappa})$ with 
$\Delta^{q\kappa}\phi\in L^{\infty}(X;\m)$ (we further impose $\phi\in L^{\infty}(X;\m)$ for $q=2$), we have 
\begin{align*}
\frac{1}{q}\int_X\Delta^{q\kappa}\phi \Gamma(f)^{\frac{q}{2}}\d\m-\int_X\phi
\Gamma(f)^{\frac{q-2}{2}}
\Gamma(f,\Delta f)\d\m
\geq \frac{1}{N}\int_X\phi\Gamma(f)^{\frac{q-2}{2}}(\Delta f)^2\d\m.
\end{align*}
The latter term is understood as $0$ if $N=\infty$.
}
\end{defn}

\begin{assumption}\label{asmp:Tamed}
{\rm We assume that $X$ satisfies ${\sf BE}_{2}(\kappa,N)$ condition for a given signed smooth measure  
$\kappa$ with $\kappa^+\in S_D({\bf X})$ and $2\kappa^-\in S_{E\!K}({\bf X})$ and $N\in[1,\infty]$.
}
\end{assumption}
Under Assumption~\ref{asmp:Tamed}, we say that $(X,\mathscr{E},\m)$ or simply $X$ is {\it tamed}. 
In fact, under $\kappa^+\in S_D({\bf X})$ and $2\kappa^-\in S_{E\!K}({\bf X})$, the condition ${\sf BE}_2(\kappa,\infty)$ is {\it equivalent} 
to ${\sf BE}_1(\kappa,\infty)$ (see Lemma~\ref{lem:BakryEmeryEquivalence} below), in particular, the heat flow $(P_t)_{t\geq0}$ 
satisfies 
\begin{align}
\Gamma(P_tf)^{1/2}\leq P_t^{\kappa}\Gamma(f)^{1/2}\quad\m\text{-a.e.~for any }f\in D(\mathscr{E})\quad \text{ and }\quad t\geq0\label{eq:gradCont}
\end{align}
(see \cite[Definition~3.3 and Theorem~3.4]{ERST}).
The inequality \eqref{eq:gradCont} plays a crucial role in our paper. 
Note that our condition $\kappa^+\in S_D({\bf X})$, $\kappa^-\in S_{E\!K}({\bf X})$ (resp.~$\kappa^+\in S_D({\bf X})$, $2\kappa^-\in S_{E\!K}({\bf X})$) is stronger than the $1$-moderate (resp.~$2$-moderate) condition treated in \cite{ERST} for the definition of tamed space.  
The $\m$-symmetric Markov process ${\bf X}$ treated in our paper may not be conservative in general. Under Assumption~\ref{asmp:Tamed}, 
a sufficient condition ({\it intrinsic stochastic completeness} called in \cite{ERST}) 
for conservativeness of ${\bf X}$ is discussed in \cite[Section~3.3]{ERST}, in 
particular, under $1\in D(\mathscr{E})$ and Assumption~\ref{asmp:Tamed}, we have the conservativeness of ${\bf X}$.

Let us introduce the Littlewood-Paley $G$-functions. To do this, we recall 
the subordination of semigroups. For $t\geq0$, define a probability measure 
$\lambda_t$ on $[0,+\infty[$ whose Laplace transform 
is given by 
\begin{align*}
\int_0^{\infty}e^{-\gamma s}\lambda_t(\d s)=e^{-\sqrt{\gamma}t},\quad \gamma\geq0.
\end{align*}
Then, for $t>0$, $\lambda_t$ has the following expression
\begin{align*}
\lambda_t(\d s):=\frac{t}{2\sqrt{\pi}}e^{-t^2/4s}s^{-3/2}\d s.
\end{align*}
For $\alpha\geq C_{\kappa}$, we define the subordination $(Q_t^{(\alpha),\kappa})_{t\geq0}$ of $(P_t^{\kappa})_{t\geq0}$ by 
\begin{align*}
Q_t^{(\alpha),\kappa}f:=\int_0^{\infty}e^{-\alpha s}P_s^{\kappa}f\,\lambda_t(\d s),\quad f\in  L^p(X;\m).
\end{align*}
When $\kappa=0$, we write $Q_t^{(\alpha)}$ instead of $Q_t^{(\alpha),0}$. Then we can easily see that for $p\in[2,+\infty]$  
\begin{align}
\|Q_t^{(\alpha),\kappa}f\|_{L^p(X;\m)}&\leq \int_0^{\infty}e^{-\alpha s}\|P_s^{\kappa}f\|_{L^p(X;\m)}\lambda_t(\d s)\notag\\
 &\leq C\left(\int_0^{\infty}e^{-(\alpha-C_{\kappa}) s}\lambda_t(\d s) \right)\|f\|_{L^p(X;\m)}\label{eq:Contra}\\
 &=Ce^{-\sqrt{\alpha-C_{\kappa}}t}\|f\|_{L^p(X;\m)}.\notag
\end{align} 
The infinitesimal generator of 
$(Q_t^{(\alpha),\kappa})_{t\geq0}$ is denoted by $-\sqrt{\alpha-\Delta_p^{\kappa}}$. We may omit the subscript $p$ for simplicity. 

For $f\in L^2(X;\m)\cap L^p(X;\m)$ and $\alpha\geq C_{\kappa}$, we define the Littlewood-Paley's $G$-functions by 
\begin{align*}
{g_{\stackrel{\rightarrow}{f}}}^{\kappa}(x,t)&:=\left|\frac{\partial}{\partial t}(Q_t^{(\alpha),\kappa}f)(x) \right|,\qquad\qquad\qquad {G_{\stackrel{\rightarrow}{f}}}^{\!\!\kappa}(x):=\left(\int_0^{\infty}t{g_{\stackrel{\rightarrow}{f}}}^{\!\!\kappa}(x,t)^2\d t  \right)^{1/2},\\
{g_f^{\uparrow}}^{\stackrel{}{\kappa}}
(x,t)&:=\left(\Gamma(Q_t^{(\alpha),\kappa}f) \right)^{1/2}(x),\qquad\qquad\quad {G_f^{\uparrow}}^{\kappa}(x):=\left(\int_0^{\infty}t{g_f^{\uparrow}}^{\kappa}(x,t)^2\d t  \right)^{1/2},\\
g_f^{\kappa}(x,t)&:=\sqrt{{g_{\stackrel{\rightarrow}{f}}}^{\kappa}(x,t)^2+
{g_f^{\uparrow}}^{\kappa}
(x,t)^2}, \qquad\quad G_f^{\;\kappa}(x):=\left(\int_0^{\infty}tg_f^{\kappa}(x,t)^2\d t \right)^{1/2}.
\end{align*}
For notational simplicity, we omit $\kappa$ in $G$-functions when we replace $P_t^{\kappa}$ with $P_t$, e.g. 
 we write ${g_f^{\rightarrow}}(x,t)$ (resp.~${G_f^{\rightarrow}}(x)$) instead of 
${g_f^{\rightarrow}}^{0}(x,t)$ (resp.~${G_f^{\rightarrow}}^{0}(x)$) and so on. 

Now we present the Littlewood-Paley-Stein inequality. In what follows, the notation 
$\|u\|_{L^p(X;\m)}\lesssim \|v\|_{L^p(X;\m)}$ stands for 
$\|u\|_{L^p(X;\m)}\leq A(\kappa)\|v\|_{L^p(X;\m)}$, 
where $A(\kappa)$ is a positive constant depending only on $\kappa$ and $p\in[1,+\infty[$. 

 
\begin{thm}\label{thm:main1}
For any $p\in]1,+\infty[$ and $\alpha\geq C_{\kappa}$ and $\alpha>0$, 
The $G$-functions $g_f^{\rightarrow}(\cdot,t)$, $g_f^{\uparrow}(\cdot,t)$, $g_f(\cdot,t)$, 
$G_f^{\rightarrow}$, $G_f^{\uparrow}$ and $G_f$ can be extended to be in $L^p(X;\m)$ for $f\in L^p(X;\m)$
and the following inequalities hold for 
$f\in L^p(X;\m)$
\begin{align}
\|G_f\|_{L^p(X;\m)}&\lesssim\|f\|_{L^p(X;\m)},\label{eq:LittlewoodPaleyStein1}\\
\|f\|_{L^p(X;\m)}&\lesssim\| G_f^{\rightarrow}\|_{L^p(X;\m)}.\label{eq:LittlewoodPaleyStein2}
\end{align}
Moreover, \eqref{eq:LittlewoodPaleyStein1} remains valid for $\alpha>0$ under $p\in]1,2]$. 
Furthermore, \eqref{eq:LittlewoodPaleyStein1} and \eqref{eq:LittlewoodPaleyStein2} remain valid for $\alpha=0$ under $\kappa^-=0$ {\rm(}hence $C_{\kappa}=0${\rm)}. 
Suppose further that $(\mathscr{E},D(\mathscr{E}))$ is transient or irreducible, $\alpha=0$ and $\kappa^-=0$. 
Then, for any $p\in]1,+\infty[$, 
the following inequalities hold for 
$f\in L^p(X;\m)\cap L^2(X;\m)$: $E_of\in L^p(X;\m)$, 
\begin{align}
\|G_f\|_{L^p(X;\m)}&\lesssim\|f-E_{o}f\|_{L^p(X;\m)},\label{eq:LittlewoodPaleyStein1+}\\
\|f-E_{o}f\|_{L^p(X;\m)}&\lesssim\| G_f^{\rightarrow}\|_{L^p(X;\m)},\label{eq:LittlewoodPaleyStein2+}\\
\|f-E_{o}f\|_{L^p(X;\m)}&\lesssim\| G_{f}^{\uparrow}\|_{L^p(X;\m)}.\label{eq:LittlewoodPaleyStein3+}
\end{align}
Here $E_{o}:=E_0-E_{0-}$ and $(E_{\lambda})_{\lambda\in\R}$ is a real resolution of the identity associated to the $L^2$-generator $\Delta$, {\rm(}in another word $E_o:=E_{\{0\}}$ and $E_{\lambda}:=E_{]-\infty,\lambda]}${\rm)}
 {\rm(}see \cite[Chapter IX, Section~5]{Yosida} for resolution of the identity{\rm)}. 
\end{thm}

In Theorem~\ref{thm:main1}, there is no estimate of $G_f^{\uparrow}$ when $\alpha>0$. 
If we assume the spectral gap for $(\mathscr{E},D(\mathscr{E}))$, i.e., there exists small $\eps>0$ such that 
$E_{]0,\eps[}=0$, then the Green operator $G_0$ 
with index $0$ is a bounded linear operator on $L^2(X;\m)$ (see \cite[(4.42)]{ShigekawaText}), so that we can deduce the same estimate 
\eqref{eq:LittlewoodPaleyStein2} for $G_f^{\uparrow}$ holds by way of the argument in \cite[Subsection~3.2.11]{ShigekawaText} (see the last statement of Subsection~\ref{subsec:LastProof} below).  

\medskip

As an application of Theorem~\ref{thm:main1}, we provide the following: 

\begin{thm}\label{thm:main2}
Let $p\in]1,+\infty[$, $q\in]1,+\infty[$ and $\alpha> C_{\kappa}$. We define 
\begin{align*}
R_{\alpha}^{(q)}(\Delta)f:=\Gamma\left((\alpha-\Delta)^{-\frac{q}{2}}f \right)^{1/2},\qquad f\in L^p(X;\m)\cap L^2(X;\m).
\end{align*}
Then we have the following statements:
\begin{enumerate}
\item[{\rm (1)}] For any $p\in[2,+\infty[$, $R_{\alpha}^{(q)}(\Delta)$ can be extended to a bounded operator on $L^p(X;\m)$. 
The operator norm $\|R_{\alpha}^{(q)}(\Delta)\|_{p,p}$ depends only on $\kappa,p,q$ and $C_{\kappa}$. This implies the inclusion
\begin{align*}
D((I-\Delta_p)^{\frac{q}{2}})\subset H^{1,p}(X;\m),
\end{align*}
where $H^{1,p}(X;\m)$ is the $\|\cdot\|_{H^{1,p}}$-completion of 
\begin{align*}
\mathscr{D}_{1,p}:=\{f\in L^p(X;\m)\cap D(\mathscr{E})\mid \Gamma(f)^{\frac12}\in L^p(X;\m)\}
\end{align*} 
with respect to the 
$(1,p)$-Sobolev norm $\|\cdot\|_{H^{1,p}}$ defined by 
\begin{align*}
\|f\|_{H^{1,p}}:=\left(\|f\|_{L^p(X;\m)}^p+\|\Gamma(f)^{\frac12}\|_{L^p(X;\m)}^p \right)^{\frac{1}{p}}. 
\end{align*}
\item[{\rm (2)}] For any $p\in[2,+\infty[$ and $q\in]1,2[$, there exists a positive constant $C_{p,q}$ such that 
\begin{align}
\|\Gamma(P_tf)^{1/2}\|_{L^p(X;\m)}\leq C_{p,q}\|R_{\alpha}^{(q)}(\Delta)\|_{p,p}(\alpha^{q/2}+t^{-q/2})
\|f\|_{L^p(X;\m)},\quad t>0,\quad f\in L^p(X;\m).
\end{align}
\end{enumerate}
\end{thm}
\begin{remark}
{\rm The $(1,p)$-Sobolev space $(H^{1,p}(X;\m),\|\cdot\|_{H^{1,p}})$ becomes a uniformly convex (hence reflexive) Banach space for any $p\in]1,+\infty[$ (see \cite[Lemma~3.1]{Kw:SobolevSpace}). 
}
\end{remark}
\begin{remark}\label{rem:forthcoming}
{\rm In the forthcoming paper \cite{EJK}, we establish Theorem~\ref{thm:main2}(1) for $q=1$ and $p\in]1,+\infty[$, i.e., so-called the boundedness of the Riesz transform $R_{\alpha}(\Delta):=R_{\alpha}^{(1)}(\Delta)$ on $L^p(X;\m)$ for $p\in]1,+\infty[$ based on the vector space calculus for tamed Dirichlet space developed by Braun~\cite{Braun:Tamed2021}.
}
\end{remark}

\section{Test functions}

The following lemmas hold. 

\begin{lem}[{\cite[Proposition~6.8]{ERST}}]\label{lem:boundedEst}
Under Assumption~\ref{asmp:Tamed}, ${\sf BE}_2(-\kappa^-,+\infty)$ holds. Moreover, for every $f\in L^2(X;\m)\cap L^{\infty}(X;\m)$ and $t>0$, it holds 
\begin{align}
\Gamma(P_tf)\leq\frac{1}{2t}\|P_t^{-\kappa^-}\|_{\infty,\infty}\cdot\|f\|_{L^{\infty}(X;\m)}^2.\label{eq:BoundedEst}
\end{align}
In particular, for $f\in L^2(X;\m)\cap L^{\infty}(X;\m)$, then $P_tf\in {\rm Test}(X)$. 
\end{lem}




\begin{lem}[{\cite[Theorems~3.4 and 3.6, Proposition~3.7 and Theorem~6.10]{ERST}}]\label{lem:BakryEmeryEquivalence}\quad\\
Under $\kappa^+\in S_D({\bf X})$ and $2\kappa^-\in S_{E\!K}({\bf X})$,  
the condition ${\sf BE}_2(\kappa,\infty)$ is equivalent 
to ${\sf BE}_1(\kappa,\infty)$. In particular, we have \eqref{eq:gradCont}. 
\end{lem}




Under Assumption~\ref{asmp:Tamed}, we now introduce ${\rm Test}(X)$ the set of test functions: 

\begin{defn}\label{def:TestFunc}
{\rm 
Let $(X,\mathscr{E},\m)$ be a tamed space. Let us define the set of {\it test functions} by 
\begin{align*}
{\rm Test}(X):&=\{f\in D(\Delta)\cap L^{\infty}(X;\m)\mid \Gamma(f)\in L^{\infty}(X;\m),\Delta f\in D(\mathscr{E})\}.
\end{align*}
}
\end{defn}


\begin{lem}[{\cite[Lemma~3.2]{Sav14}}]\label{lem:algebra}
Under Assumption~\ref{asmp:Tamed}, for every $f\in {\rm Test}(X)$, we have 
$\Gamma(f)\in D(\mathscr{E})\cap L^{\infty}(X;\m)$ and there exists $\mu=\mu^+-\mu^-$ with 
$\mu^{\pm}\in D(\mathscr{E})^*$ such that 
\begin{align}
-\mathscr{E}^{2\kappa}(u,\varphi)=\int_X\tilde{\varphi}\,\d \mu\quad\text{ for all }\quad \varphi\in D(\mathscr{E}).
\end{align}
Moreover, ${\rm Test}(X)$ is an algebra, i.e., for $f,g\in {\rm Test}(X)$, $fg\in {\rm Test}(X)$, 
if further {\boldmath$f$}$\in {\rm Test}(X)^n$, then $\Phi(${\boldmath$f$}$)\in {\rm Test}(X)$ for every smooth 
function $\Phi:\R^n\to\R$ with $\Phi(0)=0$.
\end{lem}


\begin{lem}[{cf.~\cite[Corollary~6.9]{ERST}}]\label{lem:DensenessTestFunc}
${\rm Test}(X)\cap L^p(X;\m)$ is dense in $L^p(X;\m)\cap L^2(X;\m)$ both in $L^p$-norm 
and in $L^2$-norm. Moreover, ${\rm Test}(X)$ is dense in $(\mathscr{E},D(\mathscr{E}))$. 
\end{lem}
\begin{proof}[\bf Proof]  
Take $f\in L^p(X;\m)\cap L^2(X;\m)$. We may assume $f\in L^{\infty}(X;\m)$, because 
$f$ is $L^p$(and also $L^2$)-approximated by a sequence $\{f^k\}$ of 
$L^p(X;\m)\cap L^2(X;\m)\cap L^{\infty}(X;\m)$-functions defined 
by $f^k:=(-k)\lor f\land k$. If $f\in L^p(X;\m)\cap L^2(X;\m)\cap L^{\infty}(X;\m)$, 
$P_tf\in {\rm Test}(X)$ by Lemma~\ref{lem:boundedEst} and 
$\{P_tf\}\subset {\rm Test}(X)\cap L^p(X;\m)$ converges to 
$f$ in $L^p$ and in $L^2$ as $t\to0$. If $f\in D(\mathscr{E})$, then $f$ can be approximated by 
$\{P_tf^k\}$ in $(\mathscr{E},D(\mathscr{E}))$. This shows the last statement.
\end{proof} 
\begin{remark}\label{rem:TestFunction}
{\rm 
As proved above, ${\rm Test}(X)$ forms an algebra and dense in $(\mathscr{E},D(\mathscr{E}))$ under 
Assumption~\ref{asmp:Tamed}. However, ${\rm Test}(X)$ is not necessarily a subspace of $C_b(X)$. 
When the tamed space comes from ${\sf RCD}$-space, the Sobolev-to-Lipschitz property of ${\sf RCD}$-spaces ensures 
${\rm Test}(X)\subset C_b(X)$. 
}
\end{remark}

\section{Proof of Theorem~\ref{thm:main1}}
In this section, we prove  Theorem~\ref{thm:main1} by a probabilistic method. 
The original idea is due to Meyer~\cite{Meyer}. The reader is referred to Bakry~\cite{Bakry1}, Shigekawa-Yoshida~\cite{ShigekawaYoshida}, Yoshida~\cite{YoshidaNobuo}. In these papers, they expanded 
$\Delta(Q_t^{(\alpha)}f)^p$, $f\in \mathcal{A}$, by employing the usual functional 
analytic argument in the proof of Littlewood-Paley-Stein inequality. In that calculations, 
they needed to assume the existence of a good core $\mathcal{A}$ like ${\rm Test}(X)$ in this paper. 
Though we have a good core ${\rm Test}(X)$ in the framework of tamed Dirichlet space under 
Bakry-\'Emery condition ${\sf BE}_2(\kappa,\infty)$, we will not follow their method. 
We mimic the method of the proof of Kawabi-Miyokawa~\cite{KawabiMiyokawa} in proving Theorem~\ref{thm:main1}.  
However, our curvature lower bound is not a constant in general, this gives another 
technical difficulty in proving Theorem~\ref{thm:main1}. For this, we should modify the 
method of the proof in  Kawabi-Miyokawa~\cite{KawabiMiyokawa}.  
We prove Theorem~\ref{thm:main1} for $f\in D(\Delta)\cap L^p(X;\m)$ at the beginning. Any $f\in  L^p(X;\m)\cap L^2(X;\m)$ 
with 
$p\in]1,+\infty[$ can be approximated by a sequence in $D(\Delta)\cap L^p(X;\m)$ in $L^p$-norm and in $L^2$-norm.
Then one can conclude the statement of 
Theorem~\ref{thm:main1} for general $f\in  L^p(X;\m)\cap L^2(X;\m)$ (hence $G_f^{\rightarrow}$, $G_f^{\uparrow}$ and $G_f$ can be extended for general $f\in L^p(X;\m)$) in view of the triangle inequality
$|g_{f_1}^{\rightarrow}-g_{f_2}^{\rightarrow}|\leq g_{f_1-f_2}^{\rightarrow}$, 
(resp.~$|g_{f_1}^{\uparrow}-g_{f_2}^{\uparrow}|\leq g_{f_1-f_2}^{\uparrow}$) hence $|G_{f_1}^{\rightarrow}-G_{f_2}^{\rightarrow}| \leq G_{f_1-f_2}^{\rightarrow}$ (resp.~$|G_{f_1}^{\uparrow}-G_{f_2}^{\uparrow}| \leq G_{f_1-f_2}^{\uparrow}$). 


\subsection{Preliminaries}\label{subsec:preliminary}

In this subsections, we make some preparations. We have already used the notation $(\P_x)_{x\in X}$ 
to denote the diffusion measure of ${\bf X}$ associated with the Dirichlet form $(\mathscr{E},D(\mathscr{E}))$. In this subsection, we use the notation $\P_x^{\uparrow}$ instead of $\P_x$. Let $(B_t,\P_{\stackrel{\rightarrow}{a}})$ be one-dimensional Brownian motion 
starting at $a\in \R$ with the generator $\frac{\partial^2}{\partial a^2}$. We set 
$\widehat{X}:=X\times\R$, $\hat{x}:=(x,a)\in \widehat{X}$, 
$\widehat{X}_t:=(X_t,B_t)$, $t\geq0$, $\widehat{\m}:=\m\otimes m$ 
 and $\P_{\hat{x}}:=\P_x^{\uparrow}\otimes \P_{\stackrel{\rightarrow}{a}}$. 
Then $\widehat{\bf X}:=(\widehat{X}_t,\P_{\hat{x}})$ is an $\widehat{\m}$-symmetric diffusion process on 
$\widehat{X}$ with the (formal) generator $\Delta+\frac{\partial^2}{\partial a^2}$, where $m$ is one-dimensional Lebesgue measure. We put $\P_{\m}^{\uparrow}:=\int_X\P_x^{\uparrow}\m(\d x)$, 
$\P_{\m\otimes\delta_a}:=\int_X\P_{(x,a)}\m(\d x)$ and denote the integration with respect to 
$\P_x^{\uparrow}$, $\P_{\stackrel{\rightarrow}{a}}$, $\P_{(x,a)}$ and $\P_{\m\otimes\delta_a}$ by 
$\E_x^{\uparrow}$, $\E_{\stackrel{\rightarrow}{a}}$, $\E_{(x,a)}$ and $\E_{\m\otimes\delta_a}$, respectively. 

We denote the semigroup on $L^p(\widehat{X};\widehat{\m})$ associated with the diffusion process 
$(\widehat{X}_t)_{t\geq0}$ by $(\widehat{P}_t)_{t\geq0}$ and its generator by $\widehat{\Delta}_p$. We also denote the Dirichlet form on $L^2(\widehat{X};\widehat{\m})$ associated with $\widehat{\Delta}_2$ by 
$(\widehat{\mathscr{E}}, D(\widehat{\mathscr{E}}))$. That is, 
\begin{align*}
D(\widehat{\mathscr{E}}):&=\left\{u\in L^2(\widehat{X};\widehat{\m})\;\left|\; \lim_{t\to0}\frac{1}{t}(u-\widehat{P}_tu,u)_{L^2(\widehat{X};\widehat{\m})}<+\infty\right.\right\},\\
\widehat{\mathscr{E}}(u,v):&=\lim_{t\to0}\frac{1}{t}(u-\widehat{P}_tu,v)_{L^2(\widehat{X};\widehat{\m})}\qquad\text{ for }\quad u,v\in D(\widehat{\mathscr{E}}).
\end{align*} 
Since $(\widehat{\mathscr{E}}, D(\widehat{\mathscr{E}}))$ is associated to the 
$\widehat{\m}$-symmetric Borel right process $\widehat{\bf X}$, it is quasi-regular by 
Fitzsimmons~\cite{Fitzsimmons}. 

Throughout this subsection, we assume $\kappa^+\in S_D({\bf X})$ and $\kappa^-\in S_{E\!K}({\bf X})$.  
We define $\widehat{\kappa}^{\,\pm}:=\kappa^{\pm}\otimes m$ and $\widehat{\kappa}:=
\widehat{\,\kappa}^+-\widehat{\,\kappa}^-$. 
The associated positive continuous additive functional (PCAF in short) $\widehat{A}_t^{\;\widehat{\,\kappa}^{\pm}}$ in Revuz correspondence under $\wh{\bf X}$ is given by $\widehat{A}_t^{\;\widehat{\kappa}^{\pm}}=A_t^{\kappa^{\pm}}$, in particular, $\widehat{\kappa}^{\,+}$ (resp.~$\widehat{\kappa}^{\,-}$) 
is of Dynkin (resp.~extended Kato) class smooth measure with respect to $\widehat{\bf X}$, i.e. 
$\widehat{\kappa}^{\,+}\in S_D(\widehat{\bf X})$ (resp.~$\widehat{\kappa}^{\,-}\in S_{E\!K}(\widehat{\bf X})$). 
Then we can define the following quadratic form 
$(\widehat{\mathscr{E}}^{\;\widehat{\kappa}}, D(\widehat{\mathscr{E}}^{\;\widehat{\kappa}}))$ 
on $L^2(\widehat{X};\widehat{\m})$: 
\begin{align}
D(\widehat{\mathscr{E}}^{\;\widehat{\kappa}}):=D(\widehat{\mathscr{E}}),\qquad
\widehat{\mathscr{E}}^{\;\widehat{\kappa}}(u,v):=\widehat{\mathscr{E}}(u,v)+\langle\widehat{\kappa},\tilde{u}\tilde{v}\rangle\quad\text{ for }\quad u,v\in D(\widehat{\mathscr{E}}^{\;\widehat{\kappa}}).\label{eq:quadratic}
\end{align}
Here $\tilde{u}$ denotes the $\widehat{\mathscr{E}}$-quasi-continuous $\widehat{\m}$-version of $u$ with respect to 
$(\widehat{\mathscr{E}}, D(\widehat{\mathscr{E}}))$. 
The strongly continuous semigroup $(\widehat{P}_t^{\;\widehat{\kappa}})_{t\geq0}$ on $L^2(\widehat{X};\widehat{\m})$ associated with the quadratic form 
$(\widehat{\mathscr{E}}^{\;\widehat{\kappa}}, D(\widehat{\mathscr{E}}^{\;\widehat{\kappa}}))$ 
is given by 
\begin{align}
\widehat{P}_t^{\widehat{\;\kappa}}u(\hat{x})=
\E_{\hat{x}}
[e^{-A_t^{\kappa}}u(\widehat{X}_t)]
\quad \text{ for }\quad u\in L^2(\widehat{X};\widehat{\m})\cap \mathscr{B}(\widehat{X}).\label{eq:FeynmanKac}
\end{align}

We denote by $\widehat{\mathscr{C}}:={\rm Test}
(X)\otimes C_c^{\infty}(\R)$ the totality of all linear combinations of $f\otimes\varphi$, $f\in {\rm Test}(X)$, $\varphi\in C_c^{\infty}(\R)$, where 
$(f\otimes \varphi)(y):=f(x)\varphi(a)$. Meanwhile, the space $L^2(X;\m)\otimes L^2(\R)$ and 
$D(\mathscr{E})\otimes H^{1,2}(\R)$ are usual tensor products of Hilbert spaces, where $H^{1,2}(\R)$ is the Sobolev space which consists of all functions $\varphi\in L^2(\R)$ such that the weak derivative $\varphi'$ exists and belongs to $L^2(\R)$. Then we have 

\begin{lem}\label{lem:Core}
$\widehat{\mathscr{C}}$ is dense in $D(\widehat{\mathscr{E}}
)$. Moreover, 
for $u,v\in D(\mathscr{E}
)\otimes H^{1,2}(\R)$,  we have
\begin{align}
\widehat{\mathscr{E}}(u,v)&=\int_{\R}\mathscr{E}(u(\cdot,a),v(\cdot,a))m(\d a)+
\int_X\m(\d x)\int_{\R}\frac{\partial u}{\partial a}(x,a)\frac{\partial v}{\partial a}(x,a)m(\d a).\label{eq:Identity}
\end{align}
\end{lem}
\begin{proof}[\bf Proof]  
We denote by $(T_t)_{t\geq0}$ the transition semigroup associated with $(B_t,\P_a^{\rightarrow})$. 
We can regard it as the semigroup on $L^2(\R)$. First, we denote that the following identity holds:
\begin{align}
\widehat{P}_t(f\otimes\varphi)=(P_tf)\otimes(T_t\varphi), \quad f\in L^2(X;\m),\quad \varphi\in L^2(\R).\label{eq:productSemigroup}
\end{align} 
By \eqref{eq:productSemigroup}, we can see $\widehat{\mathscr{C}}\subset D(\mathscr{E})\otimes H^{1,2}(\R)\subset D(\widehat{\mathscr{E}})$ and the identity \eqref{eq:Identity}. We also have
\begin{align}
\widehat{\mathscr{E}}_1(f\otimes \varphi,f\otimes\varphi)\leq\mathscr{E}(f,f)\|\varphi\|_{L^2(\R)}^2
+\|f\|_{L^2(X;\m)}^2\left(\|\varphi'\|_{L^2(\R)}^2+\|\varphi\|_{L^2(\R)}^2 \right)\label{eq:CartesProduct}
\end{align} 
holds for $f\in D(\mathscr{E})$, $\varphi\in H^{1,2}(\R)$. By \eqref{eq:CartesProduct}, 
we can see that $\widehat{\mathscr{C}}$ is dense in $D(\mathscr{E})\otimes H^{1,2}(\R)$ with respect 
to $\widehat{\mathscr{E}}_1$, because ${\rm Test}(X)$ and $C_c^{\infty}(\R)$ are dense in $(\mathscr{E},D(\mathscr{E}))$ and in $H^{1,2}(\R)$, respectively. 

Hence it is sufficient to show $D(\mathscr{E})\otimes H^{1,2}(\R)$ is dense in $(\widehat{\mathscr{E}}, D(\widehat{\mathscr{E}}))$. Since $L^2(X;\m)\otimes L^2(\R)$ is dense in $L^2(\widehat{X};\widehat{\m})$, 
$\bigcup_{t>0}\widehat{P}_t\left(L^2(X;\m)\otimes L^2(\R) \right)$ is dense in 
$(\widehat{\mathscr{E}}, D(\widehat{\mathscr{E}}))$. On the other hand, 
\eqref{eq:productSemigroup} also leads us to 
\begin{align*}
\bigcup_{t>0}\widehat{P}_t\left(L^2(X;\m)\otimes L^2(\R) \right)&=\bigcup_{t>0}\left(P_t(L^2(X;\m)) \right)
\otimes \left(T_t(L^2(\R))\right)\\
&\subset D(\mathscr{E})\otimes H^{1,2}(\R)\subset D(\widehat{\mathscr{E}}).
\end{align*}
This completes the proof.
\end{proof} 
Denote by $\mathscr{P}(X)$, the family of all Borel probability measures on $X$, and  
by $\mathscr{B}^*(X)$, the family of all universally measurable sets, that is, $\mathscr{B}^*(X):=\bigcap_{\nu\in\mathscr{P}(X)}\overline{\mathscr{B}(X)}^{\nu}$, where $\overline{\mathscr{B}(X)}^{\nu}$ is the $\nu$-completion of $\mathscr{B}(X)$ for $\nu\in\mathscr{P}(X)$. $\mathscr{B}^*(X)$ also denotes the family of 
universally measurable real valued functions. Moreover, $\mathscr{B}_b^*(X)$ (resp.~$\mathscr{B}_+^*(X)$) denotes the family of bounded (resp.~non-negative) universally measurable functions.  
For $f\in \mathscr{B}_b^*(X)$, or $f\in \mathscr{B}_+^*(X)$, we set
\begin{align}
q_t^{(\alpha),\kappa}f(x):=\E_x\left[\int_0^{\infty}e^{-\alpha s-A_s^{\kappa}}f(X_s)\lambda_t(\d s) \right]
\end{align}
and write $q_t^{(\alpha)}f(x)$ instead of $q_t^{(\alpha),0}f(x)$. Then $q_t^{(\alpha),\kappa}f\in \mathscr{B}_b^*(X)$ (resp.~$q_t^{(\alpha),\kappa}f\in \mathscr{B}_+^*(X)$) under $f\in \mathscr{B}_b^*(X)$ (resp.~$f\in \mathscr{B}_+^*(X)$).  
It is easy to see $q_t^{(\alpha),\kappa}1(x)\leq Ce^{-\sqrt{\alpha-C_{\kappa}}t}$ ($\alpha\geq C_{\kappa}$) and 
$q_t^{(\alpha)}1(x)\leq e^{-\sqrt{\alpha}t}$ ($\alpha\geq0$). 
Take $f\in L^2(X;\m)\cap \mathscr{B}_b^*(X)$ or $f\in L^2(X;\m)\cap \mathscr{B}_+^*(X)$.  
According to the equation 
\begin{align}
(q_t^{(\alpha),\kappa}f,g)_{\m}&=\int_0^{\infty}e^{-\alpha s}(p_s^{\kappa}f,g)_{\m}\lambda_t(\d s)\notag\\
&=\int_0^{\infty}e^{-\alpha s}(P_s^{\kappa}f,g)_{\m}\lambda_t(\d s)\label{eq:Version}\\
&=(Q_t^{(\alpha),\kappa}f,g)_{\m}\quad\text{ for any }\quad g\in L^2(X;\m),\notag
\end{align}
$q_t^{(\alpha),\kappa}f$ is an $\m$-version of $Q_t^{(\alpha),\kappa}f$.
Now we fix a function $f\in D(\Delta^{\kappa})
\cap \mathscr{B}^*(X)$. 
We  set $u(x,a):=q_a^{(\alpha),\kappa}f(x)$ $(\alpha\geq C_{\kappa})$. Then it holds that 
\begin{align}
&a\mapsto u(\cdot,a)\quad\text{ is an $L^2(X;\m)$-valued smooth function, and }\notag\\
&\left(\frac{\partial^2}{\partial a^2}+\Delta^{\kappa}
-\alpha \right)u(\cdot,a)=0\quad\text{ in }\quad L^2(X;\m).\label{eq:recursive}
\end{align}
Furthermore for $a\in\R$, we consider $v(x,a):=u(x,|a|)=q_{|a|}^{(\alpha),\kappa
}f(x)$ for $\alpha>C_{\kappa}$. 
Then by \eqref{eq:Contra}, we have 
\begin{align}
\|v\|_{L^2(\widehat{X};\widehat{\m})}\leq \left(\int_{\R}e^{-2\sqrt{\alpha-C_{\kappa}
}|a|}\|f\|_{L^2(X;\m)}^2 \d a\right)^{1/2}=(
\alpha-C_{\kappa}
)
^{-1/4}\|f\|_{L^2(X;\m)}.\label{eq:L^2Est}
\end{align}
The main purpose of this subsection is the semimartingale decomposition of $\tilde{v}(\widehat{X}_{t\land\tau})$, $t\geq0$, where $\tau:=\inf\{t>0\mid B_t=0\}$. As 
the first step, we give the following: 
\begin{lem}\label{lem:Domain}
For $f\in D(\Delta^{\kappa})
\cap \mathscr{B}^*(X)$ and $\alpha>C_{\kappa}$, 
$v(x,a):=q_{|a|}^{(\alpha),\kappa}f(x)$ satisfies 
$v\in D(\widehat{\mathscr{E}})$. 
\end{lem}
\begin{proof}[\bf Proof]  
At the beginning, we note that $L^2(\widehat{X};\widehat{\m})\cong L^2(\R,L^2(X;\m);m)$. 
According to Fubini's theorem, we have 
\begin{align}
\widehat{P}_t^{\,\widehat{\kappa}}
v(\hat{x})=\E_{\hat{x}}\left[e^{-A_t^{\kappa}}
u(X_t,|B_t|) \right]=\E_x^{\uparrow}\left[e^{-A_t^{\kappa}}
\E_a^{\rightarrow}[u(\cdot,|B_t|)] \right].\label{eq:Semigroup}
\end{align}
We recall Tanaka's formula
\begin{align*}
|B_t|=|B_0|+\int_0^t{\rm sgn}(B_s)\d B_s+L_t^0,\quad t\geq0,\quad \P_a^{\rightarrow}\text{-a.s.,}
\end{align*}
where $(L_t^0)_{t\geq0}$ is the local time of one-dimensional Brownian motion $(B_t)_{t\geq0}$ at the origin. Then by using It\^o's formula, we have 
\begin{align}
u(\cdot,|B_t|)&=u(\cdot,|B_0|)+\int_0^t
\frac{\partial u}{\partial a}(\cdot, |B_s|){\rm sgn}(B_s)\d B_s
\label{eq:Ito}\\
&\hspace{1cm}+\int_0^t\frac{\partial u}{\partial a}(\cdot, |B_s|){\rm sgn}(B_s)\d L_s^0+
\int_0^t\frac{\partial^2 u}{\partial a^2}(\cdot,|B_s|)\d s\notag\\
&=u(\cdot,|B_0|)-\int_0^t\sqrt{\alpha-\Delta^{\kappa}
} u(\cdot,|B_s|){\rm sgn}(B_s)\d B_s\notag\\
&\hspace{1cm}-\int_0^t\sqrt{\alpha-\Delta^{\kappa}
} u(\cdot,|B_s|){\rm sgn}(B_s)\d L_s^0\notag\\
&\hspace{1cm}+\int_0^t(\alpha-\Delta^{\kappa}
)u(\cdot,|B_s|)\d s.
\notag
\end{align}
Hence \eqref{eq:Ito} leads us to 
\begin{align}
\E_a^{\rightarrow}[u(\cdot,|B_t|)]&=u(\cdot,|a|)-\E_a^{\rightarrow}\left[\int_0^t\sqrt{\alpha-\Delta^{\kappa}
}u(\cdot,|B_s|)\d L_s^0 \right]\label{eq:OneIto}\\
&\hspace{1cm}+\E_a^{\rightarrow}\left[\int_0^t(\alpha-\Delta^{\kappa}
)u(\cdot,|B_s|)\d s \right].\notag
\end{align}
On the other hand, $u(\cdot,|a|)=q_{|a|}^{(\alpha),\kappa
}f(\cdot)\in D(\Delta^{\kappa}
)$ in view of 
\eqref{eq:recursive}. Hence
\begin{align*}
M_t^{[u(\cdot,|a|)]}:=e^{-A_t^{\kappa}}
(q_{|a|}^{(\alpha),\kappa
}f)(X_t)-(q_{|a|}^{(\alpha),\kappa
}f)(X_0)
-\int_0^te^{-A_s^{\kappa}}
\Delta^{\kappa}
(q_{|a|}^{(\alpha),\kappa
}f)(X_s)\d s, \quad t\geq0,
\end{align*} 
is an $L^2(\P_x^{\uparrow})$-martingale. Then we have
\begin{align}
\E_x^{\uparrow}\left[e^{-A_t^{\kappa}}
u(X_t,|a|) \right]&=(q_{|a|}^{(\alpha),\kappa
})f(x)+\int_0^tp_s^{\kappa}
(\Delta^{\kappa}
q_{|a|}^{(\alpha),\kappa
}f)(x)\d s, \quad\m\text{-a.e.~}x\in X.
\label{eq:VerticlaEq}
\end{align}
By summarizing \eqref{eq:Semigroup}, \eqref{eq:OneIto} and \eqref{eq:VerticlaEq}, we can proceed as 
\begin{align}
\frac{1}{t}&(v-\widehat{P}_t^{\widehat{\;\kappa}}
v,v)_{L^2(\widehat{X};\widehat{\m})}\notag\\
&=-\frac{1}{t}\int_{\R}\d a\int_X\left\{\int_0^t P_s^{\kappa}
(\Delta^{\kappa}
Q_{|a|}^{(\alpha),\kappa
}f)(x)\d s \right\}\cdot Q_{|a|}^{(\alpha),\kappa
}f(x)\m(\d x)\notag\\
&\hspace{1cm}+
\frac{1}{t}\int_{\R}\d a\int_X\E_x^{\uparrow}\left[e^{-A_t^{\kappa}}
\E_a^{\rightarrow}\left[
\int_0^t\sqrt{\alpha-\Delta^{\kappa}
}u(\cdot,|B_s|)\d L_s^0 \right](X_t) \right]
\cdot Q_{|a|}^{(\alpha),\kappa
}f(x)\m(\d x)\notag\\
&\hspace{1cm}-
\frac{1}{t}\int_{\R}\d a\int_X\E_x^{\uparrow}\left[e^{-A_t^{\kappa}}
\E_a^{\rightarrow}\left[
\int_0^t(\alpha-\Delta^{\kappa}
)u(\cdot,|B_s|)\d L_s^0 \right](X_t) \right]
\cdot Q_{|a|}^{(\alpha),\kappa
}f(x)\m(\d x)
\notag\\
&=-\frac{1}{t}\int_{\R}\d a\int_0^t(P_s^{\kappa}
\Delta^{\kappa}
Q_{|a|}^{(\alpha),\kappa
}f,Q_{|a|}^{(\alpha),\kappa
}f)_{L^2(X;\m)}\d s\label{eq;Dirichlet}\\
&\hspace{1cm}+
\frac{1}{t}\int_{\R}\d a\int_X\E_a^{\rightarrow}\left[\int_0^t\sqrt{\alpha-\Delta^{\kappa}
}u(x,|B_s|)\d L_s^0 \right]P_t^{\kappa}
(Q_{|a|}^{(\alpha),\kappa
}f)(x)\m(\d x)\notag\\
&\hspace{1cm}-
\frac{1}{t}\int_{\R}\d a\int_X\E_a^{\rightarrow}\left[\int_0^t(\alpha-\Delta^{\kappa}
)u(x,|B_s|)\d s \right]P_t^{\kappa}(Q_{|a|}^{(\alpha),\kappa
}f)(x)\m(\d x)\notag\\
&=:-I_1(t)+I_2(t)-I_3(t),\notag
\end{align}
where we used the symmetry of $(P_t^{\kappa}
)_{t\geq0}$ on $L^2(X;\m)$. For the terms $I_1(t)$ and $I_2(t)$,  we see the following estimates by using the contractivity \eqref{eq:KatoContraction} 
of $(P_t^{\kappa}
)_{t\geq0}$ on $L^2(X;\m)$ and  \eqref{eq:Contra}:
\begin{align}
|I_1(t)|&\leq\frac{1}{t}\int_{\R}\d a\int_0^tCe^{C_{\kappa}s}
\|\Delta^{\kappa}Q_{|a|}^{(\alpha),\kappa
}f\|_{L^2(X;\m)}\cdot\|Q_{|a|}^{(\alpha),\kappa
}f \|_{L^2(X;\m)}\d s\notag\\
&\leq 
\left(\frac{C
}{t}\int_0^te^{{C_{\kappa}s}}
\d s \right)
\int_{\R}e^{-2\sqrt{\alpha-C_{\kappa}
}|a|}
\|\Delta^{\kappa}
f\|_{L^2(X;\m)}\cdot\|f\|_{L^2(X;\m)}\d a\label{eq:I_1}\\
&\leq 
\left(\frac{C
}{t}\int_0^te^{{C_{\kappa}s}}
\d s \right)
\frac{1}{\sqrt{\alpha-C_{\kappa}
}}
\|\Delta^{\kappa}
f\|_{L^2(X;\m)}\cdot\|f\|_{L^2(X;\m)},\notag
\end{align}
\begin{align*}
|I_2(t)|&=\left|\frac{1}{t}\int_{\R}\d a\int_X\left( \sqrt{\alpha-\Delta^{\kappa}
}u(x,0)\E_a^{\rightarrow}\left[L_t^0 \right]\right)\cdot P_t^{\kappa}
(Q_{|a|}^{(\alpha),\kappa
}f)(x)\m(\d x) \right|\\
&=\frac{1}{t}\left|\int_{\R}(\sqrt{\alpha-\Delta^{\kappa}
}f, P_t^{\kappa}
Q_{|a|}^{(\alpha),\kappa
}f)_{L^2(X;\m)}\E_a^{\rightarrow}[L_t^0]\d a \right|\\
&\leq \frac{2C
}{t}e^{C_{\kappa}t}
\|\sqrt{\alpha-\Delta^{\kappa}
}f\|_{L^2(X;\m)}\cdot\|f\|_{L^2(X;\m)}
\int_0^{\infty}e^{-\sqrt{\alpha-C_{\kappa}
}a}\E_a^{\rightarrow}[L_t^0]\d a.
\end{align*}
Here we recall 
\begin{align*}
\P_a^{\rightarrow}(L_t^r\in\d y)=\frac{1}{\sqrt{\pi t}}\exp\left( -\frac{y+|r-a|^2}{4t}\right)\d y,\qquad y>0.
\end{align*}
(see \cite[p.~155]{BorodinSalminen}). Then we can continue as 
\begin{align}
|I_2(t)|&\leq \frac{2C
}{t}e^{C_{\kappa}t}
\|\sqrt{\alpha-\Delta^{\kappa}
}f\|_{L^2(X;\m)}\cdot\|f\|_{L^2(X;\m)}\notag\\
&\hspace{1cm}\times \int_0^{\infty}e^{-\sqrt{\alpha-C_{\kappa}
}a}\left\{\int_0^{\infty}y\frac{1}{\sqrt{\pi t}}\exp\left(-\frac{(a+y)^2}{4t} \right)\d y \right\}\d a\notag\\
&\leq 8Ce^{C_{\kappa}t}
\|\sqrt{\alpha-\Delta^{\kappa}
}f\|_{L^2(X;\m)}
\cdot
\|f\|_{L^2(X;\m)}\label{eq:I_2}\\
&\hspace{1cm}\times \int_0^{\infty}\frac{1}{\sqrt{2\pi}}e^{-\frac{a^2}{2}}\d a\int_0^{\infty}y e^{-\frac{y^2}{2}}\d y\notag\\
&=4Ce^{C_{\kappa}t}
\|\sqrt{\alpha-\Delta^{\kappa}
}f\|_{L^2(X;\m)}\cdot\|f\|_{L^2(X;\m)}.\notag
\end{align}
For the term $I_3(t)$, we have
\begin{align}
|I_3(t)|&\leq \frac{Ce^{C_{\kappa}t}
}{t}
\int_{\R}\left\|
\E_a^{\rightarrow}
\left[\int_0^t(\alpha-\Delta^{\kappa}
)u(\cdot,|B_s|)\d s \right]
 \right\|_{L^2(X;\m)} \|Q_{|a|}^{(\alpha),\kappa
 } f \|_{L^2(X;\m)}\d a\notag\\
 &\leq
 \frac{Ce^{C_{\kappa}t}
 }{t}\int_{\R}\E_a^{\rightarrow}\left[\int_0^t\|(\alpha-\Delta^{\kappa}
 )Q_{|B_s|}^{(\alpha),\kappa
 }f(\cdot) \|_{L^2(X;\m)}\d s \right]
\left(e^{-\sqrt{\alpha- C_{\kappa}
}|a|}\|f\|_{L^2(X;\m)} \right)\d a\label{eq:I_3}\\
&\leq \frac{Ce^{C_{\kappa}t}
}{t}\int_{\R}\E_a^{\rightarrow}\left[\int_0^t(\alpha\|f\|_{L^2(X;\m)}+\|\Delta^{\kappa}f
\|_{L^2(X;\m)})\d s \right]
\left(e^{-\sqrt{\alpha- C_{\kappa}
}|a|}\|f\|_{L^2(X;\m)} \right)\d a\notag\\
&\leq \frac{2Ce^{C_{\kappa}t}
}{\sqrt{\alpha-C_{\kappa}
}}
(\alpha \|f\|_{L^2(X;\m)}+\|\Delta^{\kappa}
f\|_{L^2(X;\m)})\|f\|_{L^2(X;\m)}\notag
\end{align}
Finally, we substitute estimates \eqref{eq:I_1}, \eqref{eq:I_2} and \eqref{eq:I_3} into \eqref{eq;Dirichlet}. Then we can easily see 
\begin{align*}
\lim_{t\to0}\frac{1}{t}(v-\widehat{P}_t^{\,\widehat{\kappa}}
v,v)_{L^2(\widehat{X};\widehat{\m})}
=\sup_{t>0}\frac{1}{t}(v-\widehat{P}_t^{\,\widehat{\kappa}}
v,v)_{L^2(\widehat{X};\widehat{\m})}<\infty.
\end{align*}
This and \eqref{eq:L^2Est} with \eqref{eq:quadratic} complete the proof. 
\end{proof} 


By Lemma~\ref{lem:Domain}, we can apply Fukushima's decomposition to 
$v(x,a):=q_{|a|}^{(\alpha),\kappa}f(x)$ for $f\in D(\Delta^{\kappa})\cap \mathscr{B}^*(X)$ and $\alpha>C_{\kappa}$. 
That is, there exists 
a martingale additive functional of finite energy $M^{[v]}$ and a continuous additive functional of zero energy $N^{[v]}$ such that 
\begin{align}
\tilde{v}(\widehat{X}_t)-\tilde{v}(\widehat{X}_0)=M_t^{[v]}+N_t^{[v]}\quad t\geq0\quad \P_{\hat{x}}\text{-a.s.~for q.e.~}\hat{x},\label{eq:FukushimaDecomp} 
\end{align}
where $\tilde{v}$ is an $\widehat{\mathscr{E}}$-quasi-continuous $\widehat{\m}$-version of 
$v\in D(\widehat{\mathscr{E}}^{\,\kappa})=
 D(\widehat{\mathscr{E}})$.
See \cite[Theorem~5.2.2]{FOT} for Fukushima's decomposition theorem. Thanks to \cite[Theorem~5.2.3]{FOT},  we know that 
\begin{align}
\langle M^{[v]}\rangle_t=\int_0^t \left\{\Gamma(v,v)(\widehat{X}_s)+\left( \frac{\partial v}{\partial a}(\widehat{X}_s)\right)^2 \right\}\d s,\quad t\geq0.\label{eq:quadraticVariation}
\end{align}
See also \cite[Theorem~5.1.3 and Example~5.2.1]{FOT} for details.  

From now on, we give the explicit expression of $N^{[v]}$. Let us define the signed measure $\nu$ on $\widehat{X}$ by 
\begin{align*}
\nu(\d x\d a):=2\sqrt{\alpha-\Delta^{\kappa}
} v(x,a)\m(\d x)\delta_0(\d a)
\end{align*}
for $\alpha>C_{\kappa}$, 
where $\delta_0$ is Dirac measure on $\R$ with unit mass at origin. The total variation of 
$\nu$ is given by 
\begin{align*}
|\nu|(\d x\d a):=2|\sqrt{\alpha-\Delta^{\kappa}
}v(x,a)|\m(\d x)\delta_0(\d a).
\end{align*}
Note here that $\nu$ depends on $f\in D(\Delta^{\kappa})\cap \mathscr{B}^*(X)$ and $\alpha>C_{\kappa}$. 
Then we have
\begin{lem}\label{lem:finiteEnergy}
Suppose $f\in D(\Delta^{\kappa})\cap \mathscr{B}^*(X)$ and $\alpha>C_{\kappa}$. Then, there 
exists a constant $C>0$ such that 
\begin{align*}
\int_{\widehat{X}}|g\otimes \varphi(x,a)|\cdot|\nu|(\d x\d a)\leq C\sqrt{\widehat{\mathscr{E}}_1(g\otimes\varphi,g\otimes\varphi)}
\end{align*}
for $g\in{\rm Test}
(X)$, $\varphi\in C_c^{\infty}(\R)$. That is, $\nu$ is of finite energy integral. For instance, see \cite[Sections~2.2 and 5.4]{FOT} for the definitions of measures 
of finite energy integrals. 
\end{lem}
\begin{proof}[\bf Proof]  
We take a positive constant $a_0$ such that ${\rm supp}[\varphi]\subset [-a_0,a_0]$. 
We first consider the case of $\varphi(0)\leq0$. Let $\eps>0$. Then for $\m$-a.e.~$x\in X$, we have 
\begin{align*}
\int_{\R}&|\varphi(a)|\sqrt{(\sqrt{\alpha-\Delta^{\kappa}}v(x,a))^2+\eps}\delta_0(\d a)\\
&=-\varphi(0)\sqrt{(\sqrt{\alpha-\Delta^{\kappa}}v(x,0))^2+\eps}\\
&=\varphi(a_0)\sqrt{(\sqrt{\alpha-\Delta^{\kappa}}v(x,a_0))^2+\eps}-\varphi(0)\sqrt{(\sqrt{\alpha-\Delta^{\kappa}}v(x,0))^2+\eps}\\
&=\int_0^{a_0}\frac{\partial}{\partial a}\left\{\varphi(a)\sqrt{(\sqrt{\alpha-\Delta^{\kappa}}v(x,a))^2+\eps} \right\}\d a\\
&=\int_0^{a_0}\varphi'(a)\sqrt{(\sqrt{\alpha-\Delta^{\kappa}}v(x,a))^2+\eps}\d a\\
&\hspace{1cm}-\int_0^{a_0}\varphi(a)\frac{\sqrt{\alpha-\Delta^{\kappa}}v(x,a)\cdot(\alpha-\Delta^{\kappa})v(x,a)}{\sqrt{(\sqrt{\alpha-\Delta^{\kappa}}v(x,a_0))^2+\eps}}\d a\\
&\leq \int_{\R}|\varphi'(a)|\sqrt{(\sqrt{\alpha-\Delta^{\kappa}}v(x,a_0))^2+\eps} \d a+
\int_{\R}|\varphi(a)|\cdot|(\alpha-\Delta^{\kappa})v(x,a)|\d a.
\end{align*}
Therefore
\begin{align*}
\int_{\widehat{X}}&|(g\otimes \varphi)(x,a)|\cdot|\nu|(\d x\d a)\\
&=2\lim_{\eps\to0}\int_X|g(x)|\left(\int_{\R}|\varphi(a)|
\sqrt{(\sqrt{\alpha-\Delta^{\kappa}}v(x,a))^2+\eps}\delta_0(\d a)
 \right)\m(\d x)\\
 &\leq
 2\varlimsup_{\eps\to0}
 \int_X|g(x)|\left(\int_{\R}|\varphi'(a)|
\sqrt{(\sqrt{\alpha-\Delta^{\kappa}}v(x,a))^2+\eps}\d a
 \right)\m(\d x)\\
 &\hspace{1cm}+2\int_X|g(x)|\left(\int_{\R}|\varphi(a)|\cdot|(\alpha-\Delta^{\kappa})v(x,a)|\d a \right)\m(\d x)\\
 &\leq 2\left(
 \|\sqrt{\alpha-\Delta^{\kappa}}v\|_{L^2(\widehat{X};\widehat{\m})}\|\varphi'\|_{L^2(\R)}+
  \|(\alpha-\Delta^{\kappa})v\|_{L^2(\widehat{X};\widehat{\m})}\|\varphi\|_{L^2(\R)}
  \right)\|g\|_{L^2(X;\m)}\\
  &\leq 2\sqrt{2}(\alpha-C_{\kappa})^{-1/4}\left(\|\sqrt{\alpha-\Delta^{\kappa}}v\|_{L^2(\widehat{X};\widehat{\m})}+
  \|(\alpha-\Delta^{\kappa})v\|_{L^2(\widehat{X};\widehat{\m})}\right)\sqrt{\widehat{\mathscr{E}}(g\otimes\varphi,g\otimes\varphi)}\\
  &=:C\sqrt{\widehat{\mathscr{E}}(g\otimes\varphi,g\otimes\varphi)},
\end{align*}
where we used \eqref{eq:L^2Est} and 
\begin{align*}
\widehat{\mathscr{E}}(g\otimes \varphi,g\otimes\varphi)=\mathscr{E}(g,g)\|\varphi\|_{L^2(\R)}^2+\|g\|_{L^2(X;\m)}^2\|\varphi'\|_{L^2(\R)}^2
\end{align*}
for the last line. This is the desired result. 

In the case of $\varphi(0)\geq0$, we easily see 
\begin{align}
\int_{\R}|\varphi(a)|\sqrt{(\sqrt{\alpha-\Delta^{\kappa}}v(x,a))^2+\eps}\delta_0(\d a)=
\int_{-a_0}^0\frac{\partial}{\partial a}\left\{\varphi(a)\sqrt{(\sqrt{\alpha-\Delta^{\kappa}}v(x,a))^2+\eps} \right\}\d a.\label{eq:negativeCase}
\end{align}
By using \eqref{eq:negativeCase}, we can follow the same argument as the case where $\varphi(0)\leq0$. Therefore the proof is completed. 
\end{proof} 

Thanks to Lemma~\ref{lem:finiteEnergy}, $\nu$ is of finite $1$-order energy integral. Then for 
each $\beta>0$, there exists a unique $U_{\beta}\nu\in D(\mathscr{E})$ such that 
the following relation holds: 
\begin{align}
\widehat{\mathscr{E}}_{\beta}(U_{\beta}\nu,g\otimes\varphi)=\int_{\widehat{X}}(g\otimes\varphi)(x,a)\nu(\d x\d a),\quad g\in {\rm Test}(X),\varphi\in C_c^{\infty}(\R).\label{eq:Potential}
\end{align}
\begin{lem}\label{lem:Potential}
We have the following under $f\in D(\Delta^{\kappa})\cap \mathscr{B}^*(X)$, $\alpha>C_{\kappa}$ and $\beta>0$: 
\begin{enumerate}
\item[{\rm (1)}] $U_{\alpha}\nu=v$. 
\item[{\rm (2)}] $U_{\beta}\nu=v-(\beta-\alpha)\widehat{R}_{\beta}v$ holds, where $(\widehat{R}_{\beta})_{\beta>0}$ is the resolvent of $(\widehat{P}_t)_{t\geq0}$. 
\end{enumerate}
\end{lem}
\begin{proof}[\bf Proof]  
(1) We need to show \eqref{eq:Potential}. By using the integration by parts formula, 
for $\m$-a.e.~$x\in X$, we have 
\begin{align}
\int_{\R}&\frac{\partial v}{\partial a}(x,a)\varphi'(a)\d a\label{eq:Poten1}\\
&=-\int_0^{\infty}\sqrt{\alpha-\Delta^{\kappa}}u(x,a)\varphi'(a)\d a+\int_0^{\infty}
\sqrt{\alpha-\Delta^{\kappa}}u(x,a)\varphi'(-a)\d a\notag\\
&=-\int_0^{\infty}\sqrt{\alpha-\Delta^{\kappa}}u(x,a)\frac{\d}{\d a}(\varphi(a)+\varphi(-a))\d a\notag\\
&=2\sqrt{\alpha-\Delta^{\kappa}}u(x,0)\varphi(0)+\int_0^{\infty}\frac{\partial}{\partial a}
\sqrt{\alpha-\Delta^{\kappa}}u(x,a)(\varphi(a)+\varphi(-a))\d a\notag\\
&=2\sqrt{\alpha-\Delta^{\kappa}}u(x,0)\varphi(0)-
\int_0^{\infty}(\alpha-\Delta^{\kappa})u(x,a)(\varphi(a)+\varphi(-a))\d a\notag\\
&=2\sqrt{\alpha-\Delta^{\kappa}}v(x,0)\varphi(0)-
\int_{\R}(\alpha-\Delta^{\kappa})v(x,a)\varphi(a)\d a.\notag
\end{align}
Then \eqref{eq:Poten1} leads us to our desired equality as follows: 
\begin{align*}
\widehat{\mathscr{E}}_{\alpha}(v,g\otimes\varphi)&=\int_{\R}\d a\varphi(a)
\int_X\sqrt{\alpha-\Delta^{\kappa}}v(x,a)\sqrt{\alpha-\Delta^{\kappa}}g(x)\m(\d x)\\
&\hspace{1cm}+
\int_X\m(\d x)g(x)\left(2\sqrt{\alpha-\Delta^{\kappa}}v(x,0)\varphi(0)-\int_{\R}(\alpha-\Delta^{\kappa})v(x,a)\varphi(a)\d a \right)\\
&=2\int_X\sqrt{\alpha-\Delta^{\kappa}}v(x,0)g(x)\varphi(0)\m(\d x)\\
&=\int_{\widehat{X}}(g\otimes\varphi)(x,a)\nu(\d x\d a).
\end{align*}
(2) We recall $\widehat{\mathscr{E}}_{\beta}(\widehat{R}_{\beta}v,g\otimes\varphi)=(v,g\otimes\varphi)_{L^2(\widehat{X};\widehat{\m})}$. Then we have 
\begin{align*}
\widehat{\mathscr{E}}_{\beta}(v-(\beta-\alpha)\widehat{R}_{\beta}v,g\otimes\varphi)&=
\widehat{\mathscr{E}}_{\beta}(v,g\otimes\varphi)-(\beta-\alpha)\cdot(v,g\otimes\varphi)_{L^2(\widehat{X};\widehat{\m})}\\
&=\widehat{\mathscr{E}}_{\alpha}(v,g\otimes\varphi)\\
&=\int_{\widehat{X}}(g\otimes\varphi)(x,a)\nu(\d x\d a),
\end{align*}
where we used (1) for the last line. Hence the proof of (2) is now completed. 
\end{proof} 
Thanks to \cite[Lemma~5.4.1]{FOT} and Lemma~\ref{lem:Potential}, we have 
\begin{align*}
N_t^{[v]}=\alpha\int_0^t\tilde{v}(\widehat{X}_s)\d s-A_t^{\nu},\quad t\geq0,
\end{align*}
where $\tilde{v}$ is an $\widehat{\mathscr{E}}$-quasi-continuous $\widehat{\m}$-version of $v$ and $A^{\nu}$ is the CAF corresponding to $\nu$. Since $\nu$ does not charge out of $X\times\{0\}$, due to \cite[Theorem~5.1.5]{FOT}, $A_{t\land\tau}^{\nu}=0$ holds. Thus we get
\begin{align}
N^{[v]}_{t\land\tau}=\alpha\int_0^{t\land\tau}\tilde{v}(\widehat{X}_s)\d s.\label{eq:CAFzero}
\end{align}
By summarizing \eqref{eq:FukushimaDecomp}, \eqref{eq:quadraticVariation} and \eqref{eq:CAFzero}, 
we have the following semi-martingale decomposition which plays a crucial role later. 

\begin{prop}\label{prop:semimartingale}
Suppose $\alpha>C_{\kappa}$, $f\in D(\Delta^{\kappa})\cap \mathscr{B}^*(X)$ and set $v(x,a):=
q_{|a|}^{(\alpha),\kappa}f(x)$ for $(x,a)\in \widehat{X}$. Then
we have the semi-martingale decomposition
\begin{align}
\tilde{v}(\widehat{X}_{t\land\tau})-\tilde{v}(\widehat{X}_0)=M_{t\land\tau}^{[v]}+\alpha\int_0^{t\land\tau}\tilde{v}(\widehat{X}_s)\d s,\quad t\geq0,\label{eq:semimartingale1}
\end{align}
under $\P_{\hat{x}}$ for q.e.~$\hat{x}=(x,a)$. Moreover it holds
\begin{align}
\langle M^{[v]}\rangle_{t\land\tau}&=\int_0^{t\land\tau}\left\{\Gamma(v,v)(\widehat{X}_s)+
\left(\frac{\partial v}{\partial a}(\widehat{X}_s) \right)^2 \right\}\d s.\label{eq:semimartingale2}
\end{align}
In particular, by setting $M_t:=M^{[v]}_{t\land\tau}$, 
\begin{align}
\E_{(x,a)}[\langle M\rangle_{\infty}]&= \E_{(x,a)}
\left[\int_0^{\tau}\left\{\Gamma(v,v)+\left(\frac{\partial v}{\partial a}\right)^2\right\}(\widehat{X}_s)
\d s \right]<\infty\label{eq:semimartingale3}
\end{align}
for $\widehat{\m}$-a.e.~$(x,a)\in \widehat{G}:=X\times]0,+\infty[$, 
because the absorbing process $\widehat{\bf X}_{\widehat{G}}$ on 
$\widehat{G}$ is an $\widehat{\m}$-symmetric transient process and 
$\Gamma(v)+\left(\frac{\partial v}{\partial a} \right)^2\in L^1(\widehat{G};\widehat{\m})$.
\end{prop}

Since $v(x,a)=u(x,a)$ holds for $a\geq0$, this proposition also gives the semi-martingale decomposition 
of $u(\widehat{X}_{t\land\tau})$. 

Before closing this subsection, we need the following lemma to allow $\m\otimes\delta_a$ is an 
initial distribution. 

\begin{lem}\label{lem:initialDist}
$\m\otimes\delta_a$ does not charge any set of zero capacity 
with respect to $(\widehat{\mathscr{E}},D(\widehat{\mathscr{E}}))$ for $m$-a.e.~$a\in\R$. 
\end{lem}
\begin{proof}[\bf Proof]  
Let $N\subset \widehat{X}$ be a set of zero capacity with respect to $(\widehat{\mathscr{E}},D(\widehat{\mathscr{E}}))$. Then by $\widehat{\rm O}$kura~\cite[Theorem~4.1(4)]{Okura}, 
$N_a$ is a set of zero capacity with respect to $(\mathscr{E},D(\mathscr{E}))$ for $m$-a.e.~$a\in \R$, where the set $N_a\subset X$ is defined by $N_a:=\{x\in X\mid (x,a)\in N\}$, $a\in\R$. Thus we have
\begin{align*}
\widehat{\m}(N)=\m(N_a)\leq{\rm Cap}_{\mathscr{E}_1}(N_a)=0.
\end{align*}
\end{proof} 



\subsection{Proof of \eqref{eq:LittlewoodPaleyStein1} for $f\in D(\Delta)\cap L^p(X;\m)\cap \mathscr{B}^*(X)$ under  $p\in]1,2[$}
\label{subsec:3.2}
In this subsection, we return to the proof of the upper estimate \eqref{eq:LittlewoodPaleyStein1} in Theorem~\ref{thm:main1} in the case of $1<p<2$.
Recall the $\widehat{\m}$-symmetric diffusion process $\widehat{\bf X}=(\widehat{X}_t,\P_{\hat{x}})$ on $\widehat{X}:=X\times\R$ with 
$\widehat{X}_t:=(X_t, B_t)$.  
We need the following identity for our later use. 
See Shigekawa~\cite[Proposition~3.10]{ShigekawaText} for the proof. 


\begin{lem}[{Shigekawa~\cite[Proposition~3.10]{ShigekawaText}}]\label{lem:ShigekawaIdntity}
Let $j:X\times[0,+\infty[\to[0,+\infty[$ be a measurable function. Then 
\begin{align}
\E_{\m\otimes\delta_a}\left[\int_0^{\tau}j(\widehat{X}_s)\d s \right]=\int_X\m(\d x)\int_0^{\infty}(a\land t)j(x,t)\d t.\label{eq:Shigekawa3.10}
\end{align}
\end{lem} 

Since $\{X_t\}_{t\geq0}$ and $\{B_t\}_{t\geq0}$ are mutually independent under $\E_{\m\otimes\delta_a}$ 
and $\m$ is the invariant measure of $\{X_t\}_{t\geq0}$, we can see the following identity for any 
$h\in \mathscr{B}_b(X)$:
\begin{align}
\E_{\m\otimes\delta_a}[h(X_{\tau})]=\int_X h(x)\m(\d x)\label{eq:invariant}
\end{align}
(see \cite[(3.47)]{ShigekawaText} for the proof. In \cite{ShigekawaText}, $\m\in\mathscr{P}(X)$ is assumed, but 
\cite[Proposition~3.10]{ShigekawaText} remains valid for general $\sigma$-finite $\m$). 

Now, we consider the case $\kappa=0$ and $v(x,a):=u(x,|a|):=q_{|a|}^{(\alpha)}f(x)$ for 
$f\in D(\Delta)\cap L^p(X;\m)\cap \mathscr{B}^*(X)$ and $\alpha>0$. 
Applying Lemma~\ref{lem:Domain}, $v\in D(\widehat{\mathscr{E}})$. 
We abbreviate $M_{t\land \tau}^{[v]}$ as $M_t$ for simplicity.
Combining Proposition~\ref{prop:semimartingale} and Lemma~\ref{lem:initialDist} under $\kappa=0$ with $\alpha>0$, there exists a non-negative sequence $\{a_n\}_{n\in\mathbb{N}}$ 
such that $\lim_{n\to\infty}a_n=+\infty$, \eqref{eq:semimartingale1} and \eqref{eq:semimartingale2} hold under $\P_{\m\otimes\delta_{a_n}}$ for any $n\in\mathbb{N}$. 

We set $V_t:=\tilde{v}(\widehat{X}_{t\land\tau})$. We apply It\^o's formula to $V_t^2$. Proposition~\ref{prop:semimartingale} implies 
\begin{align}
\d (V_t)^2&=2V_t\d M_t+2\alpha V_t^2\d t+ \d\langle M\rangle_t\notag\\
&= 2V_t\d M_t+2(g_f(\widehat{X}_t)^2+\alpha V_t^2)\d t.\label{eq:VSquare}
\end{align}
Let $\eps>0$. By applying It\^o's formula to 
$(V_t^2+\eps)^{p/2}$ again, we also have
\begin{align*}
\d(V_t^2+\eps)^{\frac{p}{2}}&=p(V_t^2+\eps)^{\frac{p}{2}-1}V_t\d M_t\\
&\hspace{1cm}+p(V_t^2+\eps)^{\frac{p}{2}-1}\left(g_f(\widehat{X}_t)^2+\alpha V_t^2 \right)\d t\\
&\hspace{2cm}+\frac{p(p-2)}{2}(V_t^2+\eps)^{\frac{p}{2}-2}V_t^2\d\langle M\rangle_t\\
&\geq p(V_t^2+\eps)^{\frac{p}{2}-1} V_t\d M_t+p(p-1)(V_t^2+\eps)^{\frac{p}{2}-1}g_f(\widehat{X}_t)^2\d t,
\end{align*}
where we used $p<2$ for the last line. 


Hence, by taking the expectation of the inequality above and using 
$u(x,a)=v(x,a)$ for $a\geq0$, we have 
\begin{align}
\E_{\m\otimes\delta_a}\left[p(p-1)\int_0^{\tau}(V_t^2+\eps)^{\frac{p}{2}-1}g_f(\widehat{X}_t)^2\d t \right]&\leq 
\E_{\m\otimes\delta_a}\left[(V_{\tau}^2+\eps)^{\frac{p}{2}}-(V_0^2+\eps)^{\frac{p}{2}} \right]\notag\\
&\leq \E_{\m\otimes\delta_a}\left[(V_{\tau}^2+\eps)^{\frac{p}{2}}\right]\notag\\
&=\E_{\m\otimes\delta_a}\left[\left(\tilde{u}(\widehat{X}_{\tau})^2+\eps\right)^{\frac{p}{2}} \right]\label{eq:VarVar}\\
&=\E_{\m\otimes\delta_a}\left[\left(f(X_{\tau})^2+\eps\right)^{\frac{p}{2}} \right]\notag\\&=
\int_X(|f(x)|^2+\eps)^{\frac{p}{2}}\m(\d x),\notag
\end{align}
where we used \eqref{eq:invariant} for the last line. Here, by recalling \eqref{eq:Shigekawa3.10}, the left hand side of \eqref{eq:VarVar} is equal to 
\begin{align*}
p(p-1)\int_X\m(\d x)\int_0^{\infty}(t\land a_n)(u(x,t)^2+\eps)^{\frac{p}{2}-1}g_f(x,t)^2\d t.
\end{align*}
Therefore, by letting $\eps\to0$ and $n\to\infty$, we have 
\begin{align}
p(p-1)\int_X\m(\d x)\int_0^{\infty}t u(x,t)^{p-2}g_f(x,t)^2\d t\leq\int_X|f(x)|^p\m(\d x).\label{eq:Desired}
\end{align}

Now, we recall the maximal ergodic inequality (see Shigekawa~\cite[Theorem~3.3]{ShigekawaText} for details)
\begin{align}
\left\| \sup_{t\geq0}|P_tf|\right\|_{L^p(X;\m)}\leq \frac{p}{p-1}\|f\|_{L^p(X;\m)},\qquad p>1.\label{eq:Desired*}
\end{align}
It leads us to
\begin{align}
\|G_f\|_{L^p(X;\m)}^p&=\int_X\m(\d x)\left\{\int_0^{\infty}t|u(x,t)|^{2-p}
|u(x,t)|^{p-2}g_f(x,t)^2\d t\right\}^{\frac{p}{2}}\notag \\
&\leq \int_X\m(\d x)\left\{\int_0^{\infty}t\left(\sup_{t\geq0}|P_tf(x)| \right)^{2-p}
|u(x,t)|^{p-2}g_f(x,t)^2\d t\right\}^{\frac{p}{2}}\notag\\
&\leq \left\{\int_X\left(\sup_{t\geq0}|P_tf(x)| \right)^{p}\m(\d x) \right\}^{\frac{2-p}{2}}\notag\\
&\hspace{1cm}\times
\left\{\int_X\int_0^{\infty}t|u(x,t)|^{p-2}g_f(x,t)^2\d t\m(\d x) \right\}^{\frac{p}{2}}\label{eq:UpperG} \\
&\hspace{-0.6cm}\stackrel{\eqref{eq:Desired},\eqref{eq:Desired*}}{\leq}\frac{p^{\frac{p(1-p)}{2}}}{(p-1)^{\frac{p(3-p)}{2}}}
\left\{\int_X|f(x)|^p\m(\d x) \right\}^{\frac{2-p}{2}}
\left\{\int_X|f(x)|^p\m(\d x) \right\}^{\frac{p}{2}}\notag\\
&=\frac{p^{\frac{p(1-p)}{2}}}{(p-1)^{\frac{p(3-p)}{2}}}
\|f\|_{L^p(X;\m)}^p.\notag
\end{align}
Thus \eqref{eq:LittlewoodPaleyStein1} holds under $p\in]1,2[$, $\alpha>0$ and $f\in D(\Delta)\cap L^p(X;\m)\cap \mathscr{B}^*(X)$. 

Next we prove that \eqref{eq:LittlewoodPaleyStein1} holds 
under $p\in]1,2[$, $\alpha=0$ and $f\in D(\Delta)\cap L^p(X;\m)\cap \mathscr{B}^*(X)$. 
We note that $\Gamma(Q_{\cdot}^{(\alpha)}f)^{\frac12}\to \Gamma(Q_{\cdot}^{(0)}f)^{\frac12}$ in 
$L^2(X\times[0,\infty[:\m\otimes e^{-t}\d t)$ as $\alpha\to0$ for $f\in D(\mathscr{E})$. 
Indeed, 
\begin{align*}
\int_0^{\infty}e^{-t}&\d t\int_X\left|\Gamma(Q_t^{(\alpha)}f)^{\frac12}-\Gamma(Q_t^{(0)}f)^{\frac12} \right|^2\d\m\\
&\leq\int_0^{\infty}e^{-t}\d t\int_X\Gamma(Q_t^{(\alpha)}f-Q_t^{(0)}f)\d\m\\
&=\int_0^{\infty}e^{-t}\d t\int_0^{\infty}
(1+\lambda)\left( e^{-\sqrt{\alpha+\lambda}t}-e^{-\sqrt{\lambda}t}\right)^2\d(E_{\lambda}f,f)\\
&=\int_0^{\infty}(1+\lambda)\left[\frac{1}{2\sqrt{\alpha+\lambda}+1}-\frac{2}{\sqrt{\alpha+\lambda}+\sqrt{\lambda}+1}+\frac{1}{2\sqrt{\lambda}+1} \right]
\d(E_{\lambda}f,f)\\
&\to 0\quad \text{ as }\quad \alpha\downarrow0.
\end{align*}
Moreover, 
$\sqrt{\alpha-\Delta}Q_{\cdot}^{(\alpha)}f\to \sqrt{-\Delta}Q_{\cdot}^{(0)}f$ in $L^2(X\times[0,\infty[;\m\otimes e^{-t}\d t)$ as $\alpha\to0$ for $f\in L^2(X;\m)$. Indeed, 
\begin{align*}
\int_0^{\infty}&e^{-t}\d t\int_X\left|\sqrt{\alpha-\Delta}Q_t^{(\alpha)}f-\sqrt{-\Delta}Q_t^{(0)}f\right|^2\d\m\\
&=\int_0^{\infty}e^{-t}\d t\int_0^{\infty}
\left|\sqrt{\alpha+\lambda}e^{-\sqrt{\alpha+\lambda}t}-\sqrt{\lambda}e^{-\sqrt{\lambda}t} \right|^2\d(E_{\lambda}f,f)\\
&=\int_0^{\infty}\left(\frac{\alpha+\lambda}{1+2\sqrt{\alpha+\lambda}}-\frac{2\sqrt{\alpha+\lambda}\sqrt{\lambda}}{1+\sqrt{\alpha+\lambda}+\sqrt{\lambda}}+\frac{\lambda}{1+2\sqrt{\lambda}} \right)\d(E_{\lambda}f,f)
&\to 0\quad \text{ as }\quad \alpha\downarrow0.
\end{align*}
Then there exists a subsequence $\{\alpha_k\}$ tending to $0$ as $k\to\infty$ such that $\Gamma(Q_t^{(\alpha_k)}f)(x)\to \Gamma(Q_t^{(0)}f)(x)$ and $\sqrt{\alpha_k-\Delta}Q_t^{(\alpha_k)}f(x) \to \sqrt{-\Delta}Q_t^{(0)}f(x)$ as $k\to\infty$ $\widehat{\m}$-a.e.~$(x,t)$.   

Therefore, we can conclude that \eqref{eq:LittlewoodPaleyStein1} holds 
under $p\in]1,2[$, $\alpha=0$ and $f\in D(\Delta)\cap L^p(X;\m)\cap \mathscr{B}^*(X)$ by way of Fatou's lemma and 
the estimate \eqref{eq:UpperG} under $\alpha>0$. 



\subsection{Proof of \eqref{eq:LittlewoodPaleyStein1} for $f\in D(\Delta)\cap L^p(X;\m)\cap \mathscr{B}^*(X)$ under  $p\in]2,+\infty[$}\label{subsec:p>2}
In this subsection, we assume Assumption~\ref{asmp:Tamed}, consequently, 
the estimate \eqref{eq:gradCont} holds for $f\in D(\mathscr{E})$.  
We still assume $\kappa\in S_D({\bf X})$ and $\kappa^-\in S_{E\!K}({\bf X})$. 
In the case of $p>2$, we need additional functions, namely $H$-functions
\begin{align*}
{H_f^{\rightarrow}}^{\kappa}:&=\left\{\int_0^{\infty}tQ_t^{(\alpha),\kappa}(g_f^{\rightarrow}(\cdot,t)^2)(x)\d t \right\}^{\frac12},\\
{H_f^{\uparrow}}^{\kappa}:&=\left\{\int_0^{\infty}tQ_t^{(\alpha),\kappa}(g_f^{\uparrow}(\cdot,t)^2)(x)\d t \right\}^{\frac12},\\
{H_f}^{\kappa}:&=\left\{\int_0^{\infty}tQ_t^{(\alpha),\kappa}(g_f(\cdot,t)^2)(x)\d t \right\}^{\frac12}
\end{align*}
under $\alpha\geq C_{\kappa}$. 
We write ${H_f^{\rightarrow}}:={H_f^{\rightarrow}}^0$, 
${H_f^{\uparrow}}:={H_f^{\uparrow}}^0$ and ${H_f}:={H_f}^0$. 
Note that ${H_f^{\rightarrow}}^{\kappa}$, ${H_f^{\uparrow}}^{\kappa}$ and 
${H_f}^{\kappa}$ (resp.~${H_f^{\rightarrow}}$, ${H_f^{\uparrow}}$ and 
${H_f}$) depend on $\alpha\geq C_{\kappa}$ (resp.~$\alpha\geq0$ if $\kappa^-=0$). 

\medskip

Not only Lemma~\ref{lem:ShigekawaIdntity}, we need the following inequality extending  
\cite[Proposition~3.11]{ShigekawaText}. 
\begin{lem}\label{lem:ShigekawaInequality}
Assume $\kappa^+\in S_D({\bf X})$ and $\kappa^-\in S_{E\!K}({\bf X})$.  
Let $j:X\times[0,+\infty[\to[0,+\infty[$ be a measurable function. Then 
\begin{align}
\E_{\m\otimes\delta_a}\left[\left. \int_0^{\tau}j(\widehat{X}_s)\d s\,\right|\, X_{\tau}\right]
\geq \int_0^{\infty}(a\land t)
Q_t^{(\alpha),\kappa}(j(\cdot,t))(X_{\tau})
\d t\label{eq:ShigekawaProp3.11}
\end{align}
holds for $\alpha\geq C_{\kappa}$. 
\end{lem}
\begin{remark}
{\rm Since $\alpha\geq C_{\kappa}>0$ under $\kappa^-\ne0$, we can not expect the equality in \eqref{eq:ShigekawaProp3.11} under $\alpha=0$.  
}
\end{remark}
\begin{proof}[\bf Proof of Lemma~{\boldmath\ref{lem:ShigekawaInequality}}]
We may assume $\alpha>C_{\kappa}$, because the case $\alpha=C_{\kappa}$ in 
\eqref{eq:ShigekawaProp3.11} 
can be deduced by the limit $\alpha\downarrow C_{\kappa}$.  By taking an increasing approximating sequence $\{j_n\}\subset L^1(X\times[0,+\infty[;\widehat{\m})$ of non-negative measurable functions to $j$, we may assume 
$j\in L^1(X\times[0,+\infty[;\widehat{\m})$. 
Thanks to the transience of $\widehat{\bf X}_{\widehat{G}}$ with the integrability of $j$, we have 
\begin{align}
\E_{(x,a)}\left[\int_0^{\tau}j(\widehat{X}_s)\d s \right]<\infty,\quad \widehat{\m}\text{-a.e.~}(x,a)\label{eq:finiteness}
\end{align} 
by \cite[(1.5.4)]{FOT}. 
To prove \eqref{eq:ShigekawaProp3.11}, it suffices to show 
\begin{align}
\E_{(x,a)}\left[f(X_{\tau})\int_0^{\tau}j(\widehat{X}_s)\d s  \right]\geq
\E_{(x,a)}\left[\int_0^{\tau} 
\tilde{v}(\widehat{X}_t)j(\widehat{X}_t)\d t \right]\quad \widehat{\m}\text{-a.e.~}(x,a).
\label{eq:Shigekawa2.46}
\end{align}
for $v(x,a):=q_{|a|}^{(\alpha),\kappa}f(x)$ with 
$f\in \mathscr{B}_+(X)$ and $\alpha>C_{\kappa}$.
Indeed, by taking an integration of \eqref{eq:Shigekawa2.46} with respect to $\m$, we have
\begin{align*}
\E_{\m\otimes\delta_a}\left[f(X_{\tau})\int_0^{\tau}j(\widehat{X}_s)\d s  \right]&\geq
\E_{\m\otimes\delta_a}\left[\int_0^{\tau}
\tilde{v}(\widehat{X}_t)j(\widehat{X}_t)\d t \right]
\\
&\hspace{-0.2cm}\stackrel{\eqref{eq:Shigekawa3.10}}{=}\int_X\m(\d x)\int_0^{\infty}(a\land t)v(x,t)j(x,t)\d t\\
&=\int_0^{\infty}(a\land t)\d t\int_X
Q_t^{(\alpha),\kappa}f(x)
j(x,t)\m(\d x)\\
&=\int_0^{\infty}(a\land t)\d t\int_X Q_t^{(\alpha),\kappa}(j(\cdot,t))(x)f(x)\m(\d x)
\quad\text{(symmetry)}\\
&=\int_0^{\infty}(a\land t)\E_{\m\otimes\delta_a}\left[Q_t^{(\alpha),\kappa}
(j(\cdot,t))(X_{\tau})
f(X_{\tau}) \right]\d t,
\end{align*}
which implies \eqref{eq:ShigekawaProp3.11}.  
Here we apply \eqref{eq:invariant} in the last equality. 
 

From now on, we prove  \eqref{eq:Shigekawa2.46} for $f\in D(\Delta^{\kappa})\cap\mathscr{B}^*_+(X)$. 
Fix such an $f$ and set $v(x,a):=u(x,|a|):=q_{|a|}^{(\alpha),\kappa}f(x)$ with $\alpha>C_{\kappa}$. 
By Lemma~\ref{lem:Domain}, $v\in D(\widehat{\mathscr{E}}^{\kappa})=D(\widehat{\mathscr{E}})$ for $\alpha>C_{\kappa}$.  As proved in Proposition~\ref{prop:semimartingale}, we have 
the semi-martingale decomposition \eqref{eq:semimartingale1} with \eqref{eq:semimartingale2} and \eqref{eq:semimartingale3}.  We set 
\begin{align*}
M_f(t):&=M_{t\land\tau}^{[v]}+\tilde{v}(\widehat{X}_0)\\
&=\tilde{v}(\widehat{X}_{t\land\tau})-\alpha\int_0^{t\land\tau}\tilde{v}(\widehat{X}_s)\d s,\quad t\geq0\quad\P_{\hat{x}}\text{-a.s.~for q.e.~}\hat{x}.
\end{align*}
Then by \eqref{eq:semimartingale3}, $M_f(t)$ is a uniformly integrable  
martingale with respect to $\P_{(x,a)}$ for q.e.~$(x,a)$. 
As $t\to\infty$, $M_f(t)$ converges to 
\begin{align*}
\tilde{v}(\widehat{X}_{\tau})-\alpha\int_0^{\tau}\tilde{v}(\widehat{X}_s)\d s=f(X_{\tau})-\alpha\int_0^{\tau}\tilde{v}(\widehat{X}_s)\d s.
\end{align*}
As a consequence, $M_f(t)$ is represented by 
\begin{align}
M_f(t)=\E_{(x,a)}\left[\left. f(X_{\tau})-\alpha\int_0^{\tau}\tilde{v}(\widehat{X}_s)\d s \,\right|\,\mathcal{F}_t\right],\quad \P_{(x,a)}\text{-a.s.~for q.e.~}(x,a).\label{eq:closedmartingale}
\end{align}
Thus 
\begin{align*}
\E_{(x,a)}\left[f(X_{\tau})\int_0^{\tau}j(\widehat{X}_s)\d s  \right]&=
\int_0^{\infty}\E_{(x,a)}\left[j(\widehat{X}_t) f(X_{\tau})\1_{\{t\leq\tau\}} \right]\d t\\
&=\int_0^{\infty}\E_{(x,a)}\left[\E_{(x,a)}[f(X_{\tau})\,|\,\mathcal{F}_t]j(\widehat{X}_t)\1_{\{t\leq\tau\}} \right]\d t\\
&\hspace{-0.2cm}\stackrel{\eqref{eq:closedmartingale}}{=}\int_0^{\infty}\E_{(x,a)}\left[\tilde{v}(\widehat{X}_t)j(\widehat{X}_t)\1_{\{t\leq\tau\}} \right]\d t\\
&\hspace{1cm}+\alpha\int_0^{\infty}\E_{(x,a)}\left[\E_{(x,a)}\left[\left.\int_{t\land\tau}^{\tau} \tilde{v}(\widehat{X}_s)\d s\,\right|\,\mathcal{F}_t \right]j(\widehat{X}_t)\1_{\{t\leq\tau\}} \right]\d t\\
&\geq \int_0^{\infty}\E_{(x,a)}\left[\tilde{v}(\widehat{X}_t)j(\widehat{X}_t)\1_{\{t\leq\tau\}} \right]\d t\\
&=\E_{(x,a)}\left[\int_0^{\tau}\tilde{v}(\widehat{X}_t)j(\widehat{X}_t)\d t \right].
\end{align*}
This shows that \eqref{eq:Shigekawa2.46} holds for $f\in D(\Delta^{\kappa})\cap\mathscr{B}_+^*(X)$ with 
$\alpha>C_{\kappa}$. 

Next we prove that \eqref{eq:Shigekawa2.46} holds $f\in L^2(X;\m)\cap \mathscr{B}_b(X)_+$ with 
$\alpha>C_{\kappa}$. For such an $f$, we set 
$f_n:=p_{1/n}^{\kappa}f$. Then $\{f_n\}$ is uniformly bounded and $f_n\in D(\Delta^{\kappa})\cap 
\mathscr{B}^*_b(X)_+$. We already prove that  \eqref{eq:Shigekawa2.46} holds for $f_n$.  
Letting $n\to\infty$ with \eqref{eq:finiteness}, 
we can conclude that \eqref{eq:Shigekawa2.46} holds for $f\in L^2(X;\m)\cap \mathscr{B}_b(X)_+$ by way of Lebesgue's dominated convergence theorem. Finally, approximating $f\in \mathscr{B}(X)_+$ by an increasing sequence in $L^2(X;\m)\cap \mathscr{B}_b(X)_+$, 
we see that  \eqref{eq:Shigekawa2.46} still holds for any $f\in \mathscr{B}(X)_+$ with $\alpha>C_{\kappa}$.
\end{proof}



We begin by the following proposition: 

\begin{prop}\label{prop:ShigekawaYoshida}
For $p>2$, the following inequality holds for any $f\in D(\Delta)\cap L^p(X;\m)\cap\mathscr{B}^*(X)$ with $\alpha\geq C_{\kappa}$ and $\alpha>0$: 
\begin{align*}
\|{H_f}^{\kappa}\|_{L^p(X;\m)}\lesssim \|f\|_{L^p(X;\m)}.
\end{align*}
\end{prop}
\begin{proof}[\bf Proof]  
By a slight modification, we can prove in the same way as the proof of Shigekawa-Yoshida~\cite[Proposition~4.2]{ShigekawaYoshida}. However we give the proof for readers' convenience.

Let us recall that $v(x,a):=q_{|a|}^{(\alpha)}f(x)$ with $f\in D(\Delta)$ satisfies $v\in D(\widehat{\mathscr{E}})$ under $\alpha>0$, 
due to \eqref{eq:VSquare}, we have 
\begin{align}
V_{t\land\tau}^2-V_0^2=2\int_0^{t\land\tau}V_s\d M_s+2\int_0^{t\land\tau}
\left(\alpha V_s^2+g_f(\widehat{X}_s)^2 \right)\d s.\label{eq:VSquareStochas}
\end{align}
Since $A_t:=2\int_0^{t\land\tau}
\left(\alpha V_s^2+g_f(\widehat{X}_s)^2 \right)\d s$, $t\geq0$, is a continuous increasing process, 
\eqref{eq:VSquareStochas} implies that $Z_t:=V_{t\land\tau}^2-V_0^2$, $t\geq0$, is a submartingale. 

Now we need an inequality for submartingale. Let $\{Z_t\}_{t\geq0}$ be a continuous submartingale with the Doob-Meyer decomposition 
$Z_t=M_t+A_t$, where $\{M_t\}_{t\geq0}$ is a continuous martingale and $\{A_t\}_{t\geq0}$ is a continuous increasing process with $A_0=0$. Due to Lenglart-L\'epingle-Pratelli~\cite{LenglartLepinglePratelli}, it holds that 
\begin{align}
\E[A_{\infty}^p]\leq(2p)^p\E\left[\sup_{t\geq0}|Z_t|^p \right], \quad p>1.\label{eq:LLP}
\end{align} 
Then by using \eqref{eq:LLP} and Doob's inequality, we have 
\begin{align}
\E_{\m\otimes\delta_{a_n}}\left[\left\{2\int_0^{\tau}\left(\alpha V_s^2+g_f(\widehat{X}_s)^2 \right)\d s \right\}^{\frac{p}{2}} \right]
&\lesssim\E_{\m\otimes\delta_{a_n}}\left[\sup_{t\geq0}|V_{t\land\tau}^2-V_0^2|^{\frac{p}{2}} \right]\notag\\
&\lesssim\E_{\m\otimes\delta_{a_n}}\left[|V_{\tau}^2-V_0^2|^{\frac{p}{2}} \right]\notag\\
&=\E_{\m\otimes\delta_{a_n}}\left[|{v}(\widehat{X}_{\tau})^2-{v}(\widehat{X}_0)^2|^{\frac{p}{2}} \right]\label{eq:Doob}\\
&=\E_{\m\otimes\delta_{a_n}}
\left[|(Q_0^{(\alpha)}
f(X_{\tau}))^2-
(Q_{a_n}^{(\alpha)}f(X_0))^2|^{\frac{p}{2}}
 \right]\notag\\
&\lesssim \E_{\m\otimes\delta_{a_n}}
\left[|Q_0^{(\alpha)}f(X_{\tau})|^p \right]+
\E_{\m\otimes\delta_{a_n}}\left[|Q_{a_n}^{(\alpha)}f(X_0)|^p \right]\notag\\
&\hspace{-0.2cm}\stackrel{\eqref{eq:invariant}}{=}
\|f\|_{L^p(X;\m)}^p+\|Q_{a_n}^{(\alpha)}f\|_{L^p(X;\m)}^p
\lesssim\|f\|_{L^p(X;\m)}^p.\notag
\end{align}
On the other hand, by using \eqref{eq:ShigekawaProp3.11}, \eqref{eq:Doob} and Jensen's inequality, we have 
\begin{align*}
\|{H_f}^{\!\!\kappa}\|_{L^p(X;\m)}^p&=\left\|\left\{\int_0^{\infty}t Q_t^{(\alpha),\kappa}(g_f(\cdot,t)^2)\d t 
\right\}^{\frac{p}{2}} \right\|_{L^1(X;\m)}\\
&=\lim_{n\to\infty}\left\|
\left\{\int_0^{\infty}(a_n\land t) Q_t^{(\alpha),\kappa}(g_f(\cdot,t)^2)\d t 
\right\}^{\frac{p}{2}}
 \right\|_{L^1(X;\m)}\\
&\hspace{-0.2cm}\stackrel{\eqref{eq:invariant}}{=}
\varliminf_{n\to\infty}\E_{\m\otimes\delta_{a_n}}\left[\left\{\int_0^{\infty}(a_n\land t)Q_t^{(\alpha),\kappa}(g_f(\cdot,t)^2)(X_{\tau})\d t \right\}^{\frac{p}{2}} \right] \notag\\
 &\hspace{-0.2cm}\stackrel{\eqref{eq:ShigekawaProp3.11}}{\leq}
 \varliminf_{n\to\infty}
 \E_{\m\otimes\delta_{a_n}}
 \left[\E_{\m\otimes\delta_{a_n}}
 \left[\left.
\int_0^{\tau}g_f(\widehat{X}_s)^2\d s
\,\right|\,X_{\tau}
\right]^{\frac{p}{2}}
\right]\\
&\leq\varliminf_{n\to\infty}\E_{\m\otimes\delta_{a_n}}\left[\E_{\m\otimes\delta_{a_n}}
\left[\left.\left(\int_0^{\tau}g_f(\widehat{X}_s)^2\d s \right)^{\frac{p}{2}} \,\right|\, X_{\tau} 
\right] \right]\\
&=\varliminf_{n\to\infty}\E_{\m\otimes\delta_{a_n}}\left[
\left(\int_0^{\tau}g_f(\widehat{X}_s)^2\d s \right)^{\frac{p}{2}}\right]\\
&\leq\varliminf_{n\to\infty}\E_{\m\otimes\delta_{a_n}}\left[
\left\{\int_0^{\tau}\left(\alpha V_s^2+g_f(\widehat{X}_s)^2 \right)\d s
 \right\}^{\frac{p}{2}}
\right]\\
&\hspace{-0.2cm}\stackrel{\eqref{eq:Doob}}{\lesssim} \|f\|_{L^p(X;\m)}^p.
\end{align*}
In applying \eqref{eq:ShigekawaProp3.11}, we use $\alpha\geq C_{\kappa}$. 
This completes the proof. 
\end{proof} 

Next we study the relationship between $G$-functions and $H$-functions. In the proof of this proposition, Assumption~\ref{asmp:Tamed} plays a key role. 

\begin{prop}\label{prop:GHEest}
\begin{enumerate}
\item[{\rm (1)}] For any $f\in D(\mathscr{E})$ and $\alpha\geq C_{\kappa}$, the following inequality 
holds: 
\begin{align*}
G_f^{\uparrow}\leq 2\sqrt{C}{H_f^{\uparrow}}^{\kappa},\quad\m\text{-a.e.}
\end{align*} 
\item[{\rm (2)}] For any $f\in D(\mathscr{E})$ and $\alpha\geq0$, the following inequality 
holds: 
\begin{align*}
G_f^{\rightarrow}\leq 2{H_f^{\rightarrow}}.
\end{align*} 
\end{enumerate}
\end{prop}
\begin{proof}[\bf Proof]  
First we show (1). 
For this, we prove 
\begin{align}
\Gamma(Q_t^{(\alpha)}f)^{\frac12}\leq\int_0^{\infty}e^{-\alpha s}\Gamma(P_sf)^{\frac12}\lambda_t(\d s),\quad \m\text{-a.e.}\label{eq:Bochner}
\end{align}
We set a $D(\mathscr{E})$-valued simple function $f_n(s)$ by 
\begin{align*}
f_n(s):=\sum_{k=0}^{2^n\cdot n-1}\1_{[\frac{k}{2^n},\frac{k+1}{2^n}[}(s)P_{\frac{k}{2^n}}f+\1_{[n,\infty[}(s)P_nf.
\end{align*}
Then for each $s\geq0$, $\{f_n(s)\}$ converges to $P_sf$ in $(\mathscr{E},D(\mathscr{E}))$ as $n\to\infty$, consequently 
$\Gamma(f_n(s))^{\frac12}$ converges to $\Gamma(P_sf)^{\frac12}$ in $L^2(X;\m)$ as $n\to\infty$. 
Similarly, setting $Q_t f_n:=\int_0^{\infty}e^{-\alpha s}f_n(s)\lambda_t(\d s)$, 
$\{Q_tf_n\}$  converges to $Q_t^{(\alpha)}f$ in 
 $(\mathscr{E},D(\mathscr{E}))$ as $n\to\infty$, and $\Gamma(Q_t f_n)^{\frac12}$ converges to $\Gamma(Q_t^{(\alpha)}f)^{\frac12}$ in $L^2(X;\m)$ as $n\to\infty$. Since 
\begin{align*}
\Gamma(f_n(s))^{\frac12}=\left\{\begin{array}{cl}\Gamma(P_{\frac{k}{2^n}}f)^{\frac12}, & \quad\text{ if }\quad  s\in[\frac{k}{2^n},\frac{k+1}{2^n}[,\quad k\in\{0,1,\cdots, 2^n\cdot n-1\},\\  \Gamma(P_nf)^{\frac12}, & \quad\text{ if }\quad s\geq n,\end{array}\right.
\end{align*} 
we can see 
\begin{align*}
\Gamma(Q_tf_n)^{\frac12}\leq \int_0^{\infty}e^{-\alpha s}\Gamma(f_n(s))^{\frac12}\lambda_t(\d s).
\end{align*} 
Letting $n\to\infty$, we obtain \eqref{eq:Bochner}.  
By \eqref{eq:Bochner}, the upper estimate \eqref{eq:gradCont} and Schwarz's inequality, we have the following estimate for any $\alpha\geq C_{\kappa}$ and $f\in D(\mathscr{E})$:
\begin{align}
\Gamma(Q_t^{(\alpha)}f)^{\frac12}&\leq\int_0^{\infty}e^{-\alpha s}\Gamma(P_sf)^{\frac12}\lambda_t(\d s)\notag\\
&\leq\int_0^{\infty}e^{-\alpha s}P_s^{\kappa}\Gamma(f)^{\frac12}\lambda_t(\d s)\label{eq:BEQ1}\\
&=Q_t^{(\alpha),\kappa}\Gamma(f)^{\frac12}=q_t^{(\alpha),\kappa}\Gamma^*(f)^{\frac12},\notag
\end{align}
where $\Gamma^*(f)$ is a Borel $\m$-version of $\Gamma(f)$. 
Then \eqref{eq:BEQ1} yields that for each $t\geq0$
\begin{align}
g_f^{\uparrow}(x,2t)^2&=\Gamma(Q_{2t}^{(\alpha)}f)(x)\notag\\
&=\Gamma\left(Q_t^{(\alpha)}(Q_t^{(\alpha)}f)\right)(x)\notag\\
&\leq \left(q_t^{(\alpha),\kappa}\Gamma^*(Q_t^{(\alpha)}f)^{\frac12}(x)\right)^2 \notag\\
&\leq q_t^{(\alpha),\kappa}1(x) \cdot Q_t^{(\alpha),\kappa}\Gamma(Q_t^{(\alpha)}f)(x)\label{eq:BEQ2}\\
&= \left(\int_0^{\infty}e^{-\alpha s}p_s^{\kappa}1(x)\lambda_t(\d s) \right)
Q_t^{(\alpha),\kappa}(g_f^{\uparrow}(\cdot,t)^2)(x)
\notag\\
&\hspace{-0.2cm}\stackrel{\eqref{eq:Contra}}{\leq} Ce^{-\sqrt{\alpha-C_{\kappa}}t}Q_t^{(\alpha),\kappa}(g_f^{\uparrow}(\cdot,t)^2)(x),\quad \m\text{-a.e.~}x\in X.\notag
\end{align}
Since $t\mapsto Q_t^{(\alpha),\kappa}f$ is a $D(\mathscr{E})$-valued continuous function, 
$t\mapsto g_f^{\uparrow}(x,2t)^2=\Gamma(Q_{2t}^{(\alpha)}f)$ is an $L^1(X;\m)$-valued continuous function, moreover, 
$t\mapsto Q_t^{(\alpha),\kappa}(g_f^{\uparrow}(\cdot,2t)^2)$ is an $L^1(X;\m)$-valued continuous function. 
Thus, $g_f^{\uparrow}(x,2t)^2\leq  Q_t^{(\alpha),\kappa}(g_f^{\uparrow}(\cdot,t)^2)(x)$ for all $t\geq0$ $\m$-a.e.~$x\in X$. 
 

Therefore we have 
\begin{align*}
\left(G_f^{\uparrow}(x)\right)^2&=4\int_0^{\infty}tg_f^{\uparrow}(x,2t)^2\d t\\
&\leq 4C\int_0^{\infty}t Q_t^{(\alpha),\kappa}(g_f^{\uparrow}(\cdot,t)^2)(x)\d t=
4C({H_f^{\uparrow}}^{\kappa}(x))^2,\quad \m\text{-a.e.~}x\in X,
\end{align*}
where we changed the variable $t$ in $2t$ in the first line and used \eqref{eq:BEQ2} for the second line. 
Next we show (2). Using the semigroup property of $(Q_t^{(\alpha)})_{t\geq0}$ under $\alpha\geq0$,  we have 
\begin{align*}
Q_{t+s}^{(\alpha)}f=Q_t^{(\alpha)}Q_s^{(\alpha)}f.
\end{align*}
Differentiating with respect to $s$ and setting $s=t$, we have 
\begin{align*}
\left.\frac{\partial}{\partial a}Q_a^{(\alpha)}f\right|_{a=2t} =
Q_t^{(\alpha)}\left.\frac{\partial}{\partial a}Q_a^{(\alpha)}f\right|_{a=t}.
\end{align*}
Therefore, from the sub-Markovian property of $(Q_t^{(\alpha)})_{t\geq0}$
\begin{align*}
g_f^{\rightarrow}(x,2t)^2&=\left(\left.\frac{\partial}{\partial a}Q_a^{(\alpha)}f(x)\right|_{a=2t} \right)^2
=\left(Q_t^{(\alpha)}\left.\frac{\partial}{\partial a}Q_a^{(\alpha)}f(x)\right|_{a=t} \right)^2\\&
\leq Q_t^{(\alpha)}\left(\left|\frac{\partial}{\partial t}Q_t^{(\alpha)}f\right|^2\right)(x)=Q_t^{(\alpha)}(g_f^{\rightarrow}(\cdot,t)^2)(x).
\end{align*}
Integrating this, we have
\begin{align*}
G_f^{\rightarrow}(x)^2=4\int_0^{\infty}tg_f^{\rightarrow}(x,2t)^2\ dt
\leq 4\int_0^{\infty}tQ_t^{(\alpha)}(g_f^{\rightarrow}(\cdot,t))(x)^2\d t=4H_f^{\rightarrow}(x)^2.
\end{align*}
This completes the proof. 
\end{proof} 

Combining Propositions~\ref{prop:ShigekawaYoshida} and \ref{prop:GHEest}, 
\eqref{eq:LittlewoodPaleyStein1} holds for 
$f\in D(\Delta)\cap L^p(X;\m)\cap \mathscr{B}^*(X)$ under $p\in]2,+\infty[$ and $\alpha\geq C_{\kappa}$ with 
$\alpha>0$.  
Under $\kappa^-=0$ and $\alpha=0$, we can see that \eqref{eq:LittlewoodPaleyStein1} holds for 
$f\in D(\Delta)\cap L^p(X;\m)\cap \mathscr{B}^*(X)$ and $p\in]2,+\infty[$ by way of the same argument 
as in the previous subsection.  





\subsection{Proof of the upper estimates \eqref{eq:LittlewoodPaleyStein1} and \eqref{eq:LittlewoodPaleyStein1+} 
under $p=2$}\label{subsec:p=2}
We prove the upper estimates \eqref{eq:LittlewoodPaleyStein1} and \eqref{eq:LittlewoodPaleyStein1+} under $p=2$. 
\begin{prop}\label{prop:GHEestp=2}
When $\alpha>0$,
\begin{align}
2\|G_f^{\rightarrow}\|_{L^2(X;\m)}=\|f\|_{L^2(X;\m)},\quad \text{ for all }\quad f\in L^2(X;\m),\label{eq:EqE+1}\\
2\|G_f^{\uparrow}\|_{L^2(X;\m)}\leq\|f\|_{L^2(X;\m)},\quad \text{ for all }\quad f\in L^2(X;\m).\label{eq:EqE+2}
\end{align}
When $\alpha=0$,
\begin{align}
2\|G_f^{\rightarrow}\|_{L^2(X;\m)}=\|f-E_{o}f\|_{L^2(X;\m)},\quad \text{ for all }\quad f\in L^2(X;\m),\label{eq:EqEo1}\\
2\|G_f^{\uparrow}\|_{L^2(X;\m)}=\|f-E_{o}f\|_{L^2(X;\m)},\quad \text{ for all }\quad f\in L^2(X;\m).\label{eq:EqEo2}
\end{align}
In particular, $2\|G_f\|_{L^2(X;\m)}\leq \|f\|_{L^2(X;\m)}$ for all $f\in L^2(X;\m)$ under $\alpha>0$, 
and $2\|G_f\|_{L^2(X;\m)}\leq \|f-E_{o}f\|_{L^2(X;\m)}$ for all $f\in L^2(X;\m)$ under $\alpha=0$. 
\end{prop}
\begin{proof}[\bf Proof]  
First note that for $f\in L^2(X;\m)$
\begin{align*}
\|G_f^{\rightarrow}\|_{L^2(X;\m)}^2&=\int_X\m(\d x)\int_0^{\infty}t\left|\frac{\partial}{\partial t}Q_t^{(\alpha)}f(x) \right|^2\d t\\
&=\int_0^{\infty}t\d t \left\|\frac{\partial}{\partial t}Q_t^{(\alpha)}f \right\|_{L^2(X;\m)}^2\\
&=\int_0^{\infty} t\d t\int_0^{\infty}(\lambda+\alpha)e^{-2\sqrt{\lambda+\alpha}t}\d(E_{\lambda}f,f)\\
&=\int_0^{\infty}\left\{\int_0^{\infty}t(\lambda+\alpha)e^{-2\sqrt{\lambda+\alpha}t}\d t \right\}\d(E_{\lambda}f,f).
\end{align*}
Now, noticing that 
\begin{align*}
\int_0^{\infty}t(\lambda+\alpha)e^{-2\sqrt{\lambda+\alpha}t}\d t=\left\{\begin{array}{cl}\frac14, & \lambda+\alpha>0, \\0, & \lambda+\alpha=0,\end{array}\right.
\end{align*}
we have 
\begin{align*}
\|G_f^{\rightarrow}\|_{L^2(X;\m)}^2=\left\{\begin{array}{lc}\frac14\|f\|_{L^2(X;\m)}^2, & \alpha>0, \\ \frac14\|f-E_{o}f\|_{L^2(X;\m)}^2, & \alpha=0.\end{array}\right.
\end{align*}
Next, let us consider $G_f^{\uparrow}$: 
\begin{align*}
\|G_f^{\uparrow}\|_{L^2(X;\m)}^2&=\int_X\m(\d x)\int_0^{\infty}t\Gamma(Q_t^{(\alpha)}f)\d t\\
&=\int_0^{\infty} t\d t\int_X\Gamma(Q_t^{(\alpha)}f)\m(\d x)\\
&=\int_0^{\infty} t\d t\int_X(-\Delta Q_t^{(\alpha)})f(x)Q_t^{(\alpha)}f(x)\m(\d x)\\
&=\int_0^{\infty}t \d t\int_0^{\infty}\lambda e^{-2\sqrt{\lambda+\alpha}t}\d (E_{\lambda}f,f)\\
&=\int_{]0,+\infty[}\frac{\lambda}{4(\lambda+\alpha)}\d (E_{\lambda}f,f).
\end{align*}
Thus, $\|G_f^{\uparrow}\|_{L^2(X;\m)}^2\leq \frac14\|f\|_{L^2(X;\m)}^2$ follows when $\alpha>0$ and 
$\|G_f^{\uparrow}\|_{L^2(X;\m)}^2=\frac14\|f-E_{o}f\|_{L^2(X;\m)}^2$ follows when $\alpha=0$. 
Therefore, $\|G_f\|_{L^2(X;\m)}^2=\|G_f^{\rightarrow}\|_{L^2(X;\m)}^2+\|G_f^{\uparrow}\|_{L^2(X;\m)}^2\leq\frac12
\|f\|_{L^2(X;\m)}^2$ for $\alpha>0$ and $\|G_f\|_{L^2(X;\m)}^2=\frac12
\|f-E_{o}f\|_{L^2(X;\m)}^2$ for $\alpha=0$. 
\end{proof} 
\begin{lem}\label{lem:SpectralMeasure}
Suppose $f\in L^2(X;\m)$. Then $E_of\in D(\mathscr{E})$ and $\mathscr{E}(E_of,E_of)=0$. 
Moreover, under $\kappa^-=0$ and $\alpha=0$, $G_{E_of}=0$ for $f\in L^2(X;\m)$. 
\end{lem}
\begin{proof}[\bf Proof]  
In view of the spectral representation, 
\begin{align*}
(T_tE_of,E_o f)&=\int_0^{\infty}e^{\lambda t}\d (E_{\lambda}E_of,E_of)\\
&=\lim_{n\to\infty}\sum_{k=0}^{2^n\cdot n-1}e^{\frac{k}{2^n}t}\int_{[\frac{k}{2^n},\frac{k+1}{2^n}[ }
\d(E_{\lambda}E_of,E_of)+e^{nt}\int_{[n,+\infty[}\d(E_{\lambda}E_of,E_of)
\\
&=\lim_{n\to\infty}\sum_{k=0}^{2^n\cdot n-1}e^{\frac{k}{2^n}t}
(E_{[\frac{k}{2^n},\frac{k+1}{2^n}[}E_{\{0\}}f,E_{\{0\}}f)
+e^{nt}(E_{[n,+\infty[}E_{\{0\}}f,E_{\{0\}}f)
\\
&=(E_{[0,\frac{1}{2^n}[}E_{\{0\}}f,E_{\{0\}}f)
=
(E_{\{0\}}f,E_{\{0\}}f)=\|E_of\|_{L^2(X;\m)}^2,
\end{align*}
which yields the first conclusion. In the same way, we can confirm $\mathscr{E}(Q_t^{(0)}\!E_of,Q_t^{(0)}E_of)=0$ for 
$f\in L^2(X;\m)$. This implies that for each $t>0$
\begin{align*}
\int_Xg_{E_o f}^{\rightarrow}(\cdot,t)^2\d\m&=\int_X|\sqrt{-\Delta}Q_t^{(0)}E_of|^2\d\m=\mathscr{E}(Q_t^{(0)}\!E_of,Q_t^{(0)}\!E_of)=0,\\
\int_Xg_{E_o f}^{\uparrow}(\cdot,t)^2\d\m&=\int_X\Gamma(Q_t^{(0)}\!E_of)\d\m=\mathscr{E}(Q_t^{(0)}\!E_of,Q_t^{(0)}\!E_of)=0,
\end{align*}
which yields the second conclusion.  
\end{proof} 
\subsection{Proof of \eqref{eq:LittlewoodPaleyStein1} 
for general $\alpha\geq C_{\kappa}$ with $\alpha>0$ and 
$p\in]1,+\infty[$}
\begin{proof}[\bf Proof of \boldmath\eqref{eq:LittlewoodPaleyStein1}]
It is clear that Subsections~\ref{subsec:3.2}, \ref{subsec:p>2} and \ref{subsec:p=2}
conclude the desired upper estimate \eqref{eq:LittlewoodPaleyStein1} for $f\in D(\Delta)\cap L^p(X;\m)$ and $p\in]1,+\infty[$. Any $f\in L^p(X;\m)\cap L^2(X;\m)$ can be approximated by $f_n:=P_{1/n}f\in D(\Delta)\cap L^p(X;\m)$ in $L^p$-norm and in $L^2$-norm. We can obtain the upper estimate \eqref{eq:LittlewoodPaleyStein1} for 
$f\in L^p(X;\m)\cap L^2(X;\m)$ under $p\in]1,+\infty[$ and $\alpha\geq C_{\kappa}$ with $\alpha>0$. 
Moreover, an $L^p$-approximation by a sequence in $L^p(X;\m)\cap L^2(X;\m)$ tells us that \eqref{eq:LittlewoodPaleyStein1} 
holds for $f\in L^p(X;\m)$ under $p\in]1,+\infty[$ and $\alpha\geq C_{\kappa}$ with $\alpha>0$. 
Similarly, \eqref{eq:LittlewoodPaleyStein1} holds for $f\in L^p(X;\m)\cap L^2(X;\m)$ under $p\in]1,+\infty[$ and $\alpha=0$ with $\kappa^-=0$. 
\end{proof}

\subsection{Proof of \eqref{eq:LittlewoodPaleyStein2} for general $\alpha\geq C_{\kappa}$ with $\alpha>0$ and 
$p\in]1,+\infty[$}
\begin{proof}[\bf Proof of {\boldmath\eqref{eq:LittlewoodPaleyStein2}}]
The proof of the lower estimate \eqref{eq:LittlewoodPaleyStein2} follows from the upper estimate 
\eqref{eq:LittlewoodPaleyStein1}, 
\eqref{eq:EqE+1} and the 
argument in \cite[Subsection~3.2.11]{ShigekawaText}. 
Moreover, an $L^p$-approximation by a sequence in $L^p(X;\m)\cap L^2(X;\m)$ tells us that \eqref{eq:LittlewoodPaleyStein2} holds for 
$f\in L^p(X;\m)$ under $p\in]1,+\infty[$ and $\alpha\geq C_{\kappa}$ with $\alpha>0$. 
Similarly, \eqref{eq:LittlewoodPaleyStein2} holds for $f\in L^p(X;\m)\cap L^2(X;\m)$ under $p\in]1,+\infty[$ and $\alpha=0$ with $\kappa^-=0$. 
\end{proof}

 

\subsection{Proofs of \eqref{eq:LittlewoodPaleyStein1+}, \eqref{eq:LittlewoodPaleyStein2+} and 
\eqref{eq:LittlewoodPaleyStein3+}
for  $\alpha=0$ and $p\in]1,+\infty[$ 
under $\kappa^-=0$}\label{subsec:LastProof}

We already obtain the estimates \eqref{eq:LittlewoodPaleyStein1+}, 
\eqref{eq:LittlewoodPaleyStein2+} and \eqref{eq:LittlewoodPaleyStein3+} under $p=2$. 
To prove  \eqref{eq:LittlewoodPaleyStein1+}, \eqref{eq:LittlewoodPaleyStein2+} and \eqref{eq:LittlewoodPaleyStein3+} for general $p\in]1,+\infty[$, we need to show $E_of\in L^p(X;\m)$ for 
$f\in  L^2(X;\m)$ at first. 
By Lemma~\ref{lem:SpectralMeasure}, we know $E_of\in D(\mathscr{E})$ and $\mathscr{E}(E_of,E_of)=0$.  When
$(\mathscr{E},D(\mathscr{E}))$ is transient, $E_of=0\in L^p(X;\m)$ in this case. 
When $(\mathscr{E},D(\mathscr{E}))$ is irreducible, then $E_of\equiv c_f\in L^2(X;\m)$. 
If $\m(X)=+\infty$, then $c_f=0$, hence $E_of=0\in L^p(X;\m)$. If $\m(X)<+\infty$, then $E_of=c_f\in L^p(X;\m)$. 

Under $\kappa^-=0$ and $\alpha=0$, we already know that \eqref{eq:LittlewoodPaleyStein1} and 
\eqref{eq:LittlewoodPaleyStein2} hold for $f\in L^p(X;\m)\cap L^2(X;\m)$ with $p\in]1,+\infty[$. Replacing $f$ with $f-E_of$ in \eqref{eq:LittlewoodPaleyStein1} and \eqref{eq:LittlewoodPaleyStein2} and noting $G_{E_of}=0$ by 
Lemma~\ref{lem:SpectralMeasure}, we can conclude that \eqref{eq:LittlewoodPaleyStein1+} and 
\eqref{eq:LittlewoodPaleyStein2+} 
 hold for $f\in L^p(X;\m)\cap L^2(X;\m)$ with $p\in]1,+\infty[$ under $\kappa^-=0$, $\alpha=0$ and the transience or irreducibility. Further, under $\kappa^-=0$ and $\alpha=0$, 
we can directly deduce the lower estimate \eqref{eq:LittlewoodPaleyStein3+} 
as discussed in \cite[Subsection~3.2.11]{ShigekawaText} 
from the upper estimate \eqref{eq:LittlewoodPaleyStein1+} by using \eqref{eq:EqEo2}.
This completes the whole proof of Theorem~\ref{thm:main1}.


\section{Proof of Theorem~\ref{thm:main2}}
The proof of Theorem~\ref{thm:main2} for $q>1$ is quite similar to the proof of 
\cite[Theorem~1.3]{KawabiMiyokawa} 
based on Theorem~\ref{thm:main1}. 
So we omit the details of the proof.
We only mention the well-definedness of $R_{\alpha}^{(q)}(\Delta)f$. Though 
the Riesz operator $f\mapsto R_{\alpha}^{(q)}(\Delta )f$ is not linear, we can confirm the following triangle inequality: for $f_1,f_2\in L^p(X;\m)\cap L^2(X;\m)$
\begin{align*}
\|R_{\alpha}^{(q)}(\Delta )f_1-R_{\alpha}^{(q)}(\Delta )f_2 \|_{L^p(X;\m)}^p&=
\int_X\left|\sqrt{\Gamma((\alpha-\Delta)^{-\frac{q}{2}}f_1)}-\sqrt{\Gamma((\alpha-\Delta)^{-\frac{q}{2}}f_2)} \right|^p\d\m\\
&\leq \int_X\Gamma((\alpha-\Delta)^{-\frac{q}{2}}(f_1-f_2))^{\frac{p}{2}}\d\m\\
&=\|R_{\alpha}^{(q)}(\Delta )(f_1-f_2) \|_{L^p(X;\m)}^p\\
&\leq \|R_{\alpha}^{(q)}(\Delta )\|_{p,p}\|f_1-f_2\|_{L^p(X;\m)},
\end{align*}
which implies the extendability of $R_{\alpha}^{(q)}(\Delta )f$ for $f\in L^p(X;\m)$. 



\section{Examples}\label{sec:examples}
It is well-known that (abstract) Wiener space $(B,H,\mu)$ satisfies Littlewood-Paley-Stein inequality (see Shigekawa~\cite[Chapter~3]{ShigekawaText}). Though our Theorems~\ref{thm:main1} and \ref{thm:main2} are not new for $(B,H,\mu)$, we observe the conditions:   
Let $(\mathscr{E}^{\rm OU},D(\mathscr{E}^{\rm OU}))$ be the Dirichlet form on 
$L^2(B;\mu)$ associated to the Ornstein-Uhlenbeck process ${\bf X}^{\rm OU}$ and $(T_t^{\rm OU})_{t\geq0}$ its associated semigroup on $L^2(B;\mu)$. 
Let $D_H$ be the $H$-derivative, i.e. $\langle D_HF(z), h\rangle_H=\lim_{\eps\to0}\frac{F(z+\eps h)-F(z)}{\eps}$, for a cylindrical function $F$. It is known that  
${\sf BE}_2(1,\infty)$-condition holds for $(B,\mathscr{E}^{\rm OU},\mu)$ (see \cite[13.2]{AES}), that is, 
$(B,\mathscr{E}^{\rm OU},\mu)$ is tamed by $\mu\in S_K({\bf X}^{\rm OU})$. 

Moreover, the contents of Theorems~\ref{thm:main1} and \ref{thm:main2} are proved in Kawabi-Miyokawa~\cite{KawabiMiyokawa} under related different conditions (our Theorems~\ref{thm:main1} and \ref{thm:main2} do not cover the results of  \cite{KawabiMiyokawa}, but are not covered by \cite{KawabiMiyokawa}).  
More concretely, let $E$ be a Hilbert space defined by $E:=L^2(\R,\R^d;e^{-2\lambda\chi(x)}\d x)$ with a fixed $\chi\in C^{\infty}(\R)$ satisfying $\chi(x)=|x|$ for $|x|\geq1$ and another Hilbert space $H:=L^2(\R,\R^d;\d x)$.   
They consider a Dirichlet form $(\mathscr{E}, D(\mathscr{E}))$ on $L^2(E;\mu)$ associated with the diffusion process ${\bf X}$ on an infinite volume 
 path space $C(\R,\R^d)$ with ($U$-)Gibbs measures $\mu$ associated with the (formal) Hamiltonian
 \begin{align*}
 \mathcal{H}(w):=\frac12\int_{\R}|w'(x)|^2_{\R^d}\d x+\int_{\R}U(w(x))\d x, 
 \end{align*}  
 where $U:\R^d\to\R$ is an interaction potential satisfying the conditions {\bf (U1)},{\bf (U2)},{\bf (U3)} depending on 
 constants $K_1\in \R$ and $K_2>0$ (see \cite[4.1]{KawabiMiyokawa}). Then the $L^2$-semigroup $(P_t)_{t\geq0}$ associated to $(\mathscr{E}, D(\mathscr{E}))$ satisfies the following gradient estimate
\begin{align*}
|D(P_tf)(w)|_H\leq e^{K_1t}P_t(|Df|_H)(w)\quad \text{ for }\quad \mu\text{-a.e.~}w\in E,
\end{align*}
which is equivalent to ${\sf BE}_2(-K_1,\infty)$-condition. 
Here $Df$ is a closed extension of the Fr\'echet derivative $Df:E\to H$ for cylindrical function $f$.    
Hence $(E,\mathscr{E},\mu)$ is tamed by $-K_1\mu$ with $|K_1|\mu\in S_K({\bf X})$. 

\bigskip 

In the rest, we expose new examples. 
\begin{example}[{RCD spaces}]
{\rm A metric measure space $(X,{\sf  d},\m)$ is a complete separable metric space $(X,{\sf d})$ with a $\sigma$-finite Borel measure with $\m(B)<\infty$ for any bounded Borel set $B$. We assume $\m$ has full topological support, i.e., $\m(G)>0$ for non-empty open set $G$. 

Any metric open ball is denoted by $B_r(x):=\{y\in X\mid {\sf d}(x,y)<r\}$ for $r>0$ and $x\in X$. 
A subset $B$ of $X$ is said to be bounded if it is included in a metric open ball.    
Denote by $C([0,1],X)$ the space of continuous curve defined on the unit interval $[0,1]$ equipped the distance ${\sf d}_{\infty}(\gamma,\eta):=\sup_{t\in[0,1]}{\sf d}(\gamma_t,\eta_t)$ for every $\gamma,\eta\in C([0,1],X)$. This turn $C([0,1],X)$ into complete separable metric space.   
Next we consider the set of $2$-absolutely continuous curves, denoted by $AC^q([0,1],X)$, is the subset of $\gamma\in C([0,1],X)$ so that there exists  $g\in L^q(0,1)$ satisfying 
\begin{align*}
{\sf d}(\gamma_t,\gamma_s)\leq\int_s^tg(r)\d r,\quad s<t\text{ in }[0,1].
\end{align*}
Recall that for any $\gamma\in AC^2([0,1],X)$, there exists a minimal a.e.~function $g\in L^2(0,1)$ satisfying the above, called {\it metric speed} denoted by $|\dot\gamma_t|$, which is defined as 
$|\dot\gamma_t|:=\lim_{h\downarrow0}{\sf d}(\gamma_{t+h}, \gamma_t)/h$ for $\gamma\in AC^2([0,1],X)$, 
$|\dot\gamma_t|:=+\infty$ otherwise. 
We define the kinetic energy functional $C([0,1],X)\ni\gamma\mapsto {\sf Ke}_t(\gamma):=\int_0^1|\dot{\gamma}_t|^2\d t$, if $\gamma\in AC^2([0,1],X)$, ${\sf Ke}_t(\gamma):=+\infty$ otherwise. 
\begin{defn}[$2$-test plan]\label{def:$q$-test}
{\rm Let $(X,{\sf d},\m)$ be a metric measure space. 
A measure {\boldmath$\pi$}$\in\mathscr{P}(C([0,1],X))$ is said to be a $2$-test plan, provided 
\begin{enumerate}
\item[(i)]\label{item:qtest1} there exists $C>0$ so that $({\sf e}_t)_{\sharp}${\boldmath$\pi$}$\leq C\m$ for every $t\in[0,1]$;
\item[(ii)]\label{item:qtest2} we have $\int_{C([0,1],X)}{\sf Ke}_2(\gamma)${\boldmath$\pi$}$(\d\gamma)<\infty$. 
\end{enumerate} 
}
\end{defn}


\begin{defn}[Sobolev space {$W^{\hspace{0.03cm}1,2}(X)$}]\label{def:W1pSobolev}
{\rm
A Borel function $f\in L^2(X;\m)$ belongs to $W^{1,2}(X)$, 
provided there exists a $G\in L^2(X;\m)$,  
called {\it $2$-weak upper gradient} of $f$ so that 
\begin{align}
\int_{C([0,1],X)}|f(\gamma_1)-f(\gamma_0)|\text{\boldmath$\pi$}(\d\gamma)\leq \int_{C([0,1],X)}\int_0^1
G(\gamma_t)|\dot{\gamma}_t|\d t\text{\boldmath$\pi$}(\d\gamma),\quad \text{$\forall${\boldmath$\pi$} 
$2$-test plan}. \label{eq:SpSobolev}
\end{align} 
The assignment $(t,\gamma)\mapsto G(\gamma_t)|\dot{\gamma}_t|$ is Borel measurable (see \cite[Remark~2.1.9]{GPLecture}) and the right hand side of 
\eqref{eq:SpSobolev} is finite for $G\in L^2(X;\m)_+$ (see \cite[(2.5)]{GigliNobili}). These shows not only the finiteness of the right hand side of \eqref{eq:SpSobolev} but also the continuity of the assignment 
$L^2(X;\m)\ni G\mapsto  \int_{C([0,1],X)}\int_0^1 G(\gamma_t)|\dot{\gamma}_t|\d t\text{\boldmath$\pi$}(\d\gamma)$. This, combined with the closedness of the convex combination of the $2$-weak upper gradient, shows that the set of $2$-weak upper gradient of a given Borel function $f$ is a closed convex subset of $L^2(X;\m)$. The minimal $p$-weak upper gradient, denoted by $|Df|_2$ is then the element of 
minimal $L^p$-norm in this class. Also, by making use of the lattice property of the set of $2$-weak 
upper gradient, such a minimality is also in the $\m$-a.e. sense (see \cite[Proposition~2.17 and Theorem~2.18]{Ch:metmeas}). 

Then, $W^{1,2}(X)$ forms a Banach space equipped with the following norm:
equipped with the norm 
\begin{align*}
\|f\|_{W^{1,2}(X)}:=\left(\|f\|^2_{L^2(X;\m)}+\| |Df|_2\|^2_{L^2(X;\m)} \right)^{\frac{1}{2}} ,\quad f\in  W^{1,2}(X). 
\end{align*}
 
}
\end{defn}

It is in general false 
that  $(W^{1,2}(X),\|\cdot\|_{W^{1,2}(X)})$ is a Hilbert space. 
When this occurs, we say that $(X,{\sf d},\m)$ is {\it infinitesimally Hilbertian} (see 
\cite{Gigli:OntheDifferentialStr}). Equivalently, we call $(X,{\sf d},\m)$  infinitesimally Hilbertian 
provided the following {\it parallelogram identity} holds:
\begin{align}
2|Df|_2^2+2|Dg|_2^2=|D(f+g)|_2^2+|D(f-g)|_2^2, \quad \m\text{-a.e.} \quad \forall f,g\in W^{1,2}(X).\label{eq:parallelogram}
\end{align}

For simplicity, when $p=2$, we omit the suffix $2$ from $|Df|_2$ for $f\in W^{1,2}(X)$, i.e. we write $|Df|$ instead of $|Df|_2$ for $f\in W^{1,2}(X)$.  
Under \eqref{eq:parallelogram}, we can give a bilinear form $\langle D\cdot ,D\cdot\rangle:W^{1,2}(X)\times W^{1,2}(X)\to L^1(X;\m)$ which is defined by 
\begin{align*}
\langle Df,Dg\rangle:=\frac14|D(f+g)|_2^2-\frac14|D(f-g)|_2^2,\quad f,g\in W^{1,2}(X).
\end{align*}
Moreover, under the infinitesimally Hilbertian condition, 
the bilinear form $(\mathscr{E},D(\mathscr{E}))$ defined by 
\begin{align*}
D(\mathscr{E}):=W^{1,2}(X),\quad \mathscr{E}(f,g):=\frac12\int_X\langle Df,Dg\rangle\d \m
\end{align*}
is a strongly local Dirichlet form on $L^2(X;\m)$. 
Denote by $(P_t)_{t\geq0}$ the $\m$-symmetric semigroup on $L^2(X;\m)$ associated with $(\mathscr{E},D(\mathscr{E}))$. 
Under \eqref{eq:parallelogram}, let $(\Delta, D(\Delta))$ be the 
$L^2$-generator associated with $(\mathscr{E},D(\mathscr{E}))$ similarly defined as in \eqref{eq:generatorL2} before. 
\begin{defn}[RCD-spaces]
{\rm A metric measure space $(X,{\sf d},\m)$ is said to be an {\it {\sf RCD}$(K,\infty)$-space} if 
it satisfies 
the following conditions: 
\begin{enumerate}
\item[\rm(1)]
$(X, {\sf d}, \m)$ is infinitesimally Hilbertian. 

\item[\rm(2)]
There exist $x_0 \in X$ and constants $c, C > 0$ such that 
$\m(B_r(x_0)) \le C \e^{c r^2}$. 

\item[\rm(3)]
If $f \in W^{1,2}(X)$ satisfies 
$| D f |_2 \le 1$ $\m$-a.e., then $f$ has a $1$-Lipschitz representative. 

\item[\rm(4)]
For any $f \in D ( \Delta )$ 
with $\Delta f \in W^{1,2}(X)$ 
and $g \in D ( \Delta ) \cap L^\infty (X; \m)$ 
with $g \ge 0$ and $\Delta g \in L^\infty (X; \m)$, 
\begin{align*}
\frac12 \int_X | D f |_2^2 \Delta g \, \d \m 
- \int_X \langle D f, D \Delta f \rangle g \, \d \m 
\ge 
K \int_X | D f |_2^2 g \, \d \m 
\end{align*}
\end{enumerate}
Let $N\in[1,+\infty[$. 
A metric measure space $(X,{\sf d},\m)$ is said to be an {\it {\sf RCD}${}^*(K,N)$-space} if 
it is an {\sf RCD}$(K,\infty)$-space and 
for any $f \in D ( \Delta )$ 
with $\Delta f \in W^{1,2}(X)$ 
and $g \in D( \Delta ) \cap L^\infty (X; \m)$ 
with $g \ge 0$ and $\Delta g \in L^\infty (X; \m)$, 
\begin{align*}
\frac12 \int_X | D f |_2^2 \Delta g \, \d \m 
- \int_X \langle D f, D \Delta f \rangle g \, \d \m 
\ge 
K \int_X | D f |_2^2 g \, \d \m 
+ \frac{1}{N} \int_X ( \Delta f )^2 g \, \d \m. 
\end{align*}

}
\end{defn}
\begin{remark}
{\rm 
\begin{enumerate}
\item It is shown in \cite{Cav-Mil} that the notion {\sf RCD}${}^*(K,N)$-space is equivalent to {\sf RCD}$(K,N)$-space, which is defined to be a {\sf CD}$(K,N)$-space having infinitesimal Hilbertian condition. 
Here {\sf CD}$(K,N)$-space is a metric measure space defined in terms of optimal mass transport theory (see \cite{AGS_Riem,EKS} for details). 
So we may say  
{\sf RCD}$(K,N)$-space instead of {\sf RCD}${}^*(K,N)$-space. Moreover, {\sf RCD}$(K,N)$-space (or {\sf RCD}${}^*(K,N)$-space) is a locally compact separable metric space, consequently, $\m$ becomes a Radon measure.  
\item If $(X,{\sf d},\m)$ is an {\sf RCD}$(K,N)$-space, it enjoys the Bishop-Gromov inequality: 
Let $\kappa:=K/(N-1)$ if $N>1$ and $\kappa:=0$ if $N=1$. We set $\omega_N:=\frac{\pi^{N/2}}{\int_0^{\infty}t^{N/2}e^{-t}\d t}$ (volume of unit bal in $\R^N$ provided $N\in\mathbb{N}$) and 
$V_{\kappa}(r):=\omega_N\int_0^r\s_{\kappa}^{N-1}(t)\d t$.  
 Then 
\begin{align}
\frac{\m(B_R(x))}{V_{\kappa}(R)}\leq \frac{\m(B_r(x))}{V_{\kappa}(r)},\qquad x\in X, \qquad 0<r<R. \label{eq:BishopGromov}
\end{align}
Here  
$\s_{\kappa}(s)$ is the solution to Jacobi equation $\s_{\kappa}''(s)+\kappa\s_{\kappa}(s)=0$ with 
$\s_{\kappa}(0)=0$, $\s_{\kappa}'(0)=1$. More concretely, $\s_{\kappa}(s)$ is given by
\begin{align*}
\s_{\kappa}(s):=\left\{\begin{array}{cc}\frac{\sin \sqrt{\kappa}s}{\sqrt{\kappa}} & \kappa>0, \\ s & \kappa=0, \\ \frac{\sinh \sqrt{-\kappa}s}{\sqrt{-\kappa}} & \kappa<0.\end{array}\right.
\end{align*}
\item If $(X,{\sf d},\m)$ is an $\mathsf{RCD}(K,\infty)$-space, it satisfies the following Bakry-\'Emery estimate:
\begin{align}
|DP_tf|_2\leq e^{-Kt}P_t|Df|_2\quad \m\text{-a.e.}\quad for \quad f\in W^{1,2}(X),\label{eq:BakryEmery}
\end{align}
in particular, $P_tf\in W^{1,2}(X)$
(see \cite[Proposition~3.1]{GigliHan}).
\item 
If $(X,{\sf d},\m)$ is an $\mathsf{RCD}(K,N)$-space with $N\in[1,+\infty[$, then,  
for any $f\in {\rm Lip}(X)$ 
${\rm lip}(f)=|Df|_2$ $\m$-a.e. holds, 
because  $\mathsf{RCD}(K,N)$ with $N\in[1,+\infty[$ admits the local volume doubling and a weak local $(1,2)$-Poincar\'e inequality (see \cite[\S5 and \S6]{Ch:metmeas}). 
\end{enumerate}
}
\end{remark}
By definition, for ${\sf RCD}(K,N)$-space $(X,{\sf d},\m)$,  $(X,\mathscr{E},\m)$ 
satisfies the  ${\sf BE}_2(K,N)$-condition and $K^{\pm}\m$ is always a Kato class smooth measure, 
hence it is a tamed Dirichlet space by $K\m$. So we can apply Theorems~\ref{thm:main1} and \ref{thm:main2} to $(X,{\sf d},\m)$. 
}
\end{example}

\begin{example}[Riemannian manifolds with boundary]
{\rm \quad Let $(M,g)$ be a smooth Riemannian manifold with boundary $\partial M$. 
Denote by $\mathfrak{v}:={\rm vol}_g$ the Riemannian volume measure induced by $g$, and by 
$\mathfrak{s}$ the surface measure on $\partial M$
(see \cite[\S1.2]{Braun:Tamed2021}). 
If $\partial M\ne \emptyset$, then $\partial M$ is a smooth co-dimension $1$ submanifold of $M$ and it becomes Riemannian 
when endowed with the pulback metric 
\begin{align*}
\langle \cdot,\cdot\rangle_j:=j^*\langle\cdot,\cdot\rangle,\qquad \langle u,v\rangle:=g(u,v)\quad\text{ for }\quad u,v\in TM.
\end{align*}  
under the natural inclusion $j:\partial M\to M$. The map $j$ induces a natural inclusion $d_j:T\partial M\to TM|_{\partial M}$ which is not surjective. In particular, the vector bundles $T\partial M$ and $TM|_{\partial M}$ do not coincide. 

Let $\m$ be a Borel measure on $M$ which is locally equivalent to $\mathfrak{v}$. 
Let $D(\mathscr{E}):=W^{1,2}(M^{\circ})$ be the Sobolev space with respect to 
$\m$ defined in the usual sense on $M^{\circ}:=M\setminus\partial M$. Define $\mathscr{E}:W^{1,2}(M^{\circ})\to [0,+\infty[$ 
by 
\begin{align*}
\mathscr{E}(f):=\int_{M^{\circ}}|\nabla f|^2\d \m
\end{align*}  
and the quantity $\mathscr{E}(f,g)$, $f,g\in W^{1,2}(M^{\circ})$, by polarization. Then $(\mathscr{E}, D(\mathscr{E}))$ becomes a strongly local regular Dirichlet form on $L^2(M;\m)$, since $C_c^{\infty}(M)$ is a dense set of $D(\mathscr{E})$. 
Let $k:M^{\circ}\to\R$ and $\ell:\partial M\to\R$ be continuous functions providing lower bounds on the Ricci curvature and the second fundamental form of $\partial M$, respectively. 
Suppose that $M$ is compact and $\m=\mathfrak{v}$. Then $(M,\mathscr{E},\m)$ is tamed by 
\begin{align*}
\kappa:=k\mathfrak{v}+\ell\mathfrak{s},
\end{align*}
because $\mathfrak{v},\mathfrak{s}\in S_K({\bf X})$ (see \cite[Theorem~2.36]{ERST} and \cite[Theorem~5.1]{Hsu:2001}). Then one can apply Theorems~\ref{thm:main1} and \ref{thm:main2} to $(M,\mathscr{E},\m)$. 
Remark that \eqref{eq:LittlewoodPaleyStein1} is proved by 
Shigekawa~\cite[Propositions~6.2 and 6.4]{Shigekawa1} in the framework of compact smooth Riemannian manifold with boundary. 

More generally, if $M$ is regularly exhaustible, i.e., there exists an increasing sequence $(X_n)_{n\in\N}$ of domains $X_n\subset M^{\circ}$ with smooth boundary $\partial X_n$ such that $g$ is smooth on $X_n$ and the following properties hold:
\begin{enumerate}
\item[(1)] The closed sets $(\overline{X}_n)_{n\in\N}$ constitute an $\mathscr{E}$-nest for $(\mathscr{E},W^{1,2}(M))$.  
\item[(2)] For all compact sets $K\subset M^{\circ}$ there exists $N\in\N$ such tht $K\subset X_n$ for all $n\geq N$.
\item[(3)] There are lower bounds $\ell_n:\partial X_n\to\R$ for the curvature of $\partial X_n$ with $\ell_n=\ell$ on $\partial M\cap \partial X_n$ such that the distributions $\kappa_n=k\mathfrak{v}_n+\ell_n\mathfrak{s}_n$ are uniformly $2$-moderate in 
the sense that 
\begin{align}
\sup_{n\in\N}\sup_{t\in[0,1]}\sup_{x\in X_n}\E_x^{(n)}\left[e^{-2A_t^{\kappa_n}} \right]<\infty,\label{eq:2moderate}
\end{align}
where $\mathfrak{v}_n$ is the volume measure of $X_n$ and $\mathfrak{s}_n$ is the surface measure of $\partial X_n$.
\end{enumerate}
Suppose $\m=\mathfrak{v}$. 
Then $(M,\mathscr{E},\m)$ is tamed by $\kappa=k\mathfrak{v}+\ell\mathfrak{s}$ 
(see \cite[Theorem~4.5]{ERST}), hence Theorems~\ref{thm:main1} and \ref{thm:main2} hold for $(M,\mathscr{E},\m)$ provided 
$\kappa^+\in S_D({\bf X})$ and $2\kappa^-\in S_{E\!K}({\bf X})$. 

Let $Y$ be the domain defined by 
\begin{align*}
Y:=\{(x,y,z)\in\R^3\mid z>\phi(\sqrt{x^2+y^2})\},
\end{align*}
where $\phi:[0,+\infty[\to[0,+\infty[$ is $C^2$ on $]0,+\infty[$ with $\phi(r):=r-r^{2-\alpha}$, $\alpha\in]0,1[$ for 
$r\in[0,1]$, $\phi$ constant for $r\geq2$ and $\phi''(r)\leq0$ for $r\in[0,+\infty[$. 
Let $\m_Y$ be the $3$-dimensional Lebesgue measure restricted to $Y$ and 
$\sigma_{\partial Y}$ the $2$-dimensional Hausdorff measure on $\partial Y$. Denote by 
$\mathscr{E}_Y$ the Dirichlet form on $L^2(Y;\m)$ with Neumann boundary conditions. 
The smallest eigenvalue of the second fundamental form of $\partial Y$ can be given by 
\begin{align*}
\ell(r,\phi)=\frac{\phi''(r)}{(1+|\phi'(r)|^2)^{3/2}} (\leq0),
\end{align*}
for $r\leq 1$ and $\ell=0$ for $r\geq2$. It is proved in \cite[Theorem~4.6]{ERST} that 
the Dirichlet space $(Y,\mathscr{E}_Y,\m_Y)$ is tamed by 
\begin{align*}
\kappa=\ell\sigma_{\partial Y}.
\end{align*}
Since $|\kappa|=|\ell|\sigma_{\partial Y}\in S_K({\bf X})$ (see \cite[Lemma~2.34, Theorem~2.36, Proof of Theorem~4.6]{ERST}), 
we can apply Theorems~\ref{thm:main1} and \ref{thm:main2} for $(Y,\mathscr{E}_Y,\m_Y)$. 
}
\end{example}
\begin{example}[Configuration space over metric measure spaces]
{\rm Let $(M,g)$ be a complete smooth Riemannian manifold without boundary.   
The configuration space $\Upsilon$ over $M$ is the space of all locally finite point measures, that is, 
\begin{align*}
\Upsilon:=\{\gamma\in\mathcal{M}(M)\mid \gamma(K)\in\N\cup\{0\}\quad \text{ for all compact sets}\quad K\subset M\}. 
\end{align*}
In the seminal paper Albeverio-Kondrachev-R\"ockner~\cite{AKR} identified a natural geometry on $\Upsilon$ by lifting the geometry of $M$ to $\Upsilon$. In particular, there exists a natural gradient $\nabla^{\Upsilon}$, divergence ${\rm div}^{\Upsilon}$ and 
Laplace operator $\Delta^{\Upsilon}$ on $\Upsilon$. It is shown in \cite{AKR} that the Poisson point measure $\pi$ on 
$\Upsilon$ is the unique (up to intensity) measure on $\Upsilon$ under which the gradient and divergence become 
dual operator in $L^2(\Upsilon;\pi)$. Hence, the Poisson measure $\pi$ is the natural volume measure on $\Upsilon$ 
and $\Upsilon$ can be seen as an infinite dimensional Riemannian manifold. The canonical Dirichlet form 
\begin{align*}
\mathscr{E}(F)=\int_{\Upsilon}|\nabla^{\Upsilon}F|_{\gamma}^2\pi(\d\gamma)
\end{align*}
 constructed in \cite[Theorem~6.1]{AKR} is quasi-regular and strongly local
 and it induces the heat semigroup $T_t^{\Upsilon}$ and a Brownian motion ${\bf X}^{\Upsilon}$ on $\Upsilon$ which can be identified with the independent infinite particle process. If ${\rm Ric}_g\geq K$ on $M$ with $K\in \R$, then $(\Upsilon,\mathscr{E}^{\Upsilon},\pi)$ is tamed by 
 $\kappa:=K\pi$ with $|\kappa|\in S_K({\bf X}^{\Upsilon})$ (see \cite[Theorem~4.7]{EKS} and \cite[Theorem~3.6]{ERST}). 
 Then one can apply Theorems~\ref{thm:main1} and \ref{thm:main2} for $(\Upsilon,\mathscr{E}^{\Upsilon},\pi)$. 
 
 More generally, in Dello Schiavo-Suzuki~\cite{DelloSuzuki:ConfigurationI}, configuration space $\Upsilon$ over proper complete and separable metric space $(X,{\sf d})$ is considered. The configuration space $\Upsilon$ is endowed with the \emph{vage topology} $\tau_V$, induced by duality with continuous compactly supported functions on $X$, and with a reference Borel probability measure $\mu$ satisfying \cite[Assumption~2.17]{DelloSuzuki:ConfigurationI}, commonly understood as the law of a proper point process on $X$. In \cite{DelloSuzuki:ConfigurationI}, 
 they constructed the strongly local Dirichlet form $\mathscr{E}^{\Upsilon}$ defined to be the $L^2(\Upsilon;\mu)$-closure of a certain pre-Dirichlet form on a class of certain cylinder functions and prove its quasi-regularity for a wide class of measures $\mu$ and base spaces (see \cite[Proposition~3.9 and Theorem~3.45]{DelloSuzuki:ConfigurationI}). Moreover, in 
 Dello Schiavo-Suzuki~\cite{DelloSuzuki:ConfigurationII}, for any fixed $K\in \R$ they prove that a Dirichlet form $(\mathscr{E},D(\mathscr{E}))$ with its 
 carr\'e-du-champ $\Gamma$ satisfies ${\sf BE}_2(K,\infty)$ if and only if the Dirichlet form $(\mathscr{E}^{\Upsilon},D(\mathscr{E}^{\Upsilon}))$ on $L^2(\Upsilon;\mu)$ with its carr\'e-du-champ $\Gamma^{\Upsilon}$ satisfies ${\sf BE}_2(K,\infty)$.
Hence, if $(X,\mathscr{E},\m)$ is tamed by $K\m$ with $|K|\m\in S_K({\bf X})$, then  $(\Upsilon,\mathscr{E}^{\Upsilon},\mu)$ is tamed by $\kappa:=K\mu$ with $|\kappa|\in S_K({\bf X}^{\Upsilon})$. 
Then one can apply Theorems~\ref{thm:main1} and \ref{thm:main2} for $(\Upsilon,\mathscr{E}^{\Upsilon},\mu)$ under the suitable class of measures $\mu$ defined in \cite[Assumption~2.17]{DelloSuzuki:ConfigurationI}.  
}
\end{example}

\providecommand{\bysame}{\leavevmode\hbox to3em{\hrulefill}\thinspace}
\providecommand{\MR}{\relax\ifhmode\unskip\space\fi MR }
% \MRhref is called by the amsart/book/proc definition of \MR.
\providecommand{\MRhref}[2]{%
  \href{http://www.ams.org/mathscinet-getitem?mr=#1}{#2}
}
\providecommand{\href}[2]{#2}
\begin{thebibliography}{99}


\bibitem{AKR}
{S. Albeverio, Yu. G. Kondratiev and M. R\"ockner},
\emph{Analysis and geometry on configuration spaces}, J. Funct. Anal. {\bf 154} (1998), no. 2, 444--500.

\bibitem{AM:AF}
{S.~Albeverio and Z.-M. Ma},
\emph{Additive functionals, nowhere Radon and Kato class smooth measures associated with Dirichlet forms},   
Osaka J. Math. {\bf 29} (1992), no. 2, 247--265.

%\bibitem{AB}
%L.~Ambrosio and J.~Bertrand, 
%\emph{DC calculus},
%Math.~Z. {\bf 288} (2018), no.~3-4, 1037--1080.
%%Preprint. Available at \textsf{arXiv}: 1505.048 17. 

%\bibitem{AGMR}
%L.~Ambrosio, N.~Gigli, A.~Mondino and T.~Rajala,
%\emph{Riemannian Ricci curvature lower bounds in metric measure spaces with $\sigma$-finite measure}, 
%Trans.\ Amer.\ Math.\ Soc.\ {\bf 367} (2015), no.~7,  4661--4701.


\bibitem{AES}
{L.~Ambrosio, M. Erbar and  G.~Savar\'e}, 
\emph{Optimal transport, Cheeger energies and contractivity of dynamic transport distances in extended spaces}, 
Nonlinear Analysis, {\bf 137}, (2016), 77--134.


\bibitem{AGS_Calc}
L.~Ambrosio, N.~Gigli and G.~Savar\'e,
\emph{Calculus and heat flow in metric measure spaces and applications to spaces with Ricci bounds from below}, 
Invent.\ Math.\ \textbf{195} (2014), no.~2, 289--391.

\bibitem{AGS_Riem}
\bysame,
\emph{Metric measure spaces with Riemannian Ricci curvature bounded from 
below}, 
Duke Math.\ J.\ \textbf{163} (2014), no.~7, 1405--1490.

\bibitem{AGS_Sobolev}
\bysame,
\emph{Density of Lipschitz functions and equivalence 
of weak gradients in metric measure spaces}, 
Rev.\ Mat.\ Iberoam.\ {\bf 29} (2013), no.~3, 969--996.

\bibitem{AGS_BakryEmery}
\bysame,
\emph{Bakry-\'Emery curvature-dimension condition and Riemannian Ricci curvature bounds}, Ann. Probab. {\bf 43} (2015), no.~1, 339--404. 

\bibitem{AMS}
{L.~Ambrosio, A.~Mondino and G.~Savar\'e}, 
\emph{Nonlinear diffusion equations and curvature conditions in metric measure spaces}, 
%Preprint. To appear in 
Memoirs Amer. Math. Soc. {\bf 262}, (2019), no. 1270.%, Available at \textsf{arXiv}: 1509.07273.

%\bibitem{ABD}
%{G. Antonelli, E. Bru\'e, and D. Semola}, 
%\emph{Volume bounds for the quantitative singular strata of non collapsed RCD metric measure spaces}, Anal. Geom. Metr. Spaces, {\bf 7} (2019), 158--178.

\bibitem{Bakry1}
{D.~Bakry}, 
\emph{Etude des transformations de Riesz dans les vari\'et\'es riemaniennes \'a courbure 
de Ricci minor\'ee}, S\'eminaire de Prob. XXI, Lecture Notes in Math. {\bf 1247}, Springer-Velrag, Berlin-Heidelberg-New York, (1987), 137--172.

\bibitem{Bakry2}
\bysame,
\emph{On Sobolev and logarithmic Sobolev inequalities for Markov semi-groups}, 
in \lq\lq New Trends in Stochastic Analysis\rq\rq\;  (K. D. Elworthy, S. Kusuoka and I. Shigekawa eds.), World Sci. Publishing, River Edge, NJ (1997), 43--75.

\bibitem{BE1}
{D.~Bakry and M.~\'Emery},
        \emph{Diffusion hypercontractives}, in: S\'em. Prob. XIX, in: Lecture Notes in Math., vol. 1123, Springer-Verlag, Berlin/New York, 1985, pp.~177--206.

%\bibitem{BQ}
%{D.~Bakry and Z.-M.~Qian},
%\emph{Volume comparison theorems without Jacobi fields}, Current trends in potential theory, 115--122, Theta Ser. Adv. Math., {\bf 4}, Theta, Bucharest, 2005.
%Available at {\tt http://www.lsp.ups-tlse.fr/Bakry.}


\bibitem{BGL_book}
{D.~Bakry, I.~Gentil and M.~Ledoux},
\emph{Analysis and geometry of Markov diffusion operators}, 
Grundlehren der Mathematischen Wissenschaften {\bf 348}, Springer, Cham, 2014.

\bibitem{BorodinSalminen}
{A.~N.~Borodin and P.~Salminen}, 
\emph{Handbook of Brownian Motion-Facts and Formulae}, 
Second edition. Probability and its Applications. Birkha\"user Verlag, Basel, 2002.


\bibitem{Braun:Tamed2021}
{M.~Braun}, 
\emph{Vector calculus for tamed Dirichlet spaces}, preprint 2021, to appear in Mem. Amer. Math. Soc. Available at {\tt  https://arxiv.org/pdf/2108.}{\tt 12374.pdf} 

\bibitem{Cav-Mil}
{F.~Cavalletti and E.~Milman}, 
\emph{The globalization theorem for the curvature dimension condition},
Inventiones mathematicae {\bf 226} (2021), no.~1, 1--137.  

%Preprint. Available at \textsf{arXiv}: 1612.07623. 

%\bibitem{Cav-St}
%{F.~Cavalletti and K.-T. Sturm},
%\emph{Local curvature-dimension condition implies measure-contraction property}, 
%J.\ Funct.\ Anal.\ \textbf{262} (2012), no.~12,  5110--5127. 

\bibitem{CFKZ:Pert} 
 {Z.-Q.~Chen, P.~J.~Fitzsimmons, K.~Kuwae and T.-S.~Zhang}, 
\emph{Perturbation of symmetric Markov processes}, Probab. Theory Related Fields {\bf 140} 
(2008), no.~1-2, 239--275.


\bibitem{CFKZ:GenPert}   
         \bysame, 
\emph{On general perturbations of symmetric Markov processes}, J. Math. Pures et 
Appliqu\'ees {\bf 92} (2009), no.~4, 363--374. 


 \bibitem{CFbook}
 {Z.-Q. Chen and M. Fukushima}, \emph{Symmetric Markov processes, time change, and
 boundary theory}, London Mathematical Society Monographs Series, {\bf 35}. Princeton University Press, Princeton, NJ, 2012.    

\bibitem{Ch:metmeas}
J.~Cheeger, \emph{Differentiability of {L}ipschitz functions on metric measure
  spaces}, Geom. Funct. Anal. \textbf{9} (1999), no.~3, 428--517.

%\bibitem{Choquet:Vol1}
%{G.~Choquet}, 
%\emph{Lectures on analysis. Vol. I: Integration and topological vector spaces}, Edited by J. Marsden, T. Lance and S. Gelbart W. A. Benjamin, Inc., New York-Amsterdam 1969 Vol. I

%\bibitem{Conway:FunctionalAnal}
%{J.~B.~Conway}, \emph{A Course in Functional Analysis}, Graduate Texts in Mathematics {\bf 96} 2nd ed.
%1985.  

\bibitem{CoulhonDuong}
{T.~Coulhon and X. T. Duong}, 
\emph{Riesz transform and related inequalities on noncompact Riemannian manifolds}, 
Comm. Pure Appl. Math. {\bf 56} (2003), 1728--1751.

%
%\bibitem{CKM}
%{M.~Cranston, W.~S.~Kendall, and P.~March},  
%\emph{The radial part of Brownian motion II. Its life and times on the cut locus}, 
%Probab. Th. Related Fields \textbf{96} (1993) no.~3, 353--368.

% \bibitem{DMG:ComparisonCR}
%  {A.~Debiard, B.~Gaveau and E.~Mazet}, 
%  \emph{Th\'eor\`emes de comparaison en g\'eom\'etrie riemannienne} C. R. Acad. Sci. Paris S\'er. A-B {\bf 281} (1975), no. 12, Aii, A455--A458. 

% \bibitem{DMG:Comparison}
%  {A.~Debiard, B.~Gaveau and E.~Mazet}, 
%  \emph{Th\'eor\`emes de comparaison en g\'eom\'etrie riemannienne}, Publ. Res. Inst. Math. Sci. {\bf 12} (1976/77), no. 2, 391--425.


%\bibitem{DelloSuzuki:LademacherSobolevToLip}
%{L. Dello Schiavo and K. Suzuki}. 
%\emph{Rademacher-type theorems and Sobolev-to-Lipschitz properties for strongly local Dirichlet spaces}, 
%J. Funct. Anal. {\bf 281} (2021), no. 11, Paper No. 109234, 63 pp.

\bibitem{DelloSuzuki:ConfigurationI} 
{L. Dello Schiavo and K. Suzuki}, 
\emph{Configuration spaces over singular spaces — I. Dirichlet-Form and
Metric Measure Geometry}, arXiv:2109.03192, 2021. 
 
\bibitem{DelloSuzuki:ConfigurationII} 
\bysame, \emph{Configuration spaces over singular spaces — II. Curvature}, arXiv:2205.013 79, 2021. 

%\bibitem{Dudley}
%{R.~M.~Dudley},
% \emph{Real Analysis and Probability}, Revised reprint of the 1989 original. Cambridge Studies in Advanced Mathematics, {\bf 74}. Cambridge University Press, Cambridge, 2002.
 
 
%\bibitem{EkelanfTemanm}
%{I.~Ekeland and R.~Temam}, 
%\emph{Convex analysis and variational problems}, Studies in Mathematics and its Applications, Vol. 1 (Second ed.). New York: North-Holland Publishing Co., Amsterdam-Oxford,American 1976. 
 
\bibitem{EH}
{M. Erbar and M. Huesmann}, 
\emph{Curvature bounds for configuration spaces}, Calc. Var. Partial Differential Equations {\bf 54} (2015), no. 1, 397--430.
 
\bibitem{EKS}
{M.~Erbar, K.~Kuwada and K.-T.~Sturm},
\emph{On the equivalence of the entropic curvature-dimension condition and Bochner's inequality on metric measure spaces}, 
Invent.\ Math.\ {\bf 201} (2015), no.~3, 993--1071.

\bibitem{ERST}
{M. Erbar, C. Rigoni, K.-T. Sturm and L. Tamanini},  
\emph{Tamed spaces — Dirichlet spaces with distribution-valued Ricci bounds}, 
 J. Math. Pures Appl. (9) {\bf 161} (2022), 1--69.
%preprint, 2020, 
%Avairable at {\tt arXiv:2009.03121, 2020.}


\bibitem{EJK}
{S.~Esaki, Z.J. Xu and K.~Kuwae}, 
\emph{Riesz transforms for Dirichlet spaces tamed by distributional curvature lower bounds}, (2023) preprint.

\bibitem{Fitzsimmons}
{P.~J.~Fitzsimmons}, 
\emph{On the quasi-regularity of semi-Dirichlet forms},
Potential Analysis {\bf 15} (2001), 151--185.

%\bibitem{Fuk:StrictDecomposition} 
%{M. Fukushima}, 
%\emph{On a strict decomposition of additive functionals for symmetric diffusion processes}, Proc. Japan Acad. {\bf 70} Ser. A (1994), no.~9, 277--281. 
% 
%\bibitem{Fuk:Semimartingale}
% {M.~Fukushima},
%\emph{On semi-martingale characterizations of functionals of symmetric Markov processes}, 
%Electron. J. Probab. {\bf 4} (1999), no.~18, 1--32.
  
\bibitem{FOT}
    {M.~Fukushima, Y.~Oshima and M.~Takeda},
   \emph{Dirichlet forms and symmetric Markov processes}, Second revised and extended edition. de Gruyter Studies in Mathematics, {\bf 19}. Walter de Gruyter \& Co., Berlin, 2011. 

% \bibitem{FST}
% {M.~Fukushima, K.~Sato and S.~Taniguchi}, 
% \emph{On the closable parts of pre-Dirichlet forms and the fine supports of underlying measures},
% Osaka J. Math. {\bf 28} (1991), no.~3, 517--535.

%\bibitem{G}
% {R.~K. Getoor},
%           \emph{Some remarks on measures associated with homogeneous
%           random measures}, Seminar on stochastic processes, 
%           1985, 94--107,
%           Birkh\"auser, Boston, 1986.

% \bibitem{Gigli:Splitting}
% {N.~Gigli}, 
% \emph{The splitting theorem in non-smooth context}, 
% Preprint. Available at \textsf{arXiv}:1302.5555v1.

% \bibitem{Gigli:AnOverview}
% {N.~Gigli}, 
% \emph{An overview of the proof of the splitting theorem in spaces with non-negative Ricci curvature}, 
% Anal. Geom. Metr. Spaces {\bf 2} (2014), 169--213. 

\bibitem{Gigli:OntheDifferentialStr}
{N.~Gigli}, 
\emph{
On the differential structure of metric measure spaces and applications}, 
Mem. Amer. Math. Soc. {\bf 236} (2015), no. 1113.

\bibitem{Gigli:NonSmoothDifferentialStr}
\bysame, 
\emph{Nonsmooth Differential Geometry--An Approach Tailored for Spaces with Ricci Curvature Bounded from Below}, 
Mem. Amer. Math. Soc. {\bf 251} (2018), no. 1196.

\bibitem{Gigli:NonSmoothDifferentialStr}
\bysame, 
\emph{Lectures notes on differential calculus on RCD spaces}, preprint, 2017. 
Available at {\tt https://cvgmt.sns.it/media/doc/paper/3373/RIMSnotes.pdf}
 


\bibitem{Gigli:2013vi}
{N.~Gigli, A.~Mondino and T.~Rajala}, 
\emph{{Euclidean spaces as weak tangents
  of infinitesimally Hilbertian metric spaces with Ricci curvature bounded
  below}}, J. Reine. Angew. Math. {\bf 705} (2015), 233--244. 


\bibitem{GigliHan}
{N.~Gigli and B.-X. Han}, 
\emph{Independence on $p$ of weak upper gradients on {\sf RCD} spaces}, 
J.\ Funct.\ Anal. {\bf 271} (2016), no.~1, 1--11.



\bibitem{GigliNobili}
{N.~Gigli and F. Nobili}, 
\emph{A first-order condition for the independence on $p$ of weak gradients}, 
J.\ Funct.\ Anal. {\bf 283} (2022), no.~11, 109686.

\bibitem{GPLecture}
{N.~Gigli and E.~Pasqualetto}, 
\emph{Lectures on Nonsmooth Differential Geometry},  Springer-Verlag, 2020. 

\bibitem{GP}
\bysame,
%{N.~Gigli and E.~Pasqualetto}, 
\emph{Behaviour of the reference measure on \textsf{RCD} spaces under charts}, 
Communications in Analysis and Geometry {\bf 29} (2021), no.~6, 1391--1414.  
%Preprint. Available at \textsf{arXiv}:1607.05188v2. 

%\bibitem{GigliTama}
%{N.~Gigli and L.~Tamanini}, 
%\emph{Second order differentiation formula on RCD${}^*(K,N)$ spaces}, Journal of the European Mathematical Society, {\bf 23} (2021), 1727--1795.



\bibitem{Hsu:2001}
{E.~P.~Hsu}, 
\emph{Stochastic analysis on manifolds}, 
 Graduate Studies in Mathematics, {\bf 38}. American Mathematical Society, Providence, RI, 2002.  

%\bibitem{HuaKellXia}
%{B.~Hua, M.~Kell and C. Xia},
%\emph{Harmonic functions on metric measure spaces}, 
%Preprint. Available at \textsf{arXiv}:1308.3607v2.

% \bibitem{Jiang:Li-Yau}
% {R.~Jiang}, 
% \emph{The Li-Yau inequality and heat kernels on metric measure spaces},  J. Math. Pures Appl. (9) {\bf 104} (2015), no. 1, 29--57.

%\bibitem{Jiang:2016wo}
%{R.~Jiang, H.~Li and H.~Zhang}, 
%\emph{Heat kernel bounds on
%  metric measure spaces and some applications}, 
% Potential Anal. {\bf 44} (2016), no.~3, 601--627. 
  %Preprint. Available at
  %\textsf{arXiv}:1407.5289, 2014.

%\bibitem{Kakutani}
%{S.~Kakutani}, \emph{Weak convergence in uniformly convex spaces}, T\^ohoku Math. J. {\bf 45} (1938), 188--193.
%

\bibitem{KawabiMiyokawa}
{H.~Kawabi and T.~Miyokawa}, 
\emph{The Littlewood-Paley-Stein inequality for diffusion processes on general metric spaces}, 
J. Math. Sci. Univ. Tokyo {\bf 14} (2007), 1--30.


%\bibitem{Keith:measurable}
%{S.~Keith}, 
%Measurable differentiable structures and the Poincar\'e inequality, 
%\emph{Indiana Univ. Math. J}. {\bf 53} (2004), no. 4, 1127--1150. 
%
%\bibitem{KZ:Poincare}
%{S.~Keith and X. Zhong}, 
%The Poincar\'e inequality is an open ended condition,  
%Annals of Mathematics, {\bf 167} (2008), 575--599.
%
%\bibitem{KellMondino}
%{M.~Kell and A.~Mondino}, 
%\emph{On the volume measure of non-smooth spaces with Ricci curvature bounded below},  
%Annali della Scuola Normale Superiore di Pisa 
%{\bf 18} (2018), no.~2, 593--610.
%%Available at \textsf{arXiv}:1607.02036v1. 
%
%\bibitem{Kendall} 
%{W.~S.~Kendall}, 
%\emph{The radial part of Brownian motion on a manifold: a semimartingale property},
%\textit{Ann. Prob.} \textbf{15} (1987), no.~4, 1491--1500.
%

%\bibitem{KK:AnalChara}
% {D.~Kim and K.~Kuwae},  
% %\bysame,  
% \emph{Analytic characterizations of gaugeability for generalized Feynman-Kac functionals}, 
%Trans. Amer. Math. Soc. \textbf{369} (2017), no.~7, 4545--4596.

%\bibitem{Kitabeppu:2015wba}
%Y.~Kitabeppu and S.~Lakzian, \emph{{Characterization of low dimensional
%  $RC\!D^*(K,N)$ spaces}}, Anal. Geom. Metr. Spaces {\bf 4} (2016), 187--215. 
  %Preprint. Available at: \textsf{arXiv}:1505.00420v2.

%    \bibitem{Kw:reflect}
%        {K.~Kuwae}, 
%          \emph{Reflected Dirichlet forms and the uniqueness of Silverstein's extension}, 
%          Potential Anal. {\bf 16} (2002), no. 3, 221--247. 


% \bibitem{Kw:maximumprinciple}
%        {K.~Kuwae}, 
%         \emph{Maximum principles for subharmonic functions 
%         via local semi-Dirichlet forms},   
%        Canadian J. Math. {\bf 60} (2008), no. 4, 822-874. 



%\bibitem{KK}
%{K.~Kuwada and K.~Kuwae},
%\emph{Radial processes on $\mathsf{RCD}^* (K,N)$ spaces},
%Jour. Math. Pures et Appl. {\bf 126} (2019), no.~9, 72--108.

\bibitem{Kw:func}
       {K.~Kuwae}, 
        \emph{Functional calculus for Dirichlet forms}, 
        \emph{Osaka J.~Math.~}{\bf 35} (1998), 683--715. 

\bibitem{Kw:SobolevSpace}
\bysame, 
\emph{$(1,p)$-Sobolev spaces based on Dirichlet forms}, 
in preparation, 2023. 

%\bibitem{Kw:stochI}
%{K.~Kuwae}, 
% \emph{Stochastic calculus 
%over symmetric Markov processes without time reversal},  Ann. Probab. {\bf 38}  (2010),  
%no.~4, 1532--1569. 

%\bibitem{Kw:stochIErrata}
%\bysame, 
%\emph{Errata: Stochastic 
% calculus over symmetric Markov processes without time reversal},  Ann. Probab. {\bf 38}  (2010), no.~4,  1532--1569, Ann. Probab. {\bf 40}  (2012), 
%no.~6, 2705--2706. 

%  \bibitem{KMS:lap}
%      {K. Kuwae, Y. Machigashira and T. Shioya}, 
%         \emph{Sobolev spaces, Laplacian, and heat kernel 
%        on Alexandrov spaces}, {Math.~Z.} {\bf 238} 
%        (2001), no.~2, 269--316.
%
%\bibitem{KwSy:splitting}
%{K.~Kuwae and T.~Shioya}, 
%\emph{A topological splitting theorem for weighted
%  Alexandrov spaces}, 
%  Tohoku Math. J. {\bf 63} (2011), no~1, 59--76.

\bibitem{LenglartLepinglePratelli}
{E.~Lenglart, D.~L\'epingle and M. Pratelli}, 
\emph{Pr\'esentation unifi\'ee de certaines in\'egalit\'es de th\'eorie des martingales}, 
S\'eminaire de Prob. XIV, Lecture Notes in Math. {\bf 784}, Springer-Verlag, Berlin-Heidelberg-New York, (1980), 26--48.

% \bibitem{Xdli:RieszTrans}
% {X.-D.~Li}, 
% \emph{Riesz transforms for symmetric diffusion operators on complete Riemannian manifolds}, 
% Rev. Mat. Iberoam. {\bf 22} (2006), no. 2, 591--648.

% \bibitem{Xdli:Liouville}
% \bysame, 
% \emph{Liouville theorems for symmetric diffusion operators on complete Riemannian manifolds},  J. Math. Pures Appl. (9) {\bf 84} (2005), no. 10, 1295--1361.

% \bibitem{LiX:PerelmanEntropyFor}
% \bysame, 
% \emph{Perelman's entropy formula for the Witten Laplacian on Riemannian manifolds via Bakry-Emery Ricci curvature}, 
% Math. Ann. {\bf 353} (2012), no. 2, 403--437.

%\bibitem{LV2}
%{J.~Lott and C.~Villani},
%\emph{Ricci curvature for metric-measure spaces via optimal transport}, 
%Ann.\ of Math.\ {\bf 169} (2009), no.~3, 903--991.

%\bibitem{Loomis:harmonicAnal}
%{L.~H.~Loomis}, 
%\emph{An introduction to abstract harmonic analysis}, D. Van Nostrand Company, Inc., Toronto-New York-London, 1953. 

\bibitem{MR}
{Z.-M.~Ma and M.~R\"ockner}, 
\emph{Introduction to the Theory of (Non-Symmetric) Dirichlet Forms}, 
Springer-Verlag, Berlin-Heidelberg-New York, 1992.

\bibitem{Meyer}
{P.~A.~Meyer}, 
\emph{D\'emonstration probabiliste de certaines in\'egalit\'es de Littlewood-Paley}, S\'eminaire de Prob. X, Lecture Notes in Math. {\bf 511}, Springer-
Velrag, Berlin-Heidelberg-New York, (1976), 125--183.

%\bibitem{MondinoNaberI}
%{A. Mondino and A. Naber},
%\emph{Structure theory of metric-measure spaces with lower Ricci curvature bounds}, {\bf 21} (2019), no.~6, 1809--1854.

%Preprint. To appear in Jour. European Math Soc. Available at \textsf{arXiv}:1405.2222.
   
% \bibitem{Rajala:Poincare}
% {T.~Rajala}, 
% \emph{Local Poincar\'e inequalities from stable curvature conditions on metric spaces},  Calc. Var. {\bf 44} (2012), no.~3, 477--494. 

\bibitem{Okura}
{H.~$\widehat{\rm O}$kura},
\emph{A new approach to the skew product of symmetric Markov processes}, Mem. Fac. Eng. and Design Kyoto Inst. Tech. {\bf 46} (1998), 1--12.

%\bibitem{Ohta}
%{S. Ohta},
%\emph{On the measure contraction property of metric measure spaces},  
%Comment. Math. Helv. \textbf{82} (2007) no.~4, 805--828. 

%\bibitem{Petrunin:LottVillani}
%{A. Petrunin}, 
%\emph{Alexandrov meets Lott--Villani--Sturm}, M\"unster J. of Math. {\bf 4} 
%(2011), 53--64.
%
%\bibitem{Rajala:Poincare2}
%{T.~Rajala}, 
%\emph{Interpolated measures with bounded density in metric spaces satisfying the curvature-dimension conditions of Sturm}, J. Funct. Anal. {\bf 263} (2012) no.~4, 896--924. 
   
%\bibitem{Max:Comparison}
%{M.~K. von Renesse},
%\emph{Heat kernel comparison on Alexandrov spaces with curvature bounded below}, Potential Anal. {\bf 21} (2004), no.~2, 151--176.
%
%\bibitem{Royden:RealAnal}
%{H.~L.~Royden}, 
%\emph{Real analysis}, Third edition. Macmillan Publishing Company, New York, 1988. 

 \bibitem{Sav14}
 {G.~Savar\'e}, 
 \emph{Self-improvement of the Bakry-\'Emery condition and
 Wasserstein contraction of the heat flow in $RC\!D (K, \infty)$ metric measure spaces}, 
 Discrete and Contin.\ Dyn.\ Syst.\ \textbf{34} (2014), 1641--1661.

 
%\bibitem{Schwarz}
%{G.~Schwarz}, 
%\emph{Hodge decomposition, --- a method for solving boundary value problems}, 
%Lecture Notes in Mathematics, 1607. Springer Verlag, Berlin, 1995. viii+155 pp.  

%\bibitem{Shanmugalingam:Newton}
%      {N.~Shanmugalingam}, 
%     \emph{Newtonian spaces: an extension of Sobolev spaces to metric measure spaces}, 
%     Rev. Mat. Iberoamericana {\bf 16} (2000), no. 2, 243--279. 
%\bibitem{Shanmugalingam:Harmonic}
%      {N.~Shanmugalingam}, 
%\emph{Harmonic functions on metric spaces}, Illinois J. Math. {\bf 45} (2001), no. 3, 1021--1050. 

\bibitem{Shigekawa1}
{I.~Shigekawa}, 
\emph{Littlewood-Paley inequality for a diffusion satisfying the logarithmic Sobolev inequality and for the Brownian motion on a Riemannian
manifold with boundary}, Osaka J. Math. {\bf 39} (2002), 897--930.

\bibitem{ShigekawaText}
\bysame,
\emph{Stochastic Analysis}, Translations of Mathematical Monographs {\bf 224}. Iwanami Series in Modern Mathematics. American Mathematical Society, Providence, RI, 2004.

\bibitem{ShigekawaYoshida}
{I.~Shigekawa and N.~Yoshida}, 
\emph{Littlewood-Paley-Stein inequality for a symmetric diffusion}, 
J. Math. Soc. Japan {\bf 44} (1992), 251--280.


% \bibitem{Schu:Fatou}
% {B.~Schmuland}, 
% \emph{Positivity preserving forms have the Fatou property}, 
% Potential Anal. {\bf 10} (1999), no.~4, 373--388.

%\bibitem{SchoenYau:LectDiffGeo}
% {R.~Schoen and S.~T.~Yau}, 
% \emph{Lectures on differential geometry}, 
% Conference Proceedings and Lecture Notes in Geometry and Topology, I. International Press, Cambridge, MA, 1994.

% \bibitem{Staff:Liouville}
% {S.~Stafford},
% \emph{A probabilistic proof of S.-Y. Cheng's Liouville theorem},  
% Ann. Probab. \textbf{18} (1990), no. 4, 1816--1822. 

%\bibitem{Stein}
%{E.~M.~Stein}, 
%\emph{Topics in Harmonic Analysis Related to Littlewood-Paley Theory}, Annals of Math. Studies {\bf 63}, Princeton Univ. Press, 1970.

%   \bibitem{St:DirII}
%         {K.-Th.~Sturm}, 
%        \emph{Analysis on local Dirichlet spaces. II. 
%       Upper Gaussian estimates for the fundamental 
%       solutions of parabolic equations},
%       Osaka J.~Math. {\bf 32} (1995), 275--312.

%\bibitem{Sturm:1995gu}
%K.-T. Sturm, \emph{{Sharp estimates for capacities and applications to
%  symmetric diffusions}}, Probab. Theory Related Fields \textbf{103} (1995),
%  no.~1, 73--89.

%\bibitem{StI} 
%{K.-T.~Sturm},
%\emph{On the geometry of metric measure spaces.~I},
%Acta Math.\ {\bf 196} (2006), 65--131.
% 
%\bibitem{StII} 
%\bysame,
%\emph{On the geometry of metric measure spaces.~II},
%Acta Math.\ {\bf 196} (2006), 133--177.

%\bibitem{Sturm_coupling} 
%\bysame, \emph{Metric measure spaces with variable Ricci bounds and couplings of Brownian motions}. Festschrift Masatoshi Fukushima, 553--575, 
%Interdiscip. Math. Sci., 17, World Sci. Publ., Hackensack, NJ, 2015. 

\bibitem{Sznitzman}
{A.~S.~Sznitman}, 
\emph{Brownian motion and obstacles and random media}, Springer-Verlag Berlin Heidelberg 1998. 

% \bibitem{book_Vil1}
% C.~Villani, \emph{Topics in optimal transportations}, Graduate studies in 
% mathematics, 58, American mathematical society, Providence, RI, 2003.

%\bibitem{Vi2} 
%{C.~Villani}, \emph{Optimal transport, old and new},  Springer-Verlag, Berlin, 2009.

%\bibitem{Yamaguchi:Luminy}
%{T. Yamaguchi}, \emph{A convergence theorem 
%in the geometry of Alexandrov spaces}, Actes
%de la Table Ronde de G\'eom\'etrie Diff\'erentielle (Luminy, 1992), S\'emin. Congr., vol. 1, Soc. Math. France, Paris, 1996, pp. 601--642.

\bibitem{YoshidaNobuo}
{N.~Yoshida}, 
\emph{Sobolev spaces on a Riemannian manifold and their equivalence}, 
J. Math. Kyoto Univ. {\bf 32} (1992), 621--654.

\bibitem{Yosida}
{K.~Yosida}, 
\emph{Functional Analysis}, Sixth Edition. Springer-Verlag. Berlin Heidelberg New York 1980.


%\bibitem{ZhangZhu:RicciAlex}
%{H. C. Zhang and X. P. Zhu}, 
%\emph{Ricci curvature on Alexandrov spaces and rigidity theorems}, 
%Comm. Anal. Geom. {\bf 18} (2010), no.~3, 503--554.



%\bibitem{AliMasRigo}
%{L.~J.~Alias, P.~Mastrolia and M.~Rigoli},
%\emph{Maximum principles and geometric applications}, Springer Monographs in Mathematics. Springer, Cham, 2016.

%{\color{black}{
%\bibitem{Azencott:Behavi}   
%{R.~Azencott}, \emph{Behavior of diffusion semi-groups at infinity}, 
%Bull. Soc. Math. France {\bf 102} (1974), 193--240.  
%}}



%\bibitem{B}
%{V.~Bayle},
%\emph{Propri\'et\'es de concavit\'e du profil isop\'erim\'etrique et applications},
%PhD Thesis, Universit\'e Joseph-Fourier-Grenoble I, 2003.

%  \bibitem{BG:Markov}
%       {R. M. Blumenthal and R. K. Getoor},
%       \emph{Markov processes and potential theory},
%       Pure and Applied Mathematics, Vol.~{\bf 29} Academic
%       Press, New York-London 1968.
%
%
%\bibitem{BRW:invariant}
%{V.~I.~Bogachev, M.~R\"ockner and F.-Y.~Wang},
%\emph{Elliptic equations for invariant measures on finite and infinite dimensional manifolds}, J. Math. Pures Appl. (9) {\bf 80} (2001), no. 2, 177--221.
%
%
%\bibitem{Brighton}
%{K.~Brighton},
%\emph{A Liouville-type theorem for smooth metric measure spaces}, J. Geom. Anal. {\bf 23} (2013), no.~2, 562--570.
%
%\bibitem{BKMW}
%{A.~Burtscher, C.~Ketterer, R.~J.~McCann and E.~Woolgar},
%\emph{Inscribed radius bounds for lower Ricci bounded metric measure spaces with mean convex boundary}, preprint 2020,
%Available from {\tt arXiv:2005.07435v4}.

%\bibitem{ChenJostQiu}
%{Q.~Chen, J.~Jost and H.~Qiu},
%\emph{Existence and Liouville theorems for $V$-harmonic maps
%from complete manifolds}, Ann. Glob. Anal. Geom. {\bf 42} (2012), 565--584.
%
%\bibitem{ChenJostWang}
%{Q.~Chen, J.~Jost and G.~Wang},
%\emph{A maximum principle for generalizations of harmonic maps in Hermitian, affine, Weyl, and Finsler geometry}, J. Geom. Anal. {\bf 25} (2015), 2407--2426.
%
%\bibitem{SYCheng}
%{S.-Y.~Cheng},
%\emph{A Liouville theorem for harmonic maps}, In Proc.
%Sympos. Pure Math. {\bf 36} (1980), 147--151. Amer. Math. Soc., Province, R. I.
%
%\bibitem{Choi}
%{H.~Choi},
%\emph{On the Liouville theorem for harmonic maps},
%Proc. Amer. Math. Soc. {\bf 85} (1982), no.~1, 91--94.
%
%\bibitem{FLZ}
%{F.~Fang, X.-D.~Li and Z.~Zhang},
%\emph{Two generalizations of Cheeger-Gromoll splitting theorem via Bakry-Emery Ricci curvature}, Ann. Inst. Fourier (Grenoble) {\bf 59} (2009), no. 2, 563--573.
%
%\bibitem{FLL}
%{A.~Futaki, H.-Z.~Li and  X.-D.~Li},
%\emph{On the first eigenvalue of the Witten Laplacian and the diameter of compact shrinking Ricci solitons}, Ann. Global Anal. Geom. {\bf 44} (2013), no.~2, 105--114.
%
%
%

%\bibitem{CK}
%{C.~B.~Croke and B.~Kleiner},
%\emph{A warped product splitting theorem},
%Duke Math. J. {\bf 67} (1992), no. 3, 571--574.

%\bibitem{Fuk:StrictDecomposition}
%{M. Fukushima},
%\emph{On a strict decomposition of additive functionals for symmetric diffusion processes}, Proc. Japan Acad. {\bf 70} Ser. A (1994), no.~9, 277--281.

%\bibitem{Fuk:Semimartingale}
% {M.~Fukushima},
%\emph{On semi-martingale characterizations of functionals of symmetric Markov processes},
%Electron. J. Probab. {\bf 4} (1999), no.~18, 1--32.

%\bibitem{FOT}
%    {M.~Fukushima, Y.~Oshima and M.~Takeda},
%   \emph{Dirichlet forms and symmetric Markov processes}, Second revised and extended edition. de Gruyter Studies in Mathematics, {\bf 19}. Walter de Gruyter \& Co., Berlin, 2011.
%
%\bibitem{GreeneWu}
%{R. Greene and H.~Wu},
%\emph{Function theory on manifolds which possess a pole}, Lecture Notes in Math. {\bf 699}. Springer Berlin.
%
% \bibitem{Get:TranRec}
%        {R.~K.~Getoor},
%        \emph{Transience and recurrence of Markov processes}
%Seminar on Probability, XIV (Paris, 1978/1979) (French), 397--409,
%Lecture Notes in Math., {\bf 784}, Springer, Berlin, Heidelberg, New York, 1980, 397--409.
%
%\bibitem{HackenhThal:1994}
%{W.~Hackenbroch and A.~Thalmaier},
%\emph{Stochastische Analysis. {\rm(}German{\rm)} {\rm[}Stochastic analysis{\rm]} Eine Einf\"uhrung in die Theorie der stetigen Semimartingale. {\rm[}An introduction to the theory of continuous semimartingales{\rm]}}, Mathematische Leitf\"aden. [Mathematical Textbooks] B. G. Teubner, Stuttgart, 1994.

%\bibitem{HK}
%{E.~Heintze and H.~Karcher},
%\emph{A general comparison theorem with applications to volume estimates for submanifolds},
%Ann. Sci. Ecole Norm. Sup. {\bf 11} (1978), 451--470.

%\bibitem{HildtJostWidemann}
%{S.~Hildebrandt, J.~Jost, K.~O.~Widman},
% \emph{Harmonic mappings and minimal submanifolds}, Invent. Math. {\bf 62} (1980),
%269--298.

%{\color{black}{
%\bibitem{Hsu:1989}
%{P. Hsu}, 
%\emph{Heat semigroup on a complete Riemannian manifold}, Ann. Probab. {\bf 17} (1989), 
%1248--1254.
%}}

%\bibitem{Hsu:2001}
%{E.~P.~Hsu},
%\emph{Stochastic analysis on manifolds},
% Graduate Studies in Mathematics, {\bf 38}. American Mathematical Society, Providence, RI, 2002.
%
%\bibitem{Hua2019}
%{B.~Hua},
%\emph{Liouville theorem for bounded harmonic functions
%on manifolds and graphs satisfying non-negative curvature dimension condition},
%Calc. Var. {\bf 58} (2019), no.~2,  42.
%
%\bibitem{HuaKellXia}
%{B.~Hua, M.~Kell and C. Xia},
%\emph{Harmonic functions on metric measure spaces},
%Preprint. Available at \textsf{arXiv}:1308.3607v2.
%
%\bibitem{K2}
%{A. Kasue},
%\emph{Ricci curvature, geodesics and some geometric properties of Riemannian manifolds with boundary},
%J. Math. Soc. Japan {\bf 35} (1983), no. 1, 117--131.

%\bibitem{K3}
%\bysame,
%\emph{On a lower bound for the first eigenvalue of the Laplace operator on a Riemannian manifold},
%Ann. Sci. \'Ecole Norm. Sup. (4) {\bf 17} (1984), no. 1, 31--44.

%\bibitem{K4}
%\bysame,
%\emph{Applications of Laplacian and Hessian Comparison Theorems},
%Geometry of geodesics and related topics (Tokyo, 1982), 333--386, Adv. Stud. Pure Math., 3, North-Holland, Amsterdam, 1984.


%\bibitem{Kendall} 
%{W.~S.~Kendall}, 
%\emph{The radial part of Brownian motion on a manifold: a semimartingale property},
%Ann. Prob. {\bf 15} (1987), 
%1491--1500.
%
%
%\bibitem{Kend:martingalemanifold}
%    \bysame,
%    \emph{Martingales on manifolds and harmonic maps, Geometry of random motion }(Ithaca, N.Y., 1987), 121--157, \text{Contemp. Math.}, \textbf{73}, Amer. Math. Soc., Providence, RI, 1988.
%
%    \bibitem{Kend:probconvI}
%    \bysame,
%     \emph{Probability, convexity, and harmonic maps with small image I:  Uniqueness and fine     existence}, \text{Proc. London Math. Soc.}, (3) \textbf{61} (1990), no. 2, 371--406.
%
%
%\bibitem{Kend:hemisphere}
%\bysame,
%\emph{Convexity and the hemisphere}, J. London Math. Soc. {\bf 43} (1991b), no.~2, 567--576.
%
%\bibitem{Kend:probconvII}
%\bysame,
%\emph{Probability, convexity, and harmonic maps with small image II:  Smoothness via probablistic gradient inequalities}, \text{Jour. Func. Anal.}, \textbf{126} (1994), no. 1, 228--257.
%
%
%\bibitem{KolMil:2017}
%{A.~Kolesnikov and E.~Milman},
%\emph{Brascamp-Lieb-type inequalities on weighted Riemannian manifolds with boundary},
%J. Geom. Anal. {\bf 27} (2017), no.~2, 1680--1702.
%
%\bibitem{K}
%{K.~Kuwada},
% \emph{A probabilistic approach to the maximal diameter theorem}, Math. Nachr. {\bf 286}  (2013),
%no.~4, 374--378.

%{\color{black}{
%  \bibitem{Kw:maximumprinciple}
%       {K.~Kuwae},
%        \emph{Maximum principles for subharmonic functions via local semi-Dirichlet forms},
%       Canadian J. Math. {\bf 60}, (2008), no. 4, 822--874.
%
%
%\bibitem{KL}
%{K.~Kuwae and X.-D.~Li},
%\emph{New Laplacian comparison theorem
%and its applications to diffusion processes
%on Riemannian manifolds}, % online published in 
% Bulletin of London Math. Soc. {\bf 54} (2022), no.~2, 404--427. 
 %http://doi.org/10.1112/blms.12568, 
 %Available from {\tt arXiv:2001.00444}.
%}}
%
%\bibitem{KSa}
%{K.~Kuwae and Y.~Sakurai},
%\emph{Rigidity phenomena on lower $N$-weighted Ricci curvature with $\eps$-range for non-symmetric Laplacian}, 
%Illinois J. Math. {\bf 65} (2021), no.~4, 847--868.
%
%
%\bibitem{KS}
%{K.~Kuwae and T.~Shukuri},
%\emph{Laplacian comparison theorem on Riemannian manifolds with modified $m$-Bakry-Emery Ricci lower bounds for $m\leq1$},  Tohoku Math. J. {\bf 74} (2022), no.~1, 1--25. 
%
%
%
%{\color{black}{
%\bibitem{KLLSa}
%{K.~Kuwae, S.~Li, X.-D.~Li and Y.~Sakurai},
%\emph{Gradient estimate of
%$\Delta_V$-harmonic maps under
% $(m,V)$-Ricci curvature bounds for non-positive $m$}, in preparation. 2021.
%}}


%\bibitem{Li12}
%\bysame,
%\emph{Perelman's entropy formula for the Witten Laplacian on Riemannian manifolds via Bakry-Emery Ricci curvature}, Math. Ann. {\bf 353} (2012), no.~2, 403--437.

%\bibitem{LL15}
%{S. Li and  X.-D. Li},
%\emph{W-entropy formula for the Witten Laplacian on manifolds with time dependent metrics and potentials}, Pacific J. Math. {\bf 278} (2015), no.~1, 173--199.

%\bibitem{LL-JFA18}
%{S. Li and X.-D. Li}, \emph{Hamilton differential Harnack inequality and W-entropy for Witten}
%\emph{Laplacian on Riemannian manifolds}, J. Funct. Anal.
% {\bf 274} (2018), no. 11, 3263--3290.
%%(2017), {\tt https://doi.org/10.1016} {\tt /j.jfa.2017.09.017}
%
%
%\bibitem{LL-AJM18}
%\bysame, \emph{On Harnack inequalities for Witten Laplacian on Riemannian manifolds with
% super Ricci flows},  Asian J. Math.  {\bf 22} (2018), no. 3,  577--598.
%
%
%\bibitem{LL-SCM19}
%\bysame,
%\emph{On the Li-Yau-Hamilton Harnack inequalities for Witten Laplacian on Ricci flows and super Ricci flows} (in Chinese). Sci. Sin.  Math,  {\bf  49} (2019), 1613--1632.
% %, doi: 10.1360/N012019-00044
%
%\bibitem{LLL}
%\bysame,
%\emph{On Liouville theorems and Harnack inequalities
%for non-symmetric  diffusion operators on complete Riemannian manifolds}, in preparation. 2021.
%

%\bibitem{LW1}
%{H.~Li and Y.~Wei},
%\emph{Rigidity theorems for diameter estimates of compact manifold with boundary},
%Int. Math. Res. Not. IMRN (2015), no. 11, 3651--3668.

%\bibitem{LW2}
%\bysame,
%\emph{$f$-minimal surface and manifold with positive $m$-Bakry-\'Emery Ricci curvature},
%J. Geom. Anal. {\bf 25} (2015), no. 1, 421--435.

% \bibitem{Xdli:Liouville}
% {X.-D.~Li},
% \emph{Liouville theorems for symmetric diffusion operators on complete Riemannian manifolds},  J. Math. Pures Appl. (9) {\bf 84} (2005), no.~10, 1295--1361.

% \bibitem{LiX:PerelmanEntropyFor}
% \bysame,
% \emph{Perelman's entropy formula for the Witten Laplacian on Riemannian manifolds via Bakry-Emery Ricci curvature},
% Math. Ann. {\bf 353} (2012), no. 2, 403--437.

%\bibitem{Li-SPA16}
%\bysame,  \emph{Hamilton's Harnack inequality and the $W$-entropy formula on complete Riemannian manifolds},  Stochastic Processes and their Applications {\bf 126} (2016), no.~4, 1264--1283.
%
%\bibitem{YiLi:2015}
%{Y.~Li},
%\emph{Li-Yau-Hamilton estimates and Bakry-Emery-Ricci
%curvature}, Nonlinear Analysis {\bf 113} (2015),
%1--32.
%
%{\color{black}{
%\bibitem{LiYau}
%{P.~Li and S.-T.~Yau}, 
%On the parabolic kernel of the Schr\"odinger operator, Acta Math. {\bf 156} (1986), 153--200.
%}}
%
%\bibitem{Lich}
%{A.~Lichnerowicz}, \emph{Vari\'et\'es riemanniennes \`a tenseur C non n\'egatif}. (French)
%C.\ R.\ Acad.\ Sci.\ Paris S\'er.\ A-B {\bf 271} (1970), A650--A653.
%
%\bibitem{Lim}
%{A.~Lim},
%\emph{The splitting theorem and topology of noncompact spaces with nonnegative $N$-Bakry \'Emery Ricci curvature}, Proc. Amer. Math. Soc. {\bf 149} (2021), no.~8,  
%3515--3529. 
%%Available from {\tt  arXiv:2001.06028v3}.
%
%\bibitem{Lo}
%{J. Lott},
%\emph{Some geometric properties of the Bakry-\'Emery Ricci tensor}, Comment. Math. Helv. {\bf 78} (2003), no.~4,
%865--883.
%
%\bibitem{Loustau}
%{B.~Loustau}, \emph{Harmonic maps from K\"ahler manifolds}, preprint, 2020, available from {\tt https://arxiv.org/pdf/2010.03545.pdf}.
%
%
%
%{\color{black}{
%
%\bibitem{MR}
%  {Z.-M.~Ma and M.~R\"ockner}, 
%          {\em Introduction to the Theory of (Non-Symmetric) Dirichlet Forms},
%           Springer Universitext, 1992.
%
%
%\bibitem{LMO:CompaFinsler}
%{Y.~Lu, E.~Minguzzi and S.~Ohta},
%\emph{Comparison theorems on weighted Finsler manifolds and spacetimes with $\varepsilon$-range},
%Anal. Geom. Metr. Spaces {\bf 10} (2022), 1--30. 
%%preprint (2020), Available from
%%{\tt arXiv:2007.00219}.
%
%}}
%
%
%\bibitem{MRS:2021}
%{M.~Magnabosco, C.~Rigoni and G.~Sosa},
%\emph{Convergence of metric measure spaces satisfying the CD condition for negative values of the dimension parameter}, preprint 2021, available from {\tt https://arxiv.org/pdf/2104.03588.pdf}.
%
%\bibitem{Mai1}
%{C.~H.~Mai},
%\emph{Rigidity for the isoperimetric inequality of negative effective dimension on weighted Riemannian manifolds},
%Geom. Dedicata {\bf 202} (2019), 213--232.
%
%\bibitem{Mai2}
%\bysame,
%\emph{On Riemannian manifolds with positive weighted Ricci curvature of negative effective dimension},
%Kyushu J. Math. {\bf 73} (2019), no. 1, 205--218.
%
%%\bibitem{M}
%%{E.~Milman},
%%\emph{Sharp isoperimetric inequalities and model spaces for curvature-dimension-diameter condition},
%%J. Eur. Math. Soc. {\bf 17} (2015), no.~5, 1041--1078.
%
%\bibitem{Mineg}
%{E.~Milman},
%\emph{Beyond traditional curvature-dimension I:
%new model spaces for isoperimetric and concentration inequalities in negative dimension},
%Trans. Amer. Math. Soc. {\bf 369} (2017), no. 5, 3605--3637.
%
%%\bibitem{Mo}
%%{F.~Morgan},
%%\emph{Manifolds with density},
%%Notices of the AMS (2005), 853--858.
%
%\bibitem{Nishikawa}
%{S.~Nishikawa},
%\emph{Variational problems in geometry}, Translations of Mathematical Monographs,
%{\bf 205}, Amer. Math. Soc. Province, R. I.
%
%\bibitem{Oh<0}
%{S. Ohta},
%\emph{$(K,N)$-convexity and the curvature-dimension condition for negative $N$},
%J.\ Geom.\ Anal.\ {\bf 26} (2016), no.~3, 2067--2096.
%
%{\color{black}{
%\bibitem{Oshima}
%{Y.~Oshima}, 
%\emph{Semi-Dirichlet forms and Markov processes}, de Gruyter Studies in Mathematics, {\bf 48}. Walter de Gruyter \& Co., Berlin, 2013.
%}}

%\bibitem{Pet:RiemannianGeo}
%{P.~Petersen},
%\emph{Riemannian geometry}, Third edition, Graduate Texts in Mathematics, vol. {\bf 171}, Springer, Cham, 2016.

%\bibitem{Qi}
%{Z.-M.~Qian},
%\emph{Estimates for weighted volumes and applications},
%Quart.\ J.\ Math.\ Oxford Ser.\ (2) {\bf 48} (1997), 235--242.
%
%\bibitem{Qian}
%\bysame,
%\emph{On conservation of probability and the Feller property}, Ann. Probab. {\bf 24} (1996), no.~1, 280--292.
%
%
%\bibitem{Qiu}
%{H.~Qiu},
%\emph{The heat flow of $V$-harmonic maps from complete manifolds into regular balls},
%Proc. Amer. Math. Soc. {\bf 145} (2017), no.~5, 2271--2280.
%
%
%  \bibitem{Sakai}
%      {T.~Sakai},
%      \emph{Riemannian geometry}.
%      Translated from the 1992 Japanese original by the author.
%     Translations of Mathematical Monographs, {\bf 149}.
%     American Mathematical Society, Providence, RI, 1996.
%

%\bibitem{Sa1}
%{Y.~Sakurai},
%\emph{Rigidity of manifolds with boundary under a lower Ricci curvature bound}, Osaka J. Math. {\bf 54} (2017), no. 1, 85--119.

%\bibitem{Sa2}
%\bysame,
%\emph{Rigidity of manifolds with boundary under a lower Bakry-\'Emery Ricci curvature bound}, Tohoku Math. J. (2) {\bf 71} (2019), no. 1, 69--109.

%\bibitem{Sak}
%{Y.~Sakurai},
%\emph{Comparison geometry of manifolds with boundary under a lower weighted Ricci curvature bound}, Canad. J. Math. {\bf 72} (2020), no. 1, 243--280.
%
%
%\bibitem{S-Y}
%{R. Schoen and S.~T.~Yau},
%\emph{Lectures on Harmonic Maps}, International Press Incorporated, Boston, 1997.

%{\color{black}{
%   \bibitem{Shar:gene}
%       {M.~Sharpe}, 
%       \emph{General theory of Markov processes}, 
%       Pure and Applied Mathematics {\bf 133}. 
%       Academic Press, Inc., Boston, MA, 1988.   
%
%}}
%\bibitem{SiuYau}
%{Y.~T.~Siu and S.~T.~Yau},
%\emph{Complete K\"ahler manifolds with nonpositive curvature of faster than quadratic decay}, Ann. Math. {\bf 105} (1977), 225--264.
%
% \bibitem{Staff:Liouville}
% {S.~Stafford},
% \emph{A probabilistic proof of S.-Y. Cheng's Liouville theorem},
% Ann. Probab. \textbf{18} (1990), no. 4, 1816--1822.
%
%\bibitem{Yau:Harmonic}
%{S.~T.~Yau},
%\emph{Harmonic functions on complete Riemannian manifolds}, Comm. Pure Appl. Math. {\bf 28} (1975), 201--228.
%
%\bibitem{FYWang}
%{F.-Y.~Wang},
%\emph{Analysis for diffusion processes on Riemannian manifolds}, Advanced Series on Statistical Science \& Applied Probability, {\bf 18}. World Scientific Publishing Co. Pte. Ltd., Hackensack, NJ, 2014.
%
%\bibitem{WW}
%{G.~Wei and W.~Wylie},
%\emph{Comparison geometry for the Bakry-Emery Ricci tensor},
%J.\ Differential Geom.\ {\bf 83} (2009), no.~2, 377--405.

%\bibitem{Wu}
%{J.-Y. Wu},
%\emph{$L^p$-Liouville theorems on complete smooth metric measure spaces},
%Bull. Sci. math. {\bf 138} (2014), no.~4, 510--539.

%
%\bibitem{Wy}
%{W.~Wylie},
%\emph{A warped product version of the Cheeger--Gromoll splitting theorem}, Trans. Amer. Math. Soc.
%{\bf 369} (2017), no. 9, 6661--6681.
%
%\bibitem{WyYero}
%{W. Wylie and D. Yeroshkin},
%\emph{On the geometry of Riemannian manifolds with density}, preprint 2016,
%Available from {\tt arXiv:1602.08000}.



\end{thebibliography}
% \bibliographystyle{amsplain}
% \bibliography{refs}
\end{document}

