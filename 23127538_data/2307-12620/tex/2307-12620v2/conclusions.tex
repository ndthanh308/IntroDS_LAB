\section{Conclusion}\label{sec:conclusions}

We have focused on temporal logic programming within the context of Temporal Equilibrium Logic over finite traces.
More precisely, we have studied a fragment close to logic programming rules in the spirit of~\cite{gabbay87a}: a past-present temporal logic program
consists of a set of rules whose body refers to the past and present while their head refers to the present.
This fragment is very interesting for implementation purposes since it can be solved by means of incremental solving techniques as implemented in \telingo{}.
%Moreover, restricting the body of the rules to only present and past formulas makes it so that the body of the rules can be seen as queries on the set of conclusions generated during the solving phase.

Contrary to the propositional case~\cite{feleli06a}, where answer sets of an arbitrary propositional formula can be captured by means of the classical models of another formula $\psi$, in the temporal case, this is impossible to do the same mapping among the temporal equilibrium models of a formula $\varphi$ and the \LTL{} models of another formula $\psi$~\cite{bozpea16a}.

In this paper, we show that past-present temporal logic programs can be effectively reduced to \LTL{} formulas by means of completion and loop formulas.
More precisely, we extend the definition of completion and temporal loop formulas in the spirit of Lin and Zhao~\cite{linjzh03a} to the temporal case, and we show that for tight past-present programs, the use of completion is sufficient to achieve a reduction to an \LTLf{} formula.
Moreover, when the program is not tight, we also show that the computation of the temporal completion and a finite number of loop formulas suffices to reduce \TELf{} to \LTLf{}.
%Finally, we consider Ferraris et al. approach~\cite{feleli06a} where the computation of the completion can be automatically replaced by the consideration of unitary loops.
%In this contribution, we extend such result to the case of past-present logic programs.

%As future work, we plan to study in detail the relation between temporal completion and loop formulas and \emph{unfounded sets}~\cite{feleli06a}, since the latter play a central role in the solving algorithm of \clingo{}~\cite{gekakasc14a}.
%Lastly, we will study the case of past-present temporal programs with variables~\cite{AguadoCPVD17} in order to get a second-order characterisation of loop formulas, in the spirit of~\cite{leemen08a}.
