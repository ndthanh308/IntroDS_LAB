\section{Proof of Theorem~\ref{thm:ppp2pcp}}


To prove Theorem~\ref{thm:ppp2pcp}, we first translate $P$  using the alternative definition $\eta^*(\mu)$ for each
subformula $\mu \in \mathit{subf}(P)$. Each definition $\eta^*(\mu)$ is then transformed into $\eta^{**}(\mu)$ by
replacing double implications by simple implications, using Lemma~\ref{lem:completion:tmp}.
Finally, $\eta^{**}(\mu)$ can be transformed into $\eta(\mu)$ as $\alwaysF\varphi$ is equivalent to
$\wnext\alwaysF\varphi \wedge \varphi$. Table~\ref{tab:ppp2tlp} shows the definitions $\eta^*$ and $\eta^{**}$.


\input{tables/table-ppp2tlp}

For practical reason, the proof uses the alternative three-valued characterization of \THT\ presented in ???.
\comment{F: add ref telf long}
Given an \HT-trace $\M$,  its associated truth valuation  is defined as a function $\trival{k}{\varphi}$ that assigns
a truth value in $\{0,1,2\}$ to formula $\varphi$ at time point $\rangeo{k}{0}{\lambda}$,  according to the
following rules:
\begin{eqnarray*}
\trival{k}{\bot} & \eqdef & 0 \\
\trival{k}{a} & \eqdef &
    \begin{cases}
    0 & \text{if} \ a \not\in T_k \\
    1 & \text{if} \ a \in T_k\setminus H_k \\
    2 & \text{if} \ a \in H_k
    \end{cases}\hspace{20pt} \text{for any atom } a\\
\trival{k}{\neg \varphi} & \eqdef &
    \begin{cases}
    2  & \text{if } \trival{k}{\varphi}=0 \\
    0  & \text{otherwise}
    \end{cases}\\
\trival{k}{\varphi \wedge \psi} & \eqdef &
    \min(\trival{k}{\varphi},\trival{k}{\psi})\\
\trival{k}{\varphi \vee   \psi} & \eqdef &
    \max(\trival{k}{\varphi},\trival{k}{\psi})\\
\trival{k}{\previous \varphi} & \eqdef &
    \begin{cases}
    0                      & \text{if } k=0 \\
    \trival{k-1}{\varphi}  & \text{if } k>0
    \end{cases}\\
\trival{k}{\varphi \since \psi} & \eqdef &
    \max\{\min(\trival{j}{\psi},\min\{\trival{i}{\varphi}\mid j<i\leq k\} ) \mid 0\leq j\leq k\}\\
\trival{k}{\varphi \trigger \psi} & \eqdef &
    \min\{\max(\trival{j}{\psi},\max\{\trival{i}{\varphi}\mid j<i\leq k\} ) \mid 0\leq j\leq k\}\\
\end{eqnarray*}

\begin{proposition}
    Let $\tuple{\H,\T}$ be an \HT-trace of length $\lambda$, ${\bm m}$ its associated valuation
    and $\rangeo{k}{0}{\lambda}$:
        \begin{itemize}
            \item $\tuple{\H,\T},k \models \varphi$ iff $\trival{k}{\varphi}=2$
            \item $\tuple{\T,\T},k \models \varphi$ iff $\trival{k}{\varphi}\neq 0$
        \end{itemize}
\end{proposition}

\vspace{1cm}

\begin{lemma}\label{lem:aux}
    Let $\varphi$ be a temporal formula over alphabet \A\ and $\tuple{\H,\T}$ an \HT-trace of length $\lambda$ over \A,
    and let $\X$ and $\Y$ be two traces of same length $\lambda$ such that $ \X\leq \Y$, $X_i\cap\A=\emptyset$ and
    $Y_i\cap\A=\emptyset$ for all $\rangeo{i}{0}{\lambda}$.
    Then, $\tuple{\H,\T}\models \varphi$ iff $\tuple{\H\cup \X,\T\cup \Y}\models \varphi$.
\end{lemma}

\begin{proofof}{Lemma~\ref{lem:aux}}
    By induction on the structure of $\varphi$.\\
\end{proofof}



We define the translation $\sigma^*$ as the present-centered program:
\begin{align*}
    \sigma^*(P) = \initial{P}\cup\final{P}\cup
    \left\lbrace \wnext\alwaysF\big(\Head{r} \leftarrow\Lab{\body{r}}\big) \mid r \in \dynamic{P} \right\rbrace
	\cup \left\lbrace \eta^*(\mu)
	\mid \mu \in \mathit{subf}(P)\right\rbrace
\end{align*}

\begin{lemma}\label{lem:nf1}
    Let $\tuple{\H,\T}$ be a \THTf{} model of length $\lambda$ of a past-present program $P$ over $\mathcal{A}$.
    Then, there exists some \THTf{}-trace $\tuple{\H',\T'}$ over alphabet $\mathcal{A}^+$
    such that $\tuple{\H,\T} = \tuple{\H',\T'}|_{\mathcal{A}}$
    and $\tuple{\H',\T'} \models \sigma^*(P)$.
\end{lemma}

\begin{proofof}{Lemma~\ref{lem:nf1}}

Take the \THTf{}-trace $\tuple{\H',\T'}$ whose three valued interpretation ${\bm m}'$ satisfies:
\begin{align}
	\trivalp{k}{\Lab{\varphi}} \overset{\label{proofofNF_equalityExtAlph}}{=} \trival{k}{\varphi}
\end{align}
for any formula $\varphi$ over $\mathcal{A}$ and for all $\rangeo{k}{0}{\lambda}$.
When $\varphi$ is an atom $a \in \mathcal{A}$ then $\trivalp{k}{a} = \trivalp{k}{\Lab{a}} = \trival{k}{a}$, which
implies that both valuations coincide for atoms, and so, $\tuple{\H',\T'} |_{\mathcal{A}} = \tuple{\H,\T}$.
It remains to be shown that $\tuple{\H',\T'} \models \sigma^*(\Gamma)$, which is equivalent to
\begin{align*}
    \tuple{\H',\T'} &\models
        \initial{P}\cup\final{P}\cup
    \left\lbrace \wnext\alwaysF\big(\Head{r} \leftarrow\Lab{\body{r}}\big) \mid r \in \dynamic{P} \right\rbrace
	\cup \left\lbrace \eta^*(\mu)
	\mid \mu \in \mathit{subf}(P)\right\rbrace\\
    \Leftrightarrow \tuple{\H',\T'} &\models
        \initial{P}\text{ and }
        \tuple{\H',\T'} \models \final{P}\text{ and }\\
    \tuple{\H',\T'}& \models
          \left\lbrace \wnext\alwaysF\big(\Head{r} \leftarrow\Lab{\body{r}}\big) \mid r \in \dynamic{P} \right\rbrace
     	  \text{ and }
        \tuple{\H',\T'} \models  \left\lbrace \eta^*(\mu \mid \mu \in \mathit{subf}(P)\right\rbrace
\end{align*}

The first two satisfaction relations follow directly from Lemma~\ref{lem:aux} as we have that $\tuple{\H,\T}$ is a model
of $P$.

Proving the third part is equivalent to proving that for all  $r \in \dynamic{P}$,
$\tuple{\H',\T'}, k \models \Head{r} \leftarrow\Lab{\body{r}} $ for all $\rangeo{k}{1}{\lambda}$.
Assume that $\tuple{\H',\T'}, k \models \Lab{\body{r}} $ for some $\rangeo{k}{1}{\lambda}$ and some $r \in \dynamic{P}$.
Then,  since $\trivalp{0}{\Lab{\gamma}}=2$ iff $\trival{0}{\gamma}=2$, we have $\tuple{\H',\T'}, k \models \body{r} $.
As $\tuple{\H,\T}$ is a model of $P$, $\tuple{\H,\T}, k \models \Head{r} \leftarrow\Lab{\body{r}}$, therefore
$\tuple{\H,\T}, k \models \Head{r}$. By Lemma~\ref{lem:aux}, we also get $\tuple{\H',\T'}, k \models \Head{r}$, and then
$\tuple{\H',\T'}, k \models \Head{r} \leftarrow\Lab{\body{r}} $.
As $k$ and $r$ where chosen arbitrarily, we get that for all  $r \in \dynamic{P}$,
$\tuple{\H',\T'}, k \models \Head{r} \leftarrow\Lab{\body{r}} $ for all $\rangeo{k}{1}{\lambda}$.

For the last part, we consider the following cases depending on the structure of the subformula $\mu$:

\begin{enumerate}
     \item
        For $\mu = \neg \varphi$ we have $\eta(\mu) = \alwaysF (\Lab{\mu} \leftrightarrow \neg\Lab{\varphi} )$ and so,
        $\tuple{\H',\T'} \models \eta(\mu)$ amounts to proving
        $\trivalp{k}{\Lab{\mu}}=\trivalp{k}{\neg\Lab{\varphi}}$ for all $\rangeo{k}{0}{\lambda}$.\\
	    In this case we have
        	\begin{align*}
	        	\trivalp{k}{\Lab{\mu}} = \trival{k}{\mu}
                &= \trival{k}{\neg\varphi}\\
                &=
                \begin{cases}
                    2  & \text{if } \trival{k}{\varphi}=0 \\
                    0  & \text{otherwise}
                \end{cases}\\
                &=
                \begin{cases}
                    2  & \text{if } \trivalp{k}{\Lab\varphi}=0 \\
                    0  & \text{otherwise}
                \end{cases}\\
                &= \trivalp{k}{\neg\Lab\varphi}\\
            \end{align*}


    \item
        For $\mu = \varphi \otimes \psi$ with $\otimes \in \{\wedge, \vee\}$ we have
        $\eta(\mu) = \alwaysF (\Lab{\mu} \leftrightarrow \Lab{\varphi} \otimes \Lab{\psi})$ and so,
        $\tuple{\H',\T'} \models \eta(\mu)$ amounts to proving
        $\trivalp{k}{\Lab{\mu}}=\trivalp{k}{\Lab{\varphi} \otimes \Lab{\psi}}$ for all $\rangeo{k}{0}{\lambda}$.\\
	    In this case we have
        	\begin{align*}
	        	\trivalp{k}{\Lab{\mu}} = \trival{k}{\mu}
                &= \trival{k}{\varphi\otimes\psi}\\
                &= f^\otimes (\trival{k}{\varphi}, \trival{k}{\psi})\\
                &= f^\otimes (\trivalp{k}{\Lab{\varphi}}, \trivalp{k}{\Lab{\psi}})\\
                &= \trivalp{k}{\Lab{\varphi} \otimes \Lab{\psi} }
            \end{align*}
        where $f^\otimes$ is the three-valued mapping associated the corresponding Boolean connective
        $\otimes \in \{\wedge, \vee, \to\}$ in the \THT{} three-valued semantics.
    \item
        For $\mu = \previous \varphi$ we have two formulas in $\eta(\mu)$:
        \begin{itemize}
            \item[-]
                For the formula $\wnext \alwaysF(\Lab\mu \leftrightarrow \previous \Lab\varphi)$ note that the prefix
                $\wnext \alwaysF$ means that the double implication must be satisfied for any $\rangeo{k}{1}{\lambda}$
                and, moreover, that this is trivially true when $\lambda=1$.
                So, we have to prove $\trivalp{k}{\Lab\mu}=\trivalp{k}{\previous \Lab\varphi}$ for all
                $\rangeo{k}{0}{\lambda}$ and may assume $\lambda>1$. The proof can be obtained as follows:
        	    \begin{align*}
		            \trivalp{k}{\Lab{\mu}} = \trival{k}{\mu}
            		&= \trival{k}{\previous \varphi}\\
            		&= \trival{k-1}{\varphi}\\
            		&= \trivalp{k-1}{\Lab\varphi}\\
            		&= \trivalp{k}{\previous \Lab{\varphi} }
            	\end{align*}
            \item[-] For satisfying the formula $\neg \Lab\mu$, this is the same than requiring $\trivalp{0}{\Lab\mu}=0$
                and this follows from $\trivalp{0}{\Lab\mu}=\trival{0}{\mu}=\trival{0}{\previous \varphi}=0$.
        \end{itemize}

    \item
        For $\mu = \varphi \since \psi$, we have to prove that
        $\trivalp{k}{\Lab{\mu}}=\trivalp{k}{\Lab{\psi} \vee ( \Lab{\varphi} \wedge \previous \Lab{\mu})}$
        for all $\rangeo{k}{0}{\lambda}$.
        This can be proved as follows:
        \begin{itemize}
            \item[--] case $k=0$
                \[\begin{array}{rll}
                    & \trivalp{0}{\Lab{\mu}}= \trival{0}{\mu} \\
                    &= \trival{0}{\varphi \since \psi}\\
                    &= \trival{0}{\psi \vee \varphi \wedge \previous \underbrace{(\varphi \since \psi)}_\mu}
                    & \text{by temporal dual of } \eqref{f:induntil}\\
                    &= \max(\trival{0}{\psi},\min(\trival{0}{\varphi},0) ) \\
                    &= \max(\trivalp{0}{\Lab\psi},\min(\trivalp{0}{\Lab\varphi},\trivalp{0}{\previous\Lab\mu}) ) \\
                    &= \trivalp{0}{\Lab\psi \vee \Lab\varphi \wedge \previous \Lab\mu}
                \end{array}\]
            \item[--] case $k>0$
                \[\begin{array}{rll}
                    & \trivalp{k}{\Lab{\mu}}= \trival{k}{\mu} \\
                    &= \trival{k}{\varphi \since \psi}\\
                    &= \trival{k}{\psi \vee \varphi \wedge \previous \underbrace{(\varphi \since \psi)}_\mu}
                    & \text{by temporal dual of } \eqref{f:induntil}\\
                    &= \max(\trival{k}{\psi},\min(\trival{k}{\varphi},\trival{k-1}{\mu}) ) \\
                    &= \max(\trivalp{k}{\Lab\psi},\min(\trivalp{k}{\Lab\varphi},\trivalp{k-1}{\Lab\mu}) ) \\
                    &= \trivalp{k}{\Lab\psi \vee \Lab\varphi \wedge \previous \Lab\mu}
                \end{array}\]
        \end{itemize}

    \item
        For $\mu = \varphi \trigger \psi$, the proof is analogous as for $\since$, exchanging the roles of $\wedge/\min$
        and $\vee/\max$ and using the temporal dual of \eqref{f:indrelease} instead of the one for \eqref{f:induntil}.
\end{enumerate}
\end{proofof}

\begin{lemma}\label{lem:nf2}
Let $\Gamma$ be a temporal theory over $\mathcal{A}$ and
let $\tuple{\H,\T}$ be a \THTf{} model of $\sigma(\Gamma)$, being ${\bm m}$ its associated three-valuation.
Then for any $\mu \in \mathit{subf}(\Gamma)$ and any $k=0..n$
we have $\trival{k}{\Lab{\mu}} = \trival{k}{\mu}$.
\end{lemma}
\begin{proofof}{Lemma~\ref{lem:nf2}}
We proceed by structural induction on $\mu$.
\begin{enumerate}
\item If $\mu$ is $\bot$ or an atom $p$, this is trivial because $\Lab\mu=\mu$ by definition.

\item If $\mu=\varphi \otimes \psi$ for any Boolean connective $\otimes \in \{\vee,\wedge,\to\}$ then:
\[\begin{array}{rll}
\trival{k}{\Lab\mu} &= \trival{k}{\Lab\varphi \otimes \Lab\psi}
& \text{using the equality in }\eta(\mu) \\
&= f^\otimes(\trival{k}{\Lab\varphi},\trival{k}{\Lab\psi}) \\
&= f^\otimes(\trival{k}{\varphi},\trival{k}{\psi})
& \text{By induction on } \varphi, \psi \\
&= \trival{k}{\varphi \otimes \psi} \\
&= \trival{k}{\mu}
\end{array}\]

\item If $\mu=\previous \varphi$ we divide into two cases:
\begin{itemize}
\item[-] If $k>0$ we can apply the first formula in $\eta(\mu)$ as follows:
\[\begin{array}{rll}
\trival{k}{\Lab\mu} &= \trival{k}{\previous \Lab\varphi}\\
&= \trival{k-1}{\Lab\varphi} & \text{(since }k>0)\\
&= \trival{k-1}{\varphi}
& \text{By induction on } \varphi \\
&= \trival{k}{\previous \varphi} \\
&= \trival{k}{\mu}
\end{array}\]
\item[-] If $k=0$ we directly use the second formula in $\eta(\mu)$ so that
    $\trival{0}{\Lab\mu}=0=\trival{0}{\previous \varphi}=\trival{0}{\mu}$.
\end{itemize}

\item If $\mu=\varphi \since \psi$ we divide into two cases:
\begin{itemize}
\item[-] If $k=0$ we directly use the second formula in $\eta(\mu)$ so that
\[\begin{array}{rll}
\trival{0}{\Lab\mu} &= \trival{0}{\Lab\psi} & \text{Using }\eta(\mu)\\
&= \trival{0}{\psi}
& \text{By induction on } \psi \\
&= \trival{0}{\varphi \since \psi}
\end{array}\]
being the last step already proved in the second case for $\since$ of Lemma~\ref{lem:nf1}.
\item[-] If $k>0$ we will proceed by a second induction on $k$.
The case $k=0$ was proved in the previous step.
So, assume it is proved up to $k-1$.
\[\begin{array}{rll}
& \trival{k}{\Lab\mu} \\
&= \trival{k}{\Lab\psi \vee (\Lab\varphi \wedge \previous \Lab\mu))}\\
&=  \max(\trival{k}{\Lab\psi}, \min(\trival{k}{\Lab\varphi},\trival{k-1}{\Lab\mu})))) \\
&=  \max(\trival{k}{\psi}, \min(\trival{k}{\varphi},\trival{k-1}{\mu}))))
& \text{Structural induction } \varphi, \psi\\
& & \text{and } k-1 \ \text{induct. on } \mu \\
&=  \trival{k}{\psi \vee \varphi \wedge \previous \mu}\\
&=  \trival{k}{\varphi \since \psi}
& \text{Temporal dual of } \eqref{f:induntil}\\
\end{array}\]
\end{itemize}

\item The case for $\mu=\varphi \trigger \psi$ is completely analogous to $\mu=\varphi \since \psi$ exchanging the roles of $\wedge/\min$ and $\vee/\max$ and using the temporal dual of \eqref{f:indrelease}.

\item If $\mu=\next \varphi$ we divide the proof into two cases:
\begin{itemize}
\item[-] If $k=n$ we use the second formula in $\eta(\mu)$ so that $\trival{n}{\Lab\mu}=0=\trival{n}{\next \varphi}=\trival{n}{\mu}$.
\item[-] If $0\leq k <n$ we can apply the first formula in $\eta(\mu)$ that guarantees $\trival{j}{\previous \Lab\mu}=\trival{j}{\Lab\varphi}$ for all $j=1..n$.
In particular, we can take $j=k+1$ and so:
\[\begin{array}{rll}
\trival{k+1}{\previous \Lab\mu} &= \trival{k+1}{\Lab\varphi}\\
&= \trival{k+1}{\varphi}
& \text{By induction on } \varphi \\
&= \trival{k}{\next \varphi} \\
&= \trival{k}{\mu}
\end{array}\]
Finally, note that $\trival{k+1}{\previous \Lab\mu}=\trival{k}{\Lab\mu}$.
\end{itemize}

\item If $\mu=\varphi \until \psi$ we have, again, two cases:
\begin{itemize}
\item[-] If $k=n$ we apply the second formula in $\eta(\mu)$ as follows
\[\begin{array}{rll}
\trival{n}{\Lab\mu} &= \trival{n}{\Lab\psi} & \text{Using }\eta(\mu)\\
&= \trival{n}{\psi}
& \text{By induction on } \psi \\
&= \trival{n}{\varphi \until \psi}
\end{array}\]
being the last step already proved in the second case for $\until$ of Lemma~\ref{lem:nf1}.


\item[-] If $0\leq k <n$ we will apply induction on $k$ backwards, from $n$ to $0$.
The case $k=n$ has been already proved.
Assume it holds for $k+1$ and we want to prove the result for $k$.
Now, the first formula in $\eta(\mu)$ that guarantees $\trival{j}{\previous \Lab\mu}=\trival{j}{\previous \Lab\psi \vee (\previous \Lab\varphi \wedge \Lab\mu)}$ for all $j=1..n$.
If we take $j=k+1$:
\[\begin{array}{rll}
\trival{k+1}{\previous \Lab\mu} &= \trival{k+1}{\previous \Lab\psi \vee (\previous \Lab\varphi \wedge \Lab\mu)}\\
&= \max(\trival{k}{\Lab\psi}),\min(\trival{k}{\Lab\varphi},\trival{k+1}{\Lab\mu}))
\end{array}\]
we can apply structural induction on subformulas $\psi$ and $\varphi$, but also integer induction on $\mu$ for case $k+1$, so we obtain that the above expression is equal to:
\[\begin{array}{rll}
&= \max(\trival{k}{\psi}),\min(\trival{k}{\varphi},\trival{k+1}{\mu})) \\
&= \trival{k}{\psi \vee \varphi \wedge \next \mu}\\
&= \trival{k}{\varphi \until \psi}
& \text{By }\eqref{f:induntil}
\end{array}\]
and, again, we have that the first line of the equations $\trival{k+1}{\previous \Lab\mu}=\trival{k}{\Lab\mu}$.
\end{itemize}

\item The case for $\mu=\varphi \release \psi$ is completely analogous to $\mu=\varphi \until \psi$ exchanging the roles of $\wedge/\min$ and $\vee/\max$ and using  \eqref{f:indrelease}.

\end{enumerate}
\end{proofof}

\begin{theorem}\label{th:thtmodels}
For any temporal theory $\Gamma$ over $\mathcal{A}$ we have that \THTf{} satisfies:
\begin{align*}
\left\lbrace  \tuple{\H,\T} \mid \tuple{\H,\T} \models \Gamma \right\rbrace =
\left\lbrace  \tuple{\H',\T'}|_\mathcal{A} \mid \tuple{\H',\T'} \models \sigma(\Gamma) \right\rbrace
\end{align*}
\end{theorem}
\begin{proofof}{th:thtmodels}
The `$\subseteq$' direction follows from Lemma~\ref{lem:nf1}.
For the `$\supseteq$' direction, take some $\tuple{\H',\T'}$ model of $\sigma(\Gamma)$ of length $n$.
This implies that its associated 3-valuation satisfies $\trivalp{0}{\Lab\varphi}=2$ for all formulas $\varphi \in \Gamma$, as those labels are included in $\sigma(\Gamma)$ as formulas.
Since $\Gamma \subseteq subf(\Gamma)$, we can apply Lemma~\ref{lem:nf2} to conclude $\trivalp{k}{\Lab\varphi}=\trivalp{k}{\varphi}$ for all $k=0..n$ but then, in particular, $2=\trivalp{0}{\Lab\varphi}=\trivalp{0}{\varphi}$ and so, $\tuple{\H',\T'}$ is also a model of $\Gamma$.
Finally, $\tuple{\H',\T'}|_\mathcal{A}$ is still a model of $\Gamma$ because the latter is a theory for vocabulary $\mathcal{A}$.
\end{proofof}

Let $\mathit{TSM}(\Gamma)$ denote the set of temporal stable models of a theory $\Gamma$.
Then:
\begin{corollary}
Let $\Gamma$ be a theory for vocabulary $\mathcal{A}$.
Then, for any theory $\Gamma'$ of the same vocabulary, \THTf{} satisfies:
\[
\mathit{TSM}(\Gamma \cup \Gamma') = \{\tuple{\H',\T'}|_\mathcal{A} \mid \tuple{\H',\T'} \in \mathit{TSM}(\sigma(\Gamma) \cup \Gamma')\}
\]
\qed
\end{corollary}

In other words, $\Gamma$ and $\sigma(\Gamma)$ are strongly equivalent, assuming we disregard the auxiliary atoms not in $\mathcal{A}$ and we use any arbitrary context $\Gamma'$ not referring to those auxiliary atoms.
%We have that \Tra{\gamma} is either of form
%\(
%\hat{\next} \alwaysF\previous \varphi \wedge \alwaysF( \finally \to \varphi)
%\)
%or
%\(
%\hat{\next} \alwaysF \varphi \wedge \alwaysF(\initially \to \varphi)
%\)
%\tbrw
%
%which means that any dynamical rule can be represented in a set of rules of the form
%\begin{align*}
%  \hat{\next} \alwaysF \alpha
%\end{align*}
%or in the form
%\begin{align*}
%  \alwaysF (\finally \to \beta),
%\end{align*}
%whereby both $\alpha$ and $\beta$ consist of literals, the unary operator $\previous$,
%the constants $\top, \bot, \initially$ and $\finally$
%and the binary operators $\wedge, \vee$ and $\to$.

%%% Local Variables:
%%% mode: latex
%%% TeX-master: "../paper"
%%% End:
