
\begin{proofof}{Theorem \ref{thm:loops}}

We first prove that if \Ttrace\ is a temporal answer set of $P$, then \Ttrace\ is a \LTLf-model of \tCF{P} and \LF{P}.
The proof for \tCF{P} is the same as for Theorem~\ref{thm:tcompletion}.
Remains to prove that \Ttrace\ is a \LTLf-model of \LF{P}.
Assume by contradiction that $\tuple{\Ttrace,\Ttrace}\not\models \LF{P}$. Two different cases must be considered:
\begin{itemize}
    \item
        there is a loop $L$ in \Graph{\dynamic{P}} such that
        $\tuple{\Ttrace,\Ttrace},i\not\models \bigvee_{a\in L} a \rightarrow \mathit{ES}_{\dynamic{P}}(L)$
        for some $\rangeo{i}{1}{\lambda}$, or
    \item
        there is a loop $L$ in \Graph{\initial{P}} such that
        $\tuple{\Ttrace,\Ttrace},0\not\models \bigvee_{a\in L} a \rightarrow \mathit{ES}_{\initial{P}}(L)$.
\end{itemize}

For the first case, let \Htrace\ be a trace of length $\lambda$ such that $H_i=T_i\setminus L$ and
$H_k=T_k$ otherwise.
We show that $\tuple{\Htrace,\Ttrace},0\models P$, which will contradict the hypothesis \Ttrace\ is a \TELf-model of $P$:
\begin{enumerate}
    \item
        $\tuple{\Htrace,\Ttrace},0\models \initial{P}$: follows from  $\tuple{\Ttrace,\Ttrace}\models \initial{P}$
        by Lemma~\ref{lem:pastocc} as $T_0\setminus H_0 = \emptyset$.
    \item
        $\tuple{\Htrace,\Ttrace},0\models \final{P}$: follows from  $\tuple{\Ttrace,\Ttrace}\models \final{P}$ as rules in \final{P} are
        constraints.
   \item
        $\tuple{\Htrace,\Ttrace},0\models \dynamic{P}$: $\tuple{\Ttrace,\Ttrace},0\models \dynamic{P}$ since $\Ttrace$ is a \TELf{}-model of $P$.
        Therefore, $\tuple{\Ttrace,\Ttrace},k \models r$ for all $\rangeo{k}{1}{\lambda}$ and for all $r\in\dynamic{P}$.
        Then, $\tuple{\Ttrace,\Ttrace},k \not\models \body{r}$ or $\tuple{\Ttrace,\Ttrace},k \models \Head{r}$.
        If $\tuple{\Ttrace,\Ttrace},k \not\models \body{r}$, by persistence, $\tuple{\Htrace,\Ttrace},k \not\models \body{r}$ and
        $\tuple{\Htrace,\Ttrace},k \models r$.
        If $\tuple{\Ttrace,\Ttrace},k \models \body{r}$, then $\tuple{\Ttrace,\Ttrace},k \models \Head{r}$.

        In the case $k\neq i$, $H_k=T_k$ and $\tuple{\Ttrace,\Ttrace}, k \models \Head{r}$ imply  $\tuple{\Htrace,\Ttrace},k \models \Head{r}$.% follows
        %from $\tuple{\Ttrace,\Ttrace},k \models \Head{r}$,
        %and then $\tuple{\Ttrace,\Ttrace},k \models r$.
%
        In the case $k=i$, we have two cases.
        \begin{itemize}
            \item
                if $\tuple{\Ttrace,\Ttrace},i\not\models\support{L}{\body{r}}$, then,
                as $(T_i\setminus H_i)\setminus L=\emptyset$ and $T_k\setminus H_k=\emptyset$ for $k<i$,
                by Lemma~\ref{lem:support},
                $\tuple{\Htrace,\Ttrace},i \not\models \body{r}$. So $\tuple{\Htrace,\Ttrace},i \models r$.
            \item if $\tuple{\Ttrace,\Ttrace},i\models\support{L}{\body{r}}$ and $\Head{r}\cap L =\emptyset$ then 
            $\tuple{\Htrace,\Ttrace},i \models \Head{r}$ follows from $\tuple{\Ttrace,\Ttrace},i \models \Head{r}$ and $\tuple{\Htrace,\Ttrace},i \models r$.
            \item if $\tuple{\Ttrace,\Ttrace},i\models\support{L}{\body{r}}$ and $\Head{r}\cap L =\emptyset$ then
                        \begin{itemize}
                            \item if there is some $p\in\Head{r}\setminus L$ such that $p\in T_i$, then $p\in H_i$. So
                                $\tuple{\Htrace,\Ttrace},i \models \Head{r}$ and then $\tuple{\Htrace,\Ttrace},i \models r$.
                            \item if there is no $p\in\Head{r}\setminus L$ such that $p\in T_i$, then
                                $\tuple{\Ttrace,\Ttrace},i\models \bigwedge_{p\in \Head{r}\setminus L} \neg p$.
                                As we also have $\tuple{\Ttrace,\Ttrace},i\models\support{L}{\body{r}}$,
                                $\tuple{\Ttrace,\Ttrace},i\models \bigvee_{a\in L} a \rightarrow \mathit{ES}_{\dynamic{P}}(L)$,
                                which contradict our hypothesis.
                        \end{itemize}
        \end{itemize}
\end{enumerate}

The proof of the second case follows a similar reasoning as for the first one.

%
%%-------------------
Next, we prove that if \Ttrace\ is a \LTLf-model of \tCF{P} and \LF{P}, then \Ttrace\ is a \TELf-model of $P$.
The proof for $\tuple{\Ttrace,\Ttrace}\models P$ is the same as for Theorem~\ref{thm:tcompletion}.
Remains to prove that there is no $\Htrace<\Ttrace$ such that $\tuple{\Htrace,\Ttrace}\models P$.

Let assume that there exists such a trace $\Htrace$, and let $i$ be the smallest time point such that $H_i\subset T_i$.
Therefore, $H_k = T_k$ for all $\rangeco{k}{0}{i}$.

\begin{itemize}
	\item If $i>0$:
	Let $a\in T_i\setminus H_i$. $\tuple{\Ttrace,\Ttrace}\models \tCF{P}$, so $\tuple{\Ttrace,\Ttrace},i \models a \leftrightarrow
	\bigvee_{r\in\dynamic{P}, a\in\Head{r}}(\body{r}\wedge\bigwedge_{p\in \Head{r}\setminus \{a\}} \neg p)$.
	As $a\in T_i$, there is some rule $r\in\dynamic{P}$ such that $a\in\Head{r}$,
	$\tuple{\Ttrace,\Ttrace},i \models \body{r}$, and $\tuple{\Ttrace,\Ttrace},i \models \bigwedge_{p\in \Head{r}\setminus \{a\}} \neg p$.
	$\tuple{\Htrace,\Ttrace}\models P$, so $\tuple{\Htrace,\Ttrace},i \models \body{r} \rightarrow a \vee\bigvee_{p\in\Head{r}\setminus\{a\}} p$.
	Then, $\tuple{\Htrace,\Ttrace},i \not\models \body{r}$ or $\tuple{\Htrace,\Ttrace},i \models a$ or
	$\tuple{\Htrace,\Ttrace},i \models \bigvee_{p\in\Head{r}\setminus\{a\}} p$.
	As $a\not\in H_i$, $\tuple{\Htrace,\Ttrace},i \not\models a$.
	As $\tuple{\Ttrace,\Ttrace},i \models \bigwedge_{p\in \Head{r}\setminus \{a\}} \neg p$,
	$\tuple{\Htrace,\Ttrace},i \not\models \bigvee_{p\in\Head{r}\setminus\{a\}} p$.
	So, $\tuple{\Htrace,\Ttrace},i \not\models \body{r}$.

$\tuple{\Ttrace,\Ttrace},i \models \body{r}$, $\tuple{\Htrace,\Ttrace},i \not\models \body{r}$ and $T_j\setminus H_j = \emptyset$ for $j<i$, so,
by Lemma~\ref{lem:pastocc}, there must be some $b\in T_i\setminus H_i$ with a present and positive occurence in $\body{r}$ that is not in the scope of negation.
Therefore, for any $a\in T_i\setminus H_i$, there is some $b\in T_i\setminus H_i$ such that $(a,b)\in G(\dynamic{P})$.
It implies a loop $L$ in $\dynamic{P}$, with $L\subseteq T_i\setminus H_i$.

The strongly connected components (SCC) of the dependency graph of $\dynamic{P}$ over $T_i\setminus H_i$ form
a directed acyclic graph, so there is some SCC $L$, such that,
for any $a\in L$, there is no $b\in (T_i\setminus H_i)\setminus L$ such that $(a,b)\in \Graph{\dynamic{P}}$.

For any $a\in T_i\setminus H_i$, $\tuple{\Htrace,\Ttrace},i\not\models \body{r}$, for all $r \in \dynamic{P}$
such that $a\in\Head{r}$ and $\tuple{\Ttrace,\Ttrace},i \models \bigwedge_{p\in \Head{r}\setminus \{a\}} \neg p$.
So $\tuple{\Htrace,\Ttrace},i\not\models \body{r}$, for all $r \in \dynamic{P}$ such that $L\cap\Head{r}\neq \emptyset$
and $\tuple{\Ttrace,\Ttrace},i \models \bigwedge_{p\in \Head{r}\setminus L} \neg p$.
Let \X\ be a trace of length $\lambda$ with $X_i=L$ and $X_j=\emptyset$ for $ j\neq i$.
For any $a\in L$ there is no $b\in (T_i\setminus H_i)\setminus L$ such that $(a,b)\in \Graph{\dynamic{P}}$,
so all positive and present occurences of atoms from $L$ in $\body{r}$ are in the scope of negation.
Then, we can apply Lemma~\ref{lem:pastocc}, and get that
$\tuple{\Ttrace\setminus \X,\Ttrace},i\not\models \body{r}$, for all $r \in \dynamic{P}$ such that $L\cap\Head{r}\neq \emptyset$
and $\tuple{\Ttrace,\Ttrace},i \models \bigwedge_{p\in \Head{r}\setminus L} \neg p$.
Then, as $X_i\setminus L = \emptyset$, by Lemma~\ref{lem:support},
$\tuple{\Ttrace,\Ttrace},i\not\models \support{L}{\body{r}}$, for all $r \in \dynamic{P}$ such that $L\cap\Head{r}\neq \emptyset$
and $\tuple{\Ttrace,\Ttrace},i \models \bigwedge_{p\in \Head{r}\setminus L} \neg p$.
So, $\tuple{\Ttrace,\Ttrace},i\not\models \bigvee_{a\in L} a \rightarrow \mathit{ES}_{\dynamic{P}}(L)$,
and then $\tuple{\Ttrace,\Ttrace}\not\models \LF{P}$. Contradiction.\\

\item Case $i=0$: we reach a contradiction in a similar way as above.
	
\end{itemize}
\end{proofof}
