\section{Introduction}\label{sec:introduction}

Reasoning about dynamic scenarios is a central problem in the areas of  Knowledge Representation~\cite{bale04} (KR) and Artificial Intelligence (AI).
Several formal approaches and systems have emerged to introduce non-monotonic reasoning features in scenarios where the formalisation of time is fundamental~\cite{BaralZ07,BaralZ08,emerson90a,Gonzalez2002,sandewall94a}.
In \emph{Answer Set Programming}~\cite{breitr11a} (ASP), former approaches to temporal reasoning use first-order encodings~\cite{lifschitz99b} where the time is represented by means of a variable whose value comes from a finite domain.
The main advantage of those approaches is that the computation of answer sets can be achieved via incremental solving~\cite{gekakaosscth08a}.
Their downside is that they require an explicit representation of time points.

\emph{Temporal Equilibrium Logic}~\cite{AguadoCDPSSV23} (\TEL{}) was proposed as a temporal extension of \emph{Equilibrium Logic}~\cite{pearce96a}
with connectives from \emph{Linear Time Temporal Logic}~\cite{pnueli77a} (\LTL{}).
Due to the computational complexity of its satisfiability problem (\textsc{ExpSpace}), finding tractable fragments of \TEL{} with good computational properties have also been a topic in the literature.
Within this context, \emph{splittable temporal logic programs}~\cite{agcapevi11a} have been proved to be a syntactic fragment of \TEL{} that allows for a reduction to \LTL{} via the use of Loop Formulas~\cite{feleli06a}.

When considering incremental solving, logics on finite traces such as \LTLf{}~\cite{giavar13a} have been shown to be more suitable.
Accordingly, \emph{Temporal Equilibrium Logic on Finite traces} (\TELf)~\cite{cakascsc18a} was created and became the foundations of the temporal ASP solver \telingo{}~\cite{cakamosc19a}.




We present a new syntactic fragment of \TELf{}, named \emph{past-present} temporal logic programs.
Inspired by Gabbay's seminal paper~\cite{gabbay87a}, where the declarative character of past temporal operators is emphasized, this language consists of a set of logic programming rules whose formulas in the head are disjunctions of atoms that reference the present, while in its body we allow for any arbitrary temporal formula without the use of future operators.
Such restriction ensures that the past remains independent of the future, which is the case in most dynamic domains, and makes this fragment advantageous for incremental solving.
%The use of only past operators in the body has the advantage that, when using incremental solving, the body of each rule can be seen as a query whose satisfiability can be checked on the (partial) incremental computation at each step.
%This advantage allows us to exploit the advantages of \telingo's API in order to reuse partial computations during the solving in order increase its performance.

As a contribution, we study the Lin-Zhao theorem~\cite{linjzh03a} within the context of past-present temporal logic programs.
More precisely, we show that when the program is \emph{tight}~\cite{erdlif03a}, extending Clark's completion~\cite{clark78a,fages94a} to the temporal case suffices to capture the answer sets of a finite past-present program as the \LTLf{}-models of a corresponding temporal formula.
We also show that, when the program is not tight, the use of loop formulas is necessary. To this purpose, we  extend the definition of loop formulas to the case of past-present programs and we prove the Lin-Zhao theorem in our setting.
%Finally, we also prove the generalisation of Lin-Zhao theorem in the sense of~\cite{feleli06a}, where the computation of the completion is be replaced by the consideration of unitary loops.

The paper is organised as follows: in Section~\ref{sec:background}, we review the formal background and we introduce the concept of past-present temporal programs.
In Section~\ref{sec:tcompletion}, we extend the completion property to the temporal case.
Section~\ref{sec:lf} is devoted to the introduction of our temporal extension of loop formulas.
In section~\ref{sec:unitary-cycles}, we shows that temporal completion can be captured in the general theory
of loop formulas by considering unitary cycles.
Finally, in Section~\ref{sec:conclusions}, we present the conclusions as well as some future research lines.

%After establishing the formal background
%The paper is organised as follows: in Section~\ref{sec:background} we introduce the logic of temporal here and there over finite traces and its equilibrium counterpart.
%In Section~\ref{sec:ppp} we introduce the concept of past-present temporal programs while, in Section~\ref{sec:tcompletion} we extend the completion property to the temporal case and we prove that, for the case of tight programs, computing the temporal completion suffices to characterise the temporal answer sets of a past-present programs in terms of \LTLf{} formulas.
%The non-tight case is studied in Section~\ref{sec:lf}, where we introduce our temporal extension of loop formulas and extend the results presented in Section~\ref{sec:tcompletion} to the case of non-tight programs.
%Our last contribution, presented in Section~\ref{sec:unitary-cycles}, shows that temporal completion can be captured in the general theory of loop formulas by considering unitary cycles.
%Finally, in Section~\ref{sec:conclusions} we present the conclusions of the paper an we outline some future research lines.
