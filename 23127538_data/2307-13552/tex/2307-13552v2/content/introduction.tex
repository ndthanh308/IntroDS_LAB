\section{Introduction}

The Rubik's Cube is a 3D puzzle game that has been widely popular since its invention in 1974. It has been a subject of interest for researchers in Artificial Intelligence (AI) due to its computational complexity and potential for developing efficient problem-solving algorithms. RC has motivated researchers to explore alternative representations that simplify the problem while preserving its complexity. Efficient algorithms have been developed to solve RC in the least number of moves, and they have been used in various applications, including robot manipulation, game theory, and machine learning. Therefore, in this paper, we aim to explore the different representations and algorithms to solve RC and evaluate their performance and effectiveness in solving this challenging puzzle.


% As artificial intelligence (AI) continues to solve problems that humans struggle to solve, there is an emerging need for humans to understand these solutions so that we can trust AI, create new educational opportunities, and even discover new knowledge. Many of these problems are present in planning where the objective is to find a sequence of actions to go from a given initial state to the goal state. AI has been successfully applied to solve the Rubik's Cube (RC) \cite{deepcube,aaai_allure_demo,joyner2008adventures,agostinelli2021designing}, but these methods used opaque learning techniques, which are hard for human RC learners to reap the benefits. Additionally, solving RC is considered a challenging puzzle \cite{van2010quest}; thus, we wanted to see the capabilities of automated planning approaches in this domain.

% \bs{Have a table showing planners, representations and heuristics. Also mention datasets there. You can introduce notations like m1, m2; 1, d2 there.}

Various solution approaches  have been proposed RC including Reinforcement Learning (RL) and search.
% Various formalisms have been proposed to represent RC, including Reinforcement Learning (RL) and SAS+. 
For instance, DeepCubeA \cite{agostinelli2019solving} uses RL to learn policies for solving RC, where the cube state is represented by an array of numerical features. Although DeepCubeA is a domain-independent puzzle solver, it employs a custom representation for RC. On the other hand, \citet{buchner2022comparison} utilized SAS+ representation to model the RC problem in a finite domain representation, which enables standard general-purpose solvers like Scorpion to be used on the RC problem. Despite the success of these approaches, no prior work has explored the use of Planning Domain Definition Language (PDDL) to encode a 3x3x3 RC problem. While a previous study\footnote{https://wu-kan.cn/2019/11/21/Planning-and-Uncertainty/} has encoded a 2x2x2 RC problem using PDDL and solved it with a Fast-Forward planner, there exists no PDDL encoding for a 3x3x3 RC problem.


\begin{table}[!t]
\centering
\begin{tabular}{|l|l|}
\hline
\textbf{Notations} & \textbf{Description} \\ \hline
RC & Rubik's Cube \\ \hline 
 \hline
PDDL &  Planning Domain Description Language \\ 
 & \cite{pddl2.1}\\ \hline
SAS+ &  \begin{tabular}[c]{@{}l@{}}State-Action-Space+ \\ \cite{fikes1971strips} \end{tabular}\\ \hline
Custom & \begin{tabular}[c]{@{}l@{}}RC representation in DeepCubeA \\ \cite{agostinelli2019solving} \end{tabular}\\ \hline \hline
Blind & FastDownward with Blind\textsuperscript{3} \\ \hline
% Max & FastDownward with Max \\ \hline
GC & FastDownward with Goal count\textsuperscript{3} \\ \hline
CG & FastDownward with Causal Graph\textsuperscript{3} \\ \hline
CEA & \begin{tabular}[c]{@{}l@{}}FastDownward with \\ Context-enhanced Additive\textsuperscript{3}\end{tabular} \\ \hline
LM-Cost & \begin{tabular}[c]{@{}l@{}}FastDownward with \\ LM-Cost Partitioning\textsuperscript{3}\end{tabular} \\ \hline
FF & FastDownward with FF\textsuperscript{3}\\ \hline \hline
M\&S & Scorpion with Merge \& Shrink\textsuperscript{4}\\ \hline
PDB-Man & Scorpion with Max Manual PDB\textsuperscript{4}\\ \hline
PDB-Sys & Scorpion with Max Systematic PDB\textsuperscript{4} \\ \hline \hline
d1 & \begin{tabular}[c]{@{}l@{}}Dataset of 200 problems generated \\ considering 12 RC actions\end{tabular} \\ \hline
d2 & \begin{tabular}[c]{@{}l@{}}Dataset of 200 problems generated \\ considering 18 RC actions\end{tabular} \\ \hline \hline
m1 & PDDL model with 12 RC actions \\ \hline
m2 & PDDL model with 18 RC actions \\ \hline
\end{tabular}
\caption{Notations or abbreviations and their descriptions.}
\label{tab:notations}
\end{table}

In this paper, we introduce a novel approach for representing RC in PDDL. We encode the initial state and goal state using a set of predicates, each of which specifies the color of a sticker on a particular cube piece or edge piece. We then define the actions that can be taken to manipulate the cube pieces and edges. Our PDDL representation enables us to model RC as a classical planning problem, which can be solved using off-the-shelf planning tools. To the best of our knowledge, this is the first attempt to represent RC formally using PDDL. We also evaluate the effectiveness of our approach by comparing it with other state-of-the-art representations in terms of the efficiency and effectiveness of problem-solving.
% While no previous PDDL encoding of a 3x3x3 RC problem is known to the authors, there is previous work\footnote{https://wu-kan.cn/2019/11/21/Planning-and-Uncertainty/} for a 2x2x2 RC setting and is solved with the Fast-Forward planner. Authors in \cite{buchner2022comparison} modeled the RC problem in finite domain representation, which enables the standard general purpose solvers to be used on the RC problem. In this work, we compare our PDDL representation for RC with the SAS+ representation by \citet{buchner2022comparison} and provide a quantitative analysis comparing these two representations.
Our major contributions are:
\begin{itemize}

    \item We develop the first PDDL formulation for the 3x3x3 Rubik's Cube, which is a novel and significant contribution to the existing literature. This PDDL formulation will enable the use of standard PDDL planners for solving Rubik's Cube problems, which was not previously possible.
     \item We bridge across hither-to incomparable RC solving approaches, compare their performance  and draw insights from results to facilitate new research.
    \item We perform a comparative analysis of two formal languages, SAS+ and PDDL, and custom one in DeepCubeA, a RL approach for solving RC on a set of common benchmark RC problems. This comparative analysis is important as it provides insights into the strengths and weaknesses of these different approaches, and helps to identify which method may be most appropriate for a given problem setting.
    % \item \textbf{(rewrite)} We evaluate the performance of different formal languages and methods in solving Rubik's Cube problems. This evaluations provides valuable insights into the trade-offs between representational choice and plan optimality that can help researchers explore and design future strategies for challenging domains.
\end{itemize}

The paper is organized as follows: we begin with giving an overview of Rubik's Cube solving ecosystem, including the RC problem, domain-independent planners and heuristics, and learning-based RC solvers. Then, we present a comparison of three different representations for RC: DeepCubeA, SAS+, and PDDL. Next, we outline the experiments conducted, including the heuristics considered and the experimental setup, followed by  results. We compare RC solvers and heuristics for the number of problems solved and plan optimality. Finally, the paper concludes with a discussion of the findings and their implications for future research in solving larger RC problems.

% We also compare the performance results from a domain-dependent solver \textit{DeepCubeA (citation)}. 

% Domain-Independent Solver with custom representation solves all 
% PDDL representation 
% SAS+ PDB with korf's patterns

% Considering Abstract heuristics which aim at providing an optimal solution show a lesser performance than a Non-Optimal planner. In our case, FF heuristic, which does not guarantee on the optimality of the solution performs much better than the Abstract heuristics

% This work can help (1) planning researchers improve their algorithms, (2) enable human RC learners to use off-the-shelf planners to find custom and optimized ways to solve any given RC configuration, and (3) make new solving opportunities possible like PDDL planners being used as a labeled data generator for use during training by learning based RC solvers, the latter has been shown to scale to more significant instances \cite{deepcube}. Our key insights from this work using planners in Table~\ref{tab:rc-planners} are:
% %\begin{itemize}
%     %\item 
%     PDDL-based encoding helps us generate competitive plans with an off-the-shelf planner, and 
%     %\item 
%     the characteristic of the planner, its heuristic strategy, and problem encoding have an impact on the performance (Table~\ref{tab:exp-results}).
% %\end{itemize}

% Moreover, it can be used as a
% labeled data generator to help training of learning based RC
% solvers which have been shown to scale to large RC problems.
% A demonstration can be seen at \cite{rc-demo}.  %\href{https://youtu.be/tp9Z0yppSJw} {here}.

% In the remainder of the paper, we start by defining the components and their formalisms employed in the RC Solver Ecosystem. We then present our PDDL model for RC followed by an illustration of its operation. Finally, we present the comparative analysis of the two RC representations and conclude with a discussion of the obtained results.

%(a) introducing the first PDDL formulation for a 3-size RC, and (b) providing an empirical insight into the role of representation and planners along with their heuristics on performance. 
