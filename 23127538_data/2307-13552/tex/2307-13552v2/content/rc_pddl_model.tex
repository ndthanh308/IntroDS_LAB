% \vspace{-0.3cm}
\section{Comparision of RC representations}

In this section, we describe and provide a comparative analysis of different RC representations comprising of RL and Planning formal languages.
% \vspace{-0.25cm}
\subsection{DeepCubeA}
The DeepCubeA algorithm adopts a unidimensional array as a representation of the Rubik's Cube (RC) state. Specifically, this array encompasses 54 elements, each of which corresponds to a unique sticker color present on a cube piece of the RC. While this array-based modeling offers computational advantages, it is limited by its inability to fully encapsulate the spatial orientation of Rubik's Cube. Furthermore, the usage of a hard-coded representation and implicit assumptions concerning the position of cubelets poses a challenge to novice users seeking to comprehend the array-based representation.
% \vspace{-0.25cm}
\subsection{SAS+}
In \citealt{buchner2022comparison}, Rubik's Cube is modeled with 18 actions in SAS+ representation as a factored effect task, with each face labeled as F, B, L, R, U, or D. The orientation of each cube piece is represented as a triple of values, and for corner cube pieces, the orientation is a permutation of \{1, 2, 3\}, while for edge cube pieces, it is a permutation of \{1, 2, \#\} (where \# represents a blank symbol). The rotation of the cube in 3D space is captured as the permutation of the respective triple for each cube piece. The SAS+ model has 20 variables, 480 fact pairs and bytes required for representing each state is 16 bytes.

% Figure environment removed

% While sticking to planning terminology of actions, problems and plans, we want to clarify terminology prevalent in RC literature that we will also refer to. A sequence of actions are called {\em macro-actions} and a collection of macro-actions are called {\em algorithms} in RC parlance. A solver may employ a strategy for sequencing macro-actions to solve the cube. 

\begin{lstlisting}[
  float=t,
  caption={Action L of Rubik's Cube modeled in PDDL},
  label={lst:pddl},
  language=PDDL, numbers=none,
  frame=lines,
  xleftmargin=\parindent, xrightmargin=\parindent
  ]
(:action L
:effect (and
;for corner cubelets
(forall(?x ?y ?z)(when (cube1 ?x ?y ?z) 
  (and (cube2 ?y ?x ?z))))
(forall(?x ?y ?z)(when (cube3 ?x ?y ?z) 
  (and (cube1 ?y ?x ?z))))
(forall(?x ?y ?z)(when (cube4 ?x ?y ?z) 
  (and (cube3 ?y ?x ?z))))
(forall(?x ?y ?z)(when (cube2 ?x ?y ?z) 
  (and (cube4 ?y ?x ?z))))
;for edge cubelets
(forall(?x ?z)(when (edge13 ?x ?z) 
  (and (edge12 ?x ?z))))
(forall(?y ?z)(when (edge34 ?y ?z) 
  (and (edge13 ?y ?z))))
(forall(?x ?z)(when (edge24 ?x ?z) 
  (and (edge34 ?x ?z))))
(forall(?y ?z)(when (edge12 ?y ?z) 
  (and (edge24 ?y ?z))))))
\end{lstlisting}
% \vspace{-0.2cm}
\subsection{PDDL}

In the PDDL domain, the Rubik's cube problem environment has been defined by assuming the cube pieces are in a fixed position and are named accordingly, as defined in Figure \ref{FIG: rubiks-cube}. These fixed cube pieces are modeled as predicates in the RC domain and the colors they possess in the three-dimensional space as parameters of these predicates. With the help of conditional effects, each action in the RC environment is defined as the change of colors on these fixed cube pieces. The 3D axis of the cube is considered as three separate parameters {\em X}, {\em Y}, and {\em Z} that specify the position of the colors on the cube's pieces. One of these axes can be connected to each face of the cube. According to the representation shown in Figure \ref{FIG: rubiks-cube}, the respective faces on each axis are: $F_{X} = \langle U,D \rangle$; $F_{Y} = \langle R,L \rangle$; $F_{Z} = \langle F,B \rangle$. These different faces of the cube can be identified by the color of the middle cube piece. We considered White, Red, and Green colors as the colors on the front(F), up(U) and right(R) faces respectively (similarly, the counter colors on the counter faces).

The following conventions regarding the RC cube pieces are considered to model the RC domain actions in the PDDL:
\begin{enumerate}
  \item The corner cube pieces of the RC are modeled as a three-color cubelet and are specified as a predicate with three parameters: {\em x}, {\em y}, and {\em z}, which indicate the piece's colors on three separate axes. There are 8 corner pieces in RC.
  \item The edge cube pieces, which are in between corner cube pieces, are modeled as two-color cubelet and is specified as a predicate with two parameters denoting the piece's colors on the two axes. There are 12 edge pieces in RC.
  \item We do not consider the rotations performed on the middle layer, as this can be resolved into rotation of right and left faces in the opposite direction. As a result, the middle cube piece of a face is unaltered.
\end{enumerate}

The predicate names define the fixed position of the cubelets that are defined with respect to the different faces of the cube. The representation considered for the cube positions is shown in Figure \ref{FIG: rubiks-cube}. One of the actions, action `L', of RC designed in PDDL from the description provided is shown in Listing \ref{lst:pddl}. In this, we refer to corner cube pieces as \textit{cubeP} and edge cube pieces as \textit{edgePQ} where \textit{P} and \textit{Q} are the numbers for the cube pieces as stated in Figure \ref{FIG: rubiks-cube}.
When the move L is applied to the RC, for example, the left face is rotated clockwise. This may be regarded as a 90-degree clockwise translation of colors from the left-face corner and edge cube pieces. Considering the RC representation shown in Figure \ref{FIG: rubiks-cube}, the colors on the pieces: cube1, cube2, cube4, and cube3, are circularly shifted towards the right. The same applies to the edge pieces. As the left face falls in the Z-plane, only the  X-axis and Y-axis colors on the cube pieces are affected. 

During execution of a problem, FastDownward first translates the domain into a SAS model. The resulting SAS version of the RC PDDL domain model has 480 variables and 960 fact pairs, with each state requiring 60 bytes for representation. 

% \begin{lstlisting}[
%   float=!b,
%   caption={Action L of Rubik's Cube modeled in PDDL},
%   label={lst:pddl},
%   language=PDDL, numbers=none,
% %   frame=single
%   ]
% (:action L
% :effect (and
% ;for corner cubelets
% (forall(?x ?y ?z)(when (cube1 ?x ?y ?z) 
%   (and (not (cube1 ?x ?y ?z)) (cube2 ?y ?x ?z))))
% (forall(?x ?y ?z)(when (cube3 ?x ?y ?z) 
%   (and (not (cube3 ?x ?y ?z)) (cube1 ?y ?x ?z))))
% (forall(?x ?y ?z)(when (cube4 ?x ?y ?z) 
%   (and (not (cube4 ?x ?y ?z)) (cube3 ?y ?x ?z))))
% (forall(?x ?y ?z)(when (cube2 ?x ?y ?z) 
%   (and (not (cube2 ?x ?y ?z)) (cube4 ?y ?x ?z))))
% ;for edge cubelets
% (forall(?x ?z)(when (edge13 ?x ?z) 
%   (and (not(edge13 ?x ?z))(edge12 ?x ?z))))
% (forall(?y ?z)(when (edge34 ?y ?z) 
%   (and (not(edge34 ?y ?z))(edge13 ?y ?z))))
% (forall(?x ?z)(when (edge24 ?x ?z) 
%   (and (not(edge24 ?x ?z))(edge34 ?x ?z))))
% (forall(?y ?z)(when (edge12 ?y ?z) 
%   (and (not(edge12 ?y ?z))(edge24 ?y ?z))))))
% \end{lstlisting}