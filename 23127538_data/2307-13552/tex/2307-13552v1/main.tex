%File: anonymous-submission-latex-2023.tex
\documentclass[]{article} % DO NOT CHANGE THIS
\usepackage[]{aaai23}  % DO NOT CHANGE THIS
\usepackage{times}  % DO NOT CHANGE THIS
\usepackage{helvet}  % DO NOT CHANGE THIS
\usepackage{courier}  % DO NOT CHANGE THIS
\usepackage[hyphens]{url}  % DO NOT CHANGE THIS
\usepackage{graphicx} % DO NOT CHANGE THIS
\urlstyle{rm} % DO NOT CHANGE THIS
\def\UrlFont{\rm}  % DO NOT CHANGE THIS
\usepackage{natbib}  % DO NOT CHANGE THIS AND DO NOT ADD ANY OPTIONS TO IT
\usepackage{caption} % DO NOT CHANGE THIS AND DO NOT ADD ANY OPTIONS TO IT
\usepackage[algo2e]{algorithm2e}
\frenchspacing  % DO NOT CHANGE THIS
\setlength{\pdfpagewidth}{8.5in} % DO NOT CHANGE THIS
\setlength{\pdfpageheight}{11in} % DO NOT CHANGE THIS
%
% These are recommended to typeset algorithms but not required. See the subsubsection on algorithms. Remove them if you don't have algorithms in your paper.
\usepackage{subcaption}
\usepackage{algorithm}
% \usepackage{algorithmic}
% \usepackage{algpseudocode}
\usepackage{color}
\newcommand{\bs}[1]{{\color{red}~{\em Comment by Biplav: #1}}}
%
% These are are recommended to typeset listings but not required. See the subsubsection on listing. Remove this block if you don't have listings in your paper.
\usepackage{newfloat}
\usepackage{listings}
% \input{content/algotexs/HDDL}
% \input{content/algotexs/JSHOP}
\input{content/algotexs/PDDL}
\DeclareCaptionStyle{ruled}{labelfont=normalfont,labelsep=colon,strut=off} % DO NOT CHANGE THIS
\lstset{%
	basicstyle={\footnotesize\ttfamily},% footnotesize acceptable for monospace
	numbers=left,numberstyle=\footnotesize,xleftmargin=2em,% show line numbers, remove this entire line if you don't want the numbers.
	aboveskip=0pt,belowskip=0pt,%
	showstringspaces=false,tabsize=2,breaklines=true}
\floatstyle{ruled}
\newfloat{listing}{tb}{lst}{}
\floatname{listing}{Listing}
%
% Keep the \pdfinfo as shown here. There's no need
% for you to add the /Title and /Author tags.
\pdfinfo{
/TemplateVersion (2023.1)
}

% DISALLOWED PACKAGES
% \usepackage{authblk} -- This package is specifically forbidden
% \usepackage{balance} -- This package is specifically forbidden
% \usepackage{color (if used in text)
% \usepackage{CJK} -- This package is specifically forbidden
% \usepackage{float} -- This package is specifically forbidden
% \usepackage{flushend} -- This package is specifically forbidden
% \usepackage{fontenc} -- This package is specifically forbidden
% \usepackage{fullpage} -- This package is specifically forbidden
% \usepackage{geometry} -- This package is specifically forbidden
% \usepackage{grffile} -- This package is specifically forbidden
% \usepackage{hyperref} -- This package is specifically forbidden
% \usepackage{navigator} -- This package is specifically forbidden
% (or any other package that embeds links such as navigator or hyperref)
% \indentfirst} -- This package is specifically forbidden
% \layout} -- This package is specifically forbidden
% \multicol} -- This package is specifically forbidden
% \nameref} -- This package is specifically forbidden
% \usepackage{savetrees} -- This package is specifically forbidden
% \usepackage{setspace} -- This package is specifically forbidden
% \usepackage{stfloats} -- This package is specifically forbidden
% \usepackage{tabu} -- This package is specifically forbidden
% \usepackage{titlesec} -- This package is specifically forbidden
% \usepackage{tocbibind} -- This package is specifically forbidden
% \usepackage{ulem} -- This package is specifically forbidden
% \usepackage{wrapfig} -- This package is specifically forbidden
% DISALLOWED COMMANDS
% \nocopyright -- Your paper will not be published if you use this command
% \addtolength -- This command may not be used
% \balance -- This command may not be used
% \baselinestretch -- Your paper will not be published if you use this command
% \clearpage -- No page breaks of any kind may be used for the final version of your paper
% \columnsep -- This command may not be used
% \newpage -- No page breaks of any kind may be used for the final version of your paper
% \pagebreak -- No page breaks of any kind may be used for the final version of your paperr
% \pagestyle -- This command may not be used
% \tiny -- This is not an acceptable font size.
% \vspace{- -- No negative value may be used in proximity of a caption, figure, table, section, subsection, subsubsection, or reference
% \vskip{- -- No negative value may be used to alter spacing above or below a caption, figure, table, section, subsection, subsubsection, or reference

\setcounter{secnumdepth}{0} %May be changed to 1 or 2 if section numbers are desired.

% The file aaai23.sty is the style file for AAAI Press
% proceedings, working notes, and technical reports.
%

% Title
% \usepackage{color,soul}
\usepackage{multirow}
% \newcommand{\bm}[1]{\textcolor{red}{[#1 \textsc{--Bharath}]}}
% \newcommand{\bips}[1]{\textcolor{blue}{[#1 \textsc{--Biplav}]}}
% \newcommand{\vp}[1]{\textcolor{green}{[#1 \textsc{--Vishal}]}}


% Your title must be in mixed case, not sentence case.
% That means all verbs (including short verbs like be, is, using,and go),
% nouns, adverbs, adjectives should be capitalized, including both words in hyphenated terms, while
% articles, conjunctions, and prepositions are lower case unless they
% directly follow a colon or long dash
\title{On Solving the Rubik's Cube with Domain-Independent Planners \\ Using Standard Representations}
\author {
    % Authors
    Bharath Muppasani,
    Vishal Pallagani,
    Biplav Srivastava,
    Forest Agostinelli
}
\affiliations {
    % Affiliations
    % \textsuperscript{\rm 1} 
    AI Institute, University of South Carolina, Columbia, South Carolina, USA\\
    \{bharath@email., vishalp@email., 
    biplav.s@, foresta@cse.\}sc.edu
}

%Example, Single Author, ->> remove \iffalse,\fi and place them surrounding AAAI title to use it
\iffalse

\author {
    Author Name
}
\affiliations{
    Affiliation\\
    Affiliation Line 2\\
    name@example.com
}
\fi

\iffalse
%Example, Multiple Authors, ->> remove \iffalse,\fi and place them surrounding AAAI title to use it
%\title{On Solving the Rubik's Cube with a  Planner}
\title{On Solving the Rubik's Cube with Domain-Independent Planners Using a Standard Representation}
\author {
    % Authors
    Bharath Muppasani,
    Vishal Pallagani,
    Biplav Srivastava,
    Forest Agostinelli
}
\affiliations {
    % Affiliations
    % \textsuperscript{\rm 1} 
    AI Institute, University of South Carolina, Columbia, South Carolina, USA\\
    \{bharath@email., vishalp@mailbox., biplav.s@, foresta@cse.\}sc.edu
}
\fi


% REMOVE THIS: bibentry
% This is only needed to show inline citations in the guidelines document. You should not need it and can safely delete it.
\usepackage{bibentry}
% END REMOVE bibentry

 \begin{document}
% The file aaai.sty is the style file for AAAI Press 
% proceedings, working notes, and technical reports.
%

%\title{Learning to Play the Rubik's Cube From Image to Solutions with PDDL}

\maketitle
\begin{abstract}


\begin{quote}

Rubik's Cube (RC) is a well-known and computationally challenging puzzle that has motivated AI researchers to explore  efficient alternative representations and problem-solving methods. The ideal situation for planning here is that a problem be solved optimally and efficiently represented in a standard notation using a general-purpose solver and heuristics. The fastest solver today for RC is DeepCubeA with a custom representation, and another approach is with Scorpion planner with State-Action-Space+ (SAS+) representation. In this paper, we present the first RC representation in the popular PDDL language so that the domain becomes more accessible to PDDL planners, competitions, and knowledge engineering tools, and is more human-readable. We then bridge across existing approaches  and compare performance.
We find that in one comparable experiment, DeepCubeA\footnote{DeepCubeA trained with 12 RC actions} solves all problems with varying complexities, albeit only 18\% are optimal plans. For the same problem set, Scorpion with SAS+ representation and pattern database heuristics solves 61.50\% problems, while FastDownward with PDDL representation and FF heuristic solves 56.50\% problems, out of which all the plans generated were optimal. Our study provides valuable insights into the trade-offs between representational choice and plan optimality that can help researchers design future strategies for challenging domains combining general-purpose solving methods (planning, reinforcement learning), heuristics, and representations (standard or custom).

% Solution method - Solver, representation, and Heuristic.

% Rubik's Cube is a well-known and computationally challenging puzzle, which has led researchers in the fields of planning and learning to explore more efficient alternative representations and methods for problem-solving. State-Action-Space+ (SAS+) and Planning Domain Definition Language (PDDL) are two common representations used in automated planning. \citet{buchner2022comparison} have developed a SAS+ representation for Rubik's Cube and discovered that pattern databases exhibit superior performance. SAS+ employs a state-variable formalism, while PDDL uses a more expressive predicate logic formalism. PDDL provides greater expressivity, standardization, tool support, and is more human-readable compared to SAS+. In this study, we present the first PDDL formulation for the 3x3 Rubik's Cube and conduct a comparative analysis of the two formal languages, along with DeepCubeA \citep{agostinelli2019solving}, a Reinforcement Learning (RL) approach to solving the Rubik's Cube. Our findings indicate that DeepCubeA solves all problems with varying complexities, albeit only 18\% are optimal plans. For the same problem set, SAS+ representation using pattern databases achieves 61.50\% optimal plans, and PDDL representation with the FF heuristic obtains 56.50\% optimal plans. Our study has yielded significant findings regarding the performance of various formal languages and methods used for solving Rubik's Cube problems. The results demonstrate the effectiveness of these different approaches and provide valuable insights into the trade-offs between representational complexity and plan optimality.



% Ultimately, these results can inform the development of more efficient algorithms and representations for solving complex planning problems beyond the Rubik's Cube.


%Rubik's Cube is a popular puzzle that is also computationally hard to solve. As a result,  researchers in both planning and learning - especially reinforcement and deep-learning, communities have attempted to solve its problems by focusing on efficiency but creating alternative representations and methods. Two popular representations for planning problems are State-Action-Space+ (SAS+) and Planning Domain Definition Language (PDDL), which are used for modeling and solving problems in the field of automated planning. \citet{buchner2022comparison} have created a SAS+ representation for Rubik's Cube and have found that pattern databases show the best performance in solving Rubik's Cube problems. SAS+ represents planning problems in a state-variable formalism, whereas PDDL is a more expressive language for representing planning problems in a predicate logic formalism. PDDL also offers more expressivity, standardization, tool support, and is human readable as opposed to SAS+. In this paper, we introduce the first PDDL formulation for the 3x3 Rubik's Cube and perform a comparative analysis of the above two formal languages as well as DeepCubeA \cite{agostinelli2019solving}, which is a Reinforcement Learning (RL) based approach for solving the Rubik's Cube. We find that DeepCubeA solves 100\% of the problems with varying complexity, although, a majority of them are non-optimal. For the same problem set, SAS+ representation with pattern databases achieves 61.50\% optimal plans and PDDL representation with FF heuristic \footnote{https://www.fast-downward.org/Doc/Evaluator#FF_heuristic} obtains 56.50\% optimal plans. 


%We then create a common set of benchmark problems and evaluate different planners and heuristics. We find that 

%Consequently, it is hard to draw on their lessons to cross-fertilize new ideas. 

%within a given time and storage budget,  the largest number of problems are solved with the Merge and Shrink abstraction heuristic while when the problem is converted to  SAS+ representation,  pattern database (PDB) heuristics are most effective. 

% --- Abstract 2 
% Rubik's Cube is a popular puzzle that is also computationally hard to solve. In the following work, we introduce the PDDL formulation for the 3-dimensional Rubik's Cube, which is first of it's type to the best of our knowledge. We evaluate the performance of our model with different heuristics which have been employed for solving Rubik's Cube. Additionally, We perform a comparative study of two different representations for Rubik's Cube, namely, PDDL and SAS\textsuperscript{+}. We see that Merge & Shrink abstraction heuristic provides the maximum success rate in case of PDDL model, while the pattern database (PDB) heuristics, the state-of-the-art for solving Rubik’s Cube, were successful with SAS\textsuperscript{+} representation.

% Our system has types of two audiences: (a) planning researchers who can explore a hard problem and improve their planning algorithms, and (b)  RC learners who want to learn how to solve the puzzle at their own pace and can now modify an initial plan (e.g., manually, using other algorithms) and see their execution.

% --- Abstract 1
 %We show how to solve Rubik's Cube with a Planner where the domain and problem are encoded in PDDL. For a planning problem, we just need to provide the criteria on the goal state and describe the problem environment with set of possible actions, and then call the planner on an initial state to find a plan or set of plans to reach the goal state. 
% Further, we show how to capture the initial state as an image and convert it to a problem specification with the aim to solve it. 
\end{quote}
\end{abstract}


% -----
% TODO: Means end analysis
% TODO: Landmark planning
% TODO: Completeness
% TODO: Soundness
\section{Introduction}
Current quantum hardware is unable to carry out universal quantum computations due to the buildup of errors that occur during the computation. 
The magnitude of the individual error is currently above the value that the Threshold Theorem requires in order to kick-start quantum error correction and fault-tolerant quantum computation~\cite[Section 10.6]{nielsen_chuang_2010}. 
Although the experimentally achieved fidelity rates are promising and the error bounds are inching closer to the required threshold, we will have to work for the foreseeable future with quantum hardware with errors that build-up during the computation.  This implies that we can only do a limited number of steps before the output of the computation has become completely uncorrelated with the intended one.

For fault-tolerant quantum computing, we repeat four steps: 
1) We apply a number of single and two-qubit quantum gates, in parallel whenever possible; 
2) We perform a syndrome measurement on a subset of the qubits; 
3) We perform fast classical computations to determine which errors have occurred and how to correct them; 
and, 4) We apply correction terms based on the classical computations.
We then repeat these four steps with a next sequence of gates. 
These four steps are essential to fault-tolerant quantum computing. 


The starting point of this work is to use the four steps outlined above, not to carry out error correction and fault-tolerant computation, but to enhance short, constant-depth, {\em uncorrected} quantum circuits that perform single qubit gates and {\em nearest-neighbor} two qubit gates. 
Since in the long run we will have to implement error-correction and fault-tolerant computation anyhow, and this is done by such a four-step process, why not make other use of this architecture? Moreover, on some of the quantum hardware platforms, these operations are already in place.
Embracing this idea we naturally arrive at the question: what is the computational power of \textit{low-depth} quantum-classical circuits organized as in the four steps outlined above? 
We thus investigate circuits that execute a small, ideally constant, number of stages, where at each stage we may apply, in parallel, single qubit gates and {\em nearest-neighbor} two qubit gates, followed by measurements, followed by low-depth classical computations of which the outcome can control quantum gates in later stages. 
It is not clear, at first, whether such circuits, especially with constant depth, can do anything remotely useful. 
But we will see that this is indeed the case: many quantum computations can be done by such circuits in constant depth. 
By parallelizing quantum computations in this way, we improve the overall computational capabilities of these circuits, as we do not incur errors on qubits that are idle, simply because qubits are not idle for a very long time. 
Furthermore, reducing the depth of quantum circuits, at the cost of increasing width, allows the circuit to be run faster even if errors occur.

The first usage of such a four-step layout, not to do error correction, but to perform computations, can be found in the paradigm of measurement-based quantum computing~\cite{gottesman1999demonstrating,raussendorf2001one,jozsa2006introduction,clark2007generalised}: 
A universal form of quantum computing where a quantum state is prepared and operations are performed by measuring qubits in different bases, depending on previous measurements and intermediate measurements.

\citeauthor{PhamSvore2013} were the first to formalize the four-step protocol for performing computations~\cite{PhamSvore2013}. They included specific hardware topologies by considering two-dimensional graphs for imposing constraints on qubit interactions. In their model, they develop circuits for particularly useful multi-qubit gates, including specifying costs in the width, number of qubits, depth, number of concurrent time steps, size, and total number of non-Identity operations.
As a result, they find an algorithm that factors integers in polylogarithmic depth.
\citeauthor{Browne:2011} showed that the main tool in the work by \citeauthor{PhamSvore2013}, the fan-out gate, can also be replaced by additional log-depth classical computations in the measurement-based quantum computing setting~\cite{Browne:2011}.

More recently, \citeauthor{Cirac:2021} introduced a scheme to implement unitary operations involving quantum circuits combined with Local Operations and Classical Communication ($\mathsf{LOCC}$) channels: $\mathsf{LOCC}$-assisted quantum circuits~\cite{Cirac:2021}. Similarly to the four-step scheme we just described, they allow for a short depth circuit to be run on the qubits, followed by one round of $\mathsf{LOCC}$, in which ancilla qubits are measured and local unitaries are applied based on the measurement outcomes. They show that in this model any 1D transitionally invariant matrix-product state (MPS) with fixed bond dimension is in the same phase of matter as the trivial state. Similar ideas can be found in~\cite{TVV_NonAbelianTopologicalOrder_2022, tantivasadakarn2021long}.

In this work, we introduce a new model, called \textit{Local Alternating Quantum-Classical Computations} ($\LAQCC$). In this model we alternate between running quantum circuits (constrained by locality), ending in the measurement of a subset of qubits, and fast classical computations based on the measurement results. The outcome of the classical computations are then used to control future quantum circuits. We allow for flexibility in this model, by giving different constraints to the power of both the quantum circuits and the classical circuits as well as the number of alternations between them. 
Most attention will be given to $\LAQCC$ containing quantum circuits of constant depth, classical circuits of logarithmic depth and at most a constant number of alternations between them. 
Any circuit constructed in this model is considered to be of constant depth. 
We restrict ourselves to logarithmic depth classical computations, as this is the first natural and non-trivial extension beyond constant-depth classical computations. 
Constant-depth classical computations do however also have an equivalent constant-depth quantum implementation.

The definition of $\LAQCC$ sharpens the original definition of \citeauthor{PhamSvore2013} by adding constraints to the intermediate classical computations. This allows us to bound the power of $\LAQCC$ from above. 

The main result of \citeauthor{Cirac:2021}, that 1D translational invariant MPS with fixed bond dimension can be prepared by $\mathsf{LOCC}$-assisted circuits, relies on local symmetries of the MPS. These symmetries allow them to prepare local states (on a constant number of qubits) and glue them together by doing one round of the appropriate entangling measurement and corrections, after which they run a round of local unitaries to get the desired result. This general scheme for preparing states that exhibit an MPS description with the appropriate local symmetries requires only geometrically local unitaries and one round of measurement and corrections an therefore is accessible in $\LAQCC$. Studying different local symmetries, known as Symmetry Protected Topological (SPT) phases of matter, to find measurement-based constant depth circuits for states is a broad ongoing field of research~\cite{TVV_NonAbelianTopologicalOrder_2022, tantivasadakarn2021long, smith2023deterministic}. 
All these schemes have a $\LAQCC$ implementation.

%$\LAQCC$-circuits also exist for general schemes of preparing local states, based on the local tensors, and gluing them together using one round of entangled measurement and corrections, based on the local symmetry. 
%The main result of \citeauthor{Cirac:2021}, that 1D translational invariant MPS with fixed bond dimension can be prepared by $\mathsf{LOCC}$-assisted circuits, relies heavily on local symmetries of the MPS and as a result also has an equivalent $\LAQCC$ implementation. 
%The corrections applied after the measurement round are local unitaries depending on the local symmetries of the MPS. 

 

%This general scheme of preparing local states, based on the local tensors, and gluing it together by doing one round of entangled measurement and corrections, based on the local symmetry, is accessible in $\LAQCC$.
Note however that \citeauthor{Cirac:2021} also suggest a circuit for the $W$-state.
This circuit uses sequentially and dependent measurement-based corrections of the ancilla qubits. 
These dependent measurements translate to sequential alternations between the quantum and classical circuits and therefore increase the total depth to linear depth, exceeding the constant-depth constraints imposed by $\LAQCC$-circuits. 

We study the power of the $\LAQCC$ model with respect to state preparation, showing that even with only constant quantum-depth and logarithmic classical depth it remains possible to prepare states with long-range entanglement.
Another surprising result is that it is unlikely that $\LAQCC$ circuits are classically simulatable. We show that any instantaneous quantum polynomial-time (IQP) circuit~\cite{Bremner2010,Shepherd2009} has an $\LAQCC$ implementation.
Classical simulation of IQP circuits implies the collapse of the polynomial hierarchy to the third level, which is not believed to be true~\cite{Bremner2017}. Therefore, we expect that $\LAQCC$ circuits are unlikely to be classically simulatable. We bound the power of $\LAQCC$ by showing that it is contained in $\QNC^1$, the class of polynomial-size, log-depth circuits.

Next, we also study the power that intermediate classical calculations can add to quantum computations, by considering a new model that alternates between polynomially many polynomial-depth quantum circuits and unbounded classical computations
We study this model by doing a complexity theoretical analysis, where we draw inspiration from the notions of complexity given by \citeauthor{RosenthalYuen:2022}, \citeauthor{MetgerYuen:2023}, and \citeauthor{Aaronson:2004}.
All three complexity notions are based on the notion of state preparation, instead of more traditional definition of complexity such as the decidability of a computational problem. 
The first two consider classes based on sequences of quantum states preparable by a polynomial-sized quantum circuit, where the circuits are uniformly generated by a computational class, for instance, the class $\mathsf{PSPACE}$, which results in the complexity class $\mathsf{StatePSPACE}$~\cite{RosenthalYuen:2022,MetgerYuen:2023}.
The third notion considers a relative complexity, where the complexity is measured between two given states, and is measured by the number of gates, from a given gate-set, required to transform one state in another state~\cite{Aaronson:2004}. 
For our definition of state preparation complexity, we drop the uniformity constraint from~\cite{RosenthalYuen:2022,MetgerYuen:2023} and define a class as $\mathsf{StateX}$, which refers to states preparable by circuits of type $\mathsf{X}$. 
As an example, if $\mathsf{X} = \QNC^0$, this results in the class $\mathsf{StateQNC^0}$, which is the set of states preparable from the $\ket{0}^n$ state by poly-size constant-depth circuits. 
This notion is similar to the relative complexity from~\cite{Aaronson:2004}, where one state is the  $\ket{0}^n$ state and instead of counting the number of gates we consider the set of states preparable by a fixed number of gates. Using this notion of complexity we show that any state preparable by an $\LAQCC^*$ circuit is also preparable by a $\mathsf{PostQPoly}$ circuit, the class of circuits of polynomial depth with an additional post-selection gate. 

All Clifford circuits have a constant-depth $\LAQCC$ implementation, implying that any stabilizer state can be implemented by a constant-depth $\LAQCC$ circuit, see Section~\ref{sec:clifford_circuits} for a proof of this statement. 
Efficient circuits for stabilizer states have been known already through measurement-based quantum computing. Therefore this paper focuses on the preparation of non-stabilizer states, and as a surprising result we find novel constant-depth protocols for four very natural classes of non-stabilizer states.
Despite the extensive research into these four classes of non-stabilizer states and the many applications of them, no efficient constant- or low-depth state preparation protocols are known yet. We specifically consider these four classes as they are all often used as initial states in other algorithms.

The first state is a uniform superposition over an arbitrary number of states. 
This state finds applications in many quantum algorithms, as they often start with a uniform superposition over multiple states. 
This superposition is often achieved by applying Hadamard gates to every qubit due to its simplicity to prepare. 
Yet, the analysis of many algorithms, such as Shor's algorithm~\cite{Shor:1997}, would benefit from a different initial superposition. 
The circuit to prepare the uniform superposition over an arbitrary number of states uses an exact version of Grover search as a subroutine, that turns a probabilistic circuit, with a known constant probability of success, into a deterministic circuit. 
We use the circuit for preparing a uniform superposition over an arbitrary number of states as a subroutine in the next two quantum state preparation protocols. 

The second state is the $W$-state, the uniform superposition over all computational basis states of Hamming-weight~$1$, a natural long-ranged entangled state that displays a fundamentally nonequivalent type of entanglement from the Greenberger–Horne–Zeilinger state~\cite{WState:2000}, for which $\LAQCC$-type constant-depth circuits were previously known~\cite{PhamSvore2013, Cirac:2021}. 
The $W$-state is often used as benchmark for new quantum hardware~\cite{Haffner2005,Neeley2010,GarciaPerez:2021}. 
A novel way to prepare the $W$-state therefore gives a new way to benchmark different quantum devices with each other. 
A circuit for preparing the $W$-state was given in~\cite{Cirac:2021}, but this implementation requires sequentially alternating measurements followed by local unitaries, which in the $\LAQCC$ model is not considered to be of constant depth. 
We improve this protocol by giving an $\LAQCC$ implementation of the $W$-state, based on a compress-uncompress method that links the one-hot and binary encoding of integers.

The third state considered is the Dicke state, a generalization of the $W$-state, a superposition over all computational basis states with Hamming-weight $k$~\cite{Dicke:1954}. 
Dicke states have relevance in various practical settings.
For instance, for quantum game theory~\cite{zdemir2007}, quantum storage~\cite{Bacon_Compress:2006,Plesch:2010}, quantum error correction~\cite{ouyang2014permutation}, quantum metrology~\cite{toth2012multipartite}, and quantum networking~\cite{prevedel2009experimental}. 
Dicke states have been used as a starting state for variational optimization algorithms, most notably Quantum Alternating Operator Ansatz (QAOA)~\cite{Hadfield2019}, to find solutions to problems such as Maximum k-vertex Cover~\cite{Brandhofer2022,cook2020quantum}.
The ground states of physical Hamiltonians describing one-dimensional chains tend to show a resemblance to Dicke states such as states resulting from the Bethe ansatz, making them an ideal starting state when investigating the ground state behavior of these Hamiltonians~\cite{TDL_BetheAnsatzDerivation:2010,B_ExcitedStateQuantumPhaseTransitions:2013,DickeTransitions:2021}. 
For instance, the algorithm by \citeauthor{van2021preparing}, who give an algorithm to prepare the Bethe ansatz eigenstates of the spin-1/2 XXZ spin chain, starts by first preparing a Dicke state~\cite{van2021preparing}. 
A Dicke-state preparation protocol based on the compress-uncompress methodology used in the $W$-state furthermore finds applications in entanglement distillation, where the entanglement of a large state is concentrated on only a few qubits. 
Efficient deterministic circuits for preparing Dicke states have been proposed by \citeauthor{bartschi2019deterministic}~\cite{bartschi2019deterministic, bartschi2022deterministic_short_depth}. 
They provide a quantum circuit of depth $\mathO(k \log(\frac{n}{k}))$, allowing arbitrary connectivity, to prepare a Dicke state, which they conjecture to be optimal when $k$ is constant. 
In this work, we provide a constant-depth $\LAQCC$ circuit below their conjectured bound already for constant $k$. 
However, this does not directly disprove their conjecture, as we allow for intermediate measurements and classical computations. 
More significantly, we even construct constant-depth $\LAQCC$ circuits for $k = \mathO(\sqrt{n})$ greatly improving their bound.
This construction extends the compress-uncompress method for the $W$-state combined with additional subroutines. 

We continue with a log-depth state preparation protocol for the Dicke-state for arbitrary $k$. 
This protocol implements an efficient transformation between the factoradic number representation and the combinatorial number representation of a positive integer. 
The combinatorial number representation relates directly to the Dicke state. 
The provided efficient transformation between number representation systems might be of independent interest. 

We conclude by modifying our protocol for preparing a Dicke-state to a protocol that prepares quantum many-body scar states in constant-depth. 
These states have low entanglement and longer coherence times than states with similar energy density.
These characteristics make many-body scar states interesting to analyze and relevant within physics.
Many-body scar states appear for instance in the AKLT model~\cite{AKLT:1987,MRBAR:2018,MRB:2018} and different spin models~\cite{SI:2019,MOBFR:2020}.
Known methods for preparing these states have polynomial-depth~\cite{Gustafson:2023}, whereas our circuit has constant depth. 

% We conclude by studying the power that intermediate classical calculations can add to quantum computations. 
% In this study, we define a new model that relaxes constant-depth quantum circuits to polynomial depth quantum circuits, log-depth classical calculations to unbounded classical computations and a constant number of alternations to a polynomial number of alternations. 
% We call this model $\LAQCC^*$. 
% We study this model by doing a complexity theoretical analysis, where we draw inspiration from the notions of complexity given by \citeauthor{RosenthalYuen:2022}, \citeauthor{MetgerYuen:2023}, and \citeauthor{Aaronson:2004}.
% All three complexity notions are based on the notion of state preparation, instead of more traditional definition of complexity such as the decidability of a computational problem. 
% The first two consider classes based on sequences of quantum states preparable by a polynomial-sized quantum circuit, where the circuits are uniformly generated by a computational class, for instance, the class $\mathsf{PSPACE}$, which results in the complexity class $\mathsf{StatePSPACE}$~\cite{RosenthalYuen:2022,MetgerYuen:2023}.
% The third notion considers a relative complexity, where the complexity is measured between two given states, and is measured by the number of gates, from a given gate-set, required to transform one state in another state~\cite{Aaronson:2004}. 
% For our definition of state preparation complexity, we drop the uniformity constraint from~\cite{RosenthalYuen:2022,MetgerYuen:2023} and define a class as $\mathsf{StateX}$, which refers to states preparable by circuits of type $\mathsf{X}$. 
% As an example, if $\mathsf{X} = \QNC^0$, this results in the class $\mathsf{StateQNC^0}$, which is the set of states preparable from the $\ket{0}^n$ state by poly-size constant-depth circuits. 
% This notion is similar to the relative complexity from~\cite{Aaronson:2004}, where one state is the  $\ket{0}^n$ state and instead of counting the number of gates we consider the set of states preparable by a fixed number of gates. Using this notion of complexity we show that any state preparable by an $\LAQCC^*$ circuit is also preparable by a $\mathsf{PostQPoly}$ circuit, the class of circuits of polynomial depth with an additional post-selection gate. 

\paragraph{Summary of results}
\begin{itemize}
    \item We give a new definition of a computational model that captures the power of the four step process: applying a constant number of layers of one- and two-qubit gates; performing a syndrome measurement; perform a fast classical computation determining corrections; apply corrections. We call this model \emph{Local Alternating Quantum Classical Computations}, or $\LAQCC$ for short. In this model we bound the allowed quantum operations, intermediate classical calculations, and number of rounds separately. In Section~\ref{sec:LAQCC_model} we define this model and give a list of operations based on results from literature contained in this computational model. In some of these operations we explicitly use that we allow for multiple, but at most constant, rounds  of corrections.
    \item  We show show that there exist $\LAQCC$ circuits that can not be weakly simulated in Section~\ref{sec:IQP_in_LAQCC}. We further show that for every $\LAQCC$ circuit there exists a $\QNC^1$ circuit simulating it perfectly, in Section~\ref{sec:LAQCC_in_QNC1}.
    \item We introduce a new type computational complexity for preparing states and show that the extension of $\LAQCC$ where we allow a polynomial number of rounds and unbounded classical computation, is contained in $\mathsf{PostQPoly}$, the class of polynomial circuits with post-selection, in Section~\ref{sec:Complexity results}.
    \item We show a protocol to prepare the uniform superposition state of size $q$ in $\LAQCC$ using $\mathO(\ceil{\log_2(q)}^2)$ qubits in Section~\ref{sec:superposition_modulo_q}. 
    \item We show a protocol to prepare the $W_n$ state in $\LAQCC$ using $\mathO(n\log(n))$ qubits in Section~\ref{sec:W_state_in_LAQCC}.
    \item We show two ways of preparing the Dicke-$(n,k)$ state. The first method is in $\LAQCC$, works up to $k = \mathO(\sqrt{n})$, uses $\mathO(n^2\log(n))$ qubits, and is found in Section~\ref{sec:dicke:small_k}. The second method is in $\LAQCC\text{-}\mathsf{LOG}$ (an extension of $\LAQCC$ allowing for logarithmic number of alterations instead of constant), works for any $k$, uses $\mathO(\text{poly}(n))$ qubits, and is found in Section~\ref{sec:Dicke_in_LAQCC_LOG}. 
    \item We extend on our $\LAQCC$ method of generating Dicke-$(n,k)$ states for $k = \mathO(\sqrt{n})$ and show a protocol to generate many-body scar states for a particular Hamiltonian in $\LAQCC$ (Section~\ref{sec:many_body_scar}). 
\end{itemize}
Summarized in a table, we provide the following state generation protocols:
\begin{table}[htb]
\centering
\begin{tabular}{l|l|l|l}
\textbf{State description} & \textbf{Width} & \textbf{Depth} & \textbf{Implementation}\\
\hline 
Uniform superposition mod $q$: $\frac{1}{\sqrt{q}} \sum_{i = 0}^{q-1}\ket{i}$ & $\mathO(\ceil{\log^2 q})$ & $\mathO(1)$ & Section~\ref{sec:superposition_modulo_q}\\

$W$-state: $\frac{1}{\sqrt{n}}\sum_{i = 0}^{n-1}\ket{e_i}$ & $\mathO(n \log n)$ & $\mathO(1)$ & Section~\ref{sec:W_state_in_LAQCC}\\

Dicke-$(n,k)$, $k = \mathO(\sqrt{n})$: $\binom{n}{k}^{-1/2}\sum_{x \in \{0,1\}^n: |x| = k} \ket{x}$ &  $\mathO(n^2\log n)$ & $\mathO(1)$ 
&Section~\ref{sec:dicke:small_k}\\

Dicke-$(n,k)$: $\binom{n}{k}^{-1/2}\sum_{x \in \{0,1\}^n: |x| = k} \ket{x}$ & $\mathO(\text{poly}(n))$ & $\mathO(\log n)$ &Section~\ref{sec:Dicke_in_LAQCC_LOG}\\

QMBS: $\ket{S_k} = \frac{1}{k! \sqrt{\mathcal N(n,k)}}(Q^\dagger)^k \ket{\Omega}$ &  $\mathO(n^2\log n)$ & $\mathO(1)$  &  Section~\ref{sec:many_body_scar}
\end{tabular}
\caption{Summary of state preparation protocols given in this paper.}
\label{tab:sate_prep}
\end{table}
In the entry for the quantum many-body scar state $Q$ denotes the raising operator and $\mathcal N(n,k)=\binom{n-k-1}{k}$. 
Section~\ref{sec:many_body_scar} will provide more details on the variables and the implementation. 

\paragraph{Organization of the paper}
\noindent We first introduce relevant preliminaries in Section~\ref{sec:preliminaries}. 
In Section~\ref{sec:LAQCC_model} we formally define the class of Local Alternating Quantum-Classical Computations ($\LAQCC$). We also show that any Clifford circuit can be implemented in constant depth $\LAQCC$ (a result based on a result from measurement-based quantum computing~\cite{jozsa2006introduction}). 
This result allows us to give many useful multi-qubit gates and routines in Section~\ref{sec:gates_created_in_LAQCC}. 
Beyond that we show that constant depth $\LAQCC$ circuits are contained in $\QNC^1$ and that any $\mathsf{IQP}$ circuit has an $\LAQCC$ implementation.
We conclude this section with an analysis of a more powerful instantiation of $\LAQCC$ and show an inclusion with respect to the class $\mathsf{PostQPoly}$, which is the class of circuits of polynomial depth with one additional post-selection gate. 
In Section~\ref{sec:state_prep_in_LAQCC} we give $\LAQCC$ circuit implementations for preparing the uniform superposition over an arbitrary number of states, the $W$-state and the Dicke state up to $k = \mathO(\sqrt{n})$. We furthermore give a log-depth circuit implementation for preparing the Dicke state for any $k$. We conclude by showing a $\LAQCC$ circuit for generating many body scar states of a particular type of Hamiltonian.


\vspacebeforesection
\section{Background}
\label{sec:background}

In this section, we provide the necessary background information to ensure a comprehensive understanding of the attack described in this paper. We start with a description of the Distributed Hash Table (DHT) used by IPFS, followed by its content resolution mechanisms. We also detail techniques for network size estimation, necessary for our attack detection and mitigation mechanisms.

\vspacebeforesection
\subsection{IPFS DHT}
\label{sec:kad_dht}

We review the features of the Kademlia DHT~\cite{maymounkov2002kademlia} and its \texttt{libp2p} implementation~\cite{libp2p_github} that are the most relevant to our attack.
To participate in the DHT, each peer generates a public/private key pair and derives an identity $\peerid \in \{0,1\}^{256}$ as the hash of its public key.
Ideally, each peer generates a random key pair and, therefore, peer IDs are distributed uniformly and independently over the space $\{0,1\}^{256}$.
While honest nodes follow this rule, malicious nodes may generate and choose from an arbitrary number of key pairs.
Each peer maintains a routing table consisting of $m=256$ buckets.
The $i$-th bucket contains the addresses of up to $k=20$ peers whose peer IDs share a common prefix of exactly $i$ bits with the peer's own peer ID. 

%
A new participant node joins the IPFS network by contacting one of the hardcoded bootstrap nodes. This bootstrap node provides the new node with some initial peers allowing it to join the DHT. The new node uses this information to perform a walk through the DHT towards its own peer ID.
The walk allows to: \textit{(i)}~make sure that there is no other node in the network with the same ID; \textit{(ii)}~discover new peers and fill the newcomer's DHT routing table. At the same time, the newcomer establishes \bitswap~\cite{de2021accelerating} connections to a subset of encountered peers (usually around 300 of them). The core role of the \bitswap protocol is to enable bilateral content transfer and to play the role of a cache for recently-accessed content.

The main DHT operation $\Call{GetClosestPeers}{\key}$ returns the $k=20$ closest peers to $\key$. 
%
In Kademlia, the distance between two keys $x$ and $y$ in the key space is given by $x \oplus y \in \{0,...,2^{256}-1\}$, where $\oplus$ denotes the bitwise XOR operation on the keys; the resulting binary string is interpreted as an integer.
%
When a client wants to find the peers with IDs closest to $\key$, it sends a request to the $\alpha=3$ peers in its routing table whose peer IDs are closest to $\key$. Each of these peers returns the $k$ closest peers to $\key$ in its own routing table and the addresses of these peers. 
%
The client again sends a request to the $\alpha$ peers closest to $\key$, among peers in its routing table and those whose addresses it just received. This process repeats until the client does not find any more peers closer to $\key$.
Due to network churn and imperfect routing tables, we observed in our experiments that successive calls to $\Call{GetClosestPeers}{\key}$ do not always return the same set of $k=20$ peers (we provide more details in \Cref{sec:evaluation}, \Cref{fig:20closest}). This is an important limitation affecting our attack.

\vspacebeforesection
\subsection{Content Resolution in IPFS}
\label{sec:ipfs}

IPFS is a content-centric network.
It allows its participant to request files without specifying their location. 
%
Content is indexed by content IDs $\cid \in \{0,1\}^{256}$ that are derived from a hash of that content.
Both peer IDs and CIDs are used as keys in the DHT.
Each node can play the role of a \provider, \downloader, or \resolver. 
The process of content advertisement and resolution is illustrated in \Cref{fig:add_get_provider}.

%
When a \provider wishes to publish content with a given $\cid$ on IPFS, it creates a \emph{provider record} that contains $cid$ and the \provider's address.
During a $\Call{Provide}{\cid}$ operation, the \provider first uses $\Call{GetClosestPeers}{\cid}$ to locate the $k=20$ peers with their peer IDs closest to $\cid$, 
%
and then sends them a $\mathsf{PutProvider}$ message including the provider record (\Cref{fig:add_get_provider}(a)).
We call the peers that hold provider records for $\cid$ the \emph{resolvers} for $\cid$.

Each CID can have several \providers. In fact, by default, each IPFS client becomes a provider for each piece of content it downloads for a fixed amount of time (12h, 24h, or 48h depending on the client version or custom configuration). As a result, the system provides an auto-scaling feature with supply automatically rising with demand.

%
When a \downloader wishes to fetch a piece of content, it first sends a request to all its \bitswap peers. If none of them has the content, the \downloader uses the DHT-based resolution system. We stress that the \bitswap protocol plays the supporting role of a cache in the dissemination of popular files. However, the mechanism does not provide reliable content resolution, in particular for new or less popular content. %

When \bitswap unstructured search fails, the \downloader resolves $\cid$ using $\Call{FindProviders}{\cid}$. This operation uses a DHT walk identical to that of $\Call{GetClosestPeers}{\cid}$ to find $k$ \resolvers but also queries encountered nodes for a provider record for $\cid$ (\Cref{fig:add_get_provider}(b)). The process terminates when either 20 \providers have been found, or all \resolvers have been asked. Querying all encountered nodes (\ie, not only the designated \resolvers) is useful because some of the encountered nodes may have a provider record in their cache.
%

Upon receiving a provider record, the client connects to the address specified in the provider record to retrieve the actual content (\Cref{fig:add_get_provider}(c)).
Provider records are not authenticated, and therefore malicious \providers may respond with incorrect provider records (or may not respond at all). However, the integrity of the content is preserved because the hash of the retrieved content can be verified against its $\cid$.
%


%

\input{img/add_get_provider.tex}

\vspacebeforesection
\subsection{Network Size Estimator}
\label{sec:netsize}

The number of nodes in a decentralized system is generally unknown due to the avoidance of centralized membership management.
This number is nonetheless useful for optimizations, deciding on individual node configurations, or security mechanisms.
Various methods were proposed for the decentralized estimation of unstructured and structured networks~\cite{eli-sohl-dht-size-estimation,kostoulas2005decentralized, manku2003symphony}.
We use in this work a mechanism developed initially by Protocol Labs as part of a mechanism for decreasing the latency of publishing content in IPFS~\cite{network-size-estimation-notion,network-size-estimation-github-pr}.

%
%
%
%
%
%
%
%
%
%

Each node in the DHT refreshes its routing table periodically (every $10$ minutes in \texttt{libp2p}). 
For this, the node samples $m$ random keys (one for each bucket of its routing table)
%
and queries the DHT to obtain the $k=20$ closest peer IDs to each key.
Using these, the node then computes the average distance between each one of these keys $\key_j$ for $j=1,\dots,m$ and their $i$-th closest peer ID for $i=1,...,k$ (with $m=256$ and $k=20$).
\begin{equation}
    \label{equ:avg-dist}
    \overline{D}_i = \frac{1}{m} \sum_{j=1}^m \operatorname{dist}(\key_j, \peerid_{j}^{(i)})
\end{equation}
where $\peerid_{j}^{(i)}$ is the $i$-th closest peer ID to $\key_j$.
With $N$ peers in the DHT and peer IDs uniformly distributed in the hash space, the expected distance between a $\key$ and its $i$-th closest peer ID is $\frac{2^{256}i}{N+1}$. The node then runs a least square regression to compute the value of $N$ for which the expected distances best fit the empirical average distances, \ie,
\begin{equation}
    \label{equ:netsize-least-squares}
    \hat{N} = \arg\min_{N} \sum_{i=1}^k \left(\overline{D}_i - \frac{2^{256}i}{N+1}\right)^2.
\end{equation}
The resulting estimate $\hat{N}$ can be computed in closed form.
%

When a node starts running, it must perform DHT queries for a few random keys to initialize its network size estimate. 
Since a larger number of queries will result in higher accuracy, making more queries than what is needed to initialize one's routing table is recommended.
Thereafter, keeping the estimate up-to-date does not require any excess DHT queries beyond what is already used for refreshing the routing table as this is done frequently (every 10 minutes).

While the network size estimate has a stochastic variance resulting from the probability distribution of the honest peer IDs, it is hard for an attacker to bias the estimate significantly. Since the estimator uses the density of peer IDs around keys chosen uniformly at random, the adversary would require numerous Sybil nodes (on the order of the whole network size) to significantly affect the peer ID density around those keys.

% \section{System Setup}
\label{sec:interface}


% Figure environment removed





Our experimental setup consists of both hardware and software components to effectively evaluate the performance of the proposed autocalibration technique with real users. We utilized a Tobii PCEye eye tracker, with a sampling rate of 60 Hz, connected to a standard Windows laptop via USB. The eye tracker tracked gaze data directed towards a 24-inch screen that displayed an on-screen visual keyboard (Fig. \ref{fig:activation}). 
The keyboard was designed as part of an UWP application built on top of the Microsoft Gaze Interaction Library \citep{gil}. The application was written using the Tobii Pro SDK (designed to offer access to gaze data from Tobii eye trackers) in C\#. It facilitates the capture and processing of x, y coordinates from the eye tracker, providing necessary data for our autocalibration technique.
The software for our autocalibrated gaze typing technique was developed in Python, which received the 2D gaze coordinates in real-time from the UWP application via Google's remote procedure call (RPC) protocol. This allowed Python scripts to access gaze data, analyze it, and apply the necessary miscalibration corrections which could be displayed back on the visual keyboard application. 

% Figure environment removed

The customized gaze typing application is functionally similar to visual keyboards available as part of standard software such as Windows Eye Control. However, it serves as a test bed with more control to process and update the miscalibrated gaze coordinates in real-time. Figure \ref{fig:activation} illustrates the mechanism for typing a character on the visual keyboard through eye gaze. A user's detected gaze location is displayed as a red dot on the screen. The user can direct their gaze at a desired character for typing. If they fixate on a key for $50$ ms, a dwell timer of $400$ms is initiated by the system. The start of the timer is depicted by a green rectangle on the key of interest (Fig. \ref{fig:activation}(a)). If the user continues to look at the same key, the rectangle slowly changes size and decreases in area over time, eventually collapsing at the center of the key (Fig. \ref{fig:activation}(b)). Otherwise, the timer is aborted and the user must fixate and dwell again to type any character subsequently. The dwell timer is a mechanism to ensure that a user is intentionally willing to type that specific character. Once the timer is completed, the user receives visual feedback (the key turns red as shown in Fig. \ref{fig:activation}(c)) that the character is typed on the text box at the top of the screen. The eye tracker can be calibrated via a standard Tobii software available with the purchase of the tracker (Fig.\ref{fig:calibration}). It uses a procedure of following at a set of dots (in sequence) on the screen that, and then return to controlling an application of interest with their calibrated gaze.
% \vspace{-0.3cm}
\section{Comparision of RC representations}

In this section, we describe and provide a comparative analysis of different RC representations comprising of RL and Planning formal languages.
% \vspace{-0.25cm}
\subsection{DeepCubeA}
The DeepCubeA algorithm adopts a unidimensional array as a representation of the Rubik's Cube (RC) state. Specifically, this array encompasses 54 elements, each of which corresponds to a unique sticker color present on a cube piece of the RC. While this array-based modeling offers computational advantages, it is limited by its inability to fully encapsulate the spatial orientation of Rubik's Cube. Furthermore, the usage of a hard-coded representation and implicit assumptions concerning the position of cubelets poses a challenge to novice users seeking to comprehend the array-based representation.
% \vspace{-0.25cm}
\subsection{SAS+}
In \citealt{buchner2022comparison}, Rubik's Cube is modeled with 18 actions in SAS+ representation as a factored effect task, with each face labeled as F, B, L, R, U, or D. The orientation of each cube piece is represented as a triple of values, and for corner cube pieces, the orientation is a permutation of \{1, 2, 3\}, while for edge cube pieces, it is a permutation of \{1, 2, \#\} (where \# represents a blank symbol). The rotation of the cube in 3D space is captured as the permutation of the respective triple for each cube piece. The SAS+ model has 20 variables, 480 fact pairs and bytes required for representing each state is 16 bytes.

% Figure environment removed

% While sticking to planning terminology of actions, problems and plans, we want to clarify terminology prevalent in RC literature that we will also refer to. A sequence of actions are called {\em macro-actions} and a collection of macro-actions are called {\em algorithms} in RC parlance. A solver may employ a strategy for sequencing macro-actions to solve the cube. 

\begin{lstlisting}[
  float=t,
  caption={Action L of Rubik's Cube modeled in PDDL},
  label={lst:pddl},
  language=PDDL, numbers=none,
  frame=lines,
  xleftmargin=\parindent, xrightmargin=\parindent
  ]
(:action L
:effect (and
;for corner cubelets
(forall(?x ?y ?z)(when (cube1 ?x ?y ?z) 
  (and (cube2 ?y ?x ?z))))
(forall(?x ?y ?z)(when (cube3 ?x ?y ?z) 
  (and (cube1 ?y ?x ?z))))
(forall(?x ?y ?z)(when (cube4 ?x ?y ?z) 
  (and (cube3 ?y ?x ?z))))
(forall(?x ?y ?z)(when (cube2 ?x ?y ?z) 
  (and (cube4 ?y ?x ?z))))
;for edge cubelets
(forall(?x ?z)(when (edge13 ?x ?z) 
  (and (edge12 ?x ?z))))
(forall(?y ?z)(when (edge34 ?y ?z) 
  (and (edge13 ?y ?z))))
(forall(?x ?z)(when (edge24 ?x ?z) 
  (and (edge34 ?x ?z))))
(forall(?y ?z)(when (edge12 ?y ?z) 
  (and (edge24 ?y ?z))))))
\end{lstlisting}
% \vspace{-0.2cm}
\subsection{PDDL}

In the PDDL domain, the Rubik's cube problem environment has been defined by assuming the cube pieces are in a fixed position and are named accordingly, as defined in Figure \ref{FIG: rubiks-cube}. These fixed cube pieces are modeled as predicates in the RC domain and the colors they possess in the three-dimensional space as parameters of these predicates. With the help of conditional effects, each action in the RC environment is defined as the change of colors on these fixed cube pieces. The 3D axis of the cube is considered as three separate parameters {\em X}, {\em Y}, and {\em Z} that specify the position of the colors on the cube's pieces. One of these axes can be connected to each face of the cube. According to the representation shown in Figure \ref{FIG: rubiks-cube}, the respective faces on each axis are: $F_{X} = \langle U,D \rangle$; $F_{Y} = \langle R,L \rangle$; $F_{Z} = \langle F,B \rangle$. These different faces of the cube can be identified by the color of the middle cube piece. We considered White, Red, and Green colors as the colors on the front(F), up(U) and right(R) faces respectively (similarly, the counter colors on the counter faces).

The following conventions regarding the RC cube pieces are considered to model the RC domain actions in the PDDL:
\begin{enumerate}
  \item The corner cube pieces of the RC are modeled as a three-color cubelet and are specified as a predicate with three parameters: {\em x}, {\em y}, and {\em z}, which indicate the piece's colors on three separate axes. There are 8 corner pieces in RC.
  \item The edge cube pieces, which are in between corner cube pieces, are modeled as two-color cubelet and is specified as a predicate with two parameters denoting the piece's colors on the two axes. There are 12 edge pieces in RC.
  \item We do not consider the rotations performed on the middle layer, as this can be resolved into rotation of right and left faces in the opposite direction. As a result, the middle cube piece of a face is unaltered.
\end{enumerate}

The predicate names define the fixed position of the cubelets that are defined with respect to the different faces of the cube. The representation considered for the cube positions is shown in Figure \ref{FIG: rubiks-cube}. One of the actions, action `L', of RC designed in PDDL from the description provided is shown in Listing \ref{lst:pddl}. In this, we refer to corner cube pieces as \textit{cubeP} and edge cube pieces as \textit{edgePQ} where \textit{P} and \textit{Q} are the numbers for the cube pieces as stated in Figure \ref{FIG: rubiks-cube}.
When the move L is applied to the RC, for example, the left face is rotated clockwise. This may be regarded as a 90-degree clockwise translation of colors from the left-face corner and edge cube pieces. Considering the RC representation shown in Figure \ref{FIG: rubiks-cube}, the colors on the pieces: cube1, cube2, cube4, and cube3, are circularly shifted towards the right. The same applies to the edge pieces. As the left face falls in the Z-plane, only the  X-axis and Y-axis colors on the cube pieces are affected. 

During execution of a problem, FastDownward first translates the domain into a SAS model. The resulting SAS version of the RC PDDL domain model has 480 variables and 960 fact pairs, with each state requiring 60 bytes for representation. 

% \begin{lstlisting}[
%   float=!b,
%   caption={Action L of Rubik's Cube modeled in PDDL},
%   label={lst:pddl},
%   language=PDDL, numbers=none,
% %   frame=single
%   ]
% (:action L
% :effect (and
% ;for corner cubelets
% (forall(?x ?y ?z)(when (cube1 ?x ?y ?z) 
%   (and (not (cube1 ?x ?y ?z)) (cube2 ?y ?x ?z))))
% (forall(?x ?y ?z)(when (cube3 ?x ?y ?z) 
%   (and (not (cube3 ?x ?y ?z)) (cube1 ?y ?x ?z))))
% (forall(?x ?y ?z)(when (cube4 ?x ?y ?z) 
%   (and (not (cube4 ?x ?y ?z)) (cube3 ?y ?x ?z))))
% (forall(?x ?y ?z)(when (cube2 ?x ?y ?z) 
%   (and (not (cube2 ?x ?y ?z)) (cube4 ?y ?x ?z))))
% ;for edge cubelets
% (forall(?x ?z)(when (edge13 ?x ?z) 
%   (and (not(edge13 ?x ?z))(edge12 ?x ?z))))
% (forall(?y ?z)(when (edge34 ?y ?z) 
%   (and (not(edge34 ?y ?z))(edge13 ?y ?z))))
% (forall(?x ?z)(when (edge24 ?x ?z) 
%   (and (not(edge24 ?x ?z))(edge34 ?x ?z))))
% (forall(?y ?z)(when (edge12 ?y ?z) 
%   (and (not(edge12 ?y ?z))(edge24 ?y ?z))))))
% \end{lstlisting}
\section{Experiments}
% \haizhou{Follow the same way of introduction as we did in Section2.}
% \noindent In this section, we will introduce datasets and experimental setups that we used. Then we evaluate our method, other self-supervised methods, and supervised methods under different distribution shifts (\ie, concept shifts and covariate shifts) under common settings (\ie, transductive, inductive settings). It has to note that we focus on node-level tasks (\eg, node classification) in this work. As for graph-level tasks, we leave it as our future work and some simple experiments can be found in Appendix~\ref{app:graph_classification}. 
In this section, we first introduce the experimental setup including datasets, training, and evaluation protocol in Section~\ref{sec:dataset}~and~\ref{sec:unsupervised}. 
% Next, we present our experimental setup and conduct extensive experiments to evaluate our method in Section~\ref{sec:unsupervised}. 
We then perform an ablation study to demonstrate the effectiveness of each proposed component in Section~\ref{sec:ablation}. 
Additionally, we analyze the impact of important hyper-parameters in Section~\ref{sec:sensitivity}. 
Subsequently, we integrate our method with various encoding models, showcasing the model-agnostic nature of our recipe in Section~\ref{sec:other_models}. 
Finally, we provide some qualitative results such as feature visualization in Section~\ref{sec:vis}.
It is important to note that we focus on node-level tasks (\eg, node classification) in this work. As for graph-level tasks, we leave it as our future work, while some simple experiments are also provided in Appendix~\ref{app:graph_classification}.

\subsection{Datasets}\label{sec:dataset}
There exist some benchmarks for evaluating graph out-of-distribution generalization~\cite{good,ji2022drugood,gds}. 
Among them, GOOD~\cite{good} is the most representative and comprehensive benchmark that curates more diverse graph datasets with diverse tasks, including single/multi-task graph classification, graph regression, and node classification involving more distribution shifts (\ie, concept shifts and covariate shifts). Hence in this work, we follow the evaluation protocol proposed in \cite{good}. Furthermore, we validate the effectiveness of our method in the datasets (\ie, Amazon-Photo, Elliptic) that are used in EERM~\cite{eerm}. The statistics and detailed introduction to these datasets can be found in Table~\ref{tab:dataset} and Appendix~\ref{app:datasets}.

\begin{table*}[htp]
\caption{The descriptions of datasets. ``Domain-Level'' means splitting by graphs, ``Time-Aware'' denotes splitting according to chronological order.``Word'' and ``Degree'' represent splitting according to word diversity and node degree respectively. ``Language'' means splitting by user language, suggesting the prediction should not be impacted by the language the user use. ``University'' denotes splitting according to the domain university, implying that the prediction of webpages should be based on word contents and link connections rather than university features. ``Color'' means that nodes are split according to node differences in covariate shift and color-label correlations in concept shift.}
\label{tab:dataset}
\centering
\begin{tabular}{cccccccc}
\toprule
Datasets     & Network Type        & \#Nodes & \#Edges & \#Attributes &\#Classes& Train/Val/Test Split     & Metric   \\
% Cora         & Artificial Transformation & 2,703   &         &              &         &                      & Accuracy \\
Amazon-Photo\footnotemark
             & Co-purchasing network      & 7,650   & 119,081   & 755          & 10      & Domain-Level         & Accuracy \\
Elliptic\footnotemark  
             & Bitcoin transactions       & 203,769 & 234,355   & 165          & 2       & Time-Aware           & F1-Score \\
GOOD-Cora    & Scientific publications    & 19,793  & 126,842   & 8,710         & 70      & Word/Degree          & Accuracy \\
% GOOD-Arxiv   & arXiv papers               & 169,343 & 2,315,598 & 128          & 40      & Time/Degree          & Accuracy \\
GOOD-Twitch  & Gamer network              & 34,120  & 892,346   & 128          & 2       & Language             & ROC-AUC  \\
GOOD-CBAS    & A BA-house graph           & 700     & 3,962     & 4             & 4       & Color                & Accuracy \\
GOOD-WebKB   & Webpage network            & 617     & 1,138     & 1,703         & 5       & University           & Accuracy \\
\bottomrule
\end{tabular}
\end{table*}
\footnotetext[5]{This dataset is adopted from~\cite{yang2016revisiting}. \cite{eerm} constructs ten graphs with different environment id’s for each graph.} 
\footnotetext[6]{The original is available on \hyperlink{https://www.kaggle.com/ellipticco/elliptic-data-set}{https://www.kaggle.com/ellipticco/elliptic-data-set}}

\subsection{Unsupervised Representation Learning}\label{sec:unsupervised}
\subsubsection{Transductive Setting}~\label{sec:trans}
% \noindent\textbf{Baselines.}\quad We conduct experiments with 12 baselines which consist of three categories: supervised methods and self-supervised generative methods, self-supervised contrastive methods. Specifically, we compare with three supervised baselines: empirical risk minimization~(ERM)~\cite{erm}, invariant risk minimization (IRM)~\cite{irm}, and a recent proposed graph OOD method dubbed EERM~\cite{eerm}. We also compare various unsupervised node-level representation learning methods: three self-supervised generative methods including GAE~\cite{gae}, VGAE~\cite{gae}, GraphMAE~\cite{gmae} and seven self-supervised contrastive methods: DGI~\cite{dgi}, MVGRL~\cite{mvgrl}, GRACE~\cite{grace}, RoSA~\cite{rosa}, BGRL~\cite{bgrl}, COSTA~\cite{costa}, SwAV~\cite{swav}. The descriptions of these methods can be found in Appendix~\ref{app:baselines}.
In this subsection, we focus on validating our proposed algorithm under the transductive setting, where the test nodes will participate in message passing~\cite{gilmer2017neural} during training following~\cite{good}. 

\noindent\textbf{Baselines.} We conduct experiments with 12 baselines from three categories: (i)~supervised methods, including empirical risk minimization~(\textbf{ERM})~\cite{erm}, invariant risk minimization (\textbf{IRM})~\cite{irm}, and a recent proposed graph OOD method \textbf{EERM}~\cite{eerm}; (ii)~self-supervised generative methods including Graph Autoencoder (\textbf{GAE})~\cite{gae}, Variational Graph Autoencoder (\textbf{VGAE})~\cite{gae}, Self-Supervised Masked Graph Autoencoders (\textbf{GraphMAE})~\cite{gmae}; (iii)~self-supervised contrastive methods including Deep Graph Infomax (\textbf{DGI})~\cite{dgi}, Contrastive Multi-View Representation Learning on Graphs (\textbf{MVGRL})~\cite{mvgrl}, Deep Graph Contrastive Representation Learning (\textbf{GRACE})~\cite{grace}, A Robust Self-Aligned Framework for Node-Node Graph Contrastive Learning (\textbf{RoSA})~\cite{rosa}, Bootstrapped Representation Learning on Graphs (\textbf{BGRL})~\cite{bgrl}, Covariance-Preserving Feature Augmentation for Graph Contrastive Learning (\textbf{COSTA})~\cite{costa}, Unsupervised Learning of Visual Features by Contrasting Cluster Assignments (\textbf{SwAV})~\cite{swav}. The detailed descriptions of these baselines can be found in Appendix~\ref{app:baselines}.

\noindent\textbf{Experimental setup.} We use the same graph encoder across different datasets for a fair comparison following~\cite{good}. We use grid search to find other hyper-parameters (\eg, learning rate, epochs) for different methods. For all experiments, we select the best checkpoints for ID and OOD tests according to results on ID and OOD validation sets following~\cite{good}, respectively. Experimental details and hyper-parameter selections are provided in Appendix~\ref{app:hyper}. For evaluating unsupervised methods, a linear classifier will be built on the frozen trained encoder after finishing pre-training. The reported results are the mean performance with standard deviation after 10 runs following~\cite{good}.

\noindent\textbf{Analysis.}\quad Based on the experimental results listed in Table~\ref{tab:trans_concept} and \ref{tab:trans_covariate}, we can draw the following conclusions: firstly, we find strong self-supervised methods (\eg, GRACE, BGRL, COSTA) are more robust to distribution shifts (concept shift in Table~\ref{tab:trans_concept} and covariate shift in Table~\ref{tab:trans_covariate}) compared to supervised methods. For instance, on GOOD-CBAS and GOOD-WebKB datasets, GRACE surpasses the best supervised method by large margins (over 6\% absolute improvement). Interestingly, we find the methods designed for OOD generalization (\ie, IRM) and graph OOD generalization (\ie, EERM) do not attain superior performance than the standard ERM on most of the datasets. For example, EERM shows superior OOD performance compared to ERM in only one experiment, and IRM outperforms ERM in four out of ten experiments across the conducted evaluations. This phenomenon is also observed in \cite{good,ahuja2020empirical,rosenfeld2021risks}, showcasing the challenge of achieving invariant prediction in non-Euclidean graph settings. 

Furthermore, our method surpasses other SOTA self-supervised methods on the OOD test set of all datasets by a considerable margin while achieving comparable performance in the in-distribution test set. For instance, on small datasets such as GOOD-CBAS and GOOD-WebKB, our method outperforms GRACE\footnote{MARIO is built up on GRACE according to our recipe. So, we make a comparison with GRACE here.} by over 2\% absolute accuracy on the OOD test set. On larger datasets such as GOOD-Cora and GOOD-Twitch, our method still outperforms other methods which shows its superiority. For instance, under covariate shift, MARIO surpasses other methods by over 7\% absolute accuracy on the GOOD-Twitch OOD test set. These statistics prove the effectiveness of our design.


\begin{table*}[htp]
\caption{Experimental results of all methods under concept shift. The bold font means the top-1 performance and the underline represents the second performance across the unsupervised methods. 'ID' represents in-distribution test performance and 'OOD' means out-of-distribution test performance. (OOM: out-of-memory on a GPU with 24GB memory)}
\label{tab:trans_concept}
\centering
\scalebox{0.95}{
\begin{tabular}{l|cc|cc|cc|cc|cc}
\toprule
\toprule
\multirow{3}{*}{concept shift} & \multicolumn{4}{c|}{GOOD-Cora}                   & \multicolumn{2}{c|}{GOOD-CBAS} & \multicolumn{2}{c|}{GOOD-Twitch} & \multicolumn{2}{c}{GOOD-WebKB} \\
                           & \multicolumn{2}{c}{word} & \multicolumn{2}{c|}{degree}& \multicolumn{2}{c|}{color}    & \multicolumn{2}{c|}{language}   & \multicolumn{2}{c}{university} \\
                           & ID         & OOD         & ID          & OOD          & ID            & OOD           & ID             & OOD            & ID            & OOD            \\
\midrule
ERM                        & 66.38±0.45 & 64.44±0.18  & 68.60±0.40  & 60.76±0.34   & 89.79±1.39    & 83.43±1.19    & 80.80±1.00     & 56.92±0.92     & 62.67±1.53    & 26.33±1.09     \\
IRM                        & 66.42±0.41 & 64.29±0.31  & 68.57±0.35  & 61.45±0.24   & 89.64±1.21    & 82.29±1.14    & 78.87±1.04     & 59.30±1.79     & 62.67±1.10    & 26.88±1.42     \\
EERM                       & 65.10±0.44 & 62.45±0.19  & 66.95±0.44  & 56.58±0.25   & 79.07±2.12    & 64.50±1.01    & OOM            & OOM            & 62.50±2.01    & 28.07±3.23      \\
\midrule
% Random-Init                & 37.53±1.74 & 32.12±1.24  & 37.82±1.71  & 27.74±1.14   &               &               &                &                & 60.33±2.21    & 27.07±1.70     \\
GAE                        & 60.65±0.89 & 58.00±0.55  & 62.59±1.11  & 53.44±0.80   & 75.28±1.36    & 68.07±2.05    & 81.25±0.81     & 51.51±1.05     & 62.17±3.34    & 25.78±1.85     \\
VGAE                       & 63.19±0.53 & 60.35±0.47  & 61.65±0.66  & 54.28±0.28   & 76.50±0.50    & 59.07±0.56    & 80.46±0.53     & 55.56±4.53     & 62.50±2.38    & 24.40±2.57     \\
GraphMAE                   & \underline{66.44±0.46} & \underline{64.87±0.30}  & 67.95±0.46  & 59.41±0.39   & 89.14±0.89    & 82.93±0.93    & 80.05±0.64     & 59.38±1.49     & 61.83±3.37    & 29.27±2.15     \\
DGI                        & 63.33±0.56 & 60.71±0.49  & 65.93±1.02  & 55.83±0.53   & 91.22±1.47    & 85.00±1.66    & 80.05±0.87     & 59.16±1.88     & 61.83±2.83    & 28.63±1.92      \\
MVGRL                      & OOM        & OOM         & OOM         & OOM          & 88.57±1.15    & 76.50±1.17    & OOM            & OOM            & 62.00±3.79    & 28.26±4.20     \\
GRACE                      & 65.61±0.61 & 63.92±0.44  & \textbf{68.59±0.35}  & 60.15±0.45   & 92.00±1.39    & 88.64±0.67    & \textbf{83.43±0.63}     & \underline{60.45±1.46}     & 64.00±3.43    & \underline{34.86±3.43}  \\
RoSA                       & 64.06±0.67 & 62.44±0.39  & 67.07±0.65  & 57.68±0.44   & 90.78±2.27    & 85.93±2.14    & 82.39±0.42     & 57.45±2.16     & 64.17±4.10    & 32.20±2.15     \\
BGRL                       & 65.18±0.43 & 63.43±0.45  & 66.83±0.80  & 59.63±0.38   & 92.36±1.16    & 87.14±1.60    & 82.52±0.60     & 55.48±1.48     & 63.67±2.33    & 31.47±3.43     \\
COSTA                      & 65.05±0.80 & 62.37±0.45  & 66.76±0.87  & 55.73±0.36   & \underline{93.50±2.62}    & \underline{89.29±3.11}    & 83.15±0.30 & 55.03±3.22     & 61.66±2.58    & 32.39±2.13 \\
% ArCL                       &            &             & 67.64±0.57  & 59.71±0.44   &               &               &                &                & 65.00±3.94    & 35.41±1.97 \\      
SwAV                       & 62.22±0.53 & 59.79±0.53  & 64.65±0.94  & 55.06±0.39   & 89.00±0.79    & 81.72±0.66    & \underline{83.32±0.15}     & 59.69±1.97     & \underline{65.17±3.76}    & 29.36±2.01    \\
\midrule
MARIO                       & \textbf{67.11±0.46} & \textbf{65.28±0.34}  & \underline{68.46±0.40}  & \textbf{61.30±0.28}   & \textbf{94.36±1.21}    & \textbf{91.28±1.10}    & 82.31±0.54     & \textbf{63.33±1.72}     & \textbf{65.67±2.81}    & \textbf{37.15±2.37}     \\
\bottomrule
\end{tabular}}
\end{table*}

\begin{table*}[htp]
\caption{Experimental results of all methods under covariate shift. The bold font means the top-1 performance and the underline represents the second performance across the unsupervised methods. 'ID' represents in-distribution test performance and 'OOD' means out-of-distribution test performance. (OOM: out-of-memory on a GPU with 24GB memory)}
\label{tab:trans_covariate}
\centering
\scalebox{0.95}{
\begin{tabular}{l|cc|cc|cc|cc|cc}
\toprule
\toprule
\multirow{3}{*}{covariate shift} & \multicolumn{4}{c|}{GOOD-Cora}                                   & \multicolumn{2}{c|}{GOOD-CBAS} & \multicolumn{2}{c|}{GOOD-Twitch} & \multicolumn{2}{c}{GOOD-WebKB} \\
                           & \multicolumn{2}{c}{word} & \multicolumn{2}{c|}{degree}& \multicolumn{2}{c|}{color}    & \multicolumn{2}{c|}{language}   & \multicolumn{2}{c}{university} \\
                           & ID         & OOD         & ID          & OOD          & ID            & OOD           & ID             & OOD            & ID            & OOD            \\
\midrule
ERM                        & 70.50±0.41 & 64.69±0.33  & 72.46±0.49  & 55.53±0.50   & 92.00±3.08    & 77.57±1.29    & 70.98±0.41     & 49.35±5.09     & 39.34±1.79    & 14.52±3.14   \\
IRM                        & 70.48±0.26 & 64.53±0.57  & 71.98±0.34  & 53.72±0.46   & 90.86±2.41    & 78.86±1.67    & 69.81±0.95     & 49.11±2.82     & 38.52±3.30    & 13.97±2.80     \\
EERM                       & OOM        & OOM         & OOM         & OOM          & 65.00±2.57    & 57.43±3.60    & OOM            & OOM            & 46.07±4.55    & 27.40±7.65     \\
\midrule
GAE                        & 56.63±0.79 & 48.93±0.93  & 66.30±0.88  & 34.01±0.87   & 73.00±2.16    & 60.86±3.01    & 67.24±1.23     & 47.65±2.49     & 45.08±6.32    & 28.02±6.29    \\
VGAE                       & 62.02±0.66 & 54.12±0.86  & 69.41±0.57  & 44.20±1.29   & 62.29±2.04    & 63.29±1.11    & 66.99±1.43     & \underline{50.48±4.58}     & 48.85±4.68    & 20.87±6.69     \\
GraphMAE                   & 68.14±0.43 & 64.00±0.33  & \textbf{73.36±0.56}  & 53.75±0.55   & 67.28±3.03    & 67.28±1.49    & 68.84±1.20     & 48.02±2.79     & 48.03±4.34    & 30.00±8.09     \\
DGI                        & 60.85±0.75 & 57.03±0.67  & 68.97±0.41  & 41.75±0.88   & 69.57±4.09    & 59.71±3.43    & 68.43±1.05     & 44.83±1.61     & 48.52±5.04    & 21.11±7.50     \\
MVGRL                      & OOM        & OOM         & OOM         & OOM          & 65.00±1.94    & 64.15±0.77    & OOM            & OOM           & \textbf{54.10±5.39}    & 16.59±6.51     \\
GRACE                      & \underline{68.77±0.33} & \underline{64.21±0.41}  & 72.69±0.34  & \underline{56.10±0.63}   & \underline{93.57±1.83}    & \underline{89.29±3.40}    & \underline{71.12±0.87} & 46.21±1.54 & 49.67±5.82    & 28.10±4.68    \\
RoSA                       & 68.19±0.56 & 62.48±0.61  & 71.04±0.62  & 52.72±0.79   & 84.71±4.14    &79.14±3.51     & 70.58±0.36     & 45.83±1.72     & 52.30±4.24    & \underline{34.24±7.92}     \\
BGRL                       & 67.23±0.43 & 61.33±0.36  & 72.11±0.39  & 49.15±0.73   & 89.00±2.56    & 79.86±3.29    & \textbf{71.43±0.53}     & 43.86±0.94     & 51.80±5.55    & 30.32±7.61    \\
COSTA                      & 65.28±0.60 & 60.33±0.53  & 70.65±0.62  & 54.03±0.28   & 92.29±1.59    & 82.71±2.74    & 69.29±1.37     & 49.07±2.13     & 50.49±3.01    & 29.84±4.75   \\
SwAV                       & 63.29±1.01 & 56.98±0.94  & 70.27±0.73  & 43.00±0.52   & 89.57±1.12    & 81.43±1.69    & 69.19±0.93     & 49.37±2.96     & 49.84±4.82    & 30.55±6.72   \\
\midrule
MARIO                       & \textbf{69.99±0.54} & \textbf{65.06±0.34}  & \underline{72.73±0.43}  & \textbf{57.73±0.45}  & \textbf{94.57±2.46}    & \textbf{91.00±2.48}     & 68.31±0.78 & \textbf{57.37±1.37}     & \underline{53.94±3.23}    & \textbf{35.24±4.98}   \\
\bottomrule
\end{tabular}}

\end{table*}

\subsubsection{Inductive Setting}
In this subsection, we conduct experiments under the inductive settings, where the test nodes are kept unseen during training. This setting is more suitable for domain generalization.
% But we think it is more convincing that conduct experiments under inductive settings which means test nodes are unseen during training. This setting is more appropriate for domain generalization.

\noindent\textbf{Baselines:} For GOOD-WebKB and GOOD-CBAS datasets, we adopt ERM, IRM, GraphMAE, and GRACE as our baselines. And for Amazon-Photo and Elliptic datasets, we select ERM, EERM, and GRACE as our baselines.

\noindent\textbf{Experimental setup:} For GOOD-WebKB and GOOD-CBAS datasets, we use the same model configuration in Section~\ref{sec:trans}.
% Besides, we add experiments on Amazon-Photo dataset~\cite{yang2016revisiting} and Elliptic~\cite{elliptic} dataset in this subsection. 
For Amazon-Photo dataset~\cite{yang2016revisiting} and Elliptic~\cite{elliptic} dataset, they consist of many snapshots (training data and testing data use different snapshots) which are naturally inductive. For Amazon-Photo dataset, we use 2-layer GCN~\cite{gcn} as the encoder and for elliptic dataset, we use 5-layer GraphSAGE~\cite{sage} as encoder following~\cite{eerm}.

% Figure environment removed

\noindent\textbf{Analysis:}
According to Figure~\ref{fig:amazon},\ref{fig:elliptic},\ref{fig:ind_con},\ref{fig:ind_cov}, we can draw following conclusions:
firstly, based on Figure~\ref{fig:amazon}, it is evident that our method outperforms other representative supervised and self-supervised methods on all test graphs (T1$\sim$T8). This superiority is reflected in the larger median value of our method compared to others. For instance, MARIO achieves over a 3\% absolute improvement compared to ERM in terms of the mean value of eight median values. Additionally, our method demonstrates higher stability across different random initializations, as indicated by the closer proximity of the first and third quartile values to the median value~(\eg, the difference of first and third quartile values of ERM, EERM, GRACE and MARIO are 4.2, 3.3, 6.7 and 1.0 on T8 respectively which indicates MARIO is much more stable than other methods). Furthermore, our method exhibits consistent performance across different graphs (\eg, The standard deviation of median values on T1$\sim$T8 for ERM, EERM, GRACE, and MARIO are 0.4, 1.1, 1.2, and 0.3, respectively.), indicating its robustness to environmental variations and its ability to extract invariant features: $g(G^e) \approx g(G^{e'})$ for all $e, e' \in \mathcal{E}^\text{train}$. In summary, our method showcases enhanced OOD generalization capabilities.
% $g(G^e)g(G^e^\prime)$ where $any e, e^\prime in \mathcal{E}^{train}$

Secondly, from the results presented in Figure~\ref{fig:elliptic}, we can observe that our method averagely harvests 10.9\% absolute improvement over GRACE and 12.5\% absolute improvement over EERM in terms of F1 scores on Elliptic dataset. This demonstrates the effectiveness of our method in handling distribution shifts and improving performance compared to existing approaches. It is worth noting that GRACE's performance worsens over time, indicating its inability to handle distribution shifts effectively. In contrast, our method consistently achieves better F1 scores, except for T9, which is caused by the dark market shutdown occurred after T7~\cite{elliptic}. The emergence of such an event introduces significant variations in data distributions, which subsequently results in performance degradation for all methods. Indeed, this event serves as an unpredictable external factor that introduces significant challenges for models trained on limited training data. The results indicate that the performance heavily depends on available training data. Nonetheless, our approach outperforms other methods even in such an extreme case. This highlights the effectiveness of our method in addressing distribution shifts and improving generalization performance.

Finally, based on the observations from Figure~\ref{fig:ind_con} and Figure~\ref{fig:ind_cov} MARIO demonstrates the best performances on both ID and OOD test sets for GOOD-WebKB and GOOD-CBAS datasets, under both concept shift and covariate shift. Notably, MARIO outperforms other methods by more than 3\% and 10\% absolute improvement on GOOD-WebKB and GOOD-CBAS, respectively, under covariate shift. We can draw similar conclusions as discussed in Section~\ref{sec:trans}. Even under the inductive setting, our method continues to demonstrate excellent OOD generalization capabilities and achieves comparable or even improved in-distribution test performance. These statistical results further validate the effectiveness of our method in handling distribution shifts and enhancing generalization performance.

Overall, the observations we have made provide strong evidence of the great capacity of our method for handling distribution shifts, validating its effectiveness and potential for real-world applications.



% Figure environment removed

% Figure environment removed


% Figure environment removed


\subsection{Ablation Studies}\label{sec:ablation}
\noindent Table~\ref{tab:aba} provides a detailed analysis of the effect of each component according to our proposed recipe for improving OOD generalization in graph contrastive learning. Let's examine the different variants of our method and their impact on performance.
Specifically, MARIO~(w/o ad) represents MARIO without  adversarial augmentation. MARIO~(w/o cmi) denotes we only maximize the mutual information between positive pairs without considering conditional mutual information. MARIO~(w/o cmi, ad) means a vanilla graph contrastive method that is similar to GRACE. 

From Table~\ref{tab:aba}, we can find MARIO~(w/o cmi) lags far behind MARIO on OOD test set which demonstrates appropriately minimizing the redundant information (\ie, conditional mutual information) is essential to improve OOD generalization of GCL methods. And adversarial augmentation can also boost OOD generalization because it can approximately serve as a supermum operator to learn more invariant features  discussed in Section~\ref{sec:aug}. Based on the analysis of these variants, it is evident that the proposed improvements on data augmentation and contrastive loss in the recipe are both effective in enhancing graph OOD generalization. Each component contributes to the overall performance improvement, and their combination leads to a stronger self-supervised graph learner in terms of graph OOD generalization. 

In short, the findings from Table~\ref{tab:aba} support the rationale behind your proposed recipe and provide empirical evidence of the effectiveness of each proposed component. By incorporating these enhancements, our method achieves superior performance in handling distribution shifts and improving graph OOD generalization in graph contrastive learning.
\begin{table*}[htp]
\caption{Ablation studies for MARIO by masking each component.}
\label{tab:aba}
\centering
\scalebox{0.9}{
\begin{tabular}{l|cc|cc|cc|cc|cc}
\toprule
\toprule
\multirow{3}{*}{concept shift} & \multicolumn{4}{c|}{GOOD-Cora}                       & \multicolumn{2}{c|}{GOOD-CBAS} & \multicolumn{2}{c|}{GOOD-Twitch} & \multicolumn{2}{c}{GOOD-WebKB} \\
                           & \multicolumn{2}{c}{word} & \multicolumn{2}{c|}{degree}& \multicolumn{2}{c|}{color}    & \multicolumn{2}{c|}{language}   & \multicolumn{2}{c}{university} \\
                           & ID         & OOD         & ID          & OOD          & ID            & OOD           & ID             & OOD            & ID            & OOD            \\
\midrule
MARIO                      & \textbf{67.11±0.46} & \textbf{65.28±0.34}  & \textbf{68.46±0.40}  & \textbf{61.30±0.28}      & \textbf{94.36±1.21}  & \textbf{91.28±1.10}    & 82.31±0.54     & \textbf{63.33±1.72}     & \textbf{65.67±2.81}    & \textbf{37.15±2.37}     \\
MARIO(w/o ad)              & 66.23±0.53 & 64.02±0.18  & 67.88±0.38  & 60.46±0.29   & 93.21±1.25    & 90.29±0.91    & 82.42±0.73     & 60.50±1.02     & 64.83±2.83    & 36.51±3.25    \\
MARIO(w/o cmi)             & 65.32±0.60 & 63.51±0.32  & 68.14±0.32  & 61.19±0.34   & 94.15±1.23    & 90.57±1.96    & \textbf{82.51±0.56}     & 61.41±2.63     & 64.50±4.35    & 35.78±2.53     \\
MARIO(w/o cmi, ad)         & 64.67±0.55 & 63.11±0.32  & 67.95±0.65  & 60.01±0.57   & 93.36±1.66    & 89.64±1.73    & 81.90±0.75     & 60.12±1.60     & 64.17±3.67    & 34.13±2.38     \\
\bottomrule
\end{tabular}}
\end{table*}
% & 65.32±0.60 & 63.51±0.32 exchange 64.67±0.55 & 63.11±0.32
% 68.14±0.32       id ood test: 60.95±0.43       ood ood test: 61.19±0.34


\subsection{Sensitivity Analysis}\label{sec:sensitivity}
\noindent In this subsection, we will analyze some important hyper-parameters of our method. We conduct sensitivity analysis on GOOD-WebKB dataset with concept shift, we chose two sensitive hyper-parameters (\ie, the coefficient $\gamma$ of condition mutual information in Equation~\ref{equ:cmi} and the number of prototypes $|C|$ in Equation~\ref{equ:pq}). The coefficient of CMI range in $[0.001, 0.01, 0.1, 0.5, 1]$ and the number of prototypes $|C|$ ranges in $[10, 50, 100, 200, 300]$. From Figure~\ref{fig:sensitivity}, we can observe that $\gamma$ reaches 0.1 and $|C|$ reaches 100 or 200 can achieve the best OOD test accuracy. Both higher and lower values of $\gamma$ result in suboptimal performance. This finding aligns with previous research such as DIB~\cite{dib}, indicating that an appropriate compression level is crucial for achieving optimal performance. Extremely high or low compression values are not ideal. 

Regarding the number of prototypes $|C|$, based on the results shown in Figure~\ref{fig:sensitivity}, it is found that setting $|C|=100$ leads to the best performance in terms of OOD test accuracy. This choice provides a moderate number of pseudo labels, which is beneficial for the learning process. 

Based on the sensitivity analysis, we determined that setting $\gamma=0.1$ and $|C|=100$ on most datasets. These hyperparameter values strike a balance between compression level and the number of prototypes, resulting in improved graph OOD generalization.
% Figure environment removed


\subsection{Integrated with Other Models}\label{sec:other_models}
% Figure environment removed

\begin{table}[htp]
\caption{Results of different learning approaches with different encoding models (\ie, GCN, GraphSAGE, GAT).}
\label{tab:others}
\centering
\scalebox{0.9}{
\begin{tabular}{cc|cc|cc}
\toprule
\toprule
\multirow{3}{*}{Model}& \multirow{3}{*}{Method} & \multicolumn{2}{c|}{GOOD-CBAS} & \multicolumn{2}{c}{GOOD-WebKB} \\
                & & \multicolumn{2}{c|}{color}    & \multicolumn{2}{c}{university} \\
                &   & ID          & OOD         & ID          & OOD            \\
\midrule
\multirow{3}{*}{GCN} 
&ERM               & 89.79±1.39 & 83.43±1.19  &  62.67±1.53 & 26.33±1.09         \\
&GRACE             & 92.00±1.39 & 88.64±0.67  &  64.00±3.43 & 34.86±3.43        \\
&MARIO             & 94.36±1.21 & 91.28±1.10  &  65.67±2.81 & 37.15±2.37        \\ \bottomrule
\multirow{3}{*}{SAGE} 
&ERM               & 95.07±1.51 & 75.14±1.19  & 73.67±2.08  & 46.33±3.42       \\
&GRACE             & 95.29±1.11 & 74.43±2.36  & 70.50±5.06  & 49.54±3.83        \\
&MARIO             & 96.00±1.07 & 76.29±3.01  & 71.00±3.82  & 51.74±4.63        \\ \bottomrule
\multirow{3}{*}{GAT} 
&ERM               & 78.64±3.63 & 72.93±2.64  & 61.33±3.71  & 28.99±2.63        \\
&GRACE             & 84.57±1.79 & 78.36±1.60  & 59.50±2.36  & 35.78±3.26        \\
&MARIO             & 84.93±1.95 & 80.43±1.89  & 62.17±4.78  & 38.17±3.10        \\
\bottomrule
\end{tabular}}
\end{table}



\noindent In the subsection, we demonstrate the model-agnostic nature of the recipe by integrating it with various graph neural network (GNN) models, including GCN, GraphSAGE, and GAT.

From Table~\ref{tab:others}, it can be observed that regardless of the specific GNN model used as the encoder, our method consistently achieves the best performance on the OOD test set. This indicates the effectiveness and robustness of our method across different GNN models.
By achieving superior performance across different GNN models, MARIO demonstrates its versatility and ability to improve the OOD generalization of various graph neural models. This highlights the broad applicability and effectiveness of our recipe in enhancing the performance of different GNN encoders.

Furthermore, we integrate our recipe with other GCL methods in Appendix~\ref{app:other_methods}. The results demonstrate our recipe can boost the OOD generalization ability of various GCL methods which means our recipe can serve as a plug-in for many current classical GCL methods.

% Figure environment removed

\subsection{Visualization}\label{sec:vis}
\subsubsection{Metric Score Curves}
We present metric score curves for ERM and MARIO, including training, ID validation, ID testing, OOD validation, and OOD testing accuracy, in Figure~\ref{fig:curve2}. Notably, MARIO demonstrates superior convergence with approximately 10\% absolute improvement on the OOD test set compared to ERM. Furthermore, MARIO effectively narrows the performance gap between in-distribution and out-of-distribution performance, showcasing its efficacy in enhancing OOD generalization for graph data. More metric score curves can be found in Appendix~\ref{app:curves}.


\subsubsection{Feature Visualization}
In order to assess the quality of learned embeddings, we adopt t-SNE~\cite{tsne} to visualize the node embedding on GOOD-Cora dataset (concept shift in word domain) using random-init of GCN, EERM, GRACE, and MARIO, where different classes have different colors in Figure~\ref{fig:vis}. For clarity, we select eight classes with the largest number of nodes to enhance the informativeness and interpretability of the visualization. We can observe that the 2D projection of node embeddings learned by MARIO has a better separation of clusters, which indicates the model can help learn representative features for downstream tasks. It has to note that we depict both ID nodes and OOD nodes in the same figure. 

Besides, we also separately visualize ID nodes and OOD nodes in the different figures in the Appendix~\ref{app:feature}. And we can find MARIO performs a clearer separation of clusters whether on ID nodes or OOD nodes compared to other methods.



\section{Experimental Results}\label{sec:results}
    \subsection{General Results}
        The basic ResSAN model is used to determine reference results which our expanded model can be compared to as it is structurally similar to ResLAN but does not possess the Lidar adaptive components of it. Further, we compare with the full-size PackNet-SAN and the unmodified NLSPN architecture. 
        As it can be seen from Tab.\,\ref{tab:sota-results}, our LiDAR-adaptive ResLAN achieves competitive performance compared to state-of-the-art standard depth completion methods, which are specialized to the unfiltered 64-beam-LiDAR. The performance differences are in the range of a few centimetres in terms of MAE, which is acceptable given the practical advantage that ResLAN can generalize to different beam patterns as will be shown below.

        Furthermore, we compared the architectures for a set of three different input types that contained 64, 32 or 16 LiDAR channels using both filter types on the metrics from the KITTI benchmark. The NLSPN model was trained for the standard depth completion task and then evaluated with different input data. As for the ResSAN models, we trained one model for each input type and tested it for the corresponding one which serve serve as the \emph{Baseline} in Tab.\,\ref{tab:overall-results}. Our ResLAN model was jointly trained for all three settings. As listed in Tab.\,\ref{tab:overall-results}, the ResLAN models outperform the challenging baseline in all metrics for FOV filtering and all but one for sparse filtering. This implies that our LiDAR adaptive model is able to outperform dedicated models in case of very sparse input depth. Fig.\,\ref{fig:comp-plot} shows this is indeed the case for 32 and even more for 16 channels. For FOV-filtered inputs with 16 channels, the ResLAN exhibits approx. $10\%$ smaller MAE than the baseline. As for the NLSPN, it becomes apparent that it is not capable of generalizing to other input types since it shows clearly worse results. The difference is especially pronounced for the FOV filtering where on average more than every fourth predicted pixel is more than $25 \%$ deviating from the ground truth\,($\delta_{1.25}$). Therefore, using a weight-adapting network in combination with differently filtered input depths allows us to train models that outperform their non-adaptive counterparts.

        \begin{table}[]
            \centering
    	    \small
            \vspace{0.4cm}
            \caption{\textbf{Depth estimation result for standard depth completion} when the ResSAN model was only trained for 64 channels and the ResLAN model for multiple tasks. The PackNet-SAN and NLSPN models were trained with the setup that was also used for our model architecture.}
            \footnotesize
            \setlength{\tabcolsep}{5pt}
            \begin{tabular}{@{}lrrrrl@{}}
            \toprule
            \multicolumn{6}{c}{\textbf{Standard LiDAR Depth Completion}}                                                                                                                         \\ \midrule
            \multicolumn{1}{l|}{Method}          & RMSE $\downarrow$            & MAE  $\downarrow$            & iRMSE $\downarrow$             & iMAE $\downarrow$ & $\delta_{1.25}$ $\uparrow$ \\
            \multicolumn{1}{l|}{}                & \multicolumn{1}{l}{{[}mm{]}} & \multicolumn{1}{l}{{[}mm{]}} & \multicolumn{1}{l}{{[}1/km{]}} & {[}1/km{]}        &                            \\ \midrule
            \multicolumn{1}{l|}{PackNet-SAN}     &  914                            &  298                            &  2.78                              &  1.4                 &  99.65 \%                          \\
            \multicolumn{1}{l|}{NLSPN}           &  \textbf{889}                            &   \textbf{263}                           &  \textbf{2.62}                              &   \textbf{1.3}                &   \textbf{99.61} \%                         \\ \midrule
            \multicolumn{1}{l|}{ResSAN (Ours)}   & 948                             &  275                            &  2.75                              &    1.4               &   99.58 \%                         \\
            \multicolumn{1}{l|}{ResLAN (Ours)} &   969                           &  283                            &   2.83                             &   1.4                &  99.56 \%                          \\ \bottomrule
            \end{tabular}
            \vspace{0.2cm}
            \label{tab:sota-results}
        \end{table}

        \begin{table}[]
    	    \centering
    	    \small
    	    \caption{\textbf{Depth estimation results of the two baseline setups and the explicit and implicit ResSAN} when evaluated on a combination of 16, 32 and 64 channel depth inputs. Please note that Specialist Methods need to train three specialized networks, one for each of the three types of inputs while our method only uses one network.}
            \footnotesize
            \setlength{\tabcolsep}{4.8pt}
            \begin{tabular}{@{}lrrrrl@{}}
                \toprule
                \multicolumn{6}{c}{\textbf{Sparse Channel Filter}}                                                                                                                                  \\ \midrule
                \multicolumn{1}{l|}{Method}        & RMSE $\downarrow$            & MAE  $\downarrow$            & iRMSE $\downarrow$             & iMAE $\downarrow$ & $\delta_{1.25}$ $\uparrow$  \\
                \multicolumn{1}{l|}{}              & \multicolumn{1}{l}{{[}mm{]}} & \multicolumn{1}{l}{{[}mm{]}} & \multicolumn{1}{l}{{[}1/km{]}} & {[}1/km{]}        &                             \\ \midrule
                \multicolumn{1}{l|}{NLSPN}         &  1396                            &  437                            & 5.54                               &  2.2                 &  98.82 \%                           \\
                \multicolumn{1}{l|}{Baseline}      & \textbf{1207}                             &  381                            & 4.41                               &  1.8                 &  \textbf{99.37} \%                           \\
                \multicolumn{1}{l|}{ResLAN (Ours)} &  1215                            &  \textbf{378}                            &  \textbf{4.27}                              &  \textbf{1.7}                 &  99.31 \%                           \\ \toprule
                \multicolumn{6}{c}{\textbf{Field-of-View Filter}}                                                                                                                                   \\ \midrule
                \multicolumn{1}{l|}{Method}        & RMSE $\downarrow$            & MAE  $\downarrow$            & iRMSE $\downarrow$             & iMAE $\downarrow$ & $\delta_{1.25}$ $\uparrow$ \\
                \multicolumn{1}{l|}{}              & \multicolumn{1}{l}{{[}mm{]}} & \multicolumn{1}{l}{{[}mm{]}} & \multicolumn{1}{l}{{[}1/km{]}} & {[}1/km{]}        &                             \\ \midrule
                \multicolumn{1}{l|}{NLSPN}         &  2738                            &  1702                            & 12.3                              &  4.3                 &  74.69 \%                           \\
                \multicolumn{1}{l|}{Baseline}      &  1556                            &  525                            &  6.8                              &  3.0                 & 98.14 \%                            \\
                \multicolumn{1}{l|}{ResLAN (Ours)} &  \textbf{1548}                            &  \textbf{519}                            &  \textbf{6.44}                              &  \textbf{2.8}                 & \textbf{98.52 \%}                            \\ \bottomrule
            \end{tabular}
            \label{tab:overall-results}
        \end{table}

        
        
        % Figure environment removed
        
        % Figure environment removed

    \subsection{Filter Effects}
        Comparing the effect of the two different types of depth input filters on the model performance, it becomes apparent that FOV filtering is the more challenging task. In that setting, reducing LiDAR channels is more detrimental to the performance than sparse filtering as it creates regions where no depth information is available. Effectively, the model is forced to perform depth prediction in these regions. These effects are highlighted in the depth images in Fig.\,\ref{fig:dense-maps} where the effect of a 16-channel sparse depth filter and a 16-channel FOV can be compared.

    \subsection{Generalization Capabilities}
        We trained three models for both filter types eaach, so the combinations and number of filtered depth inputs they receive are different. This serves the purpose of testing the generalization capabilities of the ResLAN architecture as well as the robustness to different filter settings. After training, the models were evaluated for the depth input settings they were trained for, as well as for ones they weren't exposed to. Overall, ResLAN shows good generalization capabilities. As one can gather from Fig.\,\ref{fig:explicit-comp} and Fig.\,\ref{fig:implicit-comp}, the consequences of slightly varying sets of input depth settings are limited. The most considerable deviations can be seen when the model is tasked to extrapolate. For instance, the model $\{64, 32, 16\}$ shows a noticeably higher MAE for eight-channel depth inputs than the model that was trained for it. Similar behaviour can be seen for the FOV filtering case as well for the model $\{64, 48, 32\}$ when tasked to generalize for a 16-channel input. There is no such pronounced effect for generalization tasks that lie between two filter settings the model was trained for. At most, it can be observed that models that were trained for a smaller range of filter values perform slightly better than ones that have to cover a wider range. The number of filter settings used in a fixed range does not relevantly influence the model performance, as can be seen, when comparing the two models in Fig.\,\ref{fig:implicit-comp}, which are both trained for a range of 64 to 32 channels but one with three filter settings and the other one with five.
    
    % Figure environment removed
    
    
    % Figure environment removed
% \section{Visualization for Plan Validity}
Our system's PDDL encoded RC solver, in combination with a Visualizer, generates plan actions to solve the problem using the Fast-Downward (FD) AI planner. We have used publicly available Tkinter based Visualizer code\footnote{\url{https://code-projects.org/rubiks-cube-in-python-with-source-code/}} for plan visualization. A random scrambled state of RC represented in the Visualizer is shown in the Figure \ref{FIG: shuffled-cube}. AI planners are controllable in generating the desired plans. We can specify the search algorithm and the heuristics to the Fast-Downward planner. Each different search algorithm generates different plans to reach the goal state. We employed A\textsuperscript{*} search algorithm in combination with landmark-count heuristics which supports conditional-effects in our system and gives plans in minutes.

The system architecture is shown in Figure \ref{FIG: system}. For diverse types of audiences, our solution offers two main features:
% \noindent (a) 
% \begin{enumerate}[label=(\alph*)]
%     \item 
    (a) By giving the Fast-Downward planner path together with the problem file holding the cube's initial scrambled state, the system may be used to solve any RC problem. It creates an actions plan file as a solution for the problem file given and then launches the Visualizer. The user may visually follow the actions in the plan file generated to observe the RC being solved step by step.
    %\item 
    %\noindent 
    (b) The visualizer may also be used by a user with their own problem file and actions plan file to see the RC being solved step by step from the actions in the plan file. This will assist the user in comprehending the many states being explored in order to arrive at the solved goal state from the initial state set by the user.
%&\end{enumerate}

% % Figure environment removed

% % Figure environment removed
% \vspace{-0.1in}
\section{Conclusion and Future Work}
In this work, I design corruption-robust algorithms for the Lipschitz contextual search problem. I present the \emph{agnostic checking} technique and demonstrate its effectiveness in designing corruption-robust algorithms. There are several open problems for future research. First, in the algorithm I propose for pricing loss, the schedule for agnostic checks is fixed upfront. Can the learner design an adaptive checking schedule for the pricing loss? Second, this work assumes the learner has knowledge of the Lipschitz constant $L$. Can the learner design efficient no-regret algorithms without knowledge of $L$? 


% -----



% --------------
%\section{ Acknowledgments}
%We want to thank ...

\bibliography{references/references}


% --------------
\end{document}
