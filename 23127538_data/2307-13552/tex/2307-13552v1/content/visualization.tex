\section{Visualization for Plan Validity}
Our system's PDDL encoded RC solver, in combination with a Visualizer, generates plan actions to solve the problem using the Fast-Downward (FD) AI planner. We have used publicly available Tkinter based Visualizer code\footnote{\url{https://code-projects.org/rubiks-cube-in-python-with-source-code/}} for plan visualization. A random scrambled state of RC represented in the Visualizer is shown in the Figure \ref{FIG: shuffled-cube}. AI planners are controllable in generating the desired plans. We can specify the search algorithm and the heuristics to the Fast-Downward planner. Each different search algorithm generates different plans to reach the goal state. We employed A\textsuperscript{*} search algorithm in combination with landmark-count heuristics which supports conditional-effects in our system and gives plans in minutes.

The system architecture is shown in Figure \ref{FIG: system}. For diverse types of audiences, our solution offers two main features:
% \noindent (a) 
% \begin{enumerate}[label=(\alph*)]
%     \item 
    (a) By giving the Fast-Downward planner path together with the problem file holding the cube's initial scrambled state, the system may be used to solve any RC problem. It creates an actions plan file as a solution for the problem file given and then launches the Visualizer. The user may visually follow the actions in the plan file generated to observe the RC being solved step by step.
    %\item 
    %\noindent 
    (b) The visualizer may also be used by a user with their own problem file and actions plan file to see the RC being solved step by step from the actions in the plan file. This will assist the user in comprehending the many states being explored in order to arrive at the solved goal state from the initial state set by the user.
%&\end{enumerate}

% % Figure environment removed

% % Figure environment removed
% \vspace{-0.1in}