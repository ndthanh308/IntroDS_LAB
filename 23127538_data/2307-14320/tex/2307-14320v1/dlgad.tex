%%%%%%%%%%%%%%%%%%%%%%%%%%%%%%%%%%%%%%%%%%%%%%%%%%%%%%%%%%%%%%%%%%%%%
%%                                                                 %%
%% Please do not use \input{...} to include other tex files.       %%
%% Submit your LaTeX manuscript as one .tex document.              %%
%%                                                                 %%
%% All additional figures and files should be attached             %%
%% separately and not embedded in the \TeX\ document itself.       %%
%%                                                                 %%
%%%%%%%%%%%%%%%%%%%%%%%%%%%%%%%%%%%%%%%%%%%%%%%%%%%%%%%%%%%%%%%%%%%%%

%%\documentclass[referee,sn-basic]{sn-jnl}% referee option is meant for double line spacing

%%=======================================================%%
%% to print line numbers in the margin use lineno option %%
%%=======================================================%%

%%\documentclass[lineno,sn-basic]{sn-jnl}% Basic Springer Nature Reference Style/Chemistry Reference Style

%%======================================================%%
%% to compile with pdflatex/xelatex use pdflatex option %%
%%======================================================%%

%%\documentclass[pdflatex,sn-basic]{sn-jnl}% Basic Springer Nature Reference Style/Chemistry Reference Style

%%\documentclass[sn-basic]{sn-jnl}% Basic Springer Nature Reference Style/Chemistry Reference Style
\documentclass[pdflatex,sn-mathphys]{sn-jnl}% Math and Physical Sciences Reference Style
%%\documentclass[sn-aps]{sn-jnl}% American Physical Society (APS) Reference Style
%%\documentclass[sn-vancouver]{sn-jnl}% Vancouver Reference Style
%%\documentclass[sn-apa]{sn-jnl}% APA Reference Style
%%\documentclass[sn-chicago]{sn-jnl}% Chicago-based Humanities Reference Style
%%\documentclass[sn-standardnature]{sn-jnl}% Standard Nature Portfolio Reference Style
%%\documentclass[default]{sn-jnl}% Default
%%\documentclass[default,iicol]{sn-jnl}% Default with double column layout

%%%% Standard Packages
%%<additional latex packages if required can be included here>
%%%%

%%%%%=============================================================================%%%%
%%%%  Remarks: This template is provided to aid authors with the preparation
%%%%  of original research articles intended for submission to journals published 
%%%%  by Springer Nature. The guidance has been prepared in partnership with 
%%%%  production teams to conform to Springer Nature technical requirements. 
%%%%  Editorial and presentation requirements differ among journal portfolios and 
%%%%  research disciplines. You may find sections in this template are irrelevant 
%%%%  to your work and are empowered to omit any such section if allowed by the 
%%%%  journal you intend to submit to. The submission guidelines and policies 
%%%%  of the journal take precedence. A detailed User Manual is available in the 
%%%%  template package for technical guidance.
%%%%%=============================================================================%%%%

\jyear{2022}%

%% as per the requirement new theorem styles can be included as shown below
\theoremstyle{thmstyleone}%
\newtheorem{theorem}{Theorem}%  meant for continuous numbers
%%\newtheorem{theorem}{Theorem}[section]% meant for sectionwise numbers
%% optional argument [theorem] produces theorem numbering sequence instead of independent numbers for Proposition
\newtheorem{proposition}[theorem]{Proposition}% 
%%\newtheorem{proposition}{Proposition}% to get separate numbers for theorem and proposition etc.

\theoremstyle{thmstyletwo}%
\newtheorem{example}{Example}%
\newtheorem{remark}{Remark}%

\theoremstyle{thmstylethree}%
\newtheorem{definition}{Definition}%

\raggedbottom
%%\unnumbered% uncomment this for unnumbered level heads
\usepackage{subfigure}
\usepackage{comment}
\usepackage{arydshln} % for dashed lines

\begin{document}

\title[The double-LGAD]{A new Low Gain Avalanche Diode concept: the double-LGAD}

%%=============================================================%%
%% Prefix	-> \pfx{Dr}
%% GivenName	-> \fnm{Joergen W.}
%% Particle	-> \spfx{van der} -> surname prefix
%% FamilyName	-> \sur{Ploeg}
%% Suffix	-> \sfx{IV}
%% NatureName	-> \tanm{Poet Laureate} -> Title after name
%% Degrees	-> \dgr{MSc, PhD}
%% \author*[1,2]{\pfx{Dr} \fnm{Joergen W.} \spfx{van der} \sur{Ploeg} \sfx{IV} \tanm{Poet Laureate} 
%%                 \dgr{MSc, PhD}}\email{iauthor@gmail.com}
%%=============================================================%%
\author*[1]{\fnm{F.} \sur{Carnesecchi}}\email{francesca.carnesecchi@cern.ch}
\author*[2,5]{\fnm{S.} \sur{Strazzi}}\email{sofia.strazzi2@unibo.it}
\author[2,5]{\fnm{A.} \sur{Alici}}
\author[3,4]{\fnm{R.} \sur{Arcidiacono}}
\author[3]{\fnm{N.} \sur{Cartiglia}}
\author[5]{\fnm{D.} \sur{Cavazza}}
\author[7]{\fnm{S.} \sur{Durando}}
\author[3]{\fnm{M.} \sur{Ferrero}}
\author[5]{\fnm{A.} \sur{Margotti}}
\author[3,6]{\fnm{L.} \sur{Menzio}}
\author[5]{\fnm{R.} \sur{Nania}}
\author[2,5]{\fnm{B.} \sur{Sabiu}}
\author[2,5]{\fnm{G.} \sur{Scioli}}
\author[3]{\fnm{F.} \sur{Siviero}}
\author[3,6]{\fnm{V.} \sur{Sola}}
\author[5]{\fnm{G.} \sur{Vignola}}
%\note{Now at Deutsches Elektronen-Synchrotron DESY, Hamburg, Germany}

\affil[1]{\orgname{CERN}, \orgaddress{\city{Geneva}, \country{Switzerland}}}
\affil[2]{\orgdiv{Dipartimento Fisica e Astronomia dell’Università}, \orgaddress{\city{Bologna}, \country{Italy}}}
\affil[3]{\orgname{INFN}, \orgaddress{\city{Torino}, \country{Italy}}}
\affil[4]{\orgdiv{Università del Piemonte Orientale}, \orgaddress{\city{Novara}, \country{Italy}}}
\affil[5]{\orgname{INFN}, \orgaddress{\city{Bologna}, \country{Italy}}}
\affil[6]{\orgdiv{Università degli Studi di Torino}, \orgaddress{\city{Torino}, \country{Italy}}}
\affil[7]{\orgdiv{Dipartimento di elettronica e telecomunicazioni}, \orgname{Politecnico di Torino}, \city{Torino}, \country{Italy}}
%\author[2,3]{\fnm{Second} \sur{Author}}\email{iiauthor@gmail.com}
%\equalcont{These authors contributed equally to this work.}
%\author[1,2]{\fnm{Third} \sur{Author}}\email{iiiauthor@gmail.com}
%\equalcont{These authors contributed equally to this work.}

%%==================================%%
%% sample for unstructured abstract %%
%%==================================%%

\abstract{This paper describes the new concept of the double-LGAD. The goal is to increase the charge at the input of the electronics, keeping a time resolution equal or better than a standard (single) LGAD; this has been realized by adding the charges of two coupled LGADs while still using a single front-end electronics. %to improve the time resolution in respect to a standard (single) LGAD by adding the charges of two coupled LGADs while still using a single front-end electronics. 
The study here reported has been done starting from single LGAD with a thickness of  25\,\textmu{m}, 35\,\textmu{m} and 50\,\textmu{m}.}

\keywords{LGAD, UFSD, Timing, TOF}

%%\pacs[JEL Classification]{D8, H51}

%%\pacs[MSC Classification]{35A01, 65L10, 65L12, 65L20, 65L70}

\maketitle

\section{Introduction}\label{sec:intro}
Low Gain Avalanche Detectors (LGADs)\cite{2014PELLEGRINI}, also known as Ultra Fast Silicon Detectors (UFSDs) \cite{2017Sadrozinski}, are $n$-on-$p$ diodes with an additional highly doped $p^+$-type layer (gain layer) underneath the $n$-contact, which is responsible for the charge multiplication mechanism (in reverse bias regime).
It has been proven that LGADs with a thickness of 35\,\textmu{m} combined with a gain G$\sim30$ can provide a time resolution around 22\,ps \cite{2023Carnesecchi}. \\
Thanks to the excellent timing performance, this technology is already envisioned or proposed for several detector upgrades and applications \cite{atlas,cms,alice3}.

It has already been demonstrated that the time resolution of LGADs improves with thinner designs. Nevertheless, a thinner design also implies a smaller charge at the input of the amplifier which, because of the worse S/N, might worsen the performance of the amplifier and/or its power consumption. This need gave rise to the idea of the double-LGAD.

\section{The double-LGAD} 
\label{sec:d-LGAD}
The concept of the double-LGAD (d-LGAD) is inspired by the Multigap Resistive Plate Chambers (MRPC) \cite{MRPC}. Essentially, the idea is to sum up the signal taken from a double-layer of LGADs, still using an unique front-end amplifier.\\ 
As a first step, the signals from two different LGADs have been summed using a specific PCB design (more details in Sec.\ref{sec:set_el}) and the output has been sent  to a single and
common amplifier.
In Figure \ref{fig:concept} a schematic of a d-LGAD is reported. This is currently just a proof of concept, but the natural next step would be a better integration of such a concept either in the board containing the electronics or in the detector itself (in a truly d-LGAD or e.g. using Through-Silicon Via, TSV, technique). \\
%%%%%%%%%
 % Figure environment removed
%%%%%%%%%
In the proposed scheme, given by the sum of two LGADs each with a certain thickness t, the charge of the d-LGAD, is expected to be double if compared with a single LGAD of same thickness t. 
The time resolution of such a d-LGAD is expected to be largely better than that of an equivalent single LGAD of thickness 2t. Similarly to MRPC \cite{MRPC_sim}, the time resolution of a d-LGAD is foreseen to improve also w.r.t. to a single LGAD of thickness t; however, due to different signal amplitudes in the two d-LGADs,  this improvement is not by a factor of $\sqrt{2}$.  As stated in \cite{MRPC_sim},  the time resolution will be dominated by the d-LGAD with the largest signal. In other words, in d-LGAD, the LGAD with the largest signal always dominates the time resolution. 

%At the same time, the time resolution  of such d-LGAD is expected to be comparable to the one of a single LGAD of thickness t, so largely better than an equivalent single LGAD of thickness 2t. 
%Not only, similarly to MRPC \cite{MRPC_sim}, the time resolution of a d-LGAD is actually foreseen to slightly improve w.r.t. a single LGAD of thickness t. 
%However, the improvement is not expected to scale with a factor 1/$\sqrt{\text{n}}$, where n is the number of LGAD layers, which in our case is 2, but it is estimated to be less significant. Indeed, as reported in \cite{MRPC_sim}, for each signal the time resolution will always be dominated by the LGAD, among the two, which detected the largest signal. In other words, the fastest of the two LGAD layers always dominates the time resolution.

\section{Experimental setup} 
\label{sec:setup}

\subsection{Detectors} 
\label{sec:det}

\label{sec:detectors}
The tested LGADs came from two manufacturers, Hamamatsu Photonics K.K. (HPK, Japan) and Fondazione Bruno Kessler (FBK, Italy) and have a different area (A). The sensors from HPK have a nominal thickness (T)\footnote{Usually the active thickness is around 2-3 \textmu{m} less than the nominal. } of 50 \textmu{m}, appertaining to a very uniform wafer.
The FBK LGADs\footnote{This UFSD production is called EXFLU0~\cite{exflu}.} have a nominal thickness of 25 \textmu{m} and 35 \textmu{m}, respectively. More details about the FBK LGADs characteristics and performances can be found in \cite{2023Carnesecchi}. \\
%%%%%%%%%%%%%%%%%
\begin{table}[ht]
\begin{center}
\begin{minipage}{\textwidth}
\caption{Characteristics of the Front (F) and Back (B) LGADs of each couple under test.} \label{tab:lgad_char}
\begin{tabular*}{\textwidth}{@{\extracolsep{\fill}}lcccccc@{\extracolsep{\fill}}}
\toprule
 & A (mm$^2$)  & T (\textmu{m}) & V$_{bd}$ (V) & V$_\text{applied} (V)$  & Gain\\
\midrule
FBK25-F & 1 $\times$ 1 & 25 & 132 $\pm$ 1  & 80-120   & 11-24\\
FBK25-B & 1 $\times$ 1 & 25 & 124 $\pm$ 1 & 80-120   & 12-43\\
FBK35-F & 1 $\times$ 1  & 35  & 266.5 $\pm$ 0.5  & 180-240   & 9-17\\
FBK35-B & 1 $\times$ 1  & 35 & 268 $\pm$ 1  & 190-240   & 11-27\\
HPK50-F & 1.3 $\times$ 1.3 & 50  & 224.6  $\pm$ 0.2 &  170-220 & 24-63\\
HPK50-B & 1.3 $\times$ 1.3 & 50 & 237.4  $\pm$ 0.2 &  170-220 & 25-64\\
\botrule
\end{tabular*}
\end{minipage}
\end{center}
\end{table}
%%%%%%%%%%%%%%%%%
All the tested detectors have been previously completely characterized at the INFN Bologna laboratories. The method to measure the breakdown voltage (V$_{bd}$) gain are explained in \cite{2023Carnesecchi}.
The main characteristics of the sensors are reported in Table \ref{tab:lgad_char}. 


\subsection{Beam test setup and electronics}
\label{sec:set_el}
The time resolution of the UFSDs has been studied at the T10 beamline at PS-CERN in July and November 2022. The beam was mainly composed of protons and pions with a momentum of +10 GeV/c.
For each data acquisition up to 4 carrier boards mounted on micro-mover stages were aligned to the beam in a telescope frame at a relative distance of 24 cm, and the whole setup was enclosed in a dark environment box at room temperature.

All the LGADs tested have been mounted on a board V1.4-SCIPP-08/18, containing a wide bandwidth (2 GHz) and low noise inverting amplifier with a measured amplification of factor 6. The board has been modified in order to place one LGAD on each side of the board and, thanks to a TSV in the PCB itself, connect the output of the two LGADs together and send the signal to the amplifier described above, realizing a first prototype of d-LGAD.
The output of the board was followed by a second amplification stage, with a gain factor of around 13 and 14 respectively for the HPK50 and FBK sensors \footnote{The second amplifier used for the HPK50 and for the FBK sensors were the minicircuit LEE39+ (\href{https://www.minicircuits.com/pdfs/LEE-39+.pdf}{LEE39+ datasheet}) and Gali52+ (\href{https://www.minicircuits.com/pdfs/GALI-52+.pdf}{GALI52+ datasheet}) respectively.}.

Up to 4 amplified signals were sent to a LeCroy WaveRunner 9404M-MS oscilloscope\footnote{\href{https://teledynelecroy.com/oscilloscope/waverunner-9000-oscilloscopes/waverunner-9404m-ms
}{Lecroy WaveRunner datasheet}}, with 20 Gs/s sampling rate,  4 GHz of analogue bandwidth and 8-bit vertical resolution. The contribution of the oscilloscope time resolution to the measured one was negligible.

For the trigger of the data acquisition, a threshold has been set for each channel and the coincidence of the four sensors has been used.

%%%%%%%%%%%%%%%%%%%%%%%
% Figure environment removed
%%%%%%%%%%%%%%%%%%%%%%%

For all the measurements reported in this paper, the double LGAD has always been compared with the performances of the single LGADs composing the d-LGAD under test\footnote{First we tested always the d-LGAD. Then we un-bonded the bottom LGAD in order to test the top one. Lastly, we un-bonded the top LGAD, bonding and testing the bottom one}. Therefore, every LGAD thickness will always have three different measurements: two coming from each of the LGAD composing the d-LGAD, and one from the d-LGAD itself.

The RMS of the noise (see \cite{2023Carnesecchi} for more details) and the Signal to Noise ratio (S/N) have been evaluated for each sensor and voltage. In Figure \ref{fig:rms} they are reported as a function of the applied voltage.
As can be seen, the noise between single and double sensors is compatible for all thicknesses. The S/N instead is always higher for the d-LGAD, giving already some insight into the better performances reported later in the paper.\\

\section{Results} 
\label{sec:results}
The data analysis was performed following similar procedures to those reported in \cite{2019Carne, 2023Carnesecchi}. In particular, thanks to the oscilloscope readout, the full signal waveforms were recorded and analyzed. It was then possible to use the Constant Fraction Discriminator (CFD) method to extract the DUTs time resolutions.
Moreover, to filter out the high-frequency noise, a smoothing of the LGAD signal was applied with a four-point moving average.

To extract the time resolution of a single DUT (single or double-LGAD), a system with  three sensors and three differences between the arrival time of each pair of detectors has been considered. The sigma extracted from the fit has then been used to obtain the final time resolution of the three LGADs at a given voltage and CFD.\\

In Figure \ref{fig:landau} the measured charge distributions are shown. Notice that, the comparisons between single and the double LGADs has been done at a fixed gain.
As can be observed the d-LGAD always showed an MPV of charge which was double that of the single one, as naively expected, demonstrating the success of a so-built detector and, as a consequence, the potentially less demanding electronic front-end that can be realized, thanks to the larger charge integrated in input.\\

%%%%%%%%%%%%%%%%%%%%%%%
 % Figure environment removed
%%%%%%%%%%%%%%%%%%%%%%%

In all the following plots, the measured time resolution for a fixed CFD (more details in \cite{2019Carne, 2023Carnesecchi}) has always been considered.
In Figure \ref{fig:time_v} the time resolution as a function of the applied voltage is reported.

%%%%%%%%%
 % Figure environment removed
%%%%%%%%%

First of all, if compared with \cite{2023Carnesecchi}, the results for all thicknesses are compatible.
It can be noticed that the resolutions of the single LGADs are very nicely uniform only for the 50 \textmu m couple, owing to the more uniform sensor wafer (and specifically to the more similar gains for the two sensors).
Nevertheless for all three thicknesses, for a fixed voltage an improvement of the time resolution has been observed with the d-LGADs. A final time resolution of $\sim$~20~ps has been obtained for all three thicknesses. 
For the future, new 25 and 35 \textmu m  production with increased uniformity could potentially improve the time resolution of d-LGAD, bringing time resolution below 20 ps.
%In particular, a factor of 6.6, 4.2, 4.4 of improvement was observed for the 25, 35, and 50 \textmu m thicknesses, respectively. 

In Figure \ref{fig:time_charge_gain} the time resolution as a function of the charge is then reported. The plot would be totally similar if plotted vs gain. 
As expected, the d-LGADs show higher charge wrt to single LGADs at the input of electronics, in particular for uniform wafers and couples, as in the case of the 50 \textmu m  thickness.
%It is interesting to observe that, even for the same input charge, the d-LGADs consistently show a better time resolution when compared to single ones, as expected.  This would represent a significant benefit of this approach.\\

%%%%%%%%%
 % Figure environment removed
%%%%%%%%%

\begin{comment}
Paragone 25-50?\\
Aggiunguamo anche paragone diretto 25 e 50? dipende anche da spessore finale del 25 e del 50, quindi vedrei prima di riempire la tabella 1. Inoltre bisonga capire se davvero ci vengono cosi' simili, se il gain ci torna.
Intanto l ho aggiunta in \ref{fig:charge_25_50} per fare capire cio' cui mi riferisco\\

%%%%%%%%%
 % Figure environment removed
%%%%%%%%%
\end{comment}

%%%%%%%%%%%%%%%%%
\begin{table}[ht]
\begin{center}
\begin{minipage}{\textwidth}
\caption{Time resolution for 25, 35 and 50 for a given voltage (or gain) obtained in a beam test setup at room temperature.} \label{tab:res}
\begin{tabular*}{\textwidth}{@{\extracolsep{\fill}}lccc@{\extracolsep{\fill}}}
%\begin{tabular}{\textwidth}{lccc}% \hline
\toprule
 & Voltage applied & Gain & Time resolution\\
\midrule
FBK25 Front & 120 V & 24 $\pm$ 2 & (23 $\pm$ 2) ps\\
FBK25 Back & 120 V & 43 $\pm$ 4 & (28 $\pm$ 3) ps\\
d-FBK25 & 120 V & 35 $\pm$ 4 & (20 $\pm$ 2) ps\\
%\hline
\cdashline{1-4}
FBK35 Front & 240 V & 17 $\pm$ 2 & (30 $\pm$ 3) ps\\
FBK35 Back & 240 V & 27 $\pm$ 3 & (25 $\pm$ 2) ps\\
d-FBK35 & 240 V & 15 $\pm$ 2 & (23 $\pm$ 2) ps\\
%\hline
\cdashline{1-4}
HPK50 Front & 220 V & 63 $\pm$ 6 & (28 $\pm$ 3) ps\\
HPK50 Back & 220 V & 64 $\pm$ 6 & (28 $\pm$ 3) ps\\
d-HPK50 & 220 V & 59 $\pm$ 6 & (22 $\pm$ 2) ps\\

\botrule
\end{tabular*}
\end{minipage}
\end{center}
\end{table}
%%%%%%%%%%%%%%%%%

In Table \ref{tab:res} the best time resolutions reached for the three detectors tested are summarized.


\section{Conclusions}\label{sec:conclusions}
The study presented in this paper describes a new concept for improving time resolution by coupling two LGADs connected to the same amplifier. All the results have been obtained using a 10 GeV/c beam at CERN PS. 
For three different thicknesses of sensors, the performance of the d-LGAD has been compared with that of the single LGADs composing it. The d-LGAD concept shows clear advantages over the standard single LGAD, resulting in better time resolution with the added benefit of a higher charge provided at the input of the amplifier. In particular, results demonstrate a consistent improvement in time resolution for the d-LGAD compared to the single LGADs, reaching a time resolution of $\sim$~20~ps for all three thicknesses. %with a factor of 6.6, 4.2, and 4.4 for the 25 \textmu m, 35 \textmu m and 50 \textmu m thicknesses, respectively. 
Additionally, for all the couples, the charge MPV generated by the d-LGAD is doubled compared to both single sensors, as expected, resulting in a clear advantage for the electronics.
Overall, this concept presents a promising development for LGAD’s performance and paves the way for future implementation of such sensors.


\backmatter

\bmhead{Acknowledgments}

We acknowledge the following funding agencies and collaborations: INFN – FBK agreement on sensor production; Dipartimenti di Eccellenza, Univ. of Torino (ex L. 232/2016, art. 1, cc. 314, 337); Ministero della Ricerca, Italia, PRIN 2017, Grant 2017L2XKTJ – 4DinSiDe; Ministero della Ricerca, Italia, FARE,    Grant R165xr8frt\_fare.\\
The authors wish also to thank the support of the mechanical and electronic workshops of the INFN Unit of Bologna and the CERN-PS operator team for the support. We would also like to thank the CERN Bondlab for their availability during the beam tests.

\section*{Declarations}
The study was funded by: INFN – FBK agreement on sensor production; Dipartimenti di Eccellenza, Univ. of Torino (ex L. 232/2016, art. 1, cc. 314, 337); Ministero della Ricerca, Italia, PRIN 2017, Grant 2017L2XKTJ – 4DinSiDe; Ministero della Ricerca, Italia, FARE, Grant R165xr8frt\_fare.\\
The authors received research support from institutes as specified in the author list beneath the title. \\

\section*{Data availability}
The datasets generated during and/or analysed during the current study are available from the corresponding author on reasonable request.

%%===========================================================================================%%
%% If you are submitting to one of the Nature Portfolio journals, using the eJP submission   %%
%% system, please include the references within the manuscript file itself. You may do this  %%
%% by copying the reference list from your .bbl file, paste it into the main manuscript .tex %%
%% file, and delete the associated \verb+\bibliography+ commands.                            %%
%%===========================================================================================%%

\bibliography{sn-bibliography}% common bib file
%% if required, the content of .bbl file can be included here once bbl is generated
%%\input sn-article.bbl

%% Default %%
%%\input sn-sample-bib.tex%

\end{document}
