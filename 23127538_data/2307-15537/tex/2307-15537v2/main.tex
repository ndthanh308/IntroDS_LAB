\documentclass[11pt,a4paper]{scrartcl}

% Defines default style and includes several useful packages
\usepackage{ILD}

% Useful macros for writing ILDdp notes
\usepackage[symbol]{footmisc}
\usepackage{feynmf}
\usepackage{multirow}
\usepackage{textpos}
%\usepackage{lineno}
%\linenumbers

%============================================%
% Set up the title page
%============================================%

% Set the title of the note
\title{Tuning Pythia8 for future $e^+e^-$ colliders}

% Set the ILD note number
% Numbering convention:
% TTTT = topic (phys, soft or tech)
% YYYY = year
% NNN = number
% Papers and topical papers:
%\ildphys{YYYY}{NNN} % Physics 
%\ildsoft{YYYY}{NNN} % Software
%\ildtech{YYYY}{NNN} % Technical
% ILD notes and conference proceedings:
%\ildpublic{TTTT}{YYYY}{NNN}% Public note 
%\ildinternal{TTTT}{YYYY}{NNN} % Internal note 
\ildproc{PHYS}{2023}{002} % Proceedings
% Example:
%\ildphys{2023}{NNN}
%\ildpublic{phys}{2021}{001}
%\ildproc{phys}{2021}{001}
% ILD numbering convention:
% https://confluence.desy.de/display/ILD/ILD+Publication+and+Speakers+Bureau

% Set the publication date
\date{DESY-23-111}
%\date{\formatdate{18}{6}{2014}}

% Define the authors and their institutes, they will appear exactly in the order as they are added
% Footnotes can be added using the \thanks command
\addauthor{Zhijie Zhao}{\institute{1,2}\thanks{Talk presented at the International Workshop on Future Linear Colliders (LCWS 2023), 15-19 May 2023. C23-05-15.3.}}
\addauthor{Mikael Berggren}{\institute{1}}
\addauthor{Jenny List}{\institute{1}}

\addinstitute{1}{Deutsches Elektronen-Synchrotron DESY, Germany}
\addinstitute{2}{Center for Future High Energy Physics, Institute of High Energy Physics, Chinese Academy of Sciences, China}

% Add "On behalf of ... (optional)"
%\onbehalfof{the ILD detector concept group}

% Define an abstract for the note 
\abstract{
The majority of Monte-Carlo (MC) simulation campaigns for future $e^+e^-$ colliders has so far been based on the leading-order (LO) matrix elements provided by Whizard 1.95, 
followed by parton shower and hadronization in Pythia6, 
using the tune of the OPAL experiment at LEP. 
In this contribution, we test and develop the interface between Whizard3 and Pythia8. 
As a first step, we simulate the $e^+e^-\to q\bar{q}$ process with LO matrix elements, 
and compare three tunes in Pythia8: the standard Pythia8 tune, the OPAL tune and the ALEPH tune. 
At stable-hadron level, predictions of charged and neutral hadron multiplicities of these tunes are compared to LEP data, 
since they are strongly relevant to the performance of particle flow algorithms.

The events are used to perform a full detector simulation and reconstruction of the International Large Detector concept (ILD) as an example for a particle-flow-optimised detector. 
At reconstruction level, a comparison of the jet energy resolution in these tunes is presented. 
We found good agreement with previous results that were simulated by Whizard1+Pythia6. 
In addition, the preliminary next-to-leading order (NLO) results are also presented. 
This modern MC simulation chain, with matched NLO matrix elements in the future, should be introduced to ILC or other future $e^+e^-$ colliders.
}

% Uncomment this line to remove the stamp with the ILD note number from the top right corner
% of the title page
%\notitlestamp

%============================================%
% Bibliography
%============================================%
% define the list of bibliography data files
\addbibresource{./ref.bib}

%============================================%
% Search path for images
%============================================%
\graphicspath{ {./logos/}{./figures/} }

%============================================%
% Start of the actual document
%============================================%
\begin{document}

% generates the title page
\titlepage



%\section{Introduction}
%\label{sec:intro}
%% Figure environment removed

\section{Introduction}
Automatic 3D reconstruction of clothed humans using image inputs has gained increasing significance due to its potential applications in a wide array of AR/VR scenarios. High-fidelity reconstructions typically depend on sophisticated capture systems, which are developed with dense camera arrays~\cite{collet2015high,joo2015panoptic,joo2018total}, programmable light-stages~\cite{Vlasic2009, guo2019relightables}, and depth sensors~\cite{newcombe2011kinectfusion,DoubleFusion,BodyFusion,dou2016fusion4d,newcombe2015dynamicfusion}. However, stringent capture environments equipped with complex hardware pose significant challenges for consumer-level applications.


In this context, considerable research effort has been dedicated to developing methods that allow for more flexible capture configurations, such as utilizing a few RGB inputs. Among these works, learning implicit functions \cite{iccv2020PIFu, saito2020pifuhd, hong2021stereopifu} has proven effective in achieving highly detailed reconstructions by integrating the advancements of deep neural networks. These methods employ large multi-layer perceptrons (MLPs) to predict the occupancy probability or truncated signed distance function (TSDF) value of every queried 3D point based on its associated local feature, which is extracted from images. They can recover a continuous surface at arbitrary resolutions without topology restrictions.


However, in typical MLP-based implicit networks, the occupancy or TSDF value at each location is solved independently with planar image features, rendering them less capable of addressing challenging cases such as occlusions. Consequently, these methods suffer from generalization and robustness issues, particularly when tackling strong occlusions caused by large motion or multiple interacting humans. 
Some follow-up studies  \cite{zheng2021deepmulticap,zheng2021pamir,huang2020arch} utilize an extra geometric model, SMPL~\cite{Loper2015}, to improve robustness by introducing strong shape priors. 
Their success typically relies on the assumption of geometrical similarity \cite{huang2020arch} between the shape prior and target reconstruction, making them intractable for handling complex cases with loose clothes and sensitive to errors in SMPL model fitting.



%\ping{this paragraph sounds like `TSDF is better than MLP/SMPL, and we use TSDF to solve the problem'. But in Sec 3, we are telling a different story, saying `MLP needs a 3D convolutional encoder'. We need to make these two sections consistent.}\sicong{I think in this paragraph we claim that the TSDF}


%We opt for Trucated Signed Distance Funtion (TSDF) volumetric representations as they are naturally suitable for convolution operations, which have shown remarkable performance for learning hierarchical features on 2D visual perception tasks \cite{SunXLW19}. 
%Meanwhile, TSDF also describes the gradual geometry change around shape surface, which is not reflected by occupancy volume. 

We instead revisit the 3D volumetric representation and resort to 3D convolutional neural networks (CNNs) for feature learning, due to their impressive performance in feature learning and the ability to incorporate spatial context. However, volumetric methods and 3D convolution involve discretization, which might raise concerns regarding whether a discretized volume can preserve subtle geometric details as continuous representations learned in implicit functions. We investigate the relationship between volume resolution and quantization error on synthetic data by converting target mesh objects to TSDF volumes, as shown in Figure~\ref{fig:quantization_error}. We observe that the quantization errors are significantly reduced by increasing volume resolution and become nearly negligible when reaching a relatively high resolution (e.g., 512 or higher). In other words, achieving fine-detailed reconstruction is not supposed to be restricted by the use of volume representations as long as a proper volume resolution is utilized. Therefore, we present a method with high-resolution feature volumes, e.g., 256 and 512, while traditional volumetric methods \cite{varol18_bodynet,gilbert2018volumetric} are often limited to much lower resolutions, such as 32 or 128.



On the other hand, an increase in volume resolution may lead to a cubic growth of memory overhead \cite{8100085}. Reducing memory costs while guaranteeing the granularity of volumetric representations is necessary for pursuing high-quality reconstruction. Thus, we adopt a coarse-to-fine approach and cull away irrelevant voxels to build a sparse high-resolution feature volume. At the coarse level, the network computes an initial TSDF by applying a U-Net with sparse 3D CNN \cite{3DSemanticSegmentationWithSubmanifoldSparseConvNet} on the sparse feature volume, which is carved by a visual hull. Through our experiments, it turns out that more than 95\% of the volume grids are discarded by the visual hull culling, making the sparse 3D CNN efficient. At the fine level, the network focuses on a narrow band near the zero-level set of the initial TSDF and discretizes the narrow band with smaller voxels. By employing this narrow-band culling, we further shrink the sampling space, resulting in a relatively small range of grid numbers (usually 300K--500K in our experiments) even with a high volume resolution of 512. The remaining voxels in the narrow band are associated with features that fuse high-frequency information from the computed normal maps upon the low-frequency shape from the coarse level to compute the TSDF at high resolution. The final mesh is then extracted from the TSDF using the Marching-Cube algorithm ~\cite{Lorensen87marchingcubes}.
% Different from the u-net sturcture to preserve global topology context, we then apply a shallow 3dcnn to compute the final TSDF $D_{final}$ which contain more local geometry detail.




% \ping{this paragraph can be expanded. It is an important contribution and often ignored by other works. stress on the novel idea of regressing blending weights instead of colors}

In addition to geometry, high-quality mesh texture is also a crucial factor contributing to visual appearance. Directly computing a color field in 3D space, as in \cite{iccv2020PIFu}, struggles to capture high-frequency texture details, while the neural radiance field (NeRF) \cite{yu2020pixelnerf} or the DoubleField~\cite{shao2022doublefield} require expensive per-instance optimization and are often unstable for sparse input images. In contrast, we adopt an image-based rendering approach to compute a texture atlas map, which is efficient and widely supported in existing computer graphics tools. 
Specifically, we compute a blending weight at each 3D point on the mesh surface to determine its color as a weighted average of the colors at its image projections. The blending weights can be computed at a relatively coarse resolution, e.g., 512 volume resolution in our case, and leave texture details to the high-resolution images, such as 1K or 2K. Unlike previous methods that generate blurry texturing results under sparse input, our method generalizes well on both synthetic and real data with just a few input views. 
Figure~\ref{fig:teaser} shows two examples reconstructed by our method. Despite the challenging garment, pose, and occlusion, our method recovers faithful shape, normal, and texture on the right.

%with a wide variety of poses and clothing styles, and it is also adaptive to handle input image with arbitrary resolutions.
%\sicong{For this concern we claim that when the resolution of dicretized volume meets certain threshold (which is 256 in our experiment), the quantization error can be neglected.} 



In summary, the main contributions of this paper are as follows:
\begin{itemize}
\vspace{-0.1in}
  \item 
  We revisit the 3D volumetric representation and demonstrate that it can support clothed human reconstruction with equal or even better performance compared to implicit representation. 
  \item 
  We develop a memory and computation-efficient method for high-resolution volumetric reconstruction using sophisticated sparse 3D CNN, coarse-to-fine estimation, and voxel culling by visual hull and narrow bands. 
  \item 
  We introduce a novel method to compute a texture atlas map, which captures rich appearance details from high-resolution input images.
  \item 
  We achieve impressive results on standard benchmark datasets Twindom and MultiHuman, significantly reducing the point-2-surface (P2S) precision to approximately 0.2cm from just six input views, with more than $50\%$ error reduction compared to the state-of-the-art methods, including DoubleField~\cite{shao2022doublefield} and PIFuHD~\cite{saito2020pifuhd}.
\end{itemize}

%\section{The ILD Concept and MC Samples}
%\label{sec:ILDMC}
%\section{The ILD Concept and MC Samples}
\label{sec:ILDMC}
ILD~\cite{ILDIDR, ILDESU, TDR4} is one of the proposed detector concepts for future $e^+e^-$ Higgs factories.
It is highly-hermetic multi-purpose detector designed for particle flow reconstruction.
ILD consists of a high precision vertex detector, a time projection chamber, silicon tracking detectors, a highly granular calorimeter system and a forward detector system, all placed inside of a solenoid providing a magnetic field of $3.5$\,T, surrounded by an iron yoke instrumented for muon detection.
The jet energy resolution as a function of $|\cos \theta _{\mathrm{thrust}}|$ is shown in Fig.~\ref{fig:JER}.
Details of the ILD design, as well as about the particle flow concept can be found in Refs.~\cite{TDR4, LongESU, PFA}.

% Figure environment removed


%\section{High-level Reconstruction and Event Selection}
%\label{sec:recosel}
%\section{High-level Reconstruction and Event Selection}
\label{sec:recosel}


%\section{Results}
%\label{sec:results}
%\section{Experimental Results}\label{sec:results}
    \subsection{General Results}
        The basic ResSAN model is used to determine reference results which our expanded model can be compared to as it is structurally similar to ResLAN but does not possess the Lidar adaptive components of it. Further, we compare with the full-size PackNet-SAN and the unmodified NLSPN architecture. 
        As it can be seen from Tab.\,\ref{tab:sota-results}, our LiDAR-adaptive ResLAN achieves competitive performance compared to state-of-the-art standard depth completion methods, which are specialized to the unfiltered 64-beam-LiDAR. The performance differences are in the range of a few centimetres in terms of MAE, which is acceptable given the practical advantage that ResLAN can generalize to different beam patterns as will be shown below.

        Furthermore, we compared the architectures for a set of three different input types that contained 64, 32 or 16 LiDAR channels using both filter types on the metrics from the KITTI benchmark. The NLSPN model was trained for the standard depth completion task and then evaluated with different input data. As for the ResSAN models, we trained one model for each input type and tested it for the corresponding one which serve serve as the \emph{Baseline} in Tab.\,\ref{tab:overall-results}. Our ResLAN model was jointly trained for all three settings. As listed in Tab.\,\ref{tab:overall-results}, the ResLAN models outperform the challenging baseline in all metrics for FOV filtering and all but one for sparse filtering. This implies that our LiDAR adaptive model is able to outperform dedicated models in case of very sparse input depth. Fig.\,\ref{fig:comp-plot} shows this is indeed the case for 32 and even more for 16 channels. For FOV-filtered inputs with 16 channels, the ResLAN exhibits approx. $10\%$ smaller MAE than the baseline. As for the NLSPN, it becomes apparent that it is not capable of generalizing to other input types since it shows clearly worse results. The difference is especially pronounced for the FOV filtering where on average more than every fourth predicted pixel is more than $25 \%$ deviating from the ground truth\,($\delta_{1.25}$). Therefore, using a weight-adapting network in combination with differently filtered input depths allows us to train models that outperform their non-adaptive counterparts.

        \begin{table}[]
            \centering
    	    \small
            \vspace{0.4cm}
            \caption{\textbf{Depth estimation result for standard depth completion} when the ResSAN model was only trained for 64 channels and the ResLAN model for multiple tasks. The PackNet-SAN and NLSPN models were trained with the setup that was also used for our model architecture.}
            \footnotesize
            \setlength{\tabcolsep}{5pt}
            \begin{tabular}{@{}lrrrrl@{}}
            \toprule
            \multicolumn{6}{c}{\textbf{Standard LiDAR Depth Completion}}                                                                                                                         \\ \midrule
            \multicolumn{1}{l|}{Method}          & RMSE $\downarrow$            & MAE  $\downarrow$            & iRMSE $\downarrow$             & iMAE $\downarrow$ & $\delta_{1.25}$ $\uparrow$ \\
            \multicolumn{1}{l|}{}                & \multicolumn{1}{l}{{[}mm{]}} & \multicolumn{1}{l}{{[}mm{]}} & \multicolumn{1}{l}{{[}1/km{]}} & {[}1/km{]}        &                            \\ \midrule
            \multicolumn{1}{l|}{PackNet-SAN}     &  914                            &  298                            &  2.78                              &  1.4                 &  99.65 \%                          \\
            \multicolumn{1}{l|}{NLSPN}           &  \textbf{889}                            &   \textbf{263}                           &  \textbf{2.62}                              &   \textbf{1.3}                &   \textbf{99.61} \%                         \\ \midrule
            \multicolumn{1}{l|}{ResSAN (Ours)}   & 948                             &  275                            &  2.75                              &    1.4               &   99.58 \%                         \\
            \multicolumn{1}{l|}{ResLAN (Ours)} &   969                           &  283                            &   2.83                             &   1.4                &  99.56 \%                          \\ \bottomrule
            \end{tabular}
            \vspace{0.2cm}
            \label{tab:sota-results}
        \end{table}

        \begin{table}[]
    	    \centering
    	    \small
    	    \caption{\textbf{Depth estimation results of the two baseline setups and the explicit and implicit ResSAN} when evaluated on a combination of 16, 32 and 64 channel depth inputs. Please note that Specialist Methods need to train three specialized networks, one for each of the three types of inputs while our method only uses one network.}
            \footnotesize
            \setlength{\tabcolsep}{4.8pt}
            \begin{tabular}{@{}lrrrrl@{}}
                \toprule
                \multicolumn{6}{c}{\textbf{Sparse Channel Filter}}                                                                                                                                  \\ \midrule
                \multicolumn{1}{l|}{Method}        & RMSE $\downarrow$            & MAE  $\downarrow$            & iRMSE $\downarrow$             & iMAE $\downarrow$ & $\delta_{1.25}$ $\uparrow$  \\
                \multicolumn{1}{l|}{}              & \multicolumn{1}{l}{{[}mm{]}} & \multicolumn{1}{l}{{[}mm{]}} & \multicolumn{1}{l}{{[}1/km{]}} & {[}1/km{]}        &                             \\ \midrule
                \multicolumn{1}{l|}{NLSPN}         &  1396                            &  437                            & 5.54                               &  2.2                 &  98.82 \%                           \\
                \multicolumn{1}{l|}{Baseline}      & \textbf{1207}                             &  381                            & 4.41                               &  1.8                 &  \textbf{99.37} \%                           \\
                \multicolumn{1}{l|}{ResLAN (Ours)} &  1215                            &  \textbf{378}                            &  \textbf{4.27}                              &  \textbf{1.7}                 &  99.31 \%                           \\ \toprule
                \multicolumn{6}{c}{\textbf{Field-of-View Filter}}                                                                                                                                   \\ \midrule
                \multicolumn{1}{l|}{Method}        & RMSE $\downarrow$            & MAE  $\downarrow$            & iRMSE $\downarrow$             & iMAE $\downarrow$ & $\delta_{1.25}$ $\uparrow$ \\
                \multicolumn{1}{l|}{}              & \multicolumn{1}{l}{{[}mm{]}} & \multicolumn{1}{l}{{[}mm{]}} & \multicolumn{1}{l}{{[}1/km{]}} & {[}1/km{]}        &                             \\ \midrule
                \multicolumn{1}{l|}{NLSPN}         &  2738                            &  1702                            & 12.3                              &  4.3                 &  74.69 \%                           \\
                \multicolumn{1}{l|}{Baseline}      &  1556                            &  525                            &  6.8                              &  3.0                 & 98.14 \%                            \\
                \multicolumn{1}{l|}{ResLAN (Ours)} &  \textbf{1548}                            &  \textbf{519}                            &  \textbf{6.44}                              &  \textbf{2.8}                 & \textbf{98.52 \%}                            \\ \bottomrule
            \end{tabular}
            \label{tab:overall-results}
        \end{table}

        
        
        % Figure environment removed
        
        % Figure environment removed

    \subsection{Filter Effects}
        Comparing the effect of the two different types of depth input filters on the model performance, it becomes apparent that FOV filtering is the more challenging task. In that setting, reducing LiDAR channels is more detrimental to the performance than sparse filtering as it creates regions where no depth information is available. Effectively, the model is forced to perform depth prediction in these regions. These effects are highlighted in the depth images in Fig.\,\ref{fig:dense-maps} where the effect of a 16-channel sparse depth filter and a 16-channel FOV can be compared.

    \subsection{Generalization Capabilities}
        We trained three models for both filter types eaach, so the combinations and number of filtered depth inputs they receive are different. This serves the purpose of testing the generalization capabilities of the ResLAN architecture as well as the robustness to different filter settings. After training, the models were evaluated for the depth input settings they were trained for, as well as for ones they weren't exposed to. Overall, ResLAN shows good generalization capabilities. As one can gather from Fig.\,\ref{fig:explicit-comp} and Fig.\,\ref{fig:implicit-comp}, the consequences of slightly varying sets of input depth settings are limited. The most considerable deviations can be seen when the model is tasked to extrapolate. For instance, the model $\{64, 32, 16\}$ shows a noticeably higher MAE for eight-channel depth inputs than the model that was trained for it. Similar behaviour can be seen for the FOV filtering case as well for the model $\{64, 48, 32\}$ when tasked to generalize for a 16-channel input. There is no such pronounced effect for generalization tasks that lie between two filter settings the model was trained for. At most, it can be observed that models that were trained for a smaller range of filter values perform slightly better than ones that have to cover a wider range. The number of filter settings used in a fixed range does not relevantly influence the model performance, as can be seen, when comparing the two models in Fig.\,\ref{fig:implicit-comp}, which are both trained for a range of 64 to 32 channels but one with three filter settings and the other one with five.
    
    % Figure environment removed
    
    
    % Figure environment removed

%\section{Conclusion}
%\label{sec:concl}
%\include{concl.tex}

\section{Introduction}
\label{sec:intro}

To measure the properties of Higgs boson precisely, 
there are many proposals for future colliders, 
such as ILC~\cite{ILCInternationalDevelopmentTeam:2022izu,Behnke:2013xla,ILC:2013jhg,Adolphsen:2013jya,Adolphsen:2013kya,Behnke:2013lya}, 
CLIC~\cite{Linssen:2012hp,CLICdp:2018cto}, 
CEPC~\cite{CEPCPhysicsStudyGroup:2022uwl,CEPCStudyGroup:2018rmc,CEPCStudyGroup:2018ghi} 
and FCC-ee~\cite{Bernardi:2022hny,FCC:2018evy}. 
All of them are $e^+e^-$ colliders, 
and they are designed as Higgs Factories. 
There are many features of lepton beam: 
initial state radiation (ISR), polarization, beam-strahlung, and so on.
They also give special requirements to the Monte-Carlo (MC) event generator.
Whizard~\cite{Kilian:2007gr} is a general purpose event generator that can handle these features, 
and has been widely used in analyses of processes at $e^+e^-$ colliders.

There are many detector concepts for future Higgs Factories. 
In this contribution, we take International Large Detector (ILD)~\cite{Behnke:2013lya,ILDConceptGroup:2020sfq} as an example. 
ILD is designed for $e^+e^-$ collisions between 90 GeV and 1 TeV. 
It is optimized for particle flow algorithm (PFA). 
The PFA aims at recontructing every invidual particle created in the event, i.e., charged particles, photons, and neutral hadrons. 
In particular, the energy resolution of neutral hadrons is expected to be worse than that of the other two types of particle.
In the MC simulation, the reconstruction of these particles is strongly dependent on the tuning of MC parameters.

Some physics aspects cannot be derived from first principles, 
especially in the area of soft QCD. 
Pythia~\cite{Bierlich:2022pfr} is a MC generator that contains many parameters that represent a true uncertainty in our understanding of nature. 
Tuning these parameters is important to describe data. 
A good tune should have: 
\begin{itemize}
  \item Physically sensible parameter values, with good universality.
  \item Good agreement with data (LEP1 in this contribution).
  \item Reliable uncertainties.
  \item Best fit for our observables.
\end{itemize}

At present, events for analysis of $e^+e^-$ colliders have so far been based on the leading order (LO) matrix elements provided by WHIZARD 1.95. 
Parton shower and hadronization are performed by Pythia6~\cite{Sjostrand:2006za}, 
using the tune of the OPAL experiment at the Large Electron-Positron Collider (LEP). 
However, the major versions of Whizard and Pythia have been upgraded to Whizard3 and Pythia8. 
Our first goal is upgrading the simulation chain to Whizard3+Pythia8.
We hope to get good agreement with LEP data, especially the neutral hadrons. 
Finally, next-to-leading order (NLO) matched events should be included in the future, 
because of the requirement of high precision.

This contribution is organized as given below. 
In section~\ref{sec:ahm}, we compare the average hadron multiplicities of different Pythia8 and Pythia6 tunes. 
In section~\ref{sec:jer}, the full ILD simulation is performed and the jet energy resolution (JER) is discussed. 
Section~\ref{sec:nlo} shows the preliminary results of NLO events. 
Finally, we summarize this study in section~\ref{sec:summary}.

\section{Average Hadron Multiplicities}
\label{sec:ahm}

As we mention above, hadronization rates are crucial for studying particle flow performance. 
To study it, we use the following generator setup (LEP1 condition):
\begin{itemize}
     \item Process: $e^+e^-\to q\bar{q}$ $(q=u,d,s,c,b)$.  
     \item The center of mass energy is $E_{cm}=91.19$ GeV.
     \item Beams are un-polarized.
     \item Beam-strahlung is not considered.
     \item ISR is switched on. 
\end{itemize}
The parton shower and hadronization are performed by Pythia8. 
Three tunes are considered in this context: 
\begin{itemize}
  \item The standard tune, using the default parameter set of Pythia8.
  \item The OPAL tune. 
  \item The ALEPH tune.
\end{itemize}

The dominant hadrons are pions. 
The average numbers of $\pi^0$ and $\pi^\pm$ are displayed in the left of Fig.~\ref{fig:hadrons}.
Two Pythia6 tunes, standard and OPAL, are considered for comparison.
The LEP1 data are taken from \cite{Boehrer:1996pr,ALEPH:1996oqp} as a reference. 
We can see that three Pythia8 tunes and the Pythia6 standard tune are close to each other and generally agree with the LEP1 data, 
but the Pythia6 OPAL tune gives larger numbers.
% Figure environment removed

The numbers of other hadrons are displayed in the right of Fig.~\ref{fig:hadrons}. 
For the proton and $K^0_S$, which are important in the calculation of JER, 
the usual Pythia6 OPAL tune has good agreement with LEP1 data, 
but the numbers of other hadrons have large differences to data. 
Generally, the standard Pythia8 tune is the closest one to data.

\section{ILD simulation and jet energy resolution (JER)}
\label{sec:jer}

Full Geant4-based MC simulations are crucial to optimize a well performing detector concept. 
We take ILD as an example to discuss the detector performance. 
In this context, an important parameter is the JER of ILD. 
To study it, we use the following generator setup: 
\begin{itemize}
  \item Process: $e^+e^-\to q\bar{q}$ $(q=u,d,s)$.
  \item ISR is switched off. 
  \item $E_{cm}=40,91,200,350,500$ GeV.
  \item Full simulation is performed with large model of ILD (ILD-L).
\end{itemize}

As a first step, we reproduce the results in the ILD Interim Design Report (IDR)~\cite{ILDConceptGroup:2020sfq} by the Whizard3+Pythia8 framework.
Since the OPAL tune has been used in the IDR, 
we also generate events with the OPAL tune.
Comparison between our results and previous events is shown in Fig.~\ref{fig:jer1}. 
This figure is separated to 3 regions: 
left is the full angle region, 
middle is the barrel region ($|\cos\theta_{thrust}|<0.7$), 
and right is the forward region ($|\cos\theta_{thrust}|>0.7$). 
There are slight differences between two curves, 
but basically they have good agreement.
% Figure environment removed

We also compare the JER with differnt Pythia8 tunes. 
Results are shown in Fig.~\ref{fig:jer2}. 
All of them can give JER around 3\%. 
The standard tune and the ALEPH tune have some overlap, 
and the curve of OPAL tune is lower than other two curves.
% Figure environment removed

\section{Preliminary Results of NLO Events}
\label{sec:nlo}

NLO QCD corrections can be calculated by interfacing Whizard with OpenLoops~\cite{Buccioni:2019sur}.
Whizard supports POWHEG matching~\cite{Nason:2004rx} to generate NLO events. 
The average hadron multiplicities with Pythia8 standard tune are shown in Fig.~\ref{fig:pions-nlo}.
One can observe that the numbers of hadrons at NLO are slightly lower than the LO.
% Figure environment removed 

To see the NLO effects, we use FastJet~\cite{Cacciari:2011ma} to find jets with the Durham algorithm~\cite{Catani:1991hj}.
The total number of jets is forced to 2. 
The transverse momentum distributions are displayed in Fig.~\ref{fig:pt-nlo}.
For the $e^+e^-\to u\bar{u}$ process, the QCD correction is small. 
We can see that the ratio is almost $1$ in these plots. 
Another feature is that the enhancement tends to occur at the large $p_T$ region.  
% Figure environment removed

We can also pass the NLO events to perform an ILD simulation. 
The results of JER are plotted in Fig.~\ref{fig:jer3}. 
We can see that the JER curves at LO and NLO are consistent in the full angle region and the barrel region. 
But it is very different in the forward region. 
% Figure environment removed

To figure out the reason, we plot the thrust distributions of two quarks at parton level in Fig.~\ref{fig:thrust}.
Obviously, there are fewer NLO events when the thrust tends to 0 or 1. 
In Whizard, there is a minimum $p_T$ for the hardest emission in an event. 
In our case, it is the additional gluon. 
This $p_T$ cut reduces the number of events in the forward region significantly.
% Figure environment removed

\section{Summary}
\label{sec:summary}

In this contribution, we have upgraded the MC simulation chain for $e^+e^-$ colliders with the Whizard3+Pythia8 framework. 
As a first step, we study the process $e^+e^- \to q\bar{q}$.
Three tunes of Pythia8 have been compared: the standard Pythia8 tune, the OPAL tune and the ALEPH tune. 
The hadron multiplicities and JER at ILD have been compared. 
The standard tune can give hadron multiplicities close to LEP1 data, and the best JER is obtained by the OPAL tune. 
In this context, we are in favor of the Pythia8 standard tune, 
since it has better over-all agreement with the LEP1 data. 
We also test the NLO mode of Whizard and generate events by POWHEG matching. 
Preliminary results of NLO events are shown. 

\section*{Acknowledgements}

Z. Zhao has been partially supported by a China and Germany Postdoctoral Exchange Program between the Office of China Postdoctoral Council (OCPC) and DESY.

\section*{Appendix}

In this study, three tunes of Pythia8 are considered: 1) the standard tune with the default parameter set of Pythia8, 
2) the tune from the OPAL experiment, and 3) the tune from the ALEPH experiment.
The input parameters of these tunes are listed in Table.~\ref{beta1}.
The details of these parameters are referred to~\cite{Skands:2014pea}.

\begin{center}
\begin{table}
  \begin{center}
  \begin{tabular}{l|l|c|c|c}
  \hline
  Parameter      &  name in PYTHIA8  & standard   & OPAL   &  ALEPH\\
  \hline
  \hline
 $P(qq)/P(q)$  & StringFlav:probQQtoQ  & 0.081 &  0.085 &  0.105 \\
 $P(s)/P(u)$     & StringFlav:probStoUD  & 0.217  &  0.310 &  0.283 \\
 $(P(su)/P(du))/(P(s)/P(u))$        & StringFlav:probSQtoQQ & 0.915 & 0.45  &  0.710 \\
 $\frac{1}{3}(P(ud_1)/P(ud_0))$  & StringFlav:probQQ1toQQ0 & 0.0275  &  0.025  & 0.05 \\
 $(S=1)$ d,u   & StringFlav:mesonUDvector &  0.50  &  0.60 & 0.54 \\
 $(S=1)$ s      & StringFlav:mesonSvector    &  0.55   &  0.40 & 0.46 \\
 $(S=1)$ c,b   & StringFlav:mesonCvector   &  0.88   &  0.72  & 0.65 \\
                        & StringFlav:mesonBvector   &   2.20   &  0.72  &  0.65 \\
 $S=1,s=0$ prob. & StringFlav:mesonUDL1S0J1  &  0.0 &  0.43 & 0.12\\
 $S=0,s=1$ prob.  & StringFlav:mesonUDL1S1J0 & 0.0   &  0.08 & 0.04\\
 $S=1,s=1$ prob.  & StringFlav:mesonUDL1S1J1   &  0.0 &  0.08  & 0.12 \\
 tensor mesons (L=1)  & StringFlav:mesonUDL1S1J2 & 0.0 & 0.17 & 0.20 \\
 leading baryon suppr.  & StringFlav:suppressLeadingB & off  &  on  &  on\\
                                       & StringFlav:lightLeadingBSup   & 0.5 &  1.0 & 0.58 \\
                                      & StringFlav:heavyLeadingBSup & 0.9 &  1.0 & 0.58 \\
 $\sigma$ (GeV)      &  StringPT:sigma  &  0.335  &  0.4000 &  0.362 \\
 $\eta^\prime$ suppression  & StringFlav:etaPrimeSup  & 0.12  &  0.40  &  0.27\\ 
 $a$ of LSFF    &  StringZ:aLund & 0.68 &  0.11  &  0.40\\
 $b$ of LSFF     &  StringZ:aLund & 0.98  &  0.52 & 0.824 \\
 $\Delta a$ for s quark & StringZ:aExtraSQuark &  0.0 &0.0 & 0.0   \\
 $\Delta a$ for Di-quark & StringZ:aExtraDiquark & 0.97 & 0.5 & 0.5 \\
 $\epsilon_c$    &  StringZ:usePetersonC  & off  &  on  &   on\\
                           &  StringZ:epsilonC  & 0.05        &  $-0.031$ &  0.04\\
 $\epsilon_b$    &  StringZ:usePetersonB   & off  &  on  &  on \\
                           &  StringZ:epsilonB  & 0.005       &  $-0.002$ &  0.0018\\
% $\Lambda_{QCD}$  (GeV) & PARJ(81) & CoupSM::Lambda3() \\
%                                             &                 & CoupSM::Lambda4() \\
%                                              &                 & CoupSM::Lambda5() \\
 PS QCD cut-off (GeV)  & TimeShower:pTmin  &  0.5    & 0.95  &  0.735  \\
 PS cut-off for QED  & TimeShower:pTminChgQ  &  0.5    & 0.95  &  0.735  \\
 adiation off quarks (GeV) &  &  & &  \\
  \hline
  \hline
  \end{tabular}
  \end{center}
  \caption{The input parameters of three tunes of Pythia8: the standard, OPAL and ALEPH tunes are listed here.} \label{beta1}
\end{table}
\end{center}  


% add references
\printbibliography{}
\end{document}
