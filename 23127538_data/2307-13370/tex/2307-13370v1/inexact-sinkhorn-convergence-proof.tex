For every $t \geq 0$ and $j \in \{1,\dots,k\}$, let
$\widetilde{\nu}^{j}_{t}$
be the distribution returned by the approximate Sinkhorn oracle that satisfies
the properties listed in Definition~\ref{dfn:approximate-sinkhorn-oracle}.
We follow along the lines of proof of
Theorem~\ref{thm:exact-scheme-convergence}.

First, we will establish an upper
bound on the oscillation norm of the iterates $\widetilde{\vecpsi}_{t}$.
Indeed, by the property four in
Definition~\ref{dfn:approximate-sinkhorn-oracle} we have
\begin{equation}
  \|\widetilde{\psi}^{j}_{t+1}\|_{\mathrm{osc}}
  \leq
  (1-\eta)\|\widetilde{\psi}^{j}_{t}\|_{\mathrm{osc}} + \eta
  c_{\infty}(\mathcal{X}).
\end{equation}
Since $\widetilde{\psi}^{j}_{0} = 0$, for any $t \geq 0$ we have
$\|\widetilde{\psi}^{j}_{t}\|_{\mathrm{osc}} \leq c_{\infty}(\mathcal{X})$.

Let $\tilde{\delta}_{t} = E_{\lambda,\tau}^{\vecnu, w}(\psi^{*}) -
E_{\lambda,\tau}^{\vecnu, w}(\widetilde{\vecpsi}_{t})$ be the suboptimality gap
at time $t$. Using the concavity upper bound
\eqref{eq:concave-suboptimality-gap} and the property two in
Definition~\ref{dfn:approximate-sinkhorn-oracle} we have
\begin{align}
  \tilde{\delta}_{t}
  &\leq
  2c_{\infty}(\mathcal{X})\sum_{j=1}^{k}w_{j}\|\nu^{j} -
  \nu_{t}^{j}\|_{\mathrm{TV}}
  \\
  &\leq
  \varepsilon +
  2c_{\infty}(\mathcal{X})\sum_{j=1}^{k}w_{j}\|\nu^{j} -
  \widetilde{\nu}_{t}^{j}\|_{\mathrm{TV}}
  \\
  &\leq
  \varepsilon +
  \sqrt{2}c_{\infty}(\mathcal{X})\sum_{j=1}^{k}w_{j}
  \sqrt{\kl{\nu^{j}}{\widetilde{\nu}_{t}^{j}}}
  \\
  &\leq
  \varepsilon +
  \sqrt{2}c_{\infty}(\mathcal{X})
  \sqrt{\sum_{j=1}^{k}w_{j}\kl{\nu^{j}}{\widetilde{\nu}_{t}^{j}}}.
\end{align}
Combining the property three
stated in the
Definition~\ref{dfn:approximate-sinkhorn-oracle} with
Lemma~\ref{lemma:log-Z-ratio-bound} we obtain
\begin{align}
  \tilde{\delta}_{t} - \tilde{\delta}_{t+1}
  &\geq \min(\lambda,\tau)
  \sum_{j=1}^{k}w_{j}\kl{v^{j}}{\widetilde{v}_{t}^{j}}
  -\min(\lambda,\tau)
  \log\left(
    \sum_{j=1}^{k}
    w_{j}
    \int_{\mathcal{X}}
    \frac{d\nu_{t}}{d\widetilde{\nu}_{t}}(y)
    \nu^{j}(dy)
  \right)
  \\
  &\geq
  \sum_{j=1}^{k}w_{j}\kl{v^{j}}{\widetilde{v}_{t}^{j}}
  -\min(\lambda,\tau)
  \log\left(
    1 + \varepsilon^{2}/(2c_{\infty}(\mathcal{X})^{2})
  \right)
  \\
  &\geq
  \min(\lambda,\tau)
  \sum_{j=1}^{k}w_{j}\kl{v^{j}}{\widetilde{v}_{t}^{j}}
  -\frac{\min(\lambda,\tau)}{2c_{\infty}(\mathcal{X})^{2}}
  \varepsilon^{2}
  \\
  &\geq
  \frac{\min(\lambda,\tau)}{2c_{\infty}(\mathcal{X})^{2}}
  \max\left\{0, \tilde{\delta}_{t} - \varepsilon \right\}^{2}
  -\frac{\min(\lambda,\tau)}{2c_{\infty}(\mathcal{X})^{2}}
  \varepsilon^{2}.
\end{align}
Provided that $\widetilde{\delta}_{t} \geq 2\varepsilon$ it holds that
\begin{equation}
  (\tilde{\delta}_{t} - 2\varepsilon) - (\tilde{\delta}_{t+1} - 2\varepsilon)
  \geq
  \frac{\min(\lambda,\tau)}{2c_{\infty}(\mathcal{X})}
  (\tilde{\delta}_{t} - 2\varepsilon)^{2}.
\end{equation}
Let $T$ be the first index such that $\widetilde{\delta}_{T+1} < 2\varepsilon$
and set $T = \infty$ if no such index exists.
Then, the above equation is valid for any $t \leq T$. In particular, repeating
the proof of Theorem~\ref{thm:exact-scheme-convergence}, for any
$t \leq T$ we have
\begin{equation}
  \widetilde{\delta}_{t} - 2\varepsilon \leq
  \frac{2c_{\infty}(\mathcal{X})^{2}}{\min(\lambda,\tau)}\frac{1}{t},
\end{equation}
which completes the proof of this theorem. \hfill\qed




