\section{Background and Notation}
\label{sec:background}

This section provides the background material on doubly regularized entropic
Wasserstein barycenters and introduces the notation used throughout the paper.
In the remainder of the paper, let $\mathcal{X}$ be a
compact and convex subset of $\mathbb{R}^{d}$ with a non-empty
interior. Let $\mathcal{P}(\mathcal{X})$ denote the set of probability measures on
$\mathcal{X}$ endowed with Borel sigma-algebra.
Let $c: \mathcal{X} \times \mathcal{X} \to [0, \infty)$ be a cost function such
that $c_{\infty}(\mathcal{X}) = \sup_{x,x' \in \mathcal{X}} c(x,x') < \infty$.
We denote by $\kl{\cdot}{\cdot}$ the Kullback-Leibler divergence,
$\|\cdot\|_{\mathrm{TV}}$ is the total-variation norm, and
$\|f\|_{\mathrm{osc}} = \sup_{x}f(x) - \inf_{x'}f(x')$ is the oscillation
norm.
Given two measures $\nu,\nu'$, the notation $\nu \ll \nu'$ denotes that $\nu$
is absolutely continuous with respect to the measure $\nu'$; in this case
$d\nu/d\nu'$ denotes the Radon-Nikodym derivative of $\nu$ with respect to
$\nu'$. Finally, throughout the paper $w$ denotes a vector of $k$ strictly
positive elements that sum to one.

%%%%%%%%%%%%%%%%%%%%%%%%%%%%%%%%%%%%%%%%%%%%%%%%%%%%%%%%%%%%%%%%%%%%%%%%%%%%%%%
\subsection{Entropic Optimal Transport}
\label{sec:entropic-ot}

For any $\mu,\nu \in \mathcal{P}(\mathcal{X})$ define the entropy regularized
optimal transport problem by
\begin{equation}
  \label{eq:entropic-ot-problem}
  T_{\lambda}(\mu,\nu)
  = \inf_{\gamma \in \Pi(\mu,\nu)}\mathbf{E}_{(X,Y) \sim \gamma}[c(X,Y)]
  + \lambda\kl{\gamma}{\mu \otimes \nu},
\end{equation}
where $\mathrm{KL}$ is the Kullback-Leibler divergence and
$\Pi(\mu,\nu) \subseteq \mathcal{P}(\mathcal{X}\otimes\mathcal{X})$ is the set
of probability measures such that for any $\gamma \in \Pi(\mu,\nu)$ and any
Borel subset $A$ of $\mathcal{X}$ it holds that
$\gamma(A \times \mathcal{X}) = \mu(A)$ and $\gamma(\mathcal{X} \times A) =
\nu(A)$.

Let $E_{\lambda}^{\mu,\nu} : L_{1}(\mu) \times L_{1}(\nu) \to \mathbb{R}$ be
the function defined by
\begin{align}
  \begin{split}
  \label{eq:E-dfn}
  E_{\lambda}^{\mu,\nu}(\phi, \psi)
  &= \mathbf{E}_{X \sim \mu}[\phi(X)] + \mathbf{E}_{Y \sim \nu}[\psi(Y)]
  \\&\quad\quad+ \lambda\left(1 -
    \int_{\mathcal{X}} \int_{\mathcal{X}}
    \exp\left(\frac{\phi(x) + \psi(y) - c(x,y)}{\lambda}\right)
    \nu(dy)\mu(dx)
  \right).
\end{split}
\end{align}
The entropic optimal transport problem \eqref{eq:entropic-ot-problem}
admits the following dual representation:
\begin{equation}
  \label{eq:entropic-ot-dual-representation}
  T_{\lambda}(\mu, \nu)
  = \max_{\phi,\psi}
  E_{\lambda}^{\mu,\nu}(\phi, \psi).
\end{equation}
For any $\psi$ define
\begin{equation}
  \phi_{\psi} \in \mathrm{argmax}_{\phi \in L_{1}(\mu)}
  E_{\lambda}^{\mu,\nu}(\phi, \psi).
\end{equation}
The solution is unique $\mu$-almost everywhere up to a constant; we fix a
particular choice
\begin{equation}
  \phi_{\psi}(x) =
  -\lambda\log\left(
    \int_{\mathcal{X}}
  \exp\left(\frac{\psi(y) - c(x,y)}{\lambda}\right)\nu(dy)\right).
\end{equation}
Likewise, we denote
$\psi_{\phi} = \mathrm{argmax}_{\psi \in L_{1}(\nu)}E_{\lambda}^{\mu,\nu}(\phi,
\psi)$ with the analogous expression to the one given above, interchanging the
roles of $\phi$ and $\psi$. Then,
the maximum in \eqref{eq:entropic-ot-dual-representation}
is attained by any pair $(\phi^{*}, \psi^{*})$ such that
$\phi^{*} = \phi_{\psi^{*}}$ and $\psi^{*} = \psi_{\phi^{*}}$; such a pair is
said to solve the Schr\"{o}dinger system and it is unique up to translations
$(\phi^{*}+a,\psi^{*}-a)$ by any constant $a\in \mathbb{R}$.
The optimal coupling that solves the primal problem
\eqref{eq:entropic-ot-problem} can be obtained from the pair
$(\phi^{*}, \psi^{*})$ via the primal-dual relation
\begin{equation}
  \label{eq:entropic-primal-dual-relation}
  \gamma^{*}(dx,dy) =
  \exp\left(\frac{\phi^{*}(x) + \psi^{*}(y) - c(x,y)}{\lambda}\right)
  \mu(dx)\nu(dy).
\end{equation}
We conclude this section by listing two properties of functions of the form
$\phi_{\psi}$. These properties will be used repeatedly throughout this paper.
First, for any $\psi$ we have
\begin{equation}
  \int_{\mathcal{X}}\int_{\mathcal{X}}
  \exp\left(\frac{\phi_{\psi}(x) + \psi(y) - c(x,y)}{\lambda}\right)
  \nu(dy)\mu(dx) = 1,
\end{equation}
which means, in particular, that for any $\psi$ we have
\begin{equation}
  \label{eq:semi-dual-entropic-ot}
  E_{\lambda}^{\mu,\nu}(\phi_{\psi}, \psi)
  = \mathbf{E}_{X \sim \mu}[\phi_{\psi}(X)] + \mathbf{E}_{Y \sim \nu}[\psi(Y)].
\end{equation}
The second property of interest is that for any $\psi$ and any $x,x' \in
\mathcal{X}$ it holds that
\begin{align}
  \phi_{\psi}(x) - \phi_{\psi}(x')
  &=
  -\lambda\log
  \frac{
    \int
    \exp\left(\frac{\psi(y) - c(x,y)}{\lambda}\right)\nu(dy)
  }
  {\int
    \exp\left(\frac{\psi(y) - c(x',y)}{\lambda}\right)\nu(dy)
  }
  \\
  &=
  -\lambda\log
  \frac{
    \int
    \exp\left(\frac{\psi(y) - c(x',y) + c(x',y) - c(x,y)}{\lambda}\right)\nu(dy)
  }
  {\int
    \exp\left(\frac{\psi(y) - c(x',y)}{\lambda}\right)\nu(dy)
  }
  \\
  &\leq \sup_{y \in \mathcal{X}} c(x', y) - c(x,y)
  \leq c_{\infty}(\mathcal{X}).
\end{align}
In particular, for any $\psi$ we have
\begin{equation}
  \label{eq:schroedinger-potentials-bounded}
  \|\phi_{\psi}\|_{\mathrm{osc}} = \sup_{x} \phi_{\psi}(x) -
  \inf_{x'}\phi_{\psi}(x')
  \leq c_{\infty}(\mathcal{X}).
\end{equation}

%%%%%%%%%%%%%%%%%%%%%%%%%%%%%%%%%%%%%%%%%%%%%%%%%%%%%%%%%%%%%%%%%%%%%%%%%%%%%%%
\subsection{Doubly Regularized Entropic Barycenters}
\label{sec:doubly-entropic-barycenters}

Let $\vecnu = (\nu^{1}, \dots, \nu^{k}) \in \mathcal{P}(\mathcal{X})^{k}$
be $k$ probability measures and let $w \in \mathbb{R}^{k}$ be a vector of
positive numbers that sum to one.
Given the inner regularization strength $\lambda > 0$ and the outer regularization
strength $\tau > 0$, the $(\lambda,\tau)$ barycenter $\mu_{\lambda,\tau} \in
  \mathcal{P}(\mathcal{X})$
of probability measures
$\vecnu$ with respect to the weights vector $w$ is defined as the unique
solution to the following optimization problem:
\begin{equation}
  \label{eq:doubly-regularized-barycenter-primal}
  \mu_{\lambda,\tau}
  = \mathrm{argmin}_{\mu \in \mathcal{P}(\mathcal{X})}\,
  \sum_{j=1}^{k}w_{j}
  T_{\lambda}(\mu,\nu^{j})
  + \tau\kl{\mu}{\piref},
\end{equation}
where $\piref \in \mathcal{P}(\mathcal{X})$ is a reference probability measure.

We will now describe how to obtain a concave dual maximization problem to the
primal problem \eqref{eq:doubly-regularized-barycenter-primal}, following along
the lines of \citet*[Section 2.3]{chizat2023doubly}, where the interested
reader will find a comprehensive justification of all the claims made in the
rest of this section.

First, using the semi-dual formulation of entropic optimal transport problem
\eqref{eq:semi-dual-entropic-ot}, we have, for each $j \in \{1,\dots,k\}$
\begin{equation}
  T_{\lambda}(\mu,\nu^{j})
  = \sup_{\psi^{j} \in L_{1}(\nu^{j})} \mathbf{E}_{X\sim\mu}[\phi_{\psi^{j}}(X)]
  + \mathbf{E}_{Y \sim \nu^{j}}[\psi^{j}(Y)].
\end{equation}
Denote $\vecpsi = (\psi^{1},\dots,\psi^{j}) \in L_{1}(\vecnu)$. Then, we may
rewrite the primal problem \eqref{eq:doubly-regularized-barycenter-primal} by
\begin{equation}
  \min_{\mu \in \mathcal{P}(X)} \max_{\vecpsi \in L_{1}(\vecnu)}
  \sum_{j=1}^{k}w_{j} \mathbf{E}_{Y \sim \nu^{j}}\big[\psi^{j}(Y)\big]
  +
  \mathbf{E}_{X\sim\mu}\big[\sum_{j=1}^{k}w_{j}\phi_{\psi^{j}}(X)\big]
  + \tau\kl{\mu}{\piref}.
\end{equation}
Interchanging $\min$ and $\max$, which is justified using compactness of
$\mathcal{X}$ as detailed in \cite{chizat2023doubly}, we obtain the dual optimization
objective $E_{\lambda,\tau}^{\vecnu, w} : L_{1}(\vecnu) \to \mathbb{R}$
defined by
\begin{align}
  \begin{split}
  \label{eq:doubly-entropic-dual}
    E_{\lambda,\tau}^{\vecnu, w}(\vecpsi)
    &= \min_{\mu \in \mathcal{P}(X)}
    \sum_{j=1}^{k}w_{j} \mathbf{E}_{Y \sim \nu^{j}}\big[\psi^{j}(Y)\big]
    +
    \mathbf{E}_{X\sim\mu}\big[\sum_{j=1}^{k}w_{j}\phi_{\psi^{j}}(X)\big]
    + \tau\kl{\mu}{\piref}.
    \\
    &=
    \sum_{j=1}^{k}w_{j} \mathbf{E}_{Y \sim \nu^{j}}\big[\psi^{j}(Y)\big]
    - \tau
    \log \int \exp\left(\frac{-\sum_{j=1}^{k}\phi_{\psi^{j}}(x)}
    {\tau}\right) \piref(dx).
  \end{split}
\end{align}
The infimum above is attained by the measure
\begin{equation}
  \label{eq:doubly-entropic-argmin-mu}
  \mu_{\vecpsi}(dx)
  = Z_{\vecpsi}^{-1}\exp\left(
    \frac{-\sum_{j=1}^{k}\phi_{\psi^{j}}(x)}{\tau}
  \right)\piref(dx),\quad
  Z_{\vecpsi} =
    \int \exp\left(\frac{-\sum_{j=1}^{k}\phi_{\psi^{j}}(x)}
    {\tau}\right) \piref(dx).
\end{equation}
To each dual variable $\vecpsi$ we associate
the marginal measures $\nu^{j}_{\vecpsi}(dy)$
defined for $j=1,\dots,k$ by
\begin{equation}
  \label{eq:psi-marginals}
  \nu^{j}_{\vecpsi}(dy)
  = \nu^{j}(dy)\int\exp\left(
    \frac{\phi_{\psi^{j}}(x) + \psi^{j}(y) - c(x,y)}{\lambda}
  \right)\mu_{\vecpsi}(dx).
\end{equation}
Finally, we mention that the objective
$E^{\vecnu, w}_{\lambda,\tau}$
is
concave and for any $\vecpsi, \vecpsi'$ it holds that
\begin{equation}
  \lim_{h \to 0}
  \frac{
    E^{\vecnu, w}_{\lambda,\tau}(\vecpsi + h\vecpsi')
    -
    E^{\vecnu, w}_{\lambda,\tau}(\vecpsi)
  }{h}
  =
  \sum_{j=1}^{k}w_{j}\left(
    \mathbf{E}_{\nu^{j}}[(\psi')^{j}] -
    \mathbf{E}_{\nu^{j}_{\vecpsi}}[(\psi')^{j}]
  \right)
  \frac{}{}.
\end{equation}
In particular, fixing any optimal dual variable $\vecpsi^{*}$, for any
$\vecpsi$ it holds using concavity of $E^{\vecnu,w}_{\lambda,\tau}$ that
\begin{equation}
  \label{eq:concave-suboptimality-gap}
  0 \leq E(\vecpsi^{*}) - E(\vecpsi)
  \leq \sum_{j=1}^{k}w_{k}\left(
    \mathbf{E}_{\nu^{j}}\left[(\psi^{*})^{j} - \psi^{j}\right]
    -
    \mathbf{E}_{\nu^{j}_{\vecpsi}}\left[
      (\psi^{*})^{j} - \psi^{j}
    \right]
  \right).
\end{equation}
This concludes our overview of the background material on
$(\lambda,\tau)$-barycenters.


%%%%%%%%%%%%%%%%%%%%%%%%%%%%%%%%%%%%%%%%%%%%%%%%%%%%%%%%%%%%%%%%%%%%%%%%%%%%%%%

