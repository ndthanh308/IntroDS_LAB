\section{Implementation and Evaluation}


We implement our approach as a prototype tool based on the Clang Static Analyzer~\cite{ClangStaticAnalyzer}.
The implementation includes a front-end that transforms C programs into the input form of our invariant generation solver (i.e., our back-end). 
The front-end 
first transforms affine while loops in C into the canonical form as in Figure~\ref{fig:unnestedPQandRecursive} and then converts the canonical form into an affine transition system.  
The back-end is an extension of StInG~\cite{Sting} written in C++ and uses PPL 1.2~\cite{DBLP:conf/sas/BagnaraRZH02} for polyhedra manipulation (e.g., projection, generator computation, etc.). 
The back-end generates invariants at initial location by applying invariant-generation with Farkas' Lemma and uses invariant propagation method to generate invariants at other locations whenever applicable. 

Notably, our back-end includes two additional features. The first one is the functionality   
to remove invalid transitions with unsatisfiable guard condition $\tsGuardcond$. 
The second one is the treatment of the situation of the unsatisfiability in 
the application of Farkas' Lemma 
(see $-1\ge 0$ at the bottom of Figure~\ref{tab:farkasinit} and Figure~\ref{tab:farkascons}), which is however missing in the original tool StInG~\cite{Sting}. The former can simplify the \LTS{} to improve time efficiency and the later can increase accuracy. 
A key difficulty in the second one is that we obtain polyhedra rather than polyhedral cones, and thus cannot directly apply the generator computation. To address this difficulty, we show that it suffices to consider $\mu=1$ in Figure~\ref{tab:farkascons} and include the generators of both the polytope and the polyhedral cone of the Minkowski decomposition of the polyhedron. As its correctness proof is somewhat technical, we relegate them to Appendix~\ref{sec:appendix_mu1} and Appendix~\ref{sec:appendix_minkowski}.


Below we present the experimental evaluation. We compare our approach with (i) previous related approaches on disjunctive invariant generation, (ii) relevant approaches in loop summary and (iii) state-of-the-art software verifiers including SeaHorn~\cite{SeaHorn}, CPAChecker~\cite{CPAchecker}, Veriabs~\cite{SVCOMP2023Veriabs} (the champion of the reachability track in SV-COMP 2023) and the recent recurrence analysis tool from~\citet{oopsla23}. All the experimental results are obtained from a Linux (Ubuntu 20.04 LTS) with an 11th Gen Intel Core i7 (3.20 GHz) CPU, 32 GB of memory. We choose representative benchmarks related to affine disjunctive invariants and loop summary from the literature
~\cite{FSE2022,DBLP:journals/entcs/AncourtCI10,DBLP:conf/cav/SharmaDDA11,DBLP:conf/sigsoft/XieCLLL16,DBLP:conf/vmcai/BoutonnetH19,DBLP:journals/entcs/HenryMM12} and also SV-COMP, WCET benchmark sets for evaluation. 
Our experimental results are summarized in Table~\ref{tab:related} -- Table~\ref{tab:largebm}.


In all the tables, 
"Our approach" means the results by our approach, 
"Type" means what type of results we obtained, 
"Time" means the runtime measured in seconds, 
"v.s." means the accuracy compared against the previous results. 
For the type of results, we have "Dis" means the result is an invariant (holding at the loop header) obtained by disjuncting all invariants at each location except $\tsLoc_{e}$, 
"Smry" means the result is a loop summary 
where the input variables carry the subscript $0$ (e.g., $x_0$) and the output variables do not carry subscript (e.g., $x$), 
and "LR" means the result is an invariant at the termination location $\tsLoc_{e}$ with a determined fixed-input. 
"Detailed Results" means detailed invariants or summaries for "Dis" or "Smry" or "LR" 
generated from our approach. 
For the accuracy in the column "v.s.", we have "$=$" means that our result is equal to the original result, 
"$>$" means that our result is strictly stronger, 
and "$+$" means that no existing result is available. 
For the symbol in the column "Name", we have "$\circledast$" means an affine nested loop, "$\ddagger$" means that our result is strengthened by incremental method~\cite{DBLP:conf/sat/Bradley12} which is a strategy to strengthen an invariant using previously generated invariants step-by-step, "$\star$" means that our result is obtained by invariant propagation. 
\begin{table}
    \centering
    \caption{Experimental Results on Invariant Generation}
    \label{tab:related}
    \resizebox{\linewidth}{!}{
    \begin{tabular}{|c|c|c|c|c|c|} 
    \hline
    \multicolumn{2}{|c|}{Benchmark} & \multicolumn{4}{c|}{Our Approach} \\ 
    \hline
    \multicolumn{2}{|c|}{Name} & Type & Time & v.s. & Detailed Result \\ 
    \hline
    \multirow{1}{*}{\citet{FSE2022}} & \multirow{1}{*}{fig2$\ \star$} & Dis & 0.02s & {$>$} & \begin{tabular}[c]{@{}c@{}}{(z=0 $\wedge$ 0 $\le$ x $\le$ 1000y-1 $\wedge$ 1 $\le$ y) $\vee$ (x-1000y=z $\wedge$ x-999 $\le$ 1000y $\le$ x $\wedge$ 1 $\le$ y) $\vee$} \\{(z=1000 $\wedge$ 1 $\le$ y $\wedge$ 1000y $\le$ x-1000)}\end{tabular} \\ 
    \hline
    \multirow{6}{*}{\citet{DBLP:journals/entcs/AncourtCI10}} & \multirow{2}{*}{Gopan07$\ \star$} & Dis & $<$0.01s & $+$ & \begin{tabular}[c]{@{}c@{}}(x=y~$\wedge$~0 $\le$ x $\le$ 50)~$\vee$ (x+y=102~$\wedge$ 51 $\le$ x $\le$ 102)\end{tabular} \\ 
    \cline{3-6}
     &  & LR & $<$0.01s & {$>$} & {x=102 $\wedge$ y=-1} \\ 
    \cline{2-6}
     & \multirow{2}{*}{Gulwani07$\ \star$} & Dis & 0.01s & $+$ & \begin{tabular}[c]{@{}c@{}}(y=50~$\wedge$~1 $\le$ x $\le$ 49)~$\vee$ (x=y~$\wedge$ 50 $\le$ x $\le$ 99)\end{tabular} \\ 
    \cline{3-6}
     &  & LR & 0.01s & {$>$} & {x=y=100} \\ 
    \cline{2-6}
     & \multirow{2}{*}{Halbwachs$\ \star$} & Dis & 0.01s & $+$ & 0 $\le$ y $\le$ x $\le$ 100 \\ 
    \cline{3-6}
     &  & LR & 0.01s & {$>$} & \begin{tabular}[c]{@{}c@{}}{(101~$\le$~x~$\le$~102~$\wedge$~0~$\le$~y~$\wedge$~y+2~$\le$~x)~$\vee$~(x=101~$\wedge$~1~$\le$~y~$\le$~101)}\end{tabular} \\ 
    \hline
    \multirow{8}{*}{\citet{DBLP:conf/cav/SharmaDDA11}} & \multirow{2}{*}{POPL07$\ \star$} & Dis & $<$0.01s & {$>$} & \begin{tabular}[c]{@{}c@{}}{(y=50~$\wedge$~0 $\le$ x $\le$ 49)~$\vee$ (x=y~$\wedge$ 50 $\le$ x $\le$ 99)}\end{tabular} \\ 
    \cline{3-6}
     &  & LR & $<$0.01s & {$>$} & {x=y=100} \\ 
    \cline{2-6}
     & \multirow{2}{*}{CAV06$\ \star$} & Dis & 0.01s & $+$ & \begin{tabular}[c]{@{}c@{}}(f=0~$\wedge$~x=y~$\wedge$~0~$\le$ x~$\le$ 50)~$\vee$ (f=0~$\wedge$ x+y=102~$\wedge$ 51~$\le$ x~$\le$ 101)~$\vee$ \\(f=0~$\wedge$ y=0~$\wedge$~x=102)\end{tabular} \\ 
    \cline{3-6}
     &  & LR & 0.01s & $+$ & f=1 $\wedge$ x=102 $\wedge$ y=-1 \\ 
    \cline{2-6}
     & \multirow{2}{*}{ex1$\ \star$} & Dis & 0.02s & $+$ & \begin{tabular}[c]{@{}c@{}}(f=0~$\wedge$~x=y~$\wedge$~0~$\le$~x~$\le$~48)~$\vee$~ (f=0~$\wedge$~x+y=98 $\wedge$~49 $\le$~x~$\le$~98)~$\vee$~ \\(f=0~$\wedge$~y=-1 $\wedge$~x=99)\end{tabular} \\ 
    \cline{3-6}
     &  & LR & 0.02s & $+$ & f=1 $\wedge$ x=99 $\wedge$ y=-2 \\ 
    \cline{2-6}
     & \multirow{2}{*}{ex2$\ \star$} & Dis & 0.02s & $+$ & \begin{tabular}[c]{@{}c@{}}(0 $\le$ x $\le$ 24 $\wedge$ x=y=z) $\vee$ (25 $\le$ x $\le$ 50 $\wedge$ x=y $\wedge$ 5x-100=z) $\vee$ \\(51 $\le$ x $\le$ 99 $\wedge$ x+y=102 $\wedge$ 5x-100=z)\end{tabular} \\ 
    \cline{3-6}
     &  & LR & 0.02s & $+$ & x=100 $\wedge$ y=2 $\wedge$ z=400 \\ 
    \hline
    \multirow{4}{*}{\citet{DBLP:conf/sigsoft/XieCLLL16}} & fig1a$\ \star$ & Dis & 0.01s & $+$ & \begin{tabular}[c]{@{}c@{}}(n=100~$\wedge$~0~$\le$ x~$\le$ 99~$\wedge$~x=z-1)~$\vee$ (n=100~$\wedge$ 1~$\le$~x~$\le$ 99~$\wedge$ x=z)\end{tabular} \\ 
    \cline{2-6}
     & fig6a$\ \star$ & Dis & 0.02s & $+$ & \begin{tabular}[c]{@{}c@{}}(n-1 $\ge$ m $\wedge$ j $\ge$ 0 $\wedge$ i $\ge$ 0 $\wedge$ n-i $\ge$ 1 $\wedge$ m-j $\ge$ 1) $\vee$ \\(m=j $\wedge$ n $\ge$ i+1 $\wedge$ i $\ge$ 0 $\wedge$ n-1 $\ge$ m $\wedge$ m $\ge$ 1)\end{tabular} \\ 
    \cline{2-6}
     & fig1c$\ \star$ & Dis & $<$0.01s & $+$ & 1~$\le$ j~$\le$ m-1~$\wedge$ 0~$\le$ k~$\wedge$ i~$\le$ m-1 \\ 
    \cline{2-6}
     & fig1f$\ \star$ & Dis & 0.01s & $+$ & \begin{tabular}[c]{@{}c@{}}(s=1 $\wedge$ x$_1$ = x$_2$ $\wedge$ 0 $\le$ x$_1$)~$\vee$~(s=2 $\wedge$ x$_1$ = x$_2$ + 1 $\wedge$ 1 $\le$ x$_1$) $\vee$ \\(s=3 $\wedge$ x$_1$ = x$_2$ $\wedge$ 1 $\le$ x$_1$) $\vee$ (s=4 $\wedge$ x$_1$ = x$_2$ $\wedge$ 1 $\le$ x$_1$)\end{tabular} \\ 
    \hline
    \multirow{5}{*}{\citet{DBLP:conf/vmcai/BoutonnetH19}} & eudiv$\ \ddagger \star$ & Dis & 0.01s & $+$ & r $\ge$ b $\ge$ 1 $\wedge$ a $\ge$~q+r~$\wedge$ q $\ge$ 0 \\ 
    \cline{2-6}
     & correct1$\ \star$ & Dis & $<$0.01s & $+$ & s $\ge$ 0 $\wedge$ t $\ge$ 0 $\wedge$ x=o+e \\ 
    \cline{2-6}
     & janne\_complex$\ \circledast$ & Dis & 20.86s & $+$ & \begin{tabular}[c]{@{}c@{}}(55x+11y $\le$ 1686 $\wedge$ x $\le$ y $\wedge$ 481x $\ge$ 241y) $\vee$~\\(y $\le$ 5 $\wedge$ 2x-y $\ge$ 14 $\wedge$ x-y $\le$ 12) $\vee$ (y $\le$ 5 $\wedge$ x-y $\le$ 12 $\wedge$ 65x-29y $\ge$ 420) $\vee$~\\(55x+11y-1686 $\le$ 0 $\wedge$ 1 $\le$ x-y $\le$ 12 $\wedge$ y $\ge$ 6 $\wedge$ 481x+4y $\ge$ 4842 $\wedge$ 3x-y $\ge$ 22)\end{tabular} \\ 
    \cline{2-6}
     & minver$\ \circledast \ddagger \star$ & Dis & $<$0.01s & $+$ & j $\le$ 3i $\le$ 2j $\wedge$ j $\le$ 3 \\ 
    \cline{2-6}
     & fft1$\ \circledast \star$ & Dis & 0.10s & $+$ & \begin{tabular}[c]{@{}c@{}}(n=8 $\wedge$ m=15 $\wedge$ k+2 $\le$ j $\wedge$ k $\le$ 8 $\wedge$ 9 $\le$ j $\le$ 2k $\wedge$ 1 $\le$ i $\le$ 15) $\vee$ \\(n=8 $\wedge$ m=15 $\wedge$ 2 $\le$ j $\le$ 8 $\wedge$ 2 $\le$ i $\le$ 15 $\wedge$ j $\le$ 2k $\wedge$ k $\le$ 8)\end{tabular} \\ 
    \hline
    \citet{DBLP:journals/entcs/HenryMM12} & fig1$\ \star$ & Dis & $<$0.01s & $+$ & \begin{tabular}[c]{@{}c@{}}(2x=t $\wedge$ p=0 $\wedge$ 2x $\le$ 99 $\wedge$ 0 $\le$ x) $\vee$ (2x=t+3 $\wedge$ p=1 $\wedge$ 2 $\le$ x $\le$ 51)\end{tabular} \\
    \hline
    \end{tabular}
    }
\end{table}



First, Table~\ref{tab:related} presents the experimental results on our approach with invariant propagation. Note that the runtime for all benchmarks are mostly very short (within $0.2$s) and thus we only consider the comparison in accuracy. 
In Table~\ref{tab:related}, one can observe that our approach mostly generates invariants with better accuracy. 
In detail, our approach could derive significantly tighter disjunctive invariants
for benchmarks in 
\citet{FSE2022,DBLP:conf/cav/SharmaDDA11}, and  
generate precise LR results of program in disjunctive form for benchmarks in \citet{DBLP:journals/entcs/AncourtCI10,DBLP:conf/cav/SharmaDDA11}. 
On several benchmarks 
(such as \emph{eudiv, minver} in 
\citet{DBLP:conf/vmcai/BoutonnetH19})
that require incremental method,  
we run our approach twice for which the second run generates more invariants based on those obtained in the first run and obtain tighter invariants.
Finally, our approach could also resolve nested loops with complex control flow 
such as \emph{janne\_complex, minver, fft1} in \citet{DBLP:conf/vmcai/BoutonnetH19}. 
We relegate the detailed invariants in the original papers to Appendix~\ref{sec:appendix_invariants}. 

\begin{table}
  \centering
  \caption{Experimental Results on Loop Summary}
  \label{tab:relatedsmry}
  \resizebox{\linewidth}{!}{
  \begin{tabular}{|c|c|c|c|c|c|} 
  \hline
  \multicolumn{2}{|c|}{Benchmark} & \multicolumn{4}{c|}{Our Approach} \\ 
  \hline
  \multicolumn{2}{|c|}{Name} & Type & Time & v.s. & Detailed Result \\ 
  \hline
  \multirow{1}{*}{\citet{FSE2022}} & fig2 $\star$ & \multirow{19}{*}{Smry} & 0.03s & $+$ & \begin{tabular}[c]{@{}c@{}}(x$_0$ $\le$ x $\wedge$ y=y$_0$ $\wedge$ z=z$_0$) $\vee$ \\(x-1000y=z-z$_0$ $\wedge$ y=y$_0$ $\wedge$ x$_0$+1 $\le$ 1000y \\$\wedge$ 1000y $\le$ x $\le$ 1000y+999) $\vee$ \\(z=z$_0$+1000 $\wedge$ y=y$_0$ \\$\wedge$ x$_0$ $\le$ 1000y-1 $\wedge$ 1000y $\le$ x-1000)\end{tabular} \\ 
  \cline{1-2}\cline{4-6}
  \multirow{4}{*}{\citet{DBLP:conf/sigsoft/XieCLLL16}} & fig1a $\star$ $\ddagger$ &  & 0.01s & {$>$} & \begin{tabular}[c]{@{}c@{}}(x=z=n=n$_0$~$\wedge$~x$_0$~$\le$ z$_0$-1 $\wedge$ z$_0$ $\le$ n-1) $\vee$~\\(x$_0$+1 $\le$ x=n=n$_0$ $\le$ z$_0$=z) $\vee$~\\(x=z=n=n$_0$ $\wedge$ z$_0$ $\le$ x$_0$ $\le$ n-1)\end{tabular} \\ 
  \cline{2-2}\cline{4-6}
   & fig6a $\star$ &  & 0.02s & $=$ & i$_0$=j$_0$=0 $\wedge$ m=m$_0$ $\wedge$ i=n=n$_0$ $\wedge$ j=0 $\wedge$ 1 $\le$ m $\le$ n-1 \\ 
  \cline{2-2}\cline{4-6}
   & fig1c $\star$ &  & $<$~0.01s & {$>$} & {1 $\le$ j $\le$ m-1 $\wedge$ m~$ \le$ i $\le$ 2m-2~$\wedge$ 1 $\le$ k $\le$~m-i$_0$} \\ 
  \cline{2-2}\cline{4-6}
   & fig1f $\star$ &  & 0.02s & {$>$} & \begin{tabular}[c]{@{}c@{}}(s=1~$\wedge$~x$_1$-x$_2$ = x$_{10}$-x$_{20}$ $\wedge$ x$_{10}$ $\le$~x$_1$)~$\vee$~\\(s=2~$\wedge$~x$_1$-x$_2$-1 = x$_{10}$-x$_{20}$ $\wedge$~1~$\le$~x$_1$-x$_{10}$)~$\vee$~\\(s=3~$\wedge$~x$_1$-x$_2$ = x$_{10}$-x$_{20}$ $\wedge$~1~$\le$~x$_1$-x$_{10}$)~$\vee$~\\(s=4~$\wedge$~x$_1$-x$_2$ = x$_{10}$-x$_{20}$ $\wedge$~1~$\le$~x$_1$-x$_{10}$)\end{tabular} \\ 
  \cline{1-2}\cline{4-6}
  \multirow{5}{*}{\citet{DBLP:conf/vmcai/BoutonnetH19}} & eudiv $\star$ $\ddagger$ &  & 0.01s & $=$ & a=a$_0$ $\wedge$ b=b$_0$ $\wedge$ r $\ge$ 0 $\wedge$ b $\ge$ r+1 $\wedge$ a+1~$\ge$ b+q+r~$\wedge$ q $\ge$ 1 \\ 
  \cline{2-2}\cline{4-6}
   & correct1 $\star$ $\ddagger$ &  & $<$~0.01s & $+$ & \begin{tabular}[c]{@{}c@{}}(x-x$_0$=e-e$_0$ $\wedge$ x$_0$=o+e$_0$ $\wedge$ t $\ge$ 0 $\wedge$ x-x$_0$-t+s+e$_0$ $\ge$ 1 \\$\wedge$ x$_0$ $\ge$ x+s $\wedge$ x$_0$+t $\ge$ e$_0$+x)\end{tabular} \\ 
  \cline{2-2}\cline{4-6}
   & janne\_complex $\circledast$ &  & 34.34s & $+$ & \begin{tabular}[c]{@{}c@{}}(x$_0$ $\le$ 29 $\wedge$ y$_0$ $\le$ 5 $\wedge$ y$_0$ $\le$ x$_0$ - 1 $\wedge$ x $\le$ y + 12 \\$\wedge$ -36x-12x$_0$+y-18y$_0$ $\ge$ -1811 $\wedge$ -36x-61x$_0$+y-18y$_0$ $\ge$ -3036 \\$\wedge$ -107x-52x$_0$-12y-78y$_0$+6639 $\ge$ 0 $\wedge$ 3x-y $\ge$ 22 \\$\wedge$ x $\ge$ 30 $\wedge$ 2x-2x$_0$+y$_0$ $\ge$ 12 $\wedge$ x-x$_0$ $\ge$ 4) $\vee$~\\(x$_0$ $\le$ 29 $\wedge$ y$_0$ $\ge$ x$_0$ $\wedge$ 30 $\le$ x $\le$ 31 $\wedge$ x $\le$ y+12 \\$\wedge$ 5x-5x$_0$+y-y$_0$ $\ge$ 0 $\wedge$ 13x-13x$_0$-3y+3y$_0$ $\ge$ 0 \\$\wedge$ 127x-155x$_0$-25y+25y$_0$+308 $\ge$ 0 \\$\wedge$ 297x-297x$_0$-47y+47y$_0$-1064 $\ge$ 0)\end{tabular} \\ 
  \cline{2-2}\cline{4-6}
   & minver $\star$ $\circledast$ &  & 0.01s & $+$ & \begin{tabular}[c]{@{}c@{}}(i$_0$ $\le$ 2 $\wedge$ j$_0$ $\le$ 2 $\wedge$ i=j=3) $\vee$ \\(i$_0$ $\le$ 2 $\wedge$ j$_0$ $\ge$ 3 $\wedge$ i=3 $\wedge$ j=j$_0$)\end{tabular} \\ 
  \cline{2-2}\cline{4-6}
   & fft1 $\star$ $\circledast$ &  & 0.10s & $+$ & \begin{tabular}[c]{@{}c@{}}(n=n$_0$ $\wedge$ m=m$_0$ $\wedge$ i$_0$+1 $\le$ i=m+1 $\wedge$ j$_0$ $\ge$ n+1 \\$\wedge$ k+1 $\le$ j $\le$ 2k $\wedge$ 3k+1 $\le$ j+n) $\vee$\\(n=n$_0$ $\wedge$ m=m$_0$ $\wedge$ i=m+1 $\ge$ i$_0$+2 \\$\wedge$ 2k $\ge$ j $\ge$ k+1 $\wedge$ 3k+1 $\le$ j+n $\wedge$ j$_0$ $\le$ n) $\vee$\\(k=n=n$_0$ $\wedge$ m=m$_0$ $\wedge$ i=m+1 $\ge$ i$_0$+1 \\$\wedge$ 2k $\ge$ j $\wedge$ k $\ge$ j$_0$ $\wedge$ k $\ge$ j)\end{tabular} \\ 
  \cline{1-2}\cline{4-6}
  \multirow{5}{*}{WCET\cite{Gustafsson:WCET2010:Benchmarks}} & cnt\_cover $\star$ &  & $<$~0.01s & $+$ & c=c$_0$+10 $\wedge$ cnt=cnt$_0$+10 \\ 
  \cline{2-2}\cline{4-6}
   & cnt\_minver $\star$ $\circledast$ &  & 0.05s & $+$ & \begin{tabular}[c]{@{}c@{}}(i$_0$=2 $\wedge$ j$_0$ $\le$ 2 $\wedge$ i=j=3 $\wedge$ cnt$_1$=1) $\vee$ \\(i$_0$ $\le$ 1 $\wedge$ j$_0$ $\le$ 2 $\wedge$ i=j=3) $\vee$ \\(i$_0$ $\le$ 2 $\wedge$ j $\ge$ 3 $\wedge$ i=3 $\wedge$ j=j$_0$ $\wedge$ i$_0$+cnt$_2$=3)\end{tabular} \\ 
  \cline{2-2}\cline{4-6}
   & cnt\_fft1 $\star$ $\circledast$ &  & 0.20s & $+$ & \begin{tabular}[c]{@{}c@{}}(n=n$_0$ $\wedge$ m=m$_0$ $\wedge$ i$_0$+cnt$_1$=i=m+1 $\wedge$ j$_0$ $\ge$ n+1 \\$\wedge$ k+1 $\le$ j $\le$ 2k $\wedge$ 3k+1 $\le$ j+n $\wedge$ i-i$_0$ $\ge$ cnt$_2$ $\ge$ 1) $\vee$\\(n=n$_0$ $\wedge$ m=m$_0$ $\wedge$ i=m+1=i$_0$+cnt$_1$ $\wedge$ 2k $\ge$ j $\ge$ k+1 \\$\wedge$ 3k+1 $\le$ j+n $\wedge$ j$_0$ $\le$ n $\wedge$ i-i$_0$-1 $\ge$ cnt$_2$ $\ge$ 1) $\vee$\\(k=n=n$_0$ $\wedge$ m=m$_0$ $\wedge$ i=m+1=i$_0$+cnt$_1$ $\wedge$ 2k $\ge$ j \\$\wedge$ k $\ge$ j$_0$ $\wedge$ k $\ge$ j $\wedge$ i-i$_0$-1 $\ge$ cnt$_2$ $\ge$ 0)\end{tabular} \\ 
  \cline{1-2}\cline{4-6}
  \multirow{8}{*}{SPEED\cite{DBLP:conf/popl/GulwaniMC09}} & cnt\_SimpleSingle $\star$ &  & $<$~0.01s & $+$ & \begin{tabular}[c]{@{}c@{}} (x=n$_0$ $\wedge$ cnt$_1$ = cnt$_2$ + cnt$_3$ $\wedge$ x = x$_0$ + cnt$_2$ + cnt$_3$ \\$\wedge$ x = n $\wedge$ cnt$_2$ $\ge$ 1 $\wedge$ cnt$_3$ $\ge$ 0) $\vee$ \\(x=n$_0$ $\wedge$ cnt$_1$ = cnt$_2$ + cnt$_3$ $\wedge$ x = x$_0$ + cnt$_2$ + cnt$_3$ \\$\wedge$ x = n $\wedge$ cnt$_2$ $\ge$ 0 $\wedge$ cnt$_3$ $\ge$ 1) \end{tabular} \\ 
  \cline{2-2}\cline{4-6}
    & cnt\_SimpleSingle2 $\star$ &  & 0.10s & $+$ & \begin{tabular}[c]{@{}c@{}}(cnt$_1$ = cnt$_2$ $\wedge$ m = m$_0$ $\wedge$ x = x$_0$ + cnt$_2$ $\wedge$ cnt3 = 0 \\$\wedge$ y = y$_0$ +cnt$_2$ $\wedge$ x = n $\wedge$ x = n$_0$ $\wedge$ cnt$_2$ $\ge$ 1 $\wedge$ y $\ge$ m ) $\vee$\\(y = m $\wedge$ y = m$_0$ $\wedge$ x = x$_0$ + cnt$_1$ $\wedge$ x = x$_0$ + cnt$_2$ + cnt$_3$ \\$\wedge$ x = x$_0$ + y - y$_0$ $\wedge$ x$_0$ = n - cnt$_2$ \\$\wedge$ x$_0$ = n$_0$ - cnt$_2$ $\wedge$ cnt$_2$ $\ge$ 1 $\wedge$ x $\ge$ x$_0$ + cnt$_2$ + 1) $\vee$~\\(y = m$_0$ $\wedge$ cnt$_2$ = 0 $\wedge$ x = x$_0$ +cnt$_1$ $\wedge$ x = x$_0$ + cnt$_3$ \\$\wedge$ x = x$_0$ + y - y$_0$ $\wedge$ y = m $\wedge$ n = n$_0$ $\wedge$ x $\ge$ n + 1 $\wedge$ x $\ge$ x$_0$ + 1 )\end{tabular} \\ 
  \cline{2-2}\cline{4-6}
    & cnt\_SimpleMultiple $\star$ &  & 0.05s & $+$ & \begin{tabular}[c]{@{}c@{}}(y = m$_0$ $\wedge$ y = m $\wedge$ x = x$_0$ + cnt$_1$ - cnt$_2$ $\wedge$ cnt$_1$ = cnt$_2$ + cnt$_3$ \\$\wedge$ y = y$_0$ + cnt$_2$ $\wedge$ x = n $\wedge$ x = n$_0$ $\wedge$ cnt$_1$ $\ge$ cnt$_2$ + 1 $\wedge$ cnt$_2$ $\ge$ 1) $\vee$\\(x = x$_0$ + cnt$_1$ $\wedge$ cnt$_2$ = 0 $\wedge$ m = m$_0$ $\wedge$ x = x$_0$ + cnt$_3$ \\$\wedge$ y = y$_0$ $\wedge$ x = n $\wedge$ x = n$_0$ $\wedge$ x $\ge$ x$_0$ + 1 $\wedge$ y $\ge$ m)\end{tabular} \\ 
  \cline{2-2}\cline{4-6}
    & cnt\_NestedMultiple $\star$ $\circledast$ &  & 0.03s & $+$ & \begin{tabular}[c]{@{}c@{}}(y=m$_0$ $\wedge$ cnt$_2$=1 $\wedge$ x=x$_0$+1 $\wedge$ cnt$_3$=0 \\$\wedge$ cnt$_1$=1 $\wedge$ x=n $\wedge$ x=n$_0$ $\wedge$ y=m $\wedge$ y $\ge$ y$_0$+1) $\vee$\\(y=m$_0$ $\wedge$ cnt$_2$=1 $\wedge$ x=x$_0$+cnt$_1$ $\wedge$ cnt$_1$=cnt$_3$+1 \\$\wedge$ y=m $\wedge$ x=n $\wedge$ x=n$_0$ $\wedge$ y $\ge$ y$_0$+1 $\wedge$ cnt$_1$ $\ge$ 2) $\vee$\\(x=x$_0$+cnt$_1$ $\wedge$ cnt$_2$=0 $\wedge$ m=m$_0$ $\wedge$ x=x$_0$+cnt$_3$ \\$\wedge$ y=y$_0$ $\wedge$ x=n $\wedge$ x=n$_0$ $\wedge$ x$\ge$ x$_0$+1 $\wedge$ y $\ge$ m)\end{tabular} \\ 
  \hline
  \end{tabular}
  }
\end{table}

Second, Table~\ref{tab:relatedsmry} presents the experimental results on affine disjunctive loop summary. We first compare our generated loop summaries with existing results in \citet{DBLP:conf/sigsoft/XieCLLL16} and \citet{DBLP:conf/vmcai/BoutonnetH19} (for eudiv, correct1), and find that our approach mostly generate more accurate loop summaries. Then we test our approach on WCET benchmarks in \citet{Gustafsson:WCET2010:Benchmarks} related to affine loop summary and adapt Speed benchmarks in ~\citet{DBLP:conf/popl/GulwaniMC09} to affine runtime behaviour by fixing the number of loop iterations in either the outer or the inner loop. In these benchmarks, we use a special variable \emph{$\mathsf{cnt}$} to represent the number of loop iterations of an outer/inner loop. 
The results for these benchmarks were previously not reported, and our results show that our approach generates precise affine disjunctive loop summaries for these benchmarks.

\begin{table}
  \centering
  \caption{Experiment for SeaHorn, CPAChecker, VeriAbs and OOPSLA23}
  \label{tab:seahorn}
  \resizebox{\linewidth}{!}{
  \begin{tabular}{|c|c|c|c|c|c|c|c|c|c|c|c|} 
  \hline
  \multicolumn{2}{|c|}{Benchmark} & \multicolumn{2}{c|}{Our Approach} & \multicolumn{2}{c|}{SeaHorn} & \multicolumn{2}{c|}{CPAChecker} & \multicolumn{2}{c|}{VeriAbs} & \multicolumn{2}{c|}{OOPSLA23} \\ 
  \hline
  \multicolumn{2}{|c|}{Name} & Proof & Time (s) & Proof & Time (s) & Proof & Time (s) & Proof & Time (s) & Proof & Time (s) \\ 
  \hline
  \citet{FSE2022} & fig2$\star$ & \multirow{50}{*}{T} & 0.02 & F & $>36000$ & F & 3 & F & 27 & F & 1 \\ 
  \cline{1-2}\cline{4-12}
  \multirow{3}{*}{\citet{DBLP:journals/entcs/AncourtCI10}} & Gopan07$\star$ &  & 0.01 & F & $>36000$ & T & 17 & T & 30 & F & 1 \\ 
  \cline{2-2}\cline{4-12}
   & Halbwachs$\star$ &  & 0.01 & F & $>36000$ & T & 30 & T & 33 & F & 1 \\ 
  \cline{2-2}\cline{4-12}
   & Gulwani07$\star$ &  & 0.01 & T & 1 & T & 16 & T & 18 & T & 2 \\ 
  \cline{1-2}\cline{4-12}
  \multirow{4}{*}{\citet{DBLP:conf/cav/SharmaDDA11}} & CAV06$\star$ &  & $<0.01$ & F & $>36000$ & T & 12 & T & 35 & F & 1 \\ 
  \cline{2-2}\cline{4-12}
   & ex1$\star$ &  & 0.01 & F & $>36000$ & T & 14 & T & 35 & F & 1 \\ 
  \cline{2-2}\cline{4-12}
   & POPL07$\star$ &  & 0.02 & T & 1 & T & 11 & T & 19 & T & 2 \\ 
  \cline{2-2}\cline{4-12}
   & ex2$\star$ &  & 0.02 & T & 2 & T & 13 & T & 20 & T & 2 \\ 
  \cline{1-2}\cline{4-12}
  \multirow{4}{*}{\citet{DBLP:conf/sigsoft/XieCLLL16}} & fig1a$\star$ &  & 0.01 & F & 1 & F & 10 & F & 21 & F & 4 \\ 
  \cline{2-2}\cline{4-12}
   & fig1c$\star$ &  & $<0.01$ & F & 1 & F & 10 & F & 27 & F & 1 \\ 
  \cline{2-2}\cline{4-12}
   & fig6a$\star$ &  & 0.02 & F & 1 & F & 11 & F & 874 & F & 1 \\ 
  \cline{2-2}\cline{4-12}
   & fig1f$\star$ &  & 0.01 & T & 1 & F & 445 & F & 513 & T & 3 \\ 
  \cline{1-2}\cline{4-12}
  \multirow{5}{*}{\citet{DBLP:conf/vmcai/BoutonnetH19}} & eudiv$\star \ddagger$ &  & 0.01 & F & 1 & F & 10 & F & 18 & F & 1 \\ 
  \cline{2-2}\cline{4-12}
   & janne\_complex$\circledast$ &  & 28.24 & F & 1 & F & 9 & F & 16 & F & 1 \\ 
  \cline{2-2}\cline{4-12}
   & minver$\star \ddagger \circledast$ &  & $<0.01$ & F & 1 & F & 9 & F & 898 & F & 2 \\ 
  \cline{2-2}\cline{4-12}
   & fft1$\star \circledast$ &  & 0.08 & F & 1 & F & 9 & F & 1 & F & 1 \\ 
  \cline{2-2}\cline{4-12}
   & correct1$\star$ &  & 0.01 & F & 1 & T & 10 & F & 19 & F & 2 \\ 
  \cline{1-2}\cline{4-12}
  \citet{DBLP:journals/entcs/HenryMM12} & fig1$\star$ &  & 0.01 & T & 1 & T & 17 & T & 18 & F & 2 \\ 
  \cline{1-2}\cline{4-12}
  \multirow{32}{*}{\citet{svcomp}} & benchmark44\_disjunctive.c$\star$ &  & 0.01 & F & 1 & F & 1 & F & 32 & F & 1 \\ 
  \cline{2-2}\cline{4-12}
   & count\_by\_nondet.c$\star$ &  & 0.01 & F & $>36000$ & F & 1 & F & 901 & F & 3 \\ 
  \cline{2-2}\cline{4-12}
   & mono-crafted\_6.c$\star$ &  & 0.01 & F & $>36000$ & F & 1 & T & 236 & T & 2 \\ 
  \cline{2-2}\cline{4-12}
   & mono-crafted\_9.c$\star$ &  & 0.01 & F & $>36000$ & F & 1 & T & 270 & T & 2 \\ 
  \cline{2-2}\cline{4-12}
   & mono-crafted\_13.c$\star$ &  & 0.01 & F & $>36000$ & F & 1 & T & 209 & T & 2 \\ 
  \cline{2-2}\cline{4-12}
   & Mono4\_1.c$\star$ &  & 0.01 & F & $>36000$ & F & 1 & F & 403 & T & 2 \\ 
  \cline{2-2}\cline{4-12}
   & Mono5\_1.c$\star$ &  & 0.01 & F & $>36000$ & F & 1 & F & 405 & T & 3 \\ 
  \cline{2-2}\cline{4-12}
   & Mono6\_1.c$\star$ &  & 0.02 & F & $>36000$ & F & 1 & F & 403 & T & 2 \\ 
  \cline{2-2}\cline{4-12}
   & gcnr2008.c$\star$ &  & 0.02 & F & 1 & F & 1 & F & 62 & F & 1 \\ 
  \cline{2-2}\cline{4-12}
   & gr2006.c$\star$ &  & 0.02 & F & $>36000$ & T & 17 & T & 85 & F & 1 \\ 
  \cline{2-2}\cline{4-12}
   & benchmark07\_linear.c$\star$ &  & 0.01 & T & 1 & T & 10 & F & 30 & F & 1 \\ 
  \cline{2-2}\cline{4-12}
   & benchmark21\_disjunctive.c$\star$ &  & 0.01 & T & 1 & T & 11 & T & 31 & F & 1 \\ 
  \cline{2-2}\cline{4-12}
   & benchmark32\_linear.c$\star$ &  & 0.01 & T & 1 & T & 8 & T & 17 & F & 1 \\ 
  \cline{2-2}\cline{4-12}
   & benchmark51\_polynomial.c$\star$ &  & 0.01 & T & 1 & T & 10 & T & 17 & F & 1 \\ 
  \cline{2-2}\cline{4-12}
   & afnp2014.c$\star$ &  & 0.01 & T & 1 & T & 66 & T & 22 & T & 2 \\ 
  \cline{2-2}\cline{4-12}
   & eq1.c$\star$ &  & 0.01 & T & 1 & T & 10 & F & 223 & F & 1 \\ 
  \cline{2-2}\cline{4-12}
   & gj2007.c$\star$ &  & 0.01 & T & 1 & T & 11 & T & 51 & T & 2 \\ 
  \cline{2-2}\cline{4-12}
   & nested\_5.c$\star \circledast$ &  & 0.03 & T & 2 & T & 8 & T & 23578 & F & 1 \\ 
  \cline{2-2}\cline{4-12}
   & terminator\_02-2.c$\star$ &  & 0.02 & T & 1 & T & 9 & T & 19 & F & 1 \\ 
  \cline{2-2}\cline{4-12}
   & nested\_6.c$\star \circledast$ &  & 0.02 & T & 1 & T & 9 & T & 15 & F & 1 \\ 
  \cline{2-2}\cline{4-12}
   & sum01\_bug02.c$\star$ &  & 0.01 & T & 1 & T & 9 & F & 1 & F & 1 \\ 
  \cline{2-2}\cline{4-12}
   & sum01\_bug02\_sum01\_bug02\_base.case.c$\star$ &  & 0.01 & T & 1 & T & 9 & F & 1 & F & 1 \\ 
  \cline{2-2}\cline{4-12}
   & nested\_delay\_notd2.c$\star$ &  & 0.02 & T & 1 & F & 1 & F & 208 & F & 1 \\ 
  \cline{2-2}\cline{4-12}
   & benchmark06\_conjunctive.c$\star$ &  & 0.01 & T & 2 & F & 1 & F & 853 & F & 1 \\ 
  \cline{2-2}\cline{4-12}
   & benchmark31\_disjunctive.c$\star$ &  & 0.01 & T & 2 & F & 1 & F & 30 & F & 1 \\ 
  \cline{2-2}\cline{4-12}
   & benchmark45\_disjunctive.c$\star$ &  & 0.01 & T & 2 & F & 1 & F & 28 & F & 1 \\ 
  \cline{2-2}\cline{4-12}
   & benchmark46\_disjunctive.c$\star$ &  & 0.01 & T & 2 & F & 1 & F & 29 & F & 1 \\ 
  \cline{2-2}\cline{4-12}
   & benchmark47\_linear.c$\star$ &  & 0.01 & T & 1 & F & 1 & F & 33 & F & 1 \\ 
  \cline{2-2}\cline{4-12}
   & bhmr2007.c$\star$ &  & 0.02 & T & 1 & F & 1 & F & 899 & F & 1 \\ 
  \cline{2-2}\cline{4-12}
   & cggmp2005\_variant.c$\star$ &  & 0.01 & T & 1 & F & 1 & T & 193 & T & 2 \\ 
  \cline{2-2}\cline{4-12}
   & ddlm2013.c$\star$ &  & 0.02 & T & 1 & F & 1 & F & 901 & T & 2 \\ 
  \cline{2-2}\cline{4-12}
   & half.c$\star$ &  & 0.01 & T & 1 & F & 1 & F & 811 & T & 2 \\ 
  \hline
  \end{tabular}
  }
\end{table}



Third, Table~\ref{tab:seahorn} presents the comparison with the state-of-the-art 
software verifiers 
SeaHorn~\cite{SeaHorn}, ~CPAChecker~\cite{CPAchecker}, Veriabs~\cite{SVCOMP2023Veriabs} and the tool from \citet{oopsla23} (the column "OOPSLA23" in the table) . We first have the comparison over the benchmarks in Table~\ref{tab:related} (i.e., the benchmarks except for "SV-COMP" in Table~\ref{tab:seahorn}). For this part of benchmarks, since both SeaHorn and CPAChecker require the user to provide a goal property, we feed them simple goal properties such as the equality between variables and constants (e.g., $x=y$, $x=100$, etc.) arising from the disjunctive feature of the benchmarks. Then we choose representative benchmarks with disjunctive feature from the categories \emph{loop-new, loop-lit, loop-crafted-1, loops, loop-invariants, loop-zilu, loop-simple} of SV-COMP and compare the results between our approach and Seahorn/CPAChecker. 
These benchmarks from SV-COMP covers typical disjunctive features including \emph{multi-phase loop, loop with if-else or if-else-break, loop with nondeterminism-branch, loop with switch-case, loop under non-initialized variables, mode transition, nested loops}. 
We keep the original assertions for the benchmarks from SV-COMP. 
In the table, the columns "SeaHorn"/"CPAChecker"/"Veriabs"/"OOPSLA23" mean the results generated by SeaHorn/CPAChecker/Veriabs/OOPSLA23, and the "Proof" column specifies whether the tool could verify the given assertion for which 
the symbol "F" (resp. "T") means the obtained results are incapable (resp. capable) of checking the assertions respectively. We set a time-out of 10 hours in this table.
One can observe that these tools fail on most of the benchmarks even if these benchmarks are at a small scale, while our approach succeeds in checking the assertions in all the benchmarks and is substantially more time efficient. We find that the reasons behind these tools include failure to handle \textbf{break}-statement such as \emph{Gopan07}, incapability to handle non-initialized variables such as \emph{fig1a, fig6a, fig1c, eudiv, correct1, janne\_complex, minver, fft1}, incompetence to handle disjunction such as \emph{Halbwachs}, insufficient to handle nested loops with complex control flow such as \emph{janne\_complex, minver, fft1}, 
etc. 
\begin{table}[H]
    \centering
    \caption{Experiment for Invariant Propagation}
    \label{tab:largebm}
    \resizebox{0.6\linewidth}{!}{
    \begin{tabular}{|c|c|c|c|c|c|c|} 
    \hline
    \multicolumn{4}{|c|}{\multirow{2}{*}{Benchmark}} & \multicolumn{3}{c|}{Our Approach} \\ 
    \cline{5-7}
    \multicolumn{4}{|c|}{} & No PPG & \multicolumn{2}{c|}{PPG} \\ 
    \hline
    \multicolumn{2}{|c|}{Name} & Loc & Dim & Time (s) & Time (s) & Speedup \\ 
    \hline
    \multirow{7}{*}{POPL07$\star$ \cite{DBLP:conf/cav/SharmaDDA11}} & 3p & 3 & 9 & $<$0.01 & $<$0.01 & 1.00X \\ 
    \cline{2-7}
     & 4p & 4 & 16 & 0.05 & 0.04 & 1.25X \\ 
    \cline{2-7}
     & 5p & 5 & 25 & 0.33 & 0.05 & 6.60X \\ 
    \cline{2-7}
     & 6p & 6 & 36 & 3.32 & 0.09 & 36.89X \\ 
    \cline{2-7}
     & 7p & 7 & 49 & 35.40 & 0.21 & 168.57X \\ 
    \cline{2-7}
     & 8p & 8 & 64 & 359.21 & 0.40 & 898.03X \\ 
    \cline{2-7}
     & 9p & 9 & 81 & 2900.43 & 0.84 & 3452.89X \\
    \hline
    \end{tabular}
    }
\end{table}

\smallskip
\smallskip

Finally, 
Table~\ref{tab:largebm} demonstrates the improvement of speedup by our invariant propagation technique. 
In Table~\ref{tab:largebm}, "$r$-p" means that $r$ is a benchmark-inside number to show how many locations are there in the \LTS, 
"Loc" means the number of locations under \LTS, 
"Dim" means the number of  
unknown coefficients at all locations, 
"No PPG" means using our disjunctive affine invariant generation over each location under \LTS{} without invariant propagation (i.e., following the original approach in ~\citet{oopsla22/scalable}), 
"PPG" means using our invariant propagation, 
"Time(s)" means the runtime measured in seconds, and 
"Speedup" means the ratio of time consumed by "No PPG" against "PPG". 
The experimental results in Table~\ref{tab:largebm} show that our invariant propagation could substantially improve the time efficiency over large benchmarks. 

\begin{remark}[Other Related Approaches]
We are unable to have direct comparison with the very related work \citet{DBLP:conf/vmcai/BoutonnetH19,FSE2022,DBLP:journals/entcs/HenryMM12,DBLP:conf/sigsoft/XieCLLL16,DBLP:conf/tase/LinZCSXLS21} due to the following reasons. First, the works \citet{DBLP:conf/vmcai/BoutonnetH19,DBLP:conf/sigsoft/XieCLLL16,DBLP:conf/tase/LinZCSXLS21} neither publicize their implementation nor report the detailed invariants in some key benchmarks such as \emph{janne\_complex, minver, fft1}. Second, although the tool PAGAI~\cite{DBLP:journals/entcs/HenryMM12} claims the functionality of disjunctive invariant generation, we find that this functionality could not work in the disjunctive-invariant-generation mode.
Third, the tool in~\citet{FSE2022} accepts only the smtlib format of the CHC solver and has a preprocessing on the original CHC input, making the recovery of the original loop information difficult. We have tried the submodules of SeaHorn~\cite{SeaHorn} and Eldarica~\cite{Eldarica} to transform several simple examples (e.g., \emph{Gopan07} and \emph{POPL07}) in this paper into their CHC format, but this tool does not terminate on the CHC inputs of these simple examples. 
We also note that machine learning approaches~\cite{DBLP:conf/iclr/RyanWYGJ20, DBLP:conf/pldi/YaoRWJG20, DBLP:conf/nips/SiDRNS18} could also generate disjunctive invariants, but we found robustness problem that a slight deviation in a simple program (without changing the branch structure in the loop) can cause these approaches non-terminating. Our approach is based on constraint solving and therefore does not have this robustness issue. \qed
\end{remark}
