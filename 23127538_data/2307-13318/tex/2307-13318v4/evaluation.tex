\section{Experimental Evaluation}
\label{sec:exp}
In this section, we present the evaluation of the implementation (referred to as {\ToolName}) of our approach to generate disjunctive affine invariants. We focus on the following two questions (\textbf{RQ1} and \textbf{RQ2}).


\begin{itemize}
    \item \textbf{RQ1:} How competitive is \ToolName\ when compared with other approaches?
    \item \textbf{RQ2:} How effective does invariant propagation enhance our approach?
\end{itemize}
\subsection{Experimental Setup}

\smalltitle{Implementation.} We implement our approach (including the algorithmic techniques in Section~\ref{sec:alg}) as a prototype \ToolName, dividing the implementation into front-end and back-end. The front-end utilizes Clang Static Analyzer~\cite{ClangStaticAnalyzer} to extract and transform C programs, processing programs into the format required by the back-end. The back-end is an extension of StInG~\cite{Sting} written in C++ and uses PPL 1.2~\cite{DBLP:conf/sas/BagnaraRZH02} for polyhedra manipulation (e.g., projection, generator computation, etc.), which generates invariants and propagate them to obtain a disjuntive invariant as the loop invariant.

\smalltitle{Environment.} All experiments are conducted on a machine equipped with a 12th-generation Intel(R) Core(TM) i7-12800HX CPU, 16 cores, 2304 MHz, 9.5GB RAM, running Ubuntu 20.04 (LTS). Following the competition settings of SV-COMP, for studies \textbf{RQ1} and \textbf{RQ2}, we impose a time limit of 900s. 


\smalltitle{Benchmarks.} We have a total of 114 affine programs, $38.6\%$ of which have disjunctive features,
sourced from: 1) 105 benchmarks from the SV-COMP, ReachSafety-Loop track. We excluded those with arrays, pointers, and other non-numeric features, those with modulus, division, polynomial, and other non-linear operations. 2) 9 benchmarks from the recent paper~\cite{DBLP:conf/vmcai/BoutonnetH19}, which include complex nested loops and examples with disjunctive features.

\smalltitle{Methodology.} 
In \textbf{RQ1}, we compare \ToolName\ utilizing invariant propagation techniques with several state-of-the-art software verifiers:

\begin{itemize}
    \item Veriabs~\cite{SVCOMP2023Veriabs} is a state-of-the-art software verifier that is an integration of various strategies such as fuzz testing, $k$-induction, loop shrinking, loop pruning, full-program induction, explicit state model checking and other invariant generation techniques, which is capable to deal with programs with disjunctive features.
    \item CPAChecker~\cite{CPAchecker} is a well-developed software verifier that is based on bounded model checking and interpolation and has a comprehensive ability to verify various kinds of properties. 
    \item OOPSLA23~\cite{oopsla23} is a recent recurrence analysis tool that handles only loops with the ultimate strict alternation pattern that eventually the loop will alternate between different modes periodically and performs good on such class of programs, which thus excels in the verification of disjunctive programs with alternating modes.
    \item DIG~\cite{ICSE2022dig} is an invariant generation tool considering disjunctive features in programs and utilizes front-end CIVL~\cite{DIG_CIVL} to obtain symbolic execution traces. It employs dynamic analysis along with efficient algorithms from algebra and geometry to solve numerical invariant templates, thereby generating numerical invariants at any position within a program, which is capable of extensively handling the programs with array, nonlinear, linear and disjunctive features.
    \item IKOS with \emph{Polyset} domain from PPLite~\cite{IKOS,PPLite_domain} is a classic abstract interpretation framework with various interface supports. The \emph{Polyset} abstract domain is an efficient implementation of the powerset of polyhedra and serves as an alternative to the trace partitioning strategy implemented in Astree~\cite{ASTree}.
\end{itemize}

In \textbf{RQ2}, we focus on comparing the impact of the invariant propagation technique on the time efficiency. By contrasting the tool's performance when calculating invariants for each location individually against using invariant propagation, we analyze the role of invariant propagation.

\subsection{Tool Comparison (RQ1)}

Our work primarily focuses on the generation of disjunctive invariants, whereas tools like CPAChecker and Veriabs are specifically designed as bug finders for verifying assertions. However, by integrating the PPL library~\cite{DBLP:conf/sas/BagnaraRZH02} and Z3~\cite{z3}, we use the generated invariants to verify the correctness of assertions and demonstrate the precision of the invariants generated by \ToolName.


\begin{table}[t]
\centering
\small
\resizebox{0.95\textwidth}{!}{
\begin{tabular}{|cc|ccc|ccc|ccc|}
\hline
\multicolumn{2}{|c|}{Benchmark}                                                    & \multicolumn{3}{c|}{\ToolName }                                               & \multicolumn{3}{c|}{Veriabs}                                         & \multicolumn{3}{c|}{CPAChecker}                                                           \\ \hline
\multicolumn{1}{|c|}{Source}                       & \#Num                         & \multicolumn{1}{c|}{\#Ver.} & \multicolumn{1}{c|}{\#Unk.} & Time (s)                        & \multicolumn{1}{c|}{\#Ver.} & \multicolumn{1}{c|}{\#Unk.} & Time (s) & \multicolumn{1}{c|}{\#Ver.} & \multicolumn{1}{c|}{\#Unk.} & Time (s)                      \\ \hline
\multicolumn{1}{|c|}{loop-invariants}              & 5                             & \multicolumn{1}{c|}{4}      & \multicolumn{1}{c|}{1}      & 0.47                            & \multicolumn{1}{c|}{5}      & \multicolumn{1}{c|}{0}      & 153.31   & \multicolumn{1}{c|}{4}      & \multicolumn{1}{c|}{1}      & 1001.49                       \\ \hline
\multicolumn{1}{|c|}{loop-new}                     & 2                             & \multicolumn{1}{c|}{2}      & \multicolumn{1}{c|}{0}      & 0.11                            & \multicolumn{1}{c|}{0}      & \multicolumn{1}{c|}{2}      & 959.74   & \multicolumn{1}{c|}{0}      & \multicolumn{1}{c|}{2}      & 1807.20                       \\ \hline
\multicolumn{1}{|c|}{loop-invgen}                  & 5                             & \multicolumn{1}{c|}{4}      & \multicolumn{1}{c|}{1}      & 0.41                            & \multicolumn{1}{c|}{5}      & \multicolumn{1}{c|}{0}      & 160.51   & \multicolumn{1}{c|}{0}      & \multicolumn{1}{c|}{5}      & 4518.19                       \\ \hline
\multicolumn{1}{|c|}{loops-crafted-1}              & 25                            & \multicolumn{1}{c|}{20}     & \multicolumn{1}{c|}{5}      & 4.84                            & \multicolumn{1}{c|}{25}     & \multicolumn{1}{c|}{0}      & 4010.55  & \multicolumn{1}{c|}{0}      & \multicolumn{1}{c|}{25}     & 22607.55                      \\ \hline
\multicolumn{1}{|c|}{loop-simple}                  & 2                             & \multicolumn{1}{c|}{1}      & \multicolumn{1}{c|}{1}      & 2                               & \multicolumn{1}{c|}{1}      & \multicolumn{1}{c|}{1}      & 944.69   & \multicolumn{1}{c|}{1}      & \multicolumn{1}{c|}{1}      & 919.03                        \\ \hline
\multicolumn{1}{|c|}{loop-zilu}                    & 26                            & \multicolumn{1}{c|}{26}     & \multicolumn{1}{c|}{0}      & 0.77                            & \multicolumn{1}{c|}{25}     & \multicolumn{1}{c|}{1}      & 1064.60  & \multicolumn{1}{c|}{26}     & \multicolumn{1}{c|}{0}      & 307.18                        \\ \hline
\multicolumn{1}{|c|}{loops}                        & 18                            & \multicolumn{1}{c|}{15}     & \multicolumn{1}{c|}{3}      & 3.26                            & \multicolumn{1}{c|}{17}     & \multicolumn{1}{c|}{1}      & 536.33   & \multicolumn{1}{c|}{17}     & \multicolumn{1}{c|}{1}      & 1123.35                       \\ \hline
\multicolumn{1}{|c|}{loop-lit}                     & 10                            & \multicolumn{1}{c|}{10}     & \multicolumn{1}{c|}{0}      & 22.22                           & \multicolumn{1}{c|}{10}     & \multicolumn{1}{c|}{0}      & 280.87   & \multicolumn{1}{c|}{5}      & \multicolumn{1}{c|}{5}      & 5655.33                       \\ \hline
\multicolumn{1}{|c|}{loop-acceleration}            & 10                            & \multicolumn{1}{c|}{9}      & \multicolumn{1}{c|}{1}      & 0.32                            & \multicolumn{1}{c|}{9}      & \multicolumn{1}{c|}{1}      & 493.13   & \multicolumn{1}{c|}{9}      & \multicolumn{1}{c|}{1}      & 1030.78                       \\ \hline
\multicolumn{1}{|c|}{loop-crafted}                 & 2                             & \multicolumn{1}{c|}{2}      & \multicolumn{1}{c|}{0}      & 0.09                            & \multicolumn{1}{c|}{2}      & \multicolumn{1}{c|}{0}      & 49.59    & \multicolumn{1}{c|}{2}      & \multicolumn{1}{c|}{0}      & 27.97                         \\ \hline
\multicolumn{1}{|c|}{\cite{DBLP:conf/vmcai/BoutonnetH19}} & 9                             & \multicolumn{1}{c|}{8}      & \multicolumn{1}{c|}{1}      & 2.12                            & \multicolumn{1}{c|}{9}      & \multicolumn{1}{c|}{0}      & 286.24   & \multicolumn{1}{c|}{4}      & \multicolumn{1}{c|}{5}      & 4576.98                       \\ \hline
\multicolumn{1}{|c|}{\textbf{Total}}                        & \textbf{114} & \multicolumn{1}{c|}{101}    & \multicolumn{1}{c|}{13}     & \textbf{34.65} & \multicolumn{1}{c|}{108}    & \multicolumn{1}{c|}{6}      & 8939.56  & \multicolumn{1}{c|}{68}     & \multicolumn{1}{c|}{46}     & 43573.42                      \\ \hline \hline
\multicolumn{2}{|c|}{Benchmark}                                                    & \multicolumn{3}{c|}{OOPSLA23}                                                               & \multicolumn{3}{c|}{DIG}                                             & \multicolumn{3}{c|}{IKOS + PPLite}                                                        \\ \hline
\multicolumn{1}{|c|}{Source}                       & \#Num                         & \multicolumn{1}{c|}{\#Ver.} & \multicolumn{1}{c|}{\#Unk.} & Time (s)                        & \multicolumn{1}{c|}{\#Ver.} & \multicolumn{1}{c|}{\#Unk.} & Time (s) & \multicolumn{1}{c|}{\#Ver.} & \multicolumn{1}{c|}{\#Unk.} & \multicolumn{1}{c|}{Time (s)} \\ \hline
\multicolumn{1}{|c|}{loop-invariants}              & 5                             & \multicolumn{1}{c|}{1}      & \multicolumn{1}{c|}{4}      & 14.09                           & \multicolumn{1}{c|}{0}       & \multicolumn{1}{c|}{5}       & 2344.12          & \multicolumn{1}{c|}{3}       & \multicolumn{1}{c|}{2}       & 0.88       \\ \hline
\multicolumn{1}{|c|}{loop-new}                     & 2                             & \multicolumn{1}{c|}{0}      & \multicolumn{1}{c|}{2}      & 5.83                            & \multicolumn{1}{c|}{0}       & \multicolumn{1}{c|}{2}       & 241.68         & \multicolumn{1}{c|}{1}       & \multicolumn{1}{c|}{1}       & 988.06        \\ \hline
\multicolumn{1}{|c|}{loop-invgen}                  & 5                             & \multicolumn{1}{c|}{4}      & \multicolumn{1}{c|}{1}      & 14.78                           & \multicolumn{1}{c|}{1}       & \multicolumn{1}{c|}{4}       & 264.34         & \multicolumn{1}{c|}{5}       & \multicolumn{1}{c|}{0}       & 1.03        \\ \hline
\multicolumn{1}{|c|}{loops-crafted-1}              & 25                            & \multicolumn{1}{c|}{22}     & \multicolumn{1}{c|}{3}      & 82.03                           & \multicolumn{1}{c|}{10}       & \multicolumn{1}{c|}{15}       & 6030.28         & \multicolumn{1}{c|}{0}       & \multicolumn{1}{c|}{25}       & 8270.92        \\ \hline
\multicolumn{1}{|c|}{loop-simple}                  & 2                             & \multicolumn{1}{c|}{0}      & \multicolumn{1}{c|}{2}      & 5.82                            & \multicolumn{1}{c|}{0}       & \multicolumn{1}{c|}{2}       &  493.24        & \multicolumn{1}{c|}{2}       & \multicolumn{1}{c|}{0}       & 10.01       \\ \hline
\multicolumn{1}{|c|}{loop-zilu}                    & 26                            & \multicolumn{1}{c|}{0}      & \multicolumn{1}{c|}{26}     & 68.24                           & \multicolumn{1}{c|}{19}       & \multicolumn{1}{c|}{7}       & 6878.38         & \multicolumn{1}{c|}{17}       & \multicolumn{1}{c|}{9}       & 1827.31       \\ \hline
\multicolumn{1}{|c|}{loops}                        & 18                            & \multicolumn{1}{c|}{4}      & \multicolumn{1}{c|}{14}     & 48.36                           & \multicolumn{1}{c|}{3}       & \multicolumn{1}{c|}{15}       & 5241.99         & \multicolumn{1}{c|}{7}       & \multicolumn{1}{c|}{11}       & 909.04      \\ \hline
\multicolumn{1}{|c|}{loop-lit}                     & 10                            & \multicolumn{1}{c|}{7}      & \multicolumn{1}{c|}{3}      & 29.59                           & \multicolumn{1}{c|}{3}       & \multicolumn{1}{c|}{7}       & 2024.29         & \multicolumn{1}{c|}{5}       & \multicolumn{1}{c|}{5}       & 3993.86     \\ \hline
\multicolumn{1}{|c|}{loop-acceleration}            & 10                            & \multicolumn{1}{c|}{8}      & \multicolumn{1}{c|}{1}      & 27.48                           & \multicolumn{1}{c|}{3}       & \multicolumn{1}{c|}{7}       & 2263.40         & \multicolumn{1}{c|}{6}       & \multicolumn{1}{c|}{4}       & 1.66       \\ \hline
\multicolumn{1}{|c|}{loop-crafted}                 & 2                             & \multicolumn{1}{c|}{2}      & \multicolumn{1}{c|}{0}      & 5.63                            & \multicolumn{1}{c|}{0}       & \multicolumn{1}{c|}{2}       &  503.76        & \multicolumn{1}{c|}{2}       & \multicolumn{1}{c|}{0}       & 0.33      \\ \hline
\multicolumn{1}{|c|}{\cite{DBLP:conf/vmcai/BoutonnetH19}} & 9                             & \multicolumn{1}{c|}{6}      & \multicolumn{1}{c|}{3}      & 28.98                           & \multicolumn{1}{c|}{1}       & \multicolumn{1}{c|}{8}       & 497.92         & \multicolumn{1}{c|}{8}       & \multicolumn{1}{c|}{1}       & 2.04        \\ \hline
\multicolumn{1}{|c|}{\textbf{Total}}                        & \textbf{114} & \multicolumn{1}{c|}{55}     & \multicolumn{1}{c|}{59}     & 330.83                          & \multicolumn{1}{c|}{40}       & \multicolumn{1}{c|}{74}       & 26783.40         & \multicolumn{1}{c|}{56}       & \multicolumn{1}{c|}{58}       & 16005.15        \\ \hline
\end{tabular}
}
\vspace{0.8em}
\caption{Comparisons Over 114 Benchmarks}

\label{exp:ComparisonOverTools}
\end{table}

The complete comparison results of \ToolName\ with other tools are presented in Table~\ref{exp:ComparisonOverTools}. In the table, \textit{Source} indicates the source category of the benchmark. The term \textit{\#Ver.} represents the number of examples correctly verified by the verifier, and \textit{\#Unk.} (unknown) mainly arises from the following situations: a) The front-end fails to parse correctly, resulting in program crashes. b) Returns \textbf{Unknown}. c) Timeouts. For the benchmarks from~\cite{DBLP:conf/vmcai/BoutonnetH19}, which do not contain assertions to be verified, we modify the invariants generated by our \ToolName\ as assertions and test them over the other tools to obtain results.

From the table, it is evident that \ToolName\ typically requires less than 0.3 seconds on average for verification, and its overall verification accuracy is very close to that of the SV-COMP 2023 Reachability track winner Veriabs, while significantly outperforming Veriabs in terms of time efficiency by 10X to 1000X. This is mainly because Veriabs employs a rich strategy to assist verification, granting it a stronger verification capability but also requiring more time for most examples. CPAChecker experienced a broad range of timeouts in examples with complex loops that could not be verified within a finite unfolding of loops. This is due to the intrinsic limitations of its bounded model checking approach, and its loop unwinding strategy also results in verification times on the dataset that significantly exceed those of other tools. 

Despite the fact that the tool from~\cite{oopsla23} has the second fewest number of verified benchmarks, it outperforms other tools in examples suitable for recurrence analysis. For DIG, we employ it to generate loop invariants and post conditions, and use Z3~\cite{z3} prover to verify the assertion. Nevertheless, the frontend of DIG necessitates CIVL's reliance on extracting symbolic execution paths from the program. When processing loops, it similarly depends on loop unrolling, and if it cannot fully unroll loops within a small bound, it determines that locations after the loop are unreachable. Consequently, it exhibits issues analogous to those of CPAChecker. Additionally, for some randomly assigned variables in SV-COMP, DIG lacks a suitable modeling. We have already reported several bugs via issues on GitHub. As a classical framework for abstract interpretation, IKOS with PPLite did not deliver optimal verification outcomes on the dataset. In some straightforward nested loops and more extensive loop iterations, it either failed to converge to a fixed point, or the precision of the invariants obtained upon convergence was insufficient to verify assertions, thereby causing timeouts or unknown in certain instances. 

In summary, we conclude that \ToolName\ significantly outperforms other tools such as Veriabs in time efficiency for affine numerical programs, while its verification capability is not inferior to the SV-COMP winner Veriabs. We also conducted an in-depth analysis of the cases where our \ToolName\ returns \textbf{Unknown}. The primary reasons for the issues include: a) the absence of type range constraints at the front end, b) reliance on modular arithmetic, c) the need for more complex loop generalizations, d) exceeding the computational precision of the PPL library, and e) exponential arithmetic that surpasses the modeling capabilities of linear templates. 7-8 of these cases could be further solved by optimizing implementations. In the verifiable cases, the preliminary implementation of \ToolName\ has already far surpassed existing methods in efficiency. 


\subsection{Ablation Study in Invariant Propagation (RQ2)}

\begin{table}[t]
    \centering
    \resizebox{0.9\linewidth}{!}{%
        \begin{tabular}{|cc|cccccc|}
        \hline
        \multicolumn{2}{|c|}{\multirow{2}{*}{Benchmark}} & \multicolumn{6}{c|}{\ToolName}                                                                                                                                    \\ \cline{3-8} 
        \multicolumn{2}{|c|}{}                           & \multicolumn{3}{c|}{No PPG}                                                               & \multicolumn{3}{c|}{PPG}                                             \\ \hline
        \multicolumn{1}{|c|}{Source}        & \#Num      & \multicolumn{1}{c|}{\#Ver.} & \multicolumn{1}{c|}{\#Unk.} & \multicolumn{1}{c|}{Time (s)} & \multicolumn{1}{c|}{\#Ver.} & \multicolumn{1}{c|}{\#Unk.} & Time (s) \\ \hline
        \multicolumn{1}{|c|}{SV-COMP}       & 105        & \multicolumn{1}{c|}{91}     & \multicolumn{1}{c|}{14}     & \multicolumn{1}{c|}{1825.53}  & \multicolumn{1}{c|}{93}     & \multicolumn{1}{c|}{12}     & 32.53    \\ \hline
        \multicolumn{1}{|c|}{paper}         & 9          & \multicolumn{1}{c|}{8}      & \multicolumn{1}{c|}{1}      & \multicolumn{1}{c|}{10.76}    & \multicolumn{1}{c|}{8}      & \multicolumn{1}{c|}{1}      & 2.12     \\ \hline
        \end{tabular}
    }
    \vspace{0.8em}
    \caption{Experiment for Invariant Propagation}
    \label{tab:propagation}
\end{table}
% Figure environment removed

In this section, we conduct an ablation study to evaluate the performance of the invariant propagation technique within \ToolName. In Table~\ref{tab:propagation}, we present the overall results, where we can clearly observe that the use of invariant propagation leads to a 5X-50X improvement in time efficiency.

More specifically, through the scatter plot in Figure~\ref{fig:CompareProgation}, we compared the time performance of individual examples before and after the application of invariant propagation techniques. In some cases, invariant propagation led to significant efficiency improvements (10X-1000X). This is due to the fact that for more complex programs, the size of the \LTS{} \(\Gamma\) is larger, and applying invariant propagation techniques on this basis can maximize performance optimization. Since the tool itself performs efficiently in most examples, the optimization brought by this technique is not apparent in those cases in the graph where the time is below 0.1 seconds. As the propagation itself, including the projection of sub-ATS, incurs a certain time cost, which dilutes the time optimization brought about by invariant propagation.

In conclusion, invariant propagation significantly enhances the tool's scalability and yields superior optimization results for complex examples. This also reveals that, within our constraint-solving methodology, the cost of computing invariants at any given location constitutes the principal computational bottleneck. By reducing the number of locations that need to be computed and leveraging prior results to avoid redundant polyhedral operations, we can effectively enhance efficiency.

\subsection{Caveat to Correctness}

This section elucidates configurations that may induce subtle deviations from real-world programs or alternative models during the empirical evaluation of our tool.

\begin{itemize}
    \item In our current experimental setup, we have not accounted for the behavior of machine integers during overflow conditions. Consequently, our verification process is confined to affine programs that do not encounter overflow errors.
    \item Within the context of control flow transformations, we introduce uncertainty into conditional statements by adhering to the SV-COMP guidelines. This is achieved by replacing branch conditions with functions that return random Boolean values, thereby emulating the semantics of non-deterministic branches. Nonetheless, we have yet to effectively model uncertainty in variable coefficients, specifically affine inequalities with coefficients represented as intervals.
\end{itemize}