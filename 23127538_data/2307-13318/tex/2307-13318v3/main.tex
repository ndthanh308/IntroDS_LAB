
%% For submission and review of your manuscript please change the
%% command to \documentclass[manuscript, screen, review]{acmart}.
%%
%% When submitting camera ready or to TAPS, please change the command
%% to \documentclass[sigconf]{acmart} or whichever template is required for your publication.

\documentclass[acmsmall,screen,nonacm]{acmart}
\settopmatter{printccs=false,printacmref=false}

%% Rights management information.  This information is sent to you
%% when you complete the rights form.  These commands have SAMPLE
%% values in them; it is your responsibility as an author to replace
%% the commands and values with those provided to you when you
%% complete the rights form.
% \setcopyright{acmlicensed}
% \copyrightyear{2018}
% \acmYear{2018}
% \acmDOI{XXXXXXX.XXXXXXX}

%% These commands are for a PROCEEDINGS abstract or paper.
% \acmConference[Conference acronym 'XX]{Make sure to enter the correct
%   conference title from your rights confirmation emai}{June 03--05,
%   2018}{Woodstock, NY}

%%
%%  Uncomment \acmBooktitle if the title of the proceedings is different
%%  from ``Proceedings of ...''!
%%
%%\acmBooktitle{Woodstock '18: ACM Symposium on Neural Gaze Detection,
%%  June 03--05, 2018, Woodstock, NY}
%% \acmISBN{978-1-4503-XXXX-X/18/06}


%%
%% Submission ID.
%% Use this when submitting an article to a sponsored event. You'll
%% receive a unique submission ID from the organizers
%% of the event, and this ID should be used as the parameter to this command.
%%\acmSubmissionID{123-A56-BU3}


%%
%% The majority of ACM publications use numbered citations and
%% references.  The command \citestyle{authoryear} switches to the
%% "author year" style.
%%
%% If you are preparing content for an event
%% sponsored by ACM SIGGRAPH, you must use the "author year" style of
%% citations and references.
%% Uncommenting
%% the next command will enable that style.
\citestyle{acmauthoryear}

\usepackage{booktabs}   %% For formal tables:
                        %% http://ctan.org/pkg/booktabs
\usepackage{subcaption} %% For complex figures with subfigures/subcaptions
                        %% http://ctan.org/pkg/subcaption
\renewcommand\footnotetextcopyrightpermission[1]{}
\newcommand\bmmax{0}
\let\Bbbk\relax
\usepackage{amssymb}
\usepackage{array,makecell}
\usepackage[utf8]{inputenc}
\usepackage[T1]{fontenc}
\usepackage{microtype}
\usepackage{balance}
\usepackage{multirow}
\usepackage{diagbox}
\usepackage{amsmath}
\usepackage{graphics}
\usepackage{multicol}
\usepackage{enumitem}
\usepackage{mathtools}
\usepackage{epsfig}
\usepackage{xspace}
\usepackage{extarrows}
\usepackage{bm}
\usepackage{rotating}
%\usepackage{paralist}
\usepackage{eucal}
\usepackage[symbol, flushmargin]{footmisc}
%\usepackage[vlined,boxed,ruled,linesnumbered]{algorithm2e}
\usepackage{multirow}
\usepackage{graphicx}
\usepackage{float}
%\usepackage{subfig}
\usepackage{ragged2e}
\usepackage[normalem]{ulem}
\usepackage{algorithm}
\usepackage{algorithmic}
\usepackage{setspace}
\usepackage{listings}
\usepackage[most]{tcolorbox}
\lstdefinelanguage{program}
{
morekeywords={if, then, else, fi, while, true, false, switch, case, skip, break},
sensitive = false
}

\newtheorem{example}{Example}
\newtheorem{remark}{Remark}
\newtheorem{definition}{Definition}
\newtheorem{lemma}{Lemma}
\newtheorem{proposition}{Proposition}

\newcommand\calF{\mathcal{F}}
\newcommand\calG{\mathcal{G}}
\newcommand\calM{\mathcal{M}}
\newcommand\calV{\mathcal{V}}
\newcommand\calU{\mathcal{U}}
\newcommand\calW{\mathcal{W}}
\newcommand\calP{\mathcal{P}}
\newcommand\calD{\mathbb{D}}
%%%%%%%%%%%%%%%%%
%% macros introduced by Luke 
\newcommand\mydef[1]{{\bf\em #1}}
%%%%%%%%%%%%%%%%%

\newcommand{\numviparams}{{| \lambda |}}
\newcommand{\scoreaccvars}[1]{s_1^{#1}, \ldots, s_{\numviparams}^{#1}}
\newcommand{\scoreaccvar}[2]{s_{#1}^{#2}}
\newcommand{\isdeterm}[1]{\text{Deterministic}({#1})}


\newcommand{\expect}[1]{\mathbb{E}\left[{#1}\right]}
\newcommand{\var}[1]{\mathbb{V}\left[ {#1} \right]}
\newcommand{\expectdist}[2]{\mathbb{E}_{#1}\left[ {#2} \right]}
\newcommand{\vardist}[2]{\mathbb{V}_{#1}\left[ {#2} \right]}
\newcommand{\cov}[2]{\mathbb{C}\text{ov}[{#1}][{#2}]}
\newcommand{\covv}[1]{\mathbb{C}\text{ov}[{#1}]}
\newcommand{\corr}[1]{\mathbb{C}\text{orr}[{#1}]}

\newcommand{\fix}[1]{\mathit{fix}\left({#1}\right)}
\newcommand{\sbr}[1]{\left\llbracket {#1} \right\rrbracket}
\newcommand{\ctxtype}[3]{{#1} \cong_\text{ctx} {#2} : {#3}}
\newcommand{\bigstep}[3]{{#1} \Downarrow_{#2} {#3}}


% PCF types
\newcommand{\bool}{\mathit{bool}}
\newcommand{\nat}{\mathit{nat}}

\newcommand{\ctx}[1]{\mathcal{C}\left[ {#1}\right] }
\newcommand{\pcft}[1]{\text{PCF}_{#1}}

\newcommand{\nfl}{\mathbb{N}_\bot}
\newcommand{\bfl}{\mathbb{B}_\bot}

% PCF constructs
\newcommand{\succc}[1]{\mathbf{succ}({#1})}
\newcommand{\succcn}[2]{\mathbf{succ}^{#1}({#2})}
\newcommand{\zero}{\mathbf{0}}
\newcommand{\zerotest}[1]{\mathbf{zero}\left({#1}\right)}
\newcommand{\pred}[1]{\mathbf{pred}\left( {#1} \right)}
\newcommand{\predn}[2]{\mathbf{pred}^{#1}\left( {#2} \right)}
\def\solvable{\#}

\newcommand{\true}{\mathbf{true}}
\newcommand{\false}{\mathbf{false}}
\newcommand{\pcffix}[1]{\mathbf{fix}\left({#1}\right)}
\newcommand{\pcffn}[3]{\mathbf{fn}~{#1}:{#2}\mathpunct{.}{#3}}
\newcommand{\pairtype}[2]{{#1} * {#2}}
\newcommand{\pairexp}[2]{\mathbf{pair}({#1}, {#2})}
\newcommand{\leftexp}[1]{\mathbf{left}({#1})}
\newcommand{\rightexp}[1]{\mathbf{right}({#1})}

\newcommand{\RationalPos}{\mathbb{Q}^{+}}

\newcommand{\meas}[1]{\mathbb{M}\left( {#1} \right) }
\newcommand{\integ}[1]{\sbr{#1}_I}

\newcommand{\notbigstep}[2]{{#1}~\cancel{\Downarrow}_{#2}}
\newcommand{\subtrace}[3]{{#1}^{{#2} \ldots {#3}}}
\newcommand{\supp}[1]{\textsf{supp}\left({#1}\right)}
\newcommand{\dom}[1]{\textsf{Dom}\left({#1}\right)}
\newcommand{\suppk}[2]{\textsf{Supp}^{#1}\left({#2}\right)}
\newcommand{\tracespace}{\bigcup_{n \in \mathbb{N}}[0, 1]^n}
\newcommand{\generictracespace}{\mathbb{T}}
\newcommand{\nnreals}{\mathbb{R}_{\geq 0}}
\newcommand{\posreals}{\mathbb{R}_{> 0}}
\newcommand{\reals}{\mathbb{R}}

\newcommand{\unrollkM}[2]{\textsf{unroll}_{#1}\left({#2}\right)}
\newcommand{\nphmcint}[5]{\Psi_\textsf{NP}\left({#1}, {#2}, {#3}, {#4}, {#5}\right)}

%SPCF constructs
\newcommand{\spcfvalues}{\Lambda^0_v}

\newcommand{\prevalueM}[1]{\textsf{value}^{-1}_{#1}(\spcfvalues{})}
\newcommand{\num}[1]{\underline{#1}}

% \theoremstyle{definition}
% \newtheorem{thm}{Theorem}
% \newtheorem{lem}{Lemma}
% \newtheorem{defn}{Definition}
% \newtheorem{conj}{Conjecture}
% \newtheorem{prop}{Proposition}

%\theoremstyle{definition}
%\newtheorem{defn}{Definition}[section]
%\newtheorem{example}[defn]{Example}
%
%
%\theoremstyle{plain}
%\newtheorem{thm}{Theorem}[section]
%\newtheorem{lem}[thm]{Lemma}
%\newtheorem{cor}[thm]{Corollary}
%\newtheorem{conj}[thm]{Conjecture}
%\newtheorem{prop}[thm]{Proposition}
%\newtheorem{remark}[thm]{Remark}

%% Proofs
%\let\oldproof\proof
%\renewcommand{\proof}{\color{blue}\oldproof}


\definecolor{codegreen}{rgb}{0,0.6,0}
\definecolor{codegray}{rgb}{0.5,0.5,0.5}
\definecolor{codepurple}{rgb}{0.58,0,0.82}
\definecolor{backcolour}{rgb}{0.95,0.95,0.92}

\lstdefinestyle{myStyle}{
    belowcaptionskip=1\baselineskip,
    breaklines=true,
    frame=none,
    basicstyle=\footnotesize\ttfamily,
    keywordstyle=\bfseries\color{green!40!black},
    commentstyle=\itshape\color{purple!40!black},
    identifierstyle=\color{blue},
    backgroundcolor=\color{gray!10!white},
    %backgroundcolor=\color{backcolour}, 
    numberstyle=\tiny\color{codegray},
    stringstyle=\color{codepurple},
    breakatwhitespace=false,                          
    keepspaces=true,                 
    numbers=left,       
    numbersep=5pt,                  
    showspaces=false,                
    showstringspaces=false,
    showtabs=false,                  
    tabsize=2,
}

% argmin/argmax
\DeclareMathOperator*{\argmax}{arg\,max}
\DeclareMathOperator*{\argmin}{arg\,min}

% Concatenation of lists
\newcommand\doubleplus{+\kern-1.3ex+\kern0.8ex}

% Program configurations
\newcommand{\tuple}[1]{\ensuremath{\langle #1 \rangle}}
% Rule based definitions
\newcommand{\Rule}[4][]{\ensuremath{\inferrule*[lab={\hypertarget{#2}{(\TirName{#2})}},#1]{#3}{#4}}}

% Calligraphic symbols
\newcommand{\calI}{{\mathcal I}} 
\newcommand{\calT}{{\mathcal T}}

%  Macro for new Y operator.
\newcommand{\yBounded}[3]{\mu^{#1}_{#2}\rvert_{#3}}

%%%%%%%%%%%%%%%%%
 
%%%%%%%%%%%%%%%%%

\newcommand{\expv}{\mathbb{E}}

\newcommand{\combTr}[2]{\left[\begin{matrix}
		#1\\
		#2
	\end{matrix} \right]}

\newcommand{\exType}[2]{\left\{\begin{matrix}
		#1\\
		#2
	\end{matrix} \right\}}
\newcommand{\myint}[1]{ [#1]}
\newcommand{\Uniform}{\ensuremath{\mathrm{Uniform}}}
\newcommand{\Normal}{\ensuremath{\mathrm{normal}}}
\DeclareMathOperator{\abs}{abs}
\DeclareMathOperator{\pdf}{pdf}

\newcommand{\intConf}[1]{\lceil#1\rceil}
\newcommand{\tr}{\boldsymbol{t}}

\newcommand{\sample}{\tt{sample}}
%\newcommand{\fix}{\texttt{fix}}
%\newcommand{\num}[1]{\underline{#1}}
\newcommand{\myif}{\texttt{if}}
\newcommand{\mylet}{\texttt{let} \, }
\newcommand{\myin}{\, \texttt{in} \,}
\newcommand{\mythen}{\, \texttt{then} \,}
\newcommand{\myelse}{\, \texttt{else} \,}
\newcommand{\score}{\tt{score}}
\newcommand{\tick}{\tt{tick}}

\newcommand{\term}{\tt{term}}
\newcommand{\pv}{\mathbf{v}}
\newcommand{\rv}{\mathbf{r}}

\newcommand{\interval}{\mathfrak{I}}

\newcommand{\typeReal}{\textbf{\textsf{R}}}

\newcommand{\symbolInt}{\myint{\cdot}}

\newcommand{\LambdaInterval}{\Lambda_{\interval}}
\newcommand{\LambdaSymbolic}{\Lambda_{\text{sym}}}

\newcommand{\toIntervalTerm}[1]{#1^{2\interval}}

%Others
\newcommand{\Sset}{\mathbb{S}}
\newcommand{\Iset}{\mathbb{I}}
\newcommand{\Rset}{\mathbb{R}}
\newcommand{\Nset}{\mathbb{N}}
\newcommand{\Zset}{\mathbb{Z}}

\newcommand{\Term}{\mathbb{T}}
\newcommand{\prob}{\mathbb{P}}
\newcommand{\expt}{\mathbb{E}}


\newcommand{\Leb}{\tt{Leb}}
\newcommand{\Red}{\tt{Red}}
\newcommand{\cost}{\text{cost}}

%\newcommand{\intervalab}[2]{\underline{[#1,#2]}}
\newcommand{\intervalab}{\underline{[a,b]}}
\newcommand{\interI}{\mathcal{I}}
\newcommand{\trans}{\mathcal{T}}

\newcommand{\iv}{\mathbb{I}}

% Programming language constructs
\newcommand{\lit}[1]{\underline{#1}}
\newcommand{\letIn}[1]{\mathsf{let}\,{#1}\,\mathsf{in}\,}
\newcommand{\fixLam}[2]{\mu {#1} {#2}.}
\newcommand{\ifElse}[3]{\mathsf{if} (#1 \le \num{0}) \, {#2} \,\mathsf{else}\, {#3}}

%%Basic notions
\newcommand{\pspace}{(\Omega,\mathcal{F},\probm)}
\newcommand{\probm}{\mathbb{P}}
\newcommand{\condexpv}[2]{{\expt}{\left[{#1} \mid {#2}\right]}}

\newcommand{\stdConf}[1]{(#1)}
%\newcommand{\intConf}[1]{\lceil#1\rceil}
%\newcommand{\intConf}[1]{(#1)}
%\newcommand{\symConf}[1]{\langle\!\langle  #1 \rangle\!\rangle}
%\newcommand\symPath[1]{(#1)}
\newcommand{\symPath}[1]{\langle\!\langle  #1 \rangle\!\rangle}
\newcommand\symConf[1]{(#1)}

\newcommand{\ifSimple}[3]{\mathsf{if}(#1, #2, #3)}
%\newcommand{\ifElse}[3]{\mathsf{if} (#1 \le 0) \, \allowbreak {#2} \, \allowbreak \mathsf{else}\, {#3}}
%\newcommand{\ifElse}[3]{\ifSimple{#1}{#2}{#3}}

%\newcommand{\trace}{\mathsf{s}}
%
%\newcommand\defn[1]{{\bf \em #1}}
\newcommand{\traces}{\mathbb{T}}
%
%\newcommand{\stdConf}[1]{(#1)}
%%\newcommand{\intConf}[1]{\lceil#1\rceil}
%\newcommand{\intConf}[1]{(#1)}
%%\newcommand{\symConf}[1]{\langle\!\langle  #1 \rangle\!\rangle}
%%\newcommand\symPath[1]{(#1)}
%\newcommand{\symPath}[1]{\langle\!\langle  #1 \rangle\!\rangle}
%\newcommand\symConf[1]{(#1)}

\newcommand{\valueSem}[1]{\mathsf{val}_{#1}} % value (semantics)
\newcommand{\weightSem}[1]{\mathsf{wt}_{#1}} % weight (semantics)
\newcommand{\measureSem}[1]{\llbracket #1 \rrbracket}
\newcommand{\posterior}{\mathsf{posterior}}


%%%%%%%%%
% 
%%%%%%%%
\newcommand{\loc}{\ell}
\newcommand{\locs}{\mathit{L}}
\newcommand{\blocs}{\mathit{L}_{\mathrm{b}}}

\newcommand{\iflocs}{\mathit{L}_{\mathrm{if}}}
\newcommand{\looplocs}{\mathit{L}_{\mathrm{while}}}

\newcommand{\alocs}{\mathit{L}_{\mathrm{a}}}
\newcommand{\wlocs}{\mathit{L}_{\mathrm{w}}}
\newcommand{\rlocs}{\mathit{L}_{\mathrm{r}}}
\newcommand{\Alocs}[1]{\mathit{L}_{\mathrm{A}}^{\mathsf{#1}}}
\newcommand{\Dlocs}{\mathit{L}_{\mathrm{nd}}}
\newcommand{\transitions}{{\rightarrow}}

%%% 
\newcommand{\plocs}{\mathit{L}_{\mathrm{p}}}
\newcommand{\tlocs}{\mathit{L}_{\mathrm{t}}}

\newcommand{\lin}{\loc_\mathrm{init}}
\newcommand{\lout}{\loc_\mathrm{out}}
\newcommand{\val}[1]{\mbox{\sl Val}_{#1}}

\newcommand{\pvars}{V_\mathrm{p}}
\newcommand{\rvars}{V_{\mathrm{r}}}
\newcommand{\pre}{\mathrm{pre}}

\newcommand{\sle}{\sqsubseteq}
\newcommand{\sge}{\sqsupseteq}

\newcommand{\lfp}{\mathrm{lfp}}
\newcommand{\gfp}{\mathrm{gfp}}

\newcommand{\rdvarjdis}{\mathcal D}
\newcommand{\sampset}{\textit{supp}}

\newcommand{\upd}{\mbox{\sl upd}}
\newcommand{\wet}{\mbox{\sl wt}}
\newcommand{\transset}{\mathfrak T}
\newcommand{\valin}{\pv_{\mathrm{init}}}
\newcommand{\ret}{\mbox{\sl ret}}

\newcommand{\win}{w_{\mathrm{init}}}

\newcommand{\sampdpd}{\overline{\Upsilon}}

\newcommand{\outmap}{\text{O}}
\newcommand{\sat}[1]{\langle #1 \rangle}
\newcommand{\monoid}{\mbox{\sl Monoid}}
\newcommand{\handelmanformat}{(\dagger)}

\newcommand{\trunc}{\mathcal{B}}

\newcommand{\ewt}{\mbox{\sl ewt}}
\newcommand{\statemap}{\text{St}}

\newcommand{\valrd}{{\mathbf{r}}}
\newcommand{\frmloc}{\ell^{\mathrm{src}}}
\newcommand{\toloc}{\ell^{\mathrm{dst}}}

\newcommand{\monomials}{\mathbf{M}}

\begin{document}

\title{Affine Disjunctive Invariant Generation with Farkas' Lemma}

%% Of note is the shared affiliation of the first two authors, and the
%% "authornote" and "authornotemark" commands
%% used to denote shared contribution to the research.

\author{Jingyu Ke}
\affiliation{
  \institution{Shanghai Jiao Tong University}
  \city{Shanghai}
  \country{China}
}
\email{windocotber@sjtu.edu.cn}

\author{Hongfei Fu}
\authornote{Corresponding Author}
\affiliation{
  \institution{Shanghai Jiao Tong University}
  \city{Shanghai}
  \country{China}
}
\email{jt002845@sjtu.edu.cn}

\author{Hongming Liu}
\affiliation{
  \institution{Shanghai Jiao Tong University}
  \city{Shanghai}
  \country{China}
}
\email{hm-liu@sjtu.edu.cn}

\author{Liqian Chen}
\affiliation{
  \institution{National University of Defense Technology}
  \city{Changsha}
  \country{China}
}
\email{lqchen@nudt.edu.cn}

\author{Guoqiang Li}
\affiliation{
  \institution{Shanghai Jiao Tong University}
  \city{Shanghai}
  \country{China}
}
\email{li.g@sjtu.edu.cn}


\begin{abstract}
Invariant generation is the classical problem that aims at automated generation of assertions that over-approximates the set of reachable program states in a program. We consider affine invariants over affine while loops (i.e., loops with affine loop guards, conditional branches and assignment statements), and explore the automated generation of disjunctive affine invariants. Disjunctive invariants are important to 
capture disjunctive features in programs such as multiple phases, transitions between different modes, etc., and are typically more precise than conjunctive invariants over programs with these features. 
To generate tight affine invariants, existing approaches have investigated the application of Farkas' Lemma to conjunctive affine invariant generation, but none of them considers disjunctive affine invariants. 

In this work, we introduce a novel approach to generate affine disjunctive invariants using Farkas' Lemma. Our approach employs a carefully designed control flow transformation to create an affine transition system from the original loop to which previous approaches in Farkas' Lemma apply. 
The affine transition system distinguishes paths within a loop body with their corresponding conjunctive invariants, and takes the disjunction of these conjunctive invariants. 
Furthermore, we propose optimizations to improve the scalability, accuracy and  applicability of our approach.
These optimizations include: a) an invariant propagation technique that
enables the spread of invariants within the strongly connected components of the transition system to improve scalability; b) the tackling of infeasible implication in the application of Farkas' Lemma to improve accuracy; 
c) the standard loop summarization that extends our approach to nested loops to improve applicability.  
The experimental results over a benchmark set of more than 100 affine while loops (mostly from SVCOMP2023) shows that our approach outperforms other approaches in both the number of solved instances and the time efficiency. In particular, our approach handles most instances in 100 milliseconds, achieving a speedup of 10X to 1000X while maintaining a comparable success rate when compared with the state-of-the-art tool Veriabs. 
\end{abstract}


%%
%% The code below is generated by the tool at http://dl.acm.org/ccs.cfm.
%% Please copy and paste the code instead of the example below.
%%
\begin{CCSXML}
<ccs2012>
<concept>
<concept_id>10011007.10011006.10011008</concept_id>
<concept_desc>Software and its engineering~General programming languages</concept_desc>
<concept_significance>500</concept_significance>
</concept>
<concept>
<concept_id>10003456.10003457.10003521.10003525</concept_id>
<concept_desc>Social and professional topics~History of programming languages</concept_desc>
<concept_significance>300</concept_significance>
</concept>
</ccs2012>
\end{CCSXML}
\ccsdesc[500]{Software and its engineering~General programming languages}
\ccsdesc[300]{Social and professional topics~History of programming languages}

\keywords{}

\maketitle

\section{Introduction}
Current quantum hardware is unable to carry out universal quantum computations due to the buildup of errors that occur during the computation. 
The magnitude of the individual error is currently above the value that the Threshold Theorem requires in order to kick-start quantum error correction and fault-tolerant quantum computation~\cite[Section 10.6]{nielsen_chuang_2010}. 
Although the experimentally achieved fidelity rates are promising and the error bounds are inching closer to the required threshold, we will have to work for the foreseeable future with quantum hardware with errors that build-up during the computation.  This implies that we can only do a limited number of steps before the output of the computation has become completely uncorrelated with the intended one.

For fault-tolerant quantum computing, we repeat four steps: 
1) We apply a number of single and two-qubit quantum gates, in parallel whenever possible; 
2) We perform a syndrome measurement on a subset of the qubits; 
3) We perform fast classical computations to determine which errors have occurred and how to correct them; 
and, 4) We apply correction terms based on the classical computations.
We then repeat these four steps with a next sequence of gates. 
These four steps are essential to fault-tolerant quantum computing. 


The starting point of this work is to use the four steps outlined above, not to carry out error correction and fault-tolerant computation, but to enhance short, constant-depth, {\em uncorrected} quantum circuits that perform single qubit gates and {\em nearest-neighbor} two qubit gates. 
Since in the long run we will have to implement error-correction and fault-tolerant computation anyhow, and this is done by such a four-step process, why not make other use of this architecture? Moreover, on some of the quantum hardware platforms, these operations are already in place.
Embracing this idea we naturally arrive at the question: what is the computational power of \textit{low-depth} quantum-classical circuits organized as in the four steps outlined above? 
We thus investigate circuits that execute a small, ideally constant, number of stages, where at each stage we may apply, in parallel, single qubit gates and {\em nearest-neighbor} two qubit gates, followed by measurements, followed by low-depth classical computations of which the outcome can control quantum gates in later stages. 
It is not clear, at first, whether such circuits, especially with constant depth, can do anything remotely useful. 
But we will see that this is indeed the case: many quantum computations can be done by such circuits in constant depth. 
By parallelizing quantum computations in this way, we improve the overall computational capabilities of these circuits, as we do not incur errors on qubits that are idle, simply because qubits are not idle for a very long time. 
Furthermore, reducing the depth of quantum circuits, at the cost of increasing width, allows the circuit to be run faster even if errors occur.

The first usage of such a four-step layout, not to do error correction, but to perform computations, can be found in the paradigm of measurement-based quantum computing~\cite{gottesman1999demonstrating,raussendorf2001one,jozsa2006introduction,clark2007generalised}: 
A universal form of quantum computing where a quantum state is prepared and operations are performed by measuring qubits in different bases, depending on previous measurements and intermediate measurements.

\citeauthor{PhamSvore2013} were the first to formalize the four-step protocol for performing computations~\cite{PhamSvore2013}. They included specific hardware topologies by considering two-dimensional graphs for imposing constraints on qubit interactions. In their model, they develop circuits for particularly useful multi-qubit gates, including specifying costs in the width, number of qubits, depth, number of concurrent time steps, size, and total number of non-Identity operations.
As a result, they find an algorithm that factors integers in polylogarithmic depth.
\citeauthor{Browne:2011} showed that the main tool in the work by \citeauthor{PhamSvore2013}, the fan-out gate, can also be replaced by additional log-depth classical computations in the measurement-based quantum computing setting~\cite{Browne:2011}.

More recently, \citeauthor{Cirac:2021} introduced a scheme to implement unitary operations involving quantum circuits combined with Local Operations and Classical Communication ($\mathsf{LOCC}$) channels: $\mathsf{LOCC}$-assisted quantum circuits~\cite{Cirac:2021}. Similarly to the four-step scheme we just described, they allow for a short depth circuit to be run on the qubits, followed by one round of $\mathsf{LOCC}$, in which ancilla qubits are measured and local unitaries are applied based on the measurement outcomes. They show that in this model any 1D transitionally invariant matrix-product state (MPS) with fixed bond dimension is in the same phase of matter as the trivial state. Similar ideas can be found in~\cite{TVV_NonAbelianTopologicalOrder_2022, tantivasadakarn2021long}.

In this work, we introduce a new model, called \textit{Local Alternating Quantum-Classical Computations} ($\LAQCC$). In this model we alternate between running quantum circuits (constrained by locality), ending in the measurement of a subset of qubits, and fast classical computations based on the measurement results. The outcome of the classical computations are then used to control future quantum circuits. We allow for flexibility in this model, by giving different constraints to the power of both the quantum circuits and the classical circuits as well as the number of alternations between them. 
Most attention will be given to $\LAQCC$ containing quantum circuits of constant depth, classical circuits of logarithmic depth and at most a constant number of alternations between them. 
Any circuit constructed in this model is considered to be of constant depth. 
We restrict ourselves to logarithmic depth classical computations, as this is the first natural and non-trivial extension beyond constant-depth classical computations. 
Constant-depth classical computations do however also have an equivalent constant-depth quantum implementation.

The definition of $\LAQCC$ sharpens the original definition of \citeauthor{PhamSvore2013} by adding constraints to the intermediate classical computations. This allows us to bound the power of $\LAQCC$ from above. 

The main result of \citeauthor{Cirac:2021}, that 1D translational invariant MPS with fixed bond dimension can be prepared by $\mathsf{LOCC}$-assisted circuits, relies on local symmetries of the MPS. These symmetries allow them to prepare local states (on a constant number of qubits) and glue them together by doing one round of the appropriate entangling measurement and corrections, after which they run a round of local unitaries to get the desired result. This general scheme for preparing states that exhibit an MPS description with the appropriate local symmetries requires only geometrically local unitaries and one round of measurement and corrections an therefore is accessible in $\LAQCC$. Studying different local symmetries, known as Symmetry Protected Topological (SPT) phases of matter, to find measurement-based constant depth circuits for states is a broad ongoing field of research~\cite{TVV_NonAbelianTopologicalOrder_2022, tantivasadakarn2021long, smith2023deterministic}. 
All these schemes have a $\LAQCC$ implementation.

%$\LAQCC$-circuits also exist for general schemes of preparing local states, based on the local tensors, and gluing them together using one round of entangled measurement and corrections, based on the local symmetry. 
%The main result of \citeauthor{Cirac:2021}, that 1D translational invariant MPS with fixed bond dimension can be prepared by $\mathsf{LOCC}$-assisted circuits, relies heavily on local symmetries of the MPS and as a result also has an equivalent $\LAQCC$ implementation. 
%The corrections applied after the measurement round are local unitaries depending on the local symmetries of the MPS. 

 

%This general scheme of preparing local states, based on the local tensors, and gluing it together by doing one round of entangled measurement and corrections, based on the local symmetry, is accessible in $\LAQCC$.
Note however that \citeauthor{Cirac:2021} also suggest a circuit for the $W$-state.
This circuit uses sequentially and dependent measurement-based corrections of the ancilla qubits. 
These dependent measurements translate to sequential alternations between the quantum and classical circuits and therefore increase the total depth to linear depth, exceeding the constant-depth constraints imposed by $\LAQCC$-circuits. 

We study the power of the $\LAQCC$ model with respect to state preparation, showing that even with only constant quantum-depth and logarithmic classical depth it remains possible to prepare states with long-range entanglement.
Another surprising result is that it is unlikely that $\LAQCC$ circuits are classically simulatable. We show that any instantaneous quantum polynomial-time (IQP) circuit~\cite{Bremner2010,Shepherd2009} has an $\LAQCC$ implementation.
Classical simulation of IQP circuits implies the collapse of the polynomial hierarchy to the third level, which is not believed to be true~\cite{Bremner2017}. Therefore, we expect that $\LAQCC$ circuits are unlikely to be classically simulatable. We bound the power of $\LAQCC$ by showing that it is contained in $\QNC^1$, the class of polynomial-size, log-depth circuits.

Next, we also study the power that intermediate classical calculations can add to quantum computations, by considering a new model that alternates between polynomially many polynomial-depth quantum circuits and unbounded classical computations
We study this model by doing a complexity theoretical analysis, where we draw inspiration from the notions of complexity given by \citeauthor{RosenthalYuen:2022}, \citeauthor{MetgerYuen:2023}, and \citeauthor{Aaronson:2004}.
All three complexity notions are based on the notion of state preparation, instead of more traditional definition of complexity such as the decidability of a computational problem. 
The first two consider classes based on sequences of quantum states preparable by a polynomial-sized quantum circuit, where the circuits are uniformly generated by a computational class, for instance, the class $\mathsf{PSPACE}$, which results in the complexity class $\mathsf{StatePSPACE}$~\cite{RosenthalYuen:2022,MetgerYuen:2023}.
The third notion considers a relative complexity, where the complexity is measured between two given states, and is measured by the number of gates, from a given gate-set, required to transform one state in another state~\cite{Aaronson:2004}. 
For our definition of state preparation complexity, we drop the uniformity constraint from~\cite{RosenthalYuen:2022,MetgerYuen:2023} and define a class as $\mathsf{StateX}$, which refers to states preparable by circuits of type $\mathsf{X}$. 
As an example, if $\mathsf{X} = \QNC^0$, this results in the class $\mathsf{StateQNC^0}$, which is the set of states preparable from the $\ket{0}^n$ state by poly-size constant-depth circuits. 
This notion is similar to the relative complexity from~\cite{Aaronson:2004}, where one state is the  $\ket{0}^n$ state and instead of counting the number of gates we consider the set of states preparable by a fixed number of gates. Using this notion of complexity we show that any state preparable by an $\LAQCC^*$ circuit is also preparable by a $\mathsf{PostQPoly}$ circuit, the class of circuits of polynomial depth with an additional post-selection gate. 

All Clifford circuits have a constant-depth $\LAQCC$ implementation, implying that any stabilizer state can be implemented by a constant-depth $\LAQCC$ circuit, see Section~\ref{sec:clifford_circuits} for a proof of this statement. 
Efficient circuits for stabilizer states have been known already through measurement-based quantum computing. Therefore this paper focuses on the preparation of non-stabilizer states, and as a surprising result we find novel constant-depth protocols for four very natural classes of non-stabilizer states.
Despite the extensive research into these four classes of non-stabilizer states and the many applications of them, no efficient constant- or low-depth state preparation protocols are known yet. We specifically consider these four classes as they are all often used as initial states in other algorithms.

The first state is a uniform superposition over an arbitrary number of states. 
This state finds applications in many quantum algorithms, as they often start with a uniform superposition over multiple states. 
This superposition is often achieved by applying Hadamard gates to every qubit due to its simplicity to prepare. 
Yet, the analysis of many algorithms, such as Shor's algorithm~\cite{Shor:1997}, would benefit from a different initial superposition. 
The circuit to prepare the uniform superposition over an arbitrary number of states uses an exact version of Grover search as a subroutine, that turns a probabilistic circuit, with a known constant probability of success, into a deterministic circuit. 
We use the circuit for preparing a uniform superposition over an arbitrary number of states as a subroutine in the next two quantum state preparation protocols. 

The second state is the $W$-state, the uniform superposition over all computational basis states of Hamming-weight~$1$, a natural long-ranged entangled state that displays a fundamentally nonequivalent type of entanglement from the Greenberger–Horne–Zeilinger state~\cite{WState:2000}, for which $\LAQCC$-type constant-depth circuits were previously known~\cite{PhamSvore2013, Cirac:2021}. 
The $W$-state is often used as benchmark for new quantum hardware~\cite{Haffner2005,Neeley2010,GarciaPerez:2021}. 
A novel way to prepare the $W$-state therefore gives a new way to benchmark different quantum devices with each other. 
A circuit for preparing the $W$-state was given in~\cite{Cirac:2021}, but this implementation requires sequentially alternating measurements followed by local unitaries, which in the $\LAQCC$ model is not considered to be of constant depth. 
We improve this protocol by giving an $\LAQCC$ implementation of the $W$-state, based on a compress-uncompress method that links the one-hot and binary encoding of integers.

The third state considered is the Dicke state, a generalization of the $W$-state, a superposition over all computational basis states with Hamming-weight $k$~\cite{Dicke:1954}. 
Dicke states have relevance in various practical settings.
For instance, for quantum game theory~\cite{zdemir2007}, quantum storage~\cite{Bacon_Compress:2006,Plesch:2010}, quantum error correction~\cite{ouyang2014permutation}, quantum metrology~\cite{toth2012multipartite}, and quantum networking~\cite{prevedel2009experimental}. 
Dicke states have been used as a starting state for variational optimization algorithms, most notably Quantum Alternating Operator Ansatz (QAOA)~\cite{Hadfield2019}, to find solutions to problems such as Maximum k-vertex Cover~\cite{Brandhofer2022,cook2020quantum}.
The ground states of physical Hamiltonians describing one-dimensional chains tend to show a resemblance to Dicke states such as states resulting from the Bethe ansatz, making them an ideal starting state when investigating the ground state behavior of these Hamiltonians~\cite{TDL_BetheAnsatzDerivation:2010,B_ExcitedStateQuantumPhaseTransitions:2013,DickeTransitions:2021}. 
For instance, the algorithm by \citeauthor{van2021preparing}, who give an algorithm to prepare the Bethe ansatz eigenstates of the spin-1/2 XXZ spin chain, starts by first preparing a Dicke state~\cite{van2021preparing}. 
A Dicke-state preparation protocol based on the compress-uncompress methodology used in the $W$-state furthermore finds applications in entanglement distillation, where the entanglement of a large state is concentrated on only a few qubits. 
Efficient deterministic circuits for preparing Dicke states have been proposed by \citeauthor{bartschi2019deterministic}~\cite{bartschi2019deterministic, bartschi2022deterministic_short_depth}. 
They provide a quantum circuit of depth $\mathO(k \log(\frac{n}{k}))$, allowing arbitrary connectivity, to prepare a Dicke state, which they conjecture to be optimal when $k$ is constant. 
In this work, we provide a constant-depth $\LAQCC$ circuit below their conjectured bound already for constant $k$. 
However, this does not directly disprove their conjecture, as we allow for intermediate measurements and classical computations. 
More significantly, we even construct constant-depth $\LAQCC$ circuits for $k = \mathO(\sqrt{n})$ greatly improving their bound.
This construction extends the compress-uncompress method for the $W$-state combined with additional subroutines. 

We continue with a log-depth state preparation protocol for the Dicke-state for arbitrary $k$. 
This protocol implements an efficient transformation between the factoradic number representation and the combinatorial number representation of a positive integer. 
The combinatorial number representation relates directly to the Dicke state. 
The provided efficient transformation between number representation systems might be of independent interest. 

We conclude by modifying our protocol for preparing a Dicke-state to a protocol that prepares quantum many-body scar states in constant-depth. 
These states have low entanglement and longer coherence times than states with similar energy density.
These characteristics make many-body scar states interesting to analyze and relevant within physics.
Many-body scar states appear for instance in the AKLT model~\cite{AKLT:1987,MRBAR:2018,MRB:2018} and different spin models~\cite{SI:2019,MOBFR:2020}.
Known methods for preparing these states have polynomial-depth~\cite{Gustafson:2023}, whereas our circuit has constant depth. 

% We conclude by studying the power that intermediate classical calculations can add to quantum computations. 
% In this study, we define a new model that relaxes constant-depth quantum circuits to polynomial depth quantum circuits, log-depth classical calculations to unbounded classical computations and a constant number of alternations to a polynomial number of alternations. 
% We call this model $\LAQCC^*$. 
% We study this model by doing a complexity theoretical analysis, where we draw inspiration from the notions of complexity given by \citeauthor{RosenthalYuen:2022}, \citeauthor{MetgerYuen:2023}, and \citeauthor{Aaronson:2004}.
% All three complexity notions are based on the notion of state preparation, instead of more traditional definition of complexity such as the decidability of a computational problem. 
% The first two consider classes based on sequences of quantum states preparable by a polynomial-sized quantum circuit, where the circuits are uniformly generated by a computational class, for instance, the class $\mathsf{PSPACE}$, which results in the complexity class $\mathsf{StatePSPACE}$~\cite{RosenthalYuen:2022,MetgerYuen:2023}.
% The third notion considers a relative complexity, where the complexity is measured between two given states, and is measured by the number of gates, from a given gate-set, required to transform one state in another state~\cite{Aaronson:2004}. 
% For our definition of state preparation complexity, we drop the uniformity constraint from~\cite{RosenthalYuen:2022,MetgerYuen:2023} and define a class as $\mathsf{StateX}$, which refers to states preparable by circuits of type $\mathsf{X}$. 
% As an example, if $\mathsf{X} = \QNC^0$, this results in the class $\mathsf{StateQNC^0}$, which is the set of states preparable from the $\ket{0}^n$ state by poly-size constant-depth circuits. 
% This notion is similar to the relative complexity from~\cite{Aaronson:2004}, where one state is the  $\ket{0}^n$ state and instead of counting the number of gates we consider the set of states preparable by a fixed number of gates. Using this notion of complexity we show that any state preparable by an $\LAQCC^*$ circuit is also preparable by a $\mathsf{PostQPoly}$ circuit, the class of circuits of polynomial depth with an additional post-selection gate. 

\paragraph{Summary of results}
\begin{itemize}
    \item We give a new definition of a computational model that captures the power of the four step process: applying a constant number of layers of one- and two-qubit gates; performing a syndrome measurement; perform a fast classical computation determining corrections; apply corrections. We call this model \emph{Local Alternating Quantum Classical Computations}, or $\LAQCC$ for short. In this model we bound the allowed quantum operations, intermediate classical calculations, and number of rounds separately. In Section~\ref{sec:LAQCC_model} we define this model and give a list of operations based on results from literature contained in this computational model. In some of these operations we explicitly use that we allow for multiple, but at most constant, rounds  of corrections.
    \item  We show show that there exist $\LAQCC$ circuits that can not be weakly simulated in Section~\ref{sec:IQP_in_LAQCC}. We further show that for every $\LAQCC$ circuit there exists a $\QNC^1$ circuit simulating it perfectly, in Section~\ref{sec:LAQCC_in_QNC1}.
    \item We introduce a new type computational complexity for preparing states and show that the extension of $\LAQCC$ where we allow a polynomial number of rounds and unbounded classical computation, is contained in $\mathsf{PostQPoly}$, the class of polynomial circuits with post-selection, in Section~\ref{sec:Complexity results}.
    \item We show a protocol to prepare the uniform superposition state of size $q$ in $\LAQCC$ using $\mathO(\ceil{\log_2(q)}^2)$ qubits in Section~\ref{sec:superposition_modulo_q}. 
    \item We show a protocol to prepare the $W_n$ state in $\LAQCC$ using $\mathO(n\log(n))$ qubits in Section~\ref{sec:W_state_in_LAQCC}.
    \item We show two ways of preparing the Dicke-$(n,k)$ state. The first method is in $\LAQCC$, works up to $k = \mathO(\sqrt{n})$, uses $\mathO(n^2\log(n))$ qubits, and is found in Section~\ref{sec:dicke:small_k}. The second method is in $\LAQCC\text{-}\mathsf{LOG}$ (an extension of $\LAQCC$ allowing for logarithmic number of alterations instead of constant), works for any $k$, uses $\mathO(\text{poly}(n))$ qubits, and is found in Section~\ref{sec:Dicke_in_LAQCC_LOG}. 
    \item We extend on our $\LAQCC$ method of generating Dicke-$(n,k)$ states for $k = \mathO(\sqrt{n})$ and show a protocol to generate many-body scar states for a particular Hamiltonian in $\LAQCC$ (Section~\ref{sec:many_body_scar}). 
\end{itemize}
Summarized in a table, we provide the following state generation protocols:
\begin{table}[htb]
\centering
\begin{tabular}{l|l|l|l}
\textbf{State description} & \textbf{Width} & \textbf{Depth} & \textbf{Implementation}\\
\hline 
Uniform superposition mod $q$: $\frac{1}{\sqrt{q}} \sum_{i = 0}^{q-1}\ket{i}$ & $\mathO(\ceil{\log^2 q})$ & $\mathO(1)$ & Section~\ref{sec:superposition_modulo_q}\\

$W$-state: $\frac{1}{\sqrt{n}}\sum_{i = 0}^{n-1}\ket{e_i}$ & $\mathO(n \log n)$ & $\mathO(1)$ & Section~\ref{sec:W_state_in_LAQCC}\\

Dicke-$(n,k)$, $k = \mathO(\sqrt{n})$: $\binom{n}{k}^{-1/2}\sum_{x \in \{0,1\}^n: |x| = k} \ket{x}$ &  $\mathO(n^2\log n)$ & $\mathO(1)$ 
&Section~\ref{sec:dicke:small_k}\\

Dicke-$(n,k)$: $\binom{n}{k}^{-1/2}\sum_{x \in \{0,1\}^n: |x| = k} \ket{x}$ & $\mathO(\text{poly}(n))$ & $\mathO(\log n)$ &Section~\ref{sec:Dicke_in_LAQCC_LOG}\\

QMBS: $\ket{S_k} = \frac{1}{k! \sqrt{\mathcal N(n,k)}}(Q^\dagger)^k \ket{\Omega}$ &  $\mathO(n^2\log n)$ & $\mathO(1)$  &  Section~\ref{sec:many_body_scar}
\end{tabular}
\caption{Summary of state preparation protocols given in this paper.}
\label{tab:sate_prep}
\end{table}
In the entry for the quantum many-body scar state $Q$ denotes the raising operator and $\mathcal N(n,k)=\binom{n-k-1}{k}$. 
Section~\ref{sec:many_body_scar} will provide more details on the variables and the implementation. 

\paragraph{Organization of the paper}
\noindent We first introduce relevant preliminaries in Section~\ref{sec:preliminaries}. 
In Section~\ref{sec:LAQCC_model} we formally define the class of Local Alternating Quantum-Classical Computations ($\LAQCC$). We also show that any Clifford circuit can be implemented in constant depth $\LAQCC$ (a result based on a result from measurement-based quantum computing~\cite{jozsa2006introduction}). 
This result allows us to give many useful multi-qubit gates and routines in Section~\ref{sec:gates_created_in_LAQCC}. 
Beyond that we show that constant depth $\LAQCC$ circuits are contained in $\QNC^1$ and that any $\mathsf{IQP}$ circuit has an $\LAQCC$ implementation.
We conclude this section with an analysis of a more powerful instantiation of $\LAQCC$ and show an inclusion with respect to the class $\mathsf{PostQPoly}$, which is the class of circuits of polynomial depth with one additional post-selection gate. 
In Section~\ref{sec:state_prep_in_LAQCC} we give $\LAQCC$ circuit implementations for preparing the uniform superposition over an arbitrary number of states, the $W$-state and the Dicke state up to $k = \mathO(\sqrt{n})$. We furthermore give a log-depth circuit implementation for preparing the Dicke state for any $k$. We conclude by showing a $\LAQCC$ circuit for generating many body scar states of a particular type of Hamiltonian.


\section{Preliminaries}
In this section, we describe the necessary background for automated planning and the significance of the International Planning Competition. 

% \subsection{Ontology}
% A formal ontology is typically represented as a set of concepts, relations, and axioms. A concept represents a set of objects or entities that share common properties, while a relation represents a connection or association between two or more concepts. Axioms are statements that define the relationships between concepts and relations. It is a formal representation of knowledge that is designed to facilitate automated reasoning and information processing. It acts as a structured vocabulary that describes a domain and promotes interoperability, data integration, and communication between humans and machines. Formally, an ontology $O$ can be represented as a tuple $(C, R, A)$, where $C$ is the set of concepts, $R$ is the set of relations, and $A$ is the set of axioms. Each concept \textit{c} $\in$ $C$ can be represented as a set of attributes, denoted as $Att(c)$. Similarly, each relation \textit{r} $\in$ $R$ can be represented as a set of attributes, denoted as $Att(r)$.

% Ontology is a branch of philosophy that deals with the nature of existence and being. In the field of computer science, however, ontology refers to a formal representation of knowledge that is designed to facilitate automated reasoning and information processing. It is a structured vocabulary that describes a domain and promotes interoperability, data integration, and communication between humans and machines. Various tools and methodologies, including Protege and ontology editors, are available for ontology creation. Ontologies are increasingly important in artificial intelligence, knowledge engineering, and the semantic web, and researchers are exploring their potential in diverse domains and applications.

% Figure environment removed

\subsection{Automated Planning}

Automated planning, also known as AI planning, is the process of finding a sequence of actions that will transform an initial state of the world into a desired goal state \cite{ghallab2004automated}. It involves constructing a plan or a sequence of actions that will achieve a specified objective while respecting any constraints or limitations that may be present. Formally, automated planning can be defined as a tuple $(S, A, T, I, G)$, where:
\begin{itemize}
    \item $S$ is the set of possible states of the world
    \item $A$ is the set of possible actions that can be taken
    \item $T$ is the transition function that describes the effects of taking an action on the current state of the world
    \item $I$ is the initial state of the world
    \item $G$ is the desired goal state
\end{itemize}
Using this notation, the problem of automated planning can be framed as finding a sequence of actions $\prec a_1, a_2, ..., a_k\succ$ that will transform the initial state $I$ into the goal state $G$, while respecting any constraints or limitations on the actions. 
 % In automated planning, 
 A problem is defined in terms of a domain and a problem instance. The domain defines the possible actions that can be taken and the effects of each action, while the problem instance specifies the initial state of the world and the desired goal state. 
Various techniques can be used to solve the planning problem, such as search algorithms, constraint-based reasoning, and optimization methods. These techniques involve exploring the space of possible plans and selecting the one that satisfies the objective and any constraints. Figure \ref{fig:planning_bw} illustrates an automated planning scenario for the blocksworld domain, where an initial state can be transformed into a goal state by executing a sequence of actions.

% \noindent \textbf{Attributes modeled about a domain.}
%   %\noindent \textbf{Attributes modeled in a domain file}
%  \begin{enumerate}
%      \item \textbf{Requirements:} A list of requirements that the planner must satisfy in order to solve the domain. Requirements include durative actions, conditional effects, or negative preconditions. For example, in blocksworld domain with types involved, one of the requirements is \emph{typing}.
%     \item \textbf{Predicates:} Predicates are fundamental elements in the planning domain that define the properties of the world. They are used to describe the initial and goal states, as well as the preconditions and effects of actions. Predicates are usually defined as logical expressions over a set of variables, where each variable can take on a finite number of values. In the context of planning, predicates are typically used to represent facts about the world that can be true or false, such as the location of an object or the status of a machine. For example, in blocksworld domain, the predicate \verb|(on b1 b2)| could indicate that block 'b2' is on top of block 'b1'.
%      \item \textbf{Actions:} Actions are the basic units of change in the planning domain. They represent atomic operations that can be performed to transform the world from one state to another. Each action has a name, a set of parameters, preconditions that must be satisfied before the action can be executed, and effects that describe the changes that the action makes to the world. Actions can be used to model a wide variety of operations, ranging from simple movements or transformations to complex processes such as planning or decision-making. For example, in blocksworld domain, the action \verb|unstack b2 b1| can be used to unstack block 'b2' from block 'b1'. 
     
%      \item \textbf{Preconditions:} Preconditions are the conditions that must be true before an action can be executed. They are usually defined using predicates and can involve multiple variables. Preconditions can also be negative, which means that a certain condition must not be true for an action to be executed. In planning, preconditions ensure that actions are only executed when the necessary conditions have been met, such as ensuring that a machine is turned off before it is serviced. For example, in blocksworld domain, the action \verb|unstack b2 b1| has a precondition of \verb|(on b1 b2)|, meaning that for the action to be valid, the block 'b2' should be on top of block 'b1'.
     
%      \item \textbf{Effects:} Effects describe the changes that an action makes to the world. They are usually defined using predicates and can involve multiple variables. Effects can be positive, which means that a certain condition becomes true after the action is executed, or negative, which means that a certain condition becomes false after the action is executed. In the context of planning, effects are used to model the changes that result from executing an action, such as moving an object from one location to another or turning a machine on. For example, in blocksworld domain, when the action \verb|unstack b2 b1| is executed, one of its effect is \verb|(not (on b1 b2))|, indicating that block 'b2' is no longer on top of block 'b1'.
     
%      \item \textbf{Constants:} Constants are values that are fixed and do not change during the execution of the planning problem. They are used to represent objects or entities in the world that have a fixed value, such as the speed limit on a road. Constants can be used to simplify the planning problem by reducing the number of variables that need to be considered and by providing a fixed set of values that can be used in predicates and actions. For example, in blocksworld domain, the constant \emph{table} could represent the surface on which the blocks are initially placed.
     
%      \item \textbf{Types:} Types are used to classify objects or entities in the world based on their attributes or properties. They are used to define the domain of values that a variable can take on and can be used to constrain the values that are assigned to variables. In the context of planning, types are typically used to group related objects or entities together, such as cars or bicycles, and to specify the properties that are common to all members of a type, such as their color or size. For example, in blocksworld domain with types involved, one can represent the predicate as \verb|(on ?x - block ?y - block)| stating that the parameters in the predicate are of type \emph{block}.

%  \end{enumerate}


% ######### Shorter version for AI Planning preliminaries
% \subsection{Automated Planning}

% Automated planning, also known as AI planning, finds actions transforming an initial world state into a goal state \cite{ghallab2004automated}. It involves creating a plan, respecting constraints, defined as $(S, A, T, I, G)$ where $S$ is the world states set, $A$ is the actions set, $T$ is the state transition function, $I$ is the initial state, and $G$ is the goal state. The challenge is to find actions $\prec a_1, a_2, ..., a_k\succ$ converting $I$ to $G$ under constraints. 

% A problem has a domain (defining actions and effects) and an instance (specifying initial and goal states). Various techniques can be used to solve the planning problem, such as search algorithms, constraint-based reasoning, and optimization methods. These techniques involve exploring the space of possible plans and selecting the one that satisfies the objective and any constraints. Figure \ref{fig:planning_bw} illustrates an automated planning scenario for the blocksworld domain, where an initial state can be transformed into a goal state by executing a sequence of actions.

\noindent \textbf{Attributes modeled about a domain.}
 \begin{enumerate}
     \item \textbf{Requirements:} A list of requirements that the planner must satisfy to solve the given domain, e.g., \emph{typing} in blocksworld with types.
     \item \textbf{Predicates:} Define world properties, e.g., \verb|(on b1 b2)| in blocksworld.
     \item \textbf{Actions:} Units of change with preconditions and effects, e.g., \verb|unstack b2 b1| in blocksworld.
     \item \textbf{Preconditions:} Conditions for action execution, e.g., \verb|(on b1 b2)| for \\ \verb|unstack b2 b1|.
     \item \textbf{Effects:} Post-action world changes, e.g., \verb|(not (on b1 b2))| after \\ \verb|unstack b2 b1|.
     \item \textbf{Constants:} Fixed values, e.g., \emph{table} in blocksworld.
     \item \textbf{Types:} Classifications based on attributes, e.g., \\ \verb|(on ?x - block ?y - block)| in typed blocksworld.
 \end{enumerate}

\noindent \textbf{Attributes modeled about a problem instance from a domain.}
\begin{enumerate}
    \item \textbf{Name:} The name of the planning problem.
    \item \textbf{Domain:} The name of the planning domain that the problem belongs to.
    \item \textbf{Objects:} A list of objects that are present in the planning problem. Objects are typically defined in terms of their type and name. In the example shown in Figure \ref{fig:planning_bw}, objects are b1, b2, and b3.
    \item \textbf{Initial State:} A description of the initial state of the world, including the values of all relevant predicates. Figure \ref{fig:planning_bw} represents an example initial state.
    \item \textbf{Goal State:} A description of the desired goal state of the world, including the values of all relevant predicates. Figure \ref{fig:planning_bw} represents an example goal state.
\end{enumerate}

% \vspace{2cm}
\subsection{International Planning Competition (IPC)}

% IPC serves as a significant means of assessing and comparing various planning systems. By presenting new planners and benchmark problems each year, the competitions aim to stimulate the advancement of new planning methodologies and reflect current trends and challenges in the field. The competition comprises multiple tracks, each covering various planning problems such as classical, temporal, and probabilistic planning. These tracks include benchmark problems that evaluate the performance of planners concerning parameters such as plan quality, plan length, and run time. The results of these competitions provide insights into the current state-of-the-art in planning and help identify the strengths and weaknesses of different planning systems. IPC can serve as an excellent starting point for building a planning-related ontology as the benchmark problems used in these competitions can provide a comprehensive overview of the domain and the types of problems that planners need to solve. 

IPC is pivotal for evaluating and contrasting planning systems. Introducing new planners and benchmarks, it promotes innovative planning methodologies and reflects the field's evolving challenges. The competition has multiple tracks, such as classical and probabilistic planning, with benchmarks assessing plan quality, length, and run time. IPC results offer a glimpse into the latest in planning, highlighting system pros and cons. The benchmarks from IPC are ideal for crafting a planning-related ontology, encapsulating the domain's breadth and planners' challenges.

\section{Secure Design of \puma}\label{sec:design}
In this section, we first present an overview of \puma, and present the protocols for secure $\gelu$ , $\softmax$, embedding, and $\layernorm$ used by \puma. Note that the linear layers such as matrix multiplication are straightforward in replicated secret sharing, so we mainly describe our protocols for non-linear layers in this manuscript.

\subsection{Overview of \puma}\label{sec:overview}
To achieve secure inference of Transformer models, \puma\ defines three kinds of roles: one model owner, one client, and three computing parties. The model owner and the client  provide their models or inputs to the computing parties (i.e., $P_0$, $P_1$, and $P_2$) in a secret-shared form, then the computing parties execute the MPC protocols and send the results back to the client. Note that the model owner and client can also act as one of the computing party, we describe them separately for generality. \eg, when the model owner acts as $P_0$, the client acts as  $P_1$, a third-party dealer acts as $P_2$, the system model becomes the same with \mpcformer~\citep{li2023mpcformer}.

During the secure inference process, a key invariant is maintained: For any layer, the computing parties always start with 2-out-of-3 replicated secret shares of the previous layer's output and the model weights, and end with 2-out-of-3 replicated secret shares of this layer's output. As the shares do not leak any information to each party, this ensures that the layers can be sequentially combined for arbitrary depths to obtain a secure computation scheme for any Transformer-based model.
%The main focus of \puma\ is to reduce the computation and communication costs between the computing parties while maintaining the desired level of security. 



\iffalse
\textbf{Threat Model.}
Following previous works~\citep{aby3,li2023mpcformer},
\puma\ resists a semi-honest (a.k.a., honest-but-curious) adversary in honest-majority~\citep{lindell2009proof}, where the adversary passively corrupts no more than one computing party. Such an adversary follows the protocol specification exactly, but may try to learn more information than permitted. Please note that \puma\ cannot protect against the extraction of information from the inference results, and the examination of mitigating solutions (\eg, differential privacy~\citep{abadi2016deep}) falls outside the scope of this study.
\fi 

\subsection{Protocol for Secure GeLU}\label{sec:gelu}
Most of the current approaches view the $\gelu$ function as a composition of smaller functions and try to optimize each piece of them, making them to miss the
chance of optimizing the private $\gelu$ as a whole. Given the $\gelu$ function:
\begin{equation}\label{eq:gelu}
\begin{split}
    \gelu(x) &= \frac{x}{2} \cdot \left(1 + \tanh \left( \sqrt{\frac{2}{\pi}} \cdot \left(x + 0.044715 \cdot x^3 \right) \right) \right)\\
    &\approx x\cdot \mathsf{sigmoid}(0.071355\cdot x^3 + 1.595769\cdot x) 
\end{split},
\end{equation}
these approaches~\citep{hao2022iron,characmpctranformer} focus either on designing efficient protocols for function $\tanh$
or using the existing MPC protocols of exponentiation and reciprocal for $\mathsf{sigmoid}$. 

However, none of current approaches have utilized the fact that $\gelu$ function is almost linear on the two sides (\ie, $\gelu(x)\approx 0$ for $x<-4$ and $\gelu(x)\approx x$ for $x>3$). 
Within the short interval $[-4,3]$ of $\gelu$,
we suggest a piece-wise approximation of low-degree polynomials is a more efficient and easy-to-implement choice for its secure protocol. Concretely, our piece-wise low-degree polynomials are shown as equation~(\ref{eq:geluapprox}):
\begin{equation}\label{eq:geluapprox}
\gelu(x)=
\begin{cases}
0, & x<-4 \\
F_0(x), & -4 \le x < -1.95 \\
F_1(x), & -1.95 \le x \le 3 \\
x, & x >3
\end{cases},
\end{equation}
where polynomials $F_0()$ and $F_1()$ are computed by library $\mathsf{numpy.ployfit}$\footnote{\url{https://numpy.org/doc/stable/reference/generated/numpy.polyfit.html}} as equation~(\ref{eq:f0f1}). Surprsingly, the above simple poly fit works very well and our $\mathsf{max\ error}< 0.01403$, $\mathsf{median\ error}< 4.41e-05$, and $\mathsf{mean\ error}< 0.00168$.
\begin{equation}\label{eq:f0f1}
\begin{cases}
F_0(x) &= -0.011034134030615728 x^3 -0.11807612951181953 x^2 \\
&- 0.42226581151983866 x -0.5054031199708174\\
F_1(x) &= 0.0018067462606141187x^6 -0.037688200365904236 x^4 \\
&+ 0.3603292692789629x^2 + 0.5x + 0.008526321541038084
\end{cases}
\end{equation}

Formally, given secret input $\share{x}$, our secure $\gelu$ protocol $\Pi_{\gelu}$ is constructed as algorithm~\ref{protocol:gelu}. 
\iffalse
\begin{itemize}
    \item The parties jointly compute
$\share{b_0}^2 = \Pi_{\mathsf{LT}}(\share{x}, 4)$,
$\share{b_1}^2 = \Pi_{\mathsf{LT}}(\share{x}, -1.95)$, and
$\share{b_2}^2 = \Pi_{\mathsf{LT}}(3, \share{x})$.

\item  Then, each $P_i$ locally compute
$\share{b_3}^2 = \share{b_1}^2 \oplus \share{b_2}^ \oplus 1$ and
$\share{b_4}^2 = \share{b_0}^2 \oplus \share{b_1}^2$

\item Finally, the parties compute and return 
$\share{b_2}^2 \cdot \share{x} + \share{b_4}^2 \cdot F_0(\share{x}) + \share{b_3}^2 \cdot F_1(\share{x})$, where polynomials $(F_0, F_1)$ can be computed easily using secure addition and multiplication (and its variants, \eg, secure square)~\citep{spu}. 
\end{itemize}
\fi 

\begin{algorithm}[tp]
\caption{Secure $\gelu$ Protocol $\Pi_{\mathsf{GeLU}}$}\label{protocol:gelu}
\begin{algorithmic}[1]
\REQUIRE
$P_i$ holds the 2-out-of-3 replicate secret share $\share{x}_i$ for $i\in \{0,1,2\}$ 
\ENSURE
$P_i$ gets the 2-out-of-3 replicate secret share $\share{y}_i$ for $i\in \{0,1,2\}$, where $y=\gelu(x)$.

\STATE $P_0$, $P_1$, and $P_2$ jointly compute
\begin{equation*}
\begin{split}
&\shareb{b_0} = \Pi_{\mathsf{LT}}(\share{x}, -4),~~~\vartriangleright b_0 = 1\{x<-4\}\\
&\shareb{b_1} = \Pi_{\mathsf{LT}}(\share{x}, -1.95),~~~\vartriangleright b_1 = 1\{x<-1.95\} \\
&\shareb{b_2} = \Pi_{\mathsf{LT}}(3, \share{x}),~~~~~~\vartriangleright b_2 = 1\{3<x\}
\end{split}
\end{equation*}
and compute 
$\shareb{z_0} = \shareb{b_0} \oplus \shareb{b_1}$,
$\shareb{z_1} = \shareb{b_1} \oplus \shareb{b_2} \oplus 1$, and $\shareb{z_2}=\shareb{b_2}$. Note that $z_0 = 1\{-4\le x < -1.95\}$, $z_1 = 1\{-1.95\le x\le 3\}$, and $z_2 = 1\{x>3\}$.

\STATE Jointly compute $\share{x^2} = \Pi_{\mathsf{Square}}(\share{x})$, $\share{x^3} = \Pi_{\mathsf{Mul}}(\share{x}, \share{x^2})$, $\share{x^4} = \Pi_{\mathsf{Square}}(\share{x^2})$, and $\share{x^6} = \Pi_{\mathsf{Square}}(\share{x^3})$.

\STATE Computing polynomials $\share{F_0(x)}$ and $\share{F_1(x)}$ based on $\{\share{x}, \share{x^2}, \share{x^3}, \share{x^4}, \share{x^6}\}$ as equation~(\ref{eq:geluapprox}) securely.


\RETURN$\share{y} = \Pi_{\mathsf{Mul_{BA}}}(\shareb{z_0}, \share{F_0(x)}) + \Pi_{\mathsf{Mul_{BA}}}(\shareb{z_1}, \share{F_1(x)})+\Pi_{\mathsf{Mul_{BA}}}(\shareb{z_2}, \share{x})$.

\end{algorithmic}
\end{algorithm}



\subsection{Protocol for Secure Softmax}\label{sec:secureatten}

In the function $\attention(\Q,\K,\V)=
\softmax(\Q \cdot \K^\mathsf{T} + \M) \cdot \V$, where $\M$ can be viewed as a bias matrix, the key challenge is computing function $\softmax$. For the sake of numerical stability, the $\softmax$ function is computed as
\begin{equation}\label{eq:softmax}
    \softmax(\x)[i]=\frac{\exp(\x[i] - \bar{x} - \epsilon)}{\sum_i \exp(\x[i] - \bar{x} - \epsilon)},
\end{equation}
where $\bar{x}$ is the maximum element of the input vector $\x$. 
For the normal plaintext softmax, $\epsilon=0$. For a two-dimension matrix, we apply equation~(\ref{eq:softmax}) to each of its row vector.

Formally, our detailed secure protocol  $\Pi_{\softmax}$ is illustrated in algorithm~\ref{protocol:softmax}, where we propose two optimizations:
\begin{itemize}
\item 
For the first optimization, we set $\epsilon$ in equation~\ref{eq:softmax} to a tiny and positive
value, e.g., $\epsilon =
10^{-6}$, so that the inputs to exponentiation
in equation~\ref{eq:softmax} are all negative. We exploit the negative operands
for acceleration. Particularly, we compute the exponentiation using the Taylor series~\citep{tan2021cryptgpu} with a simple clipping
\begin{equation}\label{eq:negexp}
\mathsf{negExp}(x) = \begin{cases}
    0, &x < T_{\exp} \\
    (1+\frac{x}{2^t})^{2^t}, &x\in [T_{\exp},0].
\end{cases}
\end{equation}
Indeed, we apply the less-than for the branch $x < T_{\exp}$
The division by $2^t$ can be achieved using
$\Pi_{\mathsf{Trunc}}^t$ since the input is already negative. Also, we can
compute the power-of-$2^t$ using $t$-step sequences of square function $\Pi_{\mathsf{square}}$ and $\Pi_{\mathsf{Trunc}}^f$. Suppose our MPC program uses
$18$-bit fixed-point precision. Then we set $T_{\exp}=-14$ given $\exp(-14) < 2^{-18}$, and empirically set $t = 5$.


\item 
Our second optimization is to reduce the number of divisions, which ultimately saves computation and communication costs.
To achieve this, for a vector $\x$ of size $n$, we have replaced the operation $\mathsf{Div}(\x, \mathsf{Broadcast}(y))$ with $\x \cdot  \mathsf{Broadcast}(\frac{1}{y})$, where $y=\sum_{i=1}^n\x[i]$. By making this replacement, we effectively reduce $n$ divisions to just one reciprocal operation and $n$ multiplications.
This optimization is particularly beneficial in the case of the $\softmax$ operation. The $\frac{1}{y}$ in the $\softmax$ operation is still large enough to maintain sufficient accuracy under fixed-point values. As a result, this optimization can significantly reduce the computational and communication costs while still providing accurate results.
\end{itemize}

\begin{algorithm}[tp]
\caption{Secure $\softmax$ Protocol $\Pi_{\softmax}$}\label{protocol:softmax}
\begin{algorithmic}[1]
\REQUIRE
$P_i$ holds the 2-out-of-3 replicate secret share $\share{\x}_i$ for $i\in \{0,1,2\}$, and $\x$ is a vector of size $n$. 
\ENSURE
$P_i$ gets the 2-out-of-3 replicate secret share $\share{\y}_i$ for $i\in \{0,1,2\}$, where $\y=\softmax(\x)$.

\STATE $P_0$, $P_1$, and $P_2$ jointly compute
$\shareb{\mathbf{b}} = \Pi_{\mathsf{LT}}(T_{\exp}, \share{\x})$ and the maximum $\share{\bar{x}} = \Pi_{\mathsf{Max}}(\share{\x})$.

\STATE Parties locally computes $\share{\hat{\x}} = \share{\x} - \share{\bar{x}} - \epsilon$, and jointly compute $\share{\z_0} = 1+  \Pi_{\mathsf{Trunc}}^t(\share{\hat{\x}})$.

\FOR{$j=1,2,\dots, t$}
\STATE $\share{\z_j} = \Pi_{\mathsf{Square}}(\share{\z_{j-1}})$.
\ENDFOR

\STATE Parties locally compute $\share{z} = \sum_{i=1}^n \share{\z[i]}$ and jointly compute $\share{1/z} = \Pi_{\mathsf{Recip}}(\share{z})$.

\STATE Parties jointly compute $\share{\z / z} = \Pi_{\mathsf{Mul}}(\share{\z}, \share{1/z})$

\RETURN $\share{\y} = \Pi_{\mathsf{Mul}_{\mathsf{BA}}}( \shareb{\mathbf{b}}, \share{\z / z})$.

\end{algorithmic}
\end{algorithm}

\subsection{Protocol for Secure Embedding}\label{sec:embed}


The current secure embedding procedure described in~\citep{li2023mpcformer} necessitates the client to  generate a one-hot vector using the token $\tokenid$ locally. This deviates from a plaintext Transformer workflow where the one-hot vector is generated inside the model. As a result, they have to carefully strip off the one-hot step from the pre-trained models, and add the step to the client side, which could be an obstacle for deployment. 



To address this issue, we propose a secure embedding design as follows. Assuming that the token $\tokenid\in [n]$ and all embedding vectors are denoted by $\E= (\e_1^T, \e_2^T, \dots, \e_n^T)$, the embedding can be formulated as $\e_{\tokenid} = \mathbf{E}[\tokenid]$. Given $(\tokenid, \E)$ are in secret-shared fashion, our secure embedding protocol $\Pi_{\mathsf{Embed}}$ works as follows:
\begin{itemize}
    \item The computing parties securely compute the one-hot vector $\shareb{\mathbf{o}}$ after receiving $\share{\tokenid}$ from the client. Specifically, $\shareb{\mathbf{o}[i]}=\Pi_{\mathsf{Eq}}(i,\share{\tokenid})$ for $i\in [n]$.
    \item The parties can compute the embedded vector via $\share{\e_{\tokenid}} = \Pi_{\mathsf{Mul_{BA}}}(\share{\E}, \shareb{\mathbf{o}})$, where  does not require secure truncation.
\end{itemize}
In this way, our $\Pi_{\mathsf{Embed}}$ does not require explicit modification of the workflow of plaintext Transformer models, at the cost of more $\Pi_{\mathsf{Eq}}$ and $\Pi_{\mathsf{Mul_{BA}}}$ operations. 



\subsection{Protocol for Secure LayerNorm}\label{sec:seclayernorm}
Recall that given a vector $\x$ of size $n$, $\layernorm(\x)[i] =  \gamma \cdot \frac{\x[i]-\mu}{\sqrt{\sigma}} + \beta$, where $(\gamma, \beta)$ are trained parameters, $\mu = \frac{\sum_{i=1}^n \x[i]}{n}$, and $\sigma = \sum_{i=1}^n (\x[i] - \mu)^2$. In MPC, the key challenge is the evaluation of the divide-square-root $\frac{\x[i]-\mu}{\sqrt{\sigma}}$ formula. To securely evaluate this formula, CrypTen sequentially executes the MPC protocols of square-root, reciprocal, and multiplication. However, we observe that $\frac{\x[i]-\mu}{\sqrt{\sigma}}$ is equal to $(\x[i]-\mu)\cdot \sigma^{-1/2}$. And in the MPC side, the costs of computing the inverse-square-root $\sigma^{-1/2}$ is similar to that of the square-root operation~\citep{rSqrt}. Besides, inspired by the second optimization of \S~\ref{sec:secureatten}, we can first compute $\sigma^{-1/2}$ and then $\mathsf{Broadcast}(\sigma^{-1/2})$ to support fast and secure $\layernorm(\x)$. And our formal protocol $\Pi_{\layernorm}$ is shown in algorithm~\ref{protocol:layernorm}.

\begin{algorithm}[tp]
\caption{Secure $\mathsf{LayerNorm}$ Protocol $\Pi_{\mathsf{LayerNorm}}$}\label{protocol:layernorm}
\begin{algorithmic}[1]
\REQUIRE
$P_i$ holds the 2-out-of-3 replicate secret share $\share{\x}_i$ for $i\in \{0,1,2\}$, and $\x$ is a vector of size $n$. 
\ENSURE
$P_i$ gets the 2-out-of-3 replicate secret share $\share{\y}_i$ for $i\in \{0,1,2\}$, where $\y=\mathsf{LayerNorm}(\x)$.

\STATE $P_0$, $P_1$, and $P_2$ compute $\share{\mu} = \frac{1}{n}\cdot \sum_{i=1}^n\share{\x[i]}$ and $\share{\sigma} = \sum_{i=1}^n \Pi_{\mathsf{Square}}(\share{\x} - \share{\mu})[i]$.

\STATE Parties jointly compute $\share{\sigma^{-1/2}} = \Pi_{\mathsf{rSqrt}}(\share{\sigma})$.

\STATE Parties jointly compute $\share{\mathbf{c}} = \Pi_{\mathsf{Mul}}((\share{\x} - \share{\mu}), \share{\sigma^{-1/2}})$

\RETURN $\share{\y} = \Pi_{\mathsf{Mul}}(\share{\gamma}, \share{\mathbf{c}}) + \share{\beta}$.

\end{algorithmic}
\end{algorithm}
\section{Disjunctive Affine Invariant Generation for Unnested Loops}

In this section, we present our approach for generating affine disjunctive loop invariants over unnested affine while loops. 
Throughout the section, we fix the set of program variables 
as $X=\{x_1,\dots,x_n\}$ and identify the set $X$ as the set of variables in the 
\LTS{} to be derived from the loop. We consider the canonical form of an unnested affine while loop as in Figure~\ref{fig:unnestedPQandRecursive}, where we have: 
\begin{itemize}
\item The PAP $G$ is the loop condition (or loop guard) for the while loop.
\item The vector $\mathbf{x}=(x_1,\dots,x_n)^{\mathrm{T}}$ represents the column vector of program variables, and each $\mathbf{F}_i$ ($1\le i\le m$) is an affine function, i.e.,  $\mathbf{F}_i(\mathbf{x})=\mathbf{A} \mathbf{x}+\mathbf{b}$ where $\mathbf{A}$ (resp. $\mathbf{b}$) is an $n\times n$ square matrices (resp. $n$-dimensional column vector) that specifies the affine update under the affine assertion $\phi_i$ (as a conditional branch). The assignment $\mathbf{x}:=\mathbf{F}_i(\mathbf{x})$ is considered simultaneously for the variables in $\mathbf{x}$ so that in one execution step, the current valuation $\tsEval$ is updated to $\mathbf{F}_i(\tsEval)$. 
\item The \textbf{switch} keyword represents a special conditional branching (i.e., different from its original meaning in e.g. C programming language) that if the current values of the program variables satisfy the condition $\phi_i$, then the assignment at the $i$th conditional branch (i.e., $\mathbf{x}:=\mathbf{F}_i(\mathbf{x})$) is executed. Note that the branch conditions $\phi_1,\dots,\phi_m$ need not to be pairwise disjoint (i.e., there can be some valuation $\tsEval$ that satisfies both $\phi_i,\phi_j$ ($i\ne j$)), so that our setting covers nondeterminism in imperative programs.  
\item The statements $\delta_1,\dots,\delta_m$ specify whether the loop continues after the affine update of the conditional branches $\phi_1,\dots,\phi_m$. Each statement $\delta_i$ is either the \textbf{skip} statement that does nothing (which means that the loop continues after the affine update of $\mathbf{F}_i$) or the \textbf{break} statement (which means that the loop exits after the affine update). 
\end{itemize}
A major motivation behind Figure~\ref{fig:unnestedPQandRecursive} is that we treat each top-level branch $\phi_i$ as a standalone branch location and the overall invariant is a disjunction of the invariants at these branch locations. 

\lstset{language=program}
\lstset{tabsize=3}
\newsavebox{\unnestedP}
\begin{lrbox}{\unnestedP}
\begin{lstlisting}[mathescape]
switch {
  case $\phi_{P,1}$: $\mathbf{x}:=F_{P,1}(\mathbf{x})$;$\delta_{P,1}$;
  $\cdots$
  case $\phi_{P,p}$: $\mathbf{x}:=F_{P,p}(\mathbf{x})$;$\delta_{P,p}$;
}
\end{lstlisting}
\end{lrbox}

\lstset{language=program}
\lstset{tabsize=3}
\newsavebox{\unnestedQ}
\begin{lrbox}{\unnestedQ}
\begin{lstlisting}[mathescape]
switch {
  case $\phi_{Q,1}$: $\mathbf{x}:=\mathbf{F}_{Q,1}(\mathbf{x})$;$\delta_{Q,1}$;
  $\cdots$
  case $\phi_{Q,q}$: $\mathbf{x}:=\mathbf{F}_{Q,q}(\mathbf{x})$;$\delta_{Q,q}$;
}
\end{lstlisting}
\end{lrbox}

\lstset{language=program}
\lstset{tabsize=3}
\newsavebox{\unnestedRsequential}
\begin{lrbox}{\unnestedRsequential}
\begin{lstlisting}[mathescape]
switch {
  $\cdots$
  case $\phi_{P,i}$: 
    $\mathbf{x}:=(\mathbf{F}_{P,i}(\mathbf{x}))$;
    $break$;($\mbox{if }\delta_{P,i}=\textbf{break}$)
  $\cdots$
  case $\phi_{P,i}\wedge \phi_{Q,j}[\mathbf{F}_{P,i}(\mathbf{x})/\mathbf{x}]$: 
    $\mathbf{x}:=\mathbf{F}_{Q,j} (\mathbf{F}_{P,i}(\mathbf{x}))$; 
    $\delta_{Q,j}$;($\mbox{if }\delta_{P,i}=\textbf{skip}$)
  $\cdots$
}
\end{lstlisting}
\end{lrbox}

\lstset{language=program}
\lstset{tabsize=3}
\newsavebox{\unnestedRconditional}
\begin{lrbox}{\unnestedRconditional}
\begin{lstlisting}[mathescape]
switch {
  $\cdots$
  case $\phi_{P,i}\wedge b$: 
    $\mathbf{x}:=\mathbf{F}_{P,i}(\mathbf{x})$;$\delta_{P,i}$;
  $\cdots$
  case $\phi_{Q,j}\wedge \neg b$: 
    $\mathbf{x}:=\mathbf{F}_{Q,j}(\mathbf{x})$;$\delta_{Q,j}$;
  $\cdots$
}
\end{lstlisting}
\end{lrbox}

\lstset{language=program}
\lstset{tabsize=3}
\newsavebox{\unnested}
\begin{lrbox}{\unnested}
\begin{lstlisting}[mathescape]
while $G$ {
  switch 
    case $\phi_1$: $\mathbf{x}:=\mathbf{F}_1(\mathbf{x})$;$\delta_1$;
      $\vdots$ 
    case $\phi_m$: $\mathbf{x}:=\mathbf{F}_m(\mathbf{x})$;$\delta_m$;
}
\end{lstlisting}
\end{lrbox}
\vspace{-0.2cm}
% Figure environment removed
\vspace{-0.3cm}

Any unnested affine while loop with break statement can be transformed into the canonical form in Figure~\ref{fig:unnestedPQandRecursive} by recursively examining the substructures of the loop body of the loop. 
A detailed transformation is provided in 
Appendix~\ref{appendix:transform}.
Note that although the transformation into our canonical form may cause exponential blow up in the number of conditional branches in the loop body, in practice a loop typically has a small number of conditional branches and further improvement can be carried out by removing invalid branches (i.e., those whose branch condition is unsatisfiable). Moreover, such a canonical form is often necessary to derive precise disjunctive information for a while loop. 

Below we illustrate our algorithm to generate disjunctive affine invariants on unnested affine while loops. Informally, our algorithm applies the top-level branches and follows Farkas' Lemma for affine invariant generation as in \citet{DBLP:conf/cav/ColonSS03,DBLP:conf/sas/SankaranarayananSM04,oopsla22/scalable}, and further proposes the improvement of invariant generation that is closely related to the top-level branches and has not been considered in the existing  approaches~\cite{DBLP:conf/cav/ColonSS03,DBLP:conf/sas/SankaranarayananSM04,oopsla22/scalable}. Here we first consider an unnested affine while loop $W$. The workflow of our algorithm is demonstrated as follows (\textbf{Step B1} -- \textbf{Step B3}). 

\smallskip
\noindent\textbf{Step B1.} We first transform the loop $W$ into a canonical form $\mathsf{C}_W$ w.r.t Figure~\ref{fig:unnestedPQandRecursive} as stated previously. Taking Example~\ref{eg:realcode} as a running example, the canonical form of the example is given in Example~\ref{eg:transformedcode}. 

\smallskip
\noindent\textbf{Step B2.} Then we apply the top-level branches to transform the loop $\mathsf{C}_W$ into a \LTS{}. The transformation is in a straightforward fashion that every top-level conditional branch (i.e., $\phi_i$ in Figure~\ref{fig:unnestedPQandRecursive}) corresponds to a stand-alone location, and the guard of a transition is determined by the loop condition (i.e., $G$) as well as the branch conditions of the source and target locations of the transition. Formally, we have that the \LTS{} $\Gamma_W$ derived from the loop $W$ is given as follows:
\begin{itemize}
\item The set of locations is $\{\tsLoc_1,\dots, \tsLoc_m, \tsLoc_{e}\}$, 
where each $\tsLoc_i$ ($1\le i\le m$) corresponds to the branch location with branch condition $\phi_i$ and $\tsLoc_{e}$ is the termination program counter of the loop. 
\item For each $1\le i\le m$, if $\delta_i=\mathbf{break}$, we have that transition (where we denote $\mathbf{x}':=(x'_1,\dots,x'_n)^\mathrm{T}$)
$$
\tau_{i}=(\tsLoc_i, \tsLoc_e, G \wedge \phi_i \wedge  \mathbf{x}'=\mathbf{F}_i(\mathbf{x}))
$$
that specifies the one-step jump from the branch location $\tsLoc_i$ to the termination location $\tsLoc_e$, where the guard condition is a conjunction of the loop guard G (for staying in the loop at the current loop iteration), the branch condition $\phi_i$ (that the current execution of the loop body follows the location $\tsLoc_i$) and $\mathbf{x}'=\mathbf{F}_i(\mathbf{x})$ (for the affine update). 
\item For each $1\le i,j\le m$, where $\delta_i\neq \mathbf{break}$, we have the transition 
$$
\tau_{ij}=(\tsLoc_i, \tsLoc_j, G \wedge \phi_i \wedge G[\mathbf{x}'/\mathbf{x}] \wedge \phi_j[\mathbf{x}'/\mathbf{x}] \wedge \mathbf{x}'=\mathbf{F}_i(\mathbf{x}))
$$ 
that specifies the one-step jump from the branch location $\tsLoc_i$ to the branch location $\tsLoc_j$, for which the guard condition is $ G \wedge \phi_i \wedge G[\mathbf{x}'/\mathbf{x}] \wedge \phi_j[\mathbf{x}'/\mathbf{x}]\wedge \mathbf{x}'=\mathbf{F}_i(\mathbf{x})$ since the transition needs to pass the loop guard $G$, satisfy the branch condition $\phi_i$ when staying in the location $\tsLoc_i$, have the affine update specified by $\mathbf{F}_i$ and fulfill the loop guard $G[\mathbf{x}'/\mathbf{x}]$ and the branch condition $\phi_j$ upon entering the location $\tsLoc_j$. 
\item For each $1\le i\le m$, where $\delta_i\neq \mathbf{break}$, we have the transition 
$$
\tau'_{i}=(\tsLoc_i, \tsLoc_{e}, G\wedge \phi_i\wedge (\neg G)[\mathbf{x}'/\mathbf{x}]\wedge \mathbf{x}'=\mathbf{F}_i(\mathbf{x}))
$$ 
for the one-step jump from the branch location $\tsLoc_i$ to the termination location $\tsLoc_e$ for which the guard condition is a conjunction of the loop guard $G$, the branch condition  $\phi_i$, the affine update $\mathbf{x}'=\mathbf{F}_i(\mathbf{x})$ and the negation of the loop guard (for jumping out of the loop). 
\end{itemize} 

After the transformation, we remove transitions with unsatisfiable guard condition to reduce the size of the derived \LTS{}. The transformation for the running example has been given in Example~\ref{eg:runningtransitions}. 


\smallskip
\noindent\textbf{Step B3.} After the transformation into an \LTS{}, we follow existing approaches~\cite{DBLP:conf/cav/ColonSS03,DBLP:conf/sas/SankaranarayananSM04,oopsla22/scalable} that generate affine invariants with Farkas' Lemma. In particular, we apply the recent approach~\cite{oopsla22/scalable} that has the most scalability (see Example ~\ref{eg:farkasapplication} for the running example). A slight difference is that we do not encode the constraints for the termination program location $\tsLoc_{e}$. This is because the invariant at $\tsLoc_{e}$ can be derived 
from non-termination locations to the termination location. In the following, we further propose an invariant propagation technique that takes advantage of a common feature in the top-level branches to improve the time efficiency. 

In our invariant propagation, we explore a special structure in the derived \LTS{} that often arises in the top-level branches, and propose a technique that applies to the special structure and allows one to generate invariants at only one location and obtain the invariants at other locations through a propagation process. To illustrate the invariant propagation, we first identify the special structure of non-crossing affine transition systems. 

\begin{definition}
An \LTS{} $\Gamma$ is \emph{non-crossing} if there exists a depth-first search (DFS) tree of the directed graph $\mbox{\sl DG}(\Gamma)$ (i.e., its underlying directed graph) rooted at the initial location that does not have cross edges. (Recall that a cross edge in a DFS tree is an edge whose destination location is a visited location in the DFS but not an ancestor of the source location of the edge). 
\end{definition}

An example of a non-crossing DFS tree is given in Example~\ref{eg:propagation}, while a simple example that violates the non-crossing property would be a complete directed graph. Non-crossing ATS's are common in the top-level branch form of an unnested while loop. For example, the case of multiphase invariants~\cite{DBLP:conf/cav/SharmaDDA11} is a special case of non-crossing affine transition systems where a location is never entered again once it is left. The strict alternation between branch locations 
is also a special case of non-crossing affine transition systems. In general, any affine transition system that has one outgoing-transition for every location (which arises from deterministic mode change in while loops) is non-crossing, since in its DFS tree there is no cross edges. 

We illustrate the main workflow of our invariant propagation technique. Consider an \LTS{} $\Gamma$ transformed from an unnested affine while loop. Given a DFS tree $T$ of $\mbox{\sl DG}(\Gamma)$ rooted at the initial location $\tsLoc^*$ that has the non-crossing property and a conjunctive affine invariant $\eta(\tsLoc^*)$ at the location $\tsLoc^{*}$ generated from the approach by~\citet{oopsla22/scalable}, the invariant propagation works by repeatedly propagating the invariant $\eta(\tsLoc^*)$ from the root to other locations in a breadth-first search (BFS) from the root $\tsLoc^*$. In the BFS, a single step of propagation that is from a location $\tsLoc$ in the current BFS front with the invariant $\eta(\tsLoc)$ (as a DNF PAP) computed from the prior BFS process to a location $\tsLoc'$ in the next front, considers all transitions from 
$\tsLoc$ to $\tsLoc'$; for each such transition  $\tau=(\tsLoc, \tsLoc',\tsGuardcond)$, our approach computes a DNF PAP as an invariant $I(\tau,\tsLoc')$ for the \LTS{} $\Gamma[\tsLoc',K_\tau:=\{\tsEval'\mid \exists \tsEval.(\tsEval\models \eta(\tsLoc) \wedge \tsEval,\tsEval'\models \tsGuardcond)\}]$ (see Page~\pageref{pg:selfloop} for the definition of $\Gamma[-,-]$) via the approach by~\citet{oopsla22/scalable} 
and disjuncts all these $I(\tau,\tsLoc')$'s together to obtain $\eta(\tsLoc')$. Note that in such an ATS $\Gamma[\tsLoc',K_\tau:=\{\tsEval'\mid \exists \tsEval.(\tsEval\models \eta(\tsLoc) \wedge \tsEval,\tsEval'\models \tsGuardcond)\}]$ we consider self-loop transitions at a location $\tsLoc'$ since our approach needs to cover the case that when propagated to the location $\tsLoc'$, the ATS (and the original program) may dwell at the branch location $\tsLoc'$ for a finite unbounded number of steps. The invariant at the termination location $\tsLoc_{e}$ is also obtained by performing a single propagation step from the non-termination locations. 



The details of a single propagation in the BFS is as follows. 
Consider a location $\tsLoc$ at the current BFS front with the computed PAP invariant $\eta(\tsLoc)=\bigvee_{i=1}^d \Phi_i$ where each $\Phi_i$ is an affine assertion. Then for each transition $\tau=(\tsLoc, \tsLoc',\tsGuardcond)$, we have that $I(\tau,\tsLoc')=\bigvee_{i=1}^d I(\tau,\tsLoc',i)$ where each $I(\tau,\tsLoc',i)$ is a conjunctive affine invariant of the \LTS{} $\Gamma[\tsLoc',K_{\tau,i}:=\{\tsEval'\mid \exists \tsEval.(\tsEval\models \Phi_i \wedge \tsEval,\tsEval'\models \tsGuardcond)\}]$. 
Hence, our approach calculates $I(\tau,\tsLoc')$ by computing for each $1\le i\le d$ the conjunctive affine invariant $I(\tau,\tsLoc',i)$ (over  $\Gamma[\tsLoc',K_{\tau,i}]$) by the approaches~\cite{oopsla22/scalable}. 

\begin{example}
A preliminary example of invariant propagation for our running example has been given in Example~\ref{eg:propagation}, where we have the DFS tree and the breath-first propagation from the branch location $\tsLoc_2$ to $\tsLoc_1$. We give more details for the single propagation step from $\tsLoc_2$ to $\tsLoc_1$. 
For $\Phi:=\eta(\tsLoc_2)=(y=50\wedge 0\le x\le 49)$ and transition $\tau=(\tsLoc_2, \tsLoc_1,\tsGuardcond_5)$, our approach computes $K_{\tau}:=\{\tsEval'\mid \exists \tsEval.(\tsEval\models \Phi \wedge \tsEval,\tsEval'\models \tsGuardcond_5)=\{(x,y)\mid (x=50 \wedge y=50)\}$, and further derives the invariant for $\tsLoc_1$ from the ATS $\Gamma[\tsLoc_1,K_{\tau}]$ (that comprises only the location $\tsLoc_1$ and the self-loop transition $\tsGuardcond_1$ at $\tsLoc_1$). \qed
\end{example}

To instantiate a single propagation step, we need to encode the set $K_{\tau,i}$ as an affine assertion $\Phi'_{\tau,i}$ without quantifiers that defines the set, and this can be accomplished by the projection of the polyhedron $\{(\tsEval, \tsEval')\mid \tsEval\models \Phi_i \wedge \tsEval,\tsEval'\models \tsGuardcond\}$ onto the dimensions of $\tsEval'$. However, polyhedral projection is an operation with relatively high computation cost. Below we show that these $\Phi'_{\tau,i}$'s can be computed more efficiently by the resorting to the affine updates between $\mathbf{x}$ and $\mathbf{x}'$ from the original while loop. 

Consider the task to project the polyhedron $H=\{(\tsEval, \tsEval')\mid \tsEval\models \Phi \wedge \tsEval,\tsEval'\models \tsGuardcond\}$ in the treatment  of a transition $\tau=(\tsLoc, \tsLoc',\tsGuardcond)$ stated above, where $\Phi$ is an affine assertion. Recall that the transition is derived in the way that the relationship between the variables from $\tsVars$ and $\tsVars'$ is given by some affine assignment $\mathbf{x}:=\mathbf{A}\mathbf{x}+\mathbf{b}$ (i.e., $\mathbf{x}'=\mathbf{A}\mathbf{x}+\mathbf{b}$) under some conditional branch in the canonical form of Figure~\ref{fig:unnestedPQandRecursive}. We consider two cases below.
\begin{itemize}
\item The first case is that the matrix $\mathbf{A}$ is invertible. In this case, we have that $\mathbf{x}=\mathbf{A}^{-1}\mathbf{x}'-\mathbf{A}^{-1}\mathbf{b}$, and we obtain an affine assertion $\Phi'$ over $\tsVars'$ that defines the projected polyhedron directly as 
$(\Phi\wedge \tsGuardcond)[(\mathbf{A}^{-1}\mathbf{x}'-\mathbf{A}^{-1}\mathbf{b})/\mathbf{x}]$. In this case, no polyhedral projection is needed. 
\item The second case is that the matrix $A$ is not invertible. Then we solve the system of affine equations $\mathbf{A} \mathbf{x}=\mathbf{x}'-\mathbf{b}$ by the standard method of Gaussian Elimination in elementary affine algebra 
and obtains that 
$
\textstyle\mathbf{x}= \mathbf{u}(\mathbf{x}') + \sum_{i=1}^k a_k\cdot \mathbf{v}_i~(a_1,\dots, a_k\in \mathbb{R})
$
where (i) the vector $\mathbf{u}(\mathbf{x}')$ is a solution to the non-homogenous equation $\mathbf{A}\mathbf{x}=\mathbf{x}'-\mathbf{b}$ and can be expressed as an affine combination of the entries in $\mathbf{x}'$ (i.e., $\mathbf{u}(\mathbf{x}')=\mathbf{C} \mathbf{x}'+\mathbf{d}$ for some matrix $\mathbf{C}$ and vector $\mathbf{d}$) and (ii) $\mathbf{v}_1,\dots,\mathbf{v}_k$ are the basic solution of the homogeneous equation  $\mathbf{A}\mathbf{x}=\mathbf{0}$ and are constant vectors not relying on 
$\mathbf{x}'$. The fresh variables $a_1,\dots, a_k$ are the coefficients of the basic solution and can take any real value. 
As a consequence, the projection of the affine assertion $\tsEval\models \Phi \wedge \tsEval,\tsEval'\models \tsGuardcond$ (that defines the polyhedron $H$) onto the variables $\mathbf{x}'$ can be obtained  as the projection of the affine assertion 
$
\textstyle(\Phi\wedge \tsGuardcond)[(\mathbf{u}(\mathbf{x}') + \sum_{i=1}^k a_k\cdot \mathbf{v}_i)/\mathbf{x}]
$ 
onto the variables $\mathbf{x}'$ (i.e., projecting away the dimensions of $a_1,\dots, a_k$). Note that the number of the basic solution $a_1,\dots, a_k$ is equal to $n-\mathrm{rank}(A)$ where $\mathrm{rank}(A)$ is the rank of the matrix $A$. This means that the number of variables to be projected away is smaller than $n$.
It follows that in this case, it is possible to project away much less variables compared with the original projection method (that needs to project away all the $n$ variables $x_1,\dots,x_n$ in $\mathbf{x}$), and thus can further improve the time efficiency.
\end{itemize}

The advantage of incorporating invariant propagation lies at the observation that to generate the invariants at all the locations, previous approaches consider to solve them either as a whole~\cite{DBLP:conf/sas/SankaranarayananSM04} or separately~\cite{oopsla22/scalable} via the generator computation of polyhedral cones. Thus, all these approaches require to solve the invariants at all the locations with generator computation, an operation with relative high cost and possible exponential blow-up. Invariant propagation improves the time efficiency in that when the underlying \LTS{} has a non-crossing DFS tree, then it suffices to perform generator computation only in the computation of the invariants at the initial location and in the treatment of self-loops at other locations.  


Note that non-crossing affine transition systems do not cover all cases of directed acyclic graphs, but this can be partially remedied by first computing the strongly-connected components (SCCs) of the underlying \LTS{} and then considering each SCC separately. 

In summary, the workflow of our algorithm 
over an unnested affine while loop is as follows.

\begin{itemize}
\item First, our algorithm transforms an unnested affine while loop into the canonical form in Figure~\ref{fig:unnestedPQandRecursive} and further transforms it into an affine transition system. 
\item Second, our algorithm applies the approach by~\citet{oopsla22/scalable} and our invariant propagation technique (if possible) to obtain affine invariants at the branch locations of the affine transition system. In the case that the affine transition system is non-crossing w.r.t the initial location, our algorithm applies the approach by~\citet{oopsla22/scalable} to obtain the affine invariant at the initial location and afterwards derive the invariants at other locations through invariant propagation. Otherwise (i.e., the affine transition system is not non-crossing), our algorithm follows the original approach by~\citet{oopsla22/scalable} to generate the invariants at all the locations. 
\end{itemize}


By an induction on the depth of the DFS tree, we can prove that the assertions generated from our invariant propagation are indeed invariants and are at least as tight as the invariants generated by the previous approaches \cite{DBLP:conf/sas/SankaranarayananSM04,oopsla22/scalable}. Due to space limitation, we relegate the detailed proofs to Appendix~\ref{sec:appendix_invpropagation_proof}. 


%\section{Disjunctive Affine Invariant Generation for Nested Loops}

Recall that in the previous section, we proposed a novel approach for generating disjunctive affine invariants over unnested while loops via Farkas' Lemma, top-level branches and an invariant propagation technique. 
In this section, we extend this approach to nested affine while loops. 
 

The main idea is as follows. Given a nested affine while loop $W$, our approach works by first recursively computing the loop summary $\ProcSmry_{W'}$ for each inner while loop $W'$ in $W$ (from the innermost to the outermost), 
and then 
tackling the main loop body via the top-level branches and the loop summaries $\ProcSmry_{W'}$ of the inner loops. Below we fix a nested affine while loop $W$ with variable set $\tsVars=\{\tsVar_1,\dots,\tsVar_n\}$ and present the technical details. 

The most involved part in our approach is the transformation of the main loop $W$ into its corresponding \LTS{} by the top-level branches.   
Unlike the situation of unnested while loops, a direct recursive algorithm that transforms the loop $W$ into a canonical form in Figure~\ref{fig:unnestedPQandRecursive} as in the unnested case is not possible, since one needs to tackle the loop summaries from the inner while loops in $W$.

To address the problem above, our algorithm works with the \emph{control flow graph} (CFG) $H$ of the loop body of the loop $W$ and considers the \emph{execution paths} in this CFG. The CFG $H$ is a directed graph whose vertices are the program counters of the loop body and whose edges describe the one-step jumps between these program counters. Except for the standard semantics of the jumps emitting from assignment statements and conditional branches, for a program counter that represents the entry point of an inner while loop that is not nested in other inner loops, we have the special treatment that the jump at the program counter is directed to the termination program counter of this inner loop in the loop body of $W$ (i.e., skipping the execution of this inner loop). An \emph{execution path} in the CFG $H$ is a directed path of program counters that ends in (i) either the termination program counter of the loop body of $W$ without visiting a  program counter that represents the \textbf{break} statement or (ii) a first \textbf{break} statement without visiting prior \textbf{break} statements. 
An example is as follows.

\begin{example}\label{eg:janne_cfg}
Consider the janne\_complex program from ~\citet{DBLP:conf/vmcai/BoutonnetH19} in Figure~\ref{fig:jannecomplex}.  
The CFG of the program is given in Figure~\ref{fig:janne_cfg} where the nodes correspond to the program counters, the directed edges with guards specifies the jumps and their conditions, and the affine assignments are given in the program counters $A_1,A_2,A_3$. 

\newsavebox{\jannecomplex}
\begin{lrbox}{\jannecomplex}
\begin{lstlisting}[mathescape]
while($x<30$){
    while($y<x$){
        if ($y>5$) $y=y*3$;
        else $y=y+2$;
        if (y>=10 && y<=12) $x=x+10$;
        else $x=x+1$;
    }
    $x=x+2$; $y=y-10$;
}
\end{lstlisting}
\end{lrbox}

% Figure environment removed

% Figure environment removed


We denote by $W$ the outer loop with entry point $E_{\mathrm{Outer}}$, and by $W'$ the inner loop with entry point $E_{\mathrm{inner}}$.    
The execution path starts at the \emph{Initial Condition} $\left[x, y \right]$, jumps to the next vertices along the edge whose condition is satisfied (e.g., \emph{True} is tautology, \emph{$x < 30$} is satisfied when variable $x$ value is less than $30$, etc.), and terminates in the \emph{Exit} statement. 
The only execution path for the loop body of $W$ is  
$A_{IS} \rightarrow A_{1}$,
for which we abstract the whole inner loop by $A_{IS}$. 
\qed 
\end{example}


Based on the CFG $H$ and the execution paths, our approach constructs the \LTS{} for the outer loop $W$ as follows. Since the output of an inner while loop $W'$ in $W$ cannot be exactly determined from the input to the loop $W'$, we first have fresh output variables 
$\overline{x}_{W',1},\dots,\overline{x}_{W',n'}$ to represent the output values of the variables $\overline{x}_{W',1},\dots,\overline{x}_{W',n'}$ after the execution of the inner loop $W'$. These output variables are used to express the loop summaries of these inner loops. 

Then, to get the numerical information from execution paths,  we symbolically compute the values of the program variables at each program counter in an execution path. In detail, given an execution path $\omega=\iota_1,\dots,\iota_k$ where each $\iota_i$ is a program counter of the loop body of the loop $W$, our approach computes the affine expressions $\alpha_{\tsVar,i}$ and PAPs $\beta_i$ (for $\tsVar\in \tsVars$ and $1\le i\le k$) over the program variables in $\tsVars$ (for which they represent their initial values at the start of the loop body of $W$ here) and the fresh output variables. 
The intuition is that (i) each affine expression $\alpha_{\tsVar,i}$ represents the value of the variable $\tsVar$ at the program counter $\iota_i$ along the execution path $\omega$ and (ii) each PAP $\beta_i$ specifies the condition that the program counter $\iota_i$ is reached along the execution path $\omega$. The computation is recursive on $i$ as follows.

Denote the vectors $\alpha_i:=(\alpha_{\tsVar_1,i},\dots, \alpha_{\tsVar_n,i})$ and  $\overline{x}_{W'}=(\overline{x}_{W',1},\dots,\overline{x}_{W',n'})$. For the base case when $i=1$, we have $\alpha_1=(\tsVar_1,\dots,\tsVar_n)$ and $\beta_1=\mathbf{true}$ that specifies the initial setting at the start program counter $\iota_1$ of the loop body of the original loop $W$. For the recursive case, suppose that our approach has computed the affine expressions in $\alpha_{i}$ and the PAP $\beta_i$. We classify four cases below:

\begin{itemize}
\item \emph{Case 1:} The program counter $\iota_{i}$ is an affine assignment statement $\mathbf{x}:=\mathbf{F}(\mathbf{x})$. Then we have that $\alpha_{i+1}= \alpha_i[\mathbf{F}(\mathbf{x})/\mathbf{x}]$ and $\beta_{i+1}:=\beta_i$.  
\item \emph{Case 2:} The program counter $\iota_{i}$ is a conditional branch with branch condition $b$ and the next program counter $\iota_{i+1}$ follows its \textbf{then}-branch. Then the vector $\alpha_{i+1}$ is the same as $\alpha_i$, and the PAP $\beta_{i+1}$ is obtained as $\beta_{i+1}=\beta_i\wedge b$.   
\item \emph{Case 3:} The program counter $\iota_{i}$ is a conditional branch with branch condition $b$ and the next program counter $\iota_{i+1}$ follows its \textbf{else}-branch. The only difference between this case and the previous case is that $\beta_{i+1}$ is obtained as $\beta_{i+1}:=\beta_i\wedge \neg b$.   
\item \emph{Case 4:} The program counter $\iota_{i}$ is the entry point of an inner while loop $W'$ of $W$ and $\iota_{i+1}$ is the successor program counter outside $W'$ in the loop body of $W$. Then $\alpha_{i+1}:=\overline{x}_{W'}$ and $\beta_{i+1}:=\ProcSmry_{W'}(\alpha_i, \overline{x}_{W'})$. Here we use the ouput variables to express the loop summary. Note that the loop summary $\ProcSmry_{W'}$ (see Page~\pageref{pg:loopsmry} for the definition of $\ProcSmry$) is recursively computed. 
\end{itemize}  

\begin{example}\label{eg:evolution}
Continue with the execution path in Example~\ref{eg:janne_cfg}. 
% Figure environment removed
The evolution of $\alpha_{i}$ and $\beta_{i}$ with the initial setting $\alpha_{1}=[x,y],\beta_{1}=\mathbf{true}$ is given in Figure~\ref{fig:outer_inner}.   
\qed 
\end{example}




After the $\alpha_i,\beta_i$'s are obtained for an execution path $\omega=\iota_1,\dots,\iota_k$ from the recursive computation above, we let the PAP $\Psi_\omega:=\bigwedge_{i\in I} \beta_i$ where the index set $I$ is the set of all $1\le i\le k$ such that the program counter $\iota_i$ corresponds to either a conditional branch or the entry point of an inner while loop, and the vector of affine expression $\alpha_{\omega}:=\alpha_{k+1}$. Note that the PAP $\Psi_\omega$ is the condition that the execution of the loop body follows the execution path $\omega$, and the affine expressions in the vector $\alpha_\omega$ represent the values of the program variables after the execution path $\omega$ of the loop body of $W$ in terms of the initial values of the program variables and the fresh variables for the output of the inner while loops in $W$. 


Finally, our approach constructs the \LTS{} for the loop $W$ and we only present the main points: 

\begin{itemize}
\item First, for each execution path $\omega$ of the loop body of $W$, we have a standalone location $\tsLoc_\omega$ for  this execution path. Recall that we abstract the inner loops, so that the execution paths can be finitely enumerated. 
\item Second, for all locations $\tsLoc_\omega,\tsLoc_{\omega'}$ (from the execution paths $\omega,\omega'$), we have the transition $\tau_{\omega,\omega'}:=(\tsLoc_{\omega}, \tsLoc_{\omega'}, \Psi_{\omega} \wedge \Psi'_{\omega'}\wedge \mathbf{x}'=\alpha_\omega)$ which means that if the execution path in the current iteration of the loop $W$ is $\omega$, then in the next iteration the execution path can be $\omega'$ with the guard condition $\Psi_{\omega} \wedge \Psi'_{\omega'}\wedge \mathbf{x}'=\alpha_\omega$ that comprises the conditions for the execution paths $\omega,\omega'$ and the condition $\mathbf{x}'=\alpha_\omega$ for the next values of the program variables.
\item Third, we enumerate all possible initial locations $l_\omega$, along with their corresponding initial conditions $\tsInitcond= G \wedge \Psi_{\omega}$. To derive loop summary, we follow the standard technique (see e.g.~\citet{DBLP:conf/vmcai/BoutonnetH19}) to include the input variables $\tsVars_{\mathsf{in}}$ and conjunct the affine assertion $\bigwedge_{\tsVar\in\tsVars} \tsVar=\tsVar_{\mathsf{in}}$ into each disjunctive clause of the initial condition $\tsInitcond$. 
Manually specified initial conditions can also be conjuncted into 
$\tsInitcond$. 
\end{itemize}

A detailed process that handles $\mathbf{break}$ statement is similar to the unnested situation. Again, we can remove invalid transitions by checking whether their guard condition is satisfiable or not. 


Finally, we apply the approach \cite{oopsla22/scalable} and our invariant propagation to the \LTS{} constructed above to obtain the loop summary as an invariant (over the variables in $\tsVars_{\mathsf{in}}\cup\tsVars$) generated at the termination location $\tsLoc_{e}$, and rename each variable $\tsVar\in\tsVars$ to its output 
$\tsVar_{\mathsf{out}}$. 

\begin{example}\label{eg:nested_execution}
Consider the janne\_complex program in Figure~\ref{fig:jannecomplex}. By integrating the loop summary of the inner loop, our approach constructs an affine transition system that corresponds to a while loop of $22$ top-level branches.  
For the lack of space, we relegate the detailed branches to Figure~\ref{fig:jannecomplexsmryfull} in Appendix~\ref{sec:appendix_innersmry}. \qed
\end{example}


At the end of the illustration of our algorithms, we discuss possible extensions as follows. 

\begin{remark}[Extensions]\label{rmk:extension}
Our approach can be extended in the following ways. To obtain a more precise top-level branch representation, one extension is by (i) distinguishing even/odd integer values of program variables and (ii) detecting hidden termination phases via the approach in \cite{DBLP:conf/cav/Ben-AmramG17}. To handle machine integers, another extension is by having a piecewise disjunctive treatment for the cases of overflow and non-overflow. 
Finally, our approach could be extended to floating point numbers by considering piecewise affine approximations~\cite{DBLP:conf/esop/Mine04, DBLP:conf/vmcai/Mine06}. 
\end{remark}

\section{Evaluation} \label{sec:evaluation}

\begin{table*}[tbp]
\centering
\small
\begin{tabular}{cccccccccc}
\toprule
& \multicolumn{3}{c}{\msr} & \multicolumn{3}{c}{\negc} & \multicolumn{3}{c}{\wsj} \\
& Acc. & F1 & wF1 & Acc. & F1 & wF1 & Acc. & F1 & wF1 \\ \cmidrule(lr){2-4} \cmidrule(lr){5-7} \cmidrule(lr){8-10} 
\udel & 66.86 & 56.76 & 64.3 & \textbf{80.80} & 55.45 & 77.9 & 63.74 & 64.23 & 63.2 \\
\icsi & \underline{71.19} & 64.73 & 70.4 & 80.36 & 64.53 & \underline{78.6} & 64.62 & 64.15 & 63.4 \\
\cnts & 68.59 & 61.39 & 67.2 & 78.68 & 61.62 & 76.8 & 64.31 & 64.59 & 64.4 \\
\osu & 68.02 & 60.28 & 66.6 & 79.24 & 57.04 & 76.5 & 69.20 & 69.63 & 68.9 \\
\isg & 67.05 & 58.83 & 65.3 & 77.34 & 59.52 & 75.6 & 69.15 & 69.35 & 69.2 \\ \midrule
\bert & \textbf{71.68} & \underline{66.70} & \textbf{71.4} & 77.79 & \underline{72.87} & 77.7 & \underline{80.95} & \underline{80.93} & \underline{80.9} \\
\roberta & 70.91 & \textbf{67.53} & \underline{70.7} & \textbf{80.80} & \textbf{77.29} & \textbf{80.7} & \textbf{82.61} & \textbf{82.70} & \textbf{82.6} \\ \midrule
Average & 69.19 & 62.32 & 67.99 & 79.29 & 64.05 & 77.69 & 70.65 & 70.80 & 70.37 \\
\bottomrule
\end{tabular}
\caption{\label{tab:performance} Overall accuracy (Acc.), macro-averaged F1 (F1), and weighted-macro F1 (wF1) scores of the algorithms depicted in Section~\ref{sec:algorithm}. For instance, \msr-\udel refers to a C5.0 classifier trained on the \msr~corpus, using the feature set mentioned in \citet{greenbacker-mccoy-2009-udel}.}
%Its Acc., F1 and wF1 of this model are 66.86, 56.76, and 64.3, respectively.}
\end{table*}


In this section, we introduce the evaluation protocol and report the performance of the models.

\subsection{Implementation Details} \label{sec:implementation}

For \bert and \roberta, we used \textit{bert-base-cased} and \textit{roberta-base}, both from Hugging Face. For fine-tuning, we set the batch size to 16, the learning rate to 1e-3, the dropout rate to 0.5, and the size of the output layer to 256. We ran each model for 20 epochs and used the one that achieved the highest F1 score on the development set. The implementation details of the classic ML-based models can be found in Appendix~\ref{sec:appendixML}.

\subsection{Evaluation Protocol} \label{sec:protocol}

The main evaluation metric in the GREC-MSR shared tasks was accuracy. 
In addition to accuracy, we also report macro-F1 and weighted-macro F1. We argue that different metrics evaluate algorithms from different perspectives and provide us with different meaningful insights. 
For pragmatic tasks like REG, it makes sense to ask how well an algorithm performs on naturally distributed data which is often imbalanced. For these cases, reporting accuracy and weighted F1 are logical. 
Furthermore, analogous to other classification tasks, minority categories should not be overlooked. Take as an example the class \emph{description} in the \negc corpus, which occurs only 4\%. If a model fails to produce this class, the produced document might sound unnatural. Therefore, it is important to ensure that an algorithm is not over- or under-generating certain classes. Looking into accuracy and macro-F1 together provides insights into such cases.

\subsection{Performance of the Models}\label{subsec:overallacc}

The overall accuracy of the models, their macro F1, and their weighted-macro F1 are presented in Table \ref{tab:performance}. 
We also present the ranking of the models based on these scores in Appendix~\ref{sec:app_rank}. 


\paragraph{PLM-based Models.} The best-performing models across all corpora and metrics are PLM-based models.  In six out of nine rankings, \bert and \roberta are ranked as the top two models. The sole exception is \negc, where \bert is the second worst model. The benefit of using PLMs is the largest on the \wsj corpus. For example, \roberta improves the macro F1 score from 69.63 (i.e., the performance of the best ML-based model) to 82.70.


\paragraph{ML-based Models.} In contrast to the robust performance of the PLM models, the performance of the classic ML models is more corpus-dependent. In the case of \msr and \negc, \icsi is the best-performing model, while in the case of \wsj, it is at the bottom section of the rankings. Another interesting observation is the performance of the \udel models. In terms of accuracy, \udel has the highest performance in \negc, while it has the lowest performance in both \msr and \wsj. In terms of macro-F1 rankings, the \negc \udel model dropped from first to last place, whereas \bert improved from penultimate place to second place. In general, our ML models yielded lower scores than the original models used in the GREC study \citep{belz2009generating}. This could be attributed to a variety of factors, including differences in feature engineering and model parameters.

\paragraph{Comparing Different Metrics.} 

Upon comparing average scores across the three metrics, we observe that for \msr and \negc, PLMs are clear winners only when macro-F1 is the metric in question. However, for \wsj, PLMs are winners on all three metrics. This may be because the distribution of categories in \wsj is much more balanced than in the other two corpora.
\section{Related Work}
\label{appsec: related work}
Bayesian causal discovery literature has primarily focused on inference in linear models with closed-form posteriors or marginalized parameters. Early works considered sampling directed acyclic graphs (DAGs) for discrete~\cite{cooper1992bayesian, madigan1995bayesian, heckerman2006bayesian} and Gaussian random variables~\cite{friedman2003being, tong2001active} using Markov chain Monte Carlo (MCMC) in the DAG space. However, these approaches exhibit slow mixing and convergence~\cite{eaton2012bayesian,grzegorczyk2008improving}, often requiring restrictions on number of parents~\cite{kuipers2017partition}. %Alternative exact dynamic programming methods are limited to small settings~\cite{koivisto2012advances}. 

Recent advances in variational inference~\cite{zhang2018advances} have facilitated graph inference in DAG space, with gradient-based methods employing the NOTEARS DAG penalty \cite{zheng2018dags}.\cite{annadani2021variational} samples DAGs from autoregressive adjacency matrix distributions, while \cite{lorch2021dibs} utilizes Stein variational approach \cite{liu2016stein} for DAGs and causal model parameters. \cite{cundy2021bcd} proposed a variational inference framework on node orderings using the gumbel-sinkhorn gradient estimator \cite{mena2018learning}. \cite{deleu2022bayesian,nishikawa2022bayesian} employ the GFlowNet framework \cite{bengio2021gflownet} for inferring the DAG posterior. Most methods, except\cite{lorch2021dibs} are restricted to linear models, while \cite{lorch2021dibs} has high computational costs and lacks DAG generation guarantees compared to our method.
% at least quadratic scaling complexity, both with respect to the number of nodes (due to the DAG penalty) as well as number of posterior samples. Our proposed approach instead has linear complexity with respect to number of posterior samples and does not require any additional DAG penalty.     

In contrast, \emph{quasi-Bayesian} methods, such as DAG bootstrap \cite{friedman2013data}, demonstrate competitive performance. DAG bootstrap resamples data and estimates a single DAG using PC \cite{spirtes2000causation}, GES \cite{chickering2002optimal}, or similar algorithms, weighting the obtained DAGs by their unnormalized posterior probabilities. Recent neural network-based works employ variational inference to learn DAG distributions and point estimates for nonlinear model parameters \cite{charpentier2022differentiable,geffner2022deep}.
\section{Conclusion and Future Work}
In this work, I design corruption-robust algorithms for the Lipschitz contextual search problem. I present the \emph{agnostic checking} technique and demonstrate its effectiveness in designing corruption-robust algorithms. There are several open problems for future research. First, in the algorithm I propose for pricing loss, the schedule for agnostic checks is fixed upfront. Can the learner design an adaptive checking schedule for the pricing loss? Second, this work assumes the learner has knowledge of the Lipschitz constant $L$. Can the learner design efficient no-regret algorithms without knowledge of $L$? 


%% Acknowledgments
%% acks environment is optional contents suppressed with 'anonymous'
%% Commands \grantsponsor{<sponsorID>}{<name>}{<url>} and
%% \grantnum[<url>]{<sponsorID>}{<number>} should be used to
%% acknowledge financial support and will be used by metadata
%% extraction tools.
\begin{acks}                            
  This material is based upon work supported by the \grantsponsor{GS100000001}{National Science Foundation}{http://dx.doi.org/10.13039/100000001} under Grant No.~\grantnum{GS100000001}{nnnnnnn} and Grant No.~\grantnum{GS100000001}{mmmmmmm}.  Any opinions, findings, and conclusions or recommendations expressed in this material are those of the author and do not necessarily reflect the views of the National Science Foundation.
\end{acks}
\clearpage
%% Bibliography
\bibliographystyle{ACM-Reference-Format}
\bibliography{invariants}
\clearpage
%% Appendix
\begin{comment}
\section{System Architecture}
\label{appendix:architecture}
\system has a novel modularized system architecture with three key components: 
\emph{StreamManager}, 
\emph{TxnManager} and \emph{TxnScheduler}. 
These components are instantiated in each thread locally.
The execution outline of \system is presented in Algorithm~\ref{alg:algo}.
Transactional stream processing is continuous and potentially never ends (Line 1$\sim$8).
The dependency resolution and execution of state transactions are separated into two non-overlapping phases by punctuations~\cite{Tucker:2003:EPS:776752.776780} (Line 2 and 5), which guarantees that no subsequent input event will have a smaller timestamp. 
Effectively, a batch of state transactions is collected during the first phase, and processed during the second phase.

In the first phase (i.e., stream processing phase), 
the \emph{StreamManager} conducts preprocessing for every input event ($e$). Similar to some prior works~\cite{tstream}, state transactions may be issued but not immediately processed during preprocessing (Line 3).
The \emph{pre\_processing} and \emph{post\_processing} functions are exposed as APIs to users.
The \emph{TxnManager} handles dependency resolution (Line 4) among state transactions and insert decomposed operations to construct a \tpg. We discuss the detailed two-phase \tpg construction process in Section~\ref{subsec:construction}.

In the second phase  (i.e., transaction processing phase), 
the \emph{TxnManager} is first involved again to refine (Line 6) the constructed \tpg with further dependency resolution.
The \emph{TxnScheduler} 
schedules operations for concurrent execution based on the constructed \tpg according to the three dimensions of scheduling decisions (Line 7). 
In particular, a scheduling decision model $M$ is instantiated based on the constructed \tpg (Line 14).
\textbf{\circled{1}} Guided by $M$, execution threads adopt an exploration strategy (Section~\ref{subsec:explore}) to explore the constructed \tpg for operations available to be scheduled constrained by dependencies. 
\textbf{\circled{2}} 
During exploration, one or multiple operations may be treated as the 
% basic 
unit of scheduling (Section~\ref{subsec:granularity}). 
Subsequently, \textbf{\circled{3}} every thread executes operation(s) in the unit of scheduling with various abort handling mechanisms (Section~\ref{subsec:abort_handling}).
Only when state transactions are processed (i.e., committed or aborted) can the associated input events be postprocessed (Line 8) by the \emph{StreamManager} based on transaction processing results.
\end{comment}

\begin{comment}
\begin{algorithm}
\footnotesize
    \KwData{$e$ \tcp{Input event}}
    \KwData{$txn_{ts}$ \tcp{State transaction}}
    \KwData{$G$ \tcp{The currently constructed TPG}}
    \While{!finish processing of input streams}{
        \eIf(\tcp*[h]{Phase 1}){\text{$e$ is not a $punctuation$}}{
                $txn_{ts}$ $\gets$ PRE\_Processing($e$)\;
                \textbf{TPG\_Construction}($G$, $txn_{ts}$)\; 
          }(\tcp*[h]{Phase 2}){
                \textbf{TPG\_Refinement}($G$)\; 
                \textbf{TXN\_Scheduling}($G$)\; 
                POST\_Processing()\;
          }
    }
    
    \SetKwFunction{FMain}{TPG\_Construction}
    \SetKwProg{Fn}{Function}{:}{}
    \Fn{\FMain{$G$, $txn_{ts}$}}{
        $O_{1..k}$ $\gets$ \textbf{Partition} $txn_{ts}$\;
        \ForEach{\text{operation $O_{i}$ $\in$ $O_{1..k}$}}{
            \textbf{Identify} its \ld\;
            $G$ $\gets$ $G$ + $O_{i}$ \;
        }
    }
    \SetKwFunction{FMain}{TPG\_Refinement}
    \SetKwProg{Fn}{Function}{:}{}
    \Fn{\FMain{$G$}}{
        \ForEach{\text{vertex $e_{i}$ $\in$ $G$}}{
            \textbf{Identify} its \td, \pd\;
        }
    }
    
    \SetKwFunction{FMain}{TXN\_Scheduling}
    \SetKwProg{Fn}{Function}{:}{}
    \Fn{\FMain{$G$}}{
        $M$ $\gets$ Instantiated with $G$;\tcp{A decision model}
        \While{!finish scheduling of $G$
        }{
          \textbf{\circled{2}} $Scheduling Unit$ $\gets$ \textbf{\circled{1}} \emph{Explore}($G$, $M$)\; 
            \textbf{\circled{3}} \emph{Execute with Abort Handling} ($Scheduling Unit$)\; 
        }
    }
  \caption{Execution Outline of \system}
  \label{alg:algo}
\end{algorithm}
\end{comment}
%\section{Proof for Correctness and Accuracy for our Invariant Propagation}
\label{app:ProofofCorrectnessInvPropagation}

Below we prove the theoretical properties that the affine assertions generated from our invariant propagation are indeed invariants, and are at least as tight as the invariants generated from the previous approaches~\cite{DBLP:conf/sas/SankaranarayananSM04,oopsla22/scalable}.  

\begin{proposition} 
The affine assertions generated by the invariant propagation are invariants. 
\end{proposition}
\begin{proof}
Let $\Gamma$ be an \LTS{} whose directed graph $\mbox{\sl DG}(\Gamma)$ has a non-crossing DFS tree $T$. The proof is by induction on the BFS level of the tree $T$. The base step is that the affine assertion at the root (i.e., the initial location) is correct since it is generated by the approach \cite{oopsla22/scalable}. The inductive step is to show that if the affine assertions generated at the nodes of the current level are invariants, then so are the affine assertions at the next level. The proof for the inductive step follows from the fact that any path of the \LTS{} $\Gamma$ that visits a location $\tsLoc'$ in the next BFS level should first visit some location $\tsLoc$ (with the valuation $\tsEval$ guaranteed to satisfy the invariant $\eta(\tsLoc)$) in the current BFS level, and then possibly repeatedly stays at the location $\tsLoc'$. (Note that here we use the fact that there is no crossing edge in the DFS tree $T$. 
This fact is captured by the initial condition $K_{\tau,i}$ for a transition $(\tsLoc, \tsLoc', \tsGuardcond)$ (that is obtained from the $i$th disjunctive clause $\Phi_i$ of the invariant $\eta(\tsLoc)$) and the invariant $I(\tau,\tsLoc',i)$ for the self-loop \LTS{} $\Gamma[\tsLoc', K_{\tau,i}]$ in a single propagation step.  
\end{proof}

\begin{proposition}\label{prp:propagation}
The invariant propagation generates invariants at least as tight as the previous approaches \cite{DBLP:conf/sas/SankaranarayananSM04,oopsla22/scalable}. 
\end{proposition}
\begin{proof}
The proof proceeds via an induction on the BFS level of the invariant propagation. For the base step, we have that the affine invariant generated at the root is generated directly from the previous approach~\cite{oopsla22/scalable}. Then the base step follows from the fact that the approach~\cite{oopsla22/scalable} has the same precision as the original approach \cite{DBLP:conf/sas/SankaranarayananSM04}. For the inductive step, suppose the induction hypothesis that the invariant of every node at the current BFS level in the DFS tree implies the counterpart generated by the approach~\cite{DBLP:conf/sas/SankaranarayananSM04}. We prove that the implication holds for the next BFS level. The proof can be obtained by observing that each individual affine inequality (as a conjunctive inequality in an affine assertion) in the invariants generated by the approach~\cite{DBLP:conf/sas/SankaranarayananSM04} on a location $\tsLoc'$ at the next BFS level satisfies the consecution condition derived from any transition $\tau=(\tsLoc,\tsLoc',\tsGuardcond)$ to the location $\tsLoc'$, so that each such inequality is implied by the initial condition $K_{\tau,i}$ and satisfies the possible consecution condition from the self-loop in $\Gamma[\tsLoc', K_{\tau,i}]$. Since we apply the same approach ~\cite{DBLP:conf/sas/SankaranarayananSM04} (i.e., solving the same constraints for the unknown coefficients from the consecution condition of the self-loop), the invariant $\eta(\tsLoc')$ generated by our invariant propagation implies any individual affine inequality generated by the approach~\cite{DBLP:conf/sas/SankaranarayananSM04}.  
\end{proof}

\clearpage
%\section{Full Experimental Results on Invariant Generation and Loop Summary Compared with Original Results}
\label{sec:appendix_invariants}

\begin{table}[htbp]
    \centering
    \caption{Full Experimental Results on Invariant Generation for Table~\ref{tab:related}}
    \label{tab:appendix_related}
    \resizebox{\linewidth}{!}{%
    \begin{tabular}{|c|c|c|c|c|c|c|} 
    \hline
    \multicolumn{2}{|c|}{Benchmark} & \multicolumn{5}{c|}{Comparison} \\ 
    \hline
    \multicolumn{2}{|c|}{Name} & Type & Time & v.s. & Original Result & Our Result\\ 
    \hline
    \multirow{1}{*}{\citet{FSE2022}} & \multirow{1}{*}{fig2$\ \star$} & Dis & 0.02s & {$>$} & \begin{tabular}[c]{@{}c@{}}(y$>$0 y$>$x/1000 $\implies$ z=0) $\wedge$ \\(y$>$0 y$=$x/1000 $\implies$ z=x-1000y) $\wedge$ \\(y$>$0 y$<$x/1000 $\implies$ z=1000)\end{tabular} & \begin{tabular}[c]{@{}c@{}}(z=0 $\wedge$ 0 $\le$ x $\le$ 1000y-1 $\wedge$ 1 $\le$ y) $\vee$ \\(x-1000y=z $\wedge$ x-999 $\le$ 1000y $\le$ x $\wedge$ 1 $\le$ y) $\vee$ \\(z=1000 $\wedge$ 1 $\le$ y $\wedge$ 1000y $\le$ x-1000) \end{tabular} \\ 
    \hline
    \multirow{3}{*}{\citet{DBLP:journals/entcs/AncourtCI10}} & \multirow{1}{*}{Gopan07$\ \star$}
     & LR & $<$0.01s & {$>$} & x=102 & {x=102 $\wedge$ y=-1} \\ 
    \cline{2-7}
     & \multirow{1}{*}{Gulwani07$\ \star$}
     & LR & 0.01s & {$>$} & y=100 & {x=y=100} \\ 
    \cline{2-7}
     & \multirow{1}{*}{Halbwachs$\ \star$}& LR & 0.01s & {$>$} & 2$\le$x+y $\wedge$ y$\le$x $\wedge$ x+y$\le$202 & \begin{tabular}[c]{@{}c@{}}{(101~$\le$~x~$\le$~102~$\wedge$~0~$\le$~y~$\wedge$~y+2~$\le$~x)~$\vee$~(x=101~$\wedge$~1~$\le$~y~$\le$~101)}\end{tabular} \\ 
    \hline
    \multirow{2}{*}{\citet{DBLP:conf/cav/SharmaDDA11}} & \multirow{2}{*}{POPL07$\ \star$} & Dis & $<$0.01s & {$>$} & (y=50~$\wedge$ x $\le$ 50)~$\vee$ (x=y~$\wedge$ 50 $\le$ x $\le$ 100) & \begin{tabular}[c]{@{}c@{}}{(y=50~$\wedge$~0 $\le$ x $\le$ 49)~$\vee$ (x=y~$\wedge$ 50 $\le$ x $\le$ 99)}\end{tabular} \\ 
    \cline{3-7}
     &  & LR & $<$0.01s & {$>$} & y=100 & {x=y=100} \\ 
    \hline
    \end{tabular}
    }
\end{table}

\begin{table}[htbp]
    \centering
    \caption{Full Experimental Results on Loop Summary for Table~\ref{tab:relatedsmry}}
    \label{tab:appendix_relatedsmry}
    \resizebox{\linewidth}{!}{%
    \begin{tabular}{|c|c|c|c|c|c|c|} 
    \hline
    \multicolumn{2}{|c|}{Benchmark} & \multicolumn{5}{c|}{Comparison} \\ 
    \hline
    \multicolumn{2}{|c|}{Name} & Type & Time & v.s. & Original Result & Our Result \\ 
    \hline
    \multirow{6}{*}{\citet{DBLP:conf/sigsoft/XieCLLL16}} & fig1a $\star$ $\ddagger$ & \multirow{5}{*}{Smry} & 0.01s & {$>$} & \begin{tabular}[c]{@{}c@{}}(x$_0$ $\ge$ n$_0$ $\wedge$ x=x$_0$ $\wedge$ z=z$_0$) $\vee$ \\(x$_0$ $<$n$_0$ $\le$ z$_0$ $\wedge$ x=n$_0$ $\wedge$ z=z$_0$) $\vee$ \\(x$_0$ $<$ n$_0$ $\wedge$ z$_0$ $<$ n$_0$ $\wedge$ x=z=n$_0$)\end{tabular} & \begin{tabular}[c]{@{}c@{}}(x=z=n=n$_0$~$\wedge$~x$_0$~$\le$ z$_0$-1 $\wedge$ z$_0$ $\le$ n-1) $\vee$~\\(x$_0$+1 $\le$ x=n=n$_0$ $\le$ z$_0$=z) $\vee$~\\(x=z=n=n$_0$ $\wedge$ z$_0$ $\le$ x$_0$ $\le$ n-1)\end{tabular} \\ 
    \cline{2-2}\cline{4-7}
     & fig1c $\star$ &  & $<$~0.01s & {$>$} & i $<$ m & {1 $\le$ j $\le$ m-1 $\wedge$ m~$ \le$ i $\le$ 2m-2~$\wedge$ 1 $\le$ k $\le$~m-i$_0$} \\ 
    \cline{2-2}\cline{4-7}
     & fig1f $\star$ &  & 0.02s & {$>$} & x$_{10}$ $\ge$ 0 $\wedge$ x$_{20}$ $\ge$ 0 & \begin{tabular}[c]{@{}c@{}}(s=1~$\wedge$~x$_1$-x$_2$ = x$_{10}$-x$_{20}$ $\wedge$ x$_{10}$ $\le$~x$_1$)~$\vee$~\\(s=2~$\wedge$~x$_1$-x$_2$-1 = x$_{10}$-x$_{20}$ $\wedge$~1~$\le$~x$_1$-x$_{10}$)~$\vee$~\\(s=3~$\wedge$~x$_1$-x$_2$ = x$_{10}$-x$_{20}$ $\wedge$~1~$\le$~x$_1$-x$_{10}$)~$\vee$~\\(s=4~$\wedge$~x$_1$-x$_2$ = x$_{10}$-x$_{20}$ $\wedge$~1~$\le$~x$_1$-x$_{10}$)\end{tabular} \\ 
    \hline
    \end{tabular}
    }
\end{table}
\clearpage
\section{Full Inner Caess for Figure~\ref{fig:jannecomplex}}
\label{sec:appendix_innersmry}

%\lstset{language=program}
%\lstset{tabsize=3}
\newsavebox{\jannecomplexsmryfull}
\begin{lrbox}{\jannecomplexsmryfull}
\begin{lstlisting}[mathescape]
while($x<30$){
 switch
  case $$: 
   $$;Skip;
  case $x<=29,6<=y,y<=x-1$: 
   $x'<=29,y'>=6,y'-x'<=-1,36x'-y'<=36x-3y+18,3x'-y'>=22,x'>=x+3,x'-y'<=12$;Skip;
  case $x<=29,6<=y,y<=x-1$: 
   $x'<=29,y'<=x'-1,y'<=5,36x-3y+18>=36x'-y',3x'-y'>=22,x'-x>=3,x'<=y'+12$;Skip;
  case $x<=29,6<=y,y<=x-1$: 
   $x'<=29,y'<=x'-1,y'<=5,36x-3y+18>=36x'-y',3x'-y'>=22,x'-x>=3,x'<=y'+12$;Skip;
  case $x<=29,6<=y,y<=x-1$: 
   $x'<=29,y'>=x',36x-3y+18>=36x'-y',3x'-y'>=22,x'-x>=3,x'<=y'+12$;Skip;
  case $x<=29,6<=y,y<=x-1$: 
   $x'>=30,36x-3y+18>=36x'-y',3x'-y'>=22,x'-x>=3,x'<=y'+12$;Skip;
  case $x<=29,y<=x-1,y<=5$: 
   $x'<=29,y'<=x'-1,y'<=5,x'<=y'+12,36x-18y+141>=36x'-y'-82,6x-3y+6>=6x',y'-22,2x'-2x+y-12>=0,x'-x>=4,3x'-y'>=22$;Skip;
  case $x<=29,y<=x-1,y<=5$: 
   $x'<=29,6<=y',y'<=x'-1,x'<=y'+12,36x-18y+141>=36x'-y'-82,6x-3y+6>=6x'-y'-22,2x'-2x+y-12>=0,x'-x>=4,3x'-y'>=22$;Skip;
  case $x<=29,y<=x-1,y<=5$: 
   $x'<=29,y'<=x'-1,y'<=5,x'<=y'+12,36x-18y+141>=36x'-y'-82,6x-3y+6>=6x'-y'-22,2x'-2x+y-12>=0,x'-x>=4,3x'-y'>=22$;Skip;
  case $x<=29,y<=x-1,y<=5$: 
   $x'<=29,y'>=x',x'<=y'+12,36x-18y+141>=36x'-y'-82,6x-3y+6>=6x'-y'-22,2x'-2x+y-12>=0,x'-x>=4,3x'-y'>=22$;Skip;
  case $x<=29,y<=x-1,y<=5$: 
   $x'>=30,x'<=y'+12,36x-18y+141>=36x'-y'-82,6x-3y+6>=6x'-y'-22,2x'-2x+y-12>=0,x'-x>=4,3x'-y'>=22$;Skip;  
  case $x<=29,y<=x-1,y<=5$: 
   $x'<=29,y'<=x'-1,y'<=5,x'-2x+y=0,x'=y',y+2<=x',x'<=7$;Skip;
  case $x<=29,y<=x-1,y<=5$: 
   $x'<=29,6<=y',y'<=x'-1,x'-2x+y=0,x'=y',y+2<=x',x'<=7$;Skip;
  case $x<=29,y<=x-1,y<=5$: 
   $x'<=29,y'<=x'-1,y'<=5,x'-2x+y=0,x'=y',y+2<=x',x'<=7$;Skip;
  case $x<=29,y<=x-1,y<=5$: 
   $x'<=29,y'>=x',x'-2x+y=0,x'=y',y+2<=x',x'<=7$;Skip;
  case $x<=29,y<=x-1,y<=5$: 
   $x'>=30,x'-2x+y=0,x'=y',y+2<=x',x'<=7$;Skip;
  case $x<=29,y>=x$: 
   $x'<=29,y'>=x',x'=x+2,y'=y-10$;Skip;
  case $x<=29,y>=x$: 
   $x'<=29,6<=y',y'<=x'-1,x'=x+2,y'=y-10$;Skip;
  case $x<=29,y>=x$: 
   $x'<=29,y'<=x'-1,y'<=5,x'=x+2,y'=y-10$;Skip;
  case $x<=29,y>=x$: 
   $x'<=29,y'<=x'-1,y'<=5,x'=x+2,y'=y-10$;Skip;
  case $x<=29,y>=x$: 
   $x'<=29,y'<=x'-1,y'<=5,x'=x+2,y'=y-10$;Skip;
  case $x<=29,y>=x$:
   $x'>=30,x'=x+2,y'=y-10$;Skip;
}
\end{lstlisting}
\end{lrbox}

% Figure environment removed



\end{document}
