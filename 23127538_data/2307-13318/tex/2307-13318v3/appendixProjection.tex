
Consider the task to project the polyhedron $H=\{(\tsEval, \tsEval')\mid \tsEval\models \Phi \wedge \tsEval,\tsEval'\models \tsGuardcond\}$ in the treatment  of a transition $\tau=(\tsLoc, \tsLoc',\tsGuardcond)$ stated above, where $\Phi$ is an affine assertion. Recall that the transition is derived in the way that the relationship between the variables from $\tsVars$ and $\tsVars'$ is given by some affine assignment $\mathbf{x}:=\mathbf{A}\mathbf{x}+\mathbf{b}$ (i.e., $\mathbf{x}'=\mathbf{A}\mathbf{x}+\mathbf{b}$) under some conditional branch in the canonical form of Figure~\ref{fig:unnestedPQandRecursive}. We consider two cases below.
\begin{itemize}
\item The first case is that the matrix $\mathbf{A}$ is invertible. In this case, we have that $\mathbf{x}=\mathbf{A}^{-1}\mathbf{x}'-\mathbf{A}^{-1}\mathbf{b}$, and we obtain an affine assertion $\Phi'$ over $\tsVars'$ that defines the projected polyhedron directly as 
$(\Phi\wedge \tsGuardcond)[(\mathbf{A}^{-1}\mathbf{x}'-\mathbf{A}^{-1}\mathbf{b})/\mathbf{x}]$. In this case, no polyhedral projection is needed. 
\item The second case is that the matrix $A$ is not invertible. Then we solve the system of affine equations $\mathbf{A} \mathbf{x}=\mathbf{x}'-\mathbf{b}$ by the standard method of Gaussian Elimination in elementary affine algebra 
%(see e.g. standard textbooks such as \cite[Chapter ?]{?}) 
and obtains that 
$
\textstyle\mathbf{x}= \mathbf{u}(\mathbf{x}') + \sum_{i=1}^k a_k\cdot \mathbf{v}_i~(a_1,\dots, a_k\in \mathbb{R})
$
where (i) the vector $\mathbf{u}(\mathbf{x}')$ is a solution to the non-homogenous equation $\mathbf{A}\mathbf{x}=\mathbf{x}'-\mathbf{b}$ and can be expressed as an affine combination of the entries in $\mathbf{x}'$ (i.e., $\mathbf{u}(\mathbf{x}')=\mathbf{C} \mathbf{x}'+\mathbf{d}$ for some matrix $\mathbf{C}$ and vector $\mathbf{d}$) and (ii) $\mathbf{v}_1,\dots,\mathbf{v}_k$ are the basic solution of the homogeneous equation  $\mathbf{A}\mathbf{x}=\mathbf{0}$ and are constant vectors not relying on 
$\mathbf{x}'$. The fresh variables $a_1,\dots, a_k$ are the coefficients of the basic solution and can take any real value. 
As a consequence, the projection of the affine assertion $\tsEval\models \Phi \wedge \tsEval,\tsEval'\models \tsGuardcond$ (that defines the polyhedron $H$) onto the variables $\mathbf{x}'$ can be obtained  as the projection of the affine assertion 
$
\textstyle(\Phi\wedge \tsGuardcond)[(\mathbf{u}(\mathbf{x}') + \sum_{i=1}^k a_k\cdot \mathbf{v}_i)/\mathbf{x}]
$ 
onto the variables $\mathbf{x}'$ (i.e., projecting away the dimensions of $a_1,\dots, a_k$). Note that the number of the basic solution $a_1,\dots, a_k$ is equal to $n-\mathrm{rank}(A)$ where $\mathrm{rank}(A)$ is the rank of the matrix $A$. This means that the number of variables to be projected away is smaller than $n$.
%and we have that the higher $\mathrm{rank}(A)$ is, the lower the amount is. 
It follows that in this case, it is possible to project away much less variables compared with the original projection method (that needs to project away all the $n$ variables $x_1,\dots,x_n$ in $\mathbf{x}$), and thus can further improve the time efficiency.
\end{itemize}