\section{Proof for Correctness and Accuracy for our Invariant Propagation}
\label{app:ProofofCorrectnessInvPropagation}

Below we prove the theoretical properties that the affine assertions generated from our invariant propagation are indeed invariants, and are at least as tight as the invariants generated from the previous approaches~\cite{DBLP:conf/sas/SankaranarayananSM04,oopsla22/scalable}.  

\begin{proposition} 
The affine assertions generated by the invariant propagation are invariants. 
\end{proposition}
\begin{proof}
Let $\Gamma$ be an \LTS{} whose directed graph $\mbox{\sl DG}(\Gamma)$ has a non-crossing DFS tree $T$. The proof is by induction on the BFS level of the tree $T$. The base step is that the affine assertion at the root (i.e., the initial location) is correct since it is generated by the approach \cite{oopsla22/scalable}. The inductive step is to show that if the affine assertions generated at the nodes of the current level are invariants, then so are the affine assertions at the next level. The proof for the inductive step follows from the fact that any path of the \LTS{} $\Gamma$ that visits a location $\tsLoc'$ in the next BFS level should first visit some location $\tsLoc$ (with the valuation $\tsEval$ guaranteed to satisfy the invariant $\eta(\tsLoc)$) in the current BFS level, and then possibly repeatedly stays at the location $\tsLoc'$. (Note that here we use the fact that there is no crossing edge in the DFS tree $T$. 
This fact is captured by the initial condition $K_{\tau,i}$ for a transition $(\tsLoc, \tsLoc', \tsGuardcond)$ (that is obtained from the $i$th disjunctive clause $\Phi_i$ of the invariant $\eta(\tsLoc)$) and the invariant $I(\tau,\tsLoc',i)$ for the self-loop \LTS{} $\Gamma[\tsLoc', K_{\tau,i}]$ in a single propagation step.  
\end{proof}

\begin{proposition}\label{prp:propagation}
The invariant propagation generates invariants at least as tight as the previous approaches \cite{DBLP:conf/sas/SankaranarayananSM04,oopsla22/scalable}. 
\end{proposition}
\begin{proof}
The proof proceeds via an induction on the BFS level of the invariant propagation. For the base step, we have that the affine invariant generated at the root is generated directly from the previous approach~\cite{oopsla22/scalable}. Then the base step follows from the fact that the approach~\cite{oopsla22/scalable} has the same precision as the original approach \cite{DBLP:conf/sas/SankaranarayananSM04}. For the inductive step, suppose the induction hypothesis that the invariant of every node at the current BFS level in the DFS tree implies the counterpart generated by the approach~\cite{DBLP:conf/sas/SankaranarayananSM04}. We prove that the implication holds for the next BFS level. The proof can be obtained by observing that each individual affine inequality (as a conjunctive inequality in an affine assertion) in the invariants generated by the approach~\cite{DBLP:conf/sas/SankaranarayananSM04} on a location $\tsLoc'$ at the next BFS level satisfies the consecution condition derived from any transition $\tau=(\tsLoc,\tsLoc',\tsGuardcond)$ to the location $\tsLoc'$, so that each such inequality is implied by the initial condition $K_{\tau,i}$ and satisfies the possible consecution condition from the self-loop in $\Gamma[\tsLoc', K_{\tau,i}]$. Since we apply the same approach ~\cite{DBLP:conf/sas/SankaranarayananSM04} (i.e., solving the same constraints for the unknown coefficients from the consecution condition of the self-loop), the invariant $\eta(\tsLoc')$ generated by our invariant propagation implies any individual affine inequality generated by the approach~\cite{DBLP:conf/sas/SankaranarayananSM04}.  
\end{proof}
