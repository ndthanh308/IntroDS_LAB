%% 
%% Copyright 2007-2020 Elsevier Ltd
%% 
%% This file is part of the 'Elsarticle Bundle'.
%% ---------------------------------------------
%% 
%% It may be distributed under the conditions of the LaTeX Project Public
%% License, either version 1.2 of this license or (at your option) any
%% later version.  The latest version of this license is in
%%    http://www.latex-project.org/lppl.txt
%% and version 1.2 or later is part of all distributions of LaTeX
%% version 1999/12/01 or later.
%% 
%% The list of all files belonging to the 'Elsarticle Bundle' is
%% given in the file `manifest.txt'.
%% 
%% Template article for Elsevier's document class `elsarticle'
%% with harvard style bibliographic references

\documentclass[preprint,12pt]{elsarticle}
%\documentclass[5p,times]{elsarticle}

\makeatletter
\def\ps@pprintTitle{%
 \let\@oddhead\@empty
 \let\@evenhead\@empty
 \def\@oddfoot{}%
 \let\@evenfoot\@oddfoot}
\makeatother

%% Use the option review to obtain double line spacing
%% \documentclass[preprint,review,12pt]{elsarticle}

%% Use the options 1p,twocolumn; 3p; 3p,twocolumn; 5p; or 5p,twocolumn
%% for a journal layout:
%% \documentclass[final,1p,times]{elsarticle}
%% \documentclass[final,1p,times,twocolumn]{elsarticle}
%% \documentclass[final,3p,times]{elsarticle}
%% \documentclass[final,3p,times,twocolumn]{elsarticle}
%% \documentclass[final,5p,times]{elsarticle}
%\documentclass[final,5p,times,twocolumn]{elsarticle}

%% For including figures, graphicx.sty has been loaded in
%% elsarticle.cls. If you prefer to use the old commands
%% please give \usepackage{epsfig}

%% The amssymb package provides various useful mathematical symbols
\usepackage{amssymb}
%% The amsthm package provides extended theorem environments
%% \usepackage{amsthm}

%% The lineno packages adds line numbers. Start line numbering with
%% \begin{linenumbers}, end it with \end{linenumbers}. Or switch it on
%% for the whole article with \linenumbers.
%% \usepackage{lineno}

\usepackage{float}
\usepackage{subfigure}
\usepackage{amsmath}
%\usepackage{subcaption}
\usepackage{afterpage}
\usepackage{longtable}

\usepackage{graphicx,subfigmat,etoolbox,amssymb,float}

\newcounter{subfigcount}
\setcounter{subfigcount}{0}

%\journal{Future Generation Computer Systems}

\begin{document}

\begin{frontmatter}

%% Title, authors and addresses

%% use the tnoteref command within \title for footnotes;
%% use the tnotetext command for theassociated footnote;
%% use the fnref command within \author or \address for footnotes;
%% use the fntext command for theassociated footnote;
%% use the corref command within \author for corresponding author footnotes;
%% use the cortext command for theassociated footnote;
%% use the ead command for the email address,
%% and the form \ead[url] for the home page:
%% \title{Title\tnoteref{label1}}
%% \tnotetext[label1]{}
%% \author{Name\corref{cor1}\fnref{label2}}
%% \ead{email address}
%% \ead[url]{home page}
%% \fntext[label2]{}
%% \cortext[cor1]{}
%% \affiliation{organization={},
%%             addressline={},
%%             city={},
%%             postcode={},
%%             state={},
%%             country={}}
%% \fntext[label3]{}

\title{Experimental Investigation of Airfoil Trailing Edge Noise Reduction by using TE Serrations}

%% use optional labels to link authors explicitly to addresses:
 \author[label1,label2]{Weicheng Xue}
 \affiliation[label1]{organization={Department of Computer Science and Technology, Tsinghua University},
             city={Beijing},
             postcode={100084},
             country={China}}

 \author[label2]{Bing Yang}
 \affiliation[label2]{organization={Institute of Engineering Thermophysics, Chinese Academy of Sciences},
             city={Beijing},
             postcode={100083},
             country={China}}             

%% use optional labels to link authors explicitly to addresses:
%% \author[label1,label2]{}
%% \affiliation[label1]{organization={},
%%             addressline={},
%%             city={},
%%             postcode={},
%%             state={},
%%             country={}}
%%
%% \affiliation[label2]{organization={},
%%             addressline={},
%%             city={},
%%             postcode={},
%%             state={},
%%             country={}}

\begin{abstract}
%% Text of abstract
The growing prominence of aerodynamic noise from wind turbine blades at high wind speeds has made it the primary source of noise for wind turbines, with adverse effects on nearby residents' living conditions. This study focuses on experimental research conducted in an anechoic wind tunnel to investigate the noise reduction mechanism of wind turbine blade airfoils using serrated trailing edges, aiming to contribute to the development of low-noise wind turbine blades. Three models, including two types of NACA series airfoils and one reference plate with attachable serrated trailing edges, were tested. The findings reveal that airfoils with serrated trailing edges exhibit a 3 to 6 dB reduction in the mid-high frequency wideband noise ($0.5 < S_t < 1$), with the width of the frequency band of noise reduction slightly increasing as the Reynolds number rises. Furthermore, the magnitude of noise reduction in the mid-high frequency band is greater at moderate angles of attack compared to zero degrees. The presence of serrations also eliminates multiple tones of high amplitude exceeding 10 dB. The study highlights serration height as the most influential factor for noise reduction, surpassing the significance of serration width and the ratio of width to height. Moreover, investigations into the noise reduction mechanism indicate varying degrees of reduction in streamwise fluctuating velocity spectra near the serrated trailing edge, even aligning with changes in the sound power spectra. Serrations were found to alter the turbulence length scale in the downstream flow field, potentially impacting noise generation. This study suggests that the reduction in streamwise fluctuating velocity near the serrated trailing edge plays a crucial role in noise reduction, highlighting the importance of detailed flow field measurements and analysis for a comprehensive understanding of the mechanistic relationship between flow changes and serration-induced noise reduction.
\end{abstract}

%%Graphical abstract
%\begin{graphicalabstract}
%% Figure removed
%\end{graphicalabstract}

%%Research highlights
\begin{highlights}
\item Experiments were conducted in an anechoic wind tunnel, aimed at investigating the noise reduction mechanism of wind turbine blade airfoils through the utilization of serrated trailing edges. Three distinct models were employed, consisting of two variants of NACA series airfoils and one reference plate.
\item Airfoils featuring serrated trailing edges exhibited a reduction in the mid-high frequency wideband noise, ranging from 3 to 6 dB. This reduction was observed within the frequency range of approximately $0.5 < S_t < 1$. Additionally, the width of the frequency band displaying noise reduction exhibited a slight increase with higher Reynolds numbers.
\item The magnitude of noise reduction within the mid-high frequency band was more pronounced at moderate angles of attack compared to zero degrees for airfoils employing serrated trailing edges
\item Airfoils integrated with serrations effectively eliminated multiple tones of high amplitude, surpassing 10 dB in reduction.
\item In the context of noise reduction magnitude, the serration height emerged as the most influential factor, surpassing the significance of serration width and the ratio of serration width to height.
\end{highlights}

\begin{keyword}
%% keywords here, in the form: keyword \sep keyword
wind turbine \sep serrations \sep noise reduction \sep aerodynamic noise \sep wind tunnel
%% PACS codes here, in the form: \PACS code \sep code

%% MSC codes here, in the form: \MSC code \sep code
%% or \MSC[2008] code \sep code (2000 is the default)

\end{keyword}

\end{frontmatter}

%% \linenumbers

%% main text
\section{Introduction}

The development of noise reduction technologies can facilitate an increase in the tip speed of wind turbine blades since aerodynamic noise generated from a wind turbine blade increases with the fifth power of the wind speed~\cite{SORENSEN2012225}. In order to improve the productivity and efficiency of wind turbines, the size of wind turbines and the length of blades have been increasing~\cite{kaewniam2022recent,hoen2023effects}, causing the tip speed ratio of wind turbine rotors to continuously increase~\cite{kosasih2016influence}. Given the fifth power relationship, a 15\% increase in the tip speed can reduce the energy consumption of a wind turbine by 5\% to 7\%, while a 15\% increase in the tip speed can increase the noise by 3 dB~\cite{barone2011survey}.

The main sources of noise in wind turbines are aerodynamic noise from the blades and mechanical noise from the nacelle~\cite{deshmukh2019wind}. Mechanical noise is generated from mechanical components and displays tonal characteristics~\cite{pinder1992mechanical}. Aerodynamic noise is generated from the interaction of the air and the blade, which can be classified into inflow turbulence noise and self-noise~\cite{oerlemans2011wind}. Inflow turbulence noise greatly depends on the atmospheric conditions. Based on the different mechanisms~\cite{brooks1989airfoil}, self-noise can be further divided into turbulent boundary layer trailing edge noise, laminar boundary layer trailing edge noise, stall separation noise (including local stall separation and full stall separation), blunt trailing edge vortex shedding noise, and blade tip vortex shedding noise. Since wind turbines mostly operate below their stall conditions, stall separation noise only dominates in specific situations. Also, blunt trailing edge vortex shedding noise can be avoided if using a sharp trailing edge. A significant amount of research has been conducted on the self-noise related to the two remaining mechanisms, i.e., turbulent boundary layer trailing edge noise and laminar boundary layer trailing edge noise, both related to the interaction of boundary layer with the sharp trailing edge. To reduce the trailing edge noise without adversely affecting the aerodynamic performance, modifications of the trailing edge can be applied, including brush trailing edge~\cite{herr2005experimental,herr2007design,finez2010broadband,suryadi2023identifying}, perforated trailing edge~\cite{suryadi2023identifying,geyer2010measurement,zhang2020experimental}, and serrated trailing edge~\cite{oerlemans2009reduction,moreau2011flat,moreau2013noise,avallone2018noise,celik2021aeroacoustic,pereira2022aeroacoustics}.

Despite significant progress in the research on noise reduction using trailing edge modifications, the underlying mechanisms behind the effectiveness of modified trailing edges in reducing trailing edge noise remain not fully understood. To provide a comprehensive overview of the current understanding of trailing edge noise reduction techniques, it is crucial to examine the relevant theoretical studies, experimental research, and simulation-based approaches. The comprehensive investigation will provide valuable insights into the intricate details of the noise reduction process, enabling us to further optimize the design and implementation of trailing edge serrations for noise reduction.

\subsection{Theory}

Howe made foundational contributions to the analytical solution of airfoil noise. Building upon the work of Lighthill~\cite{lighthill1954sound} and Powell~\cite{powell1964theory}, Howe further derived the ideal fluid wave equation based on stagnation enthalpy as the fundamental variable~\cite{howe1975contributions}. The derived equation suggests that vorticity sources are the primary noise source if entropy gradient sound sources are not considered. The order of magnitude of the noise is related to the rate of vorticity cutting across streamlines. Howe also theoretically predicted the influence of applying the Kutta condition on noise generation~\cite{howe1976influence}. If the Kutta condition is imposed at the trailing edge, the scattering effect between incoming vortices and the trailing edge vortex street is completely canceled out, resulting in no noise generation and zero longitudinal fluctuating velocities for all fluid particles on the airfoil surface and its wake. Furthermore, Howe also investigated the noise reduction effects of serrated trailing edges~\cite{howe1991aerodynamic,howe1991noise}. He pointed out that effective noise reduction can be achieved in the high-frequency range when the dynamic characteristic length is sufficiently small compared to the geometric parameters of the serrations. Additionally, the impact of various dimensions such as serration height, serration width, and boundary layer thickness on noise reduction must be considered. It can be derived theoretically that noise decreases as the root length of the serrations decreases, and the best noise reduction is achieved when the serrated trailing edge transitions into a brush-like shape. Nevertheless, the coherent interaction between the trailing edge and the incoming turbulent vortices can cause fluctuations in the spectrum curve for small serration configurations. However, the theory is based on ideal assumptions and does not consider the influence of other types of complex vortex motions near the airfoil trailing edge on the flow field and sound field. Therefore, it has limitations in predicting the application of serrated trailing edge noise reduction.

\subsection{Experiments}

Gruber et al.~\cite{gruber2010experimental} observed that the noise reduction using serrated trailing edges can be as much as 5 dB across a wide frequency range. However, the magnitude of the noise reduction decreases in the high-frequency range. The noise even increases above a critical frequency, independent of the serration parameters. Similar observations of increased high-frequency noise were also reported in Ref~\cite{oerlemans2009reduction}. It is found that the addition of a serrated trailing edge resulted in a reduction in the wake turbulence intensity. The turbulence length scale in the wake increases as the serration root length decreases. Gruber et al.~\cite{gruber2011mechanisms} proposed design guidelines for serrated trailing edges: $f_0 \delta/U_0 < 1$, $h/\delta > 0.5$, and maintaining a sufficiently small value of the serration root length $\lambda$. They attempted to explain the noise reduction mechanism based on three parameters: $\omega \delta/U_c < 1$, $h/\delta$, and $h/\lambda$. It is found that $\omega \delta/U_c$ strongly affects the frequency range for the noise reduction. The influence of $h/\lambda$ is less pronounced than that of $h$ on noise reduction. Although the fundamental noise reduction mechanism was not fundamentally explained, some explanations regarding the three important parameters can provide some valuable suggestions for further experimental design.

Dassen et al.~\cite{dassen1996results} investigated the noise reduction effects of serrated trailing edges with different configurations on various airfoils and a flat plate. Noise reduction of 3 to 8 dB can be achieved when using serrations. The noise reduction performance of serrated trailing edges on the flat plate was better than on the airfoils. Finez et al.~\cite{finez2011broadband} investigated the noise reduction effects of serrated trailing edges with different geometric parameters on a cascade composed of 7 NACA 651210 airfoils. Their experiment showed that the cascade effect had no significant impact on the noise reduction in the low to mid-frequency range. They attributed the noise reduction to the pressure difference between the pressure and suction sides, which increases the distance of the suction side turbulent boundary layer from the wall, ultimately leading to a reduction in the sound source efficiency. The additional microscopic fluctuation region displayed by Particle Image Velocimetry (PIV) along each serration reduces the correlation of flow near the sound source and hence reduces the sound source efficiency. 

Herr et al.~\cite{herr2005experimental,herr2007design} studied the noise reduction effects of various brush trailing edges on a flat plate and a NACA 0012 airfoil. They used brush height and displacement thickness as characteristic lengths to calculate non-dimensional frequencies and plotted the sound pressure level spectra. When using brush height as the characteristic length, their experimental results were consistent with the findings of Brooks et al.~\cite{brooks1989airfoil}. However, when using displacement thickness as the characteristic dimension, the experimental results in the low to mid-frequency range did not match the similarity rules reported in Refs~\cite{brooks1989airfoil,brooks1980prediction,brooks1981trailing}. The effects of the brush length, the gap between brushes and the flexibility of the brush are investigated. Overall, their explanation of the noise reduction mechanism as the brush trailing edge weakening the flow discontinuity caused by the pressure and suction sides of the trailing edge was merely speculative.

Nash et al.~\cite{nash1999boundary} investigated the mechanism of pure tone noise generation on the NACA 0012 airfoil under moderate Reynolds number conditions ($10^5 < Re_c < 2 \times 10^6$). They used three-dimensional Laser Doppler Anemometry (LDA) to observe the enhanced instability in the upstream boundary layer of the trailing edge, which exhibited the same frequency as the pure tone noise. Based on their observations, they proposed a new mechanism to explain pure tone noise, which involves the amplified Tollmien-Schlichting (T-S) instability caused by the deformed laminar boundary layer in the pressure side separation region. Additionally, increasing the angle of attack was found to increase the pure tone noise. This can be attributed to the decrease in adverse pressure gradient on the pressure side of the trailing edge, which allows the trailing edge to maintain a laminar state even at higher velocities.

Chong et al.~\cite{chong2013experimental} conducted extensive experimental research on unsteady pure tone noise, focusing on the effects of Tollmien-Schlichting (T-S) waves and separation bubbles. In their studies under moderate Reynolds number conditions ($10^5 < Re_c < 10^6$), they investigated the mechanism of how a serrated trailing edge reduces the instability noise of a NACA 0012 airfoil. The serrations can alter the length scale of the separation bubbles near the trailing edge, thus reducing the intensity of unsteady pure tone noise. The use of a serrated trailing edge induces a three-dimensional flow, weakens the normal direction fluctuation intensity near the trailing edge, reduces the adverse pressure gradient near the trailing edge, and suppresses the amplification effect of separation bubbles in the scattering process. The trailing edge noise exhibits a distinct characteristic of multiple narrowband pure tone noise components superimposed on a broadband spectrum. In Refs~\cite{chong2013airfoil,chong2011noise}, Chong et al. utilized non-planar serrated trailing edges. Serrated trailing edges can accelerate the transition from laminar to turbulent flow, thereby suppressing separation bubbles and achieving noise reduction. The intensity of the noise followed a power law of velocity with an exponent of 5.5 to 6, which is higher than the previously observed power law of 5, which indicates that trailing edge noise is the primary source, but other types of noise (such as leading edge noise) also contribute to the noise at higher wind speeds.

Moreau et al.~\cite{moreau2011flat,moreau2013noise} investigated the noise reduction characteristics of a serrated trailing edge on a flat plate under low to moderate Reynolds number (Re) conditions. Their experimental results indicate that the noise reduction characteristics linearly depend on the Strouhal number and also on the wavelength of the serrations. However, contrary to Howe's theory, it is observed that serrations with larger wavelengths provide better noise reduction performance compared to smaller wavelength serrations. Additionally, significant variations in the noise reduction performance of serrated trailing edges are observed under different wind speed conditions. Flow measurements indicate that the main reason for noise reduction by serrated trailing edges is the alteration of the flow field near the source location rather than changes in the radiation efficiency of the source. Hence, the assumption that serrated trailing edges do not affect the turbulent flow field is deemed invalid, which explains the significant disparity between theoretical predictions and experimental measurements.

Oerlemans et al.~\cite{oerlemans2009reduction} conducted experiments on a full-scale 2.3 MW wind turbine to investigate the effect of serrated trailing edges on reducing wind turbine noise. The results showed that serrated trailing edges can achieve a noise reduction of up to 3.2 dB below 1000 Hz. However, above 1000 Hz, an increase in noise was observed, which is attributed to the inconsistency between the installed serrations and the flow direction.

\subsection{Numerical Simulations}

Jones and Sandberg~\cite{jones2010numerical,sandberg2010reprint} conducted DNS (Direct Numerical Simulation) studies on the flow field and sound field of airfoils under low Reynolds number conditions. They found that in addition to the trailing edge noise commonly studied, there are other important sound sources. The trailing edge noise typically dominates the mid-to-low frequency range, while the high-frequency noise is mainly caused by the reattachment region on the suction side, which exhibits unsteady characteristics and results in the location of high-frequency sound sources continuously changing. They also observed that the installation of serrations leads to high-frequency noise reduction without affecting the low-frequency noise, and the flow field upstream of the serrations remains largely unchanged. They explained this as the serrations primarily altering the scattering and diffraction processes of the noise and possibly only modifying the flow near the serrations and close to the trailing edge. Their conclusion of high-frequency noise reduction without affecting the low-frequency noise is similar to Howe's theoretical findings but inconsistent with many experimental studies.

\section{Experimental Setup}

\subsection{Anechoic wind tunnel}

The anechoic wind tunnel used in the experiment in this work is a small open-circuit wind tunnel located in the Fluid and Acoustic Engineering Laboratory at Beihang University, as shown in Fig.~\ref{tunnel}. The wind tunnel outlet has a circular shape with a diameter of 150 mm. Its maximum wind speed can reach 50 m/s. The relationship between wind speed and turbulence intensity at the center of the wind tunnel outlet is given in Table~\ref{velocity_turbulence}, indicating a low level of turbulence in the wind tunnel. In this study, noise and flow measurement experiments were conducted under three different wind speeds: V = 15 m/s, 25 m/s, and 35 m/s.

% Figure environment removed

\begin{table}[H]
	\caption{Wind velocity at the centerline of the wind tunnel exit and corresponding turbulence intensity}
	\centering
	\begin{tabular}{ccccc}
		\hline
		Wind Velocity, m/s& 15.19& 25.12& 35.45& 45.69\\
		Turbulence intensity, \%& 0.036& 0.080& 0.101& 0.144\\
		\hline
	\end{tabular}
	\label{velocity_turbulence}
\end{table}

Fig.~\ref{background_airfoil} shows the spectral comparison of background noise in the anechoic chamber without an airfoil and with an airfoil (NACA 634421) at wind velocities of V = 25.12 m/s and 35.45 m/s. At low frequencies, there is a significant difference of more than 10 dB between the noise without an airfoil and with an airfoil. At high frequencies, the noise difference is around 1-3 dB. This anechoic chamber is reliable for airfoil aeroacoustic experiments.

% Figure environment removed

\subsection{Experimental Airfoils}

Two airfoil models, NACA 633418 and NACA 634421, as well as a flat plate, were used in the experiment. The chord length of all models was 74 mm, and the span was 160 mm. The airfoil or plate models were constructed by joining two halves together, which were clamped using circular discs at both ends. The trailing edge was manufactured by slotting a blunt trailing edge with a thickness of 0.6 mm, allowing for the insertion of serrations of different sizes, as shown in Fig.~\ref{airfoil_plate}.

% Figure environment removed

\subsection{Serrated Trailing Edges}

The height ($2h$) of the serrations in the experimental group was set to 8 mm, 10 mm, and 12 mm, respectively. The ratio ($\lambda/h$) of the serration wavelength ($\lambda$) to half-height ($h$) was set to 0.8, 0.4, and 0.2, respectively. The corresponding straight bars in the reference group were selected to have the same wetted area as the serrations, with a height equal to half of the serration height. Therefore, the lengths of the straight bars were 4 mm, 5 mm, and 6 mm, respectively. Fig.~\ref{serrations} shows a representation of the trailing edges used in this experiment. Table~\ref{serration_parameter} and Table~\ref{bar_parameter} provide a detailed list of the parameters for the serrations and straight bars, respectively. The experiment included a total of 12 variations, which consisted of the 8 types of serrations in Table~\ref{serration_parameter}, the 3 types of straight bars in Table~\ref{bar_parameter}, and the case without any additional trailing edge (designated as 0-0).

% Figure environment removed

\begin{table}[H]
	\caption{Serration parameters}
	\centering
	\begin{tabular}{cccc}
		\hline
		h, mm& $\lambda$, mm& $\lambda$/h& Model\\
        \hline
		4& 3.2& 0.8& 4-1S\\
		4& 1.6& 0.4& 4-2S\\  
		\hline
		5& 4.0& 0.8& 5-1S\\
		5& 2.0& 0.4& 5-2S\\  
        5& 1.0& 0.2& 5-3S\\
		\hline  
		6& 4.8& 0.8& 6-1S\\
		6& 2.4& 0.4& 6-2S\\  
        6& 1.2& 0.2& 6-3S\\
		\hline   
	\end{tabular}
	\label{serration_parameter}
\end{table}

\begin{table}[H]
	\caption{Straight bar parameters}
	\centering
	\begin{tabular}{cc}
		\hline
		h, mm& Model\\
		4& 4-0F\\
        5& 5-0F\\
        6& 6-0F\\
		\hline
	\end{tabular}
	\label{bar_parameter}
\end{table}

\subsection{Data acquisition and processing system}

The microphone used is the BSWA MPA 416 pressure transducer, with a radius of 1/4 inch and a frequency response of 20-20000 Hz, as shown in Fig.~\ref{digital_acquisition}. It can be suitable for this experiment.

The digital acquisition instrument used is a product from NI (National Instruments), consisting of a PXIe-1071 chassis, a PXIe-8102 controller, multiple PXIe-4496 data acquisition cards, and related communication cables, as shown in Fig.~\ref{digital_acquisition}. One PXIe-4496 data acquisition card can sample for 16 channels synchronously, with a single-channel synchronous sampling rate of 204.8 kS/s, meeting the sampling frequency requirements for noise measurement in this work.

% Figure environment removed

After the sound pressure signal from the microphone enters the noise analyzer, it is processed and stored using the data acquisition program written in LabVIEW, resulting in a .tdms format file. When analyzing the data, the required .tdms format file is imported into the FFT analysis program also written in Labview. The FFT analysis program has the functionality to calculate the FFT and allows for setting the starting point of the time segment, the number of time segments, the sliding window function, reference channels, and more. The interface of the FFT analysis program is shown in Fig.~\ref{fft_analysis}. Unless otherwise specified, the sampling rate in this experiment is 65536 samples/s, the time segment length is 4096, the total sampling time is 10 s for each data acquisition, and the spectral overlap rate is set to 50\% to enable smooth spectrum calculation when averaging over the total sampling time.

% Figure environment removed

\section{Sound Power Level vs Reynolds Number}

Due to the wide range of frequencies studied in this work, a narrower frequency band is more suitable to use, in order to display reliable results. Unless otherwise specified, the sound power level calculation in this work is presented for 1/3-octave bands. High-pass filtering is employed in the data processing, with the cutoff frequency being 20 Hz. The effective center frequency of the 1/3-octave bands starts from 315 Hz and comprises a total of 17 frequency bands.%, as shown in Table~\ref{band_center}.

%\begin{table}[H]
	%\caption{1/3-octave band center frequency}
	%\centering
	%\begin{tabular}{cccccccccccccccccc}
	%	\hline
	%	Number& 1& 2& 3& 4& 5& 6& 7& 8& 9& 10& 11& 12& 13& 14& 15& 16& 17\\
	%	Center frequency, Hz& 315& 400& 500& 630& 800& 1000& 1250& 1600& 2000& 2500& 3150& 4000& 5000& 6300& 8000& 10000& 12500\\
	%	\hline
	%\end{tabular}
	%\label{band_center}
%\end{table}

Noise has a continuous spectral characteristic and is suitable to be analyzed in frequency bands. Typically, two types of frequency bands can be used. The first type is a doubling frequency band with upper and lower cutoff frequencies in a 2:1 ratio. The second type divides the doubling frequency band into three equal parts, known as 1/3-octave bands. This is the approach adopted in this study. If the center frequency of a band is $f$, the upper and lower cutoff frequencies for the doubling frequency band are $\sqrt{2} f$ and $f/\sqrt{2}$ respectively, while for the 1/3-octave band, they are $\sqrt[6]{2}f$ and $f/\sqrt[6]{2}$.

\subsection{$Re = 0.7 \times 10^5$}

Fig.\ref{Re_1000} shows the sound power level curves for NACA 633418, NACA 634421 airfoils, and a flat plate with equal chord length at a Reynolds number of $0.7 \times 10^5$. For the NACA 633418 airfoil, the 0-0 model (original airfoil) exhibits a peak noise level close to 5 dB in the frequency range centered around 1250 Hz. Adding the 5-XS (X representing a number) serrations can effectively eliminate this peak noise. Additionally, compared to the original airfoil, the serrated trailing edge can reduce the wideband noise by 2-7 dB in the frequency range centered around 630-1600 Hz. However, beyond the frequency range centered around 2500 Hz, the serrated model generates more noise than the original airfoil. The advantage of the serrated model is also evident compared to the 5-0F model (with a straight bar). It can reduce the noise by 2-3 dB in the frequency range centered around 800-1250 Hz. For the NACA 634421 airfoil, the variation of sound power level follows a similar pattern to the NACA 633418 airfoil. The major difference is that the frequency range centered around 1600 Hz corresponds to the maximum noise reduction range, instead of 1250 Hz. Another difference is that the noise reduction effect of the serrated configuration for the NACA 634421 airfoil is not as significant as for the NACA 633418 airfoil, with a maximum reduction of only 5 dB. The third difference is that the frequency range in which the noise reduction by serrations is pronounced is narrower, specifically in the frequency range centered around 1000-1250 Hz. For the equal-chord-length flat plate, the serrations only reduce the noise by 2 dB in the frequency range centered around 3150 Hz compared to the 5-0F configuration (with a straight bar).

% Figure environment removed

\subsection{$Re = 1.2 \times 10^5$}

Fig.\ref{Re_1500} shows the sound power level curves for NACA 633418, NACA 634421 airfoils, and a flat plate with equal chord length at a Reynolds number of $1.2 \times 10^5$. For the NACA 633418 airfoil, the serrated trailing edge model shows a reduction of more than 3 dB in wideband noise in the frequency ranges of 800-3150 Hz and 10000-12500 Hz compared to the original airfoil. The addition of the serrated trailing edges results in lower noise levels across the entire frequency range compared to the original airfoil. Compared to the 5-0F configuration (with a straight strip), the serrated trailing edge also reduces noise levels starting from the frequency range centered around 1000 Hz. In the frequency range centered around 4000-8000 Hz, the noise reduction is between 2-5 dB. For the NACA 634421 airfoil, the trend of sound power level changes is similar to the NACA 633418 airfoil, but with slightly lower noise reduction amplitudes. For the equal-chord-length flat plate, the serrated trailing edge model shows a greater reduction in the noise (over 5 dB) compared to the 5-0F model in the frequency range centered around 5000 Hz. At higher frequencies (above the 3150 Hz range), the noise reduction pattern is similar to the previous two airfoils, but the serrated trailing edge has minimal effect on noise reduction at lower frequencies.

% Figure environment removed

\subsection{$Re = 1.6 \times 10^5$}

Fig.\ref{Re_2000} shows the sound power level curves for NACA 633418 and NACA 634421 airfoils at a Reynolds number of $1.6 \times 10^5$. For both airfoils, the noise reduction pattern after adding serrations is consistent, but with slightly different amplitudes. The serrations eliminate the peak noise at the center frequency of 3150 Hz for both airfoils, with the NACA 633418 airfoil achieving a reduction of up to 5 dB. For the NACA 633418 airfoil, the serrated trailing edge configuration reduces noise by 1-5 dB in the frequency range centered around 1600-5000 Hz and by nearly 3 dB in the frequency range centered around 10000-12500 Hz. The reduction in noise amplitude with serrations is about 1 dB lower for the NACA 634421 airfoil compared to the NACA 633418 airfoil. For both airfoils, the primary frequency range for noise reduction compared to the 5-0F configuration (with a straight strip) is centered around 8000-10000 Hz, with reductions of 3-5 dB.

% Figure environment removed

\subsection{Noise reduction frequency range vs $Re$}

As the sound power level variations of the two airfoil models in this work are quite similar across various Reynolds numbers, the NACA 634421 airfoil will be taken as a representation to investigate how the noise reduction frequency range changes with Reynolds number, as shown in Fig.~\ref{noise_reduction_intervals}. The defined intervals P1 and P2 are as follows: P1 denotes the frequency interval where the noise reduction of the added serrated airfoil type (5-XS, where X is a digit) compared to the original airfoil type (0-0) achieves or exceeds 2.5 dB, and P2 denotes the frequency interval where the noise reduction of the added serrated airfoil type compared to the added straight trailing edge type (5-0F) achieves or exceeds 2.5 dB. Two phenomena can be observed: 1. Compared to the original airfoil, adding serrations can effectively reduce noise in the low to medium frequencies as well as in the high frequencies. When compared to the airfoil with straight trailing edges, the addition of serrations provides noise reduction mainly in the mid to high frequencies; 2. The impact of serration wavelength ($\lambda$) or width-to-height ratio ($\lambda/h$) on the sound power level is not significant. However, it is observed that as $\lambda$ becomes smaller or $\lambda/h$ decreases, the noise reduction effect of serrations improves.

% Figure environment removed

In order to better study the distribution of noise reduction frequency intervals for airfoils with serrated trailing edges, a Strouhal number (St) as a function of frequency ($f$) and wind speed ($U$) is used, which is given in Eq.~\ref{St}. Contour plots of noise reduction were generated, as shown in Fig.~\ref{}. The boundaries (white lines) of the noise reduction regions show a clear dependence on the Strouhal number.

\begin{equation}
\label{St}
S_t = \frac{f \delta}{U - U_0}
\end{equation}

In Eq.~\ref{St}, $f$ represents the sound frequency, and $\delta$ is the thickness of the turbulent boundary layer, which is on the order of millimeters. In this study, $\delta$ is taken as 3 mm. $U_0$ is a velocity correction factor that accounts for the white lines not passing through the origin in Fig.~\ref{noise_reduction_f_U}. This correction is necessary due to the low signal-to-noise ratio at low frequencies (below the high-pass filter cutoff frequency) and low wind speeds. The regions between the two white lines indicate clear noise reduction areas, while the areas outside the white lines show a decrease in noise reduction or even negative noise reduction.

% Figure environment removed

From Fig.~\ref{noise_reduction_f_U}, three regularities regarding the distribution of noise reduction with respect to frequency ($f$) and wind speed ($U$) can be summarized:

\begin{enumerate}
  \item P1 and P2 zones show almost mutually exclusive relationships. The serrations can achieve good noise reduction compared to both the plain trailing edge and the trailing edge with a straight bar, making serrations almost always the most effective choice. However, serrations may not achieve noise reduction across all frequency ranges relative to the reference models.
  \item The upper and lower boundaries of the noise reduction zone approximately correspond to $St_{u} = 1$ and $St_{l} = 0.5$, respectively.
  \item The larger the Reynolds number (Re), the wider the noise reduction zone.
\end{enumerate}

\subsection{The relationship between noise reduction and angle of attack}

Fig.~\ref{noise_alpha_633418} and Fig.~\ref{noise_alpha_634421} show the relationship between noise reduction and angle of attack for NACA 633418 and NACA 634421 airfoils, respectively. The noise reduction is calculated as the difference in sound power level between the reference group (without any modification or with a straight-edged trailing edge) and the airfoil equipped with a 5-2S serrated trailing edge.

% Figure environment removed

% Figure environment removed

According to Fig.~\ref{633418_with_original_alpha}, within the frequency range of 500 $\sim$ 5000 Hz, the addition of the 5-2S serrated trailing edge can achieve certain levels of noise reduction at the four tested angles of attack: 20\textdegree, 10\textdegree, 0\textdegree, and -10\textdegree. The least effective noise reduction occurs at 20\textdegree, which may be due to flow separation, leading to a decrease in the serration's effectiveness. The best noise reduction results are obtained at 10\textdegree and -10\textdegree. The serrations can effectively eliminate pure tone noise up to 11 dB at 2000 Hz and 2500 Hz. However, within the frequency range of 5000 $\sim$ 8000 Hz, some angles of attack may experience a slight increase in noise. At 0\textdegree, there is a certain level of noise reduction across the entire frequency range, with a smooth curve indicating consistent noise reduction in the mid-frequency range. As the frequency increases, the noise reduction near 5000 Hz tends to approach 0 for all angles of attack. Then, around 8000 Hz, the noise reduction quickly increases again for all angles of attack, reaching a peak at 10000 Hz before decreasing.

According to Fig.~\ref{633418_with_bar_alpha}, when compared to the airfoil with the straight trailing edge as the reference, the serrated trailing edge shows noticeable noise reduction starting at around 1000 Hz. The 20\textdegree curve changes most smoothly, and within the 1250 $\sim$ 8000 Hz frequency range, the noise reduction is consistently positive. For 10\textdegree, 0\textdegree, and -10\textdegree, their curve patterns are similar, with the noise reduction reaching extreme values within the 5000 $\sim$ 6300 Hz frequency range. However, noticeable differences appear in the 1000 $\sim$ 3150 Hz frequency range. Their primary noise reduction regions are generally within the 1600 $\sim$ 8000 Hz frequency range. Within the 1250 $\sim$ 2000 Hz frequency range, both the 10\textdegree and -10\textdegree curves show prominent peaks, and the serrations effectively reduce these peak noises. In contrast, the 0\textdegree curve does not exhibit pure tone noise within this frequency range. The 10\textdegree curve has a denser distribution and higher amplitude, while the -10\textdegree curve has a wider distribution and lower amplitude.

According to Fig~.\ref{634421_with_original_alpha}, when compared to the original airfoil, the airfoil with the 5-2S serrated trailing edge shows noise reduction in the 315 $\sim$ 5000 Hz frequency range for all four angles of attack. For the most pronounced noise reduction at a 20\textdegree, the maximum reduction occurs at the 1600 Hz frequency, reaching up to 15 dB. This indicates that the serrated trailing edge effectively reduces the pure tone noise of the NACA 634421 airfoil under high angle of attack conditions. According to Fig~.\ref{634421_with_bar_alpha}, when using the airfoil with the straight edge (5-0F) as the reference, the serrated trailing edge maintains a relatively high level of noise reduction in a wide frequency range of 1600~8000 Hz.

In summary, the impact of the serrated trailing edge on airfoils with different angles of attack can be described as follows:

\begin{enumerate}
    \item The serrated trailing edge can eliminate peak noise at high angles of attack (before separation) and reduce broadband noise at moderate angles of attack.
    \item The noise reduction pattern of the serrated trailing edge varies significantly for different airfoil shapes and angles of attack.
\end{enumerate}

\subsection{Relationship between Sound Power Level and serration height}

Fig.~\ref{633418_13o_h} and Fig.~\ref{634421_13o_h} show the curves of Sound Power Level variation with serrated trailing edge height for NACA 633418 and NACA 634421 airfoils, respectively, under three different Reynolds numbers (Re) conditions in this study.

In engineering applications, particular attention is given to the impact of serration parameters on the noise reduction performance. In this context, relevant experimental results are provided to demonstrate that finer and longer serrations lead to better noise reduction performance. This basically aligns with Howe's findings in references [11, 12], which concluded that the higher the serrated trailing edge height, the better the noise reduction effect on the airfoil.

% Figure environment removed

% Figure environment removed

At $Re=0.7\times10^5$, for the NACA 633418 airfoil, the serration height has a slight effect only in the frequency range of 1600 $\sim$ 5000 Hz. It is observed that shorter serrations lead to better noise reduction performance in this case. For the NACA 634421 airfoil, the serration height has a minor impact only in the frequency range of 800 $\sim$ 1000 Hz. In this situation, the serration height should not be too high or too short, and the 5-2S serration type shows the best noise reduction effect. At $Re=1.2\times10^5$ and $Re=1.6\times10^5$, for both airfoil types, it is observed that the higher the serration height, the better the noise reduction effect.

In summary, the impact of serration height (2h) on airfoil noise reduction can be summarized as follows:

\begin{enumerate}
    \item The influence of serration height (2h) on sound power level is not significant.
    \item For a higher Reynolds number, the larger the serration height, the lower the noise generated by the airfoils.
\end{enumerate}

\subsection{Summary}

Based on the experiments conducted on NACA 633418, NACA 634421, and the equal chord length flat plate, the following conclusions can be drawn:

\begin{enumerate}
    \item Compared to the original airfoil, adding serrations to the trailing edge can effectively reduces noise in the low and middle frequency range as well as in the high frequency range. The non-dimensional frequency range of the noise reduction boundaries roughly conforms to $St_u$=1 and $St_l$=0.5.
    \item Serrations can simultaneously achieve significant noise reduction for both the flat plate and the wing profiles with straight-edge tails.
    \item Serrated trailing edges can eliminate peak noise at large angles of attack and broadband noise at moderate angles of attack. However, the noise reduction performance varies for different airfoils.
    \item The noise reduction patterns of the three models are not very sensitive to the serration aspect ratio, while the serration height only has an effect on low-frequency noise reduction. Generally, finer and longer serrations lead to better noise reduction.
    \item The two airfoils used in this experiment have a similar impact on the sound power level curves, while the flat plate shows significant differences in the low-frequency range but consistent trends in the high-frequency range.
\end{enumerate}
    
\section{Sound field directivity} 

\subsection{Microphone Arrangement}    

Fig.~\ref{directivity_microphones} shows the arrangement of microphones in the directivity experiment. The angle between the line connecting the microphones and the wing trailing edge and the direction of the airflow is denoted as $\phi$. The three microphones are placed at positions with radii of 2.35 m, and $\phi$ values of 45\textdegree, 60\textdegree, and 75\textdegree, respectively.

% Figure environment removed   

\subsection{Sound directivity results for three experimental models}

Fig.~\ref{633418_1000_directivity}, Fig.~\ref{633418_1500_directivity}, and Fig.~\ref{633418_2000_directivity} show the directivities of the NACA 633418 airfoil at three different Reynolds numbers ($Re=0.7\times10^5$, $Re=1.2\times10^5$, and $Re=1.6\times10^5$) for the frequency centers of 400 Hz, 1000 Hz, 3150 Hz, and 10000 Hz, respectively.

% Figure environment removed

% Figure environment removed

% Figure environment removed

At Reynolds number $Re=1.2\times10^5$, for lower frequencies (400 Hz and 1000 Hz), the received far-field sound intensity is higher for larger angles $\phi$. However, as the frequency increases to 3150 Hz, the radiation intensity is lowest at 60\textdegree, while 45\textdegree and 70\textdegree positions show similar results. At higher frequencies, the sound radiation intensity received at 45\textdegree is higher than that received at 75\textdegree. Additionally, adding the serrations does not significantly alter the far-field directivity distribution of the original airfoil or the airfoil with straight-edge end plates. The sound directivity pattern at Reynolds number $Re=0.7\times10^5$ is similar to that at $Re=1.2\times10^5$. However, there is a bending in the directivity pattern at 400 Hz. The sound directivity pattern at Reynolds number $Re=1.6\times10^5$ shows significant differences at 400 Hz compared to $Re=1.2\times10^5$. At 400 Hz, the sound radiation intensity received at 75\textdegree is smaller than that at 60\textdegree and 45\textdegree positions.

The sound directivity patterns of NACA 634421 and NACA 633418 airfoils at different Reynolds numbers and frequency ranges are nearly identical. For the sake of conciseness, this work does not present the directivity results for NACA 634421.

Fig.~\ref{plate_1500_directivity} shows the sound directivity patterns of a flat plate at Reynolds number $Re=1.2\times10^5$ for 1/3 octave band frequency center points of 400 Hz, 1000 Hz, and 3150 Hz. At Reynolds number $Re=1.2\times10^5$, the sound directivity patterns of the flat plate are similar to the airfoil models at 400 Hz, showing relatively uniform distribution along various angles. At 1000 Hz, the directivity variation with angle $\phi$ is not as pronounced as observed in the airfoil models. However, at 3150 Hz, the directivity pattern of the flat plate is different from that of the airfoil models. Instead of exhibiting a folding behavior as observed in the airfoil models, the sound radiation is strongest at an angle of 45\textdegree.

% Figure environment removed

\subsection{Summary}

Based on the sound directivity experiments conducted for NACA 633418 wing model, NACA 634421 wing model, and the flat plate, the following conclusion can be drawn:

\begin{enumerate}
    \item Adding serrations to the airfoil model does not significantly alter its acoustic directivity. The maximum noise tends to propagate in a direction close to perpendicular to the wing at low to medium frequencies and in a direction closer to downstream at high frequencies. 
    \item The directivity patterns for the flat plate are similar to those of the wing models, with only slight differences observed at certain frequencies.
\end{enumerate}

\section{Wake flow measurement}

\subsection{Velocity measurement system and measurement point locations}

Fig.\ref{hot_wire} shows the spatial arrangement of the components of the hot wire measurement system. The hot wire measurement system utilized products from Dantec Dynamics A/S company. For this experiment, the 55 P13 type of thermal wire probe was used. This probe is equipped with a 90\textdegree bend, with a hot wire diameter of 5 µm and a length of 1.25 mm. It causes minimal disturbance to the flow field and has a maximum response frequency of 90 kHz. The three-dimensional displacement device employs a helical micro-displacement principle, providing a resolution of up to 0.05 mm.

% Figure environment removed

Velocity measurements were conducted on the wake flow of three different experimental models: the original airfoil (0-0 model), the airfoil with added straight-edge (5-0F model), and three different airfoil models with serrated edges (5-XS, where X represents 1, 2, or 3). The arrangement and numbering of measurement points are shown in Fig.~\ref{measurement_locations}. Taking measurement point 12 as an example, its location is 2 mm downstream of the trailing edge in the pressure surface direction, offset by 2 mm. Other measurement points can be referenced using Fig.~\ref{measurement_locations}.

% Figure environment removed

\subsection{Flow field measurement results}

Fig.~\ref{634421_velocity} shows the average velocity and turbulence intensity for NACA 634421 airfoil with different types of trailing edges at various measurement points and at a Reynolds number of $1.2\times10^5$. The left and right halves in Fig.~\ref{634421_velocity} display the average velocity or fluctuation velocity at distances of 2 mm and 7 mm from the trailing edge apex, respectively. The horizontal axis in Fig.~\ref{634421_velocity} represents the offset distance of each measurement point from the trailing edge apex in the positive downstream direction, as shown in Fig.~\ref{measurement_locations}.

% Figure environment removed

From Fig.~\ref{634421_velocity}, it can be observed that as the distance from the airfoil trailing edge increases, the velocity also increases and approaches the mainstream velocity. Similarly, as the distance from the trailing edge directly downstream the airfoil increases, the velocity also increases. Even at 4 mm away, the flow is still influenced by the serrated trailing edge, and the velocity has not fully recovered to match the mainstream velocity. Comparing the three types of airfoils: the original airfoil, the airfoil with straight-edge extension, and the airfoils with different serrations (5-XS, X=1, 2, 3), it is observed that the original airfoil and the airfoil with straight-edge extension have the smallest average velocity at the same measurement points. Among the serrated airfoils, the one with the largest serration has the highest average velocity at the same measurement points. From Fig.~\ref{634421_fluctuation_velocity}, it is seen that as the distance from the wing trailing edge increases, the overall trend of the fluctuation velocity decreases. Adding serrations can reduce the intensity of the fluctuation at multiple measurement points.

From Fig.~\ref{634421_1500_turbulence_spectrum}, the power spectral density curves of the turbulent velocity fluctuations at measurement points "10" and "20" are shown for $Re=1.2\times10^5$. Control points "10" and "20" are located directly downstream of the airfoil trailing edge and experience the most significant effects of the airfoil on the flow field. It can be observed that the original airfoil has much higher streamwise turbulent velocity power spectral density in the wake region compared to the airfoil with serrated trailing edges, particularly in the frequency range of 300$\sim$4000 Hz, where the difference can reach up to 4 dB. However, a different behavior is observed for the airfoil with straight-edge extension as its turbulent velocity spectra at these two control points are almost the smallest, especially evident at the "10" control point closer to the trailing edge. This indicates that, compared to the straight-edge extension, the serrations enhance the streamwise turbulence in the near-wake region of the airfoil. Additionally, as the measurement points get closer to the trailing edge, the differences in the flow field become more prominent, especially in the mid-low-frequency range. It is also noticeable that the serrations have a relatively minor impact on the high-frequency streamwise turbulence in the flow field, and it roughly follows a -2.5 power law, consistent with Ref~\cite{gruber2010experimental}. The effect of the serrations on the flow field also depends on the size of the serrations.

% Figure environment removed

Fig.~\ref{634421_1500_2mm_autocorelation} and Fig.~\ref{634421_1500_7mm_autocorelation} display the autocorrelation maps of the turbulent velocity fluctuations for the airfoil with 0-0 and 5-2S serrated trailing edges, respectively, at distances of 2 mm and 7 mm downstream of the trailing edge, at $Re=1.2\times10^5$. From Fig.~\ref{634421_1500_2mm_autocorelation}, it is observed that at the 2 mm location downstream of the trailing edge, the autocorrelation map of the turbulent velocity fluctuations for the original airfoil (0-0 type) has a broader high-value region and a narrower mid-low-value region. In Fig.~\ref{634421_1500_7mm_autocorelation}, at the 7 mm location downstream of the trailing edge, there is almost no difference between the autocorrelation maps of the original airfoil and the airfoil with a serrated trailing edge, indicating that as the flow field develops further, their differences diminish.

% Figure environment removed

% Figure environment removed

The characteristic eddy scale $L_u(x)$ can be estimated using the relationship $R_uu(x, \tau_0) / R_uu(x, 0) = 0.5$ from the autocorrelation function. This equation is used to determine the size of the characteristic eddy scale at a specific position $x$ by finding the value of $\tau_0$ where the autocorrelation function drops to half of its initial value. Eq.~\ref{tau} can be used:

\begin{equation}
    \label{tau}
    L_u(x) = U_c \tau_0
\end{equation}
where $U_c$ is the velocity of the incoming eddies, which can be taken as $U_c=0.7 U_0$, with $U_0$ being the free-stream velocity. It is evident that, for the same Reynolds number (Re) condition, $L_u(x)$ depends solely on different x positions and $\tau_0$.

To better illustrate the autocorrelation functions and the sizes of characteristic eddy scales for different trailing edges, they are plotted together on the same graph, as shown in Fig.~\ref{634421_1500_turbulent_length}. The values labeled as $l_{c1}$ represent the characteristic eddy scales for each trailing edge type.

From Fig.~\ref{634421_1500_turbulent_length}, it can be observed that the turbulent length scale for the 5-0F type and 0-0 type trailing edges is the largest. Additionally, the turbulent length scale for the large serrated trailing edge is larger than that of the medium and small serrated trailing edges. Serrations have the capability to reduce the characteristic eddy scale, indicating that they can disrupt larger vortex structures. Moreover, finer and longer serrations have a stronger ability to alter the flow field vortex scale.

% Figure environment removed

Fig.~\ref{634421_velocity_2000} shows the average velocity and fluctuating velocity at different measurement points for $Re=1.6\times10^5$. At $Re=1.6\times10^5$, the variation trend of average velocity at different points is similar to that at $Re=1.2\times10^5$. The difference is that the fluctuating velocity decreases at D=2 mm. This indicates that under this condition, the flow outside D=2 mm is less affected by the serrations. The serrations primarily influence the flow near the trailing edge rather than the entire flow field.

% Figure environment removed

Fig.~\ref{634421_2000_turbulent_length} shows the autocorrelation curves at measurement point 10 for different types of serrated trailing edges at $Re=1.6\times10^5$. The parameter $l_{c2}$ represents the characteristic vortex scale. Surprisingly, at $Re=1.6\times10^5$, the serrations actually lead to an increase in the characteristic vortex scale. Additionally, shorter and wider serrations have a stronger ability to alter the flow field vortex scale. Further experiments are needed to fully explain this phenomenon. Comparing the characteristic vortex scales at the two Reynolds numbers, the scale at $Re=1.6\times10^5$ is approximately 0.7 mm smaller than that at $Re=1.2\times105$. In statistical terms, this suggests that larger-scale vortices are less prevalent at higher Reynolds numbers, while smaller-scale vortices are more abundant.

% Figure environment removed

Fig.~\ref{633418_velocity} shows that NACA 633418 airfoil shares some similarities with NACA 634421 airfoil in terms of the variation of mean velocity with distance from the trailing edge. However, there are also differences between the two airfoils. The fluctuation velocities, from largest to smallest, are observed in the following order: 0-0 type, 5-0F type, 5-2S type, 5-1S type, and 5-3S type. The high-frequency fluctuations in the flow field generally follow a -2.5 power law with respect to frequency, shown in Fig.~\ref{633418_1500_turbulence_spectrum}.

% Figure environment removed

% Figure environment removed

\subsection{Summary}

Based on the flow field measurement experiments conducted on NACA 633418 and NACA 634421 airfoils, the following conclusions can be drawn:

\begin{enumerate}
    \item The addition of serrations can reduce the fluctuation velocities at multiple measurement points.
    \item At $Re=1.2\times10^5$, the fluctuation velocity spectrum of the original airfoil is highest in the mid-low frequency range, which aligns with the noise reduction range in the acoustic power spectrum. This indicates that the reduction of streamwise fluctuation velocity near the trailing edge may be a significant factor contributing to the noise reduction. Serrations can decrease the characteristic vortex scale, suggesting their ability to disrupt larger vortex structures. Moreover, finer serrations have a stronger capability to modify the flow field vortex scale.
    \item The mechanisms, distributions, and radiation patterns of sound sources in the serrated trailing edge and wake region are quite complex. The reduction of streamwise fluctuation velocity alone cannot fully explain the noise reduction achieved by serrated trailing edges. A more detailed measurement and analysis of the flow field are necessary for a clearer understanding of the noise reduction mechanisms of serrated trailing edges.
\end{enumerate}   

\section{Conclusions}
\label{conclusions}

This study focuses on the experimental investigation of noise reduction using serrated trailing edge airfoils in a wind tunnel. It is divided into three parts: acoustic spectral characteristics, sound field directivity, and wake flow field measurements. In the first part, four factors influencing the sound power level are considered: Reynolds number (Re), angle of attack ($\alpha$), serration parameters (serration height $h$, wavelength $\lambda$, and aspect ratio $\lambda/h$), and model type (airfoil or flat plate). The second part primarily examines the effects of frequency ($f$), model type, and Reynolds number on the far-field sound radiation. The third part involves wake flow field measurements, with a focus on the fluctuating velocities at measurement points.

The main conclusions of this work can be summarized as follows:

\begin{enumerate}
    \item A comparative study of airfoils with serrated trailing edges, airfoils with straight trailing edges, and original airfoils at different angles of attack and Reynolds numbers reveals that serrated trailing edge airfoils exhibit significant noise reduction compared to the original airfoils in the mid-low and high-frequency ranges (reductions of 2 dB or more). Compared to airfoils with straight trailing edges, serrated trailing edge airfoils show noticeable noise reduction in the mid-high frequency range (reductions of 2.5 to 5 dB). The effective noise reduction ranges roughly follow $St_u$ = 1 and $St_l$ = 0.5.
    \item Analysis of sound field directivity at different frequencies indicates that serrated trailing edge airfoils almost do not alter the acoustic directivity of the original airfoils. In the mid-low frequency range, the maximum noise propagates in a direction close to the perpendicular to the airfoil, while in the high-frequency range, it propagates in a direction close to the downstream.
    \item The study of wake flow velocity spectra shows that serrated trailing edge airfoils result in reduced fluctuating velocities, and the frequency range of reduction in wake fluctuating velocity spectra is consistent with that of the sound power spectra. It is suggested that the reduction in wake fluctuating velocity may be one of the reasons for the noise reduction achieved by serrated trailing edges.
\end{enumerate}   

Future experimental research is recommended to conduct simultaneous sound field and flow field measurements of serrated trailing edge noise reduction in a larger anechoic wind tunnel with a diameter greater than 400 mm. It is also suggested to utilize flow field visualization techniques as much as possible to explore the effects of larger-sized models and serrated trailing edges on the flow field and sound field.

%% The Appendices part is started with the command \appendix;
%% appendix sections are then done as normal sections
%% \appendix

%% \section{}
%% \label{}

%% For citations use: 
%%       \citet{<label>} ==> Jones et al. [21]
%%       \citep{<label>} ==> [21]
%%

%% If you have bibdatabase file and want bibtex to generate the
%% bibitems, please use
%%
\bibliographystyle{elsarticle-num-names} 
%%  \bibliography{<your bibdatabase>}

%% else use the following coding to input the bibitems directly in the
%% TeX file.

%\begin{thebibliography}{00}

%% \bibitem[Author(year)]{label}
%% Text of bibliographic item

%\bibitem[ ()]{}

%\end{thebibliography}

\bibliography{mybib}

\end{document}

\endinput
%%
%% End of file `elsarticle-template-num-names.tex'.
