\documentclass[twocolumn,pra,aps,superscriptaddress,showpacs]{revtex4}
\usepackage{amsmath}
\usepackage{graphicx}
\usepackage{epstopdf}
\usepackage{threeparttable}
\usepackage{float}
\usepackage{amsfonts}

%%%%%%%%%%%%%%%%%%%%%%%%%%%%%%%%%%%%%%%%%%%%%%%%%%%%%%%%%%%%%%%%%%%%%%%

\begin{document}
\title{Magnetic supersolid phases in
spin-orbit coupled extended Bose-Hubbard model}\emph{}
\author{Dong-Dong Pu}
\affiliation{College of Physics and Electronic Science, Hubei
Normal University, Huangshi 435002, China}
\author{Ji-Guo Wang\footnote{Corresponding author: wangjiguo@hbnu.edu.cn}}
\affiliation{College of Physics and Electronic Science, Hubei
Normal University, Huangshi 435002, China}
\author{Ya-Fei Song\footnote{Corresponding author: q1304852625@live.com}}
\affiliation{Department of Mathematics and Physics, Shijiazhuang
TieDao University, Shijiazhuang 050043, China}
\author{Xiao-Dong Bai}
\affiliation{College of Physics, Hebei Normal University,
Shijiazhuang 050043, China}



\begin{abstract}
The study of the ultracold atomic spin systems with long-range
interaction provides the possibility to search for magnetic
supersolid phases in quantum many-body scenarios. In this paper,
we consider two-species Bose gases with spin-orbit coupling and
nearest-neighbor interaction confined in a two-dimensional optical
lattice. The competition between SOC and interactions creates rich
ground-state diagrams with supersolid phase exhibiting the phase
modulation or magnetic ordering. The combined effect of
intraspecies nearest-neighbor interaction and spin-orbit coupling
results in the phase-twisted paired checkboard supersolid and
phase-striped paired checkboard supersolid phases. The
introduction of interspecies nearest-neighbor interaction enriches
the quantum phases of the system. The phase-twisted lattice
supersolid and phase-striped lattice supersolid phases are
preferred. We find that the appearance of some ground-state phases
depending on interspecies on-site interaction. A type of lattice
supersolid phase with supersolid in one spin species but
insulating in the other that exists in the miscible domain, while
the paired stripe supersolid phase with stripe structures in each
spin species in the immiscible domain. Finally, to further
characterize each phase, we discuss their spin-dependent momentum
distributions and spin-texture structures. The magnetic textures
such as antiferromagnetic, spiral and stripe orders are shown in
supersolid phases. The results here in could help in the observe
for these magnetic supersolid phases in ultracold atomic
experiments with nearest-neighbor interaction and spin-orbit
coupling in optical lattice.

\end{abstract}

\pacs{03.75.Lm, 05.70.Fh, 67.80.bd}

 \maketitle


\section{introduction}

In the past few decades,  ultracold bosonic atoms trapped in
optical lattices has been attracted extensively attention. In the
strongly interacting regime, two quantum phases: Mott insulator
(MI) phase and superfluid (SF) phase, and MI-SF phase transition
are observed in experiments \cite{M. P. A. Fisher 1989,D. Jaksch
1998,K. Sheshadri 1993,S. Sachdev 1999,M. Greiner 2002,C. Orzel
2001}, which can be described by the standard Bose-Hubbard model
with on-site interaction and nearest-neighbor (NN) hopping of
atoms. The experimental realizations of one-dimensional (1D) to
three-dimensional (3D) Bose-Hubbard models \cite{I. Bloch 2008}
provide a platform to explore the various quantum phases and phase
transitions \cite{J. K. Freericks 1994,T. Stoferle 2004,S. Folling
2006,I. B. Spielman 2007,B. C. Sansone 2007,P. Sengupta 2007,M.
Iskin 2011,X. B. Zhang 2012,T. Ohgoe 2012,H. M. Deng 2015,D. S.
Luhmann 2016,B. Gardas 2017,O. Mansikkamaki 2021,P. Zechmann
2023}. In the two-species Bose-Hubbard model, a rich variety of
quantum phases are observed due to the interspecies interaction,
such as the paired SF (PSF) phase, super-counter-fluid (SCF)
phase, peculiar magnetic state, quantum droplet, ferromagnetic
spin phase and antiferromagnetic spin phase \cite{E. Altman
2003,A. Kuklov 2004,A. B. Kuklov 2003,A. Isacsson 2005,A. Hubener
2009,A. Hu 2009,J. Pietraszewicz 2012,J. M. Zhang 2012,W. Wang
2014,S. Basak 2021,V. E. Colussi 2022,Y. Machida 2022}. Recently,
the recent experimental realization of two-species dipolar
condensate mixtures of Er-Dy \cite{A. Trautmann 2018} stimulated
the enthusiasm of researchers to study the two-species Bose gases
in the extended optical lattices. Segregated quantum phase,
supersolid (SS) phase and spin-density wave phase appeared in the
two-species Bose-Hubbard model with NN interaction \cite{T. Mishra
2008,X. Guan 2019,R. Bai 2020,D. C. Zhang 2022}.







Ultracold atoms with spin-orbit coupling (SOC) represent an
important and active research fields in the quantum gases physics.
Recently, the artificial SOC effect in multi-species Bose systems
has been realized in the cold atomic experiments by tuning Raman
field \cite{Y.-J. Lin 2011,J. Li 2016,J.-R. Li 2017}. The form of
SOC can be of either Rashba \cite{Y. A. Bychkov 1984} or
Dresselhaus \cite{G. Dresselhaus 1955} type, both being frequently
analyzed in terms of an effective gauge force. The combination of
SOC and interaction of atoms can give rise to a variety of quantum
states. The effective super-exchange spin model with the
Dzyaloshinskii-Moriya type (DM-type)interactions can be obtained
by the second-order perturbation theory \cite{I. Dzyaloshinsky
1958,T. Moriya 1960} in the MI regime of two-dimensional (2D)
spin-orbit coupled Bose-Hubbard model. The magnetic structures,
such as the spiral, vortex crystal and skyrmion crystal can be
found by applying the classical Monte-Carlo (MC) simulations,
bosonic dynamical mean field (BDMF) theory, variational order (VO)
method and tensor network states (TNS) method \cite{W. S. Cole
2012,J. Radic 2012,Z. Cai 2012,C. H. Wong 2013,J. Z. Zhao 2015,R.
Y. Li 2015,L. He 2015,B. Xiong 2016,J. G. Wang 2016,C. Wang
2017,L. Zhang 2019}. The effects of the strength and symmetry of
SOC on the ground-state phases and phase transitions are also
investigated. The phase-twisted SF (PT-SF) phase, phase-striped SF
(PS-SF) phase and orbital-ordered SF phase are driven by SOC
\cite{A. Dutta 2013,A. T. Bolukbasi 2014,D. Toniolo 2014,C. Hickey
2014,D. Yamamoto 2017,M. Yan 2017,A. Dutta 2019,K. Suthar 2021}.
However, the comprehensive theoretical study of the phase diagrams
and phase transitions in a 2D spin-orbit coupled Bose-Hubbard
model with NN interaction is still missing.



In this work we investigate the quantum phases and phase
transitions of 2D spin-orbit coupled bose gases with NN
interaction by using the inhomogeneous dynamical guztwiller mean
field (DGMF) method. The competition between SOC and interactions
(the on-site interaction and NN interaction) gives rise to a
variety of quantum phases with phase modulation or spin ordering.
The translational symmetries of each species density are broken by
the intraspecies NN interaction and while the total density is
preserved. The paired checkboard supersolid (PCSS) phase with
checkboard structure in each species and homogeneous in total
density appeared when only considering inraspecies NN interaction.
The SOC driven the phase-twisted PCSS (PT-PCSS) and phase-striped
PCSS (PS-PCSS) phases. The introduction of interspecies NN
interaction modifies the quantum phases of the system. The
phase-twisted lattice supersolid (PT-LSS) and phase-striped
lattice supersolid (PS-LSS) phases are preferred. For the lattice
supersolid (LSS) phase, the translational symmetries of each
species density and total density are broken by the NN
interactions, the lattice structure is existed stabilized in total
density. A new type of LSS (LSS-I) phase with supersolid in one
spin species but insulating in the other that exists at large
chemical potentials in the miscible domain
($U^{2}_{\uparrow\downarrow}<U_{\uparrow\uparrow}U_{\downarrow\downarrow}$)
\cite{T. L. Ho 1996}. The paired stripe supersolid (PSSS) phase
are observed in the immiscible domain
($U^{2}_{\uparrow\downarrow}>U_{\uparrow\uparrow}U_{\downarrow\downarrow}$)
\cite{P. Ao 1998}, unlike the PCSS phase, it is characterized by
the stripe structure of each species density. The SOC also driven
the PT-PSSS and PS-PSSS phases. We find that phase transitions
depending on interspecies on-site interaction
($U_{\uparrow\downarrow}$), there is a transition from the LSS
phase to the PT-SF phase, and to the PS-SF phase in the miscible
domain, while from the LSS phase to PT-PSSS phase, and to the
PS-PSSS phase in the immiscible domain. Finally, to further
characterize each phase, we have discussed their spin-dependent
momentum distributions and spin-texture structures. The magnetic
textures such as antiferromagnetic (AFM), spiral and stripe orders
are shown in the SS phases. The results here in could help in the
observe for these magnetic SS phases in ultracold atomic
experiments with NN interaction and SOC in optical lattice.


The paper is organized as follows: In Sec. II, we introduce the
model of the spin-orbit-coupled two-species Bose gases in a 2D
optical lattices with NN interaction. In Sec. III, we display
MI-SF phase transition in spin-orbit coupled Bose-Hubbard model.
In Sec. IV, the phase diagrams and phase transitions in spin-orbit
coupled extended Bose-Hubbard model without and with interspecies
NN interaction are discussed in sections A and B, respectively. A
summary is included in Sec. V.


\section{model and Hamilton}
We consider two-species bosons in 2D optical lattices with SOC and
NN interaction, which can be described by the spin-orbit coupled
extended Bose-Hubbard model. The Hamiltonian of the system is
\begin{widetext}
\begin{equation}
\begin{split}
&\hat{H}=-\sum_{p,q,\sigma}\bigg[t_{x}(\hat{b}^{\dag\sigma}_{p,q}\hat{b}^{\sigma}_{p+1,q}+H.c.)+t_{y}(\hat{b}^{\dag\sigma}_{p,q}\hat{b}^{\sigma}_{p,q+1}+H.c.)
-\frac{U_{\sigma\sigma}}{2}n^{\sigma}_{p,q}(n^{\sigma}_{p,q}-1)-V_{\sigma\sigma}(\hat{n}^{\sigma}_{p,q}\hat{n}^{\sigma}_{p+1,q}+\hat{n}^{\sigma}_{p,q}\hat{n}^{\sigma}_{p,q+1})+\mu^{\sigma}_{p,q}
\hat{n}^{\sigma}_{p,q}\bigg]\\&-\sum_{p,q}\bigg[\gamma_{x}(\hat{b}^{\dag\uparrow}_{p,q}\hat{b}^{\downarrow}_{p+1,q}-\hat{b}^{\dag\downarrow}_{p,q}\hat{b}^{\uparrow}_{p+1,q})
+H.c.-i\gamma_{y}(\hat{b}^{\dag\downarrow}_{p,q}\hat{b}^{\uparrow}_{p,q+1}+\hat{b}^{\dag\uparrow}_{p,q}\hat{b}^{\downarrow}_{p,q+1})+H.c.
-U_{\uparrow,\downarrow}\hat{n}^{\uparrow}_{p,q}\hat{n}^{\downarrow}_{p,q}+V_{\uparrow,\downarrow}(\hat{n}^{\uparrow}_{p,q}\hat{n}^{\downarrow}_{p+1,q}+\hat{n}^{\uparrow}_{p,q}\hat{n}^{\downarrow}_{p,q+1})\bigg]
\end{split}
\end{equation}
\end{widetext}
where $\sigma=\uparrow,\downarrow$ denotes the spin-$\sigma$
species and $(p,q)$ are the sites indices.
$\hat{b}^{\dag\sigma}_{p,q} (\hat{b}^{\sigma}_{p,q})$ is bosonic
creation (annihilation) operator, $\hat{n}^{\sigma}_{p,q}$ is the
bosonic number operator and $\mu^{\sigma}_{p,q}$ is the chemical
potential of spin-$\sigma$ species at site $(p,q)$. $t_{x}
(t_{y})$ and $\gamma_{x} (\gamma_{y})$ are the hopping strength
and SOC strength along the $x$ ($y$) direction, respectively.
$U_{\sigma\sigma}$ and $V_{\sigma\sigma}$ is the intraspecies
on-site and NN interactions of spin-$\sigma$ species,
respectively. For simplicity, we choose symmetric hopping (SOC)
$t_{x}=t_{y}=t$ ($\gamma_{x}=\gamma_{y}=\gamma$), identical
intraspecies on-site (NN) interaction
$U_{\uparrow\uparrow}=U_{\downarrow\downarrow}=U$
($V_{\uparrow\uparrow}=V_{\downarrow\downarrow}=V$) and equal
chemical potential
$\mu^{\uparrow}_{p,q}=\mu^{\downarrow}_{p,q}=\mu$.
$U_{\uparrow\downarrow}$ and $V_{\uparrow\downarrow}$ are the
interspecies on-site and NN interactions, respectively.


The bosonic operators can be transformed by Fourier transformation
are
$\hat{b}^{\sigma}_{p,q}=\frac{1}{\sqrt{l}}\sum_{k}\hat{b}^{\sigma}_{k}e^{ik(p+q)}$,
and they satisfy the commutation relations
$[\hat{b}^{\sigma}_{k},\hat{b}^{\sigma}_{k^{'}}]=\delta_{kk^{'}}$.
For the system with weak atomic interactions, i.e., $U \ll t$ and
$V\ll t$, the Hamiltonian of Eq. (1) in the momentum space can be
written as
\begin{equation}
\hat{H}_{kin}=\sum_{k}
 \left(
\begin{array}{cc}
\hat{b}^{\dag\uparrow}_{k} & \hat{b}^{\dag\downarrow}_{k}
\end{array}
\right) \mathcal{H}_{k}
\left(
\begin{array}{c}
\hat{b}^{\uparrow}_{k} \\ \hat{b}^{\downarrow}_{k},
\end{array}
\right)
\end{equation}
where $\mathcal{H}_{k}=-2t(\cos k_{x}+\cos
k_{y})\hat{\mathrm{I}}+2\gamma(\sin k_{y}\hat{\sigma}_{x}-\sin
k_{x}\hat{\sigma}_{y})$. The energy eigenvalues of
$\mathcal{H}_{k}$ are
\begin{equation}
E^{\pm}_{k}=-2t(\cos k_{x}+\cos k_{y})\pm
2\gamma\sqrt{\sin^{2}k_{x}+\sin^{2}k_{y}}.
\end{equation}

The four degenerate minima in the lower branch are $\pm
\mathbf{Q}=(\pm k_{0},\pm k_{0})$ with
$k_{0}=\arctan\frac{\gamma}{\sqrt{2}t}$. The corresponding
eigenstates are
\begin{equation}
\Psi^{\pm}_{k}=\frac{1}{\sqrt{2}}e^{\pm i\mathbf{r}\cdot
\mathbf{Q}} \left(
\begin{array}{c}
1 \\ e^{\pm i\pi/4}.
\end{array}
\right)
\end{equation}
The location of the minima of Bose gases is determined by SOC in
Eq. (4), which shows the SOC effect plays an important role on the
ground-state phases of spin-orbit coupled bosonic system.

The ground-state phases of the spin-orbit coupled Bose-Hubbard
model of the Eq. (1) are obtained by using the DGMF method. Under
the mean-field decoupling approximation, the hopping and NN
interaction terms can be written as
\begin{equation}
\begin{split}
\hat{b}_{p,q}^{\dag\sigma}\hat{b}_{p',q'}^{\sigma}&=\langle\hat{b}_{p,q}^{\dag\sigma}\rangle\hat{b}_{p',q'}^{\sigma}+\hat{b}_{p,q}^{\dag\sigma}\langle\hat{b}_{p',q'}^{\sigma}\rangle-\langle\hat{b}_{p,q}^{\dag\sigma}\rangle\langle\hat{b}^{\sigma}_{p',q'}\rangle,\\
\hat{n}_{p,q}^{\sigma}\hat{n}_{p',q'}^{\sigma}&=\langle\hat{n}_{p,q}^{\sigma}\rangle\hat{n}_{p',q'}^{\sigma}+\hat{n}_{p,q}^{\sigma}\langle\hat{n}_{p',q'}^{\sigma}\rangle-\langle\hat{n}_{p,q}^{\sigma}\rangle\langle\hat{n}^{\sigma}_{p',q'}\rangle,
\end{split}
\end{equation}



% Figure environment removed

The many-body wave function of the ground state of the system is
given by
\begin{equation}
|\Psi\rangle=\prod_{p,q}|\psi\rangle_{p,q}=\prod_{p,q}\bigg(\sum^{n_{max}}_{n_{\uparrow},n_{\downarrow}}c^{n_{\uparrow},n_{\downarrow}}_{p,q}|n_{\uparrow},n_{\downarrow}\rangle_{p,q}\bigg)
\end{equation}
where $|\psi\rangle_{p,q}$ is the single site ground-state.
$|n_{\uparrow},n_{\downarrow}\rangle_{p,q}$ is the Fock state and
$c_{p,q}^{n_{\uparrow},n_{\downarrow}}$ is the the probability
amplitudes, which is normalized in our numerical simulations,
i.e., $\sum^{n_{max}}_{n_{\uparrow},n_{\downarrow}}
|c_{p,q}^{n_{\uparrow},n_{\downarrow}}|^{2}=1$. The truncation of
maximum number of bosons at each lattice site $n_{max}=6$ in the
numerical simulation. The SF order parameters of spin-$\sigma$
species at site $(p,q)$ are obtained as by using the above ansatz
\begin{equation}
\begin{split}
\Delta^{\uparrow}_{p,q}=\langle\Psi|\hat{b}^{\uparrow}_{p,q}|\Psi\rangle=\Sigma^{n_{max}}_{n_{\uparrow},n_{\downarrow}}\sqrt{n^{\uparrow}_{p,q}}c^{\ast
n_{\uparrow}-1,n_{\downarrow}}_{p,q}c^{
n_{\uparrow},n_{\downarrow}}_{p,q},\\
\Delta^{\downarrow}_{p,q}=\langle\Psi|\hat{b}^{\downarrow}_{p,q}|\Psi\rangle=\Sigma^{n_{max}}_{n_{\uparrow},n_{\downarrow}}\sqrt{n^{\downarrow}_{p,q}}c^{\ast
n_{\uparrow},n_{\downarrow}-1}_{p,q}c^{
n_{\uparrow},n_{\downarrow}}_{p,q},\\
\end{split}
\end{equation}
and the filling numbers are

\begin{equation}
\begin{split}
n^{\uparrow}_{p,q}=\langle\Psi|\hat{b}^{\dag\uparrow}_{p,q}\hat{b}^{\uparrow}_{p,q}|\Psi\rangle=\Sigma^{n_{max}}_{n_{\uparrow},n_{\downarrow}}n^{\uparrow}_{p,q}|c^{n_{\uparrow},n_{\downarrow}}_{p,q}|^{2},\\
n^{\downarrow}_{p,q}=\langle\Psi|\hat{b}^{\dag\downarrow}_{p,q}\hat{b}^{\downarrow}_{p,q}|\Psi\rangle=\Sigma^{n_{max}}_{n_{\uparrow},n_{\downarrow}}n^{\downarrow}_{p,q}|c^{n_{\uparrow},n_{\downarrow}}_{p,q}|^{2}.\\
\end{split}
\end{equation}
The $c_{p,q}^{n_{\uparrow},n_{\downarrow}}$ is complex with SOC,
therefore, the SF order parameters are complex numbers in general.
It can be rewritten in terms of the magnitude and phase, i.e.,
$\Delta^{\sigma}_{p,q}=|\Delta^{\sigma}_{p,q}|e^{i\theta^{\sigma}_{p,q}}$.
Since $U_{\uparrow\uparrow}=U_{\downarrow\downarrow}$ and
$\mu^{\uparrow}_{p,q}=\mu^{\downarrow}_{p,q}$, the SF order
parameters
$|\Delta^{\uparrow}_{p,q}|=|\Delta^{\downarrow}_{p,q}|$.





Minimization of the effective action
$\langle\Psi|i\frac{\partial}{\partial t}-\hat{H}|\Psi\rangle$
results in the equation of motion for
$c^{n_{\uparrow},n_{\downarrow}}_{p,q}$\cite{J. Zakrzewski 2005,C.
Trefzger 2011,A. Rapp 2013,Y. F. Song 2020,Y. J. Zhou 2020}
\begin{widetext}
\begin{equation}
\begin{split}
i\frac{dc^{n_{\uparrow},n_{\downarrow}}_{p,q}}{dt}&=-t
\big\{\bar{\Delta}^{\uparrow}_{p,q}\sqrt{n^{\uparrow}_{p,q}+1}c^{n_{\uparrow}+1,n_{\downarrow}}_{p,q}
+\bar{\Delta}^{\uparrow\ast}_{p,q}\sqrt{n^{\uparrow}_{p,q}}c^{n_{\uparrow}-1,n_{\downarrow}}_{p,q}
+\bar{\Delta}^{\downarrow}_{p,q}\sqrt{n^{\downarrow}_{p,q}+1}c^{n_{\uparrow},n_{\downarrow}+1}_{p,q}
+\bar{\Delta}^{\downarrow\ast}_{p,q}\sqrt{n^{\downarrow}_{p,q}}c^{n_{\uparrow},n_{\downarrow}-1}_{p,q}\big\}\\&
-\gamma
\big\{\bar{\Delta}^{\downarrow}_{p^{'},q}\sqrt{n^{\uparrow}_{p,q}+1}c^{n_{\uparrow}+1,n_{\downarrow}}_{p,q}
+\bar{\Delta}^{\downarrow\ast}_{p^{'},q}\sqrt{n^{\uparrow}_{p,q}}c^{
n_{\uparrow}-1,n_{\downarrow}}_{p,q}
-\bar{\Delta}^{\uparrow}_{p^{'},q}\sqrt{n^{\uparrow}_{p,q}+1}c^{n_{\uparrow}+1,n_{\downarrow}}_{p,q}
-\bar{\Delta}^{\uparrow\ast}_{p^{'},q}\sqrt{n^{\downarrow}_{p,q}}c^{
n_{\uparrow},n_{\downarrow}-1}_{p,q}\big\}\\&
 +i\gamma
\big\{\bar{\Delta}^{\uparrow}_{p,q^{'}}\sqrt{n^{\downarrow}_{p,q}+1}c^{n_{\uparrow},n_{\downarrow}+1}_{+,q}
-\bar{\Delta}^{\uparrow\ast}_{p,q^{'}}\sqrt{n^{\downarrow}_{p,q}}c^{
n_{\uparrow},n_{\downarrow}-1}_{p,q}
+\bar{\Delta}^{\downarrow}_{p,q^{'}}\sqrt{n^{\uparrow}_{p,q}+1}c^{n_{\uparrow}+1,n_{\downarrow}}_{p,q^{'}}
-\bar{\Delta}^{\downarrow\ast}_{p,q^{'}}\sqrt{n^{\uparrow}_{p,q}}c^{
n_{\uparrow}-1,n_{\downarrow}}_{p,q}\big\}\\&+
\big\{\sum_{\sigma}\big[\frac{U_{\sigma\sigma}}{2}n^{\sigma}_{p,q}(n^{\sigma}_{p,q}-1)+V_{\sigma\sigma}n^{\sigma}_{p,q}\bar{n}^{\sigma}_{p,q}\big]
+U_{\uparrow\downarrow}n^{\uparrow}_{p,q}n^{\downarrow}_{p,q}
+V_{\uparrow\downarrow}(n^{\uparrow}_{p,q}\bar{n}^{\downarrow}_{p,q}
+n^{\downarrow}_{p,q}\bar{n}^{\uparrow}_{p,q})-\mu \sum
n^{\sigma}_{p,q}\big\}c^{n_{\uparrow},n_{\downarrow}}_{p,q},
\end{split}
\end{equation}
\end{widetext}
where
$\bar{\Delta}^{\uparrow}_{p,q}=\Delta^{\uparrow}_{p+1,q}+\Delta^{\uparrow}_{p-1,q}+\Delta^{\uparrow}_{p,q+1}+\Delta^{\uparrow}_{p,q-1}$,
$\bar{\Delta}^{\uparrow}_{p^{'},q}=\Delta^{\uparrow}_{p+1,q}+\Delta^{\uparrow}_{p-1,q}$
and
$\bar{\Delta}^{\uparrow}_{p,q^{'}}=\Delta^{\uparrow}_{p,q+1}+\Delta^{\uparrow}_{p,q-1}$
sum over NN sites of site $(p,q)$. The system size $\Omega=L\times
L$ lattice sites with the periodic boundary conditions, here, we
choose $L=12$. The ground-state phases and phase transitions of
spin-orbit coupled extended Bose-Hubbard model can be obtained by
using the standard imaginary-time-evolution propagation\cite{W.
Bao 2002,W. Bao 2003,P. Bader 2013} in Eq. (9), i.e.,
$t\rightarrow-it$. In order to have universality, we choose the
random number as the initial Guztwiller wave function.


\section{MI-SF phase transitions in spin-orbit coupled Bose-Hubbard model}

% Figure environment removed
% Figure environment removed


We first discuss the effects of SOC on the ground-state phases and
phase transitions in the standard spin-orbit coupled Bose-Hubbard
model, i.e., $V=0$ and $V_{\uparrow\downarrow}=0$. Figure 1 shows
the phase diagrams in the $t/U-\mu/U$ plane for different values
of $\gamma$ with $U_{\uparrow,\downarrow}/U=0.8$ in (a) and
$U_{\uparrow,\downarrow}/U=1.2$ in (b). The MI phases are
characterized by MI($N$), where
$N=n_{\uparrow}+n_{\downarrow}\in\mathbb{N}$. Two quantum phases:
MI and SF phases, exhibited in the absence of SOC $\gamma=0$,
which are similar with the single-species Bose-Hubbard model
\cite{B. C. Sansone 2008}. The lobe sizes of MI($N\in 2n+1$) are
smaller than those MI($N\in 2n$) at $U_{\uparrow,\downarrow}<U$,
while the lobe sizes of MI($N$) shrink as $N$ increases at
$U_{\uparrow,\downarrow}>U$, as shown in Figs. 1(a1) and 1(b1).
The magnitudes of SF order parameters are uniform at each site,
while the phases $\theta_{p,q}$=arg$(\Delta_{p,q})$ are nonuniform
due to SOC. The phase-twisted SF (PT-SF) phase that phase varies
diagonally across the sites and phase-striped SF (PS-SF) phase
that phase exhibits stripelike patterns along the axis direction,
one can be seen in Figs. 4(a) and 4(b). The SF phases also can be
classified by using the spin-dependent momentum
$\langle\rho_{\uparrow\downarrow}(k)\rangle=\Omega^{-2}\sum_{A,B}\langle
\hat{b}^{\uparrow}_{A}\hat{b}^{\downarrow}_{B}\rangle
e^{i\textbf{k}\cdot(\textbf{r}_{A}-\textbf{r}_{B})}$\cite{A. Dutta
2019,K. Suthar 2021}, where $\textbf{r}_{A}$ ($\textbf{r}_{B}$) is
the location of $A$-th ($B$-th) site, site $A$ and site $B$ are
the NN sites. The spin-dependent momentum peak at
$\langle\rho_{\uparrow\downarrow}(-k_{0},-k_{0})\rangle$ or
$\langle\rho_{\uparrow\downarrow}(k_{0},k_{0})\rangle$ along the
diagonal direction in the PT-SF phase and
$\langle\rho_{\uparrow\downarrow}(k_{0},0)\rangle$ or
$\langle\rho_{\uparrow\downarrow}(-k_{0},0)\rangle$
($\langle\rho_{\uparrow\downarrow}(0,k_{0})\rangle$ or
$\langle\rho_{\uparrow\downarrow}(0,-k_{0})\rangle$) along the $x$
or $-x$ ($y$ or $-y$) direction in the PS-SF phase. It can be seen
that the quadruple degeneracy of $\textbf{Q}$ is spontaneously
broken by interaction, the PT-SF phase chooses the position of the
diagonal of the Brillouin zone and PS-SF phase chooses the $x-$ or
$y-$axis of k-space. The SOC shrinks the MI lobe size and it
vanishes as the SOC strength is increased beyond a critical value
$\gamma_{c}$. The phase transitions from the PT-SF phase to the
PS-SF phase, and to the zero momentum SF (ZM-SF) phase at
$U_{\uparrow\downarrow}/U=0.8$ and to the $z$-polarized
ferromagnetic SF (Z-SF) phase at $U_{\uparrow\downarrow}/U=1.2$
with the hopping strength increases.


% Figure environment removed

The critical hopping $t_{c}$ of MI-SF transition in the presence
of SOC can be given by the second-order perturbation theory
(details given in Appendix A),
\begin{equation}
\begin{split}
\frac{zt_{c}}{U}=\frac{1}{2}\bigg\{\frac{zt_{0}}{U}+\big[(\frac{zt_{0}}{U})^{2}-8(\frac{\gamma}{U})^{2}\big]^{\frac{1}{2}}\bigg\}
\end{split}
\end{equation}
where $t_{0}=t^{\uparrow}_{0}=t^{\downarrow}_{0}$ is the critical
hopping of MI-SF transition without SOC. For the MI($N\in 2n$)
phase, the occupy number $n_{\uparrow}=n_{\downarrow}$,
$\frac{1}{zt_{0}}=\frac{n_{\uparrow}+1}{Un_{\uparrow}-\mu+U_{\uparrow\downarrow}n_{\downarrow}}-
\frac{n_{\uparrow}}{U(n_{\uparrow}-1)-\mu+U_{\uparrow\downarrow}n_{\downarrow}}$
\cite{R. Bai 2020,G. H. Chen 2003}. For the MI($N\in 2n+1$) phase,
one atom at each site is chosen randomly from the two species, the
energies of system is degenerate for all the possible
combinations. The occupy number
$(n_{\uparrow},n_{\downarrow})=(\frac{N+1}{2},\frac{N-1}{2})$ or
$(\frac{N-1}{2},\frac{N+1}{2})$, therefore,
$\frac{1}{zt_{0}}=\frac{n_{\uparrow}+1}{Un_{\uparrow}-\mu+U_{\uparrow\downarrow}n_{\downarrow}}-
\frac{n_{\uparrow}}{U(n_{\uparrow}-1)-\mu+U_{\uparrow\downarrow}n_{\downarrow}}+\frac{n_{\downarrow}+1}{Un_{\downarrow}-\mu+U_{\uparrow\downarrow}n_{\uparrow}}-
\frac{n_{\downarrow}}{U(n_{\downarrow}-1)-\mu+U_{\uparrow\downarrow}n_{\uparrow}}$.
The phase boundaries (filled red circle lines) between MI-SF
phases are calculated by solving Eq. (10), we find that the
results agree well with the numerical simulation results, one can
be seen in Fig. 1.





The magnetic structures of the PT-SF and PS-SF phases are also
studied in Fig. 8(a) and (b), respectively. The spin texture is
defined by\cite{H. Y. Hui 2017}
$S_{\zeta}=\langle\Psi|\hat{a}^{\dag}\hat{\sigma}_{\zeta}\hat{a}|\Psi\rangle/|\Psi|^{2}$
($\zeta=x,y,z$), where
$\hat{a}^{\dag}=(\hat{b}_{\uparrow}^{\dag},\hat{b}_{\downarrow}^{\dag})$
and $\hat{\sigma}_{\zeta}$ is the Pauli matrix. The PT-SF and
PS-SF phases show the interesting spin configurations in the $x-y$
plane in Figs. 8(a) and 8(b), respectively. The spiral order are
exhibited in the PT-SF phase and stripe order in the PS-SF phase.
The spin texture structures are the same with the SF order phase
distributions in Figs. 4(a) and 4(b). The spiral order is the
spins having a spiral wave along the diagonal direction and stripe
order is the spins being separated by periodically spaced domain
walls along the axis direction. The values of $S_{z}$ are weak,
i.e., $S_{z}\in\{-0.12,0.12\}$ in PT-SF phase and $\{-0.34,0.04\}$
in PT-SF phase. The interplay between on-site interaction and SOC
gives rise to the distinct magnetic textures in SF phase.







\section{exotic SS phases in spin-orbit coupled extended Bose-Hubbard model}

The density translational symmetry of the system can be
spontaneously broken with the long-range NN interaction, the
quantum phases with periodic density modulations emerge, such as
the DW and SS phases. The quantum phases with exotic spin magnetic
and SF order phase structures are driven by SOC. Therefore, we
study the ground-state phases and phase transitions in spin-orbit
coupled extended Bose-Hubbard model.



% Figure environment removed

% Figure environment removed

% Figure environment removed

% Figure environment removed
\subsection{$V_{\uparrow\downarrow}=0$}

 We first discuss the effect of intraspecies NN interaction and SOC on the ground-sate phases, i.e.,
 $V_{\uparrow\downarrow}=0$. We plot the phase diagrams as functions of
 $t$ and $\mu$ for different $V$ and $\gamma$ at $U_{\uparrow\downarrow}/U=0.8$ in Fig. 2 and
$U_{\uparrow\downarrow}/U=1.2$ in Fig. 3. Due to the NN
interaction, the DW and MI phases are indicated by the NN lattice
sites occupancies $(n^{\uparrow}_{A},n^{\uparrow}_{B})$. These
phases have zero superfluid order parameter
$|\Delta^{\sigma}_{A}|=|\Delta^{\sigma}_{B}|=0$, and are
incompressible. The MI phase has integer commensurate occupy
$n^{\sigma}_{A}=n^{\sigma}_{B}\in \mathbb{N}$ while the DW phase
with integer occupy $n^{\sigma}_{A}\neq n^{\sigma}_{B}$. For the
DW phase, the relative occupy $\Delta
n_{A}=n^{\uparrow}_{A}-n^{\downarrow}_{A}=-\Delta n_{B}$, which
means that $n^{\uparrow}_{A}=n^{\downarrow}_{B}$ and
$n^{\downarrow}_{A}=n^{\uparrow}_{B}$ \cite{R. Bai 2020}. The
translational symmetry of single species is broken by intraspecies
NN interaction $V$, and both the spin-$\uparrow$ species and
spin-$\downarrow$ species have periodic density modulation. As a
result, a type of SS phase with checkerboard structure in single
species appears, and hence can be regraded as paired checkboard SS
(PCSS) phase. The density distribution of PCSS phase makes the
intraspecies NN interaction weak, for example,
$(n^{\uparrow}_{A},n^{\downarrow}_{A})=(0.1,1.1)$ and
$(n^{\uparrow}_{B},n^{\downarrow}_{B,q})=(1.1,0.1)$ in Fig. 4(c),
the intraspecies NN interaction term in Eq. (1) is weak that can
not enough to break the translational symmetry of total density.
The total density $N=n_{\uparrow}+n_{\downarrow}$ of PCSS phase is
homogeneous. When $\gamma=0$, the incompressible phases including
the DW(1,0), MI(1,1), DW(2,1) phase with $V/U=0.05$ at
$U_{\uparrow\downarrow}/U=0.8$ and in Fig. 2(a1). If the
interspecies on-site interaction or intraspecies NN interaction is
increased beyond a critical value $U_{\uparrow\downarrow}\gtrsim
U^{c}_{\uparrow\downarrow}$ or $V\gtrsim V^{c}$, only DW($N\in
\mathbb{N}$,0) phase exists, and the domain of the PCSS phase also
increases, one can be seen in Figs. 2(b1) and 3(a1). The SOC
driven the phase-twisted PCSS (PT-PCSS) and phase-striped PCSS
(PS-PCSS) phase, as shown in Figs. 4(c)-(e). In addition to the
PT-PCSS and PS-PCSS phases, the PT-SF phase or PS-SF phase is also
observed for weak intraspecies NN interaction $V/U\lesssim 0.12$.
Upon increasing $V$ further, the intraspecies NN interaction
 plays a prominent role on the ground-state phase. The phase variation of SF
 order are inhibited by intraspecies NN interaction at larger hopping
 strengths, the zero momentum PCSS (ZM-PCSS) phase (see Fig. 4(e)) with
 $\theta^{\sigma}_{p,q}=0$ occupies the most region, as shown in
 Figs. 2(c)-(d) and 3(c)-(d).





The magnetic structures of PT-SCSS phase and PS-SCSS phase are
shown in Figs. 8(d)-(e), respectively. The value of
$S_{z}\in\{-1,1\}$. The PT-PCSS phase shows the antiferromagnet
order along the $z$ axis (Z-AFM) that the neighboring spins
pointing to the opposite directions ($S_{z}=\pm 1$). The PS-PCSS
phase also show the antiferromagnet order structure, however, the
vectors form a certain angle to the $z$ axis, one can be seen in
Fig. 8(d).















\subsection{$V_{\uparrow\downarrow}\neq 0$}

The effect of the intraspecies NN interaction and SOC on the
ground-state phases of spin-orbit coupled extended BH model has
been discussed above. Two kinds of the PCSS phases, i.e., the
PT-PCSS and PS-PCSS phases with the periodic density modulation in
each species are found. We find that the intraspecies NN
interaction alone is not enough to break the translational
symmetry of total density. Here, we study the quantum phases of
spin-orbit coupled Bose-Hubbard model by adding the interspecies
NN interaction. For simplicity, we consider symmetric NN
interactions, i.e., $V_{\uparrow\downarrow}=V$.


The ground-state phase diagrams in the $t-\mu$ plane for different
$V_{\uparrow\downarrow}$ and $\gamma$ are shown in Fig. 5 with
$U_{\uparrow\downarrow}/U=0.8$
 and Fig. 6 with
$U_{\uparrow\downarrow}/U=1.2$. The DW and MI phases appear
alternately with $\mu$ increasing without SOC $\gamma/U=0$ at weak
$V_{\uparrow\downarrow}/U=0.05$, as shown in Figs. 5(a1) and
6(a1). The DW lobes are surrounded by a thin envelope of a new
kind of SS phase. It has the periodic density modulations in both
single species and total density. The total density exhibits the
lattice structure, we take this SS phase as the lattice SS (LSS)
phase. The SOC-driven the phase-twisted LSS (PT-LSS) phase and
phase-striped LSS (PS-LSS) phase, one can be seen in Figs. 7(d)
and 7(e). Two peaks of spin-dependent momentum at
$\langle\rho_{\uparrow\downarrow}(k_{0}, k_{0})\rangle$ and
$\langle\rho_{\uparrow\downarrow}(-k_{0}, -k_{0})\rangle$ with
equal heights along the diagonal direction in the PT-LSS phase and
$\langle\rho_{\uparrow\downarrow}(k_{0}, 0)\rangle$ and
$\langle\rho_{\uparrow\downarrow}(-k_{0}, 0)\rangle$
($\langle\rho_{\uparrow\downarrow}(0,k_{0})\rangle$ and
$\langle\rho_{\uparrow\downarrow}(0,-k_{0})\rangle$) along the $x$
($y$) direction in the PS-LSS phase. We find an interesting
phenomenon is that the SS phase around the DW(3,2) lobe is not the
LSS phase only at $U_{\uparrow\downarrow}/U=0.8$ with weak SOC
$\gamma/U\lesssim 0.024$, is a phase with SS in one spin species
but insulating in the other, as shown in Fig. 5(a1). The total
density of this phase also shows the lattice structure, we take it
as the LSS-I phase, the density and spin-dependent momentum of
phase-twisted LSS-I (PT-LSS-I) phase are shown in Fig. 7(a). The
DW and MI lobes shrunk with SOC increases, and MI phase survives
at larger $\gamma/U=0.04$. The reason for the existence of the MI
phase with larger SOC is that the energy consumption of the MI
phase is larger than that of the DW phase due to repulsion between
two species coexisting on the same lattice site at finite
$U_{\uparrow\downarrow}$. The combination of the NN interactions
and SOC displays completely different quantum phases in the
miscible and immiscible domains at larger hopping strength $t$.
The PT-SF and PS-SF phases emerge in the immiscible domain in Fig.
5 with $U_{\uparrow\downarrow}/U=1.2$ while the phase-twisted
paired stripe SS (PT-PSSS) and phase-striped paired stripe
(PS-PSSS) phases in the immiscible domain in Fig. 6 with
$U_{\uparrow\downarrow}/U=0.8$. For the paired stripe SS (PSSS)
phases, each species occupies opposite wave vectors of the four
states of $Q$, which leads to the stripe structures in single
species density and homogenous in total density. Two peaks of
spin-dependent momentum are found in PT-PSSS and PS-PSSS phases,
one can be seen in Figs. 7(b) and 7(c). For larger
$V_{\uparrow\downarrow}$, the PT-SF and PS-SF (PT-PSSS and
PS-PSSS) phases are replaced by the PT-LSS, PS-LSS and
zero-momentum LSS (ZM-LSS) phases. Similar with the case of $V$,
the interspecies NN interaction also inhibits the phase variation
of SF order of LSS phase, ZM-LSS phase (see Fig. 7(f)) occupies
the most region, as shown in Figs. 5(d) and 6(d).





The spin textures of the PT-PSSS, PS-PSSS, PT-LSS, and PS-LSS
phases are respectively shown in Figs. 8(e)-(h). The PT-SS phases
favor the spiral orders and PS-SS phases are the stripe orders.
The NN interactions and SOC play an important role on the spiral
order of the spatial periods. The spiral order of the PT-PSSS
phase has spatial periods 10 sites while PT-LSS phase has 5 sites,
which can be respectively denoted as spiral-10 and spiral-5
orders, as show in Figs. 8(e) and 8(g).


The interplay between the interactions and SOC induces a variety
of quantum phases with magnetic orderings. The intraspecies NN
interaction and SOC induces two finite-momentum PCSS phases, i.e.,
the PT-PCSS and PS-PCSS phases only with density periodic
modulations in each species. When considering interspecies NN
interaction, the PT-LSS and PS-LSS phases with density periodic
modulations in both each species and total density emerge. The
appearance of some ground-state phases depending on interspecies
on-site interaction. The LSS-I phase with SS in one spin species
but insulating in the other that exists in the miscible domain,
while the PSSS phase with stripe structures in each spin species
in the immiscible domain. The magnetic textures such as
antiferromagnetic, spiral and stripe orders are shown in these SS
phases.


The relation between the critical hopping $t_{c}$ and SOC $\gamma$
of MI-SS or DW-SS phase transition of spin-orbit coupled extended
Bose-Hubbard model can be obtained by using the perturbative
analysis (details are given in Appendix B),

\begin{equation}
\begin{split}
z^{2}t_{c}^{2}+\gamma^{2}&=(z^{4}t_{c}^{4}+4\gamma^{4})J^{\uparrow}_{0A}J^{\downarrow}_{0A}+2z^{2}t_{c}^{2}\gamma^{2}[(J^{\uparrow}_{0A})^{2}+(J^{\downarrow}_{0A})^{2}]
\\&+\gamma(z^{2}t_{c}^{2}-2\gamma^{2})(J^{\uparrow}_{0A}-J^{\downarrow}_{0A}).
\end{split}
\end{equation}
where
$J^{\uparrow}_{0A}=\frac{1}{t^{\uparrow}_{0A}}=\frac{n^{\uparrow}_{A}+1}{Un^{\uparrow}_{A}+U_{\uparrow\downarrow}n^{\downarrow}_{A}+zVn^{\uparrow}_{B}+zV_{\uparrow\downarrow}n^{\downarrow}_{B}-\mu}
-\frac{n^{\uparrow}_{A}}{U(n^{\uparrow}_{A}-1)+U_{\uparrow\downarrow}n^{\downarrow}_{A}+zVn^{\uparrow}_{B}+zV_{\uparrow\downarrow}n^{\downarrow}_{B}-\mu}$
and
$J^{\uparrow}_{0B}=\frac{1}{t^{\uparrow}_{0B}}=\frac{n^{\uparrow}_{B}+1}{Un^{\uparrow}_{B}+U_{\uparrow\downarrow}n^{\downarrow}_{B}+zVn^{\uparrow}_{A}+zV_{\uparrow\downarrow}n^{\downarrow}_{A}-\mu}
-\frac{n^{\uparrow}_{B}}{U(n^{\uparrow}_{B}-1)+U_{\uparrow\downarrow}n^{\downarrow}_{B}+zVn^{\uparrow}_{A}+zV_{\uparrow\downarrow}n^{\downarrow}_{A}-\mu}$,
$t^{\uparrow}_{0A}$ and $t^{\uparrow}_{0B}$ are critical hoppings
of MI-SS or DW-SS phase transition in two-species extended
Bose-Hubbard model of spin-$\sigma$ species at sites $A$ and $B$,
respectively. When $\gamma=0$, the Eq. (11) becomes the Eq. (16)
of Ref \cite{R. Bai 2020}.














\section{summary}

We have investigated the quantum phases and phase transitions of
spin-orbit coupled bose gases in a 2D extended Bose-Hubbard model
by using DGMF method. The competition between SOC and interactions
creates rich ground-state diagrams with SS phases exhibiting the
phase modulation or magnetic ordering. The combined effect of
intraspecies NN interaction and SOC results in the PT-PCSS and
PS-PCSS phases. The PCSS phase only has the periodic density
modulation in each species and uniform in total density. The
introduction of interspecies NN interaction enriches the quantum
phases of the system. The PT-LSS and PS-LSS phases with periodic
density modulation in both each species and total density are
preferred. We find that the appearance of some ground-state phases
depending on interspecies on-site interaction. The LSS-I phase
with SS in one spin species but insulating in the other that
exists in the miscible domain, while the PSSS phase with stripe
structures in each spin species in the immiscible domain. For the
PT (PS)-PSSS phase, each species occupies opposite wave vectors of
the four states of single particle energy spectrum, it shows the
stripe structures in each species and uniform in total density.
Finally, to further characterize each phase, we discuss their
spin-dependent momentum distributions and spin-texture structures.
The magnetic textures such as Z-AFM, spiral and stripe orders are
shown in these SS phases. The spiral orders also can by classified
by the spatial periods, including the spiral-10 and spiral-5
orders. The results here in could help in the observe for these
magnetic supersolid phases in ultracold atomic experiments with
nearest-neighbor interaction and spin-orbit coupling in optical
lattice.








This work is supported by the Scientific and Technological
Research Program of the Education Department of Hubei province
under Grant Nos. D20222502, the NSF of Hubei Province of China
under Grant No. 2022CFB499, the NSF of China under Grant No.
11904242 and the Talent project of Hubei Normal University under
Grant No. HS2022RC033.



\section*{APPENDIX: PERTURBATIVE TREATMENT}


 \subsection*{A: Spin-orbit coupled Bose-Hubbard model}


 We first discuss the
spin-orbit coupled Bose-Hubbard model, the hopping and SOC terms
in the single-site Hamiltonian regard as the perturbation
Hamiltonian and the on-site interaction terms with the chemical
potential as the unperturbed Hamiltonian. Therefore, the energy of
the ground state of the unperturbed Hamiltonian is given as
\begin{subequations}
\begin{align}
 E_{n^{\uparrow}_{p,q},n^{\downarrow}_{p,q}}^{a(0)}&=\frac{U}{2}\sum_{\sigma}n^{\sigma}_{p,q}(n^{\sigma}_{p,q}-1)+U_{\uparrow\downarrow}n^{\uparrow}_{p,q}n^{\downarrow}_{p,q}-\mu(n^{\uparrow}_{p,q}+n^{\downarrow}_{p,q})\tag{a1},
 \end{align}
\end{subequations}
The second-order perturbed ground-state energy can be written as
\begin{widetext}
\begin{subequations}
\begin{align}
E_{n^{\uparrow}_{p,q},n^{\downarrow}_{p,q}}^{a(2)} &
=\sum_{m^{\uparrow},m^{\downarrow}\neq
n^{\uparrow},n^{\downarrow}}\frac{|_{p,q}\langle
m^{\uparrow},m^{\downarrow}|\hat{T}^{a}_{p,q}|n^{\uparrow},n^{\downarrow}\rangle_{p,q}|^{2}}{E_{n^{\uparrow}_{p,q},n^{\downarrow}_{p,q}}^{(0)}-E_{m^{\uparrow}_{p,q},m^{\downarrow}_{p,q}}^{(0)}}
=z^{2}t^{2}|\Delta_{p,q}^{\uparrow}|^{2}J^{\uparrow}_{0}+z^{2}t^{2}|\Delta_{p,q}^{\downarrow}|^{2}J^{\downarrow}_{0}+2\gamma^{2}|\Delta_{p,q}^{\uparrow}|^{2}J^{\downarrow}_{0}+2\gamma^{2}|\Delta_{p,q}^{\downarrow}|^{2}J^{\uparrow}_{0}
\notag\\&+zt(|\Delta_{p,q}^{\uparrow}|^{2}+|\Delta_{p,q}^{\downarrow}|^{2})
= \Phi^{a\dag}
\begin{pmatrix}
z^{2}t^{2}J^{\uparrow}_{0}+2\gamma^{2}J^{\downarrow}_{0}+zt & 0 \\
0 &z^{2} t^{2}J^{\downarrow}_{0}+2\gamma^{2}J^{\uparrow}_{0}+zt
\end{pmatrix}
\Phi^{a}=\Phi^{a\dag} \mathcal{A}^{a}\Phi^{a}
=\lambda_{1}|\Delta^{\uparrow}_{p,q}|^{2}+\lambda_{2}|\Delta^{\downarrow}_{p,q}|^{2}\tag{a2}.
 \end{align}
\end{subequations}
\end{widetext}
where
$\Phi^{a\dag}=(\Delta^{\uparrow\dag}_{p,q},\Delta^{\downarrow\dag}_{p,q})$.
The perturbation Hamiltonian

\begin{subequations}
\begin{align}
\hat{T}^{a}_{p,q}&=-t\sum_{\sigma}\big[\bar{\Delta}^{\sigma}_{p,q}(\hat{b}^{\dag\sigma}_{p,q}+\hat{b}^{\sigma}_{p,q})-|\Delta^{\sigma}_{p,q}|^{2}
\big]\notag\\&
+\gamma\big[\bar{\Delta}^{\uparrow}_{p^{'},q}(\hat{b}^{\dag\downarrow}_{p,q}+\hat{b}^{\downarrow}_{p,q})-\bar{\Delta}^{\downarrow}_{p^{'},q}(\hat{b}^{\dag\uparrow}_{p,q}+\hat{b}^{\uparrow}_{p,q})\big]\notag\\&
+i\gamma\big[\bar{\Delta}^{\uparrow}_{p,q^{'}}(\hat{b}^{\dag\downarrow}_{p,q}-\hat{b}^{\downarrow}_{p,q})+\bar{\Delta}^{\downarrow}_{p,q^{'}}(\hat{b}^{\dag\uparrow}_{p,q}-\hat{b}^{\uparrow}_{p,q})\big]\tag{a3},
 \end{align}
\end{subequations}
where
$\bar{\Delta}^{\sigma}_{p,q}=\Delta^{\sigma}_{p-1,q}+\Delta^{\sigma}_{p+1,q}+\Delta^{\sigma}_{p,q-1}+\Delta^{\sigma}_{p,q+1}=z\Delta^{\sigma}_{p,q}$,
$\bar{\Delta}^{\sigma}_{p^{'},q}=\Delta^{\sigma}_{p-1,q}+\Delta^{\sigma}_{p+1,q}$
and
$\bar{\Delta}^{\sigma}_{p,q^{'}}=\Delta^{\sigma}_{p,q-1}+\Delta^{\sigma}_{p,q+1}$.
$\lambda_{1}$ and $\lambda_{2}$ are the eigenvalues of matrix
$\mathcal{A}$. The parameter
$J^{\sigma}_{0}=\frac{1}{t^{\sigma}_{0}}$, where $t^{\sigma}_{0}$
is the critical hopping of MI-SF transition in the absence of SOC
of spin-$\sigma$ species. For the MI phase
$n^{\uparrow}=n^{\downarrow}$, the boundaries
$t^{\uparrow}_{0}=t^{\downarrow}_{0}$. If we want to obtain the
ground-sate phases, we should
min$\{E_{n^{\uparrow}_{p,q},n^{\downarrow}_{p,q}}^{(2)}\}$, i.e.,
$\frac{\partial
E_{n^{\uparrow}_{p,q},n^{\downarrow}_{p,q}}^{(2)}}{\partial
\Delta^{\uparrow}_{p,q}}=0$ and $\frac{\partial
E_{n^{\uparrow}_{p,q},n^{\downarrow}_{p,q}}^{(2)}}{\partial
\Delta^{\downarrow}_{p,q}}=0$. Therefore, the eigenvalues
$\lambda_{1}=\lambda_{2}=0$.


\begin{subequations}
\begin{align}
\lambda_{1}&=z^{2}t^{2}J^{\uparrow}_{0}+2\gamma^{2}J^{\uparrow}_{0}+zt=(z^{2}t^{2}+2\gamma^{2})J^{\uparrow}_{0}+zt\notag\\&
=DJ^{\uparrow}_{0}-1=0, \tag{a5}
 \end{align}
\end{subequations}
where $D=\frac{z^{2}t^{2}+2\gamma^{2}}{zt}$, thus,
\begin{subequations}
\begin{align}
&\mu^{2}+\big[U-2Un_{p,q}^{\uparrow}-2U_{\uparrow\downarrow}n_{p,q}^{\downarrow}+D\big]\mu+Un_{p,q}^{\uparrow2}-U^{2}n_{p,q}^{\uparrow}\notag\\&
+UU_{\uparrow\downarrow}n_{p,q}^{\uparrow}n_{p,q}^{\downarrow}+U_{\uparrow\downarrow}n_{p,q}^{\downarrow2}+D(U-U_{\uparrow\downarrow}n_{p,q}^{\downarrow})=0\tag{a6}.
 \end{align}
\end{subequations}
We obtain
\begin{subequations}
\begin{align}
\mu^{\uparrow}_{p,q\pm}&=\frac{1}{2}\bigg\{U(2n_{p,q}^{\uparrow}-1)+2U_{\uparrow\downarrow}n_{p,q}^{\downarrow}-D\notag\\&\pm\big[U^{2}-2DU(2n_{p,q}^{\uparrow}+1)+D^{2}\big]^{\frac{1}{2}}\bigg\},\notag\\
\mu^{\downarrow}_{p,q\pm}&=\frac{1}{2}\bigg\{U(2n_{p,q}^{\downarrow}-1)+2U_{\uparrow\downarrow}n_{p,q}^{\uparrow}-D\notag\\&\pm\big[U^{2}-2DU(2n_{p,q}^{\downarrow}+1)+D^{2}\big]^{\frac{1}{2}}\bigg\}.\tag{a7}
 \end{align}
\end{subequations}
Here $\mu^{\uparrow}_{p,q}=\mu^{\downarrow}_{p,q}$. The critical
condition for the MI-SF transition of each species is when the
terms under the square root in Eq. (a5) vanish or when
$\mu^{\sigma}_{p,q-}=\mu^{\sigma}_{p,q+}$. We yield the critical
values of the spin-orbit coupled Bose-Hubbard model as

\begin{subequations}
\begin{align}
\frac{zt_{c}}{U}=\frac{1}{2}\bigg\{\frac{zt_{0}}{U}+\big[(\frac{zt_{0}}{U})^{2}-8(\frac{\gamma}{U})^{2}\big]^{\frac{1}{2}}\bigg\}.\tag{a8}
 \end{align}
\end{subequations}

 \subsection*{B: Spin-orbit coupled extended Bose-Hubbard model}
For the spin-orbit coupled extended Bose-Hubbard model, the
hopping and SOC terms in the single-site Hamiltonian are also the
perturbation Hamiltonian and the interactions (on-site interaction
and NN interaction) with the chemical potential are the
unperturbed Hamiltonian. The energy of the ground state of the
unperturbed Hamiltonian is given as
\begin{subequations}
\begin{align}
 E_{n^{\uparrow}_{A},n^{\downarrow}_{A}}^{b(0)}&=\sum_{\sigma}\big[\frac{U}{2}n^{\sigma}_{A}(n^{\sigma}_{A}-1)+Vn^{\sigma}_{B}n^{\sigma}_{A}\big]\notag\\&
 +U_{\uparrow\downarrow}n^{\uparrow}_{A}n^{\downarrow}_{A}+V_{\uparrow\downarrow}n^{\sigma}_{B}n^{\sigma^{'}}_{A}
 -\mu(n^{\uparrow}_{A}+n^{\downarrow}_{A})\tag{b1},
 \end{align}
\end{subequations}

The second-order perturbed ground-state energy is
\begin{widetext}
\begin{subequations}
\begin{align}
E_{n^{\uparrow}_{A},n^{\downarrow}_{A}}^{b(2)} &
=\sum_{m^{\uparrow},m^{\downarrow}\neq
n^{\uparrow},n^{\downarrow}}\frac{|_{A}\langle
m^{\uparrow},m^{\downarrow}|\hat{T}^{b}_{A}|n^{\uparrow},n^{\downarrow}\rangle_{A}|^{2}}{E_{n^{\uparrow}_{A},n^{\downarrow}_{A}}^{(0)}-E_{m^{\uparrow}_{A},m^{\downarrow}_{A}}^{(0)}}
=z^{2}t^{2}(|\Delta_{A}^{\uparrow}|^{2}J^{\uparrow}_{0B}+|\Delta_{A}^{\downarrow}|^{2}J^{\downarrow}_{0B}+|\Delta_{B}^{\uparrow}|^{2}J^{\uparrow}_{0A}+|\Delta_{B}^{\downarrow}|^{2}J^{\downarrow}_{0A})\notag\\&+2zt(\Delta_{A}^{\uparrow}\Delta_{B}^{\uparrow}+\Delta_{A}^{\downarrow}\Delta_{B}^{\downarrow})
+2\gamma^{2}(|\Delta_{A}^{\uparrow}|^{2}J^{\downarrow}_{0B}+|\Delta_{A}^{\downarrow}|^{2}J^{\uparrow}_{0B}+|\Delta_{B}^{\uparrow}|^{2}J^{\downarrow}_{0A}+|\Delta_{B}^{\downarrow}|^{2}J^{\uparrow}_{0A})+2\gamma(\Delta_{A}^{\uparrow}\Delta_{B}^{\downarrow}-\Delta_{B}^{\uparrow}\Delta_{A}^{\downarrow})
\notag\\& = \Phi^{b\dag}
\begin{pmatrix}
z^{2}t^{2}J^{\uparrow}_{0B}+2\gamma^{2}J^{\downarrow}_{0B} & 0 & zt & \gamma \\
0 & z^{2}t^{2}J^{\downarrow}_{0B}+2\gamma^{2}J^{\uparrow}_{0B} & -\gamma & zt \\
zt & -\gamma & z^{2}t^{2}J^{\uparrow}_{0A}+2\gamma^{2}J^{\downarrow}_{0A} & 0 \\
\gamma & zt & 0 &
z^{2}t^{2}J^{\downarrow}_{0A}+2\gamma^{2}J^{\uparrow}_{0A}
\end{pmatrix}
\Phi^{b}= \Phi^{b\dag} \mathcal{A}^{b}\Phi^{b}
\notag\\&=\lambda_{1}|\Delta^{\uparrow}_{A}|^{2}+\lambda_{2}|\Delta^{\downarrow}_{A}|^{2}+\lambda_{3}|\Delta^{\uparrow}_{B}|^{2}+\lambda_{4}|\Delta^{\downarrow}_{B}|^{2}\tag{b2}.
 \end{align}
\end{subequations}
\end{widetext}
where lattice sites $A$ and $B$ are the NN site, i.e., site
$A=(p,q)$ and site  $B=(p\pm1,q)$ or $(p,q\pm1)$ site and
$\Phi^{b\dag}=(\Delta^{\uparrow\dag}_{A},\Delta^{\downarrow\dag}_{A},\Delta^{\uparrow\dag}_{B},\Delta^{\downarrow\dag}_{B})$.
The perturbation Hamiltonian
\begin{widetext}
\begin{subequations}
\begin{align}
\hat{T}^{b}_{A}&=-zt\big\{\sum_{\sigma}\big[\Delta^{\sigma}_{A}(\hat{b}^{\sigma\dag}_{A}+\hat{b}^{\sigma}_{A})+\Delta^{\sigma}_{B}(\hat{b}^{\sigma\dag}_{B}+\hat{b}^{\sigma}_{B})
\big]-2(\Delta^{\uparrow}_{A}\Delta^{\uparrow}_{B}+\Delta^{\downarrow}_{A}\Delta^{\downarrow}_{B})\big\}
-\gamma\big[\Delta^{\uparrow}_{A}(\hat{b}^{\downarrow\dag}_{B}+\hat{b}^{\downarrow}_{B})+\Delta^{\downarrow}_{B}(\hat{b}^{\uparrow\dag}_{A}+\hat{b}^{\uparrow}_{A})-2\Delta^{\uparrow}_{A}\Delta^{\downarrow}_{B}\big]
 \notag\\&
 +\gamma\big[\Delta^{\downarrow}_{A}(\hat{b}^{\uparrow\dag}_{B}+\hat{b}^{\uparrow}_{B})+\Delta^{\uparrow}_{B}(\hat{b}^{\downarrow\dag}_{A}+\hat{b}^{\downarrow}_{A})-2\Delta^{\downarrow}_{A}\Delta^{\uparrow}_{B}\big]
+i\gamma\big[\Delta^{\uparrow}_{A}(\hat{b}^{\downarrow\dag}_{B}+\hat{b}^{\downarrow}_{B})+\Delta^{\downarrow}_{B}(\hat{b}^{\uparrow\dag}_{A}+\hat{b}^{\uparrow}_{A})\big]
+i\gamma\big[\Delta^{\downarrow}_{A}(\hat{b}^{\uparrow\dag}_{B}+\hat{b}^{\uparrow}_{B})+\Delta^{\uparrow}_{B}(\hat{b}^{\downarrow\dag}_{A}+\hat{b}^{\downarrow}_{A})\big]\tag{b3}.
 \end{align}
\end{subequations}
\end{widetext}

In the spin-orbit coupled extended Bose-Hubbard model, the MI and
DW phases are existed. The occupation
$n^{\uparrow}_{A}=n^{\downarrow}_{B}$ and
$n^{\downarrow}_{A}=n^{\uparrow}_{B}$ in the MI and DW phases,
which results the $J^{\uparrow}_{0A}=J^{\downarrow}_{0B}$ and
$J^{\downarrow}_{0A}=J^{\uparrow}_{0B}$. The eigenvalues of matrix
$\mathcal{A}^{b}$ are
\begin{widetext}
\begin{subequations}
\begin{align}
\lambda_{\pm}&=\frac{1}{2}\bigg\{(z^{2}t^{2}+2\gamma^{2})(J^{\uparrow}_{0A}+J^{\downarrow}_{0A})\pm\big\{\big[(z^{2}t^{2}+2\gamma^{2})(J^{\uparrow}_{0A}+J^{\downarrow}_{0A})\big]^{2}-4\big[-\gamma^{2}+(z^{2}\gamma
t^{2}-2\gamma^{3})(J^{\uparrow}_{0A}-J^{\downarrow}_{0A})\notag\\&+(4\gamma^{4}+z^{4}t^{4})J^{\uparrow}_{0A}J^{\downarrow}_{0A}
-z^{2}t^{2}+2z^{2}\gamma^{2}t^{2}((J^{\uparrow}_{0A})^{2}+(J^{\downarrow}_{0A})^{2})\big]\big\}^{\frac{1}{2}}\bigg\}\tag{b4}.
 \end{align}
\end{subequations}
\end{widetext}
The parameters are
\begin{widetext}
\begin{subequations}
\begin{align}
J^{\uparrow}_{0A}=\frac{1}{t^{\uparrow}_{0A}}&=\frac{n^{\uparrow}_{A}+1}{Un^{\uparrow}_{A}+U_{\uparrow\downarrow}n^{\downarrow}_{A}+zVn^{\uparrow}_{B}+zV_{\uparrow\downarrow}n^{\downarrow}_{B}-\mu}
-\frac{n^{\uparrow}_{A}}{U(n^{\uparrow}_{A}-1)+U_{\uparrow\downarrow}n^{\downarrow}_{A}+zVn^{\uparrow}_{B}+zV_{\uparrow\downarrow}n^{\downarrow}_{B}-\mu},\notag\\
J^{\uparrow}_{0B}=\frac{1}{t^{\uparrow}_{0B}}&=\frac{n^{\uparrow}_{B}+1}{Un^{\uparrow}_{B}+U_{\uparrow\downarrow}n^{\downarrow}_{B}+zVn^{\uparrow}_{A}+zV_{\uparrow\downarrow}n^{\downarrow}_{A}-\mu}
-\frac{n^{\uparrow}_{B}}{U(n^{\uparrow}_{B}-1)+U_{\uparrow\downarrow}n^{\downarrow}_{B}+zVn^{\uparrow}_{A}+zV_{\uparrow\downarrow}n^{\downarrow}_{A}-\mu},\tag{b5}
\end{align}
\end{subequations}
\end{widetext}
where $t^{\uparrow}_{0A}$ and $t^{\uparrow}_{0B}$ are critical
hoppings of MI-SS or DW-SS transition in the present of NN
interaction of spin-$\sigma$ species at sites $A$ and $B$,
respectively.


The ground-sate phases can be obtained by minimizing
$E_{n^{\uparrow}_{A},n^{\downarrow}_{A}}^{b(2)}$, i.e.,
$\frac{\partial
E_{n^{\uparrow}_{A},n^{\downarrow}_{A}}^{(2)}}{\partial
\Delta^{\uparrow}_{A}}=\frac{\partial
E_{n^{\uparrow}_{A},n^{\downarrow}_{A}}^{(2)}}{\partial
\Delta^{\uparrow}_{B}}=\frac{\partial
E_{n^{\uparrow}_{A},n^{\downarrow}_{A}}^{(2)}}{\partial
\Delta^{\downarrow}_{A}}=\frac{\partial
E_{n^{\uparrow}_{A},n^{\downarrow}_{A}}^{(2)}}{\partial
\Delta^{\downarrow}_{B}}=0$. Therefore, the critical hopping
$t_{c}$ and SOC $\gamma$ of the spin-orbit coupled Bose-Hubbard
model satisfy the following relation
\begin{subequations}
\begin{align}
z^{2}t_{c}^{2}+\gamma^{2}&=(z^{4}t_{c}^{4}+4\gamma^{4})J^{\uparrow}_{0A}J^{\downarrow}_{0A}+2z^{2}t_{c}^{2}\gamma^{2}[(J^{\uparrow}_{0A})^{2}+(J^{\downarrow}_{0A})^{2}]
\notag\\&+\gamma(z^{2}t_{c}^{2}-2\gamma^{2})(J^{\uparrow}_{0A}-J^{\downarrow}_{0A})\tag{b6}.
\end{align}
\end{subequations}






\begin{thebibliography}{49}




\bibitem{M. P. A. Fisher 1989} M. P. A. Fisher,P. B. Weichman,G. Grinstein, and D. S. Fisher, Phys. Rev. B \textbf{40}, 546 (1989).
\bibitem{K. Sheshadri 1993} K. Sheshadri, H. R. Krishnamurthy, R. Pandit, and T. V. Ramakrishnan, Europhys. Lett. \textbf{22}, 257 (1993).

\bibitem{D. Jaksch 1998} D. Jaksch, C. Bruder, J. I. Cirac, C. W. Gardiner,and P.Zoller, Phys. Rev.Lett. \textbf{81}, 3108 (1998).
\bibitem{S. Sachdev 1999} S. Sachdev, Quantum Phase Transitions (Cambridge University press,
Cambridge, England, 1999).
\bibitem{M. Greiner 2002} M. Greiner, O. Mandel, T. Esslinger, T. W. H\"{a}nsch,and I.Bloch,  Nature (London) \textbf{415}, 39 (2002).
\bibitem{C. Orzel 2001} C. Orzel, A. K. Tuchman, M. L. Fenselau, M. Yasuda, and M. A.
Kasevich, Science \textbf{291}, 2386 (2001).



\bibitem{I. Bloch 2008} I. Bloch, J. Dalibard, and W. Zwerger, Rev. Mod. Phys.\textbf{80}, 885-964 (2008).



\bibitem{J. K. Freericks 1994} J. K. Freericks, and H. Monien, Europhys. Lett. \textbf{26}, 545 (1994).
\bibitem{T. Stoferle 2004} T. St\"{o}ferle, H. Moritz, C. Schori, M. K\"{o}hl, and T.Esslinger, Phys. Rev. Lett. \textbf{92}, 130403 (2004).
\bibitem{S. Folling 2006} S. F\"{o}lling, A. Widera, T. M\"{u}ller, F. Gerbier, and I.Bloch, Phys. Rev. Lett. \textbf{97}, 060403 (2006).
\bibitem{I. B. Spielman 2007} I. B. Spielman, W. D. Phillips, and J. V. Porto, Phys. Rev. Lett. \textbf{98}, 080404 (2007).
\bibitem{B. C. Sansone 2007} B. C. Sansone, N. V. Prokofev, and B. V. Svistunov, Phys.Rev. B \textbf{75}, 134302 (2007).
\bibitem{P. Sengupta 2007} P. Sengupta and S. Haas,Phys. Rev. Lett. \textbf{99}, 050403 (2007).
\bibitem{M. Iskin 2011} M. Iskin, Phys. Rev. A \textbf{83}, 051606(R) (2011).
\bibitem{X. B. Zhang 2012} X. B. Zhang, C. L.  Hung, S. K. Tung, C. Chin, Science \textbf{335}, 1070 (2012).
\bibitem{T. Ohgoe 2012} T. Ohgoe, T. Suzuki, and N. Kawashima, Phys. Rev. B \textbf{86}, 054520
(2012).
\bibitem{H. M. Deng 2015} H. M. Deng, H. Dai, J. H. Huang, X. Z. Qin, J. Xu, H. H. Zhong, C.
S. He, and C. H. Lee, Phys. Rev. A \textbf{92}, 023618 (2015).
\bibitem{D. S. Luhmann 2016} D. S. L\"{u}hmann, Phys. Rev. A \textbf{94},
011603(R) (2016).
\bibitem{B. Gardas 2017} B. Gardas, J. Dziarmaga, and W. H. Zurek, Phys. Rev. B \textbf{95}, 104306
(2017).
\bibitem{O. Mansikkamaki 2021} O. Mansikkam\"{a}ki, S. Laine, and M.
Silveri, Phys. Rev. B \textbf{103}, L220202 (2021).
\bibitem{P. Zechmann 2023} P. Zechmann, E. Altman, M. Knap, and J. Feldmeier, Phys. Rev. B
\textbf{107}, 195131 (2023).




\bibitem{E. Altman 2003} E. Altman, W. Hofstetter, E. Demler and M. D. Lukin, New J. Phys.
\textbf{5}, 113 (2003).
\bibitem{A. Kuklov 2004} A. Kuklov, N. Prokofev and B. Svistunov, Phys. Rev. Lett. \textbf{92}, 050402
(2004).
\bibitem{A. B. Kuklov 2003} A. B. Kuklov and B. V. Svistunov, Phys. Rev. Lett. \textbf{90}, 100401 (2003).
\bibitem{A. Isacsson 2005} A. Isacsson, M. C. Cha, K. Sengupta and S. M. Girvin, Phys. Rev. B \textbf{72}, 184507 (2005).
\bibitem{A. Hubener 2009} A. Hubener, M. Snoek and W. Hofstetter, Phys. Rev. B \textbf{80}, 245109 (2009).
\bibitem{A. Hu 2009} A. Hu, L. Mathey, I. Danshita, E. Tiesinga, C. J. Williams and C. W. Clark, Phys. Rev. A \textbf{80}, 023619 (2009).
\bibitem{J. Pietraszewicz 2012} J. Pietraszewicz, T. Sowi\'{n}ski, M. Brewczyk, J. Zakrzewski, M.
Lewenstein, and M. Gajda, Phys. Rev. A \textbf{85}, 053638 (2012).
\bibitem{J. M. Zhang 2012} J. M. Zhang, C. Shen, and W. M. Liu, Phys. Rev. A \textbf{85}, 013637 (2012).
\bibitem{W. Wang 2014}  W. Wang, V. Penna, and B. C. Sansone, Phys. Rev. E \textbf{90}, 022116
(2014).
\bibitem{S. Basak 2021} S. Basak and H. Pu, Phys. Rev. A \textbf{104}, 053326
(2021).
\bibitem{V. E. Colussi 2022} V. E. Colussi, F. Caleffi, C. Menotti, A.
Recati, SciPost Phys. \textbf{12}, 111 (2022).
\bibitem{Y. Machida 2022} Y. Machida, I. Danshita, D. Yamamoto, and K. Kasamatsu, Phys. Rev.
A \textbf{105}, L031301 (2022).
\bibitem{A. Trautmann 2018} A. Trautmann, P. Ilzh\"{o}fer, G. Durastante, C. Politi, M. Sohmen, M. J. Mark, and F. Ferlaino, Phys. Rev. Lett. \textbf{121}, 213601
(2018).
\bibitem{T. Mishra 2008} T. Mishra, B. K. Sahoo, and R. V. Pai, Phys. Rev. A \textbf{78}, 013632
(2008).
\bibitem{X. Guan 2019} X. Guan, J. T. Fan, X. F. Zhou, G. Chen, and S. T.
 Jia, Phys. Rev. A \textbf{100}, 013617 (2019).
\bibitem{R. Bai 2020} R. Bai, D. Gaur, H. Sable, S. Bandyopadhyay, K. Suthar, and D.
Angom, Phys. Rev. A \textbf{102}, 043309 (2020).
\bibitem{D. C. Zhang 2022} D. C. Zhang, S. P. Feng, S. J. Yang, Phys. Lett. A \textbf{427},
127912, (2022).





\bibitem{Y.-J. Lin 2011} Y.-J. Lin, K. Jimenez-Garcia, and I. B. Spielman, Nature
(London) \textbf{471}, 83 (2011).
\bibitem{J. Li 2016} J. Li, W. Huang, B. Shteynas, S. Burchesky, F. \c{C}. Top, E. Su, J. Lee, A. O. Jamison, and W. Ketterle, Phys. Rev. Lett. \textbf{117}, 185301
(2016).
\bibitem{J.-R. Li 2017} J.-R. Li, J. Lee, W. Huang, S. Burchesky, B. Shteynas, F. \c{C}. Top, A. O. Jamison, and W. Ketterle, Nature
(London) \textbf{543}, 91 (2017).
\bibitem{Y. A. Bychkov 1984} Y. A. Bychkov and E. I. Rashba, J. Phys. C. \textbf{17}, 6039 (1984).
\bibitem{G. Dresselhaus 1955} G. Dresselhaus, Phys. Rev. \textbf{100}, 580 (1955).

\bibitem{I. Dzyaloshinsky 1958} I. Dzyaloshinsky, J. Phys. and Chem. Sol. \textbf{4}, 241 (1958).
\bibitem{T. Moriya 1960} T. Moriya, Phys. Rev. \textbf{120}, 91 (1960).


\bibitem{W. S. Cole 2012} W. S. Cole, S. Zhang, A. Paramekanti, and N. Trivedi, Phys.Rev. Lett. \textbf{109}, 085302 (2012).
\bibitem{J. Radic 2012} J. Radic, A. Di Ciolo, K. Sun, and V. Galitski, Phys. Rev. Lett. \textbf{109}, 085303 (2012).
\bibitem{Z. Cai 2012} Z. Cai, X. Zhou, and C. Wu, Phys. Rev. A \textbf{85}, 061605(R) (2012).
\bibitem{C. H. Wong 2013} C. H. Wong and R. A. Duine, Phys. Rev. Lett. \textbf{110}, 115301 (2013).
\bibitem{J. Z. Zhao 2015} J. Z. Zhao, S. J. Hu, and P. Zhang, Phys. Rev. Lett. \textbf{115}, 195302
(2015).
\bibitem{R. Y. Li 2015} R. Y. Li, L. He, Q. Sun, A. C. Ji, and G. S. Tian, Chin. Phys. B \textbf{24}, 056701 (2015).
\bibitem{L. He 2015} L. He, A. C. Ji, and W. Hofstette, Phys. Rev. A \textbf{92}, 023630 (2015).
\bibitem{J. G. Wang 2016} J. G. Wang, S. P. Feng and S. J. Yang, New J. Phys. \textbf{18}, 103053 (2016).
\bibitem{B. Xiong 2016} B. Xiong, J. H. Zheng, Y. J. Lin, and D. W. Wang, Phys. Rev. A \textbf{94},
063611 (2016).
\bibitem{C. Wang 2017} C. Wang, M. Gong, Y. J. Han, G. C. Guo, and L. X. He, Phys. Rev. B \textbf{96}, 115119 (2017).
\bibitem{L. Zhang 2019} L. Zhang, Y. G. Ke, and C. H. Lee, Phys. Rev. B \textbf{100}, 224420 (2019)



\bibitem{A. Dutta 2013} A. Dutta and S. Mandal, Phys. Rev. A \textbf{88}, 063619 (2013).
\bibitem{A. T. Bolukbasi 2014} A. T. Bolukbasi and M. Iskin, Phys. Rev. A \textbf{89}, 043603
(2014).
\bibitem{C. Hickey 2014} C. Hickey and A. Paramekanti, Phys. Rev. Lett. \textbf{113}, 265302 (2014).

\bibitem{D. Toniolo 2014} D. Toniolo and J. Linder, Phys. Rev. A \textbf{89}, 061605(R) (2014).
\bibitem{D. Yamamoto 2017} D. Yamamoto, I. B. Spielman, and C. A. R. S\'{a}de Melo, Phys.
Rev. A 96, 061603(R) (2017).
\bibitem{M. Yan 2017} M. Yan, Y. Qian, H. Y. Hui, M.Gong, C. Zhang, and V. W. Scarola, Phys. Rev. A \textbf{96}, 053619 (2017).


\bibitem{A. Dutta 2019} A. Dutta, A. Joshi, K. Sengupta, and P. Majumdar, Phys. Rev. B \textbf{99}, 195126 (2019).
\bibitem{K. Suthar 2021} K. Suthar, P. Kaur, S. Gautam, and D. Angom, Phys. Rev. A \textbf{104}, 043320 (2021).


\bibitem{T. L. Ho 1996} T. L. Ho and V. B. Shenoy, Phys. Rev. Lett. \textbf{77}, 3276-3279
(1996).
\bibitem{P. Ao 1998} P. Ao and S. T. Chui, Phys. Rev. A \textbf{58}, 4836-4840 (1998).


\bibitem{J. Zakrzewski 2005} J. Zakrzewski, Phys. Rev. A \textbf{71}, 043601
(2005).
\bibitem{C. Trefzger 2011} C. Trefzger, C. Menotti, B. C.Sansone and M. Lewenstein, J. Phys. B: At. Mol. Opt. Phys. \textbf{44}
193001 (2011).
\bibitem{A. Rapp 2013} \'{A}. Rapp, Phys. Rev. A \textbf{87}, 043611
(2013).
\bibitem{Y. F. Song 2020} Y. F. Song and S. J. Yang, New J. Phys. \textbf{22}, 073001 (2020).
\bibitem{Y. J. Zhou 2020} Y. J. Zhou, Y. Q. Li, R. Nath, and W. B. Li, Phys. Rev. A \textbf{101}, 013427
(2020).

\bibitem{W. Bao 2002} W. Bao, S. Jin, and P. A. Markowich, J. Comput. Phys. \textbf{175}, 487
(2002).
\bibitem{W. Bao 2003} W. Bao, D. Jaksch, and P. A. Markowich, J. Comput. Phys.
\textbf{187}, 318 (2003).
 \bibitem{P. Bader 2013} P. Bader, S. Blanes, and F. Casas, J. Chem. Phys. \textbf{139}, 124117
(2013).


 \bibitem{B. C. Sansone 2008} B. C. Sansone, \c{S}. G. S\"{o}yler, N. Prokof'ev, and B.
Svistunov, Phys. Rev. A \textbf{77}, 015602 (2008).

\bibitem{G. H. Chen 2003} G. H. Chen and Y. S. Wu, Phys. Rev. A \textbf{67}, 013606 (2003)



\bibitem{H. Y. Hui 2017} H. Y. Hui, Y. P. Zhang, C. W. Zhang, and V. W. Scarola, Phys. Rev.
A \textbf{95}, 033603 (2017).





























\end{thebibliography}
\end{document}
