% (
% {
\documentclass[final,3p,times]{elsarticle_fr}

% if you use PostScript figures in your article
% use the graphics package for simple commands
%\usepackage{graphics}
% or use the graphicx package for more complicated commands
\usepackage{graphicx}
% or use the epsfig package if you prefer to use the old commands
%\usepackage{epsfig}
% pour avoir la légende des figure sur le côté
\usepackage{sidecap}
% pour avoir les figures en tikz
\usepackage{tikz}
%\usetikzlibrary{arrows.meta,bending,positioning}

% The amssymb package provides various useful mathematical symbols
\usepackage{amssymb}
% pour avoir comment
\usepackage{version}

\usepackage[english,french]{babel}

%%%%%%%%%%%%%%%%%%%%%%%%%%%%%%%%
%% GUILLEMETS (FRENCH QUOTES) %%
%%%%%%%%%%%%%%%%%%%%%%%%%%%%%%%%
%\newcommand{\g}[1]{«~#1~»}
\def\og{\leavevmode\raise.3ex\hbox{$\scriptscriptstyle\langle\!\langle$~}}
\def\fg{\leavevmode\raise.3ex\hbox{~$\!\scriptscriptstyle\,\rangle\!\rangle$}}
\newcommand{\g}[1]{\og~#1\fg}


%%%%%%%%%%%%%%%%%%%%%%%%%%%%%%%%%
%  MACROS DEFINED BY YVAN CASTIN
%%%%%%%%%%%%%%%%%%%%%%%%%%%%%%%%%
%\usepackage[applemac]{inputenc}
\usepackage[utf8]{inputenc}
\usepackage[T1]{fontenc}
\usepackage{amsmath}
\usepackage{color}
\usepackage{hyperref}
\usepackage{ulem} % pour pouvoir barrer du texte avec \sout{}
\hypersetup{breaklinks=true}
\hypersetup{colorlinks=true}
\hypersetup{urlcolor=blue}
\hypersetup{citecolor=black}
\hypersetup{linkcolor=black}
\usepackage{cleveref}
\newcommand{\bea}{\begin{eqnarray}}
\newcommand{\eea}{\end{eqnarray}}
\newcommand{\be}{\begin{equation}}
\newcommand{\ee}{\end{equation}}
\newcommand{\rr}{\mathbf{r}}
\newcommand{\RR}{\mathbf{R}}
\newcommand{\kk}{\mathbf{k}}
\newcommand{\ev}{\mathbf{e}}
\newcommand{\QQ}{\mathbf{Q}}
\newcommand{\KK}{\mathbf{K}}
\newcommand{\PP}{\mathbf{P}}
\newcommand{\FF}{\mathbf{F}}
\newcommand{\pp}{\mathbf{p}}
\newcommand{\qq}{\mathbf{q}}
\newcommand{\uu}{\mathbf{u}}
\newcommand{\vv}{\mathbf{v}}
\newcommand{\nn}{\mathbf{n}}
\newcommand{\kf}{k_{\mathrm{F}}}
\newcommand{\mD}{\mathcal{D}}
\newcommand{\mA}{\mathcal{A}}
\newcommand{\mB}{\mathcal{B}}
\newcommand{\mE}{\mathcal{E}}
\newcommand{\mJ}{\mathcal{J}}
\newcommand{\mK}{\mathcal{K}}
\newcommand{\mT}{\mathcal{T}}
\newcommand{\mR}{\mathcal{R}}
\newcommand{\mH}{\mathcal{H}}
\newcommand{\mI}{\mathcal{I}}
\newcommand{\mC}{\mathcal{C}}
\newcommand{\mG}{\mathcal{G}}
\newcommand{\KKK}{\mathbb{K}}
\newcommand{\zero}{\mathbf{0}}
\newcommand{\epsc}{\check{\epsilon}}
\newcommand{\veps}{\varepsilon}
\newcommand{\vepsb}{\bar{\varepsilon}}
\newcommand{\omb}{\bar{\omega}}
\newcommand{\mb}{\bar{m}}
\newcommand{\qb}{\bar{q}}
\newcommand{\kb}{\bar{k}}
\newcommand{\kpb}{\bar{k}'}
\newcommand{\thetat}{\tilde{\theta}}
\newcommand{\zetat}{\tilde{\zeta}}
\newcommand{\zetac}{\check{\zeta}}
\newcommand{\sigg}{\boldsymbol{\sigma}}
\newcommand{\yvan}{\color{black}}
\newcommand{\gilles}{\color{black}}
\newcommand{\simon}{\color{black}}
\renewcommand{\vec}{\mathbf}


\newcommand{\ii}{\textrm{i}}
\newcommand{\ff}{\textrm{f}}
\newcommand{\eee}{\textrm{e}}
\newcommand{\dd}{\mathrm{d}}
% opérateur identité
\def\identit{\mbox{l\hspace{-0.55em}1}}
\DeclareMathOperator\erf{erf}
\DeclareMathOperator\erfi{erfi}

\DeclareMathOperator\asin{arcsin}
\DeclareMathOperator\acos{arccos}
\DeclareMathOperator\atan{arctan}
\DeclareMathOperator\arccotan{arccotan}
\DeclareMathOperator\argth{argth}
\DeclareMathOperator\arctanh{arctanh}
\DeclareMathOperator\argch{argch}
\DeclareMathOperator\argsh{argsh}
\DeclareMathOperator\thf{th}
\DeclareMathOperator\ch{ch}
\DeclareMathOperator\sh{sh}
\DeclareMathOperator\re{Re}
\DeclareMathOperator\im{Im}
\DeclareMathOperator\sig{signe}
\newcommand{\bleu}{\color{blue}}
\newcommand{\rouge}{\color{red}}
% pour pouvoir mettre des figures dans une minipage
\usepackage{float}

% on redefinit la fleche des vecteurs pour qu'elle soit de longueur ajustable
\newcommand{\vecl}{\overrightarrow}

%\journal{Comptes rendus - Physique}

\begin{document}

\begin{frontmatter}

% Title, authors and addresses

% use the thanksref command within \title, \author or \address for footnotes;
% use the ead command for the email address,
% and the form \ead[url] for the home page:
% \title{Title\thanksref{label1}}
% \thanks[label1]{}
% \author{Name\thanksref{label2}}
% \ead{email address}
% \ead[url]{home page}
% \thanks[label2]{}
% \address{Address\thanksref{label3}}
% \thanks[label3]{}
\selectlanguage{french}
\title{Amortissement des phonons dans un superfluide 2D : insuffisance de la règle d'or de Fermi à basse température} 

% use optional labels to link authors explicitly to addresses:
% \author[label1,label2]{}
% \address[label1]{}
% \address[label2]{}
% If all authors are at the same address, the [label1] can be suppressed

\author{Yvan Castin}
\ead{yvan.castin@lkb.ens.fr}
\author{Alan Serafin}
\ead{a.serafin@protonmail.com}
\author{Alice Sinatra}
\ead{alice.sinatra@lkb.ens.fr}

\address{Laboratoire Kastler Brossel, ENS-Universit\'e PSL, CNRS, Universit\'e Sorbonne et Coll\`ege de France, 24 rue Lhomond, 75231 Paris, France}
% If your know the dates of reception, and acceptation you can put them now;
%    idem the name of the person presenting your article

%\medskip

%\begin{center}
%{\small Received *****; accepted after revision +++++}
%\end{center}

\begin{abstract}
Il est en général admis que le gaz de phonons d'un superfluide entre toujours dans un régime de couplage faible à suffisamment basse température, quelle que soit la force des interactions entre les particules sous-jacentes (constitutives du superfluide). Ainsi, dans cette limite, on devrait pouvoir toujours calculer le taux d'amortissement des phonons thermiques en appliquant la règle d'or de Fermi à l'hamiltonien $H_3$ de couplage cubique phonon-phonon tiré de l'hydrodynamique quantique, du moins dans le cas d'une branche acoustique convexe et dans le régime faiblement collisionnel (où la pulsation propre des phonons considérés reste très supérieure au taux de thermalisation du gaz). À l'aide de la méthode des fonctions de Green à $N$ corps, nous prédisons que, contrairement au cas tridimensionnel et de manière inattendue, ceci n'est pas vrai en dimension deux. Nous confirmons cette prédiction par des simulations de champ phononique classique et par une théorie non perturbative en $H_3$, qui régularise l'ordre quatre en donnant à la main une énergie complexe aux phonons virtuels des processus collisionnels à quatre phonons. Pour un fluide en interaction faible et un mode de phonon dans la limite des grandes longueurs d'onde, nous prédisons un taux d'amortissement environ trois fois plus faible que celui de la règle d'or.
\\
\noindent{\small{\it Mots-cl\'es~:} gaz de bosons~; superfluide bidimensionnel~; gaz d'atomes froids~; amortissement Belyaev-Landau des phonons~; fonctions de Green à $N$ corps ; hydrodynamique quantique}
\selectlanguage{french}

\noindent 
\vskip 0.5\baselineskip%}{
\end{abstract} 
\end{frontmatter}

% now the Version franÁaise abrÈgÈe, if it exists
\selectlanguage{french}
%\section*{Version fran\c{c}aise abr\'eg\'ee}
% Text of your Version franÁaise abrÈgÈe here

%\selectlanguage{english}
% main text
%\newpage
\tableofcontents

%{(
\section{Présentation du problème et raisonnements généraux}
\label{sec0}

Nous considérons en dimension $d\geq 2$ un système spatialement homogène de particules avec des interactions à courte portée, entièrement superfluide dans son état fondamental et préparé à température non nulle mais arbitrairement basse, ce qui permet de limiter son étude à l'ordre dominant en température au moyen d'une théorie effective de basse énergie. Indépendamment de la statistique quantique (bosonique ou fermionique) des particules et de la force de leurs interactions (en phase gazeuse ou liquide), ce système admet une branche d'excitation acoustique, c'est-à-dire de départ linéaire en nombre d'onde (sans bande interdite), dont les quanta sont les phonons. 

Le couplage entre phonons de faible nombre d'onde (d'énergie $<\eta k_B T$, où la température $T\to 0$ puis la coupure $\eta\to+\infty$ à la fin des calculs) se déduit de l'hydrodynamique quantique de Landau et Khalatnikov \cite{LK}, l'hamiltonien effectif prenant la forme\footnote{\label{note1} L'inclusion de termes d'ordre supérieur $H_5, H_6$, etc., n'aurait pas de sens sans des corrections de gradient à la fonctionnelle énergie de l'hydrodynamique quantique \cite{SonWingate}, corrections qui font aussi apparaître le terme de courbure (ici mis à la main) dans l'équation (\ref{eq003}).}
\be
\label{eq001}
H=H_2+H_3+H_4
\ee
où le terme $H_n$ est de degré $n$ en les $\hat{b}_\kk^\dagger$ et $\hat{b}_\kk$, opérateurs bosoniques de création et d'annihilation d'un phonon de vecteur d'onde $\kk$ dans la boîte de quantification $[0,L]^d$ avec des conditions aux limites périodiques. Le terme $H_2$ représente un gaz parfait de phonons de relation de dispersion $\veps_\kk$,
\be
\label{eq002}
H_2=\sum_{\kk\neq\zero} \veps_\kk \hat{b}_\kk^\dagger\hat{b}_\kk
\ee
La forme précise des termes de couplage $H_3$ et $H_4$ dans l'équation (\ref{eq001}) importe peu à ce stade et sera donnée dans la section \ref{sec1}. Le couplage entre phonons rend le système acoustiquement non linéaire et provoque l'amortissement des ondes sonores, objet du présent travail. 

Nous limitons ici notre étude de l'amortissement (i) aux modes de vecteurs d'onde $\qq$ dans le régime faiblement collisionnel $\veps_\qq\gg\hbar\Gamma_{\rm th}$ où $\Gamma_{\rm th}$ est le taux de thermalisation du gaz de phonons ou, ce qui revient au même, le taux d'amortissement des phonons thermiques d'énergie $\veps_\kk\simeq k_B T$, et (ii) au cas d'une branche acoustique de départ convexe en nombre d'onde, que l'on peut limiter à son approximation cubique à basse température:
\be
\label{eq003}
\veps_\kk=\hbar c k \left[1+\frac{\gamma}{8} \left(\frac{\hbar k}{m c}\right)^2\right]\quad\mbox{avec}\quad \gamma>0
\ee
où $m$ est la masse d'une particule, $c$ est la vitesse du son à température nulle et $\gamma$ un paramètre de courbure sans dimension.\footnote{\label{note0} Comme en électrodynamique quantique, les paramètres apparaissant dans l'hamiltonien sont en fait les valeurs nues, différentes des valeurs effectives mesurées dans une expérience, le passage des unes aux autres s'effectuant par une procédure dite de \g{renormalisation}. Il faudrait donc parler dans l'équation (\ref{eq002}) de relation de dispersion nue $\veps_\kk^{(0)}$ et dans l'équation (\ref{eq003}) de vitesse du son nue $c^{(0)}$ et de paramètre de courbure nu $\gamma^{(0)}$. Nous ne le faisons pas ici pour alléger les notations, et parce que cette renormalisation n'intervient pas en dimension $d=2$ à l'ordre dominant en température.}\footnote{Le choix $\gamma>0$ exclut les superfluides de fermions de spin $1/2$ appariés dans la limite, dite BCS, d'une interaction faiblement attractive entre particules \cite{SDM}.} Ces hypothèses autorisent à décrire l'amortissement en termes de processus élémentaires de couplage entre phonons\footnote{Dans le régime opposé $\veps_\qq\ll\hbar\Gamma_{\rm th}$, le système a le temps de relaxer vers une juxtaposition d'états d'équilibre thermodynamique locaux pendant une période d'oscillation de l'onde sonore de vecteur d'onde $\qq$ et l'amortissement à température non nulle se décrit au travers de coefficients de transport ou de viscosité, obtenus par résolution d'équations cinétiques et qui entrent dans les équations aux dérivées partielles sur les densités, les vitesses, l'entropie volumique, etc., d'un modèle à deux fluides \cite{livreK}.} et, comme il est généralement admis, à déduire le taux d'amortissement de la règle d'or de Fermi appliquée au terme $H_3$ de l'hamiltonien.\footnote{Dans le cas $\gamma<0$ d'une branche acoustique de départ concave, les processus à trois phonons induits par $H_3$ à l'ordre un ne conservent pas l'énergie-impulsion, et il faut utiliser une formulation étendue de la règle d'or (tenant compte des processus à quatre phonons induits par $H_3$ à l'ordre deux et par $H_4$ à l'ordre un) pour obtenir un taux d'amortissement non nul, voir les références \cite{LK,Annalen} et notre section \ref{sec3.2}.} En effet, on s'attend à ce qu'à suffisamment basse température, le gaz de phonons entre toujours dans le régime de couplage faible, même si les particules constitutives du système sont en interaction forte (comme dans l'hélium 4 liquide par exemple \cite{Maris}). Nous allons montrer dans cette section, par un raisonnement simple, que ceci est vrai en dimension $d=3$ mais pas en dimension $d=2$. L'amortissement des phonons dans le cas bidimensionnel fera donc l'objet d'une étude plus poussée (au-delà de la règle d'or) dans les sections suivantes. 

Notre raisonnement simple s'appuie sur la méthode des fonctions de Green à $N$ corps \cite{FW}. Pour fixer les idées et la fonction de Green à considérer, nous supposons que le gaz, initialement à l'équilibre thermique pour l'hamiltonien $H$ (opérateur densité $\hat{\sigma}(0)=\hat{\sigma}_{\rm th}\propto \exp(-H/k_B T)$), est excité par la création soudaine d'un état cohérent de phonons dans le mode $\qq$:
\be
\label{eq010}
\hat{\sigma}(0^+) = \hat{U}_{\rm exc}\hat{\sigma}_{\rm th} \hat{U}^\dagger_{\rm exc} \quad\mbox{avec}\quad \hat{U}_{\rm exc}=\eee^{\alpha\hat{b}_\qq^\dagger-\alpha^*\hat{b}_\qq}
\ee
Expérimentalement, cette procédure d'excitation peut être réalisée par une impulsion laser de Bragg de courte durée \cite{Grynberg,Ketterle,Davidson,Vale,Cartago}. On mesure ensuite l'amplitude $\langle\hat{b}_\qq(t)\rangle$ de l'état cohérent au cours du temps, au travers par exemple de la densité du gaz (plus précisément de ses composantes de Fourier de vecteurs d'onde $\pm\qq$). Dans la limite d'une très faible excitation initiale, on entre dans le régime de réponse linéaire et l'on peut formellement développer l'opérateur unitaire d'excitation $\hat{U}_{\rm exc}$ à l'ordre un en $\alpha$ dans l'équation (\ref{eq010}) pour obtenir
\be
\label{eq011}
\langle\hat{b}_\qq(t)\rangle\stackrel{t>0}{\underset{\alpha\to 0}{=}} \alpha \langle[\hat{b}_\qq(t),\hat{b}_\qq^\dagger(0)]\rangle_{\rm th}+O(\alpha^2)
\ee
où $\hat{b}_\qq(t)$ est écrit en représentation de Heisenberg pour l'évolution non perturbée par $\hat{U}_{\rm exc}$, $\hat{b}_{\qq}(t)=\eee^{\ii H t/\hbar} \hat{b}_{\qq}(0)\eee^{-\ii H t/\hbar}$, la moyenne au premier membre est prise dans l'état perturbé $\hat{\sigma}(0^+)$ et celle au second membre dans l'état thermique initial non perturbé $\hat{\sigma}_{\rm th}$, et on a utilisé le fait que $\langle\hat{b}_{\qq}(0)\hat{b}_{\qq}(t)\rangle_{\rm th}=\langle\hat{b}_{\qq}(t)\hat{b}_{\qq}(0)\rangle_{\rm th}=0$ par non-conservation de l'impulsion totale. À l'aide de la section 31 de la référence \cite{FW}, nous relions notre signal à la fonction de Green habituelle à deux points $\mG$ et à la fonction énergie propre associée $\Sigma$ (notée $\Sigma^{\star}$ dans \cite{FW}):\footnote{Rappelons qu'à l'origine, $\mG$ et $\Sigma$ sont des fonctions de $(\rr,t)$ et $(\rr',t')$ où $\rr,\rr'$ sont deux vecteurs positions et $t,t'$ deux temps. On les réduit à des fonctions de $(\rr-\rr',t-t')$ par homogénéité spatiale et stationnarité de l'état thermique, dont les transformées de Fourier spatio-temporelles sont des fonctions de $(\kk,\omega)$; le prolongement analytique de ces dernières aux pulsations complexes $z/\hbar$ conduit aux fonctions $\mG(\kk,z)$ et $\Sigma(\kk,z)$ de l'équation (\ref{eq012}).}
\be
\label{eq012}
\boxed{s(t)\equiv \eee^{\ii\veps_\qq t/\hbar}\langle[\hat{b}_\qq(t),\hat{b}_\qq^\dagger(0)]\rangle_{\rm th}\stackrel{t>0}{=} \int_{C_+} \frac{\dd z}{2\ii\pi} \eee^{-\ii (z-\veps_\qq)t/\hbar} \mathcal{G}(\qq,z)\quad\mbox{avec}\quad \mathcal{G}(\qq,z)=\frac{1}{z-\veps_\qq-\Sigma(\qq,z)}}
\ee
où le chemin d'intégration $C_+$ court dans le demi-plan supérieur parallèlement à l'axe réel de $+\infty$ à $-\infty$ et où nous avons, par commodité, retiré l'évolution libre due à l'hamiltonien $H_2$. Dans la suite, il sera avantageux de prendre comme nouvelle variable d'énergie complexe l'écart à l'énergie de l'évolution libre, toute confusion étant évitée par mise en indice de la dépendance en vecteur d'onde:
\be
\label{eq013}
\Sigma(\qq,z)=\Sigma_\qq(\zeta)\quad\mbox{avec}\quad\zeta\equiv z-\veps_\qq
\ee
À partir de ce formalisme exact, on peut retrouver l'amortissement exponentiel de la règle d'or de Fermi $|s(t)|^2\simeq \exp(-\Gamma_q t)$ en deux étapes. (i) On effectue d'abord l'approximation du pôle \cite{CCTbordeaux} négligeant la dépendance en énergie de la fonction $\Sigma$:
\be
\label{eq014}
\Sigma_\qq(\zeta) \approx \Sigma_\qq(\ii 0^+)\quad \mbox{si bien que}\quad s(t)\simeq s_{\mbox{\scriptsize pôle}}(t)=\exp[-\ii \Sigma_\qq(\ii 0^+)t/\hbar]
\ee
(ii) Comme l'approximation du pôle n'est en général légitime que dans une limite de couplage faible entre les phonons, il suffit de calculer $\Sigma_\qq$ à l'ordre dominant en le couplage, c'est-à-dire à l'ordre deux en $H_3$, mais aussi à l'ordre un en $H_4$ si l'on a besoin de connaître $\re\Sigma_\qq$. La question centrale de cette section est de savoir si le simple fait de prendre une limite de basse température avec mise à l'échelle du nombre d'onde du mode considéré,\footnote{Faut-il le préciser, la force des interactions entre les particules constitutives du fluide ne varie pas dans la prise de limite (\ref{eq020}), si bien que la vitesse du son $c$ reste constante. Par ailleurs, la limite (\ref{eq020}) fait automatiquement entrer le mode $\qq$ dans le régime faiblement collisionnel $\veps_{\qq}\gg \hbar\Gamma_{\rm th}$ car le taux $\Gamma_{\rm th}$ de thermalisation du gaz, que l'on peut estimer par le taux d'amortissement des phonons de nombre d'onde $\approx k_B T/\hbar c$, tend vers zéro plus vite que linéairement en température, comme nous le verrons.}
\be
\label{eq020}
\epsilon \equiv k_B T/mc^2 \to 0 \quad \mbox{à}\quad \qb\equiv \hbar c q/k_B T \quad \mbox{fixé}
\ee
suffit à faire entrer le gaz de phonons dans le régime de couplage faible et à justifier les deux étapes. Nous verrons que la réponse dépend de la dimensionalité du système.

\paragraph{Étape (i)} Pour justifier l'approximation du pôle, nous nous appuyons sur les lois d'échelle décrivant le comportement de la fonction énergie propre à basse température:
\be
\label{eq021}
\Sigma_\qq(\zeta) = k_B T \epsilon^\nu\, \tilde{\Sigma}_{\qb}(\zeta/(k_BT\epsilon^\sigma))
\ee
Après sortie de l'échelle d'énergie $k_B T$, il reste deux exposants $\nu$ et $\sigma$, le premier caractérisant l'ordre de grandeur de la fonction $\Sigma_\qq(\zeta)$, le second sa largeur en énergie autour de $\zeta=0$. Il est entendu que la nouvelle fonction $\tilde{\Sigma}_{\qb}$ admet une limite finie et non nulle lorsque $\epsilon\to 0$. Après report de la forme (\ref{eq021}) dans l'expression (\ref{eq012}) du signal, nous effectuons les changements d'échelle suivants sur les variables conjuguées $\zeta$ et $t$,
\be
\label{eq022}
\zeta\equiv k_B T \epsilon^\nu \zetat \quad ; \quad t\equiv \tilde{t}\,\hbar/(k_B T \epsilon^\nu)
\ee
pour obtenir l'expression réduite
\be
\label{eq023}
\boxed{s(t)=\int_{C_+} \frac{\dd \zetat}{2\ii\pi} \frac{\eee^{-\ii \zetat \tilde{t}}}{\zetat-\tilde{\Sigma}_{\qb}(\epsilon^{\nu-\sigma}\zetat)}}
\ee
dans laquelle il reste à faire tendre $\epsilon$ vers zéro à $\tilde{t}$ fixé. On constate alors que, si $\nu>\sigma$, l'argument de la fonction $\tilde{\Sigma}_{\qb}$ tend vers zéro par parties imaginaires strictement positives lorsque $\epsilon\to 0$ ce qui autorise à le remplacer par $\ii 0^+$: on retrouve l'approximation du pôle (\ref{eq014}), exacte dans cette limite. 

\paragraph{Étape (ii)} Pour savoir s'il suffit de calculer la fonction énergie propre à l'ordre dominant en le couplage entre phonons, gardons pour simplifier seulement le couplage cubique $H_3$ (on peut vérifier que le couplage quartique $H_4$ ne change rien qualitativement), omettons les processus non résonnants qu'elle contient (termes en $\hat{b}^\dagger\hat{b}^\dagger\hat{b}^\dagger$ et $\hat{b}\hat{b}\hat{b}$) de contribution sous-dominante dans la limite $\epsilon\to 0$, et estimons l'ordre de grandeur de sa contribution d'ordre $2n$ à $\Sigma_\qq$ comme suit:\footnote{Comme $H_3$ change la parité du nombre de phonons, les contributions d'ordre impair $2n+1$ sont nulles.}
\be
\label{eq024}
\Sigma^{(2n)}_\qq(\ii 0^+)\approx \int \left(\prod_{i=1}^{n} \dd^d k_i\right) \langle\ |\mathcal{H}_3|\ \rangle \frac{1}{\Delta E_1} \langle\ |\mathcal{H}_3|\ \rangle \ldots \langle\ |\mathcal{H}_3|\ \rangle \frac{1}{\Delta E_{2n-1}} \langle\ |\mathcal{H}_3|\ \rangle
\ee
Justifions brièvement la forme (\ref{eq024}) (le lecteur est renvoyé aux sections \ref{sec2} et \ref{sec3} pour le traitement détaillé des cas $n=1$ et $n=2$). L'ordre $2n$ fait intervenir $2n$ actions de $H_3$ donc $2n$ éléments de matrice $\langle\,|H_3|\,\rangle$, chacune d'elles introduisant un nouveau vecteur d'onde de phonon indépendant $\kk_i$ (termes de Belyaev en $\hat{b}^\dagger\hat{b}^\dagger\hat{b}$ dans $H_3$) ou en faisant disparaître un (termes de Landau en $\hat{b}^\dagger\hat{b}\hat{b}$); les autres vecteurs d'onde de phonons qui apparaissent se déduisent des $\kk_i$ et de $\qq$ par conservation de l'impulsion. Comme les $2n$ actions de $H_3$ doivent conserver le nombre de phonons dans leur ensemble, il doit y avoir autant de processus de Belyaev que de Landau ce qui conduit à exactement $n$ vecteurs d'onde indépendants sur lesquels sommer. À la limite thermodynamique, on exprime $H_3$ en termes de l'hamiltonien volumique tel que $H_3=\mH_3/L^{d/2}$ et on remplace la somme sur les $\kk_i$ par une intégrale. Enfin, entre chaque élément de matrice de $H_3$ apparaît un dénominateur traduisant la propagation libre d'hamiltonien $H_2$ et donnant la différence d'énergie entre un état intermédiaire et l'état initial à un phonon $\qq$. Chaque $\Delta E_j$ ($1\leq j\leq 2n-1$) peut être exprimé comme la somme algébrique (c'est-à-dire avec un signe plus ou moins) de $j$ défauts d'énergie élémentaires $\veps_\kk+\veps_{\kk'}-\veps_{\kk''}$ correspondant à un processus Belyaev-Landau $\kk+\kk'\leftrightarrow\kk''$. Dans la limite de basse température $\epsilon\to 0$, l'intégrale (\ref{eq024}) est en général dominée par les processus \g{aux petits angles}, où {\it tous} les phonons intermédiaires (ceux de vecteurs d'onde $\kk_i$ et les autres) sont presque colinéaires à et de même sens que $\qq$ \cite{LK,Annalen}; plus précisément, on trouve que les angles $\theta_i$ entre les vecteurs $\qq$ et $\kk_i$ doivent être des $O(\gamma^{1/2}\epsilon)$ pour que les $\Delta E_j$ atteignent leur échelle d'énergie minimale et que l'intégrande soit amplifié par l'effet \g{petit dénominateur}, ce qui conduit au développement du défaut d'énergie élémentaire
\be
\label{eq025}
\veps_\kk+\veps_{\kk'}-\veps_{\kk''=\kk+\kk'}\underset{\epsilon\to 0}{\sim} k_B T\gamma\epsilon^2\left[\frac{\kb\kpb}{\kb+\kpb}\frac{(\theta-\theta')^2}{2\gamma\epsilon^2}-\frac{3}{8}\kb\kpb(\kb+\kpb)\right]
\ee
où $\gamma$ est le paramètre de courbure dans la relation de dispersion (\ref{eq003}) et l'adimensionnement de $k$ en $\kb$ par sa valeur typique $k_B T/\hbar c$ est calqué sur l'équation (\ref{eq020}). Les $\Delta E_j$ sont donc d'ordre $k_B T\epsilon^2$ et chaque élément d'intégration $\dd^dk_i$ dans (\ref{eq024}) écrit en coordonnées polaires ou sphériques fait sortir un facteur $(k_B T/\hbar c)^d (\gamma^{1/2}\epsilon)^{d-1}$. Il reste à donner l'estimation des éléments de matrice de $\mH_3$, que l'on tire aisément de leur expression (\ref{eq102}) à venir (la constante de couplage $\Lambda$ est donnée par l'équation (\ref{eq032})):
\be
\label{eq027}
\langle\ |\mathcal{H}_3|\ \rangle \approx (k\xi k'\xi k''\xi)^{1/2}(1+\Lambda)\frac{mc^2}{\rho^{1/2}} \approx \epsilon^{3/2}(1+\Lambda)\frac{mc^2 \xi^{d/2}}{(\rho\xi^d)^{1/2}}
\ee
où l'on a introduit la densité $\rho$ du fluide et la longueur dite de relaxation ou de corrélation du superfluide, 
\be
\label{eq026}
\xi=\frac{\hbar}{mc}
\ee
Nous aboutissons ainsi à l'estimation
\be
\label{eq028}
\boxed{\Sigma_\qq^{(2n)}(\ii0^+)\underset{\epsilon\to 0}{\approx} k_BT \gamma \epsilon^2 \left[\frac{\epsilon^{2d-4}(1+\Lambda)^2}{\gamma^{(5-d)/2}\rho\xi^d}\right]^n \quad\quad (n>0,d\geq 2)}
\ee
qui désigne la quantité entre crochets comme petit paramètre du développement perturbatif de $\Sigma_\qq$ en le couplage. Le degré en $\epsilon$ de l'ordre dominant non nul, correspondant au choix $n=1$ dans l'équation (\ref{eq028}), donne la valeur de l'exposant $\nu$ dans l'équation (\ref{eq021}):
\be
\label{eq004}
\nu=2d-2
\ee
La valeur du second exposant $\sigma$ s'obtient en généralisant l'estimation (\ref{eq024}) au cas $\zeta\neq 0$, ce qui revient à ajouter $\zeta$ à l'énergie de l'état initial donc à chaque dénominateur d'énergie, $\Delta E_j\to \Delta E_j+\zeta$. On l'a vu, $\Delta E_j\approx k_B T\epsilon^2$, et c'est aussi la largeur de la fonction énergie propre autour de $z=\epsilon_\qq$ (autour de $\zeta=0$), si bien que
\be
\label{eq005}
\sigma=2
\ee
indépendamment de la dimensionalité.

\paragraph{Cas tridimensionnel} En dimension $d=3$, où $\nu=4>\sigma=2$ et où $2d-4=2>0$ entre les crochets de l'équation (\ref{eq028}), les étapes (i) et (ii) sont justifiées et l'amortissement du mode $\qq$ est décrit de manière exacte par la règle d'or de Fermi dans la limite (\ref{eq020}). On trouve alors le comportement dominant à basse température de la partie imaginaire de $\Sigma_\qq(\ii 0^+)$,  
\begin{multline}
\label{eq030}
\im\Sigma_\qq^{d=3}(\ii 0^+) \stackrel{\qb\,\mbox{\scriptsize fix\'e}}{\underset{\epsilon\to 0}{\sim}} -\frac{9 k_B T \epsilon^4(1+\Lambda)^2}{32\pi\rho\xi^3}\Bigg[\int_0^{\qb}\dd\kb\, \kb^2(\qb-\kb)^2(\bar{n}^{\rm lin}_k+1/2)+\int_0^{+\infty}\dd\kb\, \kb^2(\kb+\qb)^2(\bar{n}^{\rm lin}_k-\bar{n}^{\rm lin}_{k+q})\Bigg] \\
=-\frac{9 k_B T \epsilon^4(1+\Lambda)^2}{32\pi\rho\xi^3}\Bigg\{\frac{\qb^5}{60}+48[g_5(1)-g_5(\eee^{-\qb})]-24\qb\, g_4(\eee^{-\qb})+4\qb^2[g_3(1)-g_3(\eee^{-\qb})]\Bigg\}
\end{multline}
correspondant exactement au taux d'amortissement $\Gamma_q$ donné dans la référence \cite{Annalen} (multiplié par l'habituel facteur $-\hbar/2$). Dans l'équation (\ref{eq030}), on a introduit les fonctions de Bose $g_\alpha(z)=\sum_{n=1}^{+\infty} z^n/n^\alpha$ et on a pu remplacer la relation de dispersion $\veps_\kk$ par son approximation linéaire $\hbar c k$ dans les nombres d'occupation, ce qu'indique l'exposant \g{lin}:
\be
\label{eq029}
\bar{n}_k^{\rm lin}=\frac{1}{\exp\kb-1}
\ee
On constate avec satisfaction que le préfacteur dans (\ref{eq030}) est bien celui prédit par (\ref{eq028}) pour $n=1$ (et $d=3$) ; il est en particulier indépendant du paramètre de courbure $\gamma$. Le formalisme des fonctions de Green donne en plus accès à la partie réelle, qui échappait à la référence \cite{Annalen} et qui représente le déplacement d'énergie thermique du phonon $\qq$,\footnote{On omet dans (\ref{eq031}) la composante dite quantique, indépendante des nombres d'occupation, qui est à l'ordre dominant en $\epsilon$ la somme de deux contributions, (i) une contribution mal définie, combinaison linéaire des monômes $q$, $q^3$ et $q^5$ avec des coefficients dépendant de façon divergente de la coupure ultraviolette $\eta$ de l'équation (\ref{eq103}), qui doit donc être réabsorbée dans $\veps_\qq$, c'est-à-dire dans une redéfinition de la relation de dispersion (voir aussi la note \ref{note0}), et (ii) une contribution universelle prédite quantitativement par l'hydrodynamique quantique, $\re\Sigma_{\qq,{\rm quant\ univ}}^{d=3}(\ii 0^+)=[3(1+\Lambda)^2/(160\pi^2)](\hbar^2 q^5/m\rho)\ln(\hbar q/mc)$. Une telle contribution en $q^5\ln q$ était attendue qualitativement dans la référence \cite{SonWingate}. Elle apparaît dans l'équation (35) de la référence \cite{Barc}, avec le bon coefficient numérique mais avec un facteur $c$ en trop.}
\begin{multline}
\label{eq031}
\re\Sigma_{\qq,{\rm th}}^{d=3}(\ii 0^+) \stackrel{\qb\,\mbox{\scriptsize fix\'e}}{\underset{\epsilon\to 0}{\sim}} \frac{k_B T \epsilon^4(1+\Lambda)^2}{2\pi^2\rho\xi^3}
\Bigg\{\qb\zeta(4)\left[-27\ln\frac{\eee(3\gamma)^{1/2}\epsilon}{4}+\frac{2+18\Lambda+3\Lambda_*/2}{(1+\Lambda)^2}\right] \\
+\frac{9}{8} \int_0^{+\infty} \dd\kb\, \kb^2\, \bar{n}^{\rm lin}_k \left[(\kb-\qb)^2\ln|\kb-\qb|-(\kb+\qb)^2\ln(\kb+\qb)\right] \Bigg\}
\end{multline}
où $\zeta$ est la fonction zêta de Riemann, telle que $\zeta(s)=g_s(1)$, et $\zeta(4)=\pi^4/90$. On obtient pour la partie réelle une forme plus compliquée que celle de (\ref{eq028}) car (a) les angles entre $\qq$ et $\kk$ d'ordre $\theta\approx\epsilon^0$ contribuent à la même hauteur que les petits angles $\theta\approx\epsilon$,\footnote{On remarque en effet dans l'intégration angulaire, en $d=3$, que $\sin\theta\,\dd\theta/(\Delta E/k_B T)$ est $\approx \epsilon^0$ aux grands angles et $\approx \epsilon^2/\epsilon^2=\epsilon^0$ aux petits angles, où $\Delta E$ est un dénominateur d'énergie. Ce phénomène ne se produit pas pour la partie imaginaire, compte tenu du Dirac de conservation de l'énergie $\delta(\Delta E)$ et du fait que $\Delta E$ ne s'annule qu'aux petits angles pour des phonons thermiques à basse température. En pratique, pour obtenir (\ref{eq031}), on sépare l'intégrale angulaire en deux intervalles, $\theta\in[0,\alpha]$ et $\theta\in [\alpha,\pi]$ où $\alpha\ll 1$ est fixé. Dans le premier intervalle, on garde le terme de courbure dans la relation de dispersion au dénominateur de l'intégrande mais on remplace le numérateur par son ordre dominant en $\theta$ (ainsi, $\sin\theta\simeq\theta$). Dans le second intervalle, on utilise la relation de dispersion linéaire mais on garde toute la dépendance en $\theta$ des fonctions trigonométriques. Tous calculs faits, la partie basse et la partie haute de l'intégrale comportent une divergence en $\ln\alpha$ qui disparaît exactement dans leur somme.} et (b) l'hamiltonien quartique $H_4$ contribue également, ce qui nous oblige à introduire la seconde constante de couplage $\Lambda_*$ de l'équation (\ref{eq032}). En particulier, on remarque une correction logarithmique à la loi en $k_B T \epsilon^4$. Dans la limite $\qb\to 0$, l'expression (\ref{eq031}) devient $\sim \hbar q \delta c_{\rm th}$, où le coefficient $\delta c_{\rm th}$ est la correction thermique à la vitesse du son dans le fluide. La valeur de $\delta c_{\rm th}$ tirée de la théorie cinétique d'Andreev et Khalatnikov \cite{AK}, voir l'équation (45) de la référence \cite{Barc}, est en bon accord avec la nôtre,\footnote{Il y a accord parfait si, dans l'équation (45) de \cite{Barc}, on remplace $\ln 27$ par $25/6-2C+\ln(3/2)+2\zeta'(4)/\zeta(4)=\ln(26,\!852\ldots)$ où $C=0,\!577\ldots$ est la constante d'Euler. Ce petit écart résulte d'une approximation faite par Andreev et Khalatnikov dans le calcul d'une intégrale, celle de l'équation (13) dans \cite{AK}.} mais celle déduite de l'ordre dominant de la théorie effective des champs, voir l'équation (33) de la référence \cite{Barc}, est en désaccord.

\paragraph{Cas bidimensionnel} En dimension $d=2$, où $\nu=\sigma=2$ et où $2d-4=0$, nous échouons à justifier l'étape (i) et l'étape (ii) à basse température! En particulier, l'expression entre crochets dans l'équation (\ref{eq028}) ne dépend pas de $\epsilon$ et ne devient un petit paramètre que dans la limite $\rho\xi^2\gg 1$ où le superfluide sous-jacent lui-même (et pas simplement le gaz de phonons) entre dans le régime d'interaction faible. Ainsi, en $d=2$, la prise de la limite $\epsilon\to 0$ ne rend exactes ni l'approximation du pôle ni le calcul de $\Sigma_\qq(\zeta)$ à l'ordre deux en le couplage entre phonons. Pour obtenir la bonne expression du signal dans cette limite, il faut calculer $\tilde{\Sigma}_{\qb}(\zetat)$ pour toutes les valeurs de $\zetat$ et à tous les ordres en $H_3$. Donnons quand même comme valeur de référence le comportement à basse température de $\Sigma_\qq(\ii 0^+)$ à l'ordre dominant en le couplage ($n=1$):
\begin{multline} 
\label{eq033}
\Sigma_\qq^{(2)}(\ii 0^+)\stackrel{\qb\,\mbox{\scriptsize fix\'e}}{\underset{\epsilon\to 0}{\sim}} -\ii\frac{9k_B T\epsilon^2(1+\Lambda)^2}{8\pi(3\gamma)^{1/2}\rho\xi^2} \Bigg[\int_0^{\qb}\dd\kb\, \kb(\qb-\kb)(\bar{n}^{\rm lin}_k+1/2)+\int_0^{+\infty}\dd\kb\, \kb (\kb+\qb) (\bar{n}^{\rm lin}_k-\bar{n}^{\rm lin}_{k+q})\Bigg] \\
=\boxed{-\ii\frac{9k_B T\epsilon^2(1+\Lambda)^2}{8\pi(3\gamma)^{1/2}\rho\xi^2} \Bigg[\frac{\qb^3}{12}+2\zeta(2)\qb\Bigg]\equiv -\ii \frac{\hbar\Gamma_q}{2}}
\end{multline}
avec $\zeta(2)=\pi^2/6$. À suffisamment basse température, $\Sigma^{(2)}_\qq(\ii 0^+)$ devient donc imaginaire pur et se réduit à $-\ii\hbar\Gamma_q/2$ où $\Gamma_q$ est le taux d'amortissement de la règle d'or à l'ordre dominant en $\epsilon$ (l'hamiltonien quartique $H_4$ ne fait pas apparaître ici de partie réelle car il est sous-dominant en $\epsilon$, voir la section \ref{sec2}). L'équation (\ref{eq033}) est à notre connaissance originale. Dans les limites $\qb\to 0$ et $\qb\to +\infty$, elle redonne les équations (30) et (22) de la référence \cite{CB} (fois $-\ii\hbar$ compte tenu des conventions de \cite{CB}) qui traite du gaz bidimensionnel de bosons en interaction faible pour lequel $\gamma=1$ (comme dans le spectre de Bogolioubov) et $\Lambda=0$ (comme pour l'équation d'état de champ moyen à température nulle $\mu=\rho g$ utilisée dans \cite{CB}). Dans la suite, nous irons en dimension $d=2$ au-delà de la règle d'or de Fermi avec des simulations numériques de champ classique dans la section \ref{sec1} et avec la méthode des fonctions de Green à des ordres en $H_3$ de plus en plus élevés dans les sections ultérieures: à l'ordre deux dans la section \ref{sec2}, à l'ordre quatre dans la section \ref{sec3} et de manière non perturbative dans la section \ref{sec4} ; à notre surprise, nous y sommes contraints par l'échec de l'estimation (\ref{eq028}) pour $n\geq 2$. Nous concluons dans la section \ref{sec5}.


\section{Expérience numérique dans le modèle de champ classique}
\label{sec1}

Les études expérimentales du son dans des gaz d'atomes froids superfluides bidimensionnels se sont concentrées pour l'instant sur le régime hydrodynamique \cite{HM,PCZH} ou, dans le régime faiblement collisionnel, sur l'intervalle de température $\epsilon=k_B T/mc^2\geq 1$ \cite{JDJB,Salasnich,Stringari}. Pour vérifier qu'en dimension $d=2$, la limite de basse température (\ref{eq020}) ne suffit pas à rendre exacte la décroissance exponentielle prédite par la règle d'or de Fermi, nous effectuons donc dans cette section une étude numérique de l'amortissement des phonons à tous les ordres en le couplage $H_3$ dans le modèle de champ classique. Nous rappelons d'abord l'hamiltonien de l'hydrodynamique quantique de Landau et Khalatnikov. Nous expliquons ensuite comment intégrer numériquement sa version de champ classique et introduisons une fonction de corrélation plus commode à calculer que le signal (\ref{eq012}) mais presque équivalente. Nous analysons ensuite les résultats obtenus pour différentes forces des interactions. 

\subsection{Hamiltonien de l'hydrodynamique quantique}
\label{sec1.1}

L'hydrodynamique quantique est une théorie effective de basse énergie qui permet d'obtenir des prédictions exactes à l'ordre dominant en température, même pour un fluide en interaction forte \cite{LK}. Le fluide est décomposé en petits éléments de taille $b$ tout de même mésoscopique ($b\gg\xi=\hbar/mc$). Chaque élément, de centre $\rr$, est (i) suffisamment grand pour qu'il comporte un grand nombre de particules ($\rho b^d\gg 1$) ce qui permet de lui attribuer un opérateur phase $\hat{\phi}(\rr)$ canoniquement conjugué à son opérateur nombre de particules $\hat{\rho}(\rr)b^d$, $[\hat{\rho}(\rr)b^d,\hat{\phi}(\rr)]=\ii\delta_{\rr,\rr'},$ et (ii) suffisamment petit pour être considéré aussi bien comme homogène ($b\ll\lambda_{\rm th}$ où $\lambda_{\rm th}=2\pi\hbar c/k_B T$ est la longueur d'onde de de Broglie d'un phonon thermique du fluide) que dans son état fondamental ($k_B T\ll\veps_{\rm exc}^{\rm min}$ où $\veps_{\rm exc}^{\rm min}=\hbar c (2\pi/b)$ est l'énergie d'excitation phononique minimale dans le petit élément).\footnote{La coupure ultraviolette à venir correspond peu ou prou au choix $\pi/b\approx \eta k_B T/\hbar c$. Même dans ce cas, $b$ reste très inférieure à la longueur de cohérence du fluide à suffisamment basse température. C'est évident en $d=3$ à cause de la présence d'un condensat. C'est vrai aussi en $d=2$ où la fonction de cohérence du premier ordre $g_1(\rr)$ décroît à grande distance comme $\exp[mk_B T\ln(\xi/r)/(2\pi\hbar^2\rho_s)]$ où $\rho_s\simeq \rho$ est la densité superfluide \cite{Bere,Nel,SJ}. Ceci rend plausible l'hypothèse $\hat{\vv}=(\hbar/m)\mathbf{grad}\,\hat{\phi}$ utilisée dans la référence \cite{LK} et dans l'équation (\ref{eq050}).} L'hamiltonien est alors somme des énergies cinétiques du centre de masse et de l'énergie interne de chaque petit élément:
\be
\label{eq050}
H_{\rm hydro}=\sum_\rr \frac{1}{2} m \hat{\vv}(\rr)\cdot\hat{\rho}(\rr) b^d\hat{\vv}(\rr)+b^d e_0(\hat{\rho}(\rr))
\ee
où $\hat{\vv}=(\hbar/m)\mathbf{grad}\,\hat{\phi}$ est le champ de vitesse du fluide et $e_0(\rho)$ est la densité volumique d'énergie dans l'état fondamental à la densité $\rho$. À basse température, le champ de densité $\hat{\rho}(\rr)$ fluctue peu et le champ de phase $\hat{\phi}(\rr)$ varie lentement \cite{broui}; on les sépare en leurs composantes de Fourier de vecteur d'onde nul $\hat{\rho}_0=\hat{N}/L^d\simeq \rho$ et $\hat{\phi}_0$, et en des fluctuations spatiales $\delta\hat{\rho}(\rr)$ et $\delta\hat{\phi}(\rr)$ admettant les développements modaux
\be
\label{eq052}
\delta\hat{\rho}(\rr)=\frac{1}{L^{d/2}} \sum_{\kk\neq\zero} \left(\frac{\hbar\rho k}{2mc}\right)^{1/2}(\hat{b}_\kk+\hat{b}_{-\kk}^\dagger)\,\eee^{\ii\kk\cdot\rr} \quad ; \quad \delta\hat{\phi}(\rr)=\frac{-\ii}{L^{d/2}}  \sum_{\kk\neq\zero} \left(\frac{mc}{2\hbar\rho k}\right)^{1/2} (\hat{b}_\kk-\hat{b}_{-\kk}^\dagger)\,\eee^{\ii\kk\cdot\rr}
\ee
où les $\hat{b}_\kk^\dagger$ et les $\hat{b}_\kk$ sont les opérateurs de création et d'annihilation d'un phonon de vecteur d'onde $\kk$ dans la boîte de quantification $[0,L]^d$, $c$ est la vitesse du son à température nulle, donnée par $mc^2=\rho(\dd\mu/\dd\rho)$ avec $\mu=\dd e_0/\dd\rho$ le potentiel chimique du fluide dans l'état fondamental. Comme dans toute théorie effective, il faut mettre une coupure ultraviolette dans l'espace des vecteurs d'onde, que nous prenons ici isotrope
\be
\label{eq103}
\hbar c k < \eta k_B T
\ee
et à laquelle les sommes (\ref{eq052}) sont soumises; on fera tendre la température vers zéro comme dans (\ref{eq020}) puis $\eta\to+\infty$ (dans le cas quantique) sans déclencher de divergence si tout va bien. Il reste à développer $H_{\rm hydro}$ en puissances de $\delta\hat{\rho}$ et $\mathbf{grad}\,\delta\hat{\phi}$. L'ordre zéro fournit une constante sans intérêt. L'ordre un est nul par construction.  L'ordre deux, diagonalisé par la transformation (\ref{eq052}), redonne l'hamiltonien $H_2$ de l'équation (\ref{eq002}) pour une relation de dispersion linéaire, la courbure dans l'équation (\ref{eq003}) devant être mise à la main, voir les références \cite{LK,Annalen} et notre note \ref{note1}. Les ordres trois et quatre, donnant le couplage entre phonons dans l'équation (\ref{eq001}), s'écrivent en point de vue $\rr$:
\be
\label{eq104}
H_3=b^d\sum_\rr \frac{1}{2} m\hat{\vv}(\rr)\cdot\delta\hat{\rho}(\rr)\hat{\vv}(\rr)+\frac{1}{6}\frac{\dd^2\mu}{\dd\rho^2}\delta\hat{\rho}(\rr)^3 \quad ; \quad H_4=b^d\sum_\rr\frac{1}{24} \frac{\dd^3\mu}{\dd\rho^{3}}\delta\hat{\rho}(\rr)^4
\ee
sachant que $\mathbf{grad}\,\hat{\phi}=\mathbf{grad}\,\delta\hat{\phi}$. Ils font apparaître les constantes de couplage adimensionnées
\be
\label{eq032}
\Lambda=\rho\frac{\dd^2\mu}{\dd\rho^2}/(3\frac{\dd\mu}{\dd\rho})=\rho^2\frac{\dd^2\mu}{\dd\rho^2}/(3 mc^2)\quad\mbox{et}\quad \Lambda_*=\rho^3\frac{\dd^3\mu}{\dd\rho^3}/(mc^2)
\ee
Pour révéler les processus physiques contenus dans $H_3$ et déjà mentionnés dans la section \ref{sec0}, on injecte dans (\ref{eq104}) les développements modaux (\ref{eq052}) et on regroupe les termes de même type, respectivement Belyaev, Landau et non résonnants:
\be
\label{eq100}
H_3 = H_3^{(-)} + H_3^{(+)} + H_3^{\mbox{\scriptsize (non rés.)}}
\ee
avec
\be
\label{eq101}
H_3^{(-)}= (H_3^{(+)})^\dagger = \frac{1}{2 L^{d/2}}\sum_{\kk,\kk'} \mA(\kk,\kk';\kk+\kk') \hat{b}_{\kk+\kk'}^\dagger \hat{b}_\kk \hat{b}_{\kk'} \quad ; \quad H_3^{\mbox{\scriptsize (non rés.)}} = \frac{1}{6 L^{d/2}} \sum_{\kk,\kk'} \mA(\kk,\kk';-(\kk+\kk')) (\hat{b}_{\kk}\hat{b}_{\kk'}\hat{b}_{-(\kk+\kk')}+\mbox{h.c.})
\ee
et l'amplitude de couplage\footnote{Par choix de notation, notre amplitude $\mA$  diffère d'un facteur deux de celle de la référence \cite{Annalen}.}
\be
\label{eq102}
\mA(\kk,\kk';\kk'')=mc^2\left(\frac{\hbar^3kk'k''}{8\rho m^3c^3}\right)^{1/2}\left(3\Lambda+\frac{\kk\cdot\kk'}{k k'}+\frac{\kk\cdot\kk''}{k k''}+\frac{\kk'\cdot\kk''}{k' k''}\right)
\ee
Bien entendu, tous les vecteurs d'onde $\kk$, $\kk'$, $\pm(\kk+\kk')$ apparaissant dans l'hamiltonien sont non nuls et soumis à la condition (\ref{eq103}). L'écriture de $H_4$ en point de vue $\hat{b}$ et $\hat{b}^\dagger$ ne présente guère d'intérêt ici et nous en faisons l'économie.

\subsection{Modèle de champ classique}
\label{sec1.2}

Le modèle est construit en remplaçant les opérateurs $\hat{b}_\kk$ et $\hat{b}_\kk^\dagger$ dans la section \ref{sec1.1} par des amplitudes complexes $b_\kk$ et $b_\kk^*$. Ceci n'a de sens que pour les modes de phonons fortement peuplés, ce qui impose un choix de coupure $\eta\lesssim 1$ dans l'équation (\ref{eq103}); dans les simulations, nous prendrons $\eta=1$. À l'équilibre thermique, les nombres d'occupation des modes ne sont plus donnés par la loi de Bose mais par l'équipartition $\bar{n}_\kk^{\rm clas}=k_B T/\veps_\kk$. 

Limitons notre étude numérique au cas expérimentalement pertinent d'un gaz de bosons sans spin dans le régime quasi 2D où le potentiel chimique $\mu$ dans le plan $xOy$ est inférieur à mais pas très petit devant le quantum d'excitation $\hbar\omega_\perp$ du mouvement selon $Oz$ ; dans ce cas, l'équation d'état prend la forme simple $\mu=\rho g$ avec $g=(8\pi)^{1/2}\hbar^2a_{\rm 3D}/(ma_\perp)$, où $a_{\rm 3D}$ est la longueur de diffusion à 3D et $a_\perp=(\hbar/m\omega_\perp)^{1/2}$ la taille de l'oscillateur harmonique selon $Oz$ \cite{CD,HD}, si bien que $\Lambda=\Lambda_*=0$ dans l'équation (\ref{eq032}) et $H_4\equiv 0$ dans l'équation (\ref{eq001}).\footnote{L'équation d'état de Popov d'un gaz de bosons strictement bidimensionnel dans le régime d'interaction faible correspond à une pseudo-constante de couplage dépendant du potentiel chimique, $g(\mu)=4\pi\hbar^2/[m\ln(\veps_0/\mu)]$ où $\veps_0=4\hbar^2/[ma_{\rm 2D}^2\exp(2C+1)]$, $a_{\rm 2D}$ est la longueur de diffusion à 2D et $C=0,\!577\ldots$ est la constante d'Euler \cite{Popov,Morathese}. Dans son domaine de validité \cite{MC2}, elle conduit à des valeurs non nulles mais faibles de $\Lambda$ et $\Lambda_*$, par exemple $\Lambda=mg/(12\pi\hbar^2)/[1-mg/(4\pi\hbar^2)]^2\simeq mg/(12\pi\hbar^2) \ll 1$.} 

La méthode de simulation numérique est décrite en détail dans la référence \cite{TheseAlan}. Pour échantillonner l'état initial, on tire les $b_\kk$ selon la distribution thermique (gaussienne) pour l'hamiltonien $H_2$, puis on fait évoluer le champ résultant avec l'hamiltonien complet $H$ pour atteindre le vrai équilibre thermique. L'évolution temporelle sur chaque pas $dt$ est séparée en l'évolution due à $H_2$ pendant $dt/2$, l'évolution due à $H_3$ seul pendant $dt$ puis à nouveau l'évolution due à $H_2$ pendant $dt/2$. L'évolution due à $H_2$ se calcule exactement dans l'espace des $\kk$. L'évolution due à $H_3$ est faite dans l'espace des $\rr$\footnote{Comme les équations du mouvement dérivant de $H_3$ sont locales dans l'espace des $\rr$ (ou quasi locales puisqu'elles font intervenir des gradients spatiaux), la complexité sur un pas $dt$ est de degré un en le nombre de points $\mathcal{N}$ sur la grille numérique ; dans l'espace des $\kk$, elles sont fortement non locales (chaque amplitude $b_\kk$ est couplée en gros à toutes les autres) d'où une complexité catastrophique $\propto \mathcal{N}^2$.} par la méthode de Runge-Kutta d'ordre quatre \cite{NumRec}, en pensant à filtrer le champ dans l'espace des $\kk$ comme dans l'équation (\ref{eq103}) après chaque action de $H_3$ dans l'espace des $\rr$ pour s'affranchir des processus violant la coupure qu'elle a induits.\footnote{Le passage de $\rr$ à $\kk$ est effectué par une transformation de Fourier rapide, dans laquelle les vecteurs d'onde décrivent l'ensemble $\mD\cap(2\pi/L)\mathbb{Z}^d$ avec $\mD=[-k_{\rm max},k_{\rm max}[^d$ (chaque composante cartésienne $k_\alpha$ de $\kk$ a un sens modulo $2 k_{\rm max}$, si bien que $\mD$ peut être considéré comme la première zone de Brillouin du réseau spatial $(\pi/k_{\rm max})\mathbb{Z}^d$, qui est précisément celui décrit par $\rr$, du moins après restriction à $[0,L[^d$). Le choix naïf $k_{\rm max}=\eta k_B T/\hbar c$ associé à la coupure (\ref{eq103}) ne convient pas: les processus de $H_3$ faisant sortir les phonons de la première zone de Brillouin correspondante sont repliés dans cette zone (processus Umklapp) et ne peuvent plus être filtrés. Il faut utiliser au minimum $k_{\rm max}=2\eta k_B T/\hbar c$: le filtrage nous oblige à doubler le nombre de points par dimension.}  Le pas $dt$ doit être assez petit pour que l'énergie totale sur une seule réalisation soit conservée à mieux qu'un pour cent de l'énergie d'interaction typique $\langle |H_3|\rangle$, où $\langle\ldots\rangle$ est la moyenne sur un grand nombre de réalisations, elle-même une très faible fraction de l'énergie totale. 


Il serait inefficace numériquement de déterminer l'amortissement des phonons par simulation de l'excitation de Bragg (\ref{eq010}) car ceci obligerait à refaire un grand nombre de simulations indépendantes pour chaque nouvelle valeur du vecteur d'onde $\qq$, tout en s'assurant que l'amplitude $\alpha$ est assez faible pour qu'on soit en régime de réponse linéaire.\footnote{La version de champ classique $\langle\partial b_\qq(t)/\partial b_\qq(0)\rangle_{\rm th}$ de la fonction de corrélation $\langle[\hat{b}_\qq(t),\hat{b}_\qq^\dagger(0)]\rangle_{\rm th}$, qui fait intervenir une dérivée numérique dans le mode particulier $\qq$, est confrontée aux mêmes problèmes.} Nous préférons utiliser la fonction de corrélation suivante, permettant de traiter en parallèle toutes les valeurs possibles de $\qq$ par simple moyenne sur un même jeu de réalisations indépendantes:\footnote{Le passage au troisième membre de l'équation (\ref{eq110}) est approché car nous avons remplacé au dénominateur $\langle\hat{b}_\qq(0)\hat{b}_\qq^\dagger(0)\rangle_{\rm th}$ par $1+\bar{n}_{\qq}$ où $\bar{n}_{\qq}$ est la loi de Bose à l'énergie $\veps_{\qq}$, ce qui est  une excellente approximation pour nos simulations.}
\be
\label{eq110}
s_*(t)\equiv \frac{\eee^{\ii\veps_\qq t/\hbar}\langle\hat{b}_\qq(t)\hat{b}_\qq^\dagger(0)\rangle_{\rm th}}{\langle\hat{b}_\qq(0)\hat{b}_\qq^\dagger(0)\rangle_{\rm th}}\stackrel{t>0}{\simeq} \lim_{\delta\to 0^+} \int_{-\infty}^{+\infty} 
\frac{\dd\veps}{-2\ii\pi} \eee^{-\ii(\veps-\veps_\qq)t/\hbar} \frac{\mathcal{G}(\qq,\veps+\ii\delta)+\left[\mathcal{G}(\qq,\veps+\ii\delta)-\mathcal{G}(\qq,\veps-\ii\delta)\right]/[\exp(\veps/k_B T)-1]}{1+1/[\exp(\veps_\qq/k_B T)-1]}
\ee
Il se trouve que ce signal est en pratique très proche de celui de l'équation (\ref{eq012}),
\be
\label{eq111}
s_*(t) \simeq s(t)
\ee
En effet, même pour l'interaction la plus forte considérée dans nos simulations, on dispose, pour toute valeur de l'énergie $\veps$, de la propriété $|\im\Sigma(\qq,\veps+\ii 0^+)|\ll \veps_\qq$, si bien que la différence entre les fonctions de Green $\mG(\qq,\veps\pm\ii\delta)$ entre crochets dans l'équation (\ref{eq110}), qui ne peut résulter que d'une partie imaginaire non nulle de la fonction énergie propre dans la limite $\delta\to 0^+$, est très petite sauf dans la largeur de la \g{lorentzienne} $\mathcal{G}(\qq,\veps+\ii 0^+)$ où $|\veps-\veps_\qq|\lesssim |\im\Sigma(\qq,\veps+\ii 0^+)|\ll \veps_\qq, k_B T$. C'est donc une excellente approximation que de remplacer $\veps$ par $\veps_\qq$ dans la loi de Bose au numérateur de l'intégrande dans (\ref{eq110}). Par analyticité dans le demi-plan complexe inférieur, le terme proportionnel à $\mathcal{G}(\qq,z-\ii\delta)$ donne alors une contribution nulle à l'intégrale (on le voit en refermant le chemin d'intégration par un demi-cercle à l'infini et en utilisant le théorème de Cauchy) et on aboutit à (\ref{eq111}). La même conclusion vaut dans le modèle de champ classique, où les deux fonctions exponentielles dans (\ref{eq110}) sont linéarisées.


\subsection{Les résultats numériques}
\label{sec1.3}

% Figure environment removed

% Figure environment removed

Sur la figure \ref{fig1}, nous portons l'inverse du temps de décroissance à $1/\eee$ du signal en fonction du nombre d'onde réduit $\qb=\hbar cq/k_B T$ du mode considéré, pour quatre valeurs de la force des interactions repérée par $1/(\rho\xi^2)$ et pour une ou deux valeurs de $\epsilon=k_B T/mc^2$, voir les disques. L'unité de temps choisie est $2\Gamma_u^{-1}$, où la constante $\Gamma_u$ apparaît en préfacteur dans le taux d'amortissement de la règle d'or de Fermi à l'ordre dominant en $\epsilon$, ici pour le champ classique:\footnote{Dans l'expression (\ref{eq033}) pour le champ quantique, on remplace $\bar{n}_k^{\rm lin}$ par $1/\kb$, on néglige le terme $1/2$ devant $\bar{n}_k^{\rm lin}$ dans la première intégrale et on met la coupure $\kb<\eta-\qb$ dans la seconde.}
\be
\label{eq113}
\boxed{\Gamma_q^{\mbox{\scriptsize clas}}=\Gamma_u \qb(\eta-\qb/2) \quad \mbox{avec}\quad \Gamma_u = \frac{9k_B T\epsilon^2(1+\Lambda)^2}{4\pi\hbar(3\gamma)^{1/2}\rho\xi^2}}
\ee
Comme on le voit sur les panneaux (a) et (d) de la figure \ref{fig1}, les résultats ainsi normalisés ne dépendent pas significativement de $\epsilon$ à $\qb$ fixé et la limite $\epsilon\to 0$ semble atteinte ; dans les simulations, l'inverse du temps de décroissance à $1/\eee$ varie donc bien comme $k_B T\epsilon^2$ à basse température, comme le prédisaient la règle d'or et plus généralement l'estimation (\ref{eq028}) à tous les ordres en $H_3$. En revanche, la dépendance en $\qb$ du temps d'amortissement est fort mal décrite par la règle d'or, en tireté sur la figure \ref{fig1}: comme le prévoyait le raisonnement général de la section \ref{sec0}, la limite $\epsilon\to 0$ ne suffit donc pas à 2D à faire entrer le gaz de phonons dans un régime de couplage faible. Ce que la section \ref{sec0} ne disait pas, cependant, c'est que la règle d'or reste insuffisante même si le superfluide sous-jacent entre dans le régime d'interaction faible $(\rho\xi^2)^{-1}\to 0$, voir le panneau (d) où un calcul perturbatif à l'ordre dominant en $H_3$ devrait a priori être légitime; le départ à faible $\qb$ prédit par (\ref{eq113}) y est d'ailleurs particulièrement mauvais. 

Pour rendre l'insuffisance de la règle d'or encore plus criante, nous représentons sur la figure \ref{fig2}, en fonction du temps, l'écart du signal numérique à l'approximation du pôle (\ref{eq014}) limitée à l'ordre deux en $H_3$ [voir l'équation (\ref{eq124})] et traduite en champ classique (c'est-à-dire, si l'on était à l'ordre dominant en $\epsilon$, l'écart à l'exponentielle $\exp(-\Gamma_q^{\rm clas}t/2)$ de la règle d'or), pour les mêmes forces d'interaction que précédemment mais pour une valeur fixée du nombre d'onde réduit, $\qb=1/4$, suffisamment grande pour rendre aisée l'atteinte de la limite thermodynamique dans les simulations.\footnote{Plus $\qb$ est petit, plus le taux d'amortissement du mode est faible et plus la granularité des niveaux d'énergie du gaz de phonons dans la boîte de quantification se fait sentir, et plus il faut augmenter la taille du système pour atteindre la limite thermodynamique.} Là aussi, la limite $\epsilon\to 0$ est atteinte dans les simulations (lorsque le temps $t$ est mis à l'échelle comme dans (\ref{eq022}) avec $\nu=2d-2=2$, ici en l'exprimant en unités de $2/\Gamma_q^{\rm clas}$), l'écart à la règle d'or est bien là et persiste même dans la limite d'interaction faible $(\rho\xi^2)^{-1}\to 0$, du moins sur la partie réelle.\footnote{Si l'on passe du panneau (c) au panneau (d) sur la figure \ref{fig2}, la force des interactions est divisée par $\simeq 3$, la partie imaginaire est divisée par $\simeq 2$ en valeur absolue maximale donc pourrait bien tendre vers zéro, alors que la partie réelle ne décroît que d'un facteur $\simeq 1,\!3$.}

\section{Étude à l'ordre dominant en le couplage phonon-phonon}
\label{sec2}

Les simulations de champ classique de la section \ref{sec1} ont établi l'insuffisance de la règle d'or de Fermi dans la description de l'amortissement des phonons en dimension $d=2$; nous ne l'avons pas montré ici, mais outre le désaccord sur le temps de décroissance à $1/\eee$, elles révèlent, en échelle logarithmique, une décroissance non exponentielle du signal $s(t)$ \cite{TheseAlan}. Dans cette section, nous incriminons l'approximation faite dans l'étape (i) de la section \ref{sec0}, et y renonçons en gardant toute la dépendance en énergie de la fonction énergie propre dans l'expression intégrale (\ref{eq012}) de $s(t)$, en continuant cependant à calculer $\Sigma_\qq(\zeta)$ à l'ordre dominant en le couplage phonon-phonon et en $\epsilon$ (comme dans l'étape (ii) de la section \ref{sec0}). Après un calcul diagrammatique de $\Sigma^{(2)}$, nous obtenons quelques prédictions analytiques mises sous une forme universelle puis nous comparons aux simulations.

\subsection{Calcul de la fonction énergie propre}
\label{sec2.1}

Dans l'hamiltonien $H$ de l'équation (\ref{eq001}), le couplage cubique $H_3$ semble prépondérant par rapport au couplage quartique $H_4$, puisqu'il apparaît à l'ordre $\delta\hat{\rho}^3$ plutôt que $\delta\hat{\rho}^4$ dans le développement de l'hamiltonien hydrodynamique, voir la section \ref{sec1.1}. Cependant, comme il ne conserve pas le nombre de phonons, il doit être traité dans le calcul de $\Sigma_\qq(\zeta)$ au minimum au second ordre de la théorie des perturbations, alors que $H_4$ donne une contribution non nulle dès le premier ordre, à première vue aussi importante que celle de $H_3$.\footnote{Là où apparaît dans la contribution de $H_3$ le rapport $\langle\ |\mH_3|\ \rangle^2/\Delta E\approx \delta\hat{\rho}^{2\times 3-2}$, intervient dans celle de $H_4$ l'élément de matrice $\langle\ |\mH_4|\ \rangle\approx \delta\hat{\rho}^4$, si l'on utilise les notations de l'équation (\ref{eq024}) et le fait que le dénominateur $\Delta E$, constitué de différences d'énergie de l'hamiltonien quadratique $H_2$, est $\approx\delta\hat{\rho}^2$.} Le calcul complet s'impose.

Au premier ordre de la théorie des perturbations, il n'est point besoin d'utiliser de diagramme de Feynman, la contribution à $\Sigma_\qq(\zeta)$ étant indépendante de la variable $\zeta$ et s'obtenant par la procédure simple suivante, justifiable par la méthode de l'équation pilote de la référence \cite{SCE}. On écrit l'opérateur de la perturbation, ici $H_4$, en point de vue $\hat{b},\hat{b}^\dagger$ et dans l'ordre normal en s'aidant des relations de commutation bosoniques (tous les $\hat{b}^\dagger$ mis à gauche, tous les $\hat{b}$ mis à droite). Les termes ne conservant pas le nombre de phonons ne contribuent pas. Les termes en $\hat{b}^\dagger\hat{b}$ sont absorbés dans une redéfinition des énergies propres nues $\veps_\kk$ des modes (comme nous y préparait la note \ref{note0}). Enfin, on isole dans les \g{vrais} termes de couplage quartique les contributions en $\hat{b}^\dagger_\qq\hat{b}^\dagger_\kk\hat{b}_\kk\hat{b}_\qq$ dont la moyenne thermique sur les modes $\kk\neq\qq$ donne le déplacement d'énergie du mode $\qq$. Tout ceci se résume par l'expression compacte
\be
\label{eq120}
\Sigma_\qq^{(H_4)}(\zeta)=\int\frac{\dd^dk}{(2\pi)^d} \langle\qq,\kk|\mathcal{H}_4^{\rm vrai}
%^{\hat{b}^\dagger\hat{b}^\dagger\hat{b}\hat{b}}
|\qq,\kk\rangle \bar{n}_k = \frac{\hbar^2q\Lambda_*}{4m\rho}\int\frac{\dd^dk}{(2\pi)^d} k\bar{n}_k \underset{\epsilon\to 0}{\approx} \frac{\bar{q}k_B T\epsilon^{d+1}}{\rho\xi^d}
\ee
où $\mH_4$ est l'hamiltonien volumique tel que $H_4=\mH_4/L^{d}$, la constante de couplage $\Lambda_*$ est celle de l'équation (\ref{eq032}), nous avons déduit l'élément de matrice dans l'état à deux phonons de l'expression (\ref{eq104}) de $H_4$ (\ref{eq052}) de $\delta\hat{\rho}(\rr)$ et nous avons obtenu le comportement d'échelle dans la limite (\ref{eq020}) au quatrième membre de l'équation (\ref{eq120}) comme dans la section \ref{sec0}. Comme nous le voyons sur la contribution (\ref{eq028}) de $H_3$ écrite pour $n=1$, la contribution de $H_4$ en dimension $d=2$ (mais pas en dimension trois) est sous-dominante et peut donc être négligée lorsque $\epsilon\to 0$. C'est ce que nous ferons désormais.

%%%%%%%%%%%%%%%%%%%%%%%%%%%
% FIGURA 1BOUCLE
%%%%%%%%%%%%%%%%%%%%%%%%%%%
% Figure environment removed

Au second ordre en $H_3$, il vaut mieux avoir recours à la méthode de la référence \cite{FW}.\footnote{Nos états de phonons ayant directement la symétrie bosonique, nous pouvons faire l'économie des lignes d'interaction directe et d'échange introduites dans \cite{FW}.} Cette méthode donne un développement diagrammatique de la fonction énergie propre $\Sigma(\qq,z)$ pour des valeurs particulières de la variable énergie complexe, $z=\Omega\in 2\ii\pi k_B T \mathbb{Z}$, où $\Omega$ est appelée énergie de Matsubara,\footnote{Petite entorse à la notation habituelle \cite{FW}, nos variables de Matsubara $\Omega$, $\omega$, etc., ne sont pas des fréquences réelles mais des énergies imaginaires pures, c'est-à-dire que nous y incorporons un facteur $\ii\hbar$ supplémentaire.} que l'on complète dans un deuxième temps par un prolongement analytique à $\mathbb{C}\setminus\mathbb{R}$ \cite{FW}. Ici, comme sur la figure \ref{fig6}, chaque diagramme doit comporter deux sommets (nous sommes au second ordre), chaque sommet voyant se rejoindre trois lignes de phonons ($H_3$ est cubique en les $\hat{b},\hat{b}^\dagger$). La seule topologie possible est à une boucle et, puisqu'il faut considérer comme identiques deux boucles se déduisant l'une de l'autre par rotation d'angle $\pi$ autour leur axe, les quatre orientations possibles des deux lignes internes ne conduisent qu'à trois diagrammes distincts, représentés dans l'ordre Belyaev, Landau et non résonnant sur la figure \ref{fig6} (pour reprendre la typologie de l'équation (\ref{eq100})). Compte tenu de la conservation de l'énergie de Matsubara et de l'impulsion à chaque sommet (la somme des quantités entrantes doit égaler celle des quantités sortantes), il reste une seule énergie de Matsubara $\omega$ et un seul vecteur d'onde $\kk$ indépendants, sur lesquels il faut sommer. Les règles de Feynman dans la section 25 de la référence \cite{FW} conduisent alors pour notre gaz de phonons de potentiel chimique nul à
\be
\label{eq121}
\Sigma^{(2)}(\qq,\Omega) = -k_B T \sum_{\omega\in 2\ii\pi k_B T\mathbb{Z}}\int\frac{\dd^dk}{(2\pi)^d} \Bigg\{\frac{(1/2)|\langle\kk,\qq-\kk|\mathcal{H}_3|\qq\rangle|^2}{(\omega-\veps_\kk)(\Omega-\omega-\veps_{\qq-\kk})}+\frac{|\langle\kk+\qq|\mathcal{H}_3|\kk,\qq\rangle|^2}{(\omega-\veps_\kk)(\Omega+\omega-\veps_{\kk+\qq})}+\frac{(1/2)|\langle 0|\mathcal{H}_3|\qq,\kk,-(\kk+\qq)\rangle|^2}{(\omega-\veps_\kk)(-\Omega-\omega-\veps_{-(\kk+\qq)})}
\Bigg\}
\ee
avec, pour chaque contribution, un domaine d'intégration en $\kk$ assurant que les vecteurs d'onde mis en jeu $\kk$, $\qq-\kk$, etc., satisfont à la coupure (\ref{eq103}). Sont inclus aussi les facteurs de symétrie $1/2$ lorsque les deux branches de la boucle ont la même orientation et sont donc physiquement identiques. Les sommes de Matsubara se calculent au moyen de l'expression générale \footnote{\label{notedemo} La démonstration se fait très bien par décomposition du sommande en éléments simples vis-à-vis de la variable $n$ et utilisation de l'identité $\lim_{N\to +\infty} \sum_{n=-N}^{N} 1/(2\ii\pi n-\vepsb)=-(\bar{n}_\veps+1/2)\ \forall \veps\in\mathbb{R}^*$ résultant des propriétés de la fonction digamma $\psi(z)$.}
\be
\label{eq122}
\sum_{n\in\mathbb{Z}} \frac{1}{(2\ii\pi n-\vepsb_1)(2\ii\pi n-\vepsb_2)} = -\frac{\bar{n}_{\veps_1}-\bar{n}_{\veps_2}}{\vepsb_1-\vepsb_2}
\ee
où les énergies réelles (mais pas forcément positives) $\veps_j$ sont deux à deux distinctes, on a posé $\vepsb=\veps/k_B T$ et $\bar{n}_\veps=1/(\exp\vepsb-1)$. On utilise ensuite le fait que $\Omega/k_B T\in 2\ii\pi\mathbb{Z}$ pour faire disparaître cette variable de la loi de Bose, et la relation $\bar{n}_{-\veps}=-(1+\bar{n}_\veps)$ pour se ramener à de vrais nombres d'occupation des modes (à une énergie positive). Ainsi, par exemple:
\be
\label{eq123}
\bar{n}_{\Omega-\veps_{\qq-\kk}}=\bar{n}_{-\veps_{\qq-\kk}}=-(1+\bar{n}_{\qq-\kk})
\ee
L'intégrande devient une fonction rationnelle de la variable discrète $\Omega$, dont le prolongement analytique à $\mathbb{C}\setminus\mathbb{R}$ s'effectue par la simple substitution $\Omega\to z$. Il reste
\begin{multline}
\label{eq124}
\Sigma_\qq^{(2)}(\zeta)=\int\frac{\dd^dk}{(2\pi)^d}\Bigg[\frac{(1/2)|\langle\kk,\qq-\kk|\mathcal{H}_3|\qq\rangle|^2}{\zeta+\veps_\qq-(\veps_\kk+\veps_{\qq-\kk})}(\bar{n}_\kk+\bar{n}_{\qq-\kk}+1)+\frac{|\langle\kk+\qq|\mathcal{H}_3|\kk,\qq\rangle|^2}{\zeta+\veps_\qq+\veps_\kk-\veps_{\qq+\kk}}(\bar{n}_\kk-\bar{n}_{\qq+\kk})\\
-\frac{(1/2)|\langle 0|\mathcal{H}_3|\qq,\kk,-(\kk+\qq)\rangle|^2}{\zeta+\veps_\qq+\veps_{-(\kk+\qq)}+\veps_\kk}(\bar{n}_\kk+\bar{n}_{-(\qq+\kk)}+1)\Bigg\}
\end{multline}
Il reste à passer à la limite $\epsilon\to 0$ à $\qb$ et $\zetat$ fixés comme dans les équations (\ref{eq020},\ref{eq021},\ref{eq022}) avec, rappelons-le, des exposants $\nu=\sigma=2$ en dimension $d=2$. Au travers de l'effet \g{petit dénominateur} décrit dans la section \ref{sec0}, chaque contribution à l'intégrale (\ref{eq124}) est dominée par les configurations où les trois vecteurs d'onde au dénominateur d'énergie sont quasi colinéaires et de même sens (à un angle $O(\epsilon)$ près). Dans la première contribution, il suffit d'utiliser le développement (\ref{eq025}) avec $\kk'=\qq-\kk$ (avoir $\kk$ et $\kk'$ quasi colinéaires et de même sens implique alors $k<q$); dans la seconde, on utilise (\ref{eq025}) avec $\kk'=\qq$ ce qui n'impose rien sur $k$; dans la troisième, on a affaire à une somme à coefficients positifs de trois énergies de phonons, qui ne peut donc jamais être un $O(k_B T\epsilon^2$) (elle est toujours $\approx k_B T\epsilon^0$) si bien que la contribution correspondante est sous-dominante et négligeable à basse température.\footnote{Après ces développements, l'intégrale angulaire est lorentzienne donc s'effectue très bien avec la formule de Cauchy en refermant le chemin d'intégration par un demi-cercle infini dans le plan complexe, après avoir effectué le changement de variable $\theta=\epsilon\tilde{\theta}$ et avoir envoyé les bornes d'intégration $\pm\pi/\epsilon$ sur $\tilde{\theta}$ à l'infini.} Il reste, en respectant l'ordre des contributions dans (\ref{eq124}), et en se limitant au demi-plan supérieur strict comme dans l'expression (\ref{eq012}) du signal:\footnote{\label{notegen} La valeur de $\tilde{\Sigma}_{\qb}^{(2,2)}(\zetat)$ dans le demi-plan inférieur strict s'en déduit par la relation générale valable à tous les ordres en le couplage entre phonons, $\left[\Sigma_\qq(\zeta)\right]^*=\Sigma_\qq(\zeta^*)$.}
\be
\label{eq125}
\boxed{\tilde{\Sigma}_{\qb}^{(2,2)}(\zetat)\stackrel{\im\zetat>0}{=}\frac{9(1+\Lambda)^2(2\qb)^{1/2}}{64\ii\pi\rho\xi^2} \Bigg\{\int_0^{\qb} \dd\kb\, \frac{[\kb(\qb-\kb)]^{3/2}(1+\bar{n}_k^{\rm lin}+\bar{n}_{q-k}^{\rm lin})}{\left[(3\gamma/8)\qb\kb(\qb-\kb)+\tilde{\zeta}\right]^{1/2}} +2 \int_0^{\eta-\qb} \dd\kb \frac{[\kb(\qb+\kb)]^{3/2}(\bar{n}^{\rm lin}_k-\bar{n}^{\rm lin}_{k+q})}{\left[(3\gamma/8)\qb\kb(\kb+\qb)-\tilde{\zeta}\right]^{1/2}}\Bigg\}}
\ee
où les nombres d'occupation repérés par l'exposant \g{lin} sont ceux (\ref{eq029}) des modes de phonons pour la relation de dispersion linéarisée. Au premier membre de (\ref{eq125}), le second entier $2$ dans l'exposant rappelle que $\Sigma^{(2)}$ est calculé à l'ordre deux en $\epsilon$ ; les tildes sur $\Sigma$ et $\zeta$ rappellent qu'un facteur $k_BT\epsilon^2$ a été sorti de ces quantités comme dans (\ref{eq021},\ref{eq022}). Enfin, malgré la menace toujours présente des fluctuations quantiques du champ, on peut prendre la limite d'une coupure $\eta$ infinie sans déclencher de divergence ultraviolette, donc sans avoir à renormaliser quelque quantité que ce soit. Alors, si l'on fait tendre $\zetat$ vers zéro par parties imaginaires positives dans (\ref{eq125}), on retrouve l'équation (\ref{eq033}).

\subsection{Études analytiques: universalité à un paramètre}
\label{sec2.2}

De façon remarquable, il est possible, par un simple changement de variable, de faire en sorte qu'il ne reste plus, dans la théorie à l'ordre dominant traitée dans cette section \ref{sec2}, qu'un seul paramètre $u$ contenant tous les paramètres physiques comme la force des interactions $1/\rho\xi^2$, le paramètre de courbure $\gamma$ et la constante de couplage $\Lambda$, à savoir\footnote{Pour qu'il en aille de même dans le modèle de champ classique, il faudrait de plus incorporer le paramètre de coupure $\eta$ dans $u$, c'est-à-dire introduire la quantité $u_{\rm clas}=\eta u$. C'est inutile ici car nous avons pris $\eta=1$ dans les simulations. À titre de curiosité, signalons pour le champ classique et dans la limite $\qb\ll 1$ que le signal $s^{(2,2)}(t)$ défini plus bas cesse de commencer à la valeur $1$ en $t=0$ (comme il se doit physiquement) si $u_{\rm clas}<1$, ce que nous relions mathématiquement à l'apparition d'un pôle de la fonction de Green sur l'axe réel. Par conséquent, il serait intéressant de savoir si tous les systèmes physiques bidimensionnels (même en interaction très forte, pourvu qu'elle soit à courte portée) conduisent à $u>1$.}
\be
\label{eq112}
u=\frac{\pi\rho\xi^2\gamma^{3/2}}{\sqrt{3}(1+\Lambda)^2}
\ee
En effet, les nouvelles définitions repérées par un accent tchèque,
\be
\label{eq130}
\check{t}=\frac{3\gamma\qb}{8u}\tilde{t}=\frac{3\gamma\epsilon^2 c q t}{8u}\quad ; \quad \zetac=\frac{8u}{3\gamma\qb}\zetat=\frac{8u\zeta}{3\gamma\epsilon^2\hbar c q} 
\ee
conduisent à une nouvelle fonction énergie propre et à une nouvelle expression du signal ne mettant visiblement en jeu que $u$ et les variables réduites $\qb$ et $\check{t}$:
\be
\label{eq131}
\check{\Sigma}^{(2,2)}_{\qb}(\check{\zeta}) = \frac{8u}{3\gamma\qb} \tilde{\Sigma}^{(2,2)}_{\qb}(\zetat) \stackrel{\im\zetac>0}{=} \frac{1}{\ii} \int_0^{\qb} \frac{\dd\kb}{2\qb} \frac{[\kb(\qb-\kb)]^{3/2} (1+\bar{n}^{\rm lin}_k+\bar{n}^{\rm lin}_{q-k})}{[\kb(\qb-\kb)+\check{\zeta}/u]^{1/2}}+\frac{1}{\ii} \int_0^{+\infty} \frac{\dd\kb}{\qb}\frac{[\kb(\qb+\kb)]^{3/2} (\bar{n}^{\rm lin}_k-\bar{n}^{\rm lin}_{q+k})}{[\kb(\qb+\kb)-\check{\zeta}/u]^{1/2}}
\ee
et
\be
\label{eq132}
s^{(2,2)}(t) = \int_{C_+} \frac{\dd\check{\zeta}}{2\ii\pi} \eee^{-\ii\check{\zeta}\check{t}} \check{\mathcal{G}}_{\qb}^{(2,2)}(\zetac)  \quad\mbox{avec}\quad \check{\mathcal{G}}_{\qb}^{(2,2)}(\zetac)=\frac{1}{\check{\zeta}-\check{\Sigma}^{(2,2)}_{\qb}(\check{\zeta})}
\ee
Nous parlons dans ce cas d'universalité à un paramètre.

%%%%%%%%%%%%%%%%%%%%%%%%%%%
% FIGURA CAMMINO
%%%%%%%%%%%%%%%%%%%%%%%%%%%
% Figure environment removed

Effectuons le prolongement analytique de la fonction (\ref{eq131}) du demi-plan supérieur au demi-plan inférieur à travers sa ligne de coupure sur l'axe réel.\footnote{Il y a ici une subtilité mathématique que le lecteur voudra bien nous pardonner. Implicitement, à partir de l'expression (\ref{eq131}) de $\check{\Sigma}_{\qb}^{(2,2)}(\zetac)$, valable pourtant seulement pour $\im\zetac>0$, nous définissons une fonction dans tout $\mathbb{C}\setminus\mathbb{R}$ en appliquant (\ref{eq131}) telle quelle pour $\im\zetac<0$; la ligne de coupure $[-u\qb^2/4,+\infty[$ mentionnée dans le texte et représentée sur la figure \ref{fig5}a correspond à cette fonction plutôt qu'à la vraie fonction énergie propre $\check{\Sigma}_{\qb}^{(2,2)}(\check{\zeta})$ dont la ligne de coupure est $\mathbb{R}$ tout entier et qui serait dès lors moins commode à prolonger analytiquement du demi-plan supérieur au demi-plan inférieur pour le même résultat final (on voit que c'est $\mathbb{R}$ tout entier (i) en considérant le dénominateur d'énergie $\zeta+\Delta E$ dans la contribution de Landau de l'équation (\ref{eq124}) et en vérifiant que $\Delta E$ décrit $\mathbb{R}$ lorsque $\theta/\epsilon$ et $\kb$ décrivent $\mathbb{R}^+$ dans (\ref{eq025}) avec $\kk'=\qq$, ou (ii) en utilisant la relation $\ii z^{1/2}=-\mathrm{signe}(\im z)(-z)^{1/2}$, $\forall z\in\mathbb{C}\setminus\mathbb{R}$, pour faire disparaître le facteur $\ii$ dans (\ref{eq131}) et obtenir une expression physiquement correcte sur tout $\mathbb{C}\setminus\mathbb{R}$ car satisfaisant à $\Sigma_{\qq}(\zeta)^*=\Sigma_{\qq}(\zeta^*)$, dont on montre ensuite que la ligne de coupure est $\mathbb{R}$).} Ceci va permettre, comme dans la référence \cite{CCTbordeaux}, de séparer le signal en la contribution d'un lacet autour d'une ligne de coupure déplacée (contribution décroissant en loi de puissance aux temps longs) et la contribution d'un pôle dans la fonction de Green ainsi prolongée (contribution décroissant exponentiellement en temps), voir notre figure \ref{fig5}b, ce qui améliore la compréhension physique. Il faut d'abord identifier les points de branchement (singularités immuables) de $\check{\Sigma}_{\qb}^{(2,2)}(\check{\zeta})$ sur l'axe réel. Comme les valeurs de $\check{\zeta}$ annulant l'argument de la racine carrée au dénominateur décrivent l'intervalle $[-u\qb^2/4,0]$ lorsque $\kb$ décrit $[0,\qb]$ dans la première contribution à (\ref{eq131}), et décrivent $[0,+\infty[$ lorsque $\kb$ décrit $\mathbb{R}^+$ dans la seconde, on pourrait s'attendre à avoir deux points de branchement, $\check{\zeta}=-u\qb^2/4$ et $\check{\zeta}=0$.\footnote{Par exemple, $\zetac\mapsto\int_a^b\dd x/(\zetac-x)^{1/2}=2(\zetac-b)^{1/2}-2(\zetac-a)^{1/2}$ admet les réels $a$ et $b$ comme points de branchement.} En réalité, par compensation subtile entre les deux contributions, seul $\zetac=-u\qb^2/4$ subsiste ; de ce point part la ligne de coupure $[-u\qb^2/4,+\infty[$ comme sur la figure \ref{fig5}a.\footnote{\label{note30} Si l'on revient à la variable $\zetat$, le point de branchement se trouve en $-3\gamma\qb^3/32$.} On peut alors faire tourner cette dernière d'un angle $\phi \in ]-\pi,-\pi/2[$ pour l'amener dans le troisième quadrant, par prolongement analytique de $\check{\Sigma}^{(2,2)}_{\qb}(\check{\zeta})$ du demi-plan supérieur au demi-plan inférieur, la nouvelle ligne de coupure étant la demi-droite $-u\qb^2/4+\exp(\ii\phi)\mathbb{R}^+$, voir la figure \ref{fig5}b. La fonction ainsi prolongée s'écrit, comme le montre l'\ref{app1} et en convenant que $z^{3/2}=z\sqrt{z}$:
\be
\label{eq140}
\check{\Sigma}^{(2,2)}_{\qb\downarrow}(\check{\zeta})= \frac{\eee^{3\ii\phi/2}}{\ii} \int_0^{+\infty} \frac{\dd\kb}{\qb} 
\frac{(\kb^2-\eee^{-\ii\phi}\qb^2/4)^{3/2}}{[\kb^2-\eee^{-\ii\phi}(\qb^2/4+\zetac/u)]^{1/2}} \Bigg[
\frac{1}{\exp(\eee^{\ii\phi/2}\kb-\qb/2)-1}-\frac{1}{\exp(\eee^{\ii\phi/2}\kb+\qb/2)-1} \Bigg]
\ee
En particulier, elle est analytique sur un voisinage de $\zetac=0$ (voir l'\ref{app1}, qui donne aussi sa valeur et celle de sa dérivée en ce point). En revanche, la fonction de Green $\check{\mG}_{\qb\downarrow}^{(2,2)}(\zetac)$ correspondante admet un pôle $\check{\zeta}_{\qb}$ dans le quatrième quadrant comme nous l'avons vérifié numériquement.\footnote{Si l'angle $\phi$ est trop proche de $-\pi/2$ (ce que nous éviterons dans la suite) et si $\qb$ est assez faible, un second pôle de partie imaginaire $<0$, d'interprétation physique incertaine, peut apparaître à gauche de la ligne de coupure déplacée.} 

Plutôt que d'évaluer numériquement les contributions du lacet et du pôle au signal, cherchons maintenant à extraire analytiquement de la forme (\ref{eq140}) la \g{substantifique moelle} de l'ordre deux en $H_3$, en remarquant par exemple qu'il est illusoire de calculer la position du pôle $\zetac_{\qb}$ à tous les ordres en $1/u$ puisque la fonction énergie propre est calculée ici seulement à l'ordre un en $1/u$. En revanche, se limiter à l'ordre zéro en $1/u$, ce qui reviendrait à prendre la limite $u\to+\infty$ à $\qb$ fixé, serait trop pauvre, car le signal se réduirait alors à l'exponentielle de la règle d'or de Fermi. On le voit bien sur l'écriture (\ref{eq131}), la nouvelle fonction énergie propre dépend de $u$ seulement au travers du rapport $\zetac/u$ ; aussi tend-elle vers sa valeur en $\ii 0^+$ lorsqu'on fait tendre $u$ vers $+\infty$ sous le signe intégral ($\zetac$ et $\check{t}$ étant fixés dans (\ref{eq132})). Un résultat plus intéressant s'obtient en prenant la limite $u\to+\infty$ à
\be
\label{eq150}
Q=u^{1/2}\qb \quad\mbox{et}\quad \check{t}\quad\mbox{fixés}
\ee
de façon que le point de branchement sur la figure \ref{fig5} reste fixe au lieu d'être rejeté à l'infini (ce qui ferait effectivement disparaître la contribution du lacet). On trouve alors une correction d'ordre $u^{-1/2}$ à la règle d'or de Fermi ; comme $u^{-1}\ll u^{-1/2}$ il semble entendu que l'on puisse garder cette correction sans avoir à calculer la fonction énergie propre à l'ordre suivant (l'ordre quatre) en $H_3$.\footnote{Nous verrons cependant avec surprise dans la section \ref{sec4} qu'il en va autrement.} Reportons les détails techniques dans l'\ref{app2} et donnons directement le résultat:
\be
\label{eq151}
s^{(2,2)}(t) \stackrel{Q\,\mbox{\scriptsize et}\,\check{t}\,\mbox{\scriptsize fixés}}{\underset{u\to+\infty}{=}}\eee^{-\pi^2\check{t}/3} + \frac{S(\check{t})}{u^{1/2}}+O(1/u) \quad\mbox{avec}\quad S(\check{t})=S_{\rm lacet}(\check{t})+S_{\mbox{\scriptsize pôle}}(\check{t})
\ee
La première contribution est l'exponentielle réelle de la règle d'or dans la limite $\qb\to 0$. La seconde est partagée en les contributions du lacet et du pôle établies dans l'\ref{app2}:
\be
\label{eq152}
S_{\rm lacet}(\check{t})= -\frac{\eee^{\ii Q^2\check{t}/4}}{\pi}\int_0^{\eee^{\ii(\phi+\pi)}(+\infty)} \dd z \frac{\eee^{\ii z \check{t}}\int_0^{\pi/2} \dd\alpha \left(z\sin^2\alpha+\frac{Q^2}{4}\right)^{1/2}}{(\frac{Q^2}{4}+z-\ii\frac{\pi^2}{3})^2} \quad ; \quad S_{\mbox{\scriptsize pôle}}(\check{t})=\eee^{-\pi^2\check{t}/3} (\delta \check{Z}-\ii \delta \check{\zeta}_{\qb}\check{t})
\ee
où l'angle $\phi$ est choisi librement dans $]-\pi,-\pi/2[$ (le résultat (\ref{eq152}) n'en dépend pas), $\delta\check{\zeta}_{\qb}$ est le premier écart du pôle à l'approximation du pôle $-\ii\pi^2/3$ (multiplié par $u^{1/2}$) et $\delta\check{Z}$ est le premier écart à un du résidu de $\check{\mG}_{\qb\downarrow}^{(2,2)}(\zetac)$ (multiplié lui aussi par $u^{1/2}$) dans un développement en puissances de $u^{-1/2}$:\footnote{Comme le signal (\ref{eq012}) commence exactement à un, il faut que $S(0)=0$ donc que la contribution du lacet pour $\check{t}=0$ compense exactement celle de l'écart à un du résidu du pôle. La vérification est non triviale. L'idée directrice est de ramener $S_{\rm lacet}(0)$ à une intégrale simple en intervertissant l'intégration sur $z$ et sur $\alpha$ dans (\ref{eq152}) (on connaît en effet une primitive de $(z+a)^{1/2}/(z+b)^2$ où $a$ et $b$ sont deux constantes).}
\be
\label{eq154}
\delta\check{\zeta}_{\qb} = \int_0^{+\infty} \dd K \left[1-\frac{(K^2+Q^2/4)^{1/2}}{(K^2-\ii\frac{\pi^2}{3}+\frac{Q^2}{4})^{1/2}}\right]
\quad;\quad 
\delta \check{Z}=\frac{1}{2} \int_0^{+\infty} \dd K \frac{(K^2+Q^2/4)^{1/2}}{(K^2-\ii\frac{\pi^2}{3}+\frac{Q^2}{4})^{3/2}} 
\ee
Sur la figure \ref{fig4}, nous représentons la fonction $S(\check{t})$ de l'équation (\ref{eq151}) pour $Q=1$, ainsi que les prédictions tirées numériquement de $\check{\Sigma}_{\qb}^{(2,2)}$ pour des valeurs finies de $u$. On constate que $u=20$ a déjà atteint le régime asymptotique; on s'attend donc à ce qu'il en aille de même pour les paramètres des simulations de champ classique de la figure \ref{fig2}d, qui correspondent à $u\simeq 18,\!14$ et $Q\simeq 1,\!06$. Ainsi, la fonction $S(\check{t})$ n'est pas qu'une curiosité mathématique et ses prédictions ne sont pas dénuées d'intérêt pratique. Gardons-en deux, dues au lacet: le comportement aux temps courts est en racine carrée, ce qui rend compte du départ rapide du signal sur la figure \ref{fig4}, et le comportement aux temps longs est sinusoïdal amorti comme l'inverse du temps, son raccordement forcément continu à celui aux temps courts expliquant les changements de signe sur cette même figure,\footnote{Si $\check{t}\to +\infty$, l'intégrale donnant $S_{\rm lacet}(\check{t})$ est dominée par les valeurs de $z$ proches de zéro et on peut remplacer $z$ par $0$ dans l'intégrande, partout sauf dans $\exp(\ii z\check{t})$. Si $\check{t}\to 0^+$, l'intégrale donnant $\exp(\ii Q^2\check{t}/4)S_{\rm lacet}(\check{t})-S_{\rm lacet}(0)$ est dominée par les valeurs de $z$ proches de l'infini, donc se réduit à l'intégrale de Fresnel $(-1/\pi)\int_0^{+\infty} \dd z [\exp(\ii z\check{t})-1]/z^{3/2}$ dont la valeur est connue (on a pris $\phi+\pi=0^+$ pour simplifier).}\footnote{\label{note2} On peut obtenir une expression plus explicite du signal dans la limite $Q\to 0$: $S(\check{t})=\left(\frac{\check{t}}{\pi}\right)^{1/2}\eee^{-\ii\pi/4}-\frac{\sqrt{3}}{2\pi} \eee^{-\ii\pi/4}\left(\frac{2\pi^2\check{t}}{3}-1\right)\erfi(\sqrt{\pi^2\check{t}/3})\eee^{-\pi^2\check{t}/3}$, où $\erfi$ est la fonction d'erreur imaginaire, ce qui admet le même comportement aux temps courts que dans (\ref{eq155}) mais le nouveau comportement aux temps longs $S(\check{t}) \underset{\check{t}\to +\infty}{\sim} - \frac{\eee^{-\ii\pi/4}}{2\pi^{1/2}} \left(\frac{3}{\pi^2}\right)^2 \frac{1}{\check{t}^{3/2}}$. Comme la fonction $\erfi$ est, malgré son nom, à valeurs réelles sur l'axe réel, on trouve que $\re S(\check{t})=-\im S(\check{t})$ dans cette limite, une propriété dont il reste une trace sur la figure \ref{fig4}.}
\be
\label{eq155}
S(\check{t}) \underset{\check{t}\to 0^+}{\sim} 2\eee^{-\ii\pi/4} (\check{t}/\pi)^{1/2} \quad\quad ; \quad\quad
S(\check{t})\underset{\check{t}\to +\infty}{\sim} \frac{-\ii Q/4}{(\frac{Q^2}{4}-\ii\frac{\pi^2}{3})^2} \frac{\eee^{\ii Q^2\check{t}/4}}{\check{t}}
\ee

\begin{SCfigure}
% Figure removed
\caption{Dans la théorie quantique et pour une fonction énergie propre calculée à l'ordre deux en $H_3$ et à l'ordre deux en $\epsilon=k_B T/mc^2$, écart du signal (\ref{eq012}) à la règle d'or de Fermi multiplié par $u^{1/2}$ (le paramètre universel $u$ est donné par (\ref{eq112})) et représenté en fonction du temps réduit (\ref{eq130}) en partie réelle et partie imaginaire, pour des valeurs croissantes de $u$ (le superfluide bidimensionnel sous-jacent est en interaction de plus en plus faible) mais pour un nombre d'onde mis à l'échelle comme dans l'équation (\ref{eq150}) avec $Q=1$. Courbes pour des valeurs finies de $u$: obtenues par intégration numérique de l'expression (\ref{eq132}). Courbes pour $u=+\infty$: tirées de la prédiction analytique (\ref{eq151}) (elles correspondent simplement à $\re S(\check{t})$ et $\im S(\check{t})$).}
\label{fig4}
\end{SCfigure}

\subsection{Comparaison aux simulations de champ classique}
\label{sec2.3}

Afin de comparer aux simulations de la section \ref{sec1}, nous transposons les traitements quantiques des sections \ref{sec2.1} et \ref{sec2.2} au champ classique: dans l'équation (\ref{eq125}), il faut garder le paramètre de coupure $\eta$ fini (il vaut un dans les simulations), négliger le terme $1$ devant les nombres d'occupation $\bar{n}^{\rm lin}_k$ et remplacer ceux-ci par la loi d'équipartition classique $1/\kb$. À titre de test, notons que la valeur correspondante de $\Sigma_\qq^{(2,2)}(\ii 0^+)$ s'écrit alors $-\ii\hbar\Gamma_q^{\mbox{\scriptsize clas}}/2$ où $\Gamma_q^{\mbox{\scriptsize clas}}$ est le taux d'amortissement (\ref{eq113}) comme il se doit. Dans la suite, nous gardons bien sûr toute la dépendance de $\tilde{\Sigma}_\qq^{(2,2)}(\zetat)$ en $\zetat$ dans le calcul du signal $s^{(2,2)}(t)$ par fonction de Green.

Les prédictions correspondantes pour le temps de décroissance à $1/\eee$ sont représentées en trait plein noir épais sur chaque panneau de la figure \ref{fig1}. Elles sont dans tous les cas plus proches des simulations que la règle d'or mais, paradoxalement, l'amélioration est la plus nette dans le régime d'interaction forte (a) et devient très modeste dans le régime d'interaction faible (d). Cette dégradation était cependant prévisible par le raisonnement précédant l'équation (\ref{eq150}): si $\rho\xi^2$ augmente à $\qb$ fixé, le calcul par fonction de Green à l'ordre deux en $H_3$ doit se réduire à la règle d'or donc s'écarter des simulations.

L'analyse de la figure \ref{fig2} montrant la dépendance temporelle du signal nous donne un indice supplémentaire sur ce qui se passe lorsqu'on réduit la force des interactions, $(\rho \xi^2)^{-1}\to 0$. Les prédictions tirées de $\tilde{\Sigma}_\qq^{(2,2)}(\zetat)$ sont cette fois-ci représentées en tireté (celles de la règle d'or seraient identiquement nulles). Elles se rapprochent de manière remarquable des simulations pour la partie imaginaire de l'écart $s(t)-s_{\mbox{\scriptsize pôle}}(t)$ (le tireté rouge se confond presque avec les résultats numériques sur le panneau (d)) mais elles présentent un désaccord persistant, et même croissant en valeurs relatives, sur la partie réelle, qui explique l'échec précédent sur le temps de décroissance à $1/\eee$.

L'anomalie que représente la partie réelle sur la figure \ref{fig2}d est criante si l'on se souvient des prédictions analytiques de la figure \ref{fig4} (même si elles sont pour le champ quantique): la partie réelle et la partie imaginaire, pour les valeurs de $u$ et de $Q$ correspondantes ($u\simeq 18,\!14$ et $Q\simeq 1,\!06$, rappelons-le), devraient être plus ou moins symétriques l'une de l'autre par rapport à l'axe des abscisses (voir aussi la note \ref{note2}), ce qui est loin d'être le cas dans les simulations.

En conclusion, le calcul du signal avec la fonction énergie propre $\tilde{\Sigma}^{(2,2)}(\qb,\zetat)$, par la seule faute de la partie réelle, échoue à rendre compte du premier écart à la règle d'or dans la limite $u\to +\infty$ où il avait toutes les chances d'exceller.

% Figure environment removed


\section{Étude à l'ordre quatre en $H_3$}
\label{sec3}

Pour comprendre l'écart persistant sur la partie réelle du signal entre les simulations numériques de la section \ref{sec1} et l'étude théorique à l'ordre deux en $H_3$ de la section \ref{sec2}, même quand le superfluide bidimensionnel entre dans la limite d'interaction faible $(\rho\xi^2)^{-1}\to 0$, nous calculons ici la fonction énergie propre à l'ordre quatre en $H_3$, soit au moyen de l'habituel formalisme diagrammatique dans la section \ref{sec3.1}, soit par un raisonnement à la Landau dans la section \ref{sec3.2}, qui apporte un éclairage physique intéressant et aboutit à une expression plus compacte. Nous montrons cependant, dans la section \ref{sec3.3}, que le résultat à l'ordre quatre est inutilisable dans la limite de basse température $\epsilon=k_B T/mc^2\to 0$ qui nous intéresse ici.

\subsection{Calcul avec le formalisme diagrammatique de la référence \cite{FW}}
\label{sec3.1}

\paragraph{Les deux topologies} Nous omettons directement l'hamiltonien quartique $H_4$ dans l'équation (\ref{eq001}) et les termes non résonnants $H_3^{\mbox{\scriptsize non\, r\'es}}$ dans l'hamiltonien cubique (\ref{eq100}), car leurs contributions à $\Sigma^{(4)}$ sont sous-dominantes dans la limite $\epsilon\to 0$ (c'est évident pour $H_3^{\mbox{\scriptsize non\, r\'es}}$, qui ne bénéficie pas de l'effet \g{petit dénominateur}, et ceci sera justifié pour $H_4$ dans la section \ref{sec3.2}). L'hamiltonien de couplage se réduit donc aux termes $H_3^{(+/-)}$ donnés dans l'équation (\ref{eq101}). Dans les diagrammes de Feynman d'ordre quatre correspondants, chacun des quatre sommets est le point de rencontre de trois lignes de phonons, ce qui conduit à deux topologies possibles, (i) à boucle interne et (ii) à pont, comme le montre la figure \ref{fig8}. Dans la première topologie, on lève la dégénérescence en mettant arbitrairement la boucle interne dans la branche supérieure, et on se souvient (comme dans la section \ref{sec2}) que l'échange des deux lignes de la boucle interne conduit au même diagramme ; il reste huit diagrammes distincts. Dans la seconde topologie, on lève la dégénérescence en orientant la ligne du pont vers le bas ; il reste six diagrammes distincts. En réalité, dans la terminologie de la figure \ref{fig8}, les contributions des diagrammes $B_3$ et $B_4$ à $\Sigma^{(4)}$ se doublent, ainsi que celles de $P_2$ et $P_3$ et que celles de $P_4$ et $P_6$.\footnote{Il existe en effet une symétrie non triviale reliant certains diagrammes entre eux. Pour le voir, appliquons à chaque diagramme une rotation d'axe vertical d'angle $\pi$ puis changeons l'orientation de toutes les lignes (internes et externes); ceci ne change pas la valeur de la contribution du diagramme à $\Sigma^{(4)}$. La plupart des diagrammes sont ainsi envoyés sur eux-mêmes, à l'exception des paires $\{B_3,B_4\}$, $\{P_2,P_3\}$ et $\{P_4,P_6\}$ dont les deux partenaires sont simplement échangés (on se souviendra, dans le cas des diagrammes à pont, qu'une rotation d'axe horizontal d'angle $\pi$, qui permet de rétablir l'orientation originelle de la ligne verticale, ne change pas la valeur du diagramme). On en déduit que $\Sigma_{B_3}(\qq,z)=\Sigma_{B_4}(\qq,z)$, $\Sigma_{P_2}(\qq,z)=\Sigma_{P_3}(\qq,z)$ et $\Sigma_{P_4}(\qq,z)=\Sigma_{P_6}(\qq,z)$ comme annoncé.} Il ne reste que onze diagrammes véritablement indépendants.

% où les noms $B_i$ et $P_j$ sont ceux de la figure \ref{fig8}.

%%%%%%%%%%%%%%%%%%%%%%%%%%%
% FIGURA BOUCLES ET PONT
%%%%%%%%%%%%%%%%%%%%%%%%%%%
% Figure environment removed

%%%%%%%%%%%%%%%%%%%%%%%%%%%
% FIGURA SIGNES
%%%%%%%%%%%%%%%%%%%%%%%%%%%
% Figure environment removed

\paragraph{Calcul d'un diagramme générique à boucle interne}
Il est commode de numéroter les lignes internes de $j=1$ à $5$ comme sur la partie gauche de la figure \ref{fig7}. Chaque ligne a une orientation $s_j\in\{-1;1\}$ (de gauche à droite si $s_j=1$, de droite à gauche si $s_j=-1$), un vecteur d'onde $\kk_j$ et une énergie de Matsubara $\omega_j\in 2\ii\pi k_B T \mathbb{Z}$. Sur la figure \ref{fig8}, on a fait le choix d'avoir toujours $s_3\geq s_4$ afin d'éviter le double comptage de diagramme évoqué plus haut. On choisit comme quadrivecteurs indépendants ceux des lignes 2 et 3, que l'on renomme comme suit: 
\be
\label{eq210}
(\kk_2,\omega_2)=(\kk,\omega) \quad ; \quad (\kk_3,\omega_3)=(\kk',\omega')
\ee
Les autres s'en déduisent par conservation de l'énergie-impulsion en chaque sommet. Par exemple, la somme des vecteurs d'onde qui arrivent à un sommet donné est égale à la somme des vecteurs d'onde qui en partent, ce qui donne:
\be
\label{eq211}
\kk_1=s_1(\qq-s_2\kk) \quad ; \quad \kk_4=s_4(s_2\kk-s_3\kk') \quad ; \quad \kk_5=s_2 s_5 \kk
\ee
On a les mêmes relations sur les énergies de Matsubara. Compte tenu des règles énoncées dans la section 25 de la référence \cite{FW}, la contribution à la fonction énergie propre du diagramme générique à boucle interne $B_i$ vaut:
\be
\label{eq212}
\Sigma_{B_i}^{(4)}(\qq,z) = \frac{(-k_B T)^2}{1+\delta_{s_3,s_4}} \int\frac{\dd^dk}{(2\pi)^d}\int\frac{\dd^dk'}{(2\pi)^d}
\mA(\kk_1,\kk_2;\qq)\mA(\kk_3,\kk_4;\kk_2)\mA(\kk_3,\kk_4;\kk_5)\mA(\kk_1,\kk_5;\qq)
\sum_{\omega,\omega'\in 2\ii\pi k_B T\mathbb{Z}}\frac{1}{\prod_{j=1}^{5} (\omega_j-\veps_{\kk_j})}
\ee
où l'on a temporairement $z=\Omega\in 2\ii\pi k_B T\mathbb{Z}$ (voir la section \ref{sec2.1}). Pour écrire les amplitudes de transition en chaque sommet, on a utilisé l'invariance de la fonction $\mA$ de l'équation (\ref{eq102}) par permutation de ses trois arguments et donc le fait que les éléments de matrice $\langle \kk_2,\kk_3|\mathcal{H}_3|\kk_1\rangle$, $\langle \kk_1,\kk_3|\mathcal{H}_3|\kk_2\rangle$ et $\langle \kk_1,\kk_2|\mathcal{H}_3|\kk_3\rangle$ sont tous égaux à $\mA(\kk_2,\kk_3;\kk_1)$. Le dénominateur du facteur de symétrie vaut $2$ si les deux lignes de la boucle interne ont la même orientation (pour compenser le double comptage des configurations ne différant que par l'échange des quadrivecteurs $(\kk_3,\omega_3)$ et $(\kk_4,\omega_4)$ et dans ce cas physiquement identiques); sinon, ces deux lignes sont discernables et le facteur de symétrie vaut $1$. La somme double de Matsubara est calculée explicitement dans l'\ref{app3}, qui montre aussi que son expression se simplifie à la limite des petits angles $O(\epsilon)\to 0$ entre les vecteurs d'onde, où l'on se restreint à $s_2=s_5$ (sinon, la contribution du diagramme est sous-dominante en $\epsilon$, voir l'équation (\ref{eq215}) plus bas):
\be
\label{eq213}
\sum_{\omega,\omega'\in 2\ii\pi k_B T\mathbb{Z}}\frac{(k_B T)^2}{\prod_{j=1}^{5}(\omega_j-\veps_{\kk_j})} \stackrel{\mbox{\scriptsize diag.}\, B}{\underset{\epsilon\to 0}{\sim}} -s_1 s_4 \frac{(\bar{n}_{\veps_3}-\bar{n}_{-s_3s_4\veps_4})(\bar{n}_{-s_1\veps_1}-\bar{n}_{s_2\veps_2})}{(\zeta-\zeta_1)^2(\zeta-\zeta_2)}
\ee
avec la variable $\zeta=z-\veps_\qq$ et des $\zeta_j$ tous d'ordre $k_B T\epsilon^2$,
\be
\label{eq214}
\zeta_1\equiv s_1\veps_{\kk_1}+s_2\veps_{\kk_2}-\veps_\qq \quad ; \quad \zeta_2\equiv s_1\veps_{\kk_1}+s_3\veps_{\kk_3}+s_4\veps_{\kk_4}-\veps_{\qq}
\ee
La dépendance en $\zeta$ de la forme (\ref{eq213}) au second membre est facile à interpréter physiquement puisque $\zeta_1+\veps_\qq$ et $\zeta_2+\veps_\qq$ sont les énergies des états intermédiaires repérés par les droites verticales sur la partie gauche de la figure \ref{fig7}: en effet, ce sont les sommes des énergies des lignes internes $j$ coupées par chaque droite verticale comptées algébriquement, c'est-à-dire pondérées par les orientations $s_j$ (si $s_j=-1$, le phonon sur la ligne $j$ remonte le temps et son énergie doit être comptée négativement). Ceci valide l'estimation (\ref{eq024}) en en reproduisant les dénominateurs d'énergie. 

Pour passer à la limite $\epsilon\to 0$ dans (\ref{eq212}), on met directement les angles à zéro dans chaque amplitude de couplage $\mA$ (ce qui fait sortir un facteur $3(1+\Lambda)$), puis on développe les dénominateurs d'énergie de (\ref{eq213}) sur le modèle de (\ref{eq025}) ; pour que l'effet \g{petit dénominateur} joue à plein, il faut que les vecteurs internes $\kk_j$ fassent tous un petit angle $O(\epsilon)$ avec le vecteur d'onde $\qq$ entrant, ce qui impose\footnote{Quand $\kk_2=\kk$ et $\kk_3=\kk'$ font un angle nul avec $\qq$, les autres vecteurs d'onde $\kk_j$ donnés par (\ref{eq211}) doivent satisfaire à la condition $\kk_j\cdot\qq=k_j q$.}
\be
\label{eq215}
\sig(q-s_2k)=s_1 \quad ; \quad \sig(s_2k-s_3k')=s_4 \quad ; \quad s_2s_5=1
\ee
Les deux premières conditions dans (\ref{eq215}) restreignent le domaine de variation des nombres d'onde $k$ et $k'$. La dernière, qui en est indépendante, élimine les diagrammes $B_3$ et $B_4$, de contributions sous-dominantes. En se souvenant des définitions (\ref{eq022},\ref{eq004}) de $\zetat$ et (\ref{eq020}) de $\qb$, on arrive en dimension $d=2$ au résultat (pour $\im\zetat\neq 0$ mais de signe quelconque):
\be
\label{eq220}
\tilde\Sigma^{(4,2)}_{B_i}(\qb,\tilde{\zeta})=\frac{\qb}{1+\delta_{s_3,s_4}}\left(\frac{9(1+\Lambda)^2}{8\rho\xi^2}\right)^2\int_{\mathcal{D}_1^B}\frac{\kb\dd \kb}{(2\pi)^2} \int_{\mathcal{D}_2^B(\kb)} \frac{\kb'\dd \kb'}{(2\pi)^2} \kb^2\kb'(\qb-s_2\kb)(s_2\kb-s_3\kb')(\bar{n}^{\rm lin}_{k'}-\bar{n}^{\rm lin}_{k'-s_2 s_3 k})(\bar{n}^{\rm lin}_{s_2k}-\bar{n}^{\rm lin}_{s_2k-q}) I(\kb,\kb')
\ee
avec l'intégrale angulaire\footnote{Dans l'intégrale, avant de prendre la limite $\epsilon\to 0$, on a posé $\tilde{\theta}=\theta/\epsilon$ et $\tilde{\phi}=(\theta-\theta')/\epsilon$, où $\theta$ est l'angle entre $\qq$ et $\kk$, $\theta'$ l'angle entre $\qq$ et $\kk'$.}
\be
\label{eq221}
I(\kb,\kb')=\int_{\mathbb{R}} \dd\tilde\theta \int_{\mathbb{R}} \dd\tilde\phi \frac{1}{[\tilde{\zeta}-(A_1^B+B_1^B\tilde\theta^2)]^{2}[\tilde{\zeta}-(A^B_2+B^B_1\tilde\theta^2+B_2^B\tilde\phi^2)]}
\ee
qui fait intervenir les coefficients
\be
\label{eq222}
A_1^B=\frac{3\gamma}{8}(\kb-s_2 \qb)\kb \qb \quad ; \quad A_2^B=\frac{3\gamma}{8}\kb \kb' (s_2\kb'-s_3 \kb) +A_1^{B} \quad ; \quad B_1^B=\frac{-\kb\qb/2}{\kb-s_2 \qb} \quad ; \quad B_2^B=\frac{-\kb \kb'/2}{s_2\kb'-s_3 \kb}
\ee
et les domaines d'intégrations sur $\kb$, puis sur $\kb'$ à $\kb$ fixé, satisfaisant aux conditions (\ref{eq215}) et de coupure (\ref{eq103}) $\kb_j<\eta$, $1\leq j\leq 5$. Au premier membre de (\ref{eq220}), le second entier $2$ dans l'exposant rappelle que $\Sigma^{(4)}$ est calculé à l'ordre deux en $\epsilon$ et le tilde sur $\Sigma$ signifie comme dans (\ref{eq021}) que cette quantité a été divisée par $k_BT\epsilon^2$. Enfin, on vérifie au cas par cas que l'expression (\ref{eq220}) admet une limite finie lorsque la coupure $\eta\to+\infty$ (pour le champ quantique).\footnote{Par exemple, dans le cas de $B_8$ ($s_1=s_4=-1$, $s_2=s_3=s_5=1$), les domaines $\kb\in\mathcal{D}_1^B=[\qb,\eta]$ et $\kb'\in \mathcal{D}_2^B(\kb)=[\kb,\eta]$ sont non bornés lorsque $\eta\to +\infty$ mais les facteurs d'amplification bosoniques $(\bar{n}_{k'}^{\rm lin}-\bar{n}_{k'-k}^{\rm lin})(\bar{n}_k^{\rm lin}-\bar{n}_{k-q}^{\rm lin})=O[\exp(-\kb-\kb'')]$ sont à décroissance rapide en les deux variables d'intégration découplées $(\kb,\kb''=\kb'-\kb)$ et écrasent toute divergence en loi de puissance qui pourrait provenir des autres facteurs ; dans le cas de $B_7$ ($s_1=-1$, $s_2=s_3=s_4=s_5=1$), $\mathcal{D}_1^B=[\qb,\eta]$ et $\mathcal{D}_2^B(\kb)=[0,\kb]$ restent non bornés et les facteurs bosoniques $(1+\bar{n}_{k'}^{\rm lin}+\bar{n}_{k-k'}^{\rm lin})(\bar{n}_k^{\rm lin}-\bar{n}_{k-q}^{\rm lin})=O(\exp(-\kb))$ ne sont plus à décroissance rapide qu'en la variable $\kb$ mais, comme $\kb'<\kb$, l'intégrale sur $\kb'$ à $\kb$ fixé est au plus un $O(\kb^\alpha)$, $\alpha>0$, et l'intégrale sur $\kb$ converge si $\eta\to +\infty$.}

\paragraph{Calcul d'un diagramme générique à pont} Le début ressemble beaucoup au cas précédent. On numérote les lignes internes comme sur la partie droite de la figure \ref{fig7}, avec la même convention de signe, sauf pour la ligne verticale (le \g{pont}), de signe $s_5=1$ lorsqu'elle orientée vers le bas (ce à quoi nous nous sommes limités sur la figure \ref{fig8} pour éviter un double comptage). On garde comme quadrivecteurs indépendants
\be
\label{eq230}
(\kk_1,\omega_1)=(\kk,\omega) \quad ;\quad (\kk_3,\omega_3)=(\kk',\omega')
\ee
les autres s'en déduisant par conservation de l'énergie-impulsion, par exemple pour les vecteurs d'onde:
\be
\label{eq231}
\kk_2=s_2(\qq-s_1\kk) \quad ;\quad \kk_4=s_4(\qq-s_3\kk') \quad ;\quad \kk_5=s_5(s_1\kk-s_3\kk')
\ee
La contribution du diagramme $P_i$ à la fonction énergie propre s'écrit alors \cite{FW}:
\be
\label{eq232}
\Sigma_{P_i}^{(4)}(\qq,z) = (-k_B T)^2\int\frac{\dd^dk}{(2\pi)^d}\int\frac{\dd^dk'}{(2\pi)^d}
\mA(\kk_1,\kk_2;\qq)\mA(\kk_3,\kk_5;\kk_1)\mA(\kk_2,\kk_5;\kk_4)\mA(\kk_3,\kk_4;\qq)
\sum_{\omega,\omega'\in 2\ii\pi k_B T\mathbb{Z}}\frac{1}{\prod_{j=1}^{5} (\omega_j-\veps_{\kk_j})}
\ee
Nul besoin ici de diviser par un facteur de symétrie. Le calcul explicite de la somme double de Matsubara est donné dans l'\ref{app3}, qui établit aussi son expression simplifiée à la limite des petits angles $O(\epsilon)$ entre les vecteurs d'onde: 
\be
\label{eq233}
\sum_{\omega,\omega'\in 2\ii\pi k_B T\mathbb{Z}}\frac{(k_B T)^2}{\prod_{j=1}^{5}(\omega_j-\veps_{\kk_j})} \stackrel{\mbox{\scriptsize diag.}\, P}{\underset{\epsilon\to 0}{\sim}} 
-s_1 s_2 s_4 s_5 \frac{(\bar{n}_{-s_2\veps_2}-\bar{n}_{s_1\veps_1})(\bar{n}_{\veps_3}-\bar{n}_{-s_3s_5\veps_5})}
{(\zeta-\zeta_1)(\zeta-\zeta_2)(\zeta-\zeta_4)}
-s_1 s_2 s_4 s_5 \frac{(\bar{n}_{-s_3s_5\veps_5}-\bar{n}_{-s_3s_4\veps_4})(\bar{n}_{s_1\veps_1}-\bar{n}_{-s_2\veps_2})}
{(\zeta-\zeta_1)(\zeta-\zeta_2)(\zeta-\zeta_3)}
\ee
avec la variable $\zeta$ et des $\zeta_j$ tous d'ordre $k_BT\epsilon^2$,
\be
\label{eq238}
\zeta_1\equiv s_1\veps_{\kk_1}+s_2\veps_{\kk_2}-\veps_\qq\quad ; \quad \zeta_2=s_3\veps_{\kk_3}+s_4\veps_{\kk_4}-\veps_\qq \quad ; \quad \zeta_3\equiv s_1\veps_{\kk_1}+s_4\veps_{\kk_4}-s_5\veps_{\kk_5}-\veps_{\qq} \quad ;\quad \zeta_4\equiv s_2\veps_{\kk_2}+s_3\veps_{\kk_3}+s_5\veps_{\kk_5}-\veps_{\qq}
\ee
Il y a cette fois deux contributions au second membre, dites de type I et de type II dans cet ordre. Les quantités $\zeta_j$ sont des différences d'énergie entre états intermédiaires et état initial, comme dans le cas à boucle interne, sauf qu'il faut introduire quatre droites pour les repérer, voir la partie droite de la figure \ref{fig7}: deux droites verticales pour $\zeta_1$ et $\zeta_2$, deux droites obliques pour $\zeta_3$ et $\zeta_4$. En effet, du point de vue de la droite oblique marquée I sur la figure, la ligne interne du pont est orientée de gauche à droite si $s_5=1$, et $\zeta_4$ doit comporter un signe plus devant $s_5\veps_{\kk_5}$ dans l'équation (\ref{eq238}); pour la droite oblique II, c'est le contraire, d'où le signe moins devant $s_5\veps_{\kk_5}$ dans $\zeta_3$.

Lorsque $\epsilon\to 0$, la limite des petits angles dans les dénominateurs d'énergie et dans les amplitudes de couplage $\mA$ se prend comme précédemment, si ce n'est qu'il faut distinguer les deux types. Pour la contribution de type I, on obtient (pour $\im\zetat\neq 0$ mais de signe quelconque)
\be
\label{eq240}
\tilde\Sigma^{(4,2)}_{P_{i}^{\rm I}}(\qb,\tilde{\zeta})=\qb\left(\frac{9(1+\Lambda)^2}{8\rho\xi^2}\right)^2\int_{\mathcal{D}_1^P} \frac{\kb\dd \kb}{(2\pi)^2} \int_{\mathcal{D}_2^P(\kb)} \frac{\kb'\dd \kb'}{(2\pi)^2} \kb \kb'(\qb-s_1\kb)(\qb-s_3\kb')(\kb-s_1s_3\kb')(\bar{n}^{\rm lin}_{k'}-\bar{n}^{\rm lin}_{k'-s_1 s_3 k})(\bar{n}^{\rm lin}_{s_1k}-\bar{n}^{\rm lin}_{s_1k-q}) J(\kb,\kb')
\ee
avec l'intégrale angulaire
\be
\label{eq241}
J(\kb,\kb')=\int_{\mathbb{R}} \dd\tilde\theta \int_{\mathbb{R}} \dd\tilde\theta' \frac{1}{[\tilde{\zeta}-(A_1^P+B_1^P\thetat^2)][\tilde{\zeta}-(A_2^P+B_2^P\thetat'^2)][\tilde{\zeta}-(A_4^P+B_1^P\thetat^2+B_4^P(\thetat-\thetat')^2)]}
\ee
Pour celle de type II, il vient (toujours pour $\im\zetat\neq 0$ mais de signe quelconque)
\be
\label{eq242}
\tilde\Sigma^{(4,2)}_{P_{i}^{\rm II}}(\qb,\tilde{\zeta})=\qb\left(\frac{9(1+\Lambda)^2}{8\rho\xi^2}\right)^2\int_{\mathcal{D}_1^P} \frac{\kb\dd \kb}{(2\pi)^2} \int_{\mathcal{D}_2^P(\kb)} \frac{\kb'\dd \kb'}{(2\pi)^2} \kb \kb'(\qb-s_1\kb)(\qb-s_3\kb')(\kb-s_1s_3\kb')(\bar{n}^{\rm lin}_{k'-s_3q}-\bar{n}^{\rm lin}_{k'-s_1 s_3 k})(\bar{n}^{\rm lin}_{s_1k}-\bar{n}^{\rm lin}_{s_1k-q}) K(\kb,\kb')
\ee
avec l'intégrale angulaire
\be
\label{eq243}
K(\kb,\kb')=\int_{\mathbb{R}} \dd\tilde\theta \int_{\mathbb{R}} \dd\tilde\theta' \frac{1}{[\tilde{\zeta}-(A_1^P+B_1^P\thetat^2)][\tilde{\zeta}-(A_2^P+B_2^P\thetat'^2)][\tilde{\zeta}-(A_3^P+B_2^P\thetat'^2+B_3^P(\thetat-\thetat')^2)]}
\ee
Indépendamment du type, les coefficients valent
\be
\label{eq244}
A_1^P=\frac{3\gamma}{8}(\kb-s_1\qb)\kb\qb \quad ; \quad A_2^P=\frac{3\gamma}{8}(\kb'-s_3\qb)\kb'\qb \quad ; \quad A_4^P=A_1^P+\frac{3\gamma}{8}\kb \kb' (s_1\kb'-s_3\kb) \quad ; \quad A_3^P=A_1^P+A_2^P-A_4^P
\ee
\be
\label{eq245}
B_1^P=\frac{-\kb\qb/2}{\kb-s_1\qb} \quad ; \quad B_2^P=\frac{-\kb'\qb/2}{\kb'-s_3\qb} \quad ; \quad B_4^P=\frac{-\kb\kb'/2}{s_1\kb'-s_3\kb} \quad ; \quad B_3^P=-B_4^P
\ee
Les domaines d'intégration $\mathcal{D}_1^P$ et $\mathcal{D}_2^P(\kb)$ sont eux aussi indépendants du type; ils tiennent compte de la coupure (\ref{eq103}) sur tous les vecteurs d'onde internes $\kk_j$, $1\leq j\leq 5$, ainsi que des conditions 
\be
\label{eq239}
\sig(q-s_1k)=s_2 \quad ;\quad \sig(q-s_3k')=s_4 \quad;\quad \sig(s_1k-s_3k')=s_5
\ee
assurant que $\kk_2$, $\kk_4$ et $\kk_5$ sont colinéaires à et de même sens que $\qq$ dès que $\kk$ et $\kk'$ le sont. Pour toutes les orientations $s_j$ prises sur la figure \ref{fig8}, il y a compatibilité entre les conditions (\ref{eq239}) donc aucun des diagrammes $P_i$ n'est sous-dominant en $\epsilon$. Enfin, on vérifie l'existence d'une limite finie dans (\ref{eq240},\ref{eq242}) lorsque la coupure $\eta\to +\infty$ (pour le champ quantique).

\subsection{Calcul par un raisonnement à la Landau}
\label{sec3.2}

Le calcul de $\Sigma^{(4)}$ comme dans la référence \cite{FW}, bien que systématique et rigoureux, ne dit pas explicitement quels processus collisionnels entre phonons se cachent derrière chaque diagramme de la figure \ref{fig8}. Nous développons ici une méthode plus heuristique, inspirée de la référence \cite{LK} et valable à l'ordre dominant en $\epsilon$, qui répond à la question, donne le résultat sous une forme compacte (plutôt qu'éclatée en une dizaine de contributions intégrales) et permet de montrer facilement que la contribution de $H_4$ est sous-dominante dans la limite $\epsilon\to 0$.

La référence \cite{LK} explique, pour l'hamiltonien (\ref{eq001}), comment calculer par une règle d'or de Fermi étendue le taux d'amortissement du mode de phonon $\qq$ dû à des collisions $n\to n'$, $n$ étant le nombre de phonons entrants $\qq_i$ (ils comportent un et un seul phonon de vecteur d'onde $\qq$) et $n'$ le nombre de phonons sortants $\qq'_j$ (ils ne comportent aucun phonon de vecteur d'onde $\qq$). Les cas ordinaires des processus de Belyaev et de Landau correspondent à $(n,n')=(1,2)$ et $(n,n')=(2,1)$, ils sont à $n+n'=3$ phonons ; ici, il s'agira de processus à quatre phonons, comme $(n,n')=(2,2)$. On commence par écrire des équations cinétiques sur les nombres d'occupation des phonons, provisoirement supposés hors d'équilibre, en comptant positivement le processus direct $n\to n'$ (qui tend à vider le mode $\qq$) et négativement le processus inverse $n'\to n$ (qui tend à le remplir): 
\begin{multline}
\label{eq250}
\frac{\dd}{\dd t} {n}_\qq = -\frac{1}{(n-1)!}\int\left(\prod_{i=2}^{n}\frac{\dd^dq_i}{(2\pi)^d}\right)\frac{1}{n'!}\int\left(\prod_{j=1}^{n'}\frac{\dd^dq'_j}{(2\pi)^d}\right)\frac{2\pi}{\hbar} \mA^2_{\ii\to \ff} \delta(E_\ii-E_\ff) (2\pi)^d \delta\left(\sum_{i=1}^{n}\qq_i-\sum_{j=1}^{n'}\qq'_j\right)\\
\times\left[\prod_{i=1}^{n}n_{\qq_i}\prod_{j=1}^{n'}(1+n_{\qq'_j})-\prod_{i=1}^{n} (1+n_{\qq_i}) \prod_{j=1}^{n'}n_{\qq'_j}\right]
\end{multline}
On a pris la limite thermodynamique, et on a divisé par des nombres de symétrie tenant compte de l'invariance du processus par permutation des $n-1$ derniers vecteurs d'onde entrants (on convient que $\qq_1=\qq$) et par permutation des $n'$ vecteurs d'onde sortants. On reconnaît dans (\ref{eq250}) le Dirac de conservation de l'énergie, le facteur $2\pi/\hbar$ et le carré d'une amplitude de transition $\mA_{\ii\to\ff}$ entre état initial $|\ii\rangle=|(\qq_i)_{1\leq i\leq n}\rangle$ et état final $|\ff\rangle=|(\qq'_j)_{1\leq j\leq n'}\rangle$ typiques de la règle d'or, ainsi qu'un Dirac de conservation de l'impulsion et les facteurs d'amplification bosoniques. Le principe de microréversibilité a permis de factoriser l'amplitude au carré du processus direct. 

On se place ensuite dans le cadre de la procédure d'excitation (\ref{eq010}), où tous les modes de phonons sont à l'équilibre thermique à l'exception du mode $\qq$, d'où la substitution $n_\kk\to \bar{n}_\kk$, $\kk\neq\qq$, et on linéarise (\ref{eq250}) autour de $n_\qq=\bar{n}_\qq$ pour une  faible excitation, ce qui donne l'équation de retour à l'équilibre $(\dd/\dd t)(n_\qq-\bar{n}_\qq)|_{t=0^+}=-\Gamma_\qq^{n\to n'}(n_\qq-\bar{n}_\qq)$ avec le taux
\begin{multline}
\label{eq251}
\Gamma_\qq^{n\to n'}=\frac{1}{(n-1)!}\int\left(\prod_{i=2}^{n}\frac{\dd^dq_i}{(2\pi)^d}\right)\frac{1}{n'!}\left(\int\prod_{j=1}^{n'}\frac{\dd^dq'_j}{(2\pi)^d}\right)\frac{2\pi}{\hbar} \mA^2_{\ii\to \ff} \delta(E_\ii-E_\ff) (2\pi)^d \delta\left(\sum_{i=1}^{n}\qq_i-\sum_{j=1}^{n'}\qq'_j\right) \\  \left[\prod_{i=2}^{n}\bar{n}_{\qq_i}\prod_{j=1}^{n'}(1+\bar{n}_{\qq'_j})-\prod_{i=2}^{n} (1+\bar{n}_{\qq_i}) \prod_{j=1}^{n'}\bar{n}_{\qq'_j}\right]
\end{multline}
où, au contraire de l'équation (\ref{eq250}), tous les produits sur les vecteurs d'onde entrants excluent le mode $\qq$ donc commencent à $i=2$. La différence avec la règle d'or réside dans le calcul de l'amplitude $\mA_{\ii\to\ff}$. D'habitude, on a simplement $\mA_{\ii\to\ff}=\langle\ff|\mH_3|\ii\rangle$. La référence \cite{LK} s'occupe justement du cas utile ici de collisions à quatre phonons, où $\mA_{\ii\to\ff}$ s'obtient en traitant $H_3$ au second ordre et $H_4$ au premier ordre de la théorie des perturbations:
\be
\label{eq252}
\mA_{\ii\to\ff}=\langle\ff|\mH_4|\ii\rangle + \sum_{|\lambda \rangle} \frac{\langle\ff|\mH_3|\lambda\rangle\langle\lambda|\mH_3|\ii\rangle}{E_\ii-E_\lambda}
\ee
et la somme porte sur tous les états de Fock de phonons intermédiaires possibles $|\lambda\rangle$. 
Un simple comptage de puissances comme dans la section \ref{sec0} montre que la contribution de $\mH_4$ à (\ref{eq252}) est sous-dominante dans la limite des petits angles $\epsilon\to 0$;\footnote{Au vu des équations (\ref{eq052},\ref{eq104}), la contribution de $\mH_4$ à (\ref{eq252}) est d'ordre $(mc^2/\rho)\epsilon^2$ ; au vu des équations (\ref{eq025},\ref{eq027}), celle de $\mH_3$ est d'ordre $(mc^2/\rho)\epsilon^0$.} nous l'omettons donc dans la suite. Au dénominateur d'énergie, nous avons choisi comme énergie de référence celle $E_\ii$ de l'état initial plutôt que celle $E_\ff$ de l'état final, car nous parlons d'une transition $\ii\to\ff$, comme le rappelle le choix de $\ff$ en cro- et $\ii$ en -chet dans l'élément de matrice au numérateur, ce qui ne constitue pas une raison physique sérieuse. Heureusement, ceci n'a aucune importance puisque l'amplitude est utilisée dans (\ref{eq250}) sur la couche d'énergie ($E_\ii=E_\ff$): le résultat est invariant par échange des rôles de $\ii$ et $\ff$ dans n'importe quel terme de la somme au second membre de (\ref{eq252}):
\be
\label{eq253}
\frac{\langle\ff|\mH_3|\lambda\rangle\langle\lambda|\mH_3|\ii\rangle}{E_\ii-E_\lambda} \mbox{(sens direct)} \longleftrightarrow \frac{\langle\ii|\mH_3|\lambda\rangle\langle\lambda|\mH_3|\ff\rangle}{E_\ff-E_\lambda} \mbox{(sens inverse)}
\ee
Le taux total d'amortissement s'obtient en sommant $\Gamma_\qq^{n\to n'}$ sur les processus collisionnels $n\to n'$ pertinents à l'ordre considéré. À partir des règles évidentes suivantes, $(i)$ le diagramme donnant l'amplitude de transition $\mA_{\ii\to\ff}$ est connexe, c'est-à-dire qu'il n'y a pas de découplage possible du processus $n\to n'$ en deux processus collisionnels indépendants, $(ii)$ tout état intermédiaire (entre actions successives de $\mH_3$) doit différer de l'état initial et de l'état final, on trouve à l'ordre quatre en $H_3^{(+)}+H_3^{(-)}$ comme seules possibilités les processus $1\to 3$, $2\to 2$ et $3\to 1$:\footnote{Dans la référence \cite{LK}, la branche acoustique est de départ concave: seul le processus $2\to 2$ conserve l'énergie-impulsion et donne une contribution non nulle à (\ref{eq254}); cependant, elle a été mal calculée, et l'erreur a été corrigée dans la référence \cite{Annalen}.}
\be
\label{eq254}
\Gamma^{(4)}_\qq=\Gamma_\qq^{1\to 3}+\Gamma_\qq^{2\to 2}+\Gamma_\qq^{3\to 1}
\ee

De manière heuristique, nous généralisons ce formalisme au calcul de la contribution d'ordre quatre en $H_3^{(+)}+H_3^{(-)}$ à la fonction énergie propre du mode $\qq$. Nous admettons, comme pour le taux d'amortissement, qu'elle est la somme des contributions des processus collisionnels $n\to n'$ susmentionnés:
\be
\label{eq255}
\Sigma^{(4)}(\qq,z)=\Sigma_{1\to 3}(\qq,z)+\Sigma_{2\to 2}(\qq,z)+\Sigma_{3\to 1}(\qq,z)
\ee
Sachant que le Dirac de conservation de l'énergie dans (\ref{eq251}) provient en définitive de l'identité 
\be
\label{eq256}
\lim_{\delta\to 0^+} \im \frac{1}{E_\ii+\ii\delta-E_\ff}=-\pi\delta(E_\ii-E_\ff)
\ee
et que la variable déplacée $\zeta=z-\veps_\qq$ joue le rôle de $\ii\delta$ dans l'approximation du pôle (\ref{eq014}), nous arrivons sans hésitation à la forme
\begin{multline}
\label{eq257}
\Sigma_{n\to n'}^{(4)}(\qq,z)=\frac{1}{(n-1)!} \int\left(\prod_{i=2}^{n}\frac{\dd^dq_i}{(2\pi)^d}\right)\frac{1}{n'!} \int\left(\prod_{j=1}^{n'} \frac{\dd^dq'_j}{(2\pi)^d}\right) \frac{\mathcal{A}_{\ii\to\ff}^2(\qq,z)}{\zeta+E_\ii-E_\ff}(2\pi)^d \delta\left(\sum_{i=1}^{n}\qq_i-\sum_{j=1}^{n'}\qq'_j\right) \\
\times\left[\prod_{i=2}^{n}\bar{n}_{\qq_i}\prod_{j=1}^{n'}(1+\bar{n}_{\qq'_j})-\prod_{i=2}^{n} (1+\bar{n}_{\qq_i}) \prod_{j=1}^{n'}\bar{n}_{\qq'_j}\right]
\end{multline}
Le seul piège réside dans la construction de la dépendance en $\zeta$ de l'amplitude de transition $\mA$. Le choix de $E_\ii$ ou $E_\ff$ comme énergie de référence n'est maintenant plus indifférent. Choisir $E_\ii$ comme au premier membre de (\ref{eq253}), c'est prendre le point de vue d'une transition $\ii\to\ff$ donc parcourir le temps dans le sens ordinaire ; le bon chemin d'intégration est précisément $C_+$, celui décrit par la variable $\zeta$ dans (\ref{eq012}) ; il faut donc remplacer $E_\ii$ par $\zeta+E_\ii$ dans le dénominateur d'énergie interne à $\mA$. En revanche, choisir $E_\ff$ comme au second membre de (\ref{eq253}) revient à prendre le point de vue inverse d'une transition $\ff\to\ii$ donc à remonter le temps; le bon chemin d'intégration de la fonction de Green est alors $C_-$ \cite{CCTbordeaux}, qui court dans le demi-plan inférieur parallèlement à l'axe réel de $-\infty$ à $+\infty$ et qui est décrit par la variable $-\zeta$ dans (\ref{eq012}) ; il faut donc remplacer $E_\ff$ par $-\zeta+E_\ff$ dans le dénominateur d'énergie interne à $\mA$. 

Pour trancher selon l'état intermédiaire $|\lambda\rangle$, il faut s'appuyer sur la propriété suivante. Par construction, la fonction $\Sigma^{(4)}(\qq,z)$ dépend du mode au travers de son vecteur d'onde $\qq$ mais pas de son énergie $\veps_\qq$ (lorsqu'on utilise comme variable l'énergie complexe originelle $z$), et ne peut donc pas comporter de dénominateur d'énergie contenant $\veps_\qq$. C'est le contraire qui se produit si l'on utilise la variable déplacée $\zeta$. Ceci veut dire que, si dans le sens direct de l'amplitude de transition, l'action sur $|\ii\rangle$ de l'opérateur $\mH_3^{(\pm)}$ ne fait pas disparaître le phonon de vecteur d'onde $\qq$, il faut utiliser le sens inverse de l'amplitude de transition, cette règle étant à appliquer état intermédiaire par état intermédiaire. D'où notre prescription finale sur l'amplitude de transition à inclure dans (\ref{eq257}):
\be
\label{eq258}
\boxed{\mathcal{A}_{\ii\to\ff}(\qq,z) = \sum_{|\lambda\rangle\, / \, \qq\notin|\lambda\rangle} \frac{\langle \ff| \mH_3^{(\pm)} |\lambda\rangle \langle\lambda| \mH_3^{(\pm)}|\ii\rangle}{\zeta+E_\ii-E_\lambda} 
+
\sum_{|\lambda\rangle\, / \, \qq\in|\lambda\rangle} \frac{\langle \ii| \mH_3^{(\pm)} |\lambda\rangle \langle\lambda| \mH_3^{(\pm)}|\ff\rangle}{-\zeta+E_\ff-E_\lambda}} 
\ee
De l'expression (\ref{eq258}) nous tirons l'amplitude de transition pour le processus $1\to 3$ (sachant que $\qq'_1+\qq'_2+\qq'_3=\qq$): 
\begin{multline}
\label{eq260}
\mathcal{A}_{\qq\to\qq'_1,\qq'_2,\qq'_3}(\qq,\zeta)= \frac{\mA(\qq'_1,\qq'_2;\qq'_1+\qq'_2)\mA(\qq'_1+\qq'_2,\qq'_3;\qq)}{\zeta+\epsilon_\qq-(\epsilon_{\qq'_1+\qq'_2}+\epsilon_{\qq'_3})} +\frac{\mA(\qq'_2,\qq'_3;\qq'_2+\qq'_3)\mA(\qq'_2+\qq'_3,\qq'_1;\qq)}{\zeta+\epsilon_\qq-(\epsilon_{\qq'_2+\qq'_3}+\epsilon_{\qq'_1})} \\ +\frac{\mA(\qq'_1,\qq'_3;\qq'_1+\qq'_3)\mA(\qq'_1+\qq'_3,\qq'_2;\qq)}{\zeta+\epsilon_\qq-(\epsilon_{\qq'_1+\qq'_3}+\epsilon_{\qq'_2})}
\end{multline}
puis pour le processus $2\to 2$ (sachant que $\qq'_1+\qq'_2=\qq+\qq_2$):
\begin{multline}
\label{eq262}
\mathcal{A}_{\qq,\qq_2\to\qq'_1,\qq'_2}(\qq,\zeta)= \frac{\mA(\qq'_1,\qq'_2;\qq'_1+\qq'_2)\mA(\qq,\qq_2;\qq+\qq_2)}{\zeta+\epsilon_{\qq}+\epsilon_{\qq_2}-\epsilon_{\qq+\qq_2}} -\frac{\mA(\qq'_1-\qq,\qq'_2;\qq_2)\mA(\qq'_1-\qq,\qq;\qq'_1)}{\zeta+\epsilon_{\qq}+\epsilon_{\qq'_1-\qq}-\epsilon_{\qq'_1}} \\ +\frac{\mA(\qq'_2-\qq_2,\qq_2;\qq'_2)\mA(\qq'_2-\qq_2,\qq'_1;\qq)}{\zeta+\epsilon_{\qq}-(\epsilon_{\qq'_1}+\epsilon_{\qq'_2-\qq_2})} +\frac{\mA(\qq-\qq'_2,\qq_2;\qq'_1)\mA(\qq-\qq'_2,\qq'_2;\qq)}{\zeta+\epsilon_{\qq}-(\epsilon_{\qq'_2}+\epsilon_{\qq-\qq'_2})}-\frac{\mA(\qq'_2-\qq,\qq'_1;\qq_2)\mA(\qq'_2-\qq,\qq;\qq'_2)}{\zeta+\epsilon_{\qq}+\epsilon_{\qq'_2-\qq}-\epsilon_{\qq'_2}}
\end{multline}
et enfin pour le processus $3\to 1$ (sachant que $\qq_1'=\qq+\qq_2+\qq_3$):
\begin{multline}
\label{eq264}
\mathcal{A}_{\qq,\qq_2,\qq_3\to \qq_1'}(\qq,\zeta)= -\frac{\mA(\qq_2,\qq_3;\qq_2+\qq_3)\mA(\qq_2+\qq_3,\qq;\qq_1')}{\zeta+\epsilon_{\qq}+\epsilon_{\qq_2+\qq_3}-\epsilon_{\qq_1'}}+\frac{\mA(\qq+\qq_2,\qq_3;\qq_1')\mA(\qq,\qq_2;\qq+\qq_2)}{\zeta+\epsilon_{\qq}+\epsilon_{\qq_2}-\epsilon_{\qq+\qq_2}}\\ +\frac{\mA(\qq+\qq_3,\qq_2;\qq_1')\mA(\qq,\qq_3;\qq+\qq_3)}{\zeta+\epsilon_{\qq}+\epsilon_{\qq_3}-\epsilon_{\qq+\qq_3}}
\end{multline}
Dans ces équations (\ref{eq260},\ref{eq262},\ref{eq264}), les trois contributions où il a fallu choisir le sens inverse dans (\ref{eq253}), c'est-à-dire prendre $E_\ff$ plutôt que $E_\ii$ comme énergie de référence, sont faciles à repérer car elles commencent par un signe moins. 

Notre prescription (\ref{eq257}) donne des résultats simples mais en général différents de ceux, exacts, de la théorie diagrammatique à l'ordre quatre, compte tenu des expressions des sommes doubles de Matsubara dans l'\ref{app3}. Cependant, de façon remarquable, 
une fois que ces sommes ont été remplacées par leurs expressions simplifiées (\ref{eq213},\ref{eq233}), ce qui est légitime aux petits angles, elle est en accord parfait avec la valeur de $\Sigma^{(4)}$ prédite par les diagrammes de la figure \ref{fig8}; on le vérifie à l'aide de la table de correspondance \ref{table1}, obtenue au vu des amplitudes de couplage dans (\ref{eq212},\ref{eq232}) et des dénominateurs d'énergie dans (\ref{eq213},\ref{eq233}), et qui permet d'associer à chaque ligne interne sur cette figure un phonon entrant $\qq_i$, sortant $\qq'_j$ ou virtuel (combinaison linéaire des précédents) d'un processus $n\to n'$ donc de donner un sens physique clair au diagramme correspondant. En définitive, la prescription (\ref{eq257}) conduit en $d=2$ à une expression rigoureusement équivalente mais beaucoup plus compacte de $\Sigma^{(4)}$ à l'ordre dominant en $\epsilon$:
\be
\label{eq269}
\boxed{
\tilde{\Sigma}_{\qb}^{(4,2)}(\tilde{\zeta})=\frac{1}{3}\int\!\!\!\!\int_{\mathbb{T}_{\qb}} \frac{\qb'_1\dd \qb'_1}{(2\pi)^2} \frac{\qb'_2\dd \qb'_2}{(2\pi)^2} \Phi(\qb'_1,\qb'_2) + \int\!\!\!\!\int_{\mathbb{R}^{+2}\setminus\mathbb{T}_{\qb}} \frac{\qb'_1\dd \qb'_1}{(2\pi)^2} \frac{\qb'_2\dd \qb'_2}{(2\pi)^2} \Phi(\qb'_1,\qb'_2) + \int\!\!\!\!\int_{\mathbb{R}^{+2}}\frac{\qb_2\dd \qb_2}{(2\pi)^2}\frac{\qb_3\dd \qb_3}{(2\pi)^2} \Phi(-\qb_2,-\qb_3)}
\ee
où nous avons respecté l'ordre des processus $1\to 3$, $2\to 2$ et $3\to 1$, $\mathbb{T}_{\qb}$ est le triangle $\{(\qb'_1,\qb'_2)\in\mathbb{R}^{+2}\;|\; \qb'_1+\qb'_2\leq \qb\}$  et, comme précédemment, le tilde sur $\Sigma_{\qb}$ indique une division par $k_B T\epsilon^2$. Une seule et même fonction intervient dans l'intégrande des contributions $n\to n'$, seul le domaine d'intégration change avec le processus considéré:\footnote{En ce qui concerne les facteurs bosoniques, on le montre en faisant bon usage de la relation $\bar{n}_{-k}^{\rm lin}=-(1+\bar{n}_k^{\rm lin})$, $\forall k\in\mathbb{R}^*$.}
\begin{multline}
\label{eq270}
\Phi(\qb'_1,\qb'_2)=\frac{1}{2} \qb \left(\frac{9(1+\Lambda)^2}{8\rho\xi^2}\right)^2 \qb'_1 \qb'_2 (\qb'_1+\qb'_2-\qb) [(1+\bar{n}^{\rm lin}_{q'_1}+\bar{n}^{\rm lin}_{q'_2})\bar{n}^{\rm lin}_{q'_1+q'_2-q}-\bar{n}^{\rm lin}_{q'_1}\bar{n}^{\rm lin}_{q'_2}] \\ \times \int_{\mathbb{R}}\dd\thetat_1\int_{\mathbb{R}}\dd\thetat_2\frac{\left\{\frac{\qb'_1+\qb'_2}{\zetat+\qb(\qb'_1+\qb'_2)(\qb'_1+\qb'_2-\qb)\left[\frac{(\qb'_1\thetat_1+\qb'_2\thetat_2)^2}{2(\qb'_1+\qb'_2-\qb)^2(\qb'_1+\qb'_2)^2}-\frac{3\gamma}{8}\right]}+\frac{\qb-\qb'_1}{\zetat+\qb \qb'_1(\qb-\qb'_1)\left[-\frac{\thetat_1^2}{2(\qb'_1-\qb)^2}+\frac{3\gamma}{8}\right]}+\frac{\qb-\qb'_2}{\zetat+\qb \qb'_2(\qb-\qb'_2)\left[-\frac{\thetat_2^2}{2(\qb'_2-\qb)^2}+\frac{3\gamma}{8}\right]}\right\}^2}{\zetat - \frac{\qb'_1(\qb'_2-\qb)\thetat_1^2+\qb'_2(\qb'_1-\qb)\thetat_2^2-2\qb'_1 \qb'_2\thetat_1 \thetat_2}{2(\qb'_1+\qb'_2-\qb)}+\frac{3\gamma}{8} (\qb'_1+\qb'_2)(\qb'_1-\qb)(\qb'_2-\qb)}
\end{multline}

\begin{table}[t]
\centerline{\small
\begin{tabular}{|c|c|c|c|c|c|}
$B_1$ & tous les carrés de $1\to 3$ & $P_{1}^{\rm I}$  &  tous les rectangles de $1\to 3$ & $P_{1}^{\rm II}$ & $3^{\rm e}$ terme $\times$ $4^{\rm e}$ terme de $2\to 2$\\
$B_2$ & carrés des $3^{\rm e}$ et $4^{\rm e}$ termes de $2\to 2$ & $P_{2}^{\rm I}$ & $1^{\rm er}$ terme $\times$ $3^{\rm e}$ terme de $2\to 2$ & $P_{2}^{\rm II}$ & $3^{\rm e}$ terme $\times$ $5^{\rm e}$ terme de $2\to 2$\\
$B_3$ & $2^{\rm e}$ terme $\times$ $3^{\rm e}$ terme de $2\to 2$ & $P_{3}^{\rm I}$ & $1^{\rm er}$ terme $\times$ $4^{\rm e}$ terme de $2\to 2$ & $P_{3}^{\rm II}$ & $2^{\rm e}$ terme $\times$ $4^{\rm e}$ terme de $2\to 2$\\
$B_4$ & $4^{\rm e}$ terme $\times$ $5^{\rm e}$ terme de $2\to 2$ & $P_{4}^{\rm I}$ & $1^{\rm er}$ terme $\times$ $2^{\rm e}$ terme de $2\to 2$ & $P_{4}^{\rm II}$ & $1^{\rm er}$ terme $\times$ $3^{\rm e}$ terme de $3\to 1$\\
$B_5$ & carrés des $2^{\rm e}$ et $5^{\rm e}$ termes de $2\to 2$ & $P_{5}^{\rm I}$ & $2^{\rm e}$ terme $\times$ $3^{\rm e}$ terme de $3\to 1$ & $P_{5}^{\rm II}$ & $2^{\rm e}$ terme $\times$ $5^{\rm e}$ terme de $2\to 2$\\
$B_6$ & carré du $1^{\rm er}$ terme de $3\to 1$ & $P_{6}^{\rm I}$ & $1^{\rm er}$ terme $\times$ $5^{\rm e}$ terme de $2\to 2$ & $P_{6}^{\rm II}$ & $1^{\rm er}$ terme $\times$ $2^{\rm e}$ terme de $3\to 1$\\
$B_7$ & carré du $1^{\rm er}$ terme de $2\to 2$ & & & & \\
$B_8$ & carrés des $2^{\rm e}$ et $3^{\rm e}$ termes de $3\to 1$ & & & & 
\end{tabular}
}
\caption{Dans le calcul de la fonction énergie propre à l'ordre quatre en $H_3$, correspondance entre les diagrammes de Feynman de la figure \ref{fig8} et les termes carrés ou rectangles (doubles produits) des quantités $\mA^2_{\ii\to\ff}(\qq,z)$ intervenant dans la prescription à la Landau (\ref{eq257}); les différents termes des amplitudes de transition $\mA_{\ii\to\ff}(\qq,z)$ auxquels il est fait référence sont ceux des équations (\ref{eq260},\ref{eq262},\ref{eq264}) écrits dans cet ordre. Pour les diagrammes à pont, il a fallu distinguer entre type I et type II comme dans (\ref{eq233}). La correspondance vaut après intégration sur les vecteurs d'onde comme dans (\ref{eq257}), qui fait par exemple que les carrés de chacun des termes de (\ref{eq260}), qui diffèrent par une permutation des $\qq_j'$, ou que les doubles produits des $2^{\rm e}$ et $4^{\rm e}$ termes et des $5^{\rm e}$ et $3^{\rm e}$ termes de (\ref{eq262}), qui sont envoyés l'un sur l'autre par échange de $\qq_1'$ et $\qq_2'$, donnent la même contribution.}
\label{table1}
\end{table}
% par rapport à la synthèse et aux notes, renumérotation des termes 
% avant -> après : 1->3 : 1->1 2->3 3->2
% 2->2 : 1->1 2->4 3->2 4->3 5->5
% 3->1 : 1->2 2->3 3->1

\subsection{Échec de l'ordre quatre dans la limite $\epsilon=k_B T/mc^2\to 0$}
\label{sec3.3}

\paragraph{Une divergence inattendue} En première application simple du développement de la fonction énergie propre à l'ordre quatre en $H_3$, nous avions décidé de calculer numériquement\footnote{Les intégrales angulaires sont calculées analytiquement, seule l'intégration sur les nombres d'onde est faite par ordinateur.} $\tilde{\Sigma}_{\qb}^{(4,2)}(\ii 0^+)$ pour voir si, par une meilleure approximation de $\Sigma_{\qq}(\ii 0^+)$ dans l'équation (\ref{eq014}), cela rapprochait l'approximation du pôle des résultats numériques en réduisant l'écart sur la figure \ref{fig2}d. À notre grande surprise, nous avons découvert que $\tilde{\Sigma}^{(4,2)}(\ii\delta)$ diverge comme $1/\ii\delta$ lorsque $\delta\to 0^+$! Nous avons ensuite confirmé ce résultat analytiquement, en montrant que\footnote{Pour faciliter le calcul des intégrales angulaires, nous avons établi (\ref{eq280}) d'abord dans le cas où $\zetat\to 0$ par parties imaginaires $>0$, puis nous avons étendu au cas $\im\zetat<0$ comme dans la note \ref{notegen}.}
\be
\label{eq280}
\boxed{\tilde{\Sigma}_{\qb}^{(4,2)}(\zetat)\underset{\zetat\to 0}{\sim} \frac{-2}{3\pi^2\gamma\zetat} \left(\frac{9(1+\Lambda)^2}{8\rho\xi^2}\right)^2 \int_{\qb}^{\eta} \kb\dd\kb \int_0^{\eta-\kb} \kb'\dd\kb' (\kb-\qb)(\qb+\kb')(\bar{n}_{k-q}^{\rm lin}-\bar{n}_{k}^{\rm lin})(\bar{n}_{k'+q}^{\rm lin}-\bar{n}_{k'+k}^{\rm lin})}
\ee
Cette divergence provient de l'intégrale angulaire $K(\kb,\kb')$ (\ref{eq243}) du diagramme à pont $P_5$. Analysons donc le comportement de (\ref{eq243}) dans la limite $\zetat\to 0$ pour des orientations quelconques $s_j$ des lignes internes. S'il est divergent, il est nécessairement déterminé par un voisinage du lieu d'annulation du dénominateur de l'intégrande dans le plan $(\thetat,\thetat')$ pour $\zetat=0$. Comme le paramètre de courbure $\gamma$ est ici $>0$, ce lieu est la réunion des deux droites verticales $\thetat=\veps(-A_1^P/B_1^P)^{1/2}$ ($\veps=\pm 1$) où s'annule le premier facteur dans le dénominateur, des deux droites horizontales $\thetat'=\veps'(-A_2^P/B_2^P)^{1/2}$ ($\veps'=\pm 1$) où s'annule le deuxième facteur, et de la conique annulant le troisième facteur. Le malheur est que la conique passe toujours par les points d'intersection $P_{\veps,\veps'}\equiv(\veps(-A_1^P/B_1^P)^{1/2},\veps'(-A_2^P/B_2^P)^{1/2})$ des droites horizontales et verticales, quelles que soient les nombres d'onde $\kb$, $\kb'$ et $\qb$, à la seule condition que $\veps\veps'=s_2 s_4$. Ceci résulte de l'identité sur les coefficients (\ref{eq244},\ref{eq245}):
\be
\label{eq281}
4A_1^P A_2^P B_1^P B_2^P (B_3^P)^2 = [A_2^P B_1^P(B_2^P+B_3^P)+A_1^P B_2^P B_3^P-A_3^P B_1^P B_2^P]^2
\ee
Comme l'intégrale angulaire est bidimensionnelle, une éventuelle partie divergente $K_{\rm div}(\kb,\kb')$ de $K(\kb,\kb')$ ne peut provenir que d'un petit voisinage des points de triple intersection $P_{s_2,s_4}$ et $P_{-s_2,-s_4}$, sur lequel nous remplaçons chaque facteur au dénominateur par son approximation linéaire en $\tau=\thetat-\veps(-A_1^P/B_1^P)^{1/2}$ et $\tau'=\thetat'-\veps'(-A_2^P/B_2^P)^{1/2}$. Il reste
\be
\label{eq282}
K_{\rm div}(\kb,\kb')=\int_{\mathbb{R}^2} \frac{2\dd \tau\dd \tau'}{[\zetat-(\frac{3\gamma}{4})^{1/2}s_1\qb\kb \tau][\zetat-(\frac{3\gamma}{4})^{1/2}s_3\qb\kb' \tau']\{\zetat-(\frac{3\gamma}{4})^{1/2}s_3\kb'[s_1\kb \tau +(\qb-s_1\kb) \tau']\}}=- \frac{8\pi^2(1-s_2)(1-s_1s_3)}{3\gamma\zetat \qb\kb\kb' (s_1\kb-s_3\kb')}
\ee
où nous avons pu étendre l'intégrale à $\mathbb{R}^2$ tout entier puisqu'elle converge à l'infini, et avons pu nous limiter au point $P_{s_2,s_4}$ (au prix de l'ajout d'un facteur $2$ global) par invariance par parité $(\thetat,\thetat')\to (-\thetat,-\thetat')$. Au troisième membre de (\ref{eq282}), on a calculé explicitement l'intégrale sur $\tau'$ puis sur $\tau$ par la méthode des résidus (il faut refermer le chemin d'intégration par un demi-cercle à l'infini dans le demi-plan supérieur ou inférieur contenant le nombre minimal de pôles au vu des contraintes (\ref{eq239})). La partie divergente est donc non nulle seulement si $s_2=s_1s_3=-1$, c'est-à-dire si la branche inférieure gauche de la boucle à pont est orientée vers la gauche et si les deux branches supérieures ont des orientations opposées, ce qui sélectionne comme annoncé le seul diagramme $P_5$ sur la figure \ref{fig8}. Le report de la valeur correspondante de $K_{\rm div}(\kb,\kb')$ dans (\ref{eq242}) redonne l'équation (\ref{eq280}). La même technique appliquée à la contribution de type I des diagrammes à pont conduit dans (\ref{eq241}) à\footnote{En effet, les coefficients satisfont aussi à la relation obtenue en échangeant les indices $1$ et $2$ et remplaçant l'indice $3$ par l'indice $4$ dans (\ref{eq281}).}\footnote{Dans le cas des diagrammes à boucle interne, l'existence d'un point d'annulation triple est moins surprenante puisque le dénominateur de (\ref{eq221}) comporte un trinôme au carré. En fait, il y en a quatre, qui jouent des rôles équivalents. Pour les choix de signes $\veps=s_1 s_2$ et $\veps'=s_2 s_3 s_4$, les coefficients de $\tau$ et $\tau'$ dans les formes linéarisées sont tous négatifs; en les mettant à $-1$ et en prenant $\zetat=\ii\delta$ pour simplifier, on tombe sur $\int_{[-1,1]^2} 4\dd \tau\dd \tau'/[(\ii\delta-\tau)^2 (\ii\delta -\tau-\tau')]$, qui est un $O(1)$ lorsque $\delta\to 0^+$.}
\be
\label{eq283}
J_{\rm div}(\kb,\kb')=\int_{\mathbb{R}^2} \frac{2\dd \tau\dd \tau'}{[\zetat-(\frac{3\gamma}{4})^{1/2}s_1\qb\kb \tau][\zetat-(\frac{3\gamma}{4})^{1/2}s_3\qb\kb' \tau']\{\zetat-(\frac{3\gamma}{4})^{1/2}s_1\kb[(\qb-s_3\kb')\tau +s_3\kb'\tau']\}}=\frac{8\pi^2(1-s_1)(1-s_2s_4)}{3\gamma\zetat \qb\kb\kb'(s_1\kb-s_3\kb')}
\ee
et conclut à l'absence de divergence (il faudrait que la branche supérieure gauche de la boucle à pont soit orientée vers la gauche, et que les deux branches inférieures soient d'orientations opposées, ce qui ne se produit pas sur la figure \ref{fig8}).

\paragraph{Implications physiques} À partir du développement de la fonction énergie propre à l'ordre quatre en $H_3$, et toujours à l'ordre dominant (deux) en $\epsilon$, nous formons une nouvelle approximation de la fonction de Green, 
\be
\label{eq289}
\tilde{\mathcal{G}}_{\qb}^{(\leq 4,2)}(\zetat)=\frac{1}{\zetat-\tilde{\Sigma}_{\qb}^{(2,2)}(\zetat)-\tilde{\Sigma}_{\qb}^{(4,2)}(\zetat)}
\ee
On pourrait croire qu'elle est plus précise et plus performante que celle de la section \ref{sec2}. Malheureusement, à cause de la divergence (\ref{eq280}), il n'en est rien: dans le régime d'interaction faible, pourtant le plus favorable, elle admet un pôle $\zetat^{n\phi}_{\qb}$ dans le demi-plan supérieur (et son complexe conjugué dans le demi-plan inférieur), en l'absence, faut-il le préciser, de tout prolongement analytique, d'où l'exposant $n\phi$ (\g{non physique}). Ceci viole en effet la propriété fondamentale d'analyticité de la fonction de Green sur $\mathbb{C}\setminus\mathbb{R}$ \cite{FW}, et conduit par exemple à un signal divergeant exponentiellement en temps dans (\ref{eq012}) si le chemin $C_+$ passe au-dessus de $\zetat^{n\phi}_{\qb}$, ou ne commençant pas à un s'il passe en dessous. Pour montrer l'existence de ce pôle, réduisons l'équation implicite
\be
\label{eq290}
\zetat_{\qb}^{n\phi} = \tilde{\Sigma}_{\qb}^{(2,2)}(\zetat_{\qb}^{n\phi})+\tilde{\Sigma}_{\qb}^{(4,2)}(\zetat_{\qb}^{n\phi})
\ee
à son ordre dominant en supposant que $\zetat_{\qb}^{n\phi}$ tend vers zéro comme $(\rho\xi^2)^{-1}$ (par parties imaginaires positives). Puisque $\tilde{\Sigma}_{\qb}^{(2,2)}(\zetat)$ est déjà d'ordre un en $(\rho\xi^2)^{-1}$, et que c'est une fonction dérivable de $\zetat$ en $\ii 0^+$ (voir l'\ref{app1}), on peut la remplacer par sa valeur en ce point. Dans le cas de $\tilde{\Sigma}_{\qb}^{(4,2)}(\zetat)$, il faut garder sa partie divergente (\ref{eq280}), le préfacteur d'ordre deux $(\rho\xi^2)^{-2}$ qu'elle contient étant remonté à l'ordre un par division par $\zetat$, mais on peut négliger sa partie régulière, d'ordre deux près de $\ii 0^+$. En effectuant les changements de variable universalisants (\ref{eq112},\ref{eq130},\ref{eq131}) de la section \ref{sec2}, il reste
\be
\label{eq291}
\zetac^{n\phi}_{\qb} = \check{\Sigma}_{\qb}^{(2,2)}(\ii 0^+)-\frac{(2/\qb^{2})\int_{\qb}^{\eta}\kb\,\dd\kb \int_0^{\eta-\kb}\kb'\dd\kb' (\kb-\qb)(\kb'+\qb)(\bar{n}_{\kb-\qb}^{\rm lin}-\bar{n}_{\kb}^{\rm lin})(\bar{n}_{\kb'+\qb}^{\rm lin}-\bar{n}_{\kb'+\kb}^{\rm lin})}{\zetac^{n\phi}_{\qb}}
\ee
où le premier terme au second membre figure dans (\ref{eq141}). Comme l'intégrale au numérateur est $>0$, l'équation (\ref{eq291}) sur $\zetac^{n\phi}_{\qb}$ admet une solution imaginaire pure dans le demi-plan supérieur.\footnote{Posons $\zetac^{n\phi}_{\qb}=\ii y$. Alors $y=-2B+A/y$ où les coefficients $A$ et $B$ sont réels $>0$. L'équation du second degré sur $y$ qui en résulte admet une (et une seule) racine $>0$, $y_0=A/[(A+B^2)^{1/2}+B]$. Lorsque $\qb\to 0$, $B$ a une limite finie et non nulle au vu de (\ref{eq141}), alors que $A$ diverge comme $1/\qb$, si bien que $\zetac^{n\phi}_{\qb}$ diverge comme $\qb^{-1/2}$ et $\zetat_{\qb}^{n\phi}$ tend vers zéro comme $\qb^{1/2}$. Ainsi, la condition $\im\zetat^{n\phi}_{\qb}\lll|\tilde{\Sigma}_{\qb}^{(2,2)}(\ii 0^+)|=O(\qb)$, qui rend le pôle non physique sans effet sur le signal dans sa mi-largeur temporelle, ne peut être satisfaite à nombre d'onde $\qb\lesssim 1$. Elle peut l'être à grand nombre d'onde $\qb\gg 1$, où $B\approx \qb^2$ au vu de (\ref{eq141}) et $A\sim 2\exp(-\qb)$.} Pour être complets, signalons que $\tilde{\Sigma}_{\qb}^{(4,2)}(\zetat)$ admet une autre divergence linéaire sur l'axe réel, en $\zetat_0=-3\gamma\qb^3/32$,\footnote{Remarquons que $\zetat_0$ est aussi le point de branchement de $\tilde{\Sigma}_{\qb}^{(2,2)}(\zetat)$, voir la section \ref{sec2.1} et la note \ref{note30}.} comme le montre l'\ref{app4}, moins dommageable car n'impliquant pas l'existence d'un pôle non physique de la fonction de Green dans la limite d'interaction faible (il suffit que l'on ait $|\tilde{\Sigma}_{\qb}^{(2,2)}(\ii 0^+)|\lll|\re\zetat_0|$ pour que cette divergence ait peu d'effet sur l'amortissement du signal dans sa mi-largeur). 

\paragraph{Applicabilité de l'ordre quatre} L'apparition d'un pôle non physique oblige-t-elle à remettre en question la validité du développement diagrammatique et par là, celle de la méthode des fonctions de Green à $N$ corps ? Non. La théorie était prévue pour être utilisée dans la limite d'interaction faible $(\rho\xi^2)^{-1}\to 0$ à température réduite fixée $\epsilon=k_B T/mc^2$. Nous avons cependant pris la limite $\epsilon\to 0$ à ordre de développement fixé en $(\rho\xi^2)^{-1}$, ce qui échoue en dimension $d=2$. Montrons que tout va bien, c'est-à-dire que le pôle non physique disparaît, lorsque $\rho\xi^2\epsilon^2$ est assez grand (à $\qb$ fixé). 

À cette fin, voyons comment l'analyse locale conduisant à (\ref{eq282}) est modifiée lorsque $\epsilon$ est petit mais pas infinitésimal, en nous limitant directement au diagramme $P_5$, donc en prenant $s_1=s_4=s_5=1$, $s_2=s_3=-1$. Si $\zetat=0$, le dénominateur d'énergie de type II dans (\ref{eq233}), qui est aussi celui de (\ref{eq282}), s'écrit $(-\zeta_1)(-\zeta_2)(-\zeta_3)$, où les défauts d'énergie $\zeta_i$ sont ceux de l'équation (\ref{eq238}). Le premier facteur $-\zeta_1=-(\veps_{\kk}-\veps_{\kk-\qq}-\veps_{\qq})$ est fonction du seul angle $\theta=\epsilon\thetat$ entre $\qq$ et $\kk$, et continue à s'annuler sur deux droites verticales dans le plan $(\thetat,\thetat')$, à des abscisses légèrement modifiées, ce qui change de manière négligeable (par un facteur $1+O(\epsilon)^2$) le coefficient de $\tau$ dans (\ref{eq282}); le deuxième facteur $-\zeta_2=-(-\veps_{\kk'}+\veps_{\qq+\kk'}-\veps_{\qq})$ est fonction du seul angle $\theta'=\epsilon\thetat'$ entre $\qq$ et $\kk'$, et continue à s'annuler sur deux droites horizontales, légèrement déplacées, avec les mêmes conséquences négligeables sur le coefficient de $\tau'$ dans (\ref{eq282}). En revanche, le lieu d'annulation du troisième facteur, à savoir $-\zeta_3=-(\veps_{\kk}+\veps_{\qq+\kk'}-\veps_{\kk+\kk'}-\veps_{\qq})$, n'est plus exactement une conique dans le plan $(\thetat,\thetat')$ et passe maintenant un peu à côté des points d'intersection $P_{-1,1}$ et $P_{1,-1}$ des droites horizontales et verticales: ceci change très légèrement les coefficients de la combinaison linéaire de $\tau$ et $\tau'$ dans (\ref{eq282}), mais surtout ajoute, à côté de $\zetat$, la valeur de $-\zetat_3$ en $P_{-1,1}$ désormais non nulle:
\be
\label{eq292}
(-\zetat_3)_{P_{-1,1}} \underset{\epsilon\to 0}{\sim} -\frac{27}{64}\gamma^2\epsilon^2\kb\kb'(\kb-\qb)(\kb+\kb')(\kb'+\qb)
\ee
Nous recalculons en conséquence le troisième membre de (\ref{eq282}):
\be
\label{eq293}
K_{\rm div}^{P_5}(\kb,\kb')=-\frac{32\pi^2}{3\gamma\qb\kb\kb'(\kb+\kb')} \frac{1}{\zetat-\frac{27}{64}\gamma^2\epsilon^2\qb\kb\kb'(\kb-\qb)(\kb'+\qb)}
\ee
La nouvelle équation sur le pôle non physique
\be
\label{eq294}
\zetac^{n\phi}_{\qb} = \check{\Sigma}_{\qb}^{(2,2)}(\ii 0^+)-\frac{2}{\qb^{2}}\int_{\qb}^{\eta}\kb\,\dd\kb \int_0^{\eta-\kb}\kb'\dd\kb' \frac{(\kb-\qb)(\kb'+\qb)(\bar{n}_{\kb-\qb}^{\rm lin}-\bar{n}_{\kb}^{\rm lin})(\bar{n}_{\kb'+\qb}^{\rm lin}-\bar{n}_{\kb'+\kb}^{\rm lin})}{\zetac^{n\phi}_{\qb}-\frac{9}{8}\gamma u \epsilon^2\kb\kb'(\kb-\qb)(\kb'+\qb)}
\ee
cesse d'admettre une solution dans le demi-plan supérieur lorsque $\gamma u\epsilon^2>h(\qb)$, où $u$ est le paramètre universel (\ref{eq112}) et $h(\qb)$ est fonction exclusivement de $\qb$ (et de la coupure $\eta$ si on ne la fait pas tendre vers l'infini dans la théorie quantique), comme on le voit en prenant formellement la limite $\gamma u \epsilon^2\to +\infty$ dans (\ref{eq294}). Pour le champ classique, nous trouvons numériquement que $h(\qb)$ est une fonction décroissante de $\qb$; le problème d'applicabilité de l'ordre quatre pour de faibles valeurs de $\epsilon$ subsiste donc dans la limite $\qb\to 0$.

\paragraph{Conclusion} À nombre d'onde réduit $\qb$ fixé du mode de phonon, la condition d'applicabilité de la théorie diagrammatique à l'ordre quatre en $H_3$ n'est pas seulement que le paramètre universel $u$ soit assez grand (régime d'interaction faible attendu) mais aussi que le produit $u\epsilon^2$ soit assez élevé. Elle présente donc une incompatibilité de fond avec la limite étudiée dans cette article, celle de basse température $\epsilon\to 0$ à force des interactions fixée. Ainsi, pour les simulations numériques de la figure \ref{fig2}, où $\qb=1/4$ et $\gamma=1$, nous trouvons, en suivant par la méthode de Newton jusqu'à disparition la racine de l'équation (\ref{eq294}) écrite en champ classique, qu'il faudrait avoir $u\epsilon^2>10,\!6$ soit $\rho\xi^2>23,\!4$ pour $\epsilon=1/2$ et $\rho\xi^2>52,\!6$ pour $\epsilon=1/3$. Or $\rho\xi^2$ vaut au plus $10$ sur la figure \ref{fig2} ; la théorie à l'ordre quatre développée dans cette section est donc inapplicable.


\section{Théories non perturbatives en le couplage entre phonons}
\label{sec4}

Comme nous l'avons vu dans la section \ref{sec3}, la théorie à l'ordre quatre en $H_3$ n'est pas utilisable en dimension $d=2$ dans la limite de basse température $\epsilon=k_B T/mc^2\to 0$. Pour rendre compte de l'écart entre la théorie à l'ordre deux en $H_3$ et les simulations sur le temps de décroissance à $1/\eee$ du mode de phonon $\qq$ et sur la partie réelle du signal associé (\ref{eq012}) dans le régime l'interaction faible $(\rho\xi^2)^{-1}\ll 1$, voir les figures \ref{fig1}d et \ref{fig2}d, nous proposons ici une théorie non perturbative guidée par des considérations physiques. En bref, nous régularisons la partie divergente de l'ordre quatre de la fonction énergie propre en donnant une durée de vie finie aux phonons dits virtuels (ou intermédiaires) des processus collisionnels à quatre phonons ; ceci fait apparaître une contribution qui se révèle être d'ordre deux en $H_3$ et qui vient donc corriger l'expression de l'énergie propre à l'ordre dominant en la force des interactions. Nous présentons une régularisation de type physique atomique dans la section \ref{sec4.1} (les phonons virtuels $\KK$ acquièrent une énergie complexe $\veps_\KK\to \veps_\KK +\Sigma_\KK^{(2,2)}(\ii 0^+)=\veps_\KK-(\ii\hbar/2)\Gamma_\KK$) et une régularisation de type matière condensée dans la section \ref{sec4.2} (on garde dans l'énergie complexe des phonons virtuels la dépendance en fréquence de la fonction énergie propre $\Sigma_\KK^{(2,2)}(\zeta)$). Enfin, dans la section \ref{sec4.3}, nous donnons la nouvelle forme de la théorie dans la limite $(\rho\xi^2)^{-1}\to 0$, commune aux deux régularisations et qui remplace avec succès la règle d'or de Fermi.

\subsection{Régularisation de type physique atomique}
\label{sec4.1}

Nous tirons de la section \ref{sec3.3} l'enseignement que l'ajout d'une partie complexe infinitésimale $\ii 0^+$ à $\zeta=z-\veps_\qq$, très habituel dans la méthode des fonctions de Green, ne suffit pas à donner une valeur finie à $\tilde{\Sigma}_\qq^{(4,2)}(\zeta=0)$. Or, les collisions entre les phonons, qui les rendent instables, fournissent naturellement une contribution imaginaire non infinitésimale à côté de $\zeta$ dans les dénominateurs d'énergie. En suivant la prescription de la physique atomique, nous donnons donc aux phonons virtuels une énergie complexe
\be
\label{eq300}
\veps_\KK\to\veps_\KK+\Sigma_\KK^{(2,2)}(\ii 0^{\pm})=\veps_\KK\mp \frac{\ii\hbar}{2}\Gamma_\KK
\ee
où $\Gamma_\KK$ est le taux d'amortissement des phonons $\KK$ de la règle d'or de Fermi à l'ordre deux en $\epsilon$, voir l'équation (\ref{eq033}). Les phonons virtuels, par définition ni entrants ni sortants dans la formulation de Landau de la section \ref{sec3.2}, sont ceux dont les vecteurs d'onde ne sont pas indiqués sur la figure \ref{fig8}.  Le signe à prendre dans (\ref{eq300}) devant $\Gamma_\KK$ ou dans l'argument de $\Sigma_\KK$ est celui du haut si le phonon descend le temps (ligne interne orientée vers la droite) ou celui du bas si le phonon remonte le temps (ligne interne orientée vers la gauche).\footnote{ Dans la formulation de Landau, on choisit $-\ii(\hbar/2)\Gamma_\KK$ dans la première somme de l'équation (\ref{eq258}) et son opposé dans la deuxième somme, pour les raisons  expliquées après (\ref{eq257}), ce qui conduit au même résultat. } En revanche, nous ne donnons pas une énergie complexe aux phonons entrants ou sortants car ceci remettrait en question la conservation de l'énergie du gaz lors des processus collisionnels. Par conséquent, il n'y a pas de modification à apporter à $\Sigma_{\qq}(\zeta)$ à l'ordre deux en $H_3$,  qui ne comporte aucun phonon virtuel. 

Nous appliquons la procédure (\ref{eq300}) à la partie divergente de $\tilde{\Sigma}^{(4,2)}_{\qb}$, qui provient de la contribution de type II du diagramme $P_5$, et nous omettons toutes les autres contributions à $\tilde{\Sigma}^{(4,2)}_{\qb}$, qui se révèlent être des $O((\rho\xi^2)^{-2})$ et sous-dominantes lorsque $(\rho\xi^2)^{-1}\to 0$.\footnote{En l'absence de divergence en $\zeta=0$, l'ajout d'une partie imaginaire à $\veps_\KK$ ne change pas l'ordre dominant du diagramme, qui reste d'ordre quatre en $H_3$.} Comme le montre la figure \ref{fig8}, les phonons virtuels sont alors $\kk_2=\qq-\kk$ et $\kk_3=\kk'$ dans les notations de la figure \ref{fig7}. Tous deux remontent le temps ($s_2=s_3=-1$), si bien que les quantités $\zeta_1$ et $\zeta_2$ dans (\ref{eq233}) et les coefficients $A_1^P$ et $A_2^P$ dans (\ref{eq243}) sont changés comme suit, 
\be
\label{eq301}
\zeta_1\to\zeta_1-\ii\frac{\hbar}{2}\Gamma_{\qq-\kk} \quad ;\quad  \zeta_2\to\zeta_2-\ii\frac{\hbar}{2}\Gamma_{\kk'} \quad ; \quad A_1^P\to A_1^P-\frac{\ii}{2}\tilde{\Gamma}_{\kb-\qb} \quad ; \quad A_2^P\to A_2^P-\frac{\ii}{2}\tilde{\Gamma}_{\kb'}
\ee
l'ordre des facteurs au dénominateur de (\ref{eq243}) respectant celui de (\ref{eq233}) ; ici $\tilde{\Gamma}_{\bar{K}}=\hbar\Gamma_{\KK}/k_BT\epsilon^2$ dans l'esprit de l'équation (\ref{eq022}). L'ancienne partie divergente de l'intégrale angulaire (\ref{eq282}) devient pour $\im\zetat>0$:\footnote{On continue à linéariser autour du même point de triple intersection des lieux d'annulation en regroupant les corrections complexes avec la variable $\zetat$ si bien qu'au second membre de (\ref{eq282}), on remplace $\zetat$ par $\zetat+(\ii/2)\tilde{\Gamma}_{\kb-\qb}$ dans le premier facteur et par $\zetat+(\ii/2)\tilde{\Gamma}_{\kb'}$ dans le deuxième facteur au dénominateur.}
\be
\label{eq302}
K_{\mbox{\scriptsize div. rég.}}^{P_5^{\rm II}}(\kb,\kb')=-\frac{32\pi^2}{3\gamma\qb\kb\kb'} \frac{1}{\zetat(\kb+\kb')+\frac{\ii}{2}\kb'\tilde{\Gamma}_{\kb-\qb}+\frac{\ii}{2}(\kb-\qb)\tilde{\Gamma}_{\kb'}}
\ee
Après remplacement de $K(\kb,\kb')$ par (\ref{eq302}) dans (\ref{eq242}) et ajout de la contribution d'ordre deux en $H_3$, nous aboutissons à la prescription non perturbative  pour la fonction énergie propre corrigeant le résultat de la section \ref{sec2}:
\be
\label{eq303}
\boxed{\tilde{\Sigma}_{\rm corr}^{(2,2)}(\qb,\zetat) = \tilde{\Sigma}_{\qb}^{(2,2)}(\zetat) + \tilde{\Sigma}^{(4,2)}_{\mbox{\scriptsize div. rég.}}(\qb,\zetat)}
\ee
où la première contribution est celle de l'équation (\ref{eq125}) et la seconde, régularisation de la partie divergente de $\tilde{\Sigma}_{\qb}^{(4,2)}(\zetat)$, s'écrit, toujours pour $\im\zetac>0$:
\be
\label{eq304}
\boxed{\tilde{\Sigma}^{(4,2)}_{\mbox{\scriptsize div. rég.}}(\qb,\zetat)=-\frac{2}{3\pi^2\gamma} \left(\frac{9(1+\Lambda)^2}{8\rho \xi^2}\right)^2
\int_{\qb}^\eta \dd\kb \int_0^{\eta-\kb} \dd\kb' \frac{\kb \kb' (\kb-\qb)(\kb'+\qb)(\kb+\kb')}
{\tilde{\zeta}(\kb+\kb')+\frac{\ii}{2}\kb'\tilde{\Gamma}_{\kb-\qb}+\frac{\ii}{2}(\kb-\qb)\tilde{\Gamma}_{\kb'}} 
(\bar{n}^{\rm lin}_{k'+q}-\bar{n}^{\rm lin}_{k'+k})(\bar{n}^{\rm lin}_{k-q}-\bar{n}^{\rm lin}_{k})}
\ee
En reportant la fonction (\ref{eq303}) dans l'équation (\ref{eq023}) (avec $\nu=\sigma$ en dimension $d=2$) dans sa version de champ classique, nous obtenons pour la force des interactions la plus faible un bien meilleur accord avec les simulations numériques, voir le trait plein orange sur la figure \ref{fig3}a et le trait plein noir sur la figure \ref{fig3}b, à comparer respectivement au trait plein noir sur la figure \ref{fig3}a et au tireté noir sur la figure \ref{fig3}b correspondant à l'ordre deux en $H_3$ non corrigé. 

\subsection{Régularisation de type matière condensée}
\label{sec4.2}

On sait bien en physique de la matière condensée que l'énergie complexe d'une quasi-particule dans un système en interaction dépend de la fréquence à laquelle on la regarde, c'est-à-dire de la fréquence à laquelle on mesure les fonctions de réponse. C'est d'ailleurs l'essence de ce qu'est la fonction $\Sigma$ dans (\ref{eq012}). Il faut donc remplacer dans (\ref{eq300}) l'énergie complexe fixée $\mp(\ii\hbar/2)\Gamma_\KK$ du phonon virtuel par une fonction de $\zeta$:
\be
\label{eq305} 
\veps_\kk\to \veps_\KK+\Sigma_\KK^{(2)}(\pm(\zeta+\Delta E)-\veps_\KK)
\ee
où il faut prendre le signe $+$ si le phonon $\KK$ descend le temps, et le signe $-$ s'il le remonte. La quantité $\Delta E$ s'obtient en réinterprétant le dénominateur d'énergie auquel participe le phonon $\KK$ dans $\Sigma^{(4)}$ comme celui du propagateur libre de ce phonon à l'énergie complexe $z_{\KK}=\zeta+\Delta E$. Ainsi, si le phonon $\KK$ descend le temps (signe $s_\KK=+1$ devant $\veps_\KK$  dans (\ref{eq238})), il contribuera à $\Sigma^{(4)}$ dans les formes simplifiées (\ref{eq213},\ref{eq233}) au travers du facteur $1/(\zeta+\Delta E-\veps_\KK)$, qui n'est autre que son propagateur non perturbé à l'énergie complexe $z_\KK$ donc à l'énergie complexe redéfinie comme dans (\ref{eq013}) $\zeta_\KK=z_\KK-\veps_\KK$. Nous proposons donc de le remplacer par la fonction de Green (\ref{eq012}) du phonon $\KK$, avec $\Sigma_{\KK}$ calculé à l'ordre deux en $H_3$:
\be
\label{eq306}
\frac{1}{\zeta+\Delta E-\veps_\KK}\to \frac{1}{\zeta+\Delta E-\veps_\KK-\Sigma_\KK^{(2)}(\zeta+\Delta E-\veps_\KK)}
\ee
ce qui explique la substitution (\ref{eq305}). Si le phonon virtuel remonte le temps (signe $s_\KK=-1$), nous nous ramenons au cas précédent par la chaîne de transformations
\be
\label{eq307}
\frac{1}{\zeta+\Delta E+\veps_\KK} = -\frac{1}{-(\zeta+\Delta)-\veps_\KK} \longrightarrow -\frac{1}{-(\zeta+\Delta E)-\veps_\KK-\Sigma_\KK^{(2)}(-(\zeta+\Delta E+\veps_\KK))} = \frac{1}{\zeta+\Delta E+\veps_\KK+\Sigma_\KK^{(2)}(-(\zeta+\Delta E+\veps_\KK))}
\ee
 en accord là aussi avec (\ref{eq305}). En pratique, comme dans la section \ref{sec4.1}, on garde seulement la partie divergente de $\tilde{\Sigma}^{(4,2)}_{\qb}$ (provenant de la contribution de type II du diagramme $P_5$), dans laquelle les phonons virtuels sont de vecteur d'onde $\KK=\kk-\qq$ (avec $\Delta E=\veps_{\qq}-\veps_\kk$) ou $\KK=\kk'$ (avec $\Delta E=\veps_{\qq}-\veps_{\kk'+\qq}$) et remontent tous deux le temps. Après régularisation, la partie divergente (\ref{eq282}) de l'intégrale angulaire devient
\be
\label{eq308}
K_{\mbox{\scriptsize div. rég.}}^{P_5^{\rm II}}(\kb,kb') = \int_{\mathbb{R}^2} \frac{2\dd \tau \dd \tau'}{[\zetat-\alpha \tau +\tilde{\Sigma}_{\kb-\qb}^{(2,2)}(-\zetat+\alpha \tau)] [\zetat-\alpha' \tau' + \tilde{\Sigma}_{\kb'}^{(2,2)}(-\zetat+\alpha' \tau')] (\zetat-\beta \tau -\beta' \tau')}
\ee
où les coefficients  sont ceux de $\tau$ et $\tau'$  au second membre de (\ref{eq282}): $\alpha=(3\gamma/4)^{1/2}\qb\kb$, $\alpha'=-(3\gamma/4)^{1/2}\qb\kb'$, $\beta=-(3\gamma/4)^{1/2}\kb\kb'$, $\beta'=(3\gamma/4)^{1/2}(\kb-\qb)\kb'$. Comme les fonctions $\tilde{\Sigma}^{(2,2)}$ au dénominateur de (\ref{eq308}) admettent des lignes de coupure dans le plan complexe, le théorème de Cauchy ne permet plus de calculer l'intégrale double et l'on peut seulement intégrer une fois, par exemple sur $\tau'$, en se limitant comme d'habitude à $\im\zetat>0$.\footnote{Le dénominateur de l'intégrande dans (\ref{eq308}) admet une ligne de coupure vis-à-vis de la variable $\tau$ uniquement dans le demi-plan supérieur strict car $\im(\zetat/\alpha)>0$, et vis-à-vis de la variable $\tau'$ uniquement dans le demi-plan inférieur strict car $\im(\zetat/\alpha')<0$. On peut donc intégrer sur $\tau'$ par le théorème des résidus en refermant le chemin d'intégration par un grand demi-cercle à l'infini dans le demi-plan supérieur ; ce chemin entoure nécessairement le pôle $\tau_0'=(\zetat-\beta \tau)/\beta'$ issu du troisième facteur au dénominateur de (\ref{eq308}) car $\im \tau_0'>0$. L'intégrande du résultat (\ref{eq310}) admet cette fois une ligne de coupure vis-à-vis de la variable $\tau$ aussi bien dans le demi-plan supérieur ($\im(\zetat/\alpha)>0$) que dans le demi-plan inférieur ($\im(\zetat/\alpha')<0$) ce qui empêche de refermer le chemin d'intégration seulement à l'infini.}  
En reportant la contribution correspondante à $K(\kb,\kb')$ dans (\ref{eq242}), nous aboutissons à la fonction énergie propre corrigée de même forme que (\ref{eq303}), avec cette fois
\begin{multline}
\label{eq310}
\tilde{\Sigma}^{(4,2)}_{\mbox{\scriptsize div. rég.}}(\qb,\zetat)= \qb\left(\frac{9(1+\Lambda)^2}{8\rho \xi^2}\right)^2 \\
\times \int_{\qb}^\eta \frac{\kb \dd\kb}{(2\pi)^2}
\int_0^{\eta-\kb} \frac{\kb' \dd\kb'}{(2\pi)^2}
\int_{-\infty}^{+\infty} \dd \tau 
\frac{(-4\ii\pi/\beta')\kb \kb' (\kb-\qb)(\kb'+\qb)(\kb+\kb')(\bar{n}^{\rm lin}_{k'+q}-\bar{n}^{\rm lin}_{k'+k})(\bar{n}^{\rm lin}_{k-q}-\bar{n}^{\rm lin}_{k})}
{\left[\tilde{\zeta}-\alpha \tau +\tilde{\Sigma}^{(2,2)}_{\kb-\qb}(-\tilde{\zeta}+\alpha \tau)\right]
\left[-\frac{\beta}{\beta'}(\tilde{\zeta}-\alpha'\tau)+\tilde{\Sigma}^{(2,2)}_{\kb'}
\left(\frac{\beta}{\beta'}(\tilde{\zeta}-\alpha'\tau)\right)\right]}
\end{multline}
sachant que $\alpha'-\beta'=\beta$. Les résultats de cette approche sont représentés par des étoiles orange sur la figure \ref{fig3}a et par des pointillés rouge et noir sur la figure \ref{fig3}b. \footnote{Leur obtention est assez lourde puisque chaque évaluation des fonctions $\tilde{\Sigma}_{\bar{K}}^{(2,2)}$ requiert une intégration numérique sur un nombre d'onde comme dans (\ref{eq125}). Nous avons discrétisé les nombres d'onde réduits avec un pas $1/64$ (résultats sur la figure) mais nous avons vérifié qu'un pas $1/96$ conduit à des résultats très proches.} Les deux procédures de régularisation, de type physique atomique ou matière condensée, donnent en définitive des résultats proches entre eux et proches des simulations, ce qui nous rassure sur leur validité. 

\subsection{Théorie régularisée dans la limite d'interaction faible}
\label{sec4.3}

Nous calculons ici la limite des théories régularisées lorsque $(\rho\xi^2)^{-1}\to 0$. Pour ce faire, nous mettons la variable $\zetat$ et les fonctions $\tilde{\Sigma}$ à l'échelle de leurs valeurs caractéristiques $\propto (\rho\xi^2)^{-1}$ au moyen des changement de variable (\ref{eq130}) et (\ref{eq131}), repérés par un accent tchèque (un facteur $\qb$ a été sorti en prévision d'un passage à la limite $\qb\to 0$, et tous les paramètres sont rassemblés dans la quantité $u\propto\rho\xi^2$ de l'équation (\ref{eq112})), puis nous passons à la limite $u\to+\infty$ à $\check{\zeta}$ fixé dans les équations (\ref{eq303}) et (\ref{eq304}). Contrairement à ce qui se passait à l'ordre deux en $H_3$ non corrigé (\ref{eq131}), il reste cette fois une dépendance en $\check{\zeta}$:\footnote{On montre en deux étapes que la fonction de Green associée à la fonction énergie propre (\ref{eq311}) n'admet, comme il se doit, aucun pôle dans le demi-plan complexe supérieur. Dans un premier temps, on établit la majoration $\im\check{\Sigma}_{{\rm corr}\,\infty}^{(2,2)}(\qb,\check{\zeta})\leq\im\check{\Sigma}_{{\rm corr}\,\infty}^{(2,2)}(\qb,0)\ \forall \check{\zeta}\in\mathbb{C}$, $\im\zetac\geq 0$; à cette fin, on écrit le dénominateur de l'intégrande dans (\ref{eq311}) sous la forme $x+\ii y$, avec $x\in\mathbb{R}$ et $y\in\mathbb{R}^+$, puis on utilise la chaîne $\im[-1/(x+\ii y)]=y/(x^2+y^2)\leq 1/y\leq 2/[\kb'(\kb-\qb)(\check{\Gamma}_{\kb-\qb}+\check{\Gamma}_{\kb'})]$. Dans un second temps, on vérifie que $\im\check{\Sigma}_{{\rm corr}\,\infty}^{(2,2)}(\qb,0)<0\ \forall\qb>0$. Il en résulte que l'équation $\check{\zeta}=\check{\Sigma}_{{\rm corr}\,\infty}^{(2,2)}(\qb,\check{\zeta})$ ne peut avoir de solution $\check{\zeta}$ de partie imaginaire positive sans conduire à la contradiction $\im\check{\zeta}<0$.}
\be
\label{eq311}
\boxed{\check{\Sigma}_{\rm corr}^{(2,2)}(\qb,\check{\zeta}) \underset{u \to +\infty}{\stackrel{\im\check{\zeta}\,>\,0}{\longrightarrow}} \check{\Sigma}_{{\rm corr}\,\infty}^{(2,2)}(\qb,\check{\zeta})\equiv-\frac{\ii}{2}\check{\Gamma}_{\qb}-\frac{2}{\qb}\int_{\qb}^\eta \dd\kb \int_0^{\eta-\kb} \dd\kb' \frac{\kb\kb'(\kb-\qb)(\kb'+\qb)(\kb+\kb')(\bar{n}^{\rm lin}_{k'+q}-\bar{n}^{\rm lin}_{k'+k}) (\bar{n}^{\rm lin}_{k-q}-\bar{n}^{\rm lin}_{k})} {\check{\zeta}\qb(\kb+\kb')+\frac{\ii}{2}\kb'(\kb-\qb)\left(\check{\Gamma}_{\kb-\qb}+\check{\Gamma}_{\kb'}\right)}}
\ee
Ici, $\check{\Gamma}_{\bar{K}}=2\ii\check{\Sigma}_{\bar{K}}^{(2,2)}(\ii 0^+)$ est le taux d'amortissement de la règle d'or à l'ordre dominant en $\epsilon$ et adimensionné comme dans (\ref{eq130}), $\check{\Gamma}_{\bar{K}}=8u\Gamma_{\KK}/(3\gamma\epsilon^2\hbar cK)$ ; on a $\check{\Gamma}_{\bar{K}}=\check{\Gamma}_{\bar{K}}^{\rm class}=2\eta-\bar{K}$ au vu de (\ref{eq113}) dans le cas du champ classique et $\check{\Gamma}_{\bar{K}}=4\zeta(2)+\bar{K}^2/6$ au vu de (\ref{eq033}) ou (\ref{eq141}) dans le cas du champ quantique de coupure infinie.
On obtient exactement la même expression que (\ref{eq311}) en partant de la forme régularisée (\ref{eq310}).\footnote{On effectue simultanément les changements de variables $\zetat=(3\gamma\qb/8u)\check{\zeta}$ et $\tau=(3\gamma\qb/8u)\check{\tau}$, puis on passe à la limite $u\to +\infty$ sous le signe intégral à $\check{\tau}$ fixé. Alors les lignes de coupure des fonctions $\tilde{\Sigma}^{(2,2)}_{\bar{K}}$ sont rejetées à l'infini, ces fonctions sont remplacées par des constantes, par exemple $\lim_{u\to+\infty} (8u/3\gamma\qb) \tilde{\Sigma}^{(2,2)}_{\kb-\qb}((3\gamma\qb/8u)(\alpha\check{\tau}-\check{\zeta}))=[(\kb-\qb)/\qb] \check{\Sigma}^{(2,2)}_{\kb-\qb}(\ii 0^-)=[(\kb-\qb)/\qb](\ii/2)\check{\Gamma}_{\kb-\qb}$, et l'intégrale sur $\check{\tau}$ peut être calculée par la formule de Cauchy. On retrouve (\ref{eq311}).} De façon remarquable, le signal (\ref{eq012}) associé à la fonction énergie propre (\ref{eq311}) prend des valeurs réelles à tout temps.\footnote{Ceci résulte de la propriété $[\check{\Sigma}(\check{\zeta})]^*=-\check{\Sigma}(-\check{\zeta}^*)$, $\forall \check{\zeta}\in\mathbb{C}\setminus\ii\mathbb{R}^-$, satisfaite par la fonction $\check{\Sigma}_{{\rm corr}\,\infty}^{(2,2)}(\qb,\check{\zeta})$ (si on étend l'expression au second membre telle quelle au cas $\im\check{\zeta}<0$).} Il y a donc bien, dans la limite $(\rho\xi^2)^{-1}\to 0$, un désaccord persistant avec la règle d'or de Fermi sur la partie réelle du signal, comme le laissait prévoir la figure \ref{fig3}b. La figure \ref{fig9} représente, en fonction du nombre d'onde réduit $\qb$, le temps de décroissance à $1/\eee$ du mode de phonon $\qq$ déduit de (\ref{eq311}) pour la théorie de champ quantique (coupure $\eta\to +\infty$). C'est notre prédiction à l'ordre dominant à basse température et en la force des interactions pour un superfluide bidimensionnel, qui remplace et corrige celle de la règle d'or de Fermi représentée en tireté sur la figure \ref{fig9}.

\begin{SCfigure}
% Figure removed
\caption{Dans la limite de basse température $\epsilon=k_BT/mc^2\to 0$ puis d'interaction faible $(\rho\xi^2)^{-1}\to 0$ pour le champ quantique, inverse du temps de décroissance à $1/\eee$ du signal (\ref{eq012}) pour le mode de phonon $\qq$ dans un superfluide bidimensionnel, en fonction du nombre d'onde réduit $\qb$. Trait plein : prédiction des théories non perturbatives de la section \ref{sec4}, correspondant à la fonction énergie propre corrigée (\ref{eq311}) de la section \ref{sec4.3}. Pointillé: pente à l'origine (\ref{eq313}). Tireté: règle d'or (\ref{eq033}). Le taux $\Gamma_u$ servant d'unité est donné par (\ref{eq113}).}
\label{fig9}
\end{SCfigure}

Trois prédictions analytiques simples peuvent être tirées de l'équation (\ref{eq311}). D'abord, dans la limite $\qb\to 0$, la dépendance en $\check{\zeta}$ disparaît au second membre de (\ref{eq311}) et le signal (\ref{eq012}) devient purement exponentiel, $s(t)=\exp(-A\check{t})$ avec
\be
\label{eq312}
A=\frac{1}{2}\check{\Gamma}_0+4\int_0^\eta \dd \kb \int_0^{\eta-\kb} \dd \kb'
\frac{\kb\kb'(\kb+\kb')(\bar{n}_{k'}^{\rm lin}-\bar{n}_{k'+k}^{\rm lin})\frac{\dd}{\dd \kb}\bar{n}_{k}^{\rm lin}}
{\check{\Gamma}_{\kb}+\check{\Gamma}_{\kb'}} =
\left\{ \begin{array}{ll}
[1-4(4 \ln \frac{4}{3} -1)]\eta \simeq 0,\!4\,\eta & \mbox{(champ classique)} \\
0,\!988\,618 \ldots & \mbox{(champ quantique)}
\end{array}
\right.
\ee
Si l'on quitte le temps adimensionné $\check{t}$ (\ref{eq130}), ce qui fait apparaître la quantité $\Gamma_u$ déja introduite dans (\ref{eq113}), on trouve que ceci correspond à un taux d'amortissement corrigé tendant vers zéro linéairement en $\qb$ :
\be
\label{eq313}
\boxed{
\Gamma_q^{{\rm corr}\,\infty} \underset{\qb\to 0}{\sim} A \Gamma_u \qb = A \frac{9k_B T\epsilon^2(1+\Lambda)^2}{4\pi\hbar(3\gamma)^{1/2}\rho\xi^2} \qb \sim \left\{ \begin{array}{ll} \displaystyle\frac{A}{\eta} \Gamma_q^{\rm class} \simeq 0,\!4\,\Gamma_q^{\rm class} & \mbox{(champ classique)} \\ \displaystyle\frac{A}{2\zeta(2)} \Gamma_q \simeq 0,\!3\,\Gamma_q & \mbox{(champ quantique)}\end{array} \right. }
\ee
représenté en pointillé sur la figure \ref{fig9} pour le champ quantique et sur la figure \ref{fig3}a pour le champ classique. Ici, $\Gamma_q^{\rm class}$ et $\Gamma_q$ sont les taux d'amortissement de la règle d'or pour le champ classique (\ref{eq113}) et pour le champ quantique (\ref{eq033}). On voit donc que la pente à l'origine du taux corrigé est considérablement plus faible que celle de la règle d'or. Ensuite, dans la limite $\qb\to +\infty$, bien entendu pour le champ quantique de coupure infinie, l'intégrale au second membre de (\ref{eq311}) tend uniformément vers zéro dans le demi-plan supérieur et la décroissance du signal devient purement exponentielle, avec un taux équivalent à celui de la règle d'or: \footnote{Ce résultat est aussi celui de la température nulle, ce qui explique pourquoi le troisième membre de (\ref{eq315}) est en fait indépendant de la température lorsqu'on revient au nombre d'onde $q$ non adimensionné. Comme il résulte de l'hamiltonien effectif de l'hydrodynamique quantique, il ne vaut alors qu'à l'ordre dominant en $q\xi$ c'est-à-dire dans la partie linéaire du spectre phononique.}
\be
\label{eq315}
\Gamma_q^{{\rm corr}\,\infty} \underset{\qb\to+\infty}{\sim} \Gamma_q \underset{\qb\to+\infty}{\sim} \frac{3 k_B T\epsilon^2(1+\Lambda)^2}{16\pi(3\gamma)^{1/2}\rho\xi^2} \qb^3
\ee
Enfin, dans la limite des temps longs $\check{t}\to +\infty$ à $\qb$ fixé, la décroissance du signal n'est pas exponentielle mais en loi de puissance à logarithme près, prenant même forme pour le champ classique et pour le champ quantique, comme le montre l'\ref{app5}:
\be
\label{eq314}
s^{(2,2)}_{{\rm corr}\,\infty}(t)\underset{\check{t}\to+\infty}{\sim} \frac{\qb^4(\dd\bar{n}^{\rm lin}_q/\dd\qb)\ln(\check{t}^2)}{\left[\check{\Gamma}_0\check{\Sigma}_{{\rm corr}\,\infty}^{(2,2)}(\qb,0)\check{t}\right]^2}
\ee
Pour $\qb=1/4$, on trouve numériquement à partir de la fonction énergie propre (\ref{eq311}) que le comportement asymptotique (\ref{eq314}) n'est pas atteint avant un temps $\Gamma_q t/2=100$, aussi bien pour le champ quantique que pour le champ classique, ce qui va bien au-delà de nos simulations de la figure \ref{fig3}b et nous empêche de le tester. 

\section{Conclusion}
\label{sec5}

Dans ce travail, nous avons considéré le gaz de phonons d'un superfluide bidimensionnel décrit par l'hydrodynamique quantique, une théorie effective de basse énergie, avec un couplage cubique et quartique entre phonons et une branche acoustique $\veps_{\qq}$ fonction convexe du nombre d'onde $q$. Deux paramètres importants caractérisent le système :  la température réduite $\epsilon=k_BT/mc^2$ et la force des interactions $(\rho \xi^2)^{-1}$ entre particules du superfluide, où $c$ est la vitesse du son et $\rho$ la densité du superfluide dans l'état fondamental, $m$ la masse d'une particule et $\xi=\hbar/mc$ la longueur de relaxation.  Dans ce système, nous avons étudié en régime faiblement collisionnel l'amortissement du mode de phonon $\qq$ excité par une impulsion de Bragg en réponse linéaire, en nous limitant à l'ordre dominant en $\epsilon$ car aller au-delà n'aurait pas de sens avec notre hamiltonien modèle. 

Un premier résultat inattendu est qu'en dimension deux, la prise de limite de basse température, $\epsilon\to 0$ à $(\rho \xi^2)^{-1}$ fixé, ne suffit pas à faire entrer le gaz de phonons en régime de couplage faible. Ceci met en échec l'approximation du pôle, la règle d'or de Fermi et l'hypothèse d'une décroissance exponentielle.  En effet, comme nous le montrons par un raisonnement de loi d'échelle, la fonction énergie propre $\Sigma(\qq,z)$ a une valeur typique et une échelle de variation en énergie au voisinage de $z=\varepsilon_\qq$ toutes les deux d'ordre $k_BT \epsilon^2$, alors qu'il faudrait avoir une valeur typique très inférieure à l'échelle de variation, comme c'est le cas en dimension trois, pour pouvoir appliquer la règle d'or.  Ce premier résultat théorique est confirmé par des simulations de champ classique. 

Un second résultat inattendu du cas bidimensionnel est que, même dans la limite d'interaction faible $(\rho \xi^2)^{-1}\to 0$, un calcul perturbatif ne donne pas accès à la fonction énergie propre à l'ordre dominant en $\epsilon$.  Ainsi, le calcul de $\Sigma(\qq,z)$ à l'ordre deux en le couplage cubique entre phonons ne reproduit pas l'amortissement du mode $\qq$ observé numériquement et le calcul à l'ordre quatre présente (après prise de limite $\epsilon\to 0$ à $\qb=\hbar c q/k_B T$ et $(\rho \xi^2)^{-1}$ fixés) une divergence non physique en $z=\varepsilon_\qq$ qui le rend inutilisable. 

Pour sortir de cette impasse, guidés par l'intuition physique, nous donnons une durée de vie finie aux phonons virtuels (présents  seulement dans l'état intermédiaire des processus collisionnels à quatre phonons).  Ceci supprime la divergence dans l'ordre quatre en la transformant en une contribution formellement du second ordre, ce qui corrige le calcul à l'ordre deux. La théorie régularisée se trouve être en bon accord avec les simulations  dans le régime d'interaction faible.  Elle permet de prendre la limite $(\rho \xi^2)^{-1}\to 0$ dans l'expression de $\Sigma(\qq,z)$ à l'ordre dominant en $\epsilon$ et de proposer une loi d'amortissement des phonons remplaçant celle de la règle d'or. La nouvelle loi est non exponentielle, sauf dans la limite $\qb\to 0$ où elle est caractérisé par le taux (\ref{eq313}), linéaire en nombre d'onde comme la règle d'or mais avec un coefficient environ trois fois plus faible, et dans la limite $\qb\to+\infty$ où son taux (\ref{eq315}) reproduit celui de la règle d'or.\\

\noindent {\bf Remerciements~:} Alan Serafin a bénéficié d'un financement par le projet européen de recherche et d'innovation Horizon 2020 macQsimal numéro 820393.

\appendix
\section{Prolongement analytique de $\check{\Sigma}^{(2,2)}_{\qb}(\check{\zeta})$ au demi-plan inférieur et conditions de validité de la règle d'or}
\label{app1}

Pour prolonger analytiquement du demi-plan complexe supérieur au demi-plan inférieur l'expression (\ref{eq131}) de la fonction énergie propre $\check{\Sigma}^{(2,2)}_{\qb}(\check{\zeta})$, nous procédons en deux étapes: (i) nous réduisons (\ref{eq131}) à une intégrale unique, celle d'une fonction analytique d'une variable $x$ bien choisie sur un intervalle de $\mathbb{R}$, et (ii) pour étendre cette intégrale à $\im\zetac<0$, nous déformons le chemin d'intégration sur $x$ dans le plan complexe et faisons tourner les lignes de coupure de la fonction analytique en question. En application, nous donnons des conditions de validité de la règle d'or de Fermi, nécessaires mais cependant pas suffisantes.

\paragraph{Se ramener à une seule intégrale} Dans la contribution de Landau à (\ref{eq131}), c'est-à-dire la seconde intégrale, nous effectuons le changement de variable $x=\kb(\kb+\qb)$ soit $\kb=(x+\qb^2/4)^{1/2}-\qb/2$, où $x\in\mathbb{R}^+$. Dans la contribution de Beliaev, l'intégrande est invariant par changement de $\kb$ en $\qb-\kb$, ce qui autorise à limiter l'intégration sur $\kb$ à l'intervalle $[0,\qb/2]$ au prix d'une multiplication par un facteur $2$; nous effectuons alors le changement de variable devenu injectif $x=\kb(\kb-\qb)$ soit $\kb=\qb/2-(\qb^2/4+x)^{1/2}$, où $x\in[-\qb^2/4,0]$; nous utilisons au préalable l'identité $\bar{n}_k^{\rm lin}=-(1+\bar{n}_{-k}^{\rm lin})$ pour donner au facteur d'amplification bosonique la même structure de différence de nombres d'occupation que dans Landau. Nous aboutissons dans les deux cas au même intégrande, si ce n'est que Beliaev fait apparaître un facteur global $(-x)^{1/2}/(\zetac/u-x)^{1/2}$ là où Landau contient un facteur $x^{1/2}/(x-\zetac/u)^{1/2}$ ; on peut leur donner cependant l'expression commune $(x-\ii 0^+)^{1/2}/(x-\zetac/u)^{1/2}$ au moyen des relations $(-x)^{1/2}=\ii(x-\ii 0^+)^{1/2} \forall x<0$, $x^{1/2}=(x-\ii 0^+)^{1/2} \forall x>0$, $(\zetac/u-x)^{1/2}=\ii(x-\zetac/u)^{1/2}\forall x\in\mathbb{R}$ sachant que $\im(\zetac/u)>0$. Nous aboutissons à l'intégrale unique cherchée:
\be
\label{eq1001}
\check{\Sigma}^{(2,2)}_{\qb}(\check{\zeta})= \frac{1}{2\ii\qb}\int_{-\qb^2/4}^{+\infty}\dd x \frac{x\sqrt{x-\ii 0^+}}{\sqrt{x-\zetac/u}\sqrt{x+\qb^2/4}} \left[\frac{1}{\eee^{\sqrt{x+\qb^2/4}-\qb/2}-1}-\frac{1}{\eee^{\sqrt{x+\qb^2/4}+\qb/2}-1}\right]
\ee

\begin{SCfigure}
\begin{tikzpicture}[x=6mm,y=6mm]
\draw[thick,->,>=stealth](-7,0)--(7,0);
\node(a)at(7,0)[below]{${\rm Re\,}x$};
\draw[thick,->,>=stealth](0,-5)--(0,6);
\node(a)at(0,6)[right]{${\rm Im\,}x$};
\node(a)at(-3,0)[]{$\bullet$};
\node(a)at(-3,-0.15)[above]{$-\bar{q}^2/4$};
\draw[ultra thick](-7,0)--(-3,0);
\draw[ultra thick](-7,0.7)--(0,0.7);
\node(a)at(0,0.7)[]{$\bullet$};
\node(a)at(0,0.7)[right]{${\rm i} 0^+$};
\draw[dashed,ultra thick](-3,0)--(-7,-4);
\draw[dashed,ultra thick](0,0.7)--(4,4.7);
\draw[ultra thick](-7,2)--(3,2);
\node(a)at(3,2)[]{$\bullet$};
\node(a)at(3,2)[below]{$\check{\zeta}/u$};
\draw[dashed,ultra thick](3,2)--(6,5);
\draw[thick,color=black,->,>=stealth](-3,0)+(-180:1)arc(-180:-135:1);
\node(a)at(-4.5,-0.5)[]{$\alpha$};
\draw[thick,color=black,->,>=stealth](0,0.7)+(180:0.8)arc(180:45:0.8);
\node(a)at(-1.5,1.3)[]{$\alpha-\pi$};
\draw[thick,color=black,->,>=stealth](3,2)+(180:0.8)arc(180:45:0.8);
\node(a)at(3.2,3)[]{$\alpha-\pi$};
\draw[thick,color=blue,->,>=stealth](-3,0-0.1)--(-5,-2-0.1);
\draw[thick,color=blue](-3,0-0.1)--(-7,-4-0.1);
\draw[thick,color=red,->,>=stealth](1,-0.1)+(-180:4)arc(-180:-135:4);
\draw[thick,color=red,->,>=stealth](1,-0.1)+(-135:4)arc(-135:0:4);
\draw[->,>=stealth,color=red,thick](5,-0.1)--(6,-0.1);
\node(a)at(-0.25,0)[below]{$0$};
\end{tikzpicture}
\caption{Illustration graphique de la méthode de prolongement analytique de $\check{\Sigma}_{\qb}^{(2,2)}(\zetac)$ du demi-plan supérieur $\im\zetac>0$ au demi-plan inférieur $\im\zetac<0$ à partir de la forme intégrale (\ref{eq1001}). Courbe rouge: première déformation du chemin d'intégration sur la variable $x$ dans le plan complexe. Demi-droites : lignes de singularité de l'intégrande de (\ref{eq1001}) considéré comme une fonction de $x$, en trait plein épais pour leur position originelle, en tireté épais après rotation d'un angle $\alpha$ ou $\alpha-\pi$ comme indiqué ($\alpha=\pi/4$ sur la figure). Disques noirs: points de branchement $x_0$ des fonctions racines carrées $(x-x_0)^{1/2}$ correspondantes. Demi-droite bleue: chemin d'intégration définitif sur $x$, conduisant au résultat cherché (\ref{eq140}).}
\label{fig11}
\end{SCfigure}

\paragraph{La prolonger analytiquement} L'intégrande de (\ref{eq1001}) est une fonction analytique de $x$ dans le plan complexe, si l'on excepte les trois lignes de singularités issues des lignes de coupure des trois racines carrées mises en jeu: $\ii 0^++\mathbb{R}^-$ au numérateur, $\zetac/u+\mathbb{R}^-$ et $-\qb^2/4+\mathbb{R}^-$ au dénominateur, en noir sur la figure \ref{fig11}.\footnote{L'intégrande admet également des pôles $x_n=2\ii n\pi(\qb+2\ii n\pi)$, $n\in\mathbb{Z}^*$, provenant des nombres d'occupation de Bose ($x=0$ n'est pas un pôle à cause du facteur $x$ au numérateur). Les déformations continues à venir du chemin d'intégration dans (\ref{eq1001}), en rouge puis en bleu sur la figure \ref{fig11}, ne passent sur aucun de ces pôles aux conditions suivantes, supposées vérifiées dans la suite (elles assurent que le pôle le plus dangereux $x_{n=-1}$ reste à gauche de la demi-droite bleue): (i) $-\qb^2/4>\re x_{-1}=-(2\pi)^2$ c'est-à-dire $\qb<4\pi$, (ii) $\tan\alpha>2\pi\qb/[(2\pi)^2-\qb^2/4]$.} Ceci permet de déformer à volonté le chemin d'intégration, pourvu qu'il ne croise aucune singularité de l'intégrande, voir la ligne rouge sur la figure \ref{fig11}. On pourrait croire que ceci suffit à rendre l'intégrale analytique en $\zetac/u$ dans la région comprise entre la ligne rouge et l'axe réel ; il n'en est rien car, lorsque $\zetac/u$ entre dans cette région, la ligne de coupure (horizontale) qui lui est attachée croise nécessairement la ligne rouge. Il faut donc faire mieux, en déformant aussi les lignes de coupure de l'intégrande ; le plus simple est d'introduire les nouvelles déterminations $z_\theta^{1/2}\equiv\exp(\ii\theta/2)[\exp(-\ii\theta)z]^{1/2}$ de la racine carrée, paramétrées par l'angle $\theta\in]-\pi,\pi[$ dont on a fait tourner la ligne de coupure originelle (la demi-droite $\mathbb{R}^-$ étant envoyée sur $\exp(\ii\theta)\mathbb{R}^-$). Tant que les lignes de coupure, dans leur mouvement tournant, ne balaient pas le chemin d'intégration en rouge, l'intégrande ne change pas sur ce chemin et la valeur de l'intégrale est préservée. Nous décidons de faire tourner la ligne de coupure de $(x+\qb^2/4)^{1/2}$ d'un angle arbitraire $\alpha\in ]0,\pi/2[$, et les deux autres d'un angle $\alpha-\pi\in]-\pi,-\pi/2[$ pour qu'elles soient parallèles mais de direction opposée à la première, voir les lignes en tireté sur la figure \ref{fig11}. Enfin, nous prenons comme chemin d'intégration définitif la demi-droite en bleu sur la figure \ref{fig11}, qui suit par en-dessous la nouvelle ligne de coupure de $(x+\qb^2/4)^{1/2}$,\footnote{On a pu refermer par un arc de cercle de rayon $R\to +\infty$, de contribution $\to 0$ puisque l'intégrande y est exponentiellement petit en $\sqrt{R}$.} et que nous paramétrons comme suit, $x=\exp(\ii\alpha)(-v-\ii 0^+)-\qb^2/4$, $v\in\mathbb{R}^+$. Alors
\be
\label{eq1002}
(x+\qb^2/4)^{1/2}_\alpha=-\ii\eee^{\ii\alpha/2}\sqrt{v} \ \  ; \ \  (x-\zetac/u)^{1/2}_{\alpha-\pi}=-\ii\eee^{\ii\alpha/2}[v+\eee^{-\ii\alpha}(\qb^2/4+\zetac/u)]^{1/2} \ \  ;\ \  (x-\ii 0^+)^{1/2}_{\alpha-\pi} = -\ii\eee^{\ii\alpha/2} [v+\eee^{-\ii\alpha}\qb^2/4]^{1/2}
\ee
où tous les $\ii 0^+$ ont pu être omis sans mal dans les seconds membres. On peut désormais faire descendre le nombre complexe $\zetac/u$ continûment dans le demi-plan inférieur en restant constamment à droite ou à gauche de la demi-droite bleue sans que cette dernière ne soit croisée par la ligne de coupure attachée à $\zetac/u$: on a bien prolongé analytiquement $\check{\Sigma}_{\qb}^{(2,2)}(\zetac)$ à $\im\zetac<0$. Il reste à poser $v=\kb^2$, $\kb\in\mathbb{R}^+$, pour arriver au résultat annoncé (\ref{eq140}) avec $\phi=\alpha-\pi\in]-\pi,-\pi/2[$.

\paragraph{Application} Maintenant que nous savons, au vu de l'équation (\ref{eq1001}), que $\check{\Sigma}^{(2,2)}_{\qb}(\zetac)$ n'a pas de point de branchement en $\zetac=0$, nous pouvons calculer sa valeur et sa dérivée en $\zetac=\ii 0^+$. Pour sa valeur, le plus simple est faire tendre $\zetac$ vers zéro par parties imaginaires $>0$ sous le signe intégral dans la forme (\ref{eq131}); on intègre sur $\kb$ après développement des nombres d'occupation en série entière de la variable $\exp(-\kb)$, ce qui fait apparaître des fonctions de Bose $g_\alpha(z)$ prises en $z=1$ ou $z=\exp(-\qb)$, les secondes ayant le bon goût de se compenser dans le résultat final. Pour la dérivée, le plus simple est d'utiliser le prolongement analytique (\ref{eq140}), dont l'intégrande ne présente aucune singularité sur le chemin d'intégration pour $\zetac=0$; on peut alors dériver par rapport à $\zetac$ puis prendre la limite $\zetac\to 0$ sous le signe intégral.\footnote{Si l'on s'en tient à la forme (\ref{eq131}), on ne peut pas dériver sous le signe intégral sans faire apparaître une intégrale non absolument convergente. On peut procéder de manière hybride après le choix d'une coupure infrarouge infinitésimale $\kappa$ sur $\kb$: dans la forme (\ref{eq131}), limitée à $\kb>\kappa$ dans chaque contribution, on dérive sous le signe intégral puis on fait tendre $\zetac$ vers $0$ ; dans la forme (\ref{eq1001}), limitée à $-\kappa\qb<x<\kappa\qb$, on garde l'expression exacte du facteur $(x-\ii 0^+)^{1/2}/(x-\zetac/u)^{1/2}$ mais on développe le reste de l'intégrande en puissances de $x$ (l'ordre zéro suffit en fait), et on calcule analytiquement l'intégrale résultante pour $\zetac=s+\ii 0^+$, $s$ réel (au besoin en termes de fonctions hypergéométriques) puis sa dérivée par rapport à $s$ en $s=0$. Le résultat final est le même.} Nous obtenons finalement
\be
\label{eq141}
\check{\Sigma}^{(2,2)}_{\qb}(\check{\zeta}=\ii 0^+) = \frac{1}{\ii}\left(\frac{\qb^2}{12}+2\zeta(2)\right) \quad; \quad
\frac{\dd}{\dd\check{\zeta}} \check{\Sigma}^{(2,2)}_{\qb}(\check{\zeta}=\ii 0^+) = \frac{1}{u\qb} \left(\frac{\pi}{2}+\frac{\ii\qb}{4}\right)
\ee
La valeur de la fonction n'est qu'une réécriture du résultat (\ref{eq033}) sous l'effet des changements de variable dans (\ref{eq130},\ref{eq131}). Sa dérivée intervient dans une condition de validité bien connue de la règle d'or \cite{CCTbordeaux}, $|\im(\dd/\dd\zeta)\Sigma_{\qq}(\zeta=\ii 0^+)|\ll 1$, qui s'écrit ici $1/(4u)\ll 1$ et impose, sans surprise, $u\gg 1$. Une condition plus précise exige que le résidu $Z$ de la fonction de Green, qui apparaît en facteur de l'exponentielle $\exp(-\zetac_{\qb}\check{t})$ dans l'expression exacte (\ref{eq2001}) du signal $s^{(2,2)}(t)$, soit très proche de la valeur un prise dans la règle d'or, ce qui ajoute la contrainte $|\re(\dd/\dd\zeta)\Sigma_{\qq}(\zeta=\ii 0^+)|\ll 1$, ici $\qb\gg \pi/(2 u)$. Comme le montre la figure \ref{fig1}d, sur laquelle $u\simeq 18,\!14$, ces conditions ne sont pas suffisantes. La section \ref{sec4} expliquera pourquoi.

\section{Premier écart du signal à la règle d'or pour la théorie à l'ordre deux en $H_3$}
\label{app2}

Nous établissons ici le résultat (\ref{eq151}) de la section \ref{sec2.2}, valable dans la limite d'interaction faible $u\to+\infty$, en partant de l'écriture (\ref{eq132}) du signal. À cette fin, guidés par la figure \ref{fig5}b, nous appliquons d'abord le théorème des résidus au contour formé de la réunion du chemin $C_+$, des portions de cercle à l'infini et du lacet autour de la ligne de coupure $-Q^2/4-\eee^{\ii(\phi+\pi)}\mathbb{R}^+$: l'intégrale de $\exp(-\ii\check{\zeta}\check{t})\check{\mathcal{G}}_{\qb}^{(2,2)}(\zetac)$ sur ce contour vaut $2\ii\pi$ fois le résidu en $\zetac=\zetac_{\qb}$, pôle de la fonction de Green prolongée analytiquement au demi-plan inférieur au moyen de (\ref{eq140}). Le signal (\ref{eq132}) s'écrit donc de manière exacte comme somme des contributions du pôle $\zetac_{\qb}$ et du lacet:
\be
\label{eq2001}
s^{(2,2)}(t) = \frac{\exp(-\zetac_{\qb}\check{t})}{1-\frac{\dd}{\dd\zetac}\check{\Sigma}^{(2,2)}_{\qb\downarrow}(\zetac_{\qb})}
+\eee^{\ii Q^2\check{t}/4} \int_0^{\eee^{\ii(\phi+\pi)}(+\infty)} \frac{\dd z}{2\ii\pi} \frac{(\check{\Sigma}_- - \check{\Sigma}_+)\exp(\ii z \check{t})}{(-Q^2/4-z-\check{\Sigma}_-)(-Q^2/4-z-\check{\Sigma}_+)}
\ee
où $Q=u^{1/2}\qb$ comme dans (\ref{eq150}), la variable complexe $z$ va en ligne droite de $0$ à $\exp[\ii(\phi+\pi)](+\infty)$ et l'on note $\check{\Sigma}_{\pm}\equiv\check{\Sigma}^{(2,2)}_{\qb\downarrow}(-Q^2/4-z+\eee^{\ii(\phi+\pi)}\ii 0^\pm)$ les valeurs du prolongement analytique (\ref{eq140}) de la fonction énergie propre immédiatement à gauche (signe $+$) ou à droite (signe $-$) de la ligne de coupure. Nous faisons ensuite bon usage de la propriété fondamentale\footnote{Le terme dominant s'obtient en écrivant la différence au premier membre de (\ref{eq2003}) comme une seule intégrale sur $\kb$ au moyen de (\ref{eq140}), en la multipliant par $u^{1/2}$, puis en faisant le changement de variable $\kb=K/u^{1/2}$ avant de passer à la limite $u\to +\infty$ sous le signe intégral.}
\be
\label{eq2003}
\check{\Sigma}^{(2,2)}_{\qb\downarrow}(\zetac)-\check{\Sigma}^{(2,2)}_{\qb\downarrow}(0)\stackrel{Q\,\mbox{\scriptsize fix\'e}}{\underset{u\to+\infty}{=}} \frac{\eee^{\ii\phi/2}}{\ii u^{1/2}} \int_0^{+\infty} \dd K \left\{\frac{(K^2-\eee^{-\ii\phi}Q^2/4)^{1/2}}{[K^2-\eee^{-\ii\phi}(\zetac+Q^2/4)]^{1/2}}-1\right\}+O(1/u)
\ee
Dans la contribution du lacet à (\ref{eq2001}), à l'ordre dominant en $u^{-1/2}$, nous pouvons ainsi remplacer au dénominateur $\check{\Sigma}_\pm$ par $\check{\Sigma}^{(2,2)}_{\qb\downarrow}(0)\to -2\ii\zeta(2)=-\ii\pi^2/3$ d'après (\ref{eq141}), et au numérateur la différence $\check{\Sigma}_--\check{\Sigma}_+$ par son équivalent, qui s'écrit
\be
\label{eq2004}
\check{\Sigma}_--\check{\Sigma}_+ \stackrel{Q\,\mbox{\scriptsize fix\'e}}{\underset{u\to+\infty}{\sim}} \frac{\eee^{\ii\phi/2}}{\ii u^{1/2}} \int_0^{+\infty} \dd K
\left(K^2-\eee^{-\ii\phi}Q^2/4\right)^{1/2}\left[\frac{1}{(K^2-x+\ii 0^-)^{1/2}}-\frac{1}{(K^2-x+\ii 0^+)^{1/2}}\right]
\ee
sachant que $z=\eee^{\ii(\phi+\pi)}x$, $x>0$. L'intégrande dans (\ref{eq2004}) est non nul seulement pour $K<x^{1/2}$, ce qui incite à poser $K=x^{1/2}\sin\alpha$, $\alpha\in [0,\pi/2]$, d'autant plus que ceci fait disparaître les racines carrées au dénominateur: nous retrouvons $S_{\rm lacet}(\check{t})$ dans (\ref{eq152}). Dans l'équation implicite définissant le pôle, réécrite en mettant à part son approximation d'ordre zéro, $\zetac_{\qb}-\check{\Sigma}^{(2,2)}_{\qb\downarrow}(0)=\check{\Sigma}^{(2,2)}_{\qb\downarrow}(\zetac_{\qb})-\check{\Sigma}^{(2,2)}_{\qb\downarrow}(0)$, il suffit au second membre d'utiliser (\ref{eq2003}) avec $\zetac=-\ii\pi^2/3$; le choix commode $\phi=-\pi^{+}$ (dont le résultat ne dépend pas) redonne $\delta\zetac_{\qb}$ dans (\ref{eq154}). Enfin, pour obtenir le premier écart à un du résidu, il suffit de dériver (\ref{eq2003}) par rapport à $\zetac$ (sous le signe intégral au second membre), de remplacer $\zetac$ par $-\ii\pi^2/3$ puis de prendre $\phi=-\pi^+$ pour retrouver $\delta\check{Z}$ dans (\ref{eq154}).

\section{Les sommes doubles de Matsubara dans la théorie à l'ordre quatre en $H_3$}
\label{app3}

Nous calculons ici explicitement les sommes de Matsubara contribuant à (\ref{eq212},\ref{eq232}) dans la section \ref{sec3.1} et montrons comment obtenir leurs formes simplifiées (\ref{eq213},\ref{eq233}) dans la limite des petits angles. Le calcul repose sur la généralisation de la relation (\ref{eq122}) à un nombre quelconque $J$ de facteurs dans le sommande, qui se démontre comme dans la note \ref{notedemo}:
\be
\label{eq3000}
\sum_{n\in\mathbb{Z}} \frac{1}{\prod_{j=1}^{J} (2\ii\pi n-\vepsb_j)} = \sum_{j=1}^{J} \frac{-\bar{n}_{\veps_j}}{\prod_{\stackrel{\mbox{\scriptsize$k\!=\!1$}}{k\neq j}}^{J} (\vepsb_j-\vepsb_k)} \quad\forall J\geq 2
\ee

\paragraph{Diagrammes à boucle interne} Écrivons la somme apparaissant dans (\ref{eq212}) après sortie d'un facteur $(k_B T)^5$ au dénominateur (on a posé $\veps_{\kk_j}=k_B T\vepsb_j$) et élimination des énergies de Matsubara au profit des variables indépendantes $\omega=2\ii\pi k_B T n$ et $\omega'=2\ii\pi k_B T n'$ sur le modèle des équations (\ref{eq210},\ref{eq211}), pour une énergie complexe $z$ valant temporairement $\Omega=k_B T\bar{\Omega}\in 2\ii\pi k_B T\mathbb{Z}$:
\be
\label{eq3001}
S_{B_i}(z)=(k_BT)^{-5}\sum_{n,n'\in\mathbb{Z}} \frac{1}{s_1(\bar{\Omega}-2\ii\pi s_2 n)-\vepsb_1}\frac{1}{2\ii\pi n-\vepsb_2}\frac{1}{2\ii\pi n'-\vepsb_3}\frac{1}{2\ii\pi s_4(s_2 n - s_3 n')-\vepsb_4} \frac{1}{2\ii\pi s_2 s_5 n-\vepsb_5}
\ee
Calculons au moyen de (\ref{eq122}) la somme sur $n'$ du produit des deux facteurs qui en dépendent, puis faisons disparaître la variable $\bar{\Omega}$ des nombres d'occupation comme dans la première égalité de (\ref{eq123}). Après le changement de variable astucieux $n\to s_2 n$, qui n'affecte pas le domaine de sommation puisque $s_2=\pm 1$, il reste
\be
\label{eq3002}
S_{B_i}(z)= (k_BT)^{-5}\sum_{n\in\mathbb{Z}} \frac{s_1s_2s_4s_5(\bar{n}_{\veps_3}-\bar{n}_{-s_3s_4\veps_4})}{[2\ii\pi n-(\bar{\Omega}-s_1\vepsb_1)](2\ii\pi n-s_2\vepsb_2)(2\ii\pi n -s_5 \vepsb_5)[2\ii\pi n-(s_3\vepsb_3+s_4\vepsb_4)]}
\ee
Nous avons pris soin de sortir tous les facteurs nécessaires pour mettre le résultat sous une forme simple, comme au premier membre de la relation (\ref{eq3000}), ce qui permet d'utiliser facilement cette dernière:
\begin{multline}
\label{eq3003}
S_{B_i}(z)=-(k_B T)^{-2}s_1s_2s_4s_5(\bar{n}_{\veps_3}-\bar{n}_{-s_3s_4\veps_4})\Bigg\{\frac{1}{z-z_2}\left[\frac{\bar{n}_{-s_1\veps_1}}{(z-z_1)(z-z_3)}-\frac{\bar{n}_{s_3\veps_3+s_4\veps_4}}{(z_2-z_1)(z_2-z_3)}\right] \\
+\frac{1}{z_1-z_3}\left[\frac{\bar{n}_{s_2\veps_2}}{(z_1-z)(z_1-z_2)}-\frac{\bar{n}_{s_5\veps_5}}{(z_3-z)(z_3-z_2)}\right]\Bigg\}
\end{multline}
À ce stade, nous avons prolongé analytiquement en la variable $z$ sur $\mathbb{C}\setminus\mathbb{R}$ et introduit les énergies des trois états intermédiaires (repérés par les droites verticales rouges sur la partie gauche de la figure \ref{fig7}) comptées algébriquement, $z_1=s_1\veps_1+s_2\veps_2$, $z_2=s_1\veps_1+s_3\veps_3+s_4\veps_4$ et $z_3=s_1\veps_1+s_5\veps_5$. Dans le cas $s_2=s_5$, on a aussi $\veps_2=\veps_5$ et $z_1=z_3$ puisque $\kk_2$ est alors égal à $\kk_5$ ; ceci fait apparaître une division par zéro dans la seconde contribution de (\ref{eq3003}) entre accolades, que nous prolongeons donc par continuité avec la règle de l'Hôpital. Après d'habiles réarrangements, il vient:\footnote{Nous utilisons en particulier l'identité $(z_1-z_2)^{-2}(z-z_2)^{-1}+(z_1-z_2)^{-2}(z_1-z)^{-1}+(z_1-z)^{-2}(z_1-z_2)^{-1}=(z-z_1)^{-2}(z-z_2)^{-1}$.}
\be
\label{eq3004}
%S_{B_i}^{s_2=s_5}(z)=-(k_B T)^{-2}s_1s_4(\bar{n}_{\veps_3}-\bar{n}_{-s_3s_4\veps_4})\Bigg\{\frac{1}{z-z_2}\left[\frac{\bar{n}_{-s_1\veps_1}}{(z-z_1)^2}-\frac{\bar{n}_{s_3\veps_3+s_4\veps_4}}{(z_2-z_1)^2}\right]-\frac{\bar{n}_{s_2\veps_2}}{(z_1-z_2)(z_1-z)} \left[\frac{1}{z_1-z_2}+\frac{1}{z_1-z}+\frac{(1+\bar{n}_{s_2\veps_2})}{k_B T}\right]\Bigg\}
S_{B_i}^{s_2=s_5}(z)=-(k_B T)^{-2}s_1s_4(\bar{n}_{\veps_3}-\bar{n}_{-s_3s_4\veps_4})\left[\frac{\bar{n}_{-s_1\veps_1}-\bar{n}_{s_2\veps_2}}{(z-z_1)^2(z-z_2)}+\frac{\bar{n}_{s_2\veps_2}-\bar{n}_{s_3\veps_3+s_4\veps_4}}{(z_2-z_1)^2(z-z_2)}+\frac{\bar{n}_{s_2\veps_2}(1+\bar{n}_{s_2\veps_2})}{k_B T(z_2-z_1)(z_1-z)}\right]
\ee
Remarquons que le facteur global $(k_B T)^{-2}$ de (\ref{eq3003},\ref{eq3004}) se simplifie avec le facteur $(-k_B T)^2$ dans (\ref{eq212}). Pour terminer, prenons la limite des petits angles $\epsilon\to 0$ dans (\ref{eq3004}), dans laquelle $z$, $z_1$ et $z_2$ s'écrivent $\veps_q+O(k_B T\epsilon^2)$. Entre les crochets, la deuxième contribution est sous-dominante parce que $s_3\veps_3+s_4\veps_4=s_2\veps_2+z_2-z_1=s_2\veps_2+O(k_B T\epsilon)^2$, si bien que son numérateur tend vers zéro; la troisième l'est aussi car son dénominateur est d'ordre $(k_B T)^3\epsilon^4$ au lieu de $(k_B T)^3\epsilon^6$ comme les autres. Nous retrouvons (\ref{eq213}).

\paragraph{Diagrammes à pont} Nous procédons comme précédemment, avec les mêmes notations. La somme double dans (\ref{eq232}) s'écrit
\be
\label{eq3010}
S_{P_i}(z)=(k_B T)^{-5} \sum_{n,n'\in\mathbb{Z}} \frac{1}{2\ii\pi n-\vepsb_1} \frac{1}{s_2(\bar{\Omega}-2\ii\pi s_1 n)-\vepsb_2}\frac{1}{2\ii\pi n'-\vepsb_3}\frac{1}{s_4(\bar{\Omega}-2\ii\pi s_3 n')-\vepsb_4} \frac{1}{2\ii\pi s_5(s_1 n-s_3 n')-\vepsb_5}
\ee
Au moyen de la relation (\ref{eq3000}) avec $J=3$, nous calculons la somme sur $n'$, décomposons le résultat en éléments simples pour la variable $2\ii\pi n$ puis effectuons le changement de variable $n\to s_1 n$, si bien que
\be
\label{eq3011}
S_{P_i}(z)=\frac{(k_B T)^{-5}s_1s_2s_4s_5}{\bar{\Omega}-s_3\vepsb_3-s_4\vepsb_4}\sum_{n\in\mathbb{Z}} \frac{1}{(2\ii\pi n-s_1\vepsb_1)[2\ii\pi n-(\bar{\Omega}-s_2\vepsb_2)]} \left[\frac{\bar{n}_{\veps_3}-\bar{n}_{-s_3s_5\veps_5}}{2\ii\pi n-(s_3\veps_3+s_5\veps_5)}+\frac{\bar{n}_{-s_3s_5\veps_5}-\bar{n}_{-s_3s_4\veps_4}}{2\ii\pi n-(\bar{\Omega}+s_5\veps_5-s_4\veps_4)}\right]
\ee
Nous utilisons à nouveau (\ref{eq3000}) puis prolongeons analytiquement sur $z$ à $\mathbb{C}\setminus\mathbb{R}$; il y a deux termes entre les grands crochets de (\ref{eq3011}) donc deux contributions, dites de type I et de type II dans cet ordre:
\bea
\label{eq3012}
S_{P_i^{\rm I}}(z)&=&\frac{(k_B T)^{-2}s_1s_2s_4s_5(\bar{n}_{\veps_3}-\bar{n}_{-s_3s_5\veps_5})}{z_2-z}\left[
\frac{\bar{n}_{s_1\veps_1}}{(z_1-z)(z_1-z_4)}+\frac{\bar{n}_{-s_2\veps_2}}{(z-z_1)(z-z_4)}+\frac{\bar{n}_{s_3\veps_3+s_5\veps_5}}{(z_4-z_1)(z_4-z)}\right] \\
\label{eq3013}
S_{P_i^{\rm II}}(z)&=&\frac{(k_B T)^{-2}s_1s_2s_4s_5(\bar{n}_{-s_3s_5\veps_5}-\bar{n}_{-s_3s_4\veps_4})}{z_2-z}\left[
\frac{\bar{n}_{s_1\veps_1}}{(z_1-z)(z_3-z)}+\frac{\bar{n}_{-s_2\veps_2}}{(z-z_1)(z_3-z_1)}+\frac{\bar{n}_{s_5\veps_5-s_4\veps_4}}{(z-z_3)(z_1-z_3)}\right]
\eea
avec les énergies algébriques des états intermédiaires repérés par les droites rouges sur la partie droite de la figure \ref{fig7}, $z_1=s_1\veps_1+s_2\veps_2$ et $z_2=s_3\veps_3+s_4\veps_4$ communes aux deux types, $z_4=s_2\veps_2+s_3\veps_3+s_5\veps_5$ de type I et enfin $z_3=s_1\veps_1+s_4\veps_4-s_5\veps_5$ de type II. Ces variables ne sont pas indépendantes puisque $z_1+z_2=z_3+z_4$. Il reste à prendre la limite des petits angles $\epsilon\to 0$, dans laquelle $z_j=\veps_q+O(k_B T \epsilon^2)$, $1\leq j\leq 4$. Alors $s_3\veps_3+s_5\veps_5=s_1\veps_1+O(k_BT\epsilon^2)$ et l'on peut, à l'ordre dominant, remplacer $\bar{n}_{s_3\veps_3+s_5\veps_5}$ par $\bar{n}_{s_1\veps_1}$ dans (\ref{eq3012}) ; en regroupant par nombre d'occupation et en simplifiant, on retrouve la première contribution au second membre de (\ref{eq233}). De même, on remplace $\bar{n}_{s_5\veps_5-s_4\veps_4}$ par $\bar{n}_{-s_2\veps_2}$ à l'ordre dominant dans (\ref{eq3013}), ce qui reproduit la seconde contribution à (\ref{eq233}).

\section{Une divergence linéaire de $\tilde{\Sigma}_{\qb}^{(4,2)}(\zetat)$ en dehors de l'origine}
\label{app4}

Comme nous le montrons dans cette annexe, la composante $\tilde{\Sigma}_{\qb}^{(4,2)}(\zetat)$ de la fonction énergie propre d'ordre quatre en $H_3$ et d'ordre deux en $\epsilon$, obtenue en sommant toutes les contributions (\ref{eq220}), (\ref{eq240}), (\ref{eq242}) des diagrammes d'ordre quatre de la section \ref{sec3.1}, diverge linéairement lorsque $\zetat\to\zetat_0\equiv -3\gamma\qb^3/32$ (ici par parties imaginaires positives): 
\be
\label{eq4000}
\tilde{\Sigma}_{\qb}^{(4,2)}(\zetat) \stackrel{\im\zetat>0}{\underset{\zetat\to \zetat_0}{\sim}} \frac{\qb^3(\bar{n}^{\rm lin}_{q/2}+{1}/{2})}{24\ii\pi^2\gamma(\zetat-\zetat_0)} \left(\frac{9(1+\Lambda)^2}{8\rho\xi^2}\right)^2\left[\int_0^{\qb/2} \dd\kb' \kb'(\kb'-\qb/2)(\bar{n}^{\rm lin}_{k'}-\bar{n}^{\rm lin}_{k'-{q}/{2}})+2\int_{\qb/2}^{\eta} \dd\kb'\kb'(\kb'-\qb/2)(\bar{n}^{\rm lin}_{k'}-\bar{n}^{\rm lin}_{k'-{q}/{2}})\right]
\ee
Ceci résulte de deux phénomènes. (i) À nombre d'onde réduit $\kb$ fixé, l'intégrale angulaire (\ref{eq221}) dans les diagrammes à boucle interne diverge en $\zetat=A_1^B$ avec un exposant $3/2$ car, pour cette valeur de $\zetat$, le dénominateur de l'intégrande admet $\thetat=0$ comme racine quadruple. Pour le voir quantitativement, on peut utiliser l'identité $\int_{\mathbb{R}}\dd\thetat/(z-\thetat^2)^2=-\ii\pi/(2 z^{3/2}), \forall z\in\mathbb{C}\setminus\mathbb{R}$, avec $z=(\zetat-A_1^B)/B_1^B$. (ii) Cette divergence $(\zetat-A_1^B)^{-3/2}$ se produit à une position $A_1^B$ dépendant de $\kb$, comme le montre l'équation (\ref{eq222}), donc elle est en général effacée par l'intégration sur $\kb$ à $\zetat$ fixé dans (\ref{eq220}), sauf si $\zetat$ coïncide avec un extrémum parabolique de $A_1^B(\kb)$ sur le domaine d'intégration $\mathcal{D}_1^B$. Pour le voir, choisissons un nombre d'onde réduit $\kb_0$ arbitraire dans $\mathcal{D}_1^B$, posons $\zetat=w+A_1^B(\kb_0)$ et passons à la limite $w\to 0$ par parties imaginaires positives dans (\ref{eq220}): le résultat ne peut diverger que si la valeur de l'intégrale 
\be
\label{eq4005}
\sigma_{\kb_0}(w)=\int_{\kb_0-\veps}^{\kb_0+\veps} \frac{\dd\kb}{[w+A_1^B(\kb_0)-A_1^B(\kb)]^{3/2}}
\ee
diverge dans cette limite, où $\veps>0$ est fixé. Comme on peut prendre $\veps$ aussi petit que l'on veut sans changer la conclusion, on peut remplacer $A_1^B(\kb)-A_1^B(\kb_0)$ dans (\ref{eq4005}) par son développement limité en $\kb_0$ à l'ordre dominant non nul. Si $(\dd A_1^B/\dd \kb)(\kb=\kb_0)\neq 0$, un développement à l'ordre un suffit, et il n'y a pas de divergence car $\int_{\mathbb{R}} \dd \kappa/(w-\kappa)^{3/2}=0$.\footnote{Si $\kb_0$ se trouve au bord de $\mathcal{D}_1^B$, le voisinage de $\kb_0$ se réduit à $[\kb_0,\kb_0+\veps]$ ou $[\kb_0-\veps,\kb_0]$ et il y a divergence en racine carrée car $\int_0^{+\infty} \dd \kappa/(w-\kappa)^{3/2}=-2/w^{1/2}$. Cette divergence est sous-dominante et ignorée ici.} Sinon, il faut développer à l'ordre deux, et il y a divergence linéaire $[\zetat-A_1^B(\kb_0)]^{-1}$ comme l'annonçait (\ref{eq4000}) car
\be
\label{eq4001}
\int_{\mathbb{R}} \frac{\dd \kappa}{(w-\kappa^2)^{3/2}} = \frac{2}{\ii w} \quad\mbox{et}\quad \int_{\mathbb{R}} \frac{\dd \kappa}{(w+\kappa^2)^{3/2}} = \frac{2}{w}\quad\forall w\in \mathbb{C}, \im w>0
\ee

Donnons plus de détails sur l'exemple du diagramme $B_1$ de la figure \ref{fig8}, dans lequel les signes $s_i$ des orientations des lignes internes valent tous $1$, le domaine d'intégration sur $\kb$ vaut $\mathcal{D}_1^B=[0,\qb]$ et celui sur $\kb'$ à $\kb$ fixé vaut $\mathcal{D}_2^B(\kb)=[0,\kb]$. On remarque que $A_1^B(\kb)$ atteint son minimum en $\kb_0=\qb/2$, à l'intérieur de $\mathcal{D}_1^B$ donc avec une dérivée nulle, et que ce minimum vaut justement $\zetat_0=-3\gamma\qb^3/32$ comme dans (\ref{eq4000}). Pour obtenir un équivalent de $\tilde{\Sigma}_{B_1}^{(4,2)}(\qb,\zetat)$ lorsque $\zetat\to\zetat_0$, nous pouvons traiter $\thetat^2$, $(\kb-\qb/2)^2$ et $w=\zetat-\zetat_0$ comme des infiniment petits du même ordre. Ainsi, dans l'intégrale angulaire (\ref{eq221}), nous pouvons remplacer $\zetat$ par $\zetat_0+\ii 0^+$ et $\thetat^2$ par $0$ dans le second facteur au dénominateur, et remplacer partout $\kb$ par $\kb_0=\qb/2$ sauf dans le coefficient $A_1^B(\kb)$ que l'on écrit $A_1^B(\kb)=\tilde{\zeta}_0+(3\gamma\qb/8)(\kb-\qb/2)^2$. Après calcul de l'intégrale sur $\phi$ puis sur $\thetat$ par le théorème des résidus, il reste
\be
\label{eq4002}
I(\kb,\kb') \sim \frac{-8\pi^2}{(6\gamma\qb^3)^{1/2} \kb'}\frac{1}{[w-(3\gamma/8)\qb(\kb-\qb/2)^2]^{3/2}}
\ee
ce que l'on peut reporter dans l'intégrande de (\ref{eq220}), en remplaçant de plus $\kb$ par $\kb_0=\qb/2$ dans tous les autres facteurs et dans la borne supérieure d'intégration sur $\kb'$. Dans l'intégrale sur $\kb$ à $\kb'$ fixé, il ne reste plus que
\be
\label{eq4003}
\int_0^{\qb} \frac{\dd \kb}{[w-(3\gamma/8)\qb(\kb-\qb/2)^2]^{3/2}}\sim \left(\frac{8}{3\gamma\qb}\right)^{1/2}\int_{-\infty}^{+\infty} \frac{\dd\kappa}{(w-\kappa^2)^{3/2}}=\left(\frac{8}{3\gamma\qb}\right)^{1/2} \frac{2}{\ii w}
\ee
où nous avons effectué le changement de variable $\kb=\qb/2+[8/(3\gamma\qb)]^{1/2}\kappa$, étendu les bornes d'intégration sur $\kappa$ à $\pm\infty$ sans changer le comportement dominant et utilisé l'équation (\ref{eq4001}). Nous obtenons finalement la partie divergente du diagramme $B_1$:
\be
\label{eq4004}
\tilde{\Sigma}_{B_1}^{(4,2)}(\qb,\zetat) \stackrel{\im\zetat>0}{\underset{\zetat\to \zetat_0}{\sim}} \frac{\qb^3(\bar{n}^{\rm lin}_{q/2}+{1}/{2})}{24\ii\pi^2\gamma(\zetat-\zetat_0)} \left(\frac{9(1+\Lambda)^2}{8\rho\xi^2}\right)^2\int_0^{\qb/2} \dd\kb' \kb'(\kb'-\qb/2)(\bar{n}^{\rm lin}_{k'}-\bar{n}^{\rm lin}_{k'-{q}/{2}})
\ee
Le diagramme $B_2$ se traite de la même façon et donne une contribution divergente similaire (avec un préfacteur double et l'intégrale sur $\kb'$ allant de $\qb/2$ à $\eta$). Les diagrammes $B_3$ et $B_4$ sont sous-dominants en $\epsilon$ car $s_5\neq s_2$ et doivent être omis ici. Les diagrammes $B_5$ et $B_6$ conduisent à un coefficient $A_1^B(\kb)=(3\gamma/8)\qb\kb(\kb+\qb)$ strictement croissant sur l'intervalle d'intégration $\kb\in[0,\eta-\qb]$ donc ne peuvent contribuer à la divergence linéaire (\ref{eq4000}). Les diagrammes $B_7$ et $B_8$ ont le même problème, avec $A_1^B(\kb)=(3\gamma/8)\qb\kb(\kb-\qb)$ et $\mathcal{D}_1^B=[\qb,\eta]>\qb/2$. Il reste à regrouper les contributions de $B_1$ et $B_2$ pour obtenir (\ref{eq4000}), dont l'expression entre crochets dans le cas du champ quantique tend vers $-\qb(16\pi^2+\qb^2)/48$ lorsque $\eta\to+\infty$. 

Pour terminer, expliquons pourquoi les diagrammes à pont ne contribuent pas à la divergence linéaire (\ref{eq4000}) sur l'exemple du diagramme $P_1$ (tous les $s_i=1$, $\mathcal{D}_1^P=[0,\qb]$ et $\mathcal{D}_2^P(\kb)=[0,\kb]$, type II). Comme l'avait remarqué la section \ref{sec3.3}, l'intégrale angulaire (\ref{eq243}) diverge linéairement en $\zetat=X_0(\kb,\kb')$ si les lieux d'annulation des trois facteurs au dénominateur de (\ref{eq243}) se croisent en un même point dans le plan $(\tilde{\theta},\tilde{\theta}')$. La condition de triple intersection s'obtient en remplaçant tous les coefficients $A_i^P$ par $A_i^P-X_0$ dans (\ref{eq281}); l'équation du second degré sur $X_0$ qui en résulte admet la solution nulle évidente, à laquelle s'était implicitement limitée la section \ref{sec3.3}, et une deuxième racine réelle, $X_0(\kb,\kb')=-3\gamma \kb(\kb-\qb)\kb'(\kb'-\qb)/(2\qb)$ ; cette dernière conduit à une triple intersection dans le plan $(\tilde{\theta},\tilde{\theta}')$, aux points de coordonnées $\pm[\sqrt{3\gamma}/(2\qb)]((\qb-\kb)(\qb-2\kb'),-(\qb-2\kb)(\qb-\kb'))$, si on ajoute les contraintes $\kb>\qb/2$ et $\kb'<\qb/2$ pour avoir les bons signes dans les coefficients des petits écarts $(\tau,\tau')$ à ces points (sans quoi, comme dans (\ref{eq282}), il n'y a pas de divergence). Alors, en procédant comme dans (\ref{eq282}), on obtient la partie divergente de l'intégrale angulaire (pour $\zetat\to X_0$ par parties imaginaires positives),
\be
\label{eq4006}
K_{\rm div}(\kb,\kb') = \frac{-32\pi^2}{3\gamma\qb\kb\kb'(\kb-\kb')[\zetat-X_0(\kb,\kb')]}
\ee
à intégrer sur $(\kb,\kb')\in[\qb/2,\qb]\times[0,\qb/2]$ après multiplication par la même fonction de $(\kb,\kb')$ que dans (\ref{eq242}). Or $X_0(\kb,\kb')$ admet un minimum parabolique, situé en $(\kb,\kb')=(\qb/2,\qb/2)$ et valant précisément $\zetat_0$ ! La singularité (\ref{eq4006}) survivra donc à l'intégration sur les nombres d'onde si $\zetat\to \zetat_0$ comme dans (\ref{eq4000}). Effectuons alors le changement de variables $(\kb=\kappa+\qb/2,\kb'=-\kappa'+\qb/2)$ et traitons $w=\zetat-\zetat_0$, $\kappa^2$ et $\kappa'^2$ comme des infiniment petits du même ordre. En écrivant la fonction en facteur de $[\zetat-X_0(\kb,\kb')]^{-1}$ à l'ordre dominant en $\kappa$ et $\kappa'$ et en remplaçant le pôle $X_0(\kb,\kb')$ par son approximation quadratique en $\kappa$ et $\kappa'$ près du minimum, nous obtenons l'équivalent
\be
\label{eq4008}
\tilde{\Sigma}^{(4,2)}_{P_1^{\rm II}}(\qb,\zetat)\stackrel{\im w>0}{\underset{w\to 0}{\sim}} (\qb/2)^3 \left(\frac{9(1+\Lambda)^2}{8\rho\xi^2}\right)^2 (2\bar{n}^{\rm lin}_{q/2}+1)\frac{(-32\pi^2)}{3\gamma(2\pi)^4} \int_0^{\qb/2}\dd\kappa \int_0^{\qb/2} \dd\kappa' \frac{1}{\kappa+\kappa'} \frac{1}{w-(3\gamma\qb/8)(\kappa^2+\kappa'^2)}
\ee
On peut étendre les bornes d'intégration $\qb/2$ à $+\infty$ sans perdre l'équivalence. L'intégrale double se calcule alors facilement en coordonnées polaires $(K,\alpha)$ où $K=(\kappa^2+\kappa'^2)^{1/2}$ décrit $\mathbb{R}^+$ et $\alpha$ décrit $[0,\pi/2]$. On trouve finalement une divergence en racine carrée, $\propto (\zetat-\zetat_0)^{-1/2}$, sous-dominante dans (\ref{eq4000}).

\section{Le signal aux temps longs dans la limite $(\rho\xi^2)^{-1}\to 0$ des théories non perturbatives}
\label{app5}

Nous obtenons ici le comportement asymptotique (\ref{eq314}) du signal associé à la fonction énergie propre $\check{\Sigma}_{{\rm corr}\,\infty}^{(2,2)}(\qb,\check{\zeta})$ de l'équation (\ref{eq311}). À cette fin, nous appliquons la technique de l'\ref{app2} en rabattant le chemin d'intégration $C_+$ dans (\ref{eq012}) sur la ligne de coupure de $\check{\Sigma}_{{\rm corr}\,\infty}^{(2,2)}(\qb,\check{\zeta})$ située sur $\ii\mathbb{R}^{-}$ et décrite paramétriquement par $\zeta=-\ii\check{y}$, $\check{y}\geq 0$ (cette ligne de coupure est l'ensemble des valeurs de $\zetac$ annulant le dénominateur de l'intégrande dans (\ref{eq311})). En transposant l'équation (\ref{eq2001}) au présent cas, nous obtenons 
\be
\label{eq5001}
s^{(2,2)}_{{\rm corr}\,\infty}(t)\underset{\check{t}\to+\infty}{\sim}\int_0^{+\infty}\frac{\dd\check{y}}{2\pi}\frac{\delta\sigma_{\qb}(\check{y})\exp(-\check{y}\check{t})}{\left[\check{\Sigma}_{{\rm corr}\,\infty}^{(2,2)}(\qb,0)\right]^2}\quad\mbox{où }\quad
\delta\sigma_{\qb}(\check{y})=\check{\Sigma}_{{\rm corr}\,\infty}^{(2,2)}(\qb,0^+-\ii\check{y})-\check{\Sigma}_{{\rm corr}\,\infty}^{(2,2)}(\qb,0^--\ii\check{y})
\ee
est la discontinuité de la fonction énergie propre à travers sa ligne de coupure à la distance $\check{y}$ de l'origine. L'identité au sens des distributions $1/(\check{y}+\ii 0^+)=\mathrm{v.p.}(1/\check{y})-\ii\pi\delta(\check{y})$ conduit à 
\be
\label{eq5002}
\delta\sigma_{\qb}(\check{y})=-\frac{4\pi}{\qb^2}\int_{\qb}^{\eta}\dd \kb \int_0^{\eta-\kb}\dd\kb' \kb \kb' (\kb-\qb)(\kb'+\qb) (\bar{n}_{k'+q}^{\rm lin}-\bar{n}_{k'+k}^{\rm lin})(\bar{n}_{k-q}^{\rm lin}-\bar{n}_k^{\rm lin})
\delta\left(\check{y}-\kb'(\kb-\qb)\frac{\check{\Gamma}_{\kb-\qb}+\check{\Gamma}_{\kb'}}{2\qb(\kb+\kb')}\right)
\ee
Dans la limite $\check{t}\to+\infty$, la fonction exponentielle dans (\ref{eq5001}) limite l'intégrale à des valeurs arbitrairement faibles de $\check{y}$ et il suffit de connaître le comportement dominant de $\delta\sigma_{\qb}(\check{y})$ lorsque $\check{y}\to 0^+$. Comme les taux réduits $\check{\Gamma}_{\bar{K}}$ ne peuvent pas tendre vers zéro, l'argument du Dirac dans (\ref{eq5002}) ne peut s'annuler dans la limite $\check{y}\to 0$ que si $\kb'$ ou $\kb-\qb$ est infinitésimal. Pour un choix arbitraire d'une borne $\veps$ non infinitésimale mais $\ll 1$, on peut donc limiter le domaine d'intégration sur $\kb$ et $\kb'$ aux deux parties, (i) $\kb\in[\qb,\qb+\veps]$, $\kb'\in[0,\eta-\kb]$ et (ii) $\kb\in [\qb+\veps,\eta-\veps]$, $\kb'\in[0,\veps]$.\footnote{On remarque dans la partie (ii) que, si $\kb\leq\eta-\veps$, on a $\veps\leq\eta-\kb$ donc $\kb'$, qui doit rester $\leq \eta-\kb$, peut effectivement aller jusqu'à $\veps$. On vérifie par ailleurs que le \g{bout manquant} $\kb\in[\eta-\veps,\eta]$, $\kb'\in[0,\veps]$ donne une contribution en $O(\check{y}\veps)$ négligeable à $\delta\sigma_{\qb}(\check{y})$.} Dans la partie (i) du domaine d'intégration, on a $\kb-\qb\leq\veps\ll 1$ donc on peut remplacer les différentes quantités dépendant de $\kb$ dans (\ref{eq5002}) par leur développement limité en $\kb=\qb$ à l'ordre dominant, par exemple $(\kb-\qb)(\bar{n}^{\rm lin}_{k-q}-\bar{n}^{\rm lin}_{k})\simeq 1$, $\bar{n}_{k'+q}^{\rm lin}-\bar{n}_{k'+k}^{\rm lin}\simeq(\kb-\qb)(-\dd/\dd\kb')\bar{n}_{k'+q}^{\rm lin}$, $\check{\Gamma}_{\kb-\qb}\simeq \check{\Gamma}_0$ et la borne d'intégration sur $\kb'$ peut être remplacée par $\eta-\qb$. L'argument du Dirac dans (\ref{eq5002}) devient une fonction affine de $\kb-\qb$ ce qui permet d'intégrer facilement sur $\kb$:
\be
\label{eq5003}
\delta\sigma_{\qb}^{\rm (i)}(\check{y}) \simeq -16\pi\qb\check{y}\int_0^{\eta-\bar{q}} \dd\kb'\frac{(\kb'+\qb)^3}{(\check{\Gamma}_0+\check{\Gamma}_{\kb'})^2}\left(-\frac{\dd}{\dd\kb'}\bar{n}_{k'+q}^{\rm lin}\right) Y\left(\veps-\frac{2(\kb'+\qb)\qb\check{y}}{\kb'(\check{\Gamma}_0+\check{\Gamma}_{\kb'})}\right) \simeq -16\pi\qb\check{y}\int_{\qb^2\check{y}/\veps\check{\Gamma}_0}^{\eta-\qb}\frac{\dd\kb'}{\kb'} \frac{(\kb'+\qb)^3}{(\check{\Gamma}_0+\check{\Gamma}_{\kb'})^2} \left(-\frac{\dd}{\dd\kb'}\bar{n}_{k'+q}^{\rm lin}\right)
\ee
La forme du troisième membre de (\ref{eq5003}) repose sur le raisonnement suivant: si $\check{y}\to 0$ à $\veps$ fixé, l'argument de la fonction de Heaviside $Y$ au second membre est strictement positif, sauf si $\kb'\to 0$ ; au bord du support de cette fonction $Y$, on a donc $\kb'\ll 1$, ce qui autorise à remplacer dans son argument le deuxième terme par son approximation d'ordre dominant en $\kb'$ ; ceci fait apparaître $Y(\veps-\qb^2\check{y}/\kb'\check{\Gamma}_0)$, d'où la borne inférieure annoncée dans l'intégration sur $\kb'$. Dans la partie (ii) du domaine d'intégration, on procède de même: comme $\kb'\leq\veps\ll 1$, on remplace les différentes quantités dans (\ref{eq5002}) par leur ordre dominant en $\kb'$. L'argument du Dirac devenant une fonction affine de $\kb'$, on intègre sur cette variable ; la fonction $Y(\veps-\frac{2\qb\kb\check{y}}{(\kb-\qb)(\check{\Gamma}_{\kb-\qb}+\check{\Gamma}_0)})$ qui en résulte est inopérante lorsque $\check{y}\to 0$ car $\kb-\qb$ est ici $\geq\veps$, d'où
\be
\label{eq5004}
\delta\sigma_{\qb}^{\rm (ii)}(\check{y}) \simeq -16\pi\qb\check{y} \int_{\qb+\veps}^{\eta-\veps} \frac{\dd\kb}{\kb-\qb}\frac{\kb^3}{(\check{\Gamma}_{\kb-\qb}+\check{\Gamma}_0)^2} (\bar{n}_q^{\rm lin}-\bar{n}_k^{\rm lin})(\bar{n}_{k-q}^{\rm lin}-\bar{n}_k^{\rm lin})
\ee
Il reste à passer à la limite $\veps\to 0$ dans les résultats (\ref{eq5003},\ref{eq5004}). Pour cela, on sépare chaque intégrande en sa partie divergente infrarouge $\propto 1/\kb'$ ou $\propto 1/(\kb-\qb)$ et sa partie régulière.\footnote{Pour calculer le coefficient de la partie divergente $1/(\kb-\qb)$, on utilise le fait que $\lim_{\kb\to\qb}(\bar{n}_q^{\rm lin}-\bar{n}_k^{\rm lin})(\bar{n}_{k-q}^{\rm lin}-\bar{n}_k^{\rm lin})=(-\dd/\dd\qb)\bar{n}_q^{\rm lin}.$} Dans l'intégrale des parties régulières, on peut faire tendre sans dommage $\veps$ vers zéro dans les bornes. Dans l'intégrale des parties divergentes, faire de même donne naissance à des contributions dangereuses $\pm\ln\veps$, mais qui ont le bon goût de se compenser exactement dans la somme des parties (i) et (ii). Finalement
\be
\label{eq5005}
\delta\sigma_{\qb}(\check{y})\underset{\check{y}\to 0^+}{=}\frac{4\pi\qb^4}{\check{\Gamma}_0^2} \left(-\frac{\dd}{\dd\qb}\bar{n}^{\rm lin}_{\qb}\right) \check{y}\ln\check{y} + C_{\qb} \check{y} + O(\check{y}^2\ln\check{y})
\ee
Pour alléger, nous ne donnons pas l'expression du coefficient $C_{\qb}$ de la partie linéaire dans (\ref{eq5005}), auquel notre méthode donne pourtant accès. La forme du $O(\ldots)$ dans (\ref{eq5005}) a en revanche été obtenue numériquement. Le report du terme dominant $\propto\check{y}\ln\check{y}$ dans (\ref{eq5001}) conduit au résultat annoncé (\ref{eq314}).
%)}

%%%%%%%%%%%%%%%%%%%%%%%%%%%
\begin{thebibliography}{99}
\bibitem{LK} L. Landau, I. Khalatnikov, \g{Teoriya vyazkosti Geliya-II}, Zh. Eksp. Teor. Fiz. {\bf 19}, 637 (1949).
\bibitem{SonWingate} D.T. Son, M. Wingate, \href{https://doi.org/10.1016/j.aop.2005.11.001}{\g{General coordinate invariance and conformal invariance in nonrelativistic physics: Unitary Fermi gas}}, Ann. Physics {\bf 321}, 197 (2006).
\bibitem{SDM} S. Van Loon, C.A.R. Sá de Melo, \href{https://doi.org/10.48550/arXiv.2112.09232}{\g{Effects of quantum fluctuations on the low-energy collective modes of two-dimensional superfluid Fermi gases from the BCS to the Bose Limit}}, prépublication arXiv$:$2112.09232v2 (2023).
\bibitem{livreK} I. Khalatnikov, \textit{An Introduction to the Theory of Superfluidity} (CRC Press-Taylor \& Francis, Boca Raton, 2018).
\bibitem{Annalen} H. Kurkjian, Y. Castin, A. Sinatra, \href{https://hal.archives-ouvertes.fr/hal-01392846}{\g{Three-phonon and four-phonon interaction processes in a pair-condensed Fermi gas}}, Annalen der Physik {\bf 529}, 1600352 (2017).
\bibitem{Maris} H.J. Maris, \href{https://doi.org/10.1103/RevModPhys.49.341}{Phonon-phonon interactions in liquid helium}, Rev. Mod. Phys. {\bf 49}, 341 (1977).
\bibitem{FW} A.L. Fetter, J.D. Walecka, \textit{Quantum Theory of Many-Particle Systems} (Dover, Mineola, 2003).
\bibitem{Grynberg} D.R. Meacher, D. Boiron, H. Metcalf, C. Salomon, G. Grynberg, \href{https://doi.org/10.1103/PhysRevA.50.R1992}{\g{Method for velocimetry of cold atoms}}, Phys. Rev. A {\bf 50}, R1992(R) (1994).
\bibitem{Ketterle} J. Stenger, S. Inouye, A.P. Chikkatur, D.M. Stamper-Kurn, D.E. Pritchard, W. Ketterle, \href{https://link.aps.org/doi/10.1103/PhysRevLett.82.4569}{\g{Bragg spectroscopy of a Bose-Einstein condensate}}, Phys. Rev. Lett. {\bf 82}, 4569 (1999).
\bibitem{Davidson} J. Steinhauer, R. Ozeri, N. Katz, N. Davidson, \href{https://link.aps.org/doi/10.1103/PhysRevLett.88.120407}{\g{Excitation spectrum of a Bose-Einstein condensate}}, Phys. Rev. Lett. {\bf 88}, 120407 (2002).
\bibitem{Vale} G. Veeravalli, E. Kuhnle, P. Dyke, C.J. Vale, \href{https://link.aps.org/doi/10.1103/PhysRevLett.101.250403}{\g{Bragg spectroscopy of a strongly interacting Fermi gas}}, Phys. Rev. Lett. {\bf 101}, 250403 (2008).
\bibitem{Cartago} A. Sinatra, C. Lobo, Y. Castin, \href{https://doi.org/10.1088/0953-4075/35/17/301}{\g{The truncated Wigner method for Bose condensed gases: limits of validity and applications}}, J. Phys. B {\bf 35}, 3599 (2002).
\bibitem{CCTbordeaux} C. Cohen-Tannoudji, J. Dupont-Roc, G. Grynberg, \textit{Processus d'interaction entre photons et atomes} (EDP Sciences/CNRS Éditions, Paris, 1988).
\bibitem{AK} A. Andreev, I.M. Khalatnikov, \href{http://jetp.ras.ru/cgi-bin/dn/e_017_06_1384.pdf}{\g{Sound in Liquid Helium II Near Absolute Zero}}, Zh. Eksp. Teor. Fiz. {\bf 44}, 2058 (1963) [Sov. Phys. JETP {\bf 17}, 1384 (1963)].
\bibitem{Barc} M.A. Escobedo, C. Manuel, \href{https://doi.org/10.1103/PhysRevA.82.023614}{\g{Effective field theory and dispersion law of the phonons of a nonrelativistic superfluid}}, Phys. Rev. A {\bf 82}, 023614 (2010).
\bibitem{CB} Ming-Chiang Chung, A.B. Bhattacherjee, \href{https://doi.org/10.1088/1367-2630/11/12/123012}{\g{Damping in 2D and 3D dilute Bose gases}},  New J. Phys. {\bf 11}, 123012 (2009).
\bibitem{HM} M. Bohlen, L. Sobirey, N. Luick, H. Biss, T. Enss, Th. Lompe, H. Moritz,\href{https://journals.aps.org/prl/abstract/10.1103/PhysRevLett.124.240403}{\g{Sound Propagation and Quantum-Limited Damping in a Two-Dimensional Fermi Gas}}, Phys. Rev. Lett. {\bf 124}, 240403 (2020).
\bibitem{PCZH} P. Christodoulou, M. Gałka, N. Dogra, R. Lopes, J. Schmitt, Z. Hadzibabic, \href{https://doi.org/10.1038/s41586-021-03537-9}{\g{Observation of first and second sound in a BKT superfluid}}, Nature {\bf 594}, 191 (2021).
\bibitem{JDJB} J.L. Ville, R. Saint-Jalm, \'E. Le Cerf, M. Aidelsburger, S. Nascimb\`ene, J. Dalibard, J. Beugnon, \href{https://doi.org/10.1103/PhysRevLett.121.145301}{\g{Sound Propagation in a Uniform Superfluid Two-Dimensional Bose Gas}}, Phys. Rev. Lett. {\bf 121}, 145301 (2018).
\bibitem{Salasnich} A. Cappellaro, F. Toigo, L. Salasnich, \href{https://doi.org/10.1103/PhysRevA.98.043605}{\g{Collisionless dynamics in two-dimensional bosonic gases}}, Phys. Rev. A {\bf 98}, 043605 (2018).
\bibitem{Stringari} Miki Ota, F. Larcher, F. Dalfovo, L. Pitaevskii, N.P. Proukakis, S. Stringari, \href{https://doi.org/10.1103/PhysRevLett.121.145302}{\g{Collisionless Sound in a Uniform Two-Dimensional Bose Gas}}, Phys. Rev. Lett. {\bf 121}, 145302 (2018).
\bibitem{Bere} V.L. Berezinskii, \href{http://jetp.ras.ru/cgi-bin/dn/e_034_03_0610.pdf}{\g{Destruction of long-range order in one-dimensional and two-dimensional systems having a continuous symmetry group II. Quantum systems}}, Zh. Eksp. Teor. Fiz. {\bf 61}, 1144 (1971) [Sov. Phys. JETP {\bf 34}, 610 (1972)].
\bibitem{Nel} D.R. Nelson, J.M. Kosterlitz, \href{https://doi.org/10.1103/PhysRevLett.39.1201}{\g{Universal Jump in the Superfluid Density of Two-Dimensional Superfluids}}, Phys. Rev. Lett. {\bf 39}, 1201 (1977).
\bibitem{SJ} R. Saint-Jalm, \href{https://theses.hal.science/tel-03116129}{\it Exploration de la physique à deux dimensions avec des gaz de Bose dans des potentiels à fond plat: ordre en phase et symétrie dynamique}, thèse de doctorat de École normale supérieure et de l'Université PSL (Paris, octobre 2019).
\bibitem{broui} H. Kurkjian, Y. Castin, A. Sinatra, \href{https://doi.org/10.1016/j.crhy.2016.02.005}{Brouillage thermique d'un gaz cohérent de fermions}, Comptes Rendus Physique {\bf 17}, 789 (2016).
\bibitem{CD} Y. Castin, R. Dum, \href{https://doi.org/10.1007/s100530050584}{\g{Bose-Einstein condensates with vortices in rotating traps}}, Eur. Phys. J. D {\bf 7}, 399 (1999).
\bibitem{HD} Z. Hadzibabic, J. Dalibard, \href{https://doi.org/10.1393/ncr/i2011-10066-3}{\g{Two-dimensional Bose fluids: An atomic physics perspective}}, Rivista del Nuovo Cimento {\bf 34}, 389 (2011).
\bibitem{Popov} V.N. Popov, {\it Functional Integrals in Quantum Field Theory and Statistical Physics} (Reidel, Dordrecht, 1983).
\bibitem{Morathese} C. Mora, \href{https://theses.hal.science/tel-00005472v1}{\it Gaz de bosons et de fermions condens\'es~: phases de Fulde-Ferrell-Larkin-Ovchinnikov et quasicondensats}, th\`ese de doctorat de l'Universit\'e Paris VI (Paris, mars 2004).
\bibitem{MC2} C. Mora, Y. Castin, \href{https://doi.org/10.1103/PhysRevLett.102.180404}{\g{Ground state energy of the two-dimensional weakly interacting Bose gas: First correction beyond Bogoliubov theory}}, Phys. Rev. Lett. {\bf 102}, 180404 (2009). 
\bibitem{TheseAlan} A. Serafin, \href{https://theses.hal.science/tel-04002097}{\textit{Deux études de propriétés collectives de systèmes quantiques}}, thèse de doctorat de l'Université Sorbonne (Paris, décembre 2022).
\bibitem{NumRec} W.H. Press, S.A. Teukolsky, W.T. Vetterling, B.P.  Flannery, {\it Numerical Recipes} (Cambridge University Press, Cambridge, 1988).
\bibitem{SCE} A. Sinatra, Y. Castin, E. Witkowska, \href{https://doi.org/10.1103/PhysRevA.75.033616}{\g{Nondiffusive phase spreading of a Bose-Einstein condensate at finite temperature}}, Phys. Rev. A {\bf 75}, 033616 (2007).
\end{thebibliography}
\end{document}
%}
%)
