% (
% {
\documentclass[final,3p,times]{elsarticle_en}

% if you use PostScript figures in your article
% use the graphics package for simple commands
%\usepackage{graphics}
% or use the graphicx package for more complicated commands
\usepackage{graphicx}
% or use the epsfig package if you prefer to use the old commands
%\usepackage{epsfig}
% pour avoir la légende des figure sur le côté
\usepackage{sidecap}
% pour avoir les figures en tikz
\usepackage{tikz}
%\usetikzlibrary{arrows.meta,bending,positioning}

% The amssymb package provides various useful mathematical symbols
\usepackage{amssymb}
% pour avoir comment
\usepackage{version}

\usepackage[english,french]{babel}

%%%%%%%%%%%%%%%%%%%%%%%%%%%%%%%%
%% GUILLEMETS (FRENCH QUOTES) %%
%%%%%%%%%%%%%%%%%%%%%%%%%%%%%%%%
%\newcommand{\g}[1]{«~#1~»}
\def\og{\leavevmode\raise.3ex\hbox{$\scriptscriptstyle\langle\!\langle$~}}
\def\fg{\leavevmode\raise.3ex\hbox{~$\!\scriptscriptstyle\,\rangle\!\rangle$}}
%\newcommand{\g}[1]{\og~#1\fg}
\newcommand{\g}[1]{``#1''}

%%%%%%%%%%%%%%%%%%%%%%%%%%%%%%%%%
%  MACROS DEFINED BY YVAN CASTIN
%%%%%%%%%%%%%%%%%%%%%%%%%%%%%%%%%
%\usepackage[applemac]{inputenc}
\usepackage[utf8]{inputenc}
\usepackage[T1]{fontenc}
\usepackage{amsmath}
\usepackage{xcolor}
\usepackage{hyperref}
\usepackage{ulem} % pour pouvoir barrer du texte avec \sout{}
\hypersetup{breaklinks=true}
\hypersetup{colorlinks=true}
\hypersetup{urlcolor=blue}
\hypersetup{citecolor=black}
\hypersetup{linkcolor=black}
\usepackage{cleveref}
\newcommand{\bea}{\begin{eqnarray}}
\newcommand{\eea}{\end{eqnarray}}
\newcommand{\be}{\begin{equation}}
\newcommand{\ee}{\end{equation}}
\newcommand{\rr}{\mathbf{r}}
\newcommand{\RR}{\mathbf{R}}
\newcommand{\kk}{\mathbf{k}}
\newcommand{\ev}{\mathbf{e}}
\newcommand{\QQ}{\mathbf{Q}}
\newcommand{\KK}{\mathbf{K}}
\newcommand{\PP}{\mathbf{P}}
\newcommand{\FF}{\mathbf{F}}
\newcommand{\pp}{\mathbf{p}}
\newcommand{\qq}{\mathbf{q}}
\newcommand{\uu}{\mathbf{u}}
\newcommand{\vv}{\mathbf{v}}
\newcommand{\nn}{\mathbf{n}}
\newcommand{\kf}{k_{\mathrm{F}}}
\newcommand{\mD}{\mathcal{D}}
\newcommand{\mA}{\mathcal{A}}
\newcommand{\mB}{\mathcal{B}}
\newcommand{\mE}{\mathcal{E}}
\newcommand{\mJ}{\mathcal{J}}
\newcommand{\mK}{\mathcal{K}}
\newcommand{\mT}{\mathcal{T}}
\newcommand{\mR}{\mathcal{R}}
\newcommand{\mH}{\mathcal{H}}
\newcommand{\mI}{\mathcal{I}}
\newcommand{\mC}{\mathcal{C}}
\newcommand{\mG}{\mathcal{G}}
\newcommand{\KKK}{\mathbb{K}}
\newcommand{\zero}{\mathbf{0}}
\newcommand{\epsc}{\check{\epsilon}}
\newcommand{\veps}{\varepsilon}
\newcommand{\vepsb}{\bar{\varepsilon}}
\newcommand{\omb}{\bar{\omega}}
\newcommand{\mb}{\bar{m}}
\newcommand{\qb}{\bar{q}}
\newcommand{\kb}{\bar{k}}
\newcommand{\kpb}{\bar{k}'}
\newcommand{\thetat}{\tilde{\theta}}
\newcommand{\zetat}{\tilde{\zeta}}
\newcommand{\zetac}{\check{\zeta}}
\newcommand{\sigg}{\boldsymbol{\sigma}}
\newcommand{\yvan}{\color{black}}
\newcommand{\gilles}{\color{black}}
\newcommand{\simon}{\color{black}}
\renewcommand{\vec}{\mathbf}


\newcommand{\ii}{\textrm{i}}
\newcommand{\ff}{\textrm{f}}
\newcommand{\eee}{\textrm{e}}
\newcommand{\dd}{\mathrm{d}}
% opérateur identité
\def\identit{\mbox{l\hspace{-0.55em}1}}
\DeclareMathOperator\erf{erf}
\DeclareMathOperator\erfi{erfi}

\DeclareMathOperator\asin{arcsin}
\DeclareMathOperator\acos{arccos}
\DeclareMathOperator\atan{arctan}
\DeclareMathOperator\arccotan{arccotan}
\DeclareMathOperator\argth{argth}
\DeclareMathOperator\arctanh{arctanh}
\DeclareMathOperator\argch{argch}
\DeclareMathOperator\argsh{argsh}
\DeclareMathOperator\thf{th}
\DeclareMathOperator\ch{ch}
\DeclareMathOperator\sh{sh}
\DeclareMathOperator\re{Re}
\DeclareMathOperator\im{Im}
\DeclareMathOperator\sig{sign}
\newcommand{\bleu}{\color{blue}}
\newcommand{\rouge}{\color{red}}
\newcommand{\yc}{\color{black}}
\newcommand{\ycd}{\color{black}}
\newcommand{\yct}{\color{black}}
% pour pouvoir mettre des figures dans une minipage
\usepackage{float}

% on redefinit la fleche des vecteurs pour qu'elle soit de longueur ajustable
\newcommand{\vecl}{\overrightarrow}

%\journal{Comptes rendus - Physique}

\begin{document}

\begin{frontmatter}

% Title, authors and addresses

% use the thanksref command within \title, \author or \address for footnotes;
% use the ead command for the email address,
% and the form \ead[url] for the home page:
% \title{Title\thanksref{label1}}
% \thanks[label1]{}
% \author{Name\thanksref{label2}}
% \ead{email address}
% \ead[url]{home page}
% \thanks[label2]{}
% \address{Address\thanksref{label3}}
% \thanks[label3]{}
%\selectlanguage{french}
\selectlanguage{english}
\title{Phonon damping in a 2D superfluid: insufficiency of Fermi's golden rule at low temperature}

% use optional labels to link authors explicitly to addresses:
% \author[label1,label2]{}
% \address[label1]{}
% \address[label2]{}
% If all authors are at the same address, the [label1] can be suppressed

\author{Yvan Castin}
\ead{yvan.castin@lkb.ens.fr}
\author{Alan Serafin}
\ead{a.serafin@protonmail.com}
\author{Alice Sinatra}
\ead{alice.sinatra@lkb.ens.fr}

\address{Laboratoire Kastler Brossel, ENS-Universit\'e PSL, CNRS, Universit\'e Sorbonne and Coll\`ege de France, 24 rue Lhomond, 75231 Paris, France}
% If your know the dates of reception, and acceptation you can put them now;
%    idem the name of the person presenting your article

%\medskip

%\begin{center}
%{\small Received *****; accepted after revision +++++}
%\end{center}

\begin{abstract}
It is generally accepted that the phonon gas of a superfluid always enters a weak coupling regime at sufficiently low temperatures, whatever the strength of the interactions between the underlying particles (constitutive of the superfluid). Thus, in this limit, we should always be able to calculate the damping rate of thermal phonons by applying Fermi's golden rule to the Hamiltonian {\yc $H_3$} of cubic phonon-phonon coupling taken from quantum hydrodynamics, at least in the case of a convex acoustic branch and in the collisionless regime (where the eigenfrequency of the considered phonons remains much greater than the gas thermalization rate). Using the many-body Green's function method, we predict that, unexpectedly, this is not true in two dimensions, contrary to the three-dimensional case. We confirm this prediction with classical phonon-field simulations and a non-perturbative theory in $H_3$, where the fourth order is regularized by hand, giving a complex energy to the virtual phonons of the four-phonon collisional processes. For a weakly interacting fluid and a phonon mode in the long-wavelength limit, we predict a damping rate about three times lower than that of the golden rule.
\\
\noindent{\small {\it Keywords:} Bose gas ; two-dimensional superfluid ; cold atom gas ; Beliaev-Landau phonon damping; many-body Green's functions; quantum hydrodynamics}

\noindent 
\vskip 0.5\baselineskip%}{
\end{abstract} 
\end{frontmatter}

% now the Version franÁaise abrÈgÈe, if it exists
%\selectlanguage{french}
%\section*{Version fran\c{c}aise abr\'eg\'ee}
% Text of your Version franÁaise abrÈgÈe here

\selectlanguage{english}
% main text
%\newpage
\tableofcontents

%\input papier.txt
%\input sortie.sed
%\input sortie_rouge.sed
%\input paper.txt

\section{Presentation of the problem and general reasoning}
\label{sec0}

In dimension $d\geq 2$, we consider a spatially homogeneous system of particles with short-range interactions, entirely superfluid in its ground state and prepared at a non-zero but arbitrarily low temperature, which allows us to restrict its study to leading order in temperature by means of an effective low-energy theory. Irrespective of the quantum statistics (bosonic or fermionic) of the particles and the strength of their interactions (in gas or liquid phase), this system has an acoustic excitation branch, i.e. with a linear departure in wave number (with no band gap), whose quanta are the phonons. 

The coupling between low-wavenumber phonons (of energy $<\eta k_B T$, where temperature $T\to 0$ then cutoff $\eta\to+\infty$ at the end of the calculations) is deduced from the quantum hydrodynamics of Landau and Khalatnikov \cite{LK}, the effective Hamiltonian taking the form\footnote{\label{note1} The inclusion of higher-order terms $H_5, H_6$, etc., would be meaningless without gradient corrections to the energy functional of quantum hydrodynamics \cite{SonWingate}, corrections which also bring the curvature term (here put by hand) into equation (\ref{eq003}).} 
\begin{equation}\label{eq001}
H=H_2+H_3+H_4
\end{equation} where term $H_n$ is of degree $n$ in terms of $\hat{b}_\kk^\dagger$ and $\hat{b}_\kk$, bosonic creation and annihilation operators of a phonon of wave vector $\kk$ in a quantization box $[0,L]^d$ with periodic boundary conditions. The term $H_2$ represents an ideal gas of phonons of dispersion relation $\veps_\kk$, 
\begin{equation}\label{eq002}
H_2=\sum_{\kk\neq\zero} \veps_\mathbf{k}\hat{b}_\kk^\dagger\hat{b}_\mathbf{k}\end{equation} The precise form of the coupling terms $H_3$ and $H_4$ in equation (\ref{eq001}) is of little importance at this stage and will be given in section \ref{sec1}. The coupling between phonons makes the system acoustically non-linear and causes the damping of sound waves, the subject of the present work. 

We limit our study of damping here (i) to modes of wave vector $\qq$ in the collisionless regime $\veps_\qq\gg\hbar\Gamma_{\rm th}$ where $\Gamma_{\rm th}$ is the thermalization rate of the phonon gas or, what amounts to the same thing, the damping rate of thermal phonons of energy $\veps_\kk\simeq k_B T$, and (ii) to the case of an acoustic branch with a convex start in wavenumber, which can be limited to its cubic approximation at low temperature: 
\begin{equation}\label{eq003}
\veps_\kk=\hbar c k \left[1+\frac{\gamma}{8} \left(\frac{\hbar k}{m c}\right)^2\right]\quad\mbox{with}\quad \gamma>0
\end{equation} where $m$ is the mass of a particle, $c$ is the speed of sound at zero temperature and $\gamma$ is a dimensionless curvature parameter. \footnote{\label{note0} As in quantum electrodynamics, the parameters appearing in the Hamiltonian are in fact the bare values, different from the actual values measured in an experiment, the passage from one to the other being made by a procedure known as \g{renormalisation}. Equation (\ref{eq002}) should therefore refer to a bare dispersion relation $\veps_\kk^{(0)}$ and equation (\ref{eq003}) to a bare speed of sound $c^{(0)}$ and a bare curvature parameter $\gamma^{(0)}$. We do not do so here to lighten the notations, and because this renormalization does not intervene in dimension $d=2$ at the leading order in temperature.}\footnote{The choice $\gamma>0$ excludes superfluids of paired spin $1/2$ fermions in the limit, known as BCS, of a weakly attractive interaction between particles \cite{SDM}.} These assumptions allow us to describe damping in terms of elementary phonon-coupling processes\footnote{In the opposite regime $\veps_\qq\ll\hbar\Gamma_{\rm th}$, the system has time to relax to a juxtaposition of local thermodynamic equilibrium states during an oscillation period of the sound wave of wavevector $\qq$ and damping at non-zero temperature is described through transport or viscosity coefficients, obtained by solving kinetic equations and which enter into partial differential equations on densities, velocities, volume entropy, etc, of a two-fluid model \cite{livreK}.} and, as is generally accepted, to deduce the damping rate from Fermi's golden rule applied to the $H_3$ term of the Hamiltonian.\footnote{In the case $\gamma<0$ of an acoustic branch with a concave start, the three-phonon processes induced by $H_3$ to order one do not conserve momentum-energy, and an extended formulation of the golden rule (taking into account the four-phonon processes induced by $H_3$ to order two and by $H_4$ to order one) must be used to obtain a non-zero damping rate, see references \cite{LK,Annalen} and our section \ref{sec3.2}.} Indeed, it is expected that at sufficiently low temperatures, the phonon gas will always enter the weak coupling regime, even if the constituent particles of the system are strongly interacting (as in liquid helium 4, for example \cite{Maris}). We will show in this section, by simple reasoning, that this is true in dimension $d=3$ but not in dimension $d=2$. Phonon damping in the two-dimensional case will therefore be the subject of further study (beyond the golden rule) in subsequent sections. 

Our simple reasoning is based on many-body Green’s function method \cite{FW}. To fix ideas and the Green's function to be considered, we assume that the gas, initially at thermal equilibrium for the Hamiltonian $H$ (density operator $\hat{\sigma}(0)=\hat{\sigma}_{\rm th}\propto \exp(-H/k_B T)$), is excited by the sudden creation of a coherent state of phonons in mode $\qq$: 
\begin{equation}\label{eq010}
\hat{\sigma}(0^+) = \hat{U}_{\rm exc}\hat{\sigma}_{\rm th} \hat{U}^\dagger_{\rm exc} \quad\mbox{with}\quad \hat{U}_{\rm exc}=\eee^{\alpha\hat{b}_\qq^\dagger-\alpha^*\hat{b}_\qq}
\end{equation} Experimentally, this excitation procedure can be carried out by a short-duration 
Bragg laser pulse \cite{Grynberg,Ketterle,Davidson,Vale,Cartago}. The amplitude $\langle\hat{b}_\qq(t)\rangle$ of the coherent state is then measured over time, for example through the gas density (more precisely, its Fourier components of wave vectors $\pm\qq$). In the limit of very low initial excitation, we enter the linear response regime and can formally expand the unitary excitation operator $\hat{U}_{\rm exc}$ to order one in $\alpha$ in equation (\ref{eq010}) to obtain 
\begin{equation}\label{eq011}
\langle\hat{b}_\qq(t)\rangle\stackrel{t>0}{\underset{\alpha\to 0}{=}} \alpha \langle[\hat{b}_\qq(t),\hat{b}_\qq^\dagger(0)]\rangle_{\rm th}+O(\alpha^2)
\end{equation} where $\hat{b}_\qq(t)$ is written in Heisenberg representation for the evolution unperturbed by $\hat{U}_{\rm exc}$, $\hat{b}_{\qq}(t)=\eee^{\textrm{i}H t/\hbar} \hat{b}_{\qq}(0)\eee^{-\textrm{i}H t/\hbar}$, the mean of the left-hand side is taken in the perturbed state $\hat{\sigma}(0^+)$ and that of the right-hand side in the initial unperturbed thermal state $\hat{\sigma}_{\rm th}$, and we have used the fact that $\langle\hat{b}_{\qq}(0)\hat{b}_{\qq}(t)\rangle_{\rm th}=\langle\hat{b}_{\qq}(t)\hat{b}_{\qq}(0)\rangle_{\rm th}=0$ by non-conservation of the total momentum. Using section 31 of reference \cite{FW}, we relate our signal to the two-point Green's function $\mG$ {\yc of the phonon field} and the associated self-energy $\Sigma$ (noted $\Sigma^{\star}$ in \cite{FW}):\footnote{Recall that, originally, $\mG$ and $\Sigma$ are functions of $(\rr,t)$ and $(\rr',t')$, where $\rr,\rr'$ are two position vectors and $t,t'$ are two times. {\yct By spatial homogeneity and stationarity of the thermal state,} they are reduced to functions of $(\rr-\rr',t-t')$, whose space-time Fourier transforms are functions of $(\kk,\omega)$; the analytic continuation of the latter to complex frequencies $z/\hbar$ leads to functions $\mG(\kk,z)$ and $\Sigma(\kk,z)$ of equation (\ref{eq012}).} 
\begin{equation}\label{eq012}
\boxed{s(t)\equiv \eee^{\ii\veps_\mathbf{q}t/\hbar}\langle[\hat{b}_\qq(t),\hat{b}_\qq^\dagger(0)]\rangle_{\rm th}\stackrel{t>0}{=} \int_{C_+} \frac{\mathrm{d}z}{2\ii\pi} \eee^{-\textrm{i}(z-\veps_\qq)t/\hbar} \mathcal{G}(\qq,z)\quad\mbox{with}\quad \mathcal{G}(\qq,z)=\frac{1}{z-\veps_\qq-\Sigma(\qq,z)}}
\end{equation} where the integration path $C_+$ runs in the upper half {\yc of the complex} plane parallel to the real axis from $+\infty$ to $-\infty$ and where we have, for convenience, removed the free evolution due to the Hamiltonian $H_2$. In the following, it will be advantageous to take as the new complex energy variable in $\Sigma$ the deviation from the free-evolution energy, any confusion being avoided by indexing the wave vector dependence: 
\begin{equation}\label{eq013}
\Sigma(\qq,z)=\Sigma_\qq(\zeta)\quad\mbox{with}\quad\zeta\equiv z-\veps_\mathbf{q}\quad{\ycd\mbox{hence}\quad\mathcal{G}(\qq,z)=\frac{1}{\zeta-\Sigma_\qq(\zeta)}}\end{equation} From this exact formalism, we can recover the exponential damping of Fermi's golden rule $|s(t)|^2\simeq \exp(-\Gamma_q t)$ in two steps. (i) First, the pole approximation \cite{CCTbordeaux} is performed, neglecting the energy dependence of $\Sigma$: 
\begin{equation}\label{eq014}
\Sigma_\qq(\zeta) \approx \Sigma_\qq(\textrm{i}0^+)\quad \mbox{so that}\quad s(t)\simeq s_{\mbox{\scriptsize pole}}(t)=\exp[-\textrm{i}\Sigma_\qq(\textrm{i}0^+)t/\hbar]
\end{equation} (ii) As the pole approximation is generally legitimate only in a weak coupling limit between phonons, it suffices to calculate $\Sigma_\qq$ at the leading order in the coupling, i.e. at second order in $H_3$, but also at first order in $H_4$ if we need to know $\re\Sigma_\qq$. The central question of this section is whether simply taking a low-temperature limit with proper rescaling of the wave number of the mode under consideration,\footnote{It should be pointed out that the strength of the interactions between the constituent particles of the fluid does not vary in the limit (\ref{eq020}), so that the speed of sound $c$ remains constant. On the other hand, limit (\ref{eq020}) automatically brings mode $\qq$ into the collisionless regime $\veps_{\qq}\gg \hbar\Gamma_{\rm th}$ because the rate $\Gamma_{\rm th}$ of gas thermalization, which can be estimated by the damping rate of phonons of wave number $\approx k_B T/\hbar c$, tends to zero faster than linearly in temperature, as we shall see.} 
\begin{equation}\label{eq020}
\epsilon \equiv k_B T/mc^2 \to 0 \quad \mbox{with}\quad \qb\equiv \hbar c q/k_B T \quad \mbox{fixed}
\end{equation} is enough to bring the phonon gas into the weak coupling regime and justify the two steps. We'll see that the answer depends on the dimensionality of the system. 
\paragraph{Step (i)}
 To justify the pole approximation, we rely on the scaling laws describing the behavior of the self-energy at low temperature: 
\begin{equation}\label{eq021}
\Sigma_\qq(\zeta) = k_B T \epsilon^\nu\, \tilde{\Sigma}_{\qb}(\zeta/(k_BT\epsilon^\sigma))
\end{equation} After outputting the energy scale $k_B T$, two exponents $\nu$ and $\sigma$ remain, the first characterizing the order of magnitude of function $\Sigma_\qq(\zeta)$, the second its energy width around $\zeta=0$. It is understood that the new function $\tilde{\Sigma}_{\qb}$ has a finite, non-zero limit when $\epsilon\to 0$. After inserting form (\ref{eq021}) into the signal expression (\ref{eq012}), we {\yct perform} the following scaling on the conjugate variables $\zeta$ and $t$, 
\begin{equation}\label{eq022}
\zeta\equiv k_B T \epsilon^\nu \tilde{\zeta}\quad ; \quad t\equiv \tilde{t}\,\hbar/(k_B T \epsilon^\nu)
\end{equation} to obtain the reduced expression 
\begin{equation}\label{eq023}
\boxed{s(t)=\int_{C_+} \frac{\mathrm{d}\zetat}{2\ii\pi} \frac{\eee^{-\textrm{i}\tilde{\zeta}\tilde{t}}}{\zetat-\tilde{\Sigma}_{\qb}(\epsilon^{\nu-\sigma}\zetat)}}
\end{equation} in which $\epsilon$ still has to tend towards zero at a fixed $\tilde{t}$. We then see that, if $\nu>\sigma$, the argument of function $\tilde{\Sigma}_{\qb}$ tends towards zero by positive imaginary-part values when $\epsilon\to 0$, which means it can be replaced by $\textrm{i}0^+$: we find the pole approximation (\ref{eq014}), which is exact in this limit. 

\paragraph{Step (ii)}
 To find out whether it's sufficient to calculate the self-energy to the leading order in phonon coupling, let's keep only the cubic coupling $H_3$ for simplicity (we can check that the quartic coupling $H_4$ doesn't change anything qualitatively), omit the non-resonant processes it contains (terms in $\hat{b}^\dagger\hat{b}^\dagger\hat{b}^\dagger$ and $\hat{b}\hat{b}\hat{b}$) of sub-dominant contribution in the $\epsilon\to 0$ limit, and estimate the order of magnitude of its contribution of order $2n$ to $\Sigma_\qq{\ycd(\ii 0^+)}$ as follows: \footnote{As $H_3$ changes the parity of the phonon number, the odd-order contributions $2n+1$ are zero.} 
\begin{equation}\label{eq024}
\Sigma^{(2n)}_\qq(\textrm{i}0^+)\approx \int \left(\prod_{i=1}^{n} \dd^d k_i\right) \langle\ |\mathcal{H}_3|\ \rangle \frac{1}{\Delta E_1} \langle\ |\mathcal{H}_3|\ \rangle \ldots \langle\ |\mathcal{H}_3|\ \rangle \frac{1}{\Delta E_{2n-1}} \langle\ |\mathcal{H}_3|\ \rangle
\end{equation} Let's briefly justify form (\ref{eq024}) (the reader is referred to sections \ref{sec2} and \ref{sec3} for a detailed treatment of cases $n=1$ and $n=2$). Order $2n$ involves $2n$ actions of $H_3$ so $2n$ matrix elements $\langle\,|H_3|\,\rangle$, each of which introduces a new independent phonon wave vector $\kk_i$ (Beliaev terms $\hat{b}^\dagger\hat{b}^\dagger\hat{b}$ in $H_3$) or makes one disappear (Landau terms $\hat{b}^\dagger\hat{b}\hat{b}$); the other phonon wave vectors that appear are deduced from $\kk_i$ and $\qq$ by conservation of momentum. Since the $2n$ actions of $H_3$ must conserve the number of phonons as a whole, there must be as many Beliaev processes as Landau processes, leading to exactly $n$ independent wave vectors on which to sum. In the thermodynamic limit, we express $H_3$ in terms of the {\yct volumic} Hamiltonian such that $H_3=\mH_3/L^{d/2}$ and replace the sum over the $\kk_i$ by an integral. Finally, between each matrix element of $H_3$ appears {\yc an energy} denominator representing the free propagation of Hamiltonian $H_2$ and giving the energy difference between {\yct the initial state with a phonon $\qq$ and} an intermediate state. Each $\Delta E_j$ ($1\leq j\leq 2n-1$) can be expressed as the algebraic sum (i.e. with a plus or minus sign) of $j$ elementary energy defects $\veps_\kk+\veps_{\kk'}-\veps_{\kk''}$ corresponding to a $\kk+\kk'\leftrightarrow\kk''$ Beliaev-Landau process. In the low-temperature limit $\epsilon\to 0$, the integral (\ref{eq024}) is in general dominated by \g{small-angle} processes, where {\it all} intermediate phonons (those with wave vectors $\kk_i$ and others) are almost collinear to and in the same direction as $\qq$ \cite{LK,Annalen}; more precisely, we find that the angles $\theta_i$ between vectors $\qq$ and $\kk_i$ must be $O(\gamma^{1/2}\epsilon)$ for $\Delta E_j$ to reach their minimum energy scale {\yc $k_B T\epsilon^2$} and for the integrand to be amplified by the \g{small-denominator} effect, which leads to the expansion of the elementary energy defect 
\begin{equation}\label{eq025}
\veps_\kk+\veps_{\kk'}-\veps_{\kk''=\kk+\kk'}\underset{\epsilon\to 0}{\sim} k_B T\gamma\epsilon^2\left[\frac{\kb\kpb}{\kb+\kpb}\frac{(\theta-\theta')^2}{2\gamma\epsilon^2}-\frac{3}{8}\kb\kpb(\kb+\kpb)\right]
\end{equation} where $\gamma$ is the curvature parameter in the dispersion relation (\ref{eq003}), {\yc angles $\theta=\widehat{(\qq,\kk)}$ and $\theta'=\widehat{(\qq,\kk')}$ are introduced,} and the scaling of $k$ to $\kb$ by its typical value $k_B T/\hbar c$ is based on equation (\ref{eq020}). The $\Delta E_j$ are {\yc thus effectively} of order $k_B T\epsilon^2$ and each $\dd^dk_i$ integration element in (\ref{eq024}) written in polar or spherical coordinates brings out a factor $(k_B T/\hbar c)^d (\gamma^{1/2}\epsilon)^{d-1}$. It remains to give the estimate of the matrix elements of $\mH_3$, which is easily derived from their expression (\ref{eq102}) to come (the coupling constant $\Lambda$ is given by equation (\ref{eq032})): 
\begin{equation}\label{eq027}
\langle\ |\mathcal{H}_3|\ \rangle \approx (k\xi k'\xi k''\xi)^{1/2}(1+\Lambda)\frac{mc^2}{\rho^{1/2}} \approx \epsilon^{3/2}(1+\Lambda)\frac{mc^2 \xi^{d/2}}{(\rho\xi^d)^{1/2}}
\end{equation} where we have introduced the density $\rho$ of the fluid and the so-called relaxation or correlation length of the superfluid, 
\begin{equation}\label{eq026}
\xi=\frac{\hbar}{mc}
\end{equation} We thus end up with the estimate 
\begin{equation}\label{eq028}
\boxed{\Sigma_\qq^{(2n)}(\ii0^+)\underset{\epsilon\to 0}{\approx} k_BT \gamma \epsilon^2 \left[\frac{\epsilon^{2d-4}(1+\Lambda)^2}{\gamma^{(5-d)/2}\rho\xi^d}\right]^n \quad\quad (n>0,d\geq 2)}
\end{equation} which designates the quantity in square brackets as the small parameter of the perturbative expansion of $\Sigma_\qq$ in the coupling. The degree in $\epsilon$ of the leading non-zero order, corresponding to the choice $n=1$ in equation (\ref{eq028}), gives the value of the exponent $\nu$ in equation (\ref{eq021}): 
\begin{equation}\label{eq004}
\nu=2d-2
\end{equation} The value of the second exponent $\sigma$ is obtained by generalizing estimate (\ref{eq024}) to the case $\zeta\neq 0$, which amounts to adding $\zeta$ to the energy of the initial state and therefore to each energy denominator, $\Delta E_j\to \Delta E_j+\zeta$. As we've seen, $\Delta E_j\approx k_B T\epsilon^2$; this is also the width of the self-energy around $z=\epsilon_\qq$ (around $\zeta=0$), so that
\begin{equation}\label{eq005}
\sigma=2
\end{equation} 
irrespective of dimensionality.

\paragraph{Three-dimensional case}
 In dimension $d=3$, where $\nu=4>\sigma=2$ and where $2d-4=2>0$ in the square brackets of equation (\ref{eq028}), steps (i) and (ii) are justified and the damping of mode $\qq$ is described exactly by Fermi's golden rule in the limit (\ref{eq020}). We then find the low-temperature behavior of the imaginary part of $\Sigma_\qq(\textrm{i}0^+)$, 
\begin{multline}
\label{eq030}
\im\Sigma_\qq^{d=3}(\textrm{i}0^+) \stackrel{\qb\,\mbox{\scriptsize fixed}}{\underset{\epsilon\to 0}{\sim}} -\frac{9 k_B T \epsilon^4(1+\Lambda)^2}{32\pi\rho\xi^3}\Bigg[\int_0^{\qb}\dd\kb\, \kb^2(\qb-\kb)^2(\bar{n}^{\rm lin}_k+1/2)+\int_0^{+\infty}\dd\kb\, \kb^2(\kb+\qb)^2(\bar{n}^{\rm lin}_k-\bar{n}^{\rm lin}_{k+q})\Bigg] \\
=-\frac{9 k_B T \epsilon^4(1+\Lambda)^2}{32\pi\rho\xi^3}\Bigg\{\frac{\qb^5}{60}+48[g_5(1)-g_5(\eee^{-\qb})]-24\qb\, g_4(\eee^{-\qb})+4\qb^2[g_3(1)-g_3(\eee^{-\qb})]\Bigg\}
\end{multline}
 corresponding exactly to the $\Gamma_q$ damping rate given in reference \cite{Annalen} (multiplied by the usual $-\hbar/2$ factor). In equation (\ref{eq030}), Bose functions $g_\alpha(z)=\sum_{n=1}^{+\infty} z^n/n^\alpha$ have been introduced and it has been possible to replace the dispersion relation $\veps_\kk$ by its linear approximation $\hbar c k$ in the occupation numbers, which is indicated by the exponent \g{lin}: 
\begin{equation}\label{eq029}
\bar{n}_k^{\rm lin}=\frac{1}{\exp\kb-1}
\end{equation} It is satisfying to note that the prefactor in (\ref{eq030}) is indeed that predicted by (\ref{eq028}) for $n=1$ (and $d=3$); in particular, it is independent of the curvature parameter $\gamma$. Green's function formalism also gives access to the real part, which {\yct was out of reach of} reference \cite{Annalen} and represents the thermal energy shift of phonon $\qq$,\footnote{We omit in (\ref{eq031}) the so-called quantum component, independent of occupation numbers, which is at the leading order in $\epsilon$ the sum of two contributions, (i) an ill-defined contribution, linear combination of monomials $q$, $q^3$ and $q^5$ with coefficients divergently dependent on the ultraviolet cutoff $\eta$ of equation (\ref{eq103}), which must therefore be reabsorbed in $\veps_\qq$, i.e. in a redefinition of the dispersion relation (see also footnote \ref{note0}), and (ii) a universal contribution predicted quantitatively by quantum hydrodynamics, $\re\Sigma_{\qq,{\rm quant\ univ}}^{d=3}(\textrm{i}0^+)=[3(1+\Lambda)^2/(160\pi^2)](\hbar^2 q^5/m\rho)\ln(\hbar q/mc)$. Such a contribution in $q^5\ln q$ was qualitatively expected in reference \cite{SonWingate}. It appears in equation (35) of reference \cite{Barc}, with the correct numerical coefficient but with an extra factor $c$.} 
\begin{multline}
\label{eq031}
\re\Sigma_{\qq,{\rm th}}^{d=3}(\textrm{i}0^+) \stackrel{\qb\,\mbox{\scriptsize fixed}}{\underset{\epsilon\to 0}{\sim}} \frac{k_B T \epsilon^4(1+\Lambda)^2}{2\pi^2\rho\xi^3}
\Bigg\{\qb\zeta(4)\left[-27\ln\frac{\eee(3\gamma)^{1/2}\epsilon}{4}+\frac{2+18\Lambda+3\Lambda_*/2}{(1+\Lambda)^2}\right] \\
+\frac{9}{8} \int_0^{+\infty} \dd\kb\, \kb^2\, \bar{n}^{\rm lin}_k \left[(\kb-\qb)^2\ln|\kb-\qb|-(\kb+\qb)^2\ln(\kb+\qb)\right] \Bigg\}
\end{multline}
 where $\zeta$ is the Riemann zeta function, such {\yct that} $\zeta(s)=g_s(1)$, and $\zeta(4)=\pi^4/90$. For the real part, we obtain a more complicated form than that of (\ref{eq028}) because (a) the angles between $\qq$ and $\kk$ of order $\theta\approx\epsilon^0$ contribute at the same level as the small angles $\theta\approx\epsilon$,\footnote{Indeed, in the angular integration, in $d=3$, $\sin\theta\,\dd\theta/(\Delta E/k_B T)$ is $\approx \epsilon^0$ at large angles and $\approx \epsilon^2/\epsilon^2=\epsilon^0$ at small angles, where $\Delta E$ is an energy denominator. This phenomenon does not occur for the imaginary part, given the energy-conserving Dirac distribution $\delta(\Delta E)$ and the fact that $\Delta E$ only vanishes at small angles for low-temperature thermal phonons. In practice, to obtain (\ref{eq031}), we split the angular integral into two intervals, $\theta\in[0,\alpha]$ and $\theta\in [\alpha,\pi]$, where $\alpha\ll 1$ is fixed. In the first interval, the curvature term in the dispersion relation is kept in the denominator of the integrand, but the numerator is replaced by its leading order in $\theta$ (i.e. $\sin\theta\simeq\theta$). In the second interval, we use the linear dispersion relation, but keep all the dependence in $\theta$ of the trigonometric functions. All calculations done, the lower and upper parts of the integral include a divergence in $\ln\alpha$ which disappears exactly in their sum.} and (b) the quartic Hamiltonian $H_4$ also contributes, which forces us to introduce the second coupling constant $\Lambda_*$ of equation (\ref{eq032}). In particular, we notice a logarithmic correction to the $k_B T \epsilon^4$ law. In the $\qb\to 0$ limit, expression (\ref{eq031}) becomes $\sim \hbar q \delta c_{\rm th}$, where coefficient $\delta c_{\rm th}$ is the thermal correction to the speed of {\ycd collisionless} sound in the fluid. The value of $\delta c_{\rm th}$ derived from Andreev and Khalatnikov's kinetic theory \cite{AK}, see equation (45) in reference \cite{Barc}, is in good agreement with ours,\footnote{Perfect agreement is reached if, in equation (45) of \cite{Barc}, we replace $\ln 27$ by $25/6-2C+\ln(3/2)+2\zeta'(4)/\zeta(4)=\ln(26.852\ldots)$ where $C=0.577\ldots$ is Euler's constant. This small deviation results from an approximation made by Andreev and Khalatnikov in the calculation of an integral, that of equation (13) in \cite{AK}.} but that deduced from the leading order of effective field theory, see equation (33) in reference \cite{Barc}, is in disagreement.

\paragraph{Two-dimensional case}
 In dimension $d=2$, where $\nu=\sigma=2$ and where $2d-4=0$, we fail to justify step (i) and step (ii) at low temperature! In particular, the bracketed expression in equation (\ref{eq028}) does not depend on $\epsilon$ and only becomes a small parameter in the $\rho\xi^2\gg 1$ limit where the underlying superfluid itself (and not just the phonon gas) enters the weakly interacting regime. So, in $d=2$, taking the $\epsilon\to 0$ limit makes neither the pole approximation nor the $\Sigma_\qq(\zeta)$ second-order calculation in phonon coupling accurate. To obtain the correct expression of the signal in this limit, we need to calculate $\tilde{\Sigma}_{\qb}(\zetat)$ for all values of $\zetat$ and at all orders in $H_3$. However, let's take as a reference value the low-temperature behavior of $\Sigma_\qq(\textrm{i}0^+)$ at the leading order in the coupling ($n=1$): 
\begin{multline} 
\label{eq033}
\Sigma_\qq^{(2)}(\textrm{i}0^+)\stackrel{\qb\,\mbox{\scriptsize fixed}}{\underset{\epsilon\to 0}{\sim}} -\ii\frac{9k_B T\epsilon^2(1+\Lambda)^2}{8\pi(3\gamma)^{1/2}\rho\xi^2} \Bigg[\int_0^{\qb}\dd\kb\, \kb(\qb-\kb)(\bar{n}^{\rm lin}_k+1/2)+\int_0^{+\infty}\dd\kb\, \bar{k}(\kb+\qb) (\bar{n}^{\rm lin}_k-\bar{n}^{\rm lin}_{k+q})\Bigg] \\
=\boxed{-\ii\frac{9k_B T\epsilon^2(1+\Lambda)^2}{8\pi(3\gamma)^{1/2}\rho\xi^2} \Bigg[\frac{\qb^3}{12}+2\zeta(2)\qb\Bigg]\equiv -\textrm{i}\frac{\hbar\Gamma_q}{2}}
\end{multline}
 with $\zeta(2)=\pi^2/6$. At sufficiently low temperatures, $\Sigma^{(2)}_\qq(\textrm{i}0^+)$ becomes pure imaginary and reduces to $-\ii\hbar\Gamma_q/2$ where $\Gamma_q$ is the damping rate of the golden rule at the leading order in $\epsilon$ (the quartic Hamiltonian $H_4$ does not show a real part here as it is subleading in $\epsilon$, see section \ref{sec2}). Equation (\ref{eq033}) is, as far as we know, original. In limits $\qb\to 0$ and $\qb\to +\infty$, it reproduces equations (30) and (22) of reference \cite{CB} (times $-\ii\hbar$ given the conventions of \cite{CB}), which deals with the two-dimensional gas of weakly interacting bosons for which $\gamma=1$ (as in the Bogolioubov spectrum) and $\Lambda=0$ (as in the zero-temperature mean-field equation of state $\mu=\rho g$ used in \cite{CB}). In the following, in dimension $d=2$, we go beyond Fermi's golden rule with classical field numerical simulations in section \ref{sec1} and with the Green's function method to higher and higher orders in $H_3$ in later sections: to order two in section \ref{sec2}, to order four in section \ref{sec3} and non-perturbatively in section \ref{sec4}; to our surprise, we are forced to do so by the failure of estimate (\ref{eq028}) for $n\geq 2$. {\ycd In fact, by an explicit calculation of the fourth order contribution in $H_3$ in section \ref{sec3}, we establish a troubling result that disproves (\ref{eq028}) in dimension $d=2$: $\Sigma_{\qq}^{(4)}(\ii 0^+)$ does not tend to zero as $k_B T \epsilon^2$ at low temperature $\epsilon \to 0$, but as $k_B T\epsilon^0$, see equation (\ref{eq296}). This effectively makes the perturbative expansion in powers of $H_3$ unusable in this limit, since the fourth order becomes much larger than the second order for a fixed interaction strength $(\rho\xi^2)^{-1}$ in the superfluid. To obtain an operational description of the damping, section \ref{sec4}, contrary to equation (\ref{eq028}), takes the limit $\epsilon \to 0$ with fixed $\zetat\neq 0$ in $\tilde{\Sigma}_{\qb}^{(4)}(\zetat)$, regularizes the result (divergent at $\zetat=0$) nonperturbatively by assigning finite lifetimes to the intermediate phonons of the four-phonon collisional processes, and calculates the integral over $\zetat$ in equation (\ref{eq023}), finally obtaining a damping rate scaling as $k_B T\epsilon^2$ as expected.} We conclude in section \ref{sec5}.

\section{Numerical experiment in the classical field model}
\label{sec1}

Experimental studies of sound in two-dimensional superfluid ultracold atomic gases have so far concentrated on the hydrodynamic \cite{HM,PCZH} regime or, in the collisionless regime, on the $\epsilon=k_B T/mc^2\geq 1$ \cite{JDJB,Salasnich,Stringari} temperature interval. To verify that in dimension $d=2$, the low-temperature limit (\ref{eq020}) is not sufficient to render accurate the exponential decay predicted by Fermi's golden rule, we therefore carry out in this section a numerical study of phonon damping, at all orders in coupling $H_3$, in the classical field model. We first recall the quantum hydrodynamics Hamiltonian of Landau and Khalatnikov. We then explain how to numerically integrate its classical field version and introduce a correlation function that is more convenient to calculate than signal (\ref{eq012}), but almost equivalent. We then analyze the results obtained for different interaction strengths. 

\subsection{Quantum hydrodynamics Hamiltonian}
\label{sec1.1}

Quantum hydrodynamics is an effective low-energy theory that provides accurate predictions to leading order in temperature, even for a strongly interacting fluid \cite{LK}. The fluid is decomposed into small, but still mesoscopic elements of size $b$ ($b\gg\xi=\hbar/mc$). Each element, with center $\rr$, is (i) sufficiently large that it has a large number of particles ($\rho b^d\gg 1$), allowing it to be assigned a phase operator $\hat{\phi}(\rr)$ canonically conjugated to its particle number operator $\hat{\rho}(\rr)b^d$, $[\hat{\rho}(\rr)b^d,\hat{\phi}({\yct\rr'})]=\ii\delta_{\rr,\rr'},$ and (ii) small enough to be considered both as homogeneous ($b\ll\lambda_{\rm th}$ where $\lambda_{\rm th}=2\pi\hbar c/k_B T$ is the de Broglie wavelength of a thermal phonon in the fluid) and in its ground state ($k_B T\ll\veps_{\rm exc}^{\rm min}$ where $\veps_{\rm exc}^{\rm min}=\hbar c (2\pi/b)$ is the minimum phonon excitation energy in the small element). \footnote{The ultraviolet cutoff to come corresponds more or less to the choice $\pi/b\approx \eta k_B T/\hbar c$. Even so, $b$ remains well below the coherence length of the fluid at sufficiently low temperatures. This is obvious in $d=3$ because of the presence of condensate. It is also true in $d=2$ where the first-order coherence function $g_1(\rr)$ decreases at large distances as $\exp[mk_B T\ln(\xi/r)/(2\pi\hbar^2\rho_s)]$ where $\rho_s\simeq \rho$ is the superfluid density \cite{Bere,Nel,SJ}. This makes the assumption $\hat{\vv}=(\hbar/m)\mathbf{grad}\,\hat{\phi}$ used in reference \cite{LK} and in equation (\ref{eq050}) plausible.} The Hamiltonian is then the sum of the kinetic energies of the center of mass and the internal energy of each small element: 
\begin{equation}\label{eq050}
H_{\rm hydro}=\sum_\mathbf{r}\frac{1}{2} m \hat{\vv}(\rr)\cdot\hat{\rho}(\rr) b^d\hat{\vv}(\rr)+b^d e_0(\hat{\rho}(\rr))
\end{equation} where $\hat{\vv}=(\hbar/m)\mathbf{grad}\,\hat{\phi}$ is the fluid velocity field and $e_0(\rho)$ is the bulk energy density in the ground state at density $\rho$. At low temperatures, the density field $\hat{\rho}(\rr)$ fluctuates little and the phase field $\hat{\phi}(\rr)$ varies slowly; we separate them into their Fourier components of zero wave vector $\hat{\rho}_0=\hat{N}/L^d\simeq \rho$ and $\hat{\phi}_0$, and into spatial fluctuations $\delta\hat{\rho}(\rr)$ and $\delta\hat{\phi}(\rr)$ admitting modal expansions {\yc\cite{broui}}
\begin{equation}\label{eq052}
\delta\hat{\rho}(\rr)=\frac{1}{L^{d/2}} \sum_{\kk\neq\zero} \left(\frac{\hbar\rho k}{2mc}\right)^{1/2}(\hat{b}_\kk+\hat{b}_{-\kk}^\dagger)\,\eee^{\ii\kk\cdot\rr} \quad ; \quad \delta\hat{\phi}(\rr)=\frac{-\ii}{L^{d/2}}  \sum_{\kk\neq\zero} \left(\frac{mc}{2\hbar\rho k}\right)^{1/2} (\hat{b}_\kk-\hat{b}_{-\kk}^\dagger)\,\eee^{\ii\kk\cdot\rr}
\end{equation} where $\hat{b}_\kk^\dagger$ and $\hat{b}_\kk$ are the creation and annihilation operators of a phonon of wave vector $\kk$ in the quantization box $[0,L]^d$, $c$ is the speed of sound at zero temperature, given by $mc^2=\rho(\dd\mu/\dd\rho)$ with $\mu=\mathrm{d}e_0/\dd\rho$ the chemical potential of the fluid in the ground state. As in any effective theory, we need to put an ultraviolet cutoff in the space of wave vectors, which we take here to be isotropic 
\begin{equation}\label{eq103}
\hbar c k < \eta k_B T
\end{equation} and to which the sums (\ref{eq052}) are restricted; we'll let the temperature tend towards zero as in (\ref{eq020}) and then $\eta\to+\infty$ (in the quantum case) without encountering divergences if all goes well. It remains to expand $H_{\rm hydro}$ into powers of $\delta\hat{\rho}$ and $\mathbf{grad}\,\delta\hat{\phi}$. Zero order provides a constant of no interest. Order one is zero by construction.  Order two, diagonalized by the transformation (\ref{eq052}), gives back the Hamiltonian $H_2$ of equation (\ref{eq002}) {\yct with} a linear dispersion relation, the curvature in equation (\ref{eq003}) having to be put in by hand, see references \cite{LK,Annalen} and our footnote \ref{note1}. The third and fourth orders, giving the coupling between phonons in equation (\ref{eq001}), are written in representation $\rr$: 
\begin{equation}\label{eq104}
H_3=b^d\sum_\mathbf{r}\frac{1}{2} m\hat{\vv}(\rr)\cdot\delta\hat{\rho}(\rr)\hat{\vv}(\rr)+\frac{1}{6}\frac{\dd^2\mu}{\dd\rho^2}\delta\hat{\rho}(\rr)^3 \quad ; \quad H_4=b^d\sum_\rr\frac{1}{24} \frac{\dd^3\mu}{\dd\rho^{3}}\delta\hat{\rho}(\rr)^4
\end{equation} knowing that $\mathbf{grad}\,\hat{\phi}=\mathbf{grad}\,\delta\hat{\phi}$. They involve the dimensionless coupling constants 
\begin{equation}\label{eq032}
\Lambda=\rho\frac{\dd^2\mu}{\dd\rho^2}/(3\frac{\dd\mu}{\dd\rho})=\rho^2\frac{\dd^2\mu}{\dd\rho^2}/(3 mc^2)\quad\mbox{and}\quad \Lambda_*=\rho^3\frac{\dd^3\mu}{\dd\rho^3}/(mc^2)
\end{equation} To reveal the physical processes contained in $H_3$ and already mentioned in section \ref{sec0}, we insert in (\ref{eq104}) the modal expansions (\ref{eq052}) and collect together the terms of the same type, Beliaev, Landau and non-resonant respectively: 
\begin{equation}\label{eq100}
H_3 = {\ycd H_3^{(+)} + H_3^{(-)}} + H_3^{\mbox{\yct\scriptsize (non res.)}}
\end{equation} with 
\begin{equation}\label{eq101}
H_3^{(-)}= (H_3^{(+)})^\dagger = \frac{1}{2 L^{d/2}}\sum_{\kk,\kk'} \mA(\kk,\kk';\kk+\kk') \hat{b}_{\kk+\kk'}^\dagger \hat{b}_\mathbf{k}\hat{b}_{\kk'} \quad ; \quad H_3^{\mbox{\yct\scriptsize (non res.)}} = \frac{1}{6 L^{d/2}} \sum_{\kk,\kk'} \mA(\kk,\kk';-(\kk+\kk')) (\hat{b}_{\kk}\hat{b}_{\kk'}\hat{b}_{-(\kk+\kk')}+\mbox{h.c.})
\end{equation} and {\yct the} coupling amplitude\footnote{By choice of notation, our amplitude $\mA$ differs by a factor of two from that of reference \cite{Annalen} {\yct(more precisely by a factor $2mc^2/\rho^{1/2}$)}.} 
\begin{equation}\label{eq102}
\mA(\kk,\kk';\kk'')=mc^2\left(\frac{\hbar^3kk'k''}{8\rho m^3c^3}\right)^{1/2}\left(3\Lambda+\frac{\kk\cdot\kk'}{k k'}+\frac{\kk\cdot\kk''}{k k''}+\frac{\kk'\cdot\kk''}{k' k''}\right)
\end{equation} Of course, all wave vectors $\kk$, $\kk'$, $\pm(\kk+\kk')$ appearing in the Hamiltonian are non-zero and subject to condition (\ref{eq103}). Writing $H_4$ in representation $\hat{b}$ and $\hat{b}^\dagger$ is of little interest here, and we dispense with it.

\subsection{Classical field model}
\label{sec1.2}

The model is built by replacing operators $\hat{b}_\kk$ and $\hat{b}_\kk^\dagger$ in section \ref{sec1.1} with complex amplitudes $b_\kk$ and $b_\kk^*$. This only makes sense for highly populated phonon modes, which imposes a choice of cutoff $\eta\lesssim 1$ in equation (\ref{eq103}); in the simulations, we'll take $\eta=1$. At thermal equilibrium, mode occupation numbers are no longer given by Bose's law, but by equipartition $\bar{n}_\kk^{\rm clas}=k_B T/\veps_\kk$. 

Let's restrict our numerical study to the experimentally relevant case of a {\yct gas of spinless bosons} in the quasi-2D regime, where the chemical potential $\mu$ in the $xOy$ plane is less than, but not very small compared to $\hbar\omega_\perp$, the quantum of excitation of the motion along $Oz$; in this case, the equation of state takes the simple form $\mu=\rho g$ with $g=(8\pi)^{1/2}\hbar^2a_{\rm 3D}/(ma_\perp)$, where $a_{\rm 3D}$ is the 3D scattering length and $a_\perp=(\hbar/m\omega_\perp)^{1/2}$ the size of the harmonic oscillator along $Oz$ \cite{CD,HD}, so that $\Lambda=\Lambda_*=0$ in equation (\ref{eq032}) and $H_4\equiv 0$ in equation {\yc(\ref{eq104})}. \footnote{The Popov equation of state of a strictly two-dimensional Bose gas in the weakly interacting regime corresponds to a chemical potential-dependent pseudo-coupling constant, $g(\mu)=4\pi\hbar^2/[m\ln(\veps_0/\mu)]$ where $\veps_0=4\hbar^2/[ma_{\rm 2D}^2\exp(2C+1)]$, $a_{\rm 2D}$ is the 2D scattering length and $C=0.577\ldots$ is the Euler constant \cite{Popov,Morathese}. In its range of validity \cite{MC2}, it leads to non-zero but small values of $\Lambda$ and $\Lambda_*$, for example {\ycd $\Lambda=mg(\mu)/(12\pi\hbar^2)/[1-mg(\mu)/(4\pi\hbar^2)]^2\simeq mg(\mu)/(12\pi\hbar^2) \ll 1$. The quasi- and strictly 2D regimes are connected by noting that $g=g(\mu=0.105\,965\ldots\hbar\omega_\perp)$ in view of the expression of $a_{\rm 2D}$ in terms of $a_{\rm 3D}$ and $a_\perp$ given in references \cite{Petrov,Olshanii,Ludovic}}.}

The numerical simulation method is described in detail in reference \cite{TheseAlan}. To sample the initial state, $b_\kk$ are drawn according to the (Gaussian) thermal distribution for the $H_2$ Hamiltonian, then the resulting field is evolved with the full Hamiltonian $H$ to reach true thermal equilibrium {\yc by ergodicity}. The time evolution on each $dt$ step is split into the evolution due to $H_2$ during $dt/2$, the evolution due to $H_3$ alone during $dt$ and then again the evolution due to $H_2$ during $dt/2$. The evolution due to $H_2$ is calculated exactly in $\kk$ space, while the evolution due to $H_3$ is calculated in $\rr$\footnote{Since the equations of motion deriving from $H_3$ are local in $\rr$ space (or quasi-local since they involve spatial gradients), the complexity on a step $dt$ is of degree one in the number of points $\mathcal{N}$ on the numerical grid; in $\kk$ space, they are strongly non-local (each amplitude $b_\kk$ is roughly coupled to all the others) resulting in catastrophic complexity $\propto \mathcal{N}^2$.} space by the fourth-order Runge-Kutta method \cite{NumRec}, remembering to filter the field in $\kk$ space as in equation (\ref{eq103}) after each action of $H_3$ in $\rr$ space to eliminate {\yct the cut-off-violating} processes {\yct that $H_3$} has induced.\footnote{The change from $\rr$ to $\kk$ is performed by a fast Fourier transform, in which the wave vectors span the set $\mD\cap(2\pi/L)\mathbb{Z}^d$ with $\mD=[-k_{\rm max},k_{\rm max}[^d$ (each Cartesian component $k_\alpha$ of $\kk$ has a meaning modulo $2 k_{\rm max}$, so that $\mD$ can be regarded as the first Brillouin zone of the spatial lattice $(\pi/k_{\rm max})\mathbb{Z}^d$, which is precisely that spanned by $\rr$ {\yc in the simulations with} restriction to $[0,L[^d$). The naive choice $k_{\rm max}=\eta k_B T/\hbar c$ associated with the cutoff (\ref{eq103}) is not suitable: the processes of $H_3$ bringing phonons out of the corresponding first Brillouin zone are {\yc then} folded back into this zone (Umklapp processes) and can no longer be filtered. At the very least, $k_{\rm max}=2\eta k_B T/\hbar c$ must be used: filtering forces us to double the number of points per dimension.}  The $dt$ step must be small enough for the total energy on a single realization to be conserved to better than one percent of the typical $\langle |H_3|\rangle$ interaction energy, where $\langle\ldots\rangle$ is the average over a large number of realizations, itself a very small fraction of the total energy. 

 It would be numerically inefficient to determine phonon damping by simulating {\yct a} Bragg excitation (\ref{eq010}), as this would require a large number of independent simulations for each new value of wave vector $\qq$, while ensuring that amplitude $\alpha$ is small enough to be in the linear response regime.\footnote{The classical field version $\langle\partial b_\qq(t)/\partial b_\qq(0)\rangle_{\rm th}$ of the correlation function $\langle[\hat{b}_\qq(t),\hat{b}_\qq^\dagger(0)]\rangle_{\rm th}$, which involves a numerical derivative in the particular mode $\qq$, faces the same problems.} We prefer to use the following correlation function, allowing all possible values of $\qq$ to be processed in parallel by simple averaging over the same set of independent realizations: \footnote{The obtention of the {\ycd right-hand} side of equation (\ref{eq110}) {\yc is done by means of the exact relation (31.24) of reference \cite{FW} but is here} approximated as we replaced $\langle\hat{b}_\qq(0)\hat{b}_\qq^\dagger(0)\rangle_{\rm th}$ in the denominator by $1+\bar{n}_{\qq}$, where $\bar{n}_{\qq}$ is the Bose law at energy $\veps_{\qq}$; {\yc this is however} an excellent approximation for our simulations.} 
\begin{equation}\label{eq110}
s_*(t)\equiv \frac{\eee^{\ii\veps_\mathbf{q}t/\hbar}\langle\hat{b}_\qq(t)\hat{b}_\qq^\dagger(0)\rangle_{\rm th}}{\langle\hat{b}_\qq(0)\hat{b}_\qq^\dagger(0)\rangle_{\rm th}}\stackrel{t>0}{\simeq} \lim_{\delta\to 0^+} \int_{-\infty}^{+\infty} 
\frac{\dd\veps}{-2\ii\pi} \eee^{-\ii(\veps-\veps_\qq)t/\hbar} \frac{\mathcal{G}(\qq,\veps+\ii\delta)+\left[\mathcal{G}(\qq,\veps+\ii\delta)-\mathcal{G}(\qq,\veps-\ii\delta)\right]/[\exp(\veps/k_B T)-1]}{1+1/[\exp(\veps_\qq/k_B T)-1]}
\end{equation} {\yct In practice,} this signal is very close to that of equation (\ref{eq012}), which in simulations allows us to identify it with $s(t)$, as we'll do in the figure caption:
\begin{equation}\label{eq111}
s_*(t) \simeq s(t)
\end{equation} Indeed, even for the strongest interaction considered in our simulations, we have, for any value of energy $\veps$, the property $|\im\Sigma(\qq,\veps+\textrm{i}0^+)|\ll \veps_\qq$, so that the difference between the bracketed Green's functions $\mG(\qq,\veps\pm\ii\delta)$ in equation (\ref{eq110}), which can only result from a non-zero imaginary part of the self-energy in the limit $\delta\to 0^+$, is very small except in the width of \g{Lorentzian} $\mathcal{G}(\qq,\veps+\textrm{i}0^+)$ where $|\veps-\veps_\qq|\lesssim |\im\Sigma(\qq,\veps+\textrm{i}0^+)|\ll \veps_\qq, k_B T$. It is therefore an excellent approximation to replace $\veps$ by $\veps_\qq$ in Bose's law in the numerator of the integrand in (\ref{eq110}). By analyticity in the lower complex half-plane, the term proportional to $\mathcal{G}(\qq,z-\ii\delta){\ycd\,\exp(-\ii zt/\hbar)}$ then gives a zero contribution to the integrand {\ycd for $t>0$} (this can be seen by closing the integration path with a semicircle at infinity and using Cauchy's theorem) and we end up with (\ref{eq111}). The same conclusion applies to the classical field model, where the two exponential functions in (\ref{eq110}) are linearized.

 
\subsection{Numerical results}
\label{sec1.3}
% Figure environment removed
 % Figure environment removed
 In Figure \ref{fig1}, we plot the inverse of the decay time at $1/\eee$ of the signal as a function of the reduced wavenumber $\qb=\hbar cq/k_B T$ of the mode under consideration, for four values of the interaction strength $1/(\rho\xi^2)$ and for one or two values of $\epsilon=k_B T/mc^2$, see the disks. The unit of time is $2\Gamma_u^{-1}$, where the constant $\Gamma_u$ appears as a prefactor in the damping rate of Fermi's golden rule at the leading order in $\epsilon$, here for the classical field: \footnote{In expression (\ref{eq033}) for the quantum field, we replace $\bar{n}_k^{\rm lin}$ by $1/\kb$, neglect the term $1/2$ in front of $\bar{n}_k^{\rm lin}$ in the first integral and put the cutoff $\kb<\eta-\qb$ in the second.} 
\begin{equation}\label{eq113}
\boxed{\Gamma_q^{\mbox{\scriptsize clas}}=\Gamma_u \qb(\eta-\qb/2) \quad \mbox{with}\quad \Gamma_u = \frac{9k_B T\epsilon^2(1+\Lambda)^2}{4\pi\hbar(3\gamma)^{1/2}\rho\xi^2}}
\end{equation} As can be seen in panels (a) and (d) of Figure \ref{fig1}, the results normalized in this way do not depend significantly on $\epsilon$ at $\qb$ fixed and the $\epsilon\to 0$ limit appears to be reached; in the simulations, the inverse of the decay time at $1/\eee$ therefore varies indeed as $k_B T\epsilon^2$ at low temperature, as predicted by the golden rule and more generally by estimate (\ref{eq028}) at all orders in $H_3$. On the other hand, the $\qb$ dependence of the damping time is very poorly described by the golden rule, dashed line in Figure \ref{fig1}: as predicted by the general reasoning of section \ref{sec0}, the $\epsilon\to 0$ limit is therefore not sufficient in 2D to bring the phonon gas into a weak coupling regime. What section \ref{sec0} didn't say, however, is that the golden rule remains insufficient even if the underlying superfluid enters the weakly interacting regime $(\rho\xi^2)^{-1}\to 0$, see panel (d) where a perturbative calculation at the leading order in $H_3$ should a priori be legitimate; the behavior at weak $\qb$ predicted by (\ref{eq113}) is in fact particularly bad there. 

To make the inadequacy of the golden rule even more glaring, we plot in Figure \ref{fig2}, as a function of time, the deviation of the numerical signal from the pole approximation (\ref{eq014}) limited to second order at $H_3$ {\yc but taken to all orders in $\epsilon$} [see equation (\ref{eq124})] and transformed for the classical field (if we were at the leading order in $\epsilon$, {\yc it would simply be} the deviation from the exponential $\exp(-\Gamma_q^{\rm clas}t/2)$ of the golden rule), for the same interaction strengths as before but for a fixed value of the reduced wave number, $\qb=1/4$, large enough to make it easy to reach the thermodynamic limit in simulations. \footnote{The smaller $\qb$ is, the lower the damping rate of the mode and the greater the granularity of the energy levels of the phonon gas in the quantization box, and the more we need to increase the size of the system to reach the thermodynamic limit.} Here too, the $\epsilon\to 0$ limit is reached in simulations (when time $t$ is scaled as in (\ref{eq022}) with $\nu=2d-2=2$, which is done here by expressing it in units of $2/\Gamma_q^{\rm clas}$), the {\yc expected} deviation from the golden rule is indeed present and persists even in the weakly interacting limit $(\rho\xi^2)^{-1}\to 0$, at least on the real part.\footnote{If we move from panel (c) to panel (d) in Figure \ref{fig2}, the {\yct interaction} strength is divided by $\simeq 3$, the imaginary part is divided by $\simeq 2$ in maximum absolute value so may well tend towards zero, while the real part only decreases by a factor of $\simeq 1.3$.}

\section{Study at leading order in phonon-phonon coupling}
\label{sec2}

The classical field simulations of section \ref{sec1} established the inadequacy of Fermi's golden rule in describing phonon damping in dimension $d=2$; we have not shown this here, but in addition to disagreement on the decay time at $1/\eee$, they reveal, in logarithmic scale, a non-exponential decay of the {\yct signal} $s(t)$ \cite{TheseAlan}. In this section, we incriminate the approximation made in step (i) of section \ref{sec0}, and renounce it by retaining all the energy dependence of the self-energy in the integral expression (\ref{eq012}) of $s(t)$, continuing however to calculate $\Sigma_\qq(\zeta)$ at leading order in the phonon-phonon coupling and in $\epsilon$ (as in step (ii) of section \ref{sec0}). After a diagrammatic calculation of $\Sigma^{(2)}$, we obtain some analytical predictions put into a universal form and compare them with simulations.

\subsection{Calculation of the self-energy}
\label{sec2.1}

In the Hamiltonian $H$ of equation (\ref{eq001}), the cubic coupling {\yct $H_3$} seems to dominate over the quartic coupling {\yct $H_4$}, since it appears at order $\delta\hat{\rho}^3$ rather than $\delta\hat{\rho}^4$ in the expansion of the hydrodynamic Hamiltonian, see section \ref{sec1.1}. However, as $H_3$ does not conserve the phonon number, it must be treated  at least at second order of perturbation theory in the calculation of $\Sigma_\qq(\zeta)$, whereas $H_4$ gives a non-zero first order contribution, which is at first sight as large as that of $H_3$.\footnote{The contribution of $H_3$ {\yc to $\Sigma_\qq(\ii 0^+)$} involves the ratio $\langle\ |\mH_3|\ \rangle^2/\Delta E\approx \delta\hat{\rho}^{2\times 3-2}$, that of $H_4$ involves the matrix element $\langle\ |\mH_4|\ \rangle\approx \delta\hat{\rho}^4$, where we use the notations of equation (\ref{eq024}) and the fact that the denominator $\Delta E$, made up of energy differences of the quadratic Hamiltonian $H_2$, is $\approx\delta\hat{\rho}^2$.} The full calculation is required.

At first order in perturbation theory, there's no need to use Feynman {\yct diagrams}, as the contribution to $\Sigma_\qq(\zeta)$ is independent of the $\zeta$ variable and can be obtained by the following simple procedure, justifiable by the quantum master equation method of reference \cite{SCE}. We write the perturbation operator, here $H_4$, in representation $\hat{b},\hat{b}^\dagger$ and in normal order, with the help of bosonic commutation relations (all $\hat{b}^\dagger$ put on the left, all $\hat{b}$ put on the right). Terms that do not preserve phonon number do not contribute. The terms $\hat{b}^\dagger\hat{b}$, {\yc being quadratic}, are absorbed into a redefinition of the bare $\veps_\kk$ eigenenergies of the modes (as was expected from footnote \ref{note0}). Finally, we isolate in the {\yc remaining} \g{true} quartic coupling terms the contributions $\hat{b}^\dagger_\qq\hat{b}^\dagger_\kk\hat{b}_\kk\hat{b}_\qq$, whose thermal averaging over the modes $\kk\neq\qq$ gives the $\qq$-mode energy shift. All this is summarized by the compact expression 
\begin{equation}\label{eq120}
\Sigma_\qq^{(H_4)}(\zeta)=\int\frac{\dd^dk}{(2\pi)^d} \langle\qq,\kk|\mathcal{H}_4^{\rm true}
|\qq,\kk\rangle \bar{n}_k = \frac{\hbar^2q\Lambda_*}{4m\rho}\int\frac{\dd^dk}{(2\pi)^d} k\bar{n}_k \underset{\epsilon\to 0}{\approx} \frac{\bar{q}k_B T\epsilon^{d+1}}{\rho\xi^d}
\end{equation} where $\mH_4$ is the {\yct volumic} Hamiltonian such that $H_4=\mH_4/L^{d}$, the coupling constant $\Lambda_*$ is that of equation (\ref{eq032}), we have deduced the matrix element in the two-phonon state from expression (\ref{eq104}) of $H_4$ and expression (\ref{eq052}) of $\delta\hat{\rho}(\rr)$ and we have obtained the scaling behavior in the limit (\ref{eq020}) in the {\yct right-hand} side of equation (\ref{eq120}) as in section \ref{sec0}. As we see from the contribution (\ref{eq028}) of $H_3$ written for $n=1$, the contribution of $H_4$ in dimension $d=2$ (but not in dimension three) is subleading and can therefore be neglected when $\epsilon\to 0$. This is what we'll do from now on.
% Figure environment removed

At second order in $H_3$, it's best to use the method in reference \cite{FW}.\footnote{Since our phonon states already have the bosonic symmetry, we can dispense with the direct interaction and exchange lines introduced in \cite{FW}.} This method {\yc provides} a diagrammatic expansion of the self-energy $\Sigma(\qq,z)$ for particular values of the complex energy variable, $z=\Omega\in 2\ii\pi k_B T \mathbb{Z}$, where $\Omega$ is called Matsubara energy,\footnote{As a slight departure from the usual notation \cite{FW}, our Matsubara variables $\Omega$, $\omega$, etc., are not real frequencies but pure imaginary energies, i.e. we incorporate an additional factor $\ii\hbar$.} which is then completed by an analytic continuation to $\mathbb{C}\setminus\mathbb{R}$ \cite{FW}. Here, as in Figure \ref{fig6}, each diagram must have two vertices (we're at second order), where three phonon lines meet ($H_3$ is cubic in $\hat{b},\hat{b}^\dagger$). The only possible topology is one-loop and, since two loops deduced from each other by rotation of angle $\pi$ around their axis must be considered as identical, the four possible orientations of the two internal lines lead to only three distinct diagrams, represented in the order Beliaev, Landau and non-resonant in Figure \ref{fig6} (using the typology of equation (\ref{eq100})). Given the conservation of Matsubara energy and momentum at each vertex (the sum of incoming quantities must equal that of outgoing quantities), there remains a single independent Matsubara energy $\omega$ and wave vector $\kk$, on which to sum. Feynman's rules {\yc of} section 25 of reference \cite{FW} then lead for our phonon gas of zero chemical potential to 
\begin{equation}\label{eq121}
\Sigma^{(2)}(\qq,\Omega) = -k_B T \sum_{\omega\in 2\ii\pi k_B T\mathbb{Z}}\int\frac{\dd^dk}{(2\pi)^d} \Bigg\{\frac{(1/2)|\langle\kk,\qq-\kk|\mathcal{H}_3|\qq\rangle|^2}{(\omega-\veps_\kk)(\Omega-\omega-\veps_{\qq-\kk})}+\frac{|\langle\kk+\qq|\mathcal{H}_3|\kk,\qq\rangle|^2}{(\omega-\veps_\kk)(\Omega+\omega-\veps_{\kk+\qq})}+\frac{(1/2)|\langle 0|\mathcal{H}_3|\qq,\kk,-(\kk+\qq)\rangle|^2}{(\omega-\veps_\kk)(-\Omega-\omega-\veps_{-(\kk+\qq)})}
\Bigg\}
\end{equation} 
{\yc in the thermodynamic limit with $\mathcal{H}_3=H_3 L^{d/2}$, $|0\rangle$ the phonon vacuum and}, for each contribution, a $\kk$ integration domain ensuring that the wavevectors $\kk$, $\qq-\kk$, etc., satisfy the cutoff (\ref{eq103}). Also included are symmetry factors $1/2$ when the two branches of the loop have the same orientation and are therefore physically identical. Matsubara sums are calculated by means of the general expression \footnote{\label{notedemo} This is easily demonstrated by partial fraction decomposition of the summand with respect to variable $n$ and using the identity $\lim_{N\to +\infty} \sum_{n=-N}^{N} 1/(2\ii\pi n-\vepsb)=-(\bar{n}_\veps+1/2)\ \forall \veps\in{\yc\mathbb{C}\setminus 2\ii\pi\mathbb{Z}}$ resulting from the properties of the digamma function $\psi(z)$ {\yc such as the reflection formula}.} 
\begin{equation}\label{eq122}
\sum_{n\in\mathbb{Z}} \frac{1}{(2\ii\pi n-\vepsb_1)(2\ii\pi n-\vepsb_2)} = -\frac{\bar{n}_{\veps_1}-\bar{n}_{\veps_2}}{\vepsb_1-\vepsb_2}
\end{equation} where the real {\ycd or even complex} energies $\veps_j$ are two by two distinct, and we set $\vepsb=\veps/k_B T$ and $\bar{n}_\veps=1/(\exp\vepsb-1)$. We then use the fact that $\Omega/k_B T\in 2\ii\pi\mathbb{Z}$ to make this variable disappear from Bose's law, and the relation $\bar{n}_{-\veps}=-(1+\bar{n}_\veps)$ to bring us back to true occupation numbers (at positive energy). So, for example: 
\begin{equation}\label{eq123}
\bar{n}_{\Omega-\veps_{\qq-\kk}}=\bar{n}_{-\veps_{\qq-\kk}}=-(1+\bar{n}_{\qq-\kk})
\end{equation} The integrand becomes a rational function of the discrete variable $\Omega$, and its analytic continuation to $\mathbb{C}\setminus\mathbb{R}$ is achieved by the simple substitution $\Omega\to z$. We get 
\begin{multline}
\label{eq124}
\Sigma_\qq^{(2)}(\zeta)=\int\frac{\dd^dk}{(2\pi)^d}\Bigg[\frac{(1/2)|\langle\kk,\qq-\kk|\mathcal{H}_3|\qq\rangle|^2}{\zeta+\veps_\qq-(\veps_\kk+\veps_{\qq-\kk})}(\bar{n}_\kk+\bar{n}_{\qq-\kk}+1)+\frac{|\langle\kk+\qq|\mathcal{H}_3|\kk,\qq\rangle|^2}{\zeta+\veps_\qq+\veps_\kk-\veps_{\qq+\kk}}(\bar{n}_\kk-\bar{n}_{\qq+\kk})\\
-\frac{(1/2)|\langle 0|\mathcal{H}_3|\qq,\kk,-(\kk+\qq)\rangle|^2}{\zeta+\veps_\qq+\veps_{-(\kk+\qq)}+\veps_\kk}(\bar{n}_\kk+\bar{n}_{-(\qq+\kk)}+1)\Bigg\}
\end{multline}
 It remains to take the limit $\epsilon\to 0$ at $\qb$ and $\zetat$ fixed as in equations (\ref{eq020},\ref{eq021},\ref{eq022}) with, remember, exponents $\nu=\sigma=2$ in dimension $d=2$. Through the \g{small-denominator} effect described in section \ref{sec0}, each contribution to the integral (\ref{eq124}) is dominated by configurations where the three wave vectors in the energy denominator are quasi-colinear and of the same direction (within an angle $O(\epsilon)$). In the first contribution, it suffices to use expansion (\ref{eq025}) with $\kk'=\qq-\kk$ (having $\kk$ and $\kk'$ quasicollinear and of the same direction then implies $k<q$); in the second, we use (\ref{eq025}) with $\kk'=\qq$ which imposes nothing on $k$; in the third, we are dealing with a sum with positive coefficients of three phonon energies, which can never be $O(k_B T\epsilon^2$) (it is always $\approx k_B T\epsilon^0$), so that the corresponding contribution is subleading and negligible at low temperatures. \footnote{After these expansions, the angular integral is Lorentzian and can therefore be performed very well using Cauchy's formula, by closing the integration path with an infinite semicircle in the complex plane, after performing the change of variable $\theta=\epsilon\tilde{\theta}$ {\yc on the angle $\theta$ between $\qq$ and $\kk$} and sending the integration bounds $\pm\pi/\epsilon$ over $\tilde{\theta}$ to infinity. {\ycd In the first contribution, the limiting behavior $\theta'\sim-k\theta/(q-k)$ of the angle $\theta'$ between $\qq$ and $\kk'$ was used beforehand.}} Respecting the order of the contributions in (\ref{eq124}), and restricting ourselves to the strict upper half {\yc of the complex} plane as in expression (\ref{eq012}) of the signal, we get:\footnote{\label{notegen} The value of $\tilde{\Sigma}_{\qb}^{(2,2)}(\zetat)$ in the strict lower half {\yc of the complex} plane can be deduced from the general relation valid for all orders in phonon coupling, ${\yc\Sigma_\qq(\zeta)^*}=\Sigma_\qq(\zeta^*)$.} 
\begin{equation}\label{eq125}
\boxed{\tilde{\Sigma}_{\qb}^{(2,2)}(\zetat)\stackrel{\im\zetat>0}{=}\frac{9(1+\Lambda)^2(2\qb)^{1/2}}{64\ii\pi\rho\xi^2} \Bigg\{\int_0^{\qb} \dd\kb\, \frac{[\kb(\qb-\kb)]^{3/2}(1+\bar{n}_k^{\rm lin}+\bar{n}_{q-k}^{\rm lin})}{\left[(3\gamma/8)\qb\kb(\qb-\kb)+\tilde{\zeta}\right]^{1/2}} +2 \int_0^{\eta-\qb} \dd\bar{k}\frac{[\kb(\qb+\kb)]^{3/2}(\bar{n}^{\rm lin}_k-\bar{n}^{\rm lin}_{k+q})}{\left[(3\gamma/8)\qb\kb(\kb+\qb)-\tilde{\zeta}\right]^{1/2}}\Bigg\}}
\end{equation} where the occupation numbers marked by the superscript \g{lin} are those (\ref{eq029}) of the phonon modes for the linearized dispersion relation. In the left-hand side of (\ref{eq125}), the second integer $2$ in the exponent reminds us that $\Sigma^{(2)}$ is calculated to second order in $\epsilon$; the tildes on $\Sigma$ and $\zeta$ remind us that a factor $k_BT\epsilon^2$ has been taken out of these quantities as in (\ref{eq021},\ref{eq022}). Finally, despite the ever-present threat of quantum field fluctuations, we can take the limit of an infinite $\eta$ cutoff without triggering ultraviolet divergence, so without having to renormalize any quantities whatsoever. Then, if we make $\zetat$ tend towards zero by positive imaginary parts in (\ref{eq125}), we find equation (\ref{eq033}).

\subsection{Analytical studies: one-parameter universality}
\label{sec2.2}

Remarkably, it is possible, by a simple change of variable, to ensure that in the leading-order theory {\yc in $\epsilon$ and in the coupling} treated in this section \ref{sec2}, there remains only one parameter $u$ containing all physical parameters such as the interaction strength $1/\rho\xi^2$, the curvature parameter $\gamma$ and the coupling constant $\Lambda$, i.e.\footnote{For the same to apply in the classical field model, we would also have to incorporate the cut-off parameter $\eta$ into $u$, i.e. introduce the quantity $u_{\rm clas}=\eta u$. This is unnecessary here, as we have taken $\eta=1$ in the simulations. As a curiosity, let's point out, for the classical field and in the limit $\qb\ll 1$, that the signal $s^{(2,2)}(t)$ defined below ceases to start at value $1$ in $t=0$ (as it should physically) if $u_{\rm clas}<1$, which we link mathematically to the appearance of a pole of {\yct the} Green's function on the real axis. Consequently, it would be interesting to know whether all two-dimensional physical systems (even in very strong interaction, provided it's short-range) lead to $u>1$.} 
\begin{equation}\label{eq112}
u=\frac{\pi\rho\xi^2\gamma^{3/2}}{\sqrt{3}(1+\Lambda)^2}
\end{equation} In fact, the new definitions marked with a Czech accent, 
\begin{equation}\label{eq130}
\check{t}=\frac{3\gamma\qb}{8u}\tilde{t}=\frac{3\gamma\epsilon^2 c q t}{8u}\quad ; \quad \zetac=\frac{8u}{3\gamma\qb}\zetat=\frac{8u\zeta}{3\gamma\epsilon^2\hbar c q} 
\end{equation} lead to a new self-energy and a new signal expression, clearly involving only $u$ and the reduced variables $\qb$ and $\check{t}$: 
\begin{equation}\label{eq131}
\check{\Sigma}^{(2,2)}_{\qb}(\check{\zeta}) = \frac{8u}{3\gamma\qb} \tilde{\Sigma}^{(2,2)}_{\qb}(\zetat) \stackrel{\im\zetac>0}{=} \frac{1}{\ii} \int_0^{\qb} \frac{\dd\kb}{2\qb} \frac{[\kb(\qb-\kb)]^{3/2} (1+\bar{n}^{\rm lin}_k+\bar{n}^{\rm lin}_{q-k})}{[\kb(\qb-\kb)+\check{\zeta}/u]^{1/2}}+\frac{1}{\ii} \int_0^{+\infty} \frac{\dd\kb}{\qb}\frac{[\kb(\qb+\kb)]^{3/2} (\bar{n}^{\rm lin}_k-\bar{n}^{\rm lin}_{q+k})}{[\kb(\qb+\kb)-\check{\zeta}/u]^{1/2}}
\end{equation} and 
\begin{equation}\label{eq132}
s^{(2,2)}(t) = \int_{C_+} \frac{\dd\check{\zeta}}{2\ii\pi} \eee^{-\ii\check{\zeta}\check{t}} \check{\mathcal{G}}_{\qb}^{(2,2)}(\zetac)  \quad\mbox{with}\quad \check{\mathcal{G}}_{\qb}^{(2,2)}(\zetac)=\frac{1}{\check{\zeta}-\check{\Sigma}^{(2,2)}_{\qb}(\check{\zeta})}
\end{equation} In this case, we speak of one-parameter universality.
% Figure environment removed
 


Let's carry out the analytic continuation of function (\ref{eq131}) from the upper to the lower half-plane through its branch cut on the real axis.\footnote{There's a mathematical subtlety here that the reader will forgive us for. Implicitly, from expression (\ref{eq131}) for $\check{\Sigma}_{\qb}^{(2,2)}(\zetac)$, valid only for $\im\zetac>0$, we define a function in all $\mathbb{C}\setminus\mathbb{R}$ by applying (\ref{eq131}) as it is for $\im\zetac<0$; the branch cut $[-u\qb^2/4,+\infty[$ mentioned in the text and shown in Figure \ref{fig5}a corresponds to this function rather than to the true self-energy $\check{\Sigma}_{\qb}^{(2,2)}(\check{\zeta})$, whose branch cut is $\mathbb{R}$ in its entirety and which would therefore be less convenient to analytically continuate from the upper half-plane to the lower half-plane, for the same final result (we see that it's $\mathbb{R}$ in its entirety (i) by considering the energy denominator $\zeta+\Delta E$ in the Landau contribution of equation (\ref{eq124}) and checking that $\Delta E$ spans $\mathbb{R}$ when $\theta/\epsilon$ and $\kb$ span $\mathbb{R}^+$ in (\ref{eq025}) with $\kk'=\qq$, or (ii) by using the relation $\textrm{i}z^{1/2}=-\mathrm{signe}(\im z)(-z)^{1/2}$, $\forall z\in\mathbb{C}\setminus\mathbb{R}$, to make the factor $\ii$ disappear in (\ref{eq131}) and obtain a physically correct expression over all $\mathbb{C}\setminus\mathbb{R}$ satisfying $\Sigma_{\qq}(\zeta)^*=\Sigma_{\qq}(\zeta^*)$, whose branch cut is then shown to be $\mathbb{R}$).} This will enable us, as in reference \cite{CCTbordeaux}, to separate the signal into the contribution of a contour around a displaced branch cut (power-law decreasing contribution at long times) and the contribution of a pole in the analytically continuated Green's function (exponentially decreasing contribution in time), see our Figure \ref{fig5}b, which improves physical understanding. First, we need to identify the branch points (immutable singularities) of $\check{\Sigma}_{\qb}^{(2,2)}(\check{\zeta})$ on the real axis. Since the values of $\check{\zeta}$ that make the square root argument in the denominator vanish, span the interval $[-u\qb^2/4,0]$ when $\kb$ spans $[0,\qb]$ in the first contribution to (\ref{eq131}), and span $[0,+\infty[$ when $\kb$ spans $\mathbb{R}^+$ in the second, we might expect to have two branch points, $\check{\zeta}=-u\qb^2/4$ and $\check{\zeta}=0$. \footnote{For example, $\zetac\mapsto\int_a^b\mathrm{d}x/(\zetac-x)^{1/2}=2(\zetac-b)^{1/2}-2(\zetac-a)^{1/2}$ has the real numbers $a$ and $b$ as branch points.} In reality, by subtle compensation between the two contributions, only $\zetac=-u\qb^2/4$ remains; the branch cut $[-u\qb^2/4,+\infty[$ starts from this point, as shown in Figure \ref{fig5}a.\footnote{\label{note30} If we return to variable $\zetat$, the branch point is at $-3\gamma\qb^3/32$.}. The branch cut can then be rotated by an angle $\phi \in ]-\pi,-\pi/2[$ to bring it into the third quadrant, by analytic continuation of $\check{\Sigma}^{(2,2)}_{\qb}(\check{\zeta})$ from the upper half-plane to the lower half-plane, the new branch cut being the half-line $-u\qb^2/4+\exp(\ii\phi)\mathbb{R}^+$, see Figure \ref{fig5}b. As shown in \ref{app1} and agreeing that $z^{3/2}=z\sqrt{z}$, the function continuated in this way is written: 
\begin{equation}\label{eq140}
\check{\Sigma}^{(2,2)}_{\qb\downarrow}(\check{\zeta})= \frac{\eee^{3\ii\phi/2}}{\ii} \int_0^{+\infty} \frac{\dd\kb}{\qb} 
\frac{(\kb^2-\eee^{-\ii\phi}\qb^2/4)^{3/2}}{[\kb^2-\eee^{-\ii\phi}(\qb^2/4+\zetac/u)]^{1/2}} \Bigg[
\frac{1}{\exp(\eee^{\ii\phi/2}\kb-\qb/2)-1}-\frac{1}{\exp(\eee^{\ii\phi/2}\kb+\qb/2)-1} \Bigg]
\end{equation} In particular, it is analytic on a neighborhood of $\zetac=0$ (see \ref{app1}, which also gives its value and that of its derivative at this point). On the other hand, the corresponding Green's function $\check{\mG}_{\qb\downarrow}^{(2,2)}(\zetac)$ has a pole $\check{\zeta}_{\qb}$ in the fourth quadrant, as we have verified numerically.\footnote{If angle $\phi$ is too close to $-\pi/2$ (which we'll avoid in the following) and if $\qb$ is quite small, a second pole of imaginary part $<0$, of uncertain physical interpretation, may appear to the left of the displaced branch cut.} 

Rather than numerically evaluating the contour and pole contributions to the signal, let's now extract analytically from form (\ref{eq140}) the \g{very substance} of the theory at order two in $H_3$, noting for example that it is illusory to calculate the position of the $\zetac_{\qb}$ pole at all orders in $1/u$ since the self-energy is calculated here only at order one in $1/u$. On the other hand, limiting ourselves to order zero in $1/u$, which would mean taking the $u\to+\infty$ limit at $\qb$ fixed, would be too poor, as the signal would then be reduced to the exponential of Fermi's golden rule. As can be seen from (\ref{eq131}), the new self-energy depends on $u$ only through the ratio $\zetac/u$; so it tends towards its value at $\textrm{i}0^+$ when $u$ tends to $+\infty$ under the integral sign ($\zetac$ and $\check{t}$ being fixed in (\ref{eq132})). A more interesting result is obtained by taking the limit $u\to+\infty$ for 
\begin{equation}\label{eq150}
Q=u^{1/2}\bar{q}\quad\mbox{and}\quad \check{t}\quad\mbox{fixed}
\end{equation} so that the {\yct branch} point in Figure \ref{fig5} remains fixed instead of being rejected at infinity (which would effectively make the contour contribution disappear). We then find a correction of order $u^{-1/2}$ to Fermi's golden rule {\yc signal}; as $u^{-1}\ll u^{-1/2}$ {\yct one would expect} that we can keep this correction without having to calculate the self-energy at the next order (order four) in $H_3$, {\yc even if we'll see} in section \ref{sec4} that this is not the case. Let's report the technical details in \ref{app2} and give the result directly {\yc in quantum theory}: 
\begin{equation}\label{eq151}
s^{(2,2)}(t) \stackrel{Q\,\mbox{\scriptsize {\yct and}}\,\check{t}\,\mbox{\scriptsize fixed}}{\underset{u\to+\infty}{=}}\eee^{-\pi^2\check{t}/3} + \frac{S(\check{t})}{u^{1/2}}+O(1/u) \quad\mbox{with}\quad S(\check{t})=S_{\rm contour}(\check{t})+S_{\mbox{\scriptsize pole}}(\check{t})
\end{equation} The first contribution is the real-valued exponential of the golden rule in limit $\qb\to 0$. The second is divided into the contour and pole contributions established in \ref{app2}: 
\begin{equation}\label{eq152}
S_{\rm contour}(\check{t})= -\frac{\eee^{\textrm{i}Q^2\check{t}/4}}{\pi}\int_0^{\eee^{\ii(\phi+\pi)}(+\infty)} \mathrm{d}z \frac{\eee^{\textrm{i}z \check{t}}\int_0^{\pi/2} \dd\alpha \left(z\sin^2\alpha+\frac{Q^2}{4}\right)^{1/2}}{(\frac{Q^2}{4}+z-\ii\frac{\pi^2}{3})^2} \quad ; \quad S_{\mbox{\scriptsize pole}}(\check{t})=\eee^{-\pi^2\check{t}/3} (\delta \check{Z}-\textrm{i}\delta \check{\zeta}_{\qb}\check{t})
\end{equation} where angle $\phi$ is freely chosen in $]-\pi,-\pi/2[$ (result (\ref{eq152}) does not depend on it). In a power expansion of $u^{-1/2}$, $\delta\check{\zeta}_{\qb}$ is the first deviation of the pole from the pole approximation $-\ii\pi^2/3$ (multiplied by $u^{1/2}$) and $\delta\check{Z}$ is the first deviation from one of the residue of $\check{\mG}_{\qb\downarrow}^{(2,2)}(\zetac)$ (also multiplied by $u^{1/2}$): \footnote{As signal (\ref{eq012}) starts exactly at one, $S(0)=0$ so the contour contribution for $\check{t}=0$ must exactly cancel out that of the pole residue's deviation from one. Verification is non-trivial. The main idea is to reduce $S_{\rm contour}(0)$ to a simple integral by inverting the integration on $z$ and $\alpha$ in (\ref{eq152}) (a primitive function of $(z+a)^{1/2}/(z+b)^2$ is known, where $a$ and $b$ are two constants). {\yc  Moreover, we note that $\re\delta\check{Z}>0$ for all $Q$: the first correction in $u^{-1/2}$ makes the residue $\check{Z}$ of modulus $>1$ and therefore the pole $\check{\zeta}_{\qb}$ of spectral weight $>1$.}}
\begin{equation}\label{eq154}
\delta\check{\zeta}_{\qb} = \int_0^{+\infty} \mathrm{d}K \left[1-\frac{(K^2+Q^2/4)^{1/2}}{(K^2-\ii\frac{\pi^2}{3}+\frac{Q^2}{4})^{1/2}}\right]\quad;\quad 
\delta \check{Z}=\frac{1}{2} \int_0^{+\infty} \mathrm{d}K \frac{(K^2+Q^2/4)^{1/2}}{(K^2-\ii\frac{\pi^2}{3}+\frac{Q^2}{4})^{3/2}} 
\end{equation} In Figure \ref{fig4}, we show function $S(\check{t})$ of equation (\ref{eq151}) for $Q=1$, together with predictions drawn numerically from $\check{\Sigma}_{\qb}^{(2,2)}{\yc(\check{\zeta})}$ for finite values of $u$. It can be seen that $u=20$ has already reached the asymptotic regime, so the same is expected for the parameters of the classical field simulations in Fig.~\ref{fig2}d, which correspond to $u\simeq 18.14$ and $Q\simeq 1.06$. So function $S(\check{t})$ is not just a mathematical curiosity, and its predictions are not without practical interest. Let's keep two of them, due to the contour: the behavior at short times is square-root, which accounts for the rapid rise of the signal in Figure \ref{fig4}, and the behavior at long times is sinusoidal damped as the inverse of time, its continuous connection to that at short times explaining the sign changes on this figure,\footnote{If $\check{t}\to +\infty$, the integral giving $S_{\rm contour}(\check{t})$ is dominated by values of $z$ close to zero and $z$ can be replaced by $0$ in the integrand, everywhere except in $\exp(\textrm{i}z\check{t})$. If $\check{t}\to 0^+$, the integral giving $\exp(\textrm{i}Q^2\check{t}/4)S_{\rm contour}(\check{t})-S_{\rm contour}(0)$ is dominated by values of $z$ close to infinity, so reduces to the Fresnel integral $(-1/\pi)\int_0^{+\infty} \mathrm{d}z [\exp(\textrm{i}z\check{t})-1]/z^{3/2}$ whose value is known (we've taken $\phi+\pi=0^+$ for simplicity).}\footnote{\label{note2} A more explicit expression of the signal can be obtained in the limit $Q\to 0$:
   $S(\check{t})=\left(\frac{\check{t}}{\pi}\right)^{1/2}\eee^{-\ii\pi/4}-\frac{\sqrt{3}}{2\pi} \eee^{-\ii\pi/4}\left(\frac{2\pi^2\check{t}}{3}-1\right)\erfi(\sqrt{\pi^2\check{t}/3})\eee^{-\pi^2\check{t}/3}$, where $\erfi$ is the imaginary error function, which has the same behavior at short times as in (\ref{eq155}) but the new behavior at long times $S(\check{t}) \underset{\check{t}\to +\infty}{\sim} - \frac{\eee^{-\ii\pi/4}}{2\pi^{1/2}} \left(\frac{3}{\pi^2}\right)^2 \frac{1}{\check{t}^{3/2}}$. Since function $\erfi$ is, despite its name, real-valued on the real axis, we find that $\re S(\check{t})=-\im S(\check{t})$ in this limit, a property of which a trace remains in Figure \ref{fig4}.} 
\begin{equation}\label{eq155}
S(\check{t}) \underset{\check{t}\to 0^+}{\sim} 2\eee^{-\ii\pi/4} (\check{t}/\pi)^{1/2} \quad\quad ; \quad\quad
S(\check{t})\underset{\check{t}\to +\infty}{\sim} \frac{-\textrm{i}Q/4}{(\frac{Q^2}{4}-\ii\frac{\pi^2}{3})^2} \frac{\eee^{\textrm{i}Q^2\check{t}/4}}{\check{t}}
\end{equation}
 \begin{SCfigure}
% Figure removed
\caption{In quantum theory and for a self-energy calculated at order two in $H_3$ and at order two in $\epsilon=k_B T/mc^2$, deviation of signal (\ref{eq012}) from Fermi's golden rule {\yc exponential $\exp(-\Gamma_qt/2)$} multiplied by $u^{1/2}$ (the universal parameter $u$ is given by (\ref{eq112}), {\yc the rate $\Gamma_q$ by (\ref{eq033})}) and plotted as a function of reduced time (\ref{eq130}) for real and imaginary parts, for increasing values of $u$ (the underlying two-dimensional superfluid is more and more weakly interacting) but for a scaled wave number as in (\ref{eq150}) {\yc kept constant,} $Q=1$. Curves for finite values of $u$: obtained by numerical integration of expression (\ref{eq132}). Curves for $u=+\infty$: taken from analytical prediction (\ref{eq151}) (they simply correspond to $\re S(\check{t})$ and $\im S(\check{t})$).}
\label{fig4}
\end{SCfigure}
 
\subsection{Comparison with classical field simulations}
\label{sec2.3}

In order to compare with the simulations of section \ref{sec1}, we transpose the quantum treatments of sections \ref{sec2.1} and \ref{sec2.2} to the classical field: in equation (\ref{eq125}), we must keep the cutoff parameter $\eta$ finite (it is one in the simulations), neglect the term $1$ in front of the occupation numbers $\bar{n}^{\rm lin}_k$ and replace these by the classical equipartition law $1/\kb$. As a test, note that the corresponding value of $\Sigma_\qq^{(2,2)}(\textrm{i}0^+)$ is then written as $-\ii\hbar\Gamma_q^{\mbox{\scriptsize clas}}/2$, where $\Gamma_q^{\mbox{\scriptsize clas}}$ is the damping rate (\ref{eq113}), as it should be. In what follows, we of course retain the full dependence of $\tilde{\Sigma}_\qq^{(2,2)}(\zetat)$ on $\zetat$ when calculating signal $s^{(2,2)}(t)$ by Green's function.

The corresponding predictions for the decay time at $1/\eee$ are shown in thick black solid lines on each panel of Figure \ref{fig1}. They are in all cases closer to the simulations than the golden rule, but paradoxically the improvement is clearest in the strong interaction regime (a) and becomes very modest in the weakly interacting regime (d). This deterioration was, however, to be expected from the reasoning preceding equation (\ref{eq150}): if $\rho\xi^2$ increases at a fixed $\qb$, the second-order Green's function calculation in $H_3$ must reduce to the golden rule, thus deviating from the simulations.

Analysis of Figure \ref{fig2} showing the time dependence of the signal gives us a further clue as to what happens when we reduce the {\yct interaction} strength, $(\rho \xi^2)^{-1}\to 0$. The predictions derived from $\tilde{\Sigma}_\qq^{(2,2)}(\zetat)$ are this time shown as dashed lines (those from the golden rule would be identically zero). They come remarkably close to the simulations for the imaginary part of the {\yct deviation} $s(t)-s_{\mbox{\scriptsize pole}}(t)$ (the red dashed line almost merges with the numerical results in panel (d)), but they show a persistent, and even increasing in relative values, disagreement on the real part, which explains the previous failure on the decay time at $1/\eee$.

The anomaly exhibited by the real part in Figure \ref{fig2}d is striking if we recall the analytical predictions of Figure \ref{fig4} (even if they are for the quantum field): the real part and the imaginary part, for the corresponding values of $u$ and $Q$ ($u\simeq 18.14$ and $Q\simeq 1.06$, remember), should be more or less symmetrical to each other with respect to the abscissa axis (see also footnote \ref{note2}), which is far from being the case in the simulations. In conclusion, the calculation of the signal with the self-energy $\tilde{\Sigma}^{(2,2)}_{\yc\qb}(\zetat)$, by fault of the real part, fails to account for the first deviation from the golden rule in the limit $u\to +\infty$ where it had every chance of excelling.

\section{Study at fourth order in $H_3$}
\label{sec3}

To understand the persistent discrepancy in the real part of the signal between the numerical simulations in section \ref{sec1} and the theoretical study at second-order in $H_3$ in section \ref{sec2}, even when the two-dimensional superfluid enters the weakly interacting limit $(\rho\xi^2)^{-1}\to 0$, we calculate the self-energy at fourth order in $H_3$, either {\yct with} the usual diagrammatic formalism in section \ref{sec3.1}, or by a Landau-style reasoning in section \ref{sec3.2}, which provides an interesting physical insight and leads to a more compact expression. In section \ref{sec3.3}, however, we show that the fourth-order result is useless in the low-temperature limit $\epsilon=k_B T/mc^2\to 0$ of interest here.

\subsection{Calculation with the diagrammatic formalism of reference \cite{FW}}
\label{sec3.1}

\paragraph{The two topologies}
 We directly omit the quartic Hamiltonian $H_4$ in equation (\ref{eq001}) and the non-resonant terms $H_3^{\mbox{\yct\scriptsize(non\, res.)}}$ in the cubic Hamiltonian (\ref{eq100}), as their contributions to $\Sigma^{(4)}$ are subleading in the limit $\epsilon\to 0$ (this is obvious for $H_3^{\mbox{\yct\scriptsize(non\, res.)}}$, which does not benefit from the \g{small-denominator} effect, and this will be justified for $H_4$ in section \ref{sec3.2}). The coupling Hamiltonian therefore reduces to the terms $H_3^{(+/-)}$ given in equation (\ref{eq101}). In the corresponding fourth-order Feynman diagrams, each of the four vertices is the meeting point of three phonon lines, leading to two possible topologies, (i) inner-loop and (ii) bridge, as shown in Figure \ref{fig8}. In the first topology, we lift the degeneracy by arbitrarily putting the inner loop in the upper branch, and remember (as in section \ref{sec2}) that swapping the two inner-loop lines leads to the same diagram; {\yc after orientation of internal lines,} this leaves eight distinct diagrams. In the second topology, the degeneracy is lifted by orienting the bridge line downwards; six distinct diagrams remain. In reality, in the terminology of Figure \ref{fig8}, the contributions of diagrams $B_3$ and $B_4$ to $\Sigma^{(4)}$ double, as do those of $P_2$ and $P_3$ and those of $P_4$ and $P_6$.\footnote{There is indeed a non-trivial symmetry connecting certain diagrams. To see this, let's apply a vertical axis rotation of angle $\pi$ to each diagram, then change the orientation of all lines (internal and external); this does not change the value of the diagram's contribution to $\Sigma^{(4)}$. Most diagrams are {\yct mapped} on themselves in this way, with the exception of pairs $\{B_3,B_4\}$, $\{P_2,P_3\}$ and $\{P_4,P_6\}$, whose two partners are simply exchanged (remember, in the case of bridge diagrams, that a horizontal axis rotation of angle $\pi$, which restores the original orientation of the vertical line, does not change the value of the diagram). This means that $\Sigma_{B_3}^{{\yc(4)}}(\qq,z)=\Sigma_{B_4}^{{\yc(4)}}(\qq,z)$, $\Sigma_{P_2}^{{\yc(4)}}(\qq,z)=\Sigma_{P_3}^{{\yc(4)}}(\qq,z)$ and $\Sigma_{P_4}^{{\yc(4)}}(\qq,z)=\Sigma_{P_6}^{{\yc(4)}}(\qq,z)$ as announced.} Only eleven truly independent diagrams remain.
% Figure environment removed
 % Figure environment removed
 
\paragraph{Calculating a generic inner-loop diagram}
 It's convenient to number the internal lines from $j=1$ to $5$ as on the left-hand side of Figure \ref{fig7}. Each line has an orientation $s_j\in\{-1;1\}$ (left to right if $s_j=1$, right to left if $s_j=-1$), a wave vector $\kk_j$ and a Matsubara energy $\omega_j\in 2\ii\pi k_B T \mathbb{Z}$. In Figure \ref{fig8}, we have chosen to always have $s_3\geq s_4$ in order to avoid the double counting of diagrams mentioned above. As independent quadrivectors, we've chosen those in lines 2 and 3, which we've renamed as follows:  
\begin{equation}\label{eq210}
(\kk_2,\omega_2)=(\kk,\omega) \quad ; \quad (\kk_3,\omega_3)=(\kk',\omega')
\end{equation} The others can be deduced by energy-momentum conservation at each vertex. For example, the sum of the wave vectors arriving at a given vertex is equal to the sum of the wave vectors departing from it, giving: 
\begin{equation}\label{eq211}
\kk_1=s_1(\qq-s_2\kk) \quad ; \quad \kk_4=s_4(s_2\kk-s_3\kk') \quad ; \quad \kk_5=s_2 s_5 \mathbf{k}\end{equation} The same relations apply to Matsubara energies. Given the rules set out in section 25 of reference \cite{FW}, the contribution to the self-energy of the generic inner-loop diagram $B_i$ is: 
\begin{equation}\label{eq212}
\Sigma_{B_i}^{(4)}(\qq,z) = \frac{(-k_B T)^2}{1+\delta_{s_3,s_4}} \int\frac{\dd^dk}{(2\pi)^d}\int\frac{\dd^dk'}{(2\pi)^d}
\mA(\kk_1,\kk_2;\qq)\mA(\kk_3,\kk_4;\kk_2)\mA(\kk_3,\kk_4;\kk_5)\mA(\kk_1,\kk_5;\qq)
\sum_{\omega,\omega'\in 2\ii\pi k_B T\mathbb{Z}}\frac{1}{\prod_{j=1}^{5} (\omega_j-\veps_{\kk_j})}
\end{equation} where we temporarily have $z=\Omega\in 2\ii\pi k_B T\mathbb{Z}$ (see section \ref{sec2.1}). To write the transition amplitudes at each vertex, we used the invariance of function $\mA$ of equation (\ref{eq102}) by permutation of its three arguments and hence the fact that matrix elements $\langle \kk_2,\kk_3|\mathcal{H}_3|\kk_1\rangle$, $\langle \kk_1,\kk_3|\mathcal{H}_3|\kk_2\rangle$ and $\langle \kk_1,\kk_2|\mathcal{H}_3|\kk_3\rangle$ are all equal to $\mA(\kk_2,\kk_3;\kk_1)$. The denominator of the symmetry factor {\yc in front of the integrals} is $2$ if the two lines of the inner loop have the same orientation (to compensate for the double counting of configurations differing only by the exchange of quadrivectors $(\kk_3,\omega_3)$ and $(\kk_4,\omega_4)$ and in this case physically identical); otherwise, these two lines are distinguishable and the symmetry factor is $1$. The Matsubara double sum is calculated explicitly in \ref{app3}, which also shows that its expression simplifies in the limit of small angles $O(\epsilon)\to 0$ between the wave vectors, where we can restrict to $s_2=s_5$ (otherwise, the diagram's contribution is subleading in $\epsilon$, see equation (\ref{eq215}) below): 
\begin{equation}\label{eq213}
\sum_{\omega,\omega'\in 2\ii\pi k_B T\mathbb{Z}}\frac{(k_B T)^2}{\prod_{j=1}^{5}(\omega_j-\veps_{\kk_j})} \stackrel{\mbox{\scriptsize diag.}\, B}{\underset{\epsilon\to 0}{\sim}} -s_1 s_4 \frac{(\bar{n}_{\veps_3}-\bar{n}_{-s_3s_4\veps_4})(\bar{n}_{-s_1\veps_1}-\bar{n}_{s_2\veps_2})}{(\zeta-\zeta_1)^2(\zeta-\zeta_2)}
\end{equation} with variable $\zeta=z-\veps_\qq$ and $\zeta_j$ all of order $k_B T\epsilon^2$, 
\begin{equation}\label{eq214}
\zeta_1\equiv s_1\veps_{\kk_1}+s_2\veps_{\kk_2}-\veps_\mathbf{q}\quad ; \quad \zeta_2\equiv s_1\veps_{\kk_1}+s_3\veps_{\kk_3}+s_4\veps_{\kk_4}-\veps_{\qq}
\end{equation} The dependence in $\zeta$ of form (\ref{eq213}) on the right-hand  side is easy to interpret physically since $\zeta_1+\veps_\qq$ and $\zeta_2+\veps_\qq$ are the energies of the intermediate states marked by the vertical lines on the left-hand side of Figure \ref{fig7}: in fact, they are the sums of the energies of the internal lines $j$ intersected by each vertical line counted algebraically, i.e. weighted by the orientations $s_j$ (if $s_j=-1$, the phonon on line $j$ is going backward in time and its energy must be counted negatively). This validates estimate (\ref{eq024}) by reproducing its energy denominators. 

To go to the $\epsilon\to 0$ limit in (\ref{eq212}), we directly set the angles to zero in each coupling amplitude $\mA$ (which brings out a factor $3(1+\Lambda)$), then expand the energy denominators of (\ref{eq213}) on the model of (\ref{eq025}) ; for the \g{small-denominator} effect to take full action, the internal vectors $\kk_j$ must all make a small angle $O(\epsilon)$ with the incoming wave vector $\qq$, which imposes\footnote{When $\kk_2=\kk$ and $\kk_3=\kk'$ form a zero angle with $\qq$, the other wave vectors $\kk_j$ given by (\ref{eq211}) must satisfy the condition $\kk_j\cdot\qq=k_j q$.} 
\begin{equation}\label{eq215}
\sig(q-s_2k)=s_1 \quad ; \quad \sig(s_2k-s_3k')=s_4 \quad ; \quad s_2s_5=1
\end{equation} The first two conditions in (\ref{eq215}) restrict the range of variation of wave numbers $k$ and $k'$. The last, which is independent of them, eliminates diagrams $B_3$ and $B_4$, of subleading contributions. Remembering the definitions (\ref{eq022},\ref{eq004}) of $\zetat$ and (\ref{eq020}) of $\qb$, we arrive in dimension $d=2$ at the result (for $\im\zetat\neq 0$ but of arbitrary sign): 
\begin{equation}\label{eq220}
\tilde\Sigma^{(4,2)}_{B_i}(\qb,\tilde{\zeta})=\frac{\qb}{1+\delta_{s_3,s_4}}\left(\frac{9(1+\Lambda)^2}{8\rho\xi^2}\right)^2\int_{\mathcal{D}_1^B}\frac{\kb\mathrm{d}\kb}{(2\pi)^2} \int_{\mathcal{D}_2^B(\kb)} \frac{\kb'\mathrm{d}\kb'}{(2\pi)^2} \kb^2\kb'(\qb-s_2\kb)(s_2\kb-s_3\kb')(\bar{n}^{\rm lin}_{k'}-\bar{n}^{\rm lin}_{k'-s_2 s_3 k})(\bar{n}^{\rm lin}_{s_2k}-\bar{n}^{\rm lin}_{s_2k-q}) I(\kb,\kb')
\end{equation} 
with {\yc the occupation numbers (\ref{eq029}) extended to negative wavenumbers and} the angular integral\footnote{In the integral, before taking the limit $\epsilon\to 0$, we set $\tilde{\theta}=\theta/\epsilon$ and $\tilde{\phi}=(\theta-\theta')/\epsilon$, where $\theta$ is the angle between $\qq$ and $\kk$, $\theta'$ the angle between $\qq$ and $\kk'$.} 
\begin{equation}\label{eq221}
I(\kb,\kb')=\int_{\mathbb{R}} \dd\tilde\theta \int_{\mathbb{R}} \dd\tilde\phi \frac{1}{[\tilde{\zeta}-(A_1^B+B_1^B\tilde\theta^2)]^{2}[\tilde{\zeta}-(A^B_2+B^B_1\tilde\theta^2+B_2^B\tilde\phi^2)]}
\end{equation} involving coefficients 
\begin{equation}\label{eq222}
A_1^B=\frac{3\gamma}{8}(\kb-s_2 \qb)\bar{k}\bar{q}\quad ; \quad A_2^B=\frac{3\gamma}{8}\bar{k}\kb' (s_2\kb'-s_3 \kb) +A_1^{B} \quad ; \quad B_1^B=\frac{-\kb\qb/2}{\kb-s_2 \qb} \quad ; \quad B_2^B=\frac{-\bar{k}\kb'/2}{s_2\kb'-s_3 \kb}
\end{equation} and integration domains {\yc$\mathcal{D}_1^B$} over $\kb$, then {\yc$\mathcal{D}_2^{B}(\kb)$} over $\kb'$ at fixed $\kb$, satisfying conditions (\ref{eq215}) and cut-off (\ref{eq103}) $\kb_j<\eta$, $1\leq j\leq 5$. In the left-hand side of (\ref{eq220}), the second integer in the exponent reminds us {\yc to which order in $\epsilon$ the function} $\Sigma^{(4)}$ is calculated {\yc(here to second order)}, and the tilde over $\Sigma$ means, as in (\ref{eq021}), that {\yc the result} has been divided by $k_BT\epsilon^2$. Finally, we check on a case-by-case basis that expression (\ref{eq220}) has a finite limit when the cutoff $\eta\to+\infty$ (for the quantum field).\footnote{For example, in the case of {\yc the diagram} $B_8$ ($s_1=s_4=-1$, $s_2=s_3=s_5=1$), the domains {\yc$\mathcal{D}_1^B=[\qb,\eta]$ and $\mathcal{D}_2^B(\kb)=[\kb,\eta]$} are unbounded when $\eta\to +\infty$ but the bosonic amplification factors $(\bar{n}_{k'}^{\rm lin}-\bar{n}_{k'-k}^{\rm lin})(\bar{n}_k^{\rm lin}-\bar{n}_{k-q}^{\rm lin})=O[\exp(-\kb-\kb'')]$ are {\yct rapidly decreasing} in the two decoupled integration variables $(\kb,\kb''=\kb'-\kb)$ and {\yct suppress} any power-law divergence that might arise from the other factors ; in the case of $B_7$ ($s_1=-1$, $s_2=s_3=s_4=s_5=1$), $\mathcal{D}_1^B=[\qb,\eta]$ and $\mathcal{D}_2^B(\kb)=[0,\kb]$ remain unbounded and the bosonic factors $(1+\bar{n}_{k'}^{\rm lin}+\bar{n}_{k-k'}^{\rm lin})(\bar{n}_k^{\rm lin}-\bar{n}_{k-q}^{\rm lin})=O(\exp(-\kb))$ are now {\yct rapidly decreasing} only in variable $\kb$ but, since $\kb'<\kb$, the integral over $\kb'$ at fixed $\kb$ is at most $O(\kb^\alpha)$, $\alpha>0$, and the integral over $\kb$ converges if $\eta\to +\infty$.}

\paragraph{Calculating a generic bridge diagram}
 The beginning is very similar to the previous case. We number the internal lines as on the right-hand side of Figure \ref{fig7}, with the same sign convention, except for the vertical line (the \g{bridge}), with sign $s_5=1$ when oriented downwards (which we have limited ourselves to in Figure \ref{fig8} to avoid double counting). We keep as independent quadrivectors 
\begin{equation}\label{eq230}
(\kk_1,\omega_1)=(\kk,\omega) \quad ;\quad (\kk_3,\omega_3)=(\kk',\omega')
\end{equation} the others deduced by energy-momentum conservation, e.g. for wave vectors: 
\begin{equation}\label{eq231}
\kk_2=s_2(\qq-s_1\kk) \quad ;\quad \kk_4=s_4(\qq-s_3\kk') \quad ;\quad \kk_5=s_5(s_1\kk-s_3\kk')
\end{equation} The contribution of diagram $P_i$ to the self-energy is then written \cite{FW}: 
\begin{equation}\label{eq232}
\Sigma_{P_i}^{(4)}(\qq,z) = (-k_B T)^2\int\frac{\dd^dk}{(2\pi)^d}\int\frac{\dd^dk'}{(2\pi)^d}
\mA(\kk_1,\kk_2;\qq)\mA(\kk_3,\kk_5;\kk_1)\mA(\kk_2,\kk_5;\kk_4)\mA(\kk_3,\kk_4;\qq)
\sum_{\omega,\omega'\in 2\ii\pi k_B T\mathbb{Z}}\frac{1}{\prod_{j=1}^{5} (\omega_j-\veps_{\kk_j})}
\end{equation} No need here to divide by a symmetry factor. The explicit calculation of the Matsubara double sum is given in \ref{app3}, which also establishes its simplified expression in the limit of small angles $O(\epsilon)$ between the wave vectors: 
\begin{equation}\label{eq233}
\sum_{\omega,\omega'\in 2\ii\pi k_B T\mathbb{Z}}\frac{(k_B T)^2}{\prod_{j=1}^{5}(\omega_j-\veps_{\kk_j})} \stackrel{\mbox{\scriptsize diag.}\, P}{\underset{\epsilon\to 0}{\sim}} 
-s_1 s_2 s_4 s_5 \frac{(\bar{n}_{-s_2\veps_2}-\bar{n}_{s_1\veps_1})(\bar{n}_{\veps_3}-\bar{n}_{-s_3s_5\veps_5})}
{(\zeta-\zeta_1)(\zeta-\zeta_2)(\zeta-\zeta_4)}
-s_1 s_2 s_4 s_5 \frac{(\bar{n}_{-s_3s_5\veps_5}-\bar{n}_{-s_3s_4\veps_4})(\bar{n}_{s_1\veps_1}-\bar{n}_{-s_2\veps_2})}
{(\zeta-\zeta_1)(\zeta-\zeta_2)(\zeta-\zeta_3)}
\end{equation} with variable $\zeta$ and $\zeta_j$ all of order $k_BT\epsilon^2$, 
\begin{equation}\label{eq238}
\zeta_1\equiv s_1\veps_{\kk_1}+s_2\veps_{\kk_2}-\veps_\qq\quad ; \quad \zeta_2=s_3\veps_{\kk_3}+s_4\veps_{\kk_4}-\veps_\mathbf{q}\quad ; \quad \zeta_3\equiv s_1\veps_{\kk_1}+s_4\veps_{\kk_4}-s_5\veps_{\kk_5}-\veps_{\qq} \quad ;\quad \zeta_4\equiv s_2\veps_{\kk_2}+s_3\veps_{\kk_3}+s_5\veps_{\kk_5}-\veps_{\qq}
\end{equation} There are this time two contributions to the right-hand side, called type I and type II in this order. Quantities $\zeta_j$ are energy differences between intermediate and initial states, as in the inner-loop case, except that four straight lines must be introduced to identify them, see right-hand side of Figure \ref{fig7}: two vertical lines for $\zeta_1$ and $\zeta_2$, two oblique lines for $\zeta_3$ and $\zeta_4$. Indeed, from the point of view of the oblique line marked I on the figure, the internal line of the bridge is oriented from left to right if $s_5=1$, and $\zeta_4$ must have a plus sign in front of $s_5\veps_{\kk_5}$ in equation (\ref{eq238}); for oblique line II, the opposite is true, hence the minus sign in front of $s_5\veps_{\kk_5}$ in $\zeta_3$.

When $\epsilon\to 0$, the limit of small angles in the energy denominators and in the coupling amplitudes $\mA$ is taken as before, except that a distinction must be made between the two types. For type I contribution, we obtain (for $\im\zetat\neq 0$ but of arbitrary sign) 
\begin{equation}\label{eq240}
\tilde\Sigma^{(4,2)}_{P_{i}^{\rm I}}(\qb,\tilde{\zeta})=\qb\left(\frac{9(1+\Lambda)^2}{8\rho\xi^2}\right)^2\int_{\mathcal{D}_1^P} \frac{\kb\mathrm{d}\kb}{(2\pi)^2} \int_{\mathcal{D}_2^P(\kb)} \frac{\kb'\mathrm{d}\kb'}{(2\pi)^2} \bar{k}\kb'(\qb-s_1\kb)(\qb-s_3\kb')(\kb-s_1s_3\kb')(\bar{n}^{\rm lin}_{k'}-\bar{n}^{\rm lin}_{k'-s_1 s_3 k})(\bar{n}^{\rm lin}_{s_1k}-\bar{n}^{\rm lin}_{s_1k-q}) J(\kb,\kb')
\end{equation} with the angular integral 
\begin{equation}\label{eq241}
J(\kb,\kb')=\int_{\mathbb{R}} \dd\tilde\theta \int_{\mathbb{R}} \dd\tilde\theta' \frac{1}{[\tilde{\zeta}-(A_1^P+B_1^P\thetat^2)][\tilde{\zeta}-(A_2^P+B_2^P\thetat'^2)][\tilde{\zeta}-(A_4^P+B_1^P\thetat^2+B_4^P(\thetat-\thetat')^2)]}
\end{equation} For type II contribution, we obtain (again for $\im\zetat\neq 0$ but of arbitrary sign) 
\begin{equation}\label{eq242}
\tilde\Sigma^{(4,2)}_{P_{i}^{\rm II}}(\qb,\tilde{\zeta})=\qb\left(\frac{9(1+\Lambda)^2}{8\rho\xi^2}\right)^2\int_{\mathcal{D}_1^P} \frac{\kb\mathrm{d}\kb}{(2\pi)^2} \int_{\mathcal{D}_2^P(\kb)} \frac{\kb'\mathrm{d}\kb'}{(2\pi)^2} \bar{k}\kb'(\qb-s_1\kb)(\qb-s_3\kb')(\kb-s_1s_3\kb')(\bar{n}^{\rm lin}_{k'-s_3q}-\bar{n}^{\rm lin}_{k'-s_1 s_3 k})(\bar{n}^{\rm lin}_{s_1k}-\bar{n}^{\rm lin}_{s_1k-q}) K(\kb,\kb')
\end{equation} with the angular integral 
\begin{equation}\label{eq243}
K(\kb,\kb')=\int_{\mathbb{R}} \dd\tilde\theta \int_{\mathbb{R}} \dd\tilde\theta' \frac{1}{[\tilde{\zeta}-(A_1^P+B_1^P\thetat^2)][\tilde{\zeta}-(A_2^P+B_2^P\thetat'^2)][\tilde{\zeta}-(A_3^P+B_2^P\thetat'^2+B_3^P(\thetat-\thetat')^2)]}
\end{equation} Irrespective of type, the coefficients are 
\begin{equation}\label{eq244}
A_1^P=\frac{3\gamma}{8}(\kb-s_1\qb)\kb\bar{q}\quad ; \quad A_2^P=\frac{3\gamma}{8}(\kb'-s_3\qb)\kb'\bar{q}\quad ; \quad A_4^P=A_1^P+\frac{3\gamma}{8}\bar{k}\kb' (s_1\kb'-s_3\kb) \quad ; \quad A_3^P=A_1^P+A_2^P-A_4^P
\end{equation} \begin{equation}\label{eq245}
B_1^P=\frac{-\kb\qb/2}{\kb-s_1\qb} \quad ; \quad B_2^P=\frac{-\kb'\qb/2}{\kb'-s_3\qb} \quad ; \quad B_4^P=\frac{-\kb\kb'/2}{s_1\kb'-s_3\kb} \quad ; \quad B_3^P=-B_4^P
\end{equation} The integration domains $\mathcal{D}_1^P$ and $\mathcal{D}_2^P(\kb)$ are also independent of type; they take into account the cut-off (\ref{eq103}) on all internal wave vectors $\kk_j$, $1\leq j\leq 5$, as well as conditions 
\begin{equation}\label{eq239}
\sig(q-s_1k)=s_2 \quad ;\quad \sig(q-s_3k')=s_4 \quad;\quad \sig(s_1k-s_3k')=s_5
\end{equation} ensuring that $\kk_2$, $\kk_4$ and $\kk_5$ are collinear with and in the same direction as $\qq$ as soon as $\kk$ and $\kk'$ are. For all $s_j$ orientations taken from Figure \ref{fig8}, conditions (\ref{eq239}) are compatible, so none of the $P_i$ diagrams is subleading in $\epsilon$. Finally, we verify the existence of a finite limit in (\ref{eq240},\ref{eq242}) when cutoff $\eta\to +\infty$ (for the quantum field).

\subsection{Calculation using Landau approach}
\label{sec3.2}

The calculation of $\Sigma^{(4)}$ as in reference \cite{FW}, although systematic and rigorous, does not explicitly say which collisional processes between phonons lie behind each diagram in Figure \ref{fig8}. Here we develop a more heuristic method, inspired by reference \cite{LK} and valid to the leading order in $\epsilon$, which answers the question, gives the result in compact form (rather than spread into a dozen or so integral contributions) and makes it easy to show that the contribution of $H_4$ is subleading in the $\epsilon\to 0$ limit.

Reference \cite{LK} explains, for Hamiltonian (\ref{eq001}), how to calculate by an extended Fermi golden rule the damping rate of  phonon mode $\qq$ due to $n\to n'$ collisions, $n$ being the number of incoming $\qq_i$ phonons (they include one and only one phonon of wave-vector $\qq$) and $n'$ the number of outgoing $\qq'_j$ phonons (they do not include any phonon of wave-vector $\qq$). The ordinary cases of Beliaev and Landau processes correspond to $(n,n')=(1,2)$ and $(n,n')=(2,1)$, they have $n+n'=3$ phonons; here, it will be about four-phonon processes, like $(n,n')=(2,2)$. We start by writing kinetic equations for the phonon occupation numbers, provisionally assumed to be out of equilibrium, counting positively the direct process $n\to n'$ (which tends to empty mode $\qq$) and negatively the inverse process $n'\to n$ (which tends to fill it):  
\begin{multline}
\label{eq250}
\frac{\dd}{\mathrm{d}t} {n}_\mathbf{q}= -\frac{1}{(n-1)!}\int\left(\prod_{i=2}^{n}\frac{\dd^dq_i}{(2\pi)^d}\right)\frac{1}{n'!}\int\left(\prod_{j=1}^{n'}\frac{\dd^dq'_j}{(2\pi)^d}\right)\frac{2\pi}{\hbar} \mA^2_{\ii\to \ff} \delta(E_\ii-E_\ff) (2\pi)^d \delta\left(\sum_{i=1}^{n}\qq_i-\sum_{j=1}^{n'}\qq'_j\right)\\
\times\left[\prod_{i=1}^{n}n_{\qq_i}\prod_{j=1}^{n'}(1+n_{\qq'_j})-\prod_{i=1}^{n} (1+n_{\qq_i}) \prod_{j=1}^{n'}n_{\qq'_j}\right]
\end{multline}
 We have taken the thermodynamic limit, and divided by symmetry numbers taking into account the invariance of the process by permutation of the $n-1$ last incoming wave vectors (we agree that $\qq_1=\qq$) and by permutation of the $n'$ outgoing wave vectors. We recognize in (\ref{eq250}) the Dirac distribution of energy conservation, the factor $2\pi/\hbar$ and the square of a transition amplitude $\mA_{\ii\to\ff}$ between initial state $|\ii\rangle=|(\qq_i)_{1\leq i\leq n}\rangle$ and final state $|\ff\rangle=|(\qq'_j)_{1\leq j\leq n'}\rangle$ typical of the golden rule, as well as a Dirac distribution of momentum conservation and the bosonic amplification factors. The principle of microreversibility was used to factorize the squared amplitude of the direct process. 

We then take the excitation procedure (\ref{eq010}), where all phonon modes are at thermal equilibrium except $\qq$, hence the substitution $n_\kk\to \bar{n}_\kk$, $\kk\neq\qq$, and linearize (\ref{eq250}) around $n_\qq=\bar{n}_\qq$ for weak excitation, which gives the back-to-equilibrium equation $(\dd/\mathrm{d}t)(n_\qq-\bar{n}_\qq)|_{t=0^+}=-\Gamma_\qq^{n\to n'}(n_\qq-\bar{n}_\qq)$ with rate 
\begin{multline}
\label{eq251}
\Gamma_\qq^{n\to n'}=\frac{1}{(n-1)!}\int\left(\prod_{i=2}^{n}\frac{\dd^dq_i}{(2\pi)^d}\right)\frac{1}{n'!}\left(\int\prod_{j=1}^{n'}\frac{\dd^dq'_j}{(2\pi)^d}\right)\frac{2\pi}{\hbar} \mA^2_{\ii\to \ff} \delta(E_\ii-E_\ff) (2\pi)^d \delta\left(\sum_{i=1}^{n}\qq_i-\sum_{j=1}^{n'}\qq'_j\right) \\ {\yct\times} \left[\prod_{i=2}^{n}\bar{n}_{\qq_i}\prod_{j=1}^{n'}(1+\bar{n}_{\qq'_j})-\prod_{i=2}^{n} (1+\bar{n}_{\qq_i}) \prod_{j=1}^{n'}\bar{n}_{\qq'_j}\right]
\end{multline}
 where, unlike equation (\ref{eq250}), all products on the incoming wave vectors exclude mode $\qq$ and therefore start at $i=2$. The difference with the golden rule lies in the calculation of amplitude $\mA_{\ii\to\ff}$. Usually, we simply have $\mA_{\ii\to\ff}=\langle\ff|\mH_3|\ii\rangle$. Reference \cite{LK} deals precisely with the useful case of four-phonon collisions, where $\mA_{\ii\to\ff}$ is obtained by treating {\yc$H_3=\mathcal{H}_3/L^{d/2}$} at second order and {\yc$H_4=\mathcal{H}_4/L^d$} at first order of perturbation theory: 
\begin{equation}\label{eq252}
\mA_{\ii\to\ff}=\langle\ff|\mH_4|\ii\rangle + \sum_{|\lambda \rangle} \frac{\langle\ff|\mH_3|\lambda\rangle\langle\lambda|\mH_3|\ii\rangle}{E_\ii-E_\lambda}
\end{equation} and the sum is taken over all possible intermediate phonon Fock states $|\lambda\rangle$.  A simple power counting as in section \ref{sec0} shows that $\mH_4$'s contribution to (\ref{eq252}) is subleading in the small-angle limit $\epsilon\to 0$;\footnote{In view of equations (\ref{eq052},\ref{eq104}), the contribution of $\mH_4$ to (\ref{eq252}) is of order $(mc^2/\rho)\epsilon^2$; in view of equations (\ref{eq025},\ref{eq027}), that of $\mH_3$ is of order $(mc^2/\rho)\epsilon^0$.} so we omit it in the following. In the energy denominator, we have chosen the reference energy $E_\ii$ of the initial state rather than $E_\ff$ of the final state, because we are talking about a $\ii\to\ff$ transition, as recalled by the choice of $\ff$ in bra and $\ii$ in ket in the matrix element in the numerator, which is not a serious physical reason. Fortunately, this doesn't matter, since the amplitude is used in (\ref{eq250}) on the energy shell ($E_\ii=E_\ff$): the result is invariant by exchanging the roles of $\ii$ and $\ff$ in any term of the sum in the right-hand side of (\ref{eq252}): \begin{equation}\label{eq253}
\frac{\langle\ff|\mH_3|\lambda\rangle\langle\lambda|\mH_3|\ii\rangle}{E_\ii-E_\lambda} \mbox{(direct direction)} \longleftrightarrow \frac{\langle\ii|\mH_3|\lambda\rangle\langle\lambda|\mH_3|\ff\rangle}{E_\ff-E_\lambda} \mbox{(inverse direction)}
\end{equation} The total damping rate is obtained by summing $\Gamma_\qq^{n\to n'}$ over the collisional processes $n\to n'$ relevant to the order under consideration. From the following obvious rules, (i) the diagram giving the transition amplitude $\mA_{\ii\to\ff}$ is connected, i.e. there is no possible decoupling of the $n\to n'$ process into two independent collisional processes, (ii) any intermediate state (between successive actions of $\mH_3$) must differ from the initial and final states, we find at order four in $H_3^{(+)}+H_3^{(-)}$ as the only possibilities the processes $1\to 3$, $2\to 2$ and $3\to 1$: \footnote{In reference \cite{LK}, {\yc unlike in our case,} the acoustic branch has a concave start {\yc($\gamma<0$ in equation (\ref{eq003}))}: only process $2\to 2$ preserves momentum-energy and gives a non-zero contribution to (\ref{eq254}); however, it was miscalculated, and the error has been corrected in reference \cite{Annalen}.} 
\begin{equation}\label{eq254}
\Gamma^{(4)}_\qq=\Gamma_\qq^{1\to 3}+\Gamma_\qq^{2\to 2}+\Gamma_\qq^{3\to 1}
\end{equation}
 Heuristically, we generalize this formalism to the calculation of the fourth-order contribution in $H_3^{(+)}+H_3^{(-)}$ to the self-energy of mode $\qq$. As with the damping rate, we assume that it is the sum of the contributions of the above-mentioned $n\to n'$ collisional processes: 
\begin{equation}\label{eq255}
\Sigma^{(4)}(\qq,z)=\Sigma_{1\to 3}(\qq,z)+\Sigma_{2\to 2}(\qq,z)+\Sigma_{3\to 1}(\qq,z)
\end{equation} Knowing that the energy-conserving Dirac distribution in (\ref{eq251}) ultimately comes from the identity
\begin{equation}\label{eq256}
\lim_{\delta\to 0^+} \im \frac{1}{E_\ii+\ii\delta-E_\ff}=-\pi\delta(E_\ii-E_\ff)\quad,
\end{equation} 
that the displaced variable $\zeta=z-\veps_\qq$ plays the role of $\ii\delta$ in the pole approximation (\ref{eq014}) {\yc and a self-energy differs from a rate by a factor of $-\ii\hbar/2$}, we arrive without hesitation at the form 
\begin{multline}
\label{eq257}
\Sigma_{n\to n'}^{(4)}(\qq,z)=\frac{1}{(n-1)!} \int\left(\prod_{i=2}^{n}\frac{\dd^dq_i}{(2\pi)^d}\right)\frac{1}{n'!} \int\left(\prod_{j=1}^{n'} \frac{\dd^dq'_j}{(2\pi)^d}\right) \frac{\mathcal{A}_{\ii\to\ff}^2(\qq,z)}{\zeta+E_\ii-E_\ff}(2\pi)^d \delta\left(\sum_{i=1}^{n}\qq_i-\sum_{j=1}^{n'}\qq'_j\right) \\
\times\left[\prod_{i=2}^{n}\bar{n}_{\qq_i}\prod_{j=1}^{n'}(1+\bar{n}_{\qq'_j})-\prod_{i=2}^{n} (1+\bar{n}_{\qq_i}) \prod_{j=1}^{n'}\bar{n}_{\qq'_j}\right]
\end{multline}
 The only pitfall lies in the construction of the $\zeta$ dependence of the {\yc$\mA_{\ii\to\ff}$} transition amplitude. The choice of $E_\ii$ or $E_\ff$ as reference energy is no longer irrelevant. Choosing $E_\ii$ as in the left-hand side of (\ref{eq253}) means taking the point of view of a $\ii\to\ff$ transition, i.e. travelling through time in the ordinary direction; the correct integration path is precisely $C_+$, the one spanned by variable $\zeta$ in (\ref{eq012}); $E_\ii$ must therefore be replaced by $\zeta+E_\ii$ in the energy denominator internal to {\yc$\mA_{\ii\to\ff}$}. On the other hand, choosing $E_\ff$ as in the right-hand side of (\ref{eq253}) means taking the opposite point of view of a $\ff\to\ii$ transition, i.e. going backward in time; the correct integration path for {\yct the} Green's function is then $C_-$ \cite{CCTbordeaux}, which runs in the lower half-plane parallel to the real axis from $-\infty$ to $+\infty$ and is spanned by variable $-\zeta$ in (\ref{eq012}); $E_\ff$ must therefore be replaced by $-\zeta+E_\ff$ in the energy denominator internal to {\yc$\mA_{\ii\to\ff}$}. 

To decide according to the intermediate state $|\lambda\rangle$, we need to rely on the following property. By construction, function $\Sigma^{(4)}(\qq,z)$ depends on the mode through its wave vector $\qq$ but not its energy $\veps_\qq$ (when using the original complex energy $z$ as a variable), and therefore cannot have an energy denominator containing $\veps_\qq$. The opposite is true when using the displaced variable $\zeta$. This means that, if in the direct direction of the transition amplitude, the action on $|\ii\rangle$ of operator $\mH_3^{(\pm)}$ does not cause the phonon of wave vector $\qq$ to disappear, {\ycd its energy $\veps_\qq$ appears in $E_\lambda$ and therefore disappears in $\zeta+E_\ii-E_\lambda$}, and reverse direction of the transition amplitude must be used, this rule being applied intermediate state by intermediate state. Hence our final prescription on the transition amplitude to be included in (\ref{eq257}):\footnote{{\ycd In reality, the collisional processes $n\to n'$ with $n$ incoming phonons $\qq_i$ and $n'$ outgoing phonons $\qq'_j$ occur in a thermal bath, and the modes of the virtual phonons $\kk$ (neither incoming nor outgoing) are not initially empty. The corresponding bosonic amplification factors $(n_\kk+1)^{1/2}$ and $n_\kk^{1/2}$ would therefore have to be taken into account in the expression (\ref{eq258}) of the transition amplitude, and the contribution of the new allowed processes, in which a virtual phonon $\kk$ is annihilated before being recreated (which is forbidden in vacuum), would have to be included. However, these two modifications exactly cancel each other and the expression (\ref{eq258}) remains unchanged. Let's check this on a single term of the first sum in (\ref{eq258}). To this aim, let's describe in more detail the corresponding transition $|\ii\rangle\to|\lambda\rangle\to|\ff\rangle$ by noting $|\ii\rangle=|\ii_1\rangle\otimes|\ii_2\rangle\otimes|n_\kk:\kk\rangle$ and $|\ff\rangle=|\ff_1\rangle\otimes|\ff_2\rangle\otimes|n_\kk: \kk\rangle$, where $|\ii_1\rangle$ ($|\ii_2\rangle$) are the incoming photons annihilated by the first (second) action of $\mathcal{H}_3$, $|\ff_1\rangle$ ($|\ff_2\rangle$) are the outgoing phonons created by the first (second) action of $\mathcal{H}_3$ and $|n_\kk: \kk\rangle$ the Fock state with $n_\kk$ quanta in the virtual phonon mode, so the intermediate state is written $|\lambda\rangle=|\ff_1\rangle\otimes|\ii_2\rangle\otimes|n_\kk+1:\kk\rangle$
(we don't take into account the other modes, spectators, nor the thermal occupation of incoming and outgoing modes, since their bosonic factors are included in (\ref{eq250},\ref{eq257})). In this example, $\qq\notin |\lambda\rangle$ hence $\qq\in|\ii_1\rangle$.
The components of $\mathcal{H}_3$ ensuring the transitions $|\ii\rangle\to|\lambda\rangle$ and $|\lambda\rangle\to|\ff\rangle$ are written respectively $W_1=\mathcal{A}_1\hat{b}^\dagger_\kk(\prod_{\qq'_j\in|\ff_1\rangle} \hat{b}^\dagger_{\qq'_j})(\prod_{\qq_i\in|\ii_1\rangle} \hat{b}_{\qq_i})$ and $W_2=\mathcal{A}_2(\prod_{\qq'_j\in|\ff_2\rangle} \hat{b}^\dagger_{\qq'_j})(\prod_{\qq_i\in|\ii_2\rangle} \hat{b}_{\qq_i})\hat{b}_\kk$.
In the presence of the thermal bath, (i) the numerator $\langle\ff|\mH_3^{(\pm)}|\lambda\rangle\langle\lambda|\mH_3^{(\pm)}|\ii\rangle=\langle\ff|W_2|\lambda\rangle\langle\lambda|W_1|\ii\rangle$ becomes $\mathcal{A}_1\mathcal{A}_2(1+n_\kk)$, (ii) the new associated process, corresponding to the inverse order of action of $W_1$ and $W_2$, has numerator $\langle\ff|W_1|\lambda'\rangle\langle\lambda'|W_2|\ii \rangle=\mathcal{A}_1\mathcal{A}_2n _\kk$, (iii) the new intermediate state $|\lambda'\rangle=|\ii_1\rangle\otimes|\ff_2\rangle\otimes|n_\kk-1: \kk\rangle$ contains the phonon $\qq$ so its energy denominator must be taken from the second sum of (\ref{eq258}), and happens to be the exact opposite of the denominator of the former process, the calculation giving $-\zeta+E_\ff-E_{\lambda'}=-(\zeta+E_\ii-E_\lambda)$. Ultimately, the linear contributions in $n_\kk$ of the two processes exactly cancel each other in the transition amplitude $\mathcal{A}_{\ii\to\ff}(\qq,z)$, which allows us to forget the thermal background as is done in (\ref{eq258}). Similar reasoning, but limited to $2\to 2$ processes on the energy shell, can already be found in reference \cite{Annalen}.
}}
\begin{equation}\label{eq258}
\boxed{\mathcal{A}_{\ii\to\ff}(\qq,z) = \sum_{|\lambda\rangle\, / \, \qq\notin|\lambda\rangle} \frac{\langle \ff| \mH_3^{(\pm)} |\lambda\rangle \langle\lambda| \mH_3^{(\pm)}|\ii\rangle}{\zeta+E_\ii-E_\lambda} 
+
\sum_{|\lambda\rangle\, / \, \qq\in|\lambda\rangle} \frac{\langle \ii| \mH_3^{(\pm)} |\lambda\rangle \langle\lambda| \mH_3^{(\pm)}|\ff\rangle}{-\zeta+E_\ff-E_\lambda}} 
\end{equation} From expression (\ref{eq258}) we derive the transition amplitude for process $1\to 3$ (knowing that $\qq'_1+\qq'_2+\qq'_3=\qq$): 
\begin{multline}
\label{eq260}
\mathcal{A}_{\qq\to\qq'_1,\qq'_2,\qq'_3}(\qq,\zeta)= \frac{\mA(\qq'_1,\qq'_2;\qq'_1+\qq'_2)\mA(\qq'_1+\qq'_2,\qq'_3;\qq)}{\zeta+\epsilon_\qq-(\epsilon_{\qq'_1+\qq'_2}+\epsilon_{\qq'_3})} +\frac{\mA(\qq'_2,\qq'_3;\qq'_2+\qq'_3)\mA(\qq'_2+\qq'_3,\qq'_1;\qq)}{\zeta+\epsilon_\qq-(\epsilon_{\qq'_2+\qq'_3}+\epsilon_{\qq'_1})} \\ +\frac{\mA(\qq'_1,\qq'_3;\qq'_1+\qq'_3)\mA(\qq'_1+\qq'_3,\qq'_2;\qq)}{\zeta+\epsilon_\qq-(\epsilon_{\qq'_1+\qq'_3}+\epsilon_{\qq'_2})}
\end{multline}
 then for process $2\to 2$ (knowing that $\qq'_1+\qq'_2=\qq+\qq_2$): 
\begin{multline}
\label{eq262}
\mathcal{A}_{\qq,\qq_2\to\qq'_1,\qq'_2}(\qq,\zeta)= \frac{\mA(\qq'_1,\qq'_2;\qq'_1+\qq'_2)\mA(\qq,\qq_2;\qq+\qq_2)}{\zeta+\epsilon_{\qq}+\epsilon_{\qq_2}-\epsilon_{\qq+\qq_2}} -\frac{\mA(\qq'_1-\qq,\qq'_2;\qq_2)\mA(\qq'_1-\qq,\qq;\qq'_1)}{\zeta+\epsilon_{\qq}+\epsilon_{\qq'_1-\qq}-\epsilon_{\qq'_1}} \\ +\frac{\mA(\qq'_2-\qq_2,\qq_2;\qq'_2)\mA(\qq'_2-\qq_2,\qq'_1;\qq)}{\zeta+\epsilon_{\qq}-(\epsilon_{\qq'_1}+\epsilon_{\qq'_2-\qq_2})} +\frac{\mA(\qq-\qq'_2,\qq_2;\qq'_1)\mA(\qq-\qq'_2,\qq'_2;\qq)}{\zeta+\epsilon_{\qq}-(\epsilon_{\qq'_2}+\epsilon_{\qq-\qq'_2})}-\frac{\mA(\qq'_2-\qq,\qq'_1;\qq_2)\mA(\qq'_2-\qq,\qq;\qq'_2)}{\zeta+\epsilon_{\qq}+\epsilon_{\qq'_2-\qq}-\epsilon_{\qq'_2}}
\end{multline}
 and finally for process $3\to 1$ (knowing that $\qq_1'=\qq+\qq_2+\qq_3$): 
\begin{multline}
\label{eq264}
\mathcal{A}_{\qq,\qq_2,\qq_3\to \qq_1'}(\qq,\zeta)= -\frac{\mA(\qq_2,\qq_3;\qq_2+\qq_3)\mA(\qq_2+\qq_3,\qq;\qq_1')}{\zeta+\epsilon_{\qq}+\epsilon_{\qq_2+\qq_3}-\epsilon_{\qq_1'}}+\frac{\mA(\qq+\qq_2,\qq_3;\qq_1')\mA(\qq,\qq_2;\qq+\qq_2)}{\zeta+\epsilon_{\qq}+\epsilon_{\qq_2}-\epsilon_{\qq+\qq_2}}\\ +\frac{\mA(\qq+\qq_3,\qq_2;\qq_1')\mA(\qq,\qq_3;\qq+\qq_3)}{\zeta+\epsilon_{\qq}+\epsilon_{\qq_3}-\epsilon_{\qq+\qq_3}}
\end{multline}
 In these equations (\ref{eq260},\ref{eq262},\ref{eq264}), the three contributions where we had to choose the opposite direction in (\ref{eq253}), i.e. take $E_\ff$ rather than $E_\ii$ as the reference energy, are easy to spot as they start with a minus sign. 

Our prescription (\ref{eq257},{\yc\ref{eq258}}) gives simple but generally different results from the exact fourth-order diagrammatic theory, given the expressions for Matsubara double sums in \ref{app3}. Remarkably, however, once these sums have been replaced by their simplified expressions (\ref{eq213},\ref{eq233}), which is legitimate at small angles, it is in perfect agreement with the value of $\Sigma^{(4)}$ predicted by the diagrams in Figure \ref{fig8}; this can be verified with the aid of the correspondence table \ref{table1}, obtained from the coupling amplitudes in (\ref{eq212},\ref{eq232}) and the energy denominators in (\ref{eq213},\ref{eq233}), and which makes it possible to associate with each internal line on this figure an incoming $\qq_i$, outgoing $\qq'_j$ or virtual (linear combination of the previous ones) phonon of a $n\to n'$ process, thus giving a clear physical meaning to the corresponding diagram.\footnote{{\ycd The verification is facilitated by the following result: according to our Landau prescription, each diagrammatic contribution to the Matsubara sums (\ref{eq213},\ref{eq233}) is given by $\sigma/\prod_j (\zeta-\zeta_j)$ with $\sigma=s(f_1 f_2-f_3f_4)$, where (i) the global sign $s$ is the product of the orientations $s_i$ of the virtual phonon lines (indeed, if a virtual phonon goes back in time, its energy appears with a plus sign in the denominator in the corresponding $(\zeta-\zeta_j)$ factor, so it contributes through the second sum of (\ref{eq258}) and a global minus sign must be included), (ii) each $f_k$ is the product of the bosonic amplification factors of the lines connected to the vertex $k$ (the line $i$ contributes to $f_k$ by a factor $\bar{n}_{\veps_i}$ if it goes towards the vertex, or by a factor $1+\bar{n}_{\veps_i}$ if it goes away from it, which is written $s_i(1+\bar{n}_{s_i\veps_i})$ if the line is to the right of the vertex and $s_i\bar{n}_{s_i\veps_i}$ if it is to the left), excluding, however, external lines and virtual phonon lines, whose occupation numbers do not appear in the prescription (\ref{eq257},\ref{eq258}). Diagram vertices are numbered from left to right; in the case of bridge diagrams, the degeneracy of internal vertex numbering is lifted by rotating the vertical phonon line by a small angle to the left (type I, effective orientation $s_5$) or to the right (type II, effective orientation $-s_5$). One thus finds, all signs factorized, $\sigma_B=s_1s_2s_3s_4s_5[(1+\bar{n}_{s_1\veps_1})(1+\bar{n}_{s_3\veps_3})(1+\bar{n}_{s_4\veps_4})- \bar{n}_{s_1\veps_1}\bar{n}_{s_3\veps_3}\bar{n}_{s_4\veps_4}]$ for inner-loop diagrams ($s=s_2s_5$), $\sigma_{P^{\rm I}}=s_1s_2s_3s_4s_5[(1+\bar{n}_{s_2\veps_2})(1+\bar{n}_{s_3\veps_3})(1+\bar{n}_{s_5\veps_5})-\bar{n}_{s_2\veps_2} \bar{n}_{s_3\veps_3} \bar{n}_{s_5\veps_5}]$ and $\sigma_{P^{\rm II}}=s_1s_2s_3s_4(-s_5)[(1+\bar{n}_{s_1\veps_1})(1+\bar{n}_{s_4\veps_4})(1+\bar{n}_{-s_5\veps_5})-\bar{n}_{s_1\veps_1} \bar{n}_{s_4\veps_4} \bar{n}_{-s_5\veps_5}]$ for bridge diagrams ($s=s_1s_4$ for type I, $s=s_2s_3$ for type II). We then find exactly the simplified forms (\ref{eq213},\ref{eq233}) if we use energy conservation at the second vertex, respectively $s_2\veps_2=s_3\veps_3+s_4\veps_4$, $s_1\veps_1=s_3\veps_3+s_5\veps_5$ and $s_2\veps_2=s_4\veps_4+(-s_5)\veps_5$, through the identity $\bar{n}_{a+b}=\bar{n}_a\bar{n}_b/(1+\bar{n}_a+\bar{n}_b)$, and the equality $s_2=s_5$ in the inner-loop diagrams.}} Ultimately, the prescription (\ref{eq257},{\yc\ref{eq258}}) leads in $d=2$ to a rigorously equivalent but much more compact expression of $\Sigma^{(4)}$ at leading order in $\epsilon$: 
\begin{equation}\label{eq269}
\boxed{
\tilde{\Sigma}_{\qb}^{(4,2)}(\tilde{\zeta})=\frac{1}{3}\int\!\!\!\!\int_{\mathbb{T}_{\qb}} \frac{\qb'_1\mathrm{d}\qb'_1}{(2\pi)^2} \frac{\qb'_2\mathrm{d}\qb'_2}{(2\pi)^2} \Phi(\qb'_1,\qb'_2) + \int\!\!\!\!\int_{\mathbb{R}^{+2}\setminus\mathbb{T}_{\qb}} \frac{\qb'_1\mathrm{d}\qb'_1}{(2\pi)^2} \frac{\qb'_2\mathrm{d}\qb'_2}{(2\pi)^2} \Phi(\qb'_1,\qb'_2) + \int\!\!\!\!\int_{\mathbb{R}^{+2}}\frac{\qb_2\mathrm{d}\qb_2}{(2\pi)^2}\frac{\qb_3\mathrm{d}\qb_3}{(2\pi)^2} \Phi(-\qb_2,-\qb_3)}
\end{equation} where we have kept the order of processes $1\to 3$, $2\to 2$ and $3\to 1$, $\mathbb{T}_{\qb}$ is the triangle $\{(\qb'_1,\qb'_2)\in\mathbb{R}^{+2}\;|\; \qb'_1+\qb'_2\leq \qb\}$ and, as before, the tilde on $\Sigma_{\qb}$ indicates a division by $k_B T\epsilon^2$. One and the same function is involved in the integrand of the $n\to n'$ contributions, only the integration domain changes with the process under consideration: \footnote{As far as the bosonic factors are concerned, this is shown by making use of the relation $\bar{n}_{-k}^{\rm lin}=-(1+\bar{n}_k^{\rm lin})$, $\forall k\in\mathbb{R}^*$.} 
\begin{multline}
\label{eq270}
\Phi(\qb'_1,\qb'_2)=\frac{1}{2} \bar{q}\left(\frac{9(1+\Lambda)^2}{8\rho\xi^2}\right)^2 \qb'_1 \qb'_2 (\qb'_1+\qb'_2-\qb) [(1+\bar{n}^{\rm lin}_{q'_1}+\bar{n}^{\rm lin}_{q'_2})\bar{n}^{\rm lin}_{q'_1+q'_2-q}-\bar{n}^{\rm lin}_{q'_1}\bar{n}^{\rm lin}_{q'_2}] \\ \times \int_{\mathbb{R}}\dd\thetat_1\int_{\mathbb{R}}\dd\thetat_2\frac{\left\{\frac{\qb'_1+\qb'_2}{\zetat+\qb(\qb'_1+\qb'_2)(\qb'_1+\qb'_2-\qb)\left[\frac{(\qb'_1\thetat_1+\qb'_2\thetat_2)^2}{2(\qb'_1+\qb'_2-\qb)^2(\qb'_1+\qb'_2)^2}-\frac{3\gamma}{8}\right]}+\frac{\qb-\qb'_1}{\zetat+\bar{q}\qb'_1(\qb-\qb'_1)\left[-\frac{\thetat_1^2}{2(\qb'_1-\qb)^2}+\frac{3\gamma}{8}\right]}+\frac{\qb-\qb'_2}{\zetat+\bar{q}\qb'_2(\qb-\qb'_2)\left[-\frac{\thetat_2^2}{2(\qb'_2-\qb)^2}+\frac{3\gamma}{8}\right]}\right\}^2}{\tilde{\zeta}- \frac{\qb'_1(\qb'_2-\qb)\thetat_1^2+\qb'_2(\qb'_1-\qb)\thetat_2^2-2\qb'_1 \qb'_2\thetat_1 \thetat_2}{2(\qb'_1+\qb'_2-\qb)}+\frac{3\gamma}{8} (\qb'_1+\qb'_2)(\qb'_1-\qb)(\qb'_2-\qb)}
\end{multline}

 \begin{table}[t]
\centerline{\small
\begin{tabular}{|c|c|c|c|c|c|}
$B_1$ & all the squares of $1\to 3$ & $P_{1}^{\rm I}$  &  all crossed terms of $1\to 3$ & $P_{1}^{\rm II}$ & $3^{\rm rd}$ term $\times$ $4^{\rm th}$ term of $2\to 2$\\
$B_2$ & squares of $3^{\rm rd}$ and $4^{\rm th}$ terms of $2\to 2$ & $P_{2}^{\rm I}$ & $1^{\rm st}$ term $\times$ $3^{\rm rd}$ term of $2\to 2$ & $P_{2}^{\rm II}$ & $3^{\rm rd}$ term $\times$ $5^{\rm th}$ term of $2\to 2$\\
$B_3$ & $2^{\rm nd}$ term $\times$ $3^{\rm rd}$ term of $2\to 2$ & $P_{3}^{\rm I}$ & $1^{\rm st}$ term $\times$ $4^{\rm th}$ term of $2\to 2$ & $P_{3}^{\rm II}$ & $2^{\rm nd}$ term $\times$ $4^{\rm th}$ term of $2\to 2$\\
$B_4$ & $4^{\rm th}$ term $\times$ $5^{\rm th}$ term of $2\to 2$ & $P_{4}^{\rm I}$ & $1^{\rm st}$ term $\times$ $2^{\rm nd}$ term of $2\to 2$ & $P_{4}^{\rm II}$ & $1^{\rm st}$ term $\times$ $3^{\rm rd}$ term of $3\to 1$\\
$B_5$ & squares of $2^{\rm nd}$ and $5^{\rm th}$ terms of $2\to 2$ & $P_{5}^{\rm I}$ & $2^{\rm nd}$ term $\times$ $3^{\rm rd}$ term of $3\to 1$ & $P_{5}^{\rm II}$ & $2^{\rm nd}$ term $\times$ $5^{\rm th}$ term of $2\to 2$\\
$B_6$ & square of $1^{\rm st}$ term of $3\to 1$ & $P_{6}^{\rm I}$ & $1^{\rm st}$ term $\times$ $5^{\rm th}$ term of $2\to 2$ & $P_{6}^{\rm II}$ & $1^{\rm st}$ term $\times$ $2^{\rm nd}$ term of $3\to 1$\\
$B_7$ & square of $1^{\rm st}$ term of $2\to 2$ & & & & \\
$B_8$ & squares of $2^{\rm nd}$ and $3^{\rm rd}$ terms of $3\to 1$ & & & & 
\end{tabular}
}
\caption{In the calculation of the self-energy at fourth-order in $H_3$, correspondence between the Feynman diagrams in Figure \ref{fig8} and the square or crossed terms (double products) of the quantities $\mA^2_{\ii\to\ff}(\qq,z)$ involved in Landau prescription (\ref{eq257}); the various terms of the $\mA_{\ii\to\ff}(\qq,z)$ transition amplitudes referred to are those of equations (\ref{eq260},\ref{eq262},\ref{eq264}) written in that order. For bridge diagrams, it was necessary to distinguish between type I and type II, as in (\ref{eq233}). The correspondence holds after integration on the wave vectors as in (\ref{eq257}), which makes, for example, the squares of each of the terms of (\ref{eq260}), which differ by a permutation of $\qq_j'$, or the double products of $2^{\rm nd}$ and $4^{\rm th}$ terms and $5^{\rm th}$ and $3^{\rm rd}$ terms of (\ref{eq262}), which are mapped one onto the other by exchange of $\qq_1'$ and $\qq_2'$, give the same contribution.}
\label{table1}
\end{table}
 
\subsection{Failure of order four in the limit $\epsilon=k_B T/mc^2\to 0$}
\label{sec3.3}

\paragraph{An unexpected divergence}
 As a first simple application of the expansion of the self-energy to order four in $H_3$, we had decided to calculate {\yct $\tilde{\Sigma}_{\qb}^{(4,2)}(\textrm{i}0^+)$} numerically\footnote{The angular integrals are calculated analytically, only the integration over the wavenumbers is done by computer.} to see if, through a better approximation of $\Sigma_{\qq}(\textrm{i}0^+)$ in equation (\ref{eq014}), this would bring the pole approximation closer to the numerical results by reducing the deviation in Figure \ref{fig2}d. To our surprise, we discovered that $\tilde{\Sigma}_{\yc\qb}^{(4,2)}(\ii\delta)$ diverges as $1/\ii\delta$ when $\delta\to 0^+$! We then confirmed this result analytically, showing that\footnote{To facilitate the calculation of angular integrals, we have established (\ref{eq280}) first in the case where $\zetat\to 0$ by positive imaginary parts, then extended to the case $\im\zetat<0$ as in footnote \ref{notegen}.} 
\begin{equation}\label{eq280}
\boxed{\tilde{\Sigma}_{\qb}^{(4,2)}(\zetat)\underset{\zetat\to 0}{\sim} \frac{-2}{3\pi^2\gamma\zetat} \left(\frac{9(1+\Lambda)^2}{8\rho\xi^2}\right)^2 \int_{\qb}^{\eta} \kb\dd\bar{k}\int_0^{\eta-\kb} \kb'\dd\kb' (\kb-\qb)(\qb+\kb')(\bar{n}_{k-q}^{\rm lin}-\bar{n}_{k}^{\rm lin})(\bar{n}_{k'+q}^{\rm lin}-\bar{n}_{k'+k}^{\rm lin})}
\end{equation} This divergence comes from the angular integral $K(\kb,\kb')$ (\ref{eq243}) of bridge diagram $P_5$. Let's analyze the behavior of (\ref{eq243}) in the $\zetat\to 0$ limit for any orientation $s_j$ of the internal lines. If it is divergent, it is necessarily dominated by a neighborhood of the cancellation locus of the integrand denominator in the $(\thetat,\thetat')$ plane for $\zetat=0$. As the curvature parameter $\gamma$ is $>0$, this locus is the union of the two vertical lines $\thetat=\veps(-A_1^P/B_1^P)^{1/2}$ ($\veps=\pm 1$) where the first factor in the denominator vanishes, the two horizontal lines $\thetat'=\veps'(-A_2^P/B_2^P)^{1/2}$ ($\veps'=\pm 1$) where the second factor vanishes, and the conic making the third factor vanish. The unfortunate thing is that the conic always passes through the intersection points $P_{\veps,\veps'}\equiv(\veps(-A_1^P/B_1^P)^{1/2},\veps'(-A_2^P/B_2^P)^{1/2})$ of the horizontal and vertical lines, whatever the wave numbers $\kb$, $\kb'$ and $\qb$, on the sole condition that $\veps\veps'=s_2 s_4$. This follows from the identity on the coefficients (\ref{eq244},\ref{eq245}): 
\begin{equation}\label{eq281}
4A_1^P A_2^P B_1^P B_2^P (B_3^P)^2 = [A_2^P B_1^P(B_2^P+B_3^P)+A_1^P B_2^P B_3^P-A_3^P B_1^P B_2^P]^2
\end{equation} Since the angular integral is two-dimensional, a possible divergent part $K_{\rm div}(\kb,\kb')$ of $K(\kb,\kb')$ can only come from a small neighborhood of the triple intersection points $P_{s_2,s_4}$ and $P_{-s_2,-s_4}$, where we replace each factor in the denominator by its linear approximation in $\tau=\thetat-\veps(-A_1^P/B_1^P)^{1/2}$ and $\tau'=\thetat'-\veps'(-A_2^P/B_2^P)^{1/2}$. This leaves 
\begin{equation}\label{eq282}
K_{\rm div}(\kb,\kb')=\int_{\mathbb{R}^2} \frac{2\mathrm{d}\tau\mathrm{d}\tau'}{[\zetat-(\frac{3\gamma}{4})^{1/2}s_1\qb\bar{k}\tau][\zetat-(\frac{3\gamma}{4})^{1/2}s_3\qb\kb' \tau']\{\zetat-(\frac{3\gamma}{4})^{1/2}s_3\kb'[s_1\bar{k}\tau +(\qb-s_1\kb) \tau']\}}=- \frac{8\pi^2(1-s_2)(1-s_1s_3)}{3\gamma\tilde{\zeta}\qb\kb\kb' (s_1\kb-s_3\kb')}
\end{equation} where we have been able to extend the integral to the whole $\mathbb{R}^2$ since it converges to infinity, and have been able to restrict ourselves to the point $P_{s_2,s_4}$ (at the cost of adding a global factor $2$) by parity invariance $(\thetat,\thetat')\to (-\thetat,-\thetat')$. In the right-hand side of (\ref{eq282}), we have explicitly calculated the integral over $\tau'$ and then over $\tau$ by the residue method (the integration path must be closed by a semicircle at infinity in the upper or lower half-plane containing the minimum number of poles in view of constraints (\ref{eq239})). The divergent part is therefore non-zero only if $s_2=s_1s_3=-1$, i.e. if the lower left branch of the bridge loop is oriented to the left and the two upper branches have opposite orientations, which selects as announced diagram $P_5$ in Figure \ref{fig8}. {\yct Inserting} the corresponding value of $K_{\rm div}(\kb,\kb')$ {\yct into} (\ref{eq242}) yields equation (\ref{eq280}). The same technique applied to type I contribution of bridge diagrams leads in (\ref{eq241}) to {\yc a divergent part of the form}\footnote{In fact, the coefficients also satisfy the relationship obtained by exchanging indices $1$ and $2$ and replacing index $3$ by index $4$ in (\ref{eq281}).}\footnote{In the case of inner-loop diagrams, the existence of a triple nodal point is less surprising since the denominator of (\ref{eq221}) comprises a squared trinomial. In fact, there are four of them, playing equivalent roles. For the sign choices $\veps=s_1 s_2$ and $\veps'=s_2 s_3 s_4$, the coefficients of $\tau$ and $\tau'$ in the linearized forms are all negative; setting them to $-1$ and taking $\zetat=\ii\delta$ to simplify, we get $\int_{[-1,1]^2} 4\mathrm{d}\tau\mathrm{d}\tau'/[(\ii\delta-\tau)^2 (\ii\delta -\tau-\tau')]$, which is $O(1)$ when $\delta\to 0^+$.} 
\begin{equation}\label{eq283}
J_{\rm div}(\kb,\kb')=\int_{\mathbb{R}^2} \frac{2\mathrm{d}\tau\mathrm{d}\tau'}{[\zetat-(\frac{3\gamma}{4})^{1/2}s_1\qb\bar{k}\tau][\zetat-(\frac{3\gamma}{4})^{1/2}s_3\qb\kb' \tau']\{\zetat-(\frac{3\gamma}{4})^{1/2}s_1\kb[(\qb-s_3\kb')\tau +s_3\kb'\tau']\}}=\frac{8\pi^2(1-s_1)(1-s_2s_4)}{3\gamma\tilde{\zeta}\qb\kb\kb'(s_1\kb-s_3\kb')}
\end{equation} and concludes that there is no divergence ({\yc to have $J_{\rm div}\neq 0$,} the upper left branch of the bridge loop would have to be oriented to the left, and the two lower branches would have to be of opposite orientations, which does not occur in Figure \ref{fig8}).

\paragraph{Physical implications}
 From the expansion of the self-energy to order four in $H_3$, and still to leading order (two) in $\epsilon$, we form a new approximation to {\yct the} Green's function, 
\begin{equation}\label{eq289}
\tilde{\mathcal{G}}_{\qb}^{(\leq 4,2)}(\zetat)=\frac{1}{\zetat-\tilde{\Sigma}_{\qb}^{(2,2)}(\zetat)-\tilde{\Sigma}_{\qb}^{(4,2)}(\zetat)}
\end{equation} One might think that it is more accurate and performs better than the one in section \ref{sec2}. Unfortunately, because of the divergence (\ref{eq280}), this is not the case: in the weakly interacting regime, which is the most favorable, it has a pole $\zetat^{n\phi}_{\qb}$ in the upper half-plane (and its complex conjugate in the lower half-plane), in the absence, it should be pointed out, of any analytic continuation, hence the exponent $n\phi$ (\g{non physical}). {\yc This} violates the fundamental analyticity property of {\yct the} Green's function on $\mathbb{C}\setminus\mathbb{R}$ \cite{FW}, and leads, for example, to a signal diverging exponentially in time in (\ref{eq012}) if path $C_+$ passes above $\zetat^{n\phi}_{\qb}$, or not starting at one if it passes below. To show the existence of this pole, let's reduce the implicit equation 
\begin{equation}\label{eq290}
\zetat_{\qb}^{n\phi} = \tilde{\Sigma}_{\qb}^{(2,2)}(\zetat_{\qb}^{n\phi})+\tilde{\Sigma}_{\qb}^{(4,2)}(\zetat_{\qb}^{n\phi})
\end{equation} to its leading order, assuming that $\zetat_{\qb}^{n\phi}$ tends to zero as $(\rho\xi^2)^{-1}$ (by positive imaginary parts). Since $\tilde{\Sigma}_{\qb}^{(2,2)}(\zetat)$ is already of order one in $(\rho\xi^2)^{-1}$, and is a derivable function of $\zetat$ at $\textrm{i}0^+$ (see \ref{app1}), we can replace it by its value at this point. In the case of $\tilde{\Sigma}_{\qb}^{(4,2)}(\zetat)$, we keep its divergent part (\ref{eq280}), as the second-order prefactor $(\rho\xi^2)^{-2}$ it contains is promoted to first order by division by $\zetat$, but we neglect its regular part, of second-order near $\textrm{i}0^+$. By making the universalizing change of variables (\ref{eq112},\ref{eq130},\ref{eq131}) {\yct as in} section \ref{sec2}, we are left with 
\begin{equation}\label{eq291}
\zetac^{n\phi}_{\qb} = \check{\Sigma}_{\qb}^{(2,2)}(\textrm{i}0^+)-\frac{(2/\qb^{2})\int_{\qb}^{\eta}\kb\,\dd\bar{k}\int_0^{\eta-\kb}\kb'\dd\kb'(\kb-\qb)(\kb'+\qb){\yc(\bar{n}_{k-q}^{\rm lin}-\bar{n}_{k}^{\rm lin})(\bar{n}_{k'+q}^{\rm lin}-\bar{n}_{k'+k}^{\rm lin})}}{\zetac^{n\phi}_{\qb}}
\end{equation} where the first term in the right-hand side is given in (\ref{eq141}). Since the integral in the numerator is $>0$, equation (\ref{eq291}) on $\zetac^{n\phi}_{\qb}$ has a pure imaginary solution in the upper half-plane.\footnote{Let us set $\zetac^{n\phi}_{\qb}=\textrm{i}y$. Then {\yc(\ref{eq291}) is written as} $y=-2B+A/y$ where the coefficients $A$ and $B$ are real $>0$. The resulting second-degree equation on $y$ has one (and only one) positive root, $y_0=A/[(A+B^2)^{1/2}+B]$. When $\qb\to 0$, $B$ has a finite, non-zero limit in view of (\ref{eq141}), then $A$ diverges as $1/\qb$, so that $\zetac^{n\phi}_{\qb}$ diverges as $\qb^{-1/2}$ and $\zetat_{\qb}^{n\phi}$ tends to zero as $\qb^{1/2}$. Thus, condition $\im\zetat^{n\phi}_{\qb}\lll|\tilde{\Sigma}_{\qb}^{(2,2)}(\textrm{i}0^+)|=O(\qb)$, which makes the non-physical pole without effect on the signal in its time half-width, cannot be satisfied at wavenumber $\qb\lesssim 1$. It can be satisfied at large wavenumber $\qb\gg 1$, where $B\approx \qb^2$ in view of (\ref{eq141}) and $A\sim 2\exp(-\qb)$.} For the sake of completeness, let's point out that $\tilde{\Sigma}_{\qb}^{(4,2)}(\zetat)$ has another linear divergence on the real axis, at $\zetat_0=-3\gamma\qb^3/32$,\footnote{Note that $\zetat_0$ is also the branch point of $\tilde{\Sigma}_{\qb}^{(2,2)}(\zetat)$, see section \ref{sec2.1} and footnote \ref{note30}.} as shown by \ref{app4}, which is less damaging as it does not imply the existence of a non-physical pole of {\yct the} Green's function in the weakly interacting limit (it suffices to have $|\tilde{\Sigma}_{\qb}^{(2,2)}(\textrm{i}0^+)|\lll|\re\zetat_0|$ for this divergence to have little effect on signal damping in its half-width). 

\paragraph{Applicability of order four}
 Does the appearance of a non-physical pole force us to question the validity of the diagrammatic expansion and, by extension, that of the many-body Green's function method? No. The theory was intended to be used in the $(\rho\xi^2)^{-1}\to 0$ weakly interacting limit at fixed reduced temperature $\epsilon=k_B T/mc^2$. However, we took the limit $\epsilon\to 0$ with a fixed order of expansion in $(\rho\xi^2)^{-1}$, which fails in dimension $d=2$. Let's show that everything is fine, i.e. the non-physical pole disappears, when $\rho\xi^2\epsilon^2$ is large enough (at fixed $\qb$). 

To this end, let's see how the local analysis leading to (\ref{eq282}) is modified when $\epsilon$ is small but not infinitesimal, by restricting ourselves directly to diagram $P_5$, i.e. taking $s_1=s_4=s_5=1$, $s_2=s_3=-1$. If $\zetat=0$, type II energy denominator in (\ref{eq233}), which is also that of (\ref{eq282}), is written $(-\zeta_1)(-\zeta_2)(-\zeta_3)$, where the energy defects $\zeta_i$ are those in equation (\ref{eq238}). The first factor $-\zeta_1=-(\veps_{\kk}-\veps_{\kk-\qq}-\veps_{\qq})$ is a function of the single angle $\theta=\epsilon\thetat$ between $\qq$ and $\kk$, and continues to vanish on two vertical lines in the plane $(\thetat,\thetat')$, at slightly modified abscissas, which negligibly changes (by a factor $1+O(\epsilon)^2$) the coefficient of $\tau$ in (\ref{eq282}); the second factor $-\zeta_2=-(-\veps_{\kk'}+\veps_{\qq+\kk'}-\veps_{\qq})$ is a function of the single angle $\theta'=\epsilon\thetat'$ between $\qq$ and $\kk'$, and continues to vanish on two horizontal lines, slightly displaced, with the same negligible consequences on the coefficient of $\tau'$ in (\ref{eq282}). On the other hand, the cancellation locus of the third factor, namely $-\zeta_3=-(\veps_{\kk}+\veps_{\qq+\kk'}-\veps_{\kk+\kk'}-\veps_{\qq})$, is no longer exactly a conic in the $(\thetat,\thetat')$ plane and now passes slightly to the side of the intersection points $P_{-1,1}$ and $P_{1,-1}$ of the horizontal and vertical lines: this changes very slightly the coefficients of the linear combination of $\tau$ and $\tau'$ in (\ref{eq282}), but above all adds, next to $\zetat$, the now non-zero value of $-\zetat_3$ in $P_{-1,1}$: 
\begin{equation}\label{eq292}
(-\zetat_3)_{P_{-1,1}} \underset{\epsilon\to 0}{\sim} -\frac{27}{64}\gamma^2\epsilon^2\kb\kb'(\kb-\qb)(\kb+\kb')(\kb'+\qb)
\end{equation} We recalculate the right-hand side of (\ref{eq282}) accordingly: 
\begin{equation}\label{eq293}
K_{\rm div}^{P_5}(\kb,\kb')=-\frac{32\pi^2}{3\gamma\qb\kb\kb'(\kb+\kb')} \frac{1}{\zetat-\frac{27}{64}\gamma^2\epsilon^2\qb\kb\kb'(\kb-\qb)(\kb'+\qb)}
\end{equation} The new equation on the non-physical pole 
\begin{equation}\label{eq294}
\zetac^{n\phi}_{\qb} = \check{\Sigma}_{\qb}^{(2,2)}(\textrm{i}0^+)-\frac{2}{\qb^{2}}\int_{\qb}^{\eta}\kb\,\dd\bar{k}\int_0^{\eta-\kb}\kb'\dd\kb' \frac{(\kb-\qb)(\kb'+\qb){\yc(\bar{n}_{k-q}^{\rm lin}-\bar{n}_{k}^{\rm lin})(\bar{n}_{k'+q}^{\rm lin}-\bar{n}_{k'+k}^{\rm lin})}}{\zetac^{n\phi}_{\qb}-\frac{9}{8}\gamma u \epsilon^2\kb\kb'(\kb-\qb)(\kb'+\qb)}
\end{equation} ceases to have a solution in the upper half-plane when $\gamma u\epsilon^2>h(\qb)$, where $u$ is the universal parameter (\ref{eq112}) and $h(\qb)$ is a function exclusively of $\qb$ (and of the cut-off $\eta$ if it is not sent to infinity in quantum theory), as can be seen by formally taking the limit $\gamma u \epsilon^2\to +\infty$ in (\ref{eq294}). For the classical field, we find numerically that $h(\qb)$ is a decreasing function of $\qb$; the problem of applicability of fourth-order theory for small values of $\epsilon$ therefore remains in the limit $\qb \to 0$.\footnote{{\ycd For the classical field ($\eta=1$), we have the equivalent $h(\qb)\sim \frac{8}{9}(\pi^2/2\qb)^{4/3}$ when $\qb \to 0$ (in practice $\qb^{1/3}\ll 1$). This is obtained by writing the right-hand side of (\ref{eq294}) to leading order in $\qb$ (in particular, we replace $\qb$ by zero in the first integration bound and in the denominator of the integrand, and replace the numerator by its linear approximation in $\qb$), then showing that the double integral is dominated by a neighborhood of $(\kb,\kb')=(0,0)$ of radius $h(\qb)^{-1/4}\ll 1$ when $\qb \to 0$, which allows us to replace the upper integration bounds by $+\infty$. The result is an analytically calculable double integral and the equation $\zetac^{n\phi}_{\qb}=-\ii+(1+\ii)\pi^2/[2\qb(\frac{9}{8}\gamma u\epsilon^2)^{3/4}(\zetac^{n\phi}_{\qb})^{1/4}]$. It remains to express the fact that the non-physical pole has a zero imaginary part when it disappears.}} {\ycd To see this more simply, we can evaluate $\check{\Sigma}_{\qb}^{(4)}(\ii 0^+)$ (fourth-order in $H_3$ contribution considered at all orders in $\epsilon$) in the limit $\epsilon\to 0$. All we need to do is replace $\check{\zeta}_{\qb}^{n\phi}$ by $\ii 0^+$ in the double integral in the right-hand side of (\ref{eq294}), which we can then calculate explicitly for the quantum field (cutoff $\eta\to+\infty$):
\be
\label{eq296}
\check{\Sigma}_{\qb}^{(4)}(\ii 0^+)\underset{\epsilon\to 0}{\sim} \frac{16 g_2(\eee^{-\qb})}{9\gamma u \epsilon^2 \qb^2} \quad \mbox{hence} \quad \Sigma_{\qb}^{(4)}(\ii 0^+)\underset{\epsilon\to 0}{\sim} k_B T \left[\frac{(1+\Lambda)^2}{\gamma^{3/2}\rho\xi^2}\right]^2 \frac{2 g_2(\eee^{-\qb})}{\pi^2\qb}
\ee
At low temperature $\epsilon \ll 1$, the fourth order in $H_3$ (\ref{eq296}) is therefore a small correction to the second order (\ref{eq141}) only if $\gamma u\epsilon^2$ is much larger than a certain function of $\qb$, namely $1/\qb^2$ at low $\qb$. \footnote{{\yc The result (\ref{eq296}) does not contradict equation (\ref{eq280}) because the order of the limits $\zetat\to 0$ and $\epsilon\to 0$ is not the same.}}}

\paragraph{Conclusion}
 At a fixed reduced wave number $\qb$ of the phonon mode {\yc whose damping is being studied}, the condition for the applicability of diagrammatic theory at order four in $H_3$ is not only that the universal parameter $u$ is large enough (the expected weakly interacting regime) but also that the product ${\yc\gamma}u\epsilon^2$ is large enough. The perturbative approach therefore presents a fundamental incompatibility with the limit studied in this article, the low-temperature $\epsilon\to 0$ limit with fixed interaction strength. In particular, for the numerical simulations in Figure \ref{fig2}, where $\qb=1/4$ and $\gamma=1$, we find, by following with Newton's method the root of equation (\ref{eq294}), written for the classical field, until it disappears, that we should have $u\epsilon^2>10.6$ or $\rho\xi^2>23.4$ for $\epsilon=1/2$ and $\rho\xi^2>52.6$ for $\epsilon=1/3$. However, $\rho\xi^2$ is at most $10$ in Figure \ref{fig2}; the fourth-order theory developed in this section is therefore inapplicable.

\section{Nonperturbative theories in $H_3$}
\label{sec4}

As we saw in section \ref{sec3}, the theory at fourth order in $H_3$ is not usable in dimension $d=2$ in the low-temperature limit $\epsilon=k_B T/mc^2\to 0$. To account for the discrepancy between the theory at second order in $H_3$ and simulations on the decay time at $1/\eee$ of the phonon mode $\qq$ and on the real part of the associated signal (\ref{eq012}) in the weakly interacting regime $(\rho\xi^2)^{-1}\ll 1$, see figures \ref{fig1}d and \ref{fig2}d, we propose here a non-perturbative theory guided by physical considerations. In brief, {\yc we omit the finite part and} we regularize the divergent {\yc part of the fourth-order} self-energy by giving a finite lifetime to the so-called virtual (or intermediate) phonons of four-phonon collisional processes; this brings out a contribution that turns out to be second-order in $H_3$ and thus corrects the eigenenergy expression to the leading order in the {\yct interaction} strength. We present an atomic physics-type regularization in section \ref{sec4.1} (virtual phonons $\KK$ acquire a complex energy $\veps_\KK\to \veps_\mathbf{K}+\Sigma_\KK^{(2,2)}(\textrm{i}0^+)=\veps_\KK-(\ii\hbar/2)\Gamma_\KK$) and a condensed matter-type regularization in section \ref{sec4.2} (the frequency dependence of the self-energy $\Sigma_\KK^{(2,2)}(\zeta)$ is retained in the complex energy of virtual phonons). Finally, in section \ref{sec4.3}, we give the new form of the theory in the $(\rho\xi^2)^{-1}\to 0$ limit, common to both regularizations and successfully replacing Fermi's golden rule.

\subsection{Atomic physics-type regularization}
\label{sec4.1}

We learn from section \ref{sec3.3} that the addition of an infinitesimal complex part $\textrm{i}0^+$ to $\zeta=z-\veps_\qq$, very common in the Green's function method, is not sufficient to give a finite value to $\tilde{\Sigma}_\qq^{(4,2)}(\zeta=0)$. Collisions between phonons, which render them unstable, naturally provide a non-infinitesimal imaginary contribution next to $\zeta$ in the energy denominators. Following the prescription of atomic physics, we therefore give virtual phonons a complex energy 
\begin{equation}\label{eq300}
\veps_\KK\to\veps_\KK+\Sigma_\KK^{(2,2)}(\textrm{i}0^{\pm})=\veps_\KK\mp \frac{\ii\hbar}{2}\Gamma_\mathbf{K}\end{equation} where $\Gamma_\KK$ is the phonon-$\KK$ damping rate of Fermi's golden rule to order two in $\epsilon$, see equation (\ref{eq033}). Virtual phonons, by definition neither incoming nor outgoing in Landau's formulation of section \ref{sec3.2}, are those whose wave vectors are not shown in {\yc the diagrams of} Figure \ref{fig8}.  The sign to be taken in (\ref{eq300}) in front of $\Gamma_\KK$ or in the argument of $\Sigma_\KK$ is the upper one if the phonon goes forward in time (internal line oriented to the right) or the lower one if the phonon goes backward in time (internal line oriented to the left).\footnote{In Landau's formulation, we choose $-\ii(\hbar/2)\Gamma_\KK$ in the first sum of equation (\ref{eq258}) and its opposite in the second sum, for the reasons explained below (\ref{eq257}) {\yc (in the second sum, the transition amplitudes are reversed in time),} which leads to the same result {\yc as in the diagrammatic formulation}.} On the other hand, we don't give a complex energy to incoming or outgoing phonons as this would question energy conservation in the gas during collisional processes. Consequently, there is no need to modify $\Sigma_{\qq}(\zeta)$ to order two in $H_3$, which has no virtual phonons. 

We apply procedure (\ref{eq300}) to the divergent part of $\tilde{\Sigma}^{(4,2)}_{\qb}$, which comes from the type II contribution of diagram $P_5$, and omit all other contributions to $\tilde{\Sigma}^{(4,2)}_{\qb}$, which turn out to be $O((\rho\xi^2)^{-2})$ and subleading when $(\rho\xi^2)^{-1}\to 0$.\footnote{In the absence of divergence at $\zeta=0$, adding an imaginary part to $\veps_\KK$ does not change the leading order of the diagram, which remains fourth-order in $H_3$.} As shown in Figure \ref{fig8}, the virtual phonons are then {\ycd $\kk_2=\kk-\qq$} and $\kk_3=\kk'$ in the notations of Figure \ref{fig7}. Both go backward in time ($s_2=s_3=-1$), so the quantities $\zeta_1$ and $\zeta_2$ in (\ref{eq233}), {\ycd that are written $\zeta_1=\veps_\kk-\veps_{\kk-\qq}-\veps_\qq$ and $\zeta_2=-\veps_{\kk'}+\veps_{\kk'+\qq}-\veps_{\qq}$,} and the coefficients $A_1^P$ and $A_2^P$ in (\ref{eq243}) are changed as follows, 
\begin{equation}\label{eq301}
\zeta_1\to\zeta_1-\ii\frac{\hbar}{2}\Gamma_{\yc\kk-\qq} \quad ;\quad  \zeta_2\to\zeta_2-\ii\frac{\hbar}{2}\Gamma_{\kk'} \quad ; \quad A_1^P\to A_1^P-\frac{\ii}{2}\tilde{\Gamma}_{\kb-\qb} \quad ; \quad A_2^P\to A_2^P-\frac{\ii}{2}\tilde{\Gamma}_{\kb'}
\end{equation} the order of the factors in the denominator of (\ref{eq243}) respecting that of (\ref{eq233}); here $\tilde{\Gamma}_{\bar{K}}=\hbar\Gamma_{\KK}/k_BT\epsilon^2$ in the spirit of equation (\ref{eq022}). {\yc On the other hand, the quantity $\zeta_3$ does not involve any virtual phonon and remains unchanged.} The previously diverging part of the angular integral (\ref{eq282}) becomes for $\im\zetat>0$:\footnote{We continue to linearize around the same triple intersection point of the cancellation loci, grouping the complex corrections with variable $\zetat$, so that in the {\yct integral in} (\ref{eq282}), we replace $\zetat$ by $\zetat+(\ii/2)\tilde{\Gamma}_{\kb-\qb}$ in the first factor and by $\zetat+(\ii/2)\tilde{\Gamma}_{\kb'}$ in the second factor in the denominator.} 
\begin{equation}\label{eq302}
K_{\mbox{\scriptsize div. {\yct reg.}}}^{P_5^{\rm II}}(\kb,\kb')=-\frac{32\pi^2}{3\gamma\qb\kb\kb'} \frac{1}{\zetat(\kb+\kb')+\frac{\ii}{2}\kb'\tilde{\Gamma}_{\kb-\qb}+\frac{\ii}{2}(\kb-\qb)\tilde{\Gamma}_{\kb'}}
\end{equation} After replacing $K(\kb,\kb')$ by (\ref{eq302}) in (\ref{eq242}) and adding the contribution of second order in $H_3$, we end up with the non-perturbative prescription for the self-energy correcting the result of section \ref{sec2}: 
\begin{equation}\label{eq303}
\boxed{\tilde{\Sigma}_{\rm corr}^{(2,2)}(\qb,\zetat) = \tilde{\Sigma}_{\qb}^{(2,2)}(\zetat) + \tilde{\Sigma}^{(4,2)}_{\mbox{\scriptsize div. {\yct reg.}}}(\qb,\zetat)}
\end{equation} where the first contribution is that of equation (\ref{eq125}) and the second, a regularization of the divergent part of $\tilde{\Sigma}_{\qb}^{(4,2)}(\zetat)$, is written, again for $\im\zetac>0$: 
\begin{equation}\label{eq304}
\boxed{\tilde{\Sigma}^{(4,2)}_{\mbox{\scriptsize div. {\yct reg.}}}(\qb,\zetat)=-\frac{2}{3\pi^2\gamma} \left(\frac{9(1+\Lambda)^2}{8\rho \xi^2}\right)^2
\int_{\qb}^\eta \dd\bar{k}\int_0^{\eta-\kb} \dd\kb' \frac{\bar{k}\kb' (\kb-\qb)(\kb'+\qb)(\kb+\kb')}
{\tilde{\zeta}(\kb+\kb')+\frac{\ii}{2}\kb'\tilde{\Gamma}_{\kb-\qb}+\frac{\ii}{2}(\kb-\qb)\tilde{\Gamma}_{\kb'}} 
(\bar{n}^{\rm lin}_{k'+q}-\bar{n}^{\rm lin}_{k'+k})(\bar{n}^{\rm lin}_{k-q}-\bar{n}^{\rm lin}_{k})}
\end{equation} By inserting function (\ref{eq303}) into equation (\ref{eq023}) (with $\nu=\sigma$ in dimension $d=2$) in its classical field version, we obtain a much better agreement with numerical simulations for the weakest interaction strength, see the orange solid line in Figure \ref{fig3}a and the black solid line in Figure \ref{fig3}b, to be compared respectively with the black solid line in Figure \ref{fig3}a and the black dashed line in Figure \ref{fig3}b corresponding to the uncorrected second order in $H_3$. 

% Figure environment removed

\subsection{Condensed matter regularization}
\label{sec4.2}

It is well known in condensed matter physics that the complex energy of a quasiparticle in an interacting system depends on the frequency at which it is observed, i.e. the frequency at which the response functions are measured. This is the essence of $\Sigma$ in (\ref{eq012}). In (\ref{eq300}), therefore, we need to replace the fixed complex energy $\mp(\ii\hbar/2)\Gamma_\KK$ of the virtual phonon {\yc$\KK$} with a function of {\yc the integration variable} $\zeta$: 
\begin{equation}\label{eq305} 
\veps_\kk\to \veps_\KK+\Sigma_\KK^{(2)}(\pm(\zeta+\Delta E)-\veps_\KK)
\end{equation} where we need to take the sign $+$ if the phonon $\KK$ goes forward in time, and the sign $-$ if it goes backward. The quantity $\Delta E$ is obtained by reinterpreting the energy denominator in which phonon $\KK$ participates in $\Sigma^{(4)}$ as that of the free propagator of this phonon with complex energy $z_{\KK}=\zeta+\Delta E$. So, if phonon $\KK$ goes forward in time (sign $s_\KK=+1$ before $\veps_\KK$ in (\ref{eq238})), it will contribute to $\Sigma^{(4)}$ in simplified forms (\ref{eq213},\ref{eq233}) through factor $1/(\zeta+\Delta E-\veps_\KK)$, which is precisely its unperturbed propagator at complex energy $z_\KK$ and therefore at complex energy redefined as in (\ref{eq013}) $\zeta_\KK=z_\KK-\veps_\KK$. We therefore propose to replace it by the Green's function (\ref{eq012}) of phonon $\KK$ {\yc written as in (\ref{eq013})}, with $\Sigma_{\KK}$ calculated to order two in $H_3$: \begin{equation}\label{eq306}
\frac{1}{\zeta+\Delta E-\veps_\KK}\to \frac{1}{\zeta+\Delta E-\veps_\KK-\Sigma_\KK^{(2)}(\zeta+\Delta E-\veps_\KK)}
\end{equation} which explains the substitution (\ref{eq305}). If the virtual phonon goes backward in time (sign $s_\KK=-1$), we reduce to the previous case through the chain of transformations 
\begin{equation}\label{eq307}
\frac{1}{\zeta+\Delta E+\veps_\KK} = -\frac{1}{-(\zeta+\Delta{\ycd E})-\veps_\KK} \longrightarrow -\frac{1}{-(\zeta+\Delta E)-\veps_\KK-\Sigma_\KK^{(2)}(-(\zeta+\Delta E{\yc)-}\veps_\KK)} = \frac{1}{\zeta+\Delta E+\veps_\KK+\Sigma_\KK^{(2)}(-(\zeta+\Delta E{\yc)-}\veps_\KK)}
\end{equation} again in accordance with (\ref{eq305}). In practice, as in section \ref{sec4.1}, we keep only the divergent part of $\tilde{\Sigma}^{(4,2)}_{\qb}$ (coming from the type II contribution of diagram $P_5$), in which the virtual phonons {\yc both go backward in time and} are of wave vector $\KK=\kk-\qq$ (with $\Delta E=\veps_{\qq}-\veps_\kk$ {\ycd so that $\zeta-\zeta_1=\zeta+\Delta E+\veps_{\kk-\qq}$}) or $\KK=\kk'$ (with $\Delta E=\veps_{\qq}-\veps_{\kk'+\qq}$ {\ycd so that $\zeta-\zeta_2=\zeta+\Delta E+\veps_{\kk'}$}). After regularization, the divergent part (\ref{eq282}) of the angular integral becomes 
\begin{equation}\label{eq308}
K_{\mbox{\scriptsize div. {\yct reg.}}}^{P_5^{\rm II}}(\kb,{\yct\kb'}) = \int_{\mathbb{R}^2} \frac{2\mathrm{d}\tau \mathrm{d}\tau'}{[\zetat-\alpha \tau +\tilde{\Sigma}_{\kb-\qb}^{(2,2)}(-\zetat+\alpha \tau)] [\zetat-\alpha' \tau' + \tilde{\Sigma}_{\kb'}^{(2,2)}(-\zetat+\alpha' \tau')] (\zetat-\beta \tau -\beta' \tau')}
\end{equation} where the coefficients are those of $\tau$ and $\tau'$ in the {\yct integral in} (\ref{eq282}): $\alpha=(3\gamma/4)^{1/2}\qb\kb$, $\alpha'=-(3\gamma/4)^{1/2}\qb\kb'$, $\beta=-(3\gamma/4)^{1/2}\kb\kb'$, $\beta'=(3\gamma/4)^{1/2}(\kb-\qb)\kb'$. As functions $\tilde{\Sigma}^{(2,2)}$ in the denominator of (\ref{eq308}) have branch cuts in the complex plane, {\yc see section \ref{sec2.2},} Cauchy's theorem no longer allows us to calculate the double integral and we can only integrate once, for example over $\tau'$, restricting ourselves as usual to $\im\zetat>0$.\footnote{The denominator of the integrand in (\ref{eq308}) has a branch cut with respect to variable $\tau$ only in the strict upper half-plane because $\im(\zetat/\alpha)>0$, and with respect to variable $\tau'$ only in the strict lower half-plane because $\im(\zetat/\alpha')<0$. We can therefore integrate on $\tau'$ by the residue theorem, closing the integration path with a large semicircle at infinity in the upper half-plane; this path necessarily surrounds the pole $\tau_0'=(\zetat-\beta \tau)/\beta'$ arising from the third factor in the denominator of (\ref{eq308}) because $\im \tau_0'>0$. The integrand of the result in (\ref{eq310}) this time has a branch cut with respect to variable $\tau$ both in the upper half-plane ($\im(\zetat/\alpha)>0$) and in the lower half-plane ($\im(\zetat/\alpha')<0$), which prevents the integration path from being closed only at infinity.} By inserting the corresponding contribution to $K(\kb,\kb')$ in (\ref{eq242}), we end up with the corrected self-energy of the same form as (\ref{eq303}), now with 
\begin{multline}
\label{eq310}
\tilde{\Sigma}^{(4,2)}_{\mbox{\scriptsize div. {\yct reg.}}}(\qb,\zetat)= \qb\left(\frac{9(1+\Lambda)^2}{8\rho \xi^2}\right)^2 \\
\times \int_{\qb}^\eta \frac{\bar{k}\dd\kb}{(2\pi)^2}
\int_0^{\eta-\kb} \frac{\kb' \dd\kb'}{(2\pi)^2}
\int_{-\infty}^{+\infty} \mathrm{d}\tau 
\frac{(-4\ii\pi/\beta')\bar{k}\kb' (\kb-\qb)(\kb'+\qb)(\kb+\kb')(\bar{n}^{\rm lin}_{k'+q}-\bar{n}^{\rm lin}_{k'+k})(\bar{n}^{\rm lin}_{k-q}-\bar{n}^{\rm lin}_{k})}
{\left[\tilde{\zeta}-\alpha \tau +\tilde{\Sigma}^{(2,2)}_{\kb-\qb}(-\tilde{\zeta}+\alpha \tau)\right]
\left[-\frac{\beta}{\beta'}(\tilde{\zeta}-\alpha'\tau)+\tilde{\Sigma}^{(2,2)}_{\kb'}
\left(\frac{\beta}{\beta'}(\tilde{\zeta}-\alpha'\tau)\right)\right]}
\end{multline}
 knowing that $\alpha'-\beta'=\beta$. The results of this approach are shown as orange stars in Fig.~\ref{fig3}a and as red and black dotted lines in Fig.~\ref{fig3}b. \footnote{Getting them is rather cumbersome, since each evaluation of functions $\tilde{\Sigma}_{\bar{K}}^{(2,2)}$ requires numerical integration over a wavenumber, as in (\ref{eq125}). We have discretized the reduced wavenumbers with a step of $1/64$ (results {\yct in the} figure), but we have verified that a step of $1/96$ leads to very similar results.} The two regularization procedures, atomic physics or condensed matter, ultimately give results close to each other and close to the simulations, which reassures us as to their validity. 

\subsection{Regularized theory in the weakly interacting limit}
\label{sec4.3}

Here we calculate the limit of regularized theories when $(\rho\xi^2)^{-1}\to 0$. To do this, we scale variable $\zetat$ and functions $\tilde{\Sigma}$ to their characteristic values $\propto (\rho\xi^2)^{-1}$ by means of {\yct changes} of variables (\ref{eq130}) and (\ref{eq131}), marked with a Czech accent (a factor $\qb$ has been taken out in view of taking the $\qb\to 0$ limit, and all parameters are collected together in the quantity $u\propto\rho\xi^2$ of equation (\ref{eq112})), then {\yct we take} the $u\to+\infty$ limit at fixed $\check{\zeta}$ in equations (\ref{eq303}) and (\ref{eq304}). Contrary to what happened at uncorrected second order in $H_3$ (\ref{eq131}), there remains a dependency in $\check{\zeta}$: \footnote{In two steps, we show that the Green's function associated with self-energy (\ref{eq311}) has, as it should, no pole in the upper complex half-plane. First, we establish the {\yct upper bound} $\im\check{\Sigma}_{{\rm corr}\,\infty}^{(2,2)}(\qb,\check{\zeta})\leq\im\check{\Sigma}_{{\rm corr}\,\infty}^{(2,2)}(\qb,0)\ \forall \check{\zeta}\in\mathbb{C}$, $\im\zetac\geq 0$; to this end, we write the denominator of the integrand in (\ref{eq311}) in the form $x+\textrm{i}y$, with $x\in\mathbb{R}$ and $y\in\mathbb{R}^+$, then use the chain $\im[-1/(x+\textrm{i}y)]=y/(x^2+y^2)\leq 1/y\leq 2/[\kb'(\kb-\qb)(\check{\Gamma}_{\kb-\qb}+\check{\Gamma}_{\kb'})]$. In a second step, we check that $\im\check{\Sigma}_{{\rm corr}\,\infty}^{(2,2)}(\qb,0)<0\ \forall\qb>0$. It follows that equation $\check{\zeta}=\check{\Sigma}_{{\rm corr}\,\infty}^{(2,2)}(\qb,\check{\zeta})$ cannot have a solution $\check{\zeta}$ with a positive imaginary part without leading to the contradiction $\im\check{\zeta}<0$.} 
\begin{equation}\label{eq311}
\boxed{\check{\Sigma}_{\rm corr}^{(2,2)}(\qb,\check{\zeta}) \underset{u \to +\infty}{\stackrel{\im\check{\zeta}\,>\,0}{\longrightarrow}} \check{\Sigma}_{{\rm corr}\,\infty}^{(2,2)}(\qb,\check{\zeta})\equiv-\frac{\ii}{2}\check{\Gamma}_{\qb}-\frac{2}{\qb}\int_{\qb}^\eta \dd\bar{k}\int_0^{\eta-\kb} \dd\kb' \frac{\kb\kb'(\kb-\qb)(\kb'+\qb)(\kb+\kb')(\bar{n}^{\rm lin}_{k'+q}-\bar{n}^{\rm lin}_{k'+k}) (\bar{n}^{\rm lin}_{k-q}-\bar{n}^{\rm lin}_{k})} {\check{\zeta}\qb(\kb+\kb')+\frac{\ii}{2}\kb'(\kb-\qb)\left(\check{\Gamma}_{\kb-\qb}+\check{\Gamma}_{\kb'}\right)}}
\end{equation} Here, $\check{\Gamma}_{\bar{K}}=2\ii\check{\Sigma}_{\bar{K}}^{(2,2)}(\textrm{i}0^+)$ is the golden rule damping rate at leading order in $\epsilon$ and scaled as in (\ref{eq130}), $\check{\Gamma}_{\bar{K}}=8u\Gamma_{\KK}/(3\gamma\epsilon^2\hbar cK)$; we have $\check{\Gamma}_{\bar{K}}=\check{\Gamma}_{\bar{K}}^{{\yct\rm clas}}=2\eta-\bar{K}$ given (\ref{eq113}) in the case of the classical field and $\check{\Gamma}_{\bar{K}}=4\zeta(2)+\bar{K}^2/6$ given (\ref{eq033}) or (\ref{eq141}) in the case of the infinite cut-off quantum field. Starting from the regularized form (\ref{eq310}), we obtain exactly the same expression as (\ref{eq311}).\footnote{Variable changes $\zetat=(3\gamma\qb/8u)\check{\zeta}$ and $\tau=(3\gamma\qb/8u)\check{\tau}$ are carried out simultaneously, {\yct and} the limit $u\to +\infty$ is taken under the integral sign at fixed $\check{\tau}$. Then the branch cuts of functions $\tilde{\Sigma}^{(2,2)}_{\bar{K}}$ are rejected at infinity, these functions are replaced by constants, e.g. $\lim_{u\to+\infty} (8u/3\gamma\qb) \tilde{\Sigma}^{(2,2)}_{\kb-\qb}((3\gamma\qb/8u)(\alpha\check{\tau}-\check{\zeta}))=[(\kb-\qb)/\qb] \check{\Sigma}^{(2,2)}_{\kb-\qb}(\textrm{i}0^-)=[(\kb-\qb)/\qb](\ii/2)\check{\Gamma}_{\kb-\qb}$, and the integral over $\check{\tau}$ can be calculated by Cauchy's formula. We find (\ref{eq311}).} Remarkably, signal (\ref{eq012}) associated with the self-energy (\ref{eq311}) takes real values at all times.\footnote{This results from the property ${\yc\check{\Sigma}(\check{\zeta})^*}=-\check{\Sigma}(-\check{\zeta}^*)$, $\forall\check{\zeta}\in\mathbb{C}\setminus\ii\mathbb{R}^-$, satisfied by function $\check{\Sigma}_{{\rm corr}\,\infty}^{(2,2)}(\qb,\check{\zeta})$ (if we extend the expression in the right-hand side as it is to the case $\im\check{\zeta}<0$).} {\yc When} $(\rho\xi^2)^{-1}\to 0$, therefore, there is persistent disagreement with Fermi's golden rule on the real part of the signal, as predicted in Figure \ref{fig3}b, {\ycd but not on the imaginary part, of zero limit in both theories}. Figure \ref{fig9} shows, as a function of the reduced wavenumber $\qb$, the decay time at $1/\eee$ of the phonon mode $\qq$ deduced from (\ref{eq311}) for quantum field theory (cutoff $\eta\to +\infty$). This is our prediction to leading order in temperature and interaction strength for a two-dimensional superfluid, which replaces and corrects that of Fermi's golden rule shown as a dashed line in Fig.~\ref{fig9}.
\begin{SCfigure}
% Figure removed
\caption{In the low-temperature limit $\epsilon=k_BT/mc^2\to 0$ then weakly interacting limit $(\rho\xi^2)^{-1}\to 0$ for the quantum field {\yc of infinite cut-off $\eta$}, inverse of the decay time at $1/\eee$ of signal (\ref{eq012}) for the phonon mode $\qq$ in a two-dimensional superfluid, as a function of the reduced wavenumber $\qb$. Solid line: prediction from non-perturbative theories in section \ref{sec4}, corresponding to the corrected self-energy (\ref{eq311}) in section \ref{sec4.3}. Dotted line: slope at origin (\ref{eq313}). Dashed line: golden rule (\ref{eq033}). The rate $\Gamma_u$ serving as unit is given by (\ref{eq113}).}
\label{fig9}
\end{SCfigure}
 Three simple analytical predictions can be drawn from equation (\ref{eq311}). Firstly, in the $\qb\to 0$ limit, the $\check{\zeta}$ dependence disappears in the right-hand side of (\ref{eq311}) and signal (\ref{eq012}) becomes purely exponential, $s(t)=\exp(-A\check{t})$ with 
\begin{equation}\label{eq312}
A=\frac{1}{2}\check{\Gamma}_0+4\int_0^\eta \mathrm{d}\bar{k}\int_0^{\eta-\kb} \mathrm{d}\kb'
\frac{\kb\kb'(\kb+\kb')(\bar{n}_{k'}^{\rm lin}-\bar{n}_{k'+k}^{\rm lin})\frac{\dd}{\mathrm{d}\kb}\bar{n}_{k}^{\rm lin}}
{\check{\Gamma}_{\kb}+\check{\Gamma}_{\kb'}} =
\left\{ \begin{array}{ll}
[1-4(4 \ln \frac{4}{3} -1)]\eta \simeq 0.4\,\eta & \mbox{(classical field)} \\
& \\
0.988\,618 \ldots & \mbox{(quantum field)}
\end{array}
\right.
\end{equation} If we leave the rescaled time $\check{t}$ (\ref{eq130}), which brings out the quantity $\Gamma_u$ already introduced in (\ref{eq113}) {\yc since $\check{t}=\qb\Gamma_u t/2$}, we find that this corresponds to a corrected damping rate tending towards zero linearly in $\qb$ : 
\begin{equation}\label{eq313}
\boxed{
\Gamma_q^{{\rm corr}\,\infty} \underset{\qb\to 0}{\sim} A \Gamma_u \bar{q}= A \frac{9k_B T\epsilon^2(1+\Lambda)^2}{4\pi\hbar(3\gamma)^{1/2}\rho\xi^2} \bar{q}\sim \left\{ \begin{array}{ll} \displaystyle\frac{A}{\eta} \Gamma_q^{{\yct\rm clas}} \simeq 0.4\,\Gamma_q^{{\yct\rm clas}} & \mbox{(classical field)} \\ \displaystyle\frac{A}{2\zeta(2)} \Gamma_q \simeq 0.3\,\Gamma_q & \mbox{(quantum field)}\end{array} \right.}
\end{equation} shown as a dotted line in Figure \ref{fig9} for the quantum field and in Figure \ref{fig3}a for the classical field. Here, $\Gamma_q^{{\yct\rm clas}}$ and $\Gamma_q$ are the damping rates of the golden rule for the classical field (\ref{eq113}) and for the quantum field (\ref{eq033}). This shows that the slope at the origin of the corrected rate is considerably smaller than that of the golden rule. Then, in the $\qb\to +\infty$ limit, of course for the infinite cut-off quantum field, the integral in the right-hand side of (\ref{eq311}) tends uniformly towards zero in the upper half-plane and the signal decay becomes purely exponential, with a rate equivalent to that of the golden rule: \footnote{This result is also that of zero temperature, which explains why the right-hand side of (\ref{eq315}) is in fact independent of temperature when we return to the non-{\yct scaled} wave number $q$. As it results from the effective Hamiltonian of quantum hydrodynamics, it is then only valid at the leading order in $q\xi$, i.e. in the linear part of the {\yc acoustic branch.}}
\begin{equation}\label{eq315}
\Gamma_q^{{\rm corr}\,\infty} \underset{\qb\to+\infty}{\sim} \Gamma_q \underset{\qb\to+\infty}{\sim} \frac{3 k_B T\epsilon^2(1+\Lambda)^2}{16\pi(3\gamma)^{1/2}\rho\xi^2} \qb^3
\end{equation} Finally, in the limit of long times $\check{t}\to +\infty$ at fixed $\qb$, the signal decay is not exponential but power-law up to a logarithm, taking the same form for the classical and quantum fields, as shown in \ref{app5}: 
\begin{equation}\label{eq314}
s^{(2,2)}_{{\rm corr}\,\infty}(t)\underset{\check{t}\to+\infty}{\sim} \frac{\qb^4(\dd\bar{n}^{\rm lin}_q/\dd\qb)\ln(\check{t}^2)}{\left[\check{\Gamma}_0\check{\Sigma}_{{\rm corr}\,\infty}^{(2,2)}(\qb,0)\check{t}\right]^2}
\end{equation} For $\qb=1/4$, we find numerically from the self-energy (\ref{eq311}) that the asymptotic behavior (\ref{eq314}) is not reached before a {\yc reduced} time $\Gamma_q t/2=100$, for both the quantum and classical fields, which goes far beyond our simulations in Fig.~\ref{fig3}b and prevents us from testing it. 

\section{Conclusion}
\label{sec5}

In this work, we have considered the phonon gas of a two-dimensional superfluid described by quantum hydrodynamics, an effective low-energy theory, with cubic and quartic coupling between phonons for an acoustic branch $\veps_{\qq}$ which is a convex function of wavenumber $q$. Two important parameters characterize the system: the reduced temperature $\epsilon=k_BT/mc^2$ and the {\yct interaction} strength $(\rho \xi^2)^{-1}$ between superfluid particles, where $c$ is the speed of sound and $\rho$ the density of the superfluid in the ground state, $m$ the mass of a particle and $\xi=\hbar/mc$ the relaxation length.  In this system, we have studied the damping of the phonon mode $\qq$ excited by a Bragg pulse in the linear response and collisionless regime, limiting ourselves to the leading order in $\epsilon$ as going beyond this would make no sense with our model Hamiltonian. 

A first unexpected result is that in dimension two, taking the low-temperature limit, $\epsilon\to 0$ at fixed $(\rho \xi^2)^{-1}$, is not enough to bring the phonon gas into the weak-coupling regime. This defeats the pole approximation, Fermi's golden rule and the assumption of exponential decay.  Indeed, as we show by a scaling-law reasoning, the self-energy $\Sigma(\qq,z)$ has a typical value and a scale of variation in energy in the vicinity of $z=\varepsilon_\qq$ both of order $k_BT \epsilon^2$, whereas it would be necessary to have a typical value much lower than the scale of variation {\yc to be able to apply the golden rule}  (as is the case in dimension three {\ycd where the scale of variation remains $k_B T\epsilon^2$ but the typical value is lowered to $k_B T\epsilon^4$ due to a reduction of the density of final states at low energy)}. This first theoretical result is confirmed by classical field simulations. 

A second unexpected result of the two-dimensional case is that, even in the weakly interacting limit $(\rho \xi^2)^{-1}\to 0$ {\yc in the superfluid}, a perturbative calculation does not give access to the self-energy to the leading order in $\epsilon$.  Thus, the calculation of $\Sigma(\qq,z)$ at second order in the cubic phonon coupling does not reproduce the damping of mode $\qq$ observed numerically, and the fourth-order calculation {\yct exhibits} (after taking the $\epsilon\to 0$ limit at fixed $\qb=\hbar c q/k_B T$ and $(\rho \xi^2)^{-1}$) an unphysical divergence in $z=\varepsilon_\qq$ that renders it useless. 

To break this deadlock, guided by physical intuition, we give a finite lifetime to the virtual phonons ({\yc those} present only in the intermediate state of four-phonon collisional processes).  This removes the divergence of the fourth-order contribution by transforming it into a formally second-order contribution, which corrects the second-order calculation. The regularized theory is found to be in good agreement with simulations in the weakly interacting regime.  It makes it possible to take the $(\rho \xi^2)^{-1}\to 0$ limit in the expression of $\Sigma(\qq,z)$ to the leading order in $\epsilon$ and to propose a phonon damping law replacing that of the golden rule. The new law is non-exponential, except in the $\qb\to 0$ limit where it is characterized by the rate (\ref{eq313}), linear in wavenumber like the golden rule but with a coefficient about three times smaller, and in the $\qb\to+\infty$ limit where its rate (\ref{eq315}) reproduces that of the golden rule.

\medskip
\noindent {\bf Acknowledgements~:} Alan Serafin received funding from the European Horizon 2020 research and innovation project macQsimal number 820393. {\ycd This article is part of a special issue dedicated to the CNRS Gold Medal awarded to Jean Dalibard in December 2021. The authors congratulate Jean on his award and thank him for his brilliant lectures at the Collège de France and for all the enriching discussions they had with him on cold atomic gases (Y.C. in particular recalls with emotion his teaching on Fermi's golden rule at the ENS \g{DEA de physique quantique} in 1987-1988).}

\appendix 
\section{Analytic continuation of $\check{\Sigma}^{(2,2)}_{\qb}(\check{\zeta})$ to the lower half-plane and validity conditions of the golden rule}
\label{app1}

To analytically continuate from the upper complex half-plane to the lower half-plane expression (\ref{eq131}) of the self-energy $\check{\Sigma}^{(2,2)}_{\qb}(\check{\zeta})$, we proceed in two steps: (i) we reduce (\ref{eq131}) to a single integral, that of an analytic function of a well-chosen variable $x$ over an interval of $\mathbb{R}$, and (ii) to extend this integral to $\im\zetac<0$, we deform the integration path on $x$ in the complex plane and rotate the branch cuts of the corresponding analytic function. In application, we give conditions for the validity of Fermi's golden rule, which are necessary but not sufficient.

\paragraph{Reduce to a single integral}
 In Landau's contribution to (\ref{eq131}), i.e. {\yc in} the second integral, we perform the change of variable $x=\kb(\kb+\qb)$ or $\kb=(x+\qb^2/4)^{1/2}-\qb/2$, where $x\in\mathbb{R}^+$. In Beliaev's contribution, the integrand is invariant by changing $\kb$ to $\qb-\kb$, which allows us to restrict the integration on $\kb$ to the interval $[0,\qb/2]$ at the cost of multiplying by a factor $2$; we then perform the now injective change of variable $x=\kb(\kb-\qb)$ to $\kb=\qb/2-(\qb^2/4+x)^{1/2}$, where $x\in[-\qb^2/4,0]$; we use the identity $\bar{n}_k^{\rm lin}=-(1+\bar{n}_{-k}^{\rm lin})$ to give the bosonic amplification factor the same structure of difference of occupation numbers as in Landau. In both cases, we arrive at the same integrand, except that Beliaev shows a global factor $(-x)^{1/2}/(\zetac/u-x)^{1/2}$ where Landau contains a factor $x^{1/2}/(x-\zetac/u)^{1/2}$; however, we can give them the common expression $(x-\textrm{i}0^+)^{1/2}/(x-\zetac/u)^{1/2}$ by means of the relations $(-x)^{1/2}=\ii(x-\textrm{i}0^+)^{1/2} \forall x<0$, $x^{1/2}=(x-\textrm{i}0^+)^{1/2} \forall x>0$, $(\zetac/u-x)^{1/2}=\ii(x-\zetac/u)^{1/2}\forall x\in\mathbb{R}$ knowing that $\im(\zetac/u)>0$. We arrive at the single integral we're looking for: 
\begin{equation}\label{eq1001}
\check{\Sigma}^{(2,2)}_{\qb}(\check{\zeta})= \frac{1}{2\ii\qb}\int_{-\qb^2/4}^{+\infty}\mathrm{d}x \frac{x\sqrt{x-\textrm{i}0^+}}{\sqrt{x-\zetac/u}\sqrt{x+\qb^2/4}} \left[\frac{1}{\eee^{\sqrt{x+\qb^2/4}-\qb/2}-1}-\frac{1}{\eee^{\sqrt{x+\qb^2/4}+\qb/2}-1}\right]
\end{equation}
 \begin{SCfigure}
\begin{tikzpicture}[x=6mm,y=6mm]
\draw[thick,->,>=stealth](-7,0)--(7,0);
\node(a)at(7,0)[below]{${\rm Re\,}x$};
\draw[thick,->,>=stealth](0,-5)--(0,6);
\node(a)at(0,6)[right]{${\rm Im\,}x$};
\node(a)at(-3,0)[]{$\bullet$};
\node(a)at(-3,-0.15)[above]{$-\bar{q}^2/4$};
\draw[ultra thick](-7,0)--(-3,0);
\draw[ultra thick](-7,0.7)--(0,0.7);
\node(a)at(0,0.7)[]{$\bullet$};
\node(a)at(0,0.7)[right]{${\rm i} 0^+$};
\draw[dashed,ultra thick](-3,0)--(-7,-4);
\draw[dashed,ultra thick](0,0.7)--(4,4.7);
\draw[ultra thick](-7,2)--(3,2);
\node(a)at(3,2)[]{$\bullet$};
\node(a)at(3,2)[below]{$\check{\zeta}/u$};
\draw[dashed,ultra thick](3,2)--(6,5);
\draw[thick,color=black,->,>=stealth](-3,0)+(-180:1)arc(-180:-135:1);
\node(a)at(-4.5,-0.5)[]{$\alpha$};
\draw[thick,color=black,->,>=stealth](0,0.7)+(180:0.8)arc(180:45:0.8);
\node(a)at(-1.5,1.3)[]{$\alpha-\pi$};
\draw[thick,color=black,->,>=stealth](3,2)+(180:0.8)arc(180:45:0.8);
\node(a)at(3.2,3)[]{$\alpha-\pi$};
\draw[thick,color=blue,->,>=stealth](-3,0-0.1)--(-5,-2-0.1);
\draw[thick,color=blue](-3,0-0.1)--(-7,-4-0.1);
\draw[thick,color=red,->,>=stealth](1,-0.1)+(-180:4)arc(-180:-135:4);
\draw[thick,color=red,->,>=stealth](1,-0.1)+(-135:4)arc(-135:0:4);
\draw[->,>=stealth,color=red,thick](5,-0.1)--(6,-0.1);
\node(a)at(-0.25,0)[below]{$0$};
\end{tikzpicture}
\caption{Graphical illustration of the method for analytically continuating $\check{\Sigma}_{\qb}^{(2,2)}(\zetac)$ from the upper half-plane $\im\zetac>0$ to the lower half-plane $\im\zetac<0$ using integral form (\ref{eq1001}). Red curve: first deformation of the integration path on variable $x$ in the complex plane. {\yct Black h}alf-lines: singularity lines of the integrand of (\ref{eq1001}) considered as a function of $x$, in thick solid line for their original position, in thick dashed line after rotation by an angle $\alpha$ or $\alpha-\pi$ as shown ($\alpha=\pi/4$ on the figure). Black discs: branch points $x_0$ of the corresponding $(x-x_0)^{1/2}$ square root functions. Blue half-line: final integration path on $x$, leading to the result sought (\ref{eq140}).}
\label{fig11}
\end{SCfigure}
 
\paragraph{Extending it analytically}
 Integrand of (\ref{eq1001}) is an analytic function of $x$ in the complex plane, except for the three singularity lines resulting from the branch cuts of the three square roots involved: $\textrm{i}0^++\mathbb{R}^-$ in the numerator, $\zetac/u+\mathbb{R}^-$ and $-\qb^2/4+\mathbb{R}^-$ in the denominator, shown in black in Figure \ref{fig11}. \footnote{The integrand also has poles $x_n=2\textrm{i}n\pi(\qb+2\textrm{i}n\pi)$, $n\in\mathbb{Z}^*$, coming from the Bose occupation numbers ($x=0$ is not a pole because of the factor $x$ in the numerator). Future continuous deformations of the integration path in (\ref{eq1001}), shown in red and then blue in Fig.~\ref{fig11}, do not pass over any of these poles under the following conditions, which are assumed to be verified in the following (they ensure that the most dangerous pole $x_{n=-1}$ remains to the left of the blue half-line {\yc at an angle $\alpha$}): (i) $-\qb^2/4>\re x_{-1}=-(2\pi)^2$ i.e. $\qb<4\pi$, (ii) $\tan\alpha>2\pi\qb/[(2\pi)^2-\qb^2/4]$.} This allows the integration path to be deformed at will, provided it doesn't cross any singularity of the integrand, see the red line in Figure \ref{fig11}. You'd think this would be enough to make the integrand analytic in $\zetac/u$ in the region between the red line and the real axis; but it's not, because when $\zetac/u$ enters this region, the (horizontal) branch cut attached to it necessarily crosses the red line. So we have to do better, by also deforming the branch cuts of the integrand; the simplest way to do this is to introduce the new $z_\theta^{1/2}\equiv\exp(\ii\theta/2)[\exp(-\ii\theta)z]^{1/2}$ square-root determinations, parameterized by the angle $\theta\in]-\pi,\pi[$ by which the original branch cut was rotated (the $\mathbb{R}^-$ half-line being sent to $\exp(\ii\theta)\mathbb{R}^-$). As long as the branch cuts, in their rotating motion, do not sweep the integration path in red, the integrand does not change on this path and the value of the integrand is preserved. We decide to rotate the $(x+\qb^2/4)^{1/2}$ branch cut by an arbitrary angle $\alpha\in ]0,\pi/2[$, and the other two by an angle $\alpha-\pi\in]-\pi,-\pi/2[$ so that they are parallel but in the opposite direction to the first one, see the dashed lines in Figure \ref{fig11}. Finally, we take as our final integration path the blue half-line in Figure \ref{fig11}, which follows the new branch cut of $(x+\qb^2/4)^{1/2}$\footnote{We could close the integration path with a circular arc of radius $R\to +\infty$, of contribution $\to 0$ since the integrand there is exponentially small in $\sqrt{R}$.} from below and which we parameterize as follows, $x=\exp(\ii\alpha)(-v-\textrm{i}0^+)-\qb^2/4$, $v\in\mathbb{R}^+$. Then 
\begin{equation}\label{eq1002}
(x+\qb^2/4)^{1/2}_\alpha=-\ii\eee^{\ii\alpha/2}\sqrt{v} \ \  ; \ \  (x-\zetac/u)^{1/2}_{\alpha-\pi}=-\ii\eee^{\ii\alpha/2}[v+\eee^{-\ii\alpha}(\qb^2/4+\zetac/u)]^{1/2} \ \  ;\ \  (x-\textrm{i}0^+)^{1/2}_{\alpha-\pi} = -\ii\eee^{\ii\alpha/2} {\yc(}v+\eee^{-\ii\alpha}\qb^2/4{\yc)}^{1/2}
\end{equation} where all $\textrm{i}0^+$ could be omitted without difficulty in the right-hand sides. The complex number $\zetac/u$ can now be made to descend continuously into the lower half-plane, remaining constantly to the right or left of the blue half-line without the latter being crossed by the branch cut attached to $\zetac/u$: we have indeed continuated $\check{\Sigma}_{\qb}^{(2,2)}(\zetac)$ analytically to $\im\zetac<0$. It remains to set $v=\kb^2$, $\kb\in\mathbb{R}^+$, to arrive at the announced result (\ref{eq140}) with $\phi=\alpha-\pi\in]-\pi,-\pi/2[$.

\paragraph{Application}
 Now that we know from equation (\ref{eq1001}) that $\check{\Sigma}^{(2,2)}_{\qb}(\zetac)$ has no branch point at $\zetac=0$, we can calculate its value and derivative at $\zetac=\textrm{i}0^+$. For its value, the simplest way is to make $\zetac$ tend towards zero by positive imaginary parts under the integral sign in form (\ref{eq131}); we integrate over $\kb$ after expanding the occupation numbers in integer series of variable $\exp(-\kb)$, which brings up Bose functions $g_\alpha(z)$ evaluated at $z=1$ or at $z=\exp(-\qb)$ (in the latter case, they cancel out in the final result). For the derivative, the simplest way is to use the analytic continuation (\ref{eq140}), whose integrand shows no singularity on the integration path for $\zetac=0$; we can then differentiate with respect to $\zetac$ and take the limit $\zetac\to 0$ under the integral sign.\footnote{If we stick to form (\ref{eq131}), we can't take the derivative under the integral sign without making a non-absolutely convergent integral appear. We can proceed in a hybrid way after choosing an infinitesimal infrared cut $\kappa$ over $\kb$: in form (\ref{eq131}), limited to $\kb>\kappa$ in each contribution, we differentiate under the integral sign and then let $\zetac$ tend towards $0$ ; in form (\ref{eq1001}), limited to $-\kappa\qb<x<\kappa\qb$, we keep the exact expression of the factor $(x-\textrm{i}0^+)^{1/2}/(x-\zetac/u)^{1/2}$ but expand the rest of the integrand in powers of $x$ (zero order is in fact sufficient), and we analytically calculate the resulting integral for $\zetac=s+\textrm{i}0^+$, where $s$ is a real number (if necessary in terms of hypergeometric functions) then we calculate its derivative with respect to $s$ at $s=0$. The final result is the same.} We finally obtain 
\begin{equation}\label{eq141}
\check{\Sigma}^{(2,2)}_{\qb}(\check{\zeta}=\textrm{i}0^+) = \frac{1}{\ii}\left(\frac{\qb^2}{12}+2\zeta(2)\right) \quad; \quad
\frac{\dd}{\dd\check{\zeta}} \check{\Sigma}^{(2,2)}_{\qb}(\check{\zeta}=\textrm{i}0^+) = \frac{1}{u\qb} \left(\frac{\pi}{2}+\frac{\ii\qb}{4}\right)
\end{equation} The value of the function is simply a rewriting of result (\ref{eq033}) under the effect of the change of variables in (\ref{eq130},\ref{eq131}). Its derivative is involved in a well-known validity condition of the golden rule \cite{CCTbordeaux}, $|\im(\dd/\dd\zeta)\Sigma_{\qq}(\zeta=\textrm{i}0^+)|\ll 1$, which is written here as $1/(4u)\ll 1$ and imposes, unsurprisingly, $u\gg 1$. A more precise condition requires that the residue $Z$ of {\yct the} Green's function, which appears as a factor of the exponential $\exp(-\zetac_{\qb}\check{t})$ in the exact expression (\ref{eq2001}) of the signal $s^{(2,2)}(t)$, be very close to the value one taken in the golden rule, which adds the constraint $|\re(\dd/\dd\zeta)\Sigma_{\qq}(\zeta=\textrm{i}0^+)|\ll 1$, here $\qb\gg \pi/(2 u)$. As shown in Figure \ref{fig1}d, in which $u\simeq 18.14$, these conditions are not sufficient. {\ycd Sections \ref{sec3.3} and \ref{sec4} help to understand why (in essence, a singularity appears at $\zeta=\ii 0^+$ in $\Sigma_\qq^{(4)}(\zeta)$, the fourth-order contribution in $H_3$, making the present second-order calculation irrelevant in the limit $\epsilon\to 0$).}

\section{First deviation of the signal from the golden-rule exponential for the theory at second order in $H_3$}
\label{app2}

Here we establish result (\ref{eq151}) of section \ref{sec2.2}, valid in the weakly interacting limit $u\to+\infty$ {\yc at fixed $\check{t}$ and $Q=u^{1/2}\qb$ as in (\ref{eq150})}, by starting from the writing (\ref{eq132}) of the signal. To this end, guided by Fig.~\ref{fig5}b, we first apply the residue theorem to the contour formed by the union of the path $C_+$, the portions of the circle at infinity and the contour around the branch cut $-Q^2/4-\eee^{\ii(\phi+\pi)}\mathbb{R}^+$: the integral of $\exp(-\ii\check{\zeta}\check{t})\check{\mathcal{G}}_{\qb}^{(2,2)}(\zetac)$ on this contour is $2\ii\pi$ times the residue in $\zetac=\zetac_{\qb}$, the pole of {\yct the} Green's function {\yct analytically continuated} to the lower half-plane by means of (\ref{eq140}). The signal (\ref{eq132}) can therefore be written exactly as the sum of the contributions of the pole $\zetac_{\qb}$ and the contour: 
\begin{equation}\label{eq2001}
s^{(2,2)}(t) = \frac{\exp(-{\yct\ii}\zetac_{\qb}\check{t})}{1-\frac{\dd}{\dd\zetac}\check{\Sigma}^{(2,2)}_{\qb\downarrow}(\zetac_{\qb})}
+\eee^{\textrm{i}Q^2\check{t}/4} \int_0^{\eee^{\ii(\phi+\pi)}(+\infty)} \frac{\mathrm{d}z}{2\ii\pi} \frac{(\check{\Sigma}_- - \check{\Sigma}_+)\exp(\textrm{i}z \check{t})}{(-Q^2/4-z-\check{\Sigma}_-)(-Q^2/4-z-\check{\Sigma}_+)}
\end{equation} {\yc where the} complex variable $z$ goes in a straight line from $0$ to $\exp[\ii(\phi+\pi)](+\infty)$ and we note $\check{\Sigma}_{\pm}\equiv\check{\Sigma}^{(2,2)}_{\qb\downarrow}(-Q^2/4-z+\eee^{\ii(\phi+\pi)}\textrm{i}0^\pm)$ the values of the analytic continuation (\ref{eq140}) of the self-energy immediately to the left (sign $+$) or right (sign $-$) of the branch cut. We then make use of the fundamental property\footnote{The leading term is obtained by writing the difference in the left-hand side of (\ref{eq2003}) as a single integral over $\kb$ by means of (\ref{eq140}), multiplying by $u^{1/2}$, then making the change of variable $\kb=K/u^{1/2}$ before taking the limit $u\to +\infty$ under the integral sign.} 
\begin{equation}\label{eq2003}
\check{\Sigma}^{(2,2)}_{\qb\downarrow}(\zetac)-\check{\Sigma}^{(2,2)}_{\qb\downarrow}(0)\stackrel{Q\,\mbox{\scriptsize fixed}}{\underset{u\to+\infty}{=}} \frac{\eee^{\ii\phi/2}}{\textrm{i}u^{1/2}} \int_0^{+\infty} \mathrm{d}K \left\{\frac{(K^2-\eee^{-\ii\phi}Q^2/4)^{1/2}}{[K^2-\eee^{-\ii\phi}(\zetac+Q^2/4)]^{1/2}}-1\right\}+O(1/u)
\end{equation} In the contribution of the contour to (\ref{eq2001}), at the leading order in $u^{-1/2}$, we can thus replace in the denominator $\check{\Sigma}_\pm$ by $\check{\Sigma}^{(2,2)}_{\qb\downarrow}(0)\to -2\ii\zeta(2)=-\ii\pi^2/3$ according to (\ref{eq141}), and in the numerator the difference $\check{\Sigma}_--\check{\Sigma}_+$ by its asymptotic equivalent, which is written 
\begin{equation}\label{eq2004}
\check{\Sigma}_--\check{\Sigma}_+ \stackrel{Q\,\mbox{\scriptsize fixed}}{\underset{u\to+\infty}{\sim}} \frac{\eee^{\ii\phi/2}}{\textrm{i}u^{1/2}} \int_0^{+\infty} \mathrm{d}K
\left(K^2-\eee^{-\ii\phi}Q^2/4\right)^{1/2}\left[\frac{1}{(K^2-x+\textrm{i}0^-)^{1/2}}-\frac{1}{(K^2-x+\textrm{i}0^+)^{1/2}}\right]
\end{equation} knowing that $z=\eee^{\ii(\phi+\pi)}x$, $x>0$. The integrand in (\ref{eq2004}) is non-zero only for $K<x^{1/2}$, which prompts us to put $K=x^{1/2}\sin\alpha$, $\alpha\in [0,\pi/2]$, especially as this makes the square roots in the denominator disappear: we find $S_{\rm contour}(\check{t})$ in (\ref{eq152}). In the implicit equation defining the pole, rewritten by setting aside its zero-order approximation, $\zetac_{\qb}-\check{\Sigma}^{(2,2)}_{\qb\downarrow}(0)=\check{\Sigma}^{(2,2)}_{\qb\downarrow}(\zetac_{\qb})-\check{\Sigma}^{(2,2)}_{\qb\downarrow}(0)$, it suffices for the right-hand side to use (\ref{eq2003}) with $\zetac=-\ii\pi^2/3$; the convenient choice $\phi={\ycd(-\pi)}^{+}$ (on which the result does not depend) yields $\delta\zetac_{\qb}$ in (\ref{eq154}). Finally, to obtain the first deviation from one of the residue, simply differentiate (\ref{eq2003}) with respect to $\zetac$ (under the integral sign in the right-hand side), replace $\zetac$ by $-\ii\pi^2/3$ and then take $\phi={\ycd(-\pi)}^+$: {\yct one finds} $\delta\check{Z}$ in (\ref{eq154}).

\section{Double Matsubara sums in the theory at fourth order in $H_3$}
\label{app3}

Here we explicitly calculate the Matsubara sums contributing to (\ref{eq212},\ref{eq232}) in section \ref{sec3.1} and show how to obtain their simplified forms (\ref{eq213},\ref{eq233}) in the small-angle limit. The calculation relies on the generalization of relation (\ref{eq122}) to any number $J$ of factors in the summand, which is demonstrated as in footnote \ref{notedemo}: 
\begin{equation}\label{eq3000}
\sum_{n\in\mathbb{Z}} \frac{1}{\prod_{j=1}^{J} (2\ii\pi n-\vepsb_j)} = \sum_{j=1}^{J} \frac{-\bar{n}_{\veps_j}}{\prod_{\stackrel{\mbox{\scriptsize$k\!=\!1$}}{k\neq j}}^{J} (\vepsb_j-\vepsb_k)} \quad\forall J\geq 2
\end{equation}
 
\paragraph{Internal-loop diagrams}
 Let's write the sum appearing in (\ref{eq212}) after outputting a factor $(k_B T)^5$ in the denominator (we have set $\veps_{\kk_j}=k_B T\vepsb_j$) and {\yct expressing} the Matsubara energies in {\yct terms} of the independent variables $\omega=2\ii\pi k_B T n$ and $\omega'=2\ii\pi k_B T n'$ as in equations (\ref{eq210},\ref{eq211}), for a complex energy $z$ temporarily equal to $\Omega=k_B T\bar{\Omega}\in 2\ii\pi k_B T\mathbb{Z}$: 
\begin{equation}\label{eq3001}
S_{B_i}(z)=(k_BT)^{-5}\sum_{n,n'\in\mathbb{Z}} \frac{1}{s_1(\bar{\Omega}-2\ii\pi s_2 n)-\vepsb_1}\frac{1}{2\ii\pi n-\vepsb_2}\frac{1}{2\ii\pi n'-\vepsb_3}\frac{1}{2\ii\pi s_4(s_2 n - s_3 n')-\vepsb_4} \frac{1}{2\ii\pi s_2 s_5 n-\vepsb_5}
\end{equation} Using (\ref{eq122}), let's calculate the sum over $n'$ of the product of the two factors that depend on $n'$, then let's make variable $\bar{\Omega}$ disappear from the occupation numbers as in the first equality of (\ref{eq123}). After the clever change of variable $n\to s_2 n$, which doesn't affect the summation domain since $s_2=\pm 1$, we're left with 
\begin{equation}\label{eq3002}
S_{B_i}(z)= (k_BT)^{-5}\sum_{n\in\mathbb{Z}} \frac{s_1s_2s_4s_5(\bar{n}_{\veps_3}-\bar{n}_{-s_3s_4\veps_4})}{[2\ii\pi n-(\bar{\Omega}-s_1\vepsb_1)](2\ii\pi n-s_2\vepsb_2)(2\ii\pi n -s_5 \vepsb_5)[2\ii\pi n-(s_3\vepsb_3+s_4\vepsb_4)]}
\end{equation} We've taken out all the necessary factors to put the result into a simple form, as in the left-hand side of relation (\ref{eq3000}), which makes this relation easy to use: 
\begin{multline}
\label{eq3003}
S_{B_i}(z)=-(k_B T)^{-2}s_1s_2s_4s_5(\bar{n}_{\veps_3}-\bar{n}_{-s_3s_4\veps_4})\Bigg\{\frac{1}{z-z_2}\left[\frac{\bar{n}_{-s_1\veps_1}}{(z-z_1)(z-z_3)}-\frac{\bar{n}_{s_3\veps_3+s_4\veps_4}}{(z_2-z_1)(z_2-z_3)}\right] \\
+\frac{1}{z_1-z_3}\left[\frac{\bar{n}_{s_2\veps_2}}{(z_1-z)(z_1-z_2)}-\frac{\bar{n}_{s_5\veps_5}}{(z_3-z)(z_3-z_2)}\right]\Bigg\}
\end{multline}
 At this point, we have {\yct analytically continuated the result} for variable $z$ over $\mathbb{C}\setminus\mathbb{R}$ and introduced the energies of the three intermediate states (marked by the red vertical lines on the left-hand side of Figure \ref{fig7}) counted algebraically, $z_1=s_1\veps_1+s_2\veps_2$, $z_2=s_1\veps_1+s_3\veps_3+s_4\veps_4$ and $z_3=s_1\veps_1+s_5\veps_5$. In the case of $s_2=s_5$, we also have $\veps_2=\veps_5$ and $z_1=z_3$, since $\kk_2$ is then equal to $\kk_5$; this reveals a division by zero in the second contribution of (\ref{eq3003}) between curly braces, which we therefore extend by continuity with L'Hospital's rule. After some skilful rearrangement, we arrive at:\footnote{In particular, we use the identity $(z_1-z_2)^{-2}(z-z_2)^{-1}+(z_1-z_2)^{-2}(z_1-z)^{-1}+(z_1-z)^{-2}(z_1-z_2)^{-1}=(z-z_1)^{-2}(z-z_2)^{-1}$.} 
\begin{equation}\label{eq3004}
S_{B_i}^{s_2=s_5}(z)=-(k_B T)^{-2}s_1s_4(\bar{n}_{\veps_3}-\bar{n}_{-s_3s_4\veps_4})\left[\frac{\bar{n}_{-s_1\veps_1}-\bar{n}_{s_2\veps_2}}{(z-z_1)^2(z-z_2)}+\frac{\bar{n}_{s_2\veps_2}-\bar{n}_{s_3\veps_3+s_4\veps_4}}{(z_2-z_1)^2(z-z_2)}+\frac{\bar{n}_{s_2\veps_2}(1+\bar{n}_{s_2\veps_2})}{k_B T(z_2-z_1)(z_1-z)}\right]
\end{equation} Note that the global factor $(k_B T)^{-2}$ in (\ref{eq3003},\ref{eq3004}) simplifies with the factor $(-k_B T)^2$ in (\ref{eq212}). Finally, consider the small-angle limit $\epsilon\to 0$ in (\ref{eq3004}), in which $z$, $z_1$ and $z_2$ are written as $\veps_q+O(k_B T\epsilon^2)$. Between the square brackets, the second contribution is subleading because $s_3\veps_3+s_4\veps_4=s_2\veps_2+z_2-z_1=s_2\veps_2+O(k_B T\epsilon{\yc^2)}$, so its numerator tends to zero; the third is also subleading because its denominator is of order $(k_B T)^3\epsilon^4$ instead of $(k_B T)^3\epsilon^6$ like the others. We find (\ref{eq213}).

\paragraph{Bridge diagrams}
 We proceed as before, with the same notations. The double sum in (\ref{eq232}) is written 
\begin{equation}\label{eq3010}
S_{P_i}(z)=(k_B T)^{-5} \sum_{n,n'\in\mathbb{Z}} \frac{1}{2\ii\pi n-\vepsb_1} \frac{1}{s_2(\bar{\Omega}-2\ii\pi s_1 n)-\vepsb_2}\frac{1}{2\ii\pi n'-\vepsb_3}\frac{1}{s_4(\bar{\Omega}-2\ii\pi s_3 n')-\vepsb_4} \frac{1}{2\ii\pi s_5(s_1 n-s_3 n')-\vepsb_5}
\end{equation} Using relation (\ref{eq3000}) with $J=3$, we calculate the sum over $n'$, perform a partial fraction decomposition of the result for variable $2\ii\pi n$ and then make the change of variable $n\to s_1 n$, so that 
\begin{equation}\label{eq3011}
S_{P_i}(z)=\frac{(k_B T)^{-5}s_1s_2s_4s_5}{\bar{\Omega}-s_3\vepsb_3-s_4\vepsb_4}\sum_{n\in\mathbb{Z}} \frac{1}{(2\ii\pi n-s_1\vepsb_1)[2\ii\pi n-(\bar{\Omega}-s_2\vepsb_2)]} \left[\frac{\bar{n}_{\veps_3}-\bar{n}_{-s_3s_5\veps_5}}{2\ii\pi n-(s_3\veps_3+s_5\veps_5)}+\frac{\bar{n}_{-s_3s_5\veps_5}-\bar{n}_{-s_3s_4\veps_4}}{2\ii\pi n-(\bar{\Omega}+s_5\veps_5-s_4\veps_4)}\right]
\end{equation} We use (\ref{eq3000}) again and then analytically continuate over $z$ to $\mathbb{C}\setminus\mathbb{R}$; there are two terms between the square brackets in (\ref{eq3011}), so two contributions, type I and type II in that order: 
\begin{eqnarray}\label{eq3012}
S_{P_i^{\rm I}}(z)&=&\frac{(k_B T)^{-2}s_1s_2s_4s_5(\bar{n}_{\veps_3}-\bar{n}_{-s_3s_5\veps_5})}{z_2-z}\left[
\frac{\bar{n}_{s_1\veps_1}}{(z_1-z)(z_1-z_4)}+\frac{\bar{n}_{-s_2\veps_2}}{(z-z_1)(z-z_4)}+\frac{\bar{n}_{s_3\veps_3+s_5\veps_5}}{(z_4-z_1)(z_4-z)}\right] \\
\label{eq3013}
S_{P_i^{\rm II}}(z)&=&\frac{(k_B T)^{-2}s_1s_2s_4s_5(\bar{n}_{-s_3s_5\veps_5}-\bar{n}_{-s_3s_4\veps_4})}{z_2-z}\left[
\frac{\bar{n}_{s_1\veps_1}}{(z_1-z)(z_3-z)}+\frac{\bar{n}_{-s_2\veps_2}}{(z-z_1)(z_3-z_1)}+\frac{\bar{n}_{s_5\veps_5-s_4\veps_4}}{(z-z_3)(z_1-z_3)}\right]
\end{eqnarray} with the algebraic energies of the intermediate states marked by the red lines on the right-hand side of Figure \ref{fig7}, $z_1=s_1\veps_1+s_2\veps_2$ and $z_2=s_3\veps_3+s_4\veps_4$ common to both types, $z_4=s_2\veps_2+s_3\veps_3+s_5\veps_5$ for type I and finally $z_3=s_1\veps_1+s_4\veps_4-s_5\veps_5$ for type II. These variables are not independent since $z_1+z_2=z_3+z_4$. It remains to take the limit of small angles $\epsilon\to 0$, in which {\yc $z=\veps_q+O(k_B T \epsilon^2)$ and} $z_j=\veps_q+O(k_B T \epsilon^2)$, $1\leq j\leq 4$. Then $s_3\veps_3+s_5\veps_5=s_1\veps_1+O(k_BT\epsilon^2)$ and we can, at leading order, replace $\bar{n}_{s_3\veps_3+s_5\veps_5}$ by $\bar{n}_{s_1\veps_1}$ in (\ref{eq3012}); grouping by occupation number and simplifying, we find the first contribution to the right-hand side of (\ref{eq233}). Similarly, we replace $\bar{n}_{s_5\veps_5-s_4\veps_4}$ by $\bar{n}_{-s_2\veps_2}$ at leading order in (\ref{eq3013}), which reproduces the second contribution to (\ref{eq233}).

\section{A linear divergence of $\tilde{\Sigma}_{\qb}^{(4,2)}(\zetat)$ away from the origin}
\label{app4}

As we show in this appendix, the $\tilde{\Sigma}_{\qb}^{(4,2)}(\zetat)$ component of the self-energy at fourth-order in $H_3$ and second-order in $\epsilon$, obtained by summing all the contributions (\ref{eq220}), (\ref{eq240}), (\ref{eq242}) of the fourth-order diagrams in section \ref{sec3.1}, diverges linearly when $\zetat\to\zetat_0\equiv -3\gamma\qb^3/32$ (here by positive imaginary parts): 
\begin{equation}\label{eq4000}
\tilde{\Sigma}_{\qb}^{(4,2)}(\zetat) \stackrel{\im\zetat>0}{\underset{\zetat\to \zetat_0}{\sim}} \frac{\qb^3(\bar{n}^{\rm lin}_{q/2}+{1}/{2})}{24\ii\pi^2\gamma(\zetat-\zetat_0)} \left(\frac{9(1+\Lambda)^2}{8\rho\xi^2}\right)^2\left[\int_0^{\qb/2} \dd\kb' \kb'(\kb'-\qb/2)(\bar{n}^{\rm lin}_{k'}-\bar{n}^{\rm lin}_{k'-{q}/{2}})+2\int_{\qb/2}^{\eta} \dd\kb'\kb'(\kb'-\qb/2)(\bar{n}^{\rm lin}_{k'}-\bar{n}^{\rm lin}_{k'-{q}/{2}})\right]
\end{equation} This results from two phenomena. (i) At a fixed reduced wavenumber $\kb$, the angular integral (\ref{eq221}) in inner-loop diagrams diverges at $\zetat=A_1^B$ with exponent $3/2$ because, for this value of $\zetat$, the denominator of the integrand has $\thetat=0$ as quadruple root. To see this quantitatively, we can use the identity $\int_{\mathbb{R}}\dd\thetat/(z-\thetat^2)^2=-\ii\pi\,{\ycd\sig(\im z)}/(2 z^{3/2}), \forall z\in\mathbb{C}\setminus\mathbb{R}$, with $z=(\zetat-A_1^B)/B_1^B$. (ii) This $(\zetat-A_1^B)^{-3/2}$ {\yc divergence} occurs at a position $A_1^B$ depending on $\kb$, as shown in equation (\ref{eq222}), so it is generally washed out by integration over $\kb$ at fixed $\zetat$ in (\ref{eq220}), unless $\zetat$ coincides with a parabolic extremum of $A_1^B(\kb)$ on the integration domain $\mathcal{D}_1^B$. To see this, let's choose an arbitrary reduced wavenumber $\kb_0$ in $\mathcal{D}_1^B$, let's set $\zetat=w+A_1^B(\kb_0)$ and take the limit $w\to 0$ by positive imaginary parts in (\ref{eq220}): the result can only diverge if the value of the integral 
\begin{equation}\label{eq4005}
\sigma_{\kb_0}(w)=\int_{\kb_0-\veps}^{\kb_0+\veps} \frac{\dd\kb}{[w+A_1^B(\kb_0)-A_1^B(\kb)]^{3/2}}
\end{equation} diverges in this limit, where $\veps>0$ is fixed. Since we can take $\veps$ to be as small as we like without changing the conclusion, we can replace $A_1^B(\kb)-A_1^B(\kb_0)$ in (\ref{eq4005}) by its Taylor expansion in $\kb_0$ to the non-zero leading order. If $(\mathrm{d}A_1^B/\mathrm{d}\kb)(\kb=\kb_0)\neq 0$, an expansion to order one is sufficient, and there is no divergence because $\int_{\mathbb{R}} \mathrm{d}\kappa/(w-\kappa)^{3/2}=0$. \footnote{If $\kb_0$ lies on the edge of $\mathcal{D}_1^B$, the neighborhood of $\kb_0$ reduces to $[\kb_0,\kb_0+\veps]$ or $[\kb_0-\veps,\kb_0]$ and there is square-root divergence because $\int_0^{+\infty} \mathrm{d}\kappa/(w-\kappa)^{3/2}=-2/w^{1/2}$. This divergence is subleading and ignored here.} Otherwise, expansion to second order is required, and there is linear divergence $[\zetat-A_1^B(\kb_0)]^{-1}$ as predicted by (\ref{eq4000}) because 
\begin{equation}\label{eq4001}
\int_{\mathbb{R}} \frac{\mathrm{d}\kappa}{(w-\kappa^2)^{3/2}} = \frac{2}{\textrm{i}w} \quad\mbox{and}\quad \int_{\mathbb{R}} \frac{\mathrm{d}\kappa}{(w+\kappa^2)^{3/2}} = \frac{2}{w}\quad\forall w\in \mathbb{C}, \im w>0
\end{equation}
 Let's go into more detail on the example of diagram $B_1$ in Figure \ref{fig8}, in which the signs $s_i$ of the internal line orientations are all $1$, the integration domain on $\kb$ is $\mathcal{D}_1^B=[0,\qb]$ and that on $\kb'$ at fixed $\kb$ is $\mathcal{D}_2^B(\kb)=[0,\kb]$. Note that $A_1^B(\kb)$ reaches its minimum at $\kb_0=\qb/2$, {\yct in the interior of} $\mathcal{D}_1^B$ and therefore with zero derivative, and that this minimum is precisely $\zetat_0=-3\gamma\qb^3/32$, as in (\ref{eq4000}). To obtain an equivalent of $\tilde{\Sigma}_{B_1}^{(4,2)}(\qb,\zetat)$ when $\zetat\to\zetat_0$, we can treat $\thetat^2$, $(\kb-\qb/2)^2$ and $w=\zetat-\zetat_0$ as infinitesimals of the same order. So, in the angular integral (\ref{eq221}), we can replace $\zetat$ by $\zetat_0+\textrm{i}0^+$ and $\thetat^2$ by $0$ in the second factor in the denominator, and everywhere replace $\kb$ by $\kb_0=\qb/2$ except in coefficient $A_1^B(\kb)$, which is written as $A_1^B(\kb)=\tilde{\zeta}_0+(3\gamma\qb/8)(\kb-\qb/2)^2$. After calculating the integral over $\phi$ and then over $\thetat$ using the residue theorem, it remains
\begin{equation}\label{eq4002}
I(\kb,\kb') \sim \frac{-8\pi^2}{(6\gamma\qb^3)^{1/2} \kb'}\frac{1}{[w-(3\gamma/8)\qb(\kb-\qb/2)^2]^{3/2}}
\end{equation} which can be inserted in the integrand of (\ref{eq220}), replacing $\kb$ by $\kb_0=\qb/2$ in all other factors and in the upper bound of integration over $\kb'$. {\yc We then change the order of integration.} In the {\yc now internal} integral over $\kb$ at fixed $\kb'$, it only remains
\begin{equation}\label{eq4003}
\int_0^{\qb} \frac{\mathrm{d}\kb}{[w-(3\gamma/8)\qb(\kb-\qb/2)^2]^{3/2}}\sim \left(\frac{8}{3\gamma\qb}\right)^{1/2}\int_{-\infty}^{+\infty} \frac{\dd\kappa}{(w-\kappa^2)^{3/2}}=\left(\frac{8}{3\gamma\qb}\right)^{1/2} \frac{2}{\textrm{i}w}
\end{equation} where we have made the change of variable $\kb=\qb/2+[8/(3\gamma\qb)]^{1/2}\kappa$, extended the integration bounds over $\kappa$ to $\pm\infty$ without changing the leading behavior and used equation (\ref{eq4001}). We finally obtain the divergent part of diagram $B_1$: 
\begin{equation}\label{eq4004}
\tilde{\Sigma}_{B_1}^{(4,2)}(\qb,\zetat) \stackrel{\im\zetat>0}{\underset{\zetat\to \zetat_0}{\sim}} \frac{\qb^3(\bar{n}^{\rm lin}_{q/2}+{1}/{2})}{24\ii\pi^2\gamma(\zetat-\zetat_0)} \left(\frac{9(1+\Lambda)^2}{8\rho\xi^2}\right)^2\int_0^{\qb/2} \dd\kb' \kb'(\kb'-\qb/2)(\bar{n}^{\rm lin}_{k'}-\bar{n}^{\rm lin}_{k'-{q}/{2}})
\end{equation} Diagram $B_2$ is treated in the same way and gives a similar divergent contribution (with a two times larger prefactor and the integral over $\kb'$ from $\qb/2$ to $\eta$). Diagrams $B_3$ and $B_4$ are subleading in $\epsilon$ because $s_5\neq s_2$ and must be omitted here. Diagrams $B_5$ and $B_6$ lead to a strictly increasing coefficient $A_1^B(\kb)=(3\gamma/8)\qb\kb(\kb+\qb)$ on the integration interval $\kb\in[0,\eta-\qb]$ and therefore cannot contribute to the linear divergence (\ref{eq4000}). Diagrams $B_7$ and $B_8$ have the same problem, with $A_1^B(\kb)=(3\gamma/8)\qb\kb(\kb-\qb)$ and $\mathcal{D}_1^B=[\qb,\eta]>\qb/2$. It remains to combine the contributions of $B_1$ and $B_2$ to obtain (\ref{eq4000}), whose expression between square brackets in the quantum field case tends towards $-\qb(16\pi^2+\qb^2)/48$ when $\eta\to+\infty$. 

Finally, let's explain why bridge diagrams do not contribute to the linear divergence (\ref{eq4000}) on the example of diagram $P_1$ (all $s_i=1$, $\mathcal{D}_1^P=[0,\qb]$ and $\mathcal{D}_2^P(\kb)=[0,\kb]$, type II). As noted in section \ref{sec3.3}, the angular integral (\ref{eq243}) diverges linearly at {\yc some} $\zetat=X_0(\kb,\kb')$ if the cancellation loci of the three factors in the denominator of (\ref{eq243}) intersect at the same point in the $(\tilde{\theta},\tilde{\theta}')$ plane. The triple intersection condition {\ycd in the general case $X_0\neq 0$} is obtained by replacing all coefficients $A_i^P$ by $A_i^P-X_0$ in (\ref{eq281}); the resulting second-degree equation on $X_0$ has the obvious zero solution, to which section \ref{sec3.3} was implicitly restricted, and a second real root, $X_0(\kb,\kb')=-3\gamma \kb(\kb-\qb)\kb'(\kb'-\qb)/(2\qb)$ ; the latter leads to a triple intersection in the plane $(\tilde{\theta},\tilde{\theta}')$, at points with coordinates $\pm[\sqrt{3\gamma}/(2\qb)]((\qb-\kb)(\qb-2\kb'),-(\qb-2\kb)(\qb-\kb'))$, if we add constraints $\kb>\qb/2$ and $\kb'<\qb/2$ to have the right signs in the coefficients of the small deviations $(\tau,\tau')$ at these points (otherwise, as in (\ref{eq282}), there is no divergence). Then, proceeding as in (\ref{eq282}), we obtain the divergent part of the angular integral (for $\zetat\to X_0$ by positive imaginary parts), 
\begin{equation}\label{eq4006}
K_{\rm div}(\kb,\kb') = \frac{-32\pi^2}{3\gamma\qb\kb\kb'(\kb-\kb')[\zetat-X_0(\kb,\kb')]}
\end{equation} to be integrated on $(\kb,\kb')\in[\qb/2,\qb]\times[0,\qb/2]$ after multiplication by the same function of $(\kb,\kb')$ as in (\ref{eq242}). Now $X_0(\kb,\kb')$ has a parabolic minimum, located at $(\kb,\kb')=(\qb/2,\qb/2)$ and precisely equal to $\zetat_0$! The singularity (\ref{eq4006}) will therefore survive the wave number integration if $\zetat\to \zetat_0$ as in (\ref{eq4000}). Let's then perform the change of variables $(\kb=\kappa+\qb/2,\kb'=-\kappa'+\qb/2)$ and treat $w=\zetat-\zetat_0$, $\kappa^2$ and $\kappa'^2$ as infinitesimals of the same order. By writing the function multiplying $[\zetat-X_0(\kb,\kb')]^{-1}$ to the leading order in $\kappa$ and $\kappa'$ and replacing the pole $X_0(\kb,\kb')$ by its quadratic approximation in $\kappa$ and $\kappa'$ near the minimum, we obtain the equivalent 
\begin{equation}\label{eq4008}
\tilde{\Sigma}^{(4,2)}_{P_1^{\rm II}}(\qb,\zetat)\stackrel{\im w>0}{\underset{w\to 0}{\sim}} (\qb/2)^{\ycd 4} \left(\frac{9(1+\Lambda)^2}{8\rho\xi^2}\right)^2 (2\bar{n}^{\rm lin}_{q/2}+1)\frac{(-32\pi^2)}{3\gamma(2\pi)^4} \int_0^{\qb/2}\dd\kappa \int_0^{\qb/2} \dd\kappa' \frac{1}{\kappa+\kappa'} \frac{1}{w-(3\gamma\qb/8)(\kappa^2+\kappa'^2)}
\end{equation} We can extend the integration bounds $\qb/2$ to $+\infty$ without losing equivalence. The double integral is then easily calculated in polar coordinates $(K,\alpha)$, where $K=(\kappa^2+\kappa'^2)^{1/2}$ spans $\mathbb{R}^+$ and $\alpha$ spans $[0,\pi/2]$. We finally find a square-root divergence, $\propto (\zetat-\zetat_0)^{-1/2}$, subleading in (\ref{eq4000}).

\section{The long-time behavior of the signal in the $(\rho\xi^2)^{-1}\to 0$ limit of non-perturbative theories}
\label{app5}

Here we obtain the asymptotic behavior (\ref{eq314}) of the signal associated with the self-energy $\check{\Sigma}_{{\rm corr}\,\infty}^{(2,2)}(\qb,\check{\zeta})$ of equation (\ref{eq311}). To this end, we apply the technique of \ref{app2} by folding back the integration path $C_+$ in (\ref{eq012}) onto the branch cut of $\check{\Sigma}_{{\rm corr}\,\infty}^{(2,2)}(\qb,\check{\zeta})$ located on $\ii\mathbb{R}^{-}$ and described parametrically by $\zeta=-\ii\check{y}$, $\check{y}\geq 0$ (this branch cut is the set of values of $\zetac$ cancelling the denominator of the integrand in (\ref{eq311})). Transposing equation (\ref{eq2001}) to the present case, we obtain 
\begin{equation}\label{eq5001}
s^{(2,2)}_{{\rm corr}\,\infty}(t)\underset{\check{t}\to+\infty}{\sim}\int_0^{+\infty}\frac{\dd\check{y}}{2\pi}\frac{\delta\sigma_{\qb}(\check{y})\exp(-\check{y}\check{t})}{\left[\check{\Sigma}_{{\rm corr}\,\infty}^{(2,2)}(\qb,0)\right]^2}\quad\mbox{where }\quad
\delta\sigma_{\qb}(\check{y})=\check{\Sigma}_{{\rm corr}\,\infty}^{(2,2)}(\qb,0^+-\ii\check{y})-\check{\Sigma}_{{\rm corr}\,\infty}^{(2,2)}(\qb,0^--\ii\check{y})
\end{equation} is the discontinuity of the self-energy across its branch cut at distance $\check{y}$ from the origin. The identity in the sense of distributions $1/(\check{y}+\textrm{i}0^+)=\mathrm{v.p.}(1/\check{y})-\ii\pi\delta(\check{y})$ leads to 
\begin{equation}\label{eq5002}
\delta\sigma_{\qb}(\check{y})=-\frac{4\pi}{\qb^2}\int_{\qb}^{\eta}\mathrm{d}\bar{k}\int_0^{\eta-\kb}\dd\kb' \bar{k}\kb' (\kb-\qb)(\kb'+\qb) (\bar{n}_{k'+q}^{\rm lin}-\bar{n}_{k'+k}^{\rm lin})(\bar{n}_{k-q}^{\rm lin}-\bar{n}_k^{\rm lin})
\delta\left(\check{y}-\kb'(\kb-\qb)\frac{\check{\Gamma}_{\kb-\qb}+\check{\Gamma}_{\kb'}}{2\qb(\kb+\kb')}\right)
\end{equation} In the $\check{t}\to+\infty$ limit, the exponential function in (\ref{eq5001}) restricts the integral to arbitrarily small values of $\check{y}$ and it suffices to know the leading behavior of $\delta\sigma_{\qb}(\check{y})$ when $\check{y}\to 0^+$. Since the reduced rates $\check{\Gamma}_{\bar{K}}$ cannot tend to zero, the argument of the Dirac distribution in (\ref{eq5002}) can be zero in the limit $\check{y}\to 0$ only if $\kb'$ or $\kb-\qb$ is infinitesimal. For an arbitrary choice of a bound $\veps$ that is not infinitesimal but $\ll 1$, we can then restrict the integration domain on $\kb$ and $\kb'$ to the two parts, (i) $\kb\in[\qb,\qb+\veps]$, $\kb'\in[0,\eta-\kb]$ and (ii) $\kb\in [\qb+\veps,\eta-\veps]$, $\kb'\in[0,\veps]$. \footnote{We note in part (ii) that, if $\kb\leq\eta-\veps$, we have $\veps\leq\eta-\kb$ so $\kb'$, which must remain $\leq \eta-\kb$, can indeed go as far as $\veps$. We also check that \g{missing bit} $\kb\in[\eta-\veps,\eta]$, $\kb'\in[0,\veps]$ gives a negligible contribution $O(\check{y}\veps)$ to $\delta\sigma_{\qb}(\check{y})$.} In part (i) of the integration domain, we have $\kb-\qb\leq\veps\ll 1$ so we can replace the various quantities depending on $\kb$ in (\ref{eq5002}) by their Taylor expansion at $\kb=\qb$ to leading order, e.g. $(\kb-\qb)(\bar{n}^{\rm lin}_{k-q}-\bar{n}^{\rm lin}_{k})\simeq 1$, $\bar{n}_{k'+q}^{\rm lin}-\bar{n}_{k'+k}^{\rm lin}\simeq(\kb-\qb)(-\dd/\dd\kb')\bar{n}_{k'+q}^{\rm lin}$ {\yct or} $\check{\Gamma}_{\kb-\qb}\simeq \check{\Gamma}_0$ and the {\yc upper} integration bound on $\kb'$ can be replaced by $\eta-\qb$. The argument of the Dirac distribution in (\ref{eq5002}) becomes an affine function of $\kb-\qb$, making it easy to integrate over $\kb$: 
\begin{equation}\label{eq5003}
\delta\sigma_{\qb}^{\rm (i)}(\check{y}) \simeq -16\pi\qb\check{y}\int_0^{\eta-\bar{q}} \dd\kb'\frac{(\kb'+\qb)^3}{(\check{\Gamma}_0+\check{\Gamma}_{\kb'})^2}\left(-\frac{\dd}{\dd\kb'}\bar{n}_{k'+q}^{\rm lin}\right) Y\left(\veps-\frac{2(\kb'+\qb)\qb\check{y}}{\kb'(\check{\Gamma}_0+\check{\Gamma}_{\kb'})}\right) \simeq -16\pi\qb\check{y}\int_{\qb^2\check{y}/\veps\check{\Gamma}_0}^{\eta-\qb}\frac{\dd\kb'}{\kb'} \frac{(\kb'+\qb)^3}{(\check{\Gamma}_0+\check{\Gamma}_{\kb'})^2} \left(-\frac{\dd}{\dd\kb'}\bar{n}_{k'+q}^{\rm lin}\right)
\end{equation} The form of the {\yct second integral in} (\ref{eq5003}) is based on the following reasoning: if $\check{y}\to 0$ at fixed $\veps$, the argument of the Heaviside function $Y$ in the {\yct first integral} is positive, unless $\kb'\to 0$; at the edge of the support of this function $Y$, we therefore have $\kb'\ll 1$, which allows us to replace in its argument the second term by its leading-order approximation in $\kb'$; this brings up $Y(\veps-\qb^2\check{y}/\kb'\check{\Gamma}_0)$, hence the lower bound announced in the integration over $\kb'$. In part (ii) of the integration domain, we proceed in the same way: since $\kb'\leq\veps\ll 1$, we replace the various quantities in (\ref{eq5002}) by their leading order in $\kb'$. As the argument of the Dirac distribution becomes an affine function of $\kb'$, we integrate over this variable; the resulting function $Y(\veps-\frac{2\qb\kb\check{y}}{(\kb-\qb)(\check{\Gamma}_{\kb-\qb}+\check{\Gamma}_0)})$ is inoperative when $\check{y}\to 0$ because $\kb-\qb$ is here $\geq\veps$, hence 
\begin{equation}\label{eq5004}
\delta\sigma_{\qb}^{\rm (ii)}(\check{y}) \simeq -16\pi\qb\check{y} \int_{\qb+\veps}^{\eta-\veps} \frac{\dd\kb}{\kb-\qb}\frac{\kb^3}{(\check{\Gamma}_{\kb-\qb}+\check{\Gamma}_0)^2} (\bar{n}_q^{\rm lin}-\bar{n}_k^{\rm lin})(\bar{n}_{k-q}^{\rm lin}-\bar{n}_k^{\rm lin})
\end{equation} It remains to take the limit $\veps\to 0$ in results (\ref{eq5003},\ref{eq5004}). To do this, we separate each integrand into its infrared divergent part $\propto 1/\kb'$ or $\propto 1/(\kb-\qb)$ and its regular part.\footnote{To calculate the coefficient of the divergent part $1/(\kb-\qb)$, we use the fact that $\lim_{\kb\to\qb}(\bar{n}_q^{\rm lin}-\bar{n}_k^{\rm lin})(\bar{n}_{k-q}^{\rm lin}-\bar{n}_k^{\rm lin})=(-\dd/\dd\qb)\bar{n}_q^{\rm lin}.$
} In the integral of the regular parts, we can make $\veps$ tend to zero within the integration bounds without damage. In the integral of divergent parts, doing the same gives rise to dangerous contributions $\pm\ln\veps$, but which exactly cancel out in the sum of parts (i) and (ii). Finally, 
\begin{equation}\label{eq5005}
\delta\sigma_{\qb}(\check{y})\underset{\check{y}\to 0^+}{=}\frac{4\pi\qb^4}{\check{\Gamma}_0^2} \left(-\frac{\dd}{\dd\qb}\bar{n}^{\rm lin}_{{\yct q}}\right) \check{y}\ln\check{y} + C_{\qb} \check{y} + O(\check{y}^2\ln\check{y})
\end{equation} For the sake of brevity, we do not give the expression of the coefficient $C_{\qb}$ of the linear part in (\ref{eq5005}), to which our method nevertheless gives access. The form of $O(\ldots)$ in (\ref{eq5005}), on the other hand, has been obtained numerically. Insertion of the leading term $\propto\check{y}\ln\check{y}$ in (\ref{eq5001}) leads to the announced result (\ref{eq314}).

%%%%%%%%%%%%%%%%%%%%%%%%%%%
\begin{thebibliography}{99}
%\input biblio_en.txt
\bibitem{LK} L. Landau, I. Khalatnikov, \g{Teoriya vyazkosti Geliya-II}, Zh. Eksp. Teor. Fiz. {\bf 19}, 637 (1949).
\bibitem{SonWingate} D.T. Son, M. Wingate, \href{https://doi.org/10.1016/j.aop.2005.11.001}{\g{General coordinate invariance and conformal invariance in nonrelativistic physics: Unitary Fermi gas}}, Ann. Physics {\bf 321}, 197 (2006).
\bibitem{SDM} S. Van Loon, C.A.R. Sá de Melo, \href{https://doi.org/10.1103/PhysRevLett.131.113001}{\g{Effects of quantum fluctuations on the low-energy collective modes of two-dimensional superfluid Fermi gases from the BCS to the Bose Limit}}, {\ycd Phys. Rev. Lett. {\bf 131}, 113001 (2023)}.
\bibitem{livreK} I. Khalatnikov, \textit{An Introduction to the Theory of Superfluidity} (CRC Press-Taylor \& Francis, Boca Raton, 2018).
\bibitem{Annalen} H. Kurkjian, Y. Castin, A. Sinatra, \href{https://hal.archives-ouvertes.fr/hal-01392846}{\g{Three-phonon and four-phonon interaction processes in a pair-condensed Fermi gas}}, Annalen der Physik {\bf 529}, 1600352 (2017).
\bibitem{Maris} H.J. Maris, \href{https://doi.org/10.1103/RevModPhys.49.341}{Phonon-phonon interactions in liquid helium}, Rev. Mod. Phys. {\bf 49}, 341 (1977).
\bibitem{FW} A.L. Fetter, J.D. Walecka, \textit{Quantum Theory of Many-Particle Systems} (Dover, Mineola, 2003).
\bibitem{Grynberg} D.R. Meacher, D. Boiron, H. Metcalf, C. Salomon, G. Grynberg, \href{https://doi.org/10.1103/PhysRevA.50.R1992}{\g{Method for velocimetry of cold atoms}}, Phys. Rev. A {\bf 50}, R1992(R) (1994).
\bibitem{Ketterle} J. Stenger, S. Inouye, A.P. Chikkatur, D.M. Stamper-Kurn, D.E. Pritchard, W. Ketterle, \href{https://link.aps.org/doi/10.1103/PhysRevLett.82.4569}{\g{Bragg spectroscopy of a Bose-Einstein condensate}}, Phys. Rev. Lett. {\bf 82}, 4569 (1999).
\bibitem{Davidson} J. Steinhauer, R. Ozeri, N. Katz, N. Davidson, \href{https://link.aps.org/doi/10.1103/PhysRevLett.88.120407}{\g{Excitation spectrum of a Bose-Einstein condensate}}, Phys. Rev. Lett. {\bf 88}, 120407 (2002).
\bibitem{Vale} G. Veeravalli, E. Kuhnle, P. Dyke, C.J. Vale, \href{https://link.aps.org/doi/10.1103/PhysRevLett.101.250403}{\g{Bragg spectroscopy of a strongly interacting Fermi gas}}, Phys. Rev. Lett. {\bf 101}, 250403 (2008).
\bibitem{Cartago} A. Sinatra, C. Lobo, Y. Castin, \href{https://doi.org/10.1088/0953-4075/35/17/301}{\g{The truncated Wigner method for Bose condensed gases: limits of validity and applications}}, J. Phys. B {\bf 35}, 3599 (2002).
\bibitem{CCTbordeaux} C. Cohen-Tannoudji, J. Dupont-Roc, G. Grynberg, \textit{Processus d'interaction entre photons et atomes} (EDP Sciences/CNRS Éditions, Paris, 1988).
\bibitem{Barc} {\yct M.A. Escobedo, C. Manuel, \href{https://doi.org/10.1103/PhysRevA.82.023614}{\g{Effective field theory and dispersion law of the phonons of a nonrelativistic superfluid}}, Phys. Rev. A {\bf 82}, 023614 (2010).}
\bibitem{AK} A. Andreev, I.M. Khalatnikov, \href{http://jetp.ras.ru/cgi-bin/dn/e_017_06_1384.pdf}{\g{Sound in Liquid Helium II Near Absolute Zero}}, Zh. Eksp. Teor. Fiz. {\bf 44}, 2058 (1963) [Sov. Phys. JETP {\bf 17}, 1384 (1963)].
\bibitem{CB} Ming-Chiang Chung, A.B. Bhattacherjee, \href{https://doi.org/10.1088/1367-2630/11/12/123012}{\g{Damping in 2D and 3D dilute Bose gases}},  New J. Phys. {\bf 11}, 123012 (2009).
\bibitem{HM} M. Bohlen, L. Sobirey, N. Luick, H. Biss, T. Enss, Th. Lompe, H. Moritz,\href{https://journals.aps.org/prl/abstract/10.1103/PhysRevLett.124.240403}{\g{Sound Propagation and Quantum-Limited Damping in a Two-Dimensional Fermi Gas}}, Phys. Rev. Lett. {\bf 124}, 240403 (2020).
\bibitem{PCZH} P. Christodoulou, M. Gałka, N. Dogra, R. Lopes, J. Schmitt, Z. Hadzibabic, \href{https://doi.org/10.1038/s41586-021-03537-9}{\g{Observation of first and second sound in a BKT superfluid}}, Nature {\bf 594}, 191 (2021).
\bibitem{JDJB} J.L. Ville, R. Saint-Jalm, \'E. Le Cerf, M. Aidelsburger, S. Nascimb\`ene, J. Dalibard, J. Beugnon, \href{https://doi.org/10.1103/PhysRevLett.121.145301}{\g{Sound Propagation in a Uniform Superfluid Two-Dimensional Bose Gas}}, Phys. Rev. Lett. {\bf 121}, 145301 (2018).
\bibitem{Salasnich} A. Cappellaro, F. Toigo, L. Salasnich, \href{https://doi.org/10.1103/PhysRevA.98.043605}{\g{Collisionless dynamics in two-dimensional bosonic gases}}, Phys. Rev. A {\bf 98}, 043605 (2018).
\bibitem{Stringari} Miki Ota, F. Larcher, F. Dalfovo, L. Pitaevskii, N.P. Proukakis, S. Stringari, \href{https://doi.org/10.1103/PhysRevLett.121.145302}{\g{Collisionless Sound in a Uniform Two-Dimensional Bose Gas}}, Phys. Rev. Lett. {\bf 121}, 145302 (2018).
\bibitem{Bere} V.L. Berezinskii, \href{http://jetp.ras.ru/cgi-bin/dn/e_034_03_0610.pdf}{\g{Destruction of long-range order in one-dimensional and two-dimensional systems having a continuous symmetry group II. Quantum systems}}, Zh. Eksp. Teor. Fiz. {\bf 61}, 1144 (1971) [Sov. Phys. JETP {\bf 34}, 610 (1972)].
\bibitem{Nel} D.R. Nelson, J.M. Kosterlitz, \href{https://doi.org/10.1103/PhysRevLett.39.1201}{\g{Universal Jump in the Superfluid Density of Two-Dimensional Superfluids}}, Phys. Rev. Lett. {\bf 39}, 1201 (1977).
\bibitem{SJ} R. Saint-Jalm, \href{https://theses.hal.science/tel-03116129}{\it Exploration de la physique à deux dimensions avec des gaz de Bose dans des potentiels à fond plat: ordre en phase et symétrie dynamique}, PhD thesis of École normale supérieure and PSL University (Paris, October 2019).
\bibitem{broui} H. Kurkjian, Y. Castin, A. Sinatra, \href{https://doi.org/10.1016/j.crhy.2016.02.005}{Brouillage thermique d'un gaz cohérent de fermions}, Comptes Rendus Physique {\bf 17}, 789 (2016).
\bibitem{CD} Y. Castin, R. Dum, \href{https://doi.org/10.1007/s100530050584}{\g{Bose-Einstein condensates with vortices in rotating traps}}, Eur. Phys. J. D {\bf 7}, 399 (1999).
\bibitem{HD} Z. Hadzibabic, J. Dalibard, \href{https://doi.org/10.1393/ncr/i2011-10066-3}{\g{Two-dimensional Bose fluids: An atomic physics perspective}}, Rivista del Nuovo Cimento {\bf 34}, 389 (2011).
\bibitem{Popov} V.N. Popov, {\it Functional Integrals in Quantum Field Theory and Statistical Physics} (Reidel, Dordrecht, 1983).
\bibitem{Morathese} C. Mora, \href{https://theses.hal.science/tel-00005472v1}{\it Gaz de bosons et de fermions condens\'es~: phases de Fulde-Ferrell-Larkin-Ovchinnikov et quasicondensats}, PhD thesis of Universit\'e Paris VI (Paris, March 2004).
\bibitem{MC2} C. Mora, Y. Castin, \href{https://doi.org/10.1103/PhysRevLett.102.180404}{\g{Ground state energy of the two-dimensional weakly interacting Bose gas: First correction beyond Bogoliubov theory}}, Phys. Rev. Lett. {\bf 102}, 180404 (2009). 
\bibitem{Petrov} {\yc D.S. Petrov, G.V. Shlyapnikov, \href{https://doi.org/10.1103/PhysRevA.64.012706}{\g{Interatomic collisions in a tightly confined Bose gas}}, Phys. Rev. A {\bf 64}, 012706 (2001).}
\bibitem{Olshanii} {\yc L. Pricoupenko, M. Olshanii, \href{https://doi.org/10.1088/0953-4075/40/11/009}{\g{Stability of two-dimensional Bose gases in the resonant regime}}, J. Phys. B {\bf 40}, 2065 (2007).}
\bibitem{Ludovic} {\yc L. Pricoupenko, \href{https://doi.org/10.1103/PhysRevA.83.062711}{\g{Isotropic contact forces in arbitrary representation: Heterogeneous few-body problems and low dimensions}}, Phys. Rev. A {\bf 83}, 062711 (2011).}
\bibitem{TheseAlan} A. Serafin, \href{https://theses.hal.science/tel-04002097}{\textit{Deux études de propriétés collectives de systèmes quantiques}}, PhD thesis of Sorbonne University (Paris, December 2022).
\bibitem{NumRec} W.H. Press, S.A. Teukolsky, W.T. Vetterling, B.P.  Flannery, {\it Numerical Recipes} (Cambridge University Press, Cambridge, 1988).
\bibitem{SCE} A. Sinatra, Y. Castin, E. Witkowska, \href{https://doi.org/10.1103/PhysRevA.75.033616}{\g{Nondiffusive phase spreading of a Bose-Einstein condensate at finite temperature}}, Phys. Rev. A {\bf 75}, 033616 (2007).
\end{thebibliography}
\end{document}
%}
%)
