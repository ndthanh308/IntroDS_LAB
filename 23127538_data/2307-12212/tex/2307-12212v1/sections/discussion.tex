\section{Discussion and Future Work}\label{sec:discussion}

%
%

%

Currently, IPFS does not provide an admission mechanism and \resolvers will accept any provider records until they run out of storage. After that, depending on the implementation, the \resolvers may crash or flush older, legitimate provider records. The time required for this attack depends on the bandwidth available at each \resolver and the amount of free storage. While a deeper analysis is out of the scope of this paper, introducing an admission mechanism based on the diversity of incoming traffic has the potential to eliminate this vulnerability.

While our mitigation technique (\Cref{sec:mitigation}) fully protects against the CID censorship attack, it involves querying the Sybil nodes for provider records. The Sybil nodes may return a large number of fake provider records so that \downloaders keep trying them, thereby slowing down the resolution. The impact of such an attack can be reduced if the \downloader only tries a single provider record obtained from each \resolver and prefers records obtained from \resolvers with diverse IP addresses (\ie from different /24 networks). Any attempts to significantly delay the resolution would sharply increase the attacker's cost. 

Our mitigation technique relies on region-based DHT queries. For easy integration with the current IPFS network, those queries are built on top of a regular Kademlia DHT that does not natively support them. This results in slightly higher overhead and increases resolution time. Adapting the core internals of the DHT and optimizing them for region-based queries might speed up the process and reduce its overhead. However, such deep changes make incremental deployment and compatibility with the existing version challenging. 

During this project, we initially considered an approach where \providers register their provider records on all the nodes encountered on the path towards \resolvers. Such a solution increases the chance of an honest \downloader receiving a correct provider record before reaching the region with the Sybil nodes. However, the mechanism does not provide resistance for \downloaders located close to the CID in the DHT hash space. Furthermore, the on-path registration significantly increases the storage cost of holding provider records for the entire network even when no attack is being conducted. 

The IPFS DHT, and thus our mitigation and detection mechanisms, depends on the correctness of the DHT routing. However, a powerful attack may try to disturb DHT operations by deploying a large number of uniformly distributed Sybils that only return other Sybils when queried. While costly, such an attack could be devastating for the entire ecosystem. We advocate for additional future work that improves the DHT resistance to such attacks and is practical to deploy in large-scale networks. 

%

%
