%
\vspacebeforesection
\section{Introduction}
%

%
%
%
%
%
%
%
%
%
%
%
%
%

%
%
%
%

%
Inter-Planetary FileSystem (IPFS) is the largest decentralized peer-to-peer filesystem currently in operation. The platform underpins various decentralized web applications~\cite{ecosystem}, including social networking and discussion (Discussify~\cite{discussify}, Matters News~\cite{matters}), data storage (Space~\cite{Space}, Peergos~\cite{Peergos}, Temporal~\cite{temporal}), content search (Almonit~\cite{almonit}, Deece~\cite{deece}), messaging (Berty~\cite{Berty}), content streaming (Audius~\cite{Audius}, Watchit~\cite{Watchit}, DTube~\cite{dtube}), gaming (Gala~\cite{gala}, Splinterlands~\cite{Splinterlands}), and e-commerce (Ethlance~\cite{ethlance}, dClimate~\cite{dClimate}). IPFS is widely used as external storage for blockchain-based applications, including valuable NFT platforms. Support for accessing IPFS has further been integrated into HTTP gateways (\eg, Cloudflare) and mainstream browsers such as Opera and Brave, allowing easy uptake. 
The IPFS network currently contains a steady number of 25,000 online nodes, spread across 2,700 Autonomous Systems and 152 countries, according to a recent study~\cite{trautwein2022design} that also observed widespread usage by clients with 7.1 million content retrieval operations observed from a single vantage point and during a single day.

%
%
%
IPFS is a content-centric network where each piece of content is identified by a Content Identifier (CID), similarly to BitTorrent~\cite{bittorrent} or Content-Centric Networking~\cite{zhang2014named}.
CIDs are derived by hashing the content and do not embed any network location information. %
%
Such an approach enables easy content deduplication and the retrieval of data from the closest available location, in addition to maintaining data integrity.
%
%
%
%
%
%

%
Data retrieval in a content-centric network requires mapping CIDs into network identifiers (\ie IP addresses and port numbers) of nodes hosting the content, called \emph{\providers}. 
%
Without this resolution mechanism, nodes willing to fetch data, or \emph{\downloaders}, have no means to know where to send their requests for data. 
The design of IPFS results from decades of research on how to build efficient P2P systems~\cite{lua2005survey,androutsellis2004survey}.
It uses resolution based on a Distributed Hash Table (DHT) combined with Bitswap, a flooding-based, unstructured search mechanism. Similarly to systems such as Gnutella~\cite{ripeanu2001peer}, \downloaders use Bitswap to establish connections to random peers in the network and send them content queries. Bitswap acts as a lightweight cache and speeds up the retrieval of popular content, but cannot provide discovery guarantees, in particular for newer or less popular data.

Reliable content discovery is provided by the DHT-based resolution system. Nodes hosting content advertise themselves as \providers in the network. First, they create \emph{provider records} linking their hosted content (identified by CIDs) to their network location (\ie, IP address and port number). Second, the \providers send the provider records to be stored on a fixed number of designated nodes. We refer to those nodes as \emph{\resolvers}.  
\Downloaders wishing to fetch the content contact the same \resolvers, retrieve the relevant provider records, and then directly contact the discovered \providers to download the data. The DHT guarantees to find the content if it is stored in the network.
Its proper operation is, therefore, of paramount importance to ensure content availability.
%
%
IPFS uses the \texttt{libp2p} implementation~\cite{libp2p_github} of the Kademlia DHT~\cite{maymounkov2002kademlia}.

%
%
%
%
%
%
%

\para{Contributions}
We make four main contributions.

First, we present a content censorship attack targeting the main IPFS DHT-based resolution system.
The attack relies on strategically placing Sybil identities in the network so that they replace honest \resolvers for a given CID.
As a result, \downloaders cannot discover \provider records for the target CID and are unable to download the content.
The attack can be performed from a single, resource-constrained machine at very little cost (\$4 using AWS) and makes the \provider records unavailable after a time that ranges from a few seconds to up to 48h depending on the initial setup.
%
Currently, IPFS has no mechanisms to counter the attack, threatening the 
%
%
security of systems using IPFS as a storage platform.
This includes collaborative file hosting solutions such as Filecoin~\cite{psaras2020interplanetary} and systems building upon it~\cite{huang2020secure, de2021accelerating}.
It also concerns the many proposals combining IPFS for storage with blockchain-hosted application logic, e.g., to implement social networks~\cite{xu2018building}, domain-specific data sharing applications~\cite{jianjun2020research,mukne2019land}, or decentralized equivalents to centralized services such as ride-sharing~\cite{hossan2021securing}.

%

Second, we present a reliable attack detection technique that analyzes the distribution of peer IDs in the network using the KL Divergence metric~\cite{g-test}. 
This method extends previous work~\cite{cholez2010detection} and leverages a local density-based network size estimator~\cite{eli-sohl-dht-size-estimation,kostoulas2005decentralized, manku2003symphony} to automatically adjust the detection to the dynamic size of the IPFS network.
The detection allows \providers to execute mitigation techniques, which may be more costly than the default mode, only when an attack is detected, thereby  minimizing the overhead when there is no attack. 
%
The detection can be performed by any node during regular content resolution operations and does not incur any additional message overhead.
In our experiments on the live IPFS network, our attack detection method was able to detect 99\% of the attacks with a false positive rate of 4\%, while allowing users to trade off these rates based on individual preferences.
%
A higher detection rate ensures better security while also leading to more false positives that increase the overhead when there is no attack.

Third, we introduce a mitigation technique that allows us to reliably discover provider records regardless of the number of Sybil nodes placed by an attacker around the target CID.
%
The mitigation replaces the regular \emph{put} and \emph{get} DHT operation by hash space region-based queries. 
Using these, \providers always find honest \resolvers to store their provider records and  querying nodes always discover these honest \resolvers and receive true provider records. While introducing an overhead linear
in the number of Sybil nodes placed close to the target CID, this mitigation is only enforced when suspicions exist about the existence of an attack, as indicated by our detection mechanism.
%
%
%
%

%
%
%
%
%
%

Finally, we implement the attack using a custom IPFS DHT server node, and we implement our detection and mitigation techniques on top of the \texttt{libp2p} DHT~\cite{libp2p_github}.
Importantly, the detection and mitigation implementations are fully compatible with the unmodified IPFS clients and can be incrementally deployed in the system. 
%
Therefore, while nodes that have not upgraded may remain vulnerable to the censorship attack, they continue to interoperate with nodes that have upgraded.
%
We evaluate the feasibility of the attack and the efficiency of the countermeasures using simulations as well as actual experiments on the live IPFS network.
Our solution is currently undergoing review by the Protocol Labs engineering team for deployment in the next release of the \texttt{libp2p} DHT.

\para{Related Attacks and Mitigations}
Similar DHT vulnerabilities have been previously discussed in the literature~\cite{dabek2001wide} and multiple prevention mechanisms have been proposed~\cite{dabek2001wide, danezis2009sybilinfer, dan2012centralized, danezis2005sybil}. However, mostly due to practical reasons~\cite{dabek2001wide, dan2012centralized, baumgart2007s} or unrealistic assumptions~\cite{danezis2009sybilinfer, danezis2005sybil, prunster2018holistic}, these mechanisms cannot be deployed in modern decentralized systems. We provide a detailed discussion on this topic in \Cref{sec:related}. As a result, multiple top-tier systems currently rely on a vulnerable DHT for various purposes.
The peer discovery mechanism for several blockchains (\eg, Ethereum ~\cite{eth19discovery}, Celestia~\cite{celestia}, or Polkadot~\cite{burdges2020overview}) uses the same \texttt{libp2p} DHT implementation as IPFS. File sharing in I2P~\cite{timpanaro2015evaluation} and data dissemination in Dat~\cite{dat23dat} also use the Kademlia DHT, although with a different implementation.
%
%
%
Our contributions (both the attack, its detection, and mitigation mechanisms) are expected to apply to these systems as well and more broadly to systems using Kademlia or a Kademlia-like DHT.
%
%
%
%

\para{Outline}
In \Cref{sec:ethics}, we discuss ethical considerations for our study. \Cref{sec:background} presents background on IPFS and its content resolution mechanisms. \Cref{sec:attack}, \Cref{sec:detection}, and \Cref{sec:mitigation} respectively introduce the attack, its detection, and mitigation techniques. 
In \Cref{sec:evaluation} we evaluate all these mechanisms experimentally and we discuss related work in \Cref{sec:related}. \Cref{sec:discussion} provides a discussion on the implication of the attack and its countermeasures, while \Cref{sec:conclusion} concludes the paper. 



%
%
%