%
%
%
%
\documentclass[conference,10pt]{IEEEtran}
%


\usepackage{times}
\usepackage[utf8]{inputenc}
\usepackage{url}
\usepackage{amssymb}
\usepackage{amsmath}
\usepackage{array,multirow}
\usepackage{xspace}
\usepackage{xcolor}
\usepackage{color, colortbl} 
\usepackage{balance}
\usepackage{paralist}
\usepackage[
  colorlinks = true,
  linkcolor = blue,
  urlcolor  = blue,
  citecolor = blue,
  anchorcolor = blue]{hyperref}
\usepackage{cleveref}
\usepackage{graphicx}
\usepackage{pifont}
\usepackage{array}
\usepackage{booktabs}
\usepackage{tikz}
\usepackage{bm}
%
\setlength {\marginparwidth}{2cm}
\usepackage{todonotes}
\usepackage{arydshln}
\usepackage{makecell}
\usepackage{tabularx}
%
\usepackage{caption}
\usepackage{soul,color}
\usepackage{mwe}
\usepackage{ifthen}
\usepackage{algorithm}
\usepackage[noend]{algpseudocode}
\usepackage{epsfig}
\usepackage[tight,footnotesize]{subfigure}
\usepackage{algorithm}%
\usepackage{algpseudocode}%
\usepackage{caption}
\usepackage{cite}
\usepackage{listings}
%

\usepackage{amsthm}
\newtheorem{theorem}{Theorem}
\usepackage{bbm}

%
\usepackage{noto-emoji-easy}

%
\usetikzlibrary{decorations.pathmorphing, decorations.pathreplacing, decorations.shapes}
\usetikzlibrary{calc}
\usetikzlibrary{shapes.callouts}

%
\begin{document}
%

%
%

%
%
\title{Content Censorship in the InterPlanetary File System\vspace{-0.25in}}

%
%
%

\iffalse

\author{\IEEEauthorblockN{Srivatsan Sridhar\IEEEauthorrefmark{1}, Onur Ascigil\IEEEauthorrefmark{2}, Navin Keizer\IEEEauthorrefmark{5},Michał Król\IEEEauthorrefmark{3} \\François Genon\IEEEauthorrefmark{3}, Sébastien Pierre\IEEEauthorrefmark{4}, Yiannis Psaras\IEEEauthorrefmark{6}, Etienne Rivière\IEEEauthorrefmark{4}, Michał Król\IEEEauthorrefmark{3}}

 \vspace{+0.06in}
\IEEEauthorblockA{\IEEEauthorrefmark{1}Stanford University,
}
\IEEEauthorblockA{\IEEEauthorrefmark{2}Lancaster University \\
}
\IEEEauthorblockA{\IEEEauthorrefmark{3}City, University of London,
}
\IEEEauthorblockA{\IEEEauthorrefmark{4}ICTEAM, UCLouvain, Belgium \\
}
\IEEEauthorblockA{\IEEEauthorrefmark{5}University College London
}
\IEEEauthorblockA{\IEEEauthorrefmark{6}Protocol Labs \\
}
\vspace{-0.2in}

}
\fi 

\author{
  \IEEEauthorblockN{Srivatsan Sridhar\IEEEauthorrefmark{1}, Onur Ascigil\IEEEauthorrefmark{2}, Navin Keizer\IEEEauthorrefmark{5}, François Genon\IEEEauthorrefmark{3}\\ Sébastien Pierre\IEEEauthorrefmark{3}, Yiannis Psaras\IEEEauthorrefmark{6}, Etienne Rivière\IEEEauthorrefmark{3}, Michał Król\IEEEauthorrefmark{4}}
  
  \vspace{0.1in}
  
  \IEEEauthorblockA{
    \begin{tabular}{ccc}
      \IEEEauthorrefmark{1}Stanford University & \IEEEauthorrefmark{2}Lancaster University & \IEEEauthorrefmark{5}University College London \\
      \IEEEauthorrefmark{3}ICTEAM, UCLouvain, Belgium & \IEEEauthorrefmark{4}City, University of London & \IEEEauthorrefmark{6}Protocol Labs \\
    \end{tabular}
  }

    \vspace{0.04in}
}


\maketitle

\newcommand\calF{\mathcal{F}}
\newcommand\calG{\mathcal{G}}
\newcommand\calM{\mathcal{M}}
\newcommand\calV{\mathcal{V}}
\newcommand\calU{\mathcal{U}}
\newcommand\calW{\mathcal{W}}
\newcommand\calP{\mathcal{P}}
\newcommand\calD{\mathbb{D}}
%%%%%%%%%%%%%%%%%
%% macros introduced by Luke 
\newcommand\mydef[1]{{\bf\em #1}}
%%%%%%%%%%%%%%%%%

\newcommand{\numviparams}{{| \lambda |}}
\newcommand{\scoreaccvars}[1]{s_1^{#1}, \ldots, s_{\numviparams}^{#1}}
\newcommand{\scoreaccvar}[2]{s_{#1}^{#2}}
\newcommand{\isdeterm}[1]{\text{Deterministic}({#1})}


\newcommand{\expect}[1]{\mathbb{E}\left[{#1}\right]}
\newcommand{\var}[1]{\mathbb{V}\left[ {#1} \right]}
\newcommand{\expectdist}[2]{\mathbb{E}_{#1}\left[ {#2} \right]}
\newcommand{\vardist}[2]{\mathbb{V}_{#1}\left[ {#2} \right]}
\newcommand{\cov}[2]{\mathbb{C}\text{ov}[{#1}][{#2}]}
\newcommand{\covv}[1]{\mathbb{C}\text{ov}[{#1}]}
\newcommand{\corr}[1]{\mathbb{C}\text{orr}[{#1}]}

\newcommand{\fix}[1]{\mathit{fix}\left({#1}\right)}
\newcommand{\sbr}[1]{\left\llbracket {#1} \right\rrbracket}
\newcommand{\ctxtype}[3]{{#1} \cong_\text{ctx} {#2} : {#3}}
\newcommand{\bigstep}[3]{{#1} \Downarrow_{#2} {#3}}


% PCF types
\newcommand{\bool}{\mathit{bool}}
\newcommand{\nat}{\mathit{nat}}

\newcommand{\ctx}[1]{\mathcal{C}\left[ {#1}\right] }
\newcommand{\pcft}[1]{\text{PCF}_{#1}}

\newcommand{\nfl}{\mathbb{N}_\bot}
\newcommand{\bfl}{\mathbb{B}_\bot}

% PCF constructs
\newcommand{\succc}[1]{\mathbf{succ}({#1})}
\newcommand{\succcn}[2]{\mathbf{succ}^{#1}({#2})}
\newcommand{\zero}{\mathbf{0}}
\newcommand{\zerotest}[1]{\mathbf{zero}\left({#1}\right)}
\newcommand{\pred}[1]{\mathbf{pred}\left( {#1} \right)}
\newcommand{\predn}[2]{\mathbf{pred}^{#1}\left( {#2} \right)}
\def\solvable{\#}

\newcommand{\true}{\mathbf{true}}
\newcommand{\false}{\mathbf{false}}
\newcommand{\pcffix}[1]{\mathbf{fix}\left({#1}\right)}
\newcommand{\pcffn}[3]{\mathbf{fn}~{#1}:{#2}\mathpunct{.}{#3}}
\newcommand{\pairtype}[2]{{#1} * {#2}}
\newcommand{\pairexp}[2]{\mathbf{pair}({#1}, {#2})}
\newcommand{\leftexp}[1]{\mathbf{left}({#1})}
\newcommand{\rightexp}[1]{\mathbf{right}({#1})}

\newcommand{\RationalPos}{\mathbb{Q}^{+}}

\newcommand{\meas}[1]{\mathbb{M}\left( {#1} \right) }
\newcommand{\integ}[1]{\sbr{#1}_I}

\newcommand{\notbigstep}[2]{{#1}~\cancel{\Downarrow}_{#2}}
\newcommand{\subtrace}[3]{{#1}^{{#2} \ldots {#3}}}
\newcommand{\supp}[1]{\textsf{supp}\left({#1}\right)}
\newcommand{\dom}[1]{\textsf{Dom}\left({#1}\right)}
\newcommand{\suppk}[2]{\textsf{Supp}^{#1}\left({#2}\right)}
\newcommand{\tracespace}{\bigcup_{n \in \mathbb{N}}[0, 1]^n}
\newcommand{\generictracespace}{\mathbb{T}}
\newcommand{\nnreals}{\mathbb{R}_{\geq 0}}
\newcommand{\posreals}{\mathbb{R}_{> 0}}
\newcommand{\reals}{\mathbb{R}}

\newcommand{\unrollkM}[2]{\textsf{unroll}_{#1}\left({#2}\right)}
\newcommand{\nphmcint}[5]{\Psi_\textsf{NP}\left({#1}, {#2}, {#3}, {#4}, {#5}\right)}

%SPCF constructs
\newcommand{\spcfvalues}{\Lambda^0_v}

\newcommand{\prevalueM}[1]{\textsf{value}^{-1}_{#1}(\spcfvalues{})}
\newcommand{\num}[1]{\underline{#1}}

% \theoremstyle{definition}
% \newtheorem{thm}{Theorem}
% \newtheorem{lem}{Lemma}
% \newtheorem{defn}{Definition}
% \newtheorem{conj}{Conjecture}
% \newtheorem{prop}{Proposition}

%\theoremstyle{definition}
%\newtheorem{defn}{Definition}[section]
%\newtheorem{example}[defn]{Example}
%
%
%\theoremstyle{plain}
%\newtheorem{thm}{Theorem}[section]
%\newtheorem{lem}[thm]{Lemma}
%\newtheorem{cor}[thm]{Corollary}
%\newtheorem{conj}[thm]{Conjecture}
%\newtheorem{prop}[thm]{Proposition}
%\newtheorem{remark}[thm]{Remark}

%% Proofs
%\let\oldproof\proof
%\renewcommand{\proof}{\color{blue}\oldproof}


\definecolor{codegreen}{rgb}{0,0.6,0}
\definecolor{codegray}{rgb}{0.5,0.5,0.5}
\definecolor{codepurple}{rgb}{0.58,0,0.82}
\definecolor{backcolour}{rgb}{0.95,0.95,0.92}

\lstdefinestyle{myStyle}{
    belowcaptionskip=1\baselineskip,
    breaklines=true,
    frame=none,
    basicstyle=\footnotesize\ttfamily,
    keywordstyle=\bfseries\color{green!40!black},
    commentstyle=\itshape\color{purple!40!black},
    identifierstyle=\color{blue},
    backgroundcolor=\color{gray!10!white},
    %backgroundcolor=\color{backcolour}, 
    numberstyle=\tiny\color{codegray},
    stringstyle=\color{codepurple},
    breakatwhitespace=false,                          
    keepspaces=true,                 
    numbers=left,       
    numbersep=5pt,                  
    showspaces=false,                
    showstringspaces=false,
    showtabs=false,                  
    tabsize=2,
}

% argmin/argmax
\DeclareMathOperator*{\argmax}{arg\,max}
\DeclareMathOperator*{\argmin}{arg\,min}

% Concatenation of lists
\newcommand\doubleplus{+\kern-1.3ex+\kern0.8ex}

% Program configurations
\newcommand{\tuple}[1]{\ensuremath{\langle #1 \rangle}}
% Rule based definitions
\newcommand{\Rule}[4][]{\ensuremath{\inferrule*[lab={\hypertarget{#2}{(\TirName{#2})}},#1]{#3}{#4}}}

% Calligraphic symbols
\newcommand{\calI}{{\mathcal I}} 
\newcommand{\calT}{{\mathcal T}}

%  Macro for new Y operator.
\newcommand{\yBounded}[3]{\mu^{#1}_{#2}\rvert_{#3}}

%%%%%%%%%%%%%%%%%
 
%%%%%%%%%%%%%%%%%

\newcommand{\expv}{\mathbb{E}}

\newcommand{\combTr}[2]{\left[\begin{matrix}
		#1\\
		#2
	\end{matrix} \right]}

\newcommand{\exType}[2]{\left\{\begin{matrix}
		#1\\
		#2
	\end{matrix} \right\}}
\newcommand{\myint}[1]{ [#1]}
\newcommand{\Uniform}{\ensuremath{\mathrm{Uniform}}}
\newcommand{\Normal}{\ensuremath{\mathrm{normal}}}
\DeclareMathOperator{\abs}{abs}
\DeclareMathOperator{\pdf}{pdf}

\newcommand{\intConf}[1]{\lceil#1\rceil}
\newcommand{\tr}{\boldsymbol{t}}

\newcommand{\sample}{\tt{sample}}
%\newcommand{\fix}{\texttt{fix}}
%\newcommand{\num}[1]{\underline{#1}}
\newcommand{\myif}{\texttt{if}}
\newcommand{\mylet}{\texttt{let} \, }
\newcommand{\myin}{\, \texttt{in} \,}
\newcommand{\mythen}{\, \texttt{then} \,}
\newcommand{\myelse}{\, \texttt{else} \,}
\newcommand{\score}{\tt{score}}
\newcommand{\tick}{\tt{tick}}

\newcommand{\term}{\tt{term}}
\newcommand{\pv}{\mathbf{v}}
\newcommand{\rv}{\mathbf{r}}

\newcommand{\interval}{\mathfrak{I}}

\newcommand{\typeReal}{\textbf{\textsf{R}}}

\newcommand{\symbolInt}{\myint{\cdot}}

\newcommand{\LambdaInterval}{\Lambda_{\interval}}
\newcommand{\LambdaSymbolic}{\Lambda_{\text{sym}}}

\newcommand{\toIntervalTerm}[1]{#1^{2\interval}}

%Others
\newcommand{\Sset}{\mathbb{S}}
\newcommand{\Iset}{\mathbb{I}}
\newcommand{\Rset}{\mathbb{R}}
\newcommand{\Nset}{\mathbb{N}}
\newcommand{\Zset}{\mathbb{Z}}

\newcommand{\Term}{\mathbb{T}}
\newcommand{\prob}{\mathbb{P}}
\newcommand{\expt}{\mathbb{E}}


\newcommand{\Leb}{\tt{Leb}}
\newcommand{\Red}{\tt{Red}}
\newcommand{\cost}{\text{cost}}

%\newcommand{\intervalab}[2]{\underline{[#1,#2]}}
\newcommand{\intervalab}{\underline{[a,b]}}
\newcommand{\interI}{\mathcal{I}}
\newcommand{\trans}{\mathcal{T}}

\newcommand{\iv}{\mathbb{I}}

% Programming language constructs
\newcommand{\lit}[1]{\underline{#1}}
\newcommand{\letIn}[1]{\mathsf{let}\,{#1}\,\mathsf{in}\,}
\newcommand{\fixLam}[2]{\mu {#1} {#2}.}
\newcommand{\ifElse}[3]{\mathsf{if} (#1 \le \num{0}) \, {#2} \,\mathsf{else}\, {#3}}

%%Basic notions
\newcommand{\pspace}{(\Omega,\mathcal{F},\probm)}
\newcommand{\probm}{\mathbb{P}}
\newcommand{\condexpv}[2]{{\expt}{\left[{#1} \mid {#2}\right]}}

\newcommand{\stdConf}[1]{(#1)}
%\newcommand{\intConf}[1]{\lceil#1\rceil}
%\newcommand{\intConf}[1]{(#1)}
%\newcommand{\symConf}[1]{\langle\!\langle  #1 \rangle\!\rangle}
%\newcommand\symPath[1]{(#1)}
\newcommand{\symPath}[1]{\langle\!\langle  #1 \rangle\!\rangle}
\newcommand\symConf[1]{(#1)}

\newcommand{\ifSimple}[3]{\mathsf{if}(#1, #2, #3)}
%\newcommand{\ifElse}[3]{\mathsf{if} (#1 \le 0) \, \allowbreak {#2} \, \allowbreak \mathsf{else}\, {#3}}
%\newcommand{\ifElse}[3]{\ifSimple{#1}{#2}{#3}}

%\newcommand{\trace}{\mathsf{s}}
%
%\newcommand\defn[1]{{\bf \em #1}}
\newcommand{\traces}{\mathbb{T}}
%
%\newcommand{\stdConf}[1]{(#1)}
%%\newcommand{\intConf}[1]{\lceil#1\rceil}
%\newcommand{\intConf}[1]{(#1)}
%%\newcommand{\symConf}[1]{\langle\!\langle  #1 \rangle\!\rangle}
%%\newcommand\symPath[1]{(#1)}
%\newcommand{\symPath}[1]{\langle\!\langle  #1 \rangle\!\rangle}
%\newcommand\symConf[1]{(#1)}

\newcommand{\valueSem}[1]{\mathsf{val}_{#1}} % value (semantics)
\newcommand{\weightSem}[1]{\mathsf{wt}_{#1}} % weight (semantics)
\newcommand{\measureSem}[1]{\llbracket #1 \rrbracket}
\newcommand{\posterior}{\mathsf{posterior}}


%%%%%%%%%
% 
%%%%%%%%
\newcommand{\loc}{\ell}
\newcommand{\locs}{\mathit{L}}
\newcommand{\blocs}{\mathit{L}_{\mathrm{b}}}

\newcommand{\iflocs}{\mathit{L}_{\mathrm{if}}}
\newcommand{\looplocs}{\mathit{L}_{\mathrm{while}}}

\newcommand{\alocs}{\mathit{L}_{\mathrm{a}}}
\newcommand{\wlocs}{\mathit{L}_{\mathrm{w}}}
\newcommand{\rlocs}{\mathit{L}_{\mathrm{r}}}
\newcommand{\Alocs}[1]{\mathit{L}_{\mathrm{A}}^{\mathsf{#1}}}
\newcommand{\Dlocs}{\mathit{L}_{\mathrm{nd}}}
\newcommand{\transitions}{{\rightarrow}}

%%% 
\newcommand{\plocs}{\mathit{L}_{\mathrm{p}}}
\newcommand{\tlocs}{\mathit{L}_{\mathrm{t}}}

\newcommand{\lin}{\loc_\mathrm{init}}
\newcommand{\lout}{\loc_\mathrm{out}}
\newcommand{\val}[1]{\mbox{\sl Val}_{#1}}

\newcommand{\pvars}{V_\mathrm{p}}
\newcommand{\rvars}{V_{\mathrm{r}}}
\newcommand{\pre}{\mathrm{pre}}

\newcommand{\sle}{\sqsubseteq}
\newcommand{\sge}{\sqsupseteq}

\newcommand{\lfp}{\mathrm{lfp}}
\newcommand{\gfp}{\mathrm{gfp}}

\newcommand{\rdvarjdis}{\mathcal D}
\newcommand{\sampset}{\textit{supp}}

\newcommand{\upd}{\mbox{\sl upd}}
\newcommand{\wet}{\mbox{\sl wt}}
\newcommand{\transset}{\mathfrak T}
\newcommand{\valin}{\pv_{\mathrm{init}}}
\newcommand{\ret}{\mbox{\sl ret}}

\newcommand{\win}{w_{\mathrm{init}}}

\newcommand{\sampdpd}{\overline{\Upsilon}}

\newcommand{\outmap}{\text{O}}
\newcommand{\sat}[1]{\langle #1 \rangle}
\newcommand{\monoid}{\mbox{\sl Monoid}}
\newcommand{\handelmanformat}{(\dagger)}

\newcommand{\trunc}{\mathcal{B}}

\newcommand{\ewt}{\mbox{\sl ewt}}
\newcommand{\statemap}{\text{St}}

\newcommand{\valrd}{{\mathbf{r}}}
\newcommand{\frmloc}{\ell^{\mathrm{src}}}
\newcommand{\toloc}{\ell^{\mathrm{dst}}}

\newcommand{\monomials}{\mathbf{M}}
%
\begin{abstract}
  The InterPlanetary File System (IPFS) is currently the largest decentralized storage solution in operation, with thousands of active participants and millions of daily content transfers. IPFS is used as remote data storage for numerous blockchain-based smart contracts, Non-Fungible Tokens (NFT), and decentralized applications.

  We present a content censorship attack that can be executed with minimal effort and cost, and that prevents the retrieval of any chosen content in the IPFS network. 
  The attack exploits a conceptual issue in a core component of IPFS, the Kademlia Distributed Hash Table (DHT), which is used to resolve content IDs to peer addresses.
  We provide efficient detection and mitigation mechanisms for this vulnerability. 
  Our mechanisms achieve a 99.6\% detection rate and mitigate 100\% of the detected attacks with minimal signaling and computational overhead.
  We followed responsible disclosure procedures, 
  and our countermeasures are scheduled for deployment in the future versions of IPFS.
  %
  
%
%
%
%
%
%
%
%
%
%
%
%
%
%
%
%
%
%
\end{abstract}

% Figure environment removed

\section{Introduction}
Automatic 3D reconstruction of clothed humans using image inputs has gained increasing significance due to its potential applications in a wide array of AR/VR scenarios. High-fidelity reconstructions typically depend on sophisticated capture systems, which are developed with dense camera arrays~\cite{collet2015high,joo2015panoptic,joo2018total}, programmable light-stages~\cite{Vlasic2009, guo2019relightables}, and depth sensors~\cite{newcombe2011kinectfusion,DoubleFusion,BodyFusion,dou2016fusion4d,newcombe2015dynamicfusion}. However, stringent capture environments equipped with complex hardware pose significant challenges for consumer-level applications.


In this context, considerable research effort has been dedicated to developing methods that allow for more flexible capture configurations, such as utilizing a few RGB inputs. Among these works, learning implicit functions \cite{iccv2020PIFu, saito2020pifuhd, hong2021stereopifu} has proven effective in achieving highly detailed reconstructions by integrating the advancements of deep neural networks. These methods employ large multi-layer perceptrons (MLPs) to predict the occupancy probability or truncated signed distance function (TSDF) value of every queried 3D point based on its associated local feature, which is extracted from images. They can recover a continuous surface at arbitrary resolutions without topology restrictions.


However, in typical MLP-based implicit networks, the occupancy or TSDF value at each location is solved independently with planar image features, rendering them less capable of addressing challenging cases such as occlusions. Consequently, these methods suffer from generalization and robustness issues, particularly when tackling strong occlusions caused by large motion or multiple interacting humans. 
Some follow-up studies  \cite{zheng2021deepmulticap,zheng2021pamir,huang2020arch} utilize an extra geometric model, SMPL~\cite{Loper2015}, to improve robustness by introducing strong shape priors. 
Their success typically relies on the assumption of geometrical similarity \cite{huang2020arch} between the shape prior and target reconstruction, making them intractable for handling complex cases with loose clothes and sensitive to errors in SMPL model fitting.



%\ping{this paragraph sounds like `TSDF is better than MLP/SMPL, and we use TSDF to solve the problem'. But in Sec 3, we are telling a different story, saying `MLP needs a 3D convolutional encoder'. We need to make these two sections consistent.}\sicong{I think in this paragraph we claim that the TSDF}


%We opt for Trucated Signed Distance Funtion (TSDF) volumetric representations as they are naturally suitable for convolution operations, which have shown remarkable performance for learning hierarchical features on 2D visual perception tasks \cite{SunXLW19}. 
%Meanwhile, TSDF also describes the gradual geometry change around shape surface, which is not reflected by occupancy volume. 

We instead revisit the 3D volumetric representation and resort to 3D convolutional neural networks (CNNs) for feature learning, due to their impressive performance in feature learning and the ability to incorporate spatial context. However, volumetric methods and 3D convolution involve discretization, which might raise concerns regarding whether a discretized volume can preserve subtle geometric details as continuous representations learned in implicit functions. We investigate the relationship between volume resolution and quantization error on synthetic data by converting target mesh objects to TSDF volumes, as shown in Figure~\ref{fig:quantization_error}. We observe that the quantization errors are significantly reduced by increasing volume resolution and become nearly negligible when reaching a relatively high resolution (e.g., 512 or higher). In other words, achieving fine-detailed reconstruction is not supposed to be restricted by the use of volume representations as long as a proper volume resolution is utilized. Therefore, we present a method with high-resolution feature volumes, e.g., 256 and 512, while traditional volumetric methods \cite{varol18_bodynet,gilbert2018volumetric} are often limited to much lower resolutions, such as 32 or 128.



On the other hand, an increase in volume resolution may lead to a cubic growth of memory overhead \cite{8100085}. Reducing memory costs while guaranteeing the granularity of volumetric representations is necessary for pursuing high-quality reconstruction. Thus, we adopt a coarse-to-fine approach and cull away irrelevant voxels to build a sparse high-resolution feature volume. At the coarse level, the network computes an initial TSDF by applying a U-Net with sparse 3D CNN \cite{3DSemanticSegmentationWithSubmanifoldSparseConvNet} on the sparse feature volume, which is carved by a visual hull. Through our experiments, it turns out that more than 95\% of the volume grids are discarded by the visual hull culling, making the sparse 3D CNN efficient. At the fine level, the network focuses on a narrow band near the zero-level set of the initial TSDF and discretizes the narrow band with smaller voxels. By employing this narrow-band culling, we further shrink the sampling space, resulting in a relatively small range of grid numbers (usually 300K--500K in our experiments) even with a high volume resolution of 512. The remaining voxels in the narrow band are associated with features that fuse high-frequency information from the computed normal maps upon the low-frequency shape from the coarse level to compute the TSDF at high resolution. The final mesh is then extracted from the TSDF using the Marching-Cube algorithm ~\cite{Lorensen87marchingcubes}.
% Different from the u-net sturcture to preserve global topology context, we then apply a shallow 3dcnn to compute the final TSDF $D_{final}$ which contain more local geometry detail.




% \ping{this paragraph can be expanded. It is an important contribution and often ignored by other works. stress on the novel idea of regressing blending weights instead of colors}

In addition to geometry, high-quality mesh texture is also a crucial factor contributing to visual appearance. Directly computing a color field in 3D space, as in \cite{iccv2020PIFu}, struggles to capture high-frequency texture details, while the neural radiance field (NeRF) \cite{yu2020pixelnerf} or the DoubleField~\cite{shao2022doublefield} require expensive per-instance optimization and are often unstable for sparse input images. In contrast, we adopt an image-based rendering approach to compute a texture atlas map, which is efficient and widely supported in existing computer graphics tools. 
Specifically, we compute a blending weight at each 3D point on the mesh surface to determine its color as a weighted average of the colors at its image projections. The blending weights can be computed at a relatively coarse resolution, e.g., 512 volume resolution in our case, and leave texture details to the high-resolution images, such as 1K or 2K. Unlike previous methods that generate blurry texturing results under sparse input, our method generalizes well on both synthetic and real data with just a few input views. 
Figure~\ref{fig:teaser} shows two examples reconstructed by our method. Despite the challenging garment, pose, and occlusion, our method recovers faithful shape, normal, and texture on the right.

%with a wide variety of poses and clothing styles, and it is also adaptive to handle input image with arbitrary resolutions.
%\sicong{For this concern we claim that when the resolution of dicretized volume meets certain threshold (which is 256 in our experiment), the quantization error can be neglected.} 



In summary, the main contributions of this paper are as follows:
\begin{itemize}
\vspace{-0.1in}
  \item 
  We revisit the 3D volumetric representation and demonstrate that it can support clothed human reconstruction with equal or even better performance compared to implicit representation. 
  \item 
  We develop a memory and computation-efficient method for high-resolution volumetric reconstruction using sophisticated sparse 3D CNN, coarse-to-fine estimation, and voxel culling by visual hull and narrow bands. 
  \item 
  We introduce a novel method to compute a texture atlas map, which captures rich appearance details from high-resolution input images.
  \item 
  We achieve impressive results on standard benchmark datasets Twindom and MultiHuman, significantly reducing the point-2-surface (P2S) precision to approximately 0.2cm from just six input views, with more than $50\%$ error reduction compared to the state-of-the-art methods, including DoubleField~\cite{shao2022doublefield} and PIFuHD~\cite{saito2020pifuhd}.
\end{itemize}

In this Section we will conduct the NeurIPS Ethics Review, a critical and an important step in helping ensure beneficial research outcomes. 
The research process of this paper did not cause any direct harms to person since our work is theoretical and mathematical, so our ethical analysis focuses on \emph{Societal Impact and Potential Harmful Consequences}. 

The areas of concern on the NeurIPS guidelines for social impact are  safety, security, discrimination, surveillance, deception and harassment, environmental, humans rights, and bias and fairness.

Of this list, \emph{discrimination, and bias and fairness} are the most relevant to our work. 
We will consider these notions together.

This section of the Ethics Review directs us to consider `known or anticipated consequences of research'.
The high level idea in our paper is studying and improving the robustness of fairness constrained learning in the presence of malicious noise.
Our work is a step towards models that are robust to malicious noise which would ideally result in fair \emph{and} accurate learning systems. 
This paper also exposes when models are inevitably fragile and thus direct actors towards a different solution concept or system.
These two threads would ideally have a positive benefit in mitigating discrimination and bias.

However, there exists a certain constant risk that mathematical models of bias (and models learned on data) are mismatched with
the underlying empirical reality and that when deployed supposedly unambiguously beneficial interventions can be harmful.
The malicious noise model is intentionally a worst case model with quite capacious assumptions, so we believe our results are quite general and would not fall in this trap; but these concerns are worth keeping in mind when considering a new deployment.
In particular, the field of fairness constrained learning needs more experimental verification.

Despite these risks, we thoroughly believe that it is critical to study fairness in learning since different demographic groups exist in the real world  and machine learning can affect those groups differently.  
Theoretical work like ours is critical to exposing fundamental trade-offs and 
baselines that can help craft effective fairness interventions.

We are confident our work passes high ethical standards and is a net ethical benefit to society at large. 

\vspacebeforesection
\section{Background}
\label{sec:background}

In this section, we provide the necessary background information to ensure a comprehensive understanding of the attack described in this paper. We start with a description of the Distributed Hash Table (DHT) used by IPFS, followed by its content resolution mechanisms. We also detail techniques for network size estimation, necessary for our attack detection and mitigation mechanisms.

\vspacebeforesection
\subsection{IPFS DHT}
\label{sec:kad_dht}

We review the features of the Kademlia DHT~\cite{maymounkov2002kademlia} and its \texttt{libp2p} implementation~\cite{libp2p_github} that are the most relevant to our attack.
To participate in the DHT, each peer generates a public/private key pair and derives an identity $\peerid \in \{0,1\}^{256}$ as the hash of its public key.
Ideally, each peer generates a random key pair and, therefore, peer IDs are distributed uniformly and independently over the space $\{0,1\}^{256}$.
While honest nodes follow this rule, malicious nodes may generate and choose from an arbitrary number of key pairs.
Each peer maintains a routing table consisting of $m=256$ buckets.
The $i$-th bucket contains the addresses of up to $k=20$ peers whose peer IDs share a common prefix of exactly $i$ bits with the peer's own peer ID. 

%
A new participant node joins the IPFS network by contacting one of the hardcoded bootstrap nodes. This bootstrap node provides the new node with some initial peers allowing it to join the DHT. The new node uses this information to perform a walk through the DHT towards its own peer ID.
The walk allows to: \textit{(i)}~make sure that there is no other node in the network with the same ID; \textit{(ii)}~discover new peers and fill the newcomer's DHT routing table. At the same time, the newcomer establishes \bitswap~\cite{de2021accelerating} connections to a subset of encountered peers (usually around 300 of them). The core role of the \bitswap protocol is to enable bilateral content transfer and to play the role of a cache for recently-accessed content.

The main DHT operation $\Call{GetClosestPeers}{\key}$ returns the $k=20$ closest peers to $\key$. 
%
In Kademlia, the distance between two keys $x$ and $y$ in the key space is given by $x \oplus y \in \{0,...,2^{256}-1\}$, where $\oplus$ denotes the bitwise XOR operation on the keys; the resulting binary string is interpreted as an integer.
%
When a client wants to find the peers with IDs closest to $\key$, it sends a request to the $\alpha=3$ peers in its routing table whose peer IDs are closest to $\key$. Each of these peers returns the $k$ closest peers to $\key$ in its own routing table and the addresses of these peers. 
%
The client again sends a request to the $\alpha$ peers closest to $\key$, among peers in its routing table and those whose addresses it just received. This process repeats until the client does not find any more peers closer to $\key$.
Due to network churn and imperfect routing tables, we observed in our experiments that successive calls to $\Call{GetClosestPeers}{\key}$ do not always return the same set of $k=20$ peers (we provide more details in \Cref{sec:evaluation}, \Cref{fig:20closest}). This is an important limitation affecting our attack.

\vspacebeforesection
\subsection{Content Resolution in IPFS}
\label{sec:ipfs}

IPFS is a content-centric network.
It allows its participant to request files without specifying their location. 
%
Content is indexed by content IDs $\cid \in \{0,1\}^{256}$ that are derived from a hash of that content.
Both peer IDs and CIDs are used as keys in the DHT.
Each node can play the role of a \provider, \downloader, or \resolver. 
The process of content advertisement and resolution is illustrated in \Cref{fig:add_get_provider}.

%
When a \provider wishes to publish content with a given $\cid$ on IPFS, it creates a \emph{provider record} that contains $cid$ and the \provider's address.
During a $\Call{Provide}{\cid}$ operation, the \provider first uses $\Call{GetClosestPeers}{\cid}$ to locate the $k=20$ peers with their peer IDs closest to $\cid$, 
%
and then sends them a $\mathsf{PutProvider}$ message including the provider record (\Cref{fig:add_get_provider}(a)).
We call the peers that hold provider records for $\cid$ the \emph{resolvers} for $\cid$.

Each CID can have several \providers. In fact, by default, each IPFS client becomes a provider for each piece of content it downloads for a fixed amount of time (12h, 24h, or 48h depending on the client version or custom configuration). As a result, the system provides an auto-scaling feature with supply automatically rising with demand.

%
When a \downloader wishes to fetch a piece of content, it first sends a request to all its \bitswap peers. If none of them has the content, the \downloader uses the DHT-based resolution system. We stress that the \bitswap protocol plays the supporting role of a cache in the dissemination of popular files. However, the mechanism does not provide reliable content resolution, in particular for new or less popular content. %

When \bitswap unstructured search fails, the \downloader resolves $\cid$ using $\Call{FindProviders}{\cid}$. This operation uses a DHT walk identical to that of $\Call{GetClosestPeers}{\cid}$ to find $k$ \resolvers but also queries encountered nodes for a provider record for $\cid$ (\Cref{fig:add_get_provider}(b)). The process terminates when either 20 \providers have been found, or all \resolvers have been asked. Querying all encountered nodes (\ie, not only the designated \resolvers) is useful because some of the encountered nodes may have a provider record in their cache.
%

Upon receiving a provider record, the client connects to the address specified in the provider record to retrieve the actual content (\Cref{fig:add_get_provider}(c)).
Provider records are not authenticated, and therefore malicious \providers may respond with incorrect provider records (or may not respond at all). However, the integrity of the content is preserved because the hash of the retrieved content can be verified against its $\cid$.
%


%

\input{img/add_get_provider.tex}

\vspacebeforesection
\subsection{Network Size Estimator}
\label{sec:netsize}

The number of nodes in a decentralized system is generally unknown due to the avoidance of centralized membership management.
This number is nonetheless useful for optimizations, deciding on individual node configurations, or security mechanisms.
Various methods were proposed for the decentralized estimation of unstructured and structured networks~\cite{eli-sohl-dht-size-estimation,kostoulas2005decentralized, manku2003symphony}.
We use in this work a mechanism developed initially by Protocol Labs as part of a mechanism for decreasing the latency of publishing content in IPFS~\cite{network-size-estimation-notion,network-size-estimation-github-pr}.

%
%
%
%
%
%
%
%
%
%

Each node in the DHT refreshes its routing table periodically (every $10$ minutes in \texttt{libp2p}). 
For this, the node samples $m$ random keys (one for each bucket of its routing table)
%
and queries the DHT to obtain the $k=20$ closest peer IDs to each key.
Using these, the node then computes the average distance between each one of these keys $\key_j$ for $j=1,\dots,m$ and their $i$-th closest peer ID for $i=1,...,k$ (with $m=256$ and $k=20$).
\begin{equation}
    \label{equ:avg-dist}
    \overline{D}_i = \frac{1}{m} \sum_{j=1}^m \operatorname{dist}(\key_j, \peerid_{j}^{(i)})
\end{equation}
where $\peerid_{j}^{(i)}$ is the $i$-th closest peer ID to $\key_j$.
With $N$ peers in the DHT and peer IDs uniformly distributed in the hash space, the expected distance between a $\key$ and its $i$-th closest peer ID is $\frac{2^{256}i}{N+1}$. The node then runs a least square regression to compute the value of $N$ for which the expected distances best fit the empirical average distances, \ie,
\begin{equation}
    \label{equ:netsize-least-squares}
    \hat{N} = \arg\min_{N} \sum_{i=1}^k \left(\overline{D}_i - \frac{2^{256}i}{N+1}\right)^2.
\end{equation}
The resulting estimate $\hat{N}$ can be computed in closed form.
%

When a node starts running, it must perform DHT queries for a few random keys to initialize its network size estimate. 
Since a larger number of queries will result in higher accuracy, making more queries than what is needed to initialize one's routing table is recommended.
Thereafter, keeping the estimate up-to-date does not require any excess DHT queries beyond what is already used for refreshing the routing table as this is done frequently (every 10 minutes).

While the network size estimate has a stochastic variance resulting from the probability distribution of the honest peer IDs, it is hard for an attacker to bias the estimate significantly. Since the estimator uses the density of peer IDs around keys chosen uniformly at random, the adversary would require numerous Sybil nodes (on the order of the whole network size) to significantly affect the peer ID density around those keys.


% !TEX program = pdflatex
% !TEX root = main.tex


\section{The Model}

We represent a series of interactions between $N$ individuals as a sequence of weighted directed networks with adjacency matrix $A^t$ for $t=0,1,2,\ldots,T$. For each $t$, its entry $A_{ij}^t$ is the outcome of interactions $i \rightarrow j$ suggesting that $i$ is ranked above $j$. This allows both cardinal and ordinal inputs. For instance, in team sports, $A_{ij}^t$ could be the number of points by which team $i$ beat team $j$, or we could simply set $A_{ij}^t=1$ to indicate that $i$ won and $j$ lost. We can include the case where individuals interact multiple times at time $t$ by summing the corresponding entries.

We assume that the values of $A_{ij}^t$ are influenced by a vector of real-valued ranks $\v{s}^t=(s_{1}^t,\dots, s_{N}^t)$, where $s_i^t$ is $i$'s skill, strength or prestige at time $t$.
To model these interactions, we follow SpringRank's approach of imagining the network as a physical system~\cite{de2018physical}. Specifically, each node $i$ is embedded in $\mathbb{R}$ at position $s_i^t$, and each directed edge $i \rightarrow j$ becomes an oriented spring with a non-zero resting length and displacement $s_i^t-s_j^t$. Since we are free to rescale latent space and the energy scale, we set the spring constant and resting length to $1$. The spring corresponding to an edge $i \rightarrow j$ at time $t$ then has energy
\be\label{eqn:staticH}
H_{ij}(s_i^t,s_j^t)=\f{1}{2} \bup{s_i^t-s_j^t-1}^{2} \, .
\ee
If there were no other effects, the total energy of the system at time $t$ would then be 
\be\label{eqn:totalstaticH}
H^t(\v{s}^t) = \sum_{i,j=1}^{N} A_{ij}^t \,H_{ij}(s_i^t,s_j^t) \, .
\ee
If we determined $\v{s}^t$ by minimizing $H^t$ for each $t$ separately, we would simply be applying the static SpringRank model separately to each ``snapshot'' of the network. This would ignore all previous (and future) interactions, and ignore the hypothesis that ranks change smoothly from one time-step to the next.

% Figure environment removed

To model this smoothness, we also assume a dependence between ranks at successive time-steps. Specifically, we extend the Hamiltonian~\eqref{eqn:totalstaticH} with an extra term that models the \emph{self-interaction} between past and current ranks,
\begin{equation}\label{eqn:selfH}
\Hself^t(\v{s}^t,\v{s}^{t-1}) 
= \frac{\kself}{2} \sum_{i=1}^N (s_i^t-s_i^{t-1})^2 \, .
\end{equation}
This can be seen as a set of additional ``self-springs'' that connect the rank of each individual with its own previous rank. The spring constant $\kself$ parametrizes how smoothly we want the ranks to change from one step to the next. In inference terms, $\kself$ is a hyperparameter which we tune using cross-validation.

Summing over all time-steps $0 < t \le T$ and adding this to the pairwise interactions at each time-step then gives a total energy

\begin{align}\label{eqn:fullH}
\Htotal(\{\v{s}^t\}) = \sum_{t=0}^T H^t(\v{s}^t) + \sum_{t=1}^T \Hself^t(\v{s}^t,\v{s}^{t-1}) \, .
\end{align}
We call this the dynamical SpringRank Hamiltonian. The optimal ranks $\v{s}^0,\v{s}^1,\ldots,\v{s}^T$ are those that minimize it.


There are two ways to minimize $\Htotal$. One is to proceed in an online way, moving forward in time. In this approach, we use the static SpringRank model Eq.~\eqref{eqn:totalstaticH} to find the initial ranks $\v{s}^0$ by minimizing $H^0(\v{s}^0)$. As in Ref.~\cite{de2018physical}, the energy is unchanged if we add a constant to all the ranks; we can break this translational symmetry by setting the mean initial rank $(1/N) \sum_{i=1}^N v_i^0$ to zero.
Then, at each subsequent time-step $t \ge 1$, we update the ranks by taking into account both the new pairwise interactions and the self-springs connecting the ranks with their previous values. Namely, given $\v{s}^{t-1}$ and $A^t$, we find the ranks $\v{s}^t$ that minimize $H^t(\v{s}^t) + \Hself^t(\v{s}^t,\v{s}^{t-1})$.

Since this is a convex function of $\v{s}^t$, we can find its minimum by setting its gradient to zero, or equivalently by balancing all the forces $v_i^t$. This yields a system of linear equations:
\begin{align}\label{eqn:fullsolution}
\rup{ D^{out,t}+D^{in,t}- \bup{A^t + (A^t)^\dagger}+\kself\id} \,\v{s}^t
&=\rup{D^{out,t}-D^{in,t}}\v{1} \nonumber \\& +\kself\, \v{s}^{t-1} \, . 
\end{align}

Here 
$D^{out,t}$ and $D^{in,t}$ are diagonal matrices whose entries are the weighted out- and in-degrees $D^{out,t}_{ii}=\sum_{j}A^t_{ij}$ and $D^{in,t}_{ii}=\sum_{j}A^t_{ji}$; 
$\dagger$ denotes the transpose; 
$\id$ is the identity matrix; 
and $\v{1}$ is the all-ones vector.

The matrix on the left side of~\Cref{eqn:fullsolution} is invertible if $\kself > 0$. In particular, its eigenvector $\v{1}$ has eigenvalue $N \kself$. Thus for each $A^t$ and each $\v{s}^{t-1}$, Eq.~\eqref{eqn:fullsolution} has a unique solution $\v{s}^t$. Overall, Eq.~\eqref{eqn:fullsolution} is similar to the regularized version of SpringRank~\cite{de2018physical} with regularization parameter $\alpha= \kself$. However, unlike the static model, there is a term on the right-hand side containing the previous ranks $\v{s}^{t-1}$, creating a Markovian dependence between successive time-steps. We refer to this model as \dsrfull\ (\dsr).

Importantly the online DSR approach does not actually minimize $\Htotal$, instead solving a sequence of minimization problems, one for each time step. To minimize $\Htotal$ instead, we set $\nabla \Htotal(\v{s}^t) = 0$, solving for the minimizers $\v{s}^t$ over all $N(T+1)$ ranks simultaneously, yielding the following system of equations (SI \Cref{sec:h_total_derive}):

\begin{align}\label{eqn:h_total}
\rup{ D^{out,t}+D^{in,t} - \bup{A^t+(A^t)^\dagger} + 2\kself\id}\,\v{s}^t 
&=\rup{D^{out,t}-D^{in,t}}\v{1} \nonumber\\ 
& +\kself \,\bup{\v{s}^{t-1} + \v{s}^{t+1}} \, . 
\end{align}
This differs from \Cref{eqn:fullH} in that the right-hand side now includes both past and future ranks (which doubles the contribution of $\kself$ on the left). We remove the terms $\v{s}^{t-1}$ and $\v{s}^{t+1}$ for $t=0$ and $t=T$ respectively. This entire system has translational symmetry, since the energy Eq.~\eqref{eqn:fullH} remains the same if we add the same constant to all ranks at all times, but we can again break this symmetry by setting the mean rank to zero.

Additionally, in contrast to \Cref{eqn:fullsolution}, the ranks at $t$ now depend on both $t-1$ and $t+1$, which themselves depend on ranks at adjacent time-steps, so that ranks are affected by interactions in both the past and the future. In computer science, methods like this where the entire history is provided to the algorithm are called \emph{offline}, to distinguish them from \emph{online} approaches that update their results in real time as data becomes available. Thus we refer to this model as \nmdsrfull\ (\nmdsr).  

The cost of solving \Cref{eqn:fullsolution} for a single time-step is the same as static SpringRank with only one additional parameter to be tuned using cross-validation, and there are $T$ such $N$-dimensional equations to be solved successively. On the other hand, \Cref{eqn:h_total} requires solving a single  system of dimension $NT$, whose operator consists of $T$ blocks, each of dimension $N\times N$. While these two approaches feature numbers of non-zero entries that are fundamentally determined by the number of total edges across all time steps, the cost of solving \dsr vs \nmdsr will depend on the particular choice of linear solver~\cite{peng2021solving}.

Philosophically, Eqns.~\eqref{eqn:fullsolution} and~\eqref{eqn:h_total} are trying to do two different things. If we are given all the data $A^0,A^1,\ldots,A^T$ and we want to infer retrospectively how each individual's rank changed over time, it makes sense to include both past and future interactions as in~\eqref{eqn:h_total} so that $s_i^t$ is affected by $i$'s entire history. 

In contrast, \eqref{eqn:fullsolution} can be viewed as modeling each individual's perceived rank at the time, based only on the interactions that have occurred so far.

In principle, one could envisage other ways to formally incorporate an explicit dependence on  $\v{s}^{t-1}$ into the model, and we provide one example in SI \Cref{sec:sidynl}. However, we found that the approaches presented in this Section provide a natural interpretation, result in good prediction performance on both real and synthetic datasets (see \Cref{sec:results}) and are computationally scalable. 

We close this section with two possible extensions to these models. First, in some settings we might have timestamps $t$ that are not successive integers $0,1,\ldots,T$. In this case, if the time interval between two successive times is $\Delta t$, one could scale the spring constant of the self-springs between time-steps as $\kself/\Delta t$. This corresponds to the fact that if we have $\Delta$ identical springs in series, each of which is stretched by $(s^t-s^{t-1})/\Delta$, their total energy is $(1/2)(\kself/\Delta)(s^t-s^{t-1})^2$. The same expression applies if the timestamps are real-valued so that $\Delta$ is not an integer.

Second, if we believe that not just the ranks themselves but their rates of change behave smoothly over time, one could add a momentum term to the Hamiltonian which is quadratic in the discrete second derivative of the ranks. Since
\begin{gather*}
\left( (s^{t+1}-s^t) - (s^t-s^{t-1}) \right)^2
= \left( s^{t+1} - 2 s^t + s^{t-1} \right)^2 \\
= 2 (s^t-s^{t-1})^2 + 2 (s^{t+1}-s^t)^2 - (s^{t+1} - s^{t-1})^2 \, ,
\end{gather*}
this is equivalent to adding a repulsive force, i.e., a spring with negative spring constant, between ranks two time-steps apart. Note that the system nevertheless remains convex: this momentum term is positive semidefinite, so adding it to~\eqref{eqn:fullH} keeps the coupling matrix positive definite except for translational symmetry. Of course, these terms are second-order in time. In the online approach, one would have to determine $\v{s}^0$ from the static model, $\v{s}^1$ from the first-order model~\eqref{eqn:fullsolution}, and then use the model including this momentum term for $\v{s}^t$ for $t \ge 2$. We have not pursued this here, but it may make sense for certain datasets.


\subsection{Moving-window SpringRank}\label{subsec:mwsr}

Before we test the various versions of \dsrfull\ defined above, we consider a simpler model as a baseline. 
The simplest way to extend SpringRank to a dynamical context is to apply the static model to the interactions in a series of ``windows,'' where in each window we sum the interactions over a series of consecutive time-steps. For instance, we can compute $\v{s}^t$ for each $t$ by applying the static model to a window of width $\tau$, i.e., replacing $A^t$ with $\sum_{t'=t}^{t+\tau-1} A^{t'}$. Since these windows overlap, the resulting estimates $\v{s}^t$ will be smooth to some extent, even without imposing an explicit dependence between $\v{s}^t$ and $\v{s}^{t-1}$. We use this method, which we call \mwsrfull\ (\mwsr), as a baseline to compare with the dynamical models presented above.

Roughly speaking, a larger $\tau$ is like a larger self-spring constant $\kself$, since it induces more overlap between windows and thus a stronger correlation between the inferred ranks. However, like a decaying-history approach, \mwsr\ assumes a particular kernel for the importance of past time-steps: namely, that all $t'$ in the window are equally important. In contrast, \dsrfull\ infers the importance of past time-steps by coupling $\v{s}^t$ with $\v{s}^{t-1}$.

However, both models have a free parameter that needs to be tuned, i.e., $\kself$ and $\tau$. A shorter window $\tau$ or smaller spring constant $\kself$ allows the ranks to respond quickly to new interactions, while a longer window or larger spring constant more tightly couples nearby estimates. This trade-off suggests the existence of an optimal window length $\tau_{\opt}$. We tune $\tau$ using a cross-validation procedure as explained in SI \Cref{sisec:tuning}.


\subsection{Generative Model and Synthetic Data}
\label{sec:genmod}

Analogous to a model presented in~\cite{de2018physical}, we propose a probabilistic generative model for dynamic data. It takes as input the ranks $\v{s}^t$ and generates a sequence of weighted directed networks with adjacency matrix $A^t$ at time $t$. One can also imagine models that generate the ranks, for instance with a random walk with Gaussian steps whose log-probability is the self-spring Hamiltonian~\eqref{eqn:selfH}, but we treat $\v{s}^t$ as an input since we want the user of this model to have control over how the ground-truth ranks vary with time.  For instance, in our experiments below we generate synthetic data where the ranks vary sinusoidally.

The generative model has two real-valued parameters: a signal-to-noise ratio or inverse temperature $\beta$, and an overall density of edges $c$. Given the ranks $\v{s}^t$, it generates weighted, directed edges between each pair of nodes $i,j$ independently, as follows. The probability $P_{ij}^t(\beta)$ of $i$ ``beating'' $j$ at time $t$, giving a directed edge $i \to j$, is a logistic function as in~\cite{de2018physical} or the Bradley-Terry-Luce model~\cite{bradley1952,luce1959}:
\bea
\nonumber P_{ij}^t(\beta)=\frac{1}{1+\e^{-2\beta(s_i^t-s_j^t)}} \, .
\eea
The number of such edges, which gives the integer weight $A_{ij}^t$, is then drawn from a Poisson distribution whose mean $\lambda_{ij}^t$ is $cP^t_{ij}\,(\beta)$: 
\be
\label{generative_poiss}
A^t_{ij} \sim \Poi\left(\lambda_{ij}^t=\frac{c}{1+\e^{-2\beta(s_i^t-s_j^t)}}\right).
\ee
Since $P_{ij}^t(\beta) + P_{ji}^t(\beta)=1$, for any pair $i,j$ the total number of interactions $A_{ij}^t + A_{ji}^t$ is Poisson-distributed with mean $c$. The rank differences $s_i^t-s_j^t$ are used only to choose the directions of these edges. This  is equivalent to a model where we define a random multigraph where the number of edges between $i$ and $j$ is $\Poi(c)$, and then we choose the direction of each edge independently according to $P_{ij}^t$.

This is different from the generative model proposed in the static case in~\cite{de2018physical}. In that model the probability that $i$ and $j$ interact depends on $s_i-s_j$ so that nodes are more likely to interact if their ranks are fairly close. This is consistent with SpringRank's assumption that if $i$ beats $j$ then $j$ is below $i$, but not too far below it (since the springs have resting length $1$). This assumption makes sense for some datasets but not for others. By generating synthetic data without this dependence, our intent is to pose a greater challenge to SpringRank by modeling (for example) round-robin tournaments where every team plays each other.

\subsection{Model Evaluation}
\label{sec:testing}

Assessing a ranking model on real datasets is not straightforward since we do not know the true values of the underlying ranks. Nevertheless, we may measure the extent to which inferred ranks are accurate in the sense that they can predict the outcome of new observations. 

There are several performance metrics that can be used for prediction evaluation. From coarse-grained measures capable of predicting the likely winner to more fine-grained measures that also estimate odds, we consider four main metrics in our experiments, detailed in \Cref{sisec:evaluation}. We measure prediction performance using a cross-validation protocol where datasets are divided into training and test sets. The training set is used for hyperparameter tuning and parameter estimation while performance is evaluated on the test set. In order to preserve the chronological ordering of the data, the test set contains future observations, i.e., observations that chronologically follow those used in training. Hyperparameters for each method are tuned using grid-search in order to maximize the performance metrics as described in SI \Cref{sisec:tuning}.





%%% Local Variables:
%%% mode: latex
%%% TeX-master: "main"
%%% End:


\section{Node Injection Link Stealing Attack\label{sec:attack}}

GNNs are prone to various privacy attacks that usually aim at learning as much information as possible about their underlying graph structure. GNNs inherit the potential attacks against standard neural networks such as membership inference attacks \cite{MIA_GNN, MIA_Jiayuan}, whereby the goal of the adversary is to ascertain whether a sample is included in the training dataset or not.

As introduced earlier, in this paper, we focus on a particular attack named as \textit{link stealing attack}, where an adversary without access to the adjacency matrix aims to learn whether a particular edge exists or not.

In this section, we first introduce the threat model to characterize the adversary's background knowledge. Then, we propose our node injection link stealing attack that takes advantage of the dynamicity of GNNs.


\subsection{Threat model}
\subsubsection{Environment}
As mentioned in the previous section, we consider a GNN application in which a server has already trained the GNN using a specific dataset and offers access to this GNN through a black-box API. In this context, the black-box API is an interface provided by the server that enables users to interact with the pre-trained GNN model without directly accessing its internal components, such as the model architecture, parameters, or graph structure.
Users can submit prediction queries using node IDs. If a new node needs to be added to the graph, users can employ a \textit{connect} query to attach the node to the graph before querying its prediction based on its ID.
The API processes input data into output predictions, ensuring that the model's underlying computations remain hidden from the user. Users can query this GNN for the purpose of node classification. Hence, the query consists of the queried node's ID and the output of this query is the vector of prediction scores for this particular node. The users do not have the knowledge of edges of this graph. Hence the only information that a user knows is the set of nodes' ids.

\subsubsection{Adversary's goal and knowledge}

We consider an adversary, $\mathcal{A}$, who assumes the role of a GNN user. Her objective is to determine the neighbors of a specific \textit{target node}, $v_t$, selected from a set of \textit{target nodes}, $V_{\mathcal{A}}$, within the graph. This is done based on the GNN's predictions for the node set $V_{\mathcal{A}}$. In simpler terms, $\mathcal{A}$ aims to identify the neighbors of the target node $v_t$ that are included in the target set nodes $V_{\mathcal{A}}$.

We should note that if the adversary aims to identify all the links within the graph, then the set of target nodes $V_{\mathcal{A}}$ becomes the set containing all the nodes of the graph $V$. To achieve this, the adversary may need to perform multiple node injections, targeting different nodes from the graph each time. However, the practicality of such an approach is debatable. The adversary's selection of target nodes reflects her background knowledge about these nodes. For instance, in the context of social networks, the adversary's background knowledge could include information such as users' interests. This information can guide the adversary in selecting target nodes $V_{\mathcal{A}}$ that are more likely to be connected. In our attack scenario, we choose the target nodes uniformly at random.

The adversary $\mathcal{A}$ is able to obtain the predictions of the target nodes $V_{\mathcal{A}}$ by sending the server their corresponding IDs through the provided API.
In addition, the adversary $\mathcal{A}$ is able to use the \textit{connect} query to connect a node $v_m$ to a target node $v_t$. In general, we assume that the adversary does not have access to the features of the nodes in the graph, with the exception of certain attack strategies described in Sec.~\ref{subsection:malicious features strategies}.

\subsection{Node injection link stealing attack}
\label{sec:node-injection-attack}
In this section, we formally define our NILS attack that, unlike existing link-stealing attacks, exploits the dynamic nature of the underlying GNN. Indeed, adversary $\mathcal{A}$ can \textit{connect} new nodes and further query the prediction scores of a set of nodes $V_{\mathcal{A}}$ in the graph. While adding this new node $v_m$, $\mathcal{A}$ can choose which existing node $v_t$ it actually connects to and hence try to discover its neighbors. More formally:

\begin{enumerate}
\item $\mathcal{A}$ first queries the prediction scores of the target nodes $V_{\mathcal{A}}$
and receives the corresponding prediction matrix $P$ of the target nodes $V_{\mathcal{A}}$.
    \item $\mathcal{A}$ generates malicious features of a malicious node $v_m$ based on the obtained prediction matrix $P$ (see Sec.~\ref{sec:strategies} for further details on this step).
    \item Next, $\mathcal{A}$ sends a \emph{connect} query to inject the malicious node $v_m$. The query has the following parameters: the features $x_m$ of the new node, and the ID of the target node $v_t$ the adversary wishes to connect $v_m$ to.
    \item The server adds this \text{malicious} node $v_m$ to the graph and links it to the target node $v_t$.
    \item $\mathcal{A}$ queries back the server for new prediction matrix $P'$ of the target nodes $V_{\mathcal{A}}$ and obtains it.
    \item With access to $P$ and $P'$, $\mathcal{A}$ computes the $L_1$ distance between $P(v)$ and $P'(v)$ of each node $v$ in $V_{\mathcal{A}}$.
    % Next, $\mathcal{A}$ can infer the actual links of $v_t$ by computing the $L_1$ distance between the prediction scores $P$ and $P'$ of each node $v$  in $V_{\mathcal{A}}$ before and after the injection.
    A significant change in the prediction scores of a node $v$ indicates a high probability of being a neighbor to $v_t$. If the difference exceeds a threshold $R$, the adversary infers that node $v$ is a neighbor of $v_t$.
    % Indeed, there is a high chance that the prediction scores of $v_t$'s neighbors have significantly been modified. Hence, for a given node, if this difference is large enough, there is a high chance that this particular node $v$ is a neighbor of $v_t$.
\end{enumerate}

The decision threshold $R$ is determined through an extensive parameter tuning process, aiming for an optimal trade-off between precision and recall in identifying the true neighbors of the target node. This balance is represented by the $F_1$ score. We evaluate various candidate values of $R$, selecting the one that yields the highest $F_1$ score as the optimal threshold. The results reported in our study are based on this optimal value of $R$.

 This attack strategy is depicted in Figure \ref{fig:attack_strategy} and outlined in Algorithm \ref{alg:prob_vecs_attack}.
 
% Figure environment removed

\begin{algorithm}
\caption{Node Injection Link Stealing Attack}
\label{alg:prob_vecs_attack}
\textbf{Input:} set of nodes $V_{\mathcal{A}}$ and target node $v_t$. \\
\textbf{Output:} the identified neighbors of $v_t$ by the adversary.\\

$P$ = GNN($V_{\mathcal{A}}$, $X_{V_{\mathcal{A}}}$) \Comment{Step 1}\\
Generate malicious features $x_m$ of node $v_m$ \Comment{Step 2}\\
Connect node $v_m$ to $v_t$. \Comment{Step 3-4}\\
$P'$ = GNN($V_{\mathcal{A}} \cup v_m$ , $X_{V_{\mathcal{A}}} \cup x_m$) \Comment{Step 5}\\
\For{each node $v$ in $V_{\mathcal{A}}$}{ 
$D(v)$ = $\lVert P(v) - P'(v) \rVert_1$ \Comment{Step 6} \\ 
\If{$D(v) \geq R$}{
      $v$ is a neighbor of $v_t$ \\}
\Else{
      $v$ is not a neighbor of $v_t$ \\}
}
\end{algorithm}

 \subsection{Strategies for malicious node's features\label{subsection:malicious features strategies}}
 \label{sec:strategies}
In order to evaluate how the injection of the malicious node $v_m$ influences the predictions of the GNN, we study five strategies to generate the malicious node's features $x_m$. This helps us to assess the success of our attack. These five strategies are designed with varying degrees of sparsity and stealthiness,
%\javi{[after reading the whole subsection, I'm not sure about what we mean by stealthiness. I think the key point here is whether the adversary knows the features of ]}
enabling us to explore their effectiveness in altering the model's predictions. We define the proposed strategies as follow:

\begin{enumerate}
    \item \textbf{All-ones strategy}: Generates a dense feature vector for the malicious node, containing all ones, as shown in the equation below:
    \begin{equation*}
        x_m = \mathbf{1}.
    \end{equation*}
    This strategy potentially causes significant changes in predictions but may be less stealthy due to its dense feature vector.
    
    \item \textbf{All-zeros strategy}: Creates a sparse feature vector for the malicious node, containing all zeros, as shown in the equation below:
    \begin{equation*}
        x_m = \mathbf{0}.
    \end{equation*}
    This approach may subtly alter the output of the GNN, leading to smaller changes in predictions, while offering increased stealthiness.
    
    \item \textbf{Identity strategy}: Introduces a malicious node with a feature vector identical to the target node's feature vector, as shown below:
    \begin{equation*}
        x_m = x_t.
    \end{equation*}
    This strategy causes confusion in the model's predictions for neighboring nodes and has variable stealthiness based on the similarity between injected and target nodes.
    For this strategy, we assume that $\mathcal{A}$ knows the features of the target node $x_t$.
    
    \item \textbf{Max attributes strategy}: This method creates a malicious node feature vector by computing the element-wise maximum of each attribute in the target nodes' feature matrix.
    Specifically, it considers only nodes from classes different from the target node's class, as shown below:
    \begin{equation*}
    x_{m,k} = \max_{i \in V_{\mathcal{A}}, \text{ with } C_i \neq C_t} X_{i,k}, \quad \text{for} \quad k = 1, \ldots, d.
    \end{equation*}
    Here, $C_i$ represents the class of node $i$, and $C_t$ is the class of the target node.
    This strategy potentially causes significant changes in predictions but may be less stealthy due to exaggerated features. We assume in this strategy that the adversary has access to the features of the set of target nodes $V_{\mathcal{A}}$ and also to their predicted classes by the GNN. The predicted classes are accessible to the adversary after step 1 in Algorithm \ref{alg:prob_vecs_attack}.
    
    \item \textbf{Class representative strategy}: This approach generates a malicious node feature vector by selecting the feature vector of the node with the highest confidence score for a specific class, different from the target node's class, as shown below:
    \begin{equation*}
    x_m = x_{i^*} \text{ with } i^* = \argmax_{\substack{ i \in V_{\mathcal{A}}, \\ C_i \neq C_t}} p_{i,j}.
    \end{equation*}
    In this equation, $x_m$ is the malicious node feature vector, $i^*$ is the node index with the highest confidence score for a specific class different from the target node's class, $V_{\mathcal{A}}$ is the set of target nodes, $C_i$ represents the class of node $i$, and $C_t$ is the class of the target node. This strategy leverages the model's predictions to alter the neighbors of the target node predictions, potentially offering increased stealthiness.
\end{enumerate}
Additionally, we introduce the \textit{so-called} LinkTeller \textbf{Influence} strategy as an alternative to the original method in \cite{linkteller} incorporating their feature perturbation strategy.
This strategy entails perturbing the features of the target node by adding a small real value $\delta$, as shown below:
\begin{equation*}
x_m = x_t + \mathbf{\delta}.
\end{equation*}
We assess the performance of the Influence strategy in comparison to other strategies, aiming to determine whether the attack performance gains are attributable to node injection or the crafting of malicious features. It is worth noting, however, that the Influence strategy may be easily detected if the feature $x_t$ has a discrete nature, given that $x_m$ is real-valued.


In this section, we first describe our algorithm for $\cliquedet{k, \ell}$, and then analyze its running time in some interesting cases. 

Throughout this section, we use $g(k, \ell)$ to denote our algorithm's running time exponent on the number of $\ell$-cliques of $\cliquedet{k, \ell}$, i.e., our algorithm for $\cliquedet{k, \ell}$ runs in $\tO(\Delta_\ell^{g(k, \ell)})$ time. 

\subsection{General Detection Framework}
Now we describe a generic algorithm for $\cliquedet{k, \ell}$ for $k \ge 3$ (for $k=2$, we trivially list all edges in the graph, so $g(2, 1) = 2$) in Algorithm~\ref{alg:generic_detection}.

\begin{algorithm}
\caption{Generic $
\cliquedet{k, \ell}$ algorithm.}\label{alg:generic_detection}
\begin{algorithmic}
\item \textbf{Input:} Graph $G = (V, E)$ and the list $L$ of all $\ell$-cliques. 
\item \textbf{Output:} Output {\sc yes} if $G$ contains a $k$-cliques, and {\sc no} otherwise.
\item \textbf{The Algorithm:} 
\begin{itemize}
    \item Let integers $k \ge a \geq b \geq c \ge 1$ be such that $k = a + b + c$ (the algorithm chooses $a, b, c$ optimally). Then goal is then to bound the number of $d$-cliques for $d \in \{a, b, c\}$. 
    \begin{itemize}
        \item If $d \ge \ell$, we can use Lemma~\ref{lem:simple_list_ub} to upper bound the number of $d$-cliques with $S_d = \tilde{\Theta}(\Delta_\ell^{d/\ell})$, and add these $d$-cliques to a list $L_d$ in the same time. 
        \item If $d < \ell$, for every $d$-clique $K$ with $\Delta_
        \ell (K) \leq \Delta_\ell^{x_d}$ (for some parameter $x_d \in [0, 1]$ to be chosen), we check if $K$ is in a $k$-clique by recursively running $\cliquedet{k-d, \ell-d}$ in its neighbourhood. Then, let $L_d$ denote the set of remaining $d$-cliques. Then, $S_{d} := |L_d|  = \Theta(\Delta_\ell^{1-x_d})$. The running time of this step is 
        \begin{align*}
            \tO\left(\sum_{\substack{K: d\text{-clique}\\\Delta_\ell(K) \le \Delta_\ell^{x_d}}} \Delta_\ell(K)^{g(k-d, \ell - d)}\right) & \le \tO\left(\sum_{\substack{K: d\text{-clique}\\\Delta_\ell(K) \le \Delta_\ell^{x_d}}} \Delta_\ell(K) \cdot \Delta_\ell^{x_d (g(k-d, \ell - d) - 1)}\right)\\
            & \le \tO\left(\Delta_\ell^{1+x_d (g(k-d, \ell - d) - 1)}\right). 
        \end{align*}
    \end{itemize}
    \item Finally, we conduct a usual matrix multiplication of dimensions $S_a, S_b, S_c$ in time $\MM(S_a, S_b, S_c)$ as follows. If we find a $k$-clique, output {\sc yes}, otherwise we output {\sc no.}
    \begin{itemize}
        \item Create a matrix $X$ whose rows are indexed by $a$-cliques in $L_a$ and columns are indexed by  $b$-cliques in $L_b$. Set $A[K_a, K_b] = 1$ if  the nodes of $K_a$ and $K_b$ form an $(a+ b)$-clique, and $0$ otherwise.
        \item Create a matrix $Y$ whose rows are indexed by $b$-cliques in $L_b$ and columns are indexed by $c$-cliques in $L_c$, and set the entries similarly.
        \item Compute $Z = XY$. For each pair of  remaining $a$-clique $K_a$ and $c$-clique $K_c$ that form an $(a + c)$-clique, check if $Z[K_a, K_c] > 0$. If such an entry exists, output {\sc yes}. Otherwise, output {\sc no}.
    \end{itemize}
    
\end{itemize}
\end{algorithmic}
\end{algorithm}

The correctness of this algorithm is immediate. 
We also remark that the algorithm can  be used to count the number of $k$-cliques, by replacing all the recursive calls with the counting version of the algorithm, using the matrix multiplication to count the number of $k$-cliques in the remaining graph, and properly summing up and scaling the numbers. Clearly, the counting version of the algorithm will have the same running time. 

\subsection{Examples}
\label{sec:detection_examples}
Let us give some explicit examples to illustrate the algorithm. 

\paragraph{$\cliquedet{k,1}$.} The simplest example is $\cliquedet{k,1}$ for $k \ge 3$. Let $\lfloor k/3 \rfloor \leq c \leq b \leq a \leq \lceil k/3\rceil$ be integers such that $a+b+c = k$, which is one of the possible choices of $a, b, c$ for the algorithm.
Note that $c = \lfloor k/3 \rfloor, b =  \lceil (k-1)/3\rceil, a = \lceil k/3 \rceil$. Since $a, b, c \ge \ell = 1$, the algorithm would choose to use Lemma~\ref{lem:simple_list_ub} to bound the number of cliques of sizes $a, b, c$ as $n^a, n^b, n^c$ respectively. Thus, the running time of the algorithm is $\tO(n^{\omega(a, b, c)}) = \tO(n^{\beta(k)})$, matching the previous running time \cite{eisenbrand2004complexity}.

\paragraph{$\cliquedet{k,\ell}$ for $\ell \leq \lfloor k / 3\rfloor.$}
Similar as above, let $c = \lfloor k/3 \rfloor, b =  \lceil (k-1)/3\rceil, a = \lceil k/3 \rceil$ and the algorithm would choose to use Lemma~\ref{lem:simple_list_ub} to bound the number of cliques of sizes $a, b, c$. Thus, the running time of the algorithm is  $\tO(\Delta_\ell^{\omega(a/\ell, b/\ell, c/\ell)}) \leq \tO(\Delta_\ell^{\omega(\lceil k/3 \rceil,   \lceil (k-1)/3\rceil, \lfloor k/3 \rfloor)/\ell})$. 
This running time is optimal barring improvements for $\cliquedet{k, 1}$:


\begin{table}[ht]
    \centering
    \begin{tabular}{c|c|c|c|c|c|c|c|c|c|c}
         \backslashbox{$\ell$}{$k$} &  3 & 4 & 5 & 6 & 7 & 8 & 9 & 10 & 11 & 12\\
         \hline 
         1 & 2.373 & 3.252 & 4.089 & 4.746 & 5.594 & 6.401 & 7.119 & 7.952 & 8.751 & 9.492\\
         2 & 1.408 & 1.657 & 2.058 & 2.373 & 2.797 & 3.201 & 3.559 & 3.976 & 4.376 & 4.746\\
         3 & - & 1.248 & 1.422 & 1.669 & 1.918 & 2.151 & 2.373 & 2.651 & 2.917 & 3.164\\
         4 & - & - & 1.175 & 1.298 & 1.487 & 1.657 & 1.841 & 2.028 & 2.207 & 2.373\\
         5 & - & - & - & 1.130 & 1.233 & 1.378 & 1.504 & 1.660 & 1.811 & 1.954
    \end{tabular}
    \caption{Our $\cliquedet{k, \ell}$ exponent for various values of $k, \ell$ with the best current bound on $\omega$ \cite{alman2021refined} and rectangular matrix multiplication \cite{LU18}. See also \cite{van2019dynamic} for a way to bound $\omega(a, b, c)$ for arbitrary $a, b, c > 0$ from values of $\omega(1, x, 1)$. 
    The $(k, \ell)$th entry corresponds to the exponent $\alpha$ such that the runtime to detect a $k$-clique is $\tilde{O}(\Delta_\ell^\alpha)$ , where $\Delta_\ell$ is the number of $\ell$-cliques. 
    }
    \label{tab:det_exponent}
\end{table}

\begin{proposition}
Fix any positive integers $k \ge 3$ and $\ell \le \lfloor k/3\rfloor$, and let $\beta(k) = \omega(\lceil k/3 \rceil,   \lceil (k-1)/3\rceil, \lfloor k/3 \rfloor)$.
If $\cliquedet{k, 1}$ requires $n^{\beta(k) - o(1)}$ time, then
$\cliquedet{k, \ell}$ requires $\Delta_\ell^{\beta(k)/\ell-o(1)}$ time. 
\end{proposition}
\begin{proof}
Suppose for the sake of contradiction that $\cliquedet{k, \ell}$ has an  $O(\Delta_\ell^{\beta(k)/\ell-\eps}$) time algorithm $\mathcal{A}$ for some $\eps > 0$. Then given a $\cliquedet{k, 1}$ instance, we can first use Lemma~\ref{lem:simple_list_ub} to list all $\ell$-cliques in $O(n^\ell)$ time, and the number of $\ell$-cliques is bounded by $O(n^\ell)$. Then we can use $\mathcal{A}$ to solve the $\cliquedet{k, 1}$ instance in $O((n^\ell)^{\beta(k)/\ell-\eps})=n^{\beta(k)-\eps \ell}$ time, a contradiction.
\end{proof}



\begin{example}[$\cliquedet{3, 2}$]
\label{ex:clique-det-3-2}
\em
In this case, the algorithm can only choose $a=b=c=1$, and it would naturally choose $x_a=x_b=x_c$. The time it takes to bound the number of $1$-cliques (nodes) is $\tO(\Delta_2^{1+x_a (g(2, 1) - 1)}) = \tO(m^{1+x_a})$. Then we have $S_a, S_b, S_c \le \Theta(m^{1-x_a})$. Thus, the running time for the matrix multiplication of dimensions $S_a, S_b, S_c$ is $\tO(m^{(1-x_a)\omega})$. Overall, the running time is $\tO(m^{\frac{2\omega}{\omega+1}})$ by setting $x_a = \frac{\omega-1}{\omega+1}$. This is essentially Alon, Yuster and Zwick \cite{alon1997finding}'s triangle detection algorithm for sparse graphs. 
\end{example}

\begin{example}[$\cliquedet{4, 2}$]
\label{ex:clique-det-4-2}
\em
In this case, the algorithm can only choose $a=2, b=c=1$, and it would naturally choose $x_b=x_c$. The algorithm uses Lemma~\ref{lem:simple_list_ub} to (trivially) bound the number of edges as $m$. 
The time it takes to bound the number of nodes is $\tO(\Delta_2^{1+x_b (g(3, 1) - 1)}) = \tO(m^{1+x_b(\omega-1)})$. Then we have $S_a \le \Theta(m), S_b, S_c \le \Theta(m^{1-x_b})$. Thus, the running time for the matrix multiplication of dimensions $S_a, S_b, S_c$ is $\tO(m^{\omega(1, 1-x_b, 1-x_b)})$. The algorithm chooses $x_b$ so that $1+x_b(\omega-1) = \omega(1, 1-x_b, 1-x_b)$. If we simply bound $\omega(1, 1-x_b, 1-x_b)$ by $x_b + \omega(1-x_b)$, we can get $g(4, 2) \le \frac{\omega+1}{2}$ by setting $x_b = \frac{1}{2}$. For the current best bound of rectangular matrix multiplication \cite{LU18}, we can set $x_b = 0.478$ to get an upper bound $g(4, 2) \le 1.657$. As seen in Table~\ref{table:improved_det_4_5}, this is an improvement over the previous best algorithm of Eisenbrand and Grandoni \cite{eisenbrand2004complexity}. The key difference between our algorithm and \cite{eisenbrand2004complexity}'s algorithm is that, after they perform a similar first stage, they recursively call a $\cliquedet{4, 1}$ algorithm on graphs with $S_b$ nodes, losing the information that the graph has $S_a = m$ edges to begin with. We instead utilize this information with rectangular matrix multiplication to get a better running time. 

\end{example}

\begin{example}[$\cliquedet{5, 2}$]
\label{ex:clique-det-5-2}
\em
In this case, let the algorithm  choose $a=b=2, c=1$ (the choice $a=3, b=c=1$ gives a worse bound). The algorithm uses Lemma~\ref{lem:simple_list_ub} to (trivially) bound the number of edges as $m$. 
The time it takes to bound the number of nodes is $\tO(\Delta_2^{1+x_c (g(4, 1) - 1)}) = \tO(m^{1+x_c(\omega(1, 2, 1)-1)})$. Then we have $S_a,S_b \le \Theta(m), S_c \le \Theta(m^{1-x_c})$. Thus, the running time for the matrix multiplication of dimensions $S_a, S_b, S_c$ is $\tO(m^{\omega(1, 1, 1-x_c)})$. The algorithm chooses $x_c$ so that $1+x_c(\omega(1, 2, 1)-1) = \omega(1, 1, 1-x_c)$. If we simply bound $\omega(1, 2, 1)$ by $\omega + 1$ and $\omega(1, 1, 1-x_c)$ by $2x_c + (1-x_c)\omega$, we can get $g(5, 2) \le \frac{\omega+2}{2}$ by setting $x_c = \frac{1}{2}$. For the current best bound of rectangular matrix multiplication~\cite{LU18}, we can set $x_c = 0.4698$ to get an upper bound $g(5, 2) \le 2.058$. As seen in Table~\ref{table:improved_det_4_5}, this is an improvement over the previous best known algorithm of Eisenbrand and Grandoni \cite{eisenbrand2004complexity}.
\end{example}


\begin{example}[More Small Examples]
\label{ex:more-small-examples}
\em
See Tables~\ref{tab:detection_runtime} and \ref{tab:det_exponent} for more examples of the running times of our algorithm. These running times were obtained by finding the optimal values of $a, b, c$ using dynamic programming. 

From previous examples, one might wonder whether the algorithm always sets $a, b, c$ as close to $k/3$ as possible. The following example shows that it is not the case (for $\omega = 2$). 

In $\cliquedet{8, 4}$, if the algorithm chooses $a=4, b = c = 2$, then the running time is 
$$\tO\left(\Delta_4^{1+x_b(g(6, 2)-1)}+\Delta_4^{1+x_c(g(6, 2)-1)} + \Delta_4^{\omega(1, 1-x_b, 1-x_c)}\right).$$
By setting $x_b=x_c = \frac{1}{2}$, this running time is bounded by $\tO(\Delta_4^{3/2})$ when $\omega = 2$ (See Table~\ref{tab:det_exponent} for the value of $g(6, 2)$ when $\omega = 2$). 

However, if the algorithm chooses a more balanced choice $a=b=3, c = 2$, then the running time is $$\tO\left(\Delta_4^{1+x_a(g(5, 1)-1)}+\Delta_4^{1+x_b(g(5, 1)-1)}+\Delta_4^{1+x_c(g(6, 2)-1)} + \Delta_4^{\omega(1-x_a, 1-x_b, 1-x_c)}\right).$$
One optimal way to set the parameters when $\omega = 2$ is $x_a = x_b = \frac{1}{5}$ and $x_c = \frac{3}{5}$, which only gives an $\tO(\Delta_4^{8/5})$ running time when $\omega = 2$ (See Table~\ref{tab:det_exponent} for the values of $g(5, 1)$ and $g(6, 2)$ when $\omega = 2$). 
\end{example}

\subsection{Upper Bound for \texorpdfstring{$\cliquedet{k, k - h}$}{(k, k-h)-Clique-Detection}}
\label{sec:k-h_detect_bound}

In this section, we analyze the running time of our algorithm for $\cliquedet{k, k - h}$ for some constant $h=O(1)$. For convenience, let $e_h(k) = g(k, k - h)$.

We start with the following lemma.
\begin{lemma}
\label{lem:det_exponent_monotone}
For every $k > h$, $e_h(k + 1) \le e_h(k)$. 
\end{lemma}
\begin{proof}
We prove the statement by induction. We skip the base case $k = h+1$ as it works similarly as the induction step
(except for $h=1$, in which case $e_h(2) = 2$ and $e_h(3) = \frac{2\omega}{\omega+1} \le e_h(2)$, as the algorithm handles $\cliquedet{2, 1}$ specially).  Suppose the statement is already true for all smaller $k$. 

Let $\ell = k - h$ and $\ell' = k + 1 - h$. Suppose for $\cliquedet{k, \ell}$, the optimal parameters are $a, b, c, x_a, x_b, x_c, S_a, S_b, S_c$ ($x_d$ is relevant only if $d < \ell$ for $d \in \{a, b, c\}$).  Consider $\cliquedet{k+1, \ell+1}$ with parameters $a' = a+1, b' =b, c'=c$ and $x'_{a'}, x'_{b'}, x'_{c'}, S'_{a'}, S'_{b'}, S'_{c'}$ to be determined. Let $\Delta_\ell$ be the number of $\ell$-cliques in the $\cliquedet{k, \ell}$ instance and let $\Delta'_{\ell'}$ be the number of $(\ell+1)$-cliques in the $\cliquedet{k+1, \ell+1}$ instance. 

We first compare exponents related to $S_a$ and $S'_{a'}$.
\begin{itemize}
    \item If $a \ge \ell$. Then $S_a$ in $\cliquedet{k, \ell}$ is bounded by $\tO(\Delta_\ell^{a / \ell})$. In the $\cliquedet{k + 1, \ell + 1}$ algorithm, $S_{a'}'$ is bounded  by $\tO((\Delta'_{\ell'})^{(a+1)/(\ell+1)})$, a smaller exponent. 
    \item If $a < \ell$, the exponent of the running time for bounding $S_a$ in $\cliquedet{k, \ell}$ is $1+x_a (e_h(k-a) - 1)$, and $S_a$ is bounded by $\Delta_\ell^{1-x_a}$. Let $x'_{a'}$ be equal to $x_a$ in the algorithm for $\cliquedet{k + 1, \ell + 1}$. Then notice that the exponent for running time is $1+x'_{a'}(e_h(k + 1 - a') - 1) = 1+x_a(e_h(k-a)-1)$ and the bound on $S'_{a'}$ is $(\Delta'_{\ell'})^{1-x_a}$, both with same exponents as previous bounds. 
\end{itemize}
We then compare exponents related to $S_b$ and $S'_{b'}$. 
\begin{itemize}
    \item If $b > \ell$. Then $b' = b \ge \ell+1 = \ell'$. Then $S_b$ in $\cliquedet{k, \ell}$ is bounded by $\tO(\Delta_\ell^{b / \ell})$. In the $\cliquedet{k + 1, \ell + 1}$ algorithm, $S_{b'}'$ is bounded  by $\tO((\Delta'_{\ell'})^{b/(\ell+1)})$, a smaller exponent. 
    \item If $b = \ell$. In this case, $S_b = \tO(\Delta_\ell)$ and we will have $b' < \ell'$. Let $x'_{b'} = 0$ in $\cliquedet{k + 1, \ell + 1}$. Then $S'_{b'}$ is bounded by $\tO((\Delta'_{\ell'})^1)$, the same exponent as the bound of $S_b$. Also, the cost for having this bound is $\tO((\Delta'_{\ell'})^{1+x'_{b'}(e_h(k+1-b'))}) = \tO(\Delta'_{\ell'})$, so we can ignore the cost as it is near-linear time.
    \item If $b < \ell$, the exponent of the running time for bounding $S_b$ in $\cliquedet{k, \ell}$ is $1+x_b (e_h(k-b) - 1)$, and $S_b$ is bounded by $\Delta_\ell^{1-x_b}$. Let $x'_{b'}$ be equal to $x_b$ in the algorithm for $\cliquedet{k + 1, \ell + 1}$. Then notice that the exponent for running time is $1+x'_{b'}(e_h(k + 1 - b') - 1) = 1+x_b(e_h(k-b+1)-1)$. By the induction assumption, 
    $e_h(k-b+1) \le e_h(k-b)$, so $1+x_b(e_h(k-b+1)-1)$ is upper bounded by the running time exponent of the corresponding case in $\cliquedet{k, \ell}$. Note that this case does not happen in the base case $k=h+1$, as $b < \ell = 1$ can never happen, so we can safely apply the induction assumption. 
    The bound on $S'_{b'}$ is $(\Delta'_{\ell'})^{1-x_b}$,  with the same exponent as $S_b$ in $\cliquedet{k, \ell}$. 
\end{itemize}
The comparison of the exponents related to $S_c$ and $S'_{c'}$ works similarly. Thus, $e_h(k+1) \le e_h(k)$. 
\end{proof}

\begin{proposition}
\label{prop:eh_upper_bound}
$e_h(k) = 1+O\left(1/ k^{\log_{\frac{3}{2}}(\frac{\omega}{\omega-1})}\right)$. 
\end{proposition}
\begin{proof}
Let $\ell = k - h$. 
 Let $k_0 = 100h$. For all $k \le k_0$, $e_h(k) = O(1)$. 

For $k > k_0$, we choose $a, b, c$ in our $\cliquedet{k, k - h}$ algorithm so that $\lfloor k/3\rfloor = c \le b \le a = \lceil k/3\rceil$. Clearly, $a, b, c < \ell = k - h$. The running time of the algorithm is thus 
$$\tO\left(\Delta_\ell^{1+x_a  \cdot (e_h(k-a)-1)} 
+ \Delta_\ell^{1+x_b  \cdot (e_h(k-b)-1)} 
+ \Delta_\ell^{1+x_c  \cdot (e_h(k-c)-1)} 
+ MM\left(\Delta_\ell^{1-x_a}, \Delta_\ell^{1-x_b},\Delta_\ell^{1-x_c} \right)\right).$$
By Lemma~\ref{lem:det_exponent_monotone}, $e_h(k-c) \le e_h(k-b) \le e_h(k-a)$, so the running time is bounded by 
$$\tO\left(\Delta_\ell^{1+\max\{x_a, x_b, x_c\}  \cdot (e_h(k-a)-1)} 
+MM\left(\Delta_\ell^{1-x_a}, \Delta_\ell^{1-x_b},\Delta_\ell^{1-x_c} \right)\right).$$
Set $x_a = x_b = x_c = \frac{\omega -1}{\omega + e_h(k-a) - 1}$. The running time then becomes 
$$\tO\left(\Delta_\ell^{\frac{\omega \cdot e_h(k-a)}{\omega + e_h(k-a) - 1}}\right).$$
Thus,
$e_h(k) \le \frac{\omega \cdot e_h(k-a)}{\omega + e_h(k-a) - 1}$. Consequently, $$e_h(k) - 1 \le \frac{(\omega - 1) \cdot (e_h(k-a) - 1)}{\omega + e_h(k-a) - 1} \le \frac{\omega - 1}{\omega}  \cdot (e_h(k-a) - 1) = \frac{\omega - 1}{\omega}  \cdot (e_h(k-\lceil k/3\rceil) - 1).$$
Therefore $e_h(k) - 1 \le O\left(\left(\frac{\omega - 1}{\omega}\right)^{\log_{\frac{3}{2}} k}\right) = O\left(1/ k^{\log_{\frac{3}{2}}(\frac{\omega}{\omega-1})}\right)$.
\end{proof}

We also show that our choices of $a, b, c$ are not too far away from optimal, at least when $\omega = 2$. In the following proposition, recall $e_h(k)$ is the exponent of our algorithm, instead of the best exponent for $\cliquedet{k, k - h}$. 

\begin{proposition}
$e_h(k) = 1+\Omega\left(1/ k^{\log_{\frac{3}{2}}(2)}\right)$. 
\end{proposition}
\begin{proof}
Let $\ell = k - h$,  $\rho = \log_{\frac{3}{2}}(2)$, and $f_h(k) = \frac{1}{e_h(k) - 1}$. 
 Let $k_0 = 100h$. It is not difficult to see that for all $k \le k_0$, $f_h(k) \le M k^\rho - 1$ for some sufficiently large constant $M > 1$ because our algorithm does not achieve almost linear time, i.e., it always has $e_h(k) > 1$ and thus $f_h(k) < \infty$. 
 
 Let $k > k_0$, and let $a, b, c$ be the optimal choices for $\cliquedet{k, k - h}$. We will show by induction that $f_h(k) \le M k^\rho - 1$. 
 Consider two cases. 
 
 For the first case, assume $a < \ell$. Let $x_a, x_b, x_c$ be the optimal parameters for $\cliquedet{k, k - h}$, and if there are multiple choices, we choose one set of parameters with smallest $x_a+x_b+x_c$. 
 Then, the bound of our running time is (up to $\tO(1)$ factors) 
$$\Delta_\ell^{1+x_a  \cdot (e_h(k-a)-1)} 
+ \Delta_\ell^{1+x_b  \cdot (e_h(k-b)-1)} 
+ \Delta_\ell^{1+x_c  \cdot (e_h(k-c)-1)} 
+ MM\left(\Delta_\ell^{1-x_a}, \Delta_\ell^{1-x_b},\Delta_\ell^{1-x_c}\right).$$ 

Suppose $x_a > x_b$. By Lemma~\ref{lem:det_exponent_monotone}, $e_h(k-a) \ge e_h(k-b)$. Therefore, we can slightly increase $x_b$, and the running time of the algorithm will not be worse. This contradicts with the optimality of $x_a, x_b, x_c$ and minimality of $x_a+x_b+x_c$. Thus, we must have $x_a \le x_b$. Similarly, we have $x_b \le x_c$. 

Then we can lower bound $MM\left(\Delta_\ell^{1-x_a}, \Delta_\ell^{1-x_b},\Delta_\ell^{1-x_c}\right)$ by   $\Delta_\ell^{2-x_a-x_b}$.

The optimal way to balance $\Delta_\ell^{1+x_a  \cdot (e_h(k-a)-1)}, \Delta_\ell^{1+x_b  \cdot (e_h(k-b)-1)}, \Delta_\ell^{1+x_c  \cdot (e_h(k-c)-1)}$ and $ \Delta_\ell^{2-x_a-x_b}$ is to set 
$x_a = \frac{e_h(k-b)-1}{e_h(k-a)e_h(k-b)-1}$, $x_b = \frac{e_h(k-a)-1}{e_h(k-a)e_h(k-b)-1}$ and $x_c = \min\{1, \frac{(e_h(k-a)-1)(e_h(k-b)-1)}{(e_h(k-a)e_h(k-b)-1)(e_h(k-c)-1)}\}$, which gives \[e_h(k) \ge \frac{2e_h(k-a)e_h(k-b)-e_h(k-a)-e_h(k-b)}{e_h(k-a)e_h(k-b)-1}.\] Substituting $e_h$ by $f_h$ gives the following cleaner formula:
$$f_h(k) \le 1+f_h(k-a)+f_h(k-b).$$
As the algorithm chooses the optimal $a, b, c$, we have that 
$$f_h(k) \le \max_{\substack{1 \leq c \leq b \leq a \leq k\\ a + b + c = k}} \left\{ 1 + f_h(k-a) + f_h(k-b)\right\}.$$

By Lemma~\ref{lem:det_exponent_monotone}, $f_h(k-b)$ is nondecreasing when $b$ increases, so we can pick $b$ to be as large as possible for fixed $a$. Therefore, for fixed $a$, we choose  $c = \lfloor \frac{k-a}{2} \rfloor$ and $b = \lceil \frac{k-a}{2} \rceil$. Therefore, we can rewrite
$$f_h(k) \le \max_{ k/3  \leq a \leq k-2} \left\{1 + f_h(k-a) + f_h\left(\left\lfloor \frac{k+a}{2}\right\rfloor\right)\right\}.$$

By the induction assumption, $f_h(k') \le M (k')^\rho - 1$ for all $k'<k$.

Then,
\begin{align*}
    f_h(k)&\leq \max_{k/3 \le a \le k-2} \left\{1 + f_h(k-a) + f_h\left(\left\lfloor\frac{k+a}{2}\right\rfloor\right)\right\} \\
    & \leq \max_{0 \le p \le k/3}\left\{ 1 + M \left(\frac{2k}{3} - 2p\right)^\rho + M\left(\frac{2k}{3} + p\right)^\rho - 2 \right\}\\
    & \leq Mk^\rho \cdot \max_{0 \le p' \le 1/3} \left\{\left(\frac{2}{3}-2p'\right)^\rho + \left(\frac{2}{3}+p'\right)^\rho \right\} - 1\\
    &\le Mk^\rho - 1,
\end{align*}
which completes the induction step for this case. 

For the other case, assume $a \ge \ell$. Note that we must have $b, c < \ell$ as $2\ell > k$. Let $x_b, x_c$ be the optimal parameters. Similar as before, we can assume $x_b \le x_c$. 
 Then, the bound of our running time is (up to $\tO(1)$ factors) 
\begin{align*}
&\Delta_\ell^{1+x_b  \cdot (e_h(k-b)-1)} 
+ \Delta_\ell^{1+x_c  \cdot (e_h(k-c)-1)} 
+ MM\left(\Delta_\ell^{a/\ell}, \Delta_\ell^{1-x_b},\Delta_\ell^{1-x_c}\right)\\
\ge & \Delta_\ell^{1+x_b  \cdot (e_h(k-b)-1)} 
+ \Delta_\ell^{1+x_c  \cdot (e_h(k-c)-1)} 
+ \Delta_\ell^{a/\ell + 1 - x_b}
\end{align*}
The optimal way to balance is to set $x_b = \frac{a}{\ell e_h(k-b)}$ and $x_c = \min\{1, \frac{a(e_h(k-b)-1)}{\ell e_h(k-b)(e_h(k-c)-1)}\}$. 
This gives $e_h(k) \ge \frac{ae_h(k-b)-a}{\ell e_h(k-b)}+1$. Note that it is possible that $x_b > 1$ in this setting, but if that happens, $e_h(k) > e_h(k-b)$, which by Lemma~\ref{lem:det_exponent_monotone}, can never be optimal. In terms of $f_h$, this implies that $f_h(k) \le \frac{\ell  (f_h(k-b)+1)}{a}$. 
As the algorithm chooses the optimal $a, b, c$, we have that 
$$f_h(k) \le \max_{\substack{1 \leq c \leq b < \ell \leq a \leq k\\ a + b + c = k}}  \frac{\ell  (f_h(k-b)+1)}{a}.$$
By Lemma~\ref{lem:det_exponent_monotone}, $f_h(k-b)$ is nondecreasing when $b$ increases, so we can pick $b$ to be as large as possible for fixed $a$. Therefore, for fixed $a$, we choose  $c = \lfloor \frac{k-a}{2} \rfloor$ and $b = \lceil \frac{k-a}{2} \rceil$. Thus, we can rewrite
$$f_h(k) \le \max_{ \ell \leq a \leq k-2} \frac{\ell  (f_h\left(\left\lfloor \frac{k+a}{2}\right\rfloor\right)+1)}{a} \le \max_{ \ell \leq a \leq k-2}  \left\{f_h\left(\left\lfloor \frac{k+a}{2}\right\rfloor\right)+1\right\}.$$
By induction, it can be further upper bounded by 
$$\max_{ \ell \leq a \leq k-2}  \left\{M\left(\left\lfloor \frac{k+a}{2}\right\rfloor\right)^\rho -1 +1\right\} \le M(k-1)^\rho < Mk^\rho - 1,$$
as $M, \rho > 1$. This finishes the induction step for this case. 

Overall, we have shown that $f_h(k) \le M k^\rho - 1$ for all $k$, which implies $e_h(k) = 1+\Omega\left(1/ k^{\log_{\frac{3}{2}}(2)}\right)$. 
\end{proof}

\subsection{Upper Bound for \texorpdfstring{$\cliquedet{C\ell, \ell}$}{(Cl, l)-Clique-Detection}}\label{sec:Cl_l_detectionbound}
Define a sequence of functions $(f_i)_{i \ge 0}$ as follows:
$$f_i(C) = \frac{2^i \omega^{i+1} C}{3^{i+1}(\omega-1)^i+\left(3 (2^i - 3^i) (\omega-1)^i - 2^i (\omega-1)^i \omega + 2^i \omega^{i + 1}\right)C}. $$
The functions have the following recurrence relation, whose proof we omit as it is straightforward algebra. 
\begin{claim}
$f_0(C) = \frac{\omega C}{3}$ and $f_i(C) = \frac{\omega}{1+\frac{\omega - 1}{f_{i-1}\left(\frac{2C}{3-C}\right)}}$ for $i > 0$.
\end{claim}


Then we can express the running time of $\cliquedet{C\ell, \ell}$ for sufficiently large $\ell$ in terms of the functions $f_i$:
\begin{theorem}\label{thm:det_mult_upper_bound}
Let $C > 1$ be any constant such that $\frac{1}{C} \in \left(1-\left(\frac{2}{3}\right)^i, 1-\left(\frac{2}{3}\right)^{i+1} \right]$ for some constant integer $i \ge 0$. Then for any $\ell \ge 1$ and $C\ell \le  k \le (C+o_\ell(1)) \ell$, $g(k, \ell) \le f_i(C) + o_\ell(1)$. 
\end{theorem}
\begin{proof}
We prove by induction on $i$. 

When $i = 0$, $k \ge C\ell \ge 3\ell$. Therefore, we can apply the  $\cliquedet{k,\ell}$ example in Section~\ref{sec:detection_examples} for $\ell \leq \lfloor k / 3\rfloor$  to get $g(k, \ell) \le \omega(\lceil k/3 \rceil,   \lceil (k-1)/3\rceil, \lfloor k/3 \rfloor)/\ell$. This leads to 
\begin{align*}
    g(k, \ell) &\le \omega(k/3+1, k/3+1,k/3+1) / \ell \\
    & = \frac{(k/3+1)\omega}{\ell}\\
    & \le \frac{((C+o_\ell(1)) \ell / 3 + 1) \omega}{\ell}\\
    & \le \frac{\omega C}{3} + o_\ell(1) = f_0(C) + o_\ell(1).
\end{align*}

When $i > 0$, assume the claim is correct for $i-1$. Similar to the proof of Proposition~\ref{prop:eh_upper_bound}, we choose $a, b, c$ in our $\cliquedet{k, \ell}$ algorithm so that $\lfloor k/3\rfloor = c \le b \le a = \lceil k/3\rceil$. By the same analysis, the running time exponent can then be bounded by 
$\frac{\omega \cdot g(k-a, \ell - a)}{\omega + g(k-a, \ell - a) - 1}$. Let $C' = \frac{2C}{3-C}$. It is not difficult to verify that $\frac{1}{C'} \in \left(1-\left(\frac{2}{3}\right)^{i-1}, 1-\left(\frac{2}{3}\right)^{i} \right]$. 

Also, 
\begin{align*}
    \frac{k-a}{\ell - a} &\ge \frac{k - k/3}{\ell - k/3} \ge \frac{C \ell - (C\ell) / 3}{\ell - (C \ell) / 3} = \frac{2C}{3 - C} = C',
\end{align*}
and 
\begin{align*}
    \frac{k-a}{\ell - a} &\le \frac{k - (k/3 + 1)}{\ell - (k/3 + 1)} \le \frac{(C + o_\ell(1))\ell - ((C + o_\ell(1))\ell) / 3}{\ell - ((C + o_\ell(1)) \ell) / 3} = \frac{2C + o_\ell(1)}{3 - C - o_\ell(1)} = C' + o_\ell(1).
\end{align*}
Thus, $C'(\ell - a) \le k-a \le (C'+o_\ell(1))(\ell-a)$, so $g(k-a, \ell - a) \le f_{i-1}(C') + o_\ell(1)$ by induction. Therefore, the running time exponent of $\cliquedet{k, \ell}$ can be bounded by 
\begin{align*}
    \frac{\omega \cdot g(k-a, \ell - a)}{\omega + g(k-a, \ell - a) - 1} &= \frac{\omega}{1 + \frac{\omega - 1}{g(k-a, \ell - a)}}
    \le \frac{\omega}{1 + \frac{\omega - 1}{f_{i-1}(C') + o_\ell(1)}}
     \le \frac{\omega}{1 + \frac{\omega - 1}{f_{i-1}(\frac{2C}{3-C})}} +  o_\ell(1) = f_i(C) + o_\ell(1).
\end{align*}
\end{proof}

% Figure environment removed

In Figure~\ref{fig:detection_upper_bound}, we compare the bound obtained from Theorem~\ref{thm:det_mult_upper_bound} with the actual running time of Algorithm~\ref{alg:generic_detection} computed by dynamic programming for $3 \leq k \leq 200$.
In particular, for various values of $C$, we plot the exponent of $\cliquedet{k, \lfloor k/C\rfloor}$ against the upper bound obtained from Theorem~\ref{thm:det_mult_upper_bound} (without the $o_\ell(1)$ factor). Figure~\ref{fig:detection_upper_bound} shows that the estimates given by Theorem~\ref{thm:det_mult_upper_bound} are actually quite close to the actual exponents, and the values  indeed converge to our bound. 



\vspacebeforesection
\section{Mitigation with Region-Based Queries}
\label{sec:mitigation}

%
Countering Sybil attacks in an open, decentralized system is challenging.
Traditionally, this problem is solved by binding identities to valuable resources (\eg using Proof of Work~\cite{dwork92proofofwork,baumgart2007s}), certificate authorities~\cite{castro_dht}, reputation systems~\cite{sybillimit, sybilguard, danezis2005sybil, whanau}, or diversifying the IP addresses of the peers of each node~\cite{total_eclipse}.
Proof of Work and IP address restrictions are not sufficient as our attack only requires a few Sybil peers ($e \approx 45$): these measures would only slightly increase the cost of the attack. On the other hand, certificate authorities and reputation systems hamper the decentralization and open participation model of IPFS.
%
Simply increasing the value of the number of closest peers contacted in a DHT query (currently $k=20$) also does not solve the problem as the attacker only needs to generate more Sybil peers to match the new number.
Another naive idea is that the providers modify the content by one bit to modify its CID. However, the new CID must be then advertised to potential downloaders to make the content publicly accessible. The attacker can then simply “follow” the new CIDs and continuously censor the content.
%
Further, modifying the content is unsuitable for immutable Web3.0 content (e.g., NFTs and DIDs) whose hash is already published on a blockchain.

The fundamental problem in countering the content censorship attack lies in the inability to classify \resolvers as honest or malicious. When a \downloader receives no \provider record or an inactive provider record from a \resolver (\ie, it is unable to find the referenced provider or the provider does not hold the content), this can be due to several reasons. For instance, the \resolver was offline, the record used to be correct but the \provider had since left, or there was a network failure.
%
%
As a result, the \downloader cannot draw any conclusions on the \resolver's correctness based on the received results, eliminating any attempts to gradually filter out malicious nodes by local scoring systems.
%


\input{img/mitigation-illustration.tex}

\para{Main idea}
The core observation behind our approach is that, while an attacker can spawn additional Sybil identities, it has no way of removing the honest ones from the network. As long as the \provider can send its provider record to the initial honest \resolvers, and the \downloader can communicate with these \resolvers, the censorship attack will be mitigated. To maintain communication with the initial honest \resolvers even during an attack, we propose \emph{region-based} DHT queries. Rather than communicating with the $k=20$ closest peers to a CID, that an attacker can easily control, we communicate with all the nodes in the hash space region that $k=20$ uniformly distributed peer IDs should cover (\Cref{fig:mitigation}).
The size of this region is calculated using the network size estimate and using the assumption that honest peer IDs are distributed uniformly over the key space.
%
Such an approach ensures that regardless of the number of Sybil nodes placed by an attacker, the \provider can store provider records on $\approx 20$ honest \resolvers, and the \downloader also communicates with $\approx 20$ honest \resolvers to reliably retrieve the correct provider records.
%
To prevent additional overhead when there is no attack, we run the region-based queries only when an attack is detected using the detection mechanism that we detailed in \Cref{sec:detection}.

%
%
%


\begin{algorithm}[t]
    \caption{Function to find all peers with a Common Prefix Length (CPL) $\geq \algvar{minCPL}$ with $\algvar{key}$} 
    \label{alg:region_queries}
    \begin{algorithmic}[1]
    \Procedure{FindByCPL}{$\algvar{key}, \algvar{minCPL}$}
        \State $\algvar{set} \gets \Call{GetClosestPeers}{\algvar{key}}$
        \State $\algvar{CPL} \gets \operatorname{minCommonPrefixLength}(\algvar{set}, \algvar{key})$
        \While{$\algvar{CPL} \geq \algvar{minCPL}$}
            \State $\algvar{qkey} \gets \algvar{key}[\mathbin{:} \algvar{CPL}] \mathbin\Vert \overline{\algvar{key}[\algvar{CPL}]} \mathbin\Vert \algvar{key}[\algvar{CPL}+1 \mathbin{:}]$
            %
            %
            \State $\algvar{set} \gets \algvar{set} \cup \Call{FindByCPL}{\algvar{qkey}, \algvar{CPL}+1}$
            \State $\algvar{CPL} \gets \algvar{CPL} - 1$
        \EndWhile
        \State $\operatorname{removeItemsWithPrefixLessThan}(\algvar{set}, \algvar{minCPL})$
        \State \Return $\algvar{set}$
        \EndProcedure
        \end{algorithmic} 
\end{algorithm}

\para{Algorithm}
The region-based query algorithm is described in \Cref{alg:region_queries}, with a sample execution in~\Cref{fig:region-based-illustration}.
The goal of this algorithm is to find all peer IDs that share a common prefix of at least $\algvar{minCPL}$ bits with $\key$.
Note that any two keys $\algvar{k}_1, \algvar{k}_2$ have a common prefix length (CPL) of at least $l$ iff the XOR distance between $\algvar{k}_1$ and $\algvar{k}_2$ is less than $2^{256-l}$.
Therefore, the common prefix requirement specifies a region of the key space with a distance $2^{256-\algvar{minCPL}}$ from $\key$.
%
To keep our mitigation compatible with the current version of \texttt{libp2p} DHT nodes, we use the same RPCs that the DHT nodes currently use.
Therefore, we build the algorithm using only calls to $\Call{GetClosestPeers}{\key}$ which obtains the 20 peer IDs that are the closest to $\key$, which is already available in \texttt{go-libp2p-kad-dht}~\cite{libp2p_github_get_closest_peers}.  We start with this primitive and then compute the common prefix length shared by $\key$ and all of its 20 closest peer IDs, which we note as $\CPL$.

Since we have found at least one peer ID with a common prefix length $\CPL$, we must have found all peer IDs with common prefix length $\geq \CPL+1$, as the latter are closer (in XOR distance) to $\key$ than the former (see step 0 in \Cref{fig:region-based-illustration}).
In the next step, we would like to find all peer IDs with common prefix $\geq \CPL$ with $\key$. 
Since we have already found all peer IDs with the prefix $\key[\mathbin{:} \CPL+1]$ (\ie, the first $\CPL+1$ bits match $\key$),
we only need to find all peers IDs with the prefix $\key[\mathbin{:} \CPL] \mathbin\Vert \overline{\key[\CPL]}$ (\ie the first $\CPL$ bits match $\key$ and the $(\CPL+1)$-th bit is different).
This is done recursively using our algorithm.
%
This step is repeated until all peer IDs with common prefix length $\geq \algvar{minCPL}$ with $\key$ have been found.

\input{img/region-based-illustration.tex}

While using this region-based query algorithm, we choose the value of $\algvar{minCPL}$ such that a region of the key space with common prefix length at least $\algvar{minCPL}$ with $\key=\cid$ contains at least $k=20$ honest peer IDs with high probability.
Suppose that there are a total of $N$ peer IDs in the DHT, distributed uniformly across the hash space. A common prefix length of at least $\algvar{minCPL}$ corresponds to a XOR distance $<2^{256 - \algvar{minCPL}}$. Then, the expected number of peer IDs in this region is $2^{\algvar{minCPL}} \times N$.
By setting $\algvar{minCPL} = \lceil \log_2\left(\frac{N}{k}\right) \rceil$, we have a region that contains $k$ honest peer IDs on average.
Given an estimate $\hat{N}$ of the network size (as described in \cref{sec:netsize}), we substitute $\hat{N}$ for $N$ to calculate the region size.
By a simple probabilistic bound, we can also extend this region to contain $k$ honest peer IDs with high probability.

\para{Cost Analysis}
By default, both \providers and \downloaders do one DHT lookup (using $\Call{GetClosestPeers}{\key}$) to obtain the list of $k=20$ closest peer IDs to $\key$. 
%
%
%
%
%
%
%
When a \provider or \downloader uses a region-based query, it does multiple lookups using $\Call{GetClosestPeers}{\cdot}$. 
The number of lookups required increases sub-linearly in the number of Sybil identities placed by an attacker (shown experimentally in \Cref{fig:mit-lookups-sybils}).
%
Importantly, operating a Sybil identity requires participating in the DHT routing and responding to keep-alive messages.
%
As a result, the cost for the attacker increases linearly with the number of Sybil identities.
%
When the target CID is not under attack, using the region-based query would still use more than one $\Call{GetClosestPeers}{\cdot}$ lookups, because honest peer IDs are distributed randomly, and therefore the chosen region might contain more than $20$ peer IDs.
To avoid this overhead when there is no attack, we run the region-based lookup only when the detection mechanism (\Cref{sec:detection}) detects an attack, and use the default lookup otherwise.
We evaluate the \provider's and \downloader's cost of the region-based queries (number of lookups and latency) and the attacker's cost in \Cref{sec:evaluation}.

\para{Correctness Analysis}
We prove that \Cref{alg:region_queries} indeed finds all peer IDs with a common prefix length of at least $\algvar{minCPL}$ with $\key$.
\begin{theorem}
    \label{thm:region-based-routing-proof}
    Assuming that $\Call{GetClosestPeers}{\key}$ returns the $20$ closest peer IDs to $\key$, $\Call{FindByCPL}{\algvar{key}, \algvar{minCPL}}$ returns all peer IDs with a common prefix of at least $\algvar{minCPL}$ bits with $\key$.
\end{theorem}
\begin{proof}
    We prove this claim through induction.
    For the base case, if there are $< 20$ peer IDs with a common prefix length of at least $\algvar{minCPL}$ with $\key$, then $\Call{GetClosestPeers}{\key}$ must return at least one peer with a common prefix length $ < \algvar{minCPL}$ with $\key$. Therefore, $\CPL < \algvar{minCPL}$, hence the function returns all peer IDs that have a common prefix length of at least $\algvar{minCPL}$.

    Otherwise, $\CPL \geq \algvar{minCPL}$. Since we have found at least one peer with a common prefix length $\CPL$, we have found all peers with a common prefix length $\geq \CPL + 1$, that is all peers with the prefix $\key[:\CPL+1]$. Thus, we create a new key $\algvar{qkey}$ which has the prefix $\key[:\CPL] \mathbin\Vert \overline{\key[\CPL]}$. By induction, we assume that $\Call{findByCPL}{\algvar{qkey}, \CPL+1}$ returns all peers with prefix $\key[\mathbin{:} \CPL] \mathbin\Vert \overline{\key[\CPL]}$. Together, we now have all peers with the prefix $\key[\mathbin{:} \CPL]$. After subtracting $1$ from $\CPL$, we maintain the invariant that we have found all peers with prefix $\key[\mathbin{:}\CPL+1]$. If the loop doesn’t quit, then we continue to find peers with one more bit in the common prefix in every iteration. If the loop quits, this means that $\CPL + 1 \leq \algvar{minCPL}$, therefore we have found all peers with common prefix length of at least $\algvar{minCPL}$ as promised. 
\end{proof}

Even if $\Call{GetClosestPeers}{\key}$ does not return \textit{all} of the 20 closest peer IDs to $\key$, the algorithm will still terminate (because $\CPL$ decreases at every iteration) but may not find all the peers with a common prefix length of at least $\algvar{minCPL}$.
Since we restrict ourselves to build the region-based lookup using $\Call{GetClosestPeers}{\key}$, our method is accurate only in the cases when $\Call{GetClosestPeers}{\key}$ is accurate.

%
%

%
    
%

%

\section{Evaluation} \label{sec:evaluation}

\begin{table*}[tbp]
\centering
\small
\begin{tabular}{cccccccccc}
\toprule
& \multicolumn{3}{c}{\msr} & \multicolumn{3}{c}{\negc} & \multicolumn{3}{c}{\wsj} \\
& Acc. & F1 & wF1 & Acc. & F1 & wF1 & Acc. & F1 & wF1 \\ \cmidrule(lr){2-4} \cmidrule(lr){5-7} \cmidrule(lr){8-10} 
\udel & 66.86 & 56.76 & 64.3 & \textbf{80.80} & 55.45 & 77.9 & 63.74 & 64.23 & 63.2 \\
\icsi & \underline{71.19} & 64.73 & 70.4 & 80.36 & 64.53 & \underline{78.6} & 64.62 & 64.15 & 63.4 \\
\cnts & 68.59 & 61.39 & 67.2 & 78.68 & 61.62 & 76.8 & 64.31 & 64.59 & 64.4 \\
\osu & 68.02 & 60.28 & 66.6 & 79.24 & 57.04 & 76.5 & 69.20 & 69.63 & 68.9 \\
\isg & 67.05 & 58.83 & 65.3 & 77.34 & 59.52 & 75.6 & 69.15 & 69.35 & 69.2 \\ \midrule
\bert & \textbf{71.68} & \underline{66.70} & \textbf{71.4} & 77.79 & \underline{72.87} & 77.7 & \underline{80.95} & \underline{80.93} & \underline{80.9} \\
\roberta & 70.91 & \textbf{67.53} & \underline{70.7} & \textbf{80.80} & \textbf{77.29} & \textbf{80.7} & \textbf{82.61} & \textbf{82.70} & \textbf{82.6} \\ \midrule
Average & 69.19 & 62.32 & 67.99 & 79.29 & 64.05 & 77.69 & 70.65 & 70.80 & 70.37 \\
\bottomrule
\end{tabular}
\caption{\label{tab:performance} Overall accuracy (Acc.), macro-averaged F1 (F1), and weighted-macro F1 (wF1) scores of the algorithms depicted in Section~\ref{sec:algorithm}. For instance, \msr-\udel refers to a C5.0 classifier trained on the \msr~corpus, using the feature set mentioned in \citet{greenbacker-mccoy-2009-udel}.}
%Its Acc., F1 and wF1 of this model are 66.86, 56.76, and 64.3, respectively.}
\end{table*}


In this section, we introduce the evaluation protocol and report the performance of the models.

\subsection{Implementation Details} \label{sec:implementation}

For \bert and \roberta, we used \textit{bert-base-cased} and \textit{roberta-base}, both from Hugging Face. For fine-tuning, we set the batch size to 16, the learning rate to 1e-3, the dropout rate to 0.5, and the size of the output layer to 256. We ran each model for 20 epochs and used the one that achieved the highest F1 score on the development set. The implementation details of the classic ML-based models can be found in Appendix~\ref{sec:appendixML}.

\subsection{Evaluation Protocol} \label{sec:protocol}

The main evaluation metric in the GREC-MSR shared tasks was accuracy. 
In addition to accuracy, we also report macro-F1 and weighted-macro F1. We argue that different metrics evaluate algorithms from different perspectives and provide us with different meaningful insights. 
For pragmatic tasks like REG, it makes sense to ask how well an algorithm performs on naturally distributed data which is often imbalanced. For these cases, reporting accuracy and weighted F1 are logical. 
Furthermore, analogous to other classification tasks, minority categories should not be overlooked. Take as an example the class \emph{description} in the \negc corpus, which occurs only 4\%. If a model fails to produce this class, the produced document might sound unnatural. Therefore, it is important to ensure that an algorithm is not over- or under-generating certain classes. Looking into accuracy and macro-F1 together provides insights into such cases.

\subsection{Performance of the Models}\label{subsec:overallacc}

The overall accuracy of the models, their macro F1, and their weighted-macro F1 are presented in Table \ref{tab:performance}. 
We also present the ranking of the models based on these scores in Appendix~\ref{sec:app_rank}. 


\paragraph{PLM-based Models.} The best-performing models across all corpora and metrics are PLM-based models.  In six out of nine rankings, \bert and \roberta are ranked as the top two models. The sole exception is \negc, where \bert is the second worst model. The benefit of using PLMs is the largest on the \wsj corpus. For example, \roberta improves the macro F1 score from 69.63 (i.e., the performance of the best ML-based model) to 82.70.


\paragraph{ML-based Models.} In contrast to the robust performance of the PLM models, the performance of the classic ML models is more corpus-dependent. In the case of \msr and \negc, \icsi is the best-performing model, while in the case of \wsj, it is at the bottom section of the rankings. Another interesting observation is the performance of the \udel models. In terms of accuracy, \udel has the highest performance in \negc, while it has the lowest performance in both \msr and \wsj. In terms of macro-F1 rankings, the \negc \udel model dropped from first to last place, whereas \bert improved from penultimate place to second place. In general, our ML models yielded lower scores than the original models used in the GREC study \citep{belz2009generating}. This could be attributed to a variety of factors, including differences in feature engineering and model parameters.

\paragraph{Comparing Different Metrics.} 

Upon comparing average scores across the three metrics, we observe that for \msr and \negc, PLMs are clear winners only when macro-F1 is the metric in question. However, for \wsj, PLMs are winners on all three metrics. This may be because the distribution of categories in \wsj is much more balanced than in the other two corpora.

\section{Related Work}
\label{appsec: related work}
Bayesian causal discovery literature has primarily focused on inference in linear models with closed-form posteriors or marginalized parameters. Early works considered sampling directed acyclic graphs (DAGs) for discrete~\cite{cooper1992bayesian, madigan1995bayesian, heckerman2006bayesian} and Gaussian random variables~\cite{friedman2003being, tong2001active} using Markov chain Monte Carlo (MCMC) in the DAG space. However, these approaches exhibit slow mixing and convergence~\cite{eaton2012bayesian,grzegorczyk2008improving}, often requiring restrictions on number of parents~\cite{kuipers2017partition}. %Alternative exact dynamic programming methods are limited to small settings~\cite{koivisto2012advances}. 

Recent advances in variational inference~\cite{zhang2018advances} have facilitated graph inference in DAG space, with gradient-based methods employing the NOTEARS DAG penalty \cite{zheng2018dags}.\cite{annadani2021variational} samples DAGs from autoregressive adjacency matrix distributions, while \cite{lorch2021dibs} utilizes Stein variational approach \cite{liu2016stein} for DAGs and causal model parameters. \cite{cundy2021bcd} proposed a variational inference framework on node orderings using the gumbel-sinkhorn gradient estimator \cite{mena2018learning}. \cite{deleu2022bayesian,nishikawa2022bayesian} employ the GFlowNet framework \cite{bengio2021gflownet} for inferring the DAG posterior. Most methods, except\cite{lorch2021dibs} are restricted to linear models, while \cite{lorch2021dibs} has high computational costs and lacks DAG generation guarantees compared to our method.
% at least quadratic scaling complexity, both with respect to the number of nodes (due to the DAG penalty) as well as number of posterior samples. Our proposed approach instead has linear complexity with respect to number of posterior samples and does not require any additional DAG penalty.     

In contrast, \emph{quasi-Bayesian} methods, such as DAG bootstrap \cite{friedman2013data}, demonstrate competitive performance. DAG bootstrap resamples data and estimates a single DAG using PC \cite{spirtes2000causation}, GES \cite{chickering2002optimal}, or similar algorithms, weighting the obtained DAGs by their unnormalized posterior probabilities. Recent neural network-based works employ variational inference to learn DAG distributions and point estimates for nonlinear model parameters \cite{charpentier2022differentiable,geffner2022deep}.

\section{Discussion}
\label{sec: discussion}
\kmsdelete{In this work} We study \kmsreplace{Fairness-Aware PAC learning}{Fair-ERM} in the malicious noise model, and  in some cases allow 
the learner to maintain optimal overall accuracy despite the signal in Group $B$ being almost entirely washed out.
%when we allow learners to use the
%$\PQ$ randomized expansion of the hypothesis class $\mathcal{H}$
In particular we show that different fairness constraints have fundamentally different behavior in the presence of Malicious Noise, in terms of the amount of accuracy loss that a given level of Malicious Noise could cause a fairness-constrained learner to incur. 
The key to achieving our results, which are more optimistic than those in \cite{lampert}, is allowing for improper learners using the (P,Q)-randomized expansions of the given class $\mathcal{H}$.
%We \kmsreplace{present a picture of the}{prove upper and lower bounds on}
%accuracy loss for a range of fairness notions, given \kmsreplace{this simple randomization step.}{learning over $\PQ$.
%In general our results indicate Fair-ERM (given learning over $\PQ$) is more robust than claimed in \cite{lampert}.
The type of smoothness we create by using $\PQ$ seems to be a natural property that is likely shared by many natural hypothesis classes.

Fairness notions are motivated as a response to learned disparities when there is \kmsdelete{data corruption or} systemic error affecting \kmsdelete{the data for}
one group. 
Fairness notions are supposed to mitigate this by ruling out classifiers that have worse performance on a sub-group. 
This can peg both classifiers at a lower level of performance \kmsdelete{(e.g that the lower subgroup)} in order to \emph{motivate} \cite{hardt16} improving the data collection or labelling process to obtain more reliable performance. 
%So in \kmsreplace{some}{a} sense, sensitivity of the fairness notion to poor sub-group performance caused by malicious noise is the \textit{point} of fairness constraints! 
However, it also desirable that fairness constraints perform gracefully when subject to Malicious Noise because fairness constraints will be used in contexts where the data is unreliable and noisy and this might not be known to the learner.
This tension, exposed by our work, motivates 
%a revisiting of fairness notions from first principles approach and trying to axiomatize the 
%desired properties of a fairness intervention a la cryptography and privacy. \footnote{Work in multi-calibration \cite{multicalib} is a viable direction for this problem but it is unclear how 
%that and related notions behave with unreliable data. }
on going work studying the sensitivity level of fairness constraints. 
%If we we are to take a view, if a classifier is deployed 


\section{Conclusion and Future Work}
In this work, I design corruption-robust algorithms for the Lipschitz contextual search problem. I present the \emph{agnostic checking} technique and demonstrate its effectiveness in designing corruption-robust algorithms. There are several open problems for future research. First, in the algorithm I propose for pricing loss, the schedule for agnostic checks is fixed upfront. Can the learner design an adaptive checking schedule for the pricing loss? Second, this work assumes the learner has knowledge of the Lipschitz constant $L$. Can the learner design efficient no-regret algorithms without knowledge of $L$? 

\section*{Acknowledgements}
This work was partly done when Srivatsan Sridhar was consulting for Protocol Labs. At Stanford University, Srivatsan Sridhar's research is funded by a gift from the Ethereum Foundation.

\bibliographystyle{IEEEtranS}
\bibliography{refs}

%


\end{document}

