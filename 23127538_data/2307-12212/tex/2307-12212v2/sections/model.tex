\vspacebeforesection
\section{Threat Model}\label{sec:model}

We assume $N$ DHT nodes participating in the IPFS network. Multiple nodes may share the same IP address (due to NAT or being hosted by the same physical machine)~\cite{marcus2018low}. However, two nodes cannot share the same ID.

We assume the presence of malicious actors in the network that may refuse to store valid provider records and distribute these to honest participants.
Malicious actors can spawn multiple virtual nodes within one physical machine, operate multiple physical machines, and coordinate their actions.
We assume that no honest node is fully eclipsed by malicious ones, \ie, each honest node has \emph{at least} one honest peer and the DHT routing allows honest nodes to reach any key and discover other honest peers.
IPFS already implements multiple mechanisms preventing eclipse attacks at the DHT level~\cite{total_eclipse}.

An attacker running Sybil nodes interfaces with the network using regular IPFS operations. We do not rely on bugs present in the operating system or any other components not related to the P2P network node implementation under attack. Any flaw in the P2P system protocols, however, may be exploited as these are considered part of the attack target. Non-Sybil DHT nodes (ones that are not spawned by the malicious actor) are assumed to be configured and operated as intended. Thus, importantly, the attacker only controls Sybil nodes that they create but does not corrupt or bring down any other nodes.

%
%

The goal of the attacker is to prevent \downloaders from obtaining \provider records for a target CID. This leads to \emph{content censorship} as the \downloader cannot find a \provider to obtain the content from (recall that \bitswap is only a cache for popular content while the DHT is required for reliable content discovery). We measure the attack effectiveness $\aeff$ as the ratio of unsuccessful $\Call{FindProviders}{\cid}$ queries to the total number of queries for existing content $\cid$ issued by honest \downloaders. A query is unsuccessful if it does not return any honest provider record. The attack effectiveness may vary depending on the placement of the CID and the \downloader issuing the request in the hash space. We thus always consider $\aeff$ as an average for multiple CIDs and multiple \downloaders, both uniformly spread across the hash space.

The goal of honest participants is to detect the attack and mitigate its effects, \ie to enable \downloaders to discover valid provider records and later fetch the content despite the actions of the attacker.

To assess the effectiveness of our \emph{detection} mechanism, we use the false positive $f_p$ and false negative $f_n$ rates. The false positive rate $f_p$ is the proportion of erroneous detections (\ie when there was no attack). The false negative rate $f_n$ is the proportion of attacks that are not detected. A detection leads to a mitigation action. A false positive leads, therefore, solely to additional overhead, while a false negative leads to effective censorship of content. Henceforth, we favor minimizing the false negative rate $f_n$.

The mitigation effectiveness $\meff$ is the ratio of the number of successful $\Call{FindProviders}{\cid}$ queries to the total number of queries issued by honest \downloaders when the target, existing $\cid$ is under attack and the mitigation mechanism is used.
A successful query is defined as one that returns at least one honest provider record. Similarly to the attack effectiveness, we report $\meff$ as an average for queries issued for multiple CIDs by multiple downloaders uniformly spread across the hash space. 
%

%



\begin{table}[t]
    %
    \footnotesize
    \newcolumntype{E}{>{\raggedright\arraybackslash} m{0.125\linewidth} }
    \newcolumntype{F}{>{\raggedright\arraybackslash} m{0.775\linewidth} }
    \renewcommand{\arraystretch}{1.2}
    
    %
    \begin{tabular}{EF}
    \toprule
    \multicolumn{2}{l}{\textbf{General parameters}} \\
    \midrule
    $N$ & Network size (number of nodes) \\
    $k$ & Bucket size, \resolvers per CID, and number of closest peers obtained in a $\Call{GetClosestPeers}{\cid}$ call (currently, $k=20$ in \texttt{libp2p}/IPFS) 
    %
     \\
    \midrule
    
    \multicolumn{2}{l}{\textbf{Attack general parameters}} \\
    \midrule
    $\aeff$ &  Effectiveness of the attack $[\%]$ \\
    $\numSybils$ & Number of Sybil nodes \\
    $\numSybils(\aeff)$ & Number of Sybil nodes necessary to perform the attack with effectiveness $\aeff$. \\
    \midrule
    
    \multicolumn{2}{l}{\textbf{Attack costs}} \\
    \midrule
    $s(\numSybils)$ & Brute-force attempts necessary to generate $\numSybils$ Sybil identities that are the closest to a target CID.\\
    $\cgen$ & Cost of generating $s(\numSybils)$ Sybil identities $[\$]$\\
    $\coper$ & Cost per unit time of operating $\numSybils$ Sybil nodes to attack a single CID $[{\$}/{s}]$\\
    $\catt$ & Total cost of attacking a single CID $[\$]$\\
    \midrule
    
    \multicolumn{2}{l}{\textbf{Attack performance}} \\
    \midrule
    $\twarmup$ & Warmup time during which the $\numSybils$ Sybil nodes need to be run but the attack is not yet fully effective $[s]$\\
    $\teff$ &Time after $\twarmup$ during which the attack remains fully effective. The attack maintains its effect only as long as the $\numSybils$ Sybil nodes are present in the network $[s]$. \\
    \midrule
    
    \multicolumn{2}{l}{\textbf{Detection and mitigation performance}} \\
    \midrule
    $\threshold$ & Threshold for the detection mechanism \\
    $f_n$ & Detection false negative rate $[\%]$ \\
    $f_p$ & Detection false positive rate $[\%]$ \\
    $\meff$ & Effectiveness of the mitigation $[\%]$ \\
    \bottomrule
    
    \end{tabular}
    %
    \caption{Parameters and characterization of the attack, detection, and mitigation.\vspaceaftercaption}
    \label{tab:notation}
\end{table}



%
%
%
%
%

%

%

%

%

%

%

%

%

%
%

%
%
%
%
%