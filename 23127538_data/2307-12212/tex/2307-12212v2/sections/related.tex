\vspacebeforesection
\section{Related Work}
\label{sec:related}

In this section, we review previous work focusing on solving the problem of attacks based on Sybil identities in decentralized systems and their limitations. For a more complete view of the Sybil attack and countermeasures, we refer the readers to a survey by Urdaneta \etal~\cite{survey11}.

CFS~\cite{dabek2001wide} is a storage system built over the early Chord DHT~\cite{stoica2003chord}. CFS uses node ID authentication to prevent a node from taking a specific position in the DHT ring.
%
CFS clients check that the node ID is the result of the hash of its IP address, plus a number from a small range, \eg, 1 to 10. However, this solution is less effective when an attacker has access to a large number of IP addresses (\eg using cloud providers) and it is incompatible with a large number of peers placed behind a NAT, which is the case in IPFS~\cite{trautwein2022design}. S-Chord~\cite{byzantine_chord} is an extension of Chord that can provide routing guarantees despite the presence of a number of Byzantine nodes in the network but it increases the number of messages and latency for routing by a factor logarithmic in the number of nodes. S/Kademlia~\cite{baumgart2007s} proposes Proof-of-Work (PoW) mechanisms to rate-limit the generation of new peer IDs. However, while PoW slows down the attacker, it does not fully mitigate the problem, makes the system less sustainable, and is problematic for constrained devices.

Some mechanisms make additional assumptions on trusted certificate authorities (CAs) to sign peer IDs~\cite{castro_dht} or use social trust networks~\cite{sybillimit, sybilguard, danezis2005sybil, whanau} to detect or prevent Sybil attacks.
Even though most deployed systems do rely on hardcoded bootstrap nodes, relying on CAs to control and certify all memberships would be considered incompatible with the open and decentralized environment of IPFS. 
%
%
%

Awerbuch and Scheideler~\cite{awerbuch2009} propose that the peer IDs of all honest peers in a DHT be rotated whenever a new peer joins. This can prevent an attacker's peers from concentrating in one region of the key space. Unfortunately, this solution is particularly expensive in a dynamic network where nodes constantly join and leave the system.

Cholez \etal~\cite{cholez2010detection} introduce a Sybil detection mechanism based on KL-divergence followed by removing suspected peers from the set of $k$ closest peers.
We adopt their detection mechanism but do not remove any peers.
Since in IPFS, the CID is the hash of the content, Sybil peers cannot cause a downloader to accept incorrect content. Therefore, removing Sybil peers is not required. Instead, our mitigation ensures that providers and downloaders continue to contact enough honest peers.
%

Recently, Protocol Labs introduced network indexers~\cite{ipfsindexer} allowing to resolve a CID to a list of providers in Filecoin~\cite{fisch2018scaling}. While the usage of cloud infrastructure makes the system highly efficient and resistant to Sybil attacks, the indexer is fully centralized introducing the risks of censorship and can constitute a single point of failure.

\para{Eclipse attacks}
Multiple attacks based on node \emph{eclipsing} target decentralized systems.
An attacker attempts to control all the neighbors of a specific target node in the overlay.
This differs from the content censorship attack we discuss in this paper, which targets a specific entry of a distributed directory.

Eclipse attacks are documented for Bitcoin~\cite{heilman2015eclipse,saad2021syncattack} or Ethereum~\cite{henningsen2019eclipsing, marcus2018low} allowing to \emph{partition} the blockchain network and prevent a miner from participating fairly.
The recent Gethlighting attack shows this is possible by only eclipsing a subset of a node's neighborhood~\cite{heopartitioning}.
A recent attack targets the IPFS DHT~\cite{henningsen2020mapping, total_eclipse} and allows isolating a single node from the network.
As a result, the IPFS DHT was augmented with table eviction policies and with rules restricting the number of peers with the same IP address in a routing table.
This makes the attack impractical even for a resourceful attacker~\cite{total_eclipse}.
Our censorship attack targets content rather than single nodes and works despite these changes. However, we also build upon this past work as we rely on eclipse resistance for our mitigation techniques.
%

Wang \etal~\cite{wang2008attacking} is an early example of a content censorship attack, targeting the Kad network, an implementation of Kademlia used in the eDonkey~\cite{edonkey} and eMule~\cite{emule} content-sharing networks.
These attacks exploit a vulnerability in the Kad implementation: peers were not authenticated based on their peer IDs, so an attacker could impersonate another peer.
This vulnerability does not exist in the IPFS network where peer IDs are derived by hashing the peer's public key, and where messages are signed with the corresponding secret keys.
Our attack is much simpler than the one of Wang \etal and does not require this vulnerability.

%
%
%

%
%
%
%
%
%


%
%
%
%
%

%
%
%
%
%

%
%
%
    

%

%
%
