\vspacebeforesection
\section{Conclusion}
\label{sec:conclusion}

We presented a successful censorship attack on IPFS. We showed that an attacker can easily make  any content undiscoverable in the network by strategically placing a small number of Sybil identities in the DHT. The effectiveness of the attack was confirmed by removing multiple, specifically crafted content from the live IPFS network. Importantly, our attack has a constant, negligible cost regardless of the popularity of the target content.

The attack has a significant impact on the IPFS network itself as it threatens the core functionality of the platform. However, it also impacts other systems that rely on the availability of content stored on IPFS. This includes thousands of decentralized applications and oracles deployed on various blockchains. Moreover, the DHT flaw that led to the attack is present in other currently deployed DHT-based systems.

We also presented a robust detection technique allowing us to detect the attack in real time without communication overhead and to activate our proposed mitigation mechanisms when necessary. 

Finally, we introduce a practical mitigation technique based on region-based DHT queries. While many others mitigation techniques have been proposed, none of them are practical enough to be deployed in an open decentralized system. Our approach is the first that can be deployed incrementally in a live network without requiring changes to the core DHT protocol.
It also does not require additional components and does not incur significant overhead. Importantly, our mitigation technique prevents the attack without blocking any nodes or using unreliable reputation systems. We believe that our mitigation technique can be easily integrated into other DHT-based systems.
