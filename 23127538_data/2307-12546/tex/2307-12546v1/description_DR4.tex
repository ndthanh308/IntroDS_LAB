\subsection{The 4FGL-DR4 Catalog}
\label{dr4_description}

The catalog is available online\footnote{See \url{https://fermi.gsfc.nasa.gov/ssc/data/access/lat/14yr_catalog/}.}, together with associated products.
It contains 7195 entries (7194 sources, since the Crab Nebula has two entries), among which 546 are new point sources.
The source designation remains \texttt{4FGL JHHMM.m+DDMM}.
The format is the same as DR3.
The detection threshold outside the Galactic plane decreased but remains a little above $1 \times 10^{-12}$ erg cm$^{-2}$ s$^{-1}$ in the 100~MeV to 100~GeV band.
The new sources are collectively similar to the new DR3 sources.
There are 11 with TS $>$ 100. Among those, eight are associated with blazars, and two are the excesses that we added to geometric extended sources in Carina and Cygnus (see \S~\ref{catalog_detection}). The last one (4FGL J0730.5+6720) has a $WISE$ AGN-like counterpart just below the association threshold of 0.8.

The number of new sources (546) is less than the number of sources introduced in DR2 (723) and DR3 (890). The reason why more sources entered the DR3 catalog was probably the software developments introduced on that occasion (priors on the diffuse parameters, lower TS$_{\rm curv}$ threshold for the LP spectral shape) and the fact that we have added more iterations to the detection process.
The reason why we add fewer sources in DR4 than in DR2 is less obvious. Of course the additional exposure is less in relative terms (+17\% instead of +25\%). The software changes (smooth diffuse modulation and priors on curvature) may also play a role.
