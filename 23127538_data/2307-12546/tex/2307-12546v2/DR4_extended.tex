%% Table: Extended sources
\begin{deluxetable*}{llllcl}
\tabletypesize{\scriptsize}
\tablecaption{Extended Sources Entered or Modified in the 4FGL-DR4 Analysis
\label{tbl:extended}}
\tablewidth{0pt}
\tablehead{
\colhead{DR4 Name}&
\colhead{Extended Source}&
\colhead{Origin}&
\colhead{Spatial Form}&
\colhead{Extent [deg]}&
\colhead{Reference}
}

\startdata
J0205.6+6449e & 3C 58 & New & Gaussian & 0.045 & \citet{3C58_Li18} \\
J0822.1$-$4253e & Puppis A & 3FHL & Map & 0.45 & \citet{PuppisA_Mayer22} \\
J1119.0$-$6127e & SNR G292.2$-$0.5 & New & Gaussian & 0.15 & \citet{HGPS_2018} \\
J1925.2+1618e & SNR G51.3+0.1 & New & Gaussian & 0.22 & \citet{G51.3_Araya21} \\
J1954.4+3252e & CTB 80 & New & Disk & 0.65 & \citet{CTB80_Araya21} \\
J2051.0+3049e & Cygnus Loop & 2FGL & Map & 1.65 & \citet{CygnusLoop_Tutone21} \\
\enddata

\tablecomments{~List of updated and new sources that have been modeled as spatially extended. The Origin column gives the name of the \Fermilat catalog in which that extended source was first introduced (with a different template). The Extent column indicates the radius for Disk (flat disk) sources, the 68\% containment radius for Gaussian sources, and an approximate radius for Map (external template) sources.}

\end{deluxetable*}
