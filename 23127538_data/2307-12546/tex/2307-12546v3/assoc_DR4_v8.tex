\section{Associations}
\label{dr4_assocs}

 Since the DR3 release, some changes in associations or classes have been made:

\begin{itemize}
\item \cite{Cor22} found that 4FGL J1702.7$-$5655 is a candidate redback binary system. \cite{Swi22} reported that 4FGL J1408.6$-$2917 is a black-widow candidate. These two sources are both classified as binaries.   A low-mass X-ray  binary was discovered in 4FGL J1120.0$-$2204 \citep{Swi22a};
\item pulsations have been discovered for 17 sources: 16 millisecond pulsars (MSPs), 7 of which were  previously classified as candidates, and one as a globular cluster, NGC 6652)  and one PSR. The list of detected pulsars in DR4 differs from 3PC  in that 14 PSRs and 7 MSPs are missing, while one PSR and 5 MSPs are not in 3PC. 4FGL J1846.4$-$0258  is spatially  coincident with the 3PC  PSR J1846$-$0258 but the association has been discarded on the basis of a $\gamma$-ray SED incompatible with the pulsar class;
\item  12 new MSP candidates and two young pulsar candidates have been associated using the updated West Virginia University\footnote{\url{http://astro.phys.wvu.edu/GalacticMSPs/GalacticMSPs.txt}} or ATNF\footnote{\url{http://www.atnf.csiro.au/research/pulsar/psrcat/}} catalogs; 
\item the nova YZ Ret \citep{Sok22} is now the counterpart of the new source  4FGL J0358.4$-$5446. It was previously associated with  4FGL J0358.5$-$5432, whose counterpart is now the blazar candidate CRATES J035838$-$543404;
\item four AGNs have changed classes in accordance with  4LAC-DR3  \citep{LAT22_4LACDR3}. NGC 6454 is classified as a radio galaxy, while TXS 0159+085, RX J0134.4+2638, and NVSS J121915+365718 are now classified as BL Lac objects (BLL);
\item 9 blazars of unknown types (BCUs) have been reclassified as BLLs following \cite{Rap23,paiano19,Pai20};
\item TXS 1433+205 has been found to be a distant Fanaroff-Riley II radio galaxy \citep{Pal23};
\item four extended sources  (HESS J1804$-$216, HESS J1809$-$193, W 41, SNR G106.3+02.7) with debated classes as either PWN or SNR, or showing composite features,  have been promoted to a new class, SPP (in capital letter in the CLASS1 column of the FITS file). These sources were previously classified as SPP candidates\footnote{These are sources of unknown nature but spatially overlapping with known SNRs or PWNe and thus candidate members of these classes.} (referred to as ``spp" in the FITS file).
\item two associations have a confidence level that places them in the high-confidence sample, while they were either included in the low-confidence one  (4FGL J0153.3$-$1845) or omitted (4FGL J2306.6+0940).  The most probable counterpart of 4FGL J1912.2$-$3636 was spurious and has been removed.
\end{itemize}
\noindent In addition, 6 TeV associations have been added following  new detections reported in TeVCat\footnote{\url{http://tevcat.uchicago.edu/}}.  

Concerning the new sources in DR4, the association procedure  has been performed  by means of the Bayesian and likelihood-ratio (LR) methods along the lines of earlier releases. Associations have been obtained for 236 of the 546 new  point sources\footnote{These sources can be selected by  requiring DataRelease=4 and no entry in ASSOC\_4FGL.}. The Bayesian and the LR methods yield 200 and 139 associations respectively,  103 being in common. The  association fraction of 43\% is roughly consistent with that expected for the low-TS sources making up the DR4 sample, based on earlier results.  

These associations comprise:
\begin{itemize}
\item three pulsars with detected pulsations (one young pulsar PSR J1224$-$6407 and two MSPs, PSR J0737$-$3039A and PSR J1909$-$3744) and one MSP candidate PSR J1008$-$46;
\item  the four novae listed in Table \ref{tbl:extended} plus the bright RS Oph;
\item  one PWN candidate (Kes 75);
\item  one binary star (V918 Sco);
\item  one star-forming region (SH 2-148);
\item 10 SPP candidates;
\item 15 sources of unknown nature (UNK, $|b|<10\arcdeg$ sources solely associated with the LR method from large radio and X-ray surveys);
\item 191 blazars including 25 BLLs, 28 flat-spectrum radio quasars (FSRQs) and 138 BCUs. 
The fraction of BCUs, which was 35\% in the 4FGL initial catalog, has risen to 72\% for the new blazars presented in subsequent releases (DR2 to DR4), due to the lack of available spectroscopic data for these blazar candidates. The new blazars (summed over all classes)  show a photon index distribution markedly different from that seen  in DR1:  while the latter exhibited a broad maximum about 2.2, the new-blazar  distribution peaks around 2.5, indicating the dominance of FSRQ-like sources \cite[see ][]{LAT22_4LACDR3}. This behavior is likely related to variability.  FSRQs show stronger and more frequent flares than BLLs in the LAT energy band, enhancing their detectability with respect to BLLs as livetime accumulates.     
\item 6 radio galaxies (2MASX J03204016+2727485, GB6 J1226+6406, 3C 293, PKS 1603+00, NGC 6061, LEDA 58287), one compact steep spectrum radio source (4C +76.03), one Seyfert (PKS 0000$-$160).
\end{itemize}
Two extended sources (SNR G292.2-0.5 and CTB 80) have been classified as SPPs, one as a SNR (SNR G51.3+0.1) and one as a PWN (3C 58).

We provide low-probability ($0.1<P<0.8$) associations for 43 sources and associations with {\sl Planck} counterparts for 15 others.  Three TeV associations (PKS 1413+135, RS Oph, 1RXS J195815.6$-$301119) have been added based on TeVCat.  

The overall fraction of soft Galactic unassociated sources \citep[SGUs, see][]{LAT22_4FGLDR3} has remained about the same ($\simeq$ 17\%) as in the previous two releases. A difference is observed when considering the Galactic latitude distributions, where the narrow component peaking on the Galactic plane \citep[the ``spike", see Figure 18 in][]{LAT22_4FGLDR3} is somewhat suppressed in DR4. This may indicate that, if SGUs are related to some mismodeled excess in diffuse emission, the spike excess has (at least  partially) been absorbed by the already-reported sources.   
