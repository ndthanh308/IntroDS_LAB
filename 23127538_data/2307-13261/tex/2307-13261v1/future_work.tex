\section{Future Work}
\label{sec:future_work}
In addition to the open problems mentioned throughout the paper, in this section, we discuss several natural directions for future work.


\subsection{Analysis of Probabilistic Greedy Algorithm}
There are several open questions surrounding the \greedy{p} algorithm under the oblivious adversary model:
\begin{itemize}
	\item Is it always optimal for the adversary to give the nodes with highest degree first?
	\item Does the fixed-$n$ competitive ratio (for optimal $p$) increase monotonically with $n$?
    \item What are the fixed-$n$ competitive ratios when $d=1$?
    \item Can we prove an upper bound on the general competitive ratio?
	\item Is there a better algorithm?
\end{itemize}


\subsection{Extensions}
For our bounds on the competitive ratio for $\sigma$-bounded hypercubes, we assume that the online algorithm has access to the value of $\sigma$. How would the results change if the algorithm does not know $\sigma$ in advance?
%
We also assume that the arrangement of hyperrectangles in the plane is given to the algorithm. A more difficult scenario is where only the intersection graph is revealed to the algorithm. It would be interesting to see which results would still hold in this setting.

In this work, we define dominating and non-dominating order with respect to the upper vertices of a set of hyperrectangles. One direction for future work would be to consider dominating/non-dominated order with respect to other points in the hyperrectangles (e.g., the centers or lower vertices).
%
Other input orders could be considered as well, including ordering by distances of the lower/upper vertices or centers of the hyperrectangles from the origin.

Another direction would be to require that all hyperrectangles have integer coordinates. This could be combined with a requirement that all hyperrectangles in the input are contained within the region $[0,c]^d$ for some constant $c \geq 2$.
%
We could also consider a model where, instead of only being able to accept hyperrectangles from the input, the algorithm could at each time step choose to either accept a hyperrectangle or discard a previously selected one.