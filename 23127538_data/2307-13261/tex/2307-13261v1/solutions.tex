\section{Unit Hypercubes}
\label{sec:unit}
In this section, we prove results for unit hypercubes in non-dominated and arbitrary order. The results from Section~\ref{sec:rect} on dominating order apply in this special case as well.


\subsection{Non-Dominated Order}
\subsubsection{Adaptive Offline Adversary}
The naive greedy algorithm has a competitive ratio of at most $2^d-1$.

\begin{proof}
	Note that a unit $d$-cube can be intersected by at most $2^d$ disjoint unit $d$-cubes if the input comes in non-dominated order. Let $A$ be the set of hypercubes in some optimal solution. Let $B$ be the set of hypercubes in the greedy solution. For each hypercube $x_i$ in $B \setminus A$, there are at most $2^d$ hypercubes in $A \setminus B$ that intersect it. If there is one, let $y_i$ be the hypercube that is dominated by $x_i$. Since $y_i$ was not chosen by the greedy algorithm, it must intersect some other hypercube in $B$. Thus, we have the following.
	\begin{align*}
		|A \setminus B| &\leq (2^d - 1) \cdot |B \setminus A| \\
		|A \setminus B| + |A \cap B| &\leq (2^d - 1) \cdot |B \setminus A| + |A \cap B| \\
		|A \setminus B| + |A \cap B| &\leq (2^d - 1) \cdot |B \setminus A| + (2^d - 1) \cdot |A \cap B| \\
		|A \setminus B| + |A \cap B| &\leq (2^d - 1) \cdot ( |B \setminus A| + |A \cap B| ) \\
		|A| &\leq (2^d - 1) \cdot |B|
	\end{align*}
\end{proof}


\subsubsection{Adaptive Online Adversary}
Every online algorithm must have competitive ratio at least $2^d - 1$.

\begin{proof}
	Until the algorithm selects a hypercube, the adversary gives as input disjoint, unit hypercubes. Once the algorithm selects a hypercube $x_i$, the adversary gives as input (in non-dominated order) $2^d - 1$ disjoint, unit hypercubes that intersect $x_i$ but are not dominated by $x_i$.
\end{proof}


\subsubsection{Oblivious Adversary}
\label{sec:unit:non-dominated:oblivious}
\paragraph{Proposed Algorithm}
Our proposed algorithm is ``greedy with probability $p$'': if the newest hypercube in the input does not overlap with previously accepted hypercubes, accept it with probability $p$.
For any $p \in [0,1]$, we will denote this algorithm by \greedy{p}.
%
If the adversary makes none of the input overlap, this gives us a ratio of $1/p$. so the competitive ratio is at least $1/p$ for any $p$.
%
If the adversary repeatedly gives a $d$-cube and then $2^d - 1$ disjoint hypercubes that intersect it (as in the adaptive case), then the ratio is
\[\frac{2^d - 1}{p + (1-p) (2^d - 1) p} = \frac{2^d - 1}{2^d p - (2^d - 1) p^2} \,.\]
Differentiating the denominator, we have $2^d - (2^{d+1} - 2) p$, which has a zero at $p = \frac{2^{d-1}}{2^d - 1}$, giving us a ratio of $\frac{(2^d - 1)^2}{4^{d-1}}$. Thus, the competitive ratio is at least $\frac{(2^d - 1)^2}{4^{d-1}}$ for $n = 2^d$ and $p = \frac{2^{d-1}}{2^d - 1}$. This ratio is approximately 4 for large $d$.

For $n$ hypercubes, there are $2^{T_{n-1}}$ possible intersection graphs (accounts for the ordering as well), where $T_n$ is the $n$th triangular number, so it is likely infeasible to find the optimal strategy by hand.
We implemented a brute force approach~\citep{Advani2023} that tries all possible intersection graphs to find the optimal strategy for an oblivious adversary for each $p \in \set{0.00, 0.01, \dots, 1.00}$.
%
Competitive ratios for various values of $n$ and $d \geq 2$ are shown in Table~\ref{tab:upper_bounds}. For $d=1$ they are only upper bounds. They were calculated by running the code and then manually determining valid arrangements of unit squares in the plane with the corresponding intersection graphs. Arrangements for $n = 3 \dots 5$ are shown in Appendix~\ref{sec:additional_figs}.
%
In addition, with $p=0.5$, for $n=6$ and $n=7$, we have upper bounds of 3.200000 and 3.636364, respectively, on the competitive ratio.
% corresponding `G_str` values:
% n=6: '111110110011000'
% n=7: '111110110011000110000'

\begin{table}[bhpt]
\caption{Competitive ratios for approximately optimal values of $p$ and $d \geq 2$.}
\label{tab:upper_bounds}
\centering
\begin{tabular}{cccc}
    \toprule
    $n$ & $p$ & competitive ratio & runtime (s) \\
    \midrule
    1 & 1.00 & 1.000000 & $<1$ \\
    2 & 1.00 & 1.000000 & $<1$ \\
    3 & 0.75 & 1.777778 & $<1$ \\
    4 & 0.67 & 2.250056 & 1 \\
    5 & 0.56 & 2.651001 & 38 \\
    \bottomrule
\end{tabular}
\end{table}

If we want to marginally improve these results, we can analytically find the optimal $p$ for each intersection graph and then confirm by brute force that the same graph is optimal for that new $p$. For example, for $n=5$, we need to find the $p$ that maximizes the following.
\begin{align*}
	\E[X]
	&= p + (1-p) (p + (1-p) (p \cdot (1 + p \cdot (1 + p) + (1-p) p) + (1-p) (p \cdot (1 + p) + (1-p) p))) \\
	&= p + (1-p) (p + (1-p) (p \cdot (1 + 2p) + (1-p) 2p)) \\
	&= p + (1-p) (p + (1-p) 3p) \\
	&= p + (1-p) (4p - 3p^2) \\
	&= p + 4p - 3p^2 - 4p^2 + 3p^3 \\
	&= 5p - 7p^2 + 3p^3
\end{align*}
Differentiating, we have
\[\frac{d}{dp} \E[X] = 5 - 14p + 9p^2 \,,\]
which has zeros at $\frac{5}{9}$ and $1$. The expectation is maximized at $p = \frac{5}{9}$. The code confirms that the optimal intersection graph for this value of $p$ is the same, so the competitive ratio of \greedy{\frac{5}{9}} for $n=5$ is $\frac{729}{275} \approx 2.650909$.

For general $n>5$ and $d \geq 2$, the adversary's strategy must be at least as good as using the optimal strategy for $n=5$ for each set of 5 hypercubes and then using the optimal strategy for $n \bmod 5$ for the remaining hypercubes. If we fix $p=0.56$, we have an asymptotic lower bound of 2.651001 for the competitive ratio.

One difficulty with this computational approach is that we only get the optimal intersection graph, not the arrangement of the hypercubes in space. This means that we have to manually verify if there even exists such an arrangement of unit $d$-cubes. Equivalently, we have to determine whether the cubicity of the graph is at most $d$. For $d \geq 2$, recognizing graphs of boxicity at most $d$ is NP-complete~\citep{Yannakakis1982, Kratochvil1994}, and the NP-completeness of recognizing graphs of cubicity at most $d$ follows directly. On the other hand, even though this approach does not guarantee the existence of the actual arrangement of unit hypercubes, it does give us an upper bound on the competitive ratio.

Another obstacle is that the expression for the expected value of the solution constructed by \greedy{p} is given by a polynomial of degree $n$. For large $n$, this could lead to huge changes in the value of the expression with very small changes to $p$. This won't invalidate our theoretical results, but it does mean we may miss even better algorithms given by other values of $p$ if we are not very careful in the optimization process. Perhaps there are some theoretical guarantees on how many values of $p$ we have to test before we can be reasonably sure that we have found the maximum? Is there a more practical method for maximizing univariate polynomials over a finite interval? These are two relevant questions for future work.


\paragraph{Lower Bound}
Every online algorithm must have competitive ratio at least $\frac{12}{7}$ ($\approx 1.714286$) asymptotically.

\begin{proof}
	The adversary begins in State $i$. The adversary first gives two disjoint unit hypercubes as input. Then, it (uniformly) randomly ``marks'' one (denote it by $x_{i,1}$) and gives as input two disjoint, unit hypercubes that intersect $x_{i,1}$ but no other hypercubes from the existing input. Next, it randomly marks one of these (denote it by $x_{i,2}$) and gives as input two disjoint, unit hypercubes that intersect $x_{i,1}$ and $x_{i,2}$ but no others. Finally, the adversary switches to State $i+1$ and restarts the process from $j=1$. Once $n$ hypercubes have been given as input (by the end of State $I \coloneqq \lceil n/4 \rceil$), there is an MIS of size $\bigl\lfloor \frac{n}{2} \bigr\rfloor + I$ consisting of all the hypercubes not marked by the adversary.
	
	Consider any online algorithm. For each state $i$ and level $j$, by symmetry, the probability that it selects only $x_{i,j}$ and the probability that it selects only the other hypercube are equal; denote this probability by $p_{i,j}$. Let $p_{\sup} = \sup \set{p_{i,j}}$. Then, in each state $i \ne I$, the expected size of the independent set $X_i$ constructed by the online algorithm can be bounded as follows.
	\begin{align*}
		\E[X_i]
		&\leq 1 + p_{i,1} (1 + p_{i,2}) \\
		&\leq 1 + p_{\sup} (1 + p_{\sup}) \\
		&\leq 1 + \frac{1}{2} \biggl(1 + \frac{1}{2}\biggr) \\
		&= \frac{7}{4}
	\end{align*}
	The expected size of the overall independent set $\bigcup \set{X_i}_{i=1}^{I}$ is then at most $\frac{7}{4} I = \frac{7}{4} \bigl\lceil \frac{n}{4} \bigr\rceil$.
	Finally, the competitive ratio is at least
	\[
	\frac{\bigl\lfloor \frac{n}{2} \bigr\rfloor + \lceil \frac{n}{4} \rceil}{\frac{7}{4} \bigl\lceil \frac{n}{4} \bigr\rceil}
	= \frac{\bigl\lfloor \frac{n}{2} \bigr\rfloor}{\frac{7}{4} \bigl\lceil \frac{n}{4} \bigr\rceil} + \frac{4}{7}
	\,,\]
	and is asymptotically at least $\frac{12}{7}$.
\end{proof}

An example of the adversary strategy from the proof is shown in Figure~\ref{fig:adversary_strategy_unit_1}.

% Figure environment removed


\subsection{Arbitrary Order}
\subsubsection{Adaptive Offline Adversary}
The naive greedy algorithm has a competitive ratio of at most $2^d$. This result was proved independently by \citet{De2021}.

\begin{proof}
	Note that a unit $d$-cube can be intersected by at most $2^d$ disjoint unit $d$-cubes. Let $A$ be the set of hypercubes in some optimal solution. Let $B$ be the set of hypercubes in the greedy solution. For each hypercube in $B \setminus A$, there are at most $2^d$ hypercubes in $A \setminus B$ that intersect it. By construction, every hypercube in $A \setminus B$ must intersect some hypercube in $B \setminus A$ (or it would be in $B$). Thus, we have the following.
	\begin{align*}
		|A \setminus B| &\leq 2^d \cdot |B \setminus A| \\
		|A \setminus B| + |A \cap B| &\leq 2^d \cdot |B \setminus A| + |A \cap B| \\
		|A \setminus B| + |A \cap B| &\leq 2^d \cdot |B \setminus A| + 2^d \cdot |A \cap B| \\
		|A \setminus B| + |A \cap B| &\leq 2^d \cdot ( |B \setminus A| + |A \cap B| ) \\
		|A| &\leq 2^d \cdot |B|
	\end{align*}
\end{proof}


\subsubsection{Adaptive Online Adversary}
Every online algorithm must have competitive ratio at least $2^d$. This result was proved independently by \citet{De2021}.

\begin{proof}
    Until the algorithm selects a hypercube, the adversary gives as input disjoint, unit hypercubes. Once the algorithm selects a hypercube $x_i$, the adversary gives as input $2^d$ disjoint, unit hypercubes that intersect $x_i$.
\end{proof}


\subsubsection{Oblivious Adversary}
\label{sec:unit:arbitrary:oblivious}
\paragraph{Proposed Algorithm}
Again, our proposed algorithm is \greedy{p}.
%
If the adversary makes none of the input overlap, this gives us a ratio of $1/p$. so the competitive ratio is at least $1/p$ for any $p$.
%
If the adversary repeatedly gives a $d$-cube and then $2^d$ disjoint hypercubes that intersect it (as in the adaptive case), then the ratio is
\[\frac{2^d}{p + (1-p) 2^d p} = \frac{2^d}{(2^d + 1)p - 2^d p^2} \,.\]
Differentiating the denominator, we have $2^d + 1 - 2^{d+1} p$, which has a zero at $p = \frac{2^d + 1}{2^{d+1}}$, giving us a ratio of $\frac{4^{d+1}}{(2^d + 1)^2}$. Thus, the competitive ratio is at least $\frac{4^{d+1}}{(2^d + 1)^2}$ for $n = 2^d + 1$ and $p = \frac{2^d + 1}{2^{d+1}}$. This ratio is approximately 4 for large $d$.

The same competitive ratios from Table~\ref{tab:upper_bounds} apply here as well. Again, for $d=1$ they are only upper bounds. In addition, for $n=6$, we have found an explicit arrangement (see Appendix~\ref{sec:additional_figs}), so we can state an optimal $p$ and competitive ratio (see Table~\ref{tab:upper_bound_n6}). For $n=7$, we do not have an explicit arrangement, so the value from the non-dominated case remains just an upper bound.

\begin{table}[bhpt]
\caption{Competitive ratio for $n=6$, approximately optimal value of $p$, and $d \geq 2$.}
\label{tab:upper_bound_n6}
\centering
\begin{tabular}{cccc}
	\toprule
	$n$ & $p$ & competitive ratio & runtime (s) \\
	\midrule
	6 & 0.50 & 3.200000 & 2577 \\
	\bottomrule
\end{tabular}
\end{table}

For general $n>6$ and $d \geq 2$, the adversary's strategy must be at least as good as using the optimal strategy for $n=6$ for each set of 6 hypercubes and then using the optimal strategy for $n \bmod 6$ for the remaining hypercubes. If we fix $p=0.5$, we have an asymptotic lower bound of 3.2 for the competitive ratio.

\citet{Caragiannis2007} prove a result for the similar case of unit disks in arbitrary order with an oblivious adversary that beats the optimal solution for an adaptive adversary. Unfortunately, when adapted to the setting of unit hypercubes, their technique does not improve on the optimal solution for an adaptive adversary.

%proof sketch:
%
%select a random point uniformly from $[-1,1]^2$ and recenter the coordinate system with that as the origin.
%
%algorithm: let $(x,y)$ be the lower vertex of the input square. consider the point $(x \bmod 2, y \bmod 2)$. if this point is contained in the region $[0,1)^2$, then select it as long as it doesn't intersect any already selected squares.
%
%let $\mathcal{D}$ be the input set. let $\mathcal{D}'$ be the squares that passed the algorithm's ``mod check''. the probability for each square of being in $\mathcal{D}'$ is $1/4$. let $A(\cdot)$ be the output of an optimal offline algorithm on a given input. let $B(\cdot)$ be the output of the online algorithm on a given input. we have $\E[|A(\mathcal{D}')|] \geq \E[|A(\mathcal{D}) \cap \mathcal{D}'|] = 1/4 \cdot |A(\mathcal{D})|$. note that the intersection graph of $\mathcal{D}'$ consists of disjoint cliques. in each clique, both the online algorithm and the optimal offline algorithm select exactly one square. therefore, $\E[|B(\mathcal{D}')|] = \E[|A(\mathcal{D}')|]$. this gives us an upper bound of 4 on the competitive ratio.
%
%can extend this argument to $d$ dims and get a $2^d$ upper bound.


\paragraph{Lower Bound}
Every online algorithm must have competitive ratio at least $\frac{32}{15}$ ($\approx 2.133333$) asymptotically.

\begin{proof}
	The adversary begins in State $i$. The adversary first gives two disjoint unit hypercubes as input. Then, it (uniformly) randomly ``marks'' one (denote it by $x_{i,1}$) and gives as input two disjoint, unit hypercubes that intersect $x_{i,1}$ but no other hypercubes from the existing input. Next, it randomly marks one of these (denote it by $x_{i,2}$) and gives as input two disjoint, unit hypercubes that intersect $x_{i,1}$ and $x_{i,2}$ but no others. It continues this process until it has given 6 hypercubes, at the $j$th level marking a hypercube $x_{i,j}$ and giving as input two hypercubes that intersect $x_{i,1}$ through $x_{i,j}$ but no others. Once 6 hypercubes have been given in State $i$, the adversary switches to State $i+1$ and restarts the process from $j=1$. Once $n$ hypercubes have been given as input (by the end of State $I \coloneqq \lceil n/6 \rceil$), there is an MIS of size $\bigl\lfloor \frac{n}{2} \bigr\rfloor + I$ consisting of all the hypercubes not marked by the adversary.
	
	Consider any online algorithm. For each state $i$ and level $j$, by symmetry, the probability that it selects only $x_{i,j}$ and the probability that it selects only the other hypercube are equal; denote this probability by $p_{i,j}$. Let $p_{\sup} = \sup \set{p_{i,j}}$. Then, in each state $i \ne I$, the expected size of the independent set $X_i$ constructed by the online algorithm can be bounded as follows.
	\begin{align*}
		\E[X_i]
		&\leq 1 + p_{i,1} (1 + p_{i,2} (1 + p_{i,3})) \\
		&\leq 1 + p_{\sup} (1 + p_{\sup} (1 + p_{\sup})) \\
		&\leq 1 + \frac{1}{2} \biggl(1 + \frac{1}{2} \biggl(1 + \frac{1}{2}\biggr)\biggr) \\
		&= \frac{15}{8}
	\end{align*}
	The expected size of the overall independent set $\bigcup \set{X_i}_{i=1}^{I}$ is then at most $\frac{15}{8} I = \frac{15}{8} \bigl\lceil \frac{n}{6} \bigr\rceil$.
	Finally, the competitive ratio is at least
	\[
	\frac{\bigl\lfloor \frac{n}{2} \bigr\rfloor + \lceil \frac{n}{6} \rceil}{\frac{15}{8} \bigl\lceil \frac{n}{6} \bigr\rceil}
	= \frac{\bigl\lfloor \frac{n}{2} \bigr\rfloor}{\frac{15}{8} \bigl\lceil \frac{n}{6} \bigr\rceil} + \frac{8}{15}
	\,,\]
	and is asymptotically at least $\frac{32}{15}$.
\end{proof}

An example of the adversary strategy from the proof is shown in Figure~\ref{fig:adversary_strategy_unit_2}.

% Figure environment removed






\section{\texorpdfstring{$\bm{\sigma}$}{σ}-Bounded Hypercubes}
\label{sec:sig-bounded}
For $\sigma \geq 1$, let a $\sigma$-bounded hypercube be a hypercube with side length between 1 and $\sigma$, inclusive. In this section, we prove results for $\sigma$-bounded hypercubes in non-dominated and arbitrary order. We assume that the online algorithm is given the value of $\sigma$ as part of its input. The results from Section~\ref{sec:rect} on dominating order apply in this special case as well.


\subsection{Non-Dominated Order}
\subsubsection{Adaptive Offline Adversary}
The naive greedy algorithm has a competitive ratio of at most $(\lceil\sigma\rceil + 1)^d - \lceil\sigma\rceil^d$.

\begin{proof}
	Let $A$ be the set of hypercubes in some optimal solution. Let $B$ be the set of hypercubes in the greedy solution. For each hypercube $x$ in $B \setminus A$, there are at most $(\lceil\sigma\rceil + 1)^d - \lceil\sigma\rceil^d$ hypercubes in $A \setminus B$ that intersect it and are not dominated by it. Every hypercube in $A \setminus B$ that intersects and is dominated by $x$ must intersect (and not be dominated by) some other hypercube in $B$ (or it would have been chosen by the greedy algorithm). Thus, we have the following.
	\begin{align*}
		|A \setminus B| &\leq \bigl((\lceil\sigma\rceil + 1)^d - \lceil\sigma\rceil^d\bigr) \cdot |B \setminus A| \\
		|A \setminus B| + |A \cap B| &\leq \bigl((\lceil\sigma\rceil + 1)^d - \lceil\sigma\rceil^d\bigr) \cdot |B \setminus A| + |A \cap B| \\
		|A \setminus B| + |A \cap B| &\leq \bigl((\lceil\sigma\rceil + 1)^d - \lceil\sigma\rceil^d\bigr) \cdot |B \setminus A| + \bigl((\lceil\sigma\rceil + 1)^d - \lceil\sigma\rceil^d\bigr) \cdot |A \cap B| \\
		|A \setminus B| + |A \cap B| &\leq \bigl((\lceil\sigma\rceil + 1)^d - \lceil\sigma\rceil^d\bigr) \cdot ( |B \setminus A| + |A \cap B| ) \\
		|A| &\leq \bigl((\lceil\sigma\rceil + 1)^d - \lceil\sigma\rceil^d\bigr) \cdot |B|
	\end{align*}
\end{proof}


\subsubsection{Adaptive Online Adversary}
Every online algorithm must have competitive ratio at least $(\lceil\sigma\rceil + 1)^d - \lceil\sigma\rceil^d$.

\begin{proof}
	Consider any algorithm. Until the algorithm selects a hypercube, the adversary gives disjoint hypercubes with side length $\sigma$. Once the algorithm selects one, the adversary gives as input $(\lceil\sigma\rceil + 1)^d - \lceil\sigma\rceil^d$ disjoint unit hypercubes intersecting that one.
\end{proof}


\subsubsection{Oblivious Adversary}
\label{sec:sig-bounded:non-dominated:oblivious}
These results use techniques based on those of \citet{Caragiannis2007}.


\paragraph{Upper Bound}
For any integer $k \geq 1$, let $b = \sigma^{1/k}$.
Consider the following algorithm. First, it picks an integer $i$ uniformly at random from the interval $[0, k-1]$.
Then it greedily selects each hypercube with side length in the range $[b^i, b^{i+1}]$ and ignores the rest.
This algorithm has a competitive ratio of $(\lceil b \rceil + 1)^d k - \lceil b \rceil^d k$.

\begin{proof}
	Let $A$ be the MIS constructed by an optimal offline algorithm, $B$ be the independent set constructed by the online algorithm, $B'$ be the set of hypercubes in the input with side length in the range $[b^i, b^{i+1}]$ (those not ignored by the online algorithm), and $A_i$ be a MIS of the intersection graph of $B'$.
	
	If we consider only $B'$, then the input consists of (potentially scaled-up) $b$-bounded hypercubes. We know that the greedy algorithm has a competitive ratio of $(\lceil b \rceil + 1)^d - \lceil b \rceil^d$ in this case. So the size of $A_i$ is at most $((\lceil b \rceil + 1)^d - \lceil b \rceil^d) \cdot |B|$. The size of $A_i$ is necessarily at least $|A \cap B'|$, and the expected size of $A \cap B'$ is $\frac{1}{k} |A|$. Thus, the competitive ratio of the full online algorithm is at most $(\lceil b \rceil + 1)^d k - \lceil b \rceil^d k$.
\end{proof}

The next question is how to choose the value of $k$. Notably, if we take $k=1$, we have the deterministic online algorithm used for the adaptive adversary case and the same competitive ratio of $(\lceil\sigma\rceil + 1)^d - \lceil\sigma\rceil^d$. At the other extreme, assuming $\sigma \geq 2$, if we take $k = \lceil \log_2 \sigma \rceil$, we get $\lceil b \rceil = 2$ and a competitive ratio of at most $3^d \lceil \log_2 \sigma \rceil - 2^d \lceil \log_2 \sigma \rceil$. For large $d$, the latter is a significantly better bound. The optimal $k$ is somewhere in the interval $[1, \lceil \log_2 \sigma \rceil]$; determining the exact optimum is an open problem.


\paragraph{Lower Bound}
Every online algorithm must have competitive ratio at least $\frac{\lceil \log_2 \sigma \rceil + 1}{2}$ asymptotically.

\begin{proof}
	The adversary begins in State $i$. The adversary first gives two disjoint hypercubes as input. Then, it (uniformly) randomly ``marks'' one (denote it by $x_{i,1}$) and gives as input two disjoint hypercubes that intersect $x_{i,1}$ but no other hypercubes from the existing input. Next, it randomly marks one of these (denote it by $x_{i,2}$) and gives as input two disjoint hypercubes that intersect $x_{i,1}$ and $x_{i,2}$ but no others. It continues this process until it has given $2\lceil \log_2 \sigma \rceil$ hypercubes, at the $j$th level marking a hypercube $x_{i,j}$ and giving as input two hypercubes that intersect $x_{i,1}$ through $x_{i,j}$ but no others. Once $2\lceil \log_2 \sigma \rceil$ hypercubes have been given in State $i$, the adversary switches to State $i+1$ and restarts the process from $j=1$. Once $n$ hypercubes have been given as input (by the end of State $I \coloneqq \lceil n/(2\lceil \log_2 \sigma \rceil) \rceil$), there is an MIS of size $\bigl\lfloor \frac{n}{2} \bigr\rfloor + I$ consisting of all the hypercubes not marked by the adversary.
	
	Consider any online algorithm. For each state $i$ and level $j$, by symmetry, the probability that it selects only $x_{i,j}$ and the probability that it selects only the other hypercube are equal; denote this probability by $p_{i,j}$. Let $p_{\sup} = \sup \set{p_{i,j}}$. Then, in each state $i$, the expected size of the independent set $X_i$ constructed by the online algorithm can be bounded as follows.
	\begin{align*}
		\E[X_i]
		&\leq 1 + p_{i,1} (1 + p_{i,2} (1 + \ldots + p_{i,\lceil \log_2 \sigma \rceil})\dots)) \\
		&\leq 1 + p_{i,1} (1 + p_{i,2} (1 + \dots)) \\
		&\leq 1 + p_{\sup} (1 + p_{\sup} (1 + \dots)) \\
		&= \frac{1}{1 - p_{\sup}} \\
		&\leq \frac{1}{1 - \frac{1}{2}} \\
		&= 2
	\end{align*}
	The expected size of the overall independent set $\bigcup \set{X_i}_{i=1}^{I}$ is then at most $2I = 2\lceil n/(2\lceil \log_2 \sigma \rceil) \rceil$.
	Finally, the competitive ratio is at least
	\[
	\frac{\bigl\lfloor \frac{n}{2} \bigr\rfloor + \lceil n/(2\lceil \log_2 \sigma \rceil) \rceil}{2\lceil n/(2\lceil \log_2 \sigma \rceil) \rceil}
	= \frac{\bigl\lfloor \frac{n}{2} \bigr\rfloor}{2\lceil n/(2\lceil \log_2 \sigma \rceil) \rceil} + \frac{1}{2}
	\,,\]
	which is asymptotically at least $\frac{\lceil \log_2 \sigma \rceil + 1}{2}$.
\end{proof}


\subsection{Arbitrary Order}
\subsubsection{Adaptive Offline Adversary}
The naive greedy algorithm has a competitive ratio of at most $(\lceil\sigma\rceil + 1)^d$.

\begin{proof}
	Note that a $\sigma$-bounded $d$-cube can be intersected by at most $(\lceil\sigma\rceil + 1)^d$ disjoint $\sigma$-bounded $d$-cubes. Let $A$ be the set of hypercubes in some optimal solution. Let $B$ be the set of hypercubes in the greedy solution. For each hypercube in $B \setminus A$, there are at most $(\lceil\sigma\rceil + 1)^d$ hypercubes in $A \setminus B$ that intersect it. By construction, every hypercube in $A \setminus B$ must intersect some hypercube in $B \setminus A$ (or it would be in $B$). Thus, we have the following.
	\begin{align*}
		|A \setminus B| &\leq (\lceil\sigma\rceil + 1)^d \cdot |B \setminus A| \\
		|A \setminus B| + |A \cap B| &\leq (\lceil\sigma\rceil + 1)^d \cdot |B \setminus A| + |A \cap B| \\
		|A \setminus B| + |A \cap B| &\leq (\lceil\sigma\rceil + 1)^d \cdot |B \setminus A| + (\lceil\sigma\rceil + 1)^d \cdot |A \cap B| \\
		|A \setminus B| + |A \cap B| &\leq (\lceil\sigma\rceil + 1)^d \cdot ( |B \setminus A| + |A \cap B| ) \\
		|A| &\leq (\lceil\sigma\rceil + 1)^d \cdot |B|
	\end{align*}
\end{proof}


\subsubsection{Adaptive Online Adversary}
Every online algorithm must have competitive ratio at least $(\lceil\sigma\rceil + 1)^d$.

\begin{proof}
	Consider any algorithm. Until the algorithm selects a hypercube, the adversary gives disjoint hypercubes with side length $\sigma$. Once the algorithm selects one, the adversary gives as input $(\lceil\sigma\rceil + 1)^d$ disjoint unit hypercubes intersecting that one.
\end{proof}


\subsubsection{Oblivious Adversary}
\label{sec:sig-bounded:arbitrary:oblivious}
These results use techniques based on those of \citet{Caragiannis2007}.


\paragraph{Upper Bound}
For any integer $k \geq 1$, let $b = \sigma^{1/k}$.
Consider the following algorithm. First, it picks an integer $i$ uniformly at random from the interval $[0, k-1]$.
Then it greedily selects each hypercube with side length in the range $[b^i, b^{i+1}]$ and ignores the rest.
This algorithm has a competitive ratio of $(\lceil b \rceil + 1)^d k$.

\begin{proof}
	Let $A$ be the MIS constructed by an optimal offline algorithm, $B$ be the independent set constructed by the online algorithm, $B'$ be the set of hypercubes in the input with side length in the range $[b^i, b^{i+1}]$ (those not ignored by the online algorithm), and $A_i$ be a MIS of the intersection graph of $B'$.
	
	If we consider only $B'$, then the input consists of (potentially scaled-up) $b$-bounded hypercubes. We know that the greedy algorithm has a competitive ratio of $(\lceil b \rceil + 1)^d$ in this case. So the size of $A_i$ is at most $(\lceil b \rceil + 1)^d \cdot |B|$. The size of $A_i$ is necessarily at least $|A \cap B'|$, and the expected size of $A \cap B'$ is $\frac{1}{k} |A|$. Thus, the competitive ratio of the full online algorithm is at most $(\lceil b \rceil + 1)^d k$.
\end{proof}

The next question is how to choose the value of $k$. Notably, if we take $k=1$, we have the deterministic online algorithm used for the adaptive adversary case and the same competitive ratio of $(\lceil\sigma\rceil + 1)^d$. At the other extreme, assuming $\sigma \geq 2$, if we take $k = \lceil \log_2 \sigma \rceil$, we get $\lceil b \rceil = 2$ and a competitive ratio of at most $3^d \lceil \log_2 \sigma \rceil$.
The optimal $k$ is somewhere in the interval $[1, \lceil \log_2 \sigma \rceil]$.

To find a good choice of $k$, we differentiate the upper bound on the competitive ratio with respect to $k$.
For simpler analysis, we use a looser form of the upper bound: $(b + 2)^d k$. We have the following.
\begin{align*}
	\frac{d}{dk} \bigl(\sigma^{1/k} + 2\bigr)^d k
	&= \bigl(\sigma^{1/k} + 2\bigr)^d + d \bigl(\sigma^{1/k} + 2\bigr)^{d-1} \ln(\sigma) \cdot \sigma^{1/k} \bigl(-k^{-2}\bigr) k \\
	&= \bigl(\sigma^{1/k} + 2\bigr)^d - d \bigl(\sigma^{1/k} + 2\bigr)^{d-1} \ln(\sigma) \cdot \sigma^{1/k} k^{-1} \\
	&= \bigl(\sigma^{1/k} + 2\bigr)^{d-1} \bigl(\sigma^{1/k} + 2 - d \ln(\sigma) \cdot \sigma^{1/k} k^{-1}\bigr) \\
	&= \bigl(\sigma^{1/k} + 2\bigr)^{d-1} \bigl(2 + \sigma^{1/k} \bigl(1 - d \ln(\sigma) \cdot k^{-1}\bigr)\bigr)
\end{align*}
For very small $k$, the derivative is negative, and at $k = d \ln \sigma$, the derivative is positive. It changes sign exactly once between the two points at
\[k' = \frac{\ln \sigma}{W_0\Bigl(\frac{2e^{-1/d}}{d}\Bigr) + \frac{1}{d}} \,,\]
where $W_0$ is the principal branch of the Lambert $W$ function (see Appendix~\ref{sec:proofs} for proof). Approximate values of $\Bigl(W_0\Bigl(\frac{2e^{-1/d}}{d}\Bigr) + \frac{1}{d}\Bigr)^{-1}$ in various dimensions $d$ are shown in Table~\ref{tab:k-prime}. For very small $x$, we have $W_0(x) \approx x$; thus, for large $d$, we have
\begin{equation*}
	\biggl(W_0\biggl(\frac{2e^{-1/d}}{d}\biggr) + \frac{1}{d}\biggr)^{-1}
	\approx \biggl(W_0\biggl(\frac{2}{d}\biggr) + \frac{1}{d}\biggr)^{-1}
	\approx \biggl(\frac{2}{d} + \frac{1}{d}\biggr)^{-1}
	= \frac{d}{3} \,,
\end{equation*}
which gives $k' \approx \frac{d}{3} \ln \sigma$. For large $d$, this is greater than $\lceil \log_2 \sigma \rceil$, so the optimal $k$ is likely $\lceil \log_2 \sigma \rceil$.
%
For small $d$, the easiest approach is to evaluate $(\lceil b \rceil + 1)^d k$ for several integers $k$ near $k'$ and select the best one.

\begin{table}[bhpt]
	\caption{Approximate values of $\Bigl(W_0\Bigl(\frac{2e^{-1/d}}{d}\Bigr) + \frac{1}{d}\Bigr)^{-1}$ in various dimensions $d$.}
	\label{tab:k-prime}
	\centering
	\begin{tabular}{cccccccc}
		\toprule
		$d$ & 1 & 2 & 3 & 4 & 10 & 100 & 1000 \\
		\midrule
		value & 0.683501 & 1.10537 & 1.48514 & 1.84821 & 3.92179 & 33.9903 & 333.999 \\
		\bottomrule
	\end{tabular}
\end{table}


\paragraph{Lower Bound}
The lower bound from Section~\ref{sec:sig-bounded:non-dominated:oblivious} applies in this setting as well.






\section{Unit-Volume Hyperrectangles}
\label{sec:unit-vol}
In this section, we prove results for unit-volume hyperrectangles in non-dominated and arbitrary order. The results from Section~\ref{sec:rect} on dominating order apply in this special case as well.


\subsection{Non-Dominated Order}
\subsubsection{Adaptive Online Adversary}
Every online algorithm must have competitive ratio at least $n - 1$.

\begin{proof}
	The adversary can give as input disjoint, dominating, unit-volume hyperrectangles until the algorithm selects one. Then, the adversary can have the remaining input be disjoint, unit-volume hyperrectangles in dominating order that intersect (and are not dominated by) the selected hyperrectangle.
\end{proof}

An example of the adversary strategy from the proof is shown in Figure~\ref{fig:adversary_strategy_unit-vol_1}.

% Figure environment removed


\subsubsection{Oblivious Adversary}
\label{sec:unit-vol:non-dominated:oblivious}
\paragraph{Proposed Algorithm}
All results on \greedy{p} for fixed $n$ from Section~\ref{sec:unit:non-dominated:oblivious} (unit hypercubes) hold here. In addition, we can construct an arrangement for $n=6$ (see Figure~\ref{fig:n6_unit-vol}), so the result for $n=6$ from Section~\ref{sec:unit:arbitrary:oblivious} holds as well.

% Figure environment removed


\paragraph{Lower Bound}
Every online algorithm must have competitive ratio at least $\frac{1}{2} \bigl\lfloor \frac{n}{2} \bigr\rfloor + \frac{1}{2}$.

\begin{proof}
	The adversary first gives two disjoint, unit-volume hyperrectangles as input. Then, it (uniformly) randomly ``marks'' one (denote it by $x_1$) and gives as input two disjoint hyperrectangles that intersect $x_1$ but no other hyperrectangles from the existing input. Next, it randomly marks one of these (denote it by $x_2$) and gives as input two disjoint hyperrectangles that intersect $x_1$ and $x_2$ but no others. It continues this process indefinitely, at the $k$th level marking a hyperrectangle $x_k$ and giving as input two hyperrectangles that intersect $x_1$ through $x_k$ but no others. Once $n$ hyperrectangles have been given as input, there is an MIS of size $\bigl\lfloor \frac{n}{2} \bigr\rfloor + 1$ consisting of all the hyperrectangles not marked by the adversary.
	
	Consider any online algorithm. At each level $k$, by symmetry, the probability that it selects only $x_k$ and the probability that it selects only the other hyperrectangle are equal; denote this probability by $p_k$. Let $p_{\sup} = \sup \set{p_k}$. Then, the expected size of the independent set $X$ constructed by the online algorithm can be bounded as follows.
	\begin{align*}
		\E[X]
		&\leq 1 + p_1 (1 + p_2 (1 + \dots)) \\
		&\leq 1 + p_{\sup} (1 + p_{\sup} (1 + \dots)) \\
		&= \frac{1}{1 - p_{\sup}} \\
		&\leq \frac{1}{1 - \frac{1}{2}} \\
		&= 2
	\end{align*}
	The competitive ratio is then at least
	$\frac{1}{2} \bigl\lfloor \frac{n}{2} \bigr\rfloor + \frac{1}{2}$,
	which is $\Theta(n)$.
\end{proof}

An example illustrating the adversary strategy from the proof is shown in Figure~\ref{fig:adversary_strategy_unit-vol_2}.

% Figure environment removed


\subsection{Arbitrary Order}
\subsubsection{Oblivious Adversary}
All results from Section~\ref{sec:unit-vol:non-dominated:oblivious} hold here.






\section{Arbitrary Hypercubes}
\label{sec:cube}
In this section, we prove results for arbitrary hypercubes in non-dominated and arbitrary order. The results from Section~\ref{sec:rect} on dominating order apply in this special case as well.


\subsection{Non-Dominated Order}
\subsubsection{Adaptive Online Adversary}
Every online algorithm must have competitive ratio at least $n - 1$.

\begin{proof}
	The adversary can give as input disjoint, dominating hypercubes until the algorithm selects one. Then, the adversary can have the remaining input be disjoint hypercubes in non-dominated order that intersect (and are not dominated by) the selected hypercube.
\end{proof}

An example of the adversary strategy from the proof is shown in Figure~\ref{fig:adversary_strategy_cube_1}.

% Figure environment removed


\subsubsection{Oblivious Adversary}
\label{sec:cube:non-dominated:oblivious}
\paragraph{Proposed Algorithm}
All results on \greedy{p} for fixed $n$ from Section~\ref{sec:unit:non-dominated:oblivious} (unit hypercubes) hold here. In addition, we can construct an arrangement for $n=6$ (see Figure~\ref{fig:n6_cube}), so the result for $n=6$ from Section~\ref{sec:unit:arbitrary:oblivious} holds as well.

% Figure environment removed


\paragraph{Lower Bound}
Every online algorithm must have competitive ratio at least $\frac{1}{2} \bigl\lfloor \frac{n}{2} \bigr\rfloor + \frac{1}{2}$.

\begin{proof}
	Suppose $n$ is finite.
	The adversary first gives two disjoint hypercubes as input. Then, it (uniformly) randomly ``selects'' one (denote it by $x_1$) and gives as input two disjoint hypercubes that intersect $x_1$ but no other hypercubes from the existing input. Next, it randomly selects one of these (denote it by $x_2$) and gives as input two disjoint hypercubes that intersect $x_1$ and $x_2$ but no others. It continues this process until it has given $n$ hypercubes as input, at the $k$th level selecting a hypercube $x_k$ and giving as input two hypercubes that intersect $x_1$ through $x_k$ but no others. At the end of this process, there is an MIS of size $\bigl\lfloor \frac{n}{2} \bigr\rfloor + 1$ consisting of all the hypercubes not selected by the adversary.
	
	Consider any online algorithm. At each level $k$, by symmetry, the probability that it selects only $x_k$ and the probability that it selects only the other hypercube are equal; denote this probability by $p_k$. Let $p_{\sup} = \sup \set{p_k}$. Then, the expected size of the independent set $X$ constructed by the online algorithm can be bounded as follows.
	\begin{align*}
		\E[X]
		&\leq 1 + p_1 (1 + p_2 (1 + \dots)) \\
		&\leq 1 + p_{\sup} (1 + p_{\sup} (1 + \dots)) \\
		&= \frac{1}{1 - p_{\sup}} \\
		&\leq \frac{1}{1 - \frac{1}{2}} \\
		&= 2
	\end{align*}
	The competitive ratio is then at least
	$\frac{1}{2} \bigl\lfloor \frac{n}{2} \bigr\rfloor + \frac{1}{2}$,
	which is $\Theta(n)$.
	
	If $n = \infty$, then for any claimed finite competitive ratio $\alpha$, the adversary can repeatedly execute the above process with $2\alpha$ full levels for a competitive ratio of at least $\alpha + \frac{1}{2}$. Thus, the competitive ratio must be $\infty$.
\end{proof}

An example illustrating the adversary strategy from the proof is shown in Figure~\ref{fig:adversary_strategy_cube_2}.

% Figure environment removed


\subsection{Arbitrary Order}
\subsubsection{Oblivious Adversary}
All results from Section~\ref{sec:cube:non-dominated:oblivious} hold here.






\section{Arbitrary Hyperrectangles}
\label{sec:rect}
In this section, we prove results for arbitrary hyperrectangles in dominating and arbitrary order under the adaptive offline adversary model.


\subsection{Dominating Order}
\subsubsection{Adaptive Offline Adversary}
The naive greedy algorithm is optimal.

\begin{proof}
	Note that a hyperrectangle can be intersected by any number of disjoint hyperrectangles if the input comes in dominating order, but it can be dominated by at most one of those and dominates the rest.
	%
	Let $A$ be the set of hyperrectangles in some optimal solution. Let $B$ be the set of hyperrectangles in the greedy solution.
	%
	Let $x_i \in B \setminus A$. Consider the set $Y_i$ of hyperrectangles in $A \setminus B$ that intersect $x_i$ and are dominated by it. Since each hyperrectangle in $Y_i$ was not selected by the greedy algorithm, they must each intersect some hyperrectangle in $B$ other than $x_i$.
	
	If $x_i$ is the first hyperrectangle in the input order, then $|Y_i| = 0$. Otherwise, let $z_i$ be the hyperrectangle in $B$ whose position in the input order is closest to that of $x_i$ out of those that came before $x_i$. Note that every hyperrectangle in $Y_i$ must intersect $z_i$. Since the hyperrectangles in $Y_i \subset A$ must be disjoint, $Y_i$ must have cardinality at most 1.
	
	Thus, each hyperrectangle $x_i$ in $B \setminus A$ intersects at most two hyperrectangles in $A \setminus B$. If it intersects two, then one of them dominates it and the other is dominated by it. Furthermore, the hyperrectangle that is dominated by $x_i$ must intersect (and dominate) some other hyperrectangle in $B \setminus A$ as well. Therefore, $|A| = |B|$.
\end{proof}

Essentially, in dominating order, it reduces to the 1D case, where ``earliest deadline first'' is optimal.


\subsection{Non-Dominated Order}
\subsubsection{Oblivious Adversary}
All results from Section~\ref{sec:unit-vol:non-dominated:oblivious} (unit-volume hyperrectangles) hold here.


\subsection{Arbitrary Order}
\subsubsection{Adaptive Offline Adversary}
The naive greedy algorithm has a competitive ratio at most $n - 1$.

\begin{proof}
    As long as $n > 0$, the greedy algorithm will select at least one hyperrectangle.
	Suppose some optimal solution has $n$ hyperrectangles. Then, all $n$ hyperrectangles in the input are disjoint, so the greedy algorithm will also select $n$ hyperrectangles, giving a ratio of 1. If the optimal solution has at most $n - 1$ hyperrectangles, then the ratio is at most $n - 1$.
\end{proof}


\subsubsection{Oblivious Adversary}
All results from Section~\ref{sec:unit-vol:non-dominated:oblivious} hold here.