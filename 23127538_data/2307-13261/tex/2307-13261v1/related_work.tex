\section{Related Work}
\label{sec:related_work}
In this section, we discuss key related work. Broadly, each paper can be classified as solving online MIS for different class of graphs or solving MIS of hyperrectangles under a different input framework (offline, dynamic, etc.).


\subsection{Online Interval Scheduling}
The 1D version of our problem is ``online MIS in interval graphs'' or ``online maximum disjoint set of intervals''. It can also be viewed as a form of interval scheduling.
%
A prominent early work on a similar problem is that of \citet{Lipton1994}. However, they aim to find a set of intervals of maximum total length instead of a set of maximum cardinality, and intervals are added in order of start time. A special case of the work of \citet{Faigle1995} is similar to our problem; they do optimize for maximum cardinality, but still require that intervals arrive in order of start time and also allow intervals to be discarded.

Most papers tend to assume that the intervals are added in order of start or end time, but \citet{Coffman1998} study the case of unit-length intervals added in random order. The model of \citet{Goyal2020} can be viewed as intervals added in arbitrary order (but maximizing weight, not cardinality).
%
There is also work on the streaming setting (limited memory)~\citep{Emek2016, Cabello2017}.


\subsection{Online MIS in Disk Graphs}
A disk graph is the intersection graph of a set of disks in the plane.
%
There is a deterministic online algorithm (greedy) for MIS in disk graphs. In unit disk graphs (all disks have diameter 1), the algorithm is 5-competitive. In general disk graphs, the algorithm is $(n-1)$-competitive. In $\sigma$-bounded disk graphs (all disks have diameters between 1 and $\sigma$), the algorithm is $O(\min\set{n, \sigma^2})$-competitive. In all cases, it is optimal~\citep{Erlebach2006}.

\citet{Caragiannis2007} study randomized algorithms for online MIS in disk graphs. They use the oblivious adversary model. They give a lower bound of $\Omega(n)$ on the competitive ratio for general disks. For $\sigma$-bounded disks, they give an algorithm that is $O(\min\set{n, \log\sigma})$-competitive if $\sigma$ is given as input and an algorithm that is $O\bigl(\min\set[\big]{n, \prod_{j=1}^{\log^* \sigma-1} \log^{(j)} \sigma}\bigr)$-competitive otherwise. For unit disk graphs, they give a lower bound of $2.5$ (or 3 if only the graph representation is given) and present an algorithm with a competitive ratio of $8 \sqrt{3} / \pi \approx 4.41$.
%
For the similar case of unit balls, there is a 12-competitive algorithm~\citep{De2021}.
%
In the (turnstile) streaming case for unit disks, \citet{Bakshi2020} prove the same $8 \sqrt{3} / \pi$ upper bound and also prove a $2-\epsilon$ lower bound.

\citet{Marathe1995} show that there is a 3-competitive algorithm in unit disk graphs where disks arrive in order of $x$-position. They also mention the idea of inscribing regular polygons in unit disks to get algorithms with loose competitive ratios for MIS for that type of polygon. One could use this approach to compute loose competitive ratios for unit squares arriving in order of $x$-position. Furthermore, the same idea could potentially be extended to regular polygons (squares) inscribed in general disks, irregular polygons (rectangles) inscribed in unit disks, or irregular polygons (rectangles) inscribed in general disks.


\subsection{Online MIS in General Graphs}
For certain stochastic input models, there is an algorithm with competitive ratio quadratic in the inductive independence number of the graph~\citep{Gobel2013}.
%
If the algorithm is allowed to maintain multiple candidate solutions, then there is an algorithm with sublinear competitive ratio~\citep{Halldorsson2002}.
%
Other relaxations of the online setting have been considered as well~\citep{Boyar2022}.


\subsection{Offline MIS of Rectangles}
The offline 2D version of our problem has received a lot of recent attention. To the best of our knowledge, \citet{Agarwal1998} were the first to study the problem. The first (polynomial-time) constant-factor approximation algorithm was only recently discovered~\citep{Mitchell2021}. The current state of the art has a competitive ratio of $2 + \epsilon$~\citep{Galvez2021}.
%
There is also existing work where restrictions are placed on the family of rectangles used as input~\citep{Correa2015}.


\subsection{Dynamic/Streaming/Online MIS of Hyperrectangles}
The dynamic version of MIS of hyperrectangles has been studied before~\citep{Bhore2022a}. There are also earlier results for hypercubes in a dynamic setting~\citep{Bhore2020}.
%
\citet{Bhore2022b} study MIS of rectangles in a streaming setting (limited memory). They present a one-pass streaming algorithm that achieves an approximation factor of 4 for unit-height rectangles.

There is a single work studying a subcase of the problem we study in this paper. \citet{De2021} have independently proved tight upper and lower bounds for the special case of unit $d$-cubes in arbitrary order with an adaptive adversary.
%
In addition, the inductive independence number of intersection graphs of $d$-hyperrectangles is $2d-1$~\citep{Ye2012}, so there is an algorithm with constant competitive ratio for online MIS of $d$-hyperrectangles for certain stochastic input models~\citep{Gobel2013}.