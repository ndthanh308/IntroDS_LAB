\section{Introduction}
The maximum independent set (MIS) problem is a classical NP-hard problem in theoretical computer science, with many real-world applications as well.
Many variations of the MIS problem have been studied. Recently, there has been a series of papers studying the MIS problem restricted to the family of intersection graphs of sets of axis-aligned rectangles in the plane~\citep{Mitchell2021, Galvez2021}.
In this work, we study the extension of this problem to higher dimensions in an online setting.


\subsection{Definitions}
In $d$ dimensions, an \emph{axis-aligned hyperrectangle} is a closed region $[a_1, b_1] \times [a_2, b_2] \times \dots \times [a_d, b_d]$. A $d$-hyperrectangle is a $d$-dimensional hyperrectangle. E.g., a 2-hyperrectangle is a rectangle.
All hyperrectangles henceforth are assumed to be axis aligned unless otherwise specified.
%
A \emph{hypercube} is a hyperrectangle with all side lengths equal (i.e., $b_1 - a_1 = b_2 - a_2 = \dots = b_d - a_d$). A $d$-hypercube is a $d$-dimensional hypercube. E.g., a 3-hypercube is a cube.
%
Regardless of the dimension $d$, we will use ``volume'' to refer to the Lebesgue measure of a given hyperrectangle. E.g., for $d=1$, this would be the length; for $d=2$, the area.

Every hyperrectangle has exactly one vertex with all coordinates greater than or equal to the respective coordinates of every other vertex; we will refer to this vertex as the \emph{upper} vertex. Analogously, we will refer to the vertex with all coordinates less than or equal to the respective coordinates of every other vertex as the \emph{lower} vertex of the hyperrectangle. E.g., for the hypercube $[0,1]^4$, the upper vertex is $(1,1,1,1)$ and the lower vertex is $(0,0,0,0)$.
%
A hyperrectangle $x$ \emph{dominates} a hyperrectangle $y$ if the upper vertex of $x$ has all coordinates greater than or equal to the respective coordinates of the upper vertex of $y$. E.g., $[0,2]^2$ dominates $[0,1]^2$, but neither of $[0,1] \times [0,2]$ and $[0,2] \times [0,1]$ dominates the other.

Given a set of geometric objects (e.g., hyperrectangles) in space, their intersection graph is a simple, undirected graph with a node representing each object and an edge between the nodes of two objects that intersect.

For any (possibly randomized) online algorithm with solution $\mathrm{SOL}$, if the solution of an optimal offline algorithm is $\mathrm{OPT}$ and the space of possible inputs is denoted by $\mathcal{X}$, the \emph{competitive ratio} of the online algorithm is\footnote{Here, ``$\sup$'' refers to the supremum, or least upper bound. When the maximum exists, it is equal to the supremum.}
\[\sup_{X \in \mathcal{X}} \frac{|\mathrm{OPT}|}{\E[|\mathrm{SOL}|]} \,.\]


\subsection{Problem Formulation}
Given a set of $n$ axis-aligned hyperrectangles provided in an online fashion, the goal is to select a set of disjoint hyperrectangles with (approximately) maximum cardinality.
We also note the following key aspects of the problem formulation:
\begin{itemize}
    \item The input is \emph{insertion only}: exactly one hyperrectangle is given as input in each time step, and no hyperrectangle is ever removed.
    \item The selections of the algorithm are final. In each time step, the algorithm either accepts or rejects the currently offered hyperrectangle; its decision cannot be revoked in future time steps.
    \item The algorithm has access to the coordinates of the hyperrectangles in space, not just the intersection graph associated with the input.
    \item The performance of the online algorithm is measured using its competitive ratio.
\end{itemize}


\subsubsection{Input Shape and Size}
We prove different results for various restrictions on the shapes and sizes of the input hyperrectangles. The following classes of hyperrectangles are considered (in non-decreasing order of generality):
\begin{description}
    \item[Unit hypercubes] Hypercubes with side length 1.
    \item[$\bm{\sigma}$-bounded hypercubes] Hypercubes with side length between 1 and $\sigma$, inclusive.
    \item[Unit-volume hyperrectangles] Hyperrectangle with volume 1.
    \item[Arbitrary hypercubes] Hypercubes with no other restrictions.
    \item[Arbitrary hyperrectangles] Hyperrectangles with no other restrictions.
\end{description}


\subsubsection{Input Order}
We prove results for various orderings of hyperrectangles in the input. The following input orders are considered (in increasing order of difficulty):
\begin{description}
    \item[Dominating] Each hyperrectangle dominates every hyperrectangle preceding it in the input order.
    \item[Non-dominated] Each hyperrectangle is \emph{not} dominated by any hyperrectangle preceding it in the input order.
    \item[Arbitrary] The input can arrive in any order.
\end{description}


\subsubsection{Adversary Model}
We consider the following three adversary models (in increasing order of difficulty):
\begin{description}
	\item[Oblivious] This adversary knows the algorithm's implementation but not the hyperrectangles it selects during execution.
	\item[Adaptive online] This adversary knows the algorithm's implementation and can adapt its strategy after the algorithm makes selections.
	\item[Adaptive offline] This adversary knows \emph{everything}. It knows the algorithm's implementation and the outcome of every random variable used in its selection process in advance.
\end{description}


\subsection{Contributions}
To the best of our knowledge, we are the first to present results on the MIS of hyperrectangles problem under the oblivious adversary model. We are also the first to present results under adaptive adversary models with input in dominating or non-dominated order. For input in arbitrary order, we present the first results on hyperrectangles more general than unit hypercubes; we present the first results on $\sigma$-bounded hypercubes, unit-volume hyperrectangles, arbitrary hypercubes, and arbitrary hyperrectangles.

An overview of our results for input in non-dominated and arbitrary order can be found in Tables~\ref{tab:non-dom_results}~and~\ref{tab:arbit_results}, respectively. For input in dominating order, we show that the performance of the naive greedy algorithm matches the performance of an optimal offline algorithm in all cases. We also give lower bounds on the competitive ratio of the \greedy{p} algorithm (defined in Section~\ref{sec:unit:non-dominated:oblivious}) for $d \geq 2$ under the oblivious adversary model (Table~\ref{tab:greedy_p_results}).

\begin{table}[bhpt]
\caption{Competitive ratios (or bounds when unknown) of optimal online algorithm for input in non-dominated order.}
\label{tab:non-dom_results}
\centering
\begin{tabular}{lc@{\hspace{1cm}}c}
    \toprule
    Input shape/size & Adaptive & Oblivious \\
    \midrule
    Unit hypercubes & $2^d - 1$ & $\Bigl[ \frac{12}{7}, 2^d - 1 \Bigr]$ \\
    $\sigma$-bounded hypercubes & $(\lceil \sigma \rceil + 1)^d - \lceil \sigma \rceil^d$ & $\Bigl[ \frac{\lceil \log_2 \sigma \rceil + 1}{2}, 3^d \lceil \log_2 \sigma \rceil - 2^d \lceil \log_2 \sigma \rceil \Bigr]$ \\
    Unit-volume hyperrectangles & $n - 1$ & $\Bigl[ \frac{1}{2} \Bigl\lfloor \frac{n}{2} \Bigr\rfloor + \frac{1}{2}, n - 1 \Bigr]$ \\
    Arbitrary hypercubes & $n - 1$ & $\Bigl[ \frac{1}{2} \Bigl\lfloor \frac{n}{2} \Bigr\rfloor + \frac{1}{2}, n - 1 \Bigr]$ \\
    Arbitrary hyperrectangles & $n - 1$ & $\Bigl[ \frac{1}{2} \Bigl\lfloor \frac{n}{2} \Bigr\rfloor + \frac{1}{2}, n - 1 \Bigr]$ \\
    \bottomrule
\end{tabular}
\end{table}

\begin{table}[bhpt]
\caption{Competitive ratios (or bounds when unknown) of optimal online algorithm for input in arbitrary order.}
\label{tab:arbit_results}
\centering
\begin{tabular}{lc@{\hspace{1cm}}c}
    \toprule
    Input shape/size & Adaptive & Oblivious \\
    \midrule
    Unit hypercubes & $2^d$~(\citet{De2021}) & $\Bigl[\frac{32}{15}, 2^d\Bigr]$ \\
    $\sigma$-bounded hypercubes & $(\lceil \sigma \rceil + 1)^d$ & $\Bigl[ \frac{\lceil \log_2 \sigma \rceil + 1}{2}, 3^d \lceil \log_2 \sigma \rceil \Bigr]$ \\
    Unit-volume hyperrectangles & $n - 1$ & $\Bigl[ \frac{1}{2} \Bigl\lfloor \frac{n}{2} \Bigr\rfloor + \frac{1}{2}, n - 1 \Bigr]$ \\
    Arbitrary hypercubes & $n - 1$ & $\Bigl[ \frac{1}{2} \Bigl\lfloor \frac{n}{2} \Bigr\rfloor + \frac{1}{2}, n - 1 \Bigr]$ \\
    Arbitrary hyperrectangles & $n - 1$ & $\Bigl[ \frac{1}{2} \Bigl\lfloor \frac{n}{2} \Bigr\rfloor + \frac{1}{2}, n - 1 \Bigr]$ \\
    \bottomrule
\end{tabular}
\end{table}

\begin{table}[bhpt]
\caption{Lower bounds on competitive ratios of \greedy{p} for $d \geq 2$ in the oblivious adversary model.}
\label{tab:greedy_p_results}
\centering
\begin{tabular}{lcc}
    \toprule
    Input shape/size & Non-dominated order & Arbitrary order \\
    \midrule
    Unit hypercubes & 2.65 ($p = 0.56$) & 3.20 ($p = 0.50$) \\
    Unit-volume hyperrectangles & 3.20 ($p = 0.50$) & 3.20 ($p = 0.50$) \\
    Arbitrary hypercubes & 3.20 ($p = 0.50$) & 3.20 ($p = 0.50$) \\
    Arbitrary hyperrectangles & 3.20 ($p = 0.50$) & 3.20 ($p = 0.50$) \\
    \bottomrule
\end{tabular}
\end{table}


\subsection{Organization}
We begin by surveying related work (Section~\ref{sec:related_work}).
Next, we present our results on unit hypercubes (Section~\ref{sec:unit}), $\sigma$-bounded hypercubes (Section~\ref{sec:sig-bounded}), unit-volume hyperrectangles (Section~\ref{sec:unit-vol}), arbitrary hypercubes (Section~\ref{sec:cube}), and arbitrary hyperrectangles (Section~\ref{sec:rect}).
Finally, we discuss directions for future work (Section~\ref{sec:future_work}).