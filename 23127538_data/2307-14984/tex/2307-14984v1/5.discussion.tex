\section{Discussions and Open Problems}\label{sec::discussion}


The S$^3$ system, which has been developed, represents an initial endeavor aimed at harnessing the capabilities of large language models. This is to facilitate simulation within the domain of social science.

In light of this, our analysis delves further into its application and limitations, along with promising future improvements.

\subsection{Application of S$^3$ System}

Leveraging the powerful capabilities of large language models, this system excels in agent-based simulation. The system has the following applications in the field of social science.

\begin{itemize}[leftmargin=*]

    \item \textbf{Prediction.} Prediction is the most fundamental ability of agent-based simulation. Large language model-based simulation can be utilized to predict social phenomena, trends, and individual behaviors with historically collected data. For example, in economics, language models can help forecast market trends, predict consumer behavior, or estimate the impact of policy changes. In sociology, these models can aid in predicting social movements, public opinion shifts, or the adoption of new cultural practices. 

    \item \textbf{Reasoning and explanation.} During the simulation, each agent can be easily configured, and thus the system can facilitate reasoning and explanation in social science by generating phenomena with different configurations. Comparing the simulation results can provide explain the cause of the specific phenomena. Furthermore, the agent can be observed by prompts which can reflect how a human takes actions in the social environment.

    \item \textbf{Pattern discovery and theory construction.} With repeated simulation during the extremely less cost compared with real data collection, the simulation process can reveal some patterns of the social network. By uncovering patterns, these models can contribute to the development of new theories and insights. Furthermore, researchers can configure all the agents and the social network environment, based on an assumption or theory, and observe the simulation results. Testing the simulation results can help validate whether the proposed assumption or theory is correct or not.

    \item \textbf{Policy making.} The simulation can inform evidence-based policy-making by simulating and evaluating the potential outcomes of different policy interventions. It can assess the impact of policy changes on various social factors, including individual agents and the social environment. For example, in public health, it can simulate the spread of infectious diseases to evaluate the effectiveness of different intervention strategies. In urban planning, it can simulate the impact of transportation policies on traffic congestion or air pollution, by affecting how the users select public transportation. By generating simulations, these models can aid policymakers in making informed decisions.

\end{itemize}


\subsection{Improvement on Individual-level Simulation}
The current design of individual simulation still has several limitations requiring further improvement.
  First,  the agent requires more prior knowledge of user behavior, including how real humankind senses the social environment and makes decisions.  In other words, the simulation should encompass an understanding and integration of intricate contextual elements that exert influence on human behavior. Second, while prior knowledge of user behavior is essential, simulations also need to consider the broader context in which decisions are made. This includes factors such as historical events, social conditions, and personal experiences. By enhancing the agent's capacity to perceive and interpret contextual cues, more precise simulations can be achieved.


\subsection{Improvement on Population-level Simulation}

First, it is better to combine agent-based simulation with system dynamics-based methods.
Agent-based simulation focuses on modeling individual entities and their interactions, while system dynamics focuses on modeling the behavior of the social complex system as a whole. Through the fusion of these two methodologies, we can develop simulations of heightened comprehensiveness, encompassing both micro-level interactions and macro-level systemic behavior. This integration can provide a more accurate representation of population dynamics, including the impact of individual decisions on the overall system.

Second, we can consider a broader range of social phenomena. This involves modeling various societal, economic, and cultural factors that influence human behavior and interactions. Examples of social phenomena to consider include social networks, opinion dynamics, cultural diffusion, income inequality, and infectious disease spread. By incorporating these phenomena into the simulation, we can better validate the system's effectiveness and also gain more insights into social simulation.

\subsection{Improvement on System Architecture Design}

First, we can consider incorporating other channels for social event information. It is essential to acknowledge that social-connected users are not the sole providers of information for individuals within social networks. Consequently, the integration of supplementary data sources has the potential to enrich the individual simulation. For instance, recommender systems can be integrated to gather diverse information about social events. This integration can help capture a wider range of perspectives and increase the realism of the simulation.

Second, the system architecture should consider improving efficiency, which is essential for running large-scale simulations effectively. Optimizing the system architecture and computational processes can significantly enhance the performance and speed of simulations. To this end, techniques such as parallel computing, distributed computing, and algorithmic optimizations can be employed to reduce computational complexity and advance the efficiency of simulation runs. This allows for faster and more extensive exploration of scenarios, thereby enabling researchers to gain insights faster.

Third, it is essential to add an interface for policy intervention. Including an interface that allows policymakers to interact with the simulation can be beneficial. This interface would enable policymakers to input and test various interventions and policies in a controlled environment. By simulating the potential outcomes of different policy decisions, policymakers can make more informed choices. They can also evaluate the potential impact of their interventions on the simulated population. This feature can facilitate evidence-based decision-making and identify effective strategies.