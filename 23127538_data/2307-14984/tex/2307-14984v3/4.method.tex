\section{Architecture and Methodology}\label{sec::method}

\subsection{Architecture Design}

In order to simulate the process of information propagation on the online social network, we have designed a message propagation simulation framework illustrated in Figure \ref{fig::framework} and is explained in detail below.


\textbf{Environment Construction}: The construction of the environment involves the formation of a social network on a public platform, comprising users and connections among them. For instance, users have the ability to establish mutual following relationships with their friends, or one-way following relationships with users they find interesting. Hence, the social network can be characterized as a directed graph, where the outdegree and indegree of nodes in the network represent the number of people they follow and the number of followers they possess, respectively.
The users within this network can be broadly categorized into three groups: influential users, regular users, and low-impact users. Influential users typically exhibit a significantly larger number of followers compared to the number of people they follow. Moreover, they demonstrate a tendency to share high-quality original information. Regular users, on the other hand, typically maintain a balanced proportion of followers and followings. Additionally, a considerable portion of regular users engage in mutual following relationships, which often reflect their real-life friendships. Conversely, low-impact users exhibit limited followers, infrequent message posting, and typically represent the terminal points of message propagation chains. It is important to note that within this framework, we have excluded the consideration of social bots and zombie users, despite their prevalence on social platforms.



\textbf{User Characterization}
In addition to the social relationships present within the network, each user possesses their own attribute descriptions. Certain attributes are objective and specific, encompassing factors such as gender, occupation, and age. On the other hand, other attributes are more abstract, including their attitudes towards specific events and their prevailing emotional states. The former attributes tend to exhibit minimal fluctuations over short durations, whereas the latter attributes are more dynamic, particularly when users engage in information browsing on social platforms. In such cases, their fundamental attributes, message content, and message sources consistently shape their attitudes, emotions, and other abstract attributes. 
In light of the aforementioned descriptions, we also introduce a memory pool for each user. Given the abundance of messages from diverse users on online public platforms, a multitude of messages emerge daily. It is important to acknowledge that different messages exert varying influences on distinct users. To address this, we draw inspiration from~\cite{park2023generative} and propose the concept of influence factors. These factors calculate weighted scores based on parameters such as posting time, content relevance, and message importance. By doing so, we ensure that the user's memory pool retains the most impactful messages, making them highly memorable.

\begin{itemize}[leftmargin=*]
    \item \textbf{Temporal Influence}: The recency of messages plays a significant role in human memory, with previous messages gradually fading over time. A time score is ascribed to messages using a prescribed forgetting function.
    
    \item \textbf{Content Relevance}: The relevance of message content is assessed with regard to the user's individual characteristics. Notably, younger individuals tend to exhibit a greater inclination towards entertainment-related events, whereas middle-aged individuals demonstrate heightened interest in political affairs. To quantify the degree of relevance, a relevance score is obtained by measuring the cosine similarity between a user's fundamental attributes and the content of the message.
    
    \item \textbf{Message Authenticity}: The authenticity of messages is closely related to their sources. Messages are categorized based on their origins, encompassing messages disseminated by unidirectional followers, messages shared by mutual followers, messages recommended by the platform, and messages previously posted by the user themselves. Distinct scores are assigned to messages based on their respective sources.
\end{itemize}





\textbf{Update and Evolution Mechanism}: During a social gathering, various official accounts and individual users contribute posts concerning the event, encompassing news reports and personal viewpoints. Upon encountering these messages, the users who follow them manifest diverse emotional responses. Some users may even formulate their own stances on contentious matters, either in support or opposition, subsequently engaging in online activities such as endorsing, disseminating, and creating original message. In this simulation, we employ large language models to replicate individual users, leveraging their profiles and memory pools as prompts to generate cognitive reactions and behavioral responses. Subsequently, their abstract attributes and memory pools undergo updates. Following the modification of a user's memory pool, these messages disseminate and exert influence on their followers while they peruse the content. This iterative process persists, emulating the propagation of messages and the evolution of individuals' cognitive states.

\subsection{Initialization}

\subsubsection{Social Network Construction}


Within the scope of this study, we propose an initialization approach to construct a network utilizing data acquired from real-world social media sources (refer to Table \ref{tab::overall}). Strict adherence to privacy regulations and policies is maintained throughout the collection of social media data. Our approach leverages keyword-matching techniques to effectively extract posts relevant to the simulated scenarios. Subsequently, we delve into the identification of the authors and extract them as the foundational nodes of our network. Expanding beyond the individual level, we meticulously gather socially connected users. To establish connections between users, directed edges are established if the corresponding followee exists within the extracted user set. To optimize simulation efficiency, in this work, we focus solely on this sub-graph rather than the entire graph which is too large. During the simulation, the dissemination of messages occurs exclusively between source nodes and their corresponding target nodes.


\subsubsection{User Demographics Prediction}


Expanding upon the properties of the node, specifically focusing on user demographic attributes, emerges as a pivotal stride in our endeavor towards a more exhaustive simulation. Through the incorporation of additional information regarding the users into the system, we can delve into and scrutinize their behaviors, interactions, and influence within the network, more effectively. User demographic attributes allow us to capture heterogeneity and diversity in real-world social networks. That is, demographic attributes play a significant role in shaping individual behaviors and preferences, which, in turn, influence the network's overall attitude dynamics. In our study, we chose gender, age, and occupation as the major demographic attributes. As social media data does not directly offer attributes such as gender, age, and occupation, we rely on prediction techniques to estimate these attributes. Leveraging LLMs provides a robust approach to predicting these demographic attributes. By utilizing LLMs, we can leverage the extensive contextual understanding and knowledge encoded within the models to infer user demographics based on available information, such as personal descriptions and content within posts. The technical details are as follows.



\noindent \textbf{User Demographics Prediction with LLM.}
In order to predict user gender based on personal descriptions, since the collected data lacks sufficient labels, we use a public dataset released in~\cite{10.1145/3219819.3220077,10.1145/2700398} for assistance. It allows us to extract a vast array of labeled gender and personal description relationships. We filter out data with longer than 10 words in this dataset served as the ground truth to tune the language model. Specifically, we use ChatGLM~\cite{du-etal-2022-glm} as the foundation model and employ the P-Tuning-v2~\cite{liu-etal-2022-p} methodology. We feed the model with the personal description as a prompt and let the model determine the most probable gender associated with the given description.

To predict age using users' posts, we use Blog Authorship Corpus Dataset \cite{schler2006effects} dataset to establish the expression-to-age relationship. This dataset provides us with author-age labels for corresponding textual posts. We randomly select the historical blogs in \cite{schler2006effects} and add them to the prompt as input; then, the age can be used as the label for prefix tuning. The tuned large language model can be used to predict the age label in our collected social media dataset.


Next, we predict occupations only using pre-trained LLMs. In this scenario, we directly feed users' posts and personal profile descriptions to the LLM for prediction. By examining the content of these inputs, the model showcased its capacity to comprehend and infer users' occupations, further enhancing our demographic prediction capabilities. 


\noindent \textbf{Prediction Result Evaluation}

The outcomes of our age and gender prediction analysis are presented in Table \ref{tab:dem_pre}. Our gender predictor, which relies on a fine-tuned Large Language Model (LLM), achieves satisfactory results. Despite the absence of explicit gender information in all personal descriptions, the predictor successfully generates valid predictions. Moving on to age, we select English blogs from \cite{schler2006effects} and ensured similar age distribution across the training and testing process. The results show that the mean squared error (MSE) was 128, while the mean absolute error (MAE) was around 7.53. These values indicate a 21.5\% unified percentage error (see Table \ref{tab:dem_pre}). 


As for the occupations, we initially include the posts and personal descriptions of the combined user dataset in the prompt. We then feed the prompt to pre-trained ChatGLM to obtain the occupation of each user. We leave the supervised fine-tuning for occupation prediction as future work. It results in a total of 1,016 different occupations being identified from all users. However, utilizing all occupations is not essential since some occupations are very close. Thus, we group all occupations into 10 distinct occupation categories using the LLM, of which the categories can be found in Table \ref{tab:gpt_occ}. By condensing the number of occupations into a smaller set, we are able to simplify the simulation.




\begin{table}[t]
    \centering
    \begin{minipage}{0.45\textwidth}  %
            \caption{Prediction performance of gender and age.}
        \centering
        \small
        \renewcommand{\arraystretch}{1.2} %
        \begin{tabular}{c|ccc}
            \specialrule{1.5pt}{0pt}{0pt}
            \textbf{Demographic} & \multicolumn{3}{c}{\textbf{Performance}} \\ \midrule
            \multirow{2}{*}{Gender} & \textbf{Acc} & \textbf{F1} & \textbf{AUC} \\
            & 0.710 & 0.667 & 0.708 \\ \midrule
             \multirow{2}{*}{Age} & \textbf{MSE} & \textbf{MAE} & \textbf{Avg \% Error} \\
            & 128.0 & 7.53 & 21.50 \\ \specialrule{1.5pt}{0pt}{0pt}
        \end{tabular}
        \captionsetup{singlelinecheck=off, justification=centering}
        \label{tab:dem_pre}
    \end{minipage}
    \hfill
    \begin{minipage}{0.45\textwidth}  %
        \centering
        \caption{Ten occupations.}
        \renewcommand{\arraystretch}{1.2} %
        \begin{tabular}{|l|l|}
     \hline 1& \fontfamily{ppl}\selectfont Education Practitioner\\
 2&  \fontfamily{ppl}\selectfont Administrative Manager / Officer \\
3& \fontfamily{ppl}\selectfont Unemployed / Student \\
4& \fontfamily{ppl}\selectfont Engineer \\
5& \fontfamily{ppl}\selectfont Labor Technician / Worker \\
6& \fontfamily{ppl}\selectfont Logistics Practitioner \\
7& \fontfamily{ppl}\selectfont Medical Personnel \\
8& \fontfamily{ppl}\selectfont Financial Practitioner\\
9& \fontfamily{ppl}\selectfont Media Personnel \\
10& \fontfamily{ppl}\selectfont Entertainment and Arts Practitioner \\
\hline
        \end{tabular}
        \label{tab:gpt_occ}
    \end{minipage}
\end{table}



\subsection{Emotion and Attitude Simulation}

In our emotion simulation model, we adopt a Markov chain approach to capture the dynamic process of emotional changes triggered by a user receiving a message. The simulation involves four essential inputs: user demographics, current emotion, the received post. Emotions are classified into three distinct stages: calm, moderate, and intense.
User demographics serve as supplementary information LLMs, providing a reference point to contextualize emotional responses. The current emotion represents the user's emotional status before receiving the post, while the received post acts as the actuator for prompting the LLM to determine a new emotional status.


To regulate the decrease of emotional states over time, we introduce the decaying coefficient, a hyper-parameter that controls the decay rate of emotions. Our hypothesis assumes that emotions tend to diminish gradually as time passes, influencing the emotion simulation process. Throughout this intricate mechanism, we impart these details by prompt to the LLMs, which are responsible for deciding whether the emotional state should change in response to the received post. We are trying to reduce as much manual intervention as possible, to highlight the capability of LLMs in simulating emotional changes by posts. The attitude simulation is similar to the emotion simulation.







\subsection{Behavior Simulation}

\subsubsection{Content-generation Behavior}
In our social network simulation model, we incorporate an advanced approach utilizing Large Language Models (LLMs) to reproduce the dynamic process of content creation, shaped by users' emotions and attitudes towards specific events. The simulation hinges on two vital inputs: user profile information, and their current emotional or attitudinal state towards the event. Each piece of generated content is an embodiment of a user's internal state and external influences, reflecting their unique perspective.

User profile information serves as a reference point for the LLMs, furnishing essential context to shape content responses. The current emotional or attitudinal state symbolizes the user's mindset when reacting to the event, thereby playing a vital role in the LLM's generation of potential responses.

Underpinning this sophisticated mechanism is the profound cognitive and behavioral comprehension of LLMs. The LLM is prompted with these details and is then responsible for deciding how the content should be shaped in response to the event. Our aim is to minimize manual intervention as much as possible, to highlight the capability of LLMs in simulating authentic user-generated content.

The approach mirrors the way real-world users form their posts in response to distinct events, aligning the text generation process with the emotional or attitudinal dynamics of users. In this manner, we have been successful in utilizing LLMs to emulate the content creation process on social networks with high fidelity. 

\subsubsection{Interaction Behavior}

During the simulation, when a user receives a message from one of their followees, a critical decision needs to be made---whether to repost/post or not. 
That is to say, the interaction behavior includes reposting (forwarding) the original content and posting new content about the same social event.
The user's interaction behavior plays a pivotal role in propagating messages to the user's followers, facilitating the spread of information within the social network. However, modeling the complex mechanisms governing a user's interaction behavior poses significant challenges.
To address it, we employ large language models to capture the intricate relationship between the user, post features, and interaction behavior. 


Specifically, to leverage the ability of LLMs to simulate a real user's interaction behavior, we prompt the model with information regarding the user's demographic properties, \textit{i.e.} gender, age, and occupation, in addition to the specific posts received, letting the LLM think like the user and make its decision. By such means, we enable LLM to make predictions regarding the user's inclination to repost the message or post new content.



To summarize, by employing the above approach, we can effectively harness the power of LLMs to predict users' interaction behavior, taking into account various user and post features. 


\subsection{Other Implementation Details}
The system employs various techniques for utilizing or adapting large language models to the agent-based simulation.
For prompting-driven methods, we use either GPT-3.5 API provided by OpenAI\footnote{https://platform.openai.com/overview} or a ChatGLM-6B model~\cite{du-etal-2022-glm}.
For fine-tuning methods, we conduct the tuning based on the open-source ChatGLM model.

