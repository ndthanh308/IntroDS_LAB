\section{Introduction}\label{sec::intro}


The social network, comprising interconnected individuals in society, constitutes a cornerstone of the contemporary world. 
Diverging from mathematical analysis, computer simulation offers a fresh avenue to comprehend the formation and evolution of social networks. This serves as a fundamental tool for social scientists.
Notably, in 1996, there was already a book titled \textit{Social Science Microsimulation}~\cite{troitzsch1996social} providing valuable insights about simulation from the perspective of social science. Social simulation encompasses a wide range of domains, encompassing both individual and population social activities.
At the heart of social simulation lie two perspectives~\cite{gilbert2005simulation}: 1) the dynamic feedback or interaction among individuals, and 2) the states of the population, either as a collective whole or as distinct groups. By simulating social activities, researchers and practitioners can predict the future evolution of individual and population states. In addition, they facilitate experimental environments through interventions.
Social simulation can be implemented in two forms: microlevel simulation~\cite{chopard1998cellular,park2023generative} and macrolevel simulation~\cite{kolesar1975simulation,meadows1974dynamics, forrester1993system,marsh1978using}. In macrolevel simulation, also known as system-based simulation, researchers model the dynamics of the system using equations that elucidate the changing status of the population. Conversely, microlevel simulation, or agent-based simulation, involves researchers employing either human-crafted rules or parameterized models to depict the behavior of individuals (referred to as agents) who interact with others.
Recently, with the exponential growth of the Internet, online social networks have emerged as the principal platform for societal activities. Users engage in various interactive behaviors such as chatting, posting, and sharing content. Consequently, the study of social networks has become a central research focus within the realm of social science, thereby emphasizing the criticality of simulation in this domain.


Large language models (LLMs)~\cite{brown2020language,openai2023gpt4,chowdhery2022palm, anil2023palm,touvron2023llama,zeng2022glm} are the recent advancement in the field of deep learning, characterized by the utilization of an extensive array of neural layers. These models undergo training on vast textual corpora, acquiring a remarkable fundamental capacity to comprehend, generate, and manipulate human language. 
Given their impressive prowess in text comprehension, which closely approximates human-level performance, LLMs have emerged as a particularly auspicious avenue of research for approaching general artificial intelligence. Consequently, researchers~\cite{aher2023using,horton2023large,hamalainen2023evaluating,park2023generative} leverage LLMs as agent-like entities for simulating human-like behavior, capitalizing on three fundamental capabilities.
First and foremost, LLMs possess the ability to perceive and apprehend the world, albeit restricted to environments that can be adequately described in textual form. 
Secondly, LLMs are capable of devising and organizing task schedules by leveraging reasoning techniques that incorporate both task requirements and the attendant rewards. 
Throughout this process, LLMs effectively maintain and update a memory inventory, employing appropriately guided prompts rooted in human-like reasoning patterns.
Lastly, LLMs exhibit the capacity to generate texts that bear a striking resemblance to human-produced language. These textual outputs can influence the environment and interact with other agents. Consequently, it holds significant promise to adopt an agent-based simulation paradigm that harnesses LLMs to simulate each user within a social network, thereby capturing their respective behaviors and the intricate interplay among users.


In this study, we present the Social-network Simulation System (S$^3$), which employs LLM-empowered agents to simulate users within a social network effectively. Initially, we establish an environment using real-world social network data. To ensure the authenticity of this environment, we propose a user-demographic inference module that combines prompt engineering with prompt tuning, to infer user demographics such as age, gender, and occupation. 
Within the constructed environment, users have the ability to observe content from individuals they follow, thereby influencing their own attitudes, emotions, and subsequent behaviors. Users can forward content, create new content, or remain inactive. Hence, at the individual level, we employ prompt engineering and prompt tuning methodologies to simulate attitudes, emotions, and behaviors. Notably, this simulation considers both demographics and memory of historically-posted content.
At the population level, the accumulation of individual behaviors, including content generation and forwarding, alongside the evolving internal states of attitudes and emotions, leads to the emergence of collective behavior. This behavior encompasses the propagation of information, attitudes, and emotions.


To assess the efficacy of the proposed S$^3$ system, we have chosen two exemplary scenarios, namely, \textbf{gender discrimination} and \textbf{nuclear energy}. With respect to gender discrimination, our objective is to simulate user responses to online content associated with this issue, while closely observing the dissemination patterns of related information and evolving public sentiment. Regarding nuclear energy, our aim is to simulate user reactions to online content pertaining to power policies. In addition, we aim to simulate the contentious and conflicting interactions between two opposing population groups. To evaluate the precision of our simulations, we employ metrics that measure accuracy at both the individual and population levels.
This work's main contributions can be summarized as follows.
\begin{itemize}[leftmargin=*]
    \item We take the pioneering step of simulating social networks with large language models (LLMs), which follows the agent-based simulation paradigm, and empowers the agents with the latest advances.
    \item We develop a simulation system that supports both individual-level and population-level simulations, which can learn from the collected real social network data, and simulate future states.
    \item We systematically conduct the evaluation, and the results show that the simulation system with LLM-empowered agents can achieve considerable accuracy in multiple metrics. Consequently, our system introduces a novel simulation paradigm in social science research, offering extensive support for scientific investigations and real-world applications.
\end{itemize}


To provide a comprehensive understanding of the current research landscape, we begin by reviewing relevant works in Section~\ref{sec::related}. Subsequently, we proceed to introduce the simulation system in Section~\ref{sec::system}, followed by a detailed exposition of the methodology and implementation in Section~\ref{sec::method}. In Section~\ref{sec::discussion}, we engage in discussions and analyze open challenges associated with related research and applications. Finally, we conclude our work in Section~\ref{sec::conclusion}.
