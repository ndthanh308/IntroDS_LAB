\section{S$^3$: Social Network Simulation }\label{sec::system}

\subsection{System Overview}

Our system is constructed within a social network framework, wherein the agent's capabilities are augmented through the utilization of large language models. More specifically, our primary objective is to ensure that the simulation attains a significant degree of quantitative accuracy, catering to both individual-level and population-level simulations. Regarding individual-level simulation, our aim is to replicate behaviors, attitudes, and emotions by leveraging user characteristics, the informational context within social networks, and the intricate mechanisms governing user cognitive perception and decision-making. Through the utilization of agent-based simulation, we further assess the population-level dynamics by scrutinizing the performance of simulating three pivotal social phenomena: the propagation process of information, attitude, and emotion.

% Figure environment removed

\subsection{Social Network Environment}
\begin{table}[t!]
	\caption{The utilized datasets for social network simulation.}
    \label{tab::overall}
    \vspace{0.2cm}
	\centering
	\begin{tabular}{lccc}
		\toprule
		\textbf{Scenario} & \textbf{\#Users} & \textbf{\#Relations} & \textbf{\#Posts} \\ 
		\midrule
		Gender Discrimination & 8,563 & 25,656 & 103,905 \\ 
		Nuclear Energy & 17,945 & 77,435 & 229,450 \\ 
		\bottomrule
	\end{tabular}
\end{table}

In this study, our focus is directed toward two specific focal points, namely gender discrimination and nuclear energy. These particular subjects are chosen owing to their highly controversial nature, which yielded an extensive corpus of data. More specifically, our investigation regarding nuclear energy centers on examining the prevailing attitudes of the general public toward the choice between supporting nuclear energy sources or relying on fossil fuels. As for gender discrimination, our objective is to delve into the emotional experiences of individuals and populations, particularly those elicited by incidents of gender-based discrimination, such as feelings of anger. The availability of such copious amounts of data facilitates the extraction of a substantial portion of the authentic network, thereby enabling us to gain a macroscopic perspective that closely approximates reality. To conduct this analysis, we collect the real data with users, social connections, and textual posts in social media, as detailed in Table \ref{tab::overall}. This dataset provides us with the necessary resources to delve deep into the dynamics of these contentious subjects and gain valuable insights into their impact on social networks. 

User demographics play a pivotal role in shaping user behavior, necessitating the development of a more extensive user persona to enable the realistic and plausible simulation of their actions. However, due to the limited availability of user information obtained directly from social media, it becomes imperative to extract the missing user demographics from textual data, such as user posts and personal descriptions. 
Specifically, we capture user demographic features from textual information using LLM, with a particular emphasis on predicting Age, Gender, and Occupation. By integrating demographic attributes inferred from social network data, we are able to present an enhanced and more authentic representation of users' actions and interactions.

\begin{table}[t!]
\centering
\caption{Performance of our system on prediction tasks for individual simulation.}
\vspace{0.2cm}
\label{tab::ind1}
\small
\begin{tabular}{llccc}
\toprule
\textbf{Scenario} & \textbf{Task} & \textbf{Acc} & \textbf{AUC} & \textbf{F1} \\ 
\midrule
\multirow{2}{*}{Gender Discrimination} & Emotion Level & 71.8\% & --- & --- \\
                                 & Event Propogation & 66.2\% & 0.662 & 0.667 \\
\midrule
\multirow{3}{*}{Nuclear Energy}  & Initial Attitude & 74.3\% & 0.727 & 0.834 \\ 
                                 & Attitude Change & 83.9\% & 0.865 & 0.857 \\
                                 & Event Propogation & 69.5\% & 0.681 & 0.758 \\ 
\bottomrule
\end{tabular}
\end{table}

\subsection{Individual-level Simulation}

Utilizing the initialized social network environment, the system commences the simulation at an individual level. Precisely, the user acquires awareness of the information environment, thereby influencing their emotions and attitude. Subsequently, the user is granted the option to forward (repost) observed posts, generate new content, or keep inactive. In essence, we conduct individual simulations encompassing three facets: emotion, attitude, and interaction behavior.

\subsubsection{Emotion Simulation}
In the process of disseminating real-world events, when a user with their own cognition, attitudes, and personality encounters an event, they are often triggered emotionally and express their emotions on social platforms. Emulating user emotions is crucial for social network simulations, as it significantly influences how users convey their intended messages. However, simulating emotions is challenging due to the multitude of factors and complex relationships involved in human emotions. Leveraging the rich knowledge of human behavior inherent in LLMs, we employ LLM to simulate individual emotions.

Specifically, we model the potential emotions of users towards a particular event as three levels: calm, moderate, and intense. Initially, when users are unaware of the event, their default emotion level is set to calm. However, as they become aware of the event, their emotional state begins to evolve. In order to capture this dynamic nature of emotions, we employ a Markov process. This process considers several factors, including the user's current emotion level, user profiles, user history, and the messages received at the present time step. By integrating these variables, we can predict the user's emotion level in the subsequent time step.

Our emotion simulation approach has yielded promising results at the individual level. As shown in Table~\ref{tab::ind1}, using real-world data for evaluation, our method demonstrates good performance in predicting the emotions of the next time step. We achieve an accuracy of 71.8\% in this three-classification task, thanks to the excellent modeling and understanding of human emotional expression by large language models.

\subsubsection{Attitude Simulation}
Just as emulating user emotions proves pivotal for social network simulations, simulating user attitudes carries equal weight. The reproduction of attitudes in a virtual social environment is complex yet indispensable. It is the combination of these attitudes that guide users' actions, opinions, and decisions about different topics. The challenge in this simulation lies in the multifaceted and subjective nature of attitudes, which are influenced by a wide range of internal and external factors, from individual experiences and beliefs to societal influences and perceived norms.

For our simulation, we assume that users have initial attitudes towards specific issues, which change based on unfolding events. This dynamic adaptation of attitudes is reflective of real-world social interactions, where people modify their views in response to changing circumstances, influential figures, or compelling arguments. 

In our model, much akin to the emotional state, we track the users' attitudes on a binary spectrum, which consists only of negative and positive stances towards an event. Our first step is to establish an initial state for the user's attitude. This is derived from the user profiles and user history, reflecting their predispositions based on past interactions and behaviors. Once the initial state is established, the dynamics of attitude changes are modeled as a Markov process. The subsequent evolution of these attitudes incorporates not only the user's current attitude but also their profile, history, and the messages received at the current time step. These factors are collectively employed to predict the user's attitude in the ensuing time step. Both the initial attitude and the assessment of attitude change are determined based on the LLM.

As depicted in Table~\ref{tab::ind1}, our methods have demonstrated excellent performance. In the task of predicting initial attitudes, our approach yields an accuracy of 74.3\%, an AUC score of 0.727, and an F1-Score of 0.834. In the subsequent task of attitude change prediction, our method performs even better, achieving an impressive accuracy of 83.9\%, an AUC score of 0.865, and an F1-Score of 0.857. These results can be largely attributed to the ability of LLMs to profoundly comprehend human behavior and cognition. Such understanding enables a sophisticated interpretation of user-generated content, resulting in a more accurate prediction of users' attitudes and their evolution over time.

\begin{table}[t!]
\centering
\caption{Performance of our system on conditional text generation tasks.}
\vspace{0.3cm}
\label{tab:ind2}
\begin{tabular}{ccc}
\hline
\textbf{Scenario}              & \textbf{Perplexity} & \textbf{Cosine Similarity} \\ \hline
\textbf{Gender Discrimination} & 19.289              & 0.723                              \\ \hline
\textbf{Nuclear Energy}        & 16.145              & 0.741                              \\ \hline
\end{tabular}
\end{table}

\subsubsection{Content-generation Behavior Simulation}
Within the realm of real-world social networks, users shape their content based on their prevailing attitudes and emotions towards distinct events. Emulating this content creation process is an essential, yet complex, aspect of social network simulations. Each piece of generated content acts as a mirror to the user's internal state and external influences, manifesting their individual perspective on the event at hand. The crux of the challenge is to encapsulate the wide array of expressions and styles that users employ to convey their sentiments, opinions, and reactions.

Leveraging the strengths of LLMs can significantly alleviate this challenge. These models, with their ability to generate text that closely resembles human-like language patterns, facilitate the simulation of user-generated content with high accuracy. By inputting the user's profile, along with their current attitude or emotional state, these models are capable of generating content that faithfully reproduces what a user might post in response to a particular event.

This approach, informed by the capabilities of large language models, enables us to craft a sophisticated simulation that mirrors the content generation process in real-world social networks. It thereby provides a nuanced understanding of how users' attitudes and emotions are reflected in their content, offering invaluable insights for the study of social dynamics.

As can be seen in Table~\ref{tab:ind2}, our methods yield impressive results. In the Gender Discrimination scenario, we achieved a Perplexity score of 19.289 and an average cosine similarity of 0.723 when compared with the actual user-generated text. In the case of the Nuclear Energy scenario, these figures were even more impressive, with a Perplexity score of 16.145 and an average cosine similarity of 0.741.

These results validate the effectiveness of our approach, where the LLM's profound comprehension of human cognition and behavior significantly contributes to accurately simulating user-generated content in social network simulations. Thus, our model serves as a powerful tool in understanding and predicting social dynamics in various contexts.

\subsubsection{Interactive Behavior Simulation}

During the simulation, upon receiving a message from one of their followees, the user is faced with a consequential decision: whether to engage in forwarding, posting new content or do nothing. 
Effectively modeling the decision-making process is important in simulating information propagation.

Through our data-driven approach, we utilize Large Language Models (LLMs) to simulate users' interaction behavior by capturing the intricate relationship between users and contexts. The input is the information environment that the user senses, and the LLM-empowered agent make the decision by learning from the observed real data. 

Our model has demonstrated commendable efficacy in this regard. In the scenario of Gender Discrimination, our model achieved an Accuracy of 66.2\%, AUC of 0.662, and F1-Score of 0.667. Progressing to the Nuclear Energy context, the model's performance remained robust, with an Accuracy of 69.5\%, AUC of 0.681, and F1-Score of 0.758.

These promising results not only attest to the LLM's capability in accurately simulating individual user behavior but also pave the way for exploring its potential at a larger scale. This accomplishment forms the basis for the population-level simulation, which we will delve into in the subsequent sections.

\subsection{Population-level Simulation}

In S$^3$, we capture three forms of propagation, including the propagation of information, emotion, and attitude. Here information propagation focuses on the transmission of news that describes events in social environments. Emotion propagation emphasizes the social contagion of people's feelings toward specific events or topics. Attitude propagation describes that people exchange their attitudes or viewpoints in the social network. Subsequently, we shall expound upon our comprehensive capacity to simulate these three aforementioned forms of propagation.

% Figure environment removed

% Figure environment removed

\subsubsection{Information Propagation}

With the widespread adoption of digital media, the propagation of information experiences a significant acceleration~\cite{lorenz2023systematic,luding2005information}. In the context of a simulation system designed to mimic social networks, one of its paramount functionalities lies in accurately modeling the process of information propagation and delineating crucial phase transitions~\cite{xie2021detecting,notarmuzi2022universality}. For example, Notarmuzi et al.~\cite{notarmuzi2022universality} conducted extensive empirical studies on a large scale, successfully distilling the concepts of universality, criticality, and complexity associated with information propagation in social media. Meanwhile,  Xie et al.~\cite{xie2021detecting} expanded upon the widely accepted percolation theory and skillfully captured the intricate phase transitions inherent in the spread of information on social media platforms.

Diverging from previous studies grounded in physical models, our approach adopts a LLM perspective to capture the dynamics of the information propagation process. In order to ascertain the efficacy of our proposed S$^3$ model, we have selected two typical events: (i) Eight-child Mother Event and (ii) Japan Nuclear Wastewater Release Event. The former event came to public attention in late January 2022, encompassing a range of contentious issues, such as sexual assault and feminism. The latter event entails Japan's government's decision to release nuclear wastewater into the ocean, eliciting significant global scrutiny and interest.

Utilizing our simulator as a foundation, we employ a quantitative approach to evaluate the temporal dissemination of the aforementioned occurrence. This is achieved by calculating the overall number of people who have known the events at each time step  (refer to Figure~\ref{fig::ssb} and Figure~\ref{fig::ssn}). Subsequently, through a comparative analysis with the empirical data (as illustrated in Figure~\ref{fig::tsb} and Figure~\ref{fig::tsn}), we discern that our simulator exhibits a commendable capacity for accurately forecasting the propagation patterns of both events. In particular, we notice that the rate of rise becomes gradually marginal over time, which can also be captured by our simulator.

\subsubsection{Emotion Propagation}

Another indispensable form of propagation is the transmission of emotion on social media~\cite{wang2022global,schafer2002spinning}. For example, Wang et al.~\cite{wang2022global} adopt the natural language processing techniques (BERT) and perform frequent global measurements of emotion states to gauge the impacts of pandemic and related policies. In S$^3$, we utilize the state-of-the-art LLM to extract emotions from real-world data and simulate the emotional propagation among LLM-based agents. 

To examine whether the S$^3$ simulator can also reproduce the emotion propagation process, we further simulate users' emotions expressed in the Eight-child Mother event. We extract the emotional density from the textual interactions among agents. Comparing our simulation results (Figure~\ref{fig::seb}) and the empirical observations (Figure~\ref{fig::teb}),  we find that our model can well capture the dynamic process of emotion propagation. Notably, we observe that there are two emotional peaks in the event. This suggests that if news of the event spreads more slowly across a larger community, a secondary peak in emotional intensity may occur. Based on the initialization obtained from real-world data, our model successfully reproduces these distinct peaks, thereby demonstrating  the effectiveness of our proposed S$^3$ system.

\subsubsection{Attitude Propagation}
One of today's most concerning issues is the polarization and confrontation between populations with diverging attitudes toward controversial topics or events. Great efforts have been made to quantify real-world polarization~\cite{lorenz2023systematic,flamino2023political,hohmann2023quantifying} and simulate the polarization process using co-evolution model~\cite{santos2021link,baumann2020modeling,baumann2021emergence,liu2023emergence}. In S$^3$, we use LLM to simulate propagation attitudes and predict polarization patterns in social networks.

Here we focus on the Japan Nuclear Wastewater Release Event, in which people's attitudes are polarized toward nuclear energy. As shown in Figure~\ref{fig::pn}, we can observe that with the propagation of related information, positive attitudes toward nuclear energy decline rapidly, exhibiting a salient trough. In our S$^3$ model, though modeling repeated interactions among agents, we reproduce the sudden decrease in positive attitudes and also capture their gradual increase. Overall, these observations suggest that our proposed model can not only simulate attitude propagation but also capture the critical dynamical patterns when situated in real-world scenarios.

\subsection{Comparative Evaluation}

To comprehensively evaluate the effectiveness of our S$^3$ system, we conduct comparative experiments against several baseline methods across three key propagation tasks: information propagation, opinion propagation, and emotion propagation.

\subsubsection{Information Propagation Comparison}

We compare our approach with two widely-used baseline models: the Linear Threshold (LT) model~\cite{chen2010scalable} and the Independent Cascade (IC) model~\cite{he2014minimum}. We employ two evaluation metrics: Mean Squared Error Difference (MSED) and Correlation Coefficient (Cor). MSED is defined as:

\begin{equation}
\text{MSED} = \frac{1}{T} \sum_{t=1}^{T} (S_t^{actual} - S_t^{predicted})^2
\end{equation}

where $S_t^{actual}$ and $S_t^{predicted}$ are the actual and predicted state values at time $t$, respectively.

\begin{table}[t!]
\centering
\caption{Performance comparison for information propagation simulation.}
\vspace{0.2cm}
\label{tab:info_prop}
\begin{tabular}{lccc}
\hline
\textbf{Method} & \textbf{Dataset} & \textbf{MSED} & \textbf{Cor} \\
\hline
LT & Nuclear Energy & 0.081 & 0.965 \\
IC & Nuclear Energy & 0.075 & 0.971 \\
S$^3$ & Nuclear Energy & 0.103 & 0.967 \\
\hline
LT & Gender Discrimination & 0.065 & 0.982 \\
IC & Gender Discrimination & 0.074 & 0.966 \\
S$^3$ & Gender Discrimination & 0.051 & 0.996 \\
\hline
\end{tabular}
\end{table}

As shown in Table~\ref{tab:info_prop}, our method achieves competitive performance compared to traditional models, with particularly strong results on the Gender Discrimination dataset where we achieve the lowest MSED and highest correlation.

\subsubsection{Opinion and Emotion Propagation Comparison}

For opinion and emotion propagation, we compare against five baseline methods: Voter~\cite{yildiz2010voting}, DeGroot~\cite{degroot1974reaching}, Feed-forward Neural Network (FNN), Sociologically-Informed Neural Network (SINN)~\cite{okawa2022predicting}, and Neural Dynamics on Complex Networks (NDCN)~\cite{zang2020neural}. These baseline methods require training data (50\% for training, 10\% for validation, 40\% for evaluation), while our approach operates in a zero-shot manner.

\begin{table}[t!]
\centering
\caption{Performance comparison for opinion and emotion propagation simulation.}
\vspace{0.2cm}
\label{tab:opinion_emotion}
\begin{tabular}{lcccc}
\hline
\multirow{2}{*}{\textbf{Method}} & \multicolumn{2}{c}{\textbf{Opinion Propagation}} & \multicolumn{2}{c}{\textbf{Emotion Propagation}} \\
\cline{2-3} \cline{4-5}
& \textbf{MSED} & \textbf{Cor} & \textbf{MSED} & \textbf{Cor} \\
\hline
Voter & 0.725 & -0.01 & 0.892 & -0.01 \\
DeGroot & 5.614 & 0.373 & 6.763 & 0.242 \\
FNN & 1.150 & 0.629 & 0.381 & 0.583 \\
SINN & 0.163 & 0.892 & 0.188 & 0.794 \\
NDCN & 0.060 & 0.882 & 0.060 & 0.828 \\
S$^3$ (Zero-shot) & 0.182 & 0.858 & 0.051 & 0.892 \\
\hline
\end{tabular}
\end{table}

Table~\ref{tab:opinion_emotion} demonstrates that our zero-shot approach achieves competitive performance with training-dependent baselines for opinion propagation and outperforms all baselines for emotion propagation, highlighting the effectiveness of our LLM-based approach.
