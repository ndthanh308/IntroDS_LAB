\section{Related Works}\label{sec::related}
In this section, we discuss two areas close to this work, social simulation and large language model-based simulation.

\subsection{Social Simulation}

According to~\cite{bratley1987guide}, "Simulation means driving a model of a system with suitable inputs and observing the corresponding outputs". Social simulation aims to simulate various social activities, which encompass a wide range of applications~\cite{gilbert2005simulation}.
One primary advantage of social simulation is its potential to aid social scientists in comprehending the characteristics of the social world~\cite{axelrod1997advancing}.
This is primarily attributed to the fact that the internal mechanisms driving social behaviors are not directly observable.
By employing a simulation model capable of reasonably replicating the dynamic nature of historical social behaviors, it becomes feasible to utilize the simulation tool for predicting the future of the social system.
Furthermore, social simulation can serve as a training ground, particularly for economists involved in social-economic simulations~\cite{spencer1984effect}. In this context, the economist can assume a digital persona, namely an artificial intelligence program tasked with formulating economic policies.
Moreover, social simulation can even serve as a substitute for human presence, exemplified by the emergence of digital avatars in the metaverse~\cite{lee2021all}.
From the perspective of social science research, social simulation plays a crucial role in facilitating the development of new social science theories. It achieves this by validating theoretical assumptions and enhancing theory through the application of more precise formalizations.


In spite of the promising applications, conducting social simulation is complex. The earliest works use discrete event-based simulation~\cite{kolesar1975simulation} or system dynamics~\cite{meadows1974dynamics, forrester1993system,marsh1978using} with a series of equations to approximate multiple variables over time that partly describe the system.
These early methods primarily focused on accurately predicting the variables rather than elucidating the underlying mechanisms or causal relationships. 
% Consequently, their contributions to social science research were limited.
Subsequently, drawing inspiration from the rapid development and remarkable success of simulation in other scientific domains, the utilization of agent-based simulation emerged in the field of social simulation. A notable and representative technique among these simulation methods is the employment of \textit{Cellular Automata}~\cite{chopard1998cellular}. Initially, this approach establishes a social environment composed of numerous individuals and subsequently formulates a set of rules dictating how individuals interact with one another and update their states.
Agent-based simulation can be regarded as a micro-level simulation that approximates real-world systems by describing the behavior of explicitly defined micro-level individuals. Thus, it is also referred to as microsimulation.


In recent times, owing to significant advancements in machine learning and artificial intelligence, agent-based simulation has witnessed a notable transformation. This transformation is characterized by the utilization of increasingly intricate and robust agents propelled by machine learning algorithms. These agents possess the ability to dynamically perceive their surroundings and exhibit actions that closely resemble human behavior.
The rapid progress in simulating individual agents has not only preserved the effectiveness of conventional simulation paradigms but has also resulted in significant improvements. This is particularly important for large language models, which are on the path towards achieving partial general artificial intelligence. Consequently, in this study, we embrace the microsimulation paradigm and employ meticulously guided and finely tuned large language models to govern the behavior of individuals within social networks. 



\subsection{Large Language Model-based Simulation}

Recently, relying on the strong power in understanding and generating human language, large language models such as GPT series~\cite{brown2020language,openai2023gpt4}, PaLM series~\cite{chowdhery2022palm, anil2023palm},  LLaMA~\cite{touvron2023llama}, GLM~\cite{zeng2022glm}, etc. are attracting widespread attention.

LLMs have exhibited exceptional capabilities in zero-shot scenarios, enabling rapid adaptation to diverse tasks across academic and industrial domains. The expansive language model aligns well with the agent-based simulation paradigm mentioned earlier, wherein the primary objective involves constructing an agent represented by a rule or program endowed with sufficient capacity to simulate real-world individuals.

Aher~\textit{et al.}~\cite{aher2023using} conducted a preliminary test to find that LLMs possess the capability to reproduce some classic economic, psycholinguistic, and social psychology experiments.
Horton~\textit{et al.}~\cite{horton2023large} substitute human participants with LLM agents, which are given endowments, information, preferences, etc., with prompts and then simulate the economic behaviors.
The results with LLM-empowered agents show qualitatively similar results to the original papers (with human experiments)~\cite{samuelson1988status,charness2002understanding}.
Another study ~\cite{hamalainen2023evaluating} adopts an LLM-based crowdsourcing approach by gathering feedback from LLM avatars representing actual humans, to support the research of computational social science.

Recently, Part~\textit{et al.}~\cite{park2023generative} construct a virtual town with 25 LLM-empowered agents based on a video game environment, in which the agent can plan and schedule what to do in daily life. Although the simulation is purely based on a generative paradigm without any real-data evaluation, it provides insights that LLM can serve as a powerful tool in agent-based simulation. Each agent was assigned its own identity and distinct characteristics through prompts, facilitating communication among them. It is noteworthy that this simulation was conducted exclusively within a generative paradigm, without incorporating any real-world data for evaluation. Nevertheless, the findings offer valuable insights into LLM's potential as a potent tool in agent-based simulations.
