\documentclass[10pt,sigconf,letterpaper]{acmart}
\settopmatter{printacmref=false} % Removes citation information below abstract
\renewcommand\footnotetextcopyrightpermission[1]{} % removes footnote with conference information in first column
\pagestyle{plain} % removes running headers

\usepackage[T1]{fontenc}
% T1 fonts will be used to generate the final print and online PDFs,
% so please use T1 fonts in your manuscript whenever possible.
% Other font encondings may result in incorrect characters.
%
\usepackage{graphicx}
% Used for displaying a sample figure. If possible, figure files should
% be included in EPS format.
%
% If you use the hyperref package, please uncomment the following two lines
% to display URLs in blue roman font according to Springer's eBook style:
%\usepackage{color}
%\renewcommand\UrlFont{\color{blue}\rmfamily}
%
%\usepackage{graphicx}
%\usepackage{times}  
\usepackage{hyperref}
%\usepackage{titlesec}
%\usepackage{blindtext}
\usepackage{xcolor}
%\usepackage{tikz}
%\usepackage{amsmath}
%\usepackage{enumitem}
%\usepackage{booktabs}
\usepackage[font=small,labelfont=bf, textfont=it]{caption}
\usepackage[hang,scriptsize,tight,nooneline]{subfigure}
%\usepackage{gensymb}
%\usepackage{outlines}
%\usepackage{dirtytalk}
%\usepackage{soul}

%\usepackage{floatrow}
%\usepackage{sidecap}
%\sidecaptionvpos{figure}{c}

\newcommand{\parab}[1]{\vspace{0.03in}\noindent{\bf #1}}
\newcommand{\starlinksubreddit}{\texttt{r/Starlink}}
\newcommand{\fix}{\textcolor{blue}}
\newcommand{\update}{\textcolor{blue}}
\newcommand{\aryan}{\textcolor{black}}
\newcommand{\venkat}{\textcolor{magenta}}
\newcommand{\venkatX}[1]{\textcolor{magenta}{XXX #1 XXX}}
%\hypersetup{pdfstartview=FitH,pdfpagelayout=SinglePage}
\newcommand{\cut}[1]{\st{#1}}

%\setlength\paperheight {11in}
%\setlength\paperwidth {8.5in}
%\setlength{\textwidth}{7in}
%\setlength{\textheight}{9.25in}
%\setlength{\oddsidemargin}{-.25in}
%\setlength{\evensidemargin}{-.25in}

\begin{document}
%
%\title{Contribution Title\thanks{Supported by organization x.}}
%\title{Perception Matters: Viewing ISP Networks\\through the Social Media Lens}
\title{\fontsize{18pt}{20pt}\selectfont On viewing SpaceX Starlink through the Social Media Lens}
%\subtitle{Paper $\#X$, $5$ pages $+ 1$ page of references}
%
%\titlerunning{Abbreviated paper title}
% If the paper title is too long for the running head, you can set
% an abbreviated paper title here
%
%\author{First Author\inst{1}\orcidID{0000-1111-2222-3333} \and
%Second Author\inst{2,3}\orcidID{1111-2222-3333-4444} \and
%Third Author\inst{3}\orcidID{2222--3333-4444-5555}}
%
%\authorrunning{F. Author et al.}
% First names are abbreviated in the running head.
% If there are more than two authors, 'et al.' is used.
%
%\institute{Princeton University, Princeton NJ 08544, USA \and
%Springer Heidelberg, Tiergartenstr. 17, 69121 Heidelberg, Germany
%\email{lncs@springer.com}\\
%\url{http://www.springer.com/gp/computer-science/lncs} \and
%ABC Institute, Rupert-Karls-University Heidelberg, Heidelberg, Germany\\
%\email{\{abc,lncs\}@uni-heidelberg.de}}
%\author{Paper $\#X$, $5$ pages $+ 1$ page of references}
%
%\maketitle              % typeset the header of the contribution
%
\author{Aryan Taneja, Debopam Bhattacherjee, Saikat Guha, Venkata N. Padmanabhan}
\affiliation{%
\vskip 0.5em
  \institution{Microsoft Research - India}
\vskip 1.5em
}

\begin{abstract}
\textit{Multiple low-Earth orbit satellite constellations, aimed at beaming broadband connectivity from space, are currently under active deployment. While such space-based Internet is set to augment, globally, today's terrestrial connectivity, and has managed to generate significant hype, it has been largely difficult for the community to measure, quantify, or understand the nuances of these offerings in the absence of a global measurement infrastructure -- the research community has mostly resorted to simulators, emulators, and limited measurements till now. In this paper, we identify an opportunity to use the social media `lens' to complement such measurements and mine user-centric insights on the evolving ecosystem at scale.}


\textit{To illustrate the broader opportunity here, we focus on SpaceX Starlink -- a mega-constellation that is set to eventually include tens of thousands of Low-Earth Orbit (LEO) satellites providing Internet connectivity from space. Tapping social media, we present our analyses on learning about Starlink's network events even \textit{weeks before} any public announcements on these, how some large-scale events have significantly impacted user sentiment, and how users' \aryan{perception} %expectation%
of Starlink performance has evolved over time. We discuss how the methodology presented here could enable a better, holistic understanding of these networks.}

%\keywords{Satellite Broadband  \and Low-Earth Orbit \and Social Media.}
\end{abstract}
%
%
%
\maketitle 

\section{Introduction}
Current quantum hardware is unable to carry out universal quantum computations due to the buildup of errors that occur during the computation. 
The magnitude of the individual error is currently above the value that the Threshold Theorem requires in order to kick-start quantum error correction and fault-tolerant quantum computation~\cite[Section 10.6]{nielsen_chuang_2010}. 
Although the experimentally achieved fidelity rates are promising and the error bounds are inching closer to the required threshold, we will have to work for the foreseeable future with quantum hardware with errors that build-up during the computation.  This implies that we can only do a limited number of steps before the output of the computation has become completely uncorrelated with the intended one.

For fault-tolerant quantum computing, we repeat four steps: 
1) We apply a number of single and two-qubit quantum gates, in parallel whenever possible; 
2) We perform a syndrome measurement on a subset of the qubits; 
3) We perform fast classical computations to determine which errors have occurred and how to correct them; 
and, 4) We apply correction terms based on the classical computations.
We then repeat these four steps with a next sequence of gates. 
These four steps are essential to fault-tolerant quantum computing. 


The starting point of this work is to use the four steps outlined above, not to carry out error correction and fault-tolerant computation, but to enhance short, constant-depth, {\em uncorrected} quantum circuits that perform single qubit gates and {\em nearest-neighbor} two qubit gates. 
Since in the long run we will have to implement error-correction and fault-tolerant computation anyhow, and this is done by such a four-step process, why not make other use of this architecture? Moreover, on some of the quantum hardware platforms, these operations are already in place.
Embracing this idea we naturally arrive at the question: what is the computational power of \textit{low-depth} quantum-classical circuits organized as in the four steps outlined above? 
We thus investigate circuits that execute a small, ideally constant, number of stages, where at each stage we may apply, in parallel, single qubit gates and {\em nearest-neighbor} two qubit gates, followed by measurements, followed by low-depth classical computations of which the outcome can control quantum gates in later stages. 
It is not clear, at first, whether such circuits, especially with constant depth, can do anything remotely useful. 
But we will see that this is indeed the case: many quantum computations can be done by such circuits in constant depth. 
By parallelizing quantum computations in this way, we improve the overall computational capabilities of these circuits, as we do not incur errors on qubits that are idle, simply because qubits are not idle for a very long time. 
Furthermore, reducing the depth of quantum circuits, at the cost of increasing width, allows the circuit to be run faster even if errors occur.

The first usage of such a four-step layout, not to do error correction, but to perform computations, can be found in the paradigm of measurement-based quantum computing~\cite{gottesman1999demonstrating,raussendorf2001one,jozsa2006introduction,clark2007generalised}: 
A universal form of quantum computing where a quantum state is prepared and operations are performed by measuring qubits in different bases, depending on previous measurements and intermediate measurements.

\citeauthor{PhamSvore2013} were the first to formalize the four-step protocol for performing computations~\cite{PhamSvore2013}. They included specific hardware topologies by considering two-dimensional graphs for imposing constraints on qubit interactions. In their model, they develop circuits for particularly useful multi-qubit gates, including specifying costs in the width, number of qubits, depth, number of concurrent time steps, size, and total number of non-Identity operations.
As a result, they find an algorithm that factors integers in polylogarithmic depth.
\citeauthor{Browne:2011} showed that the main tool in the work by \citeauthor{PhamSvore2013}, the fan-out gate, can also be replaced by additional log-depth classical computations in the measurement-based quantum computing setting~\cite{Browne:2011}.

More recently, \citeauthor{Cirac:2021} introduced a scheme to implement unitary operations involving quantum circuits combined with Local Operations and Classical Communication ($\mathsf{LOCC}$) channels: $\mathsf{LOCC}$-assisted quantum circuits~\cite{Cirac:2021}. Similarly to the four-step scheme we just described, they allow for a short depth circuit to be run on the qubits, followed by one round of $\mathsf{LOCC}$, in which ancilla qubits are measured and local unitaries are applied based on the measurement outcomes. They show that in this model any 1D transitionally invariant matrix-product state (MPS) with fixed bond dimension is in the same phase of matter as the trivial state. Similar ideas can be found in~\cite{TVV_NonAbelianTopologicalOrder_2022, tantivasadakarn2021long}.

In this work, we introduce a new model, called \textit{Local Alternating Quantum-Classical Computations} ($\LAQCC$). In this model we alternate between running quantum circuits (constrained by locality), ending in the measurement of a subset of qubits, and fast classical computations based on the measurement results. The outcome of the classical computations are then used to control future quantum circuits. We allow for flexibility in this model, by giving different constraints to the power of both the quantum circuits and the classical circuits as well as the number of alternations between them. 
Most attention will be given to $\LAQCC$ containing quantum circuits of constant depth, classical circuits of logarithmic depth and at most a constant number of alternations between them. 
Any circuit constructed in this model is considered to be of constant depth. 
We restrict ourselves to logarithmic depth classical computations, as this is the first natural and non-trivial extension beyond constant-depth classical computations. 
Constant-depth classical computations do however also have an equivalent constant-depth quantum implementation.

The definition of $\LAQCC$ sharpens the original definition of \citeauthor{PhamSvore2013} by adding constraints to the intermediate classical computations. This allows us to bound the power of $\LAQCC$ from above. 

The main result of \citeauthor{Cirac:2021}, that 1D translational invariant MPS with fixed bond dimension can be prepared by $\mathsf{LOCC}$-assisted circuits, relies on local symmetries of the MPS. These symmetries allow them to prepare local states (on a constant number of qubits) and glue them together by doing one round of the appropriate entangling measurement and corrections, after which they run a round of local unitaries to get the desired result. This general scheme for preparing states that exhibit an MPS description with the appropriate local symmetries requires only geometrically local unitaries and one round of measurement and corrections an therefore is accessible in $\LAQCC$. Studying different local symmetries, known as Symmetry Protected Topological (SPT) phases of matter, to find measurement-based constant depth circuits for states is a broad ongoing field of research~\cite{TVV_NonAbelianTopologicalOrder_2022, tantivasadakarn2021long, smith2023deterministic}. 
All these schemes have a $\LAQCC$ implementation.

%$\LAQCC$-circuits also exist for general schemes of preparing local states, based on the local tensors, and gluing them together using one round of entangled measurement and corrections, based on the local symmetry. 
%The main result of \citeauthor{Cirac:2021}, that 1D translational invariant MPS with fixed bond dimension can be prepared by $\mathsf{LOCC}$-assisted circuits, relies heavily on local symmetries of the MPS and as a result also has an equivalent $\LAQCC$ implementation. 
%The corrections applied after the measurement round are local unitaries depending on the local symmetries of the MPS. 

 

%This general scheme of preparing local states, based on the local tensors, and gluing it together by doing one round of entangled measurement and corrections, based on the local symmetry, is accessible in $\LAQCC$.
Note however that \citeauthor{Cirac:2021} also suggest a circuit for the $W$-state.
This circuit uses sequentially and dependent measurement-based corrections of the ancilla qubits. 
These dependent measurements translate to sequential alternations between the quantum and classical circuits and therefore increase the total depth to linear depth, exceeding the constant-depth constraints imposed by $\LAQCC$-circuits. 

We study the power of the $\LAQCC$ model with respect to state preparation, showing that even with only constant quantum-depth and logarithmic classical depth it remains possible to prepare states with long-range entanglement.
Another surprising result is that it is unlikely that $\LAQCC$ circuits are classically simulatable. We show that any instantaneous quantum polynomial-time (IQP) circuit~\cite{Bremner2010,Shepherd2009} has an $\LAQCC$ implementation.
Classical simulation of IQP circuits implies the collapse of the polynomial hierarchy to the third level, which is not believed to be true~\cite{Bremner2017}. Therefore, we expect that $\LAQCC$ circuits are unlikely to be classically simulatable. We bound the power of $\LAQCC$ by showing that it is contained in $\QNC^1$, the class of polynomial-size, log-depth circuits.

Next, we also study the power that intermediate classical calculations can add to quantum computations, by considering a new model that alternates between polynomially many polynomial-depth quantum circuits and unbounded classical computations
We study this model by doing a complexity theoretical analysis, where we draw inspiration from the notions of complexity given by \citeauthor{RosenthalYuen:2022}, \citeauthor{MetgerYuen:2023}, and \citeauthor{Aaronson:2004}.
All three complexity notions are based on the notion of state preparation, instead of more traditional definition of complexity such as the decidability of a computational problem. 
The first two consider classes based on sequences of quantum states preparable by a polynomial-sized quantum circuit, where the circuits are uniformly generated by a computational class, for instance, the class $\mathsf{PSPACE}$, which results in the complexity class $\mathsf{StatePSPACE}$~\cite{RosenthalYuen:2022,MetgerYuen:2023}.
The third notion considers a relative complexity, where the complexity is measured between two given states, and is measured by the number of gates, from a given gate-set, required to transform one state in another state~\cite{Aaronson:2004}. 
For our definition of state preparation complexity, we drop the uniformity constraint from~\cite{RosenthalYuen:2022,MetgerYuen:2023} and define a class as $\mathsf{StateX}$, which refers to states preparable by circuits of type $\mathsf{X}$. 
As an example, if $\mathsf{X} = \QNC^0$, this results in the class $\mathsf{StateQNC^0}$, which is the set of states preparable from the $\ket{0}^n$ state by poly-size constant-depth circuits. 
This notion is similar to the relative complexity from~\cite{Aaronson:2004}, where one state is the  $\ket{0}^n$ state and instead of counting the number of gates we consider the set of states preparable by a fixed number of gates. Using this notion of complexity we show that any state preparable by an $\LAQCC^*$ circuit is also preparable by a $\mathsf{PostQPoly}$ circuit, the class of circuits of polynomial depth with an additional post-selection gate. 

All Clifford circuits have a constant-depth $\LAQCC$ implementation, implying that any stabilizer state can be implemented by a constant-depth $\LAQCC$ circuit, see Section~\ref{sec:clifford_circuits} for a proof of this statement. 
Efficient circuits for stabilizer states have been known already through measurement-based quantum computing. Therefore this paper focuses on the preparation of non-stabilizer states, and as a surprising result we find novel constant-depth protocols for four very natural classes of non-stabilizer states.
Despite the extensive research into these four classes of non-stabilizer states and the many applications of them, no efficient constant- or low-depth state preparation protocols are known yet. We specifically consider these four classes as they are all often used as initial states in other algorithms.

The first state is a uniform superposition over an arbitrary number of states. 
This state finds applications in many quantum algorithms, as they often start with a uniform superposition over multiple states. 
This superposition is often achieved by applying Hadamard gates to every qubit due to its simplicity to prepare. 
Yet, the analysis of many algorithms, such as Shor's algorithm~\cite{Shor:1997}, would benefit from a different initial superposition. 
The circuit to prepare the uniform superposition over an arbitrary number of states uses an exact version of Grover search as a subroutine, that turns a probabilistic circuit, with a known constant probability of success, into a deterministic circuit. 
We use the circuit for preparing a uniform superposition over an arbitrary number of states as a subroutine in the next two quantum state preparation protocols. 

The second state is the $W$-state, the uniform superposition over all computational basis states of Hamming-weight~$1$, a natural long-ranged entangled state that displays a fundamentally nonequivalent type of entanglement from the Greenberger–Horne–Zeilinger state~\cite{WState:2000}, for which $\LAQCC$-type constant-depth circuits were previously known~\cite{PhamSvore2013, Cirac:2021}. 
The $W$-state is often used as benchmark for new quantum hardware~\cite{Haffner2005,Neeley2010,GarciaPerez:2021}. 
A novel way to prepare the $W$-state therefore gives a new way to benchmark different quantum devices with each other. 
A circuit for preparing the $W$-state was given in~\cite{Cirac:2021}, but this implementation requires sequentially alternating measurements followed by local unitaries, which in the $\LAQCC$ model is not considered to be of constant depth. 
We improve this protocol by giving an $\LAQCC$ implementation of the $W$-state, based on a compress-uncompress method that links the one-hot and binary encoding of integers.

The third state considered is the Dicke state, a generalization of the $W$-state, a superposition over all computational basis states with Hamming-weight $k$~\cite{Dicke:1954}. 
Dicke states have relevance in various practical settings.
For instance, for quantum game theory~\cite{zdemir2007}, quantum storage~\cite{Bacon_Compress:2006,Plesch:2010}, quantum error correction~\cite{ouyang2014permutation}, quantum metrology~\cite{toth2012multipartite}, and quantum networking~\cite{prevedel2009experimental}. 
Dicke states have been used as a starting state for variational optimization algorithms, most notably Quantum Alternating Operator Ansatz (QAOA)~\cite{Hadfield2019}, to find solutions to problems such as Maximum k-vertex Cover~\cite{Brandhofer2022,cook2020quantum}.
The ground states of physical Hamiltonians describing one-dimensional chains tend to show a resemblance to Dicke states such as states resulting from the Bethe ansatz, making them an ideal starting state when investigating the ground state behavior of these Hamiltonians~\cite{TDL_BetheAnsatzDerivation:2010,B_ExcitedStateQuantumPhaseTransitions:2013,DickeTransitions:2021}. 
For instance, the algorithm by \citeauthor{van2021preparing}, who give an algorithm to prepare the Bethe ansatz eigenstates of the spin-1/2 XXZ spin chain, starts by first preparing a Dicke state~\cite{van2021preparing}. 
A Dicke-state preparation protocol based on the compress-uncompress methodology used in the $W$-state furthermore finds applications in entanglement distillation, where the entanglement of a large state is concentrated on only a few qubits. 
Efficient deterministic circuits for preparing Dicke states have been proposed by \citeauthor{bartschi2019deterministic}~\cite{bartschi2019deterministic, bartschi2022deterministic_short_depth}. 
They provide a quantum circuit of depth $\mathO(k \log(\frac{n}{k}))$, allowing arbitrary connectivity, to prepare a Dicke state, which they conjecture to be optimal when $k$ is constant. 
In this work, we provide a constant-depth $\LAQCC$ circuit below their conjectured bound already for constant $k$. 
However, this does not directly disprove their conjecture, as we allow for intermediate measurements and classical computations. 
More significantly, we even construct constant-depth $\LAQCC$ circuits for $k = \mathO(\sqrt{n})$ greatly improving their bound.
This construction extends the compress-uncompress method for the $W$-state combined with additional subroutines. 

We continue with a log-depth state preparation protocol for the Dicke-state for arbitrary $k$. 
This protocol implements an efficient transformation between the factoradic number representation and the combinatorial number representation of a positive integer. 
The combinatorial number representation relates directly to the Dicke state. 
The provided efficient transformation between number representation systems might be of independent interest. 

We conclude by modifying our protocol for preparing a Dicke-state to a protocol that prepares quantum many-body scar states in constant-depth. 
These states have low entanglement and longer coherence times than states with similar energy density.
These characteristics make many-body scar states interesting to analyze and relevant within physics.
Many-body scar states appear for instance in the AKLT model~\cite{AKLT:1987,MRBAR:2018,MRB:2018} and different spin models~\cite{SI:2019,MOBFR:2020}.
Known methods for preparing these states have polynomial-depth~\cite{Gustafson:2023}, whereas our circuit has constant depth. 

% We conclude by studying the power that intermediate classical calculations can add to quantum computations. 
% In this study, we define a new model that relaxes constant-depth quantum circuits to polynomial depth quantum circuits, log-depth classical calculations to unbounded classical computations and a constant number of alternations to a polynomial number of alternations. 
% We call this model $\LAQCC^*$. 
% We study this model by doing a complexity theoretical analysis, where we draw inspiration from the notions of complexity given by \citeauthor{RosenthalYuen:2022}, \citeauthor{MetgerYuen:2023}, and \citeauthor{Aaronson:2004}.
% All three complexity notions are based on the notion of state preparation, instead of more traditional definition of complexity such as the decidability of a computational problem. 
% The first two consider classes based on sequences of quantum states preparable by a polynomial-sized quantum circuit, where the circuits are uniformly generated by a computational class, for instance, the class $\mathsf{PSPACE}$, which results in the complexity class $\mathsf{StatePSPACE}$~\cite{RosenthalYuen:2022,MetgerYuen:2023}.
% The third notion considers a relative complexity, where the complexity is measured between two given states, and is measured by the number of gates, from a given gate-set, required to transform one state in another state~\cite{Aaronson:2004}. 
% For our definition of state preparation complexity, we drop the uniformity constraint from~\cite{RosenthalYuen:2022,MetgerYuen:2023} and define a class as $\mathsf{StateX}$, which refers to states preparable by circuits of type $\mathsf{X}$. 
% As an example, if $\mathsf{X} = \QNC^0$, this results in the class $\mathsf{StateQNC^0}$, which is the set of states preparable from the $\ket{0}^n$ state by poly-size constant-depth circuits. 
% This notion is similar to the relative complexity from~\cite{Aaronson:2004}, where one state is the  $\ket{0}^n$ state and instead of counting the number of gates we consider the set of states preparable by a fixed number of gates. Using this notion of complexity we show that any state preparable by an $\LAQCC^*$ circuit is also preparable by a $\mathsf{PostQPoly}$ circuit, the class of circuits of polynomial depth with an additional post-selection gate. 

\paragraph{Summary of results}
\begin{itemize}
    \item We give a new definition of a computational model that captures the power of the four step process: applying a constant number of layers of one- and two-qubit gates; performing a syndrome measurement; perform a fast classical computation determining corrections; apply corrections. We call this model \emph{Local Alternating Quantum Classical Computations}, or $\LAQCC$ for short. In this model we bound the allowed quantum operations, intermediate classical calculations, and number of rounds separately. In Section~\ref{sec:LAQCC_model} we define this model and give a list of operations based on results from literature contained in this computational model. In some of these operations we explicitly use that we allow for multiple, but at most constant, rounds  of corrections.
    \item  We show show that there exist $\LAQCC$ circuits that can not be weakly simulated in Section~\ref{sec:IQP_in_LAQCC}. We further show that for every $\LAQCC$ circuit there exists a $\QNC^1$ circuit simulating it perfectly, in Section~\ref{sec:LAQCC_in_QNC1}.
    \item We introduce a new type computational complexity for preparing states and show that the extension of $\LAQCC$ where we allow a polynomial number of rounds and unbounded classical computation, is contained in $\mathsf{PostQPoly}$, the class of polynomial circuits with post-selection, in Section~\ref{sec:Complexity results}.
    \item We show a protocol to prepare the uniform superposition state of size $q$ in $\LAQCC$ using $\mathO(\ceil{\log_2(q)}^2)$ qubits in Section~\ref{sec:superposition_modulo_q}. 
    \item We show a protocol to prepare the $W_n$ state in $\LAQCC$ using $\mathO(n\log(n))$ qubits in Section~\ref{sec:W_state_in_LAQCC}.
    \item We show two ways of preparing the Dicke-$(n,k)$ state. The first method is in $\LAQCC$, works up to $k = \mathO(\sqrt{n})$, uses $\mathO(n^2\log(n))$ qubits, and is found in Section~\ref{sec:dicke:small_k}. The second method is in $\LAQCC\text{-}\mathsf{LOG}$ (an extension of $\LAQCC$ allowing for logarithmic number of alterations instead of constant), works for any $k$, uses $\mathO(\text{poly}(n))$ qubits, and is found in Section~\ref{sec:Dicke_in_LAQCC_LOG}. 
    \item We extend on our $\LAQCC$ method of generating Dicke-$(n,k)$ states for $k = \mathO(\sqrt{n})$ and show a protocol to generate many-body scar states for a particular Hamiltonian in $\LAQCC$ (Section~\ref{sec:many_body_scar}). 
\end{itemize}
Summarized in a table, we provide the following state generation protocols:
\begin{table}[htb]
\centering
\begin{tabular}{l|l|l|l}
\textbf{State description} & \textbf{Width} & \textbf{Depth} & \textbf{Implementation}\\
\hline 
Uniform superposition mod $q$: $\frac{1}{\sqrt{q}} \sum_{i = 0}^{q-1}\ket{i}$ & $\mathO(\ceil{\log^2 q})$ & $\mathO(1)$ & Section~\ref{sec:superposition_modulo_q}\\

$W$-state: $\frac{1}{\sqrt{n}}\sum_{i = 0}^{n-1}\ket{e_i}$ & $\mathO(n \log n)$ & $\mathO(1)$ & Section~\ref{sec:W_state_in_LAQCC}\\

Dicke-$(n,k)$, $k = \mathO(\sqrt{n})$: $\binom{n}{k}^{-1/2}\sum_{x \in \{0,1\}^n: |x| = k} \ket{x}$ &  $\mathO(n^2\log n)$ & $\mathO(1)$ 
&Section~\ref{sec:dicke:small_k}\\

Dicke-$(n,k)$: $\binom{n}{k}^{-1/2}\sum_{x \in \{0,1\}^n: |x| = k} \ket{x}$ & $\mathO(\text{poly}(n))$ & $\mathO(\log n)$ &Section~\ref{sec:Dicke_in_LAQCC_LOG}\\

QMBS: $\ket{S_k} = \frac{1}{k! \sqrt{\mathcal N(n,k)}}(Q^\dagger)^k \ket{\Omega}$ &  $\mathO(n^2\log n)$ & $\mathO(1)$  &  Section~\ref{sec:many_body_scar}
\end{tabular}
\caption{Summary of state preparation protocols given in this paper.}
\label{tab:sate_prep}
\end{table}
In the entry for the quantum many-body scar state $Q$ denotes the raising operator and $\mathcal N(n,k)=\binom{n-k-1}{k}$. 
Section~\ref{sec:many_body_scar} will provide more details on the variables and the implementation. 

\paragraph{Organization of the paper}
\noindent We first introduce relevant preliminaries in Section~\ref{sec:preliminaries}. 
In Section~\ref{sec:LAQCC_model} we formally define the class of Local Alternating Quantum-Classical Computations ($\LAQCC$). We also show that any Clifford circuit can be implemented in constant depth $\LAQCC$ (a result based on a result from measurement-based quantum computing~\cite{jozsa2006introduction}). 
This result allows us to give many useful multi-qubit gates and routines in Section~\ref{sec:gates_created_in_LAQCC}. 
Beyond that we show that constant depth $\LAQCC$ circuits are contained in $\QNC^1$ and that any $\mathsf{IQP}$ circuit has an $\LAQCC$ implementation.
We conclude this section with an analysis of a more powerful instantiation of $\LAQCC$ and show an inclusion with respect to the class $\mathsf{PostQPoly}$, which is the class of circuits of polynomial depth with one additional post-selection gate. 
In Section~\ref{sec:state_prep_in_LAQCC} we give $\LAQCC$ circuit implementations for preparing the uniform superposition over an arbitrary number of states, the $W$-state and the Dicke state up to $k = \mathO(\sqrt{n})$. We furthermore give a log-depth circuit implementation for preparing the Dicke state for any $k$. We conclude by showing a $\LAQCC$ circuit for generating many body scar states of a particular type of Hamiltonian.


%\vspace{-0.1in}
\section{Methodology}
\label{sec:methods}
%\vspace{-0.05in}

In this section, we briefly discuss the APIs and language and vision tools we use to collect, extract, and analyze publicly available information on the Reddit social platform. We also describe the data we analyze.

We used Reddit~\cite{reddit_api} and pushshift.io~\cite{pushshift_api} APIs to get all the post and comment data available on \starlinksubreddit{} for the entire period between Jan'$21$ and Dec'$22$ ($24$ months). We used the Python Reddit API Wrapper (PRAW)~\cite{praw} to access the Reddit APIs. While gathering the data, we cleaned it to get rid of user IDs and content that have been explicitly requested to be removed by users. The former step makes sure that our work does not raise any ethical concerns while the latter step conforms to data privacy guidelines. We only collected data available \texttt{publicly} on the Reddit platform.

We use the following language and computer vision tools for the analyses presented in this paper:
%\vspace{-0.1in}
\begin{itemize}
    \item Azure's Cognitive Services~\cite{azure_acs}: We use sentiment analysis and opinion mining APIs to quantify the sentiment of posts and comments. We also use ACS' optical character recognition (OCR) service to extract network measurement information from screenshots of speed-test reports shared by Redditors (Reddit users) on the platform.%\vspace{-0.1in}
    \item Natural Language Toolkit (NLTK)~\cite{nltk}: We use the libraries to filter out stop-words, extract $n$-grams, and create word clouds of posts and comments.
\end{itemize}

Rather than reinventing the wheel,
we rely on the already available language and vision tools. Our focus is on demonstrating how the capabilities of these tools could be leveraged to understand users' perceptions of networks better. The accuracy of our analyses depends on the accuracy of these tools.

% Figure environment removed

  
Nevertheless, we built a set of heuristics to extract network measurement metrics from the OCR output of semi-structured speed-test reports across all test providers. Fig.~\ref{fig:ookla_template} show two such templates from Ookla. Our relative pixel-distance-based heuristics could extract downlink and uplink speeds and latency (also jitter and packet loss rates, when available) along with the corresponding units, date and time information, network provider name, server location, etc. While we do not use speed-test location information in our current analyses, we could have used the server location as a proxy for client location, depending on the use case, as test providers like Ookla (present at $1$,$000+$ locations~\cite{ookla_server_sel}) usually pick a `closest'-on-the-network server for the measurements. The heuristics could parse all the different speed-test templates, including more complex table-structured ones (having multiple sub-reports together, like in Fig.~\ref{fig:ookla2}), which we could collect over the entire period. We discuss the related results in \S\ref{sec:bandwidth}.

While the readers already have a rough idea of the data by now, let us briefly list down the inputs to our analysis framework for clarity:
%\vspace{-0.05in}
\begin{itemize}
    \item Reddit posts: We gather post text, embedded URLs, photos, and videos, submission timestamp, post ID, and post metadata (number of comments and upvotes).%\vspace{-0.1in}
    \item Comments: We collect comment text, embedded URLs, photos, and videos, submission timestamp, comment ID, and metadata. We regenerate the comment tree for a post leveraging parent IDs in the comment metadata.%\vspace{-0.1in}
    \item Speed-test screenshots: Many people post screenshots of speed-tests (across providers) on online forums. We gather these screenshots for further analysis while getting post data.
\end{itemize}
\vspace{-0.1in}


\vspace{-0.1in}
\section{Events drive sentiment}
\label{sec:event_sentiment}
\vspace{-0.05in}

Redditors' sentiments on \starlinksubreddit{} are influenced by events. We systematically identify peaks of strong sentiment and automate the discovery of related events of interest. 

% Figure environment removed

For each day between Jan'$21$ and Dec'$22$, we analyze the sentiment of individual post content (text) using Azure's Cognitive Services.
The sentiment analysis service assigns $3$ different scores -- positive, negative, and neutral -- to each piece of text, which add up to $1$. We count the number of posts with strong positive ($\geq${}$0.7$) and negative ($\geq${}$0.7$) scores per day and plot them in Fig.~\ref{fig:sentiment_temporal}.

Our framework could discover sentiment peaks (day-wise bins), generate word clouds (using NLTK) across all posts and comments over a day for those peaks, and discover relevant news articles by searching online for the keywords (top $3$ uni-grams) appended with `Starlink' for the custom date. This pipeline enables the framework to annotate sentiment peaks with events that drive them. 

The top $3$ sentiment peaks, as shown in Fig.~\ref{fig:sentiment_temporal}, correspond to events of $3$ distinct flavors. On $9^{th}$ Feb'$21$, Redditors showed strong positive sentiment towards Starlink opening up pre-ordering of user terminals in the US, Canada, and UK~\cite{starlink_preorder}. On $24^{th}$ Nov'$21$, SpaceX's email~\cite{starlink_delay} to pre-order customers on delay in terminal delivery led to a negative sentiment peak. Fig.~\ref{fig:outage_wordcloud} shows the word cloud corresponding to the third highest peak ($22^{nd}$ Apr'$22$) which is driven by negative sentiment. The $3^{rd}$ most common word in the generated word cloud is `outage'. Interestingly, we could not find any relevant news though for this date, although Redditors from $14$ different countries (including $\sim${}$190$ reports from the US) confirmed an outage online. 

On further exploration, we could detect more such sentiment peaks which correspond to similar service outages, many of which were reported by the media~\cite{starlink_outage1,starlink_outage2,starlink_outage3}. Across such outages, negative sentiment (number of strongly negative posts and comments on outage as a fraction of all related posts and comments) varies by as much as $70\%$ depending on the nature of the outage. While some outages lasted for tens of minutes to hours, the one event that generated the most negative sentiment was on $10^{th}$ Mar'$21$ -- during the early beta-testing phase. Redditors reported~\cite{reddit_outage_thread_10Mar21} multiple short outages resulting in frequent call drops throughout the day, thus adding to their annoyance.

% Figure environment removed

% Figure environment removed

While the above approach (step $1$) is aimed at identifying sentiment peaks tied to various events, both networking and otherwise, we dive deeper and focus more on network service outages next (step $2$). We created a library of uni/bi-grams which are common in `outage' posts and comments identified following the above procedure. We manually verified that keywords in this library are meaningful in the specific context and got rid of outliers. We then identified threads with such keywords and negative sentiments (i.e., we filtered out threads with such keywords but positive or neutral sentiments) assuming outages to be `negative' events. Fig.~\ref{fig:outage_temporal} plots the day-wise frequencies of these keywords in such \textsl{negative} sentiment threads. $7^{th}$ Jan'$22$ and $30^{th}$ Aug'$22$ have the largest spikes of such keywords and correspond to large reported outages~\cite{starlink_outage7Jan22,starlink_outage30Aug22}. But more interestingly, there are numerous shorter peaks over time which correspond to local transient outages. Interestingly, in Fig.~\ref{fig:sentiment_temporal} we did not observe any sentiment peaks for the above $2$ dates when there were global outages because Redditors were also discussing other non-networking positive events that coincided with these network outages. The $2$-step process (leveraging specific keywords library and sentiment analysis) helps us zoom in on the particular event of interest -- network outage in this case.

OOkla's Downdetector~\cite{downdetector} also uses sentiment data from Twitter alongside direct reports, albeit the detailed methodology is not publicly available. Our framework, on the other hand, leverages a 2-step process, as detailed above, that could be used to analyze both networking and non-networking events of interest. Social platform-based methods could detect transient outages (probably due to network updates, misconfigurations, etc.) which might go undetected/unreported otherwise. In these initial days of LEO broadband offerings, understanding such transient downtimes is crucial in fixing performance bottlenecks and misconfigurations.

IODA~\cite{ioda_outage} relies on active probing and BGP data to detect Internet outages. While the methodology works well for the terrestrial Internet, it would be interesting to revisit the same in the context of LEO networks -- LEO terminals in a region might transiently have no satellites within the field of view and experience temporary downtime. We keep a deeper comparison with IODA for future work.



\vspace{-0.1in}
\section{Know before we know!}
\label{sec:roaming}
\vspace{-0.05in}


What are some of the most popular topics being discussed on \starlinksubreddit{} and what could we infer from them? For each month, we identify the list of popular posts and explore the topics of discussion. To do this, we rely on $2$ metrics -- the number of upvotes and the number of comments per post. Fig.~\ref{fig:post_popularity} plots the corresponding CDFs for the month of Feb'$22$. We observe that both the CDFs have long tails -- we pick popular posts which lie at the intersection of the $99^{th}$ percentiles of both the CDFs. For Feb'$22$, there are $1$,$898$ posts in total, and $8$ of them are identified to be popular based on the definition above. 

The most popular post is on the Ukraine war and the role of SpaceX Starlink in providing connectivity. We rather focus on the second most popular post which has $500+$ upvotes and $100+$ comments. We use NLTK to remove common stop words 
and create a word cloud of uni-grams for the post and the  comments on the post. Fig.~\ref{fig:roaming_wordcloud} shows that the most common uni-gram being discussed is `roaming'. As a next step, we generate a similar word cloud but this time with the bi-grams. The most common bi-gram is `roaming enabled'. We also analyze the sentiment scores of the specific post and the comments, and found them to be primarily \textsl{positive}, which increases our confidence on the fact that Starlink roaming might have been enabled in early February, $2022$.

Interestingly, while Redditors discussed Starlink roaming being functional on $23^{rd}$ Feb'$22$ already (we could also identify a smaller peak on the same topic on $15^{th}$ Feb), SpaceX officially notified~\cite{starlink_portability} Starlink subscribers on portability (roaming) mode only in May, $2022$ -- $3$ months down the line. Elon Musk, CEO of SpaceX, tweeted earlier~\cite{musk_roaming} on $3^{rd}$ March on mobile roaming being enabled, still $\sim${}$2$ weeks after we could detect the earliest signal from \starlinksubreddit{}. So, our framework could automatically detect that roaming have been enabled on the Starlink network way before any public announcements from SpaceX on the same.

Our methodology above could be used for different use cases. For example, ISPs could use such a tool to keep track of competition and be on the forefront of innovation. Also, while exploring the subreddit manually, we observed posts~\cite{reddit_link1,reddit_link2,reddit_link3,reddit_link4} which discuss deep technical aspects.
An extension to our methodology could allow ISPs to pinpoint such public discussions and learn from community expertise. As social networks get all the more ingrained in our lives, and with more people becoming comfortable discussing ideas and reporting facts online, ISPs could probably benefit from tapping into the public whiteboard discussions on social platforms. 

Also, at these early stages of LEO broadband, automatic identification of such events will allow potential customers, researchers, and Internet enthusiasts to keep track of all the features and their evolution over time. A potential customer in a remote area who `needs' the roaming capabilities might benefit from such early information on the same.
\aryan{We also observed Redditors discussing~\cite{orion1,orion2,orion3,orion4,orion5,orion6,orion7} performance bottlenecks (and opportunities) of applications, like video conferencing, over LEO broadband. These social media insights could allow application providers to pinpoint issues over these new networks and fix them early toward offering improved experience.}

\vspace{-0.1in}
\section{Following the shifting fulcrum}
\label{sec:bandwidth}
\vspace{-0.05in}

Social media users share photos and screenshots while posting online. Redditors on \starlinksubreddit{} often share screenshots (or links to them) of network performance test reports to spark discussions. We gather all such test report screenshots across test providers like Ookla~\cite{ookla_speedtest}, Fast (powered by Netflix)~\cite{fast_speedtest}, Starlink itself, and others, and extract uplink speed, downlink speed, latency information, etc. using Azure's Optical Character Recognition (OCR)~\cite{azure_ocr}. After applying a set of heuristics, as discussed in \S\ref{sec:methods}, to rule out false positives, we could identify $\sim${}$1$,$750$ reports of Starlink speed-tests being shared publicly between Jan, $2021$ and Dec, $2022$. In this section, we focus on the evolution of `observed' downlink speed during this period and users' perceptions of the same.

%\vspace{-0.2in}
% Figure environment removed


\parab{Demand vs. supply}
Fig.~\ref{fig:downlink_evolve} shows the change in observed downlink speeds with time. For each month, we plot the median speeds across all shared screenshots of Starlink speed tests. We annotate the observed speeds with the number of Starlink launches~\cite{starlink_launches} and also the reported number of Starlink users (whenever public information is available)~\cite{beta_10K,beta_69K,beta_90K,beta_140K,beta_250K,beta_400K,beta_500K} during a month. We also plot the monthly median downlink speeds with $95\%$ and $90\%$ of the monthly speed data picked uniformly at random -- the plots closely follow each other showing that the observed medians are considerably stable.


We observe that, between Jan and Sep'$21$, the median downlink speeds increased in general. SpaceX launched ($\sim${}$60$ satellites per launch) $14$ times during this period and the number of users increased from $10K$ (in Feb) to $90K$ (in Aug).  Further, between Sep'$21$ and Dec'$22$, there has been an almost steady decrease in observed speeds although SpaceX launched batches of Starlink satellites $32$ times. Note, however, that the number of reported Starlink users increased from $90K$ to $1M$ (and beyond) during the same period, resulting in more than $1000\%$ rise in downlink demand, assuming a linear increase in demand with users. Note that between Nov'$21$ and Jan'$22$, there were $4$ Starlink launches while the number of users increased by only $5K$ (announcement by Starlink on delay in terminal delivery in Nov'$21$, as shown in Fig.~\ref{fig:sentiment_temporal}). This resulted in slight improvements in observed bandwidth in Jan'$22$.
In March, April, and June'$21$, observed downlink speeds improved as new satellites were deployed steadily. But it is not always a straightforward reflection of deployment as more Starlink users are also added continually, as is evident in some other cases. On the contrary, if there aren't any new launches, and new users are added, the observed speeds decrease. Between Jun and Aug'$21$, $21K$ new users started using Starlink with no new launches happening. This is reflected in the sharp decrease in median speeds during the period. Beyond Sep'$21$, reported bandwidths have decreased almost steadily given the large increase in demand as Starlink service expanded to various countries across the globe.

The broad observation here is that more Starlink launches do not always result in higher observed bandwidth, although the aggregate bandwidth of the Starlink system should be increasing with the addition of satellites. It is a complex calculus involving both supply and demand, and it is always a race to add more satellites and cater to the ever-increasing aggregate bandwidth demands.


We did not quantify any bias inherent in such social media-based estimates. Note that unbiased absolute values of download speeds are not critical for our analyses; we rather needed relative changes in download speeds and corresponding sentiments (discussed next) from one month to the next. As part of future work, any bias arising due to demographics in a social community will hopefully reduce as we span across more social platforms. We discuss this in further depth under \S\ref{sec:discussions}.


\parab{The wheel of time}
With time, the perception of users on network performance changes. In Fig.~\ref{fig:downlink_evolve} we also plot (green, dashed) the strong positive sentiment of users on downlink speeds. To do this, we analyze the sentiment of posts (text content) that share Starlink speed-test reports using Azure's Cognitive Services. We identify posts with strong positive ($\geq${}$0.7$) or negative ($\geq${}$0.7$) scores, and define the normalized strong positive sentiment score ($Pos$) as the number of strong positive posts over sum of the number of strong positive and negative posts in a month thus filtering out edge cases when identifying the sentiment is hard.


We observe that $Pos$ broadly follows the observed downlink speed trends, but there are interesting exceptions. For instance, while downlink speed is higher in Dec'$21$ than Apr'$21$, $Pos$ is drastically lower for Dec'$21$. We believe this is because user sentiment is, in general, a reflection of both short-term and long-term conditioning -- users get acclimatized to their current network conditions and give negative sentiment for any degradation in network conditions even if such conditions are better than the past. The exact inverse of this trend is also visible -- the downlink speeds decrease between Mar'$22$ and Dec'$22$ while the $Pos$ improves over time. This demonstrates users getting conditioned to lower speeds, but not necessarily attachment and loyalty to the ISP. While we observed users frequently discussing application performance over LEO networks on social platforms, we keep such an in-depth and finer-granularity study of observed bandwidth versus user sentiment for future work.

%\vspace{-0.1in}
\section{Broader implications}
\label{sec:applications}
%\vspace{-0.05in}

While we have focused on SpaceX Starlink and the corresponding subreddit \starlinksubreddit{} in this preliminary study, our methodology and framework could span multiple online social communities and LEO providers to collect interesting insights in the absence of large-scale measurement platforms. Even beyond LEO, such a framework allows us to revisit our perception of Internet measurement.

\parab{User-centric} For relatively new offerings like LEO broadband Internet, it is critical to assess user perception and engagement to gauge the market demand and dynamics. How do the customers react and express themselves online in response to broad events such as infrastructure upgradation, outage, peering and partnership announcements, etc.? An NLP and vision services-enabled framework like this could potentially tap into the large corpus of user feedback, discussions, and debate publicly available on various online social platforms. 

\parab{Complementarity to measurement testbeds} While platforms like PlanetLab~\cite{planetlab} and M-Lab~\cite{MLab} capture Internet measurements at scale over years, LEO broadband networks with all the uniqueness and orbital dynamics also need to be measured at such global scales. But even if such an LEO measurement platform comes into existence, insights from social media capturing user perception and sentiment could complement such measurements, offering a more holistic view of these networks at the early stages of deployment.


\parab{Active user engagement} One could take a step further and engage actively with users on social platforms. It should be straightforward to share back the insights at both aggregate and granular levels with users. For example, when a user uploads a speed test screenshot, the framework could employ bots that allow the user to compare their experience with that of others. A user can opt out of such a service at will and could also request the framework to delete their data. If a user rather wants to opt in, there could be different levels of user engagement -- allowing the framework to collect more data (e.g. physical coordinates of user terminals) and run more speed tests.
Enthusiastic users could also be willing to share some of their resources for running arbitrary network experiments on a slice. A global LEO measurement testbed would need spatiotemporal diversity, given the unique dynamicity and geometry of the constellations, and benefit from such large-scale voluntary participation.


%!TEX root = ../Schur indices and line operators.tex


\section{Discussion}












\section{Related Work}
%\subsection{Cost Volume based Deep Stereo Matching}
%Stereo matching is a typical problem that has been studied for decades and a well-known four-step pipeline \cite{scharstein2002taxonomy} has been established, where cost volume construction is an indispensable step. Current state-of-the-art stereo matching methods are all cost volume based methods and they can be categorized into two types. Typically, a cost volume is a 4D tensor of height, width, disparity, and features. The first category just uses a full correlation to generate a single-feature cost volume. Such methods are usually efficient but lose much information because of the decimation of feature channels. Many previous work, including Dispnet \cite{dispnet}, MADNet \cite{madnet}, IResNet \cite{iresnet} and AANet \cite{aanet}, belong to this category. The second category usually uses concatenation \cite{gcnet} or group-wise correlation \cite{gwcnet} to generate a multi-feature 4D cost volume. Such a method can achieve better performance while requiring higher computational complexity and memory consumption. Actually, a majority of the top-performing networks in public leaderboards belong to this category, such as GANet \cite{ganet}, CSPN \cite{cspn} and ACFNet \cite{acfnet}. These methods generally employ multiple 3D convolution layers to constantly regularize the 4D cost volume and then apply softmax over the disparity dimension to produce a discrete disparity probability distribution. The final predicted disparity is obtained by softly weighting indices according to their probability, which is also called soft argmin in GCNet \cite{gcnet}. However, soft argmin leaves the output susceptible to multi-modal disparity probability distributions. ACFNet \cite{acfnet} observes this problem and proposes to directly supervise the cost volume with unimodal ground truth distributions. In contrast, we define an uncertainty estimation to quantify the degree to which the cost volume tends to be multi-modal distribution, higher implies the higher possibility of estimation error.

\subsection{Multi-scale Cost Volume based Stereo Matching}
Cost volume construction is an indispensable step in the well-known four-step pipeline for stereo matching \cite{scharstein2002taxonomy, pamisurvey1, pamisurvey2}. Typically, current state-of-the-art stereo matching methods can be categorized into two types of cost volume-based methods, where the cost volume is a 4D tensor of height, width, disparity, and features. The first category usually uses the single-feature 3D cost volume generated by full correlation, which is efficient while losing much information due to the decimation of feature channels. Many real-time methods, such as Dispnet \cite{dispnet}, MADNet \cite{madnet, madnet_pami} and AANet \cite{aanet}, belongs to the category. Moreover, two-stage refinement \cite{mcvmfc} and pyramidal towers \cite{madnet} are commonly applied in the single-feature cost volume based network to construct multi-scale cost volume. The second category usually uses the multi-feature 4D cost volume generated by concatenation \cite{gcnet} or group-wise correlation \cite{gwcnet}, which can achieve better performance with higher computational complexity and memory consumption. Most top-performing networks, including GANet \cite{ganet}, CSPN \cite{cspn} and ACFNet \cite{acfnet} belong to this category. 
% In these methods, the 4D cost volume is constantly regularized by multiple 3D convolution layers and then a discrete disparity probability distribution can be produced by softmax. Next, the final predicted disparity can be obtained by softly weighting indices according to their probability \cite{gcnet}. However, such output is susceptible to multimodal disparity probability distributions and ACFNet \cite{acfnet} gives a solution by directly supervising the cost volume with unimodal ground truth distributions to alleviate this problem. 
Recently, to alleviate the high computational complexity and memory consumption when employing multi-feature 4D cost volumes, \cite{cvpmvsnet, cascade, uscnet} propose to use cascade cost volume representation in multi-view stereo. These methods usually first predict an initial disparity at the coarsest resolution of the image and then gradually refine the disparity by narrowing down the disparity search space. More closely related to our approach is Casstereo \cite{cascade}, which first extended such representation to stereo matching. It selected to uniform sample a pre-defined range to generate the next stage’s disparity search range. Instead, we employ pixel-level uncertainty estimation to adaptively adjust the next stage disparity searching range and generate pseudo-labels for subsequent domain adaptation. Our method also shares similarities with UCSNet \cite{uscnet}, which constructs uncertainty-aware cost volume in multi-view stereo while it doesn’t employ uncertainty estimation to generate pseudo-labels.

%\subsection{Multi-scale Cost Volume based Deep Stereo Matching} 
% \subsection{Multi-scale Cost Volume based Stereo Matching} 
%Multi-scale cost volume firstly was applied in the single-feature cost volume based network with the form of two-stage refinement \cite{mcvmfc} and pyramidal towers \cite{madnet}. Recently, cascade cost volume representation \cite{cvpmvsnet, cascade, uscnet} was proposed in multi-view stereo to alleviate the high computational complexity and memory consumption when employing multi-feature 4D cost volumes. These methods generally predict an initial disparity at the coarsest resolution of the image. Then, they will narrow down the disparity search space and gradually refine the disparity. More closely related to our approach is Casstereo \cite{cascade}, which first extended such representation to stereo matching. It selected to uniform sample a pre-defined range to generate the next stage’s disparity search range. Instead, we employ uncertainty estimation to adaptively adjust the next stage pixel-level disparity searching range and push the next stage's cost volume to be predominantly unimodal.

% The single-feature cost volume based network with the form of two-stage refinement \cite{mcvmfc} and pyramidal towers \cite{madnet} first employ multi-scale cost volume for stereo matching. Recently, to alleviate the high computational complexity and memory consumption when employing multi-feature 4D cost volumes, \cite{cvpmvsnet, cascade, uscnet} propose to use cascade cost volume representation in multi-view stereo, which generally predict an initial disparity at the coarsest resolution of the image. Then, the disparity search space is narrowed down and the disparity is gradually refined. More closely related to our approach is Casstereo \cite{cascade}, which first extended such representation to stereo matching. It selected to uniform sample a pre-defined range to generate the next stage’s disparity search range. Instead, we employ uncertainty estimation to adaptively adjust the next stage pixel-level disparity searching range and push the next stage's cost volume to be predominantly unimodal.

% Figure environment removed

\subsection{Robust Stereo Matching} 
There exist three categories of generalization definitions for robust stereo matching. 1) Cross-domain Generalization: the network’s ability to perform well on unseen scenes (cannot see the image pairs of the target domain in advance). Towards this end, Jia et al \cite{sungeneralizaiton} propose to incorporate scene geometry priors into an end-to-end network. Zhang et al \cite{dsmnet} introduce a domain normalization and a trainable non-local graph-based filter to construct a domain-invariant stereo matching network. 2) Adapt Generalization: the network’s ability to adapt pre-trained models to the new domain with unlabeled target data. Previous work usually pre-trains the models on synthetic data and then adapts it to new target domains with Graph Laplacian regularization \cite{zoom}, non-adversarial progressive color transfer \cite{adastereo}, and Knowledge Reverse Distillation \cite{aohnet}. More closely related to our approach are \cite{aohnet, unsuperviseddomainadaptation} in stereo matching and Monoresmatch \cite{monoresmatch} in monocular depth estimation, which also proposes to generate a pseudo-label for domain adaptation. However, these methods all select to employ classical stereo matching methods \cite{sgm} alongside with confidence estimators, e.g., left-right consistency check to generate pseudo-labels. That is all these methods need an independent method to generate corresponding pseudo-labels. Instead, the proposed method is an end-to-end network that can generate the predicted disparity map, corresponding uncertainty map and pseudo-labels jointly, which is a more simple, yet efficient way. 
% Instead, our proposed method can employ pixel-level and area-level uncertainty estimation to self-distill the predicted disparity maps of our pre-training model and generate sparse while reliable pseudo-labels to align the domain gap, which is a more simple, yet efficient way. 
3) Joint Generalization: the network’s ability to perform well on a variety of datasets with the same model parameters. MCV-MFC \cite{mcvmfc} introduces a two-stage finetuning scheme to achieve a good trade-off between generalization and fitting capability on multiple datasets. However, it doesn’t touch the inner difference between diverse datasets, e.g, the unbalanced disparity distribution. To further address this problem, we propose a cascade cost volume to adaptively the next stage disparity searching space, where the pixel-level uncertainty estimation is at the core.

% \subsection{Monocular Depth Estimation}
% Monocular depth estimation aims to estimate depth values from a single image, instead of stereo images or multiple frames in a video. This problem is ill-posed because of the ambiguity of object sizes. However, humans could estimate the depth from a single image with prior knowledge of the scenes. Recently, learning based methods were explored to learn depth values by supervised or unsupervised learning. Eigen et al. first employed Convolutional Neural Networks (CNN) to predict depth in a coarse-to-fine manner and further improved its performance by multi-task learning. Liu et al. presented deep convolutional neural fields model by combining deep model with continuous CRF. Li et al. [22] refined deep CNN outputs with a hierarchical CRF. Multi-scale continuous CRF was formulated into a deep sequential network by Xu et al. [45] to refine depth estimation. Unsupervised methods tried to train monocular depth estimation with stereo
% image pairs or image sequences and test on single images. Garg et al. [9] used novel image view synthesis loss to train a depth estimation network in an unsupervised way. Godard et al. [11] introduced left-right consistency regularization to improve the performance of view synthesis loss. Recently, some work also propose to use the stereo matching network as a proxy to learn depth from synthetic data or directly employ traditional stereo matching methods to distill proxies labels from the target domain, which proves the feasibility of distilling stereo matching networks to learn monocular depth estimation.




%\vspace{-0.2in}
\section{Conclusion}
%\vspace{-0.1in}

We leverage recent advances in language and vision capabilities to mine social media and understand users' perceptions of SpaceX Starlink network performance and events. This framework could complement LEO broadband measurement tools by offering a user-centric view of these networks. Our framework could detect important network events days/weeks before public announcements, detect sentiment peaks and related events (like outages), and understand the evolving perception of bandwidth on the SpaceX Starlink network.

%
% ---- Bibliography ----
%
% BibTeX users should specify bibliography style 'splncs04'.
% References will then be sorted and formatted in the correct style.
%
% \bibliographystyle{splncs04}
% \bibliography{mybibliography}
%
\bibliographystyle{splncs04}
\bibliography{pam22} 
\end{document}
