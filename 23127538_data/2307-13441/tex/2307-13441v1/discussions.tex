%\vspace{-0.15in}
\section{Discussions}
\label{sec:discussions}
%\vspace{-0.1in}


\parab{AI for user sentiment} Generative AI models~\cite{gpt3, llama,openai2023gpt4} could effectively summarize and quantify user feedback and sentiment while removing harmful/biased content. Custom AI models that could understand the user pulse on social media at scale could augment the insights generated using cutting-edge vision and sentiment analysis tools.

\parab{The social network behavioral bias:} We did not analyze the bias that might occur due to a divergence between the experience of users and what they choose to report on social networks. For example, a user might be running network speed tests multiple times and reporting only the better/worse ones. Another bias might arise due to users in a public group being over-enthusiastic about the topic. For example, in the case of Reddit, discussions on \starlinksubreddit{} could have a stronger negative sentiment toward Starlink's competitors. Also, bias might arise due to the socio-demographics of users active on a particular social platform~\cite{hargittai2020potential}. As part of future work, we plan to expand across multiple social platforms and communities and combine social media data with other sources of information to quantify/mitigate this bias, if existent.

\parab{Gaming the framework:} Different stakeholders including users and ISPs could game the framework. For example, automated bots on Reddit could publish posts with strong sentiments to skew the aggregate analyses. We could rely on ongoing research on detecting bot-generated text and bot behavior~\cite{alothali2018detecting,schuchard2019bot,shi2019detecting,beskow2018bot} to tackle this problem. AI/ML-based techniques~\cite{luceri2020detecting,al2022online} to detect large-scale trolls on social media could also be plugged in to make the framework more robust to such gaming. One problem though is that on a specific social community, like \starlinksubreddit{}, the number of interactions is not too high volume, thus making it vulnerable to gaming. Nevertheless, gathering data across multiple platforms and communities might address this issue.

\parab{Tackling ethical and privacy concerns:} While we do not use user identification information in this work, one might extend the system to engage more closely with users, as discussed in \S\ref{sec:applications}. In such cases, we should carefully consider the ethical and privacy concerns that might arise while dealing with user-level data. Data should be anonymized, and user consent should precede active user engagement. One should also have a mechanism in place to permanently delete user data upon request, conforming to privacy regulations such as GDPR~\cite{gdpr}, CCPA~\cite{ccpa}, etc.

\parab{The more the better:} In this preliminary study, we only analyze data from a single platform (Reddit) and for a single ISP (Starlink). The framework could be extended to also analyze other social platforms such as Facebook, Twitter, Discord, etc., and other ISPs. The more data and the more diverse group culture explored, the easier it would be to identify sentiment peaks and important network (and non-network) events. Also, more data across groups with related but slightly different agendas might reduce the behavioral bias discussed above.

\parab{Location-based analysis (LEO specific)} More data on speed-test reports across test providers and social platforms could allow us to analyze the impact of constellations' geometry, Earth's shape, cell congestion, interference with GEO satellites, etc. on Starlink performance. This might require users to engage more with the framework and share explicit location information. 


