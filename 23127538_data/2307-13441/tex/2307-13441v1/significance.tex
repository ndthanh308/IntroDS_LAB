%\vspace{-0.1in}
\section{Broader implications}
\label{sec:applications}
%\vspace{-0.05in}

While we have focused on SpaceX Starlink and the corresponding subreddit \starlinksubreddit{} in this preliminary study, our methodology and framework could span multiple online social communities and LEO providers to collect interesting insights in the absence of large-scale measurement platforms. Even beyond LEO, such a framework allows us to revisit our perception of Internet measurement.

\parab{User-centric} For relatively new offerings like LEO broadband Internet, it is critical to assess user perception and engagement to gauge the market demand and dynamics. How do the customers react and express themselves online in response to broad events such as infrastructure upgradation, outage, peering and partnership announcements, etc.? An NLP and vision services-enabled framework like this could potentially tap into the large corpus of user feedback, discussions, and debate publicly available on various online social platforms. 

\parab{Complementarity to measurement testbeds} While platforms like PlanetLab~\cite{planetlab} and M-Lab~\cite{MLab} capture Internet measurements at scale over years, LEO broadband networks with all the uniqueness and orbital dynamics also need to be measured at such global scales. But even if such an LEO measurement platform comes into existence, insights from social media capturing user perception and sentiment could complement such measurements, offering a more holistic view of these networks at the early stages of deployment.


\parab{Active user engagement} One could take a step further and engage actively with users on social platforms. It should be straightforward to share back the insights at both aggregate and granular levels with users. For example, when a user uploads a speed test screenshot, the framework could employ bots that allow the user to compare their experience with that of others. A user can opt out of such a service at will and could also request the framework to delete their data. If a user rather wants to opt in, there could be different levels of user engagement -- allowing the framework to collect more data (e.g. physical coordinates of user terminals) and run more speed tests.
Enthusiastic users could also be willing to share some of their resources for running arbitrary network experiments on a slice. A global LEO measurement testbed would need spatiotemporal diversity, given the unique dynamicity and geometry of the constellations, and benefit from such large-scale voluntary participation.

