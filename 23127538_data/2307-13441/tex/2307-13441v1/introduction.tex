\section{Introduction}
\label{sec:intro}

Low-Earth Orbit (LEO) satellite networks like SpaceX Starlink~\cite{starlink_40K}, OneWeb~\cite{oneweb}, Amazon's Kuiper~\cite{amazon_news}, Telesat's Lightspeed~\cite{telesat}, and others are set to revolutionize the landscape of global connectivity. Multiple of these LEO satellite constellations are currently under deployment and SpaceX has already started offering Starlink Internet services in $58$ countries~\cite{beta_500K} as of June, $2023$. They already have more than $4$,$000$~\cite{starlink_count_new} satellites deployed and their long-term plan~\cite{starlink_40K} is to deploy $40K+$ LEO satellites. Recent advances in launching~\cite{reusable_boosters,10x_launch_cost_reduction} and satellite design~\cite{satellite_size} have enabled such large-scale deployments consisting of thousands of satellites. While such deployments have managed to generate significant hype~\cite{leo_popular_1,leo_popular_2,leo_popular_3,leo_popular_4,leo_popular_5}, %among the potential customers, media~\cite{}, and the research community~\cite{}, 
there is still a serious dearth of tools that could measure these networks and quantify their performance at scale. Although the first steps have been taken to simulate~\cite{kassing2020exploring} and emulate~\cite{lai2023starrynet} LEO networks and there have been reports on limited network measurement experiments~\cite{starlink_perf_firstlook,apnic_leo_cc}, the broader community, which includes potential customers, ISPs (read competitors), researchers, and Internet enthusiasts, is still in search of a platform that gives a holistic picture of this `space'. In this work, we explore if relevant information readily available on various social media could be mined to fill this gap in understanding LEO networks.

Social platforms like Facebook, Reddit, LinkedIn, Instagram, and even newer ones like Discord have seen large increases~\cite{facebook_stat,discord_stat} in user base and online communities over the last decade. Reddit, which is a leading platform for topic-based discussion, now hosts more than $100$,$000$ active communities and generates more than $350$ million posts a year~\cite{reddit_stat}. Such tremendous growth of social media has also driven, to a large extent, a lot of active research on sentiment analysis and opinion mining. As more users share their personal experiences and express their feelings online, such language capabilities could be leveraged to distill subjective product feedback across online forums~\cite{liu2012sentiment,anto2016product}. While on one hand, the language capabilities have improved significantly~\cite{devlin2018bert,dai2019transformer,yang2019xlnet,liu2019roberta,qi2020stanza,azure_acs,google_NL_AI} in the last few years, on the other hand, these tools have also been applied to diverse sentiment analysis and opinion mining use cases -- predict election results and box-office collections of movies, understand attitude towards vaccination, detect hate speech, etc. Computer vision has seen comparable advances, and today we have document scanning and parsing (OCR) capabilities on our mobile phones~\cite{google_lens,ms_lens}. We use these new-age language and vision capabilities and other techniques to analyze users' perception of SpaceX Starlink on a popular subreddit \starlinksubreddit{}\cite{starlink_subreddit}. We believe our techniques could be generalized across LEO providers and social platforms to gather useful insights into the LEO broadband space.


% Figure environment removed

Following the footsteps of past work on mining social platforms to quantify network performance and demands~\cite{qiu2010listen,hsu2011using,yang2016estimating}, and network and service failures~\cite{motoyama2010measuring,takeshita2015early}, we demonstrate here the scope of building a framework that enables the community around the `LEO' space to understand these networks better. \starlinksubreddit{} has managed to draw significant participation from enthusiasts, early adopters, and others. As, Fig.~\ref{fig:activity_plot} shows, there are $372$ posts per week on average with the peak reaching $1$,$326$ posts/week in Feb'$21$. The number of upvotes and comments, which are strong signals of user activity, are $8$,$190$ and $5$,$702$ per week on average respectively. While the initial flurry of enthusiasm (first half of $2021$) has subsided, the subreddit continues to see significant user engagement. We analyze the publicly available posts on the subreddit channel and gather aggregated insights on users' perception of Starlink service. We gather data using standard social network APIs and use standard language and computer vision tools, coupled with a set of custom heuristics, to extract structured data on network performance from publicly shared screenshots, to generate word clouds of $n$-grams, and to quantify user sentiment. Armed with this data, we explore the following $3$ representational use cases in this paper:


\vspace{-0.2in}
\begin{itemize}
    \item We analyze how user sentiment follows related events and public announcements -- this could be used by competing ISPs and the research community to evaluate the large-scale impact of changes and events -- both technical and otherwise. Also, such sentiment insights could equip potential Starlink customers to take informed decisions while subscribing to broadband services. We also show how specific events like Starlink network outages could be detected from public discussions on social media leveraging a mix of language and sentiment analysis tools.
    \item Using publicly available information and a combination of language tools, we could identify when roaming was enabled as a service on the Starlink network, weeks before any public announcement on this from Starlink. For competing ISPs, it is important to detect such signals early and act upon them. 
    \item We analyze how bandwidth on the Starlink network varies relatively over time with more launches and users, and how that drives user sentiment. Such insights could help the broad community to understand how customer expectation changes as Starlink services evolve.
\end{itemize}
\vspace{-0.05in}

We believe our analyses make the case for coming up with a framework encompassing multiple social platforms (Reddit is just one of them) and LEO broadband providers (Starlink is just one of them) that help augment the currently limited LEO measurement landscape. While here we present early-stage analyses, this framework has the potential to offer a \textbf{complementary user-centric view} of the LEO broadband space even in the presence of large-scale LEO measurement platforms in the future.
