%\vspace{-0.2in}
\section{Related work}
%\vspace{-0.1in}

Recent breakthrough advances in natural language processing~\cite{devlin2018bert,dai2019transformer,yang2019xlnet,liu2019roberta,qi2020stanza,azure_acs,google_NL_AI,openai2023gpt4} and computer vision~\cite{girshick2014rich,redmon2016you,he2017mask,brock2018large,karras2019style} have made text and image scanning, parsing, and comprehension capabilities~\cite{luan2019general,huang2019icdar2019,liu2019graph} ubiquitous among other things. Sentiment analysis, which is used for detecting sentiments underlying text, has been exhaustively applied to a multitude of use cases -- understanding product feedback and preferences~\cite{liu2012sentiment,anto2016product}, and predicting trends and outcomes of large-scale events~\cite{matalon2021using,karalevicius2018using,bermingham2011using} leveraging the corpus of publicly available user interactions on online social platforms. Leveraging social media to understand networks better is not a completely new space -- sentiment data have been used to understand/detect mobile network performance~\cite{qiu2010listen,hsu2011using} and demands~\cite{yang2016estimating}, service availability and failures~\cite{motoyama2010measuring,takeshita2015early}, and attacks and security vulnerabilities~\cite{al2013leveraging,ritter2015weakly,sabottke2015vulnerability,khandpur2017crowdsourcing,chambers2018detecting,shu2018understanding}. Motivated by this large body of past work, we explore a similar methodology to understand SpaceX Starlink service leveraging crowdsourced information available on social media.

While many measurement platforms~\cite{planetlab,MLab,ripe_atlas,ookla_speedtest,fast_speedtest} exist that gather Internet performance metrics at scale, we did not find a measurement platform yet targeted to analyze the nuances of LEO dynamics -- temporal changes in latency and bandwidth independent of the congestion, impact of the constellation geometry, weather, and interference with other similar services, etc. While the research community has already engaged in simulations~\cite{kassing2020exploring,bhattacherjee2019network,handley2018delay} and limited measurements~\cite{starlink_perf_firstlook,leo_geo_aus} of these networks, measurements at scale are still unavailable. A recent work~\cite{starlink_browserside} on gathering LEO measurements leverages a custom-built browser plugin -- our work is complementary to this work in the absence of large-scale LEO measurement platforms. Even if large-scale LEO measurement platforms become available in the future, our framework has the potential to complement such platforms by offering a user-centric view of the services. A more recent work~\cite{pan2023measuring} uses Reddit to ``crosscheck'' Starlink measurement insights, but we rather consume public data on the \starlinksubreddit{} subreddit as the primary source of performance and user sentiment measurements.
