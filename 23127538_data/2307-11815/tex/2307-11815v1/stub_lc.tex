\begin{deluxetable*}{lrrrrrrrr}
\tablewidth{0pt}\tablecaption{SN Light Curves\label{tab:lc_stub}}
\tablehead{\colhead{Name}& \colhead{BJD $-$245700.0}&\colhead{counts s$^{-1}$ }& \colhead{Uncertainty}&  \colhead{Fraction of Peak} & \colhead{Uncertainty}& \colhead{$T_{\rm mag}$}& \colhead{Unceratinty}& \colhead{Calib. Offset}}
\startdata 
SN2018exc&  1325.32891   &$-$     5.63      &     6.03      &$-$    0.0382     &    0.0409     &     19.65     &--&--\\ 
SN2018exc&  1325.34974   &     9.35      &     6.07      &    0.0634     &    0.0412     &     19.21     &     0.70      &--\\ 
SN2018exc&  1325.37057   &$-$     1.34      &     6.14      &$-$    0.0091     &    0.0416     &     19.65     &--&--\\ 
\ldots&\dots&\dots&\dots&\dots&\dots&\dots&\dots&\dots \\ 
\enddata 
\tablecomments{ A machine readable version of this table is available in the online version of this article.  For fluxes below the 8 hour detection limits (including zero and and negative flux due to noise), the magnitudes are represented by $T_{\rm lim}$  from Table~\ref{tab:physical_data} and have uncertainties marked as NaN. The Calibration Offset gives the shift applied to flux calibrate the second sector of \tess\ observations for supernova observed near the ecliptic poles (see Appendix~\ref{sec:fits}).  }
\end{deluxetable*}
