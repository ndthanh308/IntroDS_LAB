\movetabledown=2.55in
\begin{deluxetable*}{lrrccc}
\tablewidth{0pt}\tablecaption{SN with Evidence for or Against Companion Interactions \label{tab:candidates}}
\tablehead{ 
\colhead{Name}& \colhead{$\ln K$}\tablenotemark{a}& \colhead{$\Delta {\rm BIC}$}\tablenotemark{b}& \colhead{rms$_{\rm Gauss}/$rms$_{\rm observed}$}\tablenotemark{c} & \colhead{Median Radius\tablenotemark{d}} &\colhead{Median Separation\tablenotemark{e}}}
\startdata 
SN2020abqu&   5.62   &   7.22   & 0.5133 &  3.39  $\pm$1.68&  0.61  $\pm$0.30\\ 
SN2021ahmz&   4.33   &   4.56   & 0.4444 &  1.94  $\pm$0.84&  0.35  $\pm$0.15\\ 
SN2022ajw &   3.06   &  -0.71   & 0.6124 &  2.57  $\pm$1.21&  0.46  $\pm$0.22\\ 
\hline
SN2020tld &  -16.27  &  -32.31  & 0.4196 &      --       &      --       \\ 
SN2022eyw &  -15.26  &  -32.44  & 0.8696 &      --       &      --       \\ 
\enddata 
\tablecomments{ SN2020abqu, SN2021ahmz, and SN2022ajw show a slight preference for adding a companion interaction model to a curved power law, though the statistical evidence is not robust when using $\Delta$BIC. SN2020tld and SN2022eyw show robust evidence against the addition of a companion interaction model.   \tablenotetext{a}{Natural log of Bayes Factor $K$.  Larger values of K indicate a preference for companion interactions.}\tablenotetext{b}{Difference between Bayesian Information Criteria for models with and without companion interactions.  Larger values of $\Delta {\rm BIC}$ indicate a preference for companion interactions.}  \tablenotetext{c}{ Noise metric defined in \S\ref{sec:results}, which quantifies departures of the light curves from random Gaussian noise. Values near zero indicate very little improvement from binning (systematic errors dominate over random noise), while 1 indicates perfect Gaussian noise scaling  (random noise only). For SN2022eyw, the light curve is affected by scattered light but the SN signal occurs at a time isolated from the excess noise.} \tablenotetext{d}{ Companion Roche lobe radii in units of solar radii.  "Median Radius" gives the median (50th percentile) and 68\% credible region of the posterior distribution.} \tablenotetext{e}{Companion separations in units of 10$^{12}$~cm.  "Median Separation" gives the median (50th percentile) and 68\% credible region  of the posterior distribution.}}
\end{deluxetable*}
