\documentclass[12pt,reqno]{amsart}

\usepackage{nameref}
\usepackage{upgreek}     
\usepackage{enumitem}
\usepackage{graphicx}
\usepackage{mathrsfs}
\usepackage{amsthm}
\usepackage{amssymb}
\usepackage{amsmath,amsthm,amsfonts}
\usepackage{tikz-cd}
\usepackage{mathabx}
\usepackage{hhline}
\usepackage{textcomp}
\usepackage{amsmath,amscd}
\usepackage[all,cmtip]{xy}
\usepackage{mathtools}
\usepackage[margin=1in]{geometry}
\usepackage{lipsum}
\usepackage{extarrows}
\usepackage{upgreek}
\usepackage{bbm}
%\usepackage{makeid}
%\usepackage[a4paper]{geometry}
\usepackage[displaymath]{lineno}
\usepackage[singlespacing]{setspace}%
\setcounter{MaxMatrixCols}{30}
\usepackage[colorlinks=true,breaklinks=true,bookmarks=true,urlcolor=blue,
citecolor=blue,linkcolor=blue,bookmarksopen=false,draft=false]{hyperref}
%\definecolor{DarkOrchid} {cmyk}{0.40,0.80,0.20,0}
\def\tcb{\textcolor{blue}}
\def\tcm{   \textcolor{magenta}}
\def\tcr{ \textcolor{red}}
\def\tcv{ \textcolor{DarkOrchid}}
%\biboptions{sort&compress}
\providecommand{\U}[1]{\protect\rule{.1in}{.1in}}

\makeatletter
\let\orgdescriptionlabel\descriptionlabel
\renewcommand*{\descriptionlabel}[1]{%
	\let\orglabel\label
	\let\label\@gobble
	\phantomsection
	\edef\@currentlabel{#1}%
	%\edef\@currentlabelname{#1}%
	\let\label\orglabel
	\orgdescriptionlabel{#1}%
}
\makeatother







\theoremstyle{definition}
\newtheorem{theorem}{Theorem}[section]
\newtheorem{lemma}{Lemma}[section]
\newtheorem{remk}{Remark}[section]
\newtheorem{prop}{Proposition}[section]
\newtheorem{defn}{Definition}[section]
\newtheorem*{theorem*}{Theorem}
\newtheorem{cor}{Corollary}[section]
\newtheorem*{con}{Condition}

\numberwithin{equation}{section}

%    Absolute value notation
\newcommand{\abs}[1]{\lvert#1\rvert}

%    Blank box placeholder for figures (to avoid requiring any
%    particular graphics capabilities for printing this document).
\newcommand{\blankbox}[2]{%
  \parbox{\columnwidth}{\centering
%    Set fboxsep to 0 so that the actual size of the box will match the
%    given measurements more closely.
    \setlength{\fboxsep}{0pt}%
    \fbox{\raisebox{0pt}[#2]{\hspace{#1}}}%
  }%
}
\DeclareMathAlphabet{\mathpzc}{OT1}{pzc}{m}{it}
\renewcommand{\theequation}{\thesection.\arabic{equation}}

\newcommand{\ca}{\mathrm{Cat}}
\newcommand{\ftt}[1] {\mathsf{#1}}
\newcommand{\boldm}[1] {\mathversion{bold}#1\mathversion{normal}}
\newcommand{\ps}{\mathrm{({PS}})_{\mathbf{G}}}
\newcommand{\pss}{\mathrm{({PS}})_c}
\newcommand{\dit}{\displaystyle{\int}}
\newcommand{\di}{differentiable }
\newcommand{\ga}{G\^{a}teaux }
\newcommand{\va}{\varphi}
\newcommand{\cs}{continuous }
\def\derat#1{{d \over dt} \hbox{\vrule width0.5pt 
		height 5mm depth 3mm${{}\atop{{}\atop{\scriptstyle t=#1}}}$}}
\def\deras#1{{d \over ds} \hbox{\vrule width0.5pt 
		height 5mm depth 3mm${{}\atop{{}\atop{\scriptstyle s=#1}}}$}}
\def\deratau#1{{d \over d\tau} \hbox{\vrule width0.5pt 
		height 5mm depth 3mm${{}\atop{{}\atop{\scriptstyle \tau=#1}}}$}}
\newcommand{\vf}{\mathtt{V}}
\newcommand{\dd}{{\tt D}}
\newcommand{\dt}[1]{{\tt d}{#1}}
\newcommand{\fs}[1]{\mathbbm {#1}}
\newcommand{\eu}[1]{\EuScript {#1}}
\newcommand\Set[2]{\{\,#1\mid#2\,\}}
\newcommand*{\medwedge}{\mathbin{\scalebox{1.5}{\ensuremath{\wedge}}}}
\newcommand*{\medprod}{\mathbin{\scalebox{0.75}{\ensuremath{\prod}}}}
\newcommand*{\medcap}{\mathbin{\scalebox{1.5}{\ensuremath{\cap}}}}
\newcommand*{\medcup}{\mathbin{\scalebox{1.5}{\ensuremath{\cup}}}}
%\newcommand*{\medint}{\mathbin{\scalebox{1.5}{\ensuremath{\int}}}}
\newcommand{\lc}{\mathsf{L}}
\newcommand{\ls}{\mathbbm{L}}
\newcommand{\ty}{\EuScript{T}}
\newcommand{\ix}{\mathbf{I}}
\newcommand{\h}{Hausdorff }
\newcommand{\co}{continuous}
\newcommand{\mt}{\mathbbm {d}}
\newcommand{\cat}[1]{\ensuremath{\mathsf{\mathop{#1}}}}
%\newcommand{\fr}{Fr\'{e}chet }
%\newcommand{\N}{\mathbb{N}}
\newcommand{\zz}{\mathbb{Z}}
\newcommand{\Q}{\mathbb{Q}}
\newcommand{\R}{\mathbb{R}}
\newcommand{\set}[1]{\left\{#1\right\}}
%\newcommand{\abs}[1]{\left\lvert#1\right\rvert}
\newcommand{\snorm}[2][]{\left\lVert#2\right\rVert_{#1}}
\newcommand{\hlcs}  {\textsf{Hlcs}}
\newcommand{\Sem}[1]  {\textsf{Sem}(#1)}
\newcommand{\clos}[1]  {\mathsf{clos}(#1)}
%\newcommand{\norm}[1]{\left\lVert #1 \right\rVert}
% Це потрібно для скороченого запису ряду Остроградського
\newcommand{\zero}[1]{\mathbf{0}_{#1}}
\providecommand{\norm}[1]{\lVert\mspace{1mu}#1\mspace{1mu}\rVert}
\newcommand{\hh}{\mathcal{H}}
\newcommand{\nbd}{neighborhood \,}
\newcommand{\li}{Lipschitz}
\newcommand{\rr}{\mathbb{R}}
\newcommand{\nn}{\mathbb{N}}
\newcommand{\nr}{\mathcal{N}}
\newcommand{\uu}{\mathcal{U}}
\newcommand{\vv}{\mathcal{V}}
\newcommand{\ww}{\mathcal{W}}
\newcommand{\w}{\mathbbmss{w}}
\newcommand{\x}{\mathbbmss{x}}
\newcommand{\xx}{\mathcal{X}}
\newcommand{\ab}{\mathcal{A}}
\newcommand{\p}{\bf{p}}
\newcommand{\id}{\mathrm{Id}}
%\newcommand{\pr}{\mathrm{Pr}}
\newcommand{\pr}{\Pi}
\newcommand{\tn}{\pitchfork}
\newcommand{\f}{Fr\'{e}chet }
%\newcommand{\mc}{MC^k}
\newcommand{\mc}[1]{\mathtt{MC}^{#1}}
\newcommand{\bl}[1] {\mathbf {#1}}
%\DeclareMathOperator{\dd}{D}
\DeclareMathOperator{\codim}{codim}
\DeclareMathOperator{\Ind}{Ind}
\DeclareMathOperator{\Img}{Img}
\DeclareMathOperator{\Coker}{Coker}
\DeclareMathOperator{\Aut}{Aut}
\DeclareMathOperator{\Iso}{Iso}
\DeclareMathOperator{\Lip}{Lip_{loc}}
\newcommand{\bb}{\mathcal{B}}
\DeclareMathAlphabet\EuScript{U}{eus}{m}{n}
\SetMathAlphabet\EuScript{bold}{U}{eus}{b}{n}
\newcommand\opn{\ensuremath{\mathrel{\mathpalette\opncls\circ}}}
\newcommand\cls{\ensuremath{\mathrel{\mathpalette\opncls\bullet}}}
\newcommand{\opncls}[2]{
	\ooalign{$#1\subseteq$\cr
		\hidewidth\raisefix{#1}\hbox{$#1{\stylefix{#1}#2}\mkern2mu$}\cr}}
\def\raisefix#1{
	\ifx#1\displaystyle
	\raise.39ex
	\else
	\ifx#1\textstyle
	\raise.39ex
	\else
	\ifx#1\scriptstyle
	\raise.275ex
	\else
	\raise.150ex
	\fi
	\fi
	\fi
}
\def\stylefix#1{
	\ifx#1\displaystyle
	\scriptstyle
	\else
	\ifx#1\textstyle
	\scriptstyle
	\else
	\ifx#1\scriptstyle
	\scriptscriptstyle
	\else
	\scriptscriptstyle
	\fi
	\fi
	\fi
}
\DeclareFontFamily{U}{mathx}{\hyphenchar\font45}
\DeclareFontShape{U}{mathx}{m}{n}{
	<5> <6> <7> <8> <9> <10>
	<10.95> <12> <14.4> <17.28> <20.74> <24.88>
	mathx10
}{}
%\newcommand{\cat}[1]{\ensuremath{\mathsf{\mathop{#1}}}}
\newcommand{\fr}{Fr\'{e}chet }
\newcommand{\N}{\mathbb{N}}
\newcommand{\Z}{\mathbb{Z}}

\newcommand{\Osign}[1]{\mathrm{O}^{#1}}
\providecommand{\norm}[1]{\lVert\mspace{1mu}#1\mspace{1mu}\rVert}
%%%%%%%%%%%%%%%%%5
\newcommand{\Cl}[1]{\overline{#1}}
\newcommand{\Fr}{\mathop{\mathrm{Fr}}\nolimits}
\newcommand{\Int}{\mathop{\mathrm{int}}\nolimits}
\newcommand{\Id}{\mathop\mathrm{Id}\nolimits}
\renewcommand{\Re}{\mathop{\mathrm{Re}}\nolimits}
\renewcommand{\Im}{\mathop{\mathrm{Im}}\nolimits}
\newcommand{\Card}[1]{\left| #1 \right|}
\renewcommand{\emptyset}{\varnothing}
\newcommand{\Finsler}{\snorm[n]{\cdot}}
\newcommand{\s}{\mathbf{S}}


\makeatletter
\@namedef{subjclassname@2020}{%
	\textup{2020} Mathematics Subject Classification}
\makeatother
\begin{document}

\title{Geometry via sprays on Fr\'{e}chet Manifolds}

%    Information for first author
\author{Kaveh Eftekharinasab}
%\thanks{The author would like to thank the reviewer for the valuable comments  }
%    Address of record for the research reported here
\address{Topology lab.  Institute of Mathematics of National Academy of Sciences of Ukraine, Tereshchenkivska st. 3,  01024, Kyiv, Ukraine}
\email{kaveh@imath.kiev.ua}
\thanks {}

%    Information for second author



%    General info
\subjclass[2020]{58B99,  
	53C05.
}

%\date{}



\keywords{\fr manifolds, sprays, connection maps, symmetric linear connections}

\begin{abstract}
	We construct connection maps and establish linear symmetric connections on tangent and second-order tangent bundles for \fr manifolds by employing sprays. We characterize a linear symmetric connection on the tangent bundle of a \fr manifold
	 through the bilinear symmetric mapping associated with a given spray. Also, we provide an alternative characterization of linear symmetric connections on tangent bundles, employing tangent structures. Moreover, we prove that a bijective correspondence exists between linear symmetric connections on tangent bundles and connection maps induced by sprays.
\end{abstract}

\maketitle
\section{Introduction}
The geometry of \fr manifolds, particularly concerning higher-order structures on tangent bundles, has seen significant advancements in recent years, see for instance, \cite{a,a2,a3, a5,dod,dod2,dod3,dod4,su1}.
A significant challenge in handling these manifolds is that the general linear group of a \fr space is not a Lie group. Consequently, it cannot serve as a structure group for bundles. Another issue is that the space of continuous linear mappings between \fr spaces does not possess a \fr space structure. Therefore, defining structures involving this space is not a straightforward generalization of the Banach case.

An approach to address these issues involves the use of the projective limit technique as employed in the referenced papers. In this method, we consider \fr manifolds that are projective limits of Banach manifolds. This allows the construction of objects such as structure groups using projective limits of their corresponding Banach counterparts.

This approach has proven successful in deriving numerous valuable results, as outlined in the aforementioned papers. Nevertheless, in situations where we do not need a structure group, the question arises: are there alternative methods to address these problems? Because in the projective limit approach, constructing all objects through projective limit systems is not always straightforward and accessible.

In this work, we aim to establish an alternative approach that avoids the use of general linear groups for constructing geometric objects. Specifically, we will explore the application of sprays, an aspect that has received limited attention in the context of \fr manifolds.

 Despite its limitations, when attempting to define differentiable bundles on \fr manifolds without relying on a structure group, a fundamental question arises: how do we define the differentiability of transition mappings, given that they take values in a general linear group? In \cite{hamilton}, Hamilton presented one such definition, while Neeb proposed an alternative definition in \cite{neeb}. In this work, we will adopt the latter definition, which appears to be more suitable for developing the geometry of \fr manifolds. 
 
 The starting point is to prove that a given spray uniquely determines a connection map, a concept introduced by Dombrowski \cite{dom}. This is accomplished in Theorem \ref{th:con}, which stands as our principal result linking a spray with other objects, such as connections on tangent and second-order tangent bundles, facilitating further development.
 
 It is a common practice to employ parallel transport in defining connections. However, this approach faces limitations in the context of \fr manifolds, where the absence of parallel transports can be an issue. Consequently, in our setting, a notable challenge in the concept of a connection lies in its determination from a spray.
 
 In Theorem \ref{th:ch}, we prove that a given spray induces a unique symmetric linear connection, and conversely.
 Furthermore, in Theorem \ref{th:noncon}, we will achieve the same result through the utilization of a tangent structure.
 
 One of our concerns is the geometry of second-order tangent bundles, due to their significance and applications in studying ordinary differential equations (ODEs) on manifolds. A pivotal question related to second-order tangent bundles is the determination of when they exhibit a vector bundle structure. We show that the presence of a spray implies the existence of a vector bundle structure on second-order tangent bundles, as stated in Theorem \ref{th.iso}.
Furthermore, in Theorem \ref{th:ind}, we show that a linear symmetric connection on a tangent bundle induces a linear symmetric connection on the corresponding second-order tangent bundle, and vice versa.
 %Moreover, by incorporating the concept of a tangent structure, which is directly related to connection maps, we establish the coexistence of connection maps (induced by a spray) and linear symmetric connections on tangent bundles in Theorem \ref{th:noncon}.

In conclusion, for \fr manifolds, as demonstrated, sprays and vector bundles without group structure offer a clear and accessible approach for examining geometrical objects on tangent and second-order tangent bundles. Notably, we observe several parallels with the geometry of Banach manifolds. This perspective opens avenues for exploring additional significant topics, such as second-order differential equations on \fr manifolds.

    
\section{Prerequisites}\label{sec:1}
We  employ the concept of differentiable maps, known as $C^k$-maps in the Michal–Bastiani sense or Keller's $C_c^k$-maps. This choice is made as it avoids introducing any topology on spaces of continuous linear maps.

Throughout
by $ U \opn \mathsf{T} $ we mean that $ U $
is an open subset of a topological space $ \mathsf{T} $.

\begin{defn}[\cite{neeb}, Definition I.2.1]
Let $\fs{E}$ and $\fs{F}$ be locally convex spaces, $ \va: U \opn \fs{E}  \to  \fs{F}$ a mapping. Then the  derivative
	of $\va$ at $x$ in the direction $h$ is defined as 
$$ \dd\va(x)(h) := \derat0 \va(x + t h) 
= \lim_{t \to 0} {1\over t}(\va(x+th) -\va(x)) $$
whenever it exists. The function $\va$ is called differentiable at
	$x$ if $\dd \va(x)(h)$ exists for all $h \in \fs{E}$. It is called 
	continuously differentiable, if it is differentiable at all
points of $U$, and the mapping
$$ \dd \va : U \times \fs{E} \to \fs{F}, \quad (x,h) \mapsto \dd \va(x)(h) $$
is continuous. It is called a {\it $C^k$-map}, $k \in \nn \cup \{\infty\}$, 
if it is continuous, the iterated directional derivatives 
$ \dd^{j}\va(x)(h_1,\ldots, h_j)
$
exist for all integers $j \leq k$, $x \in U$ and $h_1,\ldots, h_j \in \fs{E}$, 
and all maps $\dd^j \va : U \times \fs{E}^j \to F$ are continuous. We call $ C^{\infty} $-maps smooth. 
\end{defn}
The differentiability in the sense of the following definition makes it possible to define a differentiable vector bundle without
requiring a structure group. 
\begin{defn}[\cite{neeb}, Definition II.3.1]\label{def:neeb}
	Let $ \fs{M} $ be a $ C^k, k\geq 1, $ locally convex manifold, and $ \mathrm{Diff}(\fs{M}) $ be the group of
	diffeomorphisms of $ \fs{M} $. Further, let $ \fs{N} $ be a $ C^k$ locally convex manifold. Although, in general, $ \mathrm{Diff}(\fs{M} ) $ does not possess a natural
	 Lie group structure,  a map $ \va: \fs{N} \to \mathrm{Diff}(\fs{M})$ is said to be $ C^k$, if the following map 	is of class $ C^k$:
	\begin{equation*}
	\widehat{\va} : \fs{N} \times \fs{M} \to \fs{M} \times \fs{M}, \quad (n,x) \mapsto (\va(n)(x),\va^{-1}(n)(x)).
	\end{equation*}
\end{defn}
The definition of differentiable vector bundles without structure groups, as presented in \cite{neeb}, serves as a facilitator in developing the Lie theory of Lie groups modeled on locally convex spaces. This definition proves to be instrumental in advancing the geometry of any \fr manifold equipped with a spray.
\begin{defn}[\cite{neeb}, Definition I.3.8]\label{def:neebvec}
Let $\fs{M}$ be a $ C^k $-Fr\'{e}chet manifold modeled on a \fr space $\fs{F}$, where $ k\geq1 $, and $\fs{E}$ be another \fr space. A $C^r$-{vector bundle} of type $\fs{E}$ over $\fs{M}$ is a triple $(\pr, \fs{V}, \fs{E})$, consisting a $C^r$-\fr manifold $\fs{V}$, a $ C^r $-map $\pr: \fs{V} \to \fs{M}$, and a \fr space $\fs{E}$, possessing the following properties:
\begin{description}
	\item[(VB.1)]  $ \forall m \in \fs{M} $, the fiber $ \fs{V}_m := \pr^{-1}(m) $ is a \fr space 
	isomorphic to $\fs{E} $.
	\item[(VB.2)] For each $m \in \fs{M}$, there exists an open neighborhood $U$ such that a diffeomorphism
	\begin{equation*}
	\upphi_U:\pr^{-1}(U) \to U \times \fs{E} 
	\end{equation*}
    can be established, where $\upphi_U = (\pr \rvert_U, \uppsi_U)$ and $\uppsi_U : \pr^{-1}(U) \to \fs{E}$ is linear on each $\fs{V}_m$ for $m \in U$.	
\end{description}
We then call $ U $ a trivializing subset of $ \fs{M} $ and $ \upphi_U $ a bundle chart. If $ \upphi_U $ and $ \upphi_V $ are two bundle charts and $ U \cap V \neq \emptyset $, then we obtain a diffeomorphism
\begin{equation*}
\upphi_U \circ  \upphi_V^{-1}: (U \cap V) \times \fs{E} \to (U \cap V) \times \fs{E}
\end{equation*}
of the form $ (x, v) \mapsto (x, \uppsi_{VU} (x)v) $. This leads to a map
\begin{equation*}
 \uppsi_{UV} : U \cap V \to \bl{GL}(\fs{E})
\end{equation*}
for which it does not make sense to speak about smoothness because $ \bl{GL}(\fs{E}) $ is not a Lie group. Nevertheless, $ \uppsi_{UV} $ is of class $ C^r $ in the sense of Definition \ref{def:neeb}, as the map
\begin{gather*}
\widehat{\uppsi_{UV}}: (U \cap V) \times \fs{E} \to (U \cap V) \times \fs{E} \\
(x,v) \mapsto
(\uppsi_{UV}(x)v,\uppsi_{UV}(x)^{-1}v) =(\uppsi_{UV}(x)v,\uppsi_{VU}(x)v)
\end{gather*}
is of class $ C^r $. Here, $ \bl{GL}(\fs{E}) $ is the general linear group of $ \fs{E} $.
\end{defn}
Sprays were developed for a category of \fr manifolds known as $ \mc{k} $-\fr manifolds in \cite{k3}. Another form of differentiability, namely $ \mc{k} $-differentiability, is applied within this category. However, as the type of differentiability does not impact the definitions and results required in this paper, we adopt them without presenting the proofs.
 
Henceforth, we assume that $\fs{M}$ is a $ C^k $-Fr\'{e}chet manifold modeled on a Fr\'{e}chet space $\fs{F}$, $ k\geq3 $, and $ \fs{M} $ admits a smooth partition of unity. 

We denote the tangent bundle by $ T\fs{M} $ and the tangent map of a function $ \va $  defined on $\fs{M}$ 
by $ T\va  $  (or $ \va_{*} $)). Subscripts are used to indicate fibers of bundles and the restrictions of maps to those fibers.

A second-order vector field $ \vf $ (of class $ C^k $) is a $ C^k $-mapping $\vf : T\fs{M} \to T(T\fs{M})$  such that
$$ T\pr_{\fs{M}} \circ \vf =\Id_{T\fs{M}}.$$ Let $ \mathbb{V}: \fs{N} \to \fs{M} $ be a $ C^k $-vector bundle,  $ s $ a fixed real number. Define the mapping: 
\begin{equation*}
L_{\mathbb{V}}: \fs{N} \to \fs{N},\quad v \mapsto sv.
\end{equation*}
Then, its tangent map $ TL_{\mathbb{V}}: T(\fs{N}) \to T(\fs{N}) $ is the  induced map on $ T(\fs{N}) $.
If $ \fs{N}= T\fs{M}  $, then
\begin{equation*}
(L_{T\fs{M}})_* \circ L_{T(T\fs{M})}= L_{T(T\fs{M})} \circ (L_{T\fs{M}})_*
\end{equation*}
which follows from the linearity of $ L_{T\fs{M}} $ on each fiber.

A second-order vector filed $\s : T\fs{M} \to T(T\fs{M})$ (of class $ C^{k-1} $) is said to be a {spray} if 
it satisfies the following condition:
\begin{enumerate}[label=$ \bl{(SP.\arabic*)} $,ref=SP.\arabic*]
	\item \label{eq:sp1} $\s(sv) = (L_{T\fs{M}})_*(s\s(v))$ for
	all $ s \in \rr $ and $ v \in T\fs{M} $.
\end{enumerate}
If $ \fs{M} $ admits a smooth partition of unity, then there exists a spray on $ \fs{M} $.

Let $ U \opn \fs{F} $, so that $ T(U) = U \times \fs{F} $, and
$ T(T(U)) = (U \times \fs{F} ) \times (\fs{F} \times \fs{F}) $. 
In this context, the representations of $ L_{TU} $ and $ (L_{TU})_*$ in the chart are given by the
following maps:
\begin{equation*}
L_{TU}:(x,v) \mapsto (x,sv) \quad \text{and}\quad (L_{TU})_*:(x,v,u,w)\mapsto (x,sv,u,sw).
\end{equation*}
Therefore, we get
\begin{equation*}
L_{T(TU)} \circ (L_{TU})_*(x,v,u,w)=(x,sv,su,s^2w).
\end{equation*}
In order to avoid confusion, if necessary, in a chart $ U $, we index objects by $ U $ to indicate their local representations.
Let $ {\vf}_U= ({\vf}_{U,1},{\vf}_{U,2}): (U \times \fs{F} ) \to \fs{F} \times \fs{F}$
be a local representation of $ \vf $, where each $ {\vf}_{U,i} $ maps $ U \times \fs{F} $
to $ \fs{F} $ with $ {\vf}_{U,1}(x,v)=v $. Then, it represents a spray if and only if, for
all $ s \in \rr $, the following condition holds:
\begin{equation}\label{eq:1}
{\vf}_{U,2}(x,sv)=s^2{\vf}_{U,2}(x,v).
\end{equation}
Thus, we observe that \eqref{eq:sp1} (in addition to being a second-order vector field), 
simply means that $ {\vf}_{U,2} $ is homogeneous of degree 2 in the
variable $ v $.  It follows that $ {\vf}_{U,2} $ is a quadratic
map in its second variable, and more precisely, this quadratic map is given by
\begin{equation*}
{\vf}_{U,2} (x,v) = \dfrac{1}{2}\dd_2^2 {\vf}_{U,2} (x, 0)(v,v).
\end{equation*}
Thus, the spray is induced by a symmetric bilinear map given at each point
$ x $ in a chart by
\begin{equation}\label{eq:sp3}
\mathbb{B}= \dfrac{1}{2}\dd_2^2 {\vf}_{U,2} (x, 0),
\end{equation}
where $ \dd_2^2 $ is the second partial derivative with respect to the second variable.
Conversely, suppose a map $ x \in U \mapsto \mathbb{B}(x) $ is given, where $ \mathbb{B}(x) $ is a symmetric bilinear $ C^{k-1} $-map from  $\fs{F} \times \fs{F}$ to $\fs{F} $.
For
each $ v, w \in \fs{E} $, the value of $  \mathbb{B}(x) $ at $ (v, w) $ is denoted by $ \mathbb{B}(x; v, w) $ or
$ \mathbb{B}(x)(v, w) $. Define $ {\vf}_{U,2}(x, v) = \mathbb{B}(x; v,v) $. Then, $ {\vf}_{U,2} $ is quadratic in its
second variable, and a $ C^k $-spray over $ U $ can be represented by the map $ {\vf}_{U,2} $ defined by
\begin{equation*}
{\vf}_{U,2}(x, v) = (v, (\mathbb{B}(x; t),t)) = (v, {\vf}_{U,2}(x, v)).
\end{equation*}
The mapping $ \mathbb{B} $ is called the symmetric bilinear map 
associated with the spray.

Next, we will introduce a covariant derivative; however, it's important to note that this definition is not the most general one available. There exist other definitions that are more encompassing in nature. However, this definition is seamlessly incorporated for the applications of sprays.

Let $ \mathsf{V}^{k}(\fs{M}) $ and $ \mathcal{E}^k(\fs{M}) $ denote the sets of all $C^{k}$-vector fields and $C^k$-real-valued maps on $ \fs{M} $, respectively. Let $X$ and $Y$ belong to $ \mathsf{V}^{k}(\fs{M}) $, and $ [X,Y] $ represent the bracket product. A covariant derivative $ \nabla $ is an $\rr$-bilinear map:
 \begin{equation*}
 \begin{array}{cccc}
 \nabla : \mathsf{V}^{k}(\fs{M}) \times \mathsf{V}^{k}(\fs{M}) \to \mathsf{V}^{k}(\fs{M}),\quad
 (X,Y) \to \nabla_XY
 \end{array}
 \end{equation*} 
 such that for all $ \varphi \in \mathcal{E}^k(\fs{M})$ and $ X,Y \in \mathsf{V}^{k}(\fs{M}) $ the following hold: 
 \begin{description}
 	\item [(CD.1)\label{eq:cd2}] $ \nabla_{\varphi X}Y = \varphi \nabla_{ X}Y$,
 	\item [(CD.2)\label{eq:cd3}]$\nabla_{ X}(\varphi Y) = (\mathcal{L}_{X}\varphi)Y + \varphi \nabla_{ X}Y$,
 	\item [(CD.3\label{eq:cd4})]$\nabla_XY - \nabla_YX = [X,Y]$.
 \end{description}
 
Given a spray $ \s $  on $ \fs{M} $, then there exists a unique covariant
derivative $ \nabla $ (of class $ C^{k-1} $) such that in a chart $ U $, the covariant derivative is expressed as follows:
\begin{equation}\label{eq:cd1}
(\nabla_XY)_U(x) = Y'_U(x)X_U(x)-\mathbb{B}_U(x)(X_U(x),Y_U(x)),
\end{equation}
see \cite{k3}. The last condition is imposed to ensure the existence of a bijective correspondence between sprays and covariant derivatives, as discussed in \cite[VIII, $ \S 2 $ ]{lang}.
\section{Associated Connection Maps}
In this section, we prove our departure prime theorem, Theorem \ref{th:ch}. This theorem asserts that a spray on a manifolds induces a unique connection map in the sense of \cite{eli} and \cite{vilms}. 

Let $ \pr_{\fs{M}}: T\fs{M} \to \fs{M} $ and $ \pr_{T\fs{M}}: T(T\fs{M}) \to T\fs{M} $ denote the tangent and double tangent bundles, respectively. Consider a local trivialization $(U,\phi)$ for $ \fs{M} $, where $ \phi(U) \times \fs{F} $ is the corresponding chart for $ T\fs{M} $, and $ (\phi(U) \times \fs{F}) \times (\fs{F} \times \fs{F}) $ represents the corresponding chart for $ T(T\fs{M}) $.

Define $ \pr_{\fs{M}}^*T\fs{M} $ as the pullback bundle induced by $ \pr_{\fs{M}} $, and let $ \pr_{\fs{M}}^*: \pr_{\fs{M}}^*T\fs{M} \to T\fs{M}$ be its projection,  and $ (\phi(U) \times \fs{F}) \times \fs{F} $ 
the corresponding chart for $ \pr_{\fs{M}}^*T\fs{M} $.

In charts, the tangent map $ T\pr_{T\fs{M}} $ takes the following form: $$ (\phi{(U)} \times \fs{F}) \times (\fs{F} \times \fs{F}) \to  \phi{(U)} \times \fs{F}, \quad T\pr_{T\fs{M}}(u,x,y,z) = (u,y).$$
Furthermore, the following diagram is commutative: 
\begin{equation} \label{dig:1}
\begin{tikzcd}
{(\phi{(U)}\times\mathbb{F})\times (\mathbb{F} \times \mathbb{F})} && {\phi{(U)}\times\mathbb{F}} \\
{\phi{(U)}\times \mathbb{F}} && \phi{(U)}
\arrow["{T\mathrm{Pr}_{\mathbb{M}}}", from=1-1, to=1-3]
\arrow["{\mathrm{Pr}_{\mathbb{M}}}", from=1-3, to=2-3]
\arrow["{\mathrm{Pr}_{\mathbb{M}}}", from=2-1, to=2-3]
\arrow["{\mathrm{Pr}_{T\mathbb{M}}}"', from=1-1, to=2-1]
\end{tikzcd}
\end{equation}

Define the mapping $ \mathcal{H}: T(T\fs{M}) \to \pr_{T\fs{M}}^*T\fs{M} $ such that $ T\pr_{T\fs{M}} = \pr_{\fs{M}}^* \circ \mathcal{H} $. Then, the diagram \eqref{dig:1} gives rise to the following diagram:
\begin{equation}
\begin{tikzcd}
{(\phi{(U)} \times \mathbb{F})\times \mathbb{F} \times \mathbb{F}} && {(\phi{(U)} \times \mathbb{F})\times \mathbb{F}  } && {\phi{(U)} \times \mathbb{F}} \\
{\phi{(U)}\times\mathbb{F}} && {\phi{(U)}\times\mathbb{F}} && \phi{(U)}
\arrow["{\mathrm{Id}_{T\mathbb{M}}}", from=2-1, to=2-3]
\arrow["{\mathrm{Pr}_{\fs{M}}^*}", from=1-3, to=2-3]
\arrow["{\mathrm{Pr}_{T\mathbb{M}}}", from=1-1, to=2-1]
\arrow["{\mathcal{H}}", from=1-1, to=1-3]
\arrow["{\mathrm{Pr}_{\mathbb{M}}}", from=2-3, to=2-5]
\arrow["{\mathrm{Pr}_{\fs{M}}^*}", from=1-3, to=1-5]
\arrow["{\mathrm{Pr}_{\mathbb{M}}}", from=1-5, to=2-5]
\end{tikzcd}
\end{equation}
with $ \mathcal{H}(u,x,y,z) = (u,x,y)$ and $ \pr_{\fs{M}}^*(u,y,z)=(u,z) $. Let
$ \overline{\mathcal{H}} = \pr_{\fs{M}}^* \circ \mathcal{H}$, then $ \overline{\mathcal{H}} =T\pr_{T\fs{M}} $.

Consider another chart $ V $ and an isomorphism $ \va : U \to V $. For a given spray $ \s $, we require a transformation rule for a change of charts under $ \va $. Locally, the map $ T\va $ takes the form:
\begin{equation*}
\upphi: \phi({U}) \times \fs{F}  \to \fs{F} \times \fs{F}, \quad \upphi(x,v)=(\va(x),\va'(x)v).
\end{equation*}
Then, the change of chart for $ TT\va $ is given  by
\begin{gather*}
\widehat{\upphi}: \phi{(U)} \times \fs{F} \times \fs{F} \times \fs{F} \to \phi(V) \times \fs{F} \times \fs{F} \times \fs{F}, \quad \widehat{\upphi}=(\upphi, \upphi').
\end{gather*}
The derivative  $ \upphi' $ is given as follows:
\[
\upphi'(x,v)= \begin{bmatrix}
\va'(x) & 0 \\
\va''(x)v & \va'(x)
\end{bmatrix}\mathbf{v}, \quad
\mathbf{v}=\begin{bmatrix} u \\ v \end{bmatrix}, \; u,v \in \fs{F}.
\]
Thus,
\begin{equation}
\widehat{\upphi}(x, v, u, w) \mapsto \big(\va(x), \va'(x)u, \va'(x)v, \va''(x)(u,v) + \va'(x)w \big).
\end{equation}
As mentioned in Section \ref{sec:1}, in a chart $ U $, the local representative of $ \s $ can be expressed as $ {\s_U}=({\s_{U,1}}, {\s_{U,2}}) $, where $ {\s_{U,1}}(x,v)=v $ and $ {\s_{U,2}} $ satisfies Equation \eqref{eq:1}. Similarly, in a chart $ V $, the local representatives of $ \s $ are denoted as ${\s_V}=({\s_{V,1}}, {\s_{V,2}}) $. These representatives satisfy the following equation:
\begin{equation}
\s_V(\va(x),\va'(x)v) = \big(\va'(x)v, \va''(x)(v,v)+\va'(x)\s_{U,2}(x,v)\big),
\end{equation} 
since $ w=  \s_{U,2}(x,v)$, and $ \s_{U,1}(x,v)=\s_{V,1}(x,v)= v $, we can set $ u=v $. Thus,
\begin{gather}\label{gat:1}\nonumber
\s_{V,2}\big(\va(x), \va'(x)v \big) = \va''(x)(v, v)+ h'(x)\s_{U,2}(x,v), \\ 
\mathbb{B}_U\big(\va(x);\va'(x)v,\va'(x)w \big)=\va''(x)(v,w)+\va'(x)\mathbb{B}_V\big(x;v,w \big). 
\end{gather}
\begin{lemma}\label{lem:1}
	Let $ \s $ be a given spray on $ \fs{M} $, and let $ \fs{B} $ be the symmetric bilinear map associated with $ \s $.
	Then, there exists a unique vector bundle morphism 
	\begin{equation} \label{eq:l}
		\mathcal{L}: T(TM) \to \pr_{\fs{M}}^*T\fs{M}
	\end{equation}
	over $ T\fs{M} $
	such that over a chart $ U $, we get
	\begin{equation}
	\mathcal{L}_U(x,v,u,w) = (x,v,w-\fs{B}_U(x;v,u)).
	\end{equation} 
	Furthermore, there exists a unique vector bundle morphism $ \overline{\mathcal{L}} $ such that
	\begin{equation}\label{eq:3}
\overline{\mathcal{L}}_U(x,v,u,w) = (x,w-\fs{B}_U(x;v,u)).
\end{equation} 
\end{lemma}
\begin{proof}
	Let $ V $ be another chart and $ \va : U \to V $ an isomorphism. As explained earlier, the transformation rule for $ (TT\va)_U $ under a change of chart is given by
\begin{gather*}
\widehat{\upphi}: \phi(U) \times \fs{F} \times \fs{F} \times \fs{F} \to \phi(U) \times \fs{F} \times \fs{F} \times \fs{F}\\  \widehat{\upphi}(x, v, u, w) \mapsto \big(\va(x), \va'(x)u, \va'(x)v, \va''(x)(u,v) + \va'(x)w \big).
\end{gather*}
Thus,
\begin{equation}
\mathcal{L}_V \circ \widehat{\upphi}(x, v, u, w) =\big(\va(x), \va'(x)u, \va'(x)w \big).
\end{equation}
Because $ \va''(x)(u,v) $ drops in the fourth coordinates. Therefore, the family of mappings $ \set{\mathcal{L}_U} $
defines a vector bundle morphism over $ T\fs{M} $, and its local expression reveals that it is indeed a vector bundle isomorphism over $ U $. This concludes the proof of the first part. 
To prove the second part let 
\begin{equation}\label{eq:lbar}
	\overline{\mathcal{L}} = \pr_{\fs{M}}^* \circ \mathcal{L}
\end{equation}
 Then, it is a vector bundle morphism satisfying \eqref{eq:3}.
\end{proof}
\begin{theorem}\label{th:con}
Let $ \s $ be a given spray on $ \fs{M} $, with its associated covariant derivative $ \nabla $.
Then there exists a unique vector bundle morphism 
\begin{equation}
K : T(T\fs{M}) \to T\fs{M}
\end{equation} 
such that $ \nabla = K \circ T $, and for all $ C^{k-1} $-vector fields
$ X,Y $ on $ \fs{M} $ the following diagram is commutative:
\[\begin{tikzcd}
{T\mathbb{M}} & {} & {T(T\mathbb{M})} \\
{\mathbb{M}} && {T\mathbb{M}}
\arrow["TX", from=1-1, to=1-3]
\arrow["K", from=1-3, to=2-3]
\arrow["{\nabla_YX}", from=2-1, to=2-3]
\arrow["Y"', from=2-1, to=1-1]
\end{tikzcd}\]
\end{theorem}
\begin{proof}
	The sought-after morphism is, in fact, $ \overline{\mathcal{L}} $, defined by \eqref{eq:lbar} and \eqref{eq:l}, the existence and uniqueness of which were established in Lemma \ref{lem:1}. In a chart $ U $, the expression for $ K = \overline{\mathcal{L}} $ takes the following form:
\begin{gather*}
K_U(x,v): \fs{F} \times \fs{F} \to \fs{F}\\
(u,w) \mapsto w-\fs{B}_U(x;v,u)
\end{gather*}
that fulfills the conditions of the theorem. 
\end{proof}

The expression $ \nabla = K \circ T  $ indicates that $ K $ is  a connection map in the sense defined  by Eliasson \cite{eli} and Vilms \cite{vilms}. We call it a connection map associated with the spray. 
This connection map finds complete characterization through the associated bilinear map  $ \fs{B} $. The differentiability of the spray implies that $ K $ is of class $ C^{k-1} $. 

The linearity of $ \fs{B} $ with respect to the second variable implies that $ K $ is linear on the fibers of $ T(T\fs{M}) $, and therefore $ K $ is a linear connection map. Furthermore, the symmetry of the connection map $ K $ in variables $ u $ and $ v $ results from the symmetric nature of $ \fs{B} $.
\begin{remk}
We define a connection map through the associated covariant of a spray. A natural question arises: can a connection map uniquely determine a spray? Specifically, we inquire whether a covariant derivative alone can determine a spray. In \cite[VIII, $ \S 2 $]{lang}, it was highlighted that in the case of Banach manifolds if a manifold admits a cutoff function and we employ a torsion-free covariant derivative, leading to the existence of the symmetric bilinear mapping $ \fs{B}_U $ for each chart $ U $, then a bijective correspondence emerges between sprays and covariant derivatives.

The situation remains consistent for \fr manifolds, as we utilize torsion-free covariant derivatives. Regarding the presence of a cutoff function, \fr manifolds, especially nuclear manifolds, that admit a smooth partition of unity, possess cutoff functions.
A \fr manifold that admits a smooth partition of unity also possesses a spray. Thus, within our framework, it suffices to assume that manifolds admit smooth partitions of unity. 
\end{remk}
Next, given the significance of second-order tangent bundles (refer to \cite{a,dod3}) and the feasibility of addressing them through our approach, we will delve into a detailed examination. Initially, we demonstrate that a connection map induces a vector bundle structure on the second-order tangent bundle $ T^2\fs{M}$.

Recall that the second-order tangent bundle $ T^2\fs{M} $ is the set of all 2-jets of $ \fs{M} $.
The space $ T^2\fs{M} $ has the fiber bundle structure over $ \fs{M} $. Its bundle projection $ \pr^2_{\fs{M}} : T^2\fs{M} \to \fs{M}$ is defined by $ \pr_{\fs{M}}^2(j^2_x(\gamma))=x$, where  $ \gamma : \rr \to \fs{M} $
is mapping such that $ \gamma(0)=x $. However, it is well known that, in general, $ T^2\fs{M} $ is not a vector bundle.

Let $ \fs{N} $ be a $ C^k $-\fr manifold,  and let $ \va : \fs{M} \to \fs{N} $ be a $ C^k $-mapping, $ k\geq 1 $.
The second-order tangent map $ T^2\va = T^2M \to T^2N $ is defined by $ j^2_x\gamma \mapsto j^2_{\va(x)} (\va \circ \gamma) $. 

Despite being a fiber bundle morphism, $ T^2\va $ is not always linear on the fibers of $ T^2\fs{M} $ and $ T^2\fs{N} $ due to the involvement of second-order derivatives.
Now, we prove that the associated connection map of a spray induces a vector bundle structure on $ \fs{M} $.
\begin{theorem}\label{th.iso}
	Let $ \s $ be a given spray on $ \fs{M} $, with $ K $ being its associated connection map. Then, $ K $ not only induces a vector bundle structure on $ T^2\fs{M} $ over $ \fs{M} $ but also leads to an isomorphism between this vector bundle and the direct sum vector bundle $ T\fs{M} \oplus T\fs{M}$.
\end{theorem} 
\begin{proof}
The proof is identical to the one in \cite[Theorem 4.1]{k1}. Just, in our setting, we need to consider  the following mapping:
\begin{equation}\label{eq:iso}
\Upsilon \coloneq (\pr_{T\fs{M}}, \overline{\mathcal{H}},  K) : T(T\fs{M}) \to T\fs{M} \oplus T\fs{M} \oplus T\fs{M}
\end{equation}
to obtain an isomorphism of fiber bundles over $ \fs{M} $.
\end{proof}
Conversely, the existence of a vector bundle structure on $ T^2\fs{M} $ implies the presence of a connection map. Let $ \mathbf{V}T^2\fs{M} $ be the vertical subbundle of $ T(T^2M) $. The connection map can be obtained by identifying $ \mathbf{V}T^2\fs{M} $ with the pullback of $ T^2\fs{M} $ over itself, denoted as $ (\pr^{2}{\fs{M}})^*T^2 \fs{M}$, and applying the natural mapping of the pullback, which maps $ (\pr^{2}{\fs{M}})^*T^2 \fs{M}$ onto $ T^2 \fs{M} $


%The proof of this statement is almost identical to the proof of \cite[Theorem 4.2]{k1}. We give a brief scheme of the proof with the modifications wherever is needed:
%Let $ (U,\phi_U) $ and $(\pi_{2}^{-1}(U),\Phi)$ be  local trivializations for $ \fs{M} $ and 
% $T^{2}\fs{M}$ respectively, and assume that the restriction to each fiber over $ x \in \fs{M} $ satisfies:
% $ \Phi_{U,x} =  \Phi_{U,x}^1 \times \Phi_{U,x}^2$, where $ \Phi_{U,x}^1, \Phi_{U,x}^2: T_x\fs{M} \to \fs{F} $ are the induced isomorphisms. Let $ (V, \phi_V) $ be a chart for $ \fs{M} $ with $ V \subset U $, and take
% $ \phi_U \coloneq \phi_V \circ (\Phi_{U,x}^{1} \circ (\dd_x \phi_V)^{-1}) $.
%  Now, define the symmetric bilinear mapping
% $$
% \fs{B}_U(y)(u,u) = (\Phi_{U,x}^{2} ([f,x]_2) - (\phi_U \circ f)''(0).
% $$
%where $ f $ is the curve of $ \fs{M} $ that generates the vector $ u $ with respect to $ \phi_U $.
%The values of $ \fs{B}_U(y)(v,u) $ when $ v \neq u $ is defined automatically.
%The agreements of local trivializations for  $ T^\fs{M} $ on all common areas of their domain imply that
%the mapping $ \fs{B} $ satisfy the transformation rule of the change of chart \eqref{gat:1} and therefore leads to 
%the connection map.
In \cite{dod}, it was noted that the induced vector bundle structure on second-order tangent bundles is contingent upon the selection of a connection map. Nevertheless, it was proved that this vector bundle structure remains invariant under conjugate connection maps concerning diffeomorphisms of a manifold. We encounter a similar situation in our context. For the sake of completeness, we will briefly discuss this matter, as the arguments are nearly identical.

We will adopt the concept of the conjugacy of connection maps as introduced in \cite{vas}. Consider two sprays $ \s_1 $ and $ \s_2 $ on $ \fs{M} $, with their associated connection maps being $ K_1 $ and $ K_2 $, respectively. Let $ \upmu $ be a diffeomorphism of $ \fs{M} $. The connection maps $ K_1 $ and $ K_2 $ are called as $ \upmu $-conjugate (or $\upmu$-related) if they satisfy the following condition:
\begin{equation}\label{eq:conj}
	T\upmu \circ K_1 = K_2 \circ T(T \upmu).
	\end{equation}
Let $ (U,\phi) $ and $ (V,\psi) $ be local trivializations of $ \fs{M} $, and $ \upmu_{UV} \coloneq \psi \circ \upmu \circ \phi^{-1} $ the local representation of $ T^2\upmu $. Then, locally Equation \eqref{eq:conj} becomes:
\begin{gather}\label{eq:conjloc}
\dd \upmu_{UV}(\phi(x))\big(-\fs{B}_U(\phi(x))(u,u) \big) = \\ \nonumber
\dd^2 \uptau_{UV}\big(\phi(x)\big)(u,u)-\fs{B}_V \big(\upmu_{UV}(\phi(x)) \big) \Big(\dd \upmu_{UV}(\phi(x))(u), \dd \upmu_{UV}(\phi(x))(u)\Big),
\end{gather}
for every $ (x,u)  \in U \times \fs{F}$.
We require the local representations of $ \upmu $ and its corresponding tangent map $ T\upmu $, as well as the local representations of the connection maps. Detailed computations can be found in \cite[$ \S 1.5.6 $]{dod2}, and we will omit them here. It is important to note that in \cite[$ \S 1.5.6 $]{dod2}, the local representations of connection maps are provided in terms of Christoffel symbols $ \Gamma $, and we must apply the equality
\begin{equation}
	\Gamma(x)(u,v) = - \fs{B}(x)(u,v).
\end{equation}
wherever necessary.
 Again, by using the same computations as presented in \cite[$ \S 8.3 $]{dod2}, we determine the  local representation of $ T^2\upmu $ as follows:
define the mapping  
\begin{gather*}
\lambda_{U}: \pr_{\fs{M}}^2(U) \to U \times \fs{F} \times \fs{F} \\
\lambda_{U} (j^2_x \gamma) = \Big(x, (\phi \circ \gamma)'(0), (\phi \circ \gamma)''(0) 
-\fs{B}(\phi(x)) \big( (\phi \circ \gamma)'(0), (\phi \circ \gamma)'(0)\big)
\Big),
\end{gather*}
for every $ x \in  (\pr_{\fs{M}}^2)^{-1}(U)$. Then $ (U, \phi, \Phi_U^2) $ is the local trivializations of
$ T^2\fs{M} $. Here, the diffeomorphism $ \Phi_U^2: (\pr_{\fs{M}}^2)^{-1}(U) \to \phi(U) \times \fs{F} \times \fs{F} $ is defined by
$$
\Phi_U^2 \coloneq (\phi \times \id_{\fs{F}} \times \id_{\fs{F}}) \circ \lambda_U.
$$
Then the local representation turns into:  
\begin{gather}\label{eq:find}
\Big (\Phi^2_{V, \upmu(x)} \circ T^2_x \upmu \circ (\Phi^2_{U, (x)})^{-1} \Big)(h,k) = \\ \nonumber
= \Big(\dd \upmu_{UV}(y)(h), \dd \upmu_{UV}(y)(k)+ \dd \upmu_{UV}(y)\fs{B}_U(y)(h,h)
+ \dd^2 \upmu_{UV}(y)(h,h)- \\ \nonumber
-\fs{B}_V (\upmu_{UV}(y)) \big( \dd \upmu_{UV}(y)(h), \dd \upmu_{UV}(y)(h)
\big)
\Big),
\end{gather}
for every $ (y,h,k) \in \phi(U) \times \fs{F} \times \fs{F} $ and $ x = \phi^{-1}(y) $. 

Now, if $ K_1 $ and
$ K_2 $ are $ \upmu $-conjugated, then by combining \eqref{eq:find} and \eqref{eq:conjloc} we get:

\begin{gather}\label{eq:fin}
\Big (\Phi^2_{V, \upmu(x)} \circ T^2_x \upmu \circ (\Phi^2_{U, (x)})^{-1} \Big)(h,k) = \\ \nonumber
= \big( \dd \upmu_{UV}(\phi(x))(h), \dd \upmu_{UV}(\phi(x))(k)
\big)
\end{gather}
which implies that $ T^2\upmu : T^2 \fs{M} \to T^2\fs{M} $ is linear. Moreover, it implies that
$$
x \mapsto \Phi^2_{V, \upmu(x)} \circ T^2_x \upmu \circ (\Phi^2_{U, (x)})^{-1}
$$
is differentiable, therefore, $ (T^2\upmu, \upmu) $ is a vector bundle isomorphism.
Thus, the vector bundle structures induced by $ K_1 $ and $ K_2 $ on $ T^2\fs{M} $ are isomorphic.
\section{Associated Linear Symmetric Connection}

In this section, our initial concern is to prove the existence of an injective correspondence between sprays and linear symmetric connections.
We will adapt the notion of nonlinear connection due to Barthel \cite{ba} and Vilms \cite{vilms}, originally formulated for Banach manifolds.

Henceforth, for the sake of simplicity, we assume that  $\fs{M}$ is a  smooth Fr\'{e}chet manifold. 
  Let $ \pr_{\fs{M}}: T\fs{M} \to \fs{M} $ be the tangent bundle, and $ \pr^*_{\fs{M}} T\fs{M}$  the pullback bundle induced by $ \pr_{\fs{M}} $.
  
   Define the mapping 
  \begin{gather}
  \pr_{\fs{M}}!: T(T\fs{M}) \to \pr_{\fs{M}}^*T\fs{M} \\ \nonumber
    \pr_{\fs{M}}^*(v_x)=(x, T\pr_{\fs{M},x}(v_x), \quad \forall v_x \in T_x(T\fs{M})).
  \end{gather}
  The mapping $ \pr_{\fs{M}}! $ is smooth and onto with 
  $ \ker \pr_{\fs{M}}! = \ker T\pr_{\fs{M}} = \mathbf{V}T\fs{M}$ (the vertical subbundle of $ T(T\fs{M}) $).
  Consequently, we have the following exact sequence of vector bundles over $ T\fs{M} $:
 \begin{equation}\label{eq:exact}
 0 \to \mathbf{V}T\fs{M} \xrightarrow{\eu{I}} T(T\fs{M})\xrightarrow{\pr_{\fs{M}}!} \pr^*_{\fs{M}} T\fs{M} \to 0,
 \end{equation}
 where $ \eu{I} $ is the canonical inclusion. A smooth nonlinear connection on $ T\fs{M} $ is a splitting $ \eu{C} $ on the left
 of the exact sequence \eqref{eq:exact}. It means that $ \eu{C} : T(T\fs{M}) \to \mathbf{V}T\fs{M}$ is a map such that
 \begin{equation*}
 	\eu{C} \circ \eu{I} =\Id_{\mathbf{V}T\fs{M}}. 
 \end{equation*}
  The horizontal bundle $ \mathbf{H}T\fs{M}$ is defined as the kernel $\ker \eu{C},  \mathbf{H}T\fs{M} = \ker \eu{C}$. It is well known that in a presence of a connection  we have $ T(T\fs{M})= \mathbf{H}T\fs{M} \oplus \mathbf{V}T\fs{M} $, and  conversely the existence of a subbundle $ \mathbf{H}T\fs{M} $ satisfying
 $ \mathbf{H}T\fs{M} \oplus \mathbf{V}T\fs{M} $ implies the existence of a nonlinear connection on $ T\fs{M} $.
 
 A connection  $ \eu{C} $ is called linear, if in addition,  $ \eu{C}: T(T\fs{M}) \to \mathbf{V}T\fs{M} \to T(T\fs{M}) $
 is $ (T\pr_{\fs{M}},T\pr_{\fs{M}}) $ fiberwise linear. 
 
 Moving forward, we will introduce the definition of symmetric connections.
 
 It is well known that $ T(T\fs{M}) $ possesses two vector bundle structure as a bundle over $ T\fs{M} $:
  one induced by $ T\pr_{T\fs{M}} $, and the othere with  $ T\fs{M} $ as the base manifold. There is a canonical involution $ \mathtt{Inv} : T(T\fs{M}) \to  T\fs{M}$ that interchanges two structures. 
 In \cite[Theorem 1]{sym}, it was proved that for a finite dimensional manifold $ M $, the canonical involution $ \mathtt{Inv} $ is an isomorphism between the primarily and the secondary vector bundle structures on $ T{M} $
 such that the following diagram commutes:
\[\begin{tikzcd}
{T(T{M})} & {} & {T(T{M})} \\
{T{M}} & {} & {T{M}}
\arrow["{\mathtt{Inv}}", from=1-1, to=1-3]
\arrow["{T\Pi_{M}}", from=1-3, to=2-3]
\arrow["{\mathrm{Id}}", from=2-1, to=2-3]
\arrow["{\Pi_{TM}}"', from=1-1, to=2-1]
\end{tikzcd}\]

The mapping $ \mathtt{Inv} $ is self-inverse, and its fixed points precisely correspond to symmetric vectors ($ v \in T(T{M}) $ is called symmetric if $ T\pr_{T{M}}(v) = \pr_{T{M}}(v) $). The assertions made in \cite[Theorem 1]{sym} also hold true for infinite-dimensional \fr manifolds.
 
 We call a connection $ \eu{C} $ symmetric if 
 \begin{equation}
 \eu{C} = \eu{C} \circ \mathtt{Inv}.
 \end{equation}
In the upcoming theorem, we prove that a spray uniquely determines a linear symmetric connection, and vice versa. The proof of this theorem hinges on establishing the relationship between a connection and the associated connection map of a given spray.
 \begin{theorem}\label{th:ch}	
 Suppose $\s  $ is a spray on $ \fs{M} $. Then, there exists a unique linear symmetric connection on $ T\fs{M} $ that is entirely characterized
 by the associated symmetric bilinear mappings of $ \s $. Conversely, if $ \eu{C} $ is a linear symmetric connection on
 $ T\fs{M} $, then there exists a unique spray on $ \fs{M} $ whose associated connection map is determined by $ \eu{C} $.  
 \end{theorem}
 \begin{proof}
 Let $ K: T(T\fs{M}) \to \fs{M} $ be the associated connection map of $ \s $.  Consider a chart $ U $ of $ \fs{M}$, where $ \fs{B}_U $ is the associated symmetric bilinear mapping over $ U $. Let $ K_U(x,v,u,w) = (x,w-\fs{B}_U(x;v,u))$ be the local representation of $ K $ in $ U $. 
 
 Define locally the mapping: 
 \begin{gather}
  \eu{K}: \mathbf{V}T\fs{M} \to T\fs{M},\\ \nonumber
  \eu{K}(x,u,0,v)= (x,v), \quad \forall (x,v) \in U \times \fs{F}.
 \end{gather}
 Since $ K $ is a smooth bundle morphism, taking into account the definition of $\eu{K}$, it follows that $ K $ 
uniquely factors  into $ K= \eu{K} \circ \eu{C} $, where $ \eu{C} : T(T\fs{M}) \to \mathbf{V}(T\fs{M}) $
 is a smooth morphism which is locally given by
 \begin{equation*}
 \eu{C}(x,u,v,w)=(x,u,0,w-\fs{B}_U(x)(v,u))).
 \end{equation*}
Now, let $  v= 0$, and thus $ \eu{C} \circ \eu{I} = \Id_{\mathbf{V}T\fs{M}} $, indicating that $ \eu{C} $ is a connection.
The linearity of $ \fs{B}_U $ with respect to the second variables implies that $\eu{C} $ is linear on the fibers of $ T(T\fs{M}) $, and therefore $ \eu{C} $ is linear. Also, since $ \fs{B}_U $ is symmetric we have
\begin{equation*}
\eu{C}(x,u,v,w) = \eu{C}(x,v,u,w)
\end{equation*}
which implies $ \eu{C} \circ \mathtt{Inv} = \eu{C} $,  because the mapping $ \mathtt{Inv} $ switches the middle coordinates.
Thus, $ \eu{C} $ is symmetric. 

Conversely, suppose that $ \eu{C} $ is a linear symmetric connection. By definition $ \eu{C} $ is the left splitting of the exact sequence \eqref{eq:exact} such that
\begin{equation*}
\eu{C} \circ \eu{I} =\Id_{\mathbf{V}T\fs{M}}.
\end{equation*}

 %Define $ \eu{C}_1 :  T(T\fs{M}) \to \mathbf{V}T\fs{M}$
% the morphism on fibers as identity on element of $ \mathbf{V}_uT\fs{M} $ (vertical vectors) and zero on elements of
%  $ \mathbf{H}_uT\fs{M} $ (horizontal vectors). Thus, $ \eu{C}_1 $ is the left splitting of the exact sequence \eqref{eq:exact}, i.e., $ \eu{C}_1 \circ \eu{I} = \Id_{\mathbf{V}T\fs{M}}$.
 Let $ \mathrm{Pr}_2 $ be the projection of $ T\fs{M} \times T\fs{M} $ onto the second factor. By using the well known identification $ \Bbbk: \mathbf{V}T\fs{M} \simeq T\fs{M} \times  T\fs{M}$, we can define a mapping
 $ \eu{C}_2 :  \mathbf{V}T\fs{M} \to T\fs{M} $ with $ \eu{C}_2 = \mathrm{Pr}_2 \circ \Bbbk   $, such that
 $ (\eu{C}_2, \pr_{\fs{M}}) $ is a vector morphism between $ \mathbf{V}T\fs{M} $ and $ T\fs{M} $.
 
 Define the mapping 
 \begin{equation*}
 K \coloneq \eu{C}_2 \circ \eu{C} : T(T\fs{M}) \to T\fs{M}.
 \end{equation*}
 It is evident that
 $ v \in T\fs{M} $ is horizontal ($ v \in \mathbf{H}T\fs{M} $)
 if and only if
 \begin{equation}\label{eq:cm}
 \ker K = \mathbf{H}T\fs{M}.
 \end{equation}
 We will determine the local representation of $ K $. Let $ (U,\phi) $ be a local trivialization of $ \fs{M} $.
  The exact sequence $ \eqref{eq:exact} $ locally turns into
  \begin{equation*}
  0 \to \phi(U) \times \fs{F} \times \set{0} \times \fs{F} \to \phi(U) \times \fs{F} \times \fs{F} \times \fs{F} \to
\phi(U) \times \fs{F} \times \set{0} \times \fs{F} \to 0,
  \end{equation*}
  so that $ \eu{I}: (x,v,0,w) \mapsto  (x,v,0,w)  $,  $ \pr_{\fs{M}}! (x,u,v) \mapsto (x,u,w) $, and
  $ \Bbbk : (x,u,0,w) \mapsto (x,u,w) $. Locally, $ \eu{C} $ maps $ (x,u,v,w)$ to $(x,u,0, \eu{C}_U(x,u,v,w)) $,
  where $ \eu{C}_U $ exhibit symmetry with respect to $ u  $ and $ v $ and linearity with respect to $ u, v $ and $ w $.
  
  The equality $ \eu{C} \circ \eu{I} = \Id_{\mathbf{V}T\fs{M}} $ implies that $ \eu{C}_U(x,u,0,w) =w $. Now, let us define
  $$ \widehat{\eu{C}}_U \coloneq \eu{C}(x,u,v,w)+w $$ which does not depend on $ w $, and
  \begin{equation*}
  \widehat{\eu{C}}_U(x,u,v,w_1)+\widehat{\eu{C}}_U(x,u,v,w_2) = \widehat{\eu{C}}_U(x,u,v,w_1+w_2).
  \end{equation*}
  
 %Let $(U, \phi_U)  $ be a local trivialization for $ \fs{M} $,
 % $ (\phi_U(U) \times \fs{F}, \overline{\phi}_U) $
 %the corresponding chart for $ T(\fs{M}) $, and  $ (\phi_U(U) \times \fs{F} \times \fs{F} \times \fs{F}, \widehat{\phi}_U) $  the corresponding chart for $ T(T(\fs{M})) $. Here,
 %$ \overline{\phi}_U = (\phi_U \times \id_{\fs{F}}) \circ \upphi_U (\text{tanstion function}) : \pr^{-1}_{T\fs{M}}(U) %\to  \phi_U(U) \times \fs{F} $. Also,
% \begin{gather*}
% \widehat{\phi}_U= \upphi^{-1}_{T\fs{M}}(\pr^{-1}_{T\fs{M}}) \to \phi_U(U) \times \fs{F} \times \fs{F} \times \fs{F} \\
% j^2_u \gamma \mapsto (\phi_U(u), (\phi_U \circ \gamma)'(0)),
%  \end{gather*}
Let $ K_U  $ be the local representation of $ K $ over $ U $. Then,
\begin{gather*}
K_U(x,u,v,w) = (\mathrm{Pr}_2 \circ \Bbbk )(x,u,0,\eu{C}_U(x,u,v,w)) = (x, \eu{C}_U(x,u,v,w)).
\end{gather*}
However, $ \eu{C}_U(x,u,v,w) = w - \widehat{\eu{C}}_U(x,u,v)  $, where $ \widehat{\eu{C}}_U $
is symmetric and linear in $ u $ and $ v $. Thus, we can write
\begin{equation*}
K_U(x,u,v,w) = (x,w- \fs{B}_U(x)(u,v))
\end{equation*}
where $ \fs{B}_U : \phi_U(U) \times \fs{F}  \times \fs{F} \to  \fs{F} $
is a smooth map, and  symmetric and linear in $ u $ and $ v $.
This means $ K $ is a connection map, and $ \fs{B}_U  $ induces a spray whose connection map is $ K $.
 \end{proof}

Next, we show that any linear connection  on $ T\fs{M} $ induces a linear connection on $ T^2\fs{M} $ and vice versa.
To establish this, we rely on the following lemma, which was proven in \cite{duc} for finite-dimensional manifolds, and the proof is nearly identical.

Let $ \fs{V}_1 $ and $ \fs{V}_2 $ be smooth vector bundles over $ \fs{M} $, and $ \va: \fs{M} \to \fs{M} $ a $ C^1 $-mapping. A mapping $ \upphi : \fs{V}_1 \to \fs{V}_2 $ is called a $ \va $-morphism if for any 
$ x \in \fs{M} $, there exists a vector bundle chart $ (U, \psi) $ of $ \fs{V}_1 $ at $ x $, a vector bundle chart
$ (V, \uppsi) $ of $ \fs{V}_2 $ at $ \va{(x)} $, and a $ C^1 $-mapping $ \Phi_{VU} : (U \cap \va^{-1}(U)) \times \fs{F} \to \fs{F} $ such that 
$$ \upphi_x \circ \psi^{-1}  = \uppsi^{-1}_{\va(x)} \circ \Phi_{VU}, \quad \forall x \in U \cap \va^{-1}(V).$$
\begin{lemma}\label{lem:2}
Let $ \upphi: \fs{V}_1 \to \fs{V}_2  $ be a $ \va $-morphism that is isomorphism on fibers. Then, 
$ T_u \fs{V}_1 \simeq T_{\upphi{(u)}} \fs{V}_2 $ for any $ u \in U $ ( $ U$ is a chart for $ \fs{M} $). Furthermore, 
any (linear symmetric) connection on $ \fs{V}_2 $ induces a (linear symmetric) connection on $ \fs{V}_2 $.
\end{lemma}
\begin{proof}
From the identity $ T\pr_{\fs{V}_2} \circ T \phi = T \va \, \circ T \pr_{\fs{V}_2} $ and the assumption that $ \phi $ is an isomorphism on fibers, it follows that $ T_u \fs{V}_1 \simeq T_{\upphi{(u)}} \fs{V}_2 $ for any $ u \in U $.
Now, let $ \eu{C}_2 $ be the left splitting that determines the (linear symmetric) connection on $ \fs{V}_2 $. Let
\begin{equation*}
\eu{C}_1 \coloneq (T\phi\mid_{\mathbf{V}\fs{V}_1})^{-1} \circ \eu{C}_2 \circ T \phi ,
\end{equation*}
 then it is a left splitting
for the exact sequence associated to $ \fs{V}_1 $, i.e., a (linear symmetric) connection on $ \fs{V}_1 $. 
\end{proof}
\begin{theorem}\label{th:ind}
Suppose $\s  $ is a spray on $ \fs{M} $. Any linear symmetric connection on $ T\fs{M} $ induces a linear symmetric connection on the vector bundle $ T^2\fs{M} $, and vice versa.
\end{theorem}
\begin{proof}
By Theorem \ref{th.iso}	the spray $ \s $ induces a vector bundle structure on $ T^2\fs{M} $. Let $ K $ be the associated connection map of $ \s $. 

Suppose that  $ \eu{C} $ is a linear symmetric connection on $ T\fs{M} $.
	The restriction of $ \eu{C} $ to $ T^2\fs{M} $ ($ T^2 \fs{M} $ is a submanifold of $ T(T\fs{M}) $) is a morphism
	of the vector bundles $ T^2\fs{M} $ and $ T\fs{M} $ that is isomorphism on fibers. Therefore, Lemma
	\ref{lem:2} implies that  $ \eu{C} $ induces a linear symmetric connection on $ T^2\fs{M} $. 
	
	Conversely, let $ \mathcal{C} $ be a linear connection on $ T^2\fs{M} $, and define the mapping  
	\begin{equation*}
	\mathcal{C}_1 \coloneq \mathcal{C} \circ \pr_{T^2\fs{M}} : T( T^2\fs{M}) \to T\fs{M}
	\end{equation*}
      Consider the diffeomorphism $ \Upsilon : T(T\fs{M}) \to 
	 T\fs{M} \oplus T\fs{M} \oplus T\fs{M}$ defined in Equation \eqref{eq:iso} which induces an isomorphism on fibers.
	 The second-order tangent bundle $ T^2\fs{M} $ is a submanifold of $ T(T\fs{M}) $ consisting of vectors $ v $ such that 
	 $ \pr_{T\fs{M}}(v) = T\pr_{\fs{M}}(v) $. Since
	 \begin{equation*}
	 	\Theta : T\fs{M} \oplus T\fs{M} \to T\fs{M} \oplus T\fs{M} \oplus T\fs{M}
	 \end{equation*}
	 is  the natural isomorphism  of  $ T\fs{M} \oplus T\fs{M} $ onto
	  $ (\pr_{T\fs{M}} \oplus K \oplus T\pr_{\fs{M}})(T^2\fs{M}) $, it follows that
	 \begin{equation*}
	 \Theta^{-1} \circ (\pr_{T\fs{M}} \oplus K \oplus T\pr_{\fs{M}})(T^2\fs{M}) = T\pr_{\fs{M}} \oplus K (T^2\fs{M}),
	 \end{equation*}
	 and, therefore, $ \mathbf{K} \coloneq T\pr_{\fs{M}} \circ K : T^2\fs{M} \to \fs{M} $ is a diffeomorphism, and determines the vector bundle charts. Let $ \varPi : T\fs{M} \oplus T\fs{M} \to T\fs{M} $ be the projection. Then, $  \varPi\circ \mathbf{K} \circ \mathcal{C}_1  $ gives the morphism of the bundles $ T^2\fs{M} $ and $ T\fs{M} $ 
	 which is an isomorphism on fibers. Therefore, Lemma
	 \ref{lem:2} implies that  $ \mathcal{C} $ induces a connection on $ T\fs{M} $. 
\end{proof}

Next, we characterize nonlinear connections using tangent structures.

	The vertical space $ \mathbf{V}_xT\fs{M} $ at $ v \in T \fs{M} $ is naturally identified with $ T_x\fs{M} $, $ x=\pr_{\fs{M}}(v) $. Define the mapping:
	\begin{gather}\label{eq:verlift}
	\mathtt{Ver}_v: T_x\fs{M} \to \mathbf{V}_vT\fs{M}  \\ \nonumber
	\mathtt{Ver}_v(w) = \derat0 v + t w, \quad  w \in T_x\fs{M}.
	\end{gather}
The mapping $ \mathtt{Ver}_v $ is a linear isomorphism for any $ v \in T\fs{M} $ and is called a vertical lift. Since
$ \ker K = \mathbf{H}T\fs{M} $ (Equation \eqref{eq:cm}), we can easily show that for each $ v \in T_x\fs{M} $, the connection map $ K $ maps $ T_vT\fs{M} $ to $ T_{\pr_{\fs{M}}(v)} \fs{M}$, and satisfies:
\begin{equation}\label{eq:inco}
K (\mathtt{Ver}_v(w)) = w, \quad \forall w \in T_{\pr_{\fs{M}}(v)} \fs{M}.
\end{equation}
Since, for $ h \in T\fs{M} $, the mapping $ \Pi_{\mathbb{M}} \mid_{\mathbf{H}_hT\fs{M}} $ is isomorphism, and
\begin{equation*}
T\pr_{\fs{M},x} : T_hT\fs{M} \to T_x\fs{M} (x =\pr_{T\fs{M}}(h))
\end{equation*}
 is an epimorphism, then
$ T\pr_{\fs{M},h} \mid_{\mathbf{H}_hT\fs{M}} : \mathbf{H}_hT\fs{M} \to T_x\fs{M} $ is an isomorphism. The inverse of the latter isomorphism $ \mathtt{Hor}_h : T_x\fs{M} \to \mathbf{H}_hT\fs{M} $ is called horizontal lift induced 
by the connection.

Now, define the mapping:
\begin{gather}
\jmath : \pr^*_{T\fs{M}} \to \mathbf{V}T\fs{M} \\ \nonumber
\jmath(x_1,x_2) = \mathsf{Ver}_{x_1}(x_2),\; x_1,x_2 \in T\fs{M},\; \pr_{\fs{M}}(x_1) = \pr_{\fs{M}}(x_2).
\end{gather}
For $ u \in T\fs{M} $, define the mapping $ \eu{J}_u : T_uT\fs{M} \to  \mathbf{V}_uT\fs{M}$ by 
\begin{equation*}
\eu{J}_u \coloneq \eu{I}_u \circ \jmath_u \circ \pr_{\fs{M},u}!.
\end{equation*}
The mapping $ \eu{J} $, called the tangent structure, can be considered as an $ \mathcal{E}^{\infty}(\fs{M}) $-linear mapping from $ \mathsf{V}(T\fs{M}) $ to itself, defined by 
\begin{equation*}
	\eu{J}(V)(u) = \eu{J}_u(V_{\pr_{T\fs{M}}(u)})
\end{equation*}
for $ V \in  \mathsf{V}(T\fs{M}), \; u \in T\fs{M}$. Directly, from the definition it follows that $ \eu{J}^2 =0 $, and 
\begin{equation*}
\Img (\eu{J}) = \ker (\eu{J}) =  \mathsf{Sec} (\mathbf{V}T\fs{M}),
\end{equation*}
where $  \mathsf{Sec} (\mathbf{V}T\fs{M}) $ is the space of smooth sections.

The horizontal and vertical lifts are related by 
\begin{equation}\label{eq:impor}
\eu{J}_u \circ \mathtt{Hor}_u = \mathtt{Ver}_u,
\end{equation} 
that is the following diagram is commutative 
\[\begin{tikzcd}
{T_uT\fs{M}} & {} & {T_uT\fs{M}} \\
& {T_{\Pi_\fs{M}(u)}\fs{M}}
\arrow["{\eu{J}_u}", from=1-1, to=1-3]
\arrow["{\mathtt{Ver}_u}", from=1-3, to=2-2]
\arrow["{T\Pi_{\fs{M},u}}"', from=1-1, to=2-2]
\end{tikzcd}\]

Compose  Equation \eqref{eq:impor} from the left by $K_u $ so $ K_u \circ \eu{J}_u \circ \mathtt{Hor}_u =  K_u \circ\mathtt{Ver}_u  $. By \eqref{eq:inco}, the write side is the identity $ \id_{T_{\pr_{\fs{M}}(v)} \fs{M}} $. 
Then, since $ \mathtt{Hor}_u $ is a linear isomorphism, it follows that $ K_u \circ \eu{J}_u $ is its inverse, and hence we obtain:
\begin{equation}\label{eq:last}
K_u \circ \eu{J}_u = T\pr_{\fs{M},u}.
\end{equation}
Let $ \mathsf{V}(\fs{M}) $ and $ \mathsf{V}(T\fs{M}) $ be the modules of smooth vector fields on $ \fs{M} $ and
$ T\fs{M} $, respectively. Let $ \eu{C} $ be a nonlinear (linear) connection on $ T\fs{M} $. It induces a morphism
$ \widehat{\eu{C}} : \mathsf{V}(\fs{M}) \to \mathsf{Sec} (\mathbf{V}T\fs{M}) $ (the space of smooth sections).
Define the mapping:
\begin{gather}
\mathtt{Vp} : \mathsf{V}(T\fs{M})  \to \mathsf{V}(T\fs{M}) \\ \nonumber
\mathtt{Vp}(v) \coloneq
\begin{cases}
1 & \text{if } \widehat{\eu{C}}(v) \;\text{is vertical}\\
0 & \text{if } v \;\text{is horizontal}.
\end{cases}
\end{gather}
The definition of $ \mathtt{Vp} $ directly implies that $\mathtt{Vp}^2=\mathtt{Vp}  $. Now, define
$ \mathtt{Hp} \coloneq \id_{\mathbf{V}T\fs{M}} -  \mathtt{Vp} $ so that we have $\mathtt{Hp}^2=\mathtt{Hp}  $.
We call $ \mathtt{Vp}  $  ($ \mathtt{Hp}  $) the vertical  (horizontal) projector associated to $ \eu{C} $.
\begin{theorem} \label{th:noncon}
	Any linear symmetric connection on the tangent bundle determines a  connection map, and vice versa. 
\end{theorem}
\begin{proof}
Let $ \eu{C} $ be a linear symmetric  connection, and let $ u \in T\fs{M} $. The mapping
\begin{equation*}
	\eu{J}_u\mid_{\mathbf{H}_uT\fs{M}}: \mathbf{H}_uT\fs{M} \to \mathbf{V}_uT\fs{M}
\end{equation*}
 is an isomorphism. Let $ \eu{L}_U $ be its inverse, and define its extension to $ T_uT\fs{M} $ by $ \eu{L}_u \coloneq \eu{L}_u \circ \eu{C}_u $.
Directly from the definition, it follows that $ \eu{L}^2 = 0  $ and $ \Img \eu{L} = \ker \eu{L} = \mathbf{H}T\fs{M} $.
Moreover, $ \eu{L} \circ \eu{J} =  \mathtt{Hp}$ and $ \eu{J} \circ \eu{L} = \mathtt{Vp} $. Now define the linear mapping:
\begin{gather*}
K_u : T_uT\fs{M} \to T\pr_{\fs{M}(u)}\fs{M} \\
K_u = T\pr_{\fs{M}(u)} \circ \eu{L}_u.
\end{gather*}
We need to show that $ K_u \circ \eu{J}_u = T\pr_{\fs{M}(u)} $. Since $ \eu{L}_u \circ \eu{J} = (\mathtt{Hp})_u $
and $ T\pr_{\fs{M}(u)} \circ  (\mathtt{Hp})_u = T\pr_{\fs{M}(u)}$, it follows that $ K_u \circ \eu{J}_u = T\pr_{\fs{M}(u)} $.

Conversely, let $ K_u $ be a connection map. Then, $ K_u \circ \eu{J}_u = T\pr_{\fs{M},u}$ (Equation \eqref{eq:last}).
Thus, since $ T\pr_{\fs{M},u} $ is an epimorphism, $ K_u  $ is also an epimorphism. Let $ \mathbf{H}_u T\fs{M}\coloneq  \ker K_u $
and $ \mathbf{V}_u \coloneq  \ker \eu{J}_U $. From $ K_u \circ \eu{J}_u = T\pr_{\fs{M},u}$  it follows that 
$ \mathbf{H}_u T\fs{M} \cap \mathbf{V}_u T\fs{M} = \set{0}$, and hence $$ \mathbf{H}_u T\fs{M} \oplus \mathbf{V}_u T\fs{M} = T_uT\fs{M} .$$
\end{proof}
\bibliographystyle{amsplain}
\begin{thebibliography}{10}
\bibitem{a3}
M. Aghasi and A. Suri, Splitting theorems for the double tangent bundles of \fr manifolds, Balkan Journal of Geometry and its Applications, Vol. 15, No. 2 (2010)	1-13.
\bibitem{a2}
M. Aghasi, A.R Bahari, C.T.J. Dodson,  G.N.  Galanis, and A. Suri, Second order structures for sprays and connections on \fr manifolds. http://arxiv.org/abs/0810.5261v1.	
\bibitem{a}	
M. Aghasi and C.T.J. Dodson and G.N. Galanis and A. Suri, Infinite-dimensional second order ordinary differential equations via $ T^2M $, Nonlinear Analysis: Theory, Methods \& Applications, Vol. 67, No. 10 (2007) 2829-2838.
10.1016/j.na.2006.09.043
\bibitem{a5}	
M. Aghasi and C.T.J. Dodson and G.N. Galanis and A. Suri, Conjugate connections and differential equations on infinite dimensional manifolds, VIII international colloquium on differential geometry, Santiago de Compostela, 7-11 July 2008, World Scientific, Hackensack, NJ, 227-236, 2009. 
\bibitem{ba}
W. Barthel, Nichtlineare Zusammenhänge und deren Holonomiegruppen, Journal f\"{u}r die reine und angewandte Mathematik, Vol. 212 (1963) 120-149. 
\bibitem{dod3}	
C. Dodson and G. Galanis, Second order tangent bundles of infinite dimensional manifolds,
Journal of Geometry and Physics, Vol. 52, No. 2 (2004) 127-136. 10.1016/j.geomphys.2004.02.005.
\bibitem{dod4}	
C. Dodson, G. Galanis, E. Vassiliou, A generalized second-order frame bundle for \fr manifolds,
Journal of Geometry and Physics, Vol. 55, No. 3 (2005) 291-305. 10.1016/j.geomphys.2004.12.011.
\bibitem{dod2}	
C. Dodson, G. Galanis and E. Vassiliou, E.  Geometry in a \fr context: A projective limit approach (London Mathematical Society Lecture Note Series). Cambridge: Cambridge University Press (2015).10.1017/CBO9781316556092
\bibitem{dod}
C. Dodson, G. Galanis, and E. Vassiliou, Isomorphism classes for Banach vector bundle structures of second tangents, Mathematical Proceedings of the Cambridge Philosophical Society, Vol. 141, No. 3 (2006) 489-496. 10.1017/S0305004106009467
\bibitem{dom} 
P. Dombrowski, On the Geometry of the Tangent Bundle, Journal f\"{u}r die reine und angewandte Mathematik, Vol. 210 (1962) 73-88.
\bibitem{duc} 
 T. V. Duc, Sur la g\'{e}om\'{e}trie diff\'{e}rentielle des fibr\'{e}s vectoriels, Kodai math. Sem.
Rep., Vol. 26 (1975), 349-408.
	  	
\bibitem{k1}
K. Eftekharinasab, Geometry of bounded \fr manifolds,  Rocky Mountain J. Math. Vol. 46, No. 3(2016) 895-913.  10.1216/RMJ-2016-46-3-895 
\bibitem{k3}
K. Eftekharinasab and V. Petrusenko, Finserian geodesics on \fr manifolds, Bulletin of the Transilvania University of Barsov: Mathematics, Informatics, Physics, Series 3
Vol 13, No.1 (2020)  129-151.  https://doi.org/10.31926/but.mif.2020.13.62.1.11
\bibitem{eli}
H. Eliasson, Geometry of manifolds of maps, J. Differential Geom. Vol. 1, No. (1-2) (1967) 169-194. 
10.4310/jdg/1214427887 
\bibitem{sym}
R. Fisher and H. Taquer, Second order tangent vectors in Riemannian geometry, J. Korean Math. Soc., Vol. 36, No.5 (1999) 959-1008.
\bibitem{ga1}
G. Galanis, Universal connections in \fr principal bundles. Period Math Hung, Vol. 54, No. 1
 (2007) 1-13. https://doi.org/10.1007/s-10998-007-1001-z.
 \bibitem{hamilton}
 R. Hamilton, The inverse function theorem of Nash and Moser, Bull. Amer. Math. Soc. (N.S.), Vol.7, No. 1 (1982)
  65-222.
\bibitem{lang}
S. Lang, Fundamentals of differential geometry, Springer, New York, 1999.
\bibitem{neeb}
KH Neeb,  Towards a Lie theory of locally convex groups. Jpn. J. Math. Vol.1, No. 2 (2006) 291-468.
 https://doi.org/10.1007/s11537-006-0606-y
 \bibitem{su1}
A. Suri and M. Moosaei, Bundle of frames and sprays for \fr Manifolds. International Electronic Journal of Geometry, vol.11, no.1 (2018) 1-16. 
\bibitem{vas}
E. Vassiliou, Transformations of linear connections. Period. Math. Hung., Vol. 13, No.4 (1982)289-308. https://doi.org/10.1007/BF01849241
\bibitem{vilms}
J. Vilms, Connections on tangent bundles, J. Differential Geom. Vol. 1, No. (3-4) (1967)  235-243.
10.4310/jdg/1214428091
\end{thebibliography}

\end{document}

%------------------------------------------------------------------------------
% End of journal.tex
%------------------------------------------------------------------------------
