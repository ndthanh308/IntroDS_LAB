\documentstyle{gandb}
\makeatletter
\input epsfig.sty
\makeatother
\def\oth{{\textstyle \frac{1}{3}}}
\def\oqu{{\textstyle \frac{1}{4}}}
\def\nth{{\textstyle \frac{1}{9}}}
\let\<=\langle
\let\>=\rangle
%\def\mathrm#1{{\rm #1}}
\def\vec#1{\mbox{\boldmath{$#1$}}}
\def\div{\mathop{\mathrm{div}}\nolimits}
\evensidemargin 29pt \fussy
\begin{document}
\title{IMPROVED ENVELOPE AND EMITTANCE DESCRIPTION}
{IMPROVED ENVELOPE AND EMITTANCE\protect\\
DESCRIPTION OF PARTICLE BEAMS\protect\\
USING THE FOKKER-PLANCK APPROACH}
\author{J\"URGEN STRUCKMEIER}
\authorhead{J.~STRUCKMEIER}
\address{\it Gesellschaft f\"ur Schwerionenforschung (GSI),
Postfach~11~05~52,\protect\\ 64220~Darmstadt, Germany}
\received{(Received 28 October 1993; in final form 5 April 1994)}
\abstract{Beam dynamics calculations that are based on the
Vlasov equation do not permit the the treatment of
stochastic phenomena such as intra-beam scattering.
If the nature of the stochastic process can be regarded as a
Markov process, we are allowed to use the Fokker-Planck equation
to describe the change of the phase space volume the beam occupies.
From the Fokker-Planck equation we derive
equations of motion for the beam envelopes and for the rms-emittances.
Compared to previous approaches based on Liouville's theorem,
these equations contain additional terms that describe
the temperature balancing within the beam.
Our formalism is applied to the effect of intra-beam scattering
relevant for beams circulating in storage rings near
thermodynamical equilibrium.
In this case, the Fokker-Planck coefficients can be treated as
adiabatic constants of motion.
Due to the simplified analysis based on `beam moments',
we obtain fairly simple equations that allow us to estimate
the growth rate of the beam emittance.}
\keywords{envelope equation, Fokker-Planck equation,
storage ring, intra-beam scattering\newline~\newline
\hspace*{-22mm}Published in: Part.~Acc.~\underline{45}, 229 (1994)}
\maketitle
\section{INTRODUCTION}
Analytical approaches in the optics of charged particle beams
including self-forces of the beam usually
assume that Liouville's theorem applies for the phase
space distribution function $f$ in the $6$-dimensional `$\mu$~space'.
According to this theorem, the volume of an element of phase space
populated by particles remains constant in time.
From Liouville's theorem we can derive straightforwardly the
equation of motion for $f$, commonly referred to as
the Vlasov or the `collisionless' Boltzmann equation in the general case.
Although it is practically impossible to integrate the
Vlasov equation directly, it may serve as the
starting-point for analytical investigations under certain conditions.
As an example, we quote the so-called
`Kapchinskij-Vladimirskij' model\cite{kapvla}
for the distribution function $f$, which simplifies the Vlasov
equation to yield a closed set of differential equations
for the envelopes of an unbunched beam.

Nevertheless, we have to recall that Liouville's
theorem applies only if the beam transformation can be
regarded as a deterministic process.
On the basis of Vlasov's equation,
we are thus unable to describe the variety
of stochastic phenomena of beam optics --
such as turbulent heating during
the relaxation phase of free field energy,
thermal equipartitioning within the beam,
or the effect of intra-beam scattering in storage rings.

Recently, Bohn\cite{bohn} made an attempt to extend
the self-consistent Vlasov-Poisson treatment of beam optics.
He included the Fokker-Planck equation in his analysis to describe
the dilution of the phase space due to turbulent heating for a
mismatched `sheet' beam.

Since the Fokker-Planck equation is even
more complicated than the Vlasov equation, attempts for
a direct integration without referring to the `sheet beam'
model do not appear worth while.
The complexity of the problem of directly integrating the
Vlasov equation led to attempts to simplify the
analytical description of beam optics.
A largely accepted method has been presented by
Lapostolle\cite{lapostolle} and Sacherer\cite{sacherer}
describing the beam optics in terms of `root-mean-square' (rms)
beam moments and their respective equations of motion.
In this paper, we apply this idea to derive
`moment equations' to the Fokker-Planck equation.

We thus obtain additional moment terms that --- at least in principle ---
allow us to describe any Markov process within the beam
not yet covered by the Vlasov approach.
The remaining task is to determine the Fokker-Planck
coefficients for the non-Liouvillean process to be investigated.

In this article, we refer to the work of Jansen\cite{jansen},
who calculated these coefficients for the effect of
intra-beam scattering for charged particle beams
near thermodynamical equilibrium.
Including collision effects is particularly important
for highly charged heavy ions and high phase space densities.
We restrict ourselves to cases where the effect of collisions is small
compared with external forces or the smooth part of the self-fields.
Hence we do not treat the beam's turbulent heating phase, but the
long-term behavior of `relaxed' beams undergoing intra-beam scattering,
which typically occurs for beams circulating in storage rings.

Prior to mathematical details, we discuss in section~\ref{fopla} the
restrictions under which this Fokker-Planck
approach is applicable to charged particle beams.
Proceeding further, we will set up the extended
moment equations for the frame of reference moving with the beam
(section~\ref{moments}).
These moment equations are transformed to the
laboratory system in section~\ref{labor}, using the
usual `trace space' notation.
We thus obtain the extended envelope and emittance equations
(section~\ref{rms-eq}).
After discussing the concepts of `excess field energy'
and `temperature' of ion beams in sections~\ref{excess}
and~\ref{temperature}, respectively, we review in section~\ref{coeff}
the relationship between the scattering-related Fokker-Planck
coefficients and the beam parameters.
In the following, we derive differential
equations that describe the change of beam emittances
due to a temperature balancing process.
This is performed for unbunched beams in section~\ref{unbunched}
and for the more general cases of bunched beams and coasting beams
in storage rings in section~\ref{bunched} and~\ref{coasting}.
Finally, we will discuss some numerical examples
in section~\ref{numeric}.
The implications of our theory for beam transport devices
such as quadrupole channels and storage rings will be outlined.
In addition, it is shown that the nature of {\em numerical\/} emittance
growth effects observed in computer simulations of charged particle
beams can also be explained with the help of Fokker-Planck approach.
\section{THE FOKKER-PLANCK APPROACH FOR CHARGED PARTICLE BEAMS} \label{fopla}
The 6-dimensional phase space distribution function
$f = f(\vec{x},\vec{p};t)$ that represents a charged
particle beam satisfies Liouville's theorem
\begin{equation} \label{liouville}
\frac{d f}{d t} = 0 \; ,
\end{equation}
if the particles do not interact and if the time evolution of
their coordinates follows Hamilton's canonical equations.
For charged particle beams whose electric self-fields
must be taken into account, Eq.~(\ref{liouville}) remains
true if the space charge potential $V^{\mathrm{sc}}$ is
sufficiently smooth that it can be treated analogous to
an external defocusing potential.

Writing (\ref{liouville}) explicitly and inserting the
equation of motion of a particle with the mass $m$ and charge $q$
under the influence of external focusing fields $\vec{F}^{\mathrm{ext}}$
and the smooth part of the self-fields $\vec{E}^{\mathrm{sc}}$
$$
\dot{\vec{p}} = \vec{F}^{\mathrm{ext}} + q\vec{E}^{\mathrm{sc}} \; ,
$$
we directly obtain the Vlasov equation.
The Poisson equation describes the
relationship between the distribution function $f$ and
the space-charge field $\vec{E}^{\mathrm{sc}}$:
$$
\div\vec{E}^{\mathrm{sc}} = -4\pi \, q \int f \; d^3 p \;.
$$
If we want to include irreversible effects in our analysis
of beam optics, we must give up the strictly deterministic
viewpoint of the Vlasov equation.
Examples of such effects are turbulent heating due to mismatch,
energy loss due to radiation damping, residual gas
scattering, cooling, intra-beam scattering.
All the processes are associated with a change of the
$6$-dimensional phase space volume the beam occupies,
hence they may not be analyzed on the basis of the Vlasov approach.

The action of non-Liouvillean effects can be described
mathematically replacing Eq.~(\ref{liouville})
by the Boltzmann equation:
$$
\frac{df}{dt} = \left[ \frac{\partial f}
{\partial t}\right]_{\mathrm{NL}} \;.
$$
Explicitly, this equation can be written in
its non-relativistic form as:
\begin{equation} \label{bz1}
\frac{\partial f}{\partial t} + \frac{\vec{p}}{m} \cdot
\vec{\nabla}_x f + \left( \vec{F}^{\mathrm{ext}} +
q\vec{E}^{\mathrm{sc}} \right) \cdot \vec{\nabla}_p f =
\left[ \frac{\partial f}{\partial t}\right]_{\mathrm{NL}} \;.
\end{equation}
The r.h.s.\ term thus formally describes the
change of the beam's phase space volume due to the
non-Liouvillean effects.
If this term can be neglected, Eq.~(\ref{bz1}) reduces to
the Vlasov equation.

Under certain conditions to be discussed below, the
non-Liouvillean effects can be modeled by the
Fokker-Planck equation (see References.~\cite{jansen,risken}, for example):
$$
\left[ \frac{\partial f}{\partial t} \right]_{\mathrm{NL}} =
- \sum_i \frac{\partial}{\partial p_i} \left\{ F_i(\vec{p},t) \, f \right\} +
m^2 \sum_{i,j} \frac{\partial^2}{\partial p_i \partial p_j}
\left\{ D_{ij}(\vec{p},t) \, f \right\} \;.
$$
Herein $D_{ij}$ stands for the `diffusion tensor' elements, whereas the
$F_i$ are referred to as the `drift vector' components.

The Fokker-Planck approach is based on the assumption that
there exists a specific time interval $\Delta t$
such that the phase space distribution
$f(t+\Delta t)$ at time $t+\Delta t$ only depends on $f(t)$ and
{\em not\/} on $f$ at any earlier time.
During this time interval $\Delta t$, any `memory'
of earlier events is canceled.
This means that the time interval $\Delta t$
covers a considerable number of events, e.g.\
particle-particle collisions if we want to analyze
the effect of intra-beam scattering.
During this interval, the phase space distribution function $f$
must remain approximately unchanged.
In other words, the velocity changes of all particles
due to the non-Liouvillean effect during successive time intervals
must be statistically independent.
A stochastic process that has this property is
referred to as a `Markov process'.

It can be concluded that the Fokker-Planck approach is
only applicable if $f$ changes slowly compared with the
typical time constant of the non-Liouvillean effect.

The most difficult part in our task to include
non-Liouvillean effects in our analysis of beam optics
is to determine the Fokker-Planck coefficients
$D_{ij}$ and $F_i$.
Especially the effect of turbulent heating of the beam due
to mismatch -- where the beam is far from thermodynamical
equilibrium -- implies that that these coefficients are
explicitly time dependent.
This makes it extremely complicated -- or even impossible --
to evaluate the Fokker-Planck coefficients on a rigorous basis,
hence forces to apply heuristic models\cite{bohn}.

If we want to treat cases where the beam can be assumed to
be near thermodynamical equilibrium, things become easier.
In these cases, the Fokker-Planck coefficients can be
taken to be adiabatic constants of motion.
This is true for beam circulating in storage rings
for times that are large compared to the internal beam dynamics'
time scale (for example the betatron frequency).

Eq.~(\ref{bz1}) represents the self-consistent equation
of motion for a phase space distribution $f$.
It describes a charged particle beam under the
influence of external focusing forces,
the {\em smooth\/} part of the self-fields due to its
macroscopic charge density function, and the non-Liouvillean
effects of a Markov process within the beam, which
may change the phase space volume the beam occupies.
\section{THE EXTENDED MOMENT EQUATIONS} \label{moments}
In order to circumvent a lengthy direct integration
of Eq.~(\ref{bz1}), we switch from the total information
on the beam that is contained in the phase space distribution
function $f$ to physical quantities that contain
less information on the beam but still can serve as
measures for the beam quantities of interest.
A usual way to simplify our description of beam dynamics
is to derive equations of motion for the beam's second
central moments from the equation of motion for $f$.

With $f$ as the normalized particle phase space distribution function,
the beam moments can be defined by integrals
over all phase space, $\tau$:
\begin{eqnarray*}
\<x_i^2\>   & = & \int x_i^2   \, f \; d\tau \\
\<x_i p_i\> & = & \int x_i p_i \, f \; d\tau \\
\<p_i^2\>   & = & \int p_i^2   \, f \; d\tau \\
\<x_i E_i\> & = & \int x_i E_i \, f \; d\tau \\
\<p_i E_i\> & = & \int p_i E_i \, f \; d\tau \;.
\end{eqnarray*}
We obtain -- as an example -- the time derivative of $\< x_i^2 \>$ using
\begin{equation} \label{hderi}
\frac{d}{dt} \<x_i^2\> = \int x_i^2 \,
\frac{\partial   f}{\partial   t} d\tau
\end{equation}
and inserting $\partial   f / \partial   t$ from
Eq.~(\ref{bz1}) into (\ref{hderi}).
It is physically reasonable to assume that
the phase space density $f$ as well as all
its derivatives vanish at the boundary
of the populated phase space.
As a consequence, all solvable parts of the integrals
evaluate to zero inserting the integration boundaries.
The non-vanishing moment terms assemble for each phase space plane
$i = 1,2,3$ to the following coupled set of equations:
\begin{eqnarray} \label{deri}
\frac{d}{dt} \<x_i^2\> - \frac{2}{m} \<x_i p_i\> & = & 0 \nonumber \\
\frac{d}{dt} \<x_i p_i\> - \frac{1}{m} \<p_i^2\> -
\<x_i F^{\mathrm{ext}}_i\> - q\, \<x_i E_i\> & = & \<x_i F_i \> \\
\frac{d}{dt} \<p_i^2\> - 2\, \<p_i F^{\mathrm{ext}}_i \> -
2q\, \<p_i E_i\> & = & 2\, \<p_i F_i \>+2m^2\< D_{ii} \> \nonumber \;.
\end{eqnarray}
The left-hand terms of Eqs.~(\ref{deri})
have been derived first by Sacherer\cite{sacherer},
assuming that Liouville's theorem applies.
The non-zero right-hand terms\cite{soerensen} originate in the
Fokker-Planck equation that describe the non-Liouvillean effects.
We observe that only the diagonal terms of the diffusion tensor
appear in the moment equations.
All terms of the right-hand side of (\ref{deri}) are
proportional to the Fokker-Planck coefficients.
As a consequence, they vanish
if the non-Liouvillean effects can be neglected.
The set of coupled differential equations
then reduces to Sacherer's equations, as expected.
\section{TRANSFORMATION TO THE LABORATORY SYSTEM} \label{labor}
For circular structures, the dependency of the orbital angular
velocity on the particle momentum is included defining an
`effective mass' $m_{z;b}$ for the longitudinal equation of motion.
In the center of momentum frame moving with the beam,
it is given by\cite{lawson}:
$$
m_{z;b} = \frac{m}{\eta \gamma^2} \; ,
$$
with
\begin{equation} \label{etadef}
\eta = \cases{\gamma^{-2} - \gamma_t^{-2} &for a circular orbit
\cr \noalign{\vskip3mm}
\gamma^{-2} &for a straight motion.}
\end{equation}
\vspace*{1pt}\\
Herein, $\gamma_t$ denotes the {\em transition } $\gamma$.
Thus, in the beam's frame of reference, we have
$m_{z;b} \equiv m$ for a straight motion, whereas
$m_{z;b} = m / (1-\gamma^2/\gamma_t^2)$ in a ring structure.
The definition of a constant `effective mass' for the
longitudinal beam dynamics is justified, if the time scale
of the non-Liouvillean effects is large compared to the
particle's revolution time.
If this is not true, one has to use a time dependent
$\eta$ instead of the constant one.
This replacement does not affect the structure
of the beam moment equations.

Usually we may formulate the equation of motion
(\ref{bz1}) in the non-relativistic limit,
if we refer to the beam's center of momentum frame.
In this paper, we will restrict ourselves to these cases.
On the other hand, we are interested in the beam
properties as they occur in the laboratory system.
We therefore must transform all quantities from the beam frame
(index $b$) into the laboratory system (index $\ell$).
In the latter system, the time scale is `stretched' by
the relativistic mass factor $\gamma$ compared to
the particle's proper time.
Following the usual `trace space'\cite{lawson}
formulation of beam optics, we replace the proper time $t$
as the independent variable by the distance
$s$ along the focusing structure measured in the laboratory system:
$$
s \equiv (\Delta s)_\ell = c \beta \cdot (\Delta t)_\ell =
c \beta \gamma \cdot (\Delta t)_b \equiv c\beta\gamma\cdot t \;.
$$
We further introduce the `trace space' coordinates via
$$
x_1 \equiv x_b = x_\ell \qquad , \qquad
x_2 \equiv y_b = y_\ell \qquad , \qquad
x_3 \equiv z_b = \gamma \, (\Delta z)_\ell \equiv \gamma z_\ell\; ,
$$
and
$$
p_1 \equiv p_{x;b} = m c \beta \gamma \cdot x'_\ell \quad , \quad
p_2 \equiv p_{y;b} = m c \beta \gamma \cdot y'_\ell \quad , \quad
p_3 \equiv p_{z;b} = m c \beta \eta^{-1} z'_\ell \;.
$$
The electric space charge fields as well as the external focusing
forces transform according to:
$$
E_{x;b} = \gamma^{-1} E_{x;\ell} \qquad , \qquad
E_{y;b} = \gamma^{-1} E_{y;\ell} \qquad , \qquad
E_{z;b} = E_{z;\ell} \;.
$$
The drift vector components $F_i$ stand for the frictional
forces that are part of the non-Liouvillean effects.
To first order, the drift coefficients $F_i$ can be
approximated by Stokes' law:
$$
F_i = -\beta_{f;i} \cdot p_i \; ,
$$
wherein the $\beta_{f;i}(t)$ are usually designated as
`dynamic friction coefficients'.
With the external focusing force functions $k_{x,y,z}(s)$ that
describe {\em linear\/} external focusing forces,
the moment equations (\ref{deri}) become, skipping all
indices $\ell$:
\begin{eqnarray} \label{deri2}
\frac{d}{ds} \<x^2\> - \, 2\<xx'\> & = & 0 \nonumber \\
\frac{d}{ds} \<xx'\> - \<x'^2\> + \, k_x^2(s) \<x^2\> -
\frac{q}{m c^2 \beta^2 \gamma^3} \<x E_x\> & = &
-\frac{\beta_{f;x}}{c \beta \gamma} \, \<xx'\> \\
\frac{d}{ds} \<x'^2\> + \, 2 k_x^2(s) \<xx'\> -
\frac{2q}{m c^2 \beta^2 \gamma^3} \<x' E_x\> & = &
-\frac{2\beta_{f;x}}{c \beta \gamma} \, \<x'^2\> +
\frac{2\< D_{xx}\>}{c^3 \beta^3 \gamma^3} \nonumber \;.
\end{eqnarray}
The corresponding set of equations holds for the $y$-direction.
For the $z$-direction, we obtain:
\begin{eqnarray} \label{deri3}
\frac{d}{ds} \<z^2\> - \, 2\<zz'\> & = & 0 \nonumber \\
\frac{d}{ds} \<zz'\> - \<z'^2\> + \, k_z^2(s) \<z^2\> -
\frac{q\cdot \eta}{m c^2 \beta^2 \gamma} \<z E_z\> & = &
-\frac{\beta_{f;z}}{c \beta \gamma} \, \<zz'\> \\
\frac{d}{ds} \<z'^2\> + \, 2 k_z^2(s) \<zz'\> -
\frac{2q\cdot \eta}{m c^2 \beta^2 \gamma} \<z' E_z\> & = &
-\frac{2\beta_{f;z}}{c \beta \gamma} \, \<z'^2\> +
\frac{2\< D_{zz}\>}{c^3 \beta^3 \gamma^5} \nonumber \;.
\end{eqnarray}
We conclude this section with the remark that the sets of
moment equations (\ref{deri2}) and (\ref{deri3}) are
not compatible with single particle equations of motion if
the diffusion tensor elements $\< D_{ii}\>$ cannot be neglected.
This is not astonishing if we recall that the
Fokker-Planck equation models a {\em stochastic\/} process,
i.e.\ the evolution of a probability distribution.
In contrast, the Vlasov approach is strictly {\em deterministic},
hence allows us -- at least in principle --
to trace separately the phase space motion
of each individual particle due to all forces acting on it.
\section{THE EXTENDED ENVELOPE AND RMS-EMITTANCE EQUATIONS} \label{rms-eq}
Following further Sacherer's formalism, we define
the rms-emittance $\varepsilon_x(s)$ as
\begin{equation} \label{epsrms}
\varepsilon_x^2(s) \, = \,  \<x^2\> \<x'^2\> - \<x x'\>^2
\end{equation}
and combine the first and the second equation of
(\ref{deri2}) in order to set up a differential equation
for $\sqrt{\<x^2\>}$.
This quantity is of particular importance in the study of charged
particle beams, since it is proportional to the actual
beam width in the $x$-direction:
\begin{equation} \label{kv1}
\frac{d^2}{ds^2} \sqrt{\<x^2\>} + \frac{\beta_{f;x}}{c \beta \gamma}
\frac{d}{ds} \sqrt{\<x^2\>} + k_x^2(s) \sqrt{\<x^2\>} -
\frac{q}{m c^2 \beta^2 \gamma^3} \frac{\<xE_x\>}{\sqrt{\<x^2\>}} -
\frac{\varepsilon_x^2(s)}{\sqrt{\<x^2\>^3}} = 0 \;.
\end{equation}
Comparing Eq.~(\ref{kv1}) with Sacherer's
`rms envelope equation'\cite{sacherer}, we observe that
one additional term containing the first order
derivative of $\sqrt{\<x^2\>}$ appears in our theory.

The `dynamical friction coefficients' $\beta_{f;i}$ are always positive.
Mathematically, this implies that all variations of
the envelope function $\sqrt{\<x^2\>}$ are damped.
As a consequence for periodic structures, eventual
mismatch oscillations of the beam envelopes fade away.
In this sense, the `dynamical friction' leads to a
`self-matching' of the beam to the periodic structure.
In the laminar beam limit ($\varepsilon_x \rightarrow 0$),
Eq.~(\ref{kv1}) represents a damped oscillator equation,
which becomes linear for $E_x \propto x$.
The amplitude of any solution function then
decreases with the relaxation time $\tau_f$ of
$$
\tau_f = 2/\beta_f \;.
$$
This value thus provides us with the order of magnitude for the
relaxation time of mismatch oscillations of the beam envelope, hence with
the time the beam needs to match itself to a periodic focusing structure.
We will discuss a numerical example in section~\ref{numeric}.

In general, $\<xE_x\>$ and $\varepsilon_x^2$ are not
constants of motion but unknown functions of $s$.
This accounts for the fact that Eq.~(\ref{kv1})
is not closed, hence cannot be used directly
to evaluate the envelope function.
In the following, we will discuss the conditions
under which Eq.~(\ref{kv1}) can be combined
with additional differential equations in order to
establish a closed set of equations that can be integrated.

On the basis of Eqs.~(\ref{deri2}), the derivative of the
rms-emittance (\ref{epsrms}) is readily calculated to give
\begin{equation} \label{epsderi1}
\frac{1}{\<x^2\>} \frac{d}{ds} \varepsilon_x^2(s) =
\frac{2q}{m c^2 \beta^2 \gamma^3}
\left( \<x'E_x\> - \frac{\<xx'\>}{\<x^2\>} \<xE_x\>\right) -
2\left( \frac{\beta_{f;x}}{c \beta \gamma} \frac{\varepsilon_x^2(s)}{\<x^2\>} -
\frac{\< D_{xx}\>}{c^3 \beta^3 \gamma^3} \right) \;.
\end{equation}
The equation for the $z$-direction may be written
$$
\frac{1}{\<z^2\>} \frac{d}{ds} \varepsilon_z^2(s) =
\frac{2q\eta}{m c^2 \beta^2 \gamma}
\left( \<z'E_z\> - \frac{\<zz'\>}{\<z^2\>} \<zE_z\>\right) -
2\left( \frac{\beta_{f;z}}{c \beta \gamma} \frac{\varepsilon_z^2(s)}{\<z^2\>} -
\frac{\< D_{zz}\>}{c^3 \beta^3 \gamma^5} \right) \;.
$$
In the following section, we analyze the physical meaning
of those moments that contain electric field components $E_i$.
Thereafter, the terms related to the Fokker-Planck
coefficients will be investigated.
We will show that they describe the relaxation
of temperature differences within the beam.
\section{THE `EXCESS FIELD ENERGY' OF A CHARGED PARTICLE BEAM} \label{excess}
As has been shown previously\cite{wang,host,stho},
the sum of the beam moments $\<x' E_x\>$, $\<y' E_y\>$,
and $\<z' E_z\>$ can be correlated to the change of the
electrostatic field energy $W$ constituted by
all charges of the beam:
\begin{equation} \label{dw}
\frac{d W}{ds} = -N q \, \Bigl(
\<x' E_x\> \, + \<y' E_y\> \, + \<z' E_z\> \Bigr) \,.
\end{equation}
If the charge distribution is {\em uniform },
the space charge field components
$E_x$, $E_y$, and $E_z$ are {\em linear\/}
functions of $x$, $y$, and $z$, respectively.
We conclude from (\ref{dw}), that the derivative of
the field energy $W_u$ of a uniform charge distribution
can be expressed as
$$
\frac{d W_u}{ds} = -N q \left(
\frac{\<xx'\>}{\<x^2\>} \<x E_x\> \, +
\frac{\<yy'\>}{\<y^2\>} \<y E_y\> \, +
\frac{\<zz'\>}{\<z^2\>} \<z E_z\> \right) \;.
$$
We thus obtain for the difference between a real and a uniform beam:
\begin{equation} \label{dwdwu}
\frac{d}{ds} \left( W - W_u \right) =
-N q \left( \<x' E_x\> - \frac{\<xx'\>}{\<x^2\>}\<xE_x\> +
\mbox{(similar $y$- and $z$-terms)}\right) \;.
\end{equation}
If we compare beams of the same rms-dimensions, we learn that
the {\em minimum\/} field energy applies to the {\em uniform\/}
density profile\cite{host}.
Consequently, $W-W_u$ defines the additional (`excess') field energy
the beam possesses, if its charge density is not uniform.
Since $W_u$ constitutes the minimum field energy
of a beam of given size, at maximum the `excess' part of the
total field energy can be converted into thermal beam energy.

Once this phase space reorganization process
has been completed, which means that a self-consistent
phase space distribution possessing a specific equilibrium
`excess' field energy has been established,
only fluctuations of $W-W_u$ can occur.
We conclude that the change of the `excess field energy' can be
regarded as an onset effect that does not govern the long-term
beam behavior.
Simulations\cite{strekla,wang} as well as theoretical
investigations\cite{anderson} have shown that the
thermalization process takes place very rapidly
within a few plasma periods.
\section{THE TEMPERATURE OF A CHARGED PARTICLE BEAM} \label{temperature}
As usual in statistical physics, we relate the temperature
of an ensemble of particles to their `incoherent' motion.
Since charged particle beams change their size
while passing through a focusing device,
we must subtract the `coherent' part of the particle
velocities from the total velocities in order to obtain
physically reasonable expressions for the beam temperatures.

The temperatures in all spatial directions are not necessarily the same.
We thus have to define them separately for each direction.
In order to determine the temperature $T_x$,
we must first calculate $g(x,x';s)$ denoting
the projection of the phase space distribution
function $f$ onto the $x,x'$-subspace at $s$:
$$
g(x,x';s) = \int f \; dy \, dy' \, dz \, dz' \;.
$$
The scaled `coherent' velocity $\xi'(x;s)$ in the $x$-direction
as a function of $x$ is given by:
\begin{equation} \label{xiprime}
\xi'(x;s) = \frac{\int x' g(x,x';s) \; dx'}{\int g(x,x';s) \; dx'} \;.
\end{equation}
Now the laboratory $x$-temperature $kT_x$ in energy units
can be defined as\cite{lawson}:
$$
k T_x = m c^2 \beta^2 \gamma \int (x' - \xi')^2 \, f \; d\tau \; ,
$$
which is easily simplified to
\begin{equation} \label{temp-x}
k T_x = m c^2 \beta^2 \gamma \Bigl( \< x'^2 \> - \< \xi'^2 \> \Bigr) \;.
\end{equation}
In general, it is not possible to calculate
$\xi'(x;s)$ for a given distribution function $f$
in a closed analytical form.
As an example that can be treated analytically,
we quote the `K-V' model\cite{kapvla}
for the distribution function $f$ of an unbunched beam.
It has the property that all projections of $f$ onto $2$-dimensional
subspaces lead to uniformly populated ellipses.
For this case, the integrals contained in (\ref{xiprime})
are easily solved to give:
\begin{equation} \label{xiprimekv}
\xi'(x;s) = \frac{\< x x' \>}{\< x^2 \>} \, x \qquad \leadsto \qquad
\< \xi'^2 \> = \frac{\< x x' \>^2}{\< x^2 \>} \;.
\end{equation}
For non-K-V distributions, we can use (\ref{xiprimekv}) only
as an approximation for the `coherent' share of the particle velocities.
Calculating the temperature according to Eq.~(\ref{temp-x}),
the related error vanishes together with the
`coherent' motion, i.e.\ if \mbox{$\< \xi'^2 \> = 0$}.
This occurs repeatedly at positions, where the beam width
in $x$-direction takes on a maximum or a minimum value.
We thus do not introduce a systematic error using
(\ref{xiprimekv}) for all types of phase space distributions $f$.

Under these circumstances, the temperatures are correlated
to the second order beam moments only.
We thus obtain for the transverse temperatures in the laboratory system:
$$
k T_x = \frac{\<p_{x;\ell}^2\>}{m\gamma} =
m c^2 \beta^2 \gamma \cdot \frac{\varepsilon_x^2(s)}{\< x^2 \>} \;.
$$
In the same frame, the longitudinal temperature $T_z$ is given by
$$
k T_z = \frac{1}{|\eta |} m c^2 \beta^2 \gamma \cdot
\frac{\varepsilon_z^2(s)}{\< z^2 \>} =
m c^2 \beta^2 \gamma \cdot
\frac{\bar{\varepsilon}_z^2(s)}{\< z^2 \>} \; ,
$$
defining $\bar{\varepsilon}_z(s) = \varepsilon_z(s) /\sqrt{|\eta |}$
as the `effective longitudinal emittance'.
In these definitions, we must take the absolute value of $\eta$
since the change of the longitudinal beam emittance due to
particle-particle collisions occurs in a symmetric way above
and below transition energy.

The equilibrium temperature $T$ is simply the
average of the $x$-, $y$-, and $z$-temperatures:
\begin{equation} \label{temp3}
\frac{k T}{m c^2 \beta^2 \gamma} = \frac{1}{3} \left(
\frac{\varepsilon_x^2(s)}{\<x^2\>} +
\frac{\varepsilon_y^2(s)}{\<y^2\>} +
\frac{\bar{\varepsilon}_z^2(s)}{\<z^2\>} \right) \;.
\end{equation}
If the longitudinal temperature need not to be taken into account, we
can restrict ourselves to the average of the transverse temperatures.
The transverse equilibrium temperature $T_\perp$ can then be defined as:
\begin{equation} \label{temp2}
\frac{k T_\perp}{m c^2 \beta^2 \gamma} = \frac{1}{2} \left(
\frac{\varepsilon_x^2(s)}{\<x^2\>} +
\frac{\varepsilon_y^2(s)}{\<y^2\>} \right) \;.
\end{equation}
Depending on the physics of our system, we have to decide
whether to use (\ref{temp3}) or (\ref{temp2}) as the
appropriate expression for the equilibrium beam temperature
to be inserted into Eq.~(\ref{epsderi1}).
For example, the dynamics of bunched beams as well as
beams within dispersive systems require a
complete $3$-dimensional description.
The case of an unbunched beam that passes through a
non-dispersive system allows a simplified treatment.
For this case, we can restrict ourselves to the
use of (\ref{temp2}) as the equilibrium beam temperature
of the `transverse' phase space.
\section{THE FOKKER-PLANCK COEFFICIENTS FOR INTRA-BEAM SCATTERING}\label{coeff}
So far, the procedure of deriving extended moment equations
has been performed formally by speaking in general of
`non-Liouvillean' effects to be described by the Fokker-Planck equation.
In other words, up to now nothing has been said about
the Fokker-Planck coefficients $\beta_{f;i}$ and $D_{ii}$.
As an example of a physical effect that does not obey Liouville's theorem,
we will treat the process of intra-beam scattering.
In order to learn how the Fokker-Planck coefficients for the effect
of intra-beam scattering are correlated to the physical beam properties,
it is necessary to study the process of Coulomb scattering of an
ensemble of particles in detail.
This has been done extensively by Jansen\cite{jansen},
treating the more general `many body problem of particles
interacting through an inverse square force law'.
We therefore can restrict ourselves to
results that apply in this context.

The method of evaluating the Fokker-Planck coefficients
for a intra-beam scattering effects of a charged particle
beam can be summarized as follows:
\begin{itemize}
\item in the first step, the velocity changes of a test particle
due to scattering from a single beam particle as a function of the test
particle's initial velocity and impact parameter are calculated,
\item second, the expression obtained in the first step is
averaged over all possible impact parameters,
\item finally, averaging over all particle velocities
of the beam is performed.
This means that the velocity distribution of the beam
must be taken into account.
\end{itemize}
For simplicity, we assume that the {\em equilibrium\/} particle
beam has an {\em isotropic\/} Maxwellian velocity distribution.
If the actual beam is in a state not too far from this equilibrium state,
it is justifiable to assume that friction as well as the
diffusion processes are also isotropic.
Then only one diffusion coefficient $D$ in conjunction with a single
friction coefficient $\beta_f$ appears in our equations:
$$
D \equiv \<D_{xx}\> = \<D_{yy}\> = \<D_{zz}\> \qquad , \qquad
\beta_f \equiv \beta_{f;x} = \beta_{f;y} = \beta_{f;z} \;.
$$
Under these circumstances, $D$ turns out to be proportional
to the `dynamical friction coefficient' $\beta_f$, yielding
the well-known Einstein\cite{einstein} relation:
\begin{equation} \label{diffu}
D = \beta_f \cdot \frac{k T_b}{m} = \beta_f \cdot \gamma\frac{k T_\ell}{m}
\end{equation}
As the result of the averaging procedures,
$\beta_f$ is given by\cite{jansen,reiser}:
\begin{equation} \label{friction}
\beta_f = \frac{16 \sqrt{\pi}}{3} \; n \, c
\left( \frac{q^2}{m \gamma c^2} \right) ^2 \cdot
\left( \frac{m \gamma c^2}{2kT} \right)^{3/2} \cdot \; \ln \Lambda \;.
\end{equation}
In these expressions, $kT$ denotes the equilibrium beam temperature
in energy units, $m$ the particle rest mass, $q$ its charge,
and $\gamma$ the relativistic mass factor,
$n$ stands for the average particle density and $\ln \Lambda$
for the so-called Coulomb logarithm.

As usual for phenomena involving Coulomb scattering, we must establish
a reasonable upper limit for the maximum
impact parameter in order to keep $\Lambda$ finite.
For non-neutralized systems, it is suggested by Jansen\cite{jansen}
to identify the maximum impact parameter with the average distance
between the particles rather than with the Debye length.
This leads to the following formula for $\Lambda$:
$$
\Lambda = \frac{3}{2} \cdot \frac{kT}
{q^2 \cdot n^{1/3}} \;.
$$
Since $\beta_f$ depends only logarithmically on
$\Lambda$, any result from our approach will not
depend critically on the exact value of $\Lambda$.
Inserting (\ref{diffu}) into Eq.~(\ref{epsderi1}) yields
\begin{equation} \label{epsderi1a}
\frac{1}{\<x^2\>} \frac{d}{ds} \varepsilon_x^2(s) =
\frac{2q}{m c^2 \beta^2 \gamma^3}
\left( \<x'E_x\> - \frac{\<xx'\>}{\<x^2\>} \<xE_x\>\right) -
2 k_f \left( \frac{\varepsilon_x^2(s)}{\<x^2\>} -
\frac{kT}{m c^2 \beta^2 \gamma} \right) \; ,
\end{equation}
with the abbreviation $k_f = \beta_f / c\beta\gamma$.
We observe that the term proportional to $k_f$
agrees with the temperature relaxation ansatz\cite{erlangen}
commonly used to describe temperature relaxation phenomena.

In the following sections, we insert the expressions (\ref{temp3}) and
(\ref{temp2}) for the equilibrium temperatures into Eq.~(\ref{epsderi1a}).
This will lead to fairly simple growth rate formulae suitable to
estimate the effect of intra-beam scattering on the beam emittance.
\section{THE EMITTANCE EQUATION FOR UNBUNCHED BEAMS} \label{unbunched}
Provided that the longitudinal and the transverse particle motion
can approximately be treated as decoupled, it is convenient to
define a `transverse equilibrium temperature' $T_\perp$
according to (\ref{temp2}).
We insert this expression into Eq.~(\ref{epsderi1a}):
\begin{equation} \label{epsderiunbu}
\frac{1}{\<x^2\>} \frac{d}{ds} \varepsilon_x^2(s) +
k_f \left(
\frac{\varepsilon_x^2(s)}{\<x^2\>} -
\frac{\varepsilon_y^2(s)}{\<y^2\>} \right) =
\frac{2q}{m c^2 \beta^2 \gamma^3}
\left( \<x'E_x\> - \frac{\<xx'\>}{\<x^2\>} \<xE_x\> \right) \;.
\end{equation}
The terms appearing on the right-hand side are related to
the change of `excess field energy' (\ref{dwdwu}), hence
to the `onset' effect that the beam particles rearrange
to form a stationary density profile.
If we want to analyze the `long-term' beam behavior,
we can disregard these terms.

As has been shown by Sacherer\cite{sacherer},
the moment $\< x E_x \>$ contained in Eq.~(\ref{kv1})
can be expressed in terms of second order moments,
in the case that the transverse charge density has elliptical symmetry:
\begin{equation} \label{ellisymm}
\< x E_x \> = \frac{I}{c\beta} \, \frac{\sqrt{\< x^2 \>}}
{\sqrt{\< x^2 \>} + \sqrt{\< y^2 \>}} \;.
\end{equation}
Computer simulations show that this is a reasonable assumption
for unbunched beams confined by linear focusing forces --
even if the space charge forces are non-linear.

With the help of (\ref{ellisymm}), we can
combine Eqs.~(\ref{kv1}) and (\ref{epsderiunbu}) to
establish a closed set of coupled differential equations
for the `long-term' values of envelopes and emittances
containing solely second order beam moments:
\begin{eqnarray} \label{kv2}
\frac{1}{\< x^2 \>} \frac{d}{d s} \varepsilon_x^2(s) + k_f \left(
\frac{\varepsilon_x^2(s)}{\< x^2 \>} - \frac{\varepsilon_y^2(s)}{\< y^2 \>}
\right) & = & 0 \nonumber \\
\frac{1}{\< y^2 \>} \frac{d}{d s} \varepsilon_y^2(s) + k_f \left(
\frac{\varepsilon_y^2(s)}{\< y^2 \>} - \frac{\varepsilon_x^2(s)}{\< x^2 \>}
\right) & = & 0 \nonumber \\
\frac{d^2}{d s^2} \sqrt{\<x^2\>} + k_f \frac{d}{d s} \sqrt{\<x^2\>} +
k_x^2(s) \sqrt{\<x^2\>} -
\frac{K/2}{\sqrt{\<x^2\>} + \sqrt{\<y^2\>}} -
\frac{\varepsilon_x^2(s)}{\sqrt{\<x^2\>^3}} & = & 0 \nonumber \\
\frac{d^2}{d s^2} \sqrt{\<y^2\>} + k_f \frac{d}{d s} \sqrt{\<y^2\>} +
k_y^2(s) \sqrt{\<y^2\>} -
\frac{K/2}{\sqrt{\<x^2\>} + \sqrt{\<y^2\>}} -
\frac{\varepsilon_y^2(s)}{\sqrt{\<y^2\>^3}} & = & 0 \nonumber \\
& &
\end{eqnarray}
with $K$ denoting the dimensionless `generalized perveance'
$$
K = \frac{2qI}{m c^3 \beta^3 \gamma^3} \;.
$$
In order to gain a better insight into the mathematical
implications of these equations, we introduce $r_t$
as the ratio of the transverse beam temperatures
\begin{equation} \label{temprat}
r_t(s) = \frac{T_x}{T_y} = \frac{\varepsilon_x^2(s)}{\< x^2 \>}
\cdot  \frac{\< y^2 \>}{\varepsilon_y^2(s)} \; ,
\end{equation}
and rewrite the first two equations of (\ref{kv2})
in an equivalent form
\begin{eqnarray} \label{epsderi4}
\frac{d}{d s} \ln \varepsilon_x^2(s) & = & k_f \,
\bigl( r_t^{-1}(s) - 1 \bigr) \nonumber \\
\frac{d}{d s} \ln \varepsilon_y^2(s) & = & k_f \,
\bigl( r_t(s) - 1 \bigr) \;.
\end{eqnarray}
We observe that the rms-emittance in the $x$-direction
increases as long as the temperature $T_x$
is smaller than $T_y$.
At the same time, $\varepsilon_y$ decreases until a temperature
balance in both transverse directions is achieved.
These changes take place in the opposite
direction as long as $T_x > T_y$.
If we add Eqs.~(\ref{epsderi4}), we find a quantity
that increases in any case:
\begin{equation} \label{entro}
\frac{d}{d s} \ln \varepsilon_x^2(s) \varepsilon_y^2(s) =
k_f \cdot \frac{\bigl[ 1 - r_t(s) \bigr]^2}{r_t(s)} \geq 0 \;.
\end{equation}
The product $\varepsilon_x^2(s) \varepsilon_y^2(s)$
can be regarded as the 4-dimensional transverse
rms-emit\-tance $\varepsilon_\perp^4(s)$.
It increases as long as the transverse
beam temperatures are not balanced.
This indicates that the expression
$\ln \varepsilon_x(s) \varepsilon_y(s)$
can serve as a measure for the transverse beam entropy\cite{lalaglu}.

We now formally integrate Eq.~(\ref{entro}):
\begin{equation} \label{epsint}
\frac{\varepsilon_x^2(s)\, \varepsilon_y^2(s)}
     {\varepsilon_x^2(0)\, \varepsilon_y^2(0)}
= \exp \left( k_f \int\limits_0^s
\frac{\bigl[ 1 - r_t(z) \bigr] ^2}
{r_t(z)} \, dz \right) \geq 1 \;.
\end{equation}
The inequality holds since $k_f$ as well as the
integrand cannot take on negative values.
It is not possible to solve Eq.~(\ref{epsint}) for
$\varepsilon_x\varepsilon_y$ since the integrand
itself is a function of the transverse emittances.
This dependence cancels out, if the ratio of the
transverse emittances comes close to unity
$$
\varepsilon_x(s) \approx \varepsilon_y(s) \;.
$$
We can apply this approximation in cases, where the
main temperature differences have already relaxed and the
resulting temperature fluctuations are due
to variations of the transverse beam widths.
With this restriction in mind, the temperature ratio
(\ref{temprat}) can be replaced by the `envelope ratio':
$$
r_t(s) \approx r_e(s) = \frac{\< y^2 \>}{\< x^2 \>} \; ,
$$
yielding
$$
\varepsilon_\perp^4(s) = \varepsilon_\perp^4(0) \cdot \exp \left(
k_f \int\limits_0^s
\frac{\bigl[ 1 - r_e(z) \bigr] ^2}{r_e(z)} \, dz \right) \;.
$$
For periodic beam focusing systems, it is helpful to
introduce the following dimensionless quantity $I_e$:
\begin{equation} \label{envint}
I_e = \frac{1}{S} \int\limits_0^S
\frac{\bigl[ 1 - r_e(z) \bigr] ^2}{r_e(z)} \, dz \; ,
\end{equation}
with $S$ as the period length.
In the following sections, we will refer to $I_e$
as the `ellipticity' of the focusing system.
Since the integrand cannot take on negative
values, we find that $I_e \geq 0$.
The integral depends solely on the course of
the envelope ratio $r_e(s)$ which a matched beam
defines while passing through one focusing period.
Thus, $I_e$ is a characteristic of the specific focusing system only.

Reinserting the time as the independent variable, the emittance ratio
$\varepsilon_\perp(t) / \varepsilon_\perp(0)$
increases in time as
$$
\frac{\varepsilon_\perp(t)}{\varepsilon_\perp(0)} =
\exp \left\{ \oqu \, \beta_f \, I_e \cdot t \right\} \;.
$$
Consequently, the $e$-folding time $\tau_{\mathrm{ef}}$ for
$\varepsilon_\perp(t) / \varepsilon_\perp(0)$ is given by
\begin{equation} \label{e-fold}
\tau_{\mathrm{ef}}^{-1} = \oqu \, \beta_f \, I_e \;.
\end{equation}
We observe that the rate of the emittance growth due to
intra-beam scattering depends, besides $\beta_f$, on the
`ellipticity' $I_e$ of the beam envelopes along the focusing structure.
If $I_e$ does not vanish -- as it is always the case in strong
focusing systems -- a scattering induced growth of the emittance
can never be avoided.

Note that (\ref{e-fold}) is valid only as long as the particle
density $n$, hence the `dynamical friction coefficient'
$\beta_f$, is not appreciably changed.
This means that Eq.~(\ref{e-fold}) provides us with the
{\em instantaneous\/} emittance growth rate for given beam parameters.

In section~\ref{numeric}, we will present
the results of numerical calculations.
Of course, the effect of emittance growth due to
weak particle-particle collisions can be neglected for
{\em linear\/} structures, since the time the beam resides within
the structure is small compared to $\tau_{\mathrm{ef}}$.
Nevertheless, the relation (\ref{e-fold}) gives us an insight
into the physical consequences of processes that involve statistically
fluctuating forces -- as they also occur due to simplifications
in computer simulations of charged particle beams.
We will shortly touch this subject in the last section.
\section{THE EMITTANCE EQUATION FOR BUNCHED BEAMS} \label{bunched}
For the general $3$-dimensional treatment, we insert (\ref{temp3})
for the equilibrium temperature into Eq.~(\ref{epsderi1}):
\begin{equation} \label{epsderi2}
\frac{1}{\<x^2\>} \frac{d}{ds} \varepsilon_x^2(s) =
\frac{2q}{m c^2 \beta^2 \gamma^3} \!
\left( \! \<x'E_x\> - \frac{\<xx'\>}{\<x^2\>} \<xE_x\> \! \right) -
\frac{2k_f}{3} \! \left( \!
\frac{2\varepsilon_x^2(s)}{\<x^2\>} -
\frac{\varepsilon_y^2(s)}{\<y^2\>} -
\frac{\bar{\varepsilon}_z^2(s)}{\<z^2\>} \! \right)
\end{equation}
The corresponding equations can be written for
$\varepsilon_y^2(s)$ and $\varepsilon_z^2(s)$.
If we add these three equations, we can apply
Eq.~(\ref{dwdwu}) in order to set up a differential
equation that correlates the changes of the
rms-emittances with the change of the `excess'
space charge field energy $W-W_u$ of a charged particle beam:
\begin{equation} \label{sr2}
\frac{1}{\<x^2\>} \frac{d \varepsilon_x^2(s)}{ds} +
\frac{1}{\<y^2\>} \frac{d \varepsilon_y^2(s)}{ds} +
\frac{1}{\<z^2\>} \frac{d \bar{\varepsilon}_z^2(s)}{ds} =
\frac{-2}{m c^2 \beta^2 \gamma^3 N}
\frac{d}{ds} \bigl[ W(s) - W_u(s) \bigr] \,.
\end{equation}
Note that all temperature terms contained
in Eq.~(\ref{epsderi2}) cancel.
For that reason, Eq.~(\ref{sr2}) has exactly
the same form as the equations
derived earlier\cite{wang,host,stho}, using the
Vlasov equation as the starting-point.

As before, we may neglect the field energy terms if
we want to analyse the long-term beam behavior.
Then Eq.~(\ref{epsderi2}) reduces to
\begin{equation} \label{epsderi3}
\frac{1}{\<x^2\>} \frac{d}{ds} \varepsilon_x^2(s) =
-\frac{2k_f}{3} \left(
\frac{2\varepsilon_x^2(s)}{\<x^2\>} -
\frac{\varepsilon_y^2(s)}{\<y^2\>} -
\frac{\bar{\varepsilon}_z^2(s)}{\<z^2\>} \right) \; ,
\end{equation}
again likewise for the $y$- and $z$-directions.

Together with the rms-envelope equations (\ref{kv1}) for all
the spatial coordinates, and the expressions for
$\< xE_x \>$, $\< yE_y \>$, and $\< zE_z \>$
for bunched beams\cite{sacherer}, these
differential equations form a closed set.

Eq.~(\ref{epsderi3}) can be written in an equivalent form
in terms of temperature ratios, as defined in Eq.~(\ref{temprat})
\begin{equation} \label{lnepsderi}
\frac{d}{d s} \ln \varepsilon_x^2(s) = \frac{2k_f}{3} \, \left(
\frac{T_y(s)}{T_x(s)} + \frac{T_z(s)}{T_x(s)} - 2 \right) \; ,
\end{equation}
and similar for $\varepsilon_y^2$ and $\bar{\varepsilon}_z^2$.
Depending on the actual temperature ratios, the expression
in parentheses can be positive or negative.
This means that according to the direction of the temperature
flow within the beam, the rms-emittance $\varepsilon_x^2(s)$
can increase or decrease.
In contrast, the product $\varepsilon_x^2\varepsilon_y^2\bar{\varepsilon}_z^2$
can only increase during the balancing process.
This is verified, if we add (\ref{lnepsderi}) to the
similar equations for $\varepsilon_y^2$ and $\bar{\varepsilon}_z^2$:
\begin{equation} \label{entro3}
\frac{d}{d s} \ln \varepsilon_x^2(s)
\varepsilon_y^2(s) \bar{\varepsilon}_z^2(s) =
\frac{2k_f}{3} \, \left(
\frac{(1-r_{xy})^2}{r_{xy}} +
\frac{(1-r_{xz})^2}{r_{xz}} +
\frac{(1-r_{yz})^2}{r_{yz}} \right) \geq 0 \;.
\end{equation}
For the sake of clarity,
we used abbreviations for the temperature ratios, namely
$$
r_{xy}(s) = \frac{T_y(s)}{T_x(s)} \qquad , \qquad
r_{xz}(s) = \frac{T_z(s)}{T_x(s)} \qquad , \qquad
r_{yz}(s) = \frac{T_z(s)}{T_y(s)} \;.
$$
As already demonstrated in the previous section,
Eq.~(\ref{entro3}) can be formally integrated:
\begin{equation} \label{t-growth}
\frac{\varepsilon_{\mathrm{tot}}^3(S)}{\varepsilon_{\mathrm{tot}}^3(0)} \equiv
\frac{\varepsilon_x(S) \, \varepsilon_y(S) \, \bar{\varepsilon}_z(S)}
{\varepsilon_x(0) \, \varepsilon_y(0) \, \bar{\varepsilon}_z(0)}
= \exp \, \Bigl\{ \oth k_fS \, (I_{xy} + I_{xz} + I_{yz}) \Bigr\} \; ,
\end{equation}
with $I_{xy}$, $I_{xz}$, and $I_{yz}$ denoting the three
possible integrals of the temperature ratio functions.
For example, the dimensionless quantity $I_{xy}$ is given by:
\begin{equation} \label{xy-int}
I_{xy} = \frac{1}{S} \int\limits_0^S
\frac{\bigl[ 1 - r_{xy}(s) \bigr] ^2}{r_{xy}(s)} \, ds \geq 0
\qquad , \qquad r_{xy}(s) = \frac{\varepsilon_y^2(s)}{\< y^2 \>}
\frac{\< x^2 \>}{\varepsilon_x^2(s)} \;.
\end{equation}
From Eq.~(\ref{t-growth}), the average $e$-folding time
$\tau_{\mathrm{ef}}$ for the emittance ratio
$\varepsilon_{\mathrm{tot}}(t) / \varepsilon_{\mathrm{tot}}(0)$
is calculated to give
\begin{equation} \label{tau3}
\tau_{\mathrm{ef}}^{-1} = \nth \, \beta_f \, (I_{xy} + I_{xz} + I_{yz}) \;.
\end{equation}
We learn that a temperature balancing process --
which is driven by the fluctuating component of the
interaction potential -- is always accompanied by
an increase of the total beam emittance $\varepsilon_{\mathrm{tot}}$.
If in a periodic system the temperature imbalance is
restored periodically due to the specific beam handling,
the integrals $I_{xy}$, $I_{xz}$, and $I_{yz}$ are positive, hence
a repeated, not saturating growth of the emittance occurs.
\section{THE EMITTANCE EQUATION FOR COASTING BEAMS IN STORAGE RINGS}
\label{coasting}
If two beam particles circulate within a storage ring at different
longitudinal velocities, they will take on any relative position along
the ring circumference in the course of their lifetime within the ring.
This means that we do not observe a correlation between the longitudinal
locations of the beam particles and their respective velocities:
$$
\< zz' \> = 0 \qquad \leadsto \qquad \< z^2 \> = \mathrm{const.}
$$
In place of $z'$, we use in the following the particle's relative
momentum deviation $\Delta p / p$ from the reference particle;
$z'$ can be expressed in terms of the relative revolution
time deviation as
$$
z' = -\frac{\Delta \tau}{\tau} \;.
$$
The ratio of the relative revolution time deviation to
$\Delta p / p$ is given by the factor $-\eta$,
which has already been defined in Eq.~(\ref{etadef}):
$$
\eta = -\frac{\Delta \tau}{\tau} \bigg/ \frac{\Delta p}{p} \qquad \leadsto
\qquad z' = \eta \, \frac{\Delta p}{p} \equiv \eta \,\delta \;.
$$
All expressions containing the longitudinal emittance can then
be replaced by:
$$
\bar{\varepsilon}_z^2(s) = |\eta |^{-1} \varepsilon_z^2(s) =
|\eta | \, \<z^2 \> \< \delta^2 \> \;.
$$
The coupled set of equations (\ref{epsderi3}) describing
the emittance changes can now be written as:
\begin{eqnarray}\label{eps3-d}
\frac{1}{\<x^2\>} \frac{d}{ds} \varepsilon_x^2(s) & = &
-\frac{2k_f}{3} \left(
\frac{2\varepsilon_x^2(s)}{\<x^2\>} -
\frac{ \varepsilon_y^2(s)}{\<y^2\>} -
|\eta |\, \< \delta^2 \> \right) \nonumber \\
\frac{1}{\<y^2\>} \frac{d}{ds} \varepsilon_y^2(s) & = &
-\frac{2k_f}{3} \left(
\frac{2\varepsilon_y^2(s)}{\<y^2\>} -
\frac{ \varepsilon_x^2(s)}{\<x^2\>} -
|\eta |\, \< \delta^2 \> \right) \\
|\eta |\, \frac{d}{ds}\< \delta^2 \> & = &
-\frac{2k_f}{3} \left( 2 |\eta |\, \< \delta^2 \> -
\frac{ \varepsilon_x^2(s)}{\<x^2\>} -
\frac{ \varepsilon_y^2(s)}{\<y^2\>} \right) \; , \nonumber
\end{eqnarray}
neglecting, as before, effects that change the `excess field energy'.
If we add the Eqs.~(\ref{eps3-d}), all right-hand side terms cancel,
yielding:
\begin{equation} \label{piwi1}
\frac{1}{\<x^2\>} \frac{d}{ds} \varepsilon_x^2(s) +
\frac{1}{\<y^2\>} \frac{d}{ds} \varepsilon_y^2(s) +
|\eta |\, \frac{d}{ds} \< \delta^2 \> = 0 \;.
\end{equation}
Provided that we are allowed to treat the transverse beam sizes
as adiabatic constants, Eq.~(\ref{piwi1}) can be rewritten as:
$$
\frac{d}{ds} \left[
\frac{\varepsilon_x^2(s)}{\<x^2\>} +
\frac{\varepsilon_y^2(s)}{\<y^2\>} +
|\eta |\, \< \delta^2 \> \right] = 0 \;.
$$
Integration leads to the constant of motion that has
been derived by Piwinski\cite{piwinski}:
\begin{equation} \label{piwi2}
\frac{\varepsilon_x^2(s)}{\<x^2\>} +
\frac{\varepsilon_y^2(s)}{\<y^2\>} +
|\eta |\, \< \delta^2 \> = \mathrm{const.}
\end{equation}
For strong focusing rings, this adiabatic invariant
does {\em not\/} exist, since we can no longer assume
the transverse beam dimensions to stay approximately constant
along the focusing period.
For this case, we must integrate the coupled set of equations
(\ref{eps3-d}) together with the envelope equations of (\ref{kv2})
in order to determine the change of both the transverse emittances
and the momentum spread due to intra-beam scattering effects.
Since a temperature imbalance within the beam is periodically
reestablished by the strong transverse focusing, no state of equilibrium,
as suggested by (\ref{piwi2}), is ever reached.
In terms of our approach, this means that the temperature ratio
integrals, $I_{xy}$, $I_{xz}$, and $I_{yz}$,
as defined in (\ref{xy-int}), do not vanish.
Strong focusing devices
thus always `generate' a positive $e$-folding time $\tau_{\mathrm{ef}}$
for the total beam emittance $\varepsilon_{\mathrm{tot}}(t)$:
$$
\tau_{\mathrm{ef}}^{-1} = \nth \, \beta_f \, (I_{xy} + I_{xz} + I_{yz}) \;.
$$
\section{RESULTS OF NUMERICAL CALCULATIONS} \label{numeric}
\subsection{Quadrupole Channels} \label{quadchannel}
% Figure environment removed
With quadrupole lenses, a net focusing effect can only be obtained
if we combine quadrupoles of different focusing orientation
in a doublet or other alternating gradient arrangements.
As a consequence, it is not possible to keep a beam
azimuthally round along a structure that contains quadrupoles.
It follows that the integral (\ref{envint}) is always greater than zero,
leading to a finite growth rate even for perfectly matched beams.
Since the imbalance of the transverse beam temperatures
is restored by each quadrupole, no equilibrium
can ever be reached.
We thus do not find a relaxation of the growth rate,
but always a finite emittance growth gradient.

According to (\ref{e-fold}), the growth rate
depends on the beam parameters, yielding a specific
value of the `dynamical friction coefficient'
$\beta_f$ (\ref{friction}), and the lattice parameters, which
are associated with specific value of the `ellipticity' $I_e$.
\begin{table}
\begin{center}
\begin{tabular}{|ll|} \hline
ion & $^{40}$Ar$^{1+\vphantom{^X}}$ \\
energy & $4.75$~KeV/amu \\
period length $S$ & $4.0$~m \\
zero current phase advance $\sigma_0$ & $60^\circ$ \\
depressed phase advance $\sigma$ & $15^\circ$ \\
beam current $I$ & $0.25$~mA \\
initial rms emittances $\varepsilon_{x,y}(0)$ & $3.125\times 10^{-6}$~m \\
ellipticity $I_e$ (matched beam) & $0.135$ \\
friction coefficient $\beta_f$ & $11.1$~s$^{-1}$ \\
emittance $e$-folding time $\tau_{\mathrm{ef}}$ & $2.7$~s \\ \hline
\end{tabular}
\caption{List of parameters for the Periodic Quadrupole Channel simulation}
\label{quadtab}
\end{center}
\end{table}
An example of a low energy beam transport channel is
plotted in Fig.~\ref{sim-quad}.
Using beam parameters as listed in Tab.~\ref{quadtab}
and matching the beam perfectly to the channel,
we find a value of $I_e = 0.135$ for the `ellipticity'.
These settings lead to the $e$-folding time for $\varepsilon_\perp$
of $\tau_{\mathrm{ef}} = 2.7$~s.
During this space of time, the beam would pass
$6.4\times 10^5$ focusing periods.

The growth rate of the transverse emittances
per period amounts to
$$
\frac{\varepsilon_\perp(S)}{\varepsilon_\perp(0)} - 1
\approx 1.5 \times 10^{-6} \;.
$$
\subsection{Storage Rings}
% Figure environment removed
The emittance growth effect due to temperature imbalance is not
negligible, if the time the beam resides within the focusing structure
is of the order of the $e$-folding time (\ref{e-fold}) or more.
This is the case in circular devices such as storage rings.

We base our calculations on the geometry and the beam data
of the `Experimental Storage Ring' (ESR)\cite{franzke,franzetal},
which started operation at GSI in 1990.

As an approximation, we neglect the change of the transverse
beam widths due to the ring dispersion.
Due to these simplifications, we can calculate
the envelope dynamics using the envelope equations
contained in (\ref{kv2}), together with the set of emittance
equations (\ref{eps3-d}).
Of course, (\ref{eps3-d}) can also be combined with a more
sophisticated method for evaluating the envelope functions.
Namely, the local ring dispersion $\alpha_p(s) = \Delta x / (\Delta p/p)$
should be taken into account in a refined model.

\begin{table}
\begin{center}
\begin{tabular}{|ll|} \hline
\mbox{ion} & $^{20}$Ne$^{10+\vphantom{^X}}$ \\
energy & $250$~MeV/amu \\
period length $S$ & 108.36 m \\
horizontal tune $Q_h$ & $2.31$ \\
vertical tune $Q_v$ & $2.25$ \\
$\eta = \gamma^{-2} - \gamma_t^{-2}$ & $0.493$ \\
beam current $I$ & $7$~mA \\
initial rms emittances $\varepsilon_{x,y}(0)$ & $0.25\times 10^{-6}$~m \\
initial rms momentum spread $\Delta p/p$ & $1.3\times 10^{-4}$ \\
ellipticity $I_{xy}$ & $4.710$ \\
ellipticity $I_{xz}$ & $2.689$ \\
ellipticity $I_{yz}$ & $2.382$ \\
friction coefficient $\beta_f$ & $0.1306$~s$^{-1}$ \\
horiz.\ emittance $e$-folding time $\tau_{x,\mathrm{ef}}$ & $10.5$~s \\
vert.\  emittance $e$-folding time $\tau_{y,\mathrm{ef}}$ & $13.7$~s \\
long.\  emittance $e$-folding time $\tau_{z,\mathrm{ef}}$ & $ 1.9$~s \\
total   emittance $e$-folding time $\tau_{\mathrm{ef}}$   & $ 7.0$~s \\ \hline
\end{tabular}
\caption{List of parameters for the ESR simulation}
\label{esrtab}
\end{center}
\end{table}

The cooled $^{20}$Ne$^{10+}$ beam with the parameters listed in
Tab.~\ref{esrtab} corresponds to $2.6 \times 10^9$ stored particles.
With the values of $\beta_f = 0.131$~s$^{-1}$ for the `dynamical friction
coefficient' we obtain $\tau_{\mathrm{ef}} = 7.0$~s as $e$-folding time
for the total beam emittance $\varepsilon_{\mathrm{tot}}(t)$,
i.e.\ $1.2 \times 10^7$ turns.
Note that due to the bending magnets,
the average beam envelopes in the $x$- and $y$-directions
no longer agree, as is the case for matched beams
in symmetric quadrupole or solenoid channels.
This leads to different emittance growth
rates for the transverse directions (Fig.~\ref{sim-esr}):
$$
\frac{\varepsilon_x(S)}{\varepsilon_x(0)} - 1
\approx 5.6\times 10^{-8} \; , \quad
\frac{\varepsilon_y(S)}{\varepsilon_y(0)} - 1
\approx 4.3\times 10^{-8} \; , \quad
\frac{\< (\Delta p / p)^2 \> (S)}{\< (\Delta p / p)^2 \> (0)} - 1
\approx 3.0\times 10^{-7} \;.
$$
There is one conclusion we can draw at this point.
According to Eq.~(\ref{tau3}), the $e$-folding frequency of the
total beam emittance is proportional to
the `ellipticities' $I_{xy}$, $I_{xz}$, and $I_{yz}$, which
are positive numbers that characterize the specific ring optics.
They increases the more the matched beam lacks
$x$-$y$ symmetry along the focusing period.
Designing the ring optics to yield largely smooth
envelope functions thus minimizes the intra-beam
scattering induced emittance growth rate.
\subsection{Computer Simulations of Charged Particle Beams}
% Figure environment removed
The Fokker-Planck approach to investigate the behavior
of charged particle beams improves our capability not only
to analyze effects observed in reality, but also to interpret
the results of computer simulations of such beams.

The joint effect of all simplifications necessary to keep
the computing time finite can be visualized
in the way that an additional `error field' is added to
the `true field' we would get in the ideal case.
In other words, the calculated self-field $\vec{E}_c$
can be interpreted as the sum of the true self-field
$\vec{E}_t$ generated by the real beam and a
statistically fluctuating `error field' $\vec{E}_f$:
$$
\vec{E}_c = \vec{E}_t + \vec{E}_f \;.
$$
If we describe the action of the fluctuating `error field'
$\vec{E}_f$  in the same way as above using the Fokker-Planck approach,
we end up with the same moment equations (\ref{deri2}) for the beam.
The amplitude of the fluctuating error field is certainly
much larger than field fluctuations due to the
discreteness of the beam charges.
We therefore use implicitly
a much larger `simulation friction coefficient' $\beta_f^{\, \mathrm{sim}}$,
if we simulate a charged particle beam on the basis
of `representative particles' and a `particle-in-cell'
method to determine the self-fields.
This leads to a much larger gradient in the emittance growth
calculated in our simulations than we would expect in reality.
If we simulate the beam transformation through the quadrupole
channel described in subsection~\ref{quadchannel}, we observe
a continuous growth of the transverse beam emittance,
as plotted in Fig.~\ref{sim-old}.
In view of these simulation results,
the growth rate per cell amounts to:
$$
\frac{\varepsilon_\perp(S)}{\varepsilon_\perp(0)} - 1
\approx 4\times 10^{-5} \;.
$$
We thus find for the ratio of the real beam $\beta_f$
to the `simulation friction coefficient' $\beta_f^{\, \mathrm{sim}}$
for this specific simulation example:
$$
\frac{\beta_f}{\beta_f^{\, \mathrm{sim}}} \approx
1.5 \times 10^{-6} \, / \, 4 \times 10^{-5} \approx 0.04 \;.
$$
$\beta_f^{\, \mathrm{sim}}$ depends in a rather complicated
way on the number of simulation particles, the step width
of the time integration, and the mesh size.
For that reason, it seems to be impossible to derive
an analytical expression for this ratio.

Qualitatively, the results of our simulations
exhibit the effect anticipated by the Fokker-Planck approach:
no emittance growth occurs if we transform
matched beams through solenoid channels,
since $I_e \equiv 0$ in these cases.
In contrast, we observe a steady, non-saturating growth if we
transform the equivalent beam through a periodic quadrupole channel.
Regarding Eq.~(\ref{e-fold}), this is the expected result:
we cannot avoid emittance growth due to fluctuating self-fields,
if the `ellipticity' (\ref{envint}) does not vanish.
\section{CONCLUSIONS}
The Vlasov equation serves as the equation of motion
for the phase space distribution function in the case that
Liouville's theorem applies.

This is obviously not true for all processes that may come
to effect during the lifetime of a charged particle beam.
If we additionally want to investigate non-Liouvillean effects,
we must give up the assumption that the total phase space
volume occupied by the beam remains constant.
The Fokker-Planck equation can be used to describe the
phase space dilution process in case that the underlying
physical effect can be regarded as a Markov process.

Applying Sacherer's\cite{sacherer} formalism to derive
`moment equations' to the Fokker-Planck rather than the
Vlasov equation, hence to include non-Liouvillean effects
in our `moment' analysis, we obtain additional terms that describe
\begin{itemize}
\item the process of temperature balancing within the beam,
\item the effect of `damping' of mismatch oscillations.
\end{itemize}
This rather general approach has been applied to the effect
of intra-beam scattering.
In case that the diffusion as well as the friction processes
within the beam can be regarded as approximately isotropic,
the scattering effects are related to a single coefficient $\beta_f$,
which can directly be obtained from the beam parameters.
It is referred to as the `dynamical friction coefficient'
of the Fokker-Planck equation.
Explicitly, $\beta_f$ is proportional to the particle density
and the fourth power of the charge state.

For {\em unbunched\/} beams propagating through {\em linear\/}
structures, we derived a simple formula that provides us with a
characteristic $e$-folding time $\tau_{\mathrm{ef}}$ for the
transverse beam emittance $\varepsilon_\perp$:
$$
\tau_{\mathrm{ef}} = 4 \, \bigl[ \, \beta_f \, I_e \bigr] ^{-1} \;.
$$
The dimensionless quantity $I_e$ represents
the `ellipticity' of the matched beam along the focusing period,
hence is a feature of the specific lattice only.

The effect of emittance growth due to the relaxation of
temperature differences has important consequences
for systems, where a temperature imbalance
is restored periodically by the focusing forces.
Namely, in systems containing quadrupoles,
a temperature equilibrium can never be reached.
In other words, strong focusing devices are always
associated with a positive `ellipticity' $I_e>0$,
leading to a continuous, non-saturating growth of the
transverse beam emittance.

Our formalism is easily generalized to three dimensions,
which is necessary to study collision effects within
{\em bunched\/} beams and {\em coasting\/} beams in storage rings.
Intra-beam scattering acts to relax the
temperature imbalances within the beam.
This process is always accompanied
by an increase of the total beam emittance.
In principle, intra-beam scattering would not cause any emittance
growth as long as the temperatures within the beam are balanced,
i.e.\ if the beam is in the state of thermodynamical equilibrium.
On the other hand, real focusing lattices do not enable
the beam to take on an equilibrium temperature due to the
non-continuous external forces acting on the particles.
Depending on the specific beam optics, the scattering
effect thus always `generates' a positive growth rate
of the total beam emittance.
For the $e$-folding time $\tau_{\mathrm{ef}}$ of the total
beam emittance $\varepsilon_{\mathrm{tot}}$, we found:
$$
\tau_{\mathrm{ef}} = 9 \, \bigl[ \, \beta_f \,
(I_{xy} + I_{xz} + I_{yz}) \bigr] ^{-1} \; ,
$$
wherein $I_{xy}$, $I_{xz}$, and $I_{yz}$ denote the three
possible `ellipticities', which are related to the course of the
beam temperature ratios along the focusing period
in the $x,y$-, $x,z$-, and $y,z$-planes, respectively.

If we want to reduce the scattering induced emittance
growth rate in a storage ring, we must design the beam
optics in a way that the `ellipticities' of the
envelopes along the ring circumference are minimized,
which means that the matched envelope functions
should be as smooth as possible.
\acknowledge
I wish to thank I.~Hofmann for frequent valuable
discussions during the course of this work.
I am also grateful to K.~Johnson who -- during
his leave from Los Alamos at GSI -- read the
manuscript and offered helpful criticism.
\begin{thebibliography}{99}
\bibitem{kapvla}{I.M.~Kapchinskij and V.V.~Vladimirskij, Proc.\ Int.\ Conf.\
     on High Energy Accelerators, CERN, Geneva, 247 (1959)}
\bibitem{bohn}{C.L.~Bohn,
     {\em Phys.~Rev.~Lett. } {\bf 70}, 932 (1993)}
\bibitem{lapostolle}{P.M.~Lapostolle, CERN report
     CERN-ISR/DI-70-36, (1970). P.M.~Lapostolle, {\em IEEE Trans.~Nucl.~Sci. }
 {\bf NS-18}, 1055 (1971)}
\bibitem{sacherer}{F.J.~Sacherer,
     {\em IEEE Trans.~Nucl.~Sci. } {\bf NS-18}, 1105 (1971)}
\bibitem{jansen}{G.H.~Jansen, {\em Coulomb Interaction in Particle Beams },
     Academic Press, New York (1990)}
\bibitem{risken}{H.~Risken, {\em The Fokker-Planck Equation },
     Springer-Verlag, Berlin, Heidelberg, New York (1989)}
\bibitem{soerensen}{A.H.~S\o{}rensen,
     {\em Introduction to Intrabeam Scattering },
     CERN Report 87-10 (ed.  S.~Turner), 135-151, CERN, Geneva (1987)}
\bibitem{lawson}{J.D.~Lawson,
     {\em The Physics of Charged-Particle Beams },
     Oxford University Press (1988)}
\bibitem{wang}{T.P.~Wangler, K.R.~Crandall, R.S.~Mills, and M.~Reiser,
     {\em IEEE Trans.~Nucl.~Sci. } {\bf NS-32}, 2196 (1985)}
\bibitem{host}{I.~Hofmann and J.~Struckmeier,
     {\em Particle Accelerators } {\bf 21}, 69-98 (1987)}
\bibitem{stho}{J.~Struckmeier and I.~Hofmann,
     {\em Particle Accelerators } {\bf 39}, 219-249 (1992)}
\bibitem{strekla}{J.~Struckmeier, J.~Klabunde, and M.~Reiser,
     {\em Particle Accelerators } {\bf 15}, 47-65 (1984)}
\bibitem{anderson}{O.A.~Anderson,
     {\em Particle Accelerators } {\bf 21}, 197-226 (1987)}
\bibitem{einstein}{A.~Einstein, {\em Ann.\ Physik } {\bf 17}, 549 (1905) and
     {\bf 19}, 371 (1906)}
\bibitem{reiser}{M.~Reiser,
     {\em Theory and Design of Charged Particle Beams }, Wiley \& Sons,
     to be published (1993)}
\bibitem{erlangen}{M.~Seurer, C.~Toepffer, and P.-G.~Reinhard,
     {\em Workshop on Beam Cooling and Related Topics},
     Montreux (1993)}
\bibitem{lalaglu}{J.D.~Lawson, P.M.~Lapostolle, and R.L.~Gluckstern,
     {\em Particle Accelerators } {\bf 5}, 61-65 (1973)}
\bibitem{piwinski}{A.~Piwinski, Proc.\ 9th Int.\ Conf.\
     on High Energy Accelerators, SLAC, Stanford, 405 (1974)}
\bibitem{franzke}{B.~Franzke,
     {\em Nuclear Instruments and Methods } {\bf B24/25}, 18 (1987)}
\bibitem{franzetal}{B.~Franzke et al.,
     Proc.\ Part.\ Accel.\ Conf., Washington~D.C.\ (1993)}
\end{thebibliography}
\end{document}
\typeout{get arXiv to do 4 passes: Label(s) may have changed. Rerun}
