\documentclass[preprint,eqsecnum,aps,nofootinbib]{revtex4}
%\documentclass[preprint,aps,nofootinbib]{revtex4}
\usepackage{amsfonts,amsmath,amssymb,amsthm}
\usepackage{latexsym}
\usepackage{bbm,bm}
\usepackage{graphicx}
%\usepackage[dvips]{color}
\usepackage{tikz} 
\usepackage{multirow}

%%%%%%%%%%%%%%%%%%%%%%%%%%%%%%%%%%%%%%%%%%%%%%%%%%
\newcommand{\uv}{{\bm u}}
\newcommand{\rv}{{\bm r}}
\newcommand{\ra}{\rangle}
\newcommand{\la}{\langle}
\newcommand{\vp}{\varphi}
\newcommand{\tr}{\mathrm{tr}}
\renewcommand{\o}{\otimes}
\newcommand{\ket}[1]{\lvert #1 \rangle}
\newcommand{\bra}[1]{\langle #1 \lvert}
\newcommand{\beq}{\begin{equation}}
\newcommand{\eeq}{\end{equation}}
\newcommand{\beqs}{\begin{eqnarray}}
\newcommand{\eeqs}{\end{eqnarray}}
%%%%%%%%%%%%%%%%%%%%%%%%%%%%%%%%%%%%%%%%%%%%%%%%%%%%%

\begin{document}


\title{Euclidean time method in Generalized Eigenvalue Equation }



\author{Mi-Ra Hwang$^1$, Eylee Jung$^1$, MuSeong Kim$^2$ and DaeKil Park$^{1,3}$\footnote{corresponding author, dkpark@kyungnam.ac.kr} }

\affiliation{$^1$Department of Electronic Engineering, Kyungnam University, Changwon,
                 631-701, Korea    \\
                $^2$Pharos iBio Co., Ltd. 
                Head Office: \#1408, 38, Heungan-daero 427beon-gil, Dongan-gu, Anyang, 14059, Korea \\
                $^3$Department of Physics, Kyungnam University, Changwon,
                  631-701, Korea }



\begin{abstract}
We develop the Euclidean time method of the variational quantum eigensolver for solving the generalized eigenvalue equation $A \ket{\phi_n} = \lambda_n B \ket{\phi_n}$, 
where $A$ and $B$ are hermitian operators, and $\ket{\phi_n}$ and $\lambda_n$ are called the eigenvector and the corresponding eigenvalue of this equation respectively.
For the purpose we modify the usual Euclidean time formalism, which was developed for solving the time-independent Schr\"{o}dinger equation.
We apply our formalism to three numerical examples for test. It is shown that our formalism works very well in all numerical examples. 
We also apply our formalism to the hydrogen atom and compute the electric polarizability. It turns out that our result is slightly less than that of the perturbation method. 
%The future applications to the atomic physics problems are briefly discussed. 

\end{abstract}

\maketitle

\section{Introduction}
After Feynman's suggestion on quantum computer\cite{feynman82,feynman86} few decades ago, the hardwares and the algorithms are rapidly developed recently. 
The quantum computers with few hundred qubits were constructed in several companies such as IBM and Google. Also, some quantum algorithms have been presented 
such as factoring\cite{text,shor94}, database searching\cite{text,grover96}, and matrix inversion\cite{hhl09}. However, it seems to be far away to construct the large-scale, fault-tolerant 
universal quantum computer. 

In spite of this fact, the current-stage quantum computers have their own merits when the classical computers are simultaneously used. 
In this reason, the hybrid quantum-classical algorithms play important role recently on 
noisy intermediate-scale quantum (NISQ) era. The representatives of the hybrid algorithm are the quantum approximate optimization algorithm (QAOA)\cite{farhi14}
and variational quantum eigensolver (VQE)\cite{peruzzo13}. QAOA has been used to find approximate solutions of classical Ising models\cite{ising17} and 
clustering problems formulated as MaxCut\cite{max_cut}.
VQE was first applied in Ref.\cite{peruzzo13} to compute the ground state molecular energy for helium hydride ion $HeH^+$.
The hybrid quantum-classical algorithms have been used in finding the energy spectra\cite{malley15,higgott18,endo18,vogt20}, simulating the Schr\"{o}dinger equations\cite{ying16,mahdian20,endo18-2},
and quantum machine learning\cite{benedetti19,wang20,keren18}. They were also applied to black hole physics\cite{bh21}, high-energy physics\cite{hep22}, and cosmology\cite{cosmos22}.

VQE is a variational algorithm designed to find the ground state of a system governed by a Hamiltonian $H$. Let $\ket{\phi}$ be an initial state that is easy to prepare. By applying a unitary 
operator $U({\bm \theta})$ we prepare the parameter-dependent quantum state:
\begin{equation}
\label{initial-1}
\ket{\psi ({\bm \theta})} = U({\bm \theta}) \ket{\phi}.
\end{equation}
Then, the expectation value of the Hamiltonian can be written as 
\begin{equation}
\label{expectation-1}
E({\bm \theta}) = \bra{\psi ({\bm \theta})} H \ket {\psi ({\bm \theta})}.
\end{equation} 
If $U({\bm \theta})$ is selected appropriately, the ground state energy $E_0$ can be computed by minimizing $E({\bm \theta})$:
\begin{equation}
\label{variation-1}
E_0 = E({\bm \theta}_{min}) = \min_{\bm \theta} E({\bm \theta}).
\end{equation}
Furthermore, the ground state $\ket{\psi_0}$ can be derived as $\ket{\psi_0} = \ket{\psi ({\bm \theta}_{min})}$. This is a whole story of the variational method in quantum mechanics. 
In VQE the quantum computer computes expectation value in Eq. (\ref{expectation-1}) while the minimization of $E({\bm \theta})$ is carried out in the classical computer. 

In many papers the classical computer uses the classical optimizers such as Nelder-Mead for the minimization of  $E({\bm \theta})$. However, the classical optimizers can 
yield an incorrect answer if $E({\bm \theta})$ has local minima. Even though there are several methods\cite{lo-minima} to escape the local minima problem, we think that the most physically 
appealing method is a Euclidean time method introduced in Ref.\cite{euclidean-1}, because the Euclidean time $\tau = i t$ is frequently used in the path-integral quantum mechanics\cite{feynman,kleinert}.
For example, let us consider the one-dimensional simple harmonic oscillator system. Then, the Euclidean propagator is 
\begin{equation}
\label{sho-1}
G[x_b, x_a: \tau] = \sqrt{\frac{m \omega}{2 \pi \hbar \sinh \omega \tau}} \exp\left[ - \frac{m \omega}{2 \hbar \sinh \omega \tau} 
\left\{ (x_a^2 + x_b^2) \cosh \omega \tau - 2 x_a x_b \right\} \right].
\end{equation}
If we take $\tau \rightarrow \infty$ limit, the propagator becomes 
\begin{equation}
\label{sho-2}
\lim_{\tau \rightarrow \infty} G[x_b, x_a: \tau] \sim \phi_0^* (x_b) \phi_0 (x_a) e^{-\frac{i}{\hbar} E_0 \tau}
\end{equation}
where 
\begin{equation}
\label{sho-3}
\phi_0 (x) = \left(\frac{m \omega}{\pi \hbar} \right)^{1/4} e^{-\frac{m \omega}{2 \hbar} x^2}  \hspace{1.0cm} E_0 = \frac{1}{2} \hbar \omega.
\end{equation}
These are exact eigenfunction and eigenvalue for the ground state of the system. Thus, the VQE with the Euclidean time\cite{euclidean-1} naturally yields ground state energy and the corresponding eigenvector at  large $\tau$ limit. 
This technique was used to discover Hamiltonian spectra\cite{spectra} and  is extended to the mixed state scenario\cite{mixed}.

In this paper we want to apply the Euclidean time method of VQE to the generalized eigenvalue equation(GEE)
\begin{equation}
\label{gevalue-1}
A \ket{\phi_n} = \lambda_n B \ket{\phi_n}  \hspace{1.0cm} (n = 0, 1, \cdots)
\end{equation}
where $A$ and $B$ are hermitian operators, and $\lambda_n$ is a generalized eigenvalue. 
The GEE was used in Ref.\cite{gee1969} to compute the electric polarizability in the hydrogen atom. 
GEE problems also arise in the quantum chemistry\cite{fordgee} and fluid mechanics\cite{fluidgee}.
In order to solve Eq. (\ref{gevalue-1}) in quantum computer the slightly variant of the quantum phase estimation (QPE) was suggested in Ref.\cite{qpegee}.
However, the QPE technique generally requires long coherence time and hence, is not suitable for the NISQ devices. 
In order to overcome the difficulty, the authors in Ref.\cite{lslf22} used the quantum gradient descent algorithm and solve the numerical example.

The paper is organized as follows. In section II we present a formalism, which shows how to apply the Euclidean time method to the generalized eigenvalue problem (GEP).
In section III we solve the numerical example of the GEP when $B$ is regular operator. In section IV we consider another numerical problem when $B$ is singular operator\footnote{ If $B$ is regular, the GEE (\ref{gevalue-1}) can be converted into the usual eigenvalue equation 
$\left(B^{-1} A \right)  \ket{\phi_n} = \lambda_n  \ket{\phi_n}$ by incorporating the matrix inversion algorithm \cite{hhl09} in principle. If, however, $B$ is singular, such a conversion is impossible because $B^{-1}$ does not exist.}. It is shown that 
the Euclidean time technique introduced in this paper works very well when $B$ is regular or singular. In section V we introduce another numerical example, where $A$ and $B$ are $8 \times 8$ matrices. It turns out that the eigenvalues converges very slowly with respect to the Euclidean time $\tau$ 
compared to the previous numerical examples.
In section VI we review Ref. \cite{gee1969}, where the electric polarizability of the hydrogen atom is calculated perturbatively  by applying the GEE. In section VII we explore the same atomic physics issue by applying the Euclidean time method. It turns out that the result of this section 
is slightly less than that of the perturbation method. In section VIII a brief conclusion is given. In appendix A we summarize the calculation of section VII as a Table II. 


%we discuss the application of GEP to the calculation of the polarizability for hydrogen atom by using the perturbation theory. In appendix A we briefly describe how to explore the atomic physics issue introduced in section VI in quantum computer.


\section{Formalism}
Let us consider the GEE of Eq. (\ref{gevalue-1}). 
Due to the matrix $B$ the orthogonality of the normalized eigenvectors is expressed as
\begin{equation}
\label{gevalue-2}
\bra{\phi_m} B \ket{\phi_n} = \delta_{mn}.
\end{equation}
In the following we will call the condition (\ref{gevalue-2}) by $B$-orthogonality.

 %are not orthogonal to each other, but they are $B$-orthogonal. It is convenient to normalize $\ket{\phi_n}$ as 

We start with a generalized Euclidean time-dependent Schr\"{o}dinger-like equation
\begin{equation}
\label{imaginary-1}
\frac{\partial}{\partial \tau} \ket{\psi(\tau)} = - (A - \lambda B)  \ket{\psi(\tau)}
\end{equation}
where $\tau = i t$ is an Euclidean time. If $\ket{\psi(\tau)}$ is an eigenvector of Eq. (\ref{gevalue-1}), the eigenvalue $\lambda$ in Eq. (\ref{imaginary-1})
can be written as 
\begin{equation}
\label{imaginary-2}
\lambda \rightarrow F(\tau) = \frac{\bra{\psi(\tau)} A \ket{\psi(\tau)}} {\bra{\psi(\tau)} B \ket{\psi(\tau)}}.
\end{equation}
Thus, the Euclidean time evolution of $\ket{\psi(\tau)}$ is governed by 
\begin{equation}
\label{imaginary-3}
\frac{\partial}{\partial \tau} \ket{\psi(\tau)}= - (A - F(\tau) B) \ket{\psi(\tau)}.
\end{equation}
As usual Euclidean quantum mechanics, $\ket{\psi(\tau)}$ should approach to the ground state of Eq. (\ref{gevalue-1}) in the $\tau \rightarrow \infty$ limit.
Thus, we want to solve Eq. (\ref{imaginary-3}) by applying the hybrid quantum-classical algorithm.

In order to solve Eq. (\ref{imaginary-3}) numerically, we assume $\ket{\psi(\tau)}$ as 
\begin{eqnarray}
\label{vqa-1}
&& \hspace{1.0cm} \ket{\psi(\tau)} = V ({\bm \theta}) \ket{\bar{0}}    \\    \nonumber
&& V ({\bm \theta}) = U_N (\theta_N) \cdots U_k (\theta_k) \cdots U_1 (\theta_1)
\end{eqnarray}
where $U_k$ is an unitary operator and $\theta_k$ is dependent on $\tau$. Using 
\begin{equation}
\label{vqa-2}
\frac{\partial}{\partial \tau} \ket{\psi(\tau)} = \sum_{i=1}^N \dot{\theta}_i \frac{\partial}{\partial \theta_i} \ket{\psi(\tau)} ,
\end{equation}
one can show directly 
\begin{eqnarray}
\label{vqa-3}
&& \hspace{3.0cm}  \left| \left( \frac{\partial}{\partial \tau} + A - F B \right) \ket{\psi(\tau)} \right|^2               \\   \nonumber
&& = \sum_{i,j=1}^N \dot{\theta}_i \dot{\theta}_j \left( \frac{\partial}{\partial \theta_j} \bra{\psi(\tau)} \right) \left( \frac{\partial}{\partial \theta_i} \ket{\psi(\tau)} \right) 
+ \bra{\psi(\tau)} (A - F B)^2 \ket{\psi(\tau)}                                                                                \\   \nonumber
&&\hspace{.5cm}+  \sum_{i=1}^N \dot{\theta}_i \bra{\psi(\tau)} (A - F B) \left( \frac{\partial}{\partial \theta_i} \ket{\psi(\tau)} \right) 
+ \sum_{i=1}^N \dot{\theta}_i \left( \frac{\partial}{\partial \theta_i} \bra{\psi(\tau)} \right) (A - F B)  \ket{\psi(\tau)}.
\end{eqnarray}
Applying the McLachlan's variational principle
\begin{equation}
\label{vqa-4}
\frac{\partial}{\partial \dot{\theta}_j} \left| \left( \frac{\partial}{\partial \tau} + A - F B \right) \ket{\psi(\tau)} \right| = 0,
\end{equation}
one can derive the first-order coupled differential equation of the parameter
\begin{equation}
\label{vqa-5}
\sum_{j=1}^N \Gamma_{ij} \dot{\theta}_j = C_i
\end{equation}
where 
\begin{eqnarray}
\label{vqa-6}
&&\Gamma_{ij} = \mbox{Re} \left[ \left( \frac{\partial}{\partial \theta_i} \bra{\psi(\tau)} \right) \left( \frac{\partial}{\partial \theta_j} \ket{\psi(\tau)} \right)  \right]   \\    \nonumber
&& C_i = - \mbox{Re} \left[ \left( \frac{\partial}{\partial \theta_i} \bra{\psi(\tau)} \right) (A - F B)  \ket{\psi(\tau)} \right].
\end{eqnarray}

The hybrid quantum-classical algorithm we adopt in this paper is as following. We solve the differential equation (\ref{vqa-5}) in the classical computer by making use of the Euler method
\begin{equation}
\label{euler-1}
{\bm \theta} (\tau + \delta \tau) \approx {\bm \theta} (\tau) + \Gamma^{-1} (\tau) {\bm C} (\tau) \delta \tau.
\end{equation}
The coefficients $\Gamma_{ij} (\tau)$ and $C_i (\tau)$ as well as $F(\tau)$ in Eq. (\ref{imaginary-2}) will be computed via the suitable quantum algorithms. 

%%%%%%%%%%%%%%%%%%%%%%%%%%%%%%%%%%%%%%%%%%%%%%%%%%%%%%%%%
% Figure environment removed
%%%%%%%%%%%%%%%%%%%%%%%%%%%%%%%%%%%%%%%%%%%%%%%%%%%%%%%%%%%

First, let us briefly comment how to compute $F(\tau)$. We assume that the matrices $A$ and $B$ are $2^m \times 2^m$ hermitian. Then, $A$ and $B$ can be decomposed into the linear combination 
of $m$-tensor product of the Pauli matrices:
\begin{eqnarray}
\label{pauli}
\sigma_0 = I_2 = \left(  \begin{array}{cc} 1 & 0  \\  0 & 1  \end{array}   \right)            \hspace{.5cm}
\sigma_1 = \left(  \begin{array}{cc} 0 & 1  \\  1 & 0  \end{array}   \right)            \hspace{.5cm}
\sigma_2 = \left(  \begin{array}{cc} 0 & -i  \\  i & 0  \end{array}   \right)            \hspace{.5cm}
\sigma_3 = \left(  \begin{array}{cc} 1 & 0  \\  0 & -1  \end{array}   \right).           
\end{eqnarray}
Since Pauli matrices are unitary as well as hermitian, the expectation value of each term, say $U_i$, can be computed by applying Fig. 1. 
In this way it is possible to compute $F(\tau)$ by applying the circuit of Fig. 1 repeatedly.

%%%%%%%%%%%%%%%%%%%%%%%%%%%%%%%%%%%%%%%%%%%%%%%%%%%%%%%%%
% Figure environment removed
%%%%%%%%%%%%%%%%%%%%%%%%%%%%%%%%%%%%%%%%%%%%%%%%%%%%%%%%%%%

Now, let us explain how to compute $\Gamma_{ij}$ and $C_i$ with quantum circuits. Let us express the derivative of $U_i (\theta_i)$ in a form:
\begin{equation}
\label{vqa-7}
\frac{\partial U_i (\theta_i)}{\partial \theta_i} = \sum_{k=1}^N f_{k,i} U_i (\theta_i) \sigma_{k,i},
\end{equation}
where $f_{k,i}$ is a complex number and $\sigma_{k,i}$ is unitary operator. Then, one can show easily 
\begin{equation}
\label{vqa-8}
\frac{\partial}{\partial \theta_i} \ket{\psi(\tau)} = \sum_k f_{k,i} \widetilde{V}_{k,i} \ket{\bar{0}}
\end{equation}
where 
\begin{equation}
\label{vqa-9}
\widetilde{V}_{k,i} = U_N \cdots U_i \sigma_{k,i} U_{i-1} \cdots U_1.
\end{equation}
Inserting Eq. (\ref{vqa-8}) into Eq. (\ref{vqa-6}), one can show 
\begin{eqnarray}
\label{vqa-10}
&&  \Gamma_{ij} = \mbox{Re} \left[ \sum_{k,\ell = 1}^N f_{k,i}^* f_{\ell,j} \bra{\bar{0}} \widetilde{V}_{k,i}^{\dagger} \widetilde{V}_{\ell,j} \ket{\bar{0}} \right]             \\   \nonumber
&& C_i = - \mbox{Re} \left[ \sum_{k,\alpha} f_{k,i}^* \Lambda_{\alpha} \bra{\bar{0}} \widetilde{V}_{k,i}^{\dagger} h_{\alpha} V \ket{\bar{0}} \right]
\end{eqnarray}
where we used a decomposition 
\begin{equation}
\label{vqa-11}
A - F B = \sum_{\alpha} \Lambda_{\alpha} h_{\alpha}.
\end{equation}
All the terms of the summations in $\Gamma_{ij}$ and $C_i$ are proportional to the general terms $\mbox{Re} \left[ e^{i \theta} \bra{\bar{0}} \widetilde{V}_{k,i}^{\dagger} \widetilde{V}_{\ell,j} \ket{\bar{0}} \right]$ and 
$\mbox{Re} \left[ e^{\i \theta} \bra{\bar{0}} \widetilde{V}_{k,i}^{\dagger} h_{\alpha} V \ket{\bar{0}} \right]$ respectively. These quantities can be computed by applying the quantum circuits of Fig. 2. 
In this way, it is possible to compute $\Gamma_{ij}$ and $C_i$ by applying the circuits of Fig. 2 repeatedly. 

After obtaining the ground state $\ket{g} = \ket{\phi_0}$ and corresponding eigenvalue $\lambda_0$, one can compute the first excited state $\ket{\phi_1}$ by changing $A$ as 
\begin{equation}
\label{vqa-12}
A \rightarrow A' = A + \mu B \ket{g} \bra{g} B
\end{equation}
where $\ket{g}$ is normalized as $\bra{g} B \ket{g} = 1$. The parameter $\mu$ is chosen as $\mu > \lambda_1 - \lambda_0$. Since we do not know $\lambda_1$, we should choose $\mu$ sufficiently large. 
Repeating this procedure one can compute the full spectrum of the GEE (\ref{gevalue-1}). 

\section{Numerical Example I: Case for regular $B$}

%%%%%%%%%%%%%%%%%%%%%%%%%%%%%%%%%%%%%%%%%%%%%%%%%%%%%%%%%
% Figure environment removed
%%%%%%%%%%%%%%%%%%%%%%%%%%%%%%%%%%%%%%%%%%%%%%%%%%%%%%%%%%%

In this section we apply the Euclidean time method introduced in the previous section to Eq. (\ref{gevalue-1}), where 
\begin{eqnarray}
\label{toy-1}
&&A = I_2 \otimes I_2 + 0.4 Z \otimes I_2 + 0.4 I_2 \otimes Z + 0.2 X \otimes X               \\    \nonumber
&&B = I_2 \otimes I_2 + 0.3 Z \otimes I_2 + 0.4 I_2 \otimes Z + 0.2 Z \otimes Z.
\end{eqnarray}
In this case $A - F B = \sum_{\alpha = 0}^4 \Lambda_{\alpha} h_{\alpha}$, where 
\begin{eqnarray}
\label{toy-2}
&&\Lambda_0 = 1 - F  \hspace{.4cm} \Lambda_1 = 0.4 - 0.3 F  \hspace{.4cm}  \Lambda_2 = 0.4 (1 - F)  \hspace{.4cm}  \Lambda_3 = 0.2  \hspace{.4cm}  \Lambda = -0.2 F   \\   \nonumber
&& h_0 = I_2 \otimes I_2  \hspace{.4cm} h_1 = Z \otimes I_2   \hspace{.4cm}  h_2 = I_2 \otimes Z  \hspace{.4cm}  h_3 = X \otimes X  \hspace{.4cm}  h_4 = Z \otimes Z.
\end{eqnarray}
We choose the state $\ket{\psi (\tau)}$ as four-parameter state shown in Fig. 3(a). Then, it is straightforward to construct the quantum circuits for $\Gamma_{ij}$ and $C_i$. 
For example, the quantum circuit for $\Gamma_{24}$ is plotted in Fig. 3(b).

%%%%%%%%%%%%%%%%%%%%%%%%%%%%%%%%%%%%%%%%%%%%%%%%%%%%%%%%%
% Figure environment removed
%%%%%%%%%%%%%%%%%%%%%%%%%%%%%%%%%%%%%%%%%%%%%%%%%%%%%%%%%%%

The Euclidean time evolution of the parameters $\theta_i$ and  $\lambda_0$ are plotted in Fig. 4(a) and Fig. 4(b) respectively. The lowest eigenvalue $\lambda_0$ approaches to $0.333262$ when $\tau$ 
approaches to $6$. In this limit the parameters approach 
\begin{equation}
\label{toy-3}
\theta_1 = 0.92987  \hspace{1.0cm}  \theta_2 = 1.99389  \hspace{1.0cm} \theta_3 = 2.19508  \hspace{1.0cm}  \theta_4 = 1.38469.
\end{equation}
The corresponding eigenstate is 
\begin{equation}
\label{toy4}
\ket{\phi_0} = 0.228442 \ket{00} + 0.044591 \ket{01} - 0.032439 \ket{10} - 1.340530 \ket{11}
\end{equation}
where Eq. (\ref{gevalue-2}) is used for normalization.This is very close to the exact eigenstate
\begin{equation}
\label{toy5}
\ket{\psi_0} = 0.229362 \ket{00} - 1.34168 \ket{11}.
\end{equation}
In order to examine how much $\ket{\phi_0}$ is close to $\ket{\psi_0}$, one can compute the fidelity, which results in  $|\bra{\phi_0} \psi_0 \rangle|^2 = 0.998358$, where the usual normalization is used. 


%%%%%%%%%%%%%%%%%%%%%%%%%%%%%%%%%%%%%%%%%%%%%%%%%%%%%%%%%
% Figure environment removed
%%%%%%%%%%%%%%%%%%%%%%%%%%%%%%%%%%%%%%%%%%%%%%%%%%%%%%%%%%%

\begin{center}
\begin{tabular}{c|c|c|c|c} \hline \hline
               &  $\lambda_0$   &   $\lambda_1$   &     $\lambda_2$   &   $\lambda_3$                                \\    \hline
imaginary time method  &  \hspace{.1cm}    $0.33326$ \hspace{.1cm}  &  \hspace{.1cm}  $0.97205$ \hspace{.1cm}  & \hspace{.1cm}  $1.02106$ \hspace{.1cm}  & \hspace{.1cm}  $1.56964$                                      \\  \hline 
exact values  &  $0.33162$  &   $0.97204$  & $1.01575$  &  $1.56765$   \\  \hline  \hline
\end{tabular}

\vspace{0.2cm}
Table I:Comparison of result of  the Euclidean time method with exact values.
\end{center}

In order to compute the first-excited eigenvalue we should change the matrix $A$ as $A' = A + \mu B \ket{\phi_0} \bra{\phi_0} B$, where 
\begin{eqnarray}
\label{toy6}
&& B \ket{\phi_0}\bra{\phi_0} B = 0.1599 I_2 \otimes I_2 -0.1459 X \otimes X + 0.1450 Y \otimes Y + 0.1590 Z \otimes Z     \\    \nonumber
&& \hspace{2.4cm} -0.0652 I_2 \otimes Z - 0.0652 Z \otimes I_2 + 0.0166 I_2 \otimes X - 0.0168 X \otimes I_2                                                 \\    \nonumber
&&\hspace{2.4cm}+ 0.0041 X \otimes Z - 0.0030 Z \otimes X.
\end{eqnarray}
Then, we should modify the quantum circuits for $F(\tau)$ and $C_i (\tau)$ to include Eq. (\ref{toy6}). In Fig. 5a the Euclidean time evolution of $\lambda_1$ is plotted, where 
$\delta \tau = 0.1$ and $\mu = 10$ are chosen. This figure shows that $\lambda_1$ approaches to $0.97205$ when $\tau$ approaches to $80$. The second-excited eigenvalue can be 
computed by changing $A$ as $A'' = A + \mu_1 B \ket{\phi_0} \bra{\phi_0} B + \mu_2 B \ket{\phi_1} \bra{\phi_1} B$. The Euclidean time evolution of $\lambda_2$ is plotted in 
Fig. 5b, where $\delta \tau = 0.01$ and $\mu_1 = \mu_2 = 10$ are chosen. The eigenvalue $\lambda_2$ approaches to $1.02106$ when $\tau$ approaches to $2.0$. 
Similarly, the Euclidean time evolution of $\lambda_3$ is plotted in Fig. 5c. The eigenvalue $\lambda_3$ approaches to $1.56964$ at the large $\tau$. The eigenvalues computed by the Euclidean 
time method are compared with the exact values in Table I. Table I shows that the eigenvalues computed by the Euclidean time method coincides with the exact values within $99.5 \%$. 

The satisfactory accuracy of our results is mainly due to the fact that we use the qiskit (version $0.36.2$) in classical computer. If, however, we use the real quantum computer, the discrepancy between numerical and exact results would be increased due to 
the noise effect. 
%Since the system is small for our case, this discrepancy is very small. However, if the system is large, noise can make the large discrepancy.
For this case we have to use the noise mitigation process appropriately. 
Few years ago, a quantum algorithm was proposed to exactly and efficiently discuss the effect of noise on the system\cite{wang2018} in the photosynthetic energy transfer.



  
\section{Numerical Example II: Case for singular $B$}

%%%%%%%%%%%%%%%%%%%%%%%%%%%%%%%%%%%%%%%%%%%%%%%%%%%%%%%%%
% Figure environment removed
%%%%%%%%%%%%%%%%%%%%%%%%%%%%%%%%%%%%%%%%%%%%%%%%%%%%%%%%%%%



In order to confirm that our formalism also can be applied for the singular $B$,  we consider another numerical example when $B$ is singular operator in this section. 
In order to explore this issue, let us choose $A$ and $B$ in the form:
\begin{eqnarray}
\label{singular-1}
A = \left(    \begin{array}{cccc}
              a_1   &  0  &  0  &  b            \\
              0  &  a_2  &  b  &  0             \\
              0  &  b  &  a_3  &  0             \\
              b  &  0  &  0  &  a_4             
                 \end{array}                     \right)    \hspace{1.0cm}
B = \left(    \begin{array}{cccc}
              1  &  1  &  1  &  1              \\
              1  &  1  &  1  &  1              \\
              1  &  1  &  1  &  1              \\
              1  &  1  &  1  &  1               
                  \end{array}               \right).
\end{eqnarray}
In this case the eigenvalue and corresponding eigenvector can be computed analytically. 
Unlike the usual eigenvalue equation one can show that this system generates single eigenvalue in the form
\begin{equation}
\label{s_eigenvalue}
\lambda = \frac{a_1 a_2 a_3 a_4 - (a_1 a_4 + a_2 a_3) b^2 + b^4}{Q}
\end{equation}
where
\begin{equation}
\label{s_eigenvalue-2}
Q = a_1 a_2 a_3 + a_1 a_2 a_4 + a_1 a_3 a_4 + a_2 a_3 a_4 - 2 (a_1 a_4 + a_2 a_3) b - (a_1 + a_2 + a_3 + a_4) b^2 + 4 b^3.
\end{equation}
The corresponding eigenvector can be written as 
\begin{eqnarray}
\label{s_eigenvector}
&&\ket{\Phi} = \frac{1}{Q} \Bigg[ (a_4 - b) (a_2 a_3 - b^2) \ket{00} + (a_3 - b) (a_1 a_4 - b^2) \ket{01}                    \\   \nonumber
&&    \hspace{3.0cm}                    + (a_2 - b) (a_1 a_4 - b^2) \ket{10} + (a_1 - b) (a_2 a_3 - b^2) \ket{11}   \Bigg].
\end{eqnarray} 

As a numerical example we choose 
\begin{eqnarray}
\label{s_toy-1}
&&A = I_2 \otimes I_2 + 0.4 Z \otimes I_2 + 0.4 I_2 \otimes Z + 0.2 X \otimes X               \\    \nonumber
&&B = I_2 \otimes I_2 +  I_2 \otimes X + X \otimes I_2 + X \otimes X.
\end{eqnarray}
Then, Eqs. (\ref{s_eigenvalue}) and (\ref{s_eigenvector}) gives
\begin{equation}
\label{s_exact}
\lambda_{exact} = 0.15  \hspace{1.0cm} \ket{\Phi}_{exact} = 0.125 \ket{01} + 0.125 \ket{10} + 0.75 \ket{11}.
\end{equation}

The Euclidean time evolution of the eigenvalue $\lambda$ is plotted in Fig. 6. As expected it approaches to $0.150005$ as increasing $\tau$.
Using the final values of $\theta_j$, one can derive the corresponding eigenvector, which  is $\ket{\Phi} = 0.127 \ket{01} + 0.124 \ket{10} + 0.749 \ket{11}$. It approximately coincides with $\ket{\Phi}_{exact}$.

\section{Numerical Example III: for $8 \times 8$ matrices of $A$ and $B$ }
%%%%%%%%%%%%%%%%%%%%%%%%%%%%%%%%%%%%%%%%%%%%%%%%%%%%%%%%%
% Figure environment removed
%%%%%%%%%%%%%%%%%%%%%%%%%%%%%%%%%%%%%%%%%%%%%%%%%%%%%%%%%%%
In this section we apply the Euclidean time method when $A$ and $B$ are $8 \times 8$ matrices as follows:
\begin{eqnarray}
\label{toy-1}
&&A = I_2 \otimes I_2 \otimes I_2 + 0.4 Z \otimes I_2 \otimes X + 0.4 I_2 \otimes Z \otimes X               + 0.2 X \otimes X  \otimes I_2             \\    \nonumber
&&B = I_2 \otimes I_2 \otimes I_2 + 0.3 Z \otimes I_2 \otimes Z + 0.4 I_2 \otimes Z \otimes X + 0.2 Z \otimes Z \otimes X.
\end{eqnarray}
We choose the state $\ket{\psi} (\tau)$ as six-parameter state shown in Fig. 7(a). The lowest eigenvalue $\lambda_0$ approaches to 0.2126 when $\tau$ approaches to $30$, which is shown in Fig. 7(b). 
In order to compute the first-excited eigenvalue $\lambda_1$ we change $A$ as $A'=A + \mu B \ket{\phi_0} \bra{\phi_0} B$, where $\ket{\phi_0}$ is B-orthogonal ground state given by 
$B \ket{\phi_0} \bra{\phi_0} B = \sum_{ijk=0}^3 p_{ijk} \sigma_i \otimes \sigma_j \otimes \sigma_k$ with the nonzero coefficients are 
\begin{eqnarray}
\label{new-1}
&&p_{000} = p_{330} = 0.121   \hspace{.5cm}  p_{001} = p_{033} = p_{303} = p_{331} = 0.085 \hspace{.5cm} p_{003} = p_{333}= -0.080   \\   \nonumber
&&p_{030} = p_{300} = -0.107  \hspace{.5cm} p_{031} = p_{301} = -0.072  \hspace{.5cm} p_{110} = - p_{220} = -0.048                              \\   \nonumber
&&p_{111} = - p_{221} = -0.057  \hspace{.5cm}  p_{113} = - p_{223} = 0.012  \hspace{.5cm} p_{112} = p_{212} = 0.033.
\end{eqnarray}
The coefficient $\mu$ is chosen as $5.0$. Fig. 7(c) shows that $\lambda_1$ approaches 0.3988 when $\tau$ approaches to $100$. In Fig. 7(b) and (c) the red dashed lines correspond to the exact value, which are $ 0.212465$ and $0.394698$, respectively. 
One can compute the higher eigenvalues by similar way. Since this is only tedious repetition, we skip the procedure in this paper. 





\section{application to hydrogen atom: Perturbation method}

In this section we examine how to compute the electric polarizability ${\cal P}$ of the hydrogen atom by applying the generalized eigenvalue equation (\ref{gevalue-1}).
If the external electric field is very small, it can be derived by perturbation method, which was studied in Ref. \cite{gee1969}. 
In the following we will review Ref. \cite{gee1969} and in next section same problem is analyzed by applying the Euclidean time method.

Let us consider the Schr\"{o}dinger equation for the hydrogen-like atom with atomic number $Z$.
If we set the energy eigenvalue as $E = - \alpha^2/2$, the Schr\"{o}dinger equation can be converted into the GEE (\ref{gevalue-1}), where
\begin{equation}
\label{hydro-1}
A = \frac{1}{r}, \hspace{1.0cm} B = -\frac{1}{2} {\bm \triangledown}^2 + \frac{1}{2} \alpha^2, \hspace{1.0cm} \lambda_n = \frac{1}{Z} = \frac{1}{n \alpha}.
\end{equation}
In this case the eigenvector $\ket{n, \ell, m}$ should be $B$-normalized, i.e. $\bra{n_1, \ell_1, m_1} B \ket{n_2m \ell_2, m_2} = \delta_{n_1, n_2} \delta_{\ell_1, \ell_2} \delta_{m_1,m_2}$.
Then, it is straightforward to show 
\begin{equation}
\label{hydro-wavef-1}
 \psi_{n,\ell, m} =  \sqrt{\frac{4 \alpha \Gamma(n - \ell)}{n \Gamma(n + \ell + 1)}} e^{-\alpha r} (2 \alpha r)^{\ell} L_{n - \ell - 1}^{2 \ell + 1} (2 \alpha r) Y_{\ell, m} (\theta, \phi)
\end{equation}
where $Y$ and $L$ refer to spherical harmonics and generalized Laguerre polynomials. It is worthwhile noting that the normalization constant is different from the case of usual normalization constant by a factor $\alpha$.

If we apply the external electric field ${\cal E}$ along the $z$-direction, the operator $A$ is changed into 
\begin{equation}
\label{hydro-2}
A = \frac{1}{r} + \frac{\cal E}{Z}  r \cos \theta.
\end{equation}
Since the generalized eigenvalue $\lambda_n$ has only discrete spectrum, one can apply the perturbation more easily than usual perturbation because Hamiltonian has in general both discrete and continuum spectra. 
For the case of the ground state ($n=1$), straight calculation shows $\lambda_1$ in a form:
\begin{equation}
\label{hydro-3}
\lambda_1 = \frac{1}{Z} = \frac{1}{\alpha} + \frac{9}{4 Z^2 \alpha^5} {\cal E}^2 + {\cal O} \left( {\cal E}^3 \right).
\end{equation}
Solving Eq. (\ref{hydro-3}) we can conjecture $\alpha \sim Z \left(1 + \frac{9}{4} \frac{{\cal E}^2}{Z^6}  \right)$, which results in the ground state energy $E_1$ as
\begin{equation}
\label{hydro-g}
E_1 = - \frac{1}{2} \alpha^2 \sim -\frac{Z^2}{2} - \frac{9}{4} \frac{{\cal E}^2}{Z^4}.
\end{equation}
It is interesting to note that the field-dependent term in $E_1$ is proportional to $1/Z^4$.
Thus, the electric polarizability for the hydrogen atom is 
\begin{equation}
\label{hydro-5}
{\cal P} = \frac{1}{\cal E} \frac{d}{d {\cal E}} \left(\frac{9}{4} \frac{{\cal E}^2}{Z^4} \right) \Bigg{|}_{Z=1} = \frac{9}{2}
\end{equation}
in atom units. This was derived by making use of usual perturbation method in Ref.\cite{schiff}.

In order to explore this issue in quantum computer, we need to convert $A$ and $B$ as matrix forms by using mappings to qubit. If this is possible, one can compute the electric polarizability without relying on the perturbation theory. 
However, we do not know how to derive the Jordan-Wigner\cite{jw1,jw2,jw3} or Bravyi-Kitaev\cite{bk1,bk2}
matrix forms of $A$ and $B$. 
In spite of this fact, one can apply the Euclidean time method to the same atomic physics issue by introducing the proper basis in Hilbert space. This is discussed in next section.






%We hope to apply the generalized eigenvalue problem to various physical problems including this issue in the near future. 
%In Appendix A we describe briefly how to explore this issue numerically in quantum computer without using the Jordan-Wigner or  Bravyi-Kitaev matrix forms.

\section{application to hydrogen atom: Numerical Method}

%%%%%%%%%%%%%%%%%%%%%%%%%%%%%%%%%%%%%%%%%%%%%%%%%%%%%%%%%
% Figure environment removed
%%%%%%%%%%%%%%%%%%%%%%%%%%%%%%%%%%%%%%%%%%%%%%%%%%%%%%%%%%%

In this section we would like to apply the Euclidean time method to the atomic physics issue introduced in section VI  without relying on  the Jordan-Wigner or  Bravyi-Kitaev mapping. Instead of the particular mapping to qubit, we will use the matrix representation of $A$ and $B$ 
by introducing the proper basis in Hilbert space. 
Here, let us use the simple nodeless Slater-type orbital (STO) basis\cite{hjo02}
\begin{equation}
\label{sto_basis}
\ket{n, \ell, m} = R_n (r) Y_{\ell,m} (\theta, \phi)
\end{equation}
where
\begin{equation}
\label{sto-1}
R_n (r) = \frac{(2 \xi)^{3/2}}{\sqrt{\Gamma (2 n + 1)}} (2 \xi r)^{n-1} e^{-\xi r}.
\end{equation}
The problem of the STO basis is the fact that it is not completely orthogonal with respect to the principal quantum number as follows:
\begin{equation}
\label{sto-2}
\bra{n', \ell', m'} n, \ell, m \rangle = \frac{\Gamma(n + n' + 1)}{\sqrt{\Gamma(2 n' + 1) \Gamma (2 n + 1)}} \delta_{\ell,\ell'} \delta_{m,m'}.
\end{equation}
Another problem is that this basis involves the free parameter $\xi$. Thus, we have to fix $\xi$ appropriately. Then, the following matrix representations can be derived:
\begin{eqnarray}
\label{matrix_rep}
&&\bra{n', \ell', m'} A \ket{n, \ell, m}   = \frac{\Gamma(n + n')}{\sqrt{\Gamma(2 n' + 1) \Gamma (2 n + 1)}}                     \\      \nonumber
&& \times \Bigg[ 2 \xi \delta_{\ell,\ell'} \delta_{m,m'} + \frac{{\cal E}}{Z} \frac{ (n + n' + 1) (n + n')}{2 \xi} \Bigg\{\sqrt{\frac{(\ell - m + 1) (\ell + m + 1)}{(2 \ell + 1) (2 \ell + 3)}} \delta_{\ell',\ell+1} \delta_{m,m'}                   \\     \nonumber
&& \hspace{8.0cm} + \sqrt{\frac{(\ell - m ) (\ell + m )}{(2 \ell -1) (2 \ell + 1)}} \delta_{\ell',\ell-1} \delta_{m,m'} \Bigg\} \Bigg]                                                \\    \nonumber
&&\bra{n', \ell', m'} B \ket{n, \ell, m}                 \\    \nonumber
&&= \frac{\Gamma(n + n'-1)}{2 \sqrt{\Gamma(2 n' + 1) \Gamma (2 n + 1)}} \Bigg[ \xi^2 \left\{ 4 \ell (\ell + 1) + (n + n') - (n - n')^2 \right\}                                    \\    \nonumber
&&\hspace{8.0cm} + \alpha^2 (n + n') (n + n'-1) \Bigg]  \delta_{\ell,\ell'} \delta_{m,m'}.
\end{eqnarray}
In principle, the matrix representations of $A$ and $B$ are $\infty \times \infty$ dimensional. For the numerical calculation, therefore, we need to truncate them. For example, if we truncate $n, n' \geq 3$, we have $5 \times 5$ matrices of $A$ and $B$ as follows:
\begin{eqnarray}
\label{truncate-1}
&&A = \left(     \begin{array}{ccccc}
x \alpha  &  \frac{x \alpha}{\sqrt{3}}  & 0  &  \frac{{\cal E}}{x Z \alpha}  &    0       \\
\frac{x \alpha}{\sqrt{3}}  &   \frac{x \alpha}{2}  &  0  &  \frac{5 {\cal E}}{2 \sqrt{3} x Z \alpha}  &  0    \\
0  &  0  &  \frac{x \alpha}{2}  &  0  &  0                                                                        \\
 \frac{{\cal E}}{x Z \alpha}  &  \frac{5 {\cal E}}{2 \sqrt{3} x Z \alpha}  &  0  &  \frac{x \alpha}{2}  &  0   \\  
 0  &  0  &  0  &  0  &  \frac{x \alpha}{2}
                  \end{array}      \right)                                                                                                 \\   \nonumber
&&B = \left(      \begin{array}{ccccc}
\frac{(1 + x^2) \alpha^2}{2}  &  \frac{(3 + x^2) \alpha^2}{4 \sqrt{3}}  &  0  &  0  &  0        \\
\frac{(3 + x^2) \alpha^2}{4 \sqrt{3}}  &  \frac{(3 + x^2) \alpha^2}{6}  &  0  &  0  &  0        \\
0  &  0  &  \frac{(1 + x^2) \alpha^2}{2}  &  0  &  0                                                              \\
0  &  0  &  0  &  \frac{(1 + x^2) \alpha^2}{2}  &  0                                                              \\
0  &  0  &  0  &  0  &  \frac{(1 + x^2) \alpha^2}{2}
                  \end{array}    \right)
\end{eqnarray}
where $x$ is defined as $\xi = x \alpha$.
Since the qubit system only needs $2^n \times 2^n$ matrix, we change the $5 \times 5$ matrices into $\widetilde{A} = \left( \begin{array}{cc} A  &  0  \\   0  & I_3  \end{array} \right)$ and $\widetilde{B} = \left( \begin{array}{cc} B  &  0  \\   0  & I_3  \end{array} \right)$, where 
$I_3$ is a $3 \times 3$ identity matrix.  

Now, $\widetilde{A}$ and $\widetilde{B}$ are $8 \times 8$ matrices with free parameters $x$, $\alpha$, $Z$, and ${\cal E}$. 
With aid of Mathematica 13.1 one can show that when ${\cal E} \ll 0$,  the lowest eigenvalue $\lambda_1$ of the GEE (\ref{gevalue-1}) with  $\widetilde{A}$ and $\widetilde{B}$ is similar to Eq. (\ref{hydro-3}) in a form 
\begin{equation}
\label{sto-3}
\lambda_1 = \frac{1}{Z} = g_1(x) \frac{1}{\alpha} + g_2(x) \frac{{\cal E}^2}{Z^2 \alpha^5} +  {\cal O} \left( {\cal E}^3 \right)
\end{equation}
where $g_1$ and $g_2$ depend only on $x$. Then, the electric polarizability becomes ${\cal P} (x) = \frac{2 g_2(x)}{g_1(x)^3}$.

 In the following  we will compute $g_1(x)$ and $g_2(x)$ by applying the Euclidean method as follows. We fix ${\cal E} = 0.01$ and $Z=1$ for simplicity. 
 Given $x = x_*$ we compute $\lambda_1$ by the Euclidean time method for two different $\alpha$. Solving two coupled equations of Eq. (\ref{sto-3}) one can compute $g_1(x_*)$
 and $g_2(x_*)$.  For example, Fig. 8(a) and (b) correspond to the Euclidean time evolution of $\lambda_1$ when $(x, \alpha) = (0.7, -1)$ and $(x, \alpha) = (0.7, -2)$  respectively. 
 Fig. 8(c) and (d) correspond to $x=0.8$ with same values of $\alpha$. We use the same initial state of section V given in Fig. 7(a). Our numerical result is summarized in Appendix A as a Table II.
 From Table II $2 g_2 (x) / g_1(x)^3$ is maximized at $x_* = 0.9$ and at this point we have ${\cal P} (x_*) = 4.2665$. This is slightly less than the perturbation result $4.5$. 
 Of course, different truncation yields different matrix representations of $A$ and $B$. If we truncate $n, n' \geq 101$, the dimension of $\widetilde{A}$ and $\widetilde{B}$ becomes $2^{19} \times 2^{19}$. 
In this case we need  at least $20$-qubit  quantum computer for the computation of the electric polarizability using the Euclidean time method. 

%Of course, different truncation yields different matrix representations of $A$ and $B$. If we truncate $n, n' \geq 101$, the dimension of $\widetilde{A}$ and $\widetilde{B}$ becomes $2^{19} \times 2^{19}$. With such a huge dimension the Mathematica seems to be useless. 
%In this case we need  at least $20$-qubit  quantum computer for the computation of the electric polarizability using the Euclidean time method. 



%With varying the free parameters the quantum computer can compute $\lambda_1(\xi, \alpha, Z, {\cal E})$ by applying the 
%Euclidean time method. Since the dimension of  $\widetilde{A}$ and $\widetilde{B}$ is not huge, we will continue on our discussion with using Mathematica $13.1$ instead of quantum computer. If $\xi = x \alpha$, one can show that the expression of $\lambda_1$ is similar to Eq. (\ref{hydro-3}) as 
%\begin{equation}
%\label{sto-3}
%\lambda_1 = \frac{1}{Z} = g_1(x) \frac{1}{\alpha} + g_2(x) \frac{{\cal E}^2}{Z^2 \alpha^5} +  {\cal O} \left( {\cal E}^3 \right)
%\end{equation}
%where $g_1 (x)$ and $g_2(x)$ are complicated functions of $x$.  One can show that ${\cal P} (x)$ is maximized at $x = x_* = 0.8685$, and at this point ${\cal P} (x_*) = 4.2843$.
%This is slightly less than the perturbation result $4.5$. 


\section{Conclusion}

In this paper we apply the Euclidean time method of VQE to the GEE (\ref{gevalue-1}). For the purpose of this we slightly modified the usual  imaginary time method of VQE presented in Ref.\cite{euclidean-1}.
We applied our formalism to the three numerical examples. It is shown that 
the Euclidean time technique introduced in this paper works very well for all example. 
Finally, we apply our method to the hydrogen atom system and compute the electric polarizability when the external electric field is ${\cal E} = 0.01$. 
It turns out that the polarizability is $4.2665$, which is slightly less than the perturbation result $4.5$.

There are lot a issues we need to address. How to compute the electric polarizability of the hydrogen atom when ${\cal E}$ is large? 
In this case the perturbation method is useless. How to extend our method to other atoms such as helium or lithium?
It is of interest to apply our formalism to the real physical,  chemical, and fluid problems. 


%\section{application to hydrogen atom}

%\section{application to helium atom}





{\bf Acknowledgement}:
This work was supported by the National Research Foundation of Korea(NRF) grant funded by the Korea government(MSIT) (No. 2021R1A2C1094580).

\vspace{1.0cm}


{\bf Data Availability}:
 The datasets generated during and/or analyzed during the current study are available from the corresponding author on reasonable request.
 
 \vspace{1.0cm}
 
 
{\bf Conflict of Interest}: 
The authors declare that they have no known competing financial interests or personal relationships that could have appeared to influence the work reported in this paper.


















\begin{thebibliography}{99}

\bibitem{feynman82} R. P. Feynman, {\it Simulating Physics with Computers},
Int. J. Theor. Phys. {\bf 21} (1982) 467. 
\bibitem{feynman86} R. P. Feynman, {\it Quantum Mechanical Computers},
Found. Phys. {\bf 16} (1986) 507. 
\bibitem{text} M. A. Nielsen and I. L. Chuang, Quantum Computation and Quantum Information (Cambridge
University Press, Cambridge, England, 2000).
\bibitem{shor94} P. W. Shor, {\it Algorithms for Quantum Computation: Discrete
Logarithms and Factoring}, Proc. 35th Annual Symposium on Foundations of 
Computer Science (1994) 124.
\bibitem{grover96} L. K. Grover, {\it A fast quantum mechanical algorithm for
database search}, Proc. 28th Annual ACM Symposium on the Theory of 
Computing (1996) 212 [quant-ph/9605043].
\bibitem{hhl09} A. W. Harrow, A. Hassidim, and S. Lloyd, {\it Quantum algorithm for solving linear systems of equations}, Phys. Rev. Lett. {\bf 15} (2009) 150502 [arXiv:0811.3171 (quant-ph)].
\bibitem{farhi14} E. Farhi, J. Goldstone, and S. Gutmann, {\it A Quantum Approximate Optimization Algorithm}, arXiv:1411.4028 (quant-ph).
\bibitem{peruzzo13} A. Peruzzo, J. McClean, P. Shadbolt, M-H. Yung, X-Q. Zhou, P. J. Love, A. Aspuru-Guzik, and J. L. O'Brien, {\it A variational eigenvalue solver on a quantum processor}, Nat. Commun. {\bf 5} (2014) 1 [arXiv:1304.3061 (quant-ph)].
\bibitem{ising17}  N. Moll et al, {\it Quantum optimization using variational algorithms on near-term quantum devices}, Quantum Sci. Technol. {\bf 3} (2018) 030503 [arXiv:1710.01022 (quant-ph)].
\bibitem{max_cut}  J. S. Otterbach, {\it et al}, {\it Unsupervised Machine Learning on a Hybrid Quantum Computer}, arXiv:1712.05771 (quant-ph).
\bibitem{malley15} P. J. J. O'Malley, {\it et al}, {\it Scalable Quantum Simulation of Molecular Energies}, Phys. Rev. {\bf X 6} (2016) 031007 [arXiv:1512.06860 (quant-ph)].
\bibitem{higgott18} O. Higgott, D. Wang, and S. Brierley, {\it Variational Quantum Computation of Excited States}, Quantum {\bf 3} (2019) 156 [arXiv:1805.08138 (quant-ph)].
\bibitem{endo18} S. Endo, T. Jones, S. McArdle, X. Yuan, and S. Benjamin, {\it Variational quantum algorithms for discovering Hamiltonian spectra},  Phys. Rev. {\bf A 99} (2019) 062304 [arXiv:1806.05707 (quant-ph)].
\bibitem{vogt20}  N. Vogt, S. Zanker, J-M. Reiner, T. Eckl, A. Marusczyk, and M. Marthaler, {\it Preparing symmetry broken ground states with variational quantum algorithms}, Quantum Sci. Technol. {\bf 6} (2021) 035003 [arXiv:2007.01582 (quant-ph)].
\bibitem{ying16} Y. Li and S. C. Benjamin, {\it Efficient variational quantum simulator incorporating active error minimisation}, Phys. Rev. {\bf X 7} (2017) 021050 [arXiv:1611.09301 (quant-ph)].
\bibitem{mahdian20} M. Mahdian and H. D. Yeganeh, {\it Incoherent quantum algorithm dynamics of an open system with near-term devices}, Quant. Inf. Process. {\bf 19} (2020) 285 [arXiv:2008.05344 (quant-ph)].
\bibitem{endo18-2} S. Endo, J. Sun, Y. Li, S. Benjamin, and X. Yuan, {\it Variational quantum simulation of general processes}, Phys. Rev. Lett. {\bf 125} (2020) 010501 [arXiv:1812.08778 (quant-ph)].
\bibitem{benedetti19} M. Benedetti, E. Lloyd, S. Sack, and M. Fiorentini, {\it Parameterized quantum circuits as machine learning models}, Quantum Sci. Technol. {\bf 4} (2019) 043001 [arXiv:1906.07682 (quant-ph)].
\bibitem{wang20} X. Wang, Z. Song, and Y. Wang, {\it Variational Quantum Singular Value Decomposition}, Quantum {\bf 5} (2021) 483 [arXiv:2006.02336 (quant-ph)].
\bibitem{keren18} K. Li, S. Wei, F. Zhang, P. Gao, Z. Zhou, T. Xin, X. Wang, and G. Long, {\it Optimizing a Polynomial Function on a Quantum Simulator}, npj Quantum Inf. {\bf 7} (2021) 16 [arXiv:1804.05231 (quant-ph)].
\bibitem{bh21} E. Rinaldi, X. Han, M. Hassan, Y. Feng, F. Nori, M. McGuigan, and M. Hanada, {\it Matrix-Model Simulations Using Quantum Computing, Deep Learning, and Lattice Monte Carlo}, PRX Quantum {\bf 3} (2022)  010324 [arXiv:2108.02942 (quant-ph)].
\bibitem{hep22} C. W. Bauer, {\it et al}, {\it Quantum Simulation for High Energy Physics}, arXiv:2204.03381 (quant-ph).
\bibitem{cosmos22} A. Joseph, T. White, V. Chandra, and M. McGuigan, {\it Quantum Computing of Schwarzschild-de Sitter Black Holes and Kantowski-Sachs Cosmology}, arXiv:2202.09906 (quant-ph).
\bibitem{lo-minima} D. Wierichs, C. Gogolin, and M. Kastoryano, {\it Avoiding local minima in variational quantum eigensolvers with the natural gradient optimizer}, Phys. Rev. Research {\bf 2} (2020) 043246 [arXiv:2004.14666 (quant-ph)].
\bibitem{euclidean-1}S. McArdle, T. Jones, S. Endo, Y. Li, S. Benjamin, and X. Yuan, {\it Variational ansatz-based quantum simulation of imaginary time evolution}, npj Quantum Infor. {\bf 5} (2019) 75 [arXiv:1804.03023 (quant-ph)].
\bibitem{feynman} R. P. Feynman and A. R. Hibbs, Quantum Mechanics and Path Integrals (McGraw-Hill, 1965, New York).
\bibitem{kleinert} H. Kleinert, Path integrals in Quantum Mechanics, Statistics, and Polymer Physics (World Scientific,1995,  Singapore).
\bibitem{spectra}  S. Endo, T. Jones, S. McArdle, X. Yuan, and S Benjamin, {\it Variational quantum algorithms for discovering Hamiltonian spectra}, Phys. Rev. {\bf A 99} (2019) 062304 [arXiv:1806.05707  (quant-ph)].
\bibitem{mixed} X. Yuan, S. Endo, Q. Zhao, Y. Li, and S. Benjamin, {\it Theory of variational quantum simulation}, Quantum {\bf 3} (2019) 191 [arXiv:1812.08767 (quant-ph)].
\bibitem{gee1969} A. T. Amos, C. Lauhhlin, and G. R. Moody, {A generalized eigenvalue equation for the hydrogen atom}, Chem. Phys. Lett. {\bf 3} (1969) 411.
\bibitem{fordgee} B. Ford and G. Hall, {\it The generalized eigenvalue problem in quantum chemistry}, Comput. Phys. Commun. {\bf 8} (1974) 337.
\bibitem{fluidgee} K. A. Cliffe, A. Spence, and S. J. Tavener, {\it The numerical analysis of bifurcation problems with application to fluid mechanics}, Acta Numer. {\bf 9} (2000) 39.
\bibitem{qpegee} J. B. Parker and I. Joseph, {\it Quantum phase estimation for a class of generalized eigenvalue problems}, Phys. Rev. {\bf A 102} (2020) 022422 [arXiv:2002.08497 (quant-ph)].
\bibitem{lslf22}  J-M. Liang, S-Q. Shen, M. Li, and S.-M. Fei, {\it Quantum algorithms for the generalized eigenvalue problem}, Quantum Inf. Process, {\bf  21} (2022) 23 [arXiv:2112.02554 (quant-ph)].
\bibitem{wang2018} B-X. Wang, M-J. Tao, Q. Ai, T. Xin, N. Lambert, D. Ruan, Y.-C. Cheng, F. Nori, F-G. Deng, and G.-L. Long, {\it Efficient quantum simulation of photosynthetic light harvesting}, npj Quantum Inf. {\bf 4} (2018) 52 [arXiv:1801.09475 (quant-ph)].


\bibitem{schiff} L. I. Schiff, {\it Quantum Mechanics} (McGraw-Hill, Tokyo, 1968).

\bibitem{jw1} P. Jordan and E.Wigner, {\it \"{U}ber das Paulische \"{A}quivalenzverbot}, Z. Phys. {\bf 47} (1928) 631.
\bibitem{jw2} R. Somma, G. Ortiz, J. E. Gubernatis, E. Knill, and R. Laflamme, {\it Simulating physical phenomena by quantum networks}, Phys. Rev. {\bf A 65} (2002) 042323 (2002) [quant-ph/0108146].
\bibitem{jw3} M. Nielsen, {The Fermionic canonical commutation relations and the Jordan-Wigner transform}, unpublished (2005).

\bibitem{bk1} S. Bravyi and A. Kitaev, {\it Fermionic quantum computation}, Ann. Phys. {\bf 298} (2002) 210 [quant-ph/0003137].
\bibitem{bk2} J. T. Seeley, M. J. Richard, and P. J. Love, {\it The Bravyi-Kitaev transformation for quantum computation of electronic structure}, J. Chem. Phys. {\bf 137} (2012) 224109 [arXiv:1208.5986 (quant-ph)].

\bibitem{hjo02} T. Helgaker, P. Jorgensen, and J. Olsen, {\it Molecular Electronic-Structure Theory} (Wiley, Chichester, 2002).






%\bibitem{schrodinger-35} E. Schr\"{o}dinger, {\it Die gegenw\"{a}rtige Situation in der Quantenmechanik}, Naturwissenschaften, 
%{\bf 23} (1935) 807.
%\bibitem{horodecki09} R. Horodecki, P. Horodecki, M. Horodecki, and K. Horodecki, {\it Quantum Entanglement}, Rev. Mod. Phys. 
%{\bf 81} (2009) 865 [quant-ph/0702225] and references therein.
%\bibitem{teleportation} C. H. Bennett, G. Brassard, C. Crepeau, R. Jozsa, A. Peres and W. K. Wootters, {\it Teleporting
%an Unknown Quantum State via Dual Classical and Einstein-Podolsky-Rosen Channles}, Phys.Rev. Lett. {\bf 70} (1993) 1895.
%\bibitem{Luo2019}Y. H. Luo et al., {\it Quantum Teleportation in High Dimensions}, Phys. Rev. Lett. {\bf 123} (2019) 070505 [arXiv:1906.09697 (quant-ph)].
%\bibitem{superdense} C. H. Bennett and S. J. Wiesner, {\it Communication via one- and two-particle operators on
%Einstein-Podolsky-Rosen states}, Phys. Rev. Lett. {\bf 69} (1992) 2881.
%\bibitem{clon} V. Scarani, S. Lblisdir, N. Gisin and A. Acin, {\it Quantum cloning}, Rev. Mod. Phys. {\bf 77} (2005)
%1225 [quant-ph/0511088] and references therein.
%\bibitem{cryptography} A. K. Ekert , {\it Quantum Cryptography Based on Bell’s Theorem}, Phys. Rev. Lett. {\bf 67} (1991)
%661.
%\bibitem{cryptography2} C. Kollmitzer and M. Pivk, Applied Quantum Cryptography (Springer, Heidelberg, Germany, 2010).
%\bibitem{metro17} K. Wang, X. Wang, X. Zhan, Z. Bian, J. Li, B. C. Sanders, and P. Xue, {\it Entanglement-enhanced quantum metrology in a noisy environment}, Phys. Rev. {\bf A97} (2018) 042112 [arXiv:1707.08790 (quant-ph)].
%\bibitem{qcreview} T. D. Ladd, F. Jelezko, R. Laflamme, Y. Nakamura, C. Monroe, and J. L. O'Brien, 
%{\it Quantum Computers}, Nature, {\bf 464} (2010) 45 [arXiv:1009.2267 (quant-ph)].
%\bibitem{computer} G. Vidal, {\it Efficient classical simulation of slightly entangled quantum computations}, Phys. Rev.
%Lett. {\bf 91} (2003) 147902 [quant-ph/0301063].
%\bibitem{supremacy-1} F. Arute et al.,{Quantum supremacy using a programmable superconducting processor}, Nature {\bf 574} (2019) 505. Its supplementary information is given in arXiv:1910.11333.




%\bibitem{feynman82} R. P. Feynman, {\it Simulating Physics with Computers},
%Int. J. Theor. Phys. {\bf 21} (1982) 467. 
%\bibitem{feynman86} R. P. Feynman, {\it Quantum Mechanical Computers},
%Found. Phys. {\bf 16} (1986) 507. 
%\bibitem{shor94} P. W. Shor, {\it Algorithms for Quantum Computation: Discrete
%Logarithms and Factoring}, Proc. 35th Annual Symposium on Foundations of 
%Computer Science (1994) 124.

%\bibitem{berry05} D. W. Berry, G. Ahokas, R. Cleve, and B. C. Sanders, {\it Efficient quantum algorithms for simulating sparse Hamiltonians},  Comm. Math. Phys. {\bf 270} (2007) 359 [arXiv:quant-ph/0508139 (quant-ph)].
%\bibitem{dense} L. Wossnig, Z. Zhao, and A. Prakash, {\it A quantum linear system algorithm for dense matrices}, Phys. Rev. Lett. {\bf 120} (2018) 050502 [arXiv:1704.06174 (quant-ph)].
%\bibitem{hybrid} Y. Lee, J. Joo, and S. Lee, {\it Hybrid quantum linear equation algorithm and its experimental test on IBM Quantum Experience}, Scientific Reports {\bf 9} (2019) 4778 [arXiv:1807.10651 (quant-ph)].
%\bibitem{exp-hhl-1} J. Pan, Y. Cao, X. Yao, Z. Li, C. Ju, X. Peng, S. Kais, and J. Du, {\it Experimental realization of quantum algorithm for solving linear systems of equations}, Phys. Rev. {\bf A 89} (2014) 022313 [arXiv:1302.1946  (quant-ph)].
%\bibitem{ckw} V. Coffman, J. Kundu and W. K. Wootters, {\it Distributed entanglement},
%Phys. Rev. {\bf A61} (2000) 052306 [quant-ph/9907047].









%\bibitem{grover97} L. K. Grover, {\it Quantum Mechanics helps in searching for 
%a needle in a haystack}, Phys. Rev. Lett. {\bf 79} (1997)
%325 [quant-ph/9706033].
%\bibitem{bennett97} C H. Bennett, E. Bernstein, G. Brassard, and U. Vazirani, {\it Strengths and Weaknesses of Quantum Computing}, SIAM Journal on Computing, {\bf 26} (1997) 1510 [quant-ph/9701001].
%\bibitem{boyer96} M. Boyer, G. Brassard, P. Hoeyer, and A. Tapp, {\it Tight bounds on quantum searching}, Fortschritte der Physik,  {\bf 46} (1998) 493 [quant-ph/9605034].
%\bibitem{grover98} L. K. Grover, {\it How fast can a quantum computer search?}, quant-ph/9809029.
%\bibitem{form2} S. Hill and W. K. Wootters, {\it Entanglement of a Pair of Quantum Bits},
%Phys. Rev. Lett. {\bf 78} (1997) 5022 [quant-ph/9703041].
%\bibitem{form3} W. K. Wootters, {\it Entanglement of Formation of an Arbitrary State
%of Two Qubits}, Phys. Rev. Lett. {\bf 80} (1998) 2245 [quant-ph/9709029].
%\bibitem{zalka99} C. Zalka, {\it Grover's quantum searching algorithm is optimal}, Phys.Rev. {\bf A 60} (1999) 2746 [quant-ph/9711070].

%\bibitem{dur00-1} W. D\"{u}r, G. Vidal, and J. I. Cirac, {\it Three qubits can be
%entangled in two inequivalent ways}, Phys. Rev. {\bf A62} (2000) 062314
%[quant-ph/0005115].
%\bibitem{benn96} C. H. Bennett, D. P. DiVincenzo, J. A. Smokin and W. K. Wootters,
%{\it Mixed-state entanglement and quantum error correction}, Phys. Rev. {\bf A54}
%(1996) 3824 [quant-ph/9604024].
%\bibitem{uhlmann99-1} A. Uhlmann, {\it Fidelity and concurrence of conjugate states},
%Phys. Rev. {\bf A 62} (2000) 032307 [quant-ph/9909060].

%\bibitem{tangle2} R. Lohmayer, A. Osterloh, J. Siewert and A. Uhlmann, {\it Entangled
%Three-Qubit States without Concurrence and Three-Tangle}, Phys. Rev. Lett. {\bf 97}
%(2006) 260502 [quant-ph/0606071].
%\bibitem{convex_hull} A. Osterloh, J. Siewert, and A. Uhlmann, {\it Tangles of superpositions and the convex-roof extension}, Phys. Rev. {\bf A 77} (2008) 032310 [arXiv:0710.5909 (quant-ph)].
%\bibitem{tangle3} C. Eltschka, A. Osterloh, J. Siewert and A. Uhlmann, {\it Three-tangle
%for mixtures of generalized GHZ and generalized W states}, New J. Phys. {\bf 10} (2008)
%$043014, arXiv:0711.4477 (quant-ph).
%\bibitem{tangle4} E. Jung, M. R. Hwang, D. K. Park and J. W. Son, {\it Three-tangle
%for Rank-$3$ Mixed States: Mixture of Greenberger-Horne-Zeilinger, W and flipped W states},
%Phys. Rev. {\bf A79} (2009) 024306, arXiv:0810.5403 (quant-ph).
%\bibitem{tangle5} E. Jung, M. R. Hwang, D. K. Park, and S. Tamaryan, {\it Three-Party 
%Entanglement in Tripartite Teleportation Scheme through Noisy Channels}, Quant. Inf. Comp. {\bf 40} (2010) 0377 [arXiv:0904.2807 (quant-ph)].

%\bibitem{QPE1}  R. Cleve, A. Ekert, C. Macchiavello, and M. Mosca, {\it Quantum Algorithms Revisited}, quant-ph/9708016.
%\bibitem{QPE2} A. Luis and J. Pe\u{r}ina, {\it Optimum phase-shift estimation and the quantum description of the phase difference}, Phys. Rev. {\bf A 54} (1996) 4564.
%\bibitem{QPE3} see also https://qiskit.org/textbook/ch-algorithms/quantum-phase-estimation.html.

%\bibitem{ou07-1} Y. U. Ou and H. Fan, {\it Monogamy Inequality in terms of Negativity for
%Three-Qubit States}, Phys. Rev. {\bf A75} (2007) 062308 [quant-ph/0702127].
%\bibitem{vidal01-1} G. Vidal and R. F. Werner, {\it Computable measure of entanglement},
%Phys. Rev. {\bf A65} (2002) 032314 [quant-ph/0102117].
%\bibitem{peres96} A. Peres, {\it Separability Criterion for Density Matrices}, Phys.
%Rev. Lett. {\bf 77} (1996) 1413 [quant-ph/9604005].
%\bibitem{horod96} M. Horodecki, P. Horodecki and R. Horodecki, {\it Separability of mixed
%states: necessary and sufficient conditions}, Phys. Lett. {\bf A 223} (1996) 1
%[quant-ph/9605038].
%\bibitem{horod97} P. Horodecki, {\it Separability criterion and inseparable mixed states
%with partial transposition}, Phys. Lett. {\bf A 232} (1997) 333 [quant-ph/9703004].
%\bibitem{lloyd08} S. Lloyd, {\it Enhanced Sensitivity of Photodetection via Quantum Illumination}, Science, {\bf 321}, 1463 (2008).
%\bibitem{tan08}S.-H. Tan, B. I. Erkmen, V. Giovannetti, S. Guha, S. Lloyd, L. Maccone, S. Pirandola, and J. H. Shapiro, {\it Quantum Illumination with Gaussian States}, Phys. Rev. Lett. {\bf 101}, 253601 (2008) [arXiv:0810.0534 (quant-ph)].
%\bibitem{eylee-22-1}  E. Jung and  D. K. Park, {\it Quantum Illumination with three-mode Gaussian State}, Quant. Inf. Proc. {\bf 21}  (2022) 71  [arXiv:2107.05203 (quant-ph)].
%\bibitem{discord1} C. Weedbrook, S. Pirandola, J. Thompson, V. Vedral, and M. Gu, {\it How discord underlies the noise resilience of quantum illumination}, New J. Phys. {\bf 18} (2016) 043027.
%\bibitem{discord2}M. Bradshaw, S. M. Assad, J. Y. Haw, S. H. Tan, P. K. Lam, M. Gu, {\it Overarching framework between Gaussian quantum discord and Gaussian quantum illumination}, Phys. Rev. {\bf A 95} (2017) 022333 [arXiv:1611.10020 (quant-ph)].
%\bibitem{QFT1} see also https://qiskit.org/textbook/ch-algorithms/quantum-fourier-transform.html.



%\newpage 

%\begin{appendix}{\centerline{\bf Appendix A: Atomic Physics Issue}}
%\setcounter{equation}{0}
%\renewcommand{\theequation}{A.\arabic{equation}}
%In this appendix we would like to describe briefly how to explore the atomic physics issue introduced in the conclusion section  in quantum computer. For the purpose we should convert $A$ and $B$ as  matrix forms. We will use the matrix representation of $A$ and $B$ by introducing the proper basis in Hilbert space. 
%Here, let us use the simple nodeless Slater-type orbital (STO) basis\cite{hjo02}
%\begin{equation}
%\label{sto_basis}
%\ket{n, \ell, m} = R_n (r) Y_{\ell,m} (\theta, \phi)
%\end{equation}
%where
%\begin{equation}
%\label{sto-1}
%R_n (r) = \frac{(2 \xi)^{3/2}}{\sqrt{\Gamma (2 n + 1)}} (2 \xi r)^{n-1} e^{-\xi r}.
%\end{equation}
%The problem of the STO basis is the fact that it is not completely orthogonal with respect to the principal quantum number as follows:
%\begin{equation}
%\label{sto-2}
%\bra{n', \ell', m'} n, \ell, m \rangle = \frac{\Gamma(n + n' + 1)}{\sqrt{\Gamma(2 n' + 1) \Gamma (2 n + 1)}} \delta_{\ell,\ell'} \delta_{m,m'}.
%\end{equation}
%Another problem is that this basis involves the free parameter $\xi$. Thus, we have to fix $\xi$ appropriately. Then, the following matrix representations can be derived:
%\begin{eqnarray}
%\label{matrix_rep}
%&&\bra{n', \ell', m'} A \ket{n, \ell, m}   = \frac{\Gamma(n + n')}{\sqrt{\Gamma(2 n' + 1) \Gamma (2 n + 1)}}                     \\      \nonumber
%&& \times \Bigg[ 2 \xi \delta_{\ell,\ell'} \delta_{m,m'} + \frac{{\cal E}}{Z} \frac{ (n + n' + 1) (n + n')}{2 \xi} \Bigg\{\sqrt{\frac{(\ell - m + 1) (\ell + m + 1)}{(2 \ell + 1) (2 \ell + 3)}} \delta_{\ell',\ell+1} \delta_{m,m'}                   \\     \nonumber
%&& \hspace{8.0cm} + \sqrt{\frac{(\ell - m ) (\ell + m )}{(2 \ell -1) (2 \ell + 1)}} \delta_{\ell',\ell-1} \delta_{m,m'} \Bigg\} \Bigg]                                                \\    \nonumber
%&&\bra{n', \ell', m'} B \ket{n, \ell, m}                 \\    \nonumber
%&&= \frac{\Gamma(n + n'-1)}{2 \sqrt{\Gamma(2 n' + 1) \Gamma (2 n + 1)}} \Bigg[ \xi^2 \left\{ 4 \ell (\ell + 1) + (n + n') - (n - n')^2 \right\}                                    \\    \nonumber
%&&\hspace{8.0cm} + \alpha^2 (n + n') (n + n'-1) \Bigg]  \delta_{\ell,\ell'} \delta_{m,m'}.
%\end{eqnarray}
%In principle, the matrix representations of $A$ and $B$ are $\infty \times \infty$ dimensional. For the numerical calculation, therefore, we need to truncate them. For example, if we truncate $n, n' \geq 3$, we have $5 \times 5$ matrices of $A$ and $B$. 
%Since the qubit system only needs $2^n \times 2^n$ matrix, we change the $5 \times 5$ matrices into $\widetilde{A} = \left( \begin{array}{cc} A  &  0  \\   0  & I_3  \end{array} \right)$ and $\widetilde{B} = \left( \begin{array}{cc} B  &  0  \\   0  & I_3  \end{array} \right)$, where 
%$I_3$ is a $3 \times 3$ identity matrix.

%Now, $\widetilde{A}$ and $\widetilde{B}$ are $8 \times 8$ matrices with free parameter $\xi$, $\alpha$, $Z$, and ${\cal E}$.  With varying the free parameters the quantum computer can compute $\lambda_1(\xi, \alpha, Z, {\cal E})$ by applying the 
%Euclidean time method. Since the dimension of  $\widetilde{A}$ and $\widetilde{B}$ is not huge, we will continue on our discussion with using Mathematica $13.1$ instead of quantum computer. If $\xi = x \alpha$, one can show that the expression of $\lambda_1$ is similar to Eq. (\ref{hydro-3}) as 
%\begin{equation}
%\label{sto-3}
%\lambda_1 = \frac{1}{Z} = g_1(x) \frac{1}{\alpha} + g_2(x) \frac{{\cal E}^2}{Z^2 \alpha^5} +  {\cal O} \left( {\cal E}^3 \right)
%\end{equation}
%where $g_1 (x)$ and $g_2(x)$ are complicated functions of $x$. Then, the electric polarizability becomes ${\cal P} (x) = \frac{2 g_2(x)}{g_1(x)^3}$. One can show that ${\cal P} (x)$ is maximized at $x = x_* = 0.8685$, and at this point ${\cal P} (x_*) = 4.2843$.
%This is slightly less than the perturbation result $4.5$. 

%Of course, different truncation yields different matrix representations of $A$ and $B$. If we truncate $n, n' \geq 101$, the dimension of $\widetilde{A}$ and $\widetilde{B}$ becomes $2^{19} \times 2^{19}$. With such a huge dimension the Mathematica seems to be useless. 
%In this case we need  at least $20$-qubit  quantum computer for the computation of the electric polarizability using the Euclidean time method. 



%\end{appendix}

\newpage
\begin{appendix}{\centerline{\bf Appendix A: Atomic Physics Issue}}
\setcounter{equation}{0}
\renewcommand{\theequation}{A.\arabic{equation}}
\begin{center}
Table II: $x$-dependence of $g_1(x)$ and $g_2(x)$.
\vspace{0.2cm}
\tabcolsep=0.3cm

\begin{tabular}{|c|c|c|c|c|}\hline

\hspace{0.5cm}\multirow{2}{2em}{$x$} & \hspace{0.5cm}\multirow{2}{3em}{$\alpha$} &\hspace{0.5cm} \multirow{2}{3em}{$\lambda_{1}$} &\hspace{0.5cm} \multirow{1}{4em}{$g_{1}(x)$} & \hspace{0.3cm}\multirow{2}{4em}{\Large{$\frac{2 g_2(x)}{g_1(x)^3}$}}         \\  \cline{4-4}
                                 &                                           &                                                             & \hspace{0.5cm}\multirow{1}{4em}{$g_{2}(x)$} &        \\ \hline \hline
\multirow{2}{4em}{$x=0.5$}  & $\alpha=-1$   & -0.9477  &  $g_{1}(0.5)=0.9476$   & \multirow{2}{3em}{2.2467}   \\ \cline{2-3}
                                         & $\alpha=-2$   & -0.4738  &  $g_{2}(0.5)=0.9558$   &                                             \\ \cline{1-5} 
\multirow{2}{4em}{$x=0.6$}  & $\alpha=-1$   & -0.9812  &  $g_{1}(0.6)=0.9811$   & \multirow{2}{3em}{3.0883}   \\ \cline{2-3}
                                         & $\alpha=-2$   & -0.4905  &  $g_{2}(0.6)=1.4580$   &                                             \\ \cline{1-5} 
\multirow{2}{4em}{$x=0.7$}  & $\alpha=-1$   & -0.9950  &  $g_{1}(0.7)=0.9948$   & \multirow{2}{3em}{3.7747}   \\ \cline{2-3}
                                         & $\alpha=-2$   & -0.4974  &  $g_{2}(0.7)=1.8581$   &                                             \\ \cline{1-5} 
\multirow{2}{4em}{$x=0.8$}  & $\alpha=-1$   & -0.9993  &  $g_{1}(0.8)=0.9991$   &  \multirow{2}{3em}{4.1974}  \\ \cline{2-3}
                                         & $\alpha=-2$   & -0.4996  &  $g_{2}(0.8)=2.0932$   &                                             \\ \cline{1-5} 
\multirow{2}{4em}{$x=0.9$}  & $\alpha=-1$   & -1.0002  &  $g_{1}(0.9)=1.0000$   &  \multirow{2}{3em}{4.2665}  \\ \cline{2-3}
                                         & $\alpha=-2$   & -0.5000  &  $g_{2}(0.9)= 2.1330$  &                                             \\ \cline{1-5}
\multirow{2}{4em}{$x=1.01$}& $\alpha=-1$   & -1.0002  &  $g_{1}(1.01)=1.0000$  &  \multirow{2}{3em}{3.9589} \\ \cline{2-3}
                                         & $\alpha=-2$   & -0.5000  &  $g_{2}(1.01)=1.9795$  &                                            \\ \cline{1-5} 
\multirow{2}{4em}{$x=1.1$}  & $\alpha=-1$   & -1.0002  &  $g_{1}(1.1)=1.0000$   &  \multirow{2}{3em}{3.5150} \\ \cline{2-3}
                                         & $\alpha=-2$   & -0.5000  &  $g_{2}(1.1)=1.7573$   &                                            \\ \hline
\multirow{2}{4em}{$x=1.2$}  & $\alpha=-1$   & -0.9998  &  $g_{1}(1.2)=0.9996$   &  \multirow{2}{3em}{2.9513} \\ \cline{2-3}
                                         & $\alpha=-2$   & -0.4998  &  $g_{2}(1.2)=1.4739$   &                                            \\ \hline                                        
\multirow{2}{4em}{$x=1.3$}  & $\alpha=-1$   & -0.9985  &  $g_{1}(1.3)=0.9984$   &  \multirow{2}{3em}{2.4102} \\ \cline{2-3}
                                         & $\alpha=-2$   & -0.4992  &  $g_{2}(1.3)=1.1992$   &                                            \\ \hline
\multirow{2}{4em}{$x=1.4$}  & $\alpha=-1$   & -0.9959  &  $g_{1}(1.4)=0.9958$   &  \multirow{2}{3em}{1.9412} \\ \cline{2-3}
                                         & $\alpha=-2$   & -0.4979  & $g_{2}(1.4)=0.9585$    &                                            \\ \hline
\multirow{2}{4em}{$x=1.5$}  & $\alpha=-1$   & -0.9918  & $g_{1}(1.5)=0.9917$    &  \multirow{2}{3em}{1.5572} \\ \cline{2-3}
                                         & $\alpha=-2$   & -0.4958  &  $g_{2}(1.5)=0.7593$   &                                            \\ \hline  
                                                                                                                                                         
\end{tabular}

\end{center}



\end{appendix}




\end{thebibliography}










\end{document}





