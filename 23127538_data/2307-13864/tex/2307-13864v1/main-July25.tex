\documentclass{article}
\usepackage{amsmath,amsthm,amssymb,amsfonts}
%\usepackage{nmmacro}
\usepackage{verbatim}
\usepackage{graphicx}
\usepackage{stmaryrd} %for \leftrightarroweq
\usepackage{hyperref}
%\usepackage{showlabels}


\usepackage[colorinlistoftodos]{todonotes}


\theoremstyle{plain}
\newtheorem{thm}{Theorem}
\newtheorem{lem}[thm]{Lemma}
\newtheorem{prop}[thm]{Proposition}
\newtheorem{cor}[thm]{Corollary}
\newtheorem{conj}[thm]{Conjecture}
\newtheorem{remark}[thm]{Remark}

\theoremstyle{definition}
\newtheorem{definition}[thm]{Definition}
\newtheorem{exl}[thm]{Example}

\numberwithin{thm}{section}

\newcommand{\adj}{\leftrightarrow}
\newcommand{\adjeq}{\leftrightarroweq}

\DeclareMathOperator{\id}{id}
\def\Z{{\mathbb Z}}
\def\N{{\mathbb N}}
\def\R{{\mathbb R}}
\def \Fix {Fix}

\title{Articulation Points and Freezing Sets}

\author{Laurence Boxer
\thanks{Department of Computer and Information Sciences, Niagara University, NY 14109, USA
and  \newline
Department of Computer Science and Engineering, State University of New York at Buffalo \newline
email: boxer@niagara.edu
}
}

\date{ }
\begin{document}
\maketitle{}

\begin{abstract}
We show that articulation
points are unnecessary in
freezing sets. \newline

Key words and phrases: digital topology, digital image, freezing set,
articulation point
\newline

MSC: 54B20, 54C35
\end{abstract}

\date
\maketitle

\section{Introduction}
Freezing sets are part
of the fixed point theory
of digital topology. They
were introduced
in~\cite{BxFPSets2} and
studied subsequently
in~\cite{BxConv, Bx21, BxArbDim, BxConseq, BxColdFreeze, BxLtd,BxCone}.
It is often desirable
that a freezing set be
as small as possible; 
while we have some 
knowledge from the papers 
cited above of
how a freezing set can
be determined, we do not
always know if a given
freezing set is minimal.
The current paper makes
a contribution to this
problem by showing that
articulation points may
be omitted from freezing
sets.

\section{Preliminaries}
We use $\N$ for the set of natural numbers,
$\Z$ for the set of integers, and
$\#X$ for the number of distinct members of $X$.

We typically denote a (binary) digital image
as $(X,\kappa)$, where $X \subset \Z^n$ for some
$n \in \N$ and $\kappa$ represents an adjacency
relation of pairs of points in $X$. Thus,
$(X,\kappa)$ is a graph, in which members of $X$ may be
thought of as black points, and members of $\Z^n \setminus X$
as white points, of a picture of some ``real world" 
object or scene.

\subsection{Adjacencies}
This section is largely
quoted or paraphrased
from~\cite{BxArbDim}.

Let $u,n \in \N$, $1 \le u \le n$. 
A digital image $X$ that 
satisfies $X \subset \Z^n$ and
$x = (x_1, \ldots, x_n),~y=(y_1, \ldots, y_n) \in X$ 
are $c_u$-adjacent if and only if
\begin{itemize}
    \item $x \neq y$, and
    \item for at most $u$ indices~$i$, 
          $\mid x_i - y_i \mid = 1$, and
    \item for all indices $j$ such that 
          $\mid x_j - y_j \mid \neq 1$, we have
          $x_j = y_j$.
\end{itemize}
The $c_u$ adjacencies are the adjacencies most used
in digital topology, especially $c_1$ and $c_n$.

In low dimensions, it is also common to denote a
$c_u$ adjacency by the number of points that can
have this adjacency with a given point in $\Z^n$. E.g.,
\begin{itemize}
    \item For subsets of $\Z^1$, $c_1$-adjacency is 2-adjacency.
    \item For subsets of $\Z^2$, $c_1$-adjacency is 4-adjacency and
          $c_2$-adjacency is 8-adjacency.
    \item For subsets of $\Z^3$, $c_1$-adjacency is 8-adjacency,
          $c_2$-adjacency is 18-adjacency, and
          $c_3$-adjacency is 26-adjacency.
\end{itemize}

We use the notations $y \adj_{\kappa} x$, or, when
the adjacency $\kappa$ can be assumed, $y \adj x$, to mean
$x$ and $y$ are $\kappa$-adjacent.
The notations $y \adjeq_{\kappa} x$, or, when
$\kappa$ can be assumed, $y \adjeq x$, mean either
$y=x$ or $y \adj_{\kappa} x$.
\begin{comment}
For $x \in X$, let
\[ N(X,x,\kappa) =
  \{ \, y \in X \mid x \adj_{\kappa} y \, \}.
\]
When the image $(X,\kappa)$ under discussion is clear, we
will use the notations $N(x)$ or $N_{\kappa}(x)$
as follows.
\[
N(x) = \{ \, y \in X \mid y \adjeq_{\kappa} x \, \} =
N(X,x,\kappa) \cup \{x\}.
\]
\end{comment}

A sequence $P=\{y_i\}_{i=0}^m$ in a digital image $(X,\kappa)$ is
a {\em $\kappa$-path from $a \in X$ to $b \in X$} if
$a=y_0$, $b=y_m$, and $y_i \adjeq_{\kappa} y_{i+1}$ 
for $0 \leq i < m$.

$X$ is {\em $\kappa$-connected}~\cite{Rosenfeld},
or {\em connected} when $\kappa$
is understood, if for every pair of points $a,b \in X$ there
exists a $\kappa$-path in $X$ from $a$ to $b$.

A {\em (digital) $\kappa$-closed curve} is a
path $S=\{s_i\}_{i=0}^m$ such that $s_0=s_m$,
and $0 < |i - j| < m$ 
implies $s_i \neq s_j$. If, also, $0 \le i < m$ implies
\[ N(S,x_i,\kappa)=\{x_{(i-1)\mod n},~x_{(i+1)\mod m}\}
\]
then $S$ is a {\em (digital) 
$\kappa$-simple closed curve}.

\subsection{Digitally continuous functions}
This section is largely
quoted or paraphrased
from~\cite{BxArbDim}.

Digital continuity is defined
to preserve connectedness, as at
Definition~\ref{continuous} below. By
using adjacency as our standard of ``closeness," we
get Theorem~\ref{continuityPreserveAdj} below.

\begin{definition}
\label{continuous}
{\rm ~\cite{Bx99} (generalizing a definition of~\cite{Rosenfeld})}
Let $(X,\kappa)$ and $(Y,\lambda)$ be digital images.
A function $f: X \rightarrow Y$ is 
{\em $(\kappa,\lambda)$-continuous} if for
every $\kappa$-connected $A \subset X$ we have that
$f(A)$ is a $\lambda$-connected subset of $Y$.
\end{definition}

If either of $X$ or $Y$ is a subset of the 
other, we use the abbreviation
$\kappa$-continuous for $(\kappa,\kappa)$-continuous.

When the adjacency relations are understood, we will simply say that $f$ is \emph{continuous}. Continuity can be expressed in terms of adjacency of points:

\begin{thm}
{\rm ~\cite{Rosenfeld,Bx99}}
\label{continuityPreserveAdj}
A function $f:X\to Y$ is continuous if and only if $x \adj x'$ in $X$ 
implies $f(x) \adjeq f(x')$.
\end{thm}

See also~\cite{Chen94,Chen04}, where similar notions are referred to as {\em immersions}, {\em gradually varied operators},
and {\em gradually varied mappings}.

A digital {\em isomorphism} (called {\em homeomorphism}
in~\cite{Bx94}) is a $(\kappa,\lambda)$-continuous
surjection $f: X \to Y$ such that $f^{-1}: Y \to X$ is
$(\lambda,\kappa)$-continuous.

The literature uses {\em path} polymorphically: a
$(c_1,\kappa)$-continuous 
function $f: [0,m]_{\Z} \to X$
is a $\kappa$-path if 
$f([0,m]_{\Z})$ is a 
$\kappa$-path from $f(0)$ 
to $f(m)$
as described above.

We use $\id_X$ to denote
the {\em identity function}, $\id_X(x) = x$
for all $x \in X$.

Given a digital image
$(X,\kappa)$, we denote
by $C(X,\kappa)$ the set
of $\kappa$-continuous
functions $f: X \to X$.

Given $f \in C(X,\kappa)$,
a {\em fixed point} of $f$
is a point $x \in X$ such
that $f(x)=x$.
$\Fix(f)$ will denote
the set of fixed points
of~$f$. We say
$f$ is a {\em retraction},
and the set $Y=f(X)$ is a
{\em retract of $X$}, if
$f|_Y = \id_Y$; thus, 
$Y = \Fix(f)$.

%\subsection{Freezing sets}
\begin{definition}
\label{freezeDef}
{\rm \cite{BxFPSets2}}
Let $(X,\kappa)$ be a
digital image. We say
$A \subset X$ is a 
{\em freezing set for $X$}
if given $g \in C(X,\kappa)$,
$A \subset \Fix(g)$ implies
$g=\id_X$. A freezing set
$A$ is {\em minimal} if
no proper subset of $A$
is a freezing set for
$(X,\kappa)$.
\end{definition}

\section{Articulation points and freezing sets}

An {\em articulation point}
or {\em cut point} of a
connected graph 
$(X,\kappa)$ is
a point $x \in X$ such that
$(X \setminus \{x\}, \kappa)$ is not connected
(see Figure~\ref{fig:articulation}).
In this section, we show
that if all articulation 
points are removed from
a freezing set, what is
left is still a freezing
set.

% Figure environment removed

\begin{lem}
\label{articulRetraction}
    Let $M$ be the set of
    articulation points for
    the connected digital image
    $(X,\kappa)$. Let
    $K$ be a $\kappa$-component
    of $X \setminus M$.
    Then there is a 
    $\kappa$-retraction of
    $X$ to $X \setminus K$.
\end{lem}

\begin{proof}
Without loss of generality,
$M \neq \emptyset$.

    Since $X$ is connected,
    there exists $x_0 \in X \setminus (K \cup M)$ 
    such that $x_0$ is
    $\kappa$-adjacent to 
    a point of~$M$.
    By choice of $M$, no 
    point of $K$ is 
    adjacent to $x_0$.
    Let $r: X \to X$ be 
    the function
    \[ r(x) = \left \{
    \begin{array}{ll}
        x & \mbox{if } x \in X \setminus K; \\
        x_0 & \mbox{if } x \in K.
    \end{array}
    \right .
    \]
    It is easily seen that
    $r$ is a 
    $\kappa$-retraction of
    $X$ to $X \setminus K$.
\end{proof}

\begin{lem}
\label{articPtFix}
    Let $M$ be the set of
    articulation points for
    the connected digital image
    $(X,\kappa)$.
    Let
    $x_0 \in M$. Let $K_1$
    and $K_2$ be distinct
    $\kappa$-components 
     of
    $X \setminus \{x_0\}$.
    Let
    $f \in C(X,\kappa)$
    such that for some
    $x_1 \in K_1$ and
    $x_2 \in K_2$,
    $\{x_1,x_2\} \subset \Fix(f)$. Then
    $x_0 \in \Fix(f)$.
\end{lem}

\begin{proof}
    Let $x_0 \in M$ such
    that $x_1$ and $x_2$ 
    belong to distinct
    components $K_1'$ and
    $K_2'$, respectively,
    of $X \setminus \{x_0\}$.
    Let $P_i$ be a 
    %shortest
    $\kappa$-path in $X$
    from $x_i$ to $x_0$,
    $i \in \{1,2\}$. Then
    $P_1 \cup P_2$ is a path
    from $x_1$ to $x_2$. 
    
    By choice of $x_0$ we
    must have 
    $x_0 \in f(P_j)$ where
    $j = 1$ or $j=2$.
    If $f(x_0) \neq x_0$,  $f(P_j)$ is
    a path from $x_j=f(x_j)$
    to $x_0$ to $f(x_0)$ that
    has length greater than
    that of $P_j$, which is
    impossible. The assertion
    follows.
\end{proof}

\begin{prop}
\label{exclude1ArtPt}
    Let $x_0$ be an
    articulation point for
    the connected digital image
    $(X,\kappa)$. Let
    $A$ be a freezing
    set for $(X,\kappa)$.
    Then $A \setminus \{x_0\}$ 
    is a freezing
    set for $(X,\kappa)$.
\end{prop}

\begin{proof}
By Lemma~\ref{articulRetraction},
given a $\kappa$-component $K$
of $X \setminus \{x_0\}$,
there exists a retraction
$r_K$ of $X$ to $X \setminus K$. It follows that
$A \cap K \neq \emptyset$,
for otherwise $r_K$ satisfies
$r_K|_A = \id_A$
yet $r \neq \id_X$,
contrary to $A$ being
a freezing set.

Let $K_1, K_2$ be distinct 
components of 
$X \setminus \{x_0\}$.
Let $f \in C(X,\kappa)$
such that 
    $f|_{A \setminus \{x_0\}} = \id_{A \setminus \{x_0\}}$.

By Lemma~\ref{articPtFix},
$f(x_0) = x_0$. Hence
$f|_A = \id_A$, and the
assertion follows.
\end{proof}

\begin{remark}
    Note Proposition~\ref{exclude1ArtPt} implies that
    if $(X,\kappa)$ is a
    wedge of two digital
    images,
    $(X,\kappa)=
    (X_1,\kappa) \vee
    (X_2,\kappa)$, then
    the ``wedge point" of
    $X$ does not belong
    to any minimal 
    freezing set for 
    $(X,\kappa)$.
\end{remark}

\begin{thm}
    Let $(X,\kappa)$ be a
    finite connected digital 
    image. Let $A$ be a
    freezing set for 
    $(X,\kappa)$.
    Let $M$ be the set of
    articulation points of
    $(X,\kappa)$. Then
    $A \setminus M$ is a
    freezing set for 
    $(X,\kappa)$.
\end{thm}

\begin{proof}
    Let $M = \{x_i\}_{i=1}^m$.
    By Proposition~\ref{exclude1ArtPt},
    $A \setminus \{x_1\}$
    is a freezing set for 
    $(X,\kappa)$.

    We proceed inductively.
    Suppose $A \setminus \{x_1\}_{i=1}^k$
    is a freezing set for 
    $(X,\kappa)$, where
    $1 \le k < m$.
     By Proposition~\ref{exclude1ArtPt}, \[
     A \setminus \{x_1\}_{i=1}^{k+1} =
     (A \setminus \{x_1\}_{i=1}^k) 
     \setminus \{x_{k+1}\}
     \]
     is a freezing set for 
    $(X,\kappa)$. This 
    completes the induction.

    Since $A \setminus \{x_1\}_{i=1}^m = A \setminus M$,
    our assertion is
    established.
\end{proof}

\section{Further remarks}
We have shown that any and
all articulation points
can be removed from a
freezing set~$A$ of a 
digital image; what is left
remains a freezing set.

\begin{thebibliography}{99}

\bibitem{Bx94}
L. Boxer, Digitally continuous functions,
{\em Pattern Recognition Letters}
15 (8) (1994), 833-839
%\newline
%https://www.sciencedirect.com/science/article/abs/pii/0167865594900124

\bibitem{Bx99}
L. Boxer, A classical construction for the digital fundamental group, 
{\em Journal of Mathematical Imaging and Vision} 10 (1999), 51-62.
%\newline
%https://link.springer.com/article/10.1023/A$\%$3A1008370600456

\bibitem{Bx10}
L. Boxer, Continuous maps on digital simple closed curves,
{\em Applied Mathematics} 1 (2010), 377-386.
%\newline
%https://www.scirp.org/pdf/AM20100500006$\_$47396425.pdf

\bibitem{BxFPSets2}
L. Boxer, Fixed point sets 
in digital topology, 2, {\em Applied General
Topology} 21(1) (2020),
111-133

\bibitem{BxConv}
L. Boxer, 
Convexity and freezing 
sets in digital topology, 
{\em Applied General 
Topology} 22 (1) (2021), 
121 - 137

\bibitem{Bx21}
 L. Boxer, Subsets and 
 freezing sets in the 
 digital plane,
 {\em Hacettepe Journal of 
 Mathematics and 
 Statistics} 50 (4)
 (2021), 991 - 1001

\bibitem{BxArbDim}
L. Boxer, 
 Freezing Sets for 
 arbitrary digital 
 dimension,
 {\em Mathematics} 10 
 (13) (2022), 2291. 

 \bibitem{BxConseq}
  L. Boxer, 
  Some consequences of 
  restrictions on 
  digitally continuous 
  functions, 
  {\em Note di Matematica}
  42 (1) (2022), 47 - 76.

\bibitem{BxColdFreeze}
L. Boxer, Cold and 
freezing sets in the 
digital plane,
{\em Topology Proceedings}
61 (2023), 155 - 182

\bibitem{BxLtd}
L. Boxer, Limiting sets in 
digital topology, 
submitted. 
\newline
https://arxiv.org/abs/2302.01383 

\bibitem{BxCone}
 L. Boxer, 
 Limiting sets for digital 
 cones and suspensions, 
 submitted. 
\newline
https://arxiv.org/abs/2307.04969 

\bibitem{BxSt19}
L. Boxer and P.C. Staecker,
Remarks on fixed point assertions in digital topology,
{\em Applied General Topology} 20 (1) (2019), 135-153.
%\newline
%https://polipapers.upv.es/index.php/AGT/article/view/10474/11201

%\bibitem{ChartTian}
%G. Chartrand and S. Tian, 
%Distance in digraphs. 
%{\em Computers $\&$ Mathematics with 
%Applications} 34 (11) (1997), 15 - 23
%\newline
%https://www.sciencedirect.com/science/article/pii/S0898122197002162

\bibitem{Chen94}
L. Chen, Gradually varied surfaces and its optimal 
uniform approximation, 
{\em SPIE Proceedings} 2182 (1994), 300-307. 
%\newline
%https://spie.org/Publications/Proceedings/Paper/10.1117/12.171078 \newline 
%?origin$\_$id=x4325$\&$start$\_$volume$\_$number=02100

\bibitem{Chen04}
L. Chen, {\em Discrete Surfaces and Manifolds}, Scientific Practical Computing, 
Rockville, MD, 2004

\bibitem{Rosenf79}
A. Rosenfeld,
Digital topology,
{\em The American Mathematical Monthly} 86 (8) (1979), 621 - 630

\bibitem{Rosenfeld}
A. Rosenfeld, `Continuous' functions on digital pictures, 
{\em Pattern Recognition Letters} 4, 1986, 177 - 184
%\newline
%https://www.sciencedirect.com/science/article/pii/01678655

\end{thebibliography}
\end{document}