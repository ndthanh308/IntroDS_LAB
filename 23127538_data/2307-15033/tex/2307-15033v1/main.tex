\documentclass[10pt,twocolumn,letterpaper]{article}

\usepackage{iccv}
\usepackage{times}
\usepackage{epsfig}
\usepackage{graphicx}
\usepackage{amsmath}
\usepackage{amssymb}
\usepackage{multirow}
\usepackage{enumitem}


% Include other packages here, before hyperref.

% If you comment hyperref and then uncomment it, you should delete
% egpaper.aux before re-running latex.  (Or just hit 'q' on the first latex
% run, let it finish, and you should be clear).
\usepackage[pagebackref=true,breaklinks=true,letterpaper=true,colorlinks,bookmarks=false]{hyperref}

\iccvfinalcopy % *** Uncomment this line for the final submission

\def\iccvPaperID{4957} % *** Enter the ICCV Paper ID here
\def\httilde{\mbox{\tt\raisebox{-.5ex}{\symbol{126}}}}

% Pages are numbered in submission mode, and unnumbered in camera-ready
\ificcvfinal\pagestyle{empty}\fi


\begin{document}

%%%%%%%%% TITLE - PLEASE UPDATE
\title{Diverse Inpainting and Editing with GAN Inversion}


\author{Ahmet Burak Yildirim$^{*}$ \qquad Hamza Pehlivan\thanks{Joint first authors, contributed equally.}  \qquad
Bahri Batuhan Bilecen  \qquad
Aysegul Dundar \\
Bilkent University\\
\tt\small \{a.yildirim, hamza.pehlivan, batuhan.bilecen\}@bilkent.edu.tr\\
\tt\small adundar@cs.bilkent.edu.tr\\}

\maketitle

\begin{abstract}

Recent inversion methods have shown that real images can be inverted into StyleGAN's latent space and numerous edits can be achieved on those images thanks to the semantically rich feature representations of well-trained GAN models. 
However, extensive research has also shown that image inversion is challenging due to the trade-off between high-fidelity reconstruction and editability.
In this paper, we tackle an even more difficult task, inverting erased images into GAN's latent space for realistic inpaintings and editings. 
Furthermore, by augmenting inverted latent codes with different latent samples, we achieve diverse inpaintings.
Specifically, we propose to learn an encoder and mixing network to combine encoded features from erased images with StyleGAN's mapped features from random samples. 
To encourage the mixing network to utilize both inputs, we train the networks with generated data via a novel set-up.
We also utilize higher-rate features to prevent color inconsistencies between the inpainted and unerased parts.
 We run extensive experiments and compare our method with state-of-the-art inversion and inpainting methods. Qualitative metrics and visual comparisons show significant improvements.

\end{abstract}


\section{Introduction}
Deep learning models have been widely used in many applications.
For example, BERT~\citep{devlin_bert_2019}, GPT-3~\citep{brown_language_2020}, and T5~\citep{raffel_exploring_2020} achieved state-of-the-art~(SOTA) results on different natural language processing~(NLP) tasks. 
For computer vision~(CV), Transformer-like models such as ViT~\citep{dosovitskiy_image_2021} and Swin Transformer~\citep{liu_swin_2021} deliver excellent accuracy performance upon multiple tasks. 


At the same time, training deep learning models has been a critical problem troubling the community due to the long training time, especially for those large models with billions of parameters~\citep{brown_language_2020}. 
In order to enhance the training efficiency, researchers propose some manually designed parallel training strategies~\citep{narayanan_efficient_2021,shazeer_mesh-tensorflow_2018,xu_gspmd_2021}. 
However, selecting, tuning, and combining these strategies require extensive domain knowledge in deep learning models and hardware environments. With the increasing diversity of modern hardware architectures~\cite{flynn_very_1966,flynn_computer_1972} and the rapid development of deep learning models, these manually designed approaches are bringing heavier burdens to developers. 
Hence, \emph{automatic parallelism} is introduced to automate the parallel strategy searching for training models.


There are two main categories of parallelism in deep learning models: inter-layer parallelism~\citep{huang_gpipe_2019,narayanan_pipedream_2019,narayanan_memory-efficient_2021,fan_dapple_2021,li_chimera_2021,lepikhin_gshard_2021,du_glam_2022,fedus_switch_2022} and intra-layer parallelism~\citep{li_pytorch_2020,narayanan_efficient_2021,rasley_deepspeed_2020,fairscale_authors_fairscale_2021}. 
Inter-layer parallelism partitions the model into disjoint sets on different devices without slicing tensors. 
Alternatively, intra-layer parallelism partitions tensors in a layer along one or more axes and distributes them across different devices.


Current automatic parallelism techniques focus on optimizing strategies within these two categories. However, they treat these two categories separately. 
Some methods~\citep{zhao_vpipe_2022,jia_exploring_2018,cai_tensoropt_2022,wang_supporting_2019,jia_beyond_2019,schaarschmidt_automap_2021,liu_colossal-auto_2023} overlook potential opportunities for inter- or intra-layer parallelism, the others optimize inter- and intra-layer parallelism hierarchically and sequentially~\citep{narayanan_pipedream_2019,fan_dapple_2021,he_pipetransformer_2021,tarnawski_efficient_2020,tarnawski_piper_2021,zheng_alpa_2022}. 
As a result, current automatic parallelism techniques often fail to achieve the global optima and instead become trapped in local optima. 
Therefore, a unified inter- and intra-layer approach is needed to enhance the effectiveness of automatic parallelism.


This paper aims to find the optimal parallelism strategy while simultaneously considering inter- and intra-layer parallelism. 
It enables us to search in a more extensive strategy space where the globally optimal solution lurk. 
However, unifying inter- and intra-layer parallelism in automatic parallelism brings us two challenges. 
Firstly, to adopt a unified perspective on the inter- and intra-layer automatic parallelism, we should not formalize them with separate formulations as prior works. Therefore, how can we express these parallelism strategies in a unified formulation? 
Secondly, previous methods take a long time to obtain the solution with a limited strategy space. Therefore, how can we ensure that the best solution can be obtained in a reasonable time while expanding the strategy space?


To solve the above challenges, we propose UniAP. For the first challenge, UniAP adopts the mixed integer quadratic programming~(MIQP)~\citep{lazimy_mixed_1982} to search for the globally optimal parallel strategy automatically. 
It unifies the inter- and intra-layer automatic parallelism in a single MIQP formulation. 
For the second challenge, our complexity analysis and experimental results show that UniAP can obtain the globally optimal solution in a significantly shorter time.


The contributions of this paper are summarized as follows: 
\begin{itemize}
    \item We propose UniAP, the first framework to unify inter- and intra-layer automatic parallelism in model training.
    \item The optimal parallel strategies discovered by UniAP exhibit scalability on training throughput and strategy searching time.
    \item The experimental results show that UniAP speeds up model training on four Transformer-like models by up to 1.70$\times$ and reduces the strategy searching time by up to 16$\times$, compared with the SOTA method.
\end{itemize}

\section{Related Work}
\label{sec:related}

\begin{table}[t]
\small
\centering
\caption{Comparison of our method with related settings}
\begin{tabular}{cccc}
\toprule
Setting & Detect Novel OOD Data & Semi-Supervised & Learns from Novel OOD Data \\
\midrule
SSOD & \xmark & \cmark & \xmark \\ 
Open-World OD & \cmark & \xmark & \cmark \\
Open-Set SSOD & \cmark & \cmark & \xmark \\ \midrule
\textbf{Our Method} & \cmark & \cmark & \cmark \\
\bottomrule
\end{tabular}
\label{tab:comparison}
\end{table}

\paragraph{Semi-Supervised Object Detection.} Semi-supervised object detection (SSOD) approaches have become popular to reduce the need for labeling \cite{sohn2020detection, berthelot2019mixmatch, jeong2019consistency}. Pseudo-labeling based methods such as FlexMatch \cite{zhang2021flexmatch}, TSSDL \cite{shi2018transductive}, and others \cite{iscen2019label, luo2018smooth, yan2019semi, liu2021unbiased, xu2021end}, first train a teacher model using only labeled data and then use that model to create pseudo-labels for unlabeled images. The pseudo-labels are then used along with the original labeled data to train a student model. On the other hand, consistency regularization approaches such as \cite{sajjadi2016regularization, laine2017temporal, tarvainen2017mean, liu2021certainty, luo2018smooth, jeong2019consistency, iscen2019label, liu2021unbiased, xu2021end}, aim to minimize a consistency loss between differently augmented versions of an image. All of these semi-supervised learning approaches assume a ``closed-world'' setting with a fixed set of classes in both training and testing, which is not a valid assumption in real-world applications.

\paragraph{Open-World Object Detection.} Open-world object detection enables the detection of novel objects by incrementally adding novel object classes to the set of known classes. Previous work \cite{kim2022learning, kuo2015deepbox, o2015learning, wang2020leads, Maaz2022Multimodal} has studied different methods of object proposals for novel objects by attempting to remove the notion of class (all objects are regarded the same). ORE \cite{joseph2021towards} is the first to propose an open-world object detector that identifies novel classes as ‘unknown’ and proceeds to learn the unknown classes once the labels become available. \cite{han2022expanding} aims to identify unknown objects by separating high/low-density regions in the latent space. Both these approaches work in a fully-supervised setting. Our setup goes a step further and situates the open-world problem in the context of semi-supervised learning, with limited amounts of labeled ID data \textit{only}, that more closely resembles the real-world settings. 

\paragraph{Unsupervised Object Localization.} Recently proposed methods such as CutLER \cite{wang2023cut}, FreeSolo \cite{wang2022freesolo}, LOST \cite{LOST}, and MOST \cite{rambhatla2023most} propose to localize objects in an unsupervised manner, either by segmentation masks or bounding boxes. Some of these \cite{wang2023cut, LOST, rambhatla2023most} use features from self-supervised trained transformers to localize objects in the scene. In our work, we evaluate the capabilities of such methods for localizing OOD objects, as they present open-world capabilities. Based on our evaluation (\ref{sec:expts:ablation}), we use CutLER as part of the OOD Explorer to localize OOD classes. Section \ref{sec:expts} provides the details of our evaluation. 

\paragraph{Open-Set/Open-world Semi-Supervised Object Detection.}
The open-set semi-supervised object detection problem \cite{liuopen} addressed some of the limitation of the above mentioned work. Furthermore, they address like the performance of ID classes in the presence of OOD data, but they do not learn from it or improve OOD performance. They propose an offline OOD detector to filter out OOD data, thus limiting the risk of ID performance in the presence of OOD data. In contrast, our approach \textit{both} improves performance for ID classes \textit{as well as} OOD classes, i.e., our proposed framework solves a strictly stronger problem. Specifically speaking, \cite{liuopen} solves for identifying novel classes and filters it out, but does not re-introduce the classes back into the training pipeline in order to be able to learn its features. \cite{mullappilly2024semi} addresses some of the limitations of the previous mentioned methods by extending the problem to a semi-supervised setting. However, their problem setting is similar to an incremental learning setting, access to unknown class labels is provided in subsequent tasks. Our generalized setting, on the other hand, does not require access to any unknown class labels. 

\section{Methodology}
\label{sec:method}

\subsection{Overview}
\label{sec:method_fmwk}

As shown in~\cref{fig:method_fmwk}, the proposed unsupervised MOT framework is trained with the widely-used contrastive learning technique~\cite{chen2020simple,he2020momentum}. 
\lk{Specifically, for multi-object tracking}, objects within the tracklet ($\boldsymbol{k}_{+}$) should be pulled together and different tracklets ($\boldsymbol{k}_{-}$) should be separated. It can be mathematically formulated as:

\begin{equation}
% \begin{split}
    \mathcal{L}_{cl}( \boldsymbol{q}; \boldsymbol{k}_{+}; \boldsymbol{k}_{-} )= 
    - \log \frac{\exp(\boldsymbol{q} \cdot \boldsymbol{k}_{+} / \epsilon)}{\sum_{i}\exp(\boldsymbol{q} \cdot \boldsymbol{k}_{i} / \epsilon)}
  \label{eq:method_nce}
% \end{split}  
\end{equation}

\noindent where $\mathcal{L}_{cl}$ denotes the InfoNCE~\cite{oord2018representation} loss function, and $\epsilon$ is the temperature hyper-parameter~\cite{wu2018unsupervised}. 
Within a video, following the unsupervised tracking fashion~\cite{liu2022online,shuai2022id}, the positive and negative keys mainly come from two sources, \ie pseudo-labeled historical frame and self-augmented frame. 

\lk{However, two issues occur: (1) the uncertainty reduces the accuracy of pseudo-tracklets and (2) the randomly augmented samples fail to learn the inter-frame consistency.} 
We argue the above issues are not independent. 
\lk{By leveraging the uncertainty in turn,} the accurate pseudo-tracklets can guide the qualified positive and negative augmentations.

To address these two issues, we propose an uncertainty-aware pseudo-tracklet labeling strategy in \cref{sec:method_uoap}, which integrates a verification-and-rectification mechanism into the tracklet generation. Our method significantly improves the accuracy of pseudo-identities, especially in long-term interval. 
Then we propose a tracklet-guided augmentation strategy in \cref{sec:method_ada_aug}, which brings the temporary information into spatial augmentation. The augmented samples simulates the objects' motion. A hierarchical uncertainty-based sampling strategy is proposed for hard sample mining. More details are described in the following section.


\subsection{Uncertainty-aware Tracklet-Labeling}
\label{sec:method_uoap}

Accurate pseudo tracklet is critical in \liuk{learning feature consistency}. 
However, without manual annotation, \lk{the aggravated uncertainty makes} the tracklet-labeling a huge challenge due to various interference factors, including similar appearance among objects, frequent object cross and occlusions, \etc. 
\lk{In fact, the uncertainty can also be leveraged to improve the pseudo-accuracy in turn.}
In this section, we propose an \textbf{U}ncertainty-aware \textbf{T}racklet-\textbf{L}abeling (\textbf{UTL}) strategy for better pseudo-tracklets.

Given an input video sequence $V = \{I^{1}, I^{2}, \cdots, I^{N}\}$, each frame $I^{t}$ is annotated with the bounding boxes $B^{t} = \{b_{1}^{t}, b_{2}^{t}, \cdots, b_{M^{t}}^{t}\}$ of $M^{t}$ objects in $t_{th}$ frame, where $b_{i}^{t} = (cx_{i}^{t}, cy_{i}^{t}, w_{i}^{t}, h_{i}^{t})$ is the center coordinate and shape of the $i_{th}$ object $o_{i}^{t}$. As shown in~\cref{fig:method_fmwk}, \mywork~generates accurate pseudo-tracklets in four main steps:

1) \textbf{Association}. For a certain object $o_{i}^{t}$ in frame $I^{t}$, the $\ell_2$-normalized representation $\boldsymbol{f}_{i}^{t}$ can be expressed as $\boldsymbol{f}_{i}^{t} = {\phi}(I^{t}, b_{i}^{t})$, 
% \begin{equation}
%   \boldsymbol{f}_{i}^{t} = {\phi}(I^{t}, b_{i}^{t})
%   % / {\Vert {\phi}(I^{t}, b_{i}^{t}) \Vert}_{2}
%   \label{eq:method_feat}
% \end{equation}
where the embedding encoder is denoted as $\phi$.

To associate the objects in frame $I^{t}$ with the objects or trajectories in previous $I^{t \minus 1}$, a similarity matrix is constructed with their appearance embeddings:

\begin{equation}
  \boldsymbol{C} \in \mathbb{R}^{M^{t} \times M^{t \minus 1}}, \;
  c_{i,j} = {\boldsymbol{f}_{i}^{t}} \cdot  \boldsymbol{f}_{j}^{t \minus 1}
  \label{eq:method_matrix}
\end{equation}

\noindent where $c_{i,j}$ represents the cosine similarity between the $i_{th}$ object in frame $I^{t}$ and the $j_{th}$ object (or trajectory) in frame $I^{t \minus 1}$. Then the Hungarian algorithm~\cite{kuhn1955hungarian} is adopted to generate the identity association results.

2) \textbf{Verification}. However, the appearance representations are sometimes unreliable, especially in the unsupervised scenario. To solve this issue, an uncertainty metric is proposed to evaluate the association after the first stage.

% For an object $o_{i}^{t}$ in frame $I^{t}$, the similarities against the $M^{t \minus 1}$ objects in the previous frame can be expressed as:

% \begin{equation}
%   \boldsymbol{s}_{i} = \boldsymbol{C}_{i} = [c_{i,1}, c_{i,2}, \cdots, c_{i,M^{t \minus 1}}]^T
%   \label{eq:method_svec}
% \end{equation}

% Inspired by margin-based OOD detection~\cite{hendrycks2016baseline}, we assume that the assignment ($o_{i}^{t} \!\sim\! o_{j}^{t \minus 1}$) in the association stage is not convincing under the following circumstances:

% \begin{itemize}
%     \setlength{\itemsep}{0pt}
%     \item The assigned similarity between $o_{i}^{t}$ and $o_{j}^{t \minus 1}$ is relatively low (\ie, $c_{i,j} < m_1$).
%     \item The second-highest similarity with others ($c_{i,j_{2}}$) is close to the assigned $o_{j}^{t \minus 1}$ (\ie, $c_{i,j} - c_{i,j_{2}} < m_2$).
% \end{itemize}

% Based on these assumptions, we define an association-level uncertainty metric, which is formulated as:



Object association can be viewed as multi-category classification.
And confidence-score has been proved efficient and effective on detecting mis-classified examples~\cite{hendrycks2016baseline}.
Inspired by this, we propose to detect the mis-associated objects through the similarity-scores.


Given an object $o_{i}^{t}$ associated with $o_{j}^{t \minus 1}$ in the previous frame based on \cref{eq:method_matrix}, the association ($o_{i}^{t} \!\sim\! o_{j}^{t \minus 1}$) is unconvincing in two cases: 
1) the assigned similarity $c_{i,j}$ is relatively low (\eg, partial occlusion or motion blur) and 
2) there are other objects whose similarities are close to the assigned $c_{i,j}$ (\eg, similar appearance or indistinguishable embedding).
It can be formulated as:

\begin{equation}
  c_{i,j} < m_1; \quad c_{i,j_{2}} > c_{i,j} - m_2
  \label{eq:method_margin}
\end{equation}


\noindent 
where $m_1,m_2$ are constant margins. Only the second-highest similarity with others ($c_{i,j_{2}}$) is considered for simplicity.
In an ideal association, $c_{i,j}$ should be close to 1 and $c_{i,j_{2}}$ close to 0.
We thus proposed to estimate the association \lk{risk} as:

% \sigma_{i,j} = - \left( 
% \log c_{i,j} + \log \left( 1 - c_{i,j_{2}} \right)
% + \overline{\log \left( 1 - c_{i,l} \right) }
% \right)  
\begin{equation}
  \sigma_{i,j} = - \log c_{i,j} - \log \left( 1 - c_{i,j_{2}} \right)
  \label{eq:method_energy}
\end{equation}

Detailed derivation process refers to the supplementary materials.
Combining with \cref{eq:method_margin} and \cref{eq:method_energy} , an adaptive threshold is proposed:

\begin{equation}
  % \gamma_{i,j} = -\log \left( 1 + m_2 - c_{i,j} \right) -\log m_1 \left( 1 - m_3 \right)
  \gamma_{i,j} =  -\log m_1 - \log \left( 1 + m_2 - c_{i,j} \right)
  \label{eq:method_border}
\end{equation}

As shown in~\cref{fig:method_verify}, when the \lk{risk} $\sigma_{i,j}$ is higher than the threshold $\gamma_{i,j}$, the assignment ($o_{i}^{t} \!\sim\! o_{j}^{t \minus 1}$) should be re-considered. 
\lk{The \textbf{association uncertainty} is quantified as:}

\begin{equation}
  \delta_{i,j} = \sigma_{i,j} - \gamma_{i,j}
  \label{eq:method_uncertain}
\end{equation}

The results are not sensitive to the exact margins. We set $m_1 = 0.5$ and $m_2 = 0.05$ for a slightly better performance.
% More experimental details are shown in the supplementary materials.

The uncertain pairs after the verification stage and unmatched objects after the association stage are gathered as uncertain candidates for the rectification stage.


3) \textbf{Rectification}. 
The rectification stage is performed among the uncertain candidate. The similarities between two adjacent frames are no longer convincing.
% due to irregular motion, severe occlusion, and so on. 
More information should be taken into account, including motion \lk{estimation} and appearance \lk{variation} within a tracklet. 
% Specifically, intersection-over-union (IoU)~\cite{bewley2016simple} is the widely-used motion metric. At the same time, the tracklet embeddings can provide complementary appearance information.

For the uncertain candidates, \mywork~constructs another similarity matrix for the secondary rectification. 
First, \lk{the motion constraints should be relaxed}, so the association shares overlap \lk{higher than} $\beta$ 
% in adjacent frames 
\lk{are preserved}. 
Second, \lk{the appearance should not vary extremely fast}, so we adopt the averaged similarity between object $o_{i}^{t}$ and tracklet $trk_{j} = \{o_{j}^{t \minus K}, \cdots, o_{j}^{t \minus 1}\}$ within previous $K$ frames. 
In this stage, we solve the sub-problem of global identity assignments, which can be formulated as:

\begin{equation}
\begin{split}
  \boldsymbol{C}^\prime \in \mathbb{R}^{{M^{t}}^\prime \times {M^{t \minus 1}}^\prime} & \\
  c^\prime_{i,j} = \left( \frac{1}{K} \sum_{\hat{t} = t \minus K}^{t \minus 1} {\boldsymbol{f}_{i}^{t}} \cdot  \boldsymbol{f}_{j}^{\hat{t}} \right) 
            \times \mathbb{I} & \left( \text{IoU} \left( b_{i}^{t}, b_{j}^{t \minus 1} \right) > \beta \right) 
  \label{eq:method_recti}
\end{split}
\end{equation}

\noindent where $\mathbb{I}(*)$ is the indicator function. Then the match set is updated based on the Hungarian algorithm.

\lk{
\textit{Remark.} Our core contribution is the uncertainty-based verification mechanism, rather than the specific rectification, which shall be adjusted in practice. Empirically we set $\beta=0.1$ and $K=5$.
}

% Figure environment removed

4) \textbf{Propagation}. The pseudo-tracklets are propagated frame-by-frame. As shown in~\cref{fig:method_reidacc}, our strategy brings \lk{consistently} accurate pseudo-identities, \lk{\eg, reaching 97\% accuracy across 100 frames}.
% The pseudo-tracklets are progressively updated during the training process.
The long-term intra-tracklet consistency is successfully maintained.
% by the accurate pseudo-identities.

It is worth mentioning that the \lk{verification and rectification} stages can be naturally applied to the inference process to boost the performance, \lk{which does not conflict with existing association methods}.

\subsection{Tracklet-Guided Augmentation}
\label{sec:method_ada_aug}

The accurate pseudo-tracklets can guide the sample augmentation in the unsupervised MOT framework.
To learn the \liuk{inter-frame consistency}~\cite{chen2020simple,zhang2021fairmot}, good training samples should be diverse and \liuk{temporal-aware}. 
However, as illustrated in~\cref{fig:method_ada_aug}, existing methods usually treat augmentation and multi-object tracking as two isolated tasks, leading to ineffective augmentations. 
Instead, this paper utilizes the tracklet's spatial displacements to guide the augmentation process. 
According to a properly selected anchor pair, the proposed strategy makes the augmented frames aligned to the historical frames, simulating realistic tracklet movements. The proposed method concurrently focuses on the hard negative samples.
Details \lk{of the \textbf{T}racklet-\textbf{G}uided \textbf{A}ugmentation (TGA)} are given below.

% Figure environment removed

We introduce the temporal information into spatial transformation. 
First, given a current frame $I^{t}$ with $M^{t}$ objects, we select a source-anchor object $o_{a}^{t}$, whose bounding box is denoted as $b_{a}^{t} = (cx_{a}^{t}, cy_{a}^{t}, w_{a}^{t}, h_{a}^{t})$. Then, we choose a target-anchor $o_{a}^{t \minus \tau}$ in $(t \minus \tau)_{th}$  historical frame from the pseudo-tracklet $trk_{a} = \{o_{a}^{t_0}, o_{a}^{t_1}, \cdots, o_{a}^{t}\}$. 
Finally, to augment the current $I^{t}$ to align with historical $I^{t \minus \tau}$,  a tracklet-guided affine transformation can be expressed as:

\begin{equation}
  \begin{bmatrix}
      x^{t \minus \tau} \\ y^{t \minus \tau} \\ 1
  \end{bmatrix}
  =
  \boldsymbol{M}_{t}^{t \minus \tau}
  \begin{bmatrix}
      x^{t} \\ y^{t} \\ 1
  \end{bmatrix}
  =
  \begin{bmatrix}
      m_{11} & m_{12} & m_{13} \\
      m_{21} & m_{22} & m_{23} \\
      0      & 0      & 1
  \end{bmatrix}
  \begin{bmatrix}
      x^{t} \\ y^{t} \\ 1
  \end{bmatrix}
  \label{eq:method_affine}
\end{equation}

\noindent where $x^*,y^*$ are spatial coordinates, and $\boldsymbol{M}_{t}^{t \minus \tau}$ can be solved by direct linear transform (DLT) algorithm ~\cite{detone2016deep}. 
% with the corner locations of the anchor pair $(o_{a}^{t} \!\sim\! o_{a}^{t \minus \tau})$. 
Then an augmented frame $\tilde{I}^{t}$ is generated based on the tracklet-guided affine transformation with perspective jitter, which can be expressed as $\tilde{I}^{t} = \mathcal{T}\left(I^{t}, M_{t}^{t \minus \tau} \right)$.
% \begin{equation}
%   \tilde{I}^{t} = \mathcal{T}\left(I^{t}, M_{t}^{t \minus \tau} \right)
%   \label{eq:method_aug}
% \end{equation}

Intuitively, a proper anchor-selection is vitally important for our augmentation strategy. 

First, the identity accuracy of anchor pair $(o_{a}^{t} \!\sim\! o_{a}^{t \minus \tau})$ is important. In other words, the consistency of anchor tracklet $trk_{a}$ should be guaranteed. We thus design a tracklet-level uncertain metric based on the propagated association-level uncertainty defined in \cref{eq:method_uncertain}, which is formulated as:

\begin{equation}
  \Omega_{i} = \frac{1}{n} \sum_{s=t_0}^{t} \exp (\delta_{i}^{s})
  % \Omega_{i} = \sqrt[n]{\sigma_{i}^{t_0} \cdot \sigma_{i}^{t_1} \cdots \sigma_{i}^{t}}
  \label{eq:method_tenergy}
\end{equation}

\noindent where $\Omega_{i}$ represents the uncertainty of tracklet $trk_{i}$, \lk{and $n$ is the tracklet length}.
An uncertainty-based sampling strategy is designed to select the source anchor $o_{a}^{t}$ (along with the anchor $trk_{a}$) from the $M^{t}$ objects in frame $I^{t}$, which can be formulated as:

\begin{equation}
  p\left(a=i \mid t \right) 
  % = softmax\left(-\Omega_{i}\right)
  = \frac{\exp{\left(-\Omega_{i}\right)}}{\sum_{\hat{i}=1}^{M^{t}}\exp{\left(-\Omega_{\hat{i}}\right)}}
  \label{eq:method_sel_an_src}
\end{equation}

\noindent where $p\left(a=i \mid t \right)$ represents the probability to choose the $i_{th}$ tracklet $trk_{i}$ as the anchor $trk_{a}$.
The uncertain tracklet with high $\Omega$ is less likely to be selected, avoiding dramatic augmentations from erroneous pseudo-tracklets.

Second, hard negative samples matters in discriminablity learning. We tend to choose an indistinguishable (or, high uncertain) target anchor $o_{a}^{t \minus \tau}$ along the tracklet $trk_{i}$. The selection probability can be formulated as:

\begin{equation}
  p\left(\pi=t \minus \tau \mid a \right) 
  = \frac{\exp{\left(\delta_{a}^{t \minus \tau}\right)}}{\sum_{\hat{\tau}=t_0}^{t-1}\exp{\left(\delta_{a}^{t-\hat{\tau}}\right)}}
  \label{eq:method_sel_an_tgt}
\end{equation}

\lk{A visualization example are displayed in the supplementary material to illustrate the hierarchical sampling process.}

Compared with conventional random transformation, the proposed tracklet-guided augmentation is well-directed and tracking-related. 
\lk{Together with accurate pseudo-tracklets, \mywork~successfully improves the inter-frame consistency, as shown in \cref{fig:method_disc_vis}. }


% Figure environment removed

% \subsection{Momentum Memory Dictionary}
% \label{sec:method_md}


%To reuse the encoded samples from the intermediate mini-batches, we maintain a queue for each video in the memory dictionary by enqueueing the $M^{t}$ objects in the current frame and removing the oldest samples.
%In representation learning, high-quality negative samples play an essential role~\cite{chen2020simple,he2020momentum}. However, existing unsupervised trackers only take negative samples from adjacent frames, augmented frames, and the current frame itself. The lack of negative sample diversity prevents trackers from learning discriminative representations. \mywork~adopts a momentum dictionary mechanism to alleviate this problem.

%As shown in~\cref{fig:method_fmwk}, we build a memory dictionary for each \textit{independent} video input during training. Given an input image $I^{t}$ from video $V$, we randomly sample a number of negative object samples from other videos in the memory dictionary, so as to enrich the negative sample diversity. To reuse the encoded samples from the intermediate mini-batches, we maintain a queue for each video in the memory dictionary by enqueueing the $M^{t}$ objects in the current frame and removing the oldest samples.


% % Figure environment removed


% % Figure environment removed

% % Figure environment removed

% % Figure environment removed

% % Figure environment removed

% % Figure environment removed

% Figure environment removed


% Figure environment removed

% Figure environment removed


\subsection{Implementation Details}


\paragraph{Network.} In order to disentangle shape and color latent information within the hashgrids, we split the single hash table in the NeRF network architecture of Instant-NGP~\cite{mueller2022instant} into two: a density grid $\mathcal{G}^{\sigma}$ and a color grid $\mathcal{G}^c$, with the same settings as the original density grid in the open-source PyTorch implementation torch-ngp~\cite{torch-ngp}. We do this to make it possible to make fine-grained edits of one to one of the color or geometry properties without affecting the other. The rest of the network architecture remains the same, including a sigma MLP $f^\sigma$ and a color MLP $f^c$. For a spatial point $\mathbf{x}$ with view direction $\mathbf{d}$, the network predicts volume density $\sigma$ and color $c$ as follows:
\begin{align}
    \sigma, \mathbf{z} &= f^\sigma(\mathcal{G}^{\sigma}(\mathbf{x})) \\
    c &= f^c(\mathcal{G}^c(\mathbf{x}),\mathbf{z},\mathrm{SH}(\mathbf{d}))
\end{align}
where $\mathbf{z}$ is the intermediate geometry feature, and $\mathrm{SH}$ is the spherical harmonics directional encoder~\cite{mueller2022instant}. The same as Instant-NGP's settings, $f^\sigma$ has 2 layers with hidden channel 64, $f^c$ has 3 layers with hidden channel 64, and $\mathbf{z}$ is a 15-channel feature.

We compare our modified NeRF network with the vanilla architecture in the Lego scene of NeRF Blender Synthetic dataset\cite{mildenhall2020nerf}. We train our network and the vanilla network on the scene for 30,000 iterations. The result is as follows:
\begin{itemize}
    \item Ours: training time 441s, PSNR 35.08dB
    \item Vanilla: training time 408s, PSNR 34.44dB
\end{itemize}
We observe slightly slower runtime and higher quality for our modified architecture, indicating that this modification causes negligible changes.

\paragraph{Training.}
% We use Instant-NGP\fcite{NGP} as our editing framework backbone to achieve real-time editing preview. 
We select Instant-NGP~\cite{mueller2022instant} as the NeRF backbone of our editing framework.
Our implementations are based on the open-source PyTorch implementation torch-ngp~\cite{torch-ngp}. All experiments are run on a single NVIDIA RTX 3090 GPU. Note that we make a slight modification to the original network architecture. Please refer to the supplementary material for details.

During the pretraining stage, we set $\msymbol{weight_pretrain_color}=\msymbol{weight_pretrain_sigma}=1$ and the learning rate is fixed to $0.05$. During the finetuning stage, we set $\msymbol{weight_train_color} = \msymbol{weight_train_depth} = 1$ with an initial learning rate of 0.01. 
% The bit field mask of the editing space is filled so that the editing space can be fully sampled during training. 
Starting from a pretrained NeRF model, we perform 50-100 epochs of local pretraining (for about 0.5-1 seconds) and about 50 epochs of global finetuning (for about 40-60 seconds). The number of epochs and time consumption can be adjusted according to the editing type and the complexity of the scene. Note that we test our performance in the absence of tiny-cuda-nn~\cite{tiny-cuda-nn} which achieves superior speed to our backbone, which indicates that our performance has room for further optimization.
% Note that the training speed is evaluated when tiny-cuda-nn is not enabled.

\paragraph{Datasets.}
We evaluate our editing in the synthetic\Skip{lego, chair, and ship from} NeRF Blender Dataset~\cite{mildenhall2020nerf}, and the real-world captured \Skip{family and truck from}Tanks and Temples~\cite{Knapitsch2017} and \Skip{, and scan83 from} DTU~\cite{jensen2014large} datasets. We follow the official dataset split of the frames for the training and evaluation.


% Figure environment removed

% Figure environment removed

% Figure environment removed

\subsection{Experimental Results}
\label{sec-results}
% \paragraph{Comparisons of rendering quality between teacher and student network.} 


\paragraph{Qualitative NeRF editing results.} 
We provide extensive experimental results in all kinds of editing categories we design, including bounding shape (\cref{fig-bbox,fig-bbox-elf}), brushing (\cref{fig-brush}), anchor (\cref{fig-anchor}), and color (\cref{fig-teaser}). Our method not only achieves a huge performance boost, supporting instant preview at the second level but also produces more visually realistic editing appearances, such as shading effects on the lifted side in \cref{fig-brush} and shadows on the bumped surface in \cref{fig-neumesh}. Besides, results produced by the student network can even outperform the teacher labels, \eg in \cref{fig-bbox-elf} the $F^t$ output contains floating artifacts due to view inconsistency. As analyzed in \cref{sec-train}, the distillation process manages to eliminate this. We also provide an example of object transfer (\cref{fig-bbox-baby}): the bulb in the Lego scene (of Blender dataset) is transferred to the child's head in the family scene of Tanks and Temples dataset.
% \Skip{
% We evaluate our method on all the editing types we design, \ie bounding shape, brushing and anchor, respectively:
% \begin{itemize}
%     \item Bounding shape editing. As shown in \cref{fig-bbox}, we scale the warning light on the top of the Lego model, shorten the chair leg, \zjs{TBD}, and provides plausible results.
%     \item Brushing and color editing. As shown in \cref{fig-brush}, our method edits the scene according to the user's paintings (\ie a cross sign on the chair back, a heart shape on the car logo, and \zjs{TBD}). Note that our brushing method supports simultaneous geometry lifting, as shown in the ``cross'' example. Due to our shading preservation strategy in HSL space, the edited surface can contain realistic visual effects (see the shading effects of the lifted surface).
%     \item Anchor editing. As shown in \cref{fig-anchor}, our method edits the scene according to the anchor points (\ie ship's bow, bulldozer's shovel and \zjs{TBD}) and the stretching direction. The edited geometry has consistent appearance with the anchored area.
% \end{itemize}
% }

% Figure environment removed

% Figure environment removed

\paragraph{Comparisons to baselines.} Existing works have strong restrictions on editing types, which focus on either geometry editing or appearance editing, while ours is capable of doing both simultaneously. Our brushing and anchor tools can create user-guided out-of-proxy geometry structures, which no existing methods support. We make comparisons on color and texture painting supported by NeuMesh~\cite{neumesh} and Liu \etal~\cite{liu2021editing}. 

\cref{fig-neumesh} illustrates two comparisons between our method and NeuMesh~\cite{neumesh} in scribbling and a texture painting task. Our method significantly outperforms NeuMesh, which contains noticeable color bias and artifacts in the results. In contrast, our method even succeeds in rendering the shadow effects caused by geometric bumps.

\cref{fig-neumesh-mic} illustrates the results of the same non-rigid blending applied to the Mic from NeRF Blender\cite{mildenhall2020nerf}. It clearly shows that being mesh-free, We have more details than NeuMesh\cite{neumesh}, unlimited by mesh resolution.

\cref{fig-editnerf} shows an overview of the pixel-wise editing ability of existing NeRF editing methods and ours. Liu \etal~\cite{liu2021editing}'s method does not focus on the pixel-wise editing task and only supports textureless simple objects in their paper. Their method causes an overall color deterioration within the edited object, which is highly unfavorable. This is because their latent code only models the global color feature of the scene instead of fine-grained local features. Our method supports fine-grained local edits due to our local-aware embedding grids.

% \yq{describe the difference}

% \paragraph{Artistic applications (a comic on NeRF).} Based on the four example tools we implemented, we created a comic \textit{Bob the Bulb} (Fig. \ref{fig-comic}) to show the potential applications of our editing method. This might be the first artwork created with NeRF.

% \subsection{Experiments on Bounding Shape Editing}
% \subsection{Experiments on Brush Editing}
% \subsection{Experiments on Anchor Editing}
% \subsection{Experiments on Color Shape Editing}
% \subsection{Real-world Example: NeRF-Rendered Comic}


\subsection{Ablation Studies}
\label{sec-ablation}
\paragraph{Effect of the two-stage training strategy.} To validate the effectiveness of our pretraining and finetuning strategy, we make comparisons between our full strategy (3\textsuperscript{rd} row), finetuning-only (1\textsuperscript{st} row) and pretraining-only (2\textsuperscript{nd} row) in \cref{fig-ab_pre}. Our pretraining can produce a coarse result in only 1 second, while photometric finetuning can hardly change the appearance in such a short period. The pretraining stage also enhances the subsequent finetuning, in 30 seconds our full strategy produces a more complete result. However, pretraining has a side effect of local overfitting and global degradation. Therefore, our two-stage strategy makes a good balance between both and produces optimal results.

\paragraph{MLP fixing in the pretraining stage.} In \cref{fig-ab_fix}, we validate our design of fixing all MLP parameters in the pretraining stage. The result confirms our analysis that MLP mainly contains global information so it leads to global degeneration when MLP decoders are not fixed.


We proposed a machine-learning based method to approximate diagonal as well as non-diagonal elements of the Hessian of a molecule. The representation used is specific for every internal coordinates, and takes explicitly into account the bond order, which is sensible because a single point DFT calculation is computationally considerably less expensive that the explicit calculation of the Hessian.
We trained our ML model on a relatively small dataset (subset of QM7) of less than 7000 molecules. The Hessian was computed at the B3LYP/cc-pVDZ level of theory. 
The agreement between ML and DFT was satisfactory. In particular, the calculated MAPE for bond stretching force constant was below 2\%, and was particularly small for bonds involving hydrogen atoms because they point outwards and are less affected by the chemical environment. The MAPE for bending and torsion was of 5\% and 10\%, respectively. 
From the ML model trained on QM7 we were also able to predict the Hessian of some molecules representative of the QM9 dataset. The Hessian predicted in internal coordinates was then transformed into the mass-weighted Cartesian Hessian, the diagonalization of which yields the harmonic vibrational frequencies and normal modes, that can be compared to the ones calculated  explicitly from DFT.

High frequency vibrations and normal modes were predicted accurately, while lower frequency ones were not. This behaviour is analogous to the IR spectroscopy theory, where stretchings and bendings can be identified accurately, while torsion and delocalized vibrations are more difficult to be interpreted.

The approximate Hessian obtained with ML is computational inexpensive and can be used as an initial Hessian guess for geometry optimization, or in the context of alchemical geometry relaxation \cite{Domenichini2020,domenichini2022alchemical, shiraogawa2022exploration,shiraogawa2023optimization}. 
A good starting Hessian may speed up the convergence of the geometrical optimization. An in detail assessment of the performance of the ML Hessian proposed is not yet provided, but should carefully take into account many parameters on which the optimization depends, \textit{e.g.} the type of molecule, the initial geometry, the optimization algorithm, and the Hessian update scheme.



\section*{Acknowledgement}

This work has been funded by The Scientific and Technological Research Council of Turkey (TUBITAK), 3501 Research Project under Grant
No 121E097. 




% %%%%%%%%% BODY TEXT
% In this supplementary document, we provide:

% \begin{enumerate}
%   \item Architecture and training details of the two-stage framework we propose.
%   \item Visual comparison with pSp \cite{richardson2021encoding}, HFGI \cite{wang2022high}, HyperStyle \cite{alaluf2022hyperstyle}, CoModGAN \cite{zhao2021large}, InvertFill \cite{yu2022high} and DualPath \cite{wang2022dual}. 
%   \item Visual inpainting results on AFHQ Cat and Dog datasets.
%   \item Semantic editing results on FFHQ dataset.
% \end{enumerate}

\appendix






\section{Architecture Details}
\label{sec:training}


The final architecture is given in Fig. \ref{fig:sup_overall}. We follow a two-stage training pipeline. In the first stage, we train the Encoder and Mixing network. The architectures of them are as follows:

\textbf{Encoder ($E$).}
We adopt the encoder architecture from pSp \cite{richardson2021encoding} with minor modifications. First, we increase the first layer input channel number from $3$ to $4$ for taking the mask as an additional input. Then, we disable the normalization layers since we observe they decrease the performance of the model given that many input pixels may be $0$ due to removal of them.

\textbf{Mixing Network ($Mi$).}
We equip the mixing network with a neural network and gating mechanism.

% \begin{equation}
% \label{eq:gating}
% \begin{split}
%     \text{W}^{comb}, g = \text{NN}( \text{W}^{enc} ,  \text{W}^{rand} )\\ 
%     \text{W}^{out} = \sigma(g) \cdot \text{W}^{comb} + (1-\sigma(g)) \cdot \text{W}^{rand} 
% \end{split}
% \end{equation}

The dimensions of $\text{W}^{enc}$ and $\text{W}^{rand}$ are both $14\times512$.
For the neural network $\text{NN}$, we use $14$ fully connected layers.
Each of them takes a style vector from $\text{W}^{enc}$ and $\text{W}^{rand}$ that are in $1\times512$ dimension. Fully connected layers have input and output dimensions of $512$.


% Figure environment removed

In the second stage training, we set a new encoder which we refer to as Skip Encoder (S) in Fig. \ref{fig:sup_overall}. In the second stage training, we set the first Encoder frozen. We are interested in learning high-rate features and feed them to StyleGAN generator to achieve better fidelity to input image. 
The architecture is as follows: 

\textbf{Skip Encoder ($S$).}
The Skip Encoder takes input from the final output of the first stage model. We additionally feed the mask and erased input image to the Skip Encoder ($S$). 
They are concatenated and are fed to the $S$.
% There are skip connections between S and G at $32$, $64$ and $128$ resolutions. 
$S$ starts with a convolution layer to increase the channel size from $7$ to $32$ with a filter size of $3\times3$ and padding of $1$. The $32\times 256 \times 256$ feature maps are fed into residual blocks.
The residual blocks consist of three residual layers. Each residual layer consists of two convolution layers with batch normalization and parametric ReLu activation.
Each residual block downsamples the input resolution to half in its first residual layer using max pooling layer.
At each block, the channel size increases.
The Skip Encoder decreases the resolution to $32\times32$ at the end via 3 residual blocks.
The channel size at each block are as follows $48$, $64$, $96$ in the downsampling residual layers, respectively.
After we extract the $32$, $64$, and $128$ resolution feature maps, we pass them on $2$ more convolution blocks to retrieve skip connection addition ($G_{add}$) and multiplication ($G_{mult}$) maps, whose channels are compatible with the StyleGAN at respecting resolution. We do not have an activation function for $G_{add}$, but we have a sigmoid function for extracting $G_{mult}$. Lastly, StyleGAN generator features ($G_f$) are changed as follows:
\begin{equation}
\label{eq:skip}
    G_f = G_f + G_f * G_{mult} + G_{add}
\end{equation}




\section{Training Details.} We train the first stage for $500$k iterations with batch size of $8$ on two GPUs. We use learning rate of $1 \times 10^{-4}$ for all networks. We halve the learning rates at each $50$k iterations. We use the overall objectives given below to optimize the parameters of the Encoder ($E$) and the mixing network ($Mi$) as was also given in the main paper. 
We also use the same objective to optimize the parameters of $S$ in the second stage. 


For both of the training stages, we use the same objective given in Eq. \ref{eqn:full_loss} and following hyperparameters, $\lambda_{a}=8\times10^{-2}$, $\lambda_{r1}=1$, and $\lambda_{r2}=1$.  
We set the pixel-wise and VGG reconstruction loss coefficients as $1$ and $5\times10^{-5}$, respectively for both ${L}_{rg}$ and ${L}_{gg}$.



% \section{Additional Results}
% \label{sec:results}

% We provide visual results for:
% \begin{enumerate}
%     \item FFHQ inpainting results in Fig.  \ref{fig:face_comparison_supp}.
%     \item FFHQ inpainting and editing results in  Fig. \ref{fig:editing_results_supp}.
%     \item AFHQ Cat and Dog inpainting results in Fig. \ref{fig:cat_sup}.
% \end{enumerate}

% \newcommand{\interpfigt}[1]{% Figure removed}

% % Figure environment removed

% \newcommand{\interpfigtS}[1]{% Figure removed}

% % Figure environment removed




% comparison images
% smile removal
% % Figure environment removed




























{\small
\bibliographystyle{ieee_fullname}
\bibliography{egbib}
}

%\newpage
%\input{6_supp}

\end{document}