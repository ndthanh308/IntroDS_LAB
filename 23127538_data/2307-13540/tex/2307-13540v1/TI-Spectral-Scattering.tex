\documentclass[11pt]{article}

%\usepackage[notcite,notref]{showkeys}
%\usepackage[notcite]{showkeys}

\usepackage{amsmath,amssymb,graphicx,mathtools}
\numberwithin{equation}{section}
%load extra symbols and environments
\usepackage{amsthm}
\usepackage{relsize}
\usepackage[margin=1in]{geometry} %set margins
\usepackage{enumerate}
\usepackage{yhmath}
\usepackage{tikz}
\usepackage{hyperref}
\usepackage[flushleft]{threeparttable}
\usepackage{array,booktabs,makecell}

%\usepackage[margin=1.0in]{geometry}


%%%%%%%%%%%%%%%%
%%% COMMANDS %%%
%%%%%%%%%%%%%%%%
\def\@abssec#1{\vspace{.05in}\footnotesize \parindent .2in
{\bf #1. }\ignorespaces}
%proof
\def\proof{\par{\it Proof}. \ignorespaces}
\def\endproof{{\ \vbox{\hrule\hbox{%
   \vrule height1.3ex\hskip0.8ex\vrule}\hrule
  }}\par}
\graphicspath{{/EPSF/}{Figures/}}
\DeclareGraphicsExtensions{.eps}
%\DeclareGraphicsExtensions{.pdf,.jpg}

\providecommand{\abs}[1]{\lvert#1\rvert}
\providecommand{\norm}[1]{\lVert#1\rVert}

\newtheorem{theorem}{Theorem}[section]
\newtheorem{lemma}[theorem]{Lemma}
\newtheorem{proposition}[theorem]{Proposition}
\newtheorem{corollary}[theorem]{Corollary}
\newtheorem{definition}[theorem]{Definition}
\newtheorem{remark}[theorem]{Remark}
\newtheorem{hypothesis}[theorem]{Hypothesis}
\newtheorem{notation}[theorem]{Notation}
\newtheorem{assumption}[theorem]{Assumption}


\newcommand{\gb}[1]{{{\color{blue}{GB: #1}}}}
\newcommand{\bbb}[1]{{{\color{red}{BC: #1}}}}

\def \Rm {\mathbb R}
\def \Nm {\mathbb N}
\def \Cm {\mathbb C}
\def \Zm {\mathbb Z}
\def \Sm {\mathbb S}
\def \Tm {\mathbb T}
\newcommand{\eps}{\varepsilon}
\newcommand{\E}{\mathbb E}
\newcommand{\dsum}{\displaystyle\sum}
\newcommand{\dint}{\displaystyle\int}

%\newcommand{\dfrac}{\displaystyle\frac}
\newcommand{\pdr}[2]{\dfrac{\partial{#1}}{\partial{#2}}}
\newcommand{\pdrr}[2]{\dfrac{\partial^2{#1}}{\partial{#2}^2}}
\newcommand{\pdrt}[3]{\dfrac{\partial^2{#1}}{\partial{#2}{\partial{#3}}}}
\newcommand{\dr}[2]{\dfrac{d{#1}}{d{#2}}}
\newcommand{\drr}[2]{\dfrac{d^2{#1}}{d{#2}^2}}
\newcommand{\aver}[1]{\langle {#1} \rangle}
\newcommand{\Baver}[1]{\Big\langle {#1} \Big\rangle}



\newcommand{\bb}{\mathbf b} \newcommand{\bc}{\mathbf c}
\newcommand{\be}{\mathbf e} \newcommand{\bh}{\mathbf h}
\newcommand{\bk}{\mathbf k}
\newcommand{\bl}{\mathbf l} \newcommand{\bm}{\mathbf m}
\newcommand{\bp}{\mathbf p} \newcommand{\bq}{\mathbf q}
\newcommand{\bu}{\mathbf u} \newcommand{\bv}{\mathbf v}
\newcommand{\bx}{\mathbf x} \newcommand{\by}{\mathbf y}
\newcommand{\bz}{\mathbf z} 
\newcommand{\bD}{\mathbf D} \newcommand{\bF}{\mathbf F}
\newcommand{\bG}{\mathbf G} \newcommand{\bH}{\mathbf H}
\newcommand{\bK}{\mathbf K} \newcommand{\bL}{\mathbf L}
\newcommand{\bP}{\mathbf P} \newcommand{\bQ}{\mathbf Q}
\newcommand{\bX}{\mathbf X} \newcommand{\bY}{\mathbf Y} 
\newcommand{\bZ}{\mathbf Z} 
\newcommand{\baX}{\bar X}
\newcommand{\tu}{{\tilde u}}
\newcommand{\ue}{{u_\eps}}
\newcommand{\mA}{\mathcal A}
\newcommand{\mB}{\mathcal B}
\newcommand{\mC}{\mathcal C}
\newcommand{\mF}{\mathcal F}
\newcommand{\mH}{\mathcal H}
\newcommand{\mI}{\mathcal I}
\newcommand{\mK}{\mathcal K}
\newcommand{\mL}{\mathcal L}
\newcommand{\mM}{\mathcal M}
\newcommand{\mN}{\mathcal N}
\newcommand{\mP}{\mathcal P}
\newcommand{\mT}{\mathcal T}
\newcommand{\mV}{\mathcal V}
\newcommand{\rT}{{\rm T}}
\newcommand{\h}{\mathfrak h}
\newcommand{\mk}{\mathfrak k}
\newcommand{\mn}{\mathfrak n}
\newcommand{\mr}{\mathfrak r}
\newcommand{\w}{\mathfrak w}
\newcommand{\mD}{\mathfrak D}
\newcommand{\fH}{\mathfrak H}
\newcommand{\mJ}{{\mathfrak J}}
\newcommand{\fa}{{\mathfrak a}}
\newcommand{\fc}{{\mathfrak c}}
\renewcommand{\d}{\mathfrak d}
\newcommand{\mf}{{\mathfrak f}}
\newcommand{\mg}{{\mathfrak g}}
\newcommand{\mh}{{\mathfrak h}}
\renewcommand{\k}{\mathfrak k}
\newcommand{\fs}{{\mathfrak s}}
\newcommand{\fS}{{\mathfrak S}}
\newcommand{\rH}{{\rm H}} \newcommand{\rI}{{\rm I}} \newcommand{\rJ}{{\rm J}}


\newcommand{\ind}{{\rm Index\,}}
\newcommand{\vp}{\varphi}
\newcommand{\dvp}{\dot\vp}
\newcommand{\dt}{{\rm det}}
\newcommand{\Pp[1]}{{P_{#1}}}
\newcommand{\pP[1]}{{\partial_{P_{#1}}}}
\newcommand{\bPp[1]}{{\bar P_{#1}}}
\newcommand{\pbP[1]}{{\partial_{\bar P_{#1}}}}

\newcommand{\bzero}{\mathbf 0}
\newcommand{\bGamma}{\boldsymbol\Gamma}
\newcommand{\btheta}{\boldsymbol \theta}
\newcommand{\bchi}{\boldsymbol \chi}
\newcommand{\bnu}{\boldsymbol \nu}
\newcommand{\brho}{\boldsymbol \rho}
\newcommand{\bxi}{\boldsymbol \xi}
\newcommand{\bnabla}{\boldsymbol \nabla}
\newcommand{\bOm}{\boldsymbol \Omega}
\newcommand{\bomega}{\boldsymbol \omega}
\newcommand{\cd}[2]{{#1}\!\cdot\!{#2}}
\newcommand{\cout}[1]{}
\newcommand{\cin}[1]{#1}
\newcommand{\cO}{{\mathcal O}}
\newcommand{\cS}{{\mathcal S}}

\newcommand{\dX}{\partial X}
\newcommand{\X}{\tilde X}
\newcommand{\rME}{{\rm M}(E)}

%%%% NOTATION SYMBOLS %%%%%%%%%%%%%%%%%%%%%

%\newcommand{\ind}{{\rm Index\,}}
\newcommand{\sgn}[1]{\,{\rm sign}(#1)}
\newcommand{\bsgn}[1]{\,{\rm sign}\big(#1\big)}
\newcommand{\Bsgn}[1]{\,{\rm sign}\Big(#1\Big)}
\newcommand{\sign}{{\rm sign}}
\newcommand{\ow}{{\rm Op}^w}
\newcommand{\R}{{\rm R}}
\renewcommand{\H}{{\rm H}}
\newcommand{\Tr}{{\rm Tr}}
\newcommand{\Ker}{{\rm Ker}}
\newcommand{\Ran}{{\rm Ran}}
%\newcommand{\dim}{{\rm dim}}
\newcommand{\trr}{{\rm tr}}
\newcommand{\ket}[1]{|#1\rangle}
\newcommand{\bra}[1]{\langle#1|}
%\newcommand{\deg}{{\rm deg}} %%% already defined
\newcommand{\tdeg}{\widetilde{\deg\, }}

\newcommand{\umo}{\"o}
\newcommand{\fco}{\Xi}


\newcommand{\tb}[1]{\textcolor{blue}{#1}}
\newcommand{\tr}[1]{\textcolor{red}{#1}}
\newcommand{\tg}[1]{\textcolor{red}{#1}}
\newcommand{\tm}[1]{\textcolor{magenta}{#1}}
\newcommand{\tw}[1]{\textcolor{white}{#1}}
\newcommand{\chk}{\tm{~[~Check~]~}}
\newcommand{\khc}{\tb{~[~Checked~]~}}
\newcommand{\ccc}{\tr{~[~Wrong~]~}}
%\addtolength\oddsidemargin{-2.2cm}
%\addtolength\evensidemargin{-2.2cm}
%\addtolength\textwidth{4.3cm}
%\addtolength\topmargin{-2.5cm}
%\addtolength\textheight{3.cm}
\renewcommand{\baselinestretch}{1} \renewcommand{\arraystretch}{1.5}

%%%%%%%%%%%%%
%%% TITLE %%%
%%%%%%%%%%%%%
\title{Scattering theory of topologically protected edge transport}

%%%%%%%%%%%%%%%
%%% AUTHORS %%%
%%%%%%%%%%%%%%%
\author{ Binglu Chen \thanks{Department of Mathematics, University of Chicago, Chicago, IL 60637; {\tt blchen@uchicago.edu}} \and  Guillaume Bal \thanks{Departments of Statistics and Mathematics and CCAM, University of Chicago, Chicago, IL 60637; guillaumebal@uchicago.edu}  }

%%%%%%%%%%%%%%%%%%%%%%
%%% BEGIN DOCUMENT %%%
%%%%%%%%%%%%%%%%%%%%%%
\begin{document}
 
\maketitle
%\tableofcontents

\begin{abstract}
 This paper develops a scattering theory for the asymmetric transport observed at interfaces separating two-dimensional topological insulators. Starting from the spectral decomposition of an unperturbed confining Hamiltonian, we present a limiting absorption principle and construct a generalized eigenfunction expansion for perturbed systems. We then relate the interface conductivity, a current observable quantifying the transport asymmetry, to the scattering matrix associated to the generalized eigenfunctions. In particular, we show that the interface conductivity is concretely expressed as a difference of transmission coefficients and is stable against perturbations.  We apply the theory to  systems of perturbed Dirac equations with asymptotically linear domain wall. 
\end{abstract}
 
%\begin{AMS}
%\end{AMS}

\renewcommand{\thefootnote}{\fnsymbol{footnote}}
\renewcommand{\thefootnote}{\arabic{footnote}}

%\jot 0.2 cm
\renewcommand{\arraystretch}{1.1}

%\begin{keywords}
%\end{keywords}

\noindent{\bf Keywords:} Scattering theory, spectral theory, topological insulators, asymmetric edge transport, edge conductivity.

%\begin{AMS}
%\end{AMS}

%\pagestyle{myheadings}
%\thispagestyle{plain}

%%%%%%%%%%%%%%%%%%%%%%
%%% BEGINNING TEXT %%%
%%%%%%%%%%%%%%%%%%%%%%


%%%%%%%%%%%%%%%%%%%%%%
%%% BEGINNING TEXT %%%
%%%%%%%%%%%%%%%%%%%%%%

%
%%
%%%
\section{Introduction}
%%%
%

A characteristic feature of two-dimensional topological insulators is the topologically protected asymmetric transport observed at one-dimensional interfaces separating two insulating bulks. Applications may be found in many areas in condensed matter physics, photonics, and geophysical sciences  \cite{BH, delplace, sato, Volovik, Witten}. 

If $H$ is a Hamiltonian describing transport in the two-dimensional system, the asymmetry along the edge is modeled by the following edge conductivity
\begin{equation}\label{eq:edgeconductivity}
  \sigma_I[H] = {\rm Tr} \, i[H,P]\varphi'(H)
\end{equation}
where $P=P(x)\in\fS[0,1]$ and $\varphi(E)\in \fS[0,1,E_-,E_+]$  are smooth switch functions. Here, $\fS[a,b,c,d]$ is the set of bounded (measurable) functions on $\Rm$ equal to $a$ for $x<c$ and equal to $b$ for $x>d$ while $\fS[a,b]$ is their union over (finite) $c<d$.  The operator $i[H,P]$ may be interpreted as a current operator while $0\leq \varphi'(E)$ is a density of states. Thus $\sigma_I$ models the expected value of the current operator for excitations in the system with density $\varphi'(E)$ supported in an energy interval $[E_-,E_+]$ where propagation into the bulk is suppressed. In this paper, we consider the setting where no energy $E\in\Rm$ is allowed to propagate in the bulk, so that $[E_-,E_+]$ is an arbitrary (bounded) interval in $\Rm$. The interface conductivity has been used in a variety of contexts \cite{BH, Drouot, Elbau, elgart2005equality, SB-2000}. See \cite{2, 3, bal2023topological,quinn2021approximations} for an analysis of \eqref{eq:edgeconductivity} when $H$ is a differential operator on $\Rm^2$, which is the setting we consider in this  paper. 

In such a setting, an interface current is flowing in the direction of the $x-$axis while wave-fields are concentrated in the vicinity of $y=0$. We have $2\pi\sigma_I\in \Zm$ an integer describing excitations primarily moving e.g. from left to right when $2\pi\sigma_I>0$.
We may then envision the following scattering experiment. When $H=H_0$ is an unperturbed operator invariant with respect to spatial translations in $x$, then plane waves may be identified as generalized eigenfunctions of $H_0$. We will show how $\sigma_I$ may be expressed in terms of such eigenfunctions following \cite{bal2023asymmetric}.  The topological protection of the asymmetric transport states \cite{2, 3,quinn2021approximations} that $\sigma_I[H_0]=\sigma_I[H_0+Q]$ for a large class of perturbations $Q$. The asymmetric transport may thus be interpreted as an obstruction to Anderson localization \cite{1,PS}: no matter how large $Q$ is (so long as it is localized spatially), some transmission is guaranteed by the non-vanishing current $2\pi\sigma_I\not=0$.
The main objective of this paper is to devise a scattering theory for $H=H_0+Q$ with $H_0$ an operator with a well-known spectral decomposition and $Q$ a short-range perturbation, which for us will be an operator of point-wise multiplication by $Q(x,y)$, a function that decays sufficiently rapidly to $0$ as $|x|\to\infty$. More precisely, for an energy $E\in\Rm$ within the bulk band gap, we wish to show the existence of generalized plane waves solution of $H\psi_m=E\psi_m$ and construct a scattering matrix $S$ from such functions $\psi_m$. Our final objective is then to show $2\pi\sigma_I\in\Zm$ is directly related to the coefficients of the scattering matrix $S$, see Theorem \ref{thm:sigmascattering} below.  %For concreteness and to slightly simplify notation, we will often assume that $H$ is a first-order system of differential operators. There is no theoretical obstruction to considering general (elliptic) differential operators as in \cite{bal2023topological,quinn2021approximations}. 


Section \ref{sec:current} presents our main framework. Under the assumption that generalized eigenfunctions associated to the problem $H\psi=E\psi$ exist and satisfy a priori constraints, we show that a unitary scattering matrix may be defined and that the edge conductivity \eqref{eq:edgeconductivity} can indeed be computed from the scattering coefficients. This justifies computations performed in \cite{bal2023asymmetric} for systems of Dirac equations.

The rest of the paper is devoted to providing sufficient conditions for the theory of section \ref{sec:current} to apply. This is done in section \ref{sec:lap} by appealing to the spectral theorem to obtain an appropriate decomposition of $H$, and in particular to a limiting absorption principle to obtain a detailed description of the absolutely continuous spectrum and point spectrum (and lack of singular continuous spectrum) of $H$. The construction of generalized eigenfunctions for the perturbed system is given in section \ref{sec:eigexp}. Finally, section \ref{sec:Dirac} verifies all required hypotheses for slight generalizations of the systems of Dirac operators analyzed numerically in \cite{bal2023asymmetric}.

References on scattering theory, the limiting absorption principle, and generalized eigenfunction expansions that are relevant to the current work include \cite{ASNSP_1975_4_2_2_151_0,H-II-SP-83,ikebe1960eigenfunction,kato2013perturbation,RS4,simon1982schrodinger,yamada1975eigenfunction}.


%The conditions of application of the spectral theory are verified in section \ref{sec:eigexp} for the systems of Dirac operators analyzed numerically in \cite{bal2023asymmetric}. Estimates on generalized eigenfunctions and eigenfunction expansions are presented in section \ref{sec:eigexp} and used in section \ref{sec:scattering} to relate the conductivity $\sigma_I$ to the generalized eigenfunctions.


%
%%
%%%
\section{Current conservation and edge conductivity}\label{sec:current}
%%%
%
This section proposes a framework to relate the asymmetric transport modeled by the edge conductivity to spectral information on the Hamiltonian describing the system. Section \ref{sec:assumptions} summarizes our main assumptions while section \ref{sec:currentcons} introduces a current correlation and defines a notion of current conservation and section \ref{sec:scatteringmatrix} finally relates the edge conductivity to a scattering matrix associated to the Hamiltonian.

%
\subsection{Assumptions on spectral decomposition of Hamiltonian}\label{sec:assumptions}
%
Let $\mH=L^2(\Rm^2)\otimes\Cm^q$ the space of vector-valued functions with square-integrable entries defined on the Euclidean plane with coordinates $(x,y)$. We define the following functional spaces.
\begin{definition}\label{def:LHs}
For $s\in\mathbb{R}$, we define $L^2_{s}$ the weighted Hilbert space of all complex valued functions $u(x,y)$ defined in $\mathbb{R}^2$ such that $\aver{x}^su(x,y)\in L^2(\mathbb{R}^2)$ with the norm
\begin{gather*}
\|u\|_{L^2_{s}}=\Big(\int_{\mathbb{R}^2}\aver{x}^{2s}|u(x,y)|^2dxdy\Big)^{\frac{1}{2}}.
\end{gather*}
Let $\beta\geq0$. For $p\in\mathbb{N}_*$ and $s\in\mathbb{R}$, $H^p_s$ denote the Hilbert space of $L^2_{s}$ functions with distribution derivatives in $L^2_{s}$ up to $p$-th order, with norm given by
\begin{gather}\label{eq:Hsp}
\|u\|_{H_s^p}=\Big(\int_{\mathbb{R}^2}\Big[\aver{x}^{2s}\aver{y}^{2\beta}|u|^2+\aver{x}^{2s}\sum_{|\alpha|= p}|D^\alpha u|^2\Big]dxdy\Big)^{\frac{1}{2}}.
\end{gather}
\end{definition}
%\bbb{Is $|\alpha|\leq p ?$}
%% THESE ARE EQUIVALENT DEFINITIONS. THE DERIVATIVES OF ORDER 0 and p control all intermediate derivatives (this is clear in Fourier domain )
Above, $\aver{x}:=\sqrt{1+x^2}$.
We denote $H^p = H^p_0$.
We also denote by $H^p_s$ ($L^2_s$) the space of vector-valued functions $H^p_s\otimes \Cm^q$ ($L^2_s\otimes\Cm^q$).
%For vector-valued functions, we assume that each component belongs to the above spaces.  
The value of $\beta$ will be equal to $p$ in our applications.

We start with a self-adjoint operator $H_0$ from $\mD(H_0)=H^p$ to $\mH$ that is invariant with respect to translations in $x$. We then have $H_0=\mF_{\xi\to x}^{-1} \hat H_0(\xi) \mF_{x\to\xi}$ with $\mF_{x\to\xi}$ Fourier transform in the first variable $x$ and $\Rm\ni \xi\mapsto \hat H_0(\xi)$ a family of self-adjoint operators on $L^2(\Rm)\otimes\Cm^q$. 
\begin{hypothesis}[${\rm [H1]}$] \label{hyp:H1}
%We assume that $H_0$ is an elliptic operator, in the sense that for each compact $K\subset\Rm^2$, we have
%\begin{equation}\label{eq:ellipticK}
%    \|\psi\|_{H^m(K)} \leq C(K) \big( %\|H_0\psi\|_{L^2(K)} + \|\psi\|_{L^2(K)}.
%\end{equation}
(o) We assume that $H_0$ is a self-adjoint (elliptic) differential operator with domain $H^p\otimes \Cm^q$ and resolvent operator $R_0(i)=(H_0-i)^{-1}$ bounded from $\mH$ to $H^p\otimes \Cm^q$.
\\[2mm]
(i) For each $\xi\in\Rm$, $\hat H_0(\xi)$ has a compact resolvent and hence purely discrete spectrum. We assume the existence of generalized eigenfunctions in $H^p_{-s}$ for $s>\frac12$, solutions
\begin{equation}\label{eq:unperturbedxi}
 \psi_j(x,y;\xi) = \frac{1}{\sqrt{2\pi}}  e^{i\xi x} \phi_j(y;\xi),
\end{equation}
of the eigenvalue problem $(H_0-E_j(\xi))\psi_j=0$ with $(\phi_j)_j$ an orthonormal basis of $L^2(\Rm_y)\otimes\Cm^q$, i.e., $(\phi_j,\phi_k)_{L^2(\Rm_y)\otimes\Cm^q}=\delta_{jk}$. Here, $j\in J$ with $J\simeq\Nm$. 
\\[2mm]
(ii) We assume that the branches of absolutely continuous spectrum $j\to E_j(\xi)$ are smooth and satisfy $|E_j(\xi)|\to\infty$ as $|\xi|\to\infty$ with $\xi\to (1+|E_j(\xi)|^2)^{-1}$ integrable for $j\in J$. We assume that for any interval $[a,b]$, only a finite number of branches $\xi \to E_j(\xi)$ cross $[a,b]$. 
%\bbb{is $\xi\to E_j(\xi)$ ?}
%%%Thus, they cross an interval $[E_-,E_+]$ a finite number of times and for a bounded set of wavenumbers $\xi$.  
\end{hypothesis}

For $H_0$ an elliptic operator of order $p$, which is the framework we are interested in, standard ellipticity results show that $\psi_j(x,y;\xi)$ defined in \eqref{eq:unperturbedxi} is an element in $H^p_{-s}$ for $s>\frac12$. We have by assumption the spectral decomposition
\begin{equation}\label{eq:spectraldecH0}
  \qquad   H_0  = \dsum_j \dint_{\Rm} E_j(\xi) \Pi_j(\xi) d\xi,\qquad \Pi_j(\xi) = \psi_j(\cdot;\xi) \otimes \psi_j(\cdot;\xi),
\end{equation}
%where $\Pi_{[E_-,E_+]}[H_0]$ projects $H_0$ onto the interval $[E_-,E_+]$ and 
where $\Pi_j(\xi)$ are rank-one projectors. Associated to the above decomposition is the following resolution of identity. Let $\fco=(j,\xi)\in J\times \Rm$. We define for $f\in L^2(\Rm^2)\otimes\Cm^q$ the (unperturbed) Fourier transform:
\begin{equation}\label{eq:FT}
    \hat f (\Xi) =  (\mF f)(\Xi) := \dint_{\Rm^2} \overline{\psi_j(x,y,\xi)} \cdot f(x,y) dxdy = (f,\psi_j(\cdot,\xi)),
\end{equation}
where $(f,g)=\int_{\Rm^2} f(x,y)\cdot \bar g(x,y) dxdy$ is the inner product on $\mH$,
with inverse Fourier transform:
\begin{equation}\label{eq:IFT}
    f (x,y) =  (\mF^{-1} \hat f)(x,y) := \dsum_j \dint_{\Rm} \hat f(\Xi)  \psi_j(x,y,\xi) d\xi.
\end{equation}
The Fourier transform is an isometry from $\mH=L^2(\Rm^2,dxdy)\otimes\Cm^q$ to $L^2(J\times \mathbb{R},d\fco;\Cm)$, with $d\fco$ the Cartesian product of the counting measure on $J$ and the Lebesgue measure on $\Rm$.



%Assuming $H_0$ is an elliptic operator of order $m$, then standard ellipticity results show that for each $\xi$, $\psi_j(x,y;\xi)\in H^m_{-s}$ for $s>\frac12$ and some $\beta\geq0$, where 

\medskip
{\em \noindent (iii)
The spectral elements $E_j(\xi)$ and $\Pi_j(\xi)$ are assumed to be smooth in $\xi$ with a finite number of critical values. Define
\begin{equation}\label{eq:Z}
  Z = \big\{ E \in \Rm ;\  E=E_j(\xi) \mbox{ for some } (j,\xi)\in J\times\Rm \mbox{ and } \partial_\xi E_j(\xi)=0 \big\}.
\end{equation}
We assume the set $Z$ of critical values to be  finite in each bounded interval $[E_-,E_+]$.
\\[2mm]
(iv) To set up a scattering theory, we finally assume the following completeness property: for any $E\in \Rm\backslash Z$, then any solution $\psi\in H^p_{-s}$ of $(H_0-E)\psi=0$ is a linear combination of the generalized eigenfunctions $\psi_j(x,y;\xi)$ for values of $\xi$ such that $E_j(\xi)=E$. We label $\psi_m(x,y)=\psi_j(x,y;\xi_m)$ for $1\leq m\leq \rME$ the corresponding solutions at $E$ fixed.  Up to (obvious) relabeling, we thus have
%\bbb{ $ m\in M(E)?$} \gb{We had several uses for $M$. Here, $\rME$ is now an integer giving the number of propagating modes. Below we have an infinite set $M$ for the Dirac operator and a finite set $M(E)$ of propagating modes. }
\begin{equation}\label{eq:unperturbedE}
  \psi_m(x,y;E) =\frac{1}{\sqrt{2\pi}}  e^{i\xi_m(E) x} \phi_m(y;\xi_m(E)),\qquad 1\leq m\leq \rME.
\end{equation}
} % EM


The main unperturbed operator of interest in this paper is the massive Dirac Hamiltonian
\begin{equation}\label{eq:DiracH0}
  H_0 = D_x\sigma_1+D_y\sigma_2 + m(y)\sigma_3, \qquad \hat H_0(\xi) = \xi \sigma_1+D_y\sigma_2 + m(y)\sigma_3,
\end{equation}
with $\sigma_{1,2,3}$ standard Pauli matrices and $D_a=-i\partial_a$ for $a\in\{x,y\}$ and $m(y)$ a domain wall, which for concreteness, equals $y$ up to a bounded perturbation. Then, $p=\beta=1$ and $q=2$. That the spectral decomposition \eqref{eq:spectraldecH0} and all assumptions in Hypothesis [H1] applies to $H_0$ will be revisited in section \ref{sec:decH0}; see also \cite{bal2023asymmetric}.

A second natural application of the theory developed here is for the Klein-Gordon operator
\begin{equation}\label{eq:KGH0}
  H_0 = D_x^2 + \fa^*\fa ,\qquad \fa=\partial_y + m(y),\quad \fa^*=-\partial_y+m(y)
\end{equation}
with then $p=\beta=2$ and $q=1$. This operator is topologically trivial in the sense that $\sigma_I[H_0+Q]=0$ for $Q$ short range \cite{bal2023topological}. The verification of the main hypotheses [H1]-[H2]-[H3] for that operator is carried out exactly as for the Dirac operator and will not be explored further in this paper. 

The quantization of the interface conductivity $\sigma_I$ for Dirac, Klein Gordon, and more general elliptic operators with unbounded domain walls is carried out in \cite{bal2023topological}; see also \cite{3,quinn2021approximations} for the case of bounded domain walls. In this paper, we obtain the quantization of $\sigma_I$ from knowledge of the spectral decompositions of $H$ and $H_0$; see also \cite{2,3} for related spectral flow calculations.

\medskip

The main objective of this paper is to analyze transport properties for a perturbed operator $H=H_0+Q$ where $Q$ is a short-range operator. We consider the case where $Q$ is an operator of multiplication by $Q(x,y)$ with the $q\times q$-valued (measurable) function $Q(x,y)$ such that $\aver{x}^{1+\eps} |Q|\leq C$ for some $\eps>0$. Here, $\aver{x}=\sqrt{1+x^2}$. We thus assume $Q$ sufficiently rapidly decaying (only) in the $x$ variable.

We then show that for each $\Xi=(j,\xi)$, there exist modified generalized eigenfunctions $\psi^Q_j\in H^p_{-s}$ solution of $H\psi^Q_j=E_j(\xi)\psi^Q_j$. For a fixed $\Rm\ni E\not\in Z$, we still denote by $\psi^Q_m$ the solution of the problem $(H-E)\psi^Q_m=0$ with $\psi^Q_m = \psi^Q_j(\cdot,\xi_m)$ with $E_j(\xi_m)=E$ and $1\leq m\leq \rME$. Moreover, we will justify the following assumption on the spectral decomposition
\begin{equation}[{\rm H2}]\label{eq:spectraldecH}
   \qquad H  = \dsum_n \lambda_n \Pi_n + \dsum_j\dint_{\Rm} E_j(\xi) \Pi^Q_j(\xi) d\xi,\quad \Pi^Q_j(\xi) = \psi^Q_j(\cdot;\xi) \otimes \psi^Q_j(\cdot;\xi).
\end{equation}
Here, the sum over $n$ corresponds to discrete (locally finite) spectrum with eigenvalues $\lambda_n$ and (rank-one) projectors $\Pi_n = \psi_n\otimes \psi_n$ for eigenfunctions $\psi_n\in\mH$.
%Here, $\Pi_{{\rm a.c.} [E_-,E_+]}[H]$ is the spectral projection of $H$ onto the absolutely continuous spectrum restricted to the interval $[E_-,E_+]$.
The above decomposition thus imposes that the branches of absolutely continuous spectrum for $H$ and $H_0$ be the same. This is consistent with the standard result that two operators $H_1$ and $H_2=H_1+V$ with $V$ a trace-class perturbation have unitarily equivalent absolutely continuous spectrum \cite{kato2013perturbation}. While $H_0$ has only absolutely continuous spectrum, the perturbation $H=H_0+V$ may possess discrete point spectrum. The derivation of [H2] in practice comes from obtaining a generalized Fourier transform theory where $\psi_j$  in \eqref{eq:FT} and \eqref{eq:IFT} is replaced by the generalized plane wave $\psi^Q_j$.

A final assumption on the generalized eigenfunctions is that for $|x|$ large, then $\psi^Q_m$ is approximately given by a linear combination of the unperturbed solutions $\psi_n$ for $1\leq n\leq \rME$. More precisely, define the currents
\begin{equation}\label{eq:currents}
   J_m= J_m(E) = \partial_\xi E_m(\xi_m) \not=0
\end{equation}
which do not vanish for $E\not\in Z$ not being a critical value of the branches of absolutely continuous spectrum. We then assume that  for $1\leq m\leq\rME$,
\begin{equation}[{\rm H3}]\label{eq:approxscattering}
  \qquad \psi_m^Q(x,y) \approx  \dsum_{1\leq n\leq \rME} \alpha^\pm_{mn} \psi_n(x,y) =  \dsum_{1\leq n\leq \rME} \alpha^\pm_{mn} \frac{1}{\sqrt{2\pi}} e^{i\xi_n x}  \phi_n(y) ,
\end{equation}
where $a \approx b$ means that the difference $a-b$ converges to $0$ uniformly (in $x$ as a square-integrable function in $y\in\Rm$) as $x\to\pm\infty$ and where $\alpha^\pm$ are the corresponding coefficients in these two limits.

%%%%%%


%%%We want to revisit the derivations below and put them in the right order. The first item is to look at current conservation. 
%
\subsection{Current correlations}\label{sec:currentcons}
%

Let $H$ be a differential self-adjoint operator of order $p$ as described in the preceding section and so that [H2] and [H3] hold, which we assume for the rest of section \ref{sec:current}. Let $\psi_m$ and $\psi_n$ two generalized eigenfunctions in $H^p_{-s}$ for $s>\frac12$, solutions of
\[
  H\psi_m = E_m \psi_m,\quad H\psi_n = E_n \psi_n
\]
with $E_n$ and $E_m$ in $\Rm$. Define the current correlation
\begin{equation}\label{eq:currentcorrelation}
  J_{mn}(x_0) = (\psi_n, 2\pi i[H,P(\cdot-x_0)] \psi_m ) 
\end{equation}
Here, $(\cdot,\cdot)$ denotes the inner product on $\mH$. While $\psi_n\not\in \mH$, the above integral is well-defined since $[H,P(\cdot-x_0)]$ is a differential operator with coefficients that vanish for $x-x_0$ outside of a compact set and hence mapping $H^p_{-s}$ to $L^2_t$ for any $t\in\Rm$. We recall that $P$ is a switch function in $\fS[0,1]$. On that compact set in the $x-$variable,  both $H\psi_m$ and $\psi_n$ are square integrable.

%We are in a situation where such a current is indeed defined since $\psi_{n,m}$ decay in the variable $y$ and $[H,P(x-x_0)]$ is localizing in the $x-$variable. 
\begin{lemma}\label{lem:currentconservation}
 When $E_m=E_n$, we have the current conservation
 \[
   J'_{mn}(x_0) = 0 \quad \mbox{ for all } \quad x_0\in\Rm.
 \]
\end{lemma}
\begin{proof}
   Using that $P'(\cdot-x_0)$ (unlike $P(\cdot-x_0)$) is compactly supported, we obtain that 
   \begin{align*}
     &-J_{mn}'(x_0) = (\psi_n, 2\pi i[H,P'(\cdot-x_0)] \psi_m ) = (\psi_n, 2\pi i (HP'(\cdot-x_0)-P'(\cdot-x_0) H)\psi_m) \\
     = \ & (H\psi_n,  2\pi i P'(\cdot-x_0)\psi_m) - (\psi_n, 2\pi i P'(\cdot-x_0) H\psi_m)  = (E_n-E_m) (\psi_n, 2\pi i P'(\cdot-x_0) \psi_m),
   \end{align*}
   which vanishes.
\end{proof}



Consider $H=H_0+Q$ with $Q$ rapidly decaying at infinity as described in the preceding section. For a fixed energy $E\in \Rm$, the unperturbed solutions of $(H_0-E)\psi=0$ in $H^p_{-s}$ are given by $\psi_m(x,y)$ for $1\leq m\leq \rME$ in \eqref{eq:unperturbedE} while the corresponding perturbed solutions are given by $\psi^Q_m(x,y)$.  The number of propagating modes $\rME$ equals $n_++n_-$ where $n_\pm$ corresponds to the number of currents $\pm J_m>0$ associated to each unperturbed plane wave and defined in \eqref{eq:currents}. From assumption \eqref{eq:approxscattering}, we deduce that
\begin{equation}\label{eq:psimV}
  (\psi_m^Q , 2\pi i[H,P(\cdot-x_0)] \psi_n^Q ) \approx \dsum_{1\leq p,q \leq \rME} \alpha^\pm_{mp}\bar\alpha^\pm_{nq} ( e^{i \xi_p x}\phi_p,  i[H,P(\cdot-x_0)] e^{i \xi_q x}\phi_q)
\end{equation}
where $\approx$ here is the same sense as above but now as $x_0\to\pm\infty$. All we need in the sequel is in fact that \eqref{eq:psimV} holds rather than the more constraining \eqref{eq:approxscattering}. All terms in \eqref{eq:psimV} are again clearly defined since $[H,P(\cdot-x_0)]$ is compactly supported in the $x-$vicinity of $x_0$. We wish to estimate the above right-hand side.
\begin{lemma}\label{lem:unperturbedcurrent}
  Let $P$ be a switch function in $\fS(0,1)$. 
  %such that $\hat P(\xi_m-\xi_n)\not=0$ (with $\hat P$ the Fourier transform of $P-\frac12$) for all $m,n\in M$. Such a property holds for $P$ a Heaviside function.
  %heaviside function $P(x)=\theta(x-x_0)$ 
  Then we have 
  \begin{equation}
    ( e^{i \xi_m x}\phi_m, i[H,P] e^{i \xi_n x}\phi_n) = \delta_{mn} \partial_\xi E_n(\xi_n) = \delta_{mn} J_n.
  \end{equation}
\end{lemma}
\begin{proof}
  For an operator $A$, we denote by $A(x,x')$ its Schwartz kernel.
 Let us first assume $\xi_m\not=\xi_n$. Then
\begin{align*}
  &(  i[H,P] e^{i \xi_m x}\phi_m , e^{i \xi_n x}\phi_n )  = \dint e^{-i\xi_n x} \phi_n^*(y) ( i[H_0,P])(x,x',y,y') e^{i\xi_m x'}\phi_m(y') dxdy dx' dy' \\
  =& \dint e^{-i\xi_n z} \phi_n^*(y)  iH_0(z,y,y') \big( (P(x')-P(x'+z)) e^{i(\xi_m-\xi_n) x'} \big) \phi_m(y') dz dx' dy dy' \\
  =&  \hat P(\xi_n-\xi_m) \dint  \phi_n^*(y)  (e^{-i\xi_n z} - e^{-i\xi_m z})  iH_0(z,y,y')\phi_m(y') dz dy dy' \\
  =& \hat P(\xi_n-\xi_m) \dint  \phi_n^*(y) (\hat H_0(\xi_n,y,y')-\hat H_0(\xi_m,y,y'))\phi_m(y') dy dy' \\
  =& \hat P(\xi_n-\xi_m) (E_n(\xi_n)-E_m(\xi_m)) (\phi_n,\phi_m) = 0
\end{align*}
since $E=E_n(\xi_n)=E_m(\xi_m)$ while $\hat P$, the Fourier transform of $P-\frac12$, is bounded for $\xi_n\not=\xi_m$ (decomposing $P$ as a Heaviside function plus an integrable function while $\hat P$ would equal $(\xi_m-\xi_n)^{-1}$ for $P$ the Heaviside function). Note that we may not have (and do not have in practice) $(\phi_n,\phi_m)=0$ for $n\not=m$ since $\xi_n\not=\xi_m$ for a fixed value of $E$ while the eigenfunctions $\phi_n$ are orthogonal for different values of $E_m$ at a fixed value of $\xi$.

When $\xi_n=\xi_m$, we find instead
\begin{align*}
  &(    i[H,P] e^{i \xi_m x}\phi_m, e^{i \xi_m x}\phi_n)  = \dint e^{-i\xi_m x}\phi_n^*(y) ( i[H_0,P])(x,x',y,y') e^{i\xi_m x'}\phi_m(y') dxdy dx' dy' \\
  =& \dint e^{-i\xi_m z} \phi_n^*(y)   iH_0(z,y,y') \big( P(x')-P(x'+z)  \big) \phi_m(y') dz dx' dy dy' \\
  =& \dint  \phi_n^*(y)  e^{-i\xi_m z} (-z)   iH_0(z,y,y')\phi_m(y') dz dy dy' \\
  =& \dint  \phi_n^*(y) \partial_{\xi}\hat H_0(\xi_m,y,y') \phi_m(y') dy dy' = (\phi_n,\partial_{\xi}\hat H_0(\xi_m)\phi_m) .
\end{align*}
The modes $\phi_m(\xi)$ satisfy
\[
  \hat H_0(\xi) \phi_m(\xi) = E_m(\xi) \phi_m(\xi).
\]
Since the spectral branches $\xi\to E_m(\xi)$ are assumed sufficiently smooth, this yields
\[
  \partial_\xi \hat H_0 \phi_m + \hat H_0 \partial_\xi \phi_m = \partial_\xi E_m \phi_m + E_m \partial_\xi \phi_m
\]
from which we deduce 
\[
  (\phi_n,\partial_\xi \hat H_0 \phi_m) = \partial_\xi E_m (\phi_n,\phi_m)
\] 
for any $\phi_n$ such that $(\hat H_0-E_m)\phi_n=0$. If $n\not=m$ while $E_m(\xi_m)=E_n(\xi_m)$, then we may choose the eigenvectors $\phi_n$ and $\phi_m$ as orthogonal so that $(\phi_n,\partial_\xi \hat H_0 \phi_m)=0$ then (we have that $\partial_\xi E_m\not=0$ since $E\not\in Z$ is not at a critical value of the energy branches).

As a result, when $\xi_n=\xi_m$ we have
\begin{align*}
  ( e^{i \xi_m x}\phi_n,i[H,P] e^{i \xi_m x}\phi_m)   =\delta_{mn} (\phi_m,\partial_{\xi}\hat H_0(\xi_m)\phi_m)  = \delta_{mn} \partial_\xi E_m .
\end{align*}
We used the normalization $\|\phi_m\|^2=1$.  This concludes the derivation.
\end{proof}

We thus conclude from \eqref{eq:approxscattering} and the above lemma that in the limits $x_0\to\pm\infty$,
\begin{equation}\label{eq:decpsimn}
 (\psi_m^Q,2\pi i[H,P(\cdot-x_0)]\psi_n^Q) \approx \sum_p J_p \alpha^\pm_{mp} \bar\alpha^\pm_{np}.
\end{equation}

%
%%
\subsection{Scattering matrix and edge conductivity}\label{sec:scatteringmatrix}
%%
%

We next define the refection and transmission coefficients $R^\pm_{mn}$ and $T^\pm_{mn}$ as
\begin{align}
  \alpha^+_{mn} &= \sqrt{\frac{|J_m|}{|J_n|}}T^+_{mn} \ \ \mbox{when} \ \ J_m>0 \mbox{ and } J_n>0\\
   \alpha^-_{mn} &= \sqrt{\frac{|J_m|}{|J_n|}}T^-_{mn} \ \ \mbox{when} \ \ J_m<0 \mbox{ and } J_n<0\\
   \alpha^+_{mn} &=  \sqrt{\frac{|J_m|}{|J_n|}}R^-_{mn} \ \ \mbox{when} \ \ J_m<0 \mbox{ and } J_n>0\\
    \alpha^-_{mn} &=  \sqrt{\frac{|J_m|}{|J_n|}}R^+_{mn} \ \ \mbox{when} \ \ J_m>0 \mbox{ and } J_n<0,
\end{align}
while we also have $\alpha^-_{mm}=1$ when $J_m>0$ and $\alpha^+_{mm}=1$ when $J_m<0$. All other coefficients $\alpha^\pm_{ij}$ then vanish. We then have the following result:
\begin{lemma}\label{lem:unitaryscattering}
 The $(n_++n_-)\times(n_++n_-)$ scattering matrix
 \[
  S =\begin{pmatrix} T_+ & R_- \\R_+&T_-\end{pmatrix}
\]
is unitary. Here $T_+$ is the $n_+\times n_+$ matrix with coefficients $T^+_{mn}$, etc.
\end{lemma}
\begin{proof}
  From \eqref{eq:decpsimn} evaluated at $x_0\to\pm\infty$ and the current conservation in Lemma \ref{lem:currentconservation} stating that both limits are equal, we deduce that when $J_m>0$ and $J_n>0$, then
\[
 \sum_{J_p>0} \bar T^+_{mp} T^+_{np} = \delta_{mn}  -  \sum_{J_p<0} \bar R^+_{mp} R^+_{np} .
\]
This shows the orthonormality of the first $n_+$ columns of $S$. Considering the other cases $\pm J_m>0$ and $\pm J_n>0$ provides the other orthonormality constraints and concludes the proof. 
\end{proof}
\begin{lemma} \label{lem:tracescattering}
   Let $S$ be the above scattering matrix. Then
\begin{equation}\label{eq:tracescattering}
  \trr\ T^*_+T_+ - \trr\ T^*_-T_- = n_+-n_-.
\end{equation}
\end{lemma}
\begin{proof}
 From unitarity of the scattering matrix, we get
\[
  S^*S=  \begin{pmatrix}T_+^* T_+ +R_+^*R_+ & T_+^* R_-+R^*_+ T_-  \\ R_-^*T_+ + T_-^*R_+ & R^*_- R_- + T_-^* T_-  \end{pmatrix}  = S S^* =  \begin{pmatrix}T_+ T_+^* +R_-R_-^* & T_+ R_+^*+R_- T_-^*  \\ R_+T_+^*+ T_- R_-^*  & R_+ R_+^*+ T_- T_-^*-\end{pmatrix}  = I.
\]
Looking at the diagonal terms, we obtain
\[
  \trr\  T_+^* T_+ +R_+^*R_+ = \trr\ T_+ T_+^* +R_-R_-^* = n_+,\quad
  \trr\  R^*_- R_- + T_-^* T_-  = \trr\ R_+ R_+^*+ T_- T_-^* = n_-.
\]
By cyclicity of the trace or explicit computation of the norm, we deduce that 
$\trr R_+^*R_+ = \trr R_- R_-^*= \trr R_-^* R_-$
so that $\trr T_+^* T_+ - \trr T_-^* T_- = n_+-n_-.$
This may be written using only one-sided measurements as
$n_+-n_- =  \trr\ (T_+^* T_  + +  R^*_-R_-)- n_- = n_+ - \trr\ (T_-^* T_ - +  R^*_+R_+ )$.
\end{proof}

We deduce from the spectral theorem and the decomposition \eqref{eq:spectraldecH} the following result.
\begin{proposition}\label{prop:sigmaIpsimQ}
Let $E\in \Rm\backslash Z$ and $\psi_m^Q$ the associated perturbed generalized eigenfunctions. Then:
\begin{equation}\label{eq:sigmaIpsimQ}
  %2\pi \sigma_I[H] = 
  \dsum_m   \Big| \frac{\partial\xi_m}{\partial E} \Big| (\psi_m^Q , 2\pi i[H,P] \psi_m^Q ) = n_+-n_-.
\end{equation}
\end{proposition}
\begin{proof}
We have from the above calculations and sending $x_0\to+\infty$
\begin{align*}
   &\dsum_m   \Big| \frac{\partial\xi_m}{\partial E} \Big| (\psi_m^Q , 2\pi i[H,P] \psi_m^Q )  =  \dsum_m   \Big| \frac{\partial\xi_m}{\partial E} \Big| \dsum_n |\alpha^+_{mn}|^2 J_n
   \\=&\dsum_{J_m>0} \frac1{|J_m|} \dsum_{J_n>0} |T^+_{mn}|^2 |J_m|+ \dsum_{J_m<0} \frac1{|J_m|} \Big(J_m + \dsum_{J_n>0} |R^+_{mn}|^2 |J_m|\Big)
   \\=& \sum_{J_m>0,J_n>0} |T^+_{mn}|^2  - n_-  +  \sum_{J_m<0,J_n>0} |R^-_{mn}|^2  = n_+-n_-.
\end{align*}
We use here that there are $n_+$ modes with $J_m>0$ and $n_-$ modes with $J_m<0$. 
\end{proof}
%\gb{Do we need $E\not\in Z$? The proof below shows that point spectrum does not contribute to current. We need continuity of $(\psi^Q_j(\xi),i[H,P]\psi^Q_j)$ in $\xi$ also at $\xi_m$, which is not demonstrated. Without this result, we would then need to exclude $Z$ from admissible energies.}

As a corollary of the preceding proposition, we obtain the final main result of this section:
\begin{theorem}\label{thm:sigmascattering}
 The conductivity may be recast as 
 \[
  2\pi \sigma_I =  \trr\ T^*_+T_+ - \trr\ T^*_-T_- = n_+-n_-.
 \]
\end{theorem}
\begin{proof}
From \eqref{eq:spectraldecH}, we have
\[
  \varphi'(H)= \dsum_n  \varphi'(\lambda_n) \Pi_n + \dsum_j \dint_{\Rm} \varphi'(E_j(\xi)) \Pi^Q_j(\xi) d\xi
\]
The operators $i[H,P]\Pi_n$ are trace-class with vanishing trace since
\[
  \Tr [H,P]\Pi_n = (\phi_n,[H,P]\phi_n) = (\phi_n, HP \phi_n) - (\phi_n, PH \phi_n) = \lambda_n (\phi_n, P \phi_n) - \lambda_n (\phi_n, P \phi_n)=0.
\]
Since the sum over $n$ is finite, it does not contribute to $\sigma_I[H]$. Thus, from the definition of the rank-one projectors $\Pi^Q_j$, and identifying $\psi^Q_j(\xi)$ with $\psi^Q_m(E)$ when $E_j(\xi_m)=E$, we find
\[
\begin{array}{rcl}
  2\pi \sigma_I[H] &=& \dsum_j \dint_{\Rm} \varphi'(E_j(\xi))  (\psi^Q_j(\xi),2\pi i[H,P]\psi^Q_j(\xi))  d\xi\\
    &=& \dsum_m \dint_{\Rm} \Big|\frac{\partial\xi_m}{\partial E} \Big| \varphi'(E)  (\psi^Q_m(E),2\pi i[H,P]\psi^Q_m(E))  dE \\
    &=&  \dint_{\Rm}\varphi'(E) (n_+-n_-)dE = n_+-n_- = \trr\ T^*_+T_+ - \trr\ T^*_-T_-.
  \end{array}
\]
%This result is independent of the choice of $\varphi'(h)$ by assumption [H4]; see \cite{2,3,bal2023topological,quinn2021approximations}. Therefore, from the smoothness of $E_j(\xi)$ and by the inverse function theorem, the smoothness of $\xi_m(E)$ defined so that $E_m(\xi_m(E))=E$, we deduce \eqref{eq:sigmaIpsimQ} in the limit of $\varphi'$ converging to a delta function at $E$ since $\xi\to(\psi^Q_j(\xi),i[H,P]\psi^Q_j(\xi))$ is continuous at $\xi_m(E)$.  
We used $\varphi\in \fS[0,1]$ and Lemma \ref{lem:tracescattering} to conclude. 
\end{proof}



%
%%
%
%%
\section{Spectral analysis and limiting absorption principle}\label{sec:lap}
%%
%
This section analyzes spectral properties of $H_0$ and $H=H_0+Q$ for operators satisfying the following estimates.  
We assume that $H_0$ is a self-adjoint differential operator as described in [H1](o)-(iv) above. The resolvent operator $R_0(z)=(H_0-z)^{-1}$ then displays different behaviors as $z$ approaches the real-axis with positive or negative imaginary part. %We introduce the following sets. 
\begin{definition}\label{def:Jab}
For $a<b$, we define
\begin{align*}
    J_+(a,b)&=\{\lambda\in\mathbb{C}\ |  \ a<\operatorname{Re}\lambda<b,\  0<\operatorname{Im} \lambda<1\},\\
    J_-(a,b)&=\{\lambda\in\mathbb{C}\ |\  a<\operatorname{Re}\lambda<b, \ -1<\operatorname{Im} \lambda<0\}, \\
    J(a,b)&=J_+(a,b)\cup J_-(a,b).
\end{align*}
\end{definition}
Our objective is to prove results on the spectrum of $H$ that will allow us to verify hypothesis [H2] in section \ref{sec:eigexp}. 
We recall that the spaces $L^2_s$ and $H^p_s$ are introduced in Definition \ref{def:LHs} and that $Z$ is the set of critical values of branches of spectrum of $H_0$ defined in \eqref{eq:Z}.  
\begin{remark}\label{rmk:cpt}
For $s>0$, the injection $H_s^p\xhookrightarrow{} L^2$ is compact \cite{ASNSP_1975_4_2_2_151_0}.
%This could be derived by the Rellich–Kondrachov theorem and by recalling the definition of $H_s^1$. 
\end{remark}

We make the following assumptions on the short-range perturbation $Q$ and assume the following a priori estimates.
%%%\bbb{I got a little confused here, in section 3 we assume $Q$ is $q\times q$ while in sectin 2 the setting for $Q$ is $q\times q$ ?}
%%% THEY SHOULD ALL BE q\times q NOW.
\begin{hypothesis}\label{hyp:Q}
    We assume that $Q(x,y)$ is a $q\times q$ Hermitian matrix valued function. Moreover,  $|Q(x,y)|$ is bounded (measurable) and for some $h>1$ and $C=C(h)>0$,
        \begin{gather}\label{q}
|Q(x,y)| \leq C \aver{x}^{-h} ,\qquad (x,y)\in\Rm^2.
        \end{gather}
\end{hypothesis}
We recall that $\aver{x}=\sqrt{1+x^2}$.
\begin{hypothesis}\label{hyp:H0H} Let $\Rm\ni a<b\in \Rm$. We assume the following a priori estimates.
\begin{itemize}
 \item[1.] Let $s>\frac12$. There is a constant $C=C(s,a,b)>0$ such that 
 \begin{gather}\label{eq:H_0}
    \|u\|_{H_{-s}^p}\leq C\|(H_0-\lambda)u\|_{L_s^2},
\end{gather}
for all complex numbers $\lambda\in J(a,b)$ and $u\in H_s^p$.
\item[2.] Let $s>0$, $\epsilon>0$, and $(a,b)\cap Z=\emptyset$. There is a constant $C=C(s,a,b)>0$ such that 
\begin{gather}\label{eq:H0real}
    \|u|\|_{H_{s-1-\epsilon}^p}\leq C\|(H_0-\lambda)u\|_{L_s^2},
\end{gather}
for all real numbers $\lambda\in (a,b)$ and $u\in H^p$.
%\\
%\bbb{I think here $u$ should be assumed in $H^1$ for thr proof of proposition 3.5.}

\item[3.]
Let  $[a,b]\cap Z=\emptyset$ and $[a,b]$ not containing any eigenvalue of $H$. Then for $h>s>\frac{1}{2}$, there exists a constant $C=C(s,a,b)$ such that
\begin{gather}\label{ineq:h}
    \|u\|_{H^p_{-s}}\leq C\|(H-\lambda)u\|_{L^2_s},
\end{gather}
for all $u\in H_{s}^p$ and $\lambda\in J(a,b)$.
\end{itemize}
\end{hypothesis}



The above hypotheses provide sufficient a priori estimates on $H=H_0+Q$ to characterize some of its spectral properties.  We start by showing that with the above hypotheses, then $H$ has at most discrete point spectrum. 

\begin{proposition}\label{prop:eigenvalue}
The point spectrum of $H$ is discrete. (In particular, every eigenvalue has finite multiplicity.) The only possible limiting points of families of eigenvalues are in  $Z\cup \{\pm \infty\}$.
\end{proposition}
\begin{proof}
We will show that if $u\in H^p$ is an eigen-function corresponding to $\lambda$, i.e., $Hu=\lambda u,\ a<\lambda <b$, where $a,b$ satisfies 
%the condition in Proposition \ref{est:dirac_real}, 
condition 2 in Hypothesis \ref{hyp:H0H},
then $u\in H^p_s$ for some $s>0$ and %that
\begin{gather}\label{ieq1:eigen}
    \|u\|_{H^p_s}\leq C\|u\|_{L^2}.
\end{gather}
with some constant $C$ independent of $\lambda$.

The proposition is a direct corollary. Indeed, suppose $\{u_n\}$ is a set of eigenfunctions with norm 1 in $H_s^p$. By \eqref{ieq1:eigen}, $\|u_n\|_{L^2}$ is bounded below by a positive constant. Also by Remark \ref{rmk:cpt}, the injection map from $H_s^p$ into $L^2$ is compact. As $\{u_n\}$ is orthogonal and bounded both below and above in $L^2$, it must be in a finite set. This implies that $H$ has finite eigenvalues in $[a,b]$ and the multiplicity of each eigenvalue is also finite.

To prove \eqref{ieq1:eigen}, we first show that it holds true for $s=0$. Indeed, since $Hu=\lambda u$,
\[(H_0-i) u = (\lambda-i) u + g \in L^2,\quad g=-Qu= (H_0-\lambda)u.\]
Thus $u=R_0(i)[(\lambda-i) u + g]$ with $R_0(i)$ mapping from $\mH$ to $H^p$ by ellipticity assumption [H1(o)] and with $Q$ bounded implies that \eqref{ieq1:eigen} holds with $s=0$.  Now, the operator of multiplication by $Q(x,y)$ is a continuous operator from $L^2$ (and hence $H^p$) into $L^2_h(\mathbb{R}^2)$ by hypothesis \ref{hyp:Q}. This implies
\begin{gather*}
    \|g\|_{L_h^2}\leq C_1\|u\|_{H^p} \leq C_2\|u\|_{L^2},
\end{gather*}
with some constant $C_1, C_2$ independent of $\lambda$ on a compact interval.
For $g\in L_h^2$, we apply 
condition 2 in Hypothesis \ref{hyp:H0H},
%Proposition \ref{est:dirac_real}, 
%%%  \bbb{condition 2 in Hypothesis 3.4 assumes $u\in H_s^p$, but here $u\in H^p$, should we weaken the assumption of condition 2?} \gb{Yes, absolutely. We do prove the result with $p=1$ and $s=0$ for Dirac in Lemma 5.4. So what you propose indeed works.}
to deduce that $u\in H^p_{h-1-\epsilon}$ for any $\epsilon>0$. Choosing $0<\epsilon<h-1$, we proved \eqref{ieq1:eigen}.  
\end{proof}

The estimates we obtained in the preceding section naturally yield the following corollary, one of the main results of this section.
\begin{theorem}[Principle of limiting absorption] \label{thm:pla}
Let $a,b\in\Rm$ such that $[a,b]\cap Z=\emptyset$ and $[a,b]$ does not contain any eigenvalue of $H$.
%$$\sqrt{2n}<a<b<\sqrt{2n+2}\  \text{or}\  -\sqrt{2n+2}<a<b<-\sqrt{2n}$$ for some $n\in\mathbb{N}$, and $[a,b]$ does not contain any eigenvalue. 
For $\frac{1}{2}<s<h$, $f\in L_s^2$ and $\operatorname{Im}z\neq 0$, define 
\begin{gather*}
    u_z(f)=(H-z)^{-1}f.
\end{gather*}

Then for $\lambda\in (a,b)$, there exists $u^\pm(\lambda,f)$ such that
\begin{gather*}
    u_z(f)\to u^\pm(\lambda,f)\ \text{in}\ H^p_{-s}
\end{gather*}
as $z\to\lambda\pm 0i$, respectively. Moreover, $u^\pm(\lambda,f)$ are solutions of the equation 
\begin{gather*}
    (H-\lambda)u(x)=f(x)
\end{gather*}
and they are continuous functions of $\lambda$ in the topology of $H_{-s}^p$.
\end{theorem}
\begin{proof}
We modify standard arguments developed in \cite[Corollary 4.1]{yamada}. Since similar arguments will be used in the proof of Theorem \ref{thm:estimate} below to show that condition 3 in Hypothesis \ref{hyp:H0H} holds for the Dirac operator, we also refer to that proof for additional details.
%\gb{What does this mean? A bit more detail ?}
We take an arbitrary sequence $\{z_n\}$ such that $\operatorname{Im} z_n>0$ and $z_n\to\lambda+0i$ as $n\to\infty$. 

%%%From \eqref{A_lambda} and the estimation in the proof of Proposition \ref{est:dirac}, we have $u_{z_n}\in H_s^1$. And by Theorem \ref{thm:estimate}, $u_{z_n}(f)$ is bounded in $H_{-s}^1$. 

From condition 3 in Hypothesis \ref{hyp:H0H}, $u_{z_n}(f)$ is bounded in $H_{-s}^p$. 
We can then select (using the same method as described in the proof of Theorem \ref{thm:estimate} below) a subsequence $\{u_{z'_n}\}$ from $\{u_{z_n}\}$ which converges locally in $H_0^p(|x|\leq R)$ for any $R>0$ to some function $u_0$ in $H_{-s}^p$ and $u_0$ satisfies that $(H-\lambda)u_0=0$.


Next we wish to show that 
\begin{gather*}
    u_{z_n}\to u_0\ \text{in}\ H^p_{-s}.
\end{gather*}
Assuming the contrary, we select a subsequence $\{u_{z''_n}\}$ such that
\begin{gather}
\label{u_zn}
    \|u_{z''_n}-u_0\|_{H^p_{-s}}\geq\delta>0.
\end{gather}

We then choose (by the method described in the proof of Theorem \ref{thm:estimate} below) a subsequence of $\{u_{z''_n}\}$ which we still denote by $\{u_{z''_n}\}$ such that $u_{z''_n}\to u_1$ for some function $u_1$ in $H^p_{-s}$. Defining
\begin{gather*}
    v_n=u_{z'_n}-u_{z''_n}, \quad v_0=u_0-u_1,
\end{gather*}
then $v_n\to u_0-u_1$ in $H^p_{-s}$, and $(H-\lambda)v_0=0$. As in Theorem \ref{thm:estimate}, we show that $v_0\in L^2$, which implies that  $v_0=0$ since $v_0\in L^2$. This contradicts \eqref{u_zn}.

We now show that the limit $u_0$ is independent of the choice of the $\{z_n\}$ converging to $\lambda+0i$. Take another sequence $v_n\to\lambda+0i$. Then there exists $u_2\in H^p_{-s}$ such that $u_{v_n}\to u_2$ in $H^p_{-s}$. We shall show $u_0=u_2$. To prove this, define
\begin{gather*}
    w_n=u_{z_n}-u_{v_n}, \quad w_0=u_0-u_2.
\end{gather*}
Then $w_n\to w_0$ in $H^p_{-s}$ and $(H-\lambda)w_0=0$ which implies $w_0=0$.
%using the same argument as the proof of Theorem above. 
Thus, $u_+(\lambda,f)=u_0$ is well defined.


Finally we show that $u_+(\lambda,f)$ is continuous in $\lambda$. In view of the fact that the resolvent $u_z(f)$ is a continuous function of $z\in J_+(a,b)$, it suffices to show that $u_{\lambda+i\eta}$ converges to $u_\lambda(f)$ in $H_{-s}^p$ uniformly respect to $\lambda\in J_+(a,b)$. If we assumed the contrary, there would be a $\delta_1>0$ and a sequence of $\eta_n\to 0+$ such that
\begin{gather*}
    \|u_{\lambda_n+i\eta_n}(f)-u_{\lambda_n}(f)\|_{H_{-s}^p}\geq \delta_1>0.
\end{gather*}
On the other hand, for each $n$ we could also select $\eta'_n$ such that $\eta'_n\to 0+$ and
\begin{gather*}
    \|u_{\lambda_n+i\eta'_n}(f)-u_{\lambda_n}(f)\|_{H_{-s}^p}\leq \frac{\delta_1}{2}>0.
\end{gather*}
This implies that 
\begin{gather}
\label{ineq:contr}
    \|u_{\lambda_n+i\eta_n}(f)-u_{\lambda_n+i\eta'_n}(f)\|_{H_{-s}^p}\geq\frac{\delta_1}{2}>0.
\end{gather}
Because $[a,b]$ is compact, we can select a convergent subsequence $\{\lambda'_n\}$ of $\{\lambda_n\}$ such that $\lambda'_n\to\lambda_0$. This contradicts \eqref{ineq:contr} because both $u_{\lambda_n+i\eta_n}, u_{\lambda_n+i\eta'_n}$ would converge to $u_{\lambda_0}(f)$. We thus proved that $u_+(\lambda,f)$ is continuous in $\lambda$.
\end{proof}

Our final result is the following:
\begin{theorem}\label{thm:absolute_cont}
$H$ does not have singular continuous spectrum.
\end{theorem}
\begin{proof}
%\cite[Corollary 4.2]{dirac}
Let $E(\lambda)$ be the right-continuous resolution of the identity associated with the self-adjoint operator $H$. It suffices to show that $(E(\lambda)f,f)$ for $f\in L_s^2$ is continuous when $\lambda\in\mathbb{R}$ is not an eigenvalue.  

First we assume that $$k_n<\alpha<\beta<k_{n+1}$$ for some $n\in\mathbb{Z}$, and $[\alpha,\beta]$ does not contain any eigenvalue, where $\{k_n\}$ label elements in the union of $Z$ defined in \eqref{eq:Z} and the discrete spectrum of $H$. 
%are labels defined in \eqref{kn}.


For arbitrary $a,b$ with $\alpha<a<b<\beta$, we have the following relation using, e.g., \cite[(5.32)]{kato2013perturbation}, 
\begin{gather*}
\frac{1}{2}[(E(a)+E(a-0))f,f]-\frac{1}{2}[(E(b)+E(b-0))f,f]\\
=\frac{1}{2\pi i}\ \underset{\eta\to 0}{\lim}\int_a^b((H-\mu-i\eta)^{-1}f-(H-\mu+i\eta)^{-1}f,f)d\mu,
\end{gather*}
so that
\begin{gather}\label{eaeb}
|(E(a)f,f)-(E(b)f,f)|=\frac{1}{2\pi i}\int_a^b(u^+(\mu,f)-u^-(\mu,f),f)d\mu.
\end{gather}
Then we have from Theorem \ref{thm:estimate} that
\begin{equation} \label{eq:estimsc}
    |(E(a)f,f)-(E(b)f,f)|\leq C(b-a)\|f\|_{L_s^2},
\end{equation}
where $C$ is a constant only depending on $\alpha,\beta$ and $s$, which shows the desired absolute continuity on $(\alpha,\beta)$. Finally, since $Z$ and the collection of eigenvalues of $H$ is a discrete set and the singular continuous spectrum cannot be supported on a discrete set, we conclude that $H$ does not have any singular continuous spectrum.
\end{proof}



%
%%
\section{Generalized eigenfunction expansion} \label{sec:eigexp}
%%
%

In this section, we justify the expansion \eqref{eq:spectraldecH} and the corresponding hypothesis [H2] of section \ref{sec:current}. We thus need to construct generalized eigenfunctions $\psi_j^Q(\cdot;\xi)$ for $\fco=(j,\xi)\in J\times \Rm$. %The construction is based on that of $\psi_j^Q(\cdot;E)$ for $E\in\Rm\backslash Z$. 

We know that the point spectrum $(\lambda_n)_n$ of $H$ is discrete. Define $Z_H=Z\cup\{(\lambda_n)_n\}$. Let $E\in(a,b)$ with $(a,b)\cap Z_H=\emptyset$. We know from Theorem \ref{thm:pla} in the preceding section that $R(z)=(H-z)^{-1}$ is well defined on $\mH$ and bounded uniformly for $z\in J(a,b)$. We may thus define the bounded operators:
\[R^\pm(\lambda)=(H-(\lambda\pm i0))^{-1}.\]  

For $\fco=(j,\xi)\in J\times\Rm$ as defined in section \ref{sec:current}, we defined $\psi_j(x,y;\xi)$ in \eqref{eq:unperturbedxi}.  For $z\in J(a,b)$, it is convenient to introduce the function $A_\fco(x,y;z)$ defined as
\begin{equation}\label{eq:Az}
     A_\fco(z) =(I-R(z)Q)\psi_j(\xi).
\end{equation}
%Seen as an operator from $\Cm^n$ to $\mH=L^2(\Rm^2;\Cm^n)$, we may define a formal adjoint
Associated is the following linear form defined for $f\in L^2_s$ with $s>\frac12$:
\begin{equation}\label{eq:Azstar}
     A^*_\fco f(z) =(f,A_\fco(z)) := \dint_{\Rm^2} f(x,y) \cdot \bar A_\fco(x,y;z) dxdy.
\end{equation}
We now define the perturbed generalized eigenfunctions
\begin{equation}\label{eq:perturbedxi}
  \psi_j^\pm(\xi) = A_\fco(E_j(\xi)\pm i0) = (I-R(E_j(\xi)\pm i0)Q)\psi_j(\xi).
\end{equation}
For concreteness, we define $\psi_j^Q(x,y;\xi)=\psi_j^+(x,y;\xi)$, the {\em outgoing} generalized eigenfunctions, while $\psi_j^-(x,y;\xi)$ corresponds to {\em incoming} generalized eigenfunctions.

Thanks to Theorem \ref{thm:pla}, we observe that $\psi_j^\pm(\xi)\in H^p_{-s}(\Rm^2)$. For each $E\in(a,b)$, there is a finite number of wavenumbers $\xi_m$ such that $E=E_m(\xi_m)$. We denote by $\psi_m^\pm(E)$ the corresponding generalized eigenfunctions parametrized by $(m,E)$ rather than $(j,\xi)$.

 Let $f\in L_s^2$ and $j\in J$ with $\xi\in \Xi_j$ so that $E_j(\xi)\in (E_-,E_+)\backslash Z_H$. We define the generalized Fourier transform(s) by
 \begin{gather}\label{def:fourier}
    \tilde{f}^\pm(\fco)=A_\fco^*(E_j(\xi)\pm 0i)f,
\end{gather}  
and for any $\lambda_n\in (E_-,E_+)$ an eigenvalue of $H\phi_n=\lambda_n\phi_n$,
\begin{gather}\label{def:fourierdisc}
    \tilde{f}_n = (f,\phi_n).
\end{gather}  


We then obtain the following results justifying the expansion [H2] in \eqref{eq:spectraldecH} in section \ref{sec:current}. 
We first verify the following estimate on the generalized Fourier transform.
\begin{lemma}\label{lmm:a_star}
For $\frac{h}{2}>s>\frac{1}{2}$, there exists a constant $C=C(s,a,b)$ such that
\begin{gather}
    |A^*_\fco(z)f|\leq C\|f\|_{L_s^2}
\end{gather}
for all $\fco\in J\times\mathbb{R}$, $z\in J(a,b)$ and $f\in L_s^2$.
\end{lemma}
\begin{proof}
By construction, we have
\begin{gather}
    A^*_\fco(z)f= (f, A_\fco(z))_{L^2}=(f, (I-R(z)Q)\psi_{j}(\xi))_{L^2}.
\end{gather}
The multiplication operator $Q$ is a continuous map from $L_{-s}^2$ to 
$L_{h-s}^2$
and by use of Hypothesis \ref{ineq:h}, we get
\begin{gather*}
   |(f, (I-R(z)Q)\psi_{j}(\xi))_{L^2}|\leq\|f\|_{L_s^2}\cdot\|(I-R(z)Q)\psi_{j}(\xi))\|_{L_{-s}^2}\\
   \leq C\|f\|_{L_s^2}\cdot\|\psi_{j}(\xi)\|_{L_{-s}^2}\leq C\left(\int_{\Rm}(1+|x|)^{-2s}dx\right)^{\frac{1}{2}}\|f\|_{L_s^2}
\end{gather*}
and hence the result.
\end{proof}
We next state the following properties of the resolvent operator:
\begin{proposition}
\label{prop:(h-z)a}
For $s>\frac{1}{2}$, $\operatorname{Im}z\neq 0$ and $f\in L_s^2$, we have
\begin{gather}
 \label{eq:(h-z)a}
    (H-z)A_\Xi(z)=(E_j(\xi)-z)\psi_j(\xi),\\
\label{hat_r(z)f}
    \widehat{R(z)f}(\Xi)=\frac{A^*_{\Xi}(\bar{z})f}{E_j(\xi)-\bar{z}},
\end{gather}
where $\hat{f}(\Xi)=(f,\psi_j(\xi))$ is the unperturbed Fourier transform defined in \eqref{eq:FT}. %%%, and the integral in \eqref{lmm:r(z)f} is taken in Bochner's sense.
\end{proposition} 
\begin{proof}
We have by construction
\[A_\fco(z) = (I-R(z)Q)\psi_j(\xi)\]
so that 
\[ (H-z) A_\fco(z) = (H_0-z) \psi_{\fco} = (E_j(\xi)-z) \psi_j(\xi).\]
This is \eqref{eq:(h-z)a}.
Thus, from the definition of the (unperturbed) Fourier transform \eqref{eq:FT},
\[ \Big(f,\frac{(H-z) A_\fco(z)}{E_j(\xi)-z}\Big) = \hat f(\fco).\]
From this, we deduce
\[  \widehat{R(\overline{z})f}(\fco) = \Big( R(\bar z)f,\frac{(H-z) A_\fco(z)}{E_j(\xi)-z}\Big) =  \Big(f,\frac{A_\fco(z)}{E_j(\xi)-z}\Big) = \frac{A^*_\fco(z) f}{E_j(\xi)-\bar z}.\]
\end{proof}

We now prove the following eigenfunction expansion result. By proposition \ref{prop:eigenvalue}, we know that $H$ has discrete point spectrum and each eigenvalue has a finite multiplicity. Let $\{\varphi_n\}$ be the countable set of orthonormalized eigenfunctions of $H$.

\begin{theorem}[Eigenfunction Expansion Theorem]
For $f\in L_s^2\subset\mH$, 
%with $f=\Pi_{(E_-,E_+)}[H]f$, \gb{We need the latter projection when $(E_-,E_+)$ is not the whole line} 
we define $\tilde f$ as either one of the generalized Fourier transforms $\tilde f^\pm$ in \eqref{def:fourier}. Then we have the following Parseval relation:
\begin{gather}\label{eigenfunction_expansion}
    \|f\|_{L^2}=\sum_{n=0}^\infty |(f, \varphi_n)|^2+\int_\mathbb{R}\sum_{j\in J}|\tilde{f}(\Xi)|^2d\xi.
\end{gather}
\end{theorem}
\begin{proof}
For fixed $\xi$, let 
$$J_{(\alpha,\beta)}(\xi)=\{j\in J |   \ E_j(\xi)\in (\alpha,\beta)\}.$$
Let $[\alpha,\beta]$ contain no eigenvalue of $H$. We wish to show that
\begin{gather}\label{first_assertion}
    ((E(\beta)-E(\alpha))f,f)=\int_{\Rm}\sum_{j\in J_{(\alpha, \beta)}(\xi)}|\tilde{f}(\Xi)|^2 d\xi,
\end{gather}
and for an eigenvalue $\lambda$, that
\begin{gather}\label{second_assertion}
     ((E(\lambda)-E(\lambda-))f,f)=\sum_{i=1}^k|(f, \varphi_{\lambda,i})|^2,
\end{gather}
where $\{\varphi_{\lambda,i}\}$ are the orthonormalized eigenfunctions of $H$ associated with the eigenvalue $\lambda$.

The second assertion \eqref{second_assertion} reduces to the well-known Parseval equality of Fourier series. To prove the first assertion, we first assume that 
$$k_n<\alpha<\beta<k_{n+1}$$
for some $n\in\mathbb{Z}$ where $\{k_n\}$ label elements in $Z_H$, the union of $Z$ defined in \eqref{eq:Z} with the collection of eigenvalues of $H$. We make use of
\begin{gather*}
    ((E(\beta)-E(\alpha))f,f)=\frac{1}{2\pi i}\underset{\eta\downarrow 0}{\mathlarger{\lim}}\int_\alpha^\beta (R(\mu+i\eta)f-R(\mu-i\eta)f,f)d\mu.
\end{gather*}
Since $(E(\lambda)f, f)$ is absolutely continuous with respect to $\lambda$, we have using the resolvent equation $R(z_1)-R(z_2)=(z_1-z_2)R(z_1)R(z_2)$ and \eqref{hat_r(z)f}, that
\begin{gather}\label{eq:e}
\begin{aligned}
   & ((E(\beta)-E(\alpha))f,f)=\frac{1}{2\pi i}\underset{\eta\downarrow 0}{\lim}\int_\alpha^\beta 2i\eta (R(\mu-i\eta)R(\mu+i\eta)f,f)d\mu\\
    =&\frac{1}{\pi }\underset{\eta\downarrow 0}{\lim}\int_\alpha^\beta \eta \|R(\mu+i\eta)f\|^2d\mu=\frac{1}{\pi }\underset{\eta\downarrow 0}{\lim}\int_\alpha^\beta \eta \|\widehat{R(\mu+i\eta)f}\|^2d\mu\\
   =&\frac{1}{\pi }\underset{\eta\downarrow 0}{\mathlarger{\lim}}\int_\alpha^\beta\int_{\mathbb{R}}\sum_{j}\eta\left|\frac{A^*_\Xi(\mu-i\eta)}{E_j(\xi)-\mu +i\eta}f\right|^2d\xi d\mu,
\end{aligned}
\end{gather}
using the Parseval identity for $\mF$ in \eqref{eq:FT} (i.e., $\mF$ is an isometry) and \eqref{hat_r(z)f} for the last line. To analyze the above integral, we recall the result (see \cite[Proposition 4.3]{yamada1975eigenfunction}):
\begin{proposition}\label{prop:prep}
Let $f(\mu, \eta)$  be a continuous function on $[\alpha, \beta] \times [0, \eta_0]\ (\alpha<\beta, \eta_0>0)$. Then:% the following relation holds:
\begin{gather*}
    \frac{1}{\pi}\underset{\eta\downarrow 0}{\lim}\int_\alpha^\beta\frac{\eta}{(\lambda-\mu)^2+\eta^2}f(\mu,\eta)d\mu=\begin{cases}
0, &  \lambda>\beta\ or\  \lambda<\alpha  \\
\frac12 f(\lambda,0), & \lambda=\alpha\ or\ \beta\\
f(\lambda,0), & \alpha<\lambda<\beta.
\end{cases}
\end{gather*}
Moreover, the above integral is uniformly bounded on $[\alpha, \beta] \times [0, \eta_0]$.
\end{proposition}


Lemma \ref{lmm:a_star} implies that $|A^*_\Xi(\mu-i\eta)f|^2=|\tilde f(\Xi)|^2$ is continuous in $\Xi$ and uniformly bounded  on $[\alpha, \beta]\times [0,\eta_0]$. Proposition \ref{prop:prep} then yields
\begin{gather}\label{lim:eta}
\begin{aligned}
    &\underset{\eta\to 0}{\lim} \frac1\pi\int_\alpha^\beta\eta\left|\frac{A^*_\Xi(\mu-i\eta)}{E_j(\xi)-\mu-i\eta}f\right|^2d\mu
    \\& =\underset{\eta\to 0}{\lim}\frac1\pi\int_\alpha^\beta\frac{\eta}{|E_j(\xi)-\mu|^2+\eta^2} |A^*_\Xi(\mu-i\eta)f|^2 d\mu
    \ =\ \begin{cases}
    |\tilde{f}(\Xi)|^2, &  j\in J_{(\alpha,\beta)}(\xi) \\
    \frac12 |\tilde{f}(\Xi)|^2 &  E_j(\xi)=\alpha\  \text{or}\ \beta \\
    0, &  \text{otherwise}.
    \end{cases}
\end{aligned}
\end{gather}
Moreover, this integral is uniformly bounded for $\Xi\in\mathbb{R}\times J$ and $(\mu,\eta)\in (\alpha,\beta)\times(0,1)$. 

We split $J$ into two parts:
\begin{gather*}
    J_1=\big\{j\in J|\  d(E_j(\xi),[\alpha,\beta]) \geq 1, \ \forall \xi\in \Rm
    %\rho_j\geq 1+\max(|\alpha|, |\beta|)
    \big\},\quad J_0:=J\setminus J_1.
\end{gather*}
Here, $d(x,X)$ is Euclidean distance between a point $x\in\Rm$ and an interval $X\subset \Rm$.
By hypothesis \ref{hyp:H1}, $J_0$ is finite. Let $j\in J_0$. Using \eqref{lim:eta} for $\xi$ such that $E_j(\xi)\in[\alpha-1,\beta+1]$, and $|E_j(\xi)-\mu+i\eta|^2\geq C(1+|E_j(\xi)|^2)$ otherwise, we deduce that there is a constant $C=C(\alpha,\beta)$ independent of $\xi, \mu,\eta$ such that
\begin{gather}\label{est:uniform}
%\sum_{j\in J_0}\int_\alpha^\beta\eta\left|\frac{A^*_\Xi(\mu-i\eta)}{E_j(\xi)-\mu\tb{+}i\eta}f\right|^2d\mu\leq\frac{C}{1+\xi^2}\ \text{when}\ |\xi|\to\infty.
\sum_{j\in J_0} \int_\alpha^\beta\eta\left|\frac{A^*_\Xi(\mu-i\eta)}{E_j(\xi)-\mu+i\eta}f\right|^2d\mu \ \leq\    \frac{C \|f\|_{L^2_s}}{1+|E_j(\xi)|^2} .
\end{gather}
%We also used that $|A^*_\Xi(\mu-i\eta)f|^2$ was uniformly bounded by $C\|f\|_{L^2_s}$ as shown in Lemma \ref{lmm:a_star}.
%It is because the integral is  uniformly bounded for $\xi\in\mathbb{R}$ and $(\mu,\eta)\in (\alpha,\beta)\times(0,1)$ from Proposition \ref{prop:prep}.
%And for $|\xi|\to\infty$, the estimate results from that $|A^*(\mu-i\eta,\xi,m)f|^2$ is uniformly bounded and $|E_m(\xi)-\mu-i\eta|=O(\xi^{-1})$.

For $j\in J_1$, we have by construction that $|E_j(\xi)-\mu-i\eta|\geq 1$. % and $|E_m(\xi)-\mu-i\eta|=O(\xi^{-1})$. And n
Notice that
\begin{gather*}
    \sum_{j\in J}|A^*_\Xi(z)f|^2=\sum_{j\in J}|(f, A_\Xi(z))_{L^2}|^2=\sum_{j\in J}|(f, (I-R(z)Q)\psi_j(\xi))_{L^2}|^2\\
    =\sum_{m\in M}|((I-R(\bar{z})Q)f, \psi_j(\xi))_{L^2}|^2=\|\widehat{(I-R(z)Q)f\ }(\xi,y)\|_{L^2(y)}^2.
\end{gather*}
Since the operator $f\mapsto R(z)Qf$ from $L^2_s$ to $L^2_{-(s+h)}$ is uniformly bounded for $z\in J(\alpha,\beta)$, we deduce that
$$\sum_{j\in J_1}\int_\alpha^\beta\eta\left|A^*_\Xi(\mu-i\eta)f\right|^2d\mu$$ 
is also uniformly bounded for $\xi\in\mathbb{R}$ and $(\mu,\eta)\in (\alpha,\beta)\times(0,1)$. Thus we conclude that there exists a constant $C=C(\alpha,\beta)$ independent of $\xi,\mu,\eta$ such that
\begin{gather}\label{est:uniform2}
 \sum_{j\in J_1}\int_\alpha^\beta\eta\left|\frac{A^*_\Xi(\mu-i\eta)}{E_j(\xi)-\mu+i\eta}f\right|^2d\mu\leq\frac{C\|f\|_{L^2_s}}{1+|E_j(\xi)|^2}. %\ \text{when}\ |\xi|\to\infty.
\end{gather}
The estimates in \eqref{est:uniform} and \eqref{est:uniform2} are integrable in $\xi$ by Hypothesis \ref{hyp:H1}.
Thus, by the dominated convergence theorem, we deduce from \eqref{eq:e} and \eqref{lim:eta} that
\begin{gather}\label{end}
\begin{aligned}
    &((E(\beta)-E(\alpha))f,f)
   =\frac{1}{\pi }\int_{\mathbb{R}}\underset{\eta\downarrow 0}{\mathlarger{\lim}}\int_\alpha^\beta\sum_{j\in J}\eta\left|\frac{A^*_\Xi(\mu-i\eta)}{E_j(\xi)-\mu-i\eta}f\right|^2d\mu d\xi\\
   =&  \ \int_{\Rm}\sum_{j\in J{(\alpha, \beta)}(\xi)}|\tilde{f}(\Xi)|^2d\xi.
\end{aligned}
\end{gather}
This proves the first assertion \eqref{first_assertion} when $$k_n<\alpha<\beta<k_{n+1}$$
for some $n\in\mathbb{Z}$, where $k_n<k_{n+1}$ are successive numbers in the union of $Z$ defined in \eqref{eq:Z} with the collection of eigenvalues of $H$. By monotone convergence, we deduce that 
\[
  ((E(k_{n+1}-)-E(k_{n}+))f,f) = %%%\frac{1}{\pi } %%% Don't think it is there.
   \int_{\Rm}\sum_{j\in J_{(k_n, k_{n+1})}(\xi)}|\tilde{f}(\Xi)|^2d\xi.
\]
This handles the absolutely continuous part of the spectrum of $H$. It remains to address the discrete set of points $k_n$, which as in Theorem \ref{thm:absolute_cont}, carries only point spectrum. This is taken care of by the first term on the right-hand side in \eqref{eigenfunction_expansion} as in \eqref{second_assertion}.
This concludes the proof of the theorem.
\end{proof} 

%
%
%
%
% 
%
%
%
%%
\section{Application to Dirac operators with domain walls}\label{sec:Dirac}
%%
%
The theory developed in section \ref{sec:current} requires that we prove the hypotheses [H1](o-iv), and in particular the spectral decompositions in \eqref{eq:spectraldecH0} and \eqref{eq:spectraldecH}. While such a verification can undoubtedly be performed for a large class of problems, some of the steps developed below to analyze the spectrum of $H$ are intricate computationally and hence restricted to systems of Dirac operators similar to those considered in \cite{bal2023asymmetric}.


We thus consider the Dirac operator
\begin{equation}\label{eq:Dirac}
 H=H_0 + Q, \qquad H_0 = D_x\sigma_1 + D_y\sigma_2 + m(y) \sigma_3
\end{equation}
where $D_x=-i\partial_x$ and $D_y=-i\partial_y$, where $\sigma_{1,2,3}$ are the standard Pauli matrices, and where $m(y)$ is a domain wall that we will take the form $m(y)-y$ equal to a bounded function to simplify the presentation. Here, $Q$ is the operator of multiplication by $Q(x,y)$, which takes values in $2\times 2$ Hermitian matrices.

It is convenient to recast $H$ as $UHU^*$ with $U$ a unitary matrix so that $UHU^*$, still called $H$, takes the form $H=H_0 + Q$, where
\begin{gather}\label{eq:H0}
    H_0=D_x\sigma_3-D_y\sigma_2+m(y)\sigma_1=\left(\begin{array}{cc}
-i\partial_x & \fa \\
\fa^*& i\partial_x\\
\end{array}\right),\qquad \fa := \partial_y+m(y).
\end{gather}
We recognize in $\fa$ the annihilation operator of the quantum harmonic oscillator when $m(y)=y$. We assume that the range of $m$ is infinite, and for concreteness that:
\begin{hypothesis}\label{hyp:rangem}
We assume $m(y)-y$ is a bounded function. 
\end{hypothesis}

Under this hypothesis, the operators $H$ and $H_0$ are unbounded elliptic self-adjoint operators on the Hilbert space $L^2(\Rm^2;\Cm^2)$ with domains of definition $\mD(H)=\mD(H_0)$ the subspace of functions $(\psi_1,\psi_2)^t \in L^2(\Rm^2;\Cm^2)$  such that $\nabla \psi_j\in L^2(\Rm^2;\Cm^2) $ and $y\psi_j\in L^2(\Rm^2)$. This was denoted by the space $H^1$ with $\beta=1$ and $s=0$ in \eqref{eq:Hsp}. We assume here that $Q$ decays sufficiently rapidly in $x$ as described in Hypothesis \ref{hyp:Q} for the above result to hold \cite{kato2013perturbation,RS}. 
%
%%
\subsection{Spectral decomposition of the unperturbed operator} \label{sec:decH0}
%%
%
Since $H_0$ is invariant with respect to translations in the $x-$variable, we introduce the partial Fourier transform $H_0=\mF^{-1}_{\xi\to x} \hat H_0(\xi) \mF_{x\to\xi}$ with 
\begin{gather}\label{fourier:h0}
\hat H_0(\xi)=\xi\sigma_3+i\partial_y\sigma_2+m(y)\sigma_1%=\xi\sigma_3+h_1(y)
 =\left(\begin{array}{cc}
             \xi & \fa \\
             \fa^* & -\xi\\
\end{array}\right).
\end{gather}
We then compute
\begin{gather}\label{delta}
H_0^2-\lambda^2 = (H_0+\lambda)(H_0-\lambda)=\left(\begin{array}{cc}
             -\partial_{xx}+\fa\fa^* - \lambda^2 & 0 \\
             0 &  -\partial_{xx}+\fa^*\fa - \lambda^2\\
             \end{array}\right).
\end{gather}
Thus, $H_0^2$ is not quite proportional to identity since $[D_y,m(y)]=-im'(y)\not=0$. We now present a spectral decomposition of $H_0$ as well as an explicit expression for the resolvent operator $R_0(z)=(H_0-z)^{-1}$ for $z\in \Cm$ with $\Im z\not=0$.

We first observe that since the range of $m(y)$ is unbounded, standard results on Sturm Liouville operators show that $\fa^*\fa$ and $\fa\fa^*$ have a compact resolvent and hence discrete spectrum with simple eigenvalues. It is also straightforward to observe that $\fa$ admits a kernel in $L^2(\Rm)$ of dimension one. The positive eigenvalues of $\fa^*\fa$ and $\fa\fa^*$ are the same and we thus get the existence of simple eigenvalues $\rho_0=0<\rho_n<\rho_{n+1}$ for $n\geq1$ with $\rho_n/n$ converging to $2$ as $n\to\infty$. Moreover, we have the existence of two $L^2(\Rm;dy)-$orthonormal bases $(\nu_n)_{n\geq0}$ and $(\mu_n)_{n\geq1}$ such that 
\begin{equation}\label{eq:varphipsin}
    \fa^*\fa \nu_n = \rho_n \nu_n,\quad n\geq0;\qquad \fa\fa^* \mu_n=\rho_n \mu_n,\quad n\geq1.
\end{equation}
When $m(y)=y$, then $\nu_n$ and $\mu_n$ are both Hermite functions associated to the quantum harmonic oscillator. We also verify that (for $\|\cdot\|$ the standard $L^2(\Rm;dy)$ norm)
\begin{equation}\label{eq:controlfa}
\|f\| + \|yf\| + \|D_y f\| \leq C \|\fa^* f \|,\qquad \|yf\| + \|D_y f\| \leq C (\|\fa f\| + \|f\|).
\end{equation}
The asymmetry between the above two results stems from the fact that $\fa^*$ has trivial $L^2-$kernel while $\fa\nu_0=0$.

Define $\Pi^\nu_n=\nu_n\otimes \nu_n$ for $n\geq0$ and $\Pi^\mu_n=\mu_n\otimes \mu_n$ for $n\geq1$ (with $\Pi^\mu_0=0$ to simplify notation). Then we observe that we have the spectral decomposition:
\begin{equation}
    H_0^2 -\lambda^2 = \mF_{\xi\to x}^{-1}  \dsum_{n\geq0} \dint_{\Rm} d\xi \ 
    (\xi^2+\rho_n-\lambda^2) \begin{pmatrix} \Pi^\nu_n & 0 \\ 0 & \Pi^\mu_n \end{pmatrix} \, \mF_{x\to\xi}.
\end{equation}
This provides the following explicit expression for the resolvent
\begin{eqnarray}
    &R_0(\lambda) &= (H_0-\lambda)^{-1} = (H_0+\lambda) (H_0^2 -\lambda^2)^{-1} 
     \\ && = (H_0+\lambda) \mF_{\xi\to x}^{-1}  \dsum_{n\geq0} \dint_{\Rm} d\xi \ 
    (\xi^2+\rho_n-\lambda^2)^{-1} \begin{pmatrix} \Pi^\nu_n & 0 \\ 0 & \Pi^\mu_n \end{pmatrix} \, \mF_{x\to\xi}.
\end{eqnarray}
Estimates on $R_0(\lambda)$ may thus be obtained by applying $H_0+\lambda$ to the resolvent of the operator $H_0^2$. The above construction also shows that the eigenvalues of $\hat H_0^2(\xi)$ are given explicitly by $E_n^2(\xi)= \xi^2+\rho_n$ for $n\geq0$. We now show that $\pm E_n(\xi)$ are indeed eigenvalues of $\hat H_0(\xi)$. To simplify notation, we introduce the set $M$ of indices $m=(\pm1,n)$ for $n\geq1$ and $0\equiv (-1,0)$ for $n=0$. We then define the eigenvectors $\phi_m(\xi)$ for $m=(\pm,n) \in M$ as 
\[  \phi_m = \begin{pmatrix} \varphi_m \\ \psi_m \end{pmatrix}, \qquad E_m = \pm (\xi^2+\rho_n)^{\frac12}\]
with $\phi_0=(\nu_0,0)^t$ independent of $\xi$ for $m=0$ and for $n\geq1$,
\begin{equation}\label{eq:phimxi}
    \phi_m(\xi) =  c_n \begin{pmatrix} (E_m(\xi)+\xi) \nu_n \\ \rho_n\mu_n \end{pmatrix}, \qquad c_n^{-2} = 2E_m(\xi)(E_m(\xi)+\xi) >0.
\end{equation}
We verify that $c_m$ is indeed defined and independent of $\pm$ and hence labeled $c_n$. Since the functions $\nu_n$ and $\mu_n$ form an orthonormal basis, we easily deduce that the functions $\phi_m$ also form an orthonormal basis of $L^2(\Rm;\Cm^2)$.

From this completeness result, we deduce the spectral decomposition
\begin{equation}\label{eq:spectralH0}
    H_0 = \mF_{\xi\to x}^{-1} \dsum_m \dint_{\Rm} d\xi \  E_m(\xi)\, \Pi^\phi_m(\xi)  \ \mF_{x\to\xi} ,\qquad \Pi^\phi_m(\xi) = \phi_m(\xi) \otimes \phi_m(\xi).
\end{equation}
This provides an explicit expression for \eqref{eq:spectraldecH0} in hypothesis [H1] with $\psi_m$ defined by \eqref{eq:unperturbedxi}.

\medskip

We finally turn to the construction of the generalized eigenfunctions \eqref{eq:unperturbedE} at a fixed energy level $E\in\Rm\backslash Z_D$, where for the Dirac operator, we deduce from the explicit expression in \eqref{eq:phimxi} that the set $Z$ of critical values defined in \eqref{eq:Z} is given explicitly by
\begin{equation}\label{eq:ZDirac}
  Z_D = \Big\{ \, \pm \sqrt{\rho_n}; \ n\in\Nm \ \Big\}.
\end{equation}
%Here, we define $\lambda_0=0$.  
When $m(y)=y$ is a linear domain wall, then we verify that $\rho_n=2n$. By hypothesis \ref{hyp:rangem}, we deduce from \cite[Theorem 4.10]{kato2013perturbation} that $|\rho_n-2n|$ is bounded independently of $n$.

We construct a basis of $L^2(\Rm;\Cm^2)$ called $\phi_m(E)$ with a slight abuse of notation.
%(they are related to the functions $\phi_m(\xi)$ constructed above but are different. 
The objective is to construct a basis of solutions of $(\hat H_0(\xi_m)-E)\phi_m(E)=0$. The values $\xi_m$ are defined explicitly by
\begin{equation}\label{eq:xim}
    \xi_m = \epsilon_m (E^2-\rho_n)^{\frac12}
\end{equation}
where $m=(\epsilon_m,n)$ and $(-1)^{\frac12}=i$. Thus $\xi_m$ is real-valued for $n$ sufficiently small and purely imaginary when $E^2-\rho_n<0$. Since $E\not\in Z_D$, $E^2-\rho_n\not=0$. We then define
\begin{equation}\label{eq:phimE}
    \phi_m(E) =  c_n \begin{pmatrix} \sqrt{\rho_n} \mu_n \\ 
    (E-\xi_m(E))\nu_n \end{pmatrix}, \qquad c_n^{-2} = \rho_n+|E-\xi_m|^2.
\end{equation}
%%%\gb{To be checked.} 
The functions $m\to \phi_m(y;E)$ form a basis of $L^2(\Rm;\Cm^2)$ but no longer an orthonormal one. However, the orthonormalization of this basis  is a bounded operator with bounded inverse as we now show. 

When $m=(n,\epsilon_m)$ while $q=(p,\epsilon_q)$, we verify that $(\phi_m,\phi_q)=0$ when $n\not=p$. This is a direct consequence of the orthogonality of the families $\nu_n$ and $\mu_n$. However, for $q=m':=(m,-\epsilon_m)$, then we have
\[
(\phi_m,\phi_{m'})=\frac{\rho_n+(E-\epsilon_m)(\overline{E+\epsilon_m)}}{\sqrt{\rho_n+|E-\xi_m|^2}\cdot\sqrt{\rho_n+|E+\xi_m|^2}}=\frac{1+\frac{\overline{E+\xi_m}}{E+\xi_m}}{2+2\frac{|E|^2+|\xi_m|^2}{E^2-\xi^2_m}}.
\]
As a consequence, $|(\phi_m,\phi_{m'})|<\frac12$ as we verify. The functions $(\phi_m)$ are thus linearly independent and form a basis of $L^2(\Rm;\Cm^2)$ by completeness of the families $\nu_n$ and $\mu_n$. The procedure of orthonormalization of the (normalized) basis elements $\phi_m$ is therefore a operator of norm bounded by $2$.

%
\subsection{Estimates for unperturbed operator}
%
We now verify that the assumptions made in Hypothesis \ref{hyp:H0H} hold for the unperturbed Dirac operator.

\begin{proposition}\label{prop:dirac}
Estimate (1.) in Hypothesis \ref{hyp:H0H} holds for the Dirac operator.  
\end{proposition}
\begin{proposition}\label{prop:dirac_real}
Estimate (2.) in Hypothesis \ref{hyp:H0H} holds for the Dirac operator. 
\end{proposition}
The first proposition is useful for $s>\frac12$ close to $\frac12$ while the second estimate is useful for $s-1-\eps>0$.  
%%% We first state the following elementary lemma, which is a variant of \cite[Lemma A.1]{ASNSP_1975_4_2_2_151_0}.


\begin{proof}[Proposition \ref{prop:dirac}]
We introduce
\[
 u=(H_0+\lambda) v ,\quad v=(H_0^2-\lambda^2)^{-1}(H_0-\lambda) u. 
\]
We want to show that
\begin{equation}\label{eq:vdouble}
  \|v\|_{H^2_{-s}} \leq C  \|(H_0-\lambda) u \|_{L^2_s} = \|(H_0^2-\lambda^2) v \|_{L^2_s}.
\end{equation}
Using \eqref{delta}, we find
\[ H_0^2-\lambda^2 = \begin{pmatrix} D_x^2+\fa^*\fa-\lambda^2 & 0 \\ 0 & D_x^2+\fa\fa^*-\lambda^2  \end{pmatrix}.\]
Let $\{\nu_n\},\ \{\mu_n\}$ be the (orthonormal basis of) eigenfunctions of $\fa^*\fa, \fa\fa^*$,
\[  \fa^*\fa \nu_n=\rho_n \nu_n,\quad \fa\fa^* \mu_n = \rho_n \mu_n.\]
We also have $\fa\nu_0=0$.
Consider
\[ (D_x^2+\fa^*\fa -\lambda^2) v_1=g_1\]
with the expansion
\[ v_1=\sum_{n\geq0} v_{1n}(x) \nu_n(y)\]
so that 
\[ (D_x^2+\rho_n-\lambda^2) v_{1n} = g_{1n}\]
with obvious notation.

We then use \cite[Theorem A.1 and Remark A.2]{ASNSP_1975_4_2_2_151_0} for the operator $P(D)=D^2$ (and $z=\lambda^2-\rho_n$) to obtain that
\[  \|v_{1n}\|_{H^2_{-s}} \leq C \|(D_x^2+\rho_n-\lambda^2) v_{1n}\|_{L^2_s}, \]
with a constant $C=C(s,a,b)$ independent of $v_{1n}$, $\rho_n>0$, and $\lambda\in J(a,b)$. %\gb{Needs checking. Agmon's result is for $z$ in a compact set.}

We then sum over $n$ using the orthonormality of the families $\nu_n$ and $\mu_n$ to get
\[  \|D_x^2 v_1\|_{L^2_{-s}} + \|v_1\|_{L^2_{-s}} \leq C \|(H_0^2-\lambda^2) v_1\|_{L^2_s}. \]
Now we use the relation
\[  \fa^*\fa v_1 = (H_0^2-\lambda^2)v_1 + \lambda^2v_1 - D_x^2 v_1 \]
and the above inequality to deduce from \eqref{eq:controlfa} bounds for $D_y^2 v_1$ as well $y^2v_1$ in weighted $L^2$ space. This implies that  \eqref{eq:vdouble} holds for $v_1$. 
We perform the same calculation for 
\[ v_2 = \dsum_{n\geq1} v_{2n}(x) \mu_n(y).\]
Thus, \eqref{eq:vdouble} holds. It remains to apply $(H_0+\lambda)$ to $v$, count derivatives and powers of $y$, and obtain
\begin{equation}\label{eq:Hms}
  \|u\|_{H^1_{-s}} \leq C \|(H_0-\lambda) u \|_{L^2_s}.
\end{equation}
This concludes the derivation.
\end{proof}



While the proof of Proposition \ref{prop:dirac} was based on the spectral decomposition of $\fa^*\fa$ leading to that of $H_0^2-\lambda^2$, we now base the proof of Proposition \ref{prop:dirac_real} when $\lambda\equiv E$ is real-valued on the direct plane wave expansion of $H_0-\lambda$ using the eigen-elements $\phi_m(E)$ in \eqref{eq:phimE}.

We first need the following result on the operator $D$:
\begin{lemma}\label{lmm:1dim2}
Let $u\in H^1(\mathbb{R})$, and $s\geq0$. 

When $\lambda\in i\Rm$ and $\epsilon>0$, there is $C=C(s,\epsilon)$ such that
\begin{gather}\label{eq:lambda}
    \big\|u \big\|_{H_{s-1-\epsilon}^1}\leq C(1+|\lambda|)\Big\|\Big(\frac{d}{dx}-\lambda\Big)u\Big\|_{L^2_s}.
\end{gather}

When $\lambda\in \Rm$, there is $C=C(s)$ such that
\begin{gather}\label{eq:lambda2}
\big\|u\big\|_{L^2_s} \leq \frac{C}{|\lambda|(|\lambda| \wedge 1)^s} \Big\|\Big(\frac{d}{dx}-\lambda\Big)u\Big\|_{L^2_s},\quad 
\Big\|\frac{du}{dx}\Big\|_{L^2_s} \leq \frac{C(1+|\lambda|)}{|\lambda|(|\lambda| \wedge 1)^s} \Big\|\Big(\frac{d}{dx}-\lambda\Big)u\Big\|_{L^2_s}.
\end{gather}
\end{lemma}
%%%%
\begin{proof}
    We start with $\lambda\in \Rm$. Since $u\in H^1$, we have $\underset{|x|\to\infty}{\lim}u(x)=0$ and $u$ may be expressed in two ways:
\begin{gather*}
    u(x)=\int_{-\infty}^xf(t)e^{\lambda(x-t)}dt=\int_{\infty}^xf(t)e^{\lambda(x-t)}dt,
    \qquad f(x) = \Big(\frac{d}{dx}-\lambda\Big)u(x).
\end{gather*}
When $x\leq 0$,
\begin{gather*}
    |u(x)|^2\leq \int_{-\infty}^x|f(t)|^2(1+t^2)^sdt\cdot \int_{\infty}^x(1+t^2)^{-s}dt\leq C\|f\|^2_{L_s^2}\cdot(1+x^2)^{1-2s}.
\end{gather*}
A similar estimate holds for $x\geq 0$. Multiplying with $(1+x^2)^{s-1-\epsilon}$ with $\epsilon>0$, we have $u\in L_{t}^2$ and $$\|u\|_{L_t^2}\leq C\|f\|_{L_s^2},\quad t=s-1-\epsilon.$$
Together with $u'=f+\lambda u$, we obtained the inequality \eqref{eq:lambda}.

\medskip 

Consider now the case $\lambda\in\Rm$. Denote $\partial_x\equiv\frac{d}{dx}$. The second inequality in \eqref{eq:lambda2} is a consequence of the first one and the fact that $\partial_x u =(\partial_x-\lambda)u + \lambda u$.
 We may assume $\lambda\geq0$ without loss of generality as the case $\lambda<0$ then holds after $x\to -x$. Let us define $f=(\partial_x-\lambda)u$. Let $\eps>0$ and define $w_\eps(x)=\aver{\eps x}^s$. We find for $s\geq0$ that 
 \[ \Big\|\frac{w_\eps'}{w_\eps}\Big\|_\infty \leq C\eps .\]
 The result in $L^2$ (with norm $\|\cdot\|$) for $\beta=0$ holds. Indeed in the Fourier domain, 
 \[
    \hat u(\xi) = \frac1{i\xi-\lambda} \hat f,\qquad \|u\|\leq \frac1\lambda \|f\|, 
 \]
 by the Parseval equality. 
 For $s>0$ we have
 \[
    (\partial_x-\lambda) (w_\eps u) = w_\eps f - \frac{w_\eps'}{w_\eps} w_\eps u.
 \]
 Thus
 \[
    \|w_\eps u\| \leq  \frac{1}{\lambda} ( \|w_\eps f\| + C\eps \|w_\eps u\|).
 \]
 Choosing $\eps$ so that $C\eps=\frac12(\lambda\wedge 1)$, we deduce that 
 \[
    \|w_\eps u\|  \leq \frac{C}{\lambda}  \|w_\eps f\| .
 \]
 When $\lambda\gtrsim1$, we choose $\eps\sim1$ so that $w_\eps \sim \aver{x}^s$ and the result is clear. When $\lambda\lesssim1$, we use that
 \[  \lambda^s \aver{x}^s \sim \eps^{s} \aver{x}^s  \leq w_\eps(x) \leq \aver{x}^s.
 \]
 This shows that for $\lambda\lesssim1$, $\|\aver{x}^s u\| \leq C \lambda^{-1-s} \|\aver{x^s}f\|$.
\end{proof}

\begin{proof}[Proposition \ref{prop:dirac_real}.] 
    We use the basis $\phi_m(y;E)$  at a fixed $a<\lambda = E<b$ to decompose any smooth function $(x,y)$ as 
    \[ u(x,y) = \dsum_{m\in M} u_m(x) \phi_m(y;E).\] 
  We showed that $\phi_m$ formed a basis equivalent to an orthonormal one in the sense that at each fixed $x$,
 \begin{equation}\label{eq:normbound}
    \|u(x,y) \|^2_{L^2_y} \approx \dsum_m  |u_m(x)|^2.
 \end{equation}
 Here $a\approx b$ when for some constant $C>0$ we have $C^{-1}a\leq b\leq Ca$.
 We also know that
 \[
   (\hat H_0(\xi_m) - \lambda)\phi_m=0.
 \]
 Thus,
 \[  (H_0-\lambda) u_m\phi_m = (D_x-\xi_m) u_m (\varphi_m,0)^t - (D_x+\xi_m) (0,\psi_m)^t.\]
 Hence using the above
 \[
   \|(H_0-\lambda)u \|^2_{L^2_s} \approx \dsum_m \|(D_x-\xi_m) u_m\|^2_{L^2_s} + \|(D_x+\xi_m) u_m\|^2_{L^2_s}.
 \]
 Recall that $\xi_m=\eps_m(\lambda^2-\lambda_n)^{\frac12}$.  

 When $\xi_m\in\Rm$, we use \eqref{eq:lambda} to deduce that for $s-1-\epsilon\geq0$,
 \[ \|u_m\|_{H^1_{s-1-\epsilon}}  \leq C\|(D_x-\xi_m) u_m\|_{L^2_s}. \]

 When $\xi_m\in i\Rm$, we use \eqref{eq:lambda} to deduce that
 \[ 
    \|u_m\|_{H^1_s} \leq \frac{C(1+|\xi_m|)}{|\xi_m| (|\xi_m|\wedge 1)^s}  \|(D_x-\xi_m) \|_{L^2_s}.
 \]
 Since $a<\lambda<b$ and $(a,b)\cap Z_D=\emptyset$, we deduce from the definition of $\xi_m$ that $|\xi_m(\lambda)|$ is bounded below uniformly in that interval. Note, however, that $|\xi_m(\lambda)|$ tends to $0$ as $\lambda$ approaches $Z_D$.

 Summing these equalities over $m$ and using \eqref{eq:normbound}, we deduce that
 \[
    \|u\|^2_{L^2_{s-1-\epsilon}} + \|D_x u\|^2_{L^2_{s-1-\epsilon}} \leq C \|(H_0-\lambda)u \|^2_{L^2_s}
 \]
 for $s\geq0$. The system $(H_0-\lambda)u=f$ may be recast as
 \[
   \fa^* u_2 = f_1 - (D_x-\lambda) u_1,\quad \fa u_1 = f_2 + (D_x+\lambda) u_2
 \]
 From the properties of $\fa$ and $\fa^*$, this implies that 
 \[
    \| y u_j\|_{L^2_{s-1-\epsilon}} + \|D_y u_j \|_{L^2_{s-1-\epsilon}} \leq C \|f\|_{L^2_{s}} + \|u\|_{L^2_{s-1-\epsilon}} \leq C \|f\|_{L^2_{s}}.
 \]
 This concludes the derivation of \eqref{eq:H0real}.
\end{proof}


%
\subsection{Perturbed Dirac operator}
%
%%%%\gb{Needs to be updated.}
We now derive estimates for the resolvent of the perturbed Dirac operator $H=H_0+Q$.  %The main result of this section is the following.
\begin{theorem}\label{thm:estimate}
Condition (3.) in Hypothesis \ref{hyp:H0H} holds for the perturbed Dirac operator when $Q$ satisfies the assumptions in Hypothesis \ref{hyp:Q}.
\end{theorem}

Before proving the theorem, we state the following intermediate result:
\begin{lemma}\label{lmm:qu}
For $a<b,\ h>s>\frac{1}{2}$ and $\lambda\in J(a,b)$, and for $R>0$ sufficient large, there exists a constant $C=C(s,a,b,R)$ such that
\begin{gather}
    \label{ineq:per}
       \|u\|^2_{H_{-s}^1}\leq C\Big( \|(H-\lambda)u\|_{L_s^2}^2+\int_{|x|\leq R}|u(x)|^2dx\Big)
\end{gather}
for all $u\in H_s^1$ and $\lambda \in J(a,b)$.
\end{lemma}
\begin{proof}
From Proposition \ref{prop:dirac} and \eqref{eq:H_0}, we have for $s>\frac{1}{2}$ and $C_1>0$ that 
\begin{gather}\label{nothing}
     \|u\|_{H_{-s}^1}\leq C_1\|(H_0-\lambda)u\|_{L_s^2}\leq C_1(\|(H-\lambda)u\|_{L_s^2}+\|Qu\|_{L_s^2}).
\end{gather}
It remains to estimate $\|Qu\|_{L_s^2}$.  Fix $R>0$. We first estimate $\|Qu\|_{L_s^2(\{|x|\leq R\}})$. Since $|Q(x,y)|$ is bounded,
\begin{gather*}
\|Qu\|_{L_s^2(\{|x|\leq R\})}\leq C\|u\|_{L^2(\{|x|\leq R\})}.
\end{gather*}
We next estimate $\|Qu\|_{L_s^2}(\{|x|\geq R\})$. By \eqref{q},
\begin{gather}\label{ineq:er}
\|Qu\|^2_{L_s^2(\{|x|>R\})}\leq(1+R)^{2s-2h}\iint_{\{|x|\leq R\}}|u|^2|1+x^2|^{-s}dxdy\leq (1+R)^{2s-2h}\|u\|^2_{L_{-s}^2}.
\end{gather}
Choosing $R$ large enough so that $(1+R)^{2s-2h}<\frac{1}{2C_1}$, combined with \eqref{nothing} and \eqref{ineq:er}, we obtain
\begin{gather*}
    \frac{1}{2}\|u\|^2_{H_{-s}^1}\leq 2C_1\Big(\|(H-\lambda)u\|_{L_s^2}^2+C_1(R)\int_{|x|\leq R}|u(x,y)|^2dxdy\Big).
\end{gather*}
This proves \eqref{ineq:per}.
\end{proof}


We are now ready to prove the main theorem.
\begin{proof}[Theorem \ref{thm:estimate}]
By Lemma \ref{lmm:qu}, it suffices to show that for a fixed $R>0$, 
\begin{gather}
    \|u\|_{L^2(|x|\leq R)}\leq C\|(H-\lambda)u\|_{L_s^2},
\end{gather}
with some constant $C$. Assuming the contrary, there is a sequence $\{u_n\}$ of $H_{s}^1$ and a sequence $\{\lambda_n\}$ of $J_+(a,b)$ such that
\begin{gather}
\label{fn}
    \int_{|x|\leq R} |u_n(x,y)|^2dxdy=1,\ (H-\lambda_n)u_n\to 0\ \text{in}\ L_s^2.
\end{gather}
Define
$f_n=(H-\lambda_n)u_n$ and $g_n=f_n-Qu_n$. 
We may assume that $\lambda_n\to\lambda_0$ where $\lambda_0\in [a,b]$ is a non-eigenvalue real number. Indeed, $\text{Im}\ \lambda_0>\eta_0>0$ would imply that for $n$ large enough, we would have $\|u_n\|_{L^2}\leq\eta^{-2}_0\|(H-\lambda_n)u_n\|_{L^2}$, which contradicts \eqref{fn}. From Lemma \ref{lmm:qu}, there exists a constant $C_1$ such that
\begin{gather}\label{un}
    \|u_n\|^2_{H_{-s}^1}\leq C_1\left(\|(H-\lambda_n)u_n\|_{L_s^2}+\|u_n\|_{L^2(|x|\leq R)}\right).
\end{gather}
Thus $\{u_n\}$ is bounded in $H_{-s}^1$. 
So by Rellich's theorem we can select a subsequence of $\{u_n\}$ (which we still denote by $\{u_n\}$) which converges in $L^2_{\{|x|\leq R\}}$ for any $R>0$. We denote this limit as $u_0$. Since $Q$ is bounded, $Qu_n$ also converges to $Qu_0$ in $L^2_{\{|x|\leq R\}}$. This implies that $Qu_n\to Qu_0$ in $L_s^2$ by \eqref{ineq:er} and the fact that $\{u_n\}$ is bounded in $H_{-s}^1$. Thus $g_n\to g_0$ in $L_s^2$, and
\begin{gather}\label{func:gn}
    (H_0-\lambda_n)u_n=f_n-Qu_n=g_n.
\end{gather}
By standard ellipticity results for the Dirac operator $H_0$, we have that
%It is not hard to show that \gb{THIS IS AN ELLIPTICITY CONSTRAINT? NEEDS TO BE ADDED THEN.}
\begin{gather}\label{ineq:h1}
    \|u\|_{H^1(|x|\leq R)}\leq C(R)\left(\|u\|_{L^2(|x|\leq R+1)}+\|H_0u\|_{L^2(|x|\leq R+1)}\right).
\end{gather}
Then by \eqref{func:gn}, \eqref{ineq:h1} and $g_n\to g_0$ in $L^2_s$, we have 
\begin{gather}
    u_n\to u_0\ \text{in}\  H^1(|x|\leq R).
\end{gather}
Applying Lemma \ref{lmm:qu}, we have $u_n\to  u_0\ \text{in}\   H_0^1(|x|\leq R)$
so that
\begin{gather}\label{func:g0}
     \int_{|x|\leq R} |u_0(x,y)|^2dxdy=1,\quad  (H-\lambda_0)u_0=0.
\end{gather}
If we can show $u_0\in L^2$, then $u_0=0$ since $\lambda_0$ is not an eigenvalue. %This leads to a contradiction with \eqref{func:g0}. Next we shall prove $u_0\in L^2$. 
By \eqref{func:gn},
\begin{gather}\label{eq:unn}
    (H_0-\lambda_0)u_n=g_n+(\lambda_n-\lambda_0)u_n.
\end{gather}
As constructed in section \ref{sec:decH0}, $\{\phi_m(\lambda_0)\}$ which we abbreviate as $\phi_m$ below form a basis of $L^2(\mathbb{R},\mathbb{C}^2)$. For $u\in H_s^1$ with $(H_0-\lambda_0)u=f$, we use the decomposition
\begin{gather*}
u=\sum_{m\in M}^\infty u_m(x)\phi_m(y).
\end{gather*} 
We claim that
\begin{gather}\label{c}
    \Big(\frac{d}{dx}+i\xi_m\Big)u_m(x)=\begin{cases}
-i\frac{\lambda}{\xi_{m}}f_m(x)+\frac{i(\xi_{m'}+\lambda)}{\xi_{m'}}f_{m'} &  m\neq 0  \\
-if_m & m=0.
\end{cases}
\end{gather}
It is enough to prove the result for $u\in C_c^\infty$ then pass to the limit since $C_c^\infty$ is dense in $H_s^1$. By recalling $(H_0+\lambda)(H_0-\lambda_0)$ in \eqref{delta} and that $(\hat{H}_0(\xi_m)-\lambda_0)\phi_m=0$, it follows that
\begin{gather*}
(H_0+\lambda)(H_0-\lambda)u=\sum_m(-u''_m(x)-\xi^2_m)\phi_m,\\
(H_0+\lambda)f=(-if'_m(x)-\xi_mf_m(x))\sigma_3\phi_m.
\end{gather*}
%\gb{What is $a_{(n)}$? $g_{(n)}$?}
A linear decomposition gives that, 
\begin{gather}\label{eq:linear_decomp}
    \sigma_3\phi_m=-\frac{\lambda}{\xi_m}\phi_m+\frac{\xi_m+\lambda}{\xi_m}\phi_{m'},\ m\neq 0;\ \sigma_3\phi_m=-\phi_m,\ m=0
\end{gather}
where $\{m,m'\}=(\pm,n)$, so that $\xi_{m'}=-\xi_m$.
Since $u_m, f_m, f_{m'}$ all have limit 0 at $\pm\infty$, we deduce \eqref{c}.

Applying \eqref{c} to \eqref{eq:unn}, for $m=0$, gives
\begin{gather*}
   I_{n,0}:=(u_{n,0},g_{n,0})=-\xi_0\|u_{n,0}\|_{L^2}-\overline{\lambda_n-\lambda_0}\|u_{n,0}\|_{L^2}.
\end{gather*}
For $m\neq 0$, we write, for $m\neq 0$,
\begin{gather*}
    \frac{d}{dx}\Big(\begin{array}{c}
u_{n,m}\\
u_{n,m'}
\end{array}\Big)=\Big(\begin{array}{cc}
-i\xi_{m} & 0\\
0 & i\xi_{m}
\end{array}\Big)\Big(\begin{array}{c}
u_{n,m}\\
u_{n,m'}
\end{array}\Big)+A\Big(\begin{array}{c}
g_{n,m}\\
g_{n,m'}
\end{array}\Big)+(\lambda_n-
\lambda_0)A\Big(\begin{array}{c}
u_{n,m}\\
u_{n,m'}
\end{array}\Big)
\end{gather*}
where $A=\left(\begin{array}{cc}
-\frac{i\lambda_{0}}{\xi_{m}} & \frac{i(-\xi_{m}+\lambda_{0})}{-\xi_{m}i}\\
\frac{i(\xi_{m}+\lambda_{0})}{\xi_{m}i} & \frac{i\lambda_{0}}{\xi_{m}}
\end{array}\right)$. Each entry of $A$ is bounded because $\xi_m$ is bounded away from $0$ and $|\xi_m|$ is unbounded as $n$ grows. Thus
\begin{gather*}
I_{n,m}:=
\Big(\Big(\begin{array}{c}
u_{n,m}\\
u_{n,m'}
\end{array}\Big),\sigma_3A\Big(\begin{array}{c}
g_{n,m}\\
g_{n,m'}
\end{array}\Big)\Big)\\
=\Big(\Big(\begin{array}{c}
u_{n,m}\\
u_{n,m'}
\end{array}\Big),\frac{d}{dx}\sigma_3\Big(\begin{array}{c}
u_{n,m}\\
u_{n,m'}
\end{array}\Big)+\Big(\begin{array}{cc}
i\xi_{m} & 0\\
0 & i\xi_{m}
\end{array}\Big)\Big(\begin{array}{c}
u_{n,m}\\
u_{n,m'}
\end{array}\Big)-(\lambda_n-
\lambda_0)\sigma_3A\Big(\begin{array}{c}
u_{n,m}\\
u_{n,m'}
\end{array}\Big)\Big)\\
=i\xi_m\Big(\|u_{n,m}\|^2_{L^2}+\|u_{n,m'}\|^2_{L^2}\Big)-\overline{\lambda_n-\lambda_0}\Big(\Big(\begin{array}{c}
u_{n,m}\\
u_{n,m'}
\end{array}\Big),\sigma_3 A\Big(\begin{array}{c}
u_{n,m}\\
u_{n,m'}
\end{array}\Big)\Big).
\end{gather*}
Since $\lambda_n\to\lambda_0$ and $A$ is bounded as $\xi_m$ is away from $0$, so for $n$ sufficiently large, there exists a positive constant $U>0$ such that for all $n,m$,
\begin{gather}\label{a}
    \operatorname{Im}I_{n,m}-\operatorname{Re}I_{n,m}\geq U\left(\|u_{n,m}\|^2_{L^2}+\|u_{n,m'}\|^2_{L^2}\right).
\end{gather}
Summing $I_{n,m}$ over $m$,
\begin{gather}\label{b}
\sum_{m\in M,\ \epsilon_m>0}I_{n,m}\xrightarrow[]{n\to\infty}\left(u_0, g^{\text{twist}}_0\right)
\end{gather}
where $\Big(\begin{array}{c}
g^{\text{twist}}_{n,m}\\
g^{\text{twist}}_{n,m'}
\end{array}\Big)=\sigma_3A\Big(\begin{array}{c}
g_{n,m}\\
g_{n,m'}
\end{array}\Big)$, since $g_n^{\text{twist}}\in L_s^2$ and $g_n^{\text{twist}}\to g_0^{\text{twist}}$ in $L_s^2$. Thus, by \eqref{a}, \eqref{b},
\begin{gather}\label{f}
U\|u_n\|^2_{L^2}\leq \operatorname{Im}\sum_{m\in M,\ \epsilon_m>0}I_{n,m}-\operatorname{Re}\sum_{m\in M,\ \epsilon_m>0}I_{n,m}
\xrightarrow[]{n\to\infty}\operatorname{Im}\left(u_0, g^{\text{twist}}_0\right)-\operatorname{Re}\left(u_0, g^{\text{twist}}_0\right).
\end{gather}
We deduce that $\{u_n\}$ is bounded in $L^2$, which implies that $u_0\in L^2$ and hence $u_0=0$. This concludes the proof of the theorem.
\end{proof}

%
\subsection{Scattering matrix and conductivity}\label{sec:scattering}
%
The objective of this section is to prove [H3] for the Dirac operator. 
Fix an energy $E\in\mathbb{R}\setminus Z_H$. We decompose the generalized eigenfunction $\psi^Q_m(x,y;E)$ in the basis of $\phi_m=\phi_m(y;\xi_m(E))$ as
\begin{gather}\label{eigen}
    \psi_m^Q(x,y;E)=\sum_{q\in M}A_q(x)\phi_q(y)=\sum_{q\in M}B_q(x)e^{i\xi_qx}\phi_q(y).
\end{gather}
The decomposition consists of finitely many propagating modes and countably many evanescent modes. We wish to show that in the limit $x\to\pm\infty$, $\psi^Q_m$ is well approximated as a linear combination of propagating modes.

We define $M(E)\subset M$ as the subset of propagating modes at a fixed energy $E$. The cardinality of $M(E)$ was denoted by $\rME$ in Hypothesis [H1](iv).

\begin{proposition}
The generalized eigenfunctions satisfy the following approximate decomposition:
\begin{gather}
\begin{aligned}
\psi^Q_m(x,y)\approx \sum_{q\in M(E)}\alpha_{mq}^\pm e^{i\xi_qx}\phi_q(y),
\end{aligned}
\end{gather}
with respect to norm $|||u|||=\underset{x\in\mathbb{R}}{\max}\ \|u(x,\cdot)\|_{L^2(\cdot)}$; see \eqref{eq:convergenceatinfinity} below for a more precise statement.
\end{proposition}
\begin{proof}
$\psi^Q_m(x,y;E)$ satisfies that $\left(H_0-E\right)\psi^Q_m=-Q\psi^Q_m$.
We decompose $-Q\psi^Q_m$ in the basis $\phi_m(y;\xi_m(E))$ as
$$-Q\psi^Q_m=\sum_{q\in M}a_q(x)\phi_q(y),\qquad a_q(x)=(-Q\psi^Q_m,\ \phi_q(y))_y.$$
%here $a_q(x)=(-Q\Psi_m,\ \phi_q(y))_y$. 
Since $\psi^Q_m\in H_{-s}^1$ for $s>\frac{1}{2}$ and $|Q(x,y)|=O(|x|^{-h})$ for some $h>1$, then $Q\psi^Q_m\in L_t^2$ for some $t>\frac{1}{2}$ and hence $a_q(x)\in L_t^2$.

Let $p=(-\epsilon_q,n)$ be the conjugate of $q=(\epsilon_q, n)$. Then it holds that
\begin{gather}\label{claim}
    (H_0-E)\left[A_q(x)\phi_q(y)+A_p(x)\phi_p(y)\right]=a_q(x)\phi_q(y)+a_p(x)\phi_p(y).
\end{gather}
The proof is then based on a direct computation. %Then the result is a direct corollary. 
From \eqref{claim} and by use of $(\hat H_0(\xi_m)-E)\phi_m(E)=0$ and \eqref{eq:linear_decomp}, we have
\begin{gather*}
     \left[-iA'_q(x)-\xi_qA_q(x)\right]\sigma_3\phi_q+\left[-iA'_p(x)-\xi_pA_p(x)\right]\sigma_3\phi_p=a_q(x)\phi_q(y)+a_p(x)\phi_p(y)\\
=a_q(x)\left[\frac{E}{\xi_q}\sigma_3\phi_q+(1-\frac{E}{\xi_q})\sigma_3\phi_p\right]+a_p(x)\left[(1+\frac{E}{\xi_q})\sigma_3\phi_q-\frac{E}{\xi_q}\sigma_3\phi_p\right].
\end{gather*}
Thus
\begin{gather*}
    -iA'_q(x)-\xi_qA_q(x)=\frac{E}{\xi_q}a_q(x)+\left(1+\frac{E}{\xi_q}\right)a_p(x),
\end{gather*}
which implies that
\begin{gather*}
    B_q'(x)=ie^{-i\xi_qx}\left[\frac{E}{\xi_q}a_q(x)+\left(1+\frac{E}{\xi_q}\right)a_p(x)\right].
\end{gather*}

When $\xi_q\in \mathbb{R}$, since $a_q, a_p\in L_{s}^2$ for some $s>\frac{1}{2}$, then  $a_q, a_p\in L^1$, and hence $B_q(x)$ converges to two constants as $x\to\pm\infty$, i.e., $\underset{x\to\pm\infty}{\lim}B_q(x)=\alpha_{mq}^{\pm}.$
Moreover, 
\begin{gather}\label{e1}
    |B_q(x)-\alpha_{mq}^\pm|=O(|x|^{\frac{1}{2}-t})\ \text{as}\ x\to\pm\infty.
\end{gather}

When $\xi_q\in i\mathbb{R}$, then by applying Lemma \ref{lmm:1dim2}, it follows that there exists a constant independent of $q$ such that
    $
        \|A_q(x)\|_{H_t^1}\leq C\|a_q\|_{L_t^2}+C\|a_p\|_{L_t^2}
    $
    so $\lim_{|x|\to\infty}A_q(x)=0$. Moreover, $(1+|x|)^tA_q(x)\in H^1$, and thus by Sobolev's inequality, there exists a constant $C_1$ such that
\begin{gather*}
    (1+|x|)^t|A_q(x)|\leq C_1\|a_q\|_{L_t^2}.
\end{gather*}
Together with the fact that $\phi_q$ has $L^2$ norm 1 for all $q$, we derive that
\begin{gather}\label{e2}
     \Big\|\sum_{q\in M(E)}A_q(x)\phi_q(y)\Big\|_{L_y^2}\leq \frac{C_2}{(1+|x|)^t}\sum_{\xi_q\in i\mathbb{R}}\|a_q\|_{L_s^2}.
\end{gather}

Therefore, by \eqref{e1} and \eqref{e2},
we have that as $x\to\pm\infty$, 
\begin{gather}\label{eq:convergenceatinfinity}
\Big\|\psi^Q_m(x,y)-\sum_{m\in M(E)} \alpha_{mq}^\pm\phi_q(y)\Big\|_{L_y^2}=O(|x|^{\frac{1}{2}-t})\ \text{as}\ x\to\pm\infty.
\end{gather}
This concludes the proof of the proposition.
\end{proof}
This concludes our analysis of the edge conductivity for the Dirac operator. This establishes for this model the main result of this paper, namely Theorem \ref{thm:sigmascattering}, relating the two natural notions associated to asymmetric transport: edge conductivity and the difference of transmissions in a scattering experiment. 

\section*{Acknowledgment} This work was supported in part by NSF Grants DMS-2306411 and DMS-1908736.


%%%%%%%%%%%%%%%%%
\bibliography{ref} 
\bibliographystyle{siam}


\end{document}

