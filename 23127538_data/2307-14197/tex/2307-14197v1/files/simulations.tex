\section{Simulations}
\label{sec:simulations}

\subsection{Scinet LIght-Cone Simulations}
The Scinet LIght-Cone Simulations \citep[\textsc{SLICS},][]{2015MNRAS.450.2857H,2018MNRAS.481.1337H} consist of over 900 $N$-body mocks based on noise independent initial conditions. The large number of realisations is exploited to estimate the covariance matrices for weak lensing data \citep{2017MNRAS.465.2033J, 2017MNRAS.465.1454H, 2018MNRAS.474..712M, 2022arXiv221105779H} and for combinations of weak lensing and foreground clustering data \citep{2018MNRAS.481.5189B, 2018MNRAS.476.4662V}.

The cubic mocks -- with $L_\mathrm{box} = 505\,\su$ -- simulate a flat $\Lambda$CDM cosmology, described by the cosmology of the WMAP9 + SN + BAO, i.e. ($\Omega_{\rm m}$, $\sigma_8$, $\Omega_{\rm b}$, $w_0$, $h$, $n_{\rm s}$) = (0.2905, 0.826, 0.0447, -1.0, 0.6898, 0.969). They are obtained by running the non-linear double-mesh Poisson solver \textsc{cubep$^3$m} \citep{2013MNRAS.436..540H} to gravitationally evolve $1536^3$ particles -- with a particle mass $m_\mathrm{p} = 2.88\times 10^9~M_\odot/h$ -- on a $3072^3$ grid from $z=99.0$ up to $z=0$.

The dark matter haloes have been selected by applying a spherical over-density halo-finder \citep{2013MNRAS.436..540H}. Their mass function follows precisely the \citet{2001MNRAS.323....1S} fitting function, as shown in Figure~2 of \citet{2018MNRAS.481.1337H}. The redshift of the halo catalogues included in this study is $z=1.041$.
Lastly, given that some halo catalogues have been corrupted at the run time, we are limited to only 139 independent mocks. 

This study, together with the BAM \citep{2022arXiv221110640B}, JackKnife and the DESI covariance matrix comparison papers \citep{DESI_MOCK_CHALLENGE_I} focused on the DESI Emission Line Galaxies (ELGs) sample. Thus, one must assign galaxies on the SLICS halo catalogues. To this end, a HOD model adjusted for ELGs \citep{2020MNRAS.497..581A, 2021MNRAS.504.4667A} is implemented to create a set of 139 galaxy catalogues that are used as reference in all the studies mentioned before. More details about the SLICS galaxy catalogues production can be found in the DESI covariance matrix comparison paper \citep{DESI_MOCK_CHALLENGE_I}.



\subsection{Fast Particle-Mesh}
\label{sec:fastpm}

Accelerated Particle–Mesh (PM) solvers -- such as the \textsc{FastPM} software \citep{Feng2016} -- are able to produce accurate halo populations with respect to the full $N$-body simulations. Thus, they are suitable to accurately simulate large volumes. 

\textsc{FastPM} makes use of a pencil domain-decomposition Poisson solver and Fourier-space four-point differential kernel to compute the force. Additionally, the vanilla leap-frog scheme for the time integration is adjusted to account for the acceleration of velocity during a step, allowing for the accurate tracking of the linear growth of large-scale modes regardless of the number of time steps.

For the current analysis. we have run \textsc{FastPM} with two resolutions, resulting in one set of 778 Low Resolution boxes (LR; $1296^3$ particles) and one set of 141 High Resolution (HR; $1536^3$ particles) catalogues. Both sets output snapshots at the same redshift ($z=1.041$), and have the same box side length ($L_\mathrm{box}=505\,\su$) and cosmology as the \textsc{SLICS} simulations. In contrast to \textsc{SLICS}, the particle mass of the HR simulations is $2.86444 \times 10^9~M_\odot/h$, while for LR it is $4.77\times10^9~M_\odot/h$. The resolution of the force mesh is boosted by a factor of $B=2$ compared to the number of particles per side, for both LR and HR. Lastly, 40 linear steps have been used to evolve the density field from $a=0.05$ to $a=0.96$.

Due to the small number of \textsc{SLICS} galaxy realisations, for 123 runs of the \textsc{FastPM} (LR and HR likewise), we use the \textsc{SLICS} initial conditions. This plays an important role to reduce the effect of the cosmic variance in the clustering statistics and thus in the HOD fitting. \textsc{SLICS} initial density field (initial conditions) has been estimated using the Zel'dovich approximation \citep{1970A&A.....5...84Z}: $\delta^\mathrm{HR}_\mathrm{IC}(\textbf{q}) = -\nabla_\textbf{q}\Psi_Z(\textbf{q})$, where $\Psi_Z(\textbf{q}) = q - q_\mathrm{G}$ is the difference between the Lagrangian particle coordinates $q$ and the Lagrangian coordinates $q_\mathrm{G}$ of a $1536^3$ regular grid. Lastly, the initial conditions had been downgraded to the LR by cutting in Fourier space the high frequency modes larger than the Nyquist frequency corresponding to the LR field.


The halos have been selected from the dark matter field with the Friends-of-Friends halo finder in \textsc{nbodykit} \citep{Hand2018}. During the galaxy assignment process -- Section~\ref{sec:hod_fitting} -- we only make use of halos with a minimum mass of $5.72 \times 10^{10}~M_\odot/h$.

Finally, in Section~\ref{sec:methodology}, when we mention \textsc{FastPM}, we imply for simplicity both HR and LR. We only make the distinction in the results section, i.e. Section~\ref{sec:results}.
