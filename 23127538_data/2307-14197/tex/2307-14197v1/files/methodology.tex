\section{Methodology}
\label{sec:methodology}
\subsection{Clustering computation}

\subsubsection{Two point correlation function}
Mathematically, the two-point correlation function (2PCF) is a continuous function that can describe the clustering of galaxies. However, given the discrete nature of the galaxy distribution in the Universe, the 2PCF is measured using discrete estimators. 
In the case of cubic mocks, one can implement the natural estimator \citep{1974ApJS...28...19P}:
\begin{equation}
    \xi(s, \mu) = \frac{DD(s, \mu)}{RR(s, \mu)} - 1,
\end{equation}
where $DD(s, \mu)$ and $RR(s, \mu)$ are the data and the random pair counts, respectively, as functions of the radial distance 
\begin{equation}
s=\sqrt{s_{\perp}^2 + s_{\|}^2},
\end{equation}
and the cosine of the angle between $\bf{s}$ and the line-of-sight
\begin{equation}
    \mu=\frac{s_{\|}}{s}.
\end{equation}
In the previous equations, $s_{\perp}$ and $s_{\|}$ are the perpendicular ($\perp$) and parallel ($\|$) to the line-of-sight components of $\mathbf{s}$ , respectively. While the $DD$ term is evaluated directly on the data catalogue, $RR$ is calculated theoretically.

In the present analysis, we run \textsc{pyFCFC}\footnote{\url{https://github.com/dforero0896/pyfcfc}} the \textsc{python} wrapper of the Fast Correlation Function Calculator\footnote{\url{https://github.com/cheng-zhao/FCFC}} \citep[FCFC]{2023arXiv230112557Z} to estimate the 2PCF. Lastly, we decompose the 2D 2PCF $(\xi(s, \mu))$ into 1D multipoles $(\xi_{\ell}(s))$ with the help of the Legendre polynomials $L_{\ell}(\mu)$ of order $\ell$, as follows:
\begin{equation}
    \xi_{\ell}(s) = \frac{2\ell + 1}{2}\int_{-1}^1 \xi(s, \mu) L_{\ell}(\mu) d\mu.
\end{equation}

\subsubsection{Power spectrum}
From the mathematical point of view, the power spectrum $P(\bf{k})$ is the Fourier Transform of the 2PCF. However, the limited volume of a survey or a simulation creates mode coupling and makes the two clustering measurements not completely equivalent. Consequently, $P(\bf{k})$ is computed starting from the density field in Fourier space $\delta(k)$, as follows:
\begin{equation}
    \langle \delta(\mathbf{k})\delta(\mathbf{k^{\prime}})\rangle = (2\pi)^3\delta_D(\mathbf{k} + \mathbf{k^{\prime}})P(\mathbf{k}),
\end{equation}
where $\delta_D$ is the Dirac delta function.

As for the 2PCF, we evaluate the multipoles $(P_{\ell}(k))$ of the power spectrum $(P(k, \mu))$:
\begin{equation}
    P_{\ell}(k) = \frac{2\ell + 1}{2}\int_{-1}^1 P(k, \mu) L_{\ell}(\mu) d\mu,
\end{equation}
where $\mu$ is the cosine angle between $\mathbf{k}$ and the line-of-sight, i.e., 
\begin{equation}\label{eqn:mu}
   \mu = k_{\|}/k,\;\; k=\sqrt{k_{\perp}^2 + k_{\|}^2}.
\end{equation}

In practice, we harness the versatility of \textsc{POWSPEC}\footnote{\url{https://github.com/cheng-zhao/powspec}} described in \citet{2021MNRAS.503.1149Z} through its \textsc{python} wrapper\footnote{\url{https://github.com/dforero0896/pypowspec}} to calculate the power spectra and their multipoles starting from the galaxy catalogues. We estimate the density field on a grid of size $512^3$, by applying the Cloud-In-Cell \citep[CIC;][]{10.1093/mnras/stw1229} particle assignment scheme on the catalogues of galaxies. Lastly, we exploit the grid interlacing technique \citep{10.1093/mnras/stw1229} to reduce the alias effects at small scales.

In the current analysis, we show the monopole ($\ell=0$), quadrupole ($\ell=2$) and hexadecapole ($\ell=4$) for both the 2PCF and the power spectrum.

\subsubsection{Bi-spectrum}
The power spectrum and the 2PCF are two-point clustering statistics, but higher order statistics are necessary to characterize more precisely the galaxy distributions. In this study, we also look at the three-point clustering statistics, namely the bi-spectrum $B(\mathbf{k_1}, \mathbf{k_2}, \mathbf{k_3})$, the Fourier pair of the three-point correlation function \citep[e.g. ][]{2002PhR...367....1B}:
\begin{equation}
\delta^D(\mathbf{k_1} + \mathbf{k_2} + \mathbf{k_3})B(\mathbf{k_1}, \mathbf{k_2}, \mathbf{k_3}) = \langle \delta(\mathbf{k_1})\delta(\mathbf{k_2})\delta(\mathbf{k_3})\rangle.
\end{equation}

The three vectors $\mathbf{k_1}, \mathbf{k_2}, \mathbf{k_3}$ are chosen to form a triangle whose two of the three sides are fixed ($k_1 = 0.1 \pm 0.05$ and $k_2 = 0.2 \pm 0.05$), but the angle $\theta_{12}$ between $\mathbf{k_1}$ and $\mathbf{k_2}$ is varied from 0 to $\pi$. In practice, we run the \textsc{bispec}\footnote{\url{https://github.com/cheng-zhao/bispec}} code with a grid size of $512^3$ to compute the monopole of the bispectra, .



\subsection{\textsc{FastPM} HOD model}

The galaxy population and its associated clustering covariance matrix can potentially be influenced by halo properties beyond just mass, as shown in \citet{2023arXiv230501266A}. Nonetheless, such effects are expected to be small for large volume surveys such as DESI and hence we plan to address them in future work. Additionally, the \textsc{FastPM} haloes are less accurate than the ones from a $N$-body simulation, thus we do not expect that the final HOD model and parameters maintain the same physical interpretation.

As a consequence, we can adopt the simple five-parameter HOD model described in \citet{Zheng_2005} to assign galaxies to the \textsc{FastPM} halo catalogues, as long as the resulting clustering and covariance matrix match the reference. Nevertheless, in future work one can study more complex and more adapted models for the studied ELG sample.


The current model assumes that each halo can host at most one central galaxy with a probability $\mathcal{B}(1)=\langle N_\mathrm{cen} \rangle (M_{\rm h})$ dependent on the halo mass $M_{\rm h}$, where $\mathcal{B}(x)$ denotes the Bernoulli distribution and:
\begin{equation}
    \langle N_\mathrm{cen} \rangle (M_{\rm h}) = \frac{1}{2}\left[1+ \mathrm{erf}\left( \frac{\log M_{\rm h} - \log M_\mathrm{min}}{\sigma_{\log M}} \right) \right]
\end{equation}
with $\mathrm{erf}$ the error function:
\begin{equation}    
    \mathrm{erf}(x) = \frac{2}{\sqrt{\pi}} \int_0^x e^{-u^2}du.
\end{equation}
$\log M_\mathrm{min}$ is the halo mass at which the probability to host a central galaxy is one half and $\sigma_{\log M}$ controls the steepness of the transition from a probability of one to zero. Lastly, the positions and velocities of the central galaxies are precisely the values of their parent haloes.

In contrast, the number of satellite galaxies $n_\mathrm{sat}$ per halo is sampled from a Poisson distribution $\mathcal{P}(n_\mathrm{sat} | \langle N_\mathrm{sat} \rangle (M_{\rm h}))$ with the mean:
\begin{equation}
    \label{eq:hod_satelite_model}
    \langle N_\mathrm{sat} \rangle (M_{\rm h}) = \left( \frac{M_{\rm h} - M_0}{M_1} \right)^\alpha,
\end{equation}
where $M_0$ is a minimum halo mass threshold below which haloes cannot host satellite galaxies and together with $M_1$ indicating the halo mass at which one halo hosts on average one satellite galaxy, and $\alpha$ is the power-law index. Furthermore, the positions and velocities of the satellite galaxies follow the Navarro-Frenk-White \citep[NFW]{1996ApJ...462..563N} density profile.

In the interest of adjusting the smaller scales and the quadrupole, we introduce a velocity dispersion factor ($v_\mathrm{disp}$) for the velocity parallel $(\|)$ to the line-of-sight (i.e \textbf{$\mathrm{o}Z$} in the current case) of the satellite galaxies, in addition to the five HOD parameters:
\begin{equation}
    v^\mathrm{sat, new}_{\|} =\left( v^\mathrm{sat, old}_{\|} - v^\mathrm{halo}_{\|} \right) \times v_\mathrm{disp} + v^\mathrm{halo}_{\|},
\end{equation}
where $v^\mathrm{halo}_{\|}$ is the velocity parallel to the line-of-sight of the satellites' parent halo. Finally, the six free parameters are fitted so that the resulting \textsc{FastPM} clustering matches the \textsc{SLICS} one.

\subsection{HOD fitting}
\label{sec:hod_fitting}

We would like to draw the attention of the reader to Table~\ref{tab:symbols_definitions}. It contains a summary of important symbols related to the HOD fitting.
\begin{table}
    \centering
    \begin{tabular}{c | c}
        \hline
         Notation & Meaning  \\
         \hline
         $N_\mathrm{mocks}^\mathrm{cov}=123$  & The number of \textsc{FastPM} and \textsc{SLICS} pairs \\
                                          & that share the same initial conditions.\\
                                          & These catalogues have been used \\
                                          & to compute $\textbf{C}_s$, Eq.\eqref{eq:covariance_cs}, part of $\Sigma_\mathrm{diff}$. \\
         $N_\mathrm{mocks}^\mathrm{fit}=20$  & The number of \textsc{FastPM} and \textsc{SLICS} pairs for which \\
                                        & we have computed the clustering during the  HOD  \\
                                       & fitting described in Section~\ref{sec:the_first_hod_step} and Section~\ref{sec:the_second_hod_step}.  \\
         $\Sigma_\mathrm{diag}$   &  Eq.~\eqref{eq:diagonal_cov_mat}: Diagonal matrix used during \\
                                  &  the first step of the HOD fitting, see Section~\ref{sec:the_first_hod_step}.\\
         $\sigma_{n_\mathrm{g}}$ &  Estimation of the galaxy number density noise \\
                                 & used in $\Sigma_\mathrm{diag}$. \\
                                    & Standard deviation of 139 \textsc{SLICS} mocks,\\
                                    &  divided by $\sqrt{139}$. \\ 
         $\Sigma_\mathrm{diff}$   &  Eq.~\eqref{eq:difference_cov_mat}: Difference covariance matrix used during\\
                                  &  the second step of the HOD fitting, see Section~\ref{sec:the_second_hod_step}.\\
         $\sigma'_{n_\mathrm{g}}$ &  Estimation of the galaxy number density noise\\
                                  &  used in $\Sigma_\mathrm{diff}$. \\
                                  & Standard deviation of 139 \textsc{SLICS} mocks, \\
                                  & divided by $\sqrt{N_\mathrm{mocks}^\mathrm{fit}}$. \\
          $\Sigma_\chi$           & Eq.~\eqref{eq:inverse_unbiased_covaraince}: The covariance matrix used to compute \\
                                  & the $\chi^2_\nu$, Eq.\eqref{eq:reduced_chi2}. It is not used for fitting. \\
         \hline
         
    \end{tabular}
    \caption{A summary of some of the most important and possibly confusing notations and their meaning.}
    \label{tab:symbols_definitions}
\end{table}

With the aim of finding the best-fitting \textsc{FastPM} clustering, we run a HOD Optimization Routine (\textsc{HODOR}\footnote{\url{https://github.com/Andrei-EPFL/HODOR}}). It uses the \textsc{Halotools} \citep{halotools} package to define and apply the HOD model and \textsc{PyMultiNest} \citep{2014A&A...564A.125B} the \textsc{python} wrapper of \textsc{MultiNest} \citep{2008MNRAS.384..449F,2009MNRAS.398.1601F,2019OJAp....2E..10F} to sample the six HOD parameters. 

\textsc{MultiNest} is a sampler based on Bayes' theorem that provides the maximum likelihood (best-fitting) parameters, as well as the posterior probability distribution of parameters alongside the Bayesian evidence. Bayes' theorem combines prior knowledge about the $\Theta$ parameters of a model $M$ with information from the data $D$ to calculate the posterior probability density of the $\Theta$ parameters:
\begin{equation}
    p(\Theta |D, M) = \frac{p(D | \Theta, M) p(\Theta | M)} {p(D|M)},
\end{equation}
where $p(\Theta | M)$ is the prior distribution of $\Theta$ of the model $M$, $p(D | \Theta, M)$ is the likelihood, and $p(D|M)$ is a normalizing factor called Bayesian evidence.

The uniform prior distributions that we impose on all six parameters are shown in Table~\ref{tab:parameter_priors}.
Furthermore, we approximate the likelihood by a multivariate Gaussian:
\begin{equation}
    p(D | \Theta, M) = \mathcal{L}(\Theta) \sim \mathrm{e}^{-\chi ^ 2 (\Theta) / 2},
\end{equation}
with the chi-squared:
\begin{equation}
    \chi ^ 2 (\Theta) = \mathbf{v}^\mathrm{T} \mathbf{C}^{-1} \mathbf{v},
\end{equation}
where $\mathbf{v}$ is the difference between the data and model vectors $\mathbf{v} = S_\mathrm{data} - S_\mathrm{model}(\Theta)$, and $\mathbf{C}$ is the covariance matrix.

\begin{table}
    \centering
    \begin{tabular}{c | c | c | c | c | c | c}
        \hline
         name & $\log \frac{M_\mathrm{min}}{M_\odot}$ & $\sigma_{\log M}$ & $\log \frac{M_1}{M_\odot}$ & $\kappa$ & $\alpha$ & $v_\mathrm{disp}$ \\
         \hline
         min & 11.6 & 0.01 & 9 & 0 & 0 & 0.7  \\
         max & 13.6 & 4.01 & 14 & 20 & 1.3 & 1.5  \\
         \hline
         
    \end{tabular}
    \caption{The limits of the uniform prior distributions included in the HOD fitting. Note that $M_0$ from Eq.~\eqref{eq:hod_satelite_model} is $M_0 \equiv \kappa\times M_\mathrm{min}$. $M_\odot$ denotes the solar mass.}
    \label{tab:parameter_priors}
\end{table}

The purpose of a covariance matrix $\mathbf{C}$ is to estimate the noise in the data, in the context of a noise-free model.
Nevertheless, the peculiarity of this study is that both the model ($S_\mathrm{model}(\Theta)$, \textsc{FastPM}) and the data ($S_\mathrm{data}$, \textsc{SLICS}) are affected by noise. 
Due to the small volume of the \textsc{SLICS} and \textsc{FastPM} boxes, the cosmic variance component of the noise would be larger than the expected precision of ongoing surveys such as DESI. However, since the simulations have been run with matching initial conditions, the relevant noise factor is no longer the cosmic variance but rather the difference in the gravitational evolution.
Hence, the mock covariance estimated by \textsc{SLICS} or \textsc{FastPM} substantially over-estimates the error for our fittings. The more suitable noise term is the accumulated noise due to gravitational evolution while starting with exactly the same initial conditions.

Consequently, in order to more appropriately estimate the noise, we perform a two-step HOD fitting as schematically shown in Figure~\ref{fig:flowchart_twosteps}:
\begin{enumerate}
    \item we fit the monopole and quadrupole of the 2PCF $[\xi_0, \xi_2]$ and the galaxy number density $n_\mathrm{g}$ using a diagonal covariance matrix $(\Sigma_\mathrm{diag})$ and thus obtain an initial-guess (IG) best-fitting \textsc{FastPM} galaxy catalogues (IG-\textsc{FastPM}), see Section~\ref{sec:the_first_hod_step}; 
    \item we compute the differences $[\Delta_0, \Delta_2]$ between the clustering (monopole, quadrupole) of the IG best-fitting \textsc{FastPM} and the \textsc{SLICS} galaxy catalogues; we use these differences to calculate a new covariance matrix $(\Sigma_\mathrm{diff})$ with which we perform again the fitting, see Section~\ref{sec:the_second_hod_step}.
\end{enumerate}
In both cases, we use 20 \textsc{FastPM} (F) and 20 \textsc{SLICS} (S) halo boxes $(N_\mathrm{mocks}^\mathrm{fit} = 20)$ -- sharing the same initial conditions -- for the purpose of decreasing the noise. Nonetheless, the average $\Bar{n}_\mathrm{g}^\mathrm{S}$ is computed using 139 realisations, while the average $\Bar{n}_\mathrm{g}^\mathrm{F}$ is calculated using the 20 realisations included in the HOD fitting. There are three main reasons behind this discrepancy: first, it quickly becomes expensive to apply galaxies using HOD to more than 20 \textsc{FastPM} simulations; second, the number of \textsc{SLICS} reference simulations has to be the same as for \textsc{FastPM}, so that the cosmic variance is reduced in the clustering by the shared initial conditions;  third, the noise in the galaxy number density is not reduced by the shared initial conditions, thus one needs more realisations to estimate a (practically) noiseless \textsc{SLICS} reference galaxy number density.   
The galaxy number density is an important constraint as it governs the shot-noise which has a significant role in the covariance matrix.

% Figure environment removed





\subsubsection{The First Step} 
\label{sec:the_first_hod_step}
Initially, we perform the HOD fitting on the monopole and the quadrupole of the 2PCF, together with the galaxy number density. Hence, the data vector $S_\mathrm{data}$ is formed by concatenating their respective averages for the \textsc{SLICS} (S) mocks: $S_\mathrm{data} = [\Bar{\xi}_0^\mathrm{\textsc{S}}, \Bar{\xi}_2^\mathrm{\textsc{S}}, \Bar{n}_\mathrm{g}^\mathrm{\textsc{S}}]$. Similarly, the model vector $S_\mathrm{model}$ is determined from the \textsc{FastPM} (F) boxes: $S_\mathrm{model} = [\Bar{\xi}_0^\mathrm{\textsc{F}}, \Bar{\xi}_2^\mathrm{\textsc{F}}, \Bar{n}_\mathrm{g}^\mathrm{\textsc{F}}]$. 

Considering that the computing time of clustering measurements scales with the maximum separation, we need a large enough upper-limit to constrain relevant parameters, but small enough to keep a reasonable execution time for model evaluation during the HOD fitting. Additionally, since we are interested in capturing the non-linear effects, the lower-limit is set to 0. Consequently, the monopole and the quadrupole of the 2PCF are evaluated for $s \in [0, 50]\,\su$, with a bin size of $5\,\su$. Thus, $s$ is an array containing 10 elements $(s_1, \ldots, s_{10})$. 

As previously argued, in the first step, there is no appropriate noise estimation. Therefore, we can use an approximate covariance matrix that enables us to proceed to the second step and calculate a more suitable one. In this regard, we create a diagonal covariance matrix:
\begin{equation}
\label{eq:diagonal_cov_mat}
\Sigma_\mathrm{diag}=
    \begin{pmatrix}

    \sigma^2_1 &        &               &            &        &               &                        \\
               & \ddots &               &            &        &               &                        \\
               &        & \sigma^2_{10} &            &        &               &                        \\
               &        &               & \sigma^2_1 &        &               &                        \\
               &        &               &            & \ddots &               &                        \\
               &        &               &            &        & \sigma^2_{10} &                        \\
               &        &               &            &        &               & \sigma^2_{n_\mathrm{g}}\\
    
    \end{pmatrix},
\end{equation}
where the first 20 elements are defined as follows:
\begin{equation}
    \sigma_i=\frac{3}{s^2_i}, ~ i = 1, \ldots, 10.
\end{equation}
This selection of the diagonal covariance matrix is based on an examination of the $s^2\sigma_\mathrm{SLICS}(s)$ values, where $\sigma_\mathrm{SLICS}(s)$ represents the standard deviation of the \textsc{SLICS} 2PCF. Notably, the highest value is approximately three; hence, we initially approximate all values as three for simplicity.

The last element $\sigma_{n_\mathrm{g}}$ is computed as the standard deviation of 139 \textsc{SLICS} galaxy number densities, divided by $\sqrt{139}$, so that it estimates the uncertainty corresponding to the average of $139$ realisations. The strong constraint on the $n_g$ improves the fitting time, as \textsc{HODOR} initially evaluates the goodness-of-fit based only on the $\Bar{n}_\mathrm{g}^\mathrm{F}$ and $\Bar{n}_\mathrm{g}^\mathrm{S}$, and does not compute the clustering if $\Bar{n}_\mathrm{g}^\mathrm{F}$ is $10\sigma$ away from the reference. Additionally, the lack of covariance terms in the covariance matrix should, as well, decrease the convergence time.


Finally, we apply the best-fitting HOD model to all $N_\mathrm{mocks}^\mathrm{cov}=123$ \textsc{FastPM} halo boxes that share the initial conditions with the \textsc{SLICS} mocks to obtain the IG-\textsc{FastPM}.

\subsubsection{The Second Step}
\label{sec:the_second_hod_step}
To examine the influence of smaller scales on the HOD fitting, we compute the following for both \textsc{SLICS} and \textsc{FastPM}:
\begin{enumerate}
    \item the power spectrum for $k \in [0.02,\,k_\mathrm{max}]\,\ku$, with a bin size of $0.02\,\ku$,
    \item the 2PCF for $s \in [s_\mathrm{min},\,50]\,\su$, with a bin size of $5\,\su$,
\end{enumerate}
where the values of $k_\mathrm{max}$ and $s_\mathrm{min}$ are presented in Table~\ref{tab:fitting_intervals_n_bins}.
Consequently, we create the data and model vectors as follows:
\begin{enumerate}
    \item $S_\mathrm{data} = [\Bar{P}_0^\mathrm{\textsc{S}}, \Bar{P}_2^\mathrm{\textsc{S}}, \Bar{n}_\mathrm{g}^\mathrm{\textsc{S}}]$ and $S_\mathrm{model}=[\Bar{P}_0^\mathrm{\textsc{F}}, \Bar{P}_2^\mathrm{\textsc{F}}, \Bar{n}_\mathrm{g}^\mathrm{\textsc{F}}]$;
    \item $S_\mathrm{data} = [\Bar{\xi}_0^\mathrm{\textsc{S}}, \Bar{\xi}_2^\mathrm{\textsc{S}}, \Bar{n}_\mathrm{g}^\mathrm{\textsc{S}}]$ and $S_\mathrm{model}=[\Bar{\xi}_0^\mathrm{\textsc{F}}, \Bar{\xi}_2^\mathrm{\textsc{F}}, \Bar{n}_\mathrm{g}^\mathrm{\textsc{F}}]$.
\end{enumerate}

\begin{table}
    \centering
    \begin{tabular}{c | c | c | c}
        \hline
        name & Large & Medium & Small \\
        \hline
         $k_\mathrm{max}\,[\ku]$ & 0.5 & 0.4 & 0.3 \\
         $N^\ell_\mathrm{bins}$ & 24 & 19 & 14\\
         \hline
         $s_\mathrm{min}\,[\su]$  & 0 & 5  & 10 \\
         $N^\ell_\mathrm{bins}$ & 10 & 9 & 8\\
         \hline
         
    \end{tabular}
    \caption{The fitting ranges for the HOD fitting process described in Section~\ref{sec:the_second_hod_step}: $k \in [0.02,\,k_\mathrm{max}]\,\ku$ and $s \in [s_\mathrm{min},\,50]\,\su$. $N^\ell_\mathrm{bins}$ is the number of bins per multipole $\ell$.}
    \label{tab:fitting_intervals_n_bins}
\end{table}


In order to estimate the noise in the context of shared initial conditions between \textsc{SLICS} and \textsc{FastPM}, we use the $N_\mathrm{mocks}^\mathrm{cov}$ galaxy boxes of both \textsc{SLICS} and IG-\textsc{FastPM}, along with their corresponding clustering measurements (power spectrum or 2PCF). Furthermore, we introduce $\Delta_{\ell, \mathrm{IG}}^P = P_{\ell, \mathrm{IG}}^\mathrm{\textsc{F}}(k) - P_\ell^\mathrm{\textsc{S}}(k)$ and $\Delta_{\ell, \mathrm{IG}}^\xi = \xi_{\ell, \mathrm{IG}}^\mathrm{\textsc{F}}(s) - \xi_\ell^\mathrm{\textsc{S}}(s)$, as well as the generic vector $\Delta^\mathrm{IG}(x) = [\Delta_{0,\mathrm{IG}}, \Delta_{2,\mathrm{IG}}]$ to express the difference between the \textsc{SLICS} and the IG-\textsc{FastPM} galaxy clustering that share the initial conditions. Here, the variable $x$ represents either $k$ or $s$.

Taking advantage of the previous definitions, we further define a matrix \textbf{M} with the following elements:
\begin{equation}
    \textbf{M}_{ij}=\Delta_i^\mathrm{IG}(x_j) - \Bar{\Delta}^\mathrm{IG}(x_j),~i = 1, 2, ..., N_\mathrm{mocks}^\mathrm{cov},~x_j\in[x_\mathrm{min}, x_\mathrm{max}],
\end{equation}
where $\Delta_i^\mathrm{IG}$ denotes the vector corresponding to the $i-$th (\textsc{SLICS}, IG-\textsc{FastPM}) pair, $\Bar{\Delta}^\mathrm{IG}$ represents the mean vector over all (\textsc{SLICS}, IG-\textsc{FastPM}) pairs and $[x_\mathrm{min}, x_\mathrm{max}]$ defines the interval of points involved in the fitting, see Table~\ref{tab:fitting_intervals_n_bins}. Starting from this matrix and its transpose, we calculate the sample covariance matrix $\textbf{C}_s$ as follows:
\begin{equation}
    \label{eq:covariance_cs}
    \textbf{C}_s = \frac{1}{N_\mathrm{mocks}^\mathrm{cov} - 1} \textbf{M}^\mathrm{T}\textbf{M}.
\end{equation}

Lastly, we calculate the $\sigma'_{n_\mathrm{g}}$ as the standard deviation of 139 \textsc{SLICS} galaxy number densities, divided by $\sqrt{N_\mathrm{mocks}^\mathrm{fit}}$ -- so that it estimates the uncertainty corresponding to the average of $N_\mathrm{mocks}^\mathrm{fit}$ realisations -- and we attach it to the $\textbf{C}_s$ to obtain the final covariance matrix used in the HOD fitting:
\begin{equation}
\label{eq:difference_cov_mat}
\Sigma_\mathrm{diff}\equiv
    \begin{pmatrix}

    \textbf{C}_s &     0                     \\
           0     & \sigma'^2_{n_\mathrm{g}}  \\
    \end{pmatrix}.
\end{equation}
Note that while the error estimate for the clustering is based on the difference in clustering due to matched initial condition, the error of the number density is directly computed from the \textsc{SLICS} realisations, as we aim to constrain the absolute number density, which has strong effect on the final clustering covariance.


\subsubsection{Goodness-of-fit}
\label{sec:reduced_chi2}
In this section, we define a reduced $\chi^2$ -- $\chi^2_\nu$ -- that expresses the goodness-of-fit for the average of $N_\mathrm{mocks}^\mathrm{fit}$ \textsc{FastPM} galaxy clustering realisations with respect the \textsc{SLICS} reference, i.e. the $n_\mathrm{g}$ is not included:
\begin{equation}
    \label{eq:reduced_chi2}
    \chi_\nu ^ 2 = N_\mathrm{mocks}^\mathrm{fit} \times \frac{\mathbf{\Delta}^\mathrm{T} \Sigma_\chi^{-1} \mathbf{\Delta}}{\nu},
\end{equation}
where $\mathbf{\Delta}$ denotes the difference between \textsc{FastPM} and \textsc{SLICS} clustering -- monopole and quadrupole -- and $\nu = N_\mathrm{bins} - N_\mathrm{params}$, with 
\begin{enumerate}
    \item $N_\mathrm{params}=6$ -- the number of free parameters;
    \item $N_\mathrm{bins} = 2 \times N^\ell_\mathrm{bins}$ -- the length of the $\Delta^\mathrm{IG}(x)$ vector, see Table~\ref{tab:fitting_intervals_n_bins}.
\end{enumerate}

The $\Sigma_\chi^{-1}$ is the unbiased estimate of the inverse covariance matrix \citep{2007A&A...464..399H}:
\begin{equation}
    \label{eq:inverse_unbiased_covaraince}
    \Sigma_\chi^{-1} = \textbf{C}_s^{-1}\frac{N_\mathrm{mocks}^\mathrm{cov} - N_\mathrm{bins} -2}{N_\mathrm{mocks}^\mathrm{cov} - 1}, 
\end{equation}
where $\textbf{C}_s$ is defined in Eq.~\eqref{eq:covariance_cs}. \citet{2016MNRAS.456L.132S, 2022MNRAS.510.3207P} have shown that this correction may not be the optimal choice for accurately determining the uncertainty of the parameters. However, since our main focus is on obtaining the best-fitting clustering and assessing its goodness-of-fit, it remains a reasonable correction.

Finally, as we fit the average of $N_\mathrm{mocks}^\mathrm{fit}$ realisations, we must scale the covariance matrix $\textbf{C}_s$ by a factor of $1 / N_\mathrm{mocks}^\mathrm{fit}$. As a consequence, the $N_\mathrm{mocks}^\mathrm{fit}$ factor appears in Eq.\eqref{eq:reduced_chi2}.






\subsection{Covariance matrix comparison}
\label{sec:cov_mat_constrain_power}

Given that the main goal is to have a robust estimation of the uncertainty on the cosmological parameters, we want to compare the constraining power of the covariance matrices. To this end, we fit the 123 individual \textsc{SLICS} clustering (monopole and quadrupole) with the following models:
\begin{equation}
    P^\ell_\mathrm{model}(k) = b_\ell \times \Bar{P}^\ell_\mathrm{123, SLICS}(k)
\end{equation}
and
\begin{equation}
    \xi^\ell_\mathrm{model}(s) = b_\ell \times \Bar{\xi}^\ell_\mathrm{123, SLICS}(s),
\end{equation}
where $\Bar{P}^\ell_\mathrm{123, SLICS}(k)$ and $\Bar{\xi}^\ell_\mathrm{123, SLICS}(s)$ are averages of the 123 realisations and $b_\ell$ denotes the two free parameters.

Moreover, the covariance matrices are computed similarly to the Eq.~\eqref{eq:inverse_unbiased_covaraince}, but using 778 LR \textsc{FastPM} realisations. The fitting is performed using \textsc{PyMultiNest}, for different fitting ranges ($k \in [0.02,\,\mathcal{K}]\,\ku$ and $s \in [\mathcal{S},\,200]\,\su$, see Table~\ref{tab:fitting_intervals_covariance_comparison}) for the purpose of comparing the effect of the covariance matrices at different scales. The largest fitting intervals are chosen so that they cover the nominal scales included in the BAO and RSD analyses, i.e. $\mathcal{K}\approx0.2\,\ku$ and $\mathcal{S}\approx 20 \,\su$ \citep[e.g. ][]{2020MNRAS.499.5527T, 2021MNRAS.501.5616D}. Finally, the shown values are the average ($b_\ell$) and standard deviation ($\sigma_{b_\ell}$) of the marginalised posterior $p(b_\ell)$  and covariance ($\mathcal{R}[b_0, b_2]$) of the posterior distribution of $b_0$ and $b_2$, $p(b_0, b_2)$. By construction, the values of $b_\ell$ should be one.

The main reason why we perform such a simplified test is to avoid the systematic errors that can arise due to the modelling. Consequently, the comparison between the quoted $\sigma_{b_\ell}$ and $\mathcal{R}[b_0, b_2]$ should be directly related to the differences in \textsc{FastPM} covariance matrices. We, nevertheless, reckon that these comparisons do not show how the errors on the parameters of a realistic BAO/RSD model would behave.





\begin{table}
    \centering
    \begin{tabular}{c | c | c | c | c}
        \hline
         $\mathcal{K}\,[\ku]$ & 0.1 & 0.15 & 0.2 & 0.25  \\
         \hline
         $\mathcal{S}\,[\su]$  & 15 & 20  & 25 & 30 \\
         \hline
         
    \end{tabular}
    \caption{The fitting ranges -- $k \in [0.02,\,\mathcal{K}]\,\ku$ and $s \in [\mathcal{S},\,200]\,\su$ used in the clustering fitting described in Section~\ref{sec:cov_mat_constrain_power}}
    \label{tab:fitting_intervals_covariance_comparison}
\end{table}

