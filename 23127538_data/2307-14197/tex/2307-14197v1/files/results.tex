\section{Results}
\label{sec:tension_parameter}
\label{sec:results}



One of the challenges of HOD fitting is addressing the high precision imposed by large volume surveys such as DESI because it requires prohibitively many large volume simulations. Figure~\ref{fig:clustering_error_bars} illustrates this issue as a comparison between $\sigma_\mathrm{20,SLICS}$\footnote{This would be the noise level in a hypothetical case where \textsc{SLICS} and \textsc{FastPM} would not share the initial conditions.} the noise corresponding to the average of $N_\mathrm{mocks}^\mathrm{fit}=20$ \textsc{SLICS} clustering realisations and the expected DESI Y5\footnote{The DESI Year 5 error is estimated by rescaling $\sigma_\mathrm{20,SLICS}$ to match the Y5 ELG sample volume, which is assumed to be $24\,\mathrm{Gpc}^3\,h^{-3}$.} and Y1\footnote{The DESI Year 1 error is estimated by rescaling $\sigma_\mathrm{20,SLICS}$ to match the Y1 ELG sample volume, which is assumed to be one third of the Y5 volume.} errors of the ELG sample. It is obvious that $N_\mathrm{mocks}^\mathrm{fit}$ \textsc{SLICS} realisations do not reach the required precision\footnote{A simple calculation reveals that one would need 192 \textsc{SLICS} realisations to meet the DESI Y5 precision requirements.}.

% Figure environment removed

In order to overcome this issue, we employ the novel matched initial conditions simulations (\textsc{SLICS} and \textsc{FastPM}). In this case, the effect of the cosmic variance on the clustering difference is mostly removed. Therefore, as discussed in Section~\ref{sec:hod_fitting}, the relevant error estimate is given by the covariance matrix of the clustering difference between the two simulations. Given the fact that we use $N_\mathrm{mocks}^\mathrm{fit}$ pairs to perform the HOD fitting, the covariance matrix must be rescaled by $N_\mathrm{mocks}^\mathrm{fit}$. The square root of the diagonal of the resulting covariance matrix is illustrated with an dashed orange line in Figure~\ref{fig:clustering_error_bars}. One can observe that the matched initial conditions significantly reduce the noise to values below $\sigma_\mathrm{20,SLICS}$.

Furthermore, we would like to highlight that the precision depicted by the dashed orange line is either better than or equal to DESI Y1 precision up to $k \approx 0.25\,\ku$. Consequently, the results presented in this paper are precise enough with respect to the requirements of further DESI Y1 analyses. Nonetheless, it might be necessary to readdress this study for the full DESI sample, to account for even lower noise levels. For this, one could use the 1800 \textsc{AbacusSummit} \citep{Maksimova:2021ynf} $N$-body $0.5\,\mathrm{Gpc}/h$ cubic boxes.

In addition, Figure~\ref{fig:clustering_error_bars} illustrates the comparison between $\sigma_\mathrm{DIFF}$ and the square root of the diagonal elements of $\Sigma_\mathrm{diff}$. In this context, $\sigma_\mathrm{DIFF}$ represents the standard deviation of the differences between the best-fitting \textsc{FastPM} (obtained from the second HOD fitting step) and \textsc{SLICS} clustering, further divided by $\sqrt{N_\mathrm{mocks}^\mathrm{fit}}$. Ideally, an iterative HOD fitting process should be performed to ensure a robust $\Sigma_\mathrm{diff}$, but the close agreement between $\sigma_\mathrm{DIFF}$ and the diagonal elements of $\Sigma_\mathrm{diff}$ suggests that $\Sigma_\mathrm{diff}$ has approximately converged after a single iteration. A more detailed argument in support of the convergence of $\Sigma_\mathrm{diff}$ is presented in Section~\ref{sec:appendix_robustnesstest}.



As pointed out in Section~\ref{sec:hod_fitting}, it is important that the \textsc{FastPM} galaxy catalogues reproduce the \textsc{SLICS} shot-noise. Examining the \textsc{FastPM} galaxy number densities of all HOD fitting cases, we observed that the largest deviation, $| \Bar{n}_\mathrm{g}^\mathrm{S} - \Bar{n}_\mathrm{g}^\mathrm{F} |  / \sigma_{n_\mathrm{g}}'$, is approximately $0.5\,\sigma$, but most values are below $0.2\,\sigma$. This strongly supports that the galaxy number density is well constrained and that it is safe to define a $\chi^2_\nu$ without including $n_\mathrm{g}$ -- see Eq.\eqref{eq:reduced_chi2}.

Furthermore, the values of the $\chi^2_\nu$ are subject to uncertainties due to the finite number of realisations used to estimate the covariance matrix and the limited number of HOD realisations per halo catalogue. The most significant uncertainty, $\approx 27$ per cent, arises from the limited number of HOD realisations. The remaining values are below 20 per cent, see Section~\ref{sec:appendix_chi2uncertainty} for more details. The $\chi^2_\nu$ is simply used as a metric to evaluate the goodness-of-fit. For this reason it is important to consider that it is affected by a large uncertainty when comparing its magnitude to the expected value of one.





The primary focus of this paper is to investigate the limits of the \textsc{FastPM} capabilities to model the non-linear scales captured by $N$-body simulations. Furthermore, we study the effect of fitting to successively more non-linear scales and either Fourier or configuration space statistics on the \textsc{FastPM} covariance matrix. 


\subsection{Power spectrum fitting}
\label{sec:power_spectrum_fitting}

Figure~\ref{fig:1296_1536_DIFF_PK_kmax_05_04_03} shows the results of the HOD fitting performed on the power spectrum for three different $k$ intervals, defined in Table~\ref{tab:fitting_intervals_n_bins}. The second, third and fifth rows display the difference in the clustering scaled by the difference error. We remind the reader that this error is smaller than the expected one for the given volume, due to the matched initial conditions between the two simulations, see Figure~\ref{fig:clustering_error_bars}.

% Figure environment removed


The best fitting monopoles and quadrupoles are within $\pm 1 \sigma$ for most scales. Moreover, the results for the HR \textsc{FastPM} -- presented with dashed line -- are only marginally better than the ones for LR \textsc{FastPM}. Given the modest difference between the performances of the two resolutions, we believe that the LR \textsc{FastPM} is precise enough to describe the two-point clustering to non-linear scales for the DESI Y1 ELG-like galaxies.

Considering that we only fit the first two even multipoles, there is no guarantee that the third one would match the reference. Nevertheless, the fifth row of Figure~\ref{fig:1296_1536_DIFF_PK_kmax_05_04_03} illustrates that fitting the monopole and quadrupole to smaller scales improves the agreement of the hexadecapole. For instance, fitting on the Large interval pushes the $\ell=4$ multipole within $\pm2\sigma$ for $k<0.4\,\ku$, whereas for Medium and Small intervals, the hexadecapole is placed within $\pm2\sigma$ only for $k<0.3\,\ku$ or $k<0.2\,\ku$, respectively.



Due to the fact that the power spectrum is affected by the window function, it is not obvious that a good matching in Fourier space translates as a good matching in Configuration space. Thus, we compute and display the corresponding 2PCF in the right-hand side of Figure~\ref{fig:1296_1536_DIFF_PK_kmax_05_04_03}.
Most monopoles and quadrupoles agree within $\pm2\sigma$ with \textsc{SLICS} for separations larger than $20\,\su$. This suggests that it is possible to obtain a reasonable 2PCF above a certain minimum separation, even when performing HOD fitting on the power spectrum.
However, fitting on the Medium and Large intervals, the $2\sigma$ matching goes down to a separation of $10\,\su$.

In contrast, for separations smaller than $5\,\su$, the non-linear effects become dominant, making it difficult to replicate the velocity field. This is why increasing the fitting range up to $k_\mathrm{max}=0.5$ can improve the monopole but not the quadrupole.
Lastly, the 2PCF hexadecapole exhibits a bias of over $3 \sigma$ for $s<50\,\su$ in all six cases.

After a more qualitative description of the results, we present the $\chi^2_\nu$ values in the upper panels of Figure~\ref{fig:reduced_chi2}. Generally, the HR \textsc{FastPM} produces lower $\chi^2_\nu$ values than the LR, as expected from Figure~\ref{fig:1296_1536_DIFF_PK_kmax_05_04_03}.
However, $\chi^2_\nu [P(0.02, k_\mathrm{max})]\simeq 1$, which reiterates that by fitting the monopole and quadrupole of the power spectrum up to the three $k_\mathrm{max}$ values, one can achieve a good match with the \textsc{SLICS} reference, within the DESI Y1 precision even with LR. In addition, $\chi^2_\nu [\xi(20, 50)]\simeq 2$ for the small fitting interval of the LR power spectrum, reinforcing the fact that one can get a reasonable 2PCF above a certain minimum separtion threshold when the fitting is performed on the power spectrum.


% Figure environment removed

Additionally, we can observe the behaviour of $\chi^2_\nu$ when it is estimated on different intervals than those used for the fitting. When the fitting is performed on the Large interval, the $\chi^2_\nu\simeq 1$ for all smaller intervals, regardless of the resolution. However, fitting on the Medium interval shows that the difference between HR and LR becomes more significant for $k>0.4 \,\ku$ (see also Figure~\ref{fig:1296_1536_DIFF_PK_kmax_05_04_03}): the $\chi^2_\nu\simeq 2$ for LR, while for HR, it is close to one. These findings imply that fitting up to $k\leq0.4 \,\ku$ is satisfactory for HR \textsc{FastPM}, whereas smaller scales play a more significant role in LR.

Furthermore, fitting on the Small interval shows that although $\chi^2_\nu [P(0.02, 0.3)]\simeq 1$, it is much larger for $k>0.3 \,\ku$, indicating strong clustering divergence beyond that value (see Figure~\ref{fig:1296_1536_DIFF_PK_kmax_05_04_03}). Therefore, both LR and HR benefit from considering the clustering information contained in smaller scales $k>0.3 \,\ku$.



\subsection{2PCF fitting}
\label{sec:2pcf_fitting}


When the HOD fitting is performed on the power spectrum, the minimum 2PCF $\chi^2_\nu$ is $\chi^2_\nu[\xi(10, 50)] \approx 2$. While this translates to a $2\sigma$ agreement down to the separation of $10\,\su$ between \textsc{FastPM} and \textsc{SLICS} 2PCF, we test whether fitting directly the 2PCF can improve the results.
Therefore, in this section, we analyse the outcomes of the HOD fitting performed on the 2PCF monopole and quadrupole, for $s \in [s_\mathrm{min}, 50]\,\su$, see Table~\ref{tab:fitting_intervals_n_bins}.


Figure~\ref{fig:1296_1536_DIFF_CF_smin_0_5_10} presents the monopole, quadrupole and hexadecapole of the 2PCF computed for $s \in [0, 200]\,\su$ as well as the tensions between the \textsc{FastPM} and \textsc{SLICS}. The \textsc{FastPM} clustering typically falls within $2 \sigma$ of the reference for scales larger than $50\,\su$ and is largely unaffected by the fitting scenario.
However, the HR monopoles are consistently closer to the reference than LR monopoles by approximately $0.5\sigma$ at scales larger than $\approx150\,\su$.

Including the smallest scales (Large interval) in the HOD fitting, we observe a 1 to $2\sigma$ agreement with the reference for $s < 10\,\su$ in both the monopole and quadrupole. However, at intermediate scales $s \in [10, 50]\,\su$, the monopole is significantly biased, exhibiting a deviation of $3 \sigma$. In contrast, for the Medium and Small scenarios, we notice that the tensions for the monopole and quadrupole at intermediate scales drop to $1\sigma$, while the smallest scales can get biased by more than $3 \sigma$. Nevertheless, they match better the reference than the power spectrum HOD fitting case.
Lastly, the hexadecapole does not depend on the resolution nor the fitting range and is strongly biased for $s<60\,\su$, showing no improvement compared to the power spectrum fitting.


% Figure environment removed

As in the previous subsection, we test the clustering statistics of the best-fitting \textsc{FastPM} boxes that were not included in the HOD fitting, i.e. the power spectrum in the Figure~\ref{fig:1296_1536_DIFF_CF_smin_0_5_10}. The first observation is that these \textsc{FastPM} power spectra do not fit as well the reference as the ones from Figure~\ref{fig:1296_1536_DIFF_PK_kmax_05_04_03}. On one hand, for the HR case and Medium and Small fitting intervals a $\pm1\sigma$ matching is possible up to $k=0.4\,\ku$ and $k=0.3\,\ku$, respectively. On the other hand, the LR \textsc{FastPM} allows a good matching up to $k\approx0.2\,\ku$ for the same fitting intervals.
While the $s_\mathrm{min}=0$ case has a good matching quadrupole up to $k\approx0.4\,\ku$, its monopole follows similar trend to the 2PCF monopole, i.e. the intermediate scales $k \in [0.25, 0.4]\,\ku$ are biased and the rest are mostly within $2\sigma$ deviation.
Lastly, the hexadecapole is within $\pm2\sigma$ up to $k\approx0.3\,\ku$ for the Large fitting interval and up to $k\approx0.2\,\ku$  for the other cases. 


A quantitative evidence that directly fitting the 2PCF yields superior matching of the 2PCF compared to fitting the power spectrum is displayed in Figure~\ref{fig:reduced_chi2}. The majority of the $\chi^2_\nu$ values in the lower-right panel are lower compared to those in the upper-right panel. Furthermore, fitting on the Small interval ($s_\mathrm{min} = 10$), the $\chi^2_\nu\approx1$, indicating that the 2PCF is in good agreement with the \textsc{SLICS} reference above a certain minimum separation. 
The almost constant $\chi^2_\nu$ for the Large fitting interval in the lower-right panel of Figure~\ref{fig:reduced_chi2} is explained by the discrepancy at the intermediate scales of the monopole for the Large fitting interval in Figure~\ref{fig:1296_1536_DIFF_CF_smin_0_5_10}.
Lastly, as in the previous fitting scenario, the HR \textsc{FastPM} generally provides a lower $\chi^2_\nu$ than the LR. In contrast, only the HR simulations can provide a $\chi^2_\nu<2$ to both the 2PCF and the power spectra, and only when fitting with the Medium and Small intervals to 2PCF. Although not shown in the aforementioned figure, it is important to note that $\chi^2_\nu[P(0.02, 0.2)] = 2.4$ for the 2PCF Small interval LR case.


\subsection{Bi-spectrum comparison}

Taking into account that the covariance matrix depends on the bi-spectrum \citep{2018MNRAS.480.2535B}, we aim to understand its behaviour when incorporating various scales in the HOD fitting. Figure~\ref{fig:1296_1536_bispec_20} compares the average bi-spectrum of the 20 best-fitting \textsc{FastPM} boxes with the one computed on the corresponding \textsc{SLICS} boxes. It is evident that by increasing the fitting range to include smaller scales, the \textsc{FastPM} bi-spectrum changes to the extent that for $k_\mathrm{max}=0.5$, the tension ranges from 1 to $2\sigma$. In contrast, when fitting the 2PCF the resulting bi-spectrum is more biased, i.e. the lowest deviation is $\approx5\sigma$, for $s_\mathrm{min}=0$ case. Finally, there is no significant improvement in terms of the goodness-of-fit between the HR and LR. 



% Figure environment removed


In the previous sections, we compare the HR and LR \textsc{FastPM} with \textsc{SLICS} using the two-point clustering of the 20 cubic mocks included in the HOD fitting. The HR simulations perform better than LR to model the extremely non-linear scales, such as $k\approx0.5\,\ku$, $s\approx0\,\su$. In contrast, at mildly non-linear scales ($k\approx0.3\,\ku$, $s\approx10\,\su$) that are more relevant to BAO and RSD analyses \citep[e.g. ][]{2020MNRAS.499.5527T, 2021MNRAS.501.5616D}, LR and HR show similar performance. Moreover, Figure~\ref{fig:1296_1536_bispec_20} suggests that the bi-spectrum does not depend strongly on the resolution. Nevertheless, the computing cost of HR is significantly higher than for LR and given the small difference in precision, we argue it is optimal to use LR \textsc{FastPM} for further analyses.


Furthermore, in Figure~\ref{fig:bispec_1296_780}, we compare the average bi-spectra -- computed from 778 LR \textsc{FastPM} realisations -- corresponding to the six HOD fitting cases, see Table~\ref{tab:fitting_intervals_n_bins}. In this and the next figures, we choose the $k_\mathrm{max}=0.5$ case as reference because:
\begin{enumerate}
    \item Figure~\ref{fig:reduced_chi2} shows that the best-fitting power spectrum provides $\chi^2_\nu\approx1$;
    \item Figure~\ref{fig:1296_1536_bispec_20} implies that the corresponding bi-spectrum is the closest to the \textsc{SLICS} reference. 
\end{enumerate}
% Figure environment removed

One can notice that $s_\mathrm{min}=0$ bi-spectrum is at most 5 per cent different than the reference, while the rest can reach 15 per cent discrepancies. The $k_\mathrm{max}=0.3$ and $k_\mathrm{max}=0.4$ cases are 1 to 2 per cent different from each other and similarly for $s_\mathrm{min}=5$ and $s_\mathrm{min}=10$.

\subsection{Covariance comparison}

Having studied the behaviour of the bi-spectra, we now want to understand their effect on the covariance matrices of the clustering (power spectrum and 2PCF).

\subsubsection{Power spectrum covariance}

Figure~\ref{fig:correlation_together_2pcf_pspec_1296_780} presents the correlation matrices and the corresponding standard deviations $\sigma_\ell$ for the monopole and quadrupole of the power spectrum. The following pairs ($k_\mathrm{max}=0.4$, $k_\mathrm{max}=0.3$), ($s_\mathrm{min}=5$, $s_\mathrm{min}=10$) and ($s_\mathrm{min}=0$, $k_\mathrm{max}=0.5$) have very similar correlation matrices, thus we only show three cases. However, we introduce all of them in Appendix~\ref{sec:appendix_cov_mat_comp}.

The similarity to the reference correlation matrix diminishes in the following order:  $s_\mathrm{min}=0$, $k_\mathrm{max}=0.4$, $k_\mathrm{max}=0.3$, $s_\mathrm{min}=5$ and $s_\mathrm{min}=10$. However, for the largest scales of the quadrupole ($k<0.15\,\ku$), the correlation coefficients are practically the same for all cases. 

% Figure environment removed

The standard deviations in the lowest panels show similar trends.  The $s_\mathrm{min}=0$ case is within two per cent of the reference case. The $k_\mathrm{max}=0.4$, $k_\mathrm{max}=0.3$ cases overestimate the $\sigma_\ell(k)$ by approximately two per cent for $k<0.27\,\ku$ and by $\approx 5$ per cent for smaller scales. Nevertheless, these two cases are consistent with each other within one to two per cent. In contrast, $s_\mathrm{min}=5$ and $s_\mathrm{min}=10$ can overestimate the $\sigma_\ell(k)$ by $\approx 2$ to 5 per cent for $k<0.27\,\ku$ and by 10 to 20 per cent for smaller scales. These two cases are also consistent with each other for most scales, except for the quadrupole $k>0.3\,\ku$.
These findings are in agreement with the trends observed in the bi-spectrum comparison in Figure~\ref{fig:bispec_1296_780}.

In order to quantify the differences between the covariance matrices we adopt the method described in Section~\ref{sec:cov_mat_constrain_power} and thus obtain the results displayed in Figures~\ref{fig:together_pspec_2pcf_fit_avg_b0_b2} and \ref{fig:pspec_fit_std_b0_b2}. The first one reveals that none of the six covariance matrices bias the two fitting parameters, regardless of the fitting range. We only present here the results of one fitting range, however all cases can be found in Appendix~\ref{sec:appendix_cov_mat_comp}.

% Figure environment removed

% Figure environment removed

Examining the uncertainty on $b_0$ in Figure~\ref{fig:pspec_fit_std_b0_b2}, we observe that including the smaller scales the discrepancy between the error estimates of the six covariance matrices increases, as we expect from Figure~\ref{fig:correlation_together_2pcf_pspec_1296_780}, reaching a maximum of $\approx 20$ per cent larger error estimation for the $s_\mathrm{min}=10$ covariance at $\mathcal{K}=0.25\,\ku$. Moreover, each of the following pairs ($k_\mathrm{max}=0.4$, $k_\mathrm{max}=0.3$), ($s_\mathrm{min}=5$, $s_\mathrm{min}=10$) and ($s_\mathrm{min}=0$, $k_\mathrm{max}=0.5$) provide coherent estimations of the uncertainty, which is consistent with the observations on the correlation matrices and standard deviations. Lastly, a five per cent consensus between all six covariance matrices is achieved when we fit the power spectra on the $k\in[0.02, 0.1]\,\ku$.

The agreement between covariance matrices on $\sigma_2$ is much better than $\sigma_0$. Given the error bars, the six methods estimate the uncertainty with a two per cent tolerance with each other for all $\mathcal{K}$ values.

Finally, all six covariance matrices provide values of $\mathcal{R}[b_0,b_2]$  that are consistent at the level of 5 per cent, given the error bars and up to $\mathcal{K}=0.2\,\ku$. For $\mathcal{K}=0.25\,\ku$, the largest discrepancy is shown by $s_\mathrm{min}=10$ case which underestimates the value of $\mathcal{R}[b_0,b_2]$ by almost 50 per cent. The other cases underestimate $\mathcal{R}[b_0,b_2]$ by 10 to 20 per cent. 




\subsubsection{2PCF covariance}


Comparing the correlation matrices obtained from 778 2PCF in Figure~\ref{fig:correlation_together_2pcf_pspec_1296_780}, one can observe that the largest differences occur at the smallest scales $s<30\,\su$. Similarly to the power spectrum correlation matrices, the same pairs of cases show resembling behaviours at all scales. Equivalent qualitative comments can be made about the ratios of the standard deviations. Nonetheless, all cases are within $\approx2$ per cent from each other for $s>30\,\su$, while at smaller scales, the differences can get larger than $\approx20$ per cent. 



Following the method described in Section~\ref{sec:cov_mat_constrain_power}, we obtain the results shown in Figures~\ref{fig:together_pspec_2pcf_fit_avg_b0_b2} and \ref{fig:2pcf_fit_std_b0_b2}. The first figure proves that all six covariance matrices provide unbiased measurements of $b_\ell$ parameters.

Resembling the power spectrum fitting case, the six estimations of $\sigma_{b_0}$ in Figure~\ref{fig:2pcf_fit_std_b0_b2} are in better agreement when the smallest scales are not included in the 2PCF fitting, however for $\mathcal{S}\geq20\,\su$ they are all within $\approx5$ per cent from each other. The largest discrepancy is around 10 per cent and occurs between $s_\mathrm{min}=0$ and $s_\mathrm{min}=10$ for $\mathcal{S}=15\,\su$. The values of $\sigma_{b_2}$ are all consistent within $\approx2$ per cent, given the error bars. Interestingly, including the smaller scales, the $\mathcal{R}[b_0, b_2]$ values are more coherent, such that all discrepancies are within five per cent, given the error bars and for $\mathcal{S}<30\,\su$. In contrast, when $\mathcal{S}=30\,\su$ the $s_\mathrm{min}=10$ and $s_\mathrm{min}=5$ provide values $\mathcal{R}[b_0, b_2]$ that are approximately ten per cent larger than the reference, but nevertheless consistent within the error bars.

% Figure environment removed

