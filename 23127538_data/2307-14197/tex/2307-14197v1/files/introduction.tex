
\section{Introduction}

The study of Large Scale Structure of the Universe has significantly improved in the last two decades leading to Baryon Oscillation Spectroscopic Survey \citep[BOSS;][]{2017MNRAS.470.2617A} and extended-BOSS  \citep[eBOSS;][]{2021PhRvD.103h3533A} surveys. They have published the largest 3D map of over 2 millions galaxies and quasars \citep{2021PhRvD.103h3533A}. This has allowed the measurement of cosmological parameters to a percent-level precision studying Baryonic Acoustic Oscillations (BAO) and Redshift Space Distortions (RSD). 

Currently, the Dark Energy Spectroscopic Instrument \citep[DESI;][]{2013arXiv1308.0847L,  2022AJ....164..207A} is a five years long spectroscopic survey that will outperform previous surveys by a an order of magnitude\citep{2016arXiv161100036D}, aiming to constrain the cosmological parameters with precision at a sub-percent level. With its 5000 robotically controlled optical fibres \citep{2023AJ....165....9S, 2023arXiv230606310M,2016arXiv161100037D}, DESI will scan a third of the sky to map 40 millions galaxies \citep{2023ApJ...943...68L} and quasars \citep{2023AJ....165..124A}. Only after the five-month Survey Validation \citep{2023arXiv230606307D}, DESI has measured the spectra of more than one million galaxies leading to the recent Early Data Release (EDR) \citep{2023arXiv230606308D}.

Based on the DESI Legacy Imaging Surveys \citep{2017PASP..129f4101Z,2019AJ....157..168D,dr9}, there are five types of targets that are selected \citep{2023AJ....165...50M} on which optical fibres are assigned \citep{fba} to measure and analyse their spectra \citep{2023AJ....165..144G,redrock2023,2023arXiv230510426B}: Milky Way Stars \citep[MWS;][]{2020RNAAS...4..188A,2022arXiv220808514C}, Bright Galaxies \citep[BGS;][]{2020RNAAS...4..187R,2022arXiv220808512H}, Luminous Red Galaxies \citep[LRG;][]{2020RNAAS...4..181Z,2023AJ....165...58Z}, Emission Line Galaxies \citep[ELG;][]{2020RNAAS...4..180R,2023AJ....165..126R}, quasars \citep[QSO;][]{2020RNAAS...4..179Y, 2023ApJ...944..107C}. Such a complex system requires pipelines to optimise the observations \citep{2023arXiv230606309S,expcalc}.

The sub-percent precision measurements expected from ongoing and future surveys require careful analyses of the systematic effects.
To this end, the DESI Mock Challenge was launched as a series of studies and projects to build and validate the methodology for the cosmological analysis. In particular, one must find the most robust way to estimate the uncertainty of the measurements \citep{DESI_MOCK_CHALLENGE_I}.
To achieve this goal, one needs to create multiple realistic simulations of the large-scale structure, which is required to lower the noise on covariance matrix and to describe accurately the non-linear scales.



On one hand, the $N$-body simulations  -- e.g. \citep[\textsc{SLICS};][]{2018MNRAS.481.1337H}, \citep[\textsc{UNIT};][]{Chuang:2018ega} and \citep[\textsc{AbacusSummit};][]{Maksimova:2021ynf} -- are accurate, but they are computationally expensive. Therefore, they are mainly used in testing models and systematic effects, and it becomes impractical with the increase of mapped volume to have enough realisations to estimate and test covariance matrices. Consequently, faster but less accurate techniques have been developed -- e.g. \citep[\textsc{EZmocks};][]{Chuang:2014vfa, Zarrouk:2020hha, 2021MNRAS.503.1149Z}, \citep[\textsc{PATCHY};][]{10.1093/mnrasl/slt172},  \citep[\textsc{BAM};][]{2020MNRAS.491.2565B, 2019MNRAS.483L..58B,2020MNRAS.493..586P} -- to be run multiple times and estimate robustly the uncertainty.


In this study, we investigate the possibility to tune \textsc{FastPM} catalogues to reproduce the clustering of \textsc{SLICS} reference with the final goal of estimating the covariance matrix. In contrast to the other fast methods, \textsc{FastPM} uses accelerated particle-mesh solvers to evolve the dark-matter field, that should provide a higher accuracy of the large scale structure. The additional accuracy provided by \textsc{FastPM} can be important given the unprecedented statistical power of the DESI survey. Therefore, the \textsc{FastPM} covariance matrix is compared with different methods (\textsc{BAM}, \textsc{EZmock}, Jackknife \citep{Zhang_jackknife_paper}, analytical models \citep{Xu13, Wad20a, Wad20b}) in a parallel DESI Mock Challenge paper \citep{DESI_MOCK_CHALLENGE_I}.

Fundamentally similar to standard $N$-body simulations, \textsc{FastPM} evolves the dark matter field into the cosmic web, the skeleton of the large scale structure in the Universe \citep[e.g.][]{2010gfe..book.....M,2018ARA&A..56..435W}. After the dark matter haloes are selected, one must implement galaxy-halo connection models \citep{2018ARA&A..56..435W} to assign galaxies.
There are more empirically inspired models such as the Halo Occupation Distribution \citep[HOD; e.g.][]{Benson2000,Seljak2000,Peacock2000,White2001,Berlind2002,Cooray2002} and Sub-Halo Abundance Matching \citep[SHAM; e.g.][]{2004ApJ...609...35K,2004ApJ...614..533T,2004MNRAS.353..189V} and more physically inspired ones such as full hydro-dynamical simulations \citep[e.g.][]{2010MNRAS.402.1536S,2015MNRAS.446..521S,dubois2014,2017MNRAS.465.2936M,2018MNRAS.473.4077P,2019MNRAS.486.2827D} or Semi Analytical Models \citep[SAMs; e.g.][]{2011MNRAS.413..101G,2014MNRAS.439..264G}. In this case, we adopt a HOD model as it is one the most efficient ways to create mock galaxy catalogues. 

The purpose of the current paper is to show that the galaxy assignment process on \textsc{FastPM} halo catalogues with a HOD model can be adjusted to match the reference \textsc{SLICS} galaxy clustering. We thus compare the impact of different clustering statistics and examine the effects of various scales on the HOD fitting. Finally, we calculate covariance matrices for all the studied scenarios and perform a comparison to understand the influence of the HOD modelling on the parameter uncertainty.

 
In Section~\ref{sec:simulations}, we present the \textsc{SLICS} and \textsc{FastPM} simulations. The methodology that we follow is detailed in Section~\ref{sec:methodology}. We describe our results on the HOD fitting performance and the covariance matrix comparison in Section~\ref{sec:results}. In the end, Section~\ref{sec:conclusion} concludes the article.


