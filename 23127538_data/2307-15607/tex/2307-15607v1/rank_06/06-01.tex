\subsection{Family \textnumero6.1}\label{subsection:06-01}

The toric Landau--Ginzburg model for this family is given by the Laurent polynomial
\begin{gather*}
  x + 2 y + x^{-1} y^{2} + z + 3 x^{-1} y + z^{-1} + y^{-1} + 3 x^{-1} + x^{-1} y^{-1}
\end{gather*}
(see~\cite[Family \textnumero6.1]{cheltsov2018katzarkov}). It is a Minkowski polynomial \textnumero284 (see~\cite[Appendix~B: bucket~107]{akhtar2012minkowski}).

The pencil \(\mathcal{S}\) is defined by the equation
\begin{gather*}
  X^{2} Y Z + 2 X Y^{2} Z + Y^{3} Z + X Y Z^{2} + 3 Y^{2} Z T + X Y T^{2} + X Z T^{2} + 3 Y Z T^{2} + Z T^{3} = \lambda X Y Z T.
\end{gather*}
Members \(\mathcal{S}_{\lambda}\) of the pencil are irreducible for any parameter \(\lambda \in \mathbb{P}^1\) except
\begin{gather*}
  \mathcal{S}_{\infty} = S_{(X)} + S_{(Y)} + S_{(Z)} + S_{(T)}.
\end{gather*}
The base locus of the pencil \(\mathcal{S}\) consists of the following curves:
\begin{gather*}
  C_{1} = C_{(X, Z)}, \;
  C_{2} = C_{(Y, Z)}, \;
  C_{3} = C_{(Y, T)}, \;
  C_{4} = C_{(Z, T)}, \\
  C_{5} = C_{(X, Y + T)}, \;
  C_{6} = C_{(Y, X + T)}, \;
  C_{7} = C_{(T, X Z + (X + Y)^2)}.
\end{gather*}

Their linear equivalence classes on the generic member \(\mathcal{S}_{\Bbbk}\) of the pencil satisfy the following relations:
\begin{gather*}
  \begin{pmatrix}
    [\mathcal{S}_{\Bbbk} \cdot S_{(X)}] - [H_{\mathcal{S}}] \\
    [\mathcal{S}_{\Bbbk} \cdot S_{(Y)}] - [H_{\mathcal{S}}] \\
    [\mathcal{S}_{\Bbbk} \cdot S_{(Z)}] - [H_{\mathcal{S}}] \\
    [\mathcal{S}_{\Bbbk} \cdot S_{(T)}] - [H_{\mathcal{S}}]
  \end{pmatrix} = 
  \begin{pmatrix}
    1 & 0 & 0 & 0 & 3 & 0 & 0 & -1 \\
    0 & 1 & 2 & 0 & 0 & 1 & 0 & -1 \\
    1 & 1 & 0 & 2 & 0 & 0 & 0 & -1 \\
    0 & 0 & 1 & 1 & 0 & 0 & 1 & -1
  \end{pmatrix} \cdot
  \begin{pmatrix}
    [C_1] \\ \cdots \\ [C_{7}] \\ [H_{\mathcal{S}}]
  \end{pmatrix} = 0.
\end{gather*}
We can reduce the number of linear equivalence classes using these relations:
\begin{gather*}
  \begin{pmatrix}
    [C_{1}] \\ [C_{2}] \\ [C_{6}] \\ [C_{7}]
  \end{pmatrix} = 
  \begin{pmatrix}
    0 & 0 & -3 & 1 \\
    0 & -2 & 3 & 0 \\
    -2 & 2 & -3 & 1 \\
    -1 & -1 & 0 & 1
  \end{pmatrix} \cdot
  \begin{pmatrix}
    [C_{3}] \\ [C_{4}] \\ [C_{5}] \\ [H_{\mathcal{S}}]
  \end{pmatrix}.
\end{gather*}

Put \(\mu (\mu - 1) = (\lambda + 1)^{-1}\). For a general choice of \(\lambda \in \mathbb{C}\) the surface \(\mathcal{S}_{\lambda}\) has the following singularities:
\begin{itemize}\setlength{\itemindent}{2cm}
\item[\(P_{1} = P_{(X, Y, T)}\):] type \(\mathbb{A}_2\) with the quadratic term \(X \cdot Y\);
\item[\(P_{2} = P_{(Y, Z, T)}\):] type \(\mathbb{A}_3\) with the quadratic term \(Y \cdot Z\);
\item[\(P_{3} = P_{(Y, T, X + Z)}\):] type \(\mathbb{A}_1\) with the quadratic term \(Y (X + Y + Z - \lambda T) + (Y^2 + T^2)\);
\item[\(P_{4} = P_{(Z, T, X + Y)}\):] type \(\mathbb{A}_3\) with the quadratic term \((\mu Z - (\mu - 1) T) \cdot ((\mu - 1) Z - \mu T)\);
\item[\(P_{5} = P_{(X, \mu Y + (\mu- 1) Z, Y + T)}\):] type \(\mathbb{A}_2\) with the quadratic term
  \[
    X \cdot (\mu^2 (\mu - 1) (X + Y + T) + (2 \mu - 1) ((\mu - 1) Z -  \mu T));
  \]
\item[\(P_{6} = P_{(X, (\mu - 1) Y + \mu Z, Y + T)}\):] type \(\mathbb{A}_2\) with the quadratic term
  \[
    X \cdot (\mu (\mu - 1)^2 (X + Y + T) - (2 \mu - 1) (\mu Z - (\mu - 1) T)).
  \]
\end{itemize}

The \(\mathbb{Q}\)-valued intersection matrix on the group \(A_{\mathcal{S}}\) has the following form:
\begin{table}[H]
  \renewcommand\arraystretch{1.42}
  \begin{tabular}{|c||c|c|c|c|}
    \hline
    \(\bullet\) & \(C_{3}\) & \(C_{4}\) & \(C_{5}\) & \(H_{\mathcal{S}}\) \\
    \hline
    \hline
    \(C_{3}\) & $-\frac{1}{12}$ & $\frac{1}{4}$ & $\frac{1}{3}$ & $1$ \\
    \hline
    \(C_{4}\) & $\frac{1}{4}$ & $-\frac{1}{4}$ & $0$ & $1$ \\
    \hline
    \(C_{5}\) & $\frac{1}{3}$ & $0$ & $0$ & $1$ \\
    \hline
    \(H_{\mathcal{S}}\) & $1$ & $1$ & $1$ & $4$ \\
    \hline
  \end{tabular}.
\end{table}
Note that the intersection matrix has the rank 4.

Let \(\widetilde{\mathcal{S}_{\lambda}}\) and \(\widetilde{\mathcal{S}_{\Bbbk}}\) be the minimal resolution of a general member \(\mathcal{S}_{\lambda}\) and the generic member \(\mathcal{S}_{\Bbbk}\) of the pencil, respectively. Denote by \(L_{\lambda} \subset \Pic(\widetilde{\mathcal{S}_{\lambda}})\) and \(L_{\mathcal{S}} \subset \Pic(\widetilde{\mathcal{S}_{\Bbbk}})\) the subgroups generated by linear equivalence classes of exceptional divisors of the resolution and by linear equivalence classes of strict transforms \(\widetilde{C_i}\), \(\widetilde{H_{\mathcal{S}}}\) of the above-introduced curves. By construction the subgroup \(L_{\lambda}\) is equipped with the \(\Gal(\Bbbk)\)-action, and the subgroup \(L_{\mathcal{S}}\) can be identified with the subgroup \(L_{\lambda}^{\Gal(\Bbbk)} \subset L_{\lambda}\). Its Galois orbits can be described as
\[
  E_1^1, E_1^2, E_2^1, E_2^2, E_2^3, E_3^1, E_4^1 + E_4^3, E_4^2, E_5^1 + E_6^1, E_5^2 + E_6^2, \widetilde{C_{3}}, \widetilde{C_{4}}, \widetilde{C_{5}}, \widetilde{H_{\mathcal{S}}}.
\]

The intersection matrix on the lattice \(L_{\lambda}\) has the following form:
\begin{table}[H]
  \begin{tabular}{|c||cc|ccc|c|ccc|cc|cc|cccc|}
    \hline
    \(\bullet\) & \(E_1^1\) & \(E_1^2\) & \(E_2^1\) & \(E_2^2\) & \(E_2^3\) & \(E_3^1\) & \(E_4^1\) & \(E_4^2\) & \(E_4^3\) & \(E_5^1\) & \(E_5^2\) & \(E_6^1\) & \(E_6^2\) & \(\widetilde{C_{3}}\) & \(\widetilde{C_{4}}\) & \(\widetilde{C_{5}}\) & \(\widetilde{H_{\mathcal{S}}}\) \\
    \hline
    \hline
    \(E_1^1\) & $-2$ & $1$ & $0$ & $0$ & $0$ & $0$ & $0$ & $0$ & $0$ & $0$ & $0$ & $0$ & $0$ & $1$ & $0$ & $0$ & $0$ \\
    \(E_1^2\) & $1$ & $-2$ & $0$ & $0$ & $0$ & $0$ & $0$ & $0$ & $0$ & $0$ & $0$ & $0$ & $0$ & $0$ & $0$ & $1$ & $0$ \\
    \hline
    \(E_2^1\) & $0$ & $0$ & $-2$ & $1$ & $0$ & $0$ & $0$ & $0$ & $0$ & $0$ & $0$ & $0$ & $0$ & $1$ & $0$ & $0$ & $0$ \\
    \(E_2^2\) & $0$ & $0$ & $1$ & $-2$ & $1$ & $0$ & $0$ & $0$ & $0$ & $0$ & $0$ & $0$ & $0$ & $0$ & $0$ & $0$ & $0$ \\
    \(E_2^3\) & $0$ & $0$ & $0$ & $1$ & $-2$ & $0$ & $0$ & $0$ & $0$ & $0$ & $0$ & $0$ & $0$ & $0$ & $1$ & $0$ & $0$ \\
    \hline
    \(E_3^1\) & $0$ & $0$ & $0$ & $0$ & $0$ & $-2$ & $0$ & $0$ & $0$ & $0$ & $0$ & $0$ & $0$ & $1$ & $0$ & $0$ & $0$ \\
    \hline
    \(E_4^1\) & $0$ & $0$ & $0$ & $0$ & $0$ & $0$ & $-2$ & $1$ & $0$ & $0$ & $0$ & $0$ & $0$ & $0$ & $0$ & $0$ & $0$ \\
    \(E_4^2\) & $0$ & $0$ & $0$ & $0$ & $0$ & $0$ & $1$ & $-2$ & $1$ & $0$ & $0$ & $0$ & $0$ & $0$ & $1$ & $0$ & $0$ \\
    \(E_4^3\) & $0$ & $0$ & $0$ & $0$ & $0$ & $0$ & $0$ & $1$ & $-2$ & $0$ & $0$ & $0$ & $0$ & $0$ & $0$ & $0$ & $0$ \\
    \hline
    \(E_5^1\) & $0$ & $0$ & $0$ & $0$ & $0$ & $0$ & $0$ & $0$ & $0$ & $-2$ & $1$ & $0$ & $0$ & $0$ & $0$ & $1$ & $0$ \\
    \(E_5^2\) & $0$ & $0$ & $0$ & $0$ & $0$ & $0$ & $0$ & $0$ & $0$ & $1$ & $-2$ & $0$ & $0$ & $0$ & $0$ & $0$ & $0$ \\
    \hline
    \(E_6^1\) & $0$ & $0$ & $0$ & $0$ & $0$ & $0$ & $0$ & $0$ & $0$ & $0$ & $0$ & $-2$ & $1$ & $0$ & $0$ & $1$ & $0$ \\
    \(E_6^2\) & $0$ & $0$ & $0$ & $0$ & $0$ & $0$ & $0$ & $0$ & $0$ & $0$ & $0$ & $1$ & $-2$ & $0$ & $0$ & $0$ & $0$ \\
    \hline
    \(\widetilde{C_{3}}\) & $1$ & $0$ & $1$ & $0$ & $0$ & $1$ & $0$ & $0$ & $0$ & $0$ & $0$ & $0$ & $0$ & $-2$ & $0$ & $0$ & $1$ \\
    \(\widetilde{C_{4}}\) & $0$ & $0$ & $0$ & $0$ & $1$ & $0$ & $0$ & $1$ & $0$ & $0$ & $0$ & $0$ & $0$ & $0$ & $-2$ & $0$ & $1$ \\
    \(\widetilde{C_{5}}\) & $0$ & $1$ & $0$ & $0$ & $0$ & $0$ & $0$ & $0$ & $0$ & $1$ & $0$ & $1$ & $0$ & $0$ & $0$ & $-2$ & $1$ \\
    \(\widetilde{H_{\mathcal{S}}}\) & $0$ & $0$ & $0$ & $0$ & $0$ & $0$ & $0$ & $0$ & $0$ & $0$ & $0$ & $0$ & $0$ & $1$ & $1$ & $1$ & $4$ \\
    \hline
  \end{tabular}.
\end{table}
Note that the intersection matrix is non-degenerate, hence we have \(\rk(L_{\lambda}) = 17\). Recall that \(\Pic(\widetilde{\mathcal{S}_{\lambda}})\) is generated by linear equivalence classes of the exceptional curves \(E_i^j\) and of strict transforms of curves on the surface \(\mathcal{S}_{\lambda}\).

The intersection matrix on the lattice \(L_{\mathcal{S}} = L_{\lambda}^{\Gal(\Bbbk)}\) has the following form:
\begin{table}[H]
  \begin{tabular}{|c||cc|ccc|c|cc|cc|cccc|}
    \hline
    \(\bullet\) & \(E_1^1\) & \(E_1^2\) & \(E_2^1\) & \(E_2^2\) & \(E_2^3\) & \(E_3^1\) & \(E_4^1 + E_4^3\) & \(E_4^2\) & \(E_5^1 + E_6^1\) & \(E_5^2 + E_6^2\) & \(\widetilde{C_{3}}\) & \(\widetilde{C_{4}}\) & \(\widetilde{C_{5}}\) & \(\widetilde{H_{\mathcal{S}}}\) \\
    \hline
    \hline
    \(E_1^1\) & $-2$ & $1$ & $0$ & $0$ & $0$ & $0$ & $0$ & $0$ & $0$ & $0$ & $1$ & $0$ & $0$ & $0$ \\
    \(E_1^2\) & $1$ & $-2$ & $0$ & $0$ & $0$ & $0$ & $0$ & $0$ & $0$ & $0$ & $0$ & $0$ & $1$ & $0$ \\
    \hline
    \(E_2^1\) & $0$ & $0$ & $-2$ & $1$ & $0$ & $0$ & $0$ & $0$ & $0$ & $0$ & $1$ & $0$ & $0$ & $0$ \\
    \(E_2^2\) & $0$ & $0$ & $1$ & $-2$ & $1$ & $0$ & $0$ & $0$ & $0$ & $0$ & $0$ & $0$ & $0$ & $0$ \\
    \(E_2^3\) & $0$ & $0$ & $0$ & $1$ & $-2$ & $0$ & $0$ & $0$ & $0$ & $0$ & $0$ & $1$ & $0$ & $0$ \\
    \hline
    \(E_3^1\) & $0$ & $0$ & $0$ & $0$ & $0$ & $-2$ & $0$ & $0$ & $0$ & $0$ & $1$ & $0$ & $0$ & $0$ \\
    \hline
    \(E_4^1 + E_4^3\) & $0$ & $0$ & $0$ & $0$ & $0$ & $0$ & $-4$ & $2$ & $0$ & $0$ & $0$ & $0$ & $0$ & $0$ \\
    \(E_4^2\) & $0$ & $0$ & $0$ & $0$ & $0$ & $0$ & $2$ & $-2$ & $0$ & $0$ & $0$ & $1$ & $0$ & $0$ \\
    \hline
    \(E_5^1 + E_6^1\) & $0$ & $0$ & $0$ & $0$ & $0$ & $0$ & $0$ & $0$ & $-4$ & $2$ & $0$ & $0$ & $2$ & $0$ \\
    \(E_5^2 + E_6^2\) & $0$ & $0$ & $0$ & $0$ & $0$ & $0$ & $0$ & $0$ & $2$ & $-4$ & $0$ & $0$ & $0$ & $0$ \\
    \hline
    \(\widetilde{C_{3}}\) & $1$ & $0$ & $1$ & $0$ & $0$ & $1$ & $0$ & $0$ & $0$ & $0$ & $-2$ & $0$ & $0$ & $1$ \\
    \(\widetilde{C_{4}}\) & $0$ & $0$ & $0$ & $0$ & $1$ & $0$ & $0$ & $1$ & $0$ & $0$ & $0$ & $-2$ & $0$ & $1$ \\
    \(\widetilde{C_{5}}\) & $0$ & $1$ & $0$ & $0$ & $0$ & $0$ & $0$ & $0$ & $2$ & $0$ & $0$ & $0$ & $-2$ & $1$ \\
    \(\widetilde{H_{\mathcal{S}}}\) & $0$ & $0$ & $0$ & $0$ & $0$ & $0$ & $0$ & $0$ & $0$ & $0$ & $1$ & $1$ & $1$ & $4$ \\
   \hline
  \end{tabular}.
\end{table}
Note that the intersection matrix is non-degenerate, hence we have \(\rk(L_{\mathcal{S}}) = 14\). Recall that \(\Pic(\widetilde{\mathcal{S}_{\Bbbk}})\) is generated by linear equivalence classes of the exceptional curves \(E_i^j\) and of strict transforms of curves on \(\mathcal{S}_{\Bbbk}\). We denote by \(M\) the intersection matrix on \(L_{\mathcal{S}}\).

The conjectural orthogonal complement to the lattice \(L_{\mathcal{S}}\) has the form
\[
  N = H \oplus \Pic(X), \quad
  \Pic(X) =
  \begin{pmatrix}
    -2 & 0 & 0 & 0 & -1 & 0 \\
    0 & -2 & 0 & 0 & -1 & 0 \\
    0 & 0 & -2 & 0 & -1 & 0 \\
    0 & 0 & 0 & -2 & -1 & 0 \\
    -1 & -1 & -1 & -1 & 0 & 3 \\
    0 & 0 & 0 & 0 & 3 & 2
  \end{pmatrix}.
\]
We choose the following generators of discriminant groups \(D_M\) and \(D_N\) of the lattices \(M\) and \(N\), respectively:
\begin{gather*}
  \begin{pmatrix}
    0 & 0 & 0 & 0 & 0 & 0 & 0 & 0 & \frac{1}{2} & 0 & 0 & 0 & 0 & 0 \\
    0 & 0 & 0 & 0 & 0 & 0 & 0 & 0 & \frac{1}{2} & \frac{1}{2} & 0 & 0 & 0 & 0 \\
    0 & 0 & \frac{1}{2} & 0 & \frac{1}{2} & \frac{1}{2} & 0 & \frac{1}{2} & 0 & \frac{1}{2} & 0 & 0 & 0 & 0 \\
    \frac{1}{5} & \frac{4}{5} & \frac{9}{10} & \frac{1}{5} & \frac{1}{2} & \frac{3}{10} & \frac{9}{10} & \frac{3}{10} & \frac{3}{5} & \frac{3}{10} & \frac{3}{5} & \frac{4}{5} & \frac{2}{5} & \frac{4}{5}
  \end{pmatrix}, \quad
  \begin{pmatrix}
    0 & 0 & -\frac{1}{2} & 0 & \frac{1}{2} & 0 & 0 & 0 \\
    0 & 0 & -\frac{1}{2} & 0 & 0 & \frac{1}{2} & 0 & 0 \\
    0 & 0 & -\frac{1}{2} & \frac{1}{2} & 0 & 0 & 0 & 0 \\
    0 & 0 & -\frac{1}{5} & \frac{3}{10} & \frac{3}{10} & \frac{3}{10} & \frac{2}{5} & -\frac{1}{10}
  \end{pmatrix}.
\end{gather*}
Then bilinear and quadratic discriminant forms on \(D_M\) and \(D_N\) can be represented as
\begin{gather*}
  B_M = 
  \begin{pmatrix}
    0 & \frac{1}{2} & \frac{1}{2} & \frac{1}{2} \\
    \frac{1}{2} & 0 & \frac{1}{2} & \frac{1}{2} \\
    \frac{1}{2} & \frac{1}{2} & 0 & \frac{1}{2} \\
    \frac{1}{2} & \frac{1}{2} & \frac{1}{2} & \frac{2}{5}
  \end{pmatrix}, \quad
  B_N = 
  \begin{pmatrix}
    0 & \frac{1}{2} & \frac{1}{2} & \frac{1}{2} \\
    \frac{1}{2} & 0 & \frac{1}{2} & \frac{1}{2} \\
    \frac{1}{2} & \frac{1}{2} & 0 & \frac{1}{2} \\
    \frac{1}{2} & \frac{1}{2} & \frac{1}{2} & \frac{3}{5}
  \end{pmatrix}; \quad
  Q_M = \left(1,\,1,\,1,\,\frac{7}{5}\right), \quad
  Q_N = \left(1,\,1,\,1,\,\frac{3}{5}\right).
\end{gather*}
The lattices \(M\) and \(N\) have the signature \((1, 13)\) and \((2, 6)\), respectively, hence \(M^{\perp} \simeq N\) in the K3 lattice.

%%% Local Variables:
%%% mode: latex
%%% TeX-master: "../main"
%%% End:
