\subsection{Family \textnumero8.1}\label{subsection:08-01_parametrised}

The toric Landau--Ginzburg model for this family is given by the Laurent polynomial
\begin{gather*}
  (a_4 a_7) y^{2} z^{-1} + x + (a_1 a_2 a_4 a_5 a_7 + a_4 + a_7) y + (a_1 a_2 a_4 a_5 + a_1 a_2 a_5 a_7 + 1) z + (a_1 a_2 a_5) y^{-1} z^{2} + \\ (a_1 a_3 a_4 a_7 + a_1 a_4 a_6 a_7 + 1) y z^{-1} + (a_1^{2} a_2 a_3 a_5 a_6 + a_1 a_2 + a_1 a_5) y^{-1} z + (a_1^{2} a_3 a_4 a_6 a_7 + a_1 a_3 + a_1 a_6) z^{-1} + \\ (a_1^{2} a_2 a_3 a_6 + a_1^{2} a_3 a_5 a_6 + a_1) y^{-1} + a_8 x^{-1} + (a_1^{2} a_3 a_6) y^{-1} z^{-1}
\end{gather*}
(see Subsubsection~\ref{subsubsection:dPxP1_08-01}). The pencil \(\mathcal{S}(\overline{a})\) is defined by the equation
\begin{gather*}
  (a_4 a_7) X Y^{3} + X^{2} Y Z + (a_1 a_2 a_4 a_5 a_7 + a_4 + a_7) X Y^{2} Z + (a_1 a_2 a_4 a_5 + a_1 a_2 a_5 a_7 + 1) X Y Z^{2} + (a_1 a_2 a_5) X Z^{3} + \\ (a_1 a_3 a_4 a_7 + a_1 a_4 a_6 a_7 + 1) X Y^{2} T + (a_1^{2} a_2 a_3 a_5 a_6 + a_1 a_2 + a_1 a_5) X Z^{2} T + (a_1^{2} a_3 a_4 a_6 a_7 + a_1 a_3 + a_1 a_6) X Y T^{2} + \\ (a_1^{2} a_2 a_3 a_6 + a_1^{2} a_3 a_5 a_6 + a_1) X Z T^{2} + a_8 Y Z T^{2} + (a_1^{2} a_3 a_6) X T^{3} = \lambda X Y Z T.
\end{gather*}
The following members \(\mathcal{S}(\overline{a})_{\lambda}\) of the pencil are reducible:
\begin{gather*}
  \mathcal{S}(\overline{a})_{\infty} = S_{(X)} + S_{(Y)} + S_{(Z)} + S_{(T)}.
\end{gather*}
The base locus of the pencil \(\mathcal{S}(\overline{a})\) consists of the following curves:
\begin{gather*}
  C_1 = C_{(X, Y)}, \;
  C_2 = C_{(X, Z)}, \;
  C_3 = C_{(X, T)}, \;
  C_4 = C_{(Y, a_2 Z + T)}, \;
  C_5 = C_{(Y, a_5 Z + T)}, \;
  C_6 = C_{(Y, Z + a_1 a_3 a_6 T)}, \\
  C_7 = C_{(Z, Y + a_1 a_3 T)}, \;
  C_8 = C_{(Z, Y + a_1 a_6 T)}, \;
  C_9 = C_{(Z, a_4 a_7 Y + T)}, \;
  C_{10} = C_{(T, (a_4 Y + Z) (a_7 Y + Z) (Y + a_1 a_2 a_5 Z) + X Y Z)}.
\end{gather*}

Their linear equivalence classes on the generic member \(\mathcal{S}(\overline{a})_{\Bbbk}\) of the pencil satisfy the following relations:
\begin{gather*}
  \begin{pmatrix}
    [\mathcal{S}(\overline{a})_{\Bbbk} \cdot S_{(X)}] - [H_{\mathcal{S}(\overline{a})}] \\
    [\mathcal{S}(\overline{a})_{\Bbbk} \cdot S_{(Y)}] - [H_{\mathcal{S}(\overline{a})}] \\
    [\mathcal{S}(\overline{a})_{\Bbbk} \cdot S_{(Z)}] - [H_{\mathcal{S}(\overline{a})}] \\
    [\mathcal{S}(\overline{a})_{\Bbbk} \cdot S_{(T)}] - [H_{\mathcal{S}(\overline{a})}]
  \end{pmatrix} = 
  \begin{pmatrix}
    1 & 1 & 2 & 0 & 0 & 0 & 0 & 0 & 0 & 0 & -1 \\
    1 & 0 & 0 & 1 & 1 & 1 & 0 & 0 & 0 & 0 & -1 \\
    0 & 1 & 0 & 0 & 0 & 0 & 1 & 1 & 1 & 0 & -1 \\
    0 & 0 & 1 & 0 & 0 & 0 & 0 & 0 & 0 & 1 & -1
  \end{pmatrix} \cdot
  \begin{pmatrix}
    [C_1] \\ \cdots \\ [C_{10}] \\ [H_{\mathcal{S}(\overline{a})}]
  \end{pmatrix}.
\end{gather*}
We can reduce the number of linear equivalence classes using these identities:
\begin{gather*}
  \begin{pmatrix}
    [C_{2}] \\ [C_{6}] \\ [C_{9}] \\ [C_{10}]
  \end{pmatrix} = 
  \begin{pmatrix}
    -1 & -2 & 0 & 0 & 0 & 0 & 1 \\
    -1 & 0 & -1 & -1 & 0 & 0 & 1 \\
    1 & 2 & 0 & 0 & -1 & -1 & 0 \\
    0 & -1 & 0 & 0 & 0 & 0 & 1
  \end{pmatrix} \cdot
  \begin{pmatrix}
    [C_{1}] \\ [C_{3}] \\ [C_{4}] \\ [C_{5}] \\ [C_{7}] \\ [C_{8}] \\ [H_{\mathcal{S}(\overline{a})}]
  \end{pmatrix}.
\end{gather*}

For a general choice of \(\lambda \in \mathbb{C}\) and \(\overline{a} \in (\mathbb{C}^*)^8\) the surface \(\mathcal{S}(\overline{a})_{\lambda}\) has the following singularities:
\begin{itemize}\setlength{\itemindent}{2cm}
\item[\(P_{1} = P_{(Y, Z, T)}\):] type \(\mathbb{A}_2\) with the quadratic term \(Y \cdot Z\);
\item[\(P_{2} = P_{(X, T, a_4 Y + Z)}\):] type \(\mathbb{A}_1\) with the quadratic term
  \begin{gather*}
    (a_4 - a_7) (a_1 a_2 a_4 a_5 - 1) X (a_4 Y + Z) - a_4 (X^2 + a_8 T^2) + \\
    (a_1 a_4 (a_1 a_2 a_3 a_4 a_5 a_6 + a_4 (a_2 + a_5) + a_7 (a_3 + a_6)) + a_4 \lambda + 1) X T;
  \end{gather*}
\item[\(P_{3} = P_{(X, T, a_7 Y + Z)}\):] type \(\mathbb{A}_1\) with the quadratic term
  \begin{gather*}
    (a_4 - a_7) (a_1 a_2 a_5 a_7 - 1) X (a_7 Y + Z) + a_7 (X^2 + a_8 T^2) - \\
    (a_1 a_7 (a_1 a_2 a_3 a_5 a_6 a_7 + a_4 (a_3 + a_6) + a_7 (a_2 + a_5)) + a_7 \lambda + 1) X T;
  \end{gather*}
\item[\(P_{4} = P_{(X, T, Y + a_1 a_2 a_5 Z)}\):] type \(\mathbb{A}_1\) with the quadratic term
  \begin{gather*}
    (a_1 a_2 a_4 a_5 - 1) (a_1 a_2 a_5 a_7 - 1) X (Y + a_1 a_2 a_5 Z) - a_1 a_2 a_5 (X^2 + a_8 T^2) \\
    a_1 (a_1 a_2 a_5 (a_1 a_2 a_4 a_5 a_7 (a_3 + a_6) + a_2 a_5 + a_3 a_6) + a_2 a_5 \lambda + a_2 + a_5) X T.
  \end{gather*}
\end{itemize}

The \(\mathbb{Q}\)-valued intersection matrix on the group \(A^{\overline{a}}_{\mathcal{S}}\) has the following form:
\begin{table}[H]
  \renewcommand\arraystretch{1.42}
  \begin{tabular}{|c||c|c|c|c|c|c|c|}
    \hline
    \(\bullet\) & \(C_{1}\) & \(C_{3}\) & \(C_{4}\) & \(C_{5}\) & \(C_{7}\) & \(C_{8}\) & \(H_{\mathcal{S}}\) \\
    \hline
    \hline
    \(C_{1}\) & $-2$ & $1$ & $1$ & $1$ & $0$ & $0$ & $1$ \\
    \hline
    \(C_{3}\) & $1$ & $-\frac{1}{2}$ & $0$ & $0$ & $0$ & $0$ & $1$ \\
    \hline
    \(C_{4}\) & $1$ & $0$ & $-\frac{4}{3}$ & $\frac{2}{3}$ & $\frac{1}{3}$ & $\frac{1}{3}$ & $1$ \\
    \hline
    \(C_{5}\) & $1$ & $0$ & $\frac{2}{3}$ & $-\frac{4}{3}$ & $\frac{1}{3}$ & $\frac{1}{3}$ & $1$ \\
    \hline
    \(C_{7}\) & $0$ & $0$ & $\frac{1}{3}$ & $\frac{1}{3}$ & $-\frac{4}{3}$ & $\frac{2}{3}$ & $1$ \\
    \hline
    \(C_{8}\) & $0$ & $0$ & $\frac{1}{3}$ & $\frac{1}{3}$ & $\frac{2}{3}$ & $-\frac{4}{3}$ & $1$ \\
    \hline
    \(H_{\mathcal{S}}\) & $1$ & $1$ & $1$ & $1$ & $1$ & $1$ & $4$ \\
    \hline
  \end{tabular}.
\end{table}
Note that the intersection matrix has the rank 7.

Let \(\widetilde{\mathcal{S}(\overline{a})_{\lambda}}\) and \(\widetilde{\mathcal{S}(\overline{a})_{\Bbbk}}\) be the minimal resolution of a general member \(\mathcal{S}(\overline{a})_{\lambda}\) and the generic member \(\mathcal{S}(\overline{a})_{\Bbbk}\) of the pencil, respectively. Denote by \(L^{\overline{a}}_{\lambda} \subset \Pic(\widetilde{\mathcal{S}(\overline{a})_{\lambda}})\) and \(L^{\overline{a}}_{\Bbbk} \subset \Pic(\widetilde{\mathcal{S}(\overline{a})_{\Bbbk}})\) the subgroups generated by linear equivalence classes of exceptional divisors of the resolution and by linear equivalence classes of strict transforms \(\widetilde{C_i}\), \(\widetilde{H_{\mathcal{S}(\overline{a})}}\) of the above-introduced curves. By construction \(L^{\overline{a}}_{\lambda}\) is equipped with the \(\Gal(\Bbbk)\)-action, and \(L^{\overline{a}}_{\Bbbk}\) can be identified with the subgroup \((L^{\overline{a}}_{\lambda})^{\Gal(\Bbbk)} \subset L^{\overline{a}}_{\lambda}\). In our case we have \(L^{\overline{a}}_{\lambda} = L^{\overline{a}}_{\Bbbk}\).

The intersection matrix on the lattice \(L^{\overline{a}}_{\lambda}\) has the following form:
\begin{table}[H]
  \begin{tabular}{|c||cc|c|cc|ccccccc|}
    \hline
    \(\bullet\) & \(E_1^1\) & \(E_1^2\) & \(E_2^1\) & \(E_3^1\) & \(E_4^1\) & \(\widetilde{C_{1}}\) & \(\widetilde{C_{3}}\) & \(\widetilde{C_{4}}\) & \(\widetilde{C_{5}}\) & \(\widetilde{C_{7}}\) & \(\widetilde{C_{8}}\) & \(\widetilde{H_{\mathcal{S}}}\) \\
    \hline
    \hline
    \(E_1^1\) & $-2$ & $1$ & $0$ & $0$ & $0$ & $0$ & $0$ & $1$ & $1$ & $0$ & $0$ & $0$ \\
    \(E_1^2\) & $1$ & $-2$ & $0$ & $0$ & $0$ & $0$ & $0$ & $0$ & $0$ & $1$ & $1$ & $0$ \\
    \hline
    \(E_2^1\) & $0$ & $0$ & $-2$ & $0$ & $0$ & $0$ & $1$ & $0$ & $0$ & $0$ & $0$ & $0$ \\
    \hline
    \(E_3^1\) & $0$ & $0$ & $0$ & $-2$ & $0$ & $0$ & $1$ & $0$ & $0$ & $0$ & $0$ & $0$ \\
    \(E_4^1\) & $0$ & $0$ & $0$ & $0$ & $-2$ & $0$ & $1$ & $0$ & $0$ & $0$ & $0$ & $0$ \\
    \hline
    \(\widetilde{C_{1}}\) & $0$ & $0$ & $0$ & $0$ & $0$ & $-2$ & $1$ & $1$ & $1$ & $0$ & $0$ & $1$ \\
    \(\widetilde{C_{3}}\) & $0$ & $0$ & $1$ & $1$ & $1$ & $1$ & $-2$ & $0$ & $0$ & $0$ & $0$ & $1$ \\
    \(\widetilde{C_{4}}\) & $1$ & $0$ & $0$ & $0$ & $0$ & $1$ & $0$ & $-2$ & $0$ & $0$ & $0$ & $1$ \\
    \(\widetilde{C_{5}}\) & $1$ & $0$ & $0$ & $0$ & $0$ & $1$ & $0$ & $0$ & $-2$ & $0$ & $0$ & $1$ \\
    \(\widetilde{C_{7}}\) & $0$ & $1$ & $0$ & $0$ & $0$ & $0$ & $0$ & $0$ & $0$ & $-2$ & $0$ & $1$ \\
    \(\widetilde{C_{8}}\) & $0$ & $1$ & $0$ & $0$ & $0$ & $0$ & $0$ & $0$ & $0$ & $0$ & $-2$ & $1$ \\
    \(\widetilde{H_{\mathcal{S}}}\) & $0$ & $0$ & $0$ & $0$ & $0$ & $1$ & $1$ & $1$ & $1$ & $1$ & $1$ & $4$ \\
    \hline
  \end{tabular}
\end{table}
The intersection matrix is non-degenerate, hence we have \(\rk(L^{\overline{a}}_{\lambda}) = 12\). Recall that \(\Pic(\widetilde{\mathcal{S}(\overline{a})_{\lambda}})\) is generated by linear equivalence classes of the exceptional curves \(E_i^j\) and of strict transforms of curves on \(\mathcal{S}(\overline{a})_{\lambda}\). Denote by \(M\) the corresponding intersection matrix.

The conjectural orthogonal complement to the lattice \(L_{\mathcal{S}}\) has the form
\[
  N = H \oplus \Pic(X), \quad
  \Pic(X) =
  \begin{pmatrix}
    -2 & 0 & 0 & 0 & 0 & 0 & -1 & 0 \\
    0 & -2 & 0 & 0 & 0 & 0 & -1 & 0 \\
    0 & 0 & -2 & 0 & 0 & 0 & -1 & 0 \\
    0 & 0 & 0 & -2 & 0 & 0 & -1 & 0 \\
    0 & 0 & 0 & 0 & -2 & 0 & -1 & 0 \\
    0 & 0 & 0 & 0 & 0 & -2 & -1 & 0 \\
    -1 & -1 & -1 & -1 & -1 & -1 & 0 & 3 \\
    0 & 0 & 0 & 0 & 0 & 0 & 3 & 2
  \end{pmatrix}.
\]
We choose the following generators of discriminant groups \(D_M\) and \(D_N\) of the lattices \(M\) and \(N\), respectively:
\begin{gather*}
  \begin{pmatrix}
    0 & 0 & \frac{1}{2} & \frac{1}{2} & 0 & 0 & 0 & 0 & 0 & \frac{1}{2} & \frac{1}{2} & 0 \\
    0 & 0 & \frac{1}{2} & 0 & \frac{1}{2} & 0 & 0 & 0 & 0 & \frac{1}{2} & \frac{1}{2} & 0 \\
    \frac{1}{2} & 0 & \frac{1}{2} & 0 & 0 & \frac{1}{2} & 0 & 0 & 0 & 0 & \frac{1}{2} & 0 \\
    \frac{1}{2} & 0 & \frac{1}{2} & 0 & 0 & \frac{1}{2} & 0 & 0 & 0 & \frac{1}{2} & 0 & 0 \\
    0 & 0 & 0 & 0 & 0 & 0 & 0 & \frac{1}{2} & \frac{1}{2} & 0 & 0 & 0 \\
    0 & \frac{1}{2} & 0 & 0 & 0 & \frac{1}{2} & 0 & \frac{1}{2} & 0 & 0 & 0 & \frac{1}{2} \\
    \frac{1}{3} & \frac{2}{3} & \frac{2}{3} & \frac{2}{3} & \frac{2}{3} & \frac{1}{3} & \frac{1}{3} & 0 & 0 & 0 & 0 & \frac{1}{3}
  \end{pmatrix}, \quad
  \begin{pmatrix}
    0 & 0 & \frac{1}{2} & \frac{1}{2} & \frac{1}{2} & 0 & \frac{1}{2} & \frac{1}{2} & 0 & \frac{1}{2} \\
    0 & 0 & \frac{1}{2} & \frac{1}{2} & 0 & \frac{1}{2} & \frac{1}{2} & \frac{1}{2} & 0 & \frac{1}{2} \\
    0 & 0 & 0 & \frac{1}{2} & 0 & 0 & 0 & 0 & 0 & \frac{1}{2} \\
    0 & 0 & \frac{1}{2} & 0 & 0 & 0 & 0 & 0 & 0 & \frac{1}{2} \\
    0 & 0 & \frac{1}{2} & \frac{1}{2} & 0 & 0 & \frac{1}{2} & 0 & 0 & \frac{1}{2} \\
    0 & 0 & \frac{1}{2} & \frac{1}{2} & 0 & 0 & 0 & \frac{1}{2} & 0 & \frac{1}{2} \\
    0 & 0 & \frac{2}{3} & \frac{2}{3} & \frac{2}{3} & \frac{2}{3} & \frac{2}{3} & \frac{2}{3} & \frac{2}{3} & 0
  \end{pmatrix}.
\end{gather*}
Then bilinear and quadratic discriminant forms on \(D_M\) and \(D_N\) can be represented as
\begin{gather*}
  B_M = 
  \begin{pmatrix}
    0 & \frac{1}{2} & 0 & 0 & 0 & 0 & 0 \\
    \frac{1}{2} & 0 & 0 & 0 & 0 & 0 & 0 \\
    0 & 0 & 0 & \frac{1}{2} & 0 & 0 & 0 \\
    0 & 0 & \frac{1}{2} & 0 & 0 & 0 & 0 \\
    0 & 0 & 0 & 0 & 0 & \frac{1}{2} & 0 \\
    0 & 0 & 0 & 0 & \frac{1}{2} & 0 & 0 \\
    0 & 0 & 0 & 0 & 0 & 0 & \frac{2}{3}
  \end{pmatrix}, \quad
  B_N = 
  \begin{pmatrix}
    0 & \frac{1}{2} & 0 & 0 & 0 & 0 & 0 \\
    \frac{1}{2} & 0 & 0 & 0 & 0 & 0 & 0 \\
    0 & 0 & 0 & \frac{1}{2} & 0 & 0 & 0 \\
    0 & 0 & \frac{1}{2} & 0 & 0 & 0 & 0 \\
    0 & 0 & 0 & 0 & 0 & \frac{1}{2} & 0 \\
    0 & 0 & 0 & 0 & \frac{1}{2} & 0 & 0 \\
    0 & 0 & 0 & 0 & 0 & 0 & \frac{1}{3}
  \end{pmatrix}; \\
  Q_M = \left(0,\,0,\,0,\,0,\,1,\,1,\,\frac{2}{3}\right), \quad
  Q_N = \left(0,\,0,\,0,\,0,\,1,\,1,\,\frac{4}{3}\right).
\end{gather*}
The lattices \(M\) and \(N\) have the signature \((1, 11)\) and \((2, 8)\), respectively, hence \(M^{\perp} \simeq N\) in the K3 lattice.

%%% Local Variables:
%%% mode: latex
%%% TeX-master: "../main"
%%% End:
