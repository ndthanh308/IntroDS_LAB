\subsection{Toric Landau--Ginzburg models of smooth del Pezzo surfaces with very ample anticanonical class}

Let \(S\) be a smooth del Pezzo surface. If the anticanonical class \(-K_S\) is very ample, then the surface \(S\) admits a Gorenstein toric degeneration. More precisely, it is well-known that there exist 16 Gorenstein toric del Pezzo surfaces, which correspond to 16 reflexive polygones (for example, see~\cite{batyrev2010reflexive}). All of them can be realized as toric degenerations of smooth del Pezzo surfaces (see~\cite[Section~3]{przyjalkowski2017compactification}).

\begin{notation}
  We denote by \(R_i\) the \(i\)-th reflexive polygon with respect to the numeration in PALP package (see~\cite{kreuzer2004PALP}), and by \(T_i\) the corresponding Gorenstein toric del Pezzo surface. 
\end{notation}

\begin{table}[h]
\centering
\begin{tabular}{|c|c|c|}
  \hline
  \(X\) & \((-K_X)^2\) & Gorenstein toric del Pezzo surfaces \\
  \hline
  \hline
  \(\mathbb{P}^1 \times \mathbb{P}^1\) & \(8\) & \(T_2, T_4\) \\ 
  \hline
  \(\mathbb{P}^2\) & \(9\) & \(T_1\) \\ 
  \hline
  \(\Bl_1(\mathbb{P}^2)\) & \(8\) & \(T_3\) \\
  \hline  
  \(\Bl_2(\mathbb{P}^2)\) & \(7\) & \(T_5, T_6\) \\ 
  \hline
  \(\Bl_3(\mathbb{P}^2)\) & \(6\) & \(T_7, T_8, T_9, T_{10}\) \\ 
  \hline
  \(\Bl_4(\mathbb{P}^2)\) & \(5\) & \(T_{11}, T_{12}\) \\ 
  \hline
  \(\Bl_5(\mathbb{P}^2)\) & \(4\) & \(T_{13}, T_{14}, T_{15}\) \\ 
  \hline
  \(\Bl_6(\mathbb{P}^2)\) & \(3\) & \(T_{16}\) \\ 
  \hline
  \(\Bl_7(\mathbb{P}^2)\) & \(2\) & -- \\ 
  \hline
  \(\Bl_8(\mathbb{P}^2)\) & \(1\) & -- \\ 
  \hline  
\end{tabular}
\caption{Gorenstein toric del Pezzo surfaces}
\label{table:toric-degenerations}
\end{table}

\begin{remark}
  Toric degenerations \(T_1, T_3, T_4, T_6, T_{10}\) correspond to smooth toric del Pezzo surfaces. 
\end{remark}

It follows from Table~\ref{table:toric-degenerations} that a smooth del Pezzo surface \(S\) do not admit a Gorenstein toric degeneration precisely when the anticanonical degree is equal to \(1\) or \(2\). It is well-known that such a surface is either a sextic hypersurface in \(\mathbb{P}(1, 1, 2, 3)\) or a quartic hypersurface in \(\mathbb{P}(1, 1, 1, 2)\). Then \(S\) also has a toric degeneration, but its singularities are worse than Gorenstein (see~\cite[Remark~14]{przyjalkowski2017compactification}).

We construct a toric Landau--Ginzburg model for all smooth del Pezzo surfaces \(S\) with very ample anticanonical class by applying Przyjalkowski's algorithm (see~\cite[Section~3]{przyjalkowski2017compactification}).

\begin{notation}
  For any coefficients of the form \(\alpha, \beta, \alpha_i \in \mathbb{C}\) we put \(a = e^{-\alpha}\), \(b = e^{-\beta}\), and \(a_i = e^{-\alpha_i}\).
\end{notation}

\subsubsection{Quadric surface}\label{subsubsection:dP_quadric}

Let \(S \simeq \mathbb{P}^1 \times \mathbb{P}^1\) be a quadric surface, and \(D\) be an \((\alpha, \beta)\)-divisor on \(S\). The surface \(S\) admits two Gorenstein toric degenerations: \(T_4 \simeq S\) and a quadratic cone \(T_2\).

Firstly, let us choose \(T_4\) as a Gorenstein toric degeneration \(T\) of the surface \(S\), and let \(\widetilde{D}\) be a divisor on its crepant resolution \(\widetilde{T} \simeq S\). The toric Landau--Ginzburg model for the pair \((S, D) = (\widetilde{T}, \widetilde{D})\) equals to
\[
  f_{(S, D)} = f_{(\widetilde{T}, \widetilde{D})} = x + a x^{-1} + y + b y^{-1}.
\]

Secondly, let us choose \(T_2\) as a Gorenstein toric degeneration \(T\) of the surface \(S\), and let \(\widetilde{D}\) be a divisor on its crepant resolution \(\widetilde{T}\). In this case \(\widetilde{T}\) is the second Hirzebruch surface \(\mathbb{F}_2\). Consequently, we can present \(\widetilde{D}\) in the form \(\widetilde{D} = \alpha s + \beta f\), where \(s\) is a \((-2)\)-section of \(\widetilde{T}\), and \(f\) is a fibre of the map \(\widetilde{T} \rightarrow \mathbb{P}^1\). The toric Landau--Ginzburg model for the pair \((\widetilde{T}, \widetilde{D})\) has the form
\[
  f_{(\widetilde{T}, \widetilde{D})} = y + b x^{-1} y^{-1} + a y^{-1} + x y^{-1},
\]
and the toric Landau--Ginzburg model for the pair \((S, D)\) equals to
\[
  f_{(S, D)} = y + a x^{-1} y^{-1} + (a + b) y^{-1} + b x y^{-1}.
\]

% Figure environment removed

\subsubsection{Smooth del Pezzo surface of degree 9}\label{subsubsection:dP_09}

Let \(S \simeq \mathbb{P}^2\), and \(D = \alpha_1 l \in \Pic(S) \otimes_{\mathbb{Z}} \mathbb{C}\) be a divisor on \(S\), where \(l\) is the linear equivalence class of a line. The surface \(S\) admits the unique Gorenstein toric degeneration \(T_1 \simeq S\). Let us choose \(T_1\) as a Gorenstein toric degeneration \(T\) of the surface \(S\), and let \(\widetilde{D} \in \Pic(\widetilde{T}) \otimes_{\mathbb{Z}} \mathbb{C}\) be the corresponding divisor on its crepant resolution \(\widetilde{T} \simeq S\) under the identification \(\Pic(S) \simeq \Pic(\widetilde{T})\).

The toric Landau--Ginzburg model for the pair \((S, D) = (\widetilde{T}, \widetilde{D})\) equals to
\[
  f_{(S, D)} = f_{(\widetilde{T}, \widetilde{D})} = x + y + a_1 x^{-1} y^{-1}.
\]

% Figure environment removed

\subsubsection{Smooth del Pezzo surface of degree 8}\label{subsubsection:dP_08}

Let \(S \xrightarrow{\varphi_1} S'\) be a blow-up of del Pezzo surface \(S'\) from the previous case, and \(D = \alpha_1 l + \alpha_2 e_1 \in \Pic(S) \otimes_{\mathbb{Z}} \mathbb{C}\) be a divisor on \(S\), where \(e_1\) is the \(\varphi_1\)-exceptional divisor. The surface \(S\) admits a unique Gorenstein toric degeneration \(T_3 \simeq S\). Let us choose \(T_3\) as a Gorenstein toric degeneration \(T\) of the surface \(S\), and let \(\widetilde{D} \in \Pic(\widetilde{T}) \otimes_{\mathbb{Z}} \mathbb{C}\) be the corresponding divisor on the crepant resolution \(\widetilde{T} \simeq S\) under the identification \(\Pic(S) \simeq \Pic(\widetilde{T})\).

% Figure environment removed

Coefficients corresponding to the neighbour points \(L, R\) of the point \(K\) have the form \(c_L = a_1\) and \(c_R = 1\). Consequently, the toric Landau--Ginzburg model for the pair \((S, D) = (\widetilde{T}, \widetilde{D})\) equals to
\[
  f_{(S, D)} = f_{(\widetilde{T}, \widetilde{D})} =
  f_{(\widetilde{T'}, \widetilde{D'})} + c_L c_R a_2 x^{-1} = 
  x + y + a_1 x^{-1} y^{-1} + a_1 a_2 x^{-1}.
\]
where \((\widetilde{T'}, \widetilde{D'})\) is the crepant resolution of the Gorenstein toric degeneration from the previous case.

\subsubsection{Smooth del Pezzo surface of degree 7}\label{subsubsection:dP_07}

Let \(S \xrightarrow{\varphi_2} S'\) be a blow-up of del Pezzo surface \(S'\) from the previous case, and \(D = \alpha_1 l + \sum_{i = 1}^2 \alpha_{i + 1} e_i \in \Pic(S) \otimes_{\mathbb{Z}} \mathbb{C}\) be a divisor on \(S\), where \(e_2\) is the \(\varphi_2\)-exceptional divisor. The surface \(S\) admits two Gorenstein toric degenerations: \(T_6 \simeq S\) and \(T_5\). Let us choose \(T_6\) as a Gorenstein toric degeneration \(T\) of the surface \(S\), and let \(\widetilde{D} \in \Pic(\widetilde{T}) \otimes_{\mathbb{Z}} \mathbb{C}\) be a divisor on the crepant resolution \(\widetilde{T} \simeq S\) under the identification \(\Pic(S) \simeq \Pic(\widetilde{T})\).

% Figure environment removed

Coefficients corresponding to the neighbour points \(L, R\) of the point \(K\) have the form \(c_L = 1\) and \(c_R = a_1\). Consequently, the toric Landau--Ginzburg model for the pair \((S, D) = (\widetilde{T}, \widetilde{D})\) equals to
\[
  f_{(S, D)} = f_{(\widetilde{T}, \widetilde{D})} =
  f_{(\widetilde{T'}, \widetilde{D'})} + c_L c_R a_3 y^{-1} =
  x + y + a_1 x^{-1} y^{-1} + a_1 a_2 x^{-1} + a_1 a_3 y^{-1},
\]
where \((\widetilde{T'}, \widetilde{D'})\) is the crepant resolution of the Gorenstein toric degeneration from the previous case.

\subsubsection{Smooth del Pezzo surface of degree 6}\label{subsubsection:dP_06}

Let \(S \xrightarrow{\varphi_3} S'\) be a blow-up of del Pezzo surface \(S'\) from the previous case, and \(D = \alpha_1 l + \sum_{i = 1}^3 \alpha_{i + 1} e_i \in \Pic(S) \otimes_{\mathbb{Z}} \mathbb{C}\) be a divisor on \(S\), where \(e_3\) is the \(\varphi_3\)-exceptional divisor. The surface \(S\) admits four Gorenstein toric degenerations: \(T_7\), \(T_8\), \(T_9\), and \(T_{10} \simeq S\). Let us choose \(T_{10}\) as a Gorenstein toric degeneration \(T\) of the surface \(S\), and let \(\widetilde{D} \in \Pic(\widetilde{T}) \otimes_{\mathbb{Z}} \mathbb{C}\) be a divisor on the crepant resolution \(\widetilde{T} \simeq S\) under the identification \(\Pic(S) \simeq \Pic(\widetilde{T})\).

% Figure environment removed

Coefficients corresponding to the neighbour points \(L, R\) of the point \(K\) have the form \(c_L = 1\) and \(c_R = 1\). Consequently, the toric Landau--Ginzburg model for the pair \((S, D) = (\widetilde{T}, \widetilde{D})\) equals to
\[
  f_{(S, D)} = f_{(\widetilde{T}, \widetilde{D})} =
  f_{(\widetilde{T'}, \widetilde{D'})} + c_L c_R a_4 x y =
  x + y + a_1 x^{-1} y^{-1} + a_1 a_2 x^{-1} + a_1 a_3 y^{-1} + a_4 x y.
\]
where \((\widetilde{T'}, \widetilde{D'})\) is the crepant resolution of the Gorenstein toric degeneration from the previous case.

\subsubsection{Smooth del Pezzo surface of degree 5}\label{subsubsection:dP_05}

Let \(S \xrightarrow{\varphi_4} S'\) be a blow-up of del Pezzo surface \(S'\) from the previous case, and \(D = \alpha_1 l + \sum_{i = 1}^4 \alpha_{i + 1} e_i \in \Pic(S) \otimes_{\mathbb{Z}} \mathbb{C}\) be a divisor on \(S\), where \(e_4\) is the \(\varphi_4\)-exceptional divisor. The surface \(S\) admits two Gorenstein toric degenerations: \(T_{11}\) and \(T_{12}\). Let us choose \(T_{12}\) as a Gorenstein toric degeneration \(T\) of the surface \(S\), and let \(\widetilde{D} \in \Pic(\widetilde{T}) \otimes_{\mathbb{Z}} \mathbb{C}\) be a divisor on the crepant resolution \(\widetilde{T}\) under the identification \(\Pic(S) \simeq \Pic(\widetilde{T})\).

% Figure environment removed

Coefficients corresponding to the neighbour points \(L, R\) of the point \(K\) have the form \(c_L = a_1 a_2\) and \(c_R = 1\). Consequently, the toric Landau--Ginzburg model for the pair \((\widetilde{T}, \widetilde{D})\) equals to
\[
  f_{(\widetilde{T}, \widetilde{D})} =
  f_{(\widetilde{T'}, \widetilde{D'})} + c_L c_R a_5 x^{-1} y =
  x + y + a_1 x^{-1} y^{-1} + a_1 a_2 x^{-1} + a_1 a_3 y^{-1} + a_4 x y + a_1 a_2 a_5 x^{-1} y.
\]
where \((\widetilde{T'}, \widetilde{D'})\) is the crepant resolution of the Gorenstein toric degeneration from the previous case.

To obtain the toric Landau--Ginzburg model for the pair \((S, D)\), we have to modify the coefficients of \(f_{(\widetilde{T}, \widetilde{D})}\) corresponding to non-boundary non-integral points. Let us consider the facets of the fan polygon containing these points. Their marking polynomials and corresponding modified coefficients have the form
\begin{gather*}
  a_1 a_2 a_5 s^2 + a_1 (a_2 + a_5) s + a_1 =
  c_K s^2 + \widetilde{c}_L s + c_V, \\
  a_4 s^2 + (a_1 a_2 a_4 a_5 + 1) s + a_1 a_2 a_5 =
  c_W s^2 + \widetilde{c}_R s + c_K.
\end{gather*}
Consequently, the toric Landau--Ginzburg model for the pair \((S, D)\) equals to
\[
  f_{(S, D)} = x + (a_1 a_2 a_4 a_5 + 1) y + a_1 x^{-1} y^{-1} +
  a_1 (a_2 + a_5) x^{-1} + a_1 a_3 y^{-1} + a_4 x y + a_1 a_2 a_5 x^{-1} y.
\]

\subsubsection{Smooth del Pezzo surface of degree 4}\label{subsubsection:dP_04}

Let \(S \xrightarrow{\varphi_5} S'\) be a blow-up of del Pezzo surface \(S'\) from the previous case, and \(D = \alpha_1 l + \sum_{i = 1}^5 \alpha_{i + 1} e_i \in \Pic(S) \otimes_{\mathbb{Z}} \mathbb{C}\) be a divisor on \(S\), where \(e_5\) is the \(\varphi_5\)-exceptional divisor. The surface \(S\) admits three Gorenstein toric degenerations: \(T_{13}\), \(T_{14}\) and \(T_{15}\). Let us choose \(T_{14}\) as a Gorenstein toric degeneration \(T\) of the surface \(S\), and let \(\widetilde{D} \in \Pic(\widetilde{T}) \otimes_{\mathbb{Z}} \mathbb{C}\) be a divisor on the crepant resolution \(\widetilde{T}\) under the identification \(\Pic(S) \simeq \Pic(\widetilde{T})\).

% Figure environment removed

Coefficients corresponding to the neighbour points \(L, R\) of the point \(K\) have the form \(c_L = a_1 a_3\) and \(c_R = a_1\). Consequently, the toric Landau--Ginzburg model for the pair \((\widetilde{T}, \widetilde{D})\) equals to
\begin{gather*}
  f_{(\widetilde{T}, \widetilde{D})} =
  f_{(\widetilde{T'}, \widetilde{D'})} + c_L c_R a_6 x^{-1} y^{-2} = \\
  x + y + a_1 x^{-1} y^{-1} + a_1 a_2 x^{-1} + 
  a_1 a_3 y^{-1} + a_4 x y + a_1 a_2 a_5 x^{-1} y +
  a_1^2 a_3 a_6 x^{-1} y^{-2},
\end{gather*}
where \((\widetilde{T'}, \widetilde{D'})\) is the crepant resolution of the Gorenstein toric degeneration from the previous case.

To obtain the toric Landau--Ginzburg model for the pair \((S, D)\), we have to modify the coefficients of \(f_{(\widetilde{T}, \widetilde{D})}\) corresponding to non-boundary non-integral points. Let us consider the facets of the fan polygon containing these points. Their marking polynomials and corresponding modified coefficients have the form
\begin{gather*}
  a_1 a_2 a_5 s^3 + a_1 (a_1 a_2 a_3 a_5 a_6 + a_2 + a_5) s^2 +
  a_1 (a_1 a_3 a_6 (a_2 + a_5) + 1) s + a_1^2 a_3 a_6 =
  c_U s^3 + \widetilde{c}_A s^2 + \widetilde{c}_R s + c_K, \\
  a_4 s^2 + (a_1 a_2 a_4 a_5 + 1) s + a_1 a_2 a_5 =
  c_V s^2 + \widetilde{c}_B s + c_U, \\
  a_1^2 a_3 a_6 s^2 + a_1 (a_3 + a_6) s + 1 =
  c_K s^2 + \widetilde{c}_L s + c_W.
\end{gather*}
Consequently, the toric Landau--Ginzburg model for the pair \((S, D)\) equals to
\begin{gather*}
  f_{(S, D)} = x + (a_1 a_2 a_4 a_5 + 1) y + a_1 (a_1 a_3 a_6 (a_2 + a_5) + 1) x^{-1} y^{-1} + \\ a_1 (a_1 a_2 a_3 a_5 a_6 + a_2 + a_5) x^{-1} + a_1 (a_3 + a_6) y^{-1} + a_4 x y + a_1 a_2 a_5 x^{-1} y + a_1^2 a_3 a_6 x^{-1} y^{-2}.
\end{gather*}

\subsubsection{Smooth del Pezzo surface of degree 3}\label{subsubsection:dP_03}

Let \(S \xrightarrow{\varphi_6} S'\) be a blow-up of del Pezzo surface \(S'\) from the previous case, and \(D = \alpha_1 l + \sum_{i = 1}^6 \alpha_{i + 1} e_i \in \Pic(S) \otimes_{\mathbb{Z}} \mathbb{C}\) be a divisor on \(S\), where \(e_6\) is the \(\varphi_6\)-exceptional divisor. The surface \(S\) admits a unique Gorenstein toric degenerations \(T_{16}\). Let us choose \(T_{16}\) as a Gorenstein toric degeneration \(T\) of the surface \(S\), and let \(\widetilde{D}\) be a divisor on the crepant resolution \(\widetilde{T} \in \Pic(\widetilde{T}) \otimes_{\mathbb{Z}} \mathbb{C}\) under the identification \(\Pic(S) \simeq \Pic(\widetilde{T})\).

% Figure environment removed

Coefficients corresponding to the neighbour points \(L, R\) of the point \(K\) have the form \(c_L = a_4\) and \(c_R = 1\). Consequently, the toric Landau--Ginzburg model for the pair \((\widetilde{T}, \widetilde{D})\) equals to
\begin{gather*}
  f_{(\widetilde{T}, \widetilde{D})} =
  f_{(\widetilde{T'}, \widetilde{D'})} + c_L c_R a_7 x^2 y = \\
  x + y + a_1 x^{-1} y^{-1} + a_1 a_2 x^{-1} + 
  a_1 a_3 y^{-1} + a_4 x y + a_1 a_2 a_5 x^{-1} y +
  a_1^2 a_3 a_6 x^{-1} y^{-2} + a_4 a_7 x^2 y.
\end{gather*}
where \((\widetilde{T'}, \widetilde{D'})\) is the crepant resolution of the Gorenstein toric degeneration from the previous case.

To obtain the toric Landau--Ginzburg model for the pair \((S, D)\), we have to modify the coefficients of \(f_{(\widetilde{T}, \widetilde{D})}\) corresponding to non-boundary non-integral points. Let us consider the facets of the fan polygon containing these points. Their marking polynomials and corresponding modified coefficients have the form
\begin{gather*}
  a_1 a_2 a_5 s^3 + a_1 (a_1 a_2 a_3 a_5 a_6 + a_2 + a_5) s^2 + a_1 (a_1 a_3 a_6 (a_2 + a_5) + 1) s + a_1^2 a_3 a_6 = c_V s^3 + c_D s^2 + c_C s + c_W, \\
  a_4 a_7 s^3 + (a_1 a_2 a_4 a_5 a_7 + a_4 + a_7) s^2 + (a_1 a_2 a_5 (a_4 + a_7) + 1) s + a_1 a_2 a_5 = c_K s^3 + c_L s^2 + c_A s + c_V, \\
  a_1^2 a_3 a_6 s^3 + a_1 (a_1 a_3 a_4 a_6 a_7 + a_3 + a_6) s^2 + (a_1 a_4 a_7 (a_3 + a_6) + 1) s + a_4 a_7 = c_W s^3 + c_C s^2 + c_D s + c_W.
\end{gather*}

Consequently, the toric Landau--Ginzburg model for the pair \((S, D)\) equals to
\begin{gather*}
  f_{(S, D)} = 
  (a_1 a_4 a_7 (a_3 + a_6) + 1) x + (a_1 a_2 a_5 (a_4 + a_7) + 1) y + 
  a_1 (a_1 a_3 a_6 (a_2 + a_5) + 1) x^{-1} y^{-1} + \\
  a_1 (a_1 a_2 a_3 a_5 a_6 + a_2 + a_5) x^{-1} +
  a_1 (a_1 a_3 a_4 a_6 a_7 + a_3 + a_6) y^{-1} + 
  (a_1 a_2 a_4 a_5 a_7 + a_4 + a_7) x y + \\
  a_1 a_2 a_5 x^{-1} y +
  a_1^2 a_3 a_6 x^{-1} y^{-2} +
  a_4 a_7 x^2 y.
\end{gather*}
%%% Local Variables:
%%% mode: latex
%%% TeX-master: "../main"
%%% End:
