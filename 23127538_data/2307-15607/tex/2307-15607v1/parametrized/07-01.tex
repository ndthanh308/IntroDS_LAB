\subsection{Family \textnumero7.1}\label{subsection:07-01_parametrised}

The toric Landau--Ginzburg model for this family is given by the Laurent polynomial
\begin{gather*}
  \left(a_1^{2} a_3 a_6\right) x^{2} y^{-1} + \left(a_1^{2} a_2 a_3 a_6 + a_1^{2} a_3 a_5 a_6 + a_1\right) x + \left(a_1^{2} a_2 a_3 a_5 a_6 + a_1 a_2 + a_1 a_5\right) y + \\ \left(a_1 a_2 a_5\right) x^{-1} y^{2} + z + \left(a_1 a_3 + a_1 a_6\right) x y^{-1} + \left(a_1 a_2 a_4 a_5 + 1\right) x^{-1} y + a_7 z^{-1} + y^{-1} + a_4 x^{-1}
\end{gather*}
(see Subsubsection~\ref{subsubsection:dPxP1_07-01}). The pencil \(\mathcal{S}(\overline{a})\) is defined by the equation
\begin{gather*}
  \left(a_1^{2} a_3 a_6\right) X^{3} Z + \left(a_1^{2} a_2 a_3 a_6 + a_1^{2} a_3 a_5 a_6 + a_1\right) X^{2} Y Z + \left(a_1^{2} a_2 a_3 a_5 a_6 + a_1 a_2 + a_1 a_5\right) X Y^{2} Z + \left(a_1 a_2 a_5\right) Y^{3} Z + \\ X Y Z^{2} + \left(a_1 a_3 + a_1 a_6\right) X^{2} Z T + \left(a_1 a_2 a_4 a_5 + 1\right) Y^{2} Z T + a_7 X Y T^{2} + X Z T^{2} + a_4 Y Z T^{2} = \lambda X Y Z T.
\end{gather*}
The following members \(\mathcal{S}(\overline{a})_{\lambda}\) of the pencil are reducible:
\begin{gather*}
  \mathcal{S}(\overline{a})_{\infty} = S_{(X)} + S_{(Y)} + S_{(Z)} + S_{(T)}.
\end{gather*}
The base locus of the pencil \(\mathcal{S}(\overline{a})\) consists of the following curves:
\begin{gather*}
  C_{1} = C_{(X, Y)}, \;
  C_{2} = C_{(X, Z)}, \;
  C_{3} = C_{(Y, Z)}, \;
  C_{4} = C_{(Z, T)}, \;
  C_{5} = C_{(X, Y + a_4 T)}, \\
  C_{6} = C_{(X, a_1 a_2 a_5 Y + T)}, \;
  C_{7} = C_{(Y, a_1 a_3 X + T)}, \;
  C_{8} = C_{(Y, a_1 a_6 X + T)}, \;
  C_{9} = C_{(T, a_1 (X + a_2 Y) (X + a_5 Y) (a_1 a_3 a_6 X + Y) + X Y Z)}.
\end{gather*}

Their linear equivalence classes on the generic member \(\mathcal{S}(\overline{a})_{\Bbbk}\) of the pencil satisfy the following relations:
\begin{gather*}
  \begin{pmatrix}
    [S_{\Bbbk} \cdot S_{(X)}] - [H_{\mathcal{S}}] \\
    [S_{\Bbbk} \cdot S_{(Y)}] - [H_{\mathcal{S}}] \\
    [S_{\Bbbk} \cdot S_{(Z)}] - [H_{\mathcal{S}}] \\
    [S_{\Bbbk} \cdot S_{(T)}] - [H_{\mathcal{S}}]
  \end{pmatrix} = 
  \begin{pmatrix}
    1 & 1 & 0 & 0 & 1 & 1 & 0 & 0 & 0 & -1 \\
    1 & 0 & 1 & 0 & 0 & 0 & 1 & 1 & 0 & -1 \\
    0 & 1 & 1 & 2 & 0 & 0 & 0 & 0 & 0 & -1 \\
    0 & 0 & 0 & 1 & 0 & 0 & 0 & 0 & 1 & -1
  \end{pmatrix} \cdot
  \begin{pmatrix}
    [C_1] \\ \cdots \\ [C_{9}] \\ [H_{\mathcal{S}}]
  \end{pmatrix}.
\end{gather*}
We can reduce the number of linear equivalence classes using these identities:
\begin{gather*}
  \begin{pmatrix}
    [C_{2}] \\ [C_{6}] \\ [C_{8}] \\ [C_{9}]
  \end{pmatrix} = 
  \begin{pmatrix}
    0 & -1 & -2 & 0 & 0 & 1 \\
    -1 & 1 & 2 & -1 & 0 & 0 \\
    -1 & -1 & 0 & 0 & -1 & 1 \\
    0 & 0 & -1 & 0 & 0 & 1
  \end{pmatrix} \cdot
  \begin{pmatrix}
    [C_{1}] \\ [C_{3}] \\ [C_{4}] \\ [C_{5}] \\ [C_{7}] \\ [H_{\mathcal{S}}]
  \end{pmatrix}.
\end{gather*}

For a general choice of \(\lambda \in \mathbb{C}\) and \(\overline{a} \in (\mathbb{C}^*)^7\) the surface \(\mathcal{S}(\overline{a})_{\lambda}\) has the following singularities:
\begin{itemize}\setlength{\itemindent}{2cm}
\item[\(P_{1} = P_{(X, Y, Z)}\):] type \(\mathbb{A}_1\) with the quadratic term \(a_7 X Y + a_4 Y Z + X Z\);
\item[\(P_{2} = P_{(X, Y, T)}\):] type \(\mathbb{A}_3\) with the quadratic term \(X \cdot Y\);
\item[\(P_{3} = P_{(Z, T, X + a_2 Y)}\):] type \(\mathbb{A}_1\) with the quadratic term
  \[
    a_2 (Z^2 + a_7 T^2) - (a_1 a_2 (a_2 (a_3 + a_6) + a_4 a_5) + a_2 \lambda + 1) Z T - a_1 (a_2 - a_5) (a_1 a_2 a_3 a_6 - 1) Z (X + a_2 Y);
  \]
\item[\(P_{4} = P_{(Z, T, X + a_5 Y)}\):] type \(\mathbb{A}_1\) with the quadratic term
  \[
    a_5 (Z^2 + a_7 T^2) - (a_1 a_5 (a_5 (a_3 + a_6) + a_2 a_4) + a_5 \lambda + 1) Z T + a_1 (a_2 - a_5) (a_1 a_3 a_5 a_6 - 1) Z (X + a_5 Y); 
  \]
\item[\(P_{5} = P_{(Z, T, a_1 a_3 a_6 X + Y)}\):] type \(\mathbb{A}_1\) with the quadratic term
  \[
    a_3 a_6 (Z^2 + a_7 T^2) - (a_1 (a_3 a_6)^2 (a_1 a_2 a_4 a_5 + 1) + a_3 a_6 \lambda + a_3 + a_6) Z T - (a_1 a_2 a_3 a_6 - 1) (a_1 a_3 a_5 a_6 - 1) Z (a_1 a_3 a_6 X + Y).
  \]
\end{itemize}

The \(\mathbb{Q}\)-valued intersection matrix on the group \(A^{\overline{a}}_{\mathcal{S}}\) has the following form:
\begin{table}[H]
  \renewcommand\arraystretch{1.42}
  \begin{tabular}{|c||c|c|c|c|c|c|}
    \hline
    \(\bullet\) & \(C_{1}\) & \(C_{3}\) & \(C_{4}\) & \(C_{5}\) & \(C_{7}\) & \(H_{\mathcal{S}}\) \\
    \hline
    \hline
    \(C_{1}\) & $-\frac{1}{2}$ & $\frac{1}{2}$ & $0$ & $\frac{1}{2}$ & $\frac{1}{2}$ & $1$ \\
    \hline
    \(C_{3}\) & $\frac{1}{2}$ & $-\frac{3}{2}$ & $1$ & $0$ & $1$ & $1$ \\
    \hline
    \(C_{4}\) & $0$ & $1$ & $-\frac{1}{2}$ & $0$ & $0$ & $1$ \\
    \hline
    \(C_{5}\) & $\frac{1}{2}$ & $0$ & $0$ & $-\frac{5}{4}$ & $\frac{1}{4}$ & $1$ \\
    \hline
    \(C_{7}\) & $\frac{1}{2}$ & $1$ & $0$ & $\frac{1}{4}$ & $-\frac{5}{4}$ & $1$ \\
    \hline
    \(H_{\mathcal{S}}\) & $1$ & $1$ & $1$ & $1$ & $1$ & $4$ \\
    \hline
  \end{tabular}.
\end{table}
Note that the intersection matrix has the rank 6.

Let \(\widetilde{\mathcal{S}(\overline{a})_{\lambda}}\) and \(\widetilde{\mathcal{S}(\overline{a})_{\Bbbk}}\) be the minimal resolution of a general member \(\mathcal{S}(\overline{a})_{\lambda}\) and the generic member \(\mathcal{S}(\overline{a})_{\Bbbk}\) of the pencil, respectively. Denote by \(L^{\overline{a}}_{\lambda} \subset \Pic(\widetilde{\mathcal{S}(\overline{a})_{\lambda}})\) and \(L^{\overline{a}}_{\Bbbk} \subset \Pic(\widetilde{\mathcal{S}(\overline{a})_{\Bbbk}})\) the subgroups generated by linear equivalence classes of exceptional divisors of the resolution and by linear equivalence classes of strict transforms \(\widetilde{C_i}\), \(\widetilde{H_{\mathcal{S}(\overline{a})}}\) of the above-introduced curves. By construction \(L^{\overline{a}}_{\lambda}\) is equipped with the \(\Gal(\Bbbk)\)-action, and \(L^{\overline{a}}_{\Bbbk}\) can be identified with the subgroup \((L^{\overline{a}}_{\lambda})^{\Gal(\Bbbk)} \subset L^{\overline{a}}_{\lambda}\). In our case we have \(L^{\overline{a}}_{\lambda} = L^{\overline{a}}_{\Bbbk}\).

The intersection matrix on the lattice \(L^{\overline{a}}_{\lambda}\) has the following form:
\begin{table}[H]
  \begin{tabular}{|c||c|ccc|c|c|c|cccccc|}
    \hline
    \(\bullet\) & \(E_1^1\) & \(E_2^1\) & \(E_2^2\) & \(E_2^3\) & \(E_3^1\) & \(E_4^1\) & \(E_5^1\) & \(\widetilde{C_{1}}\) & \(\widetilde{C_{3}}\) & \(\widetilde{C_{4}}\) & \(\widetilde{C_{5}}\) & \(\widetilde{C_{7}}\) & \(\widetilde{H_{\mathcal{S}}}\) \\
    \hline
    \hline
    \(E_1^1\) & $-2$ & $0$ & $0$ & $0$ & $0$ & $0$ & $0$ & $1$ & $1$ & $0$ & $0$ & $0$ & $0$ \\
    \hline
    \(E_2^1\) & $0$ & $-2$ & $1$ & $0$ & $0$ & $0$ & $0$ & $0$ & $0$ & $0$ & $1$ & $0$ & $0$ \\
    \(E_2^2\) & $0$ & $1$ & $-2$ & $1$ & $0$ & $0$ & $0$ & $1$ & $0$ & $0$ & $0$ & $0$ & $0$ \\
    \(E_2^3\) & $0$ & $0$ & $1$ & $-2$ & $0$ & $0$ & $0$ & $0$ & $0$ & $0$ & $0$ & $1$ & $0$ \\
    \hline
    \(E_3^1\) & $0$ & $0$ & $0$ & $0$ & $-2$ & $0$ & $0$ & $0$ & $0$ & $1$ & $0$ & $0$ & $0$ \\
    \hline
    \(E_4^1\) & $0$ & $0$ & $0$ & $0$ & $0$ & $-2$ & $0$ & $0$ & $0$ & $1$ & $0$ & $0$ & $0$ \\
    \hline
    \(E_5^1\) & $0$ & $0$ & $0$ & $0$ & $0$ & $0$ & $-2$ & $0$ & $0$ & $1$ & $0$ & $0$ & $0$ \\
    \hline
    \(\widetilde{C_{1}}\) & $1$ & $0$ & $1$ & $0$ & $0$ & $0$ & $0$ & $-2$ & $0$ & $0$ & $0$ & $0$ & $1$ \\
    \(\widetilde{C_{3}}\) & $1$ & $0$ & $0$ & $0$ & $0$ & $0$ & $0$ & $0$ & $-2$ & $1$ & $0$ & $1$ & $1$ \\
    \(\widetilde{C_{4}}\) & $0$ & $0$ & $0$ & $0$ & $1$ & $1$ & $1$ & $0$ & $1$ & $-2$ & $0$ & $0$ & $1$ \\
    \(\widetilde{C_{5}}\) & $0$ & $1$ & $0$ & $0$ & $0$ & $0$ & $0$ & $0$ & $0$ & $0$ & $-2$ & $0$ & $1$ \\
    \(\widetilde{C_{7}}\) & $0$ & $0$ & $0$ & $1$ & $0$ & $0$ & $0$ & $0$ & $1$ & $0$ & $0$ & $-2$ & $1$ \\
    \(\widetilde{H_{\mathcal{S}}}\) & $0$ & $0$ & $0$ & $0$ & $0$ & $0$ & $0$ & $1$ & $1$ & $1$ & $1$ & $1$ & $4$ \\
    \hline
  \end{tabular}.
\end{table}
The intersection matrix is non-degenerate, hence we have \(\rk(L^{\overline{a}}_{\lambda}) = 13\). Recall that \(\Pic(\widetilde{\mathcal{S}(\overline{a})_{\lambda}})\) is generated by linear equivalence classes of the exceptional curves \(E_i^j\) and of strict transforms of curves on \(\mathcal{S}(\overline{a})_{\lambda}\). Denote by \(M\) the corresponding intersection matrix.

The conjectural orthogonal complement to the lattice \(L_{\mathcal{S}}\) has the form
\[
  N = H \oplus \Pic(X), \quad
  \Pic(X) =
  \begin{pmatrix}
    -2 & 0 & 0 & 0 & 0 & -1 & 0 \\
    0 & -2 & 0 & 0 & 0 & -1 & 0 \\
    0 & 0 & -2 & 0 & 0 & -1 & 0 \\
    0 & 0 & 0 & -2 & 0 & -1 & 0 \\
    0 & 0 & 0 & 0 & -2 & -1 & 0 \\
    -1 & -1 & -1 & -1 & -1 & 0 & 3 \\
    0 & 0 & 0 & 0 & 0 & 3 & 2
  \end{pmatrix}.
\]
We choose the following generators of discriminant groups \(D_M\) and \(D_N\) of the lattices \(M\) and \(N\), respectively:
\begin{gather*}
  \begin{pmatrix}
    \frac{1}{2} & \frac{1}{2} & 0 & 0 & \frac{1}{2} & 0 & \frac{1}{2} & \frac{1}{2} & \frac{1}{2} & 0 & 0 & 0 & \frac{1}{2} \\
    \frac{1}{2} & 0 & \frac{1}{2} & 0 & 0 & 0 & 0 & 0 & 0 & 0 & \frac{1}{2} & \frac{1}{2} & 0 \\
    \frac{1}{2} & 0 & \frac{1}{2} & 0 & \frac{1}{2} & \frac{1}{2} & 0 & 0 & 0 & 0 & \frac{1}{2} & \frac{1}{2} & 0 \\
    0 & 0 & 0 & 0 & \frac{1}{2} & 0 & \frac{1}{2} & 0 & 0 & 0 & 0 & 0 & 0 \\
    \frac{1}{8} & \frac{3}{8} & \frac{1}{4} & \frac{5}{8} & \frac{3}{8} & \frac{3}{8} & \frac{3}{8} & \frac{1}{2} & \frac{3}{4} & \frac{3}{4} & \frac{1}{2} & 0 & \frac{5}{8}
  \end{pmatrix}, \quad
  \begin{pmatrix}
    0 & 0 & \frac{1}{2} & \frac{1}{2} & \frac{1}{2} & 0 & \frac{1}{2} & 0 & 0 \\
    0 & 0 & \frac{1}{2} & \frac{1}{2} & 0 & \frac{1}{2} & \frac{1}{2} & 0 & 0 \\
    0 & 0 & \frac{1}{2} & 0 & 0 & 0 & \frac{1}{2} & 0 & 0 \\
    0 & 0 & \frac{1}{2} & \frac{1}{2} & 0 & 0 & 0 & 0 & 0 \\
    0 & 0 & \frac{1}{8} & \frac{1}{8} & \frac{1}{8} & \frac{1}{8} & \frac{1}{8} & \frac{3}{4} & \frac{7}{8}
  \end{pmatrix}.
\end{gather*}
Then bilinear and quadratic discriminant forms on \(D_M\) and \(D_N\) can be represented as
\begin{gather*}
  B_M = 
  \begin{pmatrix}
    0 & \frac{1}{2} & 0 & 0 & 0 \\
    \frac{1}{2} & 0 & 0 & 0 & 0 \\
    0 & 0 & 0 & \frac{1}{2} & 0 \\
    0 & 0 & \frac{1}{2} & 0 & 0 \\
    0 & 0 & 0 & 0 & \frac{5}{8}
  \end{pmatrix}, \quad
  B_N = 
  \begin{pmatrix}
    0 & \frac{1}{2} & 0 & 0 & 0 \\
    \frac{1}{2} & 0 & 0 & 0 & 0 \\
    0 & 0 & 0 & \frac{1}{2} & 0 \\
    0 & 0 & \frac{1}{2} & 0 & 0 \\
    0 & 0 & 0 & 0 & \frac{3}{8}
  \end{pmatrix}; \quad
  Q_M = \left(0,\,0,\,1,\,1,\,\frac{13}{8}\right), \quad
  Q_N = \left(0,\,0,\,1,\,1,\,\frac{3}{8}\right).
\end{gather*}
The lattices \(M\) and \(N\) have the signature \((1, 12)\) and \((2, 7)\), respectively, hence \(M^{\perp} \simeq N\) in the K3 lattice.

%%% Local Variables:
%%% mode: latex
%%% TeX-master: "../main"
%%% End:
