\subsection{Family \textnumero2.7}\label{subsection:02-07}

The toric Landau--Ginzburg model for this family is given by the Laurent polynomial
\begin{gather*}
  x y z^{-1} + x + 2 y + z + x^{-1} y z + 2 x z^{-1} + 2 y z^{-1} + x y^{-1} + 2 x^{-1} y + \\ y^{-1} z + 2 x^{-1} z + x y^{-1} z^{-1} + 2 z^{-1} + x^{-1} y z^{-1} + 2 y^{-1} + 2 x^{-1} + x^{-1} y^{-1} z
\end{gather*}
(see~\cite[Family \textnumero2.7]{cheltsov2018katzarkov}). It is a Minkowski polynomial \textnumero3239 (see~\cite[Appendix~B: bucket~148]{akhtar2012minkowski}).

The pencil \(\mathcal{S}\) is defined by the equation
\begin{gather*}
  X^{2} Y^{2} + X^{2} Y Z + 2 X Y^{2} Z + X Y Z^{2} + Y^{2} Z^{2} + 2 X^{2} Y T + 2 X Y^{2} T + X^{2} Z T + 2 Y^{2} Z T + \\ X Z^{2} T + 2 Y Z^{2} T + X^{2} T^{2} + 2 X Y T^{2} + Y^{2} T^{2} + 2 X Z T^{2} + 2 Y Z T^{2} + Z^{2} T^{2} = \lambda X Y Z T.
\end{gather*}

\begin{remark}
  Note that the equation is invariant under the permutations \((X, Z) \mapsto (Z, X)\), \((Y, T) \mapsto (T, Y)\).
\end{remark}

Members \(\mathcal{S}_{\lambda}\) of the pencil are irreducible for any parameter \(\lambda \in \mathbb{P}^1\) except
\begin{gather*}
  \mathcal{S}_{\infty} = S_{(X)} + S_{(Y)} + S_{(Z)} + S_{(T)}, \quad
  \mathcal{S}_{- 5} = S_{((X + Z) (Y + T) + Y T)} + S_{((X + T) (Y + Z) + X T + Y Z)}.
\end{gather*}
The base locus of the pencil \(\mathcal{S}\) consists of the following curves:
\begin{gather*}
  C_{1} = C_{(Y, T)}, \;
  C_{2} = C_{(Y, X + Z)}, \;
  C_{3} = C_{(T, X + Z)}, \;
  C_{4} = C_{(X, Y T + Z (Y + T))}, \\
  C_{5} = C_{(Y, X Z + T (X + Z))}, \;
  C_{6} = C_{(Z, Y T + X (Y + T))}, \;
  C_{7} = C_{(T, X Z + Y (X + Z))}.
\end{gather*}

Their linear equivalence classes on the generic member \(\mathcal{S}_{\Bbbk}\) of the pencil satisfy the following relations:
\begin{gather*}
  \begin{pmatrix}
    [\mathcal{S}_{\Bbbk} \cdot S_{(X)}] - [H_{\mathcal{S}}] \\
    [\mathcal{S}_{\Bbbk} \cdot S_{(Y)}] - [H_{\mathcal{S}}] \\
    [\mathcal{S}_{\Bbbk} \cdot S_{(Z)}] - [H_{\mathcal{S}}] \\
    [\mathcal{S}_{\Bbbk} \cdot S_{(T)}] - [H_{\mathcal{S}}]
  \end{pmatrix} = 
  \begin{pmatrix}
    0 & 0 & 0 & 2 & 0 & 0 & 0 & -1 \\
    1 & 1 & 0 & 0 & 1 & 0 & 0 & -1 \\
    0 & 0 & 0 & 0 & 0 & 2 & 0 & -1 \\
    1 & 0 & 1 & 0 & 0 & 0 & 1 & -1
  \end{pmatrix} \cdot
  \begin{pmatrix}
    [C_1] \\ \cdots \\ [C_7] \\ [H_{\mathcal{S}}]
  \end{pmatrix} = 0.
\end{gather*}
We can partially reduce the number of linear equivalence classes using these relations:
\[
  \begin{pmatrix}
    [C_5] \\ 2 [C_6] \\ [C_7] \\ [H_{\mathcal{S}}]
  \end{pmatrix} =
  \begin{pmatrix}
    -1 & -1 & 0 & 2 \\
    0 & 0 & 0 & 2 \\
    -1 & 0 & -1 & 2 \\
    0 & 0 & 0 & 2
  \end{pmatrix} \cdot
  \begin{pmatrix}
    [C_1] \\ [C_2] \\ [C_3] \\ [C_4]
  \end{pmatrix}.
\]

For a general choice of \(\lambda \in \mathbb{C}\) the surface \(\mathcal{S}_{\lambda}\) has the following singularities:
\begin{itemize}\setlength{\itemindent}{2cm}
\item[\(P_{1} = P_{(X, Y, Z)}\):] type \(\mathbb{D}_4\) with the quadratic term \((X + Y + Z)^2\);
\item[\(P_{2} = P_{(X, Y, T)}\):] type \(\mathbb{A}_3\) with the quadratic term \((Y + T) \cdot (X + Y + T)\);
\item[\(P_{3} = P_{(X, Z, T)}\):] type \(\mathbb{D}_4\) with the quadratic term \((X + Z + T)^2\);
\item[\(P_{4} = P_{(Y, Z, T)}\):] type \(\mathbb{A}_3\) with the quadratic term \((Y + T) \cdot (Y + Z + T)\);
\item[\(P_{5} = P_{(Y, T, X + Z)}\):] type \(\mathbb{A}_1\) with the quadratic term \((X + Z) (Y + T) - (\lambda + 4) Y T\).
\end{itemize}

The \(\mathbb{Q}\)-valued intersection matrix on the group \(A_{\mathcal{S}}\) has the following form:
\begin{table}[H]
  \renewcommand\arraystretch{1.42}
  \begin{tabular}{|c||c|c|c|c|c|}
    \hline
    \(\bullet\) & \(C_{1}\) & \(C_{2}\) & \(C_{3}\) & \(C_{4}\) & \(C_{6}\) \\
    \hline
    \hline
    \(C_{1}\) & $0$ & $\frac{1}{2}$ & $\frac{1}{2}$ & $\frac{1}{2}$ & $\frac{1}{2}$ \\
    \hline
    \(C_{2}\) & $\frac{1}{2}$ & $-\frac{1}{2}$ & $\frac{1}{2}$ & $\frac{1}{2}$ & $\frac{1}{2}$ \\
    \hline
    \(C_{3}\) & $\frac{1}{2}$ & $\frac{1}{2}$ & $-\frac{1}{2}$ & $\frac{1}{2}$ & $\frac{1}{2}$ \\
    \hline
    \(C_{4}\) & $\frac{1}{2}$ & $\frac{1}{2}$ & $\frac{1}{2}$ & $1$ & $1$ \\
    \hline
    \(C_{6}\) & $\frac{1}{2}$ & $\frac{1}{2}$ & $\frac{1}{2}$ & $1$ & $1$ \\
    \hline
  \end{tabular}.
\end{table}
Note that the intersection matrix has the rank 3.

Let \(\widetilde{\mathcal{S}_{\lambda}}\) and \(\widetilde{\mathcal{S}_{\Bbbk}}\) be the minimal resolution of a general member \(\mathcal{S}_{\lambda}\) and the generic member \(\mathcal{S}_{\Bbbk}\) of the pencil, respectively. Denote by \(L_{\lambda} \subset \Pic(\widetilde{\mathcal{S}_{\lambda}})\) and \(L_{\mathcal{S}} \subset \Pic(\widetilde{\mathcal{S}_{\Bbbk}})\) the subgroups generated by linear equivalence classes of exceptional divisors of the resolution and by linear equivalence classes of strict transforms \(\widetilde{C_i}\), \(\widetilde{H_{\mathcal{S}}}\) of the above-introduced curves. By construction the subgroup \(L_{\lambda}\) is equipped with the \(\Gal(\Bbbk)\)-action, and the subgroup \(L_{\mathcal{S}}\) can be identified with the subgroup \((L_{\lambda})^{\Gal(\Bbbk)} \subset L_{\lambda}\). In our case we have \(L_{\lambda} = L_{\mathcal{S}}\).

The intersection matrix on the lattice \(L_{\lambda}\) has the following form:
\begin{table}[H]
  \begin{tabular}{|c||cccc|ccc|cccc|ccc|c|ccccc|}
    \hline
    \(\bullet\) & \(E_1^1\) & \(E_1^2\) & \(E_1^3\) & \(E_1^4\) & \(E_2^1\) & \(E_2^2\) & \(E_2^3\) & \(E_3^1\) & \(E_3^2\) & \(E_3^3\) & \(E_3^4\) & \(E_4^1\) & \(E_4^2\) & \(E_4^3\) & \(E_5^1\) & \(\widetilde{C_{1}}\) & \(\widetilde{C_{2}}\) & \(\widetilde{C_{3}}\) & \(\widetilde{C_{4}}\) & \(\widetilde{C_{6}}\) \\
    \hline
    \hline
    \(E_1^1\) & $-2$ & $1$ & $0$ & $0$ & $0$ & $0$ & $0$ & $0$ & $0$ & $0$ & $0$ & $0$ & $0$ & $0$ & $0$ & $0$ & $1$ & $0$ & $0$ & $0$ \\
    \(E_1^2\) & $1$ & $-2$ & $1$ & $1$ & $0$ & $0$ & $0$ & $0$ & $0$ & $0$ & $0$ & $0$ & $0$ & $0$ & $0$ & $0$ & $0$ & $0$ & $0$ & $0$ \\
    \(E_1^3\) & $0$ & $1$ & $-2$ & $0$ & $0$ & $0$ & $0$ & $0$ & $0$ & $0$ & $0$ & $0$ & $0$ & $0$ & $0$ & $0$ & $0$ & $0$ & $1$ & $0$ \\
    \(E_1^4\) & $0$ & $1$ & $0$ & $-2$ & $0$ & $0$ & $0$ & $0$ & $0$ & $0$ & $0$ & $0$ & $0$ & $0$ & $0$ & $0$ & $0$ & $0$ & $0$ & $1$ \\
    \hline
    \(E_2^1\) & $0$ & $0$ & $0$ & $0$ & $-2$ & $1$ & $0$ & $0$ & $0$ & $0$ & $0$ & $0$ & $0$ & $0$ & $0$ & $1$ & $0$ & $0$ & $0$ & $0$ \\
    \(E_2^2\) & $0$ & $0$ & $0$ & $0$ & $1$ & $-2$ & $1$ & $0$ & $0$ & $0$ & $0$ & $0$ & $0$ & $0$ & $0$ & $0$ & $0$ & $0$ & $1$ & $0$ \\
    \(E_2^3\) & $0$ & $0$ & $0$ & $0$ & $0$ & $1$ & $-2$ & $0$ & $0$ & $0$ & $0$ & $0$ & $0$ & $0$ & $0$ & $0$ & $0$ & $0$ & $0$ & $0$ \\
    \hline
    \(E_3^1\) & $0$ & $0$ & $0$ & $0$ & $0$ & $0$ & $0$ & $-2$ & $1$ & $0$ & $0$ & $0$ & $0$ & $0$ & $0$ & $0$ & $0$ & $1$ & $0$ & $0$ \\
    \(E_3^2\) & $0$ & $0$ & $0$ & $0$ & $0$ & $0$ & $0$ & $1$ & $-2$ & $1$ & $1$ & $0$ & $0$ & $0$ & $0$ & $0$ & $0$ & $0$ & $0$ & $0$ \\
    \(E_3^3\) & $0$ & $0$ & $0$ & $0$ & $0$ & $0$ & $0$ & $0$ & $1$ & $-2$ & $0$ & $0$ & $0$ & $0$ & $0$ & $0$ & $0$ & $0$ & $1$ & $0$ \\
    \(E_3^4\) & $0$ & $0$ & $0$ & $0$ & $0$ & $0$ & $0$ & $0$ & $1$ & $0$ & $-2$ & $0$ & $0$ & $0$ & $0$ & $0$ & $0$ & $0$ & $0$ & $1$ \\
    \hline
    \(E_4^1\) & $0$ & $0$ & $0$ & $0$ & $0$ & $0$ & $0$ & $0$ & $0$ & $0$ & $0$ & $-2$ & $1$ & $0$ & $0$ & $1$ & $0$ & $0$ & $0$ & $0$ \\
    \(E_4^2\) & $0$ & $0$ & $0$ & $0$ & $0$ & $0$ & $0$ & $0$ & $0$ & $0$ & $0$ & $1$ & $-2$ & $1$ & $0$ & $0$ & $0$ & $0$ & $0$ & $1$ \\
    \(E_4^3\) & $0$ & $0$ & $0$ & $0$ & $0$ & $0$ & $0$ & $0$ & $0$ & $0$ & $0$ & $0$ & $1$ & $-2$ & $0$ & $0$ & $0$ & $0$ & $0$ & $0$ \\
    \hline
    \(E_5^1\) & $0$ & $0$ & $0$ & $0$ & $0$ & $0$ & $0$ & $0$ & $0$ & $0$ & $0$ & $0$ & $0$ & $0$ & $-2$ & $1$ & $1$ & $1$ & $0$ & $0$ \\
    \hline
    \(\widetilde{C_{1}}\) & $0$ & $0$ & $0$ & $0$ & $1$ & $0$ & $0$ & $0$ & $0$ & $0$ & $0$ & $1$ & $0$ & $0$ & $1$ & $-2$ & $0$ & $0$ & $0$ & $0$ \\
    \(\widetilde{C_{2}}\) & $1$ & $0$ & $0$ & $0$ & $0$ & $0$ & $0$ & $0$ & $0$ & $0$ & $0$ & $0$ & $0$ & $0$ & $1$ & $0$ & $-2$ & $0$ & $0$ & $0$ \\
    \(\widetilde{C_{3}}\) & $0$ & $0$ & $0$ & $0$ & $0$ & $0$ & $0$ & $1$ & $0$ & $0$ & $0$ & $0$ & $0$ & $0$ & $1$ & $0$ & $0$ & $-2$ & $0$ & $0$ \\
    \(\widetilde{C_{4}}\) & $0$ & $0$ & $1$ & $0$ & $0$ & $1$ & $0$ & $0$ & $0$ & $1$ & $0$ & $0$ & $0$ & $0$ & $0$ & $0$ & $0$ & $0$ & $-2$ & $0$ \\
    \(\widetilde{C_{6}}\) & $0$ & $0$ & $0$ & $1$ & $0$ & $0$ & $0$ & $0$ & $0$ & $0$ & $1$ & $0$ & $1$ & $0$ & $0$ & $0$ & $0$ & $0$ & $0$ & $-2$ \\
    \hline
  \end{tabular}.
\end{table}
Note that the intersection matrix is degenerate. We choose the following integral basis of the lattice \(L_{\lambda}\):
\begin{align*}
  \begin{pmatrix}
    [E_4^3] \\ [\widetilde{C_{6}}]
  \end{pmatrix}^T =
  \begin{pmatrix}
    0 & 2 & 3 & 1 & -1 & 2 & 1 & 0 & 2 & 3 & 1 & -3 & -2 & -4 & -4 & -2 & -2 & 4 \\
    0 & -1 & -1 & -1 & 1 & 0 & 0 & 0 & -1 & -1 & -1 & 1 & 0 & 2 & 2 & 1 & 1 & -1
  \end{pmatrix} \cdot \\
  \begin{pmatrix}
    [E_1^1] & [E_1^2] & [E_1^3] & [E_1^4] & [E_2^1] & [E_2^2] & [E_2^3] & [E_3^1] & [E_3^2] & \\
    [E_3^3] & [E_3^4] & [E_4^1] & [E_4^2] & [E_5^1] & [\widetilde{C_{1}}] & [\widetilde{C_{2}}] & [\widetilde{C_{3}}] & [\widetilde{C_{4}}]
  \end{pmatrix}^T,
\end{align*}
hence we have \(\rk(L_{\lambda}) = 18\). Recall that \(\Pic(\widetilde{\mathcal{S}_{\lambda}})\) is generated by linear equivalence classes of the exceptional curves \(E_i^j\) and of strict transforms of curves on \(\mathcal{S}_{\lambda}\). Denote by \(M\) the corresponding intersection matrix.

The conjectural orthogonal complement to the lattice \(L_{\mathcal{S}}\) has the form
\[
  N = H \oplus \Pic(X), \quad
  \Pic(X) =
  \begin{pmatrix}
    8 & 8 \\
    8 & 6
  \end{pmatrix}.
\]
We choose the following generators of discriminant groups \(D_M\) and \(D_N\) of the lattices \(M\) and \(N\), respectively:
\begin{gather*}
  \begin{pmatrix}
    \frac{1}{2} & 0 & 0 & \frac{1}{2} & \frac{1}{2} & 0 & \frac{1}{2} & \frac{1}{2} & 0 & 0 & \frac{1}{2} & 0 & 0 & \frac{1}{2} & 0 & 0 & 0 & 0 \\
    0 & \frac{1}{4} & \frac{3}{8} & \frac{1}{8} & \frac{1}{2} & 0 & 0 & 0 & \frac{3}{4} & \frac{5}{8} & \frac{7}{8} & 0 & 0 & \frac{1}{2} & 0 & \frac{3}{4} & \frac{1}{4} & \frac{1}{2}
  \end{pmatrix}, \quad
  \begin{pmatrix}
    0 & 0 & \frac{1}{2} & \frac{1}{2} \\
    0 & 0 & \frac{7}{8} & 0
  \end{pmatrix}.
\end{gather*}
Then bilinear and quadratic discriminant forms on \(D_M\) and \(D_N\) can be represented as
\begin{gather*}
  B_M = 
  \begin{pmatrix}
    \frac{1}{2} & 0 \\
    0 & \frac{7}{8}
  \end{pmatrix}, \quad
  B_N = 
  \begin{pmatrix}
    \frac{1}{2} & 0 \\
    0 & \frac{1}{8}
  \end{pmatrix}; \quad
  Q_M = \left(\frac{1}{2},\,\frac{15}{8}\right), \quad
  Q_N = \left(\frac{3}{2},\,\frac{1}{8}\right).
\end{gather*}
The lattices \(M\) and \(N\) have the signature \((1, 17)\) and \((2, 2)\), respectively, hence \(M^{\perp} \simeq N\) in the K3 lattice.

%%% Local Variables:
%%% mode: latex
%%% TeX-master: "../main"
%%% End:
