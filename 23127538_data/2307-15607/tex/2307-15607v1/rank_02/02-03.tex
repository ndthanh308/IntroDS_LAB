\subsection{Family \textnumero2.3}\label{subsection:02-03}

The toric Landau--Ginzburg model for this family is given by the Laurent polynomial
\begin{gather*}
  (x + y + 1)^4 (z + 1) x^{-1} y^{-1} z^{-1} + z + 1
\end{gather*}
(see~\cite[Family \textnumero2.3]{cheltsov2018katzarkov}). It is not a Minkowski polynomial. Let us apply the following birational transform:
\[
  (\mathbb{C}^*)^3 \dashrightarrow (\mathbb{C}^*)^3, \quad
  (x, y, z) \mapsto (-x z, x z + x - 1, -y z^{-1} - 1).
\]

The pencil \(\mathcal{S}\) is defined by the equation
\begin{gather*}
  X^3 Y + (Y + Z) (X (Z + T) - T^2) (Y + \lambda Z) = 0.
\end{gather*}
Members \(\mathcal{S}_{\lambda}\) of the pencil are irreducible for any parameter \(\lambda \in \mathbb{P}^1\) except
\begin{gather*}
  \mathcal{S}_{\infty} = S_{(Z)} + S_{(Y + Z)} + S_{(X (Z + T) - T^2)}, \quad
  \mathcal{S}_{\lambda} = S_{(Y)} + S_{(X^3 - (Y + Z) (X (Z + T) - T^2))}.
\end{gather*}
The base locus of the pencil \(\mathcal{S}\) consists of the following curves:
\begin{gather*}
  C_{1} = C_{(X, T)}, \;
  C_{2} = C_{(Y, Z)}, \;
  C_{3} = C_{(X, Y + Z)}, \;
  C_{4} = C_{(Y, X (Z + T) - T^2)}, \;
  C_{5} = C_{(Z, X^3 + Y T (X - T))}.
\end{gather*}

Their linear equivalence classes on the generic member \(\mathcal{S}_{\Bbbk}\) of the pencil satisfy the following relations:
\begin{gather*}
  \begin{pmatrix}
    [\mathcal{S}_{\Bbbk} \cdot S_{(Z)}] - [H_{\mathcal{S}}] \\
    [\mathcal{S}_{\Bbbk} \cdot S_{(Y + Z)}] - [H_{\mathcal{S}}] \\
    [\mathcal{S}_{\Bbbk} \cdot S_{(Y)}] - [H_{\mathcal{S}}]
  \end{pmatrix} = 
  \begin{pmatrix}
    0 & 1 & 0 & 0 & 1 & -1 \\
    0 & 1 & 3 & 0 & 0 & -1 \\
    0 & 2 & 0 & 1 & 0 & -1
  \end{pmatrix} \cdot
  \begin{pmatrix}
    [C_1] \\ \cdots \\ [C_{5}] \\ [H_{\mathcal{S}}]
  \end{pmatrix} = 0.
\end{gather*}
We can reduce the number of linear equivalence classes using these relations:
\begin{gather*}
  \begin{pmatrix}
    [C_{2}] \\ [C_{4}] \\ [C_{5}]
  \end{pmatrix} = 
  \begin{pmatrix}
    0 & -3 & 1 \\
    0 & 6 & -1 \\
    0 & 3 & 0
  \end{pmatrix} \cdot
  \begin{pmatrix}
    [C_{1}] \\ [C_{3}] \\ [H_{\mathcal{S}}]
  \end{pmatrix}.
\end{gather*}

For a general choice of \(\lambda \in \mathbb{C}\) the surface \(\mathcal{S}_{\lambda}\) has the following singularities:
\begin{itemize}\setlength{\itemindent}{2cm}
\item[\(P_{1} = P_{(X, Y, Z)}\):] type \(\mathbb{A}_5\) with the quadratic term \((Y + Z) \cdot (Y + \lambda Z)\);
\item[\(P_{2} = P_{(X, Z, T)}\):] type \(\mathbb{A}_1\) with the quadratic term \(X (Z + T) - T^2\);
\item[\(P_{3} = P_{(X, T, Y + Z)}\):] type \(\mathbb{A}_5\) with the quadratic term \((\lambda - 1) X \cdot (Y + Z)\);
\item[\(P_{4} = P_{(X, T, Y + \lambda Z)}\):] type \(\mathbb{A}_5\) with the quadratic term \((\lambda - 1) X \cdot (Y + \lambda Z)\).
\end{itemize}

The \(\mathbb{Q}\)-valued intersection matrix on the group \(A_{\mathcal{S}}\) has the following form:
\begin{table}[H]
  \renewcommand\arraystretch{1.42}
  \begin{tabular}{|c||c|c|c|}
    \hline
    \(\bullet\) & \(C_{1}\) & \(C_{3}\) & \(H_{\mathcal{S}}\) \\
    \hline
    \hline
    \(C_{1}\) & $\frac{1}{6}$ & $\frac{1}{3}$ & $1$ \\
    \hline
    \(C_{3}\) & $\frac{1}{3}$ & $\frac{1}{6}$ & $1$ \\
    \hline
    \(H_{\mathcal{S}}\) & $1$ & $1$ & $4$ \\
    \hline
  \end{tabular}.
\end{table}
Note that the intersection matrix has the rank 2.

Let \(\widetilde{\mathcal{S}_{\lambda}}\) and \(\widetilde{\mathcal{S}_{\Bbbk}}\) be the minimal resolution of a general member \(\mathcal{S}_{\lambda}\) and the generic member \(\mathcal{S}_{\Bbbk}\) of the pencil, respectively. Denote by \(L_{\lambda} \subset \Pic(\widetilde{\mathcal{S}_{\lambda}})\) and \(L_{\mathcal{S}} \subset \Pic(\widetilde{\mathcal{S}_{\Bbbk}})\) the subgroups generated by linear equivalence classes of exceptional divisors of the resolution and by linear equivalence classes of strict transforms \(\widetilde{C_i}\), \(\widetilde{H_{\mathcal{S}}}\) of the above-introduced curves. By construction the subgroup \(L_{\lambda}\) is equipped with the \(\Gal(\Bbbk)\)-action, and the subgroup \(L_{\mathcal{S}}\) can be identified with the subgroup \(L_{\lambda}^{\Gal(\Bbbk)} \subset L_{\lambda}\). In our case we have \(L_{\lambda} = L_{\mathcal{S}}\).

The intersection matrix on the lattice \(L_{\lambda}\) has the following form:
\begin{table}[H]
  \begin{tabular}{|c||ccccc|c|ccccc|ccccc|ccc|}
    \hline
    \(\bullet\) & \(E_1^1\) & \(E_1^2\) & \(E_1^3\) & \(E_1^4\) & \(E_1^5\) & \(E_2^1\) & \(E_3^1\) & \(E_3^2\) & \(E_3^3\) & \(E_3^4\) & \(E_3^5\) & \(E_4^1\) & \(E_4^2\) & \(E_4^3\) & \(E_4^4\) & \(E_4^5\) & \(\widetilde{C_{1}}\) & \(\widetilde{C_{3}}\) & \(\widetilde{H_{\mathcal{S}}}\) \\
    \hline
    \hline
    \(E_1^1\) & $-2$ & $1$ & $0$ & $0$ & $0$ & $0$ & $0$ & $0$ & $0$ & $0$ & $0$ & $0$ & $0$ & $0$ & $0$ & $0$ & $0$ & $1$ & $0$ \\
    \(E_1^2\) & $1$ & $-2$ & $1$ & $0$ & $0$ & $0$ & $0$ & $0$ & $0$ & $0$ & $0$ & $0$ & $0$ & $0$ & $0$ & $0$ & $0$ & $0$ & $0$ \\
    \(E_1^3\) & $0$ & $1$ & $-2$ & $1$ & $0$ & $0$ & $0$ & $0$ & $0$ & $0$ & $0$ & $0$ & $0$ & $0$ & $0$ & $0$ & $0$ & $0$ & $0$ \\
    \(E_1^4\) & $0$ & $0$ & $1$ & $-2$ & $1$ & $0$ & $0$ & $0$ & $0$ & $0$ & $0$ & $0$ & $0$ & $0$ & $0$ & $0$ & $0$ & $0$ & $0$ \\
    \(E_1^5\) & $0$ & $0$ & $0$ & $1$ & $-2$ & $0$ & $0$ & $0$ & $0$ & $0$ & $0$ & $0$ & $0$ & $0$ & $0$ & $0$ & $0$ & $0$ & $0$ \\
    \hline
    \(E_2^1\) & $0$ & $0$ & $0$ & $0$ & $0$ & $-2$ & $0$ & $0$ & $0$ & $0$ & $0$ & $0$ & $0$ & $0$ & $0$ & $0$ & $1$ & $0$ & $0$ \\
    \hline
    \(E_3^1\) & $0$ & $0$ & $0$ & $0$ & $0$ & $0$ & $-2$ & $1$ & $0$ & $0$ & $0$ & $0$ & $0$ & $0$ & $0$ & $0$ & $1$ & $0$ & $0$ \\
    \(E_3^2\) & $0$ & $0$ & $0$ & $0$ & $0$ & $0$ & $1$ & $-2$ & $1$ & $0$ & $0$ & $0$ & $0$ & $0$ & $0$ & $0$ & $0$ & $0$ & $0$ \\
    \(E_3^3\) & $0$ & $0$ & $0$ & $0$ & $0$ & $0$ & $0$ & $1$ & $-2$ & $1$ & $0$ & $0$ & $0$ & $0$ & $0$ & $0$ & $0$ & $0$ & $0$ \\
    \(E_3^4\) & $0$ & $0$ & $0$ & $0$ & $0$ & $0$ & $0$ & $0$ & $1$ & $-2$ & $1$ & $0$ & $0$ & $0$ & $0$ & $0$ & $0$ & $1$ & $0$ \\
    \(E_3^5\) & $0$ & $0$ & $0$ & $0$ & $0$ & $0$ & $0$ & $0$ & $0$ & $1$ & $-2$ & $0$ & $0$ & $0$ & $0$ & $0$ & $0$ & $0$ & $0$ \\
    \hline
    \(E_4^1\) & $0$ & $0$ & $0$ & $0$ & $0$ & $0$ & $0$ & $0$ & $0$ & $0$ & $0$ & $-2$ & $1$ & $0$ & $0$ & $0$ & $1$ & $0$ & $0$ \\
    \(E_4^2\) & $0$ & $0$ & $0$ & $0$ & $0$ & $0$ & $0$ & $0$ & $0$ & $0$ & $0$ & $1$ & $-2$ & $1$ & $0$ & $0$ & $0$ & $0$ & $0$ \\
    \(E_4^3\) & $0$ & $0$ & $0$ & $0$ & $0$ & $0$ & $0$ & $0$ & $0$ & $0$ & $0$ & $0$ & $1$ & $-2$ & $1$ & $0$ & $0$ & $0$ & $0$ \\
    \(E_4^4\) & $0$ & $0$ & $0$ & $0$ & $0$ & $0$ & $0$ & $0$ & $0$ & $0$ & $0$ & $0$ & $0$ & $1$ & $-2$ & $1$ & $0$ & $0$ & $0$ \\
    \(E_4^5\) & $0$ & $0$ & $0$ & $0$ & $0$ & $0$ & $0$ & $0$ & $0$ & $0$ & $0$ & $0$ & $0$ & $0$ & $1$ & $-2$ & $0$ & $0$ & $0$ \\
    \hline
    \(\widetilde{C_{1}}\) & $0$ & $0$ & $0$ & $0$ & $0$ & $1$ & $1$ & $0$ & $0$ & $0$ & $0$ & $1$ & $0$ & $0$ & $0$ & $0$ & $-2$ & $0$ & $1$ \\
    \(\widetilde{C_{3}}\) & $1$ & $0$ & $0$ & $0$ & $0$ & $0$ & $0$ & $0$ & $0$ & $1$ & $0$ & $0$ & $0$ & $0$ & $0$ & $0$ & $0$ & $-2$ & $1$ \\
    \(\widetilde{H_{\mathcal{S}}}\) & $0$ & $0$ & $0$ & $0$ & $0$ & $0$ & $0$ & $0$ & $0$ & $0$ & $0$ & $0$ & $0$ & $0$ & $0$ & $0$ & $1$ & $1$ & $4$ \\
    \hline
  \end{tabular}.
\end{table}
Note that the intersection matrix is degenerate. We choose the following integral basis of the lattice \(L_{\lambda}\):
\begin{align*}
  \begin{pmatrix}
    [E_4^5]
  \end{pmatrix} =
  \begin{pmatrix}
    -5 & -4 & -3 & -2 & -1 & -3 & -7 & -8 & -9 & -10 & -5 & -5 & -4 & -3 & -2 & -6 & -6 & 3
  \end{pmatrix} \cdot \\
  \begin{pmatrix}
    [E_1^1] & [E_1^2] & [E_1^3] & [E_1^4] & [E_1^5] & [E_2^1] & [E_3^1] & [E_3^2] & [E_3^3] & \\
    [E_3^4] & [E_3^5] & [E_4^1] & [E_4^2] & [E_4^3] & [E_4^4] & [\widetilde{C_{1}}] & [\widetilde{C_{3}}] & [\widetilde{H_{\mathcal{S}}}]
  \end{pmatrix}^T,
\end{align*}
hence we have \(\rk(L_{\lambda}) = 18\). Recall that \(\Pic(\widetilde{\mathcal{S}_{\lambda}})\) is generated by linear equivalence classes of the exceptional curves \(E_i^j\) and of strict transforms of curves on \(\mathcal{S}_{\lambda}\). Denote by \(M\) the corresponding intersection matrix.

The conjectural orthogonal complement to the lattice \(L_{\mathcal{S}}\) has the form
\[
  N = H \oplus \Pic(X), \quad
  \Pic(X) =
  \begin{pmatrix}
    0 & 2 \\
    2 & 4
  \end{pmatrix}.
\]
We choose the following generators of discriminant groups \(D_M\) and \(D_N\) of the lattices \(M\) and \(N\), respectively:
\begin{gather*}
  \begin{pmatrix}
    0 & 0 & 0 & 0 & 0 & \frac{1}{2} & \frac{1}{2} & 0 & \frac{1}{2} & 0 & \frac{1}{2} & 0 & 0 & 0 & 0 & 0 & 0 & 0 \\
    \frac{1}{2} & 0 & \frac{1}{2} & 0 & \frac{1}{2} & \frac{1}{2} & 0 & 0 & 0 & 0 & 0 & 0 & 0 & 0 & 0 & 0 & 0 & \frac{1}{2}
  \end{pmatrix}, \quad
  \begin{pmatrix}
    0 & 0 & \frac{1}{2} & 0 \\
    0 & 0 & 0 & \frac{1}{2}
  \end{pmatrix}.
\end{gather*}
Then bilinear and quadratic discriminant forms on \(D_M\) and \(D_N\) can be represented as
\begin{gather*}
  B_M = 
  \begin{pmatrix}
    0 & \frac{1}{2} \\
    \frac{1}{2} & 0
  \end{pmatrix}, \quad
  B_N = 
  \begin{pmatrix}
    0 & \frac{1}{2} \\
    \frac{1}{2} & 0
  \end{pmatrix}; \quad
  Q_M = \left(0,\,1\right), \quad
  Q_N = \left(0,\,1\right).
\end{gather*}
The lattices \(M\) and \(N\) have the signature \((1, 17)\) and \((2, 2)\), respectively, hence \(M^{\perp} \simeq N\) in the K3 lattice.

%%% Local Variables:
%%% mode: latex
%%% TeX-master: "../main"
%%% End:
