\subsection{Family \textnumero2.1}\label{subsection:02-01}

The toric Landau--Ginzburg model for this family is given by the Laurent polynomial
\[
  (x + y + 1)^6 (z + 1) x^{-1} y^{-2} + z^{-1}
\]
(see~\cite[Family \textnumero2.1]{cheltsov2018katzarkov}). It is not a Minkowski polynomial. Let us apply the following birational transform:
\[
  (\mathbb{C}^*)^3 \dashrightarrow \mathbb{C}^* \times \mathbb{C}^* \times \mathbb{C}^*, \quad
  (x, y, z) \mapsto (B^{-1} - B^{-2} C^{-1} - 1, B^{-2} C^{-1}, -Y^{-1} - 1).
\]

The pencil \(\mathcal{S}\) is defined by the equation
\[
  X (X + Y) C^3 = Y (Y + \lambda (X + Y)) (A^3 - B C (A - B)). 
\]
Members \(\mathcal{S}_{\lambda}\) of the pencil are irreducible for any parameter \(\lambda \in \mathbb{P}^1\) except
\[
  \mathcal{S}_{\infty} = S_{(Y)} + S_{(X + Y)} + S_{(A^3 - B C(A - B))}, \quad
  \mathcal{S}_{-1} = S_{(X)} + S_{(Y (A^3 - B C (A - B)) + C^3 (X + Y))}.
\]
The base locus of the pencil \(\mathcal{S}\) consists of the following curves:
\[
  C_1 = C_{(Y, C)}, \;
  C_2 = C_{(A, C)}, \;
  C_3 = C_{(X, A^3 - B C (A - B))}, \;
  C_4 = C_{(X + Y, A^3 - B C (A - B))}.
\]
Their linear equivalence classes on the generic member \(\mathcal{S}_{\Bbbk}\) of the pencil satisfy the following relations:
\[
  \begin{pmatrix}
    [\mathcal{S}_{\Bbbk} \cdot S_{(Y)}] - [H_{\mathcal{S}}^{(1)}] \\
    [\mathcal{S}_{\Bbbk} \cdot S_{(X + Y)}] - [H_{\mathcal{S}}^{(1)}] \\
    [\mathcal{S}_{\Bbbk} \cdot S_{(X)}] - [H_{\mathcal{S}}^{(1)}]
  \end{pmatrix} =
  \begin{pmatrix}
    3 & 0 & 0 & 0 & -1 \\
    0 & 0 & 0 & 1 & -1 \\
    0 & 0 & 1 & 0 & -1
  \end{pmatrix} \cdot
  \begin{pmatrix}
    [C_1] \\ \cdots \\ [C_4] \\ [H_{\mathcal{S}}^{(1)}]
  \end{pmatrix}
  = 0.
\]
We can reduce the number of linear equivalence classes using these relations:
\[
  \begin{pmatrix}
    [C_{3}] \\ [C_{4}] \\ [H_{\mathcal{S}}^{(1)}]
  \end{pmatrix} = 
  \begin{pmatrix}
    3 [C_{1}]\\ 3 [C_{1}] \\ 3 [C_{1}]
  \end{pmatrix}.
\]

For a general choice of \(\lambda \in \mathbb{C}\) the surface \(\mathcal{S}_{\lambda}\) has the following singularities:
\begin{itemize}\setlength{\itemindent}{2cm}
\item[\(P_1 = P_{(Y, A, C)}\):] type \(\mathbb{A}_8\) with the quadratic term \(\lambda Y \cdot C\);
\item[\(P_2 = P_{(A, C, Y + \lambda (X + Y))}\):] type \(\mathbb{A}_8\) with the quadratic term \(\lambda (\lambda + 1) Y \cdot C\).
\end{itemize}

Let \(\widetilde{\mathcal{S}_{\lambda}}\) and \(\widetilde{\mathcal{S}_{\Bbbk}}\) be the minimal resolution of a general member \(\mathcal{S}_{\lambda}\) and the generic member \(\mathcal{S}_{\Bbbk}\) of the pencil, respectively. Denote by \(L_{\lambda} \subset \Pic(\widetilde{\mathcal{S}_{\lambda}})\) and \(L_{\mathcal{S}} \subset \Pic(\widetilde{\mathcal{S}_{\Bbbk}})\) the subgroups generated by linear equivalence classes of exceptional divisors of the resolution and by linear equivalence classes of strict transforms \(\widetilde{C_i}\), \(\widetilde{H_{\mathcal{S}}^{(j)}}\) of the above-introduced curves. By construction the subgroup \(L_{\lambda}\) is equipped with the \(\Gal(\Bbbk)\)-action, and the subgroup \(L_{\mathcal{S}}\) can be identified with the subgroup \(L_{\lambda}^{\Gal(\Bbbk)} \subset L_{\lambda}\). In our case we have \(L_{\lambda} = L_{\mathcal{S}}\).

The intersection matrix on the lattice \(L_{\lambda}\) has the following form:
\begin{table}[H]
  \begin{tabular}{|c||cccccccc|cccccccc|ccc|}
    \hline
    \(\bullet\) & \(E_1^1\) & \(E_1^2\) & \(E_1^3\) & \(E_1^4\) & \(E_1^5\) & \(E_1^6\) & \(E_1^7\) & \(E_1^8\) & \(E_2^1\) & \(E_2^2\) & \(E_2^3\) & \(E_2^4\) & \(E_2^5\) & \(E_2^6\) & \(E_2^7\) & \(E_2^8\) & \(\widetilde{C_1}\) & \(\widetilde{C_2}\) & \(\widetilde{H_{\mathcal{S}}^{(2)}}\) \\
    \hline
    \hline
    \(E_1^1\) & $-2$ & $1$ & $0$ & $0$ & $0$ & $0$ & $0$ & $0$ & $0$ & $0$ & $0$ & $0$ & $0$ & $0$ & $0$ & $0$ & $0$ & $0$ & $0$ \\
    \(E_1^2\) & $1$ & $-2$ & $1$ & $0$ & $0$ & $0$ & $0$ & $0$ & $0$ & $0$ & $0$ & $0$ & $0$ & $0$ & $0$ & $0$ & $0$ & $0$ & $0$ \\
    \(E_1^3\) & $0$ & $1$ & $-2$ & $1$ & $0$ & $0$ & $0$ & $0$ & $0$ & $0$ & $0$ & $0$ & $0$ & $0$ & $0$ & $0$ & $1$ & $0$ & $0$ \\
    \(E_1^4\) & $0$ & $0$ & $1$ & $-2$ & $1$ & $0$ & $0$ & $0$ & $0$ & $0$ & $0$ & $0$ & $0$ & $0$ & $0$ & $0$ & $0$ & $0$ & $0$ \\
    \(E_1^5\) & $0$ & $0$ & $0$ & $1$ & $-2$ & $1$ & $0$ & $0$ & $0$ & $0$ & $0$ & $0$ & $0$ & $0$ & $0$ & $0$ & $0$ & $0$ & $0$ \\
    \(E_1^6\) & $0$ & $0$ & $0$ & $0$ & $1$ & $-2$ & $1$ & $0$ & $0$ & $0$ & $0$ & $0$ & $0$ & $0$ & $0$ & $0$ & $0$ & $0$ & $0$ \\
    \(E_1^7\) & $0$ & $0$ & $0$ & $0$ & $0$ & $1$ & $-2$ & $1$ & $0$ & $0$ & $0$ & $0$ & $0$ & $0$ & $0$ & $0$ & $0$ & $0$ & $0$ \\
    \(E_1^8\) & $0$ & $0$ & $0$ & $0$ & $0$ & $0$ & $1$ & $-2$ & $0$ & $0$ & $0$ & $0$ & $0$ & $0$ & $0$ & $0$ & $0$ & $1$ & $0$ \\
    \hline
    \(E_2^1\) & $0$ & $0$ & $0$ & $0$ & $0$ & $0$ & $0$ & $0$ & $-2$ & $1$ & $0$ & $0$ & $0$ & $0$ & $0$ & $0$ & $0$ & $1$ & $0$ \\
    \(E_2^2\) & $0$ & $0$ & $0$ & $0$ & $0$ & $0$ & $0$ & $0$ & $1$ & $-2$ & $1$ & $0$ & $0$ & $0$ & $0$ & $0$ & $0$ & $0$ & $0$ \\
    \(E_2^3\) & $0$ & $0$ & $0$ & $0$ & $0$ & $0$ & $0$ & $0$ & $0$ & $1$ & $-2$ & $1$ & $0$ & $0$ & $0$ & $0$ & $0$ & $0$ & $0$ \\
    \(E_2^4\) & $0$ & $0$ & $0$ & $0$ & $0$ & $0$ & $0$ & $0$ & $0$ & $0$ & $1$ & $-2$ & $1$ & $0$ & $0$ & $0$ & $0$ & $0$ & $0$ \\
    \(E_2^5\) & $0$ & $0$ & $0$ & $0$ & $0$ & $0$ & $0$ & $0$ & $0$ & $0$ & $0$ & $1$ & $-2$ & $1$ & $0$ & $0$ & $0$ & $0$ & $0$ \\
    \(E_2^6\) & $0$ & $0$ & $0$ & $0$ & $0$ & $0$ & $0$ & $0$ & $0$ & $0$ & $0$ & $0$ & $1$ & $-2$ & $1$ & $0$ & $0$ & $0$ & $0$ \\
    \(E_2^7\) & $0$ & $0$ & $0$ & $0$ & $0$ & $0$ & $0$ & $0$ & $0$ & $0$ & $0$ & $0$ & $0$ & $1$ & $-2$ & $1$ & $0$ & $0$ & $0$ \\
    \(E_2^8\) & $0$ & $0$ & $0$ & $0$ & $0$ & $0$ & $0$ & $0$ & $0$ & $0$ & $0$ & $0$ & $0$ & $0$ & $1$ & $-2$ & $0$ & $0$ & $0$ \\
    \hline
    \(\widetilde{C_1}\) & $0$ & $0$ & $1$ & $0$ & $0$ & $0$ & $0$ & $0$ & $0$ & $0$ & $0$ & $0$ & $0$ & $0$ & $0$ & $0$ & $-2$ & $0$ & $1$ \\
    \(\widetilde{C_2}\) & $0$ & $0$ & $0$ & $0$ & $0$ & $0$ & $0$ & $1$ & $1$ & $0$ & $0$ & $0$ & $0$ & $0$ & $0$ & $0$ & $0$ & $-2$ & $0$ \\
    \(\widetilde{H_{\mathcal{S}}^{(2)}}\) & $0$ & $0$ & $0$ & $0$ & $0$ & $0$ & $0$ & $0$ & $0$ & $0$ & $0$ & $0$ & $0$ & $0$ & $0$ & $0$ & $1$ & $0$ & $2$ \\
    \hline
  \end{tabular}.
\end{table}
Note that the intersection matrix is degenerate. We choose the following integral basis of the lattice \(L_{\lambda}\):
\begin{align*}
  \begin{pmatrix}
    [E_2^8]
  \end{pmatrix} =
  \begin{pmatrix}
    -5 & -10 & -15 & -14 & -13 & -12 & -11 & -10 & -8 & -7 & -6 & -5 & -4 & -3 & -2 & -6 & -9 & 3
  \end{pmatrix} \cdot \\
  \begin{pmatrix}
    [E_1^1] & [E_1^2] & [E_1^3] & [E_1^4] & [E_1^5] & [E_1^6] & [E_1^7] & [E_1^8] & [E_2^1] & \\
    [E_2^2] & [E_2^3] & [E_2^4] & [E_2^5] & [E_2^6] & [E_2^7] & [\widetilde{C_1}] & [\widetilde{C_2}] & [\widetilde{H_{\mathcal{S}}^{(2)}}]
  \end{pmatrix}^T,
\end{align*}
hence we have \(\rk(L_{\lambda}) = 18\). Recall that \(\Pic(\widetilde{\mathcal{S}_{\lambda}})\) is generated by linear equivalence classes of the exceptional curves \(E_i^j\) and of strict transforms of curves on \(\mathcal{S}_{\lambda}\). Denote by \(M\) the corresponding intersection matrix.

The conjectural orthogonal complement to the lattice \(L_{\mathcal{S}}\) has the form
\[
  N = H \oplus \Pic(X), \quad
  \Pic(X) =
  \begin{pmatrix}
    0 & 1 \\
    1 & 2
  \end{pmatrix}.
\]
Discriminant groups of the lattices \(M\) and \(N\) are both trivial. The lattices \(M\) and \(N\) have the signature \((1, 17)\) and \((2, 2)\), respectively, hence \(M^{\perp} \simeq N\) in the K3 lattice.

%%% Local Variables:
%%% mode: latex
%%% TeX-master: "../main"
%%% End:
