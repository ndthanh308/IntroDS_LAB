\subsection{Family \textnumero2.4}\label{subsection:02-04}

The toric Landau--Ginzburg model for this family is given by the Laurent polynomial
\begin{gather*}
  x^{2} z^{-1} + x y z^{-1} + x^{2} y^{-1} + 4 x + 2 y + 2 x y^{-1} z + 4 z + x^{-1} y z + y^{-1} z^{2} + x^{-1} z^{2} + 3 x z^{-1} + \\ 2 y z^{-1} + 2 x y^{-1} + 2 x^{-1} y + 2 y^{-1} z + 3 x^{-1} z + 3 z^{-1} + x^{-1} y z^{-1} + y^{-1} + 3 x^{-1} + x^{-1} z^{-1}
\end{gather*}
(see~\cite[Family \textnumero2.4]{cheltsov2018katzarkov}). It is a Minkowski polynomial \textnumero3964 (see~\cite[Appendix~B: bucket~161]{akhtar2012minkowski}).

The pencil \(\mathcal{S}\) is defined by the equation
\begin{gather*}
  X^{3} Y + X^{2} Y^{2} + X^{3} Z + 4 X^{2} Y Z + 2 X Y^{2} Z + 2 X^{2} Z^{2} + 4 X Y Z^{2} + Y^{2} Z^{2} + X Z^{3} + Y Z^{3} + 3 X^{2} Y T + \\ 2 X Y^{2} T + 2 X^{2} Z T + 2 Y^{2} Z T + 2 X Z^{2} T + 3 Y Z^{2} T + 3 X Y T^{2} + Y^{2} T^{2} + X Z T^{2} + 3 Y Z T^{2} + Y T^{3} = \lambda X Y Z T.
\end{gather*}

\begin{remark}
  Note that the equation is invariant under the permutation \((X, Z) \mapsto (Z, X)\).
\end{remark}

Members \(\mathcal{S}_{\lambda}\) of the pencil are irreducible for any parameter \(\lambda \in \mathbb{P}^1\) except
\begin{gather*}
  \mathcal{S}_{\infty} = S_{(X)} + S_{(Y)} + S_{(Z)} + S_{(T)}, \quad
  \mathcal{S}_{- 7} = S_{(X + Z + T)} + S_{(X + Y + Z + T)} + S_{(Y (X + Z + T) + X Z)}.
\end{gather*}
The base locus of the pencil \(\mathcal{S}\) consists of the following curves:
\begin{gather*}
  C_{1} = C_{(X, Y)}, \;
  C_{2} = C_{(Y, Z)}, \;
  C_{3} = C_{(X, Z + T)}, \;
  C_{4} = C_{(Z, X + T)}, \;
  C_{5} = C_{(T, X + Z)}, \\
  C_{6} = C_{(X, Y + Z + T)}, \;
  C_{7} = C_{(Y, X + Z + T)}, \;
  C_{8} = C_{(Z, X + Y + T)}, \;
  C_{9} = C_{(T, X + Y + Z)}, \;
  C_{10} = C_{(T, X Z + Y (X + Z))}.
\end{gather*}

Their linear equivalence classes on the generic member \(\mathcal{S}_{\Bbbk}\) of the pencil satisfy the following relations:
\begin{gather*}
  \begin{pmatrix}
    [\mathcal{S}_{\Bbbk} \cdot S_{(X)}] - [H_{\mathcal{S}}] \\
    [\mathcal{S}_{\Bbbk} \cdot S_{(Y)}] - [H_{\mathcal{S}}] \\
    [\mathcal{S}_{\Bbbk} \cdot S_{(Z)}] - [H_{\mathcal{S}}] \\
    [\mathcal{S}_{\Bbbk} \cdot S_{(T)}] - [H_{\mathcal{S}}] \\
    [\mathcal{S}_{\Bbbk} \cdot S_{(X + Z + T)}] - [H_{\mathcal{S}}] \\
    [\mathcal{S}_{\Bbbk} \cdot S_{(X + Y + Z + T)}] - [H_{\mathcal{S}}]
  \end{pmatrix} = 
  \begin{pmatrix}
    1 & 0 & 2 & 0 & 0 & 1 & 0 & 0 & 0 & 0 & -1 \\
    1 & 1 & 0 & 0 & 0 & 0 & 2 & 0 & 0 & 0 & -1 \\
    0 & 1 & 0 & 2 & 0 & 0 & 0 & 1 & 0 & 0 & -1 \\
    0 & 0 & 0 & 0 & 1 & 0 & 0 & 0 & 1 & 1 & -1 \\
    0 & 0 & 1 & 1 & 1 & 0 & 1 & 0 & 0 & 0 & -1 \\
    0 & 0 & 0 & 0 & 0 & 1 & 1 & 1 & 1 & 0 & -1
  \end{pmatrix} \cdot
  \begin{pmatrix}
    [C_1] \\ \cdots \\ [C_{10}] \\ [H_{\mathcal{S}}]
  \end{pmatrix} = 0.
\end{gather*}
We can reduce the number of linear equivalence classes using these relations:
\begin{gather*}
  \begin{pmatrix}
    [C_{2}] \\ [C_{6}] \\ [C_{7}] \\ [C_{8}] \\ [C_{9}] \\ [C_{10}]
  \end{pmatrix} = 
  \begin{pmatrix}
    -1 & 2 & 2 & 2 & -1 \\
    -1 & -2 & 0 & 0 & 1 \\
    0 & -1 & -1 & -1 & 1 \\
    1 & -2 & -4 & -2 & 2 \\
    0 & 5 & 5 & 3 & -3 \\
    0 & -5 & -5 & -4 & 4
  \end{pmatrix} \cdot
  \begin{pmatrix}
    [C_{1}] \\ [C_{3}] \\ [C_{4}] \\ [C_{5}] \\ [H_{\mathcal{S}}]
  \end{pmatrix}.
\end{gather*}

For a general choice of \(\lambda \in \mathbb{C}\) the surface \(\mathcal{S}_{\lambda}\) has the following singularities:
\begin{itemize}\setlength{\itemindent}{2cm}
\item[\(P_{1} = P_{(X, Z, T)}\):] type \(\mathbb{D}_4\) with the quadratic term \((X + Z + T)^2\);
\item[\(P_{2} = P_{(X, Y, Z + T)}\):] type \(\mathbb{A}_4\) with the quadratic term \((\lambda + 7) X \cdot Y\);
\item[\(P_{3} = P_{(Y, Z, X + T)}\):] type \(\mathbb{A}_4\) with the quadratic term \((\lambda + 7) Y \cdot Z\);
\item[\(P_{4} = P_{(Y, T, X + Z)}\):] type \(\mathbb{A}_1\) with the quadratic term \((X + Z + T) (X + Y + Z + T) - (\lambda + 7) Y T\).
\end{itemize}

\begin{table}[H]
  \renewcommand\arraystretch{1.42}
  \begin{tabular}{|c||c|c|c|c|c|}
    \hline
    \(\bullet\) & \(C_{1}\) & \(C_{3}\) & \(C_{4}\) & \(C_{5}\) & \(H_{\mathcal{S}}\) \\
    \hline
    \hline
    \(C_{1}\) & $-\frac{4}{5}$ & $\frac{3}{5}$ & $0$ & $0$ & $1$ \\
    \hline
    \(C_{3}\) & $\frac{3}{5}$ & $-\frac{1}{5}$ & $\frac{1}{2}$ & $\frac{1}{2}$ & $1$ \\
    \hline
    \(C_{4}\) & $0$ & $\frac{1}{2}$ & $-\frac{1}{5}$ & $\frac{1}{2}$ & $1$ \\
    \hline
    \(C_{5}\) & $0$ & $\frac{1}{2}$ & $\frac{1}{2}$ & $-\frac{1}{2}$ & $1$ \\
    \hline
    \(H_{\mathcal{S}}\) & $1$ & $1$ & $1$ & $1$ & $4$ \\
    \hline
  \end{tabular}.
\end{table}
Note that the intersection matrix has the rank 5.

Let \(\widetilde{\mathcal{S}_{\lambda}}\) and \(\widetilde{\mathcal{S}_{\Bbbk}}\) be the minimal resolution of a general member \(\mathcal{S}_{\lambda}\) and the generic member \(\mathcal{S}_{\Bbbk}\) of the pencil, respectively. Denote by \(L_{\lambda} \subset \Pic(\widetilde{\mathcal{S}_{\lambda}})\) and \(L_{\mathcal{S}} \subset \Pic(\widetilde{\mathcal{S}_{\Bbbk}})\) the subgroups generated by linear equivalence classes of exceptional divisors of the resolution and by linear equivalence classes of strict transforms \(\widetilde{C_i}\), \(\widetilde{H_{\mathcal{S}}}\) of the above-introduced curves. By construction the subgroup \(L_{\lambda}\) is equipped with the \(\Gal(\Bbbk)\)-action, and the subgroup \(L_{\mathcal{S}}\) can be identified with the subgroup \(L_{\lambda}^{\Gal(\Bbbk)} \subset L_{\lambda}\). In our case we have \(L_{\lambda} = L_{\mathcal{S}}\).

The intersection matrix on the lattice \(L_{\lambda}\) has the following form:
\begin{table}[H]
  \begin{tabular}{|c||cccc|cccc|cccc|c|ccccc|}
    \hline
    \(\bullet\) & \(E_1^1\) & \(E_1^2\) & \(E_1^3\) & \(E_1^4\) & \(E_2^1\) & \(E_2^2\) & \(E_2^3\) & \(E_2^4\) & \(E_3^1\) & \(E_3^2\) & \(E_3^3\) & \(E_3^4\) & \(E_4^1\) & \(\widetilde{C_{1}}\) & \(\widetilde{C_{3}}\) & \(\widetilde{C_{4}}\) & \(\widetilde{C_{5}}\) & \(\widetilde{H_{\mathcal{S}}}\) \\
    \hline
    \hline
    \(E_1^1\) & $-2$ & $1$ & $0$ & $0$ & $0$ & $0$ & $0$ & $0$ & $0$ & $0$ & $0$ & $0$ & $0$ & $0$ & $1$ & $0$ & $0$ & $0$ \\
    \(E_1^2\) & $1$ & $-2$ & $1$ & $1$ & $0$ & $0$ & $0$ & $0$ & $0$ & $0$ & $0$ & $0$ & $0$ & $0$ & $0$ & $0$ & $0$ & $0$ \\
    \(E_1^3\) & $0$ & $1$ & $-2$ & $0$ & $0$ & $0$ & $0$ & $0$ & $0$ & $0$ & $0$ & $0$ & $0$ & $0$ & $0$ & $1$ & $0$ & $0$ \\
    \(E_1^4\) & $0$ & $1$ & $0$ & $-2$ & $0$ & $0$ & $0$ & $0$ & $0$ & $0$ & $0$ & $0$ & $0$ & $0$ & $0$ & $0$ & $1$ & $0$ \\
    \hline
    \(E_2^1\) & $0$ & $0$ & $0$ & $0$ & $-2$ & $1$ & $0$ & $0$ & $0$ & $0$ & $0$ & $0$ & $0$ & $0$ & $1$ & $0$ & $0$ & $0$ \\
    \(E_2^2\) & $0$ & $0$ & $0$ & $0$ & $1$ & $-2$ & $1$ & $0$ & $0$ & $0$ & $0$ & $0$ & $0$ & $1$ & $0$ & $0$ & $0$ & $0$ \\
    \(E_2^3\) & $0$ & $0$ & $0$ & $0$ & $0$ & $1$ & $-2$ & $1$ & $0$ & $0$ & $0$ & $0$ & $0$ & $0$ & $0$ & $0$ & $0$ & $0$ \\
    \(E_2^4\) & $0$ & $0$ & $0$ & $0$ & $0$ & $0$ & $1$ & $-2$ & $0$ & $0$ & $0$ & $0$ & $0$ & $0$ & $0$ & $0$ & $0$ & $0$ \\
    \hline
    \(E_3^1\) & $0$ & $0$ & $0$ & $0$ & $0$ & $0$ & $0$ & $0$ & $-2$ & $1$ & $0$ & $0$ & $0$ & $0$ & $0$ & $0$ & $0$ & $0$ \\
    \(E_3^2\) & $0$ & $0$ & $0$ & $0$ & $0$ & $0$ & $0$ & $0$ & $1$ & $-2$ & $1$ & $0$ & $0$ & $0$ & $0$ & $0$ & $0$ & $0$ \\
    \(E_3^3\) & $0$ & $0$ & $0$ & $0$ & $0$ & $0$ & $0$ & $0$ & $0$ & $1$ & $-2$ & $1$ & $0$ & $0$ & $0$ & $0$ & $0$ & $0$ \\
    \(E_3^4\) & $0$ & $0$ & $0$ & $0$ & $0$ & $0$ & $0$ & $0$ & $0$ & $0$ & $1$ & $-2$ & $0$ & $0$ & $0$ & $1$ & $0$ & $0$ \\
    \hline
    \(E_4^1\) & $0$ & $0$ & $0$ & $0$ & $0$ & $0$ & $0$ & $0$ & $0$ & $0$ & $0$ & $0$ & $-2$ & $0$ & $0$ & $0$ & $1$ & $0$ \\
    \hline
    \(\widetilde{C_{1}}\) & $0$ & $0$ & $0$ & $0$ & $0$ & $1$ & $0$ & $0$ & $0$ & $0$ & $0$ & $0$ & $0$ & $-2$ & $0$ & $0$ & $0$ & $1$ \\
    \(\widetilde{C_{3}}\) & $1$ & $0$ & $0$ & $0$ & $1$ & $0$ & $0$ & $0$ & $0$ & $0$ & $0$ & $0$ & $0$ & $0$ & $-2$ & $0$ & $0$ & $1$ \\
    \(\widetilde{C_{4}}\) & $0$ & $0$ & $1$ & $0$ & $0$ & $0$ & $0$ & $0$ & $0$ & $0$ & $0$ & $1$ & $0$ & $0$ & $0$ & $-2$ & $0$ & $1$ \\
    \(\widetilde{C_{5}}\) & $0$ & $0$ & $0$ & $1$ & $0$ & $0$ & $0$ & $0$ & $0$ & $0$ & $0$ & $0$ & $1$ & $0$ & $0$ & $0$ & $-2$ & $1$ \\
    \(\widetilde{H_{\mathcal{S}}}\) & $0$ & $0$ & $0$ & $0$ & $0$ & $0$ & $0$ & $0$ & $0$ & $0$ & $0$ & $0$ & $0$ & $1$ & $1$ & $1$ & $1$ & $4$ \\
    \hline
  \end{tabular}.
\end{table}
Note that the intersection matrix is non-degenerate, hence we have \(\rk(L_{\lambda}) = 18\). Recall that \(\Pic(\widetilde{\mathcal{S}_{\lambda}})\) is generated by linear equivalence classes of the exceptional curves \(E_i^j\) and of strict transforms of curves on \(\mathcal{S}_{\lambda}\). Denote by \(M\) the corresponding intersection matrix.

The conjectural orthogonal complement to the lattice \(L_{\mathcal{S}}\) has the form
\[
  N = H \oplus \Pic(X), \quad
  \Pic(X) =
  \begin{pmatrix}
    0 & 3 \\
    3 & 4
  \end{pmatrix}.
\]
We choose the following generators of discriminant groups \(D_M\) and \(D_N\) of the lattices \(M\) and \(N\), respectively:
\begin{gather*}
  \begin{pmatrix}
    \frac{5}{9} & \frac{1}{3} & \frac{8}{9} & \frac{2}{9} & \frac{5}{9} & \frac{1}{3} & \frac{2}{9} & \frac{1}{9} & \frac{8}{9} & \frac{7}{9} & \frac{2}{3} & \frac{5}{9} & \frac{5}{9} & \frac{8}{9} & \frac{7}{9} & \frac{4}{9} & \frac{1}{9} & \frac{4}{9}
  \end{pmatrix}, \quad
  \begin{pmatrix}
    0 & 0 & \frac{8}{9} & \frac{1}{3}
  \end{pmatrix}.
\end{gather*}
Then bilinear and quadratic discriminant forms on \(D_M\) and \(D_N\) can be represented as
\begin{gather*}
  B_M = 
  \begin{pmatrix}
    \frac{7}{9}
  \end{pmatrix}, \quad
  B_N = 
  \begin{pmatrix}
    \frac{2}{9}
  \end{pmatrix}; \quad
  Q_M = \left(\frac{16}{9}\right), \quad
  Q_N = \left(\frac{2}{9}\right).
\end{gather*}
The lattices \(M\) and \(N\) have the signature \((1, 17)\) and \((2, 2)\), respectively, hence \(M^{\perp} \simeq N\) in the K3 lattice.

%%% Local Variables:
%%% mode: latex
%%% TeX-master: "../main"
%%% End:
