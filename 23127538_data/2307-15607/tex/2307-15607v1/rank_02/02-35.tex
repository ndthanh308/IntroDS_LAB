\subsection{Family \textnumero2.35}\label{subsection:02-35}

The toric Landau--Ginzburg model for this family is given by the Laurent polynomial
\begin{gather*}
  x + y + z + x y^{-1} z^{-1} + x^{-1}
\end{gather*}
(see~\cite[Family \textnumero2.35]{cheltsov2018katzarkov}). It is a Minkowski polynomial \textnumero6 (see~\cite[Appendix~B: bucket~7]{akhtar2012minkowski}).

The pencil \(\mathcal{S}\) is defined by the equation
\begin{gather*}
  X^{2} Y Z + X Y^{2} Z + X Y Z^{2} + X^{2} T^{2} + Y Z T^{2} = \lambda X Y Z T.
\end{gather*}

\begin{remark}
  Note that the equation is invariant under the permutation \((Y, Z) \mapsto (Z, Y)\).
\end{remark}

Members \(\mathcal{S}_{\lambda}\) of the pencil are irreducible for any parameter \(\lambda \in \mathbb{P}^1\) except
\begin{gather*}
  \mathcal{S}_{\infty} = S_{(X)} + S_{(Y)} + S_{(Z)} + S_{(T)}.
\end{gather*}
The base locus of the pencil \(\mathcal{S}\) consists of the following curves:
\begin{gather*}
  C_1 = C_{(X, Y)}, \;
  C_2 = C_{(X, Z)}, \;
  C_3 = C_{(X, T)}, \;
  C_4 = C_{(Y, T)}, \;
  C_5 = C_{(Z, T)}, \;
  C_6 = C_{(T, X + Y + Z)}.
\end{gather*}

Their linear equivalence classes on the generic member \(\mathcal{S}_{\Bbbk}\) of the pencil satisfy the following relations:
\begin{gather*}
  \begin{pmatrix}
    [\mathcal{S}_{\Bbbk} \cdot S_{(X)}] - [H_{\mathcal{S}}] \\
    [\mathcal{S}_{\Bbbk} \cdot S_{(Y)}] - [H_{\mathcal{S}}] \\
    [\mathcal{S}_{\Bbbk} \cdot S_{(Z)}] - [H_{\mathcal{S}}] \\
    [\mathcal{S}_{\Bbbk} \cdot S_{(T)}] - [H_{\mathcal{S}}]
  \end{pmatrix} = 
  \begin{pmatrix}
    1 & 1 & 2 & 0 & 0 & 0 & -1 \\
    2 & 0 & 0 & 2 & 0 & 0 & -1 \\
    0 & 2 & 0 & 0 & 2 & 0 & -1 \\
    0 & 0 & 1 & 1 & 1 & 1 & -1
  \end{pmatrix} \cdot
  \begin{pmatrix}
    [C_1] \\ \cdots \\ [C_{6}] \\ [H_{\mathcal{S}}]
  \end{pmatrix} = 0.
\end{gather*}
We can partially reduce the number of linear equivalence classes using these relations.
\[
  \begin{pmatrix}
    [C_4] + [C_5] + [C_6] \\ 2 [C_5] \\ 2 [C_6] \\ [H_{\mathcal{S}}]
  \end{pmatrix} =
  \begin{pmatrix}
    1 & 1 & 1 \\
    1 & -1 & 2 \\
    2 & 2 & -2 \\
    1 & 1 & 2
  \end{pmatrix} \cdot
  \begin{pmatrix}
    [C_1] \\ [C_2] \\ [C_3]
  \end{pmatrix}.
\]

For a general choice of \(\lambda \in \mathbb{C}\) the surface \(\mathcal{S}_{\lambda}\) has the following singularities:
\begin{itemize}\setlength{\itemindent}{2cm}
\item[\(P_{1} = P_{(X, Y, Z)}\):] type \(\mathbb{A}_1\) with the quadratic term \(X^2 + Y Z\);
\item[\(P_{2} = P_{(X, Y, T)}\):] type \(\mathbb{A}_5\) with the quadratic term \(X \cdot Y \);
\item[\(P_{3} = P_{(X, Z, T)}\):] type \(\mathbb{A}_5\) with the quadratic term \(X \cdot Z\);
\item[\(P_{4} = P_{(Y, Z, T)}\):] type \(\mathbb{A}_1\) with the quadratic term \(Y Z + T^2\);
\item[\(P_{5} = P_{(X, T, Y + Z)}\):] type \(\mathbb{A}_1\) with the quadratic term \(X (X + Y + Z - \lambda T) + T^2\);
\item[\(P_{6} = P_{(Y, T, X + Z)}\):] type \(\mathbb{A}_1\) with the quadratic term \(Y (X + Y + Z - \lambda T) - T^2\);
\item[\(P_{7} = P_{(Z, T, X + Y)}\):] type \(\mathbb{A}_1\) with the quadratic term \(Z (X + Y + Z - \lambda T) - T^2\).
\end{itemize}

The \(\mathbb{Q}\)-valued intersection matrix on the group \(A_{\mathcal{S}}\) has the following form:
\begin{table}[H]
  \renewcommand\arraystretch{1.42}
  \begin{tabular}{|c||c|c|c|c|c|c|}
    \hline
    \(\bullet\) & \(C_{1}\) & \(C_{2}\) & \(C_{3}\) & \(C_{4}\) & \(C_{5}\) & \(C_{6}\) \\
    \hline
    \hline
    \(C_{1}\) & $-\frac{1}{6}$ & $\frac{1}{2}$ & $\frac{1}{3}$ & $\frac{2}{3}$ & $0$ & $0$ \\
    \hline
    \(C_{2}\) & $\frac{1}{2}$ & $-\frac{1}{6}$ & $\frac{1}{3}$ & $0$ & $\frac{2}{3}$ & $0$ \\
    \hline
    \(C_{3}\) & $\frac{1}{3}$ & $\frac{1}{3}$ & $\frac{1}{6}$ & $\frac{1}{6}$ & $\frac{1}{6}$ & $\frac{1}{2}$ \\
    \hline
    \(C_{4}\) & $\frac{2}{3}$ & $0$ & $\frac{1}{6}$ & $-\frac{1}{6}$ & $\frac{1}{2}$ & $\frac{1}{2}$ \\
    \hline
    \(C_{5}\) & $0$ & $\frac{2}{3}$ & $\frac{1}{6}$ & $\frac{1}{2}$ & $-\frac{1}{6}$ & $\frac{1}{2}$ \\
    \hline
    \(C_{6}\) & $0$ & $0$ & $\frac{1}{2}$ & $\frac{1}{2}$ & $\frac{1}{2}$ & $-\frac{1}{2}$ \\
    \hline
  \end{tabular}.
\end{table}
Note that the intersection matrix has the rank 3.


Let \(\widetilde{\mathcal{S}_{\lambda}}\) and \(\widetilde{\mathcal{S}_{\Bbbk}}\) be the minimal resolution of a general member \(\mathcal{S}_{\lambda}\) and the generic member \(\mathcal{S}_{\Bbbk}\) of the pencil, respectively. Denote by \(L_{\lambda} \subset \Pic(\widetilde{\mathcal{S}_{\lambda}})\) and \(L_{\mathcal{S}} \subset \Pic(\widetilde{\mathcal{S}_{\Bbbk}})\) the subgroups generated by linear equivalence classes of exceptional divisors of the resolution and by linear equivalence classes of strict transforms \(\widetilde{C_i}\), \(\widetilde{H_{\mathcal{S}}}\) of the above-introduced curves. By construction the subgroup \(L_{\lambda}\) is equipped with the \(\Gal(\Bbbk)\)-action, and the subgroup \(L_{\mathcal{S}}\) can be identified with the subgroup \(L_{\lambda}^{\Gal(\Bbbk)} \subset L_{\lambda}\). In our case we have \(L_{\lambda} = L_{\mathcal{S}}\).

The intersection matrix on the lattice \(L_{\lambda}\) has the following form:
\begin{table}[H]
  \begin{adjustwidth}{-0.75cm}{}
    \begin{tabular}{|c||ccccc|ccccc|c|c|c|c|c|cccccc|}
      \hline
      \(\bullet\) & \(E_1^1\) & \(E_2^1\) & \(E_2^2\) & \(E_2^3\) & \(E_2^4\) & \(E_2^5\) & \(E_3^1\) & \(E_3^2\) & \(E_3^3\) & \(E_3^4\) & \(E_3^5\) & \(E_4^1\) & \(E_5^1\) & \(E_6^1\) & \(E_7^1\) & \(\widetilde{C_{1}}\) & \(\widetilde{C_{2}}\) & \(\widetilde{C_{3}}\) & \(\widetilde{C_{4}}\) & \(\widetilde{C_{5}}\) & \(\widetilde{C_{6}}\) \\
      \hline
      \hline
      \(E_1^1\) & $-2$ & $0$ & $0$ & $0$ & $0$ & $0$ & $0$ & $0$ & $0$ & $0$ & $0$ & $0$ & $0$ & $0$ & $0$ & $1$ & $1$ & $0$ & $0$ & $0$ & $0$ \\
      \(E_2^1\) & $0$ & $-2$ & $1$ & $0$ & $0$ & $0$ & $0$ & $0$ & $0$ & $0$ & $0$ & $0$ & $0$ & $0$ & $0$ & $0$ & $0$ & $1$ & $0$ & $0$ & $0$ \\
      \(E_2^2\) & $0$ & $1$ & $-2$ & $1$ & $0$ & $0$ & $0$ & $0$ & $0$ & $0$ & $0$ & $0$ & $0$ & $0$ & $0$ & $0$ & $0$ & $0$ & $0$ & $0$ & $0$ \\
      \(E_2^3\) & $0$ & $0$ & $1$ & $-2$ & $1$ & $0$ & $0$ & $0$ & $0$ & $0$ & $0$ & $0$ & $0$ & $0$ & $0$ & $0$ & $0$ & $0$ & $0$ & $0$ & $0$ \\
      \(E_2^4\) & $0$ & $0$ & $0$ & $1$ & $-2$ & $1$ & $0$ & $0$ & $0$ & $0$ & $0$ & $0$ & $0$ & $0$ & $0$ & $1$ & $0$ & $0$ & $0$ & $0$ & $0$ \\
      \(E_2^5\) & $0$ & $0$ & $0$ & $0$ & $1$ & $-2$ & $0$ & $0$ & $0$ & $0$ & $0$ & $0$ & $0$ & $0$ & $0$ & $0$ & $0$ & $0$ & $1$ & $0$ & $0$ \\
      \hline
      \(E_3^1\) & $0$ & $0$ & $0$ & $0$ & $0$ & $0$ & $-2$ & $1$ & $0$ & $0$ & $0$ & $0$ & $0$ & $0$ & $0$ & $0$ & $0$ & $1$ & $0$ & $0$ & $0$ \\
      \(E_3^2\) & $0$ & $0$ & $0$ & $0$ & $0$ & $0$ & $1$ & $-2$ & $1$ & $0$ & $0$ & $0$ & $0$ & $0$ & $0$ & $0$ & $0$ & $0$ & $0$ & $0$ & $0$ \\
      \(E_3^3\) & $0$ & $0$ & $0$ & $0$ & $0$ & $0$ & $0$ & $1$ & $-2$ & $1$ & $0$ & $0$ & $0$ & $0$ & $0$ & $0$ & $0$ & $0$ & $0$ & $0$ & $0$ \\
      \(E_3^4\) & $0$ & $0$ & $0$ & $0$ & $0$ & $0$ & $0$ & $0$ & $1$ & $-2$ & $1$ & $0$ & $0$ & $0$ & $0$ & $0$ & $1$ & $0$ & $0$ & $0$ & $0$ \\
      \(E_3^5\) & $0$ & $0$ & $0$ & $0$ & $0$ & $0$ & $0$ & $0$ & $0$ & $1$ & $-2$ & $0$ & $0$ & $0$ & $0$ & $0$ & $0$ & $0$ & $0$ & $1$ & $0$ \\
      \hline
      \(E_4^1\) & $0$ & $0$ & $0$ & $0$ & $0$ & $0$ & $0$ & $0$ & $0$ & $0$ & $0$ & $-2$ & $0$ & $0$ & $0$ & $0$ & $0$ & $0$ & $1$ & $1$ & $0$ \\
      \hline
      \(E_5^1\) & $0$ & $0$ & $0$ & $0$ & $0$ & $0$ & $0$ & $0$ & $0$ & $0$ & $0$ & $0$ & $-2$ & $0$ & $0$ & $0$ & $0$ & $1$ & $0$ & $0$ & $1$ \\
      \hline
      \(E_6^1\) & $0$ & $0$ & $0$ & $0$ & $0$ & $0$ & $0$ & $0$ & $0$ & $0$ & $0$ & $0$ & $0$ & $-2$ & $0$ & $0$ & $0$ & $0$ & $1$ & $0$ & $1$ \\
      \hline
      \(E_7^1\) & $0$ & $0$ & $0$ & $0$ & $0$ & $0$ & $0$ & $0$ & $0$ & $0$ & $0$ & $0$ & $0$ & $0$ & $-2$ & $0$ & $0$ & $0$ & $0$ & $1$ & $1$ \\
      \hline
      \(\widetilde{C_{1}}\) & $1$ & $0$ & $0$ & $0$ & $1$ & $0$ & $0$ & $0$ & $0$ & $0$ & $0$ & $0$ & $0$ & $0$ & $0$ & $-2$ & $0$ & $0$ & $0$ & $0$ & $0$ \\
      \(\widetilde{C_{2}}\) & $1$ & $0$ & $0$ & $0$ & $0$ & $0$ & $0$ & $0$ & $0$ & $1$ & $0$ & $0$ & $0$ & $0$ & $0$ & $0$ & $-2$ & $0$ & $0$ & $0$ & $0$ \\
      \(\widetilde{C_{3}}\) & $0$ & $1$ & $0$ & $0$ & $0$ & $0$ & $1$ & $0$ & $0$ & $0$ & $0$ & $0$ & $1$ & $0$ & $0$ & $0$ & $0$ & $-2$ & $0$ & $0$ & $0$ \\
      \(\widetilde{C_{4}}\) & $0$ & $0$ & $0$ & $0$ & $0$ & $1$ & $0$ & $0$ & $0$ & $0$ & $0$ & $1$ & $0$ & $1$ & $0$ & $0$ & $0$ & $0$ & $-2$ & $0$ & $0$ \\
      \(\widetilde{C_{5}}\) & $0$ & $0$ & $0$ & $0$ & $0$ & $0$ & $0$ & $0$ & $0$ & $0$ & $1$ & $1$ & $0$ & $0$ & $1$ & $0$ & $0$ & $0$ & $0$ & $-2$ & $0$ \\
      \(\widetilde{C_{6}}\) & $0$ & $0$ & $0$ & $0$ & $0$ & $0$ & $0$ & $0$ & $0$ & $0$ & $0$ & $0$ & $1$ & $1$ & $1$ & $0$ & $0$ & $0$ & $0$ & $0$ & $-2$ \\
      \hline
    \end{tabular}.
  \end{adjustwidth}
\end{table}
Note that the intersection matrix is degenerate. We choose the following integral basis of the lattice \(L_{\lambda}\):
\begin{align*}
  \begin{pmatrix}
    [E_7^1] \\ [\widetilde{C_{2}}] \\ [\widetilde{C_{5}}]
  \end{pmatrix}^T = 
  \begin{pmatrix}
    2 & -3 & 0 & 3 & 6 & 5 & -5 & -4 & -3 & -2 & -1 & 2 & -4 & 1 & 4 & -6 & 4 & -2 \\
    0 & -1 & 0 & 1 & 2 & 2 & -2 & -2 & -2 & -2 & -1 & 1 & -1 & 1 & 1 & -2 & 2 & 0 \\
    -1 & 3 & 1 & -1 & -3 & -3 & 4 & 3 & 2 & 1 & 0 & -2 & 3 & -1 & -2 & 5 & -3 & 1
  \end{pmatrix} \cdot \\
  \begin{pmatrix}
    [E_1^1] & [E_2^1] & [E_2^2] & [E_2^3] & [E_2^4] & [E_2^5] & [E_3^1] & [E_3^2] & [E_3^3] & \\ [E_3^4] & [E_3^5] & [E_4^1] & [E_5^1] & [E_6^1] & [\widetilde{C_{1}}] & [\widetilde{C_{3}}] & [\widetilde{C_{4}}] & [\widetilde{C_{6}}]
  \end{pmatrix}^T,
\end{align*}
hence we have \(\rk(L_{\lambda}) = 18\). Recall that \(\Pic(\widetilde{\mathcal{S}_{\lambda}})\) is generated by linear equivalence classes of the exceptional curves \(E_i^j\) and of strict transforms of curves on \(\mathcal{S}_{\lambda}\). Denote by \(M\) the corresponding intersection matrix.

The conjectural orthogonal complement to the lattice \(L_{\mathcal{S}}\) has the form
\[
  N = H \oplus \Pic(X), \quad
  \Pic(X) =
  \begin{pmatrix}
    -2 & 0 \\
    0 & 4
  \end{pmatrix}.
\]
We choose the following generators of discriminant groups \(D_M\) and \(D_N\) of the lattices \(M\) and \(N\), respectively:
\begin{gather*}
  \begin{pmatrix}
    0 & \frac{1}{2} & 0 & \frac{1}{2} & 0 & \frac{1}{2} & \frac{1}{2} & 0 & \frac{1}{2} & 0 & \frac{1}{2} & \frac{1}{2} & 0 & 0 & 0 & 0 & 0 & 0 \\
    \frac{1}{2} & \frac{1}{2} & \frac{1}{2} & \frac{1}{2} & \frac{1}{2} & \frac{1}{2} & \frac{3}{4} & 0 & \frac{1}{4} & \frac{1}{2} & \frac{3}{4} & \frac{1}{4} & \frac{3}{4} & \frac{1}{4} & 0 & \frac{1}{2} & \frac{1}{2} & 0
  \end{pmatrix}, \quad
  \begin{pmatrix}
    0 & 0 & \frac{1}{2} & 0 \\
    0 & 0 & 0 & \frac{1}{4}
  \end{pmatrix}.
\end{gather*}
Then bilinear and quadratic discriminant forms on \(D_M\) and \(D_N\) can be represented as
\begin{gather*}
  B_M = 
  \begin{pmatrix}
    \frac{1}{2} & 0 \\
    0 & \frac{3}{4}
  \end{pmatrix}, \quad
  B_N = 
  \begin{pmatrix}
    \frac{1}{2} & 0 \\
    0 & \frac{1}{4}
  \end{pmatrix}; \quad
  Q_M = \left(\frac{1}{2},\,\frac{7}{4}\right), \quad
  Q_N = \left(\frac{3}{2},\,\frac{1}{4}\right).
\end{gather*}
The lattices \(M\) and \(N\) have the signature \((1, 17)\) and \((2, 2)\), respectively, hence \(M^{\perp} \simeq N\) in the K3 lattice.

%%% Local Variables:
%%% mode: latex
%%% TeX-master: "../main"
%%% End:
