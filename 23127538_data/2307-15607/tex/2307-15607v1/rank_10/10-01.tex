\subsection{Family \textnumero10.1}\label{subsection:10-01}

The toric Landau--Ginzburg model for this family is given by the Laurent polynomial
\[
  (x + y + 1)^6 x^{-1} y^{-2} + z + z^{-1}
\]
(see~\cite[Family \textnumero10.1]{cheltsov2018katzarkov}). It is not a Minkowski polynomial. Let us apply the following birational transform:
\[
  (\mathbb{C}^*)^3 \dashrightarrow \mathbb{C}^* \times \mathbb{C}^* \times \mathbb{C}^*, \quad
  (x, y, z) \mapsto (B^{-1} - B^{-2} C^{-1} - 1, B^{-2} C^{-1}, Y).
\]

The pencil \(\mathcal{S}\) is defined by the equation
\[
  X Y C^3 = (A^3 - B C (A - B)) (X^2 + Y^2 - \lambda X Y).
\]
Members \(\mathcal{S}_{\lambda}\) of the pencil are irreducible for any parameter \(\lambda \in \mathbb{P}^1\) except the following one:
\[
  \mathcal{S}_{\infty} = S_{(X)} + S_{(Y)} + S_{(A^3 - B C (A - B))}.
\]
The base locus of the pencil \(\mathcal{S}\) consists of the following curves:
\[
    C_1 = C_{(A, C)}, \;
    C_2 = C_{(X, A^3 - B C (A - B))}, \;
    C_3 = C_{(Y, A^3 - B C (A - B))}.
\]
Their linear equivalence classes on the generic member \(\mathcal{S}_{\Bbbk}\) of the pencil satisfy the following relations:
\begin{gather*}
  \begin{pmatrix}
    [\mathcal{S}_{\Bbbk} \cdot S_{(X)}] - [H_{\mathcal{S}}^{(1)}] \\
    [\mathcal{S}_{\Bbbk} \cdot S_{(Y)}] - [H_{\mathcal{S}}^{(1)}]
  \end{pmatrix} =
  \begin{pmatrix}
    0 & 1 & 0 & -1 \\
    0 & 0 & 1 & -1
  \end{pmatrix} \cdot
  \begin{pmatrix}
    [C_1] \\ [C_2] \\ [C_3] \\ [H_{\mathcal{S}}^{(1)}]
  \end{pmatrix}
  = 0.
\end{gather*}
We can reduce the number of linear equivalence classes using these relations:
\begin{gather*}
  \begin{pmatrix}
    [C_{3}] \\ [H_{\mathcal{S}}^{(1)}]
  \end{pmatrix} = 
  \begin{pmatrix}
    [C_{2}] \\ [C_{2}]
  \end{pmatrix}.
\end{gather*}

Put \(\mu (\mu - 1) = (\lambda + 4)^{-1}\). For a general choice of \(\lambda \in \mathbb{C}\) the surface \(\mathcal{S}_{\lambda}\) has the following singularities:
\begin{itemize}\setlength{\itemindent}{2cm}
\item[\(P_1 = P_{(A, C, \mu X - (\mu - 1) Y)}\):] type \(\mathbb{A}_8\);
\item[\(P_2 = P_{(A, C, (\mu - 1) X - \mu Y)}\):] type \(\mathbb{A}_8\).
\end{itemize}

Let \(\widetilde{\mathcal{S}_{\lambda}}\) and \(\widetilde{\mathcal{S}_{\Bbbk}}\) be the minimal resolution of a general member \(\mathcal{S}_{\lambda}\) and the generic member \(\mathcal{S}_{\Bbbk}\) of the pencil, respectively. Denote by \(L_{\lambda} \subset \Pic(\widetilde{\mathcal{S}_{\lambda}})\) and \(L_{\mathcal{S}} \subset \Pic(\widetilde{\mathcal{S}_{\Bbbk}})\) the subgroups generated by linear equivalence classes of exceptional divisors of the resolution and by linear equivalence classes of strict transforms \(\widetilde{C_i}\), \(\widetilde{H_{\mathcal{S}}^{(j)}}\) of the above-introduced curves. By construction the subgroup \(L_{\lambda}\) is equipped with the \(\Gal(\Bbbk)\)-action, and the subgroup \(L_{\mathcal{S}}\) can be identified with the subgroup \(L_{\lambda}^{\Gal(\Bbbk)} \subset L_{\lambda}\). Its Galois orbits can be described as
\[
  (E_1^1, E_2^1), \; (E_1^2, E_2^2), \; (E_1^3, E_2^3), \; (E_1^4, E_2^4), \; (E_1^5, E_2^5), \; (E_1^6, E_2^6), \; (E_1^7, E_2^7), \; (E_1^8, E_2^8).
\]

The intersection matrix on the lattice \(L_{\lambda}\) has the following form:
\begin{table}[H]
  \begin{tabular}{|c||cccccccc|cccccccc|ccc|}
    \hline
    \(\bullet\) & \(E_1^1\) & \(E_1^2\) & \(E_1^3\) & \(E_1^4\) & \(E_1^5\) & \(E_1^6\) & \(E_1^7\) & \(E_1^8\) & \(E_2^1\) & \(E_2^2\) & \(E_2^3\) & \(E_2^4\) & \(E_2^5\) & \(E_2^6\) & \(E_2^7\) & \(E_2^8\) & \(\widetilde{C_{1}}\) & \(\widetilde{C_{2}}\) & \(\widetilde{H_{\mathcal{S}}^{(2)}}\) \\
    \hline
    \hline
    \(E_1^1\) & $-2$ & $1$ & $0$ & $0$ & $0$ & $0$ & $0$ & $0$ & $0$ & $0$ & $0$ & $0$ & $0$ & $0$ & $0$ & $0$ & $1$ & $0$ & $0$ \\
    \(E_1^2\) & $1$ & $-2$ & $1$ & $0$ & $0$ & $0$ & $0$ & $0$ & $0$ & $0$ & $0$ & $0$ & $0$ & $0$ & $0$ & $0$ & $0$ & $0$ & $0$ \\
    \(E_1^3\) & $0$ & $1$ & $-2$ & $1$ & $0$ & $0$ & $0$ & $0$ & $0$ & $0$ & $0$ & $0$ & $0$ & $0$ & $0$ & $0$ & $0$ & $0$ & $0$ \\
    \(E_1^4\) & $0$ & $0$ & $1$ & $-2$ & $1$ & $0$ & $0$ & $0$ & $0$ & $0$ & $0$ & $0$ & $0$ & $0$ & $0$ & $0$ & $0$ & $0$ & $0$ \\
    \(E_1^5\) & $0$ & $0$ & $0$ & $1$ & $-2$ & $1$ & $0$ & $0$ & $0$ & $0$ & $0$ & $0$ & $0$ & $0$ & $0$ & $0$ & $0$ & $0$ & $0$ \\
    \(E_1^6\) & $0$ & $0$ & $0$ & $0$ & $1$ & $-2$ & $1$ & $0$ & $0$ & $0$ & $0$ & $0$ & $0$ & $0$ & $0$ & $0$ & $0$ & $0$ & $0$ \\
    \(E_1^7\) & $0$ & $0$ & $0$ & $0$ & $0$ & $1$ & $-2$ & $1$ & $0$ & $0$ & $0$ & $0$ & $0$ & $0$ & $0$ & $0$ & $0$ & $0$ & $0$ \\
    \(E_1^8\) & $0$ & $0$ & $0$ & $0$ & $0$ & $0$ & $1$ & $-2$ & $0$ & $0$ & $0$ & $0$ & $0$ & $0$ & $0$ & $0$ & $0$ & $0$ & $0$ \\
    \hline
    \(E_2^1\) & $0$ & $0$ & $0$ & $0$ & $0$ & $0$ & $0$ & $0$ & $-2$ & $1$ & $0$ & $0$ & $0$ & $0$ & $0$ & $0$ & $1$ & $0$ & $0$ \\
    \(E_2^2\) & $0$ & $0$ & $0$ & $0$ & $0$ & $0$ & $0$ & $0$ & $1$ & $-2$ & $1$ & $0$ & $0$ & $0$ & $0$ & $0$ & $0$ & $0$ & $0$ \\
    \(E_2^3\) & $0$ & $0$ & $0$ & $0$ & $0$ & $0$ & $0$ & $0$ & $0$ & $1$ & $-2$ & $1$ & $0$ & $0$ & $0$ & $0$ & $0$ & $0$ & $0$ \\
    \(E_2^4\) & $0$ & $0$ & $0$ & $0$ & $0$ & $0$ & $0$ & $0$ & $0$ & $0$ & $1$ & $-2$ & $1$ & $0$ & $0$ & $0$ & $0$ & $0$ & $0$ \\
    \(E_2^5\) & $0$ & $0$ & $0$ & $0$ & $0$ & $0$ & $0$ & $0$ & $0$ & $0$ & $0$ & $1$ & $-2$ & $1$ & $0$ & $0$ & $0$ & $0$ & $0$ \\
    \(E_2^6\) & $0$ & $0$ & $0$ & $0$ & $0$ & $0$ & $0$ & $0$ & $0$ & $0$ & $0$ & $0$ & $1$ & $-2$ & $1$ & $0$ & $0$ & $0$ & $0$ \\
    \(E_2^7\) & $0$ & $0$ & $0$ & $0$ & $0$ & $0$ & $0$ & $0$ & $0$ & $0$ & $0$ & $0$ & $0$ & $1$ & $-2$ & $1$ & $0$ & $0$ & $0$ \\
    \(E_2^8\) & $0$ & $0$ & $0$ & $0$ & $0$ & $0$ & $0$ & $0$ & $0$ & $0$ & $0$ & $0$ & $0$ & $0$ & $1$ & $-2$ & $0$ & $0$ & $0$ \\
    \hline
    \(\widetilde{C_{1}}\) & $1$ & $0$ & $0$ & $0$ & $0$ & $0$ & $0$ & $0$ & $1$ & $0$ & $0$ & $0$ & $0$ & $0$ & $0$ & $0$ & $-2$ & $1$ & $0$ \\
    \(\widetilde{C_{2}}\) & $0$ & $0$ & $0$ & $0$ & $0$ & $0$ & $0$ & $0$ & $0$ & $0$ & $0$ & $0$ & $0$ & $0$ & $0$ & $0$ & $1$ & $0$ & $3$ \\
    \(\widetilde{H_{\mathcal{S}}^{(2)}}\) & $0$ & $0$ & $0$ & $0$ & $0$ & $0$ & $0$ & $0$ & $0$ & $0$ & $0$ & $0$ & $0$ & $0$ & $0$ & $0$ & $0$ & $3$ & $2$ \\
    \hline
  \end{tabular}.
\end{table}
Note that the intersection matrix is degenerate. We choose the following integral basis of the lattice \(L_{\lambda}\):
\begin{align*}
  \begin{pmatrix}
    [E_2^8]
  \end{pmatrix} =
  \begin{pmatrix}
    -8 & -7 & -6 & -5 & -4 & -3 & -2 & -1 & -8 & -7 & -6 & -5 & -4 & -3 & -2 & -9 & -2 & 3
  \end{pmatrix} \cdot \\
  \begin{pmatrix}
    E_1^1 & E_1^2 & E_1^3 & E_1^4 & E_1^5 & E_1^6 & E_1^7 & E_1^8 & E_2^1 & E_2^2 & E_2^3 & E_2^4 & E_2^5 & E_2^6 & E_2^7 & \widetilde{C_1} & \widetilde{C_2} & \widetilde{H_{\mathcal{S}}^{(2)}}
  \end{pmatrix}^T,
\end{align*}
hence we have \(\rk(L_{\lambda}) = 18\). Recall that \(\Pic(\widetilde{\mathcal{S}_{\lambda}})\) is generated by linear equivalence classes of the exceptional curves \(E_i^j\) and of strict transforms of curves on \(\mathcal{S}_{\lambda}\).

The intersection matrix on the lattice \(L_{\mathcal{S}} = L_{\lambda}^{\Gal(\Bbbk)}\) has the following form:
\begin{table}[H]
  \begin{adjustwidth}{-0.75cm}{}
    \begin{tabular}{|c||cccccccc|ccc|}
      \hline
      \(\bullet\) & \(E_1^1 + E_2^1\) & \(E_1^2 + E_2^2\) & \(E_1^3 + E_2^3\) & \(E_1^4 + E_2^4\) & \(E_1^5 + E_2^5\) & \(E_1^6 + E_2^6\) & \(E_1^7 + E_2^7\) & \(E_1^8 + E_2^8\) & \(\widetilde{C_1}\) & \(\widetilde{C_2}\) & \(\widetilde{H_{\mathcal{S}}^{(2)}}\) \\
      \hline
      \(E_1^1 + E_2^1\) & $-4$ & $2$ & $0$ & $0$ & $0$ & $0$ & $0$ & $0$ & $2$ & $0$ & $0$ \\
      \(E_1^2 + E_2^2\) & $2$ & $-4$ & $2$ & $0$ & $0$ & $0$ & $0$ & $0$ & $0$ & $0$ & $0$ \\
      \(E_1^3 + E_2^3\) & $0$ & $2$ & $-4$ & $2$ & $0$ & $0$ & $0$ & $0$ & $0$ & $0$ & $0$ \\
      \(E_1^4 + E_2^4\) & $0$ & $0$ & $2$ & $-4$ & $2$ & $0$ & $0$ & $0$ & $0$ & $0$ & $0$ \\
      \(E_1^5 + E_2^5\) & $0$ & $0$ & $0$ & $2$ & $-4$ & $2$ & $0$ & $0$ & $0$ & $0$ & $0$ \\
      \(E_1^6 + E_2^6\) & $0$ & $0$ & $0$ & $0$ & $2$ & $-4$ & $2$ & $0$ & $0$ & $0$ & $0$ \\
      \(E_1^7 + E_2^7\) & $0$ & $0$ & $0$ & $0$ & $0$ & $2$ & $-4$ & $2$ & $0$ & $0$ & $0$ \\
      \(E_1^8 + E_2^8\) & $0$ & $0$ & $0$ & $0$ & $0$ & $0$ & $2$ & $-4$ & $0$ & $0$ & $0$ \\
      \hline
      \(\widetilde{C_1}\) & $2$ & $0$ & $0$ & $0$ & $0$ & $0$ & $0$ & $0$ & $-2$ & $1$ & $0$ \\
      \(\widetilde{C_2}\) & $0$ & $0$ & $0$ & $0$ & $0$ & $0$ & $0$ & $0$ & $1$ & $0$ & $3$ \\
      \(\widetilde{H_{\mathcal{S}}^{(2)}}\) & $0$ & $0$ & $0$ & $0$ & $0$ & $0$ & $0$ & $0$ & $0$ & $3$ & $2$ \\
      \hline
    \end{tabular}.
  \end{adjustwidth}
\end{table}

Note that the intersection matrix is degenerate. We choose the following integral basis of the lattice \(L_{\mathcal{S}}\):
\begin{align*}                                                          
  \begin{pmatrix}
    [E_1^8 + E_2^8]
  \end{pmatrix} =                                                  
  \begin{pmatrix}
    -8 & -7 & -6 & -5 & -4 & -3 & -2 & -9 & -2 & 3
  \end{pmatrix} \cdot \\
  \begin{pmatrix}
    E_1^1 + E_2^1 & E_1^2 + E_2^2 & E_1^3 + E_2^3 & E_1^4 + E_2^4 & E_1^5 + E_2^5 & E_1^6 + E_2^6 & E_1^7 + E_2^7 & \widetilde{C_1} & \widetilde{C_2} & \widetilde{H_{\mathcal{S}}^{(2)}}
  \end{pmatrix}^T,
\end{align*}
hence we have \(\rk(L_{\mathcal{S}}) = 10\). Denote by \(M\) the induced intersection matrix on \(L_{\mathcal{S}}\) in this integral basis.

The conjectural orthogonal complement to the lattice \(L_{\mathcal{S}}\) has the form
\begin{gather*}
  N = H \oplus \Pic(X), \quad
  \Pic(X) =
  \begin{pmatrix}
    -2 & 0 & 0 & 0 & 0 & 0 & 0 & 0 & -1 & 0 \\
    0 & -2 & 0 & 0 & 0 & 0 & 0 & 0 & -1 & 0 \\
    0 & 0 & -2 & 0 & 0 & 0 & 0 & 0 & -1 & 0 \\
    0 & 0 & 0 & -2 & 0 & 0 & 0 & 0 & -1 & 0 \\
    0 & 0 & 0 & 0 & -2 & 0 & 0 & 0 & -1 & 0 \\
    0 & 0 & 0 & 0 & 0 & -2 & 0 & 0 & -1 & 0 \\
    0 & 0 & 0 & 0 & 0 & 0 & -2 & 0 & -1 & 0 \\
    0 & 0 & 0 & 0 & 0 & 0 & 0 & -2 & -1 & 0 \\
    -1 & -1 & -1 & -1 & -1 & -1 & -1 & -1 & 0 & 3 \\
    0 & 0 & 0 & 0 & 0 & 0 & 0 & 0 & 3 & 2
  \end{pmatrix}.
\end{gather*}

We choose the following generators of discriminant groups \(D_M\) and \(D_N\) of the lattices \(M\) and \(N\), respectively:
\begin{gather*}
  \begin{pmatrix}
    0 & 0 & 0 & \frac{1}{2} & \frac{1}{2} & 0 & \frac{1}{2} & 0 & 0 & 0 \\
    0 & 0 & 0 & 0 & \frac{1}{2} & 0 & \frac{1}{2} & 0 & 0 & 0 \\
    0 & 0 & \frac{1}{2} & 0 & \frac{1}{2} & \frac{1}{2} & \frac{1}{2} & 0 & 0 & 0 \\
    0 & 0 & \frac{1}{2} & 0 & \frac{1}{2} & \frac{1}{2} & 0 & 0 & 0 & 0 \\
    0 & \frac{1}{2} & 0 & 0 & 0 & 0 & \frac{1}{2} & \frac{1}{2} & 0 & \frac{1}{2} \\
    0 & \frac{1}{2} & \frac{1}{2} & 0 & \frac{1}{2} & 0 & 0 & \frac{1}{2} & 0 & \frac{1}{2} \\
    \frac{1}{2} & 0 & 0 & 0 & 0 & 0 & 0 & \frac{1}{2} & 0 & \frac{1}{2} \\
    0 & 0 & 0 & 0 & 0 & 0 & 0 & \frac{1}{2} & 0 & \frac{1}{2}
  \end{pmatrix}, \quad
  \begin{pmatrix}
    0 & 0 & \frac{1}{2} & \frac{1}{2} & 0 & 0 & \frac{1}{2} & 0 & \frac{1}{2} & \frac{1}{2} & 0 & \frac{1}{2} \\
    0 & 0 & \frac{1}{2} & \frac{1}{2} & 0 & 0 & 0 & \frac{1}{2} & \frac{1}{2} & \frac{1}{2} & 0 & \frac{1}{2} \\
    0 & 0 & 0 & 0 & \frac{1}{2} & 0 & \frac{1}{2} & \frac{1}{2} & \frac{1}{2} & 0 & 0 & 0 \\
    0 & 0 & \frac{1}{2} & \frac{1}{2} & \frac{1}{2} & 0 & \frac{1}{2} & \frac{1}{2} & 0 & 0 & 0 & \frac{1}{2} \\
    0 & 0 & 0 & 0 & 0 & \frac{1}{2} & \frac{1}{2} & \frac{1}{2} & 0 & \frac{1}{2} & 0 & 0 \\
    0 & 0 & \frac{1}{2} & \frac{1}{2} & \frac{1}{2} & \frac{1}{2} & 0 & 0 & \frac{1}{2} & 0 & 0 & \frac{1}{2} \\
    0 & 0 & 0 & \frac{1}{2} & 0 & 0 & 0 & 0 & 0 & 0 & 0 & \frac{1}{2} \\
    0 & 0 & \frac{1}{2} & 0 & 0 & 0 & 0 & 0 & 0 & 0 & 0 & \frac{1}{2}
  \end{pmatrix}.
\end{gather*}

Then bilinear and quadratic discriminant forms on \(D_M\) and \(D_N\) can be represented as
\begin{gather*}
  B(M) = 
  \begin{pmatrix}
    0 & \frac{1}{2} & 0 & 0 & 0 & 0 & 0 & 0 \\
    \frac{1}{2} & 0 & 0 & 0 & 0 & 0 & 0 & 0 \\
    0 & 0 & 0 & \frac{1}{2} & 0 & 0 & 0 & 0 \\
    0 & 0 & \frac{1}{2} & 0 & 0 & 0 & 0 & 0 \\
    0 & 0 & 0 & 0 & 0 & \frac{1}{2} & 0 & 0 \\
    0 & 0 & 0 & 0 & \frac{1}{2} & 0 & 0 & 0 \\
    0 & 0 & 0 & 0 & 0 & 0 & 0 & \frac{1}{2} \\
    0 & 0 & 0 & 0 & 0 & 0 & \frac{1}{2} & 0
  \end{pmatrix}, \quad
  B(N) = 
  \begin{pmatrix}
    0 & \frac{1}{2} & 0 & 0 & 0 & 0 & 0 & 0 \\
    \frac{1}{2} & 0 & 0 & 0 & 0 & 0 & 0 & 0 \\
    0 & 0 & 0 & \frac{1}{2} & 0 & 0 & 0 & 0 \\
    0 & 0 & \frac{1}{2} & 0 & 0 & 0 & 0 & 0 \\
    0 & 0 & 0 & 0 & 0 & \frac{1}{2} & 0 & 0 \\
    0 & 0 & 0 & 0 & \frac{1}{2} & 0 & 0 & 0 \\
    0 & 0 & 0 & 0 & 0 & 0 & 0 & \frac{1}{2} \\
    0 & 0 & 0 & 0 & 0 & 0 & \frac{1}{2} & 0
  \end{pmatrix}; \\
  Q(M) = \left(0,\,0,\,0,\,0,\,0,\,0,\,0,\,0\right), \quad
  Q(N) = \left(0,\,0,\,0,\,0,\,0,\,0,\,0,\,0\right).
\end{gather*}
The lattices \(M\) and \(N\) have the signature \((1, 17)\) and \((2, 10)\), respectively, hence \(M^{\perp} \simeq N\) in the K3 lattice.

%%% Local Variables:
%%% mode: latex
%%% TeX-master: "../main"
%%% End:
