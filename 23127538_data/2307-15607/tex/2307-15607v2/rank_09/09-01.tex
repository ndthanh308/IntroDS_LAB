\subsection{Family \textnumero9.1}\label{subsection:09-01}

The pencil \(\mathcal{S}\) is defined by the equation
\(X^3 Y = (Y^2 + Z^2 - \lambda Y Z) (X (Z - T) + T^2)\).
Members \(\mathcal{S}_{\lambda}\) of the pencil are irreducible for any \(\lambda \in \mathbb{P}^1\) except
\(\mathcal{S}_{\infty} = S_{(Y)} + S_{(Z)} + S_{(X (Z - T) + T^2)}\).
The base locus of the pencil \(\mathcal{S}\) consists of the following curves:
\begin{gather*}
  C_{1} = C_{(X, T)}, \;
  C_{2} = C_{(Y, Z)}, \;
  C_{3} = C_{(Y, X (Z - T) + T^2)}, \;
  C_{4} = C_{(Z, X^3 + Y T (X - T))}.
\end{gather*}
Their linear equivalence classes on the generic member \(\mathcal{S}_{\Bbbk}\) of the pencil satisfy the following relations:
\begin{gather*}
  \begin{pmatrix}
    [C_{3}] \\ [C_{4}]
  \end{pmatrix} = 
  \begin{pmatrix}
    0 & -2 & 1 \\
    0 & -1 & 1
  \end{pmatrix} \cdot
  \begin{pmatrix}
    [C_{1}] & [C_{2}] & [H_{\mathcal{S}}]
  \end{pmatrix}^T.
\end{gather*}

Put \(\mu (\mu - 1) = (\lambda - 2)^{-1}\). For a general choice of \(\lambda \in \mathbb{C}\) the surface \(\mathcal{S}_{\lambda}\) has the following singularities:
\begin{itemize}\setlength{\itemindent}{2cm}
\item[\(P_{1} = P_{(X, Y, Z)}\):] type \(\mathbb{A}_5\) with the quadratic term \((\mu Y - (\mu - 1) Z) \cdot ((\mu - 1) Y - \mu Z)\);
\item[\(P_{2} = P_{(X, Z, T)}\):] type \(\mathbb{A}_1\) with the quadratic term \(X (Z - T) + T^2\);
\item[\(P_{3} = P_{(X, T, (\mu - 1) Y - \mu Z)}\):] type \(\mathbb{A}_5\) with the quadratic term \(X \cdot ((\mu - 1) Y - \mu Z)\);
\item[\(P_{4} = P_{(X, T, \mu Y - (\mu - 1) Z)}\):] type \(\mathbb{A}_5\) with the quadratic term \(X \cdot (\mu Y - (\mu - 1) Z)\).
\end{itemize}

The only non-trivial Galois orbits on the lattice \(L_{\lambda}\) are
\[
  E_1^1 + E_1^5, E_1^2 + E_1^4, E_3^1 + E_4^1, E_3^2 + E_4^2, E_3^3 + E_4^3, E_3^4 + E_4^4, E_3^5 + E_4^5.
\]

The intersection matrix on the lattice \(L_{\lambda}\) is represented by
\begin{table}[H]
  \begin{tabular}{|c||ccccc|c|ccccc|ccccc|ccc|}
    \hline
    \(\bullet\) & \(E_1^1\) & \(E_1^2\) & \(E_1^3\) & \(E_1^4\) & \(E_1^5\) & \(E_2^1\) & \(E_3^1\) & \(E_3^2\) & \(E_3^3\) & \(E_3^4\) & \(E_3^5\) & \(E_4^1\) & \(E_4^2\) & \(E_4^3\) & \(E_4^4\) & \(E_4^5\) & \(\widetilde{C_{1}}\) & \(\widetilde{C_{2}}\) & \(\widetilde{H_{\mathcal{S}}}\) \\
    \hline
    \hline
    \(\widetilde{C_{1}}\) & \(0\) & \(0\) & \(0\) & \(0\) & \(0\) & \(1\) & \(1\) & \(0\) & \(0\) & \(0\) & \(0\) & \(1\) & \(0\) & \(0\) & \(0\) & \(0\) & \(-2\) & \(0\) & \(1\) \\
    \(\widetilde{C_{2}}\) & \(0\) & \(0\) & \(1\) & \(0\) & \(0\) & \(0\) & \(0\) & \(0\) & \(0\) & \(0\) & \(0\) & \(0\) & \(0\) & \(0\) & \(0\) & \(0\) & \(0\) & \(-2\) & \(1\) \\
    \(\widetilde{H_{\mathcal{S}}}\) & \(0\) & \(0\) & \(0\) & \(0\) & \(0\) & \(0\) & \(0\) & \(0\) & \(0\) & \(0\) & \(0\) & \(0\) & \(0\) & \(0\) & \(0\) & \(0\) & \(1\) & \(1\) & \(4\) \\
    \hline
  \end{tabular}.
\end{table}
Note that the intersection matrix is degenerate. We choose the following integral basis of the lattice \(L_{\mathcal{S}}\):
\begin{align*}
  \begin{pmatrix}
    [\widetilde{H_{\mathcal{S}}}]
  \end{pmatrix} =
  \begin{pmatrix}
    -1 & -2 & -3 & 3 & 5 & 4 & 3 & 2 & 1 & 6 & -2
  \end{pmatrix} \cdot \\
  \begin{pmatrix}
    [E_1^1 + E_1^5] & [E_1^2 + E_1^4] & [E_1^3] & [E_2^1] & [E_3^1 + E_4^1] & [E_3^2 + E_4^2] \\ [E_3^3 + E_4^3] & [E_3^4 + E_4^4] & [E_3^5 + E_4^5] & [\widetilde{C_{1}}] & [\widetilde{C_{2}}]
  \end{pmatrix}^T.
\end{align*}

Discriminant groups and discriminant forms of the lattices \(L_{\mathcal{S}}\) and \(H \oplus \Pic(X)\) are given by
\begin{gather*}
  G' = 
  \begin{pmatrix}
    0 & 0 & 0 & 0 & \frac{1}{2} & \frac{1}{2} & 0 & \frac{1}{2} & 0 & 0 & 0 \\
    0 & 0 & 0 & 0 & 0 & 0 & \frac{1}{2} & 0 & \frac{1}{2} & 0 & 0 \\
    0 & 0 & 0 & 0 & \frac{1}{2} & 0 & 0 & \frac{1}{2} & \frac{1}{2} & 0 & 0 \\
    0 & 0 & 0 & 0 & \frac{1}{2} & 0 & 0 & 0 & \frac{1}{2} & 0 & 0 \\
    0 & \frac{1}{2} & 0 & 0 & \frac{1}{2} & 0 & \frac{1}{2} & 0 & \frac{1}{2} & 0 & 0 \\
    \frac{1}{2} & \frac{1}{2} & 0 & 0 & \frac{1}{2} & 0 & \frac{1}{2} & 0 & \frac{1}{2} & 0 & 0 \\
    \frac{1}{2} & 0 & 0 & \frac{1}{2} & \frac{1}{4} & 0 & \frac{1}{4} & 0 & \frac{3}{4} & 0 & 0
  \end{pmatrix}, \;
  G'' = 
  \begin{pmatrix}
    0 & 0 & 0 & 0 & 0 & \frac{1}{2} & 0 & 0 & 0 & 0 & \frac{1}{2} \\
    0 & 0 & 0 & 0 & \frac{1}{2} & 0 & 0 & 0 & 0 & 0 & \frac{1}{2} \\
    0 & 0 & 0 & \frac{1}{2} & \frac{1}{2} & \frac{1}{2} & \frac{1}{2} & \frac{1}{2} & 0 & 0 & \frac{1}{2} \\
    0 & 0 & 0 & 0 & \frac{1}{2} & \frac{1}{2} & \frac{1}{2} & \frac{1}{2} & \frac{1}{2} & 0 & \frac{1}{2} \\
    0 & 0 & \frac{1}{2} & \frac{1}{2} & 0 & 0 & 0 & \frac{1}{2} & \frac{1}{2} & 0 & 0 \\
    0 & 0 & \frac{1}{2} & \frac{1}{2} & \frac{1}{2} & \frac{1}{2} & 0 & 0 & \frac{1}{2} & 0 & \frac{1}{2} \\
    0 & 0 & \frac{3}{4} & \frac{1}{4} & \frac{3}{4} & \frac{3}{4} & \frac{1}{4} & \frac{3}{4} & \frac{1}{4} & \frac{1}{2} & \frac{1}{4}
  \end{pmatrix};
\end{gather*}
\begin{gather*}
  B' = 
  \begin{pmatrix}
    0 & \frac{1}{2} & 0 & 0 & 0 & 0 & 0 \\
    \frac{1}{2} & 0 & 0 & 0 & 0 & 0 & 0 \\
    0 & 0 & 0 & \frac{1}{2} & 0 & 0 & 0 \\
    0 & 0 & \frac{1}{2} & 0 & 0 & 0 & 0 \\
    0 & 0 & 0 & 0 & 0 & \frac{1}{2} & 0 \\
    0 & 0 & 0 & 0 & \frac{1}{2} & 0 & 0 \\
    0 & 0 & 0 & 0 & 0 & 0 & \frac{3}{4}
  \end{pmatrix}, \;
  B'' = 
  \begin{pmatrix}
    0 & \frac{1}{2} & 0 & 0 & 0 & 0 & 0 \\
    \frac{1}{2} & 0 & 0 & 0 & 0 & 0 & 0 \\
    0 & 0 & 0 & \frac{1}{2} & 0 & 0 & 0 \\
    0 & 0 & \frac{1}{2} & 0 & 0 & 0 & 0 \\
    0 & 0 & 0 & 0 & 0 & \frac{1}{2} & 0 \\
    0 & 0 & 0 & 0 & \frac{1}{2} & 0 & 0 \\
    0 & 0 & 0 & 0 & 0 & 0 & \frac{1}{4}
  \end{pmatrix}; \;
  \begin{pmatrix}
    Q' \\ Q''
  \end{pmatrix}
  =
  \begin{pmatrix}
    0 & 0 & 0 & 0 & 0 & 0 & \frac{7}{4} \\ 
    0 & 0 & 0 & 0 & 0 & 0 & \frac{1}{4}
  \end{pmatrix}.
\end{gather*}

%%% Local Variables:
%%% mode: latex
%%% TeX-master: "../main"
%%% End:
