\subsection{Family \textnumero8.1}\label{subsection:08-01}

The pencil \(\mathcal{S}\) is defined by the equation
\begin{gather*}
  X Y^{3} + X^{2} Y Z + 3 X Y^{2} Z + 3 X Y Z^{2} + X Z^{3} + 3 X Y^{2} T + \\ 3 X Z^{2} T + 3 X Y T^{2} + 3 X Z T^{2} + Y Z T^{2} + X T^{3} = \lambda X Y Z T.
\end{gather*}
Members \(\mathcal{S}_{\lambda}\) of the pencil are irreducible for any \(\lambda \in \mathbb{P}^1\) except
\(\mathcal{S}_{\infty} = S_{(X)} + S_{(Y)} + S_{(Z)} + S_{(T)}\).
The base locus of the pencil \(\mathcal{S}\) consists of the following curves:
\begin{gather*}
  C_{1} = C_{(X, Y)}, \;
  C_{2} = C_{(X, Z)}, \;
  C_{3} = C_{(X, T)}, \;
  C_{4} = C_{(Y, Z + T)}, \;
  C_{5} = C_{(Z, Y + T)}, \;
  C_{6} = C_{(T, (Y + Z)^3 + X Y Z)}.
\end{gather*}
Their linear equivalence classes on the generic member \(\mathcal{S}_{\Bbbk}\) of the pencil satisfy the following relations:
\begin{gather*}
  \begin{pmatrix}
    [C_{1}] \\ [C_{2}] \\ [C_{6}] \\ [H_{\mathcal{S}}]
  \end{pmatrix} = 
  \begin{pmatrix}
    -2 & 0 & 3 \\
    -2 & 3 & 0 \\
    -3 & 3 & 3 \\
    -2 & 3 & 3
  \end{pmatrix} \cdot
  \begin{pmatrix}
    [C_{3}] \\ [C_{4}] \\ [C_{5}]
  \end{pmatrix}.
\end{gather*}

Put \(\mu (\mu - 1) = (\lambda + 4)^{-1}\). For a general choice of \(\lambda \in \mathbb{C}\) the surface \(\mathcal{S}_{\lambda}\) has the following singularities:
\begin{itemize}\setlength{\itemindent}{2cm}
\item[\(P_{1} = P_{(Y, Z, T)}\):] type \(\mathbb{A}_2\) with the quadratic term \(Y \cdot Z\);
\item[\(P_{2} = P_{(X, T, Y + Z)}\):] type \(\mathbb{A}_5\) with the quadratic term \((\mu X - (\mu - 1) T) \cdot ((\mu - 1) X - \mu T)\);
\item[\(P_{3} = P_{(Y, (\mu - 1) X - \mu T, Z + T)}\):] type \(\mathbb{A}_2\) with the quadratic term \(Y \cdot ((\mu - 1) X - \mu T)\);
\item[\(P_{4} = P_{(Y, \mu X - (\mu - 1) T, Z + T)}\):] type \(\mathbb{A}_2\) with the quadratic term \(Y \cdot (\mu X - (\mu - 1) T)\);
\item[\(P_{5} = P_{(Z, (\mu - 1) X - \mu T, Y + T)}\):] type \(\mathbb{A}_2\) with the quadratic term \(Z \cdot (\mu X - (\mu - 1) T)\);
\item[\(P_{6} = P_{(Z, \mu X - (\mu - 1) T, Y + T)}\):] type \(\mathbb{A}_2\) with the quadratic term \(Z \cdot ((\mu - 1) X - \mu T)\).
\end{itemize}

The only non-trivial Galois orbits on the lattice \(L_{\lambda}\) are 
\[
  E_2^1 + E_2^5, E_2^2 + E_2^4, E_3^1 + E_4^1, E_3^2 + E_4^2, E_5^1 + E_6^1, E_5^2 + E_6^2.
\]

The intersection matrix on the lattice \(L_{\lambda}\) is represented by
\begin{table}[H]
  \begin{tabular}{|c||cc|ccccc|cc|cc|cc|cc|ccc|}
    \hline
    \(\bullet\) & \(E_1^1\) & \(E_1^2\) & \(E_2^1\) & \(E_2^2\) & \(E_2^3\) & \(E_2^4\) & \(E_2^5\) & \(E_3^1\) & \(E_3^2\) & \(E_4^1\) & \(E_4^2\) & \(E_5^1\) & \(E_5^2\) & \(E_6^1\) & \(E_6^2\) & \(\widetilde{C_{3}}\) & \(\widetilde{C_{4}}\) & \(\widetilde{C_{5}}\) \\
    \hline
    \hline
    \(\widetilde{C_{3}}\) & \(0\) & \(0\) & \(0\) & \(0\) & \(1\) & \(0\) & \(0\) & \(0\) & \(0\) & \(0\) & \(0\) & \(0\) & \(0\) & \(0\) & \(0\) & \(-2\) & \(0\) & \(0\) \\
    \(\widetilde{C_{4}}\) & \(1\) & \(0\) & \(0\) & \(0\) & \(0\) & \(0\) & \(0\) & \(1\) & \(0\) & \(1\) & \(0\) & \(0\) & \(0\) & \(0\) & \(0\) & \(0\) & \(-2\) & \(0\) \\
    \(\widetilde{C_{5}}\) & \(0\) & \(1\) & \(0\) & \(0\) & \(0\) & \(0\) & \(0\) & \(0\) & \(0\) & \(0\) & \(0\) & \(1\) & \(0\) & \(1\) & \(0\) & \(0\) & \(0\) & \(-2\) \\
    \hline
  \end{tabular}.
\end{table}
Note that the intersection matrix is non-degenerate.

Discriminant groups and discriminant forms of the lattices \(L_{\mathcal{S}}\) and \(H \oplus \Pic(X)\) are given by
\begin{gather*}
  G' = 
  \begin{pmatrix}
    0 & 0 & 0 & 0 & 0 & 0 & \frac{1}{2} & \frac{1}{2} & 0 & 0 & 0 & 0 \\
    0 & 0 & 0 & 0 & 0 & 0 & \frac{1}{2} & \frac{1}{2} & \frac{1}{2} & 0 & 0 & 0 \\
    0 & 0 & 0 & 0 & 0 & \frac{1}{2} & 0 & 0 & \frac{1}{2} & 0 & 0 & 0 \\
    0 & 0 & 0 & 0 & 0 & \frac{1}{2} & \frac{1}{2} & 0 & \frac{1}{2} & 0 & 0 & 0 \\
    0 & 0 & 0 & \frac{1}{2} & 0 & 0 & 0 & 0 & 0 & 0 & 0 & 0 \\
    0 & 0 & \frac{1}{2} & 0 & 0 & 0 & 0 & 0 & 0 & 0 & 0 & 0 \\
    \frac{2}{3} & \frac{1}{3} & 0 & 0 & 0 & \frac{2}{3} & \frac{1}{3} & \frac{1}{3} & \frac{2}{3} & 0 & 0 & 0
  \end{pmatrix}, \;
  G'' = 
  \begin{pmatrix}
    0 & 0 & \frac{1}{2} & \frac{1}{2} & \frac{1}{2} & 0 & \frac{1}{2} & \frac{1}{2} & 0 & \frac{1}{2} \\
    0 & 0 & \frac{1}{2} & \frac{1}{2} & 0 & \frac{1}{2} & \frac{1}{2} & \frac{1}{2} & 0 & \frac{1}{2} \\
    0 & 0 & 0 & \frac{1}{2} & 0 & 0 & 0 & 0 & 0 & \frac{1}{2} \\
    0 & 0 & \frac{1}{2} & 0 & 0 & 0 & 0 & 0 & 0 & \frac{1}{2} \\
    0 & 0 & \frac{1}{2} & \frac{1}{2} & 0 & 0 & \frac{1}{2} & 0 & 0 & \frac{1}{2} \\
    0 & 0 & \frac{1}{2} & \frac{1}{2} & 0 & 0 & 0 & \frac{1}{2} & 0 & \frac{1}{2} \\
    0 & 0 & \frac{2}{3} & \frac{2}{3} & \frac{2}{3} & \frac{2}{3} & \frac{2}{3} & \frac{2}{3} & \frac{2}{3} & 0
  \end{pmatrix};
\end{gather*}
\begin{gather*}
  B' = 
  \begin{pmatrix}
    0 & \frac{1}{2} & 0 & 0 & 0 & 0 & 0 \\
    \frac{1}{2} & 0 & 0 & 0 & 0 & 0 & 0 \\
    0 & 0 & 0 & \frac{1}{2} & 0 & 0 & 0 \\
    0 & 0 & \frac{1}{2} & 0 & 0 & 0 & 0 \\
    0 & 0 & 0 & 0 & 0 & \frac{1}{2} & 0 \\
    0 & 0 & 0 & 0 & \frac{1}{2} & 0 & 0 \\
    0 & 0 & 0 & 0 & 0 & 0 & \frac{2}{3}
  \end{pmatrix}, \;
  B'' = 
  \begin{pmatrix}
    0 & \frac{1}{2} & 0 & 0 & 0 & 0 & 0 \\
    \frac{1}{2} & 0 & 0 & 0 & 0 & 0 & 0 \\
    0 & 0 & 0 & \frac{1}{2} & 0 & 0 & 0 \\
    0 & 0 & \frac{1}{2} & 0 & 0 & 0 & 0 \\
    0 & 0 & 0 & 0 & 0 & \frac{1}{2} & 0 \\
    0 & 0 & 0 & 0 & \frac{1}{2} & 0 & 0 \\
    0 & 0 & 0 & 0 & 0 & 0 & \frac{1}{3}
  \end{pmatrix}; \;
  \begin{pmatrix}
    Q' \\ Q''
  \end{pmatrix}
  =
  \begin{pmatrix}
    0 & 0 & 0 & 0 & 1 & 1 & \frac{2}{3} \\
    0 & 0 & 0 & 0 & 1 & 1 & \frac{4}{3}    
  \end{pmatrix}.
\end{gather*}

%%% Local Variables:
%%% mode: latex
%%% TeX-master: "../main"
%%% End:
