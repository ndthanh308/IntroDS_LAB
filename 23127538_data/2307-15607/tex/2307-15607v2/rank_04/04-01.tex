\subsection{Family \textnumero4.1}\label{subsection:04-01}

The pencil \(\mathcal{S}\) is defined by the equation
\begin{gather*}
  X^{2} Y Z + X Y^{2} Z + X Y Z^{2} + X Y^{2} T + Y^{2} Z T + X Z^{2} T + Y Z^{2} T + \\ X Y T^{2} + Y^{2} T^{2} + X Z T^{2} + 3 Y Z T^{2} + Z^{2} T^{2} + Y T^{3} + Z T^{3} = \lambda X Y Z T.
\end{gather*}
Members \(\mathcal{S}_{\lambda}\) of the pencil are irreducible for any \(\lambda \in \mathbb{P}^1\) except
\begin{gather*}
  \mathcal{S}_{\infty} = S_{(X)} + S_{(Y)} + S_{(Z)} + S_{(T)}, \;
  \mathcal{S}_{- 4} = S_{(X + T)} + S_{(X Y Z + (Y + Z + T) (Y Z + Y T + Z T))}.
\end{gather*}
The base locus of the pencil \(\mathcal{S}\) consists of the following curves:
\begin{gather*}
  C_{1} = C_{(X, T)}, \;
  C_{2} = C_{(Y, Z)}, \;
  C_{3} = C_{(Y, T)}, \;
  C_{4} = C_{(Z, T)}, \;
  C_{5} = C_{(Y, X + T)}, \;
  C_{6} = C_{(Y, Z + T)}, \\
  C_{7} = C_{(Z, X + T)}, \;
  C_{8} = C_{(Z, Y + T)}, \;
  C_{9} = C_{(X, Y + Z + T)}, \;
  C_{10} = C_{(T, X + Y + Z)}, \;
  C_{11} = C_{(X, Y Z + T (Y + Z))}.
\end{gather*}
Their linear equivalence classes on the generic member \(\mathcal{S}_{\Bbbk}\) of the pencil satisfy the following relations:
\begin{gather*}
  \begin{pmatrix}
    [C_{6}] \\ [C_{7}] \\ [C_{8}] \\ [C_{10}] \\ [C_{11}]
  \end{pmatrix} = 
  \begin{pmatrix}
    0 & -1 & -1 & 0 & -1 & 0 & 1 \\
    -2 & 0 & 0 & 0 & -1 & 0 & 1 \\
    2 & -1 & 0 & -1 & 1 & 0 & 0 \\
    -1 & 0 & -1 & -1 & 0 & 0 & 1 \\
    -1 & 0 & 0 & 0 & 0 & -1 & 1
  \end{pmatrix} \cdot
  \begin{pmatrix}
    [C_{1}] & [C_{2}] & [C_{3}] & [C_{4}] & [C_{5}] & [C_{9}] & [H_{\mathcal{S}}]
  \end{pmatrix}^T.
\end{gather*}

For a general choice of \(\lambda \in \mathbb{C}\) the surface \(\mathcal{S}_{\lambda}\) has the following singularities:
\begin{itemize}\setlength{\itemindent}{2cm}
\item[\(P_{1} = P_{(X, Y, T)}\):] type \(\mathbb{A}_2\) with the quadratic term \((X + T) \cdot (Y + T)\);
\item[\(P_{2} = P_{(X, Z, T)}\):] type \(\mathbb{A}_2\) with the quadratic term \((X + T) \cdot (Z + T)\);
\item[\(P_{3} = P_{(Y, Z, T)}\):] type \(\mathbb{A}_3\) with the quadratic term \(Y \cdot Z\);
\item[\(P_{4} = P_{(X, T, Y + Z)}\):] type \(\mathbb{A}_1\) with the quadratic term \((X + T) (X + Y + Z + T) - (\lambda + 4) X T\);
\item[\(P_{5} = P_{(Y, Z, X + T)}\):] type \(\mathbb{A}_1\) with the quadratic term \((X + T) (Y + Z) + (\lambda + 4) Y Z\).
\end{itemize}

Galois action on the lattice \(L_{\lambda}\) is trivial. The intersection matrix on \(L_{\lambda} = L_{\mathcal{S}}\) is represented by
\begin{table}[H]
  \begin{tabular}{|c||cc|cc|ccc|c|c|ccccccc|}
    \hline
    \(\bullet\) & \(E_1^1\) & \(E_1^2\) & \(E_2^1\) & \(E_2^2\) & \(E_3^1\) & \(E_3^2\) & \(E_3^3\) & \(E_4^1\) & \(E_5^1\) & \(\widetilde{C_{1}}\) & \(\widetilde{C_{2}}\) & \(\widetilde{C_{3}}\) & \(\widetilde{C_{4}}\) & \(\widetilde{C_{5}}\) & \(\widetilde{C_{9}}\) & \(\widetilde{H_{\mathcal{S}}}\) \\
    \hline
    \hline
    \(\widetilde{C_{1}}\) & \(1\) & \(0\) & \(1\) & \(0\) & \(0\) & \(0\) & \(0\) & \(1\) & \(0\) & \(-2\) & \(0\) & \(0\) & \(0\) & \(0\) & \(0\) & \(1\) \\
    \(\widetilde{C_{2}}\) & \(0\) & \(0\) & \(0\) & \(0\) & \(0\) & \(1\) & \(0\) & \(0\) & \(1\) & \(0\) & \(-2\) & \(0\) & \(0\) & \(0\) & \(0\) & \(1\) \\
    \(\widetilde{C_{3}}\) & \(0\) & \(1\) & \(0\) & \(0\) & \(1\) & \(0\) & \(0\) & \(0\) & \(0\) & \(0\) & \(0\) & \(-2\) & \(0\) & \(0\) & \(0\) & \(1\) \\
    \(\widetilde{C_{4}}\) & \(0\) & \(0\) & \(0\) & \(1\) & \(0\) & \(0\) & \(1\) & \(0\) & \(0\) & \(0\) & \(0\) & \(0\) & \(-2\) & \(0\) & \(0\) & \(1\) \\
    \(\widetilde{C_{5}}\) & \(1\) & \(0\) & \(0\) & \(0\) & \(0\) & \(0\) & \(0\) & \(0\) & \(1\) & \(0\) & \(0\) & \(0\) & \(0\) & \(-2\) & \(0\) & \(1\) \\
    \(\widetilde{C_{9}}\) & \(0\) & \(0\) & \(0\) & \(0\) & \(0\) & \(0\) & \(0\) & \(1\) & \(0\) & \(0\) & \(0\) & \(0\) & \(0\) & \(0\) & \(-2\) & \(1\) \\
    \(\widetilde{H_{\mathcal{S}}}\) & \(0\) & \(0\) & \(0\) & \(0\) & \(0\) & \(0\) & \(0\) & \(0\) & \(0\) & \(1\) & \(1\) & \(1\) & \(1\) & \(1\) & \(1\) & \(4\) \\
    \hline
  \end{tabular}.
\end{table}
Note that the intersection matrix is non-degenerate.

Discriminant groups and discriminant forms of the lattices \(L_{\mathcal{S}}\) and \(H \oplus \Pic(X)\) are given by
\begin{gather*}
  G' = 
  \begin{pmatrix}
    0 & \frac{1}{2} & 0 & 0 & \frac{1}{2} & 0 & 0 & 0 & 0 & 0 & \frac{1}{2} & 0 & 0 & \frac{1}{2} & 0 & 0 \\
    0 & \frac{1}{2} & 0 & 0 & 0 & 0 & \frac{1}{2} & \frac{1}{2} & \frac{1}{2} & 0 & \frac{1}{2} & 0 & 0 & \frac{1}{2} & 0 & \frac{1}{2} \\
    \frac{1}{2} & 0 & \frac{1}{2} & 0 & \frac{1}{2} & \frac{1}{2} & \frac{1}{2} & \frac{1}{2} & 0 & 0 & 0 & \frac{1}{2} & \frac{1}{2} & 0 & 0 & \frac{1}{2} \\
    \frac{1}{3} & \frac{5}{6} & 0 & \frac{5}{6} & \frac{2}{3} & 0 & \frac{1}{3} & \frac{5}{6} & \frac{5}{6} & \frac{1}{6} & 0 & \frac{1}{3} & \frac{2}{3} & \frac{2}{3} & \frac{1}{2} & \frac{1}{6}
  \end{pmatrix}, \;
  B' = 
  \begin{pmatrix}
    0 & \frac{1}{2} & \frac{1}{2} & \frac{1}{2} \\
    \frac{1}{2} & 0 & \frac{1}{2} & \frac{1}{2} \\
    \frac{1}{2} & \frac{1}{2} & 0 & \frac{1}{2} \\
    \frac{1}{2} & \frac{1}{2} & \frac{1}{2} & \frac{1}{3}
  \end{pmatrix}; \\
  G'' = 
  \begin{pmatrix}
    0 & 0 & \frac{1}{2} & 0 & 0 & 0 \\
    0 & 0 & 0 & \frac{1}{2} & 0 & 0 \\
    0 & 0 & 0 & 0 & \frac{1}{2} & 0 \\
    0 & 0 & -\frac{1}{3} & -\frac{1}{3} & -\frac{1}{3} & \frac{1}{6}
  \end{pmatrix}, \;
  B'' = 
  \begin{pmatrix}
    0 & \frac{1}{2} & \frac{1}{2} & \frac{1}{2} \\
    \frac{1}{2} & 0 & \frac{1}{2} & \frac{1}{2} \\
    \frac{1}{2} & \frac{1}{2} & 0 & \frac{1}{2} \\
    \frac{1}{2} & \frac{1}{2} & \frac{1}{2} & \frac{2}{3}
  \end{pmatrix}; \;
  \begin{pmatrix}
    Q' \\ Q''
  \end{pmatrix}
  =
  \begin{pmatrix}
    0 & 0 & 0 & \frac{4}{3} \\
    0 & 0 & 0 & \frac{2}{3}
  \end{pmatrix}.
\end{gather*}

%%% Local Variables:
%%% mode: latex
%%% TeX-master: "../main"
%%% End:
