\subsection{Family \textnumero4.2}\label{subsection:04-02}

The pencil \(\mathcal{S}\) is defined by the equation
\begin{gather*}
  X^{2} Y Z + X Y^{2} Z + X^{2} Z T + Y^{2} Z T + X Z^{2} T + Y Z^{2} T + 2 X Z T^{2} + 2 Y Z T^{2} + X T^{3} + Y T^{3} = \lambda X Y Z T.
\end{gather*}
Members \(\mathcal{S}_{\lambda}\) of the pencil are irreducible for any \(\lambda \in \mathbb{P}^1\) except
\begin{gather*}
  \mathcal{S}_{\infty} = S_{(X)} + S_{(Y)} + S_{(Z)} + S_{(T)}, \;
  \mathcal{S}_{- 2} = S_{(X + Y)} + S_{(Y Z T + X Z (Y + T) + (Z + T)^2 T)}.
\end{gather*}
The base locus of the pencil \(\mathcal{S}\) consists of the following curves:
\begin{gather*}
  C_{1} = C_{(X, Y)}, \;
  C_{2} = C_{(X, T)}, \;
  C_{3} = C_{(Y, T)}, \;
  C_{4} = C_{(Z, T)}, \;
  C_{5} = C_{(Z, X + Y)}, \\
  C_{6} = C_{(T, X + Y)}, \;
  C_{7} = C_{(X, Y Z + (Z + T)^2)}, \;
  C_{8} = C_{(Y, X Z + (Z + T)^2)}.
\end{gather*}
Their linear equivalence classes on the generic member \(\mathcal{S}_{\Bbbk}\) of the pencil satisfy the following relations:
\begin{gather*}
  \begin{pmatrix}
    [C_{3}] \\ [C_{5}] \\ [C_{6}] \\ [C_{7}] \\ [C_{8}]
  \end{pmatrix} = 
  \begin{pmatrix}
    2 & -1 & -4 & 1 \\
    0 & 0 & -3 & 1 \\
    -2 & 0 & 3 & 0 \\
    -1 & -1 & 0 & 1 \\
    -3 & 1 & 4 & 0
  \end{pmatrix} \cdot
  \begin{pmatrix}
    [C_{1}] \\ [C_{2}] \\ [C_{4}] \\ [H_{\mathcal{S}}]
  \end{pmatrix}.
\end{gather*}

Put \(\mu (\mu - 1) = (\lambda - 2)^{-1}\). For a general choice of \(\lambda \in \mathbb{C}\) the surface \(\mathcal{S}_{\lambda}\) has the following singularities:
\begin{itemize}\setlength{\itemindent}{2cm}
\item[\(P_{1} = P_{(X, Y, T)}\):] type \(\mathbb{A}_3\) with the quadratic term \(T \cdot (X + Y)\);
\item[\(P_{2} = P_{(X, Z, T)}\):] type \(\mathbb{A}_2\) with the quadratic term \(Z \cdot (X + T)\);
\item[\(P_{3} = P_{(Y, Z, T)}\):] type \(\mathbb{A}_2\) with the quadratic term \(Z \cdot (Y + T)\);
\item[\(P_{4} = P_{(X, Y, Z + T)}\):] type \(\mathbb{A}_3\) with the quadratic term \((\mu X - (\mu - 1) Y) \cdot ((\mu - 1) X - \mu Y)\);
\item[\(P_{5} = P_{(Z, T, X + Y)}\):] type \(\mathbb{A}_3\) with the quadratic term \(Z \cdot (X + Y - (\lambda + 2) T)\).
\end{itemize}

The only non-trivial Galois orbit on the lattice \(L_{\lambda}\) is \(E_4^1 + E_4^3\).

The intersection matrix on the lattice \(L_{\lambda}\) is represented by
\begin{table}[H]
  \begin{tabular}{|c||ccc|cc|cc|ccc|ccc|cccc|}
    \hline
    \(\bullet\) & \(E_1^1\) & \(E_1^2\) & \(E_1^3\) & \(E_2^1\) & \(E_2^2\) & \(E_3^1\) & \(E_3^2\) & \(E_4^1\) & \(E_4^2\) & \(E_4^3\) & \(E_5^1\) & \(E_5^2\) & \(E_5^3\) & \(\widetilde{C_{1}}\) & \(\widetilde{C_{2}}\) & \(\widetilde{C_{4}}\) & \(\widetilde{H_{\mathcal{S}}}\) \\
    \hline
    \hline
    \(\widetilde{C_{1}}\) & \(1\) & \(0\) & \(0\) & \(0\) & \(0\) & \(0\) & \(0\) & \(0\) & \(1\) & \(0\) & \(0\) & \(0\) & \(0\) & \(-2\) & \(0\) & \(0\) & \(1\) \\
    \(\widetilde{C_{2}}\) & \(0\) & \(0\) & \(1\) & \(1\) & \(0\) & \(0\) & \(0\) & \(0\) & \(0\) & \(0\) & \(0\) & \(0\) & \(0\) & \(0\) & \(-2\) & \(0\) & \(1\) \\
    \(\widetilde{C_{4}}\) & \(0\) & \(0\) & \(0\) & \(0\) & \(1\) & \(1\) & \(0\) & \(0\) & \(0\) & \(0\) & \(1\) & \(0\) & \(0\) & \(0\) & \(0\) & \(-2\) & \(1\) \\
    \(\widetilde{H_{\mathcal{S}}}\) & \(0\) & \(0\) & \(0\) & \(0\) & \(0\) & \(0\) & \(0\) & \(0\) & \(0\) & \(0\) & \(0\) & \(0\) & \(0\) & \(1\) & \(1\) & \(1\) & \(4\) \\
    \hline
  \end{tabular}.
\end{table}
Note that the intersection matrix is non-degenerate.

Discriminant groups and discriminant forms of the lattices \(L_{\mathcal{S}}\) and \(H \oplus \Pic(X)\) are given by
\begin{gather*}
  G' = 
  \begin{pmatrix}
    \frac{1}{4} & \frac{1}{2} & \frac{3}{4} & 0 & 0 & 0 & 0 & 0 & \frac{1}{2} & \frac{3}{4} & \frac{1}{2} & \frac{1}{4} & 0 & 0 & 0 & \frac{1}{4} \\
    \frac{3}{8} & \frac{1}{2} & \frac{5}{8} & 0 & \frac{1}{4} & 0 & \frac{1}{2} & \frac{5}{8} & \frac{1}{4} & \frac{7}{8} & \frac{1}{4} & \frac{5}{8} & \frac{1}{4} & \frac{3}{4} & \frac{1}{2} & \frac{7}{8}
  \end{pmatrix}, \\
  G'' = 
  \begin{pmatrix}
    0 & 0 & -\frac{1}{4} & -\frac{1}{4} & \frac{1}{4} & 0 \\
    0 & 0 & -\frac{3}{8} & 0 & 0 & \frac{1}{4}
  \end{pmatrix}; \;
  B' = 
  \begin{pmatrix}
    \frac{1}{4} & \frac{1}{4} \\
    \frac{1}{4} & \frac{3}{8}
  \end{pmatrix}, \;
  B'' = 
  \begin{pmatrix}
    \frac{3}{4} & \frac{3}{4} \\
    \frac{3}{4} & \frac{5}{8}
  \end{pmatrix}; \;
  \begin{pmatrix}
    Q' \\ Q''
  \end{pmatrix}
  =
  \begin{pmatrix}
    \frac{1}{4} & \frac{3}{8} \\
    \frac{7}{4} & \frac{13}{8}
  \end{pmatrix}.
\end{gather*}

%%% Local Variables:
%%% mode: latex
%%% TeX-master: "../main"
%%% End:
