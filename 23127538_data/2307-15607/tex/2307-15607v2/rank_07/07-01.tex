\subsection{Family \textnumero7.1}\label{subsection:07-01}

The pencil \(\mathcal{S}\) is defined by the equation
\begin{gather*}
  X^{3} Z + 3 X^{2} Y Z + 3 X Y^{2} Z + Y^{3} Z + X Y Z^{2} + 2 X^{2} Z T + 2 Y^{2} Z T + X Y T^{2} + X Z T^{2} + Y Z T^{2} = \lambda X Y Z T.
\end{gather*}
Members \(\mathcal{S}_{\lambda}\) of the pencil are irreducible for any \(\lambda \in \mathbb{P}^1\) except
\(\mathcal{S}_{\infty} = S_{(X)} + S_{(Y)} + S_{(Z)} + S_{(T)}\).
The base locus of the pencil \(\mathcal{S}\) consists of the following curves:
\begin{gather*}
  C_{1} = C_{(X, Y)}, \;
  C_{2} = C_{(X, Z)}, \;
  C_{3} = C_{(Y, Z)}, \;
  C_{4} = C_{(Z, T)}, \\
  C_{5} = C_{(X, Y + T)}, \;
  C_{6} = C_{(Y, X + T)}, \;
  C_{7} = C_{(T, (X + Y)^3 + X Y Z)}.
\end{gather*}
Their linear equivalence classes on the generic member \(\mathcal{S}_{\Bbbk}\) of the pencil satisfy the following relations:
\begin{gather*}
  \begin{pmatrix}
    [C_{2}] \\ [C_{3}] \\ [C_{7}] \\ [H_{\mathcal{S}}]
  \end{pmatrix} = 
  \begin{pmatrix}
    1 & -2 & 0 & 2 \\
    1 & -2 & 2 & 0 \\
    2 & -3 & 2 & 2 \\
    2 & -2 & 2 & 2
  \end{pmatrix} \cdot
  \begin{pmatrix}
    [C_{1}] \\ [C_{4}] \\ [C_{5}] \\ [C_{6}]
  \end{pmatrix}.
\end{gather*}

Put \(\mu (\mu - 1) = (\lambda + 2)^{-1}\). For a general choice of \(\lambda \in \mathbb{C}\) the surface \(\mathcal{S}_{\lambda}\) has the following singularities:
\begin{itemize}\setlength{\itemindent}{2cm}
\item[\(P_{1} = P_{(X, Y, Z)}\):] type \(\mathbb{A}_1\) with the quadratic term \(X Y + Z (X + Y)\);
\item[\(P_{2} = P_{(X, Y, T)}\):] type \(\mathbb{A}_3\) with the quadratic term \(X \cdot Y\);
\item[\(P_{3} = P_{(Z, T, X + Y)}\):] type \(\mathbb{A}_5\) with the quadratic term \((\mu Z - (\mu - 1) T) \cdot ((\mu - 1) Z - \mu T)\);
\item[\(P_{4} = P_{(X, Y + T, (\mu - 1) Z - \mu T)}\):] type \(\mathbb{A}_1\);
\item[\(P_{5} = P_{(X, Y + T, \mu Z - (\mu - 1) T)}\):] type \(\mathbb{A}_1\);
\item[\(P_{6} = P_{(Y, X + T, (\mu - 1) Z - \mu T)}\):] type \(\mathbb{A}_1\);
\item[\(P_{7} = P_{(Y, X + T, \mu Z - (\mu - 1) T)}\):] type \(\mathbb{A}_1\).
\end{itemize}

The only non-trivial Galois orbits on the lattice \(L_{\lambda}\) are
\(E_3^1 + E_3^5, E_3^2 + E_3^4, E_4^1 + E_5^1, E_6^1 + E_7^1\).

The intersection matrix on the lattice \(L_{\lambda}\) is represented by
\begin{table}[H]
  \begin{tabular}{|c||c|ccc|ccccc|c|c|c|c|cccc|}
    \hline
    \(\bullet\) & \(E_1^1\) & \(E_2^1\) & \(E_2^2\) & \(E_2^3\) & \(E_3^1\) & \(E_3^2\) & \(E_3^3\) & \(E_3^4\) & \(E_3^5\) & \(E_4^1\) & \(E_5^1\) & \(E_6^1\) & \(E_7^1\) & \(\widetilde{C_{1}}\) & \(\widetilde{C_{4}}\) & \(\widetilde{C_{5}}\) & \(\widetilde{C_{6}}\) \\
    \hline
    \hline
    \(\widetilde{C_{1}}\) & \(1\) & \(0\) & \(1\) & \(0\) & \(0\) & \(0\) & \(0\) & \(0\) & \(0\) & \(0\) & \(0\) & \(0\) & \(0\) & \(-2\) & \(0\) & \(0\) & \(0\) \\
    \(\widetilde{C_{4}}\) & \(0\) & \(0\) & \(0\) & \(0\) & \(0\) & \(0\) & \(1\) & \(0\) & \(0\) & \(0\) & \(0\) & \(0\) & \(0\) & \(0\) & \(-2\) & \(0\) & \(0\) \\
    \(\widetilde{C_{5}}\) & \(0\) & \(1\) & \(0\) & \(0\) & \(0\) & \(0\) & \(0\) & \(0\) & \(0\) & \(1\) & \(1\) & \(0\) & \(0\) & \(0\) & \(0\) & \(-2\) & \(0\) \\
    \(\widetilde{C_{6}}\) & \(0\) & \(0\) & \(0\) & \(1\) & \(0\) & \(0\) & \(0\) & \(0\) & \(0\) & \(0\) & \(0\) & \(1\) & \(1\) & \(0\) & \(0\) & \(0\) & \(-2\) \\
    \hline
  \end{tabular}.
\end{table}
Note that the intersection matrix is non-degenerate.

Discriminant groups and discriminant forms of the lattices \(L_{\mathcal{S}}\) and \(H \oplus \Pic(X)\) are given by
\begin{gather*}
  G' = 
  \begin{pmatrix}
    \frac{1}{2} & 0 & \frac{1}{2} & 0 & 0 & 0 & 0 & 0 & 0 & 0 & 0 & \frac{1}{2} & \frac{1}{2} \\
    0 & 0 & 0 & 0 & 0 & \frac{1}{2} & 0 & \frac{1}{2} & 0 & 0 & 0 & 0 & 0 \\
    0 & 0 & 0 & 0 & 0 & \frac{1}{2} & 0 & 0 & 0 & 0 & 0 & 0 & 0 \\
    \frac{1}{2} & 0 & \frac{1}{2} & 0 & \frac{1}{2} & 0 & 0 & 0 & 0 & 0 & 0 & \frac{1}{2} & \frac{1}{2} \\
    \frac{3}{4} & \frac{1}{4} & \frac{1}{4} & \frac{3}{4} & 0 & 0 & 0 & \frac{5}{8} & \frac{3}{8} & \frac{1}{2} & 0 & \frac{1}{4} & \frac{1}{4}
  \end{pmatrix}, \;
  G'' = 
  \begin{pmatrix}
    0 & 0 & \frac{1}{2} & \frac{1}{2} & \frac{1}{2} & 0 & \frac{1}{2} & 0 & 0 \\
    0 & 0 & \frac{1}{2} & \frac{1}{2} & 0 & \frac{1}{2} & \frac{1}{2} & 0 & 0 \\
    0 & 0 & \frac{1}{2} & 0 & 0 & 0 & \frac{1}{2} & 0 & 0 \\
    0 & 0 & \frac{1}{2} & \frac{1}{2} & 0 & 0 & 0 & 0 & 0 \\
    0 & 0 & \frac{1}{8} & \frac{1}{8} & \frac{1}{8} & \frac{1}{8} & \frac{1}{8} & \frac{3}{4} & \frac{7}{8}
  \end{pmatrix};
\end{gather*}
\begin{gather*}
  B' = 
  \begin{pmatrix}
    0 & \frac{1}{2} & 0 & 0 & 0 \\
    \frac{1}{2} & 0 & 0 & 0 & 0 \\
    0 & 0 & 0 & \frac{1}{2} & 0 \\
    0 & 0 & \frac{1}{2} & 0 & 0 \\
    0 & 0 & 0 & 0 & \frac{5}{8}
  \end{pmatrix}, \;
  B'' = 
  \begin{pmatrix}
    0 & \frac{1}{2} & 0 & 0 & 0 \\
    \frac{1}{2} & 0 & 0 & 0 & 0 \\
    0 & 0 & 0 & \frac{1}{2} & 0 \\
    0 & 0 & \frac{1}{2} & 0 & 0 \\
    0 & 0 & 0 & 0 & \frac{3}{8}
  \end{pmatrix}; \;
  \begin{pmatrix}
    Q' \\ Q''
  \end{pmatrix}
  =
  \begin{pmatrix}
    0 & 0 & 1 & 1 & \frac{13}{8} \\
    0 & 0 & 1 & 1 & \frac{3}{8}
  \end{pmatrix}.
\end{gather*}

%%% Local Variables:
%%% mode: latex
%%% TeX-master: "../main"
%%% End:
