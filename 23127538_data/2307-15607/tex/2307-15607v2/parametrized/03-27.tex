\subsection{Family \textnumero3.27}\label{subsection:03-27_parametrised}

The pencil \(\mathcal{S}(\alpha)\), where \(\alpha = (a_1, a_2, a_3) \in (\mathbb{C}^*)^3\), is defined by the equation
\begin{gather*}
  X^{2} Y Z + X Y^{2} Z + X Y Z^{2} + a_3 X Y T^{2} + a_2 X Z T^{2} + a_1 Y Z T^{2} = \lambda X Y Z T.
\end{gather*}
Note that \(\mathcal{S}(\alpha)_{\infty} = S_{(X)} + S_{(Y)} + S_{(Z)} + S_{(T)}\). The base locus of the pencil \(\mathcal{S}(\alpha)\) consists of the curves
\begin{gather*}
  C_{1} = C_{(X, Y)}, \;
  C_{2} = C_{(X, Z)}, \;
  C_{3} = C_{(X, T)}, \;
  C_{4} = C_{(Y, Z)}, \;
  C_{5} = C_{(Y, T)}, \;
  C_{6} = C_{(Z, T)}, \;
  C_{7} = C_{(T, X + Y + Z)}.
\end{gather*}
Their linear equivalence classes on the generic member \(\mathcal{S}(\alpha)_{\Bbbk}\) of the pencil satisfy the following relations:
\begin{gather*}
  \begin{pmatrix}
    [C_{2}] \\ [C_{4}] \\ [C_{7}] \\ [H_{\mathcal{S}(\alpha)}]
  \end{pmatrix} = 
  \begin{pmatrix}
    1 & 0 & 2 & -2 \\
    1 & 2 & 0 & -2 \\
    2 & 1 & 1 & -3 \\
    2 & 2 & 2 & -2
  \end{pmatrix} \cdot
  \begin{pmatrix}
    [C_{1}] \\ [C_{3}] \\ [C_{5}] \\ [C_{6}]
  \end{pmatrix}.
\end{gather*}

For a general choice of \(\lambda \in \mathbb{C}\) and \(\alpha \in (\mathbb{C}^*)^3\) the surface \(\mathcal{S}(\alpha)_{\lambda}\) has the following singularities:
\begin{itemize}\setlength{\itemindent}{2cm}
\item[\(P_{1} = P_{(X, Y, Z)}\):] type \(\mathbb{A}_1\) with the quadratic term \(a_3 X Y + a_2 X Z + a_1 Y Z\);
\item[\(P_{2} = P_{(X, Y, T)}\):] type \(\mathbb{A}_3\) with the quadratic term \(X \cdot Y\);
\item[\(P_{3} = P_{(X, Z, T)}\):] type \(\mathbb{A}_3\) with the quadratic term \(X \cdot Z\);
\item[\(P_{4} = P_{(Y, Z, T)}\):] type \(\mathbb{A}_3\) with the quadratic term \(Y \cdot Z\);
\item[\(P_{5} = P_{(X, T, Y + Z)}\):] type \(\mathbb{A}_1\) with the quadratic term \(X (X + Y + Z - \lambda T) + a_1 T^2\);
\item[\(P_{6} = P_{(Y, T, X + Z)}\):] type \(\mathbb{A}_1\) with the quadratic term \(Y (X + Y + Z - \lambda T) + a_2 T^2\);
\item[\(P_{7} = P_{(Z, T, X + Y)}\):] type \(\mathbb{A}_1\) with the quadratic term \(Z (X + Y + Z - \lambda T) + a_3 T^2\).
\end{itemize}

Galois action on the lattice \(L(\alpha)_{\lambda}\) is trivial. The intersection matrix on \(L(\alpha)_{\lambda} = L(\alpha)_{\mathcal{S}}\) is represented by
\begin{table}[H]
  \begin{tabular}{|c||c|ccc|ccc|ccc|c|c|c|cccc|}
    \hline
    \(\bullet\) & \(E_1^1\) & \(E_2^1\) & \(E_2^2\) & \(E_2^3\) & \(E_3^1\) & \(E_3^2\) & \(E_3^3\) & \(E_4^1\) & \(E_4^2\) & \(E_4^3\) & \(E_5^1\) & \(E_6^1\) & \(E_7^1\) & \(\widetilde{C_{1}}\) & \(\widetilde{C_{3}}\) & \(\widetilde{C_{5}}\) & \(\widetilde{C_{6}}\) \\
    \hline
    \hline
    \(\widetilde{C_{1}}\) & \(1\) & \(0\) & \(1\) & \(0\) & \(0\) & \(0\) & \(0\) & \(0\) & \(0\) & \(0\) & \(0\) & \(0\) & \(0\) & \(-2\) & \(0\) & \(0\) & \(0\) \\
    \(\widetilde{C_{3}}\) & \(0\) & \(1\) & \(0\) & \(0\) & \(1\) & \(0\) & \(0\) & \(0\) & \(0\) & \(0\) & \(1\) & \(0\) & \(0\) & \(0\) & \(-2\) & \(0\) & \(0\) \\
    \(\widetilde{C_{5}}\) & \(0\) & \(0\) & \(0\) & \(1\) & \(0\) & \(0\) & \(0\) & \(1\) & \(0\) & \(0\) & \(0\) & \(1\) & \(0\) & \(0\) & \(0\) & \(-2\) & \(0\) \\
    \(\widetilde{C_{6}}\) & \(0\) & \(0\) & \(0\) & \(0\) & \(0\) & \(0\) & \(1\) & \(0\) & \(0\) & \(1\) & \(0\) & \(0\) & \(1\) & \(0\) & \(0\) & \(0\) & \(-2\) \\
    \hline
  \end{tabular}.
\end{table}
Note that the intersection matrix is non-degenerate.

Discriminant groups and discriminant forms of the lattices \(L(\alpha)_{\mathcal{S}}\) and \(H \oplus \Pic(X)\) are given by
\begin{gather*}
  G' =
  \begin{pmatrix}
    0 & 0 & 0 & 0 & \frac{1}{2} & 0 & \frac{1}{2} & 0 & 0 & 0 & \frac{1}{2} & 0 & \frac{1}{2} & 0 & 0 & 0 & 0 \\
    0 & \frac{1}{2} & 0 & \frac{1}{2} & 0 & 0 & 0 & 0 & 0 & 0 & \frac{1}{2} & \frac{1}{2} & 0 & 0 & 0 & 0 & 0 \\
    \frac{1}{2} & \frac{3}{4} & \frac{1}{2} & \frac{1}{4} & \frac{1}{4} & \frac{1}{2} & \frac{3}{4} & \frac{1}{4} & \frac{1}{2} & \frac{3}{4} & 0 & \frac{1}{2} & \frac{1}{2} & 0 & 0 & 0 & 0
  \end{pmatrix}, \\
  G'' =
  \begin{pmatrix}
    0 & 0 & \frac{1}{2} & 0 & 0 \\
    0 & 0 & 0 & \frac{1}{2} & 0 \\
    0 & 0 & -\frac{1}{4} & -\frac{1}{4} & \frac{1}{4}
  \end{pmatrix}; \;
  B' = 
  \begin{pmatrix}
    0 & \frac{1}{2} & 0 \\
    \frac{1}{2} & 0 & 0 \\
    0 & 0 & \frac{1}{4}
  \end{pmatrix}, \;
  B'' = 
  \begin{pmatrix}
    0 & \frac{1}{2} & 0 \\
    \frac{1}{2} & 0 & 0 \\
    0 & 0 & \frac{3}{4}
  \end{pmatrix}; \;
  \begin{pmatrix}
    Q' \\ Q''
  \end{pmatrix}
  =
  \begin{pmatrix}
    0 & 0 & \frac{1}{4} \\
    0 & 0 & \frac{7}{4}    
  \end{pmatrix}.
\end{gather*}

%%% Local Variables:
%%% mode: latex
%%% TeX-master: "../main"
%%% End:
