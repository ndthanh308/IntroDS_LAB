\subsection{Parametrized toric Landau--Ginzburg models for \texorpdfstring{\(S \times \mathbb{P}^1\)}{S x P\textasciicircum1} with \texorpdfstring{\((-K_S)^2 > 2\)}{(-K\_S)\textasciicircum 2 > 2}}\label{subsection:dPxP1_list}

In this subsection we mutate the obtained parametrized Landau--Ginzburg models to make them coincide with the standard Landau--Ginzburg models after the parameter specialization. Note that here we follow the numeration of Minkowski polynomials in~\cite{akhtar2012minkowski}, it differs from the similar numeration used in~\cite{cheltsov2018katzarkov} by adding one to the corresponding number of a Minkowski polynomial.

\subsubsection{Family \textnumero2.34}\label{subsubsection:dPxP1_02-34}

The toric Landau--Ginzburg model for this family is given by the Laurent polynomial \(x + y + a_1 x^{-1} y^{-1} + z + a_2 z^{-1}\). It is a Minkowski polynomial \textnumero5 (see~\cite[Appendix~B: bucket~10]{akhtar2012minkowski}). After the change of variables \((x, y, z) \mapsto (z, y, x)\) we obtain the Laurent polynomial \(x + y + z + a_2 x^{-1} + a_1 y^{-1} z^{-1}\).

\subsubsection{Family \textnumero3.27}\label{subsubsection:dPxP1_03-27}

The toric Landau--Ginzburg model for this family is given by the Laurent polynomial \(x + a_1 x^{-1} + y + a_2 y^{-1} + z + a_3 z^{-1}\). It is a Minkowski polynomial \textnumero31 (see~\cite[Appendix~B: bucket~45]{akhtar2012minkowski}).

\subsubsection{Family \textnumero3.28}\label{subsubsection:dPxP1_03-28}

The toric Landau--Ginzburg model for this family is given by the Laurent polynomial \(x + y + a_1 x^{-1} y^{-1} + a_1 a_2 x^{-1} + z + a_3 z^{-1}\). It is a Minkowski polynomial \textnumero30 (see~\cite[Appendix~B: bucket~28]{akhtar2012minkowski}). After the change of variables
\((x, y, z) \mapsto (x^{-1}, z, y)\) we obtain the Laurent polynomial \((a_1 a_2) x + y + z + a_1 x z^{-1} + a_3 y^{-1} + x^{-1}\).

\subsubsection{Family \textnumero4.10}\label{subsubsection:dPxP1_04-10}

The toric Landau--Ginzburg model for this family is given by the Laurent polynomial
\(x + y + a_1 x^{-1} y^{-1} + a_1 a_2 x^{-1} + a_1 a_3 y^{-1} + z + a_4 z^{-1}\).
It is a Minkowski polynomial \textnumero85 (see~\cite[Appendix~B: bucket~48]{akhtar2012minkowski}). After the change of variables
\((x, y, z) \mapsto (y^{-1}, x, z)\) we obtain the Laurent polynomial
\[
  x + \left(a_1 a_2\right) y + z + a_1 x^{-1} y + a_4 z^{-1} + y^{-1} + \left(a_1 a_3\right) x^{-1}.
\]

\subsubsection{Family \textnumero5.3}\label{subsubsection:dPxP1_05-03}

The toric Landau--Ginzburg model for this family is given by the Laurent polynomial
\(x + y + a_1 x^{-1} y^{-1} + a_1 a_2 x^{-1} + a_1 a_3 y^{-1} + a_4 x y + z + a_5^{-1}\).
It is a Minkowski polynomial \textnumero219 (see~\cite[Appendix~B: bucket~76]{akhtar2012minkowski}). After the change of variables
\((x, y, z) \mapsto (y^{-1}, z, x)\) we obtain the Laurent polynomial
\(x + (a_1 a_2) y + z + a_1 y z^{-1} + a_4 y^{-1} z + (a_1 a_3) z^{-1} + y^{-1} + a_5 x^{-1}\).

\subsubsection{Family \textnumero6.1}\label{subsubsection:dPxP1_06-01}

The toric Landau--Ginzburg model for this family is given by the Laurent polynomial
\[
  x + (a_1 a_2 a_4 a_5 + 1) y + a_1 x^{-1} y^{-1} + a_1 (a_2 + a_5) x^{-1} +
  a_1 a_3 y^{-1} + a_4 x y + a_1 a_2 a_5 x^{-1} y + z + a_6 z^{-1}.
\]
It is a Minkowski polynomial \textnumero357 (see~\cite[Appendix~B: bucket~107]{akhtar2012minkowski}). After the change of variables
\((x, y, z) \mapsto (y, x^{-1}, z)\)
we obtain the Laurent polynomial
\[
  \left(a_1 a_3\right) x + y + z + a_1 x y^{-1} + a_4 x^{-1} y + a_6 z^{-1} + \left(a_1 a_2 + a_1 a_5\right) y^{-1} + \left(a_1 a_2 a_4 a_5 + 1\right) x^{-1} + \left(a_1 a_2 a_5\right) x^{-1} y^{-1}.
\]

We apply the mutation
\[
  (x, y, z) \mapsto (x, y, z)^{(M, f, N)}, \;
  M =
  \begin{pmatrix}
    -1 & 0 & 1 \\
    0 & 1 & 0 \\
    0 & 1 & -1
  \end{pmatrix}, \;
  f = x (a_4 y + 1) (a_1 a_2 a_5 + y) + y^2, \;
  N =
  \begin{pmatrix}
    -1 & 1 & -1 \\
    0 & 1 & 0 \\
    -1 & -1 & 0    
  \end{pmatrix},
\]
to obtain a Minkowski polynomial \textnumero1353 from~\cite[Appendix~B: bucket~107]{akhtar2012minkowski}:
\begin{gather*}
  x + y + \left(a_1 a_3\right) z + \left(a_1 a_3 a_4\right) x^{-1} y + a_1 y^{-1} z + \left(a_1 a_2 + a_1 a_5\right) y^{-1} + \left(a_1^{2} a_2 a_3 a_4 a_5 + a_1 a_3 + a_1 a_4 + a_6\right) x^{-1} + \\ \left(a_4 a_6\right) x^{-2} y z^{-1} + \left(a_1^{2} a_2 a_3 a_5 + a_1^{2} a_2 a_4 a_5 + a_1\right) x^{-1} y^{-1} + \left(a_1 a_2 a_4 a_5 a_6 + a_6\right) x^{-2} z^{-1} + \left(a_1^{2} a_2 a_5\right) x^{-1} y^{-2} + \\ \left(a_1 a_2 a_5 a_6\right) x^{-2} y^{-1} z^{-1}.
\end{gather*}

We apply the mutation
\[
  (x, y, z) \mapsto (x, y, z)^{(M, f, N)}, \;
  M =
  \begin{pmatrix}
    0 & -1 & 0 \\
    1 & 1 & 0 \\
    -1 & 0 & 1
  \end{pmatrix}, \;
  f = (a_4 x y + 1) (a_1 a_2 a_5 + x y), \;
  N =
  \begin{pmatrix}
    1 & 1 & 0 \\
    -1 & 0 & 0 \\
    -1 & 0 & -1
  \end{pmatrix},
\]
to obtain a Minkowski polynomial \textnumero1231 from~\cite[Appendix~B: bucket~107]{akhtar2012minkowski}:
\begin{gather*}
  x + y + a_6 z + \left(a_1 a_3 a_4\right) x^{-1} y + \left(a_1 a_2 + a_1 a_5\right) y^{-1} + \left(a_1^{2} a_2 a_3 a_4 a_5 + a_1 a_3 + a_1 a_4 + a_6\right) x^{-1} + \\ \left(a_1 a_3 a_4\right) x^{-2} y z^{-1} + \left(a_1^{2} a_2 a_3 a_5 + a_1^{2} a_2 a_4 a_5 + a_1\right) x^{-1} y^{-1} + \left(a_1^{2} a_2 a_3 a_4 a_5 + a_1 a_3 + a_1 a_4\right) x^{-2} z^{-1} + \\ \left(a_1^{2} a_2 a_5\right) x^{-1} y^{-2} + \left(a_1^{2} a_2 a_3 a_5 + a_1^{2} a_2 a_4 a_5 + a_1\right) x^{-2} y^{-1} z^{-1} + \left(a_1^{2} a_2 a_5\right) x^{-2} y^{-2} z^{-1}.
\end{gather*}

We apply the mutation
\[
  (x, y, z) \mapsto (x, y, z)^{(M, f, N)}, \;
  M =
  \begin{pmatrix}
    0 & -1 & 1 \\
    1 & 0 & 0 \\
    -1 & 0 & -1
  \end{pmatrix}, \;
  f = x y + 1, \;
  N =
  \begin{pmatrix}
    0 & 1 & 0 \\
    -1 & -1 & -1 \\
    0 & -1 & -1    
  \end{pmatrix},
\]
to obtain a Minkowski polynomial \textnumero284 from~\cite[Appendix~B: bucket~107]{akhtar2012minkowski}:
\begin{gather*}
  x + y + a_6 z + \left(a_1 a_3 a_4\right) x^{-1} y + z^{-1} + \left(a_1 a_2 + a_1 a_5\right) y^{-1} + \left(a_1^{2} a_2 a_3 a_4 a_5 + a_1 a_3 + a_1 a_4\right) x^{-1} + \\ \left(a_1^{2} a_2 a_3 a_5 + a_1^{2} a_2 a_4 a_5 + a_1\right) x^{-1} y^{-1} + \left(a_1^{2} a_2 a_5\right) x^{-1} y^{-2}.
\end{gather*}

Finally, after the change of variables
\((x, y, z) \mapsto (x, y^-1, z)\)
we obtain the Laurent polynomial
\begin{gather*}
  x + (a_1 a_2 + a_1 a_5) y + (a_1^{2} a_2 a_5) x^{-1} y^{2} + a_6 z + (a_1^{2} a_2 a_3 a_5 + a_1^{2} a_2 a_4 a_5 + a_1) x^{-1} y + \\ z^{-1} + y^{-1} + (a_1^{2} a_2 a_3 a_4 a_5 + a_1 a_3 + a_1 a_4) x^{-1} + (a_1 a_3 a_4) x^{-1} y^{-1}.
\end{gather*}

\subsubsection{Family \textnumero7.1}\label{subsubsection:dPxP1_07-01}

The toric Landau--Ginzburg model for this family is given by the Laurent polynomial
\begin{gather*}
  x + (a_1 a_2 a_4 a_5 + 1) y + a_1 (a_1 a_3 a_6 (a_2 + a_5) + 1) x^{-1} y^{-1} + \\ a_1 (a_1 a_2 a_3 a_5 a_6 + a_2 + a_5) x^{-1} + a_1 (a_3 + a_6) y^{-1} + a_4 x y + a_1 a_2 a_5 x^{-1} y + a_1^2 a_3 a_6 x^{-1} y^{-2} + z + a_7 z^{-1}.
\end{gather*}
It is a Minkowski polynomial \textnumero506 (see~\cite[Appendix~B: bucket~136]{akhtar2012minkowski}). After the change of variables
\((x, y, z) \mapsto (y^{-1}, x^{-1} y, z)\)
we obtain the Laurent polynomial
\begin{gather*}
  \left(a_1^{2} a_3 a_6\right) x^{2} y^{-1} + \left(a_1^{2} a_2 a_3 a_6 + a_1^{2} a_3 a_5 a_6 + a_1\right) x + \left(a_1^{2} a_2 a_3 a_5 a_6 + a_1 a_2 + a_1 a_5\right) y + \\ \left(a_1 a_2 a_5\right) x^{-1} y^{2} + z + \left(a_1 a_3 + a_1 a_6\right) x y^{-1} + \left(a_1 a_2 a_4 a_5 + 1\right) x^{-1} y + a_7 z^{-1} + y^{-1} + a_4 x^{-1}
\end{gather*}

\subsubsection{Family \textnumero8.1}\label{subsubsection:dPxP1_08-01}

The toric Landau--Ginzburg model for this family is given by the Laurent polynomial
\begin{gather*}
  (a_1 a_4 a_7 (a_3 + a_6) + 1) x + (a_1 a_2 a_5 (a_4 + a_7) + 1) y + 
  a_1 (a_1 a_3 a_6 (a_2 + a_5) + 1) x^{-1} y^{-1} + \\
  a_1 (a_1 a_2 a_3 a_5 a_6 + a_2 + a_5) x^{-1} +
  a_1 (a_1 a_3 a_4 a_6 a_7 + a_3 + a_6) y^{-1} + 
  (a_1 a_2 a_4 a_5 a_7 + a_4 + a_7) x y + \\
  a_1 a_2 a_5 x^{-1} y +
  a_1^2 a_3 a_6 x^{-1} y^{-2} +
  a_4 a_7 x^2 y + z + a_8 z^{-1}.
\end{gather*}
It is a Minkowski polynomial \textnumero769 (see~\cite[Appendix~B: bucket~155]{akhtar2012minkowski}). After the change of variables
\((x, y, z) \mapsto (y z^{-1}, z, x)\)
we obtain the Laurent polynomial
\begin{gather*}
  (a_4 a_7) y^{2} z^{-1} + x + (a_1 a_2 a_4 a_5 a_7 + a_4 + a_7) y + (a_1 a_2 a_4 a_5 + a_1 a_2 a_5 a_7 + 1) z + (a_1 a_2 a_5) y^{-1} z^{2} + \\ (a_1 a_3 a_4 a_7 + a_1 a_4 a_6 a_7 + 1) y z^{-1} + (a_1^{2} a_2 a_3 a_5 a_6 + a_1 a_2 + a_1 a_5) y^{-1} z + (a_1^{2} a_3 a_4 a_6 a_7 + a_1 a_3 + a_1 a_6) z^{-1} + \\ (a_1^{2} a_2 a_3 a_6 + a_1^{2} a_3 a_5 a_6 + a_1) y^{-1} + a_8 x^{-1} + (a_1^{2} a_3 a_6) y^{-1} z^{-1}
\end{gather*}

%%% Local Variables:
%%% mode: latex
%%% TeX-master: "../main"
%%% End:
