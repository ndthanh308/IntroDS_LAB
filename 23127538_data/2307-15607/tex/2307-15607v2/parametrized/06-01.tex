\subsection{Family \textnumero6.1}\label{subsection:06-01_parametrised}

The pencil \(\mathcal{S}(\alpha)\), where \(\alpha = (a_1, \ldots, a_6) \in (\mathbb{C}^*)^6\), is defined by the equation
\begin{gather*}
  X^{2} Y Z + (a_1 a_2 + a_1 a_5) X Y^{2} Z + (a_1^{2} a_2 a_5) Y^{3} Z + a_6 X Y Z^{2} + (a_1^{2} a_2 a_3 a_5 + a_1^{2} a_2 a_4 a_5 + a_1) Y^{2} Z T + \\ X Y T^{2} + X Z T^{2} + (a_1^{2} a_2 a_3 a_4 a_5 + a_1 a_3 + a_1 a_4) Y Z T^{2} + (a_1 a_3 a_4) Z T^{3} = \lambda X Y Z T.
\end{gather*}
Note that \(\mathcal{S}(\alpha)_{\infty} = S_{(X)} + S_{(Y)} + S_{(Z)} + S_{(T)}\). The base locus of the pencil \(\mathcal{S}(\alpha)\) consists of the curves
\begin{gather*}
  C_{1} = C_{(X, Z)}, \;
  C_{2} = C_{(Y, Z)}, \;
  C_{3} = C_{(Y, T)}, \;
  C_{4} = C_{(Z, T)}, \;
  C_{5} = C_{(X, Y + a_3 T)}, \;
  C_{6} = C_{(X, Y + a_4 T)}, \\
  C_{7} = C_{(X, a_1 a_2 a_5 Y + T)}, \;
  C_{8} = C_{(Y, X + a_1 a_3 a_4 T)}, \;
  C_{9} = C_{(T, (X + a_1 a_2 Y) (X + a_1 a_5 Y) + a_6 X Z)}.
\end{gather*}
Their linear equivalence classes on the generic member \(\mathcal{S}(\alpha)_{\Bbbk}\) of the pencil satisfy the following relations:
\begin{gather*}
  \begin{pmatrix}
    [C_{2}] \\ [C_{7}] \\ [C_{8}] \\ [C_{9}]
  \end{pmatrix} = 
  \begin{pmatrix}
    -1 & 0 & -2 & 0 & 0 & 1 \\
    -1 & 0 & 0 & -1 & -1 & 1 \\
    1 & -2 & 2 & 0 & 0 & 0 \\
    0 & -1 & -1 & 0 & 0 & 1
  \end{pmatrix} \cdot
  \begin{pmatrix}
    [C_{1}] & [C_{3}] & [C_{4}] & [C_{5}] & [C_{6}] & [H_{\mathcal{S}(\alpha)}]
  \end{pmatrix}^T.
\end{gather*}

For a general choice of \(\lambda \in \mathbb{C}\) and \(\alpha \in (\mathbb{C}^*)^6\) the surface \(\mathcal{S}(\alpha)_{\lambda}\) has the following singularities:
\begin{itemize}\setlength{\itemindent}{2cm}
\item[\(P_{1} = P_{(X, Y, T)}\):] type \(\mathbb{A}_2\) with the quadratic term \(X \cdot Y\);
\item[\(P_{2} = P_{(Y, Z, T)}\):] type \(\mathbb{A}_3\) with the quadratic term \(Y \cdot Z\);
\item[\(P_{3} = P_{(Y, T, X + a_6 Z)}\):] type \(\mathbb{A}_1\) with the quadratic term \(Y (X + a_1 a_2 Y + a_6 Z - \lambda T) + (a_1 a_5 Y^2 + T^2)\);
\item[\(P_{4} = P_{(Z, T, X + a_1 a_2 Y)}\):] type \(\mathbb{A}_1\) with the quadratic term
  \[
    a_2 a_6 Z^2 - (a_1 a_2 a_5 (a_3 + a_4) + a_2 \lambda + 1) Z T + a_2 T^2 + (a_2 - a_5) Z (X + a_1 a_2 Y);
  \]
\item[\(P_{5} = P_{(Z, T, X + a_1 a_5 Y)}\):] type \(\mathbb{A}_1\) with the quadratic term
  \[
    a_5 a_6 Z^2 - (a_1 a_2 a_5 (a_3 + a_4) + a_5 \lambda + 1) Z T + a_5 T^2 - (a_2 - a_5) Z (X + a_1 a_5 Y).
  \]
\end{itemize}

Galois action on the lattice \(L(\alpha)_{\lambda}\) is trivial. The intersection matrix on \(L(\alpha)_{\lambda} = L(\alpha)_{\mathcal{S}}\) is represented by
\begin{table}[H]
  \begin{tabular}{|c||cc|ccc|c|c|c|cccccc|}
    \hline
    \(\bullet\) & \(E_1^1\) & \(E_1^2\) & \(E_2^1\) & \(E_2^2\) & \(E_2^3\) & \(E_3^1\) & \(E_4^1\) & \(E_5^1\) & \(\widetilde{C_{1}}\) & \(\widetilde{C_{3}}\) & \(\widetilde{C_{4}}\) & \(\widetilde{C_{5}}\) & \(\widetilde{C_{6}}\) & \(\widetilde{H_{\mathcal{S}}}\) \\
    \hline
    \hline
    \(\widetilde{C_{1}}\) & \(0\) & \(0\) & \(0\) & \(0\) & \(0\) & \(0\) & \(0\) & \(0\) & \(-2\) & \(0\) & \(1\) & \(1\) & \(1\) & \(1\) \\
    \(\widetilde{C_{3}}\) & \(1\) & \(0\) & \(1\) & \(0\) & \(0\) & \(1\) & \(0\) & \(0\) & \(0\) & \(-2\) & \(0\) & \(0\) & \(0\) & \(1\) \\
    \(\widetilde{C_{4}}\) & \(0\) & \(0\) & \(0\) & \(0\) & \(1\) & \(0\) & \(1\) & \(1\) & \(1\) & \(0\) & \(-2\) & \(0\) & \(0\) & \(1\) \\
    \(\widetilde{C_{5}}\) & \(0\) & \(1\) & \(0\) & \(0\) & \(0\) & \(0\) & \(0\) & \(0\) & \(1\) & \(0\) & \(0\) & \(-2\) & \(0\) & \(1\) \\
    \(\widetilde{C_{6}}\) & \(0\) & \(1\) & \(0\) & \(0\) & \(0\) & \(0\) & \(0\) & \(0\) & \(1\) & \(0\) & \(0\) & \(0\) & \(-2\) & \(1\) \\
    \(\widetilde{H_{\mathcal{S}}}\) & \(0\) & \(0\) & \(0\) & \(0\) & \(0\) & \(0\) & \(0\) & \(0\) & \(1\) & \(1\) & \(1\) & \(1\) & \(1\) & \(4\) \\
    \hline
  \end{tabular}
\end{table}
Note that the intersection matrix is non-degenerate.

Discriminant groups and discriminant forms of the lattices \(L(\alpha)_{\mathcal{S}}\) and \(H \oplus \Pic(X)\) are given by
\begin{gather*}
  G' =
  \begin{pmatrix}
    0 & 0 & \frac{1}{2} & 0 & \frac{1}{2} & \frac{1}{2} & \frac{1}{2} & 0 & 0 & 0 & 0 & 0 & 0 & 0 \\
    0 & 0 & \frac{1}{2} & 0 & \frac{1}{2} & \frac{1}{2} & 0 & \frac{1}{2} & 0 & 0 & 0 & 0 & 0 & 0 \\
    0 & 0 & 0 & 0 & 0 & 0 & 0 & 0 & 0 & 0 & 0 & \frac{1}{2} & \frac{1}{2} & 0 \\
    \frac{1}{2} & 0 & 0 & 0 & 0 & 0 & 0 & 0 & \frac{1}{2} & 0 & 0 & \frac{1}{2} & 0 & \frac{1}{2} \\
    \frac{4}{5} & \frac{2}{5} & \frac{4}{5} & \frac{2}{5} & 0 & \frac{3}{5} & \frac{4}{5} & \frac{4}{5} & \frac{2}{5} & \frac{1}{5} & \frac{3}{5} & 0 & 0 & \frac{1}{5}
  \end{pmatrix}, \;
  G'' =
  \begin{pmatrix}
    0 & 0 & 0 & \frac{1}{2} & 0 & 0 & 0 & \frac{1}{2} \\
    0 & 0 & \frac{1}{2} & 0 & 0 & 0 & 0 & \frac{1}{2} \\
    0 & 0 & \frac{1}{2} & \frac{1}{2} & \frac{1}{2} & 0 & 0 & \frac{1}{2} \\
    0 & 0 & \frac{1}{2} & \frac{1}{2} & 0 & \frac{1}{2} & 0 & \frac{1}{2} \\
    0 & 0 & \frac{2}{5} & \frac{2}{5} & \frac{2}{5} & \frac{2}{5} & \frac{1}{5} & \frac{1}{5}
  \end{pmatrix};
\end{gather*}
\begin{gather*}
  B' = 
  \begin{pmatrix}
    0 & \frac{1}{2} & 0 & 0 & 0 \\
    \frac{1}{2} & 0 & 0 & 0 & 0 \\
    0 & 0 & 0 & \frac{1}{2} & 0 \\
    0 & 0 & \frac{1}{2} & 0 & 0 \\
    0 & 0 & 0 & 0 & \frac{3}{5}
  \end{pmatrix}, \;
  B'' = 
  \begin{pmatrix}
    0 & \frac{1}{2} & 0 & 0 & 0 \\
    \frac{1}{2} & 0 & 0 & 0 & 0 \\
    0 & 0 & 0 & \frac{1}{2} & 0 \\
    0 & 0 & \frac{1}{2} & 0 & 0 \\
    0 & 0 & 0 & 0 & \frac{2}{5}
  \end{pmatrix}; \;
  \begin{pmatrix}
    Q' \\ Q''
  \end{pmatrix}
  =
  \begin{pmatrix}
    0 & 0 & 1 & 1 & \frac{8}{5} \\
    0 & 0 & 1 & 1 & \frac{2}{5}
  \end{pmatrix}.
\end{gather*}

%%% Local Variables:
%%% mode: latex
%%% TeX-master: "../main"
%%% End:
