\subsection{Family \textnumero5.3}\label{subsection:05-03_parametrised}

The pencil \(\mathcal{S}(\alpha)\), where \(\alpha = (a_1, \ldots, a_5) \in (\mathbb{C}^*)^5\), is defined by the equation
\begin{gather*}
  X^{2} Y Z + (a_1 a_2) X Y^{2} Z + X Y Z^{2} + a_1 X Y^{2} T + a_4 X Z^{2} T + (a_1 a_3) X Y T^{2} + X Z T^{2} + a_5 Y Z T^{2} = \lambda X Y Z T.
\end{gather*}
Note that \(\mathcal{S}(\alpha)_{\infty} = S_{(X)} + S_{(Y)} + S_{(Z)} + S_{(T)}\). The base locus of the pencil \(\mathcal{S}(\alpha)\) consists of the curves
\begin{gather*}
  C_{1} = C_{(X, Y)}, \;
  C_{2} = C_{(X, Z)}, \;
  C_{3} = C_{(X, T)}, \;
  C_{4} = C_{(Y, Z)}, \;
  C_{5} = C_{(Y, T)}, \\
  C_{6} = C_{(Z, T)}, \;
  C_{7} = C_{(Y, a_4 Z + T)}, \;
  C_{8} = C_{(Z, Y + a_3 T)}, \;
  C_{9} = C_{(T, X + a_1 a_2 Y + Z)}.
\end{gather*}
Their linear equivalence classes on the generic member \(\mathcal{S}(\alpha)_{\Bbbk}\) of the pencil satisfy the following relations:
\begin{gather*}
  \begin{pmatrix}
    [C_{2}] \\ [C_{7}] \\ [C_{8}] \\ [C_{9}]
  \end{pmatrix} = 
  \begin{pmatrix}
    -1 & -2 & 0 & 0 & 0 & 1 \\
    -1 & 0 & -1 & -1 & 0 & 1 \\
    1 & 2 & -1 & 0 & -1 & 0 \\
    0 & -1 & 0 & -1 & -1 & 1
  \end{pmatrix} \cdot
  \begin{pmatrix}
    [C_{1}] & [C_{3}] & [C_{4}] & [C_{5}] & [C_{6}] & [H_{\mathcal{S}(\alpha)}]
  \end{pmatrix}^T.
\end{gather*}

For a general choice of \(\lambda \in \mathbb{C}\) and \(\alpha \in (\mathbb{C}^*)^5\) the surface \(\mathcal{S}(\alpha)_{\lambda}\) has the following singularities:
\begin{itemize}\setlength{\itemindent}{2cm}
\item[\(P_{1} = P_{(X, Y, Z)}\):] type \(\mathbb{A}_1\) with the quadratic term \(a_1 a_3 X Y + a_5 Y Z + X Z\);
\item[\(P_{2} = P_{(X, Y, T)}\):] type \(\mathbb{A}_2\) with the quadratic term \(X \cdot (Y + a_4 T)\);
\item[\(P_{3} = P_{(X, Z, T)}\):] type \(\mathbb{A}_2\) with the quadratic term \(X \cdot (a_2 Z + T)\);
\item[\(P_{4} = P_{(Y, Z, T)}\):] type \(\mathbb{A}_3\) with the quadratic term \(Y \cdot Z\);
\item[\(P_{5} = P_{(X, T, a_1 a_2 Y + Z)}\):] type \(\mathbb{A}_1\) with the quadratic term
  \[
    X (a_2 (X + a_1 a_2 Y + Z - (a_1 a_2 a_4 + \lambda) T) - T) + a_2 a_5 T^2.
  \]
\end{itemize}

Galois action on the lattice \(L(\alpha)_{\lambda}\) is trivial. The intersection matrix on \(L(\alpha)_{\lambda} = L(\alpha)_{\mathcal{S}}\) is represented by
\begin{table}[H]
  \begin{tabular}{|c||c|cc|cc|ccc|c|cccccc|}
    \hline
    \(\bullet\) & \(E_1^1\) & \(E_2^1\) & \(E_2^2\) & \(E_3^1\) & \(E_3^2\) & \(E_4^1\) & \(E_4^2\) & \(E_4^3\) & \(E_5^1\) & \(\widetilde{C_{1}}\) & \(\widetilde{C_{3}}\) & \(\widetilde{C_{4}}\) & \(\widetilde{C_{5}}\) & \(\widetilde{C_{6}}\) & \(\widetilde{H_{\mathcal{S}}}\) \\
    \hline
    \hline
    \(\widetilde{C_{1}}\) & \(1\) & \(1\) & \(0\) & \(0\) & \(0\) & \(0\) & \(0\) & \(0\) & \(0\) & \(-2\) & \(0\) & \(0\) & \(0\) & \(0\) & \(1\) \\
    \(\widetilde{C_{3}}\) & \(0\) & \(1\) & \(0\) & \(1\) & \(0\) & \(0\) & \(0\) & \(0\) & \(1\) & \(0\) & \(-2\) & \(0\) & \(0\) & \(0\) & \(1\) \\
    \(\widetilde{C_{4}}\) & \(1\) & \(0\) & \(0\) & \(0\) & \(0\) & \(0\) & \(1\) & \(0\) & \(0\) & \(0\) & \(0\) & \(-2\) & \(0\) & \(0\) & \(1\) \\
    \(\widetilde{C_{5}}\) & \(0\) & \(0\) & \(1\) & \(0\) & \(0\) & \(1\) & \(0\) & \(0\) & \(0\) & \(0\) & \(0\) & \(0\) & \(-2\) & \(0\) & \(1\) \\
    \(\widetilde{C_{6}}\) & \(0\) & \(0\) & \(0\) & \(0\) & \(1\) & \(0\) & \(0\) & \(1\) & \(0\) & \(0\) & \(0\) & \(0\) & \(0\) & \(-2\) & \(1\) \\
    \(\widetilde{H_{\mathcal{S}}}\) & \(0\) & \(0\) & \(0\) & \(0\) & \(0\) & \(0\) & \(0\) & \(0\) & \(0\) & \(1\) & \(1\) & \(1\) & \(1\) & \(1\) & \(4\) \\
    \hline
  \end{tabular}.
\end{table}
The intersection matrix is non-degenerate.

Discriminant groups and discriminant forms of the lattices \(L(\alpha)_{\mathcal{S}}\) and \(H \oplus \Pic(X)\) are given by
\begin{gather*}
  G' =
  \begin{pmatrix}
    \frac{1}{2} & 0 & 0 & 0 & 0 & \frac{1}{2} & 0 & \frac{1}{2} & \frac{1}{2} & 0 & 0 & 0 & 0 & 0 & \frac{1}{2} \\
    0 & 0 & \frac{1}{2} & 0 & 0 & \frac{1}{2} & 0 & 0 & 0 & \frac{1}{2} & 0 & \frac{1}{2} & 0 & 0 & 0 \\
    \frac{1}{12} & \frac{1}{12} & \frac{1}{3} & \frac{3}{4} & \frac{5}{6} & \frac{2}{3} & \frac{3}{4} & \frac{5}{6} & \frac{1}{3} & \frac{1}{6} & \frac{2}{3} & 0 & \frac{7}{12} & \frac{11}{12} & \frac{1}{6}
  \end{pmatrix}, \;
  B' = 
  \begin{pmatrix}
    0 & \frac{1}{2} & \frac{1}{2} \\
    \frac{1}{2} & 0 & \frac{1}{2} \\
    \frac{1}{2} & \frac{1}{2} & \frac{7}{12}
  \end{pmatrix}; \\
  G'' =
  \begin{pmatrix}
    0 & 0 & -\frac{1}{2} & 0 & \frac{1}{2} & 0 & 0 \\
    0 & 0 & -\frac{1}{2} & 0 & 0 & 0 & \frac{1}{2} \\
    0 & 0 & -\frac{5}{12} & \frac{1}{12} & \frac{1}{12} & -\frac{1}{6} & \frac{1}{4}
  \end{pmatrix}, \;
  B'' = 
  \begin{pmatrix}
    0 & \frac{1}{2} & \frac{1}{2} \\
    \frac{1}{2} & 0 & \frac{1}{2} \\
    \frac{1}{2} & \frac{1}{2} & \frac{5}{12}
  \end{pmatrix}; \;
  \begin{pmatrix}
    Q' \\ Q''
  \end{pmatrix}
  =
  \begin{pmatrix}
    1 & 0 & \frac{7}{12} \\
    1 & 0 & \frac{17}{12}.
  \end{pmatrix}.
\end{gather*}

%%% Local Variables:
%%% mode: latex
%%% TeX-master: "../main"
%%% End:
