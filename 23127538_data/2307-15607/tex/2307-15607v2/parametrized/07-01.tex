\subsection{Family \textnumero7.1}\label{subsection:07-01_parametrised}

The pencil \(\mathcal{S}(\alpha)\), where \(\alpha = (a_1, \ldots, a_7) \in (\mathbb{C}^*)^7\), is defined by the equation
\begin{gather*}
  \left(a_1^{2} a_3 a_6\right) X^{3} Z + \left(a_1^{2} a_2 a_3 a_6 + a_1^{2} a_3 a_5 a_6 + a_1\right) X^{2} Y Z + \left(a_1^{2} a_2 a_3 a_5 a_6 + a_1 a_2 + a_1 a_5\right) X Y^{2} Z + \left(a_1 a_2 a_5\right) Y^{3} Z + \\ X Y Z^{2} + \left(a_1 a_3 + a_1 a_6\right) X^{2} Z T + \left(a_1 a_2 a_4 a_5 + 1\right) Y^{2} Z T + a_7 X Y T^{2} + X Z T^{2} + a_4 Y Z T^{2} = \lambda X Y Z T.
\end{gather*}
Note that \(\mathcal{S}(\alpha)_{\infty} = S_{(X)} + S_{(Y)} + S_{(Z)} + S_{(T)}\). The base locus of the pencil \(\mathcal{S}(\alpha)\) consists of the curves
\begin{gather*}
  C_{1} = C_{(X, Y)}, \;
  C_{2} = C_{(X, Z)}, \;
  C_{3} = C_{(Y, Z)}, \;
  C_{4} = C_{(Z, T)}, \;
  C_{5} = C_{(X, Y + a_4 T)}, \;
  C_{6} = C_{(X, a_1 a_2 a_5 Y + T)}, \\
  C_{7} = C_{(Y, a_1 a_3 X + T)}, \;
  C_{8} = C_{(Y, a_1 a_6 X + T)}, \;
  C_{9} = C_{(T, a_1 (X + a_2 Y) (X + a_5 Y) (a_1 a_3 a_6 X + Y) + X Y Z)}.
\end{gather*}
Their linear equivalence classes on the generic member \(\mathcal{S}(\alpha)_{\Bbbk}\) of the pencil satisfy the following relations:
\begin{gather*}
  \begin{pmatrix}
    [C_{2}] \\ [C_{6}] \\ [C_{8}] \\ [C_{9}]
  \end{pmatrix} = 
  \begin{pmatrix}
    0 & -1 & -2 & 0 & 0 & 1 \\
    -1 & 1 & 2 & -1 & 0 & 0 \\
    -1 & -1 & 0 & 0 & -1 & 1 \\
    0 & 0 & -1 & 0 & 0 & 1
  \end{pmatrix} \cdot
  \begin{pmatrix}
    [C_{1}] & [C_{3}] & [C_{4}] & [C_{5}] & [C_{7}] & [H_{\mathcal{S}}]
  \end{pmatrix}^T.
\end{gather*}

For a general choice of \(\lambda \in \mathbb{C}\) and \(\alpha \in (\mathbb{C}^*)^7\) the surface \(\mathcal{S}(\alpha)_{\lambda}\) has the following singularities:
\begin{itemize}\setlength{\itemindent}{2cm}
\item[\(P_{1} = P_{(X, Y, Z)}\):] type \(\mathbb{A}_1\) with the quadratic term \(a_7 X Y + a_4 Y Z + X Z\);
\item[\(P_{2} = P_{(X, Y, T)}\):] type \(\mathbb{A}_3\) with the quadratic term \(X \cdot Y\);
\item[\(P_{3} = P_{(Z, T, X + a_2 Y)}\):] type \(\mathbb{A}_1\) with the quadratic term
  \[
    a_2 (Z^2 + a_7 T^2) - (a_1 a_2 (a_2 (a_3 + a_6) + a_4 a_5) + a_2 \lambda + 1) Z T - a_1 (a_2 - a_5) (a_1 a_2 a_3 a_6 - 1) Z (X + a_2 Y);
  \]
\item[\(P_{4} = P_{(Z, T, X + a_5 Y)}\):] type \(\mathbb{A}_1\) with the quadratic term
  \[
    a_5 (Z^2 + a_7 T^2) - (a_1 a_5 (a_5 (a_3 + a_6) + a_2 a_4) + a_5 \lambda + 1) Z T + a_1 (a_2 - a_5) (a_1 a_3 a_5 a_6 - 1) Z (X + a_5 Y); 
  \]
\item[\(P_{5} = P_{(Z, T, a_1 a_3 a_6 X + Y)}\):] type \(\mathbb{A}_1\) with the quadratic term
  \[
    a_3 a_6 (Z^2 + a_7 T^2) - (a_1 (a_3 a_6)^2 (a_1 a_2 a_4 a_5 + 1) + a_3 a_6 \lambda + a_3 + a_6) Z T - (a_1 a_2 a_3 a_6 - 1) (a_1 a_3 a_5 a_6 - 1) Z (a_1 a_3 a_6 X + Y).
  \]
\end{itemize}

Galois action on the lattice \(L(\alpha)_{\lambda}\) is trivial. The intersection matrix on \(L(\alpha)_{\lambda} = L(\alpha)_{\mathcal{S}}\) is represented by
\begin{table}[H]
  \begin{tabular}{|c||c|ccc|c|c|c|cccccc|}
    \hline
    \(\bullet\) & \(E_1^1\) & \(E_2^1\) & \(E_2^2\) & \(E_2^3\) & \(E_3^1\) & \(E_4^1\) & \(E_5^1\) & \(\widetilde{C_{1}}\) & \(\widetilde{C_{3}}\) & \(\widetilde{C_{4}}\) & \(\widetilde{C_{5}}\) & \(\widetilde{C_{7}}\) & \(\widetilde{H_{\mathcal{S}}}\) \\
    \hline
    \hline
    \(\widetilde{C_{1}}\) & \(1\) & \(0\) & \(1\) & \(0\) & \(0\) & \(0\) & \(0\) & \(-2\) & \(0\) & \(0\) & \(0\) & \(0\) & \(1\) \\
    \(\widetilde{C_{3}}\) & \(1\) & \(0\) & \(0\) & \(0\) & \(0\) & \(0\) & \(0\) & \(0\) & \(-2\) & \(1\) & \(0\) & \(1\) & \(1\) \\
    \(\widetilde{C_{4}}\) & \(0\) & \(0\) & \(0\) & \(0\) & \(1\) & \(1\) & \(1\) & \(0\) & \(1\) & \(-2\) & \(0\) & \(0\) & \(1\) \\
    \(\widetilde{C_{5}}\) & \(0\) & \(1\) & \(0\) & \(0\) & \(0\) & \(0\) & \(0\) & \(0\) & \(0\) & \(0\) & \(-2\) & \(0\) & \(1\) \\
    \(\widetilde{C_{7}}\) & \(0\) & \(0\) & \(0\) & \(1\) & \(0\) & \(0\) & \(0\) & \(0\) & \(1\) & \(0\) & \(0\) & \(-2\) & \(1\) \\
    \(\widetilde{H_{\mathcal{S}}}\) & \(0\) & \(0\) & \(0\) & \(0\) & \(0\) & \(0\) & \(0\) & \(1\) & \(1\) & \(1\) & \(1\) & \(1\) & \(4\) \\
    \hline
  \end{tabular}.
\end{table}
Note that the intersection matrix is non-degenerate.

Discriminant groups and discriminant forms of the lattices \(L(\alpha)_{\mathcal{S}}\) and \(H \oplus \Pic(X)\) are given by
\begin{gather*}
  G' =
  \begin{pmatrix}
    \frac{1}{2} & \frac{1}{2} & 0 & 0 & \frac{1}{2} & 0 & \frac{1}{2} & \frac{1}{2} & \frac{1}{2} & 0 & 0 & 0 & \frac{1}{2} \\
    \frac{1}{2} & 0 & \frac{1}{2} & 0 & 0 & 0 & 0 & 0 & 0 & 0 & \frac{1}{2} & \frac{1}{2} & 0 \\
    \frac{1}{2} & 0 & \frac{1}{2} & 0 & \frac{1}{2} & \frac{1}{2} & 0 & 0 & 0 & 0 & \frac{1}{2} & \frac{1}{2} & 0 \\
    0 & 0 & 0 & 0 & \frac{1}{2} & 0 & \frac{1}{2} & 0 & 0 & 0 & 0 & 0 & 0 \\
    \frac{1}{8} & \frac{3}{8} & \frac{1}{4} & \frac{5}{8} & \frac{3}{8} & \frac{3}{8} & \frac{3}{8} & \frac{1}{2} & \frac{3}{4} & \frac{3}{4} & \frac{1}{2} & 0 & \frac{5}{8}
  \end{pmatrix}, \;
  G'' =
  \begin{pmatrix}
    0 & 0 & \frac{1}{2} & \frac{1}{2} & \frac{1}{2} & 0 & \frac{1}{2} & 0 & 0 \\
    0 & 0 & \frac{1}{2} & \frac{1}{2} & 0 & \frac{1}{2} & \frac{1}{2} & 0 & 0 \\
    0 & 0 & \frac{1}{2} & 0 & 0 & 0 & \frac{1}{2} & 0 & 0 \\
    0 & 0 & \frac{1}{2} & \frac{1}{2} & 0 & 0 & 0 & 0 & 0 \\
    0 & 0 & \frac{1}{8} & \frac{1}{8} & \frac{1}{8} & \frac{1}{8} & \frac{1}{8} & \frac{3}{4} & \frac{7}{8}
  \end{pmatrix};
\end{gather*}
\begin{gather*}
  B' = 
  \begin{pmatrix}
    0 & \frac{1}{2} & 0 & 0 & 0 \\
    \frac{1}{2} & 0 & 0 & 0 & 0 \\
    0 & 0 & 0 & \frac{1}{2} & 0 \\
    0 & 0 & \frac{1}{2} & 0 & 0 \\
    0 & 0 & 0 & 0 & \frac{5}{8}
  \end{pmatrix}, \;
  B'' = 
  \begin{pmatrix}
    0 & \frac{1}{2} & 0 & 0 & 0 \\
    \frac{1}{2} & 0 & 0 & 0 & 0 \\
    0 & 0 & 0 & \frac{1}{2} & 0 \\
    0 & 0 & \frac{1}{2} & 0 & 0 \\
    0 & 0 & 0 & 0 & \frac{3}{8}
  \end{pmatrix}; \;
  \begin{pmatrix}
    Q' \\ Q''
  \end{pmatrix}
  =
  \begin{pmatrix}
    0 & 0 & 1 & 1 & \frac{13}{8} \\
    0 & 0 & 1 & 1 & \frac{3}{8}
  \end{pmatrix}.
\end{gather*}

%%% Local Variables:
%%% mode: latex
%%% TeX-master: "../main"
%%% End:
