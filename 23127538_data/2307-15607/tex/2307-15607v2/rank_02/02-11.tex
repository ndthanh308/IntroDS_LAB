\subsection{Family \textnumero2.11}\label{subsection:02-11}

The pencil \(\mathcal{S}\) is defined by the equation
\begin{gather*}
  X Y^{2} Z + X Y Z^{2} + Y^{2} Z^{2} + 2 Y Z^{3} + Z^{4} + X^{2} Y T + X Y^{2} T + \\
  2 Y^{2} Z T + 2 X Z^{2} T + 2 Y Z^{2} T + X^{2} T^{2} + 2 X Y T^{2} + Y^{2} T^{2} = \lambda X Y Z T.
\end{gather*}
Members \(\mathcal{S}_{\lambda}\) of the pencil are irreducible for any \(\lambda \in \mathbb{P}^1\) except
\begin{gather*}
  \mathcal{S}_{\infty} = S_{(X)} + S_{(Y)} + S_{(Z)} + S_{(T)}, \;
  \mathcal{S}_{- 2} = S_{(Z (Y + Z) + T (X + Y))} + S_{(Z (Y + Z) + T (X + Y) + X Y)}.
\end{gather*}
The base locus of the pencil \(\mathcal{S}\) consists of the following curves:
\begin{gather*}
  C_{1} = C_{(Z, T)}, \;
  C_{2} = C_{(Z, X + Y)}, \;
  C_{3} = C_{(T, Y + Z)}, \;
  C_{4} = C_{(X, Y (Z + T) + Z^2)}, \\
  C_{5} = C_{(Y, X T + Z^2)}, \;
  C_{6} = C_{(Z, X T + Y (X + T))}, \;
  C_{7} = C_{(T, Y (X + Z) + Z^2)}.
\end{gather*}
Their linear equivalence classes on the generic member \(\mathcal{S}_{\Bbbk}\) of the pencil satisfy the following relations:
\[
  \begin{pmatrix}
    2 [C_5] \\ [C_6] \\ [C_7] \\ [H_{\mathcal{S}}]
  \end{pmatrix} =
  \begin{pmatrix}
    0 & 0 & 0 & 2 \\
    -1 & -1 & 0 & 2 \\
    -1 & 0 & -1 & 2 \\
    0 & 0 & 0 & 2
  \end{pmatrix} \cdot
  \begin{pmatrix}
    [C_1] \\ [C_2] \\ [C_3] \\ [C_4]
  \end{pmatrix}.
\]

For a general choice of \(\lambda \in \mathbb{C}\) the surface \(\mathcal{S}_{\lambda}\) has the following singularities:
\begin{itemize}\setlength{\itemindent}{2cm}
\item[\(P_{1} = P_{(X, Y, Z)}\):] type \(\mathbb{D}_6\) with the quadratic term \((X + Y)^2\);
\item[\(P_{2} = P_{(X, Z, T)}\):] type \(\mathbb{A}_3\) with the quadratic term \((Z + T) \cdot (X + Z + T)\);
\item[\(P_{3} = P_{(Y, Z, T)}\):] type \(\mathbb{A}_5\) with the quadratic term \(T \cdot (Y + T)\);
\item[\(P_{4} = P_{(X, T, Y + Z)}\):] type \(\mathbb{A}_1\) with the quadratic term \((Y + Z - T) (X - Y - Z + T) - (\lambda + 2) X T\).
\end{itemize}

Galois action on the lattice \(L_{\lambda}\) is trivial. The intersection matrix on \(L_{\lambda} = L_{\mathcal{S}}\) is represented by
\begin{table}[H]
  \setlength{\tabcolsep}{4pt}
  \begin{tabular}{|c||cccccc|ccc|ccccc|c|ccccc|}
    \hline
    \(\bullet\) & \(E_1^1\) & \(E_1^2\) & \(E_1^3\) & \(E_1^4\) & \(E_1^5\) & \(E_1^6\) & \(E_2^1\) & \(E_2^2\) & \(E_2^3\) & \(E_3^1\) & \(E_3^2\) & \(E_3^3\) & \(E_3^4\) & \(E_3^5\) & \(E_4^1\) & \(\widetilde{C_{1}}\) & \(\widetilde{C_{2}}\) & \(\widetilde{C_{3}}\) & \(\widetilde{C_{4}}\) & \(\widetilde{C_{5}}\) \\
    \hline
    \hline
    \(\widetilde{C_{1}}\) & \(0\) & \(0\) & \(0\) & \(0\) & \(0\) & \(0\) & \(1\) & \(0\) & \(0\) & \(1\) & \(0\) & \(0\) & \(0\) & \(0\) & \(0\) & \(-2\) & \(1\) & \(0\) & \(0\) & \(0\) \\
    \(\widetilde{C_{2}}\) & \(1\) & \(0\) & \(0\) & \(0\) & \(0\) & \(0\) & \(0\) & \(0\) & \(0\) & \(0\) & \(0\) & \(0\) & \(0\) & \(0\) & \(0\) & \(1\) & \(-2\) & \(0\) & \(0\) & \(0\) \\
    \(\widetilde{C_{3}}\) & \(0\) & \(0\) & \(0\) & \(0\) & \(0\) & \(0\) & \(0\) & \(0\) & \(0\) & \(1\) & \(0\) & \(0\) & \(0\) & \(0\) & \(1\) & \(0\) & \(0\) & \(-2\) & \(0\) & \(0\) \\
    \(\widetilde{C_{4}}\) & \(0\) & \(0\) & \(0\) & \(0\) & \(1\) & \(0\) & \(0\) & \(1\) & \(0\) & \(0\) & \(0\) & \(0\) & \(0\) & \(0\) & \(1\) & \(0\) & \(0\) & \(0\) & \(-2\) & \(0\) \\
    \(\widetilde{C_{5}}\) & \(0\) & \(0\) & \(0\) & \(0\) & \(0\) & \(1\) & \(0\) & \(0\) & \(0\) & \(0\) & \(0\) & \(1\) & \(0\) & \(0\) & \(0\) & \(0\) & \(0\) & \(0\) & \(0\) & \(-2\) \\
    \hline
  \end{tabular}.
\end{table}
Note that the intersection matrix is degenerate. We choose the following integral basis of the lattice \(L_{\lambda}\):
\begin{align*}
  \begin{pmatrix}
    [E_4^1] \\ [\widetilde{C_{5}}]
  \end{pmatrix} = 
  \begin{pmatrix}
    0 & -2 & -4 & -6 & -5 & -3 & 1 & -2 & -1 & 5 & 4 & 3 & 2 & 1 & 4 & 2 & 2 & -4 \\
    0 & -1 & -2 & -3 & -2 & -2 & 1 & 0 & 0 & 2 & 1 & 0 & 0 & 0 & 2 & 1 & 1 & -1
  \end{pmatrix} \cdot \\
  \begin{pmatrix}
    [E_1^1] & [E_1^2] & [E_1^3] & [E_1^4] & [E_1^5] & [E_1^6] & [E_2^1] & [E_2^2] & [E_2^3] & \\ [E_3^1] & [E_3^2] & [E_3^3] & [E_3^4] & [E_3^5] & [\widetilde{C_{1}}] & [\widetilde{C_{2}}] & [\widetilde{C_{3}}] & [\widetilde{C_{4}}]
  \end{pmatrix}^T.
\end{align*}

Discriminant groups and discriminant forms of the lattices \(L_{\mathcal{S}}\) and \(H \oplus \Pic(X)\) are given by
\begin{gather*}
  G' = 
  \begin{pmatrix}
    \frac{10}{13} & \frac{4}{13} & \frac{11}{13} & \frac{5}{13} & \frac{3}{13} & \frac{9}{13} & \frac{4}{13} & \frac{12}{13} & \frac{6}{13} & \frac{11}{13} & \frac{1}{13} & \frac{4}{13} & \frac{7}{13} & \frac{10}{13} & \frac{9}{13} & \frac{3}{13} & \frac{12}{13} & \frac{1}{13}
  \end{pmatrix}, \\
  G'' = 
  \begin{pmatrix}
    0 & 0 & \frac{7}{13} & \frac{1}{13}
  \end{pmatrix}; \;
  B' = 
  \begin{pmatrix}
    \frac{6}{13}
  \end{pmatrix}, \;
  B'' = 
  \begin{pmatrix}
    \frac{7}{13}
  \end{pmatrix}; \;
  Q' =
  \begin{pmatrix}
    \frac{6}{13}
  \end{pmatrix}; \;
  Q'' =
  \begin{pmatrix}
    \frac{20}{13}
  \end{pmatrix}.
\end{gather*}

%%% Local Variables:
%%% mode: latex
%%% TeX-master: "../main"
%%% End:
