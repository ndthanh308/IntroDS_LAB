\subsection{Family \textnumero2.1}\label{subsection:02-01}

The pencil \(\mathcal{S}\) is defined by the equation
\[
  X (X + Y) C^3 = Y (Y + \lambda (X + Y)) (A^3 - B C (A - B)). 
\]
Members \(\mathcal{S}_{\lambda}\) of the pencil are irreducible for any \(\lambda \in \mathbb{P}^1\) except
\[
  \mathcal{S}_{\infty} = S_{(Y)} + S_{(X + Y)} + S_{(A^3 - B C(A - B))}, \;
  \mathcal{S}_{-1} = S_{(X)} + S_{(Y (A^3 - B C (A - B)) + C^3 (X + Y))}.
\]
The base locus of the pencil \(\mathcal{S}\) consists of the following curves:
\[
  C_1 = C_{(Y, C)}, \;
  C_2 = C_{(A, C)}, \;
  C_3 = C_{(X, A^3 - B C (A - B))}, \;
  C_4 = C_{(X + Y, A^3 - B C (A - B))}.
\]
Their linear equivalence classes on the generic member \(\mathcal{S}_{\Bbbk}\) of the pencil satisfy the following relations:
\[
  [C_{3}] = [C_{4}] = [H_{\mathcal{S}}^{(1)}] = 3 [C_{1}].
\]

For a general choice of \(\lambda \in \mathbb{C}\) the surface \(\mathcal{S}_{\lambda}\) has the following singularities:
\begin{itemize}\setlength{\itemindent}{2cm}
\item[\(P_1 = P_{(Y, A, C)}\):] type \(\mathbb{A}_8\) with the quadratic term \(\lambda Y \cdot C\);
\item[\(P_2 = P_{(A, C, Y + \lambda (X + Y))}\):] type \(\mathbb{A}_8\) with the quadratic term \(\lambda (\lambda + 1) Y \cdot C\).
\end{itemize}

Galois action on the lattice \(L_{\lambda}\) is trivial. The intersection matrix on \(L_{\lambda} = L_{\mathcal{S}}\) is represented by
\begin{table}[H]
  \begin{tabular}{|c||cccccccc|cccccccc|ccc|}
    \hline
    \(\bullet\) & \(E_1^1\) & \(E_1^2\) & \(E_1^3\) & \(E_1^4\) & \(E_1^5\) & \(E_1^6\) & \(E_1^7\) & \(E_1^8\) & \(E_2^1\) & \(E_2^2\) & \(E_2^3\) & \(E_2^4\) & \(E_2^5\) & \(E_2^6\) & \(E_2^7\) & \(E_2^8\) & \(\widetilde{C_1}\) & \(\widetilde{C_2}\) & \(\widetilde{H_{\mathcal{S}}^{(2)}}\) \\
    \hline
    \hline
    \(\widetilde{C_1}\) & \(0\) & \(0\) & \(1\) & \(0\) & \(0\) & \(0\) & \(0\) & \(0\) & \(0\) & \(0\) & \(0\) & \(0\) & \(0\) & \(0\) & \(0\) & \(0\) & \(-2\) & \(0\) & \(1\) \\
    \(\widetilde{C_2}\) & \(0\) & \(0\) & \(0\) & \(0\) & \(0\) & \(0\) & \(0\) & \(1\) & \(1\) & \(0\) & \(0\) & \(0\) & \(0\) & \(0\) & \(0\) & \(0\) & \(0\) & \(-2\) & \(0\) \\
    \(\widetilde{H_{\mathcal{S}}^{(2)}}\) & \(0\) & \(0\) & \(0\) & \(0\) & \(0\) & \(0\) & \(0\) & \(0\) & \(0\) & \(0\) & \(0\) & \(0\) & \(0\) & \(0\) & \(0\) & \(0\) & \(1\) & \(0\) & \(2\) \\
    \hline
  \end{tabular}.
\end{table}

Note that the intersection matrix is degenerate. We choose the following integral basis of the lattice \(L_{\lambda}\):
\begin{align*}
  \begin{pmatrix}
    [E_2^8]
  \end{pmatrix} =
  \begin{pmatrix}
    -5 & -10 & -15 & -14 & -13 & -12 & -11 & -10 & -8 \\ -7 & -6 & -5 & -4 & -3 & -2 & -6 & -9 & 3
  \end{pmatrix} \cdot \\
  \begin{pmatrix}
    [E_1^1] & [E_1^2] & [E_1^3] & [E_1^4] & [E_1^5] & [E_1^6] & [E_1^7] & [E_1^8] & [E_2^1] & \\
    [E_2^2] & [E_2^3] & [E_2^4] & [E_2^5] & [E_2^6] & [E_2^7] & [\widetilde{C_1}] & [\widetilde{C_2}] & [\widetilde{H_{\mathcal{S}}^{(2)}}]
  \end{pmatrix}^T.
\end{align*}

Discriminant groups of the lattices \(L_{\mathcal{S}}\) and \(H \oplus \Pic(X)\) are both trivial.

%%% Local Variables:
%%% mode: latex
%%% TeX-master: "../main"
%%% End:
