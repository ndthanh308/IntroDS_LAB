\subsection{Family \textnumero2.14}\label{subsection:02-14}

The pencil \(\mathcal{S}\) is defined by the equation
\begin{gather*}
  X^{2} Y^{2} + X^{2} Y Z + X Y Z^{2} + 2 X Y^{2} T + Y Z^{2} T + Z^{3} T + 2 X Y T^{2} + \\ Y^{2} T^{2} + 3 Y Z T^{2} + 3 Z^{2} T^{2} + 2 Y T^{3} + 3 Z T^{3} + T^{4} = \lambda X Y Z T.
\end{gather*}
Members \(\mathcal{S}_{\lambda}\) of the pencil are irreducible for any \(\lambda \in \mathbb{P}^1\) except
\(\mathcal{S}_{\infty} = S_{(X)} + S_{(Y)} + S_{(Z)} + S_{(T)}\).
The base locus of the pencil \(\mathcal{S}\) consists of the following curves:
\begin{gather*}
  C_{1} = C_{(X, T)}, \;
  C_{2} = C_{(Y, T)}, \;
  C_{3} = C_{(Y, Z + T)}, \;
  C_{4} = C_{(X, Y + Z + T)}, \\
  C_{5} = C_{(X, Y T + (Z + T)^2)}, \;
  C_{6} = C_{(Z, Y (X + T) + T^2)}, \;
  C_{7} = C_{(T, X (Y + Z) + Z^2)}.
\end{gather*}
Their linear equivalence classes on the generic member \(\mathcal{S}_{\Bbbk}\) of the pencil satisfy the following relations:
\begin{gather*}
  \begin{pmatrix}
    [C_{1}] \\ [C_{2}] \\ [C_{5}] \\ [H_{\mathcal{S}}]
  \end{pmatrix} = 
  \begin{pmatrix}
    3 & 0 & 0 & -1 \\
    -3 & 0 & 2 & 0 \\
    -3 & -1 & 2 & 1 \\
    0 & 0 & 2 & 0
  \end{pmatrix} \cdot
  \begin{pmatrix}
    [C_{3}] \\ [C_{4}] \\ [C_{6}] \\ [C_{7}]
  \end{pmatrix}.
\end{gather*}

For a general choice of \(\lambda \in \mathbb{C}\) the surface \(\mathcal{S}_{\lambda}\) has the following singularities:
\begin{itemize}\setlength{\itemindent}{2cm}
\item[\(P_{1} = P_{(X, Z, T)}\):] type \(\mathbb{D}_5\) with the quadratic term \((X + T)^2\);
\item[\(P_{2} = P_{(Y, Z, T)}\):] type \(\mathbb{A}_3\) with the quadratic term \(Y \cdot (Y + Z)\);
\item[\(P_{3} = P_{(X, Y, Z + T)}\):] type \(\mathbb{A}_2\) with the quadratic term \(Y \cdot ((\lambda + 3) X + Y + Z + T)\);
\item[\(P_{4} = P_{(X, Z, Y + T)}\):] type \(\mathbb{A}_1\) with the quadratic term \((X - Y - Z - T) (X - Y - 2 Z - T) + (\lambda + 3) X Z\);
\item[\(P_{5} = P_{(Y, X - (\lambda + 3) T, Z + T)}\):] type \(\mathbb{A}_2\) with the quadratic term
  \[
    Y \cdot ((\lambda + 3) X - (\lambda + 4)^2 Y - (\lambda + 4) Z - ((\lambda + 4)^2 - (\lambda + 3)) T);
  \]
\item[\(P_{6} = P_{(Z, X - (\lambda + 3) T, (\lambda + 4) Y + T)}\):] type \(\mathbb{A}_1\).
\end{itemize}

Galois action on the lattice \(L_{\lambda}\) is trivial. The intersection matrix on \(L_{\lambda} = L_{\mathcal{S}}\) is represented by
\begin{table}[H]
  \begin{tabular}{|c||ccccc|ccc|cc|c|cc|c|cccc|}
    \hline
    \(\bullet\) & \(E_1^1\) & \(E_1^2\) & \(E_1^3\) & \(E_1^4\) & \(E_1^5\) & \(E_2^1\) & \(E_2^2\) & \(E_2^3\) & \(E_3^1\) & \(E_3^2\) & \(E_4^1\) & \(E_5^1\) & \(E_5^2\) & \(E_6^1\) & \(\widetilde{C_{3}}\) & \(\widetilde{C_{4}}\) & \(\widetilde{C_{6}}\) & \(\widetilde{C_{7}}\) \\
    \hline
    \hline
    \(\widetilde{C_{3}}\) & \(0\) & \(0\) & \(0\) & \(0\) & \(0\) & \(1\) & \(0\) & \(0\) & \(1\) & \(0\) & \(0\) & \(1\) & \(0\) & \(0\) & \(-2\) & \(0\) & \(0\) & \(0\) \\
    \(\widetilde{C_{4}}\) & \(0\) & \(0\) & \(0\) & \(0\) & \(0\) & \(0\) & \(0\) & \(0\) & \(0\) & \(1\) & \(1\) & \(0\) & \(0\) & \(0\) & \(0\) & \(-2\) & \(0\) & \(0\) \\
    \(\widetilde{C_{6}}\) & \(1\) & \(0\) & \(0\) & \(0\) & \(0\) & \(0\) & \(1\) & \(0\) & \(0\) & \(0\) & \(1\) & \(0\) & \(0\) & \(1\) & \(0\) & \(0\) & \(-2\) & \(0\) \\
    \(\widetilde{C_{7}}\) & \(0\) & \(0\) & \(0\) & \(0\) & \(1\) & \(0\) & \(0\) & \(1\) & \(0\) & \(0\) & \(0\) & \(0\) & \(0\) & \(0\) & \(0\) & \(0\) & \(0\) & \(-2\) \\
    \hline
  \end{tabular}.
\end{table}
Note that the intersection matrix is non-degenerate.

Discriminant groups and discriminant forms of the lattices \(L_{\mathcal{S}}\) and \(H \oplus \Pic(X)\) are given by
\begin{gather*}
  G' = 
  \begin{pmatrix}
    \frac{1}{5} & \frac{1}{5} & \frac{1}{5} & \frac{3}{5} & \frac{3}{5} & 0 & \frac{4}{5} & \frac{2}{5} & \frac{3}{5} & 0 & \frac{4}{5} & \frac{4}{5} & \frac{2}{5} & \frac{3}{5} & \frac{1}{5} & \frac{2}{5} & \frac{1}{5} & 0 \\
    \frac{1}{5} & \frac{3}{5} & 0 & 0 & \frac{2}{5} & \frac{1}{5} & \frac{3}{5} & \frac{1}{5} & \frac{1}{5} & \frac{3}{5} & \frac{2}{5} & \frac{1}{5} & \frac{3}{5} & \frac{2}{5} & \frac{4}{5} & 0 & \frac{4}{5} & \frac{4}{5}
  \end{pmatrix}, \\
  G'' = 
  \begin{pmatrix}
    0 & 0 & \frac{1}{5} & 0 \\
    0 & 0 & 0 & \frac{1}{5}
  \end{pmatrix}; \;
  B' = 
  \begin{pmatrix}
    0 & \frac{4}{5} \\
    \frac{4}{5} & \frac{3}{5}
  \end{pmatrix}, \;
  B'' = 
  \begin{pmatrix}
    0 & \frac{1}{5} \\
    \frac{1}{5} & \frac{2}{5}
  \end{pmatrix}; \;
  \begin{pmatrix}
    Q' \\ Q''
  \end{pmatrix}
  =
  \begin{pmatrix}
    0 & \frac{8}{5} \\
    0 & \frac{2}{5}
  \end{pmatrix}.
\end{gather*}

%%% Local Variables:
%%% mode: latex
%%% TeX-master: "../main"
%%% End:
