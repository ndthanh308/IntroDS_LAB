\subsection{Family \textnumero2.12}\label{subsection:02-12}

The pencil \(\mathcal{S}\) is defined by the equation
\begin{gather*}
  X^{2} Y^{2} + X^{2} Y Z + X Y^{2} Z + X Y Z^{2} + 2 X^{2} Y T + 2 X Y^{2} T + \\ X^{2} T^{2} + 2 X Y T^{2} + Y^{2} T^{2} + 2 X Z T^{2} + 2 Y Z T^{2} + Z^{2} T^{2} = \lambda X Y Z T.
\end{gather*}
Members \(\mathcal{S}_{\lambda}\) of the pencil are irreducible for any \(\lambda \in \mathbb{P}^1\) except
\(\mathcal{S}_{\infty} = S_{(X)} + S_{(Y)} + S_{(Z)} + S_{(T)}\).
The base locus of the pencil \(\mathcal{S}\) consists of the following curves:
\begin{gather*}
  C_{1} = C_{(X, T)}, \;
  C_{2} = C_{(Y, T)}, \;
  C_{3} = C_{(X, Y + Z)}, \;
  C_{4} = C_{(Y, X + Z)}, \\
  C_{5} = C_{(T, X + Z)}, \;
  C_{6} = C_{(T, Y + Z)}, \;
  C_{7} = C_{(Z, X Y + T (X + Y))}.
\end{gather*}
Their linear equivalence classes on the generic member \(\mathcal{S}_{\Bbbk}\) of the pencil satisfy the following relations:
\[
  \begin{pmatrix}
    2 [C_4] \\ [C_6] \\ 2 [C_7] \\ [H_{\mathcal{S}}]
  \end{pmatrix} =
  \begin{pmatrix}
    2 & -2 & 2 & 0 \\
    1 & -1 & 2 & -1 \\
    2 & 0 & 2 & 0 \\
    2 & 0 & 2 & 0
  \end{pmatrix} \cdot
  \begin{pmatrix}
    [C_1] \\ [C_2] \\ [C_3] \\ [C_5]
  \end{pmatrix}.
\]

Put \(\mu (\mu - 1) = (\lambda + 2)^{-1}\). For a general choice of \(\lambda \in \mathbb{C}\) the surface \(\mathcal{S}_{\lambda}\) has the following singularities:
\begin{itemize}\setlength{\itemindent}{2cm}
\item[\(P_{1} = P_{(X, Y, Z)}\):] type \(\mathbb{D}_4\) with the quadratic term \((X + Y + Z)^2\);
\item[\(P_{2} = P_{(X, Y, T)}\):] type \(\mathbb{A}_1\) with the quadratic term \(X Y + T^2\);
\item[\(P_{3} = P_{(X, Z, T)}\):] type \(\mathbb{A}_1\) with the quadratic term \(X Z + (X + T)^2\);
\item[\(P_{4} = P_{(Y, Z, T)}\):] type \(\mathbb{A}_1\) with the quadratic term \(Y Z + (X + T)^2\);
\item[\(P_{5} = P_{(X, T, Y + Z)}\):] type \(\mathbb{A}_3\) with the quadratic term \(X \cdot (Y + Z - (\lambda + 2) T)\);
\item[\(P_{6} = P_{(Y, T, X + Z)}\):] type \(\mathbb{A}_3\) with the quadratic term \(Y \cdot (X + Z - (\lambda + 2) T)\);
\item[\(P_{7} = P_{(\mu X + (\mu - 1) Y, Z, (\mu - 1) Y - T)}\):] type \(\mathbb{A}_1\);
\item[\(P_{8} = P_{((\mu - 1) X + \mu Y, Z, (\mu - 1) X - T)}\):] type \(\mathbb{A}_1\).
\end{itemize}

The only non-trivial Galois orbit on the lattice \(L_{\lambda}\) is \(E_7^1 + E_8^1\).

The intersection matrix on the lattice \(L_{\lambda}\) is represented by
\begin{table}[H]
  \setlength{\tabcolsep}{4pt}
  \begin{tabular}{|c||cccc|c|c|c|ccc|ccc|c|c|cccccc|}
    \hline
    \(\bullet\) & \(E_1^1\) & \(E_1^2\) & \(E_1^3\) & \(E_1^4\) & \(E_2^1\) & \(E_3^1\) & \(E_4^1\) & \(E_5^1\) & \(E_5^2\) & \(E_5^3\) & \(E_6^1\) & \(E_6^2\) & \(E_6^3\) & \(E_7^1\) & \(E_8^1\) & \(\widetilde{C_{1}}\) & \(\widetilde{C_{2}}\) & \(\widetilde{C_{3}}\) & \(\widetilde{C_{4}}\) & \(\widetilde{C_{5}}\) & \(\widetilde{C_{7}}\)  \\
    \hline
    \hline
    \(\widetilde{C_{1}}\) & \(0\) & \(0\) & \(0\) & \(0\) & \(1\) & \(1\) & \(0\) & \(1\) & \(0\) & \(0\) & \(0\) & \(0\) & \(0\) & \(0\) & \(0\) & \(-2\) & \(0\) & \(0\) & \(0\) & \(0\) & \(0\) \\
    \(\widetilde{C_{2}}\) & \(0\) & \(0\) & \(0\) & \(0\) & \(1\) & \(0\) & \(1\) & \(0\) & \(0\) & \(0\) & \(1\) & \(0\) & \(0\) & \(0\) & \(0\) & \(0\) & \(-2\) & \(0\) & \(0\) & \(0\) & \(0\) \\
    \(\widetilde{C_{3}}\) & \(1\) & \(0\) & \(0\) & \(0\) & \(0\) & \(0\) & \(0\) & \(1\) & \(0\) & \(0\) & \(0\) & \(0\) & \(0\) & \(0\) & \(0\) & \(0\) & \(0\) & \(-2\) & \(0\) & \(0\) & \(0\) \\
    \(\widetilde{C_{4}}\) & \(0\) & \(0\) & \(1\) & \(0\) & \(0\) & \(0\) & \(0\) & \(0\) & \(0\) & \(0\) & \(1\) & \(0\) & \(0\) & \(0\) & \(0\) & \(0\) & \(0\) & \(0\) & \(-2\) & \(0\) & \(0\) \\
    \(\widetilde{C_{5}}\) & \(0\) & \(0\) & \(0\) & \(0\) & \(0\) & \(1\) & \(0\) & \(0\) & \(0\) & \(0\) & \(0\) & \(0\) & \(1\) & \(0\) & \(0\) & \(0\) & \(0\) & \(0\) & \(0\) & \(-2\) & \(0\) \\
    \(\widetilde{C_{7}}\) & \(0\) & \(0\) & \(0\) & \(1\) & \(0\) & \(1\) & \(1\) & \(0\) & \(0\) & \(0\) & \(0\) & \(0\) & \(0\) & \(1\) & \(1\) & \(0\) & \(0\) & \(0\) & \(0\) & \(0\) & \(-2\) \\
    \hline
  \end{tabular}.
\end{table}
Note that the intersection matrix is degenerate. We choose the following integral basis of the lattice \(L_{\mathcal{S}}\):
\begin{align*}
  \begin{pmatrix}
    [E_2^1] \\ [E_3^1]
  \end{pmatrix} =
  \begin{pmatrix}
    -1 & 0 & 0 & 1 & 1 & -3 & -2 & -1 & 0 & 0 & 0 & 1 & -2 & 0 & -2 & 0 & 0 & 2 \\
    -1 & 0 & 1 & 0 & 1 & -3 & -2 & -1 & 3 & 2 & 1 & 0 & -2 & 2 & -2 & 2 & 0 & 0
  \end{pmatrix} \cdot \\
  \begin{pmatrix}
    [E_1^1] & [E_1^2] & [E_1^3] & [E_1^4] & [E_4^1] & [E_5^1] & [E_5^2] & [E_5^3] & [E_6^1] \\ [E_6^2] & [E_6^3] & [E_7^1 + E_8^1] & [\widetilde{C_{1}}] & [\widetilde{C_{2}}] & [\widetilde{C_{3}}] & [\widetilde{C_{4}}] & [\widetilde{C_{5}}] & [\widetilde{C_{7}}]
  \end{pmatrix}^T.
\end{align*}

Discriminant groups and discriminant forms of the lattices \(L_{\mathcal{S}}\) and \(H \oplus \Pic(X)\) are given by
\begin{gather*}
  G' =
  \begin{pmatrix}
    0 & 0 & 0 & 0 & 0 & 0 & 0 & 0 & 0 & 0 & 0 & \frac{1}{2} & 0 & 0 & 0 & 0 & 0 & 0 \\
    \frac{2}{5} & \frac{3}{10} & \frac{4}{5} & \frac{2}{5} & \frac{3}{5} & \frac{3}{5} & \frac{2}{5} & \frac{1}{5} & \frac{4}{5} & \frac{3}{5} & \frac{2}{5} & \frac{1}{2} & \frac{3}{10} & \frac{7}{10} & \frac{1}{2} & \frac{3}{10} & \frac{1}{5} & \frac{1}{2}
  \end{pmatrix}, \\
  G'' = 
  \begin{pmatrix}
    0 & 0 & \frac{1}{2} & 0 \\
    0 & 0 & \frac{3}{5} & \frac{1}{10}
  \end{pmatrix}; \;
  B' = 
  \begin{pmatrix}
    0 & \frac{1}{2} \\
    \frac{1}{2} & \frac{4}{5}
  \end{pmatrix}, \;
  B'' = 
  \begin{pmatrix}
    0 & \frac{1}{2} \\
    \frac{1}{2} & \frac{1}{5}
  \end{pmatrix}; \;
  \begin{pmatrix}
    Q' \\ Q''
  \end{pmatrix}
  =
  \begin{pmatrix}
    1 & \frac{9}{5} \\
    1 & \frac{1}{5}
  \end{pmatrix}.
\end{gather*}

%%% Local Variables:
%%% mode: latex
%%% TeX-master: "../main"
%%% End:
