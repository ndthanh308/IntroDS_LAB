\subsection{Family \textnumero2.4}\label{subsection:02-04}

The pencil \(\mathcal{S}\) is defined by the equation
\begin{gather*}
  X^{3} Y + X^{2} Y^{2} + X^{3} Z + 4 X^{2} Y Z + 2 X Y^{2} Z + 2 X^{2} Z^{2} + 4 X Y Z^{2} + Y^{2} Z^{2} + X Z^{3} + Y Z^{3} + 3 X^{2} Y T + \\ 2 X Y^{2} T + 2 X^{2} Z T + 2 Y^{2} Z T + 2 X Z^{2} T + 3 Y Z^{2} T + 3 X Y T^{2} + Y^{2} T^{2} + X Z T^{2} + 3 Y Z T^{2} + Y T^{3} = \lambda X Y Z T.
\end{gather*}
Members \(\mathcal{S}_{\lambda}\) of the pencil are irreducible for any \(\lambda \in \mathbb{P}^1\) except
\begin{gather*}
  \mathcal{S}_{\infty} = S_{(X)} + S_{(Y)} + S_{(Z)} + S_{(T)}, \;
  \mathcal{S}_{- 7} = S_{(X + Z + T)} + S_{(X + Y + Z + T)} + S_{(Y (X + Z + T) + X Z)}.
\end{gather*}
The base locus of the pencil \(\mathcal{S}\) consists of the following curves:
\begin{gather*}
  C_{1} = C_{(X, Y)}, \;
  C_{2} = C_{(Y, Z)}, \;
  C_{3} = C_{(X, Z + T)}, \;
  C_{4} = C_{(Z, X + T)}, \;
  C_{5} = C_{(T, X + Z)}, \\
  C_{6} = C_{(X, Y + Z + T)}, \;
  C_{7} = C_{(Y, X + Z + T)}, \;
  C_{8} = C_{(Z, X + Y + T)}, \;
  C_{9} = C_{(T, X + Y + Z)}, \;
  C_{10} = C_{(T, X Z + Y (X + Z))}.
\end{gather*}
Their linear equivalence classes on the generic member \(\mathcal{S}_{\Bbbk}\) of the pencil satisfy the following relations:
\begin{gather*}
  \begin{pmatrix}
    [C_{2}] \\ [C_{6}] \\ [C_{7}] \\ [C_{8}] \\ [C_{9}] \\ [C_{10}]
  \end{pmatrix} = 
  \begin{pmatrix}
    -1 & 2 & 2 & 2 & -1 \\
    -1 & -2 & 0 & 0 & 1 \\
    0 & -1 & -1 & -1 & 1 \\
    1 & -2 & -4 & -2 & 2 \\
    0 & 5 & 5 & 3 & -3 \\
    0 & -5 & -5 & -4 & 4
  \end{pmatrix} \cdot
  \begin{pmatrix}
    [C_{1}] \\ [C_{3}] \\ [C_{4}] \\ [C_{5}] \\ [H_{\mathcal{S}}]
  \end{pmatrix}.
\end{gather*}

For a general choice of \(\lambda \in \mathbb{C}\) the surface \(\mathcal{S}_{\lambda}\) has the following singularities:
\begin{itemize}\setlength{\itemindent}{2cm}
\item[\(P_{1} = P_{(X, Z, T)}\):] type \(\mathbb{D}_4\) with the quadratic term \((X + Z + T)^2\);
\item[\(P_{2} = P_{(X, Y, Z + T)}\):] type \(\mathbb{A}_4\) with the quadratic term \((\lambda + 7) X \cdot Y\);
\item[\(P_{3} = P_{(Y, Z, X + T)}\):] type \(\mathbb{A}_4\) with the quadratic term \((\lambda + 7) Y \cdot Z\);
\item[\(P_{4} = P_{(Y, T, X + Z)}\):] type \(\mathbb{A}_1\) with the quadratic term \((X + Z + T) (X + Y + Z + T) - (\lambda + 7) Y T\).
\end{itemize}

Galois action on the lattice \(L_{\lambda}\) is trivial. The intersection matrix on \(L_{\lambda} = L_{\mathcal{S}}\) is represented by
\begin{table}[H]
  \begin{tabular}{|c||cccc|cccc|cccc|c|ccccc|}
    \hline
    \(\bullet\) & \(E_1^1\) & \(E_1^2\) & \(E_1^3\) & \(E_1^4\) & \(E_2^1\) & \(E_2^2\) & \(E_2^3\) & \(E_2^4\) & \(E_3^1\) & \(E_3^2\) & \(E_3^3\) & \(E_3^4\) & \(E_4^1\) & \(\widetilde{C_{1}}\) & \(\widetilde{C_{3}}\) & \(\widetilde{C_{4}}\) & \(\widetilde{C_{5}}\) & \(\widetilde{H_{\mathcal{S}}}\) \\
    \hline
    \hline
    \(\widetilde{C_{1}}\) & \(0\) & \(0\) & \(0\) & \(0\) & \(0\) & \(1\) & \(0\) & \(0\) & \(0\) & \(0\) & \(0\) & \(0\) & \(0\) & \(-2\) & \(0\) & \(0\) & \(0\) & \(1\) \\
    \(\widetilde{C_{3}}\) & \(1\) & \(0\) & \(0\) & \(0\) & \(1\) & \(0\) & \(0\) & \(0\) & \(0\) & \(0\) & \(0\) & \(0\) & \(0\) & \(0\) & \(-2\) & \(0\) & \(0\) & \(1\) \\
    \(\widetilde{C_{4}}\) & \(0\) & \(0\) & \(1\) & \(0\) & \(0\) & \(0\) & \(0\) & \(0\) & \(0\) & \(0\) & \(0\) & \(1\) & \(0\) & \(0\) & \(0\) & \(-2\) & \(0\) & \(1\) \\
    \(\widetilde{C_{5}}\) & \(0\) & \(0\) & \(0\) & \(1\) & \(0\) & \(0\) & \(0\) & \(0\) & \(0\) & \(0\) & \(0\) & \(0\) & \(1\) & \(0\) & \(0\) & \(0\) & \(-2\) & \(1\) \\
    \(\widetilde{H_{\mathcal{S}}}\) & \(0\) & \(0\) & \(0\) & \(0\) & \(0\) & \(0\) & \(0\) & \(0\) & \(0\) & \(0\) & \(0\) & \(0\) & \(0\) & \(1\) & \(1\) & \(1\) & \(1\) & \(4\) \\
    \hline
  \end{tabular}.
\end{table}
Note that the intersection matrix is non-degenerate.

Discriminant groups and discriminant forms of the lattices \(L_{\mathcal{S}}\) and \(H \oplus \Pic(X)\) are given by
\begin{gather*}
  G' = 
  \begin{pmatrix}
    \frac{5}{9} & \frac{1}{3} & \frac{8}{9} & \frac{2}{9} & \frac{5}{9} & \frac{1}{3} & \frac{2}{9} & \frac{1}{9} & \frac{8}{9} & \frac{7}{9} & \frac{2}{3} & \frac{5}{9} & \frac{5}{9} & \frac{8}{9} & \frac{7}{9} & \frac{4}{9} & \frac{1}{9} & \frac{4}{9}
  \end{pmatrix}, \\
  G'' = 
  \begin{pmatrix}
    0 & 0 & \frac{8}{9} & \frac{1}{3}
  \end{pmatrix}; \;
  B' = 
  \begin{pmatrix}
    \frac{7}{9}
  \end{pmatrix}, \;
  B'' =
  \begin{pmatrix}
    \frac{2}{9}
  \end{pmatrix}; \;
  Q' =
  \begin{pmatrix}
    \frac{16}{9}
  \end{pmatrix}, \;
  Q'' =
  \begin{pmatrix}
    \frac{2}{9}
  \end{pmatrix}.
\end{gather*}

%%% Local Variables:
%%% mode: latex
%%% TeX-master: "../main"
%%% End:
