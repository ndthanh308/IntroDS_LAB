\subsection{Family \textnumero2.3}\label{subsection:02-03}

The pencil \(\mathcal{S}\) is defined by the equation
\begin{gather*}
  X^3 Y + (Y + Z) (X (Z + T) - T^2) (Y + \lambda Z) = 0.
\end{gather*}
Members \(\mathcal{S}_{\lambda}\) of the pencil are irreducible for any \(\lambda \in \mathbb{P}^1\) except
\begin{gather*}
  \mathcal{S}_{\infty} = S_{(Z)} + S_{(Y + Z)} + S_{(X (Z + T) - T^2)}, \;
  \mathcal{S}_{0} = S_{(Y)} + S_{(X^3 - (Y + Z) (X (Z + T) - T^2))}.
\end{gather*}
The base locus of the pencil \(\mathcal{S}\) consists of the following curves:
\begin{gather*}
  C_{1} = C_{(X, T)}, \;
  C_{2} = C_{(Y, Z)}, \;
  C_{3} = C_{(X, Y + Z)}, \;
  C_{4} = C_{(Y, X (Z + T) - T^2)}, \;
  C_{5} = C_{(Z, X^3 + Y T (X - T))}.
\end{gather*}
Their linear equivalence classes on the generic member \(\mathcal{S}_{\Bbbk}\) of the pencil satisfy the following relations:
\begin{gather*}
  \begin{pmatrix}
    [C_{2}] \\ [C_{4}] \\ [C_{5}]
  \end{pmatrix} = 
  \begin{pmatrix}
    0 & -3 & 1 \\
    0 & 6 & -1 \\
    0 & 3 & 0
  \end{pmatrix} \cdot
  \begin{pmatrix}
    [C_{1}] \\ [C_{3}] \\ [H_{\mathcal{S}}]
  \end{pmatrix}.
\end{gather*}

For a general choice of \(\lambda \in \mathbb{C}\) the surface \(\mathcal{S}_{\lambda}\) has the following singularities:
\begin{itemize}\setlength{\itemindent}{2cm}
\item[\(P_{1} = P_{(X, Y, Z)}\):] type \(\mathbb{A}_5\) with the quadratic term \((Y + Z) \cdot (Y + \lambda Z)\);
\item[\(P_{2} = P_{(X, Z, T)}\):] type \(\mathbb{A}_1\) with the quadratic term \(X (Z + T) - T^2\);
\item[\(P_{3} = P_{(X, T, Y + Z)}\):] type \(\mathbb{A}_5\) with the quadratic term \((\lambda - 1) X \cdot (Y + Z)\);
\item[\(P_{4} = P_{(X, T, Y + \lambda Z)}\):] type \(\mathbb{A}_5\) with the quadratic term \((\lambda - 1) X \cdot (Y + \lambda Z)\).
\end{itemize}

Galois action on the lattice \(L_{\lambda}\) is trivial. The intersection matrix on \(L_{\lambda} = L_{\mathcal{S}}\) is represented by
\begin{table}[H]
  \begin{tabular}{|c||ccccc|c|ccccc|ccccc|ccc|}
    \hline
    \(\bullet\) & \(E_1^1\) & \(E_1^2\) & \(E_1^3\) & \(E_1^4\) & \(E_1^5\) & \(E_2^1\) & \(E_3^1\) & \(E_3^2\) & \(E_3^3\) & \(E_3^4\) & \(E_3^5\) & \(E_4^1\) & \(E_4^2\) & \(E_4^3\) & \(E_4^4\) & \(E_4^5\) & \(\widetilde{C_{1}}\) & \(\widetilde{C_{3}}\) & \(\widetilde{H_{\mathcal{S}}}\) \\
    \hline
    \hline
    \(\widetilde{C_{1}}\) & \(0\) & \(0\) & \(0\) & \(0\) & \(0\) & \(1\) & \(1\) & \(0\) & \(0\) & \(0\) & \(0\) & \(1\) & \(0\) & \(0\) & \(0\) & \(0\) & \(-2\) & \(0\) & \(1\) \\
    \(\widetilde{C_{3}}\) & \(1\) & \(0\) & \(0\) & \(0\) & \(0\) & \(0\) & \(0\) & \(0\) & \(0\) & \(1\) & \(0\) & \(0\) & \(0\) & \(0\) & \(0\) & \(0\) & \(0\) & \(-2\) & \(1\) \\
    \(\widetilde{H_{\mathcal{S}}}\) & \(0\) & \(0\) & \(0\) & \(0\) & \(0\) & \(0\) & \(0\) & \(0\) & \(0\) & \(0\) & \(0\) & \(0\) & \(0\) & \(0\) & \(0\) & \(0\) & \(1\) & \(1\) & \(4\) \\
    \hline
  \end{tabular}.
\end{table}
Note that the intersection matrix is degenerate. We choose the following integral basis of the lattice \(L_{\lambda}\):
\begin{align*}
  \begin{pmatrix}
    [E_4^5]
  \end{pmatrix} =
  \begin{pmatrix}
    -5 & -4 & -3 & -2 & -1 & -3 & -7 & -8 & -9 & -10 & -5 & -5 & -4 & -3 & -2 & -6 & -6 & 3
  \end{pmatrix} \cdot \\
  \begin{pmatrix}
    [E_1^1] & [E_1^2] & [E_1^3] & [E_1^4] & [E_1^5] & [E_2^1] & [E_3^1] & [E_3^2] & [E_3^3] & \\
    [E_3^4] & [E_3^5] & [E_4^1] & [E_4^2] & [E_4^3] & [E_4^4] & [\widetilde{C_{1}}] & [\widetilde{C_{3}}] & [\widetilde{H_{\mathcal{S}}}]
  \end{pmatrix}^T.
\end{align*}

Discriminant groups and discriminant forms of the lattices \(L_{\mathcal{S}}\) and \(H \oplus \Pic(X)\) are given by
\begin{gather*}
  G' = 
  \begin{pmatrix}
    0 & 0 & 0 & 0 & 0 & \frac{1}{2} & \frac{1}{2} & 0 & \frac{1}{2} & 0 & \frac{1}{2} & 0 & 0 & 0 & 0 & 0 & 0 & 0 \\
    \frac{1}{2} & 0 & \frac{1}{2} & 0 & \frac{1}{2} & \frac{1}{2} & 0 & 0 & 0 & 0 & 0 & 0 & 0 & 0 & 0 & 0 & 0 & \frac{1}{2}
  \end{pmatrix}, \\
  G'' = 
  \begin{pmatrix}
    0 & 0 & \frac{1}{2} & 0 \\
    0 & 0 & 0 & \frac{1}{2}
  \end{pmatrix}; \;
  B' = 
  \begin{pmatrix}
    0 & \frac{1}{2} \\
    \frac{1}{2} & 0
  \end{pmatrix}, \;
  B'' = 
  \begin{pmatrix}
    0 & \frac{1}{2} \\
    \frac{1}{2} & 0
  \end{pmatrix}; \;
  \begin{pmatrix}
    Q' \\ Q''
  \end{pmatrix}
  =
  \begin{pmatrix}
    0 & 1 \\
    0 & 1 
  \end{pmatrix}.
\end{gather*}

%%% Local Variables:
%%% mode: latex
%%% TeX-master: "../main"
%%% End:
