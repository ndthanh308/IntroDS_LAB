\subsection{Family \textnumero2.7}\label{subsection:02-07}

The pencil \(\mathcal{S}\) is defined by the equation
\begin{gather*}
  X^{2} Y^{2} + X^{2} Y Z + 2 X Y^{2} Z + X Y Z^{2} + Y^{2} Z^{2} + 2 X^{2} Y T + 2 X Y^{2} T + X^{2} Z T + 2 Y^{2} Z T + \\ X Z^{2} T + 2 Y Z^{2} T + X^{2} T^{2} + 2 X Y T^{2} + Y^{2} T^{2} + 2 X Z T^{2} + 2 Y Z T^{2} + Z^{2} T^{2} = \lambda X Y Z T.
\end{gather*}
Members \(\mathcal{S}_{\lambda}\) of the pencil are irreducible for any \(\lambda \in \mathbb{P}^1\) except
\begin{gather*}
  \mathcal{S}_{\infty} = S_{(X)} + S_{(Y)} + S_{(Z)} + S_{(T)}, \;
  \mathcal{S}_{- 5} = S_{((X + Z) (Y + T) + Y T)} + S_{((X + T) (Y + Z) + X T + Y Z)}.
\end{gather*}
The base locus of the pencil \(\mathcal{S}\) consists of the following curves:
\begin{gather*}
  C_{1} = C_{(Y, T)}, \;
  C_{2} = C_{(Y, X + Z)}, \;
  C_{3} = C_{(T, X + Z)}, \;
  C_{4} = C_{(X, Y T + Z (Y + T))}, \\
  C_{5} = C_{(Y, X Z + T (X + Z))}, \;
  C_{6} = C_{(Z, Y T + X (Y + T))}, \;
  C_{7} = C_{(T, X Z + Y (X + Z))}.
\end{gather*}
Their linear equivalence classes on the generic member \(\mathcal{S}_{\Bbbk}\) of the pencil satisfy the following relations:
\[
  \begin{pmatrix}
    [C_5] \\ 2 [C_6] \\ [C_7] \\ [H_{\mathcal{S}}]
  \end{pmatrix} =
  \begin{pmatrix}
    -1 & -1 & 0 & 2 \\
    0 & 0 & 0 & 2 \\
    -1 & 0 & -1 & 2 \\
    0 & 0 & 0 & 2
  \end{pmatrix} \cdot
  \begin{pmatrix}
    [C_1] \\ [C_2] \\ [C_3] \\ [C_4]
  \end{pmatrix}.
\]

For a general choice of \(\lambda \in \mathbb{C}\) the surface \(\mathcal{S}_{\lambda}\) has the following singularities:
\begin{itemize}\setlength{\itemindent}{2cm}
\item[\(P_{1} = P_{(X, Y, Z)}\):] type \(\mathbb{D}_4\) with the quadratic term \((X + Y + Z)^2\);
\item[\(P_{2} = P_{(X, Y, T)}\):] type \(\mathbb{A}_3\) with the quadratic term \((Y + T) \cdot (X + Y + T)\);
\item[\(P_{3} = P_{(X, Z, T)}\):] type \(\mathbb{D}_4\) with the quadratic term \((X + Z + T)^2\);
\item[\(P_{4} = P_{(Y, Z, T)}\):] type \(\mathbb{A}_3\) with the quadratic term \((Y + T) \cdot (Y + Z + T)\);
\item[\(P_{5} = P_{(Y, T, X + Z)}\):] type \(\mathbb{A}_1\) with the quadratic term \((X + Z) (Y + T) - (\lambda + 4) Y T\).
\end{itemize}

Galois action on the lattice \(L_{\lambda}\) is trivial. The intersection matrix on \(L_{\lambda} = L_{\mathcal{S}}\) is represented by
\begin{table}[H]
  \setlength{\tabcolsep}{4pt}
  \begin{tabular}{|c||cccc|ccc|cccc|ccc|c|ccccc|}
    \hline
    \(\bullet\) & \(E_1^1\) & \(E_1^2\) & \(E_1^3\) & \(E_1^4\) & \(E_2^1\) & \(E_2^2\) & \(E_2^3\) & \(E_3^1\) & \(E_3^2\) & \(E_3^3\) & \(E_3^4\) & \(E_4^1\) & \(E_4^2\) & \(E_4^3\) & \(E_5^1\) & \(\widetilde{C_{1}}\) & \(\widetilde{C_{2}}\) & \(\widetilde{C_{3}}\) & \(\widetilde{C_{4}}\) & \(\widetilde{C_{6}}\) \\
    \hline
    \hline
    \(\widetilde{C_{1}}\) & \(0\) & \(0\) & \(0\) & \(0\) & \(1\) & \(0\) & \(0\) & \(0\) & \(0\) & \(0\) & \(0\) & \(1\) & \(0\) & \(0\) & \(1\) & \(-2\) & \(0\) & \(0\) & \(0\) & \(0\) \\
    \(\widetilde{C_{2}}\) & \(1\) & \(0\) & \(0\) & \(0\) & \(0\) & \(0\) & \(0\) & \(0\) & \(0\) & \(0\) & \(0\) & \(0\) & \(0\) & \(0\) & \(1\) & \(0\) & \(-2\) & \(0\) & \(0\) & \(0\) \\
    \(\widetilde{C_{3}}\) & \(0\) & \(0\) & \(0\) & \(0\) & \(0\) & \(0\) & \(0\) & \(1\) & \(0\) & \(0\) & \(0\) & \(0\) & \(0\) & \(0\) & \(1\) & \(0\) & \(0\) & \(-2\) & \(0\) & \(0\) \\
    \(\widetilde{C_{4}}\) & \(0\) & \(0\) & \(1\) & \(0\) & \(0\) & \(1\) & \(0\) & \(0\) & \(0\) & \(1\) & \(0\) & \(0\) & \(0\) & \(0\) & \(0\) & \(0\) & \(0\) & \(0\) & \(-2\) & \(0\) \\
    \(\widetilde{C_{6}}\) & \(0\) & \(0\) & \(0\) & \(1\) & \(0\) & \(0\) & \(0\) & \(0\) & \(0\) & \(0\) & \(1\) & \(0\) & \(1\) & \(0\) & \(0\) & \(0\) & \(0\) & \(0\) & \(0\) & \(-2\) \\
    \hline
  \end{tabular}.
\end{table}
Note that the intersection matrix is degenerate. We choose the following integral basis of the lattice \(L_{\lambda}\):
\begin{align*}
  \begin{pmatrix}
    [E_4^3] \\ [\widetilde{C_{6}}]
  \end{pmatrix} =
  \begin{pmatrix}
    0 & 2 & 3 & 1 & -1 & 2 & 1 & 0 & 2 & 3 & 1 & -3 & -2 & -4 & -4 & -2 & -2 & 4 \\
    0 & -1 & -1 & -1 & 1 & 0 & 0 & 0 & -1 & -1 & -1 & 1 & 0 & 2 & 2 & 1 & 1 & -1
  \end{pmatrix} \cdot \\
  \begin{pmatrix}
    [E_1^1] & [E_1^2] & [E_1^3] & [E_1^4] & [E_2^1] & [E_2^2] & [E_2^3] & [E_3^1] & [E_3^2] & \\
    [E_3^3] & [E_3^4] & [E_4^1] & [E_4^2] & [E_5^1] & [\widetilde{C_{1}}] & [\widetilde{C_{2}}] & [\widetilde{C_{3}}] & [\widetilde{C_{4}}]
  \end{pmatrix}^T.
\end{align*}

Discriminant groups and discriminant forms of the lattices \(L_{\mathcal{S}}\) and \(H \oplus \Pic(X)\) are given by
\begin{gather*}
  G' = 
  \begin{pmatrix}
    \frac{1}{2} & 0 & 0 & \frac{1}{2} & \frac{1}{2} & 0 & \frac{1}{2} & \frac{1}{2} & 0 & 0 & \frac{1}{2} & 0 & 0 & \frac{1}{2} & 0 & 0 & 0 & 0 \\
    0 & \frac{1}{4} & \frac{3}{8} & \frac{1}{8} & \frac{1}{2} & 0 & 0 & 0 & \frac{3}{4} & \frac{5}{8} & \frac{7}{8} & 0 & 0 & \frac{1}{2} & 0 & \frac{3}{4} & \frac{1}{4} & \frac{1}{2}
  \end{pmatrix}, \\
  G'' = 
  \begin{pmatrix}
    0 & 0 & \frac{1}{2} & \frac{1}{2} \\
    0 & 0 & \frac{7}{8} & 0
  \end{pmatrix}; \;
  B' = 
  \begin{pmatrix}
    \frac{1}{2} & 0 \\
    0 & \frac{7}{8}
  \end{pmatrix}, \;
  B'' = 
  \begin{pmatrix}
    \frac{1}{2} & 0 \\
    0 & \frac{1}{8}
  \end{pmatrix}; \;
  \begin{pmatrix}
    Q' \\ Q''
  \end{pmatrix}
  =
  \begin{pmatrix}
    \frac{1}{2} & \frac{15}{8} \\
    \frac{3}{2} & \frac{1}{8}    
  \end{pmatrix}.
\end{gather*}

%%% Local Variables:
%%% mode: latex
%%% TeX-master: "../main"
%%% End:
