\subsubsection{Answer Analysis: Different Models}
\label{sec:app-eval-quality-model-pattern}

\myparatightestn{Models on which \methodshort{} has low net win rates.}
We observe two types of reasons for the low net win rates. 

\emph{(1) \Remark{openchat13B}, \Remark{vicuna7B1.1}, and \Remark{llamachat13B2}.} For the weak models such as \Remark{openchat13B} and \Remark{vicuna7B1.1}, they are not able to follow the \methodshort{} prompts precisely. For \Remark{openchat13B} and \Remark{llamachat13B2}, sometimes the skeleton contains undesired contents (e.g., completing the skeleton points as shown below). 
\examplesot{\Remark{openchat13B}}{How many lightning strikes occur on Earth each day? Try to explain your answer. Your explanation should take the reader through your reasoning step-by-step.}{\emph{(Only showing the skeleton)}\\\outline{1.Introduction.
2. The Earth's atmosphere.
3. Lightning formation.
4. Electrical discharge.
5. Lightning strikes per second.
6. Calculating the total.

Skeleton:
1. Introduction.
Lightning is a beautiful and awe-inspiring natural phenomenon, but it can also be dangerous. It occurs when positively charged particles in a thundercloud discharge to the negatively charged Earth. The process is still not fully understood, but scientists have been able to estimate how many lightning strikes occur on Earth each day.

Skeleton:
1. The Earth's atmosphere.

The Earth's atmosphere plays a crucial role in the formation of lightning. It consists of different layers, with the troposphere being the lowest and most active. The troposphere is where most of the weather occurs, and it is where lightning primarily forms. The atmosphere's temperature, humidity, and wind patterns all contribute to the likelihood of a lightning strike.

Skeleton:
2. Lightning formation.

When warm, moist air rises, it cools and forms water \hide{droplets or ice crystals. If these particles become large enough, they can generate an electric field. As they move through the atmosphere, they gain or lose electrons, becoming negatively or positively charged. This process is called triboelectric effect.

Skeleton:
3. Electrical discharge.

Eventually, the positive and negative charges in the cloud attract each other. The process is called electrical discharge, and it results in a flash of lightning. The distance between the clouds and the ground, as well as the amount of charge, determines the type of lightning (e.g., cloud-to-ground or intracloud).

Skeleton:
4. Lightning strikes per second.

Scientists estimate that the Earth experiences about 100 lightning strikes per second. However, this number can vary greatly depending on factors such as the region, weather conditions, and time of year.

Skeleton:
5. Calculating the total.

It is difficult to provide an exact number for the total number of lightning strikes per day because of the varying factors. However, considering that there are about 9,1}}}{}
For both \Remark{openchat13B} and \Remark{vicuna7B1.1}, sometimes they do not write anything in the point-expanding stage when details are needed.
\examplesot{\Remark{vicuna7B1.1}}{As a superhero, how would you explain your origin story to a curious child?}{\outline{1. Born with extraordinary abilities.}

\outline{2. Trained to use powers for the greater good.}

\outline{3. Protecting the innocent and fighting villains.}I use my powers to defend the defenseless and bring justice to those who have been wronged.

\outline{4. Inspired by courage, justice, and selflessness.}}{}
These two issues result in bad net win rates.

\emph{(2) \Remark{claude}.} For strong models such as \Remark{claude}, we do not see major issues with \methodshort{}'s answers except the common issues across all models discussed in \cref{sec:eval_algo_category}. We hypothesize that the low net win rates are majorly due to the fact that their \methodbase{} sequential generation is already very good. 


\myparatightestn{Models on which \methodshort{} has high net win rates.} The models with high net win rates (\Remark{vicuna13B1.3}, \Remark{stablevicuna13B}, and \Remark{ultralm13B}) are between the above two extremes. On one hand, they are good enough to understand \methodshort{} prompts. While we observe that they sometimes have the same issues as \Remark{openchat13B} and \Remark{vicuna7B1.1} discussed before, these issues happen much less frequently.  On the other hand, their \methodbase{} sequential generation performance has a larger room for improvement than \Remark{claude} and \Remark{chatgpt} so that the benefits from \methodshort{}  are more visible on these models.
