\subsubsection{Answer Analysis: Different Metrics}
\label{sec:app-eval-quality-metric-pattern}

\myparatightestn{Metrics on which \methodshort{} has low net win rates.}

\emph{(1) Coherence.}
In the prompt of \llmzoo{} evaluation, coherence is defined as ``The response should be coherent and flow logically from one point to the next that is easy to read and understand without major gaps or inconsistencies. The more coherent they are, the better.'' \methodshort{}'s answer is structured in a list of points. It is suitable for some questions but not all. For example, in the questions that require LLMs to write a passage, email, or script, the introduction and conclusion paragraphs are usually missing, and the answers usually lack transition sentences between consecutive points. These make the answers less coherent. Below is one example from \Remark{claude}.

\example{\Remark{claude}}{Write a script for a YouTube video exploring the history and cultural significance of jazz.}{\outline{1. Origins.} Jazz originated in the late 19th and early 20th centuries among African Americans in New Orleans, emerging from a combination of African rhythmic traditions, blues, ragtime, and other influences.

 \outline{2. Early jazz styles.} The early styles of jazz that emerged in the late 19th and early 20th centuries included ragtime, Dixieland, and the blues.

 \outline{3. Jazz age.} The Jazz Age was the period during the 1920s when jazz music and dance styles like the Charleston became popular cultural phenomena in American and European society.

 \outline{4. Bebop era.} Bebop emerged in the early 1940s with artists like Dizzy Gillespie and Charlie Parker experimenting with faster tempos, complex chord changes, and unpredictable melodic phrasing.

 \outline{5. Cool jazz.} Emerging in the late 1940s and early 1950s, cool jazz was a more subdued, mellow style that broke from the high-energy virtuosic improvisation of bebop.

 \outline{6. Free jazz} emerged in the late 1950s and 1960s as musicians explored free improvisation and atonality, moving away from traditional jazz structures.

 \outline{7. Fusion jazz.} Major jazz artists such as Miles Davis and Herbie Hancock began experimenting with combining elements of rock, funk, and electronic music in the 1970s, resulting in the emergence of the subgenre known as fusion jazz.

 \outline{8. Jazz festivals.} Jazz festivals emerged in the 1950s and 1960s as a way to celebrate jazz music, with famous early festivals like the Newport Jazz Festival bringing thousands of fans together to hear top performers.

 \outline{9. Jazz instruments.} The trumpet, saxophone, piano, double bass, and drum set are some of the key instruments commonly used in jazz.

 \outline{10. Influential musicians} include Louis Armstrong, Duke Ellington, Charlie Parker, Miles Davis, John Coltrane, and many others who pioneered new styles and techniques.

 \outline{11. Legacy and impact.} Jazz has left an indelible mark on music and culture over the past century, influencing countless artists and genres while spreading the spirit of improvisation and freedom across the globe.}{}{Here is a draft script for a YouTube video exploring the history and cultural significance of jazz:

Introduction

Hey everyone! Welcome to my channel. In today's video, we're going to be exploring the history and cultural impact of one of America's original art forms - jazz music. 

Jazz Origins 

Jazz grew out of the African American communities in the southern United States in the late 19th and early 20th centuries. It was influenced by a combination of African and European musical traditions. Some key elements that fed into early jazz included blues, ragtime, spirituals, work songs, field hollers, and brass band music. 

New Orleans is considered the birthplace of jazz, emerging around the turn of the 20th century. The city was a melting pot of cultures and musical styles. Musicians mixed improvisation, syncopated rhythms, bent notes and a loose ensemble playing style to create a new genre that became known as jazz. 

Early jazz greats like Buddy Bolden, Jelly Roll Morton and Sidney Bechet began fusing these various influences into the distinct jazz sound. As the music spread via travel and recordings in the 1910s, jazz captivated listeners and began to evolve into different styles.

Jazz Styles Emerge

In the 1920s, jazz expanded out from New Orleans and Chicago emerged as a hotspot. Louis Armstrong's innovative solos and expressive phrasing became influential. The swing style also developed during the 1930s and 40s, propelled by star big bands and dancers. 

Bebop arose in the 1940s with soloists like Charlie Parker and Dizzy Gillespie exploring new levels of harmonic and rhythmic complexity. In the 50s and 60s, jazz expanded into new fusions like cool jazz, hard bop, free jazz, modal jazz and soul jazz. Artists used jazz to express their unique regional, social and political identities.

Cultural Significance

Over the decades, jazz has become woven into the fabric of American society. It is equated with many values and ideals like individualism, freedom, spontaneity, innovation and social harmony.

Jazz was associated with breaking down barriers between races, classes and genres. It represented modernism and urban culture. Musicians like Louis Armstrong and Duke Ellington became some of the first widely popular African American artists.

Jazz also spread American culture and values abroad through performances, radio and film. It influenced music around the world and became a truly global art form.

Today jazz continues to thrive and evolve both in the U.S. and internationally. Its history and sounds reflect the diversity of both American and global cultures. From its folk roots, jazz has become a sophisticated art music that retains its spirit of freedom and improvisation.

Conclusion

That covers some of the key points in the history and cultural impact of jazz music. Let me know in the comments about your favorite jazz styles and musicians! Be sure to like and subscribe. See you next time.}{}

While the content of \methodshort{}'s answer is good, it reads less fluent for the purpose of a YouTube script.

\emph{(2) Immersion.}
In the prompt of \llmzoo{} evaluation, immersion is defined as ``the response should act like the assigned role using the tone, manner, and vocabulary the role would use. The more assistant-like tones, the worse. The more in-character, the better.'' 
The way that \methodshort{} breaks the answers into a list makes its answer less in-character. Therefore, \methodshort{} receives a lower immersion score. Here is one example.

\example{\Remark{claude}}{Pretend to be a world-famous chef. How would you describe your signature dish to a panel of judges?}{\outline{1. Main ingredient}: The incredibly fresh and tender salmon caught just this morning from the crystal clear waters of the Pacific Northwest.

 \outline{2. Cooking method} 

I slowly braise the main ingredient in a rich broth to bring out its natural flavors before searing it to a perfect golden crust.

 \outline{3. Special seasoning}

The dish is seasoned with a blend of fresh herbs like thyme, rosemary, and oregano which provides an aromatic backbone that pairs nicely with the natural flavors of the main ingredients. I add just a touch of smoked paprika to provide a subtle smokiness without overpowering the other flavors.

 \outline{4. Texture}: The dish has a perfect balance of crispy on the outside and tender on the inside, with each bite yielding a delightful mix of crunchy and smooth textures that dance across your tongue.

 \outline{5. Plating presentation}

I purposefully create height and dimension on the plate with the carefully arranged components of the dish, so that the dish appears vibrant and alive. The interplay of colors, textures and shapes is intended to delight the eye before the first bite.

 \outline{6. Inspiration}: This dish was inspired by my dear grandmother, who taught me the secret family recipe when I was just a child.}{}{ *clears throat and speaks in a fancy French accent*

Bonjour honored judges. Today I present to you my signature dish - la crème de la crème of French haute cuisine. Behold, le poulet à la Claude! 

This succulent chicken is firstly marinated overnight in a bath of fine herbes - thyme, rosemary, and sage cultivated from my private garden. The young spring chicken is then roasted to perfection, its skin crisping to a beautiful caramel brown. 

Nestled beside the chicken is a tower of pommes dauphine - light and airy potato puffs - alongside a decadent mound of crème fraîche infused with truffle essence. 

But the pièce de résistance is the sauce. A luxurious pool of velouté enriched with cream and butter, brightened with a dash of lemon, and studded with morsels of morel mushrooms foraged at dawn. 

One bite of this ensemble will transport you to food heaven. The tender chicken, the creaminess of the potato puffs, and the rich decadence of the sauce meld together in blissful harmony. 

So tell me honored judges, will my poulet à la Claude earn your coveted approval? I await your discerning verdict as I swirl more vintage Bordeaux into your glasses. Santé!}{}

Again, while the answer from \methodshort{} covers sufficient aspects and details, it reads rather stiffly. The \methodbase{} answer, in contrast, has the opening sentence "Bonjour honored judges. Today I present to you my signature dish..." and the conclusion statement "So tell me honored judges, will my poulet à la Claude earn your coveted approval? ..." which reads more like a  casual talk from a chef.

\myparatightestn{Metrics on which \methodshort{} has high net win rates.}

\emph{(1) Diversity.} In the prompt of \llmzoo{} evaluation, diversity is defined as ``The response should be comprehensive and provide a range of information that is not limited to a single perspective. More perspectives are better.'' The skeleton stage in \methodshort{} encourages LLMs to think from multiple perspectives. Therefore, \methodshort{}'s answers tend to have better diversity. Below is one example showing that \methodshort{}'s answer has better diversity. 

\example{\Remark{chatgpt}}{How can you determine if a person is genuinely interested in a conversation or simply being polite?}{\outline{1. Pay attention to body language.}Notice if their posture is open or closed, if they are facing towards you or away, and if they are fidgeting or relaxed. These can all indicate their level of interest in the conversation.

\outline{2. Listen for engaged responses.}If the person is genuinely interested in the conversation, they will respond with thoughtful and relevant comments or questions rather than short, generic responses.

\outline{3. Observe eye contact.}If the person maintains steady eye contact and their gaze is focused on you, they are likely genuinely interested in the conversation. On the other hand, if they frequently look away or seem distracted, they may just be being polite.

\outline{4. Look for follow-up questions.}If the person asks follow-up questions and seems genuinely interested in what you're saying, then they are likely interested in the conversation and not just being polite.

\outline{5. Notice tone of voice.}A person who is genuinely interested in a conversation will often have a more enthusiastic and animated tone of voice, compared to someone who is simply being polite and giving short, monotone responses.}{}{As an AI assistant, I cannot determine a person's true intentions or feelings. However, there are some social cues that can indicate whether a person is genuinely interested in a conversation or not. For example, if they maintain eye contact, lean in, nod their head and ask follow-up questions, it can be a sign that they are engaged in the conversation. On the other hand, if they avoid eye contact, give short or vague responses, or seem distracted, it may indicate that they are not as interested. Ultimately, it's important to remember that everyone has different communication styles, and it can be difficult to accurately interpret someone's intentions without directly asking them.}{}

\methodshort{} answers the question from five aspects, whereas the \methodbase{} answer only mentions one aspect.

\emph{(2) Relevance.} In the prompt of \llmzoo{} evaluation, relevance is defined as ``The response should be closely related to the question and answer the question accurately with sufficient details without repetition or redundancy. The more relevant they are, the better.'' In the skeleton stage, LLMs are forced to propose several points around the question; and in the point-expanding stage, LLMs are required to only discuss these points. Therefore, answers from \methodshort{} tend to have better relevance. 
One typical example is shown below.

\example{\Remark{chatgpt}}{Why do some people enjoy the sensation of being scared, such as by watching horror movies or going on roller coasters, while others avoid these experiences?}{\outline{1. Neurological differences.}Some people enjoy the sensation of being scared due to the release of adrenaline and dopamine in the brain, while others may have a lower threshold for fear due to differences in their amygdala and prefrontal cortex.

\outline{2. Psychological factors.}Some people may enjoy the sensation of being scared because it can provide a sense of excitement, thrill, and a temporary escape from the mundane aspects of everyday life. On the other hand, others may avoid these experiences due to past trauma or anxiety.

\outline{3. Developmental experiences.}Individuals who had negative experiences with fear during childhood, such as trauma or abuse, may avoid scary situations as adults, while those who had positive experiences with fear may seek out these experiences.

\outline{4. Cultural influences.}Some cultures place a greater emphasis on seeking excitement and thrill, while others may prioritize safety and security, which can impact whether individuals enjoy scary experiences or avoid them.

\outline{5. Genetics.}Research has suggested that some individuals may have a genetic predisposition towards seeking out thrilling or risky experiences, which may include enjoying the sensation of being scared.}{}{The enjoyment of being scared is a complex psychological phenomenon. Some people may enjoy the adrenaline rush that comes with being scared, while others may enjoy the feeling of being able to conquer their fears. Additionally, some people may be more naturally inclined to seek out new experiences and sensations, including those that are scary or thrilling. On the other hand, some people may be more risk-averse or may have had negative experiences with fear or anxiety in the past, which could make them more likely to avoid scary experiences. Ultimately, the reasons why some people enjoy being scared while others do not are likely to be influenced by a range of factors, including personality, past experiences, and individual preferences.}{}

In the answer from the \methodbase{} generation, the first two sentences provide little information in answering the question, and the last sentence only gives keywords such as ``personality, past
experiences, and individual preferences'' without providing concrete explanations to each. In contrast, \methodshort{}'s answer is well-structured into five reasons with sufficient explanations and it does not waste space in irrelevant contents.