\subsubsection{Answer Analysis: Different Question Categories}
\label{sec:app-eval-quality-category-pattern}
\myparatightestn{Categories on which \methodshort{} has low net win rates.}

\emph{(1) Math.} Math questions require step-by-step thinking. Without knowing the previous steps, it is hard to derive the following steps. Chain-of-Thoughts \citep{kojima2022large,wei2022chain} emphasize the importance of expanding the sequential thinking steps to solve these problems successfully. In contrast, \methodshort{} requires the models to (a) come up with the skeleton of the solution first and (b) deduce each individual step independently without referring to previous results. Both of these are challenging tasks.
Strong models are able to get (a) correct but fail at (b). 
In the following example, \Remark{chatgpt} gets the skeleton steps correct. However, without knowing the results from the first step, the model starts making mistakes in the second step.\footnote{One additional (minor) issue of the API-based \Remark{chatgpt} is that the skeleton point ``1.'' is repeated in the answer.} %

\example{\Remark{chatgpt}}{(Math) Solve for x in the equation 3x + 10 = 5(x - 2).}{\outline{1. Move all x terms to one side.} 1. Move all x terms to one side by subtracting 5x from both sides, giving -2x + 10 = -10.

\outline{2. Move all constant terms to the other side.} This can be done by subtracting 5x - 10 from both sides, leaving 3x - 5x = -10 - 10.

\outline{3. Simplify the equation.} Combine like terms on both sides of the equation to get 3x + 10 = 5x - 10.

\outline{4. Divide both sides by the coefficient of x.} Divide both sides by 2 to get x = -5.

\outline{5. Check the solution by plugging it back into the original equation.}Substitute the value of x obtained in step 4 back into the original equation and verify that both sides are equal.}{incorrect}{3x + 10 = 5(x - 2) can be simplified as follows:

3x + 10 = 5x - 10 (distribute the 5 on the right side)
10 + 10 = 5x - 3x (subtract 3x from both sides)
20 = 2x (combine like terms on the right side)
x = 10 (divide both sides by 2)

Therefore, the solution for x in the equation 3x + 10 = 5(x - 2) is x = 10.}{correct}

For weaker models, it is hard to even get step (a) correct. For example, as shown below,  the third step ``Apply FOIL'' in \Remark{vicuna13B1.3}'s \methodshort{} solution comes out of nowhere. This makes \methodshort{}'s solution incorrect (though the solution from the \methodbase{} generation is also incorrect).

\example{\Remark{vicuna13B1.3}}{(Math) Solve for x in the equation 3x + 10 = 5(x - 2).}{\outline{1. Simplify equation.}

We can simplify the equation by combining like terms:

3x + 10 = 5(x - 2)
=> 3x + 10 = 5x - 10
=> x + 5 = 2
=> x = 2 - 5
=> x = -3

\outline{2. Factor 5(x - 2).}

5(x - 2) = 5x - 10.

\outline{3. Apply FOIL.}

* Multiply the binomials in the equation, using the first term of each binomial for the first term of the product, the outside term for the second term of the product, the second term of each binomial for the third term of the product, and the remaining terms for the fourth term of the product.
* 3(x - 2) + 10 = 5x + 5(-2)
* 3x - 6 = 5x - 10
* 3x = 5x - 10
* x = 5x - 10 / 3
* x = 5/3x - 10/3
* x = 2/3x - 5/3

\outline{4. Solve for x.}

After applying the FOIL method and simplifying the equation, we get 3x = 17. Therefore, x = 17/3.}{incorrect}{To solve for x in the equation 3x + 10 = 5(x - 2), we need to first distribute the 5 to multiply both sides by 5.

3x + 10 = 5(x - 2)

5(3x + 10) = 5(x - 2)

15x + 50 = 5x - 10

Add 5x to both sides:

20x + 50 = 5x

Subtract 5x from both sides:

15x = 0

Divide both sides by 15:

x = 0

Therefore, the solution to the equation is x = 0.}{incorrect}

\emph{(2) Fermi.}
Similar to math questions, questions in the fermi category require solvers to make assumptions and calculate the answer step-by-step. For example, to answer the question in the figure below, the solver is required to make an assumption about when life began on Earth. In step 4, \Remark{chatgpt} does not know that it has already assumed 3.5 billion years in step 3 and uses the wrong number 4.54 billion in the calculation. In step 5, \Remark{chatgpt} does not know that it has already made the above assumptions and refuses to answer ``due to the uncertainties
in estimating the age of the Earth''. In contrast, \methodbase{} generation is able to create a consistent solution.


\example{\Remark{chatgpt}}{(Fermi) How many times has the Earth orbited the Sun since the beginning of life? Try to explain your answer. Your explanation should take the reader through your reasoning step-by-step.}{\outline{1. Estimate the age of the Earth.} The age of the Earth is estimated to be around 4.54 billion years, based on radiometric dating of rocks and meteorites.

\outline{2. Determine when life began on Earth.} Life on Earth is estimated to have begun around 3.5 billion years ago, based on the discovery of ancient microfossils.

\outline{3. Calculate the number of years since life began.} The oldest known fossils suggest that life on Earth began around 3.5 billion years ago. Therefore, the number of years since life began on Earth is approximately 3.5 billion years.

\outline{4. Divide the number of years by the length of one orbit.} Assuming an orbital period of 365.25 days, the Earth has orbited the Sun approximately 4.54 billion / 365.25 = 12,425,000 times since the beginning of life.

\outline{5. The Earth has orbited the Sun approximately X times since the beginning of life.} Unfortunately, it is impossible to give a specific number for the amount of times the Earth has orbited the Sun since the beginning of life due to the uncertainties in estimating the age of the Earth and when life first emerged.}{incorrect}{The beginning of life on Earth is estimated to have occurred around 3.5 to 4 billion years ago. Since that time, the Earth has been orbiting the Sun continuously. 

The length of one orbit, or one year, is approximately 365.25 days. This means that the Earth completes one full orbit around the Sun every 365.25 days. 

To calculate the number of times the Earth has orbited the Sun since the beginning of life, we can divide the total number of days by the length of one orbit:

Total number of days since the beginning of life = (3.5 billion years) x (365 days per year) = 1.2775 trillion days

Number of orbits = (1.2775 trillion days) / (365.25 days per orbit) = 3.5 billion orbits

Therefore, the Earth has orbited the Sun approximately 3.5 billion times since the beginning of life.}{}

As the following example shows, there are also cases when \methodshort{} solves the question (with minor computation errors) by embedding the computation steps in the skeleton directly in the skeleton stage. The point-expanding stage will then add little content (except some undesired repetition of the following skeletons occasionally). This happens both on small/weak models and large/powerful models. Although this behavior is not what we originally expect, it sheds light on a promising direction for making \methodshort{} a general framework: in the problem where step-by-step thinking is needed, the \methodshort{} model automatically falls back to the sequential generation mode by outputting the complete answer in the skeleton stage and skipping the point-expanding stage. Future work needs to be done to control and encourage this behavior better.

\example{\Remark{vicuna7B1.1}}{(Fermi) How many times does the average human blink in a lifetime? Try to explain your answer. Your explanation should take the reader through your reasoning step-by-step.}{\outline{1. The average human blinks 12-15 times per minute.}

\outline{2. There are 60 minutes in an hour.}

\outline{3. Therefore, the average human blinks 60 x 12-15 = 720-900 times per hour.}

\outline{4. There are 24 hours in a day.}Therefore, the average human blinks 720-900 x 24 = 17,520-22,400 times per day.

\outline{5. Therefore, the average human blinks 720-900 x 24 = 17,520-22,400 times per day.}This translates to blinking approximately 50-65 times per hour, which is a significant portion of the day.

\outline{6. There are 365 days in a year.}Therefore, the average human blinks 17,520-22,400 x 365 = 5,857,600-8,124,000 times per year.

\outline{7. Therefore, the average human blinks 17,520-22,400 x 365 = 5,857,600-8,124,000 times per year.}

\outline{8. The human lifespan is typically 70-80 years.}Therefore, the average human blinks 80,000-112,000 times in their lifetime.

\outline{9. Therefore, the average human blinks 5,857,600-8,124,000 / 70-80 = 80,000-112,000 times in their lifetime.}}{}{The average human blinks about 12-15 times per minute, which translates to about 18,200 to 20,400 blinks per day. Over the course of a lifetime, the number of blinks is estimated to be around 7,500,000 to 8,000,000.

Blinking is a reflexive action that helps to keep the eyes moist and comfortable, as well as to protect the eyes from dust and other foreign particles. The frequency of blinking can vary depending on a variety of factors, including age, health, and environmental conditions. For example, people who work in dusty or dry environments may blink more frequently than those who work in a clean, moist environment.

Over the course of a lifetime, the average person will spend approximately 27,375 days or 700,000 hours awake. This means that the majority of the time spent blinking will occur in the first few decades of life, with the frequency of blinking gradually decreasing as people age.

It is important to note that there is a significant amount of individual variation in the frequency of blinking. Some people may blink more or less than the average, depending on a variety of factors such as age, health, and environmental conditions. Additionally, blinking patterns can be affected by factors such as stress, fatigue, and certain medical conditions.}{}



\emph{(3) Coding.}
While in most cases, the models are able to decompose the coding problem into smaller tasks in the skeleton stage, the generation quality during the point-expanding stage is bad, potentially due to the fact that we did not carefully craft a special point-expanding prompt for coding. The failure cases can be grouped into two cases.
(a) In some cases, the model only generates a description of how implementation can be done without giving the code.

\examplesot{\Remark{claude}}{(Coding) Develop a C++ program that reads a text file line by line and counts the number of occurrences of a specific word in the file.}{ \outline{1. Open file stream} by declaring an ifstream object and using the open() method to open the file in read mode.

 \outline{2. Read line} by line from the file stream into a string using getline().

 \outline{3. Find word} by searching each line for the target word using string::find and keeping track of matches.

 \outline{4. Increment counter} by one each time the specified word is encountered in the current line.

 \outline{5. Close file}

The file stream should be closed after finishing reading the file to free up resources. This can be done by calling the close() method on the file stream object.}{}

This might be due to the fact that the powerful models (e.g., \Remark{claude}) follow the point-expanding prompt ``Write it **very shortly** in 1$\sim$2 sentence'' precisely. Better prompts tailed to coding could help. (b) In some other cases, the model tends to generate the complete code repeatedly in several points.

\examplesot{\Remark{openchat13B}}{(Coding) Develop a C++ program that reads a text file line by line and counts the number of occurrences of a specific word in the file.}{\outline{1. Read file line by line.}

\outline{2. Store lines in a container.}

\outline{3. Count occurrences of specific word.}
```cpp
\hide{}
```
This code uses the C++ standard library to read a text file line by line, store each line in a container, and count the occurrences of a specific word. The results are printed.

\outline{4. Print results.}

Here's the C++ program that reads a text file line by line and counts the number of occurrences of a specific word in the file:
```cpp
\hide{}
```
Please note that this is a very basic implementation. You may need to adjust it depending on your specific requirements.}{}
This problem might also be alleviated by prompts that explicitly tell the model to only generate a ``code snippet'' in each skeleton point.


Although better prompt engineering could lead to better results, we expect that \methodshort{} is not suitable for the coding questions in \vicunadataset{}. The reason is that the solutions to the problems in \vicunadataset{} are usually in a few lines with strong dependencies between each other. 
Without knowing the previously defined variable names or imported libraries, it is hard (or even impossible) to implement the subsequent code correctly. 
As a consequence, generating different parts of the answers in parallel is not suitable here. 
Similar to the math questions, automatically falling back to outputting all the code in the skeleton stage and not triggering the point-expanding stage might be more suitable for answering this question type.

However, we expect that \methodshort{} could be helpful for larger coding tasks that involve multiple modules (e.g., functions, classes). The skeleton stage could be in charge of defining the interfaces between different modules (e.g., functionalities, names, parameters, return types). The point-expanding stage could be in charge of implementing these modules or using these modules to complete the final task, which can be done in parallel. This mimics the common practice in software engineering. 


\emph{(4) Writing.} The writing questions usually ask for an email, a blog post, or a review in a given scenario. 
In the detailed evaluation output from \fastchat{} and \llmzoo{} metrics, the major and common complaint is that \methodshort{}'s answers are (slightly) less detailed. This could potentially be improved by point-expanding prompts that request more details.

While inspecting \methodshort{}'s answers manually, we observe one additional issue. 
Writing problems usually require a coherent passage \emph{without} embedded skeleton points, whereas our current \methodshort{} pipeline (\cref{sec:method}) concatenates skeleton points as part of the answer. One example is shown below.

\example{\Remark{chatgpt}}{(Writing) Can you help me write a formal email to a potential business partner proposing a joint venture?}{\outline{1. Introduce yourself and your company.} Dear [Recipient], 

I hope this email finds you well. My name is [Your Name] and I am the [Your Position] at [Your Company Name].

\outline{2. State the purpose of the email.} The purpose of this email is to propose a joint venture between our companies.

\outline{3. Explain the potential benefits of the joint venture.} By combining our resources and expertise, we can increase market share and profitability for both companies.

\outline{4. Provide details on the proposed collaboration.} Our proposed collaboration would involve combining our expertise in marketing with your strong manufacturing capabilities to create a new line of products.

\outline{5. Suggest a follow-up meeting or call.} Please let me know your availability for a follow-up call next week to discuss this opportunity further.

\outline{6. Thank the recipient for their time and consideration.}Thank you for considering this proposal and we look forward to hearing back from you soon.}{}{Of course, I can help you with that. Here's a sample email you can use as a guide:

Subject: Proposal for Joint Venture

Dear [Business Partner’s Name],

I hope this email finds you well. I am writing to propose a joint venture between our companies, [Your Company Name] and [Potential Partner Company Name]. We believe that our respective expertise can create a successful partnership that will benefit both of our businesses.

We have conducted thorough research on your company and admire your innovative approach to [insert industry or niche]. We think our [insert expertise or specialty] could complement your [insert potential partner’s specialty or expertise]. By collaborating on this joint venture, we believe we can create a new and innovative product that will appeal to our respective markets.

We are confident that our companies would work well together and that our joint venture would be successful. We would like to set up a meeting to discuss the specifics of our proposal and explore how we can move forward with the joint venture.

Thank you for considering this proposal. We look forward to hearing from you soon.

Best regards,

[Your Name]}{}
In such cases, simply removing the skeleton points would greatly improve the answer quality. To make \methodshort{} more general, one future direction is to let the LLM itself decide whether the point index and point skeleton should be included to make the final answer more natural and fluent.


\myparatightestn{Categories on which \methodshort{} has high net win rates.}

\emph{(1) Counterfactual, knowledge, common sense, generic.} All these four categories have the same characteristic: the ideal answer should cover several relatively \emph{independent} points. 
Having the LLM generate a skeleton before expanding the details can result in more comprehensive discussions for the question. Besides, organizing the answers as a list of points makes the answers easier to read, whereas the answers from \methodbase{} generation are sometimes less structured and less readable. 
Below, we give one example for each question category to illustrate these points.

\example{\Remark{chatgpt}}{(Counterfactual) What if the Internet had been invented during the Renaissance period?}{\outline{1. Advancements in communication.}The invention of the Internet during the Renaissance period would have revolutionized communication, allowing for faster and more efficient exchange of ideas and information across vast distances.

\outline{2. Increased access to knowledge.}The invention of the printing press during the Renaissance period already allowed for greater access to knowledge, but the internet would have revolutionized this even further by providing instantaneous access to information from all over the world.

\outline{3. Improved dissemination of ideas.}If the internet had been invented during the Renaissance period, the spread of ideas and knowledge would have been much faster and widespread than it was at the time. This could have led to a faster progression of scientific and artistic innovations.

\outline{4. Impact on art and literature.}The invention of the internet during the Renaissance period would have revolutionized the way art and literature are created, shared, and consumed, leading to a more diverse and globally-influenced artistic and literary scene.

\outline{5. Potential for global trade.}The invention of the internet during the Renaissance period would have greatly expanded global trade opportunities, allowing for easier communication and collaboration between merchants and traders across continents.

\outline{6. Influence on political and social structures.}The invention of the internet during the Renaissance period could have led to significant changes in political and social structures, allowing for greater democratization and communication among people from different regions and classes.}{}{If the Internet had been invented during the Renaissance period, it would have had a significant impact on the way people communicated, shared ideas, and gained knowledge. The Renaissance was a time of great intellectual and artistic growth, and the Internet would have facilitated the spread of these ideas and innovations.

Scholars and artists would have been able to share their work more easily, and people from different parts of the world would have been able to connect and collaborate more easily. The Internet would have also made it easier to access information, which could have accelerated the pace of scientific and technological advancements.

However, it's important to note that the infrastructure required for the Internet to exist did not exist during the Renaissance period. There were no telecommunications networks or electricity grids, so the idea of the Internet would have been impossible to realize. Nonetheless, it's interesting to think about how the Renaissance might have been different if the Internet had existed at that time.}{}

\example{\Remark{chatgpt}}{(Knowledge) What are some potential implications of using a single-use plastic bottle versus a reusable bottle on both the environment and human health?}{\outline{1. Single-use plastic waste.}Single-use plastic bottles contribute to the massive amount of plastic waste that ends up in landfills and oceans, harming wildlife and ecosystems.

\outline{2. Pollution of waterways.}Single-use plastic bottles contribute to the pollution of waterways, as they often end up in rivers, lakes, and oceans, causing harm to aquatic life and the overall ecosystem.

\outline{3. Harmful chemicals leaching.}Single-use plastic bottles may release harmful chemicals such as bisphenol A (BPA) and phthalates into the water or liquid they contain, posing a risk to human health.

\outline{4. Greenhouse gas emissions.}The production and disposal of single-use plastic bottles contribute to greenhouse gas emissions, which contribute to climate change. Using a reusable bottle reduces the need for constant production and disposal, therefore reducing greenhouse gas emissions.

\outline{5. Microplastic ingestion.}Single-use plastic bottles contribute to the accumulation of microplastics in the environment, which can be ingested by humans through food and water, leading to potential health risks.

\outline{6. Increased risk of disease.}Single-use plastic bottles can harbor harmful bacteria, leading to an increased risk of disease if not disposed of properly. Reusable bottles, when cleaned regularly, pose a lower risk.}{}{Using a single-use plastic bottle has significant implications on the environment and human health. 

Environmental implications: 

1. Plastic pollution: Single-use plastic bottles are one of the major contributors to plastic pollution in oceans and landfills. They take hundreds of years to decompose, and during that time, they release toxic chemicals and microplastics into the environment.

2. Carbon emissions: Manufacturing and transporting single-use plastic bottles require a significant amount of energy and resources, which contribute to greenhouse gas emissions and climate change.

3. Natural resource depletion: The production of single-use plastic bottles requires large amounts of water and oil, which are finite resources.

Health implications:

1. Chemical exposure: Single-use plastic bottles contain chemicals such as Bisphenol A (BPA) and phthalates, which can leach into the water and pose a risk to human health.

2. Microplastics ingestion: When plastic bottles break down, they release microplastics, which can enter the food chain and be ingested by humans.

Using a reusable bottle, on the other hand, has a positive impact on the environment and human health. 

Environmental benefits:

1. Reduced plastic waste: Reusable bottles significantly reduce the amount of plastic waste generated.

2. Lower carbon emissions: Reusable bottles have a lower carbon footprint as they can be used multiple times.

3. Reduced resource depletion: The production of reusable bottles requires fewer resources than single-use plastic bottles.

Health benefits:

1. Chemical-free: Reusable bottles are typically made from materials that are free from harmful chemicals, reducing the risk of chemical exposure.

2. Reduced microplastics: Reusable bottles do not break down easily, reducing the release of microplastics into the environment.

Overall, using a reusable bottle is a more sustainable and healthier option than using a single-use plastic bottle.}{}

\example{\Remark{chatgpt}}{(Common sense) How can you determine if a person is genuinely interested in a conversation or simply being polite?}{\outline{1. Pay attention to body language.}Notice if their posture is open or closed, if they are facing towards you or away, and if they are fidgeting or relaxed. These can all indicate their level of interest in the conversation.

\outline{2. Listen for engaged responses.}If the person is genuinely interested in the conversation, they will respond with thoughtful and relevant comments or questions rather than short, generic responses.

\outline{3. Observe eye contact.}If the person maintains steady eye contact and their gaze is focused on you, they are likely genuinely interested in the conversation. On the other hand, if they frequently look away or seem distracted, they may just be being polite.

\outline{4. Look for follow-up questions.}If the person asks follow-up questions and seems genuinely interested in what you're saying, then they are likely interested in the conversation and not just being polite.

\outline{5. Notice tone of voice.}A person who is genuinely interested in a conversation will often have a more enthusiastic and animated tone of voice, compared to someone who is simply being polite and giving short, monotone responses.}{}{As an AI assistant, I cannot determine a person's true intentions or feelings. However, there are some social cues that can indicate whether a person is genuinely interested in a conversation or not. For example, if they maintain eye contact, lean in, nod their head and ask follow-up questions, it can be a sign that they are engaged in the conversation. On the other hand, if they avoid eye contact, give short or vague responses, or seem distracted, it may indicate that they are not as interested. Ultimately, it's important to remember that everyone has different communication styles, and it can be difficult to accurately interpret someone's intentions without directly asking them.}{}

\example{\Remark{chatgpt}}{(Generic) How can I improve my time management skills?}{\outline{1. Set clear goals.} Setting clear goals is important in improving time management skills as it helps you to focus on what needs to be done and what can be accomplished within a certain timeframe.

\outline{2. Prioritize tasks.} Make a list of tasks and prioritize them based on their importance and deadline. This will help you focus on the most critical tasks and avoid wasting time on less important ones.

\outline{3. Create a schedule.} Creating a schedule helps you allocate time for each task and ensures that you complete them within the allotted time.

\outline{4. Minimize distractions.} Distractions such as social media, emails, and phone notifications can waste a lot of time. Try to limit them by turning off notifications or setting specific times to check them.

\outline{5. Take breaks.} Taking breaks can actually improve productivity and help prevent burnout. It's important to schedule short breaks throughout the day to recharge and refocus.

\outline{6. Learn to say no.} It's important to prioritize your own tasks and not take on too much at once. Saying no to requests that don't align with your goals or schedule can help you manage your time more effectively.

\outline{7. Delegate tasks.} If you have too many tasks to handle, consider delegating some of them to others who can help you. This will allow you to focus on more important tasks and manage your time more effectively.

\outline{8. Use time management tools.}There are a variety of time management tools available, such as calendars, to-do lists, and productivity apps, that can help you stay organized and on track with your tasks.}{}{Improving time management skills can be a challenge, but there are several strategies that can help. Here are a few:

1. Set clear goals and priorities: Identify your most important tasks and make sure to complete them first.

2. Make a schedule: Create a schedule that includes your most important tasks and set aside time for them.

3. Use a timer: Set a timer for a specific amount of time to help you stay focused and avoid distractions.

4. Take breaks: Take short breaks throughout the day to help you recharge and avoid burnout.

5. Eliminate distractions: Try to limit distractions as much as possible, such as by turning off your phone or closing your email.

6. Delegate tasks: If possible, delegate tasks to others to free up time for more important activities.

Remember, improving time management skills takes practice and perseverance. Be patient with yourself and keep trying until you find strategies that work well for you.}{}


\emph{(2) Roleplay.} Roleplay questions require LLMs to assume a character and present an introduction or speech to other people. \llmzoo{} and \fastchat{} metrics usually favor the level of details of \methodshort{} when assigning it a better score. Similar to the writing category discussed before, \methodshort{}'s answers could be even better if the skeleton points are removed from the answer.

\example{\Remark{claude}}{(Roleplay) How would you introduce yourself as a medieval knight at a royal banquet?}{\outline{1. Name and house.} My name is Sir Percival of Camelot, knight of the Round Table and protector of King Arthur's realm.

 \outline{2. Lands and titles.} I hold fiefdom over the lands of Westmarch and Ravenshire, and bear the titles of Baron of the Western Marches and Knight of the Raven.

 \outline{3. Military feats.} I have led my men to great victories against the kingdom's enemies, winning many battles through strength of arms and tactical cunning.

 \outline{4. Allegiance to king.} I pledge my sword, my loyalty, and my life to serve and protect my noble king and this great kingdom.

 \outline{5. Request to join table.} Your Highness, it would be the greatest honor to join your exalted table and share in this magnificent feast with my fellow knights and loyal subjects.}{}{Greetings, your majesty and honored guests. I am Sir Claude, a knight in service of the king. I hail from the province of Assistantia, where I was trained in the knightly virtues of chivalry, courage and courtesy. Though I am but a humble knight, I shall endeavor to serve you and this kingdom with all the strength and valor I possess. I thank you for allowing me to join you at this royal feast, and hope only that my conduct brings further honor to my liege and this noble court. Please let me know if I can be of any service during the celebrations, as it would be my honor to assist you in any way.}{}
