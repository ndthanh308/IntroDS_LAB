%% ****** Start of file apsguide4-2.tex ****** %
%%
%%   This file is part of the APS files in the REVTeX 4.2 distribution.
%%   Version 4.2b of REVTeX, December 2018.
%%
%%   Copyright (c) 2019 The American Physical Society.
%%
%%   See the REVTeX 4.2 README file for restrictions and more information.
%%
\documentclass[onecolumn,secnumarabic,amssymb, nobibnotes, aps, prd]{revtex4}
%\documentclass{article}
%\usepackage{acrofont}%NOTE: Comment out this line for the release version!
\usepackage{amsmath} \usepackage{amssymb}
\usepackage{graphicx} %use graph format
\usepackage{epstopdf}
\usepackage{amsmath}
%\usepackage{autobreak}
\usepackage{amsmath,amssymb,amsthm,amsfonts,mathrsfs,bm,verbatim}
\usepackage{graphicx,subfigure}

\newcommand{\bea}{\begin{eqnarray}}
\newcommand{\eea}{\end{eqnarray}}
\def\nn{\nonumber}

\newcommand{\revtex}{REV\TeX\ }
\newcommand{\classoption}[1]{\texttt{#1}}
\newcommand{\macro}[1]{\texttt{\textbackslash#1}}
\newcommand{\m}[1]{\macro{#1}}
\newcommand{\env}[1]{\texttt{#1}}
\setlength{\textheight}{9.5in}

\begin{document}

\title{Fermions can also produce super-radiation phenomena}%

\author{Wen-Xiang Chen$^{a}$}
\affiliation{Department of Astronomy, School of Physics and Materials Science, GuangZhou University, Guangzhou 510006, China}
\author{Yao-Guang Zheng}
\email{hesoyam12456@163.com}
\affiliation{Department of Physics, College of Sciences, Northeastern University, Shenyang 110819, China}

%\date{December 2020}%

\begin{abstract}
According to traditional theory, it is believed that Fermions do not exhibit superradiation. However, when predetermined boundary conditions are in place, there is a possibility of combining the wave function of coupled Fermions, which can result in the emergence of superradiation phenomena. This article presents a novel perspective, proposing that Fermions have the potential to display superradiation phenomena. This implies that there is a broader scope for investigating superradiation and its relationship with boundary conditions.

Keywords: Fermions,  superradiance, Wronskian determinant 
\end{abstract}



%%%%%%%%% BODY TEXT %%%%%%%%%%%%%%%%%%%%%%%%%%%%%%%%%%%%%%%%
\maketitle
\section{Introduction}


In quantum optics, superradiation\cite{1} refers to a phenomenon that occurs when a group of N emitters, such as excited atoms, interact with a common light field. When the wavelength of the light is much larger than the distance between the emitters, they collectively and coherently interact with the light, resulting in the emission of intense light pulses. This behavior is unexpected and differs from the exponential decay typically observed in independent atoms (spontaneous emission). Superradiation has been observed in various physical and chemical systems, including quantum dot arrays and aggregates, and has recently been utilized in the development of superradiative lasers.

Rotational superradiation is associated with the acceleration or movement of nearby objects, where these objects provide energy and momentum for the effect. It can be described as the outcome of the "effective" field variation around the body, such as tidal forces. Despite the absence of an apparent classical mechanism, this allows objects with concentrated angular momentum or linear momentum to transition to a lower energy state. In this sense, the effect shares similarities with quantum tunneling, where waves and particles find a way to take advantage of energy potential tendencies even in the absence of a clear classical mechanism.

In classical physics, it is generally expected that the movement or rotation of an object in a granular medium transfers momentum and energy to the surrounding particles. Thus, as the object moves along its trajectory, there is a statistical possibility of momentum being transferred to the particles. Quantum mechanics extends this principle to cases where an object moves, accelerates, or rotates in a vacuum. In such instances, quantum fluctuations with appropriate vectors become stretched and distorted, and energy and momentum are provided by the motion of nearby objects. Through this selective amplification, real physical radiation is generated around the object.

Classically, a rotating weightless ball in a vacuum would be expected to rotate indefinitely due to the absence of frictional effects or any obvious coupling with a smooth empty environment. However, in the quantum realm, the region surrounding the vacuum is not completely smooth, and the field of the sphere interacts with quantum fluctuations, accelerating them to produce real radiation. These imaginary virtual wavefronts follow specific paths around the object and are stimulated and amplified into real physical wavefronts through the coupling process. This phenomenon is sometimes referred to as the "ticking" of these fluctuations.

In theoretical studies of black holes, this effect is sometimes explained as the result of the gravitational and tidal forces exerted by the strong gravitational body, which pulls apart virtual particle pairs. Without such forces, these virtual particle pairs would quickly annihilate each other. Consequently, a large number of real particles are produced outside the black hole's horizon.

A black hole bomb refers to an exponentially increasing instability in the interaction between a massive boson field and a rotating black hole.

In astrophysics, a potential example of superradiation is Zel'dovich radiation. The effect was first described by Yakov Zel'dovich in 1971, and further developed by Igor Novikov of Moscow University. Zel'dovich considered a case in Quantum Electrodynamics (QED), where a rotating metal ball's equatorial region is expected to emit tangential electromagnetic radiation. He proposed that similar coupling effects should occur in situations involving rotating gravitational masses, such as the Kerr black hole, leading to similar radiation emissions.

This is followed by Stephen Hawking's argument and others that accelerating observers near a black hole (e.g., an observer carefully descending toward the event horizon at the end of a rope) would perceive the area as being filled with "real" radiation, while observers further away would view the same radiation as "virtual." If an accelerating observer near the event horizon captures nearby particles and throws them to a distant observer for measurement and study, the distant observer would observe the particles as having made a transition from virtual to "real" due to the physical acceleration imparted to them.

It is known that classical superradiation instability requires black holes to satisfy one condition: the incident perturbation field must be a Bose field. This condition is necessary for generating superradiation.

In 1972, Press and Teukolsky\cite{1} proposed the concept of a black hole bomb by introducing a mirror outside a black hole. According to the current understanding, this involves a scattering process that combines classical mechanics and quantum mechanics. Regge and Wheeler demonstrated that the spherically symmetric Schwarzschild black hole is stable under perturbations. However, due to the significant influence of superradiation, the stability of rotating black holes is more complex. Superradiation effects can occur in classical and quantum scattering processes. When a boson wave interacts with a rotating black hole, under certain conditions, the wave reflected by the event horizon can be amplified if the frequency range of the wave satisfies the superradiation conditions.

On January 15, 2012, \cite{2,3}Associate Professor Yuji Hasegawa of the Vienna University of Technology and Professor Masaaki Ozawa of Nagoya University, along with other scholars, published empirical results contradicting Heisenberg's uncertainty principle. They used two instruments to measure the rotation angle of a neutron and calculated the error in the measurement results, which was smaller than the uncertainty principle's prediction. This experiment challenged the limitations imposed by the uncertainty principle. However, it is important to note that the uncertainty principle remains valid, as it represents an intrinsic quantum property of particles.

An article follows the methodology employed to study superradiation and connects the uncertainty principle with the superradiation effect\cite{2,3}. The article reveals that under the superradiation effect, the measurement limit imposed by the uncertainty principle can be further reduced. According to this article, if the boundary conditions are not pre-set, the incident interference between the black hole and the coupled wave function of the black hole results in the equality of probability flow density equations on both sides. However, when the boundary conditions of the incident fermions are predetermined, the probability flow density equations on both sides are not equal, as setting the boundary conditions implies a certain probability. According to traditional theory, fermions do not exhibit superradiation. Nevertheless, if the boundary conditions are pre-set, the combined wave function of the coupled fermions can lead to superradiation phenomena.

\section{Fermionic Scattering}

Let us consider the Dirac equation for a spin-$\frac{1}{2}$ massless fermion $\Psi$, which is minimally coupled to the same electromagnetic (EM) potential $A_{\mu}$ as described:\cite{1}
\begin{equation}
\gamma^{\mu}\Psi_{;\mu}=0.
\end{equation}
Here, $\gamma^{\mu}$ represents the four Dirac matrices that satisfy the anticommutation relation ${\gamma^{\mu},\gamma^{\nu}}=2g^{\mu\nu}$.

The solution takes the form $\Psi=e^{-i\omega t}\chi(x)$, where $\chi$ is a two-spinor given by
\begin{equation}
\chi=\begin{pmatrix}f_1(x)\\f_2(x)\end{pmatrix}.
\end{equation}

Using the representation
\begin{equation}
\gamma^0=\begin{pmatrix}i&0\\0&-i\end{pmatrix}\,,\,\,\gamma^1=\begin{pmatrix}0&i\\-i&0\end{pmatrix}\,,
\end{equation}
the functions $f_1$ and $f_2$ satisfy the following system of equations:
\begin{equation}
\frac{df_1}{dx}-i(\omega-eA_0)f_2=0,\quad \frac{df_2}{dx}-i(\omega-eA_0)f_1=0.
\end{equation}

One set of solutions can be formed by the 'in' modes, which represent a flux of particles coming from $x\to-\infty$, being partially reflected (with reflection amplitude $|\mathcal{R}|^2$) and partially transmitted at the barrier:
\begin{equation}
\left(f_1^{\text{in}},f_2^{\text{in}}\right)=\left(\mathcal{I}e^{i\omega x}-\mathcal{R} e^{-i\omega x},\mathcal{I} e^{i\omega x}+\mathcal{R} e^{-i\omega x}\right) \quad \text{as} ,, x\to-\infty
\end{equation}
\begin{equation}
\left(f_1^{\text{in}},f_2^{\text{in}}\right)=\left(\mathcal{T}e^{ikx},\mathcal{T}e^{ikx}\right) \quad \text{as},, x\to+\infty.
\end{equation}

On the other hand, the conserved current associated with the Dirac equation is given by $j^{\mu}=-e\Psi^{\dagger}\gamma^{0}\gamma^{\mu}\Psi$. By equating the current at $x\to-\infty$ and $x\to+\infty$, we can find some general relations between the reflection and transmission coefficients, in particular:
\begin{equation}
\left|\mathcal{R}\right|^2=|\mathcal{I}|^2-\left|\mathcal{T}\right|^2.
\end{equation}

Therefore, $\left|\mathcal{R}\right|^2\leq |\mathcal{I}|^2$ for any frequency, indicating that there is no superradiance for fermions. A similar relation can be derived for massive fields.

The reflection and transmission coefficients depend on the specific shape of the potential $A_0$. However, one can easily show that the Wronskian
\begin{equation}
W=\tilde{f}_1 \frac{d\tilde{f}_2}{dx}-\tilde{f}_2\frac{d\tilde{f}_1}{dx},
\end{equation}
between two independent solutions $\tilde{f}_1$ and $\tilde{f}_2$, is conserved. Additionally, if $f$ is a solution, then its complex conjugate $f^*$ is another linearly independent solution. By using these properties, we find $\left|\mathcal{R}\right|^2=|\mathcal{I}|^2-\frac{\omega-eV}{\omega}\left|\mathcal{T}\right|^2$. Thus, for $0<\omega<e V$, it is possible to have superradiant amplification of the reflected current, i.e., $\left|\mathcal{R}\right|>|\mathcal{I}|$. Other potentials can also exhibit explicit superradiation phenomena.

The difference between fermions and bosons arises from the intrinsic properties of these two types of particles. Fermions have positive definite current densities and bounded transmission amplitudes $0\leq \left|\mathcal{T}\right|^2\leq |\mathcal{I}|^2$, while bosons can have changing sign in their current density as they are partially transmitted, and the transmission amplitude can be negative, $-\infty < \frac{\omega-eV}{\omega}\left|\mathcal{T}\right|^2\leq |\mathcal{I}|^2$. From the perspective of quantum field theory, due to the presence of strong electromagnetic fields, this process can be understood as a spontaneous pair generation phenomenon. The number of spontaneously produced fermion pairs in a given state is limited by Pauli's exclusion principle, while bosons do not have this limitation.

If we pre-set the boundary conditions $e{A_0(x)} = -{y}{\omega}$ (which can be ${\mu} = {-y}{\omega}$)\cite{2,3}, we can observe that when ${y}$ is relatively large (according to the properties of fermions, ${y}$ can be very large), the inequality $\left|\mathcal{R}\right|^2\leq |\mathcal{I}|^2$ may not hold. Consequently, the uncertainty principle $\Delta x\Delta p\geq 1/2$ may not hold in natural unit system. If the boundary conditions for the incident fermions are predetermined, the probability flow density equation's two sides are not equal, since setting the boundary conditions implies a certain probability.

\section{{Fermions Can Also Exhibit Super-radiation Phenomena}}
According to traditional theory, Fermions do not produce super-radiation. However, when specific boundary conditions are set in advance, the coupling of Fermions with the wave function can lead to the production of super-radiation phenomena.

Let's consider the following equations: $\boldsymbol{b}=\boldsymbol{t} \times \boldsymbol{n}$ and $\boldsymbol{t}=\frac{\partial \boldsymbol{X}}{\partial s}=\boldsymbol{X}{s}$. Applying the Frenet-Serret (FS) formula, we obtain:\cite{4}
\begin{equation}
\boldsymbol{t}{s}=\kappa \boldsymbol{n}, \quad \boldsymbol{n}{s}=-\kappa \boldsymbol{t}+\tau \boldsymbol{b}, \quad \boldsymbol{b}{s}=-\tau \boldsymbol{n}.
\end{equation}
Here, $\tau$ represents the torsion of the curve. We can derive:
\begin{equation}
\boldsymbol{X}{t}=\boldsymbol{X}{s} \times \boldsymbol{X}_{S S}.
\end{equation}
This is the differential evolution equation of the filament.

Following Hasimoto's work, we combine the second and third FS formulas to obtain:
\begin{equation}
(\boldsymbol{n}+i \boldsymbol{b})_{s}=-\kappa \boldsymbol{t}-i \tau(\boldsymbol{n}+i \boldsymbol{b}).
\end{equation}

To simplify the equations, we introduce the complex vector:
\begin{equation}
\boldsymbol{N}=(\boldsymbol{n}+i \boldsymbol{b}) e^{i \phi}.
\end{equation}
Here, the complex phase $\phi(s, t)$ is chosen such that its derivative with respect to $s$ yields $\tau(s, t)$. Specifically, $\phi(s, t)=\int_{0}^{s} \tau\left(s^{\prime}, t\right) d s^{\prime}$. We can then derive from this:
\begin{equation}
\boldsymbol{N}{S}=-\kappa e^{i \phi} \boldsymbol{t}.
\end{equation}
Let's set:
\begin{equation}
\psi=\kappa e^{i \phi}=\kappa(s, t) e^{i \int{0}^{s} \tau\left(s^{\prime}, t\right) d s^{\prime}}.
\end{equation}
This complex function replaces the two scalar functions $\kappa(s, t)$ and $\tau(s, t)$ and governs the evolution of the filament. Consequently, the equation becomes:
\begin{equation}
\boldsymbol{N}{s}=-\psi \boldsymbol{t}.
\end{equation}
This represents a new complex FS equation, where the variables $\boldsymbol{n}(s, t)$ and $\boldsymbol{b}(s, t)$ have been replaced by the complex variable $\psi$. This transformation is known as the Hasimoto transformation. Continuing with Hasimoto's approach, we seek an equation to replace $\boldsymbol{X}{t}=\boldsymbol{X}{s} \times \boldsymbol{X}{s s}$. We first observe that:
\begin{equation}
\boldsymbol{t} \cdot \boldsymbol{t}=1, \quad \boldsymbol{N} \cdot \boldsymbol{N}^{}=2, \quad \boldsymbol{N} \cdot \boldsymbol{N}=\boldsymbol{N}^{} \cdot \boldsymbol{N}^{}=0.
\end{equation}
The set of vectors $(\boldsymbol{t}, \boldsymbol{N}, \boldsymbol{N}^{})$ forms a new basis, replacing the FS basis $(\boldsymbol{t}, \boldsymbol{n}, \boldsymbol{b})$.

Therefore, we obtain the first FS equation in terms of the new variables:
\begin{equation}
\boldsymbol{t}_{s}=\frac{1}{2}\left(\psi^{} \boldsymbol{N}+\psi \boldsymbol{N}^{}\right)=\kappa \boldsymbol{n}.
\end{equation}

Now, let's differentiate the above Equation with respect to $s$. We find:
\begin{equation}
\frac{\partial}{\partial s}\left(\frac{\partial X}{\partial t}\right)=\frac{\partial}{\partial t}\left(\frac{\partial X}{\partial s}\right)=\frac{\partial}{\partial t} \boldsymbol{t}=\kappa_{s} \boldsymbol{b}+\kappa \boldsymbol{b}{s}=\kappa{s} \boldsymbol{b}-\kappa \tau \boldsymbol{n}.
\end{equation}
We express this equation in terms of the new variables $\psi$ and $\boldsymbol{N}$. We obtain:
\begin{equation}
\psi_{s} \boldsymbol{N}^{}-\psi_{s}^{} \boldsymbol{N}=-2 i \boldsymbol{t}_{t}.
\end{equation}

Next, to express $\boldsymbol{N}{t}$ in terms of the new variables, we set:
\begin{equation}
\boldsymbol{N}{t}=\alpha \boldsymbol{N}+\beta \boldsymbol{N}^{}+\gamma \boldsymbol{t}.
\end{equation}
Using the orthogonality relations, we find:
\begin{equation}
\boldsymbol{N} \cdot \boldsymbol{N}_{t}^{}+\boldsymbol{N}{t} \cdot \boldsymbol{N}^{}=2\left(\alpha+\alpha^{}\right)=4 \operatorname{Re}(\alpha)=0.
\end{equation}
Similarly, we have:
\begin{equation}
(\boldsymbol{N} \cdot \boldsymbol{N}){t}=4 \beta=0
\end{equation}
And finally:
\begin{equation}
\boldsymbol{t} \cdot \boldsymbol{N}{t}=\gamma=-i \psi{s}.
\end{equation}

Hence, we conclude:
\begin{equation}
\boldsymbol{N}{t}=\operatorname{Im}(\alpha) \boldsymbol{N}-i \psi{s} \boldsymbol{t}
\end{equation}
Let's set $\alpha=i R$ where $R$ is a real number. We have:
\begin{equation}
\boldsymbol{N}{t}=i\left(R \boldsymbol{N}-\psi{s} \boldsymbol{t}\right).
\end{equation}

Now, by combining $\boldsymbol{t}{s}$ and $\boldsymbol{t}{t}$, we can compute $\boldsymbol{N}{s t}$ and $\boldsymbol{N}{t s}$ using two different methods. This will allow us to determine the last unknown quantity $R$ and, consequently, find the equation for $\psi$. Since $\boldsymbol{N}{s t}=\boldsymbol{N}{t s}$, we can equate the components in the basis $\left(\boldsymbol{t}, \boldsymbol{N}, \boldsymbol{N}^{}\right)$. This leads to:
\begin{equation}
\begin{aligned}
i \partial_{t} \psi &=-\psi_{s s}-R \psi \
\frac{i}{2} \psi \psi_{s}^{} &=i R_{s}-\frac{i}{2} \psi_{s} \psi^{*}.
\end{aligned}
\end{equation}

This equation gives:
\begin{equation}
R_{s}=\frac{1}{2}\left(\psi \psi_{s}^{}+\psi_{s} \psi^{}\right)=\frac{1}{2}|\psi|_{s}^{2}.
\end{equation}

Traditionally, the phenomenon of superradiation has been analogized to gravitation, with the expectation that superradiation can lead to the realization of rotating space-time in toy models. A commonly studied configuration is based on vortices, such as drain bathtubs. Recently, researchers have utilized the scattering of these vortices by surface gravity waves in water to obtain the first experimental evidence of superradiation scattering. In a recent article\cite{5}, the proposal is made to utilize the information flow from the simulation model to understand the superradiation phenomenon in the context of Bose-Einstein condensation and gravity. As mentioned earlier, it is also interesting to consider the opposite direction of this analogy. Therefore, we will adopt this perspective to reevaluate the stability of quantized vortices in BECs and investigate the particular instabilities that arise in non-uniform flow BECs, similar to hydrodynamic parallel shear flows.

Cooper's pair: We will assume that the total spin and center of mass momentum, $\hbar \vec{q}$, of the pair are constant. Therefore, the orbital wave function of the pair is given by the expression \cite{6,7}:
\begin{equation}
\psi\left(\vec{r}_1, \vec{r}_2\right)=\varphi_q(\vec{\rho}) e^{i \vec{q} \cdot \vec{R}}.
\end{equation}
Here, the center of mass and relative coordinates are defined as $\vec{R}=\left(\vec{r}_1+\vec{r}_2\right) / 2$ and $\vec{\rho}=\vec{r}_1-\vec{r}_2$, respectively. When $\vec{q} \rightarrow 0$, the relative coordinate becomes spherically symmetric. Thus, $\varphi(\vec{\rho})$ is an eigenfunction of the angular momentum with quantum numbers 1 and $\mathrm{m}$. However, if $\vec{q} \neq 0$, the quantum number 1 is not conserved, but the component of angular momentum along $\vec{q}$ and parity are. Assuming $\vec{q}=0$, the wave function $\psi$ can be expanded as follows:
\begin{equation}
\psi\left(\vec{r}_1, \vec{r}_2\right)=\varphi_q(\vec{\rho})=\sum_k a_k e^{i \vec{k} \cdot \vec{r}_1} e^{-i \vec{k} \cdot \vec{r}_2}.
\end{equation}
In this expansion, the sum is limited to states with $\varepsilon_k>0$. If we consider $e^{i \vec{k} \cdot \vec{r}_1}$ and $e^{-i \vec{k} \cdot \vec{r}_2}$ as plane wave states, the pair wave function becomes a superposition of definite pairs where $\pm \vec{k}$ are occupied.

Consequently, we obtain:
\begin{equation}
R(s, t)=\frac{1}{2}|\psi|^{2}+A(t)
\end{equation}

If $\psi$ becomes a function of $x, y, z, t$, this generalizes to:
\begin{equation}
i \hbar \frac{\partial \psi(x, y, z, t)}{\partial t}+\left(\frac{\hbar}{4 m}|\psi|^{2}+A(t)\right) \psi(x, y, z, t)=-\frac{\hbar^{2}}{2 m} \Delta \psi(x, y, z, t).
\end{equation}

Assuming that the boundary $y$ and the uncertainty principle have a smaller extremum, we can conclude that under this condition, the effective potential can be greater than zero in a flat spacetime. Therefore, fermions can exhibit superradiation phenomena.


\section{{Summary}}
 This article introduces a novel concept suggesting that Fermions have the potential to generate superradiation phenomena. This implies that there is a wider scope for research on superradiation and boundary conditions.


\begin{thebibliography}{99}
\bibitem[1]{1}Brito, Richard, Vitor Cardoso, and Paolo Pani. Superradiance. Vol. 10. Springer International Publishing, 2020.

\bibitem[2]{2}Erhart, Jacqueline, et al. "Experimental demonstration of a universally valid error–disturbance uncertainty relation in spin measurements." Nature Physics 8.3 (2012): 185-189.

\bibitem[3]{3}Chen, Wen-Xiang. "Uncertainty principle and super-radiance." Available at SSRN 3683008 (2020).

\bibitem[4]{4}Driessen, P. "Derivation of the schrödinger equation from first principles." VIXRA. URL: https://vixra. org/abs/2104.0182 (2020).

\bibitem[5]{5}Giacomelli, Luca. "Superradiant phenomena-Lessons from and for Bose-Einstein condensates." (2021): 1-177.

\bibitem[6]{6}Squire, Richard H., and Norman H. March. "Quantum Phase Transition Induced by a Preformed Pair in a Boson-Fermion Model of Fulleride Superconductivity." arXiv preprint arXiv:0804.1459 (2008).

\bibitem[7]{7}Chen, Wen-Xiang. "Superradiation of Dirac particles in KN ds black hole." arXiv preprint arXiv:2105.05394 (2021).

\end{thebibliography}

\end{document}

