\documentclass[review]{elsarticle}

%RAS Comment here

\usepackage[margin=1in]{geometry}
\linespread{1.5}
\usepackage{float}
\usepackage{lineno,hyperref}
\usepackage{color}
\usepackage{amssymb}
\usepackage{graphics}
\usepackage{graphicx}
\usepackage{pbox}
\usepackage{mhchem}
\usepackage{xcolor} %colors,yay
\usepackage[ampersand]{easylist}
\usepackage{lineno}
\usepackage{booktabs}
\usepackage{multirow}
\usepackage{subcaption}
\usepackage{amsmath}
\usepackage{paralist}
\usepackage{soul}
\usepackage{siunitx}
\usepackage{amsmath}
\usepackage{url}
\usepackage{makecell}
\usepackage{hyperref}

\newcommand{\argmin}{\mathop{\mathrm{argmin}}\limits}
\newcommand{\solve}{\mathop{\text{Solve}}\limits}
\newcommand{\cm}{\text{cm}} 
\newcommand{\g}{\text{g}}
\newcommand{\MeV}{\text{MeV}} 
\newcommand{\eff}{\text{eff}}
\newcommand{\dep}{\text{dep}}
\newcommand{\mat}{\text{mat}}
\newcommand{\air}{\text{air}}
\newcommand{\PE}{\text{PE}}
\newcommand{\CS}{\text{CS}}
\newcommand{\PP}{\text{PP}}

\newcommand{\commentarrow}[1]{\textcolor{red}{$\leftarrow$#1}}
\newcommand{\comment}[1]{\textcolor{red}{#1}}

\hypersetup{  %UNCOMENT
    colorlinks=true,
    linkcolor=blue,
    filecolor=magenta,      
    urlcolor=cyan,
    pdftitle={Sharelatex Example},
    bookmarks=true,
    pdfpagemode=FullScreen,
    citecolor=blue
    }

\urlstyle{same} 

\modulolinenumbers[1]

%\journal{}
\makeatletter
\def\ps@pprintTitle{%
 \let\@oddhead\@empty
 \let\@evenhead\@empty
 \def\@oddfoot{\centerline{\thepage}}%
 \let\@evenfoot\@oddfoot}
\makeatother

\bibliographystyle{elsarticle-num}

\begin{document}

\begin{frontmatter}

\title{Direct atomic number reconstruction of dual energy cargo radiographs using a semiempirical transparency model}

\author[MITaddress]{Peter Lalor\corref{corauthor}}
\cortext[corauthor]{Corresponding author 
\\\hspace*{13pt} Email address: plalor@mit.edu
\\\hspace*{13pt} Telephone: (925) 453-1876 
\\\hspace*{13pt} 138 Cherry St, Cambridge, MA 02139}
%\ead{plalor@mit.edu}

\author[MITaddress]{Areg Danagoulian}

\address[MITaddress]{Department of Nuclear Science and Engineering, Massachusetts Institute of Technology, Cambridge, MA 02139, USA}

\begin{abstract}
Dual energy cargo inspection systems are sensitive to both the area density and the atomic number of an imaged container due to the $Z$ dependence of photon attenuation. The ability to identify cargo contents by their atomic number enables improved detection capabilities of illicit materials. Existing methods typically classify materials into a few material classes using an empirical calibration step. However, such a coarse label discretization limits atomic number selectivity and can yield inaccurate results if a material is near the midpoint of two bins. This work introduces a high resolution atomic number prediction method by minimizing the chi-squared error between measured transparency values and a semiempirical transparency model. Our previous work showed that by incorporating calibration step, the semiempirical transparency model can capture second order effects such as scattering. This method is benchmarked using two simulated radiographic phantoms, demonstrating the ability to obtain accurate material predictions on noisy input images by incorporating an image segmentation step. Furthermore, we show that this approach can be adapted to identify shielded objects after first determining the properties of the shielding, taking advantage of the closed-form nature of the transparency model.
\end{abstract}
\begin{keyword}
Dual energy radiography \sep non-intrusive inspection  \sep atomic number discrimination \sep nuclear security
\end{keyword}
\end{frontmatter}
\begin{sloppypar}

%\linenumbers

\section{Introduction}
\label{Introduction}

Every year, over 32.7 million cargo containers enter through U.S. ports~\cite{CBP2021}. A potential security concern is that a terrorist could smuggle nuclear material through U.S. ports and subsequently assemble a nuclear weapon~\cite{Cochran2008}. A nuclear detonation at a U.S. port could result in excess of \$1 trillion in net economic costs, including infrastructure damage and trade disruption~\cite{Meade2006, Rosoff2007}. To combat these threats, the U.S. Congress passed the SAFE Port Act in 2006, which mandated 100 percent screening of U.S. bound cargo and 100 percent scanning of high-risk containers~\cite{PLAW109-347}. The U.S. shortly afterwards passed the Implementing Recommendations of the 9/11 Comission Act of 2007, invoking a 2012 deadline for a full-scale implementation requiring that all containers be scanned prior to entering the U.S.~\cite{PLAW110-53}. This deadline continues to be extended due to lack of technological solutions.

The U.S. uses radiation portal monitors (RPMs) to passively scan all containers entering through U.S. ports. RPMs detect neutron and gamma radiation which may be being emitted by nuclear materials concealed within the container~\cite{Kouzes2005, Kouzes2008}. The U.S. also scans all high risk containers (identified as approximately 5 percent of seaborne containers~\cite{CBO2016}) using non-intrusive inspection (NII) technology~\cite{NII}. These radiography systems measure the attenuation of X-rays and/or gamma rays which are directed through the container to produce a density image of the scanned cargo.

To further enhance illicit material detection capabilities, some radiography systems deploy dual energy photon beams. The Rapiscan Eagle R60${}^\text{\textregistered}$ is an example of one such commercial system, using interlaced $6/4$ MeV x-ray beams to scan rail cargo~\cite{Rapiscan_overview}. Dual energy technology enables the ability to classify objects according to their $Z$, since the attenuation of photons depends on the atomic number of the intervening material. This could improve the capabilities of these systems to identify nuclear threats or high-$Z$ shielding.

This work presents a novel method for predicting the area density and atomic number of dual energy radiographic images using a semiempirical transparency model~\cite{Lalor2023_semiempirical}. Compared to previous work, this methodology obtains more precise $Z$ predictions by removing the need for material binning.

\section{Background}
\label{Background}

\subsection{Dual Energy Radiography Overview}

When a radiography system scans a material of area density $\lambda$ and atomic number $Z$, it measures the transparency of the photon beam, defined as the detected charge in the presence of the material normalized by the open beam measurement. Using the Beer-Lambert law, we can express the photon transparency $T(\lambda, Z)$ as follows:

\begin{equation}
T(\lambda, Z) = \frac{\int_0^{\infty} D(E) \phi(E) e^{-\mu (E, Z) \lambda} dE}{\int_0^{\infty}D(E) \phi(E) dE}
\label{transparency}
\end{equation}

where $\mu(E, Z)$ is the mass attenuation coefficient, $\phi(E)$ is the differential photon beam spectrum, and $D(E)$ is the detector response function. For this work, we calculate the photon beam spectrum and detector response function from the output of Geant4 simulations, as described in Section \ref{simulation}~\cite{Geant4, grasshopper}.

We observe that the radiographic transparency (Eq. \ref{transparency}) is dependent on the atomic number of the imaged material through the mass attenuation coefficient $\mu(E, Z)$. Thus, by making multiple transparency measurements of the same object using different photon energy spectra, properties of the material $Z$ can be inferred. Using the subscripts $\{H, L\}$ to refer to the \{high, low\} energy beam, we can represent the dual energy principle as a system of two equations and two unknowns:

\begin{equation}
\hat \lambda, \hat Z = \solve\limits_{\lambda, Z}
\begin{cases} 
      T_H(\lambda, Z) = T_H \\
      T_L(\lambda, Z) = T_L
\end{cases}
\label{2x2eq}
\end{equation}

where $\{T_H, T_L\}$ are a pair of dual energy transparency measurements, and $\hat \lambda$ and $\hat Z$ are the reconstructed material area density and atomic number.

\subsection{Limitations of existing methods}
\label{limitations}

We broadly categorize existing dual energy atomic number reconstruction methods into two categories: analytic methods and empirical methods. Analytic methods solve Eq. \ref{2x2eq} directly, obtaining material predictions by comparing radiographic measurements to model predictions~\cite{Novikov1999, Zhang2005, Ogorodnikov2002, Li2016}. However, this approach is generally unsuitable for practical applications, since the accuracy of Eq. \ref{transparency} deteriorates significantly under non-ideal circumstances. For instance, Eq. \ref{transparency} assumes only noninteracting photons are detected, and thus ignores the effects of scattered radiation, which can be up to one percent of the detected X-ray beam~\cite{Chen2007}. Furthermore, Eq. \ref{transparency} requires a high precision model for the photon beam spectra and detector response function, which may not be known exactly for real applications.

Other authors instead perform material discrimination using an empirical calibration step~\cite{Budner2006, Lee2012, Lee2018}. Empirical approaches are more accurate than analytic methods because the transparency model is constructed using calibration scans, reducing inherent model bias. However, these methods can only identify materials according to a small number of material classes, significantly limiting atomic number selectivity. All materials within a given class are assigned the same label, even if the radiographic system's resolution is high enough to distinguish between them. Furthermore, materials near the midpoint of two classes can be easily misclassified if the measurement uncertainty extends between both bins.

\subsection{A semiempirical formulation of the dual energy principle}

Our past work found that the secondary effects ignored in Eq. \ref{transparency} can be accurately captured by replacing the mass attenuation coefficient with a semiempirical mass attenuation coefficient $\tilde \mu(E, Z)$~\cite{Lalor2023_semiempirical}:

\begin{align}
\begin{split}
&\tilde \mu_H(\lambda, Z) = a_H \mu_\PE(E, Z) + b_H \mu_\CS(E, Z) + c_H \mu_\PP(E, Z) \\
&\tilde \mu_L(\lambda, Z) = a_L \mu_\PE(E, Z) + b_L \mu_\CS(E, Z) + c_L \mu_\PP(E, Z)
\label{semiempirical_mass_atten}
\end{split}
\end{align}

where $a_H$, $b_H$, $c_H$, $a_L$, $b_L$, and $c_L$ are determined through a calibration step, as described in Section \ref{calculating_abc}. Values for the mass attenuation coefficients are calculated from NIST cross section tables~\cite{NIST}. We note that in practice, each detector should be calibrated independently, and thus $a$, $b$, and $c$ will be different for separate detectors. We then perform a log transform $\alpha \rightarrow -\log T$ to write:

\begin{align}
\begin{split}
&\tilde \alpha_H (\lambda, Z) = -\log \frac{\int_0^{\infty} D(E) \phi_H(E) e^{-\tilde \mu_H (E, Z) \lambda} dE}{\int_0^{\infty}D(E) \phi_H(E) dE} \\
&\tilde \alpha_L (\lambda, Z) = -\log \frac{\int_0^{\infty} D(E) \phi_L(E) e^{-\tilde \mu_L (E, Z) \lambda} dE}{\int_0^{\infty}D(E) \phi_L(E) dE}
\end{split}
\label{alpha}
\end{align}

Compared to Eq. \ref{transparency}, the transparency model in Eq. \ref{alpha} shows significantly stronger agreement with simulated transparency measurements, especially if the source energy spectra and detector response function are not known exactly~\cite{Lalor2023_semiempirical}. As a result, Eq. \ref{alpha} yields more accurate material predictions for practical cargo scanning applications.

\subsection{Calculating Ground Truth}
\label{calc_ground_truth}

Even if the precise material composition of a detected object is known exactly, determining the ground truth atomic number is a somewhat unclear task. Past authors have proposed definitions of $Z_\eff$ involving a weighted average of atomic numbers~\cite{Naydenov2013, Langeveld2017_effective_Z}. However, as described in our previous work, such definitions lead to inconsistencies~\cite{Lalor2023_limitations}. Instead, the most precise definition for $Z_\eff$ is the atomic number of a homogeneous material that would produce identical high and low energy transparency measurements as the heterogeneous material. As a result, calculating $Z_\eff$ exactly requires a perfect transparency model, which is dependent on the precise geometric configuration of any particular scanning system. Since an exact transparency model is unobtainable, we instead turn to synthetic calculations. We first adapt Eq. \ref{alpha} for the heterogeneous case: 

\begin{equation}
\begin{split}
&\tilde \alpha_H (\vec \lambda, \vec Z) = -\log \frac{\int_0^{\infty} D(E) \phi_H(E) e^{-\sum_i \tilde \mu_H (E, Z_i) \lambda_i} dE}{\int_0^{\infty}D(E) \phi_H(E) dE} \\
&\tilde \alpha_L (\vec \lambda, \vec Z) = -\log \frac{\int_0^{\infty} D(E) \phi_L(E) e^{-\sum_i \tilde \mu_L (E, Z_i) \lambda_i} dE}{\int_0^{\infty}D(E) \phi_L(E) dE}
\end{split}
\label{alpha_heterogeneous}
\end{equation}

where the photon beam passes through an array of materials $\vec Z$ with corresponding area densities $\vec \lambda$. For this work, we calculate the effective atomic number and area density of a known $\{\vec \lambda, \vec Z\}$ as the solution to Eq. \ref{ground_truth}:

\begin{equation}
\lambda_\eff, ~Z_\eff = \solve\limits_{\lambda, Z}
\begin{cases}
	\tilde \alpha_H(\lambda, Z) = \tilde \alpha_H(\vec \lambda, \vec Z) \\
	\tilde \alpha_L(\lambda, Z) = \tilde \alpha_L(\vec \lambda, \vec Z) \\
\end{cases}
\label{ground_truth}
\end{equation}

While Eq. \ref{ground_truth} is only approximate, we expect it to yield an accurate result for algorithmic benchmark purposes. Unfortunately, the solution to Eq. \ref{ground_truth} is not guaranteed to be unique, even if only one material is present in the photon beam~\cite{Lalor2023_limitations}. This property is a fundamental shortcoming of dual energy radiographic systems which cannot be resolved through better algorithms or improved statistics. For this work, in the case of solution degeneracies, we choose the lower-$Z$ solution. While this heuristic yields unintuitive ground truth results for high-$Z$ materials, it represents the best possible performance of a perfect algorithm on a noiseless image.

\section{Methodology}
\label{Methodology}

\subsection{Image Segmentation}

Pixel-by-pixel atomic number reconstruction is impractical due to the poor signal-to-noise ratio of individual pixels. Instead, it is necessary to segment the image into different regions and subsequently predict the material properties of each cluster. In this way, this work ran Felzenszwalb's algorithm on the dual energy log-transparency image with a scale of $2000$, gaussian kernel width of $0.8$ pixels, and a minimum component size of $20$~\cite{Felzenszwalb2004}. This segmentation method interprets an image as an undirected graph, where edge weights are calculated as the intensity difference between connected pixels. Then, regions are greedily merged by evaluating a boundary predicate. This segmentation routine was chosen due to its speed, flexibility, and minimal hyperparameter tuning. The algorithm to be described in the next section was performed independently on every cluster. For practical applications, a specialized image segmentation routine should be used, as the accuracy of this work is highly dependent on obtaining a precise segmentation.

%The accuracy of this work is dependent on obtaining a precise segmentation, and we thus expect that an image segmentation routine optimized for a particular application would yield improved atomic number reconstruction accuracy.

\subsection{Atomic Number Reconstruction}
\label{Z_recon}
We write $\{\alpha_{H,i}, \alpha_{L,i}\}$ to be the measured log-transparencies for pixel $i$ in a pixel cluster $C$, with corresponding uncertainties $\{\sigma_{\alpha_{H,i}}, \sigma_{\alpha_{L,i}}\}$. Furthermore, we define $(\lambda_i, Z)$ to be the area density and atomic number estimates of pixel $i$, imposing that every pixel in the cluster is assigned the same atomic number. We thus define a chi-squared objective function as follows:

\begin{equation}
\chi^2( \boldsymbol{\lambda}, Z) = \sum_{i \in C} \left[ \frac{\left(\tilde \alpha_H(\lambda_i, Z) - \alpha_{H,i} \right)^2}{\sigma_{\alpha_{H,i}}^2} + \frac{\left(\tilde \alpha_L (\lambda_i, Z) - \alpha_{L,i} \right)^2}{\sigma_{\alpha_{L,i}}^2} \right]
\label{chi2}
\end{equation}

%\begin{equation}
%\hat {\boldsymbol{\lambda}}, \hat Z = \argmin_{\boldsymbol{\lambda}, Z} \chi^2(\boldsymbol{\lambda}, Z)
%\label{argmin}
%\end{equation}

Our area density and atomic number predictions for each pixel in $C$ are thus chosen as the minimizer of Eq. \ref{chi2}. While Eq. \ref{chi2} is not guaranteed to be convex for arbitrary beam spectra and detector response, we numerically verify that it is strictly convex in the $\boldsymbol{\lambda}$ dimension for the parameters used in this study, and we expect this property to be true for any realistic application. As a result, for a fixed $Z$, we can quickly minimize Eq. \ref{chi2} using only a few iterations of Newton's method:

\begin{equation}
\boldsymbol{\lambda}^{(n+1)} = \boldsymbol{\lambda}^{(n)} - \left[\frac{\partial^2 \chi^2 (\boldsymbol{\lambda}^{(n)}, Z)}{\partial \boldsymbol{\lambda}^2}\right]^{-1} \frac{\partial \chi^2 (\boldsymbol{\lambda}^{(n)}, Z)}{\partial \boldsymbol{\lambda}}
\label{newton}
\end{equation}

Expressions for the gradient and Hessian matrix are given in Section \ref{calculus}. The Hessian matrix is diagonal, so the inversion and subsequent matrix multiplication can be performed in $\mathcal{O}(|C|)$ time, where $|C|$ is the number of pixels in the cluster. We repeat this procedure exhaustively for $1 \leq Z \leq 100$ to find the global minimum of Eq. \ref{chi2}. We can speed up the calculation by noticing that the converged value of $\boldsymbol{\lambda}$ in Eq. \ref{newton} for a given $Z$ is similar to the result for neighboring $Z$ values. Thus, we can use the converged $\boldsymbol{\lambda}$ for $Z=1$ as the initial guess when calculating $\boldsymbol{\lambda}$ for $Z = 2$, and so on. Under this procedure, after solving the $Z=1$ case, we can solve $Z=2\dots100$ each using a single Newton step. We perform a runtime analysis of the method in Section \ref{runtime analysis}, finding that minimizing Eq. \ref{chi2} for every pixel cluster in a typical radiograph containing two million pixels took less than $30$ seconds using a single core of an Intel Core i7.

Sometimes, Eq. \ref{chi2} admits two local minima. This stems from a fundamental limitation of dual energy radiographic systems in which two different materials can  produce identical transparency measurements~\cite{Lalor2023_limitations}. For this work, consistent with Section \ref{calc_ground_truth}, if multiple local optima are found, we select the lower-$Z$ solution.

\section{Analysis}
\label{Analysis}

\subsection{Cargo Phantom}

In order to benchmark the effectiveness of the atomic number reconstruction routine described in Section \ref{Methodology}, this work introduces a radiographic cargo phantom, considering a range of different materials and geometries. Seven boxes composed of carbon $(Z=6)$, aluminum $(Z=13)$, iron $(Z=26)$, silver $(Z=47)$, gadolinium $(Z=64)$, lead $(Z=82)$, and uranium $(Z=92)$ are placed inside a 0.2cm thick steel container. Below the boxes are cylinders of water $(H_2 O)$, silver chloride $(AgCl)$, and uranium oxide $(UO_2)$. Next to the cylinders is a plutonium pit $(Z=94)$ surrounded by polyethylene shielding $(CH_2)$. The simulation geometry is described in more detail in Section \ref{simulation_geom}. 

Fig. \ref{synthetic_H_phantom} shows the synthetic high energy phantom image, calculated using Eq. \ref{alpha_heterogeneous} separately for each pixel. Fig. \ref{simulation_H_phantom} shows the simulated high energy phantom, calculated using Geant4 simulations. Fig. \ref{simulation_H_phantom_noisy} shows the simulated phantom after adding $10\%$ Gaussian noise to the simulated log-transparencies, representing a more realistic, noisy image. Figure \ref{Z_phantom_true} shows the resulting ground truth $\{\lambda, Z\}$ map of the cargo phantom, calculated using Eq. \ref{ground_truth}. When displaying $\{\lambda, Z\}$ maps of phantom containers, the image opacity shows the material area density and the colorbar identifies the material atomic number.

\subsection{Atomic Number Reconstruction Accuracy}
\label{Z_accuracy}

Using the simulated radiographic phantom as a template, $1000$ noisy images were generated by adding $10\%$ Gaussian noise to the template log-transparency image. The algorithm presented in Section \ref{Methodology} was then run on each of the noisy images to obtain pixel-wise area density and atomic number estimates. As a post processing step, the area density image was smoothed using a total variation filter. The mean and standard deviation of the $1000$ runs is shown in Figures \ref{Z_phantom_mu} and \ref{Z_phantom_unc}, respectively.

Overall, the reconstructed atomic number images show strong overall agreement with ground truth, yielding a reduced chi-squared of $1.02$. We quantify these results in Table \ref{tables_phantom}, where for every object of interest in the cargo phantom, we calculate the ground truth $Z_\eff$, reconstructed $Z$, and reconstructed $\sigma_Z$, defined as the average over all pixels in the object. In all cases, the reconstructed atomic number is within the uncertainty estimate of the ground truth $Z$. Our reconstructed uncertainties are much smaller than the bin sizes of prior work. Thus, this methodology provides more precise material predictions than existing methods.

\begin{table}
\begin{centering}
\begin{tabular}{l c c}
\toprule
 & \makecell{Ground Truth $Z_\eff$ \\ (average per object)} & \makecell{Reconstructed $Z$ \\ (average per object)}  \\
\midrule
 Polyethylene shield & $8.1$ & $9.4 \pm 2.2$ \\
 Water cylinder & $8.9$ & $9.8 \pm 1.2$ \\
 Carbon box & $9.1$ & $13.8 \pm 5.7$ \\
 Aluminum box & $14.7$ & $17.1 \pm 3.9$ \\
 Iron box & $26.0$ & $28.7 \pm 4.5$ \\
 Silver chloride cylinder & $37.1$ & $38.5 \pm 3.2$ \\
 Plutonium pit + polyethylene shield & $41.9$ & $32.5 \pm 12.0$ \\
 Silver box & $44.9$ & $47.1 \pm 7.2$ \\
 Uranium oxide cylinder & $56.0^*$ & $58.4^* \pm 8.2$ \\
 Gadolinium box & $58.0^*$ & $59.1^* \pm 9.2$ \\
 Lead box & $67.8^*$ & $66.0^* \pm 8.6$ \\
 Uranium box & $68.3^*$ & $68.2^* \pm 7.3$ \\
\bottomrule
\end{tabular}
\caption{Reconstructing the $Z$ of different regions within the cargo phantom. For each region, the average value of $Z$ and the average value of $\sigma_Z$ are calculated. Note that the objects are all placed inside a steel container. \\
$^*$Two solutions were found, and the lower-$Z$ solution was chosen.}
\label{tables_phantom}
\end{centering}
\end{table}

We notice that the atomic number uncertainty is largest near material boundaries, stemming from two main causes. First, the contribution of scattered photons to the detected signal becomes less predictable, causing the accuracy of the semiempirical transparency model to deteriorate. Second, the image segmentation step was performed independently between different runs, so we expect the variation between different segmentations to contribute to the uncertainty of the resulting atomic number estimate. We further observe that the plutonium pit shows the largest atomic number uncertainty. In addition to the two previous causes, this is due to the small size of the plutonium, meaning fewer pixels were included in the pixel cluster. Furthermore, the high attenuation of the plutonium pit yielded a noisier region, since the added noise was proportional to the pixel log-transparency. In general, the uncertainty estimate depends on the image noise level, so higher resolution systems will yield lower atomic number uncertainties.

% Figure environment removed

\subsection{Identification of Shielded Objects}
\label{shielded}

A security vulnerability of dual energy radiographic systems arises in a smuggler's ability to effectively conceal high-$Z$ materials from detection by embedding them within low-$Z$ shielding. To explore this concept, a shielded phantom is designed by placing aluminum $(Z=13)$, iron $(Z=26)$, tin $(Z=50)$, and plutonium $(Z=94)$ boxes behind a large graphite shield, described further in Section \ref{simulation_geom}. Fig. \ref{simulation_H_shielded_noisy} shows the Geant4 simulated image after adding $10\%$ Gaussian noise. Figures \ref{Z_shielded_mu} and \ref{Z_shielded_unc} show the predicted atomic number and uncertainty, respectively (reduced chi-squared = 1.00). We observe that the presence of the graphite shielding significantly suppresses the ability to identify plutonium by its $Z$. Direct atomic number reconstruction yields a $Z_\eff = 29.7 \pm 3.2$ for the shielded plutonium, which could easily be mistaken for steel.

To combat this limitation, we notice that the closed form nature of the transparency model used in this work (Eq. \ref{alpha}) offers a unique way to identify shielded objects. If an object with area density $\lambda_\text{object}$ and atomic number $Z_\text{object}$ is obscured by shielding with area density $\lambda_\text{shield}$ and atomic number $Z_\text{shield}$, we can separate the semiempirical mass attenuation coefficient into two components:

\begin{equation}
\tilde \mu(E, Z) \lambda = \tilde \mu(E, Z_\text{object}) \lambda_\text{object} + \tilde \mu(E, Z_\text{shield}) \lambda_\text{shield}
\label{alpha_shielded}
\end{equation}

During a first pass, we approximate $(\lambda_\text{shield}, Z_\text{shield})$ using the methods of Section \ref{Methodology} by only considering the regions surrounding the object. We then use this result to adapt Eq. \ref{alpha} as follows:

\begin{equation}
\begin{split}
&\tilde \alpha_H (\lambda_\text{object}, Z_\text{object}) = -\log \frac{\int_0^{\infty} D(E) \phi_H(E) e^{-\tilde \mu_H (E, Z_\text{object}) \lambda_\text{object}} e^{-\tilde \mu_H (E, Z_\text{shield}) \lambda_\text{shield}} dE}{\int_0^{\infty}D(E) \phi_H(E) dE} \\
&\tilde \alpha_L (\lambda_\text{object}, Z_\text{object}) = -\log \frac{\int_0^{\infty} D(E) \phi_L(E) e^{-\tilde \mu_L (E, Z_\text{object}) \lambda_\text{object}} e^{-\tilde \mu_L (E, Z_\text{shield}) \lambda_\text{shield}} dE}{\int_0^{\infty}D(E) \phi_L(E) dE}
\end{split}
\label{alpha_shielded}
\end{equation}

We can thus solve directly for $\{\lambda_\text{object}, Z_\text{object}\}$ during a second pass, exposing objects which are hidden behind the shielding and inferring their intrinsic values of $Z$ and $\lambda$. This approach is not possible if the transparency model were instead constructed empirically, since the transparency of a shielded object cannot be inferred from separate measurements of both the shielding and the object due to beam hardening. 

This methodology was applied to the shielded phantom. Fig. \ref{Z_peeled_true} shows the ground truth $\{\lambda, Z\}$ map of the objects behind the shield, and Figures. \ref{Z_peeled_mu} and \ref{Z_peeled_unc} show the predicted atomic number and uncertainty after stripping off the graphite shield. These results are quantified in table \ref{table_shielded}. By using Eq. \ref{alpha_shielded}, we are able to obtain atomic number estimates which are consistent with the ground truth $Z$ of the unshielded objects, despite the thick shielding present in the images. This result offers a potential avenue for dual energy cargo inspection systems to be able to identify high-$Z$ materials, even in the presence of shielding.

% Figure environment removed

\begin{table}
\begin{centering}
\begin{tabular}{l c c c}
\toprule
 & \makecell{Ground Truth $Z_\eff$ \\ of unshielded object} & \makecell{Reconstructed $Z$ \\ (obscured by graphite)} & \makecell{Reconstructed $Z$ \\ (stripping graphite)}  \\
\midrule
 Aluminum box & $13$ & $8.9 \pm 1.4$ & $14.5 \pm 4.9$ \\
 Iron box & $26$ & $16.2 \pm 2.0$ & $29.6 \pm 5.8$ \\
 Tin box & $50$ & $23.9 \pm 2.6$ & $53.5 \pm 10.0$ \\
 Plutonium box & $72.9^*$ & $29.7 \pm 3.2$ & $72.7^* \pm 9.1$ \\
\bottomrule
\end{tabular}
\caption{Reconstructing the $Z$ of different objects placed behind a graphite shield. Mathematically stripping off the graphite shield allows for accurate identification of the shield object. \\
$^*$Two solutions were found, and the lower-$Z$ solution was chosen.}
\label{table_shielded}
\end{centering}
\end{table}

\section{Conclusion}
\label{Conclusion}

This work introduces a flexible and accurate method for predicting the area density and atomic number of dual energy radiographic images, capable of obtaining more precise material predictions than previous methods. We define a semiempirical transparency model, which is capable of capturing second order effects such as scattering by performing a calibration step. The image is segmented into different clusters, and the atomic number of each pixel cluster is determined by minimizing the chi-squared error between the semiempirical transparency model and the raw pixel measurements. This enables high resolution atomic number predictions without the need for material binning, which is the conventional practice in most commercial systems. This approach shows less model bias than previous analytic methods and improved atomic number selectivity compared to empirical methods. Furthermore, this work showed the ability to accurately identify shielded materials by first calculating the properties of the shielding and then mathematically removing it. Future work should apply this method to deployed radiographic imaging scenarios to verify its applicability in a realistic setting.

\section{Acknowledgements}

This work was supported by the Department of Energy Computational Science Graduate Fellowship (DOE CSGF) under grant DE-SC0020347. The authors declare no conflict of interest.
\newpage

\bibliography{References.bib}

\newpage

\appendix

\section{}
\label{appendix}

\subsection{Simulation of the Beam Spectra and Detector Response}
\label{simulation}

In order to simulate the incident photon beam spectra, $10$ and $6$ MeV electrons were directed at a 0.1cm tungsten radiator backed by 1cm of copper. This geometry was based off the work of Henderson and designed to be simple, generalizable, and representative of typical cargo scanning applications~\cite{Henderson2019}. The resulting bremsstrahlung photons were subsequently measured by a tally surface and binned according to their beam angle. $\phi_{\{H, L\}}$ were calculated from the smallest beam angle bin, subtending a half angle of ${\approx} 0.2^{\circ}$. The resulting dual energy beam spectra are shown in Fig. \ref{spectra}. While $\phi_{\{H, L\}}$ only include low-angle bremsstrahlung photons, the full angular dependence of the photon beam spectra is recorded for use in subsequent transparency simulations, as described in Section \ref{simulation_geom}.

To simulate the detector response matrix, photons with energy uniformly distributed between $0$ and $10$ MeV were directed along the long axis of a $15.0 \times 10 \times 30.0$ mm cadmium tungstate (CdWO$_4$) crystal. The incident and deposited energy of each photon was recorded and binned in order to produce the detector response matrix.

% Figure environment removed

\subsection{Transparency Simulations}
\label{simulation_geom}

To simulate transparency measurements of a target geometry, photons were directed in a fam beam towards one meter tall stack of CdWO$_4$ detectors with a $10$cm lead collimator to filter scattered radiation. The detector stack was placed three meters from the photon source, and the target was placed midway between the source and the detector. The photon beam was divided into $100$ bins based on beam angle in order to capture the angular dependence of bremsstrahlung photon energy. For each beam angle bin, photons were sampled from the corresponding energy distribution calculated in Section \ref{simulation}. Each simulation, the target was shifted by one pixel, and this was repeated across the entire length of the target in order to construct the full $2D$ radiograph.

This work considered two radiographic phantoms to benchmark the performance of the atomic number reconstruction routine. The radiographic cargo phantom consists of seven boxes composed of carbon $(Z=6,~ \lambda=30 \g/\cm^2)$, aluminum $(Z=13,~ \lambda=40 \g/\cm^2)$, iron $(Z=26,~ \lambda=79 \g/\cm^2)$, silver $(Z=47,~ \lambda=79 \g/\cm^2)$, gadolinium $(Z=64,~ \lambda = 79 \g/\cm^2)$, lead $(Z=82,~ \lambda= 79 \g/\cm^2)$, and uranium $(Z=92,~ \lambda=76 \g/\cm^2)$. The boxes are all $10\cm \times 10\cm$ and placed at $y=35\cm$. Below the boxes are cylinders of water $(H_2 O,~ r = 17.5 \cm)$, silver chloride $(AgCl,~ r=7.5 \cm)$, and uranium oxide $(UO_2,~ r=3.75 \cm)$. The cylinders all extend from $y= 2.5 \cm$ to $y = 25 \cm$. Next to the cylinders is a plutonium pit $(Z=94,~ r=2\cm)$, surrounded by a shell of polyethylene shielding $(CH_2,~ r=12.5 \cm)$. The items are all placed inside a steel container with a thickness of $0.2 \cm$. The simulation geometry is shown in Fig. \ref{cargo_geom}.

The radiographic shielded phantom consists of four $10\cm \times 10\cm$ boxes composed of aluminum $(Z=13,~ \lambda = 54 \g/\cm^2)$, iron $(Z=26,~ \lambda=79 \g/\cm^2)$, tin $(Z=50,~ \lambda=73 \g/\cm^2)$, and plutonium $(Z=94,~ \lambda=60 \g/\cm^2)$. These boxes are placed behind a $35\cm$ graphite shield $(Z=6,~\lambda = 77 \g/\cm^2)$. The simulation geometry is shown in Fig. \ref{shielded_geom}.

% Figure environment removed

\subsection{Calculation of the Calibration Parameters}
\label{calculating_abc}

In radiographic systems, the presence of scattering causes the detected transparency to diverge from an idealized free streaming photon model. Our past work showed that these secondary effects can be captured using Eq. \ref{alpha}. To calibrate the model, high resolution transparency simulations of carbon $(Z=6)$, iron $(Z=26)$, and lead $(Z=82)$ were performed at an area density of $\lambda = 100 \g/\cm^2$. Then, $a$, $b$, and $c$ are determined by minimizing the squared error between the transparency model and transparency simulations:

\begin{equation}
\begin{split}
a_H, b_H, c_H &= \argmin_{a, b, c} \sum_i \left(\tilde \alpha_H(\lambda_i, Z_i) - \alpha_{H,i} \right)^2 \\
a_L, b_L, c_L &= \argmin_{a, b, c} \sum_i \left(\tilde \alpha_L(\lambda_i, Z_i) - \alpha_{L,i} \right)^2
\end{split}
\label{calc_abc}
\end{equation}

where $\{\alpha_{H,i}, ~\alpha_{L,i}\}$ are the measured log-transparencies in the presence of calibration material $\{\lambda_i, Z_i\}$. In Eq. \ref{calc_abc}, the transparency models $\tilde \alpha_H(\lambda, Z)$ and $\tilde \alpha_L(\lambda, Z)$ (Eq. \ref{alpha}) depend implicitly on $a$, $b$, and $c$ through the semiempirical mass attenuation coefficient (Eq. \ref{semiempirical_mass_atten}). The values for the calibration parameters used in this study are shown in figure \ref{calibration}.

% Figure environment removed

\subsection{Computing the gradient and Hessian matrix}
\label{calculus}
We efficiently minimize Eq. \ref{chi2} using Newton's method, given by Eq. \ref{newton}. In order to calculate the gradient and the Hessian matrix, we first differentiate equation \ref{chi2} with respect to $\boldsymbol{\lambda}$. To simplify notation, we will drop index labels and evaluate the derivatives elementwise. Using the chain rule:

\begin{equation}
\frac{\partial \chi^2(\lambda, Z)}{\partial \lambda} = \frac{2}{\sigma_{\alpha_H}^2} \left( \tilde \alpha_H(\lambda, Z) - \alpha_H \right) \frac{\partial \tilde \alpha_H(\lambda, Z)}{\partial \lambda} + \frac{2}{\sigma_{\alpha_L}^2} \left( \tilde \alpha_L(\lambda, Z) - \alpha_L \right) \frac{\partial \tilde \alpha_L(\lambda, Z)}{\partial \lambda}
\label{chi2_d1}
\end{equation}

and the second derivative:

\begin{equation}
\begin{split}
\frac{\partial^2 \chi^2(\lambda, Z)}{\partial \lambda^2} &= \frac{2}{\sigma_{\alpha_H}^2} \left( \tilde \alpha_H(\lambda, Z) - \alpha_H \right) \frac{\partial^2 \tilde \alpha_H(\lambda, Z)}{\partial \lambda^2} + \frac{2}{\sigma_{\alpha_H}^2} \left( \frac{\partial \tilde \alpha_H(\lambda, Z)}{\partial \lambda} \right)^2 \\
&~+ \frac{2}{\sigma_{\alpha_L}^2} \left( \tilde \alpha_L(\lambda, Z) - \alpha_L \right) \frac{\partial^2 \tilde \alpha_L(\lambda, Z)}{\partial \lambda^2} + \frac{2}{\sigma_{\alpha_L}^2} \left( \frac{\partial \tilde \alpha_L(\lambda, Z)}{\partial \lambda} \right)^2
\end{split}
\label{chi2_d2}
\end{equation}

We note that the cross terms of the second derivative are zero:

\begin{equation}
\frac{\partial^2 \chi^2(\boldsymbol{\lambda}, Z)}{\partial \lambda_i \partial \lambda_j} = 0~~,~~i \neq j
\label{off-diagonal}
\end{equation}

and thus the Hessian is diagonal. To calculate $\frac{\partial {\tilde \alpha}(\lambda, Z)}{\partial \lambda}$ and $\frac{\partial^2 {\tilde \alpha}(\lambda, Z)}{\partial \lambda^2}$, we first define the following quantities:

\begin{equation}
\begin{split}
Q_H^{(0)}(\lambda, Z) &= \int_0^{\infty} D(E) \phi_H(E) e^{-\tilde \mu_H (E, Z) \lambda} dE \\
Q_L^{(0)}(\lambda, Z) &= \int_0^{\infty} D(E) \phi_L(E) e^{-\tilde \mu_L (E, Z) \lambda} dE
\end{split}
\label{Q_d0}
\end{equation}
\begin{equation}
\begin{split}
Q_H^{(1)}(\lambda, Z) &= - \int_0^{\infty} D(E) \phi_H(E) \tilde \mu_H (E, Z) e^{-\tilde \mu_H (E, Z) \lambda} dE = \frac{\partial Q_H^{(0)}(\lambda, Z)}{\partial \lambda} \\
Q_L^{(1)}(\lambda, Z) &= - \int_0^{\infty} D(E) \phi_L(E) \tilde \mu_L (E, Z) e^{-\tilde \mu_L (E, Z) \lambda} dE = \frac{\partial Q_L^{(0)}(\lambda, Z)}{\partial \lambda}
\end{split}
\label{Q_d1}
\end{equation}
\begin{equation}
\begin{split}
Q_H^{(2)}(\lambda, Z) &= \int_0^{\infty} D(E) \phi_H(E) \tilde \mu_H (E, Z)^2 e^{-\tilde \mu_H (E, Z) \lambda} dE =\frac{\partial^2 Q_H^{(0)}(\lambda, Z)}{\partial \lambda^2} \\
Q_L^{(2)}(\lambda, Z) &=  \int_0^{\infty} D(E) \phi_L(E) \tilde \mu_L (E, Z)^2 e^{-\tilde \mu_L (E, Z) \lambda} dE = \frac{\partial^2 Q_L^{(0)}(\lambda, Z)}{\partial \lambda^2}
\end{split}
\label{Q_d2}
\end{equation}

Under these definitions,$\frac{\partial \tilde \alpha(\lambda, Z)}{\partial \lambda}$ and $\frac{\partial^2 \tilde \alpha(\lambda, Z)}{\partial \lambda^2}$ can be expressed as

\begin{equation}
\begin{split}
\frac{\partial \tilde \alpha_H(\lambda, Z)}{\partial \lambda} &= -\frac{Q_H^{(1)}(\lambda, Z)}{Q_H^{(0)}(\lambda, Z)} \\
\frac{\partial \tilde \alpha_L(\lambda, Z)}{\partial \lambda} &= -\frac{Q_L^{(1)}(\lambda, Z)}{Q_L^{(0)}(\lambda, Z)}
\end{split}
\label{alpha_d1}
\end{equation}

\begin{equation}
\begin{split}
\frac{\partial^2 \tilde \alpha_H(\lambda, Z)}{\partial \lambda^2} &= \frac{Q_H^{(1)}(\lambda, Z)^2 - Q_H^{(0)}(\lambda, Z) Q_H^{(2)}(\lambda, Z) }{ Q_H^{(0)}(\lambda, Z)^2} \\
\frac{\partial^2 \tilde \alpha_L(\lambda, Z)}{\partial \lambda^2} &= \frac{Q_L^{(1)}(\lambda, Z)^2 - Q_L^{(0)}(\lambda, Z) Q_L^{(2)}(\lambda, Z) }{Q_L^{(0)}(\lambda, Z)^2}
\end{split}
\label{alpha_d2}
\end{equation}

Prior to any minimization, lookup tables of $\tilde \alpha_H(\lambda, Z)$, $\tilde \alpha_L(\lambda, Z)$, $\frac{\partial \tilde \alpha_H(\lambda, Z)}{\partial \lambda}$, $\frac{\partial \tilde \alpha_L(\lambda, Z)}{\partial \lambda}$, $\frac{\partial^2 \tilde \alpha_H(\lambda, Z)}{\partial \lambda^2}$, and $\frac{\partial^2 \tilde \alpha_L(\lambda, Z)}{\partial \lambda^2}$ (equations \ref{alpha}, \ref{alpha_d1}, and \ref{alpha_d2}) were made for a large array of $\lambda$ and $Z$ values. This was done to make the calculations of equations \ref{chi2} and \ref{newton} during minimization significantly cheaper since, instead of numerical integrals, only table lookups are required. Since each detector was calibrated independently, different lookup tables were made for different detectors.

\subsection{Runtime analysis}
\label{runtime analysis}

The feasibility of any atomic number reconstruction routine is dependent on the ability to be run in real-time for practical cargo security applications. To explore this effect, we upsampled the radiographic cargo phantom and recorded the runtime of both the atomic number reconstruction algorithm and the image segmentation routine as a function of image size. The results are shown in Fig. \ref{runtime}. We verify that both the image segmentation step and the atomic number reconstruction step scale approximately linearly with the number of pixels. Furthermore, we note that the runtime is independent of the number of pixel clusters. The analysis in this section was performed using a single core of an Intel Core i7.

% Figure environment removed

\end{sloppypar}
\end{document}
