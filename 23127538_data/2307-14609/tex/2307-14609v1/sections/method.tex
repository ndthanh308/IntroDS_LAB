\section{Method}
\label{sec:method}

\subsection{Task Definition}
\label{ssec:task}

% In the general case, the problem of conditional source separation can be formulated in the following way. We consider a mixture waveform $\mathbf{x} = \sum_{i = 1}^N \mathbf{s}_i \in \mathbb{R}^L$ consisting of $N$ sources and a size of $L$ time-samples. We also consider a condition $\mathcal{C}$ which describes a target source $\mathbf{s}_T = \sum_{j \in \mathscr{S}} \mathbf{s}_j$, where $\mathscr{S} \subseteq \{1, 2, \dots, N \}$ is the subset of sources corresponding to the condition $\mathcal{C}$. The task is then to retrieve the target source $\mathbf{s}_T$ given the mixture $\mathbf{x}$ and the condition $\mathcal{C}$, which is usually represented as a vector $\mathbf{c}$.

% Additionally, we consider that each source $\mathbf{s}_i$ in the mixture $\mathbf{x}$ is associated with a set of source attributes, which can be absolute (i.e., independent of the mixture) or relative (i.e., dependent on the mixture). For instance, the gender of the speaker can be thought of as an absolute attribute, while the source energy can be regarded as a relative attribute dependent on the accompanying mixture. 

% In this work, in order to provide an illustration we narrow our focus on a specific scenario where $N = 2$ and the attribute set is $\mathcal{A} = \{ \mathcal{G}, \mathcal{E}, \mathcal{O}, \mathcal{S}\}$, where $\mathcal{G}$ represents the speaker's gender (female/male), $\mathcal{E}$ the energy of the source (high/low), $\mathcal{O}$ the temporal order of appearance of the source (first/second) and $\mathcal{S}$ the spatial position of the source with respect to the microphone (far/near). Further, we consider conditions $\mathcal{C} \in \mathcal{A}$ and each one can be associated only with one source of the mixture. For example, a valid condition would be the ``high energy source" or the ``female speaker". Thus, each condition $\mathcal{C}$ is one source attribute exclusive to the target source and can be represented by an 8-dimensional one-hot vector. Finally, we assume a fully supervised training scenario where the ground truth sources and their corresponding attributes are available when generating the synthetic training mixtures.

In the general case, the problem of conditional source separation can be formulated in the following way. We consider a mixture waveform $\mathbf{x} = \sum_{i = 1}^N \mathbf{s}_i \in \mathbb{R}^L$ consisting of $N$ sources and a size of $L$ time-samples. We also consider a condition $\mathcal{C}$ (e.g. the amount of energy of a source) and a condition value $v \in \mathcal{C}$ (e.g. low energy or high energy) which describes a target source $\mathbf{s}_T$ where $T = j$ if $\mathcal{C}(\mathbf{s}_j) = v$. The task is then to retrieve the target source $\mathbf{s}_T$ given the mixture $\mathbf{x}$ and the condition value $v$, represented as a vector $\mathbf{c}$.

Additionally, we consider that each source $\mathbf{s}_i$ in the mixture $\mathbf{x}$ is associated with a set of source attributes values. Source attributes can be absolute (i.e., independent of the mixture) or relative (i.e., dependent on the mixture). For instance, the speaker's gender can be thought of as an absolute attribute, while the source energy can be regarded as a relative attribute dependent on the mixture. %accompanying mixture. 

In this work, in order to provide an illustration we narrow our focus on a specific scenario where $N = 2$ and the source attribute set is $\mathscr{C} = \{ \mathcal{G}, \mathcal{E}, \mathcal{O}, \mathcal{S}\}$, where $\mathcal{G}$ is the source attribute that represents the speaker's gender (with values: female/male), $\mathcal{E}$ the energy of the source (high/low), $\mathcal{O}$ the temporal order of appearance of the source (first/second) and $\mathcal{S}$ the spatial distance of the source with respect to the microphone (far/near). Further, we consider conditions $\mathcal{C} \in \mathscr{C}$ meaning each condition $\mathcal{C}$ consists of one source attribute. For example, a valid condition value would be the ``high energy source" or the ``female speaker". Moreover, we assume that every condition value $v$ can be associated only with one source of the mixture, implying that sources have complementary attributes. Thus, each condition value $v$ can be represented as an 8-dimensional one-hot vector $\mathbf{c}$, and the source attribute values of a specific source as an 8-dimensional multi-hot vector $\mathbf{c}_{\text{full}}$. Finally, we assume a fully supervised training scenario where the ground truth sources and their corresponding attributes are available when generating the synthetic training mixtures.

\subsection{Heterogeneous Target Separation}
\label{ssec:hct}

One baseline approach to tackle the problem is to train a conditional separation network using the Heterogeneous Conditional Training (HCT) \cite{tzinis22_interspeech}. In this case, we have a multi-conditional network: 
\begin{equation}
\label{eq:het}
    \begin{gathered}
    \widehat{\mathbf{s}}_T, \, \widehat{\mathbf{s}}_O = f_\theta (\mathbf{x}, \mathbf{c}),
    \end{gathered}
\end{equation}
which is trained to estimate the target $\widehat{\mathbf{s}}_T$ and the residual source $\widehat{\mathbf{s}}_O$ given a one-hot vector $\mathbf{c}$ associated with a condition value $v$. During training, for a given target source $\mathbf{s}_T$ and a uniformly sampled condition value $v$, HCT tries to minimize the following loss:
\begin{equation}
\label{eq:het}
    \begin{gathered}
    \mathcal{L}_{\text{HCT}} = \ell \left(\widehat{\mathbf{s}}_T, {\mathbf{s}}_T \right) + \ell \left(\widehat{\mathbf{s}}_O, {\mathbf{s}}_O \right),
    \end{gathered}
\end{equation}
where $\ell$ is a signal level time-domain loss function. 

\subsection{Proposed Approach}
Our motivation is that some of the target source's attributes can be more discriminative than others, which suggests that we can improve separation performance by ``completing" the user's query.

Consequently, our proposed approach consists of two stages, a completion step followed by a separation step. First, we learn a completion module $g_\phi$, which given the input mixture $\mathbf{x}$ and a sampled condition value $v$ describing the target source $\mathbf{s}_T$, represented as the partial condition vector $\mathbf{c}$, is tasked with retrieving the rest of the source attribute values associated with $\mathbf{s}_T$:  
\begin{equation}
\label{eq:comp}
    \begin{gathered}
    \widehat{\mathbf{c}}_{\text{full}} = g_\phi (\mathbf{x}, \mathbf{c}).
    \end{gathered}
\end{equation}
where $\widehat{\mathbf{c}}_{\text{full}}$ is an 8-dimensional vector consisting of the probabilities assigned by the model to each source attribute value. 

Next, after pre-training $g_\phi$, we aim to learn a conditional separation model $f_\theta$, which given the mixture $\mathbf{x}$ and the concatenation of the condition vector $\mathbf{c}$ and the estimated completed vector $\widehat{\mathbf{c}}_{\text{full}}$, can retrieve the target source $\mathbf{s}_T$ as well as the residual source $\mathbf{s}_O$:
\begin{equation}
\label{eq:sep}
    \begin{gathered}
    \widehat{\mathbf{s}}_T, \, \widehat{\mathbf{s}}_O = f_\theta (\mathbf{x}, \left[\mathbf{c}, \widehat{\mathbf{c}}_{\text{full}} \right]).
    \end{gathered}
\end{equation}
While, the separation model $f_\theta$ is being trained we keep the parameters $\phi$ of completion model $g_\phi$ frozen. By conditioning the separation model both on the original condition vector $\mathbf{c}$ as well as the estimated completion $\widehat{\mathbf{c}}_{\text{full}}$, we enable detection and correction of possibly erroneous estimates from the completion module. 