\section{Conclusion and Future Work}
In this paper, we propose \textbf{AntGPT} to investigate if large language models encode useful prior knowledge on bottom-up and top-down long-term action anticipation. Thorough experiments with two LLM variants demonstrate that LLMs are capable of inferring goals helpful for top-down LTA and also modeling the temporal dynamics of actions. Moreover, the useful encoded prior knowledge from LLMs can be distilled into very compact neural networks for efficient practical use. Our proposed method sets new state-of-the-art performances on the Ego4D LTA, EPIC-Kitchens-55, and EGTEA GAZE+ benchmarks. We further study the advantages and limitations of applying LLM on video-based action anticipation, thereby laying the groundwork for future research in this field.


\noindent\textbf{Limitations.} Although our approach provides a promising new perspective in tackling the LTA task, there are limitations that are worth pointing out. The choice of representing videos with fixed-length actions is both efficient and effective for LTA task. However, the lack of visual details may pose constraints on other tasks. Another limitation is the prompt designs of ICL and CoT are still empirical, and varying the prompt strategy may cause significant performance differences. Finally, as studied in our counterfactual experiments, the goal accuracy would have significant impact on the action recognition outputs, and an important future direction is to improve the inferred goal accuracy, and also take multiple plausible goals into account.

\noindent\textbf{Acknowledgements.} We would like to thank Nate Gillman for feedback. This work is in part supported by Honda Research Institute, Meta AI, and Samsung Advanced Institute of Technology.
