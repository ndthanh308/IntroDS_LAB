

\documentclass[12pt,leqno]{amsart}
\usepackage[dvips]{graphics}
\usepackage{amssymb}
\usepackage{verbatim}
\usepackage{hyperref}
\usepackage{color}
\newenvironment{pf}{\proof[\proofname]}{\endproof}

\numberwithin{equation}{section}
\newdimen\vintkern\vintkern12pt
\def\vint{-\kern-\vintkern\int}

\newtheorem{thm}{Theorem}[section]
\newtheorem{thmmain}{\tref{thm-main}}[section]
\renewcommand{\thethmmain}{}
\newtheorem{lem}[thm]{Lemma}

\newtheorem{cor}[thm]{Corollary}
\newtheorem{prop}[thm]{Proposition}
\newtheorem{conj}[thm]{CONJECTURE}
\newtheorem{quest}[thm]{Question}
\newtheorem{defn}[thm]{Definition}
\newtheorem{fak}[thm]{Fakt}
\newtheorem{rem}[thm]{Remark}
\newtheorem{ex}[thm]{Example}
\newtheorem{hyp}[thm]{HYPOTHESES}
\newtheorem{ft}[thm]{FAKT}
\newtheorem{satz}[thm]{SATZ}
%\theoremstyle{definition}
%\newtheorem{defn}{Definition}[section]
%\newtheorem*{proof}{Beweis}
%\theoremstyle{remark}
%\newtheorem{rem}{Remark}[section]
%\newtheorem{ex}[rem]{Example}

\newcommand{\tref}[1]{Theorem~\ref{#1}}
\newcommand{\cref}[1]{Corollary~\ref{#1}}
\newcommand{\pref}[1]{Proposition~\ref{#1}}
\newcommand{\rref}[1]{Remark~\ref{#1}}
\newcommand{\dref}[1]{Definition~\ref{#1}}
\newcommand{\secref}[1]{\S\ref{#1}}
\newcommand{\lref}[1]{Lemma~\ref{#1}}
\newcommand{\exref}[1]{Example~\ref{#1}}
\newcommand{\sref}[1]{Satz~\ref{#1}}
\newcommand{\fref}[1]{Fakt~\ref{#1}}
\newcommand{\Ant}{\mathrm{Ant}}
\newcommand{\Pol}{\mathrm{Pol}}
\newcommand{\vol}{\mathrm{vol}}
\newcommand{\pt}{\mathrm{pt}}
\newcommand{\rad}{\mathrm{rad}}
\newcommand{\scal}{\mathrm{scal}}
\newcommand{\Rad}{\mathrm{Rad}}
\newcommand{\Sub}{\mathrm{Sub}}
\newcommand{\Iso}{\mathrm{Iso}}
\newcommand{\Rg}{\mathrm{Rg}}
\newcommand{\Fix}{\mathrm{Fix}}
\newcommand{\R}{\mathbb{R}}

\newcommand{\Ss}{\mathbb{S}}

\newcommand{\N}{\mathbb{N}}
\newcommand{\hm}{{\mathcal H}}
\newcommand{\Area}{\operatorname{Area}}
\newcommand{\ap}{\operatorname{ap}}
\newcommand{\diam}{\operatorname{diam}}
\newcommand{\Id}{\mathrm{Id}}
\newcommand{\md}{\operatorname{md}}
\newcommand{\apmd}{\ap\md}
\newcommand{\lip}{\mathrm{Lip}}
\newcommand{\bilip}{\mathrm{bilip}}


\newcommand{\trace}{\operatorname{tr}}
\def\X{{X^\varepsilon}}
\def\di{{\bold div}\,}

\def\eps{\epsilon}
\newcommand{\injrad}{\operatorname{Injrad}}
\newcommand{\green}{\color{green}}
\newcommand{\red}{\color{red}}
\newcommand{\cyan}{\color{cyan}}
\newcommand{\blue}{\color{blue}}
\newcommand{\nc}{\normalcolor}
\def\co{\colon\thinspace}
\newcommand{\curv}{\mathrm{curv}}

\pagestyle{plain}

\begin{document}

	\pagebreak
	
	
	\title{Some regularity of Submetries}
	
\thanks{
	A. L. was partially supported by the DFG grants   SFB TRR 191 and SPP 2026.}

	
	\author{Alexander Lytchak}
	
	\address
{Institute of Algebra and Geometry\\ KIT\\ Englerstr. 2\\ 76131 Karlsruhe, Germany}
	\email{alexander.lytchak@kit.edu}

	

\keywords
{Alexandrov space, submetry, singular
Riemannian foliation, equidistant decomposition}
\subjclass
[2010]{53C20, 53C21, 53C23}
	
	 
	
	
	
	
	\begin{abstract}
We discuss   regularity statements for  equidistant decompositions of 
 Riemannian manifolds and for the corresponding quotient spaces. 
We show that any stratum of the quotient space
has  curvature locally bounded from both sides. 
	\end{abstract}
	
	\maketitle
	
	
	\renewcommand{\theequation}{\arabic{section}.\arabic{equation}}
	\pagenumbering{arabic}
	

	\section{Introduction}
	 A  \emph{submetry} is a map $P:X\to Y$  between metric spaces  which 
sends open balls $B_r(x)$  in $X$ onto  balls $B_r (P(x))$ of the same radius  in $Y$.
Submetries as  metric generalization of Riemannian submersions have been introduced by  Berestovskii \cite{Berest2}.   Berestovksii and  Guijarro  verified that a submetry between smooth complete Riemannian manifolds always is  a $\mathcal C^{1,1}$ Riemannian submersion, but  it does not need to be $\mathcal C^2$
 \cite{Berest}.  Another example of a submetry is provided  by a distance function 
$P:\R^n\to \R$ to a convex nowhere dense subset $C\subset \R^n$.

	

		
Submetries $P:X\to Y$ with given \emph{total space} $X$ are in one-to-one correspondence with equidistant decompositions
of  $X$. The correspondence assigns to $P$ the decomposition of $X$ into fibers of $P$ \cite[Section 2.2]{KL}.    Seen this way, submetries generalize quotient maps for isometric group actions and decompositions of a complete smooth Riemannian manifold into leaves of a singular Riemannian foliation with closed leaves \cite{Molino}, \cite{Thorb},  \cite{Alexandrino-survey}.



Recent appearances
of submetries in many unrelated settings: \cite{LPZ}, \cite{Stadler}, 
\cite{Ber-finite},
 \cite{Mendes-Rad}, \cite{Mendes-Rad2}, \cite{Grovesubmet},
  \cite{GW}, \cite{Rade}
	 \cite{GW}, 
 \cite{Lyt-buildings}, make investigations of the properties of submetries  a very natural task, especially if the total space is a Riemannian manifold.
A systematic study of properties of submetries $P:M\to Y$ with total space a sufficiently smooth Riemannian manifold has been intitiated in \cite{KL}. The present paper continues the investigations of \cite{KL} and improves some regularity statements provided there. 












If the total space  $X$ is a connected (sufficiently smooth) complete Riemannian manifold $M$,  the following structural results on the base space $Y$ of a submetry $P:M\to Y$  have been derived in \cite{KL}.

The quotient space $Y$ locally has curvature bounded from 
below,   \cite{BGP}, \cite[Proposition 3.1]{KL}. There is a canonical stratification 
$Y= \cup _{l=0} ^m Y^l$, where $m$ is the dimension of $Y$
and $Y^l$ consists of all points $y\in Y$ such that the tangent space $T_yY$ has $\R^l$ as a direct factor, \cite[Theorem 1.6]{KL}. The subset $Y^l$ is locally convex in $Y$, for any $l$, and  it is an $l$-dimensional manifold.  The maximal-dimensional stratum $Y^m$, the set of
\emph{regular points} of $Y$, is open, dense and convex in $Y$.


For any point $y\in Y$ there exists some  $r>0$, such that exponential map $\exp_y$ is a  well-defined homeomorphism $\exp _y:B_r(0) \to B_r(y)$
between the $r$-ball in the tangent cone $T_yY$ around the origin and the $r$-ball 
in $Y$ around $y$, \cite[Theorem 1.3]{KL}.
This  \emph{injectivity radius} is locally bounded from below on each stratum $Y^l$,
but it goes  to $0$, when points on $Y^l$ converge to a lower-dimensional stratum.

Our first result improves the regularity of the exponential map:

\begin{thm} \label{thm: exp}
Let $Y$ be a base of a submetry $P:M\to Y$ of a Riemannian manifold with locally bounded curvature.
Then, for any $y\in Y$, there exist $r_0,C>0$, such that for all $r<r_0$ the exponential map $\exp _y:B_r(0)\to B_r(y)$ is $(1+Cr^2)$-bilipschitz. 
\end{thm}

Here and below, we use the notion of   a \emph{Riemannian manifold with locally bounded curvature} to describe a  manifold without boundary with a continuous Riemannian metric,
which has  curvature bounded locally from above and below  in the sense of Alexandrov see \cite{Ber-Nik}, \cite{KL-both}. In particular, any $\mathcal C^{1,1}$-submanifold of a Riemannian manifold with a $\mathcal C^{1,1}$-Riemannian metric is in this class \cite[Proposition 1.7]{KL-both}.  


 Theorem \ref{thm: exp} can be informally   understood as the existence of
 a pointwise  both-sided curvature bound  at any point $y\in Y$. Indeed, for a smooth Riemannian manifold $M=Y$, the optimal number $C$ in the statement of Theorem \ref{thm: exp} is  equivalent  (up to a factor) to  the optimal bound on the norm of the sectional curvatures at $y$.


In Theorem \ref{thm: exp}, the constant  $r_0(y)$ \emph{always} goes to $0$ and $C(y)$ \emph{usually}  goes to infinity, when $y$ converges to a lower stratum, \cite[Proposition 8.9]{KL},  \cite[Theorem 1.1]{Thorb}.
But both constants can be chosen
\emph{locally uniformly}  on any stratum, Theorem \ref{prop: bil} below.  This has the following consequence, which answers \cite[Question 1.12]{KL}:

\begin{cor} \label{cor: expstr}
Let $M$ be a Riemannian manifold with curvature locally bounded curvature
 and let 
$P:M\to Y$ be a submetry.  Then, any stratum $Y^l$ is a Riemannian manifold locally bounded   curvature.
\end{cor}









In general, fibers of a submetry $P:M\to Y$ can be arbitrary subsets of positive reach in $M$ (this is a common generalization of convex subsets and 
$\mathcal C^{1,1}$ submanifolds \cite{Federer}, \cite{Ly-conv}, \cite{Rataj}).
However, most fibers are $\mathcal C^{1,1}$-submanifolds and any fiber  $L$
of $P$ contains a $\mathcal C^{1,1}$-submanifold, open and dense in $L$. 
A by-product of the proof of Theorem \ref{thm: exp} is the following result saying that for any submetry $P:M\to Y$   the projections from nearby $P$-fibers onto  a manifold $P$-fiber is \emph{almost} a submetry.  We formulate it as a global result for compact fibers and refer to 
Theorem \ref{thm: semicont} for a more general local version.

\begin{prop} \label{prop: almsubm}
Let $P:M\to Y$ be a submetry, where $M$ has locally bounded curvature.
Let $L$ be a fiber of $P$ which is a compact manifold.  Then there exist constants $C,r_0>0$
such that for all fibers  $L'$ of $P$ at distance $r<r_0$ from $L$, the closest point projection  $\Pi^L :L'\to L$ is $(1+Cr)$-Lipschitz and locally $(1+Cr)$-open.
\end{prop}

Recall that a map $f:X\to Y$ between metric spaces is \emph{ locally  $C$-open},  (other terms used are \emph{Lipschitz open} and \emph{co-lipschitz}) if  for any $z\in X$ there exists $r_0>0$, such that,
 for any $r_0>r$ and any $x\in B_{r_0} (z)$,
$$B_r(f(x))\subset f(B_{Cr} (x))\,.$$
A submetry $P:M\to Y$ is called \emph{transnormal} if all fibers of $P$ are $\mathcal C^{1,1}$-submanifolds. Thus, for transnormal submetries with compact fibers  the conclusion of Proposition \ref{prop: almsubm} is true for all fibers. 


Theorem \ref{thm: exp} implies that the local decomposition of the  base space $Y$ in strata around a point $y$ corresponds to the decomposition in strata of the 
tangent space at $y$, see Corollary \ref{lem: locinf} below. This has the following 
implication to \emph{transnormal submetries}:

\begin{cor} \label{cor: last}
Let $P:M\to Y$ be a transnormal submetry, where $M$ has locally bounded curvature.
   Let $\gamma:I\to M$ be a horizontal geodesic.  Then, up to discretely many values $t_i\in I$, the connected component of fiber of $P$ through $\gamma (t)$ has the same dimension $k=k(\gamma)$   and $P(\gamma (t))$ is contained in the same stratum $Y^l$, with $l=l(\gamma)$.
\end{cor}

%Along such horizontal geodesics of  transnormal  fibers, there is a well-defined  holonomy.

  As a related consequence of Proposition \ref{prop: almsubm}, we prove that the holonomy maps between fibers of transnormal submetries are always Lipschitz-open, see Proposition \ref{prop: hol}, below.

We mention,  that all results stated here and  below  do not require  completeness of $M$ and are valid for \emph{local submetries}, see   Section \ref{subsec: loc}.  


{\bf Acknoweldgements} The author is grateful to Ricardo Mendes for his interest and helpful comments.









\section{Manifolds with locally bounded curvature}
\subsection{Notation}
By $d$ we denote the distance in metric spaces.
For a subset $A$ of a metric space $X$ we denote by $d_A:X\to \R$ the distance function to  $A$.
A \emph{geodesic} will denote an isometric  (i.e. globally  distance preserving) embedding of an interval.
A local geodesic $\gamma:I\to X$ is a curve whose restrictions to small sub-intervals are geodesics. 

We fix $\varepsilon = 10^{-4}$ for the rest of the paper.






\subsection{Curvature bounds}
We assume some familiarity with spaces with curvature bounded below  or/and  above in the sense of Alexandrov, we refer to \cite{Ber-Nik},  \cite{AKP} for the theory of these spaces.


 By a \emph{manifold with locally bounded curvature}  $M$ we mean a length metric space homeomorphic to a manifold without boundary, such that any point $x\in M$ has a convex neighborhood in $M$ which is an Alexandrov space of curvature bounded from above by $\kappa$ and from below by $-\kappa$, for some $\kappa \in \R$.  We allow the manifold $M$ to be non-complete and the value $\kappa$ to be not globally bounded on $M$.


The \emph{distance coordinates} define a $\mathcal C^{1,1}$-atlas on  any such manifold $M$ and the Riemannian metric is Lipschitz continuous in these coordinates, \cite{Ber-Nik}.
Any $\mathcal C^{1,1}$-submanifold $N\subset M$ also has locally bounded curvature in its intrinsic metric \cite[Proposition 1.7]{KL-both}. 




In any manifold $M$ with locally bounded curvature there is the notion of parallel translation along  any Lipschitz curve \cite[Section 13]{Ber-Nik}.  Let $U$ be a relatively compact small ball in such a manifold $M$ and 
assume that the curvature is bounded on $U$ by $\pm \kappa$ in the sense of Alexandrov.
Then there exist a sequence of smooth metrics $g_n$ on $U$ such that the sectional  curvatures of $g_n$ are bounded in norm by $\kappa +\delta_n$ with $\lim_{n\to \infty} \delta _n =0$ \cite[Section 15]{Ber-Nik}. Moreover, the metric tensors $g_n$ converge to $g$ in $\mathcal C^{0,1}$
and the parallel transport for $g_n$ converges to the parallel transport in $g$ uniformly.


This approximation result allows us to prove most metric statements in the smooth case first and then to obtain the general case by a limiting procedure. Mostly, a more direct but technical explanation is available without using the approximation theorem.  
The main additional tool available in the smooth situation are Jacobi-fields, which  only have almost everywhere analogues in the general case.
Readers not aquainted 
with the theory of \cite{Ber-Nik} may always assume the  total manifold $M$ to be smooth.
We prefer to stick to the more general setting of manifolds with locally bounded curvature,  since this setting seems to be  appropriate for the study of submetries,
see \cite{KL}.



\subsection{Comparison of tangent vectors at different points}
Let $M$ be a Riemannian manifold  with locally bounded curvature. Let  $O\subset M$ be open, uniquely geodesic and  convex.  Given $x,z\in O$    and vectors $v\in T_xO, w\in T_zO$,  we define  $|v-w|$ to be the distance in $T_xO$ between $v$ and the parallel transport $w'$ of $w$ to $T_xO$ along the geodesic $xz$.

This "quasi-distance" is symmetric but satifies the triangle inequality only up to a defect depending on the geometry  of $O$, see \eqref{eq: tri} below.

For  linear subspaces $W_x\subset T_xO$ and $W_z\subset T_zO$ with $\dim (W_x)=\dim (W_z)$, we  denote by $|W_x-W_z|$ the  symmetric "quasi-distance":
$$|W_z-W_x|:= \sup \{d(W_x,w')  \}\,.$$
Here ,the distance $d(W_x,w')$ to the subspace $W_x$ is measured in $T_xO$, and the supremum
is taken over all parallel translates $w'\in W_x$ of unit vectors  $w\in W_z$ along the geodesic $xz$. 


\subsection{Almost flat domains}
We refer to \cite[Section 6]{BK} for estimates of parallel translations stated below.
Let $M$ be  a Riemannian manifold and $O$  an open 
subset of $M$.  We  say that $O$ is \emph{almost flat} if the following holds true:
  The closure $\bar O$ is compact and has diameter at most $20$. The set $O$ 
is convex, uniquely geodesic and the  curvature is  bounded above and below by
$\pm \varepsilon$.

Any sufficiently small relatively compact  ball around any point $x_0$ in a Riemannian manifold $M$ with locally bounded curvature  becomes almost flat upon rescaling.  Any open ball $B$  of radius $r$ 
in an almost flat domain $O$ is still almost flat, if it is rescaled by some $1\leq  \lambda
\leq \frac {10} r$.

For any ball $B_r(x) \subset O$, the exponential  map $\exp _x :B_r(0)\to B_r(x)$ is $(1+\varepsilon r^2)$-bilipschitz on the ball $B_r(0) \subset T_xO$.



For a triangle $\Delta=xyz \subset O$ with two sides of length $l_1,l_2$,
the holonomy along $\Delta$  of any  $v \in T_xO$ satisfies \cite[Proposition 6.2]{BK}
$$||Hol ^{\Delta} (v)-v||  \leq  \varepsilon \cdot l_1\cdot l_2 \cdot ||v||\;.$$ 
For such $\Delta$ and arbitrary 
$v_x\in T_xO, v_y\in T_yO, v_z \in T_zO$, set $a:=\min \{||v_x||,||v_y||,||v_z||\}$. 
Then a  triangle inequality holds with a defect:
\begin{equation} \label{eq: tri}
|v_x-v_z|\leq |v_x-v_y |+|v_y-v_z| +\varepsilon \cdot l_1\cdot l_2\cdot a\;.
\end{equation}

Let $x\in O$  and  $v_0,h\in T_xO$ be given. Let $b=\max \{||v_0||, ||h|| \}$.
Assume that $B_{2b} (x) \subset O$.   Set $x_1=\exp _x (h),  z=\exp _x(v_0)$.
Let $h_1$ be the parallel translate of $h$ to $T_zO$ along the geodesic $xz$.
Set $z_1=\exp _{z} (h_1)$.  Set finally $v_1:= \exp _{x_1} ^{-1} (z_1)$.  Then 
the following esitmate is obtained  by a two-fold  application  of  \cite[Proposition 6.6]{BK}:
\begin{equation} \label{eq: new}
|v_0-v_1| \leq 3\varepsilon \cdot  b \cdot ||v_0||\cdot ||h|| \,.
\end{equation} 
In particular, 
$|d(x,z) -d(z,z_1)| \leq 3\varepsilon \cdot b \cdot ||v_0||\cdot ||h|| \,.$





From the estimate \eqref{eq: new} we are going to  deduce:

\begin{lem} \label{lem: karcher}
Let $x_t:=\eta (t)$ and $z_t:=\gamma (t)$ for two geodesics $\eta, \gamma :[0,1]\to O$ 
 in an almost flat domain $O$.  Set $a_t=d(x_t,z_t)$ and
$v_t:=\exp _{x_t} ^{-1} (z_t) \in T_{x_t} O$.
 If  $10^{-2}  > 2a_0 >a_1$  then, for any 
$t \leq  10^{-1}$,
%$$||a_ -a_0||\leq  Cr$$ 
\begin{equation} \label{eq: lem}
|v_t-v_0| \leq 10 \cdot a_0 \cdot t\;.
\end{equation}
\end{lem}



\begin{proof}
 Let $h\in T_{x_0} O$ and $w\in T_{z_0} O$ be the starting directions of $\gamma $ and $\eta $ respectively.
Considering the holonomies along the triangles $x_0 x_1 z_0$ and $x_0z_1z_0$ we 
obtain the bound $|h- w|\leq 4 a_0$.
Thus, if $ h_1$ denotes  the parallel translate of $h$ to $T_{z_0} O$  and 
$\tilde z_t =\exp _{z_0} (t\cdot \tilde h)$  then 
\begin{equation}  \label{eq: verylast}
d(z_t,\tilde z_t) \leq 5\cdot a_0 \cdot t.
\end{equation}


On the other hand, the vector $\tilde v_t:= \exp _{x_t} ^{-1} (\tilde z_t)$ satisfies
  $$|\tilde v_t -v_0| \leq \varepsilon \cdot t\cdot a_0\,,$$
  by  \eqref{eq: new}. Together with \eqref{eq: verylast} this implies \eqref{eq: lem}.
\end{proof}



	
	
	
	
	

\section{Basics on submetries} 
\subsection{(Local) Submetries} \label{subsec: loc}
Recall that
 $P:X\to Y$ is a  \emph{submetry} if for any $x\in X$ and any  $r>0$ the equality
$P(B_r(x))= B_r(P(x))$ holds. 

The map $P$ is called a \emph{local submetry} if for any $x\in X$ there exists some $s>0$ such that the condition $P(B_r(z))= B_r(P(z))$ holds true for any $z\in B_s(x)$ and any $r<s$. 
 We call $X$ the \emph{total space} and $Y$ the \emph{base} of  the local submetry $P$.

By definition, $P$ is a submetry if and only if it is locally $1$-Lipschitz and locally
$1$-open.
A restriction of a (local) submetry $P:X\to Y$  to an open subset $O\subset X$ is a local submetry $P:O\to Y$.  A local submetry $P:X\to Y$ is a global submetry, if   $X$ and $Y$ are length spaces and $X$ is proper \cite[Corollary 2.9]{KL}.

Let $P:X\to Y$ be a local submetry and  let $X$ be a length space.
Replacing  $Y$ by $P(Y)$ 
 we may  assume that the local submetries are surjective.  
Replacing the metric on $Y$ by the induced length  metric, $P$ remains a local submetry
\cite[Corollary 2.10]{KL}.
%We will only consider the case that the total space $X$ is a smooth connected  Riemannian manifold $M$.  
%If $P$ is a (global) submetry then $Y$ is automatically a length space \cite[Proposition 1]{Berest2}. If $P$ is a local submetry then we may replace the metric on $Y$ by the induced length metric  and the map $P$ remains a local submetry
%\cite[Corollary 2.10]{KL}. 
 Thus, we may  assume without loss of generality that the base space $Y$ is a length space.





%Any local submetry $P:X\to Y$ is an open map which is locally $1$-Lipschitz. In particular, $P$ does not increase length of curves.



For a local submetry $P:X\to Y$, a rectifiable curve $\gamma :I\to X$  is \emph{horizontal} (with respect to $P$) if $\ell (\gamma)=\ell (P\circ \gamma)$.   %In this case, we call $\gamma$  a \emph{horizontal lift} of $P\circ \gamma$.



%Let $P:X\to Y$ is a local submetry  and let $\gamma :[0,a] \to Y$ be a rectifiable  curve in $Y$. If $x\in P^{-1} (\gamma(0))$ is such that
%the closed ball $\bar B_r (x)$ with $r=\ell (\gamma)$ is compact in $X$ then there exists a horizontal lift of $\gamma$ starting in $x$ \cite[Lemma 4.4]{Lyt-open}.


%A local submetry is a submetry if the total space $X$  is proper, in particular, if $X$ is a complete Riemannian manifold.

%	A map $P:M\to N$ between smooth Riemannian manifolds is a local submetry if and only if it is a $\mathcal C^1$-Riemannian submersion \cite{Berest}.





 
  
   
 
  
  
  
  
\subsection{Structure of the base} \label{subsec: basestructure}
From now on let $M$ denote a  manifold with locally bounded curvature.

Let $P:M\to Y$ be a surjective local submetry.  Let $y\in Y$ be arbitrary.
 Then 
there exists some $r>0$ such that the closed ball
$\bar B_r(y)$ is an Alexandrov space of curvature $\geq \kappa$, for some $\kappa =\kappa (y)$.  In particular, $\bar B_r (y)$ is convex in $Y$  \cite[Proposition 3.1]{KL}.
Moreover, any geodesic $\gamma:[0,t] \to Y$ starting in $y$ can be extended  to a geodesic $\gamma :[0,r] \to Y$ up to the distance sphere  $\partial  B_r(y)$
\cite[Theorem 1.3]{KL}.
 In this case we will say that the \emph{injectivity radius  at $y$ is at least $r$}.  Under the above assumptions, any point $y'\in B_r(y)$ is connected to 
$y$ by a unique geodesic, justifying the name.


Set $ m=\dim (Y)$.
Then  $Y$ admits a canonical  decomposition  $Y= \cup _{l=0} ^m Y^l$ into \emph{strata} $Y^l$.  Here, $Y^l$ is the set of all points $y\in Y$, for which  the tangent space $T_yY$ splits off $\R^l$ but not $\R^{l+1}$ as a direct factor.  
$Y^l$ is an $l$-dimensional  manifold with a canonical $\mathcal C^{1,1}$-atlas, which is  locally convex
 in $Y$, \cite[Theorem 1.6]{KL}. 
The metric on $Y^l$ is given  by a Lipschitz continuous Riemannian metric;  the tangent space $T_yY^l$ is the maximal Euclidean factor of 
$T_yY$ \cite[Theorem 11.1]{KL}.


For any point $y\in Y^l$, there exists some $r_0=r_0 (y)>0$ with the following properties  \cite[Lemma 10.1, Theorem 11.1]{KL}:
 the open ball $B_{2r_0}(y)$ does not contain  points in $\cup _{i=0}^{l-1}Y^i$
and, for  any $y'\in B_{r_0}(y) \cap Y^l$, the injectivity radius  at $y'$ is at least $r_0$.


% if the injectivity radius at $y$ is at least $r$, \cite[Lemma 10.1]{KL}.  Moreover, 
%For any $y\in Y^l$, we  call $Y^l$ the startum of $Y$ through $y$.  The connected component $E=E_y$ of $Y^l$ through $y$ will be called
% denote by $E$ the connected component of $Y^l$ which contains $y$ and call it
% the \emph{connected stratum of $y$}.  The closure $\bar E$  of such a connected stratum
%$E$ is a \emph{primitive extremal subset} of $Y$  \cite{Pet-Per}, \cite[Theorem 11.2]{KL}.
%For any $y\in Y^l$, we  find 
%there exists
%some $r_0>0$ with the following properties \cite[Theorem 11.1]{KL}: % The ball $B:=\bar B_{r} (y) $ in $Y$ is compact  and convex.
% For any
%$y'\in B \cap Y^l$ and any $y''\in B$ there is a unique geodesic between $y'$ and $y''$.
%The injectivity radius of any $y'\in B_{r_0}(y) \cap Y^l$ is at least $r_0$.   For the canonical $\mathcal C^{1,1}$-atlas on $Y^l$,  the distance functions to any $y'\in B_{r_0} (y)\cap Y^l$ is a $\mathcal C^{1,1}$ function on $B_{r_0}(y)\cap Y^l \setminus \{y' \}$.







\subsection{Structure of the fibers}
A locally closed subset $L\subset M$ has
\emph{positive reach} in $M$ if the closest-point projection $\Pi^L$ in uniquely defined on a neighborhood $U$ of $L$ in $M$.   In this case,  $\Pi^L$ is locally Lipschitz on $U$ and the distance function $d_L$ is $\mathcal C^{1,1}$ on $U\setminus L$ \cite{KL-both}.




% If this neighborhood  $U$ contains the tubular $r$-neighborhood around $L$,   we say $L$
 %that has \emph{positive reach at least $r$}. 


 

  Any set $L$  of positive reach is a topological manifold if and only if $L$ is a $\mathcal C^{1,1}$ submanifold of $M$, \cite[Proposition 1.4]{Ly-conv}.  On the other hand, any set $L$ of positive reach contains a subset $L'$ open and dense  in $L$, which is a $\mathcal C^{1,1}$-submanifold, possibly with components of different dimensions \cite[Theorem 7.5]{Rataj}.
  
  
  
Let $P:M\to Y$ be a surjective local submetry.  Let $y\in Y^l\subset Y$.  Then the  fiber $L=P^{-1} (y)$ and the preimage
$S=P^{-1} (Y^l)$  are subsets of positive reach in $M$ \cite[Theorems 1.1,  1.7]{KL}.


Neither $L$ nor $S$ have to be manifolds. 
However, for every $y\in Y\setminus \partial  Y$ the fiber $L=P^{-1} (y)$ is   a $\mathcal C^{1,1}$-submanifold of
$M$ \cite[Theorem 1.8]{KL}.  In particular, this applies to all
$y\in Y^m$ with $m=\dim (Y)$.


%Let $y\in Y^l$ and $x\in L=P^{-1} (y)\subset S=P^{-1} (Y^l)$ be arbitrary. There exists some $r_0>0$ and a neighborhood $U$ of $x$ in $M$, such that  
%  For any point $x\in L$ there exists an open  neighborhood $U$ of $x$ in $M$ such that
% $P(U) = B_s(y)$,  and such that the restriction $P:U\cap P^{-1} (Y^l) \to B_{r_0}(y)\cap Y^l$ is a trivial fiber bundle, \cite[Proposition 11.3]{KL}. Moreover, if $L\cap U$ is a $\mathcal C^{1,1}$-submanifold then
%	$U\cap S$ is a $\mathcal C^{1,1}$-submanifold of $M$ and the restriction 
%	$P:U\cap S\to Y^l$ is a $\mathcal C^{1,1}$ Riemannian submersion.

	





\subsection{Infinitesimal structure}
Let  $P:M\to Y$ be  a local submetry, let $x\in M$  be arbitrary,  $y=P(x)$ and denote by $L$ the fiber
$P^{-1} (y)$.  



There exists a  differential $D_xP:T_xM\to T_yY$, which is itself a submetry. % For any $\lambda >0$ and $v\in T_xM$.thus    a submetry 
%equivariant with respect to scalar multiplication:  $D_xP (\lambda \cdot v)=\lambda \cdot D_xP (v)$, for any $\lambda >0$ and $v\in T_xM$.
The tangent  space $T_xL$ is  the preimage $D_x P^{-1} (0)$ and it is a convex cone in  $T_xM$  \cite[Proposition 3.3, Corollary 3.4]{KL}.
 We call   $T_xL$  the \emph{vertical space} at  $x$ and denote it by $V^x$.


%The subset of positive reach $L\subset M$ has at $x\in M$ a well-defined tangent space $T_xL\subset T_xM$ at any $x\in L$  which is a convex subcone of the Euclidean space $T_xM$.
%The subset $L$ is a $\mathcal C^{1,1}$-submanifold (possibly not connected) 
%if and only if  the tangent space $T_zL$ is a Euclidean space, for all $z\in L$


%There exists a  differential $D_xP:T_xM\to T_yY$, which is itself an \emph{infinitesimal submetry}, thus a submetry between two cones
%equivariant with respect to scalar multiplication in the cone:  $D_xP (\lambda \cdot v)=\lambda \cdot D_xP (v)$. The tangent 
%space $T_xL$ is exaclty  the preimage $D_x P^{-1} (0)$  \cite[Proposition 3.3, Corollary 3.4]{KL}  We call  the cone $T_xL$  the \emph{vertical space} at the point $x$ and denote it by $V^x$.

The \emph{horizontal space} $H^x$ is the dual cone of $T_xL$ in $T_xM$.  The cone  $H_x$ consists of all  $h\in T_xM$ such that
$||h||= |D_xP(h)|$, where  $|\cdot |$ on the right side denotes the distance  to the origin of $T_yY$.

 

A Lipschitz curve $\gamma:I\to M$ is horizontal if and only if the vector $\gamma'(t)$ is horizontal, for almost all $t\in I$.






\section{$P$-almost flat domains}
Let $P:M\to Y$ be a local submetry, where $M$ has locally bounded curvature.  Fix $y\in Y$  and consider $L=P^{-1} (y)$.
Let $x_0\in L$ be a point such that a neighborhood of $x_0$ in $L$ is a $\mathcal C^{1,1}$-submanifold.  
Note, that the set of such points $x_0$ is open and dense in $L$. 

  We rescale $M$ and $Y$  and assume that the ball  $B:= B_{10}(x_0)$ is an almost flat domain and $B_{10} (x_0) \cap L$ is a $\mathcal C^{1,1}$-submanifold of $M$.  

Let $Y^l$ denote the stratum of $Y$ containing $y$. 
We choose  a small $r_0$ as in Subsection \ref{subsec: basestructure}.  Rescaling again and applying \cite[Proposition  11.3, Theorem 1.7]{KL} 
we may assume  the following: The stratum $Y^l\cap B_{10}(y)$
is a convex $\mathcal C^{1,1}$-manifold with a Lipschitz continuous Riemannian metric. The injectivity radius at any  $y'\in Y^l \cap B_{10} (y)$ is at least 10.
 The preimage $S:=P^{-1}(Y^l) \cap B_{10} (x_0)$ is a $\mathcal C^{1,1}$ submanifold of $M$ and the restriction $P:S\to Y^l$ is a $\mathcal C^{1,1}$ Riemannian submersion. 


%For any  $y'\in  B_{10}(y)$ there exists a unique closest-point to $y'$ on $E$.
%Finally, the injectivity radius at any $y_1\in E\cap B_{10} (y)$ is at least $10$. 

%The preimage $S=P^{-1}(E) \cap B_{10} (x_0)$ is a $\mathcal C^{1,1}$-submanifold of $M$.  Moreover, the restriction $P:S\to E$ is a $\mathcal C^{1,1}$ Riemannian submersion. 
 In particular, the vertical spaces in $S$ depend Lipschitz-continuously on the points.
Rescaling, we may assume that,
for all $x_1,x_2\in S$,
\begin{equation} \label{eq: cr}
|V^{x_1}-V^{x_2}|\leq \varepsilon \cdot d(x_1,x_2)\;.
\end{equation}

By the assumption on the injectivity radius at points in $B_{10} (y)$, for any 
$x_1\in B_{5} (x_0) \cap S $ the closest-point  projection 
$\Pi ^{L'}: B_{5}(x_1)\to L'$  onto  the fiber $L'$  of $P$ through $x_1$ is 
well-defined.  Moreover, for some $r_1>0, C_1>0$ dependent on $x_0$ but not on $x_1$, and  for all $r<r_1$ the Lipschitz
 constant of $\Pi^{L'}$  is at most
\begin{equation} \label{eq: lip}
\lip (\Pi^{L'}  :B_r (x_1)\to L') \leq 1+C_1\cdot r\,.
\end{equation}
This is a special case of \cite[Theorems 1.3, 1.6]{Ly-conv}, since the fibers $L'$ as above are uniformly $\mathcal C^{1,1}$ near $x_0$.







For any sequence $z_j \to x_0$ in $M$, any Gromov--Hausdorff limit of (a any subsequence of) the vertical spaces $V^{z_j}$ contains $V^{x_0}$,  \cite[Corollary 8.4]{KL},.  Thus, we find some $r_1 >0$ such that for any $z\in B_{r_1} (x_0)$ and any unit vector $v\in V^{x_0}$, there exists some $v'\in V^z$ with
\begin{equation} \label{eq: eps}
|v-v'| \leq \varepsilon.
\end{equation}

Upon another rescaling, we may assume $r_0=r_1=10$ and $C_1=\varepsilon$.
In this situation, we will say that $B=B_{10} (x_0) $ is a \emph{$P$-almost flat domain}.
Thus, \eqref{eq: eps}, \eqref{eq: lip} and  \eqref{eq: cr} are valid with $C_1=\varepsilon$,  for all $r\leq 10$, all 
$z\in B$, all $x_1,x_2\in S=P^{-1} (Y^l) \cap B$. 





\section{Projection onto a fiber}


We aim to strengthen  \eqref{eq: eps}, \eqref{eq: lip}, \eqref{eq: cr}.



\begin{thm} \label{thm: semicont}
 $P:M\to Y$ a local submetry.  Let $x_0\in M$ be such that a neighborhood of $x_0$ in  $L=P^{-1}(P(x_0)) $ 
%a point, let $L$ be the fiber through $x_0$. Assume that a neighborhood of $x_0$ in $L$ 
is a $\mathcal C^{1,1}$-submanifold.  Then there exist  $r_0>0, C>0$ with the following properties, for any $0<r<r_0$ and any  $z\in B_r(x_0)$.
\begin{enumerate}
\item  For  any  $v\in V^{x_0}$ there exists  $v'\in V^z$ with
$|v-v'| \leq C\cdot r \cdot ||v||$. 

\item For any $h\in H^z$ there exists  $h'\in H^{x_0}$ with
$|h-h'| \leq C\cdot r \cdot ||h||$. 


\item  For  $L^z= P^{-1} (P(z))$, the closest-point projection $\Pi^L:L^z\cap B_r(x_0)\to L$ is $(1+Cr)$-Lipschitz and  locally $(1+Cr)$-open. 
\end{enumerate}
\end{thm}

Two comments before we embark on the proof.

\begin{rem} \label{rem: 1}
The numbers $C,r_0$ depend only on the value of rescalings needed to make a neighborhood  of $x_0$ a $P$-almost flat domain.  Therefore,  
% the curvature of $M$, the extrinsic curvature  of $L$
%and   an initial weak form of the semi-continuity (1) above, as provided by \cite[Corollaries 8.4]{KL}).   As the proof  and \cite[Theorem 11.1]{KL} reveal,
 the constants $r_0,C$  can be chosen  uniformly in a neighborhood of $x_0$  within  $P^{-1} (Y^l)$, where $P(x_0)\in Y^l$.
\end{rem}


\begin{rem}
For all small $r>0$, the intrinisic metric on the   ball  $B_r(x_0) \cap L$ differs from the 
extrinsic metric at most by the factor $(1+A\cdot r^2)$, for some constant $A=A(x_0,L,M)$,
\cite[Definition 1.1, Theorem 1.3]{Ly-conv}.  Thus, in Theorem  \ref{thm: semicont}(3), it does not matter whether we consider $L$ with  the intrinsic or with the extrinsic metric.
\end{rem}


Turning to Theorem \ref{thm: semicont} and using its notation,  we first prove:

%We are going  to  verify the following weaker form of  the theorem.
\begin{lem} \label{lem: 4op}
Let $B_{10} (x_0) \subset M$ be   $P$-almost flat.
Then the restriction
 $\Pi^L:L'\cap B_2 (x_0)\to L$ is locally $2$-open, for any fiber $L'$ of $P$.
\end{lem}

\begin{proof}
Consider any $z'\in L'\cap B_ 2 (x_0)$.   Set $r= d(L,z') <2$.  Consider some $\delta <\varepsilon = 10^{-4}$, so that  $B_{10\delta r} (z')\subset B_2 (x_0)$.

 Consider an arbitrary $q'\in B_{\delta r } (z') \cap L'$.  Set $q=\Pi^L (q')$. 
Consider any $p\in L$ with $t:=d(p,q) <\delta r$.  It is sufficient to find $p'\in L'$ with
 $\Pi ^L (p')=p$ and $d(p',q')\leq 2t$.

Consider the ball $B=B_3(0)$ in the horizontal space $H^p$ and  the  subset $K =\exp _p ( B_3 (0)) \subset M$. Note  that $\Pi ^L (K)=p$.
For the distance function $f=d_K$,
we claim that 
\begin{equation} \label{eq: fq}
f(q') \leq  t \cdot (1+20\varepsilon \cdot r)  
\end{equation}
Indeed, for  $h=\exp _q^{-1} (q')$,  we have $||h||=r$.   Due to \eqref{eq: cr},  for the parallel translate $h_1$ of $h$ to $p$,  we  find a vector $\tilde h \in H^p$ with 
 $$||\tilde h- h_1||\leq 2 \varepsilon\cdot t \cdot r\,.$$
Then $d(\exp _p (\tilde h), \exp _p(h_1)) \leq  3 \varepsilon\cdot t \cdot r$. From 
\eqref{eq: new}, we deduce 
$$d(q', \exp _p (h_1)) \leq t +9\varepsilon \cdot r \cdot t \,.$$
Thus, the triangle inequality implies  \eqref{eq: fq}.  

Hence, we find  a closest point $\hat p$  to $q'$ on $K=(\Pi^{L})^{-1} (p)$, such that  
$$d(q',\hat p) \leq (1+ 20 \varepsilon \cdot r)\cdot d(q,p)\;.$$

However, $\hat p$ may not  be contained in $L'$. In order to find a point $p'\in L' \cap K$ sufficiently close to $q'$, we 
are going to show that  $f=d_K$ decreases on  $L'$   sufficiently fast. More precisely,
we claim that for some  unit vector $v'\in V^{q'}$
the differential of $f$ at $q'$ satisfies 
\begin{equation} \label{eq: dq}
D_{q'} f(v') \leq -1+100\varepsilon\;.
\end{equation}

In order to verify \eqref{eq: dq}, let $\hat p$ be a closest point of $q'$ on $K$. 
Let $u'$ be the starting  vector of $\hat p q'$ and  $ u$  the parallel translate of
$u'$ to $T_pM$.

The geodesic $\hat p q'$ is orthogonal to $K$, by the first variation formula.
Using that $\exp _p^{-1} : K \to H^p$ is $(1+9 \varepsilon)$-bilipschitz, we deduce that the angle between $\hat u$ and any vector in $H^p$  is at least $\frac  \pi 2 - 90 \varepsilon$.      


 Hence, we find a unit vector $v\in V^p$ such that the the angle between $v$ and $\hat u$ is at most $90 \varepsilon $.  Applying    \eqref{eq: cr} and \eqref{eq: eps}, we find a unit vector 
$v'\in V^{q'}$ such that the angle
between $v'$ and the starting direction of $q'p$ is at most $100 \cdot \varepsilon$.
Then the first variation formula for the distance function $f=d_K$ implies \eqref{eq: dq}.



% By \eqref{eq: fq} and 
%\eqref{eq: lip} we have
%$$\frac 4  5 \cdot t \leq d(q',\hat p) \leq \frac 4 3 \cdot t\;.$$
%Set $u':= \exp _{\hat p} ^{-1} (q')$ and let $u$ be the parallel translate of $u$
%to $T_pM$.  Then the angle between $u$ and any vector in $H^p$ is at least 






% and  $h_0:=\exp _{p}^{-1} (\hat p)$. As above,  set
%$w:=\exp _{p}^{-1}  (q')$. Finally, set $u_0:=w-h_0$.%
%We are going to relate \eqref{eq: dq} to the new claim: 
%\begin{equation} \label{eq: u0}
%d(u_0, H^p)\geq \frac  4 {5}\cdot  ||u_0||.
%\end{equation}


The same argument provides for \emph{any} point $q_1\in (B_{2t} (q') \cap L') \setminus K$, a unit vector $v_1'\in V^{q_1} =T_{q_1} L'$  with
$D_{q_1} f(v_1') \leq - 1+100 \varepsilon < -\frac 3 4$. 
 Thus, the function $f:L'\to \R$ decreases  outside of $K$ at least  with velocity  $\frac 3 4$. Hence, the open map theorem \cite[Lemma 4.1]{Lyt-open},  provides a point $p'\in L'$ with $f(p')=0$ (hence $p'\in K$) and
 $$d(p',q')\leq  f(q')  \cdot (\frac  4 3 ) <2t\;,$$ 
finishing the proof of the Lemma.
\end{proof}

Using a combination of Lemma  \ref{lem: 4op} and \ref{lem: karcher} we now  provide:
\begin{proof}[Proof of Theorem \ref{thm: semicont}]
After rescaling we may assume that $B=B_{10} (x_0)$ is a  $P$-almost flat domain around $x_0$. 

We are going to prove (3) first.  Due to \eqref{eq: lip}, it is sufficient to improve the openness constant of $\Pi^L$ provided by Lemma \ref{lem: 4op}.

Set $r_0:=10^{-2}$.
Let $r \leq  r_0$ and  $z\in B_r(x_0)$ be arbitrary. Set $x =\Pi^L (z)$ and let $x_1$ be a point  on $L$ with  $s:=d(x,x_1)<\varepsilon \cdot r$.


 We are going  to find a point $z_1\in L^z$ satisfying $\Pi ^L (z_1) =x_1$ and 
\begin{equation}  \label{eq: rs}
d(z_1,z)\leq (1+ 10\cdot r)\cdot s.
\end{equation}

 Extend the geodesic $xz$ to a point $\tilde z$ with $d(x,\tilde z) = 1$.  Apply  Lemma \ref{lem: 4op}  and find a point $\tilde z_1$ on the fiber $\tilde L$ through $\tilde z$ such that $d(\tilde z_1,\tilde  z)\leq 2s$ and $\Pi^L (\tilde z_1)=x_1$.
The geodesics $\eta:=x\tilde z$ and  $\eta_1:=x_1\tilde z_1$ are horizontal and $P\circ \eta= P\circ \eta _1$.
 
Consider the point $z_1$ on the $\eta _1$ with 
$d(z_1,x_1)=d(z,x)$. Then $P(z_1)=P(z)$, hence $z_1\in L^z$.
From  Lemma \ref{lem: karcher} we deduce \eqref{eq: rs}
 %$$d(z_1,z) \leq s +10 r s=(1+10r)\cdot s\,,$$
finishing the proof of (3).

In order to prove (1) and (2) we  fix   $r\leq 10^{-2}$ and    $z\in B_r (x_0)$. Set again $x=\Pi^L (z)$.
 Due to  \eqref{eq: cr},  we have $|V^x-V^{x_0}|\leq 2\varepsilon r$. 

%Hence,  $|H^x-H^{x_0}|\leq 2\varepsilon r$. It suffices to prove (1) and (2) for unit vectors $v,h$.

Consider an arbitrary unit vector $v\in V^{x_0}$. We find a unit 
vector $\hat v \in V^x$ with $|\hat v -v| \leq 3\varepsilon r$.


 For any sufficiently small $\delta >0$ consider a point $x_1\in L$
with $d(x_1,x)<\delta$, such that the geodesic $xx_1$ starts in 
a direction $\tilde v\in T_{x} M$ with $|\tilde v-\hat v| \leq \varepsilon r$. 

Obtain a point $z_1 \in L^z$  as in the proof of (3).   Then $d(z,z_1) <2\delta$ and the starting direction  $\tilde v'$  of $zz_1$ satisfies $|\tilde v -\tilde v '|\leq 10 r$ by Lemma \ref{lem: karcher}.  Thus,
$|\tilde v '-v| \leq 11r$. 

If $x_1$ converges to $x$ then $z_1$ converges to $z$ and the vectors $\tilde v'  \in T_{z} M$ subconverge to a vector $v'\in V^z$, proving  (1).
 

Finally, (1)  implies (2) by duality of horizontal and vertical cones.
\end{proof}

We now easily deduce:
\begin{proof}[Proof of Proposition \ref{prop: almsubm}]
We  cover  the compact manifold fiber $L$ by finitely many  balls  as provided by Theorem \ref{thm: semicont}. Choosing a tubular neighborhood $B_{r_0} (L)$  of $L$
contained in the union of these balls, we obtain the conclusion directly  from Theorem \ref{thm: semicont}(3).
\end{proof}












 












\section{Exponential map in the base}
%All statements in this chapter do not rely on the smoothness of $M$ and are valid in the more general setting of \cite{KL}, where 
%$M$ is only assumed to have curvature locally bounded from both sides.










\subsection{Exponential map in the base}
The following result is a generalization of Theorem \ref{thm: exp} to local submetries. 


\begin{thm} \label{prop: bil}
Let $M$ be a manifold with locally bounded curvature and 
 $P:M\to Y$ a surjective local submetry.
Let $y_0 \in Y$ be an arbitrary point in the startum $Y^l$.   There exist  $r_0=r_0(y_0),C =C(y_0) >0$ such that, for all $y\in B_{r_0} (y_0) \cap Y^l$ and for all $r<r_0$, the exponential map
$\exp _y :B_r(0) \to B_r(y)$ is $(1+C r^2)$-bilipschitz on the ball $B_r(0)\subset T_yY$.
\end{thm}
 


\begin{proof}
 We find  $x_0 \in 
L =P^{-1} (y_0)$ such that a neighborhood of $x_0$ in $L^{x_0}$ is a $\mathcal C^{1,1}$-submanifold.  Upon rescaling we may assume that $B_{10} (x)$ is a $P$-almost flat domain.   Taking 
 into account Remark
\ref{rem: 1} and rescaling again, we may  assume that the conclusions of Theorem \ref{thm: semicont} are valid
for any $x\in B_{10} (x_0) \cap P^{-1} (L)$, with the same $C=\varepsilon$ and $r_0=10$.

%Upon another rescaling we may assume that the conclusions of Theorem \ref{thm: semicont} are sastisfied with $C=\varepsilon$ and $r_0=10$.  Taking into account Remark
%\ref{rem: 1}, we can assume that the conclusions of Theorem \ref{thm: semicont} are valid
%for any $x\in B_{10} (x_0)$, with the same $C=\varepsilon$ and $r_0=10$.

We fix an arbitrary $y\in B_{1} (y_0) \cap Y^l$ and  some $x\in B_{1} (x_0) $ with $P(x)=y$.
The ball $\bar B_1 (y)\subset Y$ is has curvature $\geq -\varepsilon$, \cite[Proposition 3.1]{KL}. By Toponogov's theorem, 
the exponential map $\exp _y$ is $(1+2\varepsilon r^2)$-Lipschitz  on $B_r(0)\subset T_yY$, for any $r\leq 1$.
It remains to bound the Lipschitz constant of $\exp_ y^{-1}$ on $B_r(y)$.



Due to  the almost flatness, 
the exponential map $\exp _x :B_r(0) \to B_r(x)$ is $(1+2\varepsilon  r^2)$-bilipschitz on the ball $B_r(0)\subset T_xM$, for any $r\leq 1$.   
 We consider the ball $Q=B_r(0) \subset H^{x}$   and its exponential image 
$Z=\exp _x (Q)\subset M$.   For all $h\in Q\subset H^x$ we have, \cite[Proposition 7.3]{KL}:
$$\exp _y\circ D_xP (h) =P\circ \exp _x (h)\,.$$
 Thus, map $P:Z\to Y$ can be written as
$$P=  \exp _y \circ D_xP   \circ \exp _x^{-1}\,.$$ 

The map $\exp _x^{-1}:Z\to Q$ is locally $(1+\varepsilon r^2)$-bilipschitz and 
the map $D_xP:H^x\to T_yY$ is a local submetry.  If we knew that $P$ is locally 
$(1+r^2)$-open, we could deduce 
 that 
$\exp _y$ is locally $(1+2r^2)$-open.  
 Since $\exp _y$ is a homeomorphism and $(1+\varepsilon \cdot r^2)$-Lipschitz,  this would
prove that $\exp _y$ is locally $(1+2r^2)$-bilipschitz.
 

It remains to prove that $P:Z\to Y$ is locally $(1+r^2)$-open.


  Thus, consider any $h\in Q$ and $z=\exp _x(h)\in Z$.
 Set $t_0:= \varepsilon \cdot (r- |h|)$.
Let $z_1 = \exp _x (h_1) \in Z$ with $d(z,z_1) <t_0$ be given.  Set $y_1= P(z_1)$ and let
$y_2 \in Y$ be such that $t:= d(y_1,y_2)<t_0$.  We need to find a point $z_2\in Z\cap P^{-1} (y_2)$, such that  $d(z_1,z_2) \leq (1+r^2) \cdot t$.

Consider  $L'=P^{-1} (y_2)$ and the distance function $f:=d_{L'}$.
Then 
$$f(z_1)=d(L', z_1)=d(y_1,y_2)=t\;.$$

Consider a closest point $p$ of $z_1$ on $L'$. Then  $z_1p$ is horizontal geodesic and its starting (unit) direction $u\in T_{z_1} M$ is contained in $H^{z_1}$.

Applying Theorem \ref{thm: semicont}, we find a vector $u'\in H^x$, with
$|u-u'|\leq \varepsilon \cdot r$.  Consider the Lipschitz curve $\eta:[0,t_0]\to Z$ starting at $z_1$:
$$\eta (s):=\exp _x   (h_1+ s\cdot u')  \subset Z\;.$$ 
 Then (any of) the initial unit directions  $u''$ of the curve $\eta$ at  $z_1$ satisfies  
$$||u''-u|| \leq 2 \varepsilon r\;.$$
Therefore, by the first formula of variation, the distance 
function $f$ decreases at $z_1$ in the direction of $u''$ at least with velocity
$$D_{z_1} f(u'')  \leq -\cos (2\varepsilon r) \leq -1+ \frac 1 {4} (2\varepsilon r)^2 < -1+\varepsilon r^2\;.$$ 

The last argument applies to any point $z_1'  \in  Z\setminus L'$ showing that the distance function $f$, restricted  to $Z$ decreases at any point on $Z\setminus L'$
at least with velocity $1-\varepsilon r^2$.  


By the open map theorem \cite[Lemma 4.1]{Lyt-open}, we find a point $z_2\in Z\setminus L'$ with $d(z_1,z_2) \leq (1+2\varepsilon r^2)\cdot t$.









This finishes the proof of the claim and of the theorem.
\end{proof}



%The proof of Theorem \ref{prop: bil} only depends on the conclusions  and the constants obtained  in Theorem \ref{thm: semicont}.  Since the constants provided by  Theorem
%\ref{thm: semicont} can be chosen locally uniformly bounded on any stratum $Y^l$ of $Y$,
%see Remark \ref{rem: 1}, we directly deduce:

%\begin{cor} \label{cor: bil}
%Let $M$ be of locally bounded curvature and $P:M\to Y$ be a surjective local submetry.
%Assume  that $y_0\in Y$ is contained in the stratum $Y^l$ of $Y$.  Then, there are $r_0,C>0$ such that, for any $y'\in Y^l \cap B_{r_0}  (y_0)$ and any $r<r_0$, the exponential map $\exp_y:B_r(0) \to B_r(y)$ is $(1+Cr^2)$-bilipschitz on the ball 
%$B_r(0)\in T_yY$.
% one find$r_0$ and $C$ can be chosen such that
%\end{cor}



\subsection{Applications}
As a consequence of Theorem \ref{prop: bil} we derive a local generalization of Corollary \ref{cor:  expstr}:

\begin{cor}
Let $P:M\to Y$ be a surjective local submetry. Then any startum   $Y^l$  of $Y$ is a manifold with locally bounded curvature.
\end{cor}

\begin{proof}
Let $y_0\in Y^l$ be arbitrary. 
% We may assume without loss of generality that $Y^l$ is
%the maximal stratum $Y^m$ and $Y=Y^l$. Indeed,
 Consider a point $x_0\in L=P^{-1}(y_0)$
such that a neighborhood of $y_0$ in $L$ is a $\mathcal C^{1,1}$-submanifold. 
Upon rescaling, we may assume that $B_{10} (x_0)$ is a $P$-almost flat domain.

Then $S:=P^{-1} (Y^l) \cap B_{10} (x_0)$ is a  $\mathcal C^{1,1}$-submanifold of $M$ and  $P:S\to P(S)\subset Y^l$ is a $\mathcal C^{1,1}$-Riemannian submersion.  Since $S$ has  bounded curvature, \cite[Proposition 1.7]{KL-both}, we may replace $M$ by $S$ and $Y$ by $P(S) \subset Y^l$,
and  assume without loss of generality that $Y=Y^l$.


The compact ball $B:=B_{1}(y_0)$ has curvature bounded from below by $-\varepsilon$   \cite[Proposition 3.1]{KL}. Applying Theorem \ref{prop: bil} and rescaling, 
we may assume  that, for any ball 
$B_r(y)$ contained in $B$, the exponential map $\exp _y:B_r(0)\to B_r(y)$
is $(1+\varepsilon r^2 )$-bilipschitz on $B_r(0)\subset T_yY$.






%Let $S$ be a small neighborhood of $x_0$ in $P^ {-1} (Y^l)$.  Then $S$ is a $\mathcal C^{1,1}$-submanifold of $M$, \cite[Theorem 11.1]{KL} and $P:S\to Y^l$ remains  a local submetry,  if $S$ is equipped with its intrinsic metric, \cite[Corollary 2.10]{KL}.  
%Since $S$ has  bounded curvature, \cite[Proposition 1.7]{KL-both}, we may replace $M$ by $S$ and $Y$ by $P(S) \subset Y^l$. 




%We find some  $r_0, C>0$ as provided by Corollary \ref{cor: bil}  and such that $\bar B_r(y)$ is strictly convex and compact, for all $y\in B_{r_0}(y_0)$ and any $r<r_0$.  Upon rescaling we may assume $r_0=1$ and $C=\varepsilon$ in Corollary \ref{cor: bil}
%and that $B_{10} (x_0)$ is a $P$-almost flat domain, for some $x_0\in  P^{-1} (y_0)$.
  It suffices to prove that $\bar B_{\frac 1 3} (y_0)$  is $CAT(1)$.   
%This can be deduced from \cite{LW-iso} or, much more directly, as follows.
Consider 3 points $y,p,q $ in this ball. Let $\bar y, \bar p, \bar q$ be three points in the round sphere   $\mathbb S^l$ of dimension $l$ and curvature $1$, such that $d(y,p)=d(\bar y, \bar p)$, $d(y,q)=d(\bar p, \bar q)$ and such that $\angle pyq =\angle \bar p \bar y \bar q$.   We need to prove 
$d(p,q) \geq d(\bar p, \bar q)$.

Identify the tangent spaces $T_yM$ and $T_{\bar y} \mathbb S ^l$ through an isometry 
$I$, which sends the starting directions of $yp$ and $yq$ on the starting directions of $\bar y\bar p$ and $\bar y \bar q$ respectively.  It suffices to prove that  the map
$$f:=\exp _{\bar y} \circ I\circ \exp _y ^{-1} :B_{1}(y)\to B_{1}  (\bar y)$$
is $1$-Lipschitz, since $\bar p =f(p)$ and $\bar q=f(q)$.

Due to Theorem \ref{prop: bil} the map $f$ is bilipschitz.  By construction, $f$ sends   geodesics starting at $y$ to geodesics starting at $\bar y$.  Hence $f$ sends spheres around $y$ onto spheres around 
$\bar y$ of the same radius.  

The  restriction of $\exp _{\bar y}$ to  the  concentric sphere
$\partial B_s (0)$ in $T_{\bar y} \mathbb S^n$    is $(1-\frac 1 2 s^2)$-Lipschitz, if we equipp 
this sphere with its intrinsic metric.  Thus the restriction 
$f:\partial B_s (y) \to \partial B_s (\bar y)$ is $1$-Lipschitz, if both spheres are equipped with their intrinsic metrics. 

The bilipschitz map $f$ is differentiable almost everywhere with linear differential, by Rademacher's theorem.  By above, at any point $z$ at which $f$ is differentiable, the differential $D_z f$ is $1$-Lipschitz.  We claim that this is enough to conclude that $f$ is $1$-Lipschitz.

Indeed, the ball $B_{\frac 1 3}(y_0)$ can be considered as a Euclidean subset $O\subset \R^l$ with a Lipschitz continuous Riemannian metric.  For any vector $v$ in $\R^l$, Fubini's theorem implies that  for \emph{almost every}  segment $\gamma$ in $O$ in direction of $v$, the length
of $\gamma$ in $Y$ is not less than the length of $f\circ \gamma$ in $\mathbb S^n$. On segments parallel 
to $v$ in $O\subset Y$, the length functional is \emph{continuous} with respect to uniform
convergence.  On the other hand,  the length of the images $f\circ \gamma$  is (as always) 
lower semi-continuous.  Thus, by a limiting procedure, the length of $f\circ \gamma$
is not larger than the length of $\gamma$ for \emph{every} segment $\gamma$ in the direction of $v$.  Therefore, the map $f$ is $1$-Lipschitz and $B_{\frac 1 3} (y_0)$  has curvature at most $1$.
\end{proof}


Another consequence of Theorem \ref{prop: bil} is the following:


\begin{cor} \label{lem: locinf}
Let $P:M\to Y$ be a surjective local submetry as above.
 Let $y\in Y$ be an arbitrary point, let $r$ be smaller than the injectivity radius of $y$ and let 
$v\in T_yY$ be a vector with $|v| <r$.  Then the tangent cones  $T_v (T_yY)$ and 
$T_{\exp _y(v)} Y$ are isometric.

In particular, if $\exp _y (v)$ is contained in the $l$-dimensional stratum $Y^l$ then 
$v$ is contained in the $l$-dimensional stratum $(T_yY)^l$.
\end{cor}




\begin{proof}
Consider the geodesic $\gamma _v:[0,r)\to Y$ in the direction of $v$ parametrized by arclength.    For $t\in (0,r)$, the tangent spaces at  $\gamma _v (t)$   do not depend on $t$,   \cite{Petruninpar}.  Moreover,  the tangent space $T_v (T_yY)$ in the Euclidean cone $T_yY$  is isometric  $T_{s\cdot v} (T_yY)$, for all $s>0$.

  

Due to Proposition \ref{prop: bil},  for small $s>0$,  a neighborhood of $(s \cdot v)$ in $T_yY$ is $(1+C s^2)$-bilipschitz to a neighborhood of $\exp _y(s\cdot v)$ in $Y$, for some $C$ independent of $s$.    Rescaling, letting $s$ go to $0$ and using that the tangent cones at $s\cdot v$ respectively at $\exp_y (s\cdot v)$ do not depend on $s$, we deduce the claim.
\end{proof}
















\section{Transnormal submetries}
%\subsection{Existence of vertical vector fields}
%A local submetry $P:M\to Y$ is called  transnormal if all fibers  of $P$ are $\mathcal C^{1,1}$-submanifold of $M$.



% any local  geodesic  $\gamma :I\to M$ is horizontal once  $\gamma'(t)$ is horizontal for some $t\in I$.
%starting normally to a leaf remains normal to all leaves it intersects.  
%This is equivalent to the statement that any fiber  of $P$ is  a $\mathcal C^{1,1}$-%submanifold of $M$ \cite[Proposition 12.5]{KL}.


%We first prove  the following local generalization of \ref{prop: srf}. 
%\begin{prop}
%Let $P:M\to Y$ be a transnormal local submetry. Then for any $x\in M$ and any $v\in V^x$
%there exists a continuous vector field $\nu$ on $M$  such that
%$\nu (x)=x$ and $\nu (z)\in V^z$, for all $z\in M$.
%\end{prop}


%\begin{proof}
%The vertical subspaces vary on $M$ semi-continuously, Theorem \ref{thm: semicont}(1).  We fix a trivialization of the tangent bundle of $M$ in a neighborhood $O$ of $x$. The existence of a continuous vector
%field $\tilde \nu$ on $O$,  with $\tilde \nu (z)\in V^z$, for all $z\in O$ and $\tilde \nu (x)=v$,  becomes a special case of \cite[Theorem 3.2]{Michael}.

%We multple this vector field  $\tilde \nu$ with a smooth function $f:O\to \R$,  supported in a compact subset of $O$ and satisfying $f(x)=1$.  Extending the arising vector field by $0$ on $M\setminus O$, provides the required  field $\nu$.
%\end{proof}


\subsection{Horizontal geodesics}
Recall that a local submetry $P:M\to Y$ is  transnormal if all fibers  of $P$ are $\mathcal C^{1,1}$-submanifold of $M$.
A local submetry $P:M\to Y$ is 
%called 
 transnormal if and only if any local  geodesic  $\gamma :I\to M$ is horizontal once  $\gamma'(t)$ is horizontal for some $t\in I$,
%starting normally to a leaf remains normal to all leaves it intersects.  
%This is equivalent to the statement that any fiber  of $P$ is  a $\mathcal C^{1,1}$-submanifold of $M
 \cite[Proposition 12.5]{KL}.  In this case, 
for any horizontal local geodesic $\gamma :I\to M$, the projection $\bar \gamma := P\circ \gamma :I\to Y$ is 
%a quasi-geodesic in $Y$, which is
a discrete concatenation of geodesics in $Y$, \cite[Corollary 7.2]{KL}.



The following Lemma is  stated as  \cite[Proposition 12.7]{KL}  for global submetries, but the proof remains unchanged in the local case:
\begin{lem} \label{lem: eqproj}
Let $P:M\to Y$ be a transnormal local submetry. Let $\gamma _1, \gamma _2:I\to M$ be  horizontal local geodesics.  Set $\bar \gamma _i:= P\circ \gamma _i$.
Assume that, for some $t\in I$, we have $\bar  \gamma _1(t)=\bar  \gamma _2 (t) $ and $\bar  \gamma _1 '(t)=\bar  \gamma _2 ' (t) $.  
Then $\bar \gamma _1$ and $\bar \gamma _2$ coincide on $I$.
\end{lem}






%Let  $P$ be transnormal. If $\gamma_1, \gamma _2:[0,a]\to M$ be horizontal local geodesics, such that the projections $P\circ \gamma _i$ coinicide initially then $P\circ \gamma _i$ coincide on $[0,a]$.  This is shown in \cite[Proposition 12.7]{KL} in the case of global submetries, but the proof remains unchanged in the local case.

We clarify the structure of projections of  horizontal local geodesics:



\begin{prop} \label{prop: constant}
Let $\gamma :I\to M$ be a horizontal local geodesic for a transnormal local submetry $P:M\to Y$ and let 
$\bar \gamma =P\circ \gamma$.
Then, for all $t\in I$, the iterated tangent cones $T_{\bar \gamma '(t)} (T_{\bar \gamma (t)} )Y)$ are pairwise isometric. 

If $l$ is the dimension of the maximal  Euclidean factor of  $T_{\bar \gamma '(t)} (T_{\bar \gamma (t)}  Y)$ 
then, for  all but discretely many times $t\in I$,
the point $P(\gamma (t))$ is contained in $Y^l$.
\end{prop}


\begin{proof}
The statements are local. Thus, we may assume that $I$ is a compact interval $[a,b]$.
The projection $\bar \gamma :[a,b]\to Y$ is a finite concatenation of geodesics $\bar \gamma :[s_1,s_{i+1}] \to Y$, for $a=s_0<s_1...<s_k=b$.
% Considering largest parts of $\bar \gamma$ which are local geodesics, we obtain 
%the uniquely defined coarsest  subdivision
%$a=s_0< s_1 <...<s_l=b$, such that, for any $i$, the restriction $\bar \gamma|_{[s_i,s_{i+1})]}$ is a local geodesic.

 %Fix some $0\leq i\leq l-1$. Then  $\bar \gamma|_{[s_i,s_{i+1})]}$ is a local geodesic.  
For all $t\in (s_i,s_{i+1})$ the 
tangent spaces $T_{\bar \gamma (t)} Y$ are pairwise isometric  \cite[Theorem 1.1]{Petruninpar}. By  Corollary \ref{lem: locinf}, they are also isometric to  $T_{\bar \gamma ^+(s_i)} (T_{\bar \gamma (s_i)} Y)$ and to $T_{\bar \gamma ^- (s_{i+1})} (T_{\bar \gamma (s_{i+1})} Y)$.
 Here and below $\bar \gamma ^{\pm} (s)$ denotes the outcoming and the incoming direction of $\bar \gamma $  in $T_{\bar \gamma (s)} Y$.

On the other hand, for any $t\in (s_i,s_{i+1})$, the incoming and the outgoing directions
$\bar \gamma ^{\pm (t)}$ are  contained in the line factor of $T_{\bar \gamma (t)} Y$.
Hence,  $T_{\bar \gamma '(t)} (T_{\bar \gamma (t)}  Y)$ is isometric to $T_{\bar \gamma (t)} Y$, for any such $t$.

It only remains to prove that for any $s=s_1,...,s_{k-1}$,  the two tangent cones 
$T_{\bar \gamma ^{\pm}(s)} (T_{\bar \gamma (s)} Y)$ are isometric two each other.  In order to prove this, it suffices to find, for any such $s$ an isometry $I: T_{\bar \gamma (s)} Y\to T_{\bar \gamma (s)} Y$ which sends $\bar \gamma ^{+}$ to $\bar \gamma ^{-}$.








In order to find  such $I$, we consider $x:=\gamma (s)\in M$ and the differential $D_{x} P: T_xM \to T_yY$. 
 The restriction of $D_xP$   to the unit sphere $K$ in the horizontal space $H^x$ is a transnormal submetry $D_xP:K\to \Sigma _y Y$, onto the space of directions at $y$ \cite[Proposition 12.5]{KL}.  

The incoming and the outgoing  directions $\gamma^{\pm} (t)\in K$ satisfy 
$\gamma^+(t)=-\gamma ^-(t)$ and $D_xP (\gamma ^{\pm } (t))= \bar \gamma ^{\pm} (t)$.

Due to  \cite[Proposition 12.7]{KL},
the decomposition of $K$ into the fibers of the submetry $D_xP$ is  equivariant under the multiplication of $K$ with $-1$.
Thus, $-Id: K\to K$ induces an isometry $\overline{- Id}:\Sigma _y Y\to \Sigma _y Y$.
The cone over this isometry $\overline {-Id}$ is the required isometry $I:T_y Y\to T_y Y$, which satisfies  $I(\bar \gamma ^{+})=\bar \gamma ^{-}$.


This proves the claim and implies that  the spaces $T_{\bar \gamma '(t)} (T_{\bar \gamma (t)} ) Y$ are pairwise  isometric.

Let now  $l$ denote the dimension of the maximal Euclidean factor of the pairwise isometric spaces 
 $T_{\bar \gamma '(t)} (T_{\bar \gamma (t)}  Y)$. Then, for all $t\neq s_0,s_1,...,s_k$ as above,  the iterated tangent cone   $T_{\bar \gamma '(t)} (T_{\bar \gamma (t)}  Y)$ is isometric to $T_{\bar \gamma (t)}  Y$. By definition, $\gamma (t)$ is contained in 
$Y^l$, for all such $t$.
\end{proof}





\subsection{Holonomy map along a horizontal geodesic}
Let $P:M\to Y$ be a transnormal local submetry. Let $\gamma :[a,b] \to M$ be a horizontal local geodesic
with projection $\bar \gamma =P\circ \gamma$.
Due to Proposition \ref{prop: constant}, there exist some $1\leq l\leq m$, such that 
$\bar \gamma (t)\in Y^l$, for  all but finitely many times $t\in [a,b]$.
For $t\in [a,b]$, consider the  fiber
$$L^t:= L^{\gamma (t)} =P^{-1} (P(\gamma (t)))\;.$$




Set $S=P^{-1} (Y^l)$.
Let $t\in [a,b]$ be such that   $\bar \gamma (t) \in Y^l$.  Set $x=\gamma (t) \in L^{t} \subset S$. Then
 $S$ is a $\mathcal C^{1,1}$-submanifold of $M$ and the restriction $P:S\to Y^l$ is a $\mathcal C^{1,1}$ Riemannian submersion.  Therefore, the normal vector $\gamma'(t)\in T_x S$ to $L^t$ 
 extends to
a unique  locally Lipschitz continuous normal field $z\to \nu _z \in T_zS$ along $L^{t}$, such that 
$$D_zP (\nu _z)= D_xP (\nu _x)=D_xP (\gamma'(t))\;.$$


Denote by $Q^t$ the set of all   $z\in L^{t}$ such that the geodesic
$\gamma ^{z} :[a,b]  \to M$ with $(\gamma^{z}) ' (t)=\nu _z$ is defined.
Then $Q^t$ is  an open subset of $L^t$ and, if $M$ is complete, $Q^t=L^t$.
Due to Lemma \ref{lem: eqproj}, for all $z\in Q^t$ 
$$P\circ \gamma ^{z} =P\circ \gamma =\bar \gamma\;.$$
%by  Lemma \ref{lem: eqproj}.


For all $s\in [a,b]$ we   obtain a map $Hol^{\gamma}_{t,s} : Q^t\to L^s$, 
 \emph{the holonomy along $\gamma$}, 
 given as
$$Hol ^{\gamma } _{t,s} ( z):= \gamma ^z (s)\;.$$

Since $\nu$ and the exponential map on $M$ are locally Lipschitz, 
the map $Hol^{\gamma} _{t,s}$  is locally Lipschitz.  

 If $\gamma (s)\in Y^l$, then $Hol^{\gamma} _{t,s} (Q^t)=Q^s$ and 
 $Hol^{\gamma} _{t,s} $ and $Hol^{\gamma} _{s,t}$ are inverse to each other.
Thus,  $Hol^{\gamma} _{t,s} $ is locally bilipschitz in this case.

Let now $r\in [a,b]$ be arbitrary. Find some $s\in [a,b]$ such that $\bar \gamma (s) \in Y^l$ and $|s-r|$ is smaller than the injectivity radius at $\bar \gamma (r)$.
 Then 
$$Hol ^{\gamma} _{s,r} \circ Hol^{\gamma} _{t,s} = Hol^{\gamma} _{t,r}\;.$$
The  map $Hol^{\gamma}_{t,s}$ is locally bilipschitz, as we have seen above.
And the map $Hol^{\gamma} _{s,r} :Q^s\to L^r$ is the closest-point projection
to $L^r$.     Once $s$ has been chosen close enough to $r$, we can apply Theorem \ref{thm: semicont} and deduce that the map $Hol^{\gamma}_{s,r}:Q^s\to L^r$ is locally Lipschitz open. 

Alltogether we have verified the following










\begin{prop} \label{prop: hol}
In the notation above, the holonomy map  along $\gamma$
$Hol^{\gamma} _{t,r}: Q^t\to L^r$ is locally Lipschitz continuous and locally Lipschitz open.
%For any point $z\in Q$ there exists a neighbborhood $U$ of $z$ in $Q$ and a homeomorphism  $\Phi: U\to V\times \R^s$, with some $s\geq 0$ and $V= Hol^{\gamma} _t( U)$, such that $$Hol^{\gamma}_t = Proj _V\circ \Phi\,$$
%where $Proj _v$ is the projection of $V\times \R^s$ onto the first factor.
If $\bar \gamma (r) \in Y^l$, then $Hol^{\gamma }_{t,r}$ is locally bilipschitz.
\end{prop}




 Combining Proposition \ref{prop: hol} and Proposition \ref{prop: constant}, we obtain a proof of Corollary \ref{cor: last} from the introduction.













%\section{Smoothness of fibers}












\bibliographystyle{alpha}
\bibliography{regular}













\end{document}

























