
\section{The Unique Games Hardness}
\label{sec:hardness}

In this section, we consider the special case of \wtd when there are
only two weight classes. Assume wlog that the two distinct weights are
1 and $W$, where $W \gg 1.$ Our first main result shows that getting a
good approximation algorithm with $(2-\eps)$-resource augmentation for
any constant $\eps > 0$ is as hard as getting a better-than-two
approximation for the vertex cover problem. 

\Hardness*

\begin{proof}
  We give a reduction from the \vc problem. Let $d = d(\eps)$ be a
  constant to be fixed later, and let $c < 2$ be a constant as in the
  statement of the theorem.  Let $\cI = (G=(V,E), t)$ be an
  instance of the \vc problem on $n$ vertices. We know that it is
  UG-hard to distinguish between the following two cases: (i) $G$ has
  a vertex cover of size at most $t$, or (ii) every vertex cover of $G$
  must have size strictly larger than $ct$.
  % where $\delta > 0$ is a
  % positive constant.

  We map $\cI$ to an instance $\cI'$ of \wtd as follows: the set of
  points in $\cI'$ is given by $V \cup \{v_0\}$, where $v_0$ is a
  special vertex. There are $t$ servers of weight $W = n^{d}$ and one
  server of unit weight. Let the edges in $E$ be $e_1, \ldots, e_m$. A
  subsequence of the request sequence consists of $m$ {\em phases},
  where we have a phase for each edge $e_i$. During phase $i$
  corresponding to edge $e_i=(u_i, v_i)$, the request sequence toggles
  between $u_i$ and $v_i$ for $W$ times. Finally, the subsequence is
  repeated $W$ times.  In other words, it is the following sequence
  $$ \big( \underbrace{u_1, v_1, u_1, v_1, \ldots, u_1, v_1}_{W \
    \mbox{times}}, \ldots,\underbrace{u_m, v_m, u_m, v_m, \ldots, u_m,
    v_m}_{W \ \mbox{times}} \big)^W.  $$ We also have to specify the
  initial location of the servers. Assume that all servers are at
  location $v_0$ in the beginning.  This completes the description of
  the instance $\cI'$. Observe that $N$, the number of requests in
  instance $\cI'$ is $O(m \cdot n^{2d})$.


  \begin{claim}
    \label{cl:case1}
    Suppose $G$ has a vertex cover of size at most $t$. Then the cost of
    the optimal solution for $\cI'$ is at most $2mW$.
  \end{claim}
  \begin{proof}
    Let $V' \subseteq V$ be a vertex cover of size $t$. Consider the
    following solution: we move the $t$ heavy servers from $v_0$ to
    $V'$ at the beginning. From now on, the heavy servers will not
    move at all. During a phase corresponding to an edge
    $e_i= (u_i, v_i)$, we know that at least one of these vertices
    will be occupied by a heavy server. If the other end-point, say
    $v_i$, is not occupied by a heavy server, we move the server of
    weight 1 to $v_i$. Now we have two servers occupying $u_i$ and
    $v_i$ respectively until the end of this phase. The total movement
    cost is incurred either at the beginning (which is $tW$ overall),
    or at the beginning of each phase (when the cost is 1). Since
    there are $mW$ phases, the overall cost is at most
    $tW + mW \leq 2m W$.
  \end{proof}

  \begin{claim}
    \label{cl:case2}
    Suppose every vertex cover in $G$ has size strictly larger than
    $ct$. Then cost of the optimal solution for $\cI'$, even with
    $c$-resource augmentation, is at least $W^2$.
  \end{claim}
  \begin{proof}
    Consider any solution for $\cI'$. Recall that the input consists
    of $W$ subsequences, call these $S_1, \ldots, S_{W}$, where each
    subsequence $S_j$ consists of $m$ phases, one for each edge of
    $G$. We claim that during each such subsequence $S_j$, the
    solution must pay movement cost of at least $W$. Indeed, consider
    a subsequence $S_j$. If the solution moves a heavy server during
    $S_j$, then the claim follows directly. Else observe that the size
    of any vertex cover is strictly larger than the number of heavy
    servers $ct$, so there is some edge $e_i=(u_i, v_i)$ not covered
    by the heavy servers during $S_j$. Now the phase for $e_i$ in
    $S_j$ would require the unit weight server to toggle between $u_i$
    and $v_i$ for $W$ times. In either case, the cost of each
    subsequence is at least $W$, and the overall cost of the solution
    is at least $W^2$.
  \end{proof}

  The above two results along with the UG-hardness result for \vc
  impliy that it is UG-hard to obtain a
  $\frac{W^2}{2mW}$-approximation for \wtd with two weight
  classes. This ratio is equal to
  $\frac{W}{2m} \geq n^{d-2} \geq N^{\nf{1}{2}-\eps}$, assuming $d$ is
  $\Omega(\nf{1}{\eps})$, which proves \Cref{thm:hard2}.
\end{proof}

%%% Local Variables:
%%% mode: latex
%%% TeX-master: "main"
%%% End:
