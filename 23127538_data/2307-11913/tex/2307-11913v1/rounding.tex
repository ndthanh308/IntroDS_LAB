\newcommand{\xI}{x_i^j(I)}
\newcommand{\yI}{y_{v,j,I}}
\newcommand{\yIs}{{\widetilde y}_{v,j,I}}
\newcommand{\yId}{{\thickbar y}_{v,j,I}}
\newcommand{\yIp}[1]{y_{#1,j,I}}
\newcommand{\xt}[1][v]{x_{#1,j,t}}
\newcommand{\xIp}[1]{x_{#1,j,I}}
\newcommand{\xIs}{\widetilde{x}_i^j(I)}
%\newcommand{\xt}{x_i^j(t)}
\newcommand{\xtminus}{x_{v,j,t^-}}
\newcommand{\xts}[1][v]{\widetilde{x}_{#1,j,t}}
\newcommand{\xtsp}[1]{{\widetilde{x}_{v,j,{#1}}}}
\newcommand{\xtsv}[1]{{\widetilde{x}_{#1,j,t}}}
\newcommand{\xtdv}[1]{{\thickbar{x}_{#1,j,t}}}
\newcommand{\xtd}{\thickbar{x}_{v,j,t}}
\newcommand{\xtminuss}{\widetilde{x}_{v,j,t^-}}
\newcommand{\xtdminus}{\thickbar{x}_{v,j,t^-}}
%\newcommand{\yI}{y_i^j(I)}
\newcommand{\tyI}{\widetilde{y}_{v,j,I}}
\newcommand{\yt}{y_i^j(t)}
\newcommand{\ytminus}{y_i^j(t^-)}
\newcommand{\yts}{\widetilde{y}_i^j(t)}
\newcommand{\ytminuss}{\widetilde{y}_{v,j,t^{-}}}

\newcommand{\RS}{{\mathsf {ScaleRound}}}
\newcommand{\halfeps}{\nicefrac{\eps}{2}}
\newcommand{\lastevent}{{\mathsf {LastEvent}}}
\newcommand{\downevent}{{\mathsf {DOWN}}\xspace}
\newcommand{\upevent}{{\mathsf {UP}}\xspace}
\newcommand{\lasttime}{{\mathsf {LastTime}}}
\newcommand{\tr}{{\mathsf{trunc}}}


\section{An Offline Algorithm via LP Rounding}
\label{sec:offline}

We now show an algorithm for the offline setting, that rounds any
fractional solution to the 
LP relaxation~\eqref{LP:tag}, and achieves the following guarantee:
%
\Offline*

Instead of working with the relaxation~\eqref{LP:tag}, we work with an
equivalent relaxation which turns out to be easier to interpret. For
each vertex $v \in V$, index $j \in [\ell]$ and time interval $I$, we
have a variable $y_{v,j,I}$, which denotes the fractional mass of
server of weight $W_j$ residing at $v$ during the entire time interval
$I$. The variable $x_{v,j,t}$ used in~\eqref{LP:tag} can be expressed as follows: 
\begin{align}
    \label{eq:xyrelation}
    x_{v,j,t} = \sum_{I: t \in I} y_{v,j,I}.
\end{align}
Let $\Int$ denote the set of all intervals during the request timeline. 
The new linear program relaxation for \wtd is the following:
\begin{alignat}{2}
  \label{lp:original}
   \tag{LP2}
  \min \nf12 \sum_{j \in [\ell] } W_j \; \sum_{I \in \Int} \sum_{v \in V}  &\yI  && \\
  \text{s.t.  } \sum_{j \in [\ell]} \sum_{I: t \in I}  \yIp{\sigma_t} &\ge 1 & \qquad \qquad
  &\forall t \label{eq:covLP} \\
  \sum_{v \in V} \sum_{I: t \in I}  \yI &\le k_j &&\forall  t, j \in [\ell] \label{eq:packLP}\\
  \notag \yI & \geq 0 \qquad \qquad &&  \forall t, j \in [\ell], v \in V. 
\end{alignat}
Note that the covering constraint~(\ref{eq:covLP}) enforces having at
least one unit of (fractional) server mass at the location $\sigma_t$
requested for each time $t$. The packing constraint~(\ref{eq:packLP})
enforces that the total (fractional) server mass of weight $W_j$ used
at any time $t$ is at most the number of servers of this weight,
namely $k_j$. Given a solution $\yI$ to~\ref{lp:original}, the
variables $\xt$ defined using~\eqref{eq:xyrelation} define a feasible
solution to~\ref{LP:tag} of the same cost.

Fix any constant $\eps\in (0, 1)$. We now prove~\Cref{thm:main} by rounding an optimal fractional solution $\yI$ to~\ref{lp:original}.  
% to obtain an integral solution that uses at most $(2+\eps)\ell k_j$ servers of weight $W_j$ and has cost of $O(1/\eps)$ times that of $\xI$. 
The rounding algorithm has two stages. The first stage scales and
discretizes the LP variables to integers such that
\begin{enumerate}[nosep]
\item the packing constraints are satisfied up to a factor of
  $(2+\eps)\ell$,
\item the covering constraints are satisfied with a
  scaled covering requirement of $\ell$ instead of 1, i.e.,
  $\sum_j \sum_{I: t\in I} \yIp{\sigma_t} \ge \ell,$ for all times $t$,
  and 
\item the cost of the fractional solution increases by a factor of
  $O(\ell/\eps)$.
\end{enumerate}
In the second stage, we remove the packing constraints from the LP;
this results in the resulting interval covering LP being
integral. Next, we scale the solution from the first stage down by
$\ell$, getting a feasible fractional solution to the standard LP
relaxation for the interval covering problem.
Finally, we use the integrality of the interval covering LP relaxation
to obtain an integral solution for~\ref{lp:original}. We present these
two stages in the next two sections.

\subsection{Stage I: Scaling and Discretization}

\begin{algorithm}[t]
  \caption{Procedure $\RS(x,y,v,W_j)$. }
  \label{algo:rs}
  {\bf Input:} A fractional solution $(\yI, \xt)$ to~\ref{lp:original}, a  location $v$ and a weight $W_j$\; 
  Initialize variables $\yId$ to 0 for all intervals $I$. \;
  {\bf (Scale):}   Define $\yIs = \left(2+\halfeps\right)\ell \cdot \yI$ and therefore,  \quad \quad $\xts = \sum_{I:t\in I} \yIs = \left(2+\halfeps\right)\ell \cdot \xt$ for each $I \in \Int$. \;
  {\bf (Round):} \For{$h =1, 2, \ldots, \ell$ \label{l:for}}{
   Initialize $\lastevent = \downevent, \lasttime = 0$. \;
   \Repeat{ we have reached the end of the timeline $[0,T]$}{
     \If{ $\lastevent = \upevent$}{
           Let $t$ be the first $\downevent$ after $\lastevent$ \;
           Update $\lastevent = \downevent, \lasttime = t. $
     }
     \Else{ ($\lastevent = \downevent$) Let $t$ be the first $\upevent$ after $\lastevent$ \;
        Add $I=[\lasttime,t)$ to $\Int_{v,j}(h)$ and increase $\yId$ by 1. \;
        Update $\lastevent = \downevent, \lasttime = t. $
     }
     }
      }
\end{algorithm}

The first stage of the rounding algorithm operates independently on
each location $v \in V$ and for each server weight $W_j$; the formal
algorithm $\RS(x,y,v,W_j)$ is given in~\Cref{algo:rs}. We work with
both the $y_{v,j,I}$ variables and the equivalent $\xt$ variables
defined in~\eqref{eq:xyrelation}; this representational flexibility
makes it convenient to explain the algorithm. To begin, we scale the
LP variables $\yI$ by a factor $(2+\halfeps)\ell$ to obtain $\yIs$ (we also
define the auxiliary variables $\xts$ by scaling $\xt$ similarly).


\emph{Discretization.} Next we discretize the scaled variables $\yIs$ and
$\xts$ to nonnegative integers $\yId$ and $\xtd$ respectively. To
start, let us describe the discretization of $\xts$ to obtain
$\xtd$. Intuitively, we would like to define $\xtd$ as
$\lfloor\xts\rfloor$, i.e., the largest step function with unit step
sizes entirely contained in $\xts$, but this can amplify small
fluctuations around integer values, and hence may increases the cost.
To avoid this, we introduce {\em hysteresis} in our discretization, by
setting different thresholds for increasing and decreasing the value
of $\xts$. We view $\xts$ as a time-varying profile and define horizontal {\em slabs} in it corresponding to the restriction of the range of $\xts$ to $[h,h+1)$ for some integer $h$. For each such slab, we identify intervals $I$ of width at most 1 and at least $\nf12$ and set  the increase the corresponding $\yId$ value by 1.
In more detail, for each such level $h$, we identify a subset
$\Int_{v,j}(h)$ of intervals for which the corresponding $\yId$
variable is to be increased by 1. 
We identify an alternating sequence of {\em up} and {\em down} events
in the timeline $[0,T]$ as follows:
\begin{itemize}
\item $\upevent$ event: At time $t$, there is an $\upevent$ event at
  level $h$ if $\xtminuss < h$ and $\xts \ge h$, and the previous
  event at level $h$ was a $\downevent$ event.
\item $\downevent$ event: At time $t$, there is a $\downevent$ event
  at level $h$ if the previous event at level $h$ was an $\upevent$,
  and $\xtminuss > h - \halfeps$ and $\xts \le h - \halfeps$, or
  $t=T$, the end of the timeline. (The reader should think of
  $\halfeps$ as the ``hysteresis gap'' between the up and down events
  at any level.)
\end{itemize}    
To make the definition complete, we set $\xts$ to 0 at $t = 0^-$ and
at $t=T^+$, and start with a $\downevent$ at time 0.  Finally, we add intervals stretching from each $\upevent$
to the next $\downevent$ to the set $\Int_{v,j}(h)$ of intervals. By
construction, these intervals are mutually disjoint. Finally, whenever
an interval $I$ is added to such a set $\Int_{v,j}(h)$, we increment
the corresponding variable $\yId$. Thus we have:
\[
    \yId = |\{h: I \in \Int_{v,j}(h) \}|, \text{ and correspondingly, } \xtd = \sum_{I: t\in I} \yId. 
\]

The next lemma shows that $\xtd$ can be thought of as a discretized form of $\xts$:
\begin{lemma}\label{lem:range}
The following holds for variables $\xtd$:
\begin{equation}\label{eq:range}
    \xts - 1 < \xtd < \xts + \halfeps.
\end{equation}
\end{lemma}
\begin{proof}
  Suppose $\xts \in [r,r+1)$. Consider the {\bf for} loop in
  line~\ref{l:for} in~\Cref{algo:rs} for a value $h \leq r$. We claim
  that at time $t$, the value of the variable $\lastevent$ must be
  $\upevent$. Suppose not. Let $t'$ be the value of $\lasttime$ at
  time $t$ (i.e., $t'$ is the last time before and including $t$ when
  an $\upevent$ or a $\downevent$ occurred). Since a $\downevent$
  event happened at time $t'$, $\xtsp{t'} < h$. Since $\xts \geq h$,
  an $\upevent$ event must occur during $(t',t]$, a
  contradiction. Therefore must have added an interval containing time
  $t$ to $\Int_{v,j}(h)$. Thus, $\xtd$ gets increased during each such
  iteration, i.e., $\xtd \geq r > \xts-1$. This proves the first
  inequality in~\eqref{eq:range}.

  We now prove the second inequality.
  Let $h$ be an integer satisfying $h \geq \xts + \halfeps.$ Consider
  the iteration of the {\bf for loop} in~\Cref{algo:rs} for this
  particular value of $h$. We claim that the value of the variable
  $\lastevent$ at time $t$ must be $\downevent$. Suppose not, and let
  $t'$ denote the value of the variable $\lasttime$. Then an
  $\upevent$ happened at time $t'$ and so $\xtsp{t'} \geq h$.  Since
  $\xtsp{t} \leq h-\halfeps$, a $\downevent$ event must have happened
  during $(t',t]$, a contradiction. Hence, we do not add any interval
  containing time $t$ to the set $\Int_{v,j}(h)$. Therefore,
  $\xtd < \xts + \halfeps$, which proves the second inequality
  in~\eqref{eq:range}.
\end{proof}

The next lemma establishes the key properties of the variables $\yId$ and $\xtd$.
\begin{lemma}\label{lem:discrete}
  The following properties hold the for the variables $\yId$:
  \begin{enumerate}[nosep,label=(\roman*)]
  \item (Cost) The LP cost increases by at most $O(\ell/\eps)$ when the original variables $\yI$ are replaced by the new variables $\yId$:
    \[
      \sum_{v, j, I} W_j \cdot \yId \le O(\ell/\eps) \cdot \sum_{v, j, I} W_j \cdot \yI.
    \]
  \item (Covering) The variables $\yId$ satisfy the scaled covering constraints of~\eqref{lp:original}
    \[
      \sum_{j, I: t \in I}   \yId \ge \ell \quad \forall t.
    \]           
  \item (Packing) The variables $\yId$ approximately satisfy the packing constraints of~\eqref{lp:original}:
    \[
      \sum_{v, I: t \in I} \yId \le (2+\eps)\ell k_j \quad \forall j \in [\ell], t.
    \]
  \end{enumerate}
\end{lemma}
\begin{proof}
  We first prove the cost bound: the cost of the solution $\yId$ is
  the weight of all intervals added to the sets $\Int_{v,j}(h)$
  for all $v,j,h$. I.e., 
  \begin{align}
    \label{eq:costy}
    \sum_{v, j, I} W_j \cdot \yId = \sum_{v,j} W_j \cdot \sum_{h \in [\ell]} |\Int_{v,j}(h)|. 
  \end{align}
  Fix a vertex $v$ and indices $j,h$.  For a non-negative number $x$,
  and non-negative integer $h$, define the \emph{$h$-level truncation}
  of $x$ to be $\tr_h(x):= \min(1, (x-h)^+)$, where $(a)^+ := \max(a,0)$ for any real $a$. Observe that
  $x = \sum_{h \geq 0} \tr_h(x)$.  In fact, for any two non-negative
  integers $x$ and $y$:
  \begin{align}
      \label{eq:trunc}
      |x-y| = \sum_{h' \geq 0} |\tr_{h'}(x)-\tr_{h'}(y)|. 
  \end{align}
  Now let $I_1=[s_1, t_1), \ldots, I_u=[s_u, t_u)$ be the intervals
  added to $\Int_{v,j}(h)$ (in left to right order). Define $t_0 =
  0$. We know that for any $i \in [u]$, an $\upevent$ happens at $s_u$
  and a $\downevent$ happens at $t_u$. Therefore,
  $\tr_h(\xtsp{s_u})- \tr_h(\xtsp{t_{u-1}}) \geq \halfeps$. Hence,
  \begin{align*}
    \varepsilon W_j/2 \cdot  |\Int_{v,j}(h)|
    & \leq W_J \cdot \sum_{i=1}^u |\tr_h(\xtsp{s_u})- \tr_h(\xtsp{t_{u-1}})| \\
    & \leq W_j \cdot  \sum_{t'=1}^T  |\tr_h(\xtsp{t-1})-\tr_h(\xts)|,
  \end{align*}
  where the last inequality follows from triangle inequality. Summing
  over all $h$ and using~\eqref{eq:trunc}, we get
  $$ \varepsilon   W_j/2 \cdot \yId \leq W_j \cdot  \sum_{t'=1}^T  |\xtsp{t-1})-\xts|. $$
  Summing over all vertices $v$ and indices $j \in [\ell]$, we see
  that the cost of the solution $\yId$ is at most $2/\varepsilon$
  times that of $\yIs$. Finally, the fact that $\yIs$ are obtained by
  scaling $\yI$ by a factor $(2+\halfeps) \ell$, we get the desired
  bound on the cost of $\yId$ solution.
  
  Next, we prove the covering property. Since $\xt$ is a feasible
  solution to~\ref{lp:original}, we have for any time $t$:
  \[
    \sum_j x_{\sigma_t, j, t} \ge 1, \text{ and therefore, } \sum_j \xtsv{\sigma_t}  \ge (2+\halfeps)\ell.
  \]
  Using \Cref{lem:range}, we have $\xtsv{\sigma_t} <  \xtdv{\sigma_t}
  + 1$, so
  \[
    \sum_{j \in \ell} \left(\xtdv{\sigma_t}+ 1\right) > (2+\halfeps)\ell, \text{ and therefore, }
    \sum_j \xtdv{\sigma_t} > \ell.
  \]
  
  Finally, we prove the packing property. Since $\xt$ is a feasible
  solution to the LP, we have for any $j \in [\ell]$ and time $t$,
  \[
    \sum_v \xt  \le k_j, \text{ and therefore, } \sum_v \xts  \le (2+\halfeps)\ell k_j.
  \]
  Again \Cref{lem:range} gives $\xts >  \xtd - \halfeps$, which implies
  \begin{equation}\label{eq:pack}
    \sum_j \left(\xtd - \halfeps\right)^+ < (2+\halfeps)\ell k_j.
  \end{equation}
  Since $\xtd$ is a
  nonnegative integer,
  \[
    \xtd > 0 \implies \xtd \ge 1 \stackrel{\small{\mbox{\Cref{lem:range}}}}{\implies} \xts > \xtd - \halfeps \ge 1 - \halfeps.
  \]
  Since $\sum_v \xts \le k_j$, it follows that the number of locations
  $v$ for which $\xtd > 0$ is at most $\frac{k_j}{1-\halfeps} < 2k_j$,
  if $\eps < 1$.  Using this fact in \Cref{eq:pack}, we get
\begin{align*}
\sum_v \xtd 
&= \sum_{v: \xtd > 0} \xtd 
= \sum_{v: \xtd > 0} \left(\xtd - \halfeps\right) + \sum_{v:\xtd > 0} \halfeps \\
&\le \sum_v \left(\xtd - \halfeps\right)^+ + 2k_j\cdot \halfeps 
\le (2+\halfeps) \ell k_j + \eps k_j.
\end{align*}
Since $\ell \ge 2$ (otherwise, we have the unweighted problem), we get
\[
\sum_v \xtd 
\le (2+\eps)\ell k_j. \qedhere
\]
\end{proof}

\subsection{Stage II: Weighted Interval Cover}

In the second stage of the rounding algorithm, we first scale the
(integer-valued) variables $\yId$ down by a factor of $\ell$ to obtain
new variables $\yI^*$:
\begin{gather}
  \yI^* := \yId/\ell \text{ and therefore, } \xt^* = \sum_{I:t\in I}
  \yI^* = \xtd/\ell. \label{eq:ystar}
\end{gather}
Our goal is to round the fractional variables $\yI^*$ to $\{0,1\}$
values.  In fact, our rounding will ensure that if the rounded value
equals $1$ then
$\yI^* > 0$. 
Since $\yId$ is
integral, the packing property in~\Cref{lem:discrete} implies that for
any time $t$, vertex $v$, and index $j \in [\ell]$, there are at most
$(2+\eps)\ell k_j$ intervals $I \ni t$ for which $\yId > 0$. The
rounding property of our algorithm will ensure that the final integral
solution, which lies in the support of $\yI^*$, will also satisfy that there are at most $(2+\varepsilon) \ell k_j$ intervals containing any time $t$. Since we are allowed a resource augmentation of
$(2+\varepsilon) \ell$ factor in the number of servers of weight
$W_j$, we can serve the requests with the set of available
servers. Henceforth, we can ignore the packing
constraint~\eqref{eq:packLP} for our rounded solution. As a result,
the relaxation~\ref{lp:original} decouples into $n$ independent
relaxations, one for each location $v \in V$.

In this decoupled instance, we get the following LP relaxation for
each location $v$, where for each class $j \in [\ell]$, we define
$\Int_{v,j}:=  \{I \mid \yI^* > 0\}$ as the set of intervals $I$ with a
nonzero value of $\yI^*$ and $\cR(v)$ as the set of times $t$ when $v$ is requested: 
\begin{alignat}{2}
  \label{lp:covering} \tag{LP$_v$}
  \min \nicefrac12 \sum_{j \in [\ell]}  W_j \cdot \sum_{I\in \Int_{v,j}}  &\yI \\
  \text{ s.t. } \sum_j \sum_{I\in \Int_{v,j}: t\in I} \yI &\ge 1
  &\quad \quad& \forall t \in \cR_v \notag\\
  \yI &\geq 0. \notag
\end{alignat}
By the covering property of \Cref{lem:discrete}, the variables $\yI^*$
defined in~(\ref{eq:ystar}) are feasible solutions for
\eqref{lp:covering} for all locations $v$. Furthermore, by the lemma's
cost property (and the scaling down by $\ell$), the total cost
$\sum_v \sum_j W_j \cdot \sum_I \yI^*$ is at most $O(1/\eps)$ times
the optimal cost of \eqref{lp:original}.

Finally, the constraint matrix for \eqref{lp:covering} satisfies the
consecutive-ones property: if the constraints are ordered
chronologically, then a variable $\yI$ appears in the constraints
corresponding to times $t\in I$ where $\sigma_t = v$, which is a
contiguous subsequence of all times $t$ where $\sigma_v =
j$. Constraint matrices with this property are totally unimodular
(see, e.g., \cite{FulkersonG65}). Therefore, each of the solutions $\{\yI^*: j \in [\ell], I \in \Int_{v,j} \}$ for \ref{lp:covering} can be rounded to a feasible integral solution without any increase in cost, 
which proves~\Cref{thm:main}.

%%% Local Variables:
%%% mode: latex
%%% TeX-master: "main"
%%% End:
