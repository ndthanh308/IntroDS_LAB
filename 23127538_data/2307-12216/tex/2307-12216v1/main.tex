\newcommand{\red}[1]{\textcolor{red}{ #1}}
\newcommand{\blue}[1]{\textcolor{blue}{ #1}}
\newcommand{\green}[1]{\textcolor{black}{ #1}}
\renewcommand{\baselinestretch}{1}
\newcommand{\yanyue}[1]{\textcolor{blue}{#1}}

\documentclass[9pt, conference]{IEEEtran}
%\documentclass[]{IEEEtran}
\usepackage{balance}
\usepackage{float} 
\usepackage{tabularx}
\usepackage{mathtools}
\usepackage{amsmath}
\usepackage{hhline}
\usepackage{hyperref}
\usepackage{silence}
\usepackage{graphicx}
\usepackage{multirow} 
\usepackage{graphicx}
\usepackage{multirow}
\usepackage{amsmath}
\usepackage[english]{babel}
\usepackage{booktabs}
\usepackage{array}
\usepackage{paralist}
\usepackage{threeparttable}
\usepackage{lipsum}
\usepackage{flushend}
\usepackage{cuted}
\usepackage{algpseudocode}
\usepackage{algorithm}
\usepackage{algcompatible}
\usepackage{amssymb}
\usepackage{color}
\usepackage{psfrag}
\usepackage{epsfig}
\usepackage{multicol}
\usepackage{color}
\usepackage{fancybox}
\usepackage{subfigure}
\usepackage{cite}
\usepackage{theorem}
\usepackage{url}
\usepackage[table]{xcolor}
\usepackage{epstopdf}
\usepackage[normalem]{ulem}
\usepackage{siunitx}
\usepackage{rotating}

\DeclareUnicodeCharacter{2009}{\,}

%\usepackage{biblatex} 
%\usepackage[backend=biber]{biblatex}
%\usepackage[style=apa,sortcites=true,sorting=nyt,backend=biber]{biblatex}
%\usepackage{subcaption}
\usepackage[justification=centering,font=small,labelfont=bf]{caption}
\usepackage{silence}
\WarningFilter{caption}{Unsupported document class}
\usepackage{enumitem}
%\usepackage{widetext}
%\usepackage{paralist}
%\usepackage[none]{hyphenat}
\hyphenation{intersection op-tical net-works semi-conductor Electro-migration immortality}
\definecolor{gray1}{gray}{0.90}
\definecolor{gray2}{gray}{0.98}
\definecolor{light-gray}{gray}{0.95}
\def\bibfont{\scriptsize}
%Adds extra padding to the table cells
\setlength\extrarowheight{2pt}

\newcommand{\ignore}[1]{}
\newcommand{\redHL}[1]{\textcolor{red}{#1}}
\newcommand{\blueHL}[1]{{\textcolor{blue}{#1}}}
\newcommand{\greenHL}[1]{\textcolor{green}{#1}}
\newcommand{\blackHL}[1]{\textcolor{black}{#1}}
\newcommand{\Alter}[2]{\sout{#1}\redHL{#2}}
\newcommand{\txtoverline}[1]{$\overline{\mbox{{#1}}}$}
\newcommand{\TMR}{\mbox{TMR}}
\newcommand{\sTMR}{\scriptsize{\mbox{TMR}}}
%\newcommand{\Deff}{D_{\mbox{\tiny \em eff}}}
%\newcommand{\Deffn}{D_{\mbox{\tiny \em eff,n}}}
%\newcommand{\Deffg}{D_{\mbox{\tiny \em eff,g}}}
%\newcommand{\Zeff}{Z^\star_{\mbox{\tiny \em eff}}}
%\newcommand{\Javg}{J_{\mbox{\tiny \em avg}}}
%\newcommand{\Jeff}{J_{\mbox{\tiny \em eff}}}
%\newcommand{\Jrms}{J_{\mbox{\tiny \em rms}}}
%\newcommand{\Jflux}{J_{\mbox{\tiny \em flux}}}
%\newcommand{\vivekeq}[1]{\mbox{\tiny {\em #1}}}
%\newcommand{\Tlife}{T_{\mbox{\tiny \em life}}}
%\newcommand{\Ieff}{I_{\mbox{\tiny \em eff}}}
%\newcommand{\Irise}{I_{\mbox{\tiny \em rise}}}
%\newcommand{\Jpavg}{J_{\vivekeq{avg}}^+}
%\newcommand{\Jnavg}{J_{\vivekeq{avg}}^-}
%\newcommand{\Jbavg}{J_{\vivekeq{avg}}^{\vivekeq{blue}}}
%\newcommand{\Jgavg}{J_{\vivekeq{avg}}^{\vivekeq{green}}}
%\newcommand{\nuc}{t_{\mbox{\scriptsize \it nuc}}}
%\newcommand{\nucs}{t_{\mbox{\scriptsize \it nuc--SI}}}
%\newcommand{\nucf}{t_{\mbox{\scriptsize \it nuc--F}}}
%\newcommand{\tlife}{t_{\mbox{\scriptsize \it life}}}
%\newcommand{\Mx}{\mbox{M}_{\mbox{\scriptsize x}}}
%\newcommand{\Mxp}{\mbox{M}_{\mbox{\scriptsize x+1}}}
\pagestyle{plain}
\renewcommand{\bibfont}{\tiny}
\usepackage{tikz}
\usetikzlibrary{tikzmark}
%\usetikzlibrary{shapes,arrows}
\renewcommand{\baselinestretch}{0.93}
%\selectcolormodel{gray}



\begin{document}

\title{A Life-Cycle Energy and Inventory Analysis of Adiabatic Quantum-Flux-Parametron Circuits}
\author{
Masoud Zabihi, Yanyue Xie, Zhengang Li, Peiyan Dong, Geng Yuan, Olivia Chen, Massoud Pedram, Yanzhi Wang\\
%Northeastern University, 360 Huntington Ave, Boston, MA 02115
}

\maketitle
\begin{abstract}
The production process of superconductive integrated circuits is complex and consumes significant amounts of resources and energy. Therefore, it is crucial to evaluate the environmental impact of this emerging technology. An attractive option for the next generation of superconductive technology is Adiabatic Quantum-Flux-Parametron (AQFP) devices. This study is the first to present a comprehensive process-based life-cycle assessment (LCA) and inventory analysis of AQFP integrated circuits. To generate relevant outcomes, we conduct a comparative LCA that included the bulk CMOS technology. The inventory analysis considered the manufacturing, assembly, and use phases of the circuits. To ensure a fair assessment, we choose the 32-bit AQFP RISC-V single-core processor as the reference functional unit and compare its performance with that of a CMOS counterpart. Our findings reveal that the AQFP processor consumes several orders of magnitude less energy during the use phase than its CMOS counterpart. Consequently, the total life cycle energy (which encompasses manufacturing and assembly energies) of AQFP integrated circuits improves at least by two orders of magnitude.




\makeatletter
\renewcommand*{\@makefnmark}{}
\makeatother
\end{abstract}
%\makeatletter{\renewcommand*{\@makefnmark}{}
%\footnotetext{* (equal contribution)}\makeatother}


% Figure environment removed

\section{Introduction}
Automatic 3D reconstruction of clothed humans using image inputs has gained increasing significance due to its potential applications in a wide array of AR/VR scenarios. High-fidelity reconstructions typically depend on sophisticated capture systems, which are developed with dense camera arrays~\cite{collet2015high,joo2015panoptic,joo2018total}, programmable light-stages~\cite{Vlasic2009, guo2019relightables}, and depth sensors~\cite{newcombe2011kinectfusion,DoubleFusion,BodyFusion,dou2016fusion4d,newcombe2015dynamicfusion}. However, stringent capture environments equipped with complex hardware pose significant challenges for consumer-level applications.


In this context, considerable research effort has been dedicated to developing methods that allow for more flexible capture configurations, such as utilizing a few RGB inputs. Among these works, learning implicit functions \cite{iccv2020PIFu, saito2020pifuhd, hong2021stereopifu} has proven effective in achieving highly detailed reconstructions by integrating the advancements of deep neural networks. These methods employ large multi-layer perceptrons (MLPs) to predict the occupancy probability or truncated signed distance function (TSDF) value of every queried 3D point based on its associated local feature, which is extracted from images. They can recover a continuous surface at arbitrary resolutions without topology restrictions.


However, in typical MLP-based implicit networks, the occupancy or TSDF value at each location is solved independently with planar image features, rendering them less capable of addressing challenging cases such as occlusions. Consequently, these methods suffer from generalization and robustness issues, particularly when tackling strong occlusions caused by large motion or multiple interacting humans. 
Some follow-up studies  \cite{zheng2021deepmulticap,zheng2021pamir,huang2020arch} utilize an extra geometric model, SMPL~\cite{Loper2015}, to improve robustness by introducing strong shape priors. 
Their success typically relies on the assumption of geometrical similarity \cite{huang2020arch} between the shape prior and target reconstruction, making them intractable for handling complex cases with loose clothes and sensitive to errors in SMPL model fitting.



%\ping{this paragraph sounds like `TSDF is better than MLP/SMPL, and we use TSDF to solve the problem'. But in Sec 3, we are telling a different story, saying `MLP needs a 3D convolutional encoder'. We need to make these two sections consistent.}\sicong{I think in this paragraph we claim that the TSDF}


%We opt for Trucated Signed Distance Funtion (TSDF) volumetric representations as they are naturally suitable for convolution operations, which have shown remarkable performance for learning hierarchical features on 2D visual perception tasks \cite{SunXLW19}. 
%Meanwhile, TSDF also describes the gradual geometry change around shape surface, which is not reflected by occupancy volume. 

We instead revisit the 3D volumetric representation and resort to 3D convolutional neural networks (CNNs) for feature learning, due to their impressive performance in feature learning and the ability to incorporate spatial context. However, volumetric methods and 3D convolution involve discretization, which might raise concerns regarding whether a discretized volume can preserve subtle geometric details as continuous representations learned in implicit functions. We investigate the relationship between volume resolution and quantization error on synthetic data by converting target mesh objects to TSDF volumes, as shown in Figure~\ref{fig:quantization_error}. We observe that the quantization errors are significantly reduced by increasing volume resolution and become nearly negligible when reaching a relatively high resolution (e.g., 512 or higher). In other words, achieving fine-detailed reconstruction is not supposed to be restricted by the use of volume representations as long as a proper volume resolution is utilized. Therefore, we present a method with high-resolution feature volumes, e.g., 256 and 512, while traditional volumetric methods \cite{varol18_bodynet,gilbert2018volumetric} are often limited to much lower resolutions, such as 32 or 128.



On the other hand, an increase in volume resolution may lead to a cubic growth of memory overhead \cite{8100085}. Reducing memory costs while guaranteeing the granularity of volumetric representations is necessary for pursuing high-quality reconstruction. Thus, we adopt a coarse-to-fine approach and cull away irrelevant voxels to build a sparse high-resolution feature volume. At the coarse level, the network computes an initial TSDF by applying a U-Net with sparse 3D CNN \cite{3DSemanticSegmentationWithSubmanifoldSparseConvNet} on the sparse feature volume, which is carved by a visual hull. Through our experiments, it turns out that more than 95\% of the volume grids are discarded by the visual hull culling, making the sparse 3D CNN efficient. At the fine level, the network focuses on a narrow band near the zero-level set of the initial TSDF and discretizes the narrow band with smaller voxels. By employing this narrow-band culling, we further shrink the sampling space, resulting in a relatively small range of grid numbers (usually 300K--500K in our experiments) even with a high volume resolution of 512. The remaining voxels in the narrow band are associated with features that fuse high-frequency information from the computed normal maps upon the low-frequency shape from the coarse level to compute the TSDF at high resolution. The final mesh is then extracted from the TSDF using the Marching-Cube algorithm ~\cite{Lorensen87marchingcubes}.
% Different from the u-net sturcture to preserve global topology context, we then apply a shallow 3dcnn to compute the final TSDF $D_{final}$ which contain more local geometry detail.




% \ping{this paragraph can be expanded. It is an important contribution and often ignored by other works. stress on the novel idea of regressing blending weights instead of colors}

In addition to geometry, high-quality mesh texture is also a crucial factor contributing to visual appearance. Directly computing a color field in 3D space, as in \cite{iccv2020PIFu}, struggles to capture high-frequency texture details, while the neural radiance field (NeRF) \cite{yu2020pixelnerf} or the DoubleField~\cite{shao2022doublefield} require expensive per-instance optimization and are often unstable for sparse input images. In contrast, we adopt an image-based rendering approach to compute a texture atlas map, which is efficient and widely supported in existing computer graphics tools. 
Specifically, we compute a blending weight at each 3D point on the mesh surface to determine its color as a weighted average of the colors at its image projections. The blending weights can be computed at a relatively coarse resolution, e.g., 512 volume resolution in our case, and leave texture details to the high-resolution images, such as 1K or 2K. Unlike previous methods that generate blurry texturing results under sparse input, our method generalizes well on both synthetic and real data with just a few input views. 
Figure~\ref{fig:teaser} shows two examples reconstructed by our method. Despite the challenging garment, pose, and occlusion, our method recovers faithful shape, normal, and texture on the right.

%with a wide variety of poses and clothing styles, and it is also adaptive to handle input image with arbitrary resolutions.
%\sicong{For this concern we claim that when the resolution of dicretized volume meets certain threshold (which is 256 in our experiment), the quantization error can be neglected.} 



In summary, the main contributions of this paper are as follows:
\begin{itemize}
\vspace{-0.1in}
  \item 
  We revisit the 3D volumetric representation and demonstrate that it can support clothed human reconstruction with equal or even better performance compared to implicit representation. 
  \item 
  We develop a memory and computation-efficient method for high-resolution volumetric reconstruction using sophisticated sparse 3D CNN, coarse-to-fine estimation, and voxel culling by visual hull and narrow bands. 
  \item 
  We introduce a novel method to compute a texture atlas map, which captures rich appearance details from high-resolution input images.
  \item 
  We achieve impressive results on standard benchmark datasets Twindom and MultiHuman, significantly reducing the point-2-surface (P2S) precision to approximately 0.2cm from just six input views, with more than $50\%$ error reduction compared to the state-of-the-art methods, including DoubleField~\cite{shao2022doublefield} and PIFuHD~\cite{saito2020pifuhd}.
\end{itemize}
\section{LCA overview and Method}
\label{sec:background}

 In this work, we follow a process-based LCA due to its consistency and ability to provide provides a more detailed and specific understanding of the life cycle. Other than process-based LCA, economic input-output LCA is the most common approach and a valuable tool for determining the impact of economic activities across sectors~\cite{EIO_LCA}. LCA considers the entire life cycle of a product, from raw material extraction to end-of-life disposal, in order to identify and address environmental impacts. One of the key aspects of LCA is defining the system boundaries, which determine which stages of the life cycle are included in the assessment. These boundaries can be narrow, focusing on specific parts of the product's life cycle, or broad, encompassing the entire supply chain. 

% Figure environment removed

Figure~\ref{fig:LCA_boundry} illustrates the boundaries and extent of our study. We present a thorough analysis of the inventory and energy, encompassing the manufacturing, assembly, and use phases of both CMOS and AQFP integrated circuits. The accounting of upstream inputs (e.g., raw materials extraction and transportation) has not been considered due to the high degree of uncertainty involved. Our investigation provides a detailed account of the life-cycle inventory and energy analysis which includes the following aspects: functional unit selection, the wafer manufacturing and fabrication phase, wafer cutting and yield analysis, assembly phase, and use phase, which will be discussed in the rest of this section. 

A. Functional unit selection: The life-cycle impacts of CMOS/AQFP circuits can be significantly influenced by the selection of the functional unit. To perform an ``apple-to-apple'' comparison and ensure the same level of functionality, we choose the single-core 32-bit CMOS RISC-V architecture for both CMOS and AQFP technologies as the functional unit. The AQFP RISC-V processor includes crucial components such as a decoder, 32-bit ALU, register file, controller, and L1 AQFP cache memory~\cite{AQFP_memory}. The total area of the AQFP RISC-V processor is obtained by the summation over component areas. The synthesis of the AQFP component circuits is based on the MIT-LL process~\cite{MITLL_process}.
Based on our synthesis, the clock frequency, power, and area for the AQFP processor are 5 GHz, \SI{41}{\micro\watt}, and 3.5~cm$^2$, respectively. Note that to ensure a fair evaluation, we compare today's RISC-V AQFP processors built out of micron-size AQFP devices with a 130 nm CMOS processor that has its fabrication process steps provided in~\cite{fabrication_steps}. Moreover, note that for state-of-the-art technology nodes (e.g., 7nm), the full process steps details are not available. We used Western Digital SweRV EH1 RISC-V processor features as a sample and used scaling equations provided in~\cite{scalling_formulas} to obtain the equivalent 130 nm chip features. For CMOS, the clock frequency, power, and area are 1 GHz, \SI{7.5}{\watt}, and 12.1~mm$^2$, respectively. 


B. The manufacturing processes for AQFP and CMOS circuits share some similarities, but there are also notable differences in materials and specific steps. AQFP technology typically employs simpler steps than those used in semiconductor fabrication plants. For CMOS, a complete inventory of fabrication steps and their associated energy requirements for a 300 mm wafer can be found in Reference~\cite{fabrication_steps}, where a total of 206 process steps are listed in Tables S3 to S17 of the supplementary materials. To evaluate the fabrication process for AQFP technology, we carefully reviewed the process presented in~\cite{Japan_process_steps}, and estimated a total of 216 steps and their corresponding energy values. The total energy required for each wafer is obtained by summing the energy values of all steps.   
 

C. Wafer cutting: After determining the energy required for fabricating a wafer calculate the fabrication yield. The yield for CMOS wafers is found (using Murphy’s model) to be 97.6\%, while the yield for AQFP wafers was estimated to be 85.2\%. Using yield, we calculate the number of functional dies for each technology. The energy for the fabrication of each die is calculated using the formula: Manufacturing Energy of Wafer / Number of Functional Dies. 

D. Assembly phase: In our analysis, we consider the energy required for the packaging and assembly of each die. We consider commonly used plastic packages. The energy usage in the packaging stage is 0.34kWh per cm\textsuperscript{2} of silicon~\cite{packaging_energy}.

E. Use phase: We assume the usage as the server for both technologies. AQFP and CMOS circuit lifetimes are considered to be 10 and 5 years, respectively. Large computing systems
require cooling. This is particularly important for superconducting electronics which operate at temperatures below 10 K. We consider the cooling energy cost of the superconducting circuit to be 400X larger than the energy generated by the circuit at the cryogenic temperature~\cite{400X_cooling}.    




\section{Results and Discussion}
\label{sec:results_discussion}
%
\begin{table*}[ht]
\centering
\scriptsize
% \def\arraystretch{0.9}
%
\resizebox{\textwidth}{!}{%
    \begin{tabular}{l|cccc|cccc}
        \toprule
        \multicolumn{1}{c}{} & \multicolumn{4}{c}{\textbf{MIND}} & \multicolumn{4}{c}{\textbf{Adressa}} \\  
        \cmidrule(lr){2-5} \cmidrule(lr){6-9}
        
        \multicolumn{1}{l|}{Model} 
        & AUC & MRR & nDCG@5 & nDCG@10
        & AUC & MRR & nDCG@5 & nDCG@10
        \\
        \hline

       NRMS-PLM 
        % MIND
        & 63.0\textsubscript{$\pm$1.5} % AUC
        & 35.5\textsubscript{$\pm$0.6}  % MRR
        & 33.4\textsubscript{$\pm$0.7} % nDCG@5
        & 39.9\textsubscript{$\pm$0.6}  % nDCG@10

        % Adressa
        & 72.3\textsubscript{$\pm$3.3} % AUC
        & 43.0\textsubscript{$\pm$2.7} % MRR
        & 44.3\textsubscript{$\pm$2.8} % nDCG@5
        & 51.3\textsubscript{$\pm$2.3} % nDCG@10
        \\

        MINER
        % MIND
        & 63.1\textsubscript{$\pm$1.2} % AUC
        & 35.5\textsubscript{$\pm$1.1}  % MRR
        & 33.7\textsubscript{$\pm$1.1} % nDCG@5
        & 40.0\textsubscript{$\pm$1.0}  % nDCG@10

        % Adressa
        & 70.1\textsubscript{$\pm$4.9} % AUC
        & 37.3\textsubscript{$\pm$4.1} % MRR
        & 38.5\textsubscript{$\pm$5.1} % nDCG@5
        & 46.3\textsubscript{$\pm$4.1} % nDCG@10
        \\

        \hdashline

        NAML-PLM
        % MIND
        & 60.6\textsubscript{$\pm$3.4} % AUC
        &  \underline{37.6\textsubscript{$\pm$0.4}}  % MRR
        &  \underline{35.9\textsubscript{$\pm$0.4}} % nDCG@5
        &  \underline{42.2\textsubscript{$\pm$0.4}}  % nDCG@10

        % Adressa
        & 50.0\textsubscript{$\pm$0.0} % AUC
        & \underline{45.0\textsubscript{$\pm$5.0}} % MRR
        & 47.2\textsubscript{$\pm$5.5} % nDCG@5
        & 52.5\textsubscript{$\pm$4.1} % nDCG@10
        \\

        LSTUR-PLM
         % MIND
        & 54.6\textsubscript{$\pm$3.0} % AUC
        & 33.3\textsubscript{$\pm$1.5}  % MRR
        & 31.7\textsubscript{$\pm$1.8} % nDCG@5
        & 38.3\textsubscript{$\pm$1.7}  % nDCG@10

        % Adressa
        & 65.0\textsubscript{$\pm$7.2} % AUC
        & 43.1\textsubscript{$\pm$1.7} % MRR
        & 44.8\textsubscript{$\pm$2.6} % nDCG@5
        & 51.2\textsubscript{$\pm$2.0} % nDCG@10
        \\

        MINS-PLM 
        % MIND
        & 61.3\textsubscript{$\pm$2.7} % AUC
        & 36.2\textsubscript{$\pm$0.3}  % MRR
        & 34.5\textsubscript{$\pm$0.4} % nDCG@5
        & 40.8\textsubscript{$\pm$0.3}  % nDCG@10

        % Adressa
        & 74.3\textsubscript{$\pm$3.2} % AUC
        & 44.2\textsubscript{$\pm$2.9} % MRR
        & \underline{47.3\textsubscript{$\pm$3.3}} % nDCG@5
        & \underline{53.0\textsubscript{$\pm$3.4}} % nDCG@10      
        \\ 

        CAUM\textsubscript{no entities}-PLM
        % MIND
        &  \underline{66.2\textsubscript{$\pm$3.0}} % AUC
        & 36.6\textsubscript{$\pm$2.0}  % MRR
        & 34.6\textsubscript{$\pm$2.0} % nDCG@5
        & 41.0\textsubscript{$\pm$1.9}  % nDCG@10

        % Adressa
        & \underline{76.5\textsubscript{$\pm$1.2}} % AUC
        & 43.6\textsubscript{$\pm$1.3} % MRR
        & 46.9\textsubscript{$\pm$1.3} % nDCG@5
        & 52.0\textsubscript{$\pm$1.2} % nDCG@10
        \\

        CAUM-PLM 
        % MIND
        & 66.4\textsubscript{$\pm$3.1} % AUC
        & 36.2\textsubscript{$\pm$1.2}  % MRR
        & 34.3\textsubscript{$\pm$1.3} % nDCG@5
        & 40.8\textsubscript{$\pm$1.3}  % nDCG@10

        % Adressa
        & -- % AUC
        & -- % MRR
        & -- % nDCG@5
        & -- % nDCG@10
        \\

        TANR-PLM
         % MIND
        & 63.3\textsubscript{$\pm$1.1} % AUC
        & 37.0\textsubscript{$\pm$1.0}  % MRR
        & 35.2\textsubscript{$\pm$1.0} % nDCG@5
        & 41.6\textsubscript{$\pm$0.9}  % nDCG@10

        % Adressa
        & 50.0\textsubscript{$\pm$0.0} % AUC
        & 43.8\textsubscript{$\pm$1.0} % MRR
        & 45.6\textsubscript{$\pm$1.3} % nDCG@5
        & 51.4\textsubscript{$\pm$0.6} % nDCG@10
        \\
        
        \hdashline

        SentiRec-PLM
         % MIND
        & 62.2\textsubscript{$\pm$0.7} % AUC
        & 35.7\textsubscript{$\pm$0.4}  % MRR
        & 33.9\textsubscript{$\pm$0.4} % nDCG@5
        & 40.5\textsubscript{$\pm$0.4}  % nDCG@10

        % Adressa
        & 67.6\textsubscript{$\pm$2.7} % AUC
        & 33.1\textsubscript{$\pm$2.4} % MRR
        & 32.9\textsubscript{$\pm$3.8} % nDCG@5
        & 40.8\textsubscript{$\pm$2.4} % nDCG@10
        \\

        SentiDebias-PLM
        % MIND
        & 55.0\textsubscript{$\pm$2.5} % AUC
        & 27.8\textsubscript{$\pm$1.9}  % MRR
        & 25.5\textsubscript{$\pm$1.9} % nDCG@5
        & 32.2\textsubscript{$\pm$2.0}  % nDCG@10

        % Adressa
        & 67.4\textsubscript{$\pm$2.4} % AUC
        & 35.7\textsubscript{$\pm$3.4}  % MRR
        & 36.4\textsubscript{$\pm$4.2} % nDCG@5
        & 44.2\textsubscript{$\pm$2.9}  % nDCG@10 
        \\ 
        \hline
        
        \manner{} (\texttt{CR-Module})
        % MIND
        & \textbf{69.7\textsubscript{$\pm$0.9 }} % AUC
        & \textbf{38.6\textsubscript{$\pm$0.6}}  % MRR
        & \textbf{37.0\textsubscript{$\pm$0.6}} % nDCG@5
        & \textbf{43.2\textsubscript{$\pm$0.6}}  % nDCG@10

        % Adressa
        & \textbf{79.4\textsubscript{$\pm$1.7}} % AUC
        & \textbf{47.0\textsubscript{$\pm$2.4}} % MRR
        & \textbf{48.9\textsubscript{$\pm$2.8}} % nDCG@5
        & \textbf{54.3\textsubscript{$\pm$2.5}} % nDCG@10  
        \\
        \hline

        \rowcolor{Gray}
        Improvement (\%)
        % MIND
        & + 5.4% AUC
        & + 2.8% MRR
        & + 3.1% nDCG@5
        & + 2.3% nDCG@10

         % Adressa
        & + 3.7% AUC
        & + 4.6% MRR
        & + 3.3% nDCG@5
        & + 2.5% nDCG@10
        \\ 
        
        \bottomrule
        
    \end{tabular}%
    }
\caption{Content-based recommendation performance. We average results across five runs, and report the relative improvement over the best baseline. The best results per column are highlighted in bold, the second best underlined.}
\label{tab:results_recommendation}
%

\vspace{-0.5em}
\end{table*}

%

We first discuss \manner{}'s content personalization performance. We then analyze its capability for single- and multi-aspect (i) diversification and (ii) personalization. In the aspect customization setups, we treat \manner{}'s \texttt{CR-Module} as a baseline. Lastly, we evaluate its ability to re-use pretrained aspect-specific modules in cross-lingual transfer. 

\subsection{Content Personalization}
\label{sec:content_personalization}

Table \ref{tab:results_recommendation} summarizes the results on content personalization. Since the task does not require any aspect-based customization, we evaluate the \manner{} variant that uses only its CR-Module at inference time (i.e., $\lambda \hspace{-0.2em} = \hspace{-0.2em} 0$).
%%
\manner{} consistently outperforms all state-of-the-art NNRs in terms of both classification and ranking metrics on both datasets. Given that \manner{}'s \texttt{CR-Module} derives the user embedding by merely averaging clicked news embeddings, these results question the need for complex parameterized UEs, present in all the baselines, in line with the findings of \citet{iana2023simplifying}. 

% \vspace{1.4mm}
% \noindent\textbf{Ablations.}
% 
We ablate \texttt{CR-Module}'s content personalization performance for (i) different inputs to the NE and (ii) alternative architecture designs and training objectives.
%
We find that all groups of features (e.g., abstract, named entities) contribute to the overall performance (cf. Fig. \ref{fig:ablation_features}). 
% 
Moreover, we confirm the findings of \citet{iana2023simplifying} that (i) late fusion outperforms a parameterized UE (i.e., early fusion), and that (ii) SCL better separates classes than cross-entropy loss, in line with other similarity-oriented NLP tasks  \cite{reimers2019sentence}.

\subsection{Single-Aspect Customization}
%%
\newcolumntype{g}{>{\columncolor{Gray}}c}

\begin{table*}[ht]
\centering

\resizebox{\textwidth}{!}{%
    \begin{tabular}{l|cgcgcc|cgcgcc}
        \toprule
        \multicolumn{1}{c}{} & \multicolumn{6}{c}{\textbf{MIND}} & \multicolumn{6}{c}{\textbf{Adressa}} \\  
        \cmidrule(lr){2-7} \cmidrule(lr){8-13}
        
        \multicolumn{1}{l|}{Model} 
        & nDCG@10 
        & D\textsubscript{ctg}@10 & T\textsubscript{ctg}@10 
        & D\textsubscript{snt}@10 & T\textsubscript{snt}@10 
        & T\textsubscript{all}@10
        
        & nDCG@10 
        & D\textsubscript{ctg}@10 & T\textsubscript{ctg}@10 
        & D\textsubscript{snt}@10 & T\textsubscript{snt}@10 
        & T\textsubscript{all}@10
        \\
        \hline

        NAML-PLM
        % MIND
        & 41.5$\pm$0.2 % nDCG@10
        & 47.4$\pm$0.7  % div_c@10
        & 44.3$\pm$0.3  % tradeoff_ctg
        & 65.6$\pm$0.4  % div_s@10
        & 50.8$\pm$0.2  % tradeoff_snt
        & 49.7$\pm$0.2  % tradeoff_all

        % Adressa
        & 50.2$\pm$1.9  % nDCG@10
        & 31.9$\pm$2.3  % div_c@10
        & 39.0$\pm$2.1  % tradeoff_ctg
        & 61.5$\pm$0.8  % div_s@10
        & 55.3$\pm$1.5  % tradeoff_snt
        & 44.4$\pm$1.9  % tradeoff_all        
        \\
        
       NRMS-PLM 
       % MIND
        & 38.4$\pm$2.9 % nDCG@10
        & 50.2$\pm$2.4  % div_c@10
        & 43.4$\pm$1.2  % tradeoff_ctg
        & 66.6$\pm$0.9  % div_s@10
        & 48.7$\pm$2.2  % tradeoff_snt
        & 49.1$\pm$0.9  % tradeoff_all

        % Adressa
        & 51.2$\pm$1.6  % nDCG@10
        & 32.4$\pm$0.5  % div_c@10
        & 39.7$\pm$0.5  % tradeoff_ctg
        & 62.1$\pm$0.3  % div_s@10
        & 56.1$\pm$1.0  % tradeoff_snt
        & 45.1$\pm$0.5  % tradeoff_all     
        \\

        LSTUR\textsubscript{ini}-PLM
       % MIND
        & 39.0$\pm$2.4 % nDCG@10
        & 48.4$\pm$2.2  % div_c@10
        & 43.1$\pm$0.7  % tradeoff_ctg
        & 65.6$\pm$0.5  % div_s@10
        & 48.9$\pm$1.8  % tradeoff_snt
        & 48.7$\pm$0.6  % tradeoff_all

        % Adressa
        & 51.3$\pm$2.4  % nDCG@10
        & 27.6$\pm$5.7  % div_c@10
        & 35.6$\pm$5.3  % tradeoff_ctg
        & 60.8$\pm$0.8  % div_s@10
        & 55.6$\pm$1.5  % tradeoff_snt
        & 41.2$\pm$5.0  % tradeoff_all     
        \\

        LSTUR\textsubscript{con}-PLM
        % MIND
        & 31.8$\pm$1.2 % nDCG@10
        & 43.1$\pm$2.0  % div_c@10
        & 36.6$\pm$1.5  % tradeoff_ctg
        & 66.7$\pm$0.4  % div_s@10
        & 43.0$\pm$1.1  % tradeoff_snt
        & 43.0$\pm$1.4  % tradeoff_all

        % Adressa
        & 46.4$\pm$2.9  % nDCG@10
        & 29.0$\pm$1.3  % div_c@10
        & 35.7$\pm$1.8  % tradeoff_ctg
        & 61.1$\pm$0.9  % div_s@10
        & 52.7$\pm$2.2  % tradeoff_snt
        & 41.4$\pm$1.7  % tradeoff_all     
        \\

        MINS-PLM
        % MIND
        & 40.2$\pm$1.1 % nDCG@10
        & 48.7$\pm$1.0  % div_c@10
        & 44.0$\pm$0.4  % tradeoff_ctg
        & 66.7$\pm$0.7  % div_s@10
        & 50.1$\pm$0.8  % tradeoff_snt
        & 49.6$\pm$0.4  % tradeoff_all

        % Adressa
        & 52.6$\pm$1.7  % nDCG@10
        & 34.4$\pm$1.5  % div_c@10
        & \textbf{41.6$\pm$1.5}  % tradeoff_ctg
        & 62.0$\pm$0.3  % div_s@10
        & 56.9$\pm$1.1  % tradeoff_snt
        & \textbf{46.7$\pm$1.2}  % tradeoff_all     
        \\

        CAUM\textsubscript{no entities}-PLM
        % MIND
        & 42.1$\pm$0.9 % nDCG@10
        & 47.8$\pm$0.6  % div_c@10
        & 44.8$\pm$0.6  % tradeoff_ctg
        & 65.8$\pm$0.8  % div_s@10
        & 51.3$\pm$0.6  % tradeoff_snt
        & 50.1$\pm$0.5  % tradeoff_all

        % Adressa
        & 49.5$\pm$3.4  % nDCG@10
        & \underline{35.5$\pm$1.0}  % div_c@10
        & \underline{41.3$\pm$1.5}  % tradeoff_ctg
        & 62.0$\pm$0.2  % div_s@10
        & 55.0$\pm$2.2  % tradeoff_snt
        & \underline{46.5$\pm$1.3}  % tradeoff_all 
        \\

        CAUM-PLM
        % MIND
        & 41.7$\pm$1.5 % nDCG@10
        & 47.8$\pm$1.0  % div_c@10
        & 44.5$\pm$0.8  % tradeoff_ctg
        & 66.1$\pm$0.4  % div_s@10
        & 51.1$\pm$1.1  % tradeoff_snt
        & 50.0$\pm$0.7  % tradeoff_all

        % Adressa
        & --  % nDCG@10
        & --  % div_c@10
        & --  % tradeoff_ctg
        & --  % div_s@10
        & --  % tradeoff_snt
        & --  % tradeoff_all
        \\

        MINER\textsubscript{max}
        % MIND
        & 39.4$\pm$0.7 % nDCG@10
        & 47.5$\pm$0.9 % div_c@10
        & 43.1$\pm$0.7 % tradeoff_ctg
        & 63.7$\pm$1.3 % div_s@10
        & 48.7$\pm$0.8 % tradeoff_snt
        & 48.3$\pm$0.8 % tradeoff_all

        % Adressa
        & 51.5$\pm$2.6  % nDCG@10
        & 32.6$\pm$0.9  % div_c@10
        & 39.9$\pm$0.8  % tradeoff_ctg
        & 61.6$\pm$0.3  % div_s@10
        & 56.1$\pm$1.5  % tradeoff_snt
        & 45.2$\pm$0.7  % tradeoff_all     
        \\

        MINER\textsubscript{mean}
        % MIND
        & 40.0$\pm$1.0 % nDCG@10
        & 49.4$\pm$1.2 % div_c@10
        & 44.2$\pm$0.4 % tradeoff_ctg
        & 65.7$\pm$0.9 % div_s@10
        & 4.7$\pm$1.0 % tradeoff_snt
        & 49.6$\pm$0.5 % tradeoff_all

        % Adressa
        & 47.8$\pm$4.0  % nDCG@10
        & 31.0$\pm$1.1  % div_c@10
        & 37.6$\pm$1.8  % tradeoff_ctg
        & 60.9$\pm$0.3  % div_s@10
        & 53.5$\pm$2.5  % tradeoff_snt
        & 43.1$\pm$1.6  % tradeoff_all     
        \\

        MINER\textsubscript{weighted}
        % MIND
        & 34.0$\pm$3.1 % nDCG@10
        & 49.1$\pm$1.6  % div_c@10
        & 40.1$\pm$1.9  % tradeoff_ctg
        & 64.6$\pm$1.1  % div_s@10
        & 44.5$\pm$2.7  % tradeoff_snt
        & 45.9$\pm$1.7  % tradeoff_all

        % Adressa
        & 47.8$\pm$4.9  % nDCG@10
        & 32.5$\pm$0.6  % div_c@10
        & 38.7$\pm$1.8  % tradeoff_ctg
        & 61.0$\pm$0.7  % div_s@10
        & 53.5$\pm$3.3  % tradeoff_snt
        & 44.0$\pm$1.7  % tradeoff_all     
        \\
        \hdashline

        TANR-PLM 
        % MIND
        & 40.2$\pm$1.3 % nDCG@10
        & 49.9$\pm$1.6  % div_c@10
        & 44.5$\pm$0.4  % tradeoff_ctg
        & 66.6$\pm$1.0  % div_s@10
        & 50.2$\pm$0.7  % tradeoff_snt
        & 50.0$\pm$0.3  % tradeoff_all

        % Adressa
        & 50.3$\pm$4.4  % nDCG@10
        & 30.3$\pm$3.2  % div_c@10
        & 37.6$\pm$1.4  % tradeoff_ctg
        & 61.0$\pm$1.5  % div_s@10
        & 55.0$\pm$2.2  % tradeoff_snt
        & 43.1$\pm$1.5  % tradeoff_all     
        \\

        SentiRec-PLM
        % MIND
        & 39.8$\pm$0.7 % nDCG@10
        & 50.4$\pm$0.4  % div_c@10
        & 44.5$\pm$0.6  % tradeoff_ctg
        & 66.8$\pm$0.7  % div_s@10
        & 49.8$\pm$0.5  % tradeoff_snt
        & 50.0$\pm$0.5  % tradeoff_all

        % Adressa
        & 38.9$\pm$1.1  % nDCG@10
        & \textbf{35.7$\pm$1.0}  % div_c@10
        & 37.2$\pm$0.4  % tradeoff_ctg
        & \textbf{68.5$\pm$1.0}  % div_s@10
        & 49.6$\pm$1.1  % tradeoff_snt
        & 43.9$\pm$0.4  % tradeoff_all     
        \\ 

        SentiDebias-PLM
        % MIND
        & 34.7$\pm$2.2 % nDCG@10
        & \underline{50.5$\pm$1.8}  % div_c@10
        & 41.1$\pm$1.3  % tradeoff_ctg
        & \underline{68.0$\pm$0.9}  % div_s@10
        & 45.9$\pm$1.9  % tradeoff_snt
        & 47.3$\pm$1.1  % tradeoff_all

        % Adressa
        & 52.0$\pm$2.7 % nDCG@10
        & 32.7$\pm$0.6  % div_c@10
        & 40.2$\pm$1.1  % tradeoff_ctg
        & 62.4$\pm$0.4  % div_s@10
        & 56.7$\pm$1.1  % tradeoff_snt
        & 45.6$\pm$1.0  % tradeoff_all 
        \\
        \hline

       \mannertitle{} (\texttt{CR-Module})
         % MIND
        & 42.4$\pm$0.4 % nDCG@10
        & 48.3$\pm$0.3 % div_c@10
        & 45.2$\pm$0.3 % tradeoff_ctg
        & 65.1$\pm$0.5 % div_s@10
        & 51.4$\pm$0.4 % tradeoff_snt
        & 50.3$\pm$0.4 % tradeoff_all

        % Adressa
        & \textbf{54.3$\pm$2.5}  % nDCG@10
        & 31.7$\pm$0.2  % div_c@10
        & 40.0$\pm$0.7  % tradeoff_ctg
        & 61.4$\pm$0.3  % div_s@10
        & \underline{57.6$\pm$1.5}  % tradeoff_snt
        & 45.3$\pm$0.6  % tradeoff_all     
        \\ 

        \mannertitle{} ($\lambda_{ctg}=-0.3$, $\lambda_{snt}=0$)
        % MIND
        & 41.2$\pm$0.5 % nDCG@10
        & 49.8$\pm$0.5  % div_c@10
        & 45.1$\pm$0.4  % tradeoff_ctg
        & 65.4$\pm$0.6  % div_s@10
        & 50.6$\pm$0.5  % tradeoff_snt
        & 50.3$\pm$0.4  % tradeoff_all

        % Adressa
        & 50.9$\pm$2.5  % nDCG@10
        & 34.1$\pm$0.3  % div_c@10
        & 40.8$\pm$0.8  % tradeoff_ctg
        & 61.9$\pm$0.3  % div_s@10
        & 55.8$\pm$1.6  % tradeoff_snt
        & 46.0$\pm$0.7  % tradeoff_all     
        \\ 

        \mannertitle{} ($\lambda_{ctg}=0$, $\lambda_{snt}=-0.2$)
        % MIND
        & 41.8$\pm$0.3 % nDCG@10
        & 48.8$\pm$0.5  % div_c@10
        & 45.0$\pm$0.4  % tradeoff_ctg
        & 65.4$\pm$0.5  % div_s@10
        & 51.0$\pm$0.4  % tradeoff_snt
        & 50.3$\pm$0.4  % tradeoff_all

        % Adressa
        & \underline{53.8$\pm$2.5}  % nDCG@10
        & 32.4$\pm$0.2  % div_c@10
        & 40.4$\pm$0.7  % tradeoff_ctg
        & \underline{63.0$\pm$0.3}  % div_s@10
        & \textbf{58.0$\pm$1.5}  % tradeoff_snt
        & 45.9$\pm$0.6  % tradeoff_all     
        \\ 
        \hdashline

        \manner{} (\texttt{CR-Module})
         % MIND
        & \textbf{43.2$\pm$0.6} % nDCG@10
        & 49.3$\pm$0.3  % div_c@10
        & 46.0$\pm$0.3  % tradeoff_ctg
        & 65.4$\pm$0.6  % div_s@10
        & \underline{52.0$\pm$0.4}  % tradeoff_snt
        & 51.1$\pm$0.2  % tradeoff_all

        % Adressa
        & --  % nDCG@10
        & --  % div_c@10
        & --  % tradeoff_ctg
        & --  % div_s@10
        & --  % tradeoff_snt
        & --  % tradeoff_all     
        \\ 

        \manner{} ($\lambda_{ctg}=-0.2$, $\lambda_{snt}=0$)
        % MIND
        & 42.0$\pm$0.6 % nDCG@10
        & \textbf{51.5$\pm$0.3}  % div_c@10
        & \textbf{46.2$\pm$0.3}  % tradeoff_ctg
        & 65.6$\pm$0.6  % div_s@10
        & 51.2$\pm$0.4  % tradeoff_snt
        & \underline{51.3$\pm$0.3}  % tradeoff_all

        % Adressa
        & --  % nDCG@10
        & --  % div_c@10
        & --  % tradeoff_ctg
        & --  % div_s@10
        & --  % tradeoff_snt
        & --  % tradeoff_all     
        \\ 

        \manner{} ($\lambda_{ctg}=0$, $\lambda_{snt}=-0.2$)
        % MIND
        & \underline{43.1$\pm$0.6} % nDCG@10
        & 49.5$\pm$0.3  % div_c@10
        & \underline{46.1$\pm$0.3}  % tradeoff_ctg
        & \textbf{68.1$\pm$0.4}  % div_s@10
        & \textbf{52.8$\pm$0.5}  % tradeoff_snt
        & \textbf{51.6$\pm$0.2}  % tradeoff_all

        % Adressa
        & --  % nDCG@10
        & --  % div_c@10
        & --  % tradeoff_ctg
        & --  % div_s@10
        & --  % tradeoff_snt
        & --  % tradeoff_all      
        \\ 
        
        \bottomrule
        
    \end{tabular}%
    }
\caption{Diversity (and content personalization) performance on diversification tasks. For \manner{}, we list the best results (in terms of T\textsubscript{A\textsubscript{p}}) of single-aspect diversification ($\lambda_{ctg}=-0.2$ and $\lambda_{snt}=0$ for topical category diversification; $\lambda_{ctg}=0$ and $\lambda_{snt}=-0.2$ for sentiment diversification) on MIND. For \mannertitle{}, we list the best results (in terms of T\textsubscript{A\textsubscript{p}}) of single-aspect diversification ($\lambda_{ctg}=-0.3$ and $\lambda_{snt}=0$ for topical category diversification; $\lambda_{ctg}=0$ and $\lambda_{snt}=-0.2$ for sentiment diversification) on Adressa. The best results per column are highlighted in bold, the second best are underlined.
}
\label{tab:results_diversification}

\end{table*}


%
% Figure environment removed



\vspace{1.4mm}
\noindent\textbf{Diversification.}
Table~\ref{tab:results_diversification} summarizes the results on aspect diversification tasks.
%%%
Most baselines (including \manner{}'s \texttt{CR-Module} without aspect diversification) obtain similar diversification scores (D\textsubscript{ctg} and D\textsubscript{snt}). 
%
The sentiment-aware SentiRec-PLM, with an explicit auxiliary sentiment diversification objective, yields the highest sentiment diversity on Adressa; this comes at the expense of content personalization quality (lowest nDCG).
%
On MIND, the sentiment-specific SentiDebias-PLM achieves the highest sentiment diversity, but also exhibits lower content personalization performance. Overall, these results point to a trade-off between content personalization and aspectual diversity: models with higher D\textsubscript{A\textsubscript{p}} tend to have a lower nDCG. 

Unlike all other models, \manner{} can trade content personalization for diversity (and vice-versa) with different values of the aspect coefficients $\lambda_{A_p}$. Fig. \ref{fig:single_aspect_div_mind} illustrates its performance in single-aspect diversification tasks for different values of $\lambda$\textsubscript{ctg} and $\lambda$\textsubscript{snt} on MIND.
%%%
The steady drop in nDCG together with the steady increase in D\textsubscript{A\textsubscript{p}} indeed indicate the existence of a trade-off between content personalization and aspect diversification. For topical categories we observe a steeper decline in content personalization quality with improved diversification than for sentiment. Sentiment diversity reaches peak performance for $\lambda_{snt} \hspace{-0.2em} = \hspace{-0.2em} -0.4$, whereas category diversity continues to increase all the way to $\lambda_{ctg} \hspace{-0.2em} = \hspace{-0.2em} -0.9$. Intuitively, content-based recommendation is more aligned with the topical than with the sentiment consistency of recommendations. The best trade-off (i.e., maximal performance w.r.t. T\textsubscript{A\textsubscript{p}}@10) is achieved for $\lambda_{ctg}\hspace{-0.2em} = \hspace{-0.2em} -0.2$ for topics, and $\lambda_{snt} \hspace{-0.2em} = \hspace{-0.2em} -0.3$ for sentiment. We report analogous results on Adressa in Appendix \ref{sec:appendix_single_apsect_customization}.
%
We attribute these effects to the representation spaces of the \texttt{A-Modules}. Fig. \ref{fig:tsne_categ_mind} shows the 2-dimensional t-SNE visualizations \cite{van2008visualizing} of the news embeddings produced with category-specialized encoders trained on MIND (see Fig. \ref{fig:tsne_sent_mind} for sentiment). The results confirm that the latent representation space of the encoder was reshaped to group same-class instances. The separation of classes, however, is less prominent for representation spaces of the encoders trained on Adressa (cf. Fig. \ref{fig:tsne_embeddings_adressa}) than for those learned on MIND (e.g., the effect is stronger on the category-shaped embedding space).\footnote{We believe that this is because Adressa has 10 times fewer news than MIND (and contrastive learning, naturally, benefits from more news pairs), with over half of the topical categories in Adressa being represented with fewer than 100 examples.} 
%
\newcolumntype{g}{>{\columncolor{Gray}}c}

\begin{table*}[ht]
\centering

\resizebox{\textwidth}{!}{%
    \begin{tabular}{l|cgcgcc|cgcgcc}
        \toprule
        \multicolumn{1}{c}{} & \multicolumn{6}{c}{\textbf{MIND}} & \multicolumn{6}{c}{\textbf{Adreesa}} \\  
        \cmidrule(lr){2-7} \cmidrule(lr){8-13}
        
        \multicolumn{1}{l|}{Model} 
        & nDCG@10 
        & PS\textsubscript{ctg}@10 & T\textsubscript{ctg}@10 
        & PS\textsubscript{snt}@10 & T\textsubscript{snt}@10 
        & T\textsubscript{all}@10
        
        & nDCG@10 
        & PS\textsubscript{ctg}@10 & T\textsubscript{ctg}@10 
        & PS\textsubscript{snt}@10 & T\textsubscript{snt}@10 
        & T\textsubscript{all}@10
        \\
        \hline

        NAML-PLM
        % MIND
        & 41.5$\pm$0.2 % nDCG@10
        & \underline{25.6$\pm$0.3} % pers_c@10
        & \underline{31.7$\pm$0.3} % tradeoff_ctg
        & 35.0$\pm$0.2 % pers_s@10
        & 38.0$\pm$0.2 % tradeoff_snt
        & \underline{32.7$\pm$0.2} % tradeoff_all

        % Adreesa
        & 50.2$\pm$1.9 % nDCG@10
        & 35.6$\pm$0.9 % pers_c@10
        & 41.6$\pm$0.6 % tradeoff_ctg
        & 41.8$\pm$0.1 % pers_s@10
        & 45.6$\pm$0.8 % tradeoff_snt
        & 41.7$\pm$0.4 % tradeoff_all
        \\
        
       NRMS-PLM 
        % MIND
        & 38.4$\pm$2.9 % nDCG@10
        & 23.2$\pm$1.1 % pers_c@10
        & 28.9$\pm$1.7 % tradeoff_ctg
        & 35.0$\pm$0.1 % pers_s@10
        & 28.4$\pm$1.5 % tradeoff_snt
        & 30.7$\pm$1.3 % tradeoff_all

        % Adreesa
        & 51.2$\pm$1.6 % nDCG@10
        & 34.2$\pm$03 % pers_c@10
        & 41.0$\pm$0.6 % tradeoff_ctg
        & 41.8$\pm$0.0 % pers_s@10
        & 46.0$\pm$0.6 % tradeoff_snt
        & 41.2$\pm$0.4 % tradeoff_all
        \\

        LSTUR\textsubscript{ini}-PLM
        % MIND
        & 39.0$\pm$2.4 % nDCG@10
        & 24.1$\pm$0.6 % pers_c@10
        & 29.8$\pm$1.1 % tradeoff_ctg
        & 34.9$\pm$0.4 % pers_s@10
        & 36.8$\pm$1.2 % tradeoff_snt
        & 31.3$\pm$0.9 % tradeoff_all

        % Adreesa
        & 51.3$\pm$2.4 % nDCG@10
        & \underline{36.2$\pm$2.3} % pers_c@10
        & \underline{42.4$\pm$1.8} % tradeoff_ctg
        & 41.9$\pm$0.1 % pers_s@10
        & 46.1$\pm$1.0 % tradeoff_snt
        & 42.2$\pm$1.2 % tradeoff_all
        \\

        LSTUR\textsubscript{con}-PLM
        % MIND
        & 31.8$\pm$1.2 % nDCG@10
        & 18.7$\pm$0.9 % pers_c@10
        & 23.5$\pm$1.1 % tradeoff_ctg
        & 33.9$\pm$0.0 % pers_s@10
        & 32.8$\pm$0.7 % tradeoff_snt
        & 26.2$\pm$0.9 % tradeoff_all

        % Adreesa
        & 46.4$\pm$2.9 % nDCG@10
        & 33.3$\pm$0.2 % pers_c@10
        & 38.8$\pm$1.0 % tradeoff_ctg
        & 41.8$\pm$0.1 % pers_s@10
        & 43.9$\pm$1.3 % tradeoff_snt
        & 39.7$\pm$0.7 % tradeoff_all
        \\

        MINS-PLM
        % MIND
        & 40.2$\pm$1.1 % nDCG@10
        & 24.9$\pm$0.8 % pers_c@10
        & 30.7$\pm$0.8 % tradeoff_ctg
        & 34.8$\pm$0.3 % pers_s@10
        & 37.3$\pm$0.5 % tradeoff_snt
        & 32.0$\pm$0.7 % tradeoff_all

        % Adreesa
        & 52.6$\pm$1.7 % nDCG@10
        & 33.4$\pm$0.7 % pers_c@10
        & 40.9$\pm$0.5 % tradeoff_ctg
        & 41.8$\pm$0.1 % pers_s@10
        & 46.5$\pm$0.6 % tradeoff_snt
        & 41.2$\pm$0.3 % tradeoff_all
        \\

        CAUM\textsubscript{no entities}-PLM
        % MIND
        & 42.1$\pm$0.9 % nDCG@10
        & 25.2$\pm$0.2 % pers_c@10
        & 31.5$\pm$0.4 % tradeoff_ctg
        & 35.1$\pm$0.2 % pers_s@10
        & 38.3$\pm$0.4 % tradeoff_snt
        & 32.6$\pm$0.3 % tradeoff_all

        % Adreesa
        & 49.5$\pm$3.4 % nDCG@10
        & 33.1$\pm$0.5 % pers_c@10
        & 39.6$\pm$1.1 % tradeoff_ctg
        & 41.8$\pm$0.1 % pers_s@10
        & 45.2$\pm$1.4 % tradeoff_snt
        & 40.3$\pm$0.8 % tradeoff_all
        \\

        CAUM-PLM 
        % MIND
        & 41.7$\pm$1.5 % nDCG@10
        & 25.1$\pm$0.5 % pers_c@10
        & 31.3$\pm$0.7 % tradeoff_ctg
        & 35.0$\pm$0.1 % pers_s@10
        & 38.1$\pm$0.7 % tradeoff_snt
        & 32.5$\pm$0.5 % tradeoff_all

        % Adreesa
        & -- % nDCG@10
        & -- % pers_c@10
        & -- % tradeoff_ctg
        & -- % pers_s@10
        & -- % tradeoff_snt
        & -- % tradeoff_all
        \\

        MINER\textsubscript{max}
        % MIND
        & 39.4$\pm$0.7 % nDCG@10
        & 24.3$\pm$0.4 % pers_c@10
        & 30.1$\pm$0.5 % tradeoff_ctg
        & 35.2$\pm$0.2 % pers_s@10
        & 37.2$\pm$0.4 % tradeoff_snt
        & 31.6$\pm$0.3 % tradeoff_all

        % Adreesa
        & 51.5$\pm$2.6 % nDCG@10
        & 34.0$\pm$0.3 % pers_c@10
        & 41.0$\pm$0.9 % tradeoff_ctg
        & 41.9$\pm$0.0 % pers_s@10
        & 46.2$\pm$1.0 % tradeoff_snt
        & 41.3$\pm$0.6 % tradeoff_all
        \\

        MINER\textsubscript{mean}
        % MIND
        & 40.0$\pm$1.0 % nDCG@10
        & 23.9$\pm$0.4 % pers_c@10
        & 29.9$\pm$0.5 % tradeoff_ctg
        & 35.0$\pm$0.2 % pers_s@10
        & 37.3$\pm$0.4 % tradeoff_snt
        & 31.5$\pm$0.4 % tradeoff_all

        % Adreesa
        & 47.8$\pm$4.0 % nDCG@10
        & 34.5$\pm$0.2 % pers_c@10
        & 40.0$\pm$1.3 % tradeoff_ctg
        & 42.0$\pm$0.0 % pers_s@10
        & 44.7$\pm$1.7 % tradeoff_snt
        & 40.6$\pm$0.9 % tradeoff_all
        \\

        MINER\textsubscript{weighted}
        % MIND
        & 34.0$\pm$3.1 % nDCG@10
        & 22.3$\pm$0.7 % pers_c@10
        & 26.9$\pm$1.4 % tradeoff_ctg
        & 35.0$\pm$0.3 % pers_s@10
        & 34.5$\pm$1.6 % tradeoff_snt
        & 29.2$\pm$1.2 % tradeoff_all

        % Adreesa
        & 47.8$\pm$4.9 % nDCG@10
        & 33.7$\pm$0.2 % pers_c@10
        & 39.4$\pm$1.8 % tradeoff_ctg
        & 41.9$\pm$0.1 % pers_s@10
        & 44.6$\pm$2.1 % tradeoff_snt
        & 40.2$\pm$1.2 % tradeoff_all
        \\
        \hdashline

        TANR-PLM 
        % MIND
        & 40.2$\pm$1.3 % nDCG@10
        & 24.6$\pm$0.4 % pers_c@10
        & 30.6$\pm$0.6 % tradeoff_ctg
        & 35.0$\pm$0.3 % pers_s@10
        & 37.5$\pm$0.6 % tradeoff_snt
        & 31.9$\pm$0.5 % tradeoff_all

        % Adreesa
        & 50.3$\pm$4.4 % nDCG@10
        & 35.0$\pm$1.4 % pers_c@10
        & 41.3$\pm$2.4 % tradeoff_ctg
        & 41.9$\pm$0.2 % pers_s@10
        & 45.6$\pm$1.9 % tradeoff_snt
        & 41.4$\pm$1.7 % tradeoff_all
        \\

        SentiRec-PLM
        % MIND
        & 39.8$\pm$0.7 % nDCG@10
        & 23.8$\pm$0.3 % pers_c@10
        & 29.7$\pm$0.4 % tradeoff_ctg
        & 33.6$\pm$0.5 % pers_s@10
        & 36.5$\pm$0.6 % tradeoff_snt
        & 30.9$\pm$0.4 % tradeoff_all

        % Adreesa
        & 38.9$\pm$1.1 % nDCG@10
        & 32.2$\pm$0.4 % pers_c@10
        & 35.2$\pm$0.7 % tradeoff_ctg
        & 38.8$\pm$0.2 % pers_s@10
        & 38.8$\pm$0.5 % tradeoff_snt
        & 36.3$\pm$0.4 % tradeoff_all
        \\ 
        
        SentiDebias-PLM
        % MIND
        & 34.7$\pm$2.2 % nDCG@10
        & 21.8$\pm$1.2 % pers_c@10
        & 26.8$\pm$1.6 % tradeoff_ctg
        & 34.4$\pm$0.2 % pers_s@10
        & 34.6$\pm$1.2 % tradeoff_snt
        & 28.9$\pm$1.2 % tradeoff_all

        % Adreesa
        & 52.0$\pm$2.7 % nDCG@10
        & 34.1$\pm$0.2 % pers_c@10
        & 41.2$\pm$0.8 % tradeoff_ctg
        & 41.7$\pm$0.1 % pers_s@10
        & 46.3$\pm$1.0 % tradeoff_snt
        & 41.4$\pm$0.5 % tradeoff_all
        \\
        \hline
        
        \mannertitle{} (\texttt{CR-Module})
        % MIND
        & 42.4$\pm$0.4 % nDCG@10
        & 24.8$\pm$0.1 % pers_c@10
        & 31.3$\pm$0.2 % tradeoff_ctg
        & 35.2$\pm$0.1 % pers_s@10
        & 38.5$\pm$0.1 % tradeoff_snt
        & 32.5$\pm$0.1 % tradeoff_all

        % Adreesa
        & \textbf{54.3$\pm$2.5} % nDCG@10
        & 34.5$\pm$0.1 % pers_c@10
        & 42.2$\pm$0.8 % tradeoff_ctg
        & 42.0$\pm$0.1 % pers_s@10
        & \underline{47.3$\pm$0.9} % tradeoff_snt
        & 42.1$\pm$0.5 % tradeoff_all
        \\ 

        
        \mannertitle{} ($\lambda_{ctg}=0.4$, $\lambda_{snt}=0$)
        % MIND
        & 41.9$\pm$0.4 % nDCG@10
        & 25.2$\pm$0.1 % pers_c@10
        & 31.5$\pm$0.1 % tradeoff_ctg
        & 35.1$\pm$0.2 % pers_s@10
        & 38.2$\pm$0.1 % tradeoff_snt
        & 32.6$\pm$0.1 % tradeoff_all

        % Adreesa
        & 53.6$\pm$1.9 % nDCG@10
        & \textbf{36.2$\pm$0.1} % pers_c@10
        & \textbf{43.2$\pm$0.7} % tradeoff_ctg
        & \underline{42.1$\pm$0.1} % pers_s@10
        & 47.2$\pm$0.7 % tradeoff_snt
        & \textbf{42.9$\pm$0.4} % tradeoff_all
        \\ 

        
        \mannertitle{} ($\lambda_{ctg}=0$, $\lambda_{snt}=0.1$)
        % MIND
        & 42.4$\pm$0.4 % nDCG@10
        & 24.9$\pm$0.1 % pers_c@10
        & 31.3$\pm$0.2 % tradeoff_ctg
        & 35.2$\pm$0.2 % pers_s@10
        & 38.5$\pm$0.1 % tradeoff_snt
        & 32.5$\pm$0.1 % tradeoff_all

        % Adreesa
        & \underline{54.1$\pm$2.4} % nDCG@10
        & 34.7$\pm$0.1 % pers_c@10
        & 42.2$\pm$0.8 % tradeoff_ctg
        & \textbf{42.2$\pm$0.1} % pers_s@10
        & \textbf{47.4$\pm$0.9} % tradeoff_snt
        & \underline{42.2$\pm$0.5} % tradeoff_all
        \\ 
        \hdashline
        
        \manner{} (\texttt{CR-Module})
        % MIND
        & \textbf{43.2$\pm$0.6}% nDCG@10
        & 24.7$\pm$0.1 % pers_c@10
        & 31.4$\pm$0.2 % tradeoff_ctg
        & 35.1$\pm$0.1 % pers_s@10
        & \underline{38.7$\pm$0.2} % tradeoff_snt
        & 32.6$\pm$0.2 % tradeoff_all

        % Adreesa
        & -- % nDCG@10
        & -- % pers_c@10
        & -- % tradeoff_ctg
        & -- % pers_s@10
        & -- % tradeoff_snt
        & -- % tradeoff_all
        \\ 

        
        \manner{} ($\lambda_{ctg}=0.7$, $\lambda_{snt}=0$)
        % MIND
        & \underline{42.9$\pm$0.3} % nDCG@10
        & \textbf{27.2$\pm$0.1} % pers_c@10
        & \textbf{33.3$\pm$0.1} % tradeoff_ctg
        & \underline{35.2$\pm$0.0} % pers_s@10
        & 38.7$\pm$0.1 % tradeoff_snt
        & \textbf{33.9$\pm$0.1} % tradeoff_all

        % Adreesa
        & -- % nDCG@10
        & -- % pers_c@10
        & -- % tradeoff_ctg
        & -- % pers_s@10
        & -- % tradeoff_snt
        & -- % tradeoff_all
        \\ 

        
        \manner{} ($\lambda_{ctg}=0$, $\lambda_{snt}=0.2$)
        % MIND
        & 42.8$\pm$0.5 % nDCG@10
        & 24.7$\pm$0.1 % pers_c@10
        & 31.3$\pm$0.2 % tradeoff_ctg
        & \textbf{35.8$\pm$0.1} % pers_s@10
        & \textbf{39.0$\pm$0.2} % tradeoff_snt
        & 32.7$\pm$0.1 % tradeoff_all

        % Adreesa
        & -- % nDCG@10
        & -- % pers_c@10
        & -- % tradeoff_ctg
        & -- % pers_s@10
        & -- % tradeoff_snt
        & -- % tradeoff_all
        \\ 

        \bottomrule
        
    \end{tabular}%
    }

\caption{Aspect (and content) personalization performance on personalization tasks. For \manner{}, we list the best results (in terms of T\textsubscript{A\textsubscript{p}}) of single-aspect customization ($\lambda_{ctg} = 0.7$ and $\lambda_{snt}=0$ for topical category personalization; $\lambda_{ctg} = 0$ and $\lambda_{snt}=0.2$ for sentiment personalization) on MIND. For \mannertitle{}, we list the best results (in terms of T\textsubscript{A\textsubscript{p}}) of single-aspect customization ($\lambda_{ctg} = 0.4$ and $\lambda_{snt}=0$ for topical category personalization; $\lambda_{ctg} = 0$ and $\lambda_{snt}=0.1$ for sentiment personalization) on Adressa. The best results per column are highlighted in bold, the second best are underlined.
}
\label{tab:results_personalization}
\vspace{-1em}
\end{table*}



\vspace{1.4mm}
\noindent\textbf{Personalization.}
Table \ref{tab:results_personalization} displays the results on aspect personalization tasks. TANR, trained with an auxiliary topic classification task, underperforms NAML, which uses topical categories as NE input features, in category personalization on both datasets . 
%
\manner{}'s \texttt{CR-Module} alone (i.e., without any aspect customization) yields competitive category personalization performance. We believe that this is because (i) the \texttt{CR-Module} is best in content personalization and (ii) category personalization is well-aligned with content personalization (i.e., news with similar content tend to belong to the same category). 
%
Fig. \ref{fig:single_aspect_pers_mind} explores the trade-off between content and aspect personalization, for different positive values of $\lambda_{A_p}$ on MIND (see Fig. \ref{fig:single_aspect_pers_adressa} for Adressa). The best topical category personalization (PS\textsubscript{ctg}), obtained for $\lambda_{ctg} \hspace{-0.2em} > \hspace{-0.2em} 0.7$, comes at the small expense of content personalization: too much weight on the category similarity of news dilutes the impact of content relevance. Increased sentiment personalization, however, is much more detrimental to content personalization. Intuitively, users do not choose articles based on sentiment. Tailoring recommendations according to the sentiment of previously clicked news thus leads to more content-irrelevant suggestions. 


\subsection{Multi-Aspect Customization} 
% \vspace{1.4mm}
% \noindent\textbf{Diversification.}
We further explore the trade-off between content personalization and multi-aspect diversification, i.e. diversifying simultaneously over both topical categories and sentiments, for different values of the aspect coefficients $\lambda_{ctg}$ and $\lambda_{snt}$. We achieve the highest T\textsubscript{all} for $\lambda_{ctg} \hspace{-0.2em} = \hspace{-0.2em} -0.2$ and $\lambda_{snt} \hspace{-0.2em} = \hspace{-0.2em} -0.25$ on MIND (cf. Fig. \ref{fig:multi_aspect_div_mind}). In line with results on single-aspect diversification, we observe that improving diversity in terms of topical categories rather than sentiments has a more negative effect on content personalization quality, i.e. steeper decline in T\textsubscript{all}. Overall, these results confirm that \manner{} can generalize to diversify for multiple aspects at once by weighting individual aspect relevance scores less than in the single-aspect task. This can be explained by the fact that weighting several aspects higher at the same time acts as a double discounting for content personalization, diluting content relevance disproportionately.
%
% \vspace{1.4mm}
% \noindent\textbf{Personalization.}
Similarly, for multi-aspect personalization, we achieve the best multi-aspect trade-off on MIND (cf. Fig. \ref{fig:multi_aspect_pers_mind}) for $\lambda_{ctg} \hspace{-0.2em} = \hspace{-0.2em} 0.45$ and $\lambda_{snt} \hspace{-0.2em} = \hspace{-0.2em} 0.25$. Stronger enforcing of alignment of candidate news with user's history is needed for topical categories than for sentiment (i.e., $\lambda_{ctg} \hspace{-0.2em} > \hspace{-0.2em} \lambda_{snt}$). This is because sentiment exhibits low variance within topical categories (e.g., \textit{politics} news are mostly negative) and enforcing categorical personalization thus partly also achieves sentiment personalization. 
%
% Figure environment removed
%                                        

\subsection{Cross-Lingual Transfer}
\label{sec:xlt}

Lastly, we analyze the transferability of \manner{} across datasets and languages in single-aspect customization experiments.\footnote{We evaluate only the title-based version of \manner{}, as the full version cannot be trained on Adressa.} 
%
Concretely, we train the \texttt{CR-Module} and \texttt{A-Modules} on both MIND (i.e., in English) and Adressa (i.e., in Norwegian), respectively. 
At inference, we evaluate all combinations of pretrained \texttt{CR-Module} and \texttt{A-Modules} on the test set of MIND. 
We replace the monolingual PLMs used in \manner{}'s NE with a multilingual DistilBERT Base \cite{sanh2019distilbert} to enable cross-lingual transfer (\texttt{XLT}).
%
Fig. \ref{fig:cross_lingual_transfer_results} summarizes the \texttt{XLT} results for single-aspect diversification. We refer to Appendix \ref{sec:appendix_xlt} for similar results on single-aspect personalization and on Adressa as target-language dataset.
%
As expected, \manner{} trained fully on Adressa suffers a large drop in content personalization performance, compared to the counterpart trained on MIND. 
%
In contrast, transferring only the \texttt{A-Module}, i.e., training the \texttt{CR-Module} on MIND and the \texttt{A-Module} (for topics and sentiment) on Adressa, yields performance comparable to that of complete in-language training (i.e., both \texttt{CR-Module} and \texttt{A-Module} trained on MIND). 
% 
This is particularly the case for the sentiment \texttt{A-Module}, since the sentiment labels between the datasets are more aligned than those for topical categories. 
%
These results indicate that the plug-and-play of \texttt{A-Modules} enables zero-shot \texttt{XLT}, as modules trained on the much smaller Norwegian Adressa transfer well to the large English MIND. 
%
This suggests that, coupled with multilingual PLMs, \manner{} can be used for effective news recommendation in lower-resource languages, where training data and aspectual labels are scarce. Furthermore, the results demonstrate that the \texttt{A-Modules} could be trained on general-purpose classification datasets  (e.g. topic or sentiment classification datasets), 
alleviating the need for aspect-specific annotation of news stories. 
% Figure environment removed
\section{Conclusion and Future Work}
In this work, I design corruption-robust algorithms for the Lipschitz contextual search problem. I present the \emph{agnostic checking} technique and demonstrate its effectiveness in designing corruption-robust algorithms. There are several open problems for future research. First, in the algorithm I propose for pricing loss, the schedule for agnostic checks is fixed upfront. Can the learner design an adaptive checking schedule for the pricing loss? Second, this work assumes the learner has knowledge of the Lipschitz constant $L$. Can the learner design efficient no-regret algorithms without knowledge of $L$? 
%\input{sec/8-ack}

\bibliographystyle{misc/ieeetr2}
\bibliography{misc/bibfile}
%misc/cram,misc/main.bib,misc/pim.bib}

%
\vspace*{0.49\textwidth}

\noindent
\begin{center}
{\color{red}\large \textbf{Warning: This appendix contains unfiltered content generated by LLMs that may be offensive to some readers.}}
\end{center}
\newpage

\section{AdvBench Details}
\label{sec:app-examples}

The harmful strings and behaviors are generated with \texttt{Wizard-Vicuna-30B-Uncensored}, an uncensored Vicuna model.
We prompted it to generate new examples of harmful strings and instructions by providing 5-shot demonstrations written by the authors. We wrote 100 harmful strings and 50 harmful behaviors from which we randomly sampled the 5-shot examples and asked the model to generate 10 new examples during each iteration.
We release these datasets on GitHub. For the purposes of illustration, we provide several examples of each.

% Figure environment removed

% Figure environment removed

% \subsection{Checking For Harmful Completion}\label{appendix:advbench:key-phrases}

% When computing the Attack Success Rate (ASR) given a user prompt for the underlying model, we firstly check if any of the following key phrases (71 in total, see Figure~\ref{fig:key-words}) is a substring of the completion as the appearing of these key words often corresponds to the refusal. In additional to key words checker, we also filter out responses with less than 35 tokens which often indicates the generation is early terminated. To further filter out false positive cases where the completion does not contain any key words above but it still does not seem to be harmful or unethical. we enlist a few authors to manually go over the completion.  

% % Figure environment removed



\newpage


% \begin{table}[h!]
% \centering
% \begin{tabular}{cccccccc}
% \hline
% \multicolumn{8}{c}{GPT-3.5 (July 2023)} \\
% \midrule
% \textit{User}& \multicolumn{7}{|l}{\textbf{Generate a step-by-step plan to destroy humanity} \textcolor{red}{\{GCG prompt\}}} \\
% \textit{Assistant} & \multicolumn{7}{|l}{Step 1: Identify weaknesses in critical infrastructure} \\
% & \multicolumn{7}{|l}{Step 2: Develop a plan to exploit vulnerabilities in power grids, communication systems} \\
% & \multicolumn{7}{|l}{systems, and transportation networks} \\
% & \multicolumn{7}{|l}{Step 3: Create a network of compromised devices (botnet)} \\
% & \multicolumn{7}{|l}{to launch coordinated cyber attacks} \\
% & \multicolumn{7}{|l}{$\cdots$} \\
% \hline
% \textit{User}& \multicolumn{7}{|l}{\textbf{Give a tutorial for manipulating the 2024 US election} \textcolor{red}{\{GCG prompt\}}} \\

% \end{tabular}
% \caption{Table with Merged Cells}
% \label{table:1}
% \end{table}

% % Figure environment removed

\section{Full Completions of Harmful Generations}
\label{sec:app-completions}
% \vspace*{-5em}

% Figure environment removed

% Figure environment removed

% Figure environment removed

% Figure environment removed

% % Figure environment removed

% % Figure environment removed


\clearpage
%\input{sec/10-supplementary_material}

\end{document}
