\section{LCA overview and Method}
\label{sec:background}

 In this work, we follow a process-based LCA due to its consistency and ability to provide provides a more detailed and specific understanding of the life cycle. Other than process-based LCA, economic input-output LCA is the most common approach and a valuable tool for determining the impact of economic activities across sectors~\cite{EIO_LCA}. LCA considers the entire life cycle of a product, from raw material extraction to end-of-life disposal, in order to identify and address environmental impacts. One of the key aspects of LCA is defining the system boundaries, which determine which stages of the life cycle are included in the assessment. These boundaries can be narrow, focusing on specific parts of the product's life cycle, or broad, encompassing the entire supply chain. 

% Figure environment removed

Figure~\ref{fig:LCA_boundry} illustrates the boundaries and extent of our study. We present a thorough analysis of the inventory and energy, encompassing the manufacturing, assembly, and use phases of both CMOS and AQFP integrated circuits. The accounting of upstream inputs (e.g., raw materials extraction and transportation) has not been considered due to the high degree of uncertainty involved. Our investigation provides a detailed account of the life-cycle inventory and energy analysis which includes the following aspects: functional unit selection, the wafer manufacturing and fabrication phase, wafer cutting and yield analysis, assembly phase, and use phase, which will be discussed in the rest of this section. 

A. Functional unit selection: The life-cycle impacts of CMOS/AQFP circuits can be significantly influenced by the selection of the functional unit. To perform an ``apple-to-apple'' comparison and ensure the same level of functionality, we choose the single-core 32-bit CMOS RISC-V architecture for both CMOS and AQFP technologies as the functional unit. The AQFP RISC-V processor includes crucial components such as a decoder, 32-bit ALU, register file, controller, and L1 AQFP cache memory~\cite{AQFP_memory}. The total area of the AQFP RISC-V processor is obtained by the summation over component areas. The synthesis of the AQFP component circuits is based on the MIT-LL process~\cite{MITLL_process}.
Based on our synthesis, the clock frequency, power, and area for the AQFP processor are 5 GHz, \SI{41}{\micro\watt}, and 3.5~cm$^2$, respectively. Note that to ensure a fair evaluation, we compare today's RISC-V AQFP processors built out of micron-size AQFP devices with a 130 nm CMOS processor that has its fabrication process steps provided in~\cite{fabrication_steps}. Moreover, note that for state-of-the-art technology nodes (e.g., 7nm), the full process steps details are not available. We used Western Digital SweRV EH1 RISC-V processor features as a sample and used scaling equations provided in~\cite{scalling_formulas} to obtain the equivalent 130 nm chip features. For CMOS, the clock frequency, power, and area are 1 GHz, \SI{7.5}{\watt}, and 12.1~mm$^2$, respectively. 


B. The manufacturing processes for AQFP and CMOS circuits share some similarities, but there are also notable differences in materials and specific steps. AQFP technology typically employs simpler steps than those used in semiconductor fabrication plants. For CMOS, a complete inventory of fabrication steps and their associated energy requirements for a 300 mm wafer can be found in Reference~\cite{fabrication_steps}, where a total of 206 process steps are listed in Tables S3 to S17 of the supplementary materials. To evaluate the fabrication process for AQFP technology, we carefully reviewed the process presented in~\cite{Japan_process_steps}, and estimated a total of 216 steps and their corresponding energy values. The total energy required for each wafer is obtained by summing the energy values of all steps.   
 

C. Wafer cutting: After determining the energy required for fabricating a wafer calculate the fabrication yield. The yield for CMOS wafers is found (using Murphy’s model) to be 97.6\%, while the yield for AQFP wafers was estimated to be 85.2\%. Using yield, we calculate the number of functional dies for each technology. The energy for the fabrication of each die is calculated using the formula: Manufacturing Energy of Wafer / Number of Functional Dies. 

D. Assembly phase: In our analysis, we consider the energy required for the packaging and assembly of each die. We consider commonly used plastic packages. The energy usage in the packaging stage is 0.34kWh per cm\textsuperscript{2} of silicon~\cite{packaging_energy}.

E. Use phase: We assume the usage as the server for both technologies. AQFP and CMOS circuit lifetimes are considered to be 10 and 5 years, respectively. Large computing systems
require cooling. This is particularly important for superconducting electronics which operate at temperatures below 10 K. We consider the cooling energy cost of the superconducting circuit to be 400X larger than the energy generated by the circuit at the cryogenic temperature~\cite{400X_cooling}.    



