\section{Results and Discussions}
\label{sec:ir-drop-similarity}


Table~\ref{table:comparison} presents a concise summary of the life-cycle energy consumption results for AQFP and CMOS technologies. Notably, the cooling cost of AQFP circuits is taken into account in our report. In the case of CMOS technology, the dominant energy component is the use of phase energy, which is considerably greater than manufacturing and assembly energies. However, for AQFP technology, manufacturing and assembly energies are significantly higher than the use phase energy.

A comparison of the energy components of the two technologies reveals that the manufacturing and assembly energies for AQFP processors are respectively 9.5X and 14.8X larger than those for a CMOS processor. This disparity can be attributed to the fact that the AQFP chip has a 28.9X larger area than that of CMOS. On the other hand, AQFP technology is significantly more energy-efficient than CMOS technology during the use phase, even when the cooling effect is taken into account. Without cooling cost, AQFP technology is six orders of magnitude more energy-efficient than CMOS during the use phase, and with cooling cost, it is four orders of magnitude more energy-efficient. The AQFP technology's overall superiority over CMOS technology is primarily due to its extreme energy efficiency during the use phase, which significantly outperforms the CMOS counterpart. Overall, considering all phases, AQFP processor is 237X and 205X more energy-efficient than CMOS processor in with and without considering the cooling effect, respectively. . 

In comparison to CMOC, AQFP technology is still in its nascent stages. It's important to note that for this assessment, we utilized today's micron-size AQFP technology against 130 nm CMOS technology node. However, Our investigation suggests that reducing the size of AQFP chips in the near future is inevitable, for example, due to advancements in AQFP device technology and more efficient, area-aware routing algorithms. Even a small downscaling (e.g., 2X) of the AQFP chip area can result in a significant improvement in fabrication yield and overall life cycle energy.  Our findings indicate that a 2X downscaling of AQFP chip area can improve wafer yield from 85.2\% in today's AQFP technology to 92.1\% in the near future AQFP technology. Moreover, we anticipate that the combined manufacturing and assembly energy requirements will decrease by 52\%, from a total of 2.8 KWh in today's AQFP technology to 1.3 KWh in the near future AQFP technology. This is expected to result in an overall improvement (compared to CMOS) in life cycle energy efficiencies of 498X and 378X without and with cooling costs, respectively.





%%%% The comparison table including RRAM technology
\ignore{
\begin{table*}[!ht]
\renewcommand{\arraystretch}{1.3}
\caption{Comparison of Energy Efficiency Between CMOS and AQFP RISC-V Processors for General Purpose Computing and RRAM/AQFP Crossbars for Special Purpose Computing.}
\label{table:comparison}
\centering
\begin{tabular}{c|c||c|c|c|c||c}
\hline
Application & Technology & Manufacturing Energy & Assembly Energy & Use Phase Energy & Total Energy & Improvement \\ 
\hline\hline
 \raisebox{-2ex} {GPC} & CMOS CPU & 0.79 KWh & 0.34 KWh & 722.70 KWh & 1,447.67 KWh &  \\ 
\cline{2-6}
 &\raisebox{1ex}{AQFP RISC-V } & \raisebox{1ex}{5.94 KWh} & \raisebox{1ex}{1.67 KWh} & \shortstack{\\[0.1pt]0.034 KWh\\(with cooling 3.48 KWh)} & \shortstack{7.61 KWh\\(with cooling 11.06 KWh)} & \raisebox{2ex} {\shortstack{190X\\(with cooling 130X)}} \\ 
\hline\hline
\raisebox{-2ex} {SPC} & RRAM crossbar & 7.05e-5 KWh & 3.48e-5 KWh &  25.68 KWh & 51.37 KWh &  \\ 
\cline{2-6}
 & \raisebox{1ex}{AQFP crossbar} & \raisebox{1ex}{2.91 KWh} & \raisebox{1ex}{ 1.43 KWh} & \shortstack{\\[0.1pt]5.60e-4 KWh\\(with cooling 0.22 KWh)} & \shortstack{4.35 KWh\\(with cooling 4.57 KWh)} & \raisebox{2ex} {\shortstack{12 X\\(with cooling 11X)}} \\ 
\hline

\end{tabular}
\end{table*}
}



\noindent