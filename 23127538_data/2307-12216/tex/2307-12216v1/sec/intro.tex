\section{Introduction} \label{sec:intro}
%%% Rewriting %%%%
%\noindent

Compared to traditional CMOS technology, superconductive electronic circuits (and particularly AQFP-based logic) have advantages such as significantly lower power consumption~\cite{superconductive_advantage}. It is forecasted that AQFP-based logic can reach tens of thousands of times energy-per-operation advantages over state-of-the-art CMOS counterparts. As a result, AQFP superconductive electronics have the potential to revolutionize supercomputing systems by reducing energy consumption and improving performance~\cite{Olivia_paper}. 


Evaluating the resources and energy of emerging circuits during manufacturing, assembly, and use phases is a vital part of environmental impact mitigation, and LCA is increasingly used for this purpose. While LCA for CMOS circuits has been thoroughly researched~\cite{nature_LCA}, there is still much to be learned about the environmental impact of AQFP circuits. Performing a LCA for both CMOS and AQFP circuits is crucial to determine which technology is superior overall and to ensure the industry's sustainable and efficient use of resources. 




This paper presents a comprehensive life-cycle energy and inventory analysis of AQFP superconductive integrated circuits, which are compared to CMOS technology circuits. The analysis encompasses the manufacturing, assembly, and use phases, utilizing 32-bit processors based on RISC-V architecture for both AQFP and CMOS circuits as the functional unit. The contribution of the paper is multifold: (i) presenting the first LCA of the AQFP circuit and including key parameters such as cooling cost in the LCA flow, (ii) analyzing the manufacturing steps of the AQFP wafer and providing comparisons with CMOS counterparts, (iii) yield analysis for both technologies, and examining the manufacturing, assembly, and use phase energies of the RISC-V AQFP-based processor and its CMOS counterpart, (v) investigation of the downscaling effect of the AQFP chip area on the LCA.    


This paper is structured as follows: a description of the LCA method is provided in Section II, followed by a discussion of the results for life-cycle energy and inventory analysis of AQFP integrated circuits in Section III. The conclusion is presented in Section IV.


%%%% The comparison table
\begin{table*}[!ht]
\renewcommand{\arraystretch}{1.3}
\caption{Comparison of CMOS and AQFP RISC-V Processors}
\label{table:comparison}
\centering
\begin{tabular}{c||c|c|c|c||c}
\hline
Processor & Manufacturing Energy & Assembly Energy & Use Phase Energy & Total Energy & Overall Improvement \\ 
\hline\hline
CMOS RISC-V & 0.17 KWh & 0.08 KWh & 665.23 KWh & 665.48 KWh & \\ 
\cline{1-5}
\raisebox{1ex}{AQFP RISC-V} & \raisebox{1ex}{1.61 KWh} & \raisebox{1ex}{1.19 KWh} & \shortstack{\\[0.1pt]0.001 KWh\\(with cooling 0.42 KWh)} & \shortstack{2.81 KWh\\(with cooling 3.23 KWh)} & \raisebox{2ex} {\shortstack{237X\\(with cooling 205X)}} \\ 
\hline
\end{tabular}
\end{table*}



