%% This document created by Scientific Word (R) Version 3.5

\documentclass[a4paper]{article}%
\usepackage{amsmath}
\usepackage{graphicx}
\usepackage{amsfonts}
\usepackage{amssymb}%
\setcounter{MaxMatrixCols}{30}
%TCIDATA{OutputFilter=latex2.dll}
%TCIDATA{Version=5.00.0.2606}
%TCIDATA{CSTFile=LaTeX article (bright).cst}
%TCIDATA{Created=Saturday, February 08, 2003 06:15:52}
%TCIDATA{LastRevised=Wednesday, July 26, 2023 10:01:57}
%TCIDATA{<META NAME="GraphicsSave" CONTENT="32">}
%TCIDATA{<META NAME="PrintViewPercent" CONTENT="100">}
%TCIDATA{<META NAME="SaveForMode" CONTENT="1">}
%TCIDATA{BibliographyScheme=Manual}
%TCIDATA{<META NAME="DocumentShell" CONTENT="Articles\SW\Standard LaTeX Article">}
%TCIDATA{Language=American English}
%TCIDATA{PageSetup=72,72,72,72,0}
%TCIDATA{Counters=arabic,1}
%TCIDATA{AllPages=
%H=36
%F=36
%}
\newtheorem{theorem}{Theorem}
\newtheorem{acknowledgement}[theorem]{Acknowledgement}
\newtheorem{algorithm}[theorem]{Algorithm}
\newtheorem{axiom}[theorem]{Axiom}
\newtheorem{case}[theorem]{Case}
\newtheorem{claim}[theorem]{Claim}
\newtheorem{conclusion}[theorem]{Conclusion}
\newtheorem{condition}[theorem]{Condition}
\newtheorem{conjecture}[theorem]{Conjecture}
\newtheorem{corollary}[theorem]{Corollary}
\newtheorem{criterion}[theorem]{Criterion}
\newtheorem{definition}[theorem]{Definition}
\newtheorem{example}[theorem]{Example}
\newtheorem{exercise}[theorem]{Exercise}
\newtheorem{lemma}[theorem]{Lemma}
\newtheorem{notation}[theorem]{Notation}
\newtheorem{problem}[theorem]{Problem}
\newtheorem{proposition}[theorem]{Proposition}
\newtheorem{remark}[theorem]{Remark}
\newtheorem{solution}[theorem]{Solution}
\newtheorem{summary}[theorem]{Summary}
\newenvironment{proof}[1][Proof]{\textbf{#1.} }{\ \rule{0.5em}{0.5em}}
\begin{document}

\title{Schubert calculus in Lie groups}
\author{Haibao Duan}
\date{}
\maketitle

\begin{abstract}
Let $G$ be a simple Lie group with a maximal torus $T$. Combining Schubert
calculus in the flag manifold $G/T$ with the Serre spectral sequence of the
fibration $G\rightarrow G/T$, we construct the integral cohomology ring
$H^{\ast}(G)$ uniformly for all simple $G$.

\begin{description}
\item[2000 Mathematical Subject Classification: ] 14M15; 55T10

\item[Key words and phrases:] Lie groups; Cohomology, Schubert calculus

\end{description}
\end{abstract}

\section{Introduction}

The problem of computing the cohomology of Lie groups was raised by E. Cartan
in 1929 \cite{C,K}. It is a focus of algebraic topology owing to the
fundamental role of Lie groups playing in geometry and topology \cite{D,MT,Sa}%
. However, despite achievements by many mathematicians in about one century,
notably \cite{B2,B3,BC,Br,G,H,K,M2,Po}, the integral cohomology of Lie groups
remains incomplete. On the other hand, Schubert calculus is the enumerative
geometry of 19th century. Making this calculus rigorous was stated by Hilbert
as his 15th problem. In the course of securing the foundation of algebraic
geometry, Van der Waerden \cite{Wa} and Weil \cite[p.331]{W} related the
problem to the determination of the intersection theory of flag manifolds, see
\cite{DZ3,DZ4} for surveys on the history. In this paper we bring a connection
between these two topics with distinguished backgrounds, and illustrate how
Schubert calculus in flag manifolds can be extended to obtain a unified
construction of the integral cohomology rings of simple Lie groups, see
Theorem B in Section \S 1.3, and Remark 5.4.

According to Cartan's classification on compact Lie groups, the
simply-connected simple Lie groups fall into three infinite families of
classical Lie groups $SU(n)$, $Spin(n),Sp(n)$, as well as the five exceptional
ones $G_{2},F_{4},E_{6},E_{7},E_{8}$. In this paper we assume that $G$ is one
of these Lie groups, and that the cohomology is over the ring $\mathbb{Z}$ of
integers, unless otherwise stated.

\subsection{Preliminaries in Schubert calculus}

Fixing a maximal torus $T$ on $G$ the inclusion $T\subset G$ induces the fiber sequence

\begin{enumerate}
\item[(1.1)] $T\rightarrow G\overset{\pi}{\rightarrow}G/T\overset
{c}{\rightarrow}B_{T}$,
\end{enumerate}

\noindent where $B_{T}$ is the classifying space of the group $T$, and where
the quotient space $G/T$ is canonically a complex projective manifold, called
\textsl{the complete flag manifold} of $G$ \cite{DZ2,DZ3}. Suppose that the
rank of $G$ is $\dim T=n$, and let $\{\omega_{1},\cdots,\omega_{n}\}\subset
H^{2}(B_{T})$ be a set of fundamental dominant weights of the group $G$
\cite{BH}. Then the cohomology $H^{\ast}(B_{T})$ is isomorphic to the free
polynomial ring $\mathbb{Z}[\omega_{1},\cdots,\omega_{n}]$, while the induced
map of $c$ on the cohomology is the \textsl{Borel characteristic map} of $G$
\cite{De1,De2}

\begin{enumerate}
\item[(1.2)] $c^{\ast}:\mathbb{Z}[\omega_{1},\cdots,\omega_{n}]\rightarrow
H^{\ast}(G/T)$.
\end{enumerate}

\noindent Furthermore, since the ring $H^{\ast}(G/T)$ is torsion free
\cite{BS}, the second page of the Serre spectral sequence $\{E_{r}^{\ast,\ast
}(G),d_{r}\}$ of $\pi$ has the presentation

\begin{enumerate}
\item[(1.3)] $E_{2}^{\ast,\ast}(G)=H^{\ast}(G/T)\otimes H^{\ast}(T)$,
\end{enumerate}

\noindent on which the $d_{2}$-action has been determined in \cite[Corollary
3.1]{D1}.

\bigskip

\noindent\textbf{Lemma 1.1. }\textsl{There exists a basis }$\left\{
\lambda_{1},\cdots,\lambda_{n}\right\}  $\textsl{\ of }$H^{1}(T)$%
\textsl{\ such that }

\textsl{i)} $H^{\ast}(T)=$\textsl{\ }$\Lambda(\lambda_{1},\cdots,\lambda_{n}%
)$\textsl{;}

\textsl{ii) }$d_{2}(x\otimes1)=0$\textsl{, }$d_{2}(x\otimes\lambda
_{i})=c^{\ast}(\omega_{i})\cdot x\otimes1$\textsl{, }$1\leq i\leq n$\textsl{,}

\noindent\textsl{where }$x\in H^{\ast}(G/T)$\textsl{,} \textsl{and }%
$\Lambda(\lambda_{1},\cdots,\lambda_{n})$\textsl{ is the exterior algebra
generated by }$\lambda_{1},\cdots,\lambda_{n}.$\hfill$\square$\noindent

\bigskip

Our aim is to construct the cohomology $H^{\ast}(G)$ from $E_{2}^{\ast,\ast
}(G)$. To this end we need a concise description of the ring $H^{\ast}(G/T)$.
Early in the 1950's Borel and Chevalley computed the algebra $H^{\ast
}(G/T;\mathbb{R})$ with real coefficients. Precisely, assume that the Weyl
group of $G$ is $W\subset Aut(H^{2}(B_{T}))$, and let $\{q_{1},\cdots
,q_{n}\}\in\mathbb{R}[\omega_{1},\cdots,\omega_{n}]^{W}$ be a basic set of
Weyl invariants of $G$ over $\mathbb{R}$ \cite{K}. Let $l_{i}$ be the degree
of $q_{i}$ as a polynomial in $\omega_{1},\cdots,\omega_{n}$. We may assume
that $l_{1}\leq\cdots\leq l_{n}$ and set $q(G)$ to be the degree sequence
$\{l_{1},\cdots,l_{n}\}$. Borel and Chevalley \cite{B1,C} showed that

\bigskip

\noindent\textbf{Lemma 1.2.} \textsl{The map }$c^{\ast}$ \textsl{induces an
isomorphism of algebras}

\begin{enumerate}
\item[(1.4)] $H^{\ast}(G/T;\mathbb{R})\cong\mathbb{R}[\omega_{1},\cdots
,\omega_{n}]/\left\langle q_{1},\cdots,q_{n}\right\rangle $\textsl{,}
\end{enumerate}

\noindent\textsl{where} \textsl{the} \textsl{sequence }$q(G)$ \textsl{is an
invariant of }$G$\textsl{, which is listed below:}

\begin{center}
{\normalsize Table 1. The degree sequences }$q(G)$ {\normalsize of simple Lie
groups }${\normalsize G}$%

\begin{tabular}
[c]{l||l}\hline
$G$ & $q(G)=\{l_{1},\cdots,l_{n}\}$\\\hline\hline
$SU(n)$ & $\{2,3,\cdots,n\}$\\\hline
$Sp(n),Spin(2n+1)$ & $\{2,4,\cdots,2n\}$\\\hline
$Spin(2n)$ & $\{2,4,\cdots,2n-2,n\}$\\\hline
$G_{2}$ & $\{2,6\}$\\\hline
$F_{4}$ & $\{2,6,8,12\}$\\\hline
$E_{6}$ & $\{2,5,6,8,9,12\}$\\\hline
$E_{7}$ & $\{2,6,8,10,12,14,18\}$\\\hline
$E_{8}$ & $\{2,8,12,14,18,20,24,30\}$\\\hline
\end{tabular}
.$\square$
\end{center}

Turning to the integral cohomology $H^{\ast}(G/T)$ we resorts to the classical
Schubert calculus. According to Chevalley \cite{Ch2}, or
Bernstein-Gel'fand-Gel'fand \cite{BGG}, the flag manifold $G/T$ admits a
partition into the Schubert cells $S_{w}$, parameterized by the elements of
the Weyl group $W$,

\begin{enumerate}
\item[(1.5)] $G/T=\cup_{w\in W}S_{w},\quad\dim S_{w}=2\iota(w)$,
\end{enumerate}

\noindent where $\iota:W\rightarrow\mathbb{Z}$ is the length function on $W$
\cite{BGG}. Since only even dimensional cells are involved in the partition,
the set $\{[S_{w}],w\in W\}$ of fundamental classes forms a basis of the
homology $H_{\ast}(G/T)$. The co-cycle class $s_{w}\in H^{\ast}(G/T)$
Kronecker dual to the basis element $[S_{w}]$ (i.e. $\left\langle s_{w}%
,[S_{u}]\right\rangle =\delta_{w,u},~w,u\in W$) is called the \textsl{Schubert
class} associated to $w\in W$. Formula (1.5) implies the following result,
well-known as "the basis theorem of Schubert calculus" \cite{BGG,Ch2}.

\bigskip

\noindent\textbf{Lemma 1.3.} \textsl{The cohomology }$H^{\ast}(G/T)$
\textsl{is torsion free, concentrated in even degrees, and has} \textsl{a
basis\ consisting of the Schubert classes }$\{s_{w},w\in W\}$\textsl{.}

\textsl{Moreover, the Schubert basis of} $H^{2}(G/T)$ \textsl{is} $\{c^{\ast
}(\omega_{1}),\cdots,c^{\ast}(\omega_{n})\}$\textsl{\ (\cite{D3}).}%
\hfill$\square$\noindent

\bigskip

By the second assertion of Lemma 1.3 we may reserve the notion $\omega_{i}$
for $c^{\ast}(\omega_{i})\in H^{2}(G/T)$, $1\leq i\leq n$, and write
$\left\langle \omega_{1},\cdots,\omega_{n}\right\rangle _{c}$ to denote the
ideal of $H^{\ast}(G/T)$ generated by the $\omega_{i}$'s. Carrying on the
results of Lemma 1.3 we show that

\bigskip

\noindent\textbf{Lemma 1.4.} \textsl{For each Lie group} $G$ \textsl{there
exists a minimal set of Schubert classes }$y_{t_{1}},\cdots,y_{t_{k}}$
\textsl{on} $G/T$\textsl{ with} $\deg y_{t_{i}}=t_{i}$\textsl{ and}
$2<t_{1}\leq\cdots\leq t_{k}$\textsl{,} \textsl{which, together with}
\textsl{the weights} $\omega_{1},\cdots,\omega_{n}$\textsl{, generates}
\textsl{the ring} $H^{\ast}(G/T)$\textsl{\ multiplicatively.}

\textsl{In addition, for each }$y_{t_{i}}$ \textsl{there exists a pair of}
\textsl{integers }$p_{i}$\textsl{,} $r_{i}>1$\textsl{, such that the following
relations hold in }$H^{\ast}(G/T)$

\begin{enumerate}
\item[(1.6)] $p_{i}y_{t_{i}}$\textsl{,} $y_{t_{i}}^{r_{i}}\in\left\langle
\omega_{1},\cdots,\omega_{n}\right\rangle _{c}$\textsl{.}
\end{enumerate}

\noindent\textbf{Proof. }Let $H^{+}(G/T)\subset H^{\ast}(G/T)$ be the subring
consisting of the homogeneous elements in positive degrees, and let
$D(H^{\ast}(G/T)):=H^{+}(G/T)\cdot H^{+}(G/T)$ be the ideal of the
decomposable elements of $H^{+}(G/T)$. Since $H^{\ast}(G/T)$ has a basis
consisting of Schubert classes, there exist Schubert classes $s_{1}%
,\cdots,s_{m}$, ordered by $\deg s_{1}\leq\cdots\leq$ $\deg s_{m}$, that
correspond to a basis of the quotient group $H^{+}(G/T)/D(H^{\ast}(G/T))$. It
follows that $\left\{  s_{1},\cdots,s_{m}\right\}  $ is a minimal set of
generators of the ring $H^{\ast}(G/T)$, while the number $m$ is an invariant
of $G$.

By Lemma 1.3 the elements of $D(H^{\ast}(G/T))$ have degrees $\geq4$. In particular

\begin{quote}
$H^{2}(G/T)\subset H^{+}(G/T)/D(H^{\ast}(G/T))$.
\end{quote}

\noindent This implies by Lemma 1.3 that $m\geq n$, and that we can take
$s_{i}=\omega_{i}$, $1\leq i\leq n$. The proof of the first statement is
completed by taking $y_{t_{j}}:=s_{n+j}$, where $t_{j}=\deg s_{n+j}$, $1\leq
j\leq k=m-n$.

By Lemma 1.2 the map $c^{\ast}$ surjects over $\mathbb{R}$. It implies for any
$w\in W$ that there exist positive integers $p_{w}$ and $r_{w}$ so that

\begin{quote}
$p_{w}s_{w}$, $s_{w}^{r_{w}}\in\left\langle \omega_{1},\cdots,\omega
_{n}\right\rangle _{c}$.
\end{quote}

\noindent It imply that $s_{w}\equiv0\operatorname{mod}D(H^{\ast}(G/T))$ if
either $p_{w}=1$ or $r_{w}=1$. We obtain (1.6) from $y_{t_{i}}\neq
0\operatorname{mod}D(H^{\ast}(G/T))$.\hfill$\square$\noindent

\bigskip

\noindent\textbf{Definition 1.5.} Following \cite{DZ2} we call a set of
Schubert classes $\left\{  y_{t_{1}},\cdots,y_{t_{k}}\right\}  $ satisfying
Lemma 1.4 \textsl{a set of special Schubert classes} on $G/T$.

For a special Schubert class $y_{t_{i}}$ denote by $tor(y_{t_{i}})$ and
$cl(y_{t_{i}})$ the least multiple $p_{i}$ and the least power $r_{i}$
satisfying the relations (1.6), to be called \textsl{the torsion index} and
\textsl{the cup length} of $y_{t_{i}}$, respectively.\hfill$\square$\noindent

\bigskip

\noindent\textbf{Example 1.6. }In the light of the algorithm indicated by the
proof of Lemma 1.4 and in the context of Schubert calculus, we have selected
in \cite{DZ2} for every Lie group $G$ a set of special Schubert classes
$\left\{  y_{t_{1}},\cdots,y_{t_{k}}\right\}  $ on $G/T$, whose degrees
sequence $\left\{  t_{1},\cdots,t_{k}\right\}  $, as well as the indices
$tor(y_{t_{i}})$ and lengths $cl(y_{t_{i}})$, are recorded in Table 2 below.
As we will see in Theorem 4.1 that, these numbers are independent of the
choice of a set of special Schubert classes on $G/T$, hence are in fact
further invariants of the group $G$, in addition to $q(G)$.

\begin{center}
{\normalsize Table 2. The degrees, torsion indices and cup lengths of the
special Schubert classes on }${\normalsize G/T}$

{\footnotesize
\begin{tabular}
[c]{l||l|l|l|l}\hline
$G$ & $SU(n)$ & $Sp(n)$ & $Spin(2n)$ & $Spin(2n+1)$\\\hline\hline
$k$ & $0$ & $0$ & $[\frac{n-2}{2}]$ & $[\frac{n-1}{2}]$\\\hline
$\{t_{1},\cdots,t_{k}\}$ &  &  & $\{4i+2,1\leq i\leq\lbrack\frac{n-2}{2}]\}$ &
$\{4i+2,1\leq i\leq\lbrack\frac{n-1}{2}]\}$\\\hline
$\{tor(y_{t_{i}})\}$ &  &  & $\{2,\cdots,2\}$ & $\{2,\cdots,2\}$\\\hline
$\{cl(y_{t_{i}})\}$ &  &  & $\{2^{[\ln\frac{n-1}{2i+1}]+1},1\leq i\leq
\lbrack\frac{n-2}{2}]\}$ & $\{2^{[\ln\frac{n}{2i+1}]+1},1\leq i\leq
\lbrack\frac{n-1}{2}]\}$\\\hline
\end{tabular}
}

\bigskip

{\footnotesize
\begin{tabular}
[c]{l}\hline%
\begin{tabular}
[c]{l||l|l|l|l|l}%
$G$ & $G_{2}$ & $F_{4}$ & $E_{6}$ & $E_{7}$ & $E_{8}$\\\hline\hline
$k$ & $1$ & $2$ & $2$ & $4$ & $7$\\\hline
$\{t_{1},\cdots,t_{k}\}$ & $\{6\}$ & $\{6,8\}$ & $\{6,8\}$ & $\{6,8,10,18\}$ &
$\{6,8,10,12,18,20,30\}$\\\hline
$\{tor(y_{t_{i}})\}$ & $\{2\}$ & $\{2,3\}$ & $\{2,3\}$ & $\{2,3,2,2\}$ &
$\{2,3,2,5,2,3,2\}$\\\hline
$\{cl(y_{t_{i}})\}$ & $\{2\}$ & $\{2,3\}$ & $\{2,3\}$ & $\{2,3,2,2\}$ &
$\{8,3,4,5,2,3,2\}$\\\hline
\end{tabular}
\\\hline
\end{tabular}
}.$\square$\noindent
\end{center}

In this paper we shall fix, once and for all, for each Lie group $G$ a set of
special Schubert classes $\left\{  y_{t_{1}},\cdots,y_{t_{k}}\right\}  $ on
$G/T$. By Lemma 1.4 the inclusions $\omega_{i},y_{t_{i}}\in H^{\ast}(G/T)$
extend to an epimorphism

\begin{quote}
$h:\mathbb{Z}[\omega_{1},\cdots,\omega_{n},y_{t_{1}},\cdots,y_{t_{k}%
}]\rightarrow H^{\ast}(G/T)$.
\end{quote}

\noindent As the ring $\mathbb{Z}[\omega_{1},\cdots,\omega_{n},y_{t_{1}%
},\cdots,y_{t_{k}}]$ is graded by $\deg\omega_{i}=2$ and $\deg y_{t}=t$, we
may use $l(g)$ to denote the degree of a homogeneous polynomial $g$ therein.
By the Hilbert basis theorem, there exists a minimal sequence of homogeneous
polynomials $\left\{  r_{1},\ldots,r_{m}\right\}  $ such that $\ker
h=\left\langle r_{1},\ldots,r_{m}\right\rangle $. In particular, $h$ induces a
ring isomorphism

\begin{quote}
$H^{\ast}(G/T)\cong\mathbb{Z}[\omega_{1},\ldots,\omega_{n},y_{t_{1}}%
,\ldots,y_{t_{k}}]/\left\langle r_{1},\ldots,r_{m}\right\rangle $
\end{quote}

\noindent which is compatible with the Borel-Chevalley formula (1.4) for the
real cohomology $H^{\ast}(G/T;\mathbb{R})$. The following result, established
in \cite[Theorem 1.2]{DZ2} (see also \cite[Theorem 5.5]{DZ3}), singles out
useful properties of the polynomials $r_{1},\cdots,r_{m}$.

\bigskip

\noindent\textbf{Theorem A.}\textsl{\ For a Lie group }$G$\textsl{ there exist
a set of homogenous polynomials }$\{f_{i},g_{i},e_{j}\}_{1\leq i\leq k,1\leq
j\leq h}\subset\ker h$\textsl{\ such that }

\begin{enumerate}
\item[(1.7)] $H^{\ast}(G/T)=\mathbb{Z}[\omega_{1},\ldots,\omega_{n},y_{t_{1}%
},\ldots,y_{t_{k}}]/\left\langle f_{i},g_{i},e_{j}\right\rangle _{1\leq i\leq
k,1\leq j\leq h;}$
\end{enumerate}

\noindent\textsl{in which}

\textsl{i) for each }$1\leq i\leq k$ \textsl{the pair\ of polynomials
}$\left\{  f_{i},g_{i}\right\}  $ \textsl{is related to the special Schubert
class }$y_{t_{i}}$\textsl{ in the fashions}

\begin{quote}
$\quad\quad f_{i}$\textsl{\ }$=$\textsl{\ }$p_{i}\cdot y_{t_{i}}-a_{i}%
$\textsl{, }$g_{i}=y_{t_{i}}^{r_{i}}-b_{i}$\textsl{, }
\end{quote}

\noindent\textsl{where }$p_{i}=tor(y_{t_{i}})$\textsl{,} $r_{i}=cl(y_{t_{i}}%
)$\textsl{\ and }$a_{i},b_{i}\in\left\langle \omega_{1},\ldots,\omega
_{n}\right\rangle $\textsl{;}

\textsl{ii) }$e_{j}\in\left\langle \omega_{1},\ldots,\omega_{n}\right\rangle
$\textsl{, }$1\leq j\leq h$\textsl{;}

\textsl{iii) the degree map} $l:\left\{  g_{i},e_{j}\right\}  _{1\leq i\leq
k,1\leq j\leq h}\rightarrow\mathbb{Z}$ \textsl{surjects onto} $2\cdot
q(G)$\textsl{, which is also one to one with the only exception}

\begin{enumerate}
\item[(1.8)] $l^{-1}(2l_{8})=\{g_{4},g_{6},g_{7}\}$\textsl{ for} $G=E_{8}%
$\textsl{.}\hfill$\square$
\end{enumerate}

\bigskip

For each special Schubert class $y_{t_{i}}$ the necessity of the pair
$\left\{  f_{i},g_{i}\right\}  $ of relations is transparent by (1.6). In
addition, we note by the contents in the fourth rows of Table 2 that

\bigskip

\noindent\textbf{Corollary 1.7.} \textsl{The torsion index }$p_{i}$ \textsl{of
a\textsl{ special }Schubert class }$y_{t_{i}}$ \textsl{is always a prime,
which takes value only in} $\{2,3,5\}$\textsl{.}\hfill$\square$

\bigskip

The dimensions of the simply Lie groups $G$ are well-known to be

\begin{center}%
\begin{tabular}
[c]{l|llllllll}\hline\hline
$G$ & $SU(n)$ & $Spin(n)$ & $\qquad Sp(n)$ & $G_{2}$ & $F_{4}$ & $E_{6}$ &
$E_{7}$ & $E_{8}$\\\hline
$\dim G$ & $n^{2}-1$ & $\frac{n(n-1)}{2}$ & $\ n(2n+1)$ & $\ 14$ & $52$ & $78$
& $133$ & $248$\\\hline\hline
\end{tabular}
.
\end{center}

\noindent Property iii) of Theorem A indicates a relationship between $\dim G$
and the degrees of the polynomials $g_{i},e_{j}$ in (1.7). In particular, we have

\begin{quote}
$k+h=n$ for $G\neq E_{8}$, but $k+h=10>8$ for $G=E_{8}$,
\end{quote}

\noindent while the routine formula due to Chevalley \cite{Ch1}

\begin{quote}
$\dim G=(2l_{1}-1)+\cdots+(2l_{n}-1)$
\end{quote}

\noindent goes over to

\bigskip

\noindent\textbf{Corollary 1.8.} $\dim G=\underset{1\leq j\leq h}{\Sigma
}(l(e_{j})-1)+\Sigma(l(g_{i})-1)$\textsl{, where the second sum ranges over
}$1\leq i\leq k$\textsl{, with the only exception that }$i\neq4,7$\textsl{ for
}$G=E_{8}$\textsl{.}

\bigskip

\noindent\textbf{Remark 1.9. }For $G=E_{8}$ we have $k=7$ by Table 2, and

\begin{quote}
$\deg g_{4}=\deg g_{6}=\deg g_{7}=60$
\end{quote}

\noindent by (1.8). In the context of Schubert calculus on $E_{8}/T$ it was
shown \cite[(6.1), (6.2)]{DZ2} that there exists a polynomial of the form

\begin{quote}
$\phi=2y_{12}^{5}-y_{20}^{3}+y_{30}^{2}+\beta$ with $\beta\in\left\langle
\omega_{1},\cdots,\omega_{8}\right\rangle $,
\end{quote}

\noindent such that the three polynomials $g_{4},g_{6}$ and $g_{7}$ are
subject to the relations

\begin{enumerate}
\item[(1.9)] $\left\{
\begin{tabular}
[c]{l}%
$g_{4}=-12\phi+5y_{12}^{4}f_{4}-4y_{20}^{2}f_{6}+6y_{30}f_{7}$;\\
$g_{6}=-10\phi+4y_{12}^{4}f_{4}-3y_{20}^{2}f_{6}+5y_{30}f_{7}$;\\
$g_{7}=15\phi-6y_{12}^{4}f_{4}+5y_{20}^{2}f_{6}-7y_{30}f_{7}$.
\end{tabular}
\right.  $
\end{enumerate}

\noindent This phenomenon will cause a few additional concerns for the group
$G=E_{8}$ in our unified approach to $H^{\ast}(G)$.\hfill$\square$

\subsection{Constructing the generators of the ring $H^{\ast}(G)$\hfill}

Let $G$ be a Lie group with a fixed set $\left\{  y_{t_{1}},\cdots,y_{t_{k}%
}\right\}  $ of special Schubert classes on $G/T$.\textsl{ }Inputting the
formula (1.7) into (1.3) we obtain a concise expression of the bi-graded ring

\begin{quote}
$E_{2}^{\ast,\ast}(G)=\frac{\mathbb{Z}[\omega_{1},\ldots,\omega_{n},y_{t_{1}%
},\cdots,y_{t_{k}}]}{\left\langle f_{i},g_{i},e_{j},\text{ }1\leq i\leq
k;1\leq j\leq h\right\rangle }\otimes\Lambda(\lambda_{1},\cdots,\lambda_{n}).$
\end{quote}

\noindent It enables us to construct a minimal system of generators of the
cohomology $H^{\ast}(G)$, uniformly for all $G$. We begin by introducing two
maps, critical for the construction.

Firstly, in term of the monomials basis $\{\omega_{1}^{k_{1}}\cdots\omega
_{n}^{k_{n}}y^{\alpha},k_{1}+\cdots+k_{n}\geq1\}$ of the ideal $\left\langle
\omega_{1},\cdots,\omega_{n}\right\rangle $ consider the linear map of degree
$-1$

\begin{enumerate}
\item[(1.10)] $\mathcal{D}:\left\langle \omega_{1},\cdots,\omega
_{n}\right\rangle \rightarrow E_{2}^{\ast,1}(G)=H^{\ast}(G/T)\otimes
\Lambda^{1}(\lambda_{1},\cdots,\lambda_{n})$
\end{enumerate}

\noindent defined by the rule

\begin{quote}
$\mathcal{D}(\omega_{1}^{k_{1}}\cdots\omega_{n}^{k_{n}}\cdot y^{\alpha
})=h(\omega_{s}^{k_{s}-1}\cdots\omega_{n}^{k_{n}}\cdot y^{\alpha}%
)\otimes\lambda_{s}$,
\end{quote}

\noindent where $s\in\{1,\cdots,n\}$ is the least index such that $k_{s}\geq
1$, and where $y^{\alpha}$ denotes an arbitrary monomial in $y_{t_{1}}%
,\cdots,y_{t_{k}}$. By Lemma 1.1 the map $\mathcal{D}$ fits into the
commutative triangle

\begin{quote}
$%
\begin{array}
[c]{ccc}%
\mathcal{D} &  & E_{2}^{\ast,1}(G)=H^{\ast}(G/T)\otimes\Lambda^{1}(\lambda
_{1},\cdots,\lambda_{n})\\
& \nearrow & \downarrow d_{2}\\
\left\langle \omega_{1},\cdots,\omega_{n}\right\rangle  & \overset{h^{\prime}%
}{\rightarrow} & E_{2}^{\ast,0}(G)=H^{\ast}(G/T)\text{,}%
\end{array}
$
\end{quote}

\noindent where $h^{\prime}$ is the restriction of the map $h$ to
$\left\langle \omega_{1},\cdots,\omega_{n}\right\rangle $. For a $d_{2}%
$-cocycle\textsl{ }$\theta\in\ker d_{2}$ write $[\theta]\in E_{3}^{\ast,\ast
}(G)$\textsl{ }to denote its cohomology class. Observe that

\bigskip

\noindent\textbf{Lemma 1.10.} \textsl{For any }$a\in\left\langle \omega
_{1},\cdots,\omega_{n}\right\rangle \cap\ker h$\textsl{ and }$b,b^{\prime}%
\in\left\langle \omega_{1},\cdots,\omega_{n}\right\rangle $ \textsl{we have}

\textsl{i) }$\mathcal{D}(a)\in\ker d_{2}$\textsl{; }

\textsl{ii) }$\mathcal{D}(b\cdot b^{\prime})-h(b)\cdot\mathcal{D}(b^{\prime
})\in\operatorname{Im}d_{2}.$

\noindent\textsl{In particular, }$[\mathcal{D}(b\cdot b^{\prime})]=0$\textsl{
if either }$b$\textsl{ or }$b^{\prime}\in\ker h$\textsl{.}

\bigskip

\noindent\textbf{Proof. }If $a\in\left\langle \omega_{1},\cdots,\omega
_{n}\right\rangle \cap\ker h$ we get from the commutative triangle
that\textbf{ }

\begin{quote}
$d_{2}(\mathcal{D}(a))=h^{\prime}(a)=h(a)=0$.
\end{quote}

\noindent This shows i). For ii) it suffices to check the cases where
$b,b^{\prime}$ are monomials in $\omega_{1},\cdots,\omega_{n}$, and the result
follows directly from the definition of $\mathcal{D}$.\hfill$\square$

\bigskip

Next, the relations $d_{r}(E_{r}^{\ast,1})=0$ for all $r\geq3$ give rise to a
sequence of quotient maps

\begin{enumerate}
\item[(1.11)] $\kappa:E_{3}^{2k,1}(G)\twoheadrightarrow\cdots
\twoheadrightarrow E_{\infty}^{2k,1}(G)=\mathcal{F}^{2k}(H^{2k+1}(G))\subset
H^{2k+1}(G)$,
\end{enumerate}

\noindent that interprets elements of $E_{3}^{2k,1}(G)$ directly as cohomology
classes of $G$, where $\mathcal{F}$ is the filtration on $H^{\ast}(G)$ induced
by $\pi$ (for details, see (2.4)). Granted with the maps $\mathcal{D}$ and
$\kappa$ above we introduce below three types of elements of $H^{\ast}(G)$.

\bigskip

\noindent\textbf{Definition 1.11. }Let $\pi^{\ast}:H^{\ast}(G/T)\rightarrow
H^{\ast}(G)$ be the induced map of $\pi$. For a special Schubert classes
$y_{t_{i}}$ on $G/T$ define the \textsl{Schubert cocycle} $x_{t_{i}}$ on $G$ by

\begin{enumerate}
\item[(1.12)] $x_{t_{i}}:=\pi^{\ast}(y_{t_{i}})\in H^{t_{i}}(G)$, $1\leq i\leq
k$.\hfill$\square$
\end{enumerate}

\bigskip

\noindent\textbf{Definition 1.12. }By ii) of Theorem A the polynomials $e_{j}$
satisfies the relation

\begin{quote}
$e_{j}\in$ $\ker h\cap\left\langle \omega_{1},\ldots,\omega_{n}\right\rangle
$, $1\leq j\leq h$.
\end{quote}

\noindent Since $\mathcal{D}(e_{j})\in\ker d_{2}$ by Lemma 1.10 we obtain the
cohomology classes

\begin{enumerate}
\item[(1.13)] $\rho_{l(e_{j})-1}:=\kappa\lbrack\mathcal{D}(e_{j})]\in
H^{l(e_{j})-1}(G),$ $1\leq j\leq h$.
\end{enumerate}

Similarly, by i) of Theorem A the pair\ of polynomials $f_{i},g_{i}\in\ker h$
gives rise to an element

\begin{quote}
$y_{t_{i}}^{r_{i}-1}\cdot f_{i}-p_{i}\cdot g_{i}=y_{t_{i}}^{r_{i}-1}\cdot
a_{i}-p_{i}\cdot b_{i}\in\ker h\cap\left\langle \omega_{1},\ldots,\omega
_{n}\right\rangle $.
\end{quote}

\noindent By i) of Lemma 1.10 we obtain the cohomology class:

\begin{enumerate}
\item[(1.14)] $\rho_{l(g_{i})-1}:=\kappa\lbrack\mathcal{D}(y_{i}^{r_{i}%
-1}\cdot a_{i}-p_{i}\cdot b_{i})]\in H^{l(g_{i})-1}(G)$,
\end{enumerate}

\noindent where $1\leq i\leq k$, with the exception that $i\neq4,7$ if
$G=E_{8}$ (see (1.8)).

For convenience, we refer the classes $\rho_{l(e_{j})-1}$, $\rho_{l(g_{i})-1}$
defined by (1.13) and (1.14) as \textsl{the primary classes }of the cohomology
$H^{\ast}(G)$.\hfill$\square$

\bigskip

A special Schubert class $y_{t}$ on $G/T$ is called $p$\textsl{-special }if
$tor(y_{t})=p$. Let $D_{1}(G,p)$ be the degree sequence of the $p$-special
Schubert classes on $G/T$. As examples, one finds from the last two columns of
Table 2 that

\begin{center}%
\begin{tabular}
[c]{l||l|l|l|l}\hline
$p$ & $2$ & $3$ & $5$ & $>5$\\\hline\hline
$D_{1}(E_{7},p)$ & $\left\{  6,10,18\right\}  $ & $\left\{  8\right\}  $ &
$\emptyset$ & $\emptyset$\\\hline
$D_{1}(E_{8},p)$ & $\left\{  6,10,18,30\right\}  $ & $\left\{  8,20\right\}  $
& $\left\{  12\right\}  $ & $\emptyset$\\\hline
\end{tabular}
.
\end{center}

\noindent\textbf{Definition 1.13. }For each $t_{i}\in D_{1}(G,p)$ consider the
polynomial $f_{i}$\textsl{\ }$=$\textsl{\ }$p\cdot y_{t_{i}}-a_{i}$ related to
the Schubert class $y_{t_{i}}$. By $f_{i}\in\ker h$ and $a_{i}\in\left\langle
\omega_{1},\ldots,\omega_{n}\right\rangle $ one gets

\begin{quote}
$f_{i}\equiv a_{i}\operatorname{mod}p\in\ker h\cap\left\langle \omega
_{1},\ldots,\omega_{n}\right\rangle $.
\end{quote}

\noindent Thus, letting $\kappa^{\prime}$ and $\mathcal{D}^{\prime}$ be
respectively the $\mathbb{F}_{p}$-analogues of the maps $\kappa$ and
$\mathcal{D}$, we obtain the $\operatorname{mod}p$ cohomology class of $G$

\begin{enumerate}
\item[(1.15)] $\theta_{t_{i}-1}:=\kappa^{\prime}[\mathcal{D}^{\prime}%
(a_{i})]\in$ $H^{t_{i}-1}(G;\mathbb{F}_{p})$.
\end{enumerate}

A subsequence $I\subseteq D_{1}(G,p)$ is called $p$\textsl{-monotone} if
$\left\vert I\right\vert \geq2$. For such a sequence $I$ we formulate, in the
sequel to (1.15), the $p$-torsion element of $H^{\ast}(G)$:

\begin{enumerate}
\item[(1.16)] $\mathcal{C}_{I}:=\beta_{p}(\theta_{I})\in H^{\ast}(G)$,
\end{enumerate}

\noindent where $\theta_{I}=\Pi_{s\in I}\theta_{s-1}$, and $\beta_{p}%
:H^{r}(G;\mathbb{F}_{p})\rightarrow H^{r+1}(G)$ is the Bockstein
homomorphism.\hfill$\square$

\bigskip

\noindent\textbf{Example 1.14. }Since\textbf{ }the sequence $q(G)$ is strictly
increasing by Table 1, the set of all primary classes $\{\rho_{l(e_{j}%
)-1},\rho_{l(g_{i})-1}\}$ of $G$ can be rephrased as $\{\rho_{2l-1},l\in
q(G)\}$ by iii) of Theorem A. As a result, we get by Corollary 1.8 that

\begin{enumerate}
\item[(1.17)] $\dim G=\Sigma_{l\in q(G)}\deg\rho_{2l-1}$.
\end{enumerate}

In addition, in terms of the invariants of the group $G$ given by Table 2, one
can enumerate all the Schubert cocycles $x_{s}$, and the torsion elements
$\mathcal{C}_{I}$ of $H^{\ast}(G)$. For example, if $G=E_{7}$, these classes
are $\left\{  x_{6},x_{8},x_{10},x_{18}\right\}  $ and $\mathcal{C}_{I}$,
where $I\subseteq\left\{  6,10,18\right\}  $ are $2$\textsl{-}monotone.\hfill
$\square$

\subsection{The cohomology of exceptional Lie groups}

The integral cohomology of a finite $CW$--complex $X$ admits the decomposition

\begin{enumerate}
\item[(1.18)] $H^{\ast}(X)=\mathcal{F}(X)\oplus_{p}\tau_{p}(X)$,
\end{enumerate}

\noindent where $\mathcal{F}(X):=H^{\ast}(X)/TorH^{\ast}(X)$ is \textsl{the
free part} of the cohomology $H^{\ast}(X)$, the direct sum $\oplus$ is over
all primes $p\geq2$, and where $\tau_{p}(X)$ is the $p$-\textsl{primary
component }of $H^{\ast}(X)$ defined by

\begin{quote}
\textsl{\ }$\tau_{p}(X):=\{x\in H^{\ast}(X)\mid p^{r}x=0,$ $r\geq1\}$.
\end{quote}

Given a sequence of graded elements $z_{1},\cdots,z_{m}$ with $\deg z_{1}%
\leq\cdots\leq\deg z_{m}$, denote by $\Delta(z_{1},\cdots,z_{m})$
\footnote{This notion $\Delta(z_{1},\cdots,z_{m})$ is due to Borel, who called
it the group in the simple system of generators $z_{1},\cdots,z_{m}$.} the
free $\mathbb{Z}$-module with the monomial basis $z_{1}^{\varepsilon_{1}%
}\cdots z_{m}^{\varepsilon_{m}}$, where $\varepsilon_{i}\in\{0,1\}$ and
$z_{i}^{0}=1$. In addition, if

\begin{quote}
$B=B^{0}\oplus B^{1}\oplus B^{2}\oplus\cdots$
\end{quote}

\noindent is a graded group (resp. ring), we use $B^{+}$ to denote its
subgroup (resp. subring) $B^{1}\oplus B^{2}\oplus\cdots$.

The integral cohomologies $H^{\ast}(G)$ have been computed for $G=SU(n)$,
$Sp(n)$ by Borel \cite{B1}, and for $G=Spin(n)$ by Pittie \cite{P}. Our main
result presents the cohomologies of the exceptional Lie groups, merely using
the classes $x_{t}$, $\varrho_{2l-1}$ and $\mathcal{C}_{I}$ constructed in
Section \S 1.2.

\bigskip

\noindent\textbf{Theorem B.} \textsl{The integral cohomology of the
exceptional Lie groups are:}

\begin{enumerate}
\item[i)] $H^{\ast}(G_{2})=\Delta(\varrho_{3})\otimes\Lambda(\varrho
_{11})\oplus\tau_{2}(G_{2})$\textsl{, where}

$\qquad\tau_{2}(G_{2})=\mathbb{F}_{2}[x_{6}]^{+}/\left\langle x_{6}%
^{2}\right\rangle \otimes\Delta(\varrho_{3})$\textsl{, }

\textsl{and where the generators are subject to the relations}

$\qquad\varrho_{3}^{2}=x_{6}$\textsl{,} $\varrho_{11}\cdot x_{6}=0$\textsl{.}

\item[ii)] $H^{\ast}(F_{4})=\Delta(\varrho_{3})\otimes\Lambda(\varrho
_{11},\varrho_{15},\varrho_{23})\oplus\tau_{2}(F_{4})\oplus\tau_{3}(F_{4}%
)$\textsl{, where}

$\qquad\tau_{2}(F_{4})=\mathbb{F}_{2}[x_{6}]^{+}/\left\langle x_{6}%
^{2}\right\rangle \otimes\Delta(\varrho_{3})\otimes\Lambda(\varrho
_{15},\varrho_{23})$\textsl{,}

$\qquad\tau_{3}(F_{4})=\mathbb{F}_{3}[x_{8}]^{+}/\left\langle x_{8}%
^{3}\right\rangle \otimes\Lambda(\varrho_{3},\varrho_{11},\varrho_{15}%
)$\textsl{,}

\textsl{and} \textsl{where the generators are subject to the relations}

$\qquad\varrho_{3}^{2}=x_{6}$\textsl{, } $\varrho_{11}\cdot x_{6}=0$\textsl{,
}$\varrho_{23}\cdot x_{8}=0$\textsl{.}

\item[iii)] $H^{\ast}(E_{6})=\Delta(\varrho_{3})\otimes\Lambda(\varrho
_{9},\varrho_{11},\varrho_{15},\varrho_{17},\varrho_{23})\oplus\tau_{2}%
(E_{6})\oplus\tau_{3}(E_{6})$\textsl{, where}

$\qquad\tau_{2}(E_{6})=\mathbb{F}_{2}[x_{6}]^{+}/\left\langle x_{6}%
^{2}\right\rangle \otimes\Delta(\varrho_{3})\otimes\Lambda(\varrho_{9}%
,\varrho_{15},\varrho_{17},\varrho_{23})$\textsl{,}

$\qquad\tau_{3}(E_{6})=\mathbb{F}_{3}[x_{8}]^{+}/\left\langle x_{8}%
^{3}\right\rangle \otimes\Lambda(\varrho_{3},\varrho_{9},\varrho_{11}%
,\varrho_{15},\varrho_{17})$\textsl{,}

\textsl{and where the generators are subject to the relations}

$\qquad\varrho_{3}^{2}=x_{6}$\textsl{, }$\varrho_{11}\cdot x_{6}=0$\textsl{,
}$\varrho_{23}\cdot x_{8}=0$\textsl{.}

\item[iv)] $H^{\ast}(E_{7})=\Delta(\varrho_{3})\otimes\Lambda(\varrho
_{11},\varrho_{15},\varrho_{19},\varrho_{23},\varrho_{27},\varrho
_{35})\underset{p=2,3}{\oplus}\tau_{p}(E_{7})$\textsl{, where}

$\qquad\tau_{2}(E_{7})=\frac{\mathbb{F}_{2}[x_{6},x_{10},x_{18},\mathcal{C}%
_{I}]^{+}}{\left\langle x_{6}^{2},x_{10}^{2},x_{18}^{2},\mathcal{R}%
_{J},\mathcal{S}_{K,L}\right\rangle }\otimes\Delta(\varrho_{3})\otimes
\Lambda(\varrho_{15},\varrho_{23},\varrho_{27})$

\textsl{with }$I,J,K,L\subseteq\{6,10,18\}$\textsl{, }$\left\vert I\right\vert
,\left\vert J\right\vert ,\left\vert K\right\vert \geq2$\textsl{,}

$\qquad\tau_{3}(E_{7})=\frac{\mathbb{F}_{3}[x_{8}]^{+}}{\left\langle x_{8}%
^{3}\right\rangle }\otimes\Lambda(\varrho_{3},\varrho_{11},\varrho
_{15},\varrho_{19},\varrho_{27},\varrho_{35})$\textsl{,}

\textsl{and} \textsl{where the generators are subject to the relations}

$\qquad\varrho_{3}^{2}=x_{6}$\textsl{, }$\varrho_{23}\cdot x_{8}=0,$

\qquad$\mathcal{H}_{i,I}\in\tau_{2}(E_{7})$ \textsl{with} $i\in\{6,10,18\}$%
\textsl{,} $I\subseteq\{6,10,18\}$.

\item[v)] $H^{\ast}(E_{8})=\Delta(\varrho_{3},\varrho_{15},\varrho
_{23})\otimes\Lambda(\varrho_{27},\varrho_{35},\varrho_{39},\varrho
_{47},\varrho_{59})\underset{p=2,3,5}{\oplus}\tau_{p}(E_{8})$\textsl{, where}

$\qquad\tau_{2}(E_{8})=\frac{\mathbb{F}_{2}[x_{6},x_{10},x_{18},x_{30}%
,\mathcal{C}_{I}]^{+}}{\left\langle x_{6}^{8},x_{10}^{4},x_{18}^{2},x_{30}%
^{2},\mathcal{R}_{J},\mathcal{S}_{K,L}\right\rangle }\otimes\Delta(\varrho
_{3},\varrho_{15},\varrho_{23})\otimes\Lambda(\varrho_{27})$

\textsl{with} $I$\textsl{,}$J,K\subseteq\{6,10,18,30\}$\textsl{, }$\left\vert
I\right\vert ,\left\vert J\right\vert ,\left\vert K\right\vert \geq
2$\textsl{,}

$\qquad\tau_{3}(E_{8})=\frac{\mathbb{F}_{3}[x_{8},x_{20},\mathcal{C}%
_{\{8,20\}}]^{+}}{\left\langle x_{8}^{3},x_{20}^{3},x_{8}^{2}x_{20}%
^{2}\mathcal{C}_{\{8,20\}},\mathcal{C}_{\{8,20\}}^{2}\right\rangle }%
\otimes\Lambda(\varrho_{3},\varrho_{15},\varrho_{27},\varrho_{35},\varrho
_{39},\varrho_{47})$\textsl{,}

$\qquad\tau_{5}(E_{8})=\frac{\mathbb{F}_{5}[x_{12}]^{+}}{\left\langle
x_{12}^{5}\right\rangle }\otimes\Lambda(\varrho_{3},\varrho_{15},\varrho
_{23},\varrho_{27},\varrho_{35},\varrho_{39},\varrho_{47})$\textsl{,}

\textsl{and where the generators are subject to the relations}

$\qquad\varrho_{3}^{2}=x_{6}$\textsl{, }$\varrho_{15}^{2}=x_{30}$\textsl{,
}$\varrho_{23}^{2}=x_{6}^{6}x_{10}$\textsl{, }$\varrho_{59}\cdot x_{12}=0$;

$\qquad\varrho_{23}\cdot x_{8}=0$, $\rho_{23}\cdot x_{20}=x_{8}^{2}%
\cdot\mathcal{C}_{\left\{  8,20\right\}  }$, $\rho_{23}\cdot\mathcal{C}%
_{\left\{  8,20\right\}  }=0$;

$\qquad\rho_{59}\cdot x_{20}=0$, $\rho_{59}\cdot x_{8}=x_{20}^{2}%
\mathcal{C}_{\left\{  8,20\right\}  }$, $\rho_{59}\cdot\mathcal{C}_{\left\{
8,20\right\}  }=0$;

\qquad$\mathcal{H}_{i,I}\in\tau_{2}(E_{8})$ \textsl{with} $i\in\{6,10,18,30\}$%
\textsl{,} $I\subseteq\{6,10,18,30\}$.
\end{enumerate}

\noindent\textsl{In addition, the relations of the types} $\mathcal{R}%
_{K},\mathcal{S}_{I,J}$ \textsl{and} $\mathcal{H}_{i,I}$\textsl{,}
\textsl{concerning only} \textsl{the }$2$\textsl{ torsion ideals }$\tau
_{2}(E_{7})$ \textsl{and} $\tau_{2}(E_{8})$\textsl{, are too many to be listed
explicitly. Instead, their general formulae are stated in (4.3), (4.4) and
(4.7), respectively.}

\bigskip

The remaining sections of the paper are arranged as following. Section \S 2
develops properties of the cohomology of certain Koszul complexes, useful for
solving the extension problem from $E_{3}^{\ast,\ast}(G)$ to $H^{\ast}(G)$.
Section \S 3 recalls from \cite{DZ1} a presentation of the algebra $H^{\ast
}(G;\mathbb{F}_{p})$, and calculates the $\operatorname{mod}p$ Bockstein
cohomology $H_{\beta}^{\ast}(G;\mathbb{F}_{p})$. In Section \S 4 we determine
the structures of three basic components of the integral cohomology $H^{\ast
}(G)$: the subring $\operatorname{Im}\pi^{\ast}$, the free part $\mathcal{F}%
(G)$, and the torsion ideals $\tau_{p}(G)$. As applications, the proof of
Theorem B is completed in Section \S 4.4.

The study of the cohomology theory of Lie groups has a long and outstanding
history. Section \S 5 is created to recall early works on the topic, make
comparison of our approach with the classical ones. It serves also the purpose
to illustrate the necessity of the conceptual developments of the present work.

\section{The cohomology of certain Koszul complexes}

Let $A$ be a graded ring (or algebra), and let $\{z_{1},\ldots,z_{k}\}$ be a
sequence of homogeneous elements of $A$. The \textsl{Koszul complex}
associated to the pair $\{A;(z_{1},\ldots,z_{k})\}$, written $K(A;z_{1}%
,\ldots,z_{k})$, is the cochain complex $\left\{  C,\delta\right\}  $ defined by

\begin{quote}
i) $C=A\otimes\Delta(\theta_{1},\cdots,\theta_{k})$, $\deg\theta_{t}=\deg
z_{t}-1$;

ii) $\delta$ is the derivation of degree $1$ on $C$ given by

$\qquad\delta(\theta_{t})=z_{t}$ and $\delta(z)=0$, $z\in A$.
\end{quote}

\noindent With $\delta\circ\delta=0$ the cohomology $H^{\ast}(K(A;z_{1}%
,\ldots,z_{k}))$ of the complex $\left\{  C,\delta\right\}  $ is defined to be
the graded quotient group $\ker\delta/\operatorname{Im}\delta$.

For example, by Lemma 1.1 the second page $\left\{  E_{2}^{\ast,\ast}%
(G),d_{2}\right\}  $ of the Serre spectral sequence of $\pi$ is the Koszul
complex $K(H^{\ast}(G/T);\omega_{1},\ldots,\omega_{n})$, whose cohomology is
the third page $E_{3}^{\ast,\ast}(G)$.

\subsection{The Koszul complex $\left\{  E_{2}^{\ast,\ast}(G),d_{2}\right\}
$}

Let $\mathcal{F}^{p}$ be the filtration on $H^{\ast}(G)$ defined by the map
$\pi$, namely (\cite[P.146]{Mc})

\begin{center}
$0=\mathcal{F}^{r+1}(H^{r}(G))\subseteq\mathcal{F}^{r}(H^{r}(G))\subseteq
\cdots\subseteq\mathcal{F}^{0}(H^{r}(G))=H^{r}(G)$
\end{center}

\noindent with

\begin{enumerate}
\item[(2.1)] $E_{\infty}^{p,q}(G)=\mathcal{F}^{p}(H^{p+q}(G))/\mathcal{F}%
^{p+1}(H^{p+q}(G))$.
\end{enumerate}

\noindent The routine property $d_{r}(E_{r}^{\ast,0}(G))=0$, $r\geq2$, yields
a sequence of quotient maps

\begin{center}
$H^{r}(G/T)=E_{2}^{r,0}\rightarrow E_{3}^{r,0}\rightarrow\cdots\rightarrow
E_{\infty}^{r,0}=\mathcal{F}^{r}(H^{r}(G))\subseteq H^{r}(G)$
\end{center}

\noindent whose composition agrees with $\pi^{\ast}:H^{\ast}(G/T)\rightarrow
H^{\ast}(G)$ \cite[P.147]{Mc}. We may therefore reserve $\pi^{\ast}$ for the surjection:

\begin{enumerate}
\item[(2.2)] $\pi^{\ast}:E_{3}^{\ast,0}(G)\twoheadrightarrow\mathcal{F}%
^{r}(H^{r}(G))\subset H^{\ast}(G)$.
\end{enumerate}

\noindent In addition, since any Schubert class $s_{w}$ satisfies $d_{2}%
(s_{w}\otimes1)=0$ by Lemma 1.1, we may reserve the notion $s_{w}$ to simplify
the class $[s_{w}\otimes1]\in E_{3}^{\ast,0}(G)$.

\bigskip

\noindent\textbf{Lemma 2.1.} \textsl{If }$\left\{  y_{t_{1}},\cdots,y_{t_{k}%
}\right\}  $\textsl{ is a set of special Schubert classes on }$G/T$\textsl{,
then}

\begin{quote}
$E_{3}^{\ast,0}(G)=\mathbb{Z}[y_{t_{1}},\cdots,y_{t_{k}}]/\left\langle
p_{i}y_{t_{i}},y_{t_{i}}^{r_{i}}\right\rangle _{1\leq i\leq k}$,
\end{quote}

\noindent\textsl{where} $p_{i}=tor(y_{t_{i}})$\textsl{,} $r_{i}=cl(y_{t_{i}})$.

\bigskip

\noindent\textbf{Proof. }The group $E_{3}^{\ast,0}(G)$ is the cokernel of the differential

\begin{quote}
$d_{2}:E_{2}^{\ast,1}(G)=H^{\ast}(G/T)\otimes\Lambda^{1}(t_{1},\cdots
,t_{n})\rightarrow E_{2}^{\ast,0}(G)=H^{\ast}(G/T)$.
\end{quote}

\noindent Since $\operatorname{Im}d_{2}=\left\langle \omega_{1},\cdots
,\omega_{n}\right\rangle _{c}\subseteq H^{\ast}(G/T)$ by Lemma 1.1, the result
is shown by setting $\omega_{1}=\cdots=\omega_{n}=0$ in (1.7).\hfill$\square$

\bigskip

The fact $H^{2s+1}(G/T)=0$ by Lemma 1.3 implies that

\begin{enumerate}
\item[(2.3)] $E_{r}^{2s+1,q}(G)=0$ for all $s,q\geq0$ and $r\geq2$.
\end{enumerate}

\noindent In particular, taking $(r,q)=(\infty,0)$ we get by (2.1) that

\begin{quote}
$\mathcal{F}^{2s+1}(H^{2s+1}(G))=\mathcal{F}^{2s+2}(H^{2s+1}(G))=0$.
\end{quote}

\noindent It implies, again by (2.1), that

\begin{quote}
$E_{\infty}^{2s,1}(G)=\mathcal{F}^{2s}(H^{2s+1}(G))\subset H^{2s+1}(G)$.
\end{quote}

\noindent Combining this with $d_{r}(E_{r}^{\ast,1})=0$ for $r\geq3$ gives
rise to the empimorphism onto $E_{\infty}^{\ast,1}(G)$, already mentioned in (1.11):

\begin{enumerate}
\item[(2.4)] $\kappa:E_{3}^{\ast,1}(G)\twoheadrightarrow E_{\infty}^{\ast
,1}(G)=\mathcal{F}^{2s}(H^{2s+1}(G))\subset H^{\ast}(G)$.
\end{enumerate}

\noindent With respect to the product inherited from that on $E_{2}^{\ast
,\ast}(G)$ the third page $E_{3}^{\ast,\ast}(G)$ is naturally a bi-graded ring
\cite[P.668]{Wh}. Concerning the relationship between the two maps $\pi^{\ast
}$ and $\kappa$ we have the next two results.

\bigskip

\noindent\textbf{Lemma 2.2. }\textsl{For any }$\rho\in E_{3}^{2s,1}%
(G)$\textsl{\ one has} $\kappa(\rho)^{2}\in\operatorname{Im}\pi^{\ast}\cap
\tau_{2}(G)$\textsl{.}

\bigskip

\noindent\textbf{Proof.} For an $\rho\in E_{3}^{2s,1}(G)$ the obvious relation
$\rho^{2}=0$ in

\begin{quote}
$E_{3}^{4s,2}(G)\twoheadrightarrow E_{\infty}^{4s,2}(G)=\mathcal{F}%
^{4s}(H^{4s+2}(G))/\mathcal{F}^{4s+1}(H^{4s+2}(G))$
\end{quote}

\noindent implies that $\kappa(\rho)^{2}\in\mathcal{F}^{4s+1}(H^{4s+2}(G))$. From

\begin{quote}
$\mathcal{F}^{4s+1}(H^{4s+2}(G))/\mathcal{F}^{4s+2}(H^{4s+2}(G))=E_{\infty
}^{4s+1,1}(G)=0$ (by (2.3))
\end{quote}

\noindent one finds further that

\begin{quote}
$\kappa(\rho)^{2}\in\mathcal{F}^{4s+2}(H^{4s+2}(G))=\operatorname{Im}\pi
^{\ast}$ (by (2.2)).
\end{quote}

\noindent On the other hand, with $\deg\kappa(\rho)=2s+1$ we must have
$2\kappa(\rho)^{2}=0$, showing that $\kappa(\rho)^{2}\in\tau_{2}(G)$%
.\hfill$\square$

\bigskip

For a prime $p$ consider the Bockstein exact sequence associated to the short
exact sequence $0\rightarrow\mathbb{Z}\rightarrow\mathbb{Z}\rightarrow
\mathbb{F}_{p}\rightarrow0$ of coefficients groups

\begin{enumerate}
\item[(2.5)] $\cdots\rightarrow H^{r}(G)\overset{\cdot p}{\rightarrow}%
H^{r}(G)\overset{r_{p}}{\rightarrow}H^{r}(G;\mathbb{F}_{p})\overset{\beta_{p}%
}{\rightarrow}H^{r+1}(G)\rightarrow\cdots$,
\end{enumerate}

\noindent where $\beta_{p}$ is the Bockstein homomorphism, $r_{p}$ is the
$\operatorname{mod}p$ reduction. Since $H^{\ast}(G)$ (resp. $H^{\ast
}(G;\mathbb{F}_{p})$) is a module over the ring $\operatorname{Im}\pi^{\ast}$,
we can regard (2.5) as an exact sequence in the $\operatorname{Im}\pi^{\ast}%
$-modules. On the other hand, since the cohomology $H^{\ast}(G/T)$ is torsion
free, we have the short exact sequence of $d_{2}$-complexes

\begin{enumerate}
\item[(2.6)] $0\rightarrow E_{2}^{\ast,\ast}(G)\overset{\cdot p}{\rightarrow
}E_{2}^{\ast,\ast}(G)\overset{r_{p}}{\rightarrow}E_{2}^{\ast,\ast
}(G;\mathbb{F}_{p})\rightarrow0$,
\end{enumerate}

\noindent whose cohomological exact sequence contains the section

\begin{enumerate}
\item[(2.7)] $\cdots\rightarrow E_{3}^{\ast,1}(G)\overset{\cdot p}%
{\rightarrow}E_{3}^{\ast,1}(G)\overset{r_{p}}{\rightarrow}E_{3}^{\ast
,1}(G;\mathbb{F}_{p})\overset{\overline{\beta}_{p}}{\rightarrow}E_{3}^{\ast
,0}(G)\overset{\cdot p}{\rightarrow}E_{3}^{\ast,0}(G)\rightarrow\cdots$
\end{enumerate}

\noindent which may be viewed as an exact sequence in the $E_{3}^{\ast,0}%
(G)$-modules. Since both $\operatorname{Im}\pi^{\ast}$ and $\operatorname{Im}%
\kappa$ have been identified with appropriate subgroups of $H^{\ast}(G)$ by
(2.2) and (2.4), we have by (2.5) and (2.7) the following exact ladder

\begin{center}%
\begin{tabular}
[c]{llllllll}%
$E_{3}^{\ast,1}(G)$ & $\overset{\cdot p}{\rightarrow}$ & $E_{3}^{\ast,1}(G)$ &
$\overset{r_{p}}{\rightarrow}$ & $E_{3}^{\ast,1}(G;\mathbb{F}_{p})$ &
$\overset{\overline{\beta}_{p}}{\rightarrow}$ & $E_{3}^{\ast,0}(G)$ &
$\rightarrow\cdots$\\
$\kappa\downarrow$ &  & $\kappa\downarrow$ &  & $\kappa^{\prime}\downarrow$ &
& $\pi^{\ast}\downarrow$ & \\
$H^{r}(G)$ & $\overset{\cdot p}{\rightarrow}$ & $H^{r}(G)$ & $\overset{r_{p}%
}{\rightarrow}$ & $H^{r}(G;\mathbb{F}_{p})$ & $\overset{\beta_{p}}%
{\rightarrow}$ & $H^{r+1}(G)$ & $\rightarrow\cdots$%
\end{tabular}
,
\end{center}

\noindent where $\kappa^{\prime}$ is the $\operatorname{mod}p$ analogue of the
map $\kappa$. This shows that

\bigskip

\noindent\textbf{Lemma 2.3. }\textsl{For any }$x\in E_{3}^{\ast,0}(G)$\textsl{
and} $\rho\in E_{3}^{\ast,1}(G)$\textsl{ one has }

\begin{enumerate}
\item[(2.8)] $\kappa(x\cdot\rho)=\pi^{\ast}x\cdot\kappa(\rho)$;

\item[(2.9)] $\beta_{p}\circ\kappa^{\prime}=\pi^{\ast}\circ\overline{\beta
}_{p}$.\hfill$\square$
\end{enumerate}

\subsection{The Koszul complex of a polynomial algebra}

Let $p$ be a prime. Given a sequence of $r$ evenly graded elements $\left\{
y_{1},\cdots,y_{r}\right\}  $, and a sequence of $r$ positive integers
$\left\{  k_{1},\cdots,k_{r}\right\}  $, consider the (truncated) polynomial
algebra over the field $\mathbb{F}_{p}$

\begin{enumerate}
\item[(2.10)] $A=\mathbb{F}_{p}[y_{1},\cdots,y_{r}]/\left\langle y_{1}^{k_{1}%
},\cdots,y_{r}^{k_{r}}\right\rangle $.
\end{enumerate}

\noindent The Koszul complex\textsl{ }$K(A;y_{1},\ldots,y_{r})$, abbreviated
by $K(A)=\{C,\delta\}$, will be called \textsl{the Koszul complex associated
to} \textsl{the polynomial algebra} $A$. The cohomology of $K(A)$, as well as
the sub $A$-module $\operatorname{Im}\delta$ of $C$, can be explicitly presented.

Regard the cochain group $C=A\otimes\Delta(\theta_{1},\cdots,\theta_{r})$ as
an $A$-module with the basis $\left\{  1,\theta_{I},I\subseteq\{1,\cdots
,r\}\right\}  $, where $\theta_{I}:=\Pi_{t\in I}\theta_{t}$. Accordingly, for
a multi-index $I\subseteq\{1,\cdots,r\}$ introduce in $C$ the next elements

\begin{enumerate}
\item[(2.11)] $g_{I}:=(\Pi_{i\in I}y_{i}^{k_{i}-1})\theta_{I}$, $C_{I}%
:=\delta(\theta_{I})$, $R_{I}=(\Pi_{i\in I}y_{i}^{k_{i}-1})C_{I}$,
\end{enumerate}

\noindent and let $A\cdot\{1,C_{I}\}$ be the sub $A$-module of $C$ with the
basis $\{1,C_{I}\}$, where $I\subseteq\{1,\cdots,r\}$.

\bigskip

\noindent\textbf{Theorem 2.4.} \textsl{As a subspace} \textsl{of }$C$\textsl{
the cohomology }$H^{\ast}(K(A))$\textsl{ has the basis }$\{1,g_{I}\}$\textsl{,
}$I\subseteq\{1,\cdots,r\}$\textsl{. In particular, }$\dim H^{\ast
}(K(A))=2^{r}$.

\textsl{In addition,, as a module over }$A$\textsl{,} \textsl{the group
}$\operatorname{Im}\delta$\textsl{ has the presentation}

\begin{enumerate}
\item[(2.12)] $\operatorname{Im}\delta=\frac{A\cdot\{1,C_{I}\}^{+}%
}{\left\langle R_{J}\right\rangle }$\textsl{, where }$I,J\subseteq
\{1,\cdots,r\}$\textsl{, }$\left\vert I\right\vert ,\left\vert J\right\vert
\geq2$\textsl{,}
\end{enumerate}

\noindent\textsl{and where} $\left\langle R_{J}\right\rangle $ \textsl{is the
submodule} \textsl{spanned over }$A^{+}$\textsl{ by the elements }$R_{J}%
$\textsl{'s.}

\bigskip

\noindent\textbf{Proof.} For a $1\leq t\leq r$ set $C_{t}:=(\mathbb{F}%
_{p}[y_{t}]/\left\langle y_{t}^{k_{t}}\right\rangle )\otimes\Delta(\theta
_{t})$. Then the cochain group $C$ has the decomposition $C=\otimes_{1\leq
t\leq r}C_{t}$, in which each factor $C_{t}$ is an invariant subspace of
$\delta$. Consequently, letting $\delta_{t}$ be the restriction of $\delta$ on
$C_{t}$ we get by the K\"{u}nneth formula that

\begin{quote}
$H^{\ast}(K(A))=\otimes_{1\leq t\leq r}H^{\ast}(C_{t},\delta_{t})$.
\end{quote}

\noindent The first assertion is shown by the fact that the cohomology
$H^{\ast}(C_{t},\delta_{t})$ has a basis consisting of the only two elements
$1,y_{t}^{k_{t}-1}\theta_{t}\in C_{t}$.

To show (2.12) we note by the short exact sequence

\begin{quote}
$0\rightarrow\ker\delta\rightarrow C\overset{\delta}{\rightarrow
}\operatorname{Im}\delta\rightarrow0$
\end{quote}

\noindent that $\delta$ induces a degree $1$ isomorphism of $A$-modules

\begin{quote}
$\overline{\delta}:C/\ker\delta\overset{\cong}{\rightarrow}\operatorname{Im}%
\delta$.
\end{quote}

\noindent Since

$C=A\cdot\{1,\theta_{I}\}_{I\subseteq\{1,\cdots,r\}}$, $\ker\delta=H^{\ast
}(K(A))\oplus\operatorname{Im}\delta$,

\noindent and since

a) $H^{\ast}(K(A))$ has the basis\textsl{ }$\{1,g_{I},$ $I\subseteq
\{1,\cdots,r\}\}$,

b) $\operatorname{Im}\delta$ is spanned over $A$ by the set $\{C_{J}%
=\delta(\theta_{J}),J\subseteq\{1,\cdots,r\}$, $\left\vert J\right\vert
\geq2\}$,

\noindent we get the presentation

\begin{quote}
$C/\ker\delta=\frac{A\cdot\{1,\theta_{I}\}^{+}}{\left\langle C_{J},\text{
}g_{K}\right\rangle }$, where $I,J,K\subseteq\{1,\cdots,r\}$.
\end{quote}

\noindent Applying the isomorphism $\overline{\delta}$ to both sides we obtain
formula (2.12) from the obvious relations

\begin{quote}
$\overline{\delta}(\theta_{I})=C_{I}$, $\quad\overline{\delta}(C_{J})=0$,
$\quad\overline{\delta}(g_{K})=R_{K}$,
\end{quote}

\noindent together with the fact that, if $I=\{t\}$ is a singleton, then

\begin{quote}
$C_{I}=y_{t}\in A$,$\quad R_{I}=y_{t}^{k_{t}}=0$.\hfill$\square$
\end{quote}

\bigskip

We emphasize at this stage that, if $b=(b_{1},\cdots,b_{r})\subset A$ is a
sequence of homogeneous elements that satisfies the degree constraints\textsl{
}

\begin{quote}
$\deg b_{i}=2\deg\theta_{i}$, $1\leq i\leq r$,
\end{quote}

\noindent then the cochain group $C=A\otimes\Delta(\theta_{1},\cdots
,\theta_{r})$ can be furnished with the product $\theta_{i}^{2}=b_{i}$ making
it the graded polynomial algebra

\begin{quote}
$C\cong\frac{\mathbb{F}_{p}[y_{1},\cdots,y_{r},\theta_{1},\cdots,\theta_{r}%
]}{\left\langle y_{1}^{k_{1}},\cdots,y_{r}^{k_{r}},\theta_{1}^{2}-b_{1}%
,\cdots,\theta_{r}^{2}-b_{r}\right\rangle }$.
\end{quote}

\noindent Consequently, the complex $K(A)=\{C,\delta\}$\textsl{ }%
becomes\textsl{ a graded differential algebra. }Regarding this fact we refer
the sequence $b$ as an\textsl{ algebra structure} on the complex $K(A)$.

\bigskip

\noindent\textbf{Theorem 2.5}\textsl{. If the Koszul complex }$K(A)$\textsl{
is furnished with an algebra structure }$b$\textsl{, then }$\operatorname{Im}%
\delta$\textsl{ is isomorphic to the (truncated) polynomial algebra}

\begin{enumerate}
\item[(2.13)] $\operatorname{Im}\delta=\frac{\mathbb{F}_{p}[y_{1},\cdots
,y_{r},C_{I}]^{+}}{\left\langle y_{1}^{k_{1}},\cdots,y_{r}^{k_{r}}%
,R_{I},S_{J,K}\right\rangle }$\textsl{, }
\end{enumerate}

\noindent\textsl{where} $I,J,K\subseteq\{1,\cdots,r\}$\textsl{, }$\left\vert
I\right\vert ,\left\vert J\right\vert ,\left\vert K\right\vert \geq2$\textsl{,
and where if }$J=\{j_{1},\cdots,j_{k}\}$\textsl{,}

\begin{enumerate}
\item[(2.14)] $S_{J,K}=C_{J}C_{K}-\underset{i_{s}\in J}{\Sigma}(-1)^{s-1}%
y_{i_{s}}b_{J_{s}\cap K}\cdot C_{\left\langle J_{s},K\right\rangle }%
$\textsl{,}
\end{enumerate}

\noindent\textsl{here} $J_{s}$ \textsl{denotes the complement of} $i_{s}\in
J$\textsl{,} $b_{L}=\Pi_{s\in L}b_{s}\in A$\textsl{, and}

\begin{quote}
$\left\langle J,K\right\rangle =\{t\in J\cup K\mid t\notin J\cap
K\}$\textsl{.}
\end{quote}

\noindent\textbf{Proof. }Granted with the\textbf{ }algebra structure $b$ on
$K(A)$\textsl{ }the product of\textsl{ }any two elements\textbf{ }$C_{J}%
,C_{K}\in C$ can be expanded as an $A$-linear combination in the $C_{I}$'s. Precisely,

\begin{quote}
$C_{J}\cdot C_{K}=\delta(\theta_{J})\cdot\delta(\theta_{K})=\delta
(\delta(\theta_{J})\cdot\theta_{K})$ (by $\delta^{2}=0$)

$=\delta((\Sigma_{i_{s}\in J}(-1)^{s}y_{i_{s}}\theta_{J_{s}})\cdot\theta_{K})$

$=\delta(\Sigma_{i_{s}\in J}(-1)^{s}y_{i_{s}}(b_{J_{s}\cap K}\cdot
\theta_{\left\langle J_{s},K\right\rangle }))$ (by $\theta_{i}^{2}=b_{i}$)

$=\Sigma_{i_{s}\in J}(-1)^{s}y_{i_{s}}b_{J_{s}\cap K}\cdot C_{\left\langle
J_{s},K\right\rangle }$
\end{quote}

\noindent(by $\delta(\theta_{K})=C_{K}$ and $\delta(a)=0$ for $a\in A$). It
implies that, if we introduce in the algebra $\mathbb{F}_{p}[y_{1}%
,\cdots,y_{r},C_{I}]^{+}$ the polynomials

\begin{quote}
$S_{J,K}=C_{J}\cdot C_{K}-\Sigma_{i_{s}\in J}(-1)^{s}y_{i_{s}}b_{J_{s}\cap
K}\cdot C_{\left\langle J_{s},K\right\rangle },$ $J,K\subseteq\{1,\cdots,r\}$,
\end{quote}

\noindent then, additively,

\begin{quote}
$\frac{\mathbb{F}_{p}[y_{1},\cdots,y_{r},C_{I}]^{+}}{\left\langle y_{1}%
^{k_{1}},\cdots,y_{r}^{k_{r}},R_{J}\right\rangle }\operatorname{mod}%
S_{J,K}\equiv\frac{A\cdot\{1,C_{I}\}^{+}}{\left\langle R_{J}\right\rangle }$.
\end{quote}

\noindent Thus, we have derived (2.13) from (2.12).\hfill$\square$

\bigskip

\noindent\textbf{Remark 2.6.} If the complex $C$ happens to be a subgroup of
the $\operatorname{mod}p$ cohomology $H^{\ast}(X;\mathbb{F}_{p})$ of a
topological space $X$, and if $p\equiv1\operatorname{mod}2$, then with respect
to the cup product on $H^{\ast}(X;\mathbb{F}_{p})$ we must have $\theta
_{i}^{2}=b_{i}=0$ for the degree reason $\deg\theta_{i}\equiv
1\operatorname{mod}2$. This implies, instead of Theorem 2.5, that

i) $C$ is isomorphic to the subalgebra $A\otimes\Lambda(\theta_{1}%
,\cdots,\theta_{r})$ of $H^{\ast}(X;\mathbb{F}_{p})$;

ii) $\operatorname{Im}\delta=\frac{\mathbb{F}_{p}[y_{1},\cdots,y_{r}%
,C_{I}]^{+}}{\left\langle y_{1}^{k_{1}},\cdots,y_{r}^{k_{r}},R_{I}%
,S_{J,K}\right\rangle }$, where

\begin{quote}
$S_{J,K}=C_{J}\cdot C_{K}$ or $C_{J}\cdot C_{K}-C_{\left\langle
J,K\right\rangle }$
\end{quote}

\noindent in accordance to $J\cap K\neq\emptyset$ or $J\cap K=\emptyset
$.\hfill$\square$

\section{The algebra $H^{\ast}(G;\mathbb{F}_{p})$}

Granted with Theorem A we have deduce in \cite{DZ1} a presentation of the
$\operatorname{mod}p$ cohomology $H^{\ast}(G/T;\mathbb{F}_{p})$ and
determined, in the sequel, the structure of the algebra $H^{\ast}%
(G;\mathbb{F}_{p})$ as an Hopf algebra over the Steenrod algebra
$\mathcal{A}_{p}$. To facilitate calculation with the integral cohomology
$H^{\ast}(G)$ in the current work, we recall in this section some relevant
results from \cite{DZ1}. These will be applied in Theorem 3.7 to show that,
for a Lie group $G$ of rank $n$, $\dim H_{\beta}^{\ast}(G;\mathbb{F}%
_{p})=2^{n}$, where $H_{\beta}^{\ast}(G;\mathbb{F}_{p})$ is the
$\operatorname{mod}p$ Bockstein cohomology of $G$, and to deduce a closed
formula of $\operatorname{Im}\delta_{p}$, where $\delta_{p}$ is Bockstein
operator on $H^{\ast}(G;\mathbb{F}_{p})$.

\subsection{The algebra $H^{\ast}(G/T;\mathbb{F}_{p})$}

For a Lie group $G$ with rank $n$ and for a prime $p$, recall by Definition
1.3 that the degree sequence of the $p$-special Schubert classes on $G/T$ is

\begin{quote}
$D_{1}(G,p):=\{t_{i}\in\{t_{1},\cdots,t_{k}\},tor(y_{t_{i}})=p\}$.
\end{quote}

\noindent For $G\neq E_{8}$ we let $\overline{D}_{1}(G,p)$ be the complement
of $D_{1}(G,p)\subset\{t_{1},\cdots,t_{k}\}$, but define $\overline{D}%
_{1}(E_{8},p)$ to be the following subsets of $\{6,8,10,12,18,20,30\}$:

\begin{center}%
\begin{tabular}
[c]{l||l|l|l|l}\hline
$p$ & $2$ & $3$ & $5$ & $>5$\\\hline
$\overline{D}_{1}(E_{8},p)$ & $\{8\}$ & $\{6,10,18\}$ & $\{6,8,10,18\}$ &
$\{6,8,10,18,20\}$\\\hline
\end{tabular}
.
\end{center}

Since the ring $H^{\ast}(G/T)$ is torsion free, one can deduce from Theorem A
a formula of $H^{\ast}(G/T;\mathbb{F}_{p})$ using the isomorphism

\begin{quote}
$H^{\ast}(G/T;\mathbb{F}_{p})=H^{\ast}(G/T)\otimes\mathbb{F}_{p}$.
\end{quote}

\noindent Precisely, the epimorphism by (1.7)

\begin{quote}
$h_{p}\equiv h\operatorname{mod}p:\mathbb{F}_{p}[\omega_{1},\cdots,\omega
_{n},y_{t_{1}},\cdots,y_{t_{k}}]\rightarrow H^{\ast}(G/T;\mathbb{F}_{p})$.
\end{quote}

\noindent satisfies, by property i) of Theorem A, that

\begin{enumerate}
\item[(3.1)] $h_{p}(f_{i})\equiv-h_{p}(a_{i})$ if $t_{i}\in D_{1}(G,p)$,

\item[(3.2)] $h_{p}(y_{t_{i}})\equiv q_{i}h_{p}(a_{i})$ if $t_{i}\notin
D_{1}(G,p)$,
\end{enumerate}

\noindent where $q_{i}>0$ is the least prime such that $q_{i}\cdot p_{i}%
\equiv1\operatorname{mod}p$. It follows from (1.7) and (3.2) that $h_{p}$
restricts to an epimorphism

\begin{quote}
$\overline{h}_{p}:\mathbb{F}_{p}[\omega_{1},\cdots,\omega_{n},y_{t_{i}%
}]_{t_{i}\in D_{1}(G,p)}\rightarrow H^{\ast}(G/T;\mathbb{F}_{p})$
\end{quote}

\noindent such that

\begin{quote}
$\ker\overline{h}_{p}=\left\langle a_{i}^{(p)},y_{t_{i}}^{r_{i}}-b_{i}%
^{(p)},g_{s}^{(p)},e_{j}^{(p)}\right\rangle _{t_{i}\in D_{1}(G,p),t_{s}%
\in\overline{D}_{1}(G,p),1\leq j\leq h}$,
\end{quote}

\noindent where $a_{i}^{(p)},b_{i}^{(p)},g_{s}^{(p)},e_{j}^{(p)}$ are the
polynomials obtained respectively from the polynomials $a_{i},b_{i}%
,g_{s},e_{j}$ in Theorem A, by eliminating those special Schubert classes
$y_{t_{j}}$ with $t_{j}\notin D_{1}(G,p)$ using $q_{j}\cdot a_{j}%
\in\left\langle \omega_{1},\ldots,\omega_{n}\right\rangle $ by (3.2). This
shows that (compare also with \cite[Lemma 2.1]{DZ1}):

\bigskip

\noindent\textbf{Lemma 3.1.} \textsl{The map }$\overline{h}_{p}$
\textsl{induces an isomorphism of algebras}

\begin{enumerate}
\item[(3.3)] $H^{\ast}(G/T;\mathbb{F}_{p})=\frac{\mathbb{F}_{p}[\omega
_{1},\cdots,\omega_{n},\text{ }y_{t_{i}}]_{t_{i}\in D_{1}(G,p)}}{\left\langle
a_{i}^{(p)},\text{ }y_{t_{i}}^{r_{i}}-b_{i}^{(p)},\text{ }g_{s}^{(p)},\text{
}e_{j}^{(p)}\right\rangle _{t_{i}\in D_{1}(G,p),t_{s}\in\overline{D}%
_{1}(G,p),1\leq j\leq h}}$\textsl{,}
\end{enumerate}

\noindent\textsl{where, if we set }$\left\langle \omega_{1},\ldots,\omega
_{n}\right\rangle _{\mathbb{F}_{p}}:=\left\langle \omega_{1},\ldots,\omega
_{n}\right\rangle \otimes\mathbb{F}_{p}$\textsl{, then}

\begin{quote}
\textsl{i) }$a_{i}^{(p)},b_{i}^{(p)},g_{s}^{(p)},e_{j}^{(p)}\in\left\langle
\omega_{1},\ldots,\omega_{n}\right\rangle _{\mathbb{F}_{p}}$,

\textsl{ii)} $\#D_{1}(G,p)+\#\overline{D}_{1}(G,p)+h=n$\textsl{.}
\end{quote}

\noindent\textsl{In particular,} $\{a_{i}^{(p)},g_{s}^{(p)},e_{j}^{(p)}%
\}\in\left\langle \omega_{1},\ldots,\omega_{n}\right\rangle _{\mathbb{F}_{p}%
}\cap\ker\overline{h}_{p}$\textsl{.}\hfill$\square$

\bigskip

\noindent\textbf{Remark 3.2. }Let $D(G,p)$ be the degree set of the $n$
polynomials $\{a_{i}^{(p)},g_{s}^{(p)},e_{j}^{(p)}\}$ in (3.3). Since $\deg
a_{i}^{(p)}=t_{i}\in D_{1}(G,p)$ the set\textbf{ }$D(G,p)$ admits the partition

\begin{enumerate}
\item[(3.4)] $D(G,p)=D_{1}(G,p)\sqcup D_{2}(G,p)$,
\end{enumerate}

\noindent in which $D_{2}(G,p)=D(G,p)\cap2\cdot q(G)$ by iii) of Theorem A.
Consequently, the set $D(G,p)$ consists of precisely $n$ distinct integers.
For instance, we get from the contents of Tables 1 and 2 that

a) If either $G=SU(n)$, $Sp(n)$ or $p\notin\{2,3,5\}$, then

\begin{quote}
$D_{1}(G,p)=\emptyset$ and $D(G,p)=D_{2}(G,p)$ $=2\cdot q(G)$.
\end{quote}

b) If $G=E_{8}$ and $p\in\{2,3,5\}$, then the decomposition (3.4) is

\begin{quote}
$D(E_{8},2)=\left\{  6,10,18,30\right\}  \sqcup\{4,16,24,28\}$;

$D(E_{8},3)=\left\{  8,20\right\}  \sqcup\{4,16,28,36,40,48\}$;

$D(E_{8},5)=\left\{  12\right\}  \sqcup\{4,16,24,28,36,40,48\}$.\hfill
$\square$
\end{quote}

\subsection{The cohomology $H^{\ast}(G;\mathbb{F}_{p})$}

Using the $\operatorname{mod}p$ analogues of the maps $\mathcal{D}$ and
$\kappa$ introduced in Section \S 1.2:

\begin{quote}
$\mathcal{D}^{\prime}:\left\langle \omega_{1},\ldots,\omega_{n}\right\rangle
_{\mathbb{F}_{p}}\rightarrow E_{2}^{\ast,1}(G;\mathbb{F}_{p})=H^{\ast
}(G/T;\mathbb{F}_{p})\otimes\Lambda^{1}(t_{1},\cdots,t_{n}),$

$\kappa^{\prime}:$ $E_{3}^{2k,1}(G;\mathbb{F}_{p})\twoheadrightarrow
E_{\infty}^{2k,1}(G;\mathbb{F}_{p})\subset H^{2k+1}(G;\mathbb{F}_{p})$,
\end{quote}

\noindent we construct, in terms of the formula (3.3) of $H^{\ast
}(G/T;\mathbb{F}_{p})$, a set of generators of the algebra $H^{\ast
}(G;\mathbb{F}_{p})$.

\bigskip

\noindent\textbf{Definition 3.3.} Let $\pi_{p}^{\ast}:H^{\ast}(G/T;\mathbb{F}%
_{p})\rightarrow H^{\ast}(G;\mathbb{F}_{p})$ be the map induced by the bundle
map $\pi$. For a $p$-special Schubert classes $y_{t}$ define the
$\operatorname{mod}p$\textsl{ Schubert cocycles} on $G$ by

\begin{enumerate}
\item[(3.5)] $\overline{x}_{t}:=\pi_{p}^{\ast}(y_{t})\in H^{t}(G;\mathbb{F}%
_{p})$, $t\in D_{1}(G,p)$.
\end{enumerate}

By Remark 3.2, for an $m\in D(G,p)$ there exists a unique polynomial $c_{m}%
\in\{a_{i}^{(p)},g_{s}^{(p)},e_{j}^{(p)}\}$ with $\deg c_{m}=m$. Since
$c_{m}\in\ker h_{p}\cap\left\langle \omega_{1},\ldots,\omega_{n}\right\rangle
_{\mathbb{F}_{p}}$ by Lemma 3.1 the composition $\kappa^{\prime}\circ
\lbrack\mathcal{D}^{\prime}]$ is applicable to $c_{m}$ to yield the cohomology class

\begin{enumerate}
\item[(3.6)] $\zeta_{m-1}:=\kappa^{\prime}[\mathcal{D}^{\prime}(c_{m})]\in$
$H^{m-1}(G;\mathbb{F}_{p})$,
\end{enumerate}

\noindent to be called a $p$\textsl{-primary class} of $H^{\ast}%
(G;\mathbb{F}_{p})$. Note that if $m\in D_{1}(G,p)$, then $\zeta_{m-1}%
=\theta_{m-1}$ by (1.15).\hfill$\square$

\bigskip

In terms of the $\operatorname{mod}p$ Schubert cocycles $\overline{x}_{t_{i}}%
$, and the $p$-primary classes $\zeta_{m-1}$ just defined, the algebra
$H^{\ast}(G;\mathbb{F}_{p})$ is presented uniformly in the following result.

\bigskip

\noindent\textbf{Theorem 3.4.} \textsl{The inclusion }$\zeta_{m-1}$\textsl{,
}$x_{t}\in H^{\ast}(G;\mathbb{F}_{p})$\textsl{\ induces an isomorphism of
algebras}

\begin{enumerate}
\item[(3.7)] $H^{\ast}(G;\mathbb{F}_{p})=\operatorname{Im}\pi_{p}^{\ast
}\otimes\Delta(\zeta_{m-1})_{m\in D(G,p)}$\textsl{,}
\end{enumerate}

\noindent\textsl{where}

\textsl{i) }$\operatorname{Im}\pi_{p}^{\ast}=\mathbb{F}_{p}[\overline{x}%
_{t}]/\left\langle \overline{x}_{t}^{r_{t}}\right\rangle _{t\in D_{1}(G,p)}%
$\textsl{,} $r_{t}=cl(y_{t})$\textsl{,}

\textsl{ii) if }$p\neq2$\textsl{ the group} $\Delta(\zeta_{m-1})_{m\in
D(G,p)}$ \textsl{can be replaced} \textsl{by the exterior algebra}
$\Lambda(\zeta_{m-1})_{m\in D(G,p)}$\textsl{;}

\textsl{iii) if }$p=2$ \textsl{and} $G$\textsl{ is exceptional, then }%
$\zeta_{m-1}^{2}=0$\textsl{ with the following exceptions}

\begin{quote}
$\zeta_{3}^{2}=\overline{x}_{6}$\textsl{ for }$G=G_{2},F_{4},E_{6},E_{7}%
,E_{8}$\textsl{;}

$\zeta_{5}^{2}=\overline{x}_{10},\quad\zeta_{9}^{2}=\overline{x}_{18}$\textsl{
for }$G=E_{7},E_{8}$\textsl{;}

$\zeta_{15}^{2}=\overline{x}_{30},$\textsl{ }$\zeta_{23}^{2}=\overline{x}%
_{6}^{6}\overline{x}_{10}$\textsl{ for }$G=E_{8}$\textsl{.}
\end{quote}

\noindent\textbf{Proof.} It have been shown in \cite[Lemma 3.2]{DZ1} that

\begin{enumerate}
\item[(3.8)] $H^{\ast}(G;\mathbb{F}_{p})=\operatorname{Im}\pi_{p}^{\ast
}\otimes\Delta(\alpha_{m-1})_{m\in D(G,p)}$,

$\operatorname{Im}\pi_{p}^{\ast}=\mathbb{F}_{p}[\overline{x}_{t}]/\left\langle
\overline{x}_{t}^{r_{t}}\right\rangle _{t\in D_{1}(G,p)}$,
\end{enumerate}

\noindent where $\{\alpha_{m-1},m\in D(G,p)\}$ is a set of $p$%
\textsl{-transgressive generators} of $H^{\ast}(G;\mathbb{F}_{p})$ (see
\cite[Definition 3.3]{DZ1}). It has also been shown that each $p$-primary
classes $\zeta_{m-1}$ is a $\operatorname{Im}\pi_{p}^{\ast}$-linear
combination of the $p$-transgressive ones with leading term $\alpha_{m-1}$
(see \cite[formula (4.2)]{DZ1})

\begin{quote}
$\zeta_{m-1}=\alpha_{m-1}+\underset{i<m,i\in D(G,p)}{\Sigma}g_{i}\cdot
\alpha_{i-1}$, $g_{i}\in\operatorname{Im}\pi_{p}^{\ast}$.
\end{quote}

\noindent Therefore, we obtain (3.7), together with the formula of
$\operatorname{Im}\pi_{p}^{\ast}$, from (3.8).

Assume that $p\neq2$. Since $\deg\zeta_{m-1}\equiv1\operatorname{mod}2$ we get
ii) from $\zeta_{m-1}^{2}\equiv0$ by $2\zeta_{m-1}^{2}\equiv
0\operatorname{mod}p$.

To show iii) assume that $p=2$ and that $G$ is exceptional. By \cite[Theorem
4.1]{DZ1} we have $\zeta_{2s-1}=\alpha_{2s-1}$\ with the following exceptions:

\begin{quote}
$\zeta_{15}=\alpha_{15}+\overline{x}_{6}\alpha_{9}$\textsl{; }$\zeta
_{27}=\alpha_{27}+\overline{x}_{10}\alpha_{17}$\textsl{\ }in\textsl{ }%
$E_{7},E_{8}$\textsl{,}

$\zeta_{23}=\alpha_{23}+\overline{x}_{6}\alpha_{17}$\textsl{\ }in\textsl{
}$E_{7}$\textsl{;}

$\zeta_{23}=\alpha_{23}+\overline{x}_{6}\alpha_{17}+x_{6}^{3}\alpha_{5}%
$\textsl{; }$\quad\zeta_{29}=\alpha_{29}+\overline{x}_{6}^{2}\alpha_{17}%
$\textsl{\ }in\textsl{ }$E_{8}$\textsl{.}
\end{quote}

\noindent The squares $\zeta_{m-1}^{2}$ stated in iii) are verified by the
formula in \cite[(4.8)]{DZ1}:

\begin{quote}
$\alpha_{2s-1}^{2}=\left\{
\begin{tabular}
[c]{l}%
$\overline{x}_{6}$ for $s=2$,\\
$\overline{x}_{4s-2}$ for $s=3,5$ and in $E_{7},E_{8}$,\\
$\overline{x}_{30}+\overline{x}_{6}^{2}\overline{x}_{18}$ for $s=8$ and in
$E_{8},$\\
$0$ in the remaining cases.
\end{tabular}
\right.  $.\hfill$\square$
\end{quote}

\subsection{The Bockstein operator $\delta_{p}$ on $H^{\ast}(G;\mathbb{F}%
_{p})$}

In term of the exact sequence (2.5) the\textsl{ }$\operatorname{mod}p$
\textsl{Bockstein operator }on the algebra\textsl{ }$H^{\ast}(G;\mathbb{F}%
_{p})$ is the composition

\begin{quote}
$\delta_{p}=r_{p}\circ\beta_{p}:$ $H^{\ast}(G;\mathbb{F}_{p})\rightarrow
H^{\ast}(G;\mathbb{F}_{p})$
\end{quote}

\noindent which satisfies, as a differential of degree $1$, that

\begin{enumerate}
\item[(3.9)] $\delta_{p}(a\cup r_{p}(c))=\delta_{p}(a)\cup r_{p}(c),a\in
H^{\ast}(G;\mathbb{F}_{p})$, $c\in H^{\ast}(G)$.
\end{enumerate}

\noindent The $\operatorname{mod}p$ Bockstein cohomology $H_{\beta}^{\ast
}(G;\mathbb{F}_{p})$ of $G$ is defined to be the graded quotient group
$\ker\delta_{p}/\operatorname{Im}\delta_{p}$. As preparation to compute $\dim
H_{\beta}^{\ast}(G;\mathbb{F}_{p})$ and the subgroup $\operatorname{Im}%
\delta_{p}\subset H^{\ast}(G;\mathbb{F}_{p})$, we clarify the
$\operatorname{mod}p$ reduction $r_{p}$, as well as the Bockstein\textsl{
}$\beta_{p}$, in the next two lemmas.

\bigskip

\noindent\textbf{Lemma 3.5.} \textsl{With respect to the presentation (3.7)
of} $H^{\ast}(G;\mathbb{F}_{p})$\textsl{,} \textsl{the reduction }$r_{p}$
\textsl{satisfies that}

\begin{quote}
\textsl{i)} $r_{p}(x_{t_{i}})\equiv\overline{x}_{t_{i}}$ \textsl{if} $t_{i}\in
D_{1}(G,p)$\textsl{, }$0$ \textsl{otherwise;}

\textsl{ii)} $r_{p}(\rho_{l(e_{j})-1})\equiv\zeta_{l(e_{j})-1}$ \textsl{for
}$1\leq j\leq h$\textsl{;}

\textsl{iii)} $r_{p}(\rho_{l(g_{i})-1})\equiv-\overline{x}_{t_{i}}^{r_{i}%
-1}\cdot\theta_{t_{i}-1}$ \textsl{for} $t_{i}\in D_{1}(G,p)$\textsl{;}

\textsl{iv)} $r_{p}(\rho_{l(g_{i})-1})\equiv-q_{i}\cdot\zeta_{l(g_{i})-1}$
\textsl{for} $t_{i}\notin D_{1}(G,p)$\textsl{;}

\textsl{v)} $r_{p}(\mathcal{C}_{I})\equiv\delta_{p}(\theta_{I})$\textsl{ for
}$I\subseteq D_{1}(G,p)$\textsl{,}
\end{quote}

\noindent\textsl{where }$r_{i}=cl(y_{t_{i}}),$ $\theta_{t-1}=\zeta_{t-1}%
$\textsl{, and where }$q_{i}$\textsl{ is co-prime to }$p$\textsl{.}

\bigskip

\noindent\textbf{Proof.} If $t_{i}\in D_{1}(G,p)$ we get $r_{p}(x_{t_{i}%
})=\overline{x}_{t_{i}}$ by $r_{p}\circ\pi^{\ast}=\pi_{p}^{\ast}\circ r_{p}$.
On the other hand, if $t_{i}\notin D_{1}(G,p)$, we find by (3.2) that

\begin{quote}
$r_{p}(x_{t_{i}})\equiv\pi_{p}^{\ast}\circ r_{p}(y_{k_{i}})\equiv\pi_{p}%
^{\ast}(q_{i}a_{i})\equiv0$,
\end{quote}

\noindent where the last equality follows from $a_{i}\in\left\langle
\omega_{1},\cdots,\omega_{n}\right\rangle =\ker\pi^{\ast}$. This shows i).

The relations ii), iii) and iv) will be proven in the same way. The obvious
relations in $E_{2}^{\ast,0}(G;\mathbb{F}_{p})=H^{\ast}(G/T;\mathbb{F}_{p})$

\begin{quote}
$r_{p}(e_{j})\equiv e_{j}^{(p)}$ for $1\leq j\leq h$,

$r_{p}(y_{t_{i}}^{r_{i}-1}\cdot f_{i}-p_{i}\cdot g_{i})\equiv-y_{i}^{r_{i}%
-1}\cdot a_{i}^{(p)}$ for $t_{i}\in D_{1}(G,p)$,

$r_{p}(y_{t_{i}}^{r_{i}-1}\cdot f_{i}-p_{i}\cdot g_{i})\equiv-q_{i}\cdot
g_{i}^{(p)}$ for $t_{i}\notin D_{1}(G,p)$
\end{quote}

\noindent go over, respectively, to the following relations in $E_{3}^{\ast
,1}(G;\mathbb{F}_{p})$

\begin{quote}
a) $r_{p}[\mathcal{D}(e_{j})]\equiv\lbrack\mathcal{D}^{\prime}(e_{j}^{(p)})]$,

b) $r_{p}[\mathcal{D}(y_{t_{i}}^{r_{i}-1}\cdot f_{i}-p_{i}\cdot g_{i}%
)]\equiv-[y_{i}^{r_{i}-1}\cdot\mathcal{D}^{\prime}(a_{i}^{(p)})]$ if $t_{i}\in
D_{1}(G,p)$,

c) $r_{p}[\mathcal{D}(y_{t_{i}}^{r_{i}-1}\cdot f_{i}-p_{i}\cdot g_{i}%
)]\equiv-q_{i}[\mathcal{D}^{\prime}(g_{i}^{(p)})]$ if $t_{i}\notin D_{1}(G,p)$.
\end{quote}

\noindent Since the map $\kappa$ is natural with respect to the reduction
$r_{p}$ (i.e. $r_{p}\circ\kappa=\kappa^{\prime}\circ r_{p}$), applying
$\kappa^{\prime}$ to both sides of a), b) and c) we obtain ii), iii) and iv)
respectively from the definition (3.6) of the classes $\zeta_{m-1}$. Note
that, in the case iii), the formula (2.8) is required to derive that

\begin{quote}
$\kappa^{\prime}[y_{i}^{r_{i}-1}\cdot\mathcal{D}^{\prime}(a_{i}^{(p)})]$
$=\overline{x}_{t_{i}}^{r_{i}-1}\cdot\theta_{t_{i}-1}$.
\end{quote}

\noindent Finally, relation v) follows from the definition (1.16) of
$\mathcal{C}_{I}$.\hfill$\square$

\bigskip

\noindent\textbf{Lemma 3.6. }\textsl{With respect to the presentation (3.7)
of} $H^{\ast}(G;\mathbb{F}_{p})$\textsl{, the Bockstein homomorphism }%
$\beta_{p}$\textsl{ satisfies that}

\begin{quote}
\textsl{i)} $\beta_{p}(\theta_{t_{i}-1})=x_{t_{i}}$ \textsl{for} $t_{i}\in
D_{1}(G,p)$\textsl{;}

\textsl{ii)} $\beta_{p}(\zeta_{m-1})=0$ \textsl{for }$m\in D_{2}%
(G,p)$\textsl{;}

\textsl{iii)} $\beta_{p}(\overline{x}_{t_{i}})=0$ \textsl{for }$t_{i}\in
D_{1}(G,p)$\textsl{.}
\end{quote}

\noindent\textbf{Proof. }If $t_{i}\in D_{1}(G,p)$ the relation $f_{i}=p\cdot
y_{t_{i}}-a_{i}$ on $H^{\ast}(G/T)$ (see in Theorem A) gives rise to the
diagram chasing in the short exact sequence (2.6) in $d_{2}$-complexes:

\begin{quote}
$%
\begin{array}
[c]{ccccc}
&  & \mathcal{D}(a_{i}) & \rightarrow & \mathcal{D}^{\prime}(a_{i}^{(p)})\\
&  & d_{2}\downarrow\quad &  & d_{2}\downarrow\quad\\
y_{t_{i}} & \overset{p}{\longrightarrow} & a_{i} &  & 0
\end{array}
$.
\end{quote}

\noindent It implies in the exact sequence (2.7) that $y_{t_{i}}%
=\overline{\beta}_{p}[\mathcal{D}^{\prime}(a_{i}^{(p)})]$ by the convention
$y_{t_{i}}=[y_{t_{i}}\otimes1]$ in Lemma 2.1. Applying $\pi^{\ast}$ to both
sides we get i) from the relation (2.9).

For ii) consider a $p$-primary class $\zeta_{m-1}\in H^{\ast}(G;\mathbb{F}%
_{p})$ with $m\in D_{2}(G,p)$. By properties ii) and iv) of Lemma 3.4, the
integral primary class $\rho_{m-1}$ satisfies $r_{p}(\rho_{m-1})=q\cdot
\zeta_{m-1}$, where $q$ is co-prime to $p$. We obtain ii) from the exactness
of (2.5).

Finally, with $r_{p}(x_{t_{i}})=\overline{x}_{t_{i}}$ for $t_{i}\in
D_{1}(G,p)$ (by i) of Lemma 3.5), we obtain iii) by the exactness of (2.5),
too.\hfill$\square$

\bigskip

For a subsequence $I\subseteq D_{1}(G,p)$ we put $C_{I}=\delta_{p}(\theta
_{I})$. Accordingly, introduce in the polynomial algebra $\mathbb{F}%
_{p}[\overline{x}_{t_{i}},C_{I}]^{+}$ the following elements

\begin{quote}
$R_{I}=(\Pi_{t_{i}\in I}\overline{x}_{t_{i}}^{r_{i}-1})C_{I}$;

$S_{J,K}=C_{J}C_{K}-\underset{i_{s}\in J=\{i_{1},\cdots,i_{h})}{\Sigma
}(-1)^{s-1}\overline{x}_{i_{s}}b_{J_{s}\cap K}\cdot C_{\left\langle
J_{s},K\right\rangle }$,
\end{quote}

\noindent where $b_{I}=\Pi_{t\in I}\zeta_{t-1}^{2}\in\operatorname{Im}\pi
_{p}^{\ast}$ (see (3.12) and (3.13) below). Applying Theorem 2.5, together
with Lemmas 3.5 and 3.6, we prove the main result of this section.

\bigskip

\noindent\textbf{Theorem 3.7. }\textsl{For a Lie group }$G$\textsl{ of rank
}$n$\textsl{ we have}

\textsl{i)} $\dim H_{\beta}^{\ast}(G;\mathbb{F}_{p})=2^{n}$\textsl{. }

\textsl{ii) the algebra }$\operatorname{Im}\delta_{p}$ \textsl{has the
presentation}

\begin{enumerate}
\item[(3.10)] $\operatorname{Im}\delta_{p}=\frac{\mathbb{F}_{p}[\overline
{x}_{t_{i}},C_{I}]^{+}}{\left\langle \overline{x}_{t_{i}}^{r_{i}}%
,R_{I},S_{J,K}\right\rangle }\otimes\Delta(\zeta_{m-1})_{m\in D_{2}(G,p)}%
$\textsl{,}
\end{enumerate}

\noindent\textsl{where }$t_{i}\in D_{1}(G,p)$\textsl{,} \textsl{and where}
$I,J,K\subseteq D_{1}(G,p)$\textsl{ are }$p$\textsl{-monotone.}

\bigskip

\noindent\textbf{Proof. }In term of (3.7) the algebra $H^{\ast}(G;\mathbb{F}%
_{p})$ has the decomposition

\begin{enumerate}
\item[(3.11)] $H^{\ast}(G;\mathbb{F}_{p})=(\operatorname{Im}\pi_{p}^{\ast
}\otimes\Delta(\zeta_{t_{i}-1})_{t_{i}\in D_{1}(G,p)})\otimes\Delta
(\zeta_{m-1})_{m\in D_{2}(G,p)}$
\end{enumerate}

\noindent on which the $\delta_{p}$-action satisfies, by Lemmas 3.5 and 3.6, that

\begin{quote}
$\delta_{p}(x)=0$ for $x\in\operatorname{Im}\pi_{p}^{\ast}$,

$\delta_{p}(\zeta_{t_{i}-1})=\overline{x}_{t_{i}}\in\operatorname{Im}\pi
_{p}^{\ast}$ for $t_{i}\in D_{1}(G,p)$,

$\delta_{p}(\zeta_{m-1})=0$ for $m\in D_{2}(G,p)$.
\end{quote}

\noindent These imply that

\begin{quote}
a) the factor $\operatorname{Im}\pi_{p}^{\ast}\otimes\Delta(\zeta_{t_{i}%
-1})_{t_{i}\in D_{1}(G,p)}$ is an invariant subspace of $\delta_{p}$, which
can be identified with the Koszul complex $(K(\operatorname{Im}\pi_{p}^{\ast
}),\delta)$ associated to the polynomial algebra $\operatorname{Im}\pi
_{p}^{\ast}$ (see in Section \S 2.2).

b) $\Delta(\zeta_{m-1})_{m\in D_{2}(G,p)}$ is an invariant subspace on which
$\delta_{p}=0$.
\end{quote}

\noindent In particular,

\begin{quote}
c) $\operatorname{Im}\delta_{p}=\operatorname{Im}\delta\otimes\Delta
(\zeta_{m-1})_{m\in D_{2}(G,p)}$.
\end{quote}

To see i) we note by a) and b) that

\begin{quote}
$H_{\beta}^{\ast}(G;\mathbb{F}_{p})=H^{\ast}(K(\operatorname{Im}\pi_{p}^{\ast
}),\delta)\otimes\Delta(\zeta_{m-1})_{m\in D_{2}(G,p)}$.
\end{quote}

\noindent We obtain i) form $\dim H^{\ast}(K(\operatorname{Im}\pi_{p}^{\ast
});\mathbb{\delta})=2^{\left\vert D_{1}(G,p)\right\vert }$ by Theorem 2.4, and
$\left\vert D_{1}(G,p)\right\vert +\left\vert D_{2}(G,p)\right\vert =n$ by
Example 3.2.

For ii) we note that the cup product on $H^{\ast}(G;\mathbb{F}_{p})$ furnishes
the complex $K(\operatorname{Im}\pi_{p}^{\ast})$ with the canonical
differential algebra structure $b=\{\zeta_{m-1}^{2}$, $m\in D_{1}(G,p)\}$ in which

\begin{enumerate}
\item[(3.12)] $\zeta_{m-1}^{2}=0\in\operatorname{Im}\pi_{p}^{\ast}$ if
$p\equiv1\operatorname{mod}2$ (see Remark 2.6);

\item[(3.13)] $\zeta_{m-1}^{2}\in\operatorname{Im}\pi_{2}^{\ast}$ if $p=2$ (by
the $\operatorname{mod}p$ analogue of Lemma 2.2).
\end{enumerate}

\noindent Thus, we obtain formula (3.10) by the relation c) and formula
(2.13).\hfill$\square$

\section{The structure of the ring $H^{\ast}(G)$}

In terms of the cohomology classes $x_{t_{i}}$, $\rho_{2l-1}$ and
$\mathcal{C}_{I}$ constructed in Section \S 1.2, we establish in Theorems 4.1,
4.4 and 4.6 general expressions of the three basic components of the
cohomology $H^{\ast}(G)$: the subring $\operatorname{Im}\pi^{\ast}$, the free
part $\mathcal{F}(G)$ and the torsion ideal $\tau_{p}(G)$, respectively. These
results are summarized in Section \S 4.4 to give a proof of Theorem B.

\subsection{The subring $\operatorname{Im}\pi^{\ast}\subset H^{\ast}(G)$}

For a prime $p$ the proof of Lemma 2.1 is valid to show by (3.3) that

\begin{quote}
$E_{3}^{\ast,0}(G;\mathbb{F}_{p})=\mathbb{F}_{p}[y_{t_{i}}]/\left\langle
y_{t_{i}}^{r_{i}}\right\rangle _{i\in D_{1}(G,p)}$.
\end{quote}

\noindent Moreover, with $\pi_{p}^{\ast}(y_{t_{i}})=\overline{x}_{t_{i}}$ by
(3.5) we find by formula (3.7) that the map $\pi_{p}^{\ast}$ induces an isomorphism

\begin{quote}
$E_{3}^{\ast,0}(G;\mathbb{F}_{p})\overset{\cong}{\rightarrow}\operatorname{Im}%
\pi_{p}^{\ast}=\mathbb{F}_{p}[\overline{x}_{t_{i}}]/\left\langle \overline
{x}_{t_{i}}^{r_{t_{i}}}\right\rangle _{i\in D_{1}(G,p)}$.
\end{quote}

\noindent Thus, we can deduce from Lemma 2.1, as well as the relation $\pi
_{p}^{\ast}\circ r_{p}=r_{p}\circ\pi^{\ast}$ on $E_{3}^{\ast,0}(G)$, the
following result.

\bigskip

\noindent\textbf{Theorem 4.1. }\textsl{The map }$\pi^{\ast}:E_{3}^{\ast
,0}(G)\rightarrow\operatorname{Im}\pi^{\ast}$ \textsl{in (2.2) is an
isomorphism.}

\textsl{In particular, if }$\left\{  y_{t_{1}},\cdots,y_{t_{k}}\right\}
$\textsl{ is a set of special Schubert classes on }$G/T$\textsl{,}

\begin{enumerate}
\item[(4.1)] $\operatorname{Im}\pi^{\ast}=\mathbb{Z}[x_{t_{1}},\cdots
,x_{t_{k}}]/\left\langle p_{i}x_{t_{i}},x_{t_{i}}^{r_{i}}\right\rangle _{1\leq
i\leq k}$
\end{enumerate}

\noindent\textsl{where} $p_{i}=tor(y_{t_{i}})$\textsl{,} $r_{i}=cl(y_{t_{i}}%
)$\textsl{.}\hfill$\square$

\bigskip

It is straightforward to see from (4.1) that

\bigskip

\noindent\textbf{Corollary 4.2. }\textsl{The reduction }$r_{p}:H^{\ast
}(G)\rightarrow H^{\ast}(G;\mathbb{F}_{p})$\textsl{ restricts to an
isomorphism}

\begin{enumerate}
\item[(4.2)] $r_{p}:\operatorname{Im}\pi^{\ast}\cap\tau_{p}(G)\rightarrow
\operatorname{Im}(\pi_{p}^{\ast})^{+}$.\hfill$\square$
\end{enumerate}

\bigskip

\noindent\textbf{Example 4.3.} In \cite{G} Grothendieck defined \textsl{Chow
ring} of the reductive algebraic group $G^{c}$ corresponding to $G$ to be the subring

\begin{quote}
$\mathcal{A}(G):=\operatorname{Im}\left\{  \pi^{\ast}:H^{\ast}(G/T)\rightarrow
H^{\ast}(G)\right\}  $.
\end{quote}

\noindent Since the Chow ring $\mathcal{A}(G/T)$ of the projective variety
$G/T$ is canonically isomorphic to the integral cohomology $H^{\ast}(G/T)$,
the subring $\mathcal{A}(G)$ of $H^{\ast}(G)$ consists of those elements of
$H^{\ast}(G)$ that can be realized by the algebraic cocycles on $G/T$ via the
map $\pi^{\ast}$.

In view of formula (4.1), the degree sequence $\left\{  t_{1},\cdots
,t_{k}\right\}  $, the torsion indices $tor(y_{t_{i}})$ and the cup lengths
$cl(y_{t_{i}})$ are the numerical invariants of $G$ required to express the
ring $\mathcal{A}(G)=\operatorname{Im}\pi^{\ast}$ algebraically. In
particular, inputting the values of these invariants given by Table 2, we get
the following expressions of the Chow ring $\mathcal{A}(G)$ in terms of the
Schubert cocycles on $G$.

\begin{center}
{\normalsize Table 3. The Chow rings of exceptional Lie groups}%

\begin{tabular}
[c]{l||l}\hline
$G$ & $\mathcal{A}(G)=\operatorname{Im}\pi^{\ast}\{H^{\ast}(G/T)\rightarrow
H^{\ast}(G)\}$\\\hline\hline
$G_{2}$ & $\frac{\mathbb{Z}[x_{6}]}{\left\langle 2x_{6},x_{6}^{2}\right\rangle
}$\\\hline
$F_{4}$ & $\frac{\mathbb{Z}[x_{6},x_{8}]}{\left\langle 2x_{6},x_{6}^{2}%
,3x_{8},x_{8}^{3}\right\rangle }$\\\hline
$E_{6}$ & $\frac{\mathbb{Z}[x_{6},x_{8}]}{\left\langle 2x_{6},x_{6}^{2}%
,3x_{8},x_{8}^{3}\right\rangle }$\\\hline
$E_{7}$ & $\frac{\mathbb{Z}[x_{6},x_{8},x_{10},x_{18}]}{\left\langle
2x_{6},3x_{8},2x_{10},2x_{18},x_{6}^{2},x_{8}^{3},x_{10}^{2},x_{18}%
^{2}\right\rangle }$\\\hline
$E_{8}$ & $\frac{\mathbb{Z}[x_{6},x_{8},x_{10},x_{12},x_{18},x_{20},x_{30}%
]}{\left\langle 2x_{6},3x_{8},2x_{10},5x_{12},2x_{18},3x_{20},2x_{30}%
,x_{6}^{8},x_{8}^{3},x_{10}^{4},x_{12}^{5},x_{18}^{2},x_{20}^{3},x_{30}%
^{2}\right\rangle }$\\\hline
\end{tabular}
.
\end{center}

\subsection{The free part $\mathcal{F}(G)\subseteq H^{\ast}(G)$}

According to Example 1.14, if $G$ is a Lie group with rank $n$, the totality
of the primary classes of $G$ can be denoted by $\{\rho_{2l-1}\in H^{\ast
}(G),l\in q(G)\}$.

\bigskip

\noindent\textbf{Theorem 4.4. }$\mathcal{F}(G)=\Delta(\rho_{2l_{1}-1}%
,\cdots,\rho_{2l_{n}-1})$\textsl{, where} $q(G)=\{l_{1},\cdots,l_{n}\}.$

\bigskip

\noindent\textbf{Proof. }Recall that the group $\Delta(\rho_{2l_{1}-1}%
,\cdots,\rho_{2l_{n}-1})$ has the basis consisting of all the square free
monomials in $\rho_{2l_{1}-1},\cdots,\rho_{2l_{n}-1}$:

\begin{quote}
$\Phi=\{1,\rho_{I}=\rho_{2l_{s_{1}}-1}\cdots\rho_{2l_{s_{k}}-1},$
$I=\{s_{1},\ldots,s_{k}\}\subseteq\{1,2,\ldots,n\}\}$.
\end{quote}

\noindent It suffices to show that $\Phi$ is a basis of $\mathcal{F}(G)$.
Assume that $m=\dim G$.

By (1.17) the full product $\rho=\rho_{2l_{1}-1}\cdots\rho_{2l_{n}-1}\in\Phi$
belongs to the top degree cohomology group $H^{m}(G)=\mathbb{Z}$. On the other
hand, by formula (3.7), for any prime $p$ the group $H^{m}(G;\mathbb{F}%
_{p})=\mathbb{F}_{p}$ is generated by the product

\begin{quote}
$\omega_{p}=\underset{i\in\Gamma(G,p)}{\Pi}\overline{x}_{t_{i}}^{r_{t_{i}}%
-1}\underset{m\in D(G,p)}{\Pi}\zeta_{m-1}$.
\end{quote}

\noindent From ii), iii) and iv) of Lemma 3.5 one finds that

\begin{quote}
$r_{p}(\rho)=r_{p}(\rho_{2l_{1}-1})\cdot\cdots\cdot r_{p}(\rho_{2l_{n}%
-1})\equiv q\cdot\omega_{p}$,
\end{quote}

\noindent where $q$ is an integer co-prime to $p$. This shows that $\rho$ is a
generator of the group $H^{m}(G)=\mathbb{Z}$. In particular, the set $\Phi$ of
$2^{n}$ monomials is linearly independent in $H^{\ast}(G)$. Since $\dim
H^{\ast}(G)\otimes\mathbb{Q=}2^{n}$ it suffices to show that the set $\Phi$
spans a direct summand of $H^{\ast}(G)$.

Assume, on the contrary, that there exist a monomial $\rho_{I}\in\Phi$, a
class $\xi\in H^{\ast}(G)$, as well as some integer $a>1$, so that a relation
of the form $\rho_{I}=a\cdot\xi$ holds in $H^{\ast}(G)$. Letting $\overline
{I}$ be the complement of $I\subseteq\{1,\cdots,n\}$, and multiplying both
sides by the class $\rho_{\overline{I}}$ yield

\begin{quote}
$\rho=(-1)^{r}a\cdot(\xi\cup\rho_{\overline{I}})$ for some $r\in\mathbb{Z}$.
\end{quote}

\noindent This contradicts to that $\rho$ is a generator of $H^{m}%
(G)=\mathbb{Z}$.\hfill$\square$

\bigskip

\noindent\textbf{Example 4.5.} As a supplement to Theorem 4.3 we show that, if
$G$ is an exceptional Lie group, then

\begin{quote}
$\mathcal{F}(G_{2})=\Delta(\varrho_{3})\otimes\Lambda(\varrho_{11})$

$\mathcal{F}(F_{4})=\Delta(\varrho_{3})\otimes\Lambda(\varrho_{11}%
,\varrho_{15},\varrho_{23})$

$\mathcal{F}(E_{6})=\Delta(\varrho_{3})\otimes\Lambda(\varrho_{9},\varrho
_{11},\varrho_{15},\varrho_{17},\varrho_{23})$

$\mathcal{F}(E_{7})=\Delta(\varrho_{3})\otimes\Lambda(\varrho_{11}%
,\varrho_{15},\varrho_{19},\varrho_{23},\varrho_{27},\varrho_{35})$

$\mathcal{F}(E_{8})=\Delta(\varrho_{3},\varrho_{15},\varrho_{23}%
)\otimes\Lambda(\varrho_{27},\varrho_{35},\varrho_{39},\varrho_{47}%
,\varrho_{59})$
\end{quote}

\noindent where

\begin{quote}
$\rho_{3}^{2}=x_{6}$ for $G=G_{2},F_{4},E_{6},E_{7},E_{8}$;

$\rho_{15}^{2}=x_{30}$; $\rho_{23}^{2}=x_{6}^{6}x_{10}$ for $G=E_{8}$.
\end{quote}

The results for the cases $G\neq E_{8}$ are fairly straightforward. We may
therefore focus on the relatively nontrivial case $G=E_{8}$, for which we
have, by the contents in the last column of Table 2, that

\begin{quote}
$\tau_{2}(E_{8})\cap\operatorname{Im}\pi^{\ast}=\frac{\mathbb{F}_{2}%
[x_{6},x_{10},x_{18},x_{30}]^{+}}{\left\langle x_{6}^{8},x_{10}^{4},x_{18}%
^{2},x_{30}^{2}\right\rangle }$.
\end{quote}

\noindent Since $\rho_{2l-1}^{2}\in\tau_{2}(E_{8})\cap\operatorname{Im}%
\pi^{\ast}$ by Lemma 2.2 and since, by Corollary 4.2, $r_{2}$ maps $\tau
_{2}(E_{8})\cap\operatorname{Im}\pi^{\ast}$ isomorphically onto

\begin{quote}
$(\operatorname{Im}\pi_{2}^{\ast})^{+}=$ $\frac{\mathbb{F}_{2}[\overline
{x}_{6},\overline{x}_{10},\overline{x}_{18},\overline{x}_{30}]^{+}%
}{\left\langle \overline{x}_{6}^{8},\overline{x}_{10}^{4},\overline{x}%
_{18}^{2},\overline{x}_{30}^{2}\right\rangle }$,
\end{quote}

\noindent the squares $\rho_{2l-1}^{2}$ are determined by $r_{2}(\rho
_{2l-1})^{2}$. The proof is completed by the results tabulated below

\begin{center}%
\begin{tabular}
[c]{l||l|l|l|l|l|l|l|l}%
$\rho_{2l-1}$ & $\rho_{3}$ & $\rho_{15}$ & $\rho_{23}$ & $\rho_{27}$ &
$\rho_{35}$ & $\rho_{39}$ & $\rho_{47}$ & $\rho_{59}$\\\hline\hline
$r_{2}(\rho_{2l-1})$ & $\zeta_{3}$ & $\zeta_{15}$ & $\zeta_{23}$ & $\zeta
_{27}$ & $\overline{x}_{18}\zeta_{17}$ & $\overline{x}_{10}^{3}\zeta_{9}$ &
$\overline{x}_{6}^{7}\zeta_{5}$ & $\overline{x}_{30}\zeta_{29}$\\\hline
$r_{2}(\rho_{2l-1})^{2}$ & $\overline{x}_{6}$ & $\overline{x}_{15}$ &
$\overline{x}_{6}^{6}\overline{x}_{10}$ & $0$ & $0$ & $0$ & $0$ & $0$\\\hline
\end{tabular}
,
\end{center}

\noindent where the contents in the second row follow from properties ii),
iii) and iv) of Lemma 3.5, and where the results in the third row have been
shown in iii) of Theorem 3.4.\hfill$\square$\hfill

\subsection{The torsion ideal $\tau_{p}(G)\subset H^{\ast}(G)$}

Let $p$ be a prime. By Theorem 4.1, for each $t_{i}\in D_{1}(G,p)$ the
Schubert cocycle $x_{t_{i}}\in\operatorname{Im}\pi^{\ast}$ satisfies
$x_{t_{i}}\in\tau_{p}(G)$. As in (1.16), for each $p$-monotone sequence
$I\subseteq D_{1}(G,p)$ we put

\begin{quote}
$\mathcal{C}_{I}=\beta_{p}(\theta_{I})\in\tau_{p}(G)$.
\end{quote}

\noindent Accordingly, consider in the polynomial algebra $\mathbb{F}%
_{p}[x_{t_{i}},\mathcal{C}_{I}]_{t_{i}\in D_{1}(G,p),\text{ }I\subseteq
D_{1}(G,p)}^{+}$ the following elements

\begin{enumerate}
\item[(4.3)] $\mathcal{R}_{I}=(\Pi_{t_{i}\in I}x_{t_{i}}^{r_{i}-1}%
)\mathcal{C}_{I}$,

\item[(4.4)] $\mathcal{S}_{J,K}=\mathcal{C}_{J}\mathcal{C}_{K}-\underset
{i_{s}\in J=\{i_{1},\cdots,i_{h})}{\Sigma}(-1)^{s-1}x_{i_{s}}\cdot
b_{J_{s}\cap K}^{\ast}\cdot\mathcal{C}_{\left\langle J_{s},K\right\rangle }$,
\end{enumerate}

\noindent where $r_{j}=cl(y_{t_{j}})$, and where $b_{H}^{\ast}\in
\operatorname{Im}\pi^{\ast}\cap\tau_{p}(G)$ is the unique element that
satisfies by Corollary 4.2 the relation (see (3.12) and (3.13))

\begin{quote}
$r_{p}(b_{H}^{\ast})=b_{H}$ ($=\Pi_{t\in H}\zeta_{t-1}^{2}$) $\in
(\operatorname{Im}\pi_{p}^{\ast})^{+}$.
\end{quote}

\bigskip

\noindent\textbf{Theorem 4.6. }\textsl{For a Lie group }$G$\textsl{ and prime
}$p$\textsl{ we have}

\begin{enumerate}
\item[(4.5)] $\tau_{p}(G)=\frac{\mathbb{F}_{p}[x_{t_{i}},\mathcal{C}_{I}]^{+}%
}{\left\langle x_{t_{i}}^{r_{i}},\mathcal{R}_{I},\mathcal{S}_{J,K}%
\right\rangle }\otimes\Delta(\rho_{m-1})_{m\in D_{2}(G,p)}$\textsl{,}
\end{enumerate}

\noindent\textsl{where }$t_{i}\in D_{1}(G,p)$\textsl{, and }$I,J,K\subseteq
D_{1}(G,p)$\textsl{ are }$p$\textsl{-monotone.}

\bigskip

\noindent\textbf{Proof.} It is known that if $X$ is a finite CW-complex for
which $\dim H^{\ast}(X;\mathbb{Q})=\dim H_{\beta}^{\ast}(X;\mathbb{F}_{p})$,
then $p\cdot\tau_{p}(X)=0$. In particular, the reduction $r_{p}$ restricts to
an isomorphism

\begin{enumerate}
\item[(4.6)] $r_{p}:\tau_{p}(X)\overset{\cong}{\rightarrow}\operatorname{Im}%
\delta_{p}$,
\end{enumerate}

\noindent where $\delta_{p}$ is the Bockstein operator on $H^{\ast
}(X;\mathbb{F}_{p})$.

If $X$ is a Lie group $G$ with rank $n$ we have by Theorems 3.7 and 4.4 that

\begin{quote}
$\dim H^{\ast}(G;\mathbb{Q})=\dim H_{\beta}^{\ast}(G;\mathbb{F}_{p})=2^{n}$.
\end{quote}

\noindent Since

\begin{quote}
i) $r_{p}(\mathcal{C}_{I})\equiv\delta_{p}(\theta_{I})=C_{I}$,

ii) $r_{p}(x_{t})\equiv\overline{x}_{t}$ for $t\in D_{1}(G,p)$ (by i) of Lemma 3.5),

iii) $r_{p}(\rho_{m-1})\equiv q_{m}\cdot\zeta_{m-1}$ for $m\in D_{2}(G,p)$ (by
ii), iv) of Lemma 3.5), where $q_{m}$ is co-prime to $p$,
\end{quote}

\noindent the isomorphism (4.6) is applicable to translate the formula (3.10)
of $\operatorname{Im}\delta_{p}$ to the desired formula (4.5) of $\tau_{p}%
(X)$.\hfill$\square$

\bigskip

\noindent\textbf{Example 4.7.} Let $G$ be an exceptional Lie group. Applying
Theorem 4.6 we justify the expressions of the torsion ideals $\tau_{p}(G)$
stated in Theorem B. It would be convenient to divide the calculation into
three cases.

\textbf{Case 1.} If $(G,p)=(G_{2},2),(F_{4},2),(F_{4},3),(E_{6},2),(E_{6}%
,3),(E_{7},3)$ or $(E_{8},5)$ then the set $D_{1}(G,p)$ is a singleton $\{t\}$
by Table 2, implying in particular that the generators $\mathcal{C}_{I}\in
\tau_{p}(G)$ (with $I$ $p$-monotone) are absent, and that formula (4.5) turns
to be

\begin{quote}
$\tau_{p}(G)=\frac{\mathbb{F}_{p}[x_{t}]^{+}}{\left\langle x_{t}%
^{r}\right\rangle }\otimes\Delta(\rho_{m-1})_{m\in D_{2}(G,p)}$, $r=cl(y_{t})$.
\end{quote}

\noindent In addition, if either $p=3$ or $5$, we can replace the factor
$\Delta(\rho_{m-1})_{m\in D_{2}(G,p)}$ by the exterior algebra $\Lambda
(\rho_{m-1})_{m\in D_{2}(G,p)}$, because $\deg(\rho_{m-1})\equiv
1\operatorname{mod}2$ implies that $\rho_{m-1}^{2}\equiv0\operatorname{mod}p$.
Finally, if $p=2$, the squares $\rho_{m-1}^{2}$ with $m\in D_{2}(G,p)$ have
been evaluated as that in Example 4.5. This justified the presentations of
$\tau_{p}(G)$ recorded in Theorem B in the current situation.

\textbf{Case 2.} If $(G,p)=(E_{8},3)$ then $D_{1}(G,p)=\{8,20\}$ by Table 2.
Formula (4.5) turns to be

\begin{quote}
$\tau_{3}(E_{8})=\frac{\mathbb{F}_{3}[x_{8},x_{20},\mathcal{C}_{\{8,20\}}%
]^{+}}{\left\langle x_{8}^{3},x_{20}^{3},\mathcal{R}_{\{8,20\}},\mathcal{S}%
_{\{8,20\},\{8,20\}}\right\rangle }\otimes\Delta(\varrho_{3},\varrho
_{15},\varrho_{27},\varrho_{35},\varrho_{39},\varrho_{47})$,
\end{quote}

\noindent where $\mathcal{R}_{\{8,20\}}=x_{8}^{2}x_{20}^{2}\mathcal{C}%
_{\{8,20\}}$ by the definition of $\mathcal{R}_{I}$, $\mathcal{S}%
_{\{8,20\},\{8,20\}}=\mathcal{C}_{\{8,20\}}^{2}$ by ii) of Remark 2.6, and
where $\varrho_{2l-1}^{2}\equiv0\operatorname{mod}3$ because of $2\varrho
_{2l-1}^{2}=0$ by Lemma 2.2. These verify the formula of $\tau_{3}(E_{8})$
stated in v) of Theorem B.

\textbf{Case 3. }For the remaining cases $(G,p)=(E_{7},2)$ or $(E_{8},2)$ one
reads from Table 2 that

\begin{quote}
$D_{1}(E_{7},2)=\{6,10,18\}$ and $D_{1}(E_{8},2)=\{6,10,18,30\}$,
\end{quote}

\noindent indicating that the generators $\mathcal{C}_{I}$ of $\tau_{2}(G)$,
henceforth the relations $\mathcal{R}_{I},\mathcal{S}_{J,K}$ on $\tau_{2}(G)$,
are too many to be presented explicitly. Instead, we conclude this case by
pointing out that, for any $2$-monotone sequences $J,K\subseteq D_{1}(G,2)$,
formula (4.4) is practical to write the relations $\mathcal{S}_{J,K}$ on
$\tau_{2}(G)$ as explicit polynomial in $x_{t_{i}}$ and $\mathcal{C}_{I}$. For
examples, the calculations in $H^{\ast}(E_{8},\mathbb{F}_{2})$

\begin{quote}
$C_{\{6,10\}}C_{\{6,10\}}=\overline{x}_{6}\zeta_{9}^{2}C_{\{6\}}+\overline
{x}_{10}\zeta_{5}^{2}C_{\{10\}}=\overline{x}_{6}^{2}\overline{x}%
_{18}+\overline{x}_{10}^{3}$,

$C_{(6,10)}C_{(6,18)}=\overline{x}_{10}\zeta_{5}^{2}C_{\{18\}}+\overline
{x}_{6}C_{\{6,10,18\}}=\overline{x}_{10}^{2}\overline{x}_{18}+\overline{x}%
_{6}C_{\{6,10,18\}}$
\end{quote}

\noindent corresponds, under the isomorphism (4.6), to the relations on
$\tau_{2}(G)$

\begin{quote}
$\mathcal{S}_{\{6,10\},\{6,10\}}=\mathcal{C}_{\{6,10\}}\mathcal{C}%
_{\{6,10\}}+x_{6}^{2}x_{18}+x_{10}^{3}$;

$\mathcal{S}_{\{6,10\},\{6,18\}}=\mathcal{C}_{(6,10)}\mathcal{C}%
_{(6,18)}+x_{10}^{2}x_{18}+x_{6}\mathcal{C}_{\{6,10,18\}}$,
\end{quote}

\noindent respectively. These computations indicate also that the relations of
the type $\mathcal{S}_{J,K}$ may be highly nontrivial.\hfill$\square$

\subsection{The action of the free part $\mathcal{F}(G)$ on the ideal
$\tau_{p}(G)$}

Since $\tau_{p}(G)$ is an ideal, the cup product on $H^{\ast}(G)$ defines an
action of the free part $\mathcal{F}(G)$ on $\tau_{p}(G)$:

\begin{quote}
$\mathcal{F}(G)\times\tau_{p}(G)\rightarrow\tau_{p}(G)$.
\end{quote}

\noindent In view of the formulae of $\mathcal{F}(G)$ and $\tau_{p}(G)$ given
respectively in Theorems 4.4 and 4.6, to clarifies this action it suffices to
find a formula $\mathcal{H}_{i,I}$ that expresses each product $\rho
_{l(g_{i})-1}\cdot\mathcal{C}_{I}$ with $t_{i}\in D_{1}(G,p)$ and $I\subseteq
D_{1}(G,p)$ as an element in $\tau_{p}(G)$. Here, if $I=\{t\}$ is a singleton,
then $\mathcal{C}_{I}:=\beta_{p}(\theta_{t-1})=x_{t}$ by i) of Lemma 3.6.

\bigskip

\noindent\textbf{Theorem 4.8.} \textsl{For each }$t_{i}\in D_{1}(G,p)$\textsl{
and }$I\subseteq D_{1}(G,p)$\textsl{ the relation }$\mathcal{H}_{i,I}$
\textsl{is given by the three possibilities}

\begin{enumerate}
\item[(4.7)] $\rho_{l(g_{i})-1}\cdot\mathcal{C}_{I}=\left\{
\begin{tabular}
[c]{l}%
$x_{t_{i}}^{r_{i}-1}\cdot\mathcal{C}_{I\cup\{t_{i}\}}$ \textsl{if}
$t_{i}\notin I$;\\
$0$ \textsl{if }$p$ \textsl{is odd and }$t_{i}\in I$;\\
$x_{t_{i}}^{r_{i}-1}\cdot(\theta_{t_{i}-1}^{2})^{\ast}\cdot\mathcal{C}%
_{I_{t_{i}}}$ \textsl{if} $p=2$ \textsl{and} $t_{i}\in I$\textsl{,}%
\end{tabular}
\ \right.  $
\end{enumerate}

\noindent\textsl{where }$I_{t_{i}}$ \textsl{is the complement of }$t_{i}\in
I$\textsl{, and where} $(\theta_{t_{i}-1}^{2})^{\ast}\in\operatorname{Im}%
\pi^{\ast}\cap\tau_{2}(G)$ \textsl{is the unique element that satisfies, under
the isomorphism (4.2), that}

\begin{quote}
$r_{2}(\theta_{t_{i}-1}^{2})^{\ast}=\theta_{t_{i}-1}^{2}\in\operatorname{Im}%
(\pi_{2}^{\ast})^{+}$\textsl{.}
\end{quote}

\noindent\textbf{Proof.} Since the reduction $r_{p}$ restricts to an
isomorphism $\tau_{p}(G)\rightarrow\operatorname{Im}\delta_{p}$ by (4.6), we
can find the expression of the product $\rho_{l(g_{i})-1}\cdot\mathcal{C}%
_{I}\in\tau_{p}(G)$ by computing with its $r_{p}$ image in $H^{\ast
}(G;\mathbb{F}_{p})$. Precisely, from $C_{I}=\delta_{p}(\theta_{I})$ and
$r_{p}(\rho_{l(g_{i})-1})\equiv-\overline{x}_{t_{i}}^{r_{\overline{t}}-1}%
\cdot\theta_{t_{i}-1}$ (by iii) of Lemma 3.5) one finds that

\begin{quote}
$r_{p}(\rho_{l(g_{i})-1}\cdot\mathcal{C}_{I})\equiv-\overline{x}_{t_{i}%
}^{r_{\overline{t}}-1}\delta_{p}(\theta_{t_{i}-1}\theta_{I})$.
\end{quote}

\noindent The proof of (4.7) is completed by

\begin{quote}
$\theta_{t_{i}-1}\theta_{I}\equiv\left\{
\begin{tabular}
[c]{l}%
$\theta_{I\cup\{t_{i}\}}\text{ if }t_{i}\notin I\text{;\quad}$\\
$0\text{ if}$ $p$ is odd and $t_{i}\in I\text{;}$\\
$\theta_{t_{i}-1}^{2}\theta_{I_{t_{i}}}\text{ if }p=2\text{ and }t_{i}\in I$.
\end{tabular}
\ \right.  $.\hfill$\square$
\end{quote}

\bigskip

We are ready to show Theorem B stated in Section \S 1.3.

\bigskip

\noindent\textbf{Proof of Theorem B. }In Theorem B, the expressions of the
cohomology $H^{\ast}(G)$ in the form

\begin{quote}
$H^{\ast}(G)=\mathcal{F}(G)\oplus_{p}\tau_{p}(G)$,
\end{quote}

\noindent together with the formulae of $\mathcal{F}(G)$ and $\tau_{p}(G)$,
have been shown by Examples 4.5 and 4.7. It remains to clarify those relations
$\mathcal{H}_{i,I}$ that describe the action of $\mathcal{F}(G)$ on $\tau
_{p}(G)$, where $t_{i}\in D_{1}(G,p)$ and $I\subseteq D_{1}(G,p)$. To this end
we divide the discussion into three cases, and apply formula (4.7).

i) For $(G,p)=(G_{2},2),(F_{4},2),(E_{6},2),(F_{4},3),(E_{6},3),(E_{7},3)$ or
$(E_{8},5)$ the set $D_{1}(G,p)$ is a singleton $\{t_{i}\}$. Thus, according
to (4.7), for each $(G,p)$ there is just one relation $\mathcal{H}_{i,I}$
which takes the form $\rho_{l(g_{i})-1}\cdot x_{t_{i}}=0$. Precisely, these
relations are

\begin{center}%
\begin{tabular}
[c]{l||l|l|l|l|l|l|l}\hline
$(G,p)$ & $(G_{2},2)$ & $(F_{4},2)$ & $(E_{6},2)$ & $(F_{4},3)$ & $(E_{6},3)$
& $(E_{7},3)$ & $(E_{8},5)$\\\hline
$\mathcal{H}_{i,I}$ & $x_{6}\varrho_{11}$ & $x_{6}\varrho_{11}$ &
$x_{6}\varrho_{11}$ & $x_{8}\varrho_{23}$ & $x_{8}\varrho_{23}$ &
$x_{8}\varrho_{23}$ & $x_{12}\varrho_{59}$\\\hline
\end{tabular}
,
\end{center}

\noindent as that recorded in Theorem B.

ii) For $(G,p)=(E_{8},3)$ we have $D_{1}(G,p)=\{8,20\}$. By (4.7) there are
$6$ relations of the type $\mathcal{H}_{i,I}$, which are presented below:

\begin{center}
{\footnotesize \renewcommand{\arraystretch}{0.8}
\begin{tabular}
[c]{l||l|l|l|l|l|l}\hline
$(i,I)$ & $(8,\{8\})$ & $(8,\{20\})$ & $(8,\{8,20\})$ & $(20,\{8\})$ &
$(20,\{20\})$ & $(20,\{8,20\})$\\\hline
$\mathcal{H}_{i,I}$ & $\rho_{23}\cdot x_{8}$ & $\rho_{23}\cdot x_{20}%
-x_{8}^{2}\cdot\mathcal{C}_{\left\{  8,20\right\}  }$ & $\rho_{23}%
\cdot\mathcal{C}_{\left\{  8,20\right\}  }$ & $\rho_{59}\cdot x_{8}-x_{20}%
^{2}\cdot\mathcal{C}_{\left\{  8,20\right\}  }$ & $\rho_{59}\cdot x_{20}$ &
$\rho_{59}\cdot\mathcal{C}_{\left\{  8,20\right\}  }$\\\hline
\end{tabular}
.}
\end{center}

\noindent These relations have been recorded in v) of Theorem B.

iii) Finally, for the remaining cases $(G,p)=(E_{7},2)$ or $(E_{8},2)$, we
have $D_{1}(E_{7},2)=\{6,10,18\}$ and $D_{1}(E_{8},2)=\{6,10,18,30\}$,
respectively. Since the relations of the form $\mathcal{H}_{i,I}$ are too
many, we have simply referred them, respectively in iv) and v) of Theorem B,
to (4.7).

The proof of theorem B has now been completed.\hfill$\square$

\section{Historical comments}

\textbf{Remark 5.1. }The study of the cohomology of Lie groups begins with
coefficients over a field $\mathbb{F}$. Notably, Hopf \cite{H} observed, for
the first time, that the group product $\mu:G\times G\rightarrow G$ on $G$
induces a map of algebras

\begin{enumerate}
\item[(1.2)] $\mu^{\ast}:H^{\ast}(G;\mathbb{F})\rightarrow H^{\ast}(G\times
G;\mathbb{F})\cong H^{\ast}(G;\mathbb{F})\otimes H^{\ast}(G;\mathbb{F})$,
\end{enumerate}

\noindent where the isomorphism follows from the K\"{u}nneth formula. It
furnishes the cohomology $H^{\ast}(G;\mathbb{F})$ with an additional
co-produce $\mu^{\ast}$, making the pair $\left\{  H^{\ast}(G;\mathbb{F}%
),\mu^{\ast}\right\}  $ an \textsl{Hopf algebra} over $\mathbb{F}$. In term of
$\mu^{\ast}$ one can specify the set of so called "\textsl{primitive
elements}"\textsl{ }of\textsl{ }$H^{\ast}(G;\mathbb{F})$

\begin{quote}
$P(G;\mathbb{F})=\{a\in H^{\ast}(G;\mathbb{F})\mid\mu^{\ast}(a)=a\otimes1$
$\oplus1\otimes a\}$,
\end{quote}

\noindent which is clearly a linear subspace of $H^{\ast}(G;\mathbb{F})$. For
the case $\mathbb{F}$ is the field $\mathbb{R}$ of reals Hopf \cite{H} proved that

\bigskip

\noindent\textbf{Theorem 5.1. }\textsl{If }$\left\{  z_{1},\cdots
,z_{k}\right\}  $\textsl{ is a homogeneous basis of }$P(G;\mathbb{R}%
)$\textsl{, then}

\begin{enumerate}
\item[(5.2)] $H^{\ast}(G;\mathbb{R})=\Lambda_{\mathbb{R}}(z_{1},\cdots,z_{k}%
)$\textsl{, }$\deg z_{i}\equiv1\operatorname{mod}2$.\hfill$\square$
\end{enumerate}

Borel \cite{B1} initiated the investigation of the algebra $H^{\ast
}(G;\mathbb{F}_{p})$ over a finite field $\mathbb{F}_{p}$. Following the idea
of Hopf he began with a classification on the Hopf algebra $H^{\ast
}(G;\mathbb{F}_{p})$.

\bigskip

\noindent\textbf{Theorem} \textbf{5.2}. \textsl{If }$\left\{  z_{1}%
,\cdots,z_{k}\right\}  $\textsl{ is a homogeneous basis of} $P(G;\mathbb{F}%
_{p})$\textsl{, then}

\begin{enumerate}
\item[(5.3)] $H^{\ast}(G;\mathbb{F}_{p})=B(z_{1})\otimes\cdots\otimes
B(z_{k})$\textsl{,}
\end{enumerate}

\noindent\textsl{where each factor }$B(z_{i})$\textsl{ is one of the next
"monogenic Hopf algebra" over }$\mathbb{F}_{p}$\textsl{:}

\begin{quote}%
\begin{tabular}
[c]{l|l|l}\hline\hline
$B(z)$ & $p=2$ & $p\neq2$\\\hline
$\deg(z)=even$ & $\mathbb{F}_{2}(x)/\left\langle x^{2^{r}}\right\rangle $ &
$\mathbb{F}_{p}(x)/\left\langle x^{p^{r}}\right\rangle $\\\hline
$\deg(z)=odd$ & $\mathbb{F}_{2}(x)/\left\langle x^{2^{r}}\right\rangle $ &
$\Lambda_{\mathbb{F}_{p}}(x)$\\\hline\hline
\end{tabular}
.\hfill$\square$
\end{quote}

Based on Theorems 5.1 and 5.2 the algebras $H^{\ast}(G;\mathbb{F})$ were
largely computed by Borel \cite{B2,B3}, Chevalley \cite{Ch0}, Araki
\cite{AS,A1,A2,A3}, Toda, Kono, Mimura and Shimada \cite{T,KM1,KM2,KM3,KMS,Ko}%
, case by case.

The integral cohomology $H^{\ast}(G)$ is much more subtle, implies the result
for any field coefficients, but in general fails to be an Hopf ring.
Nevertheless, based on the Schubert presentation (1.7) of the ring $H^{\ast
}(G/T)$, one can determine the structure of $H^{\ast}(G;\mathbb{F}_{p})$ as an
Hopf algebra over the Steenrod algebra $\mathcal{A}_{p}$ (e.g. \cite[Theorems
1.1 and 1.2]{DZ1}), and construct the ring $H^{\ast}(G)$ uniformly for all $G$
(e.g. Theorem B).

\bigskip

\textbf{Remark 5.2.} The approach to the topology of Lie group, using the
Serre spectral sequence of $\pi:G\rightarrow G/T$, has been interested by
several authors.

In \cite{L} Leray proved that the Serre spectral sequence of $\pi$ satisfies
$E_{3}^{\ast,\ast}(G;\mathbb{R})=E_{\infty}^{\ast,\ast}(G;\mathbb{R})$. Later
on, Ka\v{c} and Marlin \cite{K,M2} conjectured that there exist additive isomorphisms

\begin{enumerate}
\item[(5.4)] $E_{3}^{\ast,\ast}(G)\cong E_{\infty}^{\ast,\ast}(G)$ and
$E_{\infty}^{\ast,\ast}(G)\cong H^{\ast}(G)$.
\end{enumerate}

\noindent Since our construction of $H^{\ast}(G)$ factor through $E_{3}%
^{\ast,\ast}(G)$ (see in Section \S 1.2), these conjectures are transparent
within our approach.

Combining Schubert calculus on $G/T$ with the invariant theory of Lie groups
\cite{Ch1}, Reeder \cite[(6.2) Theorem]{R} recovered the following computation
of Borel, Chevalley and Leray \cite{B1,Ch0,L}:

\begin{enumerate}
\item[(5.5)] $H^{\ast}(G;\mathbb{R})=\Lambda_{\mathbb{R}}(\xi_{2l_{1}%
-1},\cdots,\xi_{2l_{n}-1})$, $q(G)=\{l_{1},\cdots,l_{n}\}$.
\end{enumerate}

\noindent In our context, taking $\xi_{2l-1}=\rho_{2l-1}\otimes1\in H^{\ast
}(G)\otimes\mathbb{R}$, formula (5.5) is shown by Theorem 4.1.

\bigskip

\textbf{Remark 5.3. }A classical problem of topology (resp. geometry) is to
express the cohomology of a manifold (resp. the Chow ring of a projective
variety) by a minimal system of explicit generators. In this regard Marlin
\cite{M1} obtained presentations of the Chow rings $\mathcal{A}(G)$ for
$G=SO(n)$ and $Spin(n)$ in term of explicit Schubert classes on $G/T$. On the
other hand, from earlier knowledge of the algebra $H^{\ast}(G,\mathbb{F}_{p})$
Ka\v{c} \cite[Theorem 3]{K} derived a general formula of the algebra
$\mathcal{A}(G)\otimes\mathbb{F}_{p}$, where the generators are specified only
by their degrees. In our approach the generators $x_{t}=\pi^{\ast}(y_{t})$ of
the Chow ring $\mathcal{A}(G)$ can be given by explicit Schubert classes
$y_{t}$ on $G/T$. For example, if $G$ is exceptional, then in term of the Weyl
coordinates of Schubert classes, these generators are given by the table below
(e.g. \cite[Table 1]{DZ1}):

\begin{center}%
\begin{tabular}
[c]{|l|l|l|l|l|}\hline
$y_{i}$ & $G_{2}/T$ & $F_{4}/T$ & $E_{n}/T,\text{ }n=6,7,8$ & $p$%
\\\hline\hline
$y_{3}$ & $\sigma_{\lbrack1,2,1]}$ & $\sigma_{\lbrack3,2,1]}$ & $\sigma
_{\lbrack5,4,2]}\text{, }n=6,7,8$ & $2$\\
$y_{4}$ &  & $\sigma_{\lbrack4,3,2,1]}$ & $\sigma_{\lbrack6,5,4,2]}\text{,
}n=6,7,8$ & $3$\\
$y_{5}$ &  &  & $\sigma_{\lbrack{7,6,5,4,2}]}\text{, }n=7,8$ & $2$\\
$y_{6}$ &  &  & $\sigma_{\lbrack1,3,6,5,4,2]}\text{, }n=8$ & $5$\\
$y_{9}$ &  &  & $\sigma_{\lbrack1,5,4,3,7,6,5,4,2]}$,$\text{ }n=7,8$ & $2$\\
$y_{10}$ &  &  & $\sigma_{\lbrack{1,6,5,4,3,7,6,5,4,2}]}\text{, }n=8$ & $3$\\
$y_{15}$ &  &  & $\sigma_{\lbrack5,4,2,3,1,6,5,4,3,8,7,6,5,4,2]}\text{, }n=8$
& $2$\\\hline
\end{tabular}


{\small The }${\small p}$--{\small special Schubert classes on }${\small G/T}%
${\small \ and their abbreviations}
\end{center}

\textbf{Remark 5.4.} A compact Lie group $K$ is called \textsl{simple} if it
is isomorphic to the quotient group $G/L$, where $G$ is simply-connected and
simple, and $L\subset G$ is a subgroup of the center of $G$. In particular,
the orthogonal group $SO(n)$, the projective unitary group
$PU(n):=SU(n)/\mathbb{Z}_{n}$, the adjoint exceptional Lie groups $Ad(E_{6})$
and $Ad(E_{7})$, are example non simply-connected simple Lie groups, e.g.
\cite{BB}.

To simplify the presentation we have assumed, at the very beginning, that the
Lie groups under consideration are the simply-connected and simple ones. We
mention at this point that the method developed in this paper applies equally
well to construct the cohomology of all simple Lie groups. Precisely, let
$T^{\prime}$ be a maximal torus of $K$, and let $T$ be a maximal torus of $G$
that corresponds to $T^{\prime}$ under the universal covering $G\rightarrow
K$. Then, in view of the canonical isomorphism $K/T^{\prime}=G/T$ in flag
manifolds, the second page of the Serre spectral sequence $\{E_{r}^{\ast,\ast
}(K),d_{r}^{\prime}\}$ of $K\rightarrow K/T^{\prime}$ is

\begin{quote}
$E_{2}^{\ast,\ast}(K)=H^{\ast}(G/T)\otimes H^{\ast}(T^{\prime})$,
\end{quote}

\noindent where

i) the Schubert presentation of the ring $H^{\ast}(G/T)$ is available by (1.7);

ii) the differential $d_{2}^{\prime}:H^{1}(T^{\prime})\rightarrow H^{2}(G/T)$
can be formulated in term of the fundamental group $\pi_{1}(K)=L$ (e.g.
\cite[Theorem 2.3]{D1}).

\noindent Combining these ideas the cohomologies $H^{\ast}(K)$ of the simple
Lie groups $K=PU(n)$, $Ad(E_{6})$, $Ad(E_{7})$ have been presented using
explicitly constructed generators in \cite{D2,D3}.

For further applications of Schubert calculus to computing with the integral
cohomology of homogeneous spaces, see \cite[Section \S 5]{DZ0}.

\begin{thebibliography}{99}                                                                                               %


\bibitem {AS}S. Araki, Y. Shikata, Cohomology $\operatorname{mod}2$ of the
compact exceptional group $E_{8}$, Proc. Japan Acad. 37(1961), 619--622.

\bibitem {A1}S. Araki, Cohomology modulo $2$ of the compact exceptional groups
$E_{6}$ and $E_{7}$, J. Math. Osaka City Univ. 12(1961), 43--65.

\bibitem {A2}S. Araki, Differential Hopf algebras and the cohomology
$\operatorname{mod}3$ of the compact exceptional groups $E_{7}$ and $E_{8}$,
Ann. Math. 73(1961), 404-436.

\bibitem {A3}S. Araki, On the non-commutativity of Pontryagin rings mod 3 of
some compact exceptional groups, Nagoya Math. J. 17(1960), 225--60.
Correction. Ibid. 19(1961), 195-197.

\bibitem {BB}P. F. Baum, W. Browder, The cohomology of quotients of classical
groups. Topology 3(1965), 305--336.

\bibitem {BGG}I.N. Bernstein, I.M. Gel'fand, S.I. Gel'fand, Schubert cells and
cohomology of the spaces $G/P$, Russian Math. Surveys, 28 : 3 (1973) pp. 1-26.

\bibitem {B1}A. Borel, Sur la cohomologie des espaces fibres principaux et des
espaces homogenes de groupes de Lie compacts, Ann. math. 57(1953), 115--207.

\bibitem {B2}A. Borel, Sur l'homologie et la cohomologie des groupes de Lie
compacts connexes, Amer. J. Math. 76(1954), 273--342.

\bibitem {B3}A. Borel, Homology and cohomology of compact connected Lie
groups. Proc. Nat. Acad. Sci. U. S. A. 39(1953). 1142--1146.

\bibitem {B4}A. Borel, Topics in the homology theory of fiber bundles, Berlin,
Springer, 1967.

\bibitem {BC}A. Borel, C. Chevalley, The Betti numbers of the exceptional
groups, Mem. Amer. Math. Soc. no. 14(1955), 1--9.

\bibitem {BH}A. Borel; F Hirzebruch, Characteristic classes and homogeneous
space I, J. Amer. Math. Soc., vol 80 (1958), 458-538.

\bibitem {Br}R. Brauer, Sur les invariants int\'{e}graux des vari\'{e}t\'{e}s
de Lie simples clos. C. R. Acad. Sci. Paris, 201(1935), 419--421.

\bibitem {BS}R. Bott and H. Samelson, The cohomology ring of $G/T$, Nat. Acad.
Sci. 41(1955), 490-492.

\bibitem {C}E. Cartan, La topologie des groupes de Lie, \ L'Enseignement Math.
35 (1936), 177-200.

\bibitem {Ch0}C. Chevalley, The Betti numbers of exceptional Lie groups, Proc.
ICM 2, (1950), 21-24.

\bibitem {Ch1}C. Chevalley, Invariants of finite groups generated by
reflections, Amer. J. Math. 77(1955), 778-782.

\bibitem {Ch2}C. Chevalley, Sur les D\'{e}compositions Cellulaires des Espaces
G/B, in Algebraic groups and their generalizations: Classical methods, W.
Haboush ed. Proc. Symp. in Pure Math. 56 (part 1) (1994), 1-26.

\bibitem {De1}M. Demazure, Invariants sym\'{e}riques entiers des groupes de
Weyl et torsion, Invent. Math. 21(1973), 287-301.

\bibitem {De2}M. Demazure, Desingularisation des vari\'{e}t\'{e}s de Schubert
g\'{e}n\'{e}ralis\'{e}es, Ann. Sci. \'{E}s. Norm. Sup\'{e}r. (4)7(1974), 53-88.

\bibitem {D}J. Dieudonn\'{e}, A History of Algebraic and Differential Topology
1900--1960, Boston; Basel; Birkh\"{a}user, 1989.

\bibitem {D1}H. Duan, On the Borel transgression in the fibration
$G\rightarrow G/T$, Homology, Homotopy and Applications 20(1) (2018), 79-86.

\bibitem {D2}H. Duan, The cohomology and K-theory of the projective unitary
groups PU(n), arXiv:1710.09222 [math.AT].

\bibitem {D3}H. Duan, Schubert calculus and cohomology of Lie groups. Part II.
Compact Lie groups, arXiv:1502.00410 [math.AT].

\bibitem {DZ0}H. Duan, X. Zhao, The Chow rings of generalized Grassmannians,
Found. Math. Comput. 10(2010), no.3, 245--274.

\bibitem {DZ1}H. Duan, X. Zhao, Schubert calculus and the Hopf algebra
structures of exceptional Lie groups, Forum. Math. Vol.26, no.1(2014), 113-140.

\bibitem {DZ2}H. Duan, X. Zhao, Schubert presentation of the cohomology ring
of flag manifolds $G/T$, LMS J. Comput. Math. 18(2015), no.1, 489-506.

\bibitem {DZ3}H. Duan, X. Zhao, On Schubert's Problem of Characteristics,
Springer proceedings in Mathematics and Statistics, Vol. 332(2020), 43--71.

\bibitem {DZ4}H. Duan, X. Zhao, Schubert calculus and intersection theory of
flag manifolds, Uspekhi Matematicheskikh Nauk, Volume 77, Issue 4(2022),
173--196 (Russian Math. Surveys, 77:4 (2022), 729-751).

\bibitem {G}A. Grothendieck, Torsion homologique et sections rationnelles,
Sem. C. Chevalley, ENS 1958, expos\'{e} 5, Secreatariat Math. IHP, Paris, 1958.

\bibitem {H}H. Hopf, \"{U}ber die topologie der gruppenmannigfaltigkeiten und
ibre veraligemeinerungen, Ann. of Math. 42(1941), 22--52.

\bibitem {K}V.G. Ka\v{c}, Torsion in cohomology of compact Lie groups and Chow
rings of reductive algebraic groups, Invent. Math. 80(1985), no. 1, 69--79.

\bibitem {KM1}A. Kono and M. Mimura, Cohomology mod 2 of the classifying space
of the compact connected Lie group of type $E_{8}$, J. Pure Appl. Algebra
6(1975), 61--81.

\bibitem {KM2}A. Kono and M. Mimura, On the cohomology mod 3 of the
classifying space of the 1-connected exceptional Lie group $E_{6}$, Preprint
series of Aarhus University, 1975.

\bibitem {KM3}A. Kono and M. Mimura, Cohomology operations and the Hopf
algebra structures of the compact, exceptional Lie groups $E_{7}$ and $E_{8}$,
Proc. London Math. Soc. 35(1977), 345-359.

\bibitem {KMS}A. Kono, M. Mimura and N. Shimada, On the cohomology mod 2 of
the classifying space of the 1-connected exceptional Lie group $E_{7}$, J.
Pure Appl. Algebra 8(1976), 267--283.

\bibitem {Ko}A. Kono, On the cohomology $\operatorname{mod}2$ of $E_{8}$, J.
Math. Kyoto Univ. 24(1984), 275--280.

\bibitem {L}J. Leray, Sur l'homologie des groupes de Lie, des espaces
homog\`{e}nes et des espaces fibr\'{e} principaux, Colloque de topologie
Alg\'{e}brique, Bruxelles 1950, 101--115.

\bibitem {M1}R. Marlin, Une conjecture sur les anneaux de Chow $A(G,\mathbb{Z}%
)$ renforc\'{e}e par un calcul formel, Effective methods in algebraic geometry
(Castiglioncello, 1990), 299--311, Progr. Math., 94, Birkh\"{a}user Boston,
Boston, MA, 1991.

\bibitem {M2}R. Marlin, Anneaux de Chow des groupes alg\'{e}riques $SU(n)$,
$Sp(n)$, $SO(n)$, $Spin(n),G_{2},F_{4}$; torsion, C. R. Acad. Sci. Paris, A
279(1974), 119--122.

\bibitem {Mc}J. McCleary, A user's guide to spectral sequences, Second
edition. Cambridge Studies in Advanced Mathematics, 58. Cambridge University
Press, Cambridge, 2001.

\bibitem {MT}M. Mimura and H. Toda, Topology of Lie groups. I, II. Translated
from the 1978 Japanese edition by the authors, Translations of Mathematical
Monographs, 91, American Mathematical Society, Providence, RI, 1991.

\bibitem {Po}L.S. Pontrjagin, Homologies in compact Lie groups, Rec. Math.
N.S. 6(48), (1939), 389--422.

\bibitem {P}H. Pittie, The integral homology and cohomology rings of $SO(n)$
and $Spin(n)$, J. Pure Appl. Algebra 73(1991), 105-153.

\bibitem {R}M. Reeder, On the cohomology of compact Lie groups. Enseign.
Math., II. 41, No. 3-4, 181-200 (1995).

\bibitem {Sa}H. Samelson, Topology of Lie groups, Bull. Amer. Math. Soc. Vol.
58, no.1 (1952), 2--37.

\bibitem {T}H. Toda, Cohomology of the classifying space of exceptional Lie
groups, Manifolds, Proc. int. Conf. Manifolds relat. Top. Topol., Tokyo 1973,
265-271 (1975).

\bibitem {Wa}B. L. van der Waerden, Topologische Begr\"{u}ndung des
Kalk\"{u}ls der abz\"{a}hlenden Geometrie, Math. Ann. 102 (1930), no. 1, 337--362.

\bibitem {W}A. Weil, Foundations of algebraic geometry, American Mathematical
Society, Providence, R.I. 1962.

\bibitem {Wh}G.W. Whitehead, Elements of homotopy theory, Graduate Texts in
Mathematics, 61. Springer-Verlag, New York-Berlin, 1978.
\end{thebibliography}

\bigskip

\begin{quote}
Haibao Duan,

dhb@math.ac.cn,

Yau Mathematical Science Center, Tsinghua University, Beijing 100084;

Academy of Mathematics and Systems Sciences, Chinese Academy of Sciences,
Beijing 100190.

School of Mathematical Sciences, Dalian University of Technology, Dalian 116024.
\end{quote}


\end{document}