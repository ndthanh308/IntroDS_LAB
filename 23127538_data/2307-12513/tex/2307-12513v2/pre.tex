\section{Preliminaries}
\label{sec:pre}

For a matrix $A$, we use $A_{ij}$ to denote the entry at the intersection of the $i$th row and the $j$th column. We use $[n]$ to denote the set $\{x\in\mathbb{N}\colon\, 1\leq x\leq n\}$. 
Let $x_1,x_2,\dots, x_n$ be $n$ integers. The $n\times n$ Vandermonde matrix of $x_1,x_2,\dots, x_n$ is defined to be the matrix $V$ with $V_{ij}=x_i^{j-1}$; i.e.

\[V=\left(\begin{array}{ccccc}
    1 & x_1 & x_1^2 & \dots & x_1^{n-1} \\
    1 & x_2 & x_2^2 & \dots & x_2^{n-1} \\
    \vdots & \vdots & \vdots & \ddots & \vdots \\
    1 & x_n & x_n^2 & \dots & x_n^{n-1} \\ 
\end{array}\right)\]

\medskip
\noindent
The determinant of the $n\times n$ Vandermonde matrix is 
\begin{equation}
\label{eq:det_vand}
    \prod_{1\leq i<j\leq n}(x_j-x_i).
\end{equation}

\noindent
% It is known that a  matrix $A$ is invertible if and only if $\det(A)\neq 0$ (cf.~\cite{FrInSp18}). 
Let $p$ be a prime greater than $n$, and let $x_1, x_2,\dots, x_n$ be $n$ distinct integers less than $p$. Consider the matrix  $X$, with $0\leq X_{ij}<p$ and 
\[X_{ij}\equiv x_i^{j-1}\pmod{p}.\]

\noindent
By~\eqref{eq:det_vand}
\[\det(X)\equiv \prod_{1\leq i<j\leq n}(x_j-x_i)\pmod{p}.\]
Since $p$ is prime, $\det(X)\neq 0$. Thus, $X$ is invertible over $\mathbb{R}$. The following lemma is an immediate consequence of the discussion above.

\begin{lemma}
\label{lem:key}
    Let $p$ be a prime, and let $x_1, x_2,\dots, x_n$ be $n$ distinct integers less than $p$. For the $n\times n$ matrix $X$, with $0\leq X_{ij}<p$ and $X_{ij}\equiv x_i^{j-1} \pmod{p}$, and a column vector  $\vectorbold{y}\in\mathbb{R}^n$,  
    \[X\vectorbold{y} = \mathbf{0} \iff \vectorbold{y} = \mathbf{0}.\]
\end{lemma}

\medskip
% \begin{remark}
%     The matrix $X$ can be viewed as a Vandermonde matrix of $x_1, x_2,\dots, x_n$ over $\mathbb{Z}/p\mathbb{Z}$. The nonsingularity of $X$ (i.e. $\det(X)\neq 0$ over $\mathbb{Z}/p\mathbb{Z}$) was also shown in~\cite{KIMBREL1993107,Lipson80}.
% \end{remark}
% This equivalence holds not only for matrices over integers, but also for those over any field, as shown in~\cite[Corollary of Theorem~4.7]{FrInSp18}.

% \begin{proposition}[Friedberg, Insel, and Spence~\cite{FrInSp18}]
% \label{prop:inver_det}
%     Over any field, a matrix $A$ is invertible if and only if $\det(A)\neq 0$.
% \end{proposition}

% It follows that $V$ is invertible if and only if $x_i\neq x_j$ for distinct $i,j\in[n]$. This property can be generalized to any field. We summarize this property in the following proposition.

% \begin{proposition}[\cite{KIMBREL1993107,Lipson80}]
% \label{prop:lipson}
%     Over any field, every $n\times n$ Vandermode matrix of $n$ distinct  elements is invertible.
% \end{proposition}

% \noindent
% The matrix $V$ is a Vandermode matrix over $\mathbb{Z}/p\mathbb{Z}$, which is a field. By Proposition~\ref{prop:lipson} $V$ is invertible. We claim that this matrix is also invertible over $\mathbb{Z}$.



% \begin{proof}
%     By Proposition~\ref{prop:inver_det}, it suffices to show that $\det(V)\neq 0$. The matrix $V$ is a Vandermode matrix over $\mathbb{Z}/p\mathbb{Z}$ of $n$ distinct elements, so by Proposition~\ref{prop:lipson} $V$ is invertible. Thus $\det(V)\not\equiv 0\; (\bmod\; p)$. It follows that $\det(V)\neq 0$. 
% \end{proof}

% As a immediate consequence, we have the following corollary. 

% \[\left(\begin{array}{ccccc}
%     \overline{1} & \overline{1^1} & \overline{1^2} & \dots & \overline{1^n} \\
%     \overline{1} & \overline{2^1} & \overline{1^2} & \dots & \overline{2^n} \\
%     \vdots & \vdots & \vdots & \ddots & \vdots \\
%     \overline{1} & \overline{n^1} & \overline{n^2} & \dots & \overline{n^n} \\
% \end{array}\right)\]

% \begin{corollary}
% \label{cor:key}
%     \[V\vectorbold{x} = \mathbf{0} \iff \vectorbold{x} = \mathbf{0}.\]
% \end{corollary}