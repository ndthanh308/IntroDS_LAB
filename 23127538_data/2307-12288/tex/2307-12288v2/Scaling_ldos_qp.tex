\documentclass[prl,aps,twocolumn,floats,nofootinbib,superscriptaddress]{revtex4-1}
%\documentclass[prl,aps,twocolumn,floats,nofootinbib]{revtex4}
\usepackage{amsmath,amssymb,graphicx,psfrag}
\usepackage[colorlinks,linkcolor={blue},citecolor={blue},urlcolor={blue}]{hyperref}% add hypertext capabilities
%\usepackage{babel}
\newcommand{\fig}[2]{% Figure removed}
\begin{document}
\renewcommand{\ni}{{\noindent}}
\newcommand{\dprime}{{\prime\prime}}
\newcommand{\be}{\begin{equation}}
\newcommand{\ee}{\end{equation}}
\newcommand{\bea}{\begin{eqnarray}} 
\newcommand{\eea}{\end{eqnarray}}
\newcommand{\la}{\langle}
\newcommand{\ra}{\rangle} 

\newcommand{\dg}{\dagger}
\newcommand\lbs{\left[}
\newcommand\rbs{\right]}
\newcommand\lbr{\left(}
\newcommand\rbr{\right)}
\newcommand\f{\frac}
\newcommand\e{\epsilon}
\newcommand\ua{\uparrow}
\newcommand\da{\downarrow}
\newcommand{\bcen}{\begin{center}}
\newcommand{\ecen}{\end{center}}
\newcommand{\btab}{\begin{tabular}}
\newcommand{\etab}{\end{tabular}}
\newcommand{\bdes}{\begin{description}}
\newcommand{\edes}{\end{description}}
\newcommand{\mc}{\multicolumn}
\newcommand{\ul}{\underline}
\newcommand{\non}{\nonumber}
\newcommand{\etal}{et.~al.\ }
\newcommand{\half}{\frac{1}{2}}
\newcommand{\bary}{\begin{array}}
\newcommand{\eary}{\end{array}}
\newcommand{\benum}{\begin{enumerate}}
\newcommand{\eenum}{\end{enumerate}}
\newcommand{\bitem}{\begin{itemize}}
\newcommand{\eitem}{\end{itemize}}
\newcommand{\cuup}[1]{c_{#1 \uparrow}}
\newcommand{\cdown}[1]{c_{#1 \downarrow}}
\newcommand{\cdup}[1]{c^\dagger_{#1 \uparrow}}
\newcommand{\cddown}[1]{c^\dagger_{#1 \downarrow}}
%
% bold greek characters
%
\newcommand{\beps}{\mbox{\boldmath $ \epsilon $}}
\newcommand{\bsig}{\mbox{\boldmath $ \sigma $}}
\newcommand{\bpi}{\mbox{\boldmath $ \pi $}}
\newcommand{\bkap}{\mbox{\boldmath $ \kappa $}}
\newcommand{\bgam}{\mbox{\boldmath $ \gamma $}}
\newcommand{\bphi}{\mbox{\boldmath $ \phi $}}
\newcommand{\balp}{\mbox{\boldmath $ \alpha $}}
\newcommand{\beot}{\mbox{\boldmath $ \eta $}}
\newcommand{\btau}{\mbox{\boldmath $ \tau $}}
\newcommand{\blam}{\mbox{\boldmath $ \lambda $}}
\newcommand{\bomg}{\mbox{\boldmath $ \omega $}}
\newcommand{\bOmg}{\mbox{\boldmath $ \Omega $}}
\newcommand{\bxhi}{\mbox{\boldmath $ \xi $}}
\newcommand{\bmu} {\mbox{\boldmath $ \mu $}}
\newcommand{\bnu} {\mbox{\boldmath $ \nu $}}
\newcommand{\bdelta}{{\boldsymbol{\delta}}}
\newcommand{\bTheta}{{\boldsymbol{\Theta}}}
\newcommand{\bpsi}{\mbox{\boldmath $ \psi $}}
\newcommand{\brho}{\mbox{\boldmath $ \rho $}}
\newcommand{\bGam}{\mbox{\boldmath $ \Gamma $}}
\newcommand{\bLam}{\mbox{\boldmath $ \Lambda $}}
\newcommand{\bPhi}{\mbox{\boldmath $ \Phi $}}
%
% bold latin
%
%\newcommand{\ba} { \mbox{\boldmath $a$}}
\newcommand{\ba} { \bm{a} }
\newcommand{\bb} { \mbox{\boldmath $b$}}
%\newcommand{\bc} { \mbox{\boldmath $c$}}
\newcommand{\bc} { {\mathbf c} }
\newcommand{\bd} { \mbox{\boldmath $d$}}
\newcommand{\bff}{ \mbox{\boldmath $f$}}
\newcommand{\bg} { \mbox{\boldmath $g$}}
\newcommand{\bh} { \mbox{\boldmath $h$}}
\newcommand{\bi} { \mbox{\boldmath $i$}}
\newcommand{\bj} { \mbox{\boldmath $j$}}
\newcommand{\bk} { \bm{k} }
%\newcommand{\bk} { \mbox{\boldmath $k$}}
\newcommand{\bl} { \mbox{\boldmath $l$}} 
\newcommand{\bmm} { \mbox{\boldmath $m$}}
\newcommand{\bn} { \mbox{\boldmath $n$}}
\newcommand{\bo} { \mbox{\boldmath $o$}}
\newcommand{\bp} { \bm{p} }
\newcommand{\bq} { \bm{q} }
%\newcommand{\bq} { \mbox{\boldmath $q$}}
\newcommand{\br} { \boldsymbol{r}}
\newcommand{\bs} { \mbox{\boldmath $s$}}
\newcommand{\bt} {\boldsymbol{t}} 
\newcommand{\bu} { \mbox{\boldmath $u$}}
\newcommand{\bv} { \mbox{\boldmath $v$}}
\newcommand{\bw} { \mbox{\boldmath $w$}}
\newcommand{\bx} { \mbox{\boldmath $x$}}
\newcommand{\by} { \mbox{\boldmath $y$}}
\newcommand{\bz} { \mbox{\boldmath $z$}}
\newcommand{\bA} { \mbox{\boldmath $A$}}
\newcommand{\bB} { \mbox{\boldmath $B$}}
\newcommand{\bC} { \mbox{\boldmath $C$}}
\newcommand{\bD} { \mbox{\boldmath $D$}}
\newcommand{\bF} { \mbox{\boldmath $F$}}
\newcommand{\bG} { \mbox{\boldmath $G$}}
\newcommand{\bH} { \mbox{\boldmath $H$}}
\newcommand{\bI} { \mbox{\boldmath $I$}}
\newcommand{\bJ} { \mbox{\boldmath $J$}}
\newcommand{\bK} { \mbox{\boldmath $K$}}
\newcommand{\bL} { \mbox{\boldmath $L$}}
\newcommand{\bM} { \mbox{\boldmath $M$}}
\newcommand{\bN} { \mbox{\boldmath $N$}}
\newcommand{\bO} { \mbox{\boldmath $O$}}
\newcommand{\bP} { \mbox{\boldmath $P$}}
\newcommand{\bQ} { \boldsymbol{Q} }
%\newcommand{\bQ} { \mbox{\boldmath $Q$}}
\newcommand{\bR} { {\mathbf R} }
%\newcommand{\bR} { \mbox{\boldmath $R$}}
\newcommand{\bS} { \mbox{\boldmath $S$}}
\newcommand{\bT} { \mbox{\boldmath $T$}}
\newcommand{\bU} { \mbox{\boldmath $U$}}
\newcommand{\bV} { \mbox{\boldmath $V$}}
\newcommand{\bW} { \mbox{\boldmath $W$}}
\newcommand{\bX} { \mbox{\boldmath $X$}}
\newcommand{\bY} { \mbox{\boldmath $Y$}}
\newcommand{\bZ} { \mbox{\boldmath $Z$}}
\newcommand{\bzero} { \mbox{\boldmath $0$}}
\newcommand{\bfell} {\mbox{\boldmath $ \ell $}}


%
% special math symbols
%
\newcommand{\dou}{\partial}
\newcommand{\leftjb} {[\![}
\newcommand{\rightjb} {]\!]}
\newcommand{\ju}[1]{ \leftjb #1 \rightjb }
\newcommand{\D}[1]{\mbox{d}{#1}} 
\newcommand{\grad}{\mbox{\boldmath $\nabla$}}
\newcommand{\modulus}[1]{|#1|}
\renewcommand{\div}[1]{\grad \cdot #1}
\newcommand{\curl}[1]{\grad \times #1}
\newcommand{\mean}[1]{\langle #1 \rangle}
\newcommand{\bra}[1]{{\langle #1 |}}
\newcommand{\ket}[1]{| #1 \rangle}
\newcommand{\braket}[2]{\langle #1 | #2 \rangle}
\newcommand{\dbdou}[2]{\frac{\dou #1}{\dou #2}}
\newcommand{\dbdsq}[2]{\frac{\dou^2 #1}{\dou #2^2}}
\newcommand{\Pint}[2]{ P \!\!\!\!\!\!\!\int_{#1}^{#2}}
\newcommand{\Itwo}{{\mathds{1}}}
\newcommand{\Hds}{{\mathds{H}}}
\newcommand{\cH}{{\cal H}}
\newcommand{\cS}{{\cal S}}


%
% Abbreviations for equations etc
% 
\newcommand{\prn}[1] {(\ref{#1})}
\newcommand{\sect}[1] {Section~\ref{#1}}
\newcommand{\Sect}[1] {Section~\ref{#1}}


%
% Other utilities
%
\newcommand{\uncon}[1]{\centerline{\epsfysize=#1 \epsfbox{/usr2/yogeshwar/styles/construction.pdf}}}
\newcommand{\checkup}[1]{{(\tt #1)}\typeout{#1}}
\newcommand{\ttd}[1]{{\color[rgb]{1,0,0}{\bf #1}}}
%\newcommand{\ttd}[1]{}
\newcommand{\red}[1]{{\color[rgb]{1,0,0}{\protect{#1}}}}
\newcommand{\blue}[1]{{\color[rgb]{0,0,1}{#1}}}
\newcommand{\green}[1]{{\color[rgb]{0.0,0.5,0.0}{#1}}}
\newcommand{\citebyname}[1]{\citeauthor{#1}\cite{#1}}
\newcommand{\myfigwidth}{0.95\columnwidth}
\newcommand{\myhalffig}{0.475\columnwidth}
\newcommand{\mythirdfig}{0.33\columnwidth}
\newcommand{\signum}[0]{\mathop{\mathrm{sign}}}
\newcommand{\skup}{\ket{s \uparrow}}
\newcommand{\skdn}{\ket{s \downarrow}}
\newcommand{\pkup}{\ket{p \uparrow}}
\newcommand{\pkdn}{\ket{p \downarrow}}
\newcommand{\sbup}{\bra{s \uparrow}}
\newcommand{\sbdn}{\bra{s \downarrow}}
\newcommand{\pbup}{\bra{p \uparrow}}
\newcommand{\pbdn}{\bra{p \downarrow}}

% Abbreviations for equations etc
\newcommand{\Eqn}[1] {Eqn.~(\ref{#1})}
%\newcommand{\Sect}[1] {Section~\ref{#1}}
\newcommand{\Fig}[1]{Fig.~\ref{#1}}

\title{Single-particle excitations across the localization and many-body localization transition in quasi-periodic systems}
\author{Yogeshwar Prasad}
\affiliation{Department of Liberal Studies, Kangwon National University, Samcheok, 25913, Republic of Korea}
\affiliation{Theory Division, Saha Institute of Nuclear Physics, 1/AF Bidhannagar, Kolkata 700 064, India}
\author{Arti Garg}
\affiliation{Theory Division, Saha Institute of Nuclear Physics, 1/AF Bidhannagar, Kolkata 700 064, India}
 \affiliation{Homi Bhabha National Institute, Training School Complex, Anushaktinagar, Mumbai 400094, India}
\vspace{0.2cm}
\begin{abstract}
\vspace{0.3cm}
       {We study localization and many-body localization transition in one dimensional systems in the presence of deterministic quasi-periodic potentials. We focus on single-particle excitations obtained through single-particle Green's function in real space. A single parameter scaling analysis of the ratio of the typical to average value of the local density of states (LDOS) of single particle excitations shows that the critical exponent with which the correlation length $\xi$ diverges at the transition point $\xi \sim |h-h_c|^{-\nu}$, coming from the localized side, satisfies the inequality $\nu \ge 1$ for the non-interacting Aubry-Andre (AA) model as well as for the deterministic aperodic model studied here. For the interacting systems, we study single particle excitations produced in highly excited many-body eigenstates across the MBL transition and found that the critical exponent obtained from finite-size scaling of the ratio of the typical to average value of the LDOS $\nu \ge 1$ here as well. This analysis of local density of states shows that the localization and MBL transition in systems with quasi-periodic potentials belong to a different universality class than the localization and MBL transition in systems with random disorder where $\nu \ge 2$. This is in complete contrast to the level spacing ratio which is known to support the same universality class for MBL transitions in systems with quasiperiodic as well as random disorder potentials. For the interacting systems with quasiperiodic potentials, though finite-size scaling of the level spacing ratio shows a transition at $h_c^{lsr}$ which is close to the transition point obtained from LDOS within numerical precision, the critical exponent obtained from finite-size scaling of level spacing ratio is $\nu <1$ in close similarity to the MBL systems with random disorder.}
        
\end{abstract} 
\maketitle
\section{I. Introduction}
Transport properties of a system are significantly influenced by disorder. Any tiny quantity of disorder is sufficient to localise all the single particle states in a non-interacting system in one dimension with quenched random disorder~\cite{Anderson}. But it is feasible to see a localization to delocalization transition even in one dimension in models with deterministic quasiperiodic potentials, such as the Aubry-Andre (AA) model~\cite{AA} and models with Fibonacci potentials~\cite{Giamarchi}. There are also examples of deterministic aperiodic potentials~\cite{Fishman,Sarma1990,Sarma_nonint} which have single particle mobility edges in one-dimension separating the localized states from the extended states. These deterministic models have lately been investigated in the setting of many-body localization (MBL) both theoretically~\cite{Huse,Sdsarma,Subroto,khemani,sdsarma2019,RG_QP,garg,Mirlin_Imb,Soumya,Yevgeny-AA,yp_nee,Piotr_AA,yp}  and experimentally~\cite{expt1,expt3,expt4,expt5,expt6} in the presence of interactions.  Despite active ongoing research in the field, the nature of the delocalization to MBL transition in these systems remains a mystery.


Finite-size scaling characteristics of the MBL transition in systems with AA potential have been studied previously using numerical exact diagonalization~\cite{khemani,Sheng,Piotr_AA,Yin}. Finite-size scaling of bipartite entanglement entropy and the standard deviation in entanglement entropy, under the assumption that the correlation length diverges as a power-law at the transition point $\xi \sim |h-h_c|^{-\nu}$, was performed to find the critical exponent $\nu < 1$ but close to one~\cite{khemani}. Later studies on finite-size scaling of entanglement entropy and sublattice fluctuations in magnetisation obtained $\nu =1.1$ for similar system sizes~\cite{Sheng}. In a more recent analysis of the finite-size scaling of the level spacing ratio across the MBL transition in interacting AA model a much smaller critical exponent of $\nu=0.54$ was obtained~\cite{Piotr_AA}. In complete contrast to these numerical studies, real space renormalization group calculation predicted a critical exponent $\nu =2.4$~\cite{RG_QP} and a similar exponent was obtained from the analysis based on local integral of motion for quasiperiodic MBL systems~\cite{LIOM_QP}. Thus, there is no consensus on the value of the correlation length critical exponent and the universality class to which the MBL transition in quasiperiodic systems belongs to.

A major issue in understanding the nature of the MBL transition in quasiperiodic systems is the lack of any generalized criterion like  Chayes-Chayes-Fisher-Spencer (CCFS)~\cite{CCFS} which is applicable for continuous transitions in systems with fully random disorder. For systems with quasiperiodic potential, CCFS criterion does not hold but there is an analogue of the relevance-irrelevance Harris-criterion~\cite{Harris} for quasiperiodic systems known as Luck criterion~\cite{Luck}. According to Luck criterion, a continuous transition in a clean d-dimensional system is stable with respect to the quasi-periodicity in the system if the correlation-length critical exponent $\nu > 1/d$~\cite{Luck}. But situations in which the transition is caused by quasi-periodicity itself and the clean system does not undergo any transition, relevance-irrelevance Luck criterion should not be applied. The MBL transition in quasiperiodic systems should, in principle, belong to a different universality class than the MBL transition in random disordered systems. Unlike the deterministic quasiperiodic potential, MBL systems with random disorder occasionally exhibit regions of very weak and large disorder. In systems with random disorder, these uncommon regions are known to be significant close to the MBL transition~\cite{RG_Ehud,RG_Potter,RG_Dumit,Huevneers,khemani_PRX} while these rare regions are absent in deterministic quasiperiodic systems. Second, while the delocalized side of systems with quasiperiodic potential has ballistic dynamics for the non-interacting case and super-diffusive dynamics for the interacting case~\cite{Dhar,Yevgeny-AA,yp_nee}, the delocalized side of systems with random disorder exhibits diffusive dynamics. MBL transition in quasiperiodic systems must therefore be distinct from MBL transition in random systems.


% Figure environment removed


In this work we study a one-dimensional system of fermions in the presence of AA potential~\cite{AA} and a generalized AA potential, labelled as aperiodic potential, which is known to feature single particle mobility edges~\cite{Fishman,Sarma1990,Sarma_nonint}. We examine these models in the limit of no interaction as well as in the presence of nearest neighbour interactions. We mainly explore the universal properties of  single-particle excitations across the localization and MBL  transitions. To the best of our knowledge, single particle excitations have not been explored so far for MBL systems with AA and aperiodic potential. For the many-body interacting system we compute single-particle Green's function in real space in highly excited many-body eigenstates of the system and analyze the corresponding local density of states (LDOS).  
We demonstrate that for both the models studied, the finite-size scaling of the ratio of the typical to average value of the LDOS yields correlation length critical exponent $\nu \ge 1$ for the interacting as well as non-interacting system. This should be compared with the finite-size scaling of single-particle LDOS for the MBL system with random disorder where $\nu_{rand} \ge 2$ and obey the CCFS criterion~\cite{Atanu2}. Further for MBL systems with quasiperiodic potentials finite-size scaling of the level spacing ratio shows $\nu <1$ which is consistent with earlier study on the AA model~\cite{Piotr_AA} and is also very close to the exponent obtained for MBL systems with random disorder~\cite{Piotr,Atanu2}. Additionally, the transition point obtained from level spacing ratio $h_{lsr}$ is smaller than that from the finite size scaling of LDOS $h_{c}$. This difference in transition points was also observed for MBL systems with random disorder~\cite{Atanu2} although the width $h_{c}-h_{lsr}$ of the intermediate phase is comparatively smaller for MBL systems with quasi-periodic potentials. 

The rest of the paper is organised as follows. In Section II, we introduce the models explored in this work. In section III, we study single particle LDOS for the non-interacting models and perform finite-size scaling analysis using the cost function.  In section IV we study single particle LDOS and level spacing ratio for the interacting models. We perform finite-size scaling for both the quantities. In section V we summarize our results and conclude with some remarks and open questions. 
% Figure environment removed

\section{II. Model}
We study a model of spin-less fermions in one-dimension described by the following Hamiltonian  
\be
H=-t\sum_{i}[c^\dagger_ic_{i+1}+h.c.] + \sum_i h_i n_i +\sum_{i} V n_in_{i+1}
\label{model}
\ee
with periodic boundary conditions. Here $t$ is the nearest neighbor hopping amplitude fixed to be one here, $V$ is the strength of nearest neighbour repulsion between Fermions and $h_i$ is the on-site potential of the form $h_i=h \cos(2\pi\beta i^n+\phi)$ where $n$ is a real number, $\beta = \frac{\sqrt{5}-1}{2}$ is an irrational number and $\phi$ is an offset. We study this model at half-filling of fermions. This model has been studied extensively~\cite{AA,Hashomoto,Fishman,Sarma1990,Sarma_nonint} in the non-interacting limit ($V=0$). For $n=1$ case, $h_i$ corresponds to the Aubry-Andre potential~\cite{AA} which has a delocalization-localization transition at $h=2$. For $0 \le n < 1$, the model is known to have single particle mobility edges at $\pm |2t-h|$ for $h <2t$ while for $h >2t$, all the single particle states are localized~\cite{Fishman,Sarma1990}.  For $n \ge 2$, the system behaves very much like the Anderson model with fully random disorder though the regime $1<n<2$ is not so well studied~\cite{Fishman}.

We study this model for $n=1$, that is the AA model and  $n=1/2$ where we label it as Aperiodic potential for $V=0$ and $V=t$.
The model in \Eqn{model} can also be mapped to a model of interacting spin-1/2 particles by the Jordan-Wigner transformation and has been studied extensively in the context of MBL, both, for $n=1$~\cite{Huse,khemani,Sheng,sdsarma2019,RG_QP,Mirlin_Imb,Soumya,Yevgeny-AA,Piotr_AA,Yin} and $n=1/2$~\cite{Subroto,garg,yp_nee,yp}. %and a transition from the delocalized phase to the MBL phase is seen as the disorder strength $h$ is increased. 
However, to the best of our knowledge, the analysis of the single-particle excitations and the corresponding LDOS to characterize the MBL phase and to understand the nature of the localization transition has not been explored so far for these models and this is the main focus of our work.


%The local density of states is defined as $\rho_{n}(i, \omega) = \left( - \frac{1}{\pi} \right ) Im \left [ G_{n}(i, i, \omega) \right]$.
%and the self energy matrix is obtained from generalized Dyson equation $\boldsymbol {\Sigma}_n (\omega)\equiv \mbf{{{{G^{-1}_{0}(\omega) - G^{-1}_{n}(\omega)}}}}$ for the $nth$ eigenstate.
%Here, $\mbf{G_0}(\omega)$ is the non-interacting Green's function matrix of the disordered system in \Eqn{model}. The scattering rate is identified 
%as ${\Gamma}_n(i, \omega) = - Im \left[\Sigma_n (i,i,\omega)\right]$.
\section{III. LDOS for Non-Interacting quasiperiodic models}
In this section we discuss local density of states for the non-interacting AA and aperiodic models of \Eqn{model}. The LDOS for the non-interacting model is given by $\rho_i(\omega)=\sum_n|\Psi_n(i)|^2\delta(\omega-E_n)$ where $E_n$ are the eigenvalues of the Hamiltonian in \Eqn{model} for $V=0$ and $\Psi_n(i)$ are the corresponding eigenfunctions. We introduce a small infinitesimal $\eta$ to broaden the
delta functions into Lorentzians such that $\rho_i(\omega)=\frac{1}{\pi}\frac{\eta|\Psi_n(i)|^2}{(\omega-E_n)^2+\eta^2}$. We chose $\eta$ to be a few times the average eigen value spacing. The typical value of LDOS $\rho_{typ}(\omega)$ is obtained by calculating the geometric average over the lattice sites, energy bin and various independent disorder configurations while the average value $\rho_{avg}(\omega)$ is obtained by simple arithmetic average over sites, energy bin and a large number of independent disorder configurations. Fig.~\ref{AA_nonint} shows the ratio of typical to average value of the LDOS $\rho_{typ}(\omega)/\rho_{avg}(\omega)$ for $\omega\sim 0$ as a function of disorder strength $h$. For small values of $h$, $\rho_{typ} \sim \rho_{avg}$ and as $h$ increases, typical value reduces faster than the average value. For $h\ge 2$, $\rho_{typ}/\rho_{avg}$ is very small and shows a clear decrease as the chain size increases. Thus, single-particle excitations are suppressed on the localized side resulting in exponentially small values of the typical LDOS while the excitation typically propagate over large length scale on the delocalized side. One sees a clear transition at $h=2$ as expected in this non-interacting AA model and $\rho_{typ}/\rho_{avg}$ acts as a good order parameter to distinguish the delocalized phase from the localized phase.
% Figure environment removed

We perform finite-size scaling of $\rho_{typ}(\omega=0)/\rho_{avg}(\omega=0)$ assuming that the  characteristic length scale diverges with a power law $\xi \sim |h-h_c|^{-\nu}$ at the localization transition point $h_c$.
 As a result a normalized observable $X$ obeys the scaling $X[\delta,L] \sim \bar{X}(\delta L^{1/\nu})$ with $\delta=h-h_c$. To have a quantitative estimate of the scaling collapse, we calculate the cost-function  defined as ~\cite{Prosen,Somen,Atanu2}. 
\be
C_X=\frac{\sum_{j=1}^{N_{total}-1}|X_{j+1}-X_j|}{max\{X_j\}-min\{X_j\}} -1
\label{cost}
\ee
Here $N_{total}$ is the total number of values of $\{X_i\}$ for various values of disorder $h$ and system sizes $L$.  We use the known value of the transition point, $h_c=2$, and obtain the critical exponent at the transition point by calculating the minimum of the cost function. Middle panel in Fig.~\ref{AA_nonint} shows the cost function $C_X$ vs the critical exponent $\nu$. The cost function has a minimum at $\nu \sim 1.42 > 1$ which is consistent with earlier works on AA model which predicted $\nu \ge 1$ based on finite-size scaling of inverse participation ratio (IPR)~\cite{Hashomoto} . In the third panel of Fig.~\ref{AA_nonint} scaling collapse of $\rho_{typ}(\omega=0)/\rho_{avg}(\omega=0)$ for $\nu=1.422$ and $h_c=2$ is presented. A good quality of scaling collapse is obtained close to the transition point and even away from it on the localised side of the transition.


Next we study LDOS for the non-interacting limit of the aperiodic model with $n=0.5$. The ratio of typical to average LDOS $\rho_{typ}(\omega=0)/\rho_{avg}(\omega=0)$ remains close to one for a larger range of $h$ compared to the AA model as shown in Fig.~\ref{Ap_nonint}. The system undergoes a delocalization-localization transition at $h=2$ with $\rho_{typ}(\omega=0)/\rho_{avg}(\omega=0)$ becoming very strongly system size dependent for $h >2$. In this regime, the typical value of LDOS reduces much faster than the average value as $L$ and $h$ increase. Again we calculated the cost function as in Eq,~\ref{cost} and determined the value of the critical exponent $\nu$ at the transition point $h_c=2.0t$ from the minimum of the cost function. Interestingly the cost function minimum occurs at $\nu=1.392$  which is very close to the value of $\nu$  obtained for AA model. Eventhough the aperiodic model has single particle mobility edges while all the single particle states are delocalized for $h<h_c$ for the AA model, in the non-interacting limit both the models belong to the same universality class. This is because both the models have been studied for the same irrational parameter $\beta$ in the onsite potential~\cite{Hashomoto}.

But there is a significant distinction. For the AA model, LDOS for any frequency $\omega$ shows the delocalization to localization transition at $h_c=2$ with the same value of the critical exponent $\nu$ because all the single-particle states are localized for $h>2$ and all the states are extended for $h <2$. For the aperiodic model, the system is known to have single particle mobility edges at $E_c=\pm |2t-h|$ for $h < 2t$. We studied single particle excitations at $\omega=E_c$ and analysed the finite size scaling of $\rho_{typ}(\omega=E_c)/\rho_{avg}(\omega=E_c)$. As shown in the Appendix A, for $E_c=-0.4$, $\rho_{typ}(\omega=E_c)/\rho_{avg}(\omega=E_c)$ shows a transition at $h=1.6$, for $E_c=-0.8$ the transition occurs at $h=1.2$ and so on. But the single parameter scaling does not work very well for these transitions at $h_{ME}=2t-|E_c|$  and the $\nu$ corresponding to the minimum of the cost function does not provide a good scaling collapse of the data. One of the reason for this is that though the single-particle states with $|E| > |E_c|$ get localised for $ h \ge h_{ME}$, the single particle states of $|E| < |E_c|$ are still extended along with  multifractal states close to the mobility edges. The details about finite-size scaling of LDOS at $\omega=E_c$ with $E_c$ finite are given in Appendix A. 

% Figure environment removed
\section{IV. Nature of MBL transition in quasiperiodic systems}
In this section, we study single-particle Green's function in real space for interacting models in~\Eqn{model}. We will mainly focus on the Green's function calculated for the  eigenstates in the middle of the many-body spectrum. This is because MBL transition involves highly excited many-body eigenstates~\cite{Alet_rev, Abanin_rev,Huse_rev}. Many-body eigenstates in the middle of the spectrum get localized in the end as the disorder strength is increased for a fixed strength of interaction. Density of states for the model in \Eqn{model} is sharply peaked in the middle of the spectrum and hence an infinite temperature limit, which basically gives average over the entire spectrum, will also have a dominant contribution from states in the middle of the spectrum.

Thus, we calculate single particle Green's function for $E=\frac{E_n-E_{min}}{E_{max}-E_{min}} \sim 0.5$.
The Green's function in the $nth$ eigenstate is defined as $ G_{n}(i,j, t) = -i \Theta(t) \langle \Psi_n  \vert \{ c_i(t), c_j^\dg(0)\} \vert \Psi_n \rangle$
where $i,j$ are lattice site indices. The Fourier transform of $G_n(i,j,t)$ in the Lehmann representation can be written  as
% Figure environment removed
\be
G_n(i,i,\omega) = \sum_m \frac{|\la\Psi_m|c^\dagger_i|\Psi_n\ra|^2}{\omega+i\eta-E_m+E_n} +\frac{|\la\Psi_m|c_i|\Psi_n\ra|^2}{\omega+i\eta+E_m-E_n}
\label{Gn}
\ee
Here if $|\Psi_n\ra$ is the $nth$ eigenstate of the Hamiltonian in \Eqn{model} for $N_e$ particles in the chain, states $|\Psi_m\ra$ used in the 
first (second) terms in \Eqn{Gn} are obtained from the diagonalization of $N_e+1$ ($N_e-1$) particle systems. $\eta$ is a positive infinitesimal 
and is set to be a small finite value for sufficient broadening of the delta function into Lorentzian. Note that though we are calculating $G_n(i,i,\omega)$ mainly for many-body eigenstates in the middle of the spectrum, the Lehmann sum in Eqn~\ref{Gn} still runs over all values of $E_m$.  All the data shown for the interacting model is for $V=t$. 

\subsection{Interacting AA model}
Fig.~\ref{AA_int} shows the ratio of the typical to average value of the LDOS for interacting AA model as a function of disorder strength $h$ for various system sizes. In close similarity to the non-interacting case, for weak disorder $\rho_{typ}(\omega=0) \sim \rho_{avg}(\omega=0)$  with the ratio being close to one. However, the ratio increases as the system size increases for small values of $h$ in contrast to the non-interacting weak disorder case. As the disorder strength increases, $\rho_{typ}(\omega=0)/\rho_{avg}(\omega=0)$ decreases and becomes vanishingly small in the MBL phase.

We further calculate the cost function defined in Eqn.~\ref{cost} in the $h-\nu$ plane as one needs to determine the transition point $h_c$ as well as the critical exponent $\nu$ for the interacting model. To determine the exact position of the transition point $h_c$ and the critical exponent at the transition point, the cost-function $C(h_c,\nu)$ is minimised w.r.t $\nu$ for each value of $h_c$. This results in $C_{min}^{\nu}$  which has been plotted as a function of $h_c$ in the left panels of Fig.~\ref{minm_AA}. The global minima w.r.t $h_c$ is obtained by finding the minima of $C_{min}^{\nu}$ as a function of $h_c$. Similarly, minimising the cost function w.r.t $h_c$ for each value of $\nu$ results in $C_{min}^h$ vs $\nu$ plot shown in the right panels. The global minima w.r.t $\nu$ is obtained by finding the minima of $C_{min}^h$ w.r.t $\nu$. As one can see from Fig.~\ref{minm_AA}, that the cost function has a minimum at $h_c \sim 3.94$ and $\nu\sim 1.21$. This value of the critical exponent is close to the one obtained from the finite-size scaling of the bipartite entanglement entropy and sublattice magnetization for the interacting AA model~\cite{Sheng}. With these values of $h_c$ and $\nu$ we obtain a very good quality scaling collapse of the data as shown in the third panel of Fig.~\ref{AA_int}, especially in the vicinity of the transition. This shows that even in the interacting AA model, critical exponent $\nu > 1$. Interestingly the value of $\nu$ obtained is very close to that obtained in earlier work from finite-size scaling of bipartitie entanglement entropy and sublattice magnetisation~\cite{Sheng}.
% Figure environment removed
% Figure environment removed

We also analyzed the finite-size scaling of the level spacing ratio averaged over the entire spectrum, which is frequently used to study the MBL transition. As shown in Appendix B, Fig.~\ref{lsr_AA}, the cost function has a minimum at $\nu < 1$ as observed in earlier studies~\cite{Piotr_AA}. A proper minimisation procedure of the cost function, shows that the minimum occurs at $h_{lsr}=3.527$ and $\nu_{lsr}=0.537$ (Fig.~\ref{lsr_minmAA}). This value of exponent $\nu$ is consistent with the recent work on AA model~\cite{Piotr_AA} and is also very close to the value obtained for the MBL systems with random disorder~\cite{Atanu2}.  Though the transition point obtained from level spacing ratio is not very different from the one obtained from LDOS, the critical exponents are very different.  We believe that the level spacing ratio obeys dimensionality of the effective Anderson model in Fock space and hence might have critical exponent $\nu \ge 1/d_F$ where $d_F \sim L$ is the typical connectivity of the Fock space configurations as suggested earlier~\cite{Atanu2}. 
It is important to notice that the scaling collapse for level spacing ratio using the parameters obtained from the minimization of the cost function is not as good as the collapse for the ratio of typical to average values of LDOS and may be improved by studying larger system sizes. 
% Figure environment removed

There are two important observations to be made here. Firstly, the transtion point $h_c$ obtained from the finite-size scaling of LDOS is consistent with the predictions from local integrals of motion for the interacting AA model~\cite{LIOM_QP} and is slightly larger than the transition point obtained from level spacing ratio $h_{lsr}$. Studies on spin chains using time dependent variation principle have also found the transition point from density imbalance to be larger than the one obtained from level spacing ratio~\cite{Mirlin_Imb}. One possible explanation for this may be that the two quantities have different approach to the thermodynamic limit and the two transition points should come closer for bigger system sizes.

% Figure environment removed

% Figure environment removed
\subsection{ MBL transition in interacting Aperiodic  Model}
In this section, we analyse the LDOS and the level spacing ratio for the interacting system with aperiodic potential. At half-filling, for $h <2$ this model is known to be delocalized even in the presence of weak interactions even though the non-interacting system has some localized states~\cite{garg}. In order to determine the transition point and the nature of the transition in the interacting aperiodic model with a finite interaction $V=1$, we again analyse the single-particle excitations produced in highly excited many-body eigenstates with $E\sim 0.5$. We calculate the ratio of the typical to average value of the LDOS $\rho_{typ}(\omega=0)/\rho_{avg}(\omega=0)$  vs disorder for various system sizes. The behaviour of LDOS for the interacting aperiodic model is very similar to the one seen for the interacting AA model with single particle excitations being typically allowed on the delocalized  side for weak disorder while the single particle excitations are exponentially suppressed for larger values of $h$ resulting in vanishing small values of $\rho_{typ}$ in the MBL phase. The cost function in $(h_c,\nu)$ plane is shown in the middle panel of Fig.~\ref{GAA_int} which clearly has minima at a slightly larger values of $h_c$ and $\nu$ compared to the interacting AA model. Systematic minimisation of the cost function indicates that the minimum occurs at $h_c \sim 5.545$ and $\nu \sim 1.763$ as shown in Fig~\ref{GAA_minmint}.  We get a very good scaling collapse of the ratio of typical to average value of the LDOS with these values of $h_c$ and $\nu$ as shown in the third panel of Fig.~\ref{GAA_int}. We would like to emphasize that even for the interacting aperiodic model the critical exponent $\nu \ge 1$ while the finite-size scaling of LDOS for the interacting one-dimensional model with random disorder shows $\nu \ge 2$~\cite{Atanu2}.

We further studied the finite-size scaling of the level spacing ratio. As shown in Fig.~\ref{lsr_GAAint}, the cost function has a minimum at $\nu =0.68$ and $h_{lsr}=4.378$. In complete similarity with the interacting AA model discussed above and the system with random disorder~\cite{Atanu2}, the critical exponent obtained from level spacing ratio for the interacting aperiodic model is much smaller than the exponent obtained from the finite-size scaling of the LDOS. Furthermore, both the quasiperiodic models and the random disorder model of MBL, have $\nu_{lsr}$ extremely close to each other with $\nu_{lsr} \in (0.53:0.64)$. This low value of exponent from level spacing ratio in comparison to the exponent obtained from other physical quantities has been observed in many earlier works consistently~\cite{Piotr,Atanu2,Piotr_AA} and to the best of our knowledge has not been understood yet. As we mentioned before, a possible explanation is that the level spacing ratio obeys dimensionality of the effective Anderson model in Fock space and hence might have critical exponent $\nu \ge 1/d_F$ where $d_F \sim L$ is the typical connectivity of a Fock space configuration~\cite{Atanu2}.

Coming to the difference in transition points, for the interacting aperiodic model the difference between $h_{lsr}$ and $h_c$ obtained from finite-size scaling of LDOS is even larger than that for the interacting AA model though it is smaller in comparison to the system with random disorder~\cite{Atanu2}. 
Though one would expect that the two transitions should merge in the thermodynamic limit, with the current analysis, we can not also rule out the possibility of two genuine transitions, first at $h_{lsr}$ at which ergodic to non-ergodic transition takes place followed up the second transition to MBL phase at $h_c$. On the other hand, it is hard to come-up with any intuitive reason for this difference in the two transition points. In MBL systems with random disorder, where $h_c-h_{lsr}$ is much larger~\cite{Atanu2} compared to what we have observed for the interacting aperiodic model, rare region effects~\cite{RG_Ehud,RG_Potter,RG_Dumit,Huevneers,khemani_PRX} are supposed to be the reason behind this intermediate phase. This difference in the two transition points for the MBL systems with random disorder is also consistent with the scenario of system-wide rare resonances~\cite{avalanche}. But in the  deterministic model that we are studying here, there are no rare regions of disorder and hence it is not easy to come up with an explanation for the mechanism behind an intermediate phase or a cascade of transitions for the interacting AA model. Thus, we believe that the two transition points should merge for larger system sizes.

\subsection{Broadening dependence of the typical LDOS}
So far we have presented results for LDOS for some finite small value of the broadening $\eta$. In this section, we justify our choice of the broadening used in this work. Typical value of the LDOS depends on the broadening $\eta$ used in the single particle Green's function (Eqn.~\ref{Gn}). In the localized phase, the typical value of the LDOS scales proportionally to $\eta$. For a system in the thermodynamic limit, in the delocalized phase the typical LDOS is independent of $\eta$. For a finite size system, this independence of typical LDOS is seen for a range of $\eta$ between the level spacing in the system of size $L$ and the level spacing in the system of size equal to the correlation length~\cite{Mirlin_LDOS,Altshuler_LDOS,LDOS3}. In this work, we have shown data for a particular value of $\eta$ which has been determined keeping above mentioned conditions in mind. For a finite size system,we fix eta and increase the system size in order to achieve the limit of $\eta \rightarrow 0$ which is consistent with the approach used in most of the numerical works e.g.~\cite{Parisi}. For the interacting aperiodic model, Fig.~\ref{LDOS_eta} shows the $\eta$ dependence of the typical value of the LDOS.  The first panel of Fig.~\ref{LDOS_eta} shows the typical value of LDOS vs $h$ for various values of $\eta$. For $h \ll h_c$, $\rho_{typ}(\omega=0)$ is almost independent of $\eta$ for $\eta> 2 \delta$ where $\delta$ is the average level spacing of the system under consideration and depends upon $h$ and $L$. This behaviour of LDOS for $h \ll h_c$ is shown more clearly in the middle panel of Fig.~\ref{LDOS_eta} where $\rho_{typ}(\omega=0)$ has been plotted as a function of $\eta$ for a couple of $h$ values in the weak disorder limit. The right panel shows the LDOS vs $\eta$ for $h \gg h_c$. As one can see for $0.005\le \eta \le 0.03$ typical LDOS increases monotonically with $\eta$ for $h > h_c$ while it is almost independent of $\eta$ in the delocalized phase.
Based on this analysis, we have studied finite-size scaling of the ratio of typical to average value of the LDOS for the interacting aperiodic model for $\eta=0.005$. We have chosen $\eta$ for other cases also in the similar fashion. To provide an estimate of the $\eta$ dependence of the transition point and the critical exponent obtained from LDOS, in Fig.~\ref{eta0.01} we have shown the finite size scaling of the ratio of typical to average LDOS for $\eta=0.01$ as well. As one can see, there is a slight change in the value of the transition point and the critical exponent for $\eta=0.01$ in comparison to that obtained from $\eta=0.005$ but for both the cases $\nu \ge 1$ and not close to the critical exponent obtained from the finite size scaling of the level spacing ratio. Further, the transition points obtained for both the values of $\eta$  are close to each other within numerical precision. Upon further increasing $\eta$, $\rho_{typ}(\omega=0)$ becomes almost system size independent for the entire range of the disorder strength, which indicates that $\eta\ge 0.035$ is too large and will wash out the intrinsic features of the LDOS.

% Figure environment removed


\section{V. Conclusions and Discussions}
In this work, we investigated the characteristics of the localization and MBL transition in deterministic quasiperiodic and aperiodic chains. In particular, we calculated the local density of states (LDOS) of single-particle excitations across the delocalization-localization transition in the interacting as well as non-interacting systems. For the interacting system, we analysed single-particle excitations generated in highly excited many-body eigenstates. We performed finite-size scaling of the ratio of the typical to average value of the LDOS, assuming that the characteristic length scale $\xi$ diverges as a power law at the transition point $\xi \sim |h-h_c|^{-\nu}$.  In the non-interacting as well as interacting quasiperiodic and aperiodic systems, we found the critical exponent $\nu \ge 1$ which is consistent with the IPR scaling for non-interacting AA model~\cite{Hashomoto} and the finite-size scaling of bipartite entanglement entropy for the interacting AA model~\cite{Sheng}.

We also studied level spacing ratio of consecutive eigenvalues for MBL systems with quasiperiodic and aperiodic potentials. Finite-size scaling of level spacing ratio under the assumption of power-law divergence of the correlation length gives $\nu \in[0.54,0.64]$ for both the deterministic models studied. Interestingly, the value of $\nu$ for deterministic systems  is very close to that obtained for the MBL system with random disorder~\cite{Piotr,Atanu2}. It is well known that $\nu$ resulting from level spacing ratio severely violates the CCFS criterion for MBL systems with random disorder. 
One of the potential causes of this might be that the level spacing ratio obeys dimensionality of the effective Anderson model in the Fock space~\cite{Atanu2} rather than the  physical dimension of the system. Also the transition point obtained from level spacing ratio $h_{lsr}$  is always smaller than $h_c$  obtained from LDOS for systems with quasiperiodic as well we random disorder. Whether the two transitions will merge in the thermodynamic limit needs to be explored.

Intuitively, one would expect the MBL transition in quasiperiodic systems to be in a different universality class than the MBL transition in systems with random disorder. MBL systems with random disorder have rare regions of very weak and large disorder while the deterministic quasiperiodic potential does not have those rare regions. The delocalized side of the random disorder systems have diffusive dynamics while the delocalized side of the systems with quasiperiodic potentials have super-diffusive dynamics.
From the point of view of the level spacing ratio, the MBL systems with random disorder and quasiperiodic potential seems to belong to the same universality class though one would not expect it to be so. Our finite-size scaling analysis of the LDOS reveals  a clear distinction between the MBL systems with quasiperiodic and random potentials. According to the finite-size scaling of LDOS $\nu\ge 1$ for MBL systems with quasiperiodic and aperiodic potential,  whereas $\nu \ge 2$ for MBL systems with random disorder~\cite{Atanu2}. 
 It would be interesting to come up with more physical quantities which can support these findings and can help in understanding the nature of the MBL transition in quasiperiodic systems.

% Figure environment removed


\section{Acknowledgements}
A.G. would like to acknowledge A. D. Mirlin for many insightful discussions and critical comments. A.G. would also like to acknowledge V. Ravi Chandra and Atanu Jana for discussions on related projects. Y. P. would like to acknowledge the National Research Foundation of Korea for the financial assistance.  We acknowledge National Supercomputing Mission (NSM) for providing computing resources of PARAM Shakti at IIT Kharagpur, which is implemented by C-DAC and supported by the Ministry of Electronics and Information Technology (MeitY) and Department of Science and Technology (DST), Government of India and SINP central cluster facilities.

\section{Appendix A. Finite-size scaling across the single-particle mobility edges in systems with  Aperiodic potential}
In the main text we studied single particle excitation with $\omega=0$ for the non-interacting model with aperiodic potential. However, this system is known to have single particle mobility edges at $E_c=\pm|2t-h|$ for $h<2t$ in the non-interacting limit. A natural question of interest is to find out whether mobility edges $\pm |E_c|$ can be captured by the single-particle LDOS and what is the finite-size scaling of the LDOS around $E_c$. Hence, we calculated the single-particle LDOS for $\omega=-0.4$ and $-0.8$ which are expected to undergo delocalization-localization transition at $h_c=1.6$ and $1.2$ respectively. Fig.~\ref{GAA_Ec} shows the ratio of the typical to average value of the local density of states $\rho_{typ}(\omega=-0.4)/\rho_{avg}(\omega=-0.4)$ as a function of disorder $h$ for various system sizes. For $h<1.6$, $\rho_{typ}(\omega=-0.4) \sim \rho_{avg}(\omega=-0.4)$ with the ratio being close to one. For $h>1.6$, the typical value decreases faster than the average value with increase of $h$ as well as the system size, such that the ratio becomes vanishingly small in the thermodynamic limit. We calculated the cost function using Eqn.~\ref{cost} in the main text and following the assumption that the correlation length has a power-law divergence at the transition point. Though, the cost function at $h=1.6$ has a nice minimum at $\nu =1.62$, the scaling collapse shown in the third panel of the Fig.~\ref{GAA_Ec} with $\nu=1.62$ and $h_c=1.6$ is not as good as shown in Fig[2] for the excitations at $\omega=0$. Similar behaviour is seen for excitations at $\omega=-0.8$ and LDOS for these excitations also do not show a good scaling collapse. One of the reason for this may be that though the single-particle states with $|E|> |E_c|$ undergo a localization transition at $h_{ME}=2t-|E_c|$, the system at $h_{ME}$ still has very large number of extended states for $E < |E_c|$ and multifractal states close to the mobility edges. Since at $h_{ME}$ the system does not really undergo a complete delocalization to localization transition, it might not obey the same scaling as the localization transition at $h_c=2t$.  

% Figure environment removed
\section{Appendix B. Finite-size scaling of level spacing ratio for the interacting AA model}
In this Appendix, we discuss the finite-size scaling of the level spacing ratio for the MBL system with Aubry-Andre potential. Fig.~\ref{lsr_AA} shows the level spacing ratio $r_n=\frac{min\{\Delta_n,\Delta_{n+1}\}}{max\{\Delta_n,\Delta_{n+1}\}}$ with $\Delta_n=E_{n+1}-E_n$ as a function of $h$ for various system sizes. The data shown has been obtained by averaging over the entire spectrum for a given disorder configuration and then averaged over a large number of independent disorder configurations. As expected, for weak disorder $r_{avg}$ is close to the average value for Wigner-Dyson distribution and for strong disorder $r_{avg}$ is close to the average value for a Poisson distribution. The cost function calculated using Eqn. [2] in the main text, is shown in the middle panel of Fig.~\ref{lsr_AA}. A two step minimization of the cost function as described in the main text shows that the cost function has a minimum at $h_{lsr} \sim 3.527$ and $\nu \sim 0.537$. Finite-size scaling of $r_{avg}$ shows a reasonably good scaling collapse albeit the quality of the collapse is not as good as for the LDOS. 




\bibliography{bibliography}


\end{document}
