\documentclass[prl,aps,twocolumn,floats,nofootinbib,superscriptaddress]{revtex4-1}
%\documentclass[prl,aps,twocolumn,floats,nofootinbib]{revtex4}
\usepackage{amsmath,amssymb,graphicx,psfrag}
\usepackage[colorlinks,linkcolor={blue},citecolor={blue},urlcolor={blue}]{hyperref}% add hypertext capabilities
%\usepackage{babel}
\newcommand{\fig}[2]{% Figure removed}
\begin{document}
\renewcommand{\ni}{{\noindent}}
\newcommand{\dprime}{{\prime\prime}}
\newcommand{\be}{\begin{equation}}
\newcommand{\ee}{\end{equation}}
\newcommand{\bea}{\begin{eqnarray}} 
\newcommand{\eea}{\end{eqnarray}}
\newcommand{\la}{\langle}
\newcommand{\ra}{\rangle} 

\newcommand{\dg}{\dagger}
\newcommand\lbs{\left[}
\newcommand\rbs{\right]}
\newcommand\lbr{\left(}
\newcommand\rbr{\right)}
\newcommand\f{\frac}
\newcommand\e{\epsilon}
\newcommand\ua{\uparrow}
\newcommand\da{\downarrow}
\newcommand{\bcen}{\begin{center}}
\newcommand{\ecen}{\end{center}}
\newcommand{\btab}{\begin{tabular}}
\newcommand{\etab}{\end{tabular}}
\newcommand{\bdes}{\begin{description}}
\newcommand{\edes}{\end{description}}
\newcommand{\mc}{\multicolumn}
\newcommand{\ul}{\underline}
\newcommand{\non}{\nonumber}
\newcommand{\etal}{et.~al.\ }
\newcommand{\half}{\frac{1}{2}}
\newcommand{\bary}{\begin{array}}
\newcommand{\eary}{\end{array}}
\newcommand{\benum}{\begin{enumerate}}
\newcommand{\eenum}{\end{enumerate}}
\newcommand{\bitem}{\begin{itemize}}
\newcommand{\eitem}{\end{itemize}}
\newcommand{\cuup}[1]{c_{#1 \uparrow}}
\newcommand{\cdown}[1]{c_{#1 \downarrow}}
\newcommand{\cdup}[1]{c^\dagger_{#1 \uparrow}}
\newcommand{\cddown}[1]{c^\dagger_{#1 \downarrow}}
%
% bold greek characters
%
\newcommand{\beps}{\mbox{\boldmath $ \epsilon $}}
\newcommand{\bsig}{\mbox{\boldmath $ \sigma $}}
\newcommand{\bpi}{\mbox{\boldmath $ \pi $}}
\newcommand{\bkap}{\mbox{\boldmath $ \kappa $}}
\newcommand{\bgam}{\mbox{\boldmath $ \gamma $}}
\newcommand{\bphi}{\mbox{\boldmath $ \phi $}}
\newcommand{\balp}{\mbox{\boldmath $ \alpha $}}
\newcommand{\beot}{\mbox{\boldmath $ \eta $}}
\newcommand{\btau}{\mbox{\boldmath $ \tau $}}
\newcommand{\blam}{\mbox{\boldmath $ \lambda $}}
\newcommand{\bomg}{\mbox{\boldmath $ \omega $}}
\newcommand{\bOmg}{\mbox{\boldmath $ \Omega $}}
\newcommand{\bxhi}{\mbox{\boldmath $ \xi $}}
\newcommand{\bmu} {\mbox{\boldmath $ \mu $}}
\newcommand{\bnu} {\mbox{\boldmath $ \nu $}}
\newcommand{\bdelta}{{\boldsymbol{\delta}}}
\newcommand{\bTheta}{{\boldsymbol{\Theta}}}
\newcommand{\bpsi}{\mbox{\boldmath $ \psi $}}
\newcommand{\brho}{\mbox{\boldmath $ \rho $}}
\newcommand{\bGam}{\mbox{\boldmath $ \Gamma $}}
\newcommand{\bLam}{\mbox{\boldmath $ \Lambda $}}
\newcommand{\bPhi}{\mbox{\boldmath $ \Phi $}}
%
% bold latin
%
%\newcommand{\ba} { \mbox{\boldmath $a$}}
\newcommand{\ba} { \bm{a} }
\newcommand{\bb} { \mbox{\boldmath $b$}}
%\newcommand{\bc} { \mbox{\boldmath $c$}}
\newcommand{\bc} { {\mathbf c} }
\newcommand{\bd} { \mbox{\boldmath $d$}}
\newcommand{\bff}{ \mbox{\boldmath $f$}}
\newcommand{\bg} { \mbox{\boldmath $g$}}
\newcommand{\bh} { \mbox{\boldmath $h$}}
\newcommand{\bi} { \mbox{\boldmath $i$}}
\newcommand{\bj} { \mbox{\boldmath $j$}}
\newcommand{\bk} { \bm{k} }
%\newcommand{\bk} { \mbox{\boldmath $k$}}
\newcommand{\bl} { \mbox{\boldmath $l$}} 
\newcommand{\bmm} { \mbox{\boldmath $m$}}
\newcommand{\bn} { \mbox{\boldmath $n$}}
\newcommand{\bo} { \mbox{\boldmath $o$}}
\newcommand{\bp} { \bm{p} }
\newcommand{\bq} { \bm{q} }
%\newcommand{\bq} { \mbox{\boldmath $q$}}
\newcommand{\br} { \boldsymbol{r}}
\newcommand{\bs} { \mbox{\boldmath $s$}}
\newcommand{\bt} {\boldsymbol{t}} 
\newcommand{\bu} { \mbox{\boldmath $u$}}
\newcommand{\bv} { \mbox{\boldmath $v$}}
\newcommand{\bw} { \mbox{\boldmath $w$}}
\newcommand{\bx} { \mbox{\boldmath $x$}}
\newcommand{\by} { \mbox{\boldmath $y$}}
\newcommand{\bz} { \mbox{\boldmath $z$}}
\newcommand{\bA} { \mbox{\boldmath $A$}}
\newcommand{\bB} { \mbox{\boldmath $B$}}
\newcommand{\bC} { \mbox{\boldmath $C$}}
\newcommand{\bD} { \mbox{\boldmath $D$}}
\newcommand{\bF} { \mbox{\boldmath $F$}}
\newcommand{\bG} { \mbox{\boldmath $G$}}
\newcommand{\bH} { \mbox{\boldmath $H$}}
\newcommand{\bI} { \mbox{\boldmath $I$}}
\newcommand{\bJ} { \mbox{\boldmath $J$}}
\newcommand{\bK} { \mbox{\boldmath $K$}}
\newcommand{\bL} { \mbox{\boldmath $L$}}
\newcommand{\bM} { \mbox{\boldmath $M$}}
\newcommand{\bN} { \mbox{\boldmath $N$}}
\newcommand{\bO} { \mbox{\boldmath $O$}}
\newcommand{\bP} { \mbox{\boldmath $P$}}
\newcommand{\bQ} { \boldsymbol{Q} }
%\newcommand{\bQ} { \mbox{\boldmath $Q$}}
\newcommand{\bR} { {\mathbf R} }
%\newcommand{\bR} { \mbox{\boldmath $R$}}
\newcommand{\bS} { \mbox{\boldmath $S$}}
\newcommand{\bT} { \mbox{\boldmath $T$}}
\newcommand{\bU} { \mbox{\boldmath $U$}}
\newcommand{\bV} { \mbox{\boldmath $V$}}
\newcommand{\bW} { \mbox{\boldmath $W$}}
\newcommand{\bX} { \mbox{\boldmath $X$}}
\newcommand{\bY} { \mbox{\boldmath $Y$}}
\newcommand{\bZ} { \mbox{\boldmath $Z$}}
\newcommand{\bzero} { \mbox{\boldmath $0$}}
\newcommand{\bfell} {\mbox{\boldmath $ \ell $}}


%
% special math symbols
%
\newcommand{\dou}{\partial}
\newcommand{\leftjb} {[\![}
\newcommand{\rightjb} {]\!]}
\newcommand{\ju}[1]{ \leftjb #1 \rightjb }
\newcommand{\D}[1]{\mbox{d}{#1}} 
\newcommand{\grad}{\mbox{\boldmath $\nabla$}}
\newcommand{\modulus}[1]{|#1|}
\renewcommand{\div}[1]{\grad \cdot #1}
\newcommand{\curl}[1]{\grad \times #1}
\newcommand{\mean}[1]{\langle #1 \rangle}
\newcommand{\bra}[1]{{\langle #1 |}}
\newcommand{\ket}[1]{| #1 \rangle}
\newcommand{\braket}[2]{\langle #1 | #2 \rangle}
\newcommand{\dbdou}[2]{\frac{\dou #1}{\dou #2}}
\newcommand{\dbdsq}[2]{\frac{\dou^2 #1}{\dou #2^2}}
\newcommand{\Pint}[2]{ P \!\!\!\!\!\!\!\int_{#1}^{#2}}
\newcommand{\Itwo}{{\mathds{1}}}
\newcommand{\Hds}{{\mathds{H}}}
\newcommand{\cH}{{\cal H}}
\newcommand{\cS}{{\cal S}}


%
% Abbreviations for equations etc
% 
\newcommand{\prn}[1] {(\ref{#1})}
\newcommand{\sect}[1] {Section~\ref{#1}}
\newcommand{\Sect}[1] {Section~\ref{#1}}


%
% Other utilities
%
\newcommand{\uncon}[1]{\centerline{\epsfysize=#1 \epsfbox{/usr2/yogeshwar/styles/construction.pdf}}}
\newcommand{\checkup}[1]{{(\tt #1)}\typeout{#1}}
\newcommand{\ttd}[1]{{\color[rgb]{1,0,0}{\bf #1}}}
%\newcommand{\ttd}[1]{}
\newcommand{\red}[1]{{\color[rgb]{1,0,0}{\protect{#1}}}}
\newcommand{\blue}[1]{{\color[rgb]{0,0,1}{#1}}}
\newcommand{\green}[1]{{\color[rgb]{0.0,0.5,0.0}{#1}}}
\newcommand{\citebyname}[1]{\citeauthor{#1}\cite{#1}}
\newcommand{\myfigwidth}{0.95\columnwidth}
\newcommand{\myhalffig}{0.475\columnwidth}
\newcommand{\mythirdfig}{0.33\columnwidth}
\newcommand{\signum}[0]{\mathop{\mathrm{sign}}}
\newcommand{\skup}{\ket{s \uparrow}}
\newcommand{\skdn}{\ket{s \downarrow}}
\newcommand{\pkup}{\ket{p \uparrow}}
\newcommand{\pkdn}{\ket{p \downarrow}}
\newcommand{\sbup}{\bra{s \uparrow}}
\newcommand{\sbdn}{\bra{s \downarrow}}
\newcommand{\pbup}{\bra{p \uparrow}}
\newcommand{\pbdn}{\bra{p \downarrow}}

% Abbreviations for equations etc
\newcommand{\Eqn}[1] {Eqn.~(\ref{#1})}
%\newcommand{\Sect}[1] {Section~\ref{#1}}
\newcommand{\Fig}[1]{Fig.~\ref{#1}}

\title{Single-particle excitations across the many-body localization transition in quasi-periodic systems}
\author{Yogeshwar Prasad}
\affiliation{Department of Liberal Studies, Kangwon National University, Samcheok, 25913, Republic of Korea}
\affiliation{Theory Division, Saha Institute of Nuclear Physics, 1/AF Bidhannagar, Kolkata 700 064, India}
\author{Arti Garg}
\affiliation{Theory Division, Saha Institute of Nuclear Physics, 1/AF Bidhannagar, Kolkata 700 064, India}
 \affiliation{Homi Bhabha National Institute, Training School Complex, Anushaktinagar, Mumbai 400094, India}
\vspace{0.2cm}
\begin{abstract}
\vspace{0.3cm}
       {We study many-body localization transition in one dimensional systems in the presence of a deterministic quasi-periodic potential. We focus on single-particle excitations produced in highly excited many-body eigenstates obtained through single-particle Green's function in real space. A finite-size scaling analysis of the ratio of the typical to average value of the local density of states of single particle excitations is performed assuming that the correlation length $\xi$ diverges at the transition point with a power-law $\xi \sim |h-h_c|^{-\nu}$. Both for the Aubry-Andre (AA) model and the generalized AA model, the finite size scaling of the local density of states obeys the single parameter scaling. A good quality scaling collapse is obtained for $\nu \ge 1$ which satisfies the generalized Luck's criterion for quasiperiodic systems. This analysis supports the continuous nature of the many-body localization transition in systems with AA and generalized AA potentials.}

         %We study localization and many-body localization transition in a one dimensional system in the presence of deterministic aperiodic potentials. We focus on single particle excitations obtained through single particle Green's function in real space, both, for the non-interacting and the interacting models across the transitions. We do finite-size scaling of the ratio of the typical to average value of the local density of states of single particle excitations, assuming that the correlation length $\xi$ diverges at the transition point with a power-law $\xi \sim (h-h_c)^{-\nu}$. In the non-interacting as well as in the interacting case, both for the Aubry-Andre (AA) model and the generalized AA model, the finite size scaling of the local density of states obeys this single parameter scaling showing a good scaling collapse for $\nu \ge 1$ satisfying the generalized Luck's criterion for systems with quasiperiodic potentials. This analysis supports the continuous nature of the localization and many-body localization transition in systems with AA and generalized AA potentials.} 
         
\vspace{0.cm}
\end{abstract} 
\maketitle
\section{I. Introduction}
Transport properties of a system are significantly influenced by disorder. Any tiny quantity of disorder is sufficient to localise all the single particle states in a non-interacting system in one dimension with quenched random disorder~\cite{Anderson}. But it is feasible to see a localization to delocalization transition even in one dimension in models with deterministic quasiperiodic potentials, such as the Aubry-Andre (AA) model~\cite{AA} and models with Fibonacci potentials~\cite{Giamarchi}. There are also examples of deterministic aperiodic potentials~\cite{Fishman,Sarma1990,Sarma_nonint} which have single particle mobility edges in one-dimension separating the localized states from the extended states. These deterministic models have lately been investigated in the setting of many-body localization (MBL) both theoretically~\cite{Huse,Sdsarma,Subroto,khemani,sdsarma2019,RG_yao,garg,Mirlin_Imb,Soumya,Yevgeny-AA,yp_nee,Piotr_AA,yp}  and experimentally~\cite{expt1,expt3,expt4} in the presence of interactions.  Despite ongoing research in the field, the nature of the delocalization to MBL transition in these systems remains a mystery.

MBL transition does not follow the standard classification scheme of the conventional phase transitions in statistical physics. In the MBL systems with random disorder, a criterion by Chayes-Chayes-Fisher-Spencer (CCFS)~\cite{CCFS} is used to understand the nature of the MBL transition. According to the CCFS criterion, for all systems with quenched random disorder that undergo a continuous transition with power-law divergence of the correlation length at the transition point $\xi \sim |h-h_c|^{-\nu}$,  $\nu \ge 2/d$, where $d$ is the physical dimension for the system, irrespective of whether there is an analogous transition in the clean system~\cite{CCFS,Laumann}.  This is a much stronger criterion than the original one by Harris~\cite{Harris} according to which if $\nu \ge 2/d$ for a clean system then disorder is irrelevant. Higher dimensional  Anderson model ($3d$ and above) have been shown to satisfy this bound for the critical exponent~\cite{Boris,Kramer,Slevin,Biroli,Huse_AM}. For MBL systems, though real-space renormalization group studies predicted a critical exponent $\nu \sim 3$~\cite{RG_Ehud,RG_Potter,RG_zhang,RG_Dumit}, numerical studies based on level spacing ratio and entanglement entropy, found the critical exponent $\nu \sim 1$ violating the CCFS bound~\cite{Bardarson,Luitz,khemani,Piotr}. Only very recently, finite-size scaling of single-particle local density of states and scattering rates obtained from single particle Green's function and self energy in real space, have been shown to provide the critical exponent $\nu$ which satisfies the CCFS inequality for a class of models with random disorder~\cite{Atanu2}.

% Figure environment removed


For systems with quasiperiodic potential, CCFS criterion does not hold and there is no known generalization of the CCFS criterion. For systems with quasiperiodic potential, there is an analogue of the Harris criterion known as the Luck criterion, which states that if a clean system has a continuous transition with power-law divergence of the correlation length, the transition in the clean system is stable with respect to the quasi-periodicity in the system for the critical exponent $\nu \ge 1/d$~\cite{Luck}. Therefore, the Luck criterion should not be applied to a system where the transition is caused by quasi-periodicity itself and the clean system does not have a transition. Based on qualitative justifications for systems with deterministic potentials, we generalised the CCFS criterion and argue that $\nu \ge 1/d$ is true for continuous transitions in the presence of quasiperiodic deterministic potential. This generalised criterion is now referred to as the Luck-CCFS criterion. For AA model with nearest neighbour interactions, real space renormalization group calculation predicted a critical exponent $\nu =2.4$~\cite{RG_QP} and a similar exponent was obtained from the analysis based on local integral of motion~\cite{LIOM_QP}. But  numerical studies which focused mainly on the finite-size scaling of level spacing ratio and entanglement entropy found the critical exponent $\nu < 1$ \cite{khemani,Piotr_AA}. Thus, from the point of view of level spacing ratio, MBL transition in systems with random disorder and quasiperiodic potential belong to the same universality class. The same can be concluded from the renormalization group approaches for the two class of MBL systems. But MBL system with random disorder has rare regions of very weak and large disorder while the deterministic potential does not have those rare regions. These rare regions are known to be significant near the MBL transition in systems with random disorder~\cite{RG_Ehud,RG_Potter,RG_Dumit,Huevneers,khemani_PRX}. Hence, one would expect that MBL transition in quasiperiodic systems must be different from the MBL transition in random systems.

With this goal, in this work we study a one-dimensional system of fermions in the presence of AA potential~\cite{AA} and a generalized AA potential which is known to feature single particle mobility edges~\cite{Fishman,Sarma1990,Sarma_nonint}. We investigate these models both in the non-interacting limit as well as in the presence of nearest neighbour interactions. We mainly explore the universal properties of  single-particle excitations across the localization and MBL  transitions. For the many-body interacting system, we compute single-particle Green's function in real space in highly excited many-body eigenstates of the system and analyse the corresponding local density of states(LDOS). %In a recent work on MBL systems with random disorder, 
We demonstrate that finite-size scaling of the ratio of typical to average value of LDOS for both the interacting and non-interacting systems yields $\nu \ge 1$ for both the AA and the generalised AA models, being consistent with Luck-CCFS criterion. This suggests that the MBL transition in these systems with deterministic quasiperiodic potentials  is continuous in nature and belongs to a different universality class than the MBL transition in systems with random disorder. Further, for both the deterministic MBL models the transition point obtained from level spacing ratio $h_{lsr}$ is smaller than that from the finite size scaling of LDOS $h_{c}$. This difference in transition points was also observed for MBL systems with random disorder~\cite{Atanu2} although the width $h_{c}-h_{lsr}$ of the intermediate phase is comparatively smaller for MBL systems with quasi-periodic potentials. 

The rest of the paper is organised as follows. In Section II, we introduce the models explored in this work. In section III, we study single particle LDOS for the non-interacting models and perform finite-size scaling analysis using the cost function.  In section IV we study single particle LDOS and level spacing ratio for the interacting models. We perform finite-size scaling for both the quantities and show that though $\nu$ extracted from LDOS satisfies Luck-CCFS criterion, $\nu_{LSR} \le 1$. The two quantities also undergo transition at different values of the disorder strength. In section V finally we summarize our results and conclude with some remarks and open questions. 
% Figure environment removed

\section{II. Model}
We study a model of spin-less fermions in one-dimension described by the following Hamiltonian  
\be
H=-t\sum_{i}[c^\dagger_ic_{i+1}+h.c.] + \sum_i h_i n_i +\sum_{i} V n_in_{i+1}
\label{model}
\ee
with periodic boundary conditions. Here $t$ is the nearest neighbor hopping amplitude fixed to be one here, and $h_i$ is the on-site potential of the form $h_i=h \cos(2\pi\beta i^n+\phi)$ where $n$ is a real number, $\beta = \frac{\sqrt{5}-1}{2}$ is an irrational number and $\phi$ is an offset. We study this model at half-filling of fermions. This model has been studied extensively~\cite{Fishman,Sarma1990,Sarma_nonint} in the non-interacting limit. For $n=1$ case, $h_i$ corresponds to the Aubry-Andre potential~\cite{AA} which has a delocalization-localization transition at $h=2$. For $0 \le n < 1$, the model is known to have single particle mobility edges at $\pm |2t-h|$ for $h <2t$ while for $h >2t$, all the single particle states are localized.  For $n \ge 2$, the system behaves very much like the Anderson model with fully random disorder though the regime $1<n<2$ is not so well studied.

We study this model for $n=1$, that is the AA model and  $n=1/2$ where we label it as Aperiodic potential.
The model in \Eqn{model}, which can also be mapped to a model of interacting spin-1/2 particles by the Jordan-Wigner transformation, has been studied in the context of MBL, both, for $n=1$ as well as $n=1/2$ and a transition from the delocalized phase to the MBL phase is seen as the disorder strength $h$ increases. 
However, to the best of our knowledge, the analysis of the single-particle excitations and their LDOS for this model to look for signatures of MBL and to understand the nature of the localization transition has not been attempted before and this is the main focus of our work.


%The local density of states is defined as $\rho_{n}(i, \omega) = \left( - \frac{1}{\pi} \right ) Im \left [ G_{n}(i, i, \omega) \right]$.
%and the self energy matrix is obtained from generalized Dyson equation $\boldsymbol {\Sigma}_n (\omega)\equiv \mbf{{{{G^{-1}_{0}(\omega) - G^{-1}_{n}(\omega)}}}}$ for the $nth$ eigenstate.
%Here, $\mbf{G_0}(\omega)$ is the non-interacting Green's function matrix of the disordered system in \Eqn{model}. The scattering rate is identified 
%as ${\Gamma}_n(i, \omega) = - Im \left[\Sigma_n (i,i,\omega)\right]$.
\section{III. LDOS for Non-Interacting AA Model and aperiodic model}
In this section we present results for the local density of states for the non-interacting limit of model in \Eqn{model}. The LDOS for the non-interacting model is given by $\rho_i(\omega)=\sum_n|\Psi_n(i)|^2\delta(\omega-E_n)$ where $E_n$ are the eigenvalues of the Hamiltonian in \Eqn{model} for $V=0$ and $\Psi_n(i)$ are the corresponding single particle eigenstates. We introduce a small infinitesimal $\eta$ to broaden the
delta functions into Lorentzians such that $\rho_i(\omega)=\frac{1}{\pi}\frac{\eta|\Psi_n(i)|^2}{(\omega-E_n)^2+\eta^2}$. We chose $\eta$ to be a few times the average eigen value spacing. The typical value of LDOS $\rho_{typ}(\omega)$ is obtained by calculating the geometric average over the lattice sites, energy bin and various independent disorder configurations while the average value $\rho_{avg}(\omega)$ is obtained by simple arithmetic average over sites, energy bin and a large number of independent disorder configurations. Fig.~\ref{AA_nonint} shows the ratio of typical to average value of the LDOS $\rho_{typ}(\omega)/\rho_{avg}(\omega)$ for $\omega\sim 0$ as a function of disorder strength $h$. For small values of $h$, $\rho_{typ} \sim \rho_{avg}$ and as $h$ increases, typical value reduces faster than the average value. For $h\ge 2$, $\rho_{typ}/\rho_{avg}$ is very small and shows a clear decrease as the chain size increases. Thus, single-particle excitations are suppressed on the localized side resulting in exponentially small values of the typical LDOS while the excitation typically propagate over large length scale on the delocalized side. One sees a clear transition at $h=2$ as expected in this non-interacting AA model and $\rho_{typ}/\rho_{avg}$ acts as a good order parameter to distinguish the delocalized phase from the localized phase.
% Figure environment removed

We perform finite-size scaling of $\rho_{typ}/\rho_{avg}$ at $\omega=0$ assuming that the  characteristic length scale diverges with a power law at the localization transition point $\xi \sim |h-h_c|^{-\nu}$.
 As a result a normalized observable $X$ obeys the scaling $X[\delta,L] \sim \bar{X}(\delta L^{1/\nu})$ with $\delta=h-h_c$. To have a quantitative estimate of the scaling collapse, we calculate the cost-function  defined as ~\cite{Prosen,Somen,Atanu2}. 
\be
C_X=\frac{\sum_{j=1}^{N_{total}-1}|X_{j+1}-X_j|}{max\{X_j\}-min\{X_j\}} -1
\label{cost}
\ee
Here $N_{total}$ is the total number of values of $\{X_i\}$ for various values of disorder $h$ and system sizes $L$.  We use the known value of the transition point, $h_c=2$, and obtain the critical exponent at the transition point by calculating the minimum of the cost function. Middle panel in Fig.~\ref{AA_nonint} shows the cost function $C_X$ vs the critical exponent $\nu$. The cost function has a minimum at $\nu \sim 1.42 > 1$ and satisfies the generalized Luck-CCFS criterion. In the third panel of Fig.~\ref{AA_nonint} scaling collapse of $\rho_{typ}/\rho_{avg}$ for $\nu=1.422$ and $h_c=2$ is presented. A good quality of scaling collapse is obtained close to the transition point and even away from it on the localised side of the transition point, indicating that the nature of transition in a non-interacting AA model is continuous and satisfies Luck-CCFS criterion.
%We arrange all $N_{total}$ values of $\{X_i\}$ according to increasing values of $(W-W_C)L^{1/\nu}$. $C_X$ should be zero close for a perfect data collapse but for the finite size data that we have, we look for a minimum of the cost function with respect to the exponent $\nu$ for $W_c$ values which are close to the intuitive guess of the transition point. 
%We study the ratios of typical to average LDOS and scattering rates introduced earlier using a single parameter scaling form $(X[\delta,L] \sim \bar{X}(\delta L^{1/\nu}))$, 
%which has also been used to study scaling properties of other quantities relevant in context of MBL~\cite{Luitz,Bardarson,khemani,Piotr}. As we will show shortly,  
%this scaling ansatz results in very good scaling collapse for these quantities. 

Next we study LDOS for the non-interacting limit of the aperiodic model with $n=0.5$. Since this model has both localized and extended states for $h <2$ with mobility edges at $E_c=\pm|2t-h|$, $\rho_{typ}(\omega=0)/\rho_{avg}(\omega=0)$ remains close to one for a larger range of $h$ compared to the AA model as shown in Fig.~\ref{Ap_nonint}. One can clearly see the system undergoing a delocalization-localization transition at $h=2$ with $\rho_{typ}(\omega=0)/\rho_{avg}(\omega=0)$ becoming very strongly system size dependent for $h >2$. In this regime, the typical value of LDOS reduces much faster than the average value as $L$ and $h$ increase. Again we calculated the cost function as in Eq,~\ref{cost} and using the value $h_c=2$, we determined the value of the critical exponent at the transition point from the minimum of the cost function. Interestingly the cost function minimum occurs at $\nu=1.392$  which is very close to the value of $\nu$  obtained for AA model.  It seems that though the aperiodic model has single particle mobility edges, in the non-interacting limit both the models belong to the same universality class and satisfy the Luck-CCFS criterion.

However, there is a crucial difference. For AA model, LDOS for any frequency $\omega$ will show transition at $h_c=2$ with same value of the critical exponent because all the single particle states are localized for $h>2$ and all are extended for $h <2$. For the aperiodic model, the system is known to have single particle mobility edges at $E_c=\pm |2t-h|$ for $h < 2t$. We  tried to explore the nature of the transition  near the single particle mobility edges by doing a finite size scaling of $\rho_{typ}(\omega=E_c)/\rho_{avg}(\omega=E_c)$. As shown in the Appendix A, for $E_c=-0.4$, the transition takes place at $h=1.6$, for $E_c=-0.8$ the transition occurs at $h=1.2$ and so on. We noticed that the single parameter scaling does not work very well for transitions at $h <2$  and the $\nu$ corresponding to minimum of the cost function does not give a good scaling collapse. One of the reason for this is that though  single particle states of $E \le E_c$ and $E > -E_c$ get localised for $ h>2t-|E_c|$, the single particle states of $|E| < |E_c|$ are still extended. The details about finite-size scaling of LDOS at $\omega=E_c$ with $E_c$ finite are given in Appendix A. 
%% Figure environment removed
% Figure environment removed
\section{IV. Nature of MBL transition in quasiperiodic systems}
In this section, we study Green's function for interacting models in~\Eqn{model}. We will mainly focus on the Green's function calculated for the  eigenstates in the middle of the many-body spectrum to study the MBL transition. This is because MBL transition includes highly excited many-body eigenstates~\cite{Alet_rev, Abanin_rev,Huse_rev} and many-body eigenstates in the middle of the spectrum get localized in the end as the disorder strength increases for a fixed strength of interaction. Density of states of the model in \Eqn{model} is sharply peaked in the middle of the spectrum for sufficiently large system, and hence an infinite temperature limit, which basically gives the average over the entire spectrum, will also have a dominant contribution from states in the middle of the spectrum.

Thus, we calculate single particle Green's function for $E=\frac{E_n-E_{min}}{E_{max}-E_{min}} \sim 0.5$, though the Lehmann sum in Eqn~\ref{Gn} still runs over all values of $E_m$. 
The Green's function in the $nth$ eigenstate is defined as $ G_{n}(i,j, t) = -i \Theta(t) \langle \Psi_n  \vert \{ c_i(t), c_j^\dg(0)\} \vert \Psi_n \rangle$
where $i,j$ are lattice site indices. The Fourier transform of $G_n(i,j,t)$ in the Lehmann representation, can written  as
% Figure environment removed
\be
G_n(i,i,\omega) = \sum_m \frac{|\la\Psi_m|c^\dagger_i|\Psi_n\ra|^2}{\omega+i\eta-E_m+E_n} +\frac{|\la\Psi_m|c_i|\Psi_n\ra|^2}{\omega+i\eta+E_m-E_n}
\label{Gn}
\ee
Here if $|\Psi_n\ra$ is the $nth$ eigenstate of the Hamiltonian in \Eqn{model} for $N_e$ particles in the chain, states $|\Psi_m\ra$ used in the 
first (second) terms in \Eqn{Gn} are obtained from the diagonalization of $N_e+1$ ($N_e-1$) particle systems. $\eta$ is a positive infinitesimal 
and is set to be a small finite value for sufficient broadening of the delta function into Lorentzian. All the data shown for the interacting model is for $V=t$. 

\subsection{Interacting AA model}
Fig.~\ref{AA_int} shows the ratio of the typical to average value of the LDOS for interacting AA model as a function of disorder strength $h$ for various system sizes. Again for weak disorder $\rho_{typ}(\omega=0) \sim \rho_{avg}(\omega=0)$  with the ratio being close to one. Also the ratio increases as the system size increases for small values of $h$. As the disorder strength increases, $\rho_{typ}(\omega=0)/\rho_{avg}(\omega=0)$ decreases and becomes vanishingly small in the MBL phase. Now we calculate the cost function as in Eqn.~\ref{cost} in the $h-\nu$ plane as one needs to determine the transition point $h_c$ as well as the critical exponent $\nu$. To determine the exact position of the transition point $h_c$ and the critical exponent at the transition point, the cost-function $C(h_c,\nu)$ is minimised w.r.t $\nu$ for each value of $h_c$. This results in $C_{min}^{\nu}$  which has been plotted as a function of $h_c$ in the left panels of Fig.~\ref{minm_AA}. The global minima w.r.t $h_c$ is obtained by finding the minima of $C_{min}^{\nu}$ as a function of $h_c$. Similarly, minimising the cost function w.r.t $h_c$ for each value of $\nu$ results in $C_{min}^h$ vs $\nu$ plot shown in the right panels. The global minima w.r.t $\nu$ is obtained by finding the minima of $C_{min}^h$ w.r.t $\nu$. As one can see from Fig.~\ref{minm_AA}, that the cost function has a minimum at $h_c \sim 3.94$ and $\nu\sim 1.21$. With these values of $h_c$ and $\nu$ we obtain a very good quality scaling collapse of the data as shown in the third panel of Fig.~\ref{AA_int}, especially in the vicinity of the transition. This shows that even in the interacting AA model, critical exponent $\nu > 1$ satisfying the Luck-CCFS criterion hinting towards the continuous nature of the MBL transition in these systems. 
% Figure environment removed
% Figure environment removed

We also studied the finite-size scaling of the level spacing ratio averaged over the entire spectrum and many independent disorder configurations, which has been studied in various earlier works for interacting AA model. As shown in Appendix B, Fig.~\ref{lsr_AA}, the cost function has a minimum at $\nu < 1$. A proper minimisation procedure of the cost function, shows that the minimum occurs at $h_{lsr}=3.527$ and $\nu_{lsr}=0.537$ (Fig.~\ref{lsr_minmAA}). This value of exponent $\nu$ is consistent with the recent work on AA model~\cite{Piotr_AA} and is also very close to the value obtained for the MBL systems with random disorder~\cite{Atanu2}, though it is almost half of the value predicted in some of the earlier numerical works~\cite{khemani} on AA model. It should be noted that earlier works which predicted $\nu \sim 1$ for interacting AA model did not use any quantitative analysis to check the quality of the scaling collapse while our works performs a systematic minimisation of the cost function to obtain the critical exponent and the transition point following some of the more recent works~\cite{Prosen,Piotr_AA}.  It is important to notice that the scaling collapse for level spacing ratio using the parameters obtained from the minimization of the cost function is not as good as the collapse for the ratio of typical to average values of LDOS and may be improved by studying larger system sizes. 
% Figure environment removed

There are two important point to be noted here. Firstly, the transtion point $h_c$ obtained from the finite-size scaling of LDOS is consistent with the predictions from local integrals of motion for the interacting AA model~\cite{LIOM_QP} and is larger than the transition point obtained from level spacing ratio $h_{lsr}$. Studies on spin chains using time dependent variation principle have also found the transition point from density imbalance to be much larger than the one obtained from level spacing ratio~\cite{Mirlin_Imb}. One possible explanation for this may be that the two quantities have different approach to the thermodynamic limit though we can not rule out the possibility of two transitions, first at $h_{lsr}$ at which ergodic to non-ergodic transition takes place followed up the second transition to MBL phase at $h_c$. In MBL systems with random disorder, where a much broader intermediate phase is observed~\cite{Atanu2}, rare region effects are supposed to be the reason behind this intermediate phase. But in deterministic model that we are studying here, there are no rare regions of disorder and hence it is not easy to come up with an explanation for the mechanism behind an intermediate phase preceding the MBL phase.

\subsection{ MBL transition in interacting Aperiodic  Model}
We further calculate the LDOS and the level spacing ratio for the interacting system with aperiodic potential. At half-filling, for $h <2$ this model is known to be delocalized even in the presence of weak interactions even though the non-interacting system has some localized states~\cite{garg}. In order to determine the transition point and the nature of the transition in the interacting aperiodic model with a finite interaction $V=1$, we again analyse the single-particle excitations produced in highly excited many-body eigenstates with $E\sim 0.5$. We calculate the ratio of the typical to average value of the LDOS $\rho_{typ}(\omega=0)/\rho_{avg}(\omega=0)$  vs disorder for various system sizes. The behaviour of LDOS for the interacting aperiodic model is very similar to the one seen for the interacting AA model with single particle excitations being typically allowed on the delocalized  side for weak disorder while the single particle excitations are exponentially suppressed for larger values of $h$ resulting in vanishing small values of $\rho_{typ}$ in the MBL phase. The cost function in $(h_c,\nu)$ plane is shown in the middle panel of Fig.~\ref{GAA_int} which clearly has minima at a slightly larger values of $h_c$ and $\nu$ compared to the interacting AA model. Systematic minimisation of the cost function indicates that the minimum occurs at $h_c \sim 5.56$ and $\nu \sim 1.75$ as shown in Fig~\ref{GAA_minmint}.  We get a very good scaling collapse of the ratio of typical to average value of the LDOS with these values of $h_c$ and $\nu$ as shown in the third panel of Fig.~\ref{GAA_int}. We would like to emphasize that even for the interacting aperiodic model the critical exponent $\nu \ge 1$ satisfying the Luck-CCFS criterion indicating towards the continuous nature of the MBL transition in this system.

We also studied the finite-size scaling of the level spacing ratio. As shown in Fig.~\ref{lsr_GAAint}, the cost function has a minimum at $\nu =0.68$ and $h_{lsr}=4.378$. Thus, the critical exponent obtained from level spacing ratio for the interacting aperiodic model violates Luck-CCFS criterion. Further, for the interacting aperiodic model the difference between $h_{lsr}$ and $h_c$ obtained from finite-size scaling of LDOS is even larger than that for the interacting AA model though both the models are deterministic and are not supposed to have any rare region Griffith phase.   The most plausible explanation would be that the two transition points may merge for larger system sizes.  
One final thing to keep in mind is that the scaling collapse for level spacing ratio using the parameters obtained from the minimization of the cost function is not as good as for the LDOS which might be due to either too small system sizes or a genuine effect of the physical quantity under consideration which might not follow single parameter scaling or might not undergo a continuous transition at $h_{lsr}$. But our analysis clearly indicates that the LDOS undergo a continuous transition from the delocalized phase to the MBL phase of the interacting aperiodic model with the critical exponent $\nu$ satisfying Luck-CCFS criterion.




%% Figure environment removed

%% Figure environment removed

%% Figure environment removed


%\section{V. Interacting GAA Model}

%% Figure environment removed








\section{V. Conclusions and Discussions}
In this work, we investigated the characteristics of the MBL transition in deterministic quasiperiodic and aperiodic potential systems. In particular, we calculated the local density of states (LDOS) of single-particle excitations generated in highly excited many-body eigenstates. We performed finite-size scaling of the ratio of the typical to average value of the LDOS, assuming that the characteristic length scale $\xi$ diverges as a power law at the transition point $\xi \sim |h-h_c|^{-\nu}$.  In quasiperiodic and aperiodic systems, we found a strong evidence in favour of the continuous character of the MBL transition, with the critical exponent $\nu \ge 1$ consistent with the Luck-CCFS criterion.  it is interesting to note that for systems with quasiperiodic potential, renormalization group calculations and the local integrals of motion approach anticipated $\nu \ge 2$ though we always found $\nu <2$ for quasiperiodic systems from the finite-size scaling of the LDOS.
We studied LDOS both for the interacting and non-interacting systems with Aubry-Andre and aperiodic potentials. Even in the non-interacting models we found $\nu \ge 1$ and satisfies the Luck-CCFS criterion.

We also studied level spacing ratio of consecutive eigenvalues for MBL systems with quasiperiodic and aperiodic potentials. Finite-size scaling of level spacing ratio under the assumption of power-law divergence of the correlation length gives $\nu <1$ for both the deterministic models studied. Interestingly, the value of $\nu$ for deterministic systems  is very close to $\nu$ obtained for the MBL system with random disorder~\cite{Atanu2}. It is well known that $\nu$ resulting from level spacing ratio severely violates the CCFS requirement for MBL systems with random disorder and the Luck-CCFS criterion for MBL systems with quasiperiodic potential. 
One of the potential causes of this might be that the level spacing ratio obeys dimensionality of the effective Anderson model in the Fock space~\cite{Atanu2} rather than the  physical dimension of the system. Also the transition point obtained from level spacing ratio $h_{lsr}$  is always smaller than $h_c$  obtained from LDOS for systems with quasiperiodic as well we random disorder. Whether the two transitions will merge in the thermodynamic limit needs to be explored.

 From the point of view of the level spacing ratio, the MBL systems with random disorder and quasiperiodic potential seems to belong to the same universality class though one would not expect it to be so. Our finite-size scaling analysis of the LDOS reveals  a clear distinction between the MBL systems with quasiperiodic and random potentials. According to the finite-size scaling of LDOS $1 \le \nu <2$ for MBL systems with quasiperiodic and aperiodic potential,  whereas $\nu \ge 2$ for MBL systems with random disorder (as shown in one of the earlier works~\cite{Atanu2}). The critical exponent $\nu$ complies with the Luck-CCFS criterion for MBL systems with quasiperiodic and aperiodic disorder and satisfies the CCFS bound for MBL systems with random disorder.
 It would be interesting to come up with more physical quantities which can support these findings and can help in understanding the nature of the MBL transition.
%finite size scaling of LDOS for MBL systems with random disorder  \nu \ge 2 satisfying CCFS criterion. Here, we show from finite size scaling of LDOS in highly excited many body eigen states that \nu \ge 1 satisfying Luck CCFS criterion for systems with AA and aperiodic potentials.

%Even for the non interacting limit \nu \ge 1 for both the models studied indicating continuous nature of delocalization to localization transition in both the interacting and non interacting systems with quasiperiodic and aperiodic potentials.  

\section{Acknowledgements}
A.G. would like to acknowledge V. Ravi Chandra and Atanu Jana for discussions on related projects. A.G. would also like to acknowledge Science and Engineering Research Board (SERB) of Department of Science and Technology (DST), India under grant No. CRG/2018/003269 for financial support and National Supercomputing Mission (NSM) for providing computing resources of PARAM Shakti at IIT Kharagpur, which is implemented by C-DAC and supported by the Ministry of Electronics and Information Technology (MeitY) and Department of Science and Technology (DST), Government of India. Y. P. would like to acknowledge the National Research Foundation of Korea for the financial assistance. We also thank SINP central cluster facilities.
% Figure environment removed
\section{Appendix A. Finite-size scaling across the single-particle mobility edges in systems with  Aperiodic potential}
In the main text we studied single particle LDOS for the non-interacting model with aperiodic potential only for $\omega=0$. Since single particle states at $E=0$ undergo delocalization to localization transition at h=2, finite-size scaling of the LDOS shows a transition at h=2. However, this system is known to have single particle mobility edges at $E_c=\pm|2t-h|$ for $h<2t$ in the non-interacting limit. What kind of transition occurs for single particle states with $|E| > |E_c|$ is therefore a logical question to address. Hence, we calculated the single particle LDOS for $\omega=-0.4 and -0.8$ which are expected to have a transition at $h_c=1.6$ and $1.2$ respectively. Fig.~\ref{GAA_Ec} shows the ratio of the typical to average value of the local density of states as a function of disorder $h$ for various system sizes. For $h<1.6$, $\rho_{typ}(\omega=-0.4) \sim \rho_{avg}(\omega=-0.4)$ with the ratio being close to one. For $h>1.6$, the typical value decreases faster than the average value with increase of $h$ as well as the system size, such that the ratio becomes vanishingly small in the thermodynamic limit. We calculated the cost function using Eqn.~\ref{cost} in the main text and following the assumption that the correlation length has a power-law divergence at the transition point. Though, the cost function at $h=1.6$ has a nice minimum at $\nu =1.62$, the scaling collapse shown in the third panel of the Fig.~\ref{GAA_Ec} with $\nu=1.62$ and $h_c=1/6$ is not as good as shown in Fig[2] for the excitations at $\omega=0$. Similar behaviour is seen for excitations at $\omega=-0.8$ and LDOS for these excitations also do not show very good scaling collapse. One of the reason for this may be that the system at $h <2t$ still has very large number of extended states between $E < |E_c|$ and multifractal states close to the mobility edges. Since at $h<2t$ the system does not really undergo a complete delocalization to localization transition, it might not obey the same scaling as the localization transition at $h=2t$.  
%% Figure environment removed
% Figure environment removed

\section{Appendix B. Finite-size scaling of level spacing ratio for the interacting AA model}
In this section we show the finite-size scaling of the level spacing ratio of the MBL system with Aubry-Andre potential. Fig.~\ref{lsr_AA} shows the level spacing ratio $r_n=\frac{min\{\Delta_n,\Delta_{n+1}\}}{max\{\Delta_n,\Delta_{n+1}\}}$ with $\Delta_n=E_{n+1}-E_n$ as a function of $h$ for various system sizes. Data shown has been obtained by averaging over the entire spectrum for a given disorder configuration and then averaged over a large number of independent disorder configurations. As expected, for weak disorder $r_{avg}$ is close to the average value for Wigner-Dyson distribution and for strong disorder $r_{avg}$ is close to the average value for a Poisson distribution. The cost function calculated using Eqn. [2] in the main text, is shown in the middle panel of Fig.~\ref{lsr_AA}. A two step minimization of the cost function as described in the main text shows that the cost function has a minimum at $h_{lsr} \sim 3.527$ and $\nu \sim 0.537$. Finite-size scaling of $r_{avg}$ shows a reasonably good scaling collapse albeit the quality of the collapse is not as good as for the LDOS. 



% Figure environment removed

\bibliography{bibliography}


\end{document}
