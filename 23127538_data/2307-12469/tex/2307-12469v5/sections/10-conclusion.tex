\vspace{-2pt}
\section{Conclusion}
\revision{
Our study centers around answering fundamental issues of LLM-based fuzz driver generation's effectiveness.
% , including the effectiveness, the basic challenges, and the pros and cons of several designed query strategies.
To do that, we designed a dataset and six prompt strategies, and did extensive evaluation on different models and temperatures.
% (0.7+ million generated drivers, 0.85+ billion token costs).
% built a framework for evaluation in scale, and compared generated drivers with industrial used ones.
% to study in scale.
% Over 0.7 million of generated drivers are evaluated with the cost of 0.85 billion token costs (\$8,000+ charged tokens).
Our study not only established the basic understanding on this direction but also indicates the potential future improvements.
%The insights are three-fold:
%\ding{182} LLM-based generation has promising potential but also faces challenges towards high practicality;  
%% the results demonstrated the promising practicality of LLM-based generation as well as the challenges towards high practicality;
%\ding{183} the fundamental challenge roots in tackling the API-specific usage particulars while three key beneficial designs for prompting are identified and analyzed;
%\ding{184} the LLM-generated drivers can provide comparable fuzzing outcomes as the industrial used ones.
%However, larges spaces for further improvements still exist.
Furthermore, our insights have been applied into industrial practical fuzz driver generation platform.
}
\compactline
% there are still large space for further improvements, such as automatic semantic correctness validation, API usage extension, and semantic oracle generation.

% and there are three key designs can help significantly: repeatedly querying, querying with examples, and iteratively querying;
% Besides, three key designs significantly help here: 
% revealing their potential for facilitating fuzzing of API targets.
% However, practical application of LLM-generated drivers requires careful filtering to ensure effectiveness and avoid false alarms.
% We have also analyzed the effects, advantages, and disadvantages of five different types of strategies.
% Comparing the generated drivers with manually written ones, we found that LLM-generated drivers can produce competent fuzzing outcomes, but there is still room for further improvement.

\section{Data Availability}

% To facilitate the future research, we have released all the code and data involved in our study~\cite{fuzz-drvier-study-website}.
The source code and data involved in our study can be found at~\cite{fuzz-drvier-study-website}.
% to facilitate the community.
% To facilitate the community, we will open source our quiz, evaluation framework, and the data involved in the study at~\cite{fuzz-drvier-study-website}.

% \section*{Acknowledgements}

% We thank all reviewers for their insightful comments.
% This research is supported by the National Research Foundation, Singapore, the Cyber Security Agency under its National Cybersecurity R\&D Programme (NCRP25-P04-TAICeN) and AI Singapore Programme (AISG2-RP-2020-019), by Research on Intelligent Vulnerability Mining and Verification Methods Enhanced by Large Models (E4Z00211B1), by Chinese National Natural Science Foundation (Grant \#62302500).
% Any statement expressed in this material are of the author(s) and do not reflect the views of funding agencies.

% \clearpage
