\section{Conclusion}
Our study targets on establishing fundamental understandings of LLM-based fuzz driver generation.
% , including the effectiveness, the basic challenges, and the pros and cons of several designed query strategies.
To do that, we designed and analyzed query strategies systematically, built a framework for evaluation in scale, and compared generated drivers with industrial used ones.
% to study in scale.
The main insights are three-fold:
\ding{182} the results demonstrated the promising practicality of LLM-based generation;
\ding{183} the key challenge is the API-specific usage particulars required for generation and there are three key designs can help significantly: repeatedly querying, querying with examples, and iteratively querying;
% Besides, three key designs significantly help here: 
\ding{184} there are still large space for further improvements, such as automatic semantic correctness validation, API usage extension, and semantic oracle generation.

% revealing their potential for facilitating fuzzing of API targets.
% However, practical application of LLM-generated drivers requires careful filtering to ensure effectiveness and avoid false alarms.
% We have also analyzed the effects, advantages, and disadvantages of five different types of strategies.
% Comparing the generated drivers with manually written ones, we found that LLM-generated drivers can produce competent fuzzing outcomes, but there is still room for further improvement.

% \clearpage
