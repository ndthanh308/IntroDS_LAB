%%
%% This is file `sample-sigconf.tex',
%% generated with the docstrip utility.
%%
%% The original source files were:
%%
%% samples.dtx  (with options: `sigconf')
%% 
%% IMPORTANT NOTICE:
%% 
%% For the copyright see the source file.
%% 
%% Any modified versions of this file must be renamed
%% with new filenames distinct from sample-sigconf.tex.
%% 
%% For distribution of the original source see the terms
%% for copying and modification in the file samples.dtx.
%% 
%% This generated file may be distributed as long as the
%% original source files, as listed above, are part of the
%% same distribution. (The sources need not necessarily be
%% in the same archive or directory.)
%%
%%
%% Commands for TeXCount
%TC:macro \cite [option:text,text]
%TC:macro \citep [option:text,text]
%TC:macro \citet [option:text,text]
%TC:envir table 0 1
%TC:envir table* 0 1
%TC:envir tabular [ignore] word
%TC:envir displaymath 0 word
%TC:envir math 0 word
%TC:envir comment 0 0
%%
%%
%% The first command in your LaTeX source must be the \documentclass
%% command.
%%
%% For submission and review of your manuscript please change the
%% command to \documentclass[manuscript, screen, review]{acmart}.
%%
%% When submitting camera ready or to TAPS, please change the command
%% to \documentclass[sigconf]{acmart} or whichever template is required
%% for your publication.
%%
%%
\documentclass[sigconf,authorversion,nonacm]{acmart}
% \documentclass[sigconf, review, anonymous]{acmart}
% \documentclass[sigconf,anonymous]{acmart}
% \acmConference[ISSTA 2024]{ACM SIGSOFT International Symposium on Software Testing and Analysis}{16-20 September, 2024}{Vienna, Austria}
% \documentclass[sigconf]{acmart}
% \documentclass[sigconf,review,anonymous]{acmart}
% \documentclass[sigconf,review,anonymous,nonacm]{acmart}

% \acmConference[ICSE 2024]{46th International Conference on Software Engineering}{April 2024}{Lisbon, Portugal}

%%
%% \BibTeX command to typeset BibTeX logo in the docs
\AtBeginDocument{%
  \providecommand\BibTeX{{%
    Bib\TeX}}}

%% Rights management information.  This information is sent to you
%% when you complete the rights form.  These commands have SAMPLE
%% values in them; it is your responsibility as an author to replace
%% the commands and values with those provided to you when you
%% complete the rights form.
% \setcopyright{acmcopyright}
% \copyrightyear{2018}
% \acmYear{2018}
% \acmDOI{XXXXXXX.XXXXXXX}

\setcopyright{none}
% remove ACM Reference Format:
\settopmatter{printacmref=false} % Removes citation information below abstract
\renewcommand\footnotetextcopyrightpermission[1]{} % removes footnote with conference information in first column

%% These commands are for a PROCEEDINGS abstract or paper.
% \acmConference[Conference acronym 'XX]{Make sure to enter the correct
  % conference title from your rights confirmation emai}{June 03--05,
  % 2018}{Woodstock, NY}
%%
%%  Uncomment \acmBooktitle if the title of the proceedings is different
%%  from ``Proceedings of ...''!
%%
%%\acmBooktitle{Woodstock '18: ACM Symposium on Neural Gaze Detection,
%%  June 03--05, 2018, Woodstock, NY}
% \acmPrice{15.00}
% \acmISBN{978-1-4503-XXXX-X/18/06}

\usepackage{pifont}
% \usepackage{amsmath,amssymb,amsfonts}
\usepackage{algorithm, algpseudocode}
\usepackage{amsmath,amsfonts}
% \usepackage{amssymb}
% \usepackage{algorithmic}
\usepackage{graphicx}
\usepackage{textcomp}
\usepackage{tabularx}
\usepackage{color}
\usepackage{xcolor}
\usepackage{tcolorbox}
\usepackage{colortbl}
\usepackage{fvextra}
% \usepackage{minted}
\def\BibTeX{{\rm B\kern-.05em{\sc i\kern-.025em b}\kern-.08em
    T\kern-.1667em\lower.7ex\hbox{E}\kern-.125emX}}
\usepackage{subcaption}
\usepackage{xspace}
\usepackage{longtable}

% compact itemize
\usepackage{enumitem}
% \setlist{topsep=0pt, leftmargin=*}
% \setlist{noitemsep,topsep=0pt,parsep=0pt,partopsep=0pt}
\setlist{itemsep=1pt,topsep=1pt,parsep=0pt,partopsep=0pt}

\usepackage{multirow}
\usepackage{soul}


\definecolor{templateyellow}{RGB}{255, 250, 205}
\definecolor{templateblue}{RGB}{0, 0, 135}
\definecolor{templategrey}{RGB}{204, 204, 204}

\renewcommand{\algorithmicrequire}{\textbf{Input:}}
\renewcommand{\algorithmicensure}{\textbf{Output:}}
\algrenewcommand\algorithmicindent{1.5em}
\algtext*{EndWhile}% Remove "end while" text
\algtext*{EndIf}% Remove "end if" text
\algtext*{EndFor}% Remove "end if" text
\renewcommand\algorithmicthen{}
\renewcommand\algorithmicdo{}
% https://tex.stackexchange.com/questions/238574/algorithm-return-statement-does-not-begin-on-new-line
\let\oldReturn\Return
\renewcommand{\Return}{\State\oldReturn}

\makeatletter
\newlength{\trianglerightwidth}
\settowidth{\trianglerightwidth}{$\triangleright$~}
\algnewcommand{\LineComment}[1]{\Statex \hskip\ALG@thistlm $\triangleright$ #1}
\algnewcommand{\IndentLineComment}[1]{\Statex \hskip\ALG@tlm \(\triangleright\) #1}
\algnewcommand{\LineCommentCont}[1]{\Statex \hskip\ALG@thistlm%
  \parbox[t]{\dimexpr\linewidth-\ALG@thistlm}{\hangindent=\trianglerightwidth \hangafter=1 \strut$\triangleright$ #1\strut}}
\algnewcommand{\IndentLineCommentCont}[1]{\Statex \hskip\ALG@tlm%
  \parbox[t]{\dimexpr\linewidth-\ALG@tlm}{\hangindent=\trianglerightwidth \hangafter=1 \strut$\triangleright$ #1\strut}}
\makeatother

% \newcommand{\revision}[1]{{\sethlcolor{templateyellow}\hl{{#1}}}}

\newcommand{\code}[1]{{\fontfamily{cmtt}\fontseries{m}\fontshape{n}\selectfont\small{#1}}}
\newcommand*{\algofont}{\textit}

\newcommand*{\figu}{{Fig.}\xspace}
\newcommand*{\sect}{{Section}\xspace}
\newcommand*{\tabl}{{Tbl.}\xspace}
\newcommand*{\appe}{{Appendix}\xspace}

\newcommand{\tab}{\hspace*{1em}}
\newcommand{\compactline}{\looseness=-1}

% \newcommand{\revision}[1]{\textcolor{red}{{#1}}}
\newcommand{\revision}[1]{{#1}}
\newcommand{\zhc}[1]{\textcolor{cyan}{{[zhc: #1]}}}
\newcommand{\yaowen}[1]{\textcolor{red}{{[yaowen: #1]}}}
\newcommand{\lyt}[1]{\textcolor{blue}{{[lyt: #1]}}}

\usepackage{xltabular}
\newcommand{\tabincell}[2]{\begin{tabular}{@{}#1@{}}#2\end{tabular}}

\tcbuselibrary{breakable}

%%
%% Submission ID.
%% Use this when submitting an article to a sponsored event. You'll
%% receive a unique submission ID from the organizers
%% of the event, and this ID should be used as the parameter to this command.
%%\acmSubmissionID{123-A56-BU3}

%%
%% For managing citations, it is recommended to use bibliography
%% files in BibTeX format.
%%
%% You can then either use BibTeX with the ACM-Reference-Format style,
%% or BibLaTeX with the acmnumeric or acmauthoryear sytles, that include
%% support for advanced citation of software artefact from the
%% biblatex-software package, also separately available on CTAN.
%%
%% Look at the sample-*-biblatex.tex files for templates showcasing
%% the biblatex styles.
%%

%%
%% The majority of ACM publications use numbered citations and
%% references.  The command \citestyle{authoryear} switches to the
%% "author year" style.
%%
%% If you are preparing content for an event
%% sponsored by ACM SIGGRAPH, you must use the "author year" style of
%% citations and references.
%% Uncommenting
%% the next command will enable that style.
%%\citestyle{acmauthoryear}


%%
%% end of the preamble, start of the body of the document source.
\begin{document}

%%
%% The "title" command has an optional parameter,
%% allowing the author to define a "short title" to be used in page headers.
% \title{When Fuzz Driver Generation Meets Large Language Model}
\title{\revision{How Effective Are They? Exploring Large Language Model Based Fuzz Driver Generation}}

%%
%% The "author" command and its associated commands are used to define
%% the authors and their affiliations.
%% Of note is the shared affiliation of the first two authors, and the
%% "authornote" and "authornotemark" commands
%% used to denote shared contribution to the research.


\author{
\rm Cen Zhang \textsuperscript{\dag}
\quad Yaowen Zheng \textsuperscript{\dag}
\quad Mingqiang Bai \textsuperscript{\P}
\quad Yeting Li \textsuperscript{\P} \textsuperscript{\ddag}
\quad Wei Ma \textsuperscript{\dag}
\quad Xiaofei Xie \textsuperscript{\pounds}
\quad Yuekang Li \textsuperscript{\S}
\quad Limin Sun \textsuperscript{\P}
\quad Yang Liu \textsuperscript{\dag}
\\
\normalsize{
cen001@e.ntu.edu.sg
\quad baimingqiang22@mails.ucas.ac.cn
\quad yaowen.zheng@ntu.edu.sg
\quad liyeting@iie.ac.cn
\quad xfxie@smu.edu.sg
\quad yli044@e.ntu.edu.sg
\quad ma\_wei@ntu.edu.sg
\quad sunlimin@iie.ac.cn
\quad yangliu@ntu.edu.sg
}
\\
\normalsize {
\textsuperscript{\dag}Nanyang Technological University
\quad \textsuperscript{\P} Institute of Information Engineering, CAS
\quad \textsuperscript{\ddag} School of Cyber Security, University of Chinese Academy of Sciences
\quad \textsuperscript{\pounds} Singapore Management University
\quad \textsuperscript{\S} University of New South Wales
}
}


% \email{\{cen001,yaowen.zheng\}@e.ntu.edu.sg}
% \affiliation{%
%   \institution{Nanyang Technological University}
%   \country{Singapore}
% }
% \affiliation{%
%   \institution{Nanyang Technological University}
%   \country{Singapore}
% }
%\orcid{1234-5678-9012}
%\author{G.K.M. Tobin}
%\authornotemark[1]
%\email{webmaster@marysville-ohio.com}
%  \streetaddress{P.O. Box 1212}
%  \city{Dublin}
%  \state{Ohio}
%  \country{USA}
%  \postcode{43017-6221}
%
%\author{Lars Th{\o}rv{\"a}ld}
%\affiliation{%
%  \institution{The Th{\o}rv{\"a}ld Group}
%  \streetaddress{1 Th{\o}rv{\"a}ld Circle}
%  \city{Hekla}
%  \country{Iceland}}
%\email{larst@affiliation.org}
%
%\author{Valerie B\'eranger}
%\affiliation{%
%  \institution{Inria Paris-Rocquencourt}
%  \city{Rocquencourt}
%  \country{France}
%}
%
%\author{Aparna Patel}
%\affiliation{%
% \institution{Rajiv Gandhi University}
% \streetaddress{Rono-Hills}
% \city{Doimukh}
% \state{Arunachal Pradesh}
% \country{India}}
%
%\author{Huifen Chan}
%\affiliation{%
%  \institution{Tsinghua University}
%  \streetaddress{30 Shuangqing Rd}
%  \city{Haidian Qu}
%  \state{Beijing Shi}
%  \country{China}}
%
%\author{Charles Palmer}
%\affiliation{%
%  \institution{Palmer Research Laboratories}
%  \streetaddress{8600 Datapoint Drive}
%  \city{San Antonio}
%  \state{Texas}
%  \country{USA}
%  \postcode{78229}}
%\email{cpalmer@prl.com}
%
%\author{John Smith}
%\affiliation{%
%  \institution{The Th{\o}rv{\"a}ld Group}
%  \streetaddress{1 Th{\o}rv{\"a}ld Circle}
%  \city{Hekla}
%  \country{Iceland}}
%\email{jsmith@affiliation.org}
%
%\author{Julius P. Kumquat}
%\affiliation{%
%  \institution{The Kumquat Consortium}
%  \city{New York}
%  \country{USA}}
%\email{jpkumquat@consortium.net}

%%
%% By default, the full list of authors will be used in the page
%% headers. Often, this list is too long, and will overlap
%% other information printed in the page headers. This command allows
%% the author to define a more concise list
%% of authors' names for this purpose.
% \renewcommand{\shortauthors}{Trovato et al.}

%%
%% The abstract is a short summary of the work to be presented in the
%% article.
\begin{abstract}
\begin{abstract}

This paper presents a low-cost network architecture for training large language models (LLMs) at hyperscale. We study the optimal parallelization strategy of LLMs and propose a novel datacenter network design tailored to LLM's unique communication pattern. We show that LLM training generates sparse communication patterns in the network and, therefore, does not require any-to-any full-bisection network to complete efficiently. As a result, our design eliminates the spine layer in traditional GPU clusters. We name this design a \textit{Rail-only} network and demonstrate that it achieves the same training performance while reducing the network cost by 38\% to 77\% and network power consumption by 37\% to 75\% compared to a conventional GPU datacenter. Our architecture also supports Mixture-of-Expert (MoE) models with all-to-all communication through forwarding, with only 8.2\% to 11.2\% completion time overhead for all-to-all traffic. We study the failure robustness of Rail-only networks and provide insights into the performance impact of different network and training parameters. \looseness=-1


\end{abstract}


\end{abstract}

%%
%% The code below is generated by the tool at http://dl.acm.org/ccs.cfm.
%% Please copy and paste the code instead of the example below.
%%
%\begin{CCSXML}
%<ccs2012>
% <concept>
%  <concept_id>10010520.10010553.10010562</concept_id>
%  <concept_desc>Computer systems organization~Embedded systems</concept_desc>
%  <concept_significance>500</concept_significance>
% </concept>
% <concept>
%  <concept_id>10010520.10010575.10010755</concept_id>
%  <concept_desc>Computer systems organization~Redundancy</concept_desc>
%  <concept_significance>300</concept_significance>
% </concept>
% <concept>
%  <concept_id>10010520.10010553.10010554</concept_id>
%  <concept_desc>Computer systems organization~Robotics</concept_desc>
%  <concept_significance>100</concept_significance>
% </concept>
% <concept>
%  <concept_id>10003033.10003083.10003095</concept_id>
%  <concept_desc>Networks~Network reliability</concept_desc>
%  <concept_significance>100</concept_significance>
% </concept>
%</ccs2012>
%\end{CCSXML}
%
%\ccsdesc[500]{Computer systems organization~Embedded systems}
%\ccsdesc[300]{Computer systems organization~Redundancy}
%\ccsdesc{Computer systems organization~Robotics}
%\ccsdesc[100]{Networks~Network reliability}

%%
%% Keywords. The author(s) should pick words that accurately describe
%% the work being presented. Separate the keywords with commas.
\keywords{Fuzz Driver Generation, Fuzz Testing, Large Language Model}
%% A "teaser" image appears between the author and affiliation
%% information and the body of the document, and typically spans the
%% page.
%\begin{teaserfigure}
%  % Figure removed
%  \caption{Seattle Mariners at Spring Training, 2010.}
%  \Description{Enjoying the baseball game from the third-base
%  seats. Ichiro Suzuki preparing to bat.}
%  \label{fig:teaser}
%\end{teaserfigure}

%\received{20 February 2007}
%\received[revised]{12 March 2009}
%\received[accepted]{5 June 2009}

%%
%% This command processes the author and affiliation and title
%% information and builds the first part of the formatted document.
\maketitle
% \onecolumn
\pagestyle{plain}
% \pagestyle{empty}

\section{Introduction}
\label{sec:introduction}

The recent surge of Large Language Models (LLMs), such as GPT-3.5/4~\cite{bubeck_sparks_2023}, PaLM~\cite{chowdhery_palm_2022}, FLAN-T5~\cite{chung_scaling_2022}, and Alpaca~\cite{taori_stanford_2023}, has shown a promising trend of large pre-trained models to do a variety of tasks in a zero-shot setting (\ie without any new training data). Example tasks include question answering~\cite{omar2023chatgpt,robinson2023leveraging}, logic reasoning~\cite{wei_chain--thought_2023,zhou_least--most_2023}, machine translation~\cite{brants2007large,gulcehre2017integrating} \etc\ 
A number of experiments have revealed that, built on hundreds of billions of parameters, these LLMs have started to show the capability to understand the human common sense beneath the natural language and do proper reasoning and inference accordingly~\cite{bubeck_sparks_2023,nori_capabilities_2023}.

Among different applications, one particular question yet to be answered is how well LLMs can understand human mental health states through natural language.
Mental health problems represent a significant burden for individuals and societies worldwide.
A recent report suggested that more than 20\% of adults in the U.S. would experience at least one mental disorder in their lifetime~\cite{mental2022state} and 5.6\% of adults experienced a serious psychotic disorder that significantly impairs functioning~\cite{mental2023stats}. The global economy loses around \$1 trillion annually in productivity due to depression and anxiety alone~\cite{mentalcost2023}.

In the past decade, there has been a plethora of research in natural language processing (NLP) and computational social science on detecting mental health issues via online text data such as social media~(\eg \cite{guntuku_detecting_2017,eichstaedt2018facebook,coppersmith_clpsych_2015,de_choudhury_social_2013,de_choudhury_mental_2014}). However, most of these studies have focused on building domain-specific machine learning (ML) models (\ie one model for one particular task, such as stress detection~\cite{nijhawan2022stress,guntuku2019understanding}, depression prediction~\cite{eichstaedt2018facebook,tadesse2019detection,xu_leveraging_2019}, or suicide risk assessment~\cite{de_choudhury_discovering_2016,coppersmith2018natural}). Even for traditional pre-trained language models such as BERT, it needs to be finetuned for specific downstream tasks~\cite{devlin_bert_2019,liu_roberta_2019}.
Since natural language is a major component of mental health assessment and treatment~\cite{sharma2018mental,gkotsis2016language}, LLMs might be a potentially powerful tool to understand end-users' mental states based on the language users' wrote. These instruction-finetuned and general-purpose models can understand a variety of inputs and obviate the need to train multiple models for different tasks. Thus, we can envision using one LLM for a variety of mental-health-related tasks, such as multiple question-answering, reasoning, and inference.
Such a vision opens up a wide range of opportunities for UbiComp, Human-Computer Interaction (HCI), and mental health communities, such as online public health monitoring systems~\cite{patel2018psyheal,graham2019artificial}, intelligent assistants for mental counselors and supporters~\cite{sharma_towards_2021,sharma_humanai_2023}, mental-health-aware personal chatbots~\cite{abd2021perceptions,denecke2020mental}, to just name a few.
However, there is a lack of investigation into understanding, evaluating, and improving the capability of LLMs for mental health prediction tasks.

There are few very recent studies on the evaluation of LLMs (\eg ChatGPT) on mental-health-related tasks, most of which are in zero-shot settings with simple prompt engineering~\cite{yang_evaluations_2023,amin_will_2023,lamichhane_evaluation_2023}. Researchers have shown preliminary results that LLMs have some initial capability of predicting mental health disorders with natural language with some promising but still limited performance compared to state-of-the-art domain-specific NLP models~\cite{yang_evaluations_2023,lamichhane_evaluation_2023}.
This remaining gap is expected since existing general-purpose LLMs are not specifically trained on mental health tasks.
However, to achieve our vision of leveraging LLMs for mental health support and assistance, we need to answer the research question: \textbf{How to empower LLMs with more mental health domain knowledge and become an expert}?

We conducted a series of experiments with multiple LLMs, including Alpaca~\cite{noauthor_stanford_2023}, Alpaca-LoRA~\cite{hu_lora_2021}, and GPT-3.5~\cite{noauthor_introducing_2022}.
Considering the data availability, we focused on online social media data with high-quality human-generated mental health labels.
Our experiments contained three stages: (1) zero-shot prompting, where we experimented with various prompts related to mental health, (2) few-shot prompting, where we inserted examples into prompt inputs, and (3) instruction-finetuning, where we finetuned LLMs on multiple mental-health datasets with various tasks.

Our results indicate that zero-shot obtained promising but limited performance on multiple mental health prediction tasks across all models. GPT-3.5 had relatively better results since it has a larger scale. But their performance is still far from task-specific models. 
Meanwhile, providing a few shots in the prompt can improve the model performance to some extent ($\overline{\Delta}$ = 4.7\%), but the advantage is limited.
Finally and most importantly, we found that instruction-finetuning can significantly improve the model performance across multiple mental-health-related tasks at the same time. Our finetuned Alpaca, namely \textbf{Mental-Alpaca}, significantly outperforms the original GPT-3.5 ($\times$25 times of model size) by an average of 16.7\% on balance accuracy. 
Meanwhile, Mental-Alpaca can further perform on par with the task-specific state-of-the-art Mental-RoBERTa~\cite{ji_mentalbert_2021}. It is noteworthy that Mental-RoBERTa needs to be trained on each task individually, 
while our Mental-Alpaca can solve different tasks off the shelf. 
% We open-source our training code and model at [github link].
Our experiments present the first comprehensive evaluation of various techniques to enhance LLMs' capability in the mental health domain.

The contribution of our paper can be summarized as follows:
\begin{s_enumerate}
\item We present the first comprehensive evaluation of prompt engineering, few-shot, and finetuning techniques on multiple LLMs in the mental health domain.
\item With online social media data, our results reveal that finetuning on a variety of datasets can significantly improve LLM's capability on multiple mental-health-specific tasks simultaneously.
% We release our model \textbf{Mental-Alpaca} as the first open-source LLM targeted at mental health prediction tasks.
\item We provide a few technical guidelines for future researchers and developers on turning LLMs into experts in specific domains.
\end{s_enumerate}

\section{Preliminaries}
\label{sec:preliminaries}

% Figure environment removed
\noindent

\noindent
\textbf{Fuzz Driver Basics} \tab 
% key aspects of a fuzz driver
% The prerequisites can include the initialization of the target API arguments and the setup of the correct execution context.
The key components of a fuzz driver are illustrated in Figure~\ref{fig:fuzz-driver-internal}.
A typical fuzz driver includes at least three components: prerequisites initialization, execution, and post-cleaning of the target API, as mentioned in lines 3, 4, and 7.
In addition, there are three optional components listed in lines 2, 5, and 6 that can improve a driver's effectiveness.
The component mentioned in line 2 allows the driver to reject inputs with large sizes to reduce execution costs, or to split input data into several parts for testing multiple arguments of the target API.
The component mentioned in line 5 enables the driver to call additional APIs to trigger more program behaviors, which helps reveal more bugs.
Finally, the component mentioned in line 6 allows the driver to add semantic oracles to find more logical bugs.
These oracles are similar to \texttt{assert} statements used in unit tests, which abort execution when certain properties of the program are not satisfied.
Since the fuzz drivers are extensively executed with randomly mutated input data, there is a high requirement on the correctness and robustness of its API usage.
The incorrect or unrobust usage can lead to both false positives and negatives.
For instance, if a driver failed to feed the mutated data into the API, it can never find any bug inside the target.
Or if a driver passed an incorrectly initialized argument to the API, false crashes may be raised.
% In this paper, an effective fuzz driver represents the drivers which have correct API usage and produce no false positives.
% Precisely validating the effectiveness of fuzz drivers is crucial for evaluating fuzz driver generation methods.
% However, general validation techniques do not work well due to the diverse semantics on the API usages.

\noindent
\textbf{Minimum Requirements of an Effective Fuzz Driver} \tab 
% what is an effective fuzz driver
To be effective, a fuzz driver must correctly use the API and produce no false positive results.
The minimal requirements for an effective fuzz driver includes satisfying the necessary control flow dependencies and initializing the arguments correctly.
Argument initialization can be one of the following cases (in the order of simplicity):
\ding{182} \textbf{C1}: If the argument value is irrelevant or should be a naive value such as \texttt{0} or \texttt{NULL}, a variable is declared or a literal constant is used directly;
\ding{183} \textbf{C2}: If the argument is supposed to be a macro or a global variable that is already defined in common libraries or the target API's project, it is located and used;
\ding{184} \textbf{C3}: If creating the argument requires the use of common library APIs, such as creating a file and writing specific content, common practices are followed;
\ding{185} \textbf{C4}: If initializing the argument requires the output of other APIs within the project, those APIs are initialized first following the above initialization cases.

\noindent
\textbf{LLM-Based Code Generation} \tab 
LLMs provide a natural language interface that allows users to generate code through conversational queries and answers.
With this interface, code generation tasks can be completed more efficiently and with less technical expertise required.
In this study, prompt represents the content of a single query while the conversation represents one or more rounds of queries and answers sharing the same communication context.

% The type of prompt involved in the study is \textit{prefix prompt}~\cite{prompt-engineering-survey}, which expects the LLM to continue the content
% llm is query based, in a conversational style
% what is prompt, what is query, what is conversation
% we only consider the second form of prompt

\section{Methodology}

% Figure environment removed

% This section is for: How we conduct our study to answer these RQs?
% Problem scope declaration: effective fuzz driver generation, ...

\subsection{Overview of Study}

\noindent
\textbf{Overall Design} \tab 
Figure~\ref{fig:overview-for-study} shows the overall workflow of this study.
To understand the effectiveness of different LLM-based fuzz driver generation strategies, we first constructed a quiz, which contains a set of evaluation questions and the effectiveness criteria.
Each question requires the LLMs to generate a fuzz driver according to a given API and the effectiveness of the generated driver will be evaluated based on the criteria.
% An API is qualified if its project is significant, \textit{i.e.}, tested by OSS-Fuzz, and it is the core API inside the existing fuzz drivers.
% The effectiveness criteria of an API are the conditions or checks to distinguish ineffective drivers and effective ones. 
Then we built an evaluation framework to maximize the automation of the evaluation.
% The framework will add general contents of the prompts, launch LLM queries, validate the effectiveness of replies, and classify the failed validation.
Upon these, we designed and evaluated the effectiveness of different query strategies from basic to enhanced (the first two \textbf{RQs}).
% the first two RQs are studied by comparing and analyzing the evaluation results of different query strategies.
In \textbf{RQ1}, we designed and explored the effectiveness of basic strategies which use fundamental API information and have simple interactions with LLMs;
In \textbf{RQ2}, enhanced strategies which leverage extended API usage information and interactive queries are studied.
Necessary in-depth analysis were conducted to explain the benefits and limitations of these strategies.
Lastly, in \textbf{RQ3}, we compared LLM-generated fuzz drivers with OSS-Fuzz drivers to understand their positions, advantages, and disadvantages.
% The comparison involves the static metrics of the code such as the number of unique APIs and dynamic metrics including the testing coverage and unique crashes.

\noindent
\textbf{Evaluated Language Models} \tab 
As shown in Table~\ref{tab:evaluated_models}, two state-of-the-art language models of OpenAI have been studied~\cite{openai-models}.
% All the LLM-generated fuzz drivers evaluated in this study are retrieved via ChatGPT web interfaces~\cite{chatgpt-website} (release version 23 Mar 2023) based on \texttt{chatgpt-wrapper}\cite{chatgpt-wrapper} v0.7.1 (\texttt{dabe72101b}).
All the LLM-generated fuzz drivers evaluated in this study are retrieved via \texttt{chatgpt-wrapper}~\cite{chatgpt-wrapper} v0.7.1 (\texttt{dabe72101b}).
The parameters of the models are kept same as the default values in the ChatGPT official website~\cite{chatgpt-default-config}.
% , \textit{e.g.}, temperature is set as 0.9 and top\_p is 1.

% Figure environment removed

\begin{table}[!t]
\centering
\caption{LLMs Evaluated in This Work.}
\label{tab:evaluated_models}
\resizebox{0.8\linewidth}{!}{
\begin{tabular}{lllll}
\toprule
 Model & Abbr & ChatGPT Version & Max Tokens & Training Data \tabularnewline
\midrule 
\rowcolor{black!10}
gpt3.5-turbo & \texttt{gpt3.5} & 23 Mar 2023 & 4,096 tokens & Up to Sep 2021 \tabularnewline
gpt4 & \texttt{gpt4} & 23 Mar 2023 & 8,192 tokens & Up to Sep 2021 \tabularnewline
\bottomrule
\end{tabular}
}
\end{table}

\subsection{Quiz Construction}
\label{sec:quiz-construction}

\noindent
\textbf{Qualified APIs Collection} \tab 
% What is a question for fuzz driver generation
The question in quiz is designed as generating fuzz drivers for one given API.
This is based on the intuition that any fuzz driver can be divided into one or more simpler fuzz drivers targeting different APIs.
In other words, complicate, multi-purpose fuzz drivers are essentially the combination code which fuzzes multiple API targets simultaneously.
Therefore, as the first practical evaluation on LLM-based fuzz driver generation, we focused on the more fundamental scenario.

% why selecting an appropriate question is not naive
Not all APIs are suitable to be set as questions.
Naively setting all APIs as questions will lead to the creation of meaningless or confusing questions which influences the evaluation result.
Specifically, some APIs, such as \texttt{void libxxx\_init(void)}, are meaningless fuzz targets since the code executed behind the APIs can not be affected by any input data.
Some APIs can only be supplemental APIs rather than the main fuzzing target due to the nature of their functionalities.
For example, given two APIs \texttt{object *parse\_from\_str(char *input)} and \texttt{void free\_object(object *obj)}, feeding the mutated data into \texttt{input} is apparently a better choice than feeding a meaningless pointer to the \texttt{obj} argument.
However, calling the latter API when fuzzing the former is meaningful since 
\ding{182} it may uncover the hidden bug when the former does not correctly initialize the \texttt{object *} by freeing the members of \texttt{object};
\ding{183} it releases the resources allocated in this iteration of fuzzing, which prevents the falsely reported memory leak issue.

% how did we select qualified APIs
To build a high quality quiz, we need to collect a set of qualified APIs which are both representative and suitable to be set as fuzzing targets.
Our intuition is to collect core APIs of the existing fuzz drivers from popular projects fuzzed by OSS-Fuzz.
A driver's core APIs are located based on the following criteria:
\ding{182} they are the target APIs explicitly pointed out by the author in its driver file name or the code comments, \textit{e.g.}, \texttt{dns\_name\_fromwire} API is the core API of driver \textit{dns\_name\_fromwire.c};
\ding{183} otherwise, we pick the basic APIs as the core rather than the supplemental ones.
For example, we pick the former between parse and use/free APIs.
Some fuzz drivers are composite drivers which fuzzes multiple APIs simultaneously, we identified multiple core APIs from them.
Specifically, we randomly selected 30 projects from OSS-Fuzz (commit hash \texttt{135b000926}) C projects, manually extracted 86 core APIs from 51 fuzz drivers.
More details are listed in \appe~\ref{sec:quiz-questions-detail}.

\noindent
\textbf{Effectiveness Criteria Establishment} \tab 
% What is an effective fuzz driver
% general validation method is hard
% can greatly influence the evaluation result
% our approach
% -> semi-automatic criteria: automatic + manually built checkers
% An effective fuzz driver represents the drivers which have correct API usage and produce no false positives.
% Precisely validating the effectiveness of fuzz drivers is crucial for evaluating fuzz driver generation methods.
% However, it is hard to propose a general validation technique since the effectiveness
% general validation techniques do not work well due to the diverse semantics on the API usages.
An effective fuzz driver should use the API effectively while producing no false positives.
Determining the effectiveness of a fuzz driver is important but challengeable since it requires the correct classification of its false positives (bugs caused by the driver code) and negatives (ineffective usage).
To validate precisely, we adopted a semi-automatic validation method.
As shown in Figure~\ref{fig:validation-checker}, it has four checkers.
The first two checks require manual configuration for each project while the rest two need to be configured per API, \textit{a.k.a.} per question in our quiz.
The first check examines the grammatical correctness of a driver using compiler, while the second one checks the existence of abnormal fuzzing behaviours via short-term fuzzing.
It checks whether the driver reports any bugs, \textit{i.e.}, crashes or timeout, or does not have any coverage progress in a short time period with a default fuzzing setup (empty seed, no dictionary, etc).
The intuition behind this is that, under a default setup, neither the zero coverage progress nor the quick identification of bugs are common cases.
In our study, we set the time period as one minute.
Obviously, the second check can make both false positive and negative decisions.
The rest two checks are introduced to refine the check results.
By fuzzing the OSS-Fuzz provided drivers, we collected the signatures of real bugs that can be found quickly.
We filtered these bugs in the third check.
Lastly, based on the examination of the API usage, we summarized semantic constraints that a correct driver should obey on a given API.
These constraints are written as tests to the fuzz drivers.
For example, assuming an API require the mutated input be stored in a file, one semantic test will be hook the API and check whether the filename argument represents a valid file and contains the mutated input.
Our validation framework can taxonomize the failures (Section~\ref{subsec:evaluation-framework}), which can help us iteratively improve the inexact checks when validating more fuzz drivers.

\subsection{Evaluation Framework}
\label{subsec:evaluation-framework}

% what the evaluation framework is and how to use
To boost the evaluation of a large amount of fuzz drivers, we developed a framework to maximize the automation.
As shown in Figure~\ref{fig:overview-for-study}, the framework takes a prompt generated by a query strategy as input, and outputs the classified validation result.
It also includes a website to ease the manual analysis.
In total, the framework is written in 9,342/1,542 lines of Python/HTML, and 3,857 lines of JSON, YAML, and Bash scripts.
Instead of illustrating every detail, we discussed significant design choices as follows.

\noindent
\textbf{Prompt Standardization} \tab 
% controlling the output
% extract code part only
% preparing headers
% What are already assumed to be correctly provided, and how we postprocessing the answers
Unless specifically mentioned in the strategy, a sentence is inserted at the beginning of the prompt to help standardize the output format of LLMs.
The sentence is listed as follows (Python literal), which instructs LLMs to reply in code.
\begin{tcolorbox}[size=title, opacityfill=0.1]
\small  \texttt{// The following is a fuzz driver written in C language, complete the implementation. Output the continued code in reply only.{\textbackslash}n{\textbackslash}n}
\end{tcolorbox}

\noindent
\textbf{LLM Query} \tab 
% token limitation, how many left for query and how many for answer
% perhaps discuss this in iteration section is better
% one query one conversation
% restful deisgn?
LLMs limit the maximum token numbers for the sum of tokens in query and answer.
We set 6,000 tokens for the prompt of \texttt{gpt4}, and 3,600 tokens for \texttt{gpt3.5}.
Prompts exceeding the limit need to be adjusted in strategy.

\noindent
\textbf{Effectiveness Validation} \tab 
Due to the non-deterministic nature of the LLM, some replies still can contain both code and text.
We summarized the patterns and extracted the code in replies.
Besides, to focus on evaluating the effectiveness of the generated code, our framework automatically configures the required header files and build options for a driver.
This is done by manually configuring the rules of header inclusion, and build options for the selected 30 projects, which guarantees all compilation and link errors are caused by the incorrect code rather than the unsuitable configurations.
Besides, each validation of a fuzz driver is in a fresh, isolated container, excluding the environment disturbances to the results.

\noindent
\textbf{Failure Taxonomy} \tab 
% automatic taxonomy
% semi-automatically taxonomy
The taxonomy of root causes for ineffective results also follows a semi-automatic approach.
We first did a rough categorization of the compilation, link, and fuzz errors based on the string patterns of the errors outputted by \texttt{clang}~\cite{clang-lex-diagnostics, clang-parse-diagnostics, clang-sema-diagnostics} and \texttt{libfuzzer}~\cite{libfuzzer, sanitizers}.
Then for each API, we manually identified the root causes per category and mapped each category into the final categories. 
The manual mappings were written in code and will be iteratively improved once we found a new unclassified failure. 

\section{Overall Effectiveness (\textbf{RQ1})}

% % Figure environment removed

 % \multirow{2}{*}{Index} & \multirow{2}{*}{Question} & \multirow{2}{*}{Score} &   \multicolumn{6}{c}{GPT4} &  \multicolumn{6}{c}{GPT3.5} \\

\begin{table}[!t]
\centering
\caption{Overall Evaluation Result.
\footnotesize{K represents 40, "-" means failed to retrieve full query results or not applicable for the given model.}
}
\vspace{-10pt}
\label{tab:overall_eva_rslt}
\resizebox{1.0\linewidth}{!}{
\setlength\arrayrulewidth{0.1pt}
\begin{tabular}{cl!{\color{white}\vrule}c!{\color{white}\vrule}c!{\color{white}\vrule}c!{\color{white}\vrule}c!{\color{white}\vrule}c}
& & \multicolumn{5}{c}{Temperature} \tabularnewline
\multicolumn{2}{c}{Strategy, Model} & \cellcolor{black!20}\texttt{0.0} & \cellcolor{black!20}\texttt{0.5} & \cellcolor{black!20}\texttt{1.0} & \cellcolor{black!20}\texttt{1.5} & \cellcolor{black!20}\texttt{2.0} \tabularnewline
\arrayrulecolor{white}\hline\hline
\cellcolor{black!0}& \cellcolor{black!20} {\normalsize gpt-4-0613}& \cellcolor{green!10}{\large 9}/{\footnotesize 86}& \cellcolor{green!10}{\large 9}/{\footnotesize 86}& \cellcolor{green!10}{\large 9}/{\footnotesize 86}& \cellcolor{green!0}{\large 0}/{\footnotesize 86}& {\large -}{\tiny -} \tabularnewline\arrayrulecolor{white}\hline
\cellcolor{black!0}& \cellcolor{black!20} {\normalsize gpt-3.5-turbo-0613}& \cellcolor{green!0}{\large 0}/{\footnotesize 86}& \cellcolor{green!10}{\large 1}/{\footnotesize 86}& \cellcolor{green!0}{\large 0}/{\footnotesize 86}& \cellcolor{green!0}{\large 0}/{\footnotesize 86}& \cellcolor{green!0}{\large 0}/{\footnotesize 86} \tabularnewline\arrayrulecolor{white}\hline
\cellcolor{black!0}& \cellcolor{black!20} {\normalsize wizardcoder-15b-v1.0}& \cellcolor{green!10}{\large 3}/{\footnotesize 86}& \cellcolor{green!10}{\large 1}/{\footnotesize 86}& \cellcolor{green!10}{\large 1}/{\footnotesize 86}& \cellcolor{green!0}{\large 0}/{\footnotesize 86}& \cellcolor{green!0}{\large 0}/{\footnotesize 86} \tabularnewline\arrayrulecolor{white}\hline
\cellcolor{black!0}& \cellcolor{black!20} {\normalsize text-bison-001}& \cellcolor{green!10}{\large 2}/{\footnotesize 86}& \cellcolor{green!10}{\large 2}/{\footnotesize 86}& \cellcolor{green!10}{\large 1}/{\footnotesize 86}& {\large -}{\tiny -}& {\large -}{\tiny -} \tabularnewline\arrayrulecolor{white}\hline
\multirow{-5}{*}{\cellcolor{black!0} \rotatebox[origin=c]{90}{{NAIVE-1}}}& \cellcolor{black!20} {\normalsize codellama-34b-instruct}& \cellcolor{green!0}{\large 0}/{\footnotesize 86}& \cellcolor{green!0}{\large 0}/{\footnotesize 86}& \cellcolor{green!0}{\large 0}/{\footnotesize 86}& \cellcolor{green!0}{\large 0}/{\footnotesize 86}& \cellcolor{green!0}{\large 0}/{\footnotesize 86} \tabularnewline\arrayrulecolor{white}\hline
\arrayrulecolor{white}\hline\hline
\cellcolor{black!0}& \cellcolor{black!20} {\normalsize gpt-4-0613}& \cellcolor{green!20}{\large 12}/{\footnotesize 86}& \cellcolor{green!40}{\large 30}/{\footnotesize 86}& \cellcolor{green!40}{\large 30}/{\footnotesize 86}& \cellcolor{green!10}{\large 5}/{\footnotesize 86}& {\large -}{\tiny -} \tabularnewline\arrayrulecolor{white}\hline
\cellcolor{black!0}& \cellcolor{black!20} {\normalsize gpt-3.5-turbo-0613}& \cellcolor{green!0}{\large 0}/{\footnotesize 86}& \cellcolor{green!10}{\large 6}/{\footnotesize 86}& \cellcolor{green!10}{\large 8}/{\footnotesize 86}& \cellcolor{green!10}{\large 8}/{\footnotesize 86}& \cellcolor{green!0}{\large 0}/{\footnotesize 86} \tabularnewline\arrayrulecolor{white}\hline
\cellcolor{black!0}& \cellcolor{black!20} {\normalsize wizardcoder-15b-v1.0}& \cellcolor{green!10}{\large 3}/{\footnotesize 86}& \cellcolor{green!10}{\large 8}/{\footnotesize 86}& \cellcolor{green!20}{\large 11}/{\footnotesize 86}& \cellcolor{green!10}{\large 1}/{\footnotesize 86}& \cellcolor{green!0}{\large 0}/{\footnotesize 86} \tabularnewline\arrayrulecolor{white}\hline
\cellcolor{black!0}& \cellcolor{black!20} {\normalsize text-bison-001}& \cellcolor{green!10}{\large 2}/{\footnotesize 86}& \cellcolor{green!10}{\large 5}/{\footnotesize 86}& \cellcolor{green!10}{\large 5}/{\footnotesize 86}& {\large -}{\tiny -}& {\large -}{\tiny -} \tabularnewline\arrayrulecolor{white}\hline
\multirow{-5}{*}{\cellcolor{black!0} \rotatebox[origin=c]{90}{{NAIVE-K}}}& \cellcolor{black!20} {\normalsize codellama-34b-instruct}& \cellcolor{green!0}{\large 0}/{\footnotesize 86}& \cellcolor{green!10}{\large 1}/{\footnotesize 86}& \cellcolor{green!10}{\large 3}/{\footnotesize 86}& \cellcolor{green!0}{\large 0}/{\footnotesize 86}& \cellcolor{green!0}{\large 0}/{\footnotesize 86} \tabularnewline\arrayrulecolor{white}\hline
\arrayrulecolor{white}\hline\hline
\cellcolor{black!0}& \cellcolor{black!20} {\normalsize gpt-4-0613}& \cellcolor{green!40}{\large 29}/{\footnotesize 86}& \cellcolor{green!50}{\large 41}/{\footnotesize 86}& \cellcolor{green!50}{\large 41}/{\footnotesize 86}& \cellcolor{green!30}{\large 21}/{\footnotesize 86}& {\large -}{\tiny -} \tabularnewline\arrayrulecolor{white}\hline
\cellcolor{black!0}& \cellcolor{black!20} {\normalsize gpt-3.5-turbo-0613}& \cellcolor{green!20}{\large 12}/{\footnotesize 86}& \cellcolor{green!40}{\large 29}/{\footnotesize 86}& \cellcolor{green!40}{\large 30}/{\footnotesize 86}& \cellcolor{green!30}{\large 24}/{\footnotesize 86}& \cellcolor{green!10}{\large 1}/{\footnotesize 86} \tabularnewline\arrayrulecolor{white}\hline
\cellcolor{black!0}& \cellcolor{black!20} {\normalsize wizardcoder-15b-v1.0}& \cellcolor{green!10}{\large 7}/{\footnotesize 86}& \cellcolor{green!30}{\large 23}/{\footnotesize 86}& \cellcolor{green!30}{\large 25}/{\footnotesize 86}& \cellcolor{green!20}{\large 17}/{\footnotesize 86}& \cellcolor{green!0}{\large 0}/{\footnotesize 86} \tabularnewline\arrayrulecolor{white}\hline
\cellcolor{black!0}& \cellcolor{black!20} {\normalsize text-bison-001}& \cellcolor{green!10}{\large 7}/{\footnotesize 86}& \cellcolor{green!20}{\large 13}/{\footnotesize 86}& \cellcolor{green!20}{\large 15}/{\footnotesize 86}& {\large -}{\tiny -}& {\large -}{\tiny -} \tabularnewline\arrayrulecolor{white}\hline
\multirow{-5}{*}{\cellcolor{black!0} \rotatebox[origin=c]{90}{{BACTX-K}}}& \cellcolor{black!20} {\normalsize codellama-34b-instruct}& \cellcolor{green!0}{\large 0}/{\footnotesize 86}& \cellcolor{green!10}{\large 1}/{\footnotesize 86}& \cellcolor{green!20}{\large 11}/{\footnotesize 86}& \cellcolor{green!0}{\large 0}/{\footnotesize 86}& \cellcolor{green!0}{\large 0}/{\footnotesize 86} \tabularnewline\arrayrulecolor{white}\hline
\arrayrulecolor{white}\hline\hline
\cellcolor{black!0}& \cellcolor{black!20} {\normalsize gpt-4-0613}& \cellcolor{green!40}{\large 29}/{\footnotesize 86}& \cellcolor{green!50}{\large 40}/{\footnotesize 86}& \cellcolor{green!50}{\large 41}/{\footnotesize 86}& \cellcolor{green!30}{\large 22}/{\footnotesize 86}& {\large -}{\tiny -} \tabularnewline\arrayrulecolor{white}\hline
\cellcolor{black!0}& \cellcolor{black!20} {\normalsize gpt-3.5-turbo-0613}& \cellcolor{green!20}{\large 11}/{\footnotesize 86}& \cellcolor{green!30}{\large 22}/{\footnotesize 86}& \cellcolor{green!40}{\large 29}/{\footnotesize 86}& \cellcolor{green!30}{\large 24}/{\footnotesize 86}& \cellcolor{green!10}{\large 1}/{\footnotesize 86} \tabularnewline\arrayrulecolor{white}\hline
\cellcolor{black!0}& \cellcolor{black!20} {\normalsize wizardcoder-15b-v1.0}& \cellcolor{green!10}{\large 7}/{\footnotesize 86}& \cellcolor{green!30}{\large 24}/{\footnotesize 86}& \cellcolor{green!30}{\large 25}/{\footnotesize 86}& \cellcolor{green!20}{\large 12}/{\footnotesize 86}& \cellcolor{green!0}{\large 0}/{\footnotesize 86} \tabularnewline\arrayrulecolor{white}\hline
\cellcolor{black!0}& \cellcolor{black!20} {\normalsize text-bison-001}& \cellcolor{green!10}{\large 9}/{\footnotesize 86}& \cellcolor{green!20}{\large 14}/{\footnotesize 86}& \cellcolor{green!20}{\large 14}/{\footnotesize 86}& {\large -}{\tiny -}& {\large -}{\tiny -} \tabularnewline\arrayrulecolor{white}\hline
\multirow{-5}{*}{\cellcolor{black!0} \rotatebox[origin=c]{90}{{DOCTX-K}}}& \cellcolor{black!20} {\normalsize codellama-34b-instruct}& \cellcolor{green!0}{\large 0}/{\footnotesize 86}& \cellcolor{green!10}{\large 9}/{\footnotesize 86}& \cellcolor{green!20}{\large 13}/{\footnotesize 86}& \cellcolor{green!10}{\large 1}/{\footnotesize 86}& \cellcolor{green!0}{\large 0}/{\footnotesize 86} \tabularnewline\arrayrulecolor{white}\hline
\arrayrulecolor{white}\hline\hline
\cellcolor{black!0}& \cellcolor{black!20} {\normalsize gpt-4-0613}& \cellcolor{green!70}{\large 55}/{\footnotesize 86}& \cellcolor{green!80}{\large 63}/{\footnotesize 86}& \cellcolor{green!80}{\large 62}/{\footnotesize 86}& \cellcolor{green!30}{\large 26}/{\footnotesize 86}& {\large -}{\tiny -} \tabularnewline\arrayrulecolor{white}\hline
\cellcolor{black!0}& \cellcolor{black!20} {\normalsize gpt-3.5-turbo-0613}& \cellcolor{green!40}{\large 30}/{\footnotesize 86}& \cellcolor{green!60}{\large 47}/{\footnotesize 86}& \cellcolor{green!50}{\large 43}/{\footnotesize 86}& \cellcolor{green!40}{\large 31}/{\footnotesize 86}& \cellcolor{green!0}{\large 0}/{\footnotesize 86} \tabularnewline\arrayrulecolor{white}\hline
\cellcolor{black!0}& \cellcolor{black!20} {\normalsize wizardcoder-15b-v1.0}& \cellcolor{green!50}{\large 39}/{\footnotesize 86}& \cellcolor{green!60}{\large 50}/{\footnotesize 86}& \cellcolor{green!60}{\large 48}/{\footnotesize 86}& \cellcolor{green!20}{\large 13}/{\footnotesize 86}& \cellcolor{green!0}{\large 0}/{\footnotesize 86} \tabularnewline\arrayrulecolor{white}\hline
\cellcolor{black!0}& \cellcolor{black!20} {\normalsize text-bison-001}& \cellcolor{green!30}{\large 21}/{\footnotesize 86}& \cellcolor{green!40}{\large 27}/{\footnotesize 86}& \cellcolor{green!50}{\large 38}/{\footnotesize 86}& {\large -}{\tiny -}& {\large -}{\tiny -} \tabularnewline\arrayrulecolor{white}\hline
\multirow{-5}{*}{\cellcolor{black!0} \rotatebox[origin=c]{90}{{UGCTX-K}}}& \cellcolor{black!20} {\normalsize codellama-34b-instruct}& \cellcolor{green!0}{\large 0}/{\footnotesize 86}& \cellcolor{green!10}{\large 8}/{\footnotesize 86}& \cellcolor{green!30}{\large 21}/{\footnotesize 86}& \cellcolor{green!0}{\large 0}/{\footnotesize 86}& \cellcolor{green!0}{\large 0}/{\footnotesize 86} \tabularnewline\arrayrulecolor{white}\hline
\arrayrulecolor{white}\hline\hline
\cellcolor{black!0}& \cellcolor{black!20} {\normalsize gpt-4-0613}& \cellcolor{green!70}{\large 56}/{\footnotesize 86}& \cellcolor{green!70}{\large 57}/{\footnotesize 86}& \cellcolor{green!80}{\large 62}/{\footnotesize 86}& \cellcolor{green!30}{\large 23}/{\footnotesize 86}& {\large -}{\tiny -} \tabularnewline\arrayrulecolor{white}\hline
\cellcolor{black!0}& \cellcolor{black!20} {\normalsize gpt-3.5-turbo-0613}& \cellcolor{green!40}{\large 32}/{\footnotesize 86}& \cellcolor{green!60}{\large 47}/{\footnotesize 86}& \cellcolor{green!50}{\large 43}/{\footnotesize 86}& \cellcolor{green!40}{\large 28}/{\footnotesize 86}& \cellcolor{green!10}{\large 2}/{\footnotesize 86} \tabularnewline\arrayrulecolor{white}\hline
\cellcolor{black!0}& \cellcolor{black!20} {\normalsize wizardcoder-15b-v1.0}& \cellcolor{green!10}{\large 8}/{\footnotesize 86}& \cellcolor{green!30}{\large 24}/{\footnotesize 86}& \cellcolor{green!50}{\large 37}/{\footnotesize 86}& \cellcolor{green!20}{\large 13}/{\footnotesize 86}& \cellcolor{green!0}{\large 0}/{\footnotesize 86} \tabularnewline\arrayrulecolor{white}\hline
\cellcolor{black!0}& \cellcolor{black!20} {\normalsize text-bison-001}& \cellcolor{green!10}{\large 9}/{\footnotesize 86}& \cellcolor{green!20}{\large 15}/{\footnotesize 86}& \cellcolor{green!30}{\large 20}/{\footnotesize 86}& {\large -}{\tiny -}& {\large -}{\tiny -} \tabularnewline\arrayrulecolor{white}\hline
\multirow{-5}{*}{\cellcolor{black!0} \rotatebox[origin=c]{90}{{BA-ITER-K}}}& \cellcolor{black!20} {\normalsize codellama-34b-instruct}& \cellcolor{green!10}{\large 6}/{\footnotesize 86}& \cellcolor{green!40}{\large 28}/{\footnotesize 86}& \cellcolor{green!30}{\large 22}/{\footnotesize 86}& \cellcolor{green!0}{\large 0}/{\footnotesize 86}& \cellcolor{green!0}{\large 0}/{\footnotesize 86} \tabularnewline\arrayrulecolor{white}\hline
\arrayrulecolor{white}\hline\hline
\cellcolor{black!0}& \cellcolor{black!20} {\normalsize gpt-4-0613}& \cellcolor{green!90}{\large 77}/{\footnotesize 86}& \cellcolor{green!90}{\large 78}/{\footnotesize 86}& \cellcolor{green!90}{\large 76}/{\footnotesize 86}& \cellcolor{green!30}{\large 25}/{\footnotesize 86}& {\large -}{\tiny -} \tabularnewline\arrayrulecolor{white}\hline
\cellcolor{black!0}& \cellcolor{black!20} {\normalsize gpt-3.5-turbo-0613}& \cellcolor{green!80}{\large 65}/{\footnotesize 86}& \cellcolor{green!80}{\large 68}/{\footnotesize 86}& \cellcolor{green!80}{\large 65}/{\footnotesize 86}& \cellcolor{green!50}{\large 37}/{\footnotesize 86}& \cellcolor{green!0}{\large 0}/{\footnotesize 86} \tabularnewline\arrayrulecolor{white}\hline
\cellcolor{black!0}& \cellcolor{black!20} {\normalsize wizardcoder-15b-v1.0}& \cellcolor{green!50}{\large 41}/{\footnotesize 86}& \cellcolor{green!60}{\large 48}/{\footnotesize 86}& \cellcolor{green!70}{\large 53}/{\footnotesize 86}& \cellcolor{green!20}{\large 11}/{\footnotesize 86}& \cellcolor{green!0}{\large 0}/{\footnotesize 86} \tabularnewline\arrayrulecolor{white}\hline
\cellcolor{black!0}& \cellcolor{black!20} {\normalsize text-bison-001}& \cellcolor{green!30}{\large 21}/{\footnotesize 86}& \cellcolor{green!50}{\large 37}/{\footnotesize 86}& \cellcolor{green!50}{\large 42}/{\footnotesize 86}& {\large -}{\tiny -}& {\large -}{\tiny -} \tabularnewline\arrayrulecolor{white}\hline
\multirow{-5}{*}{\cellcolor{black!0} \rotatebox[origin=c]{90}{{ALL-ITER-K}}}& \cellcolor{black!20} {\normalsize codellama-34b-instruct}& \cellcolor{green!20}{\large 13}/{\footnotesize 86}& \cellcolor{green!20}{\large 18}/{\footnotesize 86}& \cellcolor{green!30}{\large 26}/{\footnotesize 86}& \cellcolor{green!10}{\large 1}/{\footnotesize 86}& \cellcolor{green!0}{\large 0}/{\footnotesize 86} \tabularnewline\arrayrulecolor{white}\hline
\arrayrulecolor{white}\hline\hline

\end{tabular}
}
\vspace{-10pt}
\end{table}

Table~\ref{tab:overall_eva_rslt} presents the results of all evaluated configurations.
The principal data displayed in the table are the question solve rates, formatted as X/Y, where X denotes the number of questions a language model successfully solves, and Y represents the total number of questions presented.
A configuration is considered to have solved a question if at least one effective fuzz driver has been generated.
For gpt-4-0613, the results of temperature 2.0 were incomplete due to the services' slow response time in extreme temperature settings\footnote{OpenAI and Azure gpt-4-0613 API frequently raise timeout for requests with high temperature setting. Temperature 2.0 results will be updated at~\cite{fuzz-drvier-study-website} once completed.}.
For the text-bison-001, Google's query API limits the requests with a temperature setting above 1.0.
Nevertheless, given the poor or even zero performance of all other models with a temperatures setting 2.0, the data absence does not substantially affect our evaluation.
\revision{
This table only lists the number of solved questions while \cite{fuzz-drvier-study-website} posts full evaluation details of each model such as success rate per question.
}
\compactline

% \noindent
% \textbf{Performance Overview} \tab 
% promising best performance, many settings have high question solve rate
% high variance range
% observation -> changing any factor can significantly affect the result
% observation -> the key for getting a decent result is avoid the bad choice of all three bad factor
\textbf{Overall, the results offer promising evidence of the practicality of utilizing language model-based fuzz driver generation.}
The optimal configurations, namely <gpt-4-0613, ALL-ITER-K, 0.5>, achieved impressive success rates, effectively generating fuzz drivers that solved about 91\% (78/86) questions.
Moreover, three out of five LLMs assessed -- including an open-source option -- and half of the strategies explored can resolve over half of the questions.

\textbf{The substantial variation in success rates across different configurations underscores the significant influence of the three factors.}
By analyzing the data, we observe that results can greatly fluctuate when varying a single factor -- such as changing the temperature in a row, or switching models or prompting strategies in a column.
For example, <gpt-3.5-turbo-0613, NAIVE-1, 0.0> failed to solve any questions, whereas <gpt-3.5-turbo-0613, ALL-ITER-K, 0.0> managed to correctly address 76\% (65/86) of them. 
This indicates that achieving a high solve rate relies heavily on avoiding suboptimal combinations of factors.
Given that the table is sorted to reflect performance trends, the better outcomes tend to cluster in 'green areas', highlighting configurations where all contributing factors are well-adjusted.

\subsection{Analysis of Effectiveness Factors}

\noindent
\textbf{Prompt Strategies} \tab 
% varies much (NAIVE-1 -> ALL-ITER-K)
% iterative & usage example 
The observed impacts of different prompting strategies exceeded our initial expectations during their design phase.
A comparison between NAIVE-1 and ALL-ITER-K showcases a dramatic improvement in optimal question solve rates, soaring from 10\% to 90\%, emphasizing the critical role of prompt design on tool effectiveness. 
To better understand the performance trends, Table~\ref{tab:overall_eva_rslt} presents the prompting strategies ranked by their overall effectiveness.
The trends in the results are intuitive: \textbf{in general, strategies that more comprehensively leverage available information tend to yield superior results}.
For example, the strategy UGCTX-K markedly outperforms BACTX-K.
This can be attributed to UGCTX-K's inclusion of example code snippets that illustrate certain usage of the target API.
A notable performance discrepancy is also seen when comparing BA-ITER-K with BACTX-K.
Despite starting with the same initial information, BA-ITER-K significantly surpasses BACTX-K.
The reason for this performance difference lies in BA-ITER-K's iterative method -- collecting debugging information to guide the model to fix the previous fuzz driver if it is ineffective.
Among all the strategies, ALL-ITER-K stands out as the most effective across different combinations of temperature settings and models.
This makes sense considering that ALL-ITER-K not only incorporates all extended API information but also adopts a recursive problem-solving methodology.
Conclusively, its design leads to the superior performance in our evaluation.
The detailed analysis of these strategies are discussed in Section~\ref{sec:rq3}.

\noindent
\textbf{Temperatures} \tab 
% 0.0 - 1.0
% the nature of the task or evaluation does not require high randomness of the reply, as long as it can provide one effective answer 
Table~\ref{tab:overall_eva_rslt} clearly demonstrates that \textbf{configurations with a temperature setting of 0.5 tend to achieve the highest success rates}.
In contrast, models under a temperature setting above 1.0 experience a noticeable drop in performance.
Interestingly, it appears that \textbf{in general, lower temperatures, especially below the threshold of 1.0, show substantial performance advantage compared to models operating at higher temperatures}.
A surprising outcome is that models with 0.0 temperature perform remarkably well.
For instance, both <gpt-4-0613, ALL-ITER-K, 0.0> and <gpt-3.5-turbo-0613, BA-ITER-K, 0.0> stand out as second-best configuration when compared across the various temperature settings.
These results are reasonable considering the nature of fuzz driver generation task.
With a lower temperature setting, models tend to generate more consistent and predictable outputs, which benefits the synthesis of high-quality code.
High temperatures, while fostering creativity and randomness, may not provide any notable advantages in this context.
Specifically, these features are either substituted by the randomness contained in prompt strategies or deemed irrelevant by the assessment criteria.
For example, a prompting strategy like ALL-ITER-K inherently contains a built-in search process that brings the randomness from model input.
And the evaluation strictly assesses the quantity of effective drivers without considering the API usage diversity.
This criteria fits our evaluation goal, but discounts the creative diversity that could be introduced by higher temperatures.
% Instead, it strictly assesses the quantity of effective ones.

\noindent
\textbf{Open-Source LLMs vs Closed-Source LLMs} \tab 
% controversial discussion, concentrated area that how open-source LLMs can achieve comparing with openai models
% open-source LLMs can achieve adequate performance
As commonly understood in the industry, closed-source LLMs tend to outperform their open-source counterparts.
Among these, gpt-4-0613 is considered the front-runner in terms of generation capabilities.
Following closely behind is gpt-3.5-turbo-0613, which offers a cost-effective alternative due to its significantly lower token pricing.
However, it's worth noting that in the open-source domain, wizardcoder-15b-v1.0 has made remarkable strides, even surpassing Google's closed-source model, text-bison-001.
While wizardcoder-15b-v1.0 is nearly on par with gpt-3.5-turbo-0613, certain performance gaps can still be observed, but it stands as a commendable achievement for an open-source model.

% \noindent
% \begin{tcolorbox}[size=title, opacityfill=0.1, nobeforeafter, breakable]
% \begin{tcolorbox}[size=title, opacityfill=0.1, breakable]
% Each factor affects the performance significantly.
% \end{tcolorbox}


\subsection{How Far Are We to Total Practicality?}
% breakdown of the overall effectiveness based on questions
The above evaluation indicates that with the optimal configuration <gpt-4-0613, ALL-ITER-K, 0.5>, the LLM can solve 91\% predefined questions.
In other words, it can produce at least one effective fuzz driver for 78 out of 86 APIs examined.
However, this does not necessarily mean that LLMs are ready to be used in production.
Upon further examination of our APIs, we identified three primary challenges and detailed them as follows.

% Figure environment removed

% \begin{table}[!t]
% \centering
% \caption{
% Percentage of High Cost Questions in Top-10 Evaluation Configuration.
% \footnotesize{
% Question Cost = 1 / (success rate in evaluation),
% Percentage = (\# of solved questions with high cost) / (\# of all solved questions).
% \zhc{change to pie chart?}
% }
% }
% \label{tab:topx-high-cost-Q}
% % \resizebox{1.0\linewidth}{!}{
% \begin{tabular}{llllll}
% \toprule
% Question Cost & $\geq$ 2 & $\geq$ 5 & $\geq$ 10 & $\geq$ 20 & $\geq$ 40 \tabularnewline
% \midrule 
% \rowcolor{black!10}
% Avg Percentage & 90\% & 64\% & 45\% & 31\% & 18\% \tabularnewline
% \bottomrule
% \end{tabular}
% % }
% \end{table}

\noindent
\textbf{C1: High Token Cost in Fuzz Driver Generation} \tab 
% xx tokens for one correct driver
Though many configurations have shown a high rate of successful problem resolution, our analysis indicates the results come with substantial costs.
The data in Figure~\ref{fig:topx-high-cost-pie} details the percentage of high cost questions for all evaluated configurations.
Remarkably, it reveals that on average, resolving 45\% questions entail costs exceeding 10.
This suggests that for 50\% of the resolved questions, a prompt-based strategy may yield just one effective fuzz driver after repeating the entire query process 10 times or more.
When considering only the questions with costs surpassing 20 or even 40, the percentages remain notable at 31\% and 18\%, respectively.
% Another supportive fact for the high cost is that some questions\zhc{edge_connect, dns_message_checksig} have one of ten thousands of success rate in the whole evaluation.
These findings underscore a strong incentive for further research into cost reduction techniques.
Reducing costs is not only a practical concern with direct financial consequences but also essential for improving the efficiency of LLM-based fuzz driver generation.
% \zhc{make more concise}

\noindent
\textbf{C2: Ensuring Semantic Correctness of API Usage} \tab 
% xx% questions cannot be fully automatically solved
In our evaluation, we found that there is a discrepancy for approximately 34\% (29/86) of the APIs -- assuming LLMs can successfully create at least one effective fuzz driver for each in the evaluation setting, this success cannot be translated into the full automation of fuzz driver generation for them.
The issue at hand lies in the potential misuse of APIs within the generated drivers, which requires validation to ensure semantic correctness.
For example, LLMs may incorrectly initializing the argument of an API, such as passing a mutated filename to the API instead of passing a created file first and then mutating its content for fuzzing, or missing some condition checks before calling API.
In our evaluation process, we manually implemented semantic checkers to identify such API misuses for accurate assessment (details on our semantic checkers are provided in Section~\ref{sec:eval-framework}). 
However, fully automating the validation of semantic correctness remains a significant hurdle.
Consequently, even though it is feasible to generate effective fuzz drivers with the help of these LLMs, distinguishing them from the ineffective ones can be problematic due to the absence of automated methods for validating semantic correctness.
This challenge underscores the need for developing robust techniques to automatically ensure the semantic accuracy of generated fuzz drivers before they can be reliably deployed in production.

% For around 34\% APIs (29 out of 86) estimated in evaluation, there exists a gap between successfully generating one effective fuzz driver for them in evaluation with capable of fully automatically generating fuzz drivers for them in production.
% The root of the gap here is that the fuzz drivers of these APIs may contain API usage misuse which require the validation of semantic correctness.
% For example, xxx.
% In evaluation, we can manually prepare semantic checkers to help determine  these API misuses for correct evaluation (see detail of our semantic checkers in Section~\ref{sec:xxx}).
% While in production phase, it is challenging to automatically validate them.
% This causes a fact that, even it is possible to generate effective fuzz drivers using these LLMs, we cannot correctly pick it out from ineffective ones due to the lack of automatic semantic correctness validation methods.

% \noindent
\textbf{C3: Satisfying Complex API Usage Dependencies} \tab 
% contextual deps
% complex data/control deps
Overall, there are five questions cannot be resolved by any assessed configurations.
These questions are challengeable since their driver generation requires the deep understanding of specific contexts.
For instance, generating the driver for \texttt{tmux}~\cite{tmux-ossfuzz-driver-link} requires the construction of various concepts, such as session, window, pane, etc, and their relationships.
Similarly, for network-related questions~\cite{libmodbus-ossfuzz-driver-link, civetweb-ossfuzz-driver-link}, a standby network server or client is required to be created before calling the target API.
The effective drivers can only be generated by respecting these specific contextual requirements.

% \noindent
% \textbf{Quiz-Level Performance} \tab 
% Figure~\ref{fig:results-on-correct-questions} plots the overall performance for the proposed strategies.
% Specifically, 
% Figure~\ref{fig:corr-qstn-per-round} illustrates the amount of correct questions per round while the Figure~\ref{fig:stacked-corr-qstn-per-round} details the stacked amount.
% In both plots, the line style is used to distinguish the model, \textit{i.e.}, solid/dotted for \texttt{gpt4}/\texttt{gpt3.5}, and the line colors represent the type of prompt templates, \textit{i.e.}, black/red for \texttt{BACTX-K}/\texttt{NAIVE-K}.
% % \zhc{add full detail table in the appendix and mention that here}
% Apparently, \texttt{gpt4-BACTX-K} shows its performance superiority than the rest three strategies while \texttt{gpt3.5-NAIVE-K} have the lowest performance.
% This is intuitive since \texttt{gpt4-BACTX-K} is configured with a model expected to be more powerful and generates more descriptive prompt for the question.
% In general, LLM model, prompt template, and the degree of repetitive query are three independent key factors:
% \ding{182} solid lines are almost always higher than the paired dotted lines, which means that, keeping other settings the same, \texttt{gpt4} almost always perform better than \texttt{gpt3.5};
% \ding{183} black lines are significantly higher than paired red lines, indicating that, using the same LLM, prompt template of \texttt{BACTX-K} always have a significant overall performance than \texttt{NAIVE-K}'s;
% \ding{184} the stacked results in Figure~\ref{fig:stacked-corr-qstn-per-round} shows clear performance advantage than one round performance, indicating the effectiveness of repeatedly query.
% \ding{184} while the lines in Figure~\ref{fig:corr-qstn-per-round} show a relative stability on one round performance, the stacked results in Figure~\ref{fig:stacked-corr-qstn-per-round} reveal that combining the answers of all rounds can remarkably improve the overall performance, regardless of the used model and prompt template.

% \begin{table}[t]
% \centering
% \caption{Outcome Comparison Between Different K Values.}
% \label{tab:cmp-different-k}
% \resizebox{1.0\linewidth}{!}{
% \begin{tabular}{lllll}
% % \hline
% \toprule
%  & \texttt{gpt3.5-NAIVE-K} 
%  & \texttt{gpt3.5-BACTX-K} 
%  & \texttt{gpt4-NAIVE-K} 
%  & \texttt{gpt4-BACTX-K} \\
% \midrule
% \rowcolor{black!10}
% R (K=40/K=1) & 3.25 (13/4) & 2.73 (30/11) & 9.33 (28/3) & 1.76 (44/25) \\
% P (K=6/K=40) & 76.92\% (10/13) & 86.67\% (26/30) & 64.29\% (18/28) & 88.64\% (39/44) \\
% % \hline
% \bottomrule
% \end{tabular}
% }
% \label{tab:cmp-different-k}
% \end{table}

%% The observation \ding{184} indicates the usefulness of repetitive queries.
%% properly exploiting the randomness in LLM replies can significantly improve the overall performance.
%For \ding{184}, though the ratio of total corrected questions for 40 rounds to one round can more than nine, the benefits of the repetitive queries rapidly decrease after certain rounds of repeated queries, roughly following the Pareto Principle~\cite{pareto-principle}.
%Specifically, as shown in Table~\ref{tab:cmp-different-k}, roughly 80\% of the overall performance of repetitive queries are contributed in the initial 20\% rounds of queries. 
%% all settings have reached nearly or more than 80\% performance at the sixth round comparing with the total outcomes of 40 rounds.
%In general, a simple stop condition can be proposed, \textit{e.g.}, no performance increase in last X rounds, to balance the benefits and query costs.
%% the number of repeat times, \textit{a.k.a.} the value of \textbf{K}, can be adaptively determined according to the past statistics of the new correct answers.

% % Figure environment removed

% \begin{table}[t]
% \centering
% \caption{Outcome Comparison Between Different K Values.}
% \label{tab:cmp-different-k}
% \resizebox{1.0\linewidth}{!}{
% \begin{tabular}{llllllll}
% % \hline
% \toprule
% Strategy & S2 & S3 & S4 & S5 & S6 & S7 & S8\\
% \midrule
% \texttt{gpt3.5-NAIVE-K} & -         &-          &-      &-      &-      &-      &0.00\% \\
% \texttt{gpt4-NAIVE-K}   & -         &-          &10.50\%&-      &0.00\% &0.00\% &0.00\% \\
% \texttt{gpt3.5-BACTX-K} & -         &2.50\%     &-      &0.00\% &-      &0.00\% &0.00\% \\
% \texttt{gpt4-BACTX-K}   & 28.41\%   &-          &-      &0.00\% &0.00\% &0.00\% &0.00\% \\
% % \hline
% \bottomrule
% \end{tabular}
% }
% \label{tab:cmp-different-k}
% \end{table}

% \noindent
% \textbf{Question-Level Performance} \tab 
% % Besides discussing the overall performance, we analyzed these strategies on question-level statistics.
% Figure~\ref{fig:upset-plot-for-simple-strategies} shows the UpSet plot.
% % analyzing the solved questions sets for the above strategies.
% Among all 86 questions, 35 of them (40.70\%) have not been solved by any basic strategy, which shows a large space for further improvement.
% An interesting observation is that most strategies have uniquely solved questions: \texttt{gpt3.5-BACTX-K}, \texttt{gpt4-NAIVE-K}, and \texttt{gpt4-BACTX-K} have 2, 5, and 11 uniquely solved questions respectively. 
% On the one hand, it is an evidence that the current most effective strategy \texttt{gpt4-BACTX-K} still cannot outperform the rests in all respects. 
% On the other hand, it reminds us the probabilistic nature of the language models.
% For instance, though \texttt{gpt4-NAIVE-K} have same prompts/model as \texttt{gpt4-BACTX-K} and contain less descriptive information about the target API, it still can solve 5 questions that \texttt{gpt4-BACTX-K} failed to solve in 40 rounds.
% One possible explanation is that, for these questions, the less descriptive prompts generated by \texttt{gpt4-NAIVE-K} can happenly guide the \texttt{gpt4} model produce effective answers.
% A supportive finding is that all 5 questions uniquely solved by \texttt{gpt4-NAIVE-K} are of one project and their API usage follow similar design patterns.
% Besides, the average query success rate of \texttt{gpt4-BACTX-K} is higher than the rest questions, which indicates that \texttt{gpt4-BACTX-K} not only solves more questions but also solves them more reliably.
% % Specifically, the 11 questions uniquely solved by \texttt{gpt4-BACTX-K} cover 6 projects and have an average success rate 28.41\% on query, while the \texttt{gpt3.5-BACTX-K} and \texttt{gpt4-NAIVE-K}'s 2/5 questions cover 1/1 projects, with a success rate of 2.50\%/10.50\%.
% % \zhc{add full detail in appendix and add reference here}

% it reminds us the statistical nature of LLM-based methods.
% We cannot debug or reason the LLM-based methods in traditional way for one or two single questions.
% For instance, though \texttt{gpt4-NAIVE-K} have same prompts/model as \texttt{gpt4-BACTX-K} and contain less information about the target API, it still can solve 5 questions that \texttt{gpt4-BACTX-K} failed to solve.

%We also analyzed question-level performance by comparing the uniquely solved question sets.
%A main conclusion is that though most strategies have their uniquely solved questions, \texttt{gpt4-BACTX-K} outperforms others in most projects, which behaves more reliable.
%% outperforms others in most projects, which behaves more reliable.
%Due to the page limit, we put the detailed data at website~\cite{fuzz-drvier-study-website}.
%
\noindent
% \begin{tcolorbox}[size=title, opacityfill=0.1, nobeforeafter, breakable]
\begin{tcolorbox}[size=title, opacityfill=0.1, breakable]
While LLM-based generation has shown promising potential, it still faces certain challenges towards high practicality.
\end{tcolorbox}


\section{Fundamental Challenge (\textbf{RQ2})}

% % Figure environment removed

% \noindent
% \textbf{Boundaries of Effectiveness} \tab 

% \subsection{Boundaries of Effectiveness}
\subsection{Links Between Question and Performance}

% Figure environment removed

% The difference among LLM replies for the same query is the power source of the effectiveness for repetitive queries.
% Specifically, a second time answer from the LLM may add or remove certain condition checks, change the macro used in the option argument of an API, and even switch the entire API usage comparing with previous answer.
% Our main observation is that LLMs' performance degrades as they are required to consider more API specific usages when writing a driver.
% Before delving into a detailed discussion, we will first outline the minimum API usages required for writing a fuzz driver.
% Then we examine the challenges that LLMs may encounter when attempting to correctly synthesize these usages.
% Finally, evidence are presented to support our findings.
% We first hypothesized that LLMs' performance is related to complexity of API usage. 
% Then we examined the
% To test the hypothesis, we propose the evaluation steps as follows: first, we identify the minimum API usage required for fuzz driver creation; second, we examine the potential difficulties LLMs may encounter when accurately synthesizing these usages; and finally, we design a score system to quantify the degree of the minimum API specific usage, and associate it with the performance of fuzz driver generation.

To investigate the core difficulties in generating fuzz drivers with LLMs, we scrutinized the outcomes of the BACTX-K strategy.
\revision{
This strategy is a proper starting point for understanding the fundamental challenges since it merely uses generally accessible information and has simple query workflow.
}
% This strategy has 
% The analysis results of this strategy are suitable to represent the fundamental challenge since it only utilizes basic information and conducts basic query workflow.
% Comparing with other strategies which may include target specific examples, \texttt{BACTX-K} only utilizes basic information and conducts basic query workflow.
% Besides, nearly all strategies are built upon \texttt{BACTX-K} except \texttt{NAIVE-K}.
% Therefore, understanding it can help better understand the characteristics of other advanced strategies.
In Figure~\ref{fig:rq2-query-succ-rates-per-score}, \textbf{there is a clear inverse proportion relationship between the query success rate and the complexity of a question, irrespective of the used models and temperatures}.
The complexity of a question is measured by first constructing the minimal fuzz driver of each question and then quantifying the API specific usage contained in the minimized code.
A minimal effective driver for a question is created based on the OSS-Fuzz driver by removing the unnecessary part of the code and replacing the argument initialization into a simpler solution according to the cases enumerated in Section~\ref{sec:preliminaries}.
Then the complexity is quantified as the sum of the count of the following elements inside code:
\ding{182} unique project APIs;
\ding{183} unique common API usage patterns;
\ding{184} unique identifiers including non-zero literals and project global variables excluding the common API usage code;
\ding{185} branches and loops excluding the common API usage code.
Note that all branches of one condition will be counted as one.
Overall, \ding{182}, \ding{184} measure API specific vocabularies while \ding{185} for API specific control flow dependencies'.
We put detailed calculation examples at \cite{fuzz-drvier-study-website}.
\compactline

% as follows:
% \ding{182} using the OSS-Fuzz driver as initial one;
% \ding{183} removing the unnecessary code.
% A piece of code is unnecessary if its removal does not affect the driver's effectiveness;
% \ding{184} simplifying the remaining code by replacing the argument initialization to a simpler form.
% The level of the simplicity is based on the cases classified in Section~\ref{sec:preliminaries}.
% We classified the initialization to the four cases mentioned in Section~\ref{sec:preliminaries},
% and the ascending order on simplicity for these cases are: \ding{172}, \ding{173}, \ding{174}, and \ding{175}.
% Figure~\ref{fig:qstn-succ-rate-per-score}/~\ref{fig:query-succ-rate-per-score} presents the success rate of questions/queries in given score buckets.
% Note that each question contains 40 repeated queries in our experiments.
% There is a stronger relationship in query success rate.
% Note that the query success rate shows a stronger relationship.
% This is reasonable since:
% \ding{182} query success rate is a more direct and fine-grained metric reflecting question difficulty;
% \ding{183} the query success rate is calculated based on 40 repeated queries per question, providing more data and leading to less fluctuation in the overall success rate calculation.
% Specifically, when score $\geq$ 7, all query success rates $<$ 10\%, and the rate drops to zero when score $\in [13, \infty]$.


% Our observation is that \textbf{LLMs' performance degrades when they need to consider more API specific usages during driver generation.}

% Although LLMs may employ different mechanisms for generating code compared to the traditional workflow of driver composition, they still must meet certain minimum requirements.
Considering the generation process, it is intuitive that \textbf{LLMs' performance degrades when the complexity of target API specific usage increases}.
To generate effective drivers, LLMs should at least generate code satisfying minimal requirements.
In other words, they must accurately predict the API argument usage and control flow dependencies.
However, this is challenging since LLMs cannot validate their predictions against documentation or implementations as humans do.
It is reasonable to assume that LLMs have learned the language basics and common programming practices due to their training on vast amounts of code.
But the API specific usage, such as the semantic constraints on the argument, cannot be assumed.
On one hand, there may only have limited data about this in training.
On the other hand, details can be lost during preprocessing or the learning stage while the accurate generation is required.
% , such as its definitions, implementations, documentation, and usage examples, comprises only a negligible proportion of the training data.
% Furthermore, details may have been lost during preprocessing or the learning stage.
Therefore, the more API usage a LLM needs to predict, the greater the likelihood of errors, particularly for less common usages that do not follow the mainstream design patterns or have special semantic constraints.
Such situations are common in C projects, whose APIs often contain low-level project-specific details.

% Figure~\ref{fig:rq2-query-succ-rates-per-score} shows the relationship between query success rate and score.
% The plot only considers the data of the best temperature of \texttt{BACTX-K} since this is an ideal case for understanding 
% \textbf{The plot reveals a clear inverse proportion relationship between success rate and score, irrespective of the used models.}

\noindent
% \begin{tcolorbox}[size=title, opacityfill=0.1, nobeforeafter, breakable]
\begin{tcolorbox}[size=title, opacityfill=0.1, breakable]
% \textbf{Key Understanding:}
The performance of LLM-based generation declines significantly when the complexity of API specific usage increases.
% Our score can be used to quantify this association.
\end{tcolorbox}

% Figure environment removed


\subsection{Failure Analysis}
\label{sec:failure-analysis}

%	analyze failures on hard to solve/unsolved questions
% 
%	-> have common blockers/obstacles 
% 
%	-> analyze/classify the root cause of the blockers
%
%	1. nonintuitive API usage
% 
%	2. indirectly dependent API usage
%
%        case studies for repetitive queries, do we need that?

% \noindent
% \textbf{Overview} \tab 
% what failure analysis can bring
% Although we have established understanding on why, it is still unclear of the detailed reasons making the generation fail.
To understand how the generation fails on API specifics, we conducted failure analysis on \texttt{BACTX-K}.
The direct failure reason of the driver is collected to reveal the current main blockers for generation.
Specifically, the runtime errors, involving 11,095 drivers, are semi-automatically analyzed while the compilation and link errors are categorized based on the compiler outputs.
In total, 52,824 ineffective drivers were analyzed.
% the direct reason of the error instead of all potential root causes.
% It is technically challenging to exhaustively count root causes as one error can be fixed in multiple ways, resulting in numerous enumeration list.
% there can exist multiple effective drivers for a given question, resulting in numerous enumeration lists.
% Instead, focusing on the direct root cause helps reveal the current main blockers for generation.
% follows the semi-automatic approach mentioned in Section~\ref{subsec:evaluation-framework}. 
% \zhc{fix the ungrouped 276!}

\noindent
\textbf{General Taxonomies} \tab 
% results of failure analysis
Figure~\ref{fig:rq2-failure-taxonomy} details the root cause taxonomy.
There are nine root causes fallen into two categories:
the grammatical errors reported by compilers in build stage, and the semantic errors which are abnormal runtime behaviors identified from the short-term fuzzing results.
\ding{182} \textit{G1 - Corrupted Code}, the drivers do not contain a complete function of code due to either the token limitation or mismatched brackets;
\ding{183} \textit{G2 - Language Basics Violation}, the code violates the language basics like variable redefinition, parentheses mismatch, incomplete expressions, etc;
\ding{184} \textit{G3 - Non-Existing Identifier}, the code refers to non-existing things such as header files, macros, global variables, members of a \texttt{struct}, etc;
\ding{185} \textit{G4 - Type Error}. One main subcategory here is the code passes mismatched number of arguments to a function.
The rest are either unsupported type conversions or operations such as calling non-callable object, assigning \texttt{void} to a variable, allocating an incomplete \texttt{struct}, etc;
\ding{186} \textit{S1 - Incorrect Input Arrangement}, the input size check either is missed when required or contains an incorrect condition;
\ding{187} \textit{S2 - Misinitialized Function Args}, the value or inner status of initialized argument does not fit the requirements of callee function.
Typical cases are closing a file handle before passing it, using wrong enumeration value as option parameter, missing required APIs for proper initialization, etc;
\ding{188} \textit{S3 - Inexact Ctrl-Flow Deps}, the control-flow dependencies of a function does not properly implemented.
Typical cases are missing condition checks such as ensuring a pointer is not \texttt{NULL}, missing APIs for setting up execution context, missing APIs for ignoring project internal abort, using incorrect conditions, etc.
\ding{189} \textit{S4 - Improper Resource Cleaning}, the cleaning API such as \texttt{xxxfree} is either missing when required or is used without proper condition checks; 
\ding{190} \textit{S5 - Failure on Common Practices}, the code failed on standard libraries function usage like messing up memory boundary in \texttt{memcpy}, passing read-only buffer to \texttt{mkstemp}, etc.
Examples of these categories are shown in ~\cite{fuzz-drvier-study-website}.

%\noindent
%\textbf{Failure Distributions} \tab 
%Figure~\ref{fig:rq1-failure-taxonomy}b, ~\ref{fig:rq1-failure-taxonomy}c show the root cause distribution.
%The plots only distinguish prompt templates since the distributions among models for a given template do not have significant difference.
%Therefore, they are omitted here for brevity.
%% , a more detailed version can be found at \zhc{add ref to more detailed version}.
%Overall, the proportion of the grammatical and semantic errors are quite different in \texttt{NAIVE-K} and \texttt{BACTX-K}.
%More than 90\% of failures in \texttt{NAIVE-K} is of grammatical error while only 54\% in \texttt{BACTX-K}.
%This indicates that most drivers generated by \texttt{NAIVE-K} failed at early stage.
%In \texttt{BACTX-K}, \texttt{G3}, \texttt{G4}, \texttt{S2}, and \texttt{S3} are top four types of root causes and together they take up 93\% of all.
%From the prospective of fix, all these four types indicate that there is a need for providing more API specific knowledge to the models.
%% , including both grammatical and semantic usages.

% For instance, for <gpt-4-0613, 0.5, BACTX-K>, the percentage of its grammatical error is less than 50\%.
% , is automatically retrievable via code analysis approaches. 
% it corresponds to information required for the fix, \textit{e.g.}, function signatures and type declaration, is relatively easy to be automatically retrieved.
Note that the percentage shown in the taxonomy represents the union results of all BACTX-K configurations.
For each configuration, its distribution breakdown may be different.
Generally, both grammatical criteria and the API semantics are common mistakes made by these LLMs.
For grammatical mistakes, many of them root in misuse of indirect usages of the target API, such as passing mismatched number of arguments to a dependent API or referencing a non-existing \texttt{struct} member.
Mostly, it is easy to identify and correct these failures since the symptoms directly map to the root causes, \textit{e.g.}, messing up the function or type declarations.
As for semantic errors, these failures are challenging for locating root causes and correction usages due to the challenges on handling program semantics.
However, reporting the existence of an error with a relevant usage description may still be helpful.
% Few cases (4\% for \texttt{G4}, \texttt{S5}) fail on coding basics or practices while a negligible part (1\% for \texttt{G1}) are of incomplete code.
% For semantic errors, 21\% cases are of \texttt{S1-4}, indicating that \textbf{nearly half of the cases failed to satisfy the semantic constraints on API usage}.
% Similarly, the errors appear on most APIs.

Overall, the failures cover API usages in various dimensions: from grammatical level detail to semantic level direction, and from target API control flow conditions to dependent APIs' declarations.
Improving this is challengeable since:
\ding{182} \textbf{the involved usage is too broad to be fully put into one prompt}, which may either exceed the token limitation or distract the model;
\ding{183} \textbf{the useful usage for generating one driver cannot be fully predetermined}.
On one hand, models are inherently blackbox and probabilistics, whose mistakes cannot be fully predicted.
On the other hand, there are usually multiple implementation choices for a given API.

% For example, models are inherently blackbox and probabilistic, whose mistakes cannot be fully predicted.
% Besides, for a given API, models usually have multiple feasible choices during implementation.

% 27\% of the \texttt{BACTX-K} cases failed to figure out the correct way to initialize the arguments of API declared in the project.

% Besides, 16\% of failures are caused by missing contextual or control flow dependencies (\texttt{S2} and \texttt{S3}).
% This type of failure is 

% \textbf{xx\% failures are indirectly related with the target API.}
% While more than 90\% of the failures can be reasoned as the lack of detailed usage information, \zhc{xx} of them are indirectly dependent by the target API.

\noindent
% \begin{tcolorbox}[size=title, opacityfill=0.1, nobeforeafter, breakable]
\begin{tcolorbox}[size=title, opacityfill=0.1, breakable]
% \textbf{Key Challenges:}
\revision{
Most failures are of mistakes in API usage specifics.
The broadness of the involved usage is the major challenge.
}
% They are too broad to be generally handled.
% from target API to its dependents, and from grammatical details to semantic constraints.
% The broadness of the involved usages raises challenge.
% for further improvement.
% including not only usages of the API itself but also its dependencies.
% One major type of error is grammatical, which can be easily identified and corrected, while another is of semantic error that are harder to locate and fix automatically.
\end{tcolorbox}
\section{Characteristics of Key Design (\textbf{RQ3})}

% Figure environment removed

\subsection{Repeatedly Query}
% most are solved in the repeat process
% effectiveness of repeat degrades with the effectiveness of the configuration
% repeat 6 times
Repeated querying is a critical aspect of prompt strategies, greatly enhancing the success rate in generating fuzz drivers regardless of employed models, temperatures, and prompt designs.
Specifically, for the optimal configuration <gpt-4-0613, 0.5, ALL-ITER-K>, approximately 47.44\% of the issues were resolved by reinitiating the query process (37 out of 78 total resolved issues were solved upon repetition).
For the top-20 configurations, this contribution remains significantly high at an average of 67.50\%.
% all configurations: 56.63\%

Figure~\ref{fig:eff-repeatedly-query-a} displays the count of questions resolved through repeated querying across all evaluated configurations, ranked by their overall effectiveness as detailed in Table~\ref{tab:overall_eva_rslt}.
This demonstrates a direct correlation between the benefit of repeated queries and the efficacy of the configuration—\textbf{the more effective a configuration, the greater the gains from repeating the queries}.

Additionally, Figure~\ref{fig:eff-repeatedly-query-b} presents the average percentage of questions resolved in each subsequent round of querying for the top-20 configurations.
Here, the percentage for round X is determined by $\frac{Rslt(X) - Rslt(X - 1)}{Rslt(1)}$, with $Rslt(X)$ indicating the number of questions resolved by round X.
The X-axis starting from round two, highlighting that the first round corresponds to the initial query.
This data shows that \textbf{the gain of repeated queries drops significantly after the initial few rounds}.
From our evaluation, we recommend limiting repeated queries to no more than six, where the sixth round still manages to resolve an additional 20\% of questions compared to the results of the first round.

% Repeatedly query is one essential design in prompt strategies which significantly contributes to the overall effectiveness of generating fuzz drivers.
% Specifically, for the best configuration <gpt-4-0613, 0.5, ALL-ITER-K>, around 47.44\% questions are solved by repeating the queries (37 solved by repeat out of 78 solved in total).
% For the top-20 configurations, the average figure keep as high as xx\%.

% Figure~\ref{fig:eff-repeatedly-query-a} lists the number of questions solved by repeatedly query for all evaluated configurations.
% The configurations are sorted by the number of their solved questions shown in Table~\ref{tab:eval_full}.
% The result clearly shows that the effectiveness of repeatedly query is proportional to the effectiveness of the configuration. 
% In other words, \textbf{more effective a configuration is, more benefits you will earn by employing repeatedly query design}.

% Figure~\ref{fig:eff-repeatedly-query-b} details the average percentage of solved question for each repeat round in top-20 configurations.
% Given one configuration, the percentage of round X is calculated by $\frac{Rslt(X) - Rslt(X - 1)}{Rslt(1)}$, where $Rslt(X)$ represents the number of solved questions in round X.
% Note that the round number in figure X-axis starting from two, \textit{i.e.}, the second round, which is the first round of repeatedly query.
% The plot illustrates that, \textbf{the benefits of repeatedly query significantly drops after the first several rounds of the repeat}.
% According to our evaluation, we suggest the time of repeatedly query is set less than six, the last round which has additionally solved 20\% number of questions comparing to the first round's result.

% top 1, 5, 10, 20 configurations
% Pct. solved by repeat: 0.430233, 0.532558, 0.493023, 0.448837
% solved by repeat: 37, 45.8, 42.4, 38.6
%
% suggested repeated time: 5 - 8 (0.2 & 0.1)
%
%
% TOP 20
%
%model: gpt-4-0613, temp: 0.5, strategy: ALL-ITER-K, final_tp: 78, rounds: 40
%model: gpt-4-0613, temp: 0.0, strategy: ALL-ITER-K, final_tp: 77, rounds: 40
%model: gpt-4-0613, temp: 1.0, strategy: ALL-ITER-K, final_tp: 76, rounds: 40
%model: gpt-3.5-turbo-0613, temp: 0.5, strategy: ALL-ITER-K, final_tp: 68, rounds: 40
%model: gpt-3.5-turbo-0613, temp: 0.0, strategy: ALL-ITER-K, final_tp: 65, rounds: 40
%model: gpt-3.5-turbo-0613, temp: 1.0, strategy: ALL-ITER-K, final_tp: 65, rounds: 40
%model: gpt-4-0613, temp: 0.5, strategy: UGCTX-K, final_tp: 63, rounds: 40
%model: gpt-4-0613, temp: 1.0, strategy: UGCTX-K, final_tp: 62, rounds: 40
%model: gpt-4-0613, temp: 1.0, strategy: BA-ITER-K, final_tp: 62, rounds: 40
%model: gpt-4-0613, temp: 0.5, strategy: BA-ITER-K, final_tp: 57, rounds: 40
%model: gpt-4-0613, temp: 0.0, strategy: BA-ITER-K, final_tp: 56, rounds: 40
%model: gpt-4-0613, temp: 0.0, strategy: UGCTX-K, final_tp: 55, rounds: 40
%model: wizardcoder-15b-v1.0, temp: 1.0, strategy: ALL-ITER-K, final_tp: 53, rounds: 40
%model: wizardcoder-15b-v1.0, temp: 0.5, strategy: UGCTX-K, final_tp: 50, rounds: 40
%model: wizardcoder-15b-v1.0, temp: 0.5, strategy: ALL-ITER-K, final_tp: 48, rounds: 40
%model: wizardcoder-15b-v1.0, temp: 1.0, strategy: UGCTX-K, final_tp: 48, rounds: 40
%model: gpt-3.5-turbo-0613, temp: 0.5, strategy: UGCTX-K, final_tp: 47, rounds: 40
%model: gpt-3.5-turbo-0613, temp: 0.5, strategy: BA-ITER-K, final_tp: 47, rounds: 40
%model: gpt-3.5-turbo-0613, temp: 1.0, strategy: UGCTX-K, final_tp: 43, rounds: 40
%model: gpt-3.5-turbo-0613, temp: 1.0, strategy: BA-ITER-K, final_tp: 43, rounds: 40


% \begin{table}[t]
% \centering
% \caption{Questions Solved by Repeat in Top-X <Model, Temperature, Strategy>.}
% \label{tab:solved-by-repeat-in-top-x}
% \resizebox{0.8\linewidth}{!}{
% \begin{tabular}{lllll}
% % \hline
% \toprule
% Top-X 
 % & X=1
 % & X=5
 % & X=10
 % & X=20 \\
% \midrule
% % \rowcolor{black!10}
% Solved by Repeat & 43\% (37) & 53\% (46) & 49\% (42) & 45\% (39) \\
% % \hline
% \bottomrule
% \end{tabular}
% }
% \label{tab:cmp-different-k}
% \end{table}

% The observation \ding{184} indicates the usefulness of repetitive queries.
% properly exploiting the randomness in LLM replies can significantly improve the overall performance.
% For \ding{184}, though the ratio of total corrected questions for 40 rounds to one round can more than nine, the benefits of the repetitive queries rapidly decrease after certain rounds of repeated queries, roughly following the Pareto Principle~\cite{pareto-principle}.
% Specifically, as shown in Table~\ref{tab:cmp-different-k}, roughly 80\% of the overall performance of repetitive queries are contributed in the initial 20\% rounds of queries. 
% all settings have reached nearly or more than 80\% performance at the sixth round comparing with the total outcomes of 40 rounds.
% In general, a simple stop condition can be proposed, \textit{e.g.}, no performance increase in last X rounds, to balance the benefits and query costs.
% the number of repeat times, \textit{a.k.a.} the value of \textbf{K}, can be adaptively determined according to the past statistics of the new correct answers.

% \noindent
% \textbf{Question-Level Performance} \tab 
% % Besides discussing the overall performance, we analyzed these strategies on question-level statistics.
% Figure~\ref{fig:upset-plot-for-simple-strategies} shows the UpSet plot.
% % analyzing the solved questions sets for the above strategies.
% Among all 86 questions, 35 of them (40.70\%) have not been solved by any basic strategy, which shows a large space for further improvement.
% An interesting observation is that most strategies have uniquely solved questions: \texttt{gpt3.5-BACTX-K}, \texttt{gpt4-NAIVE-K}, and \texttt{gpt4-BACTX-K} have 2, 5, and 11 uniquely solved questions respectively. 
% On the one hand, it is an evidence that the current most effective strategy \texttt{gpt4-BACTX-K} still cannot outperform the rests in all respects. 
% On the other hand, it reminds us the probabilistic nature of the language models.
% For instance, though \texttt{gpt4-NAIVE-K} have same prompts/model as \texttt{gpt4-BACTX-K} and contain less descriptive information about the target API, it still can solve 5 questions that \texttt{gpt4-BACTX-K} failed to solve in 40 rounds.
% One possible explanation is that, for these questions, the less descriptive prompts generated by \texttt{gpt4-NAIVE-K} can happenly guide the \texttt{gpt4} model produce effective answers.
% A supportive finding is that all 5 questions uniquely solved by \texttt{gpt4-NAIVE-K} are of one project and their API usage follow similar design patterns.
% Besides, the average query success rate of \texttt{gpt4-BACTX-K} is higher than the rest questions, which indicates that \texttt{gpt4-BACTX-K} not only solves more questions but also solves them more reliably.
% % Specifically, the 11 questions uniquely solved by \texttt{gpt4-BACTX-K} cover 6 projects and have an average success rate 28.41\% on query, while the \texttt{gpt3.5-BACTX-K} and \texttt{gpt4-NAIVE-K}'s 2/5 questions cover 1/1 projects, with a success rate of 2.50\%/10.50\%.


% Figure environment removed

\subsection{Query With Extended Information}
% QSTN: EX, NOEX, IN, OUT
% 0.673333 0.359167 0.9275 0.143333
% QUERY: EX, NOEX, IN, OUT
% 0.300833 0.0875 0.193333 0.0491667

\noindent
\textbf{Querying With API Documentation.}
\tab
By comparing DOCTX-K and BACTX-K, we found that \textbf{there is no significant changes between their results in the metrics of resolved questions}.
On one hand, a significant percentage (43\%) of APIs in the evaluated questions do not have API documentation (49 out of 86 have).
When there is no documentation for an API, the DOCTX-K queries are identical to BACTX-K's.
On the other hand, adding API documentation in the queries may not provide enough details directly stating the API usage.
This is because these API documentations usually contain a high-level description of the usage, typically a summary of main functionality with one-sentence explanations for arguments. 
However, the blocker-solving usage information discussed in Section~\ref{sec:failure-analysis}, such as low level argument initialization specifics, control flow dependencies, or the usages of its dependent APIs, is usually not included.
% \zhc{add some data on these 49 Qs?}
% For all 49 questions that have API documentation, we compared the performance of \texttt{DOCTX-K} with that of \texttt{BACTX-K} under \texttt{gpt3.5} and \texttt{gpt4} models to understand its effectiveness.
% The first row of Figure~\ref{fig:upset-plot-for-extended} shows upset plot and Figure~\ref{fig:extended-info-succ-rate-plots-per-score}'s plots the comparison in metrics of question and query success rate.
% Line color represent strategy type while line style stands for the model type.
% The black lines represent the performance of \texttt{gpt4} while the red for \texttt{gpt3.5}'s.
% The lines with cross sign markers are of \texttt{BACTX-K} and triangle markers are of \texttt{DOCTX-K}.
% Overall, \texttt{DOCTX-K} slightly outperforms \texttt{BACTX-K} in \texttt{gpt3.5} model while performs nearly identical in \texttt{gpt4}.
% under \texttt{gpt4} model, both strategies solve the same number of questions and their query success rates are nearly identical.
% In \texttt{gpt3.5}, it uniquely solves three more questions than \texttt{BACTX-K}.
% only had a slightly better performance than \texttt{BACTX-K} by uniquely solving four questions, whereas \texttt{BACTX-K} could only solve one unique question.
% in the documentation.
% usage information required to solve the blockers analyzed in

% % \zhc{add a case analysis here to support the conclusions in detail.}
\noindent
% \begin{tcolorbox}[size=title, opacityfill=0.1, nobeforeafter, breakable]
\begin{tcolorbox}[size=title, opacityfill=0.1, breakable]
API documentation has minor performance benefits due to the limited usage description it contained.
\end{tcolorbox}
% while adding API documentation in prompt causes no negative effects on the effectiveness, it can only improve the performance in limited cases.}

%% Figure environment removed



% 7712/32367
% example sources analysis
\noindent
\textbf{Querying With Example Code Snippets.} 
\tab
When comparing the results of BACTX-K and UGCTX-K presented in Table~\ref{tab:overall_eva_rslt}, we can clearly observe that incorporating example code snippets substantially enhances performance in most configurations.
In particular, the addition of example snippets results in an average resolution of 104\% more questions across the 22 evaluated configurations, which includes five models and five different temperature settings.

Nonetheless, further analysis reveals that \textbf{the inclusion of usage examples incurs a much higher token cost, with an average increase of tenfold}.
The ratio of token costs for these two approaches varies from 4.20 to 39.71 across all configurations, with an average ratio of 14.65.
Notably, the UGCTX-K approach demands an average of 32,367 tokens to generate a single correct solution.

Figure~\ref{fig:succ-rate-of-different-ex-sources} depicts our investigation into the impact of different sources of example snippets on the quality of solutions.
This figure assesses the success rates of queries/questions associated with various example sources, which are categorized in two distinct manners based on their file paths: first, as \ding{182} \textit{External} vs. \textit{Internal}, with \textit{Internal} comprising the target project and its variations, and \textit{External} consisting of all other sources; second, as \ding{183} \textit{Test \& Example} vs. \textit{Others}, where the first group includes files with paths that contain "test" or "example" in any capitalization.
The underlying data for these plots stems from questions that were solved by UGCTX-K but not by BACTX-K across all tested configurations.
According to this analysis, it is clear that \textbf{both \textit{Internal} and \textit{Test \& Example} sources are associated with significantly higher quality example snippets in comparison to their counterparts}.

% \zhc{analysis on the distraction?}

% The plots states that, for all quiz questions, the number of correct fuzz drivers generated from the prompts built based on examples from internal source is two times more than the external's.
% For the questions that are not 
% \zhc{missing interpretation on data}
% \zhc{do we need to mention how we did this source identification? in appendix?}
% high effectiveness
% cost high
% the effectiveness depends on the quality of example sources, we identified two high quality sources
% This indicates that example snippets are not free to add since unhelpful code can distract or even confuse the model, leading to unsuccessful generation.
% In buckets have higher score, this disturbance is not observable since the query success rate of \texttt{BACTX-K} is typically low.
% \texttt{UGCTX-K} is evaluated on all quiz questions.
% The second row of Figure~\ref{fig:upset-plot-for-extended} shows the upset plot while Figure~\ref{fig:extended-info-succ-rate-plots-per-score}'s compares in metrics of question and query success rate.
% Similarly, line colors/styles distinguishes prompt templates/models.
% Overall, \texttt{UGCTX-K} clearly outperforms \texttt{BACTX-K}:
% in terms of overall effectiveness.
% The union results of \texttt{gpt3.5-UGCTX-K} and \texttt{gpt4-UGCTX-K} correctly generates effective fuzz drivers for 
% Considering the union results of \texttt{UGCTX-K},
% \texttt{gpt3.5-UGCTX-K} and \texttt{gpt4-UGCTX-K} 
% it solved 70 questions which covers almost all questions solved by \texttt{BACTX-K}. 
% For each model, Figure~\ref{fig:extended-qstn-succ-rate-per-score} shows that \texttt{UGCTX-K} (red lines) is always higher than or equal to corresponding \texttt{BACTX-K} (black lines).
% , indicating a clear advantage for \texttt{UGCTX-K}.
% Interestingly, Figure~\ref{fig:extended-query-succ-rate-per-score} shows that \texttt{UGCTX-K} has lower query success rates than \texttt{BACTX-K} in low score questions.
% \yaowen{need to check} Additionally, the intersection point (the third score bucket $[7,9]$) is the point where the question success rates of \texttt{BACTX-K} drop significantly. 
% This effect is particularly pronounced for questions that are likely to be solved without the help of snippets (buckets with lower scores).
% significantly affected by the questions it cannot solve.



\noindent
\textbf{Case Studies.}
\tab
% Two cases are discussed to show how usage snippets help in solving the common blockers.
% counter-intuitive case: 
\# 9 \texttt{wc\_Str\_conv\_with\_detect}
\tab
This case is challenging due to the unintuitiveness of its API usage.
The API declaration is "\texttt{Str wc\_Str\_conv\_with\_detect(Str is,wc\_ces * f\_ces,wc\_ces hint,wc\_ces t\_ces)}".
It is used for converting the input stream \texttt{is} from one CES (character encoding scheme, \texttt{f\_ces}) to another (\texttt{t\_ces}).
Most basic strategy drivers made mistakes on the creations of either \texttt{is} (the confusing type \texttt{Str}) or CESs, where \texttt{is} has to be created using particular APIs like \texttt{Strnew\_charp\_n} and CESs should be specific macros or carefully initialized \texttt{struct}.
Example helps here by directly providing the usage to models.

% control flow conditions: parse_xxx
\# 37 \texttt{igraph\_read\_graph\_graphdb}
\tab
The hardest part in this case is the implicit control flow dependency it required.
Besides correctly initializing the arguments, it has to call an API to mute the builtin error handlers.
By default, the API will abort immediately when any abnormal input is detected, which causes frequent false crashes blocking the fuzzing progress.
% when an abnormal input is detected, the API will abort frequently which blocks the fuzzing progress and raise large amount of false crashes.
To mute it, the driver needs to custmoize the error handler, \textit{e.g.}, call \texttt{igraph\_set\_error\_handler\-(igraph\-\_error\_handler\_ignore)}.
% a customized error handler needs to be 
% the driver has to set a silent error handler, such as calling 
This requirement is hard to be inferred beforehand due to its semantic nature and few inference clues.
However, some unit tests in project such as \texttt{foreign\_empty.c} contain this usage, which directly instructs the generation.


% Figure environment removed
\noindent
% \begin{tcolorbox}[size=title, opacityfill=0.1, nobeforeafter, breakable]
\begin{tcolorbox}[size=title, opacityfill=0.1, breakable]
    \revision{
Example code snippets can greatly enhance model performance by providing direct insights on API usage.
    }
"test/example files", "code files from the target/variant projects" are high quality sources.
% In summary, example code snippets can significantly improve the overall performance.
% They directly provides the usages which models usually failed to infer.
% However, adding usage snippets can cause performance decline if low quality candidates are used.
\end{tcolorbox}

\subsection{Iterative Query}
% performance is better than others
% average cost is even higher 
% 159% baiter improves basic
% 23% alliter improves usage
The iterative query strategy is another key design that can lead to significant improvements in performance.
Referring to Table~\ref{tab:overall_eva_rslt}, we find that, on average, incorporating an iterative query strategy into BACTX-K -- that is, adopting the BA-ITER-K approach -- helps solve 159\% more questions.
Similarly, ALL-ITER-K resolves 23\% more questions than UGCTX-K. 
However, this strategy does come at a cost.
\textbf{The inclusion of iterative design tends to lead to higher token usage when generating correct solutions}.
On average, the iterative strategy increases token costs by 57\% for BACTX-K per successful driver generation and by 17\% for UGCTX-K.
% Specifically, the token cost ratio between BA-ITER-K and BACTX-K varies from 1.29 to 2.45 across all configurations, while the ratio of ALL-ITER-K to UGCTX-K ranges from 0.67 to 1.85.

The effectiveness of the iterative strategy can be attributed to two key factors.
Firstly, \textbf{it leverages a wider array of information}, including error data generated from validating previously generated drivers.
Secondly, \textbf{it tackles the problem incrementally}, employing a step-by-step, divide-and-conquer approach that simplifies the complexity of the generation task.
This methodology is exemplified in the case studies that follow, illustrating how the iterative strategy typically operates through practical examples.

% 
% 
%% Figure environment removed
%
%% Figure environment removed

%% Figure environment removed

% \noindent
% \textbf{Effectiveness of Iterative Query}
% \tab
% which questions are compared
% introduce the plots
% Due to the search nature and high cost of the evaluation, we use partial questions for evaluation.
% In our experiments, both the maximum iteration round and the \texttt{K} are set as 20.
% Figure~\ref{fig:all-iter-succ-rate-plots-per-score} and ~\ref{fig:upset-plot-for-iter} provide comparison results.
% compare \texttt{ITER-K} and \texttt{UGCTX-K} and the two iterative strategies. 
% conclusions draw from plots
% Generally, \texttt{ITER-K} strategies have clear performance benefits.
% They solve almost all questions solved by \texttt{UGCTX-K} and uniquely solves nine questions, leaving only six questions unsolved.
% Besides, its question success rate is always higher than its counterpart's (red lines are higher than black lines).
% It is worth to mention that mostly \texttt{ITER-K} has lower query success rate than \texttt{UGCTX-K}, indicating its higher average search cost in generation.
% overall performance advantage -> can solve hard question
% low success rate -> high cost 
% Indeed, averagely the number of queries required by iterative strategies is 3-5 times to non-iterative strategies (see detailed data in~\cite{fuzz-drvier-study-website}).

\noindent
\textbf{Case Studies.}
\tab
% Two cases are discussed to demonstrate how this strategy solves the questions others failed to.
\#5 \texttt{md\_html}
\tab
This API requires the preparation of a customized callback function pointer as the argument, where all previous strategies failed to figure out.
% Previous strategies failed to provide a valid callback function pointer as the argument in this case.
% The API usage required in this case is not complex but the previous strategies failed to provide a valid callback function pointer as the argument.
The callback function is used to handle the output data of API.
% To correctly execute the API, the driver has to prepare a function which handles the outputted data of the API and passes it to the API.
All the drivers generated by \texttt{UGCTX} either pass a \texttt{NULL} pointer or a non-existing function name.
Iterative query guides the fix by providing the link error highlighting that this referred function is undefined.

\#73 \texttt{pj\_stun\_msg\_decode}
\tab
This is another typical case why iterative strategy works.
% suits the multi-round iterative solution.
% one of the iterative query found the correct answer in five iteration rounds.
% The initialization of its first argument, a pointer to a memory pool used for runtime memory management, has multi-level API dependencies.
The initialization of its first argument has multi-level API dependencies.
The dependency chain is: 
\ding{182} the API ->
\ding{183} \texttt{pj\_pool\-\_create} -> 
\ding{184} \texttt{pj\_caching\_pool\-\_init}, where -> means depends.
All non-iterative strategies failed to prepare a driver with all correct usage detail of these indirect dependencies while iterative strategies solve this by providing error related feedback to LLMs and solving multiple errors one by one.
% identifying the error related API and providing correct usage for correction. and it solves multiple errors one by one.
% requires the calls of the two APIs in order:
% \ding{182}  for initializing caching pool object \texttt{pj\_ca\-ching\_pool};
% \ding{183} \texttt{pj\_pool\-\_create} for initializing memory pool object \texttt{pj\_pool\_t}.
% The iterative query solves this by first correcting the wrongly used API for initializing caching pool
% first corrects the wrongly used API for initializing caching pool object (\ding{182}), then figures out the mismatched type error for calling \ding{183}.
In one of the solved iterative query, it first corrects the incorrect used API of \ding{184}, then figures out the mismatched type error when calling \ding{183}.
Lastly, for the driver's runtime crash, LLMs use two rounds to fix according to the assertion code located from crash stacks.
% related implementation code of failed assertion lines located by crash stacks, the iterative query uses two iterations to test out the correct size check conditions for the mutated input data.

% md html, shows the iterative advantage (for fixing easy errors)
% a hard question, shows the necessity of the iterative for solving this question

\noindent
% % \begin{tcolorbox}[size=title, opacityfill=0.1, nobeforeafter, breakable]
\begin{tcolorbox}[size=title, opacityfill=0.1, breakable]
    \revision{
Iterative query helps in utilizing more diverse information and solving the problem in a step-by-step manner.
However, it has higher token cost and increased complexity.
    }
\end{tcolorbox}
%  \begin{table*}[t]
%  \centering
%  \caption{Comparison of Metrics for LLM-Generated and OSS-Fuzz Drivers.
%  \footnotesize{
%  X-axis represents the question id.
%  For clarity, the value in x-axis is omitted.
%  All subfigures should be interpreted separately since they are evaluated on different question id groups.
%  Y-axis represents the number of used APIs, coverage, and the number of crashes for the 1st/2nd/3rd row respectively.
%  The y-axis of the 3rd row is in log scale.
%  }
%  }
%  \label{fig:rq3-cmps}
%  \resizebox{1.0\linewidth}{!}{
% \begin{tabular}{llllllllllllllllllllllllllllllllllllllllllllllllllllllllll}
% % \hline
% \toprule
% 11&11&11&11&11&11&11&11&11&11&11&11&11&11&11&11&11&11&11&11&11&11&11&11&11&11&11&11&11&11&11&11&11&11&11&11&11&11&11&11&11&11&11&11&11&11&11&11&11&11&11&11&11&11&11&11&11
% \\
% % \begin{tabular}[c]{@{}c@{}}Time\\(sec)\end{tabular} \\ 
% % \hline
% \midrule
% % \grayrow
% \bottomrule
% \end{tabular}
% }
%  \end{table*}

% % Figure environment removed

\section{OSS-Fuzz Driver Comparison (\textbf{RQ4})}
\label{sec:rq3}

\noindent
\textbf{Comparison Overview}
\tab
% three groups:
% BACTX
% UGCTX
% ITER
% In previous sections, we analyzed the effectiveness of different query strategies.
% Most fuzz drivers in OSS-Fuzz are manually written, applied practically on industrial projects, and continuously maintained and improved for years.
% strategies covered
% The comparison includes four types of query strategies: \texttt{BACTX-K}, \texttt{DOCTX-K}, \texttt{UGCTX-K}, and \texttt{ITER-K}, where the \texttt{K} remains the same as in the previous section.
We compared LLM-generated drivers with OSS-Fuzz's to obtain more practical insights.
Note that OSS-Fuzz drivers are practically used in industry for continuous fuzzing and most of them are manually written and improved for years.
Particularly, LLM-generated drivers under comparison are from gpt-4-0613 and wizardcoder-15b-v1.0 using iterative strategies with temperature 0.5.
These two configurations are the best representative for closed-source and open-source LLMs.
In total, we evaluated 53 questions which are both resolved by all configurations.
Multiple drivers of one question are merged as one to ease the comparison.
This is done by adding a wrapper snippet which links the seed scheduling with the selection of the executed logic from merged drivers.
Specifically, a switch structure is added to determine which driver it will execute based on a part of the input data.
During each fuzzing iteration, only the logic of one merged driver is executed.
Besides, some compound OSS-Fuzz drivers are designed to fuzz multiple APIs.
For clear comparison, we merged all drivers of questions involved in one compound driver as one.
In total, we prepared 38 drivers for each assessed LLM or OSS-Fuzz.
The comparisons cover both code and fuzzing metrics such as the number of used APIs, oracles, coverage, and crashes.
% which is a switch structure selecting the execution of each original driver based on the input (inspired coalescing method in FuzzGen~\cite{fuzzgen}).
% The wrapper is mainly a switch statement calling the logic of each original driver based on the first byte of the input data.
% By using this approach, fuzzer can automatically control the exploration of the logics from multiple drivers during its seed scheduling.
% The compared drivers and questions are not in one-to-one relationship, \textit{e.g.}, some OSS-Fuzz drivers are compound drivers fuzzing multiple APIs and multiple drivers can exist for fuzzing one API.
% Note that each strategy can only provide effective fuzz drivers for a part of the questions in the quiz.
% questions covered
% To investigate the main features of each strategy, the comparison focuses on each strategy's good performance questions.
% The good performance question for a strategy is the one that at least one effective fuzz driver has been generated on all evaluated LLMs in previous evaluation.
% In total, 72 questions are evaluated.
% Details of the question groups are listed in \appe~\ref{rq3-question-group-detail}.
% \zhc{or website?}
% prepare the compared driver
% metrics covered
% The comparisons covers both static and dynamic metrics including the number of used APIs, the number of oracles, the fuzzing coverage, and the number of  crashes.
% \ding{182} the number of unique APIs;
% \ding{183} the number of oracles;
% \ding{184} the fuzzing coverage;
% \ding{185} and the number of unique crashes.

\noindent
\textbf{Fuzzing Setup}
\tab
% how dynamic experiments are conducted
% For dynamic metrics \ding{184} and \ding{185}, fuzzing experiments are conducted.
Considering the randomness of fuzzing, we followed the suggestions from ~\cite{fuzz-eval}: 
the fuzzing experiments are conducted with five times of repeat for collecting average coverage information and the fuzzing of each driver lasts for 24 hours.
We used \texttt{libfuzzer}~\cite{libfuzzer} and \texttt{AFL++}~\cite{fioraldi2020afl++} as fuzzers with empty initial seed and dictionary.
"\texttt{-close\_fd\_mask=3 -rss\_limit\_mb=2048 -timeout=30}" is used for \texttt{libfuzzer} while \texttt{AFL++}'s is the default setup of \texttt{aflpp\_driver}.
For fair comparison, the coverage of fuzz driver itself is excluded in post-fuzzing data collection stage (the merged driver can have thousands of lines of code) but kept in fuzzing stage for obtaining coverage feedback.
% the fuzz driver coverage feedback during the fuzzing.
In total, the experiments took 3.75 CPU year.
% Due to page limit, Figure~\ref{fig:rq4-cmps} shows overall comparison results while full detailed data can be found at~\cite{fuzz-drvier-study-website}.

% static: 
%   api num (base & extended)
%   oracles comparison

% dynamic:
% cov
% crash

% \subsection{Comparison on Code Metrics}

\noindent
\textbf{Code Metric: API Usage}
\tab
% number of unique APIs (divided as Basic APIs and Extended APIs)
% Figure~\ref{fig:cmp-apinum} shows the results.
% Note that all \texttt{gpt4-BACTX-K} drivers (ttl 35) are additionally added in plot for supporting the findings.
% most \texttt{gpt4-ITER-K} drivers (86\%, 49/57, red bars) have used more or equal project APIs compared with OSS-Fuzz's (black bars).
% in most cases, \texttt{gpt4-ITER-K} (red bars) are higher than or equal to the OSS-Fuzz (black bars).
% In most questions, \texttt{BACTX-K} and \texttt{DOCTX-K} used less APIs than OSS-Fuzz while \texttt{UGCTX-K} and \texttt{ITER-K} contain at least the same amount of APIs.
% This is intuitive since the latter two can provide examples guiding models to use more diverse API usages while the former two cannot.
% conservative
% Mostly, it code will soon end the execution after the call of the target API with some necessary cleaning steps.
The API usage is measured by the number of unique project APIs used in the fuzz driver.
Overall, 14\% (17/35) \texttt{gpt-4-0613} drivers have used less project APIs than OSS-Fuzz's while 39\% for \texttt{wizardcoder-15b-v1.0}.
By manually investigating these drivers, we found that \textbf{LLMs conservatively use APIs in driver generation if no explicit guidance in prompts}.
For instance, some drivers only contain necessary usages such as argument initialization.
And the API usage is hardly extended such as adding APIs to use an object after parsing it.
This is a reasonable strategy since aggressively extending APIs increases the risk of generating invalid drivers.
Adding example snippets in the prompt can alleviate this situation.
% for example, src/igraph/tests/unit/foreign_empty.c provides many APIs
% models will use the API usage explicitly shown in the 
As for OSS-Fuzz drivers, the API usage diversity is case by case since they are from different contributors.
Some drivers, \textit{e.g.},~\cite{croaring-ossfuzz-driver-link} are minimally composed and some are extensively exploring more features of the target, \textit{e.g.},~\cite{lua-ossfuzz-driver-link}.
One interesting finding is that some OSS-Fuzz drivers are modified from the test files rather than written from scratch, which is a quite similar process as querying LLM with examples.
For example, \texttt{kamailio} driver~\cite{kamailio-ossfuzz-driver-link} is modified from test file~\cite{kamailio-test-example-link}.
Prompting with this example, LLM can generate similar driver code.

% an example for example code can provide high diversity of API usage
% intersting finding

% 
% case study mention the ugctx, kamailio thing (similar approach as human experts, based on example code)
% usage & iter can have greater number of apis, but its structure may not be as good as human written drivers 

\noindent
\textbf{Code Metric: Oracle}
\tab
% oracles comparison & taxonomy
% manual written has many oracles
% generated has very few (is the generated oracle correct?)
% give some cases on what are usually patterns of manually written oracles
We did statistics on the oracles of the drivers.
The result is quite clear:
in all 78 questions resolved by LLMs, OSS-Fuzz drivers of 15 questions contain at least one oracle which can detect semantic bugs, while there are \textbf{no LLM-generated drivers have oracles}.
The used semantic oracles can be categorized as following:
\ding{182} check whether the return value or output content of an API is expected, \textit{e.g.}, ~\cite{bind9-api-output-oracle};
\ding{183} check whether the project internal status has expected value, \textit{e.g.}, ~\cite{igraph-internal-status-oracle};
\ding{184} compare whether the outputs of multiple APIs conform to specific relationships, \textit{e.g.}, ~\cite{bind9-check-two-apis-oracle}.

% brief intro on oracles

\noindent
% \begin{tcolorbox}[size=title, opacityfill=0.1, nobeforeafter, breakable]
\begin{tcolorbox}[size=title, opacityfill=0.1, breakable]
LLMs tend to generate fuzz drivers with minimal API usages, significant space are left for further improvement such as extending the use of API outputs or adding semantic oracles.
% for better semantic bug detection.
\end{tcolorbox}

% \subsection{Comparison on Fuzzing Metrics}

\noindent
\textbf{Fuzzing Metric: Coverage and Crash}
\tab
% average value
% the relation between coverage and API usage
% the conclusion is it is comparable with human written drivers
% however, this also with the help of manually filtering
Figure~\ref{fig:cmp-cov},~\ref{fig:cmp-crash} plot the coverage and crash comparison results.
% Considering that the comparison experiments involve results of 228 different drivers, 
Instead of presenting every detail of the experiments for hundreds of drivers, the plots lists the comparison in certain metrics while the full experiment details can be found at~\cite{fuzz-drvier-study-website}.
Overall, \textbf{in most questions, the LLM-generated drivers demonstrate similar or better performance in metrics of both coverage and the number of uniquely found crashes}.
% Particularly, the drivers of \texttt{UGCTX-K} and \texttt{ITER-K} reach similar coverage in most questions.
% One interesting finding is that the performance of the merged fuzz driver can be decreased due to the poor usage driver it incorporates.
% Specifically, in Figure~\ref{fig:cmp-ugctx}, there is a segment of the relatively flat line, where the \texttt{UGCTX-K} performs significantly worse than OSS-Fuzz.
% However, according to its corresponding segment in Figure~\ref{fig:cmp-ugctx}, the drivers have used as many unique APIs as OSS-Fuzz's.
% The reason is that there are dozens of drivers insides the merged drivers for these questions and the large portion of poor usages distracted the fuzzer.
% This indicates that handling various LLM-generated drivers is not a naive thing. 
% Proper filtering for retrieving high quality drivers may be necessary for driver effectiveness improvement.
% the potential spam issue that LLM-based driver generation can face.
% We discuss this in Section~\ref{xxx} in detail.
% \noindent
% \textbf{Fuzzing Metric: Crash}
% \tab
% Figure~\ref{fig:cmp-crash} shows the crash results.
% In plots, the LLM-generated fuzz drivers show qualified bug finding abilities.
% In most questions that can be found bugs in our experiments, LLM-generated fuzz drivers show better crash finding outcomes.
Note that there are no false positive since the generated fuzz drivers are already filtered by the semantic checkers provided from our evaluation framework.
If only the fully automatic validation process are adopted, \textit{i.e.}, removing the last two checkers in Figure~\ref{fig:validation-checker}, the fuzzing outcome will be messed with huge number of false positives, incurring significant manual analysis efforts.
% our evaluation framework, which has manually written semantic checkers.
% one invalid or unsound usage incorporated in the merged driver can quickly raise large number of false positives, which require the additional processing efforts.

\noindent
% \begin{tcolorbox}[size=title, opacityfill=0.1, nobeforeafter, breakable]
\begin{tcolorbox}[size=title, opacityfill=0.1, breakable]
LLM-generated drivers can produce comparable fuzzing outcomes as OSS-Fuzz drivers.
In large scale application, how to practically pick effective fuzz drivers is the major challenge.
\end{tcolorbox}

% \section{Discussion}
% \label{sec:discussion}
\section{Discussion}
\label{sec:ttv}

\revision{
\noindent
\textbf{Relationships With OSS-Fuzz-Gen}
\tab
The Google OSS-Fuzz team has undertaken a parallel work called OSS-Fuzz-Gen~\cite{oss-fuzz-gen} for leveraging LLMs to generate fuzz drivers.
Based on their public information, \textit{i.e.}, one security blog~\cite{oss-fuzz-gen-blog} and the source code repository~\cite{oss-fuzz-gen}, our work is complementary to theirs.
Overall, current stage of OSS-Fuzz-Gen puts high efforts on filling the engineering gap between LLM interfaces and OSS-Fuzz projects.
Their experiments are conducted on top commercial LLMs, aiming to showcase that LLM-generated fuzz drivers can help in finding zero-day vulnerabilities and reaching new testing coverage.
However, there is no experiments or documentation discussing fundamental generation questions such as the design choices behind their prompt strategy, what are the pros and cons for different strategies, how the effectiveness varies for different models and model parameters, and what are the inherent challenges and potential future directions.
Our study, on the other hand, complements theirs by exploring these fundamental issues.
We carefully designed prompt strategies, evaluated them on various models (open and commerical LLMs) and temperatures, and distilled findings from the results. 
}

\revision{
\noindent
\textbf{Contributing to OSS-Fuzz-Gen}
\tab
%To turn our insights into practical industry values,
We carefully examined the prompt strategies of OSS-Fuzz-Gen from their implementation and validated where our insights can help.
Interestingly, their current strategy support part of our insights.
For instance, they adopted 10 time repeat results~\cite{oss-exp-repeat} and used a lower temperature (0.4) in experiments~\cite{oss-default-temperature}.
Besides, we found that OSS-Fuzz-Gen only identifies and fixes build errors while ignoring the runtime errors caused by driver.
Their generation ends when a compilable fuzz driver is synthesized and then they manually checks the validity of these drivers.
To improve this, we implemented our strategies for drivers with fuzzing runtime errors in their platform, including the identification (automatic part of validation process)~\cite{oss-pr-191, oss-pr-187, oss-pr-199, oss-pr-185}, categorization, and the corresponding iterative fix procedure~\cite{oss-pr-204, oss-pr-198}.
%The implementation has lasts over one month.
% and includes the implementation of fuzzing runtime error identification~\cite{oss-pr-191, oss-pr-187, oss-pr-199, oss-pr-185} and the expansion of their iterative logic to rectify drivers with fuzzing errors~\cite{oss-pr-204, oss-pr-198}.
These enhancements added new functionalities refining the generation results, where the cases showing its effectiveness are quickly identified~\cite{oss-pr-effective-case, oss-pr-fp-filter-case} during their benchmark tests (29 APIs, 18 projects).
Currently, the improvement is merged into the main branch and is actively used to fuzz all 282 supported projects, marking a significant milestone to us.
We are keeping refine our commitments, such as integrating more fine-grained error information during fix. 
%\compactline
}

\revision{
\noindent
\textbf{Potential Improvements}
\tab
% domain knowledge understanding
%  - agent, rag
%  - combination with traditional methods
% model fundamental ability
From our perspective, to improve the performance of LLM-based fuzz driver generation, efforts from three dimensions can be further explored.
First, the domain knowledge contained inside the target scope can be modeled and utilized for better generation.
For instance, to test network protocol APIs, the communication state machine of that protocol can be learned first and then used to guide the driver generation.
Besides, more sophisticated prompt-based solutions can be explored, such as hybrid approaches combining traditional program analysis and prompt strategies, or agent-based approaches.
Lastly, fine-tuning based methods is also a promising direction since this can enhance both the generation's effectiveness and efficiency from a model level.
%\compactline
}

% However, they do not provide a systematic analysis on the fundamental issues of LLM-based fuzz driver generation, such as the effectiveness on different model settings or parameters.
% Different from them, we aim to provide a comprehensive understanding on the fundamental issues of LLM-based fuzz driver generation.  

%\noindent
%\textbf{Evaluating More LLMs}
%\tab
%Previous sections have analyzed the results based on two state-of-the-art OpenAI LLMs.
%They are selected due to their superior performance comparing to other LLMs.
%Besides, we also evaluated Google Palm2 \texttt{text-bison-001}~\cite{anil2023palm}.
%% , CodeGen 1 (2b/8b/16b)~\cite{xxx}, and Codex (code-cushman-002/code-davinci-002)~\cite{xxx}.
%% The latter two series are significantly worse than gpt-3.5-turbo-0314 and 
%On our quiz questions, Palm2 has shown a close performance with \texttt{gpt-3.5-turbo\-0301} but still performed observably worse than it.
%And the main conclusions draw from Palm2 data is consistent with the above presented.
%We are continuously evaluating more code related LLMs such as CodeGen 2~\cite{nijkamp2023codegen2}, Llama 2~\cite{touvron2023llama}, etc.
%All these additional results are updated at~\cite{fuzz-drvier-study-website}.

% \noindent
% \textbf{Practical Application}
% \tab
% While the evaluations show high question success rates for LLM-based generation approaches, this does not necessarily mean they are ready to be applied at scale.
% The main challenge lies in identifying effective fuzz drivers from the large number of generated ones.
% This is difficult since precise identification needs correct classification of false positives (bugs caused by the driver) and negatives (ineffective usage), which require semantic understanding of API usage.
% Future directions for addressing this challenge include developing advanced automatic approaches and balancing automation with human interventions.
% Root cause analysis methods and project-specific domain knowledge can be useful in building automatic approaches.
% Human-assisted approaches require a well balance between automation requirements and manual efforts.
% Recent examples of this approach include Github Copilot~\cite{copilot}.
% Adopting or improving on this approach to support security experts could be a meaningful and interesting direction.

%
% the validation problem, spam issue 
%
% domain knowledge, human intervention, 
%  iterative collection or refinement
%

% \noindent
% \textbf{Improvements on Query Strategies}
% \tab
% % \textbf{Unsolved Questions}
% % Written with high level planning
% % Missing Contextual Knowledge
% % \textbf{Further Improvements: More Languages, Better Refinements, Fine-tuned Model, ...}
% % The query strategies evaluated can be further improved in two directions:
% % towards generating fuzz drivers
% Besides the directions discussed in Section~\ref{sec:rq3}, the contextual issue reflected by these high score questions which cannot be solved by any strategy is also an interesting direction.
% These questions failed since their driver generation requires the understanding of specific contexts.
% For instance, generating the driver for \texttt{tmux}~\cite{tmux-ossfuzz-driver-link} requires the understanding of various concepts, such as session, window, pane, etc, and their relationships.
% Similarly, for network-related questions~\cite{libmodbus-ossfuzz-driver-link, civetweb-ossfuzz-driver-link}, a standby network server or client is required to be created before calling the target API.
% The effective drivers can only be generated by respecting these specific contextual requirements.
% % Ignoring these contextual requirements can never 
% % Thus, 
% One future direction can be exploring advanced methods for these contextual challenges.
% % Besides, the direction discussed in 

% \zhc{xx} questions are still cannot be answered by any strategy in evaluation.
% The common features of these questions indicates that it is  to correctly generate the fuzz driver without a global understanding of the context requirement.
% For example, for \texttt{tmux} driver~\cite{tmux-ossfuzz-driver-link}, it requires the understanding of general concepts in \texttt{tmux} programs like the relationships between session, window, pane, etc.
% For the questions related with fuzzing network functionalities~\cite{libmodbus-ossfuzz-driver-link, civetweb-ossfuzz-driver-link}, they require the creation of a standby network server or client before calling the target API.
% Without a global understanding and planning, these contextual requirements will be ignored by the models, resulting in failed generation of effective drivers.
% More advanced strategies can be explored on solving these questions.

% Extending fuzz drivers?
% Oracles

\noindent
\textbf{Threat to Validity}
\tab
% Also discuss clone detection here?
One internal threat comes from the effectiveness validation of the generated drivers.
% There are more than 35,000 fuzz drivers have been validated based on our effectiveness criteria.
To address this, we carefully examined the APIs and manually wrote tests for them to check whether the semantic constraints of a specific API have been satisfied or not.
Another threat to validity comes from the fact that some OSS-Fuzz drivers, \textit{e.g.}, code written before Sep 2021, may already be contained in the model training data, which raises a question that whether the driver is directly memorized by the model from the training data.
Though it is infeasible to thoroughly prove its generation ability, which requires the retrain of LLMs, we found several evidences that supports the answers provided by these models are not memorized:
Many generated drivers contain APIs that do not appear in the OSS-Fuzz drivers, especially for those drivers hinted by example usage snippets or iteratively fixed by usage and error information.
Besides, the generated drivers share a distinct coding style as OSS-Fuzz drivers.
For example, the generated code are commented with explanation on why the API is used and what it is used for, etc.
The main external threat to validity comes from our evaluation datasets.
Our study focused on C projects while the insights may not be necessarily generalizable to other languages.
\compactline

% First, many drivers contain comments for the usage of each line of code, which does not exist in the manually written driver. 
% Second, not only the coding patterns between two kinds of drivers are different, but also the generated drivers can contain APIs that do not appear in the OSS-Fuzz drivers.

% \noindent
% \textbf{Pros and Cons with SOTA Methods}
% \tab

\section{Related Work}

\noindent
\textbf{Fuzz Driver Generation.} \tab 
Several works~\cite{fuzzgen, fudge, apicraft, winnie, intelligen, rubick, utopia, chen2023hopper} have focused on developing automatic approaches to generate fuzz drivers.
Most of these works follow a common methodology, which involves generating fuzz drivers based on the API usage existed in consumer programs, \textit{i.e.,} programs containing code that uses these APIs.
For instance, by abstracting the API usage as specific models such as trees~\cite{apicraft}, graphs~\cite{fuzzgen}, and automatons~\cite{rubick}, several works propose program analysis-based methods to learn the usage models from consumer programs and conduct model-based driver synthesis.
In addition, a recent work~\cite{utopia} emphasizes that unit tests are high quality consumer programs and proposes techniques to convert existing unit tests to fuzz drivers.
Though these approaches can produce effective fuzz drivers, their heavy requirements on the quality of the consumer programs, \textit{i.e.,} the consumers must contain complete API usage and are statically/dynamically analyzable, limit their generality.
Furthermore, synthesized code often lacks human readability and maintainability, limiting their practical application.
\revision{
Some parallel works~\cite{oss-fuzz-gen,prompt-fuzz} have also explored the LLM-based fuzz driver generation.
However, their main goal is to build tools demonstrating the potential of LLM-based generation.
Our study complements them by focusing on delivering the first comprehensive understanding of the fundamental issues in this direction.
}
\compactline

% utopia, rubick, apicraft, ...

\noindent
\textbf{LLM for Generative Tasks.} \tab 
Recent works have explored the potential of LLM models for various generative tasks, such as code completion~\cite{wei2023magicoder}, test case generation~\cite{guifill, zhanglingming-llm-are-zero-shot-fuzzers, yang2023white, deng2023large, schafer2023adaptive, siddiq2023exploring, xia2023universal} and code repairing~\cite{sp-repair, abhik-repair, xia2023keep}.
These works utilize the natural language processing capabilities of LLM models and employ specific prompt designs to achieve their respective tasks.
To further improve the models' performance, some works incorporate iterative/conversational strategies or use fine-tuning/in-context learning techniques.
In test case generation, previous research works have primarily targeted on testing deep learning libraries~\cite{zhanglingming-llm-are-zero-shot-fuzzers, deng2023large} and unit test generation~\cite{schafer2023adaptive, siddiq2023exploring}. 
Considering the intrinsic differences between fuzz drivers and other tests and the difference on studied programming languages, these works cannot answer the fundamental effectiveness issues of LLM-based fuzz driver generation, indicating the unique values of our study.
% On one hand, unit test has distinct robustness requirements of the API usage compared with fuzz driver.
% on testing specific APIs, they are different in the underlying design logic.
% The former focuses on validating the program's outputs (mainly based on \texttt{assert} statements) under certain given inputs while fuzz driver is a different scenario where the code has to work properly for any randomly mutated input.
% This further incurs the distinct robustness requirements on the API usages.
% On the other hand, existing works mainly target on memory-safe languages while our study focuses on the C language.
% these works evaluate on text generation or code generation in memory-safe languages while our study targets on C languages.
% On the other hand, they 
% cannot directly answer the fundamental issues about the effectiveness on applying LLM models for fuzz driver generation.
% there is a significant lack of research on the effectiveness of LLM models for fuzz test generation in C projects. 
% Given the importance of generating fuzz driver for C project, this represents a critical area for further investigation.


% unit test generation
% evaluate copilot
% code repair
% ...

\vspace{-2pt}
\section{Conclusion}
\revision{
Our study centers around fundamental issues of LLM-based fuzz driver generation's effectiveness.
% , including the effectiveness, the basic challenges, and the pros and cons of several designed query strategies.
To do that, we designed six prompt strategies, extensively evaluated them on different models and temperatures.
% built a framework for evaluation in scale, and compared generated drivers with industrial used ones.
% to study in scale.
The insights are three-fold:
\ding{182} LLM-based generation has promising potential but also faces challenges towards high practicality;  
% the results demonstrated the promising practicality of LLM-based generation as well as the challenges towards high practicality;
\ding{183} the fundamental challenge roots in tackling the API-specific usage particulars while three key beneficial designs for prompting are identified and analyzed;
\ding{184} the LLM-generated drivers can provide comparable fuzzing outcomes as the industrial used ones.
However, larges spaces for further improvements still exist.
Our insights have been applied in industrial practical fuzz driver generation.
}
\compactline
% there are still large space for further improvements, such as automatic semantic correctness validation, API usage extension, and semantic oracle generation.

% and there are three key designs can help significantly: repeatedly querying, querying with examples, and iteratively querying;
% Besides, three key designs significantly help here: 
% revealing their potential for facilitating fuzzing of API targets.
% However, practical application of LLM-generated drivers requires careful filtering to ensure effectiveness and avoid false alarms.
% We have also analyzed the effects, advantages, and disadvantages of five different types of strategies.
% Comparing the generated drivers with manually written ones, we found that LLM-generated drivers can produce competent fuzzing outcomes, but there is still room for further improvement.

\vspace{-2pt}
\section{Data Availability}

% To facilitate the future research, we have released all the code and data involved in our study~\cite{fuzz-drvier-study-website}.
The source code and data involved in our study can be found at~\cite{fuzz-drvier-study-website}.
% to facilitate the community.
% To facilitate the community, we will open source our quiz, evaluation framework, and the data involved in the study at~\cite{fuzz-drvier-study-website}.

% \clearpage

% \section{Details of Proposed Fix Prompt Templates}
\label{sec:fix-templates-detail}
Table~\ref{tab:fix-templates-detail} lists the detailed design of seven prompt templates used for fixing seven types of failures.
Concrete examples can be found in ~\cite{fuzz-drvier-study-website}.

\section{Detail of Quiz Questions}
\label{sec:quiz-questions-detail}
Table~\ref{tab:quiz} shows the full list of the 86 questions included in our quiz.

\section{Full List of Evaluation Results for All Query Strategies}
\label{sec:evaluation-results-full-list}
Table~\ref{tab:eval_full} posts the detailed statistics of evaluation results.

\onecolumn

\begin{table*}[h]
\caption{Detail of Different Fix Prompt Templates.}
\label{tab:fix-templates-detail}
\resizebox{0.9\textwidth}{!}{
\begin{tabular}{ll}
\toprule
Template Name & Template Content\\
\midrule

\rowcolor{black!10}

\begin{tabular}[t]{l}
\multirow{10}{*}{\texttt{FIX\_PRSE\_ERR}} \\
\end{tabular}
&
\begin{tabular}[t]{l}
\Verb|```| \\
\Verb|${DRIVER_CODE}| \\
\Verb|```| \\
\Verb|The above C code has compilation error.| \\
\Verb|The error line is:| \\
\Verb|${ERR_LINE_CODE}| \\
\Verb|The error description is:| \\
\Verb|${ERR_DESCRIPTION}| \\
\Verb|${SUPPLEMENTAL_INFO}| \\
\Verb|Based on the above information, fix the code.| \\
\end{tabular}
\\
\midrule

\begin{tabular}[t]{l}
\multirow{6}{*}{\texttt{FIX\_LINK\_ERR}} \\
\end{tabular}
&
\begin{tabular}[t]{l}
\Verb|```| \\
\Verb|${DRIVER_CODE}| \\
\Verb|```| \\
\Verb|The above C code calls a non-existing API ${API_NAME}.| \\
\Verb|${SUPPLEMENTAL_INFO}| \\
\Verb|Based on the above information, fix the code.| \\
\end{tabular}
\\
\midrule

\rowcolor{black!10}

\begin{tabular}[t]{l}
\multirow{6}{*}{\texttt{FIX\_FUZZ\_MEMLEAK}} \\
\end{tabular}
&
\begin{tabular}[t]{l}
\Verb|```| \\
\Verb|${DRIVER_CODE}| \\
\Verb|```| \\
\Verb|The above C code can be built successfully but has runtime memory leak.| \\
\Verb|${SUPPLEMENTAL_INFO}| \\
\Verb|Based on the above information, fix the code.| \\
\end{tabular}
\\
\midrule

\begin{tabular}[t]{l}
\multirow{6}{*}{\texttt{FIX\_FUZZ\_OOM}} \\
\end{tabular}
&
\begin{tabular}[t]{l}
\Verb|```| \\
\Verb|${DRIVER_CODE}| \\
\Verb|```| \\
\Verb|The above C code can be built successfully but meet out-of-memory, perhaps due to memory leak.| \\
\Verb|${SUPPLEMENTAL_INFO}| \\
\Verb|Based on the above information, fix the code.| \\
\end{tabular}
\\
\midrule

\rowcolor{black!10}

\begin{tabular}[t]{l}
\multirow{10}{*}{\texttt{FIX\_FUZZ\_TIMEOUT}} \\
\end{tabular}
&
\begin{tabular}[t]{l}
\Verb|```| \\
\Verb|${DRIVER_CODE}| \\
\Verb|```| \\
\Verb|The above C code can be built successfully but will stuck (timeout).| \\
\Verb|The possible stuck line is:| \\
\Verb|${ERR_LINE_CODE}| \\
\Verb|The frames of the stack are:| \\
\Verb|${ERR_STACK}| \\
\Verb|${SUPPLEMENTAL_INFO}| \\
\Verb|Based on the above information, fix the code.| \\
\end{tabular}
\\
\midrule

\begin{tabular}[t]{l}
\multirow{10}{*}{\texttt{FIX\_FUZZ\_CRASH}} \\
\end{tabular}
&
\begin{tabular}[t]{l}
\Verb|```| \\
\Verb|${DRIVER_CODE}| \\
\Verb|```| \\
\Verb|The above C code can be built successfully but will crash (${CRASH_SYMPTOM}).| \\
\Verb|The crash line is:| \\
\Verb|${ERR_LINE_CODE}| \\
\Verb|The frames of the stack are:| \\
\Verb|${ERR_DESCRIPTION}| \\
\Verb|${SUPPLEMENTAL_INFO}| \\
\Verb|Based on the above information, fix the code.| \\
\end{tabular}
\\
\midrule

\rowcolor{black!10}

\begin{tabular}[t]{l}
\multirow{6}{*}{\texttt{FIX\_FUZZ\_NONEFF}} \\
\end{tabular}
&
\begin{tabular}[t]{l}
\Verb|```| \\
\Verb|${DRIVER_CODE}| \\
\Verb|```| \\
\Verb|The above C code can be built successfully but its fuzzing seems not effective since the \ |\\
\Verb|  coverage never change.| \\
\Verb|Based on the above information, fix the code if necessary.| \\
\end{tabular}
\\
\bottomrule
\end{tabular}
}
\end{table*}

\clearpage

\onecolumn

\renewcommand\arraystretch{1.1}

%\begin{table*}
\begin{xltabular}[h]{\textwidth}{ccccc}
%\begin{tabular}{ccccc}
\caption{Detail of Quiz Questions. \label{tab:quiz}}\\
\toprule
Index & Project & API & Score & Drivers \\
  \hline
  \endfirsthead
  \multicolumn{5}{c}{Continued from previous page} \\
  \hline
Index & Project & API & Score & Drivers \\
  \hline
  \endhead
  \hline
  \multicolumn{5}{c}{Continued on next page} \\
  \endfoot
  \hline
  \endlastfoot
\cellcolor{black!10}\# 1 & \cellcolor{black!10}coturn & \cellcolor{black!10}stun\_is\_command\_message\_full\_check\_str & \cellcolor{black!10}1 & \cellcolor{black!10}FuzzStun.c \\
\# 2 & kamailio & parse\_uri & 1 & fuzz\_uri.c \\
\cellcolor{black!10}\# 3 & \cellcolor{black!10}coturn & \cellcolor{black!10}stun\_check\_message\_integrity\_str & \cellcolor{black!10}2 & \cellcolor{black!10}FuzzStun.c \\
\# 4 & libiec61850 & MmsValue\_decodeMmsData & 2 & fuzz\_mms\_decode.c \\
\cellcolor{black!10}\# 5 & \cellcolor{black!10}md4c & \cellcolor{black!10}md\_html & \cellcolor{black!10}2 & \cellcolor{black!10}fuzz-mdhtml.c \\
\# 6 & spdk & spdk\_json\_parse & 2 & parse\_json\_fuzzer.cc \\
\cellcolor{black!10}\# 7 & \cellcolor{black!10}croaring & \cellcolor{black!10}roaring\_bitmap\_portable\_deserialize\_safe & \cellcolor{black!10}3 & \cellcolor{black!10}croaring\_fuzzer.c \\
\# 8 & lua & luaL\_loadbufferx & 3 & fuzz\_lua.c \\
\cellcolor{black!10}\# 9 & \cellcolor{black!10}w3m & \cellcolor{black!10}wc\_Str\_conv\_with\_detect & \cellcolor{black!10}3 & \cellcolor{black!10}fuzz-conv.c \\
\# 10 & bind9 & dns\_name\_fromwire & 4 & dns\_name\_fromwire.c \\
\cellcolor{black!10}\# 11 & \cellcolor{black!10}gdk-pixbuf & \cellcolor{black!10}gdk\_pixbuf\_animation\_new\_from\_file & \cellcolor{black!10}4 & \cellcolor{black!10}animation\_fuzzer.c \\
\# 12 & gdk-pixbuf & gdk\_pixbuf\_new\_from\_data & 4 & pixbuf\_cons\_fuzzer.c \\
\cellcolor{black!10}\# 13 & \cellcolor{black!10}gdk-pixbuf & \cellcolor{black!10}gdk\_pixbuf\_new\_from\_file & \cellcolor{black!10}4 & \cellcolor{black!10}pixbuf\_file\_fuzzer.c \\
\# 14 & gdk-pixbuf & gdk\_pixbuf\_new\_from\_stream & 4 & stream\_fuzzer.c \\
\cellcolor{black!10}\# 15 & \cellcolor{black!10}gpac & \cellcolor{black!10}gf\_isom\_open\_file & \cellcolor{black!10}4 & \cellcolor{black!10}fuzz\_parse.c \\
\# 16 & libbpf & bpf\_object\_\_open\_mem & 4 & bpf-object-fuzzer.c \\
\cellcolor{black!10}\# 17 & \cellcolor{black!10}libpg\_query & \cellcolor{black!10}pg\_query\_parse & \cellcolor{black!10}4 & \cellcolor{black!10}fuzz\_parser.c \\
\# 18 & libucl & ucl\_parser\_add\_string & 4 & ucl\_add\_string\_fuzzer.c \\
\cellcolor{black!10}\# 19 & \cellcolor{black!10}oniguruma & \cellcolor{black!10}onig\_new & \cellcolor{black!10}4 & \cellcolor{black!10}base.c \\
\# 20 & pupnp & ixmlLoadDocumentEx & 4 & FuzzIxml.c \\
\cellcolor{black!10}\# 21 & \cellcolor{black!10}gdk-pixbuf & \cellcolor{black!10}gdk\_pixbuf\_new\_from\_file\_at\_scale & \cellcolor{black!10}5 & \cellcolor{black!10}pixbuf\_scale\_fuzzer.c \\
\# 22 & inchi & GetINCHIKeyFromINCHI & 5 & inchi\_input\_fuzzer.c \\
\cellcolor{black!10}\# 23 & \cellcolor{black!10}libdwarf & \cellcolor{black!10}dwarf\_init\_b & \cellcolor{black!10}5 & \cellcolor{black!10}fuzz\_init\_binary.c \\
\# 24 & libdwarf & dwarf\_init\_path & 5 & fuzz\_init\_path.c \\
\cellcolor{black!10}\# 25 & \cellcolor{black!10}liblouis & \cellcolor{black!10}lou\_compileString & \cellcolor{black!10}5 & \cellcolor{black!10}table\_fuzzer.cc \\
\# 26 & selinux & cil\_compile & 5 & secilc-fuzzer.c \\
\cellcolor{black!10}\# 27 & \cellcolor{black!10}bind9 & \cellcolor{black!10}dns\_name\_fromtext & \cellcolor{black!10}6 & \cellcolor{black!10}dns\_name\_fromtext\_target.c \\
\# 28 & bind9 & dns\_rdata\_fromwire & 6 & dns\_rdata\_fromwire\_text.c \\
\cellcolor{black!10}\# 29 & \cellcolor{black!10}coturn & \cellcolor{black!10}stun\_is\_binding\_response & \cellcolor{black!10}6 & \cellcolor{black!10}FuzzStunClient.c \\
\# 30 & coturn & stun\_is\_command\_message & 6 & FuzzStunClient.c \\
\cellcolor{black!10}\# 31 & \cellcolor{black!10}coturn & \cellcolor{black!10}stun\_is\_response & \cellcolor{black!10}6 & \cellcolor{black!10}FuzzStunClient.c \\
\# 32 & coturn & stun\_is\_success\_response & 6 & FuzzStunClient.c \\
\cellcolor{black!10}\# 33 & \cellcolor{black!10}hiredis & \cellcolor{black!10}redisFormatCommand & \cellcolor{black!10}6 & \cellcolor{black!10}format\_command\_fuzzer.c \\
\# 34 & igraph & igraph\_read\_graph\_dl & 6 & read\_dl\_fuzzer.cpp \\
\cellcolor{black!10}\# 35 & \cellcolor{black!10}igraph & \cellcolor{black!10}igraph\_read\_graph\_edgelist & \cellcolor{black!10}6 & \cellcolor{black!10}read\_edgelist\_fuzzer.cpp \\
\# 36 & igraph & igraph\_read\_graph\_gml & 6 & read\_gml\_fuzzer.cpp \\
\cellcolor{black!10}\# 37 & \cellcolor{black!10}igraph & \cellcolor{black!10}igraph\_read\_graph\_graphdb & \cellcolor{black!10}6 & \cellcolor{black!10}read\_graphdb\_fuzzer.cpp \\
\# 38 & igraph & igraph\_read\_graph\_graphml & 6 & read\_graphml\_fuzzer.cpp \\
\cellcolor{black!10}\# 39 & \cellcolor{black!10}igraph & \cellcolor{black!10}igraph\_read\_graph\_lgl & \cellcolor{black!10}6 & \cellcolor{black!10}read\_lgl\_fuzzer.cpp \\
\# 40 & igraph & igraph\_read\_graph\_pajek & 6 & read\_pajek\_fuzzer.cpp \\
\cellcolor{black!10}\# 41 & \cellcolor{black!10}inchi & \cellcolor{black!10}GetINCHIfromINCHI & \cellcolor{black!10}6 & \cellcolor{black!10}inchi\_input\_fuzzer.c \\
\# 42 & inchi & GetStructFromINCHI & 6 & inchi\_input\_fuzzer.c \\
\cellcolor{black!10}\# 43 & \cellcolor{black!10}kamailio & \cellcolor{black!10}parse\_msg & \cellcolor{black!10}6 & \cellcolor{black!10}fuzz\_parse\_msg.c \\
\# 44 & libyang & lys\_parse\_mem & 6 & lys\_parse\_mem.c \\
\cellcolor{black!10}\# 45 & \cellcolor{black!10}proftpd & \cellcolor{black!10}pr\_json\_object\_from\_text & \cellcolor{black!10}6 & \cellcolor{black!10}fuzzer.c \\
\# 46 & selinux & policydb\_read & 6 & binpolicy-fuzzer.c \\
\cellcolor{black!10}\# 47 & \cellcolor{black!10}kamailio & \cellcolor{black!10}get\_src\_address\_socket & \cellcolor{black!10}7 & \cellcolor{black!10}fuzz\_parse\_msg.c \\
\# 48 & kamailio & get\_src\_uri & 7 & fuzz\_parse\_msg.c \\
\cellcolor{black!10}\# 49 & \cellcolor{black!10}kamailio & \cellcolor{black!10}parse\_content\_disposition & \cellcolor{black!10}7 & \cellcolor{black!10}fuzz\_parse\_msg.c \\
\# 50 & kamailio & parse\_diversion\_header & 7 & fuzz\_parse\_msg.c \\
\cellcolor{black!10}\# 51 & \cellcolor{black!10}kamailio & \cellcolor{black!10}parse\_from\_header & \cellcolor{black!10}7 & \cellcolor{black!10}fuzz\_parse\_msg.c \\
\# 52 & kamailio & parse\_from\_uri & 7 & fuzz\_parse\_msg.c \\
\cellcolor{black!10}\# 53 & \cellcolor{black!10}kamailio & \cellcolor{black!10}parse\_headers & \cellcolor{black!10}7 & \cellcolor{black!10}fuzz\_parse\_msg.c \\
\# 54 & kamailio & parse\_identityinfo\_header & 7 & fuzz\_parse\_msg.c \\
\cellcolor{black!10}\# 55 & \cellcolor{black!10}kamailio & \cellcolor{black!10}parse\_pai\_header & \cellcolor{black!10}7 & \cellcolor{black!10}fuzz\_parse\_msg.c \\
\# 56 & kamailio & parse\_privacy & 7 & fuzz\_parse\_msg.c \\
\cellcolor{black!10}\# 57 & \cellcolor{black!10}kamailio & \cellcolor{black!10}parse\_record\_route\_headers & \cellcolor{black!10}7 & \cellcolor{black!10}fuzz\_parse\_msg.c \\
\# 58 & kamailio & parse\_refer\_to\_header & 7 & fuzz\_parse\_msg.c \\
\cellcolor{black!10}\# 59 & \cellcolor{black!10}kamailio & \cellcolor{black!10}parse\_route\_headers & \cellcolor{black!10}7 & \cellcolor{black!10}fuzz\_parse\_msg.c \\
\# 60 & kamailio & parse\_to\_header & 7 & fuzz\_parse\_msg.c \\
\cellcolor{black!10}\# 61 & \cellcolor{black!10}kamailio & \cellcolor{black!10}parse\_to\_uri & \cellcolor{black!10}7 & \cellcolor{black!10}fuzz\_parse\_msg.c \\
\# 62 & libyang & lyd\_parse\_data\_mem & 7 & \tabincell{c}{lyd\_parse\_mem\_json.c \\ lyd\_parse\_mem\_xml.c} \\
\cellcolor{black!10}\# 63 & \cellcolor{black!10}bind9 & \cellcolor{black!10}dns\_message\_parse & \cellcolor{black!10}8 & \cellcolor{black!10}dns\_message\_parse.c \\
\# 64 & igraph & igraph\_read\_graph\_ncol & 8 & read\_ncol\_fuzzer.cpp \\
\cellcolor{black!10}\# 65 & \cellcolor{black!10}pjsip & \cellcolor{black!10}pj\_json\_parse & \cellcolor{black!10}8 & \cellcolor{black!10}fuzz-json.c \\
\# 66 & pjsip & pj\_xml\_parse & 8 & fuzz-xml.c \\
\cellcolor{black!10}\# 67 & \cellcolor{black!10}pjsip & \cellcolor{black!10}pjmedia\_sdp\_parse & \cellcolor{black!10}8 & \cellcolor{black!10}fuzz-sdp.c \\
\# 68 & quickjs & lre\_compile & 8 & fuzz\_regexp.c \\
\cellcolor{black!10}\# 69 & \cellcolor{black!10}bind9 & \cellcolor{black!10}isc\_lex\_getmastertoken & \cellcolor{black!10}9 & \cellcolor{black!10}isc\_lex\_getmastertoken.c \\
\# 70 & bind9 & isc\_lex\_gettoken & 9 & isc\_lex\_gettoken.c \\
\cellcolor{black!10}\# 71 & \cellcolor{black!10}quickjs & \cellcolor{black!10}JS\_Eval & \cellcolor{black!10}9 & \cellcolor{black!10}fuzz\_eval.c, fuzz\_compile.c \\
\# 72 & igraph & igraph\_edge\_connectivity & 10 & edge\_connectivity\_fuzzer.cpp \\
\cellcolor{black!10}\# 73 & \cellcolor{black!10}pjsip & \cellcolor{black!10}pj\_stun\_msg\_decode & \cellcolor{black!10}10 & \cellcolor{black!10}fuzz-stun.c \\
\# 74 & bind9 & dns\_message\_checksig & 11 & dns\_message\_checksig.c \\
\cellcolor{black!10}\# 75 & \cellcolor{black!10}libzip & \cellcolor{black!10}zip\_fread & \cellcolor{black!10}11 & \cellcolor{black!10}zip\_read\_fuzzer.cc \\
\# 76 & bind9 & dns\_rdata\_fromtext & 12 & \tabincell{c}{dns\_rdata\_fromwire\_text.c \\ dns\_rdata\_fromtext.c}  \\
\cellcolor{black!10}\# 77 & \cellcolor{black!10}igraph & \cellcolor{black!10}igraph\_all\_minimal\_st\_separators & \cellcolor{black!10}12 & \cellcolor{black!10}vertex\_separators\_fuzzer.cpp \\
\# 78 & igraph & igraph\_minimum\_size\_separators & 12 & vertex\_separators\_fuzzer.cpp \\
\cellcolor{black!10}\# 79 & \cellcolor{black!10}pjsip & \cellcolor{black!10}pjsip\_parse\_msg & \cellcolor{black!10}12 & \cellcolor{black!10}fuzz-sip.c \\
\# 80 & igraph & igraph\_automorphism\_group & 13 & bliss\_fuzzer.cpp \\
\cellcolor{black!10}\# 81 & \cellcolor{black!10}libmodbus & \cellcolor{black!10}modbus\_read\_bits & \cellcolor{black!10}15 & \cellcolor{black!10}FuzzClient.c \\
\# 82 & libmodbus & modbus\_read\_registers & 15 & FuzzClient.c \\
\cellcolor{black!10}\# 83 & \cellcolor{black!10}civetweb & \cellcolor{black!10}mg\_get\_response & \cellcolor{black!10}17 & \cellcolor{black!10}fuzzmain.c \\
\# 84 & bind9 & dns\_master\_loadbuffer & 20 & dns\_master\_load.c \\
\cellcolor{black!10}\# 85 & \cellcolor{black!10}libmodbus & \cellcolor{black!10}modbus\_receive & \cellcolor{black!10}33 & \cellcolor{black!10}FuzzServer.c \\
\# 86 & tmux & input\_parse\_buffer & 42 & input-fuzzer.c \\
\bottomrule
\end{xltabular}
%\end{tabular}}
%\end{table*}
\twocolumn

% \section{Detail of Score Calculation}
% \label{sec:score-calculation-detail}

\begin{table*}
\caption{Full List of Evaluation Results}
\label{tab:eval_full}
\renewcommand\arraystretch{1}
\scriptsize
\setlength\arrayrulewidth{2pt}\arrayrulecolor{black}
\resizebox{\textwidth}{!}{
\begin{tabular}{cp{3.6cm} 
        >{\columncolor[gray]{1}[0.8\tabcolsep]}c
        >{\columncolor[gray]{1}[0.8\tabcolsep]}c
        >{\columncolor[gray]{1}[0.8\tabcolsep]}c
        >{\columncolor[gray]{1}[0.8\tabcolsep]}c
        >{\columncolor[gray]{1}[0.8\tabcolsep]}c
        >{\columncolor[gray]{1}[0.8\tabcolsep]}c
        >{\columncolor[gray]{1}[0.8\tabcolsep]}c
        >{\columncolor[gray]{1}[0.8\tabcolsep]}c
        >{\columncolor[gray]{1}[0.8\tabcolsep]}c
        >{\columncolor[gray]{1}[0.8\tabcolsep]}c
        >{\columncolor[gray]{1}[0.8\tabcolsep]}c
        >{\columncolor[gray]{1}[0.8\tabcolsep]}c
        >{\columncolor[gray]{1}[0.8\tabcolsep]}c
    }


\toprule
% Question & SCORE & gpt4-NAIVE & gpt4-BACTX & gpt4-UGCTX & gpt4-DOCTX & gpt4-BA-ITER & gpt4-ALL-ITER & gpt3.5-NAIVE & gpt3.5-BACTX & gpt3.5-UGCTX & gpt3.5-DOCTX & gpt3.5-BA-ITER & gpt3.5-ALL-ITER \\
 \multirow{2}{*}{Index} & \multirow{2}{*}{Question} & \multirow{2}{*}{Score} &   \multicolumn{6}{c}{GPT4} &  \multicolumn{6}{c}{GPT3.5} \\
&  & & NAIVE & BACTX & UGCTX & DOCTX & BA-ITER & EX-ITER &  NAIVE &  BACTX &  UGCTX &  DOCTX &  BA-ITER & EX-ITER \\
\midrule
\cellcolor{black!10}\# 1 & \cellcolor{black!10}\url{coturn_stun_is_command_message_full_check_str} & \cellcolor{black!10}1 & 0/40 & \cellcolor{green!72.50}29/40 & 0/40 & - & - & - & 0/40 & \cellcolor{green!67.50}27/40 & \cellcolor{green!12.50}5/40 & - & - & - \\
\# 2 & \url{kamailio_parse_uri} & 1 & 0/40 & \cellcolor{green!100.00}40/40 & \cellcolor{green!55.00}22/40 & - & - & - & 0/40 & \cellcolor{green!85.00}34/40 & \cellcolor{green!52.50}21/40 & - & - & - \\
\cellcolor{black!10}\# 3 & \cellcolor{black!10}\url{coturn_stun_check_message_integrity_str} & \cellcolor{black!10}2 & 0/40 & \cellcolor{green!30.00}12/40 & \cellcolor{green!20.00}8/40 & - & - & - & 0/40 & \cellcolor{green!20.00}8/40 & \cellcolor{green!10}2/40 & - & - & - \\
\# 4 & \url{libiec61850_MmsValue_decodeMmsData} & 2 & 0/40 & \cellcolor{green!97.50}39/40 & \cellcolor{green!27.50}11/40 & \cellcolor{green!92.50}37/40 & - & - & 0/40 & \cellcolor{green!92.50}37/40 & \cellcolor{green!30.00}12/40 & \cellcolor{green!67.50}27/40 & - & - \\
\cellcolor{black!10}\# 5 & \cellcolor{black!10}\url{md4c_md_html} & \cellcolor{black!10}2 & 0/40 & 0/40 & 0/40 & 0/40 & \cellcolor{green!50.00}20/40 & \cellcolor{green!17.65}12/68 & 0/40 & 0/40 & 0/40 & 0/40 & \cellcolor{green!26.23}16/61 & \cellcolor{green!10}2/113 \\
\# 6 & \url{spdk_spdk_json_parse} & 2 & 4/40 & \cellcolor{green!87.50}35/40 & \cellcolor{green!15.00}6/40 & - & - & - & 0/40 & \cellcolor{green!75.00}30/40 & \cellcolor{green!15.00}6/40 & - & - & - \\
\cellcolor{black!10}\# 7 & \cellcolor{black!10}\url{croaring_roaring_bitmap_portable_deserialize_safe} & \cellcolor{black!10}3 & \cellcolor{green!20.00}8/40 & \cellcolor{green!100.00}40/40 & \cellcolor{green!72.50}29/40 & \cellcolor{green!70.00}28/40 & - & - & \cellcolor{green!27.50}11/40 & \cellcolor{green!55.00}22/40 & \cellcolor{green!25.00}10/40 & \cellcolor{green!75.00}30/40 & - & - \\
\# 8 & \url{lua_luaL_loadbufferx} & 3 & \cellcolor{green!67.50}27/40 & \cellcolor{green!97.50}39/40 & \cellcolor{green!60.00}24/40 & \cellcolor{green!100.00}40/40 & - & - & \cellcolor{green!77.50}31/40 & \cellcolor{green!82.50}33/40 & \cellcolor{green!42.50}17/40 & \cellcolor{green!85.00}34/40 & - & - \\
\cellcolor{black!10}\# 9 & \cellcolor{black!10}\url{w3m_wc_Str_conv_with_detect} & \cellcolor{black!10}3 & 0/40 & 0/40 & \cellcolor{green!25.00}10/40 & - & - & - & 0/40 & 0/40 & \cellcolor{green!25.00}10/40 & - & - & - \\
\# 10 & \url{bind9_dns_name_fromwire} & 4 & 0/40 & \cellcolor{green!20.00}8/40 & \cellcolor{green!10}3/40 & - & - & - & 0/40 & 0/40 & \cellcolor{green!10}1/40 & - & - & - \\
\cellcolor{black!10}\# 11 & \cellcolor{black!10}\url{gdk-pixbuf_gdk_pixbuf_animation_new_from_file} & \cellcolor{black!10}4 & \cellcolor{green!15.00}6/40 & \cellcolor{green!82.50}33/40 & \cellcolor{green!37.50}15/40 & \cellcolor{green!67.50}27/40 & - & - & \cellcolor{green!10}3/40 & \cellcolor{green!27.50}11/40 & \cellcolor{green!10}3/40 & \cellcolor{green!22.50}9/40 & - & - \\
\# 12 & \url{gdk-pixbuf_gdk_pixbuf_new_from_data} & 4 & \cellcolor{green!10}2/40 & \cellcolor{green!40.00}16/40 & \cellcolor{green!27.50}11/40 & \cellcolor{green!27.50}11/40 & - & - & \cellcolor{green!45.00}18/40 & \cellcolor{green!62.50}25/40 & \cellcolor{green!15.00}6/40 & \cellcolor{green!57.50}23/40 & - & - \\
\cellcolor{black!10}\# 13 & \cellcolor{black!10}\url{gdk-pixbuf_gdk_pixbuf_new_from_file} & \cellcolor{black!10}4 & \cellcolor{green!25.00}10/40 & \cellcolor{green!97.50}39/40 & \cellcolor{green!50.00}20/40 & \cellcolor{green!92.50}37/40 & - & - & \cellcolor{green!12.50}5/40 & \cellcolor{green!35.00}14/40 & \cellcolor{green!12.50}5/40 & \cellcolor{green!20.00}8/40 & - & - \\
\# 14 & \url{gdk-pixbuf_gdk_pixbuf_new_from_stream} & 4 & \cellcolor{green!12.50}5/40 & \cellcolor{green!75.00}30/40 & \cellcolor{green!60.00}24/40 & \cellcolor{green!65.00}26/40 & - & - & \cellcolor{green!52.50}21/40 & \cellcolor{green!87.50}35/40 & \cellcolor{green!45.00}18/40 & \cellcolor{green!80.00}32/40 & - & - \\
\cellcolor{black!10}\# 15 & \cellcolor{black!10}\url{gpac_gf_isom_open_file} & \cellcolor{black!10}4 & \cellcolor{green!10}1/40 & \cellcolor{green!67.50}27/40 & \cellcolor{green!50.00}20/40 & - & - & - & 0/40 & \cellcolor{green!12.50}5/40 & 0/40 & - & - & - \\
\# 16 & \url{libbpf_bpf_object__open_mem} & 4 & \cellcolor{green!10}1/40 & \cellcolor{green!15.00}6/40 & 4/40 & \cellcolor{green!15.00}6/40 & - & - & 0/40 & \cellcolor{green!27.50}11/40 & \cellcolor{green!15.00}6/40 & \cellcolor{green!12.50}5/40 & - & - \\
\cellcolor{black!10}\# 17 & \cellcolor{black!10}\url{libpg_query_pg_query_parse} & \cellcolor{black!10}4 & 4/40 & \cellcolor{green!90.00}36/40 & \cellcolor{green!95.00}38/40 & - & - & - & \cellcolor{green!15.00}6/40 & \cellcolor{green!42.50}17/40 & \cellcolor{green!65.00}26/40 & - & - & - \\
\# 18 & \url{libucl_ucl_parser_add_string} & 4 & \cellcolor{green!20.00}8/40 & \cellcolor{green!47.50}19/40 & \cellcolor{green!50.00}20/40 & \cellcolor{green!72.50}29/40 & - & - & \cellcolor{green!17.50}7/40 & \cellcolor{green!20.00}8/40 & \cellcolor{green!12.50}5/40 & \cellcolor{green!45.00}18/40 & - & - \\
\cellcolor{black!10}\# 19 & \cellcolor{black!10}\url{oniguruma_onig_new} & \cellcolor{black!10}4 & \cellcolor{green!50.00}20/40 & \cellcolor{green!87.50}35/40 & \cellcolor{green!55.00}22/40 & \cellcolor{green!82.50}33/40 & - & - & \cellcolor{green!30.00}12/40 & \cellcolor{green!45.00}18/40 & \cellcolor{green!12.50}5/40 & \cellcolor{green!37.50}15/40 & - & - \\
\# 20 & \url{pupnp_ixmlLoadDocumentEx} & 4 & 0/40 & \cellcolor{green!65.00}26/40 & \cellcolor{green!75.00}15/20 & \cellcolor{green!40.00}16/40 & - & - & 0/40 & \cellcolor{green!17.50}7/40 & \cellcolor{green!20.00}4/20 & \cellcolor{green!10}1/40 & - & - \\
\cellcolor{black!10}\# 21 & \cellcolor{black!10}\url{gdk-pixbuf_gdk_pixbuf_new_from_file_at_scale} & \cellcolor{black!10}5 & \cellcolor{green!45.00}18/40 & \cellcolor{green!72.50}29/40 & \cellcolor{green!40.00}16/40 & \cellcolor{green!70.00}28/40 & - & - & \cellcolor{green!10}2/40 & \cellcolor{green!10}1/40 & \cellcolor{green!10}3/40 & \cellcolor{green!17.50}7/40 & - & - \\
\# 22 & \url{inchi_GetINCHIKeyFromINCHI} & 5 & 0/40 & \cellcolor{green!75.00}30/40 & \cellcolor{green!22.50}9/40 & \cellcolor{green!77.50}31/40 & - & - & 0/40 & \cellcolor{green!40.00}16/40 & \cellcolor{green!25.00}10/40 & \cellcolor{green!60.00}24/40 & - & - \\
\cellcolor{black!10}\# 23 & \cellcolor{black!10}\url{libdwarf_dwarf_init_b} & \cellcolor{black!10}5 & 0/40 & 0/40 & \cellcolor{green!32.50}13/40 & \cellcolor{green!10}1/40 & - & - & 0/40 & 0/40 & \cellcolor{green!27.50}11/40 & \cellcolor{green!12.50}5/40 & - & - \\
\# 24 & \url{libdwarf_dwarf_init_path} & 5 & 0/40 & 0/40 & \cellcolor{green!20.00}8/40 & 0/40 & - & - & 0/40 & 0/40 & 4/40 & 0/40 & - & - \\
\cellcolor{black!10}\# 25 & \cellcolor{black!10}\url{liblouis_lou_compileString} & \cellcolor{black!10}5 & 0/40 & \cellcolor{green!22.50}9/40 & \cellcolor{green!50.00}20/40 & \cellcolor{green!20.00}8/40 & - & - & 0/40 & \cellcolor{green!17.50}7/40 & \cellcolor{green!12.50}5/40 & \cellcolor{green!30.00}12/40 & - & - \\
\# 26 & \url{selinux_cil_compile} & 5 & 0/40 & 0/40 & \cellcolor{green!62.50}25/40 & - & - & - & 0/40 & 0/40 & \cellcolor{green!27.50}11/40 & - & - & - \\
\cellcolor{black!10}\# 27 & \cellcolor{black!10}\url{bind9_dns_name_fromtext} & \cellcolor{black!10}6 & 0/40 & \cellcolor{green!55.00}22/40 & \cellcolor{green!15.00}6/40 & - & - & - & 0/40 & 0/40 & \cellcolor{green!10}3/40 & - & - & - \\
\# 28 & \url{bind9_dns_rdata_fromwire} & 6 & 0/40 & 0/40 & \cellcolor{green!10}1/40 & - & - & - & 0/40 & 0/40 & \cellcolor{green!10}1/40 & - & - & - \\
\cellcolor{black!10}\# 29 & \cellcolor{black!10}\url{coturn_stun_is_binding_response} & \cellcolor{black!10}6 & 0/40 & \cellcolor{green!60.00}24/40 & \cellcolor{green!30.00}12/40 & - & - & - & 0/40 & 0/40 & \cellcolor{green!35.00}14/40 & - & - & - \\
\# 30 & \url{coturn_stun_is_command_message} & 6 & 0/40 & \cellcolor{green!30.00}12/40 & \cellcolor{green!37.50}15/40 & \cellcolor{green!27.50}11/40 & - & - & 0/40 & 0/40 & \cellcolor{green!42.50}17/40 & 0/40 & - & - \\
\cellcolor{black!10}\# 31 & \cellcolor{black!10}\url{coturn_stun_is_response} & \cellcolor{black!10}6 & 0/40 & \cellcolor{green!35.00}14/40 & 0/40 & - & - & - & 0/40 & 0/40 & \cellcolor{green!25.00}10/40 & - & - & - \\
\# 32 & \url{coturn_stun_is_success_response} & 6 & 0/40 & \cellcolor{green!40.00}16/40 & \cellcolor{green!37.50}15/40 & - & - & - & 0/40 & 0/40 & \cellcolor{green!15.00}6/40 & - & - & - \\
\cellcolor{black!10}\# 33 & \cellcolor{black!10}\url{hiredis_redisFormatCommand} & \cellcolor{black!10}6 & \cellcolor{green!32.50}13/40 & \cellcolor{green!100.00}40/40 & \cellcolor{green!22.50}9/40 & - & - & - & \cellcolor{green!10}2/40 & \cellcolor{green!77.50}31/40 & \cellcolor{green!20.00}8/40 & - & - & - \\
\# 34 & \url{igraph_igraph_read_graph_dl} & 6 & \cellcolor{green!22.50}9/40 & 0/40 & 4/40 & 0/40 & - & - & 0/40 & 0/40 & 0/40 & 0/40 & - & - \\
\cellcolor{black!10}\# 35 & \cellcolor{black!10}\url{igraph_igraph_read_graph_edgelist} & \cellcolor{black!10}6 & \cellcolor{green!15.00}6/40 & 0/40 & \cellcolor{green!10}2/40 & \cellcolor{green!10}1/40 & - & - & 0/40 & 0/40 & 0/40 & 0/40 & - & - \\
\# 36 & \url{igraph_igraph_read_graph_gml} & 6 & \cellcolor{green!10}3/40 & \cellcolor{green!10}2/40 & \cellcolor{green!10}1/40 & 0/40 & \cellcolor{green!10}9/187 & \cellcolor{green!10}9/98 & 0/40 & 0/40 & 0/40 & 0/40 & \cellcolor{green!10}4/214 & \cellcolor{green!10}2/193 \\
\cellcolor{black!10}\# 37 & \cellcolor{black!10}\url{igraph_igraph_read_graph_graphdb} & \cellcolor{black!10}6 & \cellcolor{green!10}3/40 & 0/40 & \cellcolor{green!15.00}6/40 & 0/40 & - & - & 0/40 & 0/40 & \cellcolor{green!10}1/40 & 0/40 & - & - \\
\# 38 & \url{igraph_igraph_read_graph_graphml} & 6 & \cellcolor{green!10}3/40 & \cellcolor{green!10}1/40 & \cellcolor{green!32.50}13/40 & \cellcolor{green!10}1/40 & - & - & 0/40 & 0/40 & \cellcolor{green!10}2/40 & 0/40 & - & - \\
\cellcolor{black!10}\# 39 & \cellcolor{black!10}\url{igraph_igraph_read_graph_lgl} & \cellcolor{black!10}6 & \cellcolor{green!10}1/40 & 0/40 & \cellcolor{green!32.50}13/40 & 0/40 & \cellcolor{green!10}1/207 & \cellcolor{green!10}7/107 & 0/40 & 0/40 & 0/40 & 0/40 & \cellcolor{green!10}2/251 & \cellcolor{green!10}1/150 \\
\# 40 & \url{igraph_igraph_read_graph_pajek} & 6 & \cellcolor{green!15.00}6/40 & \cellcolor{green!10}1/40 & \cellcolor{green!10}2/40 & 0/40 & \cellcolor{green!10}8/178 & \cellcolor{green!10.71}9/84 & 0/40 & 0/40 & 0/40 & 0/40 & \cellcolor{green!10}3/199 & \cellcolor{green!10}2/166 \\
\cellcolor{black!10}\# 41 & \cellcolor{black!10}\url{inchi_GetINCHIfromINCHI} & \cellcolor{black!10}6 & \cellcolor{green!20.00}8/40 & \cellcolor{green!67.50}27/40 & \cellcolor{green!47.50}19/40 & \cellcolor{green!80.00}32/40 & - & - & 0/40 & \cellcolor{green!10}1/40 & \cellcolor{green!20.00}8/40 & \cellcolor{green!12.50}5/40 & - & - \\
\# 42 & \url{inchi_GetStructFromINCHI} & 6 & 0/40 & \cellcolor{green!52.50}21/40 & \cellcolor{green!22.50}9/40 & \cellcolor{green!22.50}9/40 & - & - & 0/40 & \cellcolor{green!10}1/40 & 4/40 & \cellcolor{green!10}2/40 & - & - \\
\cellcolor{black!10}\# 43 & \cellcolor{black!10}\url{kamailio_parse_msg} & \cellcolor{black!10}6 & 0/40 & \cellcolor{green!57.50}23/40 & \cellcolor{green!22.50}9/40 & - & - & - & 0/40 & \cellcolor{green!32.50}13/40 & \cellcolor{green!30.00}12/40 & - & - & - \\
\# 44 & \url{libyang_lys_parse_mem} & 6 & \cellcolor{green!10}1/40 & \cellcolor{green!10}2/40 & \cellcolor{green!20.00}8/40 & 4/40 & - & - & 0/40 & 0/40 & \cellcolor{green!10}1/40 & 0/40 & - & - \\
\cellcolor{black!10}\# 45 & \cellcolor{black!10}\url{proftpd_pr_json_object_from_text} & \cellcolor{black!10}6 & 0/40 & 0/40 & \cellcolor{green!72.50}29/40 & - & - & - & 0/40 & 0/40 & \cellcolor{green!12.50}5/40 & - & - & - \\
\# 46 & \url{selinux_policydb_read} & 6 & 0/40 & \cellcolor{green!30.00}12/40 & \cellcolor{green!42.50}17/40 & - & - & - & 0/40 & \cellcolor{green!12.50}5/40 & \cellcolor{green!10}2/40 & - & - & - \\
\cellcolor{black!10}\# 47 & \cellcolor{black!10}\url{kamailio_get_src_address_socket} & \cellcolor{black!10}7 & 0/40 & \cellcolor{green!10}2/40 & \cellcolor{green!27.50}11/40 & 0/40 & - & - & 0/40 & 0/40 & \cellcolor{green!35.00}14/40 & 0/40 & - & - \\
\# 48 & \url{kamailio_get_src_uri} & 7 & 0/40 & 0/40 & \cellcolor{green!22.50}9/40 & \cellcolor{green!10}1/40 & - & - & 0/40 & \cellcolor{green!10}1/40 & \cellcolor{green!20.00}8/40 & 0/40 & - & - \\
\cellcolor{black!10}\# 49 & \cellcolor{black!10}\url{kamailio_parse_content_disposition} & \cellcolor{black!10}7 & 0/40 & 0/40 & 4/40 & 0/40 & - & - & 0/40 & 0/40 & 0/40 & 0/40 & - & - \\
\# 50 & \url{kamailio_parse_diversion_header} & 7 & 0/40 & 0/40 & \cellcolor{green!27.50}11/40 & 0/40 & - & - & 0/40 & 0/40 & \cellcolor{green!10}3/40 & 0/40 & - & - \\
\cellcolor{black!10}\# 51 & \cellcolor{black!10}\url{kamailio_parse_from_header} & \cellcolor{black!10}7 & 0/40 & 0/40 & 0/40 & - & - & - & 0/40 & 0/40 & 0/40 & - & - & - \\
\# 52 & \url{kamailio_parse_from_uri} & 7 & 0/40 & 0/40 & \cellcolor{green!10}1/40 & - & - & - & 0/40 & 0/40 & 0/40 & - & - & - \\
\cellcolor{black!10}\# 53 & \cellcolor{black!10}\url{kamailio_parse_headers} & \cellcolor{black!10}7 & 0/40 & 0/40 & 0/40 & - & - & - & 0/40 & 0/40 & 0/40 & - & - & - \\
\# 54 & \url{kamailio_parse_identityinfo_header} & 7 & 0/40 & 0/40 & \cellcolor{green!50.00}20/40 & - & - & - & 0/40 & 0/40 & \cellcolor{green!17.50}7/40 & - & - & - \\
\cellcolor{black!10}\# 55 & \cellcolor{black!10}\url{kamailio_parse_pai_header} & \cellcolor{black!10}7 & 0/40 & 0/40 & 4/40 & - & - & - & 0/40 & 0/40 & \cellcolor{green!10}2/40 & - & - & - \\
\# 56 & \url{kamailio_parse_privacy} & 7 & 0/40 & 0/40 & \cellcolor{green!22.50}9/40 & 0/40 & - & - & 0/40 & 0/40 & \cellcolor{green!10}2/40 & 0/40 & - & - \\
\cellcolor{black!10}\# 57 & \cellcolor{black!10}\url{kamailio_parse_record_route_headers} & \cellcolor{black!10}7 & 0/40 & 0/40 & \cellcolor{green!100.00}40/40 & - & - & - & 0/40 & 0/40 & \cellcolor{green!12.50}5/40 & - & - & - \\
\# 58 & \url{kamailio_parse_refer_to_header} & 7 & 0/40 & 0/40 & \cellcolor{green!17.50}7/40 & - & - & - & 0/40 & 0/40 & 4/40 & - & - & - \\
\cellcolor{black!10}\# 59 & \cellcolor{black!10}\url{kamailio_parse_route_headers} & \cellcolor{black!10}7 & 0/40 & 0/40 & \cellcolor{green!87.50}35/40 & - & - & - & 0/40 & \cellcolor{green!10}1/40 & \cellcolor{green!87.50}35/40 & - & - & - \\
\# 60 & \url{kamailio_parse_to_header} & 7 & 0/40 & 0/40 & \cellcolor{green!12.50}5/40 & - & - & - & 0/40 & 0/40 & \cellcolor{green!12.50}5/40 & - & - & - \\
\cellcolor{black!10}\# 61 & \cellcolor{black!10}\url{kamailio_parse_to_uri} & \cellcolor{black!10}7 & 0/40 & 0/40 & \cellcolor{green!10}3/40 & - & - & - & 0/40 & 0/40 & 0/40 & - & - & - \\
\# 62 & \url{libyang_lyd_parse_data_mem} & 7 & 0/40 & \cellcolor{green!22.50}9/40 & \cellcolor{green!45.00}18/40 & \cellcolor{green!25.00}10/40 & \cellcolor{green!19.44}14/72 & \cellcolor{green!25.00}13/52 & 0/40 & 0/40 & 0/40 & 0/40 & \cellcolor{green!10}5/184 & \cellcolor{green!10}2/47 \\
\cellcolor{black!10}\# 63 & \cellcolor{black!10}\url{bind9_dns_message_parse} & \cellcolor{black!10}8 & 0/40 & 0/40 & 0/40 & - & - & - & 0/40 & 0/40 & \cellcolor{green!10}1/40 & - & - & - \\
\# 64 & \url{igraph_igraph_read_graph_ncol} & 8 & \cellcolor{green!10}2/40 & 0/40 & \cellcolor{green!10}1/40 & 0/40 & 0/239 & \cellcolor{green!10}4/103 & 0/40 & 0/40 & 0/40 & 0/40 & \cellcolor{green!10}4/184 & 0/98 \\
\cellcolor{black!10}\# 65 & \cellcolor{black!10}\url{pjsip_pj_json_parse} & \cellcolor{black!10}8 & 0/40 & \cellcolor{green!10}3/40 & 4/40 & \cellcolor{green!10}2/40 & \cellcolor{green!10}13/150 & \cellcolor{green!10}10/115 & 0/40 & 0/40 & 0/40 & \cellcolor{green!10}3/40 & \cellcolor{green!10}1/260 & \cellcolor{green!10}5/155 \\
\# 66 & \url{pjsip_pj_xml_parse} & 8 & 0/40 & \cellcolor{green!25.00}10/40 & 4/40 & \cellcolor{green!25.00}10/40 & - & - & 0/40 & 0/40 & 0/40 & \cellcolor{green!17.50}7/40 & - & - \\
\cellcolor{black!10}\# 67 & \cellcolor{black!10}\url{pjsip_pjmedia_sdp_parse} & \cellcolor{black!10}8 & \cellcolor{green!10}2/40 & \cellcolor{green!22.50}9/40 & \cellcolor{green!10}1/40 & \cellcolor{green!35.00}14/40 & - & - & 0/40 & \cellcolor{green!10}3/40 & \cellcolor{green!10}1/40 & 4/40 & - & - \\
\# 68 & \url{quickjs_lre_compile} & 8 & 0/40 & 0/40 & 0/40 & - & 0/180 & \cellcolor{green!10}2/91 & 0/40 & 0/40 & 0/40 & - & 0/260 & 0/122 \\
\cellcolor{black!10}\# 69 & \cellcolor{black!10}\url{bind9_isc_lex_getmastertoken} & \cellcolor{black!10}9 & 0/40 & 0/40 & 0/40 & - & \cellcolor{green!10}2/128 & \cellcolor{green!10}1/70 & 0/40 & 0/40 & 0/40 & - & 0/121 & 0/86 \\
\# 70 & \url{bind9_isc_lex_gettoken} & 9 & 0/40 & 0/40 & 0/40 & - & \cellcolor{green!10}3/227 & \cellcolor{green!10}2/62 & 0/40 & 0/40 & 0/40 & - & 0/168 & \cellcolor{green!10}1/95 \\
\cellcolor{black!10}\# 71 & \cellcolor{black!10}\url{quickjs_JS_Eval} & \cellcolor{black!10}9 & \cellcolor{green!17.50}7/40 & \cellcolor{green!95.00}38/40 & \cellcolor{green!25.00}10/40 & - & - & - & \cellcolor{green!17.50}7/40 & \cellcolor{green!32.50}13/40 & \cellcolor{green!10}3/40 & - & - & - \\
\# 72 & \url{igraph_igraph_edge_connectivity} & 10 & 0/40 & 0/40 & 0/40 & 0/40 & 0/176 & 0/62 & 0/40 & 0/40 & 0/40 & 0/40 & 0/176 & 0/105 \\
\cellcolor{black!10}\# 73 & \cellcolor{black!10}\url{pjsip_pj_stun_msg_decode} & \cellcolor{black!10}10 & 0/40 & 0/40 & 0/40 & 0/40 & \cellcolor{green!10}12/189 & \cellcolor{green!14.47}11/76 & 0/40 & 0/40 & 0/40 & 0/40 & 0/242 & \cellcolor{green!10}1/122 \\
\# 74 & \url{bind9_dns_message_checksig} & 11 & 0/40 & 0/40 & \cellcolor{green!10}1/40 & - & 0/141 & \cellcolor{green!10}1/86 & 0/40 & 0/40 & 0/40 & - & 0/204 & 0/137 \\
\cellcolor{black!10}\# 75 & \cellcolor{black!10}\url{libzip_zip_fread} & \cellcolor{black!10}11 & \cellcolor{green!57.50}23/40 & \cellcolor{green!45.00}18/40 & \cellcolor{green!55.00}22/40 & \cellcolor{green!32.50}13/40 & - & - & \cellcolor{green!10}1/40 & \cellcolor{green!10}2/40 & 4/40 & \cellcolor{green!10}3/40 & - & - \\
\# 76 & \url{bind9_dns_rdata_fromtext} & 12 & 0/40 & 0/40 & 0/40 & - & 0/159 & \cellcolor{green!10}2/59 & 0/40 & 0/40 & 0/40 & - & 0/110 & 0/49 \\
\cellcolor{black!10}\# 77 & \cellcolor{black!10}\url{igraph_igraph_all_minimal_st_separators} & \cellcolor{black!10}12 & 0/40 & \cellcolor{green!12.50}5/40 & \cellcolor{green!17.50}7/40 & \cellcolor{green!10}1/40 & 12/120 & \cellcolor{green!12.00}9/75 & 0/40 & 0/40 & 0/40 & \cellcolor{green!10}1/40 & \cellcolor{green!10}6/225 & \cellcolor{green!10}1/126 \\
\# 78 & \url{igraph_igraph_minimum_size_separators} & 12 & \cellcolor{green!10}2/40 & 4/40 & \cellcolor{green!37.50}15/40 & 4/40 & - & - & 0/40 & 0/40 & 0/40 & 0/40 & - & - \\
\cellcolor{black!10}\# 79 & \cellcolor{black!10}\url{pjsip_pjsip_parse_msg} & \cellcolor{black!10}12 & 0/40 & 0/40 & \cellcolor{green!10}1/40 & 0/40 & 0/230 & \cellcolor{green!10}2/71 & 0/40 & 0/40 & 0/40 & 0/40 & 0/289 & 0/92 \\
\# 80 & \url{igraph_igraph_automorphism_group} & 13 & 0/40 & 0/40 & \cellcolor{green!62.50}25/40 & 0/40 & 0/255 & \cellcolor{green!13.39}15/112 & 0/40 & 0/40 & 0/40 & 0/40 & 0/138 & 0/97 \\
\cellcolor{black!10}\# 81 & \cellcolor{black!10}\url{libmodbus_modbus_read_bits} & \cellcolor{black!10}15 & 0/40 & 0/40 & 0/40 & 0/40 & 0/44 & 0/35 & 0/40 & 0/40 & 0/40 & 0/40 & 0/74 & 0/73 \\
\# 82 & \url{libmodbus_modbus_read_registers} & 15 & 0/40 & 0/40 & 0/40 & 0/40 & 0/64 & 0/35 & 0/40 & 0/40 & 0/40 & 0/40 & 0/88 & 0/41 \\
\cellcolor{black!10}\# 83 & \cellcolor{black!10}\url{civetweb_mg_get_response} & \cellcolor{black!10}17 & 0/40 & 0/40 & 0/40 & 0/40 & 0/208 & 0/47 & 0/40 & 0/40 & 0/40 & 0/40 & 0/159 & 0/91 \\
\# 84 & \url{bind9_dns_master_loadbuffer} & 20 & 0/40 & 0/40 & 0/40 & - & \cellcolor{green!10}1/206 & 0/103 & 0/40 & 0/40 & 0/40 & - & 0/157 & 0/80 \\
\cellcolor{black!10}\# 85 & \cellcolor{black!10}\url{libmodbus_modbus_receive} & \cellcolor{black!10}33 & 0/40 & 0/40 & 0/40 & 0/40 & 0/44 & 0/47 & 0/40 & 0/40 & 0/40 & 0/40 & 0/72 & 0/57 \\
\# 86 & \url{tmux_input_parse_buffer} & 42 & 0/40 & 0/40 & 0/40 & - & 0/189 & 0/165 & 0/40 & 0/40 & 0/40 & - & 0/213 & 0/146 \\
\bottomrule

\end{tabular}

}
\end{table*}



\begin{table*}
\caption{Full List of Evaluation Results}
\label{tab:eval_full}
\renewcommand\arraystretch{1}
\scriptsize
\setlength\arrayrulewidth{2pt}\arrayrulecolor{black}
\resizebox{\textwidth}{!}{
\begin{tabular}{cp{3.6cm} 
        >{\columncolor[gray]{1}[0.8\tabcolsep]}c
        >{\columncolor[gray]{1}[0.8\tabcolsep]}c
        >{\columncolor[gray]{1}[0.8\tabcolsep]}c
        >{\columncolor[gray]{1}[0.8\tabcolsep]}c
        >{\columncolor[gray]{1}[0.8\tabcolsep]}c
        >{\columncolor[gray]{1}[0.8\tabcolsep]}c
        >{\columncolor[gray]{1}[0.8\tabcolsep]}c
        >{\columncolor[gray]{1}[0.8\tabcolsep]}c
        >{\columncolor[gray]{1}[0.8\tabcolsep]}c
        >{\columncolor[gray]{1}[0.8\tabcolsep]}c
        >{\columncolor[gray]{1}[0.8\tabcolsep]}c
        >{\columncolor[gray]{1}[0.8\tabcolsep]}c
        >{\columncolor[gray]{1}[0.8\tabcolsep]}c
    }


\toprule
% Question & SCORE & gpt4-NAIVE & gpt4-BACTX & gpt4-UGCTX & gpt4-DOCTX & gpt4-BA-ITER & gpt4-ALL-ITER & gpt3.5-NAIVE & gpt3.5-BACTX & gpt3.5-UGCTX & gpt3.5-DOCTX & gpt3.5-BA-ITER & gpt3.5-ALL-ITER \\
 \multirow{2}{*}{Index} & \multirow{2}{*}{Question} & \multirow{2}{*}{Score} &   \multicolumn{6}{c}{GPT4} &  \multicolumn{6}{c}{GPT3.5} \\
&  & & NAIVE & BACTX & UGCTX & DOCTX & BA-ITER & EX-ITER &  NAIVE &  BACTX &  UGCTX &  DOCTX &  BA-ITER & EX-ITER \\
\midrule
\cellcolor{black!10}\# 1 & \cellcolor{black!10}\url{coturn_stun_is_command_message_full_check_str} & \cellcolor{black!10}1 & 0/40 & \cellcolor{green!72.50}29/40 & 0/40 & - & \cellcolor{green!53.23}33/62 & \cellcolor{green!16.04}17/106 & 0/40 & \cellcolor{green!67.50}27/40 & \cellcolor{green!12.50}5/40 & - & \cellcolor{green!42.25}30/71 & \cellcolor{green!32.00}24/75 \\
\# 2 & \url{kamailio_parse_uri} & 1 & 0/40 & \cellcolor{green!100.00}40/40 & \cellcolor{green!55.00}22/40 & - & \cellcolor{green!35.96}32/89 & \cellcolor{green!23.68}27/114 & 0/40 & \cellcolor{green!85.00}34/40 & \cellcolor{green!52.50}21/40 & - & \cellcolor{green!35.63}31/87 & \cellcolor{green!10}10/155 \\
\cellcolor{black!10}\# 3 & \cellcolor{black!10}\url{coturn_stun_check_message_integrity_str} & \cellcolor{black!10}2 & 0/40 & \cellcolor{green!30.00}12/40 & \cellcolor{green!20.00}8/40 & - & \cellcolor{green!10}13/231 & \cellcolor{green!10}7/151 & 0/40 & \cellcolor{green!20.00}8/40 & \cellcolor{green!10}2/40 & - & \cellcolor{green!10}9/156 & \cellcolor{green!10}4/93 \\
\# 4 & \url{libiec61850_MmsValue_decodeMmsData} & 2 & 0/40 & \cellcolor{green!97.50}39/40 & \cellcolor{green!27.50}11/40 & \cellcolor{green!92.50}37/40 & \cellcolor{green!40.96}34/83 & \cellcolor{green!15.38}18/117 & 0/40 & \cellcolor{green!92.50}37/40 & \cellcolor{green!30.00}12/40 & \cellcolor{green!67.50}27/40 & \cellcolor{green!33.98}35/103 & \cellcolor{green!11.11}12/108 \\
\cellcolor{black!10}\# 5 & \cellcolor{black!10}\url{md4c_md_html} & \cellcolor{black!10}2 & 0/40 & 0/40 & 0/40 & 0/40 & \cellcolor{green!48.78}40/82 & \cellcolor{green!18.03}22/122 & 0/40 & 0/40 & 0/40 & 0/40 & \cellcolor{green!32.38}34/105 & \cellcolor{green!10}4/172 \\
\# 6 & \url{spdk_spdk_json_parse} & 2 & \cellcolor{green!10.00}4/40 & \cellcolor{green!87.50}35/40 & \cellcolor{green!15.00}6/40 & - & \cellcolor{green!46.48}33/71 & \cellcolor{green!20.43}19/93 & 0/40 & \cellcolor{green!75.00}30/40 & \cellcolor{green!15.00}6/40 & - & \cellcolor{green!44.16}34/77 & \cellcolor{green!10.43}12/115 \\
\cellcolor{black!10}\# 7 & \cellcolor{black!10}\url{croaring_roaring_bitmap_portable_deserialize_safe} & \cellcolor{black!10}3 & \cellcolor{green!20.00}8/40 & \cellcolor{green!100.00}40/40 & \cellcolor{green!72.50}29/40 & \cellcolor{green!70.00}28/40 & \cellcolor{green!32.29}31/96 & \cellcolor{green!24.75}25/101 & \cellcolor{green!27.50}11/40 & \cellcolor{green!55.00}22/40 & \cellcolor{green!25.00}10/40 & \cellcolor{green!75.00}30/40 & \cellcolor{green!27.45}28/102 & \cellcolor{green!17.60}22/125 \\
\# 8 & \url{lua_luaL_loadbufferx} & 3 & \cellcolor{green!67.50}27/40 & \cellcolor{green!97.50}39/40 & \cellcolor{green!60.00}24/40 & \cellcolor{green!100.00}40/40 & \cellcolor{green!72.00}36/50 & \cellcolor{green!41.33}31/75 & \cellcolor{green!77.50}31/40 & \cellcolor{green!82.50}33/40 & \cellcolor{green!42.50}17/40 & \cellcolor{green!85.00}34/40 & \cellcolor{green!54.72}29/53 & \cellcolor{green!13.51}15/111 \\
\cellcolor{black!10}\# 9 & \cellcolor{black!10}\url{w3m_wc_Str_conv_with_detect} & \cellcolor{black!10}3 & 0/40 & 0/40 & \cellcolor{green!25.00}10/40 & - & \cellcolor{green!10}2/348 & \cellcolor{green!10}17/178 & 0/40 & 0/40 & \cellcolor{green!25.00}10/40 & - & \cellcolor{green!10}1/367 & \cellcolor{green!10}10/230 \\
\# 10 & \url{bind9_dns_name_fromwire} & 4 & 0/40 & \cellcolor{green!20.00}8/40 & \cellcolor{green!10}3/40 & - & \cellcolor{green!10}7/323 & \cellcolor{green!10}12/215 & 0/40 & 0/40 & \cellcolor{green!10}1/40 & - & 0/265 & \cellcolor{green!10}1/164 \\
\cellcolor{black!10}\# 11 & \cellcolor{black!10}\url{gdk-pixbuf_gdk_pixbuf_animation_new_from_file} & \cellcolor{black!10}4 & \cellcolor{green!15.00}6/40 & \cellcolor{green!82.50}33/40 & \cellcolor{green!37.50}15/40 & \cellcolor{green!67.50}27/40 & \cellcolor{green!32.20}19/59 & \cellcolor{green!10}7/77 & \cellcolor{green!10}3/40 & \cellcolor{green!27.50}11/40 & \cellcolor{green!10}3/40 & \cellcolor{green!22.50}9/40 & \cellcolor{green!10}7/74 & \cellcolor{green!10}4/79 \\
\# 12 & \url{gdk-pixbuf_gdk_pixbuf_new_from_data} & 4 & \cellcolor{green!10}2/40 & \cellcolor{green!40.00}16/40 & \cellcolor{green!27.50}11/40 & \cellcolor{green!27.50}11/40 & \cellcolor{green!24.75}25/101 & \cellcolor{green!14.10}11/78 & \cellcolor{green!45.00}18/40 & \cellcolor{green!62.50}25/40 & \cellcolor{green!15.00}6/40 & \cellcolor{green!57.50}23/40 & \cellcolor{green!25.00}21/84 & \cellcolor{green!12.62}13/103 \\
\cellcolor{black!10}\# 13 & \cellcolor{black!10}\url{gdk-pixbuf_gdk_pixbuf_new_from_file} & \cellcolor{black!10}4 & \cellcolor{green!25.00}10/40 & \cellcolor{green!97.50}39/40 & \cellcolor{green!50.00}20/40 & \cellcolor{green!92.50}37/40 & \cellcolor{green!30.88}21/68 & \cellcolor{green!18.03}11/61 & \cellcolor{green!12.50}5/40 & \cellcolor{green!35.00}14/40 & \cellcolor{green!12.50}5/40 & \cellcolor{green!20.00}8/40 & \cellcolor{green!10}4/85 & \cellcolor{green!10}2/78 \\
\# 14 & \url{gdk-pixbuf_gdk_pixbuf_new_from_stream} & 4 & \cellcolor{green!12.50}5/40 & \cellcolor{green!75.00}30/40 & \cellcolor{green!60.00}24/40 & \cellcolor{green!65.00}26/40 & \cellcolor{green!60.00}33/55 & \cellcolor{green!37.88}25/66 & \cellcolor{green!52.50}21/40 & \cellcolor{green!87.50}35/40 & \cellcolor{green!45.00}18/40 & \cellcolor{green!80.00}32/40 & \cellcolor{green!36.71}29/79 & \cellcolor{green!18.68}17/91 \\
\cellcolor{black!10}\# 15 & \cellcolor{black!10}\url{gpac_gf_isom_open_file} & \cellcolor{black!10}4 & \cellcolor{green!10}1/40 & \cellcolor{green!67.50}27/40 & \cellcolor{green!50.00}20/40 & - & \cellcolor{green!10}5/172 & \cellcolor{green!10}5/109 & 0/40 & \cellcolor{green!12.50}5/40 & 0/40 & - & 0/257 & 0/158 \\
\# 16 & \url{libbpf_bpf_object__open_mem} & 4 & \cellcolor{green!10}1/40 & \cellcolor{green!15.00}6/40 & \cellcolor{green!10.00}4/40 & \cellcolor{green!15.00}6/40 & \cellcolor{green!10.50}19/181 & \cellcolor{green!10}13/172 & 0/40 & \cellcolor{green!27.50}11/40 & \cellcolor{green!15.00}6/40 & \cellcolor{green!12.50}5/40 & \cellcolor{green!11.11}13/117 & \cellcolor{green!10}10/116 \\
\cellcolor{black!10}\# 17 & \cellcolor{black!10}\url{libpg_query_pg_query_parse} & \cellcolor{black!10}4 & \cellcolor{green!10.00}4/40 & \cellcolor{green!90.00}36/40 & \cellcolor{green!95.00}38/40 & - & \cellcolor{green!34.09}30/88 & \cellcolor{green!34.62}27/78 & \cellcolor{green!15.00}6/40 & \cellcolor{green!42.50}17/40 & \cellcolor{green!65.00}26/40 & - & \cellcolor{green!21.95}27/123 & \cellcolor{green!14.89}21/141 \\
\# 18 & \url{libucl_ucl_parser_add_string} & 4 & \cellcolor{green!20.00}8/40 & \cellcolor{green!47.50}19/40 & \cellcolor{green!50.00}20/40 & \cellcolor{green!72.50}29/40 & \cellcolor{green!22.61}26/115 & \cellcolor{green!16.95}20/118 & \cellcolor{green!17.50}7/40 & \cellcolor{green!20.00}8/40 & \cellcolor{green!12.50}5/40 & \cellcolor{green!45.00}18/40 & \cellcolor{green!16.99}26/153 & \cellcolor{green!15.91}21/132 \\
\cellcolor{black!10}\# 19 & \cellcolor{black!10}\url{oniguruma_onig_new} & \cellcolor{black!10}4 & \cellcolor{green!50.00}20/40 & \cellcolor{green!87.50}35/40 & \cellcolor{green!55.00}22/40 & \cellcolor{green!82.50}33/40 & \cellcolor{green!21.66}34/157 & \cellcolor{green!26.47}18/68 & \cellcolor{green!30.00}12/40 & \cellcolor{green!45.00}18/40 & \cellcolor{green!12.50}5/40 & \cellcolor{green!37.50}15/40 & \cellcolor{green!22.88}27/118 & \cellcolor{green!12.96}14/108 \\
\# 20 & \url{pupnp_ixmlLoadDocumentEx} & 4 & 0/40 & \cellcolor{green!65.00}26/40 & \cellcolor{green!75.00}15/20 & \cellcolor{green!40.00}16/40 & \cellcolor{green!11.43}20/175 & \cellcolor{green!10}11/128 & 0/40 & \cellcolor{green!17.50}7/40 & \cellcolor{green!20.00}4/20 & \cellcolor{green!10}1/40 & \cellcolor{green!10}5/135 & \cellcolor{green!10}2/107 \\
\cellcolor{black!10}\# 21 & \cellcolor{black!10}\url{gdk-pixbuf_gdk_pixbuf_new_from_file_at_scale} & \cellcolor{black!10}5 & \cellcolor{green!45.00}18/40 & \cellcolor{green!72.50}29/40 & \cellcolor{green!40.00}16/40 & \cellcolor{green!70.00}28/40 & \cellcolor{green!12.77}12/94 & \cellcolor{green!10}8/87 & \cellcolor{green!10}2/40 & \cellcolor{green!10}1/40 & \cellcolor{green!10}3/40 & \cellcolor{green!17.50}7/40 & 0/108 & \cellcolor{green!10}1/98 \\
\# 22 & \url{inchi_GetINCHIKeyFromINCHI} & 5 & 0/40 & \cellcolor{green!75.00}30/40 & \cellcolor{green!22.50}9/40 & \cellcolor{green!77.50}31/40 & \cellcolor{green!35.11}33/94 & \cellcolor{green!26.04}25/96 & 0/40 & \cellcolor{green!40.00}16/40 & \cellcolor{green!25.00}10/40 & \cellcolor{green!60.00}24/40 & \cellcolor{green!20.95}31/148 & \cellcolor{green!10.96}16/146 \\
\cellcolor{black!10}\# 23 & \cellcolor{black!10}\url{libdwarf_dwarf_init_b} & \cellcolor{black!10}5 & 0/40 & 0/40 & \cellcolor{green!32.50}13/40 & \cellcolor{green!10}1/40 & \cellcolor{green!10}15/273 & \cellcolor{green!16.07}18/112 & 0/40 & 0/40 & \cellcolor{green!27.50}11/40 & \cellcolor{green!12.50}5/40 & \cellcolor{green!10}4/241 & \cellcolor{green!10}9/91 \\
\# 24 & \url{libdwarf_dwarf_init_path} & 5 & 0/40 & 0/40 & \cellcolor{green!20.00}8/40 & 0/40 & 0/371 & \cellcolor{green!10}2/122 & 0/40 & 0/40 & \cellcolor{green!10.00}4/40 & 0/40 & 0/391 & \cellcolor{green!10}1/89 \\
\cellcolor{black!10}\# 25 & \cellcolor{black!10}\url{liblouis_lou_compileString} & \cellcolor{black!10}5 & 0/40 & \cellcolor{green!22.50}9/40 & \cellcolor{green!50.00}20/40 & \cellcolor{green!20.00}8/40 & \cellcolor{green!10}18/182 & \cellcolor{green!11.68}16/137 & 0/40 & \cellcolor{green!17.50}7/40 & \cellcolor{green!12.50}5/40 & \cellcolor{green!30.00}12/40 & \cellcolor{green!10}4/186 & \cellcolor{green!10}3/129 \\
\# 26 & \url{selinux_cil_compile} & 5 & 0/40 & 0/40 & \cellcolor{green!62.50}25/40 & - & \cellcolor{green!10}3/266 & \cellcolor{green!26.04}25/96 & 0/40 & 0/40 & \cellcolor{green!27.50}11/40 & - & \cellcolor{green!10}1/298 & \cellcolor{green!10}9/91 \\
\cellcolor{black!10}\# 27 & \cellcolor{black!10}\url{bind9_dns_name_fromtext} & \cellcolor{black!10}6 & 0/40 & \cellcolor{green!55.00}22/40 & \cellcolor{green!15.00}6/40 & - & \cellcolor{green!10}16/173 & \cellcolor{green!10}14/161 & 0/40 & 0/40 & \cellcolor{green!10}3/40 & - & 0/301 & \cellcolor{green!10}3/169 \\
\# 28 & \url{bind9_dns_rdata_fromwire} & 6 & 0/40 & 0/40 & \cellcolor{green!10}1/40 & - & \cellcolor{green!10}2/332 & \cellcolor{green!10}7/224 & 0/40 & 0/40 & \cellcolor{green!10}1/40 & - & 0/220 & 0/193 \\
\cellcolor{black!10}\# 29 & \cellcolor{black!10}\url{coturn_stun_is_binding_response} & \cellcolor{black!10}6 & 0/40 & \cellcolor{green!60.00}24/40 & \cellcolor{green!30.00}12/40 & - & \cellcolor{green!16.42}22/134 & \cellcolor{green!13.68}16/117 & 0/40 & 0/40 & \cellcolor{green!35.00}14/40 & - & \cellcolor{green!10}5/189 & \cellcolor{green!10}6/83 \\
\# 30 & \url{coturn_stun_is_command_message} & 6 & 0/40 & \cellcolor{green!30.00}12/40 & \cellcolor{green!37.50}15/40 & \cellcolor{green!27.50}11/40 & \cellcolor{green!10.06}16/159 & \cellcolor{green!12.70}16/126 & 0/40 & 0/40 & \cellcolor{green!42.50}17/40 & 0/40 & \cellcolor{green!10}1/193 & \cellcolor{green!12.79}11/86 \\
\cellcolor{black!10}\# 31 & \cellcolor{black!10}\url{coturn_stun_is_response} & \cellcolor{black!10}6 & 0/40 & \cellcolor{green!35.00}14/40 & 0/40 & - & \cellcolor{green!10}13/153 & \cellcolor{green!10}8/115 & 0/40 & 0/40 & \cellcolor{green!25.00}10/40 & - & 0/216 & \cellcolor{green!10}2/125 \\
\# 32 & \url{coturn_stun_is_success_response} & 6 & 0/40 & \cellcolor{green!40.00}16/40 & \cellcolor{green!37.50}15/40 & - & \cellcolor{green!12.59}18/143 & \cellcolor{green!15.73}14/89 & 0/40 & 0/40 & \cellcolor{green!15.00}6/40 & - & \cellcolor{green!10}4/180 & \cellcolor{green!10}6/87 \\
\cellcolor{black!10}\# 33 & \cellcolor{black!10}\url{hiredis_redisFormatCommand} & \cellcolor{black!10}6 & \cellcolor{green!32.50}13/40 & \cellcolor{green!100.00}40/40 & \cellcolor{green!22.50}9/40 & - & \cellcolor{green!10.55}27/256 & \cellcolor{green!13.73}21/153 & \cellcolor{green!10}2/40 & \cellcolor{green!77.50}31/40 & \cellcolor{green!20.00}8/40 & - & \cellcolor{green!10}13/311 & \cellcolor{green!10}8/185 \\
\# 34 & \url{igraph_igraph_read_graph_dl} & 6 & \cellcolor{green!22.50}9/40 & 0/40 & \cellcolor{green!10.00}4/40 & 0/40 & \cellcolor{green!10}8/237 & \cellcolor{green!10}11/173 & 0/40 & 0/40 & 0/40 & 0/40 & \cellcolor{green!10}3/293 & \cellcolor{green!10}2/262 \\
\cellcolor{black!10}\# 35 & \cellcolor{black!10}\url{igraph_igraph_read_graph_edgelist} & \cellcolor{black!10}6 & \cellcolor{green!15.00}6/40 & 0/40 & \cellcolor{green!10}2/40 & \cellcolor{green!10}1/40 & \cellcolor{green!10}11/309 & \cellcolor{green!10}4/203 & 0/40 & 0/40 & 0/40 & 0/40 & \cellcolor{green!10}2/295 & \cellcolor{green!10}2/241 \\
\# 36 & \url{igraph_igraph_read_graph_gml} & 6 & \cellcolor{green!10}3/40 & \cellcolor{green!10}2/40 & \cellcolor{green!10}1/40 & 0/40 & \cellcolor{green!10}17/345 & \cellcolor{green!10}12/240 & 0/40 & 0/40 & 0/40 & 0/40 & \cellcolor{green!10}5/399 & \cellcolor{green!10}2/376 \\
\cellcolor{black!10}\# 37 & \cellcolor{black!10}\url{igraph_igraph_read_graph_graphdb} & \cellcolor{black!10}6 & \cellcolor{green!10}3/40 & 0/40 & \cellcolor{green!15.00}6/40 & 0/40 & \cellcolor{green!10}7/211 & \cellcolor{green!10}14/178 & 0/40 & 0/40 & \cellcolor{green!10}1/40 & 0/40 & \cellcolor{green!10}3/256 & \cellcolor{green!10}6/245 \\
\# 38 & \url{igraph_igraph_read_graph_graphml} & 6 & \cellcolor{green!10}3/40 & \cellcolor{green!10}1/40 & \cellcolor{green!32.50}13/40 & \cellcolor{green!10}1/40 & \cellcolor{green!10}11/262 & \cellcolor{green!15.75}20/127 & 0/40 & 0/40 & \cellcolor{green!10}2/40 & 0/40 & \cellcolor{green!10}10/243 & \cellcolor{green!10}8/151 \\
\cellcolor{black!10}\# 39 & \cellcolor{black!10}\url{igraph_igraph_read_graph_lgl} & \cellcolor{black!10}6 & \cellcolor{green!10}1/40 & 0/40 & \cellcolor{green!32.50}13/40 & 0/40 & \cellcolor{green!10}7/371 & \cellcolor{green!10}19/205 & 0/40 & 0/40 & 0/40 & 0/40 & \cellcolor{green!10}8/450 & \cellcolor{green!10}3/260 \\
\# 40 & \url{igraph_igraph_read_graph_pajek} & 6 & \cellcolor{green!15.00}6/40 & \cellcolor{green!10}1/40 & \cellcolor{green!10}2/40 & 0/40 & \cellcolor{green!10}19/294 & \cellcolor{green!10}17/172 & 0/40 & 0/40 & 0/40 & 0/40 & \cellcolor{green!10}5/431 & \cellcolor{green!10}3/339 \\
\cellcolor{black!10}\# 41 & \cellcolor{black!10}\url{inchi_GetINCHIfromINCHI} & \cellcolor{black!10}6 & \cellcolor{green!20.00}8/40 & \cellcolor{green!67.50}27/40 & \cellcolor{green!47.50}19/40 & \cellcolor{green!80.00}32/40 & \cellcolor{green!10}5/338 & 0/238 & 0/40 & \cellcolor{green!10}1/40 & \cellcolor{green!20.00}8/40 & \cellcolor{green!12.50}5/40 & 0/350 & \cellcolor{green!10}2/257 \\
\# 42 & \url{inchi_GetStructFromINCHI} & 6 & 0/40 & \cellcolor{green!52.50}21/40 & \cellcolor{green!22.50}9/40 & \cellcolor{green!22.50}9/40 & \cellcolor{green!10}5/366 & \cellcolor{green!10}5/220 & 0/40 & \cellcolor{green!10}1/40 & \cellcolor{green!10.00}4/40 & \cellcolor{green!10}2/40 & \cellcolor{green!10}1/316 & \cellcolor{green!10}1/387 \\
\cellcolor{black!10}\# 43 & \cellcolor{black!10}\url{kamailio_parse_msg} & \cellcolor{black!10}6 & 0/40 & \cellcolor{green!57.50}23/40 & \cellcolor{green!22.50}9/40 & - & \cellcolor{green!16.36}18/110 & \cellcolor{green!15.08}19/126 & 0/40 & \cellcolor{green!32.50}13/40 & \cellcolor{green!30.00}12/40 & - & \cellcolor{green!10}10/169 & \cellcolor{green!10}11/183 \\
\# 44 & \url{libyang_lys_parse_mem} & 6 & \cellcolor{green!10}1/40 & \cellcolor{green!10}2/40 & \cellcolor{green!20.00}8/40 & \cellcolor{green!10.00}4/40 & \cellcolor{green!10}19/287 & \cellcolor{green!13.14}18/137 & 0/40 & 0/40 & \cellcolor{green!10}1/40 & 0/40 & \cellcolor{green!10}9/295 & \cellcolor{green!10}10/191 \\
\cellcolor{black!10}\# 45 & \cellcolor{black!10}\url{proftpd_pr_json_object_from_text} & \cellcolor{black!10}6 & 0/40 & 0/40 & \cellcolor{green!72.50}29/40 & - & \cellcolor{green!10}8/303 & \cellcolor{green!10}12/161 & 0/40 & 0/40 & \cellcolor{green!12.50}5/40 & - & \cellcolor{green!10}1/317 & \cellcolor{green!10}3/152 \\
\# 46 & \url{selinux_policydb_read} & 6 & 0/40 & \cellcolor{green!30.00}12/40 & \cellcolor{green!42.50}17/40 & - & \cellcolor{green!10.95}15/137 & \cellcolor{green!13.01}16/123 & 0/40 & \cellcolor{green!12.50}5/40 & \cellcolor{green!10}2/40 & - & \cellcolor{green!10}6/163 & \cellcolor{green!10}7/117 \\
\cellcolor{black!10}\# 47 & \cellcolor{black!10}\url{kamailio_get_src_address_socket} & \cellcolor{black!10}7 & 0/40 & \cellcolor{green!10}2/40 & \cellcolor{green!27.50}11/40 & 0/40 & \cellcolor{green!10}4/160 & \cellcolor{green!10}6/170 & 0/40 & 0/40 & \cellcolor{green!35.00}14/40 & 0/40 & 0/161 & \cellcolor{green!10}8/193 \\
\# 48 & \url{kamailio_get_src_uri} & 7 & 0/40 & 0/40 & \cellcolor{green!22.50}9/40 & \cellcolor{green!10}1/40 & \cellcolor{green!10}3/162 & \cellcolor{green!10}8/156 & 0/40 & \cellcolor{green!10}1/40 & \cellcolor{green!20.00}8/40 & 0/40 & \cellcolor{green!10}1/144 & \cellcolor{green!10}2/189 \\
\cellcolor{black!10}\# 49 & \cellcolor{black!10}\url{kamailio_parse_content_disposition} & \cellcolor{black!10}7 & 0/40 & 0/40 & \cellcolor{green!10.00}4/40 & 0/40 & 0/229 & \cellcolor{green!10}3/182 & 0/40 & 0/40 & 0/40 & 0/40 & 0/236 & \cellcolor{green!10}2/160 \\
\# 50 & \url{kamailio_parse_diversion_header} & 7 & 0/40 & 0/40 & \cellcolor{green!27.50}11/40 & 0/40 & 0/247 & \cellcolor{green!10}8/126 & 0/40 & 0/40 & \cellcolor{green!10}3/40 & 0/40 & 0/216 & \cellcolor{green!10}1/177 \\
\cellcolor{black!10}\# 51 & \cellcolor{black!10}\url{kamailio_parse_from_header} & \cellcolor{black!10}7 & 0/40 & 0/40 & 0/40 & - & 0/230 & 0/173 & 0/40 & 0/40 & 0/40 & - & 0/203 & \cellcolor{green!10}1/173 \\
\# 52 & \url{kamailio_parse_from_uri} & 7 & 0/40 & 0/40 & \cellcolor{green!10}1/40 & - & \cellcolor{green!10}1/292 & \cellcolor{green!10}1/222 & 0/40 & 0/40 & 0/40 & - & 0/203 & 0/137 \\
\cellcolor{black!10}\# 53 & \cellcolor{black!10}\url{kamailio_parse_headers} & \cellcolor{black!10}7 & 0/40 & 0/40 & 0/40 & - & 0/118 & \cellcolor{green!10}1/120 & 0/40 & 0/40 & 0/40 & - & 0/195 & 0/196 \\
\# 54 & \url{kamailio_parse_identityinfo_header} & 7 & 0/40 & 0/40 & \cellcolor{green!50.00}20/40 & - & 0/212 & \cellcolor{green!10.22}14/137 & 0/40 & 0/40 & \cellcolor{green!17.50}7/40 & - & 0/223 & \cellcolor{green!10}4/169 \\
\cellcolor{black!10}\# 55 & \cellcolor{black!10}\url{kamailio_parse_pai_header} & \cellcolor{black!10}7 & 0/40 & 0/40 & \cellcolor{green!10.00}4/40 & - & 0/187 & \cellcolor{green!10}4/208 & 0/40 & 0/40 & \cellcolor{green!10}2/40 & - & 0/192 & \cellcolor{green!10}1/192 \\
\# 56 & \url{kamailio_parse_privacy} & 7 & 0/40 & 0/40 & \cellcolor{green!22.50}9/40 & 0/40 & 0/203 & \cellcolor{green!10}6/111 & 0/40 & 0/40 & \cellcolor{green!10}2/40 & 0/40 & 0/281 & 0/187 \\
\cellcolor{black!10}\# 57 & \cellcolor{black!10}\url{kamailio_parse_record_route_headers} & \cellcolor{black!10}7 & 0/40 & 0/40 & \cellcolor{green!100.00}40/40 & - & \cellcolor{green!10}1/220 & \cellcolor{green!21.13}15/71 & 0/40 & 0/40 & \cellcolor{green!12.50}5/40 & - & 0/240 & \cellcolor{green!10}6/122 \\
\# 58 & \url{kamailio_parse_refer_to_header} & 7 & 0/40 & 0/40 & \cellcolor{green!17.50}7/40 & - & 0/216 & \cellcolor{green!10}5/131 & 0/40 & 0/40 & \cellcolor{green!10.00}4/40 & - & 0/213 & \cellcolor{green!10}2/198 \\
\cellcolor{black!10}\# 59 & \cellcolor{black!10}\url{kamailio_parse_route_headers} & \cellcolor{black!10}7 & 0/40 & 0/40 & \cellcolor{green!87.50}35/40 & - & 0/173 & \cellcolor{green!11.85}16/135 & 0/40 & \cellcolor{green!10}1/40 & \cellcolor{green!87.50}35/40 & - & 0/243 & \cellcolor{green!10}5/134 \\
\# 60 & \url{kamailio_parse_to_header} & 7 & 0/40 & 0/40 & \cellcolor{green!12.50}5/40 & - & 0/236 & \cellcolor{green!10}5/129 & 0/40 & 0/40 & \cellcolor{green!12.50}5/40 & - & 0/173 & 0/212 \\
\cellcolor{black!10}\# 61 & \cellcolor{black!10}\url{kamailio_parse_to_uri} & \cellcolor{black!10}7 & 0/40 & 0/40 & \cellcolor{green!10}3/40 & - & \cellcolor{green!10}1/215 & 0/150 & 0/40 & 0/40 & 0/40 & - & 0/225 & 0/156 \\
\# 62 & \url{libyang_lyd_parse_data_mem} & 7 & 0/40 & \cellcolor{green!22.50}9/40 & \cellcolor{green!45.00}18/40 & \cellcolor{green!25.00}10/40 & \cellcolor{green!20.25}33/163 & \cellcolor{green!29.47}28/95 & 0/40 & 0/40 & 0/40 & 0/40 & \cellcolor{green!10}16/329 & \cellcolor{green!10}3/94 \\
\cellcolor{black!10}\# 63 & \cellcolor{black!10}\url{bind9_dns_message_parse} & \cellcolor{black!10}8 & 0/40 & 0/40 & 0/40 & - & \cellcolor{green!10}5/410 & \cellcolor{green!10}4/212 & 0/40 & 0/40 & \cellcolor{green!10}1/40 & - & 0/330 & \cellcolor{green!10}1/183 \\
\# 64 & \url{igraph_igraph_read_graph_ncol} & 8 & \cellcolor{green!10}2/40 & 0/40 & \cellcolor{green!10}1/40 & 0/40 & \cellcolor{green!10}1/464 & \cellcolor{green!10}4/156 & 0/40 & 0/40 & 0/40 & 0/40 & \cellcolor{green!10}4/453 & \cellcolor{green!10}1/235 \\
\cellcolor{black!10}\# 65 & \cellcolor{black!10}\url{pjsip_pj_json_parse} & \cellcolor{black!10}8 & 0/40 & \cellcolor{green!10}3/40 & \cellcolor{green!10.00}4/40 & \cellcolor{green!10}2/40 & \cellcolor{green!10}22/323 & \cellcolor{green!10}20/214 & 0/40 & 0/40 & 0/40 & \cellcolor{green!10}3/40 & \cellcolor{green!10}4/444 & \cellcolor{green!10}7/301 \\
\# 66 & \url{pjsip_pj_xml_parse} & 8 & 0/40 & \cellcolor{green!25.00}10/40 & \cellcolor{green!10.00}4/40 & \cellcolor{green!25.00}10/40 & \cellcolor{green!10}18/258 & \cellcolor{green!10}14/258 & 0/40 & 0/40 & 0/40 & \cellcolor{green!17.50}7/40 & \cellcolor{green!10}8/358 & \cellcolor{green!10}2/365 \\
\cellcolor{black!10}\# 67 & \cellcolor{black!10}\url{pjsip_pjmedia_sdp_parse} & \cellcolor{black!10}8 & \cellcolor{green!10}2/40 & \cellcolor{green!22.50}9/40 & \cellcolor{green!10}1/40 & \cellcolor{green!35.00}14/40 & \cellcolor{green!10}19/318 & \cellcolor{green!10}19/236 & 0/40 & \cellcolor{green!10}3/40 & \cellcolor{green!10}1/40 & \cellcolor{green!10.00}4/40 & \cellcolor{green!10}8/351 & \cellcolor{green!10}2/253 \\
\# 68 & \url{quickjs_lre_compile} & 8 & 0/40 & 0/40 & 0/40 & - & 0/322 & \cellcolor{green!10}7/203 & 0/40 & 0/40 & 0/40 & - & 0/505 & 0/253 \\
\cellcolor{black!10}\# 69 & \cellcolor{black!10}\url{bind9_isc_lex_getmastertoken} & \cellcolor{black!10}9 & 0/40 & 0/40 & 0/40 & - & \cellcolor{green!10}6/327 & \cellcolor{green!10}3/161 & 0/40 & 0/40 & 0/40 & - & 0/273 & 0/144 \\
\# 70 & \url{bind9_isc_lex_gettoken} & 9 & 0/40 & 0/40 & 0/40 & - & \cellcolor{green!10}8/447 & \cellcolor{green!10}5/131 & 0/40 & 0/40 & 0/40 & - & 0/342 & \cellcolor{green!10}1/146 \\
\cellcolor{black!10}\# 71 & \cellcolor{black!10}\url{quickjs_JS_Eval} & \cellcolor{black!10}9 & \cellcolor{green!17.50}7/40 & \cellcolor{green!95.00}38/40 & \cellcolor{green!25.00}10/40 & - & \cellcolor{green!27.78}30/108 & \cellcolor{green!10}14/198 & \cellcolor{green!17.50}7/40 & \cellcolor{green!32.50}13/40 & \cellcolor{green!10}3/40 & - & \cellcolor{green!17.90}29/162 & \cellcolor{green!12.96}14/108 \\
\# 72 & \url{igraph_igraph_edge_connectivity} & 10 & 0/40 & 0/40 & 0/40 & 0/40 & 0/378 & 0/146 & 0/40 & 0/40 & 0/40 & 0/40 & 0/377 & 0/220 \\
\cellcolor{black!10}\# 73 & \cellcolor{black!10}\url{pjsip_pj_stun_msg_decode} & \cellcolor{black!10}10 & 0/40 & 0/40 & 0/40 & 0/40 & \cellcolor{green!10}18/348 & \cellcolor{green!10}16/194 & 0/40 & 0/40 & 0/40 & 0/40 & \cellcolor{green!10}1/484 & \cellcolor{green!10}4/202 \\
\# 74 & \url{bind9_dns_message_checksig} & 11 & 0/40 & 0/40 & \cellcolor{green!10}1/40 & - & 0/274 & \cellcolor{green!10}1/178 & 0/40 & 0/40 & 0/40 & - & 0/392 & 0/270 \\
\cellcolor{black!10}\# 75 & \cellcolor{black!10}\url{libzip_zip_fread} & \cellcolor{black!10}11 & \cellcolor{green!57.50}23/40 & \cellcolor{green!45.00}18/40 & \cellcolor{green!55.00}22/40 & \cellcolor{green!32.50}13/40 & \cellcolor{green!10.09}23/228 & \cellcolor{green!12.50}16/128 & \cellcolor{green!10}1/40 & \cellcolor{green!10}2/40 & \cellcolor{green!10.00}4/40 & \cellcolor{green!10}3/40 & \cellcolor{green!10}3/247 & \cellcolor{green!10}5/100 \\
\# 76 & \url{bind9_dns_rdata_fromtext} & 12 & 0/40 & 0/40 & 0/40 & - & 0/329 & \cellcolor{green!10}4/116 & 0/40 & 0/40 & 0/40 & - & 0/240 & 0/110 \\
\cellcolor{black!10}\# 77 & \cellcolor{black!10}\url{igraph_igraph_all_minimal_st_separators} & \cellcolor{black!10}12 & 0/40 & \cellcolor{green!12.50}5/40 & \cellcolor{green!17.50}7/40 & \cellcolor{green!10}1/40 & \cellcolor{green!12.58}20/159 & \cellcolor{green!10.27}15/146 & 0/40 & 0/40 & 0/40 & \cellcolor{green!10}1/40 & \cellcolor{green!10}10/421 & \cellcolor{green!10}3/216 \\
\# 78 & \url{igraph_igraph_minimum_size_separators} & 12 & \cellcolor{green!10}2/40 & \cellcolor{green!10.00}4/40 & \cellcolor{green!37.50}15/40 & \cellcolor{green!10.00}4/40 & \cellcolor{green!10}13/313 & \cellcolor{green!10}9/167 & 0/40 & 0/40 & 0/40 & 0/40 & \cellcolor{green!10}3/215 & \cellcolor{green!10}2/153 \\
\cellcolor{black!10}\# 79 & \cellcolor{black!10}\url{pjsip_pjsip_parse_msg} & \cellcolor{black!10}12 & 0/40 & 0/40 & \cellcolor{green!10}1/40 & 0/40 & 0/453 & \cellcolor{green!10}4/175 & 0/40 & 0/40 & 0/40 & 0/40 & 0/527 & \cellcolor{green!10}1/255 \\
\# 80 & \url{igraph_igraph_automorphism_group} & 13 & 0/40 & 0/40 & \cellcolor{green!62.50}25/40 & 0/40 & 0/440 & \cellcolor{green!11.32}24/212 & 0/40 & 0/40 & 0/40 & 0/40 & 0/348 & \cellcolor{green!10}1/167 \\
\cellcolor{black!10}\# 81 & \cellcolor{black!10}\url{libmodbus_modbus_read_bits} & \cellcolor{black!10}15 & 0/40 & 0/40 & 0/40 & 0/40 & 0/82 & 0/82 & 0/40 & 0/40 & 0/40 & 0/40 & 0/141 & 0/132 \\
\# 82 & \url{libmodbus_modbus_read_registers} & 15 & 0/40 & 0/40 & 0/40 & 0/40 & 0/98 & 0/73 & 0/40 & 0/40 & 0/40 & 0/40 & 0/154 & 0/91 \\
\cellcolor{black!10}\# 83 & \cellcolor{black!10}\url{civetweb_mg_get_response} & \cellcolor{black!10}17 & 0/40 & 0/40 & 0/40 & 0/40 & 0/373 & 0/107 & 0/40 & 0/40 & 0/40 & 0/40 & 0/285 & 0/136 \\
\# 84 & \url{bind9_dns_master_loadbuffer} & 20 & 0/40 & 0/40 & 0/40 & - & \cellcolor{green!10}1/403 & \cellcolor{green!10}2/275 & 0/40 & 0/40 & 0/40 & - & 0/311 & 0/191 \\
\cellcolor{black!10}\# 85 & \cellcolor{black!10}\url{libmodbus_modbus_receive} & \cellcolor{black!10}33 & 0/40 & 0/40 & 0/40 & 0/40 & 0/96 & 0/95 & 0/40 & 0/40 & 0/40 & 0/40 & 0/144 & 0/105 \\
\# 86 & \url{tmux_input_parse_buffer} & 42 & 0/40 & 0/40 & 0/40 & - & 0/396 & 0/319 & 0/40 & 0/40 & 0/40 & - & 0/390 & 0/282 \\
\bottomrule

\end{tabular}

}
\end{table*}




\begin{table*}
\caption{Full List of Evaluation Results}
\label{tab:eval_full}
\renewcommand\arraystretch{1}
\scriptsize
\setlength\arrayrulewidth{2pt}\arrayrulecolor{black}
\resizebox{\textwidth}{!}{
\begin{tabular}{cp{3.6cm} 
        >{\columncolor[gray]{1}[0.8\tabcolsep]}c
        >{\columncolor[gray]{1}[0.8\tabcolsep]}c
        >{\columncolor[gray]{1}[0.8\tabcolsep]}c
        >{\columncolor[gray]{1}[0.8\tabcolsep]}c
        >{\columncolor[gray]{1}[0.8\tabcolsep]}c
        >{\columncolor[gray]{1}[0.8\tabcolsep]}c
        >{\columncolor[gray]{1}[0.8\tabcolsep]}c
        >{\columncolor[gray]{1}[0.8\tabcolsep]}c
        >{\columncolor[gray]{1}[0.8\tabcolsep]}c
        >{\columncolor[gray]{1}[0.8\tabcolsep]}c
        >{\columncolor[gray]{1}[0.8\tabcolsep]}c
        >{\columncolor[gray]{1}[0.8\tabcolsep]}c
        >{\columncolor[gray]{1}[0.8\tabcolsep]}c
    }


\toprule
% Question & SCORE & gpt4-NAIVE & gpt4-BACTX & gpt4-UGCTX & gpt4-DOCTX & gpt4-BA-ITER & gpt4-ALL-ITER & gpt3.5-NAIVE & gpt3.5-BACTX & gpt3.5-UGCTX & gpt3.5-DOCTX & gpt3.5-BA-ITER & gpt3.5-ALL-ITER \\
 \multirow{2}{*}{Index} & \multirow{2}{*}{Question} & \multirow{2}{*}{Score} &   \multicolumn{6}{c}{GPT4} &  \multicolumn{6}{c}{GPT3.5} \\
&  & & NAIVE & BACTX & UGCTX & DOCTX & BA-ITER & EX-ITER &  NAIVE &  BACTX &  UGCTX &  DOCTX &  BA-ITER & EX-ITER \\
\midrule
\cellcolor{black!10}\# 1 & \cellcolor{black!10}\url{coturn_stun_is_command_message_full_check_str} & \cellcolor{black!10}1 & 0.00\% & \cellcolor{green!72.50}72.50\% & 0.00\% & - & \cellcolor{green!53.23}53.23\% & \cellcolor{green!16.04}16.04\% & 0.00\% & \cellcolor{green!67.50}67.50\% & \cellcolor{green!12.50}12.50\% & - & \cellcolor{green!42.25}42.25\% & \cellcolor{green!32.00}32.00\% \\
\# 2 & \url{kamailio_parse_uri} & 1 & 0.00\% & \cellcolor{green!100.00}100.00\% & \cellcolor{green!55.00}55.00\% & - & \cellcolor{green!35.96}35.96\% & \cellcolor{green!23.68}23.68\% & 0.00\% & \cellcolor{green!85.00}85.00\% & \cellcolor{green!52.50}52.50\% & - & \cellcolor{green!35.63}35.63\% & \cellcolor{green!10}6.45\% \\
\cellcolor{black!10}\# 3 & \cellcolor{black!10}\url{coturn_stun_check_message_integrity_str} & \cellcolor{black!10}2 & 0.00\% & \cellcolor{green!30.00}30.00\% & \cellcolor{green!20.00}20.00\% & - & \cellcolor{green!10}5.63\% & \cellcolor{green!10}4.64\% & 0.00\% & \cellcolor{green!20.00}20.00\% & \cellcolor{green!10}5.00\% & - & \cellcolor{green!10}5.77\% & \cellcolor{green!10}4.30\% \\
\# 4 & \url{libiec61850_MmsValue_decodeMmsData} & 2 & 0.00\% & \cellcolor{green!97.50}97.50\% & \cellcolor{green!27.50}27.50\% & \cellcolor{green!92.50}92.50\% & \cellcolor{green!40.96}40.96\% & \cellcolor{green!15.38}15.38\% & 0.00\% & \cellcolor{green!92.50}92.50\% & \cellcolor{green!30.00}30.00\% & \cellcolor{green!67.50}67.50\% & \cellcolor{green!33.98}33.98\% & \cellcolor{green!11.11}11.11\% \\
\cellcolor{black!10}\# 5 & \cellcolor{black!10}\url{md4c_md_html} & \cellcolor{black!10}2 & 0.00\% & 0.00\% & 0.00\% & 0.00\% & \cellcolor{green!48.78}48.78\% & \cellcolor{green!18.03}18.03\% & 0.00\% & 0.00\% & 0.00\% & 0.00\% & \cellcolor{green!32.38}32.38\% & \cellcolor{green!10}2.33\% \\
\# 6 & \url{spdk_spdk_json_parse} & 2 & \cellcolor{green!10.00}10.00\% & \cellcolor{green!87.50}87.50\% & \cellcolor{green!15.00}15.00\% & - & \cellcolor{green!46.48}46.48\% & \cellcolor{green!20.43}20.43\% & 0.00\% & \cellcolor{green!75.00}75.00\% & \cellcolor{green!15.00}15.00\% & - & \cellcolor{green!44.16}44.16\% & \cellcolor{green!10.43}10.43\% \\
\cellcolor{black!10}\# 7 & \cellcolor{black!10}\url{croaring_roaring_bitmap_portable_deserialize_safe} & \cellcolor{black!10}3 & \cellcolor{green!20.00}20.00\% & \cellcolor{green!100.00}100.00\% & \cellcolor{green!72.50}72.50\% & \cellcolor{green!70.00}70.00\% & \cellcolor{green!32.29}32.29\% & \cellcolor{green!24.75}24.75\% & \cellcolor{green!27.50}27.50\% & \cellcolor{green!55.00}55.00\% & \cellcolor{green!25.00}25.00\% & \cellcolor{green!75.00}75.00\% & \cellcolor{green!27.45}27.45\% & \cellcolor{green!17.60}17.60\% \\
\# 8 & \url{lua_luaL_loadbufferx} & 3 & \cellcolor{green!67.50}67.50\% & \cellcolor{green!97.50}97.50\% & \cellcolor{green!60.00}60.00\% & \cellcolor{green!100.00}100.00\% & \cellcolor{green!72.00}72.00\% & \cellcolor{green!41.33}41.33\% & \cellcolor{green!77.50}77.50\% & \cellcolor{green!82.50}82.50\% & \cellcolor{green!42.50}42.50\% & \cellcolor{green!85.00}85.00\% & \cellcolor{green!54.72}54.72\% & \cellcolor{green!13.51}13.51\% \\
\cellcolor{black!10}\# 9 & \cellcolor{black!10}\url{w3m_wc_Str_conv_with_detect} & \cellcolor{black!10}3 & 0.00\% & 0.00\% & \cellcolor{green!25.00}25.00\% & - & \cellcolor{green!10}0.57\% & \cellcolor{green!10}9.55\% & 0.00\% & 0.00\% & \cellcolor{green!25.00}25.00\% & - & \cellcolor{green!10}0.27\% & \cellcolor{green!10}4.35\% \\
\# 10 & \url{bind9_dns_name_fromwire} & 4 & 0.00\% & \cellcolor{green!20.00}20.00\% & \cellcolor{green!10}7.50\% & - & \cellcolor{green!10}2.17\% & \cellcolor{green!10}5.58\% & 0.00\% & 0.00\% & \cellcolor{green!10}2.50\% & - & 0.00\% & \cellcolor{green!10}0.61\% \\
\cellcolor{black!10}\# 11 & \cellcolor{black!10}\url{gdk-pixbuf_gdk_pixbuf_animation_new_from_file} & \cellcolor{black!10}4 & \cellcolor{green!15.00}15.00\% & \cellcolor{green!82.50}82.50\% & \cellcolor{green!37.50}37.50\% & \cellcolor{green!67.50}67.50\% & \cellcolor{green!32.20}32.20\% & \cellcolor{green!10}9.09\% & \cellcolor{green!10}7.50\% & \cellcolor{green!27.50}27.50\% & \cellcolor{green!10}7.50\% & \cellcolor{green!22.50}22.50\% & \cellcolor{green!10}9.46\% & \cellcolor{green!10}5.06\% \\
\# 12 & \url{gdk-pixbuf_gdk_pixbuf_new_from_data} & 4 & \cellcolor{green!10}5.00\% & \cellcolor{green!40.00}40.00\% & \cellcolor{green!27.50}27.50\% & \cellcolor{green!27.50}27.50\% & \cellcolor{green!24.75}24.75\% & \cellcolor{green!14.10}14.10\% & \cellcolor{green!45.00}45.00\% & \cellcolor{green!62.50}62.50\% & \cellcolor{green!15.00}15.00\% & \cellcolor{green!57.50}57.50\% & \cellcolor{green!25.00}25.00\% & \cellcolor{green!12.62}12.62\% \\
\cellcolor{black!10}\# 13 & \cellcolor{black!10}\url{gdk-pixbuf_gdk_pixbuf_new_from_file} & \cellcolor{black!10}4 & \cellcolor{green!25.00}25.00\% & \cellcolor{green!97.50}97.50\% & \cellcolor{green!50.00}50.00\% & \cellcolor{green!92.50}92.50\% & \cellcolor{green!30.88}30.88\% & \cellcolor{green!18.03}18.03\% & \cellcolor{green!12.50}12.50\% & \cellcolor{green!35.00}35.00\% & \cellcolor{green!12.50}12.50\% & \cellcolor{green!20.00}20.00\% & \cellcolor{green!10}4.71\% & \cellcolor{green!10}2.56\% \\
\# 14 & \url{gdk-pixbuf_gdk_pixbuf_new_from_stream} & 4 & \cellcolor{green!12.50}12.50\% & \cellcolor{green!75.00}75.00\% & \cellcolor{green!60.00}60.00\% & \cellcolor{green!65.00}65.00\% & \cellcolor{green!60.00}60.00\% & \cellcolor{green!37.88}37.88\% & \cellcolor{green!52.50}52.50\% & \cellcolor{green!87.50}87.50\% & \cellcolor{green!45.00}45.00\% & \cellcolor{green!80.00}80.00\% & \cellcolor{green!36.71}36.71\% & \cellcolor{green!18.68}18.68\% \\
\cellcolor{black!10}\# 15 & \cellcolor{black!10}\url{gpac_gf_isom_open_file} & \cellcolor{black!10}4 & \cellcolor{green!10}2.50\% & \cellcolor{green!67.50}67.50\% & \cellcolor{green!50.00}50.00\% & - & \cellcolor{green!10}2.91\% & \cellcolor{green!10}4.59\% & 0.00\% & \cellcolor{green!12.50}12.50\% & 0.00\% & - & 0.00\% & 0.00\% \\
\# 16 & \url{libbpf_bpf_object__open_mem} & 4 & \cellcolor{green!10}2.50\% & \cellcolor{green!15.00}15.00\% & \cellcolor{green!10.00}10.00\% & \cellcolor{green!15.00}15.00\% & \cellcolor{green!10.50}10.50\% & \cellcolor{green!10}7.56\% & 0.00\% & \cellcolor{green!27.50}27.50\% & \cellcolor{green!15.00}15.00\% & \cellcolor{green!12.50}12.50\% & \cellcolor{green!11.11}11.11\% & \cellcolor{green!10}8.62\% \\
\cellcolor{black!10}\# 17 & \cellcolor{black!10}\url{libpg_query_pg_query_parse} & \cellcolor{black!10}4 & \cellcolor{green!10.00}10.00\% & \cellcolor{green!90.00}90.00\% & \cellcolor{green!95.00}95.00\% & - & \cellcolor{green!34.09}34.09\% & \cellcolor{green!34.62}34.62\% & \cellcolor{green!15.00}15.00\% & \cellcolor{green!42.50}42.50\% & \cellcolor{green!65.00}65.00\% & - & \cellcolor{green!21.95}21.95\% & \cellcolor{green!14.89}14.89\% \\
\# 18 & \url{libucl_ucl_parser_add_string} & 4 & \cellcolor{green!20.00}20.00\% & \cellcolor{green!47.50}47.50\% & \cellcolor{green!50.00}50.00\% & \cellcolor{green!72.50}72.50\% & \cellcolor{green!22.61}22.61\% & \cellcolor{green!16.95}16.95\% & \cellcolor{green!17.50}17.50\% & \cellcolor{green!20.00}20.00\% & \cellcolor{green!12.50}12.50\% & \cellcolor{green!45.00}45.00\% & \cellcolor{green!16.99}16.99\% & \cellcolor{green!15.91}15.91\% \\
\cellcolor{black!10}\# 19 & \cellcolor{black!10}\url{oniguruma_onig_new} & \cellcolor{black!10}4 & \cellcolor{green!50.00}50.00\% & \cellcolor{green!87.50}87.50\% & \cellcolor{green!55.00}55.00\% & \cellcolor{green!82.50}82.50\% & \cellcolor{green!21.66}21.66\% & \cellcolor{green!26.47}26.47\% & \cellcolor{green!30.00}30.00\% & \cellcolor{green!45.00}45.00\% & \cellcolor{green!12.50}12.50\% & \cellcolor{green!37.50}37.50\% & \cellcolor{green!22.88}22.88\% & \cellcolor{green!12.96}12.96\% \\
\# 20 & \url{pupnp_ixmlLoadDocumentEx} & 4 & 0.00\% & \cellcolor{green!65.00}65.00\% & \cellcolor{green!75.00}75.00\% & \cellcolor{green!40.00}40.00\% & \cellcolor{green!11.43}11.43\% & \cellcolor{green!10}8.59\% & 0.00\% & \cellcolor{green!17.50}17.50\% & \cellcolor{green!20.00}20.00\% & \cellcolor{green!10}2.50\% & \cellcolor{green!10}3.70\% & \cellcolor{green!10}1.87\% \\
\cellcolor{black!10}\# 21 & \cellcolor{black!10}\url{gdk-pixbuf_gdk_pixbuf_new_from_file_at_scale} & \cellcolor{black!10}5 & \cellcolor{green!45.00}45.00\% & \cellcolor{green!72.50}72.50\% & \cellcolor{green!40.00}40.00\% & \cellcolor{green!70.00}70.00\% & \cellcolor{green!12.77}12.77\% & \cellcolor{green!10}9.20\% & \cellcolor{green!10}5.00\% & \cellcolor{green!10}2.50\% & \cellcolor{green!10}7.50\% & \cellcolor{green!17.50}17.50\% & 0.00\% & \cellcolor{green!10}1.02\% \\
\# 22 & \url{inchi_GetINCHIKeyFromINCHI} & 5 & 0.00\% & \cellcolor{green!75.00}75.00\% & \cellcolor{green!22.50}22.50\% & \cellcolor{green!77.50}77.50\% & \cellcolor{green!35.11}35.11\% & \cellcolor{green!26.04}26.04\% & 0.00\% & \cellcolor{green!40.00}40.00\% & \cellcolor{green!25.00}25.00\% & \cellcolor{green!60.00}60.00\% & \cellcolor{green!20.95}20.95\% & \cellcolor{green!10.96}10.96\% \\
\cellcolor{black!10}\# 23 & \cellcolor{black!10}\url{libdwarf_dwarf_init_b} & \cellcolor{black!10}5 & 0.00\% & 0.00\% & \cellcolor{green!32.50}32.50\% & \cellcolor{green!10}2.50\% & \cellcolor{green!10}5.49\% & \cellcolor{green!16.07}16.07\% & 0.00\% & 0.00\% & \cellcolor{green!27.50}27.50\% & \cellcolor{green!12.50}12.50\% & \cellcolor{green!10}1.66\% & \cellcolor{green!10}9.89\% \\
\# 24 & \url{libdwarf_dwarf_init_path} & 5 & 0.00\% & 0.00\% & \cellcolor{green!20.00}20.00\% & 0.00\% & 0.00\% & \cellcolor{green!10}1.64\% & 0.00\% & 0.00\% & \cellcolor{green!10.00}10.00\% & 0.00\% & 0.00\% & \cellcolor{green!10}1.12\% \\
\cellcolor{black!10}\# 25 & \cellcolor{black!10}\url{liblouis_lou_compileString} & \cellcolor{black!10}5 & 0.00\% & \cellcolor{green!22.50}22.50\% & \cellcolor{green!50.00}50.00\% & \cellcolor{green!20.00}20.00\% & \cellcolor{green!10}9.89\% & \cellcolor{green!11.68}11.68\% & 0.00\% & \cellcolor{green!17.50}17.50\% & \cellcolor{green!12.50}12.50\% & \cellcolor{green!30.00}30.00\% & \cellcolor{green!10}2.15\% & \cellcolor{green!10}2.33\% \\
\# 26 & \url{selinux_cil_compile} & 5 & 0.00\% & 0.00\% & \cellcolor{green!62.50}62.50\% & - & \cellcolor{green!10}1.13\% & \cellcolor{green!26.04}26.04\% & 0.00\% & 0.00\% & \cellcolor{green!27.50}27.50\% & - & \cellcolor{green!10}0.34\% & \cellcolor{green!10}9.89\% \\
\cellcolor{black!10}\# 27 & \cellcolor{black!10}\url{bind9_dns_name_fromtext} & \cellcolor{black!10}6 & 0.00\% & \cellcolor{green!55.00}55.00\% & \cellcolor{green!15.00}15.00\% & - & \cellcolor{green!10}9.25\% & \cellcolor{green!10}8.70\% & 0.00\% & 0.00\% & \cellcolor{green!10}7.50\% & - & 0.00\% & \cellcolor{green!10}1.78\% \\
\# 28 & \url{bind9_dns_rdata_fromwire} & 6 & 0.00\% & 0.00\% & \cellcolor{green!10}2.50\% & - & \cellcolor{green!10}0.60\% & \cellcolor{green!10}3.12\% & 0.00\% & 0.00\% & \cellcolor{green!10}2.50\% & - & 0.00\% & 0.00\% \\
\cellcolor{black!10}\# 29 & \cellcolor{black!10}\url{coturn_stun_is_binding_response} & \cellcolor{black!10}6 & 0.00\% & \cellcolor{green!60.00}60.00\% & \cellcolor{green!30.00}30.00\% & - & \cellcolor{green!16.42}16.42\% & \cellcolor{green!13.68}13.68\% & 0.00\% & 0.00\% & \cellcolor{green!35.00}35.00\% & - & \cellcolor{green!10}2.65\% & \cellcolor{green!10}7.23\% \\
\# 30 & \url{coturn_stun_is_command_message} & 6 & 0.00\% & \cellcolor{green!30.00}30.00\% & \cellcolor{green!37.50}37.50\% & \cellcolor{green!27.50}27.50\% & \cellcolor{green!10.06}10.06\% & \cellcolor{green!12.70}12.70\% & 0.00\% & 0.00\% & \cellcolor{green!42.50}42.50\% & 0.00\% & \cellcolor{green!10}0.52\% & \cellcolor{green!12.79}12.79\% \\
\cellcolor{black!10}\# 31 & \cellcolor{black!10}\url{coturn_stun_is_response} & \cellcolor{black!10}6 & 0.00\% & \cellcolor{green!35.00}35.00\% & 0.00\% & - & \cellcolor{green!10}8.50\% & \cellcolor{green!10}6.96\% & 0.00\% & 0.00\% & \cellcolor{green!25.00}25.00\% & - & 0.00\% & \cellcolor{green!10}1.60\% \\
\# 32 & \url{coturn_stun_is_success_response} & 6 & 0.00\% & \cellcolor{green!40.00}40.00\% & \cellcolor{green!37.50}37.50\% & - & \cellcolor{green!12.59}12.59\% & \cellcolor{green!15.73}15.73\% & 0.00\% & 0.00\% & \cellcolor{green!15.00}15.00\% & - & \cellcolor{green!10}2.22\% & \cellcolor{green!10}6.90\% \\
\cellcolor{black!10}\# 33 & \cellcolor{black!10}\url{hiredis_redisFormatCommand} & \cellcolor{black!10}6 & \cellcolor{green!32.50}32.50\% & \cellcolor{green!100.00}100.00\% & \cellcolor{green!22.50}22.50\% & - & \cellcolor{green!10.55}10.55\% & \cellcolor{green!13.73}13.73\% & \cellcolor{green!10}5.00\% & \cellcolor{green!77.50}77.50\% & \cellcolor{green!20.00}20.00\% & - & \cellcolor{green!10}4.18\% & \cellcolor{green!10}4.32\% \\
\# 34 & \url{igraph_igraph_read_graph_dl} & 6 & \cellcolor{green!22.50}22.50\% & 0.00\% & \cellcolor{green!10.00}10.00\% & 0.00\% & \cellcolor{green!10}3.38\% & \cellcolor{green!10}6.36\% & 0.00\% & 0.00\% & 0.00\% & 0.00\% & \cellcolor{green!10}1.02\% & \cellcolor{green!10}0.76\% \\
\cellcolor{black!10}\# 35 & \cellcolor{black!10}\url{igraph_igraph_read_graph_edgelist} & \cellcolor{black!10}6 & \cellcolor{green!15.00}15.00\% & 0.00\% & \cellcolor{green!10}5.00\% & \cellcolor{green!10}2.50\% & \cellcolor{green!10}3.56\% & \cellcolor{green!10}1.97\% & 0.00\% & 0.00\% & 0.00\% & 0.00\% & \cellcolor{green!10}0.68\% & \cellcolor{green!10}0.83\% \\
\# 36 & \url{igraph_igraph_read_graph_gml} & 6 & \cellcolor{green!10}7.50\% & \cellcolor{green!10}5.00\% & \cellcolor{green!10}2.50\% & 0.00\% & \cellcolor{green!10}4.93\% & \cellcolor{green!10}5.00\% & 0.00\% & 0.00\% & 0.00\% & 0.00\% & \cellcolor{green!10}1.25\% & \cellcolor{green!10}0.53\% \\
\cellcolor{black!10}\# 37 & \cellcolor{black!10}\url{igraph_igraph_read_graph_graphdb} & \cellcolor{black!10}6 & \cellcolor{green!10}7.50\% & 0.00\% & \cellcolor{green!15.00}15.00\% & 0.00\% & \cellcolor{green!10}3.32\% & \cellcolor{green!10}7.87\% & 0.00\% & 0.00\% & \cellcolor{green!10}2.50\% & 0.00\% & \cellcolor{green!10}1.17\% & \cellcolor{green!10}2.45\% \\
\# 38 & \url{igraph_igraph_read_graph_graphml} & 6 & \cellcolor{green!10}7.50\% & \cellcolor{green!10}2.50\% & \cellcolor{green!32.50}32.50\% & \cellcolor{green!10}2.50\% & \cellcolor{green!10}4.20\% & \cellcolor{green!15.75}15.75\% & 0.00\% & 0.00\% & \cellcolor{green!10}5.00\% & 0.00\% & \cellcolor{green!10}4.12\% & \cellcolor{green!10}5.30\% \\
\cellcolor{black!10}\# 39 & \cellcolor{black!10}\url{igraph_igraph_read_graph_lgl} & \cellcolor{black!10}6 & \cellcolor{green!10}2.50\% & 0.00\% & \cellcolor{green!32.50}32.50\% & 0.00\% & \cellcolor{green!10}1.89\% & \cellcolor{green!10}9.27\% & 0.00\% & 0.00\% & 0.00\% & 0.00\% & \cellcolor{green!10}1.78\% & \cellcolor{green!10}1.15\% \\
\# 40 & \url{igraph_igraph_read_graph_pajek} & 6 & \cellcolor{green!15.00}15.00\% & \cellcolor{green!10}2.50\% & \cellcolor{green!10}5.00\% & 0.00\% & \cellcolor{green!10}6.46\% & \cellcolor{green!10}9.88\% & 0.00\% & 0.00\% & 0.00\% & 0.00\% & \cellcolor{green!10}1.16\% & \cellcolor{green!10}0.88\% \\
\cellcolor{black!10}\# 41 & \cellcolor{black!10}\url{inchi_GetINCHIfromINCHI} & \cellcolor{black!10}6 & \cellcolor{green!20.00}20.00\% & \cellcolor{green!67.50}67.50\% & \cellcolor{green!47.50}47.50\% & \cellcolor{green!80.00}80.00\% & \cellcolor{green!10}1.48\% & 0.00\% & 0.00\% & \cellcolor{green!10}2.50\% & \cellcolor{green!20.00}20.00\% & \cellcolor{green!12.50}12.50\% & 0.00\% & \cellcolor{green!10}0.78\% \\
\# 42 & \url{inchi_GetStructFromINCHI} & 6 & 0.00\% & \cellcolor{green!52.50}52.50\% & \cellcolor{green!22.50}22.50\% & \cellcolor{green!22.50}22.50\% & \cellcolor{green!10}1.37\% & \cellcolor{green!10}2.27\% & 0.00\% & \cellcolor{green!10}2.50\% & \cellcolor{green!10.00}10.00\% & \cellcolor{green!10}5.00\% & \cellcolor{green!10}0.32\% & \cellcolor{green!10}0.26\% \\
\cellcolor{black!10}\# 43 & \cellcolor{black!10}\url{kamailio_parse_msg} & \cellcolor{black!10}6 & 0.00\% & \cellcolor{green!57.50}57.50\% & \cellcolor{green!22.50}22.50\% & - & \cellcolor{green!16.36}16.36\% & \cellcolor{green!15.08}15.08\% & 0.00\% & \cellcolor{green!32.50}32.50\% & \cellcolor{green!30.00}30.00\% & - & \cellcolor{green!10}5.92\% & \cellcolor{green!10}6.01\% \\
\# 44 & \url{libyang_lys_parse_mem} & 6 & \cellcolor{green!10}2.50\% & \cellcolor{green!10}5.00\% & \cellcolor{green!20.00}20.00\% & \cellcolor{green!10.00}10.00\% & \cellcolor{green!10}6.62\% & \cellcolor{green!13.14}13.14\% & 0.00\% & 0.00\% & \cellcolor{green!10}2.50\% & 0.00\% & \cellcolor{green!10}3.05\% & \cellcolor{green!10}5.24\% \\
\cellcolor{black!10}\# 45 & \cellcolor{black!10}\url{proftpd_pr_json_object_from_text} & \cellcolor{black!10}6 & 0.00\% & 0.00\% & \cellcolor{green!72.50}72.50\% & - & \cellcolor{green!10}2.64\% & \cellcolor{green!10}7.45\% & 0.00\% & 0.00\% & \cellcolor{green!12.50}12.50\% & - & \cellcolor{green!10}0.32\% & \cellcolor{green!10}1.97\% \\
\# 46 & \url{selinux_policydb_read} & 6 & 0.00\% & \cellcolor{green!30.00}30.00\% & \cellcolor{green!42.50}42.50\% & - & \cellcolor{green!10.95}10.95\% & \cellcolor{green!13.01}13.01\% & 0.00\% & \cellcolor{green!12.50}12.50\% & \cellcolor{green!10}5.00\% & - & \cellcolor{green!10}3.68\% & \cellcolor{green!10}5.98\% \\
\cellcolor{black!10}\# 47 & \cellcolor{black!10}\url{kamailio_get_src_address_socket} & \cellcolor{black!10}7 & 0.00\% & \cellcolor{green!10}5.00\% & \cellcolor{green!27.50}27.50\% & 0.00\% & \cellcolor{green!10}2.50\% & \cellcolor{green!10}3.53\% & 0.00\% & 0.00\% & \cellcolor{green!35.00}35.00\% & 0.00\% & 0.00\% & \cellcolor{green!10}4.15\% \\
\# 48 & \url{kamailio_get_src_uri} & 7 & 0.00\% & 0.00\% & \cellcolor{green!22.50}22.50\% & \cellcolor{green!10}2.50\% & \cellcolor{green!10}1.85\% & \cellcolor{green!10}5.13\% & 0.00\% & \cellcolor{green!10}2.50\% & \cellcolor{green!20.00}20.00\% & 0.00\% & \cellcolor{green!10}0.69\% & \cellcolor{green!10}1.06\% \\
\cellcolor{black!10}\# 49 & \cellcolor{black!10}\url{kamailio_parse_content_disposition} & \cellcolor{black!10}7 & 0.00\% & 0.00\% & \cellcolor{green!10.00}10.00\% & 0.00\% & 0.00\% & \cellcolor{green!10}1.65\% & 0.00\% & 0.00\% & 0.00\% & 0.00\% & 0.00\% & \cellcolor{green!10}1.25\% \\
\# 50 & \url{kamailio_parse_diversion_header} & 7 & 0.00\% & 0.00\% & \cellcolor{green!27.50}27.50\% & 0.00\% & 0.00\% & \cellcolor{green!10}6.35\% & 0.00\% & 0.00\% & \cellcolor{green!10}7.50\% & 0.00\% & 0.00\% & \cellcolor{green!10}0.56\% \\
\cellcolor{black!10}\# 51 & \cellcolor{black!10}\url{kamailio_parse_from_header} & \cellcolor{black!10}7 & 0.00\% & 0.00\% & 0.00\% & - & 0.00\% & 0.00\% & 0.00\% & 0.00\% & 0.00\% & - & 0.00\% & \cellcolor{green!10}0.58\% \\
\# 52 & \url{kamailio_parse_from_uri} & 7 & 0.00\% & 0.00\% & \cellcolor{green!10}2.50\% & - & \cellcolor{green!10}0.34\% & \cellcolor{green!10}0.45\% & 0.00\% & 0.00\% & 0.00\% & - & 0.00\% & 0.00\% \\
\cellcolor{black!10}\# 53 & \cellcolor{black!10}\url{kamailio_parse_headers} & \cellcolor{black!10}7 & 0.00\% & 0.00\% & 0.00\% & - & 0.00\% & \cellcolor{green!10}0.83\% & 0.00\% & 0.00\% & 0.00\% & - & 0.00\% & 0.00\% \\
\# 54 & \url{kamailio_parse_identityinfo_header} & 7 & 0.00\% & 0.00\% & \cellcolor{green!50.00}50.00\% & - & 0.00\% & \cellcolor{green!10.22}10.22\% & 0.00\% & 0.00\% & \cellcolor{green!17.50}17.50\% & - & 0.00\% & \cellcolor{green!10}2.37\% \\
\cellcolor{black!10}\# 55 & \cellcolor{black!10}\url{kamailio_parse_pai_header} & \cellcolor{black!10}7 & 0.00\% & 0.00\% & \cellcolor{green!10.00}10.00\% & - & 0.00\% & \cellcolor{green!10}1.92\% & 0.00\% & 0.00\% & \cellcolor{green!10}5.00\% & - & 0.00\% & \cellcolor{green!10}0.52\% \\
\# 56 & \url{kamailio_parse_privacy} & 7 & 0.00\% & 0.00\% & \cellcolor{green!22.50}22.50\% & 0.00\% & 0.00\% & \cellcolor{green!10}5.41\% & 0.00\% & 0.00\% & \cellcolor{green!10}5.00\% & 0.00\% & 0.00\% & 0.00\% \\
\cellcolor{black!10}\# 57 & \cellcolor{black!10}\url{kamailio_parse_record_route_headers} & \cellcolor{black!10}7 & 0.00\% & 0.00\% & \cellcolor{green!100.00}100.00\% & - & \cellcolor{green!10}0.45\% & \cellcolor{green!21.13}21.13\% & 0.00\% & 0.00\% & \cellcolor{green!12.50}12.50\% & - & 0.00\% & \cellcolor{green!10}4.92\% \\
\# 58 & \url{kamailio_parse_refer_to_header} & 7 & 0.00\% & 0.00\% & \cellcolor{green!17.50}17.50\% & - & 0.00\% & \cellcolor{green!10}3.82\% & 0.00\% & 0.00\% & \cellcolor{green!10.00}10.00\% & - & 0.00\% & \cellcolor{green!10}1.01\% \\
\cellcolor{black!10}\# 59 & \cellcolor{black!10}\url{kamailio_parse_route_headers} & \cellcolor{black!10}7 & 0.00\% & 0.00\% & \cellcolor{green!87.50}87.50\% & - & 0.00\% & \cellcolor{green!11.85}11.85\% & 0.00\% & \cellcolor{green!10}2.50\% & \cellcolor{green!87.50}87.50\% & - & 0.00\% & \cellcolor{green!10}3.73\% \\
\# 60 & \url{kamailio_parse_to_header} & 7 & 0.00\% & 0.00\% & \cellcolor{green!12.50}12.50\% & - & 0.00\% & \cellcolor{green!10}3.88\% & 0.00\% & 0.00\% & \cellcolor{green!12.50}12.50\% & - & 0.00\% & 0.00\% \\
\cellcolor{black!10}\# 61 & \cellcolor{black!10}\url{kamailio_parse_to_uri} & \cellcolor{black!10}7 & 0.00\% & 0.00\% & \cellcolor{green!10}7.50\% & - & \cellcolor{green!10}0.47\% & 0.00\% & 0.00\% & 0.00\% & 0.00\% & - & 0.00\% & 0.00\% \\
\# 62 & \url{libyang_lyd_parse_data_mem} & 7 & 0.00\% & \cellcolor{green!22.50}22.50\% & \cellcolor{green!45.00}45.00\% & \cellcolor{green!25.00}25.00\% & \cellcolor{green!20.25}20.25\% & \cellcolor{green!29.47}29.47\% & 0.00\% & 0.00\% & 0.00\% & 0.00\% & \cellcolor{green!10}4.86\% & \cellcolor{green!10}3.19\% \\
\cellcolor{black!10}\# 63 & \cellcolor{black!10}\url{bind9_dns_message_parse} & \cellcolor{black!10}8 & 0.00\% & 0.00\% & 0.00\% & - & \cellcolor{green!10}1.22\% & \cellcolor{green!10}1.89\% & 0.00\% & 0.00\% & \cellcolor{green!10}2.50\% & - & 0.00\% & \cellcolor{green!10}0.55\% \\
\# 64 & \url{igraph_igraph_read_graph_ncol} & 8 & \cellcolor{green!10}5.00\% & 0.00\% & \cellcolor{green!10}2.50\% & 0.00\% & \cellcolor{green!10}0.22\% & \cellcolor{green!10}2.56\% & 0.00\% & 0.00\% & 0.00\% & 0.00\% & \cellcolor{green!10}0.88\% & \cellcolor{green!10}0.43\% \\
\cellcolor{black!10}\# 65 & \cellcolor{black!10}\url{pjsip_pj_json_parse} & \cellcolor{black!10}8 & 0.00\% & \cellcolor{green!10}7.50\% & \cellcolor{green!10.00}10.00\% & \cellcolor{green!10}5.00\% & \cellcolor{green!10}6.81\% & \cellcolor{green!10}9.35\% & 0.00\% & 0.00\% & 0.00\% & \cellcolor{green!10}7.50\% & \cellcolor{green!10}0.90\% & \cellcolor{green!10}2.33\% \\
\# 66 & \url{pjsip_pj_xml_parse} & 8 & 0.00\% & \cellcolor{green!25.00}25.00\% & \cellcolor{green!10.00}10.00\% & \cellcolor{green!25.00}25.00\% & \cellcolor{green!10}6.98\% & \cellcolor{green!10}5.43\% & 0.00\% & 0.00\% & 0.00\% & \cellcolor{green!17.50}17.50\% & \cellcolor{green!10}2.23\% & \cellcolor{green!10}0.55\% \\
\cellcolor{black!10}\# 67 & \cellcolor{black!10}\url{pjsip_pjmedia_sdp_parse} & \cellcolor{black!10}8 & \cellcolor{green!10}5.00\% & \cellcolor{green!22.50}22.50\% & \cellcolor{green!10}2.50\% & \cellcolor{green!35.00}35.00\% & \cellcolor{green!10}5.97\% & \cellcolor{green!10}8.05\% & 0.00\% & \cellcolor{green!10}7.50\% & \cellcolor{green!10}2.50\% & \cellcolor{green!10.00}10.00\% & \cellcolor{green!10}2.28\% & \cellcolor{green!10}0.79\% \\
\# 68 & \url{quickjs_lre_compile} & 8 & 0.00\% & 0.00\% & 0.00\% & - & 0.00\% & \cellcolor{green!10}3.45\% & 0.00\% & 0.00\% & 0.00\% & - & 0.00\% & 0.00\% \\
\cellcolor{black!10}\# 69 & \cellcolor{black!10}\url{bind9_isc_lex_getmastertoken} & \cellcolor{black!10}9 & 0.00\% & 0.00\% & 0.00\% & - & \cellcolor{green!10}1.83\% & \cellcolor{green!10}1.86\% & 0.00\% & 0.00\% & 0.00\% & - & 0.00\% & 0.00\% \\
\# 70 & \url{bind9_isc_lex_gettoken} & 9 & 0.00\% & 0.00\% & 0.00\% & - & \cellcolor{green!10}1.79\% & \cellcolor{green!10}3.82\% & 0.00\% & 0.00\% & 0.00\% & - & 0.00\% & \cellcolor{green!10}0.68\% \\
\cellcolor{black!10}\# 71 & \cellcolor{black!10}\url{quickjs_JS_Eval} & \cellcolor{black!10}9 & \cellcolor{green!17.50}17.50\% & \cellcolor{green!95.00}95.00\% & \cellcolor{green!25.00}25.00\% & - & \cellcolor{green!27.78}27.78\% & \cellcolor{green!10}7.07\% & \cellcolor{green!17.50}17.50\% & \cellcolor{green!32.50}32.50\% & \cellcolor{green!10}7.50\% & - & \cellcolor{green!17.90}17.90\% & \cellcolor{green!12.96}12.96\% \\
\# 72 & \url{igraph_igraph_edge_connectivity} & 10 & 0.00\% & 0.00\% & 0.00\% & 0.00\% & 0.00\% & 0.00\% & 0.00\% & 0.00\% & 0.00\% & 0.00\% & 0.00\% & 0.00\% \\
\cellcolor{black!10}\# 73 & \cellcolor{black!10}\url{pjsip_pj_stun_msg_decode} & \cellcolor{black!10}10 & 0.00\% & 0.00\% & 0.00\% & 0.00\% & \cellcolor{green!10}5.17\% & \cellcolor{green!10}8.25\% & 0.00\% & 0.00\% & 0.00\% & 0.00\% & \cellcolor{green!10}0.21\% & \cellcolor{green!10}1.98\% \\
\# 74 & \url{bind9_dns_message_checksig} & 11 & 0.00\% & 0.00\% & \cellcolor{green!10}2.50\% & - & 0.00\% & \cellcolor{green!10}0.56\% & 0.00\% & 0.00\% & 0.00\% & - & 0.00\% & 0.00\% \\
\cellcolor{black!10}\# 75 & \cellcolor{black!10}\url{libzip_zip_fread} & \cellcolor{black!10}11 & \cellcolor{green!57.50}57.50\% & \cellcolor{green!45.00}45.00\% & \cellcolor{green!55.00}55.00\% & \cellcolor{green!32.50}32.50\% & \cellcolor{green!10.09}10.09\% & \cellcolor{green!12.50}12.50\% & \cellcolor{green!10}2.50\% & \cellcolor{green!10}5.00\% & \cellcolor{green!10.00}10.00\% & \cellcolor{green!10}7.50\% & \cellcolor{green!10}1.21\% & \cellcolor{green!10}5.00\% \\
\# 76 & \url{bind9_dns_rdata_fromtext} & 12 & 0.00\% & 0.00\% & 0.00\% & - & 0.00\% & \cellcolor{green!10}3.45\% & 0.00\% & 0.00\% & 0.00\% & - & 0.00\% & 0.00\% \\
\cellcolor{black!10}\# 77 & \cellcolor{black!10}\url{igraph_igraph_all_minimal_st_separators} & \cellcolor{black!10}12 & 0.00\% & \cellcolor{green!12.50}12.50\% & \cellcolor{green!17.50}17.50\% & \cellcolor{green!10}2.50\% & \cellcolor{green!12.58}12.58\% & \cellcolor{green!10.27}10.27\% & 0.00\% & 0.00\% & 0.00\% & \cellcolor{green!10}2.50\% & \cellcolor{green!10}2.38\% & \cellcolor{green!10}1.39\% \\
\# 78 & \url{igraph_igraph_minimum_size_separators} & 12 & \cellcolor{green!10}5.00\% & \cellcolor{green!10.00}10.00\% & \cellcolor{green!37.50}37.50\% & \cellcolor{green!10.00}10.00\% & \cellcolor{green!10}4.15\% & \cellcolor{green!10}5.39\% & 0.00\% & 0.00\% & 0.00\% & 0.00\% & \cellcolor{green!10}1.40\% & \cellcolor{green!10}1.31\% \\
\cellcolor{black!10}\# 79 & \cellcolor{black!10}\url{pjsip_pjsip_parse_msg} & \cellcolor{black!10}12 & 0.00\% & 0.00\% & \cellcolor{green!10}2.50\% & 0.00\% & 0.00\% & \cellcolor{green!10}2.29\% & 0.00\% & 0.00\% & 0.00\% & 0.00\% & 0.00\% & \cellcolor{green!10}0.39\% \\
\# 80 & \url{igraph_igraph_automorphism_group} & 13 & 0.00\% & 0.00\% & \cellcolor{green!62.50}62.50\% & 0.00\% & 0.00\% & \cellcolor{green!11.32}11.32\% & 0.00\% & 0.00\% & 0.00\% & 0.00\% & 0.00\% & \cellcolor{green!10}0.60\% \\
\cellcolor{black!10}\# 81 & \cellcolor{black!10}\url{libmodbus_modbus_read_bits} & \cellcolor{black!10}15 & 0.00\% & 0.00\% & 0.00\% & 0.00\% & 0.00\% & 0.00\% & 0.00\% & 0.00\% & 0.00\% & 0.00\% & 0.00\% & 0.00\% \\
\# 82 & \url{libmodbus_modbus_read_registers} & 15 & 0.00\% & 0.00\% & 0.00\% & 0.00\% & 0.00\% & 0.00\% & 0.00\% & 0.00\% & 0.00\% & 0.00\% & 0.00\% & 0.00\% \\
\cellcolor{black!10}\# 83 & \cellcolor{black!10}\url{civetweb_mg_get_response} & \cellcolor{black!10}17 & 0.00\% & 0.00\% & 0.00\% & 0.00\% & 0.00\% & 0.00\% & 0.00\% & 0.00\% & 0.00\% & 0.00\% & 0.00\% & 0.00\% \\
\# 84 & \url{bind9_dns_master_loadbuffer} & 20 & 0.00\% & 0.00\% & 0.00\% & - & \cellcolor{green!10}0.25\% & \cellcolor{green!10}0.73\% & 0.00\% & 0.00\% & 0.00\% & - & 0.00\% & 0.00\% \\
\cellcolor{black!10}\# 85 & \cellcolor{black!10}\url{libmodbus_modbus_receive} & \cellcolor{black!10}33 & 0.00\% & 0.00\% & 0.00\% & 0.00\% & 0.00\% & 0.00\% & 0.00\% & 0.00\% & 0.00\% & 0.00\% & 0.00\% & 0.00\% \\
\# 86 & \url{tmux_input_parse_buffer} & 42 & 0.00\% & 0.00\% & 0.00\% & - & 0.00\% & 0.00\% & 0.00\% & 0.00\% & 0.00\% & - & 0.00\% & 0.00\% \\

\bottomrule

\end{tabular}

}
\end{table*}
% % model: gpt-4-0613, temp: 0.0

\onecolumn
{\small %
\begin{xltabular}[h]{\textwidth}{ccccccccc}
%\begin{table*}[!t]
%\centering
\caption{Evaluation Result of model gpt-4-0613 with temperature 0.0.} \\
%\resizebox{1.0\linewidth}{!}{
%\begin{tabular}{cccccccccc}
\toprule
Index & Question & Score & NAIVE-40 & BACTX-40 & DOCTX-40 & UGCTX-40 & BA-ITER-40 & ALL-ITER-40 \tabularnewline
\midrule
\rowcolor{black!10} 1 & coturn\_stun\_is\_command\_message\_full\_check\_str & 1 & \cellcolor{green!0}{\large 0}/{\footnotesize 40} & \cellcolor{green!100}{\large 40}/{\footnotesize 40} & \cellcolor{green!0}{\large -}{\tiny -} & \cellcolor{green!90}{\large 36}/{\footnotesize 40} & \cellcolor{green!100}{\large 40}/{\footnotesize 40} & \cellcolor{green!80}{\large 38}/{\footnotesize 51} \tabularnewline
2 & kamailio\_parse\_uri & 1 & \cellcolor{green!0}{\large 0}/{\footnotesize 40} & \cellcolor{green!100}{\large 40}/{\footnotesize 40} & \cellcolor{green!0}{\large -}{\tiny -} & \cellcolor{green!100}{\large 39}/{\footnotesize 40} & \cellcolor{green!100}{\large 40}/{\footnotesize 40} & \cellcolor{green!80}{\large 38}/{\footnotesize 50} \tabularnewline
\rowcolor{black!10} 3 & coturn\_stun\_check\_message\_integrity\_str & 2 & \cellcolor{green!0}{\large 0}/{\footnotesize 40} & \cellcolor{green!0}{\large 0}/{\footnotesize 40} & \cellcolor{green!0}{\large -}{\tiny -} & \cellcolor{green!40}{\large 14}/{\footnotesize 40} & \cellcolor{green!40}{\large 37}/{\footnotesize 91} & \cellcolor{green!10}{\large 15}/{\footnotesize 149} \tabularnewline
4 & libiec61850\_MmsValue\_decodeMmsData & 2 & \cellcolor{green!0}{\large 0}/{\footnotesize 40} & \cellcolor{green!70}{\large 25}/{\footnotesize 40} & \cellcolor{green!100}{\large 40}/{\footnotesize 40} & \cellcolor{green!60}{\large 23}/{\footnotesize 40} & \cellcolor{green!70}{\large 40}/{\footnotesize 62} & \cellcolor{green!40}{\large 33}/{\footnotesize 91} \tabularnewline
\rowcolor{black!10} 5 & md4c\_md\_html & 2 & \cellcolor{green!0}{\large 0}/{\footnotesize 40} & \cellcolor{green!0}{\large 0}/{\footnotesize 40} & \cellcolor{green!0}{\large 0}/{\footnotesize 40} & \cellcolor{green!0}{\large 0}/{\footnotesize 40} & \cellcolor{green!50}{\large 40}/{\footnotesize 80} & \cellcolor{green!20}{\large 18}/{\footnotesize 153} \tabularnewline
6 & spdk\_spdk\_json\_parse & 2 & \cellcolor{green!0}{\large 0}/{\footnotesize 40} & \cellcolor{green!100}{\large 40}/{\footnotesize 40} & \cellcolor{green!0}{\large -}{\tiny -} & \cellcolor{green!50}{\large 18}/{\footnotesize 40} & \cellcolor{green!100}{\large 40}/{\footnotesize 40} & \cellcolor{green!40}{\large 30}/{\footnotesize 91} \tabularnewline
\rowcolor{black!10} 7 & croaring\_roaring\_bitmap\_portable\_deserialize\_safe & 3 & \cellcolor{green!60}{\large 24}/{\footnotesize 40} & \cellcolor{green!100}{\large 40}/{\footnotesize 40} & \cellcolor{green!100}{\large 40}/{\footnotesize 40} & \cellcolor{green!70}{\large 26}/{\footnotesize 40} & \cellcolor{green!100}{\large 40}/{\footnotesize 40} & \cellcolor{green!70}{\large 38}/{\footnotesize 59} \tabularnewline
8 & lua\_luaL\_loadbufferx & 3 & \cellcolor{green!100}{\large 40}/{\footnotesize 40} & \cellcolor{green!100}{\large 37}/{\footnotesize 40} & \cellcolor{green!100}{\large 40}/{\footnotesize 40} & \cellcolor{green!80}{\large 29}/{\footnotesize 40} & \cellcolor{green!100}{\large 40}/{\footnotesize 40} & \cellcolor{green!70}{\large 40}/{\footnotesize 59} \tabularnewline
\rowcolor{black!10} 9 & w3m\_wc\_Str\_conv\_with\_detect & 3 & \cellcolor{green!0}{\large 0}/{\footnotesize 40} & \cellcolor{green!30}{\large 11}/{\footnotesize 40} & \cellcolor{green!0}{\large -}{\tiny -} & \cellcolor{green!70}{\large 26}/{\footnotesize 40} & \cellcolor{green!10}{\large 9}/{\footnotesize 173} & \cellcolor{green!40}{\large 32}/{\footnotesize 87} \tabularnewline
10 & bind9\_dns\_name\_fromwire & 4 & \cellcolor{green!0}{\large 0}/{\footnotesize 40} & \cellcolor{green!90}{\large 36}/{\footnotesize 40} & \cellcolor{green!0}{\large -}{\tiny -} & \cellcolor{green!50}{\large 19}/{\footnotesize 40} & \cellcolor{green!80}{\large 38}/{\footnotesize 48} & \cellcolor{green!50}{\large 35}/{\footnotesize 74} \tabularnewline
\rowcolor{black!10} 11 & gdk-pixbuf\_gdk\_pixbuf\_animation\_new\_from\_file & 4 & \cellcolor{green!30}{\large 9}/{\footnotesize 40} & \cellcolor{green!90}{\large 33}/{\footnotesize 40} & \cellcolor{green!60}{\large 23}/{\footnotesize 40} & \cellcolor{green!20}{\large 5}/{\footnotesize 40} & \cellcolor{green!90}{\large 33}/{\footnotesize 40} & \cellcolor{green!20}{\large 8}/{\footnotesize 47} \tabularnewline
12 & gdk-pixbuf\_gdk\_pixbuf\_new\_from\_data & 4 & \cellcolor{green!0}{\large 0}/{\footnotesize 40} & \cellcolor{green!60}{\large 21}/{\footnotesize 40} & \cellcolor{green!0}{\large 0}/{\footnotesize 40} & \cellcolor{green!60}{\large 23}/{\footnotesize 40} & \cellcolor{green!70}{\large 40}/{\footnotesize 61} & \cellcolor{green!30}{\large 27}/{\footnotesize 114} \tabularnewline
\rowcolor{black!10} 13 & gdk-pixbuf\_gdk\_pixbuf\_new\_from\_file & 4 & \cellcolor{green!70}{\large 28}/{\footnotesize 40} & \cellcolor{green!100}{\large 39}/{\footnotesize 40} & \cellcolor{green!40}{\large 15}/{\footnotesize 40} & \cellcolor{green!10}{\large 4}/{\footnotesize 40} & \cellcolor{green!100}{\large 40}/{\footnotesize 40} & \cellcolor{green!20}{\large 7}/{\footnotesize 48} \tabularnewline
14 & gdk-pixbuf\_gdk\_pixbuf\_new\_from\_stream & 4 & \cellcolor{green!100}{\large 40}/{\footnotesize 40} & \cellcolor{green!100}{\large 39}/{\footnotesize 40} & \cellcolor{green!100}{\large 39}/{\footnotesize 40} & \cellcolor{green!90}{\large 36}/{\footnotesize 40} & \cellcolor{green!80}{\large 40}/{\footnotesize 53} & \cellcolor{green!90}{\large 39}/{\footnotesize 45} \tabularnewline
\rowcolor{black!10} 15 & gpac\_gf\_isom\_open\_file & 4 & \cellcolor{green!80}{\large 30}/{\footnotesize 40} & \cellcolor{green!100}{\large 39}/{\footnotesize 40} & \cellcolor{green!0}{\large -}{\tiny -} & \cellcolor{green!80}{\large 32}/{\footnotesize 40} & \cellcolor{green!30}{\large 20}/{\footnotesize 82} & \cellcolor{green!30}{\large 17}/{\footnotesize 79} \tabularnewline
16 & libbpf\_bpf\_object\_\_open\_mem & 4 & \cellcolor{green!0}{\large 0}/{\footnotesize 40} & \cellcolor{green!0}{\large 0}/{\footnotesize 40} & \cellcolor{green!0}{\large 0}/{\footnotesize 40} & \cellcolor{green!20}{\large 8}/{\footnotesize 40} & \cellcolor{green!0}{\large 1}/{\footnotesize 200} & \cellcolor{green!20}{\large 16}/{\footnotesize 144} \tabularnewline
\rowcolor{black!10} 17 & libpg\_query\_pg\_query\_parse & 4 & \cellcolor{green!0}{\large 0}/{\footnotesize 40} & \cellcolor{green!100}{\large 40}/{\footnotesize 40} & \cellcolor{green!0}{\large -}{\tiny -} & \cellcolor{green!100}{\large 38}/{\footnotesize 40} & \cellcolor{green!100}{\large 40}/{\footnotesize 40} & \cellcolor{green!60}{\large 35}/{\footnotesize 61} \tabularnewline
18 & libucl\_ucl\_parser\_add\_string & 4 & \cellcolor{green!70}{\large 27}/{\footnotesize 40} & \cellcolor{green!0}{\large 0}/{\footnotesize 40} & \cellcolor{green!0}{\large 0}/{\footnotesize 40} & \cellcolor{green!30}{\large 11}/{\footnotesize 40} & \cellcolor{green!50}{\large 38}/{\footnotesize 82} & \cellcolor{green!30}{\large 29}/{\footnotesize 94} \tabularnewline
\rowcolor{black!10} 19 & oniguruma\_onig\_new & 4 & \cellcolor{green!30}{\large 12}/{\footnotesize 40} & \cellcolor{green!100}{\large 37}/{\footnotesize 40} & \cellcolor{green!100}{\large 40}/{\footnotesize 40} & \cellcolor{green!70}{\large 28}/{\footnotesize 40} & \cellcolor{green!100}{\large 40}/{\footnotesize 42} & \cellcolor{green!50}{\large 35}/{\footnotesize 72} \tabularnewline
20 & pupnp\_ixmlLoadDocumentEx & 4 & \cellcolor{green!0}{\large 0}/{\footnotesize 40} & \cellcolor{green!0}{\large 0}/{\footnotesize 40} & \cellcolor{green!0}{\large 0}/{\footnotesize 40} & \cellcolor{green!10}{\large 3}/{\footnotesize 40} & \cellcolor{green!0}{\large 0}/{\footnotesize 40} & \cellcolor{green!0}{\large 0}/{\footnotesize 62} \tabularnewline
\rowcolor{black!10} 21 & gdk-pixbuf\_gdk\_pixbuf\_new\_from\_file\_at\_scale & 5 & \cellcolor{green!60}{\large 24}/{\footnotesize 40} & \cellcolor{green!100}{\large 40}/{\footnotesize 40} & \cellcolor{green!100}{\large 40}/{\footnotesize 40} & \cellcolor{green!70}{\large 26}/{\footnotesize 40} & \cellcolor{green!100}{\large 40}/{\footnotesize 40} & \cellcolor{green!30}{\large 18}/{\footnotesize 72} \tabularnewline
22 & inchi\_GetINCHIKeyFromINCHI & 5 & \cellcolor{green!0}{\large 0}/{\footnotesize 40} & \cellcolor{green!60}{\large 22}/{\footnotesize 40} & \cellcolor{green!100}{\large 40}/{\footnotesize 40} & \cellcolor{green!30}{\large 10}/{\footnotesize 40} & \cellcolor{green!30}{\large 25}/{\footnotesize 109} & \cellcolor{green!40}{\large 33}/{\footnotesize 93} \tabularnewline
\rowcolor{black!10} 23 & libdwarf\_dwarf\_init\_b & 5 & \cellcolor{green!0}{\large 0}/{\footnotesize 40} & \cellcolor{green!0}{\large 0}/{\footnotesize 40} & \cellcolor{green!70}{\large 26}/{\footnotesize 40} & \cellcolor{green!50}{\large 17}/{\footnotesize 40} & \cellcolor{green!20}{\large 24}/{\footnotesize 181} & \cellcolor{green!30}{\large 29}/{\footnotesize 106} \tabularnewline
24 & libdwarf\_dwarf\_init\_path & 5 & \cellcolor{green!0}{\large 0}/{\footnotesize 40} & \cellcolor{green!0}{\large 0}/{\footnotesize 40} & \cellcolor{green!0}{\large 0}/{\footnotesize 40} & \cellcolor{green!10}{\large 3}/{\footnotesize 40} & \cellcolor{green!0}{\large 0}/{\footnotesize 160} & \cellcolor{green!10}{\large 2}/{\footnotesize 104} \tabularnewline
\rowcolor{black!10} 25 & liblouis\_lou\_compileString & 5 & \cellcolor{green!0}{\large 0}/{\footnotesize 40} & \cellcolor{green!0}{\large 0}/{\footnotesize 40} & \cellcolor{green!0}{\large 0}/{\footnotesize 40} & \cellcolor{green!50}{\large 17}/{\footnotesize 40} & \cellcolor{green!10}{\large 2}/{\footnotesize 195} & \cellcolor{green!30}{\large 25}/{\footnotesize 104} \tabularnewline
26 & selinux\_cil\_compile & 5 & \cellcolor{green!0}{\large 0}/{\footnotesize 40} & \cellcolor{green!0}{\large 0}/{\footnotesize 40} & \cellcolor{green!0}{\large -}{\tiny -} & \cellcolor{green!100}{\large 39}/{\footnotesize 40} & \cellcolor{green!40}{\large 37}/{\footnotesize 95} & \cellcolor{green!80}{\large 38}/{\footnotesize 51} \tabularnewline
\rowcolor{black!10} 27 & bind9\_dns\_name\_fromtext & 6 & \cellcolor{green!0}{\large 0}/{\footnotesize 40} & \cellcolor{green!90}{\large 33}/{\footnotesize 40} & \cellcolor{green!0}{\large -}{\tiny -} & \cellcolor{green!50}{\large 19}/{\footnotesize 40} & \cellcolor{green!70}{\large 40}/{\footnotesize 57} & \cellcolor{green!40}{\large 32}/{\footnotesize 95} \tabularnewline
28 & bind9\_dns\_rdata\_fromwire & 6 & \cellcolor{green!0}{\large 0}/{\footnotesize 40} & \cellcolor{green!0}{\large 0}/{\footnotesize 40} & \cellcolor{green!0}{\large -}{\tiny -} & \cellcolor{green!10}{\large 4}/{\footnotesize 40} & \cellcolor{green!0}{\large 0}/{\footnotesize 200} & \cellcolor{green!10}{\large 14}/{\footnotesize 163} \tabularnewline
\rowcolor{black!10} 29 & coturn\_stun\_is\_binding\_response & 6 & \cellcolor{green!0}{\large 0}/{\footnotesize 40} & \cellcolor{green!100}{\large 40}/{\footnotesize 40} & \cellcolor{green!0}{\large -}{\tiny -} & \cellcolor{green!60}{\large 24}/{\footnotesize 40} & \cellcolor{green!100}{\large 40}/{\footnotesize 40} & \cellcolor{green!30}{\large 26}/{\footnotesize 89} \tabularnewline
30 & coturn\_stun\_is\_command\_message & 6 & \cellcolor{green!0}{\large 0}/{\footnotesize 40} & \cellcolor{green!100}{\large 40}/{\footnotesize 40} & \cellcolor{green!40}{\large 15}/{\footnotesize 40} & \cellcolor{green!60}{\large 21}/{\footnotesize 40} & \cellcolor{green!100}{\large 40}/{\footnotesize 40} & \cellcolor{green!40}{\large 29}/{\footnotesize 82} \tabularnewline
\rowcolor{black!10} 31 & coturn\_stun\_is\_response & 6 & \cellcolor{green!0}{\large 0}/{\footnotesize 40} & \cellcolor{green!100}{\large 40}/{\footnotesize 40} & \cellcolor{green!0}{\large -}{\tiny -} & \cellcolor{green!60}{\large 22}/{\footnotesize 40} & \cellcolor{green!100}{\large 40}/{\footnotesize 40} & \cellcolor{green!50}{\large 29}/{\footnotesize 60} \tabularnewline
32 & coturn\_stun\_is\_success\_response & 6 & \cellcolor{green!0}{\large 0}/{\footnotesize 40} & \cellcolor{green!100}{\large 40}/{\footnotesize 40} & \cellcolor{green!0}{\large -}{\tiny -} & \cellcolor{green!60}{\large 22}/{\footnotesize 40} & \cellcolor{green!100}{\large 40}/{\footnotesize 40} & \cellcolor{green!40}{\large 31}/{\footnotesize 81} \tabularnewline
\rowcolor{black!10} 33 & hiredis\_redisFormatCommand & 6 & \cellcolor{green!0}{\large 0}/{\footnotesize 40} & \cellcolor{green!80}{\large 32}/{\footnotesize 40} & \cellcolor{green!0}{\large -}{\tiny -} & \cellcolor{green!30}{\large 12}/{\footnotesize 40} & \cellcolor{green!30}{\large 37}/{\footnotesize 144} & \cellcolor{green!20}{\large 19}/{\footnotesize 165} \tabularnewline
34 & igraph\_igraph\_read\_graph\_dl & 6 & \cellcolor{green!0}{\large 0}/{\footnotesize 40} & \cellcolor{green!0}{\large 0}/{\footnotesize 40} & \cellcolor{green!0}{\large 0}/{\footnotesize 40} & \cellcolor{green!0}{\large 0}/{\footnotesize 40} & \cellcolor{green!10}{\large 1}/{\footnotesize 80} & \cellcolor{green!20}{\large 22}/{\footnotesize 146} \tabularnewline
\rowcolor{black!10} 35 & igraph\_igraph\_read\_graph\_edgelist & 6 & \cellcolor{green!10}{\large 1}/{\footnotesize 40} & \cellcolor{green!0}{\large 0}/{\footnotesize 40} & \cellcolor{green!0}{\large 0}/{\footnotesize 40} & \cellcolor{green!0}{\large 0}/{\footnotesize 40} & \cellcolor{green!0}{\large 0}/{\footnotesize 182} & \cellcolor{green!10}{\large 4}/{\footnotesize 189} \tabularnewline
36 & igraph\_igraph\_read\_graph\_gml & 6 & \cellcolor{green!0}{\large 0}/{\footnotesize 40} & \cellcolor{green!0}{\large 0}/{\footnotesize 40} & \cellcolor{green!0}{\large 0}/{\footnotesize 40} & \cellcolor{green!0}{\large 0}/{\footnotesize 40} & \cellcolor{green!30}{\large 31}/{\footnotesize 109} & \cellcolor{green!10}{\large 14}/{\footnotesize 161} \tabularnewline
\rowcolor{black!10} 37 & igraph\_igraph\_read\_graph\_graphdb & 6 & \cellcolor{green!0}{\large 0}/{\footnotesize 40} & \cellcolor{green!0}{\large 0}/{\footnotesize 40} & \cellcolor{green!0}{\large 0}/{\footnotesize 40} & \cellcolor{green!0}{\large 0}/{\footnotesize 40} & \cellcolor{green!0}{\large 0}/{\footnotesize 167} & \cellcolor{green!20}{\large 20}/{\footnotesize 151} \tabularnewline
38 & igraph\_igraph\_read\_graph\_graphml & 6 & \cellcolor{green!0}{\large 0}/{\footnotesize 40} & \cellcolor{green!0}{\large 0}/{\footnotesize 40} & \cellcolor{green!0}{\large 0}/{\footnotesize 40} & \cellcolor{green!40}{\large 15}/{\footnotesize 40} & \cellcolor{green!40}{\large 38}/{\footnotesize 113} & \cellcolor{green!20}{\large 20}/{\footnotesize 138} \tabularnewline
\rowcolor{black!10} 39 & igraph\_igraph\_read\_graph\_lgl & 6 & \cellcolor{green!0}{\large 0}/{\footnotesize 40} & \cellcolor{green!0}{\large 0}/{\footnotesize 40} & \cellcolor{green!0}{\large 0}/{\footnotesize 40} & \cellcolor{green!0}{\large 0}/{\footnotesize 40} & \cellcolor{green!10}{\large 3}/{\footnotesize 153} & \cellcolor{green!20}{\large 18}/{\footnotesize 149} \tabularnewline
40 & igraph\_igraph\_read\_graph\_pajek & 6 & \cellcolor{green!20}{\large 6}/{\footnotesize 40} & \cellcolor{green!0}{\large 0}/{\footnotesize 40} & \cellcolor{green!0}{\large 0}/{\footnotesize 40} & \cellcolor{green!0}{\large 0}/{\footnotesize 40} & \cellcolor{green!50}{\large 40}/{\footnotesize 90} & \cellcolor{green!10}{\large 8}/{\footnotesize 181} \tabularnewline
\rowcolor{black!10} 41 & inchi\_GetINCHIfromINCHI & 6 & \cellcolor{green!0}{\large 0}/{\footnotesize 40} & \cellcolor{green!30}{\large 9}/{\footnotesize 40} & \cellcolor{green!30}{\large 11}/{\footnotesize 40} & \cellcolor{green!70}{\large 27}/{\footnotesize 40} & \cellcolor{green!10}{\large 9}/{\footnotesize 200} & \cellcolor{green!20}{\large 23}/{\footnotesize 197} \tabularnewline
42 & inchi\_GetStructFromINCHI & 6 & \cellcolor{green!0}{\large 0}/{\footnotesize 40} & \cellcolor{green!30}{\large 10}/{\footnotesize 40} & \cellcolor{green!30}{\large 12}/{\footnotesize 40} & \cellcolor{green!40}{\large 13}/{\footnotesize 40} & \cellcolor{green!20}{\large 33}/{\footnotesize 195} & \cellcolor{green!10}{\large 17}/{\footnotesize 200} \tabularnewline
\rowcolor{black!10} 43 & kamailio\_parse\_msg & 6 & \cellcolor{green!0}{\large 0}/{\footnotesize 40} & \cellcolor{green!0}{\large 0}/{\footnotesize 40} & \cellcolor{green!0}{\large -}{\tiny -} & \cellcolor{green!50}{\large 18}/{\footnotesize 40} & \cellcolor{green!20}{\large 26}/{\footnotesize 126} & \cellcolor{green!40}{\large 33}/{\footnotesize 93} \tabularnewline
44 & libyang\_lys\_parse\_mem & 6 & \cellcolor{green!0}{\large 0}/{\footnotesize 40} & \cellcolor{green!0}{\large 0}/{\footnotesize 40} & \cellcolor{green!0}{\large 0}/{\footnotesize 40} & \cellcolor{green!10}{\large 2}/{\footnotesize 40} & \cellcolor{green!30}{\large 33}/{\footnotesize 151} & \cellcolor{green!40}{\large 38}/{\footnotesize 112} \tabularnewline
\rowcolor{black!10} 45 & proftpd\_pr\_json\_object\_from\_text & 6 & \cellcolor{green!0}{\large 0}/{\footnotesize 40} & \cellcolor{green!0}{\large 0}/{\footnotesize 40} & \cellcolor{green!0}{\large -}{\tiny -} & \cellcolor{green!20}{\large 6}/{\footnotesize 40} & \cellcolor{green!0}{\large 1}/{\footnotesize 197} & \cellcolor{green!30}{\large 30}/{\footnotesize 113} \tabularnewline
46 & selinux\_policydb\_read & 6 & \cellcolor{green!0}{\large 0}/{\footnotesize 40} & \cellcolor{green!0}{\large 0}/{\footnotesize 40} & \cellcolor{green!0}{\large -}{\tiny -} & \cellcolor{green!40}{\large 13}/{\footnotesize 40} & \cellcolor{green!10}{\large 9}/{\footnotesize 139} & \cellcolor{green!30}{\large 24}/{\footnotesize 108} \tabularnewline
\rowcolor{black!10} 47 & kamailio\_get\_src\_address\_socket & 7 & \cellcolor{green!0}{\large 0}/{\footnotesize 40} & \cellcolor{green!0}{\large 0}/{\footnotesize 40} & \cellcolor{green!0}{\large 0}/{\footnotesize 40} & \cellcolor{green!40}{\large 13}/{\footnotesize 40} & \cellcolor{green!0}{\large 1}/{\footnotesize 171} & \cellcolor{green!10}{\large 8}/{\footnotesize 154} \tabularnewline
48 & kamailio\_get\_src\_uri & 7 & \cellcolor{green!0}{\large 0}/{\footnotesize 40} & \cellcolor{green!0}{\large 0}/{\footnotesize 40} & \cellcolor{green!0}{\large 0}/{\footnotesize 40} & \cellcolor{green!30}{\large 9}/{\footnotesize 40} & \cellcolor{green!0}{\large 0}/{\footnotesize 156} & \cellcolor{green!10}{\large 3}/{\footnotesize 104} \tabularnewline
\rowcolor{black!10} 49 & kamailio\_parse\_content\_disposition & 7 & \cellcolor{green!0}{\large 0}/{\footnotesize 40} & \cellcolor{green!0}{\large 0}/{\footnotesize 40} & \cellcolor{green!0}{\large 0}/{\footnotesize 40} & \cellcolor{green!30}{\large 10}/{\footnotesize 40} & \cellcolor{green!0}{\large 0}/{\footnotesize 180} & \cellcolor{green!10}{\large 5}/{\footnotesize 174} \tabularnewline
50 & kamailio\_parse\_diversion\_header & 7 & \cellcolor{green!0}{\large 0}/{\footnotesize 40} & \cellcolor{green!0}{\large 0}/{\footnotesize 40} & \cellcolor{green!0}{\large 0}/{\footnotesize 40} & \cellcolor{green!40}{\large 16}/{\footnotesize 40} & \cellcolor{green!0}{\large 0}/{\footnotesize 165} & \cellcolor{green!10}{\large 9}/{\footnotesize 150} \tabularnewline
\rowcolor{black!10} 51 & kamailio\_parse\_from\_header & 7 & \cellcolor{green!0}{\large 0}/{\footnotesize 40} & \cellcolor{green!0}{\large 0}/{\footnotesize 40} & \cellcolor{green!0}{\large -}{\tiny -} & \cellcolor{green!0}{\large 0}/{\footnotesize 40} & \cellcolor{green!0}{\large 0}/{\footnotesize 165} & \cellcolor{green!0}{\large 1}/{\footnotesize 163} \tabularnewline
52 & kamailio\_parse\_from\_uri & 7 & \cellcolor{green!0}{\large 0}/{\footnotesize 40} & \cellcolor{green!0}{\large 0}/{\footnotesize 40} & \cellcolor{green!0}{\large -}{\tiny -} & \cellcolor{green!10}{\large 2}/{\footnotesize 40} & \cellcolor{green!0}{\large 0}/{\footnotesize 191} & \cellcolor{green!10}{\large 3}/{\footnotesize 172} \tabularnewline
\rowcolor{black!10} 53 & kamailio\_parse\_headers & 7 & \cellcolor{green!0}{\large 0}/{\footnotesize 40} & \cellcolor{green!0}{\large 0}/{\footnotesize 40} & \cellcolor{green!0}{\large -}{\tiny -} & \cellcolor{green!0}{\large 0}/{\footnotesize 40} & \cellcolor{green!0}{\large 0}/{\footnotesize 66} & \cellcolor{green!10}{\large 2}/{\footnotesize 122} \tabularnewline
54 & kamailio\_parse\_identityinfo\_header & 7 & \cellcolor{green!0}{\large 0}/{\footnotesize 40} & \cellcolor{green!0}{\large 0}/{\footnotesize 40} & \cellcolor{green!0}{\large -}{\tiny -} & \cellcolor{green!50}{\large 20}/{\footnotesize 40} & \cellcolor{green!0}{\large 0}/{\footnotesize 116} & \cellcolor{green!30}{\large 21}/{\footnotesize 90} \tabularnewline
\rowcolor{black!10} 55 & kamailio\_parse\_pai\_header & 7 & \cellcolor{green!0}{\large 0}/{\footnotesize 40} & \cellcolor{green!0}{\large 0}/{\footnotesize 40} & \cellcolor{green!0}{\large -}{\tiny -} & \cellcolor{green!20}{\large 7}/{\footnotesize 40} & \cellcolor{green!0}{\large 0}/{\footnotesize 125} & \cellcolor{green!10}{\large 7}/{\footnotesize 131} \tabularnewline
56 & kamailio\_parse\_privacy & 7 & \cellcolor{green!0}{\large 0}/{\footnotesize 40} & \cellcolor{green!0}{\large 0}/{\footnotesize 40} & \cellcolor{green!0}{\large 0}/{\footnotesize 40} & \cellcolor{green!20}{\large 6}/{\footnotesize 40} & \cellcolor{green!0}{\large 0}/{\footnotesize 121} & \cellcolor{green!10}{\large 9}/{\footnotesize 132} \tabularnewline
\rowcolor{black!10} 57 & kamailio\_parse\_record\_route\_headers & 7 & \cellcolor{green!0}{\large 0}/{\footnotesize 40} & \cellcolor{green!0}{\large 0}/{\footnotesize 40} & \cellcolor{green!0}{\large -}{\tiny -} & \cellcolor{green!100}{\large 40}/{\footnotesize 40} & \cellcolor{green!0}{\large 1}/{\footnotesize 107} & \cellcolor{green!50}{\large 27}/{\footnotesize 59} \tabularnewline
58 & kamailio\_parse\_refer\_to\_header & 7 & \cellcolor{green!0}{\large 0}/{\footnotesize 40} & \cellcolor{green!0}{\large 0}/{\footnotesize 40} & \cellcolor{green!0}{\large -}{\tiny -} & \cellcolor{green!20}{\large 8}/{\footnotesize 40} & \cellcolor{green!0}{\large 0}/{\footnotesize 116} & \cellcolor{green!10}{\large 8}/{\footnotesize 147} \tabularnewline
\rowcolor{black!10} 59 & kamailio\_parse\_route\_headers & 7 & \cellcolor{green!0}{\large 0}/{\footnotesize 40} & \cellcolor{green!0}{\large 0}/{\footnotesize 40} & \cellcolor{green!0}{\large -}{\tiny -} & \cellcolor{green!100}{\large 40}/{\footnotesize 40} & \cellcolor{green!0}{\large 0}/{\footnotesize 166} & \cellcolor{green!40}{\large 31}/{\footnotesize 76} \tabularnewline
60 & kamailio\_parse\_to\_header & 7 & \cellcolor{green!0}{\large 0}/{\footnotesize 40} & \cellcolor{green!0}{\large 0}/{\footnotesize 40} & \cellcolor{green!0}{\large -}{\tiny -} & \cellcolor{green!20}{\large 6}/{\footnotesize 40} & \cellcolor{green!0}{\large 0}/{\footnotesize 140} & \cellcolor{green!10}{\large 3}/{\footnotesize 152} \tabularnewline
\rowcolor{black!10} 61 & kamailio\_parse\_to\_uri & 7 & \cellcolor{green!0}{\large 0}/{\footnotesize 40} & \cellcolor{green!0}{\large 0}/{\footnotesize 40} & \cellcolor{green!0}{\large -}{\tiny -} & \cellcolor{green!10}{\large 2}/{\footnotesize 40} & \cellcolor{green!0}{\large 0}/{\footnotesize 124} & \cellcolor{green!10}{\large 6}/{\footnotesize 143} \tabularnewline
62 & libyang\_lyd\_parse\_data\_mem & 7 & \cellcolor{green!0}{\large 0}/{\footnotesize 40} & \cellcolor{green!0}{\large 0}/{\footnotesize 40} & \cellcolor{green!0}{\large 0}/{\footnotesize 40} & \cellcolor{green!0}{\large 0}/{\footnotesize 40} & \cellcolor{green!50}{\large 40}/{\footnotesize 83} & \cellcolor{green!30}{\large 30}/{\footnotesize 121} \tabularnewline
\rowcolor{black!10} 63 & bind9\_dns\_message\_parse & 8 & \cellcolor{green!0}{\large 0}/{\footnotesize 40} & \cellcolor{green!0}{\large 0}/{\footnotesize 40} & \cellcolor{green!0}{\large -}{\tiny -} & \cellcolor{green!0}{\large 0}/{\footnotesize 40} & \cellcolor{green!0}{\large 0}/{\footnotesize 200} & \cellcolor{green!10}{\large 3}/{\footnotesize 185} \tabularnewline
64 & igraph\_igraph\_read\_graph\_ncol & 8 & \cellcolor{green!0}{\large 0}/{\footnotesize 40} & \cellcolor{green!0}{\large 0}/{\footnotesize 40} & \cellcolor{green!0}{\large 0}/{\footnotesize 40} & \cellcolor{green!0}{\large 0}/{\footnotesize 40} & \cellcolor{green!0}{\large 0}/{\footnotesize 200} & \cellcolor{green!10}{\large 2}/{\footnotesize 192} \tabularnewline
\rowcolor{black!10} 65 & pjsip\_pj\_json\_parse & 8 & \cellcolor{green!0}{\large 0}/{\footnotesize 40} & \cellcolor{green!0}{\large 0}/{\footnotesize 40} & \cellcolor{green!0}{\large 0}/{\footnotesize 40} & \cellcolor{green!0}{\large 0}/{\footnotesize 40} & \cellcolor{green!10}{\large 20}/{\footnotesize 196} & \cellcolor{green!20}{\large 21}/{\footnotesize 169} \tabularnewline
66 & pjsip\_pj\_xml\_parse & 8 & \cellcolor{green!0}{\large 0}/{\footnotesize 40} & \cellcolor{green!100}{\large 38}/{\footnotesize 40} & \cellcolor{green!10}{\large 4}/{\footnotesize 40} & \cellcolor{green!0}{\large 0}/{\footnotesize 40} & \cellcolor{green!90}{\large 40}/{\footnotesize 45} & \cellcolor{green!10}{\large 11}/{\footnotesize 161} \tabularnewline
\rowcolor{black!10} 67 & pjsip\_pjmedia\_sdp\_parse & 8 & \cellcolor{green!0}{\large 0}/{\footnotesize 40} & \cellcolor{green!100}{\large 38}/{\footnotesize 40} & \cellcolor{green!100}{\large 40}/{\footnotesize 40} & \cellcolor{green!0}{\large 0}/{\footnotesize 40} & \cellcolor{green!100}{\large 40}/{\footnotesize 41} & \cellcolor{green!20}{\large 24}/{\footnotesize 131} \tabularnewline
68 & quickjs\_lre\_compile & 8 & \cellcolor{green!0}{\large 0}/{\footnotesize 40} & \cellcolor{green!0}{\large 0}/{\footnotesize 40} & \cellcolor{green!0}{\large -}{\tiny -} & \cellcolor{green!0}{\large 0}/{\footnotesize 40} & \cellcolor{green!0}{\large 0}/{\footnotesize 200} & \cellcolor{green!10}{\large 6}/{\footnotesize 198} \tabularnewline
\rowcolor{black!10} 69 & bind9\_isc\_lex\_getmastertoken & 9 & \cellcolor{green!0}{\large 0}/{\footnotesize 40} & \cellcolor{green!0}{\large 0}/{\footnotesize 40} & \cellcolor{green!0}{\large -}{\tiny -} & \cellcolor{green!0}{\large 0}/{\footnotesize 40} & \cellcolor{green!10}{\large 17}/{\footnotesize 178} & \cellcolor{green!10}{\large 4}/{\footnotesize 191} \tabularnewline
70 & bind9\_isc\_lex\_gettoken & 9 & \cellcolor{green!0}{\large 0}/{\footnotesize 40} & \cellcolor{green!0}{\large 0}/{\footnotesize 40} & \cellcolor{green!0}{\large -}{\tiny -} & \cellcolor{green!0}{\large 0}/{\footnotesize 40} & \cellcolor{green!10}{\large 12}/{\footnotesize 191} & \cellcolor{green!10}{\large 4}/{\footnotesize 179} \tabularnewline
\rowcolor{black!10} 71 & quickjs\_JS\_Eval & 9 & \cellcolor{green!0}{\large 0}/{\footnotesize 40} & \cellcolor{green!100}{\large 38}/{\footnotesize 40} & \cellcolor{green!0}{\large -}{\tiny -} & \cellcolor{green!50}{\large 17}/{\footnotesize 40} & \cellcolor{green!60}{\large 34}/{\footnotesize 66} & \cellcolor{green!40}{\large 34}/{\footnotesize 89} \tabularnewline
72 & igraph\_igraph\_edge\_connectivity & 10 & \cellcolor{green!0}{\large 0}/{\footnotesize 40} & \cellcolor{green!0}{\large 0}/{\footnotesize 40} & \cellcolor{green!0}{\large 0}/{\footnotesize 40} & \cellcolor{green!0}{\large 0}/{\footnotesize 40} & \cellcolor{green!0}{\large 0}/{\footnotesize 89} & \cellcolor{green!0}{\large 0}/{\footnotesize 133} \tabularnewline
\rowcolor{black!10} 73 & pjsip\_pj\_stun\_msg\_decode & 10 & \cellcolor{green!0}{\large 0}/{\footnotesize 40} & \cellcolor{green!0}{\large 0}/{\footnotesize 40} & \cellcolor{green!0}{\large 0}/{\footnotesize 40} & \cellcolor{green!0}{\large 0}/{\footnotesize 40} & \cellcolor{green!20}{\large 20}/{\footnotesize 175} & \cellcolor{green!20}{\large 23}/{\footnotesize 160} \tabularnewline
74 & bind9\_dns\_message\_checksig & 11 & \cellcolor{green!0}{\large 0}/{\footnotesize 40} & \cellcolor{green!0}{\large 0}/{\footnotesize 40} & \cellcolor{green!0}{\large -}{\tiny -} & \cellcolor{green!0}{\large 0}/{\footnotesize 40} & \cellcolor{green!0}{\large 0}/{\footnotesize 200} & \cellcolor{green!0}{\large 0}/{\footnotesize 183} \tabularnewline
\rowcolor{black!10} 75 & libzip\_zip\_fread & 11 & \cellcolor{green!100}{\large 40}/{\footnotesize 40} & \cellcolor{green!90}{\large 35}/{\footnotesize 40} & \cellcolor{green!0}{\large 0}/{\footnotesize 40} & \cellcolor{green!30}{\large 9}/{\footnotesize 40} & \cellcolor{green!90}{\large 40}/{\footnotesize 48} & \cellcolor{green!40}{\large 35}/{\footnotesize 110} \tabularnewline
76 & bind9\_dns\_rdata\_fromtext & 12 & \cellcolor{green!0}{\large 0}/{\footnotesize 40} & \cellcolor{green!0}{\large 0}/{\footnotesize 40} & \cellcolor{green!0}{\large -}{\tiny -} & \cellcolor{green!10}{\large 1}/{\footnotesize 40} & \cellcolor{green!10}{\large 3}/{\footnotesize 195} & \cellcolor{green!10}{\large 4}/{\footnotesize 194} \tabularnewline
\rowcolor{black!10} 77 & igraph\_igraph\_all\_minimal\_st\_separators & 12 & \cellcolor{green!0}{\large 0}/{\footnotesize 40} & \cellcolor{green!0}{\large 0}/{\footnotesize 40} & \cellcolor{green!0}{\large 0}/{\footnotesize 40} & \cellcolor{green!0}{\large 0}/{\footnotesize 40} & \cellcolor{green!10}{\large 10}/{\footnotesize 171} & \cellcolor{green!10}{\large 12}/{\footnotesize 174} \tabularnewline
78 & igraph\_igraph\_minimum\_size\_separators & 12 & \cellcolor{green!0}{\large 0}/{\footnotesize 40} & \cellcolor{green!0}{\large 0}/{\footnotesize 40} & \cellcolor{green!10}{\large 2}/{\footnotesize 40} & \cellcolor{green!0}{\large 0}/{\footnotesize 40} & \cellcolor{green!10}{\large 6}/{\footnotesize 183} & \cellcolor{green!10}{\large 12}/{\footnotesize 174} \tabularnewline
\rowcolor{black!10} 79 & pjsip\_pjsip\_parse\_msg & 12 & \cellcolor{green!0}{\large 0}/{\footnotesize 40} & \cellcolor{green!0}{\large 0}/{\footnotesize 40} & \cellcolor{green!0}{\large 0}/{\footnotesize 40} & \cellcolor{green!0}{\large 0}/{\footnotesize 40} & \cellcolor{green!0}{\large 0}/{\footnotesize 200} & \cellcolor{green!0}{\large 0}/{\footnotesize 200} \tabularnewline
80 & igraph\_igraph\_automorphism\_group & 13 & \cellcolor{green!0}{\large 0}/{\footnotesize 40} & \cellcolor{green!0}{\large 0}/{\footnotesize 40} & \cellcolor{green!0}{\large 0}/{\footnotesize 40} & \cellcolor{green!0}{\large 0}/{\footnotesize 40} & \cellcolor{green!10}{\large 7}/{\footnotesize 200} & \cellcolor{green!10}{\large 16}/{\footnotesize 166} \tabularnewline
\rowcolor{black!10} 81 & libmodbus\_modbus\_read\_bits & 15 & \cellcolor{green!0}{\large 0}/{\footnotesize 40} & \cellcolor{green!0}{\large 0}/{\footnotesize 40} & \cellcolor{green!0}{\large 0}/{\footnotesize 40} & \cellcolor{green!0}{\large 0}/{\footnotesize 40} & \cellcolor{green!0}{\large 0}/{\footnotesize 68} & \cellcolor{green!0}{\large 0}/{\footnotesize 86} \tabularnewline
82 & libmodbus\_modbus\_read\_registers & 15 & \cellcolor{green!0}{\large 0}/{\footnotesize 40} & \cellcolor{green!0}{\large 0}/{\footnotesize 40} & \cellcolor{green!0}{\large 0}/{\footnotesize 40} & \cellcolor{green!0}{\large 0}/{\footnotesize 40} & \cellcolor{green!0}{\large 0}/{\footnotesize 48} & \cellcolor{green!0}{\large 0}/{\footnotesize 62} \tabularnewline
\rowcolor{black!10} 83 & civetweb\_mg\_get\_response & 17 & \cellcolor{green!0}{\large 0}/{\footnotesize 40} & \cellcolor{green!0}{\large 0}/{\footnotesize 40} & \cellcolor{green!0}{\large 0}/{\footnotesize 40} & \cellcolor{green!0}{\large 0}/{\footnotesize 40} & \cellcolor{green!0}{\large 0}/{\footnotesize 178} & \cellcolor{green!0}{\large 0}/{\footnotesize 121} \tabularnewline
84 & bind9\_dns\_master\_loadbuffer & 20 & \cellcolor{green!0}{\large 0}/{\footnotesize 40} & \cellcolor{green!0}{\large 0}/{\footnotesize 40} & \cellcolor{green!0}{\large -}{\tiny -} & \cellcolor{green!0}{\large 0}/{\footnotesize 40} & \cellcolor{green!0}{\large 0}/{\footnotesize 198} & \cellcolor{green!10}{\large 2}/{\footnotesize 198} \tabularnewline
\rowcolor{black!10} 85 & libmodbus\_modbus\_receive & 33 & \cellcolor{green!0}{\large 0}/{\footnotesize 40} & \cellcolor{green!0}{\large 0}/{\footnotesize 40} & \cellcolor{green!0}{\large 0}/{\footnotesize 40} & \cellcolor{green!0}{\large 0}/{\footnotesize 40} & \cellcolor{green!0}{\large 0}/{\footnotesize 176} & \cellcolor{green!0}{\large 0}/{\footnotesize 111} \tabularnewline
86 & tmux\_input\_parse\_buffer & 42 & \cellcolor{green!0}{\large 0}/{\footnotesize 40} & \cellcolor{green!0}{\large 0}/{\footnotesize 40} & \cellcolor{green!0}{\large -}{\tiny -} & \cellcolor{green!0}{\large 0}/{\footnotesize 40} & \cellcolor{green!0}{\large 0}/{\footnotesize 198} & \cellcolor{green!0}{\large 0}/{\footnotesize 192} \tabularnewline

\bottomrule
%\end{tabular}
%}
%\end{table*}
\end{xltabular}
}
\twocolumn



% model: gpt-4-0613, temp: 0.5

\onecolumn
{\small %
\begin{xltabular}[h]{\textwidth}{ccccccccc}
%\begin{table*}[!t]
%\centering
\caption{Evaluation Result of model gpt-4-0613 with temperature 0.5.} \\
%\resizebox{1.0\linewidth}{!}{
%\begin{tabular}{cccccccccc}
\toprule
Index & Question & Score & NAIVE-40 & BACTX-40 & DOCTX-40 & UGCTX-40 & BA-ITER-40 & ALL-ITER-40 \tabularnewline
\midrule
\rowcolor{black!10} 1 & coturn\_stun\_is\_command\_message\_full\_check\_str & 1 & \cellcolor{green!0}{\large 0}/{\footnotesize 40} & \cellcolor{green!100}{\large 37}/{\footnotesize 40} & \cellcolor{green!0}{\large -}{\tiny -} & \cellcolor{green!90}{\large 35}/{\footnotesize 40} & \cellcolor{green!100}{\large 40}/{\footnotesize 43} & \cellcolor{green!70}{\large 36}/{\footnotesize 59} \tabularnewline
2 & kamailio\_parse\_uri & 1 & \cellcolor{green!0}{\large 0}/{\footnotesize 40} & \cellcolor{green!100}{\large 39}/{\footnotesize 40} & \cellcolor{green!0}{\large -}{\tiny -} & \cellcolor{green!90}{\large 35}/{\footnotesize 40} & \cellcolor{green!100}{\large 40}/{\footnotesize 41} & \cellcolor{green!70}{\large 37}/{\footnotesize 58} \tabularnewline
\rowcolor{black!10} 3 & coturn\_stun\_check\_message\_integrity\_str & 2 & \cellcolor{green!0}{\large 0}/{\footnotesize 40} & \cellcolor{green!10}{\large 1}/{\footnotesize 40} & \cellcolor{green!0}{\large -}{\tiny -} & \cellcolor{green!20}{\large 7}/{\footnotesize 40} & \cellcolor{green!20}{\large 28}/{\footnotesize 144} & \cellcolor{green!10}{\large 16}/{\footnotesize 153} \tabularnewline
4 & libiec61850\_MmsValue\_decodeMmsData & 2 & \cellcolor{green!0}{\large 0}/{\footnotesize 40} & \cellcolor{green!80}{\large 30}/{\footnotesize 40} & \cellcolor{green!70}{\large 28}/{\footnotesize 40} & \cellcolor{green!40}{\large 15}/{\footnotesize 40} & \cellcolor{green!70}{\large 39}/{\footnotesize 56} & \cellcolor{green!50}{\large 36}/{\footnotesize 73} \tabularnewline
\rowcolor{black!10} 5 & md4c\_md\_html & 2 & \cellcolor{green!0}{\large 0}/{\footnotesize 40} & \cellcolor{green!0}{\large 0}/{\footnotesize 40} & \cellcolor{green!0}{\large 0}/{\footnotesize 40} & \cellcolor{green!0}{\large 0}/{\footnotesize 40} & \cellcolor{green!50}{\large 40}/{\footnotesize 83} & \cellcolor{green!30}{\large 32}/{\footnotesize 126} \tabularnewline
6 & spdk\_spdk\_json\_parse & 2 & \cellcolor{green!10}{\large 2}/{\footnotesize 40} & \cellcolor{green!90}{\large 35}/{\footnotesize 40} & \cellcolor{green!0}{\large -}{\tiny -} & \cellcolor{green!40}{\large 15}/{\footnotesize 40} & \cellcolor{green!80}{\large 38}/{\footnotesize 49} & \cellcolor{green!30}{\large 30}/{\footnotesize 99} \tabularnewline
\rowcolor{black!10} 7 & croaring\_roaring\_bitmap\_portable\_deserialize\_safe & 3 & \cellcolor{green!40}{\large 14}/{\footnotesize 40} & \cellcolor{green!100}{\large 39}/{\footnotesize 40} & \cellcolor{green!100}{\large 37}/{\footnotesize 40} & \cellcolor{green!70}{\large 28}/{\footnotesize 40} & \cellcolor{green!100}{\large 40}/{\footnotesize 41} & \cellcolor{green!40}{\large 32}/{\footnotesize 84} \tabularnewline
8 & lua\_luaL\_loadbufferx & 3 & \cellcolor{green!100}{\large 40}/{\footnotesize 40} & \cellcolor{green!100}{\large 39}/{\footnotesize 40} & \cellcolor{green!90}{\large 35}/{\footnotesize 40} & \cellcolor{green!60}{\large 22}/{\footnotesize 40} & \cellcolor{green!100}{\large 40}/{\footnotesize 41} & \cellcolor{green!50}{\large 39}/{\footnotesize 89} \tabularnewline
\rowcolor{black!10} 9 & w3m\_wc\_Str\_conv\_with\_detect & 3 & \cellcolor{green!0}{\large 0}/{\footnotesize 40} & \cellcolor{green!10}{\large 1}/{\footnotesize 40} & \cellcolor{green!0}{\large -}{\tiny -} & \cellcolor{green!40}{\large 16}/{\footnotesize 40} & \cellcolor{green!10}{\large 5}/{\footnotesize 186} & \cellcolor{green!30}{\large 31}/{\footnotesize 102} \tabularnewline
10 & bind9\_dns\_name\_fromwire & 4 & \cellcolor{green!0}{\large 0}/{\footnotesize 40} & \cellcolor{green!50}{\large 17}/{\footnotesize 40} & \cellcolor{green!0}{\large -}{\tiny -} & \cellcolor{green!30}{\large 12}/{\footnotesize 40} & \cellcolor{green!20}{\large 21}/{\footnotesize 142} & \cellcolor{green!20}{\large 23}/{\footnotesize 119} \tabularnewline
\rowcolor{black!10} 11 & gdk-pixbuf\_gdk\_pixbuf\_animation\_new\_from\_file & 4 & \cellcolor{green!30}{\large 10}/{\footnotesize 40} & \cellcolor{green!50}{\large 19}/{\footnotesize 40} & \cellcolor{green!50}{\large 18}/{\footnotesize 40} & \cellcolor{green!30}{\large 12}/{\footnotesize 40} & \cellcolor{green!30}{\large 14}/{\footnotesize 49} & \cellcolor{green!30}{\large 13}/{\footnotesize 50} \tabularnewline
12 & gdk-pixbuf\_gdk\_pixbuf\_new\_from\_data & 4 & \cellcolor{green!10}{\large 4}/{\footnotesize 40} & \cellcolor{green!70}{\large 26}/{\footnotesize 40} & \cellcolor{green!30}{\large 12}/{\footnotesize 40} & \cellcolor{green!50}{\large 18}/{\footnotesize 40} & \cellcolor{green!50}{\large 35}/{\footnotesize 70} & \cellcolor{green!30}{\large 29}/{\footnotesize 117} \tabularnewline
\rowcolor{black!10} 13 & gdk-pixbuf\_gdk\_pixbuf\_new\_from\_file & 4 & \cellcolor{green!70}{\large 27}/{\footnotesize 40} & \cellcolor{green!60}{\large 21}/{\footnotesize 40} & \cellcolor{green!80}{\large 32}/{\footnotesize 40} & \cellcolor{green!10}{\large 1}/{\footnotesize 40} & \cellcolor{green!80}{\large 32}/{\footnotesize 40} & \cellcolor{green!20}{\large 8}/{\footnotesize 62} \tabularnewline
14 & gdk-pixbuf\_gdk\_pixbuf\_new\_from\_stream & 4 & \cellcolor{green!60}{\large 22}/{\footnotesize 40} & \cellcolor{green!90}{\large 33}/{\footnotesize 40} & \cellcolor{green!90}{\large 33}/{\footnotesize 40} & \cellcolor{green!100}{\large 37}/{\footnotesize 40} & \cellcolor{green!80}{\large 40}/{\footnotesize 54} & \cellcolor{green!70}{\large 37}/{\footnotesize 60} \tabularnewline
\rowcolor{black!10} 15 & gpac\_gf\_isom\_open\_file & 4 & \cellcolor{green!50}{\large 19}/{\footnotesize 40} & \cellcolor{green!80}{\large 31}/{\footnotesize 40} & \cellcolor{green!0}{\large -}{\tiny -} & \cellcolor{green!80}{\large 29}/{\footnotesize 40} & \cellcolor{green!20}{\large 13}/{\footnotesize 99} & \cellcolor{green!30}{\large 18}/{\footnotesize 79} \tabularnewline
16 & libbpf\_bpf\_object\_\_open\_mem & 4 & \cellcolor{green!10}{\large 1}/{\footnotesize 40} & \cellcolor{green!10}{\large 3}/{\footnotesize 40} & \cellcolor{green!10}{\large 2}/{\footnotesize 40} & \cellcolor{green!20}{\large 5}/{\footnotesize 40} & \cellcolor{green!10}{\large 13}/{\footnotesize 158} & \cellcolor{green!20}{\large 16}/{\footnotesize 137} \tabularnewline
\rowcolor{black!10} 17 & libpg\_query\_pg\_query\_parse & 4 & \cellcolor{green!10}{\large 2}/{\footnotesize 40} & \cellcolor{green!90}{\large 34}/{\footnotesize 40} & \cellcolor{green!0}{\large -}{\tiny -} & \cellcolor{green!90}{\large 34}/{\footnotesize 40} & \cellcolor{green!90}{\large 40}/{\footnotesize 46} & \cellcolor{green!70}{\large 39}/{\footnotesize 56} \tabularnewline
18 & libucl\_ucl\_parser\_add\_string & 4 & \cellcolor{green!60}{\large 21}/{\footnotesize 40} & \cellcolor{green!10}{\large 4}/{\footnotesize 40} & \cellcolor{green!0}{\large 0}/{\footnotesize 40} & \cellcolor{green!30}{\large 11}/{\footnotesize 40} & \cellcolor{green!50}{\large 36}/{\footnotesize 86} & \cellcolor{green!40}{\large 33}/{\footnotesize 98} \tabularnewline
\rowcolor{black!10} 19 & oniguruma\_onig\_new & 4 & \cellcolor{green!90}{\large 34}/{\footnotesize 40} & \cellcolor{green!90}{\large 33}/{\footnotesize 40} & \cellcolor{green!90}{\large 36}/{\footnotesize 40} & \cellcolor{green!50}{\large 18}/{\footnotesize 40} & \cellcolor{green!70}{\large 39}/{\footnotesize 56} & \cellcolor{green!60}{\large 33}/{\footnotesize 64} \tabularnewline
20 & pupnp\_ixmlLoadDocumentEx & 4 & \cellcolor{green!10}{\large 3}/{\footnotesize 40} & \cellcolor{green!20}{\large 6}/{\footnotesize 40} & \cellcolor{green!30}{\large 9}/{\footnotesize 40} & \cellcolor{green!40}{\large 16}/{\footnotesize 40} & \cellcolor{green!0}{\large 0}/{\footnotesize 69} & \cellcolor{green!0}{\large 0}/{\footnotesize 97} \tabularnewline
\rowcolor{black!10} 21 & gdk-pixbuf\_gdk\_pixbuf\_new\_from\_file\_at\_scale & 5 & \cellcolor{green!60}{\large 23}/{\footnotesize 40} & \cellcolor{green!90}{\large 35}/{\footnotesize 40} & \cellcolor{green!70}{\large 28}/{\footnotesize 40} & \cellcolor{green!30}{\large 10}/{\footnotesize 40} & \cellcolor{green!90}{\large 37}/{\footnotesize 41} & \cellcolor{green!30}{\large 17}/{\footnotesize 61} \tabularnewline
22 & inchi\_GetINCHIKeyFromINCHI & 5 & \cellcolor{green!10}{\large 1}/{\footnotesize 40} & \cellcolor{green!40}{\large 14}/{\footnotesize 40} & \cellcolor{green!100}{\large 38}/{\footnotesize 40} & \cellcolor{green!70}{\large 25}/{\footnotesize 40} & \cellcolor{green!20}{\large 24}/{\footnotesize 125} & \cellcolor{green!40}{\large 34}/{\footnotesize 96} \tabularnewline
\rowcolor{black!10} 23 & libdwarf\_dwarf\_init\_b & 5 & \cellcolor{green!0}{\large 0}/{\footnotesize 40} & \cellcolor{green!10}{\large 1}/{\footnotesize 40} & \cellcolor{green!30}{\large 10}/{\footnotesize 40} & \cellcolor{green!40}{\large 16}/{\footnotesize 40} & \cellcolor{green!20}{\large 21}/{\footnotesize 183} & \cellcolor{green!30}{\large 29}/{\footnotesize 102} \tabularnewline
24 & libdwarf\_dwarf\_init\_path & 5 & \cellcolor{green!0}{\large 0}/{\footnotesize 40} & \cellcolor{green!0}{\large 0}/{\footnotesize 40} & \cellcolor{green!0}{\large 0}/{\footnotesize 40} & \cellcolor{green!10}{\large 2}/{\footnotesize 40} & \cellcolor{green!0}{\large 0}/{\footnotesize 171} & \cellcolor{green!10}{\large 6}/{\footnotesize 112} \tabularnewline
\rowcolor{black!10} 25 & liblouis\_lou\_compileString & 5 & \cellcolor{green!0}{\large 0}/{\footnotesize 40} & \cellcolor{green!0}{\large 0}/{\footnotesize 40} & \cellcolor{green!10}{\large 3}/{\footnotesize 40} & \cellcolor{green!40}{\large 15}/{\footnotesize 40} & \cellcolor{green!10}{\large 9}/{\footnotesize 178} & \cellcolor{green!20}{\large 25}/{\footnotesize 120} \tabularnewline
26 & selinux\_cil\_compile & 5 & \cellcolor{green!0}{\large 0}/{\footnotesize 40} & \cellcolor{green!10}{\large 2}/{\footnotesize 40} & \cellcolor{green!0}{\large -}{\tiny -} & \cellcolor{green!90}{\large 36}/{\footnotesize 40} & \cellcolor{green!20}{\large 18}/{\footnotesize 136} & \cellcolor{green!70}{\large 37}/{\footnotesize 54} \tabularnewline
\rowcolor{black!10} 27 & bind9\_dns\_name\_fromtext & 6 & \cellcolor{green!10}{\large 1}/{\footnotesize 40} & \cellcolor{green!50}{\large 17}/{\footnotesize 40} & \cellcolor{green!0}{\large -}{\tiny -} & \cellcolor{green!50}{\large 19}/{\footnotesize 40} & \cellcolor{green!50}{\large 36}/{\footnotesize 78} & \cellcolor{green!40}{\large 32}/{\footnotesize 99} \tabularnewline
28 & bind9\_dns\_rdata\_fromwire & 6 & \cellcolor{green!0}{\large 0}/{\footnotesize 40} & \cellcolor{green!10}{\large 2}/{\footnotesize 40} & \cellcolor{green!0}{\large -}{\tiny -} & \cellcolor{green!0}{\large 0}/{\footnotesize 40} & \cellcolor{green!10}{\large 6}/{\footnotesize 187} & \cellcolor{green!10}{\large 8}/{\footnotesize 171} \tabularnewline
\rowcolor{black!10} 29 & coturn\_stun\_is\_binding\_response & 6 & \cellcolor{green!0}{\large 0}/{\footnotesize 40} & \cellcolor{green!50}{\large 20}/{\footnotesize 40} & \cellcolor{green!0}{\large -}{\tiny -} & \cellcolor{green!50}{\large 19}/{\footnotesize 40} & \cellcolor{green!40}{\large 33}/{\footnotesize 90} & \cellcolor{green!20}{\large 19}/{\footnotesize 119} \tabularnewline
30 & coturn\_stun\_is\_command\_message & 6 & \cellcolor{green!0}{\large 0}/{\footnotesize 40} & \cellcolor{green!40}{\large 13}/{\footnotesize 40} & \cellcolor{green!30}{\large 9}/{\footnotesize 40} & \cellcolor{green!60}{\large 22}/{\footnotesize 40} & \cellcolor{green!30}{\large 31}/{\footnotesize 109} & \cellcolor{green!40}{\large 30}/{\footnotesize 79} \tabularnewline
\rowcolor{black!10} 31 & coturn\_stun\_is\_response & 6 & \cellcolor{green!0}{\large 0}/{\footnotesize 40} & \cellcolor{green!40}{\large 13}/{\footnotesize 40} & \cellcolor{green!0}{\large -}{\tiny -} & \cellcolor{green!50}{\large 19}/{\footnotesize 40} & \cellcolor{green!20}{\large 16}/{\footnotesize 107} & \cellcolor{green!20}{\large 16}/{\footnotesize 98} \tabularnewline
32 & coturn\_stun\_is\_success\_response & 6 & \cellcolor{green!0}{\large 0}/{\footnotesize 40} & \cellcolor{green!50}{\large 17}/{\footnotesize 40} & \cellcolor{green!0}{\large -}{\tiny -} & \cellcolor{green!70}{\large 27}/{\footnotesize 40} & \cellcolor{green!40}{\large 32}/{\footnotesize 89} & \cellcolor{green!40}{\large 29}/{\footnotesize 90} \tabularnewline
\rowcolor{black!10} 33 & hiredis\_redisFormatCommand & 6 & \cellcolor{green!10}{\large 3}/{\footnotesize 40} & \cellcolor{green!60}{\large 21}/{\footnotesize 40} & \cellcolor{green!0}{\large -}{\tiny -} & \cellcolor{green!40}{\large 14}/{\footnotesize 40} & \cellcolor{green!30}{\large 34}/{\footnotesize 138} & \cellcolor{green!20}{\large 20}/{\footnotesize 145} \tabularnewline
34 & igraph\_igraph\_read\_graph\_dl & 6 & \cellcolor{green!20}{\large 7}/{\footnotesize 40} & \cellcolor{green!0}{\large 0}/{\footnotesize 40} & \cellcolor{green!0}{\large 0}/{\footnotesize 40} & \cellcolor{green!10}{\large 1}/{\footnotesize 40} & \cellcolor{green!20}{\large 23}/{\footnotesize 143} & \cellcolor{green!20}{\large 18}/{\footnotesize 158} \tabularnewline
\rowcolor{black!10} 35 & igraph\_igraph\_read\_graph\_edgelist & 6 & \cellcolor{green!10}{\large 3}/{\footnotesize 40} & \cellcolor{green!0}{\large 0}/{\footnotesize 40} & \cellcolor{green!0}{\large 0}/{\footnotesize 40} & \cellcolor{green!0}{\large 0}/{\footnotesize 40} & \cellcolor{green!10}{\large 14}/{\footnotesize 164} & \cellcolor{green!10}{\large 6}/{\footnotesize 189} \tabularnewline
36 & igraph\_igraph\_read\_graph\_gml & 6 & \cellcolor{green!20}{\large 5}/{\footnotesize 40} & \cellcolor{green!0}{\large 0}/{\footnotesize 40} & \cellcolor{green!0}{\large 0}/{\footnotesize 40} & \cellcolor{green!0}{\large 0}/{\footnotesize 40} & \cellcolor{green!20}{\large 21}/{\footnotesize 140} & \cellcolor{green!10}{\large 11}/{\footnotesize 161} \tabularnewline
\rowcolor{black!10} 37 & igraph\_igraph\_read\_graph\_graphdb & 6 & \cellcolor{green!10}{\large 3}/{\footnotesize 40} & \cellcolor{green!0}{\large 0}/{\footnotesize 40} & \cellcolor{green!0}{\large 0}/{\footnotesize 40} & \cellcolor{green!10}{\large 2}/{\footnotesize 40} & \cellcolor{green!0}{\large 1}/{\footnotesize 198} & \cellcolor{green!20}{\large 21}/{\footnotesize 149} \tabularnewline
38 & igraph\_igraph\_read\_graph\_graphml & 6 & \cellcolor{green!30}{\large 10}/{\footnotesize 40} & \cellcolor{green!0}{\large 0}/{\footnotesize 40} & \cellcolor{green!0}{\large 0}/{\footnotesize 40} & \cellcolor{green!40}{\large 16}/{\footnotesize 40} & \cellcolor{green!30}{\large 29}/{\footnotesize 131} & \cellcolor{green!20}{\large 18}/{\footnotesize 148} \tabularnewline
\rowcolor{black!10} 39 & igraph\_igraph\_read\_graph\_lgl & 6 & \cellcolor{green!0}{\large 0}/{\footnotesize 40} & \cellcolor{green!0}{\large 0}/{\footnotesize 40} & \cellcolor{green!0}{\large 0}/{\footnotesize 40} & \cellcolor{green!0}{\large 0}/{\footnotesize 40} & \cellcolor{green!10}{\large 2}/{\footnotesize 194} & \cellcolor{green!10}{\large 17}/{\footnotesize 161} \tabularnewline
40 & igraph\_igraph\_read\_graph\_pajek & 6 & \cellcolor{green!10}{\large 3}/{\footnotesize 40} & \cellcolor{green!0}{\large 0}/{\footnotesize 40} & \cellcolor{green!0}{\large 0}/{\footnotesize 40} & \cellcolor{green!10}{\large 1}/{\footnotesize 40} & \cellcolor{green!20}{\large 28}/{\footnotesize 137} & \cellcolor{green!10}{\large 15}/{\footnotesize 164} \tabularnewline
\rowcolor{black!10} 41 & inchi\_GetINCHIfromINCHI & 6 & \cellcolor{green!20}{\large 5}/{\footnotesize 40} & \cellcolor{green!20}{\large 7}/{\footnotesize 40} & \cellcolor{green!30}{\large 9}/{\footnotesize 40} & \cellcolor{green!70}{\large 27}/{\footnotesize 40} & \cellcolor{green!10}{\large 12}/{\footnotesize 193} & \cellcolor{green!10}{\large 19}/{\footnotesize 195} \tabularnewline
42 & inchi\_GetStructFromINCHI & 6 & \cellcolor{green!10}{\large 1}/{\footnotesize 40} & \cellcolor{green!20}{\large 7}/{\footnotesize 40} & \cellcolor{green!10}{\large 4}/{\footnotesize 40} & \cellcolor{green!20}{\large 5}/{\footnotesize 40} & \cellcolor{green!10}{\large 15}/{\footnotesize 196} & \cellcolor{green!10}{\large 15}/{\footnotesize 192} \tabularnewline
\rowcolor{black!10} 43 & kamailio\_parse\_msg & 6 & \cellcolor{green!0}{\large 0}/{\footnotesize 40} & \cellcolor{green!30}{\large 12}/{\footnotesize 40} & \cellcolor{green!0}{\large -}{\tiny -} & \cellcolor{green!60}{\large 21}/{\footnotesize 40} & \cellcolor{green!30}{\large 32}/{\footnotesize 104} & \cellcolor{green!40}{\large 32}/{\footnotesize 98} \tabularnewline
44 & libyang\_lys\_parse\_mem & 6 & \cellcolor{green!0}{\large 0}/{\footnotesize 40} & \cellcolor{green!0}{\large 0}/{\footnotesize 40} & \cellcolor{green!0}{\large 0}/{\footnotesize 40} & \cellcolor{green!10}{\large 2}/{\footnotesize 40} & \cellcolor{green!20}{\large 31}/{\footnotesize 151} & \cellcolor{green!30}{\large 36}/{\footnotesize 119} \tabularnewline
\rowcolor{black!10} 45 & proftpd\_pr\_json\_object\_from\_text & 6 & \cellcolor{green!0}{\large 0}/{\footnotesize 40} & \cellcolor{green!0}{\large 0}/{\footnotesize 40} & \cellcolor{green!0}{\large -}{\tiny -} & \cellcolor{green!30}{\large 10}/{\footnotesize 40} & \cellcolor{green!10}{\large 14}/{\footnotesize 171} & \cellcolor{green!30}{\large 29}/{\footnotesize 124} \tabularnewline
46 & selinux\_policydb\_read & 6 & \cellcolor{green!10}{\large 1}/{\footnotesize 40} & \cellcolor{green!20}{\large 7}/{\footnotesize 40} & \cellcolor{green!0}{\large -}{\tiny -} & \cellcolor{green!30}{\large 9}/{\footnotesize 40} & \cellcolor{green!10}{\large 12}/{\footnotesize 135} & \cellcolor{green!30}{\large 24}/{\footnotesize 113} \tabularnewline
\rowcolor{black!10} 47 & kamailio\_get\_src\_address\_socket & 7 & \cellcolor{green!0}{\large 0}/{\footnotesize 40} & \cellcolor{green!0}{\large 0}/{\footnotesize 40} & \cellcolor{green!0}{\large 0}/{\footnotesize 40} & \cellcolor{green!40}{\large 13}/{\footnotesize 40} & \cellcolor{green!0}{\large 0}/{\footnotesize 153} & \cellcolor{green!10}{\large 6}/{\footnotesize 129} \tabularnewline
48 & kamailio\_get\_src\_uri & 7 & \cellcolor{green!0}{\large 0}/{\footnotesize 40} & \cellcolor{green!0}{\large 0}/{\footnotesize 40} & \cellcolor{green!0}{\large 0}/{\footnotesize 40} & \cellcolor{green!30}{\large 9}/{\footnotesize 40} & \cellcolor{green!0}{\large 0}/{\footnotesize 98} & \cellcolor{green!10}{\large 7}/{\footnotesize 95} \tabularnewline
\rowcolor{black!10} 49 & kamailio\_parse\_content\_disposition & 7 & \cellcolor{green!0}{\large 0}/{\footnotesize 40} & \cellcolor{green!0}{\large 0}/{\footnotesize 40} & \cellcolor{green!0}{\large 0}/{\footnotesize 40} & \cellcolor{green!20}{\large 8}/{\footnotesize 40} & \cellcolor{green!0}{\large 0}/{\footnotesize 182} & \cellcolor{green!10}{\large 5}/{\footnotesize 175} \tabularnewline
50 & kamailio\_parse\_diversion\_header & 7 & \cellcolor{green!0}{\large 0}/{\footnotesize 40} & \cellcolor{green!0}{\large 0}/{\footnotesize 40} & \cellcolor{green!0}{\large 0}/{\footnotesize 40} & \cellcolor{green!40}{\large 15}/{\footnotesize 40} & \cellcolor{green!0}{\large 0}/{\footnotesize 160} & \cellcolor{green!10}{\large 13}/{\footnotesize 125} \tabularnewline
\rowcolor{black!10} 51 & kamailio\_parse\_from\_header & 7 & \cellcolor{green!0}{\large 0}/{\footnotesize 40} & \cellcolor{green!0}{\large 0}/{\footnotesize 40} & \cellcolor{green!0}{\large -}{\tiny -} & \cellcolor{green!10}{\large 1}/{\footnotesize 40} & \cellcolor{green!0}{\large 0}/{\footnotesize 169} & \cellcolor{green!0}{\large 0}/{\footnotesize 151} \tabularnewline
52 & kamailio\_parse\_from\_uri & 7 & \cellcolor{green!0}{\large 0}/{\footnotesize 40} & \cellcolor{green!0}{\large 0}/{\footnotesize 40} & \cellcolor{green!0}{\large -}{\tiny -} & \cellcolor{green!10}{\large 4}/{\footnotesize 40} & \cellcolor{green!0}{\large 0}/{\footnotesize 181} & \cellcolor{green!10}{\large 4}/{\footnotesize 168} \tabularnewline
\rowcolor{black!10} 53 & kamailio\_parse\_headers & 7 & \cellcolor{green!0}{\large 0}/{\footnotesize 40} & \cellcolor{green!0}{\large 0}/{\footnotesize 40} & \cellcolor{green!0}{\large -}{\tiny -} & \cellcolor{green!0}{\large 0}/{\footnotesize 40} & \cellcolor{green!0}{\large 0}/{\footnotesize 132} & \cellcolor{green!0}{\large 1}/{\footnotesize 119} \tabularnewline
54 & kamailio\_parse\_identityinfo\_header & 7 & \cellcolor{green!0}{\large 0}/{\footnotesize 40} & \cellcolor{green!0}{\large 0}/{\footnotesize 40} & \cellcolor{green!0}{\large -}{\tiny -} & \cellcolor{green!50}{\large 17}/{\footnotesize 40} & \cellcolor{green!0}{\large 0}/{\footnotesize 165} & \cellcolor{green!20}{\large 18}/{\footnotesize 104} \tabularnewline
\rowcolor{black!10} 55 & kamailio\_parse\_pai\_header & 7 & \cellcolor{green!0}{\large 0}/{\footnotesize 40} & \cellcolor{green!0}{\large 0}/{\footnotesize 40} & \cellcolor{green!0}{\large -}{\tiny -} & \cellcolor{green!20}{\large 8}/{\footnotesize 40} & \cellcolor{green!0}{\large 0}/{\footnotesize 136} & \cellcolor{green!10}{\large 11}/{\footnotesize 127} \tabularnewline
56 & kamailio\_parse\_privacy & 7 & \cellcolor{green!0}{\large 0}/{\footnotesize 40} & \cellcolor{green!0}{\large 0}/{\footnotesize 40} & \cellcolor{green!0}{\large 0}/{\footnotesize 40} & \cellcolor{green!30}{\large 9}/{\footnotesize 40} & \cellcolor{green!0}{\large 0}/{\footnotesize 180} & \cellcolor{green!10}{\large 5}/{\footnotesize 165} \tabularnewline
\rowcolor{black!10} 57 & kamailio\_parse\_record\_route\_headers & 7 & \cellcolor{green!0}{\large 0}/{\footnotesize 40} & \cellcolor{green!0}{\large 0}/{\footnotesize 40} & \cellcolor{green!0}{\large -}{\tiny -} & \cellcolor{green!100}{\large 38}/{\footnotesize 40} & \cellcolor{green!0}{\large 1}/{\footnotesize 135} & \cellcolor{green!60}{\large 31}/{\footnotesize 51} \tabularnewline
58 & kamailio\_parse\_refer\_to\_header & 7 & \cellcolor{green!0}{\large 0}/{\footnotesize 40} & \cellcolor{green!0}{\large 0}/{\footnotesize 40} & \cellcolor{green!0}{\large -}{\tiny -} & \cellcolor{green!20}{\large 5}/{\footnotesize 40} & \cellcolor{green!0}{\large 0}/{\footnotesize 157} & \cellcolor{green!10}{\large 11}/{\footnotesize 134} \tabularnewline
\rowcolor{black!10} 59 & kamailio\_parse\_route\_headers & 7 & \cellcolor{green!0}{\large 0}/{\footnotesize 40} & \cellcolor{green!0}{\large 0}/{\footnotesize 40} & \cellcolor{green!0}{\large -}{\tiny -} & \cellcolor{green!100}{\large 39}/{\footnotesize 40} & \cellcolor{green!0}{\large 0}/{\footnotesize 155} & \cellcolor{green!40}{\large 28}/{\footnotesize 80} \tabularnewline
60 & kamailio\_parse\_to\_header & 7 & \cellcolor{green!0}{\large 0}/{\footnotesize 40} & \cellcolor{green!0}{\large 0}/{\footnotesize 40} & \cellcolor{green!0}{\large -}{\tiny -} & \cellcolor{green!20}{\large 5}/{\footnotesize 40} & \cellcolor{green!0}{\large 0}/{\footnotesize 162} & \cellcolor{green!10}{\large 8}/{\footnotesize 137} \tabularnewline
\rowcolor{black!10} 61 & kamailio\_parse\_to\_uri & 7 & \cellcolor{green!0}{\large 0}/{\footnotesize 40} & \cellcolor{green!0}{\large 0}/{\footnotesize 40} & \cellcolor{green!0}{\large -}{\tiny -} & \cellcolor{green!10}{\large 2}/{\footnotesize 40} & \cellcolor{green!0}{\large 0}/{\footnotesize 164} & \cellcolor{green!10}{\large 6}/{\footnotesize 144} \tabularnewline
62 & libyang\_lyd\_parse\_data\_mem & 7 & \cellcolor{green!10}{\large 1}/{\footnotesize 40} & \cellcolor{green!0}{\large 0}/{\footnotesize 40} & \cellcolor{green!0}{\large 0}/{\footnotesize 40} & \cellcolor{green!10}{\large 2}/{\footnotesize 40} & \cellcolor{green!30}{\large 35}/{\footnotesize 118} & \cellcolor{green!20}{\large 31}/{\footnotesize 151} \tabularnewline
\rowcolor{black!10} 63 & bind9\_dns\_message\_parse & 8 & \cellcolor{green!0}{\large 0}/{\footnotesize 40} & \cellcolor{green!0}{\large 0}/{\footnotesize 40} & \cellcolor{green!0}{\large -}{\tiny -} & \cellcolor{green!0}{\large 0}/{\footnotesize 40} & \cellcolor{green!0}{\large 0}/{\footnotesize 200} & \cellcolor{green!10}{\large 3}/{\footnotesize 181} \tabularnewline
64 & igraph\_igraph\_read\_graph\_ncol & 8 & \cellcolor{green!0}{\large 0}/{\footnotesize 40} & \cellcolor{green!0}{\large 0}/{\footnotesize 40} & \cellcolor{green!0}{\large 0}/{\footnotesize 40} & \cellcolor{green!0}{\large 0}/{\footnotesize 40} & \cellcolor{green!0}{\large 0}/{\footnotesize 197} & \cellcolor{green!10}{\large 4}/{\footnotesize 194} \tabularnewline
\rowcolor{black!10} 65 & pjsip\_pj\_json\_parse & 8 & \cellcolor{green!0}{\large 0}/{\footnotesize 40} & \cellcolor{green!10}{\large 4}/{\footnotesize 40} & \cellcolor{green!0}{\large 0}/{\footnotesize 40} & \cellcolor{green!0}{\large 0}/{\footnotesize 40} & \cellcolor{green!20}{\large 25}/{\footnotesize 150} & \cellcolor{green!20}{\large 26}/{\footnotesize 140} \tabularnewline
66 & pjsip\_pj\_xml\_parse & 8 & \cellcolor{green!0}{\large 0}/{\footnotesize 40} & \cellcolor{green!50}{\large 18}/{\footnotesize 40} & \cellcolor{green!30}{\large 11}/{\footnotesize 40} & \cellcolor{green!0}{\large 0}/{\footnotesize 40} & \cellcolor{green!50}{\large 38}/{\footnotesize 89} & \cellcolor{green!20}{\large 19}/{\footnotesize 151} \tabularnewline
\rowcolor{black!10} 67 & pjsip\_pjmedia\_sdp\_parse & 8 & \cellcolor{green!0}{\large 0}/{\footnotesize 40} & \cellcolor{green!40}{\large 16}/{\footnotesize 40} & \cellcolor{green!30}{\large 11}/{\footnotesize 40} & \cellcolor{green!10}{\large 2}/{\footnotesize 40} & \cellcolor{green!30}{\large 27}/{\footnotesize 119} & \cellcolor{green!20}{\large 25}/{\footnotesize 138} \tabularnewline
68 & quickjs\_lre\_compile & 8 & \cellcolor{green!0}{\large 0}/{\footnotesize 40} & \cellcolor{green!0}{\large 0}/{\footnotesize 40} & \cellcolor{green!0}{\large -}{\tiny -} & \cellcolor{green!0}{\large 0}/{\footnotesize 40} & \cellcolor{green!0}{\large 0}/{\footnotesize 200} & \cellcolor{green!10}{\large 2}/{\footnotesize 191} \tabularnewline
\rowcolor{black!10} 69 & bind9\_isc\_lex\_getmastertoken & 9 & \cellcolor{green!10}{\large 1}/{\footnotesize 40} & \cellcolor{green!0}{\large 0}/{\footnotesize 40} & \cellcolor{green!0}{\large -}{\tiny -} & \cellcolor{green!0}{\large 0}/{\footnotesize 40} & \cellcolor{green!10}{\large 7}/{\footnotesize 193} & \cellcolor{green!10}{\large 8}/{\footnotesize 185} \tabularnewline
70 & bind9\_isc\_lex\_gettoken & 9 & \cellcolor{green!0}{\large 0}/{\footnotesize 40} & \cellcolor{green!0}{\large 0}/{\footnotesize 40} & \cellcolor{green!0}{\large -}{\tiny -} & \cellcolor{green!10}{\large 1}/{\footnotesize 40} & \cellcolor{green!10}{\large 5}/{\footnotesize 195} & \cellcolor{green!10}{\large 6}/{\footnotesize 185} \tabularnewline
\rowcolor{black!10} 71 & quickjs\_JS\_Eval & 9 & \cellcolor{green!0}{\large 0}/{\footnotesize 40} & \cellcolor{green!80}{\large 29}/{\footnotesize 40} & \cellcolor{green!0}{\large -}{\tiny -} & \cellcolor{green!40}{\large 16}/{\footnotesize 40} & \cellcolor{green!50}{\large 35}/{\footnotesize 69} & \cellcolor{green!40}{\large 35}/{\footnotesize 101} \tabularnewline
72 & igraph\_igraph\_edge\_connectivity & 10 & \cellcolor{green!0}{\large 0}/{\footnotesize 40} & \cellcolor{green!0}{\large 0}/{\footnotesize 40} & \cellcolor{green!0}{\large 0}/{\footnotesize 40} & \cellcolor{green!0}{\large 0}/{\footnotesize 40} & \cellcolor{green!0}{\large 0}/{\footnotesize 114} & \cellcolor{green!0}{\large 0}/{\footnotesize 131} \tabularnewline
\rowcolor{black!10} 73 & pjsip\_pj\_stun\_msg\_decode & 10 & \cellcolor{green!0}{\large 0}/{\footnotesize 40} & \cellcolor{green!0}{\large 0}/{\footnotesize 40} & \cellcolor{green!0}{\large 0}/{\footnotesize 40} & \cellcolor{green!0}{\large 0}/{\footnotesize 40} & \cellcolor{green!10}{\large 12}/{\footnotesize 182} & \cellcolor{green!20}{\large 21}/{\footnotesize 157} \tabularnewline
74 & bind9\_dns\_message\_checksig & 11 & \cellcolor{green!0}{\large 0}/{\footnotesize 40} & \cellcolor{green!0}{\large 0}/{\footnotesize 40} & \cellcolor{green!0}{\large -}{\tiny -} & \cellcolor{green!0}{\large 0}/{\footnotesize 40} & \cellcolor{green!0}{\large 0}/{\footnotesize 196} & \cellcolor{green!0}{\large 1}/{\footnotesize 189} \tabularnewline
\rowcolor{black!10} 75 & libzip\_zip\_fread & 11 & \cellcolor{green!80}{\large 31}/{\footnotesize 40} & \cellcolor{green!80}{\large 31}/{\footnotesize 40} & \cellcolor{green!20}{\large 7}/{\footnotesize 40} & \cellcolor{green!50}{\large 20}/{\footnotesize 40} & \cellcolor{green!80}{\large 40}/{\footnotesize 56} & \cellcolor{green!30}{\large 28}/{\footnotesize 107} \tabularnewline
76 & bind9\_dns\_rdata\_fromtext & 12 & \cellcolor{green!0}{\large 0}/{\footnotesize 40} & \cellcolor{green!0}{\large 0}/{\footnotesize 40} & \cellcolor{green!0}{\large -}{\tiny -} & \cellcolor{green!0}{\large 0}/{\footnotesize 40} & \cellcolor{green!10}{\large 2}/{\footnotesize 187} & \cellcolor{green!10}{\large 7}/{\footnotesize 186} \tabularnewline
\rowcolor{black!10} 77 & igraph\_igraph\_all\_minimal\_st\_separators & 12 & \cellcolor{green!0}{\large 0}/{\footnotesize 40} & \cellcolor{green!10}{\large 2}/{\footnotesize 40} & \cellcolor{green!10}{\large 4}/{\footnotesize 40} & \cellcolor{green!10}{\large 1}/{\footnotesize 40} & \cellcolor{green!20}{\large 25}/{\footnotesize 130} & \cellcolor{green!20}{\large 17}/{\footnotesize 149} \tabularnewline
78 & igraph\_igraph\_minimum\_size\_separators & 12 & \cellcolor{green!10}{\large 3}/{\footnotesize 40} & \cellcolor{green!20}{\large 5}/{\footnotesize 40} & \cellcolor{green!10}{\large 2}/{\footnotesize 40} & \cellcolor{green!10}{\large 3}/{\footnotesize 40} & \cellcolor{green!20}{\large 22}/{\footnotesize 154} & \cellcolor{green!20}{\large 24}/{\footnotesize 151} \tabularnewline
\rowcolor{black!10} 79 & pjsip\_pjsip\_parse\_msg & 12 & \cellcolor{green!0}{\large 0}/{\footnotesize 40} & \cellcolor{green!0}{\large 0}/{\footnotesize 40} & \cellcolor{green!0}{\large 0}/{\footnotesize 40} & \cellcolor{green!0}{\large 0}/{\footnotesize 40} & \cellcolor{green!0}{\large 0}/{\footnotesize 199} & \cellcolor{green!10}{\large 3}/{\footnotesize 194} \tabularnewline
80 & igraph\_igraph\_automorphism\_group & 13 & \cellcolor{green!0}{\large 0}/{\footnotesize 40} & \cellcolor{green!0}{\large 0}/{\footnotesize 40} & \cellcolor{green!0}{\large 0}/{\footnotesize 40} & \cellcolor{green!10}{\large 2}/{\footnotesize 40} & \cellcolor{green!0}{\large 0}/{\footnotesize 198} & \cellcolor{green!10}{\large 16}/{\footnotesize 152} \tabularnewline
\rowcolor{black!10} 81 & libmodbus\_modbus\_read\_bits & 15 & \cellcolor{green!0}{\large 0}/{\footnotesize 40} & \cellcolor{green!0}{\large 0}/{\footnotesize 40} & \cellcolor{green!0}{\large 0}/{\footnotesize 40} & \cellcolor{green!0}{\large 0}/{\footnotesize 40} & \cellcolor{green!0}{\large 0}/{\footnotesize 65} & \cellcolor{green!0}{\large 0}/{\footnotesize 79} \tabularnewline
82 & libmodbus\_modbus\_read\_registers & 15 & \cellcolor{green!0}{\large 0}/{\footnotesize 40} & \cellcolor{green!0}{\large 0}/{\footnotesize 40} & \cellcolor{green!0}{\large 0}/{\footnotesize 40} & \cellcolor{green!0}{\large 0}/{\footnotesize 40} & \cellcolor{green!0}{\large 0}/{\footnotesize 69} & \cellcolor{green!0}{\large 0}/{\footnotesize 66} \tabularnewline
\rowcolor{black!10} 83 & civetweb\_mg\_get\_response & 17 & \cellcolor{green!0}{\large 0}/{\footnotesize 40} & \cellcolor{green!0}{\large 0}/{\footnotesize 40} & \cellcolor{green!0}{\large 0}/{\footnotesize 40} & \cellcolor{green!0}{\large 0}/{\footnotesize 40} & \cellcolor{green!0}{\large 0}/{\footnotesize 151} & \cellcolor{green!0}{\large 0}/{\footnotesize 125} \tabularnewline
84 & bind9\_dns\_master\_loadbuffer & 20 & \cellcolor{green!0}{\large 0}/{\footnotesize 40} & \cellcolor{green!0}{\large 0}/{\footnotesize 40} & \cellcolor{green!0}{\large -}{\tiny -} & \cellcolor{green!0}{\large 0}/{\footnotesize 40} & \cellcolor{green!0}{\large 0}/{\footnotesize 200} & \cellcolor{green!10}{\large 2}/{\footnotesize 196} \tabularnewline
\rowcolor{black!10} 85 & libmodbus\_modbus\_receive & 33 & \cellcolor{green!0}{\large 0}/{\footnotesize 40} & \cellcolor{green!0}{\large 0}/{\footnotesize 40} & \cellcolor{green!0}{\large 0}/{\footnotesize 40} & \cellcolor{green!0}{\large 0}/{\footnotesize 40} & \cellcolor{green!0}{\large 0}/{\footnotesize 121} & \cellcolor{green!0}{\large 0}/{\footnotesize 101} \tabularnewline
86 & tmux\_input\_parse\_buffer & 42 & \cellcolor{green!0}{\large 0}/{\footnotesize 40} & \cellcolor{green!0}{\large 0}/{\footnotesize 40} & \cellcolor{green!0}{\large -}{\tiny -} & \cellcolor{green!0}{\large 0}/{\footnotesize 40} & \cellcolor{green!0}{\large 0}/{\footnotesize 200} & \cellcolor{green!0}{\large 0}/{\footnotesize 197} \tabularnewline

\bottomrule
%\end{tabular}
%}
%\end{table*}
\end{xltabular}
}
\twocolumn



% model: gpt-4-0613, temp: 1.0

\onecolumn
{\small %
\begin{xltabular}[h]{\textwidth}{ccccccccc}
%\begin{table*}[!t]
%\centering
\caption{Evaluation Result of model gpt-4-0613 with temperature 1.0.} \\
%\resizebox{1.0\linewidth}{!}{
%\begin{tabular}{cccccccccc}
\toprule
Index & Question & Score & NAIVE-40 & BACTX-40 & DOCTX-40 & UGCTX-40 & BA-ITER-40 & ALL-ITER-40 \tabularnewline
\midrule
\rowcolor{black!10} 1 & coturn\_stun\_is\_command\_message\_full\_check\_str & 1 & \cellcolor{green!0}{\large 0}/{\footnotesize 40} & \cellcolor{green!90}{\large 36}/{\footnotesize 40} & \cellcolor{green!0}{\large -}{\tiny -} & \cellcolor{green!70}{\large 25}/{\footnotesize 40} & \cellcolor{green!70}{\large 38}/{\footnotesize 54} & \cellcolor{green!60}{\large 35}/{\footnotesize 58} \tabularnewline
2 & kamailio\_parse\_uri & 1 & \cellcolor{green!0}{\large 0}/{\footnotesize 40} & \cellcolor{green!90}{\large 34}/{\footnotesize 40} & \cellcolor{green!0}{\large -}{\tiny -} & \cellcolor{green!80}{\large 31}/{\footnotesize 40} & \cellcolor{green!100}{\large 40}/{\footnotesize 41} & \cellcolor{green!80}{\large 37}/{\footnotesize 52} \tabularnewline
\rowcolor{black!10} 3 & coturn\_stun\_check\_message\_integrity\_str & 2 & \cellcolor{green!0}{\large 0}/{\footnotesize 40} & \cellcolor{green!0}{\large 0}/{\footnotesize 40} & \cellcolor{green!0}{\large -}{\tiny -} & \cellcolor{green!10}{\large 4}/{\footnotesize 40} & \cellcolor{green!10}{\large 13}/{\footnotesize 176} & \cellcolor{green!20}{\large 17}/{\footnotesize 149} \tabularnewline
4 & libiec61850\_MmsValue\_decodeMmsData & 2 & \cellcolor{green!0}{\large 0}/{\footnotesize 40} & \cellcolor{green!70}{\large 27}/{\footnotesize 40} & \cellcolor{green!60}{\large 24}/{\footnotesize 40} & \cellcolor{green!40}{\large 15}/{\footnotesize 40} & \cellcolor{green!60}{\large 36}/{\footnotesize 66} & \cellcolor{green!30}{\large 32}/{\footnotesize 108} \tabularnewline
\rowcolor{black!10} 5 & md4c\_md\_html & 2 & \cellcolor{green!0}{\large 0}/{\footnotesize 40} & \cellcolor{green!0}{\large 0}/{\footnotesize 40} & \cellcolor{green!0}{\large 0}/{\footnotesize 40} & \cellcolor{green!0}{\large 0}/{\footnotesize 40} & \cellcolor{green!40}{\large 37}/{\footnotesize 104} & \cellcolor{green!20}{\large 27}/{\footnotesize 129} \tabularnewline
6 & spdk\_spdk\_json\_parse & 2 & \cellcolor{green!20}{\large 5}/{\footnotesize 40} & \cellcolor{green!80}{\large 32}/{\footnotesize 40} & \cellcolor{green!0}{\large -}{\tiny -} & \cellcolor{green!40}{\large 13}/{\footnotesize 40} & \cellcolor{green!70}{\large 37}/{\footnotesize 54} & \cellcolor{green!20}{\large 24}/{\footnotesize 128} \tabularnewline
\rowcolor{black!10} 7 & croaring\_roaring\_bitmap\_portable\_deserialize\_safe & 3 & \cellcolor{green!40}{\large 14}/{\footnotesize 40} & \cellcolor{green!80}{\large 32}/{\footnotesize 40} & \cellcolor{green!80}{\large 32}/{\footnotesize 40} & \cellcolor{green!80}{\large 30}/{\footnotesize 40} & \cellcolor{green!80}{\large 38}/{\footnotesize 51} & \cellcolor{green!60}{\large 37}/{\footnotesize 68} \tabularnewline
8 & lua\_luaL\_loadbufferx & 3 & \cellcolor{green!90}{\large 35}/{\footnotesize 40} & \cellcolor{green!80}{\large 31}/{\footnotesize 40} & \cellcolor{green!80}{\large 32}/{\footnotesize 40} & \cellcolor{green!60}{\large 21}/{\footnotesize 40} & \cellcolor{green!90}{\large 40}/{\footnotesize 45} & \cellcolor{green!50}{\large 37}/{\footnotesize 74} \tabularnewline
\rowcolor{black!10} 9 & w3m\_wc\_Str\_conv\_with\_detect & 3 & \cellcolor{green!0}{\large 0}/{\footnotesize 40} & \cellcolor{green!0}{\large 0}/{\footnotesize 40} & \cellcolor{green!0}{\large -}{\tiny -} & \cellcolor{green!50}{\large 20}/{\footnotesize 40} & \cellcolor{green!10}{\large 2}/{\footnotesize 194} & \cellcolor{green!20}{\large 24}/{\footnotesize 121} \tabularnewline
10 & bind9\_dns\_name\_fromwire & 4 & \cellcolor{green!0}{\large 0}/{\footnotesize 40} & \cellcolor{green!20}{\large 6}/{\footnotesize 40} & \cellcolor{green!0}{\large -}{\tiny -} & \cellcolor{green!20}{\large 8}/{\footnotesize 40} & \cellcolor{green!10}{\large 7}/{\footnotesize 170} & \cellcolor{green!10}{\large 12}/{\footnotesize 156} \tabularnewline
\rowcolor{black!10} 11 & gdk-pixbuf\_gdk\_pixbuf\_animation\_new\_from\_file & 4 & \cellcolor{green!40}{\large 13}/{\footnotesize 40} & \cellcolor{green!50}{\large 19}/{\footnotesize 40} & \cellcolor{green!50}{\large 19}/{\footnotesize 40} & \cellcolor{green!20}{\large 5}/{\footnotesize 40} & \cellcolor{green!40}{\large 19}/{\footnotesize 52} & \cellcolor{green!20}{\large 9}/{\footnotesize 54} \tabularnewline
12 & gdk-pixbuf\_gdk\_pixbuf\_new\_from\_data & 4 & \cellcolor{green!30}{\large 10}/{\footnotesize 40} & \cellcolor{green!50}{\large 19}/{\footnotesize 40} & \cellcolor{green!20}{\large 8}/{\footnotesize 40} & \cellcolor{green!40}{\large 16}/{\footnotesize 40} & \cellcolor{green!50}{\large 35}/{\footnotesize 75} & \cellcolor{green!20}{\large 23}/{\footnotesize 119} \tabularnewline
\rowcolor{black!10} 13 & gdk-pixbuf\_gdk\_pixbuf\_new\_from\_file & 4 & \cellcolor{green!50}{\large 19}/{\footnotesize 40} & \cellcolor{green!40}{\large 16}/{\footnotesize 40} & \cellcolor{green!40}{\large 16}/{\footnotesize 40} & \cellcolor{green!20}{\large 5}/{\footnotesize 40} & \cellcolor{green!30}{\large 17}/{\footnotesize 56} & \cellcolor{green!30}{\large 14}/{\footnotesize 48} \tabularnewline
14 & gdk-pixbuf\_gdk\_pixbuf\_new\_from\_stream & 4 & \cellcolor{green!50}{\large 19}/{\footnotesize 40} & \cellcolor{green!90}{\large 33}/{\footnotesize 40} & \cellcolor{green!90}{\large 33}/{\footnotesize 40} & \cellcolor{green!90}{\large 36}/{\footnotesize 40} & \cellcolor{green!90}{\large 39}/{\footnotesize 47} & \cellcolor{green!70}{\large 39}/{\footnotesize 58} \tabularnewline
\rowcolor{black!10} 15 & gpac\_gf\_isom\_open\_file & 4 & \cellcolor{green!20}{\large 7}/{\footnotesize 40} & \cellcolor{green!40}{\large 13}/{\footnotesize 40} & \cellcolor{green!0}{\large -}{\tiny -} & \cellcolor{green!10}{\large 3}/{\footnotesize 40} & \cellcolor{green!10}{\large 8}/{\footnotesize 121} & \cellcolor{green!10}{\large 4}/{\footnotesize 134} \tabularnewline
16 & libbpf\_bpf\_object\_\_open\_mem & 4 & \cellcolor{green!0}{\large 0}/{\footnotesize 40} & \cellcolor{green!0}{\large 0}/{\footnotesize 40} & \cellcolor{green!10}{\large 2}/{\footnotesize 40} & \cellcolor{green!30}{\large 9}/{\footnotesize 40} & \cellcolor{green!10}{\large 13}/{\footnotesize 169} & \cellcolor{green!20}{\large 19}/{\footnotesize 138} \tabularnewline
\rowcolor{black!10} 17 & libpg\_query\_pg\_query\_parse & 4 & \cellcolor{green!10}{\large 4}/{\footnotesize 40} & \cellcolor{green!80}{\large 30}/{\footnotesize 40} & \cellcolor{green!0}{\large -}{\tiny -} & \cellcolor{green!70}{\large 26}/{\footnotesize 40} & \cellcolor{green!60}{\large 39}/{\footnotesize 65} & \cellcolor{green!50}{\large 35}/{\footnotesize 69} \tabularnewline
18 & libucl\_ucl\_parser\_add\_string & 4 & \cellcolor{green!30}{\large 10}/{\footnotesize 40} & \cellcolor{green!40}{\large 16}/{\footnotesize 40} & \cellcolor{green!20}{\large 6}/{\footnotesize 40} & \cellcolor{green!30}{\large 11}/{\footnotesize 40} & \cellcolor{green!40}{\large 37}/{\footnotesize 101} & \cellcolor{green!40}{\large 35}/{\footnotesize 107} \tabularnewline
\rowcolor{black!10} 19 & oniguruma\_onig\_new & 4 & \cellcolor{green!70}{\large 28}/{\footnotesize 40} & \cellcolor{green!60}{\large 22}/{\footnotesize 40} & \cellcolor{green!60}{\large 24}/{\footnotesize 40} & \cellcolor{green!40}{\large 15}/{\footnotesize 40} & \cellcolor{green!60}{\large 38}/{\footnotesize 68} & \cellcolor{green!50}{\large 34}/{\footnotesize 73} \tabularnewline
20 & pupnp\_ixmlLoadDocumentEx & 4 & \cellcolor{green!10}{\large 3}/{\footnotesize 40} & \cellcolor{green!20}{\large 6}/{\footnotesize 40} & \cellcolor{green!20}{\large 7}/{\footnotesize 40} & \cellcolor{green!30}{\large 12}/{\footnotesize 40} & \cellcolor{green!10}{\large 1}/{\footnotesize 96} & \cellcolor{green!0}{\large 0}/{\footnotesize 101} \tabularnewline
\rowcolor{black!10} 21 & gdk-pixbuf\_gdk\_pixbuf\_new\_from\_file\_at\_scale & 5 & \cellcolor{green!50}{\large 17}/{\footnotesize 40} & \cellcolor{green!50}{\large 17}/{\footnotesize 40} & \cellcolor{green!40}{\large 14}/{\footnotesize 40} & \cellcolor{green!10}{\large 4}/{\footnotesize 40} & \cellcolor{green!50}{\large 23}/{\footnotesize 55} & \cellcolor{green!20}{\large 10}/{\footnotesize 64} \tabularnewline
22 & inchi\_GetINCHIKeyFromINCHI & 5 & \cellcolor{green!10}{\large 4}/{\footnotesize 40} & \cellcolor{green!50}{\large 17}/{\footnotesize 40} & \cellcolor{green!80}{\large 30}/{\footnotesize 40} & \cellcolor{green!40}{\large 16}/{\footnotesize 40} & \cellcolor{green!30}{\large 28}/{\footnotesize 118} & \cellcolor{green!50}{\large 35}/{\footnotesize 82} \tabularnewline
\rowcolor{black!10} 23 & libdwarf\_dwarf\_init\_b & 5 & \cellcolor{green!0}{\large 0}/{\footnotesize 40} & \cellcolor{green!10}{\large 3}/{\footnotesize 40} & \cellcolor{green!20}{\large 5}/{\footnotesize 40} & \cellcolor{green!40}{\large 16}/{\footnotesize 40} & \cellcolor{green!10}{\large 9}/{\footnotesize 178} & \cellcolor{green!20}{\large 25}/{\footnotesize 122} \tabularnewline
24 & libdwarf\_dwarf\_init\_path & 5 & \cellcolor{green!0}{\large 0}/{\footnotesize 40} & \cellcolor{green!0}{\large 0}/{\footnotesize 40} & \cellcolor{green!0}{\large 0}/{\footnotesize 40} & \cellcolor{green!20}{\large 5}/{\footnotesize 40} & \cellcolor{green!0}{\large 0}/{\footnotesize 175} & \cellcolor{green!10}{\large 3}/{\footnotesize 136} \tabularnewline
\rowcolor{black!10} 25 & liblouis\_lou\_compileString & 5 & \cellcolor{green!10}{\large 1}/{\footnotesize 40} & \cellcolor{green!10}{\large 3}/{\footnotesize 40} & \cellcolor{green!10}{\large 3}/{\footnotesize 40} & \cellcolor{green!40}{\large 16}/{\footnotesize 40} & \cellcolor{green!10}{\large 11}/{\footnotesize 173} & \cellcolor{green!20}{\large 20}/{\footnotesize 121} \tabularnewline
26 & selinux\_cil\_compile & 5 & \cellcolor{green!0}{\large 0}/{\footnotesize 40} & \cellcolor{green!0}{\large 0}/{\footnotesize 40} & \cellcolor{green!0}{\large -}{\tiny -} & \cellcolor{green!60}{\large 24}/{\footnotesize 40} & \cellcolor{green!10}{\large 7}/{\footnotesize 175} & \cellcolor{green!50}{\large 32}/{\footnotesize 76} \tabularnewline
\rowcolor{black!10} 27 & bind9\_dns\_name\_fromtext & 6 & \cellcolor{green!10}{\large 3}/{\footnotesize 40} & \cellcolor{green!40}{\large 13}/{\footnotesize 40} & \cellcolor{green!0}{\large -}{\tiny -} & \cellcolor{green!50}{\large 19}/{\footnotesize 40} & \cellcolor{green!40}{\large 33}/{\footnotesize 99} & \cellcolor{green!30}{\large 26}/{\footnotesize 111} \tabularnewline
28 & bind9\_dns\_rdata\_fromwire & 6 & \cellcolor{green!0}{\large 0}/{\footnotesize 40} & \cellcolor{green!0}{\large 0}/{\footnotesize 40} & \cellcolor{green!0}{\large -}{\tiny -} & \cellcolor{green!10}{\large 2}/{\footnotesize 40} & \cellcolor{green!10}{\large 3}/{\footnotesize 191} & \cellcolor{green!10}{\large 9}/{\footnotesize 168} \tabularnewline
\rowcolor{black!10} 29 & coturn\_stun\_is\_binding\_response & 6 & \cellcolor{green!0}{\large 0}/{\footnotesize 40} & \cellcolor{green!40}{\large 13}/{\footnotesize 40} & \cellcolor{green!0}{\large -}{\tiny -} & \cellcolor{green!40}{\large 16}/{\footnotesize 40} & \cellcolor{green!30}{\large 28}/{\footnotesize 124} & \cellcolor{green!30}{\large 28}/{\footnotesize 98} \tabularnewline
30 & coturn\_stun\_is\_command\_message & 6 & \cellcolor{green!0}{\large 0}/{\footnotesize 40} & \cellcolor{green!30}{\large 10}/{\footnotesize 40} & \cellcolor{green!30}{\large 12}/{\footnotesize 40} & \cellcolor{green!50}{\large 18}/{\footnotesize 40} & \cellcolor{green!20}{\large 24}/{\footnotesize 140} & \cellcolor{green!30}{\large 26}/{\footnotesize 102} \tabularnewline
\rowcolor{black!10} 31 & coturn\_stun\_is\_response & 6 & \cellcolor{green!0}{\large 0}/{\footnotesize 40} & \cellcolor{green!20}{\large 8}/{\footnotesize 40} & \cellcolor{green!0}{\large -}{\tiny -} & \cellcolor{green!30}{\large 12}/{\footnotesize 40} & \cellcolor{green!10}{\large 10}/{\footnotesize 132} & \cellcolor{green!10}{\large 12}/{\footnotesize 111} \tabularnewline
32 & coturn\_stun\_is\_success\_response & 6 & \cellcolor{green!0}{\large 0}/{\footnotesize 40} & \cellcolor{green!30}{\large 11}/{\footnotesize 40} & \cellcolor{green!0}{\large -}{\tiny -} & \cellcolor{green!50}{\large 18}/{\footnotesize 40} & \cellcolor{green!30}{\large 27}/{\footnotesize 124} & \cellcolor{green!20}{\large 24}/{\footnotesize 125} \tabularnewline
\rowcolor{black!10} 33 & hiredis\_redisFormatCommand & 6 & \cellcolor{green!20}{\large 7}/{\footnotesize 40} & \cellcolor{green!50}{\large 20}/{\footnotesize 40} & \cellcolor{green!0}{\large -}{\tiny -} & \cellcolor{green!50}{\large 18}/{\footnotesize 40} & \cellcolor{green!20}{\large 29}/{\footnotesize 153} & \cellcolor{green!20}{\large 25}/{\footnotesize 150} \tabularnewline
34 & igraph\_igraph\_read\_graph\_dl & 6 & \cellcolor{green!10}{\large 4}/{\footnotesize 40} & \cellcolor{green!0}{\large 0}/{\footnotesize 40} & \cellcolor{green!0}{\large 0}/{\footnotesize 40} & \cellcolor{green!10}{\large 1}/{\footnotesize 40} & \cellcolor{green!20}{\large 18}/{\footnotesize 157} & \cellcolor{green!10}{\large 17}/{\footnotesize 163} \tabularnewline
\rowcolor{black!10} 35 & igraph\_igraph\_read\_graph\_edgelist & 6 & \cellcolor{green!10}{\large 1}/{\footnotesize 40} & \cellcolor{green!0}{\large 0}/{\footnotesize 40} & \cellcolor{green!0}{\large 0}/{\footnotesize 40} & \cellcolor{green!0}{\large 0}/{\footnotesize 40} & \cellcolor{green!10}{\large 4}/{\footnotesize 190} & \cellcolor{green!10}{\large 6}/{\footnotesize 181} \tabularnewline
36 & igraph\_igraph\_read\_graph\_gml & 6 & \cellcolor{green!10}{\large 3}/{\footnotesize 40} & \cellcolor{green!0}{\large 0}/{\footnotesize 40} & \cellcolor{green!0}{\large 0}/{\footnotesize 40} & \cellcolor{green!10}{\large 1}/{\footnotesize 40} & \cellcolor{green!10}{\large 11}/{\footnotesize 164} & \cellcolor{green!10}{\large 10}/{\footnotesize 176} \tabularnewline
\rowcolor{black!10} 37 & igraph\_igraph\_read\_graph\_graphdb & 6 & \cellcolor{green!0}{\large 0}/{\footnotesize 40} & \cellcolor{green!0}{\large 0}/{\footnotesize 40} & \cellcolor{green!0}{\large 0}/{\footnotesize 40} & \cellcolor{green!10}{\large 1}/{\footnotesize 40} & \cellcolor{green!10}{\large 3}/{\footnotesize 182} & \cellcolor{green!10}{\large 13}/{\footnotesize 166} \tabularnewline
38 & igraph\_igraph\_read\_graph\_graphml & 6 & \cellcolor{green!20}{\large 5}/{\footnotesize 40} & \cellcolor{green!10}{\large 1}/{\footnotesize 40} & \cellcolor{green!10}{\large 1}/{\footnotesize 40} & \cellcolor{green!30}{\large 10}/{\footnotesize 40} & \cellcolor{green!10}{\large 13}/{\footnotesize 178} & \cellcolor{green!20}{\large 23}/{\footnotesize 126} \tabularnewline
\rowcolor{black!10} 39 & igraph\_igraph\_read\_graph\_lgl & 6 & \cellcolor{green!0}{\large 0}/{\footnotesize 40} & \cellcolor{green!0}{\large 0}/{\footnotesize 40} & \cellcolor{green!0}{\large 0}/{\footnotesize 40} & \cellcolor{green!10}{\large 4}/{\footnotesize 40} & \cellcolor{green!10}{\large 2}/{\footnotesize 195} & \cellcolor{green!10}{\large 7}/{\footnotesize 175} \tabularnewline
40 & igraph\_igraph\_read\_graph\_pajek & 6 & \cellcolor{green!20}{\large 8}/{\footnotesize 40} & \cellcolor{green!0}{\large 0}/{\footnotesize 40} & \cellcolor{green!0}{\large 0}/{\footnotesize 40} & \cellcolor{green!0}{\large 0}/{\footnotesize 40} & \cellcolor{green!20}{\large 21}/{\footnotesize 158} & \cellcolor{green!10}{\large 11}/{\footnotesize 178} \tabularnewline
\rowcolor{black!10} 41 & inchi\_GetINCHIfromINCHI & 6 & \cellcolor{green!10}{\large 1}/{\footnotesize 40} & \cellcolor{green!10}{\large 4}/{\footnotesize 40} & \cellcolor{green!30}{\large 12}/{\footnotesize 40} & \cellcolor{green!50}{\large 18}/{\footnotesize 40} & \cellcolor{green!10}{\large 20}/{\footnotesize 185} & \cellcolor{green!10}{\large 20}/{\footnotesize 185} \tabularnewline
42 & inchi\_GetStructFromINCHI & 6 & \cellcolor{green!0}{\large 0}/{\footnotesize 40} & \cellcolor{green!10}{\large 4}/{\footnotesize 40} & \cellcolor{green!20}{\large 6}/{\footnotesize 40} & \cellcolor{green!20}{\large 5}/{\footnotesize 40} & \cellcolor{green!10}{\large 14}/{\footnotesize 190} & \cellcolor{green!10}{\large 13}/{\footnotesize 184} \tabularnewline
\rowcolor{black!10} 43 & kamailio\_parse\_msg & 6 & \cellcolor{green!0}{\large 0}/{\footnotesize 40} & \cellcolor{green!20}{\large 7}/{\footnotesize 40} & \cellcolor{green!0}{\large -}{\tiny -} & \cellcolor{green!50}{\large 19}/{\footnotesize 40} & \cellcolor{green!30}{\large 24}/{\footnotesize 96} & \cellcolor{green!40}{\large 29}/{\footnotesize 90} \tabularnewline
44 & libyang\_lys\_parse\_mem & 6 & \cellcolor{green!0}{\large 0}/{\footnotesize 40} & \cellcolor{green!0}{\large 0}/{\footnotesize 40} & \cellcolor{green!0}{\large 0}/{\footnotesize 40} & \cellcolor{green!10}{\large 2}/{\footnotesize 40} & \cellcolor{green!10}{\large 13}/{\footnotesize 176} & \cellcolor{green!20}{\large 27}/{\footnotesize 150} \tabularnewline
\rowcolor{black!10} 45 & proftpd\_pr\_json\_object\_from\_text & 6 & \cellcolor{green!0}{\large 0}/{\footnotesize 40} & \cellcolor{green!0}{\large 0}/{\footnotesize 40} & \cellcolor{green!0}{\large -}{\tiny -} & \cellcolor{green!30}{\large 11}/{\footnotesize 40} & \cellcolor{green!10}{\large 9}/{\footnotesize 185} & \cellcolor{green!30}{\large 29}/{\footnotesize 121} \tabularnewline
46 & selinux\_policydb\_read & 6 & \cellcolor{green!10}{\large 1}/{\footnotesize 40} & \cellcolor{green!20}{\large 7}/{\footnotesize 40} & \cellcolor{green!0}{\large -}{\tiny -} & \cellcolor{green!30}{\large 11}/{\footnotesize 40} & \cellcolor{green!20}{\large 15}/{\footnotesize 123} & \cellcolor{green!30}{\large 27}/{\footnotesize 100} \tabularnewline
\rowcolor{black!10} 47 & kamailio\_get\_src\_address\_socket & 7 & \cellcolor{green!0}{\large 0}/{\footnotesize 40} & \cellcolor{green!0}{\large 0}/{\footnotesize 40} & \cellcolor{green!0}{\large 0}/{\footnotesize 40} & \cellcolor{green!20}{\large 7}/{\footnotesize 40} & \cellcolor{green!0}{\large 1}/{\footnotesize 142} & \cellcolor{green!10}{\large 10}/{\footnotesize 116} \tabularnewline
48 & kamailio\_get\_src\_uri & 7 & \cellcolor{green!0}{\large 0}/{\footnotesize 40} & \cellcolor{green!0}{\large 0}/{\footnotesize 40} & \cellcolor{green!0}{\large 0}/{\footnotesize 40} & \cellcolor{green!20}{\large 8}/{\footnotesize 40} & \cellcolor{green!0}{\large 0}/{\footnotesize 127} & \cellcolor{green!10}{\large 6}/{\footnotesize 118} \tabularnewline
\rowcolor{black!10} 49 & kamailio\_parse\_content\_disposition & 7 & \cellcolor{green!0}{\large 0}/{\footnotesize 40} & \cellcolor{green!0}{\large 0}/{\footnotesize 40} & \cellcolor{green!0}{\large 0}/{\footnotesize 40} & \cellcolor{green!10}{\large 4}/{\footnotesize 40} & \cellcolor{green!0}{\large 0}/{\footnotesize 176} & \cellcolor{green!10}{\large 5}/{\footnotesize 166} \tabularnewline
50 & kamailio\_parse\_diversion\_header & 7 & \cellcolor{green!0}{\large 0}/{\footnotesize 40} & \cellcolor{green!0}{\large 0}/{\footnotesize 40} & \cellcolor{green!0}{\large 0}/{\footnotesize 40} & \cellcolor{green!30}{\large 10}/{\footnotesize 40} & \cellcolor{green!0}{\large 0}/{\footnotesize 187} & \cellcolor{green!10}{\large 9}/{\footnotesize 143} \tabularnewline
\rowcolor{black!10} 51 & kamailio\_parse\_from\_header & 7 & \cellcolor{green!0}{\large 0}/{\footnotesize 40} & \cellcolor{green!0}{\large 0}/{\footnotesize 40} & \cellcolor{green!0}{\large -}{\tiny -} & \cellcolor{green!0}{\large 0}/{\footnotesize 40} & \cellcolor{green!0}{\large 0}/{\footnotesize 187} & \cellcolor{green!0}{\large 1}/{\footnotesize 173} \tabularnewline
52 & kamailio\_parse\_from\_uri & 7 & \cellcolor{green!0}{\large 0}/{\footnotesize 40} & \cellcolor{green!0}{\large 0}/{\footnotesize 40} & \cellcolor{green!0}{\large -}{\tiny -} & \cellcolor{green!10}{\large 1}/{\footnotesize 40} & \cellcolor{green!0}{\large 0}/{\footnotesize 192} & \cellcolor{green!0}{\large 0}/{\footnotesize 178} \tabularnewline
\rowcolor{black!10} 53 & kamailio\_parse\_headers & 7 & \cellcolor{green!0}{\large 0}/{\footnotesize 40} & \cellcolor{green!0}{\large 0}/{\footnotesize 40} & \cellcolor{green!0}{\large -}{\tiny -} & \cellcolor{green!0}{\large 0}/{\footnotesize 40} & \cellcolor{green!0}{\large 0}/{\footnotesize 141} & \cellcolor{green!0}{\large 1}/{\footnotesize 135} \tabularnewline
54 & kamailio\_parse\_identityinfo\_header & 7 & \cellcolor{green!0}{\large 0}/{\footnotesize 40} & \cellcolor{green!0}{\large 0}/{\footnotesize 40} & \cellcolor{green!0}{\large -}{\tiny -} & \cellcolor{green!40}{\large 13}/{\footnotesize 40} & \cellcolor{green!0}{\large 0}/{\footnotesize 175} & \cellcolor{green!10}{\large 10}/{\footnotesize 147} \tabularnewline
\rowcolor{black!10} 55 & kamailio\_parse\_pai\_header & 7 & \cellcolor{green!0}{\large 0}/{\footnotesize 40} & \cellcolor{green!0}{\large 0}/{\footnotesize 40} & \cellcolor{green!0}{\large -}{\tiny -} & \cellcolor{green!40}{\large 15}/{\footnotesize 40} & \cellcolor{green!0}{\large 0}/{\footnotesize 161} & \cellcolor{green!10}{\large 10}/{\footnotesize 126} \tabularnewline
56 & kamailio\_parse\_privacy & 7 & \cellcolor{green!0}{\large 0}/{\footnotesize 40} & \cellcolor{green!0}{\large 0}/{\footnotesize 40} & \cellcolor{green!0}{\large 0}/{\footnotesize 40} & \cellcolor{green!20}{\large 5}/{\footnotesize 40} & \cellcolor{green!0}{\large 0}/{\footnotesize 191} & \cellcolor{green!10}{\large 8}/{\footnotesize 150} \tabularnewline
\rowcolor{black!10} 57 & kamailio\_parse\_record\_route\_headers & 7 & \cellcolor{green!0}{\large 0}/{\footnotesize 40} & \cellcolor{green!0}{\large 0}/{\footnotesize 40} & \cellcolor{green!0}{\large -}{\tiny -} & \cellcolor{green!80}{\large 32}/{\footnotesize 40} & \cellcolor{green!0}{\large 0}/{\footnotesize 161} & \cellcolor{green!30}{\large 26}/{\footnotesize 95} \tabularnewline
58 & kamailio\_parse\_refer\_to\_header & 7 & \cellcolor{green!0}{\large 0}/{\footnotesize 40} & \cellcolor{green!0}{\large 0}/{\footnotesize 40} & \cellcolor{green!0}{\large -}{\tiny -} & \cellcolor{green!10}{\large 1}/{\footnotesize 40} & \cellcolor{green!0}{\large 0}/{\footnotesize 162} & \cellcolor{green!10}{\large 11}/{\footnotesize 140} \tabularnewline
\rowcolor{black!10} 59 & kamailio\_parse\_route\_headers & 7 & \cellcolor{green!0}{\large 0}/{\footnotesize 40} & \cellcolor{green!0}{\large 0}/{\footnotesize 40} & \cellcolor{green!0}{\large -}{\tiny -} & \cellcolor{green!80}{\large 31}/{\footnotesize 40} & \cellcolor{green!0}{\large 0}/{\footnotesize 173} & \cellcolor{green!30}{\large 26}/{\footnotesize 100} \tabularnewline
60 & kamailio\_parse\_to\_header & 7 & \cellcolor{green!0}{\large 0}/{\footnotesize 40} & \cellcolor{green!0}{\large 0}/{\footnotesize 40} & \cellcolor{green!0}{\large -}{\tiny -} & \cellcolor{green!10}{\large 2}/{\footnotesize 40} & \cellcolor{green!0}{\large 0}/{\footnotesize 153} & \cellcolor{green!10}{\large 6}/{\footnotesize 147} \tabularnewline
\rowcolor{black!10} 61 & kamailio\_parse\_to\_uri & 7 & \cellcolor{green!0}{\large 0}/{\footnotesize 40} & \cellcolor{green!0}{\large 0}/{\footnotesize 40} & \cellcolor{green!0}{\large -}{\tiny -} & \cellcolor{green!10}{\large 1}/{\footnotesize 40} & \cellcolor{green!0}{\large 0}/{\footnotesize 156} & \cellcolor{green!0}{\large 1}/{\footnotesize 167} \tabularnewline
62 & libyang\_lyd\_parse\_data\_mem & 7 & \cellcolor{green!10}{\large 2}/{\footnotesize 40} & \cellcolor{green!10}{\large 3}/{\footnotesize 40} & \cellcolor{green!10}{\large 2}/{\footnotesize 40} & \cellcolor{green!10}{\large 1}/{\footnotesize 40} & \cellcolor{green!20}{\large 20}/{\footnotesize 157} & \cellcolor{green!20}{\large 24}/{\footnotesize 149} \tabularnewline
\rowcolor{black!10} 63 & bind9\_dns\_message\_parse & 8 & \cellcolor{green!0}{\large 0}/{\footnotesize 40} & \cellcolor{green!0}{\large 0}/{\footnotesize 40} & \cellcolor{green!0}{\large -}{\tiny -} & \cellcolor{green!0}{\large 0}/{\footnotesize 40} & \cellcolor{green!0}{\large 1}/{\footnotesize 193} & \cellcolor{green!10}{\large 2}/{\footnotesize 182} \tabularnewline
64 & igraph\_igraph\_read\_graph\_ncol & 8 & \cellcolor{green!10}{\large 1}/{\footnotesize 40} & \cellcolor{green!0}{\large 0}/{\footnotesize 40} & \cellcolor{green!0}{\large 0}/{\footnotesize 40} & \cellcolor{green!0}{\large 0}/{\footnotesize 40} & \cellcolor{green!0}{\large 1}/{\footnotesize 194} & \cellcolor{green!10}{\large 4}/{\footnotesize 190} \tabularnewline
\rowcolor{black!10} 65 & pjsip\_pj\_json\_parse & 8 & \cellcolor{green!0}{\large 0}/{\footnotesize 40} & \cellcolor{green!10}{\large 2}/{\footnotesize 40} & \cellcolor{green!0}{\large 0}/{\footnotesize 40} & \cellcolor{green!0}{\large 0}/{\footnotesize 40} & \cellcolor{green!20}{\large 20}/{\footnotesize 160} & \cellcolor{green!20}{\large 20}/{\footnotesize 159} \tabularnewline
66 & pjsip\_pj\_xml\_parse & 8 & \cellcolor{green!10}{\large 1}/{\footnotesize 40} & \cellcolor{green!20}{\large 6}/{\footnotesize 40} & \cellcolor{green!10}{\large 2}/{\footnotesize 40} & \cellcolor{green!0}{\large 0}/{\footnotesize 40} & \cellcolor{green!20}{\large 27}/{\footnotesize 135} & \cellcolor{green!10}{\large 15}/{\footnotesize 160} \tabularnewline
\rowcolor{black!10} 67 & pjsip\_pjmedia\_sdp\_parse & 8 & \cellcolor{green!10}{\large 4}/{\footnotesize 40} & \cellcolor{green!30}{\large 9}/{\footnotesize 40} & \cellcolor{green!30}{\large 9}/{\footnotesize 40} & \cellcolor{green!10}{\large 3}/{\footnotesize 40} & \cellcolor{green!20}{\large 24}/{\footnotesize 142} & \cellcolor{green!20}{\large 26}/{\footnotesize 131} \tabularnewline
68 & quickjs\_lre\_compile & 8 & \cellcolor{green!0}{\large 0}/{\footnotesize 40} & \cellcolor{green!0}{\large 0}/{\footnotesize 40} & \cellcolor{green!0}{\large -}{\tiny -} & \cellcolor{green!0}{\large 0}/{\footnotesize 40} & \cellcolor{green!0}{\large 0}/{\footnotesize 200} & \cellcolor{green!0}{\large 0}/{\footnotesize 200} \tabularnewline
\rowcolor{black!10} 69 & bind9\_isc\_lex\_getmastertoken & 9 & \cellcolor{green!0}{\large 0}/{\footnotesize 40} & \cellcolor{green!0}{\large 0}/{\footnotesize 40} & \cellcolor{green!0}{\large -}{\tiny -} & \cellcolor{green!0}{\large 0}/{\footnotesize 40} & \cellcolor{green!10}{\large 2}/{\footnotesize 180} & \cellcolor{green!0}{\large 1}/{\footnotesize 193} \tabularnewline
70 & bind9\_isc\_lex\_gettoken & 9 & \cellcolor{green!10}{\large 1}/{\footnotesize 40} & \cellcolor{green!0}{\large 0}/{\footnotesize 40} & \cellcolor{green!0}{\large -}{\tiny -} & \cellcolor{green!10}{\large 1}/{\footnotesize 40} & \cellcolor{green!10}{\large 5}/{\footnotesize 193} & \cellcolor{green!10}{\large 3}/{\footnotesize 168} \tabularnewline
\rowcolor{black!10} 71 & quickjs\_JS\_Eval & 9 & \cellcolor{green!0}{\large 0}/{\footnotesize 40} & \cellcolor{green!50}{\large 17}/{\footnotesize 40} & \cellcolor{green!0}{\large -}{\tiny -} & \cellcolor{green!20}{\large 8}/{\footnotesize 40} & \cellcolor{green!40}{\large 35}/{\footnotesize 92} & \cellcolor{green!20}{\large 26}/{\footnotesize 130} \tabularnewline
72 & igraph\_igraph\_edge\_connectivity & 10 & \cellcolor{green!0}{\large 0}/{\footnotesize 40} & \cellcolor{green!0}{\large 0}/{\footnotesize 40} & \cellcolor{green!0}{\large 0}/{\footnotesize 40} & \cellcolor{green!0}{\large 0}/{\footnotesize 40} & \cellcolor{green!0}{\large 1}/{\footnotesize 134} & \cellcolor{green!0}{\large 0}/{\footnotesize 123} \tabularnewline
\rowcolor{black!10} 73 & pjsip\_pj\_stun\_msg\_decode & 10 & \cellcolor{green!0}{\large 0}/{\footnotesize 40} & \cellcolor{green!10}{\large 2}/{\footnotesize 40} & \cellcolor{green!0}{\large 0}/{\footnotesize 40} & \cellcolor{green!0}{\large 0}/{\footnotesize 40} & \cellcolor{green!10}{\large 8}/{\footnotesize 182} & \cellcolor{green!10}{\large 12}/{\footnotesize 175} \tabularnewline
74 & bind9\_dns\_message\_checksig & 11 & \cellcolor{green!0}{\large 0}/{\footnotesize 40} & \cellcolor{green!0}{\large 0}/{\footnotesize 40} & \cellcolor{green!0}{\large -}{\tiny -} & \cellcolor{green!0}{\large 0}/{\footnotesize 40} & \cellcolor{green!0}{\large 0}/{\footnotesize 192} & \cellcolor{green!0}{\large 0}/{\footnotesize 197} \tabularnewline
\rowcolor{black!10} 75 & libzip\_zip\_fread & 11 & \cellcolor{green!50}{\large 18}/{\footnotesize 40} & \cellcolor{green!50}{\large 17}/{\footnotesize 40} & \cellcolor{green!30}{\large 9}/{\footnotesize 40} & \cellcolor{green!20}{\large 8}/{\footnotesize 40} & \cellcolor{green!40}{\large 34}/{\footnotesize 89} & \cellcolor{green!30}{\large 29}/{\footnotesize 115} \tabularnewline
76 & bind9\_dns\_rdata\_fromtext & 12 & \cellcolor{green!0}{\large 0}/{\footnotesize 40} & \cellcolor{green!0}{\large 0}/{\footnotesize 40} & \cellcolor{green!0}{\large -}{\tiny -} & \cellcolor{green!0}{\large 0}/{\footnotesize 40} & \cellcolor{green!0}{\large 1}/{\footnotesize 188} & \cellcolor{green!10}{\large 2}/{\footnotesize 189} \tabularnewline
\rowcolor{black!10} 77 & igraph\_igraph\_all\_minimal\_st\_separators & 12 & \cellcolor{green!0}{\large 0}/{\footnotesize 40} & \cellcolor{green!10}{\large 4}/{\footnotesize 40} & \cellcolor{green!10}{\large 1}/{\footnotesize 40} & \cellcolor{green!0}{\large 0}/{\footnotesize 40} & \cellcolor{green!20}{\large 25}/{\footnotesize 124} & \cellcolor{green!10}{\large 11}/{\footnotesize 160} \tabularnewline
78 & igraph\_igraph\_minimum\_size\_separators & 12 & \cellcolor{green!0}{\large 0}/{\footnotesize 40} & \cellcolor{green!20}{\large 8}/{\footnotesize 40} & \cellcolor{green!10}{\large 3}/{\footnotesize 40} & \cellcolor{green!10}{\large 1}/{\footnotesize 40} & \cellcolor{green!10}{\large 17}/{\footnotesize 157} & \cellcolor{green!10}{\large 10}/{\footnotesize 175} \tabularnewline
\rowcolor{black!10} 79 & pjsip\_pjsip\_parse\_msg & 12 & \cellcolor{green!0}{\large 0}/{\footnotesize 40} & \cellcolor{green!0}{\large 0}/{\footnotesize 40} & \cellcolor{green!0}{\large 0}/{\footnotesize 40} & \cellcolor{green!0}{\large 0}/{\footnotesize 40} & \cellcolor{green!0}{\large 0}/{\footnotesize 193} & \cellcolor{green!0}{\large 1}/{\footnotesize 189} \tabularnewline
80 & igraph\_igraph\_automorphism\_group & 13 & \cellcolor{green!0}{\large 0}/{\footnotesize 40} & \cellcolor{green!10}{\large 1}/{\footnotesize 40} & \cellcolor{green!10}{\large 1}/{\footnotesize 40} & \cellcolor{green!0}{\large 0}/{\footnotesize 40} & \cellcolor{green!10}{\large 2}/{\footnotesize 190} & \cellcolor{green!10}{\large 10}/{\footnotesize 173} \tabularnewline
\rowcolor{black!10} 81 & libmodbus\_modbus\_read\_bits & 15 & \cellcolor{green!0}{\large 0}/{\footnotesize 40} & \cellcolor{green!0}{\large 0}/{\footnotesize 40} & \cellcolor{green!0}{\large 0}/{\footnotesize 40} & \cellcolor{green!0}{\large 0}/{\footnotesize 40} & \cellcolor{green!0}{\large 0}/{\footnotesize 75} & \cellcolor{green!0}{\large 0}/{\footnotesize 100} \tabularnewline
82 & libmodbus\_modbus\_read\_registers & 15 & \cellcolor{green!0}{\large 0}/{\footnotesize 40} & \cellcolor{green!0}{\large 0}/{\footnotesize 40} & \cellcolor{green!0}{\large 0}/{\footnotesize 40} & \cellcolor{green!0}{\large 0}/{\footnotesize 40} & \cellcolor{green!0}{\large 0}/{\footnotesize 77} & \cellcolor{green!0}{\large 0}/{\footnotesize 98} \tabularnewline
\rowcolor{black!10} 83 & civetweb\_mg\_get\_response & 17 & \cellcolor{green!0}{\large 0}/{\footnotesize 40} & \cellcolor{green!0}{\large 0}/{\footnotesize 40} & \cellcolor{green!0}{\large 0}/{\footnotesize 40} & \cellcolor{green!0}{\large 0}/{\footnotesize 40} & \cellcolor{green!0}{\large 0}/{\footnotesize 154} & \cellcolor{green!0}{\large 0}/{\footnotesize 128} \tabularnewline
84 & bind9\_dns\_master\_loadbuffer & 20 & \cellcolor{green!0}{\large 0}/{\footnotesize 40} & \cellcolor{green!0}{\large 0}/{\footnotesize 40} & \cellcolor{green!0}{\large -}{\tiny -} & \cellcolor{green!0}{\large 0}/{\footnotesize 40} & \cellcolor{green!0}{\large 0}/{\footnotesize 186} & \cellcolor{green!0}{\large 1}/{\footnotesize 183} \tabularnewline
\rowcolor{black!10} 85 & libmodbus\_modbus\_receive & 33 & \cellcolor{green!0}{\large 0}/{\footnotesize 40} & \cellcolor{green!0}{\large 0}/{\footnotesize 40} & \cellcolor{green!0}{\large 0}/{\footnotesize 40} & \cellcolor{green!0}{\large 0}/{\footnotesize 40} & \cellcolor{green!0}{\large 0}/{\footnotesize 121} & \cellcolor{green!0}{\large 0}/{\footnotesize 123} \tabularnewline
86 & tmux\_input\_parse\_buffer & 42 & \cellcolor{green!0}{\large 0}/{\footnotesize 40} & \cellcolor{green!0}{\large 0}/{\footnotesize 40} & \cellcolor{green!0}{\large -}{\tiny -} & \cellcolor{green!0}{\large 0}/{\footnotesize 40} & \cellcolor{green!0}{\large 0}/{\footnotesize 194} & \cellcolor{green!0}{\large 0}/{\footnotesize 171} \tabularnewline

\bottomrule
%\end{tabular}
%}
%\end{table*}
\end{xltabular}
}
\twocolumn



% model: gpt-4-0613, temp: 1.5

\onecolumn
{\small %
\begin{xltabular}[h]{\textwidth}{ccccccccc}
%\begin{table*}[!t]
%\centering
\caption{Evaluation Result of model gpt-4-0613 with temperature 1.5.} \\
%\resizebox{1.0\linewidth}{!}{
%\begin{tabular}{cccccccccc}
\toprule
Index & Question & Score & NAIVE-40 & BACTX-40 & DOCTX-40 & UGCTX-40 & BA-ITER-40 & ALL-ITER-40 \tabularnewline
\midrule
\rowcolor{black!10} 1 & coturn\_stun\_is\_command\_message\_full\_check\_str & 1 & \cellcolor{green!0}{\large 0}/{\footnotesize 40} & \cellcolor{green!30}{\large 11}/{\footnotesize 40} & \cellcolor{green!0}{\large -}{\tiny -} & \cellcolor{green!10}{\large 4}/{\footnotesize 40} & \cellcolor{green!30}{\large 15}/{\footnotesize 50} & \cellcolor{green!20}{\large 8}/{\footnotesize 52} \tabularnewline
2 & kamailio\_parse\_uri & 1 & \cellcolor{green!0}{\large 0}/{\footnotesize 40} & \cellcolor{green!30}{\large 10}/{\footnotesize 40} & \cellcolor{green!0}{\large -}{\tiny -} & \cellcolor{green!20}{\large 5}/{\footnotesize 40} & \cellcolor{green!30}{\large 11}/{\footnotesize 51} & \cellcolor{green!30}{\large 12}/{\footnotesize 51} \tabularnewline
\rowcolor{black!10} 3 & coturn\_stun\_check\_message\_integrity\_str & 2 & \cellcolor{green!0}{\large 0}/{\footnotesize 40} & \cellcolor{green!0}{\large 0}/{\footnotesize 40} & \cellcolor{green!0}{\large -}{\tiny -} & \cellcolor{green!0}{\large 0}/{\footnotesize 40} & \cellcolor{green!0}{\large 0}/{\footnotesize 45} & \cellcolor{green!0}{\large 0}/{\footnotesize 47} \tabularnewline
4 & libiec61850\_MmsValue\_decodeMmsData & 2 & \cellcolor{green!0}{\large 0}/{\footnotesize 40} & \cellcolor{green!10}{\large 3}/{\footnotesize 40} & \cellcolor{green!10}{\large 4}/{\footnotesize 40} & \cellcolor{green!10}{\large 3}/{\footnotesize 40} & \cellcolor{green!10}{\large 2}/{\footnotesize 51} & \cellcolor{green!10}{\large 4}/{\footnotesize 54} \tabularnewline
\rowcolor{black!10} 5 & md4c\_md\_html & 2 & \cellcolor{green!0}{\large 0}/{\footnotesize 40} & \cellcolor{green!0}{\large 0}/{\footnotesize 40} & \cellcolor{green!0}{\large 0}/{\footnotesize 40} & \cellcolor{green!0}{\large 0}/{\footnotesize 40} & \cellcolor{green!10}{\large 3}/{\footnotesize 52} & \cellcolor{green!0}{\large 0}/{\footnotesize 53} \tabularnewline
6 & spdk\_spdk\_json\_parse & 2 & \cellcolor{green!0}{\large 0}/{\footnotesize 40} & \cellcolor{green!10}{\large 4}/{\footnotesize 40} & \cellcolor{green!0}{\large -}{\tiny -} & \cellcolor{green!10}{\large 2}/{\footnotesize 40} & \cellcolor{green!10}{\large 5}/{\footnotesize 59} & \cellcolor{green!10}{\large 5}/{\footnotesize 55} \tabularnewline
\rowcolor{black!10} 7 & croaring\_roaring\_bitmap\_portable\_deserialize\_safe & 3 & \cellcolor{green!10}{\large 1}/{\footnotesize 40} & \cellcolor{green!20}{\large 8}/{\footnotesize 40} & \cellcolor{green!20}{\large 8}/{\footnotesize 40} & \cellcolor{green!20}{\large 6}/{\footnotesize 40} & \cellcolor{green!20}{\large 9}/{\footnotesize 59} & \cellcolor{green!20}{\large 7}/{\footnotesize 46} \tabularnewline
8 & lua\_luaL\_loadbufferx & 3 & \cellcolor{green!20}{\large 5}/{\footnotesize 40} & \cellcolor{green!10}{\large 3}/{\footnotesize 40} & \cellcolor{green!20}{\large 5}/{\footnotesize 40} & \cellcolor{green!10}{\large 2}/{\footnotesize 40} & \cellcolor{green!10}{\large 5}/{\footnotesize 47} & \cellcolor{green!10}{\large 2}/{\footnotesize 49} \tabularnewline
\rowcolor{black!10} 9 & w3m\_wc\_Str\_conv\_with\_detect & 3 & \cellcolor{green!0}{\large 0}/{\footnotesize 40} & \cellcolor{green!0}{\large 0}/{\footnotesize 40} & \cellcolor{green!0}{\large -}{\tiny -} & \cellcolor{green!0}{\large 0}/{\footnotesize 40} & \cellcolor{green!0}{\large 0}/{\footnotesize 46} & \cellcolor{green!0}{\large 0}/{\footnotesize 50} \tabularnewline
10 & bind9\_dns\_name\_fromwire & 4 & \cellcolor{green!0}{\large 0}/{\footnotesize 40} & \cellcolor{green!0}{\large 0}/{\footnotesize 40} & \cellcolor{green!0}{\large -}{\tiny -} & \cellcolor{green!0}{\large 0}/{\footnotesize 40} & \cellcolor{green!0}{\large 0}/{\footnotesize 49} & \cellcolor{green!0}{\large 0}/{\footnotesize 49} \tabularnewline
\rowcolor{black!10} 11 & gdk-pixbuf\_gdk\_pixbuf\_animation\_new\_from\_file & 4 & \cellcolor{green!0}{\large 0}/{\footnotesize 40} & \cellcolor{green!10}{\large 1}/{\footnotesize 40} & \cellcolor{green!10}{\large 2}/{\footnotesize 40} & \cellcolor{green!0}{\large 0}/{\footnotesize 40} & \cellcolor{green!10}{\large 2}/{\footnotesize 52} & \cellcolor{green!0}{\large 0}/{\footnotesize 49} \tabularnewline
12 & gdk-pixbuf\_gdk\_pixbuf\_new\_from\_data & 4 & \cellcolor{green!0}{\large 0}/{\footnotesize 40} & \cellcolor{green!0}{\large 0}/{\footnotesize 40} & \cellcolor{green!10}{\large 1}/{\footnotesize 40} & \cellcolor{green!10}{\large 1}/{\footnotesize 40} & \cellcolor{green!10}{\large 1}/{\footnotesize 50} & \cellcolor{green!0}{\large 0}/{\footnotesize 52} \tabularnewline
\rowcolor{black!10} 13 & gdk-pixbuf\_gdk\_pixbuf\_new\_from\_file & 4 & \cellcolor{green!10}{\large 2}/{\footnotesize 40} & \cellcolor{green!10}{\large 1}/{\footnotesize 40} & \cellcolor{green!0}{\large 0}/{\footnotesize 40} & \cellcolor{green!10}{\large 1}/{\footnotesize 40} & \cellcolor{green!10}{\large 1}/{\footnotesize 53} & \cellcolor{green!0}{\large 0}/{\footnotesize 47} \tabularnewline
14 & gdk-pixbuf\_gdk\_pixbuf\_new\_from\_stream & 4 & \cellcolor{green!10}{\large 2}/{\footnotesize 40} & \cellcolor{green!10}{\large 3}/{\footnotesize 40} & \cellcolor{green!20}{\large 7}/{\footnotesize 40} & \cellcolor{green!10}{\large 3}/{\footnotesize 40} & \cellcolor{green!10}{\large 2}/{\footnotesize 49} & \cellcolor{green!10}{\large 3}/{\footnotesize 52} \tabularnewline
\rowcolor{black!10} 15 & gpac\_gf\_isom\_open\_file & 4 & \cellcolor{green!0}{\large 0}/{\footnotesize 40} & \cellcolor{green!0}{\large 0}/{\footnotesize 40} & \cellcolor{green!0}{\large -}{\tiny -} & \cellcolor{green!0}{\large 0}/{\footnotesize 40} & \cellcolor{green!0}{\large 0}/{\footnotesize 51} & \cellcolor{green!0}{\large 0}/{\footnotesize 45} \tabularnewline
16 & libbpf\_bpf\_object\_\_open\_mem & 4 & \cellcolor{green!0}{\large 0}/{\footnotesize 40} & \cellcolor{green!0}{\large 0}/{\footnotesize 40} & \cellcolor{green!0}{\large 0}/{\footnotesize 40} & \cellcolor{green!10}{\large 3}/{\footnotesize 40} & \cellcolor{green!0}{\large 0}/{\footnotesize 58} & \cellcolor{green!10}{\large 5}/{\footnotesize 56} \tabularnewline
\rowcolor{black!10} 17 & libpg\_query\_pg\_query\_parse & 4 & \cellcolor{green!0}{\large 0}/{\footnotesize 40} & \cellcolor{green!20}{\large 6}/{\footnotesize 40} & \cellcolor{green!0}{\large -}{\tiny -} & \cellcolor{green!20}{\large 7}/{\footnotesize 40} & \cellcolor{green!20}{\large 11}/{\footnotesize 55} & \cellcolor{green!20}{\large 11}/{\footnotesize 59} \tabularnewline
18 & libucl\_ucl\_parser\_add\_string & 4 & \cellcolor{green!10}{\large 1}/{\footnotesize 40} & \cellcolor{green!10}{\large 1}/{\footnotesize 40} & \cellcolor{green!10}{\large 2}/{\footnotesize 40} & \cellcolor{green!0}{\large 0}/{\footnotesize 40} & \cellcolor{green!0}{\large 0}/{\footnotesize 66} & \cellcolor{green!0}{\large 0}/{\footnotesize 62} \tabularnewline
\rowcolor{black!10} 19 & oniguruma\_onig\_new & 4 & \cellcolor{green!0}{\large 0}/{\footnotesize 40} & \cellcolor{green!10}{\large 1}/{\footnotesize 40} & \cellcolor{green!0}{\large 0}/{\footnotesize 40} & \cellcolor{green!0}{\large 0}/{\footnotesize 40} & \cellcolor{green!10}{\large 2}/{\footnotesize 54} & \cellcolor{green!0}{\large 0}/{\footnotesize 50} \tabularnewline
20 & pupnp\_ixmlLoadDocumentEx & 4 & \cellcolor{green!0}{\large 0}/{\footnotesize 40} & \cellcolor{green!0}{\large 0}/{\footnotesize 40} & \cellcolor{green!0}{\large 0}/{\footnotesize 40} & \cellcolor{green!10}{\large 2}/{\footnotesize 40} & \cellcolor{green!0}{\large 0}/{\footnotesize 51} & \cellcolor{green!0}{\large 0}/{\footnotesize 58} \tabularnewline
\rowcolor{black!10} 21 & gdk-pixbuf\_gdk\_pixbuf\_new\_from\_file\_at\_scale & 5 & \cellcolor{green!0}{\large 0}/{\footnotesize 40} & \cellcolor{green!0}{\large 0}/{\footnotesize 40} & \cellcolor{green!10}{\large 1}/{\footnotesize 40} & \cellcolor{green!0}{\large 0}/{\footnotesize 40} & \cellcolor{green!0}{\large 0}/{\footnotesize 42} & \cellcolor{green!0}{\large 0}/{\footnotesize 42} \tabularnewline
22 & inchi\_GetINCHIKeyFromINCHI & 5 & \cellcolor{green!0}{\large 0}/{\footnotesize 40} & \cellcolor{green!10}{\large 2}/{\footnotesize 40} & \cellcolor{green!10}{\large 3}/{\footnotesize 40} & \cellcolor{green!0}{\large 0}/{\footnotesize 40} & \cellcolor{green!10}{\large 4}/{\footnotesize 51} & \cellcolor{green!10}{\large 2}/{\footnotesize 53} \tabularnewline
\rowcolor{black!10} 23 & libdwarf\_dwarf\_init\_b & 5 & \cellcolor{green!0}{\large 0}/{\footnotesize 40} & \cellcolor{green!0}{\large 0}/{\footnotesize 40} & \cellcolor{green!0}{\large 0}/{\footnotesize 40} & \cellcolor{green!0}{\large 0}/{\footnotesize 40} & \cellcolor{green!0}{\large 0}/{\footnotesize 47} & \cellcolor{green!0}{\large 0}/{\footnotesize 48} \tabularnewline
24 & libdwarf\_dwarf\_init\_path & 5 & \cellcolor{green!0}{\large 0}/{\footnotesize 40} & \cellcolor{green!0}{\large 0}/{\footnotesize 40} & \cellcolor{green!0}{\large 0}/{\footnotesize 40} & \cellcolor{green!0}{\large 0}/{\footnotesize 40} & \cellcolor{green!0}{\large 0}/{\footnotesize 47} & \cellcolor{green!0}{\large 0}/{\footnotesize 44} \tabularnewline
\rowcolor{black!10} 25 & liblouis\_lou\_compileString & 5 & \cellcolor{green!0}{\large 0}/{\footnotesize 40} & \cellcolor{green!10}{\large 1}/{\footnotesize 40} & \cellcolor{green!10}{\large 4}/{\footnotesize 40} & \cellcolor{green!20}{\large 5}/{\footnotesize 40} & \cellcolor{green!10}{\large 2}/{\footnotesize 59} & \cellcolor{green!10}{\large 3}/{\footnotesize 51} \tabularnewline
26 & selinux\_cil\_compile & 5 & \cellcolor{green!0}{\large 0}/{\footnotesize 40} & \cellcolor{green!0}{\large 0}/{\footnotesize 40} & \cellcolor{green!0}{\large -}{\tiny -} & \cellcolor{green!10}{\large 1}/{\footnotesize 40} & \cellcolor{green!0}{\large 0}/{\footnotesize 49} & \cellcolor{green!10}{\large 2}/{\footnotesize 47} \tabularnewline
\rowcolor{black!10} 27 & bind9\_dns\_name\_fromtext & 6 & \cellcolor{green!0}{\large 0}/{\footnotesize 40} & \cellcolor{green!0}{\large 0}/{\footnotesize 40} & \cellcolor{green!0}{\large -}{\tiny -} & \cellcolor{green!10}{\large 3}/{\footnotesize 40} & \cellcolor{green!0}{\large 0}/{\footnotesize 52} & \cellcolor{green!0}{\large 0}/{\footnotesize 50} \tabularnewline
28 & bind9\_dns\_rdata\_fromwire & 6 & \cellcolor{green!0}{\large 0}/{\footnotesize 40} & \cellcolor{green!0}{\large 0}/{\footnotesize 40} & \cellcolor{green!0}{\large -}{\tiny -} & \cellcolor{green!0}{\large 0}/{\footnotesize 40} & \cellcolor{green!0}{\large 0}/{\footnotesize 47} & \cellcolor{green!0}{\large 0}/{\footnotesize 52} \tabularnewline
\rowcolor{black!10} 29 & coturn\_stun\_is\_binding\_response & 6 & \cellcolor{green!0}{\large 0}/{\footnotesize 40} & \cellcolor{green!0}{\large 0}/{\footnotesize 40} & \cellcolor{green!0}{\large -}{\tiny -} & \cellcolor{green!10}{\large 3}/{\footnotesize 40} & \cellcolor{green!10}{\large 4}/{\footnotesize 64} & \cellcolor{green!10}{\large 4}/{\footnotesize 50} \tabularnewline
30 & coturn\_stun\_is\_command\_message & 6 & \cellcolor{green!0}{\large 0}/{\footnotesize 40} & \cellcolor{green!10}{\large 3}/{\footnotesize 40} & \cellcolor{green!10}{\large 4}/{\footnotesize 40} & \cellcolor{green!10}{\large 4}/{\footnotesize 40} & \cellcolor{green!10}{\large 5}/{\footnotesize 64} & \cellcolor{green!10}{\large 5}/{\footnotesize 48} \tabularnewline
\rowcolor{black!10} 31 & coturn\_stun\_is\_response & 6 & \cellcolor{green!0}{\large 0}/{\footnotesize 40} & \cellcolor{green!10}{\large 2}/{\footnotesize 40} & \cellcolor{green!0}{\large -}{\tiny -} & \cellcolor{green!10}{\large 2}/{\footnotesize 40} & \cellcolor{green!10}{\large 3}/{\footnotesize 63} & \cellcolor{green!10}{\large 4}/{\footnotesize 51} \tabularnewline
32 & coturn\_stun\_is\_success\_response & 6 & \cellcolor{green!0}{\large 0}/{\footnotesize 40} & \cellcolor{green!10}{\large 4}/{\footnotesize 40} & \cellcolor{green!0}{\large -}{\tiny -} & \cellcolor{green!10}{\large 4}/{\footnotesize 40} & \cellcolor{green!10}{\large 1}/{\footnotesize 69} & \cellcolor{green!10}{\large 2}/{\footnotesize 58} \tabularnewline
\rowcolor{black!10} 33 & hiredis\_redisFormatCommand & 6 & \cellcolor{green!0}{\large 0}/{\footnotesize 40} & \cellcolor{green!10}{\large 4}/{\footnotesize 40} & \cellcolor{green!0}{\large -}{\tiny -} & \cellcolor{green!10}{\large 2}/{\footnotesize 40} & \cellcolor{green!10}{\large 5}/{\footnotesize 61} & \cellcolor{green!0}{\large 0}/{\footnotesize 52} \tabularnewline
34 & igraph\_igraph\_read\_graph\_dl & 6 & \cellcolor{green!0}{\large 0}/{\footnotesize 40} & \cellcolor{green!0}{\large 0}/{\footnotesize 40} & \cellcolor{green!0}{\large 0}/{\footnotesize 40} & \cellcolor{green!0}{\large 0}/{\footnotesize 40} & \cellcolor{green!10}{\large 1}/{\footnotesize 63} & \cellcolor{green!0}{\large 0}/{\footnotesize 56} \tabularnewline
\rowcolor{black!10} 35 & igraph\_igraph\_read\_graph\_edgelist & 6 & \cellcolor{green!0}{\large 0}/{\footnotesize 40} & \cellcolor{green!0}{\large 0}/{\footnotesize 40} & \cellcolor{green!0}{\large 0}/{\footnotesize 40} & \cellcolor{green!0}{\large 0}/{\footnotesize 40} & \cellcolor{green!0}{\large 0}/{\footnotesize 55} & \cellcolor{green!0}{\large 0}/{\footnotesize 52} \tabularnewline
36 & igraph\_igraph\_read\_graph\_gml & 6 & \cellcolor{green!0}{\large 0}/{\footnotesize 40} & \cellcolor{green!0}{\large 0}/{\footnotesize 40} & \cellcolor{green!0}{\large 0}/{\footnotesize 40} & \cellcolor{green!0}{\large 0}/{\footnotesize 40} & \cellcolor{green!0}{\large 0}/{\footnotesize 64} & \cellcolor{green!0}{\large 0}/{\footnotesize 55} \tabularnewline
\rowcolor{black!10} 37 & igraph\_igraph\_read\_graph\_graphdb & 6 & \cellcolor{green!0}{\large 0}/{\footnotesize 40} & \cellcolor{green!0}{\large 0}/{\footnotesize 40} & \cellcolor{green!0}{\large 0}/{\footnotesize 40} & \cellcolor{green!0}{\large 0}/{\footnotesize 40} & \cellcolor{green!0}{\large 0}/{\footnotesize 54} & \cellcolor{green!0}{\large 0}/{\footnotesize 52} \tabularnewline
38 & igraph\_igraph\_read\_graph\_graphml & 6 & \cellcolor{green!0}{\large 0}/{\footnotesize 40} & \cellcolor{green!0}{\large 0}/{\footnotesize 40} & \cellcolor{green!0}{\large 0}/{\footnotesize 40} & \cellcolor{green!0}{\large 0}/{\footnotesize 40} & \cellcolor{green!10}{\large 1}/{\footnotesize 59} & \cellcolor{green!10}{\large 2}/{\footnotesize 53} \tabularnewline
\rowcolor{black!10} 39 & igraph\_igraph\_read\_graph\_lgl & 6 & \cellcolor{green!0}{\large 0}/{\footnotesize 40} & \cellcolor{green!0}{\large 0}/{\footnotesize 40} & \cellcolor{green!0}{\large 0}/{\footnotesize 40} & \cellcolor{green!0}{\large 0}/{\footnotesize 40} & \cellcolor{green!0}{\large 0}/{\footnotesize 48} & \cellcolor{green!0}{\large 0}/{\footnotesize 46} \tabularnewline
40 & igraph\_igraph\_read\_graph\_pajek & 6 & \cellcolor{green!0}{\large 0}/{\footnotesize 40} & \cellcolor{green!0}{\large 0}/{\footnotesize 40} & \cellcolor{green!10}{\large 1}/{\footnotesize 40} & \cellcolor{green!0}{\large 0}/{\footnotesize 40} & \cellcolor{green!0}{\large 0}/{\footnotesize 56} & \cellcolor{green!0}{\large 0}/{\footnotesize 63} \tabularnewline
\rowcolor{black!10} 41 & inchi\_GetINCHIfromINCHI & 6 & \cellcolor{green!0}{\large 0}/{\footnotesize 40} & \cellcolor{green!0}{\large 0}/{\footnotesize 40} & \cellcolor{green!10}{\large 1}/{\footnotesize 40} & \cellcolor{green!0}{\large 0}/{\footnotesize 40} & \cellcolor{green!0}{\large 0}/{\footnotesize 53} & \cellcolor{green!0}{\large 0}/{\footnotesize 59} \tabularnewline
42 & inchi\_GetStructFromINCHI & 6 & \cellcolor{green!0}{\large 0}/{\footnotesize 40} & \cellcolor{green!10}{\large 1}/{\footnotesize 40} & \cellcolor{green!0}{\large 0}/{\footnotesize 40} & \cellcolor{green!0}{\large 0}/{\footnotesize 40} & \cellcolor{green!0}{\large 0}/{\footnotesize 71} & \cellcolor{green!0}{\large 0}/{\footnotesize 55} \tabularnewline
\rowcolor{black!10} 43 & kamailio\_parse\_msg & 6 & \cellcolor{green!0}{\large 0}/{\footnotesize 40} & \cellcolor{green!10}{\large 3}/{\footnotesize 40} & \cellcolor{green!0}{\large -}{\tiny -} & \cellcolor{green!10}{\large 4}/{\footnotesize 40} & \cellcolor{green!10}{\large 3}/{\footnotesize 62} & \cellcolor{green!10}{\large 2}/{\footnotesize 56} \tabularnewline
44 & libyang\_lys\_parse\_mem & 6 & \cellcolor{green!0}{\large 0}/{\footnotesize 40} & \cellcolor{green!0}{\large 0}/{\footnotesize 40} & \cellcolor{green!0}{\large 0}/{\footnotesize 40} & \cellcolor{green!0}{\large 0}/{\footnotesize 40} & \cellcolor{green!0}{\large 0}/{\footnotesize 54} & \cellcolor{green!0}{\large 0}/{\footnotesize 46} \tabularnewline
\rowcolor{black!10} 45 & proftpd\_pr\_json\_object\_from\_text & 6 & \cellcolor{green!0}{\large 0}/{\footnotesize 40} & \cellcolor{green!0}{\large 0}/{\footnotesize 40} & \cellcolor{green!0}{\large -}{\tiny -} & \cellcolor{green!0}{\large 0}/{\footnotesize 40} & \cellcolor{green!0}{\large 0}/{\footnotesize 48} & \cellcolor{green!10}{\large 1}/{\footnotesize 49} \tabularnewline
46 & selinux\_policydb\_read & 6 & \cellcolor{green!0}{\large 0}/{\footnotesize 40} & \cellcolor{green!10}{\large 1}/{\footnotesize 40} & \cellcolor{green!0}{\large -}{\tiny -} & \cellcolor{green!0}{\large 0}/{\footnotesize 40} & \cellcolor{green!0}{\large 0}/{\footnotesize 59} & \cellcolor{green!10}{\large 1}/{\footnotesize 53} \tabularnewline
\rowcolor{black!10} 47 & kamailio\_get\_src\_address\_socket & 7 & \cellcolor{green!0}{\large 0}/{\footnotesize 40} & \cellcolor{green!0}{\large 0}/{\footnotesize 40} & \cellcolor{green!0}{\large 0}/{\footnotesize 40} & \cellcolor{green!0}{\large 0}/{\footnotesize 40} & \cellcolor{green!0}{\large 0}/{\footnotesize 52} & \cellcolor{green!0}{\large 0}/{\footnotesize 55} \tabularnewline
48 & kamailio\_get\_src\_uri & 7 & \cellcolor{green!0}{\large 0}/{\footnotesize 40} & \cellcolor{green!0}{\large 0}/{\footnotesize 40} & \cellcolor{green!0}{\large 0}/{\footnotesize 40} & \cellcolor{green!0}{\large 0}/{\footnotesize 40} & \cellcolor{green!0}{\large 0}/{\footnotesize 46} & \cellcolor{green!0}{\large 0}/{\footnotesize 50} \tabularnewline
\rowcolor{black!10} 49 & kamailio\_parse\_content\_disposition & 7 & \cellcolor{green!0}{\large 0}/{\footnotesize 40} & \cellcolor{green!0}{\large 0}/{\footnotesize 40} & \cellcolor{green!0}{\large 0}/{\footnotesize 40} & \cellcolor{green!10}{\large 2}/{\footnotesize 40} & \cellcolor{green!0}{\large 0}/{\footnotesize 61} & \cellcolor{green!10}{\large 1}/{\footnotesize 52} \tabularnewline
50 & kamailio\_parse\_diversion\_header & 7 & \cellcolor{green!0}{\large 0}/{\footnotesize 40} & \cellcolor{green!0}{\large 0}/{\footnotesize 40} & \cellcolor{green!0}{\large 0}/{\footnotesize 40} & \cellcolor{green!10}{\large 1}/{\footnotesize 40} & \cellcolor{green!0}{\large 0}/{\footnotesize 56} & \cellcolor{green!10}{\large 1}/{\footnotesize 47} \tabularnewline
\rowcolor{black!10} 51 & kamailio\_parse\_from\_header & 7 & \cellcolor{green!0}{\large 0}/{\footnotesize 40} & \cellcolor{green!0}{\large 0}/{\footnotesize 40} & \cellcolor{green!0}{\large -}{\tiny -} & \cellcolor{green!0}{\large 0}/{\footnotesize 40} & \cellcolor{green!0}{\large 0}/{\footnotesize 53} & \cellcolor{green!0}{\large 0}/{\footnotesize 52} \tabularnewline
52 & kamailio\_parse\_from\_uri & 7 & \cellcolor{green!0}{\large 0}/{\footnotesize 40} & \cellcolor{green!0}{\large 0}/{\footnotesize 40} & \cellcolor{green!0}{\large -}{\tiny -} & \cellcolor{green!0}{\large 0}/{\footnotesize 40} & \cellcolor{green!0}{\large 0}/{\footnotesize 57} & \cellcolor{green!0}{\large 0}/{\footnotesize 49} \tabularnewline
\rowcolor{black!10} 53 & kamailio\_parse\_headers & 7 & \cellcolor{green!0}{\large 0}/{\footnotesize 40} & \cellcolor{green!0}{\large 0}/{\footnotesize 40} & \cellcolor{green!0}{\large -}{\tiny -} & \cellcolor{green!0}{\large 0}/{\footnotesize 40} & \cellcolor{green!0}{\large 0}/{\footnotesize 53} & \cellcolor{green!0}{\large 0}/{\footnotesize 51} \tabularnewline
54 & kamailio\_parse\_identityinfo\_header & 7 & \cellcolor{green!0}{\large 0}/{\footnotesize 40} & \cellcolor{green!0}{\large 0}/{\footnotesize 40} & \cellcolor{green!0}{\large -}{\tiny -} & \cellcolor{green!10}{\large 1}/{\footnotesize 40} & \cellcolor{green!0}{\large 0}/{\footnotesize 58} & \cellcolor{green!10}{\large 1}/{\footnotesize 49} \tabularnewline
\rowcolor{black!10} 55 & kamailio\_parse\_pai\_header & 7 & \cellcolor{green!0}{\large 0}/{\footnotesize 40} & \cellcolor{green!0}{\large 0}/{\footnotesize 40} & \cellcolor{green!0}{\large -}{\tiny -} & \cellcolor{green!0}{\large 0}/{\footnotesize 40} & \cellcolor{green!0}{\large 0}/{\footnotesize 59} & \cellcolor{green!0}{\large 0}/{\footnotesize 50} \tabularnewline
56 & kamailio\_parse\_privacy & 7 & \cellcolor{green!0}{\large 0}/{\footnotesize 40} & \cellcolor{green!0}{\large 0}/{\footnotesize 40} & \cellcolor{green!0}{\large 0}/{\footnotesize 40} & \cellcolor{green!0}{\large 0}/{\footnotesize 40} & \cellcolor{green!0}{\large 0}/{\footnotesize 52} & \cellcolor{green!10}{\large 1}/{\footnotesize 51} \tabularnewline
\rowcolor{black!10} 57 & kamailio\_parse\_record\_route\_headers & 7 & \cellcolor{green!0}{\large 0}/{\footnotesize 40} & \cellcolor{green!0}{\large 0}/{\footnotesize 40} & \cellcolor{green!0}{\large -}{\tiny -} & \cellcolor{green!0}{\large 0}/{\footnotesize 40} & \cellcolor{green!0}{\large 0}/{\footnotesize 59} & \cellcolor{green!10}{\large 1}/{\footnotesize 49} \tabularnewline
58 & kamailio\_parse\_refer\_to\_header & 7 & \cellcolor{green!0}{\large 0}/{\footnotesize 40} & \cellcolor{green!0}{\large 0}/{\footnotesize 40} & \cellcolor{green!0}{\large -}{\tiny -} & \cellcolor{green!0}{\large 0}/{\footnotesize 40} & \cellcolor{green!0}{\large 0}/{\footnotesize 54} & \cellcolor{green!0}{\large 0}/{\footnotesize 47} \tabularnewline
\rowcolor{black!10} 59 & kamailio\_parse\_route\_headers & 7 & \cellcolor{green!0}{\large 0}/{\footnotesize 40} & \cellcolor{green!0}{\large 0}/{\footnotesize 40} & \cellcolor{green!0}{\large -}{\tiny -} & \cellcolor{green!0}{\large 0}/{\footnotesize 40} & \cellcolor{green!0}{\large 0}/{\footnotesize 51} & \cellcolor{green!0}{\large 0}/{\footnotesize 51} \tabularnewline
60 & kamailio\_parse\_to\_header & 7 & \cellcolor{green!0}{\large 0}/{\footnotesize 40} & \cellcolor{green!0}{\large 0}/{\footnotesize 40} & \cellcolor{green!0}{\large -}{\tiny -} & \cellcolor{green!0}{\large 0}/{\footnotesize 40} & \cellcolor{green!0}{\large 0}/{\footnotesize 54} & \cellcolor{green!0}{\large 0}/{\footnotesize 56} \tabularnewline
\rowcolor{black!10} 61 & kamailio\_parse\_to\_uri & 7 & \cellcolor{green!0}{\large 0}/{\footnotesize 40} & \cellcolor{green!0}{\large 0}/{\footnotesize 40} & \cellcolor{green!0}{\large -}{\tiny -} & \cellcolor{green!0}{\large 0}/{\footnotesize 40} & \cellcolor{green!0}{\large 0}/{\footnotesize 57} & \cellcolor{green!0}{\large 0}/{\footnotesize 53} \tabularnewline
62 & libyang\_lyd\_parse\_data\_mem & 7 & \cellcolor{green!0}{\large 0}/{\footnotesize 40} & \cellcolor{green!0}{\large 0}/{\footnotesize 40} & \cellcolor{green!0}{\large 0}/{\footnotesize 40} & \cellcolor{green!0}{\large 0}/{\footnotesize 40} & \cellcolor{green!0}{\large 0}/{\footnotesize 48} & \cellcolor{green!0}{\large 0}/{\footnotesize 48} \tabularnewline
\rowcolor{black!10} 63 & bind9\_dns\_message\_parse & 8 & \cellcolor{green!0}{\large 0}/{\footnotesize 40} & \cellcolor{green!0}{\large 0}/{\footnotesize 40} & \cellcolor{green!0}{\large -}{\tiny -} & \cellcolor{green!0}{\large 0}/{\footnotesize 40} & \cellcolor{green!0}{\large 0}/{\footnotesize 50} & \cellcolor{green!0}{\large 0}/{\footnotesize 46} \tabularnewline
64 & igraph\_igraph\_read\_graph\_ncol & 8 & \cellcolor{green!0}{\large 0}/{\footnotesize 40} & \cellcolor{green!0}{\large 0}/{\footnotesize 40} & \cellcolor{green!0}{\large 0}/{\footnotesize 40} & \cellcolor{green!10}{\large 1}/{\footnotesize 40} & \cellcolor{green!0}{\large 0}/{\footnotesize 48} & \cellcolor{green!0}{\large 0}/{\footnotesize 51} \tabularnewline
\rowcolor{black!10} 65 & pjsip\_pj\_json\_parse & 8 & \cellcolor{green!0}{\large 0}/{\footnotesize 40} & \cellcolor{green!0}{\large 0}/{\footnotesize 40} & \cellcolor{green!0}{\large 0}/{\footnotesize 40} & \cellcolor{green!0}{\large 0}/{\footnotesize 40} & \cellcolor{green!0}{\large 0}/{\footnotesize 48} & \cellcolor{green!0}{\large 0}/{\footnotesize 54} \tabularnewline
66 & pjsip\_pj\_xml\_parse & 8 & \cellcolor{green!0}{\large 0}/{\footnotesize 40} & \cellcolor{green!0}{\large 0}/{\footnotesize 40} & \cellcolor{green!0}{\large 0}/{\footnotesize 40} & \cellcolor{green!0}{\large 0}/{\footnotesize 40} & \cellcolor{green!0}{\large 0}/{\footnotesize 48} & \cellcolor{green!0}{\large 0}/{\footnotesize 52} \tabularnewline
\rowcolor{black!10} 67 & pjsip\_pjmedia\_sdp\_parse & 8 & \cellcolor{green!0}{\large 0}/{\footnotesize 40} & \cellcolor{green!0}{\large 0}/{\footnotesize 40} & \cellcolor{green!0}{\large 0}/{\footnotesize 40} & \cellcolor{green!0}{\large 0}/{\footnotesize 40} & \cellcolor{green!0}{\large 0}/{\footnotesize 51} & \cellcolor{green!0}{\large 0}/{\footnotesize 45} \tabularnewline
68 & quickjs\_lre\_compile & 8 & \cellcolor{green!0}{\large 0}/{\footnotesize 40} & \cellcolor{green!0}{\large 0}/{\footnotesize 40} & \cellcolor{green!0}{\large -}{\tiny -} & \cellcolor{green!0}{\large 0}/{\footnotesize 40} & \cellcolor{green!0}{\large 0}/{\footnotesize 60} & \cellcolor{green!0}{\large 0}/{\footnotesize 55} \tabularnewline
\rowcolor{black!10} 69 & bind9\_isc\_lex\_getmastertoken & 9 & \cellcolor{green!0}{\large 0}/{\footnotesize 40} & \cellcolor{green!0}{\large 0}/{\footnotesize 40} & \cellcolor{green!0}{\large -}{\tiny -} & \cellcolor{green!0}{\large 0}/{\footnotesize 40} & \cellcolor{green!0}{\large 0}/{\footnotesize 48} & \cellcolor{green!0}{\large 0}/{\footnotesize 50} \tabularnewline
70 & bind9\_isc\_lex\_gettoken & 9 & \cellcolor{green!0}{\large 0}/{\footnotesize 40} & \cellcolor{green!0}{\large 0}/{\footnotesize 40} & \cellcolor{green!0}{\large -}{\tiny -} & \cellcolor{green!0}{\large 0}/{\footnotesize 40} & \cellcolor{green!0}{\large 0}/{\footnotesize 50} & \cellcolor{green!0}{\large 0}/{\footnotesize 45} \tabularnewline
\rowcolor{black!10} 71 & quickjs\_JS\_Eval & 9 & \cellcolor{green!0}{\large 0}/{\footnotesize 40} & \cellcolor{green!0}{\large 0}/{\footnotesize 40} & \cellcolor{green!0}{\large -}{\tiny -} & \cellcolor{green!0}{\large 0}/{\footnotesize 40} & \cellcolor{green!0}{\large 0}/{\footnotesize 45} & \cellcolor{green!0}{\large 0}/{\footnotesize 46} \tabularnewline
72 & igraph\_igraph\_edge\_connectivity & 10 & \cellcolor{green!0}{\large 0}/{\footnotesize 40} & \cellcolor{green!0}{\large 0}/{\footnotesize 40} & \cellcolor{green!0}{\large 0}/{\footnotesize 40} & \cellcolor{green!0}{\large 0}/{\footnotesize 40} & \cellcolor{green!0}{\large 0}/{\footnotesize 45} & \cellcolor{green!0}{\large 0}/{\footnotesize 46} \tabularnewline
\rowcolor{black!10} 73 & pjsip\_pj\_stun\_msg\_decode & 10 & \cellcolor{green!0}{\large 0}/{\footnotesize 40} & \cellcolor{green!0}{\large 0}/{\footnotesize 40} & \cellcolor{green!0}{\large 0}/{\footnotesize 40} & \cellcolor{green!0}{\large 0}/{\footnotesize 40} & \cellcolor{green!0}{\large 0}/{\footnotesize 44} & \cellcolor{green!0}{\large 0}/{\footnotesize 45} \tabularnewline
74 & bind9\_dns\_message\_checksig & 11 & \cellcolor{green!0}{\large 0}/{\footnotesize 40} & \cellcolor{green!0}{\large 0}/{\footnotesize 40} & \cellcolor{green!0}{\large -}{\tiny -} & \cellcolor{green!0}{\large 0}/{\footnotesize 40} & \cellcolor{green!0}{\large 0}/{\footnotesize 54} & \cellcolor{green!0}{\large 0}/{\footnotesize 48} \tabularnewline
\rowcolor{black!10} 75 & libzip\_zip\_fread & 11 & \cellcolor{green!0}{\large 0}/{\footnotesize 40} & \cellcolor{green!0}{\large 0}/{\footnotesize 40} & \cellcolor{green!0}{\large 0}/{\footnotesize 40} & \cellcolor{green!0}{\large 0}/{\footnotesize 40} & \cellcolor{green!0}{\large 0}/{\footnotesize 49} & \cellcolor{green!0}{\large 0}/{\footnotesize 43} \tabularnewline
76 & bind9\_dns\_rdata\_fromtext & 12 & \cellcolor{green!0}{\large 0}/{\footnotesize 40} & \cellcolor{green!0}{\large 0}/{\footnotesize 40} & \cellcolor{green!0}{\large -}{\tiny -} & \cellcolor{green!0}{\large 0}/{\footnotesize 40} & \cellcolor{green!0}{\large 0}/{\footnotesize 48} & \cellcolor{green!0}{\large 0}/{\footnotesize 44} \tabularnewline
\rowcolor{black!10} 77 & igraph\_igraph\_all\_minimal\_st\_separators & 12 & \cellcolor{green!0}{\large 0}/{\footnotesize 40} & \cellcolor{green!0}{\large 0}/{\footnotesize 40} & \cellcolor{green!0}{\large 0}/{\footnotesize 40} & \cellcolor{green!10}{\large 1}/{\footnotesize 40} & \cellcolor{green!0}{\large 0}/{\footnotesize 43} & \cellcolor{green!0}{\large 0}/{\footnotesize 45} \tabularnewline
78 & igraph\_igraph\_minimum\_size\_separators & 12 & \cellcolor{green!0}{\large 0}/{\footnotesize 40} & \cellcolor{green!0}{\large 0}/{\footnotesize 40} & \cellcolor{green!0}{\large 0}/{\footnotesize 40} & \cellcolor{green!0}{\large 0}/{\footnotesize 40} & \cellcolor{green!0}{\large 0}/{\footnotesize 47} & \cellcolor{green!0}{\large 0}/{\footnotesize 43} \tabularnewline
\rowcolor{black!10} 79 & pjsip\_pjsip\_parse\_msg & 12 & \cellcolor{green!0}{\large 0}/{\footnotesize 40} & \cellcolor{green!0}{\large 0}/{\footnotesize 40} & \cellcolor{green!0}{\large 0}/{\footnotesize 40} & \cellcolor{green!0}{\large 0}/{\footnotesize 40} & \cellcolor{green!0}{\large 0}/{\footnotesize 44} & \cellcolor{green!0}{\large 0}/{\footnotesize 45} \tabularnewline
80 & igraph\_igraph\_automorphism\_group & 13 & \cellcolor{green!0}{\large 0}/{\footnotesize 40} & \cellcolor{green!0}{\large 0}/{\footnotesize 40} & \cellcolor{green!0}{\large 0}/{\footnotesize 40} & \cellcolor{green!0}{\large 0}/{\footnotesize 40} & \cellcolor{green!0}{\large 0}/{\footnotesize 51} & \cellcolor{green!0}{\large 0}/{\footnotesize 45} \tabularnewline
\rowcolor{black!10} 81 & libmodbus\_modbus\_read\_bits & 15 & \cellcolor{green!0}{\large 0}/{\footnotesize 40} & \cellcolor{green!0}{\large 0}/{\footnotesize 40} & \cellcolor{green!0}{\large 0}/{\footnotesize 40} & \cellcolor{green!0}{\large 0}/{\footnotesize 40} & \cellcolor{green!0}{\large 0}/{\footnotesize 46} & \cellcolor{green!0}{\large 0}/{\footnotesize 43} \tabularnewline
82 & libmodbus\_modbus\_read\_registers & 15 & \cellcolor{green!0}{\large 0}/{\footnotesize 40} & \cellcolor{green!0}{\large 0}/{\footnotesize 40} & \cellcolor{green!0}{\large 0}/{\footnotesize 40} & \cellcolor{green!0}{\large 0}/{\footnotesize 40} & \cellcolor{green!0}{\large 0}/{\footnotesize 47} & \cellcolor{green!0}{\large 0}/{\footnotesize 47} \tabularnewline
\rowcolor{black!10} 83 & civetweb\_mg\_get\_response & 17 & \cellcolor{green!0}{\large 0}/{\footnotesize 40} & \cellcolor{green!0}{\large 0}/{\footnotesize 40} & \cellcolor{green!0}{\large 0}/{\footnotesize 40} & \cellcolor{green!0}{\large 0}/{\footnotesize 40} & \cellcolor{green!0}{\large 0}/{\footnotesize 52} & \cellcolor{green!0}{\large 0}/{\footnotesize 45} \tabularnewline
84 & bind9\_dns\_master\_loadbuffer & 20 & \cellcolor{green!0}{\large 0}/{\footnotesize 40} & \cellcolor{green!0}{\large 0}/{\footnotesize 40} & \cellcolor{green!0}{\large -}{\tiny -} & \cellcolor{green!0}{\large 0}/{\footnotesize 40} & \cellcolor{green!0}{\large 0}/{\footnotesize 47} & \cellcolor{green!0}{\large 0}/{\footnotesize 46} \tabularnewline
\rowcolor{black!10} 85 & libmodbus\_modbus\_receive & 33 & \cellcolor{green!0}{\large 0}/{\footnotesize 40} & \cellcolor{green!0}{\large 0}/{\footnotesize 40} & \cellcolor{green!0}{\large 0}/{\footnotesize 40} & \cellcolor{green!0}{\large 0}/{\footnotesize 40} & \cellcolor{green!0}{\large 0}/{\footnotesize 50} & \cellcolor{green!0}{\large 0}/{\footnotesize 53} \tabularnewline
86 & tmux\_input\_parse\_buffer & 42 & \cellcolor{green!0}{\large 0}/{\footnotesize 40} & \cellcolor{green!0}{\large 0}/{\footnotesize 40} & \cellcolor{green!0}{\large -}{\tiny -} & \cellcolor{green!0}{\large 0}/{\footnotesize 40} & \cellcolor{green!0}{\large 0}/{\footnotesize 62} & \cellcolor{green!0}{\large 0}/{\footnotesize 55} \tabularnewline

\bottomrule
%\end{tabular}
%}
%\end{table*}
\end{xltabular}
}
\twocolumn



% model: gpt-3.5-turbo-0613, temp: 0.0

\onecolumn
{\small %
\begin{xltabular}[h]{\textwidth}{ccccccccc}
%\begin{table*}[!t]
%\centering
\caption{Evaluation Result of model gpt-3.5-turbo-0613 with temperature 0.0.} \\
%\resizebox{1.0\linewidth}{!}{
%\begin{tabular}{cccccccccc}
\toprule
Index & Question & Score & NAIVE-40 & BACTX-40 & DOCTX-40 & UGCTX-40 & BA-ITER-40 & ALL-ITER-40 \tabularnewline
\midrule
\rowcolor{black!10} 1 & coturn\_stun\_is\_command\_message\_full\_check\_str & 1 & \cellcolor{green!0}{\large 0}/{\footnotesize 40} & \cellcolor{green!100}{\large 40}/{\footnotesize 40} & \cellcolor{green!0}{\large -}{\tiny -} & \cellcolor{green!0}{\large 0}/{\footnotesize 40} & \cellcolor{green!100}{\large 40}/{\footnotesize 40} & \cellcolor{green!50}{\large 20}/{\footnotesize 40} \tabularnewline
2 & kamailio\_parse\_uri & 1 & \cellcolor{green!0}{\large 0}/{\footnotesize 40} & \cellcolor{green!100}{\large 40}/{\footnotesize 40} & \cellcolor{green!0}{\large -}{\tiny -} & \cellcolor{green!70}{\large 26}/{\footnotesize 40} & \cellcolor{green!100}{\large 40}/{\footnotesize 40} & \cellcolor{green!40}{\large 30}/{\footnotesize 87} \tabularnewline
\rowcolor{black!10} 3 & coturn\_stun\_check\_message\_integrity\_str & 2 & \cellcolor{green!0}{\large 0}/{\footnotesize 40} & \cellcolor{green!0}{\large 0}/{\footnotesize 40} & \cellcolor{green!0}{\large -}{\tiny -} & \cellcolor{green!0}{\large 0}/{\footnotesize 40} & \cellcolor{green!10}{\large 2}/{\footnotesize 198} & \cellcolor{green!10}{\large 3}/{\footnotesize 116} \tabularnewline
4 & libiec61850\_MmsValue\_decodeMmsData & 2 & \cellcolor{green!0}{\large 0}/{\footnotesize 40} & \cellcolor{green!100}{\large 40}/{\footnotesize 40} & \cellcolor{green!0}{\large 0}/{\footnotesize 40} & \cellcolor{green!40}{\large 13}/{\footnotesize 40} & \cellcolor{green!100}{\large 40}/{\footnotesize 40} & \cellcolor{green!20}{\large 21}/{\footnotesize 134} \tabularnewline
\rowcolor{black!10} 5 & md4c\_md\_html & 2 & \cellcolor{green!0}{\large 0}/{\footnotesize 40} & \cellcolor{green!0}{\large 0}/{\footnotesize 40} & \cellcolor{green!0}{\large 0}/{\footnotesize 40} & \cellcolor{green!0}{\large 0}/{\footnotesize 40} & \cellcolor{green!50}{\large 40}/{\footnotesize 80} & \cellcolor{green!10}{\large 4}/{\footnotesize 172} \tabularnewline
6 & spdk\_spdk\_json\_parse & 2 & \cellcolor{green!0}{\large 0}/{\footnotesize 40} & \cellcolor{green!100}{\large 40}/{\footnotesize 40} & \cellcolor{green!0}{\large -}{\tiny -} & \cellcolor{green!20}{\large 8}/{\footnotesize 40} & \cellcolor{green!100}{\large 40}/{\footnotesize 40} & \cellcolor{green!20}{\large 16}/{\footnotesize 140} \tabularnewline
\rowcolor{black!10} 7 & croaring\_roaring\_bitmap\_portable\_deserialize\_safe & 3 & \cellcolor{green!0}{\large 0}/{\footnotesize 40} & \cellcolor{green!100}{\large 40}/{\footnotesize 40} & \cellcolor{green!100}{\large 40}/{\footnotesize 40} & \cellcolor{green!60}{\large 23}/{\footnotesize 40} & \cellcolor{green!100}{\large 40}/{\footnotesize 40} & \cellcolor{green!40}{\large 29}/{\footnotesize 81} \tabularnewline
8 & lua\_luaL\_loadbufferx & 3 & \cellcolor{green!0}{\large 0}/{\footnotesize 40} & \cellcolor{green!100}{\large 40}/{\footnotesize 40} & \cellcolor{green!100}{\large 40}/{\footnotesize 40} & \cellcolor{green!50}{\large 19}/{\footnotesize 40} & \cellcolor{green!100}{\large 40}/{\footnotesize 40} & \cellcolor{green!30}{\large 28}/{\footnotesize 107} \tabularnewline
\rowcolor{black!10} 9 & w3m\_wc\_Str\_conv\_with\_detect & 3 & \cellcolor{green!0}{\large 0}/{\footnotesize 40} & \cellcolor{green!0}{\large 0}/{\footnotesize 40} & \cellcolor{green!0}{\large -}{\tiny -} & \cellcolor{green!50}{\large 17}/{\footnotesize 40} & \cellcolor{green!0}{\large 0}/{\footnotesize 200} & \cellcolor{green!30}{\large 26}/{\footnotesize 120} \tabularnewline
10 & bind9\_dns\_name\_fromwire & 4 & \cellcolor{green!0}{\large 0}/{\footnotesize 40} & \cellcolor{green!0}{\large 0}/{\footnotesize 40} & \cellcolor{green!0}{\large -}{\tiny -} & \cellcolor{green!10}{\large 1}/{\footnotesize 40} & \cellcolor{green!0}{\large 0}/{\footnotesize 200} & \cellcolor{green!0}{\large 0}/{\footnotesize 193} \tabularnewline
\rowcolor{black!10} 11 & gdk-pixbuf\_gdk\_pixbuf\_animation\_new\_from\_file & 4 & \cellcolor{green!0}{\large 0}/{\footnotesize 40} & \cellcolor{green!0}{\large 0}/{\footnotesize 40} & \cellcolor{green!0}{\large 0}/{\footnotesize 40} & \cellcolor{green!30}{\large 10}/{\footnotesize 40} & \cellcolor{green!0}{\large 0}/{\footnotesize 112} & \cellcolor{green!10}{\large 4}/{\footnotesize 125} \tabularnewline
12 & gdk-pixbuf\_gdk\_pixbuf\_new\_from\_data & 4 & \cellcolor{green!0}{\large 0}/{\footnotesize 40} & \cellcolor{green!100}{\large 40}/{\footnotesize 40} & \cellcolor{green!100}{\large 40}/{\footnotesize 40} & \cellcolor{green!30}{\large 9}/{\footnotesize 40} & \cellcolor{green!100}{\large 40}/{\footnotesize 40} & \cellcolor{green!30}{\large 23}/{\footnotesize 105} \tabularnewline
\rowcolor{black!10} 13 & gdk-pixbuf\_gdk\_pixbuf\_new\_from\_file & 4 & \cellcolor{green!0}{\large 0}/{\footnotesize 40} & \cellcolor{green!0}{\large 0}/{\footnotesize 40} & \cellcolor{green!0}{\large 0}/{\footnotesize 40} & \cellcolor{green!10}{\large 1}/{\footnotesize 40} & \cellcolor{green!0}{\large 0}/{\footnotesize 93} & \cellcolor{green!10}{\large 4}/{\footnotesize 124} \tabularnewline
14 & gdk-pixbuf\_gdk\_pixbuf\_new\_from\_stream & 4 & \cellcolor{green!0}{\large 0}/{\footnotesize 40} & \cellcolor{green!100}{\large 40}/{\footnotesize 40} & \cellcolor{green!60}{\large 23}/{\footnotesize 40} & \cellcolor{green!70}{\large 26}/{\footnotesize 40} & \cellcolor{green!100}{\large 40}/{\footnotesize 40} & \cellcolor{green!40}{\large 28}/{\footnotesize 83} \tabularnewline
\rowcolor{black!10} 15 & gpac\_gf\_isom\_open\_file & 4 & \cellcolor{green!0}{\large 0}/{\footnotesize 40} & \cellcolor{green!20}{\large 6}/{\footnotesize 40} & \cellcolor{green!0}{\large -}{\tiny -} & \cellcolor{green!0}{\large 0}/{\footnotesize 40} & \cellcolor{green!10}{\large 3}/{\footnotesize 129} & \cellcolor{green!0}{\large 1}/{\footnotesize 153} \tabularnewline
16 & libbpf\_bpf\_object\_\_open\_mem & 4 & \cellcolor{green!0}{\large 0}/{\footnotesize 40} & \cellcolor{green!0}{\large 0}/{\footnotesize 40} & \cellcolor{green!0}{\large 0}/{\footnotesize 40} & \cellcolor{green!30}{\large 9}/{\footnotesize 40} & \cellcolor{green!10}{\large 6}/{\footnotesize 175} & \cellcolor{green!10}{\large 12}/{\footnotesize 139} \tabularnewline
\rowcolor{black!10} 17 & libpg\_query\_pg\_query\_parse & 4 & \cellcolor{green!0}{\large 0}/{\footnotesize 40} & \cellcolor{green!0}{\large 0}/{\footnotesize 40} & \cellcolor{green!0}{\large -}{\tiny -} & \cellcolor{green!40}{\large 13}/{\footnotesize 40} & \cellcolor{green!30}{\large 31}/{\footnotesize 144} & \cellcolor{green!40}{\large 34}/{\footnotesize 104} \tabularnewline
18 & libucl\_ucl\_parser\_add\_string & 4 & \cellcolor{green!0}{\large 0}/{\footnotesize 40} & \cellcolor{green!0}{\large 0}/{\footnotesize 40} & \cellcolor{green!20}{\large 7}/{\footnotesize 40} & \cellcolor{green!0}{\large 0}/{\footnotesize 40} & \cellcolor{green!40}{\large 39}/{\footnotesize 111} & \cellcolor{green!20}{\large 22}/{\footnotesize 147} \tabularnewline
\rowcolor{black!10} 19 & oniguruma\_onig\_new & 4 & \cellcolor{green!0}{\large 0}/{\footnotesize 40} & \cellcolor{green!0}{\large 0}/{\footnotesize 40} & \cellcolor{green!0}{\large 0}/{\footnotesize 40} & \cellcolor{green!10}{\large 2}/{\footnotesize 40} & \cellcolor{green!50}{\large 40}/{\footnotesize 80} & \cellcolor{green!20}{\large 21}/{\footnotesize 154} \tabularnewline
20 & pupnp\_ixmlLoadDocumentEx & 4 & \cellcolor{green!0}{\large 0}/{\footnotesize 40} & \cellcolor{green!0}{\large 0}/{\footnotesize 40} & \cellcolor{green!0}{\large 0}/{\footnotesize 40} & \cellcolor{green!0}{\large 0}/{\footnotesize 40} & \cellcolor{green!0}{\large 0}/{\footnotesize 80} & \cellcolor{green!0}{\large 0}/{\footnotesize 88} \tabularnewline
\rowcolor{black!10} 21 & gdk-pixbuf\_gdk\_pixbuf\_new\_from\_file\_at\_scale & 5 & \cellcolor{green!0}{\large 0}/{\footnotesize 40} & \cellcolor{green!0}{\large 0}/{\footnotesize 40} & \cellcolor{green!0}{\large 0}/{\footnotesize 40} & \cellcolor{green!0}{\large 0}/{\footnotesize 40} & \cellcolor{green!0}{\large 0}/{\footnotesize 91} & \cellcolor{green!0}{\large 0}/{\footnotesize 146} \tabularnewline
22 & inchi\_GetINCHIKeyFromINCHI & 5 & \cellcolor{green!0}{\large 0}/{\footnotesize 40} & \cellcolor{green!100}{\large 40}/{\footnotesize 40} & \cellcolor{green!0}{\large 0}/{\footnotesize 40} & \cellcolor{green!30}{\large 11}/{\footnotesize 40} & \cellcolor{green!100}{\large 40}/{\footnotesize 40} & \cellcolor{green!20}{\large 18}/{\footnotesize 138} \tabularnewline
\rowcolor{black!10} 23 & libdwarf\_dwarf\_init\_b & 5 & \cellcolor{green!0}{\large 0}/{\footnotesize 40} & \cellcolor{green!0}{\large 0}/{\footnotesize 40} & \cellcolor{green!0}{\large 0}/{\footnotesize 40} & \cellcolor{green!0}{\large 0}/{\footnotesize 40} & \cellcolor{green!10}{\large 9}/{\footnotesize 200} & \cellcolor{green!10}{\large 5}/{\footnotesize 82} \tabularnewline
24 & libdwarf\_dwarf\_init\_path & 5 & \cellcolor{green!0}{\large 0}/{\footnotesize 40} & \cellcolor{green!0}{\large 0}/{\footnotesize 40} & \cellcolor{green!0}{\large 0}/{\footnotesize 40} & \cellcolor{green!0}{\large 0}/{\footnotesize 40} & \cellcolor{green!0}{\large 0}/{\footnotesize 192} & \cellcolor{green!0}{\large 0}/{\footnotesize 143} \tabularnewline
\rowcolor{black!10} 25 & liblouis\_lou\_compileString & 5 & \cellcolor{green!0}{\large 0}/{\footnotesize 40} & \cellcolor{green!0}{\large 0}/{\footnotesize 40} & \cellcolor{green!0}{\large 0}/{\footnotesize 40} & \cellcolor{green!0}{\large 0}/{\footnotesize 40} & \cellcolor{green!0}{\large 0}/{\footnotesize 200} & \cellcolor{green!10}{\large 7}/{\footnotesize 190} \tabularnewline
26 & selinux\_cil\_compile & 5 & \cellcolor{green!0}{\large 0}/{\footnotesize 40} & \cellcolor{green!0}{\large 0}/{\footnotesize 40} & \cellcolor{green!0}{\large -}{\tiny -} & \cellcolor{green!20}{\large 8}/{\footnotesize 40} & \cellcolor{green!0}{\large 0}/{\footnotesize 200} & \cellcolor{green!10}{\large 12}/{\footnotesize 142} \tabularnewline
\rowcolor{black!10} 27 & bind9\_dns\_name\_fromtext & 6 & \cellcolor{green!0}{\large 0}/{\footnotesize 40} & \cellcolor{green!0}{\large 0}/{\footnotesize 40} & \cellcolor{green!0}{\large -}{\tiny -} & \cellcolor{green!0}{\large 0}/{\footnotesize 40} & \cellcolor{green!0}{\large 0}/{\footnotesize 200} & \cellcolor{green!10}{\large 3}/{\footnotesize 168} \tabularnewline
28 & bind9\_dns\_rdata\_fromwire & 6 & \cellcolor{green!0}{\large 0}/{\footnotesize 40} & \cellcolor{green!0}{\large 0}/{\footnotesize 40} & \cellcolor{green!0}{\large -}{\tiny -} & \cellcolor{green!0}{\large 0}/{\footnotesize 40} & \cellcolor{green!0}{\large 0}/{\footnotesize 200} & \cellcolor{green!10}{\large 2}/{\footnotesize 182} \tabularnewline
\rowcolor{black!10} 29 & coturn\_stun\_is\_binding\_response & 6 & \cellcolor{green!0}{\large 0}/{\footnotesize 40} & \cellcolor{green!0}{\large 0}/{\footnotesize 40} & \cellcolor{green!0}{\large -}{\tiny -} & \cellcolor{green!20}{\large 8}/{\footnotesize 40} & \cellcolor{green!10}{\large 5}/{\footnotesize 195} & \cellcolor{green!20}{\large 19}/{\footnotesize 118} \tabularnewline
30 & coturn\_stun\_is\_command\_message & 6 & \cellcolor{green!0}{\large 0}/{\footnotesize 40} & \cellcolor{green!0}{\large 0}/{\footnotesize 40} & \cellcolor{green!0}{\large 0}/{\footnotesize 40} & \cellcolor{green!100}{\large 40}/{\footnotesize 40} & \cellcolor{green!0}{\large 0}/{\footnotesize 200} & \cellcolor{green!50}{\large 32}/{\footnotesize 72} \tabularnewline
\rowcolor{black!10} 31 & coturn\_stun\_is\_response & 6 & \cellcolor{green!0}{\large 0}/{\footnotesize 40} & \cellcolor{green!0}{\large 0}/{\footnotesize 40} & \cellcolor{green!0}{\large -}{\tiny -} & \cellcolor{green!50}{\large 19}/{\footnotesize 40} & \cellcolor{green!0}{\large 0}/{\footnotesize 200} & \cellcolor{green!20}{\large 19}/{\footnotesize 97} \tabularnewline
32 & coturn\_stun\_is\_success\_response & 6 & \cellcolor{green!0}{\large 0}/{\footnotesize 40} & \cellcolor{green!0}{\large 0}/{\footnotesize 40} & \cellcolor{green!0}{\large -}{\tiny -} & \cellcolor{green!30}{\large 9}/{\footnotesize 40} & \cellcolor{green!10}{\large 8}/{\footnotesize 193} & \cellcolor{green!10}{\large 7}/{\footnotesize 127} \tabularnewline
\rowcolor{black!10} 33 & hiredis\_redisFormatCommand & 6 & \cellcolor{green!0}{\large 0}/{\footnotesize 40} & \cellcolor{green!100}{\large 40}/{\footnotesize 40} & \cellcolor{green!0}{\large -}{\tiny -} & \cellcolor{green!30}{\large 12}/{\footnotesize 40} & \cellcolor{green!20}{\large 32}/{\footnotesize 180} & \cellcolor{green!20}{\large 17}/{\footnotesize 151} \tabularnewline
34 & igraph\_igraph\_read\_graph\_dl & 6 & \cellcolor{green!0}{\large 0}/{\footnotesize 40} & \cellcolor{green!0}{\large 0}/{\footnotesize 40} & \cellcolor{green!0}{\large 0}/{\footnotesize 40} & \cellcolor{green!0}{\large 0}/{\footnotesize 40} & \cellcolor{green!10}{\large 4}/{\footnotesize 191} & \cellcolor{green!10}{\large 2}/{\footnotesize 196} \tabularnewline
\rowcolor{black!10} 35 & igraph\_igraph\_read\_graph\_edgelist & 6 & \cellcolor{green!0}{\large 0}/{\footnotesize 40} & \cellcolor{green!0}{\large 0}/{\footnotesize 40} & \cellcolor{green!0}{\large 0}/{\footnotesize 40} & \cellcolor{green!0}{\large 0}/{\footnotesize 40} & \cellcolor{green!10}{\large 4}/{\footnotesize 197} & \cellcolor{green!0}{\large 0}/{\footnotesize 196} \tabularnewline
36 & igraph\_igraph\_read\_graph\_gml & 6 & \cellcolor{green!0}{\large 0}/{\footnotesize 40} & \cellcolor{green!0}{\large 0}/{\footnotesize 40} & \cellcolor{green!0}{\large 0}/{\footnotesize 40} & \cellcolor{green!0}{\large 0}/{\footnotesize 40} & \cellcolor{green!20}{\large 21}/{\footnotesize 109} & \cellcolor{green!10}{\large 4}/{\footnotesize 176} \tabularnewline
\rowcolor{black!10} 37 & igraph\_igraph\_read\_graph\_graphdb & 6 & \cellcolor{green!0}{\large 0}/{\footnotesize 40} & \cellcolor{green!0}{\large 0}/{\footnotesize 40} & \cellcolor{green!0}{\large 0}/{\footnotesize 40} & \cellcolor{green!0}{\large 0}/{\footnotesize 40} & \cellcolor{green!10}{\large 8}/{\footnotesize 184} & \cellcolor{green!10}{\large 4}/{\footnotesize 193} \tabularnewline
38 & igraph\_igraph\_read\_graph\_graphml & 6 & \cellcolor{green!0}{\large 0}/{\footnotesize 40} & \cellcolor{green!0}{\large 0}/{\footnotesize 40} & \cellcolor{green!0}{\large 0}/{\footnotesize 40} & \cellcolor{green!30}{\large 12}/{\footnotesize 40} & \cellcolor{green!50}{\large 40}/{\footnotesize 94} & \cellcolor{green!10}{\large 14}/{\footnotesize 153} \tabularnewline
\rowcolor{black!10} 39 & igraph\_igraph\_read\_graph\_lgl & 6 & \cellcolor{green!0}{\large 0}/{\footnotesize 40} & \cellcolor{green!0}{\large 0}/{\footnotesize 40} & \cellcolor{green!0}{\large 0}/{\footnotesize 40} & \cellcolor{green!0}{\large 0}/{\footnotesize 40} & \cellcolor{green!0}{\large 0}/{\footnotesize 196} & \cellcolor{green!0}{\large 1}/{\footnotesize 194} \tabularnewline
40 & igraph\_igraph\_read\_graph\_pajek & 6 & \cellcolor{green!0}{\large 0}/{\footnotesize 40} & \cellcolor{green!0}{\large 0}/{\footnotesize 40} & \cellcolor{green!0}{\large 0}/{\footnotesize 40} & \cellcolor{green!0}{\large 0}/{\footnotesize 40} & \cellcolor{green!20}{\large 22}/{\footnotesize 138} & \cellcolor{green!10}{\large 17}/{\footnotesize 175} \tabularnewline
\rowcolor{black!10} 41 & inchi\_GetINCHIfromINCHI & 6 & \cellcolor{green!0}{\large 0}/{\footnotesize 40} & \cellcolor{green!0}{\large 0}/{\footnotesize 40} & \cellcolor{green!0}{\large 0}/{\footnotesize 40} & \cellcolor{green!10}{\large 2}/{\footnotesize 40} & \cellcolor{green!20}{\large 25}/{\footnotesize 192} & \cellcolor{green!10}{\large 15}/{\footnotesize 164} \tabularnewline
42 & inchi\_GetStructFromINCHI & 6 & \cellcolor{green!0}{\large 0}/{\footnotesize 40} & \cellcolor{green!0}{\large 0}/{\footnotesize 40} & \cellcolor{green!0}{\large 0}/{\footnotesize 40} & \cellcolor{green!0}{\large 0}/{\footnotesize 40} & \cellcolor{green!0}{\large 0}/{\footnotesize 52} & \cellcolor{green!10}{\large 3}/{\footnotesize 121} \tabularnewline
\rowcolor{black!10} 43 & kamailio\_parse\_msg & 6 & \cellcolor{green!0}{\large 0}/{\footnotesize 40} & \cellcolor{green!0}{\large 0}/{\footnotesize 40} & \cellcolor{green!0}{\large -}{\tiny -} & \cellcolor{green!30}{\large 12}/{\footnotesize 40} & \cellcolor{green!10}{\large 12}/{\footnotesize 164} & \cellcolor{green!30}{\large 31}/{\footnotesize 112} \tabularnewline
44 & libyang\_lys\_parse\_mem & 6 & \cellcolor{green!0}{\large 0}/{\footnotesize 40} & \cellcolor{green!0}{\large 0}/{\footnotesize 40} & \cellcolor{green!0}{\large 0}/{\footnotesize 40} & \cellcolor{green!0}{\large 0}/{\footnotesize 40} & \cellcolor{green!0}{\large 0}/{\footnotesize 200} & \cellcolor{green!10}{\large 7}/{\footnotesize 180} \tabularnewline
\rowcolor{black!10} 45 & proftpd\_pr\_json\_object\_from\_text & 6 & \cellcolor{green!0}{\large 0}/{\footnotesize 40} & \cellcolor{green!0}{\large 0}/{\footnotesize 40} & \cellcolor{green!0}{\large -}{\tiny -} & \cellcolor{green!0}{\large 0}/{\footnotesize 40} & \cellcolor{green!0}{\large 0}/{\footnotesize 200} & \cellcolor{green!0}{\large 0}/{\footnotesize 168} \tabularnewline
46 & selinux\_policydb\_read & 6 & \cellcolor{green!0}{\large 0}/{\footnotesize 40} & \cellcolor{green!0}{\large 0}/{\footnotesize 40} & \cellcolor{green!0}{\large -}{\tiny -} & \cellcolor{green!0}{\large 0}/{\footnotesize 40} & \cellcolor{green!30}{\large 38}/{\footnotesize 124} & \cellcolor{green!10}{\large 11}/{\footnotesize 148} \tabularnewline
\rowcolor{black!10} 47 & kamailio\_get\_src\_address\_socket & 7 & \cellcolor{green!0}{\large 0}/{\footnotesize 40} & \cellcolor{green!0}{\large 0}/{\footnotesize 40} & \cellcolor{green!0}{\large 0}/{\footnotesize 40} & \cellcolor{green!30}{\large 9}/{\footnotesize 40} & \cellcolor{green!0}{\large 0}/{\footnotesize 200} & \cellcolor{green!10}{\large 11}/{\footnotesize 146} \tabularnewline
48 & kamailio\_get\_src\_uri & 7 & \cellcolor{green!0}{\large 0}/{\footnotesize 40} & \cellcolor{green!0}{\large 0}/{\footnotesize 40} & \cellcolor{green!0}{\large 0}/{\footnotesize 40} & \cellcolor{green!20}{\large 7}/{\footnotesize 40} & \cellcolor{green!0}{\large 0}/{\footnotesize 200} & \cellcolor{green!10}{\large 9}/{\footnotesize 150} \tabularnewline
\rowcolor{black!10} 49 & kamailio\_parse\_content\_disposition & 7 & \cellcolor{green!0}{\large 0}/{\footnotesize 40} & \cellcolor{green!0}{\large 0}/{\footnotesize 40} & \cellcolor{green!0}{\large 0}/{\footnotesize 40} & \cellcolor{green!0}{\large 0}/{\footnotesize 40} & \cellcolor{green!0}{\large 0}/{\footnotesize 195} & \cellcolor{green!0}{\large 1}/{\footnotesize 175} \tabularnewline
50 & kamailio\_parse\_diversion\_header & 7 & \cellcolor{green!0}{\large 0}/{\footnotesize 40} & \cellcolor{green!0}{\large 0}/{\footnotesize 40} & \cellcolor{green!0}{\large 0}/{\footnotesize 40} & \cellcolor{green!0}{\large 0}/{\footnotesize 40} & \cellcolor{green!0}{\large 0}/{\footnotesize 200} & \cellcolor{green!10}{\large 5}/{\footnotesize 167} \tabularnewline
\rowcolor{black!10} 51 & kamailio\_parse\_from\_header & 7 & \cellcolor{green!0}{\large 0}/{\footnotesize 40} & \cellcolor{green!0}{\large 0}/{\footnotesize 40} & \cellcolor{green!0}{\large -}{\tiny -} & \cellcolor{green!0}{\large 0}/{\footnotesize 40} & \cellcolor{green!0}{\large 0}/{\footnotesize 200} & \cellcolor{green!0}{\large 0}/{\footnotesize 169} \tabularnewline
52 & kamailio\_parse\_from\_uri & 7 & \cellcolor{green!0}{\large 0}/{\footnotesize 40} & \cellcolor{green!0}{\large 0}/{\footnotesize 40} & \cellcolor{green!0}{\large -}{\tiny -} & \cellcolor{green!0}{\large 0}/{\footnotesize 40} & \cellcolor{green!0}{\large 0}/{\footnotesize 200} & \cellcolor{green!10}{\large 3}/{\footnotesize 186} \tabularnewline
\rowcolor{black!10} 53 & kamailio\_parse\_headers & 7 & \cellcolor{green!0}{\large 0}/{\footnotesize 40} & \cellcolor{green!0}{\large 0}/{\footnotesize 40} & \cellcolor{green!0}{\large -}{\tiny -} & \cellcolor{green!0}{\large 0}/{\footnotesize 40} & \cellcolor{green!0}{\large 0}/{\footnotesize 200} & \cellcolor{green!0}{\large 0}/{\footnotesize 154} \tabularnewline
54 & kamailio\_parse\_identityinfo\_header & 7 & \cellcolor{green!0}{\large 0}/{\footnotesize 40} & \cellcolor{green!0}{\large 0}/{\footnotesize 40} & \cellcolor{green!0}{\large -}{\tiny -} & \cellcolor{green!0}{\large 0}/{\footnotesize 40} & \cellcolor{green!0}{\large 0}/{\footnotesize 194} & \cellcolor{green!10}{\large 6}/{\footnotesize 178} \tabularnewline
\rowcolor{black!10} 55 & kamailio\_parse\_pai\_header & 7 & \cellcolor{green!0}{\large 0}/{\footnotesize 40} & \cellcolor{green!0}{\large 0}/{\footnotesize 40} & \cellcolor{green!0}{\large -}{\tiny -} & \cellcolor{green!0}{\large 0}/{\footnotesize 40} & \cellcolor{green!0}{\large 0}/{\footnotesize 140} & \cellcolor{green!10}{\large 5}/{\footnotesize 160} \tabularnewline
56 & kamailio\_parse\_privacy & 7 & \cellcolor{green!0}{\large 0}/{\footnotesize 40} & \cellcolor{green!0}{\large 0}/{\footnotesize 40} & \cellcolor{green!0}{\large 0}/{\footnotesize 40} & \cellcolor{green!0}{\large 0}/{\footnotesize 40} & \cellcolor{green!0}{\large 0}/{\footnotesize 200} & \cellcolor{green!10}{\large 6}/{\footnotesize 182} \tabularnewline
\rowcolor{black!10} 57 & kamailio\_parse\_record\_route\_headers & 7 & \cellcolor{green!0}{\large 0}/{\footnotesize 40} & \cellcolor{green!0}{\large 0}/{\footnotesize 40} & \cellcolor{green!0}{\large -}{\tiny -} & \cellcolor{green!0}{\large 0}/{\footnotesize 40} & \cellcolor{green!0}{\large 0}/{\footnotesize 200} & \cellcolor{green!20}{\large 21}/{\footnotesize 137} \tabularnewline
58 & kamailio\_parse\_refer\_to\_header & 7 & \cellcolor{green!0}{\large 0}/{\footnotesize 40} & \cellcolor{green!0}{\large 0}/{\footnotesize 40} & \cellcolor{green!0}{\large -}{\tiny -} & \cellcolor{green!0}{\large 0}/{\footnotesize 40} & \cellcolor{green!0}{\large 0}/{\footnotesize 200} & \cellcolor{green!10}{\large 3}/{\footnotesize 168} \tabularnewline
\rowcolor{black!10} 59 & kamailio\_parse\_route\_headers & 7 & \cellcolor{green!0}{\large 0}/{\footnotesize 40} & \cellcolor{green!0}{\large 0}/{\footnotesize 40} & \cellcolor{green!0}{\large -}{\tiny -} & \cellcolor{green!0}{\large 0}/{\footnotesize 40} & \cellcolor{green!0}{\large 0}/{\footnotesize 200} & \cellcolor{green!20}{\large 21}/{\footnotesize 137} \tabularnewline
60 & kamailio\_parse\_to\_header & 7 & \cellcolor{green!0}{\large 0}/{\footnotesize 40} & \cellcolor{green!0}{\large 0}/{\footnotesize 40} & \cellcolor{green!0}{\large -}{\tiny -} & \cellcolor{green!20}{\large 8}/{\footnotesize 40} & \cellcolor{green!0}{\large 0}/{\footnotesize 200} & \cellcolor{green!10}{\large 2}/{\footnotesize 158} \tabularnewline
\rowcolor{black!10} 61 & kamailio\_parse\_to\_uri & 7 & \cellcolor{green!0}{\large 0}/{\footnotesize 40} & \cellcolor{green!0}{\large 0}/{\footnotesize 40} & \cellcolor{green!0}{\large -}{\tiny -} & \cellcolor{green!0}{\large 0}/{\footnotesize 40} & \cellcolor{green!0}{\large 0}/{\footnotesize 200} & \cellcolor{green!0}{\large 1}/{\footnotesize 185} \tabularnewline
62 & libyang\_lyd\_parse\_data\_mem & 7 & \cellcolor{green!0}{\large 0}/{\footnotesize 40} & \cellcolor{green!0}{\large 0}/{\footnotesize 40} & \cellcolor{green!0}{\large 0}/{\footnotesize 40} & \cellcolor{green!0}{\large 0}/{\footnotesize 40} & \cellcolor{green!0}{\large 0}/{\footnotesize 200} & \cellcolor{green!10}{\large 3}/{\footnotesize 188} \tabularnewline
\rowcolor{black!10} 63 & bind9\_dns\_message\_parse & 8 & \cellcolor{green!0}{\large 0}/{\footnotesize 40} & \cellcolor{green!0}{\large 0}/{\footnotesize 40} & \cellcolor{green!0}{\large -}{\tiny -} & \cellcolor{green!0}{\large 0}/{\footnotesize 40} & \cellcolor{green!0}{\large 0}/{\footnotesize 200} & \cellcolor{green!0}{\large 0}/{\footnotesize 155} \tabularnewline
64 & igraph\_igraph\_read\_graph\_ncol & 8 & \cellcolor{green!0}{\large 0}/{\footnotesize 40} & \cellcolor{green!0}{\large 0}/{\footnotesize 40} & \cellcolor{green!0}{\large 0}/{\footnotesize 40} & \cellcolor{green!0}{\large 0}/{\footnotesize 40} & \cellcolor{green!0}{\large 0}/{\footnotesize 200} & \cellcolor{green!0}{\large 1}/{\footnotesize 192} \tabularnewline
\rowcolor{black!10} 65 & pjsip\_pj\_json\_parse & 8 & \cellcolor{green!0}{\large 0}/{\footnotesize 40} & \cellcolor{green!0}{\large 0}/{\footnotesize 40} & \cellcolor{green!0}{\large 0}/{\footnotesize 40} & \cellcolor{green!0}{\large 0}/{\footnotesize 40} & \cellcolor{green!0}{\large 0}/{\footnotesize 200} & \cellcolor{green!10}{\large 11}/{\footnotesize 182} \tabularnewline
66 & pjsip\_pj\_xml\_parse & 8 & \cellcolor{green!0}{\large 0}/{\footnotesize 40} & \cellcolor{green!0}{\large 0}/{\footnotesize 40} & \cellcolor{green!0}{\large 0}/{\footnotesize 40} & \cellcolor{green!0}{\large 0}/{\footnotesize 40} & \cellcolor{green!0}{\large 0}/{\footnotesize 200} & \cellcolor{green!10}{\large 9}/{\footnotesize 185} \tabularnewline
\rowcolor{black!10} 67 & pjsip\_pjmedia\_sdp\_parse & 8 & \cellcolor{green!0}{\large 0}/{\footnotesize 40} & \cellcolor{green!0}{\large 0}/{\footnotesize 40} & \cellcolor{green!0}{\large 0}/{\footnotesize 40} & \cellcolor{green!0}{\large 0}/{\footnotesize 40} & \cellcolor{green!0}{\large 0}/{\footnotesize 200} & \cellcolor{green!10}{\large 8}/{\footnotesize 190} \tabularnewline
68 & quickjs\_lre\_compile & 8 & \cellcolor{green!0}{\large 0}/{\footnotesize 40} & \cellcolor{green!0}{\large 0}/{\footnotesize 40} & \cellcolor{green!0}{\large -}{\tiny -} & \cellcolor{green!0}{\large 0}/{\footnotesize 40} & \cellcolor{green!0}{\large 0}/{\footnotesize 200} & \cellcolor{green!0}{\large 0}/{\footnotesize 200} \tabularnewline
\rowcolor{black!10} 69 & bind9\_isc\_lex\_getmastertoken & 9 & \cellcolor{green!0}{\large 0}/{\footnotesize 40} & \cellcolor{green!0}{\large 0}/{\footnotesize 40} & \cellcolor{green!0}{\large -}{\tiny -} & \cellcolor{green!0}{\large 0}/{\footnotesize 40} & \cellcolor{green!0}{\large 0}/{\footnotesize 200} & \cellcolor{green!0}{\large 0}/{\footnotesize 175} \tabularnewline
70 & bind9\_isc\_lex\_gettoken & 9 & \cellcolor{green!0}{\large 0}/{\footnotesize 40} & \cellcolor{green!0}{\large 0}/{\footnotesize 40} & \cellcolor{green!0}{\large -}{\tiny -} & \cellcolor{green!0}{\large 0}/{\footnotesize 40} & \cellcolor{green!0}{\large 0}/{\footnotesize 200} & \cellcolor{green!10}{\large 2}/{\footnotesize 186} \tabularnewline
\rowcolor{black!10} 71 & quickjs\_JS\_Eval & 9 & \cellcolor{green!0}{\large 0}/{\footnotesize 40} & \cellcolor{green!0}{\large 0}/{\footnotesize 40} & \cellcolor{green!0}{\large -}{\tiny -} & \cellcolor{green!10}{\large 2}/{\footnotesize 40} & \cellcolor{green!0}{\large 0}/{\footnotesize 200} & \cellcolor{green!20}{\large 21}/{\footnotesize 170} \tabularnewline
72 & igraph\_igraph\_edge\_connectivity & 10 & \cellcolor{green!0}{\large 0}/{\footnotesize 40} & \cellcolor{green!0}{\large 0}/{\footnotesize 40} & \cellcolor{green!0}{\large 0}/{\footnotesize 40} & \cellcolor{green!0}{\large 0}/{\footnotesize 40} & \cellcolor{green!0}{\large 0}/{\footnotesize 200} & \cellcolor{green!0}{\large 0}/{\footnotesize 123} \tabularnewline
\rowcolor{black!10} 73 & pjsip\_pj\_stun\_msg\_decode & 10 & \cellcolor{green!0}{\large 0}/{\footnotesize 40} & \cellcolor{green!0}{\large 0}/{\footnotesize 40} & \cellcolor{green!0}{\large 0}/{\footnotesize 40} & \cellcolor{green!0}{\large 0}/{\footnotesize 40} & \cellcolor{green!0}{\large 1}/{\footnotesize 198} & \cellcolor{green!10}{\large 6}/{\footnotesize 181} \tabularnewline
74 & bind9\_dns\_message\_checksig & 11 & \cellcolor{green!0}{\large 0}/{\footnotesize 40} & \cellcolor{green!0}{\large 0}/{\footnotesize 40} & \cellcolor{green!0}{\large -}{\tiny -} & \cellcolor{green!0}{\large 0}/{\footnotesize 40} & \cellcolor{green!0}{\large 0}/{\footnotesize 200} & \cellcolor{green!0}{\large 0}/{\footnotesize 145} \tabularnewline
\rowcolor{black!10} 75 & libzip\_zip\_fread & 11 & \cellcolor{green!0}{\large 0}/{\footnotesize 40} & \cellcolor{green!0}{\large 0}/{\footnotesize 40} & \cellcolor{green!0}{\large 0}/{\footnotesize 40} & \cellcolor{green!0}{\large 0}/{\footnotesize 40} & \cellcolor{green!0}{\large 0}/{\footnotesize 200} & \cellcolor{green!10}{\large 6}/{\footnotesize 172} \tabularnewline
76 & bind9\_dns\_rdata\_fromtext & 12 & \cellcolor{green!0}{\large 0}/{\footnotesize 40} & \cellcolor{green!0}{\large 0}/{\footnotesize 40} & \cellcolor{green!0}{\large -}{\tiny -} & \cellcolor{green!0}{\large 0}/{\footnotesize 40} & \cellcolor{green!0}{\large 0}/{\footnotesize 138} & \cellcolor{green!10}{\large 4}/{\footnotesize 133} \tabularnewline
\rowcolor{black!10} 77 & igraph\_igraph\_all\_minimal\_st\_separators & 12 & \cellcolor{green!0}{\large 0}/{\footnotesize 40} & \cellcolor{green!0}{\large 0}/{\footnotesize 40} & \cellcolor{green!0}{\large 0}/{\footnotesize 40} & \cellcolor{green!20}{\large 5}/{\footnotesize 40} & \cellcolor{green!20}{\large 23}/{\footnotesize 196} & \cellcolor{green!10}{\large 11}/{\footnotesize 168} \tabularnewline
78 & igraph\_igraph\_minimum\_size\_separators & 12 & \cellcolor{green!0}{\large 0}/{\footnotesize 40} & \cellcolor{green!10}{\large 1}/{\footnotesize 40} & \cellcolor{green!0}{\large 0}/{\footnotesize 40} & \cellcolor{green!20}{\large 8}/{\footnotesize 40} & \cellcolor{green!10}{\large 17}/{\footnotesize 191} & \cellcolor{green!10}{\large 7}/{\footnotesize 148} \tabularnewline
\rowcolor{black!10} 79 & pjsip\_pjsip\_parse\_msg & 12 & \cellcolor{green!0}{\large 0}/{\footnotesize 40} & \cellcolor{green!0}{\large 0}/{\footnotesize 40} & \cellcolor{green!0}{\large 0}/{\footnotesize 40} & \cellcolor{green!0}{\large 0}/{\footnotesize 40} & \cellcolor{green!0}{\large 0}/{\footnotesize 191} & \cellcolor{green!0}{\large 0}/{\footnotesize 168} \tabularnewline
80 & igraph\_igraph\_automorphism\_group & 13 & \cellcolor{green!0}{\large 0}/{\footnotesize 40} & \cellcolor{green!0}{\large 0}/{\footnotesize 40} & \cellcolor{green!0}{\large 0}/{\footnotesize 40} & \cellcolor{green!0}{\large 0}/{\footnotesize 40} & \cellcolor{green!0}{\large 0}/{\footnotesize 200} & \cellcolor{green!0}{\large 0}/{\footnotesize 200} \tabularnewline
\rowcolor{black!10} 81 & libmodbus\_modbus\_read\_bits & 15 & \cellcolor{green!0}{\large 0}/{\footnotesize 40} & \cellcolor{green!0}{\large 0}/{\footnotesize 40} & \cellcolor{green!0}{\large 0}/{\footnotesize 40} & \cellcolor{green!0}{\large 0}/{\footnotesize 40} & \cellcolor{green!0}{\large 0}/{\footnotesize 187} & \cellcolor{green!0}{\large 0}/{\footnotesize 130} \tabularnewline
82 & libmodbus\_modbus\_read\_registers & 15 & \cellcolor{green!0}{\large 0}/{\footnotesize 40} & \cellcolor{green!0}{\large 0}/{\footnotesize 40} & \cellcolor{green!0}{\large 0}/{\footnotesize 40} & \cellcolor{green!0}{\large 0}/{\footnotesize 40} & \cellcolor{green!0}{\large 0}/{\footnotesize 93} & \cellcolor{green!0}{\large 0}/{\footnotesize 110} \tabularnewline
\rowcolor{black!10} 83 & civetweb\_mg\_get\_response & 17 & \cellcolor{green!0}{\large 0}/{\footnotesize 40} & \cellcolor{green!0}{\large 0}/{\footnotesize 40} & \cellcolor{green!0}{\large 0}/{\footnotesize 40} & \cellcolor{green!0}{\large 0}/{\footnotesize 40} & \cellcolor{green!0}{\large 0}/{\footnotesize 200} & \cellcolor{green!0}{\large 0}/{\footnotesize 136} \tabularnewline
84 & bind9\_dns\_master\_loadbuffer & 20 & \cellcolor{green!0}{\large 0}/{\footnotesize 40} & \cellcolor{green!0}{\large 0}/{\footnotesize 40} & \cellcolor{green!0}{\large -}{\tiny -} & \cellcolor{green!0}{\large 0}/{\footnotesize 40} & \cellcolor{green!0}{\large 0}/{\footnotesize 200} & \cellcolor{green!0}{\large 0}/{\footnotesize 200} \tabularnewline
\rowcolor{black!10} 85 & libmodbus\_modbus\_receive & 33 & \cellcolor{green!0}{\large 0}/{\footnotesize 40} & \cellcolor{green!0}{\large 0}/{\footnotesize 40} & \cellcolor{green!0}{\large 0}/{\footnotesize 40} & \cellcolor{green!0}{\large 0}/{\footnotesize 40} & \cellcolor{green!0}{\large 0}/{\footnotesize 162} & \cellcolor{green!0}{\large 0}/{\footnotesize 147} \tabularnewline
86 & tmux\_input\_parse\_buffer & 42 & \cellcolor{green!0}{\large 0}/{\footnotesize 40} & \cellcolor{green!0}{\large 0}/{\footnotesize 40} & \cellcolor{green!0}{\large -}{\tiny -} & \cellcolor{green!0}{\large 0}/{\footnotesize 40} & \cellcolor{green!0}{\large 0}/{\footnotesize 200} & \cellcolor{green!0}{\large 0}/{\footnotesize 197} \tabularnewline

\bottomrule
%\end{tabular}
%}
%\end{table*}
\end{xltabular}
}
\twocolumn



% model: gpt-3.5-turbo-0613, temp: 0.5

\onecolumn
{\small %
\begin{xltabular}[h]{\textwidth}{ccccccccc}
%\begin{table*}[!t]
%\centering
\caption{Evaluation Result of model gpt-3.5-turbo-0613 with temperature 0.5.} \\
%\resizebox{1.0\linewidth}{!}{
%\begin{tabular}{cccccccccc}
\toprule
Index & Question & Score & NAIVE-40 & BACTX-40 & DOCTX-40 & UGCTX-40 & BA-ITER-40 & ALL-ITER-40 \tabularnewline
\midrule
\rowcolor{black!10} 1 & coturn\_stun\_is\_command\_message\_full\_check\_str & 1 & \cellcolor{green!0}{\large 0}/{\footnotesize 40} & \cellcolor{green!80}{\large 29}/{\footnotesize 40} & \cellcolor{green!0}{\large -}{\tiny -} & \cellcolor{green!10}{\large 2}/{\footnotesize 40} & \cellcolor{green!70}{\large 35}/{\footnotesize 57} & \cellcolor{green!30}{\large 21}/{\footnotesize 74} \tabularnewline
2 & kamailio\_parse\_uri & 1 & \cellcolor{green!0}{\large 0}/{\footnotesize 40} & \cellcolor{green!100}{\large 39}/{\footnotesize 40} & \cellcolor{green!0}{\large -}{\tiny -} & \cellcolor{green!50}{\large 17}/{\footnotesize 40} & \cellcolor{green!90}{\large 39}/{\footnotesize 45} & \cellcolor{green!30}{\large 23}/{\footnotesize 107} \tabularnewline
\rowcolor{black!10} 3 & coturn\_stun\_check\_message\_integrity\_str & 2 & \cellcolor{green!0}{\large 0}/{\footnotesize 40} & \cellcolor{green!10}{\large 2}/{\footnotesize 40} & \cellcolor{green!0}{\large -}{\tiny -} & \cellcolor{green!10}{\large 1}/{\footnotesize 40} & \cellcolor{green!10}{\large 8}/{\footnotesize 181} & \cellcolor{green!10}{\large 7}/{\footnotesize 140} \tabularnewline
4 & libiec61850\_MmsValue\_decodeMmsData & 2 & \cellcolor{green!0}{\large 0}/{\footnotesize 40} & \cellcolor{green!90}{\large 34}/{\footnotesize 40} & \cellcolor{green!40}{\large 15}/{\footnotesize 40} & \cellcolor{green!20}{\large 6}/{\footnotesize 40} & \cellcolor{green!100}{\large 39}/{\footnotesize 42} & \cellcolor{green!20}{\large 19}/{\footnotesize 146} \tabularnewline
\rowcolor{black!10} 5 & md4c\_md\_html & 2 & \cellcolor{green!0}{\large 0}/{\footnotesize 40} & \cellcolor{green!0}{\large 0}/{\footnotesize 40} & \cellcolor{green!0}{\large 0}/{\footnotesize 40} & \cellcolor{green!0}{\large 0}/{\footnotesize 40} & \cellcolor{green!40}{\large 37}/{\footnotesize 113} & \cellcolor{green!10}{\large 9}/{\footnotesize 174} \tabularnewline
6 & spdk\_spdk\_json\_parse & 2 & \cellcolor{green!0}{\large 0}/{\footnotesize 40} & \cellcolor{green!100}{\large 39}/{\footnotesize 40} & \cellcolor{green!0}{\large -}{\tiny -} & \cellcolor{green!20}{\large 5}/{\footnotesize 40} & \cellcolor{green!90}{\large 40}/{\footnotesize 45} & \cellcolor{green!20}{\large 16}/{\footnotesize 115} \tabularnewline
\rowcolor{black!10} 7 & croaring\_roaring\_bitmap\_portable\_deserialize\_safe & 3 & \cellcolor{green!0}{\large 0}/{\footnotesize 40} & \cellcolor{green!90}{\large 35}/{\footnotesize 40} & \cellcolor{green!100}{\large 37}/{\footnotesize 40} & \cellcolor{green!30}{\large 12}/{\footnotesize 40} & \cellcolor{green!100}{\large 40}/{\footnotesize 43} & \cellcolor{green!60}{\large 29}/{\footnotesize 54} \tabularnewline
8 & lua\_luaL\_loadbufferx & 3 & \cellcolor{green!10}{\large 2}/{\footnotesize 40} & \cellcolor{green!90}{\large 36}/{\footnotesize 40} & \cellcolor{green!90}{\large 35}/{\footnotesize 40} & \cellcolor{green!30}{\large 12}/{\footnotesize 40} & \cellcolor{green!100}{\large 39}/{\footnotesize 41} & \cellcolor{green!40}{\large 34}/{\footnotesize 84} \tabularnewline
\rowcolor{black!10} 9 & w3m\_wc\_Str\_conv\_with\_detect & 3 & \cellcolor{green!0}{\large 0}/{\footnotesize 40} & \cellcolor{green!0}{\large 0}/{\footnotesize 40} & \cellcolor{green!0}{\large -}{\tiny -} & \cellcolor{green!40}{\large 16}/{\footnotesize 40} & \cellcolor{green!0}{\large 0}/{\footnotesize 190} & \cellcolor{green!20}{\large 19}/{\footnotesize 148} \tabularnewline
10 & bind9\_dns\_name\_fromwire & 4 & \cellcolor{green!0}{\large 0}/{\footnotesize 40} & \cellcolor{green!0}{\large 0}/{\footnotesize 40} & \cellcolor{green!0}{\large -}{\tiny -} & \cellcolor{green!0}{\large 0}/{\footnotesize 40} & \cellcolor{green!0}{\large 0}/{\footnotesize 200} & \cellcolor{green!0}{\large 0}/{\footnotesize 174} \tabularnewline
\rowcolor{black!10} 11 & gdk-pixbuf\_gdk\_pixbuf\_animation\_new\_from\_file & 4 & \cellcolor{green!0}{\large 0}/{\footnotesize 40} & \cellcolor{green!10}{\large 3}/{\footnotesize 40} & \cellcolor{green!10}{\large 1}/{\footnotesize 40} & \cellcolor{green!10}{\large 2}/{\footnotesize 40} & \cellcolor{green!10}{\large 1}/{\footnotesize 94} & \cellcolor{green!10}{\large 4}/{\footnotesize 106} \tabularnewline
12 & gdk-pixbuf\_gdk\_pixbuf\_new\_from\_data & 4 & \cellcolor{green!10}{\large 1}/{\footnotesize 40} & \cellcolor{green!70}{\large 28}/{\footnotesize 40} & \cellcolor{green!80}{\large 29}/{\footnotesize 40} & \cellcolor{green!40}{\large 13}/{\footnotesize 40} & \cellcolor{green!50}{\large 28}/{\footnotesize 61} & \cellcolor{green!20}{\large 15}/{\footnotesize 120} \tabularnewline
\rowcolor{black!10} 13 & gdk-pixbuf\_gdk\_pixbuf\_new\_from\_file & 4 & \cellcolor{green!10}{\large 1}/{\footnotesize 40} & \cellcolor{green!10}{\large 1}/{\footnotesize 40} & \cellcolor{green!0}{\large 0}/{\footnotesize 40} & \cellcolor{green!10}{\large 1}/{\footnotesize 40} & \cellcolor{green!10}{\large 3}/{\footnotesize 93} & \cellcolor{green!10}{\large 3}/{\footnotesize 108} \tabularnewline
14 & gdk-pixbuf\_gdk\_pixbuf\_new\_from\_stream & 4 & \cellcolor{green!0}{\large 0}/{\footnotesize 40} & \cellcolor{green!90}{\large 34}/{\footnotesize 40} & \cellcolor{green!60}{\large 21}/{\footnotesize 40} & \cellcolor{green!70}{\large 25}/{\footnotesize 40} & \cellcolor{green!80}{\large 36}/{\footnotesize 48} & \cellcolor{green!40}{\large 30}/{\footnotesize 85} \tabularnewline
\rowcolor{black!10} 15 & gpac\_gf\_isom\_open\_file & 4 & \cellcolor{green!0}{\large 0}/{\footnotesize 40} & \cellcolor{green!20}{\large 5}/{\footnotesize 40} & \cellcolor{green!0}{\large -}{\tiny -} & \cellcolor{green!0}{\large 0}/{\footnotesize 40} & \cellcolor{green!10}{\large 6}/{\footnotesize 160} & \cellcolor{green!0}{\large 1}/{\footnotesize 161} \tabularnewline
16 & libbpf\_bpf\_object\_\_open\_mem & 4 & \cellcolor{green!0}{\large 0}/{\footnotesize 40} & \cellcolor{green!30}{\large 9}/{\footnotesize 40} & \cellcolor{green!30}{\large 11}/{\footnotesize 40} & \cellcolor{green!10}{\large 3}/{\footnotesize 40} & \cellcolor{green!20}{\large 21}/{\footnotesize 132} & \cellcolor{green!10}{\large 12}/{\footnotesize 123} \tabularnewline
\rowcolor{black!10} 17 & libpg\_query\_pg\_query\_parse & 4 & \cellcolor{green!10}{\large 1}/{\footnotesize 40} & \cellcolor{green!20}{\large 7}/{\footnotesize 40} & \cellcolor{green!0}{\large -}{\tiny -} & \cellcolor{green!30}{\large 9}/{\footnotesize 40} & \cellcolor{green!30}{\large 32}/{\footnotesize 106} & \cellcolor{green!50}{\large 35}/{\footnotesize 81} \tabularnewline
18 & libucl\_ucl\_parser\_add\_string & 4 & \cellcolor{green!0}{\large 0}/{\footnotesize 40} & \cellcolor{green!10}{\large 3}/{\footnotesize 40} & \cellcolor{green!40}{\large 13}/{\footnotesize 40} & \cellcolor{green!10}{\large 2}/{\footnotesize 40} & \cellcolor{green!20}{\large 25}/{\footnotesize 150} & \cellcolor{green!20}{\large 21}/{\footnotesize 154} \tabularnewline
\rowcolor{black!10} 19 & oniguruma\_onig\_new & 4 & \cellcolor{green!20}{\large 6}/{\footnotesize 40} & \cellcolor{green!20}{\large 5}/{\footnotesize 40} & \cellcolor{green!0}{\large 0}/{\footnotesize 40} & \cellcolor{green!20}{\large 7}/{\footnotesize 40} & \cellcolor{green!30}{\large 27}/{\footnotesize 118} & \cellcolor{green!20}{\large 18}/{\footnotesize 144} \tabularnewline
20 & pupnp\_ixmlLoadDocumentEx & 4 & \cellcolor{green!0}{\large 0}/{\footnotesize 40} & \cellcolor{green!20}{\large 6}/{\footnotesize 40} & \cellcolor{green!0}{\large 0}/{\footnotesize 40} & \cellcolor{green!10}{\large 1}/{\footnotesize 40} & \cellcolor{green!10}{\large 2}/{\footnotesize 99} & \cellcolor{green!0}{\large 0}/{\footnotesize 119} \tabularnewline
\rowcolor{black!10} 21 & gdk-pixbuf\_gdk\_pixbuf\_new\_from\_file\_at\_scale & 5 & \cellcolor{green!0}{\large 0}/{\footnotesize 40} & \cellcolor{green!10}{\large 1}/{\footnotesize 40} & \cellcolor{green!0}{\large 0}/{\footnotesize 40} & \cellcolor{green!10}{\large 2}/{\footnotesize 40} & \cellcolor{green!10}{\large 2}/{\footnotesize 95} & \cellcolor{green!10}{\large 2}/{\footnotesize 122} \tabularnewline
22 & inchi\_GetINCHIKeyFromINCHI & 5 & \cellcolor{green!0}{\large 0}/{\footnotesize 40} & \cellcolor{green!70}{\large 26}/{\footnotesize 40} & \cellcolor{green!20}{\large 7}/{\footnotesize 40} & \cellcolor{green!30}{\large 11}/{\footnotesize 40} & \cellcolor{green!40}{\large 34}/{\footnotesize 88} & \cellcolor{green!40}{\large 31}/{\footnotesize 100} \tabularnewline
\rowcolor{black!10} 23 & libdwarf\_dwarf\_init\_b & 5 & \cellcolor{green!0}{\large 0}/{\footnotesize 40} & \cellcolor{green!0}{\large 0}/{\footnotesize 40} & \cellcolor{green!10}{\large 1}/{\footnotesize 40} & \cellcolor{green!20}{\large 5}/{\footnotesize 40} & \cellcolor{green!10}{\large 10}/{\footnotesize 175} & \cellcolor{green!10}{\large 6}/{\footnotesize 97} \tabularnewline
24 & libdwarf\_dwarf\_init\_path & 5 & \cellcolor{green!0}{\large 0}/{\footnotesize 40} & \cellcolor{green!0}{\large 0}/{\footnotesize 40} & \cellcolor{green!0}{\large 0}/{\footnotesize 40} & \cellcolor{green!10}{\large 1}/{\footnotesize 40} & \cellcolor{green!0}{\large 0}/{\footnotesize 194} & \cellcolor{green!0}{\large 0}/{\footnotesize 126} \tabularnewline
\rowcolor{black!10} 25 & liblouis\_lou\_compileString & 5 & \cellcolor{green!0}{\large 0}/{\footnotesize 40} & \cellcolor{green!10}{\large 1}/{\footnotesize 40} & \cellcolor{green!0}{\large 0}/{\footnotesize 40} & \cellcolor{green!10}{\large 1}/{\footnotesize 40} & \cellcolor{green!10}{\large 6}/{\footnotesize 180} & \cellcolor{green!10}{\large 5}/{\footnotesize 189} \tabularnewline
26 & selinux\_cil\_compile & 5 & \cellcolor{green!0}{\large 0}/{\footnotesize 40} & \cellcolor{green!0}{\large 0}/{\footnotesize 40} & \cellcolor{green!0}{\large -}{\tiny -} & \cellcolor{green!40}{\large 13}/{\footnotesize 40} & \cellcolor{green!0}{\large 0}/{\footnotesize 200} & \cellcolor{green!20}{\large 19}/{\footnotesize 128} \tabularnewline
\rowcolor{black!10} 27 & bind9\_dns\_name\_fromtext & 6 & \cellcolor{green!0}{\large 0}/{\footnotesize 40} & \cellcolor{green!0}{\large 0}/{\footnotesize 40} & \cellcolor{green!0}{\large -}{\tiny -} & \cellcolor{green!10}{\large 1}/{\footnotesize 40} & \cellcolor{green!0}{\large 1}/{\footnotesize 198} & \cellcolor{green!10}{\large 4}/{\footnotesize 170} \tabularnewline
28 & bind9\_dns\_rdata\_fromwire & 6 & \cellcolor{green!0}{\large 0}/{\footnotesize 40} & \cellcolor{green!0}{\large 0}/{\footnotesize 40} & \cellcolor{green!0}{\large -}{\tiny -} & \cellcolor{green!0}{\large 0}/{\footnotesize 40} & \cellcolor{green!0}{\large 0}/{\footnotesize 200} & \cellcolor{green!10}{\large 4}/{\footnotesize 177} \tabularnewline
\rowcolor{black!10} 29 & coturn\_stun\_is\_binding\_response & 6 & \cellcolor{green!0}{\large 0}/{\footnotesize 40} & \cellcolor{green!0}{\large 0}/{\footnotesize 40} & \cellcolor{green!0}{\large -}{\tiny -} & \cellcolor{green!10}{\large 4}/{\footnotesize 40} & \cellcolor{green!10}{\large 2}/{\footnotesize 198} & \cellcolor{green!20}{\large 14}/{\footnotesize 127} \tabularnewline
30 & coturn\_stun\_is\_command\_message & 6 & \cellcolor{green!0}{\large 0}/{\footnotesize 40} & \cellcolor{green!0}{\large 0}/{\footnotesize 40} & \cellcolor{green!0}{\large 0}/{\footnotesize 40} & \cellcolor{green!60}{\large 22}/{\footnotesize 40} & \cellcolor{green!10}{\large 6}/{\footnotesize 187} & \cellcolor{green!30}{\large 23}/{\footnotesize 95} \tabularnewline
\rowcolor{black!10} 31 & coturn\_stun\_is\_response & 6 & \cellcolor{green!0}{\large 0}/{\footnotesize 40} & \cellcolor{green!0}{\large 0}/{\footnotesize 40} & \cellcolor{green!0}{\large -}{\tiny -} & \cellcolor{green!40}{\large 15}/{\footnotesize 40} & \cellcolor{green!0}{\large 0}/{\footnotesize 194} & \cellcolor{green!20}{\large 16}/{\footnotesize 103} \tabularnewline
32 & coturn\_stun\_is\_success\_response & 6 & \cellcolor{green!0}{\large 0}/{\footnotesize 40} & \cellcolor{green!0}{\large 0}/{\footnotesize 40} & \cellcolor{green!0}{\large -}{\tiny -} & \cellcolor{green!20}{\large 5}/{\footnotesize 40} & \cellcolor{green!10}{\large 4}/{\footnotesize 196} & \cellcolor{green!10}{\large 11}/{\footnotesize 131} \tabularnewline
\rowcolor{black!10} 33 & hiredis\_redisFormatCommand & 6 & \cellcolor{green!0}{\large 0}/{\footnotesize 40} & \cellcolor{green!60}{\large 21}/{\footnotesize 40} & \cellcolor{green!0}{\large -}{\tiny -} & \cellcolor{green!40}{\large 14}/{\footnotesize 40} & \cellcolor{green!10}{\large 9}/{\footnotesize 192} & \cellcolor{green!10}{\large 8}/{\footnotesize 164} \tabularnewline
34 & igraph\_igraph\_read\_graph\_dl & 6 & \cellcolor{green!0}{\large 0}/{\footnotesize 40} & \cellcolor{green!0}{\large 0}/{\footnotesize 40} & \cellcolor{green!0}{\large 0}/{\footnotesize 40} & \cellcolor{green!0}{\large 0}/{\footnotesize 40} & \cellcolor{green!10}{\large 17}/{\footnotesize 167} & \cellcolor{green!10}{\large 5}/{\footnotesize 193} \tabularnewline
\rowcolor{black!10} 35 & igraph\_igraph\_read\_graph\_edgelist & 6 & \cellcolor{green!0}{\large 0}/{\footnotesize 40} & \cellcolor{green!10}{\large 1}/{\footnotesize 40} & \cellcolor{green!0}{\large 0}/{\footnotesize 40} & \cellcolor{green!0}{\large 0}/{\footnotesize 40} & \cellcolor{green!0}{\large 1}/{\footnotesize 198} & \cellcolor{green!0}{\large 1}/{\footnotesize 200} \tabularnewline
36 & igraph\_igraph\_read\_graph\_gml & 6 & \cellcolor{green!0}{\large 0}/{\footnotesize 40} & \cellcolor{green!0}{\large 0}/{\footnotesize 40} & \cellcolor{green!0}{\large 0}/{\footnotesize 40} & \cellcolor{green!0}{\large 0}/{\footnotesize 40} & \cellcolor{green!10}{\large 11}/{\footnotesize 177} & \cellcolor{green!10}{\large 4}/{\footnotesize 194} \tabularnewline
\rowcolor{black!10} 37 & igraph\_igraph\_read\_graph\_graphdb & 6 & \cellcolor{green!0}{\large 0}/{\footnotesize 40} & \cellcolor{green!0}{\large 0}/{\footnotesize 40} & \cellcolor{green!0}{\large 0}/{\footnotesize 40} & \cellcolor{green!0}{\large 0}/{\footnotesize 40} & \cellcolor{green!10}{\large 2}/{\footnotesize 196} & \cellcolor{green!0}{\large 0}/{\footnotesize 192} \tabularnewline
38 & igraph\_igraph\_read\_graph\_graphml & 6 & \cellcolor{green!0}{\large 0}/{\footnotesize 40} & \cellcolor{green!0}{\large 0}/{\footnotesize 40} & \cellcolor{green!0}{\large 0}/{\footnotesize 40} & \cellcolor{green!20}{\large 5}/{\footnotesize 40} & \cellcolor{green!20}{\large 19}/{\footnotesize 163} & \cellcolor{green!10}{\large 14}/{\footnotesize 162} \tabularnewline
\rowcolor{black!10} 39 & igraph\_igraph\_read\_graph\_lgl & 6 & \cellcolor{green!0}{\large 0}/{\footnotesize 40} & \cellcolor{green!0}{\large 0}/{\footnotesize 40} & \cellcolor{green!0}{\large 0}/{\footnotesize 40} & \cellcolor{green!0}{\large 0}/{\footnotesize 40} & \cellcolor{green!0}{\large 0}/{\footnotesize 198} & \cellcolor{green!0}{\large 0}/{\footnotesize 196} \tabularnewline
40 & igraph\_igraph\_read\_graph\_pajek & 6 & \cellcolor{green!0}{\large 0}/{\footnotesize 40} & \cellcolor{green!0}{\large 0}/{\footnotesize 40} & \cellcolor{green!0}{\large 0}/{\footnotesize 40} & \cellcolor{green!0}{\large 0}/{\footnotesize 40} & \cellcolor{green!20}{\large 22}/{\footnotesize 156} & \cellcolor{green!10}{\large 8}/{\footnotesize 182} \tabularnewline
\rowcolor{black!10} 41 & inchi\_GetINCHIfromINCHI & 6 & \cellcolor{green!0}{\large 0}/{\footnotesize 40} & \cellcolor{green!0}{\large 0}/{\footnotesize 40} & \cellcolor{green!0}{\large 0}/{\footnotesize 40} & \cellcolor{green!10}{\large 3}/{\footnotesize 40} & \cellcolor{green!10}{\large 12}/{\footnotesize 200} & \cellcolor{green!10}{\large 14}/{\footnotesize 179} \tabularnewline
42 & inchi\_GetStructFromINCHI & 6 & \cellcolor{green!0}{\large 0}/{\footnotesize 40} & \cellcolor{green!0}{\large 0}/{\footnotesize 40} & \cellcolor{green!0}{\large 0}/{\footnotesize 40} & \cellcolor{green!0}{\large 0}/{\footnotesize 40} & \cellcolor{green!10}{\large 15}/{\footnotesize 188} & \cellcolor{green!10}{\large 4}/{\footnotesize 192} \tabularnewline
\rowcolor{black!10} 43 & kamailio\_parse\_msg & 6 & \cellcolor{green!0}{\large 0}/{\footnotesize 40} & \cellcolor{green!10}{\large 4}/{\footnotesize 40} & \cellcolor{green!0}{\large -}{\tiny -} & \cellcolor{green!30}{\large 11}/{\footnotesize 40} & \cellcolor{green!20}{\large 21}/{\footnotesize 122} & \cellcolor{green!20}{\large 23}/{\footnotesize 110} \tabularnewline
44 & libyang\_lys\_parse\_mem & 6 & \cellcolor{green!0}{\large 0}/{\footnotesize 40} & \cellcolor{green!0}{\large 0}/{\footnotesize 40} & \cellcolor{green!0}{\large 0}/{\footnotesize 40} & \cellcolor{green!10}{\large 1}/{\footnotesize 40} & \cellcolor{green!0}{\large 0}/{\footnotesize 200} & \cellcolor{green!10}{\large 6}/{\footnotesize 185} \tabularnewline
\rowcolor{black!10} 45 & proftpd\_pr\_json\_object\_from\_text & 6 & \cellcolor{green!0}{\large 0}/{\footnotesize 40} & \cellcolor{green!0}{\large 0}/{\footnotesize 40} & \cellcolor{green!0}{\large -}{\tiny -} & \cellcolor{green!10}{\large 3}/{\footnotesize 40} & \cellcolor{green!0}{\large 0}/{\footnotesize 196} & \cellcolor{green!10}{\large 4}/{\footnotesize 156} \tabularnewline
46 & selinux\_policydb\_read & 6 & \cellcolor{green!0}{\large 0}/{\footnotesize 40} & \cellcolor{green!20}{\large 5}/{\footnotesize 40} & \cellcolor{green!0}{\large -}{\tiny -} & \cellcolor{green!0}{\large 0}/{\footnotesize 40} & \cellcolor{green!20}{\large 18}/{\footnotesize 139} & \cellcolor{green!10}{\large 8}/{\footnotesize 162} \tabularnewline
\rowcolor{black!10} 47 & kamailio\_get\_src\_address\_socket & 7 & \cellcolor{green!0}{\large 0}/{\footnotesize 40} & \cellcolor{green!0}{\large 0}/{\footnotesize 40} & \cellcolor{green!0}{\large 0}/{\footnotesize 40} & \cellcolor{green!20}{\large 7}/{\footnotesize 40} & \cellcolor{green!0}{\large 0}/{\footnotesize 179} & \cellcolor{green!10}{\large 6}/{\footnotesize 156} \tabularnewline
48 & kamailio\_get\_src\_uri & 7 & \cellcolor{green!0}{\large 0}/{\footnotesize 40} & \cellcolor{green!0}{\large 0}/{\footnotesize 40} & \cellcolor{green!0}{\large 0}/{\footnotesize 40} & \cellcolor{green!30}{\large 9}/{\footnotesize 40} & \cellcolor{green!0}{\large 0}/{\footnotesize 180} & \cellcolor{green!10}{\large 5}/{\footnotesize 154} \tabularnewline
\rowcolor{black!10} 49 & kamailio\_parse\_content\_disposition & 7 & \cellcolor{green!0}{\large 0}/{\footnotesize 40} & \cellcolor{green!0}{\large 0}/{\footnotesize 40} & \cellcolor{green!0}{\large 0}/{\footnotesize 40} & \cellcolor{green!0}{\large 0}/{\footnotesize 40} & \cellcolor{green!0}{\large 0}/{\footnotesize 194} & \cellcolor{green!10}{\large 3}/{\footnotesize 164} \tabularnewline
50 & kamailio\_parse\_diversion\_header & 7 & \cellcolor{green!0}{\large 0}/{\footnotesize 40} & \cellcolor{green!0}{\large 0}/{\footnotesize 40} & \cellcolor{green!0}{\large 0}/{\footnotesize 40} & \cellcolor{green!10}{\large 2}/{\footnotesize 40} & \cellcolor{green!0}{\large 0}/{\footnotesize 189} & \cellcolor{green!10}{\large 2}/{\footnotesize 168} \tabularnewline
\rowcolor{black!10} 51 & kamailio\_parse\_from\_header & 7 & \cellcolor{green!0}{\large 0}/{\footnotesize 40} & \cellcolor{green!0}{\large 0}/{\footnotesize 40} & \cellcolor{green!0}{\large -}{\tiny -} & \cellcolor{green!0}{\large 0}/{\footnotesize 40} & \cellcolor{green!0}{\large 0}/{\footnotesize 199} & \cellcolor{green!0}{\large 0}/{\footnotesize 162} \tabularnewline
52 & kamailio\_parse\_from\_uri & 7 & \cellcolor{green!0}{\large 0}/{\footnotesize 40} & \cellcolor{green!0}{\large 0}/{\footnotesize 40} & \cellcolor{green!0}{\large -}{\tiny -} & \cellcolor{green!0}{\large 0}/{\footnotesize 40} & \cellcolor{green!0}{\large 0}/{\footnotesize 197} & \cellcolor{green!10}{\large 2}/{\footnotesize 168} \tabularnewline
\rowcolor{black!10} 53 & kamailio\_parse\_headers & 7 & \cellcolor{green!0}{\large 0}/{\footnotesize 40} & \cellcolor{green!0}{\large 0}/{\footnotesize 40} & \cellcolor{green!0}{\large -}{\tiny -} & \cellcolor{green!0}{\large 0}/{\footnotesize 40} & \cellcolor{green!0}{\large 0}/{\footnotesize 188} & \cellcolor{green!0}{\large 1}/{\footnotesize 144} \tabularnewline
54 & kamailio\_parse\_identityinfo\_header & 7 & \cellcolor{green!0}{\large 0}/{\footnotesize 40} & \cellcolor{green!0}{\large 0}/{\footnotesize 40} & \cellcolor{green!0}{\large -}{\tiny -} & \cellcolor{green!10}{\large 4}/{\footnotesize 40} & \cellcolor{green!0}{\large 0}/{\footnotesize 183} & \cellcolor{green!10}{\large 6}/{\footnotesize 161} \tabularnewline
\rowcolor{black!10} 55 & kamailio\_parse\_pai\_header & 7 & \cellcolor{green!0}{\large 0}/{\footnotesize 40} & \cellcolor{green!0}{\large 0}/{\footnotesize 40} & \cellcolor{green!0}{\large -}{\tiny -} & \cellcolor{green!0}{\large 0}/{\footnotesize 40} & \cellcolor{green!0}{\large 0}/{\footnotesize 178} & \cellcolor{green!10}{\large 2}/{\footnotesize 168} \tabularnewline
56 & kamailio\_parse\_privacy & 7 & \cellcolor{green!0}{\large 0}/{\footnotesize 40} & \cellcolor{green!0}{\large 0}/{\footnotesize 40} & \cellcolor{green!0}{\large 0}/{\footnotesize 40} & \cellcolor{green!0}{\large 0}/{\footnotesize 40} & \cellcolor{green!0}{\large 0}/{\footnotesize 200} & \cellcolor{green!10}{\large 5}/{\footnotesize 176} \tabularnewline
\rowcolor{black!10} 57 & kamailio\_parse\_record\_route\_headers & 7 & \cellcolor{green!0}{\large 0}/{\footnotesize 40} & \cellcolor{green!0}{\large 0}/{\footnotesize 40} & \cellcolor{green!0}{\large -}{\tiny -} & \cellcolor{green!10}{\large 2}/{\footnotesize 40} & \cellcolor{green!0}{\large 0}/{\footnotesize 194} & \cellcolor{green!10}{\large 10}/{\footnotesize 142} \tabularnewline
58 & kamailio\_parse\_refer\_to\_header & 7 & \cellcolor{green!0}{\large 0}/{\footnotesize 40} & \cellcolor{green!0}{\large 0}/{\footnotesize 40} & \cellcolor{green!0}{\large -}{\tiny -} & \cellcolor{green!0}{\large 0}/{\footnotesize 40} & \cellcolor{green!0}{\large 0}/{\footnotesize 186} & \cellcolor{green!10}{\large 3}/{\footnotesize 161} \tabularnewline
\rowcolor{black!10} 59 & kamailio\_parse\_route\_headers & 7 & \cellcolor{green!0}{\large 0}/{\footnotesize 40} & \cellcolor{green!0}{\large 0}/{\footnotesize 40} & \cellcolor{green!0}{\large -}{\tiny -} & \cellcolor{green!20}{\large 6}/{\footnotesize 40} & \cellcolor{green!0}{\large 0}/{\footnotesize 187} & \cellcolor{green!20}{\large 15}/{\footnotesize 136} \tabularnewline
60 & kamailio\_parse\_to\_header & 7 & \cellcolor{green!0}{\large 0}/{\footnotesize 40} & \cellcolor{green!0}{\large 0}/{\footnotesize 40} & \cellcolor{green!0}{\large -}{\tiny -} & \cellcolor{green!0}{\large 0}/{\footnotesize 40} & \cellcolor{green!0}{\large 0}/{\footnotesize 187} & \cellcolor{green!10}{\large 3}/{\footnotesize 174} \tabularnewline
\rowcolor{black!10} 61 & kamailio\_parse\_to\_uri & 7 & \cellcolor{green!0}{\large 0}/{\footnotesize 40} & \cellcolor{green!0}{\large 0}/{\footnotesize 40} & \cellcolor{green!0}{\large -}{\tiny -} & \cellcolor{green!0}{\large 0}/{\footnotesize 40} & \cellcolor{green!0}{\large 0}/{\footnotesize 184} & \cellcolor{green!0}{\large 0}/{\footnotesize 175} \tabularnewline
62 & libyang\_lyd\_parse\_data\_mem & 7 & \cellcolor{green!0}{\large 0}/{\footnotesize 40} & \cellcolor{green!0}{\large 0}/{\footnotesize 40} & \cellcolor{green!0}{\large 0}/{\footnotesize 40} & \cellcolor{green!0}{\large 0}/{\footnotesize 40} & \cellcolor{green!10}{\large 2}/{\footnotesize 194} & \cellcolor{green!10}{\large 4}/{\footnotesize 183} \tabularnewline
\rowcolor{black!10} 63 & bind9\_dns\_message\_parse & 8 & \cellcolor{green!0}{\large 0}/{\footnotesize 40} & \cellcolor{green!0}{\large 0}/{\footnotesize 40} & \cellcolor{green!0}{\large -}{\tiny -} & \cellcolor{green!10}{\large 1}/{\footnotesize 40} & \cellcolor{green!0}{\large 0}/{\footnotesize 200} & \cellcolor{green!10}{\large 3}/{\footnotesize 145} \tabularnewline
64 & igraph\_igraph\_read\_graph\_ncol & 8 & \cellcolor{green!0}{\large 0}/{\footnotesize 40} & \cellcolor{green!0}{\large 0}/{\footnotesize 40} & \cellcolor{green!0}{\large 0}/{\footnotesize 40} & \cellcolor{green!0}{\large 0}/{\footnotesize 40} & \cellcolor{green!10}{\large 5}/{\footnotesize 187} & \cellcolor{green!10}{\large 3}/{\footnotesize 184} \tabularnewline
\rowcolor{black!10} 65 & pjsip\_pj\_json\_parse & 8 & \cellcolor{green!0}{\large 0}/{\footnotesize 40} & \cellcolor{green!0}{\large 0}/{\footnotesize 40} & \cellcolor{green!0}{\large 0}/{\footnotesize 40} & \cellcolor{green!0}{\large 0}/{\footnotesize 40} & \cellcolor{green!0}{\large 1}/{\footnotesize 193} & \cellcolor{green!10}{\large 4}/{\footnotesize 192} \tabularnewline
66 & pjsip\_pj\_xml\_parse & 8 & \cellcolor{green!0}{\large 0}/{\footnotesize 40} & \cellcolor{green!0}{\large 0}/{\footnotesize 40} & \cellcolor{green!0}{\large 0}/{\footnotesize 40} & \cellcolor{green!0}{\large 0}/{\footnotesize 40} & \cellcolor{green!10}{\large 4}/{\footnotesize 186} & \cellcolor{green!10}{\large 5}/{\footnotesize 186} \tabularnewline
\rowcolor{black!10} 67 & pjsip\_pjmedia\_sdp\_parse & 8 & \cellcolor{green!0}{\large 0}/{\footnotesize 40} & \cellcolor{green!10}{\large 3}/{\footnotesize 40} & \cellcolor{green!0}{\large 0}/{\footnotesize 40} & \cellcolor{green!10}{\large 3}/{\footnotesize 40} & \cellcolor{green!10}{\large 6}/{\footnotesize 181} & \cellcolor{green!10}{\large 3}/{\footnotesize 167} \tabularnewline
68 & quickjs\_lre\_compile & 8 & \cellcolor{green!0}{\large 0}/{\footnotesize 40} & \cellcolor{green!0}{\large 0}/{\footnotesize 40} & \cellcolor{green!0}{\large -}{\tiny -} & \cellcolor{green!0}{\large 0}/{\footnotesize 40} & \cellcolor{green!0}{\large 0}/{\footnotesize 191} & \cellcolor{green!0}{\large 0}/{\footnotesize 194} \tabularnewline
\rowcolor{black!10} 69 & bind9\_isc\_lex\_getmastertoken & 9 & \cellcolor{green!0}{\large 0}/{\footnotesize 40} & \cellcolor{green!0}{\large 0}/{\footnotesize 40} & \cellcolor{green!0}{\large -}{\tiny -} & \cellcolor{green!0}{\large 0}/{\footnotesize 40} & \cellcolor{green!0}{\large 0}/{\footnotesize 195} & \cellcolor{green!0}{\large 0}/{\footnotesize 182} \tabularnewline
70 & bind9\_isc\_lex\_gettoken & 9 & \cellcolor{green!0}{\large 0}/{\footnotesize 40} & \cellcolor{green!0}{\large 0}/{\footnotesize 40} & \cellcolor{green!0}{\large -}{\tiny -} & \cellcolor{green!0}{\large 0}/{\footnotesize 40} & \cellcolor{green!0}{\large 0}/{\footnotesize 199} & \cellcolor{green!10}{\large 2}/{\footnotesize 180} \tabularnewline
\rowcolor{black!10} 71 & quickjs\_JS\_Eval & 9 & \cellcolor{green!0}{\large 0}/{\footnotesize 40} & \cellcolor{green!0}{\large 0}/{\footnotesize 40} & \cellcolor{green!0}{\large -}{\tiny -} & \cellcolor{green!10}{\large 2}/{\footnotesize 40} & \cellcolor{green!10}{\large 12}/{\footnotesize 168} & \cellcolor{green!10}{\large 17}/{\footnotesize 155} \tabularnewline
72 & igraph\_igraph\_edge\_connectivity & 10 & \cellcolor{green!0}{\large 0}/{\footnotesize 40} & \cellcolor{green!0}{\large 0}/{\footnotesize 40} & \cellcolor{green!0}{\large 0}/{\footnotesize 40} & \cellcolor{green!0}{\large 0}/{\footnotesize 40} & \cellcolor{green!0}{\large 0}/{\footnotesize 173} & \cellcolor{green!0}{\large 0}/{\footnotesize 175} \tabularnewline
\rowcolor{black!10} 73 & pjsip\_pj\_stun\_msg\_decode & 10 & \cellcolor{green!0}{\large 0}/{\footnotesize 40} & \cellcolor{green!10}{\large 1}/{\footnotesize 40} & \cellcolor{green!0}{\large 0}/{\footnotesize 40} & \cellcolor{green!0}{\large 0}/{\footnotesize 40} & \cellcolor{green!0}{\large 1}/{\footnotesize 200} & \cellcolor{green!10}{\large 5}/{\footnotesize 184} \tabularnewline
74 & bind9\_dns\_message\_checksig & 11 & \cellcolor{green!0}{\large 0}/{\footnotesize 40} & \cellcolor{green!0}{\large 0}/{\footnotesize 40} & \cellcolor{green!0}{\large -}{\tiny -} & \cellcolor{green!0}{\large 0}/{\footnotesize 40} & \cellcolor{green!0}{\large 0}/{\footnotesize 200} & \cellcolor{green!0}{\large 0}/{\footnotesize 113} \tabularnewline
\rowcolor{black!10} 75 & libzip\_zip\_fread & 11 & \cellcolor{green!0}{\large 0}/{\footnotesize 40} & \cellcolor{green!10}{\large 1}/{\footnotesize 40} & \cellcolor{green!0}{\large 0}/{\footnotesize 40} & \cellcolor{green!10}{\large 1}/{\footnotesize 40} & \cellcolor{green!10}{\large 4}/{\footnotesize 181} & \cellcolor{green!10}{\large 5}/{\footnotesize 175} \tabularnewline
76 & bind9\_dns\_rdata\_fromtext & 12 & \cellcolor{green!0}{\large 0}/{\footnotesize 40} & \cellcolor{green!0}{\large 0}/{\footnotesize 40} & \cellcolor{green!0}{\large -}{\tiny -} & \cellcolor{green!10}{\large 1}/{\footnotesize 40} & \cellcolor{green!0}{\large 0}/{\footnotesize 175} & \cellcolor{green!0}{\large 0}/{\footnotesize 127} \tabularnewline
\rowcolor{black!10} 77 & igraph\_igraph\_all\_minimal\_st\_separators & 12 & \cellcolor{green!10}{\large 1}/{\footnotesize 40} & \cellcolor{green!20}{\large 7}/{\footnotesize 40} & \cellcolor{green!0}{\large 0}/{\footnotesize 40} & \cellcolor{green!0}{\large 0}/{\footnotesize 40} & \cellcolor{green!10}{\large 10}/{\footnotesize 176} & \cellcolor{green!10}{\large 6}/{\footnotesize 183} \tabularnewline
78 & igraph\_igraph\_minimum\_size\_separators & 12 & \cellcolor{green!0}{\large 0}/{\footnotesize 40} & \cellcolor{green!30}{\large 12}/{\footnotesize 40} & \cellcolor{green!10}{\large 2}/{\footnotesize 40} & \cellcolor{green!10}{\large 2}/{\footnotesize 40} & \cellcolor{green!10}{\large 14}/{\footnotesize 159} & \cellcolor{green!10}{\large 3}/{\footnotesize 148} \tabularnewline
\rowcolor{black!10} 79 & pjsip\_pjsip\_parse\_msg & 12 & \cellcolor{green!0}{\large 0}/{\footnotesize 40} & \cellcolor{green!0}{\large 0}/{\footnotesize 40} & \cellcolor{green!0}{\large 0}/{\footnotesize 40} & \cellcolor{green!0}{\large 0}/{\footnotesize 40} & \cellcolor{green!0}{\large 0}/{\footnotesize 200} & \cellcolor{green!0}{\large 1}/{\footnotesize 196} \tabularnewline
80 & igraph\_igraph\_automorphism\_group & 13 & \cellcolor{green!0}{\large 0}/{\footnotesize 40} & \cellcolor{green!0}{\large 0}/{\footnotesize 40} & \cellcolor{green!0}{\large 0}/{\footnotesize 40} & \cellcolor{green!10}{\large 4}/{\footnotesize 40} & \cellcolor{green!0}{\large 0}/{\footnotesize 197} & \cellcolor{green!10}{\large 7}/{\footnotesize 176} \tabularnewline
\rowcolor{black!10} 81 & libmodbus\_modbus\_read\_bits & 15 & \cellcolor{green!0}{\large 0}/{\footnotesize 40} & \cellcolor{green!0}{\large 0}/{\footnotesize 40} & \cellcolor{green!0}{\large 0}/{\footnotesize 40} & \cellcolor{green!0}{\large 0}/{\footnotesize 40} & \cellcolor{green!0}{\large 0}/{\footnotesize 161} & \cellcolor{green!0}{\large 0}/{\footnotesize 125} \tabularnewline
82 & libmodbus\_modbus\_read\_registers & 15 & \cellcolor{green!0}{\large 0}/{\footnotesize 40} & \cellcolor{green!0}{\large 0}/{\footnotesize 40} & \cellcolor{green!0}{\large 0}/{\footnotesize 40} & \cellcolor{green!0}{\large 0}/{\footnotesize 40} & \cellcolor{green!0}{\large 0}/{\footnotesize 148} & \cellcolor{green!0}{\large 0}/{\footnotesize 133} \tabularnewline
\rowcolor{black!10} 83 & civetweb\_mg\_get\_response & 17 & \cellcolor{green!0}{\large 0}/{\footnotesize 40} & \cellcolor{green!0}{\large 0}/{\footnotesize 40} & \cellcolor{green!0}{\large 0}/{\footnotesize 40} & \cellcolor{green!0}{\large 0}/{\footnotesize 40} & \cellcolor{green!0}{\large 0}/{\footnotesize 196} & \cellcolor{green!0}{\large 0}/{\footnotesize 144} \tabularnewline
84 & bind9\_dns\_master\_loadbuffer & 20 & \cellcolor{green!0}{\large 0}/{\footnotesize 40} & \cellcolor{green!0}{\large 0}/{\footnotesize 40} & \cellcolor{green!0}{\large -}{\tiny -} & \cellcolor{green!0}{\large 0}/{\footnotesize 40} & \cellcolor{green!0}{\large 0}/{\footnotesize 200} & \cellcolor{green!0}{\large 0}/{\footnotesize 200} \tabularnewline
\rowcolor{black!10} 85 & libmodbus\_modbus\_receive & 33 & \cellcolor{green!0}{\large 0}/{\footnotesize 40} & \cellcolor{green!0}{\large 0}/{\footnotesize 40} & \cellcolor{green!0}{\large 0}/{\footnotesize 40} & \cellcolor{green!0}{\large 0}/{\footnotesize 40} & \cellcolor{green!0}{\large 0}/{\footnotesize 174} & \cellcolor{green!0}{\large 0}/{\footnotesize 147} \tabularnewline
86 & tmux\_input\_parse\_buffer & 42 & \cellcolor{green!0}{\large 0}/{\footnotesize 40} & \cellcolor{green!0}{\large 0}/{\footnotesize 40} & \cellcolor{green!0}{\large -}{\tiny -} & \cellcolor{green!0}{\large 0}/{\footnotesize 40} & \cellcolor{green!0}{\large 0}/{\footnotesize 190} & \cellcolor{green!0}{\large 0}/{\footnotesize 183} \tabularnewline

\bottomrule
%\end{tabular}
%}
%\end{table*}
\end{xltabular}
}
\twocolumn



% model: gpt-3.5-turbo-0613, temp: 1.0

\onecolumn
{\small %
\begin{xltabular}[h]{\textwidth}{ccccccccc}
%\begin{table*}[!t]
%\centering
\caption{Evaluation Result of model gpt-3.5-turbo-0613 with temperature 1.0.} \\
%\resizebox{1.0\linewidth}{!}{
%\begin{tabular}{cccccccccc}
\toprule
Index & Question & Score & NAIVE-40 & BACTX-40 & DOCTX-40 & UGCTX-40 & BA-ITER-40 & ALL-ITER-40 \tabularnewline
\midrule
\rowcolor{black!10} 1 & coturn\_stun\_is\_command\_message\_full\_check\_str & 1 & \cellcolor{green!0}{\large 0}/{\footnotesize 40} & \cellcolor{green!60}{\large 23}/{\footnotesize 40} & \cellcolor{green!0}{\large -}{\tiny -} & \cellcolor{green!30}{\large 10}/{\footnotesize 40} & \cellcolor{green!60}{\large 31}/{\footnotesize 55} & \cellcolor{green!40}{\large 22}/{\footnotesize 70} \tabularnewline
2 & kamailio\_parse\_uri & 1 & \cellcolor{green!0}{\large 0}/{\footnotesize 40} & \cellcolor{green!80}{\large 29}/{\footnotesize 40} & \cellcolor{green!0}{\large -}{\tiny -} & \cellcolor{green!20}{\large 7}/{\footnotesize 40} & \cellcolor{green!80}{\large 37}/{\footnotesize 47} & \cellcolor{green!20}{\large 19}/{\footnotesize 103} \tabularnewline
\rowcolor{black!10} 3 & coturn\_stun\_check\_message\_integrity\_str & 2 & \cellcolor{green!0}{\large 0}/{\footnotesize 40} & \cellcolor{green!0}{\large 0}/{\footnotesize 40} & \cellcolor{green!0}{\large -}{\tiny -} & \cellcolor{green!10}{\large 1}/{\footnotesize 40} & \cellcolor{green!10}{\large 15}/{\footnotesize 144} & \cellcolor{green!10}{\large 5}/{\footnotesize 135} \tabularnewline
4 & libiec61850\_MmsValue\_decodeMmsData & 2 & \cellcolor{green!0}{\large 0}/{\footnotesize 40} & \cellcolor{green!70}{\large 26}/{\footnotesize 40} & \cellcolor{green!40}{\large 13}/{\footnotesize 40} & \cellcolor{green!10}{\large 4}/{\footnotesize 40} & \cellcolor{green!70}{\large 36}/{\footnotesize 57} & \cellcolor{green!20}{\large 20}/{\footnotesize 117} \tabularnewline
\rowcolor{black!10} 5 & md4c\_md\_html & 2 & \cellcolor{green!0}{\large 0}/{\footnotesize 40} & \cellcolor{green!0}{\large 0}/{\footnotesize 40} & \cellcolor{green!0}{\large 0}/{\footnotesize 40} & \cellcolor{green!0}{\large 0}/{\footnotesize 40} & \cellcolor{green!30}{\large 28}/{\footnotesize 125} & \cellcolor{green!10}{\large 14}/{\footnotesize 150} \tabularnewline
6 & spdk\_spdk\_json\_parse & 2 & \cellcolor{green!0}{\large 0}/{\footnotesize 40} & \cellcolor{green!80}{\large 31}/{\footnotesize 40} & \cellcolor{green!0}{\large -}{\tiny -} & \cellcolor{green!20}{\large 8}/{\footnotesize 40} & \cellcolor{green!60}{\large 34}/{\footnotesize 62} & \cellcolor{green!20}{\large 18}/{\footnotesize 108} \tabularnewline
\rowcolor{black!10} 7 & croaring\_roaring\_bitmap\_portable\_deserialize\_safe & 3 & \cellcolor{green!10}{\large 1}/{\footnotesize 40} & \cellcolor{green!40}{\large 16}/{\footnotesize 40} & \cellcolor{green!70}{\large 26}/{\footnotesize 40} & \cellcolor{green!30}{\large 11}/{\footnotesize 40} & \cellcolor{green!50}{\large 33}/{\footnotesize 69} & \cellcolor{green!40}{\large 28}/{\footnotesize 80} \tabularnewline
8 & lua\_luaL\_loadbufferx & 3 & \cellcolor{green!10}{\large 4}/{\footnotesize 40} & \cellcolor{green!90}{\large 33}/{\footnotesize 40} & \cellcolor{green!70}{\large 28}/{\footnotesize 40} & \cellcolor{green!30}{\large 9}/{\footnotesize 40} & \cellcolor{green!90}{\large 38}/{\footnotesize 43} & \cellcolor{green!30}{\large 24}/{\footnotesize 110} \tabularnewline
\rowcolor{black!10} 9 & w3m\_wc\_Str\_conv\_with\_detect & 3 & \cellcolor{green!0}{\large 0}/{\footnotesize 40} & \cellcolor{green!0}{\large 0}/{\footnotesize 40} & \cellcolor{green!0}{\large -}{\tiny -} & \cellcolor{green!10}{\large 4}/{\footnotesize 40} & \cellcolor{green!0}{\large 0}/{\footnotesize 200} & \cellcolor{green!20}{\large 18}/{\footnotesize 136} \tabularnewline
10 & bind9\_dns\_name\_fromwire & 4 & \cellcolor{green!0}{\large 0}/{\footnotesize 40} & \cellcolor{green!0}{\large 0}/{\footnotesize 40} & \cellcolor{green!0}{\large -}{\tiny -} & \cellcolor{green!0}{\large 0}/{\footnotesize 40} & \cellcolor{green!0}{\large 0}/{\footnotesize 192} & \cellcolor{green!10}{\large 2}/{\footnotesize 186} \tabularnewline
\rowcolor{black!10} 11 & gdk-pixbuf\_gdk\_pixbuf\_animation\_new\_from\_file & 4 & \cellcolor{green!0}{\large 0}/{\footnotesize 40} & \cellcolor{green!20}{\large 8}/{\footnotesize 40} & \cellcolor{green!10}{\large 4}/{\footnotesize 40} & \cellcolor{green!10}{\large 2}/{\footnotesize 40} & \cellcolor{green!10}{\large 4}/{\footnotesize 68} & \cellcolor{green!10}{\large 3}/{\footnotesize 97} \tabularnewline
12 & gdk-pixbuf\_gdk\_pixbuf\_new\_from\_data & 4 & \cellcolor{green!0}{\large 0}/{\footnotesize 40} & \cellcolor{green!60}{\large 24}/{\footnotesize 40} & \cellcolor{green!50}{\large 20}/{\footnotesize 40} & \cellcolor{green!30}{\large 9}/{\footnotesize 40} & \cellcolor{green!40}{\large 27}/{\footnotesize 78} & \cellcolor{green!30}{\large 22}/{\footnotesize 101} \tabularnewline
\rowcolor{black!10} 13 & gdk-pixbuf\_gdk\_pixbuf\_new\_from\_file & 4 & \cellcolor{green!10}{\large 1}/{\footnotesize 40} & \cellcolor{green!20}{\large 5}/{\footnotesize 40} & \cellcolor{green!10}{\large 3}/{\footnotesize 40} & \cellcolor{green!10}{\large 3}/{\footnotesize 40} & \cellcolor{green!20}{\large 11}/{\footnotesize 75} & \cellcolor{green!10}{\large 3}/{\footnotesize 99} \tabularnewline
14 & gdk-pixbuf\_gdk\_pixbuf\_new\_from\_stream & 4 & \cellcolor{green!10}{\large 1}/{\footnotesize 40} & \cellcolor{green!60}{\large 24}/{\footnotesize 40} & \cellcolor{green!60}{\large 23}/{\footnotesize 40} & \cellcolor{green!50}{\large 19}/{\footnotesize 40} & \cellcolor{green!60}{\large 35}/{\footnotesize 62} & \cellcolor{green!30}{\large 25}/{\footnotesize 82} \tabularnewline
\rowcolor{black!10} 15 & gpac\_gf\_isom\_open\_file & 4 & \cellcolor{green!0}{\large 0}/{\footnotesize 40} & \cellcolor{green!10}{\large 2}/{\footnotesize 40} & \cellcolor{green!0}{\large -}{\tiny -} & \cellcolor{green!0}{\large 0}/{\footnotesize 40} & \cellcolor{green!0}{\large 0}/{\footnotesize 179} & \cellcolor{green!0}{\large 0}/{\footnotesize 146} \tabularnewline
16 & libbpf\_bpf\_object\_\_open\_mem & 4 & \cellcolor{green!10}{\large 1}/{\footnotesize 40} & \cellcolor{green!40}{\large 13}/{\footnotesize 40} & \cellcolor{green!30}{\large 12}/{\footnotesize 40} & \cellcolor{green!10}{\large 4}/{\footnotesize 40} & \cellcolor{green!10}{\large 14}/{\footnotesize 132} & \cellcolor{green!10}{\large 13}/{\footnotesize 128} \tabularnewline
\rowcolor{black!10} 17 & libpg\_query\_pg\_query\_parse & 4 & \cellcolor{green!0}{\large 0}/{\footnotesize 40} & \cellcolor{green!10}{\large 2}/{\footnotesize 40} & \cellcolor{green!0}{\large -}{\tiny -} & \cellcolor{green!40}{\large 15}/{\footnotesize 40} & \cellcolor{green!20}{\large 23}/{\footnotesize 117} & \cellcolor{green!30}{\large 29}/{\footnotesize 104} \tabularnewline
18 & libucl\_ucl\_parser\_add\_string & 4 & \cellcolor{green!0}{\large 0}/{\footnotesize 40} & \cellcolor{green!20}{\large 5}/{\footnotesize 40} & \cellcolor{green!30}{\large 12}/{\footnotesize 40} & \cellcolor{green!10}{\large 2}/{\footnotesize 40} & \cellcolor{green!20}{\large 21}/{\footnotesize 127} & \cellcolor{green!20}{\large 24}/{\footnotesize 148} \tabularnewline
\rowcolor{black!10} 19 & oniguruma\_onig\_new & 4 & \cellcolor{green!10}{\large 3}/{\footnotesize 40} & \cellcolor{green!20}{\large 8}/{\footnotesize 40} & \cellcolor{green!20}{\large 5}/{\footnotesize 40} & \cellcolor{green!20}{\large 6}/{\footnotesize 40} & \cellcolor{green!20}{\large 22}/{\footnotesize 131} & \cellcolor{green!10}{\large 16}/{\footnotesize 147} \tabularnewline
20 & pupnp\_ixmlLoadDocumentEx & 4 & \cellcolor{green!0}{\large 0}/{\footnotesize 40} & \cellcolor{green!10}{\large 4}/{\footnotesize 40} & \cellcolor{green!10}{\large 3}/{\footnotesize 40} & \cellcolor{green!0}{\large 0}/{\footnotesize 40} & \cellcolor{green!0}{\large 0}/{\footnotesize 110} & \cellcolor{green!0}{\large 0}/{\footnotesize 107} \tabularnewline
\rowcolor{black!10} 21 & gdk-pixbuf\_gdk\_pixbuf\_new\_from\_file\_at\_scale & 5 & \cellcolor{green!0}{\large 0}/{\footnotesize 40} & \cellcolor{green!10}{\large 1}/{\footnotesize 40} & \cellcolor{green!0}{\large 0}/{\footnotesize 40} & \cellcolor{green!10}{\large 1}/{\footnotesize 40} & \cellcolor{green!10}{\large 4}/{\footnotesize 89} & \cellcolor{green!0}{\large 0}/{\footnotesize 104} \tabularnewline
22 & inchi\_GetINCHIKeyFromINCHI & 5 & \cellcolor{green!0}{\large 0}/{\footnotesize 40} & \cellcolor{green!40}{\large 14}/{\footnotesize 40} & \cellcolor{green!20}{\large 8}/{\footnotesize 40} & \cellcolor{green!20}{\large 5}/{\footnotesize 40} & \cellcolor{green!30}{\large 26}/{\footnotesize 119} & \cellcolor{green!20}{\large 18}/{\footnotesize 135} \tabularnewline
\rowcolor{black!10} 23 & libdwarf\_dwarf\_init\_b & 5 & \cellcolor{green!0}{\large 0}/{\footnotesize 40} & \cellcolor{green!10}{\large 1}/{\footnotesize 40} & \cellcolor{green!10}{\large 2}/{\footnotesize 40} & \cellcolor{green!10}{\large 3}/{\footnotesize 40} & \cellcolor{green!10}{\large 4}/{\footnotesize 178} & \cellcolor{green!10}{\large 8}/{\footnotesize 100} \tabularnewline
24 & libdwarf\_dwarf\_init\_path & 5 & \cellcolor{green!0}{\large 0}/{\footnotesize 40} & \cellcolor{green!0}{\large 0}/{\footnotesize 40} & \cellcolor{green!0}{\large 0}/{\footnotesize 40} & \cellcolor{green!10}{\large 1}/{\footnotesize 40} & \cellcolor{green!0}{\large 0}/{\footnotesize 183} & \cellcolor{green!0}{\large 0}/{\footnotesize 133} \tabularnewline
\rowcolor{black!10} 25 & liblouis\_lou\_compileString & 5 & \cellcolor{green!0}{\large 0}/{\footnotesize 40} & \cellcolor{green!10}{\large 1}/{\footnotesize 40} & \cellcolor{green!10}{\large 3}/{\footnotesize 40} & \cellcolor{green!0}{\large 0}/{\footnotesize 40} & \cellcolor{green!10}{\large 7}/{\footnotesize 174} & \cellcolor{green!10}{\large 4}/{\footnotesize 177} \tabularnewline
26 & selinux\_cil\_compile & 5 & \cellcolor{green!0}{\large 0}/{\footnotesize 40} & \cellcolor{green!0}{\large 0}/{\footnotesize 40} & \cellcolor{green!0}{\large -}{\tiny -} & \cellcolor{green!20}{\large 8}/{\footnotesize 40} & \cellcolor{green!0}{\large 0}/{\footnotesize 193} & \cellcolor{green!20}{\large 14}/{\footnotesize 117} \tabularnewline
\rowcolor{black!10} 27 & bind9\_dns\_name\_fromtext & 6 & \cellcolor{green!0}{\large 0}/{\footnotesize 40} & \cellcolor{green!0}{\large 0}/{\footnotesize 40} & \cellcolor{green!0}{\large -}{\tiny -} & \cellcolor{green!10}{\large 1}/{\footnotesize 40} & \cellcolor{green!0}{\large 0}/{\footnotesize 195} & \cellcolor{green!10}{\large 3}/{\footnotesize 165} \tabularnewline
28 & bind9\_dns\_rdata\_fromwire & 6 & \cellcolor{green!0}{\large 0}/{\footnotesize 40} & \cellcolor{green!0}{\large 0}/{\footnotesize 40} & \cellcolor{green!0}{\large -}{\tiny -} & \cellcolor{green!0}{\large 0}/{\footnotesize 40} & \cellcolor{green!0}{\large 0}/{\footnotesize 186} & \cellcolor{green!10}{\large 2}/{\footnotesize 172} \tabularnewline
\rowcolor{black!10} 29 & coturn\_stun\_is\_binding\_response & 6 & \cellcolor{green!0}{\large 0}/{\footnotesize 40} & \cellcolor{green!0}{\large 0}/{\footnotesize 40} & \cellcolor{green!0}{\large -}{\tiny -} & \cellcolor{green!20}{\large 5}/{\footnotesize 40} & \cellcolor{green!10}{\large 2}/{\footnotesize 194} & \cellcolor{green!10}{\large 5}/{\footnotesize 119} \tabularnewline
30 & coturn\_stun\_is\_command\_message & 6 & \cellcolor{green!0}{\large 0}/{\footnotesize 40} & \cellcolor{green!0}{\large 0}/{\footnotesize 40} & \cellcolor{green!0}{\large 0}/{\footnotesize 40} & \cellcolor{green!20}{\large 7}/{\footnotesize 40} & \cellcolor{green!0}{\large 1}/{\footnotesize 191} & \cellcolor{green!20}{\large 11}/{\footnotesize 89} \tabularnewline
\rowcolor{black!10} 31 & coturn\_stun\_is\_response & 6 & \cellcolor{green!0}{\large 0}/{\footnotesize 40} & \cellcolor{green!0}{\large 0}/{\footnotesize 40} & \cellcolor{green!0}{\large -}{\tiny -} & \cellcolor{green!20}{\large 6}/{\footnotesize 40} & \cellcolor{green!0}{\large 0}/{\footnotesize 180} & \cellcolor{green!10}{\large 6}/{\footnotesize 92} \tabularnewline
32 & coturn\_stun\_is\_success\_response & 6 & \cellcolor{green!0}{\large 0}/{\footnotesize 40} & \cellcolor{green!0}{\large 0}/{\footnotesize 40} & \cellcolor{green!0}{\large -}{\tiny -} & \cellcolor{green!10}{\large 4}/{\footnotesize 40} & \cellcolor{green!10}{\large 2}/{\footnotesize 193} & \cellcolor{green!10}{\large 11}/{\footnotesize 129} \tabularnewline
\rowcolor{black!10} 33 & hiredis\_redisFormatCommand & 6 & \cellcolor{green!10}{\large 2}/{\footnotesize 40} & \cellcolor{green!60}{\large 23}/{\footnotesize 40} & \cellcolor{green!0}{\large -}{\tiny -} & \cellcolor{green!20}{\large 7}/{\footnotesize 40} & \cellcolor{green!10}{\large 6}/{\footnotesize 183} & \cellcolor{green!10}{\large 11}/{\footnotesize 143} \tabularnewline
34 & igraph\_igraph\_read\_graph\_dl & 6 & \cellcolor{green!0}{\large 0}/{\footnotesize 40} & \cellcolor{green!0}{\large 0}/{\footnotesize 40} & \cellcolor{green!0}{\large 0}/{\footnotesize 40} & \cellcolor{green!0}{\large 0}/{\footnotesize 40} & \cellcolor{green!10}{\large 9}/{\footnotesize 173} & \cellcolor{green!0}{\large 1}/{\footnotesize 187} \tabularnewline
\rowcolor{black!10} 35 & igraph\_igraph\_read\_graph\_edgelist & 6 & \cellcolor{green!0}{\large 0}/{\footnotesize 40} & \cellcolor{green!10}{\large 1}/{\footnotesize 40} & \cellcolor{green!0}{\large 0}/{\footnotesize 40} & \cellcolor{green!0}{\large 0}/{\footnotesize 40} & \cellcolor{green!0}{\large 0}/{\footnotesize 196} & \cellcolor{green!0}{\large 0}/{\footnotesize 189} \tabularnewline
36 & igraph\_igraph\_read\_graph\_gml & 6 & \cellcolor{green!0}{\large 0}/{\footnotesize 40} & \cellcolor{green!0}{\large 0}/{\footnotesize 40} & \cellcolor{green!0}{\large 0}/{\footnotesize 40} & \cellcolor{green!0}{\large 0}/{\footnotesize 40} & \cellcolor{green!10}{\large 6}/{\footnotesize 179} & \cellcolor{green!0}{\large 0}/{\footnotesize 179} \tabularnewline
\rowcolor{black!10} 37 & igraph\_igraph\_read\_graph\_graphdb & 6 & \cellcolor{green!0}{\large 0}/{\footnotesize 40} & \cellcolor{green!0}{\large 0}/{\footnotesize 40} & \cellcolor{green!0}{\large 0}/{\footnotesize 40} & \cellcolor{green!0}{\large 0}/{\footnotesize 40} & \cellcolor{green!10}{\large 2}/{\footnotesize 191} & \cellcolor{green!10}{\large 2}/{\footnotesize 172} \tabularnewline
38 & igraph\_igraph\_read\_graph\_graphml & 6 & \cellcolor{green!0}{\large 0}/{\footnotesize 40} & \cellcolor{green!0}{\large 0}/{\footnotesize 40} & \cellcolor{green!0}{\large 0}/{\footnotesize 40} & \cellcolor{green!10}{\large 3}/{\footnotesize 40} & \cellcolor{green!10}{\large 9}/{\footnotesize 179} & \cellcolor{green!10}{\large 10}/{\footnotesize 173} \tabularnewline
\rowcolor{black!10} 39 & igraph\_igraph\_read\_graph\_lgl & 6 & \cellcolor{green!0}{\large 0}/{\footnotesize 40} & \cellcolor{green!0}{\large 0}/{\footnotesize 40} & \cellcolor{green!0}{\large 0}/{\footnotesize 40} & \cellcolor{green!0}{\large 0}/{\footnotesize 40} & \cellcolor{green!0}{\large 0}/{\footnotesize 187} & \cellcolor{green!0}{\large 0}/{\footnotesize 192} \tabularnewline
40 & igraph\_igraph\_read\_graph\_pajek & 6 & \cellcolor{green!0}{\large 0}/{\footnotesize 40} & \cellcolor{green!0}{\large 0}/{\footnotesize 40} & \cellcolor{green!0}{\large 0}/{\footnotesize 40} & \cellcolor{green!0}{\large 0}/{\footnotesize 40} & \cellcolor{green!10}{\large 16}/{\footnotesize 159} & \cellcolor{green!10}{\large 5}/{\footnotesize 172} \tabularnewline
\rowcolor{black!10} 41 & inchi\_GetINCHIfromINCHI & 6 & \cellcolor{green!0}{\large 0}/{\footnotesize 40} & \cellcolor{green!0}{\large 0}/{\footnotesize 40} & \cellcolor{green!0}{\large 0}/{\footnotesize 40} & \cellcolor{green!10}{\large 2}/{\footnotesize 40} & \cellcolor{green!10}{\large 4}/{\footnotesize 193} & \cellcolor{green!10}{\large 3}/{\footnotesize 150} \tabularnewline
42 & inchi\_GetStructFromINCHI & 6 & \cellcolor{green!0}{\large 0}/{\footnotesize 40} & \cellcolor{green!0}{\large 0}/{\footnotesize 40} & \cellcolor{green!0}{\large 0}/{\footnotesize 40} & \cellcolor{green!0}{\large 0}/{\footnotesize 40} & \cellcolor{green!10}{\large 6}/{\footnotesize 190} & \cellcolor{green!10}{\large 3}/{\footnotesize 182} \tabularnewline
\rowcolor{black!10} 43 & kamailio\_parse\_msg & 6 & \cellcolor{green!0}{\large 0}/{\footnotesize 40} & \cellcolor{green!20}{\large 5}/{\footnotesize 40} & \cellcolor{green!0}{\large -}{\tiny -} & \cellcolor{green!20}{\large 5}/{\footnotesize 40} & \cellcolor{green!20}{\large 20}/{\footnotesize 130} & \cellcolor{green!20}{\large 17}/{\footnotesize 111} \tabularnewline
44 & libyang\_lys\_parse\_mem & 6 & \cellcolor{green!0}{\large 0}/{\footnotesize 40} & \cellcolor{green!0}{\large 0}/{\footnotesize 40} & \cellcolor{green!0}{\large 0}/{\footnotesize 40} & \cellcolor{green!0}{\large 0}/{\footnotesize 40} & \cellcolor{green!0}{\large 0}/{\footnotesize 195} & \cellcolor{green!10}{\large 8}/{\footnotesize 185} \tabularnewline
\rowcolor{black!10} 45 & proftpd\_pr\_json\_object\_from\_text & 6 & \cellcolor{green!0}{\large 0}/{\footnotesize 40} & \cellcolor{green!0}{\large 0}/{\footnotesize 40} & \cellcolor{green!0}{\large -}{\tiny -} & \cellcolor{green!10}{\large 2}/{\footnotesize 40} & \cellcolor{green!0}{\large 1}/{\footnotesize 194} & \cellcolor{green!10}{\large 3}/{\footnotesize 151} \tabularnewline
46 & selinux\_policydb\_read & 6 & \cellcolor{green!0}{\large 0}/{\footnotesize 40} & \cellcolor{green!10}{\large 3}/{\footnotesize 40} & \cellcolor{green!0}{\large -}{\tiny -} & \cellcolor{green!0}{\large 0}/{\footnotesize 40} & \cellcolor{green!10}{\large 13}/{\footnotesize 140} & \cellcolor{green!10}{\large 11}/{\footnotesize 158} \tabularnewline
\rowcolor{black!10} 47 & kamailio\_get\_src\_address\_socket & 7 & \cellcolor{green!0}{\large 0}/{\footnotesize 40} & \cellcolor{green!0}{\large 0}/{\footnotesize 40} & \cellcolor{green!0}{\large 0}/{\footnotesize 40} & \cellcolor{green!10}{\large 3}/{\footnotesize 40} & \cellcolor{green!0}{\large 0}/{\footnotesize 162} & \cellcolor{green!10}{\large 4}/{\footnotesize 148} \tabularnewline
48 & kamailio\_get\_src\_uri & 7 & \cellcolor{green!0}{\large 0}/{\footnotesize 40} & \cellcolor{green!0}{\large 0}/{\footnotesize 40} & \cellcolor{green!0}{\large 0}/{\footnotesize 40} & \cellcolor{green!0}{\large 0}/{\footnotesize 40} & \cellcolor{green!0}{\large 0}/{\footnotesize 171} & \cellcolor{green!10}{\large 4}/{\footnotesize 137} \tabularnewline
\rowcolor{black!10} 49 & kamailio\_parse\_content\_disposition & 7 & \cellcolor{green!0}{\large 0}/{\footnotesize 40} & \cellcolor{green!0}{\large 0}/{\footnotesize 40} & \cellcolor{green!0}{\large 0}/{\footnotesize 40} & \cellcolor{green!10}{\large 1}/{\footnotesize 40} & \cellcolor{green!0}{\large 0}/{\footnotesize 183} & \cellcolor{green!10}{\large 4}/{\footnotesize 149} \tabularnewline
50 & kamailio\_parse\_diversion\_header & 7 & \cellcolor{green!0}{\large 0}/{\footnotesize 40} & \cellcolor{green!0}{\large 0}/{\footnotesize 40} & \cellcolor{green!0}{\large 0}/{\footnotesize 40} & \cellcolor{green!0}{\large 0}/{\footnotesize 40} & \cellcolor{green!0}{\large 0}/{\footnotesize 184} & \cellcolor{green!10}{\large 8}/{\footnotesize 163} \tabularnewline
\rowcolor{black!10} 51 & kamailio\_parse\_from\_header & 7 & \cellcolor{green!0}{\large 0}/{\footnotesize 40} & \cellcolor{green!0}{\large 0}/{\footnotesize 40} & \cellcolor{green!0}{\large -}{\tiny -} & \cellcolor{green!0}{\large 0}/{\footnotesize 40} & \cellcolor{green!0}{\large 0}/{\footnotesize 189} & \cellcolor{green!0}{\large 0}/{\footnotesize 159} \tabularnewline
52 & kamailio\_parse\_from\_uri & 7 & \cellcolor{green!0}{\large 0}/{\footnotesize 40} & \cellcolor{green!0}{\large 0}/{\footnotesize 40} & \cellcolor{green!0}{\large -}{\tiny -} & \cellcolor{green!10}{\large 1}/{\footnotesize 40} & \cellcolor{green!0}{\large 0}/{\footnotesize 190} & \cellcolor{green!10}{\large 2}/{\footnotesize 137} \tabularnewline
\rowcolor{black!10} 53 & kamailio\_parse\_headers & 7 & \cellcolor{green!0}{\large 0}/{\footnotesize 40} & \cellcolor{green!0}{\large 0}/{\footnotesize 40} & \cellcolor{green!0}{\large -}{\tiny -} & \cellcolor{green!0}{\large 0}/{\footnotesize 40} & \cellcolor{green!0}{\large 0}/{\footnotesize 166} & \cellcolor{green!0}{\large 0}/{\footnotesize 153} \tabularnewline
54 & kamailio\_parse\_identityinfo\_header & 7 & \cellcolor{green!0}{\large 0}/{\footnotesize 40} & \cellcolor{green!0}{\large 0}/{\footnotesize 40} & \cellcolor{green!0}{\large -}{\tiny -} & \cellcolor{green!10}{\large 3}/{\footnotesize 40} & \cellcolor{green!0}{\large 0}/{\footnotesize 184} & \cellcolor{green!10}{\large 7}/{\footnotesize 142} \tabularnewline
\rowcolor{black!10} 55 & kamailio\_parse\_pai\_header & 7 & \cellcolor{green!0}{\large 0}/{\footnotesize 40} & \cellcolor{green!0}{\large 0}/{\footnotesize 40} & \cellcolor{green!0}{\large -}{\tiny -} & \cellcolor{green!0}{\large 0}/{\footnotesize 40} & \cellcolor{green!0}{\large 0}/{\footnotesize 188} & \cellcolor{green!10}{\large 4}/{\footnotesize 162} \tabularnewline
56 & kamailio\_parse\_privacy & 7 & \cellcolor{green!0}{\large 0}/{\footnotesize 40} & \cellcolor{green!0}{\large 0}/{\footnotesize 40} & \cellcolor{green!0}{\large 0}/{\footnotesize 40} & \cellcolor{green!10}{\large 4}/{\footnotesize 40} & \cellcolor{green!0}{\large 0}/{\footnotesize 184} & \cellcolor{green!10}{\large 5}/{\footnotesize 159} \tabularnewline
\rowcolor{black!10} 57 & kamailio\_parse\_record\_route\_headers & 7 & \cellcolor{green!0}{\large 0}/{\footnotesize 40} & \cellcolor{green!0}{\large 0}/{\footnotesize 40} & \cellcolor{green!0}{\large -}{\tiny -} & \cellcolor{green!30}{\large 11}/{\footnotesize 40} & \cellcolor{green!0}{\large 0}/{\footnotesize 180} & \cellcolor{green!10}{\large 12}/{\footnotesize 134} \tabularnewline
58 & kamailio\_parse\_refer\_to\_header & 7 & \cellcolor{green!0}{\large 0}/{\footnotesize 40} & \cellcolor{green!0}{\large 0}/{\footnotesize 40} & \cellcolor{green!0}{\large -}{\tiny -} & \cellcolor{green!10}{\large 2}/{\footnotesize 40} & \cellcolor{green!0}{\large 0}/{\footnotesize 169} & \cellcolor{green!0}{\large 1}/{\footnotesize 147} \tabularnewline
\rowcolor{black!10} 59 & kamailio\_parse\_route\_headers & 7 & \cellcolor{green!0}{\large 0}/{\footnotesize 40} & \cellcolor{green!0}{\large 0}/{\footnotesize 40} & \cellcolor{green!0}{\large -}{\tiny -} & \cellcolor{green!20}{\large 5}/{\footnotesize 40} & \cellcolor{green!0}{\large 0}/{\footnotesize 186} & \cellcolor{green!10}{\large 12}/{\footnotesize 143} \tabularnewline
60 & kamailio\_parse\_to\_header & 7 & \cellcolor{green!0}{\large 0}/{\footnotesize 40} & \cellcolor{green!0}{\large 0}/{\footnotesize 40} & \cellcolor{green!0}{\large -}{\tiny -} & \cellcolor{green!0}{\large 0}/{\footnotesize 40} & \cellcolor{green!0}{\large 0}/{\footnotesize 164} & \cellcolor{green!10}{\large 2}/{\footnotesize 153} \tabularnewline
\rowcolor{black!10} 61 & kamailio\_parse\_to\_uri & 7 & \cellcolor{green!0}{\large 0}/{\footnotesize 40} & \cellcolor{green!0}{\large 0}/{\footnotesize 40} & \cellcolor{green!0}{\large -}{\tiny -} & \cellcolor{green!10}{\large 2}/{\footnotesize 40} & \cellcolor{green!0}{\large 0}/{\footnotesize 176} & \cellcolor{green!0}{\large 1}/{\footnotesize 147} \tabularnewline
62 & libyang\_lyd\_parse\_data\_mem & 7 & \cellcolor{green!0}{\large 0}/{\footnotesize 40} & \cellcolor{green!0}{\large 0}/{\footnotesize 40} & \cellcolor{green!0}{\large 0}/{\footnotesize 40} & \cellcolor{green!0}{\large 0}/{\footnotesize 40} & \cellcolor{green!10}{\large 4}/{\footnotesize 182} & \cellcolor{green!10}{\large 7}/{\footnotesize 176} \tabularnewline
\rowcolor{black!10} 63 & bind9\_dns\_message\_parse & 8 & \cellcolor{green!0}{\large 0}/{\footnotesize 40} & \cellcolor{green!0}{\large 0}/{\footnotesize 40} & \cellcolor{green!0}{\large -}{\tiny -} & \cellcolor{green!0}{\large 0}/{\footnotesize 40} & \cellcolor{green!0}{\large 0}/{\footnotesize 189} & \cellcolor{green!0}{\large 1}/{\footnotesize 160} \tabularnewline
64 & igraph\_igraph\_read\_graph\_ncol & 8 & \cellcolor{green!0}{\large 0}/{\footnotesize 40} & \cellcolor{green!10}{\large 1}/{\footnotesize 40} & \cellcolor{green!0}{\large 0}/{\footnotesize 40} & \cellcolor{green!0}{\large 0}/{\footnotesize 40} & \cellcolor{green!0}{\large 1}/{\footnotesize 198} & \cellcolor{green!10}{\large 2}/{\footnotesize 185} \tabularnewline
\rowcolor{black!10} 65 & pjsip\_pj\_json\_parse & 8 & \cellcolor{green!0}{\large 0}/{\footnotesize 40} & \cellcolor{green!0}{\large 0}/{\footnotesize 40} & \cellcolor{green!0}{\large 0}/{\footnotesize 40} & \cellcolor{green!0}{\large 0}/{\footnotesize 40} & \cellcolor{green!10}{\large 2}/{\footnotesize 193} & \cellcolor{green!10}{\large 5}/{\footnotesize 190} \tabularnewline
66 & pjsip\_pj\_xml\_parse & 8 & \cellcolor{green!0}{\large 0}/{\footnotesize 40} & \cellcolor{green!10}{\large 1}/{\footnotesize 40} & \cellcolor{green!10}{\large 3}/{\footnotesize 40} & \cellcolor{green!0}{\large 0}/{\footnotesize 40} & \cellcolor{green!10}{\large 3}/{\footnotesize 193} & \cellcolor{green!10}{\large 6}/{\footnotesize 173} \tabularnewline
\rowcolor{black!10} 67 & pjsip\_pjmedia\_sdp\_parse & 8 & \cellcolor{green!0}{\large 0}/{\footnotesize 40} & \cellcolor{green!10}{\large 1}/{\footnotesize 40} & \cellcolor{green!10}{\large 2}/{\footnotesize 40} & \cellcolor{green!0}{\large 0}/{\footnotesize 40} & \cellcolor{green!10}{\large 3}/{\footnotesize 183} & \cellcolor{green!0}{\large 1}/{\footnotesize 158} \tabularnewline
68 & quickjs\_lre\_compile & 8 & \cellcolor{green!0}{\large 0}/{\footnotesize 40} & \cellcolor{green!0}{\large 0}/{\footnotesize 40} & \cellcolor{green!0}{\large -}{\tiny -} & \cellcolor{green!0}{\large 0}/{\footnotesize 40} & \cellcolor{green!0}{\large 0}/{\footnotesize 187} & \cellcolor{green!0}{\large 1}/{\footnotesize 177} \tabularnewline
\rowcolor{black!10} 69 & bind9\_isc\_lex\_getmastertoken & 9 & \cellcolor{green!0}{\large 0}/{\footnotesize 40} & \cellcolor{green!0}{\large 0}/{\footnotesize 40} & \cellcolor{green!0}{\large -}{\tiny -} & \cellcolor{green!0}{\large 0}/{\footnotesize 40} & \cellcolor{green!0}{\large 0}/{\footnotesize 187} & \cellcolor{green!0}{\large 0}/{\footnotesize 173} \tabularnewline
70 & bind9\_isc\_lex\_gettoken & 9 & \cellcolor{green!0}{\large 0}/{\footnotesize 40} & \cellcolor{green!0}{\large 0}/{\footnotesize 40} & \cellcolor{green!0}{\large -}{\tiny -} & \cellcolor{green!0}{\large 0}/{\footnotesize 40} & \cellcolor{green!0}{\large 0}/{\footnotesize 178} & \cellcolor{green!0}{\large 1}/{\footnotesize 167} \tabularnewline
\rowcolor{black!10} 71 & quickjs\_JS\_Eval & 9 & \cellcolor{green!0}{\large 0}/{\footnotesize 40} & \cellcolor{green!10}{\large 4}/{\footnotesize 40} & \cellcolor{green!0}{\large -}{\tiny -} & \cellcolor{green!10}{\large 4}/{\footnotesize 40} & \cellcolor{green!10}{\large 15}/{\footnotesize 149} & \cellcolor{green!10}{\large 9}/{\footnotesize 169} \tabularnewline
72 & igraph\_igraph\_edge\_connectivity & 10 & \cellcolor{green!0}{\large 0}/{\footnotesize 40} & \cellcolor{green!0}{\large 0}/{\footnotesize 40} & \cellcolor{green!0}{\large 0}/{\footnotesize 40} & \cellcolor{green!0}{\large 0}/{\footnotesize 40} & \cellcolor{green!0}{\large 0}/{\footnotesize 173} & \cellcolor{green!0}{\large 0}/{\footnotesize 174} \tabularnewline
\rowcolor{black!10} 73 & pjsip\_pj\_stun\_msg\_decode & 10 & \cellcolor{green!0}{\large 0}/{\footnotesize 40} & \cellcolor{green!10}{\large 1}/{\footnotesize 40} & \cellcolor{green!0}{\large 0}/{\footnotesize 40} & \cellcolor{green!0}{\large 0}/{\footnotesize 40} & \cellcolor{green!0}{\large 0}/{\footnotesize 180} & \cellcolor{green!0}{\large 0}/{\footnotesize 167} \tabularnewline
74 & bind9\_dns\_message\_checksig & 11 & \cellcolor{green!0}{\large 0}/{\footnotesize 40} & \cellcolor{green!0}{\large 0}/{\footnotesize 40} & \cellcolor{green!0}{\large -}{\tiny -} & \cellcolor{green!0}{\large 0}/{\footnotesize 40} & \cellcolor{green!0}{\large 0}/{\footnotesize 200} & \cellcolor{green!0}{\large 0}/{\footnotesize 140} \tabularnewline
\rowcolor{black!10} 75 & libzip\_zip\_fread & 11 & \cellcolor{green!0}{\large 0}/{\footnotesize 40} & \cellcolor{green!10}{\large 2}/{\footnotesize 40} & \cellcolor{green!0}{\large 0}/{\footnotesize 40} & \cellcolor{green!10}{\large 1}/{\footnotesize 40} & \cellcolor{green!10}{\large 11}/{\footnotesize 176} & \cellcolor{green!10}{\large 9}/{\footnotesize 158} \tabularnewline
76 & bind9\_dns\_rdata\_fromtext & 12 & \cellcolor{green!0}{\large 0}/{\footnotesize 40} & \cellcolor{green!0}{\large 0}/{\footnotesize 40} & \cellcolor{green!0}{\large -}{\tiny -} & \cellcolor{green!0}{\large 0}/{\footnotesize 40} & \cellcolor{green!0}{\large 0}/{\footnotesize 179} & \cellcolor{green!0}{\large 0}/{\footnotesize 127} \tabularnewline
\rowcolor{black!10} 77 & igraph\_igraph\_all\_minimal\_st\_separators & 12 & \cellcolor{green!10}{\large 1}/{\footnotesize 40} & \cellcolor{green!0}{\large 0}/{\footnotesize 40} & \cellcolor{green!0}{\large 0}/{\footnotesize 40} & \cellcolor{green!0}{\large 0}/{\footnotesize 40} & \cellcolor{green!10}{\large 3}/{\footnotesize 189} & \cellcolor{green!10}{\large 3}/{\footnotesize 172} \tabularnewline
78 & igraph\_igraph\_minimum\_size\_separators & 12 & \cellcolor{green!0}{\large 0}/{\footnotesize 40} & \cellcolor{green!0}{\large 0}/{\footnotesize 40} & \cellcolor{green!10}{\large 1}/{\footnotesize 40} & \cellcolor{green!0}{\large 0}/{\footnotesize 40} & \cellcolor{green!10}{\large 5}/{\footnotesize 182} & \cellcolor{green!10}{\large 4}/{\footnotesize 155} \tabularnewline
\rowcolor{black!10} 79 & pjsip\_pjsip\_parse\_msg & 12 & \cellcolor{green!0}{\large 0}/{\footnotesize 40} & \cellcolor{green!0}{\large 0}/{\footnotesize 40} & \cellcolor{green!0}{\large 0}/{\footnotesize 40} & \cellcolor{green!0}{\large 0}/{\footnotesize 40} & \cellcolor{green!0}{\large 0}/{\footnotesize 186} & \cellcolor{green!0}{\large 0}/{\footnotesize 176} \tabularnewline
80 & igraph\_igraph\_automorphism\_group & 13 & \cellcolor{green!0}{\large 0}/{\footnotesize 40} & \cellcolor{green!0}{\large 0}/{\footnotesize 40} & \cellcolor{green!0}{\large 0}/{\footnotesize 40} & \cellcolor{green!10}{\large 1}/{\footnotesize 40} & \cellcolor{green!0}{\large 0}/{\footnotesize 185} & \cellcolor{green!10}{\large 2}/{\footnotesize 155} \tabularnewline
\rowcolor{black!10} 81 & libmodbus\_modbus\_read\_bits & 15 & \cellcolor{green!0}{\large 0}/{\footnotesize 40} & \cellcolor{green!0}{\large 0}/{\footnotesize 40} & \cellcolor{green!0}{\large 0}/{\footnotesize 40} & \cellcolor{green!0}{\large 0}/{\footnotesize 40} & \cellcolor{green!0}{\large 0}/{\footnotesize 129} & \cellcolor{green!0}{\large 0}/{\footnotesize 115} \tabularnewline
82 & libmodbus\_modbus\_read\_registers & 15 & \cellcolor{green!0}{\large 0}/{\footnotesize 40} & \cellcolor{green!0}{\large 0}/{\footnotesize 40} & \cellcolor{green!0}{\large 0}/{\footnotesize 40} & \cellcolor{green!0}{\large 0}/{\footnotesize 40} & \cellcolor{green!0}{\large 0}/{\footnotesize 147} & \cellcolor{green!0}{\large 0}/{\footnotesize 117} \tabularnewline
\rowcolor{black!10} 83 & civetweb\_mg\_get\_response & 17 & \cellcolor{green!0}{\large 0}/{\footnotesize 40} & \cellcolor{green!0}{\large 0}/{\footnotesize 40} & \cellcolor{green!0}{\large 0}/{\footnotesize 40} & \cellcolor{green!0}{\large 0}/{\footnotesize 40} & \cellcolor{green!0}{\large 0}/{\footnotesize 195} & \cellcolor{green!0}{\large 0}/{\footnotesize 142} \tabularnewline
84 & bind9\_dns\_master\_loadbuffer & 20 & \cellcolor{green!0}{\large 0}/{\footnotesize 40} & \cellcolor{green!0}{\large 0}/{\footnotesize 40} & \cellcolor{green!0}{\large -}{\tiny -} & \cellcolor{green!0}{\large 0}/{\footnotesize 40} & \cellcolor{green!0}{\large 0}/{\footnotesize 197} & \cellcolor{green!0}{\large 0}/{\footnotesize 181} \tabularnewline
\rowcolor{black!10} 85 & libmodbus\_modbus\_receive & 33 & \cellcolor{green!0}{\large 0}/{\footnotesize 40} & \cellcolor{green!0}{\large 0}/{\footnotesize 40} & \cellcolor{green!0}{\large 0}/{\footnotesize 40} & \cellcolor{green!0}{\large 0}/{\footnotesize 40} & \cellcolor{green!0}{\large 0}/{\footnotesize 167} & \cellcolor{green!0}{\large 0}/{\footnotesize 143} \tabularnewline
86 & tmux\_input\_parse\_buffer & 42 & \cellcolor{green!0}{\large 0}/{\footnotesize 40} & \cellcolor{green!0}{\large 0}/{\footnotesize 40} & \cellcolor{green!0}{\large -}{\tiny -} & \cellcolor{green!0}{\large 0}/{\footnotesize 40} & \cellcolor{green!0}{\large 0}/{\footnotesize 183} & \cellcolor{green!0}{\large 0}/{\footnotesize 185} \tabularnewline

\bottomrule
%\end{tabular}
%}
%\end{table*}
\end{xltabular}
}
\twocolumn



% model: gpt-3.5-turbo-0613, temp: 1.5

\onecolumn
{\small %
\begin{xltabular}[h]{\textwidth}{ccccccccc}
%\begin{table*}[!t]
%\centering
\caption{Evaluation Result of model gpt-3.5-turbo-0613 with temperature 1.5.} \\
%\resizebox{1.0\linewidth}{!}{
%\begin{tabular}{cccccccccc}
\toprule
Index & Question & Score & NAIVE-40 & BACTX-40 & DOCTX-40 & UGCTX-40 & BA-ITER-40 & ALL-ITER-40 \tabularnewline
\midrule
\rowcolor{black!10} 1 & coturn\_stun\_is\_command\_message\_full\_check\_str & 1 & \cellcolor{green!0}{\large 0}/{\footnotesize 40} & \cellcolor{green!50}{\large 20}/{\footnotesize 40} & \cellcolor{green!0}{\large -}{\tiny -} & \cellcolor{green!20}{\large 7}/{\footnotesize 40} & \cellcolor{green!30}{\large 18}/{\footnotesize 66} & \cellcolor{green!20}{\large 9}/{\footnotesize 61} \tabularnewline
2 & kamailio\_parse\_uri & 1 & \cellcolor{green!0}{\large 0}/{\footnotesize 40} & \cellcolor{green!40}{\large 16}/{\footnotesize 40} & \cellcolor{green!0}{\large -}{\tiny -} & \cellcolor{green!10}{\large 4}/{\footnotesize 40} & \cellcolor{green!60}{\large 31}/{\footnotesize 58} & \cellcolor{green!10}{\large 6}/{\footnotesize 78} \tabularnewline
\rowcolor{black!10} 3 & coturn\_stun\_check\_message\_integrity\_str & 2 & \cellcolor{green!0}{\large 0}/{\footnotesize 40} & \cellcolor{green!10}{\large 2}/{\footnotesize 40} & \cellcolor{green!0}{\large -}{\tiny -} & \cellcolor{green!0}{\large 0}/{\footnotesize 40} & \cellcolor{green!0}{\large 0}/{\footnotesize 90} & \cellcolor{green!0}{\large 0}/{\footnotesize 68} \tabularnewline
4 & libiec61850\_MmsValue\_decodeMmsData & 2 & \cellcolor{green!0}{\large 0}/{\footnotesize 40} & \cellcolor{green!30}{\large 9}/{\footnotesize 40} & \cellcolor{green!20}{\large 5}/{\footnotesize 40} & \cellcolor{green!10}{\large 4}/{\footnotesize 40} & \cellcolor{green!20}{\large 18}/{\footnotesize 91} & \cellcolor{green!10}{\large 3}/{\footnotesize 69} \tabularnewline
\rowcolor{black!10} 5 & md4c\_md\_html & 2 & \cellcolor{green!0}{\large 0}/{\footnotesize 40} & \cellcolor{green!0}{\large 0}/{\footnotesize 40} & \cellcolor{green!0}{\large 0}/{\footnotesize 40} & \cellcolor{green!0}{\large 0}/{\footnotesize 40} & \cellcolor{green!10}{\large 12}/{\footnotesize 116} & \cellcolor{green!10}{\large 5}/{\footnotesize 90} \tabularnewline
6 & spdk\_spdk\_json\_parse & 2 & \cellcolor{green!0}{\large 0}/{\footnotesize 40} & \cellcolor{green!60}{\large 23}/{\footnotesize 40} & \cellcolor{green!0}{\large -}{\tiny -} & \cellcolor{green!10}{\large 2}/{\footnotesize 40} & \cellcolor{green!40}{\large 23}/{\footnotesize 66} & \cellcolor{green!10}{\large 7}/{\footnotesize 75} \tabularnewline
\rowcolor{black!10} 7 & croaring\_roaring\_bitmap\_portable\_deserialize\_safe & 3 & \cellcolor{green!0}{\large 0}/{\footnotesize 40} & \cellcolor{green!20}{\large 8}/{\footnotesize 40} & \cellcolor{green!30}{\large 11}/{\footnotesize 40} & \cellcolor{green!20}{\large 7}/{\footnotesize 40} & \cellcolor{green!30}{\large 21}/{\footnotesize 82} & \cellcolor{green!20}{\large 10}/{\footnotesize 74} \tabularnewline
8 & lua\_luaL\_loadbufferx & 3 & \cellcolor{green!10}{\large 2}/{\footnotesize 40} & \cellcolor{green!60}{\large 22}/{\footnotesize 40} & \cellcolor{green!60}{\large 21}/{\footnotesize 40} & \cellcolor{green!30}{\large 11}/{\footnotesize 40} & \cellcolor{green!40}{\large 24}/{\footnotesize 62} & \cellcolor{green!20}{\large 13}/{\footnotesize 82} \tabularnewline
\rowcolor{black!10} 9 & w3m\_wc\_Str\_conv\_with\_detect & 3 & \cellcolor{green!0}{\large 0}/{\footnotesize 40} & \cellcolor{green!0}{\large 0}/{\footnotesize 40} & \cellcolor{green!0}{\large -}{\tiny -} & \cellcolor{green!0}{\large 0}/{\footnotesize 40} & \cellcolor{green!0}{\large 0}/{\footnotesize 100} & \cellcolor{green!0}{\large 0}/{\footnotesize 71} \tabularnewline
10 & bind9\_dns\_name\_fromwire & 4 & \cellcolor{green!0}{\large 0}/{\footnotesize 40} & \cellcolor{green!0}{\large 0}/{\footnotesize 40} & \cellcolor{green!0}{\large -}{\tiny -} & \cellcolor{green!0}{\large 0}/{\footnotesize 40} & \cellcolor{green!0}{\large 0}/{\footnotesize 79} & \cellcolor{green!10}{\large 1}/{\footnotesize 60} \tabularnewline
\rowcolor{black!10} 11 & gdk-pixbuf\_gdk\_pixbuf\_animation\_new\_from\_file & 4 & \cellcolor{green!10}{\large 2}/{\footnotesize 40} & \cellcolor{green!20}{\large 6}/{\footnotesize 40} & \cellcolor{green!10}{\large 3}/{\footnotesize 40} & \cellcolor{green!0}{\large 0}/{\footnotesize 40} & \cellcolor{green!10}{\large 2}/{\footnotesize 78} & \cellcolor{green!0}{\large 0}/{\footnotesize 68} \tabularnewline
12 & gdk-pixbuf\_gdk\_pixbuf\_new\_from\_data & 4 & \cellcolor{green!10}{\large 1}/{\footnotesize 40} & \cellcolor{green!20}{\large 8}/{\footnotesize 40} & \cellcolor{green!20}{\large 6}/{\footnotesize 40} & \cellcolor{green!20}{\large 5}/{\footnotesize 40} & \cellcolor{green!30}{\large 17}/{\footnotesize 73} & \cellcolor{green!10}{\large 4}/{\footnotesize 70} \tabularnewline
\rowcolor{black!10} 13 & gdk-pixbuf\_gdk\_pixbuf\_new\_from\_file & 4 & \cellcolor{green!10}{\large 1}/{\footnotesize 40} & \cellcolor{green!10}{\large 3}/{\footnotesize 40} & \cellcolor{green!0}{\large 0}/{\footnotesize 40} & \cellcolor{green!0}{\large 0}/{\footnotesize 40} & \cellcolor{green!0}{\large 0}/{\footnotesize 79} & \cellcolor{green!0}{\large 0}/{\footnotesize 65} \tabularnewline
14 & gdk-pixbuf\_gdk\_pixbuf\_new\_from\_stream & 4 & \cellcolor{green!10}{\large 2}/{\footnotesize 40} & \cellcolor{green!30}{\large 10}/{\footnotesize 40} & \cellcolor{green!40}{\large 14}/{\footnotesize 40} & \cellcolor{green!20}{\large 6}/{\footnotesize 40} & \cellcolor{green!30}{\large 19}/{\footnotesize 75} & \cellcolor{green!20}{\large 9}/{\footnotesize 53} \tabularnewline
\rowcolor{black!10} 15 & gpac\_gf\_isom\_open\_file & 4 & \cellcolor{green!0}{\large 0}/{\footnotesize 40} & \cellcolor{green!0}{\large 0}/{\footnotesize 40} & \cellcolor{green!0}{\large -}{\tiny -} & \cellcolor{green!0}{\large 0}/{\footnotesize 40} & \cellcolor{green!0}{\large 0}/{\footnotesize 109} & \cellcolor{green!0}{\large 0}/{\footnotesize 91} \tabularnewline
16 & libbpf\_bpf\_object\_\_open\_mem & 4 & \cellcolor{green!0}{\large 0}/{\footnotesize 40} & \cellcolor{green!20}{\large 7}/{\footnotesize 40} & \cellcolor{green!10}{\large 4}/{\footnotesize 40} & \cellcolor{green!20}{\large 6}/{\footnotesize 40} & \cellcolor{green!10}{\large 6}/{\footnotesize 105} & \cellcolor{green!10}{\large 7}/{\footnotesize 82} \tabularnewline
\rowcolor{black!10} 17 & libpg\_query\_pg\_query\_parse & 4 & \cellcolor{green!0}{\large 0}/{\footnotesize 40} & \cellcolor{green!10}{\large 2}/{\footnotesize 40} & \cellcolor{green!0}{\large -}{\tiny -} & \cellcolor{green!20}{\large 5}/{\footnotesize 40} & \cellcolor{green!10}{\large 8}/{\footnotesize 109} & \cellcolor{green!20}{\large 11}/{\footnotesize 88} \tabularnewline
18 & libucl\_ucl\_parser\_add\_string & 4 & \cellcolor{green!0}{\large 0}/{\footnotesize 40} & \cellcolor{green!10}{\large 4}/{\footnotesize 40} & \cellcolor{green!10}{\large 3}/{\footnotesize 40} & \cellcolor{green!10}{\large 3}/{\footnotesize 40} & \cellcolor{green!10}{\large 7}/{\footnotesize 138} & \cellcolor{green!10}{\large 12}/{\footnotesize 111} \tabularnewline
\rowcolor{black!10} 19 & oniguruma\_onig\_new & 4 & \cellcolor{green!0}{\large 0}/{\footnotesize 40} & \cellcolor{green!20}{\large 6}/{\footnotesize 40} & \cellcolor{green!10}{\large 2}/{\footnotesize 40} & \cellcolor{green!10}{\large 1}/{\footnotesize 40} & \cellcolor{green!10}{\large 4}/{\footnotesize 102} & \cellcolor{green!10}{\large 6}/{\footnotesize 95} \tabularnewline
20 & pupnp\_ixmlLoadDocumentEx & 4 & \cellcolor{green!0}{\large 0}/{\footnotesize 40} & \cellcolor{green!10}{\large 3}/{\footnotesize 40} & \cellcolor{green!10}{\large 3}/{\footnotesize 40} & \cellcolor{green!10}{\large 1}/{\footnotesize 40} & \cellcolor{green!10}{\large 1}/{\footnotesize 84} & \cellcolor{green!0}{\large 0}/{\footnotesize 108} \tabularnewline
\rowcolor{black!10} 21 & gdk-pixbuf\_gdk\_pixbuf\_new\_from\_file\_at\_scale & 5 & \cellcolor{green!10}{\large 1}/{\footnotesize 40} & \cellcolor{green!10}{\large 1}/{\footnotesize 40} & \cellcolor{green!0}{\large 0}/{\footnotesize 40} & \cellcolor{green!10}{\large 1}/{\footnotesize 40} & \cellcolor{green!10}{\large 2}/{\footnotesize 89} & \cellcolor{green!0}{\large 0}/{\footnotesize 77} \tabularnewline
22 & inchi\_GetINCHIKeyFromINCHI & 5 & \cellcolor{green!0}{\large 0}/{\footnotesize 40} & \cellcolor{green!20}{\large 5}/{\footnotesize 40} & \cellcolor{green!20}{\large 5}/{\footnotesize 40} & \cellcolor{green!10}{\large 1}/{\footnotesize 40} & \cellcolor{green!20}{\large 14}/{\footnotesize 108} & \cellcolor{green!10}{\large 8}/{\footnotesize 77} \tabularnewline
\rowcolor{black!10} 23 & libdwarf\_dwarf\_init\_b & 5 & \cellcolor{green!0}{\large 0}/{\footnotesize 40} & \cellcolor{green!10}{\large 2}/{\footnotesize 40} & \cellcolor{green!10}{\large 1}/{\footnotesize 40} & \cellcolor{green!0}{\large 0}/{\footnotesize 40} & \cellcolor{green!0}{\large 0}/{\footnotesize 105} & \cellcolor{green!10}{\large 5}/{\footnotesize 67} \tabularnewline
24 & libdwarf\_dwarf\_init\_path & 5 & \cellcolor{green!0}{\large 0}/{\footnotesize 40} & \cellcolor{green!0}{\large 0}/{\footnotesize 40} & \cellcolor{green!0}{\large 0}/{\footnotesize 40} & \cellcolor{green!0}{\large 0}/{\footnotesize 40} & \cellcolor{green!0}{\large 0}/{\footnotesize 121} & \cellcolor{green!10}{\large 1}/{\footnotesize 75} \tabularnewline
\rowcolor{black!10} 25 & liblouis\_lou\_compileString & 5 & \cellcolor{green!0}{\large 0}/{\footnotesize 40} & \cellcolor{green!10}{\large 3}/{\footnotesize 40} & \cellcolor{green!20}{\large 5}/{\footnotesize 40} & \cellcolor{green!10}{\large 3}/{\footnotesize 40} & \cellcolor{green!10}{\large 7}/{\footnotesize 131} & \cellcolor{green!10}{\large 7}/{\footnotesize 99} \tabularnewline
26 & selinux\_cil\_compile & 5 & \cellcolor{green!0}{\large 0}/{\footnotesize 40} & \cellcolor{green!0}{\large 0}/{\footnotesize 40} & \cellcolor{green!0}{\large -}{\tiny -} & \cellcolor{green!30}{\large 9}/{\footnotesize 40} & \cellcolor{green!0}{\large 0}/{\footnotesize 107} & \cellcolor{green!10}{\large 6}/{\footnotesize 68} \tabularnewline
\rowcolor{black!10} 27 & bind9\_dns\_name\_fromtext & 6 & \cellcolor{green!0}{\large 0}/{\footnotesize 40} & \cellcolor{green!10}{\large 1}/{\footnotesize 40} & \cellcolor{green!0}{\large -}{\tiny -} & \cellcolor{green!10}{\large 1}/{\footnotesize 40} & \cellcolor{green!0}{\large 0}/{\footnotesize 92} & \cellcolor{green!0}{\large 0}/{\footnotesize 81} \tabularnewline
28 & bind9\_dns\_rdata\_fromwire & 6 & \cellcolor{green!0}{\large 0}/{\footnotesize 40} & \cellcolor{green!0}{\large 0}/{\footnotesize 40} & \cellcolor{green!0}{\large -}{\tiny -} & \cellcolor{green!0}{\large 0}/{\footnotesize 40} & \cellcolor{green!0}{\large 0}/{\footnotesize 67} & \cellcolor{green!0}{\large 0}/{\footnotesize 71} \tabularnewline
\rowcolor{black!10} 29 & coturn\_stun\_is\_binding\_response & 6 & \cellcolor{green!0}{\large 0}/{\footnotesize 40} & \cellcolor{green!0}{\large 0}/{\footnotesize 40} & \cellcolor{green!0}{\large -}{\tiny -} & \cellcolor{green!10}{\large 3}/{\footnotesize 40} & \cellcolor{green!0}{\large 0}/{\footnotesize 131} & \cellcolor{green!10}{\large 4}/{\footnotesize 68} \tabularnewline
30 & coturn\_stun\_is\_command\_message & 6 & \cellcolor{green!0}{\large 0}/{\footnotesize 40} & \cellcolor{green!0}{\large 0}/{\footnotesize 40} & \cellcolor{green!0}{\large 0}/{\footnotesize 40} & \cellcolor{green!30}{\large 11}/{\footnotesize 40} & \cellcolor{green!10}{\large 3}/{\footnotesize 138} & \cellcolor{green!30}{\large 12}/{\footnotesize 56} \tabularnewline
\rowcolor{black!10} 31 & coturn\_stun\_is\_response & 6 & \cellcolor{green!0}{\large 0}/{\footnotesize 40} & \cellcolor{green!0}{\large 0}/{\footnotesize 40} & \cellcolor{green!0}{\large -}{\tiny -} & \cellcolor{green!20}{\large 7}/{\footnotesize 40} & \cellcolor{green!0}{\large 0}/{\footnotesize 131} & \cellcolor{green!10}{\large 4}/{\footnotesize 67} \tabularnewline
32 & coturn\_stun\_is\_success\_response & 6 & \cellcolor{green!0}{\large 0}/{\footnotesize 40} & \cellcolor{green!0}{\large 0}/{\footnotesize 40} & \cellcolor{green!0}{\large -}{\tiny -} & \cellcolor{green!20}{\large 5}/{\footnotesize 40} & \cellcolor{green!0}{\large 0}/{\footnotesize 137} & \cellcolor{green!10}{\large 3}/{\footnotesize 53} \tabularnewline
\rowcolor{black!10} 33 & hiredis\_redisFormatCommand & 6 & \cellcolor{green!10}{\large 1}/{\footnotesize 40} & \cellcolor{green!30}{\large 11}/{\footnotesize 40} & \cellcolor{green!0}{\large -}{\tiny -} & \cellcolor{green!20}{\large 5}/{\footnotesize 40} & \cellcolor{green!10}{\large 3}/{\footnotesize 96} & \cellcolor{green!10}{\large 4}/{\footnotesize 77} \tabularnewline
34 & igraph\_igraph\_read\_graph\_dl & 6 & \cellcolor{green!0}{\large 0}/{\footnotesize 40} & \cellcolor{green!0}{\large 0}/{\footnotesize 40} & \cellcolor{green!0}{\large 0}/{\footnotesize 40} & \cellcolor{green!0}{\large 0}/{\footnotesize 40} & \cellcolor{green!0}{\large 1}/{\footnotesize 115} & \cellcolor{green!0}{\large 0}/{\footnotesize 87} \tabularnewline
\rowcolor{black!10} 35 & igraph\_igraph\_read\_graph\_edgelist & 6 & \cellcolor{green!0}{\large 0}/{\footnotesize 40} & \cellcolor{green!0}{\large 0}/{\footnotesize 40} & \cellcolor{green!0}{\large 0}/{\footnotesize 40} & \cellcolor{green!0}{\large 0}/{\footnotesize 40} & \cellcolor{green!0}{\large 0}/{\footnotesize 100} & \cellcolor{green!0}{\large 0}/{\footnotesize 83} \tabularnewline
36 & igraph\_igraph\_read\_graph\_gml & 6 & \cellcolor{green!0}{\large 0}/{\footnotesize 40} & \cellcolor{green!0}{\large 0}/{\footnotesize 40} & \cellcolor{green!0}{\large 0}/{\footnotesize 40} & \cellcolor{green!0}{\large 0}/{\footnotesize 40} & \cellcolor{green!0}{\large 0}/{\footnotesize 112} & \cellcolor{green!0}{\large 0}/{\footnotesize 92} \tabularnewline
\rowcolor{black!10} 37 & igraph\_igraph\_read\_graph\_graphdb & 6 & \cellcolor{green!0}{\large 0}/{\footnotesize 40} & \cellcolor{green!0}{\large 0}/{\footnotesize 40} & \cellcolor{green!0}{\large 0}/{\footnotesize 40} & \cellcolor{green!0}{\large 0}/{\footnotesize 40} & \cellcolor{green!0}{\large 0}/{\footnotesize 104} & \cellcolor{green!0}{\large 0}/{\footnotesize 71} \tabularnewline
38 & igraph\_igraph\_read\_graph\_graphml & 6 & \cellcolor{green!0}{\large 0}/{\footnotesize 40} & \cellcolor{green!0}{\large 0}/{\footnotesize 40} & \cellcolor{green!0}{\large 0}/{\footnotesize 40} & \cellcolor{green!10}{\large 1}/{\footnotesize 40} & \cellcolor{green!0}{\large 0}/{\footnotesize 111} & \cellcolor{green!10}{\large 3}/{\footnotesize 78} \tabularnewline
\rowcolor{black!10} 39 & igraph\_igraph\_read\_graph\_lgl & 6 & \cellcolor{green!0}{\large 0}/{\footnotesize 40} & \cellcolor{green!0}{\large 0}/{\footnotesize 40} & \cellcolor{green!0}{\large 0}/{\footnotesize 40} & \cellcolor{green!0}{\large 0}/{\footnotesize 40} & \cellcolor{green!0}{\large 0}/{\footnotesize 109} & \cellcolor{green!0}{\large 0}/{\footnotesize 77} \tabularnewline
40 & igraph\_igraph\_read\_graph\_pajek & 6 & \cellcolor{green!0}{\large 0}/{\footnotesize 40} & \cellcolor{green!0}{\large 0}/{\footnotesize 40} & \cellcolor{green!0}{\large 0}/{\footnotesize 40} & \cellcolor{green!0}{\large 0}/{\footnotesize 40} & \cellcolor{green!0}{\large 0}/{\footnotesize 101} & \cellcolor{green!0}{\large 0}/{\footnotesize 87} \tabularnewline
\rowcolor{black!10} 41 & inchi\_GetINCHIfromINCHI & 6 & \cellcolor{green!0}{\large 0}/{\footnotesize 40} & \cellcolor{green!0}{\large 0}/{\footnotesize 40} & \cellcolor{green!0}{\large 0}/{\footnotesize 40} & \cellcolor{green!0}{\large 0}/{\footnotesize 40} & \cellcolor{green!0}{\large 1}/{\footnotesize 119} & \cellcolor{green!10}{\large 2}/{\footnotesize 91} \tabularnewline
42 & inchi\_GetStructFromINCHI & 6 & \cellcolor{green!0}{\large 0}/{\footnotesize 40} & \cellcolor{green!0}{\large 0}/{\footnotesize 40} & \cellcolor{green!0}{\large 0}/{\footnotesize 40} & \cellcolor{green!0}{\large 0}/{\footnotesize 40} & \cellcolor{green!0}{\large 0}/{\footnotesize 106} & \cellcolor{green!0}{\large 0}/{\footnotesize 102} \tabularnewline
\rowcolor{black!10} 43 & kamailio\_parse\_msg & 6 & \cellcolor{green!0}{\large 0}/{\footnotesize 40} & \cellcolor{green!20}{\large 7}/{\footnotesize 40} & \cellcolor{green!0}{\large -}{\tiny -} & \cellcolor{green!10}{\large 4}/{\footnotesize 40} & \cellcolor{green!10}{\large 9}/{\footnotesize 94} & \cellcolor{green!10}{\large 6}/{\footnotesize 84} \tabularnewline
44 & libyang\_lys\_parse\_mem & 6 & \cellcolor{green!0}{\large 0}/{\footnotesize 40} & \cellcolor{green!0}{\large 0}/{\footnotesize 40} & \cellcolor{green!0}{\large 0}/{\footnotesize 40} & \cellcolor{green!0}{\large 0}/{\footnotesize 40} & \cellcolor{green!0}{\large 0}/{\footnotesize 94} & \cellcolor{green!10}{\large 1}/{\footnotesize 90} \tabularnewline
\rowcolor{black!10} 45 & proftpd\_pr\_json\_object\_from\_text & 6 & \cellcolor{green!0}{\large 0}/{\footnotesize 40} & \cellcolor{green!0}{\large 0}/{\footnotesize 40} & \cellcolor{green!0}{\large -}{\tiny -} & \cellcolor{green!0}{\large 0}/{\footnotesize 40} & \cellcolor{green!0}{\large 0}/{\footnotesize 119} & \cellcolor{green!10}{\large 1}/{\footnotesize 72} \tabularnewline
46 & selinux\_policydb\_read & 6 & \cellcolor{green!0}{\large 0}/{\footnotesize 40} & \cellcolor{green!10}{\large 1}/{\footnotesize 40} & \cellcolor{green!0}{\large -}{\tiny -} & \cellcolor{green!0}{\large 0}/{\footnotesize 40} & \cellcolor{green!10}{\large 3}/{\footnotesize 83} & \cellcolor{green!10}{\large 2}/{\footnotesize 75} \tabularnewline
\rowcolor{black!10} 47 & kamailio\_get\_src\_address\_socket & 7 & \cellcolor{green!0}{\large 0}/{\footnotesize 40} & \cellcolor{green!0}{\large 0}/{\footnotesize 40} & \cellcolor{green!0}{\large 0}/{\footnotesize 40} & \cellcolor{green!20}{\large 5}/{\footnotesize 40} & \cellcolor{green!0}{\large 0}/{\footnotesize 116} & \cellcolor{green!10}{\large 1}/{\footnotesize 87} \tabularnewline
48 & kamailio\_get\_src\_uri & 7 & \cellcolor{green!0}{\large 0}/{\footnotesize 40} & \cellcolor{green!0}{\large 0}/{\footnotesize 40} & \cellcolor{green!0}{\large 0}/{\footnotesize 40} & \cellcolor{green!0}{\large 0}/{\footnotesize 40} & \cellcolor{green!0}{\large 0}/{\footnotesize 112} & \cellcolor{green!0}{\large 0}/{\footnotesize 88} \tabularnewline
\rowcolor{black!10} 49 & kamailio\_parse\_content\_disposition & 7 & \cellcolor{green!0}{\large 0}/{\footnotesize 40} & \cellcolor{green!0}{\large 0}/{\footnotesize 40} & \cellcolor{green!0}{\large 0}/{\footnotesize 40} & \cellcolor{green!10}{\large 1}/{\footnotesize 40} & \cellcolor{green!0}{\large 0}/{\footnotesize 129} & \cellcolor{green!0}{\large 0}/{\footnotesize 89} \tabularnewline
50 & kamailio\_parse\_diversion\_header & 7 & \cellcolor{green!0}{\large 0}/{\footnotesize 40} & \cellcolor{green!0}{\large 0}/{\footnotesize 40} & \cellcolor{green!0}{\large 0}/{\footnotesize 40} & \cellcolor{green!0}{\large 0}/{\footnotesize 40} & \cellcolor{green!0}{\large 0}/{\footnotesize 136} & \cellcolor{green!0}{\large 0}/{\footnotesize 79} \tabularnewline
\rowcolor{black!10} 51 & kamailio\_parse\_from\_header & 7 & \cellcolor{green!0}{\large 0}/{\footnotesize 40} & \cellcolor{green!0}{\large 0}/{\footnotesize 40} & \cellcolor{green!0}{\large -}{\tiny -} & \cellcolor{green!0}{\large 0}/{\footnotesize 40} & \cellcolor{green!0}{\large 0}/{\footnotesize 134} & \cellcolor{green!0}{\large 0}/{\footnotesize 89} \tabularnewline
52 & kamailio\_parse\_from\_uri & 7 & \cellcolor{green!0}{\large 0}/{\footnotesize 40} & \cellcolor{green!0}{\large 0}/{\footnotesize 40} & \cellcolor{green!0}{\large -}{\tiny -} & \cellcolor{green!0}{\large 0}/{\footnotesize 40} & \cellcolor{green!0}{\large 0}/{\footnotesize 127} & \cellcolor{green!0}{\large 0}/{\footnotesize 86} \tabularnewline
\rowcolor{black!10} 53 & kamailio\_parse\_headers & 7 & \cellcolor{green!0}{\large 0}/{\footnotesize 40} & \cellcolor{green!0}{\large 0}/{\footnotesize 40} & \cellcolor{green!0}{\large -}{\tiny -} & \cellcolor{green!0}{\large 0}/{\footnotesize 40} & \cellcolor{green!0}{\large 0}/{\footnotesize 130} & \cellcolor{green!0}{\large 0}/{\footnotesize 72} \tabularnewline
54 & kamailio\_parse\_identityinfo\_header & 7 & \cellcolor{green!0}{\large 0}/{\footnotesize 40} & \cellcolor{green!0}{\large 0}/{\footnotesize 40} & \cellcolor{green!0}{\large -}{\tiny -} & \cellcolor{green!10}{\large 1}/{\footnotesize 40} & \cellcolor{green!0}{\large 0}/{\footnotesize 125} & \cellcolor{green!10}{\large 2}/{\footnotesize 85} \tabularnewline
\rowcolor{black!10} 55 & kamailio\_parse\_pai\_header & 7 & \cellcolor{green!0}{\large 0}/{\footnotesize 40} & \cellcolor{green!0}{\large 0}/{\footnotesize 40} & \cellcolor{green!0}{\large -}{\tiny -} & \cellcolor{green!10}{\large 1}/{\footnotesize 40} & \cellcolor{green!0}{\large 0}/{\footnotesize 128} & \cellcolor{green!0}{\large 0}/{\footnotesize 91} \tabularnewline
56 & kamailio\_parse\_privacy & 7 & \cellcolor{green!0}{\large 0}/{\footnotesize 40} & \cellcolor{green!0}{\large 0}/{\footnotesize 40} & \cellcolor{green!0}{\large 0}/{\footnotesize 40} & \cellcolor{green!10}{\large 1}/{\footnotesize 40} & \cellcolor{green!0}{\large 0}/{\footnotesize 143} & \cellcolor{green!0}{\large 1}/{\footnotesize 105} \tabularnewline
\rowcolor{black!10} 57 & kamailio\_parse\_record\_route\_headers & 7 & \cellcolor{green!0}{\large 0}/{\footnotesize 40} & \cellcolor{green!0}{\large 0}/{\footnotesize 40} & \cellcolor{green!0}{\large -}{\tiny -} & \cellcolor{green!0}{\large 0}/{\footnotesize 40} & \cellcolor{green!0}{\large 0}/{\footnotesize 137} & \cellcolor{green!10}{\large 3}/{\footnotesize 86} \tabularnewline
58 & kamailio\_parse\_refer\_to\_header & 7 & \cellcolor{green!0}{\large 0}/{\footnotesize 40} & \cellcolor{green!0}{\large 0}/{\footnotesize 40} & \cellcolor{green!0}{\large -}{\tiny -} & \cellcolor{green!0}{\large 0}/{\footnotesize 40} & \cellcolor{green!0}{\large 0}/{\footnotesize 122} & \cellcolor{green!0}{\large 0}/{\footnotesize 90} \tabularnewline
\rowcolor{black!10} 59 & kamailio\_parse\_route\_headers & 7 & \cellcolor{green!0}{\large 0}/{\footnotesize 40} & \cellcolor{green!0}{\large 0}/{\footnotesize 40} & \cellcolor{green!0}{\large -}{\tiny -} & \cellcolor{green!20}{\large 6}/{\footnotesize 40} & \cellcolor{green!0}{\large 0}/{\footnotesize 141} & \cellcolor{green!10}{\large 2}/{\footnotesize 84} \tabularnewline
60 & kamailio\_parse\_to\_header & 7 & \cellcolor{green!0}{\large 0}/{\footnotesize 40} & \cellcolor{green!0}{\large 0}/{\footnotesize 40} & \cellcolor{green!0}{\large -}{\tiny -} & \cellcolor{green!0}{\large 0}/{\footnotesize 40} & \cellcolor{green!0}{\large 0}/{\footnotesize 129} & \cellcolor{green!0}{\large 0}/{\footnotesize 77} \tabularnewline
\rowcolor{black!10} 61 & kamailio\_parse\_to\_uri & 7 & \cellcolor{green!0}{\large 0}/{\footnotesize 40} & \cellcolor{green!0}{\large 0}/{\footnotesize 40} & \cellcolor{green!0}{\large -}{\tiny -} & \cellcolor{green!0}{\large 0}/{\footnotesize 40} & \cellcolor{green!0}{\large 0}/{\footnotesize 116} & \cellcolor{green!0}{\large 0}/{\footnotesize 69} \tabularnewline
62 & libyang\_lyd\_parse\_data\_mem & 7 & \cellcolor{green!0}{\large 0}/{\footnotesize 40} & \cellcolor{green!0}{\large 0}/{\footnotesize 40} & \cellcolor{green!10}{\large 1}/{\footnotesize 40} & \cellcolor{green!0}{\large 0}/{\footnotesize 40} & \cellcolor{green!0}{\large 0}/{\footnotesize 103} & \cellcolor{green!0}{\large 0}/{\footnotesize 93} \tabularnewline
\rowcolor{black!10} 63 & bind9\_dns\_message\_parse & 8 & \cellcolor{green!0}{\large 0}/{\footnotesize 40} & \cellcolor{green!0}{\large 0}/{\footnotesize 40} & \cellcolor{green!0}{\large -}{\tiny -} & \cellcolor{green!0}{\large 0}/{\footnotesize 40} & \cellcolor{green!0}{\large 0}/{\footnotesize 110} & \cellcolor{green!0}{\large 0}/{\footnotesize 67} \tabularnewline
64 & igraph\_igraph\_read\_graph\_ncol & 8 & \cellcolor{green!0}{\large 0}/{\footnotesize 40} & \cellcolor{green!0}{\large 0}/{\footnotesize 40} & \cellcolor{green!0}{\large 0}/{\footnotesize 40} & \cellcolor{green!0}{\large 0}/{\footnotesize 40} & \cellcolor{green!0}{\large 0}/{\footnotesize 79} & \cellcolor{green!0}{\large 0}/{\footnotesize 70} \tabularnewline
\rowcolor{black!10} 65 & pjsip\_pj\_json\_parse & 8 & \cellcolor{green!0}{\large 0}/{\footnotesize 40} & \cellcolor{green!0}{\large 0}/{\footnotesize 40} & \cellcolor{green!0}{\large 0}/{\footnotesize 40} & \cellcolor{green!0}{\large 0}/{\footnotesize 40} & \cellcolor{green!0}{\large 0}/{\footnotesize 79} & \cellcolor{green!0}{\large 0}/{\footnotesize 80} \tabularnewline
66 & pjsip\_pj\_xml\_parse & 8 & \cellcolor{green!0}{\large 0}/{\footnotesize 40} & \cellcolor{green!0}{\large 0}/{\footnotesize 40} & \cellcolor{green!10}{\large 1}/{\footnotesize 40} & \cellcolor{green!0}{\large 0}/{\footnotesize 40} & \cellcolor{green!10}{\large 1}/{\footnotesize 91} & \cellcolor{green!0}{\large 0}/{\footnotesize 77} \tabularnewline
\rowcolor{black!10} 67 & pjsip\_pjmedia\_sdp\_parse & 8 & \cellcolor{green!0}{\large 0}/{\footnotesize 40} & \cellcolor{green!0}{\large 0}/{\footnotesize 40} & \cellcolor{green!0}{\large 0}/{\footnotesize 40} & \cellcolor{green!0}{\large 0}/{\footnotesize 40} & \cellcolor{green!0}{\large 0}/{\footnotesize 103} & \cellcolor{green!0}{\large 0}/{\footnotesize 64} \tabularnewline
68 & quickjs\_lre\_compile & 8 & \cellcolor{green!0}{\large 0}/{\footnotesize 40} & \cellcolor{green!0}{\large 0}/{\footnotesize 40} & \cellcolor{green!0}{\large -}{\tiny -} & \cellcolor{green!0}{\large 0}/{\footnotesize 40} & \cellcolor{green!0}{\large 0}/{\footnotesize 137} & \cellcolor{green!0}{\large 0}/{\footnotesize 97} \tabularnewline
\rowcolor{black!10} 69 & bind9\_isc\_lex\_getmastertoken & 9 & \cellcolor{green!0}{\large 0}/{\footnotesize 40} & \cellcolor{green!0}{\large 0}/{\footnotesize 40} & \cellcolor{green!0}{\large -}{\tiny -} & \cellcolor{green!0}{\large 0}/{\footnotesize 40} & \cellcolor{green!0}{\large 0}/{\footnotesize 85} & \cellcolor{green!0}{\large 0}/{\footnotesize 60} \tabularnewline
70 & bind9\_isc\_lex\_gettoken & 9 & \cellcolor{green!0}{\large 0}/{\footnotesize 40} & \cellcolor{green!0}{\large 0}/{\footnotesize 40} & \cellcolor{green!0}{\large -}{\tiny -} & \cellcolor{green!0}{\large 0}/{\footnotesize 40} & \cellcolor{green!0}{\large 0}/{\footnotesize 96} & \cellcolor{green!0}{\large 0}/{\footnotesize 65} \tabularnewline
\rowcolor{black!10} 71 & quickjs\_JS\_Eval & 9 & \cellcolor{green!0}{\large 0}/{\footnotesize 40} & \cellcolor{green!0}{\large 0}/{\footnotesize 40} & \cellcolor{green!0}{\large -}{\tiny -} & \cellcolor{green!0}{\large 0}/{\footnotesize 40} & \cellcolor{green!10}{\large 6}/{\footnotesize 105} & \cellcolor{green!10}{\large 1}/{\footnotesize 77} \tabularnewline
72 & igraph\_igraph\_edge\_connectivity & 10 & \cellcolor{green!0}{\large 0}/{\footnotesize 40} & \cellcolor{green!0}{\large 0}/{\footnotesize 40} & \cellcolor{green!0}{\large 0}/{\footnotesize 40} & \cellcolor{green!0}{\large 0}/{\footnotesize 40} & \cellcolor{green!0}{\large 0}/{\footnotesize 81} & \cellcolor{green!0}{\large 0}/{\footnotesize 70} \tabularnewline
\rowcolor{black!10} 73 & pjsip\_pj\_stun\_msg\_decode & 10 & \cellcolor{green!0}{\large 0}/{\footnotesize 40} & \cellcolor{green!0}{\large 0}/{\footnotesize 40} & \cellcolor{green!0}{\large 0}/{\footnotesize 40} & \cellcolor{green!0}{\large 0}/{\footnotesize 40} & \cellcolor{green!0}{\large 0}/{\footnotesize 78} & \cellcolor{green!0}{\large 0}/{\footnotesize 56} \tabularnewline
74 & bind9\_dns\_message\_checksig & 11 & \cellcolor{green!0}{\large 0}/{\footnotesize 40} & \cellcolor{green!0}{\large 0}/{\footnotesize 40} & \cellcolor{green!0}{\large -}{\tiny -} & \cellcolor{green!0}{\large 0}/{\footnotesize 40} & \cellcolor{green!0}{\large 0}/{\footnotesize 88} & \cellcolor{green!0}{\large 0}/{\footnotesize 80} \tabularnewline
\rowcolor{black!10} 75 & libzip\_zip\_fread & 11 & \cellcolor{green!10}{\large 1}/{\footnotesize 40} & \cellcolor{green!0}{\large 0}/{\footnotesize 40} & \cellcolor{green!0}{\large 0}/{\footnotesize 40} & \cellcolor{green!0}{\large 0}/{\footnotesize 40} & \cellcolor{green!10}{\large 3}/{\footnotesize 117} & \cellcolor{green!10}{\large 1}/{\footnotesize 77} \tabularnewline
76 & bind9\_dns\_rdata\_fromtext & 12 & \cellcolor{green!0}{\large 0}/{\footnotesize 40} & \cellcolor{green!0}{\large 0}/{\footnotesize 40} & \cellcolor{green!0}{\large -}{\tiny -} & \cellcolor{green!0}{\large 0}/{\footnotesize 40} & \cellcolor{green!0}{\large 0}/{\footnotesize 76} & \cellcolor{green!0}{\large 0}/{\footnotesize 52} \tabularnewline
\rowcolor{black!10} 77 & igraph\_igraph\_all\_minimal\_st\_separators & 12 & \cellcolor{green!0}{\large 0}/{\footnotesize 40} & \cellcolor{green!0}{\large 0}/{\footnotesize 40} & \cellcolor{green!0}{\large 0}/{\footnotesize 40} & \cellcolor{green!0}{\large 0}/{\footnotesize 40} & \cellcolor{green!0}{\large 0}/{\footnotesize 70} & \cellcolor{green!0}{\large 0}/{\footnotesize 57} \tabularnewline
78 & igraph\_igraph\_minimum\_size\_separators & 12 & \cellcolor{green!0}{\large 0}/{\footnotesize 40} & \cellcolor{green!0}{\large 0}/{\footnotesize 40} & \cellcolor{green!0}{\large 0}/{\footnotesize 40} & \cellcolor{green!0}{\large 0}/{\footnotesize 40} & \cellcolor{green!10}{\large 1}/{\footnotesize 69} & \cellcolor{green!0}{\large 0}/{\footnotesize 65} \tabularnewline
\rowcolor{black!10} 79 & pjsip\_pjsip\_parse\_msg & 12 & \cellcolor{green!0}{\large 0}/{\footnotesize 40} & \cellcolor{green!0}{\large 0}/{\footnotesize 40} & \cellcolor{green!0}{\large 0}/{\footnotesize 40} & \cellcolor{green!0}{\large 0}/{\footnotesize 40} & \cellcolor{green!0}{\large 0}/{\footnotesize 81} & \cellcolor{green!0}{\large 0}/{\footnotesize 72} \tabularnewline
80 & igraph\_igraph\_automorphism\_group & 13 & \cellcolor{green!0}{\large 0}/{\footnotesize 40} & \cellcolor{green!0}{\large 0}/{\footnotesize 40} & \cellcolor{green!0}{\large 0}/{\footnotesize 40} & \cellcolor{green!0}{\large 0}/{\footnotesize 40} & \cellcolor{green!0}{\large 0}/{\footnotesize 58} & \cellcolor{green!0}{\large 0}/{\footnotesize 64} \tabularnewline
\rowcolor{black!10} 81 & libmodbus\_modbus\_read\_bits & 15 & \cellcolor{green!0}{\large 0}/{\footnotesize 40} & \cellcolor{green!0}{\large 0}/{\footnotesize 40} & \cellcolor{green!0}{\large 0}/{\footnotesize 40} & \cellcolor{green!0}{\large 0}/{\footnotesize 40} & \cellcolor{green!0}{\large 0}/{\footnotesize 84} & \cellcolor{green!0}{\large 0}/{\footnotesize 84} \tabularnewline
82 & libmodbus\_modbus\_read\_registers & 15 & \cellcolor{green!0}{\large 0}/{\footnotesize 40} & \cellcolor{green!0}{\large 0}/{\footnotesize 40} & \cellcolor{green!0}{\large 0}/{\footnotesize 40} & \cellcolor{green!0}{\large 0}/{\footnotesize 40} & \cellcolor{green!0}{\large 0}/{\footnotesize 98} & \cellcolor{green!0}{\large 0}/{\footnotesize 82} \tabularnewline
\rowcolor{black!10} 83 & civetweb\_mg\_get\_response & 17 & \cellcolor{green!0}{\large 0}/{\footnotesize 40} & \cellcolor{green!0}{\large 0}/{\footnotesize 40} & \cellcolor{green!0}{\large 0}/{\footnotesize 40} & \cellcolor{green!0}{\large 0}/{\footnotesize 40} & \cellcolor{green!0}{\large 0}/{\footnotesize 103} & \cellcolor{green!0}{\large 0}/{\footnotesize 68} \tabularnewline
84 & bind9\_dns\_master\_loadbuffer & 20 & \cellcolor{green!0}{\large 0}/{\footnotesize 40} & \cellcolor{green!0}{\large 0}/{\footnotesize 40} & \cellcolor{green!0}{\large -}{\tiny -} & \cellcolor{green!0}{\large 0}/{\footnotesize 40} & \cellcolor{green!0}{\large 0}/{\footnotesize 93} & \cellcolor{green!0}{\large 0}/{\footnotesize 80} \tabularnewline
\rowcolor{black!10} 85 & libmodbus\_modbus\_receive & 33 & \cellcolor{green!0}{\large 0}/{\footnotesize 40} & \cellcolor{green!0}{\large 0}/{\footnotesize 40} & \cellcolor{green!0}{\large 0}/{\footnotesize 40} & \cellcolor{green!0}{\large 0}/{\footnotesize 40} & \cellcolor{green!0}{\large 0}/{\footnotesize 102} & \cellcolor{green!0}{\large 0}/{\footnotesize 75} \tabularnewline
86 & tmux\_input\_parse\_buffer & 42 & \cellcolor{green!0}{\large 0}/{\footnotesize 40} & \cellcolor{green!0}{\large 0}/{\footnotesize 40} & \cellcolor{green!0}{\large -}{\tiny -} & \cellcolor{green!0}{\large 0}/{\footnotesize 40} & \cellcolor{green!0}{\large 0}/{\footnotesize 133} & \cellcolor{green!0}{\large 0}/{\footnotesize 125} \tabularnewline

\bottomrule
%\end{tabular}
%}
%\end{table*}
\end{xltabular}
}
\twocolumn



% model: gpt-3.5-turbo-0613, temp: 2.0

\onecolumn
{\small %
\begin{xltabular}[h]{\textwidth}{ccccccccc}
%\begin{table*}[!t]
%\centering
\caption{Evaluation Result of model gpt-3.5-turbo-0613 with temperature 2.0.} \\
%\resizebox{1.0\linewidth}{!}{
%\begin{tabular}{cccccccccc}
\toprule
Index & Question & Score & NAIVE-40 & BACTX-40 & DOCTX-40 & UGCTX-40 & BA-ITER-40 & ALL-ITER-40 \tabularnewline
\midrule
\rowcolor{black!10} 1 & coturn\_stun\_is\_command\_message\_full\_check\_str & 1 & \cellcolor{green!0}{\large 0}/{\footnotesize 40} & \cellcolor{green!0}{\large 0}/{\footnotesize 40} & \cellcolor{green!0}{\large -}{\tiny -} & \cellcolor{green!0}{\large 0}/{\footnotesize 40} & \cellcolor{green!10}{\large 1}/{\footnotesize 45} & \cellcolor{green!0}{\large 0}/{\footnotesize 44} \tabularnewline
2 & kamailio\_parse\_uri & 1 & \cellcolor{green!0}{\large 0}/{\footnotesize 40} & \cellcolor{green!0}{\large 0}/{\footnotesize 40} & \cellcolor{green!0}{\large -}{\tiny -} & \cellcolor{green!0}{\large 0}/{\footnotesize 40} & \cellcolor{green!10}{\large 1}/{\footnotesize 42} & \cellcolor{green!0}{\large 0}/{\footnotesize 45} \tabularnewline
\rowcolor{black!10} 3 & coturn\_stun\_check\_message\_integrity\_str & 2 & \cellcolor{green!0}{\large 0}/{\footnotesize 40} & \cellcolor{green!0}{\large 0}/{\footnotesize 40} & \cellcolor{green!0}{\large -}{\tiny -} & \cellcolor{green!0}{\large 0}/{\footnotesize 40} & \cellcolor{green!0}{\large 0}/{\footnotesize 44} & \cellcolor{green!0}{\large 0}/{\footnotesize 41} \tabularnewline
4 & libiec61850\_MmsValue\_decodeMmsData & 2 & \cellcolor{green!0}{\large 0}/{\footnotesize 40} & \cellcolor{green!0}{\large 0}/{\footnotesize 40} & \cellcolor{green!0}{\large 0}/{\footnotesize 40} & \cellcolor{green!0}{\large 0}/{\footnotesize 40} & \cellcolor{green!0}{\large 0}/{\footnotesize 45} & \cellcolor{green!0}{\large 0}/{\footnotesize 44} \tabularnewline
\rowcolor{black!10} 5 & md4c\_md\_html & 2 & \cellcolor{green!0}{\large 0}/{\footnotesize 40} & \cellcolor{green!0}{\large 0}/{\footnotesize 40} & \cellcolor{green!0}{\large 0}/{\footnotesize 40} & \cellcolor{green!0}{\large 0}/{\footnotesize 40} & \cellcolor{green!0}{\large 0}/{\footnotesize 44} & \cellcolor{green!0}{\large 0}/{\footnotesize 43} \tabularnewline
6 & spdk\_spdk\_json\_parse & 2 & \cellcolor{green!0}{\large 0}/{\footnotesize 40} & \cellcolor{green!0}{\large 0}/{\footnotesize 40} & \cellcolor{green!0}{\large -}{\tiny -} & \cellcolor{green!0}{\large 0}/{\footnotesize 40} & \cellcolor{green!0}{\large 0}/{\footnotesize 46} & \cellcolor{green!0}{\large 0}/{\footnotesize 43} \tabularnewline
\rowcolor{black!10} 7 & croaring\_roaring\_bitmap\_portable\_deserialize\_safe & 3 & \cellcolor{green!0}{\large 0}/{\footnotesize 40} & \cellcolor{green!0}{\large 0}/{\footnotesize 40} & \cellcolor{green!0}{\large 0}/{\footnotesize 40} & \cellcolor{green!0}{\large 0}/{\footnotesize 40} & \cellcolor{green!0}{\large 0}/{\footnotesize 43} & \cellcolor{green!0}{\large 0}/{\footnotesize 43} \tabularnewline
8 & lua\_luaL\_loadbufferx & 3 & \cellcolor{green!0}{\large 0}/{\footnotesize 40} & \cellcolor{green!0}{\large 0}/{\footnotesize 40} & \cellcolor{green!0}{\large 0}/{\footnotesize 40} & \cellcolor{green!0}{\large 0}/{\footnotesize 40} & \cellcolor{green!0}{\large 0}/{\footnotesize 42} & \cellcolor{green!0}{\large 0}/{\footnotesize 42} \tabularnewline
\rowcolor{black!10} 9 & w3m\_wc\_Str\_conv\_with\_detect & 3 & \cellcolor{green!0}{\large 0}/{\footnotesize 40} & \cellcolor{green!0}{\large 0}/{\footnotesize 40} & \cellcolor{green!0}{\large -}{\tiny -} & \cellcolor{green!0}{\large 0}/{\footnotesize 40} & \cellcolor{green!0}{\large 0}/{\footnotesize 42} & \cellcolor{green!0}{\large 0}/{\footnotesize 45} \tabularnewline
10 & bind9\_dns\_name\_fromwire & 4 & \cellcolor{green!0}{\large 0}/{\footnotesize 40} & \cellcolor{green!0}{\large 0}/{\footnotesize 40} & \cellcolor{green!0}{\large -}{\tiny -} & \cellcolor{green!0}{\large 0}/{\footnotesize 40} & \cellcolor{green!0}{\large 0}/{\footnotesize 44} & \cellcolor{green!0}{\large 0}/{\footnotesize 42} \tabularnewline
\rowcolor{black!10} 11 & gdk-pixbuf\_gdk\_pixbuf\_animation\_new\_from\_file & 4 & \cellcolor{green!0}{\large 0}/{\footnotesize 40} & \cellcolor{green!0}{\large 0}/{\footnotesize 40} & \cellcolor{green!0}{\large 0}/{\footnotesize 40} & \cellcolor{green!0}{\large 0}/{\footnotesize 40} & \cellcolor{green!0}{\large 0}/{\footnotesize 45} & \cellcolor{green!0}{\large 0}/{\footnotesize 42} \tabularnewline
12 & gdk-pixbuf\_gdk\_pixbuf\_new\_from\_data & 4 & \cellcolor{green!0}{\large 0}/{\footnotesize 40} & \cellcolor{green!0}{\large 0}/{\footnotesize 40} & \cellcolor{green!0}{\large 0}/{\footnotesize 40} & \cellcolor{green!0}{\large 0}/{\footnotesize 40} & \cellcolor{green!0}{\large 0}/{\footnotesize 44} & \cellcolor{green!0}{\large 0}/{\footnotesize 43} \tabularnewline
\rowcolor{black!10} 13 & gdk-pixbuf\_gdk\_pixbuf\_new\_from\_file & 4 & \cellcolor{green!0}{\large 0}/{\footnotesize 40} & \cellcolor{green!0}{\large 0}/{\footnotesize 40} & \cellcolor{green!0}{\large 0}/{\footnotesize 40} & \cellcolor{green!0}{\large 0}/{\footnotesize 40} & \cellcolor{green!0}{\large 0}/{\footnotesize 45} & \cellcolor{green!0}{\large 0}/{\footnotesize 42} \tabularnewline
14 & gdk-pixbuf\_gdk\_pixbuf\_new\_from\_stream & 4 & \cellcolor{green!0}{\large 0}/{\footnotesize 40} & \cellcolor{green!0}{\large 0}/{\footnotesize 40} & \cellcolor{green!0}{\large 0}/{\footnotesize 40} & \cellcolor{green!0}{\large 0}/{\footnotesize 40} & \cellcolor{green!0}{\large 0}/{\footnotesize 43} & \cellcolor{green!0}{\large 0}/{\footnotesize 44} \tabularnewline
\rowcolor{black!10} 15 & gpac\_gf\_isom\_open\_file & 4 & \cellcolor{green!0}{\large 0}/{\footnotesize 40} & \cellcolor{green!0}{\large 0}/{\footnotesize 40} & \cellcolor{green!0}{\large -}{\tiny -} & \cellcolor{green!0}{\large 0}/{\footnotesize 40} & \cellcolor{green!0}{\large 0}/{\footnotesize 42} & \cellcolor{green!0}{\large 0}/{\footnotesize 42} \tabularnewline
16 & libbpf\_bpf\_object\_\_open\_mem & 4 & \cellcolor{green!0}{\large 0}/{\footnotesize 40} & \cellcolor{green!0}{\large 0}/{\footnotesize 40} & \cellcolor{green!0}{\large 0}/{\footnotesize 40} & \cellcolor{green!0}{\large 0}/{\footnotesize 40} & \cellcolor{green!0}{\large 0}/{\footnotesize 47} & \cellcolor{green!0}{\large 0}/{\footnotesize 43} \tabularnewline
\rowcolor{black!10} 17 & libpg\_query\_pg\_query\_parse & 4 & \cellcolor{green!0}{\large 0}/{\footnotesize 40} & \cellcolor{green!0}{\large 0}/{\footnotesize 40} & \cellcolor{green!0}{\large -}{\tiny -} & \cellcolor{green!0}{\large 0}/{\footnotesize 40} & \cellcolor{green!0}{\large 0}/{\footnotesize 45} & \cellcolor{green!0}{\large 0}/{\footnotesize 45} \tabularnewline
18 & libucl\_ucl\_parser\_add\_string & 4 & \cellcolor{green!0}{\large 0}/{\footnotesize 40} & \cellcolor{green!0}{\large 0}/{\footnotesize 40} & \cellcolor{green!0}{\large 0}/{\footnotesize 40} & \cellcolor{green!0}{\large 0}/{\footnotesize 40} & \cellcolor{green!0}{\large 0}/{\footnotesize 45} & \cellcolor{green!0}{\large 0}/{\footnotesize 43} \tabularnewline
\rowcolor{black!10} 19 & oniguruma\_onig\_new & 4 & \cellcolor{green!0}{\large 0}/{\footnotesize 40} & \cellcolor{green!0}{\large 0}/{\footnotesize 40} & \cellcolor{green!0}{\large 0}/{\footnotesize 40} & \cellcolor{green!0}{\large 0}/{\footnotesize 40} & \cellcolor{green!0}{\large 0}/{\footnotesize 45} & \cellcolor{green!0}{\large 0}/{\footnotesize 44} \tabularnewline
20 & pupnp\_ixmlLoadDocumentEx & 4 & \cellcolor{green!0}{\large 0}/{\footnotesize 40} & \cellcolor{green!0}{\large 0}/{\footnotesize 40} & \cellcolor{green!0}{\large 0}/{\footnotesize 40} & \cellcolor{green!0}{\large 0}/{\footnotesize 40} & \cellcolor{green!0}{\large 0}/{\footnotesize 45} & \cellcolor{green!0}{\large 0}/{\footnotesize 45} \tabularnewline
\rowcolor{black!10} 21 & gdk-pixbuf\_gdk\_pixbuf\_new\_from\_file\_at\_scale & 5 & \cellcolor{green!0}{\large 0}/{\footnotesize 40} & \cellcolor{green!0}{\large 0}/{\footnotesize 40} & \cellcolor{green!0}{\large 0}/{\footnotesize 40} & \cellcolor{green!0}{\large 0}/{\footnotesize 40} & \cellcolor{green!0}{\large 0}/{\footnotesize 45} & \cellcolor{green!0}{\large 0}/{\footnotesize 43} \tabularnewline
22 & inchi\_GetINCHIKeyFromINCHI & 5 & \cellcolor{green!0}{\large 0}/{\footnotesize 40} & \cellcolor{green!0}{\large 0}/{\footnotesize 40} & \cellcolor{green!0}{\large 0}/{\footnotesize 40} & \cellcolor{green!0}{\large 0}/{\footnotesize 40} & \cellcolor{green!0}{\large 0}/{\footnotesize 42} & \cellcolor{green!0}{\large 0}/{\footnotesize 42} \tabularnewline
\rowcolor{black!10} 23 & libdwarf\_dwarf\_init\_b & 5 & \cellcolor{green!0}{\large 0}/{\footnotesize 40} & \cellcolor{green!0}{\large 0}/{\footnotesize 40} & \cellcolor{green!0}{\large 0}/{\footnotesize 40} & \cellcolor{green!0}{\large 0}/{\footnotesize 40} & \cellcolor{green!0}{\large 0}/{\footnotesize 42} & \cellcolor{green!0}{\large 0}/{\footnotesize 45} \tabularnewline
24 & libdwarf\_dwarf\_init\_path & 5 & \cellcolor{green!0}{\large 0}/{\footnotesize 40} & \cellcolor{green!0}{\large 0}/{\footnotesize 40} & \cellcolor{green!0}{\large 0}/{\footnotesize 40} & \cellcolor{green!0}{\large 0}/{\footnotesize 40} & \cellcolor{green!0}{\large 0}/{\footnotesize 43} & \cellcolor{green!0}{\large 0}/{\footnotesize 40} \tabularnewline
\rowcolor{black!10} 25 & liblouis\_lou\_compileString & 5 & \cellcolor{green!0}{\large 0}/{\footnotesize 40} & \cellcolor{green!0}{\large 0}/{\footnotesize 40} & \cellcolor{green!0}{\large 0}/{\footnotesize 40} & \cellcolor{green!0}{\large 0}/{\footnotesize 40} & \cellcolor{green!0}{\large 0}/{\footnotesize 45} & \cellcolor{green!0}{\large 0}/{\footnotesize 43} \tabularnewline
26 & selinux\_cil\_compile & 5 & \cellcolor{green!0}{\large 0}/{\footnotesize 40} & \cellcolor{green!0}{\large 0}/{\footnotesize 40} & \cellcolor{green!0}{\large -}{\tiny -} & \cellcolor{green!0}{\large 0}/{\footnotesize 40} & \cellcolor{green!0}{\large 0}/{\footnotesize 42} & \cellcolor{green!0}{\large 0}/{\footnotesize 42} \tabularnewline
\rowcolor{black!10} 27 & bind9\_dns\_name\_fromtext & 6 & \cellcolor{green!0}{\large 0}/{\footnotesize 40} & \cellcolor{green!0}{\large 0}/{\footnotesize 40} & \cellcolor{green!0}{\large -}{\tiny -} & \cellcolor{green!0}{\large 0}/{\footnotesize 40} & \cellcolor{green!0}{\large 0}/{\footnotesize 45} & \cellcolor{green!0}{\large 0}/{\footnotesize 42} \tabularnewline
28 & bind9\_dns\_rdata\_fromwire & 6 & \cellcolor{green!0}{\large 0}/{\footnotesize 40} & \cellcolor{green!0}{\large 0}/{\footnotesize 40} & \cellcolor{green!0}{\large -}{\tiny -} & \cellcolor{green!0}{\large 0}/{\footnotesize 40} & \cellcolor{green!0}{\large 0}/{\footnotesize 41} & \cellcolor{green!0}{\large 0}/{\footnotesize 41} \tabularnewline
\rowcolor{black!10} 29 & coturn\_stun\_is\_binding\_response & 6 & \cellcolor{green!0}{\large 0}/{\footnotesize 40} & \cellcolor{green!0}{\large 0}/{\footnotesize 40} & \cellcolor{green!0}{\large -}{\tiny -} & \cellcolor{green!0}{\large 0}/{\footnotesize 40} & \cellcolor{green!0}{\large 0}/{\footnotesize 42} & \cellcolor{green!0}{\large 0}/{\footnotesize 42} \tabularnewline
30 & coturn\_stun\_is\_command\_message & 6 & \cellcolor{green!0}{\large 0}/{\footnotesize 40} & \cellcolor{green!0}{\large 0}/{\footnotesize 40} & \cellcolor{green!0}{\large 0}/{\footnotesize 40} & \cellcolor{green!0}{\large 0}/{\footnotesize 40} & \cellcolor{green!0}{\large 0}/{\footnotesize 44} & \cellcolor{green!0}{\large 0}/{\footnotesize 41} \tabularnewline
\rowcolor{black!10} 31 & coturn\_stun\_is\_response & 6 & \cellcolor{green!0}{\large 0}/{\footnotesize 40} & \cellcolor{green!0}{\large 0}/{\footnotesize 40} & \cellcolor{green!0}{\large -}{\tiny -} & \cellcolor{green!0}{\large 0}/{\footnotesize 40} & \cellcolor{green!0}{\large 0}/{\footnotesize 45} & \cellcolor{green!0}{\large 0}/{\footnotesize 43} \tabularnewline
32 & coturn\_stun\_is\_success\_response & 6 & \cellcolor{green!0}{\large 0}/{\footnotesize 40} & \cellcolor{green!0}{\large 0}/{\footnotesize 40} & \cellcolor{green!0}{\large -}{\tiny -} & \cellcolor{green!0}{\large 0}/{\footnotesize 40} & \cellcolor{green!0}{\large 0}/{\footnotesize 45} & \cellcolor{green!0}{\large 0}/{\footnotesize 42} \tabularnewline
\rowcolor{black!10} 33 & hiredis\_redisFormatCommand & 6 & \cellcolor{green!0}{\large 0}/{\footnotesize 40} & \cellcolor{green!0}{\large 0}/{\footnotesize 40} & \cellcolor{green!0}{\large -}{\tiny -} & \cellcolor{green!0}{\large 0}/{\footnotesize 40} & \cellcolor{green!0}{\large 0}/{\footnotesize 40} & \cellcolor{green!0}{\large 0}/{\footnotesize 41} \tabularnewline
34 & igraph\_igraph\_read\_graph\_dl & 6 & \cellcolor{green!0}{\large 0}/{\footnotesize 40} & \cellcolor{green!0}{\large 0}/{\footnotesize 40} & \cellcolor{green!0}{\large 0}/{\footnotesize 40} & \cellcolor{green!0}{\large 0}/{\footnotesize 40} & \cellcolor{green!0}{\large 0}/{\footnotesize 43} & \cellcolor{green!0}{\large 0}/{\footnotesize 43} \tabularnewline
\rowcolor{black!10} 35 & igraph\_igraph\_read\_graph\_edgelist & 6 & \cellcolor{green!0}{\large 0}/{\footnotesize 40} & \cellcolor{green!0}{\large 0}/{\footnotesize 40} & \cellcolor{green!0}{\large 0}/{\footnotesize 40} & \cellcolor{green!0}{\large 0}/{\footnotesize 40} & \cellcolor{green!0}{\large 0}/{\footnotesize 43} & \cellcolor{green!0}{\large 0}/{\footnotesize 42} \tabularnewline
36 & igraph\_igraph\_read\_graph\_gml & 6 & \cellcolor{green!0}{\large 0}/{\footnotesize 40} & \cellcolor{green!0}{\large 0}/{\footnotesize 40} & \cellcolor{green!0}{\large 0}/{\footnotesize 40} & \cellcolor{green!0}{\large 0}/{\footnotesize 40} & \cellcolor{green!0}{\large 0}/{\footnotesize 42} & \cellcolor{green!0}{\large 0}/{\footnotesize 44} \tabularnewline
\rowcolor{black!10} 37 & igraph\_igraph\_read\_graph\_graphdb & 6 & \cellcolor{green!0}{\large 0}/{\footnotesize 40} & \cellcolor{green!0}{\large 0}/{\footnotesize 40} & \cellcolor{green!0}{\large 0}/{\footnotesize 40} & \cellcolor{green!0}{\large 0}/{\footnotesize 40} & \cellcolor{green!0}{\large 0}/{\footnotesize 46} & \cellcolor{green!0}{\large 0}/{\footnotesize 43} \tabularnewline
38 & igraph\_igraph\_read\_graph\_graphml & 6 & \cellcolor{green!0}{\large 0}/{\footnotesize 40} & \cellcolor{green!0}{\large 0}/{\footnotesize 40} & \cellcolor{green!0}{\large 0}/{\footnotesize 40} & \cellcolor{green!0}{\large 0}/{\footnotesize 40} & \cellcolor{green!0}{\large 0}/{\footnotesize 43} & \cellcolor{green!0}{\large 0}/{\footnotesize 48} \tabularnewline
\rowcolor{black!10} 39 & igraph\_igraph\_read\_graph\_lgl & 6 & \cellcolor{green!0}{\large 0}/{\footnotesize 40} & \cellcolor{green!0}{\large 0}/{\footnotesize 40} & \cellcolor{green!0}{\large 0}/{\footnotesize 40} & \cellcolor{green!0}{\large 0}/{\footnotesize 40} & \cellcolor{green!0}{\large 0}/{\footnotesize 44} & \cellcolor{green!0}{\large 0}/{\footnotesize 44} \tabularnewline
40 & igraph\_igraph\_read\_graph\_pajek & 6 & \cellcolor{green!0}{\large 0}/{\footnotesize 40} & \cellcolor{green!0}{\large 0}/{\footnotesize 40} & \cellcolor{green!0}{\large 0}/{\footnotesize 40} & \cellcolor{green!0}{\large 0}/{\footnotesize 40} & \cellcolor{green!0}{\large 0}/{\footnotesize 46} & \cellcolor{green!0}{\large 0}/{\footnotesize 41} \tabularnewline
\rowcolor{black!10} 41 & inchi\_GetINCHIfromINCHI & 6 & \cellcolor{green!0}{\large 0}/{\footnotesize 40} & \cellcolor{green!0}{\large 0}/{\footnotesize 40} & \cellcolor{green!0}{\large 0}/{\footnotesize 40} & \cellcolor{green!0}{\large 0}/{\footnotesize 40} & \cellcolor{green!0}{\large 0}/{\footnotesize 40} & \cellcolor{green!0}{\large 0}/{\footnotesize 43} \tabularnewline
42 & inchi\_GetStructFromINCHI & 6 & \cellcolor{green!0}{\large 0}/{\footnotesize 40} & \cellcolor{green!0}{\large 0}/{\footnotesize 40} & \cellcolor{green!0}{\large 0}/{\footnotesize 40} & \cellcolor{green!0}{\large 0}/{\footnotesize 40} & \cellcolor{green!0}{\large 0}/{\footnotesize 47} & \cellcolor{green!0}{\large 0}/{\footnotesize 42} \tabularnewline
\rowcolor{black!10} 43 & kamailio\_parse\_msg & 6 & \cellcolor{green!0}{\large 0}/{\footnotesize 40} & \cellcolor{green!10}{\large 1}/{\footnotesize 40} & \cellcolor{green!0}{\large -}{\tiny -} & \cellcolor{green!0}{\large 0}/{\footnotesize 40} & \cellcolor{green!0}{\large 0}/{\footnotesize 42} & \cellcolor{green!0}{\large 0}/{\footnotesize 45} \tabularnewline
44 & libyang\_lys\_parse\_mem & 6 & \cellcolor{green!0}{\large 0}/{\footnotesize 40} & \cellcolor{green!0}{\large 0}/{\footnotesize 40} & \cellcolor{green!0}{\large 0}/{\footnotesize 40} & \cellcolor{green!0}{\large 0}/{\footnotesize 40} & \cellcolor{green!0}{\large 0}/{\footnotesize 40} & \cellcolor{green!0}{\large 0}/{\footnotesize 41} \tabularnewline
\rowcolor{black!10} 45 & proftpd\_pr\_json\_object\_from\_text & 6 & \cellcolor{green!0}{\large 0}/{\footnotesize 40} & \cellcolor{green!0}{\large 0}/{\footnotesize 40} & \cellcolor{green!0}{\large -}{\tiny -} & \cellcolor{green!0}{\large 0}/{\footnotesize 40} & \cellcolor{green!0}{\large 0}/{\footnotesize 41} & \cellcolor{green!0}{\large 0}/{\footnotesize 46} \tabularnewline
46 & selinux\_policydb\_read & 6 & \cellcolor{green!0}{\large 0}/{\footnotesize 40} & \cellcolor{green!0}{\large 0}/{\footnotesize 40} & \cellcolor{green!0}{\large -}{\tiny -} & \cellcolor{green!0}{\large 0}/{\footnotesize 40} & \cellcolor{green!0}{\large 0}/{\footnotesize 45} & \cellcolor{green!0}{\large 0}/{\footnotesize 41} \tabularnewline
\rowcolor{black!10} 47 & kamailio\_get\_src\_address\_socket & 7 & \cellcolor{green!0}{\large 0}/{\footnotesize 40} & \cellcolor{green!0}{\large 0}/{\footnotesize 40} & \cellcolor{green!0}{\large 0}/{\footnotesize 40} & \cellcolor{green!0}{\large 0}/{\footnotesize 40} & \cellcolor{green!0}{\large 0}/{\footnotesize 42} & \cellcolor{green!0}{\large 0}/{\footnotesize 43} \tabularnewline
48 & kamailio\_get\_src\_uri & 7 & \cellcolor{green!0}{\large 0}/{\footnotesize 40} & \cellcolor{green!0}{\large 0}/{\footnotesize 40} & \cellcolor{green!0}{\large 0}/{\footnotesize 40} & \cellcolor{green!0}{\large 0}/{\footnotesize 40} & \cellcolor{green!0}{\large 0}/{\footnotesize 42} & \cellcolor{green!0}{\large 0}/{\footnotesize 45} \tabularnewline
\rowcolor{black!10} 49 & kamailio\_parse\_content\_disposition & 7 & \cellcolor{green!0}{\large 0}/{\footnotesize 40} & \cellcolor{green!0}{\large 0}/{\footnotesize 40} & \cellcolor{green!0}{\large 0}/{\footnotesize 40} & \cellcolor{green!0}{\large 0}/{\footnotesize 40} & \cellcolor{green!0}{\large 0}/{\footnotesize 43} & \cellcolor{green!0}{\large 0}/{\footnotesize 41} \tabularnewline
50 & kamailio\_parse\_diversion\_header & 7 & \cellcolor{green!0}{\large 0}/{\footnotesize 40} & \cellcolor{green!0}{\large 0}/{\footnotesize 40} & \cellcolor{green!0}{\large 0}/{\footnotesize 40} & \cellcolor{green!0}{\large 0}/{\footnotesize 40} & \cellcolor{green!0}{\large 0}/{\footnotesize 41} & \cellcolor{green!0}{\large 0}/{\footnotesize 42} \tabularnewline
\rowcolor{black!10} 51 & kamailio\_parse\_from\_header & 7 & \cellcolor{green!0}{\large 0}/{\footnotesize 40} & \cellcolor{green!0}{\large 0}/{\footnotesize 40} & \cellcolor{green!0}{\large -}{\tiny -} & \cellcolor{green!0}{\large 0}/{\footnotesize 40} & \cellcolor{green!0}{\large 0}/{\footnotesize 41} & \cellcolor{green!0}{\large 0}/{\footnotesize 43} \tabularnewline
52 & kamailio\_parse\_from\_uri & 7 & \cellcolor{green!0}{\large 0}/{\footnotesize 40} & \cellcolor{green!0}{\large 0}/{\footnotesize 40} & \cellcolor{green!0}{\large -}{\tiny -} & \cellcolor{green!0}{\large 0}/{\footnotesize 40} & \cellcolor{green!0}{\large 0}/{\footnotesize 41} & \cellcolor{green!0}{\large 0}/{\footnotesize 44} \tabularnewline
\rowcolor{black!10} 53 & kamailio\_parse\_headers & 7 & \cellcolor{green!0}{\large 0}/{\footnotesize 40} & \cellcolor{green!0}{\large 0}/{\footnotesize 40} & \cellcolor{green!0}{\large -}{\tiny -} & \cellcolor{green!0}{\large 0}/{\footnotesize 40} & \cellcolor{green!0}{\large 0}/{\footnotesize 41} & \cellcolor{green!0}{\large 0}/{\footnotesize 45} \tabularnewline
54 & kamailio\_parse\_identityinfo\_header & 7 & \cellcolor{green!0}{\large 0}/{\footnotesize 40} & \cellcolor{green!0}{\large 0}/{\footnotesize 40} & \cellcolor{green!0}{\large -}{\tiny -} & \cellcolor{green!0}{\large 0}/{\footnotesize 40} & \cellcolor{green!0}{\large 0}/{\footnotesize 41} & \cellcolor{green!0}{\large 0}/{\footnotesize 42} \tabularnewline
\rowcolor{black!10} 55 & kamailio\_parse\_pai\_header & 7 & \cellcolor{green!0}{\large 0}/{\footnotesize 40} & \cellcolor{green!0}{\large 0}/{\footnotesize 40} & \cellcolor{green!0}{\large -}{\tiny -} & \cellcolor{green!0}{\large 0}/{\footnotesize 40} & \cellcolor{green!0}{\large 0}/{\footnotesize 41} & \cellcolor{green!0}{\large 0}/{\footnotesize 41} \tabularnewline
56 & kamailio\_parse\_privacy & 7 & \cellcolor{green!0}{\large 0}/{\footnotesize 40} & \cellcolor{green!0}{\large 0}/{\footnotesize 40} & \cellcolor{green!0}{\large 0}/{\footnotesize 40} & \cellcolor{green!0}{\large 0}/{\footnotesize 40} & \cellcolor{green!0}{\large 0}/{\footnotesize 50} & \cellcolor{green!0}{\large 0}/{\footnotesize 43} \tabularnewline
\rowcolor{black!10} 57 & kamailio\_parse\_record\_route\_headers & 7 & \cellcolor{green!0}{\large 0}/{\footnotesize 40} & \cellcolor{green!0}{\large 0}/{\footnotesize 40} & \cellcolor{green!0}{\large -}{\tiny -} & \cellcolor{green!0}{\large 0}/{\footnotesize 40} & \cellcolor{green!0}{\large 0}/{\footnotesize 42} & \cellcolor{green!0}{\large 0}/{\footnotesize 40} \tabularnewline
58 & kamailio\_parse\_refer\_to\_header & 7 & \cellcolor{green!0}{\large 0}/{\footnotesize 40} & \cellcolor{green!0}{\large 0}/{\footnotesize 40} & \cellcolor{green!0}{\large -}{\tiny -} & \cellcolor{green!0}{\large 0}/{\footnotesize 40} & \cellcolor{green!0}{\large 0}/{\footnotesize 43} & \cellcolor{green!0}{\large 0}/{\footnotesize 41} \tabularnewline
\rowcolor{black!10} 59 & kamailio\_parse\_route\_headers & 7 & \cellcolor{green!0}{\large 0}/{\footnotesize 40} & \cellcolor{green!0}{\large 0}/{\footnotesize 40} & \cellcolor{green!0}{\large -}{\tiny -} & \cellcolor{green!0}{\large 0}/{\footnotesize 40} & \cellcolor{green!0}{\large 0}/{\footnotesize 43} & \cellcolor{green!0}{\large 0}/{\footnotesize 44} \tabularnewline
60 & kamailio\_parse\_to\_header & 7 & \cellcolor{green!0}{\large 0}/{\footnotesize 40} & \cellcolor{green!0}{\large 0}/{\footnotesize 40} & \cellcolor{green!0}{\large -}{\tiny -} & \cellcolor{green!0}{\large 0}/{\footnotesize 40} & \cellcolor{green!0}{\large 0}/{\footnotesize 41} & \cellcolor{green!0}{\large 0}/{\footnotesize 43} \tabularnewline
\rowcolor{black!10} 61 & kamailio\_parse\_to\_uri & 7 & \cellcolor{green!0}{\large 0}/{\footnotesize 40} & \cellcolor{green!0}{\large 0}/{\footnotesize 40} & \cellcolor{green!0}{\large -}{\tiny -} & \cellcolor{green!0}{\large 0}/{\footnotesize 40} & \cellcolor{green!0}{\large 0}/{\footnotesize 44} & \cellcolor{green!0}{\large 0}/{\footnotesize 45} \tabularnewline
62 & libyang\_lyd\_parse\_data\_mem & 7 & \cellcolor{green!0}{\large 0}/{\footnotesize 40} & \cellcolor{green!0}{\large 0}/{\footnotesize 40} & \cellcolor{green!0}{\large 0}/{\footnotesize 40} & \cellcolor{green!0}{\large 0}/{\footnotesize 40} & \cellcolor{green!0}{\large 0}/{\footnotesize 45} & \cellcolor{green!0}{\large 0}/{\footnotesize 44} \tabularnewline
\rowcolor{black!10} 63 & bind9\_dns\_message\_parse & 8 & \cellcolor{green!0}{\large 0}/{\footnotesize 40} & \cellcolor{green!0}{\large 0}/{\footnotesize 40} & \cellcolor{green!0}{\large -}{\tiny -} & \cellcolor{green!0}{\large 0}/{\footnotesize 40} & \cellcolor{green!0}{\large 0}/{\footnotesize 44} & \cellcolor{green!0}{\large 0}/{\footnotesize 44} \tabularnewline
64 & igraph\_igraph\_read\_graph\_ncol & 8 & \cellcolor{green!0}{\large 0}/{\footnotesize 40} & \cellcolor{green!0}{\large 0}/{\footnotesize 40} & \cellcolor{green!0}{\large 0}/{\footnotesize 40} & \cellcolor{green!0}{\large 0}/{\footnotesize 40} & \cellcolor{green!0}{\large 0}/{\footnotesize 43} & \cellcolor{green!0}{\large 0}/{\footnotesize 42} \tabularnewline
\rowcolor{black!10} 65 & pjsip\_pj\_json\_parse & 8 & \cellcolor{green!0}{\large 0}/{\footnotesize 40} & \cellcolor{green!0}{\large 0}/{\footnotesize 40} & \cellcolor{green!0}{\large 0}/{\footnotesize 40} & \cellcolor{green!0}{\large 0}/{\footnotesize 40} & \cellcolor{green!0}{\large 0}/{\footnotesize 47} & \cellcolor{green!0}{\large 0}/{\footnotesize 46} \tabularnewline
66 & pjsip\_pj\_xml\_parse & 8 & \cellcolor{green!0}{\large 0}/{\footnotesize 40} & \cellcolor{green!0}{\large 0}/{\footnotesize 40} & \cellcolor{green!0}{\large 0}/{\footnotesize 40} & \cellcolor{green!0}{\large 0}/{\footnotesize 40} & \cellcolor{green!0}{\large 0}/{\footnotesize 42} & \cellcolor{green!0}{\large 0}/{\footnotesize 42} \tabularnewline
\rowcolor{black!10} 67 & pjsip\_pjmedia\_sdp\_parse & 8 & \cellcolor{green!0}{\large 0}/{\footnotesize 40} & \cellcolor{green!0}{\large 0}/{\footnotesize 40} & \cellcolor{green!0}{\large 0}/{\footnotesize 40} & \cellcolor{green!0}{\large 0}/{\footnotesize 40} & \cellcolor{green!0}{\large 0}/{\footnotesize 45} & \cellcolor{green!0}{\large 0}/{\footnotesize 40} \tabularnewline
68 & quickjs\_lre\_compile & 8 & \cellcolor{green!0}{\large 0}/{\footnotesize 40} & \cellcolor{green!0}{\large 0}/{\footnotesize 40} & \cellcolor{green!0}{\large -}{\tiny -} & \cellcolor{green!0}{\large 0}/{\footnotesize 40} & \cellcolor{green!0}{\large 0}/{\footnotesize 44} & \cellcolor{green!0}{\large 0}/{\footnotesize 40} \tabularnewline
\rowcolor{black!10} 69 & bind9\_isc\_lex\_getmastertoken & 9 & \cellcolor{green!0}{\large 0}/{\footnotesize 40} & \cellcolor{green!0}{\large 0}/{\footnotesize 40} & \cellcolor{green!0}{\large -}{\tiny -} & \cellcolor{green!0}{\large 0}/{\footnotesize 40} & \cellcolor{green!0}{\large 0}/{\footnotesize 43} & \cellcolor{green!0}{\large 0}/{\footnotesize 44} \tabularnewline
70 & bind9\_isc\_lex\_gettoken & 9 & \cellcolor{green!0}{\large 0}/{\footnotesize 40} & \cellcolor{green!0}{\large 0}/{\footnotesize 40} & \cellcolor{green!0}{\large -}{\tiny -} & \cellcolor{green!0}{\large 0}/{\footnotesize 40} & \cellcolor{green!0}{\large 0}/{\footnotesize 44} & \cellcolor{green!0}{\large 0}/{\footnotesize 40} \tabularnewline
\rowcolor{black!10} 71 & quickjs\_JS\_Eval & 9 & \cellcolor{green!0}{\large 0}/{\footnotesize 40} & \cellcolor{green!0}{\large 0}/{\footnotesize 40} & \cellcolor{green!0}{\large -}{\tiny -} & \cellcolor{green!0}{\large 0}/{\footnotesize 40} & \cellcolor{green!0}{\large 0}/{\footnotesize 43} & \cellcolor{green!0}{\large 0}/{\footnotesize 44} \tabularnewline
72 & igraph\_igraph\_edge\_connectivity & 10 & \cellcolor{green!0}{\large 0}/{\footnotesize 40} & \cellcolor{green!0}{\large 0}/{\footnotesize 40} & \cellcolor{green!0}{\large 0}/{\footnotesize 40} & \cellcolor{green!0}{\large 0}/{\footnotesize 40} & \cellcolor{green!0}{\large 0}/{\footnotesize 43} & \cellcolor{green!0}{\large 0}/{\footnotesize 40} \tabularnewline
\rowcolor{black!10} 73 & pjsip\_pj\_stun\_msg\_decode & 10 & \cellcolor{green!0}{\large 0}/{\footnotesize 40} & \cellcolor{green!0}{\large 0}/{\footnotesize 40} & \cellcolor{green!0}{\large 0}/{\footnotesize 40} & \cellcolor{green!0}{\large 0}/{\footnotesize 40} & \cellcolor{green!0}{\large 0}/{\footnotesize 44} & \cellcolor{green!0}{\large 0}/{\footnotesize 42} \tabularnewline
74 & bind9\_dns\_message\_checksig & 11 & \cellcolor{green!0}{\large 0}/{\footnotesize 40} & \cellcolor{green!0}{\large 0}/{\footnotesize 40} & \cellcolor{green!0}{\large -}{\tiny -} & \cellcolor{green!0}{\large 0}/{\footnotesize 40} & \cellcolor{green!0}{\large 0}/{\footnotesize 44} & \cellcolor{green!0}{\large 0}/{\footnotesize 42} \tabularnewline
\rowcolor{black!10} 75 & libzip\_zip\_fread & 11 & \cellcolor{green!0}{\large 0}/{\footnotesize 40} & \cellcolor{green!0}{\large 0}/{\footnotesize 40} & \cellcolor{green!0}{\large 0}/{\footnotesize 40} & \cellcolor{green!0}{\large 0}/{\footnotesize 40} & \cellcolor{green!0}{\large 0}/{\footnotesize 40} & \cellcolor{green!0}{\large 0}/{\footnotesize 42} \tabularnewline
76 & bind9\_dns\_rdata\_fromtext & 12 & \cellcolor{green!0}{\large 0}/{\footnotesize 40} & \cellcolor{green!0}{\large 0}/{\footnotesize 40} & \cellcolor{green!0}{\large -}{\tiny -} & \cellcolor{green!0}{\large 0}/{\footnotesize 40} & \cellcolor{green!0}{\large 0}/{\footnotesize 41} & \cellcolor{green!0}{\large 0}/{\footnotesize 42} \tabularnewline
\rowcolor{black!10} 77 & igraph\_igraph\_all\_minimal\_st\_separators & 12 & \cellcolor{green!0}{\large 0}/{\footnotesize 40} & \cellcolor{green!0}{\large 0}/{\footnotesize 40} & \cellcolor{green!0}{\large 0}/{\footnotesize 40} & \cellcolor{green!0}{\large 0}/{\footnotesize 40} & \cellcolor{green!0}{\large 0}/{\footnotesize 45} & \cellcolor{green!0}{\large 0}/{\footnotesize 48} \tabularnewline
78 & igraph\_igraph\_minimum\_size\_separators & 12 & \cellcolor{green!0}{\large 0}/{\footnotesize 40} & \cellcolor{green!0}{\large 0}/{\footnotesize 40} & \cellcolor{green!0}{\large 0}/{\footnotesize 40} & \cellcolor{green!0}{\large 0}/{\footnotesize 40} & \cellcolor{green!0}{\large 0}/{\footnotesize 43} & \cellcolor{green!0}{\large 0}/{\footnotesize 44} \tabularnewline
\rowcolor{black!10} 79 & pjsip\_pjsip\_parse\_msg & 12 & \cellcolor{green!0}{\large 0}/{\footnotesize 40} & \cellcolor{green!0}{\large 0}/{\footnotesize 40} & \cellcolor{green!0}{\large 0}/{\footnotesize 40} & \cellcolor{green!0}{\large 0}/{\footnotesize 40} & \cellcolor{green!0}{\large 0}/{\footnotesize 40} & \cellcolor{green!0}{\large 0}/{\footnotesize 47} \tabularnewline
80 & igraph\_igraph\_automorphism\_group & 13 & \cellcolor{green!0}{\large 0}/{\footnotesize 40} & \cellcolor{green!0}{\large 0}/{\footnotesize 40} & \cellcolor{green!0}{\large 0}/{\footnotesize 40} & \cellcolor{green!0}{\large 0}/{\footnotesize 40} & \cellcolor{green!0}{\large 0}/{\footnotesize 43} & \cellcolor{green!0}{\large 0}/{\footnotesize 44} \tabularnewline
\rowcolor{black!10} 81 & libmodbus\_modbus\_read\_bits & 15 & \cellcolor{green!0}{\large 0}/{\footnotesize 40} & \cellcolor{green!0}{\large 0}/{\footnotesize 40} & \cellcolor{green!0}{\large 0}/{\footnotesize 40} & \cellcolor{green!0}{\large 0}/{\footnotesize 40} & \cellcolor{green!0}{\large 0}/{\footnotesize 44} & \cellcolor{green!0}{\large 0}/{\footnotesize 41} \tabularnewline
82 & libmodbus\_modbus\_read\_registers & 15 & \cellcolor{green!0}{\large 0}/{\footnotesize 40} & \cellcolor{green!0}{\large 0}/{\footnotesize 40} & \cellcolor{green!0}{\large 0}/{\footnotesize 40} & \cellcolor{green!0}{\large 0}/{\footnotesize 40} & \cellcolor{green!0}{\large 0}/{\footnotesize 42} & \cellcolor{green!0}{\large 0}/{\footnotesize 44} \tabularnewline
\rowcolor{black!10} 83 & civetweb\_mg\_get\_response & 17 & \cellcolor{green!0}{\large 0}/{\footnotesize 40} & \cellcolor{green!0}{\large 0}/{\footnotesize 40} & \cellcolor{green!0}{\large 0}/{\footnotesize 40} & \cellcolor{green!0}{\large 0}/{\footnotesize 40} & \cellcolor{green!0}{\large 0}/{\footnotesize 42} & \cellcolor{green!0}{\large 0}/{\footnotesize 43} \tabularnewline
84 & bind9\_dns\_master\_loadbuffer & 20 & \cellcolor{green!0}{\large 0}/{\footnotesize 40} & \cellcolor{green!0}{\large 0}/{\footnotesize 40} & \cellcolor{green!0}{\large -}{\tiny -} & \cellcolor{green!0}{\large 0}/{\footnotesize 40} & \cellcolor{green!0}{\large 0}/{\footnotesize 42} & \cellcolor{green!0}{\large 0}/{\footnotesize 43} \tabularnewline
\rowcolor{black!10} 85 & libmodbus\_modbus\_receive & 33 & \cellcolor{green!0}{\large 0}/{\footnotesize 40} & \cellcolor{green!0}{\large 0}/{\footnotesize 40} & \cellcolor{green!0}{\large 0}/{\footnotesize 40} & \cellcolor{green!0}{\large 0}/{\footnotesize 40} & \cellcolor{green!0}{\large 0}/{\footnotesize 45} & \cellcolor{green!0}{\large 0}/{\footnotesize 42} \tabularnewline
86 & tmux\_input\_parse\_buffer & 42 & \cellcolor{green!0}{\large 0}/{\footnotesize 40} & \cellcolor{green!0}{\large 0}/{\footnotesize 40} & \cellcolor{green!0}{\large -}{\tiny -} & \cellcolor{green!0}{\large 0}/{\footnotesize 40} & \cellcolor{green!0}{\large 0}/{\footnotesize 44} & \cellcolor{green!0}{\large 0}/{\footnotesize 41} \tabularnewline

\bottomrule
%\end{tabular}
%}
%\end{table*}
\end{xltabular}
}
\twocolumn



% model: codellama-34b-instruct, temp: 0.0

\onecolumn
{\small %
\begin{xltabular}[h]{\textwidth}{ccccccccc}
%\begin{table*}[!t]
%\centering
\caption{Evaluation Result of model codellama-34b-instruct with temperature 0.0.} \\
%\resizebox{1.0\linewidth}{!}{
%\begin{tabular}{cccccccccc}
\toprule
Index & Question & Score & NAIVE-40 & BACTX-40 & DOCTX-40 & UGCTX-40 & BA-ITER-40 & ALL-ITER-40 \tabularnewline
\midrule
\rowcolor{black!10} 1 & coturn\_stun\_is\_command\_message\_full\_check\_str & 1 & \cellcolor{green!0}{\large 0}/{\footnotesize 40} & \cellcolor{green!0}{\large 0}/{\footnotesize 40} & \cellcolor{green!0}{\large -}{\tiny -} & \cellcolor{green!0}{\large 0}/{\footnotesize 40} & \cellcolor{green!50}{\large 39}/{\footnotesize 80} & \cellcolor{green!20}{\large 10}/{\footnotesize 66} \tabularnewline
2 & kamailio\_parse\_uri & 1 & \cellcolor{green!0}{\large 0}/{\footnotesize 40} & \cellcolor{green!0}{\large 0}/{\footnotesize 40} & \cellcolor{green!0}{\large -}{\tiny -} & \cellcolor{green!0}{\large 0}/{\footnotesize 40} & \cellcolor{green!50}{\large 40}/{\footnotesize 80} & \cellcolor{green!10}{\large 6}/{\footnotesize 56} \tabularnewline
\rowcolor{black!10} 3 & coturn\_stun\_check\_message\_integrity\_str & 2 & \cellcolor{green!0}{\large 0}/{\footnotesize 40} & \cellcolor{green!0}{\large 0}/{\footnotesize 40} & \cellcolor{green!0}{\large -}{\tiny -} & \cellcolor{green!0}{\large 0}/{\footnotesize 40} & \cellcolor{green!0}{\large 0}/{\footnotesize 157} & \cellcolor{green!0}{\large 0}/{\footnotesize 57} \tabularnewline
4 & libiec61850\_MmsValue\_decodeMmsData & 2 & \cellcolor{green!0}{\large 0}/{\footnotesize 40} & \cellcolor{green!0}{\large 0}/{\footnotesize 40} & \cellcolor{green!0}{\large 0}/{\footnotesize 40} & \cellcolor{green!0}{\large 0}/{\footnotesize 40} & \cellcolor{green!50}{\large 40}/{\footnotesize 80} & \cellcolor{green!10}{\large 2}/{\footnotesize 61} \tabularnewline
\rowcolor{black!10} 5 & md4c\_md\_html & 2 & \cellcolor{green!0}{\large 0}/{\footnotesize 40} & \cellcolor{green!0}{\large 0}/{\footnotesize 40} & \cellcolor{green!0}{\large 0}/{\footnotesize 40} & \cellcolor{green!0}{\large 0}/{\footnotesize 40} & \cellcolor{green!0}{\large 0}/{\footnotesize 80} & \cellcolor{green!0}{\large 0}/{\footnotesize 54} \tabularnewline
6 & spdk\_spdk\_json\_parse & 2 & \cellcolor{green!0}{\large 0}/{\footnotesize 40} & \cellcolor{green!0}{\large 0}/{\footnotesize 40} & \cellcolor{green!0}{\large -}{\tiny -} & \cellcolor{green!0}{\large 0}/{\footnotesize 40} & \cellcolor{green!0}{\large 0}/{\footnotesize 40} & \cellcolor{green!0}{\large 0}/{\footnotesize 40} \tabularnewline
\rowcolor{black!10} 7 & croaring\_roaring\_bitmap\_portable\_deserialize\_safe & 3 & \cellcolor{green!0}{\large 0}/{\footnotesize 40} & \cellcolor{green!0}{\large 0}/{\footnotesize 40} & \cellcolor{green!0}{\large 0}/{\footnotesize 40} & \cellcolor{green!0}{\large 0}/{\footnotesize 40} & \cellcolor{green!50}{\large 40}/{\footnotesize 80} & \cellcolor{green!20}{\large 10}/{\footnotesize 56} \tabularnewline
8 & lua\_luaL\_loadbufferx & 3 & \cellcolor{green!0}{\large 0}/{\footnotesize 40} & \cellcolor{green!0}{\large 0}/{\footnotesize 40} & \cellcolor{green!0}{\large 0}/{\footnotesize 40} & \cellcolor{green!0}{\large 0}/{\footnotesize 40} & \cellcolor{green!0}{\large 0}/{\footnotesize 80} & \cellcolor{green!10}{\large 8}/{\footnotesize 75} \tabularnewline
\rowcolor{black!10} 9 & w3m\_wc\_Str\_conv\_with\_detect & 3 & \cellcolor{green!0}{\large 0}/{\footnotesize 40} & \cellcolor{green!0}{\large 0}/{\footnotesize 40} & \cellcolor{green!0}{\large -}{\tiny -} & \cellcolor{green!0}{\large 0}/{\footnotesize 40} & \cellcolor{green!0}{\large 0}/{\footnotesize 200} & \cellcolor{green!0}{\large 0}/{\footnotesize 96} \tabularnewline
10 & bind9\_dns\_name\_fromwire & 4 & \cellcolor{green!0}{\large 0}/{\footnotesize 40} & \cellcolor{green!0}{\large 0}/{\footnotesize 40} & \cellcolor{green!0}{\large -}{\tiny -} & \cellcolor{green!0}{\large 0}/{\footnotesize 40} & \cellcolor{green!0}{\large 0}/{\footnotesize 40} & \cellcolor{green!0}{\large 0}/{\footnotesize 43} \tabularnewline
\rowcolor{black!10} 11 & gdk-pixbuf\_gdk\_pixbuf\_animation\_new\_from\_file & 4 & \cellcolor{green!0}{\large 0}/{\footnotesize 40} & \cellcolor{green!0}{\large 0}/{\footnotesize 40} & \cellcolor{green!0}{\large 0}/{\footnotesize 40} & \cellcolor{green!0}{\large 0}/{\footnotesize 40} & \cellcolor{green!0}{\large 0}/{\footnotesize 115} & \cellcolor{green!0}{\large 0}/{\footnotesize 50} \tabularnewline
12 & gdk-pixbuf\_gdk\_pixbuf\_new\_from\_data & 4 & \cellcolor{green!0}{\large 0}/{\footnotesize 40} & \cellcolor{green!0}{\large 0}/{\footnotesize 40} & \cellcolor{green!0}{\large 0}/{\footnotesize 40} & \cellcolor{green!0}{\large 0}/{\footnotesize 40} & \cellcolor{green!0}{\large 0}/{\footnotesize 170} & \cellcolor{green!0}{\large 0}/{\footnotesize 52} \tabularnewline
\rowcolor{black!10} 13 & gdk-pixbuf\_gdk\_pixbuf\_new\_from\_file & 4 & \cellcolor{green!0}{\large 0}/{\footnotesize 40} & \cellcolor{green!0}{\large 0}/{\footnotesize 40} & \cellcolor{green!0}{\large 0}/{\footnotesize 40} & \cellcolor{green!0}{\large 0}/{\footnotesize 40} & \cellcolor{green!0}{\large 0}/{\footnotesize 180} & \cellcolor{green!0}{\large 0}/{\footnotesize 53} \tabularnewline
14 & gdk-pixbuf\_gdk\_pixbuf\_new\_from\_stream & 4 & \cellcolor{green!0}{\large 0}/{\footnotesize 40} & \cellcolor{green!0}{\large 0}/{\footnotesize 40} & \cellcolor{green!0}{\large 0}/{\footnotesize 40} & \cellcolor{green!0}{\large 0}/{\footnotesize 40} & \cellcolor{green!40}{\large 32}/{\footnotesize 80} & \cellcolor{green!10}{\large 4}/{\footnotesize 54} \tabularnewline
\rowcolor{black!10} 15 & gpac\_gf\_isom\_open\_file & 4 & \cellcolor{green!0}{\large 0}/{\footnotesize 40} & \cellcolor{green!0}{\large 0}/{\footnotesize 40} & \cellcolor{green!0}{\large -}{\tiny -} & \cellcolor{green!0}{\large 0}/{\footnotesize 40} & \cellcolor{green!0}{\large 0}/{\footnotesize 200} & \cellcolor{green!0}{\large 0}/{\footnotesize 88} \tabularnewline
16 & libbpf\_bpf\_object\_\_open\_mem & 4 & \cellcolor{green!0}{\large 0}/{\footnotesize 40} & \cellcolor{green!0}{\large 0}/{\footnotesize 40} & \cellcolor{green!0}{\large 0}/{\footnotesize 40} & \cellcolor{green!0}{\large 0}/{\footnotesize 40} & \cellcolor{green!0}{\large 0}/{\footnotesize 80} & \cellcolor{green!10}{\large 1}/{\footnotesize 55} \tabularnewline
\rowcolor{black!10} 17 & libpg\_query\_pg\_query\_parse & 4 & \cellcolor{green!0}{\large 0}/{\footnotesize 40} & \cellcolor{green!0}{\large 0}/{\footnotesize 40} & \cellcolor{green!0}{\large -}{\tiny -} & \cellcolor{green!0}{\large 0}/{\footnotesize 40} & \cellcolor{green!0}{\large 0}/{\footnotesize 200} & \cellcolor{green!0}{\large 0}/{\footnotesize 82} \tabularnewline
18 & libucl\_ucl\_parser\_add\_string & 4 & \cellcolor{green!0}{\large 0}/{\footnotesize 40} & \cellcolor{green!0}{\large 0}/{\footnotesize 40} & \cellcolor{green!0}{\large 0}/{\footnotesize 40} & \cellcolor{green!0}{\large 0}/{\footnotesize 40} & \cellcolor{green!0}{\large 0}/{\footnotesize 80} & \cellcolor{green!10}{\large 1}/{\footnotesize 51} \tabularnewline
\rowcolor{black!10} 19 & oniguruma\_onig\_new & 4 & \cellcolor{green!0}{\large 0}/{\footnotesize 40} & \cellcolor{green!0}{\large 0}/{\footnotesize 40} & \cellcolor{green!0}{\large 0}/{\footnotesize 40} & \cellcolor{green!0}{\large 0}/{\footnotesize 40} & \cellcolor{green!0}{\large 0}/{\footnotesize 145} & \cellcolor{green!10}{\large 1}/{\footnotesize 56} \tabularnewline
20 & pupnp\_ixmlLoadDocumentEx & 4 & \cellcolor{green!0}{\large 0}/{\footnotesize 40} & \cellcolor{green!0}{\large 0}/{\footnotesize 40} & \cellcolor{green!0}{\large 0}/{\footnotesize 40} & \cellcolor{green!0}{\large 0}/{\footnotesize 40} & \cellcolor{green!0}{\large 0}/{\footnotesize 80} & \cellcolor{green!0}{\large 0}/{\footnotesize 114} \tabularnewline
\rowcolor{black!10} 21 & gdk-pixbuf\_gdk\_pixbuf\_new\_from\_file\_at\_scale & 5 & \cellcolor{green!0}{\large 0}/{\footnotesize 40} & \cellcolor{green!0}{\large 0}/{\footnotesize 40} & \cellcolor{green!0}{\large 0}/{\footnotesize 40} & \cellcolor{green!0}{\large 0}/{\footnotesize 40} & \cellcolor{green!0}{\large 0}/{\footnotesize 80} & \cellcolor{green!0}{\large 0}/{\footnotesize 45} \tabularnewline
22 & inchi\_GetINCHIKeyFromINCHI & 5 & \cellcolor{green!0}{\large 0}/{\footnotesize 40} & \cellcolor{green!0}{\large 0}/{\footnotesize 40} & \cellcolor{green!0}{\large 0}/{\footnotesize 40} & \cellcolor{green!0}{\large 0}/{\footnotesize 40} & \cellcolor{green!0}{\large 0}/{\footnotesize 80} & \cellcolor{green!10}{\large 3}/{\footnotesize 66} \tabularnewline
\rowcolor{black!10} 23 & libdwarf\_dwarf\_init\_b & 5 & \cellcolor{green!0}{\large 0}/{\footnotesize 40} & \cellcolor{green!0}{\large 0}/{\footnotesize 40} & \cellcolor{green!0}{\large 0}/{\footnotesize 40} & \cellcolor{green!0}{\large 0}/{\footnotesize 40} & \cellcolor{green!0}{\large 0}/{\footnotesize 187} & \cellcolor{green!0}{\large 0}/{\footnotesize 46} \tabularnewline
24 & libdwarf\_dwarf\_init\_path & 5 & \cellcolor{green!0}{\large 0}/{\footnotesize 40} & \cellcolor{green!0}{\large 0}/{\footnotesize 40} & \cellcolor{green!0}{\large 0}/{\footnotesize 40} & \cellcolor{green!0}{\large 0}/{\footnotesize 40} & \cellcolor{green!0}{\large 0}/{\footnotesize 200} & \cellcolor{green!0}{\large 0}/{\footnotesize 53} \tabularnewline
\rowcolor{black!10} 25 & liblouis\_lou\_compileString & 5 & \cellcolor{green!0}{\large 0}/{\footnotesize 40} & \cellcolor{green!0}{\large 0}/{\footnotesize 40} & \cellcolor{green!0}{\large 0}/{\footnotesize 40} & \cellcolor{green!0}{\large 0}/{\footnotesize 40} & \cellcolor{green!0}{\large 0}/{\footnotesize 200} & \cellcolor{green!0}{\large 0}/{\footnotesize 77} \tabularnewline
26 & selinux\_cil\_compile & 5 & \cellcolor{green!0}{\large 0}/{\footnotesize 40} & \cellcolor{green!0}{\large 0}/{\footnotesize 40} & \cellcolor{green!0}{\large -}{\tiny -} & \cellcolor{green!0}{\large 0}/{\footnotesize 40} & \cellcolor{green!0}{\large 0}/{\footnotesize 200} & \cellcolor{green!0}{\large 0}/{\footnotesize 71} \tabularnewline
\rowcolor{black!10} 27 & bind9\_dns\_name\_fromtext & 6 & \cellcolor{green!0}{\large 0}/{\footnotesize 40} & \cellcolor{green!0}{\large 0}/{\footnotesize 40} & \cellcolor{green!0}{\large -}{\tiny -} & \cellcolor{green!0}{\large 0}/{\footnotesize 40} & \cellcolor{green!0}{\large 0}/{\footnotesize 120} & \cellcolor{green!0}{\large 0}/{\footnotesize 59} \tabularnewline
28 & bind9\_dns\_rdata\_fromwire & 6 & \cellcolor{green!0}{\large 0}/{\footnotesize 40} & \cellcolor{green!0}{\large 0}/{\footnotesize 40} & \cellcolor{green!0}{\large -}{\tiny -} & \cellcolor{green!0}{\large 0}/{\footnotesize 40} & \cellcolor{green!0}{\large 0}/{\footnotesize 120} & \cellcolor{green!0}{\large 0}/{\footnotesize 54} \tabularnewline
\rowcolor{black!10} 29 & coturn\_stun\_is\_binding\_response & 6 & \cellcolor{green!0}{\large 0}/{\footnotesize 40} & \cellcolor{green!0}{\large 0}/{\footnotesize 40} & \cellcolor{green!0}{\large -}{\tiny -} & \cellcolor{green!0}{\large 0}/{\footnotesize 40} & \cellcolor{green!0}{\large 0}/{\footnotesize 200} & \cellcolor{green!0}{\large 0}/{\footnotesize 77} \tabularnewline
30 & coturn\_stun\_is\_command\_message & 6 & \cellcolor{green!0}{\large 0}/{\footnotesize 40} & \cellcolor{green!0}{\large 0}/{\footnotesize 40} & \cellcolor{green!0}{\large 0}/{\footnotesize 40} & \cellcolor{green!0}{\large 0}/{\footnotesize 40} & \cellcolor{green!0}{\large 0}/{\footnotesize 194} & \cellcolor{green!0}{\large 0}/{\footnotesize 59} \tabularnewline
\rowcolor{black!10} 31 & coturn\_stun\_is\_response & 6 & \cellcolor{green!0}{\large 0}/{\footnotesize 40} & \cellcolor{green!0}{\large 0}/{\footnotesize 40} & \cellcolor{green!0}{\large -}{\tiny -} & \cellcolor{green!0}{\large 0}/{\footnotesize 40} & \cellcolor{green!0}{\large 0}/{\footnotesize 200} & \cellcolor{green!0}{\large 0}/{\footnotesize 65} \tabularnewline
32 & coturn\_stun\_is\_success\_response & 6 & \cellcolor{green!0}{\large 0}/{\footnotesize 40} & \cellcolor{green!0}{\large 0}/{\footnotesize 40} & \cellcolor{green!0}{\large -}{\tiny -} & \cellcolor{green!0}{\large 0}/{\footnotesize 40} & \cellcolor{green!0}{\large 0}/{\footnotesize 146} & \cellcolor{green!0}{\large 0}/{\footnotesize 62} \tabularnewline
\rowcolor{black!10} 33 & hiredis\_redisFormatCommand & 6 & \cellcolor{green!0}{\large 0}/{\footnotesize 40} & \cellcolor{green!0}{\large 0}/{\footnotesize 40} & \cellcolor{green!0}{\large -}{\tiny -} & \cellcolor{green!0}{\large 0}/{\footnotesize 40} & \cellcolor{green!10}{\large 4}/{\footnotesize 128} & \cellcolor{green!10}{\large 2}/{\footnotesize 71} \tabularnewline
34 & igraph\_igraph\_read\_graph\_dl & 6 & \cellcolor{green!0}{\large 0}/{\footnotesize 40} & \cellcolor{green!0}{\large 0}/{\footnotesize 40} & \cellcolor{green!0}{\large 0}/{\footnotesize 40} & \cellcolor{green!0}{\large 0}/{\footnotesize 40} & \cellcolor{green!0}{\large 0}/{\footnotesize 200} & \cellcolor{green!0}{\large 0}/{\footnotesize 44} \tabularnewline
\rowcolor{black!10} 35 & igraph\_igraph\_read\_graph\_edgelist & 6 & \cellcolor{green!0}{\large 0}/{\footnotesize 40} & \cellcolor{green!0}{\large 0}/{\footnotesize 40} & \cellcolor{green!0}{\large 0}/{\footnotesize 40} & \cellcolor{green!0}{\large 0}/{\footnotesize 40} & \cellcolor{green!0}{\large 0}/{\footnotesize 200} & \cellcolor{green!0}{\large 0}/{\footnotesize 65} \tabularnewline
36 & igraph\_igraph\_read\_graph\_gml & 6 & \cellcolor{green!0}{\large 0}/{\footnotesize 40} & \cellcolor{green!0}{\large 0}/{\footnotesize 40} & \cellcolor{green!0}{\large 0}/{\footnotesize 40} & \cellcolor{green!0}{\large 0}/{\footnotesize 40} & \cellcolor{green!0}{\large 0}/{\footnotesize 200} & \cellcolor{green!0}{\large 0}/{\footnotesize 62} \tabularnewline
\rowcolor{black!10} 37 & igraph\_igraph\_read\_graph\_graphdb & 6 & \cellcolor{green!0}{\large 0}/{\footnotesize 40} & \cellcolor{green!0}{\large 0}/{\footnotesize 40} & \cellcolor{green!0}{\large 0}/{\footnotesize 40} & \cellcolor{green!0}{\large 0}/{\footnotesize 40} & \cellcolor{green!0}{\large 0}/{\footnotesize 200} & \cellcolor{green!0}{\large 0}/{\footnotesize 55} \tabularnewline
38 & igraph\_igraph\_read\_graph\_graphml & 6 & \cellcolor{green!0}{\large 0}/{\footnotesize 40} & \cellcolor{green!0}{\large 0}/{\footnotesize 40} & \cellcolor{green!0}{\large 0}/{\footnotesize 40} & \cellcolor{green!0}{\large 0}/{\footnotesize 40} & \cellcolor{green!0}{\large 0}/{\footnotesize 200} & \cellcolor{green!0}{\large 0}/{\footnotesize 56} \tabularnewline
\rowcolor{black!10} 39 & igraph\_igraph\_read\_graph\_lgl & 6 & \cellcolor{green!0}{\large 0}/{\footnotesize 40} & \cellcolor{green!0}{\large 0}/{\footnotesize 40} & \cellcolor{green!0}{\large 0}/{\footnotesize 40} & \cellcolor{green!0}{\large 0}/{\footnotesize 40} & \cellcolor{green!0}{\large 0}/{\footnotesize 200} & \cellcolor{green!0}{\large 0}/{\footnotesize 73} \tabularnewline
40 & igraph\_igraph\_read\_graph\_pajek & 6 & \cellcolor{green!0}{\large 0}/{\footnotesize 40} & \cellcolor{green!0}{\large 0}/{\footnotesize 40} & \cellcolor{green!0}{\large 0}/{\footnotesize 40} & \cellcolor{green!0}{\large 0}/{\footnotesize 40} & \cellcolor{green!0}{\large 0}/{\footnotesize 200} & \cellcolor{green!0}{\large 0}/{\footnotesize 78} \tabularnewline
\rowcolor{black!10} 41 & inchi\_GetINCHIfromINCHI & 6 & \cellcolor{green!0}{\large 0}/{\footnotesize 40} & \cellcolor{green!0}{\large 0}/{\footnotesize 40} & \cellcolor{green!0}{\large 0}/{\footnotesize 40} & \cellcolor{green!0}{\large 0}/{\footnotesize 40} & \cellcolor{green!0}{\large 0}/{\footnotesize 174} & \cellcolor{green!0}{\large 0}/{\footnotesize 93} \tabularnewline
42 & inchi\_GetStructFromINCHI & 6 & \cellcolor{green!0}{\large 0}/{\footnotesize 40} & \cellcolor{green!0}{\large 0}/{\footnotesize 40} & \cellcolor{green!0}{\large 0}/{\footnotesize 40} & \cellcolor{green!0}{\large 0}/{\footnotesize 40} & \cellcolor{green!0}{\large 0}/{\footnotesize 200} & \cellcolor{green!0}{\large 0}/{\footnotesize 89} \tabularnewline
\rowcolor{black!10} 43 & kamailio\_parse\_msg & 6 & \cellcolor{green!0}{\large 0}/{\footnotesize 40} & \cellcolor{green!0}{\large 0}/{\footnotesize 40} & \cellcolor{green!0}{\large -}{\tiny -} & \cellcolor{green!0}{\large 0}/{\footnotesize 40} & \cellcolor{green!0}{\large 0}/{\footnotesize 123} & \cellcolor{green!10}{\large 1}/{\footnotesize 62} \tabularnewline
44 & libyang\_lys\_parse\_mem & 6 & \cellcolor{green!0}{\large 0}/{\footnotesize 40} & \cellcolor{green!0}{\large 0}/{\footnotesize 40} & \cellcolor{green!0}{\large 0}/{\footnotesize 40} & \cellcolor{green!0}{\large 0}/{\footnotesize 40} & \cellcolor{green!0}{\large 0}/{\footnotesize 158} & \cellcolor{green!10}{\large 1}/{\footnotesize 61} \tabularnewline
\rowcolor{black!10} 45 & proftpd\_pr\_json\_object\_from\_text & 6 & \cellcolor{green!0}{\large 0}/{\footnotesize 40} & \cellcolor{green!0}{\large 0}/{\footnotesize 40} & \cellcolor{green!0}{\large -}{\tiny -} & \cellcolor{green!0}{\large 0}/{\footnotesize 40} & \cellcolor{green!0}{\large 0}/{\footnotesize 91} & \cellcolor{green!0}{\large 0}/{\footnotesize 54} \tabularnewline
46 & selinux\_policydb\_read & 6 & \cellcolor{green!0}{\large 0}/{\footnotesize 40} & \cellcolor{green!0}{\large 0}/{\footnotesize 40} & \cellcolor{green!0}{\large -}{\tiny -} & \cellcolor{green!0}{\large 0}/{\footnotesize 40} & \cellcolor{green!0}{\large 0}/{\footnotesize 200} & \cellcolor{green!0}{\large 0}/{\footnotesize 74} \tabularnewline
\rowcolor{black!10} 47 & kamailio\_get\_src\_address\_socket & 7 & \cellcolor{green!0}{\large 0}/{\footnotesize 40} & \cellcolor{green!0}{\large 0}/{\footnotesize 40} & \cellcolor{green!0}{\large 0}/{\footnotesize 40} & \cellcolor{green!0}{\large 0}/{\footnotesize 40} & \cellcolor{green!0}{\large 0}/{\footnotesize 112} & \cellcolor{green!0}{\large 0}/{\footnotesize 52} \tabularnewline
48 & kamailio\_get\_src\_uri & 7 & \cellcolor{green!0}{\large 0}/{\footnotesize 40} & \cellcolor{green!0}{\large 0}/{\footnotesize 40} & \cellcolor{green!0}{\large 0}/{\footnotesize 40} & \cellcolor{green!0}{\large 0}/{\footnotesize 40} & \cellcolor{green!0}{\large 0}/{\footnotesize 177} & \cellcolor{green!0}{\large 0}/{\footnotesize 83} \tabularnewline
\rowcolor{black!10} 49 & kamailio\_parse\_content\_disposition & 7 & \cellcolor{green!0}{\large 0}/{\footnotesize 40} & \cellcolor{green!0}{\large 0}/{\footnotesize 40} & \cellcolor{green!0}{\large 0}/{\footnotesize 40} & \cellcolor{green!0}{\large 0}/{\footnotesize 40} & \cellcolor{green!0}{\large 0}/{\footnotesize 146} & \cellcolor{green!0}{\large 0}/{\footnotesize 59} \tabularnewline
50 & kamailio\_parse\_diversion\_header & 7 & \cellcolor{green!0}{\large 0}/{\footnotesize 40} & \cellcolor{green!0}{\large 0}/{\footnotesize 40} & \cellcolor{green!0}{\large 0}/{\footnotesize 40} & \cellcolor{green!0}{\large 0}/{\footnotesize 40} & \cellcolor{green!0}{\large 0}/{\footnotesize 120} & \cellcolor{green!0}{\large 0}/{\footnotesize 77} \tabularnewline
\rowcolor{black!10} 51 & kamailio\_parse\_from\_header & 7 & \cellcolor{green!0}{\large 0}/{\footnotesize 40} & \cellcolor{green!0}{\large 0}/{\footnotesize 40} & \cellcolor{green!0}{\large -}{\tiny -} & \cellcolor{green!0}{\large 0}/{\footnotesize 40} & \cellcolor{green!0}{\large 0}/{\footnotesize 192} & \cellcolor{green!0}{\large 0}/{\footnotesize 61} \tabularnewline
52 & kamailio\_parse\_from\_uri & 7 & \cellcolor{green!0}{\large 0}/{\footnotesize 40} & \cellcolor{green!0}{\large 0}/{\footnotesize 40} & \cellcolor{green!0}{\large -}{\tiny -} & \cellcolor{green!0}{\large 0}/{\footnotesize 40} & \cellcolor{green!0}{\large 0}/{\footnotesize 200} & \cellcolor{green!0}{\large 0}/{\footnotesize 63} \tabularnewline
\rowcolor{black!10} 53 & kamailio\_parse\_headers & 7 & \cellcolor{green!0}{\large 0}/{\footnotesize 40} & \cellcolor{green!0}{\large 0}/{\footnotesize 40} & \cellcolor{green!0}{\large -}{\tiny -} & \cellcolor{green!0}{\large 0}/{\footnotesize 40} & \cellcolor{green!0}{\large 0}/{\footnotesize 146} & \cellcolor{green!0}{\large 0}/{\footnotesize 50} \tabularnewline
54 & kamailio\_parse\_identityinfo\_header & 7 & \cellcolor{green!0}{\large 0}/{\footnotesize 40} & \cellcolor{green!0}{\large 0}/{\footnotesize 40} & \cellcolor{green!0}{\large -}{\tiny -} & \cellcolor{green!0}{\large 0}/{\footnotesize 40} & \cellcolor{green!0}{\large 0}/{\footnotesize 124} & \cellcolor{green!0}{\large 0}/{\footnotesize 46} \tabularnewline
\rowcolor{black!10} 55 & kamailio\_parse\_pai\_header & 7 & \cellcolor{green!0}{\large 0}/{\footnotesize 40} & \cellcolor{green!0}{\large 0}/{\footnotesize 40} & \cellcolor{green!0}{\large -}{\tiny -} & \cellcolor{green!0}{\large 0}/{\footnotesize 40} & \cellcolor{green!0}{\large 0}/{\footnotesize 120} & \cellcolor{green!0}{\large 0}/{\footnotesize 63} \tabularnewline
56 & kamailio\_parse\_privacy & 7 & \cellcolor{green!0}{\large 0}/{\footnotesize 40} & \cellcolor{green!0}{\large 0}/{\footnotesize 40} & \cellcolor{green!0}{\large 0}/{\footnotesize 40} & \cellcolor{green!0}{\large 0}/{\footnotesize 40} & \cellcolor{green!0}{\large 0}/{\footnotesize 200} & \cellcolor{green!0}{\large 0}/{\footnotesize 78} \tabularnewline
\rowcolor{black!10} 57 & kamailio\_parse\_record\_route\_headers & 7 & \cellcolor{green!0}{\large 0}/{\footnotesize 40} & \cellcolor{green!0}{\large 0}/{\footnotesize 40} & \cellcolor{green!0}{\large -}{\tiny -} & \cellcolor{green!0}{\large 0}/{\footnotesize 40} & \cellcolor{green!0}{\large 0}/{\footnotesize 200} & \cellcolor{green!0}{\large 0}/{\footnotesize 100} \tabularnewline
58 & kamailio\_parse\_refer\_to\_header & 7 & \cellcolor{green!0}{\large 0}/{\footnotesize 40} & \cellcolor{green!0}{\large 0}/{\footnotesize 40} & \cellcolor{green!0}{\large -}{\tiny -} & \cellcolor{green!0}{\large 0}/{\footnotesize 40} & \cellcolor{green!0}{\large 0}/{\footnotesize 40} & \cellcolor{green!0}{\large 0}/{\footnotesize 47} \tabularnewline
\rowcolor{black!10} 59 & kamailio\_parse\_route\_headers & 7 & \cellcolor{green!0}{\large 0}/{\footnotesize 40} & \cellcolor{green!0}{\large 0}/{\footnotesize 40} & \cellcolor{green!0}{\large -}{\tiny -} & \cellcolor{green!0}{\large 0}/{\footnotesize 40} & \cellcolor{green!0}{\large 0}/{\footnotesize 200} & \cellcolor{green!0}{\large 0}/{\footnotesize 75} \tabularnewline
60 & kamailio\_parse\_to\_header & 7 & \cellcolor{green!0}{\large 0}/{\footnotesize 40} & \cellcolor{green!0}{\large 0}/{\footnotesize 40} & \cellcolor{green!0}{\large -}{\tiny -} & \cellcolor{green!0}{\large 0}/{\footnotesize 40} & \cellcolor{green!0}{\large 0}/{\footnotesize 160} & \cellcolor{green!0}{\large 0}/{\footnotesize 71} \tabularnewline
\rowcolor{black!10} 61 & kamailio\_parse\_to\_uri & 7 & \cellcolor{green!0}{\large 0}/{\footnotesize 40} & \cellcolor{green!0}{\large 0}/{\footnotesize 40} & \cellcolor{green!0}{\large -}{\tiny -} & \cellcolor{green!0}{\large 0}/{\footnotesize 40} & \cellcolor{green!0}{\large 0}/{\footnotesize 200} & \cellcolor{green!0}{\large 0}/{\footnotesize 62} \tabularnewline
62 & libyang\_lyd\_parse\_data\_mem & 7 & \cellcolor{green!0}{\large 0}/{\footnotesize 40} & \cellcolor{green!0}{\large 0}/{\footnotesize 40} & \cellcolor{green!0}{\large 0}/{\footnotesize 40} & \cellcolor{green!0}{\large 0}/{\footnotesize 40} & \cellcolor{green!0}{\large 0}/{\footnotesize 170} & \cellcolor{green!0}{\large 0}/{\footnotesize 48} \tabularnewline
\rowcolor{black!10} 63 & bind9\_dns\_message\_parse & 8 & \cellcolor{green!0}{\large 0}/{\footnotesize 40} & \cellcolor{green!0}{\large 0}/{\footnotesize 40} & \cellcolor{green!0}{\large -}{\tiny -} & \cellcolor{green!0}{\large 0}/{\footnotesize 40} & \cellcolor{green!0}{\large 0}/{\footnotesize 200} & \cellcolor{green!0}{\large 0}/{\footnotesize 70} \tabularnewline
64 & igraph\_igraph\_read\_graph\_ncol & 8 & \cellcolor{green!0}{\large 0}/{\footnotesize 40} & \cellcolor{green!0}{\large 0}/{\footnotesize 40} & \cellcolor{green!0}{\large 0}/{\footnotesize 40} & \cellcolor{green!0}{\large 0}/{\footnotesize 40} & \cellcolor{green!0}{\large 0}/{\footnotesize 200} & \cellcolor{green!0}{\large 0}/{\footnotesize 69} \tabularnewline
\rowcolor{black!10} 65 & pjsip\_pj\_json\_parse & 8 & \cellcolor{green!0}{\large 0}/{\footnotesize 40} & \cellcolor{green!0}{\large 0}/{\footnotesize 40} & \cellcolor{green!0}{\large 0}/{\footnotesize 40} & \cellcolor{green!0}{\large 0}/{\footnotesize 40} & \cellcolor{green!0}{\large 0}/{\footnotesize 80} & \cellcolor{green!0}{\large 0}/{\footnotesize 61} \tabularnewline
66 & pjsip\_pj\_xml\_parse & 8 & \cellcolor{green!0}{\large 0}/{\footnotesize 40} & \cellcolor{green!0}{\large 0}/{\footnotesize 40} & \cellcolor{green!0}{\large 0}/{\footnotesize 40} & \cellcolor{green!0}{\large 0}/{\footnotesize 40} & \cellcolor{green!0}{\large 0}/{\footnotesize 200} & \cellcolor{green!0}{\large 0}/{\footnotesize 56} \tabularnewline
\rowcolor{black!10} 67 & pjsip\_pjmedia\_sdp\_parse & 8 & \cellcolor{green!0}{\large 0}/{\footnotesize 40} & \cellcolor{green!0}{\large 0}/{\footnotesize 40} & \cellcolor{green!0}{\large 0}/{\footnotesize 40} & \cellcolor{green!0}{\large 0}/{\footnotesize 40} & \cellcolor{green!0}{\large 0}/{\footnotesize 170} & \cellcolor{green!0}{\large 0}/{\footnotesize 52} \tabularnewline
68 & quickjs\_lre\_compile & 8 & \cellcolor{green!0}{\large 0}/{\footnotesize 40} & \cellcolor{green!0}{\large 0}/{\footnotesize 40} & \cellcolor{green!0}{\large -}{\tiny -} & \cellcolor{green!0}{\large 0}/{\footnotesize 40} & \cellcolor{green!0}{\large 0}/{\footnotesize 120} & \cellcolor{green!0}{\large 0}/{\footnotesize 62} \tabularnewline
\rowcolor{black!10} 69 & bind9\_isc\_lex\_getmastertoken & 9 & \cellcolor{green!0}{\large 0}/{\footnotesize 40} & \cellcolor{green!0}{\large 0}/{\footnotesize 40} & \cellcolor{green!0}{\large -}{\tiny -} & \cellcolor{green!0}{\large 0}/{\footnotesize 40} & \cellcolor{green!0}{\large 0}/{\footnotesize 136} & \cellcolor{green!0}{\large 0}/{\footnotesize 56} \tabularnewline
70 & bind9\_isc\_lex\_gettoken & 9 & \cellcolor{green!0}{\large 0}/{\footnotesize 40} & \cellcolor{green!0}{\large 0}/{\footnotesize 40} & \cellcolor{green!0}{\large -}{\tiny -} & \cellcolor{green!0}{\large 0}/{\footnotesize 40} & \cellcolor{green!0}{\large 0}/{\footnotesize 200} & \cellcolor{green!0}{\large 0}/{\footnotesize 52} \tabularnewline
\rowcolor{black!10} 71 & quickjs\_JS\_Eval & 9 & \cellcolor{green!0}{\large 0}/{\footnotesize 40} & \cellcolor{green!0}{\large 0}/{\footnotesize 40} & \cellcolor{green!0}{\large -}{\tiny -} & \cellcolor{green!0}{\large 0}/{\footnotesize 40} & \cellcolor{green!0}{\large 0}/{\footnotesize 149} & \cellcolor{green!0}{\large 0}/{\footnotesize 61} \tabularnewline
72 & igraph\_igraph\_edge\_connectivity & 10 & \cellcolor{green!0}{\large 0}/{\footnotesize 40} & \cellcolor{green!0}{\large 0}/{\footnotesize 40} & \cellcolor{green!0}{\large 0}/{\footnotesize 40} & \cellcolor{green!0}{\large 0}/{\footnotesize 40} & \cellcolor{green!0}{\large 0}/{\footnotesize 120} & \cellcolor{green!0}{\large 0}/{\footnotesize 108} \tabularnewline
\rowcolor{black!10} 73 & pjsip\_pj\_stun\_msg\_decode & 10 & \cellcolor{green!0}{\large 0}/{\footnotesize 40} & \cellcolor{green!0}{\large 0}/{\footnotesize 40} & \cellcolor{green!0}{\large 0}/{\footnotesize 40} & \cellcolor{green!0}{\large 0}/{\footnotesize 40} & \cellcolor{green!0}{\large 0}/{\footnotesize 80} & \cellcolor{green!0}{\large 0}/{\footnotesize 44} \tabularnewline
74 & bind9\_dns\_message\_checksig & 11 & \cellcolor{green!0}{\large 0}/{\footnotesize 40} & \cellcolor{green!0}{\large 0}/{\footnotesize 40} & \cellcolor{green!0}{\large -}{\tiny -} & \cellcolor{green!0}{\large 0}/{\footnotesize 40} & \cellcolor{green!0}{\large 0}/{\footnotesize 200} & \cellcolor{green!0}{\large 0}/{\footnotesize 79} \tabularnewline
\rowcolor{black!10} 75 & libzip\_zip\_fread & 11 & \cellcolor{green!0}{\large 0}/{\footnotesize 40} & \cellcolor{green!0}{\large 0}/{\footnotesize 40} & \cellcolor{green!0}{\large 0}/{\footnotesize 40} & \cellcolor{green!0}{\large 0}/{\footnotesize 40} & \cellcolor{green!0}{\large 0}/{\footnotesize 120} & \cellcolor{green!0}{\large 0}/{\footnotesize 48} \tabularnewline
76 & bind9\_dns\_rdata\_fromtext & 12 & \cellcolor{green!0}{\large 0}/{\footnotesize 40} & \cellcolor{green!0}{\large 0}/{\footnotesize 40} & \cellcolor{green!0}{\large -}{\tiny -} & \cellcolor{green!0}{\large 0}/{\footnotesize 40} & \cellcolor{green!0}{\large 0}/{\footnotesize 40} & \cellcolor{green!0}{\large 0}/{\footnotesize 50} \tabularnewline
\rowcolor{black!10} 77 & igraph\_igraph\_all\_minimal\_st\_separators & 12 & \cellcolor{green!0}{\large 0}/{\footnotesize 40} & \cellcolor{green!0}{\large 0}/{\footnotesize 40} & \cellcolor{green!0}{\large 0}/{\footnotesize 40} & \cellcolor{green!0}{\large 0}/{\footnotesize 40} & \cellcolor{green!0}{\large 0}/{\footnotesize 120} & \cellcolor{green!0}{\large 0}/{\footnotesize 63} \tabularnewline
78 & igraph\_igraph\_minimum\_size\_separators & 12 & \cellcolor{green!0}{\large 0}/{\footnotesize 40} & \cellcolor{green!0}{\large 0}/{\footnotesize 40} & \cellcolor{green!0}{\large 0}/{\footnotesize 40} & \cellcolor{green!0}{\large 0}/{\footnotesize 40} & \cellcolor{green!0}{\large 0}/{\footnotesize 182} & \cellcolor{green!0}{\large 0}/{\footnotesize 79} \tabularnewline
\rowcolor{black!10} 79 & pjsip\_pjsip\_parse\_msg & 12 & \cellcolor{green!0}{\large 0}/{\footnotesize 40} & \cellcolor{green!0}{\large 0}/{\footnotesize 40} & \cellcolor{green!0}{\large 0}/{\footnotesize 40} & \cellcolor{green!0}{\large 0}/{\footnotesize 40} & \cellcolor{green!0}{\large 0}/{\footnotesize 80} & \cellcolor{green!0}{\large 0}/{\footnotesize 59} \tabularnewline
80 & igraph\_igraph\_automorphism\_group & 13 & \cellcolor{green!0}{\large 0}/{\footnotesize 40} & \cellcolor{green!0}{\large 0}/{\footnotesize 40} & \cellcolor{green!0}{\large 0}/{\footnotesize 40} & \cellcolor{green!0}{\large 0}/{\footnotesize 40} & \cellcolor{green!0}{\large 0}/{\footnotesize 113} & \cellcolor{green!0}{\large 0}/{\footnotesize 60} \tabularnewline
\rowcolor{black!10} 81 & libmodbus\_modbus\_read\_bits & 15 & \cellcolor{green!0}{\large 0}/{\footnotesize 40} & \cellcolor{green!0}{\large 0}/{\footnotesize 40} & \cellcolor{green!0}{\large 0}/{\footnotesize 40} & \cellcolor{green!0}{\large 0}/{\footnotesize 40} & \cellcolor{green!0}{\large 0}/{\footnotesize 151} & \cellcolor{green!0}{\large 0}/{\footnotesize 64} \tabularnewline
82 & libmodbus\_modbus\_read\_registers & 15 & \cellcolor{green!0}{\large 0}/{\footnotesize 40} & \cellcolor{green!0}{\large 0}/{\footnotesize 40} & \cellcolor{green!0}{\large 0}/{\footnotesize 40} & \cellcolor{green!0}{\large 0}/{\footnotesize 40} & \cellcolor{green!0}{\large 0}/{\footnotesize 80} & \cellcolor{green!0}{\large 0}/{\footnotesize 51} \tabularnewline
\rowcolor{black!10} 83 & civetweb\_mg\_get\_response & 17 & \cellcolor{green!0}{\large 0}/{\footnotesize 40} & \cellcolor{green!0}{\large 0}/{\footnotesize 40} & \cellcolor{green!0}{\large 0}/{\footnotesize 40} & \cellcolor{green!0}{\large 0}/{\footnotesize 40} & \cellcolor{green!0}{\large 0}/{\footnotesize 200} & \cellcolor{green!0}{\large 0}/{\footnotesize 48} \tabularnewline
84 & bind9\_dns\_master\_loadbuffer & 20 & \cellcolor{green!0}{\large 0}/{\footnotesize 40} & \cellcolor{green!0}{\large 0}/{\footnotesize 40} & \cellcolor{green!0}{\large -}{\tiny -} & \cellcolor{green!0}{\large 0}/{\footnotesize 40} & \cellcolor{green!0}{\large 0}/{\footnotesize 158} & \cellcolor{green!0}{\large 0}/{\footnotesize 79} \tabularnewline
\rowcolor{black!10} 85 & libmodbus\_modbus\_receive & 33 & \cellcolor{green!0}{\large 0}/{\footnotesize 40} & \cellcolor{green!0}{\large 0}/{\footnotesize 40} & \cellcolor{green!0}{\large 0}/{\footnotesize 40} & \cellcolor{green!0}{\large 0}/{\footnotesize 40} & \cellcolor{green!0}{\large 0}/{\footnotesize 40} & \cellcolor{green!0}{\large 0}/{\footnotesize 42} \tabularnewline
86 & tmux\_input\_parse\_buffer & 42 & \cellcolor{green!0}{\large 0}/{\footnotesize 40} & \cellcolor{green!0}{\large 0}/{\footnotesize 40} & \cellcolor{green!0}{\large -}{\tiny -} & \cellcolor{green!0}{\large 0}/{\footnotesize 40} & \cellcolor{green!0}{\large 0}/{\footnotesize 80} & \cellcolor{green!0}{\large 0}/{\footnotesize 85} \tabularnewline

\bottomrule
%\end{tabular}
%}
%\end{table*}
\end{xltabular}
}
\twocolumn



% model: codellama-34b-instruct, temp: 0.5

\onecolumn
{\small %
\begin{xltabular}[h]{\textwidth}{ccccccccc}
%\begin{table*}[!t]
%\centering
\caption{Evaluation Result of model codellama-34b-instruct with temperature 0.5.} \\
%\resizebox{1.0\linewidth}{!}{
%\begin{tabular}{cccccccccc}
\toprule
Index & Question & Score & NAIVE-40 & BACTX-40 & DOCTX-40 & UGCTX-40 & BA-ITER-40 & ALL-ITER-40 \tabularnewline
\midrule
\rowcolor{black!10} 1 & coturn\_stun\_is\_command\_message\_full\_check\_str & 1 & \cellcolor{green!0}{\large 0}/{\footnotesize 40} & \cellcolor{green!0}{\large 0}/{\footnotesize 40} & \cellcolor{green!0}{\large -}{\tiny -} & \cellcolor{green!20}{\large 6}/{\footnotesize 40} & \cellcolor{green!30}{\large 16}/{\footnotesize 73} & \cellcolor{green!30}{\large 15}/{\footnotesize 63} \tabularnewline
2 & kamailio\_parse\_uri & 1 & \cellcolor{green!0}{\large 0}/{\footnotesize 40} & \cellcolor{green!0}{\large 0}/{\footnotesize 40} & \cellcolor{green!0}{\large -}{\tiny -} & \cellcolor{green!0}{\large 0}/{\footnotesize 40} & \cellcolor{green!10}{\large 5}/{\footnotesize 98} & \cellcolor{green!10}{\large 1}/{\footnotesize 58} \tabularnewline
\rowcolor{black!10} 3 & coturn\_stun\_check\_message\_integrity\_str & 2 & \cellcolor{green!0}{\large 0}/{\footnotesize 40} & \cellcolor{green!0}{\large 0}/{\footnotesize 40} & \cellcolor{green!0}{\large -}{\tiny -} & \cellcolor{green!10}{\large 1}/{\footnotesize 40} & \cellcolor{green!10}{\large 7}/{\footnotesize 103} & \cellcolor{green!10}{\large 1}/{\footnotesize 58} \tabularnewline
4 & libiec61850\_MmsValue\_decodeMmsData & 2 & \cellcolor{green!0}{\large 0}/{\footnotesize 40} & \cellcolor{green!0}{\large 0}/{\footnotesize 40} & \cellcolor{green!0}{\large 0}/{\footnotesize 40} & \cellcolor{green!0}{\large 0}/{\footnotesize 40} & \cellcolor{green!20}{\large 21}/{\footnotesize 102} & \cellcolor{green!10}{\large 1}/{\footnotesize 54} \tabularnewline
\rowcolor{black!10} 5 & md4c\_md\_html & 2 & \cellcolor{green!0}{\large 0}/{\footnotesize 40} & \cellcolor{green!0}{\large 0}/{\footnotesize 40} & \cellcolor{green!0}{\large 0}/{\footnotesize 40} & \cellcolor{green!0}{\large 0}/{\footnotesize 40} & \cellcolor{green!10}{\large 2}/{\footnotesize 135} & \cellcolor{green!10}{\large 1}/{\footnotesize 75} \tabularnewline
6 & spdk\_spdk\_json\_parse & 2 & \cellcolor{green!0}{\large 0}/{\footnotesize 40} & \cellcolor{green!0}{\large 0}/{\footnotesize 40} & \cellcolor{green!0}{\large -}{\tiny -} & \cellcolor{green!0}{\large 0}/{\footnotesize 40} & \cellcolor{green!10}{\large 3}/{\footnotesize 76} & \cellcolor{green!0}{\large 0}/{\footnotesize 49} \tabularnewline
\rowcolor{black!10} 7 & croaring\_roaring\_bitmap\_portable\_deserialize\_safe & 3 & \cellcolor{green!0}{\large 0}/{\footnotesize 40} & \cellcolor{green!0}{\large 0}/{\footnotesize 40} & \cellcolor{green!0}{\large 0}/{\footnotesize 40} & \cellcolor{green!0}{\large 0}/{\footnotesize 40} & \cellcolor{green!30}{\large 23}/{\footnotesize 84} & \cellcolor{green!0}{\large 0}/{\footnotesize 53} \tabularnewline
8 & lua\_luaL\_loadbufferx & 3 & \cellcolor{green!0}{\large 0}/{\footnotesize 40} & \cellcolor{green!10}{\large 1}/{\footnotesize 40} & \cellcolor{green!10}{\large 1}/{\footnotesize 40} & \cellcolor{green!0}{\large 0}/{\footnotesize 40} & \cellcolor{green!30}{\large 23}/{\footnotesize 91} & \cellcolor{green!20}{\large 9}/{\footnotesize 67} \tabularnewline
\rowcolor{black!10} 9 & w3m\_wc\_Str\_conv\_with\_detect & 3 & \cellcolor{green!0}{\large 0}/{\footnotesize 40} & \cellcolor{green!0}{\large 0}/{\footnotesize 40} & \cellcolor{green!0}{\large -}{\tiny -} & \cellcolor{green!0}{\large 0}/{\footnotesize 40} & \cellcolor{green!0}{\large 0}/{\footnotesize 132} & \cellcolor{green!10}{\large 1}/{\footnotesize 92} \tabularnewline
10 & bind9\_dns\_name\_fromwire & 4 & \cellcolor{green!0}{\large 0}/{\footnotesize 40} & \cellcolor{green!0}{\large 0}/{\footnotesize 40} & \cellcolor{green!0}{\large -}{\tiny -} & \cellcolor{green!0}{\large 0}/{\footnotesize 40} & \cellcolor{green!0}{\large 0}/{\footnotesize 105} & \cellcolor{green!0}{\large 0}/{\footnotesize 51} \tabularnewline
\rowcolor{black!10} 11 & gdk-pixbuf\_gdk\_pixbuf\_animation\_new\_from\_file & 4 & \cellcolor{green!10}{\large 1}/{\footnotesize 40} & \cellcolor{green!0}{\large 0}/{\footnotesize 40} & \cellcolor{green!10}{\large 4}/{\footnotesize 40} & \cellcolor{green!0}{\large 0}/{\footnotesize 40} & \cellcolor{green!0}{\large 0}/{\footnotesize 95} & \cellcolor{green!0}{\large 0}/{\footnotesize 66} \tabularnewline
12 & gdk-pixbuf\_gdk\_pixbuf\_new\_from\_data & 4 & \cellcolor{green!0}{\large 0}/{\footnotesize 40} & \cellcolor{green!0}{\large 0}/{\footnotesize 40} & \cellcolor{green!20}{\large 7}/{\footnotesize 40} & \cellcolor{green!0}{\large 0}/{\footnotesize 40} & \cellcolor{green!20}{\large 11}/{\footnotesize 89} & \cellcolor{green!10}{\large 2}/{\footnotesize 56} \tabularnewline
\rowcolor{black!10} 13 & gdk-pixbuf\_gdk\_pixbuf\_new\_from\_file & 4 & \cellcolor{green!0}{\large 0}/{\footnotesize 40} & \cellcolor{green!0}{\large 0}/{\footnotesize 40} & \cellcolor{green!0}{\large 0}/{\footnotesize 40} & \cellcolor{green!0}{\large 0}/{\footnotesize 40} & \cellcolor{green!0}{\large 0}/{\footnotesize 106} & \cellcolor{green!0}{\large 0}/{\footnotesize 66} \tabularnewline
14 & gdk-pixbuf\_gdk\_pixbuf\_new\_from\_stream & 4 & \cellcolor{green!0}{\large 0}/{\footnotesize 40} & \cellcolor{green!0}{\large 0}/{\footnotesize 40} & \cellcolor{green!10}{\large 2}/{\footnotesize 40} & \cellcolor{green!0}{\large 0}/{\footnotesize 40} & \cellcolor{green!40}{\large 31}/{\footnotesize 79} & \cellcolor{green!10}{\large 5}/{\footnotesize 58} \tabularnewline
\rowcolor{black!10} 15 & gpac\_gf\_isom\_open\_file & 4 & \cellcolor{green!0}{\large 0}/{\footnotesize 40} & \cellcolor{green!0}{\large 0}/{\footnotesize 40} & \cellcolor{green!0}{\large -}{\tiny -} & \cellcolor{green!0}{\large 0}/{\footnotesize 40} & \cellcolor{green!0}{\large 0}/{\footnotesize 112} & \cellcolor{green!0}{\large 0}/{\footnotesize 60} \tabularnewline
16 & libbpf\_bpf\_object\_\_open\_mem & 4 & \cellcolor{green!0}{\large 0}/{\footnotesize 40} & \cellcolor{green!0}{\large 0}/{\footnotesize 40} & \cellcolor{green!10}{\large 1}/{\footnotesize 40} & \cellcolor{green!10}{\large 1}/{\footnotesize 40} & \cellcolor{green!10}{\large 9}/{\footnotesize 121} & \cellcolor{green!0}{\large 0}/{\footnotesize 58} \tabularnewline
\rowcolor{black!10} 17 & libpg\_query\_pg\_query\_parse & 4 & \cellcolor{green!0}{\large 0}/{\footnotesize 40} & \cellcolor{green!0}{\large 0}/{\footnotesize 40} & \cellcolor{green!0}{\large -}{\tiny -} & \cellcolor{green!0}{\large 0}/{\footnotesize 40} & \cellcolor{green!10}{\large 6}/{\footnotesize 124} & \cellcolor{green!10}{\large 1}/{\footnotesize 77} \tabularnewline
18 & libucl\_ucl\_parser\_add\_string & 4 & \cellcolor{green!0}{\large 0}/{\footnotesize 40} & \cellcolor{green!0}{\large 0}/{\footnotesize 40} & \cellcolor{green!0}{\large 0}/{\footnotesize 40} & \cellcolor{green!0}{\large 0}/{\footnotesize 40} & \cellcolor{green!10}{\large 9}/{\footnotesize 122} & \cellcolor{green!10}{\large 1}/{\footnotesize 62} \tabularnewline
\rowcolor{black!10} 19 & oniguruma\_onig\_new & 4 & \cellcolor{green!0}{\large 0}/{\footnotesize 40} & \cellcolor{green!0}{\large 0}/{\footnotesize 40} & \cellcolor{green!10}{\large 4}/{\footnotesize 40} & \cellcolor{green!0}{\large 0}/{\footnotesize 40} & \cellcolor{green!10}{\large 4}/{\footnotesize 99} & \cellcolor{green!10}{\large 1}/{\footnotesize 50} \tabularnewline
20 & pupnp\_ixmlLoadDocumentEx & 4 & \cellcolor{green!0}{\large 0}/{\footnotesize 40} & \cellcolor{green!0}{\large 0}/{\footnotesize 40} & \cellcolor{green!10}{\large 1}/{\footnotesize 40} & \cellcolor{green!0}{\large 0}/{\footnotesize 40} & \cellcolor{green!10}{\large 3}/{\footnotesize 154} & \cellcolor{green!0}{\large 0}/{\footnotesize 94} \tabularnewline
\rowcolor{black!10} 21 & gdk-pixbuf\_gdk\_pixbuf\_new\_from\_file\_at\_scale & 5 & \cellcolor{green!0}{\large 0}/{\footnotesize 40} & \cellcolor{green!0}{\large 0}/{\footnotesize 40} & \cellcolor{green!20}{\large 5}/{\footnotesize 40} & \cellcolor{green!10}{\large 1}/{\footnotesize 40} & \cellcolor{green!0}{\large 0}/{\footnotesize 82} & \cellcolor{green!10}{\large 1}/{\footnotesize 56} \tabularnewline
22 & inchi\_GetINCHIKeyFromINCHI & 5 & \cellcolor{green!0}{\large 0}/{\footnotesize 40} & \cellcolor{green!0}{\large 0}/{\footnotesize 40} & \cellcolor{green!0}{\large 0}/{\footnotesize 40} & \cellcolor{green!0}{\large 0}/{\footnotesize 40} & \cellcolor{green!10}{\large 6}/{\footnotesize 95} & \cellcolor{green!10}{\large 1}/{\footnotesize 53} \tabularnewline
\rowcolor{black!10} 23 & libdwarf\_dwarf\_init\_b & 5 & \cellcolor{green!0}{\large 0}/{\footnotesize 40} & \cellcolor{green!0}{\large 0}/{\footnotesize 40} & \cellcolor{green!0}{\large 0}/{\footnotesize 40} & \cellcolor{green!0}{\large 0}/{\footnotesize 40} & \cellcolor{green!10}{\large 4}/{\footnotesize 123} & \cellcolor{green!0}{\large 0}/{\footnotesize 56} \tabularnewline
24 & libdwarf\_dwarf\_init\_path & 5 & \cellcolor{green!0}{\large 0}/{\footnotesize 40} & \cellcolor{green!0}{\large 0}/{\footnotesize 40} & \cellcolor{green!0}{\large 0}/{\footnotesize 40} & \cellcolor{green!0}{\large 0}/{\footnotesize 40} & \cellcolor{green!0}{\large 0}/{\footnotesize 102} & \cellcolor{green!0}{\large 0}/{\footnotesize 51} \tabularnewline
\rowcolor{black!10} 25 & liblouis\_lou\_compileString & 5 & \cellcolor{green!0}{\large 0}/{\footnotesize 40} & \cellcolor{green!0}{\large 0}/{\footnotesize 40} & \cellcolor{green!10}{\large 1}/{\footnotesize 40} & \cellcolor{green!0}{\large 0}/{\footnotesize 40} & \cellcolor{green!10}{\large 2}/{\footnotesize 146} & \cellcolor{green!10}{\large 1}/{\footnotesize 81} \tabularnewline
26 & selinux\_cil\_compile & 5 & \cellcolor{green!0}{\large 0}/{\footnotesize 40} & \cellcolor{green!0}{\large 0}/{\footnotesize 40} & \cellcolor{green!0}{\large -}{\tiny -} & \cellcolor{green!0}{\large 0}/{\footnotesize 40} & \cellcolor{green!0}{\large 0}/{\footnotesize 145} & \cellcolor{green!0}{\large 0}/{\footnotesize 47} \tabularnewline
\rowcolor{black!10} 27 & bind9\_dns\_name\_fromtext & 6 & \cellcolor{green!0}{\large 0}/{\footnotesize 40} & \cellcolor{green!0}{\large 0}/{\footnotesize 40} & \cellcolor{green!0}{\large -}{\tiny -} & \cellcolor{green!10}{\large 1}/{\footnotesize 40} & \cellcolor{green!0}{\large 0}/{\footnotesize 85} & \cellcolor{green!0}{\large 0}/{\footnotesize 61} \tabularnewline
28 & bind9\_dns\_rdata\_fromwire & 6 & \cellcolor{green!0}{\large 0}/{\footnotesize 40} & \cellcolor{green!0}{\large 0}/{\footnotesize 40} & \cellcolor{green!0}{\large -}{\tiny -} & \cellcolor{green!0}{\large 0}/{\footnotesize 40} & \cellcolor{green!0}{\large 0}/{\footnotesize 82} & \cellcolor{green!0}{\large 0}/{\footnotesize 51} \tabularnewline
\rowcolor{black!10} 29 & coturn\_stun\_is\_binding\_response & 6 & \cellcolor{green!0}{\large 0}/{\footnotesize 40} & \cellcolor{green!0}{\large 0}/{\footnotesize 40} & \cellcolor{green!0}{\large -}{\tiny -} & \cellcolor{green!0}{\large 0}/{\footnotesize 40} & \cellcolor{green!10}{\large 3}/{\footnotesize 133} & \cellcolor{green!10}{\large 1}/{\footnotesize 52} \tabularnewline
30 & coturn\_stun\_is\_command\_message & 6 & \cellcolor{green!0}{\large 0}/{\footnotesize 40} & \cellcolor{green!0}{\large 0}/{\footnotesize 40} & \cellcolor{green!0}{\large 0}/{\footnotesize 40} & \cellcolor{green!0}{\large 0}/{\footnotesize 40} & \cellcolor{green!10}{\large 3}/{\footnotesize 143} & \cellcolor{green!0}{\large 0}/{\footnotesize 72} \tabularnewline
\rowcolor{black!10} 31 & coturn\_stun\_is\_response & 6 & \cellcolor{green!0}{\large 0}/{\footnotesize 40} & \cellcolor{green!0}{\large 0}/{\footnotesize 40} & \cellcolor{green!0}{\large -}{\tiny -} & \cellcolor{green!10}{\large 2}/{\footnotesize 40} & \cellcolor{green!0}{\large 1}/{\footnotesize 163} & \cellcolor{green!0}{\large 0}/{\footnotesize 74} \tabularnewline
32 & coturn\_stun\_is\_success\_response & 6 & \cellcolor{green!0}{\large 0}/{\footnotesize 40} & \cellcolor{green!0}{\large 0}/{\footnotesize 40} & \cellcolor{green!0}{\large -}{\tiny -} & \cellcolor{green!10}{\large 1}/{\footnotesize 40} & \cellcolor{green!10}{\large 4}/{\footnotesize 128} & \cellcolor{green!10}{\large 3}/{\footnotesize 62} \tabularnewline
\rowcolor{black!10} 33 & hiredis\_redisFormatCommand & 6 & \cellcolor{green!0}{\large 0}/{\footnotesize 40} & \cellcolor{green!0}{\large 0}/{\footnotesize 40} & \cellcolor{green!0}{\large -}{\tiny -} & \cellcolor{green!0}{\large 0}/{\footnotesize 40} & \cellcolor{green!10}{\large 6}/{\footnotesize 143} & \cellcolor{green!0}{\large 0}/{\footnotesize 68} \tabularnewline
34 & igraph\_igraph\_read\_graph\_dl & 6 & \cellcolor{green!0}{\large 0}/{\footnotesize 40} & \cellcolor{green!0}{\large 0}/{\footnotesize 40} & \cellcolor{green!0}{\large 0}/{\footnotesize 40} & \cellcolor{green!0}{\large 0}/{\footnotesize 40} & \cellcolor{green!0}{\large 0}/{\footnotesize 140} & \cellcolor{green!0}{\large 0}/{\footnotesize 69} \tabularnewline
\rowcolor{black!10} 35 & igraph\_igraph\_read\_graph\_edgelist & 6 & \cellcolor{green!0}{\large 0}/{\footnotesize 40} & \cellcolor{green!0}{\large 0}/{\footnotesize 40} & \cellcolor{green!0}{\large 0}/{\footnotesize 40} & \cellcolor{green!0}{\large 0}/{\footnotesize 40} & \cellcolor{green!0}{\large 0}/{\footnotesize 127} & \cellcolor{green!0}{\large 0}/{\footnotesize 70} \tabularnewline
36 & igraph\_igraph\_read\_graph\_gml & 6 & \cellcolor{green!0}{\large 0}/{\footnotesize 40} & \cellcolor{green!0}{\large 0}/{\footnotesize 40} & \cellcolor{green!0}{\large 0}/{\footnotesize 40} & \cellcolor{green!0}{\large 0}/{\footnotesize 40} & \cellcolor{green!0}{\large 0}/{\footnotesize 120} & \cellcolor{green!0}{\large 0}/{\footnotesize 75} \tabularnewline
\rowcolor{black!10} 37 & igraph\_igraph\_read\_graph\_graphdb & 6 & \cellcolor{green!0}{\large 0}/{\footnotesize 40} & \cellcolor{green!0}{\large 0}/{\footnotesize 40} & \cellcolor{green!0}{\large 0}/{\footnotesize 40} & \cellcolor{green!0}{\large 0}/{\footnotesize 40} & \cellcolor{green!0}{\large 0}/{\footnotesize 128} & \cellcolor{green!0}{\large 0}/{\footnotesize 82} \tabularnewline
38 & igraph\_igraph\_read\_graph\_graphml & 6 & \cellcolor{green!0}{\large 0}/{\footnotesize 40} & \cellcolor{green!0}{\large 0}/{\footnotesize 40} & \cellcolor{green!0}{\large 0}/{\footnotesize 40} & \cellcolor{green!0}{\large 0}/{\footnotesize 40} & \cellcolor{green!0}{\large 0}/{\footnotesize 130} & \cellcolor{green!0}{\large 0}/{\footnotesize 61} \tabularnewline
\rowcolor{black!10} 39 & igraph\_igraph\_read\_graph\_lgl & 6 & \cellcolor{green!0}{\large 0}/{\footnotesize 40} & \cellcolor{green!0}{\large 0}/{\footnotesize 40} & \cellcolor{green!0}{\large 0}/{\footnotesize 40} & \cellcolor{green!0}{\large 0}/{\footnotesize 40} & \cellcolor{green!0}{\large 0}/{\footnotesize 124} & \cellcolor{green!0}{\large 0}/{\footnotesize 79} \tabularnewline
40 & igraph\_igraph\_read\_graph\_pajek & 6 & \cellcolor{green!0}{\large 0}/{\footnotesize 40} & \cellcolor{green!0}{\large 0}/{\footnotesize 40} & \cellcolor{green!0}{\large 0}/{\footnotesize 40} & \cellcolor{green!0}{\large 0}/{\footnotesize 40} & \cellcolor{green!0}{\large 0}/{\footnotesize 150} & \cellcolor{green!0}{\large 0}/{\footnotesize 71} \tabularnewline
\rowcolor{black!10} 41 & inchi\_GetINCHIfromINCHI & 6 & \cellcolor{green!0}{\large 0}/{\footnotesize 40} & \cellcolor{green!0}{\large 0}/{\footnotesize 40} & \cellcolor{green!0}{\large 0}/{\footnotesize 40} & \cellcolor{green!0}{\large 0}/{\footnotesize 40} & \cellcolor{green!0}{\large 0}/{\footnotesize 151} & \cellcolor{green!0}{\large 0}/{\footnotesize 70} \tabularnewline
42 & inchi\_GetStructFromINCHI & 6 & \cellcolor{green!0}{\large 0}/{\footnotesize 40} & \cellcolor{green!0}{\large 0}/{\footnotesize 40} & \cellcolor{green!0}{\large 0}/{\footnotesize 40} & \cellcolor{green!0}{\large 0}/{\footnotesize 40} & \cellcolor{green!0}{\large 0}/{\footnotesize 160} & \cellcolor{green!0}{\large 0}/{\footnotesize 86} \tabularnewline
\rowcolor{black!10} 43 & kamailio\_parse\_msg & 6 & \cellcolor{green!0}{\large 0}/{\footnotesize 40} & \cellcolor{green!0}{\large 0}/{\footnotesize 40} & \cellcolor{green!0}{\large -}{\tiny -} & \cellcolor{green!0}{\large 0}/{\footnotesize 40} & \cellcolor{green!10}{\large 5}/{\footnotesize 113} & \cellcolor{green!10}{\large 1}/{\footnotesize 71} \tabularnewline
44 & libyang\_lys\_parse\_mem & 6 & \cellcolor{green!0}{\large 0}/{\footnotesize 40} & \cellcolor{green!0}{\large 0}/{\footnotesize 40} & \cellcolor{green!0}{\large 0}/{\footnotesize 40} & \cellcolor{green!0}{\large 0}/{\footnotesize 40} & \cellcolor{green!0}{\large 0}/{\footnotesize 136} & \cellcolor{green!0}{\large 0}/{\footnotesize 70} \tabularnewline
\rowcolor{black!10} 45 & proftpd\_pr\_json\_object\_from\_text & 6 & \cellcolor{green!0}{\large 0}/{\footnotesize 40} & \cellcolor{green!0}{\large 0}/{\footnotesize 40} & \cellcolor{green!0}{\large -}{\tiny -} & \cellcolor{green!0}{\large 0}/{\footnotesize 40} & \cellcolor{green!0}{\large 1}/{\footnotesize 136} & \cellcolor{green!0}{\large 0}/{\footnotesize 54} \tabularnewline
46 & selinux\_policydb\_read & 6 & \cellcolor{green!0}{\large 0}/{\footnotesize 40} & \cellcolor{green!0}{\large 0}/{\footnotesize 40} & \cellcolor{green!0}{\large -}{\tiny -} & \cellcolor{green!0}{\large 0}/{\footnotesize 40} & \cellcolor{green!10}{\large 5}/{\footnotesize 135} & \cellcolor{green!0}{\large 0}/{\footnotesize 56} \tabularnewline
\rowcolor{black!10} 47 & kamailio\_get\_src\_address\_socket & 7 & \cellcolor{green!0}{\large 0}/{\footnotesize 40} & \cellcolor{green!0}{\large 0}/{\footnotesize 40} & \cellcolor{green!0}{\large 0}/{\footnotesize 40} & \cellcolor{green!0}{\large 0}/{\footnotesize 40} & \cellcolor{green!0}{\large 0}/{\footnotesize 124} & \cellcolor{green!0}{\large 0}/{\footnotesize 78} \tabularnewline
48 & kamailio\_get\_src\_uri & 7 & \cellcolor{green!0}{\large 0}/{\footnotesize 40} & \cellcolor{green!0}{\large 0}/{\footnotesize 40} & \cellcolor{green!0}{\large 0}/{\footnotesize 40} & \cellcolor{green!0}{\large 0}/{\footnotesize 40} & \cellcolor{green!0}{\large 0}/{\footnotesize 150} & \cellcolor{green!0}{\large 0}/{\footnotesize 84} \tabularnewline
\rowcolor{black!10} 49 & kamailio\_parse\_content\_disposition & 7 & \cellcolor{green!0}{\large 0}/{\footnotesize 40} & \cellcolor{green!0}{\large 0}/{\footnotesize 40} & \cellcolor{green!0}{\large 0}/{\footnotesize 40} & \cellcolor{green!0}{\large 0}/{\footnotesize 40} & \cellcolor{green!0}{\large 0}/{\footnotesize 126} & \cellcolor{green!0}{\large 0}/{\footnotesize 63} \tabularnewline
50 & kamailio\_parse\_diversion\_header & 7 & \cellcolor{green!0}{\large 0}/{\footnotesize 40} & \cellcolor{green!0}{\large 0}/{\footnotesize 40} & \cellcolor{green!0}{\large 0}/{\footnotesize 40} & \cellcolor{green!0}{\large 0}/{\footnotesize 40} & \cellcolor{green!0}{\large 0}/{\footnotesize 137} & \cellcolor{green!0}{\large 0}/{\footnotesize 58} \tabularnewline
\rowcolor{black!10} 51 & kamailio\_parse\_from\_header & 7 & \cellcolor{green!0}{\large 0}/{\footnotesize 40} & \cellcolor{green!0}{\large 0}/{\footnotesize 40} & \cellcolor{green!0}{\large -}{\tiny -} & \cellcolor{green!0}{\large 0}/{\footnotesize 40} & \cellcolor{green!0}{\large 0}/{\footnotesize 137} & \cellcolor{green!0}{\large 0}/{\footnotesize 53} \tabularnewline
52 & kamailio\_parse\_from\_uri & 7 & \cellcolor{green!0}{\large 0}/{\footnotesize 40} & \cellcolor{green!0}{\large 0}/{\footnotesize 40} & \cellcolor{green!0}{\large -}{\tiny -} & \cellcolor{green!0}{\large 0}/{\footnotesize 40} & \cellcolor{green!0}{\large 0}/{\footnotesize 145} & \cellcolor{green!0}{\large 0}/{\footnotesize 51} \tabularnewline
\rowcolor{black!10} 53 & kamailio\_parse\_headers & 7 & \cellcolor{green!0}{\large 0}/{\footnotesize 40} & \cellcolor{green!0}{\large 0}/{\footnotesize 40} & \cellcolor{green!0}{\large -}{\tiny -} & \cellcolor{green!0}{\large 0}/{\footnotesize 40} & \cellcolor{green!0}{\large 0}/{\footnotesize 119} & \cellcolor{green!0}{\large 0}/{\footnotesize 59} \tabularnewline
54 & kamailio\_parse\_identityinfo\_header & 7 & \cellcolor{green!0}{\large 0}/{\footnotesize 40} & \cellcolor{green!0}{\large 0}/{\footnotesize 40} & \cellcolor{green!0}{\large -}{\tiny -} & \cellcolor{green!0}{\large 0}/{\footnotesize 40} & \cellcolor{green!0}{\large 0}/{\footnotesize 127} & \cellcolor{green!0}{\large 0}/{\footnotesize 73} \tabularnewline
\rowcolor{black!10} 55 & kamailio\_parse\_pai\_header & 7 & \cellcolor{green!0}{\large 0}/{\footnotesize 40} & \cellcolor{green!0}{\large 0}/{\footnotesize 40} & \cellcolor{green!0}{\large -}{\tiny -} & \cellcolor{green!0}{\large 0}/{\footnotesize 40} & \cellcolor{green!0}{\large 0}/{\footnotesize 127} & \cellcolor{green!0}{\large 0}/{\footnotesize 53} \tabularnewline
56 & kamailio\_parse\_privacy & 7 & \cellcolor{green!0}{\large 0}/{\footnotesize 40} & \cellcolor{green!0}{\large 0}/{\footnotesize 40} & \cellcolor{green!0}{\large 0}/{\footnotesize 40} & \cellcolor{green!0}{\large 0}/{\footnotesize 40} & \cellcolor{green!0}{\large 0}/{\footnotesize 147} & \cellcolor{green!0}{\large 0}/{\footnotesize 87} \tabularnewline
\rowcolor{black!10} 57 & kamailio\_parse\_record\_route\_headers & 7 & \cellcolor{green!0}{\large 0}/{\footnotesize 40} & \cellcolor{green!0}{\large 0}/{\footnotesize 40} & \cellcolor{green!0}{\large -}{\tiny -} & \cellcolor{green!0}{\large 0}/{\footnotesize 40} & \cellcolor{green!0}{\large 0}/{\footnotesize 136} & \cellcolor{green!0}{\large 0}/{\footnotesize 85} \tabularnewline
58 & kamailio\_parse\_refer\_to\_header & 7 & \cellcolor{green!0}{\large 0}/{\footnotesize 40} & \cellcolor{green!0}{\large 0}/{\footnotesize 40} & \cellcolor{green!0}{\large -}{\tiny -} & \cellcolor{green!0}{\large 0}/{\footnotesize 40} & \cellcolor{green!0}{\large 0}/{\footnotesize 135} & \cellcolor{green!0}{\large 0}/{\footnotesize 54} \tabularnewline
\rowcolor{black!10} 59 & kamailio\_parse\_route\_headers & 7 & \cellcolor{green!0}{\large 0}/{\footnotesize 40} & \cellcolor{green!0}{\large 0}/{\footnotesize 40} & \cellcolor{green!0}{\large -}{\tiny -} & \cellcolor{green!0}{\large 0}/{\footnotesize 40} & \cellcolor{green!0}{\large 0}/{\footnotesize 126} & \cellcolor{green!0}{\large 0}/{\footnotesize 94} \tabularnewline
60 & kamailio\_parse\_to\_header & 7 & \cellcolor{green!0}{\large 0}/{\footnotesize 40} & \cellcolor{green!0}{\large 0}/{\footnotesize 40} & \cellcolor{green!0}{\large -}{\tiny -} & \cellcolor{green!0}{\large 0}/{\footnotesize 40} & \cellcolor{green!0}{\large 0}/{\footnotesize 108} & \cellcolor{green!0}{\large 0}/{\footnotesize 50} \tabularnewline
\rowcolor{black!10} 61 & kamailio\_parse\_to\_uri & 7 & \cellcolor{green!0}{\large 0}/{\footnotesize 40} & \cellcolor{green!0}{\large 0}/{\footnotesize 40} & \cellcolor{green!0}{\large -}{\tiny -} & \cellcolor{green!0}{\large 0}/{\footnotesize 40} & \cellcolor{green!0}{\large 0}/{\footnotesize 122} & \cellcolor{green!0}{\large 0}/{\footnotesize 50} \tabularnewline
62 & libyang\_lyd\_parse\_data\_mem & 7 & \cellcolor{green!0}{\large 0}/{\footnotesize 40} & \cellcolor{green!0}{\large 0}/{\footnotesize 40} & \cellcolor{green!0}{\large 0}/{\footnotesize 40} & \cellcolor{green!10}{\large 1}/{\footnotesize 40} & \cellcolor{green!10}{\large 3}/{\footnotesize 122} & \cellcolor{green!0}{\large 0}/{\footnotesize 62} \tabularnewline
\rowcolor{black!10} 63 & bind9\_dns\_message\_parse & 8 & \cellcolor{green!0}{\large 0}/{\footnotesize 40} & \cellcolor{green!0}{\large 0}/{\footnotesize 40} & \cellcolor{green!0}{\large -}{\tiny -} & \cellcolor{green!0}{\large 0}/{\footnotesize 40} & \cellcolor{green!0}{\large 0}/{\footnotesize 128} & \cellcolor{green!0}{\large 0}/{\footnotesize 69} \tabularnewline
64 & igraph\_igraph\_read\_graph\_ncol & 8 & \cellcolor{green!0}{\large 0}/{\footnotesize 40} & \cellcolor{green!0}{\large 0}/{\footnotesize 40} & \cellcolor{green!0}{\large 0}/{\footnotesize 40} & \cellcolor{green!0}{\large 0}/{\footnotesize 40} & \cellcolor{green!0}{\large 0}/{\footnotesize 109} & \cellcolor{green!0}{\large 0}/{\footnotesize 58} \tabularnewline
\rowcolor{black!10} 65 & pjsip\_pj\_json\_parse & 8 & \cellcolor{green!0}{\large 0}/{\footnotesize 40} & \cellcolor{green!0}{\large 0}/{\footnotesize 40} & \cellcolor{green!0}{\large 0}/{\footnotesize 40} & \cellcolor{green!0}{\large 0}/{\footnotesize 40} & \cellcolor{green!0}{\large 0}/{\footnotesize 137} & \cellcolor{green!0}{\large 0}/{\footnotesize 72} \tabularnewline
66 & pjsip\_pj\_xml\_parse & 8 & \cellcolor{green!0}{\large 0}/{\footnotesize 40} & \cellcolor{green!0}{\large 0}/{\footnotesize 40} & \cellcolor{green!0}{\large 0}/{\footnotesize 40} & \cellcolor{green!0}{\large 0}/{\footnotesize 40} & \cellcolor{green!0}{\large 0}/{\footnotesize 145} & \cellcolor{green!0}{\large 0}/{\footnotesize 68} \tabularnewline
\rowcolor{black!10} 67 & pjsip\_pjmedia\_sdp\_parse & 8 & \cellcolor{green!0}{\large 0}/{\footnotesize 40} & \cellcolor{green!0}{\large 0}/{\footnotesize 40} & \cellcolor{green!0}{\large 0}/{\footnotesize 40} & \cellcolor{green!0}{\large 0}/{\footnotesize 40} & \cellcolor{green!0}{\large 0}/{\footnotesize 128} & \cellcolor{green!0}{\large 0}/{\footnotesize 50} \tabularnewline
68 & quickjs\_lre\_compile & 8 & \cellcolor{green!0}{\large 0}/{\footnotesize 40} & \cellcolor{green!0}{\large 0}/{\footnotesize 40} & \cellcolor{green!0}{\large -}{\tiny -} & \cellcolor{green!0}{\large 0}/{\footnotesize 40} & \cellcolor{green!0}{\large 0}/{\footnotesize 140} & \cellcolor{green!0}{\large 0}/{\footnotesize 63} \tabularnewline
\rowcolor{black!10} 69 & bind9\_isc\_lex\_getmastertoken & 9 & \cellcolor{green!0}{\large 0}/{\footnotesize 40} & \cellcolor{green!0}{\large 0}/{\footnotesize 40} & \cellcolor{green!0}{\large -}{\tiny -} & \cellcolor{green!0}{\large 0}/{\footnotesize 40} & \cellcolor{green!0}{\large 0}/{\footnotesize 130} & \cellcolor{green!0}{\large 0}/{\footnotesize 58} \tabularnewline
70 & bind9\_isc\_lex\_gettoken & 9 & \cellcolor{green!0}{\large 0}/{\footnotesize 40} & \cellcolor{green!0}{\large 0}/{\footnotesize 40} & \cellcolor{green!0}{\large -}{\tiny -} & \cellcolor{green!0}{\large 0}/{\footnotesize 40} & \cellcolor{green!0}{\large 0}/{\footnotesize 153} & \cellcolor{green!0}{\large 0}/{\footnotesize 58} \tabularnewline
\rowcolor{black!10} 71 & quickjs\_JS\_Eval & 9 & \cellcolor{green!0}{\large 0}/{\footnotesize 40} & \cellcolor{green!0}{\large 0}/{\footnotesize 40} & \cellcolor{green!0}{\large -}{\tiny -} & \cellcolor{green!0}{\large 0}/{\footnotesize 40} & \cellcolor{green!0}{\large 1}/{\footnotesize 153} & \cellcolor{green!0}{\large 0}/{\footnotesize 51} \tabularnewline
72 & igraph\_igraph\_edge\_connectivity & 10 & \cellcolor{green!0}{\large 0}/{\footnotesize 40} & \cellcolor{green!0}{\large 0}/{\footnotesize 40} & \cellcolor{green!0}{\large 0}/{\footnotesize 40} & \cellcolor{green!0}{\large 0}/{\footnotesize 40} & \cellcolor{green!0}{\large 0}/{\footnotesize 129} & \cellcolor{green!0}{\large 0}/{\footnotesize 91} \tabularnewline
\rowcolor{black!10} 73 & pjsip\_pj\_stun\_msg\_decode & 10 & \cellcolor{green!0}{\large 0}/{\footnotesize 40} & \cellcolor{green!0}{\large 0}/{\footnotesize 40} & \cellcolor{green!0}{\large 0}/{\footnotesize 40} & \cellcolor{green!0}{\large 0}/{\footnotesize 40} & \cellcolor{green!0}{\large 0}/{\footnotesize 82} & \cellcolor{green!0}{\large 0}/{\footnotesize 59} \tabularnewline
74 & bind9\_dns\_message\_checksig & 11 & \cellcolor{green!0}{\large 0}/{\footnotesize 40} & \cellcolor{green!0}{\large 0}/{\footnotesize 40} & \cellcolor{green!0}{\large -}{\tiny -} & \cellcolor{green!0}{\large 0}/{\footnotesize 40} & \cellcolor{green!0}{\large 0}/{\footnotesize 152} & \cellcolor{green!0}{\large 0}/{\footnotesize 65} \tabularnewline
\rowcolor{black!10} 75 & libzip\_zip\_fread & 11 & \cellcolor{green!0}{\large 0}/{\footnotesize 40} & \cellcolor{green!0}{\large 0}/{\footnotesize 40} & \cellcolor{green!0}{\large 0}/{\footnotesize 40} & \cellcolor{green!0}{\large 0}/{\footnotesize 40} & \cellcolor{green!0}{\large 0}/{\footnotesize 130} & \cellcolor{green!0}{\large 0}/{\footnotesize 73} \tabularnewline
76 & bind9\_dns\_rdata\_fromtext & 12 & \cellcolor{green!0}{\large 0}/{\footnotesize 40} & \cellcolor{green!0}{\large 0}/{\footnotesize 40} & \cellcolor{green!0}{\large -}{\tiny -} & \cellcolor{green!0}{\large 0}/{\footnotesize 40} & \cellcolor{green!0}{\large 0}/{\footnotesize 60} & \cellcolor{green!0}{\large 0}/{\footnotesize 46} \tabularnewline
\rowcolor{black!10} 77 & igraph\_igraph\_all\_minimal\_st\_separators & 12 & \cellcolor{green!0}{\large 0}/{\footnotesize 40} & \cellcolor{green!0}{\large 0}/{\footnotesize 40} & \cellcolor{green!0}{\large 0}/{\footnotesize 40} & \cellcolor{green!0}{\large 0}/{\footnotesize 40} & \cellcolor{green!0}{\large 0}/{\footnotesize 106} & \cellcolor{green!0}{\large 0}/{\footnotesize 76} \tabularnewline
78 & igraph\_igraph\_minimum\_size\_separators & 12 & \cellcolor{green!0}{\large 0}/{\footnotesize 40} & \cellcolor{green!0}{\large 0}/{\footnotesize 40} & \cellcolor{green!0}{\large 0}/{\footnotesize 40} & \cellcolor{green!0}{\large 0}/{\footnotesize 40} & \cellcolor{green!0}{\large 0}/{\footnotesize 106} & \cellcolor{green!0}{\large 0}/{\footnotesize 58} \tabularnewline
\rowcolor{black!10} 79 & pjsip\_pjsip\_parse\_msg & 12 & \cellcolor{green!0}{\large 0}/{\footnotesize 40} & \cellcolor{green!0}{\large 0}/{\footnotesize 40} & \cellcolor{green!0}{\large 0}/{\footnotesize 40} & \cellcolor{green!0}{\large 0}/{\footnotesize 40} & \cellcolor{green!0}{\large 0}/{\footnotesize 139} & \cellcolor{green!0}{\large 0}/{\footnotesize 56} \tabularnewline
80 & igraph\_igraph\_automorphism\_group & 13 & \cellcolor{green!0}{\large 0}/{\footnotesize 40} & \cellcolor{green!0}{\large 0}/{\footnotesize 40} & \cellcolor{green!0}{\large 0}/{\footnotesize 40} & \cellcolor{green!0}{\large 0}/{\footnotesize 40} & \cellcolor{green!0}{\large 0}/{\footnotesize 79} & \cellcolor{green!0}{\large 0}/{\footnotesize 67} \tabularnewline
\rowcolor{black!10} 81 & libmodbus\_modbus\_read\_bits & 15 & \cellcolor{green!0}{\large 0}/{\footnotesize 40} & \cellcolor{green!0}{\large 0}/{\footnotesize 40} & \cellcolor{green!0}{\large 0}/{\footnotesize 40} & \cellcolor{green!0}{\large 0}/{\footnotesize 40} & \cellcolor{green!0}{\large 0}/{\footnotesize 120} & \cellcolor{green!0}{\large 0}/{\footnotesize 49} \tabularnewline
82 & libmodbus\_modbus\_read\_registers & 15 & \cellcolor{green!0}{\large 0}/{\footnotesize 40} & \cellcolor{green!0}{\large 0}/{\footnotesize 40} & \cellcolor{green!0}{\large 0}/{\footnotesize 40} & \cellcolor{green!0}{\large 0}/{\footnotesize 40} & \cellcolor{green!0}{\large 0}/{\footnotesize 113} & \cellcolor{green!0}{\large 0}/{\footnotesize 61} \tabularnewline
\rowcolor{black!10} 83 & civetweb\_mg\_get\_response & 17 & \cellcolor{green!0}{\large 0}/{\footnotesize 40} & \cellcolor{green!0}{\large 0}/{\footnotesize 40} & \cellcolor{green!0}{\large 0}/{\footnotesize 40} & \cellcolor{green!0}{\large 0}/{\footnotesize 40} & \cellcolor{green!0}{\large 0}/{\footnotesize 155} & \cellcolor{green!0}{\large 0}/{\footnotesize 46} \tabularnewline
84 & bind9\_dns\_master\_loadbuffer & 20 & \cellcolor{green!0}{\large 0}/{\footnotesize 40} & \cellcolor{green!0}{\large 0}/{\footnotesize 40} & \cellcolor{green!0}{\large -}{\tiny -} & \cellcolor{green!0}{\large 0}/{\footnotesize 40} & \cellcolor{green!0}{\large 0}/{\footnotesize 98} & \cellcolor{green!0}{\large 0}/{\footnotesize 60} \tabularnewline
\rowcolor{black!10} 85 & libmodbus\_modbus\_receive & 33 & \cellcolor{green!0}{\large 0}/{\footnotesize 40} & \cellcolor{green!0}{\large 0}/{\footnotesize 40} & \cellcolor{green!0}{\large 0}/{\footnotesize 40} & \cellcolor{green!0}{\large 0}/{\footnotesize 40} & \cellcolor{green!0}{\large 0}/{\footnotesize 113} & \cellcolor{green!0}{\large 0}/{\footnotesize 60} \tabularnewline
86 & tmux\_input\_parse\_buffer & 42 & \cellcolor{green!0}{\large 0}/{\footnotesize 40} & \cellcolor{green!0}{\large 0}/{\footnotesize 40} & \cellcolor{green!0}{\large -}{\tiny -} & \cellcolor{green!0}{\large 0}/{\footnotesize 40} & \cellcolor{green!0}{\large 0}/{\footnotesize 152} & \cellcolor{green!0}{\large 0}/{\footnotesize 83} \tabularnewline

\bottomrule
%\end{tabular}
%}
%\end{table*}
\end{xltabular}
}
\twocolumn



% model: codellama-34b-instruct, temp: 1.0

\onecolumn
{\small %
\begin{xltabular}[h]{\textwidth}{ccccccccc}
%\begin{table*}[!t]
%\centering
\caption{Evaluation Result of model codellama-34b-instruct with temperature 1.0.} \\
%\resizebox{1.0\linewidth}{!}{
%\begin{tabular}{cccccccccc}
\toprule
Index & Question & Score & NAIVE-40 & BACTX-40 & DOCTX-40 & UGCTX-40 & BA-ITER-40 & ALL-ITER-40 \tabularnewline
\midrule
\rowcolor{black!10} 1 & coturn\_stun\_is\_command\_message\_full\_check\_str & 1 & \cellcolor{green!0}{\large 0}/{\footnotesize 40} & \cellcolor{green!10}{\large 1}/{\footnotesize 40} & \cellcolor{green!0}{\large -}{\tiny -} & \cellcolor{green!20}{\large 5}/{\footnotesize 40} & \cellcolor{green!20}{\large 10}/{\footnotesize 68} & \cellcolor{green!10}{\large 4}/{\footnotesize 65} \tabularnewline
2 & kamailio\_parse\_uri & 1 & \cellcolor{green!0}{\large 0}/{\footnotesize 40} & \cellcolor{green!0}{\large 0}/{\footnotesize 40} & \cellcolor{green!0}{\large -}{\tiny -} & \cellcolor{green!0}{\large 0}/{\footnotesize 40} & \cellcolor{green!10}{\large 6}/{\footnotesize 89} & \cellcolor{green!0}{\large 0}/{\footnotesize 56} \tabularnewline
\rowcolor{black!10} 3 & coturn\_stun\_check\_message\_integrity\_str & 2 & \cellcolor{green!0}{\large 0}/{\footnotesize 40} & \cellcolor{green!10}{\large 1}/{\footnotesize 40} & \cellcolor{green!0}{\large -}{\tiny -} & \cellcolor{green!0}{\large 0}/{\footnotesize 40} & \cellcolor{green!10}{\large 2}/{\footnotesize 73} & \cellcolor{green!10}{\large 1}/{\footnotesize 56} \tabularnewline
4 & libiec61850\_MmsValue\_decodeMmsData & 2 & \cellcolor{green!0}{\large 0}/{\footnotesize 40} & \cellcolor{green!10}{\large 2}/{\footnotesize 40} & \cellcolor{green!10}{\large 1}/{\footnotesize 40} & \cellcolor{green!10}{\large 1}/{\footnotesize 40} & \cellcolor{green!20}{\large 9}/{\footnotesize 69} & \cellcolor{green!10}{\large 2}/{\footnotesize 67} \tabularnewline
\rowcolor{black!10} 5 & md4c\_md\_html & 2 & \cellcolor{green!0}{\large 0}/{\footnotesize 40} & \cellcolor{green!0}{\large 0}/{\footnotesize 40} & \cellcolor{green!0}{\large 0}/{\footnotesize 40} & \cellcolor{green!0}{\large 0}/{\footnotesize 40} & \cellcolor{green!10}{\large 3}/{\footnotesize 108} & \cellcolor{green!10}{\large 1}/{\footnotesize 69} \tabularnewline
6 & spdk\_spdk\_json\_parse & 2 & \cellcolor{green!0}{\large 0}/{\footnotesize 40} & \cellcolor{green!10}{\large 1}/{\footnotesize 40} & \cellcolor{green!0}{\large -}{\tiny -} & \cellcolor{green!0}{\large 0}/{\footnotesize 40} & \cellcolor{green!10}{\large 3}/{\footnotesize 80} & \cellcolor{green!0}{\large 0}/{\footnotesize 52} \tabularnewline
\rowcolor{black!10} 7 & croaring\_roaring\_bitmap\_portable\_deserialize\_safe & 3 & \cellcolor{green!10}{\large 4}/{\footnotesize 40} & \cellcolor{green!10}{\large 3}/{\footnotesize 40} & \cellcolor{green!10}{\large 3}/{\footnotesize 40} & \cellcolor{green!10}{\large 3}/{\footnotesize 40} & \cellcolor{green!20}{\large 11}/{\footnotesize 77} & \cellcolor{green!10}{\large 2}/{\footnotesize 59} \tabularnewline
8 & lua\_luaL\_loadbufferx & 3 & \cellcolor{green!10}{\large 1}/{\footnotesize 40} & \cellcolor{green!10}{\large 2}/{\footnotesize 40} & \cellcolor{green!20}{\large 8}/{\footnotesize 40} & \cellcolor{green!0}{\large 0}/{\footnotesize 40} & \cellcolor{green!20}{\large 15}/{\footnotesize 72} & \cellcolor{green!10}{\large 2}/{\footnotesize 58} \tabularnewline
\rowcolor{black!10} 9 & w3m\_wc\_Str\_conv\_with\_detect & 3 & \cellcolor{green!0}{\large 0}/{\footnotesize 40} & \cellcolor{green!0}{\large 0}/{\footnotesize 40} & \cellcolor{green!0}{\large -}{\tiny -} & \cellcolor{green!10}{\large 1}/{\footnotesize 40} & \cellcolor{green!0}{\large 0}/{\footnotesize 88} & \cellcolor{green!10}{\large 1}/{\footnotesize 71} \tabularnewline
10 & bind9\_dns\_name\_fromwire & 4 & \cellcolor{green!0}{\large 0}/{\footnotesize 40} & \cellcolor{green!0}{\large 0}/{\footnotesize 40} & \cellcolor{green!0}{\large -}{\tiny -} & \cellcolor{green!0}{\large 0}/{\footnotesize 40} & \cellcolor{green!0}{\large 0}/{\footnotesize 66} & \cellcolor{green!0}{\large 0}/{\footnotesize 70} \tabularnewline
\rowcolor{black!10} 11 & gdk-pixbuf\_gdk\_pixbuf\_animation\_new\_from\_file & 4 & \cellcolor{green!0}{\large 0}/{\footnotesize 40} & \cellcolor{green!0}{\large 0}/{\footnotesize 40} & \cellcolor{green!10}{\large 2}/{\footnotesize 40} & \cellcolor{green!10}{\large 1}/{\footnotesize 40} & \cellcolor{green!0}{\large 0}/{\footnotesize 68} & \cellcolor{green!10}{\large 2}/{\footnotesize 65} \tabularnewline
12 & gdk-pixbuf\_gdk\_pixbuf\_new\_from\_data & 4 & \cellcolor{green!0}{\large 0}/{\footnotesize 40} & \cellcolor{green!10}{\large 1}/{\footnotesize 40} & \cellcolor{green!10}{\large 4}/{\footnotesize 40} & \cellcolor{green!0}{\large 0}/{\footnotesize 40} & \cellcolor{green!10}{\large 6}/{\footnotesize 74} & \cellcolor{green!0}{\large 0}/{\footnotesize 55} \tabularnewline
\rowcolor{black!10} 13 & gdk-pixbuf\_gdk\_pixbuf\_new\_from\_file & 4 & \cellcolor{green!0}{\large 0}/{\footnotesize 40} & \cellcolor{green!10}{\large 1}/{\footnotesize 40} & \cellcolor{green!0}{\large 0}/{\footnotesize 40} & \cellcolor{green!0}{\large 0}/{\footnotesize 40} & \cellcolor{green!0}{\large 0}/{\footnotesize 87} & \cellcolor{green!0}{\large 0}/{\footnotesize 65} \tabularnewline
14 & gdk-pixbuf\_gdk\_pixbuf\_new\_from\_stream & 4 & \cellcolor{green!10}{\large 2}/{\footnotesize 40} & \cellcolor{green!20}{\large 5}/{\footnotesize 40} & \cellcolor{green!10}{\large 4}/{\footnotesize 40} & \cellcolor{green!10}{\large 3}/{\footnotesize 40} & \cellcolor{green!20}{\large 9}/{\footnotesize 77} & \cellcolor{green!20}{\large 10}/{\footnotesize 56} \tabularnewline
\rowcolor{black!10} 15 & gpac\_gf\_isom\_open\_file & 4 & \cellcolor{green!0}{\large 0}/{\footnotesize 40} & \cellcolor{green!0}{\large 0}/{\footnotesize 40} & \cellcolor{green!0}{\large -}{\tiny -} & \cellcolor{green!10}{\large 1}/{\footnotesize 40} & \cellcolor{green!0}{\large 0}/{\footnotesize 104} & \cellcolor{green!0}{\large 0}/{\footnotesize 56} \tabularnewline
16 & libbpf\_bpf\_object\_\_open\_mem & 4 & \cellcolor{green!0}{\large 0}/{\footnotesize 40} & \cellcolor{green!0}{\large 0}/{\footnotesize 40} & \cellcolor{green!0}{\large 0}/{\footnotesize 40} & \cellcolor{green!0}{\large 0}/{\footnotesize 40} & \cellcolor{green!10}{\large 2}/{\footnotesize 75} & \cellcolor{green!0}{\large 0}/{\footnotesize 57} \tabularnewline
\rowcolor{black!10} 17 & libpg\_query\_pg\_query\_parse & 4 & \cellcolor{green!0}{\large 0}/{\footnotesize 40} & \cellcolor{green!0}{\large 0}/{\footnotesize 40} & \cellcolor{green!0}{\large -}{\tiny -} & \cellcolor{green!10}{\large 2}/{\footnotesize 40} & \cellcolor{green!0}{\large 1}/{\footnotesize 120} & \cellcolor{green!0}{\large 0}/{\footnotesize 77} \tabularnewline
18 & libucl\_ucl\_parser\_add\_string & 4 & \cellcolor{green!0}{\large 0}/{\footnotesize 40} & \cellcolor{green!0}{\large 0}/{\footnotesize 40} & \cellcolor{green!10}{\large 1}/{\footnotesize 40} & \cellcolor{green!10}{\large 1}/{\footnotesize 40} & \cellcolor{green!10}{\large 4}/{\footnotesize 91} & \cellcolor{green!10}{\large 1}/{\footnotesize 67} \tabularnewline
\rowcolor{black!10} 19 & oniguruma\_onig\_new & 4 & \cellcolor{green!0}{\large 0}/{\footnotesize 40} & \cellcolor{green!10}{\large 2}/{\footnotesize 40} & \cellcolor{green!10}{\large 1}/{\footnotesize 40} & \cellcolor{green!10}{\large 1}/{\footnotesize 40} & \cellcolor{green!10}{\large 2}/{\footnotesize 97} & \cellcolor{green!0}{\large 0}/{\footnotesize 53} \tabularnewline
20 & pupnp\_ixmlLoadDocumentEx & 4 & \cellcolor{green!0}{\large 0}/{\footnotesize 40} & \cellcolor{green!0}{\large 0}/{\footnotesize 40} & \cellcolor{green!10}{\large 2}/{\footnotesize 40} & \cellcolor{green!0}{\large 0}/{\footnotesize 40} & \cellcolor{green!0}{\large 0}/{\footnotesize 123} & \cellcolor{green!0}{\large 0}/{\footnotesize 91} \tabularnewline
\rowcolor{black!10} 21 & gdk-pixbuf\_gdk\_pixbuf\_new\_from\_file\_at\_scale & 5 & \cellcolor{green!0}{\large 0}/{\footnotesize 40} & \cellcolor{green!10}{\large 1}/{\footnotesize 40} & \cellcolor{green!10}{\large 3}/{\footnotesize 40} & \cellcolor{green!0}{\large 0}/{\footnotesize 40} & \cellcolor{green!0}{\large 0}/{\footnotesize 70} & \cellcolor{green!0}{\large 0}/{\footnotesize 56} \tabularnewline
22 & inchi\_GetINCHIKeyFromINCHI & 5 & \cellcolor{green!0}{\large 0}/{\footnotesize 40} & \cellcolor{green!0}{\large 0}/{\footnotesize 40} & \cellcolor{green!0}{\large 0}/{\footnotesize 40} & \cellcolor{green!0}{\large 0}/{\footnotesize 40} & \cellcolor{green!10}{\large 2}/{\footnotesize 78} & \cellcolor{green!0}{\large 0}/{\footnotesize 63} \tabularnewline
\rowcolor{black!10} 23 & libdwarf\_dwarf\_init\_b & 5 & \cellcolor{green!0}{\large 0}/{\footnotesize 40} & \cellcolor{green!0}{\large 0}/{\footnotesize 40} & \cellcolor{green!0}{\large 0}/{\footnotesize 40} & \cellcolor{green!0}{\large 0}/{\footnotesize 40} & \cellcolor{green!0}{\large 0}/{\footnotesize 78} & \cellcolor{green!0}{\large 0}/{\footnotesize 50} \tabularnewline
24 & libdwarf\_dwarf\_init\_path & 5 & \cellcolor{green!0}{\large 0}/{\footnotesize 40} & \cellcolor{green!0}{\large 0}/{\footnotesize 40} & \cellcolor{green!0}{\large 0}/{\footnotesize 40} & \cellcolor{green!0}{\large 0}/{\footnotesize 40} & \cellcolor{green!0}{\large 0}/{\footnotesize 72} & \cellcolor{green!0}{\large 0}/{\footnotesize 55} \tabularnewline
\rowcolor{black!10} 25 & liblouis\_lou\_compileString & 5 & \cellcolor{green!0}{\large 0}/{\footnotesize 40} & \cellcolor{green!0}{\large 0}/{\footnotesize 40} & \cellcolor{green!0}{\large 0}/{\footnotesize 40} & \cellcolor{green!0}{\large 0}/{\footnotesize 40} & \cellcolor{green!10}{\large 3}/{\footnotesize 92} & \cellcolor{green!10}{\large 4}/{\footnotesize 75} \tabularnewline
26 & selinux\_cil\_compile & 5 & \cellcolor{green!0}{\large 0}/{\footnotesize 40} & \cellcolor{green!0}{\large 0}/{\footnotesize 40} & \cellcolor{green!0}{\large -}{\tiny -} & \cellcolor{green!10}{\large 1}/{\footnotesize 40} & \cellcolor{green!10}{\large 1}/{\footnotesize 87} & \cellcolor{green!10}{\large 1}/{\footnotesize 50} \tabularnewline
\rowcolor{black!10} 27 & bind9\_dns\_name\_fromtext & 6 & \cellcolor{green!0}{\large 0}/{\footnotesize 40} & \cellcolor{green!0}{\large 0}/{\footnotesize 40} & \cellcolor{green!0}{\large -}{\tiny -} & \cellcolor{green!10}{\large 2}/{\footnotesize 40} & \cellcolor{green!0}{\large 0}/{\footnotesize 88} & \cellcolor{green!0}{\large 0}/{\footnotesize 64} \tabularnewline
28 & bind9\_dns\_rdata\_fromwire & 6 & \cellcolor{green!0}{\large 0}/{\footnotesize 40} & \cellcolor{green!0}{\large 0}/{\footnotesize 40} & \cellcolor{green!0}{\large -}{\tiny -} & \cellcolor{green!0}{\large 0}/{\footnotesize 40} & \cellcolor{green!0}{\large 0}/{\footnotesize 79} & \cellcolor{green!0}{\large 0}/{\footnotesize 54} \tabularnewline
\rowcolor{black!10} 29 & coturn\_stun\_is\_binding\_response & 6 & \cellcolor{green!0}{\large 0}/{\footnotesize 40} & \cellcolor{green!0}{\large 0}/{\footnotesize 40} & \cellcolor{green!0}{\large -}{\tiny -} & \cellcolor{green!0}{\large 0}/{\footnotesize 40} & \cellcolor{green!10}{\large 2}/{\footnotesize 98} & \cellcolor{green!10}{\large 1}/{\footnotesize 66} \tabularnewline
30 & coturn\_stun\_is\_command\_message & 6 & \cellcolor{green!0}{\large 0}/{\footnotesize 40} & \cellcolor{green!0}{\large 0}/{\footnotesize 40} & \cellcolor{green!0}{\large 0}/{\footnotesize 40} & \cellcolor{green!10}{\large 1}/{\footnotesize 40} & \cellcolor{green!0}{\large 1}/{\footnotesize 108} & \cellcolor{green!10}{\large 1}/{\footnotesize 69} \tabularnewline
\rowcolor{black!10} 31 & coturn\_stun\_is\_response & 6 & \cellcolor{green!0}{\large 0}/{\footnotesize 40} & \cellcolor{green!0}{\large 0}/{\footnotesize 40} & \cellcolor{green!0}{\large -}{\tiny -} & \cellcolor{green!0}{\large 0}/{\footnotesize 40} & \cellcolor{green!0}{\large 1}/{\footnotesize 109} & \cellcolor{green!0}{\large 0}/{\footnotesize 62} \tabularnewline
32 & coturn\_stun\_is\_success\_response & 6 & \cellcolor{green!0}{\large 0}/{\footnotesize 40} & \cellcolor{green!0}{\large 0}/{\footnotesize 40} & \cellcolor{green!0}{\large -}{\tiny -} & \cellcolor{green!0}{\large 0}/{\footnotesize 40} & \cellcolor{green!10}{\large 3}/{\footnotesize 115} & \cellcolor{green!10}{\large 2}/{\footnotesize 66} \tabularnewline
\rowcolor{black!10} 33 & hiredis\_redisFormatCommand & 6 & \cellcolor{green!0}{\large 0}/{\footnotesize 40} & \cellcolor{green!0}{\large 0}/{\footnotesize 40} & \cellcolor{green!0}{\large -}{\tiny -} & \cellcolor{green!10}{\large 1}/{\footnotesize 40} & \cellcolor{green!0}{\large 0}/{\footnotesize 105} & \cellcolor{green!10}{\large 1}/{\footnotesize 69} \tabularnewline
34 & igraph\_igraph\_read\_graph\_dl & 6 & \cellcolor{green!0}{\large 0}/{\footnotesize 40} & \cellcolor{green!0}{\large 0}/{\footnotesize 40} & \cellcolor{green!0}{\large 0}/{\footnotesize 40} & \cellcolor{green!0}{\large 0}/{\footnotesize 40} & \cellcolor{green!0}{\large 0}/{\footnotesize 78} & \cellcolor{green!0}{\large 0}/{\footnotesize 61} \tabularnewline
\rowcolor{black!10} 35 & igraph\_igraph\_read\_graph\_edgelist & 6 & \cellcolor{green!0}{\large 0}/{\footnotesize 40} & \cellcolor{green!0}{\large 0}/{\footnotesize 40} & \cellcolor{green!0}{\large 0}/{\footnotesize 40} & \cellcolor{green!0}{\large 0}/{\footnotesize 40} & \cellcolor{green!0}{\large 0}/{\footnotesize 80} & \cellcolor{green!0}{\large 0}/{\footnotesize 64} \tabularnewline
36 & igraph\_igraph\_read\_graph\_gml & 6 & \cellcolor{green!0}{\large 0}/{\footnotesize 40} & \cellcolor{green!0}{\large 0}/{\footnotesize 40} & \cellcolor{green!0}{\large 0}/{\footnotesize 40} & \cellcolor{green!0}{\large 0}/{\footnotesize 40} & \cellcolor{green!0}{\large 0}/{\footnotesize 89} & \cellcolor{green!0}{\large 0}/{\footnotesize 73} \tabularnewline
\rowcolor{black!10} 37 & igraph\_igraph\_read\_graph\_graphdb & 6 & \cellcolor{green!0}{\large 0}/{\footnotesize 40} & \cellcolor{green!0}{\large 0}/{\footnotesize 40} & \cellcolor{green!0}{\large 0}/{\footnotesize 40} & \cellcolor{green!0}{\large 0}/{\footnotesize 40} & \cellcolor{green!0}{\large 0}/{\footnotesize 104} & \cellcolor{green!0}{\large 0}/{\footnotesize 64} \tabularnewline
38 & igraph\_igraph\_read\_graph\_graphml & 6 & \cellcolor{green!0}{\large 0}/{\footnotesize 40} & \cellcolor{green!0}{\large 0}/{\footnotesize 40} & \cellcolor{green!0}{\large 0}/{\footnotesize 40} & \cellcolor{green!10}{\large 1}/{\footnotesize 40} & \cellcolor{green!0}{\large 0}/{\footnotesize 92} & \cellcolor{green!0}{\large 0}/{\footnotesize 57} \tabularnewline
\rowcolor{black!10} 39 & igraph\_igraph\_read\_graph\_lgl & 6 & \cellcolor{green!0}{\large 0}/{\footnotesize 40} & \cellcolor{green!0}{\large 0}/{\footnotesize 40} & \cellcolor{green!0}{\large 0}/{\footnotesize 40} & \cellcolor{green!0}{\large 0}/{\footnotesize 40} & \cellcolor{green!0}{\large 0}/{\footnotesize 112} & \cellcolor{green!10}{\large 1}/{\footnotesize 69} \tabularnewline
40 & igraph\_igraph\_read\_graph\_pajek & 6 & \cellcolor{green!0}{\large 0}/{\footnotesize 40} & \cellcolor{green!0}{\large 0}/{\footnotesize 40} & \cellcolor{green!0}{\large 0}/{\footnotesize 40} & \cellcolor{green!0}{\large 0}/{\footnotesize 40} & \cellcolor{green!0}{\large 0}/{\footnotesize 100} & \cellcolor{green!10}{\large 1}/{\footnotesize 62} \tabularnewline
\rowcolor{black!10} 41 & inchi\_GetINCHIfromINCHI & 6 & \cellcolor{green!0}{\large 0}/{\footnotesize 40} & \cellcolor{green!0}{\large 0}/{\footnotesize 40} & \cellcolor{green!0}{\large 0}/{\footnotesize 40} & \cellcolor{green!0}{\large 0}/{\footnotesize 40} & \cellcolor{green!0}{\large 0}/{\footnotesize 97} & \cellcolor{green!10}{\large 1}/{\footnotesize 65} \tabularnewline
42 & inchi\_GetStructFromINCHI & 6 & \cellcolor{green!0}{\large 0}/{\footnotesize 40} & \cellcolor{green!0}{\large 0}/{\footnotesize 40} & \cellcolor{green!0}{\large 0}/{\footnotesize 40} & \cellcolor{green!0}{\large 0}/{\footnotesize 40} & \cellcolor{green!0}{\large 0}/{\footnotesize 95} & \cellcolor{green!0}{\large 0}/{\footnotesize 77} \tabularnewline
\rowcolor{black!10} 43 & kamailio\_parse\_msg & 6 & \cellcolor{green!0}{\large 0}/{\footnotesize 40} & \cellcolor{green!0}{\large 0}/{\footnotesize 40} & \cellcolor{green!0}{\large -}{\tiny -} & \cellcolor{green!0}{\large 0}/{\footnotesize 40} & \cellcolor{green!0}{\large 0}/{\footnotesize 100} & \cellcolor{green!10}{\large 1}/{\footnotesize 67} \tabularnewline
44 & libyang\_lys\_parse\_mem & 6 & \cellcolor{green!0}{\large 0}/{\footnotesize 40} & \cellcolor{green!0}{\large 0}/{\footnotesize 40} & \cellcolor{green!0}{\large 0}/{\footnotesize 40} & \cellcolor{green!10}{\large 1}/{\footnotesize 40} & \cellcolor{green!0}{\large 0}/{\footnotesize 94} & \cellcolor{green!0}{\large 0}/{\footnotesize 68} \tabularnewline
\rowcolor{black!10} 45 & proftpd\_pr\_json\_object\_from\_text & 6 & \cellcolor{green!0}{\large 0}/{\footnotesize 40} & \cellcolor{green!0}{\large 0}/{\footnotesize 40} & \cellcolor{green!0}{\large -}{\tiny -} & \cellcolor{green!10}{\large 1}/{\footnotesize 40} & \cellcolor{green!0}{\large 0}/{\footnotesize 87} & \cellcolor{green!0}{\large 0}/{\footnotesize 59} \tabularnewline
46 & selinux\_policydb\_read & 6 & \cellcolor{green!0}{\large 0}/{\footnotesize 40} & \cellcolor{green!0}{\large 0}/{\footnotesize 40} & \cellcolor{green!0}{\large -}{\tiny -} & \cellcolor{green!0}{\large 0}/{\footnotesize 40} & \cellcolor{green!0}{\large 0}/{\footnotesize 121} & \cellcolor{green!0}{\large 0}/{\footnotesize 54} \tabularnewline
\rowcolor{black!10} 47 & kamailio\_get\_src\_address\_socket & 7 & \cellcolor{green!0}{\large 0}/{\footnotesize 40} & \cellcolor{green!0}{\large 0}/{\footnotesize 40} & \cellcolor{green!0}{\large 0}/{\footnotesize 40} & \cellcolor{green!0}{\large 0}/{\footnotesize 40} & \cellcolor{green!0}{\large 0}/{\footnotesize 112} & \cellcolor{green!0}{\large 0}/{\footnotesize 66} \tabularnewline
48 & kamailio\_get\_src\_uri & 7 & \cellcolor{green!0}{\large 0}/{\footnotesize 40} & \cellcolor{green!0}{\large 0}/{\footnotesize 40} & \cellcolor{green!0}{\large 0}/{\footnotesize 40} & \cellcolor{green!0}{\large 0}/{\footnotesize 40} & \cellcolor{green!0}{\large 0}/{\footnotesize 94} & \cellcolor{green!0}{\large 0}/{\footnotesize 56} \tabularnewline
\rowcolor{black!10} 49 & kamailio\_parse\_content\_disposition & 7 & \cellcolor{green!0}{\large 0}/{\footnotesize 40} & \cellcolor{green!0}{\large 0}/{\footnotesize 40} & \cellcolor{green!0}{\large 0}/{\footnotesize 40} & \cellcolor{green!0}{\large 0}/{\footnotesize 40} & \cellcolor{green!0}{\large 0}/{\footnotesize 105} & \cellcolor{green!0}{\large 0}/{\footnotesize 60} \tabularnewline
50 & kamailio\_parse\_diversion\_header & 7 & \cellcolor{green!0}{\large 0}/{\footnotesize 40} & \cellcolor{green!0}{\large 0}/{\footnotesize 40} & \cellcolor{green!0}{\large 0}/{\footnotesize 40} & \cellcolor{green!0}{\large 0}/{\footnotesize 40} & \cellcolor{green!0}{\large 0}/{\footnotesize 124} & \cellcolor{green!0}{\large 0}/{\footnotesize 49} \tabularnewline
\rowcolor{black!10} 51 & kamailio\_parse\_from\_header & 7 & \cellcolor{green!0}{\large 0}/{\footnotesize 40} & \cellcolor{green!0}{\large 0}/{\footnotesize 40} & \cellcolor{green!0}{\large -}{\tiny -} & \cellcolor{green!0}{\large 0}/{\footnotesize 40} & \cellcolor{green!0}{\large 0}/{\footnotesize 108} & \cellcolor{green!0}{\large 0}/{\footnotesize 58} \tabularnewline
52 & kamailio\_parse\_from\_uri & 7 & \cellcolor{green!0}{\large 0}/{\footnotesize 40} & \cellcolor{green!0}{\large 0}/{\footnotesize 40} & \cellcolor{green!0}{\large -}{\tiny -} & \cellcolor{green!0}{\large 0}/{\footnotesize 40} & \cellcolor{green!0}{\large 0}/{\footnotesize 94} & \cellcolor{green!0}{\large 0}/{\footnotesize 58} \tabularnewline
\rowcolor{black!10} 53 & kamailio\_parse\_headers & 7 & \cellcolor{green!0}{\large 0}/{\footnotesize 40} & \cellcolor{green!0}{\large 0}/{\footnotesize 40} & \cellcolor{green!0}{\large -}{\tiny -} & \cellcolor{green!0}{\large 0}/{\footnotesize 40} & \cellcolor{green!0}{\large 0}/{\footnotesize 93} & \cellcolor{green!0}{\large 0}/{\footnotesize 52} \tabularnewline
54 & kamailio\_parse\_identityinfo\_header & 7 & \cellcolor{green!0}{\large 0}/{\footnotesize 40} & \cellcolor{green!0}{\large 0}/{\footnotesize 40} & \cellcolor{green!0}{\large -}{\tiny -} & \cellcolor{green!10}{\large 1}/{\footnotesize 40} & \cellcolor{green!0}{\large 0}/{\footnotesize 107} & \cellcolor{green!0}{\large 0}/{\footnotesize 68} \tabularnewline
\rowcolor{black!10} 55 & kamailio\_parse\_pai\_header & 7 & \cellcolor{green!0}{\large 0}/{\footnotesize 40} & \cellcolor{green!0}{\large 0}/{\footnotesize 40} & \cellcolor{green!0}{\large -}{\tiny -} & \cellcolor{green!0}{\large 0}/{\footnotesize 40} & \cellcolor{green!0}{\large 0}/{\footnotesize 102} & \cellcolor{green!0}{\large 0}/{\footnotesize 58} \tabularnewline
56 & kamailio\_parse\_privacy & 7 & \cellcolor{green!0}{\large 0}/{\footnotesize 40} & \cellcolor{green!0}{\large 0}/{\footnotesize 40} & \cellcolor{green!0}{\large 0}/{\footnotesize 40} & \cellcolor{green!10}{\large 1}/{\footnotesize 40} & \cellcolor{green!0}{\large 0}/{\footnotesize 95} & \cellcolor{green!10}{\large 1}/{\footnotesize 60} \tabularnewline
\rowcolor{black!10} 57 & kamailio\_parse\_record\_route\_headers & 7 & \cellcolor{green!0}{\large 0}/{\footnotesize 40} & \cellcolor{green!0}{\large 0}/{\footnotesize 40} & \cellcolor{green!0}{\large -}{\tiny -} & \cellcolor{green!10}{\large 1}/{\footnotesize 40} & \cellcolor{green!0}{\large 0}/{\footnotesize 114} & \cellcolor{green!10}{\large 1}/{\footnotesize 65} \tabularnewline
58 & kamailio\_parse\_refer\_to\_header & 7 & \cellcolor{green!0}{\large 0}/{\footnotesize 40} & \cellcolor{green!0}{\large 0}/{\footnotesize 40} & \cellcolor{green!0}{\large -}{\tiny -} & \cellcolor{green!10}{\large 1}/{\footnotesize 40} & \cellcolor{green!0}{\large 0}/{\footnotesize 98} & \cellcolor{green!0}{\large 0}/{\footnotesize 54} \tabularnewline
\rowcolor{black!10} 59 & kamailio\_parse\_route\_headers & 7 & \cellcolor{green!0}{\large 0}/{\footnotesize 40} & \cellcolor{green!0}{\large 0}/{\footnotesize 40} & \cellcolor{green!0}{\large -}{\tiny -} & \cellcolor{green!0}{\large 0}/{\footnotesize 40} & \cellcolor{green!0}{\large 0}/{\footnotesize 95} & \cellcolor{green!10}{\large 1}/{\footnotesize 62} \tabularnewline
60 & kamailio\_parse\_to\_header & 7 & \cellcolor{green!0}{\large 0}/{\footnotesize 40} & \cellcolor{green!0}{\large 0}/{\footnotesize 40} & \cellcolor{green!0}{\large -}{\tiny -} & \cellcolor{green!0}{\large 0}/{\footnotesize 40} & \cellcolor{green!0}{\large 0}/{\footnotesize 101} & \cellcolor{green!0}{\large 0}/{\footnotesize 58} \tabularnewline
\rowcolor{black!10} 61 & kamailio\_parse\_to\_uri & 7 & \cellcolor{green!0}{\large 0}/{\footnotesize 40} & \cellcolor{green!0}{\large 0}/{\footnotesize 40} & \cellcolor{green!0}{\large -}{\tiny -} & \cellcolor{green!0}{\large 0}/{\footnotesize 40} & \cellcolor{green!0}{\large 0}/{\footnotesize 94} & \cellcolor{green!0}{\large 0}/{\footnotesize 50} \tabularnewline
62 & libyang\_lyd\_parse\_data\_mem & 7 & \cellcolor{green!0}{\large 0}/{\footnotesize 40} & \cellcolor{green!0}{\large 0}/{\footnotesize 40} & \cellcolor{green!0}{\large 0}/{\footnotesize 40} & \cellcolor{green!0}{\large 0}/{\footnotesize 40} & \cellcolor{green!0}{\large 0}/{\footnotesize 91} & \cellcolor{green!0}{\large 0}/{\footnotesize 52} \tabularnewline
\rowcolor{black!10} 63 & bind9\_dns\_message\_parse & 8 & \cellcolor{green!0}{\large 0}/{\footnotesize 40} & \cellcolor{green!0}{\large 0}/{\footnotesize 40} & \cellcolor{green!0}{\large -}{\tiny -} & \cellcolor{green!0}{\large 0}/{\footnotesize 40} & \cellcolor{green!0}{\large 0}/{\footnotesize 96} & \cellcolor{green!0}{\large 0}/{\footnotesize 64} \tabularnewline
64 & igraph\_igraph\_read\_graph\_ncol & 8 & \cellcolor{green!0}{\large 0}/{\footnotesize 40} & \cellcolor{green!0}{\large 0}/{\footnotesize 40} & \cellcolor{green!0}{\large 0}/{\footnotesize 40} & \cellcolor{green!0}{\large 0}/{\footnotesize 40} & \cellcolor{green!0}{\large 0}/{\footnotesize 74} & \cellcolor{green!0}{\large 0}/{\footnotesize 66} \tabularnewline
\rowcolor{black!10} 65 & pjsip\_pj\_json\_parse & 8 & \cellcolor{green!0}{\large 0}/{\footnotesize 40} & \cellcolor{green!0}{\large 0}/{\footnotesize 40} & \cellcolor{green!0}{\large 0}/{\footnotesize 40} & \cellcolor{green!0}{\large 0}/{\footnotesize 40} & \cellcolor{green!0}{\large 0}/{\footnotesize 104} & \cellcolor{green!0}{\large 0}/{\footnotesize 65} \tabularnewline
66 & pjsip\_pj\_xml\_parse & 8 & \cellcolor{green!0}{\large 0}/{\footnotesize 40} & \cellcolor{green!0}{\large 0}/{\footnotesize 40} & \cellcolor{green!0}{\large 0}/{\footnotesize 40} & \cellcolor{green!0}{\large 0}/{\footnotesize 40} & \cellcolor{green!0}{\large 0}/{\footnotesize 93} & \cellcolor{green!0}{\large 0}/{\footnotesize 67} \tabularnewline
\rowcolor{black!10} 67 & pjsip\_pjmedia\_sdp\_parse & 8 & \cellcolor{green!0}{\large 0}/{\footnotesize 40} & \cellcolor{green!0}{\large 0}/{\footnotesize 40} & \cellcolor{green!0}{\large 0}/{\footnotesize 40} & \cellcolor{green!0}{\large 0}/{\footnotesize 40} & \cellcolor{green!0}{\large 0}/{\footnotesize 103} & \cellcolor{green!0}{\large 0}/{\footnotesize 59} \tabularnewline
68 & quickjs\_lre\_compile & 8 & \cellcolor{green!0}{\large 0}/{\footnotesize 40} & \cellcolor{green!0}{\large 0}/{\footnotesize 40} & \cellcolor{green!0}{\large -}{\tiny -} & \cellcolor{green!0}{\large 0}/{\footnotesize 40} & \cellcolor{green!0}{\large 0}/{\footnotesize 111} & \cellcolor{green!0}{\large 0}/{\footnotesize 55} \tabularnewline
\rowcolor{black!10} 69 & bind9\_isc\_lex\_getmastertoken & 9 & \cellcolor{green!0}{\large 0}/{\footnotesize 40} & \cellcolor{green!0}{\large 0}/{\footnotesize 40} & \cellcolor{green!0}{\large -}{\tiny -} & \cellcolor{green!0}{\large 0}/{\footnotesize 40} & \cellcolor{green!0}{\large 0}/{\footnotesize 88} & \cellcolor{green!0}{\large 0}/{\footnotesize 61} \tabularnewline
70 & bind9\_isc\_lex\_gettoken & 9 & \cellcolor{green!0}{\large 0}/{\footnotesize 40} & \cellcolor{green!0}{\large 0}/{\footnotesize 40} & \cellcolor{green!0}{\large -}{\tiny -} & \cellcolor{green!0}{\large 0}/{\footnotesize 40} & \cellcolor{green!0}{\large 0}/{\footnotesize 96} & \cellcolor{green!0}{\large 0}/{\footnotesize 63} \tabularnewline
\rowcolor{black!10} 71 & quickjs\_JS\_Eval & 9 & \cellcolor{green!0}{\large 0}/{\footnotesize 40} & \cellcolor{green!0}{\large 0}/{\footnotesize 40} & \cellcolor{green!0}{\large -}{\tiny -} & \cellcolor{green!0}{\large 0}/{\footnotesize 40} & \cellcolor{green!0}{\large 1}/{\footnotesize 109} & \cellcolor{green!0}{\large 0}/{\footnotesize 60} \tabularnewline
72 & igraph\_igraph\_edge\_connectivity & 10 & \cellcolor{green!0}{\large 0}/{\footnotesize 40} & \cellcolor{green!0}{\large 0}/{\footnotesize 40} & \cellcolor{green!0}{\large 0}/{\footnotesize 40} & \cellcolor{green!0}{\large 0}/{\footnotesize 40} & \cellcolor{green!0}{\large 0}/{\footnotesize 88} & \cellcolor{green!0}{\large 0}/{\footnotesize 76} \tabularnewline
\rowcolor{black!10} 73 & pjsip\_pj\_stun\_msg\_decode & 10 & \cellcolor{green!0}{\large 0}/{\footnotesize 40} & \cellcolor{green!0}{\large 0}/{\footnotesize 40} & \cellcolor{green!0}{\large 0}/{\footnotesize 40} & \cellcolor{green!0}{\large 0}/{\footnotesize 40} & \cellcolor{green!0}{\large 0}/{\footnotesize 79} & \cellcolor{green!0}{\large 0}/{\footnotesize 51} \tabularnewline
74 & bind9\_dns\_message\_checksig & 11 & \cellcolor{green!0}{\large 0}/{\footnotesize 40} & \cellcolor{green!0}{\large 0}/{\footnotesize 40} & \cellcolor{green!0}{\large -}{\tiny -} & \cellcolor{green!0}{\large 0}/{\footnotesize 40} & \cellcolor{green!0}{\large 0}/{\footnotesize 91} & \cellcolor{green!0}{\large 0}/{\footnotesize 71} \tabularnewline
\rowcolor{black!10} 75 & libzip\_zip\_fread & 11 & \cellcolor{green!0}{\large 0}/{\footnotesize 40} & \cellcolor{green!0}{\large 0}/{\footnotesize 40} & \cellcolor{green!0}{\large 0}/{\footnotesize 40} & \cellcolor{green!0}{\large 0}/{\footnotesize 40} & \cellcolor{green!0}{\large 0}/{\footnotesize 81} & \cellcolor{green!0}{\large 0}/{\footnotesize 71} \tabularnewline
76 & bind9\_dns\_rdata\_fromtext & 12 & \cellcolor{green!0}{\large 0}/{\footnotesize 40} & \cellcolor{green!0}{\large 0}/{\footnotesize 40} & \cellcolor{green!0}{\large -}{\tiny -} & \cellcolor{green!0}{\large 0}/{\footnotesize 40} & \cellcolor{green!0}{\large 0}/{\footnotesize 51} & \cellcolor{green!0}{\large 0}/{\footnotesize 47} \tabularnewline
\rowcolor{black!10} 77 & igraph\_igraph\_all\_minimal\_st\_separators & 12 & \cellcolor{green!0}{\large 0}/{\footnotesize 40} & \cellcolor{green!0}{\large 0}/{\footnotesize 40} & \cellcolor{green!0}{\large 0}/{\footnotesize 40} & \cellcolor{green!0}{\large 0}/{\footnotesize 40} & \cellcolor{green!0}{\large 0}/{\footnotesize 85} & \cellcolor{green!10}{\large 1}/{\footnotesize 62} \tabularnewline
78 & igraph\_igraph\_minimum\_size\_separators & 12 & \cellcolor{green!0}{\large 0}/{\footnotesize 40} & \cellcolor{green!0}{\large 0}/{\footnotesize 40} & \cellcolor{green!0}{\large 0}/{\footnotesize 40} & \cellcolor{green!0}{\large 0}/{\footnotesize 40} & \cellcolor{green!0}{\large 0}/{\footnotesize 83} & \cellcolor{green!10}{\large 1}/{\footnotesize 63} \tabularnewline
\rowcolor{black!10} 79 & pjsip\_pjsip\_parse\_msg & 12 & \cellcolor{green!0}{\large 0}/{\footnotesize 40} & \cellcolor{green!0}{\large 0}/{\footnotesize 40} & \cellcolor{green!0}{\large 0}/{\footnotesize 40} & \cellcolor{green!0}{\large 0}/{\footnotesize 40} & \cellcolor{green!0}{\large 0}/{\footnotesize 92} & \cellcolor{green!0}{\large 0}/{\footnotesize 56} \tabularnewline
80 & igraph\_igraph\_automorphism\_group & 13 & \cellcolor{green!0}{\large 0}/{\footnotesize 40} & \cellcolor{green!0}{\large 0}/{\footnotesize 40} & \cellcolor{green!0}{\large 0}/{\footnotesize 40} & \cellcolor{green!0}{\large 0}/{\footnotesize 40} & \cellcolor{green!0}{\large 0}/{\footnotesize 70} & \cellcolor{green!10}{\large 1}/{\footnotesize 59} \tabularnewline
\rowcolor{black!10} 81 & libmodbus\_modbus\_read\_bits & 15 & \cellcolor{green!0}{\large 0}/{\footnotesize 40} & \cellcolor{green!0}{\large 0}/{\footnotesize 40} & \cellcolor{green!0}{\large 0}/{\footnotesize 40} & \cellcolor{green!0}{\large 0}/{\footnotesize 40} & \cellcolor{green!0}{\large 0}/{\footnotesize 96} & \cellcolor{green!0}{\large 0}/{\footnotesize 55} \tabularnewline
82 & libmodbus\_modbus\_read\_registers & 15 & \cellcolor{green!0}{\large 0}/{\footnotesize 40} & \cellcolor{green!0}{\large 0}/{\footnotesize 40} & \cellcolor{green!0}{\large 0}/{\footnotesize 40} & \cellcolor{green!0}{\large 0}/{\footnotesize 40} & \cellcolor{green!0}{\large 0}/{\footnotesize 96} & \cellcolor{green!0}{\large 0}/{\footnotesize 56} \tabularnewline
\rowcolor{black!10} 83 & civetweb\_mg\_get\_response & 17 & \cellcolor{green!0}{\large 0}/{\footnotesize 40} & \cellcolor{green!0}{\large 0}/{\footnotesize 40} & \cellcolor{green!0}{\large 0}/{\footnotesize 40} & \cellcolor{green!0}{\large 0}/{\footnotesize 40} & \cellcolor{green!0}{\large 0}/{\footnotesize 80} & \cellcolor{green!0}{\large 0}/{\footnotesize 57} \tabularnewline
84 & bind9\_dns\_master\_loadbuffer & 20 & \cellcolor{green!0}{\large 0}/{\footnotesize 40} & \cellcolor{green!0}{\large 0}/{\footnotesize 40} & \cellcolor{green!0}{\large -}{\tiny -} & \cellcolor{green!0}{\large 0}/{\footnotesize 40} & \cellcolor{green!0}{\large 0}/{\footnotesize 75} & \cellcolor{green!0}{\large 0}/{\footnotesize 62} \tabularnewline
\rowcolor{black!10} 85 & libmodbus\_modbus\_receive & 33 & \cellcolor{green!0}{\large 0}/{\footnotesize 40} & \cellcolor{green!0}{\large 0}/{\footnotesize 40} & \cellcolor{green!0}{\large 0}/{\footnotesize 40} & \cellcolor{green!0}{\large 0}/{\footnotesize 40} & \cellcolor{green!0}{\large 0}/{\footnotesize 92} & \cellcolor{green!0}{\large 0}/{\footnotesize 64} \tabularnewline
86 & tmux\_input\_parse\_buffer & 42 & \cellcolor{green!0}{\large 0}/{\footnotesize 40} & \cellcolor{green!0}{\large 0}/{\footnotesize 40} & \cellcolor{green!0}{\large -}{\tiny -} & \cellcolor{green!0}{\large 0}/{\footnotesize 40} & \cellcolor{green!0}{\large 0}/{\footnotesize 104} & \cellcolor{green!0}{\large 0}/{\footnotesize 73} \tabularnewline

\bottomrule
%\end{tabular}
%}
%\end{table*}
\end{xltabular}
}
\twocolumn



% model: codellama-34b-instruct, temp: 1.5

\onecolumn
{\small %
\begin{xltabular}[h]{\textwidth}{ccccccccc}
%\begin{table*}[!t]
%\centering
\caption{Evaluation Result of model codellama-34b-instruct with temperature 1.5.} \\
%\resizebox{1.0\linewidth}{!}{
%\begin{tabular}{cccccccccc}
\toprule
Index & Question & Score & NAIVE-40 & BACTX-40 & DOCTX-40 & UGCTX-40 & BA-ITER-40 & ALL-ITER-40 \tabularnewline
\midrule
\rowcolor{black!10} 1 & coturn\_stun\_is\_command\_message\_full\_check\_str & 1 & \cellcolor{green!0}{\large 0}/{\footnotesize 40} & \cellcolor{green!0}{\large 0}/{\footnotesize 40} & \cellcolor{green!0}{\large -}{\tiny -} & \cellcolor{green!0}{\large 0}/{\footnotesize 40} & \cellcolor{green!0}{\large 0}/{\footnotesize 40} & \cellcolor{green!0}{\large 0}/{\footnotesize 40} \tabularnewline
2 & kamailio\_parse\_uri & 1 & \cellcolor{green!0}{\large 0}/{\footnotesize 40} & \cellcolor{green!0}{\large 0}/{\footnotesize 40} & \cellcolor{green!0}{\large -}{\tiny -} & \cellcolor{green!0}{\large 0}/{\footnotesize 40} & \cellcolor{green!0}{\large 0}/{\footnotesize 40} & \cellcolor{green!0}{\large 0}/{\footnotesize 40} \tabularnewline
\rowcolor{black!10} 3 & coturn\_stun\_check\_message\_integrity\_str & 2 & \cellcolor{green!0}{\large 0}/{\footnotesize 40} & \cellcolor{green!0}{\large 0}/{\footnotesize 40} & \cellcolor{green!0}{\large -}{\tiny -} & \cellcolor{green!0}{\large 0}/{\footnotesize 40} & \cellcolor{green!0}{\large 0}/{\footnotesize 40} & \cellcolor{green!0}{\large 0}/{\footnotesize 40} \tabularnewline
4 & libiec61850\_MmsValue\_decodeMmsData & 2 & \cellcolor{green!0}{\large 0}/{\footnotesize 40} & \cellcolor{green!0}{\large 0}/{\footnotesize 40} & \cellcolor{green!10}{\large 1}/{\footnotesize 40} & \cellcolor{green!0}{\large 0}/{\footnotesize 40} & \cellcolor{green!0}{\large 0}/{\footnotesize 40} & \cellcolor{green!0}{\large 0}/{\footnotesize 40} \tabularnewline
\rowcolor{black!10} 5 & md4c\_md\_html & 2 & \cellcolor{green!0}{\large 0}/{\footnotesize 40} & \cellcolor{green!0}{\large 0}/{\footnotesize 40} & \cellcolor{green!0}{\large 0}/{\footnotesize 40} & \cellcolor{green!0}{\large 0}/{\footnotesize 40} & \cellcolor{green!0}{\large 0}/{\footnotesize 42} & \cellcolor{green!0}{\large 0}/{\footnotesize 40} \tabularnewline
6 & spdk\_spdk\_json\_parse & 2 & \cellcolor{green!0}{\large 0}/{\footnotesize 40} & \cellcolor{green!0}{\large 0}/{\footnotesize 40} & \cellcolor{green!0}{\large -}{\tiny -} & \cellcolor{green!0}{\large 0}/{\footnotesize 40} & \cellcolor{green!0}{\large 0}/{\footnotesize 40} & \cellcolor{green!0}{\large 0}/{\footnotesize 41} \tabularnewline
\rowcolor{black!10} 7 & croaring\_roaring\_bitmap\_portable\_deserialize\_safe & 3 & \cellcolor{green!0}{\large 0}/{\footnotesize 40} & \cellcolor{green!0}{\large 0}/{\footnotesize 40} & \cellcolor{green!0}{\large 0}/{\footnotesize 40} & \cellcolor{green!0}{\large 0}/{\footnotesize 40} & \cellcolor{green!0}{\large 0}/{\footnotesize 40} & \cellcolor{green!0}{\large 0}/{\footnotesize 40} \tabularnewline
8 & lua\_luaL\_loadbufferx & 3 & \cellcolor{green!0}{\large 0}/{\footnotesize 40} & \cellcolor{green!0}{\large 0}/{\footnotesize 40} & \cellcolor{green!0}{\large 0}/{\footnotesize 40} & \cellcolor{green!0}{\large 0}/{\footnotesize 40} & \cellcolor{green!0}{\large 0}/{\footnotesize 40} & \cellcolor{green!0}{\large 0}/{\footnotesize 41} \tabularnewline
\rowcolor{black!10} 9 & w3m\_wc\_Str\_conv\_with\_detect & 3 & \cellcolor{green!0}{\large 0}/{\footnotesize 40} & \cellcolor{green!0}{\large 0}/{\footnotesize 40} & \cellcolor{green!0}{\large -}{\tiny -} & \cellcolor{green!0}{\large 0}/{\footnotesize 40} & \cellcolor{green!0}{\large 0}/{\footnotesize 41} & \cellcolor{green!0}{\large 0}/{\footnotesize 41} \tabularnewline
10 & bind9\_dns\_name\_fromwire & 4 & \cellcolor{green!0}{\large 0}/{\footnotesize 40} & \cellcolor{green!0}{\large 0}/{\footnotesize 40} & \cellcolor{green!0}{\large -}{\tiny -} & \cellcolor{green!0}{\large 0}/{\footnotesize 40} & \cellcolor{green!0}{\large 0}/{\footnotesize 40} & \cellcolor{green!0}{\large 0}/{\footnotesize 40} \tabularnewline
\rowcolor{black!10} 11 & gdk-pixbuf\_gdk\_pixbuf\_animation\_new\_from\_file & 4 & \cellcolor{green!0}{\large 0}/{\footnotesize 40} & \cellcolor{green!0}{\large 0}/{\footnotesize 40} & \cellcolor{green!0}{\large 0}/{\footnotesize 40} & \cellcolor{green!0}{\large 0}/{\footnotesize 40} & \cellcolor{green!0}{\large 0}/{\footnotesize 41} & \cellcolor{green!0}{\large 0}/{\footnotesize 40} \tabularnewline
12 & gdk-pixbuf\_gdk\_pixbuf\_new\_from\_data & 4 & \cellcolor{green!0}{\large 0}/{\footnotesize 40} & \cellcolor{green!0}{\large 0}/{\footnotesize 40} & \cellcolor{green!0}{\large 0}/{\footnotesize 40} & \cellcolor{green!0}{\large 0}/{\footnotesize 40} & \cellcolor{green!0}{\large 0}/{\footnotesize 42} & \cellcolor{green!0}{\large 0}/{\footnotesize 40} \tabularnewline
\rowcolor{black!10} 13 & gdk-pixbuf\_gdk\_pixbuf\_new\_from\_file & 4 & \cellcolor{green!0}{\large 0}/{\footnotesize 40} & \cellcolor{green!0}{\large 0}/{\footnotesize 40} & \cellcolor{green!0}{\large 0}/{\footnotesize 40} & \cellcolor{green!0}{\large 0}/{\footnotesize 40} & \cellcolor{green!0}{\large 0}/{\footnotesize 40} & \cellcolor{green!0}{\large 0}/{\footnotesize 40} \tabularnewline
14 & gdk-pixbuf\_gdk\_pixbuf\_new\_from\_stream & 4 & \cellcolor{green!0}{\large 0}/{\footnotesize 40} & \cellcolor{green!0}{\large 0}/{\footnotesize 40} & \cellcolor{green!0}{\large 0}/{\footnotesize 40} & \cellcolor{green!0}{\large 0}/{\footnotesize 40} & \cellcolor{green!0}{\large 0}/{\footnotesize 40} & \cellcolor{green!10}{\large 1}/{\footnotesize 40} \tabularnewline
\rowcolor{black!10} 15 & gpac\_gf\_isom\_open\_file & 4 & \cellcolor{green!0}{\large 0}/{\footnotesize 40} & \cellcolor{green!0}{\large 0}/{\footnotesize 40} & \cellcolor{green!0}{\large -}{\tiny -} & \cellcolor{green!0}{\large 0}/{\footnotesize 40} & \cellcolor{green!0}{\large 0}/{\footnotesize 42} & \cellcolor{green!0}{\large 0}/{\footnotesize 40} \tabularnewline
16 & libbpf\_bpf\_object\_\_open\_mem & 4 & \cellcolor{green!0}{\large 0}/{\footnotesize 40} & \cellcolor{green!0}{\large 0}/{\footnotesize 40} & \cellcolor{green!0}{\large 0}/{\footnotesize 40} & \cellcolor{green!0}{\large 0}/{\footnotesize 40} & \cellcolor{green!0}{\large 0}/{\footnotesize 41} & \cellcolor{green!0}{\large 0}/{\footnotesize 41} \tabularnewline
\rowcolor{black!10} 17 & libpg\_query\_pg\_query\_parse & 4 & \cellcolor{green!0}{\large 0}/{\footnotesize 40} & \cellcolor{green!0}{\large 0}/{\footnotesize 40} & \cellcolor{green!0}{\large -}{\tiny -} & \cellcolor{green!0}{\large 0}/{\footnotesize 40} & \cellcolor{green!0}{\large 0}/{\footnotesize 40} & \cellcolor{green!0}{\large 0}/{\footnotesize 40} \tabularnewline
18 & libucl\_ucl\_parser\_add\_string & 4 & \cellcolor{green!0}{\large 0}/{\footnotesize 40} & \cellcolor{green!0}{\large 0}/{\footnotesize 40} & \cellcolor{green!0}{\large 0}/{\footnotesize 40} & \cellcolor{green!0}{\large 0}/{\footnotesize 40} & \cellcolor{green!0}{\large 0}/{\footnotesize 40} & \cellcolor{green!0}{\large 0}/{\footnotesize 41} \tabularnewline
\rowcolor{black!10} 19 & oniguruma\_onig\_new & 4 & \cellcolor{green!0}{\large 0}/{\footnotesize 40} & \cellcolor{green!0}{\large 0}/{\footnotesize 40} & \cellcolor{green!0}{\large 0}/{\footnotesize 40} & \cellcolor{green!0}{\large 0}/{\footnotesize 40} & \cellcolor{green!0}{\large 0}/{\footnotesize 40} & \cellcolor{green!0}{\large 0}/{\footnotesize 41} \tabularnewline
20 & pupnp\_ixmlLoadDocumentEx & 4 & \cellcolor{green!0}{\large 0}/{\footnotesize 40} & \cellcolor{green!0}{\large 0}/{\footnotesize 40} & \cellcolor{green!0}{\large 0}/{\footnotesize 40} & \cellcolor{green!0}{\large 0}/{\footnotesize 40} & \cellcolor{green!0}{\large 0}/{\footnotesize 40} & \cellcolor{green!0}{\large 0}/{\footnotesize 40} \tabularnewline
\rowcolor{black!10} 21 & gdk-pixbuf\_gdk\_pixbuf\_new\_from\_file\_at\_scale & 5 & \cellcolor{green!0}{\large 0}/{\footnotesize 40} & \cellcolor{green!0}{\large 0}/{\footnotesize 40} & \cellcolor{green!0}{\large 0}/{\footnotesize 40} & \cellcolor{green!0}{\large 0}/{\footnotesize 40} & \cellcolor{green!0}{\large 0}/{\footnotesize 40} & \cellcolor{green!0}{\large 0}/{\footnotesize 41} \tabularnewline
22 & inchi\_GetINCHIKeyFromINCHI & 5 & \cellcolor{green!0}{\large 0}/{\footnotesize 40} & \cellcolor{green!0}{\large 0}/{\footnotesize 40} & \cellcolor{green!0}{\large 0}/{\footnotesize 40} & \cellcolor{green!0}{\large 0}/{\footnotesize 40} & \cellcolor{green!0}{\large 0}/{\footnotesize 40} & \cellcolor{green!0}{\large 0}/{\footnotesize 40} \tabularnewline
\rowcolor{black!10} 23 & libdwarf\_dwarf\_init\_b & 5 & \cellcolor{green!0}{\large 0}/{\footnotesize 40} & \cellcolor{green!0}{\large 0}/{\footnotesize 40} & \cellcolor{green!0}{\large 0}/{\footnotesize 40} & \cellcolor{green!0}{\large 0}/{\footnotesize 40} & \cellcolor{green!0}{\large 0}/{\footnotesize 41} & \cellcolor{green!0}{\large 0}/{\footnotesize 40} \tabularnewline
24 & libdwarf\_dwarf\_init\_path & 5 & \cellcolor{green!0}{\large 0}/{\footnotesize 40} & \cellcolor{green!0}{\large 0}/{\footnotesize 40} & \cellcolor{green!0}{\large 0}/{\footnotesize 40} & \cellcolor{green!0}{\large 0}/{\footnotesize 40} & \cellcolor{green!0}{\large 0}/{\footnotesize 40} & \cellcolor{green!0}{\large 0}/{\footnotesize 43} \tabularnewline
\rowcolor{black!10} 25 & liblouis\_lou\_compileString & 5 & \cellcolor{green!0}{\large 0}/{\footnotesize 40} & \cellcolor{green!0}{\large 0}/{\footnotesize 40} & \cellcolor{green!0}{\large 0}/{\footnotesize 40} & \cellcolor{green!0}{\large 0}/{\footnotesize 40} & \cellcolor{green!0}{\large 0}/{\footnotesize 40} & \cellcolor{green!0}{\large 0}/{\footnotesize 40} \tabularnewline
26 & selinux\_cil\_compile & 5 & \cellcolor{green!0}{\large 0}/{\footnotesize 40} & \cellcolor{green!0}{\large 0}/{\footnotesize 40} & \cellcolor{green!0}{\large -}{\tiny -} & \cellcolor{green!0}{\large 0}/{\footnotesize 40} & \cellcolor{green!0}{\large 0}/{\footnotesize 40} & \cellcolor{green!0}{\large 0}/{\footnotesize 40} \tabularnewline
\rowcolor{black!10} 27 & bind9\_dns\_name\_fromtext & 6 & \cellcolor{green!0}{\large 0}/{\footnotesize 40} & \cellcolor{green!0}{\large 0}/{\footnotesize 40} & \cellcolor{green!0}{\large -}{\tiny -} & \cellcolor{green!0}{\large 0}/{\footnotesize 40} & \cellcolor{green!0}{\large 0}/{\footnotesize 40} & \cellcolor{green!0}{\large 0}/{\footnotesize 41} \tabularnewline
28 & bind9\_dns\_rdata\_fromwire & 6 & \cellcolor{green!0}{\large 0}/{\footnotesize 40} & \cellcolor{green!0}{\large 0}/{\footnotesize 40} & \cellcolor{green!0}{\large -}{\tiny -} & \cellcolor{green!0}{\large 0}/{\footnotesize 40} & \cellcolor{green!0}{\large 0}/{\footnotesize 41} & \cellcolor{green!0}{\large 0}/{\footnotesize 40} \tabularnewline
\rowcolor{black!10} 29 & coturn\_stun\_is\_binding\_response & 6 & \cellcolor{green!0}{\large 0}/{\footnotesize 40} & \cellcolor{green!0}{\large 0}/{\footnotesize 40} & \cellcolor{green!0}{\large -}{\tiny -} & \cellcolor{green!0}{\large 0}/{\footnotesize 40} & \cellcolor{green!0}{\large 0}/{\footnotesize 40} & \cellcolor{green!0}{\large 0}/{\footnotesize 40} \tabularnewline
30 & coturn\_stun\_is\_command\_message & 6 & \cellcolor{green!0}{\large 0}/{\footnotesize 40} & \cellcolor{green!0}{\large 0}/{\footnotesize 40} & \cellcolor{green!0}{\large 0}/{\footnotesize 40} & \cellcolor{green!0}{\large 0}/{\footnotesize 40} & \cellcolor{green!0}{\large 0}/{\footnotesize 40} & \cellcolor{green!0}{\large 0}/{\footnotesize 40} \tabularnewline
\rowcolor{black!10} 31 & coturn\_stun\_is\_response & 6 & \cellcolor{green!0}{\large 0}/{\footnotesize 40} & \cellcolor{green!0}{\large 0}/{\footnotesize 40} & \cellcolor{green!0}{\large -}{\tiny -} & \cellcolor{green!0}{\large 0}/{\footnotesize 40} & \cellcolor{green!0}{\large 0}/{\footnotesize 40} & \cellcolor{green!0}{\large 0}/{\footnotesize 41} \tabularnewline
32 & coturn\_stun\_is\_success\_response & 6 & \cellcolor{green!0}{\large 0}/{\footnotesize 40} & \cellcolor{green!0}{\large 0}/{\footnotesize 40} & \cellcolor{green!0}{\large -}{\tiny -} & \cellcolor{green!0}{\large 0}/{\footnotesize 40} & \cellcolor{green!0}{\large 0}/{\footnotesize 43} & \cellcolor{green!0}{\large 0}/{\footnotesize 40} \tabularnewline
\rowcolor{black!10} 33 & hiredis\_redisFormatCommand & 6 & \cellcolor{green!0}{\large 0}/{\footnotesize 40} & \cellcolor{green!0}{\large 0}/{\footnotesize 40} & \cellcolor{green!0}{\large -}{\tiny -} & \cellcolor{green!0}{\large 0}/{\footnotesize 40} & \cellcolor{green!0}{\large 0}/{\footnotesize 43} & \cellcolor{green!0}{\large 0}/{\footnotesize 40} \tabularnewline
34 & igraph\_igraph\_read\_graph\_dl & 6 & \cellcolor{green!0}{\large 0}/{\footnotesize 40} & \cellcolor{green!0}{\large 0}/{\footnotesize 40} & \cellcolor{green!0}{\large 0}/{\footnotesize 40} & \cellcolor{green!0}{\large 0}/{\footnotesize 40} & \cellcolor{green!0}{\large 0}/{\footnotesize 40} & \cellcolor{green!0}{\large 0}/{\footnotesize 40} \tabularnewline
\rowcolor{black!10} 35 & igraph\_igraph\_read\_graph\_edgelist & 6 & \cellcolor{green!0}{\large 0}/{\footnotesize 40} & \cellcolor{green!0}{\large 0}/{\footnotesize 40} & \cellcolor{green!0}{\large 0}/{\footnotesize 40} & \cellcolor{green!0}{\large 0}/{\footnotesize 40} & \cellcolor{green!0}{\large 0}/{\footnotesize 40} & \cellcolor{green!0}{\large 0}/{\footnotesize 41} \tabularnewline
36 & igraph\_igraph\_read\_graph\_gml & 6 & \cellcolor{green!0}{\large 0}/{\footnotesize 40} & \cellcolor{green!0}{\large 0}/{\footnotesize 40} & \cellcolor{green!0}{\large 0}/{\footnotesize 40} & \cellcolor{green!0}{\large 0}/{\footnotesize 40} & \cellcolor{green!0}{\large 0}/{\footnotesize 40} & \cellcolor{green!0}{\large 0}/{\footnotesize 40} \tabularnewline
\rowcolor{black!10} 37 & igraph\_igraph\_read\_graph\_graphdb & 6 & \cellcolor{green!0}{\large 0}/{\footnotesize 40} & \cellcolor{green!0}{\large 0}/{\footnotesize 40} & \cellcolor{green!0}{\large 0}/{\footnotesize 40} & \cellcolor{green!0}{\large 0}/{\footnotesize 40} & \cellcolor{green!0}{\large 0}/{\footnotesize 41} & \cellcolor{green!0}{\large 0}/{\footnotesize 41} \tabularnewline
38 & igraph\_igraph\_read\_graph\_graphml & 6 & \cellcolor{green!0}{\large 0}/{\footnotesize 40} & \cellcolor{green!0}{\large 0}/{\footnotesize 40} & \cellcolor{green!0}{\large 0}/{\footnotesize 40} & \cellcolor{green!0}{\large 0}/{\footnotesize 40} & \cellcolor{green!0}{\large 0}/{\footnotesize 40} & \cellcolor{green!0}{\large 0}/{\footnotesize 40} \tabularnewline
\rowcolor{black!10} 39 & igraph\_igraph\_read\_graph\_lgl & 6 & \cellcolor{green!0}{\large 0}/{\footnotesize 40} & \cellcolor{green!0}{\large 0}/{\footnotesize 40} & \cellcolor{green!0}{\large 0}/{\footnotesize 40} & \cellcolor{green!0}{\large 0}/{\footnotesize 40} & \cellcolor{green!0}{\large 0}/{\footnotesize 40} & \cellcolor{green!0}{\large 0}/{\footnotesize 40} \tabularnewline
40 & igraph\_igraph\_read\_graph\_pajek & 6 & \cellcolor{green!0}{\large 0}/{\footnotesize 40} & \cellcolor{green!0}{\large 0}/{\footnotesize 40} & \cellcolor{green!0}{\large 0}/{\footnotesize 40} & \cellcolor{green!0}{\large 0}/{\footnotesize 40} & \cellcolor{green!0}{\large 0}/{\footnotesize 40} & \cellcolor{green!0}{\large 0}/{\footnotesize 41} \tabularnewline
\rowcolor{black!10} 41 & inchi\_GetINCHIfromINCHI & 6 & \cellcolor{green!0}{\large 0}/{\footnotesize 40} & \cellcolor{green!0}{\large 0}/{\footnotesize 40} & \cellcolor{green!0}{\large 0}/{\footnotesize 40} & \cellcolor{green!0}{\large 0}/{\footnotesize 40} & \cellcolor{green!0}{\large 0}/{\footnotesize 42} & \cellcolor{green!0}{\large 0}/{\footnotesize 41} \tabularnewline
42 & inchi\_GetStructFromINCHI & 6 & \cellcolor{green!0}{\large 0}/{\footnotesize 40} & \cellcolor{green!0}{\large 0}/{\footnotesize 40} & \cellcolor{green!0}{\large 0}/{\footnotesize 40} & \cellcolor{green!0}{\large 0}/{\footnotesize 40} & \cellcolor{green!0}{\large 0}/{\footnotesize 41} & \cellcolor{green!0}{\large 0}/{\footnotesize 40} \tabularnewline
\rowcolor{black!10} 43 & kamailio\_parse\_msg & 6 & \cellcolor{green!0}{\large 0}/{\footnotesize 40} & \cellcolor{green!0}{\large 0}/{\footnotesize 40} & \cellcolor{green!0}{\large -}{\tiny -} & \cellcolor{green!0}{\large 0}/{\footnotesize 40} & \cellcolor{green!0}{\large 0}/{\footnotesize 41} & \cellcolor{green!0}{\large 0}/{\footnotesize 40} \tabularnewline
44 & libyang\_lys\_parse\_mem & 6 & \cellcolor{green!0}{\large 0}/{\footnotesize 40} & \cellcolor{green!0}{\large 0}/{\footnotesize 40} & \cellcolor{green!0}{\large 0}/{\footnotesize 40} & \cellcolor{green!0}{\large 0}/{\footnotesize 40} & \cellcolor{green!0}{\large 0}/{\footnotesize 40} & \cellcolor{green!0}{\large 0}/{\footnotesize 40} \tabularnewline
\rowcolor{black!10} 45 & proftpd\_pr\_json\_object\_from\_text & 6 & \cellcolor{green!0}{\large 0}/{\footnotesize 40} & \cellcolor{green!0}{\large 0}/{\footnotesize 40} & \cellcolor{green!0}{\large -}{\tiny -} & \cellcolor{green!0}{\large 0}/{\footnotesize 40} & \cellcolor{green!0}{\large 0}/{\footnotesize 40} & \cellcolor{green!0}{\large 0}/{\footnotesize 40} \tabularnewline
46 & selinux\_policydb\_read & 6 & \cellcolor{green!0}{\large 0}/{\footnotesize 40} & \cellcolor{green!0}{\large 0}/{\footnotesize 40} & \cellcolor{green!0}{\large -}{\tiny -} & \cellcolor{green!0}{\large 0}/{\footnotesize 40} & \cellcolor{green!0}{\large 0}/{\footnotesize 40} & \cellcolor{green!0}{\large 0}/{\footnotesize 40} \tabularnewline
\rowcolor{black!10} 47 & kamailio\_get\_src\_address\_socket & 7 & \cellcolor{green!0}{\large 0}/{\footnotesize 40} & \cellcolor{green!0}{\large 0}/{\footnotesize 40} & \cellcolor{green!0}{\large 0}/{\footnotesize 40} & \cellcolor{green!0}{\large 0}/{\footnotesize 40} & \cellcolor{green!0}{\large 0}/{\footnotesize 40} & \cellcolor{green!0}{\large 0}/{\footnotesize 40} \tabularnewline
48 & kamailio\_get\_src\_uri & 7 & \cellcolor{green!0}{\large 0}/{\footnotesize 40} & \cellcolor{green!0}{\large 0}/{\footnotesize 40} & \cellcolor{green!0}{\large 0}/{\footnotesize 40} & \cellcolor{green!0}{\large 0}/{\footnotesize 40} & \cellcolor{green!0}{\large 0}/{\footnotesize 41} & \cellcolor{green!0}{\large 0}/{\footnotesize 40} \tabularnewline
\rowcolor{black!10} 49 & kamailio\_parse\_content\_disposition & 7 & \cellcolor{green!0}{\large 0}/{\footnotesize 40} & \cellcolor{green!0}{\large 0}/{\footnotesize 40} & \cellcolor{green!0}{\large 0}/{\footnotesize 40} & \cellcolor{green!0}{\large 0}/{\footnotesize 40} & \cellcolor{green!0}{\large 0}/{\footnotesize 40} & \cellcolor{green!0}{\large 0}/{\footnotesize 41} \tabularnewline
50 & kamailio\_parse\_diversion\_header & 7 & \cellcolor{green!0}{\large 0}/{\footnotesize 40} & \cellcolor{green!0}{\large 0}/{\footnotesize 40} & \cellcolor{green!0}{\large 0}/{\footnotesize 40} & \cellcolor{green!0}{\large 0}/{\footnotesize 40} & \cellcolor{green!0}{\large 0}/{\footnotesize 42} & \cellcolor{green!0}{\large 0}/{\footnotesize 40} \tabularnewline
\rowcolor{black!10} 51 & kamailio\_parse\_from\_header & 7 & \cellcolor{green!0}{\large 0}/{\footnotesize 40} & \cellcolor{green!0}{\large 0}/{\footnotesize 40} & \cellcolor{green!0}{\large -}{\tiny -} & \cellcolor{green!0}{\large 0}/{\footnotesize 40} & \cellcolor{green!0}{\large 0}/{\footnotesize 40} & \cellcolor{green!0}{\large 0}/{\footnotesize 40} \tabularnewline
52 & kamailio\_parse\_from\_uri & 7 & \cellcolor{green!0}{\large 0}/{\footnotesize 40} & \cellcolor{green!0}{\large 0}/{\footnotesize 40} & \cellcolor{green!0}{\large -}{\tiny -} & \cellcolor{green!0}{\large 0}/{\footnotesize 40} & \cellcolor{green!0}{\large 0}/{\footnotesize 40} & \cellcolor{green!0}{\large 0}/{\footnotesize 40} \tabularnewline
\rowcolor{black!10} 53 & kamailio\_parse\_headers & 7 & \cellcolor{green!0}{\large 0}/{\footnotesize 40} & \cellcolor{green!0}{\large 0}/{\footnotesize 40} & \cellcolor{green!0}{\large -}{\tiny -} & \cellcolor{green!0}{\large 0}/{\footnotesize 40} & \cellcolor{green!0}{\large 0}/{\footnotesize 43} & \cellcolor{green!0}{\large 0}/{\footnotesize 41} \tabularnewline
54 & kamailio\_parse\_identityinfo\_header & 7 & \cellcolor{green!0}{\large 0}/{\footnotesize 40} & \cellcolor{green!0}{\large 0}/{\footnotesize 40} & \cellcolor{green!0}{\large -}{\tiny -} & \cellcolor{green!0}{\large 0}/{\footnotesize 40} & \cellcolor{green!0}{\large 0}/{\footnotesize 41} & \cellcolor{green!0}{\large 0}/{\footnotesize 40} \tabularnewline
\rowcolor{black!10} 55 & kamailio\_parse\_pai\_header & 7 & \cellcolor{green!0}{\large 0}/{\footnotesize 40} & \cellcolor{green!0}{\large 0}/{\footnotesize 40} & \cellcolor{green!0}{\large -}{\tiny -} & \cellcolor{green!0}{\large 0}/{\footnotesize 40} & \cellcolor{green!0}{\large 0}/{\footnotesize 40} & \cellcolor{green!0}{\large 0}/{\footnotesize 40} \tabularnewline
56 & kamailio\_parse\_privacy & 7 & \cellcolor{green!0}{\large 0}/{\footnotesize 40} & \cellcolor{green!0}{\large 0}/{\footnotesize 40} & \cellcolor{green!0}{\large 0}/{\footnotesize 40} & \cellcolor{green!0}{\large 0}/{\footnotesize 40} & \cellcolor{green!0}{\large 0}/{\footnotesize 41} & \cellcolor{green!0}{\large 0}/{\footnotesize 40} \tabularnewline
\rowcolor{black!10} 57 & kamailio\_parse\_record\_route\_headers & 7 & \cellcolor{green!0}{\large 0}/{\footnotesize 40} & \cellcolor{green!0}{\large 0}/{\footnotesize 40} & \cellcolor{green!0}{\large -}{\tiny -} & \cellcolor{green!0}{\large 0}/{\footnotesize 40} & \cellcolor{green!0}{\large 0}/{\footnotesize 40} & \cellcolor{green!0}{\large 0}/{\footnotesize 40} \tabularnewline
58 & kamailio\_parse\_refer\_to\_header & 7 & \cellcolor{green!0}{\large 0}/{\footnotesize 40} & \cellcolor{green!0}{\large 0}/{\footnotesize 40} & \cellcolor{green!0}{\large -}{\tiny -} & \cellcolor{green!0}{\large 0}/{\footnotesize 40} & \cellcolor{green!0}{\large 0}/{\footnotesize 40} & \cellcolor{green!0}{\large 0}/{\footnotesize 40} \tabularnewline
\rowcolor{black!10} 59 & kamailio\_parse\_route\_headers & 7 & \cellcolor{green!0}{\large 0}/{\footnotesize 40} & \cellcolor{green!0}{\large 0}/{\footnotesize 40} & \cellcolor{green!0}{\large -}{\tiny -} & \cellcolor{green!0}{\large 0}/{\footnotesize 40} & \cellcolor{green!0}{\large 0}/{\footnotesize 40} & \cellcolor{green!0}{\large 0}/{\footnotesize 40} \tabularnewline
60 & kamailio\_parse\_to\_header & 7 & \cellcolor{green!0}{\large 0}/{\footnotesize 40} & \cellcolor{green!0}{\large 0}/{\footnotesize 40} & \cellcolor{green!0}{\large -}{\tiny -} & \cellcolor{green!0}{\large 0}/{\footnotesize 40} & \cellcolor{green!0}{\large 0}/{\footnotesize 42} & \cellcolor{green!0}{\large 0}/{\footnotesize 40} \tabularnewline
\rowcolor{black!10} 61 & kamailio\_parse\_to\_uri & 7 & \cellcolor{green!0}{\large 0}/{\footnotesize 40} & \cellcolor{green!0}{\large 0}/{\footnotesize 40} & \cellcolor{green!0}{\large -}{\tiny -} & \cellcolor{green!0}{\large 0}/{\footnotesize 40} & \cellcolor{green!0}{\large 0}/{\footnotesize 40} & \cellcolor{green!0}{\large 0}/{\footnotesize 40} \tabularnewline
62 & libyang\_lyd\_parse\_data\_mem & 7 & \cellcolor{green!0}{\large 0}/{\footnotesize 40} & \cellcolor{green!0}{\large 0}/{\footnotesize 40} & \cellcolor{green!0}{\large 0}/{\footnotesize 40} & \cellcolor{green!0}{\large 0}/{\footnotesize 40} & \cellcolor{green!0}{\large 0}/{\footnotesize 42} & \cellcolor{green!0}{\large 0}/{\footnotesize 40} \tabularnewline
\rowcolor{black!10} 63 & bind9\_dns\_message\_parse & 8 & \cellcolor{green!0}{\large 0}/{\footnotesize 40} & \cellcolor{green!0}{\large 0}/{\footnotesize 40} & \cellcolor{green!0}{\large -}{\tiny -} & \cellcolor{green!0}{\large 0}/{\footnotesize 40} & \cellcolor{green!0}{\large 0}/{\footnotesize 40} & \cellcolor{green!0}{\large 0}/{\footnotesize 40} \tabularnewline
64 & igraph\_igraph\_read\_graph\_ncol & 8 & \cellcolor{green!0}{\large 0}/{\footnotesize 40} & \cellcolor{green!0}{\large 0}/{\footnotesize 40} & \cellcolor{green!0}{\large 0}/{\footnotesize 40} & \cellcolor{green!0}{\large 0}/{\footnotesize 40} & \cellcolor{green!0}{\large 0}/{\footnotesize 40} & \cellcolor{green!0}{\large 0}/{\footnotesize 40} \tabularnewline
\rowcolor{black!10} 65 & pjsip\_pj\_json\_parse & 8 & \cellcolor{green!0}{\large 0}/{\footnotesize 40} & \cellcolor{green!0}{\large 0}/{\footnotesize 40} & \cellcolor{green!0}{\large 0}/{\footnotesize 40} & \cellcolor{green!0}{\large 0}/{\footnotesize 40} & \cellcolor{green!0}{\large 0}/{\footnotesize 41} & \cellcolor{green!0}{\large 0}/{\footnotesize 40} \tabularnewline
66 & pjsip\_pj\_xml\_parse & 8 & \cellcolor{green!0}{\large 0}/{\footnotesize 40} & \cellcolor{green!0}{\large 0}/{\footnotesize 40} & \cellcolor{green!0}{\large 0}/{\footnotesize 40} & \cellcolor{green!0}{\large 0}/{\footnotesize 40} & \cellcolor{green!0}{\large 0}/{\footnotesize 41} & \cellcolor{green!0}{\large 0}/{\footnotesize 40} \tabularnewline
\rowcolor{black!10} 67 & pjsip\_pjmedia\_sdp\_parse & 8 & \cellcolor{green!0}{\large 0}/{\footnotesize 40} & \cellcolor{green!0}{\large 0}/{\footnotesize 40} & \cellcolor{green!0}{\large 0}/{\footnotesize 40} & \cellcolor{green!0}{\large 0}/{\footnotesize 40} & \cellcolor{green!0}{\large 0}/{\footnotesize 40} & \cellcolor{green!0}{\large 0}/{\footnotesize 41} \tabularnewline
68 & quickjs\_lre\_compile & 8 & \cellcolor{green!0}{\large 0}/{\footnotesize 40} & \cellcolor{green!0}{\large 0}/{\footnotesize 40} & \cellcolor{green!0}{\large -}{\tiny -} & \cellcolor{green!0}{\large 0}/{\footnotesize 40} & \cellcolor{green!0}{\large 0}/{\footnotesize 40} & \cellcolor{green!0}{\large 0}/{\footnotesize 40} \tabularnewline
\rowcolor{black!10} 69 & bind9\_isc\_lex\_getmastertoken & 9 & \cellcolor{green!0}{\large 0}/{\footnotesize 40} & \cellcolor{green!0}{\large 0}/{\footnotesize 40} & \cellcolor{green!0}{\large -}{\tiny -} & \cellcolor{green!0}{\large 0}/{\footnotesize 40} & \cellcolor{green!0}{\large 0}/{\footnotesize 40} & \cellcolor{green!0}{\large 0}/{\footnotesize 40} \tabularnewline
70 & bind9\_isc\_lex\_gettoken & 9 & \cellcolor{green!0}{\large 0}/{\footnotesize 40} & \cellcolor{green!0}{\large 0}/{\footnotesize 40} & \cellcolor{green!0}{\large -}{\tiny -} & \cellcolor{green!0}{\large 0}/{\footnotesize 40} & \cellcolor{green!0}{\large 0}/{\footnotesize 41} & \cellcolor{green!0}{\large 0}/{\footnotesize 40} \tabularnewline
\rowcolor{black!10} 71 & quickjs\_JS\_Eval & 9 & \cellcolor{green!0}{\large 0}/{\footnotesize 40} & \cellcolor{green!0}{\large 0}/{\footnotesize 40} & \cellcolor{green!0}{\large -}{\tiny -} & \cellcolor{green!0}{\large 0}/{\footnotesize 40} & \cellcolor{green!0}{\large 0}/{\footnotesize 41} & \cellcolor{green!0}{\large 0}/{\footnotesize 40} \tabularnewline
72 & igraph\_igraph\_edge\_connectivity & 10 & \cellcolor{green!0}{\large 0}/{\footnotesize 40} & \cellcolor{green!0}{\large 0}/{\footnotesize 40} & \cellcolor{green!0}{\large 0}/{\footnotesize 40} & \cellcolor{green!0}{\large 0}/{\footnotesize 40} & \cellcolor{green!0}{\large 0}/{\footnotesize 40} & \cellcolor{green!0}{\large 0}/{\footnotesize 41} \tabularnewline
\rowcolor{black!10} 73 & pjsip\_pj\_stun\_msg\_decode & 10 & \cellcolor{green!0}{\large 0}/{\footnotesize 40} & \cellcolor{green!0}{\large 0}/{\footnotesize 40} & \cellcolor{green!0}{\large 0}/{\footnotesize 40} & \cellcolor{green!0}{\large 0}/{\footnotesize 40} & \cellcolor{green!0}{\large 0}/{\footnotesize 40} & \cellcolor{green!0}{\large 0}/{\footnotesize 40} \tabularnewline
74 & bind9\_dns\_message\_checksig & 11 & \cellcolor{green!0}{\large 0}/{\footnotesize 40} & \cellcolor{green!0}{\large 0}/{\footnotesize 40} & \cellcolor{green!0}{\large -}{\tiny -} & \cellcolor{green!0}{\large 0}/{\footnotesize 40} & \cellcolor{green!0}{\large 0}/{\footnotesize 40} & \cellcolor{green!0}{\large 0}/{\footnotesize 40} \tabularnewline
\rowcolor{black!10} 75 & libzip\_zip\_fread & 11 & \cellcolor{green!0}{\large 0}/{\footnotesize 40} & \cellcolor{green!0}{\large 0}/{\footnotesize 40} & \cellcolor{green!0}{\large 0}/{\footnotesize 40} & \cellcolor{green!0}{\large 0}/{\footnotesize 40} & \cellcolor{green!0}{\large 0}/{\footnotesize 40} & \cellcolor{green!0}{\large 0}/{\footnotesize 40} \tabularnewline
76 & bind9\_dns\_rdata\_fromtext & 12 & \cellcolor{green!0}{\large 0}/{\footnotesize 40} & \cellcolor{green!0}{\large 0}/{\footnotesize 40} & \cellcolor{green!0}{\large -}{\tiny -} & \cellcolor{green!0}{\large 0}/{\footnotesize 40} & \cellcolor{green!0}{\large 0}/{\footnotesize 41} & \cellcolor{green!0}{\large 0}/{\footnotesize 40} \tabularnewline
\rowcolor{black!10} 77 & igraph\_igraph\_all\_minimal\_st\_separators & 12 & \cellcolor{green!0}{\large 0}/{\footnotesize 40} & \cellcolor{green!0}{\large 0}/{\footnotesize 40} & \cellcolor{green!0}{\large 0}/{\footnotesize 40} & \cellcolor{green!0}{\large 0}/{\footnotesize 40} & \cellcolor{green!0}{\large 0}/{\footnotesize 40} & \cellcolor{green!0}{\large 0}/{\footnotesize 42} \tabularnewline
78 & igraph\_igraph\_minimum\_size\_separators & 12 & \cellcolor{green!0}{\large 0}/{\footnotesize 40} & \cellcolor{green!0}{\large 0}/{\footnotesize 40} & \cellcolor{green!0}{\large 0}/{\footnotesize 40} & \cellcolor{green!0}{\large 0}/{\footnotesize 40} & \cellcolor{green!0}{\large 0}/{\footnotesize 40} & \cellcolor{green!0}{\large 0}/{\footnotesize 40} \tabularnewline
\rowcolor{black!10} 79 & pjsip\_pjsip\_parse\_msg & 12 & \cellcolor{green!0}{\large 0}/{\footnotesize 40} & \cellcolor{green!0}{\large 0}/{\footnotesize 40} & \cellcolor{green!0}{\large 0}/{\footnotesize 40} & \cellcolor{green!0}{\large 0}/{\footnotesize 40} & \cellcolor{green!0}{\large 0}/{\footnotesize 40} & \cellcolor{green!0}{\large 0}/{\footnotesize 40} \tabularnewline
80 & igraph\_igraph\_automorphism\_group & 13 & \cellcolor{green!0}{\large 0}/{\footnotesize 40} & \cellcolor{green!0}{\large 0}/{\footnotesize 40} & \cellcolor{green!0}{\large 0}/{\footnotesize 40} & \cellcolor{green!0}{\large 0}/{\footnotesize 40} & \cellcolor{green!0}{\large 0}/{\footnotesize 41} & \cellcolor{green!0}{\large 0}/{\footnotesize 40} \tabularnewline
\rowcolor{black!10} 81 & libmodbus\_modbus\_read\_bits & 15 & \cellcolor{green!0}{\large 0}/{\footnotesize 40} & \cellcolor{green!0}{\large 0}/{\footnotesize 40} & \cellcolor{green!0}{\large 0}/{\footnotesize 40} & \cellcolor{green!0}{\large 0}/{\footnotesize 40} & \cellcolor{green!0}{\large 0}/{\footnotesize 40} & \cellcolor{green!0}{\large 0}/{\footnotesize 42} \tabularnewline
82 & libmodbus\_modbus\_read\_registers & 15 & \cellcolor{green!0}{\large 0}/{\footnotesize 40} & \cellcolor{green!0}{\large 0}/{\footnotesize 40} & \cellcolor{green!0}{\large 0}/{\footnotesize 40} & \cellcolor{green!0}{\large 0}/{\footnotesize 40} & \cellcolor{green!0}{\large 0}/{\footnotesize 40} & \cellcolor{green!0}{\large 0}/{\footnotesize 40} \tabularnewline
\rowcolor{black!10} 83 & civetweb\_mg\_get\_response & 17 & \cellcolor{green!0}{\large 0}/{\footnotesize 40} & \cellcolor{green!0}{\large 0}/{\footnotesize 40} & \cellcolor{green!0}{\large 0}/{\footnotesize 40} & \cellcolor{green!0}{\large 0}/{\footnotesize 40} & \cellcolor{green!0}{\large 0}/{\footnotesize 40} & \cellcolor{green!0}{\large 0}/{\footnotesize 40} \tabularnewline
84 & bind9\_dns\_master\_loadbuffer & 20 & \cellcolor{green!0}{\large 0}/{\footnotesize 40} & \cellcolor{green!0}{\large 0}/{\footnotesize 40} & \cellcolor{green!0}{\large -}{\tiny -} & \cellcolor{green!0}{\large 0}/{\footnotesize 40} & \cellcolor{green!0}{\large 0}/{\footnotesize 40} & \cellcolor{green!0}{\large 0}/{\footnotesize 40} \tabularnewline
\rowcolor{black!10} 85 & libmodbus\_modbus\_receive & 33 & \cellcolor{green!0}{\large 0}/{\footnotesize 40} & \cellcolor{green!0}{\large 0}/{\footnotesize 40} & \cellcolor{green!0}{\large 0}/{\footnotesize 40} & \cellcolor{green!0}{\large 0}/{\footnotesize 40} & \cellcolor{green!0}{\large 0}/{\footnotesize 40} & \cellcolor{green!0}{\large 0}/{\footnotesize 40} \tabularnewline
86 & tmux\_input\_parse\_buffer & 42 & \cellcolor{green!0}{\large 0}/{\footnotesize 40} & \cellcolor{green!0}{\large 0}/{\footnotesize 40} & \cellcolor{green!0}{\large -}{\tiny -} & \cellcolor{green!0}{\large 0}/{\footnotesize 40} & \cellcolor{green!0}{\large 0}/{\footnotesize 41} & \cellcolor{green!0}{\large 0}/{\footnotesize 40} \tabularnewline

\bottomrule
%\end{tabular}
%}
%\end{table*}
\end{xltabular}
}
\twocolumn



% model: codellama-34b-instruct, temp: 2.0

\onecolumn
{\small %
\begin{xltabular}[h]{\textwidth}{ccccccccc}
%\begin{table*}[!t]
%\centering
\caption{Evaluation Result of model codellama-34b-instruct with temperature 2.0.} \\
%\resizebox{1.0\linewidth}{!}{
%\begin{tabular}{cccccccccc}
\toprule
Index & Question & Score & NAIVE-40 & BACTX-40 & DOCTX-40 & UGCTX-40 & BA-ITER-40 & ALL-ITER-40 \tabularnewline
\midrule
\rowcolor{black!10} 1 & coturn\_stun\_is\_command\_message\_full\_check\_str & 1 & \cellcolor{green!0}{\large 0}/{\footnotesize 40} & \cellcolor{green!0}{\large 0}/{\footnotesize 40} & \cellcolor{green!0}{\large -}{\tiny -} & \cellcolor{green!0}{\large 0}/{\footnotesize 40} & \cellcolor{green!0}{\large 0}/{\footnotesize 41} & \cellcolor{green!0}{\large 0}/{\footnotesize 40} \tabularnewline
2 & kamailio\_parse\_uri & 1 & \cellcolor{green!0}{\large 0}/{\footnotesize 40} & \cellcolor{green!0}{\large 0}/{\footnotesize 40} & \cellcolor{green!0}{\large -}{\tiny -} & \cellcolor{green!0}{\large 0}/{\footnotesize 40} & \cellcolor{green!0}{\large 0}/{\footnotesize 41} & \cellcolor{green!0}{\large 0}/{\footnotesize 40} \tabularnewline
\rowcolor{black!10} 3 & coturn\_stun\_check\_message\_integrity\_str & 2 & \cellcolor{green!0}{\large 0}/{\footnotesize 40} & \cellcolor{green!0}{\large 0}/{\footnotesize 40} & \cellcolor{green!0}{\large -}{\tiny -} & \cellcolor{green!0}{\large 0}/{\footnotesize 40} & \cellcolor{green!0}{\large 0}/{\footnotesize 40} & \cellcolor{green!0}{\large 0}/{\footnotesize 41} \tabularnewline
4 & libiec61850\_MmsValue\_decodeMmsData & 2 & \cellcolor{green!0}{\large 0}/{\footnotesize 40} & \cellcolor{green!0}{\large 0}/{\footnotesize 40} & \cellcolor{green!0}{\large 0}/{\footnotesize 40} & \cellcolor{green!0}{\large 0}/{\footnotesize 40} & \cellcolor{green!0}{\large 0}/{\footnotesize 40} & \cellcolor{green!0}{\large 0}/{\footnotesize 40} \tabularnewline
\rowcolor{black!10} 5 & md4c\_md\_html & 2 & \cellcolor{green!0}{\large 0}/{\footnotesize 40} & \cellcolor{green!0}{\large 0}/{\footnotesize 40} & \cellcolor{green!0}{\large 0}/{\footnotesize 40} & \cellcolor{green!0}{\large 0}/{\footnotesize 40} & \cellcolor{green!0}{\large 0}/{\footnotesize 41} & \cellcolor{green!0}{\large 0}/{\footnotesize 40} \tabularnewline
6 & spdk\_spdk\_json\_parse & 2 & \cellcolor{green!0}{\large 0}/{\footnotesize 40} & \cellcolor{green!0}{\large 0}/{\footnotesize 40} & \cellcolor{green!0}{\large -}{\tiny -} & \cellcolor{green!0}{\large 0}/{\footnotesize 40} & \cellcolor{green!0}{\large 0}/{\footnotesize 40} & \cellcolor{green!0}{\large 0}/{\footnotesize 40} \tabularnewline
\rowcolor{black!10} 7 & croaring\_roaring\_bitmap\_portable\_deserialize\_safe & 3 & \cellcolor{green!0}{\large 0}/{\footnotesize 40} & \cellcolor{green!0}{\large 0}/{\footnotesize 40} & \cellcolor{green!0}{\large 0}/{\footnotesize 40} & \cellcolor{green!0}{\large 0}/{\footnotesize 40} & \cellcolor{green!0}{\large 0}/{\footnotesize 40} & \cellcolor{green!0}{\large 0}/{\footnotesize 40} \tabularnewline
8 & lua\_luaL\_loadbufferx & 3 & \cellcolor{green!0}{\large 0}/{\footnotesize 40} & \cellcolor{green!0}{\large 0}/{\footnotesize 40} & \cellcolor{green!0}{\large 0}/{\footnotesize 40} & \cellcolor{green!0}{\large 0}/{\footnotesize 40} & \cellcolor{green!0}{\large 0}/{\footnotesize 40} & \cellcolor{green!0}{\large 0}/{\footnotesize 41} \tabularnewline
\rowcolor{black!10} 9 & w3m\_wc\_Str\_conv\_with\_detect & 3 & \cellcolor{green!0}{\large 0}/{\footnotesize 40} & \cellcolor{green!0}{\large 0}/{\footnotesize 40} & \cellcolor{green!0}{\large -}{\tiny -} & \cellcolor{green!0}{\large 0}/{\footnotesize 40} & \cellcolor{green!0}{\large 0}/{\footnotesize 40} & \cellcolor{green!0}{\large 0}/{\footnotesize 40} \tabularnewline
10 & bind9\_dns\_name\_fromwire & 4 & \cellcolor{green!0}{\large 0}/{\footnotesize 40} & \cellcolor{green!0}{\large 0}/{\footnotesize 40} & \cellcolor{green!0}{\large -}{\tiny -} & \cellcolor{green!0}{\large 0}/{\footnotesize 40} & \cellcolor{green!0}{\large 0}/{\footnotesize 40} & \cellcolor{green!0}{\large 0}/{\footnotesize 40} \tabularnewline
\rowcolor{black!10} 11 & gdk-pixbuf\_gdk\_pixbuf\_animation\_new\_from\_file & 4 & \cellcolor{green!0}{\large 0}/{\footnotesize 40} & \cellcolor{green!0}{\large 0}/{\footnotesize 40} & \cellcolor{green!0}{\large 0}/{\footnotesize 40} & \cellcolor{green!0}{\large 0}/{\footnotesize 40} & \cellcolor{green!0}{\large 0}/{\footnotesize 40} & \cellcolor{green!0}{\large 0}/{\footnotesize 40} \tabularnewline
12 & gdk-pixbuf\_gdk\_pixbuf\_new\_from\_data & 4 & \cellcolor{green!0}{\large 0}/{\footnotesize 40} & \cellcolor{green!0}{\large 0}/{\footnotesize 40} & \cellcolor{green!0}{\large 0}/{\footnotesize 40} & \cellcolor{green!0}{\large 0}/{\footnotesize 40} & \cellcolor{green!0}{\large 0}/{\footnotesize 40} & \cellcolor{green!0}{\large 0}/{\footnotesize 40} \tabularnewline
\rowcolor{black!10} 13 & gdk-pixbuf\_gdk\_pixbuf\_new\_from\_file & 4 & \cellcolor{green!0}{\large 0}/{\footnotesize 40} & \cellcolor{green!0}{\large 0}/{\footnotesize 40} & \cellcolor{green!0}{\large 0}/{\footnotesize 40} & \cellcolor{green!0}{\large 0}/{\footnotesize 40} & \cellcolor{green!0}{\large 0}/{\footnotesize 40} & \cellcolor{green!0}{\large 0}/{\footnotesize 40} \tabularnewline
14 & gdk-pixbuf\_gdk\_pixbuf\_new\_from\_stream & 4 & \cellcolor{green!0}{\large 0}/{\footnotesize 40} & \cellcolor{green!0}{\large 0}/{\footnotesize 40} & \cellcolor{green!0}{\large 0}/{\footnotesize 40} & \cellcolor{green!0}{\large 0}/{\footnotesize 40} & \cellcolor{green!0}{\large 0}/{\footnotesize 40} & \cellcolor{green!0}{\large 0}/{\footnotesize 40} \tabularnewline
\rowcolor{black!10} 15 & gpac\_gf\_isom\_open\_file & 4 & \cellcolor{green!0}{\large 0}/{\footnotesize 40} & \cellcolor{green!0}{\large 0}/{\footnotesize 40} & \cellcolor{green!0}{\large -}{\tiny -} & \cellcolor{green!0}{\large 0}/{\footnotesize 40} & \cellcolor{green!0}{\large 0}/{\footnotesize 40} & \cellcolor{green!0}{\large 0}/{\footnotesize 40} \tabularnewline
16 & libbpf\_bpf\_object\_\_open\_mem & 4 & \cellcolor{green!0}{\large 0}/{\footnotesize 40} & \cellcolor{green!0}{\large 0}/{\footnotesize 40} & \cellcolor{green!0}{\large 0}/{\footnotesize 40} & \cellcolor{green!0}{\large 0}/{\footnotesize 40} & \cellcolor{green!0}{\large 0}/{\footnotesize 40} & \cellcolor{green!0}{\large 0}/{\footnotesize 40} \tabularnewline
\rowcolor{black!10} 17 & libpg\_query\_pg\_query\_parse & 4 & \cellcolor{green!0}{\large 0}/{\footnotesize 40} & \cellcolor{green!0}{\large 0}/{\footnotesize 40} & \cellcolor{green!0}{\large -}{\tiny -} & \cellcolor{green!0}{\large 0}/{\footnotesize 40} & \cellcolor{green!0}{\large 0}/{\footnotesize 40} & \cellcolor{green!0}{\large 0}/{\footnotesize 40} \tabularnewline
18 & libucl\_ucl\_parser\_add\_string & 4 & \cellcolor{green!0}{\large 0}/{\footnotesize 40} & \cellcolor{green!0}{\large 0}/{\footnotesize 40} & \cellcolor{green!0}{\large 0}/{\footnotesize 40} & \cellcolor{green!0}{\large 0}/{\footnotesize 40} & \cellcolor{green!0}{\large 0}/{\footnotesize 40} & \cellcolor{green!0}{\large 0}/{\footnotesize 40} \tabularnewline
\rowcolor{black!10} 19 & oniguruma\_onig\_new & 4 & \cellcolor{green!0}{\large 0}/{\footnotesize 40} & \cellcolor{green!0}{\large 0}/{\footnotesize 40} & \cellcolor{green!0}{\large 0}/{\footnotesize 40} & \cellcolor{green!0}{\large 0}/{\footnotesize 40} & \cellcolor{green!0}{\large 0}/{\footnotesize 40} & \cellcolor{green!0}{\large 0}/{\footnotesize 41} \tabularnewline
20 & pupnp\_ixmlLoadDocumentEx & 4 & \cellcolor{green!0}{\large 0}/{\footnotesize 40} & \cellcolor{green!0}{\large 0}/{\footnotesize 40} & \cellcolor{green!0}{\large 0}/{\footnotesize 40} & \cellcolor{green!0}{\large 0}/{\footnotesize 40} & \cellcolor{green!0}{\large 0}/{\footnotesize 40} & \cellcolor{green!0}{\large 0}/{\footnotesize 40} \tabularnewline
\rowcolor{black!10} 21 & gdk-pixbuf\_gdk\_pixbuf\_new\_from\_file\_at\_scale & 5 & \cellcolor{green!0}{\large 0}/{\footnotesize 40} & \cellcolor{green!0}{\large 0}/{\footnotesize 40} & \cellcolor{green!0}{\large 0}/{\footnotesize 40} & \cellcolor{green!0}{\large 0}/{\footnotesize 40} & \cellcolor{green!0}{\large 0}/{\footnotesize 41} & \cellcolor{green!0}{\large 0}/{\footnotesize 40} \tabularnewline
22 & inchi\_GetINCHIKeyFromINCHI & 5 & \cellcolor{green!0}{\large 0}/{\footnotesize 40} & \cellcolor{green!0}{\large 0}/{\footnotesize 40} & \cellcolor{green!0}{\large 0}/{\footnotesize 40} & \cellcolor{green!0}{\large 0}/{\footnotesize 40} & \cellcolor{green!0}{\large 0}/{\footnotesize 40} & \cellcolor{green!0}{\large 0}/{\footnotesize 40} \tabularnewline
\rowcolor{black!10} 23 & libdwarf\_dwarf\_init\_b & 5 & \cellcolor{green!0}{\large 0}/{\footnotesize 40} & \cellcolor{green!0}{\large 0}/{\footnotesize 40} & \cellcolor{green!0}{\large 0}/{\footnotesize 40} & \cellcolor{green!0}{\large 0}/{\footnotesize 40} & \cellcolor{green!0}{\large 0}/{\footnotesize 40} & \cellcolor{green!0}{\large 0}/{\footnotesize 40} \tabularnewline
24 & libdwarf\_dwarf\_init\_path & 5 & \cellcolor{green!0}{\large 0}/{\footnotesize 40} & \cellcolor{green!0}{\large 0}/{\footnotesize 40} & \cellcolor{green!0}{\large 0}/{\footnotesize 40} & \cellcolor{green!0}{\large 0}/{\footnotesize 40} & \cellcolor{green!0}{\large 0}/{\footnotesize 40} & \cellcolor{green!0}{\large 0}/{\footnotesize 40} \tabularnewline
\rowcolor{black!10} 25 & liblouis\_lou\_compileString & 5 & \cellcolor{green!0}{\large 0}/{\footnotesize 40} & \cellcolor{green!0}{\large 0}/{\footnotesize 40} & \cellcolor{green!0}{\large 0}/{\footnotesize 40} & \cellcolor{green!0}{\large 0}/{\footnotesize 40} & \cellcolor{green!0}{\large 0}/{\footnotesize 40} & \cellcolor{green!0}{\large 0}/{\footnotesize 41} \tabularnewline
26 & selinux\_cil\_compile & 5 & \cellcolor{green!0}{\large 0}/{\footnotesize 40} & \cellcolor{green!0}{\large 0}/{\footnotesize 40} & \cellcolor{green!0}{\large -}{\tiny -} & \cellcolor{green!0}{\large 0}/{\footnotesize 40} & \cellcolor{green!0}{\large 0}/{\footnotesize 40} & \cellcolor{green!0}{\large 0}/{\footnotesize 40} \tabularnewline
\rowcolor{black!10} 27 & bind9\_dns\_name\_fromtext & 6 & \cellcolor{green!0}{\large 0}/{\footnotesize 40} & \cellcolor{green!0}{\large 0}/{\footnotesize 40} & \cellcolor{green!0}{\large -}{\tiny -} & \cellcolor{green!0}{\large 0}/{\footnotesize 40} & \cellcolor{green!0}{\large 0}/{\footnotesize 40} & \cellcolor{green!0}{\large 0}/{\footnotesize 40} \tabularnewline
28 & bind9\_dns\_rdata\_fromwire & 6 & \cellcolor{green!0}{\large 0}/{\footnotesize 40} & \cellcolor{green!0}{\large 0}/{\footnotesize 40} & \cellcolor{green!0}{\large -}{\tiny -} & \cellcolor{green!0}{\large 0}/{\footnotesize 40} & \cellcolor{green!0}{\large 0}/{\footnotesize 40} & \cellcolor{green!0}{\large 0}/{\footnotesize 40} \tabularnewline
\rowcolor{black!10} 29 & coturn\_stun\_is\_binding\_response & 6 & \cellcolor{green!0}{\large 0}/{\footnotesize 40} & \cellcolor{green!0}{\large 0}/{\footnotesize 40} & \cellcolor{green!0}{\large -}{\tiny -} & \cellcolor{green!0}{\large 0}/{\footnotesize 40} & \cellcolor{green!0}{\large 0}/{\footnotesize 41} & \cellcolor{green!0}{\large 0}/{\footnotesize 40} \tabularnewline
30 & coturn\_stun\_is\_command\_message & 6 & \cellcolor{green!0}{\large 0}/{\footnotesize 40} & \cellcolor{green!0}{\large 0}/{\footnotesize 40} & \cellcolor{green!0}{\large 0}/{\footnotesize 40} & \cellcolor{green!0}{\large 0}/{\footnotesize 40} & \cellcolor{green!0}{\large 0}/{\footnotesize 41} & \cellcolor{green!0}{\large 0}/{\footnotesize 40} \tabularnewline
\rowcolor{black!10} 31 & coturn\_stun\_is\_response & 6 & \cellcolor{green!0}{\large 0}/{\footnotesize 40} & \cellcolor{green!0}{\large 0}/{\footnotesize 40} & \cellcolor{green!0}{\large -}{\tiny -} & \cellcolor{green!0}{\large 0}/{\footnotesize 40} & \cellcolor{green!0}{\large 0}/{\footnotesize 40} & \cellcolor{green!0}{\large 0}/{\footnotesize 40} \tabularnewline
32 & coturn\_stun\_is\_success\_response & 6 & \cellcolor{green!0}{\large 0}/{\footnotesize 40} & \cellcolor{green!0}{\large 0}/{\footnotesize 40} & \cellcolor{green!0}{\large -}{\tiny -} & \cellcolor{green!0}{\large 0}/{\footnotesize 40} & \cellcolor{green!0}{\large 0}/{\footnotesize 40} & \cellcolor{green!0}{\large 0}/{\footnotesize 40} \tabularnewline
\rowcolor{black!10} 33 & hiredis\_redisFormatCommand & 6 & \cellcolor{green!0}{\large 0}/{\footnotesize 40} & \cellcolor{green!0}{\large 0}/{\footnotesize 40} & \cellcolor{green!0}{\large -}{\tiny -} & \cellcolor{green!0}{\large 0}/{\footnotesize 40} & \cellcolor{green!0}{\large 0}/{\footnotesize 40} & \cellcolor{green!0}{\large 0}/{\footnotesize 40} \tabularnewline
34 & igraph\_igraph\_read\_graph\_dl & 6 & \cellcolor{green!0}{\large 0}/{\footnotesize 40} & \cellcolor{green!0}{\large 0}/{\footnotesize 40} & \cellcolor{green!0}{\large 0}/{\footnotesize 40} & \cellcolor{green!0}{\large 0}/{\footnotesize 40} & \cellcolor{green!0}{\large 0}/{\footnotesize 40} & \cellcolor{green!0}{\large 0}/{\footnotesize 40} \tabularnewline
\rowcolor{black!10} 35 & igraph\_igraph\_read\_graph\_edgelist & 6 & \cellcolor{green!0}{\large 0}/{\footnotesize 40} & \cellcolor{green!0}{\large 0}/{\footnotesize 40} & \cellcolor{green!0}{\large 0}/{\footnotesize 40} & \cellcolor{green!0}{\large 0}/{\footnotesize 40} & \cellcolor{green!0}{\large 0}/{\footnotesize 41} & \cellcolor{green!0}{\large 0}/{\footnotesize 40} \tabularnewline
36 & igraph\_igraph\_read\_graph\_gml & 6 & \cellcolor{green!0}{\large 0}/{\footnotesize 40} & \cellcolor{green!0}{\large 0}/{\footnotesize 40} & \cellcolor{green!0}{\large 0}/{\footnotesize 40} & \cellcolor{green!0}{\large 0}/{\footnotesize 40} & \cellcolor{green!0}{\large 0}/{\footnotesize 40} & \cellcolor{green!0}{\large 0}/{\footnotesize 40} \tabularnewline
\rowcolor{black!10} 37 & igraph\_igraph\_read\_graph\_graphdb & 6 & \cellcolor{green!0}{\large 0}/{\footnotesize 40} & \cellcolor{green!0}{\large 0}/{\footnotesize 40} & \cellcolor{green!0}{\large 0}/{\footnotesize 40} & \cellcolor{green!0}{\large 0}/{\footnotesize 40} & \cellcolor{green!0}{\large 0}/{\footnotesize 40} & \cellcolor{green!0}{\large 0}/{\footnotesize 40} \tabularnewline
38 & igraph\_igraph\_read\_graph\_graphml & 6 & \cellcolor{green!0}{\large 0}/{\footnotesize 40} & \cellcolor{green!0}{\large 0}/{\footnotesize 40} & \cellcolor{green!0}{\large 0}/{\footnotesize 40} & \cellcolor{green!0}{\large 0}/{\footnotesize 40} & \cellcolor{green!0}{\large 0}/{\footnotesize 40} & \cellcolor{green!0}{\large 0}/{\footnotesize 40} \tabularnewline
\rowcolor{black!10} 39 & igraph\_igraph\_read\_graph\_lgl & 6 & \cellcolor{green!0}{\large 0}/{\footnotesize 40} & \cellcolor{green!0}{\large 0}/{\footnotesize 40} & \cellcolor{green!0}{\large 0}/{\footnotesize 40} & \cellcolor{green!0}{\large 0}/{\footnotesize 40} & \cellcolor{green!0}{\large 0}/{\footnotesize 40} & \cellcolor{green!0}{\large 0}/{\footnotesize 40} \tabularnewline
40 & igraph\_igraph\_read\_graph\_pajek & 6 & \cellcolor{green!0}{\large 0}/{\footnotesize 40} & \cellcolor{green!0}{\large 0}/{\footnotesize 40} & \cellcolor{green!0}{\large 0}/{\footnotesize 40} & \cellcolor{green!0}{\large 0}/{\footnotesize 40} & \cellcolor{green!0}{\large 0}/{\footnotesize 40} & \cellcolor{green!0}{\large 0}/{\footnotesize 40} \tabularnewline
\rowcolor{black!10} 41 & inchi\_GetINCHIfromINCHI & 6 & \cellcolor{green!0}{\large 0}/{\footnotesize 40} & \cellcolor{green!0}{\large 0}/{\footnotesize 40} & \cellcolor{green!0}{\large 0}/{\footnotesize 40} & \cellcolor{green!0}{\large 0}/{\footnotesize 40} & \cellcolor{green!0}{\large 0}/{\footnotesize 40} & \cellcolor{green!0}{\large 0}/{\footnotesize 40} \tabularnewline
42 & inchi\_GetStructFromINCHI & 6 & \cellcolor{green!0}{\large 0}/{\footnotesize 40} & \cellcolor{green!0}{\large 0}/{\footnotesize 40} & \cellcolor{green!0}{\large 0}/{\footnotesize 40} & \cellcolor{green!0}{\large 0}/{\footnotesize 40} & \cellcolor{green!0}{\large 0}/{\footnotesize 41} & \cellcolor{green!0}{\large 0}/{\footnotesize 40} \tabularnewline
\rowcolor{black!10} 43 & kamailio\_parse\_msg & 6 & \cellcolor{green!0}{\large 0}/{\footnotesize 40} & \cellcolor{green!0}{\large 0}/{\footnotesize 40} & \cellcolor{green!0}{\large -}{\tiny -} & \cellcolor{green!0}{\large 0}/{\footnotesize 40} & \cellcolor{green!0}{\large 0}/{\footnotesize 40} & \cellcolor{green!0}{\large 0}/{\footnotesize 41} \tabularnewline
44 & libyang\_lys\_parse\_mem & 6 & \cellcolor{green!0}{\large 0}/{\footnotesize 40} & \cellcolor{green!0}{\large 0}/{\footnotesize 40} & \cellcolor{green!0}{\large 0}/{\footnotesize 40} & \cellcolor{green!0}{\large 0}/{\footnotesize 40} & \cellcolor{green!0}{\large 0}/{\footnotesize 40} & \cellcolor{green!0}{\large 0}/{\footnotesize 40} \tabularnewline
\rowcolor{black!10} 45 & proftpd\_pr\_json\_object\_from\_text & 6 & \cellcolor{green!0}{\large 0}/{\footnotesize 40} & \cellcolor{green!0}{\large 0}/{\footnotesize 40} & \cellcolor{green!0}{\large -}{\tiny -} & \cellcolor{green!0}{\large 0}/{\footnotesize 40} & \cellcolor{green!0}{\large 0}/{\footnotesize 40} & \cellcolor{green!0}{\large 0}/{\footnotesize 40} \tabularnewline
46 & selinux\_policydb\_read & 6 & \cellcolor{green!0}{\large 0}/{\footnotesize 40} & \cellcolor{green!0}{\large 0}/{\footnotesize 40} & \cellcolor{green!0}{\large -}{\tiny -} & \cellcolor{green!0}{\large 0}/{\footnotesize 40} & \cellcolor{green!0}{\large 0}/{\footnotesize 40} & \cellcolor{green!0}{\large 0}/{\footnotesize 40} \tabularnewline
\rowcolor{black!10} 47 & kamailio\_get\_src\_address\_socket & 7 & \cellcolor{green!0}{\large 0}/{\footnotesize 40} & \cellcolor{green!0}{\large 0}/{\footnotesize 40} & \cellcolor{green!0}{\large 0}/{\footnotesize 40} & \cellcolor{green!0}{\large 0}/{\footnotesize 40} & \cellcolor{green!0}{\large 0}/{\footnotesize 40} & \cellcolor{green!0}{\large 0}/{\footnotesize 40} \tabularnewline
48 & kamailio\_get\_src\_uri & 7 & \cellcolor{green!0}{\large 0}/{\footnotesize 40} & \cellcolor{green!0}{\large 0}/{\footnotesize 40} & \cellcolor{green!0}{\large 0}/{\footnotesize 40} & \cellcolor{green!0}{\large 0}/{\footnotesize 40} & \cellcolor{green!0}{\large 0}/{\footnotesize 40} & \cellcolor{green!0}{\large 0}/{\footnotesize 40} \tabularnewline
\rowcolor{black!10} 49 & kamailio\_parse\_content\_disposition & 7 & \cellcolor{green!0}{\large 0}/{\footnotesize 40} & \cellcolor{green!0}{\large 0}/{\footnotesize 40} & \cellcolor{green!0}{\large 0}/{\footnotesize 40} & \cellcolor{green!0}{\large 0}/{\footnotesize 40} & \cellcolor{green!0}{\large 0}/{\footnotesize 41} & \cellcolor{green!0}{\large 0}/{\footnotesize 40} \tabularnewline
50 & kamailio\_parse\_diversion\_header & 7 & \cellcolor{green!0}{\large 0}/{\footnotesize 40} & \cellcolor{green!0}{\large 0}/{\footnotesize 40} & \cellcolor{green!0}{\large 0}/{\footnotesize 40} & \cellcolor{green!0}{\large 0}/{\footnotesize 40} & \cellcolor{green!0}{\large 0}/{\footnotesize 40} & \cellcolor{green!0}{\large 0}/{\footnotesize 40} \tabularnewline
\rowcolor{black!10} 51 & kamailio\_parse\_from\_header & 7 & \cellcolor{green!0}{\large 0}/{\footnotesize 40} & \cellcolor{green!0}{\large 0}/{\footnotesize 40} & \cellcolor{green!0}{\large -}{\tiny -} & \cellcolor{green!0}{\large 0}/{\footnotesize 40} & \cellcolor{green!0}{\large 0}/{\footnotesize 40} & \cellcolor{green!0}{\large 0}/{\footnotesize 41} \tabularnewline
52 & kamailio\_parse\_from\_uri & 7 & \cellcolor{green!0}{\large 0}/{\footnotesize 40} & \cellcolor{green!0}{\large 0}/{\footnotesize 40} & \cellcolor{green!0}{\large -}{\tiny -} & \cellcolor{green!0}{\large 0}/{\footnotesize 40} & \cellcolor{green!0}{\large 0}/{\footnotesize 40} & \cellcolor{green!0}{\large 0}/{\footnotesize 40} \tabularnewline
\rowcolor{black!10} 53 & kamailio\_parse\_headers & 7 & \cellcolor{green!0}{\large 0}/{\footnotesize 40} & \cellcolor{green!0}{\large 0}/{\footnotesize 40} & \cellcolor{green!0}{\large -}{\tiny -} & \cellcolor{green!0}{\large 0}/{\footnotesize 40} & \cellcolor{green!0}{\large 0}/{\footnotesize 40} & \cellcolor{green!0}{\large 0}/{\footnotesize 40} \tabularnewline
54 & kamailio\_parse\_identityinfo\_header & 7 & \cellcolor{green!0}{\large 0}/{\footnotesize 40} & \cellcolor{green!0}{\large 0}/{\footnotesize 40} & \cellcolor{green!0}{\large -}{\tiny -} & \cellcolor{green!0}{\large 0}/{\footnotesize 40} & \cellcolor{green!0}{\large 0}/{\footnotesize 40} & \cellcolor{green!0}{\large 0}/{\footnotesize 40} \tabularnewline
\rowcolor{black!10} 55 & kamailio\_parse\_pai\_header & 7 & \cellcolor{green!0}{\large 0}/{\footnotesize 40} & \cellcolor{green!0}{\large 0}/{\footnotesize 40} & \cellcolor{green!0}{\large -}{\tiny -} & \cellcolor{green!0}{\large 0}/{\footnotesize 40} & \cellcolor{green!0}{\large 0}/{\footnotesize 40} & \cellcolor{green!0}{\large 0}/{\footnotesize 41} \tabularnewline
56 & kamailio\_parse\_privacy & 7 & \cellcolor{green!0}{\large 0}/{\footnotesize 40} & \cellcolor{green!0}{\large 0}/{\footnotesize 40} & \cellcolor{green!0}{\large 0}/{\footnotesize 40} & \cellcolor{green!0}{\large 0}/{\footnotesize 40} & \cellcolor{green!0}{\large 0}/{\footnotesize 40} & \cellcolor{green!0}{\large 0}/{\footnotesize 40} \tabularnewline
\rowcolor{black!10} 57 & kamailio\_parse\_record\_route\_headers & 7 & \cellcolor{green!0}{\large 0}/{\footnotesize 40} & \cellcolor{green!0}{\large 0}/{\footnotesize 40} & \cellcolor{green!0}{\large -}{\tiny -} & \cellcolor{green!0}{\large 0}/{\footnotesize 40} & \cellcolor{green!0}{\large 0}/{\footnotesize 40} & \cellcolor{green!0}{\large 0}/{\footnotesize 40} \tabularnewline
58 & kamailio\_parse\_refer\_to\_header & 7 & \cellcolor{green!0}{\large 0}/{\footnotesize 40} & \cellcolor{green!0}{\large 0}/{\footnotesize 40} & \cellcolor{green!0}{\large -}{\tiny -} & \cellcolor{green!0}{\large 0}/{\footnotesize 40} & \cellcolor{green!0}{\large 0}/{\footnotesize 40} & \cellcolor{green!0}{\large 0}/{\footnotesize 40} \tabularnewline
\rowcolor{black!10} 59 & kamailio\_parse\_route\_headers & 7 & \cellcolor{green!0}{\large 0}/{\footnotesize 40} & \cellcolor{green!0}{\large 0}/{\footnotesize 40} & \cellcolor{green!0}{\large -}{\tiny -} & \cellcolor{green!0}{\large 0}/{\footnotesize 40} & \cellcolor{green!0}{\large 0}/{\footnotesize 40} & \cellcolor{green!0}{\large 0}/{\footnotesize 40} \tabularnewline
60 & kamailio\_parse\_to\_header & 7 & \cellcolor{green!0}{\large 0}/{\footnotesize 40} & \cellcolor{green!0}{\large 0}/{\footnotesize 40} & \cellcolor{green!0}{\large -}{\tiny -} & \cellcolor{green!0}{\large 0}/{\footnotesize 40} & \cellcolor{green!0}{\large 0}/{\footnotesize 40} & \cellcolor{green!0}{\large 0}/{\footnotesize 40} \tabularnewline
\rowcolor{black!10} 61 & kamailio\_parse\_to\_uri & 7 & \cellcolor{green!0}{\large 0}/{\footnotesize 40} & \cellcolor{green!0}{\large 0}/{\footnotesize 40} & \cellcolor{green!0}{\large -}{\tiny -} & \cellcolor{green!0}{\large 0}/{\footnotesize 40} & \cellcolor{green!0}{\large 0}/{\footnotesize 40} & \cellcolor{green!0}{\large 0}/{\footnotesize 40} \tabularnewline
62 & libyang\_lyd\_parse\_data\_mem & 7 & \cellcolor{green!0}{\large 0}/{\footnotesize 40} & \cellcolor{green!0}{\large 0}/{\footnotesize 40} & \cellcolor{green!0}{\large 0}/{\footnotesize 40} & \cellcolor{green!0}{\large 0}/{\footnotesize 40} & \cellcolor{green!0}{\large 0}/{\footnotesize 40} & \cellcolor{green!0}{\large 0}/{\footnotesize 40} \tabularnewline
\rowcolor{black!10} 63 & bind9\_dns\_message\_parse & 8 & \cellcolor{green!0}{\large 0}/{\footnotesize 40} & \cellcolor{green!0}{\large 0}/{\footnotesize 40} & \cellcolor{green!0}{\large -}{\tiny -} & \cellcolor{green!0}{\large 0}/{\footnotesize 40} & \cellcolor{green!0}{\large 0}/{\footnotesize 40} & \cellcolor{green!0}{\large 0}/{\footnotesize 40} \tabularnewline
64 & igraph\_igraph\_read\_graph\_ncol & 8 & \cellcolor{green!0}{\large 0}/{\footnotesize 40} & \cellcolor{green!0}{\large 0}/{\footnotesize 40} & \cellcolor{green!0}{\large 0}/{\footnotesize 40} & \cellcolor{green!0}{\large 0}/{\footnotesize 40} & \cellcolor{green!0}{\large 0}/{\footnotesize 40} & \cellcolor{green!0}{\large 0}/{\footnotesize 40} \tabularnewline
\rowcolor{black!10} 65 & pjsip\_pj\_json\_parse & 8 & \cellcolor{green!0}{\large 0}/{\footnotesize 40} & \cellcolor{green!0}{\large 0}/{\footnotesize 40} & \cellcolor{green!0}{\large 0}/{\footnotesize 40} & \cellcolor{green!0}{\large 0}/{\footnotesize 40} & \cellcolor{green!0}{\large 0}/{\footnotesize 40} & \cellcolor{green!0}{\large 0}/{\footnotesize 40} \tabularnewline
66 & pjsip\_pj\_xml\_parse & 8 & \cellcolor{green!0}{\large 0}/{\footnotesize 40} & \cellcolor{green!0}{\large 0}/{\footnotesize 40} & \cellcolor{green!0}{\large 0}/{\footnotesize 40} & \cellcolor{green!0}{\large 0}/{\footnotesize 40} & \cellcolor{green!0}{\large 0}/{\footnotesize 40} & \cellcolor{green!0}{\large 0}/{\footnotesize 40} \tabularnewline
\rowcolor{black!10} 67 & pjsip\_pjmedia\_sdp\_parse & 8 & \cellcolor{green!0}{\large 0}/{\footnotesize 40} & \cellcolor{green!0}{\large 0}/{\footnotesize 40} & \cellcolor{green!0}{\large 0}/{\footnotesize 40} & \cellcolor{green!0}{\large 0}/{\footnotesize 40} & \cellcolor{green!0}{\large 0}/{\footnotesize 40} & \cellcolor{green!0}{\large 0}/{\footnotesize 40} \tabularnewline
68 & quickjs\_lre\_compile & 8 & \cellcolor{green!0}{\large 0}/{\footnotesize 40} & \cellcolor{green!0}{\large 0}/{\footnotesize 40} & \cellcolor{green!0}{\large -}{\tiny -} & \cellcolor{green!0}{\large 0}/{\footnotesize 40} & \cellcolor{green!0}{\large 0}/{\footnotesize 40} & \cellcolor{green!0}{\large 0}/{\footnotesize 40} \tabularnewline
\rowcolor{black!10} 69 & bind9\_isc\_lex\_getmastertoken & 9 & \cellcolor{green!0}{\large 0}/{\footnotesize 40} & \cellcolor{green!0}{\large 0}/{\footnotesize 40} & \cellcolor{green!0}{\large -}{\tiny -} & \cellcolor{green!0}{\large 0}/{\footnotesize 40} & \cellcolor{green!0}{\large 0}/{\footnotesize 40} & \cellcolor{green!0}{\large 0}/{\footnotesize 41} \tabularnewline
70 & bind9\_isc\_lex\_gettoken & 9 & \cellcolor{green!0}{\large 0}/{\footnotesize 40} & \cellcolor{green!0}{\large 0}/{\footnotesize 40} & \cellcolor{green!0}{\large -}{\tiny -} & \cellcolor{green!0}{\large 0}/{\footnotesize 40} & \cellcolor{green!0}{\large 0}/{\footnotesize 40} & \cellcolor{green!0}{\large 0}/{\footnotesize 40} \tabularnewline
\rowcolor{black!10} 71 & quickjs\_JS\_Eval & 9 & \cellcolor{green!0}{\large 0}/{\footnotesize 40} & \cellcolor{green!0}{\large 0}/{\footnotesize 40} & \cellcolor{green!0}{\large -}{\tiny -} & \cellcolor{green!0}{\large 0}/{\footnotesize 40} & \cellcolor{green!0}{\large 0}/{\footnotesize 40} & \cellcolor{green!0}{\large 0}/{\footnotesize 40} \tabularnewline
72 & igraph\_igraph\_edge\_connectivity & 10 & \cellcolor{green!0}{\large 0}/{\footnotesize 40} & \cellcolor{green!0}{\large 0}/{\footnotesize 40} & \cellcolor{green!0}{\large 0}/{\footnotesize 40} & \cellcolor{green!0}{\large 0}/{\footnotesize 40} & \cellcolor{green!0}{\large 0}/{\footnotesize 40} & \cellcolor{green!0}{\large 0}/{\footnotesize 40} \tabularnewline
\rowcolor{black!10} 73 & pjsip\_pj\_stun\_msg\_decode & 10 & \cellcolor{green!0}{\large 0}/{\footnotesize 40} & \cellcolor{green!0}{\large 0}/{\footnotesize 40} & \cellcolor{green!0}{\large 0}/{\footnotesize 40} & \cellcolor{green!0}{\large 0}/{\footnotesize 40} & \cellcolor{green!0}{\large 0}/{\footnotesize 40} & \cellcolor{green!0}{\large 0}/{\footnotesize 40} \tabularnewline
74 & bind9\_dns\_message\_checksig & 11 & \cellcolor{green!0}{\large 0}/{\footnotesize 40} & \cellcolor{green!0}{\large 0}/{\footnotesize 40} & \cellcolor{green!0}{\large -}{\tiny -} & \cellcolor{green!0}{\large 0}/{\footnotesize 40} & \cellcolor{green!0}{\large 0}/{\footnotesize 40} & \cellcolor{green!0}{\large 0}/{\footnotesize 41} \tabularnewline
\rowcolor{black!10} 75 & libzip\_zip\_fread & 11 & \cellcolor{green!0}{\large 0}/{\footnotesize 40} & \cellcolor{green!0}{\large 0}/{\footnotesize 40} & \cellcolor{green!0}{\large 0}/{\footnotesize 40} & \cellcolor{green!0}{\large 0}/{\footnotesize 40} & \cellcolor{green!0}{\large 0}/{\footnotesize 40} & \cellcolor{green!0}{\large 0}/{\footnotesize 40} \tabularnewline
76 & bind9\_dns\_rdata\_fromtext & 12 & \cellcolor{green!0}{\large 0}/{\footnotesize 40} & \cellcolor{green!0}{\large 0}/{\footnotesize 40} & \cellcolor{green!0}{\large -}{\tiny -} & \cellcolor{green!0}{\large 0}/{\footnotesize 40} & \cellcolor{green!0}{\large 0}/{\footnotesize 41} & \cellcolor{green!0}{\large 0}/{\footnotesize 40} \tabularnewline
\rowcolor{black!10} 77 & igraph\_igraph\_all\_minimal\_st\_separators & 12 & \cellcolor{green!0}{\large 0}/{\footnotesize 40} & \cellcolor{green!0}{\large 0}/{\footnotesize 40} & \cellcolor{green!0}{\large 0}/{\footnotesize 40} & \cellcolor{green!0}{\large 0}/{\footnotesize 40} & \cellcolor{green!0}{\large 0}/{\footnotesize 40} & \cellcolor{green!0}{\large 0}/{\footnotesize 41} \tabularnewline
78 & igraph\_igraph\_minimum\_size\_separators & 12 & \cellcolor{green!0}{\large 0}/{\footnotesize 40} & \cellcolor{green!0}{\large 0}/{\footnotesize 40} & \cellcolor{green!0}{\large 0}/{\footnotesize 40} & \cellcolor{green!0}{\large 0}/{\footnotesize 40} & \cellcolor{green!0}{\large 0}/{\footnotesize 40} & \cellcolor{green!0}{\large 0}/{\footnotesize 40} \tabularnewline
\rowcolor{black!10} 79 & pjsip\_pjsip\_parse\_msg & 12 & \cellcolor{green!0}{\large 0}/{\footnotesize 40} & \cellcolor{green!0}{\large 0}/{\footnotesize 40} & \cellcolor{green!0}{\large 0}/{\footnotesize 40} & \cellcolor{green!0}{\large 0}/{\footnotesize 40} & \cellcolor{green!0}{\large 0}/{\footnotesize 40} & \cellcolor{green!0}{\large 0}/{\footnotesize 40} \tabularnewline
80 & igraph\_igraph\_automorphism\_group & 13 & \cellcolor{green!0}{\large 0}/{\footnotesize 40} & \cellcolor{green!0}{\large 0}/{\footnotesize 40} & \cellcolor{green!0}{\large 0}/{\footnotesize 40} & \cellcolor{green!0}{\large 0}/{\footnotesize 40} & \cellcolor{green!0}{\large 0}/{\footnotesize 40} & \cellcolor{green!0}{\large 0}/{\footnotesize 40} \tabularnewline
\rowcolor{black!10} 81 & libmodbus\_modbus\_read\_bits & 15 & \cellcolor{green!0}{\large 0}/{\footnotesize 40} & \cellcolor{green!0}{\large 0}/{\footnotesize 40} & \cellcolor{green!0}{\large 0}/{\footnotesize 40} & \cellcolor{green!0}{\large 0}/{\footnotesize 40} & \cellcolor{green!0}{\large 0}/{\footnotesize 40} & \cellcolor{green!0}{\large 0}/{\footnotesize 40} \tabularnewline
82 & libmodbus\_modbus\_read\_registers & 15 & \cellcolor{green!0}{\large 0}/{\footnotesize 40} & \cellcolor{green!0}{\large 0}/{\footnotesize 40} & \cellcolor{green!0}{\large 0}/{\footnotesize 40} & \cellcolor{green!0}{\large 0}/{\footnotesize 40} & \cellcolor{green!0}{\large 0}/{\footnotesize 40} & \cellcolor{green!0}{\large 0}/{\footnotesize 40} \tabularnewline
\rowcolor{black!10} 83 & civetweb\_mg\_get\_response & 17 & \cellcolor{green!0}{\large 0}/{\footnotesize 40} & \cellcolor{green!0}{\large 0}/{\footnotesize 40} & \cellcolor{green!0}{\large 0}/{\footnotesize 40} & \cellcolor{green!0}{\large 0}/{\footnotesize 40} & \cellcolor{green!0}{\large 0}/{\footnotesize 41} & \cellcolor{green!0}{\large 0}/{\footnotesize 40} \tabularnewline
84 & bind9\_dns\_master\_loadbuffer & 20 & \cellcolor{green!0}{\large 0}/{\footnotesize 40} & \cellcolor{green!0}{\large 0}/{\footnotesize 40} & \cellcolor{green!0}{\large -}{\tiny -} & \cellcolor{green!0}{\large 0}/{\footnotesize 40} & \cellcolor{green!0}{\large 0}/{\footnotesize 41} & \cellcolor{green!0}{\large 0}/{\footnotesize 40} \tabularnewline
\rowcolor{black!10} 85 & libmodbus\_modbus\_receive & 33 & \cellcolor{green!0}{\large 0}/{\footnotesize 40} & \cellcolor{green!0}{\large 0}/{\footnotesize 40} & \cellcolor{green!0}{\large 0}/{\footnotesize 40} & \cellcolor{green!0}{\large 0}/{\footnotesize 40} & \cellcolor{green!0}{\large 0}/{\footnotesize 41} & \cellcolor{green!0}{\large 0}/{\footnotesize 40} \tabularnewline
86 & tmux\_input\_parse\_buffer & 42 & \cellcolor{green!0}{\large 0}/{\footnotesize 40} & \cellcolor{green!0}{\large 0}/{\footnotesize 40} & \cellcolor{green!0}{\large -}{\tiny -} & \cellcolor{green!0}{\large 0}/{\footnotesize 40} & \cellcolor{green!0}{\large 0}/{\footnotesize 40} & \cellcolor{green!0}{\large 0}/{\footnotesize 40} \tabularnewline

\bottomrule
%\end{tabular}
%}
%\end{table*}
\end{xltabular}
}
\twocolumn



% model: wizardcoder-15b-v1.0, temp: 0.0

\onecolumn
{\small %
\begin{xltabular}[h]{\textwidth}{ccccccccc}
%\begin{table*}[!t]
%\centering
\caption{Evaluation Result of model wizardcoder-15b-v1.0 with temperature 0.0.} \\
%\resizebox{1.0\linewidth}{!}{
%\begin{tabular}{cccccccccc}
\toprule
Index & Question & Score & NAIVE-40 & BACTX-40 & DOCTX-40 & UGCTX-40 & BA-ITER-40 & ALL-ITER-40 \tabularnewline
\midrule
\rowcolor{black!10} 1 & coturn\_stun\_is\_command\_message\_full\_check\_str & 1 & \cellcolor{green!0}{\large 0}/{\footnotesize 40} & \cellcolor{green!100}{\large 40}/{\footnotesize 40} & \cellcolor{green!0}{\large -}{\tiny -} & \cellcolor{green!50}{\large 17}/{\footnotesize 40} & \cellcolor{green!100}{\large 40}/{\footnotesize 40} & \cellcolor{green!50}{\large 27}/{\footnotesize 53} \tabularnewline
2 & kamailio\_parse\_uri & 1 & \cellcolor{green!0}{\large 0}/{\footnotesize 40} & \cellcolor{green!100}{\large 40}/{\footnotesize 40} & \cellcolor{green!0}{\large -}{\tiny -} & \cellcolor{green!100}{\large 37}/{\footnotesize 40} & \cellcolor{green!100}{\large 40}/{\footnotesize 40} & \cellcolor{green!90}{\large 38}/{\footnotesize 45} \tabularnewline
\rowcolor{black!10} 3 & coturn\_stun\_check\_message\_integrity\_str & 2 & \cellcolor{green!0}{\large 0}/{\footnotesize 40} & \cellcolor{green!0}{\large 0}/{\footnotesize 40} & \cellcolor{green!0}{\large -}{\tiny -} & \cellcolor{green!30}{\large 11}/{\footnotesize 40} & \cellcolor{green!0}{\large 0}/{\footnotesize 200} & \cellcolor{green!10}{\large 4}/{\footnotesize 75} \tabularnewline
4 & libiec61850\_MmsValue\_decodeMmsData & 2 & \cellcolor{green!0}{\large 0}/{\footnotesize 40} & \cellcolor{green!0}{\large 0}/{\footnotesize 40} & \cellcolor{green!100}{\large 40}/{\footnotesize 40} & \cellcolor{green!60}{\large 21}/{\footnotesize 40} & \cellcolor{green!0}{\large 0}/{\footnotesize 170} & \cellcolor{green!20}{\large 12}/{\footnotesize 74} \tabularnewline
\rowcolor{black!10} 5 & md4c\_md\_html & 2 & \cellcolor{green!0}{\large 0}/{\footnotesize 40} & \cellcolor{green!0}{\large 0}/{\footnotesize 40} & \cellcolor{green!0}{\large 0}/{\footnotesize 40} & \cellcolor{green!0}{\large 0}/{\footnotesize 40} & \cellcolor{green!0}{\large 0}/{\footnotesize 200} & \cellcolor{green!0}{\large 0}/{\footnotesize 83} \tabularnewline
6 & spdk\_spdk\_json\_parse & 2 & \cellcolor{green!0}{\large 0}/{\footnotesize 40} & \cellcolor{green!0}{\large 0}/{\footnotesize 40} & \cellcolor{green!0}{\large -}{\tiny -} & \cellcolor{green!40}{\large 13}/{\footnotesize 40} & \cellcolor{green!0}{\large 0}/{\footnotesize 200} & \cellcolor{green!10}{\large 7}/{\footnotesize 104} \tabularnewline
\rowcolor{black!10} 7 & croaring\_roaring\_bitmap\_portable\_deserialize\_safe & 3 & \cellcolor{green!0}{\large 0}/{\footnotesize 40} & \cellcolor{green!0}{\large 0}/{\footnotesize 40} & \cellcolor{green!0}{\large 0}/{\footnotesize 40} & \cellcolor{green!30}{\large 9}/{\footnotesize 40} & \cellcolor{green!0}{\large 0}/{\footnotesize 188} & \cellcolor{green!10}{\large 8}/{\footnotesize 95} \tabularnewline
8 & lua\_luaL\_loadbufferx & 3 & \cellcolor{green!100}{\large 40}/{\footnotesize 40} & \cellcolor{green!100}{\large 40}/{\footnotesize 40} & \cellcolor{green!100}{\large 40}/{\footnotesize 40} & \cellcolor{green!50}{\large 20}/{\footnotesize 40} & \cellcolor{green!100}{\large 40}/{\footnotesize 40} & \cellcolor{green!60}{\large 28}/{\footnotesize 49} \tabularnewline
\rowcolor{black!10} 9 & w3m\_wc\_Str\_conv\_with\_detect & 3 & \cellcolor{green!0}{\large 0}/{\footnotesize 40} & \cellcolor{green!0}{\large 0}/{\footnotesize 40} & \cellcolor{green!0}{\large -}{\tiny -} & \cellcolor{green!0}{\large 0}/{\footnotesize 40} & \cellcolor{green!0}{\large 0}/{\footnotesize 200} & \cellcolor{green!0}{\large 0}/{\footnotesize 131} \tabularnewline
10 & bind9\_dns\_name\_fromwire & 4 & \cellcolor{green!0}{\large 0}/{\footnotesize 40} & \cellcolor{green!0}{\large 0}/{\footnotesize 40} & \cellcolor{green!0}{\large -}{\tiny -} & \cellcolor{green!0}{\large 0}/{\footnotesize 40} & \cellcolor{green!0}{\large 0}/{\footnotesize 200} & \cellcolor{green!0}{\large 0}/{\footnotesize 115} \tabularnewline
\rowcolor{black!10} 11 & gdk-pixbuf\_gdk\_pixbuf\_animation\_new\_from\_file & 4 & \cellcolor{green!100}{\large 40}/{\footnotesize 40} & \cellcolor{green!0}{\large 0}/{\footnotesize 40} & \cellcolor{green!0}{\large 0}/{\footnotesize 40} & \cellcolor{green!0}{\large 0}/{\footnotesize 40} & \cellcolor{green!0}{\large 0}/{\footnotesize 200} & \cellcolor{green!0}{\large 0}/{\footnotesize 84} \tabularnewline
12 & gdk-pixbuf\_gdk\_pixbuf\_new\_from\_data & 4 & \cellcolor{green!0}{\large 0}/{\footnotesize 40} & \cellcolor{green!100}{\large 40}/{\footnotesize 40} & \cellcolor{green!0}{\large 0}/{\footnotesize 40} & \cellcolor{green!40}{\large 16}/{\footnotesize 40} & \cellcolor{green!100}{\large 40}/{\footnotesize 40} & \cellcolor{green!30}{\large 14}/{\footnotesize 59} \tabularnewline
\rowcolor{black!10} 13 & gdk-pixbuf\_gdk\_pixbuf\_new\_from\_file & 4 & \cellcolor{green!0}{\large 0}/{\footnotesize 40} & \cellcolor{green!0}{\large 0}/{\footnotesize 40} & \cellcolor{green!0}{\large 0}/{\footnotesize 40} & \cellcolor{green!0}{\large 0}/{\footnotesize 40} & \cellcolor{green!0}{\large 0}/{\footnotesize 200} & \cellcolor{green!0}{\large 0}/{\footnotesize 120} \tabularnewline
14 & gdk-pixbuf\_gdk\_pixbuf\_new\_from\_stream & 4 & \cellcolor{green!0}{\large 0}/{\footnotesize 40} & \cellcolor{green!0}{\large 0}/{\footnotesize 40} & \cellcolor{green!0}{\large 0}/{\footnotesize 40} & \cellcolor{green!90}{\large 33}/{\footnotesize 40} & \cellcolor{green!0}{\large 0}/{\footnotesize 200} & \cellcolor{green!60}{\large 32}/{\footnotesize 61} \tabularnewline
\rowcolor{black!10} 15 & gpac\_gf\_isom\_open\_file & 4 & \cellcolor{green!0}{\large 0}/{\footnotesize 40} & \cellcolor{green!0}{\large 0}/{\footnotesize 40} & \cellcolor{green!0}{\large -}{\tiny -} & \cellcolor{green!0}{\large 0}/{\footnotesize 40} & \cellcolor{green!0}{\large 0}/{\footnotesize 200} & \cellcolor{green!0}{\large 0}/{\footnotesize 101} \tabularnewline
16 & libbpf\_bpf\_object\_\_open\_mem & 4 & \cellcolor{green!0}{\large 0}/{\footnotesize 40} & \cellcolor{green!0}{\large 0}/{\footnotesize 40} & \cellcolor{green!0}{\large 0}/{\footnotesize 40} & \cellcolor{green!50}{\large 17}/{\footnotesize 40} & \cellcolor{green!0}{\large 0}/{\footnotesize 200} & \cellcolor{green!20}{\large 15}/{\footnotesize 83} \tabularnewline
\rowcolor{black!10} 17 & libpg\_query\_pg\_query\_parse & 4 & \cellcolor{green!0}{\large 0}/{\footnotesize 40} & \cellcolor{green!0}{\large 0}/{\footnotesize 40} & \cellcolor{green!0}{\large -}{\tiny -} & \cellcolor{green!70}{\large 27}/{\footnotesize 40} & \cellcolor{green!0}{\large 0}/{\footnotesize 200} & \cellcolor{green!40}{\large 27}/{\footnotesize 68} \tabularnewline
18 & libucl\_ucl\_parser\_add\_string & 4 & \cellcolor{green!0}{\large 0}/{\footnotesize 40} & \cellcolor{green!0}{\large 0}/{\footnotesize 40} & \cellcolor{green!0}{\large 0}/{\footnotesize 40} & \cellcolor{green!0}{\large 0}/{\footnotesize 40} & \cellcolor{green!30}{\large 40}/{\footnotesize 169} & \cellcolor{green!10}{\large 3}/{\footnotesize 114} \tabularnewline
\rowcolor{black!10} 19 & oniguruma\_onig\_new & 4 & \cellcolor{green!100}{\large 40}/{\footnotesize 40} & \cellcolor{green!60}{\large 21}/{\footnotesize 40} & \cellcolor{green!10}{\large 1}/{\footnotesize 40} & \cellcolor{green!20}{\large 5}/{\footnotesize 40} & \cellcolor{green!20}{\large 18}/{\footnotesize 128} & \cellcolor{green!10}{\large 9}/{\footnotesize 92} \tabularnewline
20 & pupnp\_ixmlLoadDocumentEx & 4 & \cellcolor{green!0}{\large 0}/{\footnotesize 40} & \cellcolor{green!0}{\large 0}/{\footnotesize 40} & \cellcolor{green!0}{\large 0}/{\footnotesize 40} & \cellcolor{green!0}{\large 0}/{\footnotesize 40} & \cellcolor{green!0}{\large 0}/{\footnotesize 200} & \cellcolor{green!0}{\large 0}/{\footnotesize 149} \tabularnewline
\rowcolor{black!10} 21 & gdk-pixbuf\_gdk\_pixbuf\_new\_from\_file\_at\_scale & 5 & \cellcolor{green!0}{\large 0}/{\footnotesize 40} & \cellcolor{green!0}{\large 0}/{\footnotesize 40} & \cellcolor{green!0}{\large 0}/{\footnotesize 40} & \cellcolor{green!0}{\large 0}/{\footnotesize 40} & \cellcolor{green!0}{\large 0}/{\footnotesize 200} & \cellcolor{green!10}{\large 1}/{\footnotesize 89} \tabularnewline
22 & inchi\_GetINCHIKeyFromINCHI & 5 & \cellcolor{green!0}{\large 0}/{\footnotesize 40} & \cellcolor{green!100}{\large 40}/{\footnotesize 40} & \cellcolor{green!100}{\large 40}/{\footnotesize 40} & \cellcolor{green!70}{\large 28}/{\footnotesize 40} & \cellcolor{green!100}{\large 40}/{\footnotesize 40} & \cellcolor{green!90}{\large 36}/{\footnotesize 44} \tabularnewline
\rowcolor{black!10} 23 & libdwarf\_dwarf\_init\_b & 5 & \cellcolor{green!0}{\large 0}/{\footnotesize 40} & \cellcolor{green!0}{\large 0}/{\footnotesize 40} & \cellcolor{green!0}{\large 0}/{\footnotesize 40} & \cellcolor{green!30}{\large 12}/{\footnotesize 40} & \cellcolor{green!0}{\large 0}/{\footnotesize 200} & \cellcolor{green!30}{\large 12}/{\footnotesize 56} \tabularnewline
24 & libdwarf\_dwarf\_init\_path & 5 & \cellcolor{green!0}{\large 0}/{\footnotesize 40} & \cellcolor{green!0}{\large 0}/{\footnotesize 40} & \cellcolor{green!0}{\large 0}/{\footnotesize 40} & \cellcolor{green!0}{\large 0}/{\footnotesize 40} & \cellcolor{green!0}{\large 0}/{\footnotesize 89} & \cellcolor{green!0}{\large 0}/{\footnotesize 63} \tabularnewline
\rowcolor{black!10} 25 & liblouis\_lou\_compileString & 5 & \cellcolor{green!0}{\large 0}/{\footnotesize 40} & \cellcolor{green!0}{\large 0}/{\footnotesize 40} & \cellcolor{green!0}{\large 0}/{\footnotesize 40} & \cellcolor{green!0}{\large 0}/{\footnotesize 40} & \cellcolor{green!0}{\large 0}/{\footnotesize 200} & \cellcolor{green!0}{\large 0}/{\footnotesize 99} \tabularnewline
26 & selinux\_cil\_compile & 5 & \cellcolor{green!0}{\large 0}/{\footnotesize 40} & \cellcolor{green!0}{\large 0}/{\footnotesize 40} & \cellcolor{green!0}{\large -}{\tiny -} & \cellcolor{green!0}{\large 0}/{\footnotesize 40} & \cellcolor{green!0}{\large 0}/{\footnotesize 200} & \cellcolor{green!0}{\large 0}/{\footnotesize 60} \tabularnewline
\rowcolor{black!10} 27 & bind9\_dns\_name\_fromtext & 6 & \cellcolor{green!0}{\large 0}/{\footnotesize 40} & \cellcolor{green!0}{\large 0}/{\footnotesize 40} & \cellcolor{green!0}{\large -}{\tiny -} & \cellcolor{green!20}{\large 6}/{\footnotesize 40} & \cellcolor{green!0}{\large 0}/{\footnotesize 200} & \cellcolor{green!10}{\large 3}/{\footnotesize 86} \tabularnewline
28 & bind9\_dns\_rdata\_fromwire & 6 & \cellcolor{green!0}{\large 0}/{\footnotesize 40} & \cellcolor{green!0}{\large 0}/{\footnotesize 40} & \cellcolor{green!0}{\large -}{\tiny -} & \cellcolor{green!0}{\large 0}/{\footnotesize 40} & \cellcolor{green!0}{\large 0}/{\footnotesize 200} & \cellcolor{green!0}{\large 0}/{\footnotesize 79} \tabularnewline
\rowcolor{black!10} 29 & coturn\_stun\_is\_binding\_response & 6 & \cellcolor{green!0}{\large 0}/{\footnotesize 40} & \cellcolor{green!0}{\large 0}/{\footnotesize 40} & \cellcolor{green!0}{\large -}{\tiny -} & \cellcolor{green!70}{\large 26}/{\footnotesize 40} & \cellcolor{green!0}{\large 0}/{\footnotesize 200} & \cellcolor{green!40}{\large 24}/{\footnotesize 70} \tabularnewline
30 & coturn\_stun\_is\_command\_message & 6 & \cellcolor{green!0}{\large 0}/{\footnotesize 40} & \cellcolor{green!0}{\large 0}/{\footnotesize 40} & \cellcolor{green!0}{\large 0}/{\footnotesize 40} & \cellcolor{green!100}{\large 40}/{\footnotesize 40} & \cellcolor{green!0}{\large 0}/{\footnotesize 200} & \cellcolor{green!40}{\large 29}/{\footnotesize 84} \tabularnewline
\rowcolor{black!10} 31 & coturn\_stun\_is\_response & 6 & \cellcolor{green!0}{\large 0}/{\footnotesize 40} & \cellcolor{green!0}{\large 0}/{\footnotesize 40} & \cellcolor{green!0}{\large -}{\tiny -} & \cellcolor{green!60}{\large 21}/{\footnotesize 40} & \cellcolor{green!0}{\large 0}/{\footnotesize 200} & \cellcolor{green!30}{\large 21}/{\footnotesize 72} \tabularnewline
32 & coturn\_stun\_is\_success\_response & 6 & \cellcolor{green!0}{\large 0}/{\footnotesize 40} & \cellcolor{green!0}{\large 0}/{\footnotesize 40} & \cellcolor{green!0}{\large -}{\tiny -} & \cellcolor{green!10}{\large 3}/{\footnotesize 40} & \cellcolor{green!0}{\large 0}/{\footnotesize 200} & \cellcolor{green!10}{\large 3}/{\footnotesize 100} \tabularnewline
\rowcolor{black!10} 33 & hiredis\_redisFormatCommand & 6 & \cellcolor{green!0}{\large 0}/{\footnotesize 40} & \cellcolor{green!100}{\large 40}/{\footnotesize 40} & \cellcolor{green!0}{\large -}{\tiny -} & \cellcolor{green!80}{\large 32}/{\footnotesize 40} & \cellcolor{green!100}{\large 40}/{\footnotesize 40} & \cellcolor{green!20}{\large 20}/{\footnotesize 145} \tabularnewline
34 & igraph\_igraph\_read\_graph\_dl & 6 & \cellcolor{green!0}{\large 0}/{\footnotesize 40} & \cellcolor{green!0}{\large 0}/{\footnotesize 40} & \cellcolor{green!0}{\large 0}/{\footnotesize 40} & \cellcolor{green!0}{\large 0}/{\footnotesize 40} & \cellcolor{green!0}{\large 0}/{\footnotesize 200} & \cellcolor{green!0}{\large 0}/{\footnotesize 105} \tabularnewline
\rowcolor{black!10} 35 & igraph\_igraph\_read\_graph\_edgelist & 6 & \cellcolor{green!0}{\large 0}/{\footnotesize 40} & \cellcolor{green!0}{\large 0}/{\footnotesize 40} & \cellcolor{green!0}{\large 0}/{\footnotesize 40} & \cellcolor{green!0}{\large 0}/{\footnotesize 40} & \cellcolor{green!0}{\large 0}/{\footnotesize 200} & \cellcolor{green!0}{\large 0}/{\footnotesize 98} \tabularnewline
36 & igraph\_igraph\_read\_graph\_gml & 6 & \cellcolor{green!0}{\large 0}/{\footnotesize 40} & \cellcolor{green!0}{\large 0}/{\footnotesize 40} & \cellcolor{green!0}{\large 0}/{\footnotesize 40} & \cellcolor{green!0}{\large 0}/{\footnotesize 40} & \cellcolor{green!0}{\large 0}/{\footnotesize 200} & \cellcolor{green!0}{\large 0}/{\footnotesize 94} \tabularnewline
\rowcolor{black!10} 37 & igraph\_igraph\_read\_graph\_graphdb & 6 & \cellcolor{green!0}{\large 0}/{\footnotesize 40} & \cellcolor{green!0}{\large 0}/{\footnotesize 40} & \cellcolor{green!0}{\large 0}/{\footnotesize 40} & \cellcolor{green!0}{\large 0}/{\footnotesize 40} & \cellcolor{green!0}{\large 0}/{\footnotesize 200} & \cellcolor{green!0}{\large 0}/{\footnotesize 137} \tabularnewline
38 & igraph\_igraph\_read\_graph\_graphml & 6 & \cellcolor{green!0}{\large 0}/{\footnotesize 40} & \cellcolor{green!0}{\large 0}/{\footnotesize 40} & \cellcolor{green!0}{\large 0}/{\footnotesize 40} & \cellcolor{green!10}{\large 4}/{\footnotesize 40} & \cellcolor{green!0}{\large 0}/{\footnotesize 200} & \cellcolor{green!10}{\large 2}/{\footnotesize 102} \tabularnewline
\rowcolor{black!10} 39 & igraph\_igraph\_read\_graph\_lgl & 6 & \cellcolor{green!0}{\large 0}/{\footnotesize 40} & \cellcolor{green!0}{\large 0}/{\footnotesize 40} & \cellcolor{green!0}{\large 0}/{\footnotesize 40} & \cellcolor{green!0}{\large 0}/{\footnotesize 40} & \cellcolor{green!0}{\large 0}/{\footnotesize 200} & \cellcolor{green!0}{\large 0}/{\footnotesize 95} \tabularnewline
40 & igraph\_igraph\_read\_graph\_pajek & 6 & \cellcolor{green!0}{\large 0}/{\footnotesize 40} & \cellcolor{green!0}{\large 0}/{\footnotesize 40} & \cellcolor{green!0}{\large 0}/{\footnotesize 40} & \cellcolor{green!0}{\large 0}/{\footnotesize 40} & \cellcolor{green!0}{\large 0}/{\footnotesize 200} & \cellcolor{green!0}{\large 0}/{\footnotesize 110} \tabularnewline
\rowcolor{black!10} 41 & inchi\_GetINCHIfromINCHI & 6 & \cellcolor{green!0}{\large 0}/{\footnotesize 40} & \cellcolor{green!0}{\large 0}/{\footnotesize 40} & \cellcolor{green!0}{\large 0}/{\footnotesize 40} & \cellcolor{green!0}{\large 0}/{\footnotesize 40} & \cellcolor{green!0}{\large 0}/{\footnotesize 200} & \cellcolor{green!0}{\large 0}/{\footnotesize 124} \tabularnewline
42 & inchi\_GetStructFromINCHI & 6 & \cellcolor{green!0}{\large 0}/{\footnotesize 40} & \cellcolor{green!0}{\large 0}/{\footnotesize 40} & \cellcolor{green!0}{\large 0}/{\footnotesize 40} & \cellcolor{green!0}{\large 0}/{\footnotesize 40} & \cellcolor{green!0}{\large 0}/{\footnotesize 200} & \cellcolor{green!0}{\large 0}/{\footnotesize 127} \tabularnewline
\rowcolor{black!10} 43 & kamailio\_parse\_msg & 6 & \cellcolor{green!0}{\large 0}/{\footnotesize 40} & \cellcolor{green!0}{\large 0}/{\footnotesize 40} & \cellcolor{green!0}{\large -}{\tiny -} & \cellcolor{green!30}{\large 10}/{\footnotesize 40} & \cellcolor{green!0}{\large 0}/{\footnotesize 200} & \cellcolor{green!10}{\large 10}/{\footnotesize 96} \tabularnewline
44 & libyang\_lys\_parse\_mem & 6 & \cellcolor{green!0}{\large 0}/{\footnotesize 40} & \cellcolor{green!0}{\large 0}/{\footnotesize 40} & \cellcolor{green!0}{\large 0}/{\footnotesize 40} & \cellcolor{green!0}{\large 0}/{\footnotesize 40} & \cellcolor{green!0}{\large 0}/{\footnotesize 200} & \cellcolor{green!0}{\large 0}/{\footnotesize 147} \tabularnewline
\rowcolor{black!10} 45 & proftpd\_pr\_json\_object\_from\_text & 6 & \cellcolor{green!0}{\large 0}/{\footnotesize 40} & \cellcolor{green!0}{\large 0}/{\footnotesize 40} & \cellcolor{green!0}{\large -}{\tiny -} & \cellcolor{green!10}{\large 1}/{\footnotesize 40} & \cellcolor{green!0}{\large 0}/{\footnotesize 200} & \cellcolor{green!0}{\large 1}/{\footnotesize 110} \tabularnewline
46 & selinux\_policydb\_read & 6 & \cellcolor{green!0}{\large 0}/{\footnotesize 40} & \cellcolor{green!0}{\large 0}/{\footnotesize 40} & \cellcolor{green!0}{\large -}{\tiny -} & \cellcolor{green!10}{\large 1}/{\footnotesize 40} & \cellcolor{green!0}{\large 0}/{\footnotesize 40} & \cellcolor{green!10}{\large 2}/{\footnotesize 77} \tabularnewline
\rowcolor{black!10} 47 & kamailio\_get\_src\_address\_socket & 7 & \cellcolor{green!0}{\large 0}/{\footnotesize 40} & \cellcolor{green!0}{\large 0}/{\footnotesize 40} & \cellcolor{green!0}{\large 0}/{\footnotesize 40} & \cellcolor{green!30}{\large 11}/{\footnotesize 40} & \cellcolor{green!0}{\large 0}/{\footnotesize 200} & \cellcolor{green!20}{\large 12}/{\footnotesize 77} \tabularnewline
48 & kamailio\_get\_src\_uri & 7 & \cellcolor{green!0}{\large 0}/{\footnotesize 40} & \cellcolor{green!0}{\large 0}/{\footnotesize 40} & \cellcolor{green!0}{\large 0}/{\footnotesize 40} & \cellcolor{green!40}{\large 16}/{\footnotesize 40} & \cellcolor{green!0}{\large 0}/{\footnotesize 200} & \cellcolor{green!10}{\large 6}/{\footnotesize 95} \tabularnewline
\rowcolor{black!10} 49 & kamailio\_parse\_content\_disposition & 7 & \cellcolor{green!0}{\large 0}/{\footnotesize 40} & \cellcolor{green!0}{\large 0}/{\footnotesize 40} & \cellcolor{green!0}{\large 0}/{\footnotesize 40} & \cellcolor{green!30}{\large 11}/{\footnotesize 40} & \cellcolor{green!0}{\large 0}/{\footnotesize 200} & \cellcolor{green!10}{\large 6}/{\footnotesize 81} \tabularnewline
50 & kamailio\_parse\_diversion\_header & 7 & \cellcolor{green!0}{\large 0}/{\footnotesize 40} & \cellcolor{green!0}{\large 0}/{\footnotesize 40} & \cellcolor{green!0}{\large 0}/{\footnotesize 40} & \cellcolor{green!30}{\large 9}/{\footnotesize 40} & \cellcolor{green!0}{\large 0}/{\footnotesize 200} & \cellcolor{green!20}{\large 9}/{\footnotesize 79} \tabularnewline
\rowcolor{black!10} 51 & kamailio\_parse\_from\_header & 7 & \cellcolor{green!0}{\large 0}/{\footnotesize 40} & \cellcolor{green!0}{\large 0}/{\footnotesize 40} & \cellcolor{green!0}{\large -}{\tiny -} & \cellcolor{green!0}{\large 0}/{\footnotesize 40} & \cellcolor{green!0}{\large 0}/{\footnotesize 200} & \cellcolor{green!0}{\large 0}/{\footnotesize 96} \tabularnewline
52 & kamailio\_parse\_from\_uri & 7 & \cellcolor{green!0}{\large 0}/{\footnotesize 40} & \cellcolor{green!0}{\large 0}/{\footnotesize 40} & \cellcolor{green!0}{\large -}{\tiny -} & \cellcolor{green!10}{\large 1}/{\footnotesize 40} & \cellcolor{green!0}{\large 0}/{\footnotesize 200} & \cellcolor{green!10}{\large 1}/{\footnotesize 96} \tabularnewline
\rowcolor{black!10} 53 & kamailio\_parse\_headers & 7 & \cellcolor{green!0}{\large 0}/{\footnotesize 40} & \cellcolor{green!0}{\large 0}/{\footnotesize 40} & \cellcolor{green!0}{\large -}{\tiny -} & \cellcolor{green!0}{\large 0}/{\footnotesize 40} & \cellcolor{green!0}{\large 0}/{\footnotesize 200} & \cellcolor{green!0}{\large 0}/{\footnotesize 79} \tabularnewline
54 & kamailio\_parse\_identityinfo\_header & 7 & \cellcolor{green!0}{\large 0}/{\footnotesize 40} & \cellcolor{green!0}{\large 0}/{\footnotesize 40} & \cellcolor{green!0}{\large -}{\tiny -} & \cellcolor{green!60}{\large 22}/{\footnotesize 40} & \cellcolor{green!0}{\large 0}/{\footnotesize 200} & \cellcolor{green!30}{\large 19}/{\footnotesize 73} \tabularnewline
\rowcolor{black!10} 55 & kamailio\_parse\_pai\_header & 7 & \cellcolor{green!0}{\large 0}/{\footnotesize 40} & \cellcolor{green!0}{\large 0}/{\footnotesize 40} & \cellcolor{green!0}{\large -}{\tiny -} & \cellcolor{green!20}{\large 6}/{\footnotesize 40} & \cellcolor{green!0}{\large 0}/{\footnotesize 200} & \cellcolor{green!10}{\large 4}/{\footnotesize 99} \tabularnewline
56 & kamailio\_parse\_privacy & 7 & \cellcolor{green!0}{\large 0}/{\footnotesize 40} & \cellcolor{green!0}{\large 0}/{\footnotesize 40} & \cellcolor{green!0}{\large 0}/{\footnotesize 40} & \cellcolor{green!30}{\large 11}/{\footnotesize 40} & \cellcolor{green!0}{\large 0}/{\footnotesize 200} & \cellcolor{green!10}{\large 11}/{\footnotesize 115} \tabularnewline
\rowcolor{black!10} 57 & kamailio\_parse\_record\_route\_headers & 7 & \cellcolor{green!0}{\large 0}/{\footnotesize 40} & \cellcolor{green!0}{\large 0}/{\footnotesize 40} & \cellcolor{green!0}{\large -}{\tiny -} & \cellcolor{green!100}{\large 40}/{\footnotesize 40} & \cellcolor{green!0}{\large 0}/{\footnotesize 200} & \cellcolor{green!20}{\large 19}/{\footnotesize 124} \tabularnewline
58 & kamailio\_parse\_refer\_to\_header & 7 & \cellcolor{green!0}{\large 0}/{\footnotesize 40} & \cellcolor{green!0}{\large 0}/{\footnotesize 40} & \cellcolor{green!0}{\large -}{\tiny -} & \cellcolor{green!30}{\large 11}/{\footnotesize 40} & \cellcolor{green!0}{\large 0}/{\footnotesize 200} & \cellcolor{green!20}{\large 8}/{\footnotesize 57} \tabularnewline
\rowcolor{black!10} 59 & kamailio\_parse\_route\_headers & 7 & \cellcolor{green!0}{\large 0}/{\footnotesize 40} & \cellcolor{green!0}{\large 0}/{\footnotesize 40} & \cellcolor{green!0}{\large -}{\tiny -} & \cellcolor{green!100}{\large 40}/{\footnotesize 40} & \cellcolor{green!0}{\large 0}/{\footnotesize 200} & \cellcolor{green!40}{\large 28}/{\footnotesize 88} \tabularnewline
60 & kamailio\_parse\_to\_header & 7 & \cellcolor{green!0}{\large 0}/{\footnotesize 40} & \cellcolor{green!0}{\large 0}/{\footnotesize 40} & \cellcolor{green!0}{\large -}{\tiny -} & \cellcolor{green!10}{\large 1}/{\footnotesize 40} & \cellcolor{green!0}{\large 0}/{\footnotesize 200} & \cellcolor{green!10}{\large 2}/{\footnotesize 89} \tabularnewline
\rowcolor{black!10} 61 & kamailio\_parse\_to\_uri & 7 & \cellcolor{green!0}{\large 0}/{\footnotesize 40} & \cellcolor{green!0}{\large 0}/{\footnotesize 40} & \cellcolor{green!0}{\large -}{\tiny -} & \cellcolor{green!10}{\large 2}/{\footnotesize 40} & \cellcolor{green!0}{\large 0}/{\footnotesize 200} & \cellcolor{green!10}{\large 2}/{\footnotesize 82} \tabularnewline
62 & libyang\_lyd\_parse\_data\_mem & 7 & \cellcolor{green!0}{\large 0}/{\footnotesize 40} & \cellcolor{green!0}{\large 0}/{\footnotesize 40} & \cellcolor{green!0}{\large 0}/{\footnotesize 40} & \cellcolor{green!0}{\large 0}/{\footnotesize 40} & \cellcolor{green!0}{\large 0}/{\footnotesize 200} & \cellcolor{green!0}{\large 0}/{\footnotesize 129} \tabularnewline
\rowcolor{black!10} 63 & bind9\_dns\_message\_parse & 8 & \cellcolor{green!0}{\large 0}/{\footnotesize 40} & \cellcolor{green!0}{\large 0}/{\footnotesize 40} & \cellcolor{green!0}{\large -}{\tiny -} & \cellcolor{green!0}{\large 0}/{\footnotesize 40} & \cellcolor{green!0}{\large 0}/{\footnotesize 200} & \cellcolor{green!0}{\large 0}/{\footnotesize 136} \tabularnewline
64 & igraph\_igraph\_read\_graph\_ncol & 8 & \cellcolor{green!0}{\large 0}/{\footnotesize 40} & \cellcolor{green!0}{\large 0}/{\footnotesize 40} & \cellcolor{green!0}{\large 0}/{\footnotesize 40} & \cellcolor{green!0}{\large 0}/{\footnotesize 40} & \cellcolor{green!0}{\large 0}/{\footnotesize 200} & \cellcolor{green!0}{\large 0}/{\footnotesize 125} \tabularnewline
\rowcolor{black!10} 65 & pjsip\_pj\_json\_parse & 8 & \cellcolor{green!0}{\large 0}/{\footnotesize 40} & \cellcolor{green!0}{\large 0}/{\footnotesize 40} & \cellcolor{green!0}{\large 0}/{\footnotesize 40} & \cellcolor{green!0}{\large 0}/{\footnotesize 40} & \cellcolor{green!0}{\large 0}/{\footnotesize 200} & \cellcolor{green!0}{\large 0}/{\footnotesize 137} \tabularnewline
66 & pjsip\_pj\_xml\_parse & 8 & \cellcolor{green!0}{\large 0}/{\footnotesize 40} & \cellcolor{green!0}{\large 0}/{\footnotesize 40} & \cellcolor{green!0}{\large 0}/{\footnotesize 40} & \cellcolor{green!0}{\large 0}/{\footnotesize 40} & \cellcolor{green!0}{\large 0}/{\footnotesize 200} & \cellcolor{green!0}{\large 0}/{\footnotesize 114} \tabularnewline
\rowcolor{black!10} 67 & pjsip\_pjmedia\_sdp\_parse & 8 & \cellcolor{green!0}{\large 0}/{\footnotesize 40} & \cellcolor{green!0}{\large 0}/{\footnotesize 40} & \cellcolor{green!0}{\large 0}/{\footnotesize 40} & \cellcolor{green!0}{\large 0}/{\footnotesize 40} & \cellcolor{green!0}{\large 0}/{\footnotesize 200} & \cellcolor{green!0}{\large 0}/{\footnotesize 112} \tabularnewline
68 & quickjs\_lre\_compile & 8 & \cellcolor{green!0}{\large 0}/{\footnotesize 40} & \cellcolor{green!0}{\large 0}/{\footnotesize 40} & \cellcolor{green!0}{\large -}{\tiny -} & \cellcolor{green!0}{\large 0}/{\footnotesize 40} & \cellcolor{green!0}{\large 0}/{\footnotesize 200} & \cellcolor{green!0}{\large 0}/{\footnotesize 94} \tabularnewline
\rowcolor{black!10} 69 & bind9\_isc\_lex\_getmastertoken & 9 & \cellcolor{green!0}{\large 0}/{\footnotesize 40} & \cellcolor{green!0}{\large 0}/{\footnotesize 40} & \cellcolor{green!0}{\large -}{\tiny -} & \cellcolor{green!0}{\large 0}/{\footnotesize 40} & \cellcolor{green!0}{\large 0}/{\footnotesize 200} & \cellcolor{green!0}{\large 0}/{\footnotesize 100} \tabularnewline
70 & bind9\_isc\_lex\_gettoken & 9 & \cellcolor{green!0}{\large 0}/{\footnotesize 40} & \cellcolor{green!0}{\large 0}/{\footnotesize 40} & \cellcolor{green!0}{\large -}{\tiny -} & \cellcolor{green!0}{\large 0}/{\footnotesize 40} & \cellcolor{green!0}{\large 0}/{\footnotesize 200} & \cellcolor{green!0}{\large 0}/{\footnotesize 93} \tabularnewline
\rowcolor{black!10} 71 & quickjs\_JS\_Eval & 9 & \cellcolor{green!0}{\large 0}/{\footnotesize 40} & \cellcolor{green!0}{\large 0}/{\footnotesize 40} & \cellcolor{green!0}{\large -}{\tiny -} & \cellcolor{green!10}{\large 1}/{\footnotesize 40} & \cellcolor{green!0}{\large 0}/{\footnotesize 200} & \cellcolor{green!10}{\large 2}/{\footnotesize 100} \tabularnewline
72 & igraph\_igraph\_edge\_connectivity & 10 & \cellcolor{green!0}{\large 0}/{\footnotesize 40} & \cellcolor{green!0}{\large 0}/{\footnotesize 40} & \cellcolor{green!0}{\large 0}/{\footnotesize 40} & \cellcolor{green!0}{\large 0}/{\footnotesize 40} & \cellcolor{green!0}{\large 0}/{\footnotesize 186} & \cellcolor{green!0}{\large 0}/{\footnotesize 102} \tabularnewline
\rowcolor{black!10} 73 & pjsip\_pj\_stun\_msg\_decode & 10 & \cellcolor{green!0}{\large 0}/{\footnotesize 40} & \cellcolor{green!0}{\large 0}/{\footnotesize 40} & \cellcolor{green!0}{\large 0}/{\footnotesize 40} & \cellcolor{green!0}{\large 0}/{\footnotesize 40} & \cellcolor{green!0}{\large 0}/{\footnotesize 200} & \cellcolor{green!0}{\large 0}/{\footnotesize 94} \tabularnewline
74 & bind9\_dns\_message\_checksig & 11 & \cellcolor{green!0}{\large 0}/{\footnotesize 40} & \cellcolor{green!0}{\large 0}/{\footnotesize 40} & \cellcolor{green!0}{\large -}{\tiny -} & \cellcolor{green!0}{\large 0}/{\footnotesize 40} & \cellcolor{green!0}{\large 0}/{\footnotesize 40} & \cellcolor{green!0}{\large 0}/{\footnotesize 77} \tabularnewline
\rowcolor{black!10} 75 & libzip\_zip\_fread & 11 & \cellcolor{green!0}{\large 0}/{\footnotesize 40} & \cellcolor{green!0}{\large 0}/{\footnotesize 40} & \cellcolor{green!0}{\large 0}/{\footnotesize 40} & \cellcolor{green!10}{\large 4}/{\footnotesize 40} & \cellcolor{green!0}{\large 0}/{\footnotesize 200} & \cellcolor{green!10}{\large 4}/{\footnotesize 83} \tabularnewline
76 & bind9\_dns\_rdata\_fromtext & 12 & \cellcolor{green!0}{\large 0}/{\footnotesize 40} & \cellcolor{green!0}{\large 0}/{\footnotesize 40} & \cellcolor{green!0}{\large -}{\tiny -} & \cellcolor{green!0}{\large 0}/{\footnotesize 40} & \cellcolor{green!0}{\large 0}/{\footnotesize 200} & \cellcolor{green!0}{\large 0}/{\footnotesize 68} \tabularnewline
\rowcolor{black!10} 77 & igraph\_igraph\_all\_minimal\_st\_separators & 12 & \cellcolor{green!0}{\large 0}/{\footnotesize 40} & \cellcolor{green!0}{\large 0}/{\footnotesize 40} & \cellcolor{green!0}{\large 0}/{\footnotesize 40} & \cellcolor{green!0}{\large 0}/{\footnotesize 40} & \cellcolor{green!0}{\large 0}/{\footnotesize 40} & \cellcolor{green!0}{\large 0}/{\footnotesize 75} \tabularnewline
78 & igraph\_igraph\_minimum\_size\_separators & 12 & \cellcolor{green!0}{\large 0}/{\footnotesize 40} & \cellcolor{green!0}{\large 0}/{\footnotesize 40} & \cellcolor{green!0}{\large 0}/{\footnotesize 40} & \cellcolor{green!0}{\large 0}/{\footnotesize 40} & \cellcolor{green!0}{\large 0}/{\footnotesize 40} & \cellcolor{green!0}{\large 0}/{\footnotesize 40} \tabularnewline
\rowcolor{black!10} 79 & pjsip\_pjsip\_parse\_msg & 12 & \cellcolor{green!0}{\large 0}/{\footnotesize 40} & \cellcolor{green!0}{\large 0}/{\footnotesize 40} & \cellcolor{green!0}{\large 0}/{\footnotesize 40} & \cellcolor{green!0}{\large 0}/{\footnotesize 40} & \cellcolor{green!0}{\large 0}/{\footnotesize 200} & \cellcolor{green!0}{\large 0}/{\footnotesize 77} \tabularnewline
80 & igraph\_igraph\_automorphism\_group & 13 & \cellcolor{green!0}{\large 0}/{\footnotesize 40} & \cellcolor{green!0}{\large 0}/{\footnotesize 40} & \cellcolor{green!0}{\large 0}/{\footnotesize 40} & \cellcolor{green!0}{\large 0}/{\footnotesize 40} & \cellcolor{green!0}{\large 0}/{\footnotesize 200} & \cellcolor{green!0}{\large 0}/{\footnotesize 88} \tabularnewline
\rowcolor{black!10} 81 & libmodbus\_modbus\_read\_bits & 15 & \cellcolor{green!0}{\large 0}/{\footnotesize 40} & \cellcolor{green!0}{\large 0}/{\footnotesize 40} & \cellcolor{green!0}{\large 0}/{\footnotesize 40} & \cellcolor{green!0}{\large 0}/{\footnotesize 40} & \cellcolor{green!0}{\large 0}/{\footnotesize 142} & \cellcolor{green!0}{\large 0}/{\footnotesize 75} \tabularnewline
82 & libmodbus\_modbus\_read\_registers & 15 & \cellcolor{green!0}{\large 0}/{\footnotesize 40} & \cellcolor{green!0}{\large 0}/{\footnotesize 40} & \cellcolor{green!0}{\large 0}/{\footnotesize 40} & \cellcolor{green!0}{\large 0}/{\footnotesize 40} & \cellcolor{green!0}{\large 0}/{\footnotesize 80} & \cellcolor{green!0}{\large 0}/{\footnotesize 106} \tabularnewline
\rowcolor{black!10} 83 & civetweb\_mg\_get\_response & 17 & \cellcolor{green!0}{\large 0}/{\footnotesize 40} & \cellcolor{green!0}{\large 0}/{\footnotesize 40} & \cellcolor{green!0}{\large 0}/{\footnotesize 40} & \cellcolor{green!0}{\large 0}/{\footnotesize 40} & \cellcolor{green!0}{\large 0}/{\footnotesize 128} & \cellcolor{green!0}{\large 0}/{\footnotesize 71} \tabularnewline
84 & bind9\_dns\_master\_loadbuffer & 20 & \cellcolor{green!0}{\large 0}/{\footnotesize 40} & \cellcolor{green!0}{\large 0}/{\footnotesize 40} & \cellcolor{green!0}{\large -}{\tiny -} & \cellcolor{green!0}{\large 0}/{\footnotesize 40} & \cellcolor{green!0}{\large 0}/{\footnotesize 200} & \cellcolor{green!0}{\large 0}/{\footnotesize 125} \tabularnewline
\rowcolor{black!10} 85 & libmodbus\_modbus\_receive & 33 & \cellcolor{green!0}{\large 0}/{\footnotesize 40} & \cellcolor{green!0}{\large 0}/{\footnotesize 40} & \cellcolor{green!0}{\large 0}/{\footnotesize 40} & \cellcolor{green!0}{\large 0}/{\footnotesize 40} & \cellcolor{green!0}{\large 0}/{\footnotesize 80} & \cellcolor{green!0}{\large 0}/{\footnotesize 69} \tabularnewline
86 & tmux\_input\_parse\_buffer & 42 & \cellcolor{green!0}{\large 0}/{\footnotesize 40} & \cellcolor{green!0}{\large 0}/{\footnotesize 40} & \cellcolor{green!0}{\large -}{\tiny -} & \cellcolor{green!0}{\large 0}/{\footnotesize 40} & \cellcolor{green!0}{\large 0}/{\footnotesize 200} & \cellcolor{green!0}{\large 0}/{\footnotesize 158} \tabularnewline

\bottomrule
%\end{tabular}
%}
%\end{table*}
\end{xltabular}
}
\twocolumn



% model: wizardcoder-15b-v1.0, temp: 0.5

\onecolumn
{\small %
\begin{xltabular}[h]{\textwidth}{ccccccccc}
%\begin{table*}[!t]
%\centering
\caption{Evaluation Result of model wizardcoder-15b-v1.0 with temperature 0.5.} \\
%\resizebox{1.0\linewidth}{!}{
%\begin{tabular}{cccccccccc}
\toprule
Index & Question & Score & NAIVE-40 & BACTX-40 & DOCTX-40 & UGCTX-40 & BA-ITER-40 & ALL-ITER-40 \tabularnewline
\midrule
\rowcolor{black!10} 1 & coturn\_stun\_is\_command\_message\_full\_check\_str & 1 & \cellcolor{green!0}{\large 0}/{\footnotesize 40} & \cellcolor{green!80}{\large 31}/{\footnotesize 40} & \cellcolor{green!0}{\large -}{\tiny -} & \cellcolor{green!50}{\large 18}/{\footnotesize 40} & \cellcolor{green!80}{\large 35}/{\footnotesize 47} & \cellcolor{green!50}{\large 27}/{\footnotesize 59} \tabularnewline
2 & kamailio\_parse\_uri & 1 & \cellcolor{green!0}{\large 0}/{\footnotesize 40} & \cellcolor{green!100}{\large 37}/{\footnotesize 40} & \cellcolor{green!0}{\large -}{\tiny -} & \cellcolor{green!80}{\large 32}/{\footnotesize 40} & \cellcolor{green!100}{\large 40}/{\footnotesize 40} & \cellcolor{green!50}{\large 31}/{\footnotesize 61} \tabularnewline
\rowcolor{black!10} 3 & coturn\_stun\_check\_message\_integrity\_str & 2 & \cellcolor{green!0}{\large 0}/{\footnotesize 40} & \cellcolor{green!10}{\large 3}/{\footnotesize 40} & \cellcolor{green!0}{\large -}{\tiny -} & \cellcolor{green!10}{\large 2}/{\footnotesize 40} & \cellcolor{green!10}{\large 9}/{\footnotesize 152} & \cellcolor{green!10}{\large 3}/{\footnotesize 113} \tabularnewline
4 & libiec61850\_MmsValue\_decodeMmsData & 2 & \cellcolor{green!0}{\large 0}/{\footnotesize 40} & \cellcolor{green!50}{\large 19}/{\footnotesize 40} & \cellcolor{green!70}{\large 26}/{\footnotesize 40} & \cellcolor{green!40}{\large 14}/{\footnotesize 40} & \cellcolor{green!30}{\large 23}/{\footnotesize 93} & \cellcolor{green!30}{\large 19}/{\footnotesize 84} \tabularnewline
\rowcolor{black!10} 5 & md4c\_md\_html & 2 & \cellcolor{green!0}{\large 0}/{\footnotesize 40} & \cellcolor{green!0}{\large 0}/{\footnotesize 40} & \cellcolor{green!0}{\large 0}/{\footnotesize 40} & \cellcolor{green!0}{\large 0}/{\footnotesize 40} & \cellcolor{green!0}{\large 0}/{\footnotesize 195} & \cellcolor{green!0}{\large 0}/{\footnotesize 105} \tabularnewline
6 & spdk\_spdk\_json\_parse & 2 & \cellcolor{green!0}{\large 0}/{\footnotesize 40} & \cellcolor{green!40}{\large 14}/{\footnotesize 40} & \cellcolor{green!0}{\large -}{\tiny -} & \cellcolor{green!30}{\large 12}/{\footnotesize 40} & \cellcolor{green!40}{\large 34}/{\footnotesize 89} & \cellcolor{green!10}{\large 6}/{\footnotesize 91} \tabularnewline
\rowcolor{black!10} 7 & croaring\_roaring\_bitmap\_portable\_deserialize\_safe & 3 & \cellcolor{green!0}{\large 0}/{\footnotesize 40} & \cellcolor{green!50}{\large 19}/{\footnotesize 40} & \cellcolor{green!40}{\large 16}/{\footnotesize 40} & \cellcolor{green!20}{\large 8}/{\footnotesize 40} & \cellcolor{green!30}{\large 28}/{\footnotesize 100} & \cellcolor{green!10}{\large 10}/{\footnotesize 103} \tabularnewline
8 & lua\_luaL\_loadbufferx & 3 & \cellcolor{green!70}{\large 25}/{\footnotesize 40} & \cellcolor{green!100}{\large 40}/{\footnotesize 40} & \cellcolor{green!100}{\large 38}/{\footnotesize 40} & \cellcolor{green!40}{\large 16}/{\footnotesize 40} & \cellcolor{green!90}{\large 39}/{\footnotesize 44} & \cellcolor{green!40}{\large 25}/{\footnotesize 78} \tabularnewline
\rowcolor{black!10} 9 & w3m\_wc\_Str\_conv\_with\_detect & 3 & \cellcolor{green!0}{\large 0}/{\footnotesize 40} & \cellcolor{green!0}{\large 0}/{\footnotesize 40} & \cellcolor{green!0}{\large -}{\tiny -} & \cellcolor{green!10}{\large 1}/{\footnotesize 40} & \cellcolor{green!0}{\large 0}/{\footnotesize 188} & \cellcolor{green!0}{\large 0}/{\footnotesize 135} \tabularnewline
10 & bind9\_dns\_name\_fromwire & 4 & \cellcolor{green!0}{\large 0}/{\footnotesize 40} & \cellcolor{green!0}{\large 0}/{\footnotesize 40} & \cellcolor{green!0}{\large -}{\tiny -} & \cellcolor{green!0}{\large 0}/{\footnotesize 40} & \cellcolor{green!0}{\large 0}/{\footnotesize 195} & \cellcolor{green!0}{\large 1}/{\footnotesize 123} \tabularnewline
\rowcolor{black!10} 11 & gdk-pixbuf\_gdk\_pixbuf\_animation\_new\_from\_file & 4 & \cellcolor{green!10}{\large 3}/{\footnotesize 40} & \cellcolor{green!0}{\large 0}/{\footnotesize 40} & \cellcolor{green!0}{\large 0}/{\footnotesize 40} & \cellcolor{green!0}{\large 0}/{\footnotesize 40} & \cellcolor{green!0}{\large 0}/{\footnotesize 117} & \cellcolor{green!10}{\large 1}/{\footnotesize 100} \tabularnewline
12 & gdk-pixbuf\_gdk\_pixbuf\_new\_from\_data & 4 & \cellcolor{green!10}{\large 1}/{\footnotesize 40} & \cellcolor{green!80}{\large 30}/{\footnotesize 40} & \cellcolor{green!80}{\large 29}/{\footnotesize 40} & \cellcolor{green!50}{\large 19}/{\footnotesize 40} & \cellcolor{green!50}{\large 28}/{\footnotesize 66} & \cellcolor{green!30}{\large 17}/{\footnotesize 65} \tabularnewline
\rowcolor{black!10} 13 & gdk-pixbuf\_gdk\_pixbuf\_new\_from\_file & 4 & \cellcolor{green!20}{\large 5}/{\footnotesize 40} & \cellcolor{green!10}{\large 1}/{\footnotesize 40} & \cellcolor{green!0}{\large 0}/{\footnotesize 40} & \cellcolor{green!0}{\large 0}/{\footnotesize 40} & \cellcolor{green!0}{\large 0}/{\footnotesize 120} & \cellcolor{green!10}{\large 3}/{\footnotesize 92} \tabularnewline
14 & gdk-pixbuf\_gdk\_pixbuf\_new\_from\_stream & 4 & \cellcolor{green!10}{\large 4}/{\footnotesize 40} & \cellcolor{green!80}{\large 32}/{\footnotesize 40} & \cellcolor{green!40}{\large 15}/{\footnotesize 40} & \cellcolor{green!70}{\large 27}/{\footnotesize 40} & \cellcolor{green!30}{\large 26}/{\footnotesize 100} & \cellcolor{green!40}{\large 28}/{\footnotesize 75} \tabularnewline
\rowcolor{black!10} 15 & gpac\_gf\_isom\_open\_file & 4 & \cellcolor{green!0}{\large 0}/{\footnotesize 40} & \cellcolor{green!20}{\large 5}/{\footnotesize 40} & \cellcolor{green!0}{\large -}{\tiny -} & \cellcolor{green!10}{\large 1}/{\footnotesize 40} & \cellcolor{green!10}{\large 4}/{\footnotesize 173} & \cellcolor{green!10}{\large 1}/{\footnotesize 96} \tabularnewline
16 & libbpf\_bpf\_object\_\_open\_mem & 4 & \cellcolor{green!0}{\large 0}/{\footnotesize 40} & \cellcolor{green!20}{\large 5}/{\footnotesize 40} & \cellcolor{green!20}{\large 7}/{\footnotesize 40} & \cellcolor{green!30}{\large 9}/{\footnotesize 40} & \cellcolor{green!10}{\large 3}/{\footnotesize 185} & \cellcolor{green!20}{\large 10}/{\footnotesize 82} \tabularnewline
\rowcolor{black!10} 17 & libpg\_query\_pg\_query\_parse & 4 & \cellcolor{green!0}{\large 0}/{\footnotesize 40} & \cellcolor{green!30}{\large 9}/{\footnotesize 40} & \cellcolor{green!0}{\large -}{\tiny -} & \cellcolor{green!60}{\large 23}/{\footnotesize 40} & \cellcolor{green!10}{\large 8}/{\footnotesize 166} & \cellcolor{green!30}{\large 23}/{\footnotesize 95} \tabularnewline
18 & libucl\_ucl\_parser\_add\_string & 4 & \cellcolor{green!10}{\large 1}/{\footnotesize 40} & \cellcolor{green!30}{\large 9}/{\footnotesize 40} & \cellcolor{green!30}{\large 10}/{\footnotesize 40} & \cellcolor{green!20}{\large 7}/{\footnotesize 40} & \cellcolor{green!20}{\large 23}/{\footnotesize 165} & \cellcolor{green!10}{\large 2}/{\footnotesize 113} \tabularnewline
\rowcolor{black!10} 19 & oniguruma\_onig\_new & 4 & \cellcolor{green!20}{\large 7}/{\footnotesize 40} & \cellcolor{green!40}{\large 16}/{\footnotesize 40} & \cellcolor{green!30}{\large 12}/{\footnotesize 40} & \cellcolor{green!30}{\large 12}/{\footnotesize 40} & \cellcolor{green!20}{\large 16}/{\footnotesize 138} & \cellcolor{green!30}{\large 18}/{\footnotesize 83} \tabularnewline
20 & pupnp\_ixmlLoadDocumentEx & 4 & \cellcolor{green!0}{\large 0}/{\footnotesize 40} & \cellcolor{green!10}{\large 1}/{\footnotesize 40} & \cellcolor{green!0}{\large 0}/{\footnotesize 40} & \cellcolor{green!0}{\large 0}/{\footnotesize 40} & \cellcolor{green!0}{\large 0}/{\footnotesize 133} & \cellcolor{green!0}{\large 0}/{\footnotesize 115} \tabularnewline
\rowcolor{black!10} 21 & gdk-pixbuf\_gdk\_pixbuf\_new\_from\_file\_at\_scale & 5 & \cellcolor{green!10}{\large 2}/{\footnotesize 40} & \cellcolor{green!0}{\large 0}/{\footnotesize 40} & \cellcolor{green!0}{\large 0}/{\footnotesize 40} & \cellcolor{green!10}{\large 1}/{\footnotesize 40} & \cellcolor{green!10}{\large 1}/{\footnotesize 87} & \cellcolor{green!0}{\large 0}/{\footnotesize 92} \tabularnewline
22 & inchi\_GetINCHIKeyFromINCHI & 5 & \cellcolor{green!0}{\large 0}/{\footnotesize 40} & \cellcolor{green!70}{\large 27}/{\footnotesize 40} & \cellcolor{green!40}{\large 16}/{\footnotesize 40} & \cellcolor{green!50}{\large 18}/{\footnotesize 40} & \cellcolor{green!30}{\large 26}/{\footnotesize 95} & \cellcolor{green!30}{\large 20}/{\footnotesize 69} \tabularnewline
\rowcolor{black!10} 23 & libdwarf\_dwarf\_init\_b & 5 & \cellcolor{green!0}{\large 0}/{\footnotesize 40} & \cellcolor{green!0}{\large 0}/{\footnotesize 40} & \cellcolor{green!0}{\large 0}/{\footnotesize 40} & \cellcolor{green!20}{\large 7}/{\footnotesize 40} & \cellcolor{green!0}{\large 0}/{\footnotesize 174} & \cellcolor{green!10}{\large 5}/{\footnotesize 82} \tabularnewline
24 & libdwarf\_dwarf\_init\_path & 5 & \cellcolor{green!0}{\large 0}/{\footnotesize 40} & \cellcolor{green!0}{\large 0}/{\footnotesize 40} & \cellcolor{green!0}{\large 0}/{\footnotesize 40} & \cellcolor{green!0}{\large 0}/{\footnotesize 40} & \cellcolor{green!0}{\large 0}/{\footnotesize 157} & \cellcolor{green!0}{\large 0}/{\footnotesize 95} \tabularnewline
\rowcolor{black!10} 25 & liblouis\_lou\_compileString & 5 & \cellcolor{green!0}{\large 0}/{\footnotesize 40} & \cellcolor{green!10}{\large 2}/{\footnotesize 40} & \cellcolor{green!10}{\large 4}/{\footnotesize 40} & \cellcolor{green!10}{\large 2}/{\footnotesize 40} & \cellcolor{green!10}{\large 3}/{\footnotesize 143} & \cellcolor{green!10}{\large 3}/{\footnotesize 106} \tabularnewline
26 & selinux\_cil\_compile & 5 & \cellcolor{green!0}{\large 0}/{\footnotesize 40} & \cellcolor{green!0}{\large 0}/{\footnotesize 40} & \cellcolor{green!0}{\large -}{\tiny -} & \cellcolor{green!10}{\large 3}/{\footnotesize 40} & \cellcolor{green!0}{\large 0}/{\footnotesize 189} & \cellcolor{green!10}{\large 4}/{\footnotesize 67} \tabularnewline
\rowcolor{black!10} 27 & bind9\_dns\_name\_fromtext & 6 & \cellcolor{green!0}{\large 0}/{\footnotesize 40} & \cellcolor{green!10}{\large 1}/{\footnotesize 40} & \cellcolor{green!0}{\large -}{\tiny -} & \cellcolor{green!10}{\large 4}/{\footnotesize 40} & \cellcolor{green!0}{\large 1}/{\footnotesize 179} & \cellcolor{green!10}{\large 5}/{\footnotesize 106} \tabularnewline
28 & bind9\_dns\_rdata\_fromwire & 6 & \cellcolor{green!0}{\large 0}/{\footnotesize 40} & \cellcolor{green!0}{\large 0}/{\footnotesize 40} & \cellcolor{green!0}{\large -}{\tiny -} & \cellcolor{green!0}{\large 0}/{\footnotesize 40} & \cellcolor{green!0}{\large 0}/{\footnotesize 175} & \cellcolor{green!0}{\large 0}/{\footnotesize 119} \tabularnewline
\rowcolor{black!10} 29 & coturn\_stun\_is\_binding\_response & 6 & \cellcolor{green!0}{\large 0}/{\footnotesize 40} & \cellcolor{green!0}{\large 0}/{\footnotesize 40} & \cellcolor{green!0}{\large -}{\tiny -} & \cellcolor{green!40}{\large 13}/{\footnotesize 40} & \cellcolor{green!0}{\large 0}/{\footnotesize 184} & \cellcolor{green!20}{\large 9}/{\footnotesize 81} \tabularnewline
30 & coturn\_stun\_is\_command\_message & 6 & \cellcolor{green!0}{\large 0}/{\footnotesize 40} & \cellcolor{green!0}{\large 0}/{\footnotesize 40} & \cellcolor{green!0}{\large 0}/{\footnotesize 40} & \cellcolor{green!60}{\large 23}/{\footnotesize 40} & \cellcolor{green!0}{\large 0}/{\footnotesize 183} & \cellcolor{green!30}{\large 18}/{\footnotesize 78} \tabularnewline
\rowcolor{black!10} 31 & coturn\_stun\_is\_response & 6 & \cellcolor{green!0}{\large 0}/{\footnotesize 40} & \cellcolor{green!0}{\large 0}/{\footnotesize 40} & \cellcolor{green!0}{\large -}{\tiny -} & \cellcolor{green!40}{\large 14}/{\footnotesize 40} & \cellcolor{green!0}{\large 0}/{\footnotesize 182} & \cellcolor{green!10}{\large 9}/{\footnotesize 89} \tabularnewline
32 & coturn\_stun\_is\_success\_response & 6 & \cellcolor{green!0}{\large 0}/{\footnotesize 40} & \cellcolor{green!0}{\large 0}/{\footnotesize 40} & \cellcolor{green!0}{\large -}{\tiny -} & \cellcolor{green!30}{\large 9}/{\footnotesize 40} & \cellcolor{green!0}{\large 0}/{\footnotesize 183} & \cellcolor{green!10}{\large 6}/{\footnotesize 86} \tabularnewline
\rowcolor{black!10} 33 & hiredis\_redisFormatCommand & 6 & \cellcolor{green!0}{\large 0}/{\footnotesize 40} & \cellcolor{green!80}{\large 29}/{\footnotesize 40} & \cellcolor{green!0}{\large -}{\tiny -} & \cellcolor{green!80}{\large 30}/{\footnotesize 40} & \cellcolor{green!20}{\large 25}/{\footnotesize 128} & \cellcolor{green!10}{\large 17}/{\footnotesize 160} \tabularnewline
34 & igraph\_igraph\_read\_graph\_dl & 6 & \cellcolor{green!0}{\large 0}/{\footnotesize 40} & \cellcolor{green!0}{\large 0}/{\footnotesize 40} & \cellcolor{green!0}{\large 0}/{\footnotesize 40} & \cellcolor{green!10}{\large 1}/{\footnotesize 40} & \cellcolor{green!0}{\large 0}/{\footnotesize 184} & \cellcolor{green!10}{\large 3}/{\footnotesize 122} \tabularnewline
\rowcolor{black!10} 35 & igraph\_igraph\_read\_graph\_edgelist & 6 & \cellcolor{green!0}{\large 0}/{\footnotesize 40} & \cellcolor{green!0}{\large 0}/{\footnotesize 40} & \cellcolor{green!0}{\large 0}/{\footnotesize 40} & \cellcolor{green!0}{\large 0}/{\footnotesize 40} & \cellcolor{green!0}{\large 0}/{\footnotesize 174} & \cellcolor{green!0}{\large 0}/{\footnotesize 124} \tabularnewline
36 & igraph\_igraph\_read\_graph\_gml & 6 & \cellcolor{green!0}{\large 0}/{\footnotesize 40} & \cellcolor{green!0}{\large 0}/{\footnotesize 40} & \cellcolor{green!0}{\large 0}/{\footnotesize 40} & \cellcolor{green!0}{\large 0}/{\footnotesize 40} & \cellcolor{green!0}{\large 0}/{\footnotesize 184} & \cellcolor{green!0}{\large 0}/{\footnotesize 142} \tabularnewline
\rowcolor{black!10} 37 & igraph\_igraph\_read\_graph\_graphdb & 6 & \cellcolor{green!0}{\large 0}/{\footnotesize 40} & \cellcolor{green!0}{\large 0}/{\footnotesize 40} & \cellcolor{green!0}{\large 0}/{\footnotesize 40} & \cellcolor{green!10}{\large 1}/{\footnotesize 40} & \cellcolor{green!0}{\large 0}/{\footnotesize 185} & \cellcolor{green!0}{\large 0}/{\footnotesize 147} \tabularnewline
38 & igraph\_igraph\_read\_graph\_graphml & 6 & \cellcolor{green!0}{\large 0}/{\footnotesize 40} & \cellcolor{green!0}{\large 0}/{\footnotesize 40} & \cellcolor{green!0}{\large 0}/{\footnotesize 40} & \cellcolor{green!10}{\large 3}/{\footnotesize 40} & \cellcolor{green!0}{\large 0}/{\footnotesize 195} & \cellcolor{green!10}{\large 5}/{\footnotesize 109} \tabularnewline
\rowcolor{black!10} 39 & igraph\_igraph\_read\_graph\_lgl & 6 & \cellcolor{green!0}{\large 0}/{\footnotesize 40} & \cellcolor{green!0}{\large 0}/{\footnotesize 40} & \cellcolor{green!0}{\large 0}/{\footnotesize 40} & \cellcolor{green!0}{\large 0}/{\footnotesize 40} & \cellcolor{green!0}{\large 0}/{\footnotesize 199} & \cellcolor{green!0}{\large 0}/{\footnotesize 127} \tabularnewline
40 & igraph\_igraph\_read\_graph\_pajek & 6 & \cellcolor{green!0}{\large 0}/{\footnotesize 40} & \cellcolor{green!0}{\large 0}/{\footnotesize 40} & \cellcolor{green!0}{\large 0}/{\footnotesize 40} & \cellcolor{green!0}{\large 0}/{\footnotesize 40} & \cellcolor{green!0}{\large 0}/{\footnotesize 178} & \cellcolor{green!0}{\large 0}/{\footnotesize 115} \tabularnewline
\rowcolor{black!10} 41 & inchi\_GetINCHIfromINCHI & 6 & \cellcolor{green!0}{\large 0}/{\footnotesize 40} & \cellcolor{green!0}{\large 0}/{\footnotesize 40} & \cellcolor{green!20}{\large 5}/{\footnotesize 40} & \cellcolor{green!10}{\large 2}/{\footnotesize 40} & \cellcolor{green!0}{\large 0}/{\footnotesize 154} & \cellcolor{green!0}{\large 0}/{\footnotesize 117} \tabularnewline
42 & inchi\_GetStructFromINCHI & 6 & \cellcolor{green!0}{\large 0}/{\footnotesize 40} & \cellcolor{green!0}{\large 0}/{\footnotesize 40} & \cellcolor{green!10}{\large 1}/{\footnotesize 40} & \cellcolor{green!0}{\large 0}/{\footnotesize 40} & \cellcolor{green!0}{\large 1}/{\footnotesize 153} & \cellcolor{green!0}{\large 0}/{\footnotesize 127} \tabularnewline
\rowcolor{black!10} 43 & kamailio\_parse\_msg & 6 & \cellcolor{green!0}{\large 0}/{\footnotesize 40} & \cellcolor{green!20}{\large 5}/{\footnotesize 40} & \cellcolor{green!0}{\large -}{\tiny -} & \cellcolor{green!40}{\large 16}/{\footnotesize 40} & \cellcolor{green!10}{\large 10}/{\footnotesize 157} & \cellcolor{green!10}{\large 7}/{\footnotesize 116} \tabularnewline
44 & libyang\_lys\_parse\_mem & 6 & \cellcolor{green!0}{\large 0}/{\footnotesize 40} & \cellcolor{green!0}{\large 0}/{\footnotesize 40} & \cellcolor{green!0}{\large 0}/{\footnotesize 40} & \cellcolor{green!0}{\large 0}/{\footnotesize 40} & \cellcolor{green!0}{\large 0}/{\footnotesize 199} & \cellcolor{green!0}{\large 0}/{\footnotesize 146} \tabularnewline
\rowcolor{black!10} 45 & proftpd\_pr\_json\_object\_from\_text & 6 & \cellcolor{green!0}{\large 0}/{\footnotesize 40} & \cellcolor{green!0}{\large 0}/{\footnotesize 40} & \cellcolor{green!0}{\large -}{\tiny -} & \cellcolor{green!10}{\large 1}/{\footnotesize 40} & \cellcolor{green!0}{\large 0}/{\footnotesize 196} & \cellcolor{green!10}{\large 3}/{\footnotesize 130} \tabularnewline
46 & selinux\_policydb\_read & 6 & \cellcolor{green!0}{\large 0}/{\footnotesize 40} & \cellcolor{green!10}{\large 3}/{\footnotesize 40} & \cellcolor{green!0}{\large -}{\tiny -} & \cellcolor{green!10}{\large 2}/{\footnotesize 40} & \cellcolor{green!10}{\large 3}/{\footnotesize 163} & \cellcolor{green!10}{\large 3}/{\footnotesize 121} \tabularnewline
\rowcolor{black!10} 47 & kamailio\_get\_src\_address\_socket & 7 & \cellcolor{green!0}{\large 0}/{\footnotesize 40} & \cellcolor{green!0}{\large 0}/{\footnotesize 40} & \cellcolor{green!0}{\large 0}/{\footnotesize 40} & \cellcolor{green!20}{\large 8}/{\footnotesize 40} & \cellcolor{green!0}{\large 0}/{\footnotesize 188} & \cellcolor{green!10}{\large 8}/{\footnotesize 103} \tabularnewline
48 & kamailio\_get\_src\_uri & 7 & \cellcolor{green!0}{\large 0}/{\footnotesize 40} & \cellcolor{green!0}{\large 0}/{\footnotesize 40} & \cellcolor{green!0}{\large 0}/{\footnotesize 40} & \cellcolor{green!10}{\large 2}/{\footnotesize 40} & \cellcolor{green!0}{\large 0}/{\footnotesize 181} & \cellcolor{green!10}{\large 4}/{\footnotesize 138} \tabularnewline
\rowcolor{black!10} 49 & kamailio\_parse\_content\_disposition & 7 & \cellcolor{green!0}{\large 0}/{\footnotesize 40} & \cellcolor{green!0}{\large 0}/{\footnotesize 40} & \cellcolor{green!0}{\large 0}/{\footnotesize 40} & \cellcolor{green!10}{\large 3}/{\footnotesize 40} & \cellcolor{green!0}{\large 0}/{\footnotesize 196} & \cellcolor{green!10}{\large 6}/{\footnotesize 111} \tabularnewline
50 & kamailio\_parse\_diversion\_header & 7 & \cellcolor{green!0}{\large 0}/{\footnotesize 40} & \cellcolor{green!0}{\large 0}/{\footnotesize 40} & \cellcolor{green!0}{\large 0}/{\footnotesize 40} & \cellcolor{green!10}{\large 4}/{\footnotesize 40} & \cellcolor{green!0}{\large 0}/{\footnotesize 188} & \cellcolor{green!10}{\large 5}/{\footnotesize 109} \tabularnewline
\rowcolor{black!10} 51 & kamailio\_parse\_from\_header & 7 & \cellcolor{green!0}{\large 0}/{\footnotesize 40} & \cellcolor{green!0}{\large 0}/{\footnotesize 40} & \cellcolor{green!0}{\large -}{\tiny -} & \cellcolor{green!10}{\large 1}/{\footnotesize 40} & \cellcolor{green!0}{\large 0}/{\footnotesize 192} & \cellcolor{green!0}{\large 0}/{\footnotesize 117} \tabularnewline
52 & kamailio\_parse\_from\_uri & 7 & \cellcolor{green!0}{\large 0}/{\footnotesize 40} & \cellcolor{green!0}{\large 0}/{\footnotesize 40} & \cellcolor{green!0}{\large -}{\tiny -} & \cellcolor{green!10}{\large 2}/{\footnotesize 40} & \cellcolor{green!0}{\large 0}/{\footnotesize 199} & \cellcolor{green!0}{\large 1}/{\footnotesize 124} \tabularnewline
\rowcolor{black!10} 53 & kamailio\_parse\_headers & 7 & \cellcolor{green!0}{\large 0}/{\footnotesize 40} & \cellcolor{green!0}{\large 0}/{\footnotesize 40} & \cellcolor{green!0}{\large -}{\tiny -} & \cellcolor{green!10}{\large 1}/{\footnotesize 40} & \cellcolor{green!0}{\large 0}/{\footnotesize 174} & \cellcolor{green!10}{\large 2}/{\footnotesize 94} \tabularnewline
54 & kamailio\_parse\_identityinfo\_header & 7 & \cellcolor{green!0}{\large 0}/{\footnotesize 40} & \cellcolor{green!0}{\large 0}/{\footnotesize 40} & \cellcolor{green!0}{\large -}{\tiny -} & \cellcolor{green!50}{\large 17}/{\footnotesize 40} & \cellcolor{green!0}{\large 0}/{\footnotesize 194} & \cellcolor{green!10}{\large 11}/{\footnotesize 102} \tabularnewline
\rowcolor{black!10} 55 & kamailio\_parse\_pai\_header & 7 & \cellcolor{green!0}{\large 0}/{\footnotesize 40} & \cellcolor{green!0}{\large 0}/{\footnotesize 40} & \cellcolor{green!0}{\large -}{\tiny -} & \cellcolor{green!10}{\large 4}/{\footnotesize 40} & \cellcolor{green!0}{\large 0}/{\footnotesize 183} & \cellcolor{green!10}{\large 5}/{\footnotesize 116} \tabularnewline
56 & kamailio\_parse\_privacy & 7 & \cellcolor{green!0}{\large 0}/{\footnotesize 40} & \cellcolor{green!0}{\large 0}/{\footnotesize 40} & \cellcolor{green!0}{\large 0}/{\footnotesize 40} & \cellcolor{green!20}{\large 7}/{\footnotesize 40} & \cellcolor{green!0}{\large 0}/{\footnotesize 198} & \cellcolor{green!10}{\large 11}/{\footnotesize 116} \tabularnewline
\rowcolor{black!10} 57 & kamailio\_parse\_record\_route\_headers & 7 & \cellcolor{green!0}{\large 0}/{\footnotesize 40} & \cellcolor{green!0}{\large 0}/{\footnotesize 40} & \cellcolor{green!0}{\large -}{\tiny -} & \cellcolor{green!70}{\large 28}/{\footnotesize 40} & \cellcolor{green!0}{\large 0}/{\footnotesize 192} & \cellcolor{green!30}{\large 22}/{\footnotesize 99} \tabularnewline
58 & kamailio\_parse\_refer\_to\_header & 7 & \cellcolor{green!0}{\large 0}/{\footnotesize 40} & \cellcolor{green!0}{\large 0}/{\footnotesize 40} & \cellcolor{green!0}{\large -}{\tiny -} & \cellcolor{green!30}{\large 9}/{\footnotesize 40} & \cellcolor{green!0}{\large 0}/{\footnotesize 195} & \cellcolor{green!10}{\large 6}/{\footnotesize 117} \tabularnewline
\rowcolor{black!10} 59 & kamailio\_parse\_route\_headers & 7 & \cellcolor{green!0}{\large 0}/{\footnotesize 40} & \cellcolor{green!0}{\large 0}/{\footnotesize 40} & \cellcolor{green!0}{\large -}{\tiny -} & \cellcolor{green!80}{\large 32}/{\footnotesize 40} & \cellcolor{green!0}{\large 0}/{\footnotesize 196} & \cellcolor{green!20}{\large 19}/{\footnotesize 108} \tabularnewline
60 & kamailio\_parse\_to\_header & 7 & \cellcolor{green!0}{\large 0}/{\footnotesize 40} & \cellcolor{green!0}{\large 0}/{\footnotesize 40} & \cellcolor{green!0}{\large -}{\tiny -} & \cellcolor{green!10}{\large 1}/{\footnotesize 40} & \cellcolor{green!0}{\large 0}/{\footnotesize 193} & \cellcolor{green!10}{\large 1}/{\footnotesize 88} \tabularnewline
\rowcolor{black!10} 61 & kamailio\_parse\_to\_uri & 7 & \cellcolor{green!0}{\large 0}/{\footnotesize 40} & \cellcolor{green!0}{\large 0}/{\footnotesize 40} & \cellcolor{green!0}{\large -}{\tiny -} & \cellcolor{green!10}{\large 2}/{\footnotesize 40} & \cellcolor{green!0}{\large 0}/{\footnotesize 194} & \cellcolor{green!10}{\large 2}/{\footnotesize 103} \tabularnewline
62 & libyang\_lyd\_parse\_data\_mem & 7 & \cellcolor{green!0}{\large 0}/{\footnotesize 40} & \cellcolor{green!0}{\large 0}/{\footnotesize 40} & \cellcolor{green!0}{\large 0}/{\footnotesize 40} & \cellcolor{green!0}{\large 0}/{\footnotesize 40} & \cellcolor{green!0}{\large 0}/{\footnotesize 171} & \cellcolor{green!0}{\large 0}/{\footnotesize 114} \tabularnewline
\rowcolor{black!10} 63 & bind9\_dns\_message\_parse & 8 & \cellcolor{green!0}{\large 0}/{\footnotesize 40} & \cellcolor{green!0}{\large 0}/{\footnotesize 40} & \cellcolor{green!0}{\large -}{\tiny -} & \cellcolor{green!0}{\large 0}/{\footnotesize 40} & \cellcolor{green!0}{\large 0}/{\footnotesize 200} & \cellcolor{green!0}{\large 0}/{\footnotesize 130} \tabularnewline
64 & igraph\_igraph\_read\_graph\_ncol & 8 & \cellcolor{green!0}{\large 0}/{\footnotesize 40} & \cellcolor{green!0}{\large 0}/{\footnotesize 40} & \cellcolor{green!0}{\large 0}/{\footnotesize 40} & \cellcolor{green!0}{\large 0}/{\footnotesize 40} & \cellcolor{green!0}{\large 0}/{\footnotesize 183} & \cellcolor{green!0}{\large 0}/{\footnotesize 106} \tabularnewline
\rowcolor{black!10} 65 & pjsip\_pj\_json\_parse & 8 & \cellcolor{green!0}{\large 0}/{\footnotesize 40} & \cellcolor{green!0}{\large 0}/{\footnotesize 40} & \cellcolor{green!0}{\large 0}/{\footnotesize 40} & \cellcolor{green!0}{\large 0}/{\footnotesize 40} & \cellcolor{green!0}{\large 0}/{\footnotesize 200} & \cellcolor{green!0}{\large 0}/{\footnotesize 129} \tabularnewline
66 & pjsip\_pj\_xml\_parse & 8 & \cellcolor{green!0}{\large 0}/{\footnotesize 40} & \cellcolor{green!0}{\large 0}/{\footnotesize 40} & \cellcolor{green!0}{\large 0}/{\footnotesize 40} & \cellcolor{green!0}{\large 0}/{\footnotesize 40} & \cellcolor{green!0}{\large 0}/{\footnotesize 195} & \cellcolor{green!0}{\large 0}/{\footnotesize 129} \tabularnewline
\rowcolor{black!10} 67 & pjsip\_pjmedia\_sdp\_parse & 8 & \cellcolor{green!0}{\large 0}/{\footnotesize 40} & \cellcolor{green!0}{\large 0}/{\footnotesize 40} & \cellcolor{green!0}{\large 0}/{\footnotesize 40} & \cellcolor{green!0}{\large 0}/{\footnotesize 40} & \cellcolor{green!0}{\large 0}/{\footnotesize 191} & \cellcolor{green!0}{\large 0}/{\footnotesize 121} \tabularnewline
68 & quickjs\_lre\_compile & 8 & \cellcolor{green!0}{\large 0}/{\footnotesize 40} & \cellcolor{green!0}{\large 0}/{\footnotesize 40} & \cellcolor{green!0}{\large -}{\tiny -} & \cellcolor{green!0}{\large 0}/{\footnotesize 40} & \cellcolor{green!0}{\large 0}/{\footnotesize 186} & \cellcolor{green!0}{\large 0}/{\footnotesize 111} \tabularnewline
\rowcolor{black!10} 69 & bind9\_isc\_lex\_getmastertoken & 9 & \cellcolor{green!0}{\large 0}/{\footnotesize 40} & \cellcolor{green!0}{\large 0}/{\footnotesize 40} & \cellcolor{green!0}{\large -}{\tiny -} & \cellcolor{green!0}{\large 0}/{\footnotesize 40} & \cellcolor{green!0}{\large 0}/{\footnotesize 192} & \cellcolor{green!0}{\large 0}/{\footnotesize 98} \tabularnewline
70 & bind9\_isc\_lex\_gettoken & 9 & \cellcolor{green!0}{\large 0}/{\footnotesize 40} & \cellcolor{green!0}{\large 0}/{\footnotesize 40} & \cellcolor{green!0}{\large -}{\tiny -} & \cellcolor{green!0}{\large 0}/{\footnotesize 40} & \cellcolor{green!0}{\large 0}/{\footnotesize 198} & \cellcolor{green!0}{\large 0}/{\footnotesize 112} \tabularnewline
\rowcolor{black!10} 71 & quickjs\_JS\_Eval & 9 & \cellcolor{green!0}{\large 0}/{\footnotesize 40} & \cellcolor{green!10}{\large 2}/{\footnotesize 40} & \cellcolor{green!0}{\large -}{\tiny -} & \cellcolor{green!10}{\large 2}/{\footnotesize 40} & \cellcolor{green!10}{\large 4}/{\footnotesize 162} & \cellcolor{green!0}{\large 1}/{\footnotesize 120} \tabularnewline
72 & igraph\_igraph\_edge\_connectivity & 10 & \cellcolor{green!0}{\large 0}/{\footnotesize 40} & \cellcolor{green!0}{\large 0}/{\footnotesize 40} & \cellcolor{green!0}{\large 0}/{\footnotesize 40} & \cellcolor{green!0}{\large 0}/{\footnotesize 40} & \cellcolor{green!0}{\large 0}/{\footnotesize 131} & \cellcolor{green!0}{\large 0}/{\footnotesize 143} \tabularnewline
\rowcolor{black!10} 73 & pjsip\_pj\_stun\_msg\_decode & 10 & \cellcolor{green!0}{\large 0}/{\footnotesize 40} & \cellcolor{green!0}{\large 0}/{\footnotesize 40} & \cellcolor{green!0}{\large 0}/{\footnotesize 40} & \cellcolor{green!0}{\large 0}/{\footnotesize 40} & \cellcolor{green!0}{\large 0}/{\footnotesize 190} & \cellcolor{green!0}{\large 0}/{\footnotesize 96} \tabularnewline
74 & bind9\_dns\_message\_checksig & 11 & \cellcolor{green!0}{\large 0}/{\footnotesize 40} & \cellcolor{green!0}{\large 0}/{\footnotesize 40} & \cellcolor{green!0}{\large -}{\tiny -} & \cellcolor{green!0}{\large 0}/{\footnotesize 40} & \cellcolor{green!0}{\large 0}/{\footnotesize 166} & \cellcolor{green!0}{\large 0}/{\footnotesize 122} \tabularnewline
\rowcolor{black!10} 75 & libzip\_zip\_fread & 11 & \cellcolor{green!0}{\large 0}/{\footnotesize 40} & \cellcolor{green!0}{\large 0}/{\footnotesize 40} & \cellcolor{green!10}{\large 1}/{\footnotesize 40} & \cellcolor{green!10}{\large 1}/{\footnotesize 40} & \cellcolor{green!10}{\large 2}/{\footnotesize 160} & \cellcolor{green!10}{\large 2}/{\footnotesize 96} \tabularnewline
76 & bind9\_dns\_rdata\_fromtext & 12 & \cellcolor{green!0}{\large 0}/{\footnotesize 40} & \cellcolor{green!0}{\large 0}/{\footnotesize 40} & \cellcolor{green!0}{\large -}{\tiny -} & \cellcolor{green!0}{\large 0}/{\footnotesize 40} & \cellcolor{green!0}{\large 0}/{\footnotesize 141} & \cellcolor{green!0}{\large 0}/{\footnotesize 57} \tabularnewline
\rowcolor{black!10} 77 & igraph\_igraph\_all\_minimal\_st\_separators & 12 & \cellcolor{green!0}{\large 0}/{\footnotesize 40} & \cellcolor{green!0}{\large 0}/{\footnotesize 40} & \cellcolor{green!0}{\large 0}/{\footnotesize 40} & \cellcolor{green!0}{\large 0}/{\footnotesize 40} & \cellcolor{green!0}{\large 0}/{\footnotesize 65} & \cellcolor{green!0}{\large 0}/{\footnotesize 90} \tabularnewline
78 & igraph\_igraph\_minimum\_size\_separators & 12 & \cellcolor{green!0}{\large 0}/{\footnotesize 40} & \cellcolor{green!0}{\large 0}/{\footnotesize 40} & \cellcolor{green!0}{\large 0}/{\footnotesize 40} & \cellcolor{green!0}{\large 0}/{\footnotesize 40} & \cellcolor{green!0}{\large 0}/{\footnotesize 103} & \cellcolor{green!0}{\large 0}/{\footnotesize 85} \tabularnewline
\rowcolor{black!10} 79 & pjsip\_pjsip\_parse\_msg & 12 & \cellcolor{green!0}{\large 0}/{\footnotesize 40} & \cellcolor{green!0}{\large 0}/{\footnotesize 40} & \cellcolor{green!0}{\large 0}/{\footnotesize 40} & \cellcolor{green!0}{\large 0}/{\footnotesize 40} & \cellcolor{green!0}{\large 0}/{\footnotesize 197} & \cellcolor{green!0}{\large 0}/{\footnotesize 105} \tabularnewline
80 & igraph\_igraph\_automorphism\_group & 13 & \cellcolor{green!0}{\large 0}/{\footnotesize 40} & \cellcolor{green!0}{\large 0}/{\footnotesize 40} & \cellcolor{green!0}{\large 0}/{\footnotesize 40} & \cellcolor{green!0}{\large 0}/{\footnotesize 40} & \cellcolor{green!0}{\large 0}/{\footnotesize 117} & \cellcolor{green!0}{\large 0}/{\footnotesize 80} \tabularnewline
\rowcolor{black!10} 81 & libmodbus\_modbus\_read\_bits & 15 & \cellcolor{green!0}{\large 0}/{\footnotesize 40} & \cellcolor{green!0}{\large 0}/{\footnotesize 40} & \cellcolor{green!0}{\large 0}/{\footnotesize 40} & \cellcolor{green!0}{\large 0}/{\footnotesize 40} & \cellcolor{green!0}{\large 0}/{\footnotesize 136} & \cellcolor{green!0}{\large 0}/{\footnotesize 97} \tabularnewline
82 & libmodbus\_modbus\_read\_registers & 15 & \cellcolor{green!0}{\large 0}/{\footnotesize 40} & \cellcolor{green!0}{\large 0}/{\footnotesize 40} & \cellcolor{green!0}{\large 0}/{\footnotesize 40} & \cellcolor{green!0}{\large 0}/{\footnotesize 40} & \cellcolor{green!0}{\large 0}/{\footnotesize 132} & \cellcolor{green!0}{\large 0}/{\footnotesize 97} \tabularnewline
\rowcolor{black!10} 83 & civetweb\_mg\_get\_response & 17 & \cellcolor{green!0}{\large 0}/{\footnotesize 40} & \cellcolor{green!0}{\large 0}/{\footnotesize 40} & \cellcolor{green!0}{\large 0}/{\footnotesize 40} & \cellcolor{green!0}{\large 0}/{\footnotesize 40} & \cellcolor{green!0}{\large 0}/{\footnotesize 155} & \cellcolor{green!0}{\large 0}/{\footnotesize 94} \tabularnewline
84 & bind9\_dns\_master\_loadbuffer & 20 & \cellcolor{green!0}{\large 0}/{\footnotesize 40} & \cellcolor{green!0}{\large 0}/{\footnotesize 40} & \cellcolor{green!0}{\large -}{\tiny -} & \cellcolor{green!0}{\large 0}/{\footnotesize 40} & \cellcolor{green!0}{\large 0}/{\footnotesize 189} & \cellcolor{green!0}{\large 0}/{\footnotesize 119} \tabularnewline
\rowcolor{black!10} 85 & libmodbus\_modbus\_receive & 33 & \cellcolor{green!0}{\large 0}/{\footnotesize 40} & \cellcolor{green!0}{\large 0}/{\footnotesize 40} & \cellcolor{green!0}{\large 0}/{\footnotesize 40} & \cellcolor{green!0}{\large 0}/{\footnotesize 40} & \cellcolor{green!0}{\large 0}/{\footnotesize 115} & \cellcolor{green!0}{\large 0}/{\footnotesize 81} \tabularnewline
86 & tmux\_input\_parse\_buffer & 42 & \cellcolor{green!0}{\large 0}/{\footnotesize 40} & \cellcolor{green!0}{\large 0}/{\footnotesize 40} & \cellcolor{green!0}{\large -}{\tiny -} & \cellcolor{green!0}{\large 0}/{\footnotesize 40} & \cellcolor{green!0}{\large 0}/{\footnotesize 187} & \cellcolor{green!0}{\large 0}/{\footnotesize 133} \tabularnewline

\bottomrule
%\end{tabular}
%}
%\end{table*}
\end{xltabular}
}
\twocolumn



% model: wizardcoder-15b-v1.0, temp: 1.0

\onecolumn
{\small %
\begin{xltabular}[h]{\textwidth}{ccccccccc}
%\begin{table*}[!t]
%\centering
\caption{Evaluation Result of model wizardcoder-15b-v1.0 with temperature 1.0.} \\
%\resizebox{1.0\linewidth}{!}{
%\begin{tabular}{cccccccccc}
\toprule
Index & Question & Score & NAIVE-40 & BACTX-40 & DOCTX-40 & UGCTX-40 & BA-ITER-40 & ALL-ITER-40 \tabularnewline
\midrule
\rowcolor{black!10} 1 & coturn\_stun\_is\_command\_message\_full\_check\_str & 1 & \cellcolor{green!0}{\large 0}/{\footnotesize 40} & \cellcolor{green!40}{\large 16}/{\footnotesize 40} & \cellcolor{green!0}{\large -}{\tiny -} & \cellcolor{green!20}{\large 6}/{\footnotesize 40} & \cellcolor{green!40}{\large 25}/{\footnotesize 68} & \cellcolor{green!30}{\large 18}/{\footnotesize 59} \tabularnewline
2 & kamailio\_parse\_uri & 1 & \cellcolor{green!0}{\large 0}/{\footnotesize 40} & \cellcolor{green!50}{\large 20}/{\footnotesize 40} & \cellcolor{green!0}{\large -}{\tiny -} & \cellcolor{green!40}{\large 15}/{\footnotesize 40} & \cellcolor{green!50}{\large 29}/{\footnotesize 57} & \cellcolor{green!40}{\large 22}/{\footnotesize 60} \tabularnewline
\rowcolor{black!10} 3 & coturn\_stun\_check\_message\_integrity\_str & 2 & \cellcolor{green!0}{\large 0}/{\footnotesize 40} & \cellcolor{green!10}{\large 1}/{\footnotesize 40} & \cellcolor{green!0}{\large -}{\tiny -} & \cellcolor{green!10}{\large 1}/{\footnotesize 40} & \cellcolor{green!10}{\large 2}/{\footnotesize 116} & \cellcolor{green!0}{\large 0}/{\footnotesize 83} \tabularnewline
4 & libiec61850\_MmsValue\_decodeMmsData & 2 & \cellcolor{green!0}{\large 0}/{\footnotesize 40} & \cellcolor{green!30}{\large 10}/{\footnotesize 40} & \cellcolor{green!50}{\large 17}/{\footnotesize 40} & \cellcolor{green!40}{\large 16}/{\footnotesize 40} & \cellcolor{green!50}{\large 27}/{\footnotesize 63} & \cellcolor{green!20}{\large 15}/{\footnotesize 84} \tabularnewline
\rowcolor{black!10} 5 & md4c\_md\_html & 2 & \cellcolor{green!0}{\large 0}/{\footnotesize 40} & \cellcolor{green!0}{\large 0}/{\footnotesize 40} & \cellcolor{green!0}{\large 0}/{\footnotesize 40} & \cellcolor{green!0}{\large 0}/{\footnotesize 40} & \cellcolor{green!0}{\large 1}/{\footnotesize 165} & \cellcolor{green!0}{\large 0}/{\footnotesize 102} \tabularnewline
6 & spdk\_spdk\_json\_parse & 2 & \cellcolor{green!0}{\large 0}/{\footnotesize 40} & \cellcolor{green!40}{\large 14}/{\footnotesize 40} & \cellcolor{green!0}{\large -}{\tiny -} & \cellcolor{green!30}{\large 10}/{\footnotesize 40} & \cellcolor{green!30}{\large 25}/{\footnotesize 95} & \cellcolor{green!20}{\large 13}/{\footnotesize 90} \tabularnewline
\rowcolor{black!10} 7 & croaring\_roaring\_bitmap\_portable\_deserialize\_safe & 3 & \cellcolor{green!10}{\large 2}/{\footnotesize 40} & \cellcolor{green!50}{\large 17}/{\footnotesize 40} & \cellcolor{green!30}{\large 11}/{\footnotesize 40} & \cellcolor{green!30}{\large 12}/{\footnotesize 40} & \cellcolor{green!30}{\large 24}/{\footnotesize 83} & \cellcolor{green!20}{\large 12}/{\footnotesize 89} \tabularnewline
8 & lua\_luaL\_loadbufferx & 3 & \cellcolor{green!40}{\large 13}/{\footnotesize 40} & \cellcolor{green!80}{\large 30}/{\footnotesize 40} & \cellcolor{green!80}{\large 30}/{\footnotesize 40} & \cellcolor{green!30}{\large 10}/{\footnotesize 40} & \cellcolor{green!60}{\large 34}/{\footnotesize 60} & \cellcolor{green!20}{\large 15}/{\footnotesize 75} \tabularnewline
\rowcolor{black!10} 9 & w3m\_wc\_Str\_conv\_with\_detect & 3 & \cellcolor{green!0}{\large 0}/{\footnotesize 40} & \cellcolor{green!0}{\large 0}/{\footnotesize 40} & \cellcolor{green!0}{\large -}{\tiny -} & \cellcolor{green!10}{\large 1}/{\footnotesize 40} & \cellcolor{green!0}{\large 0}/{\footnotesize 146} & \cellcolor{green!10}{\large 2}/{\footnotesize 93} \tabularnewline
10 & bind9\_dns\_name\_fromwire & 4 & \cellcolor{green!0}{\large 0}/{\footnotesize 40} & \cellcolor{green!0}{\large 0}/{\footnotesize 40} & \cellcolor{green!0}{\large -}{\tiny -} & \cellcolor{green!10}{\large 1}/{\footnotesize 40} & \cellcolor{green!0}{\large 0}/{\footnotesize 136} & \cellcolor{green!10}{\large 3}/{\footnotesize 116} \tabularnewline
\rowcolor{black!10} 11 & gdk-pixbuf\_gdk\_pixbuf\_animation\_new\_from\_file & 4 & \cellcolor{green!10}{\large 1}/{\footnotesize 40} & \cellcolor{green!10}{\large 3}/{\footnotesize 40} & \cellcolor{green!10}{\large 1}/{\footnotesize 40} & \cellcolor{green!0}{\large 0}/{\footnotesize 40} & \cellcolor{green!10}{\large 1}/{\footnotesize 90} & \cellcolor{green!10}{\large 1}/{\footnotesize 84} \tabularnewline
12 & gdk-pixbuf\_gdk\_pixbuf\_new\_from\_data & 4 & \cellcolor{green!10}{\large 2}/{\footnotesize 40} & \cellcolor{green!40}{\large 13}/{\footnotesize 40} & \cellcolor{green!50}{\large 18}/{\footnotesize 40} & \cellcolor{green!40}{\large 14}/{\footnotesize 40} & \cellcolor{green!40}{\large 21}/{\footnotesize 63} & \cellcolor{green!20}{\large 9}/{\footnotesize 77} \tabularnewline
\rowcolor{black!10} 13 & gdk-pixbuf\_gdk\_pixbuf\_new\_from\_file & 4 & \cellcolor{green!10}{\large 2}/{\footnotesize 40} & \cellcolor{green!10}{\large 1}/{\footnotesize 40} & \cellcolor{green!10}{\large 1}/{\footnotesize 40} & \cellcolor{green!0}{\large 0}/{\footnotesize 40} & \cellcolor{green!10}{\large 2}/{\footnotesize 72} & \cellcolor{green!0}{\large 0}/{\footnotesize 92} \tabularnewline
14 & gdk-pixbuf\_gdk\_pixbuf\_new\_from\_stream & 4 & \cellcolor{green!10}{\large 2}/{\footnotesize 40} & \cellcolor{green!50}{\large 17}/{\footnotesize 40} & \cellcolor{green!50}{\large 17}/{\footnotesize 40} & \cellcolor{green!30}{\large 12}/{\footnotesize 40} & \cellcolor{green!30}{\large 25}/{\footnotesize 89} & \cellcolor{green!40}{\large 28}/{\footnotesize 76} \tabularnewline
\rowcolor{black!10} 15 & gpac\_gf\_isom\_open\_file & 4 & \cellcolor{green!0}{\large 0}/{\footnotesize 40} & \cellcolor{green!20}{\large 6}/{\footnotesize 40} & \cellcolor{green!0}{\large -}{\tiny -} & \cellcolor{green!10}{\large 3}/{\footnotesize 40} & \cellcolor{green!0}{\large 1}/{\footnotesize 141} & \cellcolor{green!0}{\large 0}/{\footnotesize 112} \tabularnewline
16 & libbpf\_bpf\_object\_\_open\_mem & 4 & \cellcolor{green!10}{\large 1}/{\footnotesize 40} & \cellcolor{green!10}{\large 3}/{\footnotesize 40} & \cellcolor{green!20}{\large 8}/{\footnotesize 40} & \cellcolor{green!20}{\large 8}/{\footnotesize 40} & \cellcolor{green!10}{\large 7}/{\footnotesize 110} & \cellcolor{green!10}{\large 5}/{\footnotesize 75} \tabularnewline
\rowcolor{black!10} 17 & libpg\_query\_pg\_query\_parse & 4 & \cellcolor{green!0}{\large 0}/{\footnotesize 40} & \cellcolor{green!10}{\large 2}/{\footnotesize 40} & \cellcolor{green!0}{\large -}{\tiny -} & \cellcolor{green!30}{\large 12}/{\footnotesize 40} & \cellcolor{green!10}{\large 11}/{\footnotesize 115} & \cellcolor{green!40}{\large 23}/{\footnotesize 73} \tabularnewline
18 & libucl\_ucl\_parser\_add\_string & 4 & \cellcolor{green!10}{\large 1}/{\footnotesize 40} & \cellcolor{green!30}{\large 10}/{\footnotesize 40} & \cellcolor{green!40}{\large 14}/{\footnotesize 40} & \cellcolor{green!20}{\large 6}/{\footnotesize 40} & \cellcolor{green!30}{\large 26}/{\footnotesize 99} & \cellcolor{green!20}{\large 12}/{\footnotesize 96} \tabularnewline
\rowcolor{black!10} 19 & oniguruma\_onig\_new & 4 & \cellcolor{green!10}{\large 1}/{\footnotesize 40} & \cellcolor{green!20}{\large 7}/{\footnotesize 40} & \cellcolor{green!20}{\large 7}/{\footnotesize 40} & \cellcolor{green!20}{\large 8}/{\footnotesize 40} & \cellcolor{green!10}{\large 11}/{\footnotesize 132} & \cellcolor{green!10}{\large 10}/{\footnotesize 93} \tabularnewline
20 & pupnp\_ixmlLoadDocumentEx & 4 & \cellcolor{green!0}{\large 0}/{\footnotesize 40} & \cellcolor{green!0}{\large 0}/{\footnotesize 40} & \cellcolor{green!0}{\large 0}/{\footnotesize 40} & \cellcolor{green!10}{\large 1}/{\footnotesize 40} & \cellcolor{green!0}{\large 0}/{\footnotesize 123} & \cellcolor{green!0}{\large 0}/{\footnotesize 101} \tabularnewline
\rowcolor{black!10} 21 & gdk-pixbuf\_gdk\_pixbuf\_new\_from\_file\_at\_scale & 5 & \cellcolor{green!10}{\large 1}/{\footnotesize 40} & \cellcolor{green!0}{\large 0}/{\footnotesize 40} & \cellcolor{green!0}{\large 0}/{\footnotesize 40} & \cellcolor{green!10}{\large 1}/{\footnotesize 40} & \cellcolor{green!10}{\large 3}/{\footnotesize 105} & \cellcolor{green!10}{\large 1}/{\footnotesize 65} \tabularnewline
22 & inchi\_GetINCHIKeyFromINCHI & 5 & \cellcolor{green!0}{\large 0}/{\footnotesize 40} & \cellcolor{green!40}{\large 13}/{\footnotesize 40} & \cellcolor{green!30}{\large 9}/{\footnotesize 40} & \cellcolor{green!20}{\large 7}/{\footnotesize 40} & \cellcolor{green!30}{\large 23}/{\footnotesize 102} & \cellcolor{green!10}{\large 9}/{\footnotesize 100} \tabularnewline
\rowcolor{black!10} 23 & libdwarf\_dwarf\_init\_b & 5 & \cellcolor{green!0}{\large 0}/{\footnotesize 40} & \cellcolor{green!0}{\large 0}/{\footnotesize 40} & \cellcolor{green!0}{\large 0}/{\footnotesize 40} & \cellcolor{green!10}{\large 4}/{\footnotesize 40} & \cellcolor{green!0}{\large 0}/{\footnotesize 110} & \cellcolor{green!20}{\large 10}/{\footnotesize 74} \tabularnewline
24 & libdwarf\_dwarf\_init\_path & 5 & \cellcolor{green!0}{\large 0}/{\footnotesize 40} & \cellcolor{green!0}{\large 0}/{\footnotesize 40} & \cellcolor{green!0}{\large 0}/{\footnotesize 40} & \cellcolor{green!0}{\large 0}/{\footnotesize 40} & \cellcolor{green!0}{\large 0}/{\footnotesize 117} & \cellcolor{green!0}{\large 0}/{\footnotesize 78} \tabularnewline
\rowcolor{black!10} 25 & liblouis\_lou\_compileString & 5 & \cellcolor{green!0}{\large 0}/{\footnotesize 40} & \cellcolor{green!10}{\large 2}/{\footnotesize 40} & \cellcolor{green!10}{\large 1}/{\footnotesize 40} & \cellcolor{green!20}{\large 5}/{\footnotesize 40} & \cellcolor{green!10}{\large 11}/{\footnotesize 111} & \cellcolor{green!10}{\large 3}/{\footnotesize 82} \tabularnewline
26 & selinux\_cil\_compile & 5 & \cellcolor{green!0}{\large 0}/{\footnotesize 40} & \cellcolor{green!0}{\large 0}/{\footnotesize 40} & \cellcolor{green!0}{\large -}{\tiny -} & \cellcolor{green!10}{\large 3}/{\footnotesize 40} & \cellcolor{green!0}{\large 1}/{\footnotesize 114} & \cellcolor{green!10}{\large 6}/{\footnotesize 69} \tabularnewline
\rowcolor{black!10} 27 & bind9\_dns\_name\_fromtext & 6 & \cellcolor{green!0}{\large 0}/{\footnotesize 40} & \cellcolor{green!10}{\large 2}/{\footnotesize 40} & \cellcolor{green!0}{\large -}{\tiny -} & \cellcolor{green!10}{\large 4}/{\footnotesize 40} & \cellcolor{green!0}{\large 1}/{\footnotesize 124} & \cellcolor{green!10}{\large 7}/{\footnotesize 82} \tabularnewline
28 & bind9\_dns\_rdata\_fromwire & 6 & \cellcolor{green!0}{\large 0}/{\footnotesize 40} & \cellcolor{green!0}{\large 0}/{\footnotesize 40} & \cellcolor{green!0}{\large -}{\tiny -} & \cellcolor{green!0}{\large 0}/{\footnotesize 40} & \cellcolor{green!10}{\large 2}/{\footnotesize 128} & \cellcolor{green!10}{\large 2}/{\footnotesize 89} \tabularnewline
\rowcolor{black!10} 29 & coturn\_stun\_is\_binding\_response & 6 & \cellcolor{green!0}{\large 0}/{\footnotesize 40} & \cellcolor{green!0}{\large 0}/{\footnotesize 40} & \cellcolor{green!0}{\large -}{\tiny -} & \cellcolor{green!20}{\large 8}/{\footnotesize 40} & \cellcolor{green!10}{\large 2}/{\footnotesize 131} & \cellcolor{green!10}{\large 7}/{\footnotesize 82} \tabularnewline
30 & coturn\_stun\_is\_command\_message & 6 & \cellcolor{green!0}{\large 0}/{\footnotesize 40} & \cellcolor{green!0}{\large 0}/{\footnotesize 40} & \cellcolor{green!0}{\large 0}/{\footnotesize 40} & \cellcolor{green!20}{\large 5}/{\footnotesize 40} & \cellcolor{green!0}{\large 1}/{\footnotesize 124} & \cellcolor{green!10}{\large 8}/{\footnotesize 89} \tabularnewline
\rowcolor{black!10} 31 & coturn\_stun\_is\_response & 6 & \cellcolor{green!0}{\large 0}/{\footnotesize 40} & \cellcolor{green!0}{\large 0}/{\footnotesize 40} & \cellcolor{green!0}{\large -}{\tiny -} & \cellcolor{green!30}{\large 9}/{\footnotesize 40} & \cellcolor{green!0}{\large 0}/{\footnotesize 123} & \cellcolor{green!10}{\large 3}/{\footnotesize 88} \tabularnewline
32 & coturn\_stun\_is\_success\_response & 6 & \cellcolor{green!0}{\large 0}/{\footnotesize 40} & \cellcolor{green!0}{\large 0}/{\footnotesize 40} & \cellcolor{green!0}{\large -}{\tiny -} & \cellcolor{green!20}{\large 7}/{\footnotesize 40} & \cellcolor{green!0}{\large 1}/{\footnotesize 117} & \cellcolor{green!10}{\large 5}/{\footnotesize 82} \tabularnewline
\rowcolor{black!10} 33 & hiredis\_redisFormatCommand & 6 & \cellcolor{green!10}{\large 1}/{\footnotesize 40} & \cellcolor{green!50}{\large 19}/{\footnotesize 40} & \cellcolor{green!0}{\large -}{\tiny -} & \cellcolor{green!50}{\large 17}/{\footnotesize 40} & \cellcolor{green!10}{\large 7}/{\footnotesize 125} & \cellcolor{green!10}{\large 3}/{\footnotesize 117} \tabularnewline
34 & igraph\_igraph\_read\_graph\_dl & 6 & \cellcolor{green!0}{\large 0}/{\footnotesize 40} & \cellcolor{green!0}{\large 0}/{\footnotesize 40} & \cellcolor{green!0}{\large 0}/{\footnotesize 40} & \cellcolor{green!0}{\large 0}/{\footnotesize 40} & \cellcolor{green!0}{\large 1}/{\footnotesize 152} & \cellcolor{green!10}{\large 1}/{\footnotesize 89} \tabularnewline
\rowcolor{black!10} 35 & igraph\_igraph\_read\_graph\_edgelist & 6 & \cellcolor{green!0}{\large 0}/{\footnotesize 40} & \cellcolor{green!0}{\large 0}/{\footnotesize 40} & \cellcolor{green!0}{\large 0}/{\footnotesize 40} & \cellcolor{green!0}{\large 0}/{\footnotesize 40} & \cellcolor{green!0}{\large 0}/{\footnotesize 150} & \cellcolor{green!0}{\large 0}/{\footnotesize 102} \tabularnewline
36 & igraph\_igraph\_read\_graph\_gml & 6 & \cellcolor{green!0}{\large 0}/{\footnotesize 40} & \cellcolor{green!0}{\large 0}/{\footnotesize 40} & \cellcolor{green!0}{\large 0}/{\footnotesize 40} & \cellcolor{green!0}{\large 0}/{\footnotesize 40} & \cellcolor{green!10}{\large 2}/{\footnotesize 157} & \cellcolor{green!0}{\large 1}/{\footnotesize 123} \tabularnewline
\rowcolor{black!10} 37 & igraph\_igraph\_read\_graph\_graphdb & 6 & \cellcolor{green!0}{\large 0}/{\footnotesize 40} & \cellcolor{green!0}{\large 0}/{\footnotesize 40} & \cellcolor{green!0}{\large 0}/{\footnotesize 40} & \cellcolor{green!0}{\large 0}/{\footnotesize 40} & \cellcolor{green!0}{\large 0}/{\footnotesize 137} & \cellcolor{green!0}{\large 0}/{\footnotesize 95} \tabularnewline
38 & igraph\_igraph\_read\_graph\_graphml & 6 & \cellcolor{green!0}{\large 0}/{\footnotesize 40} & \cellcolor{green!0}{\large 0}/{\footnotesize 40} & \cellcolor{green!0}{\large 0}/{\footnotesize 40} & \cellcolor{green!10}{\large 1}/{\footnotesize 40} & \cellcolor{green!0}{\large 0}/{\footnotesize 141} & \cellcolor{green!10}{\large 3}/{\footnotesize 94} \tabularnewline
\rowcolor{black!10} 39 & igraph\_igraph\_read\_graph\_lgl & 6 & \cellcolor{green!0}{\large 0}/{\footnotesize 40} & \cellcolor{green!0}{\large 0}/{\footnotesize 40} & \cellcolor{green!0}{\large 0}/{\footnotesize 40} & \cellcolor{green!0}{\large 0}/{\footnotesize 40} & \cellcolor{green!0}{\large 0}/{\footnotesize 155} & \cellcolor{green!10}{\large 2}/{\footnotesize 104} \tabularnewline
40 & igraph\_igraph\_read\_graph\_pajek & 6 & \cellcolor{green!0}{\large 0}/{\footnotesize 40} & \cellcolor{green!0}{\large 0}/{\footnotesize 40} & \cellcolor{green!0}{\large 0}/{\footnotesize 40} & \cellcolor{green!0}{\large 0}/{\footnotesize 40} & \cellcolor{green!0}{\large 0}/{\footnotesize 124} & \cellcolor{green!0}{\large 0}/{\footnotesize 108} \tabularnewline
\rowcolor{black!10} 41 & inchi\_GetINCHIfromINCHI & 6 & \cellcolor{green!0}{\large 0}/{\footnotesize 40} & \cellcolor{green!10}{\large 1}/{\footnotesize 40} & \cellcolor{green!10}{\large 2}/{\footnotesize 40} & \cellcolor{green!20}{\large 5}/{\footnotesize 40} & \cellcolor{green!0}{\large 1}/{\footnotesize 147} & \cellcolor{green!10}{\large 2}/{\footnotesize 109} \tabularnewline
42 & inchi\_GetStructFromINCHI & 6 & \cellcolor{green!0}{\large 0}/{\footnotesize 40} & \cellcolor{green!0}{\large 0}/{\footnotesize 40} & \cellcolor{green!0}{\large 0}/{\footnotesize 40} & \cellcolor{green!10}{\large 2}/{\footnotesize 40} & \cellcolor{green!0}{\large 0}/{\footnotesize 125} & \cellcolor{green!10}{\large 2}/{\footnotesize 106} \tabularnewline
\rowcolor{black!10} 43 & kamailio\_parse\_msg & 6 & \cellcolor{green!0}{\large 0}/{\footnotesize 40} & \cellcolor{green!30}{\large 9}/{\footnotesize 40} & \cellcolor{green!0}{\large -}{\tiny -} & \cellcolor{green!40}{\large 14}/{\footnotesize 40} & \cellcolor{green!20}{\large 15}/{\footnotesize 89} & \cellcolor{green!30}{\large 17}/{\footnotesize 73} \tabularnewline
44 & libyang\_lys\_parse\_mem & 6 & \cellcolor{green!0}{\large 0}/{\footnotesize 40} & \cellcolor{green!0}{\large 0}/{\footnotesize 40} & \cellcolor{green!0}{\large 0}/{\footnotesize 40} & \cellcolor{green!0}{\large 0}/{\footnotesize 40} & \cellcolor{green!0}{\large 0}/{\footnotesize 162} & \cellcolor{green!0}{\large 0}/{\footnotesize 115} \tabularnewline
\rowcolor{black!10} 45 & proftpd\_pr\_json\_object\_from\_text & 6 & \cellcolor{green!0}{\large 0}/{\footnotesize 40} & \cellcolor{green!0}{\large 0}/{\footnotesize 40} & \cellcolor{green!0}{\large -}{\tiny -} & \cellcolor{green!10}{\large 1}/{\footnotesize 40} & \cellcolor{green!0}{\large 0}/{\footnotesize 131} & \cellcolor{green!0}{\large 0}/{\footnotesize 130} \tabularnewline
46 & selinux\_policydb\_read & 6 & \cellcolor{green!0}{\large 0}/{\footnotesize 40} & \cellcolor{green!10}{\large 1}/{\footnotesize 40} & \cellcolor{green!0}{\large -}{\tiny -} & \cellcolor{green!10}{\large 3}/{\footnotesize 40} & \cellcolor{green!0}{\large 1}/{\footnotesize 137} & \cellcolor{green!10}{\large 5}/{\footnotesize 99} \tabularnewline
\rowcolor{black!10} 47 & kamailio\_get\_src\_address\_socket & 7 & \cellcolor{green!0}{\large 0}/{\footnotesize 40} & \cellcolor{green!0}{\large 0}/{\footnotesize 40} & \cellcolor{green!0}{\large 0}/{\footnotesize 40} & \cellcolor{green!10}{\large 1}/{\footnotesize 40} & \cellcolor{green!0}{\large 0}/{\footnotesize 149} & \cellcolor{green!10}{\large 2}/{\footnotesize 101} \tabularnewline
48 & kamailio\_get\_src\_uri & 7 & \cellcolor{green!0}{\large 0}/{\footnotesize 40} & \cellcolor{green!10}{\large 1}/{\footnotesize 40} & \cellcolor{green!0}{\large 0}/{\footnotesize 40} & \cellcolor{green!10}{\large 3}/{\footnotesize 40} & \cellcolor{green!0}{\large 0}/{\footnotesize 130} & \cellcolor{green!10}{\large 5}/{\footnotesize 102} \tabularnewline
\rowcolor{black!10} 49 & kamailio\_parse\_content\_disposition & 7 & \cellcolor{green!0}{\large 0}/{\footnotesize 40} & \cellcolor{green!0}{\large 0}/{\footnotesize 40} & \cellcolor{green!0}{\large 0}/{\footnotesize 40} & \cellcolor{green!10}{\large 1}/{\footnotesize 40} & \cellcolor{green!0}{\large 0}/{\footnotesize 139} & \cellcolor{green!10}{\large 4}/{\footnotesize 89} \tabularnewline
50 & kamailio\_parse\_diversion\_header & 7 & \cellcolor{green!0}{\large 0}/{\footnotesize 40} & \cellcolor{green!0}{\large 0}/{\footnotesize 40} & \cellcolor{green!0}{\large 0}/{\footnotesize 40} & \cellcolor{green!10}{\large 4}/{\footnotesize 40} & \cellcolor{green!0}{\large 0}/{\footnotesize 124} & \cellcolor{green!0}{\large 1}/{\footnotesize 119} \tabularnewline
\rowcolor{black!10} 51 & kamailio\_parse\_from\_header & 7 & \cellcolor{green!0}{\large 0}/{\footnotesize 40} & \cellcolor{green!0}{\large 0}/{\footnotesize 40} & \cellcolor{green!0}{\large -}{\tiny -} & \cellcolor{green!0}{\large 0}/{\footnotesize 40} & \cellcolor{green!0}{\large 0}/{\footnotesize 137} & \cellcolor{green!10}{\large 1}/{\footnotesize 86} \tabularnewline
52 & kamailio\_parse\_from\_uri & 7 & \cellcolor{green!0}{\large 0}/{\footnotesize 40} & \cellcolor{green!0}{\large 0}/{\footnotesize 40} & \cellcolor{green!0}{\large -}{\tiny -} & \cellcolor{green!0}{\large 0}/{\footnotesize 40} & \cellcolor{green!0}{\large 0}/{\footnotesize 139} & \cellcolor{green!10}{\large 2}/{\footnotesize 93} \tabularnewline
\rowcolor{black!10} 53 & kamailio\_parse\_headers & 7 & \cellcolor{green!0}{\large 0}/{\footnotesize 40} & \cellcolor{green!0}{\large 0}/{\footnotesize 40} & \cellcolor{green!0}{\large -}{\tiny -} & \cellcolor{green!0}{\large 0}/{\footnotesize 40} & \cellcolor{green!0}{\large 0}/{\footnotesize 124} & \cellcolor{green!0}{\large 0}/{\footnotesize 79} \tabularnewline
54 & kamailio\_parse\_identityinfo\_header & 7 & \cellcolor{green!0}{\large 0}/{\footnotesize 40} & \cellcolor{green!0}{\large 0}/{\footnotesize 40} & \cellcolor{green!0}{\large -}{\tiny -} & \cellcolor{green!20}{\large 6}/{\footnotesize 40} & \cellcolor{green!0}{\large 0}/{\footnotesize 140} & \cellcolor{green!10}{\large 4}/{\footnotesize 78} \tabularnewline
\rowcolor{black!10} 55 & kamailio\_parse\_pai\_header & 7 & \cellcolor{green!0}{\large 0}/{\footnotesize 40} & \cellcolor{green!0}{\large 0}/{\footnotesize 40} & \cellcolor{green!0}{\large -}{\tiny -} & \cellcolor{green!10}{\large 3}/{\footnotesize 40} & \cellcolor{green!0}{\large 0}/{\footnotesize 128} & \cellcolor{green!10}{\large 1}/{\footnotesize 99} \tabularnewline
56 & kamailio\_parse\_privacy & 7 & \cellcolor{green!0}{\large 0}/{\footnotesize 40} & \cellcolor{green!0}{\large 0}/{\footnotesize 40} & \cellcolor{green!0}{\large 0}/{\footnotesize 40} & \cellcolor{green!10}{\large 3}/{\footnotesize 40} & \cellcolor{green!0}{\large 0}/{\footnotesize 119} & \cellcolor{green!10}{\large 2}/{\footnotesize 94} \tabularnewline
\rowcolor{black!10} 57 & kamailio\_parse\_record\_route\_headers & 7 & \cellcolor{green!0}{\large 0}/{\footnotesize 40} & \cellcolor{green!0}{\large 0}/{\footnotesize 40} & \cellcolor{green!0}{\large -}{\tiny -} & \cellcolor{green!40}{\large 14}/{\footnotesize 40} & \cellcolor{green!0}{\large 1}/{\footnotesize 142} & \cellcolor{green!10}{\large 8}/{\footnotesize 89} \tabularnewline
58 & kamailio\_parse\_refer\_to\_header & 7 & \cellcolor{green!0}{\large 0}/{\footnotesize 40} & \cellcolor{green!0}{\large 0}/{\footnotesize 40} & \cellcolor{green!0}{\large -}{\tiny -} & \cellcolor{green!10}{\large 2}/{\footnotesize 40} & \cellcolor{green!0}{\large 0}/{\footnotesize 134} & \cellcolor{green!10}{\large 2}/{\footnotesize 103} \tabularnewline
\rowcolor{black!10} 59 & kamailio\_parse\_route\_headers & 7 & \cellcolor{green!0}{\large 0}/{\footnotesize 40} & \cellcolor{green!0}{\large 0}/{\footnotesize 40} & \cellcolor{green!0}{\large -}{\tiny -} & \cellcolor{green!20}{\large 8}/{\footnotesize 40} & \cellcolor{green!0}{\large 0}/{\footnotesize 135} & \cellcolor{green!20}{\large 10}/{\footnotesize 87} \tabularnewline
60 & kamailio\_parse\_to\_header & 7 & \cellcolor{green!0}{\large 0}/{\footnotesize 40} & \cellcolor{green!0}{\large 0}/{\footnotesize 40} & \cellcolor{green!0}{\large -}{\tiny -} & \cellcolor{green!0}{\large 0}/{\footnotesize 40} & \cellcolor{green!0}{\large 0}/{\footnotesize 135} & \cellcolor{green!10}{\large 1}/{\footnotesize 84} \tabularnewline
\rowcolor{black!10} 61 & kamailio\_parse\_to\_uri & 7 & \cellcolor{green!0}{\large 0}/{\footnotesize 40} & \cellcolor{green!0}{\large 0}/{\footnotesize 40} & \cellcolor{green!0}{\large -}{\tiny -} & \cellcolor{green!10}{\large 1}/{\footnotesize 40} & \cellcolor{green!0}{\large 1}/{\footnotesize 146} & \cellcolor{green!0}{\large 1}/{\footnotesize 107} \tabularnewline
62 & libyang\_lyd\_parse\_data\_mem & 7 & \cellcolor{green!0}{\large 0}/{\footnotesize 40} & \cellcolor{green!0}{\large 0}/{\footnotesize 40} & \cellcolor{green!0}{\large 0}/{\footnotesize 40} & \cellcolor{green!0}{\large 0}/{\footnotesize 40} & \cellcolor{green!10}{\large 2}/{\footnotesize 145} & \cellcolor{green!0}{\large 0}/{\footnotesize 98} \tabularnewline
\rowcolor{black!10} 63 & bind9\_dns\_message\_parse & 8 & \cellcolor{green!0}{\large 0}/{\footnotesize 40} & \cellcolor{green!0}{\large 0}/{\footnotesize 40} & \cellcolor{green!0}{\large -}{\tiny -} & \cellcolor{green!0}{\large 0}/{\footnotesize 40} & \cellcolor{green!0}{\large 0}/{\footnotesize 159} & \cellcolor{green!0}{\large 0}/{\footnotesize 110} \tabularnewline
64 & igraph\_igraph\_read\_graph\_ncol & 8 & \cellcolor{green!0}{\large 0}/{\footnotesize 40} & \cellcolor{green!0}{\large 0}/{\footnotesize 40} & \cellcolor{green!0}{\large 0}/{\footnotesize 40} & \cellcolor{green!0}{\large 0}/{\footnotesize 40} & \cellcolor{green!0}{\large 0}/{\footnotesize 149} & \cellcolor{green!0}{\large 0}/{\footnotesize 91} \tabularnewline
\rowcolor{black!10} 65 & pjsip\_pj\_json\_parse & 8 & \cellcolor{green!0}{\large 0}/{\footnotesize 40} & \cellcolor{green!0}{\large 0}/{\footnotesize 40} & \cellcolor{green!0}{\large 0}/{\footnotesize 40} & \cellcolor{green!0}{\large 0}/{\footnotesize 40} & \cellcolor{green!0}{\large 0}/{\footnotesize 167} & \cellcolor{green!0}{\large 0}/{\footnotesize 116} \tabularnewline
66 & pjsip\_pj\_xml\_parse & 8 & \cellcolor{green!0}{\large 0}/{\footnotesize 40} & \cellcolor{green!0}{\large 0}/{\footnotesize 40} & \cellcolor{green!0}{\large 0}/{\footnotesize 40} & \cellcolor{green!0}{\large 0}/{\footnotesize 40} & \cellcolor{green!0}{\large 0}/{\footnotesize 160} & \cellcolor{green!0}{\large 0}/{\footnotesize 112} \tabularnewline
\rowcolor{black!10} 67 & pjsip\_pjmedia\_sdp\_parse & 8 & \cellcolor{green!0}{\large 0}/{\footnotesize 40} & \cellcolor{green!0}{\large 0}/{\footnotesize 40} & \cellcolor{green!0}{\large 0}/{\footnotesize 40} & \cellcolor{green!0}{\large 0}/{\footnotesize 40} & \cellcolor{green!0}{\large 0}/{\footnotesize 157} & \cellcolor{green!0}{\large 0}/{\footnotesize 106} \tabularnewline
68 & quickjs\_lre\_compile & 8 & \cellcolor{green!0}{\large 0}/{\footnotesize 40} & \cellcolor{green!0}{\large 0}/{\footnotesize 40} & \cellcolor{green!0}{\large -}{\tiny -} & \cellcolor{green!0}{\large 0}/{\footnotesize 40} & \cellcolor{green!0}{\large 0}/{\footnotesize 146} & \cellcolor{green!0}{\large 0}/{\footnotesize 91} \tabularnewline
\rowcolor{black!10} 69 & bind9\_isc\_lex\_getmastertoken & 9 & \cellcolor{green!0}{\large 0}/{\footnotesize 40} & \cellcolor{green!0}{\large 0}/{\footnotesize 40} & \cellcolor{green!0}{\large -}{\tiny -} & \cellcolor{green!0}{\large 0}/{\footnotesize 40} & \cellcolor{green!0}{\large 0}/{\footnotesize 113} & \cellcolor{green!0}{\large 0}/{\footnotesize 100} \tabularnewline
70 & bind9\_isc\_lex\_gettoken & 9 & \cellcolor{green!0}{\large 0}/{\footnotesize 40} & \cellcolor{green!0}{\large 0}/{\footnotesize 40} & \cellcolor{green!0}{\large -}{\tiny -} & \cellcolor{green!0}{\large 0}/{\footnotesize 40} & \cellcolor{green!0}{\large 0}/{\footnotesize 146} & \cellcolor{green!0}{\large 0}/{\footnotesize 92} \tabularnewline
\rowcolor{black!10} 71 & quickjs\_JS\_Eval & 9 & \cellcolor{green!0}{\large 0}/{\footnotesize 40} & \cellcolor{green!10}{\large 4}/{\footnotesize 40} & \cellcolor{green!0}{\large -}{\tiny -} & \cellcolor{green!10}{\large 1}/{\footnotesize 40} & \cellcolor{green!10}{\large 12}/{\footnotesize 130} & \cellcolor{green!10}{\large 5}/{\footnotesize 93} \tabularnewline
72 & igraph\_igraph\_edge\_connectivity & 10 & \cellcolor{green!0}{\large 0}/{\footnotesize 40} & \cellcolor{green!0}{\large 0}/{\footnotesize 40} & \cellcolor{green!0}{\large 0}/{\footnotesize 40} & \cellcolor{green!0}{\large 0}/{\footnotesize 40} & \cellcolor{green!0}{\large 0}/{\footnotesize 120} & \cellcolor{green!0}{\large 0}/{\footnotesize 101} \tabularnewline
\rowcolor{black!10} 73 & pjsip\_pj\_stun\_msg\_decode & 10 & \cellcolor{green!0}{\large 0}/{\footnotesize 40} & \cellcolor{green!0}{\large 0}/{\footnotesize 40} & \cellcolor{green!0}{\large 0}/{\footnotesize 40} & \cellcolor{green!0}{\large 0}/{\footnotesize 40} & \cellcolor{green!0}{\large 0}/{\footnotesize 165} & \cellcolor{green!0}{\large 0}/{\footnotesize 74} \tabularnewline
74 & bind9\_dns\_message\_checksig & 11 & \cellcolor{green!0}{\large 0}/{\footnotesize 40} & \cellcolor{green!0}{\large 0}/{\footnotesize 40} & \cellcolor{green!0}{\large -}{\tiny -} & \cellcolor{green!0}{\large 0}/{\footnotesize 40} & \cellcolor{green!0}{\large 0}/{\footnotesize 122} & \cellcolor{green!0}{\large 0}/{\footnotesize 88} \tabularnewline
\rowcolor{black!10} 75 & libzip\_zip\_fread & 11 & \cellcolor{green!0}{\large 0}/{\footnotesize 40} & \cellcolor{green!0}{\large 0}/{\footnotesize 40} & \cellcolor{green!0}{\large 0}/{\footnotesize 40} & \cellcolor{green!0}{\large 0}/{\footnotesize 40} & \cellcolor{green!10}{\large 4}/{\footnotesize 122} & \cellcolor{green!10}{\large 3}/{\footnotesize 86} \tabularnewline
76 & bind9\_dns\_rdata\_fromtext & 12 & \cellcolor{green!0}{\large 0}/{\footnotesize 40} & \cellcolor{green!0}{\large 0}/{\footnotesize 40} & \cellcolor{green!0}{\large -}{\tiny -} & \cellcolor{green!0}{\large 0}/{\footnotesize 40} & \cellcolor{green!0}{\large 0}/{\footnotesize 112} & \cellcolor{green!0}{\large 0}/{\footnotesize 61} \tabularnewline
\rowcolor{black!10} 77 & igraph\_igraph\_all\_minimal\_st\_separators & 12 & \cellcolor{green!0}{\large 0}/{\footnotesize 40} & \cellcolor{green!0}{\large 0}/{\footnotesize 40} & \cellcolor{green!0}{\large 0}/{\footnotesize 40} & \cellcolor{green!0}{\large 0}/{\footnotesize 40} & \cellcolor{green!0}{\large 0}/{\footnotesize 104} & \cellcolor{green!0}{\large 0}/{\footnotesize 88} \tabularnewline
78 & igraph\_igraph\_minimum\_size\_separators & 12 & \cellcolor{green!0}{\large 0}/{\footnotesize 40} & \cellcolor{green!0}{\large 0}/{\footnotesize 40} & \cellcolor{green!0}{\large 0}/{\footnotesize 40} & \cellcolor{green!10}{\large 1}/{\footnotesize 40} & \cellcolor{green!0}{\large 0}/{\footnotesize 110} & \cellcolor{green!10}{\large 1}/{\footnotesize 77} \tabularnewline
\rowcolor{black!10} 79 & pjsip\_pjsip\_parse\_msg & 12 & \cellcolor{green!0}{\large 0}/{\footnotesize 40} & \cellcolor{green!0}{\large 0}/{\footnotesize 40} & \cellcolor{green!0}{\large 0}/{\footnotesize 40} & \cellcolor{green!0}{\large 0}/{\footnotesize 40} & \cellcolor{green!0}{\large 0}/{\footnotesize 159} & \cellcolor{green!0}{\large 0}/{\footnotesize 90} \tabularnewline
80 & igraph\_igraph\_automorphism\_group & 13 & \cellcolor{green!0}{\large 0}/{\footnotesize 40} & \cellcolor{green!0}{\large 0}/{\footnotesize 40} & \cellcolor{green!0}{\large 0}/{\footnotesize 40} & \cellcolor{green!10}{\large 1}/{\footnotesize 40} & \cellcolor{green!0}{\large 0}/{\footnotesize 94} & \cellcolor{green!10}{\large 3}/{\footnotesize 69} \tabularnewline
\rowcolor{black!10} 81 & libmodbus\_modbus\_read\_bits & 15 & \cellcolor{green!0}{\large 0}/{\footnotesize 40} & \cellcolor{green!0}{\large 0}/{\footnotesize 40} & \cellcolor{green!0}{\large 0}/{\footnotesize 40} & \cellcolor{green!0}{\large 0}/{\footnotesize 40} & \cellcolor{green!0}{\large 0}/{\footnotesize 122} & \cellcolor{green!0}{\large 0}/{\footnotesize 78} \tabularnewline
82 & libmodbus\_modbus\_read\_registers & 15 & \cellcolor{green!0}{\large 0}/{\footnotesize 40} & \cellcolor{green!0}{\large 0}/{\footnotesize 40} & \cellcolor{green!0}{\large 0}/{\footnotesize 40} & \cellcolor{green!0}{\large 0}/{\footnotesize 40} & \cellcolor{green!0}{\large 0}/{\footnotesize 117} & \cellcolor{green!0}{\large 0}/{\footnotesize 86} \tabularnewline
\rowcolor{black!10} 83 & civetweb\_mg\_get\_response & 17 & \cellcolor{green!0}{\large 0}/{\footnotesize 40} & \cellcolor{green!0}{\large 0}/{\footnotesize 40} & \cellcolor{green!0}{\large 0}/{\footnotesize 40} & \cellcolor{green!0}{\large 0}/{\footnotesize 40} & \cellcolor{green!0}{\large 0}/{\footnotesize 127} & \cellcolor{green!0}{\large 0}/{\footnotesize 74} \tabularnewline
84 & bind9\_dns\_master\_loadbuffer & 20 & \cellcolor{green!0}{\large 0}/{\footnotesize 40} & \cellcolor{green!0}{\large 0}/{\footnotesize 40} & \cellcolor{green!0}{\large -}{\tiny -} & \cellcolor{green!0}{\large 0}/{\footnotesize 40} & \cellcolor{green!0}{\large 0}/{\footnotesize 123} & \cellcolor{green!0}{\large 0}/{\footnotesize 99} \tabularnewline
\rowcolor{black!10} 85 & libmodbus\_modbus\_receive & 33 & \cellcolor{green!0}{\large 0}/{\footnotesize 40} & \cellcolor{green!0}{\large 0}/{\footnotesize 40} & \cellcolor{green!0}{\large 0}/{\footnotesize 40} & \cellcolor{green!0}{\large 0}/{\footnotesize 40} & \cellcolor{green!0}{\large 0}/{\footnotesize 97} & \cellcolor{green!0}{\large 0}/{\footnotesize 83} \tabularnewline
86 & tmux\_input\_parse\_buffer & 42 & \cellcolor{green!0}{\large 0}/{\footnotesize 40} & \cellcolor{green!0}{\large 0}/{\footnotesize 40} & \cellcolor{green!0}{\large -}{\tiny -} & \cellcolor{green!0}{\large 0}/{\footnotesize 40} & \cellcolor{green!0}{\large 0}/{\footnotesize 144} & \cellcolor{green!0}{\large 0}/{\footnotesize 114} \tabularnewline

\bottomrule
%\end{tabular}
%}
%\end{table*}
\end{xltabular}
}
\twocolumn



% model: wizardcoder-15b-v1.0, temp: 1.5

\onecolumn
{\small %
\begin{xltabular}[h]{\textwidth}{ccccccccc}
%\begin{table*}[!t]
%\centering
\caption{Evaluation Result of model wizardcoder-15b-v1.0 with temperature 1.5.} \\
%\resizebox{1.0\linewidth}{!}{
%\begin{tabular}{cccccccccc}
\toprule
Index & Question & Score & NAIVE-40 & BACTX-40 & DOCTX-40 & UGCTX-40 & BA-ITER-40 & ALL-ITER-40 \tabularnewline
\midrule
\rowcolor{black!10} 1 & coturn\_stun\_is\_command\_message\_full\_check\_str & 1 & \cellcolor{green!0}{\large 0}/{\footnotesize 40} & \cellcolor{green!20}{\large 5}/{\footnotesize 40} & \cellcolor{green!0}{\large -}{\tiny -} & \cellcolor{green!10}{\large 1}/{\footnotesize 40} & \cellcolor{green!10}{\large 1}/{\footnotesize 55} & \cellcolor{green!10}{\large 2}/{\footnotesize 44} \tabularnewline
2 & kamailio\_parse\_uri & 1 & \cellcolor{green!0}{\large 0}/{\footnotesize 40} & \cellcolor{green!20}{\large 6}/{\footnotesize 40} & \cellcolor{green!0}{\large -}{\tiny -} & \cellcolor{green!10}{\large 4}/{\footnotesize 40} & \cellcolor{green!10}{\large 3}/{\footnotesize 59} & \cellcolor{green!10}{\large 2}/{\footnotesize 48} \tabularnewline
\rowcolor{black!10} 3 & coturn\_stun\_check\_message\_integrity\_str & 2 & \cellcolor{green!0}{\large 0}/{\footnotesize 40} & \cellcolor{green!0}{\large 0}/{\footnotesize 40} & \cellcolor{green!0}{\large -}{\tiny -} & \cellcolor{green!0}{\large 0}/{\footnotesize 40} & \cellcolor{green!0}{\large 0}/{\footnotesize 55} & \cellcolor{green!0}{\large 0}/{\footnotesize 42} \tabularnewline
4 & libiec61850\_MmsValue\_decodeMmsData & 2 & \cellcolor{green!0}{\large 0}/{\footnotesize 40} & \cellcolor{green!10}{\large 2}/{\footnotesize 40} & \cellcolor{green!10}{\large 2}/{\footnotesize 40} & \cellcolor{green!10}{\large 2}/{\footnotesize 40} & \cellcolor{green!10}{\large 1}/{\footnotesize 58} & \cellcolor{green!10}{\large 2}/{\footnotesize 49} \tabularnewline
\rowcolor{black!10} 5 & md4c\_md\_html & 2 & \cellcolor{green!0}{\large 0}/{\footnotesize 40} & \cellcolor{green!0}{\large 0}/{\footnotesize 40} & \cellcolor{green!0}{\large 0}/{\footnotesize 40} & \cellcolor{green!0}{\large 0}/{\footnotesize 40} & \cellcolor{green!10}{\large 1}/{\footnotesize 64} & \cellcolor{green!0}{\large 0}/{\footnotesize 59} \tabularnewline
6 & spdk\_spdk\_json\_parse & 2 & \cellcolor{green!0}{\large 0}/{\footnotesize 40} & \cellcolor{green!10}{\large 1}/{\footnotesize 40} & \cellcolor{green!0}{\large -}{\tiny -} & \cellcolor{green!0}{\large 0}/{\footnotesize 40} & \cellcolor{green!10}{\large 2}/{\footnotesize 66} & \cellcolor{green!0}{\large 0}/{\footnotesize 53} \tabularnewline
\rowcolor{black!10} 7 & croaring\_roaring\_bitmap\_portable\_deserialize\_safe & 3 & \cellcolor{green!0}{\large 0}/{\footnotesize 40} & \cellcolor{green!10}{\large 3}/{\footnotesize 40} & \cellcolor{green!10}{\large 2}/{\footnotesize 40} & \cellcolor{green!10}{\large 4}/{\footnotesize 40} & \cellcolor{green!10}{\large 3}/{\footnotesize 56} & \cellcolor{green!10}{\large 1}/{\footnotesize 49} \tabularnewline
8 & lua\_luaL\_loadbufferx & 3 & \cellcolor{green!10}{\large 1}/{\footnotesize 40} & \cellcolor{green!20}{\large 6}/{\footnotesize 40} & \cellcolor{green!20}{\large 7}/{\footnotesize 40} & \cellcolor{green!10}{\large 3}/{\footnotesize 40} & \cellcolor{green!20}{\large 7}/{\footnotesize 57} & \cellcolor{green!0}{\large 0}/{\footnotesize 44} \tabularnewline
\rowcolor{black!10} 9 & w3m\_wc\_Str\_conv\_with\_detect & 3 & \cellcolor{green!0}{\large 0}/{\footnotesize 40} & \cellcolor{green!0}{\large 0}/{\footnotesize 40} & \cellcolor{green!0}{\large -}{\tiny -} & \cellcolor{green!0}{\large 0}/{\footnotesize 40} & \cellcolor{green!0}{\large 0}/{\footnotesize 50} & \cellcolor{green!0}{\large 0}/{\footnotesize 48} \tabularnewline
10 & bind9\_dns\_name\_fromwire & 4 & \cellcolor{green!0}{\large 0}/{\footnotesize 40} & \cellcolor{green!0}{\large 0}/{\footnotesize 40} & \cellcolor{green!0}{\large -}{\tiny -} & \cellcolor{green!0}{\large 0}/{\footnotesize 40} & \cellcolor{green!0}{\large 0}/{\footnotesize 54} & \cellcolor{green!0}{\large 0}/{\footnotesize 48} \tabularnewline
\rowcolor{black!10} 11 & gdk-pixbuf\_gdk\_pixbuf\_animation\_new\_from\_file & 4 & \cellcolor{green!0}{\large 0}/{\footnotesize 40} & \cellcolor{green!10}{\large 1}/{\footnotesize 40} & \cellcolor{green!0}{\large 0}/{\footnotesize 40} & \cellcolor{green!0}{\large 0}/{\footnotesize 40} & \cellcolor{green!0}{\large 0}/{\footnotesize 61} & \cellcolor{green!0}{\large 0}/{\footnotesize 51} \tabularnewline
12 & gdk-pixbuf\_gdk\_pixbuf\_new\_from\_data & 4 & \cellcolor{green!0}{\large 0}/{\footnotesize 40} & \cellcolor{green!10}{\large 1}/{\footnotesize 40} & \cellcolor{green!10}{\large 1}/{\footnotesize 40} & \cellcolor{green!10}{\large 3}/{\footnotesize 40} & \cellcolor{green!10}{\large 3}/{\footnotesize 51} & \cellcolor{green!10}{\large 3}/{\footnotesize 46} \tabularnewline
\rowcolor{black!10} 13 & gdk-pixbuf\_gdk\_pixbuf\_new\_from\_file & 4 & \cellcolor{green!0}{\large 0}/{\footnotesize 40} & \cellcolor{green!0}{\large 0}/{\footnotesize 40} & \cellcolor{green!0}{\large 0}/{\footnotesize 40} & \cellcolor{green!0}{\large 0}/{\footnotesize 40} & \cellcolor{green!0}{\large 0}/{\footnotesize 61} & \cellcolor{green!0}{\large 0}/{\footnotesize 53} \tabularnewline
14 & gdk-pixbuf\_gdk\_pixbuf\_new\_from\_stream & 4 & \cellcolor{green!0}{\large 0}/{\footnotesize 40} & \cellcolor{green!20}{\large 5}/{\footnotesize 40} & \cellcolor{green!0}{\large 0}/{\footnotesize 40} & \cellcolor{green!10}{\large 2}/{\footnotesize 40} & \cellcolor{green!10}{\large 1}/{\footnotesize 56} & \cellcolor{green!0}{\large 0}/{\footnotesize 48} \tabularnewline
\rowcolor{black!10} 15 & gpac\_gf\_isom\_open\_file & 4 & \cellcolor{green!0}{\large 0}/{\footnotesize 40} & \cellcolor{green!0}{\large 0}/{\footnotesize 40} & \cellcolor{green!0}{\large -}{\tiny -} & \cellcolor{green!0}{\large 0}/{\footnotesize 40} & \cellcolor{green!0}{\large 0}/{\footnotesize 52} & \cellcolor{green!0}{\large 0}/{\footnotesize 49} \tabularnewline
16 & libbpf\_bpf\_object\_\_open\_mem & 4 & \cellcolor{green!0}{\large 0}/{\footnotesize 40} & \cellcolor{green!10}{\large 3}/{\footnotesize 40} & \cellcolor{green!10}{\large 2}/{\footnotesize 40} & \cellcolor{green!10}{\large 1}/{\footnotesize 40} & \cellcolor{green!0}{\large 0}/{\footnotesize 55} & \cellcolor{green!10}{\large 1}/{\footnotesize 49} \tabularnewline
\rowcolor{black!10} 17 & libpg\_query\_pg\_query\_parse & 4 & \cellcolor{green!0}{\large 0}/{\footnotesize 40} & \cellcolor{green!0}{\large 0}/{\footnotesize 40} & \cellcolor{green!0}{\large -}{\tiny -} & \cellcolor{green!10}{\large 1}/{\footnotesize 40} & \cellcolor{green!10}{\large 1}/{\footnotesize 68} & \cellcolor{green!10}{\large 1}/{\footnotesize 59} \tabularnewline
18 & libucl\_ucl\_parser\_add\_string & 4 & \cellcolor{green!0}{\large 0}/{\footnotesize 40} & \cellcolor{green!10}{\large 1}/{\footnotesize 40} & \cellcolor{green!0}{\large 0}/{\footnotesize 40} & \cellcolor{green!10}{\large 2}/{\footnotesize 40} & \cellcolor{green!0}{\large 0}/{\footnotesize 64} & \cellcolor{green!10}{\large 2}/{\footnotesize 54} \tabularnewline
\rowcolor{black!10} 19 & oniguruma\_onig\_new & 4 & \cellcolor{green!0}{\large 0}/{\footnotesize 40} & \cellcolor{green!0}{\large 0}/{\footnotesize 40} & \cellcolor{green!0}{\large 0}/{\footnotesize 40} & \cellcolor{green!0}{\large 0}/{\footnotesize 40} & \cellcolor{green!0}{\large 0}/{\footnotesize 62} & \cellcolor{green!0}{\large 0}/{\footnotesize 50} \tabularnewline
20 & pupnp\_ixmlLoadDocumentEx & 4 & \cellcolor{green!0}{\large 0}/{\footnotesize 40} & \cellcolor{green!0}{\large 0}/{\footnotesize 40} & \cellcolor{green!0}{\large 0}/{\footnotesize 40} & \cellcolor{green!0}{\large 0}/{\footnotesize 40} & \cellcolor{green!0}{\large 0}/{\footnotesize 68} & \cellcolor{green!0}{\large 0}/{\footnotesize 57} \tabularnewline
\rowcolor{black!10} 21 & gdk-pixbuf\_gdk\_pixbuf\_new\_from\_file\_at\_scale & 5 & \cellcolor{green!0}{\large 0}/{\footnotesize 40} & \cellcolor{green!0}{\large 0}/{\footnotesize 40} & \cellcolor{green!0}{\large 0}/{\footnotesize 40} & \cellcolor{green!0}{\large 0}/{\footnotesize 40} & \cellcolor{green!0}{\large 0}/{\footnotesize 59} & \cellcolor{green!0}{\large 0}/{\footnotesize 49} \tabularnewline
22 & inchi\_GetINCHIKeyFromINCHI & 5 & \cellcolor{green!0}{\large 0}/{\footnotesize 40} & \cellcolor{green!10}{\large 1}/{\footnotesize 40} & \cellcolor{green!0}{\large 0}/{\footnotesize 40} & \cellcolor{green!0}{\large 0}/{\footnotesize 40} & \cellcolor{green!10}{\large 1}/{\footnotesize 52} & \cellcolor{green!0}{\large 0}/{\footnotesize 48} \tabularnewline
\rowcolor{black!10} 23 & libdwarf\_dwarf\_init\_b & 5 & \cellcolor{green!0}{\large 0}/{\footnotesize 40} & \cellcolor{green!0}{\large 0}/{\footnotesize 40} & \cellcolor{green!0}{\large 0}/{\footnotesize 40} & \cellcolor{green!0}{\large 0}/{\footnotesize 40} & \cellcolor{green!0}{\large 0}/{\footnotesize 59} & \cellcolor{green!0}{\large 0}/{\footnotesize 45} \tabularnewline
24 & libdwarf\_dwarf\_init\_path & 5 & \cellcolor{green!0}{\large 0}/{\footnotesize 40} & \cellcolor{green!0}{\large 0}/{\footnotesize 40} & \cellcolor{green!0}{\large 0}/{\footnotesize 40} & \cellcolor{green!0}{\large 0}/{\footnotesize 40} & \cellcolor{green!0}{\large 0}/{\footnotesize 63} & \cellcolor{green!0}{\large 0}/{\footnotesize 54} \tabularnewline
\rowcolor{black!10} 25 & liblouis\_lou\_compileString & 5 & \cellcolor{green!0}{\large 0}/{\footnotesize 40} & \cellcolor{green!0}{\large 0}/{\footnotesize 40} & \cellcolor{green!0}{\large 0}/{\footnotesize 40} & \cellcolor{green!10}{\large 1}/{\footnotesize 40} & \cellcolor{green!10}{\large 3}/{\footnotesize 64} & \cellcolor{green!10}{\large 1}/{\footnotesize 48} \tabularnewline
26 & selinux\_cil\_compile & 5 & \cellcolor{green!0}{\large 0}/{\footnotesize 40} & \cellcolor{green!0}{\large 0}/{\footnotesize 40} & \cellcolor{green!0}{\large -}{\tiny -} & \cellcolor{green!0}{\large 0}/{\footnotesize 40} & \cellcolor{green!0}{\large 0}/{\footnotesize 58} & \cellcolor{green!0}{\large 0}/{\footnotesize 48} \tabularnewline
\rowcolor{black!10} 27 & bind9\_dns\_name\_fromtext & 6 & \cellcolor{green!0}{\large 0}/{\footnotesize 40} & \cellcolor{green!0}{\large 0}/{\footnotesize 40} & \cellcolor{green!0}{\large -}{\tiny -} & \cellcolor{green!0}{\large 0}/{\footnotesize 40} & \cellcolor{green!0}{\large 0}/{\footnotesize 51} & \cellcolor{green!0}{\large 0}/{\footnotesize 46} \tabularnewline
28 & bind9\_dns\_rdata\_fromwire & 6 & \cellcolor{green!0}{\large 0}/{\footnotesize 40} & \cellcolor{green!0}{\large 0}/{\footnotesize 40} & \cellcolor{green!0}{\large -}{\tiny -} & \cellcolor{green!0}{\large 0}/{\footnotesize 40} & \cellcolor{green!0}{\large 0}/{\footnotesize 50} & \cellcolor{green!0}{\large 0}/{\footnotesize 42} \tabularnewline
\rowcolor{black!10} 29 & coturn\_stun\_is\_binding\_response & 6 & \cellcolor{green!0}{\large 0}/{\footnotesize 40} & \cellcolor{green!0}{\large 0}/{\footnotesize 40} & \cellcolor{green!0}{\large -}{\tiny -} & \cellcolor{green!0}{\large 0}/{\footnotesize 40} & \cellcolor{green!0}{\large 0}/{\footnotesize 64} & \cellcolor{green!0}{\large 0}/{\footnotesize 52} \tabularnewline
30 & coturn\_stun\_is\_command\_message & 6 & \cellcolor{green!0}{\large 0}/{\footnotesize 40} & \cellcolor{green!10}{\large 1}/{\footnotesize 40} & \cellcolor{green!0}{\large 0}/{\footnotesize 40} & \cellcolor{green!0}{\large 0}/{\footnotesize 40} & \cellcolor{green!0}{\large 0}/{\footnotesize 62} & \cellcolor{green!0}{\large 0}/{\footnotesize 54} \tabularnewline
\rowcolor{black!10} 31 & coturn\_stun\_is\_response & 6 & \cellcolor{green!0}{\large 0}/{\footnotesize 40} & \cellcolor{green!0}{\large 0}/{\footnotesize 40} & \cellcolor{green!0}{\large -}{\tiny -} & \cellcolor{green!0}{\large 0}/{\footnotesize 40} & \cellcolor{green!0}{\large 0}/{\footnotesize 75} & \cellcolor{green!10}{\large 1}/{\footnotesize 48} \tabularnewline
32 & coturn\_stun\_is\_success\_response & 6 & \cellcolor{green!0}{\large 0}/{\footnotesize 40} & \cellcolor{green!0}{\large 0}/{\footnotesize 40} & \cellcolor{green!0}{\large -}{\tiny -} & \cellcolor{green!0}{\large 0}/{\footnotesize 40} & \cellcolor{green!0}{\large 0}/{\footnotesize 70} & \cellcolor{green!0}{\large 0}/{\footnotesize 44} \tabularnewline
\rowcolor{black!10} 33 & hiredis\_redisFormatCommand & 6 & \cellcolor{green!0}{\large 0}/{\footnotesize 40} & \cellcolor{green!10}{\large 2}/{\footnotesize 40} & \cellcolor{green!0}{\large -}{\tiny -} & \cellcolor{green!10}{\large 1}/{\footnotesize 40} & \cellcolor{green!0}{\large 0}/{\footnotesize 57} & \cellcolor{green!0}{\large 0}/{\footnotesize 60} \tabularnewline
34 & igraph\_igraph\_read\_graph\_dl & 6 & \cellcolor{green!0}{\large 0}/{\footnotesize 40} & \cellcolor{green!0}{\large 0}/{\footnotesize 40} & \cellcolor{green!0}{\large 0}/{\footnotesize 40} & \cellcolor{green!0}{\large 0}/{\footnotesize 40} & \cellcolor{green!0}{\large 0}/{\footnotesize 60} & \cellcolor{green!0}{\large 0}/{\footnotesize 47} \tabularnewline
\rowcolor{black!10} 35 & igraph\_igraph\_read\_graph\_edgelist & 6 & \cellcolor{green!0}{\large 0}/{\footnotesize 40} & \cellcolor{green!0}{\large 0}/{\footnotesize 40} & \cellcolor{green!0}{\large 0}/{\footnotesize 40} & \cellcolor{green!0}{\large 0}/{\footnotesize 40} & \cellcolor{green!0}{\large 0}/{\footnotesize 61} & \cellcolor{green!0}{\large 0}/{\footnotesize 49} \tabularnewline
36 & igraph\_igraph\_read\_graph\_gml & 6 & \cellcolor{green!0}{\large 0}/{\footnotesize 40} & \cellcolor{green!0}{\large 0}/{\footnotesize 40} & \cellcolor{green!0}{\large 0}/{\footnotesize 40} & \cellcolor{green!0}{\large 0}/{\footnotesize 40} & \cellcolor{green!0}{\large 0}/{\footnotesize 60} & \cellcolor{green!0}{\large 0}/{\footnotesize 50} \tabularnewline
\rowcolor{black!10} 37 & igraph\_igraph\_read\_graph\_graphdb & 6 & \cellcolor{green!0}{\large 0}/{\footnotesize 40} & \cellcolor{green!0}{\large 0}/{\footnotesize 40} & \cellcolor{green!0}{\large 0}/{\footnotesize 40} & \cellcolor{green!0}{\large 0}/{\footnotesize 40} & \cellcolor{green!0}{\large 0}/{\footnotesize 61} & \cellcolor{green!0}{\large 0}/{\footnotesize 47} \tabularnewline
38 & igraph\_igraph\_read\_graph\_graphml & 6 & \cellcolor{green!0}{\large 0}/{\footnotesize 40} & \cellcolor{green!0}{\large 0}/{\footnotesize 40} & \cellcolor{green!0}{\large 0}/{\footnotesize 40} & \cellcolor{green!0}{\large 0}/{\footnotesize 40} & \cellcolor{green!0}{\large 0}/{\footnotesize 54} & \cellcolor{green!0}{\large 0}/{\footnotesize 51} \tabularnewline
\rowcolor{black!10} 39 & igraph\_igraph\_read\_graph\_lgl & 6 & \cellcolor{green!0}{\large 0}/{\footnotesize 40} & \cellcolor{green!0}{\large 0}/{\footnotesize 40} & \cellcolor{green!0}{\large 0}/{\footnotesize 40} & \cellcolor{green!0}{\large 0}/{\footnotesize 40} & \cellcolor{green!0}{\large 0}/{\footnotesize 58} & \cellcolor{green!0}{\large 0}/{\footnotesize 49} \tabularnewline
40 & igraph\_igraph\_read\_graph\_pajek & 6 & \cellcolor{green!0}{\large 0}/{\footnotesize 40} & \cellcolor{green!0}{\large 0}/{\footnotesize 40} & \cellcolor{green!0}{\large 0}/{\footnotesize 40} & \cellcolor{green!0}{\large 0}/{\footnotesize 40} & \cellcolor{green!0}{\large 0}/{\footnotesize 59} & \cellcolor{green!0}{\large 0}/{\footnotesize 49} \tabularnewline
\rowcolor{black!10} 41 & inchi\_GetINCHIfromINCHI & 6 & \cellcolor{green!0}{\large 0}/{\footnotesize 40} & \cellcolor{green!0}{\large 0}/{\footnotesize 40} & \cellcolor{green!0}{\large 0}/{\footnotesize 40} & \cellcolor{green!0}{\large 0}/{\footnotesize 40} & \cellcolor{green!0}{\large 0}/{\footnotesize 49} & \cellcolor{green!0}{\large 0}/{\footnotesize 49} \tabularnewline
42 & inchi\_GetStructFromINCHI & 6 & \cellcolor{green!0}{\large 0}/{\footnotesize 40} & \cellcolor{green!0}{\large 0}/{\footnotesize 40} & \cellcolor{green!0}{\large 0}/{\footnotesize 40} & \cellcolor{green!0}{\large 0}/{\footnotesize 40} & \cellcolor{green!0}{\large 0}/{\footnotesize 52} & \cellcolor{green!0}{\large 0}/{\footnotesize 45} \tabularnewline
\rowcolor{black!10} 43 & kamailio\_parse\_msg & 6 & \cellcolor{green!0}{\large 0}/{\footnotesize 40} & \cellcolor{green!10}{\large 3}/{\footnotesize 40} & \cellcolor{green!0}{\large -}{\tiny -} & \cellcolor{green!10}{\large 2}/{\footnotesize 40} & \cellcolor{green!10}{\large 3}/{\footnotesize 69} & \cellcolor{green!0}{\large 0}/{\footnotesize 58} \tabularnewline
44 & libyang\_lys\_parse\_mem & 6 & \cellcolor{green!0}{\large 0}/{\footnotesize 40} & \cellcolor{green!0}{\large 0}/{\footnotesize 40} & \cellcolor{green!0}{\large 0}/{\footnotesize 40} & \cellcolor{green!0}{\large 0}/{\footnotesize 40} & \cellcolor{green!0}{\large 0}/{\footnotesize 53} & \cellcolor{green!0}{\large 0}/{\footnotesize 52} \tabularnewline
\rowcolor{black!10} 45 & proftpd\_pr\_json\_object\_from\_text & 6 & \cellcolor{green!0}{\large 0}/{\footnotesize 40} & \cellcolor{green!0}{\large 0}/{\footnotesize 40} & \cellcolor{green!0}{\large -}{\tiny -} & \cellcolor{green!0}{\large 0}/{\footnotesize 40} & \cellcolor{green!0}{\large 0}/{\footnotesize 56} & \cellcolor{green!0}{\large 0}/{\footnotesize 45} \tabularnewline
46 & selinux\_policydb\_read & 6 & \cellcolor{green!0}{\large 0}/{\footnotesize 40} & \cellcolor{green!0}{\large 0}/{\footnotesize 40} & \cellcolor{green!0}{\large -}{\tiny -} & \cellcolor{green!0}{\large 0}/{\footnotesize 40} & \cellcolor{green!0}{\large 0}/{\footnotesize 51} & \cellcolor{green!0}{\large 0}/{\footnotesize 48} \tabularnewline
\rowcolor{black!10} 47 & kamailio\_get\_src\_address\_socket & 7 & \cellcolor{green!0}{\large 0}/{\footnotesize 40} & \cellcolor{green!0}{\large 0}/{\footnotesize 40} & \cellcolor{green!0}{\large 0}/{\footnotesize 40} & \cellcolor{green!0}{\large 0}/{\footnotesize 40} & \cellcolor{green!0}{\large 0}/{\footnotesize 62} & \cellcolor{green!0}{\large 0}/{\footnotesize 49} \tabularnewline
48 & kamailio\_get\_src\_uri & 7 & \cellcolor{green!0}{\large 0}/{\footnotesize 40} & \cellcolor{green!0}{\large 0}/{\footnotesize 40} & \cellcolor{green!0}{\large 0}/{\footnotesize 40} & \cellcolor{green!0}{\large 0}/{\footnotesize 40} & \cellcolor{green!0}{\large 0}/{\footnotesize 60} & \cellcolor{green!0}{\large 0}/{\footnotesize 42} \tabularnewline
\rowcolor{black!10} 49 & kamailio\_parse\_content\_disposition & 7 & \cellcolor{green!0}{\large 0}/{\footnotesize 40} & \cellcolor{green!0}{\large 0}/{\footnotesize 40} & \cellcolor{green!0}{\large 0}/{\footnotesize 40} & \cellcolor{green!0}{\large 0}/{\footnotesize 40} & \cellcolor{green!0}{\large 0}/{\footnotesize 65} & \cellcolor{green!0}{\large 0}/{\footnotesize 52} \tabularnewline
50 & kamailio\_parse\_diversion\_header & 7 & \cellcolor{green!0}{\large 0}/{\footnotesize 40} & \cellcolor{green!0}{\large 0}/{\footnotesize 40} & \cellcolor{green!0}{\large 0}/{\footnotesize 40} & \cellcolor{green!0}{\large 0}/{\footnotesize 40} & \cellcolor{green!0}{\large 0}/{\footnotesize 74} & \cellcolor{green!0}{\large 0}/{\footnotesize 47} \tabularnewline
\rowcolor{black!10} 51 & kamailio\_parse\_from\_header & 7 & \cellcolor{green!0}{\large 0}/{\footnotesize 40} & \cellcolor{green!10}{\large 1}/{\footnotesize 40} & \cellcolor{green!0}{\large -}{\tiny -} & \cellcolor{green!0}{\large 0}/{\footnotesize 40} & \cellcolor{green!0}{\large 0}/{\footnotesize 65} & \cellcolor{green!0}{\large 0}/{\footnotesize 52} \tabularnewline
52 & kamailio\_parse\_from\_uri & 7 & \cellcolor{green!0}{\large 0}/{\footnotesize 40} & \cellcolor{green!0}{\large 0}/{\footnotesize 40} & \cellcolor{green!0}{\large -}{\tiny -} & \cellcolor{green!0}{\large 0}/{\footnotesize 40} & \cellcolor{green!0}{\large 0}/{\footnotesize 68} & \cellcolor{green!0}{\large 0}/{\footnotesize 50} \tabularnewline
\rowcolor{black!10} 53 & kamailio\_parse\_headers & 7 & \cellcolor{green!0}{\large 0}/{\footnotesize 40} & \cellcolor{green!0}{\large 0}/{\footnotesize 40} & \cellcolor{green!0}{\large -}{\tiny -} & \cellcolor{green!0}{\large 0}/{\footnotesize 40} & \cellcolor{green!0}{\large 0}/{\footnotesize 58} & \cellcolor{green!0}{\large 0}/{\footnotesize 52} \tabularnewline
54 & kamailio\_parse\_identityinfo\_header & 7 & \cellcolor{green!0}{\large 0}/{\footnotesize 40} & \cellcolor{green!0}{\large 0}/{\footnotesize 40} & \cellcolor{green!0}{\large -}{\tiny -} & \cellcolor{green!0}{\large 0}/{\footnotesize 40} & \cellcolor{green!0}{\large 0}/{\footnotesize 62} & \cellcolor{green!0}{\large 0}/{\footnotesize 47} \tabularnewline
\rowcolor{black!10} 55 & kamailio\_parse\_pai\_header & 7 & \cellcolor{green!0}{\large 0}/{\footnotesize 40} & \cellcolor{green!0}{\large 0}/{\footnotesize 40} & \cellcolor{green!0}{\large -}{\tiny -} & \cellcolor{green!0}{\large 0}/{\footnotesize 40} & \cellcolor{green!0}{\large 0}/{\footnotesize 64} & \cellcolor{green!0}{\large 0}/{\footnotesize 49} \tabularnewline
56 & kamailio\_parse\_privacy & 7 & \cellcolor{green!0}{\large 0}/{\footnotesize 40} & \cellcolor{green!0}{\large 0}/{\footnotesize 40} & \cellcolor{green!0}{\large 0}/{\footnotesize 40} & \cellcolor{green!0}{\large 0}/{\footnotesize 40} & \cellcolor{green!0}{\large 0}/{\footnotesize 69} & \cellcolor{green!10}{\large 1}/{\footnotesize 50} \tabularnewline
\rowcolor{black!10} 57 & kamailio\_parse\_record\_route\_headers & 7 & \cellcolor{green!0}{\large 0}/{\footnotesize 40} & \cellcolor{green!0}{\large 0}/{\footnotesize 40} & \cellcolor{green!0}{\large -}{\tiny -} & \cellcolor{green!0}{\large 0}/{\footnotesize 40} & \cellcolor{green!0}{\large 0}/{\footnotesize 69} & \cellcolor{green!0}{\large 0}/{\footnotesize 50} \tabularnewline
58 & kamailio\_parse\_refer\_to\_header & 7 & \cellcolor{green!0}{\large 0}/{\footnotesize 40} & \cellcolor{green!0}{\large 0}/{\footnotesize 40} & \cellcolor{green!0}{\large -}{\tiny -} & \cellcolor{green!0}{\large 0}/{\footnotesize 40} & \cellcolor{green!0}{\large 0}/{\footnotesize 70} & \cellcolor{green!0}{\large 0}/{\footnotesize 51} \tabularnewline
\rowcolor{black!10} 59 & kamailio\_parse\_route\_headers & 7 & \cellcolor{green!0}{\large 0}/{\footnotesize 40} & \cellcolor{green!0}{\large 0}/{\footnotesize 40} & \cellcolor{green!0}{\large -}{\tiny -} & \cellcolor{green!0}{\large 0}/{\footnotesize 40} & \cellcolor{green!0}{\large 0}/{\footnotesize 61} & \cellcolor{green!0}{\large 0}/{\footnotesize 49} \tabularnewline
60 & kamailio\_parse\_to\_header & 7 & \cellcolor{green!0}{\large 0}/{\footnotesize 40} & \cellcolor{green!0}{\large 0}/{\footnotesize 40} & \cellcolor{green!0}{\large -}{\tiny -} & \cellcolor{green!0}{\large 0}/{\footnotesize 40} & \cellcolor{green!0}{\large 0}/{\footnotesize 62} & \cellcolor{green!0}{\large 0}/{\footnotesize 47} \tabularnewline
\rowcolor{black!10} 61 & kamailio\_parse\_to\_uri & 7 & \cellcolor{green!0}{\large 0}/{\footnotesize 40} & \cellcolor{green!0}{\large 0}/{\footnotesize 40} & \cellcolor{green!0}{\large -}{\tiny -} & \cellcolor{green!0}{\large 0}/{\footnotesize 40} & \cellcolor{green!0}{\large 0}/{\footnotesize 54} & \cellcolor{green!0}{\large 0}/{\footnotesize 48} \tabularnewline
62 & libyang\_lyd\_parse\_data\_mem & 7 & \cellcolor{green!0}{\large 0}/{\footnotesize 40} & \cellcolor{green!0}{\large 0}/{\footnotesize 40} & \cellcolor{green!0}{\large 0}/{\footnotesize 40} & \cellcolor{green!0}{\large 0}/{\footnotesize 40} & \cellcolor{green!0}{\large 0}/{\footnotesize 55} & \cellcolor{green!0}{\large 0}/{\footnotesize 47} \tabularnewline
\rowcolor{black!10} 63 & bind9\_dns\_message\_parse & 8 & \cellcolor{green!0}{\large 0}/{\footnotesize 40} & \cellcolor{green!0}{\large 0}/{\footnotesize 40} & \cellcolor{green!0}{\large -}{\tiny -} & \cellcolor{green!0}{\large 0}/{\footnotesize 40} & \cellcolor{green!0}{\large 0}/{\footnotesize 54} & \cellcolor{green!0}{\large 0}/{\footnotesize 48} \tabularnewline
64 & igraph\_igraph\_read\_graph\_ncol & 8 & \cellcolor{green!0}{\large 0}/{\footnotesize 40} & \cellcolor{green!0}{\large 0}/{\footnotesize 40} & \cellcolor{green!0}{\large 0}/{\footnotesize 40} & \cellcolor{green!0}{\large 0}/{\footnotesize 40} & \cellcolor{green!0}{\large 0}/{\footnotesize 51} & \cellcolor{green!0}{\large 0}/{\footnotesize 47} \tabularnewline
\rowcolor{black!10} 65 & pjsip\_pj\_json\_parse & 8 & \cellcolor{green!0}{\large 0}/{\footnotesize 40} & \cellcolor{green!0}{\large 0}/{\footnotesize 40} & \cellcolor{green!0}{\large 0}/{\footnotesize 40} & \cellcolor{green!0}{\large 0}/{\footnotesize 40} & \cellcolor{green!0}{\large 0}/{\footnotesize 55} & \cellcolor{green!0}{\large 0}/{\footnotesize 56} \tabularnewline
66 & pjsip\_pj\_xml\_parse & 8 & \cellcolor{green!0}{\large 0}/{\footnotesize 40} & \cellcolor{green!0}{\large 0}/{\footnotesize 40} & \cellcolor{green!0}{\large 0}/{\footnotesize 40} & \cellcolor{green!0}{\large 0}/{\footnotesize 40} & \cellcolor{green!0}{\large 0}/{\footnotesize 55} & \cellcolor{green!0}{\large 0}/{\footnotesize 50} \tabularnewline
\rowcolor{black!10} 67 & pjsip\_pjmedia\_sdp\_parse & 8 & \cellcolor{green!0}{\large 0}/{\footnotesize 40} & \cellcolor{green!0}{\large 0}/{\footnotesize 40} & \cellcolor{green!0}{\large 0}/{\footnotesize 40} & \cellcolor{green!0}{\large 0}/{\footnotesize 40} & \cellcolor{green!0}{\large 0}/{\footnotesize 57} & \cellcolor{green!0}{\large 0}/{\footnotesize 42} \tabularnewline
68 & quickjs\_lre\_compile & 8 & \cellcolor{green!0}{\large 0}/{\footnotesize 40} & \cellcolor{green!0}{\large 0}/{\footnotesize 40} & \cellcolor{green!0}{\large -}{\tiny -} & \cellcolor{green!0}{\large 0}/{\footnotesize 40} & \cellcolor{green!0}{\large 0}/{\footnotesize 62} & \cellcolor{green!0}{\large 0}/{\footnotesize 52} \tabularnewline
\rowcolor{black!10} 69 & bind9\_isc\_lex\_getmastertoken & 9 & \cellcolor{green!0}{\large 0}/{\footnotesize 40} & \cellcolor{green!0}{\large 0}/{\footnotesize 40} & \cellcolor{green!0}{\large -}{\tiny -} & \cellcolor{green!0}{\large 0}/{\footnotesize 40} & \cellcolor{green!0}{\large 0}/{\footnotesize 51} & \cellcolor{green!0}{\large 0}/{\footnotesize 46} \tabularnewline
70 & bind9\_isc\_lex\_gettoken & 9 & \cellcolor{green!0}{\large 0}/{\footnotesize 40} & \cellcolor{green!0}{\large 0}/{\footnotesize 40} & \cellcolor{green!0}{\large -}{\tiny -} & \cellcolor{green!0}{\large 0}/{\footnotesize 40} & \cellcolor{green!0}{\large 0}/{\footnotesize 57} & \cellcolor{green!0}{\large 0}/{\footnotesize 49} \tabularnewline
\rowcolor{black!10} 71 & quickjs\_JS\_Eval & 9 & \cellcolor{green!0}{\large 0}/{\footnotesize 40} & \cellcolor{green!0}{\large 0}/{\footnotesize 40} & \cellcolor{green!0}{\large -}{\tiny -} & \cellcolor{green!0}{\large 0}/{\footnotesize 40} & \cellcolor{green!0}{\large 0}/{\footnotesize 57} & \cellcolor{green!0}{\large 0}/{\footnotesize 47} \tabularnewline
72 & igraph\_igraph\_edge\_connectivity & 10 & \cellcolor{green!0}{\large 0}/{\footnotesize 40} & \cellcolor{green!0}{\large 0}/{\footnotesize 40} & \cellcolor{green!0}{\large 0}/{\footnotesize 40} & \cellcolor{green!0}{\large 0}/{\footnotesize 40} & \cellcolor{green!0}{\large 0}/{\footnotesize 51} & \cellcolor{green!0}{\large 0}/{\footnotesize 49} \tabularnewline
\rowcolor{black!10} 73 & pjsip\_pj\_stun\_msg\_decode & 10 & \cellcolor{green!0}{\large 0}/{\footnotesize 40} & \cellcolor{green!0}{\large 0}/{\footnotesize 40} & \cellcolor{green!0}{\large 0}/{\footnotesize 40} & \cellcolor{green!0}{\large 0}/{\footnotesize 40} & \cellcolor{green!0}{\large 0}/{\footnotesize 54} & \cellcolor{green!0}{\large 0}/{\footnotesize 44} \tabularnewline
74 & bind9\_dns\_message\_checksig & 11 & \cellcolor{green!0}{\large 0}/{\footnotesize 40} & \cellcolor{green!0}{\large 0}/{\footnotesize 40} & \cellcolor{green!0}{\large -}{\tiny -} & \cellcolor{green!0}{\large 0}/{\footnotesize 40} & \cellcolor{green!0}{\large 0}/{\footnotesize 53} & \cellcolor{green!0}{\large 0}/{\footnotesize 42} \tabularnewline
\rowcolor{black!10} 75 & libzip\_zip\_fread & 11 & \cellcolor{green!0}{\large 0}/{\footnotesize 40} & \cellcolor{green!0}{\large 0}/{\footnotesize 40} & \cellcolor{green!0}{\large 0}/{\footnotesize 40} & \cellcolor{green!0}{\large 0}/{\footnotesize 40} & \cellcolor{green!0}{\large 0}/{\footnotesize 62} & \cellcolor{green!0}{\large 0}/{\footnotesize 44} \tabularnewline
76 & bind9\_dns\_rdata\_fromtext & 12 & \cellcolor{green!0}{\large 0}/{\footnotesize 40} & \cellcolor{green!0}{\large 0}/{\footnotesize 40} & \cellcolor{green!0}{\large -}{\tiny -} & \cellcolor{green!0}{\large 0}/{\footnotesize 40} & \cellcolor{green!0}{\large 0}/{\footnotesize 46} & \cellcolor{green!0}{\large 0}/{\footnotesize 42} \tabularnewline
\rowcolor{black!10} 77 & igraph\_igraph\_all\_minimal\_st\_separators & 12 & \cellcolor{green!0}{\large 0}/{\footnotesize 40} & \cellcolor{green!10}{\large 1}/{\footnotesize 40} & \cellcolor{green!0}{\large 0}/{\footnotesize 40} & \cellcolor{green!0}{\large 0}/{\footnotesize 40} & \cellcolor{green!0}{\large 0}/{\footnotesize 49} & \cellcolor{green!0}{\large 0}/{\footnotesize 48} \tabularnewline
78 & igraph\_igraph\_minimum\_size\_separators & 12 & \cellcolor{green!0}{\large 0}/{\footnotesize 40} & \cellcolor{green!0}{\large 0}/{\footnotesize 40} & \cellcolor{green!0}{\large 0}/{\footnotesize 40} & \cellcolor{green!0}{\large 0}/{\footnotesize 40} & \cellcolor{green!0}{\large 0}/{\footnotesize 47} & \cellcolor{green!0}{\large 0}/{\footnotesize 48} \tabularnewline
\rowcolor{black!10} 79 & pjsip\_pjsip\_parse\_msg & 12 & \cellcolor{green!0}{\large 0}/{\footnotesize 40} & \cellcolor{green!0}{\large 0}/{\footnotesize 40} & \cellcolor{green!0}{\large 0}/{\footnotesize 40} & \cellcolor{green!0}{\large 0}/{\footnotesize 40} & \cellcolor{green!0}{\large 0}/{\footnotesize 51} & \cellcolor{green!0}{\large 0}/{\footnotesize 48} \tabularnewline
80 & igraph\_igraph\_automorphism\_group & 13 & \cellcolor{green!0}{\large 0}/{\footnotesize 40} & \cellcolor{green!0}{\large 0}/{\footnotesize 40} & \cellcolor{green!0}{\large 0}/{\footnotesize 40} & \cellcolor{green!0}{\large 0}/{\footnotesize 40} & \cellcolor{green!0}{\large 0}/{\footnotesize 44} & \cellcolor{green!0}{\large 0}/{\footnotesize 44} \tabularnewline
\rowcolor{black!10} 81 & libmodbus\_modbus\_read\_bits & 15 & \cellcolor{green!0}{\large 0}/{\footnotesize 40} & \cellcolor{green!0}{\large 0}/{\footnotesize 40} & \cellcolor{green!0}{\large 0}/{\footnotesize 40} & \cellcolor{green!0}{\large 0}/{\footnotesize 40} & \cellcolor{green!0}{\large 0}/{\footnotesize 66} & \cellcolor{green!0}{\large 0}/{\footnotesize 45} \tabularnewline
82 & libmodbus\_modbus\_read\_registers & 15 & \cellcolor{green!0}{\large 0}/{\footnotesize 40} & \cellcolor{green!0}{\large 0}/{\footnotesize 40} & \cellcolor{green!0}{\large 0}/{\footnotesize 40} & \cellcolor{green!0}{\large 0}/{\footnotesize 40} & \cellcolor{green!0}{\large 0}/{\footnotesize 55} & \cellcolor{green!0}{\large 0}/{\footnotesize 49} \tabularnewline
\rowcolor{black!10} 83 & civetweb\_mg\_get\_response & 17 & \cellcolor{green!0}{\large 0}/{\footnotesize 40} & \cellcolor{green!0}{\large 0}/{\footnotesize 40} & \cellcolor{green!0}{\large 0}/{\footnotesize 40} & \cellcolor{green!0}{\large 0}/{\footnotesize 40} & \cellcolor{green!0}{\large 0}/{\footnotesize 53} & \cellcolor{green!0}{\large 0}/{\footnotesize 44} \tabularnewline
84 & bind9\_dns\_master\_loadbuffer & 20 & \cellcolor{green!0}{\large 0}/{\footnotesize 40} & \cellcolor{green!0}{\large 0}/{\footnotesize 40} & \cellcolor{green!0}{\large -}{\tiny -} & \cellcolor{green!0}{\large 0}/{\footnotesize 40} & \cellcolor{green!0}{\large 0}/{\footnotesize 50} & \cellcolor{green!0}{\large 0}/{\footnotesize 46} \tabularnewline
\rowcolor{black!10} 85 & libmodbus\_modbus\_receive & 33 & \cellcolor{green!0}{\large 0}/{\footnotesize 40} & \cellcolor{green!0}{\large 0}/{\footnotesize 40} & \cellcolor{green!0}{\large 0}/{\footnotesize 40} & \cellcolor{green!0}{\large 0}/{\footnotesize 40} & \cellcolor{green!0}{\large 0}/{\footnotesize 65} & \cellcolor{green!0}{\large 0}/{\footnotesize 53} \tabularnewline
86 & tmux\_input\_parse\_buffer & 42 & \cellcolor{green!0}{\large 0}/{\footnotesize 40} & \cellcolor{green!0}{\large 0}/{\footnotesize 40} & \cellcolor{green!0}{\large -}{\tiny -} & \cellcolor{green!0}{\large 0}/{\footnotesize 40} & \cellcolor{green!0}{\large 0}/{\footnotesize 56} & \cellcolor{green!0}{\large 0}/{\footnotesize 57} \tabularnewline

\bottomrule
%\end{tabular}
%}
%\end{table*}
\end{xltabular}
}
\twocolumn



% model: wizardcoder-15b-v1.0, temp: 2.0

\onecolumn
{\small %
\begin{xltabular}[h]{\textwidth}{ccccccccc}
%\begin{table*}[!t]
%\centering
\caption{Evaluation Result of model wizardcoder-15b-v1.0 with temperature 2.0.} \\
%\resizebox{1.0\linewidth}{!}{
%\begin{tabular}{cccccccccc}
\toprule
Index & Question & Score & NAIVE-40 & BACTX-40 & DOCTX-40 & UGCTX-40 & BA-ITER-40 & ALL-ITER-40 \tabularnewline
\midrule
\rowcolor{black!10} 1 & coturn\_stun\_is\_command\_message\_full\_check\_str & 1 & \cellcolor{green!0}{\large 0}/{\footnotesize 40} & \cellcolor{green!0}{\large 0}/{\footnotesize 40} & \cellcolor{green!0}{\large -}{\tiny -} & \cellcolor{green!0}{\large 0}/{\footnotesize 40} & \cellcolor{green!0}{\large 0}/{\footnotesize 40} & \cellcolor{green!0}{\large 0}/{\footnotesize 40} \tabularnewline
2 & kamailio\_parse\_uri & 1 & \cellcolor{green!0}{\large 0}/{\footnotesize 40} & \cellcolor{green!0}{\large 0}/{\footnotesize 40} & \cellcolor{green!0}{\large -}{\tiny -} & \cellcolor{green!0}{\large 0}/{\footnotesize 40} & \cellcolor{green!0}{\large 0}/{\footnotesize 40} & \cellcolor{green!0}{\large 0}/{\footnotesize 41} \tabularnewline
\rowcolor{black!10} 3 & coturn\_stun\_check\_message\_integrity\_str & 2 & \cellcolor{green!0}{\large 0}/{\footnotesize 40} & \cellcolor{green!0}{\large 0}/{\footnotesize 40} & \cellcolor{green!0}{\large -}{\tiny -} & \cellcolor{green!0}{\large 0}/{\footnotesize 40} & \cellcolor{green!0}{\large 0}/{\footnotesize 40} & \cellcolor{green!0}{\large 0}/{\footnotesize 40} \tabularnewline
4 & libiec61850\_MmsValue\_decodeMmsData & 2 & \cellcolor{green!0}{\large 0}/{\footnotesize 40} & \cellcolor{green!0}{\large 0}/{\footnotesize 40} & \cellcolor{green!0}{\large 0}/{\footnotesize 40} & \cellcolor{green!0}{\large 0}/{\footnotesize 40} & \cellcolor{green!0}{\large 0}/{\footnotesize 41} & \cellcolor{green!0}{\large 0}/{\footnotesize 41} \tabularnewline
\rowcolor{black!10} 5 & md4c\_md\_html & 2 & \cellcolor{green!0}{\large 0}/{\footnotesize 40} & \cellcolor{green!0}{\large 0}/{\footnotesize 40} & \cellcolor{green!0}{\large 0}/{\footnotesize 40} & \cellcolor{green!0}{\large 0}/{\footnotesize 40} & \cellcolor{green!0}{\large 0}/{\footnotesize 40} & \cellcolor{green!0}{\large 0}/{\footnotesize 41} \tabularnewline
6 & spdk\_spdk\_json\_parse & 2 & \cellcolor{green!0}{\large 0}/{\footnotesize 40} & \cellcolor{green!0}{\large 0}/{\footnotesize 40} & \cellcolor{green!0}{\large -}{\tiny -} & \cellcolor{green!0}{\large 0}/{\footnotesize 40} & \cellcolor{green!0}{\large 0}/{\footnotesize 40} & \cellcolor{green!0}{\large 0}/{\footnotesize 41} \tabularnewline
\rowcolor{black!10} 7 & croaring\_roaring\_bitmap\_portable\_deserialize\_safe & 3 & \cellcolor{green!0}{\large 0}/{\footnotesize 40} & \cellcolor{green!0}{\large 0}/{\footnotesize 40} & \cellcolor{green!0}{\large 0}/{\footnotesize 40} & \cellcolor{green!0}{\large 0}/{\footnotesize 40} & \cellcolor{green!0}{\large 0}/{\footnotesize 42} & \cellcolor{green!0}{\large 0}/{\footnotesize 40} \tabularnewline
8 & lua\_luaL\_loadbufferx & 3 & \cellcolor{green!0}{\large 0}/{\footnotesize 40} & \cellcolor{green!0}{\large 0}/{\footnotesize 40} & \cellcolor{green!0}{\large 0}/{\footnotesize 40} & \cellcolor{green!0}{\large 0}/{\footnotesize 40} & \cellcolor{green!0}{\large 0}/{\footnotesize 40} & \cellcolor{green!0}{\large 0}/{\footnotesize 40} \tabularnewline
\rowcolor{black!10} 9 & w3m\_wc\_Str\_conv\_with\_detect & 3 & \cellcolor{green!0}{\large 0}/{\footnotesize 40} & \cellcolor{green!0}{\large 0}/{\footnotesize 40} & \cellcolor{green!0}{\large -}{\tiny -} & \cellcolor{green!0}{\large 0}/{\footnotesize 40} & \cellcolor{green!0}{\large 0}/{\footnotesize 40} & \cellcolor{green!0}{\large 0}/{\footnotesize 41} \tabularnewline
10 & bind9\_dns\_name\_fromwire & 4 & \cellcolor{green!0}{\large 0}/{\footnotesize 40} & \cellcolor{green!0}{\large 0}/{\footnotesize 40} & \cellcolor{green!0}{\large -}{\tiny -} & \cellcolor{green!0}{\large 0}/{\footnotesize 40} & \cellcolor{green!0}{\large 0}/{\footnotesize 40} & \cellcolor{green!0}{\large 0}/{\footnotesize 40} \tabularnewline
\rowcolor{black!10} 11 & gdk-pixbuf\_gdk\_pixbuf\_animation\_new\_from\_file & 4 & \cellcolor{green!0}{\large 0}/{\footnotesize 40} & \cellcolor{green!0}{\large 0}/{\footnotesize 40} & \cellcolor{green!0}{\large 0}/{\footnotesize 40} & \cellcolor{green!0}{\large 0}/{\footnotesize 40} & \cellcolor{green!0}{\large 0}/{\footnotesize 43} & \cellcolor{green!0}{\large 0}/{\footnotesize 40} \tabularnewline
12 & gdk-pixbuf\_gdk\_pixbuf\_new\_from\_data & 4 & \cellcolor{green!0}{\large 0}/{\footnotesize 40} & \cellcolor{green!0}{\large 0}/{\footnotesize 40} & \cellcolor{green!0}{\large 0}/{\footnotesize 40} & \cellcolor{green!0}{\large 0}/{\footnotesize 40} & \cellcolor{green!0}{\large 0}/{\footnotesize 41} & \cellcolor{green!0}{\large 0}/{\footnotesize 40} \tabularnewline
\rowcolor{black!10} 13 & gdk-pixbuf\_gdk\_pixbuf\_new\_from\_file & 4 & \cellcolor{green!0}{\large 0}/{\footnotesize 40} & \cellcolor{green!0}{\large 0}/{\footnotesize 40} & \cellcolor{green!0}{\large 0}/{\footnotesize 40} & \cellcolor{green!0}{\large 0}/{\footnotesize 40} & \cellcolor{green!0}{\large 0}/{\footnotesize 40} & \cellcolor{green!0}{\large 0}/{\footnotesize 40} \tabularnewline
14 & gdk-pixbuf\_gdk\_pixbuf\_new\_from\_stream & 4 & \cellcolor{green!0}{\large 0}/{\footnotesize 40} & \cellcolor{green!0}{\large 0}/{\footnotesize 40} & \cellcolor{green!0}{\large 0}/{\footnotesize 40} & \cellcolor{green!0}{\large 0}/{\footnotesize 40} & \cellcolor{green!0}{\large 0}/{\footnotesize 43} & \cellcolor{green!0}{\large 0}/{\footnotesize 40} \tabularnewline
\rowcolor{black!10} 15 & gpac\_gf\_isom\_open\_file & 4 & \cellcolor{green!0}{\large 0}/{\footnotesize 40} & \cellcolor{green!0}{\large 0}/{\footnotesize 40} & \cellcolor{green!0}{\large -}{\tiny -} & \cellcolor{green!0}{\large 0}/{\footnotesize 40} & \cellcolor{green!0}{\large 0}/{\footnotesize 40} & \cellcolor{green!0}{\large 0}/{\footnotesize 42} \tabularnewline
16 & libbpf\_bpf\_object\_\_open\_mem & 4 & \cellcolor{green!0}{\large 0}/{\footnotesize 40} & \cellcolor{green!0}{\large 0}/{\footnotesize 40} & \cellcolor{green!0}{\large 0}/{\footnotesize 40} & \cellcolor{green!0}{\large 0}/{\footnotesize 40} & \cellcolor{green!0}{\large 0}/{\footnotesize 40} & \cellcolor{green!0}{\large 0}/{\footnotesize 41} \tabularnewline
\rowcolor{black!10} 17 & libpg\_query\_pg\_query\_parse & 4 & \cellcolor{green!0}{\large 0}/{\footnotesize 40} & \cellcolor{green!0}{\large 0}/{\footnotesize 40} & \cellcolor{green!0}{\large -}{\tiny -} & \cellcolor{green!0}{\large 0}/{\footnotesize 40} & \cellcolor{green!0}{\large 0}/{\footnotesize 40} & \cellcolor{green!0}{\large 0}/{\footnotesize 40} \tabularnewline
18 & libucl\_ucl\_parser\_add\_string & 4 & \cellcolor{green!0}{\large 0}/{\footnotesize 40} & \cellcolor{green!0}{\large 0}/{\footnotesize 40} & \cellcolor{green!0}{\large 0}/{\footnotesize 40} & \cellcolor{green!0}{\large 0}/{\footnotesize 40} & \cellcolor{green!0}{\large 0}/{\footnotesize 40} & \cellcolor{green!0}{\large 0}/{\footnotesize 40} \tabularnewline
\rowcolor{black!10} 19 & oniguruma\_onig\_new & 4 & \cellcolor{green!0}{\large 0}/{\footnotesize 40} & \cellcolor{green!0}{\large 0}/{\footnotesize 40} & \cellcolor{green!0}{\large 0}/{\footnotesize 40} & \cellcolor{green!0}{\large 0}/{\footnotesize 40} & \cellcolor{green!0}{\large 0}/{\footnotesize 40} & \cellcolor{green!0}{\large 0}/{\footnotesize 43} \tabularnewline
20 & pupnp\_ixmlLoadDocumentEx & 4 & \cellcolor{green!0}{\large 0}/{\footnotesize 40} & \cellcolor{green!0}{\large 0}/{\footnotesize 40} & \cellcolor{green!0}{\large 0}/{\footnotesize 40} & \cellcolor{green!0}{\large 0}/{\footnotesize 40} & \cellcolor{green!0}{\large 0}/{\footnotesize 43} & \cellcolor{green!0}{\large 0}/{\footnotesize 42} \tabularnewline
\rowcolor{black!10} 21 & gdk-pixbuf\_gdk\_pixbuf\_new\_from\_file\_at\_scale & 5 & \cellcolor{green!0}{\large 0}/{\footnotesize 40} & \cellcolor{green!0}{\large 0}/{\footnotesize 40} & \cellcolor{green!0}{\large 0}/{\footnotesize 40} & \cellcolor{green!0}{\large 0}/{\footnotesize 40} & \cellcolor{green!0}{\large 0}/{\footnotesize 42} & \cellcolor{green!0}{\large 0}/{\footnotesize 40} \tabularnewline
22 & inchi\_GetINCHIKeyFromINCHI & 5 & \cellcolor{green!0}{\large 0}/{\footnotesize 40} & \cellcolor{green!0}{\large 0}/{\footnotesize 40} & \cellcolor{green!0}{\large 0}/{\footnotesize 40} & \cellcolor{green!0}{\large 0}/{\footnotesize 40} & \cellcolor{green!0}{\large 0}/{\footnotesize 42} & \cellcolor{green!0}{\large 0}/{\footnotesize 40} \tabularnewline
\rowcolor{black!10} 23 & libdwarf\_dwarf\_init\_b & 5 & \cellcolor{green!0}{\large 0}/{\footnotesize 40} & \cellcolor{green!0}{\large 0}/{\footnotesize 40} & \cellcolor{green!0}{\large 0}/{\footnotesize 40} & \cellcolor{green!0}{\large 0}/{\footnotesize 40} & \cellcolor{green!0}{\large 0}/{\footnotesize 44} & \cellcolor{green!0}{\large 0}/{\footnotesize 40} \tabularnewline
24 & libdwarf\_dwarf\_init\_path & 5 & \cellcolor{green!0}{\large 0}/{\footnotesize 40} & \cellcolor{green!0}{\large 0}/{\footnotesize 40} & \cellcolor{green!0}{\large 0}/{\footnotesize 40} & \cellcolor{green!0}{\large 0}/{\footnotesize 40} & \cellcolor{green!0}{\large 0}/{\footnotesize 41} & \cellcolor{green!0}{\large 0}/{\footnotesize 41} \tabularnewline
\rowcolor{black!10} 25 & liblouis\_lou\_compileString & 5 & \cellcolor{green!0}{\large 0}/{\footnotesize 40} & \cellcolor{green!0}{\large 0}/{\footnotesize 40} & \cellcolor{green!0}{\large 0}/{\footnotesize 40} & \cellcolor{green!0}{\large 0}/{\footnotesize 40} & \cellcolor{green!0}{\large 0}/{\footnotesize 40} & \cellcolor{green!0}{\large 0}/{\footnotesize 40} \tabularnewline
26 & selinux\_cil\_compile & 5 & \cellcolor{green!0}{\large 0}/{\footnotesize 40} & \cellcolor{green!0}{\large 0}/{\footnotesize 40} & \cellcolor{green!0}{\large -}{\tiny -} & \cellcolor{green!0}{\large 0}/{\footnotesize 40} & \cellcolor{green!0}{\large 0}/{\footnotesize 40} & \cellcolor{green!0}{\large 0}/{\footnotesize 42} \tabularnewline
\rowcolor{black!10} 27 & bind9\_dns\_name\_fromtext & 6 & \cellcolor{green!0}{\large 0}/{\footnotesize 40} & \cellcolor{green!0}{\large 0}/{\footnotesize 40} & \cellcolor{green!0}{\large -}{\tiny -} & \cellcolor{green!0}{\large 0}/{\footnotesize 40} & \cellcolor{green!0}{\large 0}/{\footnotesize 43} & \cellcolor{green!0}{\large 0}/{\footnotesize 40} \tabularnewline
28 & bind9\_dns\_rdata\_fromwire & 6 & \cellcolor{green!0}{\large 0}/{\footnotesize 40} & \cellcolor{green!0}{\large 0}/{\footnotesize 40} & \cellcolor{green!0}{\large -}{\tiny -} & \cellcolor{green!0}{\large 0}/{\footnotesize 40} & \cellcolor{green!0}{\large 0}/{\footnotesize 41} & \cellcolor{green!0}{\large 0}/{\footnotesize 41} \tabularnewline
\rowcolor{black!10} 29 & coturn\_stun\_is\_binding\_response & 6 & \cellcolor{green!0}{\large 0}/{\footnotesize 40} & \cellcolor{green!0}{\large 0}/{\footnotesize 40} & \cellcolor{green!0}{\large -}{\tiny -} & \cellcolor{green!0}{\large 0}/{\footnotesize 40} & \cellcolor{green!0}{\large 0}/{\footnotesize 41} & \cellcolor{green!0}{\large 0}/{\footnotesize 41} \tabularnewline
30 & coturn\_stun\_is\_command\_message & 6 & \cellcolor{green!0}{\large 0}/{\footnotesize 40} & \cellcolor{green!0}{\large 0}/{\footnotesize 40} & \cellcolor{green!0}{\large 0}/{\footnotesize 40} & \cellcolor{green!0}{\large 0}/{\footnotesize 40} & \cellcolor{green!0}{\large 0}/{\footnotesize 42} & \cellcolor{green!0}{\large 0}/{\footnotesize 40} \tabularnewline
\rowcolor{black!10} 31 & coturn\_stun\_is\_response & 6 & \cellcolor{green!0}{\large 0}/{\footnotesize 40} & \cellcolor{green!0}{\large 0}/{\footnotesize 40} & \cellcolor{green!0}{\large -}{\tiny -} & \cellcolor{green!0}{\large 0}/{\footnotesize 40} & \cellcolor{green!0}{\large 0}/{\footnotesize 40} & \cellcolor{green!0}{\large 0}/{\footnotesize 40} \tabularnewline
32 & coturn\_stun\_is\_success\_response & 6 & \cellcolor{green!0}{\large 0}/{\footnotesize 40} & \cellcolor{green!0}{\large 0}/{\footnotesize 40} & \cellcolor{green!0}{\large -}{\tiny -} & \cellcolor{green!0}{\large 0}/{\footnotesize 40} & \cellcolor{green!0}{\large 0}/{\footnotesize 42} & \cellcolor{green!0}{\large 0}/{\footnotesize 40} \tabularnewline
\rowcolor{black!10} 33 & hiredis\_redisFormatCommand & 6 & \cellcolor{green!0}{\large 0}/{\footnotesize 40} & \cellcolor{green!0}{\large 0}/{\footnotesize 40} & \cellcolor{green!0}{\large -}{\tiny -} & \cellcolor{green!0}{\large 0}/{\footnotesize 40} & \cellcolor{green!0}{\large 0}/{\footnotesize 40} & \cellcolor{green!0}{\large 0}/{\footnotesize 40} \tabularnewline
34 & igraph\_igraph\_read\_graph\_dl & 6 & \cellcolor{green!0}{\large 0}/{\footnotesize 40} & \cellcolor{green!0}{\large 0}/{\footnotesize 40} & \cellcolor{green!0}{\large 0}/{\footnotesize 40} & \cellcolor{green!0}{\large 0}/{\footnotesize 40} & \cellcolor{green!0}{\large 0}/{\footnotesize 41} & \cellcolor{green!0}{\large 0}/{\footnotesize 40} \tabularnewline
\rowcolor{black!10} 35 & igraph\_igraph\_read\_graph\_edgelist & 6 & \cellcolor{green!0}{\large 0}/{\footnotesize 40} & \cellcolor{green!0}{\large 0}/{\footnotesize 40} & \cellcolor{green!0}{\large 0}/{\footnotesize 40} & \cellcolor{green!0}{\large 0}/{\footnotesize 40} & \cellcolor{green!0}{\large 0}/{\footnotesize 43} & \cellcolor{green!0}{\large 0}/{\footnotesize 40} \tabularnewline
36 & igraph\_igraph\_read\_graph\_gml & 6 & \cellcolor{green!0}{\large 0}/{\footnotesize 40} & \cellcolor{green!0}{\large 0}/{\footnotesize 40} & \cellcolor{green!0}{\large 0}/{\footnotesize 40} & \cellcolor{green!0}{\large 0}/{\footnotesize 40} & \cellcolor{green!0}{\large 0}/{\footnotesize 40} & \cellcolor{green!0}{\large 0}/{\footnotesize 40} \tabularnewline
\rowcolor{black!10} 37 & igraph\_igraph\_read\_graph\_graphdb & 6 & \cellcolor{green!0}{\large 0}/{\footnotesize 40} & \cellcolor{green!0}{\large 0}/{\footnotesize 40} & \cellcolor{green!0}{\large 0}/{\footnotesize 40} & \cellcolor{green!0}{\large 0}/{\footnotesize 40} & \cellcolor{green!0}{\large 0}/{\footnotesize 40} & \cellcolor{green!0}{\large 0}/{\footnotesize 41} \tabularnewline
38 & igraph\_igraph\_read\_graph\_graphml & 6 & \cellcolor{green!0}{\large 0}/{\footnotesize 40} & \cellcolor{green!0}{\large 0}/{\footnotesize 40} & \cellcolor{green!0}{\large 0}/{\footnotesize 40} & \cellcolor{green!0}{\large 0}/{\footnotesize 40} & \cellcolor{green!0}{\large 0}/{\footnotesize 41} & \cellcolor{green!0}{\large 0}/{\footnotesize 42} \tabularnewline
\rowcolor{black!10} 39 & igraph\_igraph\_read\_graph\_lgl & 6 & \cellcolor{green!0}{\large 0}/{\footnotesize 40} & \cellcolor{green!0}{\large 0}/{\footnotesize 40} & \cellcolor{green!0}{\large 0}/{\footnotesize 40} & \cellcolor{green!0}{\large 0}/{\footnotesize 40} & \cellcolor{green!0}{\large 0}/{\footnotesize 40} & \cellcolor{green!0}{\large 0}/{\footnotesize 41} \tabularnewline
40 & igraph\_igraph\_read\_graph\_pajek & 6 & \cellcolor{green!0}{\large 0}/{\footnotesize 40} & \cellcolor{green!0}{\large 0}/{\footnotesize 40} & \cellcolor{green!0}{\large 0}/{\footnotesize 40} & \cellcolor{green!0}{\large 0}/{\footnotesize 40} & \cellcolor{green!0}{\large 0}/{\footnotesize 42} & \cellcolor{green!0}{\large 0}/{\footnotesize 40} \tabularnewline
\rowcolor{black!10} 41 & inchi\_GetINCHIfromINCHI & 6 & \cellcolor{green!0}{\large 0}/{\footnotesize 40} & \cellcolor{green!0}{\large 0}/{\footnotesize 40} & \cellcolor{green!0}{\large 0}/{\footnotesize 40} & \cellcolor{green!0}{\large 0}/{\footnotesize 40} & \cellcolor{green!0}{\large 0}/{\footnotesize 41} & \cellcolor{green!0}{\large 0}/{\footnotesize 42} \tabularnewline
42 & inchi\_GetStructFromINCHI & 6 & \cellcolor{green!0}{\large 0}/{\footnotesize 40} & \cellcolor{green!0}{\large 0}/{\footnotesize 40} & \cellcolor{green!0}{\large 0}/{\footnotesize 40} & \cellcolor{green!0}{\large 0}/{\footnotesize 40} & \cellcolor{green!0}{\large 0}/{\footnotesize 42} & \cellcolor{green!0}{\large 0}/{\footnotesize 40} \tabularnewline
\rowcolor{black!10} 43 & kamailio\_parse\_msg & 6 & \cellcolor{green!0}{\large 0}/{\footnotesize 40} & \cellcolor{green!0}{\large 0}/{\footnotesize 40} & \cellcolor{green!0}{\large -}{\tiny -} & \cellcolor{green!0}{\large 0}/{\footnotesize 40} & \cellcolor{green!0}{\large 0}/{\footnotesize 41} & \cellcolor{green!0}{\large 0}/{\footnotesize 41} \tabularnewline
44 & libyang\_lys\_parse\_mem & 6 & \cellcolor{green!0}{\large 0}/{\footnotesize 40} & \cellcolor{green!0}{\large 0}/{\footnotesize 40} & \cellcolor{green!0}{\large 0}/{\footnotesize 40} & \cellcolor{green!0}{\large 0}/{\footnotesize 40} & \cellcolor{green!0}{\large 0}/{\footnotesize 40} & \cellcolor{green!0}{\large 0}/{\footnotesize 40} \tabularnewline
\rowcolor{black!10} 45 & proftpd\_pr\_json\_object\_from\_text & 6 & \cellcolor{green!0}{\large 0}/{\footnotesize 40} & \cellcolor{green!0}{\large 0}/{\footnotesize 40} & \cellcolor{green!0}{\large -}{\tiny -} & \cellcolor{green!0}{\large 0}/{\footnotesize 40} & \cellcolor{green!0}{\large 0}/{\footnotesize 40} & \cellcolor{green!0}{\large 0}/{\footnotesize 40} \tabularnewline
46 & selinux\_policydb\_read & 6 & \cellcolor{green!0}{\large 0}/{\footnotesize 40} & \cellcolor{green!0}{\large 0}/{\footnotesize 40} & \cellcolor{green!0}{\large -}{\tiny -} & \cellcolor{green!0}{\large 0}/{\footnotesize 40} & \cellcolor{green!0}{\large 0}/{\footnotesize 41} & \cellcolor{green!0}{\large 0}/{\footnotesize 40} \tabularnewline
\rowcolor{black!10} 47 & kamailio\_get\_src\_address\_socket & 7 & \cellcolor{green!0}{\large 0}/{\footnotesize 40} & \cellcolor{green!0}{\large 0}/{\footnotesize 40} & \cellcolor{green!0}{\large 0}/{\footnotesize 40} & \cellcolor{green!0}{\large 0}/{\footnotesize 40} & \cellcolor{green!0}{\large 0}/{\footnotesize 41} & \cellcolor{green!0}{\large 0}/{\footnotesize 40} \tabularnewline
48 & kamailio\_get\_src\_uri & 7 & \cellcolor{green!0}{\large 0}/{\footnotesize 40} & \cellcolor{green!0}{\large 0}/{\footnotesize 40} & \cellcolor{green!0}{\large 0}/{\footnotesize 40} & \cellcolor{green!0}{\large 0}/{\footnotesize 40} & \cellcolor{green!0}{\large 0}/{\footnotesize 40} & \cellcolor{green!0}{\large 0}/{\footnotesize 40} \tabularnewline
\rowcolor{black!10} 49 & kamailio\_parse\_content\_disposition & 7 & \cellcolor{green!0}{\large 0}/{\footnotesize 40} & \cellcolor{green!0}{\large 0}/{\footnotesize 40} & \cellcolor{green!0}{\large 0}/{\footnotesize 40} & \cellcolor{green!0}{\large 0}/{\footnotesize 40} & \cellcolor{green!0}{\large 0}/{\footnotesize 41} & \cellcolor{green!0}{\large 0}/{\footnotesize 41} \tabularnewline
50 & kamailio\_parse\_diversion\_header & 7 & \cellcolor{green!0}{\large 0}/{\footnotesize 40} & \cellcolor{green!0}{\large 0}/{\footnotesize 40} & \cellcolor{green!0}{\large 0}/{\footnotesize 40} & \cellcolor{green!0}{\large 0}/{\footnotesize 40} & \cellcolor{green!0}{\large 0}/{\footnotesize 40} & \cellcolor{green!0}{\large 0}/{\footnotesize 42} \tabularnewline
\rowcolor{black!10} 51 & kamailio\_parse\_from\_header & 7 & \cellcolor{green!0}{\large 0}/{\footnotesize 40} & \cellcolor{green!0}{\large 0}/{\footnotesize 40} & \cellcolor{green!0}{\large -}{\tiny -} & \cellcolor{green!0}{\large 0}/{\footnotesize 40} & \cellcolor{green!0}{\large 0}/{\footnotesize 40} & \cellcolor{green!0}{\large 0}/{\footnotesize 40} \tabularnewline
52 & kamailio\_parse\_from\_uri & 7 & \cellcolor{green!0}{\large 0}/{\footnotesize 40} & \cellcolor{green!0}{\large 0}/{\footnotesize 40} & \cellcolor{green!0}{\large -}{\tiny -} & \cellcolor{green!0}{\large 0}/{\footnotesize 40} & \cellcolor{green!0}{\large 0}/{\footnotesize 41} & \cellcolor{green!0}{\large 0}/{\footnotesize 40} \tabularnewline
\rowcolor{black!10} 53 & kamailio\_parse\_headers & 7 & \cellcolor{green!0}{\large 0}/{\footnotesize 40} & \cellcolor{green!0}{\large 0}/{\footnotesize 40} & \cellcolor{green!0}{\large -}{\tiny -} & \cellcolor{green!0}{\large 0}/{\footnotesize 40} & \cellcolor{green!0}{\large 0}/{\footnotesize 42} & \cellcolor{green!0}{\large 0}/{\footnotesize 40} \tabularnewline
54 & kamailio\_parse\_identityinfo\_header & 7 & \cellcolor{green!0}{\large 0}/{\footnotesize 40} & \cellcolor{green!0}{\large 0}/{\footnotesize 40} & \cellcolor{green!0}{\large -}{\tiny -} & \cellcolor{green!0}{\large 0}/{\footnotesize 40} & \cellcolor{green!0}{\large 0}/{\footnotesize 41} & \cellcolor{green!0}{\large 0}/{\footnotesize 40} \tabularnewline
\rowcolor{black!10} 55 & kamailio\_parse\_pai\_header & 7 & \cellcolor{green!0}{\large 0}/{\footnotesize 40} & \cellcolor{green!0}{\large 0}/{\footnotesize 40} & \cellcolor{green!0}{\large -}{\tiny -} & \cellcolor{green!0}{\large 0}/{\footnotesize 40} & \cellcolor{green!0}{\large 0}/{\footnotesize 41} & \cellcolor{green!0}{\large 0}/{\footnotesize 40} \tabularnewline
56 & kamailio\_parse\_privacy & 7 & \cellcolor{green!0}{\large 0}/{\footnotesize 40} & \cellcolor{green!0}{\large 0}/{\footnotesize 40} & \cellcolor{green!0}{\large 0}/{\footnotesize 40} & \cellcolor{green!0}{\large 0}/{\footnotesize 40} & \cellcolor{green!0}{\large 0}/{\footnotesize 41} & \cellcolor{green!0}{\large 0}/{\footnotesize 40} \tabularnewline
\rowcolor{black!10} 57 & kamailio\_parse\_record\_route\_headers & 7 & \cellcolor{green!0}{\large 0}/{\footnotesize 40} & \cellcolor{green!0}{\large 0}/{\footnotesize 40} & \cellcolor{green!0}{\large -}{\tiny -} & \cellcolor{green!0}{\large 0}/{\footnotesize 40} & \cellcolor{green!0}{\large 0}/{\footnotesize 43} & \cellcolor{green!0}{\large 0}/{\footnotesize 40} \tabularnewline
58 & kamailio\_parse\_refer\_to\_header & 7 & \cellcolor{green!0}{\large 0}/{\footnotesize 40} & \cellcolor{green!0}{\large 0}/{\footnotesize 40} & \cellcolor{green!0}{\large -}{\tiny -} & \cellcolor{green!0}{\large 0}/{\footnotesize 40} & \cellcolor{green!0}{\large 0}/{\footnotesize 41} & \cellcolor{green!0}{\large 0}/{\footnotesize 41} \tabularnewline
\rowcolor{black!10} 59 & kamailio\_parse\_route\_headers & 7 & \cellcolor{green!0}{\large 0}/{\footnotesize 40} & \cellcolor{green!0}{\large 0}/{\footnotesize 40} & \cellcolor{green!0}{\large -}{\tiny -} & \cellcolor{green!0}{\large 0}/{\footnotesize 40} & \cellcolor{green!0}{\large 0}/{\footnotesize 41} & \cellcolor{green!0}{\large 0}/{\footnotesize 40} \tabularnewline
60 & kamailio\_parse\_to\_header & 7 & \cellcolor{green!0}{\large 0}/{\footnotesize 40} & \cellcolor{green!0}{\large 0}/{\footnotesize 40} & \cellcolor{green!0}{\large -}{\tiny -} & \cellcolor{green!0}{\large 0}/{\footnotesize 40} & \cellcolor{green!0}{\large 0}/{\footnotesize 40} & \cellcolor{green!0}{\large 0}/{\footnotesize 40} \tabularnewline
\rowcolor{black!10} 61 & kamailio\_parse\_to\_uri & 7 & \cellcolor{green!0}{\large 0}/{\footnotesize 40} & \cellcolor{green!0}{\large 0}/{\footnotesize 40} & \cellcolor{green!0}{\large -}{\tiny -} & \cellcolor{green!0}{\large 0}/{\footnotesize 40} & \cellcolor{green!0}{\large 0}/{\footnotesize 45} & \cellcolor{green!0}{\large 0}/{\footnotesize 40} \tabularnewline
62 & libyang\_lyd\_parse\_data\_mem & 7 & \cellcolor{green!0}{\large 0}/{\footnotesize 40} & \cellcolor{green!0}{\large 0}/{\footnotesize 40} & \cellcolor{green!0}{\large 0}/{\footnotesize 40} & \cellcolor{green!0}{\large 0}/{\footnotesize 40} & \cellcolor{green!0}{\large 0}/{\footnotesize 41} & \cellcolor{green!0}{\large 0}/{\footnotesize 41} \tabularnewline
\rowcolor{black!10} 63 & bind9\_dns\_message\_parse & 8 & \cellcolor{green!0}{\large 0}/{\footnotesize 40} & \cellcolor{green!0}{\large 0}/{\footnotesize 40} & \cellcolor{green!0}{\large -}{\tiny -} & \cellcolor{green!0}{\large 0}/{\footnotesize 40} & \cellcolor{green!0}{\large 0}/{\footnotesize 40} & \cellcolor{green!0}{\large 0}/{\footnotesize 40} \tabularnewline
64 & igraph\_igraph\_read\_graph\_ncol & 8 & \cellcolor{green!0}{\large 0}/{\footnotesize 40} & \cellcolor{green!0}{\large 0}/{\footnotesize 40} & \cellcolor{green!0}{\large 0}/{\footnotesize 40} & \cellcolor{green!0}{\large 0}/{\footnotesize 40} & \cellcolor{green!0}{\large 0}/{\footnotesize 40} & \cellcolor{green!0}{\large 0}/{\footnotesize 40} \tabularnewline
\rowcolor{black!10} 65 & pjsip\_pj\_json\_parse & 8 & \cellcolor{green!0}{\large 0}/{\footnotesize 40} & \cellcolor{green!0}{\large 0}/{\footnotesize 40} & \cellcolor{green!0}{\large 0}/{\footnotesize 40} & \cellcolor{green!0}{\large 0}/{\footnotesize 40} & \cellcolor{green!0}{\large 0}/{\footnotesize 40} & \cellcolor{green!0}{\large 0}/{\footnotesize 40} \tabularnewline
66 & pjsip\_pj\_xml\_parse & 8 & \cellcolor{green!0}{\large 0}/{\footnotesize 40} & \cellcolor{green!0}{\large 0}/{\footnotesize 40} & \cellcolor{green!0}{\large 0}/{\footnotesize 40} & \cellcolor{green!0}{\large 0}/{\footnotesize 40} & \cellcolor{green!0}{\large 0}/{\footnotesize 40} & \cellcolor{green!0}{\large 0}/{\footnotesize 40} \tabularnewline
\rowcolor{black!10} 67 & pjsip\_pjmedia\_sdp\_parse & 8 & \cellcolor{green!0}{\large 0}/{\footnotesize 40} & \cellcolor{green!0}{\large 0}/{\footnotesize 40} & \cellcolor{green!0}{\large 0}/{\footnotesize 40} & \cellcolor{green!0}{\large 0}/{\footnotesize 40} & \cellcolor{green!0}{\large 0}/{\footnotesize 42} & \cellcolor{green!0}{\large 0}/{\footnotesize 40} \tabularnewline
68 & quickjs\_lre\_compile & 8 & \cellcolor{green!0}{\large 0}/{\footnotesize 40} & \cellcolor{green!0}{\large 0}/{\footnotesize 40} & \cellcolor{green!0}{\large -}{\tiny -} & \cellcolor{green!0}{\large 0}/{\footnotesize 40} & \cellcolor{green!0}{\large 0}/{\footnotesize 42} & \cellcolor{green!0}{\large 0}/{\footnotesize 41} \tabularnewline
\rowcolor{black!10} 69 & bind9\_isc\_lex\_getmastertoken & 9 & \cellcolor{green!0}{\large 0}/{\footnotesize 40} & \cellcolor{green!0}{\large 0}/{\footnotesize 40} & \cellcolor{green!0}{\large -}{\tiny -} & \cellcolor{green!0}{\large 0}/{\footnotesize 40} & \cellcolor{green!0}{\large 0}/{\footnotesize 40} & \cellcolor{green!0}{\large 0}/{\footnotesize 41} \tabularnewline
70 & bind9\_isc\_lex\_gettoken & 9 & \cellcolor{green!0}{\large 0}/{\footnotesize 40} & \cellcolor{green!0}{\large 0}/{\footnotesize 40} & \cellcolor{green!0}{\large -}{\tiny -} & \cellcolor{green!0}{\large 0}/{\footnotesize 40} & \cellcolor{green!0}{\large 0}/{\footnotesize 40} & \cellcolor{green!0}{\large 0}/{\footnotesize 42} \tabularnewline
\rowcolor{black!10} 71 & quickjs\_JS\_Eval & 9 & \cellcolor{green!0}{\large 0}/{\footnotesize 40} & \cellcolor{green!0}{\large 0}/{\footnotesize 40} & \cellcolor{green!0}{\large -}{\tiny -} & \cellcolor{green!0}{\large 0}/{\footnotesize 40} & \cellcolor{green!0}{\large 0}/{\footnotesize 40} & \cellcolor{green!0}{\large 0}/{\footnotesize 40} \tabularnewline
72 & igraph\_igraph\_edge\_connectivity & 10 & \cellcolor{green!0}{\large 0}/{\footnotesize 40} & \cellcolor{green!0}{\large 0}/{\footnotesize 40} & \cellcolor{green!0}{\large 0}/{\footnotesize 40} & \cellcolor{green!0}{\large 0}/{\footnotesize 40} & \cellcolor{green!0}{\large 0}/{\footnotesize 42} & \cellcolor{green!0}{\large 0}/{\footnotesize 40} \tabularnewline
\rowcolor{black!10} 73 & pjsip\_pj\_stun\_msg\_decode & 10 & \cellcolor{green!0}{\large 0}/{\footnotesize 40} & \cellcolor{green!0}{\large 0}/{\footnotesize 40} & \cellcolor{green!0}{\large 0}/{\footnotesize 40} & \cellcolor{green!0}{\large 0}/{\footnotesize 40} & \cellcolor{green!0}{\large 0}/{\footnotesize 40} & \cellcolor{green!0}{\large 0}/{\footnotesize 40} \tabularnewline
74 & bind9\_dns\_message\_checksig & 11 & \cellcolor{green!0}{\large 0}/{\footnotesize 40} & \cellcolor{green!0}{\large 0}/{\footnotesize 40} & \cellcolor{green!0}{\large -}{\tiny -} & \cellcolor{green!0}{\large 0}/{\footnotesize 40} & \cellcolor{green!0}{\large 0}/{\footnotesize 42} & \cellcolor{green!0}{\large 0}/{\footnotesize 40} \tabularnewline
\rowcolor{black!10} 75 & libzip\_zip\_fread & 11 & \cellcolor{green!0}{\large 0}/{\footnotesize 40} & \cellcolor{green!0}{\large 0}/{\footnotesize 40} & \cellcolor{green!0}{\large 0}/{\footnotesize 40} & \cellcolor{green!0}{\large 0}/{\footnotesize 40} & \cellcolor{green!0}{\large 0}/{\footnotesize 40} & \cellcolor{green!0}{\large 0}/{\footnotesize 40} \tabularnewline
76 & bind9\_dns\_rdata\_fromtext & 12 & \cellcolor{green!0}{\large 0}/{\footnotesize 40} & \cellcolor{green!0}{\large 0}/{\footnotesize 40} & \cellcolor{green!0}{\large -}{\tiny -} & \cellcolor{green!0}{\large 0}/{\footnotesize 40} & \cellcolor{green!0}{\large 0}/{\footnotesize 40} & \cellcolor{green!0}{\large 0}/{\footnotesize 40} \tabularnewline
\rowcolor{black!10} 77 & igraph\_igraph\_all\_minimal\_st\_separators & 12 & \cellcolor{green!0}{\large 0}/{\footnotesize 40} & \cellcolor{green!0}{\large 0}/{\footnotesize 40} & \cellcolor{green!0}{\large 0}/{\footnotesize 40} & \cellcolor{green!0}{\large 0}/{\footnotesize 40} & \cellcolor{green!0}{\large 0}/{\footnotesize 40} & \cellcolor{green!0}{\large 0}/{\footnotesize 40} \tabularnewline
78 & igraph\_igraph\_minimum\_size\_separators & 12 & \cellcolor{green!0}{\large 0}/{\footnotesize 40} & \cellcolor{green!0}{\large 0}/{\footnotesize 40} & \cellcolor{green!0}{\large 0}/{\footnotesize 40} & \cellcolor{green!0}{\large 0}/{\footnotesize 40} & \cellcolor{green!0}{\large 0}/{\footnotesize 41} & \cellcolor{green!0}{\large 0}/{\footnotesize 41} \tabularnewline
\rowcolor{black!10} 79 & pjsip\_pjsip\_parse\_msg & 12 & \cellcolor{green!0}{\large 0}/{\footnotesize 40} & \cellcolor{green!0}{\large 0}/{\footnotesize 40} & \cellcolor{green!0}{\large 0}/{\footnotesize 40} & \cellcolor{green!0}{\large 0}/{\footnotesize 40} & \cellcolor{green!0}{\large 0}/{\footnotesize 41} & \cellcolor{green!0}{\large 0}/{\footnotesize 40} \tabularnewline
80 & igraph\_igraph\_automorphism\_group & 13 & \cellcolor{green!0}{\large 0}/{\footnotesize 40} & \cellcolor{green!0}{\large 0}/{\footnotesize 40} & \cellcolor{green!0}{\large 0}/{\footnotesize 40} & \cellcolor{green!0}{\large 0}/{\footnotesize 40} & \cellcolor{green!0}{\large 0}/{\footnotesize 41} & \cellcolor{green!0}{\large 0}/{\footnotesize 40} \tabularnewline
\rowcolor{black!10} 81 & libmodbus\_modbus\_read\_bits & 15 & \cellcolor{green!0}{\large 0}/{\footnotesize 40} & \cellcolor{green!0}{\large 0}/{\footnotesize 40} & \cellcolor{green!0}{\large 0}/{\footnotesize 40} & \cellcolor{green!0}{\large 0}/{\footnotesize 40} & \cellcolor{green!0}{\large 0}/{\footnotesize 41} & \cellcolor{green!0}{\large 0}/{\footnotesize 41} \tabularnewline
82 & libmodbus\_modbus\_read\_registers & 15 & \cellcolor{green!0}{\large 0}/{\footnotesize 40} & \cellcolor{green!0}{\large 0}/{\footnotesize 40} & \cellcolor{green!0}{\large 0}/{\footnotesize 40} & \cellcolor{green!0}{\large 0}/{\footnotesize 40} & \cellcolor{green!0}{\large 0}/{\footnotesize 41} & \cellcolor{green!0}{\large 0}/{\footnotesize 40} \tabularnewline
\rowcolor{black!10} 83 & civetweb\_mg\_get\_response & 17 & \cellcolor{green!0}{\large 0}/{\footnotesize 40} & \cellcolor{green!0}{\large 0}/{\footnotesize 40} & \cellcolor{green!0}{\large 0}/{\footnotesize 40} & \cellcolor{green!0}{\large 0}/{\footnotesize 40} & \cellcolor{green!0}{\large 0}/{\footnotesize 40} & \cellcolor{green!0}{\large 0}/{\footnotesize 40} \tabularnewline
84 & bind9\_dns\_master\_loadbuffer & 20 & \cellcolor{green!0}{\large 0}/{\footnotesize 40} & \cellcolor{green!0}{\large 0}/{\footnotesize 40} & \cellcolor{green!0}{\large -}{\tiny -} & \cellcolor{green!0}{\large 0}/{\footnotesize 40} & \cellcolor{green!0}{\large 0}/{\footnotesize 42} & \cellcolor{green!0}{\large 0}/{\footnotesize 40} \tabularnewline
\rowcolor{black!10} 85 & libmodbus\_modbus\_receive & 33 & \cellcolor{green!0}{\large 0}/{\footnotesize 40} & \cellcolor{green!0}{\large 0}/{\footnotesize 40} & \cellcolor{green!0}{\large 0}/{\footnotesize 40} & \cellcolor{green!0}{\large 0}/{\footnotesize 40} & \cellcolor{green!0}{\large 0}/{\footnotesize 43} & \cellcolor{green!0}{\large 0}/{\footnotesize 40} \tabularnewline
86 & tmux\_input\_parse\_buffer & 42 & \cellcolor{green!0}{\large 0}/{\footnotesize 40} & \cellcolor{green!0}{\large 0}/{\footnotesize 40} & \cellcolor{green!0}{\large -}{\tiny -} & \cellcolor{green!0}{\large 0}/{\footnotesize 40} & \cellcolor{green!0}{\large 0}/{\footnotesize 43} & \cellcolor{green!0}{\large 0}/{\footnotesize 40} \tabularnewline

\bottomrule
%\end{tabular}
%}
%\end{table*}
\end{xltabular}
}
\twocolumn



% model: text-bison-001, temp: 0.0

\onecolumn
{\small %
\begin{xltabular}[h]{\textwidth}{ccccccccc}
%\begin{table*}[!t]
%\centering
\caption{Evaluation Result of model text-bison-001 with temperature 0.0.} \\
%\resizebox{1.0\linewidth}{!}{
%\begin{tabular}{cccccccccc}
\toprule
Index & Question & Score & NAIVE-40 & BACTX-40 & DOCTX-40 & UGCTX-40 & BA-ITER-40 & ALL-ITER-40 \tabularnewline
\midrule
\rowcolor{black!10} 1 & coturn\_stun\_is\_command\_message\_full\_check\_str & 1 & \cellcolor{green!0}{\large 0}/{\footnotesize 40} & \cellcolor{green!100}{\large 40}/{\footnotesize 40} & \cellcolor{green!0}{\large -}{\tiny -} & \cellcolor{green!0}{\large 0}/{\footnotesize 40} & \cellcolor{green!100}{\large 40}/{\footnotesize 40} & \cellcolor{green!20}{\large 8}/{\footnotesize 47} \tabularnewline
2 & kamailio\_parse\_uri & 1 & \cellcolor{green!0}{\large 0}/{\footnotesize 40} & \cellcolor{green!0}{\large 0}/{\footnotesize 40} & \cellcolor{green!0}{\large -}{\tiny -} & \cellcolor{green!10}{\large 3}/{\footnotesize 40} & \cellcolor{green!40}{\large 31}/{\footnotesize 80} & \cellcolor{green!10}{\large 7}/{\footnotesize 99} \tabularnewline
\rowcolor{black!10} 3 & coturn\_stun\_check\_message\_integrity\_str & 2 & \cellcolor{green!0}{\large 0}/{\footnotesize 40} & \cellcolor{green!0}{\large 0}/{\footnotesize 40} & \cellcolor{green!0}{\large -}{\tiny -} & \cellcolor{green!0}{\large 0}/{\footnotesize 40} & \cellcolor{green!0}{\large 0}/{\footnotesize 200} & \cellcolor{green!0}{\large 0}/{\footnotesize 81} \tabularnewline
4 & libiec61850\_MmsValue\_decodeMmsData & 2 & \cellcolor{green!0}{\large 0}/{\footnotesize 40} & \cellcolor{green!0}{\large 0}/{\footnotesize 40} & \cellcolor{green!100}{\large 40}/{\footnotesize 40} & \cellcolor{green!10}{\large 2}/{\footnotesize 40} & \cellcolor{green!0}{\large 0}/{\footnotesize 80} & \cellcolor{green!10}{\large 6}/{\footnotesize 93} \tabularnewline
\rowcolor{black!10} 5 & md4c\_md\_html & 2 & \cellcolor{green!0}{\large 0}/{\footnotesize 40} & \cellcolor{green!0}{\large 0}/{\footnotesize 40} & \cellcolor{green!0}{\large 0}/{\footnotesize 40} & \cellcolor{green!0}{\large 0}/{\footnotesize 40} & \cellcolor{green!0}{\large 0}/{\footnotesize 162} & \cellcolor{green!0}{\large 0}/{\footnotesize 151} \tabularnewline
6 & spdk\_spdk\_json\_parse & 2 & \cellcolor{green!0}{\large 0}/{\footnotesize 40} & \cellcolor{green!0}{\large 0}/{\footnotesize 40} & \cellcolor{green!0}{\large -}{\tiny -} & \cellcolor{green!10}{\large 2}/{\footnotesize 40} & \cellcolor{green!0}{\large 0}/{\footnotesize 40} & \cellcolor{green!10}{\large 3}/{\footnotesize 60} \tabularnewline
\rowcolor{black!10} 7 & croaring\_roaring\_bitmap\_portable\_deserialize\_safe & 3 & \cellcolor{green!0}{\large 0}/{\footnotesize 40} & \cellcolor{green!0}{\large 0}/{\footnotesize 40} & \cellcolor{green!100}{\large 40}/{\footnotesize 40} & \cellcolor{green!70}{\large 25}/{\footnotesize 40} & \cellcolor{green!30}{\large 19}/{\footnotesize 80} & \cellcolor{green!70}{\large 30}/{\footnotesize 49} \tabularnewline
8 & lua\_luaL\_loadbufferx & 3 & \cellcolor{green!100}{\large 40}/{\footnotesize 40} & \cellcolor{green!100}{\large 40}/{\footnotesize 40} & \cellcolor{green!100}{\large 40}/{\footnotesize 40} & \cellcolor{green!50}{\large 20}/{\footnotesize 40} & \cellcolor{green!100}{\large 40}/{\footnotesize 40} & \cellcolor{green!40}{\large 30}/{\footnotesize 75} \tabularnewline
\rowcolor{black!10} 9 & w3m\_wc\_Str\_conv\_with\_detect & 3 & \cellcolor{green!0}{\large 0}/{\footnotesize 40} & \cellcolor{green!0}{\large 0}/{\footnotesize 40} & \cellcolor{green!0}{\large -}{\tiny -} & \cellcolor{green!0}{\large 0}/{\footnotesize 40} & \cellcolor{green!0}{\large 0}/{\footnotesize 200} & \cellcolor{green!0}{\large 0}/{\footnotesize 200} \tabularnewline
10 & bind9\_dns\_name\_fromwire & 4 & \cellcolor{green!0}{\large 0}/{\footnotesize 40} & \cellcolor{green!0}{\large 0}/{\footnotesize 40} & \cellcolor{green!0}{\large -}{\tiny -} & \cellcolor{green!0}{\large 0}/{\footnotesize 40} & \cellcolor{green!0}{\large 0}/{\footnotesize 200} & \cellcolor{green!10}{\large 2}/{\footnotesize 159} \tabularnewline
\rowcolor{black!10} 11 & gdk-pixbuf\_gdk\_pixbuf\_animation\_new\_from\_file & 4 & \cellcolor{green!0}{\large 0}/{\footnotesize 40} & \cellcolor{green!0}{\large 0}/{\footnotesize 40} & \cellcolor{green!0}{\large 0}/{\footnotesize 40} & \cellcolor{green!0}{\large 0}/{\footnotesize 40} & \cellcolor{green!0}{\large 0}/{\footnotesize 40} & \cellcolor{green!0}{\large 0}/{\footnotesize 81} \tabularnewline
12 & gdk-pixbuf\_gdk\_pixbuf\_new\_from\_data & 4 & \cellcolor{green!0}{\large 0}/{\footnotesize 40} & \cellcolor{green!100}{\large 40}/{\footnotesize 40} & \cellcolor{green!100}{\large 40}/{\footnotesize 40} & \cellcolor{green!60}{\large 22}/{\footnotesize 40} & \cellcolor{green!100}{\large 40}/{\footnotesize 40} & \cellcolor{green!40}{\large 23}/{\footnotesize 60} \tabularnewline
\rowcolor{black!10} 13 & gdk-pixbuf\_gdk\_pixbuf\_new\_from\_file & 4 & \cellcolor{green!0}{\large 0}/{\footnotesize 40} & \cellcolor{green!0}{\large 0}/{\footnotesize 40} & \cellcolor{green!0}{\large 0}/{\footnotesize 40} & \cellcolor{green!0}{\large 0}/{\footnotesize 40} & \cellcolor{green!0}{\large 0}/{\footnotesize 40} & \cellcolor{green!0}{\large 0}/{\footnotesize 69} \tabularnewline
14 & gdk-pixbuf\_gdk\_pixbuf\_new\_from\_stream & 4 & \cellcolor{green!100}{\large 40}/{\footnotesize 40} & \cellcolor{green!0}{\large 0}/{\footnotesize 40} & \cellcolor{green!100}{\large 40}/{\footnotesize 40} & \cellcolor{green!70}{\large 28}/{\footnotesize 40} & \cellcolor{green!0}{\large 0}/{\footnotesize 40} & \cellcolor{green!60}{\large 30}/{\footnotesize 50} \tabularnewline
\rowcolor{black!10} 15 & gpac\_gf\_isom\_open\_file & 4 & \cellcolor{green!0}{\large 0}/{\footnotesize 40} & \cellcolor{green!0}{\large 0}/{\footnotesize 40} & \cellcolor{green!0}{\large -}{\tiny -} & \cellcolor{green!0}{\large 0}/{\footnotesize 40} & \cellcolor{green!0}{\large 0}/{\footnotesize 200} & \cellcolor{green!0}{\large 0}/{\footnotesize 185} \tabularnewline
16 & libbpf\_bpf\_object\_\_open\_mem & 4 & \cellcolor{green!0}{\large 0}/{\footnotesize 40} & \cellcolor{green!100}{\large 40}/{\footnotesize 40} & \cellcolor{green!100}{\large 40}/{\footnotesize 40} & \cellcolor{green!40}{\large 16}/{\footnotesize 40} & \cellcolor{green!100}{\large 40}/{\footnotesize 40} & \cellcolor{green!40}{\large 21}/{\footnotesize 63} \tabularnewline
\rowcolor{black!10} 17 & libpg\_query\_pg\_query\_parse & 4 & \cellcolor{green!0}{\large 0}/{\footnotesize 40} & \cellcolor{green!0}{\large 0}/{\footnotesize 40} & \cellcolor{green!0}{\large -}{\tiny -} & \cellcolor{green!10}{\large 3}/{\footnotesize 40} & \cellcolor{green!0}{\large 0}/{\footnotesize 128} & \cellcolor{green!10}{\large 6}/{\footnotesize 124} \tabularnewline
18 & libucl\_ucl\_parser\_add\_string & 4 & \cellcolor{green!0}{\large 0}/{\footnotesize 40} & \cellcolor{green!0}{\large 0}/{\footnotesize 40} & \cellcolor{green!0}{\large 0}/{\footnotesize 40} & \cellcolor{green!0}{\large 0}/{\footnotesize 40} & \cellcolor{green!0}{\large 0}/{\footnotesize 200} & \cellcolor{green!10}{\large 6}/{\footnotesize 154} \tabularnewline
\rowcolor{black!10} 19 & oniguruma\_onig\_new & 4 & \cellcolor{green!0}{\large 0}/{\footnotesize 40} & \cellcolor{green!0}{\large 0}/{\footnotesize 40} & \cellcolor{green!0}{\large 0}/{\footnotesize 40} & \cellcolor{green!10}{\large 3}/{\footnotesize 40} & \cellcolor{green!0}{\large 0}/{\footnotesize 198} & \cellcolor{green!10}{\large 5}/{\footnotesize 130} \tabularnewline
20 & pupnp\_ixmlLoadDocumentEx & 4 & \cellcolor{green!0}{\large 0}/{\footnotesize 40} & \cellcolor{green!0}{\large 0}/{\footnotesize 40} & \cellcolor{green!0}{\large 0}/{\footnotesize 40} & \cellcolor{green!0}{\large 0}/{\footnotesize 40} & \cellcolor{green!0}{\large 0}/{\footnotesize 80} & \cellcolor{green!0}{\large 0}/{\footnotesize 47} \tabularnewline
\rowcolor{black!10} 21 & gdk-pixbuf\_gdk\_pixbuf\_new\_from\_file\_at\_scale & 5 & \cellcolor{green!0}{\large 0}/{\footnotesize 40} & \cellcolor{green!0}{\large 0}/{\footnotesize 40} & \cellcolor{green!0}{\large 0}/{\footnotesize 40} & \cellcolor{green!0}{\large 0}/{\footnotesize 40} & \cellcolor{green!0}{\large 0}/{\footnotesize 200} & \cellcolor{green!0}{\large 0}/{\footnotesize 83} \tabularnewline
22 & inchi\_GetINCHIKeyFromINCHI & 5 & \cellcolor{green!0}{\large 0}/{\footnotesize 40} & \cellcolor{green!100}{\large 40}/{\footnotesize 40} & \cellcolor{green!0}{\large 0}/{\footnotesize 40} & \cellcolor{green!10}{\large 3}/{\footnotesize 40} & \cellcolor{green!100}{\large 40}/{\footnotesize 40} & \cellcolor{green!10}{\large 7}/{\footnotesize 97} \tabularnewline
\rowcolor{black!10} 23 & libdwarf\_dwarf\_init\_b & 5 & \cellcolor{green!0}{\large 0}/{\footnotesize 40} & \cellcolor{green!0}{\large 0}/{\footnotesize 40} & \cellcolor{green!0}{\large 0}/{\footnotesize 40} & \cellcolor{green!30}{\large 12}/{\footnotesize 40} & \cellcolor{green!0}{\large 0}/{\footnotesize 87} & \cellcolor{green!20}{\large 13}/{\footnotesize 85} \tabularnewline
24 & libdwarf\_dwarf\_init\_path & 5 & \cellcolor{green!0}{\large 0}/{\footnotesize 40} & \cellcolor{green!0}{\large 0}/{\footnotesize 40} & \cellcolor{green!0}{\large 0}/{\footnotesize 40} & \cellcolor{green!10}{\large 2}/{\footnotesize 40} & \cellcolor{green!0}{\large 0}/{\footnotesize 40} & \cellcolor{green!10}{\large 2}/{\footnotesize 93} \tabularnewline
\rowcolor{black!10} 25 & liblouis\_lou\_compileString & 5 & \cellcolor{green!0}{\large 0}/{\footnotesize 40} & \cellcolor{green!0}{\large 0}/{\footnotesize 40} & \cellcolor{green!0}{\large 0}/{\footnotesize 40} & \cellcolor{green!0}{\large 0}/{\footnotesize 40} & \cellcolor{green!0}{\large 0}/{\footnotesize 120} & \cellcolor{green!0}{\large 0}/{\footnotesize 163} \tabularnewline
26 & selinux\_cil\_compile & 5 & \cellcolor{green!0}{\large 0}/{\footnotesize 40} & \cellcolor{green!0}{\large 0}/{\footnotesize 40} & \cellcolor{green!0}{\large -}{\tiny -} & \cellcolor{green!10}{\large 3}/{\footnotesize 40} & \cellcolor{green!0}{\large 0}/{\footnotesize 40} & \cellcolor{green!10}{\large 5}/{\footnotesize 48} \tabularnewline
\rowcolor{black!10} 27 & bind9\_dns\_name\_fromtext & 6 & \cellcolor{green!0}{\large 0}/{\footnotesize 40} & \cellcolor{green!0}{\large 0}/{\footnotesize 40} & \cellcolor{green!0}{\large -}{\tiny -} & \cellcolor{green!20}{\large 6}/{\footnotesize 40} & \cellcolor{green!0}{\large 0}/{\footnotesize 200} & \cellcolor{green!0}{\large 1}/{\footnotesize 115} \tabularnewline
28 & bind9\_dns\_rdata\_fromwire & 6 & \cellcolor{green!0}{\large 0}/{\footnotesize 40} & \cellcolor{green!0}{\large 0}/{\footnotesize 40} & \cellcolor{green!0}{\large -}{\tiny -} & \cellcolor{green!0}{\large 0}/{\footnotesize 40} & \cellcolor{green!0}{\large 0}/{\footnotesize 200} & \cellcolor{green!0}{\large 0}/{\footnotesize 144} \tabularnewline
\rowcolor{black!10} 29 & coturn\_stun\_is\_binding\_response & 6 & \cellcolor{green!0}{\large 0}/{\footnotesize 40} & \cellcolor{green!0}{\large 0}/{\footnotesize 40} & \cellcolor{green!0}{\large -}{\tiny -} & \cellcolor{green!0}{\large 0}/{\footnotesize 40} & \cellcolor{green!0}{\large 0}/{\footnotesize 80} & \cellcolor{green!0}{\large 0}/{\footnotesize 97} \tabularnewline
30 & coturn\_stun\_is\_command\_message & 6 & \cellcolor{green!0}{\large 0}/{\footnotesize 40} & \cellcolor{green!0}{\large 0}/{\footnotesize 40} & \cellcolor{green!0}{\large 0}/{\footnotesize 40} & \cellcolor{green!0}{\large 0}/{\footnotesize 40} & \cellcolor{green!0}{\large 0}/{\footnotesize 80} & \cellcolor{green!0}{\large 0}/{\footnotesize 60} \tabularnewline
\rowcolor{black!10} 31 & coturn\_stun\_is\_response & 6 & \cellcolor{green!0}{\large 0}/{\footnotesize 40} & \cellcolor{green!0}{\large 0}/{\footnotesize 40} & \cellcolor{green!0}{\large -}{\tiny -} & \cellcolor{green!0}{\large 0}/{\footnotesize 40} & \cellcolor{green!0}{\large 0}/{\footnotesize 80} & \cellcolor{green!0}{\large 0}/{\footnotesize 70} \tabularnewline
32 & coturn\_stun\_is\_success\_response & 6 & \cellcolor{green!0}{\large 0}/{\footnotesize 40} & \cellcolor{green!0}{\large 0}/{\footnotesize 40} & \cellcolor{green!0}{\large -}{\tiny -} & \cellcolor{green!0}{\large 0}/{\footnotesize 40} & \cellcolor{green!0}{\large 0}/{\footnotesize 80} & \cellcolor{green!0}{\large 0}/{\footnotesize 81} \tabularnewline
\rowcolor{black!10} 33 & hiredis\_redisFormatCommand & 6 & \cellcolor{green!0}{\large 0}/{\footnotesize 40} & \cellcolor{green!100}{\large 40}/{\footnotesize 40} & \cellcolor{green!0}{\large -}{\tiny -} & \cellcolor{green!60}{\large 22}/{\footnotesize 40} & \cellcolor{green!10}{\large 15}/{\footnotesize 147} & \cellcolor{green!10}{\large 4}/{\footnotesize 110} \tabularnewline
34 & igraph\_igraph\_read\_graph\_dl & 6 & \cellcolor{green!0}{\large 0}/{\footnotesize 40} & \cellcolor{green!0}{\large 0}/{\footnotesize 40} & \cellcolor{green!0}{\large 0}/{\footnotesize 40} & \cellcolor{green!0}{\large 0}/{\footnotesize 40} & \cellcolor{green!0}{\large 0}/{\footnotesize 200} & \cellcolor{green!0}{\large 0}/{\footnotesize 152} \tabularnewline
\rowcolor{black!10} 35 & igraph\_igraph\_read\_graph\_edgelist & 6 & \cellcolor{green!0}{\large 0}/{\footnotesize 40} & \cellcolor{green!0}{\large 0}/{\footnotesize 40} & \cellcolor{green!0}{\large 0}/{\footnotesize 40} & \cellcolor{green!0}{\large 0}/{\footnotesize 40} & \cellcolor{green!0}{\large 0}/{\footnotesize 200} & \cellcolor{green!0}{\large 0}/{\footnotesize 158} \tabularnewline
36 & igraph\_igraph\_read\_graph\_gml & 6 & \cellcolor{green!0}{\large 0}/{\footnotesize 40} & \cellcolor{green!0}{\large 0}/{\footnotesize 40} & \cellcolor{green!0}{\large 0}/{\footnotesize 40} & \cellcolor{green!0}{\large 0}/{\footnotesize 40} & \cellcolor{green!0}{\large 0}/{\footnotesize 200} & \cellcolor{green!0}{\large 0}/{\footnotesize 149} \tabularnewline
\rowcolor{black!10} 37 & igraph\_igraph\_read\_graph\_graphdb & 6 & \cellcolor{green!0}{\large 0}/{\footnotesize 40} & \cellcolor{green!0}{\large 0}/{\footnotesize 40} & \cellcolor{green!0}{\large 0}/{\footnotesize 40} & \cellcolor{green!0}{\large 0}/{\footnotesize 40} & \cellcolor{green!0}{\large 0}/{\footnotesize 200} & \cellcolor{green!0}{\large 0}/{\footnotesize 159} \tabularnewline
38 & igraph\_igraph\_read\_graph\_graphml & 6 & \cellcolor{green!0}{\large 0}/{\footnotesize 40} & \cellcolor{green!0}{\large 0}/{\footnotesize 40} & \cellcolor{green!0}{\large 0}/{\footnotesize 40} & \cellcolor{green!0}{\large 0}/{\footnotesize 40} & \cellcolor{green!0}{\large 0}/{\footnotesize 200} & \cellcolor{green!0}{\large 0}/{\footnotesize 165} \tabularnewline
\rowcolor{black!10} 39 & igraph\_igraph\_read\_graph\_lgl & 6 & \cellcolor{green!0}{\large 0}/{\footnotesize 40} & \cellcolor{green!0}{\large 0}/{\footnotesize 40} & \cellcolor{green!0}{\large 0}/{\footnotesize 40} & \cellcolor{green!0}{\large 0}/{\footnotesize 40} & \cellcolor{green!0}{\large 0}/{\footnotesize 200} & \cellcolor{green!0}{\large 0}/{\footnotesize 110} \tabularnewline
40 & igraph\_igraph\_read\_graph\_pajek & 6 & \cellcolor{green!0}{\large 0}/{\footnotesize 40} & \cellcolor{green!0}{\large 0}/{\footnotesize 40} & \cellcolor{green!0}{\large 0}/{\footnotesize 40} & \cellcolor{green!0}{\large 0}/{\footnotesize 40} & \cellcolor{green!0}{\large 0}/{\footnotesize 80} & \cellcolor{green!0}{\large 0}/{\footnotesize 135} \tabularnewline
\rowcolor{black!10} 41 & inchi\_GetINCHIfromINCHI & 6 & \cellcolor{green!0}{\large 0}/{\footnotesize 40} & \cellcolor{green!0}{\large 0}/{\footnotesize 40} & \cellcolor{green!0}{\large 0}/{\footnotesize 40} & \cellcolor{green!0}{\large 0}/{\footnotesize 40} & \cellcolor{green!0}{\large 0}/{\footnotesize 200} & \cellcolor{green!0}{\large 0}/{\footnotesize 144} \tabularnewline
42 & inchi\_GetStructFromINCHI & 6 & \cellcolor{green!0}{\large 0}/{\footnotesize 40} & \cellcolor{green!0}{\large 0}/{\footnotesize 40} & \cellcolor{green!0}{\large 0}/{\footnotesize 40} & \cellcolor{green!0}{\large 0}/{\footnotesize 40} & \cellcolor{green!0}{\large 0}/{\footnotesize 200} & \cellcolor{green!0}{\large 0}/{\footnotesize 200} \tabularnewline
\rowcolor{black!10} 43 & kamailio\_parse\_msg & 6 & \cellcolor{green!0}{\large 0}/{\footnotesize 40} & \cellcolor{green!20}{\large 7}/{\footnotesize 40} & \cellcolor{green!0}{\large -}{\tiny -} & \cellcolor{green!20}{\large 8}/{\footnotesize 40} & \cellcolor{green!10}{\large 9}/{\footnotesize 116} & \cellcolor{green!10}{\large 6}/{\footnotesize 95} \tabularnewline
44 & libyang\_lys\_parse\_mem & 6 & \cellcolor{green!0}{\large 0}/{\footnotesize 40} & \cellcolor{green!0}{\large 0}/{\footnotesize 40} & \cellcolor{green!0}{\large 0}/{\footnotesize 40} & \cellcolor{green!10}{\large 1}/{\footnotesize 40} & \cellcolor{green!0}{\large 0}/{\footnotesize 200} & \cellcolor{green!0}{\large 0}/{\footnotesize 150} \tabularnewline
\rowcolor{black!10} 45 & proftpd\_pr\_json\_object\_from\_text & 6 & \cellcolor{green!0}{\large 0}/{\footnotesize 40} & \cellcolor{green!0}{\large 0}/{\footnotesize 40} & \cellcolor{green!0}{\large -}{\tiny -} & \cellcolor{green!0}{\large 0}/{\footnotesize 40} & \cellcolor{green!0}{\large 0}/{\footnotesize 200} & \cellcolor{green!0}{\large 0}/{\footnotesize 92} \tabularnewline
46 & selinux\_policydb\_read & 6 & \cellcolor{green!0}{\large 0}/{\footnotesize 40} & \cellcolor{green!0}{\large 0}/{\footnotesize 40} & \cellcolor{green!0}{\large -}{\tiny -} & \cellcolor{green!10}{\large 1}/{\footnotesize 40} & \cellcolor{green!0}{\large 0}/{\footnotesize 200} & \cellcolor{green!0}{\large 0}/{\footnotesize 125} \tabularnewline
\rowcolor{black!10} 47 & kamailio\_get\_src\_address\_socket & 7 & \cellcolor{green!0}{\large 0}/{\footnotesize 40} & \cellcolor{green!0}{\large 0}/{\footnotesize 40} & \cellcolor{green!0}{\large 0}/{\footnotesize 40} & \cellcolor{green!0}{\large 0}/{\footnotesize 40} & \cellcolor{green!0}{\large 0}/{\footnotesize 200} & \cellcolor{green!0}{\large 0}/{\footnotesize 146} \tabularnewline
48 & kamailio\_get\_src\_uri & 7 & \cellcolor{green!0}{\large 0}/{\footnotesize 40} & \cellcolor{green!0}{\large 0}/{\footnotesize 40} & \cellcolor{green!0}{\large 0}/{\footnotesize 40} & \cellcolor{green!0}{\large 0}/{\footnotesize 40} & \cellcolor{green!0}{\large 0}/{\footnotesize 200} & \cellcolor{green!0}{\large 0}/{\footnotesize 159} \tabularnewline
\rowcolor{black!10} 49 & kamailio\_parse\_content\_disposition & 7 & \cellcolor{green!0}{\large 0}/{\footnotesize 40} & \cellcolor{green!0}{\large 0}/{\footnotesize 40} & \cellcolor{green!0}{\large 0}/{\footnotesize 40} & \cellcolor{green!0}{\large 0}/{\footnotesize 40} & \cellcolor{green!0}{\large 0}/{\footnotesize 200} & \cellcolor{green!0}{\large 0}/{\footnotesize 99} \tabularnewline
50 & kamailio\_parse\_diversion\_header & 7 & \cellcolor{green!0}{\large 0}/{\footnotesize 40} & \cellcolor{green!0}{\large 0}/{\footnotesize 40} & \cellcolor{green!0}{\large 0}/{\footnotesize 40} & \cellcolor{green!0}{\large 0}/{\footnotesize 40} & \cellcolor{green!0}{\large 0}/{\footnotesize 120} & \cellcolor{green!0}{\large 0}/{\footnotesize 80} \tabularnewline
\rowcolor{black!10} 51 & kamailio\_parse\_from\_header & 7 & \cellcolor{green!0}{\large 0}/{\footnotesize 40} & \cellcolor{green!0}{\large 0}/{\footnotesize 40} & \cellcolor{green!0}{\large -}{\tiny -} & \cellcolor{green!10}{\large 1}/{\footnotesize 40} & \cellcolor{green!0}{\large 0}/{\footnotesize 200} & \cellcolor{green!0}{\large 0}/{\footnotesize 129} \tabularnewline
52 & kamailio\_parse\_from\_uri & 7 & \cellcolor{green!0}{\large 0}/{\footnotesize 40} & \cellcolor{green!0}{\large 0}/{\footnotesize 40} & \cellcolor{green!0}{\large -}{\tiny -} & \cellcolor{green!0}{\large 0}/{\footnotesize 40} & \cellcolor{green!0}{\large 0}/{\footnotesize 200} & \cellcolor{green!0}{\large 0}/{\footnotesize 143} \tabularnewline
\rowcolor{black!10} 53 & kamailio\_parse\_headers & 7 & \cellcolor{green!0}{\large 0}/{\footnotesize 40} & \cellcolor{green!0}{\large 0}/{\footnotesize 40} & \cellcolor{green!0}{\large -}{\tiny -} & \cellcolor{green!0}{\large 0}/{\footnotesize 40} & \cellcolor{green!0}{\large 0}/{\footnotesize 200} & \cellcolor{green!0}{\large 0}/{\footnotesize 158} \tabularnewline
54 & kamailio\_parse\_identityinfo\_header & 7 & \cellcolor{green!0}{\large 0}/{\footnotesize 40} & \cellcolor{green!0}{\large 0}/{\footnotesize 40} & \cellcolor{green!0}{\large -}{\tiny -} & \cellcolor{green!0}{\large 0}/{\footnotesize 40} & \cellcolor{green!0}{\large 0}/{\footnotesize 200} & \cellcolor{green!0}{\large 0}/{\footnotesize 114} \tabularnewline
\rowcolor{black!10} 55 & kamailio\_parse\_pai\_header & 7 & \cellcolor{green!0}{\large 0}/{\footnotesize 40} & \cellcolor{green!0}{\large 0}/{\footnotesize 40} & \cellcolor{green!0}{\large -}{\tiny -} & \cellcolor{green!0}{\large 0}/{\footnotesize 40} & \cellcolor{green!0}{\large 0}/{\footnotesize 200} & \cellcolor{green!0}{\large 0}/{\footnotesize 122} \tabularnewline
56 & kamailio\_parse\_privacy & 7 & \cellcolor{green!0}{\large 0}/{\footnotesize 40} & \cellcolor{green!0}{\large 0}/{\footnotesize 40} & \cellcolor{green!0}{\large 0}/{\footnotesize 40} & \cellcolor{green!0}{\large 0}/{\footnotesize 40} & \cellcolor{green!0}{\large 0}/{\footnotesize 200} & \cellcolor{green!0}{\large 0}/{\footnotesize 129} \tabularnewline
\rowcolor{black!10} 57 & kamailio\_parse\_record\_route\_headers & 7 & \cellcolor{green!0}{\large 0}/{\footnotesize 40} & \cellcolor{green!0}{\large 0}/{\footnotesize 40} & \cellcolor{green!0}{\large -}{\tiny -} & \cellcolor{green!0}{\large 0}/{\footnotesize 40} & \cellcolor{green!0}{\large 0}/{\footnotesize 200} & \cellcolor{green!0}{\large 0}/{\footnotesize 96} \tabularnewline
58 & kamailio\_parse\_refer\_to\_header & 7 & \cellcolor{green!0}{\large 0}/{\footnotesize 40} & \cellcolor{green!0}{\large 0}/{\footnotesize 40} & \cellcolor{green!0}{\large -}{\tiny -} & \cellcolor{green!0}{\large 0}/{\footnotesize 40} & \cellcolor{green!0}{\large 0}/{\footnotesize 200} & \cellcolor{green!0}{\large 0}/{\footnotesize 139} \tabularnewline
\rowcolor{black!10} 59 & kamailio\_parse\_route\_headers & 7 & \cellcolor{green!0}{\large 0}/{\footnotesize 40} & \cellcolor{green!0}{\large 0}/{\footnotesize 40} & \cellcolor{green!0}{\large -}{\tiny -} & \cellcolor{green!0}{\large 0}/{\footnotesize 40} & \cellcolor{green!0}{\large 0}/{\footnotesize 200} & \cellcolor{green!0}{\large 0}/{\footnotesize 92} \tabularnewline
60 & kamailio\_parse\_to\_header & 7 & \cellcolor{green!0}{\large 0}/{\footnotesize 40} & \cellcolor{green!0}{\large 0}/{\footnotesize 40} & \cellcolor{green!0}{\large -}{\tiny -} & \cellcolor{green!10}{\large 3}/{\footnotesize 40} & \cellcolor{green!0}{\large 0}/{\footnotesize 80} & \cellcolor{green!10}{\large 2}/{\footnotesize 114} \tabularnewline
\rowcolor{black!10} 61 & kamailio\_parse\_to\_uri & 7 & \cellcolor{green!0}{\large 0}/{\footnotesize 40} & \cellcolor{green!0}{\large 0}/{\footnotesize 40} & \cellcolor{green!0}{\large -}{\tiny -} & \cellcolor{green!0}{\large 0}/{\footnotesize 40} & \cellcolor{green!0}{\large 0}/{\footnotesize 80} & \cellcolor{green!0}{\large 0}/{\footnotesize 110} \tabularnewline
62 & libyang\_lyd\_parse\_data\_mem & 7 & \cellcolor{green!0}{\large 0}/{\footnotesize 40} & \cellcolor{green!0}{\large 0}/{\footnotesize 40} & \cellcolor{green!0}{\large 0}/{\footnotesize 40} & \cellcolor{green!0}{\large 0}/{\footnotesize 40} & \cellcolor{green!0}{\large 0}/{\footnotesize 200} & \cellcolor{green!0}{\large 0}/{\footnotesize 133} \tabularnewline
\rowcolor{black!10} 63 & bind9\_dns\_message\_parse & 8 & \cellcolor{green!0}{\large 0}/{\footnotesize 40} & \cellcolor{green!0}{\large 0}/{\footnotesize 40} & \cellcolor{green!0}{\large -}{\tiny -} & \cellcolor{green!0}{\large 0}/{\footnotesize 40} & \cellcolor{green!0}{\large 0}/{\footnotesize 80} & \cellcolor{green!0}{\large 0}/{\footnotesize 73} \tabularnewline
64 & igraph\_igraph\_read\_graph\_ncol & 8 & \cellcolor{green!0}{\large 0}/{\footnotesize 40} & \cellcolor{green!0}{\large 0}/{\footnotesize 40} & \cellcolor{green!0}{\large 0}/{\footnotesize 40} & \cellcolor{green!0}{\large 0}/{\footnotesize 40} & \cellcolor{green!0}{\large 0}/{\footnotesize 200} & \cellcolor{green!0}{\large 0}/{\footnotesize 127} \tabularnewline
\rowcolor{black!10} 65 & pjsip\_pj\_json\_parse & 8 & \cellcolor{green!0}{\large 0}/{\footnotesize 40} & \cellcolor{green!0}{\large 0}/{\footnotesize 40} & \cellcolor{green!0}{\large 0}/{\footnotesize 40} & \cellcolor{green!0}{\large 0}/{\footnotesize 40} & \cellcolor{green!0}{\large 0}/{\footnotesize 200} & \cellcolor{green!0}{\large 0}/{\footnotesize 167} \tabularnewline
66 & pjsip\_pj\_xml\_parse & 8 & \cellcolor{green!0}{\large 0}/{\footnotesize 40} & \cellcolor{green!0}{\large 0}/{\footnotesize 40} & \cellcolor{green!0}{\large 0}/{\footnotesize 40} & \cellcolor{green!0}{\large 0}/{\footnotesize 40} & \cellcolor{green!0}{\large 0}/{\footnotesize 200} & \cellcolor{green!0}{\large 0}/{\footnotesize 138} \tabularnewline
\rowcolor{black!10} 67 & pjsip\_pjmedia\_sdp\_parse & 8 & \cellcolor{green!0}{\large 0}/{\footnotesize 40} & \cellcolor{green!0}{\large 0}/{\footnotesize 40} & \cellcolor{green!0}{\large 0}/{\footnotesize 40} & \cellcolor{green!0}{\large 0}/{\footnotesize 40} & \cellcolor{green!0}{\large 0}/{\footnotesize 200} & \cellcolor{green!0}{\large 0}/{\footnotesize 86} \tabularnewline
68 & quickjs\_lre\_compile & 8 & \cellcolor{green!0}{\large 0}/{\footnotesize 40} & \cellcolor{green!0}{\large 0}/{\footnotesize 40} & \cellcolor{green!0}{\large -}{\tiny -} & \cellcolor{green!0}{\large 0}/{\footnotesize 40} & \cellcolor{green!0}{\large 0}/{\footnotesize 200} & \cellcolor{green!0}{\large 0}/{\footnotesize 125} \tabularnewline
\rowcolor{black!10} 69 & bind9\_isc\_lex\_getmastertoken & 9 & \cellcolor{green!0}{\large 0}/{\footnotesize 40} & \cellcolor{green!0}{\large 0}/{\footnotesize 40} & \cellcolor{green!0}{\large -}{\tiny -} & \cellcolor{green!0}{\large 0}/{\footnotesize 40} & \cellcolor{green!0}{\large 0}/{\footnotesize 200} & \cellcolor{green!0}{\large 0}/{\footnotesize 133} \tabularnewline
70 & bind9\_isc\_lex\_gettoken & 9 & \cellcolor{green!0}{\large 0}/{\footnotesize 40} & \cellcolor{green!0}{\large 0}/{\footnotesize 40} & \cellcolor{green!0}{\large -}{\tiny -} & \cellcolor{green!0}{\large 0}/{\footnotesize 40} & \cellcolor{green!0}{\large 0}/{\footnotesize 200} & \cellcolor{green!0}{\large 0}/{\footnotesize 101} \tabularnewline
\rowcolor{black!10} 71 & quickjs\_JS\_Eval & 9 & \cellcolor{green!0}{\large 0}/{\footnotesize 40} & \cellcolor{green!0}{\large 0}/{\footnotesize 40} & \cellcolor{green!0}{\large -}{\tiny -} & \cellcolor{green!0}{\large 0}/{\footnotesize 40} & \cellcolor{green!0}{\large 0}/{\footnotesize 200} & \cellcolor{green!0}{\large 0}/{\footnotesize 161} \tabularnewline
72 & igraph\_igraph\_edge\_connectivity & 10 & \cellcolor{green!0}{\large 0}/{\footnotesize 40} & \cellcolor{green!0}{\large 0}/{\footnotesize 40} & \cellcolor{green!0}{\large 0}/{\footnotesize 40} & \cellcolor{green!0}{\large 0}/{\footnotesize 40} & \cellcolor{green!0}{\large 0}/{\footnotesize 200} & \cellcolor{green!0}{\large 0}/{\footnotesize 196} \tabularnewline
\rowcolor{black!10} 73 & pjsip\_pj\_stun\_msg\_decode & 10 & \cellcolor{green!0}{\large 0}/{\footnotesize 40} & \cellcolor{green!0}{\large 0}/{\footnotesize 40} & \cellcolor{green!0}{\large 0}/{\footnotesize 40} & \cellcolor{green!0}{\large 0}/{\footnotesize 40} & \cellcolor{green!0}{\large 0}/{\footnotesize 197} & \cellcolor{green!0}{\large 0}/{\footnotesize 95} \tabularnewline
74 & bind9\_dns\_message\_checksig & 11 & \cellcolor{green!0}{\large 0}/{\footnotesize 40} & \cellcolor{green!0}{\large 0}/{\footnotesize 40} & \cellcolor{green!0}{\large -}{\tiny -} & \cellcolor{green!0}{\large 0}/{\footnotesize 40} & \cellcolor{green!0}{\large 0}/{\footnotesize 200} & \cellcolor{green!0}{\large 0}/{\footnotesize 95} \tabularnewline
\rowcolor{black!10} 75 & libzip\_zip\_fread & 11 & \cellcolor{green!0}{\large 0}/{\footnotesize 40} & \cellcolor{green!0}{\large 0}/{\footnotesize 40} & \cellcolor{green!0}{\large 0}/{\footnotesize 40} & \cellcolor{green!0}{\large 0}/{\footnotesize 40} & \cellcolor{green!0}{\large 0}/{\footnotesize 200} & \cellcolor{green!0}{\large 0}/{\footnotesize 106} \tabularnewline
76 & bind9\_dns\_rdata\_fromtext & 12 & \cellcolor{green!0}{\large 0}/{\footnotesize 40} & \cellcolor{green!0}{\large 0}/{\footnotesize 40} & \cellcolor{green!0}{\large -}{\tiny -} & \cellcolor{green!0}{\large 0}/{\footnotesize 40} & \cellcolor{green!0}{\large 0}/{\footnotesize 40} & \cellcolor{green!0}{\large 0}/{\footnotesize 59} \tabularnewline
\rowcolor{black!10} 77 & igraph\_igraph\_all\_minimal\_st\_separators & 12 & \cellcolor{green!0}{\large 0}/{\footnotesize 40} & \cellcolor{green!0}{\large 0}/{\footnotesize 40} & \cellcolor{green!0}{\large 0}/{\footnotesize 40} & \cellcolor{green!0}{\large 0}/{\footnotesize 40} & \cellcolor{green!0}{\large 0}/{\footnotesize 80} & \cellcolor{green!0}{\large 0}/{\footnotesize 183} \tabularnewline
78 & igraph\_igraph\_minimum\_size\_separators & 12 & \cellcolor{green!0}{\large 0}/{\footnotesize 40} & \cellcolor{green!0}{\large 0}/{\footnotesize 40} & \cellcolor{green!0}{\large 0}/{\footnotesize 40} & \cellcolor{green!0}{\large 0}/{\footnotesize 40} & \cellcolor{green!0}{\large 0}/{\footnotesize 120} & \cellcolor{green!0}{\large 0}/{\footnotesize 133} \tabularnewline
\rowcolor{black!10} 79 & pjsip\_pjsip\_parse\_msg & 12 & \cellcolor{green!0}{\large 0}/{\footnotesize 40} & \cellcolor{green!0}{\large 0}/{\footnotesize 40} & \cellcolor{green!0}{\large 0}/{\footnotesize 40} & \cellcolor{green!0}{\large 0}/{\footnotesize 40} & \cellcolor{green!0}{\large 0}/{\footnotesize 200} & \cellcolor{green!0}{\large 0}/{\footnotesize 110} \tabularnewline
80 & igraph\_igraph\_automorphism\_group & 13 & \cellcolor{green!0}{\large 0}/{\footnotesize 40} & \cellcolor{green!0}{\large 0}/{\footnotesize 40} & \cellcolor{green!0}{\large 0}/{\footnotesize 40} & \cellcolor{green!0}{\large 0}/{\footnotesize 40} & \cellcolor{green!0}{\large 0}/{\footnotesize 200} & \cellcolor{green!0}{\large 0}/{\footnotesize 96} \tabularnewline
\rowcolor{black!10} 81 & libmodbus\_modbus\_read\_bits & 15 & \cellcolor{green!0}{\large 0}/{\footnotesize 40} & \cellcolor{green!0}{\large 0}/{\footnotesize 40} & \cellcolor{green!0}{\large 0}/{\footnotesize 40} & \cellcolor{green!0}{\large 0}/{\footnotesize 40} & \cellcolor{green!0}{\large 0}/{\footnotesize 40} & \cellcolor{green!0}{\large 0}/{\footnotesize 95} \tabularnewline
82 & libmodbus\_modbus\_read\_registers & 15 & \cellcolor{green!0}{\large 0}/{\footnotesize 40} & \cellcolor{green!0}{\large 0}/{\footnotesize 40} & \cellcolor{green!0}{\large 0}/{\footnotesize 40} & \cellcolor{green!0}{\large 0}/{\footnotesize 40} & \cellcolor{green!0}{\large 0}/{\footnotesize 127} & \cellcolor{green!0}{\large 0}/{\footnotesize 140} \tabularnewline
\rowcolor{black!10} 83 & civetweb\_mg\_get\_response & 17 & \cellcolor{green!0}{\large 0}/{\footnotesize 40} & \cellcolor{green!0}{\large 0}/{\footnotesize 40} & \cellcolor{green!0}{\large 0}/{\footnotesize 40} & \cellcolor{green!0}{\large 0}/{\footnotesize 40} & \cellcolor{green!0}{\large 0}/{\footnotesize 200} & \cellcolor{green!0}{\large 0}/{\footnotesize 95} \tabularnewline
84 & bind9\_dns\_master\_loadbuffer & 20 & \cellcolor{green!0}{\large 0}/{\footnotesize 40} & \cellcolor{green!0}{\large 0}/{\footnotesize 40} & \cellcolor{green!0}{\large -}{\tiny -} & \cellcolor{green!0}{\large 0}/{\footnotesize 40} & \cellcolor{green!0}{\large 0}/{\footnotesize 200} & \cellcolor{green!0}{\large 0}/{\footnotesize 100} \tabularnewline
\rowcolor{black!10} 85 & libmodbus\_modbus\_receive & 33 & \cellcolor{green!0}{\large 0}/{\footnotesize 40} & \cellcolor{green!0}{\large 0}/{\footnotesize 40} & \cellcolor{green!0}{\large 0}/{\footnotesize 40} & \cellcolor{green!0}{\large 0}/{\footnotesize 40} & \cellcolor{green!0}{\large 0}/{\footnotesize 120} & \cellcolor{green!0}{\large 0}/{\footnotesize 96} \tabularnewline
86 & tmux\_input\_parse\_buffer & 42 & \cellcolor{green!0}{\large 0}/{\footnotesize 40} & \cellcolor{green!0}{\large 0}/{\footnotesize 40} & \cellcolor{green!0}{\large -}{\tiny -} & \cellcolor{green!0}{\large 0}/{\footnotesize 40} & \cellcolor{green!0}{\large 0}/{\footnotesize 125} & \cellcolor{green!0}{\large 0}/{\footnotesize 183} \tabularnewline

\bottomrule
%\end{tabular}
%}
%\end{table*}
\end{xltabular}
}
\twocolumn



% model: text-bison-001, temp: 0.5

\onecolumn
{\small %
\begin{xltabular}[h]{\textwidth}{ccccccccc}
%\begin{table*}[!t]
%\centering
\caption{Evaluation Result of model text-bison-001 with temperature 0.5.} \\
%\resizebox{1.0\linewidth}{!}{
%\begin{tabular}{cccccccccc}
\toprule
Index & Question & Score & NAIVE-40 & BACTX-40 & DOCTX-40 & UGCTX-40 & BA-ITER-40 & ALL-ITER-40 \tabularnewline
\midrule
\rowcolor{black!10} 1 & coturn\_stun\_is\_command\_message\_full\_check\_str & 1 & \cellcolor{green!0}{\large 0}/{\footnotesize 40} & \cellcolor{green!100}{\large 39}/{\footnotesize 40} & \cellcolor{green!0}{\large -}{\tiny -} & \cellcolor{green!20}{\large 5}/{\footnotesize 40} & \cellcolor{green!100}{\large 40}/{\footnotesize 40} & \cellcolor{green!30}{\large 13}/{\footnotesize 43} \tabularnewline
2 & kamailio\_parse\_uri & 1 & \cellcolor{green!0}{\large 0}/{\footnotesize 40} & \cellcolor{green!10}{\large 3}/{\footnotesize 40} & \cellcolor{green!0}{\large -}{\tiny -} & \cellcolor{green!20}{\large 6}/{\footnotesize 40} & \cellcolor{green!50}{\large 39}/{\footnotesize 94} & \cellcolor{green!10}{\large 9}/{\footnotesize 112} \tabularnewline
\rowcolor{black!10} 3 & coturn\_stun\_check\_message\_integrity\_str & 2 & \cellcolor{green!0}{\large 0}/{\footnotesize 40} & \cellcolor{green!0}{\large 0}/{\footnotesize 40} & \cellcolor{green!0}{\large -}{\tiny -} & \cellcolor{green!0}{\large 0}/{\footnotesize 40} & \cellcolor{green!0}{\large 0}/{\footnotesize 163} & \cellcolor{green!0}{\large 0}/{\footnotesize 63} \tabularnewline
4 & libiec61850\_MmsValue\_decodeMmsData & 2 & \cellcolor{green!0}{\large 0}/{\footnotesize 40} & \cellcolor{green!30}{\large 9}/{\footnotesize 40} & \cellcolor{green!50}{\large 18}/{\footnotesize 40} & \cellcolor{green!20}{\large 5}/{\footnotesize 40} & \cellcolor{green!20}{\large 13}/{\footnotesize 108} & \cellcolor{green!20}{\large 15}/{\footnotesize 111} \tabularnewline
\rowcolor{black!10} 5 & md4c\_md\_html & 2 & \cellcolor{green!0}{\large 0}/{\footnotesize 40} & \cellcolor{green!0}{\large 0}/{\footnotesize 40} & \cellcolor{green!0}{\large 0}/{\footnotesize 40} & \cellcolor{green!0}{\large 0}/{\footnotesize 40} & \cellcolor{green!0}{\large 0}/{\footnotesize 188} & \cellcolor{green!0}{\large 0}/{\footnotesize 131} \tabularnewline
6 & spdk\_spdk\_json\_parse & 2 & \cellcolor{green!0}{\large 0}/{\footnotesize 40} & \cellcolor{green!50}{\large 17}/{\footnotesize 40} & \cellcolor{green!0}{\large -}{\tiny -} & \cellcolor{green!10}{\large 1}/{\footnotesize 40} & \cellcolor{green!40}{\large 16}/{\footnotesize 40} & \cellcolor{green!10}{\large 5}/{\footnotesize 67} \tabularnewline
\rowcolor{black!10} 7 & croaring\_roaring\_bitmap\_portable\_deserialize\_safe & 3 & \cellcolor{green!50}{\large 18}/{\footnotesize 40} & \cellcolor{green!50}{\large 17}/{\footnotesize 40} & \cellcolor{green!80}{\large 31}/{\footnotesize 40} & \cellcolor{green!50}{\large 17}/{\footnotesize 40} & \cellcolor{green!40}{\large 29}/{\footnotesize 74} & \cellcolor{green!50}{\large 28}/{\footnotesize 64} \tabularnewline
8 & lua\_luaL\_loadbufferx & 3 & \cellcolor{green!90}{\large 35}/{\footnotesize 40} & \cellcolor{green!100}{\large 40}/{\footnotesize 40} & \cellcolor{green!100}{\large 39}/{\footnotesize 40} & \cellcolor{green!70}{\large 27}/{\footnotesize 40} & \cellcolor{green!80}{\large 35}/{\footnotesize 45} & \cellcolor{green!50}{\large 29}/{\footnotesize 67} \tabularnewline
\rowcolor{black!10} 9 & w3m\_wc\_Str\_conv\_with\_detect & 3 & \cellcolor{green!0}{\large 0}/{\footnotesize 40} & \cellcolor{green!0}{\large 0}/{\footnotesize 40} & \cellcolor{green!0}{\large -}{\tiny -} & \cellcolor{green!0}{\large 0}/{\footnotesize 40} & \cellcolor{green!0}{\large 0}/{\footnotesize 200} & \cellcolor{green!0}{\large 0}/{\footnotesize 199} \tabularnewline
10 & bind9\_dns\_name\_fromwire & 4 & \cellcolor{green!0}{\large 0}/{\footnotesize 40} & \cellcolor{green!0}{\large 0}/{\footnotesize 40} & \cellcolor{green!0}{\large -}{\tiny -} & \cellcolor{green!0}{\large 0}/{\footnotesize 40} & \cellcolor{green!0}{\large 0}/{\footnotesize 192} & \cellcolor{green!10}{\large 3}/{\footnotesize 166} \tabularnewline
\rowcolor{black!10} 11 & gdk-pixbuf\_gdk\_pixbuf\_animation\_new\_from\_file & 4 & \cellcolor{green!0}{\large 0}/{\footnotesize 40} & \cellcolor{green!0}{\large 0}/{\footnotesize 40} & \cellcolor{green!0}{\large 0}/{\footnotesize 40} & \cellcolor{green!10}{\large 1}/{\footnotesize 40} & \cellcolor{green!0}{\large 0}/{\footnotesize 132} & \cellcolor{green!0}{\large 0}/{\footnotesize 104} \tabularnewline
12 & gdk-pixbuf\_gdk\_pixbuf\_new\_from\_data & 4 & \cellcolor{green!0}{\large 0}/{\footnotesize 40} & \cellcolor{green!100}{\large 40}/{\footnotesize 40} & \cellcolor{green!100}{\large 40}/{\footnotesize 40} & \cellcolor{green!60}{\large 24}/{\footnotesize 40} & \cellcolor{green!100}{\large 40}/{\footnotesize 40} & \cellcolor{green!60}{\large 31}/{\footnotesize 54} \tabularnewline
\rowcolor{black!10} 13 & gdk-pixbuf\_gdk\_pixbuf\_new\_from\_file & 4 & \cellcolor{green!10}{\large 2}/{\footnotesize 40} & \cellcolor{green!0}{\large 0}/{\footnotesize 40} & \cellcolor{green!10}{\large 1}/{\footnotesize 40} & \cellcolor{green!0}{\large 0}/{\footnotesize 40} & \cellcolor{green!0}{\large 0}/{\footnotesize 147} & \cellcolor{green!0}{\large 0}/{\footnotesize 96} \tabularnewline
14 & gdk-pixbuf\_gdk\_pixbuf\_new\_from\_stream & 4 & \cellcolor{green!40}{\large 16}/{\footnotesize 40} & \cellcolor{green!100}{\large 38}/{\footnotesize 40} & \cellcolor{green!90}{\large 36}/{\footnotesize 40} & \cellcolor{green!90}{\large 35}/{\footnotesize 40} & \cellcolor{green!90}{\large 38}/{\footnotesize 44} & \cellcolor{green!60}{\large 30}/{\footnotesize 57} \tabularnewline
\rowcolor{black!10} 15 & gpac\_gf\_isom\_open\_file & 4 & \cellcolor{green!0}{\large 0}/{\footnotesize 40} & \cellcolor{green!0}{\large 0}/{\footnotesize 40} & \cellcolor{green!0}{\large -}{\tiny -} & \cellcolor{green!0}{\large 0}/{\footnotesize 40} & \cellcolor{green!0}{\large 0}/{\footnotesize 200} & \cellcolor{green!0}{\large 0}/{\footnotesize 177} \tabularnewline
16 & libbpf\_bpf\_object\_\_open\_mem & 4 & \cellcolor{green!0}{\large 0}/{\footnotesize 40} & \cellcolor{green!100}{\large 37}/{\footnotesize 40} & \cellcolor{green!80}{\large 29}/{\footnotesize 40} & \cellcolor{green!30}{\large 12}/{\footnotesize 40} & \cellcolor{green!80}{\large 37}/{\footnotesize 52} & \cellcolor{green!20}{\large 13}/{\footnotesize 91} \tabularnewline
\rowcolor{black!10} 17 & libpg\_query\_pg\_query\_parse & 4 & \cellcolor{green!0}{\large 0}/{\footnotesize 40} & \cellcolor{green!0}{\large 0}/{\footnotesize 40} & \cellcolor{green!0}{\large -}{\tiny -} & \cellcolor{green!30}{\large 9}/{\footnotesize 40} & \cellcolor{green!0}{\large 0}/{\footnotesize 179} & \cellcolor{green!10}{\large 13}/{\footnotesize 126} \tabularnewline
18 & libucl\_ucl\_parser\_add\_string & 4 & \cellcolor{green!0}{\large 0}/{\footnotesize 40} & \cellcolor{green!0}{\large 0}/{\footnotesize 40} & \cellcolor{green!0}{\large 0}/{\footnotesize 40} & \cellcolor{green!0}{\large 0}/{\footnotesize 40} & \cellcolor{green!0}{\large 1}/{\footnotesize 200} & \cellcolor{green!10}{\large 2}/{\footnotesize 165} \tabularnewline
\rowcolor{black!10} 19 & oniguruma\_onig\_new & 4 & \cellcolor{green!0}{\large 0}/{\footnotesize 40} & \cellcolor{green!0}{\large 0}/{\footnotesize 40} & \cellcolor{green!0}{\large 0}/{\footnotesize 40} & \cellcolor{green!10}{\large 1}/{\footnotesize 40} & \cellcolor{green!0}{\large 0}/{\footnotesize 196} & \cellcolor{green!10}{\large 6}/{\footnotesize 128} \tabularnewline
20 & pupnp\_ixmlLoadDocumentEx & 4 & \cellcolor{green!0}{\large 0}/{\footnotesize 40} & \cellcolor{green!0}{\large 0}/{\footnotesize 40} & \cellcolor{green!0}{\large 0}/{\footnotesize 40} & \cellcolor{green!0}{\large 0}/{\footnotesize 40} & \cellcolor{green!0}{\large 0}/{\footnotesize 96} & \cellcolor{green!0}{\large 0}/{\footnotesize 73} \tabularnewline
\rowcolor{black!10} 21 & gdk-pixbuf\_gdk\_pixbuf\_new\_from\_file\_at\_scale & 5 & \cellcolor{green!0}{\large 0}/{\footnotesize 40} & \cellcolor{green!0}{\large 0}/{\footnotesize 40} & \cellcolor{green!0}{\large 0}/{\footnotesize 40} & \cellcolor{green!0}{\large 0}/{\footnotesize 40} & \cellcolor{green!0}{\large 0}/{\footnotesize 189} & \cellcolor{green!0}{\large 0}/{\footnotesize 120} \tabularnewline
22 & inchi\_GetINCHIKeyFromINCHI & 5 & \cellcolor{green!0}{\large 0}/{\footnotesize 40} & \cellcolor{green!30}{\large 11}/{\footnotesize 40} & \cellcolor{green!10}{\large 3}/{\footnotesize 40} & \cellcolor{green!10}{\large 3}/{\footnotesize 40} & \cellcolor{green!10}{\large 8}/{\footnotesize 125} & \cellcolor{green!10}{\large 3}/{\footnotesize 116} \tabularnewline
\rowcolor{black!10} 23 & libdwarf\_dwarf\_init\_b & 5 & \cellcolor{green!0}{\large 0}/{\footnotesize 40} & \cellcolor{green!0}{\large 0}/{\footnotesize 40} & \cellcolor{green!0}{\large 0}/{\footnotesize 40} & \cellcolor{green!30}{\large 10}/{\footnotesize 40} & \cellcolor{green!0}{\large 1}/{\footnotesize 155} & \cellcolor{green!20}{\large 13}/{\footnotesize 96} \tabularnewline
24 & libdwarf\_dwarf\_init\_path & 5 & \cellcolor{green!0}{\large 0}/{\footnotesize 40} & \cellcolor{green!0}{\large 0}/{\footnotesize 40} & \cellcolor{green!0}{\large 0}/{\footnotesize 40} & \cellcolor{green!10}{\large 2}/{\footnotesize 40} & \cellcolor{green!0}{\large 0}/{\footnotesize 116} & \cellcolor{green!0}{\large 0}/{\footnotesize 110} \tabularnewline
\rowcolor{black!10} 25 & liblouis\_lou\_compileString & 5 & \cellcolor{green!0}{\large 0}/{\footnotesize 40} & \cellcolor{green!0}{\large 0}/{\footnotesize 40} & \cellcolor{green!0}{\large 0}/{\footnotesize 40} & \cellcolor{green!10}{\large 1}/{\footnotesize 40} & \cellcolor{green!10}{\large 3}/{\footnotesize 177} & \cellcolor{green!0}{\large 1}/{\footnotesize 160} \tabularnewline
26 & selinux\_cil\_compile & 5 & \cellcolor{green!0}{\large 0}/{\footnotesize 40} & \cellcolor{green!0}{\large 0}/{\footnotesize 40} & \cellcolor{green!0}{\large -}{\tiny -} & \cellcolor{green!20}{\large 5}/{\footnotesize 40} & \cellcolor{green!0}{\large 0}/{\footnotesize 152} & \cellcolor{green!10}{\large 4}/{\footnotesize 55} \tabularnewline
\rowcolor{black!10} 27 & bind9\_dns\_name\_fromtext & 6 & \cellcolor{green!0}{\large 0}/{\footnotesize 40} & \cellcolor{green!0}{\large 0}/{\footnotesize 40} & \cellcolor{green!0}{\large -}{\tiny -} & \cellcolor{green!10}{\large 2}/{\footnotesize 40} & \cellcolor{green!0}{\large 0}/{\footnotesize 188} & \cellcolor{green!10}{\large 5}/{\footnotesize 108} \tabularnewline
28 & bind9\_dns\_rdata\_fromwire & 6 & \cellcolor{green!0}{\large 0}/{\footnotesize 40} & \cellcolor{green!0}{\large 0}/{\footnotesize 40} & \cellcolor{green!0}{\large -}{\tiny -} & \cellcolor{green!0}{\large 0}/{\footnotesize 40} & \cellcolor{green!0}{\large 0}/{\footnotesize 196} & \cellcolor{green!0}{\large 0}/{\footnotesize 137} \tabularnewline
\rowcolor{black!10} 29 & coturn\_stun\_is\_binding\_response & 6 & \cellcolor{green!0}{\large 0}/{\footnotesize 40} & \cellcolor{green!0}{\large 0}/{\footnotesize 40} & \cellcolor{green!0}{\large -}{\tiny -} & \cellcolor{green!10}{\large 1}/{\footnotesize 40} & \cellcolor{green!0}{\large 0}/{\footnotesize 131} & \cellcolor{green!0}{\large 0}/{\footnotesize 72} \tabularnewline
30 & coturn\_stun\_is\_command\_message & 6 & \cellcolor{green!0}{\large 0}/{\footnotesize 40} & \cellcolor{green!0}{\large 0}/{\footnotesize 40} & \cellcolor{green!0}{\large 0}/{\footnotesize 40} & \cellcolor{green!0}{\large 0}/{\footnotesize 40} & \cellcolor{green!0}{\large 0}/{\footnotesize 125} & \cellcolor{green!10}{\large 1}/{\footnotesize 61} \tabularnewline
\rowcolor{black!10} 31 & coturn\_stun\_is\_response & 6 & \cellcolor{green!0}{\large 0}/{\footnotesize 40} & \cellcolor{green!0}{\large 0}/{\footnotesize 40} & \cellcolor{green!0}{\large -}{\tiny -} & \cellcolor{green!0}{\large 0}/{\footnotesize 40} & \cellcolor{green!0}{\large 0}/{\footnotesize 89} & \cellcolor{green!0}{\large 0}/{\footnotesize 73} \tabularnewline
32 & coturn\_stun\_is\_success\_response & 6 & \cellcolor{green!0}{\large 0}/{\footnotesize 40} & \cellcolor{green!0}{\large 0}/{\footnotesize 40} & \cellcolor{green!0}{\large -}{\tiny -} & \cellcolor{green!10}{\large 2}/{\footnotesize 40} & \cellcolor{green!0}{\large 0}/{\footnotesize 119} & \cellcolor{green!10}{\large 2}/{\footnotesize 73} \tabularnewline
\rowcolor{black!10} 33 & hiredis\_redisFormatCommand & 6 & \cellcolor{green!40}{\large 13}/{\footnotesize 40} & \cellcolor{green!100}{\large 40}/{\footnotesize 40} & \cellcolor{green!0}{\large -}{\tiny -} & \cellcolor{green!50}{\large 19}/{\footnotesize 40} & \cellcolor{green!20}{\large 35}/{\footnotesize 196} & \cellcolor{green!10}{\large 4}/{\footnotesize 109} \tabularnewline
34 & igraph\_igraph\_read\_graph\_dl & 6 & \cellcolor{green!0}{\large 0}/{\footnotesize 40} & \cellcolor{green!0}{\large 0}/{\footnotesize 40} & \cellcolor{green!0}{\large 0}/{\footnotesize 40} & \cellcolor{green!0}{\large 0}/{\footnotesize 40} & \cellcolor{green!0}{\large 0}/{\footnotesize 178} & \cellcolor{green!0}{\large 0}/{\footnotesize 151} \tabularnewline
\rowcolor{black!10} 35 & igraph\_igraph\_read\_graph\_edgelist & 6 & \cellcolor{green!0}{\large 0}/{\footnotesize 40} & \cellcolor{green!0}{\large 0}/{\footnotesize 40} & \cellcolor{green!0}{\large 0}/{\footnotesize 40} & \cellcolor{green!0}{\large 0}/{\footnotesize 40} & \cellcolor{green!0}{\large 0}/{\footnotesize 165} & \cellcolor{green!0}{\large 0}/{\footnotesize 153} \tabularnewline
36 & igraph\_igraph\_read\_graph\_gml & 6 & \cellcolor{green!0}{\large 0}/{\footnotesize 40} & \cellcolor{green!0}{\large 0}/{\footnotesize 40} & \cellcolor{green!0}{\large 0}/{\footnotesize 40} & \cellcolor{green!0}{\large 0}/{\footnotesize 40} & \cellcolor{green!0}{\large 0}/{\footnotesize 187} & \cellcolor{green!0}{\large 0}/{\footnotesize 148} \tabularnewline
\rowcolor{black!10} 37 & igraph\_igraph\_read\_graph\_graphdb & 6 & \cellcolor{green!0}{\large 0}/{\footnotesize 40} & \cellcolor{green!0}{\large 0}/{\footnotesize 40} & \cellcolor{green!0}{\large 0}/{\footnotesize 40} & \cellcolor{green!0}{\large 0}/{\footnotesize 40} & \cellcolor{green!0}{\large 0}/{\footnotesize 193} & \cellcolor{green!0}{\large 0}/{\footnotesize 154} \tabularnewline
38 & igraph\_igraph\_read\_graph\_graphml & 6 & \cellcolor{green!0}{\large 0}/{\footnotesize 40} & \cellcolor{green!0}{\large 0}/{\footnotesize 40} & \cellcolor{green!0}{\large 0}/{\footnotesize 40} & \cellcolor{green!0}{\large 0}/{\footnotesize 40} & \cellcolor{green!0}{\large 0}/{\footnotesize 168} & \cellcolor{green!0}{\large 0}/{\footnotesize 158} \tabularnewline
\rowcolor{black!10} 39 & igraph\_igraph\_read\_graph\_lgl & 6 & \cellcolor{green!0}{\large 0}/{\footnotesize 40} & \cellcolor{green!0}{\large 0}/{\footnotesize 40} & \cellcolor{green!0}{\large 0}/{\footnotesize 40} & \cellcolor{green!0}{\large 0}/{\footnotesize 40} & \cellcolor{green!0}{\large 0}/{\footnotesize 183} & \cellcolor{green!0}{\large 0}/{\footnotesize 106} \tabularnewline
40 & igraph\_igraph\_read\_graph\_pajek & 6 & \cellcolor{green!0}{\large 0}/{\footnotesize 40} & \cellcolor{green!0}{\large 0}/{\footnotesize 40} & \cellcolor{green!0}{\large 0}/{\footnotesize 40} & \cellcolor{green!0}{\large 0}/{\footnotesize 40} & \cellcolor{green!0}{\large 0}/{\footnotesize 177} & \cellcolor{green!0}{\large 0}/{\footnotesize 131} \tabularnewline
\rowcolor{black!10} 41 & inchi\_GetINCHIfromINCHI & 6 & \cellcolor{green!0}{\large 0}/{\footnotesize 40} & \cellcolor{green!0}{\large 0}/{\footnotesize 40} & \cellcolor{green!0}{\large 0}/{\footnotesize 40} & \cellcolor{green!10}{\large 3}/{\footnotesize 40} & \cellcolor{green!0}{\large 0}/{\footnotesize 180} & \cellcolor{green!0}{\large 0}/{\footnotesize 128} \tabularnewline
42 & inchi\_GetStructFromINCHI & 6 & \cellcolor{green!0}{\large 0}/{\footnotesize 40} & \cellcolor{green!0}{\large 0}/{\footnotesize 40} & \cellcolor{green!0}{\large 0}/{\footnotesize 40} & \cellcolor{green!0}{\large 0}/{\footnotesize 40} & \cellcolor{green!0}{\large 0}/{\footnotesize 184} & \cellcolor{green!0}{\large 1}/{\footnotesize 181} \tabularnewline
\rowcolor{black!10} 43 & kamailio\_parse\_msg & 6 & \cellcolor{green!0}{\large 0}/{\footnotesize 40} & \cellcolor{green!10}{\large 1}/{\footnotesize 40} & \cellcolor{green!0}{\large -}{\tiny -} & \cellcolor{green!30}{\large 11}/{\footnotesize 40} & \cellcolor{green!10}{\large 8}/{\footnotesize 174} & \cellcolor{green!20}{\large 12}/{\footnotesize 96} \tabularnewline
44 & libyang\_lys\_parse\_mem & 6 & \cellcolor{green!0}{\large 0}/{\footnotesize 40} & \cellcolor{green!0}{\large 0}/{\footnotesize 40} & \cellcolor{green!0}{\large 0}/{\footnotesize 40} & \cellcolor{green!0}{\large 0}/{\footnotesize 40} & \cellcolor{green!0}{\large 0}/{\footnotesize 196} & \cellcolor{green!0}{\large 0}/{\footnotesize 142} \tabularnewline
\rowcolor{black!10} 45 & proftpd\_pr\_json\_object\_from\_text & 6 & \cellcolor{green!0}{\large 0}/{\footnotesize 40} & \cellcolor{green!0}{\large 0}/{\footnotesize 40} & \cellcolor{green!0}{\large -}{\tiny -} & \cellcolor{green!0}{\large 0}/{\footnotesize 40} & \cellcolor{green!0}{\large 0}/{\footnotesize 200} & \cellcolor{green!0}{\large 1}/{\footnotesize 101} \tabularnewline
46 & selinux\_policydb\_read & 6 & \cellcolor{green!0}{\large 0}/{\footnotesize 40} & \cellcolor{green!10}{\large 2}/{\footnotesize 40} & \cellcolor{green!0}{\large -}{\tiny -} & \cellcolor{green!10}{\large 1}/{\footnotesize 40} & \cellcolor{green!0}{\large 0}/{\footnotesize 193} & \cellcolor{green!10}{\large 4}/{\footnotesize 117} \tabularnewline
\rowcolor{black!10} 47 & kamailio\_get\_src\_address\_socket & 7 & \cellcolor{green!0}{\large 0}/{\footnotesize 40} & \cellcolor{green!0}{\large 0}/{\footnotesize 40} & \cellcolor{green!0}{\large 0}/{\footnotesize 40} & \cellcolor{green!0}{\large 0}/{\footnotesize 40} & \cellcolor{green!0}{\large 0}/{\footnotesize 187} & \cellcolor{green!10}{\large 3}/{\footnotesize 128} \tabularnewline
48 & kamailio\_get\_src\_uri & 7 & \cellcolor{green!0}{\large 0}/{\footnotesize 40} & \cellcolor{green!0}{\large 0}/{\footnotesize 40} & \cellcolor{green!0}{\large 0}/{\footnotesize 40} & \cellcolor{green!0}{\large 0}/{\footnotesize 40} & \cellcolor{green!0}{\large 0}/{\footnotesize 193} & \cellcolor{green!0}{\large 0}/{\footnotesize 162} \tabularnewline
\rowcolor{black!10} 49 & kamailio\_parse\_content\_disposition & 7 & \cellcolor{green!0}{\large 0}/{\footnotesize 40} & \cellcolor{green!0}{\large 0}/{\footnotesize 40} & \cellcolor{green!0}{\large 0}/{\footnotesize 40} & \cellcolor{green!10}{\large 2}/{\footnotesize 40} & \cellcolor{green!0}{\large 0}/{\footnotesize 171} & \cellcolor{green!0}{\large 1}/{\footnotesize 105} \tabularnewline
50 & kamailio\_parse\_diversion\_header & 7 & \cellcolor{green!0}{\large 0}/{\footnotesize 40} & \cellcolor{green!0}{\large 0}/{\footnotesize 40} & \cellcolor{green!0}{\large 0}/{\footnotesize 40} & \cellcolor{green!10}{\large 1}/{\footnotesize 40} & \cellcolor{green!0}{\large 0}/{\footnotesize 176} & \cellcolor{green!10}{\large 2}/{\footnotesize 135} \tabularnewline
\rowcolor{black!10} 51 & kamailio\_parse\_from\_header & 7 & \cellcolor{green!0}{\large 0}/{\footnotesize 40} & \cellcolor{green!0}{\large 0}/{\footnotesize 40} & \cellcolor{green!0}{\large -}{\tiny -} & \cellcolor{green!0}{\large 0}/{\footnotesize 40} & \cellcolor{green!0}{\large 0}/{\footnotesize 184} & \cellcolor{green!10}{\large 2}/{\footnotesize 140} \tabularnewline
52 & kamailio\_parse\_from\_uri & 7 & \cellcolor{green!0}{\large 0}/{\footnotesize 40} & \cellcolor{green!0}{\large 0}/{\footnotesize 40} & \cellcolor{green!0}{\large -}{\tiny -} & \cellcolor{green!0}{\large 0}/{\footnotesize 40} & \cellcolor{green!0}{\large 0}/{\footnotesize 199} & \cellcolor{green!0}{\large 0}/{\footnotesize 166} \tabularnewline
\rowcolor{black!10} 53 & kamailio\_parse\_headers & 7 & \cellcolor{green!0}{\large 0}/{\footnotesize 40} & \cellcolor{green!0}{\large 0}/{\footnotesize 40} & \cellcolor{green!0}{\large -}{\tiny -} & \cellcolor{green!0}{\large 0}/{\footnotesize 40} & \cellcolor{green!0}{\large 0}/{\footnotesize 176} & \cellcolor{green!0}{\large 0}/{\footnotesize 155} \tabularnewline
54 & kamailio\_parse\_identityinfo\_header & 7 & \cellcolor{green!0}{\large 0}/{\footnotesize 40} & \cellcolor{green!0}{\large 0}/{\footnotesize 40} & \cellcolor{green!0}{\large -}{\tiny -} & \cellcolor{green!0}{\large 0}/{\footnotesize 40} & \cellcolor{green!0}{\large 0}/{\footnotesize 175} & \cellcolor{green!0}{\large 1}/{\footnotesize 138} \tabularnewline
\rowcolor{black!10} 55 & kamailio\_parse\_pai\_header & 7 & \cellcolor{green!0}{\large 0}/{\footnotesize 40} & \cellcolor{green!0}{\large 0}/{\footnotesize 40} & \cellcolor{green!0}{\large -}{\tiny -} & \cellcolor{green!10}{\large 4}/{\footnotesize 40} & \cellcolor{green!0}{\large 0}/{\footnotesize 149} & \cellcolor{green!0}{\large 1}/{\footnotesize 123} \tabularnewline
56 & kamailio\_parse\_privacy & 7 & \cellcolor{green!0}{\large 0}/{\footnotesize 40} & \cellcolor{green!0}{\large 0}/{\footnotesize 40} & \cellcolor{green!0}{\large 0}/{\footnotesize 40} & \cellcolor{green!0}{\large 0}/{\footnotesize 40} & \cellcolor{green!0}{\large 0}/{\footnotesize 172} & \cellcolor{green!0}{\large 1}/{\footnotesize 121} \tabularnewline
\rowcolor{black!10} 57 & kamailio\_parse\_record\_route\_headers & 7 & \cellcolor{green!0}{\large 0}/{\footnotesize 40} & \cellcolor{green!0}{\large 0}/{\footnotesize 40} & \cellcolor{green!0}{\large -}{\tiny -} & \cellcolor{green!0}{\large 0}/{\footnotesize 40} & \cellcolor{green!0}{\large 0}/{\footnotesize 185} & \cellcolor{green!10}{\large 1}/{\footnotesize 94} \tabularnewline
58 & kamailio\_parse\_refer\_to\_header & 7 & \cellcolor{green!0}{\large 0}/{\footnotesize 40} & \cellcolor{green!0}{\large 0}/{\footnotesize 40} & \cellcolor{green!0}{\large -}{\tiny -} & \cellcolor{green!0}{\large 0}/{\footnotesize 40} & \cellcolor{green!0}{\large 0}/{\footnotesize 173} & \cellcolor{green!0}{\large 1}/{\footnotesize 156} \tabularnewline
\rowcolor{black!10} 59 & kamailio\_parse\_route\_headers & 7 & \cellcolor{green!0}{\large 0}/{\footnotesize 40} & \cellcolor{green!0}{\large 0}/{\footnotesize 40} & \cellcolor{green!0}{\large -}{\tiny -} & \cellcolor{green!0}{\large 0}/{\footnotesize 40} & \cellcolor{green!0}{\large 0}/{\footnotesize 189} & \cellcolor{green!10}{\large 3}/{\footnotesize 90} \tabularnewline
60 & kamailio\_parse\_to\_header & 7 & \cellcolor{green!0}{\large 0}/{\footnotesize 40} & \cellcolor{green!0}{\large 0}/{\footnotesize 40} & \cellcolor{green!0}{\large -}{\tiny -} & \cellcolor{green!0}{\large 0}/{\footnotesize 40} & \cellcolor{green!0}{\large 0}/{\footnotesize 163} & \cellcolor{green!0}{\large 1}/{\footnotesize 127} \tabularnewline
\rowcolor{black!10} 61 & kamailio\_parse\_to\_uri & 7 & \cellcolor{green!0}{\large 0}/{\footnotesize 40} & \cellcolor{green!0}{\large 0}/{\footnotesize 40} & \cellcolor{green!0}{\large -}{\tiny -} & \cellcolor{green!0}{\large 0}/{\footnotesize 40} & \cellcolor{green!0}{\large 0}/{\footnotesize 188} & \cellcolor{green!0}{\large 0}/{\footnotesize 127} \tabularnewline
62 & libyang\_lyd\_parse\_data\_mem & 7 & \cellcolor{green!0}{\large 0}/{\footnotesize 40} & \cellcolor{green!0}{\large 0}/{\footnotesize 40} & \cellcolor{green!0}{\large 0}/{\footnotesize 40} & \cellcolor{green!0}{\large 0}/{\footnotesize 40} & \cellcolor{green!0}{\large 0}/{\footnotesize 196} & \cellcolor{green!0}{\large 0}/{\footnotesize 141} \tabularnewline
\rowcolor{black!10} 63 & bind9\_dns\_message\_parse & 8 & \cellcolor{green!0}{\large 0}/{\footnotesize 40} & \cellcolor{green!0}{\large 0}/{\footnotesize 40} & \cellcolor{green!0}{\large -}{\tiny -} & \cellcolor{green!0}{\large 0}/{\footnotesize 40} & \cellcolor{green!0}{\large 0}/{\footnotesize 186} & \cellcolor{green!0}{\large 0}/{\footnotesize 116} \tabularnewline
64 & igraph\_igraph\_read\_graph\_ncol & 8 & \cellcolor{green!0}{\large 0}/{\footnotesize 40} & \cellcolor{green!0}{\large 0}/{\footnotesize 40} & \cellcolor{green!0}{\large 0}/{\footnotesize 40} & \cellcolor{green!0}{\large 0}/{\footnotesize 40} & \cellcolor{green!0}{\large 0}/{\footnotesize 174} & \cellcolor{green!0}{\large 0}/{\footnotesize 93} \tabularnewline
\rowcolor{black!10} 65 & pjsip\_pj\_json\_parse & 8 & \cellcolor{green!0}{\large 0}/{\footnotesize 40} & \cellcolor{green!0}{\large 0}/{\footnotesize 40} & \cellcolor{green!0}{\large 0}/{\footnotesize 40} & \cellcolor{green!0}{\large 0}/{\footnotesize 40} & \cellcolor{green!0}{\large 0}/{\footnotesize 194} & \cellcolor{green!0}{\large 0}/{\footnotesize 176} \tabularnewline
66 & pjsip\_pj\_xml\_parse & 8 & \cellcolor{green!0}{\large 0}/{\footnotesize 40} & \cellcolor{green!0}{\large 0}/{\footnotesize 40} & \cellcolor{green!0}{\large 0}/{\footnotesize 40} & \cellcolor{green!0}{\large 0}/{\footnotesize 40} & \cellcolor{green!0}{\large 0}/{\footnotesize 193} & \cellcolor{green!0}{\large 0}/{\footnotesize 137} \tabularnewline
\rowcolor{black!10} 67 & pjsip\_pjmedia\_sdp\_parse & 8 & \cellcolor{green!0}{\large 0}/{\footnotesize 40} & \cellcolor{green!0}{\large 0}/{\footnotesize 40} & \cellcolor{green!0}{\large 0}/{\footnotesize 40} & \cellcolor{green!0}{\large 0}/{\footnotesize 40} & \cellcolor{green!0}{\large 0}/{\footnotesize 181} & \cellcolor{green!10}{\large 1}/{\footnotesize 100} \tabularnewline
68 & quickjs\_lre\_compile & 8 & \cellcolor{green!0}{\large 0}/{\footnotesize 40} & \cellcolor{green!0}{\large 0}/{\footnotesize 40} & \cellcolor{green!0}{\large -}{\tiny -} & \cellcolor{green!0}{\large 0}/{\footnotesize 40} & \cellcolor{green!0}{\large 0}/{\footnotesize 200} & \cellcolor{green!0}{\large 0}/{\footnotesize 156} \tabularnewline
\rowcolor{black!10} 69 & bind9\_isc\_lex\_getmastertoken & 9 & \cellcolor{green!0}{\large 0}/{\footnotesize 40} & \cellcolor{green!0}{\large 0}/{\footnotesize 40} & \cellcolor{green!0}{\large -}{\tiny -} & \cellcolor{green!0}{\large 0}/{\footnotesize 40} & \cellcolor{green!0}{\large 0}/{\footnotesize 192} & \cellcolor{green!0}{\large 0}/{\footnotesize 144} \tabularnewline
70 & bind9\_isc\_lex\_gettoken & 9 & \cellcolor{green!0}{\large 0}/{\footnotesize 40} & \cellcolor{green!0}{\large 0}/{\footnotesize 40} & \cellcolor{green!0}{\large -}{\tiny -} & \cellcolor{green!0}{\large 0}/{\footnotesize 40} & \cellcolor{green!0}{\large 0}/{\footnotesize 200} & \cellcolor{green!0}{\large 0}/{\footnotesize 119} \tabularnewline
\rowcolor{black!10} 71 & quickjs\_JS\_Eval & 9 & \cellcolor{green!0}{\large 0}/{\footnotesize 40} & \cellcolor{green!0}{\large 0}/{\footnotesize 40} & \cellcolor{green!0}{\large -}{\tiny -} & \cellcolor{green!0}{\large 0}/{\footnotesize 40} & \cellcolor{green!0}{\large 0}/{\footnotesize 191} & \cellcolor{green!0}{\large 0}/{\footnotesize 137} \tabularnewline
72 & igraph\_igraph\_edge\_connectivity & 10 & \cellcolor{green!0}{\large 0}/{\footnotesize 40} & \cellcolor{green!0}{\large 0}/{\footnotesize 40} & \cellcolor{green!0}{\large 0}/{\footnotesize 40} & \cellcolor{green!0}{\large 0}/{\footnotesize 40} & \cellcolor{green!0}{\large 0}/{\footnotesize 186} & \cellcolor{green!0}{\large 0}/{\footnotesize 193} \tabularnewline
\rowcolor{black!10} 73 & pjsip\_pj\_stun\_msg\_decode & 10 & \cellcolor{green!0}{\large 0}/{\footnotesize 40} & \cellcolor{green!0}{\large 0}/{\footnotesize 40} & \cellcolor{green!0}{\large 0}/{\footnotesize 40} & \cellcolor{green!0}{\large 0}/{\footnotesize 40} & \cellcolor{green!0}{\large 0}/{\footnotesize 189} & \cellcolor{green!0}{\large 0}/{\footnotesize 91} \tabularnewline
74 & bind9\_dns\_message\_checksig & 11 & \cellcolor{green!0}{\large 0}/{\footnotesize 40} & \cellcolor{green!0}{\large 0}/{\footnotesize 40} & \cellcolor{green!0}{\large -}{\tiny -} & \cellcolor{green!0}{\large 0}/{\footnotesize 40} & \cellcolor{green!0}{\large 0}/{\footnotesize 189} & \cellcolor{green!0}{\large 0}/{\footnotesize 99} \tabularnewline
\rowcolor{black!10} 75 & libzip\_zip\_fread & 11 & \cellcolor{green!0}{\large 0}/{\footnotesize 40} & \cellcolor{green!0}{\large 0}/{\footnotesize 40} & \cellcolor{green!0}{\large 0}/{\footnotesize 40} & \cellcolor{green!0}{\large 0}/{\footnotesize 40} & \cellcolor{green!0}{\large 0}/{\footnotesize 193} & \cellcolor{green!0}{\large 0}/{\footnotesize 127} \tabularnewline
76 & bind9\_dns\_rdata\_fromtext & 12 & \cellcolor{green!0}{\large 0}/{\footnotesize 40} & \cellcolor{green!0}{\large 0}/{\footnotesize 40} & \cellcolor{green!0}{\large -}{\tiny -} & \cellcolor{green!0}{\large 0}/{\footnotesize 40} & \cellcolor{green!0}{\large 0}/{\footnotesize 160} & \cellcolor{green!0}{\large 0}/{\footnotesize 85} \tabularnewline
\rowcolor{black!10} 77 & igraph\_igraph\_all\_minimal\_st\_separators & 12 & \cellcolor{green!0}{\large 0}/{\footnotesize 40} & \cellcolor{green!0}{\large 0}/{\footnotesize 40} & \cellcolor{green!0}{\large 0}/{\footnotesize 40} & \cellcolor{green!0}{\large 0}/{\footnotesize 40} & \cellcolor{green!0}{\large 0}/{\footnotesize 173} & \cellcolor{green!0}{\large 0}/{\footnotesize 188} \tabularnewline
78 & igraph\_igraph\_minimum\_size\_separators & 12 & \cellcolor{green!0}{\large 0}/{\footnotesize 40} & \cellcolor{green!0}{\large 0}/{\footnotesize 40} & \cellcolor{green!0}{\large 0}/{\footnotesize 40} & \cellcolor{green!0}{\large 0}/{\footnotesize 40} & \cellcolor{green!0}{\large 0}/{\footnotesize 163} & \cellcolor{green!0}{\large 0}/{\footnotesize 113} \tabularnewline
\rowcolor{black!10} 79 & pjsip\_pjsip\_parse\_msg & 12 & \cellcolor{green!0}{\large 0}/{\footnotesize 40} & \cellcolor{green!0}{\large 0}/{\footnotesize 40} & \cellcolor{green!0}{\large 0}/{\footnotesize 40} & \cellcolor{green!0}{\large 0}/{\footnotesize 40} & \cellcolor{green!0}{\large 0}/{\footnotesize 197} & \cellcolor{green!0}{\large 0}/{\footnotesize 110} \tabularnewline
80 & igraph\_igraph\_automorphism\_group & 13 & \cellcolor{green!0}{\large 0}/{\footnotesize 40} & \cellcolor{green!0}{\large 0}/{\footnotesize 40} & \cellcolor{green!0}{\large 0}/{\footnotesize 40} & \cellcolor{green!0}{\large 0}/{\footnotesize 40} & \cellcolor{green!0}{\large 0}/{\footnotesize 164} & \cellcolor{green!0}{\large 0}/{\footnotesize 76} \tabularnewline
\rowcolor{black!10} 81 & libmodbus\_modbus\_read\_bits & 15 & \cellcolor{green!0}{\large 0}/{\footnotesize 40} & \cellcolor{green!0}{\large 0}/{\footnotesize 40} & \cellcolor{green!0}{\large 0}/{\footnotesize 40} & \cellcolor{green!0}{\large 0}/{\footnotesize 40} & \cellcolor{green!0}{\large 0}/{\footnotesize 130} & \cellcolor{green!0}{\large 0}/{\footnotesize 108} \tabularnewline
82 & libmodbus\_modbus\_read\_registers & 15 & \cellcolor{green!0}{\large 0}/{\footnotesize 40} & \cellcolor{green!0}{\large 0}/{\footnotesize 40} & \cellcolor{green!0}{\large 0}/{\footnotesize 40} & \cellcolor{green!0}{\large 0}/{\footnotesize 40} & \cellcolor{green!0}{\large 0}/{\footnotesize 124} & \cellcolor{green!0}{\large 0}/{\footnotesize 159} \tabularnewline
\rowcolor{black!10} 83 & civetweb\_mg\_get\_response & 17 & \cellcolor{green!0}{\large 0}/{\footnotesize 40} & \cellcolor{green!0}{\large 0}/{\footnotesize 40} & \cellcolor{green!0}{\large 0}/{\footnotesize 40} & \cellcolor{green!0}{\large 0}/{\footnotesize 40} & \cellcolor{green!0}{\large 0}/{\footnotesize 200} & \cellcolor{green!0}{\large 0}/{\footnotesize 90} \tabularnewline
84 & bind9\_dns\_master\_loadbuffer & 20 & \cellcolor{green!0}{\large 0}/{\footnotesize 40} & \cellcolor{green!0}{\large 0}/{\footnotesize 40} & \cellcolor{green!0}{\large -}{\tiny -} & \cellcolor{green!0}{\large 0}/{\footnotesize 40} & \cellcolor{green!0}{\large 0}/{\footnotesize 171} & \cellcolor{green!0}{\large 0}/{\footnotesize 131} \tabularnewline
\rowcolor{black!10} 85 & libmodbus\_modbus\_receive & 33 & \cellcolor{green!0}{\large 0}/{\footnotesize 40} & \cellcolor{green!0}{\large 0}/{\footnotesize 40} & \cellcolor{green!0}{\large 0}/{\footnotesize 40} & \cellcolor{green!0}{\large 0}/{\footnotesize 40} & \cellcolor{green!0}{\large 0}/{\footnotesize 154} & \cellcolor{green!0}{\large 0}/{\footnotesize 116} \tabularnewline
86 & tmux\_input\_parse\_buffer & 42 & \cellcolor{green!0}{\large 0}/{\footnotesize 40} & \cellcolor{green!0}{\large 0}/{\footnotesize 40} & \cellcolor{green!0}{\large -}{\tiny -} & \cellcolor{green!0}{\large 0}/{\footnotesize 40} & \cellcolor{green!0}{\large 0}/{\footnotesize 182} & \cellcolor{green!0}{\large 0}/{\footnotesize 200} \tabularnewline

\bottomrule
%\end{tabular}
%}
%\end{table*}
\end{xltabular}
}
\twocolumn



% model: text-bison-001, temp: 1.0

\onecolumn
{\small %
\begin{xltabular}[h]{\textwidth}{ccccccccc}
%\begin{table*}[!t]
%\centering
\caption{Evaluation Result of model text-bison-001 with temperature 1.0.} \\
%\resizebox{1.0\linewidth}{!}{
%\begin{tabular}{cccccccccc}
\toprule
Index & Question & Score & NAIVE-40 & BACTX-40 & DOCTX-40 & UGCTX-40 & BA-ITER-40 & ALL-ITER-40 \tabularnewline
\midrule
\rowcolor{black!10} 1 & coturn\_stun\_is\_command\_message\_full\_check\_str & 1 & \cellcolor{green!0}{\large 0}/{\footnotesize 40} & \cellcolor{green!100}{\large 39}/{\footnotesize 40} & \cellcolor{green!0}{\large -}{\tiny -} & \cellcolor{green!30}{\large 11}/{\footnotesize 40} & \cellcolor{green!100}{\large 39}/{\footnotesize 40} & \cellcolor{green!30}{\large 20}/{\footnotesize 77} \tabularnewline
2 & kamailio\_parse\_uri & 1 & \cellcolor{green!0}{\large 0}/{\footnotesize 40} & \cellcolor{green!40}{\large 13}/{\footnotesize 40} & \cellcolor{green!0}{\large -}{\tiny -} & \cellcolor{green!20}{\large 6}/{\footnotesize 40} & \cellcolor{green!30}{\large 28}/{\footnotesize 101} & \cellcolor{green!10}{\large 12}/{\footnotesize 124} \tabularnewline
\rowcolor{black!10} 3 & coturn\_stun\_check\_message\_integrity\_str & 2 & \cellcolor{green!0}{\large 0}/{\footnotesize 40} & \cellcolor{green!0}{\large 0}/{\footnotesize 40} & \cellcolor{green!0}{\large -}{\tiny -} & \cellcolor{green!10}{\large 2}/{\footnotesize 40} & \cellcolor{green!0}{\large 0}/{\footnotesize 168} & \cellcolor{green!10}{\large 1}/{\footnotesize 95} \tabularnewline
4 & libiec61850\_MmsValue\_decodeMmsData & 2 & \cellcolor{green!0}{\large 0}/{\footnotesize 40} & \cellcolor{green!30}{\large 10}/{\footnotesize 40} & \cellcolor{green!50}{\large 19}/{\footnotesize 40} & \cellcolor{green!30}{\large 11}/{\footnotesize 40} & \cellcolor{green!20}{\large 19}/{\footnotesize 108} & \cellcolor{green!20}{\large 17}/{\footnotesize 98} \tabularnewline
\rowcolor{black!10} 5 & md4c\_md\_html & 2 & \cellcolor{green!0}{\large 0}/{\footnotesize 40} & \cellcolor{green!0}{\large 0}/{\footnotesize 40} & \cellcolor{green!0}{\large 0}/{\footnotesize 40} & \cellcolor{green!0}{\large 0}/{\footnotesize 40} & \cellcolor{green!10}{\large 2}/{\footnotesize 185} & \cellcolor{green!0}{\large 1}/{\footnotesize 145} \tabularnewline
6 & spdk\_spdk\_json\_parse & 2 & \cellcolor{green!0}{\large 0}/{\footnotesize 40} & \cellcolor{green!90}{\large 34}/{\footnotesize 40} & \cellcolor{green!0}{\large -}{\tiny -} & \cellcolor{green!10}{\large 1}/{\footnotesize 40} & \cellcolor{green!80}{\large 37}/{\footnotesize 49} & \cellcolor{green!10}{\large 6}/{\footnotesize 96} \tabularnewline
\rowcolor{black!10} 7 & croaring\_roaring\_bitmap\_portable\_deserialize\_safe & 3 & \cellcolor{green!10}{\large 4}/{\footnotesize 40} & \cellcolor{green!50}{\large 20}/{\footnotesize 40} & \cellcolor{green!80}{\large 29}/{\footnotesize 40} & \cellcolor{green!50}{\large 18}/{\footnotesize 40} & \cellcolor{green!50}{\large 35}/{\footnotesize 80} & \cellcolor{green!40}{\large 29}/{\footnotesize 77} \tabularnewline
8 & lua\_luaL\_loadbufferx & 3 & \cellcolor{green!70}{\large 28}/{\footnotesize 40} & \cellcolor{green!90}{\large 36}/{\footnotesize 40} & \cellcolor{green!80}{\large 32}/{\footnotesize 40} & \cellcolor{green!40}{\large 14}/{\footnotesize 40} & \cellcolor{green!60}{\large 36}/{\footnotesize 63} & \cellcolor{green!30}{\large 21}/{\footnotesize 91} \tabularnewline
\rowcolor{black!10} 9 & w3m\_wc\_Str\_conv\_with\_detect & 3 & \cellcolor{green!0}{\large 0}/{\footnotesize 40} & \cellcolor{green!0}{\large 0}/{\footnotesize 40} & \cellcolor{green!0}{\large -}{\tiny -} & \cellcolor{green!0}{\large 0}/{\footnotesize 40} & \cellcolor{green!0}{\large 0}/{\footnotesize 188} & \cellcolor{green!0}{\large 1}/{\footnotesize 190} \tabularnewline
10 & bind9\_dns\_name\_fromwire & 4 & \cellcolor{green!0}{\large 0}/{\footnotesize 40} & \cellcolor{green!0}{\large 0}/{\footnotesize 40} & \cellcolor{green!0}{\large -}{\tiny -} & \cellcolor{green!0}{\large 0}/{\footnotesize 40} & \cellcolor{green!0}{\large 1}/{\footnotesize 168} & \cellcolor{green!10}{\large 2}/{\footnotesize 159} \tabularnewline
\rowcolor{black!10} 11 & gdk-pixbuf\_gdk\_pixbuf\_animation\_new\_from\_file & 4 & \cellcolor{green!0}{\large 0}/{\footnotesize 40} & \cellcolor{green!10}{\large 1}/{\footnotesize 40} & \cellcolor{green!10}{\large 1}/{\footnotesize 40} & \cellcolor{green!0}{\large 0}/{\footnotesize 40} & \cellcolor{green!0}{\large 0}/{\footnotesize 118} & \cellcolor{green!0}{\large 0}/{\footnotesize 138} \tabularnewline
12 & gdk-pixbuf\_gdk\_pixbuf\_new\_from\_data & 4 & \cellcolor{green!0}{\large 0}/{\footnotesize 40} & \cellcolor{green!100}{\large 38}/{\footnotesize 40} & \cellcolor{green!100}{\large 40}/{\footnotesize 40} & \cellcolor{green!50}{\large 20}/{\footnotesize 40} & \cellcolor{green!100}{\large 39}/{\footnotesize 41} & \cellcolor{green!30}{\large 22}/{\footnotesize 72} \tabularnewline
\rowcolor{black!10} 13 & gdk-pixbuf\_gdk\_pixbuf\_new\_from\_file & 4 & \cellcolor{green!10}{\large 1}/{\footnotesize 40} & \cellcolor{green!10}{\large 2}/{\footnotesize 40} & \cellcolor{green!0}{\large 0}/{\footnotesize 40} & \cellcolor{green!10}{\large 1}/{\footnotesize 40} & \cellcolor{green!0}{\large 1}/{\footnotesize 160} & \cellcolor{green!0}{\large 1}/{\footnotesize 106} \tabularnewline
14 & gdk-pixbuf\_gdk\_pixbuf\_new\_from\_stream & 4 & \cellcolor{green!20}{\large 8}/{\footnotesize 40} & \cellcolor{green!60}{\large 22}/{\footnotesize 40} & \cellcolor{green!50}{\large 20}/{\footnotesize 40} & \cellcolor{green!70}{\large 27}/{\footnotesize 40} & \cellcolor{green!60}{\large 32}/{\footnotesize 60} & \cellcolor{green!50}{\large 31}/{\footnotesize 63} \tabularnewline
\rowcolor{black!10} 15 & gpac\_gf\_isom\_open\_file & 4 & \cellcolor{green!0}{\large 0}/{\footnotesize 40} & \cellcolor{green!0}{\large 0}/{\footnotesize 40} & \cellcolor{green!0}{\large -}{\tiny -} & \cellcolor{green!10}{\large 1}/{\footnotesize 40} & \cellcolor{green!0}{\large 0}/{\footnotesize 196} & \cellcolor{green!0}{\large 0}/{\footnotesize 167} \tabularnewline
16 & libbpf\_bpf\_object\_\_open\_mem & 4 & \cellcolor{green!0}{\large 0}/{\footnotesize 40} & \cellcolor{green!60}{\large 23}/{\footnotesize 40} & \cellcolor{green!80}{\large 30}/{\footnotesize 40} & \cellcolor{green!40}{\large 13}/{\footnotesize 40} & \cellcolor{green!30}{\large 23}/{\footnotesize 106} & \cellcolor{green!20}{\large 20}/{\footnotesize 103} \tabularnewline
\rowcolor{black!10} 17 & libpg\_query\_pg\_query\_parse & 4 & \cellcolor{green!0}{\large 0}/{\footnotesize 40} & \cellcolor{green!0}{\large 0}/{\footnotesize 40} & \cellcolor{green!0}{\large -}{\tiny -} & \cellcolor{green!30}{\large 9}/{\footnotesize 40} & \cellcolor{green!10}{\large 2}/{\footnotesize 177} & \cellcolor{green!20}{\large 18}/{\footnotesize 121} \tabularnewline
18 & libucl\_ucl\_parser\_add\_string & 4 & \cellcolor{green!0}{\large 0}/{\footnotesize 40} & \cellcolor{green!0}{\large 0}/{\footnotesize 40} & \cellcolor{green!0}{\large 0}/{\footnotesize 40} & \cellcolor{green!0}{\large 0}/{\footnotesize 40} & \cellcolor{green!10}{\large 3}/{\footnotesize 197} & \cellcolor{green!10}{\large 4}/{\footnotesize 141} \tabularnewline
\rowcolor{black!10} 19 & oniguruma\_onig\_new & 4 & \cellcolor{green!0}{\large 0}/{\footnotesize 40} & \cellcolor{green!0}{\large 0}/{\footnotesize 40} & \cellcolor{green!0}{\large 0}/{\footnotesize 40} & \cellcolor{green!10}{\large 1}/{\footnotesize 40} & \cellcolor{green!0}{\large 0}/{\footnotesize 180} & \cellcolor{green!10}{\large 6}/{\footnotesize 122} \tabularnewline
20 & pupnp\_ixmlLoadDocumentEx & 4 & \cellcolor{green!0}{\large 0}/{\footnotesize 40} & \cellcolor{green!0}{\large 0}/{\footnotesize 40} & \cellcolor{green!0}{\large 0}/{\footnotesize 40} & \cellcolor{green!0}{\large 0}/{\footnotesize 40} & \cellcolor{green!0}{\large 0}/{\footnotesize 159} & \cellcolor{green!0}{\large 0}/{\footnotesize 131} \tabularnewline
\rowcolor{black!10} 21 & gdk-pixbuf\_gdk\_pixbuf\_new\_from\_file\_at\_scale & 5 & \cellcolor{green!0}{\large 0}/{\footnotesize 40} & \cellcolor{green!0}{\large 0}/{\footnotesize 40} & \cellcolor{green!0}{\large 0}/{\footnotesize 40} & \cellcolor{green!0}{\large 0}/{\footnotesize 40} & \cellcolor{green!0}{\large 0}/{\footnotesize 176} & \cellcolor{green!0}{\large 0}/{\footnotesize 125} \tabularnewline
22 & inchi\_GetINCHIKeyFromINCHI & 5 & \cellcolor{green!0}{\large 0}/{\footnotesize 40} & \cellcolor{green!20}{\large 5}/{\footnotesize 40} & \cellcolor{green!10}{\large 4}/{\footnotesize 40} & \cellcolor{green!20}{\large 5}/{\footnotesize 40} & \cellcolor{green!10}{\large 7}/{\footnotesize 154} & \cellcolor{green!10}{\large 4}/{\footnotesize 137} \tabularnewline
\rowcolor{black!10} 23 & libdwarf\_dwarf\_init\_b & 5 & \cellcolor{green!0}{\large 0}/{\footnotesize 40} & \cellcolor{green!0}{\large 0}/{\footnotesize 40} & \cellcolor{green!0}{\large 0}/{\footnotesize 40} & \cellcolor{green!20}{\large 6}/{\footnotesize 40} & \cellcolor{green!0}{\large 0}/{\footnotesize 172} & \cellcolor{green!20}{\large 15}/{\footnotesize 113} \tabularnewline
24 & libdwarf\_dwarf\_init\_path & 5 & \cellcolor{green!0}{\large 0}/{\footnotesize 40} & \cellcolor{green!0}{\large 0}/{\footnotesize 40} & \cellcolor{green!0}{\large 0}/{\footnotesize 40} & \cellcolor{green!0}{\large 0}/{\footnotesize 40} & \cellcolor{green!0}{\large 0}/{\footnotesize 133} & \cellcolor{green!0}{\large 1}/{\footnotesize 122} \tabularnewline
\rowcolor{black!10} 25 & liblouis\_lou\_compileString & 5 & \cellcolor{green!0}{\large 0}/{\footnotesize 40} & \cellcolor{green!10}{\large 1}/{\footnotesize 40} & \cellcolor{green!10}{\large 1}/{\footnotesize 40} & \cellcolor{green!10}{\large 2}/{\footnotesize 40} & \cellcolor{green!10}{\large 3}/{\footnotesize 157} & \cellcolor{green!0}{\large 1}/{\footnotesize 156} \tabularnewline
26 & selinux\_cil\_compile & 5 & \cellcolor{green!0}{\large 0}/{\footnotesize 40} & \cellcolor{green!0}{\large 0}/{\footnotesize 40} & \cellcolor{green!0}{\large -}{\tiny -} & \cellcolor{green!20}{\large 6}/{\footnotesize 40} & \cellcolor{green!0}{\large 0}/{\footnotesize 155} & \cellcolor{green!10}{\large 6}/{\footnotesize 84} \tabularnewline
\rowcolor{black!10} 27 & bind9\_dns\_name\_fromtext & 6 & \cellcolor{green!0}{\large 0}/{\footnotesize 40} & \cellcolor{green!0}{\large 0}/{\footnotesize 40} & \cellcolor{green!0}{\large -}{\tiny -} & \cellcolor{green!10}{\large 3}/{\footnotesize 40} & \cellcolor{green!0}{\large 1}/{\footnotesize 181} & \cellcolor{green!10}{\large 4}/{\footnotesize 118} \tabularnewline
28 & bind9\_dns\_rdata\_fromwire & 6 & \cellcolor{green!0}{\large 0}/{\footnotesize 40} & \cellcolor{green!0}{\large 0}/{\footnotesize 40} & \cellcolor{green!0}{\large -}{\tiny -} & \cellcolor{green!0}{\large 0}/{\footnotesize 40} & \cellcolor{green!0}{\large 0}/{\footnotesize 196} & \cellcolor{green!0}{\large 0}/{\footnotesize 155} \tabularnewline
\rowcolor{black!10} 29 & coturn\_stun\_is\_binding\_response & 6 & \cellcolor{green!0}{\large 0}/{\footnotesize 40} & \cellcolor{green!0}{\large 0}/{\footnotesize 40} & \cellcolor{green!0}{\large -}{\tiny -} & \cellcolor{green!0}{\large 0}/{\footnotesize 40} & \cellcolor{green!0}{\large 0}/{\footnotesize 167} & \cellcolor{green!10}{\large 2}/{\footnotesize 76} \tabularnewline
30 & coturn\_stun\_is\_command\_message & 6 & \cellcolor{green!0}{\large 0}/{\footnotesize 40} & \cellcolor{green!0}{\large 0}/{\footnotesize 40} & \cellcolor{green!0}{\large 0}/{\footnotesize 40} & \cellcolor{green!10}{\large 3}/{\footnotesize 40} & \cellcolor{green!0}{\large 0}/{\footnotesize 173} & \cellcolor{green!10}{\large 1}/{\footnotesize 77} \tabularnewline
\rowcolor{black!10} 31 & coturn\_stun\_is\_response & 6 & \cellcolor{green!0}{\large 0}/{\footnotesize 40} & \cellcolor{green!0}{\large 0}/{\footnotesize 40} & \cellcolor{green!0}{\large -}{\tiny -} & \cellcolor{green!10}{\large 1}/{\footnotesize 40} & \cellcolor{green!0}{\large 0}/{\footnotesize 141} & \cellcolor{green!0}{\large 0}/{\footnotesize 66} \tabularnewline
32 & coturn\_stun\_is\_success\_response & 6 & \cellcolor{green!0}{\large 0}/{\footnotesize 40} & \cellcolor{green!0}{\large 0}/{\footnotesize 40} & \cellcolor{green!0}{\large -}{\tiny -} & \cellcolor{green!10}{\large 1}/{\footnotesize 40} & \cellcolor{green!0}{\large 0}/{\footnotesize 140} & \cellcolor{green!10}{\large 2}/{\footnotesize 76} \tabularnewline
\rowcolor{black!10} 33 & hiredis\_redisFormatCommand & 6 & \cellcolor{green!40}{\large 13}/{\footnotesize 40} & \cellcolor{green!100}{\large 40}/{\footnotesize 40} & \cellcolor{green!0}{\large -}{\tiny -} & \cellcolor{green!60}{\large 22}/{\footnotesize 40} & \cellcolor{green!20}{\large 36}/{\footnotesize 195} & \cellcolor{green!10}{\large 10}/{\footnotesize 161} \tabularnewline
34 & igraph\_igraph\_read\_graph\_dl & 6 & \cellcolor{green!0}{\large 0}/{\footnotesize 40} & \cellcolor{green!0}{\large 0}/{\footnotesize 40} & \cellcolor{green!0}{\large 0}/{\footnotesize 40} & \cellcolor{green!0}{\large 0}/{\footnotesize 40} & \cellcolor{green!0}{\large 0}/{\footnotesize 165} & \cellcolor{green!0}{\large 0}/{\footnotesize 142} \tabularnewline
\rowcolor{black!10} 35 & igraph\_igraph\_read\_graph\_edgelist & 6 & \cellcolor{green!0}{\large 0}/{\footnotesize 40} & \cellcolor{green!0}{\large 0}/{\footnotesize 40} & \cellcolor{green!0}{\large 0}/{\footnotesize 40} & \cellcolor{green!0}{\large 0}/{\footnotesize 40} & \cellcolor{green!0}{\large 0}/{\footnotesize 180} & \cellcolor{green!0}{\large 0}/{\footnotesize 161} \tabularnewline
36 & igraph\_igraph\_read\_graph\_gml & 6 & \cellcolor{green!0}{\large 0}/{\footnotesize 40} & \cellcolor{green!0}{\large 0}/{\footnotesize 40} & \cellcolor{green!0}{\large 0}/{\footnotesize 40} & \cellcolor{green!0}{\large 0}/{\footnotesize 40} & \cellcolor{green!0}{\large 0}/{\footnotesize 187} & \cellcolor{green!0}{\large 0}/{\footnotesize 171} \tabularnewline
\rowcolor{black!10} 37 & igraph\_igraph\_read\_graph\_graphdb & 6 & \cellcolor{green!0}{\large 0}/{\footnotesize 40} & \cellcolor{green!0}{\large 0}/{\footnotesize 40} & \cellcolor{green!0}{\large 0}/{\footnotesize 40} & \cellcolor{green!0}{\large 0}/{\footnotesize 40} & \cellcolor{green!0}{\large 0}/{\footnotesize 188} & \cellcolor{green!0}{\large 0}/{\footnotesize 159} \tabularnewline
38 & igraph\_igraph\_read\_graph\_graphml & 6 & \cellcolor{green!0}{\large 0}/{\footnotesize 40} & \cellcolor{green!0}{\large 0}/{\footnotesize 40} & \cellcolor{green!0}{\large 0}/{\footnotesize 40} & \cellcolor{green!0}{\large 0}/{\footnotesize 40} & \cellcolor{green!0}{\large 0}/{\footnotesize 183} & \cellcolor{green!0}{\large 1}/{\footnotesize 138} \tabularnewline
\rowcolor{black!10} 39 & igraph\_igraph\_read\_graph\_lgl & 6 & \cellcolor{green!0}{\large 0}/{\footnotesize 40} & \cellcolor{green!0}{\large 0}/{\footnotesize 40} & \cellcolor{green!0}{\large 0}/{\footnotesize 40} & \cellcolor{green!0}{\large 0}/{\footnotesize 40} & \cellcolor{green!0}{\large 0}/{\footnotesize 188} & \cellcolor{green!0}{\large 0}/{\footnotesize 144} \tabularnewline
40 & igraph\_igraph\_read\_graph\_pajek & 6 & \cellcolor{green!0}{\large 0}/{\footnotesize 40} & \cellcolor{green!0}{\large 0}/{\footnotesize 40} & \cellcolor{green!0}{\large 0}/{\footnotesize 40} & \cellcolor{green!0}{\large 0}/{\footnotesize 40} & \cellcolor{green!0}{\large 0}/{\footnotesize 184} & \cellcolor{green!0}{\large 1}/{\footnotesize 143} \tabularnewline
\rowcolor{black!10} 41 & inchi\_GetINCHIfromINCHI & 6 & \cellcolor{green!0}{\large 0}/{\footnotesize 40} & \cellcolor{green!0}{\large 0}/{\footnotesize 40} & \cellcolor{green!0}{\large 0}/{\footnotesize 40} & \cellcolor{green!10}{\large 3}/{\footnotesize 40} & \cellcolor{green!0}{\large 0}/{\footnotesize 182} & \cellcolor{green!0}{\large 0}/{\footnotesize 153} \tabularnewline
42 & inchi\_GetStructFromINCHI & 6 & \cellcolor{green!0}{\large 0}/{\footnotesize 40} & \cellcolor{green!0}{\large 0}/{\footnotesize 40} & \cellcolor{green!0}{\large 0}/{\footnotesize 40} & \cellcolor{green!0}{\large 0}/{\footnotesize 40} & \cellcolor{green!0}{\large 0}/{\footnotesize 167} & \cellcolor{green!10}{\large 2}/{\footnotesize 194} \tabularnewline
\rowcolor{black!10} 43 & kamailio\_parse\_msg & 6 & \cellcolor{green!0}{\large 0}/{\footnotesize 40} & \cellcolor{green!20}{\large 6}/{\footnotesize 40} & \cellcolor{green!0}{\large -}{\tiny -} & \cellcolor{green!30}{\large 10}/{\footnotesize 40} & \cellcolor{green!10}{\large 12}/{\footnotesize 156} & \cellcolor{green!20}{\large 19}/{\footnotesize 121} \tabularnewline
44 & libyang\_lys\_parse\_mem & 6 & \cellcolor{green!0}{\large 0}/{\footnotesize 40} & \cellcolor{green!0}{\large 0}/{\footnotesize 40} & \cellcolor{green!0}{\large 0}/{\footnotesize 40} & \cellcolor{green!10}{\large 1}/{\footnotesize 40} & \cellcolor{green!0}{\large 0}/{\footnotesize 189} & \cellcolor{green!0}{\large 0}/{\footnotesize 150} \tabularnewline
\rowcolor{black!10} 45 & proftpd\_pr\_json\_object\_from\_text & 6 & \cellcolor{green!0}{\large 0}/{\footnotesize 40} & \cellcolor{green!0}{\large 0}/{\footnotesize 40} & \cellcolor{green!0}{\large -}{\tiny -} & \cellcolor{green!10}{\large 1}/{\footnotesize 40} & \cellcolor{green!0}{\large 0}/{\footnotesize 197} & \cellcolor{green!0}{\large 0}/{\footnotesize 117} \tabularnewline
46 & selinux\_policydb\_read & 6 & \cellcolor{green!0}{\large 0}/{\footnotesize 40} & \cellcolor{green!0}{\large 0}/{\footnotesize 40} & \cellcolor{green!0}{\large -}{\tiny -} & \cellcolor{green!10}{\large 2}/{\footnotesize 40} & \cellcolor{green!0}{\large 1}/{\footnotesize 179} & \cellcolor{green!10}{\large 4}/{\footnotesize 131} \tabularnewline
\rowcolor{black!10} 47 & kamailio\_get\_src\_address\_socket & 7 & \cellcolor{green!0}{\large 0}/{\footnotesize 40} & \cellcolor{green!0}{\large 0}/{\footnotesize 40} & \cellcolor{green!0}{\large 0}/{\footnotesize 40} & \cellcolor{green!20}{\large 6}/{\footnotesize 40} & \cellcolor{green!0}{\large 0}/{\footnotesize 192} & \cellcolor{green!10}{\large 2}/{\footnotesize 166} \tabularnewline
48 & kamailio\_get\_src\_uri & 7 & \cellcolor{green!0}{\large 0}/{\footnotesize 40} & \cellcolor{green!0}{\large 0}/{\footnotesize 40} & \cellcolor{green!0}{\large 0}/{\footnotesize 40} & \cellcolor{green!10}{\large 3}/{\footnotesize 40} & \cellcolor{green!0}{\large 0}/{\footnotesize 188} & \cellcolor{green!0}{\large 1}/{\footnotesize 169} \tabularnewline
\rowcolor{black!10} 49 & kamailio\_parse\_content\_disposition & 7 & \cellcolor{green!0}{\large 0}/{\footnotesize 40} & \cellcolor{green!0}{\large 0}/{\footnotesize 40} & \cellcolor{green!0}{\large 0}/{\footnotesize 40} & \cellcolor{green!0}{\large 0}/{\footnotesize 40} & \cellcolor{green!0}{\large 0}/{\footnotesize 188} & \cellcolor{green!0}{\large 1}/{\footnotesize 142} \tabularnewline
50 & kamailio\_parse\_diversion\_header & 7 & \cellcolor{green!0}{\large 0}/{\footnotesize 40} & \cellcolor{green!0}{\large 0}/{\footnotesize 40} & \cellcolor{green!0}{\large 0}/{\footnotesize 40} & \cellcolor{green!10}{\large 2}/{\footnotesize 40} & \cellcolor{green!0}{\large 0}/{\footnotesize 185} & \cellcolor{green!10}{\large 5}/{\footnotesize 148} \tabularnewline
\rowcolor{black!10} 51 & kamailio\_parse\_from\_header & 7 & \cellcolor{green!0}{\large 0}/{\footnotesize 40} & \cellcolor{green!0}{\large 0}/{\footnotesize 40} & \cellcolor{green!0}{\large -}{\tiny -} & \cellcolor{green!0}{\large 0}/{\footnotesize 40} & \cellcolor{green!0}{\large 0}/{\footnotesize 199} & \cellcolor{green!0}{\large 0}/{\footnotesize 157} \tabularnewline
52 & kamailio\_parse\_from\_uri & 7 & \cellcolor{green!0}{\large 0}/{\footnotesize 40} & \cellcolor{green!0}{\large 0}/{\footnotesize 40} & \cellcolor{green!0}{\large -}{\tiny -} & \cellcolor{green!0}{\large 0}/{\footnotesize 40} & \cellcolor{green!0}{\large 0}/{\footnotesize 194} & \cellcolor{green!0}{\large 0}/{\footnotesize 149} \tabularnewline
\rowcolor{black!10} 53 & kamailio\_parse\_headers & 7 & \cellcolor{green!0}{\large 0}/{\footnotesize 40} & \cellcolor{green!0}{\large 0}/{\footnotesize 40} & \cellcolor{green!0}{\large -}{\tiny -} & \cellcolor{green!0}{\large 0}/{\footnotesize 40} & \cellcolor{green!0}{\large 0}/{\footnotesize 167} & \cellcolor{green!0}{\large 0}/{\footnotesize 137} \tabularnewline
54 & kamailio\_parse\_identityinfo\_header & 7 & \cellcolor{green!0}{\large 0}/{\footnotesize 40} & \cellcolor{green!0}{\large 0}/{\footnotesize 40} & \cellcolor{green!0}{\large -}{\tiny -} & \cellcolor{green!20}{\large 6}/{\footnotesize 40} & \cellcolor{green!0}{\large 0}/{\footnotesize 182} & \cellcolor{green!10}{\large 5}/{\footnotesize 115} \tabularnewline
\rowcolor{black!10} 55 & kamailio\_parse\_pai\_header & 7 & \cellcolor{green!0}{\large 0}/{\footnotesize 40} & \cellcolor{green!0}{\large 0}/{\footnotesize 40} & \cellcolor{green!0}{\large -}{\tiny -} & \cellcolor{green!10}{\large 2}/{\footnotesize 40} & \cellcolor{green!0}{\large 0}/{\footnotesize 172} & \cellcolor{green!0}{\large 0}/{\footnotesize 165} \tabularnewline
56 & kamailio\_parse\_privacy & 7 & \cellcolor{green!0}{\large 0}/{\footnotesize 40} & \cellcolor{green!0}{\large 0}/{\footnotesize 40} & \cellcolor{green!0}{\large 0}/{\footnotesize 40} & \cellcolor{green!10}{\large 2}/{\footnotesize 40} & \cellcolor{green!0}{\large 0}/{\footnotesize 184} & \cellcolor{green!10}{\large 3}/{\footnotesize 122} \tabularnewline
\rowcolor{black!10} 57 & kamailio\_parse\_record\_route\_headers & 7 & \cellcolor{green!0}{\large 0}/{\footnotesize 40} & \cellcolor{green!0}{\large 0}/{\footnotesize 40} & \cellcolor{green!0}{\large -}{\tiny -} & \cellcolor{green!20}{\large 8}/{\footnotesize 40} & \cellcolor{green!0}{\large 0}/{\footnotesize 172} & \cellcolor{green!20}{\large 9}/{\footnotesize 78} \tabularnewline
58 & kamailio\_parse\_refer\_to\_header & 7 & \cellcolor{green!0}{\large 0}/{\footnotesize 40} & \cellcolor{green!0}{\large 0}/{\footnotesize 40} & \cellcolor{green!0}{\large -}{\tiny -} & \cellcolor{green!0}{\large 0}/{\footnotesize 40} & \cellcolor{green!0}{\large 0}/{\footnotesize 179} & \cellcolor{green!0}{\large 0}/{\footnotesize 155} \tabularnewline
\rowcolor{black!10} 59 & kamailio\_parse\_route\_headers & 7 & \cellcolor{green!0}{\large 0}/{\footnotesize 40} & \cellcolor{green!0}{\large 0}/{\footnotesize 40} & \cellcolor{green!0}{\large -}{\tiny -} & \cellcolor{green!20}{\large 8}/{\footnotesize 40} & \cellcolor{green!0}{\large 0}/{\footnotesize 186} & \cellcolor{green!10}{\large 3}/{\footnotesize 95} \tabularnewline
60 & kamailio\_parse\_to\_header & 7 & \cellcolor{green!0}{\large 0}/{\footnotesize 40} & \cellcolor{green!0}{\large 0}/{\footnotesize 40} & \cellcolor{green!0}{\large -}{\tiny -} & \cellcolor{green!0}{\large 0}/{\footnotesize 40} & \cellcolor{green!0}{\large 0}/{\footnotesize 187} & \cellcolor{green!0}{\large 1}/{\footnotesize 153} \tabularnewline
\rowcolor{black!10} 61 & kamailio\_parse\_to\_uri & 7 & \cellcolor{green!0}{\large 0}/{\footnotesize 40} & \cellcolor{green!0}{\large 0}/{\footnotesize 40} & \cellcolor{green!0}{\large -}{\tiny -} & \cellcolor{green!10}{\large 3}/{\footnotesize 40} & \cellcolor{green!0}{\large 0}/{\footnotesize 181} & \cellcolor{green!0}{\large 0}/{\footnotesize 147} \tabularnewline
62 & libyang\_lyd\_parse\_data\_mem & 7 & \cellcolor{green!0}{\large 0}/{\footnotesize 40} & \cellcolor{green!0}{\large 0}/{\footnotesize 40} & \cellcolor{green!0}{\large 0}/{\footnotesize 40} & \cellcolor{green!0}{\large 0}/{\footnotesize 40} & \cellcolor{green!0}{\large 0}/{\footnotesize 196} & \cellcolor{green!0}{\large 0}/{\footnotesize 163} \tabularnewline
\rowcolor{black!10} 63 & bind9\_dns\_message\_parse & 8 & \cellcolor{green!0}{\large 0}/{\footnotesize 40} & \cellcolor{green!0}{\large 0}/{\footnotesize 40} & \cellcolor{green!0}{\large -}{\tiny -} & \cellcolor{green!0}{\large 0}/{\footnotesize 40} & \cellcolor{green!0}{\large 0}/{\footnotesize 181} & \cellcolor{green!0}{\large 0}/{\footnotesize 144} \tabularnewline
64 & igraph\_igraph\_read\_graph\_ncol & 8 & \cellcolor{green!0}{\large 0}/{\footnotesize 40} & \cellcolor{green!0}{\large 0}/{\footnotesize 40} & \cellcolor{green!0}{\large 0}/{\footnotesize 40} & \cellcolor{green!0}{\large 0}/{\footnotesize 40} & \cellcolor{green!0}{\large 0}/{\footnotesize 168} & \cellcolor{green!0}{\large 0}/{\footnotesize 116} \tabularnewline
\rowcolor{black!10} 65 & pjsip\_pj\_json\_parse & 8 & \cellcolor{green!0}{\large 0}/{\footnotesize 40} & \cellcolor{green!0}{\large 0}/{\footnotesize 40} & \cellcolor{green!0}{\large 0}/{\footnotesize 40} & \cellcolor{green!0}{\large 0}/{\footnotesize 40} & \cellcolor{green!0}{\large 0}/{\footnotesize 194} & \cellcolor{green!0}{\large 0}/{\footnotesize 176} \tabularnewline
66 & pjsip\_pj\_xml\_parse & 8 & \cellcolor{green!0}{\large 0}/{\footnotesize 40} & \cellcolor{green!0}{\large 0}/{\footnotesize 40} & \cellcolor{green!0}{\large 0}/{\footnotesize 40} & \cellcolor{green!0}{\large 0}/{\footnotesize 40} & \cellcolor{green!0}{\large 0}/{\footnotesize 185} & \cellcolor{green!0}{\large 0}/{\footnotesize 154} \tabularnewline
\rowcolor{black!10} 67 & pjsip\_pjmedia\_sdp\_parse & 8 & \cellcolor{green!0}{\large 0}/{\footnotesize 40} & \cellcolor{green!0}{\large 0}/{\footnotesize 40} & \cellcolor{green!0}{\large 0}/{\footnotesize 40} & \cellcolor{green!0}{\large 0}/{\footnotesize 40} & \cellcolor{green!0}{\large 0}/{\footnotesize 168} & \cellcolor{green!0}{\large 0}/{\footnotesize 118} \tabularnewline
68 & quickjs\_lre\_compile & 8 & \cellcolor{green!0}{\large 0}/{\footnotesize 40} & \cellcolor{green!0}{\large 0}/{\footnotesize 40} & \cellcolor{green!0}{\large -}{\tiny -} & \cellcolor{green!0}{\large 0}/{\footnotesize 40} & \cellcolor{green!0}{\large 0}/{\footnotesize 194} & \cellcolor{green!0}{\large 0}/{\footnotesize 129} \tabularnewline
\rowcolor{black!10} 69 & bind9\_isc\_lex\_getmastertoken & 9 & \cellcolor{green!0}{\large 0}/{\footnotesize 40} & \cellcolor{green!0}{\large 0}/{\footnotesize 40} & \cellcolor{green!0}{\large -}{\tiny -} & \cellcolor{green!0}{\large 0}/{\footnotesize 40} & \cellcolor{green!0}{\large 0}/{\footnotesize 188} & \cellcolor{green!0}{\large 0}/{\footnotesize 148} \tabularnewline
70 & bind9\_isc\_lex\_gettoken & 9 & \cellcolor{green!0}{\large 0}/{\footnotesize 40} & \cellcolor{green!0}{\large 0}/{\footnotesize 40} & \cellcolor{green!0}{\large -}{\tiny -} & \cellcolor{green!0}{\large 0}/{\footnotesize 40} & \cellcolor{green!0}{\large 0}/{\footnotesize 196} & \cellcolor{green!0}{\large 0}/{\footnotesize 141} \tabularnewline
\rowcolor{black!10} 71 & quickjs\_JS\_Eval & 9 & \cellcolor{green!0}{\large 0}/{\footnotesize 40} & \cellcolor{green!0}{\large 0}/{\footnotesize 40} & \cellcolor{green!0}{\large -}{\tiny -} & \cellcolor{green!10}{\large 1}/{\footnotesize 40} & \cellcolor{green!0}{\large 0}/{\footnotesize 198} & \cellcolor{green!0}{\large 1}/{\footnotesize 159} \tabularnewline
72 & igraph\_igraph\_edge\_connectivity & 10 & \cellcolor{green!0}{\large 0}/{\footnotesize 40} & \cellcolor{green!0}{\large 0}/{\footnotesize 40} & \cellcolor{green!0}{\large 0}/{\footnotesize 40} & \cellcolor{green!0}{\large 0}/{\footnotesize 40} & \cellcolor{green!0}{\large 0}/{\footnotesize 170} & \cellcolor{green!0}{\large 0}/{\footnotesize 177} \tabularnewline
\rowcolor{black!10} 73 & pjsip\_pj\_stun\_msg\_decode & 10 & \cellcolor{green!0}{\large 0}/{\footnotesize 40} & \cellcolor{green!0}{\large 0}/{\footnotesize 40} & \cellcolor{green!0}{\large 0}/{\footnotesize 40} & \cellcolor{green!0}{\large 0}/{\footnotesize 40} & \cellcolor{green!0}{\large 0}/{\footnotesize 192} & \cellcolor{green!0}{\large 0}/{\footnotesize 101} \tabularnewline
74 & bind9\_dns\_message\_checksig & 11 & \cellcolor{green!0}{\large 0}/{\footnotesize 40} & \cellcolor{green!0}{\large 0}/{\footnotesize 40} & \cellcolor{green!0}{\large -}{\tiny -} & \cellcolor{green!0}{\large 0}/{\footnotesize 40} & \cellcolor{green!0}{\large 0}/{\footnotesize 181} & \cellcolor{green!0}{\large 0}/{\footnotesize 102} \tabularnewline
\rowcolor{black!10} 75 & libzip\_zip\_fread & 11 & \cellcolor{green!0}{\large 0}/{\footnotesize 40} & \cellcolor{green!0}{\large 0}/{\footnotesize 40} & \cellcolor{green!0}{\large 0}/{\footnotesize 40} & \cellcolor{green!0}{\large 0}/{\footnotesize 40} & \cellcolor{green!0}{\large 0}/{\footnotesize 180} & \cellcolor{green!0}{\large 0}/{\footnotesize 124} \tabularnewline
76 & bind9\_dns\_rdata\_fromtext & 12 & \cellcolor{green!0}{\large 0}/{\footnotesize 40} & \cellcolor{green!0}{\large 0}/{\footnotesize 40} & \cellcolor{green!0}{\large -}{\tiny -} & \cellcolor{green!0}{\large 0}/{\footnotesize 40} & \cellcolor{green!0}{\large 0}/{\footnotesize 164} & \cellcolor{green!0}{\large 0}/{\footnotesize 88} \tabularnewline
\rowcolor{black!10} 77 & igraph\_igraph\_all\_minimal\_st\_separators & 12 & \cellcolor{green!0}{\large 0}/{\footnotesize 40} & \cellcolor{green!0}{\large 0}/{\footnotesize 40} & \cellcolor{green!0}{\large 0}/{\footnotesize 40} & \cellcolor{green!0}{\large 0}/{\footnotesize 40} & \cellcolor{green!0}{\large 0}/{\footnotesize 148} & \cellcolor{green!0}{\large 0}/{\footnotesize 193} \tabularnewline
78 & igraph\_igraph\_minimum\_size\_separators & 12 & \cellcolor{green!0}{\large 0}/{\footnotesize 40} & \cellcolor{green!0}{\large 0}/{\footnotesize 40} & \cellcolor{green!0}{\large 0}/{\footnotesize 40} & \cellcolor{green!0}{\large 0}/{\footnotesize 40} & \cellcolor{green!0}{\large 0}/{\footnotesize 160} & \cellcolor{green!0}{\large 0}/{\footnotesize 109} \tabularnewline
\rowcolor{black!10} 79 & pjsip\_pjsip\_parse\_msg & 12 & \cellcolor{green!0}{\large 0}/{\footnotesize 40} & \cellcolor{green!0}{\large 0}/{\footnotesize 40} & \cellcolor{green!0}{\large 0}/{\footnotesize 40} & \cellcolor{green!0}{\large 0}/{\footnotesize 40} & \cellcolor{green!0}{\large 0}/{\footnotesize 184} & \cellcolor{green!0}{\large 0}/{\footnotesize 124} \tabularnewline
80 & igraph\_igraph\_automorphism\_group & 13 & \cellcolor{green!0}{\large 0}/{\footnotesize 40} & \cellcolor{green!0}{\large 0}/{\footnotesize 40} & \cellcolor{green!0}{\large 0}/{\footnotesize 40} & \cellcolor{green!0}{\large 0}/{\footnotesize 40} & \cellcolor{green!0}{\large 0}/{\footnotesize 154} & \cellcolor{green!0}{\large 0}/{\footnotesize 78} \tabularnewline
\rowcolor{black!10} 81 & libmodbus\_modbus\_read\_bits & 15 & \cellcolor{green!0}{\large 0}/{\footnotesize 40} & \cellcolor{green!0}{\large 0}/{\footnotesize 40} & \cellcolor{green!0}{\large 0}/{\footnotesize 40} & \cellcolor{green!0}{\large 0}/{\footnotesize 40} & \cellcolor{green!0}{\large 0}/{\footnotesize 140} & \cellcolor{green!0}{\large 0}/{\footnotesize 110} \tabularnewline
82 & libmodbus\_modbus\_read\_registers & 15 & \cellcolor{green!0}{\large 0}/{\footnotesize 40} & \cellcolor{green!0}{\large 0}/{\footnotesize 40} & \cellcolor{green!0}{\large 0}/{\footnotesize 40} & \cellcolor{green!0}{\large 0}/{\footnotesize 40} & \cellcolor{green!0}{\large 0}/{\footnotesize 158} & \cellcolor{green!0}{\large 0}/{\footnotesize 145} \tabularnewline
\rowcolor{black!10} 83 & civetweb\_mg\_get\_response & 17 & \cellcolor{green!0}{\large 0}/{\footnotesize 40} & \cellcolor{green!0}{\large 0}/{\footnotesize 40} & \cellcolor{green!0}{\large 0}/{\footnotesize 40} & \cellcolor{green!0}{\large 0}/{\footnotesize 40} & \cellcolor{green!0}{\large 0}/{\footnotesize 199} & \cellcolor{green!0}{\large 0}/{\footnotesize 87} \tabularnewline
84 & bind9\_dns\_master\_loadbuffer & 20 & \cellcolor{green!0}{\large 0}/{\footnotesize 40} & \cellcolor{green!0}{\large 0}/{\footnotesize 40} & \cellcolor{green!0}{\large -}{\tiny -} & \cellcolor{green!0}{\large 0}/{\footnotesize 40} & \cellcolor{green!0}{\large 0}/{\footnotesize 159} & \cellcolor{green!0}{\large 0}/{\footnotesize 127} \tabularnewline
\rowcolor{black!10} 85 & libmodbus\_modbus\_receive & 33 & \cellcolor{green!0}{\large 0}/{\footnotesize 40} & \cellcolor{green!0}{\large 0}/{\footnotesize 40} & \cellcolor{green!0}{\large 0}/{\footnotesize 40} & \cellcolor{green!0}{\large 0}/{\footnotesize 40} & \cellcolor{green!0}{\large 0}/{\footnotesize 138} & \cellcolor{green!0}{\large 0}/{\footnotesize 113} \tabularnewline
86 & tmux\_input\_parse\_buffer & 42 & \cellcolor{green!0}{\large 0}/{\footnotesize 40} & \cellcolor{green!0}{\large 0}/{\footnotesize 40} & \cellcolor{green!0}{\large -}{\tiny -} & \cellcolor{green!0}{\large 0}/{\footnotesize 40} & \cellcolor{green!0}{\large 0}/{\footnotesize 199} & \cellcolor{green!0}{\large 0}/{\footnotesize 195} \tabularnewline

\bottomrule
%\end{tabular}
%}
%\end{table*}
\end{xltabular}
}
\twocolumn





%%
%% The acknowledgments section is defined using the "acks" environment
%% (and NOT an unnumbered section). This ensures the proper
%% identification of the section in the article metadata, and the
%% consistent spelling of the heading.
%\begin{acks}
%To Robert, for the bagels and explaining CMYK and color spaces.
%\end{acks}

%%
%% The next two lines define the bibliography style to be used, and
%% the bibliography file.
\bibliographystyle{ACM-Reference-Format}
\bibliography{reference}

%%
%% If your work has an appendix, this is the place to put it.
% \pagebreak
% \appendix
% \section{Details of Proposed Fix Prompt Templates}
\label{sec:fix-templates-detail}
Table~\ref{tab:fix-templates-detail} lists the detailed design of seven prompt templates used for fixing seven types of failures.
Concrete examples can be found in ~\cite{fuzz-drvier-study-website}.

\section{Detail of Quiz Questions}
\label{sec:quiz-questions-detail}
Table~\ref{tab:quiz} shows the full list of the 86 questions included in our quiz.

\section{Full List of Evaluation Results for All Query Strategies}
\label{sec:evaluation-results-full-list}
Table~\ref{tab:eval_full} posts the detailed statistics of evaluation results.

\onecolumn

\begin{table*}[h]
\caption{Detail of Different Fix Prompt Templates.}
\label{tab:fix-templates-detail}
\resizebox{0.9\textwidth}{!}{
\begin{tabular}{ll}
\toprule
Template Name & Template Content\\
\midrule

\rowcolor{black!10}

\begin{tabular}[t]{l}
\multirow{10}{*}{\texttt{FIX\_PRSE\_ERR}} \\
\end{tabular}
&
\begin{tabular}[t]{l}
\Verb|```| \\
\Verb|${DRIVER_CODE}| \\
\Verb|```| \\
\Verb|The above C code has compilation error.| \\
\Verb|The error line is:| \\
\Verb|${ERR_LINE_CODE}| \\
\Verb|The error description is:| \\
\Verb|${ERR_DESCRIPTION}| \\
\Verb|${SUPPLEMENTAL_INFO}| \\
\Verb|Based on the above information, fix the code.| \\
\end{tabular}
\\
\midrule

\begin{tabular}[t]{l}
\multirow{6}{*}{\texttt{FIX\_LINK\_ERR}} \\
\end{tabular}
&
\begin{tabular}[t]{l}
\Verb|```| \\
\Verb|${DRIVER_CODE}| \\
\Verb|```| \\
\Verb|The above C code calls a non-existing API ${API_NAME}.| \\
\Verb|${SUPPLEMENTAL_INFO}| \\
\Verb|Based on the above information, fix the code.| \\
\end{tabular}
\\
\midrule

\rowcolor{black!10}

\begin{tabular}[t]{l}
\multirow{6}{*}{\texttt{FIX\_FUZZ\_MEMLEAK}} \\
\end{tabular}
&
\begin{tabular}[t]{l}
\Verb|```| \\
\Verb|${DRIVER_CODE}| \\
\Verb|```| \\
\Verb|The above C code can be built successfully but has runtime memory leak.| \\
\Verb|${SUPPLEMENTAL_INFO}| \\
\Verb|Based on the above information, fix the code.| \\
\end{tabular}
\\
\midrule

\begin{tabular}[t]{l}
\multirow{6}{*}{\texttt{FIX\_FUZZ\_OOM}} \\
\end{tabular}
&
\begin{tabular}[t]{l}
\Verb|```| \\
\Verb|${DRIVER_CODE}| \\
\Verb|```| \\
\Verb|The above C code can be built successfully but meet out-of-memory, perhaps due to memory leak.| \\
\Verb|${SUPPLEMENTAL_INFO}| \\
\Verb|Based on the above information, fix the code.| \\
\end{tabular}
\\
\midrule

\rowcolor{black!10}

\begin{tabular}[t]{l}
\multirow{10}{*}{\texttt{FIX\_FUZZ\_TIMEOUT}} \\
\end{tabular}
&
\begin{tabular}[t]{l}
\Verb|```| \\
\Verb|${DRIVER_CODE}| \\
\Verb|```| \\
\Verb|The above C code can be built successfully but will stuck (timeout).| \\
\Verb|The possible stuck line is:| \\
\Verb|${ERR_LINE_CODE}| \\
\Verb|The frames of the stack are:| \\
\Verb|${ERR_STACK}| \\
\Verb|${SUPPLEMENTAL_INFO}| \\
\Verb|Based on the above information, fix the code.| \\
\end{tabular}
\\
\midrule

\begin{tabular}[t]{l}
\multirow{10}{*}{\texttt{FIX\_FUZZ\_CRASH}} \\
\end{tabular}
&
\begin{tabular}[t]{l}
\Verb|```| \\
\Verb|${DRIVER_CODE}| \\
\Verb|```| \\
\Verb|The above C code can be built successfully but will crash (${CRASH_SYMPTOM}).| \\
\Verb|The crash line is:| \\
\Verb|${ERR_LINE_CODE}| \\
\Verb|The frames of the stack are:| \\
\Verb|${ERR_DESCRIPTION}| \\
\Verb|${SUPPLEMENTAL_INFO}| \\
\Verb|Based on the above information, fix the code.| \\
\end{tabular}
\\
\midrule

\rowcolor{black!10}

\begin{tabular}[t]{l}
\multirow{6}{*}{\texttt{FIX\_FUZZ\_NONEFF}} \\
\end{tabular}
&
\begin{tabular}[t]{l}
\Verb|```| \\
\Verb|${DRIVER_CODE}| \\
\Verb|```| \\
\Verb|The above C code can be built successfully but its fuzzing seems not effective since the \ |\\
\Verb|  coverage never change.| \\
\Verb|Based on the above information, fix the code if necessary.| \\
\end{tabular}
\\
\bottomrule
\end{tabular}
}
\end{table*}

\clearpage

\onecolumn

\renewcommand\arraystretch{1.1}

%\begin{table*}
\begin{xltabular}[h]{\textwidth}{ccccc}
%\begin{tabular}{ccccc}
\caption{Detail of Quiz Questions. \label{tab:quiz}}\\
\toprule
Index & Project & API & Score & Drivers \\
  \hline
  \endfirsthead
  \multicolumn{5}{c}{Continued from previous page} \\
  \hline
Index & Project & API & Score & Drivers \\
  \hline
  \endhead
  \hline
  \multicolumn{5}{c}{Continued on next page} \\
  \endfoot
  \hline
  \endlastfoot
\cellcolor{black!10}\# 1 & \cellcolor{black!10}coturn & \cellcolor{black!10}stun\_is\_command\_message\_full\_check\_str & \cellcolor{black!10}1 & \cellcolor{black!10}FuzzStun.c \\
\# 2 & kamailio & parse\_uri & 1 & fuzz\_uri.c \\
\cellcolor{black!10}\# 3 & \cellcolor{black!10}coturn & \cellcolor{black!10}stun\_check\_message\_integrity\_str & \cellcolor{black!10}2 & \cellcolor{black!10}FuzzStun.c \\
\# 4 & libiec61850 & MmsValue\_decodeMmsData & 2 & fuzz\_mms\_decode.c \\
\cellcolor{black!10}\# 5 & \cellcolor{black!10}md4c & \cellcolor{black!10}md\_html & \cellcolor{black!10}2 & \cellcolor{black!10}fuzz-mdhtml.c \\
\# 6 & spdk & spdk\_json\_parse & 2 & parse\_json\_fuzzer.cc \\
\cellcolor{black!10}\# 7 & \cellcolor{black!10}croaring & \cellcolor{black!10}roaring\_bitmap\_portable\_deserialize\_safe & \cellcolor{black!10}3 & \cellcolor{black!10}croaring\_fuzzer.c \\
\# 8 & lua & luaL\_loadbufferx & 3 & fuzz\_lua.c \\
\cellcolor{black!10}\# 9 & \cellcolor{black!10}w3m & \cellcolor{black!10}wc\_Str\_conv\_with\_detect & \cellcolor{black!10}3 & \cellcolor{black!10}fuzz-conv.c \\
\# 10 & bind9 & dns\_name\_fromwire & 4 & dns\_name\_fromwire.c \\
\cellcolor{black!10}\# 11 & \cellcolor{black!10}gdk-pixbuf & \cellcolor{black!10}gdk\_pixbuf\_animation\_new\_from\_file & \cellcolor{black!10}4 & \cellcolor{black!10}animation\_fuzzer.c \\
\# 12 & gdk-pixbuf & gdk\_pixbuf\_new\_from\_data & 4 & pixbuf\_cons\_fuzzer.c \\
\cellcolor{black!10}\# 13 & \cellcolor{black!10}gdk-pixbuf & \cellcolor{black!10}gdk\_pixbuf\_new\_from\_file & \cellcolor{black!10}4 & \cellcolor{black!10}pixbuf\_file\_fuzzer.c \\
\# 14 & gdk-pixbuf & gdk\_pixbuf\_new\_from\_stream & 4 & stream\_fuzzer.c \\
\cellcolor{black!10}\# 15 & \cellcolor{black!10}gpac & \cellcolor{black!10}gf\_isom\_open\_file & \cellcolor{black!10}4 & \cellcolor{black!10}fuzz\_parse.c \\
\# 16 & libbpf & bpf\_object\_\_open\_mem & 4 & bpf-object-fuzzer.c \\
\cellcolor{black!10}\# 17 & \cellcolor{black!10}libpg\_query & \cellcolor{black!10}pg\_query\_parse & \cellcolor{black!10}4 & \cellcolor{black!10}fuzz\_parser.c \\
\# 18 & libucl & ucl\_parser\_add\_string & 4 & ucl\_add\_string\_fuzzer.c \\
\cellcolor{black!10}\# 19 & \cellcolor{black!10}oniguruma & \cellcolor{black!10}onig\_new & \cellcolor{black!10}4 & \cellcolor{black!10}base.c \\
\# 20 & pupnp & ixmlLoadDocumentEx & 4 & FuzzIxml.c \\
\cellcolor{black!10}\# 21 & \cellcolor{black!10}gdk-pixbuf & \cellcolor{black!10}gdk\_pixbuf\_new\_from\_file\_at\_scale & \cellcolor{black!10}5 & \cellcolor{black!10}pixbuf\_scale\_fuzzer.c \\
\# 22 & inchi & GetINCHIKeyFromINCHI & 5 & inchi\_input\_fuzzer.c \\
\cellcolor{black!10}\# 23 & \cellcolor{black!10}libdwarf & \cellcolor{black!10}dwarf\_init\_b & \cellcolor{black!10}5 & \cellcolor{black!10}fuzz\_init\_binary.c \\
\# 24 & libdwarf & dwarf\_init\_path & 5 & fuzz\_init\_path.c \\
\cellcolor{black!10}\# 25 & \cellcolor{black!10}liblouis & \cellcolor{black!10}lou\_compileString & \cellcolor{black!10}5 & \cellcolor{black!10}table\_fuzzer.cc \\
\# 26 & selinux & cil\_compile & 5 & secilc-fuzzer.c \\
\cellcolor{black!10}\# 27 & \cellcolor{black!10}bind9 & \cellcolor{black!10}dns\_name\_fromtext & \cellcolor{black!10}6 & \cellcolor{black!10}dns\_name\_fromtext\_target.c \\
\# 28 & bind9 & dns\_rdata\_fromwire & 6 & dns\_rdata\_fromwire\_text.c \\
\cellcolor{black!10}\# 29 & \cellcolor{black!10}coturn & \cellcolor{black!10}stun\_is\_binding\_response & \cellcolor{black!10}6 & \cellcolor{black!10}FuzzStunClient.c \\
\# 30 & coturn & stun\_is\_command\_message & 6 & FuzzStunClient.c \\
\cellcolor{black!10}\# 31 & \cellcolor{black!10}coturn & \cellcolor{black!10}stun\_is\_response & \cellcolor{black!10}6 & \cellcolor{black!10}FuzzStunClient.c \\
\# 32 & coturn & stun\_is\_success\_response & 6 & FuzzStunClient.c \\
\cellcolor{black!10}\# 33 & \cellcolor{black!10}hiredis & \cellcolor{black!10}redisFormatCommand & \cellcolor{black!10}6 & \cellcolor{black!10}format\_command\_fuzzer.c \\
\# 34 & igraph & igraph\_read\_graph\_dl & 6 & read\_dl\_fuzzer.cpp \\
\cellcolor{black!10}\# 35 & \cellcolor{black!10}igraph & \cellcolor{black!10}igraph\_read\_graph\_edgelist & \cellcolor{black!10}6 & \cellcolor{black!10}read\_edgelist\_fuzzer.cpp \\
\# 36 & igraph & igraph\_read\_graph\_gml & 6 & read\_gml\_fuzzer.cpp \\
\cellcolor{black!10}\# 37 & \cellcolor{black!10}igraph & \cellcolor{black!10}igraph\_read\_graph\_graphdb & \cellcolor{black!10}6 & \cellcolor{black!10}read\_graphdb\_fuzzer.cpp \\
\# 38 & igraph & igraph\_read\_graph\_graphml & 6 & read\_graphml\_fuzzer.cpp \\
\cellcolor{black!10}\# 39 & \cellcolor{black!10}igraph & \cellcolor{black!10}igraph\_read\_graph\_lgl & \cellcolor{black!10}6 & \cellcolor{black!10}read\_lgl\_fuzzer.cpp \\
\# 40 & igraph & igraph\_read\_graph\_pajek & 6 & read\_pajek\_fuzzer.cpp \\
\cellcolor{black!10}\# 41 & \cellcolor{black!10}inchi & \cellcolor{black!10}GetINCHIfromINCHI & \cellcolor{black!10}6 & \cellcolor{black!10}inchi\_input\_fuzzer.c \\
\# 42 & inchi & GetStructFromINCHI & 6 & inchi\_input\_fuzzer.c \\
\cellcolor{black!10}\# 43 & \cellcolor{black!10}kamailio & \cellcolor{black!10}parse\_msg & \cellcolor{black!10}6 & \cellcolor{black!10}fuzz\_parse\_msg.c \\
\# 44 & libyang & lys\_parse\_mem & 6 & lys\_parse\_mem.c \\
\cellcolor{black!10}\# 45 & \cellcolor{black!10}proftpd & \cellcolor{black!10}pr\_json\_object\_from\_text & \cellcolor{black!10}6 & \cellcolor{black!10}fuzzer.c \\
\# 46 & selinux & policydb\_read & 6 & binpolicy-fuzzer.c \\
\cellcolor{black!10}\# 47 & \cellcolor{black!10}kamailio & \cellcolor{black!10}get\_src\_address\_socket & \cellcolor{black!10}7 & \cellcolor{black!10}fuzz\_parse\_msg.c \\
\# 48 & kamailio & get\_src\_uri & 7 & fuzz\_parse\_msg.c \\
\cellcolor{black!10}\# 49 & \cellcolor{black!10}kamailio & \cellcolor{black!10}parse\_content\_disposition & \cellcolor{black!10}7 & \cellcolor{black!10}fuzz\_parse\_msg.c \\
\# 50 & kamailio & parse\_diversion\_header & 7 & fuzz\_parse\_msg.c \\
\cellcolor{black!10}\# 51 & \cellcolor{black!10}kamailio & \cellcolor{black!10}parse\_from\_header & \cellcolor{black!10}7 & \cellcolor{black!10}fuzz\_parse\_msg.c \\
\# 52 & kamailio & parse\_from\_uri & 7 & fuzz\_parse\_msg.c \\
\cellcolor{black!10}\# 53 & \cellcolor{black!10}kamailio & \cellcolor{black!10}parse\_headers & \cellcolor{black!10}7 & \cellcolor{black!10}fuzz\_parse\_msg.c \\
\# 54 & kamailio & parse\_identityinfo\_header & 7 & fuzz\_parse\_msg.c \\
\cellcolor{black!10}\# 55 & \cellcolor{black!10}kamailio & \cellcolor{black!10}parse\_pai\_header & \cellcolor{black!10}7 & \cellcolor{black!10}fuzz\_parse\_msg.c \\
\# 56 & kamailio & parse\_privacy & 7 & fuzz\_parse\_msg.c \\
\cellcolor{black!10}\# 57 & \cellcolor{black!10}kamailio & \cellcolor{black!10}parse\_record\_route\_headers & \cellcolor{black!10}7 & \cellcolor{black!10}fuzz\_parse\_msg.c \\
\# 58 & kamailio & parse\_refer\_to\_header & 7 & fuzz\_parse\_msg.c \\
\cellcolor{black!10}\# 59 & \cellcolor{black!10}kamailio & \cellcolor{black!10}parse\_route\_headers & \cellcolor{black!10}7 & \cellcolor{black!10}fuzz\_parse\_msg.c \\
\# 60 & kamailio & parse\_to\_header & 7 & fuzz\_parse\_msg.c \\
\cellcolor{black!10}\# 61 & \cellcolor{black!10}kamailio & \cellcolor{black!10}parse\_to\_uri & \cellcolor{black!10}7 & \cellcolor{black!10}fuzz\_parse\_msg.c \\
\# 62 & libyang & lyd\_parse\_data\_mem & 7 & \tabincell{c}{lyd\_parse\_mem\_json.c \\ lyd\_parse\_mem\_xml.c} \\
\cellcolor{black!10}\# 63 & \cellcolor{black!10}bind9 & \cellcolor{black!10}dns\_message\_parse & \cellcolor{black!10}8 & \cellcolor{black!10}dns\_message\_parse.c \\
\# 64 & igraph & igraph\_read\_graph\_ncol & 8 & read\_ncol\_fuzzer.cpp \\
\cellcolor{black!10}\# 65 & \cellcolor{black!10}pjsip & \cellcolor{black!10}pj\_json\_parse & \cellcolor{black!10}8 & \cellcolor{black!10}fuzz-json.c \\
\# 66 & pjsip & pj\_xml\_parse & 8 & fuzz-xml.c \\
\cellcolor{black!10}\# 67 & \cellcolor{black!10}pjsip & \cellcolor{black!10}pjmedia\_sdp\_parse & \cellcolor{black!10}8 & \cellcolor{black!10}fuzz-sdp.c \\
\# 68 & quickjs & lre\_compile & 8 & fuzz\_regexp.c \\
\cellcolor{black!10}\# 69 & \cellcolor{black!10}bind9 & \cellcolor{black!10}isc\_lex\_getmastertoken & \cellcolor{black!10}9 & \cellcolor{black!10}isc\_lex\_getmastertoken.c \\
\# 70 & bind9 & isc\_lex\_gettoken & 9 & isc\_lex\_gettoken.c \\
\cellcolor{black!10}\# 71 & \cellcolor{black!10}quickjs & \cellcolor{black!10}JS\_Eval & \cellcolor{black!10}9 & \cellcolor{black!10}fuzz\_eval.c, fuzz\_compile.c \\
\# 72 & igraph & igraph\_edge\_connectivity & 10 & edge\_connectivity\_fuzzer.cpp \\
\cellcolor{black!10}\# 73 & \cellcolor{black!10}pjsip & \cellcolor{black!10}pj\_stun\_msg\_decode & \cellcolor{black!10}10 & \cellcolor{black!10}fuzz-stun.c \\
\# 74 & bind9 & dns\_message\_checksig & 11 & dns\_message\_checksig.c \\
\cellcolor{black!10}\# 75 & \cellcolor{black!10}libzip & \cellcolor{black!10}zip\_fread & \cellcolor{black!10}11 & \cellcolor{black!10}zip\_read\_fuzzer.cc \\
\# 76 & bind9 & dns\_rdata\_fromtext & 12 & \tabincell{c}{dns\_rdata\_fromwire\_text.c \\ dns\_rdata\_fromtext.c}  \\
\cellcolor{black!10}\# 77 & \cellcolor{black!10}igraph & \cellcolor{black!10}igraph\_all\_minimal\_st\_separators & \cellcolor{black!10}12 & \cellcolor{black!10}vertex\_separators\_fuzzer.cpp \\
\# 78 & igraph & igraph\_minimum\_size\_separators & 12 & vertex\_separators\_fuzzer.cpp \\
\cellcolor{black!10}\# 79 & \cellcolor{black!10}pjsip & \cellcolor{black!10}pjsip\_parse\_msg & \cellcolor{black!10}12 & \cellcolor{black!10}fuzz-sip.c \\
\# 80 & igraph & igraph\_automorphism\_group & 13 & bliss\_fuzzer.cpp \\
\cellcolor{black!10}\# 81 & \cellcolor{black!10}libmodbus & \cellcolor{black!10}modbus\_read\_bits & \cellcolor{black!10}15 & \cellcolor{black!10}FuzzClient.c \\
\# 82 & libmodbus & modbus\_read\_registers & 15 & FuzzClient.c \\
\cellcolor{black!10}\# 83 & \cellcolor{black!10}civetweb & \cellcolor{black!10}mg\_get\_response & \cellcolor{black!10}17 & \cellcolor{black!10}fuzzmain.c \\
\# 84 & bind9 & dns\_master\_loadbuffer & 20 & dns\_master\_load.c \\
\cellcolor{black!10}\# 85 & \cellcolor{black!10}libmodbus & \cellcolor{black!10}modbus\_receive & \cellcolor{black!10}33 & \cellcolor{black!10}FuzzServer.c \\
\# 86 & tmux & input\_parse\_buffer & 42 & input-fuzzer.c \\
\bottomrule
\end{xltabular}
%\end{tabular}}
%\end{table*}
\twocolumn

% \section{Detail of Score Calculation}
% \label{sec:score-calculation-detail}

\begin{table*}
\caption{Full List of Evaluation Results}
\label{tab:eval_full}
\renewcommand\arraystretch{1}
\scriptsize
\setlength\arrayrulewidth{2pt}\arrayrulecolor{black}
\resizebox{\textwidth}{!}{
\begin{tabular}{cp{3.6cm} 
        >{\columncolor[gray]{1}[0.8\tabcolsep]}c
        >{\columncolor[gray]{1}[0.8\tabcolsep]}c
        >{\columncolor[gray]{1}[0.8\tabcolsep]}c
        >{\columncolor[gray]{1}[0.8\tabcolsep]}c
        >{\columncolor[gray]{1}[0.8\tabcolsep]}c
        >{\columncolor[gray]{1}[0.8\tabcolsep]}c
        >{\columncolor[gray]{1}[0.8\tabcolsep]}c
        >{\columncolor[gray]{1}[0.8\tabcolsep]}c
        >{\columncolor[gray]{1}[0.8\tabcolsep]}c
        >{\columncolor[gray]{1}[0.8\tabcolsep]}c
        >{\columncolor[gray]{1}[0.8\tabcolsep]}c
        >{\columncolor[gray]{1}[0.8\tabcolsep]}c
        >{\columncolor[gray]{1}[0.8\tabcolsep]}c
    }


\toprule
% Question & SCORE & gpt4-NAIVE & gpt4-BACTX & gpt4-UGCTX & gpt4-DOCTX & gpt4-BA-ITER & gpt4-ALL-ITER & gpt3.5-NAIVE & gpt3.5-BACTX & gpt3.5-UGCTX & gpt3.5-DOCTX & gpt3.5-BA-ITER & gpt3.5-ALL-ITER \\
 \multirow{2}{*}{Index} & \multirow{2}{*}{Question} & \multirow{2}{*}{Score} &   \multicolumn{6}{c}{GPT4} &  \multicolumn{6}{c}{GPT3.5} \\
&  & & NAIVE & BACTX & UGCTX & DOCTX & BA-ITER & EX-ITER &  NAIVE &  BACTX &  UGCTX &  DOCTX &  BA-ITER & EX-ITER \\
\midrule
\cellcolor{black!10}\# 1 & \cellcolor{black!10}\url{coturn_stun_is_command_message_full_check_str} & \cellcolor{black!10}1 & 0/40 & \cellcolor{green!72.50}29/40 & 0/40 & - & - & - & 0/40 & \cellcolor{green!67.50}27/40 & \cellcolor{green!12.50}5/40 & - & - & - \\
\# 2 & \url{kamailio_parse_uri} & 1 & 0/40 & \cellcolor{green!100.00}40/40 & \cellcolor{green!55.00}22/40 & - & - & - & 0/40 & \cellcolor{green!85.00}34/40 & \cellcolor{green!52.50}21/40 & - & - & - \\
\cellcolor{black!10}\# 3 & \cellcolor{black!10}\url{coturn_stun_check_message_integrity_str} & \cellcolor{black!10}2 & 0/40 & \cellcolor{green!30.00}12/40 & \cellcolor{green!20.00}8/40 & - & - & - & 0/40 & \cellcolor{green!20.00}8/40 & \cellcolor{green!10}2/40 & - & - & - \\
\# 4 & \url{libiec61850_MmsValue_decodeMmsData} & 2 & 0/40 & \cellcolor{green!97.50}39/40 & \cellcolor{green!27.50}11/40 & \cellcolor{green!92.50}37/40 & - & - & 0/40 & \cellcolor{green!92.50}37/40 & \cellcolor{green!30.00}12/40 & \cellcolor{green!67.50}27/40 & - & - \\
\cellcolor{black!10}\# 5 & \cellcolor{black!10}\url{md4c_md_html} & \cellcolor{black!10}2 & 0/40 & 0/40 & 0/40 & 0/40 & \cellcolor{green!50.00}20/40 & \cellcolor{green!17.65}12/68 & 0/40 & 0/40 & 0/40 & 0/40 & \cellcolor{green!26.23}16/61 & \cellcolor{green!10}2/113 \\
\# 6 & \url{spdk_spdk_json_parse} & 2 & 4/40 & \cellcolor{green!87.50}35/40 & \cellcolor{green!15.00}6/40 & - & - & - & 0/40 & \cellcolor{green!75.00}30/40 & \cellcolor{green!15.00}6/40 & - & - & - \\
\cellcolor{black!10}\# 7 & \cellcolor{black!10}\url{croaring_roaring_bitmap_portable_deserialize_safe} & \cellcolor{black!10}3 & \cellcolor{green!20.00}8/40 & \cellcolor{green!100.00}40/40 & \cellcolor{green!72.50}29/40 & \cellcolor{green!70.00}28/40 & - & - & \cellcolor{green!27.50}11/40 & \cellcolor{green!55.00}22/40 & \cellcolor{green!25.00}10/40 & \cellcolor{green!75.00}30/40 & - & - \\
\# 8 & \url{lua_luaL_loadbufferx} & 3 & \cellcolor{green!67.50}27/40 & \cellcolor{green!97.50}39/40 & \cellcolor{green!60.00}24/40 & \cellcolor{green!100.00}40/40 & - & - & \cellcolor{green!77.50}31/40 & \cellcolor{green!82.50}33/40 & \cellcolor{green!42.50}17/40 & \cellcolor{green!85.00}34/40 & - & - \\
\cellcolor{black!10}\# 9 & \cellcolor{black!10}\url{w3m_wc_Str_conv_with_detect} & \cellcolor{black!10}3 & 0/40 & 0/40 & \cellcolor{green!25.00}10/40 & - & - & - & 0/40 & 0/40 & \cellcolor{green!25.00}10/40 & - & - & - \\
\# 10 & \url{bind9_dns_name_fromwire} & 4 & 0/40 & \cellcolor{green!20.00}8/40 & \cellcolor{green!10}3/40 & - & - & - & 0/40 & 0/40 & \cellcolor{green!10}1/40 & - & - & - \\
\cellcolor{black!10}\# 11 & \cellcolor{black!10}\url{gdk-pixbuf_gdk_pixbuf_animation_new_from_file} & \cellcolor{black!10}4 & \cellcolor{green!15.00}6/40 & \cellcolor{green!82.50}33/40 & \cellcolor{green!37.50}15/40 & \cellcolor{green!67.50}27/40 & - & - & \cellcolor{green!10}3/40 & \cellcolor{green!27.50}11/40 & \cellcolor{green!10}3/40 & \cellcolor{green!22.50}9/40 & - & - \\
\# 12 & \url{gdk-pixbuf_gdk_pixbuf_new_from_data} & 4 & \cellcolor{green!10}2/40 & \cellcolor{green!40.00}16/40 & \cellcolor{green!27.50}11/40 & \cellcolor{green!27.50}11/40 & - & - & \cellcolor{green!45.00}18/40 & \cellcolor{green!62.50}25/40 & \cellcolor{green!15.00}6/40 & \cellcolor{green!57.50}23/40 & - & - \\
\cellcolor{black!10}\# 13 & \cellcolor{black!10}\url{gdk-pixbuf_gdk_pixbuf_new_from_file} & \cellcolor{black!10}4 & \cellcolor{green!25.00}10/40 & \cellcolor{green!97.50}39/40 & \cellcolor{green!50.00}20/40 & \cellcolor{green!92.50}37/40 & - & - & \cellcolor{green!12.50}5/40 & \cellcolor{green!35.00}14/40 & \cellcolor{green!12.50}5/40 & \cellcolor{green!20.00}8/40 & - & - \\
\# 14 & \url{gdk-pixbuf_gdk_pixbuf_new_from_stream} & 4 & \cellcolor{green!12.50}5/40 & \cellcolor{green!75.00}30/40 & \cellcolor{green!60.00}24/40 & \cellcolor{green!65.00}26/40 & - & - & \cellcolor{green!52.50}21/40 & \cellcolor{green!87.50}35/40 & \cellcolor{green!45.00}18/40 & \cellcolor{green!80.00}32/40 & - & - \\
\cellcolor{black!10}\# 15 & \cellcolor{black!10}\url{gpac_gf_isom_open_file} & \cellcolor{black!10}4 & \cellcolor{green!10}1/40 & \cellcolor{green!67.50}27/40 & \cellcolor{green!50.00}20/40 & - & - & - & 0/40 & \cellcolor{green!12.50}5/40 & 0/40 & - & - & - \\
\# 16 & \url{libbpf_bpf_object__open_mem} & 4 & \cellcolor{green!10}1/40 & \cellcolor{green!15.00}6/40 & 4/40 & \cellcolor{green!15.00}6/40 & - & - & 0/40 & \cellcolor{green!27.50}11/40 & \cellcolor{green!15.00}6/40 & \cellcolor{green!12.50}5/40 & - & - \\
\cellcolor{black!10}\# 17 & \cellcolor{black!10}\url{libpg_query_pg_query_parse} & \cellcolor{black!10}4 & 4/40 & \cellcolor{green!90.00}36/40 & \cellcolor{green!95.00}38/40 & - & - & - & \cellcolor{green!15.00}6/40 & \cellcolor{green!42.50}17/40 & \cellcolor{green!65.00}26/40 & - & - & - \\
\# 18 & \url{libucl_ucl_parser_add_string} & 4 & \cellcolor{green!20.00}8/40 & \cellcolor{green!47.50}19/40 & \cellcolor{green!50.00}20/40 & \cellcolor{green!72.50}29/40 & - & - & \cellcolor{green!17.50}7/40 & \cellcolor{green!20.00}8/40 & \cellcolor{green!12.50}5/40 & \cellcolor{green!45.00}18/40 & - & - \\
\cellcolor{black!10}\# 19 & \cellcolor{black!10}\url{oniguruma_onig_new} & \cellcolor{black!10}4 & \cellcolor{green!50.00}20/40 & \cellcolor{green!87.50}35/40 & \cellcolor{green!55.00}22/40 & \cellcolor{green!82.50}33/40 & - & - & \cellcolor{green!30.00}12/40 & \cellcolor{green!45.00}18/40 & \cellcolor{green!12.50}5/40 & \cellcolor{green!37.50}15/40 & - & - \\
\# 20 & \url{pupnp_ixmlLoadDocumentEx} & 4 & 0/40 & \cellcolor{green!65.00}26/40 & \cellcolor{green!75.00}15/20 & \cellcolor{green!40.00}16/40 & - & - & 0/40 & \cellcolor{green!17.50}7/40 & \cellcolor{green!20.00}4/20 & \cellcolor{green!10}1/40 & - & - \\
\cellcolor{black!10}\# 21 & \cellcolor{black!10}\url{gdk-pixbuf_gdk_pixbuf_new_from_file_at_scale} & \cellcolor{black!10}5 & \cellcolor{green!45.00}18/40 & \cellcolor{green!72.50}29/40 & \cellcolor{green!40.00}16/40 & \cellcolor{green!70.00}28/40 & - & - & \cellcolor{green!10}2/40 & \cellcolor{green!10}1/40 & \cellcolor{green!10}3/40 & \cellcolor{green!17.50}7/40 & - & - \\
\# 22 & \url{inchi_GetINCHIKeyFromINCHI} & 5 & 0/40 & \cellcolor{green!75.00}30/40 & \cellcolor{green!22.50}9/40 & \cellcolor{green!77.50}31/40 & - & - & 0/40 & \cellcolor{green!40.00}16/40 & \cellcolor{green!25.00}10/40 & \cellcolor{green!60.00}24/40 & - & - \\
\cellcolor{black!10}\# 23 & \cellcolor{black!10}\url{libdwarf_dwarf_init_b} & \cellcolor{black!10}5 & 0/40 & 0/40 & \cellcolor{green!32.50}13/40 & \cellcolor{green!10}1/40 & - & - & 0/40 & 0/40 & \cellcolor{green!27.50}11/40 & \cellcolor{green!12.50}5/40 & - & - \\
\# 24 & \url{libdwarf_dwarf_init_path} & 5 & 0/40 & 0/40 & \cellcolor{green!20.00}8/40 & 0/40 & - & - & 0/40 & 0/40 & 4/40 & 0/40 & - & - \\
\cellcolor{black!10}\# 25 & \cellcolor{black!10}\url{liblouis_lou_compileString} & \cellcolor{black!10}5 & 0/40 & \cellcolor{green!22.50}9/40 & \cellcolor{green!50.00}20/40 & \cellcolor{green!20.00}8/40 & - & - & 0/40 & \cellcolor{green!17.50}7/40 & \cellcolor{green!12.50}5/40 & \cellcolor{green!30.00}12/40 & - & - \\
\# 26 & \url{selinux_cil_compile} & 5 & 0/40 & 0/40 & \cellcolor{green!62.50}25/40 & - & - & - & 0/40 & 0/40 & \cellcolor{green!27.50}11/40 & - & - & - \\
\cellcolor{black!10}\# 27 & \cellcolor{black!10}\url{bind9_dns_name_fromtext} & \cellcolor{black!10}6 & 0/40 & \cellcolor{green!55.00}22/40 & \cellcolor{green!15.00}6/40 & - & - & - & 0/40 & 0/40 & \cellcolor{green!10}3/40 & - & - & - \\
\# 28 & \url{bind9_dns_rdata_fromwire} & 6 & 0/40 & 0/40 & \cellcolor{green!10}1/40 & - & - & - & 0/40 & 0/40 & \cellcolor{green!10}1/40 & - & - & - \\
\cellcolor{black!10}\# 29 & \cellcolor{black!10}\url{coturn_stun_is_binding_response} & \cellcolor{black!10}6 & 0/40 & \cellcolor{green!60.00}24/40 & \cellcolor{green!30.00}12/40 & - & - & - & 0/40 & 0/40 & \cellcolor{green!35.00}14/40 & - & - & - \\
\# 30 & \url{coturn_stun_is_command_message} & 6 & 0/40 & \cellcolor{green!30.00}12/40 & \cellcolor{green!37.50}15/40 & \cellcolor{green!27.50}11/40 & - & - & 0/40 & 0/40 & \cellcolor{green!42.50}17/40 & 0/40 & - & - \\
\cellcolor{black!10}\# 31 & \cellcolor{black!10}\url{coturn_stun_is_response} & \cellcolor{black!10}6 & 0/40 & \cellcolor{green!35.00}14/40 & 0/40 & - & - & - & 0/40 & 0/40 & \cellcolor{green!25.00}10/40 & - & - & - \\
\# 32 & \url{coturn_stun_is_success_response} & 6 & 0/40 & \cellcolor{green!40.00}16/40 & \cellcolor{green!37.50}15/40 & - & - & - & 0/40 & 0/40 & \cellcolor{green!15.00}6/40 & - & - & - \\
\cellcolor{black!10}\# 33 & \cellcolor{black!10}\url{hiredis_redisFormatCommand} & \cellcolor{black!10}6 & \cellcolor{green!32.50}13/40 & \cellcolor{green!100.00}40/40 & \cellcolor{green!22.50}9/40 & - & - & - & \cellcolor{green!10}2/40 & \cellcolor{green!77.50}31/40 & \cellcolor{green!20.00}8/40 & - & - & - \\
\# 34 & \url{igraph_igraph_read_graph_dl} & 6 & \cellcolor{green!22.50}9/40 & 0/40 & 4/40 & 0/40 & - & - & 0/40 & 0/40 & 0/40 & 0/40 & - & - \\
\cellcolor{black!10}\# 35 & \cellcolor{black!10}\url{igraph_igraph_read_graph_edgelist} & \cellcolor{black!10}6 & \cellcolor{green!15.00}6/40 & 0/40 & \cellcolor{green!10}2/40 & \cellcolor{green!10}1/40 & - & - & 0/40 & 0/40 & 0/40 & 0/40 & - & - \\
\# 36 & \url{igraph_igraph_read_graph_gml} & 6 & \cellcolor{green!10}3/40 & \cellcolor{green!10}2/40 & \cellcolor{green!10}1/40 & 0/40 & \cellcolor{green!10}9/187 & \cellcolor{green!10}9/98 & 0/40 & 0/40 & 0/40 & 0/40 & \cellcolor{green!10}4/214 & \cellcolor{green!10}2/193 \\
\cellcolor{black!10}\# 37 & \cellcolor{black!10}\url{igraph_igraph_read_graph_graphdb} & \cellcolor{black!10}6 & \cellcolor{green!10}3/40 & 0/40 & \cellcolor{green!15.00}6/40 & 0/40 & - & - & 0/40 & 0/40 & \cellcolor{green!10}1/40 & 0/40 & - & - \\
\# 38 & \url{igraph_igraph_read_graph_graphml} & 6 & \cellcolor{green!10}3/40 & \cellcolor{green!10}1/40 & \cellcolor{green!32.50}13/40 & \cellcolor{green!10}1/40 & - & - & 0/40 & 0/40 & \cellcolor{green!10}2/40 & 0/40 & - & - \\
\cellcolor{black!10}\# 39 & \cellcolor{black!10}\url{igraph_igraph_read_graph_lgl} & \cellcolor{black!10}6 & \cellcolor{green!10}1/40 & 0/40 & \cellcolor{green!32.50}13/40 & 0/40 & \cellcolor{green!10}1/207 & \cellcolor{green!10}7/107 & 0/40 & 0/40 & 0/40 & 0/40 & \cellcolor{green!10}2/251 & \cellcolor{green!10}1/150 \\
\# 40 & \url{igraph_igraph_read_graph_pajek} & 6 & \cellcolor{green!15.00}6/40 & \cellcolor{green!10}1/40 & \cellcolor{green!10}2/40 & 0/40 & \cellcolor{green!10}8/178 & \cellcolor{green!10.71}9/84 & 0/40 & 0/40 & 0/40 & 0/40 & \cellcolor{green!10}3/199 & \cellcolor{green!10}2/166 \\
\cellcolor{black!10}\# 41 & \cellcolor{black!10}\url{inchi_GetINCHIfromINCHI} & \cellcolor{black!10}6 & \cellcolor{green!20.00}8/40 & \cellcolor{green!67.50}27/40 & \cellcolor{green!47.50}19/40 & \cellcolor{green!80.00}32/40 & - & - & 0/40 & \cellcolor{green!10}1/40 & \cellcolor{green!20.00}8/40 & \cellcolor{green!12.50}5/40 & - & - \\
\# 42 & \url{inchi_GetStructFromINCHI} & 6 & 0/40 & \cellcolor{green!52.50}21/40 & \cellcolor{green!22.50}9/40 & \cellcolor{green!22.50}9/40 & - & - & 0/40 & \cellcolor{green!10}1/40 & 4/40 & \cellcolor{green!10}2/40 & - & - \\
\cellcolor{black!10}\# 43 & \cellcolor{black!10}\url{kamailio_parse_msg} & \cellcolor{black!10}6 & 0/40 & \cellcolor{green!57.50}23/40 & \cellcolor{green!22.50}9/40 & - & - & - & 0/40 & \cellcolor{green!32.50}13/40 & \cellcolor{green!30.00}12/40 & - & - & - \\
\# 44 & \url{libyang_lys_parse_mem} & 6 & \cellcolor{green!10}1/40 & \cellcolor{green!10}2/40 & \cellcolor{green!20.00}8/40 & 4/40 & - & - & 0/40 & 0/40 & \cellcolor{green!10}1/40 & 0/40 & - & - \\
\cellcolor{black!10}\# 45 & \cellcolor{black!10}\url{proftpd_pr_json_object_from_text} & \cellcolor{black!10}6 & 0/40 & 0/40 & \cellcolor{green!72.50}29/40 & - & - & - & 0/40 & 0/40 & \cellcolor{green!12.50}5/40 & - & - & - \\
\# 46 & \url{selinux_policydb_read} & 6 & 0/40 & \cellcolor{green!30.00}12/40 & \cellcolor{green!42.50}17/40 & - & - & - & 0/40 & \cellcolor{green!12.50}5/40 & \cellcolor{green!10}2/40 & - & - & - \\
\cellcolor{black!10}\# 47 & \cellcolor{black!10}\url{kamailio_get_src_address_socket} & \cellcolor{black!10}7 & 0/40 & \cellcolor{green!10}2/40 & \cellcolor{green!27.50}11/40 & 0/40 & - & - & 0/40 & 0/40 & \cellcolor{green!35.00}14/40 & 0/40 & - & - \\
\# 48 & \url{kamailio_get_src_uri} & 7 & 0/40 & 0/40 & \cellcolor{green!22.50}9/40 & \cellcolor{green!10}1/40 & - & - & 0/40 & \cellcolor{green!10}1/40 & \cellcolor{green!20.00}8/40 & 0/40 & - & - \\
\cellcolor{black!10}\# 49 & \cellcolor{black!10}\url{kamailio_parse_content_disposition} & \cellcolor{black!10}7 & 0/40 & 0/40 & 4/40 & 0/40 & - & - & 0/40 & 0/40 & 0/40 & 0/40 & - & - \\
\# 50 & \url{kamailio_parse_diversion_header} & 7 & 0/40 & 0/40 & \cellcolor{green!27.50}11/40 & 0/40 & - & - & 0/40 & 0/40 & \cellcolor{green!10}3/40 & 0/40 & - & - \\
\cellcolor{black!10}\# 51 & \cellcolor{black!10}\url{kamailio_parse_from_header} & \cellcolor{black!10}7 & 0/40 & 0/40 & 0/40 & - & - & - & 0/40 & 0/40 & 0/40 & - & - & - \\
\# 52 & \url{kamailio_parse_from_uri} & 7 & 0/40 & 0/40 & \cellcolor{green!10}1/40 & - & - & - & 0/40 & 0/40 & 0/40 & - & - & - \\
\cellcolor{black!10}\# 53 & \cellcolor{black!10}\url{kamailio_parse_headers} & \cellcolor{black!10}7 & 0/40 & 0/40 & 0/40 & - & - & - & 0/40 & 0/40 & 0/40 & - & - & - \\
\# 54 & \url{kamailio_parse_identityinfo_header} & 7 & 0/40 & 0/40 & \cellcolor{green!50.00}20/40 & - & - & - & 0/40 & 0/40 & \cellcolor{green!17.50}7/40 & - & - & - \\
\cellcolor{black!10}\# 55 & \cellcolor{black!10}\url{kamailio_parse_pai_header} & \cellcolor{black!10}7 & 0/40 & 0/40 & 4/40 & - & - & - & 0/40 & 0/40 & \cellcolor{green!10}2/40 & - & - & - \\
\# 56 & \url{kamailio_parse_privacy} & 7 & 0/40 & 0/40 & \cellcolor{green!22.50}9/40 & 0/40 & - & - & 0/40 & 0/40 & \cellcolor{green!10}2/40 & 0/40 & - & - \\
\cellcolor{black!10}\# 57 & \cellcolor{black!10}\url{kamailio_parse_record_route_headers} & \cellcolor{black!10}7 & 0/40 & 0/40 & \cellcolor{green!100.00}40/40 & - & - & - & 0/40 & 0/40 & \cellcolor{green!12.50}5/40 & - & - & - \\
\# 58 & \url{kamailio_parse_refer_to_header} & 7 & 0/40 & 0/40 & \cellcolor{green!17.50}7/40 & - & - & - & 0/40 & 0/40 & 4/40 & - & - & - \\
\cellcolor{black!10}\# 59 & \cellcolor{black!10}\url{kamailio_parse_route_headers} & \cellcolor{black!10}7 & 0/40 & 0/40 & \cellcolor{green!87.50}35/40 & - & - & - & 0/40 & \cellcolor{green!10}1/40 & \cellcolor{green!87.50}35/40 & - & - & - \\
\# 60 & \url{kamailio_parse_to_header} & 7 & 0/40 & 0/40 & \cellcolor{green!12.50}5/40 & - & - & - & 0/40 & 0/40 & \cellcolor{green!12.50}5/40 & - & - & - \\
\cellcolor{black!10}\# 61 & \cellcolor{black!10}\url{kamailio_parse_to_uri} & \cellcolor{black!10}7 & 0/40 & 0/40 & \cellcolor{green!10}3/40 & - & - & - & 0/40 & 0/40 & 0/40 & - & - & - \\
\# 62 & \url{libyang_lyd_parse_data_mem} & 7 & 0/40 & \cellcolor{green!22.50}9/40 & \cellcolor{green!45.00}18/40 & \cellcolor{green!25.00}10/40 & \cellcolor{green!19.44}14/72 & \cellcolor{green!25.00}13/52 & 0/40 & 0/40 & 0/40 & 0/40 & \cellcolor{green!10}5/184 & \cellcolor{green!10}2/47 \\
\cellcolor{black!10}\# 63 & \cellcolor{black!10}\url{bind9_dns_message_parse} & \cellcolor{black!10}8 & 0/40 & 0/40 & 0/40 & - & - & - & 0/40 & 0/40 & \cellcolor{green!10}1/40 & - & - & - \\
\# 64 & \url{igraph_igraph_read_graph_ncol} & 8 & \cellcolor{green!10}2/40 & 0/40 & \cellcolor{green!10}1/40 & 0/40 & 0/239 & \cellcolor{green!10}4/103 & 0/40 & 0/40 & 0/40 & 0/40 & \cellcolor{green!10}4/184 & 0/98 \\
\cellcolor{black!10}\# 65 & \cellcolor{black!10}\url{pjsip_pj_json_parse} & \cellcolor{black!10}8 & 0/40 & \cellcolor{green!10}3/40 & 4/40 & \cellcolor{green!10}2/40 & \cellcolor{green!10}13/150 & \cellcolor{green!10}10/115 & 0/40 & 0/40 & 0/40 & \cellcolor{green!10}3/40 & \cellcolor{green!10}1/260 & \cellcolor{green!10}5/155 \\
\# 66 & \url{pjsip_pj_xml_parse} & 8 & 0/40 & \cellcolor{green!25.00}10/40 & 4/40 & \cellcolor{green!25.00}10/40 & - & - & 0/40 & 0/40 & 0/40 & \cellcolor{green!17.50}7/40 & - & - \\
\cellcolor{black!10}\# 67 & \cellcolor{black!10}\url{pjsip_pjmedia_sdp_parse} & \cellcolor{black!10}8 & \cellcolor{green!10}2/40 & \cellcolor{green!22.50}9/40 & \cellcolor{green!10}1/40 & \cellcolor{green!35.00}14/40 & - & - & 0/40 & \cellcolor{green!10}3/40 & \cellcolor{green!10}1/40 & 4/40 & - & - \\
\# 68 & \url{quickjs_lre_compile} & 8 & 0/40 & 0/40 & 0/40 & - & 0/180 & \cellcolor{green!10}2/91 & 0/40 & 0/40 & 0/40 & - & 0/260 & 0/122 \\
\cellcolor{black!10}\# 69 & \cellcolor{black!10}\url{bind9_isc_lex_getmastertoken} & \cellcolor{black!10}9 & 0/40 & 0/40 & 0/40 & - & \cellcolor{green!10}2/128 & \cellcolor{green!10}1/70 & 0/40 & 0/40 & 0/40 & - & 0/121 & 0/86 \\
\# 70 & \url{bind9_isc_lex_gettoken} & 9 & 0/40 & 0/40 & 0/40 & - & \cellcolor{green!10}3/227 & \cellcolor{green!10}2/62 & 0/40 & 0/40 & 0/40 & - & 0/168 & \cellcolor{green!10}1/95 \\
\cellcolor{black!10}\# 71 & \cellcolor{black!10}\url{quickjs_JS_Eval} & \cellcolor{black!10}9 & \cellcolor{green!17.50}7/40 & \cellcolor{green!95.00}38/40 & \cellcolor{green!25.00}10/40 & - & - & - & \cellcolor{green!17.50}7/40 & \cellcolor{green!32.50}13/40 & \cellcolor{green!10}3/40 & - & - & - \\
\# 72 & \url{igraph_igraph_edge_connectivity} & 10 & 0/40 & 0/40 & 0/40 & 0/40 & 0/176 & 0/62 & 0/40 & 0/40 & 0/40 & 0/40 & 0/176 & 0/105 \\
\cellcolor{black!10}\# 73 & \cellcolor{black!10}\url{pjsip_pj_stun_msg_decode} & \cellcolor{black!10}10 & 0/40 & 0/40 & 0/40 & 0/40 & \cellcolor{green!10}12/189 & \cellcolor{green!14.47}11/76 & 0/40 & 0/40 & 0/40 & 0/40 & 0/242 & \cellcolor{green!10}1/122 \\
\# 74 & \url{bind9_dns_message_checksig} & 11 & 0/40 & 0/40 & \cellcolor{green!10}1/40 & - & 0/141 & \cellcolor{green!10}1/86 & 0/40 & 0/40 & 0/40 & - & 0/204 & 0/137 \\
\cellcolor{black!10}\# 75 & \cellcolor{black!10}\url{libzip_zip_fread} & \cellcolor{black!10}11 & \cellcolor{green!57.50}23/40 & \cellcolor{green!45.00}18/40 & \cellcolor{green!55.00}22/40 & \cellcolor{green!32.50}13/40 & - & - & \cellcolor{green!10}1/40 & \cellcolor{green!10}2/40 & 4/40 & \cellcolor{green!10}3/40 & - & - \\
\# 76 & \url{bind9_dns_rdata_fromtext} & 12 & 0/40 & 0/40 & 0/40 & - & 0/159 & \cellcolor{green!10}2/59 & 0/40 & 0/40 & 0/40 & - & 0/110 & 0/49 \\
\cellcolor{black!10}\# 77 & \cellcolor{black!10}\url{igraph_igraph_all_minimal_st_separators} & \cellcolor{black!10}12 & 0/40 & \cellcolor{green!12.50}5/40 & \cellcolor{green!17.50}7/40 & \cellcolor{green!10}1/40 & 12/120 & \cellcolor{green!12.00}9/75 & 0/40 & 0/40 & 0/40 & \cellcolor{green!10}1/40 & \cellcolor{green!10}6/225 & \cellcolor{green!10}1/126 \\
\# 78 & \url{igraph_igraph_minimum_size_separators} & 12 & \cellcolor{green!10}2/40 & 4/40 & \cellcolor{green!37.50}15/40 & 4/40 & - & - & 0/40 & 0/40 & 0/40 & 0/40 & - & - \\
\cellcolor{black!10}\# 79 & \cellcolor{black!10}\url{pjsip_pjsip_parse_msg} & \cellcolor{black!10}12 & 0/40 & 0/40 & \cellcolor{green!10}1/40 & 0/40 & 0/230 & \cellcolor{green!10}2/71 & 0/40 & 0/40 & 0/40 & 0/40 & 0/289 & 0/92 \\
\# 80 & \url{igraph_igraph_automorphism_group} & 13 & 0/40 & 0/40 & \cellcolor{green!62.50}25/40 & 0/40 & 0/255 & \cellcolor{green!13.39}15/112 & 0/40 & 0/40 & 0/40 & 0/40 & 0/138 & 0/97 \\
\cellcolor{black!10}\# 81 & \cellcolor{black!10}\url{libmodbus_modbus_read_bits} & \cellcolor{black!10}15 & 0/40 & 0/40 & 0/40 & 0/40 & 0/44 & 0/35 & 0/40 & 0/40 & 0/40 & 0/40 & 0/74 & 0/73 \\
\# 82 & \url{libmodbus_modbus_read_registers} & 15 & 0/40 & 0/40 & 0/40 & 0/40 & 0/64 & 0/35 & 0/40 & 0/40 & 0/40 & 0/40 & 0/88 & 0/41 \\
\cellcolor{black!10}\# 83 & \cellcolor{black!10}\url{civetweb_mg_get_response} & \cellcolor{black!10}17 & 0/40 & 0/40 & 0/40 & 0/40 & 0/208 & 0/47 & 0/40 & 0/40 & 0/40 & 0/40 & 0/159 & 0/91 \\
\# 84 & \url{bind9_dns_master_loadbuffer} & 20 & 0/40 & 0/40 & 0/40 & - & \cellcolor{green!10}1/206 & 0/103 & 0/40 & 0/40 & 0/40 & - & 0/157 & 0/80 \\
\cellcolor{black!10}\# 85 & \cellcolor{black!10}\url{libmodbus_modbus_receive} & \cellcolor{black!10}33 & 0/40 & 0/40 & 0/40 & 0/40 & 0/44 & 0/47 & 0/40 & 0/40 & 0/40 & 0/40 & 0/72 & 0/57 \\
\# 86 & \url{tmux_input_parse_buffer} & 42 & 0/40 & 0/40 & 0/40 & - & 0/189 & 0/165 & 0/40 & 0/40 & 0/40 & - & 0/213 & 0/146 \\
\bottomrule

\end{tabular}

}
\end{table*}



\begin{table*}
\caption{Full List of Evaluation Results}
\label{tab:eval_full}
\renewcommand\arraystretch{1}
\scriptsize
\setlength\arrayrulewidth{2pt}\arrayrulecolor{black}
\resizebox{\textwidth}{!}{
\begin{tabular}{cp{3.6cm} 
        >{\columncolor[gray]{1}[0.8\tabcolsep]}c
        >{\columncolor[gray]{1}[0.8\tabcolsep]}c
        >{\columncolor[gray]{1}[0.8\tabcolsep]}c
        >{\columncolor[gray]{1}[0.8\tabcolsep]}c
        >{\columncolor[gray]{1}[0.8\tabcolsep]}c
        >{\columncolor[gray]{1}[0.8\tabcolsep]}c
        >{\columncolor[gray]{1}[0.8\tabcolsep]}c
        >{\columncolor[gray]{1}[0.8\tabcolsep]}c
        >{\columncolor[gray]{1}[0.8\tabcolsep]}c
        >{\columncolor[gray]{1}[0.8\tabcolsep]}c
        >{\columncolor[gray]{1}[0.8\tabcolsep]}c
        >{\columncolor[gray]{1}[0.8\tabcolsep]}c
        >{\columncolor[gray]{1}[0.8\tabcolsep]}c
    }


\toprule
% Question & SCORE & gpt4-NAIVE & gpt4-BACTX & gpt4-UGCTX & gpt4-DOCTX & gpt4-BA-ITER & gpt4-ALL-ITER & gpt3.5-NAIVE & gpt3.5-BACTX & gpt3.5-UGCTX & gpt3.5-DOCTX & gpt3.5-BA-ITER & gpt3.5-ALL-ITER \\
 \multirow{2}{*}{Index} & \multirow{2}{*}{Question} & \multirow{2}{*}{Score} &   \multicolumn{6}{c}{GPT4} &  \multicolumn{6}{c}{GPT3.5} \\
&  & & NAIVE & BACTX & UGCTX & DOCTX & BA-ITER & EX-ITER &  NAIVE &  BACTX &  UGCTX &  DOCTX &  BA-ITER & EX-ITER \\
\midrule
\cellcolor{black!10}\# 1 & \cellcolor{black!10}\url{coturn_stun_is_command_message_full_check_str} & \cellcolor{black!10}1 & 0/40 & \cellcolor{green!72.50}29/40 & 0/40 & - & \cellcolor{green!53.23}33/62 & \cellcolor{green!16.04}17/106 & 0/40 & \cellcolor{green!67.50}27/40 & \cellcolor{green!12.50}5/40 & - & \cellcolor{green!42.25}30/71 & \cellcolor{green!32.00}24/75 \\
\# 2 & \url{kamailio_parse_uri} & 1 & 0/40 & \cellcolor{green!100.00}40/40 & \cellcolor{green!55.00}22/40 & - & \cellcolor{green!35.96}32/89 & \cellcolor{green!23.68}27/114 & 0/40 & \cellcolor{green!85.00}34/40 & \cellcolor{green!52.50}21/40 & - & \cellcolor{green!35.63}31/87 & \cellcolor{green!10}10/155 \\
\cellcolor{black!10}\# 3 & \cellcolor{black!10}\url{coturn_stun_check_message_integrity_str} & \cellcolor{black!10}2 & 0/40 & \cellcolor{green!30.00}12/40 & \cellcolor{green!20.00}8/40 & - & \cellcolor{green!10}13/231 & \cellcolor{green!10}7/151 & 0/40 & \cellcolor{green!20.00}8/40 & \cellcolor{green!10}2/40 & - & \cellcolor{green!10}9/156 & \cellcolor{green!10}4/93 \\
\# 4 & \url{libiec61850_MmsValue_decodeMmsData} & 2 & 0/40 & \cellcolor{green!97.50}39/40 & \cellcolor{green!27.50}11/40 & \cellcolor{green!92.50}37/40 & \cellcolor{green!40.96}34/83 & \cellcolor{green!15.38}18/117 & 0/40 & \cellcolor{green!92.50}37/40 & \cellcolor{green!30.00}12/40 & \cellcolor{green!67.50}27/40 & \cellcolor{green!33.98}35/103 & \cellcolor{green!11.11}12/108 \\
\cellcolor{black!10}\# 5 & \cellcolor{black!10}\url{md4c_md_html} & \cellcolor{black!10}2 & 0/40 & 0/40 & 0/40 & 0/40 & \cellcolor{green!48.78}40/82 & \cellcolor{green!18.03}22/122 & 0/40 & 0/40 & 0/40 & 0/40 & \cellcolor{green!32.38}34/105 & \cellcolor{green!10}4/172 \\
\# 6 & \url{spdk_spdk_json_parse} & 2 & \cellcolor{green!10.00}4/40 & \cellcolor{green!87.50}35/40 & \cellcolor{green!15.00}6/40 & - & \cellcolor{green!46.48}33/71 & \cellcolor{green!20.43}19/93 & 0/40 & \cellcolor{green!75.00}30/40 & \cellcolor{green!15.00}6/40 & - & \cellcolor{green!44.16}34/77 & \cellcolor{green!10.43}12/115 \\
\cellcolor{black!10}\# 7 & \cellcolor{black!10}\url{croaring_roaring_bitmap_portable_deserialize_safe} & \cellcolor{black!10}3 & \cellcolor{green!20.00}8/40 & \cellcolor{green!100.00}40/40 & \cellcolor{green!72.50}29/40 & \cellcolor{green!70.00}28/40 & \cellcolor{green!32.29}31/96 & \cellcolor{green!24.75}25/101 & \cellcolor{green!27.50}11/40 & \cellcolor{green!55.00}22/40 & \cellcolor{green!25.00}10/40 & \cellcolor{green!75.00}30/40 & \cellcolor{green!27.45}28/102 & \cellcolor{green!17.60}22/125 \\
\# 8 & \url{lua_luaL_loadbufferx} & 3 & \cellcolor{green!67.50}27/40 & \cellcolor{green!97.50}39/40 & \cellcolor{green!60.00}24/40 & \cellcolor{green!100.00}40/40 & \cellcolor{green!72.00}36/50 & \cellcolor{green!41.33}31/75 & \cellcolor{green!77.50}31/40 & \cellcolor{green!82.50}33/40 & \cellcolor{green!42.50}17/40 & \cellcolor{green!85.00}34/40 & \cellcolor{green!54.72}29/53 & \cellcolor{green!13.51}15/111 \\
\cellcolor{black!10}\# 9 & \cellcolor{black!10}\url{w3m_wc_Str_conv_with_detect} & \cellcolor{black!10}3 & 0/40 & 0/40 & \cellcolor{green!25.00}10/40 & - & \cellcolor{green!10}2/348 & \cellcolor{green!10}17/178 & 0/40 & 0/40 & \cellcolor{green!25.00}10/40 & - & \cellcolor{green!10}1/367 & \cellcolor{green!10}10/230 \\
\# 10 & \url{bind9_dns_name_fromwire} & 4 & 0/40 & \cellcolor{green!20.00}8/40 & \cellcolor{green!10}3/40 & - & \cellcolor{green!10}7/323 & \cellcolor{green!10}12/215 & 0/40 & 0/40 & \cellcolor{green!10}1/40 & - & 0/265 & \cellcolor{green!10}1/164 \\
\cellcolor{black!10}\# 11 & \cellcolor{black!10}\url{gdk-pixbuf_gdk_pixbuf_animation_new_from_file} & \cellcolor{black!10}4 & \cellcolor{green!15.00}6/40 & \cellcolor{green!82.50}33/40 & \cellcolor{green!37.50}15/40 & \cellcolor{green!67.50}27/40 & \cellcolor{green!32.20}19/59 & \cellcolor{green!10}7/77 & \cellcolor{green!10}3/40 & \cellcolor{green!27.50}11/40 & \cellcolor{green!10}3/40 & \cellcolor{green!22.50}9/40 & \cellcolor{green!10}7/74 & \cellcolor{green!10}4/79 \\
\# 12 & \url{gdk-pixbuf_gdk_pixbuf_new_from_data} & 4 & \cellcolor{green!10}2/40 & \cellcolor{green!40.00}16/40 & \cellcolor{green!27.50}11/40 & \cellcolor{green!27.50}11/40 & \cellcolor{green!24.75}25/101 & \cellcolor{green!14.10}11/78 & \cellcolor{green!45.00}18/40 & \cellcolor{green!62.50}25/40 & \cellcolor{green!15.00}6/40 & \cellcolor{green!57.50}23/40 & \cellcolor{green!25.00}21/84 & \cellcolor{green!12.62}13/103 \\
\cellcolor{black!10}\# 13 & \cellcolor{black!10}\url{gdk-pixbuf_gdk_pixbuf_new_from_file} & \cellcolor{black!10}4 & \cellcolor{green!25.00}10/40 & \cellcolor{green!97.50}39/40 & \cellcolor{green!50.00}20/40 & \cellcolor{green!92.50}37/40 & \cellcolor{green!30.88}21/68 & \cellcolor{green!18.03}11/61 & \cellcolor{green!12.50}5/40 & \cellcolor{green!35.00}14/40 & \cellcolor{green!12.50}5/40 & \cellcolor{green!20.00}8/40 & \cellcolor{green!10}4/85 & \cellcolor{green!10}2/78 \\
\# 14 & \url{gdk-pixbuf_gdk_pixbuf_new_from_stream} & 4 & \cellcolor{green!12.50}5/40 & \cellcolor{green!75.00}30/40 & \cellcolor{green!60.00}24/40 & \cellcolor{green!65.00}26/40 & \cellcolor{green!60.00}33/55 & \cellcolor{green!37.88}25/66 & \cellcolor{green!52.50}21/40 & \cellcolor{green!87.50}35/40 & \cellcolor{green!45.00}18/40 & \cellcolor{green!80.00}32/40 & \cellcolor{green!36.71}29/79 & \cellcolor{green!18.68}17/91 \\
\cellcolor{black!10}\# 15 & \cellcolor{black!10}\url{gpac_gf_isom_open_file} & \cellcolor{black!10}4 & \cellcolor{green!10}1/40 & \cellcolor{green!67.50}27/40 & \cellcolor{green!50.00}20/40 & - & \cellcolor{green!10}5/172 & \cellcolor{green!10}5/109 & 0/40 & \cellcolor{green!12.50}5/40 & 0/40 & - & 0/257 & 0/158 \\
\# 16 & \url{libbpf_bpf_object__open_mem} & 4 & \cellcolor{green!10}1/40 & \cellcolor{green!15.00}6/40 & \cellcolor{green!10.00}4/40 & \cellcolor{green!15.00}6/40 & \cellcolor{green!10.50}19/181 & \cellcolor{green!10}13/172 & 0/40 & \cellcolor{green!27.50}11/40 & \cellcolor{green!15.00}6/40 & \cellcolor{green!12.50}5/40 & \cellcolor{green!11.11}13/117 & \cellcolor{green!10}10/116 \\
\cellcolor{black!10}\# 17 & \cellcolor{black!10}\url{libpg_query_pg_query_parse} & \cellcolor{black!10}4 & \cellcolor{green!10.00}4/40 & \cellcolor{green!90.00}36/40 & \cellcolor{green!95.00}38/40 & - & \cellcolor{green!34.09}30/88 & \cellcolor{green!34.62}27/78 & \cellcolor{green!15.00}6/40 & \cellcolor{green!42.50}17/40 & \cellcolor{green!65.00}26/40 & - & \cellcolor{green!21.95}27/123 & \cellcolor{green!14.89}21/141 \\
\# 18 & \url{libucl_ucl_parser_add_string} & 4 & \cellcolor{green!20.00}8/40 & \cellcolor{green!47.50}19/40 & \cellcolor{green!50.00}20/40 & \cellcolor{green!72.50}29/40 & \cellcolor{green!22.61}26/115 & \cellcolor{green!16.95}20/118 & \cellcolor{green!17.50}7/40 & \cellcolor{green!20.00}8/40 & \cellcolor{green!12.50}5/40 & \cellcolor{green!45.00}18/40 & \cellcolor{green!16.99}26/153 & \cellcolor{green!15.91}21/132 \\
\cellcolor{black!10}\# 19 & \cellcolor{black!10}\url{oniguruma_onig_new} & \cellcolor{black!10}4 & \cellcolor{green!50.00}20/40 & \cellcolor{green!87.50}35/40 & \cellcolor{green!55.00}22/40 & \cellcolor{green!82.50}33/40 & \cellcolor{green!21.66}34/157 & \cellcolor{green!26.47}18/68 & \cellcolor{green!30.00}12/40 & \cellcolor{green!45.00}18/40 & \cellcolor{green!12.50}5/40 & \cellcolor{green!37.50}15/40 & \cellcolor{green!22.88}27/118 & \cellcolor{green!12.96}14/108 \\
\# 20 & \url{pupnp_ixmlLoadDocumentEx} & 4 & 0/40 & \cellcolor{green!65.00}26/40 & \cellcolor{green!75.00}15/20 & \cellcolor{green!40.00}16/40 & \cellcolor{green!11.43}20/175 & \cellcolor{green!10}11/128 & 0/40 & \cellcolor{green!17.50}7/40 & \cellcolor{green!20.00}4/20 & \cellcolor{green!10}1/40 & \cellcolor{green!10}5/135 & \cellcolor{green!10}2/107 \\
\cellcolor{black!10}\# 21 & \cellcolor{black!10}\url{gdk-pixbuf_gdk_pixbuf_new_from_file_at_scale} & \cellcolor{black!10}5 & \cellcolor{green!45.00}18/40 & \cellcolor{green!72.50}29/40 & \cellcolor{green!40.00}16/40 & \cellcolor{green!70.00}28/40 & \cellcolor{green!12.77}12/94 & \cellcolor{green!10}8/87 & \cellcolor{green!10}2/40 & \cellcolor{green!10}1/40 & \cellcolor{green!10}3/40 & \cellcolor{green!17.50}7/40 & 0/108 & \cellcolor{green!10}1/98 \\
\# 22 & \url{inchi_GetINCHIKeyFromINCHI} & 5 & 0/40 & \cellcolor{green!75.00}30/40 & \cellcolor{green!22.50}9/40 & \cellcolor{green!77.50}31/40 & \cellcolor{green!35.11}33/94 & \cellcolor{green!26.04}25/96 & 0/40 & \cellcolor{green!40.00}16/40 & \cellcolor{green!25.00}10/40 & \cellcolor{green!60.00}24/40 & \cellcolor{green!20.95}31/148 & \cellcolor{green!10.96}16/146 \\
\cellcolor{black!10}\# 23 & \cellcolor{black!10}\url{libdwarf_dwarf_init_b} & \cellcolor{black!10}5 & 0/40 & 0/40 & \cellcolor{green!32.50}13/40 & \cellcolor{green!10}1/40 & \cellcolor{green!10}15/273 & \cellcolor{green!16.07}18/112 & 0/40 & 0/40 & \cellcolor{green!27.50}11/40 & \cellcolor{green!12.50}5/40 & \cellcolor{green!10}4/241 & \cellcolor{green!10}9/91 \\
\# 24 & \url{libdwarf_dwarf_init_path} & 5 & 0/40 & 0/40 & \cellcolor{green!20.00}8/40 & 0/40 & 0/371 & \cellcolor{green!10}2/122 & 0/40 & 0/40 & \cellcolor{green!10.00}4/40 & 0/40 & 0/391 & \cellcolor{green!10}1/89 \\
\cellcolor{black!10}\# 25 & \cellcolor{black!10}\url{liblouis_lou_compileString} & \cellcolor{black!10}5 & 0/40 & \cellcolor{green!22.50}9/40 & \cellcolor{green!50.00}20/40 & \cellcolor{green!20.00}8/40 & \cellcolor{green!10}18/182 & \cellcolor{green!11.68}16/137 & 0/40 & \cellcolor{green!17.50}7/40 & \cellcolor{green!12.50}5/40 & \cellcolor{green!30.00}12/40 & \cellcolor{green!10}4/186 & \cellcolor{green!10}3/129 \\
\# 26 & \url{selinux_cil_compile} & 5 & 0/40 & 0/40 & \cellcolor{green!62.50}25/40 & - & \cellcolor{green!10}3/266 & \cellcolor{green!26.04}25/96 & 0/40 & 0/40 & \cellcolor{green!27.50}11/40 & - & \cellcolor{green!10}1/298 & \cellcolor{green!10}9/91 \\
\cellcolor{black!10}\# 27 & \cellcolor{black!10}\url{bind9_dns_name_fromtext} & \cellcolor{black!10}6 & 0/40 & \cellcolor{green!55.00}22/40 & \cellcolor{green!15.00}6/40 & - & \cellcolor{green!10}16/173 & \cellcolor{green!10}14/161 & 0/40 & 0/40 & \cellcolor{green!10}3/40 & - & 0/301 & \cellcolor{green!10}3/169 \\
\# 28 & \url{bind9_dns_rdata_fromwire} & 6 & 0/40 & 0/40 & \cellcolor{green!10}1/40 & - & \cellcolor{green!10}2/332 & \cellcolor{green!10}7/224 & 0/40 & 0/40 & \cellcolor{green!10}1/40 & - & 0/220 & 0/193 \\
\cellcolor{black!10}\# 29 & \cellcolor{black!10}\url{coturn_stun_is_binding_response} & \cellcolor{black!10}6 & 0/40 & \cellcolor{green!60.00}24/40 & \cellcolor{green!30.00}12/40 & - & \cellcolor{green!16.42}22/134 & \cellcolor{green!13.68}16/117 & 0/40 & 0/40 & \cellcolor{green!35.00}14/40 & - & \cellcolor{green!10}5/189 & \cellcolor{green!10}6/83 \\
\# 30 & \url{coturn_stun_is_command_message} & 6 & 0/40 & \cellcolor{green!30.00}12/40 & \cellcolor{green!37.50}15/40 & \cellcolor{green!27.50}11/40 & \cellcolor{green!10.06}16/159 & \cellcolor{green!12.70}16/126 & 0/40 & 0/40 & \cellcolor{green!42.50}17/40 & 0/40 & \cellcolor{green!10}1/193 & \cellcolor{green!12.79}11/86 \\
\cellcolor{black!10}\# 31 & \cellcolor{black!10}\url{coturn_stun_is_response} & \cellcolor{black!10}6 & 0/40 & \cellcolor{green!35.00}14/40 & 0/40 & - & \cellcolor{green!10}13/153 & \cellcolor{green!10}8/115 & 0/40 & 0/40 & \cellcolor{green!25.00}10/40 & - & 0/216 & \cellcolor{green!10}2/125 \\
\# 32 & \url{coturn_stun_is_success_response} & 6 & 0/40 & \cellcolor{green!40.00}16/40 & \cellcolor{green!37.50}15/40 & - & \cellcolor{green!12.59}18/143 & \cellcolor{green!15.73}14/89 & 0/40 & 0/40 & \cellcolor{green!15.00}6/40 & - & \cellcolor{green!10}4/180 & \cellcolor{green!10}6/87 \\
\cellcolor{black!10}\# 33 & \cellcolor{black!10}\url{hiredis_redisFormatCommand} & \cellcolor{black!10}6 & \cellcolor{green!32.50}13/40 & \cellcolor{green!100.00}40/40 & \cellcolor{green!22.50}9/40 & - & \cellcolor{green!10.55}27/256 & \cellcolor{green!13.73}21/153 & \cellcolor{green!10}2/40 & \cellcolor{green!77.50}31/40 & \cellcolor{green!20.00}8/40 & - & \cellcolor{green!10}13/311 & \cellcolor{green!10}8/185 \\
\# 34 & \url{igraph_igraph_read_graph_dl} & 6 & \cellcolor{green!22.50}9/40 & 0/40 & \cellcolor{green!10.00}4/40 & 0/40 & \cellcolor{green!10}8/237 & \cellcolor{green!10}11/173 & 0/40 & 0/40 & 0/40 & 0/40 & \cellcolor{green!10}3/293 & \cellcolor{green!10}2/262 \\
\cellcolor{black!10}\# 35 & \cellcolor{black!10}\url{igraph_igraph_read_graph_edgelist} & \cellcolor{black!10}6 & \cellcolor{green!15.00}6/40 & 0/40 & \cellcolor{green!10}2/40 & \cellcolor{green!10}1/40 & \cellcolor{green!10}11/309 & \cellcolor{green!10}4/203 & 0/40 & 0/40 & 0/40 & 0/40 & \cellcolor{green!10}2/295 & \cellcolor{green!10}2/241 \\
\# 36 & \url{igraph_igraph_read_graph_gml} & 6 & \cellcolor{green!10}3/40 & \cellcolor{green!10}2/40 & \cellcolor{green!10}1/40 & 0/40 & \cellcolor{green!10}17/345 & \cellcolor{green!10}12/240 & 0/40 & 0/40 & 0/40 & 0/40 & \cellcolor{green!10}5/399 & \cellcolor{green!10}2/376 \\
\cellcolor{black!10}\# 37 & \cellcolor{black!10}\url{igraph_igraph_read_graph_graphdb} & \cellcolor{black!10}6 & \cellcolor{green!10}3/40 & 0/40 & \cellcolor{green!15.00}6/40 & 0/40 & \cellcolor{green!10}7/211 & \cellcolor{green!10}14/178 & 0/40 & 0/40 & \cellcolor{green!10}1/40 & 0/40 & \cellcolor{green!10}3/256 & \cellcolor{green!10}6/245 \\
\# 38 & \url{igraph_igraph_read_graph_graphml} & 6 & \cellcolor{green!10}3/40 & \cellcolor{green!10}1/40 & \cellcolor{green!32.50}13/40 & \cellcolor{green!10}1/40 & \cellcolor{green!10}11/262 & \cellcolor{green!15.75}20/127 & 0/40 & 0/40 & \cellcolor{green!10}2/40 & 0/40 & \cellcolor{green!10}10/243 & \cellcolor{green!10}8/151 \\
\cellcolor{black!10}\# 39 & \cellcolor{black!10}\url{igraph_igraph_read_graph_lgl} & \cellcolor{black!10}6 & \cellcolor{green!10}1/40 & 0/40 & \cellcolor{green!32.50}13/40 & 0/40 & \cellcolor{green!10}7/371 & \cellcolor{green!10}19/205 & 0/40 & 0/40 & 0/40 & 0/40 & \cellcolor{green!10}8/450 & \cellcolor{green!10}3/260 \\
\# 40 & \url{igraph_igraph_read_graph_pajek} & 6 & \cellcolor{green!15.00}6/40 & \cellcolor{green!10}1/40 & \cellcolor{green!10}2/40 & 0/40 & \cellcolor{green!10}19/294 & \cellcolor{green!10}17/172 & 0/40 & 0/40 & 0/40 & 0/40 & \cellcolor{green!10}5/431 & \cellcolor{green!10}3/339 \\
\cellcolor{black!10}\# 41 & \cellcolor{black!10}\url{inchi_GetINCHIfromINCHI} & \cellcolor{black!10}6 & \cellcolor{green!20.00}8/40 & \cellcolor{green!67.50}27/40 & \cellcolor{green!47.50}19/40 & \cellcolor{green!80.00}32/40 & \cellcolor{green!10}5/338 & 0/238 & 0/40 & \cellcolor{green!10}1/40 & \cellcolor{green!20.00}8/40 & \cellcolor{green!12.50}5/40 & 0/350 & \cellcolor{green!10}2/257 \\
\# 42 & \url{inchi_GetStructFromINCHI} & 6 & 0/40 & \cellcolor{green!52.50}21/40 & \cellcolor{green!22.50}9/40 & \cellcolor{green!22.50}9/40 & \cellcolor{green!10}5/366 & \cellcolor{green!10}5/220 & 0/40 & \cellcolor{green!10}1/40 & \cellcolor{green!10.00}4/40 & \cellcolor{green!10}2/40 & \cellcolor{green!10}1/316 & \cellcolor{green!10}1/387 \\
\cellcolor{black!10}\# 43 & \cellcolor{black!10}\url{kamailio_parse_msg} & \cellcolor{black!10}6 & 0/40 & \cellcolor{green!57.50}23/40 & \cellcolor{green!22.50}9/40 & - & \cellcolor{green!16.36}18/110 & \cellcolor{green!15.08}19/126 & 0/40 & \cellcolor{green!32.50}13/40 & \cellcolor{green!30.00}12/40 & - & \cellcolor{green!10}10/169 & \cellcolor{green!10}11/183 \\
\# 44 & \url{libyang_lys_parse_mem} & 6 & \cellcolor{green!10}1/40 & \cellcolor{green!10}2/40 & \cellcolor{green!20.00}8/40 & \cellcolor{green!10.00}4/40 & \cellcolor{green!10}19/287 & \cellcolor{green!13.14}18/137 & 0/40 & 0/40 & \cellcolor{green!10}1/40 & 0/40 & \cellcolor{green!10}9/295 & \cellcolor{green!10}10/191 \\
\cellcolor{black!10}\# 45 & \cellcolor{black!10}\url{proftpd_pr_json_object_from_text} & \cellcolor{black!10}6 & 0/40 & 0/40 & \cellcolor{green!72.50}29/40 & - & \cellcolor{green!10}8/303 & \cellcolor{green!10}12/161 & 0/40 & 0/40 & \cellcolor{green!12.50}5/40 & - & \cellcolor{green!10}1/317 & \cellcolor{green!10}3/152 \\
\# 46 & \url{selinux_policydb_read} & 6 & 0/40 & \cellcolor{green!30.00}12/40 & \cellcolor{green!42.50}17/40 & - & \cellcolor{green!10.95}15/137 & \cellcolor{green!13.01}16/123 & 0/40 & \cellcolor{green!12.50}5/40 & \cellcolor{green!10}2/40 & - & \cellcolor{green!10}6/163 & \cellcolor{green!10}7/117 \\
\cellcolor{black!10}\# 47 & \cellcolor{black!10}\url{kamailio_get_src_address_socket} & \cellcolor{black!10}7 & 0/40 & \cellcolor{green!10}2/40 & \cellcolor{green!27.50}11/40 & 0/40 & \cellcolor{green!10}4/160 & \cellcolor{green!10}6/170 & 0/40 & 0/40 & \cellcolor{green!35.00}14/40 & 0/40 & 0/161 & \cellcolor{green!10}8/193 \\
\# 48 & \url{kamailio_get_src_uri} & 7 & 0/40 & 0/40 & \cellcolor{green!22.50}9/40 & \cellcolor{green!10}1/40 & \cellcolor{green!10}3/162 & \cellcolor{green!10}8/156 & 0/40 & \cellcolor{green!10}1/40 & \cellcolor{green!20.00}8/40 & 0/40 & \cellcolor{green!10}1/144 & \cellcolor{green!10}2/189 \\
\cellcolor{black!10}\# 49 & \cellcolor{black!10}\url{kamailio_parse_content_disposition} & \cellcolor{black!10}7 & 0/40 & 0/40 & \cellcolor{green!10.00}4/40 & 0/40 & 0/229 & \cellcolor{green!10}3/182 & 0/40 & 0/40 & 0/40 & 0/40 & 0/236 & \cellcolor{green!10}2/160 \\
\# 50 & \url{kamailio_parse_diversion_header} & 7 & 0/40 & 0/40 & \cellcolor{green!27.50}11/40 & 0/40 & 0/247 & \cellcolor{green!10}8/126 & 0/40 & 0/40 & \cellcolor{green!10}3/40 & 0/40 & 0/216 & \cellcolor{green!10}1/177 \\
\cellcolor{black!10}\# 51 & \cellcolor{black!10}\url{kamailio_parse_from_header} & \cellcolor{black!10}7 & 0/40 & 0/40 & 0/40 & - & 0/230 & 0/173 & 0/40 & 0/40 & 0/40 & - & 0/203 & \cellcolor{green!10}1/173 \\
\# 52 & \url{kamailio_parse_from_uri} & 7 & 0/40 & 0/40 & \cellcolor{green!10}1/40 & - & \cellcolor{green!10}1/292 & \cellcolor{green!10}1/222 & 0/40 & 0/40 & 0/40 & - & 0/203 & 0/137 \\
\cellcolor{black!10}\# 53 & \cellcolor{black!10}\url{kamailio_parse_headers} & \cellcolor{black!10}7 & 0/40 & 0/40 & 0/40 & - & 0/118 & \cellcolor{green!10}1/120 & 0/40 & 0/40 & 0/40 & - & 0/195 & 0/196 \\
\# 54 & \url{kamailio_parse_identityinfo_header} & 7 & 0/40 & 0/40 & \cellcolor{green!50.00}20/40 & - & 0/212 & \cellcolor{green!10.22}14/137 & 0/40 & 0/40 & \cellcolor{green!17.50}7/40 & - & 0/223 & \cellcolor{green!10}4/169 \\
\cellcolor{black!10}\# 55 & \cellcolor{black!10}\url{kamailio_parse_pai_header} & \cellcolor{black!10}7 & 0/40 & 0/40 & \cellcolor{green!10.00}4/40 & - & 0/187 & \cellcolor{green!10}4/208 & 0/40 & 0/40 & \cellcolor{green!10}2/40 & - & 0/192 & \cellcolor{green!10}1/192 \\
\# 56 & \url{kamailio_parse_privacy} & 7 & 0/40 & 0/40 & \cellcolor{green!22.50}9/40 & 0/40 & 0/203 & \cellcolor{green!10}6/111 & 0/40 & 0/40 & \cellcolor{green!10}2/40 & 0/40 & 0/281 & 0/187 \\
\cellcolor{black!10}\# 57 & \cellcolor{black!10}\url{kamailio_parse_record_route_headers} & \cellcolor{black!10}7 & 0/40 & 0/40 & \cellcolor{green!100.00}40/40 & - & \cellcolor{green!10}1/220 & \cellcolor{green!21.13}15/71 & 0/40 & 0/40 & \cellcolor{green!12.50}5/40 & - & 0/240 & \cellcolor{green!10}6/122 \\
\# 58 & \url{kamailio_parse_refer_to_header} & 7 & 0/40 & 0/40 & \cellcolor{green!17.50}7/40 & - & 0/216 & \cellcolor{green!10}5/131 & 0/40 & 0/40 & \cellcolor{green!10.00}4/40 & - & 0/213 & \cellcolor{green!10}2/198 \\
\cellcolor{black!10}\# 59 & \cellcolor{black!10}\url{kamailio_parse_route_headers} & \cellcolor{black!10}7 & 0/40 & 0/40 & \cellcolor{green!87.50}35/40 & - & 0/173 & \cellcolor{green!11.85}16/135 & 0/40 & \cellcolor{green!10}1/40 & \cellcolor{green!87.50}35/40 & - & 0/243 & \cellcolor{green!10}5/134 \\
\# 60 & \url{kamailio_parse_to_header} & 7 & 0/40 & 0/40 & \cellcolor{green!12.50}5/40 & - & 0/236 & \cellcolor{green!10}5/129 & 0/40 & 0/40 & \cellcolor{green!12.50}5/40 & - & 0/173 & 0/212 \\
\cellcolor{black!10}\# 61 & \cellcolor{black!10}\url{kamailio_parse_to_uri} & \cellcolor{black!10}7 & 0/40 & 0/40 & \cellcolor{green!10}3/40 & - & \cellcolor{green!10}1/215 & 0/150 & 0/40 & 0/40 & 0/40 & - & 0/225 & 0/156 \\
\# 62 & \url{libyang_lyd_parse_data_mem} & 7 & 0/40 & \cellcolor{green!22.50}9/40 & \cellcolor{green!45.00}18/40 & \cellcolor{green!25.00}10/40 & \cellcolor{green!20.25}33/163 & \cellcolor{green!29.47}28/95 & 0/40 & 0/40 & 0/40 & 0/40 & \cellcolor{green!10}16/329 & \cellcolor{green!10}3/94 \\
\cellcolor{black!10}\# 63 & \cellcolor{black!10}\url{bind9_dns_message_parse} & \cellcolor{black!10}8 & 0/40 & 0/40 & 0/40 & - & \cellcolor{green!10}5/410 & \cellcolor{green!10}4/212 & 0/40 & 0/40 & \cellcolor{green!10}1/40 & - & 0/330 & \cellcolor{green!10}1/183 \\
\# 64 & \url{igraph_igraph_read_graph_ncol} & 8 & \cellcolor{green!10}2/40 & 0/40 & \cellcolor{green!10}1/40 & 0/40 & \cellcolor{green!10}1/464 & \cellcolor{green!10}4/156 & 0/40 & 0/40 & 0/40 & 0/40 & \cellcolor{green!10}4/453 & \cellcolor{green!10}1/235 \\
\cellcolor{black!10}\# 65 & \cellcolor{black!10}\url{pjsip_pj_json_parse} & \cellcolor{black!10}8 & 0/40 & \cellcolor{green!10}3/40 & \cellcolor{green!10.00}4/40 & \cellcolor{green!10}2/40 & \cellcolor{green!10}22/323 & \cellcolor{green!10}20/214 & 0/40 & 0/40 & 0/40 & \cellcolor{green!10}3/40 & \cellcolor{green!10}4/444 & \cellcolor{green!10}7/301 \\
\# 66 & \url{pjsip_pj_xml_parse} & 8 & 0/40 & \cellcolor{green!25.00}10/40 & \cellcolor{green!10.00}4/40 & \cellcolor{green!25.00}10/40 & \cellcolor{green!10}18/258 & \cellcolor{green!10}14/258 & 0/40 & 0/40 & 0/40 & \cellcolor{green!17.50}7/40 & \cellcolor{green!10}8/358 & \cellcolor{green!10}2/365 \\
\cellcolor{black!10}\# 67 & \cellcolor{black!10}\url{pjsip_pjmedia_sdp_parse} & \cellcolor{black!10}8 & \cellcolor{green!10}2/40 & \cellcolor{green!22.50}9/40 & \cellcolor{green!10}1/40 & \cellcolor{green!35.00}14/40 & \cellcolor{green!10}19/318 & \cellcolor{green!10}19/236 & 0/40 & \cellcolor{green!10}3/40 & \cellcolor{green!10}1/40 & \cellcolor{green!10.00}4/40 & \cellcolor{green!10}8/351 & \cellcolor{green!10}2/253 \\
\# 68 & \url{quickjs_lre_compile} & 8 & 0/40 & 0/40 & 0/40 & - & 0/322 & \cellcolor{green!10}7/203 & 0/40 & 0/40 & 0/40 & - & 0/505 & 0/253 \\
\cellcolor{black!10}\# 69 & \cellcolor{black!10}\url{bind9_isc_lex_getmastertoken} & \cellcolor{black!10}9 & 0/40 & 0/40 & 0/40 & - & \cellcolor{green!10}6/327 & \cellcolor{green!10}3/161 & 0/40 & 0/40 & 0/40 & - & 0/273 & 0/144 \\
\# 70 & \url{bind9_isc_lex_gettoken} & 9 & 0/40 & 0/40 & 0/40 & - & \cellcolor{green!10}8/447 & \cellcolor{green!10}5/131 & 0/40 & 0/40 & 0/40 & - & 0/342 & \cellcolor{green!10}1/146 \\
\cellcolor{black!10}\# 71 & \cellcolor{black!10}\url{quickjs_JS_Eval} & \cellcolor{black!10}9 & \cellcolor{green!17.50}7/40 & \cellcolor{green!95.00}38/40 & \cellcolor{green!25.00}10/40 & - & \cellcolor{green!27.78}30/108 & \cellcolor{green!10}14/198 & \cellcolor{green!17.50}7/40 & \cellcolor{green!32.50}13/40 & \cellcolor{green!10}3/40 & - & \cellcolor{green!17.90}29/162 & \cellcolor{green!12.96}14/108 \\
\# 72 & \url{igraph_igraph_edge_connectivity} & 10 & 0/40 & 0/40 & 0/40 & 0/40 & 0/378 & 0/146 & 0/40 & 0/40 & 0/40 & 0/40 & 0/377 & 0/220 \\
\cellcolor{black!10}\# 73 & \cellcolor{black!10}\url{pjsip_pj_stun_msg_decode} & \cellcolor{black!10}10 & 0/40 & 0/40 & 0/40 & 0/40 & \cellcolor{green!10}18/348 & \cellcolor{green!10}16/194 & 0/40 & 0/40 & 0/40 & 0/40 & \cellcolor{green!10}1/484 & \cellcolor{green!10}4/202 \\
\# 74 & \url{bind9_dns_message_checksig} & 11 & 0/40 & 0/40 & \cellcolor{green!10}1/40 & - & 0/274 & \cellcolor{green!10}1/178 & 0/40 & 0/40 & 0/40 & - & 0/392 & 0/270 \\
\cellcolor{black!10}\# 75 & \cellcolor{black!10}\url{libzip_zip_fread} & \cellcolor{black!10}11 & \cellcolor{green!57.50}23/40 & \cellcolor{green!45.00}18/40 & \cellcolor{green!55.00}22/40 & \cellcolor{green!32.50}13/40 & \cellcolor{green!10.09}23/228 & \cellcolor{green!12.50}16/128 & \cellcolor{green!10}1/40 & \cellcolor{green!10}2/40 & \cellcolor{green!10.00}4/40 & \cellcolor{green!10}3/40 & \cellcolor{green!10}3/247 & \cellcolor{green!10}5/100 \\
\# 76 & \url{bind9_dns_rdata_fromtext} & 12 & 0/40 & 0/40 & 0/40 & - & 0/329 & \cellcolor{green!10}4/116 & 0/40 & 0/40 & 0/40 & - & 0/240 & 0/110 \\
\cellcolor{black!10}\# 77 & \cellcolor{black!10}\url{igraph_igraph_all_minimal_st_separators} & \cellcolor{black!10}12 & 0/40 & \cellcolor{green!12.50}5/40 & \cellcolor{green!17.50}7/40 & \cellcolor{green!10}1/40 & \cellcolor{green!12.58}20/159 & \cellcolor{green!10.27}15/146 & 0/40 & 0/40 & 0/40 & \cellcolor{green!10}1/40 & \cellcolor{green!10}10/421 & \cellcolor{green!10}3/216 \\
\# 78 & \url{igraph_igraph_minimum_size_separators} & 12 & \cellcolor{green!10}2/40 & \cellcolor{green!10.00}4/40 & \cellcolor{green!37.50}15/40 & \cellcolor{green!10.00}4/40 & \cellcolor{green!10}13/313 & \cellcolor{green!10}9/167 & 0/40 & 0/40 & 0/40 & 0/40 & \cellcolor{green!10}3/215 & \cellcolor{green!10}2/153 \\
\cellcolor{black!10}\# 79 & \cellcolor{black!10}\url{pjsip_pjsip_parse_msg} & \cellcolor{black!10}12 & 0/40 & 0/40 & \cellcolor{green!10}1/40 & 0/40 & 0/453 & \cellcolor{green!10}4/175 & 0/40 & 0/40 & 0/40 & 0/40 & 0/527 & \cellcolor{green!10}1/255 \\
\# 80 & \url{igraph_igraph_automorphism_group} & 13 & 0/40 & 0/40 & \cellcolor{green!62.50}25/40 & 0/40 & 0/440 & \cellcolor{green!11.32}24/212 & 0/40 & 0/40 & 0/40 & 0/40 & 0/348 & \cellcolor{green!10}1/167 \\
\cellcolor{black!10}\# 81 & \cellcolor{black!10}\url{libmodbus_modbus_read_bits} & \cellcolor{black!10}15 & 0/40 & 0/40 & 0/40 & 0/40 & 0/82 & 0/82 & 0/40 & 0/40 & 0/40 & 0/40 & 0/141 & 0/132 \\
\# 82 & \url{libmodbus_modbus_read_registers} & 15 & 0/40 & 0/40 & 0/40 & 0/40 & 0/98 & 0/73 & 0/40 & 0/40 & 0/40 & 0/40 & 0/154 & 0/91 \\
\cellcolor{black!10}\# 83 & \cellcolor{black!10}\url{civetweb_mg_get_response} & \cellcolor{black!10}17 & 0/40 & 0/40 & 0/40 & 0/40 & 0/373 & 0/107 & 0/40 & 0/40 & 0/40 & 0/40 & 0/285 & 0/136 \\
\# 84 & \url{bind9_dns_master_loadbuffer} & 20 & 0/40 & 0/40 & 0/40 & - & \cellcolor{green!10}1/403 & \cellcolor{green!10}2/275 & 0/40 & 0/40 & 0/40 & - & 0/311 & 0/191 \\
\cellcolor{black!10}\# 85 & \cellcolor{black!10}\url{libmodbus_modbus_receive} & \cellcolor{black!10}33 & 0/40 & 0/40 & 0/40 & 0/40 & 0/96 & 0/95 & 0/40 & 0/40 & 0/40 & 0/40 & 0/144 & 0/105 \\
\# 86 & \url{tmux_input_parse_buffer} & 42 & 0/40 & 0/40 & 0/40 & - & 0/396 & 0/319 & 0/40 & 0/40 & 0/40 & - & 0/390 & 0/282 \\
\bottomrule

\end{tabular}

}
\end{table*}




\begin{table*}
\caption{Full List of Evaluation Results}
\label{tab:eval_full}
\renewcommand\arraystretch{1}
\scriptsize
\setlength\arrayrulewidth{2pt}\arrayrulecolor{black}
\resizebox{\textwidth}{!}{
\begin{tabular}{cp{3.6cm} 
        >{\columncolor[gray]{1}[0.8\tabcolsep]}c
        >{\columncolor[gray]{1}[0.8\tabcolsep]}c
        >{\columncolor[gray]{1}[0.8\tabcolsep]}c
        >{\columncolor[gray]{1}[0.8\tabcolsep]}c
        >{\columncolor[gray]{1}[0.8\tabcolsep]}c
        >{\columncolor[gray]{1}[0.8\tabcolsep]}c
        >{\columncolor[gray]{1}[0.8\tabcolsep]}c
        >{\columncolor[gray]{1}[0.8\tabcolsep]}c
        >{\columncolor[gray]{1}[0.8\tabcolsep]}c
        >{\columncolor[gray]{1}[0.8\tabcolsep]}c
        >{\columncolor[gray]{1}[0.8\tabcolsep]}c
        >{\columncolor[gray]{1}[0.8\tabcolsep]}c
        >{\columncolor[gray]{1}[0.8\tabcolsep]}c
    }


\toprule
% Question & SCORE & gpt4-NAIVE & gpt4-BACTX & gpt4-UGCTX & gpt4-DOCTX & gpt4-BA-ITER & gpt4-ALL-ITER & gpt3.5-NAIVE & gpt3.5-BACTX & gpt3.5-UGCTX & gpt3.5-DOCTX & gpt3.5-BA-ITER & gpt3.5-ALL-ITER \\
 \multirow{2}{*}{Index} & \multirow{2}{*}{Question} & \multirow{2}{*}{Score} &   \multicolumn{6}{c}{GPT4} &  \multicolumn{6}{c}{GPT3.5} \\
&  & & NAIVE & BACTX & UGCTX & DOCTX & BA-ITER & EX-ITER &  NAIVE &  BACTX &  UGCTX &  DOCTX &  BA-ITER & EX-ITER \\
\midrule
\cellcolor{black!10}\# 1 & \cellcolor{black!10}\url{coturn_stun_is_command_message_full_check_str} & \cellcolor{black!10}1 & 0.00\% & \cellcolor{green!72.50}72.50\% & 0.00\% & - & \cellcolor{green!53.23}53.23\% & \cellcolor{green!16.04}16.04\% & 0.00\% & \cellcolor{green!67.50}67.50\% & \cellcolor{green!12.50}12.50\% & - & \cellcolor{green!42.25}42.25\% & \cellcolor{green!32.00}32.00\% \\
\# 2 & \url{kamailio_parse_uri} & 1 & 0.00\% & \cellcolor{green!100.00}100.00\% & \cellcolor{green!55.00}55.00\% & - & \cellcolor{green!35.96}35.96\% & \cellcolor{green!23.68}23.68\% & 0.00\% & \cellcolor{green!85.00}85.00\% & \cellcolor{green!52.50}52.50\% & - & \cellcolor{green!35.63}35.63\% & \cellcolor{green!10}6.45\% \\
\cellcolor{black!10}\# 3 & \cellcolor{black!10}\url{coturn_stun_check_message_integrity_str} & \cellcolor{black!10}2 & 0.00\% & \cellcolor{green!30.00}30.00\% & \cellcolor{green!20.00}20.00\% & - & \cellcolor{green!10}5.63\% & \cellcolor{green!10}4.64\% & 0.00\% & \cellcolor{green!20.00}20.00\% & \cellcolor{green!10}5.00\% & - & \cellcolor{green!10}5.77\% & \cellcolor{green!10}4.30\% \\
\# 4 & \url{libiec61850_MmsValue_decodeMmsData} & 2 & 0.00\% & \cellcolor{green!97.50}97.50\% & \cellcolor{green!27.50}27.50\% & \cellcolor{green!92.50}92.50\% & \cellcolor{green!40.96}40.96\% & \cellcolor{green!15.38}15.38\% & 0.00\% & \cellcolor{green!92.50}92.50\% & \cellcolor{green!30.00}30.00\% & \cellcolor{green!67.50}67.50\% & \cellcolor{green!33.98}33.98\% & \cellcolor{green!11.11}11.11\% \\
\cellcolor{black!10}\# 5 & \cellcolor{black!10}\url{md4c_md_html} & \cellcolor{black!10}2 & 0.00\% & 0.00\% & 0.00\% & 0.00\% & \cellcolor{green!48.78}48.78\% & \cellcolor{green!18.03}18.03\% & 0.00\% & 0.00\% & 0.00\% & 0.00\% & \cellcolor{green!32.38}32.38\% & \cellcolor{green!10}2.33\% \\
\# 6 & \url{spdk_spdk_json_parse} & 2 & \cellcolor{green!10.00}10.00\% & \cellcolor{green!87.50}87.50\% & \cellcolor{green!15.00}15.00\% & - & \cellcolor{green!46.48}46.48\% & \cellcolor{green!20.43}20.43\% & 0.00\% & \cellcolor{green!75.00}75.00\% & \cellcolor{green!15.00}15.00\% & - & \cellcolor{green!44.16}44.16\% & \cellcolor{green!10.43}10.43\% \\
\cellcolor{black!10}\# 7 & \cellcolor{black!10}\url{croaring_roaring_bitmap_portable_deserialize_safe} & \cellcolor{black!10}3 & \cellcolor{green!20.00}20.00\% & \cellcolor{green!100.00}100.00\% & \cellcolor{green!72.50}72.50\% & \cellcolor{green!70.00}70.00\% & \cellcolor{green!32.29}32.29\% & \cellcolor{green!24.75}24.75\% & \cellcolor{green!27.50}27.50\% & \cellcolor{green!55.00}55.00\% & \cellcolor{green!25.00}25.00\% & \cellcolor{green!75.00}75.00\% & \cellcolor{green!27.45}27.45\% & \cellcolor{green!17.60}17.60\% \\
\# 8 & \url{lua_luaL_loadbufferx} & 3 & \cellcolor{green!67.50}67.50\% & \cellcolor{green!97.50}97.50\% & \cellcolor{green!60.00}60.00\% & \cellcolor{green!100.00}100.00\% & \cellcolor{green!72.00}72.00\% & \cellcolor{green!41.33}41.33\% & \cellcolor{green!77.50}77.50\% & \cellcolor{green!82.50}82.50\% & \cellcolor{green!42.50}42.50\% & \cellcolor{green!85.00}85.00\% & \cellcolor{green!54.72}54.72\% & \cellcolor{green!13.51}13.51\% \\
\cellcolor{black!10}\# 9 & \cellcolor{black!10}\url{w3m_wc_Str_conv_with_detect} & \cellcolor{black!10}3 & 0.00\% & 0.00\% & \cellcolor{green!25.00}25.00\% & - & \cellcolor{green!10}0.57\% & \cellcolor{green!10}9.55\% & 0.00\% & 0.00\% & \cellcolor{green!25.00}25.00\% & - & \cellcolor{green!10}0.27\% & \cellcolor{green!10}4.35\% \\
\# 10 & \url{bind9_dns_name_fromwire} & 4 & 0.00\% & \cellcolor{green!20.00}20.00\% & \cellcolor{green!10}7.50\% & - & \cellcolor{green!10}2.17\% & \cellcolor{green!10}5.58\% & 0.00\% & 0.00\% & \cellcolor{green!10}2.50\% & - & 0.00\% & \cellcolor{green!10}0.61\% \\
\cellcolor{black!10}\# 11 & \cellcolor{black!10}\url{gdk-pixbuf_gdk_pixbuf_animation_new_from_file} & \cellcolor{black!10}4 & \cellcolor{green!15.00}15.00\% & \cellcolor{green!82.50}82.50\% & \cellcolor{green!37.50}37.50\% & \cellcolor{green!67.50}67.50\% & \cellcolor{green!32.20}32.20\% & \cellcolor{green!10}9.09\% & \cellcolor{green!10}7.50\% & \cellcolor{green!27.50}27.50\% & \cellcolor{green!10}7.50\% & \cellcolor{green!22.50}22.50\% & \cellcolor{green!10}9.46\% & \cellcolor{green!10}5.06\% \\
\# 12 & \url{gdk-pixbuf_gdk_pixbuf_new_from_data} & 4 & \cellcolor{green!10}5.00\% & \cellcolor{green!40.00}40.00\% & \cellcolor{green!27.50}27.50\% & \cellcolor{green!27.50}27.50\% & \cellcolor{green!24.75}24.75\% & \cellcolor{green!14.10}14.10\% & \cellcolor{green!45.00}45.00\% & \cellcolor{green!62.50}62.50\% & \cellcolor{green!15.00}15.00\% & \cellcolor{green!57.50}57.50\% & \cellcolor{green!25.00}25.00\% & \cellcolor{green!12.62}12.62\% \\
\cellcolor{black!10}\# 13 & \cellcolor{black!10}\url{gdk-pixbuf_gdk_pixbuf_new_from_file} & \cellcolor{black!10}4 & \cellcolor{green!25.00}25.00\% & \cellcolor{green!97.50}97.50\% & \cellcolor{green!50.00}50.00\% & \cellcolor{green!92.50}92.50\% & \cellcolor{green!30.88}30.88\% & \cellcolor{green!18.03}18.03\% & \cellcolor{green!12.50}12.50\% & \cellcolor{green!35.00}35.00\% & \cellcolor{green!12.50}12.50\% & \cellcolor{green!20.00}20.00\% & \cellcolor{green!10}4.71\% & \cellcolor{green!10}2.56\% \\
\# 14 & \url{gdk-pixbuf_gdk_pixbuf_new_from_stream} & 4 & \cellcolor{green!12.50}12.50\% & \cellcolor{green!75.00}75.00\% & \cellcolor{green!60.00}60.00\% & \cellcolor{green!65.00}65.00\% & \cellcolor{green!60.00}60.00\% & \cellcolor{green!37.88}37.88\% & \cellcolor{green!52.50}52.50\% & \cellcolor{green!87.50}87.50\% & \cellcolor{green!45.00}45.00\% & \cellcolor{green!80.00}80.00\% & \cellcolor{green!36.71}36.71\% & \cellcolor{green!18.68}18.68\% \\
\cellcolor{black!10}\# 15 & \cellcolor{black!10}\url{gpac_gf_isom_open_file} & \cellcolor{black!10}4 & \cellcolor{green!10}2.50\% & \cellcolor{green!67.50}67.50\% & \cellcolor{green!50.00}50.00\% & - & \cellcolor{green!10}2.91\% & \cellcolor{green!10}4.59\% & 0.00\% & \cellcolor{green!12.50}12.50\% & 0.00\% & - & 0.00\% & 0.00\% \\
\# 16 & \url{libbpf_bpf_object__open_mem} & 4 & \cellcolor{green!10}2.50\% & \cellcolor{green!15.00}15.00\% & \cellcolor{green!10.00}10.00\% & \cellcolor{green!15.00}15.00\% & \cellcolor{green!10.50}10.50\% & \cellcolor{green!10}7.56\% & 0.00\% & \cellcolor{green!27.50}27.50\% & \cellcolor{green!15.00}15.00\% & \cellcolor{green!12.50}12.50\% & \cellcolor{green!11.11}11.11\% & \cellcolor{green!10}8.62\% \\
\cellcolor{black!10}\# 17 & \cellcolor{black!10}\url{libpg_query_pg_query_parse} & \cellcolor{black!10}4 & \cellcolor{green!10.00}10.00\% & \cellcolor{green!90.00}90.00\% & \cellcolor{green!95.00}95.00\% & - & \cellcolor{green!34.09}34.09\% & \cellcolor{green!34.62}34.62\% & \cellcolor{green!15.00}15.00\% & \cellcolor{green!42.50}42.50\% & \cellcolor{green!65.00}65.00\% & - & \cellcolor{green!21.95}21.95\% & \cellcolor{green!14.89}14.89\% \\
\# 18 & \url{libucl_ucl_parser_add_string} & 4 & \cellcolor{green!20.00}20.00\% & \cellcolor{green!47.50}47.50\% & \cellcolor{green!50.00}50.00\% & \cellcolor{green!72.50}72.50\% & \cellcolor{green!22.61}22.61\% & \cellcolor{green!16.95}16.95\% & \cellcolor{green!17.50}17.50\% & \cellcolor{green!20.00}20.00\% & \cellcolor{green!12.50}12.50\% & \cellcolor{green!45.00}45.00\% & \cellcolor{green!16.99}16.99\% & \cellcolor{green!15.91}15.91\% \\
\cellcolor{black!10}\# 19 & \cellcolor{black!10}\url{oniguruma_onig_new} & \cellcolor{black!10}4 & \cellcolor{green!50.00}50.00\% & \cellcolor{green!87.50}87.50\% & \cellcolor{green!55.00}55.00\% & \cellcolor{green!82.50}82.50\% & \cellcolor{green!21.66}21.66\% & \cellcolor{green!26.47}26.47\% & \cellcolor{green!30.00}30.00\% & \cellcolor{green!45.00}45.00\% & \cellcolor{green!12.50}12.50\% & \cellcolor{green!37.50}37.50\% & \cellcolor{green!22.88}22.88\% & \cellcolor{green!12.96}12.96\% \\
\# 20 & \url{pupnp_ixmlLoadDocumentEx} & 4 & 0.00\% & \cellcolor{green!65.00}65.00\% & \cellcolor{green!75.00}75.00\% & \cellcolor{green!40.00}40.00\% & \cellcolor{green!11.43}11.43\% & \cellcolor{green!10}8.59\% & 0.00\% & \cellcolor{green!17.50}17.50\% & \cellcolor{green!20.00}20.00\% & \cellcolor{green!10}2.50\% & \cellcolor{green!10}3.70\% & \cellcolor{green!10}1.87\% \\
\cellcolor{black!10}\# 21 & \cellcolor{black!10}\url{gdk-pixbuf_gdk_pixbuf_new_from_file_at_scale} & \cellcolor{black!10}5 & \cellcolor{green!45.00}45.00\% & \cellcolor{green!72.50}72.50\% & \cellcolor{green!40.00}40.00\% & \cellcolor{green!70.00}70.00\% & \cellcolor{green!12.77}12.77\% & \cellcolor{green!10}9.20\% & \cellcolor{green!10}5.00\% & \cellcolor{green!10}2.50\% & \cellcolor{green!10}7.50\% & \cellcolor{green!17.50}17.50\% & 0.00\% & \cellcolor{green!10}1.02\% \\
\# 22 & \url{inchi_GetINCHIKeyFromINCHI} & 5 & 0.00\% & \cellcolor{green!75.00}75.00\% & \cellcolor{green!22.50}22.50\% & \cellcolor{green!77.50}77.50\% & \cellcolor{green!35.11}35.11\% & \cellcolor{green!26.04}26.04\% & 0.00\% & \cellcolor{green!40.00}40.00\% & \cellcolor{green!25.00}25.00\% & \cellcolor{green!60.00}60.00\% & \cellcolor{green!20.95}20.95\% & \cellcolor{green!10.96}10.96\% \\
\cellcolor{black!10}\# 23 & \cellcolor{black!10}\url{libdwarf_dwarf_init_b} & \cellcolor{black!10}5 & 0.00\% & 0.00\% & \cellcolor{green!32.50}32.50\% & \cellcolor{green!10}2.50\% & \cellcolor{green!10}5.49\% & \cellcolor{green!16.07}16.07\% & 0.00\% & 0.00\% & \cellcolor{green!27.50}27.50\% & \cellcolor{green!12.50}12.50\% & \cellcolor{green!10}1.66\% & \cellcolor{green!10}9.89\% \\
\# 24 & \url{libdwarf_dwarf_init_path} & 5 & 0.00\% & 0.00\% & \cellcolor{green!20.00}20.00\% & 0.00\% & 0.00\% & \cellcolor{green!10}1.64\% & 0.00\% & 0.00\% & \cellcolor{green!10.00}10.00\% & 0.00\% & 0.00\% & \cellcolor{green!10}1.12\% \\
\cellcolor{black!10}\# 25 & \cellcolor{black!10}\url{liblouis_lou_compileString} & \cellcolor{black!10}5 & 0.00\% & \cellcolor{green!22.50}22.50\% & \cellcolor{green!50.00}50.00\% & \cellcolor{green!20.00}20.00\% & \cellcolor{green!10}9.89\% & \cellcolor{green!11.68}11.68\% & 0.00\% & \cellcolor{green!17.50}17.50\% & \cellcolor{green!12.50}12.50\% & \cellcolor{green!30.00}30.00\% & \cellcolor{green!10}2.15\% & \cellcolor{green!10}2.33\% \\
\# 26 & \url{selinux_cil_compile} & 5 & 0.00\% & 0.00\% & \cellcolor{green!62.50}62.50\% & - & \cellcolor{green!10}1.13\% & \cellcolor{green!26.04}26.04\% & 0.00\% & 0.00\% & \cellcolor{green!27.50}27.50\% & - & \cellcolor{green!10}0.34\% & \cellcolor{green!10}9.89\% \\
\cellcolor{black!10}\# 27 & \cellcolor{black!10}\url{bind9_dns_name_fromtext} & \cellcolor{black!10}6 & 0.00\% & \cellcolor{green!55.00}55.00\% & \cellcolor{green!15.00}15.00\% & - & \cellcolor{green!10}9.25\% & \cellcolor{green!10}8.70\% & 0.00\% & 0.00\% & \cellcolor{green!10}7.50\% & - & 0.00\% & \cellcolor{green!10}1.78\% \\
\# 28 & \url{bind9_dns_rdata_fromwire} & 6 & 0.00\% & 0.00\% & \cellcolor{green!10}2.50\% & - & \cellcolor{green!10}0.60\% & \cellcolor{green!10}3.12\% & 0.00\% & 0.00\% & \cellcolor{green!10}2.50\% & - & 0.00\% & 0.00\% \\
\cellcolor{black!10}\# 29 & \cellcolor{black!10}\url{coturn_stun_is_binding_response} & \cellcolor{black!10}6 & 0.00\% & \cellcolor{green!60.00}60.00\% & \cellcolor{green!30.00}30.00\% & - & \cellcolor{green!16.42}16.42\% & \cellcolor{green!13.68}13.68\% & 0.00\% & 0.00\% & \cellcolor{green!35.00}35.00\% & - & \cellcolor{green!10}2.65\% & \cellcolor{green!10}7.23\% \\
\# 30 & \url{coturn_stun_is_command_message} & 6 & 0.00\% & \cellcolor{green!30.00}30.00\% & \cellcolor{green!37.50}37.50\% & \cellcolor{green!27.50}27.50\% & \cellcolor{green!10.06}10.06\% & \cellcolor{green!12.70}12.70\% & 0.00\% & 0.00\% & \cellcolor{green!42.50}42.50\% & 0.00\% & \cellcolor{green!10}0.52\% & \cellcolor{green!12.79}12.79\% \\
\cellcolor{black!10}\# 31 & \cellcolor{black!10}\url{coturn_stun_is_response} & \cellcolor{black!10}6 & 0.00\% & \cellcolor{green!35.00}35.00\% & 0.00\% & - & \cellcolor{green!10}8.50\% & \cellcolor{green!10}6.96\% & 0.00\% & 0.00\% & \cellcolor{green!25.00}25.00\% & - & 0.00\% & \cellcolor{green!10}1.60\% \\
\# 32 & \url{coturn_stun_is_success_response} & 6 & 0.00\% & \cellcolor{green!40.00}40.00\% & \cellcolor{green!37.50}37.50\% & - & \cellcolor{green!12.59}12.59\% & \cellcolor{green!15.73}15.73\% & 0.00\% & 0.00\% & \cellcolor{green!15.00}15.00\% & - & \cellcolor{green!10}2.22\% & \cellcolor{green!10}6.90\% \\
\cellcolor{black!10}\# 33 & \cellcolor{black!10}\url{hiredis_redisFormatCommand} & \cellcolor{black!10}6 & \cellcolor{green!32.50}32.50\% & \cellcolor{green!100.00}100.00\% & \cellcolor{green!22.50}22.50\% & - & \cellcolor{green!10.55}10.55\% & \cellcolor{green!13.73}13.73\% & \cellcolor{green!10}5.00\% & \cellcolor{green!77.50}77.50\% & \cellcolor{green!20.00}20.00\% & - & \cellcolor{green!10}4.18\% & \cellcolor{green!10}4.32\% \\
\# 34 & \url{igraph_igraph_read_graph_dl} & 6 & \cellcolor{green!22.50}22.50\% & 0.00\% & \cellcolor{green!10.00}10.00\% & 0.00\% & \cellcolor{green!10}3.38\% & \cellcolor{green!10}6.36\% & 0.00\% & 0.00\% & 0.00\% & 0.00\% & \cellcolor{green!10}1.02\% & \cellcolor{green!10}0.76\% \\
\cellcolor{black!10}\# 35 & \cellcolor{black!10}\url{igraph_igraph_read_graph_edgelist} & \cellcolor{black!10}6 & \cellcolor{green!15.00}15.00\% & 0.00\% & \cellcolor{green!10}5.00\% & \cellcolor{green!10}2.50\% & \cellcolor{green!10}3.56\% & \cellcolor{green!10}1.97\% & 0.00\% & 0.00\% & 0.00\% & 0.00\% & \cellcolor{green!10}0.68\% & \cellcolor{green!10}0.83\% \\
\# 36 & \url{igraph_igraph_read_graph_gml} & 6 & \cellcolor{green!10}7.50\% & \cellcolor{green!10}5.00\% & \cellcolor{green!10}2.50\% & 0.00\% & \cellcolor{green!10}4.93\% & \cellcolor{green!10}5.00\% & 0.00\% & 0.00\% & 0.00\% & 0.00\% & \cellcolor{green!10}1.25\% & \cellcolor{green!10}0.53\% \\
\cellcolor{black!10}\# 37 & \cellcolor{black!10}\url{igraph_igraph_read_graph_graphdb} & \cellcolor{black!10}6 & \cellcolor{green!10}7.50\% & 0.00\% & \cellcolor{green!15.00}15.00\% & 0.00\% & \cellcolor{green!10}3.32\% & \cellcolor{green!10}7.87\% & 0.00\% & 0.00\% & \cellcolor{green!10}2.50\% & 0.00\% & \cellcolor{green!10}1.17\% & \cellcolor{green!10}2.45\% \\
\# 38 & \url{igraph_igraph_read_graph_graphml} & 6 & \cellcolor{green!10}7.50\% & \cellcolor{green!10}2.50\% & \cellcolor{green!32.50}32.50\% & \cellcolor{green!10}2.50\% & \cellcolor{green!10}4.20\% & \cellcolor{green!15.75}15.75\% & 0.00\% & 0.00\% & \cellcolor{green!10}5.00\% & 0.00\% & \cellcolor{green!10}4.12\% & \cellcolor{green!10}5.30\% \\
\cellcolor{black!10}\# 39 & \cellcolor{black!10}\url{igraph_igraph_read_graph_lgl} & \cellcolor{black!10}6 & \cellcolor{green!10}2.50\% & 0.00\% & \cellcolor{green!32.50}32.50\% & 0.00\% & \cellcolor{green!10}1.89\% & \cellcolor{green!10}9.27\% & 0.00\% & 0.00\% & 0.00\% & 0.00\% & \cellcolor{green!10}1.78\% & \cellcolor{green!10}1.15\% \\
\# 40 & \url{igraph_igraph_read_graph_pajek} & 6 & \cellcolor{green!15.00}15.00\% & \cellcolor{green!10}2.50\% & \cellcolor{green!10}5.00\% & 0.00\% & \cellcolor{green!10}6.46\% & \cellcolor{green!10}9.88\% & 0.00\% & 0.00\% & 0.00\% & 0.00\% & \cellcolor{green!10}1.16\% & \cellcolor{green!10}0.88\% \\
\cellcolor{black!10}\# 41 & \cellcolor{black!10}\url{inchi_GetINCHIfromINCHI} & \cellcolor{black!10}6 & \cellcolor{green!20.00}20.00\% & \cellcolor{green!67.50}67.50\% & \cellcolor{green!47.50}47.50\% & \cellcolor{green!80.00}80.00\% & \cellcolor{green!10}1.48\% & 0.00\% & 0.00\% & \cellcolor{green!10}2.50\% & \cellcolor{green!20.00}20.00\% & \cellcolor{green!12.50}12.50\% & 0.00\% & \cellcolor{green!10}0.78\% \\
\# 42 & \url{inchi_GetStructFromINCHI} & 6 & 0.00\% & \cellcolor{green!52.50}52.50\% & \cellcolor{green!22.50}22.50\% & \cellcolor{green!22.50}22.50\% & \cellcolor{green!10}1.37\% & \cellcolor{green!10}2.27\% & 0.00\% & \cellcolor{green!10}2.50\% & \cellcolor{green!10.00}10.00\% & \cellcolor{green!10}5.00\% & \cellcolor{green!10}0.32\% & \cellcolor{green!10}0.26\% \\
\cellcolor{black!10}\# 43 & \cellcolor{black!10}\url{kamailio_parse_msg} & \cellcolor{black!10}6 & 0.00\% & \cellcolor{green!57.50}57.50\% & \cellcolor{green!22.50}22.50\% & - & \cellcolor{green!16.36}16.36\% & \cellcolor{green!15.08}15.08\% & 0.00\% & \cellcolor{green!32.50}32.50\% & \cellcolor{green!30.00}30.00\% & - & \cellcolor{green!10}5.92\% & \cellcolor{green!10}6.01\% \\
\# 44 & \url{libyang_lys_parse_mem} & 6 & \cellcolor{green!10}2.50\% & \cellcolor{green!10}5.00\% & \cellcolor{green!20.00}20.00\% & \cellcolor{green!10.00}10.00\% & \cellcolor{green!10}6.62\% & \cellcolor{green!13.14}13.14\% & 0.00\% & 0.00\% & \cellcolor{green!10}2.50\% & 0.00\% & \cellcolor{green!10}3.05\% & \cellcolor{green!10}5.24\% \\
\cellcolor{black!10}\# 45 & \cellcolor{black!10}\url{proftpd_pr_json_object_from_text} & \cellcolor{black!10}6 & 0.00\% & 0.00\% & \cellcolor{green!72.50}72.50\% & - & \cellcolor{green!10}2.64\% & \cellcolor{green!10}7.45\% & 0.00\% & 0.00\% & \cellcolor{green!12.50}12.50\% & - & \cellcolor{green!10}0.32\% & \cellcolor{green!10}1.97\% \\
\# 46 & \url{selinux_policydb_read} & 6 & 0.00\% & \cellcolor{green!30.00}30.00\% & \cellcolor{green!42.50}42.50\% & - & \cellcolor{green!10.95}10.95\% & \cellcolor{green!13.01}13.01\% & 0.00\% & \cellcolor{green!12.50}12.50\% & \cellcolor{green!10}5.00\% & - & \cellcolor{green!10}3.68\% & \cellcolor{green!10}5.98\% \\
\cellcolor{black!10}\# 47 & \cellcolor{black!10}\url{kamailio_get_src_address_socket} & \cellcolor{black!10}7 & 0.00\% & \cellcolor{green!10}5.00\% & \cellcolor{green!27.50}27.50\% & 0.00\% & \cellcolor{green!10}2.50\% & \cellcolor{green!10}3.53\% & 0.00\% & 0.00\% & \cellcolor{green!35.00}35.00\% & 0.00\% & 0.00\% & \cellcolor{green!10}4.15\% \\
\# 48 & \url{kamailio_get_src_uri} & 7 & 0.00\% & 0.00\% & \cellcolor{green!22.50}22.50\% & \cellcolor{green!10}2.50\% & \cellcolor{green!10}1.85\% & \cellcolor{green!10}5.13\% & 0.00\% & \cellcolor{green!10}2.50\% & \cellcolor{green!20.00}20.00\% & 0.00\% & \cellcolor{green!10}0.69\% & \cellcolor{green!10}1.06\% \\
\cellcolor{black!10}\# 49 & \cellcolor{black!10}\url{kamailio_parse_content_disposition} & \cellcolor{black!10}7 & 0.00\% & 0.00\% & \cellcolor{green!10.00}10.00\% & 0.00\% & 0.00\% & \cellcolor{green!10}1.65\% & 0.00\% & 0.00\% & 0.00\% & 0.00\% & 0.00\% & \cellcolor{green!10}1.25\% \\
\# 50 & \url{kamailio_parse_diversion_header} & 7 & 0.00\% & 0.00\% & \cellcolor{green!27.50}27.50\% & 0.00\% & 0.00\% & \cellcolor{green!10}6.35\% & 0.00\% & 0.00\% & \cellcolor{green!10}7.50\% & 0.00\% & 0.00\% & \cellcolor{green!10}0.56\% \\
\cellcolor{black!10}\# 51 & \cellcolor{black!10}\url{kamailio_parse_from_header} & \cellcolor{black!10}7 & 0.00\% & 0.00\% & 0.00\% & - & 0.00\% & 0.00\% & 0.00\% & 0.00\% & 0.00\% & - & 0.00\% & \cellcolor{green!10}0.58\% \\
\# 52 & \url{kamailio_parse_from_uri} & 7 & 0.00\% & 0.00\% & \cellcolor{green!10}2.50\% & - & \cellcolor{green!10}0.34\% & \cellcolor{green!10}0.45\% & 0.00\% & 0.00\% & 0.00\% & - & 0.00\% & 0.00\% \\
\cellcolor{black!10}\# 53 & \cellcolor{black!10}\url{kamailio_parse_headers} & \cellcolor{black!10}7 & 0.00\% & 0.00\% & 0.00\% & - & 0.00\% & \cellcolor{green!10}0.83\% & 0.00\% & 0.00\% & 0.00\% & - & 0.00\% & 0.00\% \\
\# 54 & \url{kamailio_parse_identityinfo_header} & 7 & 0.00\% & 0.00\% & \cellcolor{green!50.00}50.00\% & - & 0.00\% & \cellcolor{green!10.22}10.22\% & 0.00\% & 0.00\% & \cellcolor{green!17.50}17.50\% & - & 0.00\% & \cellcolor{green!10}2.37\% \\
\cellcolor{black!10}\# 55 & \cellcolor{black!10}\url{kamailio_parse_pai_header} & \cellcolor{black!10}7 & 0.00\% & 0.00\% & \cellcolor{green!10.00}10.00\% & - & 0.00\% & \cellcolor{green!10}1.92\% & 0.00\% & 0.00\% & \cellcolor{green!10}5.00\% & - & 0.00\% & \cellcolor{green!10}0.52\% \\
\# 56 & \url{kamailio_parse_privacy} & 7 & 0.00\% & 0.00\% & \cellcolor{green!22.50}22.50\% & 0.00\% & 0.00\% & \cellcolor{green!10}5.41\% & 0.00\% & 0.00\% & \cellcolor{green!10}5.00\% & 0.00\% & 0.00\% & 0.00\% \\
\cellcolor{black!10}\# 57 & \cellcolor{black!10}\url{kamailio_parse_record_route_headers} & \cellcolor{black!10}7 & 0.00\% & 0.00\% & \cellcolor{green!100.00}100.00\% & - & \cellcolor{green!10}0.45\% & \cellcolor{green!21.13}21.13\% & 0.00\% & 0.00\% & \cellcolor{green!12.50}12.50\% & - & 0.00\% & \cellcolor{green!10}4.92\% \\
\# 58 & \url{kamailio_parse_refer_to_header} & 7 & 0.00\% & 0.00\% & \cellcolor{green!17.50}17.50\% & - & 0.00\% & \cellcolor{green!10}3.82\% & 0.00\% & 0.00\% & \cellcolor{green!10.00}10.00\% & - & 0.00\% & \cellcolor{green!10}1.01\% \\
\cellcolor{black!10}\# 59 & \cellcolor{black!10}\url{kamailio_parse_route_headers} & \cellcolor{black!10}7 & 0.00\% & 0.00\% & \cellcolor{green!87.50}87.50\% & - & 0.00\% & \cellcolor{green!11.85}11.85\% & 0.00\% & \cellcolor{green!10}2.50\% & \cellcolor{green!87.50}87.50\% & - & 0.00\% & \cellcolor{green!10}3.73\% \\
\# 60 & \url{kamailio_parse_to_header} & 7 & 0.00\% & 0.00\% & \cellcolor{green!12.50}12.50\% & - & 0.00\% & \cellcolor{green!10}3.88\% & 0.00\% & 0.00\% & \cellcolor{green!12.50}12.50\% & - & 0.00\% & 0.00\% \\
\cellcolor{black!10}\# 61 & \cellcolor{black!10}\url{kamailio_parse_to_uri} & \cellcolor{black!10}7 & 0.00\% & 0.00\% & \cellcolor{green!10}7.50\% & - & \cellcolor{green!10}0.47\% & 0.00\% & 0.00\% & 0.00\% & 0.00\% & - & 0.00\% & 0.00\% \\
\# 62 & \url{libyang_lyd_parse_data_mem} & 7 & 0.00\% & \cellcolor{green!22.50}22.50\% & \cellcolor{green!45.00}45.00\% & \cellcolor{green!25.00}25.00\% & \cellcolor{green!20.25}20.25\% & \cellcolor{green!29.47}29.47\% & 0.00\% & 0.00\% & 0.00\% & 0.00\% & \cellcolor{green!10}4.86\% & \cellcolor{green!10}3.19\% \\
\cellcolor{black!10}\# 63 & \cellcolor{black!10}\url{bind9_dns_message_parse} & \cellcolor{black!10}8 & 0.00\% & 0.00\% & 0.00\% & - & \cellcolor{green!10}1.22\% & \cellcolor{green!10}1.89\% & 0.00\% & 0.00\% & \cellcolor{green!10}2.50\% & - & 0.00\% & \cellcolor{green!10}0.55\% \\
\# 64 & \url{igraph_igraph_read_graph_ncol} & 8 & \cellcolor{green!10}5.00\% & 0.00\% & \cellcolor{green!10}2.50\% & 0.00\% & \cellcolor{green!10}0.22\% & \cellcolor{green!10}2.56\% & 0.00\% & 0.00\% & 0.00\% & 0.00\% & \cellcolor{green!10}0.88\% & \cellcolor{green!10}0.43\% \\
\cellcolor{black!10}\# 65 & \cellcolor{black!10}\url{pjsip_pj_json_parse} & \cellcolor{black!10}8 & 0.00\% & \cellcolor{green!10}7.50\% & \cellcolor{green!10.00}10.00\% & \cellcolor{green!10}5.00\% & \cellcolor{green!10}6.81\% & \cellcolor{green!10}9.35\% & 0.00\% & 0.00\% & 0.00\% & \cellcolor{green!10}7.50\% & \cellcolor{green!10}0.90\% & \cellcolor{green!10}2.33\% \\
\# 66 & \url{pjsip_pj_xml_parse} & 8 & 0.00\% & \cellcolor{green!25.00}25.00\% & \cellcolor{green!10.00}10.00\% & \cellcolor{green!25.00}25.00\% & \cellcolor{green!10}6.98\% & \cellcolor{green!10}5.43\% & 0.00\% & 0.00\% & 0.00\% & \cellcolor{green!17.50}17.50\% & \cellcolor{green!10}2.23\% & \cellcolor{green!10}0.55\% \\
\cellcolor{black!10}\# 67 & \cellcolor{black!10}\url{pjsip_pjmedia_sdp_parse} & \cellcolor{black!10}8 & \cellcolor{green!10}5.00\% & \cellcolor{green!22.50}22.50\% & \cellcolor{green!10}2.50\% & \cellcolor{green!35.00}35.00\% & \cellcolor{green!10}5.97\% & \cellcolor{green!10}8.05\% & 0.00\% & \cellcolor{green!10}7.50\% & \cellcolor{green!10}2.50\% & \cellcolor{green!10.00}10.00\% & \cellcolor{green!10}2.28\% & \cellcolor{green!10}0.79\% \\
\# 68 & \url{quickjs_lre_compile} & 8 & 0.00\% & 0.00\% & 0.00\% & - & 0.00\% & \cellcolor{green!10}3.45\% & 0.00\% & 0.00\% & 0.00\% & - & 0.00\% & 0.00\% \\
\cellcolor{black!10}\# 69 & \cellcolor{black!10}\url{bind9_isc_lex_getmastertoken} & \cellcolor{black!10}9 & 0.00\% & 0.00\% & 0.00\% & - & \cellcolor{green!10}1.83\% & \cellcolor{green!10}1.86\% & 0.00\% & 0.00\% & 0.00\% & - & 0.00\% & 0.00\% \\
\# 70 & \url{bind9_isc_lex_gettoken} & 9 & 0.00\% & 0.00\% & 0.00\% & - & \cellcolor{green!10}1.79\% & \cellcolor{green!10}3.82\% & 0.00\% & 0.00\% & 0.00\% & - & 0.00\% & \cellcolor{green!10}0.68\% \\
\cellcolor{black!10}\# 71 & \cellcolor{black!10}\url{quickjs_JS_Eval} & \cellcolor{black!10}9 & \cellcolor{green!17.50}17.50\% & \cellcolor{green!95.00}95.00\% & \cellcolor{green!25.00}25.00\% & - & \cellcolor{green!27.78}27.78\% & \cellcolor{green!10}7.07\% & \cellcolor{green!17.50}17.50\% & \cellcolor{green!32.50}32.50\% & \cellcolor{green!10}7.50\% & - & \cellcolor{green!17.90}17.90\% & \cellcolor{green!12.96}12.96\% \\
\# 72 & \url{igraph_igraph_edge_connectivity} & 10 & 0.00\% & 0.00\% & 0.00\% & 0.00\% & 0.00\% & 0.00\% & 0.00\% & 0.00\% & 0.00\% & 0.00\% & 0.00\% & 0.00\% \\
\cellcolor{black!10}\# 73 & \cellcolor{black!10}\url{pjsip_pj_stun_msg_decode} & \cellcolor{black!10}10 & 0.00\% & 0.00\% & 0.00\% & 0.00\% & \cellcolor{green!10}5.17\% & \cellcolor{green!10}8.25\% & 0.00\% & 0.00\% & 0.00\% & 0.00\% & \cellcolor{green!10}0.21\% & \cellcolor{green!10}1.98\% \\
\# 74 & \url{bind9_dns_message_checksig} & 11 & 0.00\% & 0.00\% & \cellcolor{green!10}2.50\% & - & 0.00\% & \cellcolor{green!10}0.56\% & 0.00\% & 0.00\% & 0.00\% & - & 0.00\% & 0.00\% \\
\cellcolor{black!10}\# 75 & \cellcolor{black!10}\url{libzip_zip_fread} & \cellcolor{black!10}11 & \cellcolor{green!57.50}57.50\% & \cellcolor{green!45.00}45.00\% & \cellcolor{green!55.00}55.00\% & \cellcolor{green!32.50}32.50\% & \cellcolor{green!10.09}10.09\% & \cellcolor{green!12.50}12.50\% & \cellcolor{green!10}2.50\% & \cellcolor{green!10}5.00\% & \cellcolor{green!10.00}10.00\% & \cellcolor{green!10}7.50\% & \cellcolor{green!10}1.21\% & \cellcolor{green!10}5.00\% \\
\# 76 & \url{bind9_dns_rdata_fromtext} & 12 & 0.00\% & 0.00\% & 0.00\% & - & 0.00\% & \cellcolor{green!10}3.45\% & 0.00\% & 0.00\% & 0.00\% & - & 0.00\% & 0.00\% \\
\cellcolor{black!10}\# 77 & \cellcolor{black!10}\url{igraph_igraph_all_minimal_st_separators} & \cellcolor{black!10}12 & 0.00\% & \cellcolor{green!12.50}12.50\% & \cellcolor{green!17.50}17.50\% & \cellcolor{green!10}2.50\% & \cellcolor{green!12.58}12.58\% & \cellcolor{green!10.27}10.27\% & 0.00\% & 0.00\% & 0.00\% & \cellcolor{green!10}2.50\% & \cellcolor{green!10}2.38\% & \cellcolor{green!10}1.39\% \\
\# 78 & \url{igraph_igraph_minimum_size_separators} & 12 & \cellcolor{green!10}5.00\% & \cellcolor{green!10.00}10.00\% & \cellcolor{green!37.50}37.50\% & \cellcolor{green!10.00}10.00\% & \cellcolor{green!10}4.15\% & \cellcolor{green!10}5.39\% & 0.00\% & 0.00\% & 0.00\% & 0.00\% & \cellcolor{green!10}1.40\% & \cellcolor{green!10}1.31\% \\
\cellcolor{black!10}\# 79 & \cellcolor{black!10}\url{pjsip_pjsip_parse_msg} & \cellcolor{black!10}12 & 0.00\% & 0.00\% & \cellcolor{green!10}2.50\% & 0.00\% & 0.00\% & \cellcolor{green!10}2.29\% & 0.00\% & 0.00\% & 0.00\% & 0.00\% & 0.00\% & \cellcolor{green!10}0.39\% \\
\# 80 & \url{igraph_igraph_automorphism_group} & 13 & 0.00\% & 0.00\% & \cellcolor{green!62.50}62.50\% & 0.00\% & 0.00\% & \cellcolor{green!11.32}11.32\% & 0.00\% & 0.00\% & 0.00\% & 0.00\% & 0.00\% & \cellcolor{green!10}0.60\% \\
\cellcolor{black!10}\# 81 & \cellcolor{black!10}\url{libmodbus_modbus_read_bits} & \cellcolor{black!10}15 & 0.00\% & 0.00\% & 0.00\% & 0.00\% & 0.00\% & 0.00\% & 0.00\% & 0.00\% & 0.00\% & 0.00\% & 0.00\% & 0.00\% \\
\# 82 & \url{libmodbus_modbus_read_registers} & 15 & 0.00\% & 0.00\% & 0.00\% & 0.00\% & 0.00\% & 0.00\% & 0.00\% & 0.00\% & 0.00\% & 0.00\% & 0.00\% & 0.00\% \\
\cellcolor{black!10}\# 83 & \cellcolor{black!10}\url{civetweb_mg_get_response} & \cellcolor{black!10}17 & 0.00\% & 0.00\% & 0.00\% & 0.00\% & 0.00\% & 0.00\% & 0.00\% & 0.00\% & 0.00\% & 0.00\% & 0.00\% & 0.00\% \\
\# 84 & \url{bind9_dns_master_loadbuffer} & 20 & 0.00\% & 0.00\% & 0.00\% & - & \cellcolor{green!10}0.25\% & \cellcolor{green!10}0.73\% & 0.00\% & 0.00\% & 0.00\% & - & 0.00\% & 0.00\% \\
\cellcolor{black!10}\# 85 & \cellcolor{black!10}\url{libmodbus_modbus_receive} & \cellcolor{black!10}33 & 0.00\% & 0.00\% & 0.00\% & 0.00\% & 0.00\% & 0.00\% & 0.00\% & 0.00\% & 0.00\% & 0.00\% & 0.00\% & 0.00\% \\
\# 86 & \url{tmux_input_parse_buffer} & 42 & 0.00\% & 0.00\% & 0.00\% & - & 0.00\% & 0.00\% & 0.00\% & 0.00\% & 0.00\% & - & 0.00\% & 0.00\% \\

\bottomrule

\end{tabular}

}
\end{table*}

\end{document}
\endinput
%%
%% End of file `sample-sigconf.tex'.
