\vspace{-2pt}
\section{Conclusion}
\revision{
Our study centers around fundamental issues of LLM-based fuzz driver generation's effectiveness.
% , including the effectiveness, the basic challenges, and the pros and cons of several designed query strategies.
To do that, we designed six prompt strategies, extensively evaluated them on different models and temperatures.
% built a framework for evaluation in scale, and compared generated drivers with industrial used ones.
% to study in scale.
The insights are three-fold:
\ding{182} LLM-based generation has promising potential but also faces challenges towards high practicality;  
% the results demonstrated the promising practicality of LLM-based generation as well as the challenges towards high practicality;
\ding{183} the fundamental challenge roots in tackling the API-specific usage particulars while three key beneficial designs for prompting are identified and analyzed;
\ding{184} the LLM-generated drivers can provide comparable fuzzing outcomes as the industrial used ones.
However, larges spaces for further improvements still exist.
Our insights have been applied in industrial practical fuzz driver generation.
}
\compactline
% there are still large space for further improvements, such as automatic semantic correctness validation, API usage extension, and semantic oracle generation.

% and there are three key designs can help significantly: repeatedly querying, querying with examples, and iteratively querying;
% Besides, three key designs significantly help here: 
% revealing their potential for facilitating fuzzing of API targets.
% However, practical application of LLM-generated drivers requires careful filtering to ensure effectiveness and avoid false alarms.
% We have also analyzed the effects, advantages, and disadvantages of five different types of strategies.
% Comparing the generated drivers with manually written ones, we found that LLM-generated drivers can produce competent fuzzing outcomes, but there is still room for further improvement.

\vspace{-2pt}
\section{Data Availability}

% To facilitate the future research, we have released all the code and data involved in our study~\cite{fuzz-drvier-study-website}.
The source code and data involved in our study can be found at~\cite{fuzz-drvier-study-website}.
% to facilitate the community.
% To facilitate the community, we will open source our quiz, evaluation framework, and the data involved in the study at~\cite{fuzz-drvier-study-website}.

% \clearpage
