\section{Related Work}

\noindent
\textbf{Fuzz Driver Generation} \tab 
Recently, several works~\cite{fuzzgen, fudge, apicraft, winnie, intelligen, rubick, utopia} have focused on developing automatic approaches to generate fuzz drivers.
Most of these works follow a common methodology, which involves generating fuzz drivers based on the API usage existed in consumer programs, \textit{i.e.,} programs containing code that uses these APIs.
For instance, by abstracting the API usage as specific models such as trees~\cite{apicraft}, graphs~\cite{fuzzgen}, and automatons~\cite{rubick}, several works propose program analysis-based methods to learn the usage models from consumer programs and conduct model-based driver synthesis.
In addition, a recent work~\cite{utopia} emphasizes that unit tests are high quality consumer programs and proposes techniques to convert existing unit tests to fuzz drivers.
Though these approaches can produce effective fuzz drivers, their heavy requirements on the quality of the consumer programs, \textit{i.e.,} the consumers must contain complete API usage and are statically/dynamically analyzable, limit their generality.
Furthermore, code synthesized by many works lacks human readability and maintainability, which may also limit their practical application.

% utopia, rubick, apicraft, ...

\noindent
\textbf{LLM for Generative Tasks} \tab 
Recent works have explored the potential of LLM models for various generative tasks, such as test case generation~\cite{guifill, zhanglingming-llm-are-zero-shot-fuzzers, deng2023large, schafer2023adaptive} and code repair generation~\cite{sp-repair, abhik-repair, xia2023keep}. These works utilize the natural language processing capabilities of LLM models and employ specific prompt designs to achieve their respective tasks. To further improve the models' performance, some works incorporate iterative/conversational strategies and use fine-tuning/in-context learning techniques.

In the field of test case generation, previous research works have primarily targeted deep learning libraries~\cite{zhanglingming-llm-are-zero-shot-fuzzers, deng2023large}, JavaScript~\cite{schafer2023adaptive} and GUI applications~\cite{guifill}. 
However, there is a significant lack of research on the effectiveness of LLM models for fuzz test generation in C projects. 
Given the importance of generating fuzz driver for  C project, this represents a critical area for further investigation.


% unit test generation
% evaluate copilot
% code repair
% ...
