    
\documentclass[prl, aps, twocolumn, amsmath, amssymb, graphicx,superscriptaddress]{revtex4-2}

\usepackage{bm}
\usepackage{graphicx}
\usepackage{braket}
\usepackage[normalem]{ulem}

\usepackage{color}
\usepackage[usenames, dvipsnames]{xcolor}
\usepackage[colorlinks=true]{hyperref}
\hypersetup{linktoc=page, colorlinks=true, linkcolor=MidnightBlue, citecolor=MidnightBlue, urlcolor=OliveGreen}

\usepackage[]{cleveref}
\Crefname{section}{Sec.}{Secs.}
\Crefname{equation}{Eq.}{Eqs.}
\Crefname{figure}{Fig.}{Figs.}
\Crefname{tabular}{Tab.}{Tabs.}

\usepackage{siunitx}
\usepackage{soul}
\usepackage{url}


\usepackage{tabularx}
\usepackage{array}
\setlength{\extrarowheight}{0.5ex}

\newcolumntype{I}{>{\hsize=0.36\hsize}X}
\newcolumntype{C}{>{\hsize=0.3\hsize\centering\arraybackslash}X}
\newcolumntype{U}{>{\hsize=0.35\hsize\centering\arraybackslash}X}
\newcolumntype{V}{>{\hsize=0.4\hsize\centering\arraybackslash}X}
\newcolumntype{W}{>{\hsize=0.44\hsize\centering\arraybackslash}X}
\newcolumntype{Y}{>{\hsize=0.49\hsize\centering\arraybackslash}X}

\usepackage{multirow}
\usepackage{makecell}

\usepackage{comment}

\usepackage[dvipsnames]{xcolor}

\newcommand{\muB}{\mu_\mathrm{\scriptscriptstyle B}}
\newcommand{\kB}{k_\mathrm{\scriptscriptstyle B}}
\newcommand{\kF}{k_\mathrm{\scriptscriptstyle F}}
\newcommand{\EF}{E_\mathrm{\scriptscriptstyle F}}

\renewcommand{\L}{\mathrm{\scriptscriptstyle L}}
\newcommand{\R}{\mathrm{\scriptscriptstyle R}}
\newcommand{\LL}{\mathrm{\scriptscriptstyle LL}}
\newcommand{\RR}{\mathrm{\scriptscriptstyle RR}}
\newcommand{\LR}{\mathrm{\scriptscriptstyle LR}}
\newcommand{\RL}{\mathrm{\scriptscriptstyle RL}}
\newcommand{\GLL}{G_\LL}
\newcommand{\GRR}{G_\RR}
\newcommand{\GLR}{G_\LR}
\newcommand{\GRL}{G_\RL}
\newcommand{\Gag}{G_\mathrm{\scriptscriptstyle N}}
\newcommand{\Vp}{V_\mathrm{p}}
\newcommand{\Vb}{V_\mathrm{b}}
\newcommand{\Vac}{V_\mathrm{\scriptscriptstyle AC}}
\newcommand{\Vg}{V_\mathrm{g}}
\newcommand{\Vlc}{V_\mathrm{lc}}
\newcommand{\Vrc}{V_\mathrm{rc}}
\newcommand{\Vdep}{V_\mathrm{dep}}
\newcommand{\DeltaT}{\Delta_\mathrm{\scriptscriptstyle T}}
\newcommand{\xiT}{\xi_\mathrm{\scriptscriptstyle T}}
\newcommand{\ellLoc}{\ell_\mathrm{loc}}

\newcommand{\SM}{\mathrm{\scriptscriptstyle SM}}
\newcommand{\SC}{\mathrm{\scriptscriptstyle SC}}
\newcommand{\DeltaMax}{\Delta_\mathrm{topo}^\mathrm{max}}
\newcommand{\meanDeltaMax}{{\bar\Delta}_\mathrm{topo}^\mathrm{max}}
\newcommand{\DeltaInd}{\Delta_\mathrm{ind}}
\newcommand{\DeltaAl}{\Delta_\mathrm{Al}}
\newcommand{\xiAl}{\xi_\mathrm{Al}}

\newcommand{\NL}{\mathrm{\scriptscriptstyle NL}}

\renewcommand{\sc}{n_{\mathrm{{\scriptscriptstyle 2D}, int}}}
\newcommand{\scu}{10^{12}/\text{cm}^2}

\newcommand{\volSOI}{\mathcal{V}_\mathrm{\scriptscriptstyle SOI2}}
\newcommand{\BavSOI}{B_\mathrm{\scriptscriptstyle SOI2}}
\newcommand{\volSOIbar}{\bar{\mathcal{V}}_\mathrm{\scriptscriptstyle SOI2}}
\newcommand{\BavSOIbar}{\bar{B}_\mathrm{\scriptscriptstyle SOI2}}

\newcommand{\Iinput}{dI^0}
\newcommand{\Vinput}{dV^0}
\newcommand{\Isample}{dI}
\newcommand{\Vsample}{dV}

\newcommand{\CM}{\textbf{C}}
\newcommand{\VM}{\textbf{V}}
\newcommand{\QM}{\textbf{Q}}
\newcommand{\ICM}{\textbf{C}^{-1}}
\renewcommand{\bra}[1]{\langle #1|}
\renewcommand{\ket}[1]{|#1 \rangle}

\newcommand{\LU}{\ket {\rm{L}\uparrow}}
\newcommand{\LD}{\ket {\rm{L}\downarrow}}
\newcommand{\RU}{\ket {\rm{R}\uparrow}}
\newcommand{\RD}{\ket {\rm{R}\downarrow}}

\newcommand{\bLU}{\bra {\rm{L\uparrow}}}
\newcommand{\bLD}{\bra {\rm{L\downarrow}}}
\newcommand{\bRU}{\bra {\rm{R\uparrow}}}
\newcommand{\bRD}{\bra {\rm{R\downarrow}}}

\newcommand{\cre}{\hat a^\dagger}
\newcommand{\ann}{\hat a}
\newcommand{\avg}[1]{\langle #1 \rangle}
\newcommand{\tso}{t_{\rm so}}
\newcommand{\Tr}{{\rm Tr}}
\newcommand{\IP}{{\rm I}}

\newcommand{\dt}[1]{\frac{\partial{#1}}{\partial t}}
\newcommand{\dz}[1]{\frac{\partial{#1}}{\partial z}}
\newcommand{\dzz}[1]{\frac{\partial^2{#1}}{\partial z^2}}

\newcommand{\Vv}{\textbf{V}}
\newcommand{\Iv}{\textbf{I}}

\newcommand{\Mm}{\textbf{M}}

\renewcommand{\thesection}{\arabic{section}}
\renewcommand\thesubsection{\thesection.\arabic{subsection}}
\renewcommand\thesubsubsection{\thesubsection.\arabic{subsubsection}}

\makeatletter
\renewcommand*{\p@subsection}{}
\renewcommand*{\p@subsubsection}{}
\makeatother


\begin{document}

\title{Comment on Hess et al. Phys. Rev. Lett. {\bf 130}, 207001 (2023)}

\author{A. Antipov}
\affiliation{Microsoft Quantum, Station Q, Santa Barbara, CA 93111}
\author{W. Cole}
\affiliation{Microsoft Quantum, Redmond, WA 98052}
\author{K. Kalashnikov}
\affiliation{Microsoft Quantum, Redmond, WA 98052}
\author{F. Karimi}
\affiliation{Microsoft Quantum, Station Q, Santa Barbara, CA 93111}
\author{R. Lutchyn}
\affiliation{Microsoft Quantum, Station Q, Santa Barbara, CA 93111}
\author{C. Nayak}
\affiliation{Microsoft Quantum, Station Q, Santa Barbara, CA 93111}
\author{D. Pikulin}
\affiliation{Microsoft Quantum, Redmond, WA 98052}
\author{G. Winkler}
\affiliation{Microsoft Quantum, Station Q, Santa Barbara, CA 93111}




\date{\today{}}

\begin{abstract}

\end{abstract}


\maketitle

In this comment, we show that the model introduced in Ref.~\onlinecite{Hess23} fails the topological gap protocol (TGP)~\cite{Pikulin21,Aghaee23}. In addition, we discuss this model in the broader context of how the TGP
has been benchmarked.
%This refutes the claim \cite{Hess23} that ``The simultaneous occurrence of a trivial bulk reopening signature and zero-bias peaks mimics the basic features required to pass the so-called `topological gap protocol.' Our results therefore provide a topologically trivial minimal model by which the applicability of the protocol can be benchmarked.''

The TGP is designed to identify topological regions in the phase diagram of a device using a measurement and analysis protocol, as detailed in Refs.~\onlinecite{Pikulin21,Aghaee23}.
Note that the TGP does not directly
calculate the topological invariant
(which is not accessible in
transport data) and
serves as a practical experimental proxy for it.
Its accuracy was tested in simulations
in which we can separately compute the topological
invariant and compare it with the output
of the TGP.
The TGP will have a low false discovery rate (FDR) ``provided that
the simulated data [used to test the TGP] is drawn from the same probability distribution as the data produced by real devices.''\cite{Aghaee23}
It is possible to design a Hamiltonian that will give a false positive TGP result; its contribution
to the false discovery rate will be weighted by the probability that it occurs in a realistic disorder model.
We characterized disorder in our devices (see Fig. 6 in Ref.~\onlinecite{Aghaee23}) and established that our disorder model is realistic and consistent with the measurements. In other words, we checked our ability to sample in simulations from the same disorder distribution as is present in our physical devices. 
The model in Ref.~\onlinecite{Hess23} is based on a specific theoretical configuration of periodic Andreev states that has vanishing probability
of occurring in realistic devices.
Therefore, it could not contribute to the estimated FDR.

Furthermore, the model of Ref.~\onlinecite{Hess23} fails the TGP. We have made the TGP available \cite{code_and_data} to enable researchers to analyze their experimental data and theoretical models with it. Since this analysis was not carried out in Ref.~\onlinecite{Hess23}, we report the outcome here: it does not pass the TGP. The model's ZBPs and gap closings are insufficient to satisfy all of the TGP's requirements.
Since the model relies on fine-tuning to produce ZBPs on either side (as opposed to a quasi-Majorana scenario that can be more stable~\cite{Prada12, Kells12, Tewari14, LiuC17, Vuik19, Pan21b}) the ZBPs are not stable to variations in the junction region, i.e. changing $\mu_L$ or $\mu_R$ by a little will split the ZBPs. The TGP demands stability to changes of gate voltages that affect the junction region, so the ZBPs in this model would be identified as trivial ZBPs.
If we ignore this for the moment and consider only
one junction configuration, we find the TGP phase diagram shown in \Cref{fig:Hess-phase-diagram-clean}. For the chemical potential value $\mu = 2$ meV  that is plotted in Ref.~\onlinecite{Hess23} as a putative example of a false positive, the ZBP at the right junction is split, so the TGP identifies this region as trivial. 
There is a more promising region at higher chemical potential with ZBPs at both sides, but less than $50\%$ of its boundary is gapless, so it doesn’t pass the TGP there either. If we now vary the junction voltages, as the TGP requires, the ZBPs
disappear and the orange region in \Cref{fig:Hess-phase-diagram-clean}(a) becomes blue. A different family of Hamiltonians could be designed to produce false positive TGP results. However, as we emphasized above, in order to shed light on the protocol's FDR, we must also determine the probability that it occurs for a realistic disorder distribution.
At the 95\% confidence level,
the results of Ref.~\onlinecite{Aghaee23} indicate that this probability will be less than 8\%.


% Figure environment removed



\bibliographystyle{apsrev}
\bibliography{comment-refs}

\end{document}
