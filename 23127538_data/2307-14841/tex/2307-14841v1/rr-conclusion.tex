In this registered report we presented our proposal for a study of the suitability of HFH for empirical studies.
The study comprises a feature-based framework comparison to characterize the HFH functionality together with an analysis of the mechanisms to retrieve information on how such features are used. 
This allows evaluating the suitability of HFH from a feature availability perspective.
Besides this feature-level study we propose to conduct a second one, more quantitative one, based on the study of volume and diversity of the data stored on the HFH.
We conduct this study both at the platform and project-levels, looking at the overall volume and richness of the data and on how the average project uses the platform.

Beyond a deeper understanding on how collaborative development of ML-related projects takes place on the HFH, the conclusion of this report is expected to be a discussion on whether HFH can be a suitable data source to perform empirical studies.
Also, given that empirical studies usually focus on a specific characteristic of code hosting platforms, as we mentioned in Section~\ref{sec:variablesQualitative}, 
beyond a boolean answer, the goal is to discuss what types of empirical studies could benefit from HFH data, either as a standalone data source or in combination with GitHub or other data sources.