The {\color{\mycolor}goal} of our registered report is to assess {\color{\mycolor}the current state} of HFH and analyze its adequacy to be used in empirical studies.
In particular, we plan to address the following research questions: 

\vspace{0.25em} 
\noindent\emph{\textbf{RQ1. What features does HFH provide as a code hosting platform {\color{\mycolor}to enable empirical studies}?  }}
We plan to comprehend the key features that characterize HFH both for individual projects (i.e. features oriented towards end-users planning to use HFH for their software development projects) and at the platform level (i.e. to facilitate the retrieval and analysis of global HFH usage information). 
This analysis will allow characterizing the platform and identifying potential use cases for empirical studies.
Thus, we subdivide RQ1 further into:

\vspace{0.25em}
\textit{RQ1.1 What features HFH offers to facilitate the collaborative development of ML-oriented projects?}
This research question performs an exploratory study of the features offered by HFH to projects hosted in the platform. 
In this RQ, we focus on the features serving project development tasks, such as pull requests for managing code contributions or issue trackers for notifying bugs or requests.
To this aim, we plan to study current code hosting platforms to define a feature framework to be used as a reference framework to analyze the platform. 

\vspace{0.25em}
\textit{RQ1.2 What features HFH offers at the platform level to facilitate access to the hosted projects' data?}
In this research question we examine the features provided by HFH aimed at retrieving its internal data, derived from the activity of projects hosted in it.
Indeed, note that these features are not necessarily aimed at developing software projects in the platform (as it is the case of the features studied in RQ1.1) but at enabling the data collection from them.
We plan to include features to cover the platform infrastructure offered by HFH to access the data, such as APIs, and whether there are other solutions built by the community, such as datasets.
Furthermore, we are interested on identifying whether such infrastructure enables to collect information from each of the HFH features identified in RQ1.1. 
We believe the availability and easy access to the data in a code hosting platform is a relevant factor for researchers when selecting platforms for their empirical studies.

\vspace{0.5em}
\noindent\emph{\textbf{RQ2. How is HFH currently being exploited?}}
We are interested in studying how HFH is so far being used at platform and project levels.
In each level, we will analyze the data within two perspectives: volume and diversity.
To measure the volume we will define quantitative variables, such as the number of repositories and users at platform level; or the number of files, contributors and commits at project level.
On the other hand, to measure diversity we will define categorical variables, such as the programming languages used in the repositories or the type of contributions (i.e., issues or discussions) in the projects.
Note that while RQ1 focuses on the features provided by the platform, RQ2 analyzes its current usage, thus allowing to better understand the platform dynamics.
We subdivide RQ2 further into:

\vspace{0.25em}
\textit{RQ2.1 What is the current state of the platform data in HFH?}
In this research question we explore 
{\color{\mycolor} how HFH is used as a whole}.
Some examples of variables to be used in this research question are the number of repositories and the level of dependency between them as an example of volume and diversity. 


\vspace{0.25em}
\textit{RQ2.2 What is the current state of the project data in HFH?}
In this research question we explore {\color{\mycolor} the usage of HFH at project level}. 
Thus, instead of the platform, the repository becomes the unit of study. 
The goal is to characterize the average (or averages if we detect different typologies) project on HFH via the analysis of their number of files and commits, number of users, its temporal evolution, etc.\looseness-1

{\color{\mycolor} To identify the goal, research questions and metrics we followed an approach similar to GQM~\cite{DBLP:books/daglib/0029933} methodology.
Section~\ref{sec:variables} addresses the metrics designed to answer each research question.}