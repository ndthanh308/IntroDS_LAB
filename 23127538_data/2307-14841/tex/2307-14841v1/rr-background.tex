Hugging Face, the company behind HFH, is an AI company originally known for its Natural Language Processing (NLP) model called Hierarchical Multi-Task Learning (HMTL)~\cite{DBLP:conf/aaai/SanhWR19} or for the Transformers library~\cite{DBLP:conf/emnlp/WolfDSCDMCRLFDS20}, which provides APIs and tools to easily download and train state-of-the-art pretrained models.
Nevertheless, it became a household name thanks to the creation of HFH, its ML-based hosting platform, with the goal of building the largest open-source collection of ML artifacts to advance and democratize the access to ML for everyone.

HFH is a Git-based {\color{\mycolor}online} code hosting platform aimed at providing a hosting site for all kinds of ML artifacts, namely: 
(1) models, pretrained models that can be used with the Transformers library; 
(2) datasets, which can be used to train ML models; and 
(3) spaces, demo apps to showcase ML models.

The storage for these artifacts relies on Git repositories, where each repository is presented on the HFH website via three tabs, namely: card, files and community.
The repository card is the front face of the repository, and it is different for each repository type.
For model repositories, the card display the content of the \texttt{README} file, the downloads by month, the repository dependencies (datasets used for training, and spaces displaying the model) and there is an Inference API\footnote{https://huggingface.co/docs/api-inference/index} interface to test and evaluate the model.
For dataset repositories, it shows the repository dependencies, the \texttt{README} file and its downloads, along with a preview of the data.
The card for space repositories is the most different and changes from one space to another, as it is designed to provide a demo of an ML model.
Next to the repository card, the file tab displays the repository files and their commit history while the community tab hosts the discussions and pull requests threads arisen during the development of the repository.

Since its creation, HFH has been rapidly evolving and incorporating new features.
For instance, the last tab regarding discussions was just released on May 2022.
{\color{\mycolor} 
To illustrate the growing evolution of the platform, Figures~\ref{fig:numProjectsComparison:numprojectsHF} and~\ref{fig:numProjectsComparison:cumulativeHF} illustrate the natural and cumulative growth of new project registrations by month in HFH, respectively.
To study the growth of the platform, we relied on the Diffusion of Innovation (DoI) theory~\cite{DBLP:books/daglib/0012785}, which helps to explain how a product gains or loses momentum in a system.
In Figure~\ref{fig:numProjectsComparison:cumulativeHF} the point indicates the month with the maximum growth.
As can be seen, the point is located in the last month of registered activity, which indicates that no momentum lost is detected, and therefore the platform is still growing.
For the sake of comparison, Figures~\ref{fig:numProjectsComparison:numprojectsGH} and~\ref{fig:numProjectsComparison:cumulativeGH} show the same growth for \gh, which was reported by Squire~\cite{DBLP:conf/wikis/Squire17}.
Note that HFH follows a growth similar to \textsc{GitHub}.
In the fifth year, the number of new projects registered in the platform was roughly the same.
Even though HFH is showing such a growing behavior, to the best of our knowledge, the number of research papers targeting empirical studies based on the platform is still very scarce.
}

% Figure environment removed
