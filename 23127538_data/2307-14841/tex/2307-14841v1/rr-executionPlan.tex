We devise an execution plan to obtain the evidences required to answer our research questions.
The plan has the following steps that will be executed sequentially:

\begin{enumerate}
    \item Analyze features from current code hosting platforms. 
    \item Construct a feature framework from the features identified in the previous step.
    \item Characterize HFH with the feature framework.
    \item Analyze the options offered by HFH to retrieve the data.
    \item Conduct the data extraction process.
    \item Perform the data analysis of the HFH data.
\end{enumerate}

The execution plan detail is as follows.
\textcolor{\mycolor}{The first step is to analyze the features from current code hosting platforms.}
\textcolor{\mycolor}{Based on this analysis, we will categorize the identified features and use them as the basis for the comparison dimensions of a feature framework.}
\textcolor{\mycolor}{We will then instantiate HFH using this framework to compare HFH offering with those of other platforms}. These first three steps will address RQ1.1.

The next step is to analyze the exposure mechanisms offered by HFH to access the hosted dta.
The intention is to determine whether the data originated from projects using HFH features can be queried and, if so, {\color{\mycolor}the easiness of data retrieval (cf. Section~\ref{sec:analysisPlan}).} 
This will answer RQ1.2. 

Once addressed RQ1, the next two steps will target RQ2. 
The way to target RQ2 will depend on the results of the analysis of the RQ1.2. 
If the situation has not recently changed, it is likely that the best starting point for RQ2 is \hfc~\cite{hfc}, a curated dataset that periodically publishes up-to-date HFH data. 
Regardless of the source, we will complete the study with the data extraction and subsequent computation of several relevant metrics to understand the volume and diversity of HFH data, according to the perspectives mentioned in Section~\ref{sec:rqs}.
