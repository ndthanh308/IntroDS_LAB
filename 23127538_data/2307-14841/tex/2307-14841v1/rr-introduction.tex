The development of empirical studies in Open-Source Software (OSS) requires large amounts of data regarding software development events and developer actions, which are typically collected from code hosting platforms.
Code hosting platforms are built on top of a version control system, such as Git, and provide collaboration tools such as issue trackers, discussions, and wikis; as well as social features such as the possibility to watch, follow and like other users and projects.
Among them, \gh has emerged as the largest code hosting site in the world, with more than 80 million users and 200 million repositories.

The emergence of Machine Learning (ML) has led to the development of platforms specifically designed for developing ML-based projects, being \hfh (HFH) one of the most popular ones.
HFH is a place where developers can publish and share their ML-based projects, as well as reuse datasets, pre-trained models and other ML artifacts.
As of April 2023, the platform hosts more than 250k repositories, and this number is growing fast.

In the last months, HFH has been evolving and incorporating features which are typically found in \gh, such the ability to create discussions or submit pull requests enabling more complex interactions and development workflows.
This evolution, its growing popularity and the ML-specific features make HFH a promising source of data for empirical studies.
Although the usage of HFH in empirical studies is promising, the current status of the platform may involve relevant perils.

In this paper, we propose a registered report to study the current state of HFH and its suitability to be used as a source for empirical studies.
We understand as suitability the amount and adequacy of the features to enable software development practices and the sufficient quantity of data to enable the conduction of empirical studies about such practices.
To this aim, we propose an execution plan where we analyze the set of features provided by HFH and then study the availability and quality of the data available in HFH.
Later, in the analysis plan, we discuss the results to evaluate their impact in different scenarios commonly found in empirical studies

The rest of the paper is structured as follows.
{\color{\mycolor}Sections~\ref{sec:background} and~\ref{sec:relatedwork} provide the background and related work, respectively.}
Section~\ref{sec:rqs} presents the research questions.
Section~\ref{sec:executionPlan} describes the execution plan, while Section~\ref{sec:variables} presents the variables identified in our report.
Sections~\ref{sec:analysisPlan} and~\ref{sec:threats} describe the analysis plan and the threats to validity, respectively.
Finally, Section~\ref{sec:conclusion} concludes the paper. 
