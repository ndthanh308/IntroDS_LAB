To address our research question we plan to conduct a qualitative and quantitative analysis of HFH.
The former will address RQ1, as it will focus on identifying the features of HFH and the options available to retrieve HFH data. 
The latter will address RQ2, and will allow us to analyze the data available in HFH via the reported data retrieval solution.
In the following we present the variables involved in each analysis.


\subsection{Qualitative analysis}
\label{sec:variablesQualitative}
During the qualitative analysis we will build a feature framework aimed at identifying which characteristics define a code hosting platform.
Features will include both characteristics offered to develop software projects and options provided to retrieve data from the platform.
{\color{\mycolor}The framework will be built by analyzing different code hosting platforms and identifying the features offered by each platform.}
{\color{\mycolor}We plan to revise a number of platforms, leveraging on existing literature (e.g.,~\cite{DBLP:journals/jsw/AlamerA17,DBLP:conf/wikis/Squire17}), but in the context of his paper we focus on \gh and \gitlab due to their relevance and wide user base.}
This will result in a superset of features \textcolor{\mycolor}{which will be backed by the literature from relevant venues} to underline the importance of some specific features in empirical studies.
{\color{\mycolor} Furthermore, we will validate the resulting set of features with the platform communities by conducting semi-structured interviews with relevant actors of each analyzed platform.} 
Note that the feature framework will help characterize HFH and, potentially, any code hosting platform by the features they offer.


Table~\ref{tab:featureFramework} shows a preliminar version of the feature framework, which may evolve during the realization of the registered report.
Features are identified as qualitative variables, and grouped according to topics.
The first five topics, namely: coding, social, user management, project management and project add-ons will address RQ1.1; while the last topic (i.e., platform) will address R1.2.
In the following, we describe each identified topic and motivate them in the context of empirical studies. 

\vspace{0.25em}
\noindent \textbf{Coding.} 
This topic includes variables addressing typical developers' needs to perform coding tasks, namely: usage of a version control software, and support for forks and pull requests, among others.
This would be the topic most related to the development process of contributors and has allowed the execution of empirical analysis of forking (e.g., \cite{DBLP:conf/icse/ZhouVK20,DBLP:conf/icis/NegoitaVSL19}), pull requests (e.g., \cite{DBLP:conf/icse/Subramanian20,DBLP:journals/ese/MezouarZZ19,DBLP:conf/apsec/LiRLZJ18}) or branching (e.g., \cite{DBLP:conf/qrs/ZouZ0H019})

\vspace{0.25em}
\noindent \textbf{Social.} 
This topic includes variables identifying user interaction and communication during development such as the creation of issues or Q\&A threads, and the ability to follow and like projects.
Addressing this kind of features has enabled studies on how users participate in discussions (e.g., \cite{DBLP:journals/ese/HataNBKT22}) or how issues are labelled during the development process (e.g., \cite{DBLP:journals/access/KimL21e,DBLP:conf/apsec/LiRLZJ18}). 

\vspace{0.25em}
\noindent \textbf{User management.} 
This topic is related to the ability of creating and managing groups of users with the purpose of sharing projects between multiple users.
Inside groups, a hierarchy structure can appear, thus defining roles inside the groups of users or inside a specific repository.
It can also enhance studies on how projects organize themselves~\cite{DBLP:conf/icse/VasilescuFS15,DBLP:conf/iisa/ChatziasimidisS15}.
Closely related to OSS development, these features may also help to conduct studies of OSS development roles inside repositories~\cite{DBLP:journals/ese/IzquierdoC22}. 

\vspace{0.25em}
\noindent \textbf{Project management.} 
The projects, or repositories, are the root element in social code hosting platforms.
This topic includes the study of support for project assets such as documentation or management tools such as milestones, labelling, or stream analytics.
Furthermore, we could also consider the chance of having different repository types, and relationships between them.
This kind of features have been used in empirical studies addressing survivability of projects~\cite{DBLP:conf/msr/AitIC22}, the use of milestones~\cite{DBLP:journals/smr/ZhangWWHW20} or checking security concerns~\cite{DBLP:conf/scored/BenedettiVM22}.

\vspace{0.25em}
\noindent \textbf{Project Add-ons.} 
Social code hosting platforms usually allow projects to integrate with apps, available via a marketplace; and communicate with external services via webhooks (e.g., \gh Actions).
This topic covers these features, which have enabled empirical studies on \gh \textsc{Actions}~\cite{DBLP:conf/icsm/DecanMMG22} or marketplace~\cite{DBLP:conf/sast/SouzaC0GB21}.

\vspace{0.25em}
\noindent \textbf{Platform.} 
This topic covers those auxiliary and technical tools to enable the collection of data for empirical studies. 
Thus, they aim at facilitating the use of the platform, including the existence of a platform API, an integrated CLI to access the platform or the indexation of the platform content, such a search mechanism.
Furthermore, we consider the existence of external datasets gathering data from the platform.

\begin{table*}[t]
\centering
\begin{threeparttable}
\caption{Qualitative analysis variables used in RQ1.}
\label{tab:featureFramework}
% \fontsize{7.4pt}{7.4pt}\selectfont
\begin{tabularx}{\textwidth}{lllX}
\textsc{RQ}             & \textsc{Topic}                      & \textsc{Feature}           & \textsc{Description}                                                               \\ % & \textsc{Evidences} \\
\toprule
\multirow{28}{*}{RQ1.1} & \multirow{11}{*}{Coding}            & CVS                        & Control version system (e.g., Git, Mercurial, Subversion, etc.)                    \\ % & \\
                        &                                     & Forking                    & Creation of a copy of other projects                                               \\ % & \cite{DBLP:conf/icse/ZhouVK20,DBLP:conf/icis/NegoitaVSL19}\\       
                        &                                     & Pull Request               & Submission of contributions to other projects                                      \\ % & \cite{DBLP:conf/icse/Subramanian20,DBLP:journals/ese/MezouarZZ19,DBLP:conf/apsec/LiRLZJ18}\\
                        &                                     & Code Review                & Discussion about changes in project files                                          \\ % & \cite{DBLP:journals/infsof/YuWYW16}\\ % Discuss the difference w/ PRs
                        &                                     & Release                    & Identification and management of project releases                                  \\ % & \cite{DBLP:conf/iwpc/WuHXGZ22}\\
                        &                                     & Packages                   & Management and release of \github packages                                         \\ % & \\
                        &                                     & Snippets                   & Upload of fragments of code to share (e.g., \github Gist)                          \\ % & \cite{DBLP:conf/msr/WangPWG15,DBLP:conf/icsm/HortonP18}\\
                        &                                     & Branches                   & Navigation and management of CVS branches                                          \\ % & \cite{DBLP:conf/qrs/ZouZ0H019}\\
                        &                                     & External integrations      & Integration with external services such as Campfire, Jira or Slack                 \\ % & \\ % https://docs.gitlab.com/ee/integration/
                        &                                     & Collaborative/Cloud coding & Online development of project files (e.g., \github Codespaces, HTH resources)      \\ % & \\
                        &                                     & Website publishing         & Support for serving HTML pages (e.g., \github Pages)                               \\ % & \\
\cmidrule{2-4}
                        & \multirow{3}{*}{Social}             & Issues                     & Reporting of bugs and requests                                                     \\ % & \cite{DBLP:journals/access/KimL21e,DBLP:conf/apsec/LiRLZJ18}\\
                        &                                     & Q\&A                       & Discussions                                                                        \\ % & \cite{DBLP:journals/ese/HataNBKT22}\\
                        &                                     & Following                  & Support for stars and following platform users                                     \\ % & \\
\cmidrule{2-4}  
                        & \multirow{2}{*}{User Management}    & Groups                     & Support for defining teams of users in projects                                    \\ % & \\ %Cite: https://link.springer.com/article/10.1186/s41469-017-0020-3
                        &                                     & Roles                      & Roles inside the repository                                                        \\ % & \\
\cmidrule{2-4}  
                        & \multirow{9}{*}{Project Management} & Milestone                  & Similar to Coding / Release                                                        \\ % & \\
                        &                                     & Wiki                       & Wiki-based system for project's documentation                                      \\ % & \\
                        &                                     & Work management            & Agile-like boards to organize tasks (e.g., \github projects; \gitlab To-do lists)  \\ % & \\
                        &                                     & Stream analytics           & Project insights (e.g., \github analytics and repository insights)                 \\ % & \cite{DBLP:conf/msr/AitIC22}\\
                        &                                     & Tagging                    & Project's tag definition and management                                            \\ % & \\
                        &                                     & Repository Type            & Classification of projects according to their purpose                              \\ % & \\
                        &                                     & Project Relations          & Definition of link between projects (e.g., dependencies, etc.)                     \\ % & \\ 
                        &                                     & Development workflows      & Continuous Integration and Development                                             \\ % & \cite{DBLP:conf/uss/KoishybayevNZMR22}\\ %, GH Actions; Alternative name: Automation and CI/CD \\
                        &                                     & Licensing                  & License identification for projects                                                \\ % & \\
                        &                                     & Security                   & Access control to project's assets (e.g., visibility, code control, etc.)          \\ % & \\ % Admittance Control, Execution Control, Code Control, and Access to Secrets (GHA)
\cmidrule{2-4}  
                        & \multirow{2}{*}{Project Add-ons}    & Webhooks                   & Integration with external applications (e.g., \github Actions)                     \\ % & \cite{DBLP:conf/qrs/ChenZCWW21}\\ 
                        &                                     & Marketplace                & Catalogue of external integrations                                                 \\ % & \\
\midrule        
\multirow{4}{*}{RQ1.2}  & \multirow{4}{*}{Platfom}            & Search                     & Search function for platform assets (e.g., repositories, files, users, etc.)       \\ % & \\%HFH has a full text search: it looks even into repo cards \\      
                        &                                     & API                        & Support for accessing the platform programmatically                                \\ % & \\
                        &                                     & Integrated CLI             & Tool to interact with the platform from the command line                           \\ % & \\
                        &                                     & Datasets                   & Existing datasets to query the platform                                           \\ % & \\
                                    % & Platform guide             & Documentation about the platform                                                   \\ % & \\

% \midrule
% This will go to RQ2
% Companion tools   & API               & \\
% Companion tools   & Integrated CLI    & \\
% Companion tools   & Platform guide    & \\
\bottomrule
\end{tabularx}
\end{threeparttable}
\end{table*}



\subsection{Quantitative analysis}
\label{sec:variablesQuantitative}

In the quantitative analysis we will examine the HFH data to provide an overview of the current usage of the platform.
We plan to study the usage of HFH at platform and project level.
The former will show the actual usage of the features identified in the previous research question and conclude on the level of exploitation of such features.
The latter will give an insight of how the development process is currently carried out in HFH, thus favoring the comprehension of why the users use this platform.
To address this analysis we leverage on the data provided by \hfc, an open-source tool that collects data from HFH and Git repositories, and stores it in a relational database to facilitate their analysis.
The database includes more than 250k repositories hosted in HFH.

We will define quantitative variables to analyze the platform and the repositories.
The variables are presented in Table~\ref{tab:variablesRQ2}.
The variables of the category \emph{Platform} will address the RQ2.1, while the variables of the category \emph{Project} will address RQ2.2.
The \emph{Platform} category is designed to characterize the HFH environment.
The HFH platform is specifically designed for ML-based artifacts, thus the platform aspects may differ, such as having repositories designed for each type of ML artifact.
Consequently, there can be variables specific for HFH.
For instance, we can study the nature of a repository type (i.e., a pre-trained model or a dataset) or the dependencies between the repositories, as one dataset can be used by multiple models.
We can measure the amount of repositories of each type or the proportion between datasets and models.

Regarding the \emph{Project} category, we will intend to give an insight on the status of the repositories.
As aforementioned, empirical analysis usually rely on a reduced subset of the repositories in a code hosting platform.
Therefore, we find appropriate to perform an analysis from a repository perspective and identify whether there is a way to select prolific repositories to perform empirical analysis as it is done on other code hosting platforms.
The variables in this category will intend to describe project specific characteristics, namely: 
\textcolor{\mycolor}{the purpose of the users to use this platform, by analyzing their activity and involvement in projects; the development process followed in a repository, via the information of activity and content; or the communication between contributors.
Furthermore, we propose demographic measurements to better describe a project, namely: the age, the project's artifact type, the number and type of dependent repositories, and the popularity, understood as the number of likes and downloads.}


\begin{table*}[t]
    \centering
    % \begin{threeparttable}
    \caption{Quantitative analysis variables used in RQ2.}
    \label{tab:variablesRQ2}
    % \fontsize{7.4pt}{7.4pt}\selectfont
    \begin{tabularx}{\textwidth}{llcX}
    \textsc{Category}                   & \textsc{Variable}                            & \textsc{Type}              & \textsc{Description} \\
    \toprule
    \multirow{4}{*}{Platform}           & Number of repositories                       & Q                          & Amount of projects in the platform \\                        
                                        & Diversity of repositories                    & C                          & \textcolor{\mycolor}{Distribution projects according to their category} in the platform \\
                                        & Number of users                              & Q                          & Amount of users in the platform \\   
                                        & Dependency of repositories                   & C                          & Communities \textcolor{\mycolor}{identified} by the dependency graph \\
    \midrule
    \multirow{8}{*}{Project}            & Activity                                     & Q                          & Activity events \textcolor{\mycolor}{(e.g., commits, files, users)} over time \\                     
                                        & Content                                      & C                          & \textcolor{\mycolor}{Distribution of file composition} of repositories \\
                                        & Involvement                                  & Q                          & Amount of users contributing in the repository \\
                                        & Interactions                                 & Q                          & Communication \textcolor{\mycolor}{(e.g., number of comments)} of users within the repositories \\
                                        & \textcolor{\mycolor}{Age}                    & \textcolor{\mycolor}{Q}    & \textcolor{\mycolor}{Time span of project's life} \\
                                        & \textcolor{\mycolor}{Artifact type}          & \textcolor{\mycolor}{C}    & \textcolor{\mycolor}{The ML task addressed by the proposed artifact (e.g., image classification, text generation, etc.)} \\ %\tnote{a}\adem{footnote for us, maybe remove it in the last version}} \\
                                        & \textcolor{\mycolor}{Dependent Repositories} & \textcolor{\mycolor}{Q/C}  & \textcolor{\mycolor}{Amount and type of dependent repositories of the project} \\
                                        & \textcolor{\mycolor}{Popularity}             & \textcolor{\mycolor}{Q}    & \textcolor{\mycolor}{Amount of likes and downloads} \\
    \bottomrule
    \multicolumn{4}{l}{Q: Quantitative. C: Categorical.}
    \end{tabularx}
    % \begin{tablenotes}
        % \item [a] {\color{\mycolor}https://huggingface.co/tasks}
    % \end{tablenotes}
    % \end{threeparttable}
\end{table*}

