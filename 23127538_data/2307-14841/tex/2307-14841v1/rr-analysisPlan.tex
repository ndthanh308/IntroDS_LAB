Once we execute our study, we plan to analyze the results to explore the suitability of HFH for empirical studies. 
Our intention is to discuss how the current situation of HFH, assessed by the research questions, can enable empirical studies and help researchers to identify potential working lines in the platform.
To this aim, the following procedures will be followed.

\vspace{0.25em}
\noindent\textbf{RQ1.}
The outcome of RQ1 will characterize HFH in terms of features, and will allow us to analyze their importance in the context of empirical studies.
We plan to study the intersection of these features with the ones offered by other code hosting platforms, such as \gh and \gitlab, by analyzing existing literature.
The intersection can bring three scenarios:
(1) shared features, which would enable replicating in HFH empirical studies performed in other platforms;
(2) features not available in HFH, where we will study its impact and significance with regard to existing empirical studies;
and (3) exclusive features of HFH, where we can discuss opportunities where unique HFH features may open new applications for empirical studies.


RQ1.2 targets the retrieval data process from HFH, and during the analysis we plan to compare the means offered by the platform with those available in other code hosting platforms.
Being a relatively young platform, it is expected that HFH does not offer the same amount of prepackaged or curated datasets as other consolidated platforms, but we will study current options to perform this process.
{\color{\mycolor} We define the easiness of data retrieval with different aspects: 
(1) Usability of the mechanisms, which values the amount of existing solutions, such as APIs or datasets; 
(2) the accessibility of such solutions, whether they are open to everyone or if they require some sort of login, for instance; 
(3) the limitations, such as a token-based restriction;
 and (4) the recentness of the solutions, which focus on whether there is a temporal gap between the retrieved information and the platform data at the moment of extraction.}

{\color{\mycolor} The objective of the qualitative analysis for RQ1 is to derive conclusions from the data, keeping a clear chain of evidence~\cite{DBLP:books/daglib/0029933}. 
Our intention is to build and apply the feature framework, which will allow us to apply hypothesis generation techniques.
These techniques are intended to find hypothesis from the data, and the result of these techniques are the hypothesis as such.
Examples of hypotheses generating techniques are ``constant comparisons'' and ``cross-case analysis''~\cite{DBLP:journals/tse/Seaman99}.
The analysis will be conducted with an editing approach formalism, which means codes are defined based on findings of the researcher during the analysis.
}

\vspace{0.25em}
\noindent\textbf{RQ2.}
The results from RQ2 will be used to analyze HFH in terms of the identified categories, namely: platform (i.e., RQ2.1) and projects (i.e., RQ2.2).
The analysis of the former will help us to provide an overview of the behavior and evolution of the platform, while the latter may help us to characterize the typical HFH project.

Regarding the \emph{Platform} category, we expect to gather the absolute data and perform an analysis process to extract descriptive statistic indicators.
The descriptive statistic indicators will reflect the current status of HFH and the adequacy in terms of platform data.
Another aspect we will consider is the ecosystem characterization.
HFH may constitute ecosystems in a particular way, leveraging on its own features, as \gh does with its topics.

Once analyzed the platform, we will also analyze the project as a study unit.
Covered by the variables in the \emph{Project} category, we aim to visualize the exploitation of the repository features by its users.
Thus, the development process may differ from other code hosting platforms as this platform is designed for ML artifacts.
Also, the nature of the repository may show different signs and focus of activity.
For instance, instead of having continuous coding development, the repositories could also be used as a hosting site, being the community features (discussion threads) the source of activity of the repository.

{\color{\mycolor} With regard to the analysis of the quantitative variables for RQ2 we plan to perform the quantitative data interpretation proposed per Wholin et al.~\cite{DBLP:books/daglib/0029933}, which identify three steps, namely:
(1) descriptive statistics, where the data is characterized using descriptive statistics;
(2) dataset reduction, in which abnormal or false data points are excluded;
and (3) hypothesis testing, where the hypotheses of the experiment are evaluated statistically, at a given level of significance.

We plan to leverage on descriptive statistics to get a general view of the HFH data.
For instance, measures of central tendency, such as the mean, mode, and median, may help to understand the behavior of users in the platform as in the involvement of users in a project. 
As the target of this report is an exploratory study, instead of just excluding abnormal data points, we also plan to analyze them to bring some indicators (e.g., the number of empty repositories in the platform). 
After the exclusion of the abnormal data points, we plan to apply the research questions as presented previously. 
}

It is expected to find that HFH would not be suitable for longitudinal studies for long time projects or a massive number of projects, as we believe HFH is not mature enough to be treated as a large code hosting platform.
The conclusions of this research question will aid us to be able to sustent such discussions.

