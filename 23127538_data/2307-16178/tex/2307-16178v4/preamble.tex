\usepackage{epsfig,color}
\usepackage{amsthm}
\usepackage{amsmath}    %For theorems
\usepackage{bm}
\usepackage{epstopdf}
\usepackage{threeparttable}
\usepackage{amssymb}
\usepackage{url}
\usepackage{enumitem}
\usepackage{multirow}
\usepackage{hhline}
\usepackage{booktabs}

% \IEEEoverridecommandlockouts

\usepackage[linesnumbered,boxed,commentsnumbered,ruled,vlined,longend]{algorithm2e}
\SetAlgorithmName{Procedure}
\usepackage{comment}
\setlist[itemize]{leftmargin=*}
\newcommand{\eps}{\varepsilon}
\newcommand{\kron}{\otimes}
\DeclareMathOperator{\diag}{diag}
\DeclareMathOperator{\trace}{trace}
\DeclareMathOperator{\rank}{rank}
\DeclareMathOperator{\tspan}{span}
\DeclareMathOperator*{\minimize}{minimize}
\DeclareMathOperator*{\maximize}{maximize}
\DeclareMathOperator*{\find}{find}
\DeclareMathOperator*{\subjectto}{subject\ to}
\DeclareMathOperator{\vecc}{vec}
\makeatother
\DeclareMathAlphabet\mathbfcal{OMS}{cmsy}{b}{n}
\renewcommand\qedsymbol{$\blacksquare$}
\usepackage{float}
\usepackage[caption = false]{subfig}
%\usepackage{natbib}
%\bibliographystyle{agsm}
\hyphenation{op-tical net-works semi-conduc-tor}
\usepackage{caption}
\newtheorem{theorem}{Theorem}
\newtheorem{mydef}{Definition}
\newtheorem{mylem}{Lemma}
\newtheorem{mypro}{Property}
\newtheorem{myrem}{Remark}
\newtheorem{asmp}{Assumption}
\newtheorem{mycor}{Corollary}
\newtheorem{myprs}{Proposition}
\newtheorem{exmpl}{Example}
% Algorithmic modifications
\renewcommand{\thefootnote}{\arabic{footnote}}

\makeatletter
\newcommand{\algorithmicbreak}{\textbf{break}}
\newcommand{\BREAK}{\STATE \algorithmicbreak}
\makeatother

\usepackage{stackengine}
\newcommand\barbelow[1]{\stackunder[1.2pt]{$#1$}{\rule{.8ex}{.075ex}}}

\newcommand{\mat}[1]{\boldsymbol{#1}}
\renewcommand{\vec}[1]{\boldsymbol{\mathrm{#1}}}
\newcommand{\vecalt}[1]{\boldsymbol{#1}}

\newcommand{\conj}[1]{\overline{#1}}

\newcommand{\normof}[1]{\|#1\|}
\newcommand{\onormof}[2]{\|#1\|_{#2}}

\newcommand{\MIN}[2]{\begin{array}{ll} \displaystyle \minimize_{#1} & {#2} \end{array}}
\newcommand{\MINone}[3]{\begin{array}{ll} \displaystyle \minimize_{#1} & {#2} \\ \subjectto & {#3} \end{array}}
\newcommand{\OPTone}{\MINone}

\newcommand{\itr}[2]{#1^{(#2)}}
\newcommand{\itn}[1]{^{(#1)}}

\newcommand{\prob}{\mathbb{P}}
\newcommand{\probof}[1]{\prob\left\{ #1 \right\}}

\newcommand{\pmat}[1]{\begin{pmatrix} #1 \end{pmatrix}}
\newcommand{\bmat}[1]{\begin{bmatrix} #1 \end{bmatrix}}
\newcommand{\spmat}[1]{\left(\begin{smallmatrix} #1 \end{smallmatrix}\right)}
\newcommand{\sbmat}[1]{\left[\begin{smallmatrix} #1 \end{smallmatrix}\right]}

\newcommand{\RR}{\mathbb{R}}
\newcommand{\CC}{\mathbb{C}}

\providecommand{\eye}{\mat{I}}
\providecommand{\mA}{\ensuremath{\mat{A}}}
\providecommand{\mB}{\ensuremath{\mat{B}}}
\providecommand{\mC}{\ensuremath{\mat{C}}}
\providecommand{\mD}{\ensuremath{\mat{D}}}
\providecommand{\mE}{\ensuremath{\mat{E}}}
\providecommand{\mF}{\ensuremath{\mat{F}}}
\providecommand{\mG}{\ensuremath{\mat{G}}}
\providecommand{\mH}{\ensuremath{\mat{H}}}
\providecommand{\mI}{\ensuremath{\mat{I}}}
\providecommand{\mJ}{\ensuremath{\mat{J}}}
\providecommand{\mK}{\ensuremath{\mat{K}}}
\providecommand{\mL}{\ensuremath{\mat{L}}}
\providecommand{\mM}{\ensuremath{\mat{M}}}
\providecommand{\mN}{\ensuremath{\mat{N}}}
\providecommand{\mO}{\ensuremath{\mat{O}}}
\providecommand{\mP}{\ensuremath{\mat{P}}}
\providecommand{\mQ}{\ensuremath{\mat{Q}}}
\providecommand{\mR}{\ensuremath{\mat{R}}}
\providecommand{\mS}{\ensuremath{\mat{S}}}
\providecommand{\mT}{\ensuremath{\mat{T}}}
\providecommand{\mU}{\ensuremath{\mat{U}}}
\providecommand{\mV}{\ensuremath{\mat{V}}}
\providecommand{\mW}{\ensuremath{\mat{W}}}
\providecommand{\mX}{\ensuremath{\mat{X}}}
\providecommand{\mY}{\ensuremath{\mat{Y}}}
\providecommand{\mZ}{\ensuremath{\mat{Z}}}
\providecommand{\mLambda}{\ensuremath{\mat{\Lambda}}}
\providecommand{\mPbar}{\bar{\mP}}

\providecommand{\ones}{\vec{e}}
\providecommand{\va}{\ensuremath{\vec{a}}}
\providecommand{\vb}{\ensuremath{\vec{b}}}
\providecommand{\vc}{\ensuremath{\vec{c}}}
\providecommand{\vd}{\ensuremath{\vec{d}}}
\providecommand{\ve}{\ensuremath{\vec{e}}}
\providecommand{\vf}{\ensuremath{\vec{f}}}
\providecommand{\vg}{\ensuremath{\vec{g}}}
\providecommand{\vh}{\ensuremath{\vec{h}}}
\providecommand{\vi}{\ensuremath{\vec{i}}}
\providecommand{\vj}{\ensuremath{\vec{j}}}
\providecommand{\vk}{\ensuremath{\vec{k}}}
\providecommand{\vl}{\ensuremath{\vec{l}}}
\providecommand{\vm}{\ensuremath{\vec{l}}}
\providecommand{\vn}{\ensuremath{\vec{n}}}
\providecommand{\vo}{\ensuremath{\vec{o}}}
\providecommand{\vp}{\ensuremath{\vec{p}}}
\providecommand{\vq}{\ensuremath{\vec{q}}}
\providecommand{\vr}{\ensuremath{\vec{r}}}
\providecommand{\vs}{\ensuremath{\vec{s}}}
\providecommand{\vt}{\ensuremath{\vec{t}}}
\providecommand{\vu}{\ensuremath{\vec{u}}}
%\providecommand{\vv}{\ensuremath{\vec{v}}}
\providecommand{\vw}{\ensuremath{\vec{w}}}
\providecommand{\vx}{\ensuremath{\vec{x}}}
\providecommand{\vy}{\ensuremath{\vec{y}}}
\providecommand{\vz}{\ensuremath{\vec{z}}}
\providecommand{\vpi}{\ensuremath{\vecalt{\pi}}}


\def\comment{\textcolor{red}}
\def\add{\textcolor{blue}}

\newcommand{\tr}{{\rm Tr}}
\newcommand{\st}{{\rm s.t.}}
\newcommand{\by}{\mathbf{y}}
\newcommand{\bx}{\mathbf{x}}
\newcommand{\bX}{\mathbf{X}}
\newcommand{\bP}{\mathbf{P}}
\newcommand{\bp}{\mathbf{p}}
\newcommand{\bs}{\mathbf{s}}
\newcommand{\bS}{\mathbf{S}}
\newcommand{\bq}{\mathbf{q}}
\newcommand{\bi}{\mathbf{i}}
\newcommand{\br}{\mathbf{r}}
\newcommand{\bQ}{\mathbf{Q}}
\newcommand{\bI}{\mathbf{I}}
\newcommand{\bM}{\mathbf{M}}
\newcommand{\bW}{\mathbf{W}}
\newcommand{\bZ}{\mathbf{Z}}
\newcommand{\bV}{\mathbf{V}}
\newcommand{\bA}{\mathbf{A}}
\newcommand{\bB}{\mathbf{B}}
\newcommand{\bC}{\mathbf{C}}
\newcommand{\bD}{\mathbf{D}}
\newcommand{\bn}{\mathbf{n}}
\newcommand{\bY}{\mathbf{Y}}
%\newcommand{\eye}{{\rm j\;}}
\newcommand{\bDelta}{\boldsymbol{\Delta}}


\DeclareMathOperator{\range}{range}
\DeclareMathOperator{\sign}{sign}
\DeclareMathOperator{\nullspace}{null}
\DeclareMathOperator{\pr}{Pr}
\DeclareMathOperator{\Beta}{B}
\DeclareMathOperator{\real}{\Re}
\DeclareMathOperator{\imag}{\Im}
\DeclareMathOperator{\maximum}{max}
\DeclareMathOperator{\minimum}{min}
\DeclareMathOperator{\fl}{fl}       % flow
\DeclareMathOperator{\tve}{tve}     % flow
\DeclareMathOperator{\flatprof}{flat}
\DeclareMathOperator{\MLE}{MLE}

\newcommand{\cD}{{\cal D}}
\newcommand{\cJ}{{\cal J}}
\newcommand{\cL}{{\cal{L}}}
\newcommand{\cN}{{\cal N}}
\newcommand{\cP}{{\cal P}}
\newcommand{\cs}{{\cal s}}
\newcommand{\cbs}{{\utwi{\cal s}}}
\newcommand{\cG}{{\cal G}}
\newcommand{\cA}{{\cal A}}
\newcommand{\cS}{{\cal S}}
\newcommand{\cR}{{\cal R}}
\newcommand{\cT}{{\cal T}}
\newcommand{\cC}{{\cal C}}
\newcommand{\cE}{{\cal E}}
\newcommand{\cI}{{\cal I}}
\newcommand{\cB}{{\cal B}}
\newcommand{\cU}{{\cal U}}
\newcommand{\cM}{{\cal M}}
\newcommand{\cbP}{{\utwi{\cal P}}}

\newcommand{\cH}{{\cal H}}

\newcommand{\cK}{{\cal K}}
\newcommand{\cW}{{\cal W}}

\newcommand{\balpha}{{\utwi{\alpha}}}
\newcommand{\btheta}{{\utwi{\theta}}}
\newcommand{\bepsilon}{{\utwi{\epsilon}}}
\newcommand{\bgamma}{{\utwi{\gamma}}}
\newcommand{\beps}{{\utwi{\epsilon}}}
\newcommand{\bdeta}{{\utwi{\eta}}}
\newcommand{\bnu}{{\utwi{\nu}}}
\newcommand{\bpi}{{\utwi{\pi}}}
\newcommand{\bpsi}{{\utwi{\psi}}}
\newcommand{\bPsi}{{\utwi{\Psi}}}
\newcommand{\bphi}{{\utwi{\phi}}}
\newcommand{\bPhi}{{\bm{\Phi}}}
\newcommand{\bxi}{{\utwi{\xi}}}
\newcommand{\bSigma}{{\utwi{\Sigma}}}
\newcommand{\bvarphi}{{\utwi{\varphi}}}
\newcommand{\bchi}{{\utwi{\chi}}}
\newcommand{\bmu}{{\utwi{\mu}}}
\newcommand{\bzeta}{{\utwi{\zeta}}}
\newcommand{\bvarpi}{{\utwi{\varpi}}}
\newcommand{\bvartheta}{{\utwi{\vartheta}}}
\newcommand{\bGamma}{{\utwi{\Gamma}}}
\newcommand{\bLambda}{{\utwi{\Lambda}}}
\newcommand{\m}{\boldsymbol}
\allowdisplaybreaks[4]
%\pdfminorversion=4
\usepackage[colorlinks = true,
linkcolor = blue,
urlcolor  = blue,
citecolor = blue,
anchorcolor = blue]{hyperref}
%\renewcommand*{\thefootnote}{\fnsymbol{footnote}}
%\newcommand{\eqname}[1]{\tag*{#1}}% Tag equation with name
\newcommand{\mb}[1]{\mathbf{#1}}
\newcommand{\mc}[1]{\mathcal{#1}}
\newcommand{\mbb}[1]{\mathbb{#1}}
\newcommand{\mr}[1]{\mathrm{#1}}
\usepackage[framemethod=TikZ]{mdframed}
\mdfdefinestyle{MyFrame}{%
	linecolor=black,
	outerlinewidth=0.5pt,
	roundcorner=1pt,
	innerrightmargin=5pt,
	innerleftmargin=5pt,}
	
	%***************** Packages needed for numtests:(Hafez)**********%
\usepackage{graphicx}
%\usepackage{epstopdf}
%\epstopdfsetup{update}
%\usepackage{ifpdf}
%\ifpdf
%\DeclareGraphicsExtensions{.eps}
%\else
%\DeclareGraphicsExtensions{.eps}
%\fi
%\usepackage{subfig}

%\usepackage[draft]{hyperref}
%
%\let\oldbibitem\bibitem
%\def\bibitem{\vfill\oldbibitem}

%\usepackage{cleveref}
%\newcommand{\crefrangeconjunction}{--}

%\def\squad{\hskip.5em\relax}
%\def\ssquad{\hskip.25em\relax}
\usepackage{diagbox}
\usepackage{colortbl}