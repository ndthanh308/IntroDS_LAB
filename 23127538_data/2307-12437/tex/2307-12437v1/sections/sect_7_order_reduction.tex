\section{Comparative study of order reduction methods}
%
% Figure environment removed  
%
In this section, we study the conservativeness of the proposed zonotope order reduction method ReaZOR with respect to other known order reduction methods. While other methods can not replace ReaZOR as a part of a convex optimization program, it is still important to analyze its performance. We compare it with order reduction methods implemented in software package CORA \cite{althoff2015introduction, CORA2020}: 1) Girard's method proposed in \cite{girard2005reachability}, 2) Combastel's method proposed in \cite{combastel2003state}, 3-4) volume minimization methods (denoted as Method A and Method C in the original paper) \cite{althoff2010reachability}, 5) method proposed by Scott et. al. \cite{scott2016constrained}, and
6) principal component analysis (PCA)-based method reported in \cite{kopetzki2017methods}.
        %  = {'girard', 'combastel', 'methA', 'methC', 'scott', 'pca', 'ReaZOR', 'not reduced'}
        
The comparative study is performed as follows. We take a control design solution for inverted pendulum with a wall discussed in the Section \ref{sec_InvertedPendulum} represented by a sequence of zonotopes $\mathbb{X}_k$ and $\mathbb{U}_k$. To each zonotope $\mathbb{X}_k$ we apply propagation $\mathbb{X}_{k,i}^* = A_i(k) \mathbb{X}_k + B_i(k) \mathbb{U}_k + d(k)$ individually for all $[A_i, B_i, d_i] \in \mathcal{V}_k$. Then we find overapproximation \eqref{eq_convhull} of the convex hull of resulting zonotopes $\mathbb{X}_{k,i}^*$, denoted as $\mathbb{X}_k^*$. To follow algorithm \eqref{ConvexOptimization} we need to perform order reduction on $\mathbb{X}_k^*$ and apply Minkowski sum to it, thus finding $\mathbb{X}_k$.

Zonotope $\mathbb{X}_k^*$ has $r$ columns, and we reduce it to $p$ columns using all methods listed above. Let $V_{k,h}$ be the volume of the zonotope $\mathbb{X}_k^*$ after reduction by method \#$h$, where the first six methods have been listed above, ReaZOR is the method number 7; $V_{k,8}$ corresponds to the volume of $\mathbb{X}_k^*$ without reduction. Let $\nu_{k}$ be the percentage difference between the volume of a reduced zonotope and the volume of the zonotope without reduction $V_{k,8}$:
%
\begin{equation}
    \nu_{k, h} = \frac{| V_{k,h} - V_{k,8} |}{V_{k,8}} \cdot 100\%
\end{equation}
%\underset{h = 1, .., 8}{\text{max}} 
%
We will refer to $\nu_{k, h}$ as volume errors. In this experiment, we use zonotope propagation for the uncertain hybrid dynamics described in Section \ref{sec_InvertedPendulum}; the resulting sequence of shown in Figure \ref{fig:convhull_reduction}. The goal of the next experiment is to show how various order reduction techniques differ in terms of volume error. 
%
\begin{table}[t]
\caption{Comparison of mean and maximum values of volume errors of the reduced zonotopes for different order reduction methods}
\label{table_experiment1}
\begin{center}
\begin{tabular}{|c|c|c|c|c|}
\hline
\# & Method & Reference &  $\underset{k}{\text{mean}}( \nu_{k, h} )$ & $\underset{k}{\text{max}}( \nu_{k, h} )$ \\
\hline
1 & Girard's & \cite{girard2005reachability, CORA2020} & $0.673$\% & $1.995$\% \\
\hline
2 & Combastel's & \cite{combastel2003state, CORA2020} & $0.673$\% & $1.995$\% \\
\hline
3 & Method A & \cite{althoff2010reachability, CORA2020} & $0.673$\% & $1.995$\% \\
\hline
4 & Method C & \cite{althoff2010reachability, CORA2020} & $0.673$\% & $1.995$\% \\
\hline
5 & Scott et. al. & \cite{scott2016constrained, CORA2020} & $0.674$\% & $1.995$\% \\
\hline
6 & PCA-based & \cite{kopetzki2017methods, CORA2020} & $0.673$\% & $1.995$\% \\
\hline
7 & ReaZOR & & $0.675$\% & $1.981$\% \\
\hline
\end{tabular}
\end{center}
\end{table}
%
Table \ref{table_experiment1} shows the mean and maximum values of $\nu_{k, h}$ across the whole trajectory for each method included in the comparison. As we can see, ReaZOR is slightly better than the other methods in terms of the maximum value of volume error and is slightly worse than the others in terms of the mean value of volume error; but in both cases, the differences are negligible compared with the errors themselves. This indicates that the volume errors are dominated by the geometry of the sets rather than by particular features of the order reduction methods. This allows us to conjecture that in terms of volume error, all methods show very similar performance, meaning that other metrics can be used in choosing the preferred order reduction method; in our case, we value numerical properties that the methods exhibit when used as a part of a convex optimization problem.
