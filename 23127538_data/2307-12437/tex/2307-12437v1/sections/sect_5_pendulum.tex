
\section{Case-Study: Inverted Pendulum with a Wall} \label{sec_InvertedPendulum}

For the validation of the proposed methods, we take an example of a hybrid system - an inverted pendulum that interacts with an elastic wall. This system has been used for a similar purpose in \cite{sadraddini2019sampling,marcucci2017approximate}.

\subsection{Experimental setup}
%
Our setup consists of an 18V BLDC motor GYEMS RMD-L-50 and its driver GYEMS DRC-06, allowing current control; we use Renishaw MHA7 for position measurements and CAN BUS for communications, with the control commands updated on 250Hz. The program is being run on Raspberry PI 4 ModelB. A 20x20x410mm aluminum profile is attached to the shaft of the motor; the profile has sockets for attaching an additional mass at different distances from the motor shaft.

\subsection{Mathematical description}

The position of the pendulum is described by angle $q$. The dynamics of the system has two hybrid states: with contact ($q \geq q_c$) and without it ($q < q_c$), where $q_c$ is the angle at which the contact with the undeformed wall takes place:

\begin{equation*}
\begin{cases}
        I \ddot q + \mu_f \dot q + m g l \sin(q) = c_{\tau} i & \text{if} \ q < q_c \\
        I \ddot q + \mu_c \dot q + m g l \sin(q) + k(q - q_c)= c_{\tau} i & \text{if} \ q \geq q_c \\
\end{cases}
\end{equation*}
%
where $I$, $m$, $l$ are the moment of inertia, mass, and length of the pendulum, $\mu_f$ and $\mu_c$ are viscous friction forces, $g$ is the gravitational constant, $c_{\tau}$ is the torque coefficient, and $k$ is the stiffness coefficient of the wall.

% Figure environment removed

Some of the parameters are known exactly: $m = 0.126$ kg,  $\mu_f = 0.001$ N/ms, $c_{\tau} = 0.03$ Nm/A. Others are known to be in the intervals: $I \in [0.0116, 0.0203] \ \text{m}^2$, $k \in [116.1, 141.9]$ N/rad, $\mu_c \in [0.41, 0.51]$ N/ms and $l \in [0.12, 0.18]$ m. This allows us to define two sets of vertices $\mathcal{V}_{k, 1}$ for the case $q \geq q_c$, containing variations of parameters $I$ and $l$ and $\mathcal{V}_{k, 2}$ for the case $q < q_c$, containing variations of parameters $k$ and $\mu_c$, with the assumption that when the pendulum touches the wall, these two parameters dominate its dynamics. In both cases $\mathcal{V}_{k, i}$ will have four elements.

Initial and final sets $\mathbb{X}_0$ and $\mathbb{X}_f$ are constrained as: $\mathbb{X}_0 = \mathbb{X}_f \in \Zo{0_{2 \times 1}}{\text{diag}(0.02, 0.4)}$, where $\text{diag}$ is an operator that returns a matrix with its inputs on the diagonal, and $0_{2 \times 1}$ is a vector of zeros. Additive disturbance $\mathbb{W}$ is chosen as $\mathbb{W} = \Zo{0_{2 \times 1}}{\text{diag}(10^{-4}, 10^{-3})}$, torque limits are set implicitly as $\mathbb{H}^u = \Zo{0}{20}$. Parameter $z$ (zonotope order) in the order reduction algorithm is chosen as 6.

The control law was applied to the experimental setup. The first experiment was performed for $ l = 0.13 $ m, the second for $ l = 0.17 $ m. Results of the control design and the experiments are shown together in Fig. \ref{fig:phase_pendulum}. It was possible to successfully design a control policy by using proposed methods as well as verify its performance on the experimental setup.



