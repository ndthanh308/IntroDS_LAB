\section{State of the Art}

Zonotope set representation has been successfully used in various areas control: in reachability analysis \cite{alanwar2021data}, formal system verification \cite{althoff2008verification}, control design based on invariant sets \cite{wan2009numerical, hamdi2017stabilization, han2016enlarging}, fault detection \cite{xu2013relationship, yang2020novel}, and state estimation \cite{alamo2005guaranteed, le2013zonotopic}, among other applications. The questions of robustness of these methods to model uncertainties and uncertain inputs have been studied as well \cite{le2013zonotopic, combastel2015zonotopes}.

Recently, a control design for piecewise affine systems was proposed based on zonotope set propagation \cite{sadraddini2019sampling}. It can be seen as similar to explicit MPC resulting from considering finite-time stability of a time-varying linear system with ellipsoidal state sets \cite{amato2014finite}. There are however a number of differences between the two approaches. The primal difference is that the FTS-based control design makes use of Lyapunov equations and casts the problem as an SDP, whereas the zonotope propagation method directly applies dynamics equations to the state set, casting the problem as an LP, QP, or SOCP. The latter has many advantages but lacks techniques for dealing with multiplicative model uncertainty; relevant LMI-based methods do not have zonotope-based analogs. This paper aims to alleviate part of this problem by proposing a control design method robust to polytopic model uncertainties.

There are a number of classical results in robust control of systems with norm-bounded and polytopic uncertainty \cite{amato2006robust}. Quadratic stability of a linear time-invariant (LTI) system with polytopic model uncertainty (where the set of all possible models is a polytope) can be proven by simultaneously solving Lyapunov equations for the vertices of the polytopic set of possible models, which can be cast as a single LMI \cite{amato2006robust}; this result allows to design robust control both for uncertain LTI and linear parameter varying systems and can be naturally extended to control design. For systems with norm-bounded model uncertainty, the problem can be reformulated as convex using linear-fractional transformation and S-procedure, also enabling control design as a single LMI \cite{amato2014finite}. This approach can be extended to include input and output constraints, leading to a robust explicit MPC formulation \cite{de2004robust}. While these results have vast practical significance, the appeal of zonotope propagation-based techniques is the lower computational complexity, which gives them the ability to handle hybrid systems while using mixed-integer solvers as a back-end \cite{sadraddini2019sampling}, as well as the possibility to directly account for additive disturbance using Minkowski sum of zonotopes.

Control design based on zonotope set propagation faces a number of well-known problems associated with numerical operations on zonotopes in general, and with the use of Minkowski sums in particular. One of these problems is zonotope containment. There are a number of works on the topic, with an array of algorithms proposed \cite{han2016enlarging, sadraddini2019linear, kulmburg2021co}. However, since the control design is required to be cast as convex optimization, the range of possible zonotope containment algorithms is limited to the ones that can be presented as linear or SOCP problems. Such an algorithm has been presented in \cite{sadraddini2019linear}. A critical study of this encoding can be found in \cite{kulmburg2021co}.

Another problem is zonotope order reduction methods. The result of Minkowski summation of two zonotopes is a zonotope with a larger generator; this means that iterative application of Minkowski sums leads to a linear increase in the number of elements in the zonotope generators, which in turn means a linear increase in the number of continuous variables in the resulting optimization problem. It has been observed that replacing a zonotope with a new one whose generator has fewer columns leads to a better performance in terms of zonotope propagation and control design \cite{sadraddini2019sampling}. A number of order reduction methods have been proposed \cite{althoff2015introduction, girard2005reachability, combastel2003state, althoff2010reachability, scott2016constrained}. However, those methods are not optimized for an iterative application. In this paper, we exploit the iterative nature of the state propagation problem to propose a novel order reduction method which 1) facilitates the diagonal blocks in zonotopes, 2) prevents the increase of zonotope order after Minkowski summation for a certain class of disturbances.

Thus, the main contributions of the paper are the following:

\begin{itemize}
    \item An extension of the existing zonotope-based explicit MPC to cover the case of time-varying polytopic parametric uncertainty.
    \item A novel convex zonotope order reduction method, that takes advantage of the iterative structure of the convex program which implements the zonotope-based explicit MPC.
    \item A parallelotope-based time-free control policy choice algorithm that solves the time initialization problem (i.e., the problem of starting the motion from an arbitrary part of the trajectory, rather than from its beginning) and avoids chattering.
\end{itemize}