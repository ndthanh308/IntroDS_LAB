\section{Problem Formulation}

Consider an uncertain discrete affine time-variant (LTV) system:
%
\begin{equation}
\label{eq:LTV}
    \begin{matrix}
    x_{k+1}  =  A(k) x_k + B(k) u_k + d(k) + w_k
    \\[0.5em]
    [A(k) \ B(k) \ d(k)] \in \Omega_k
    \\[0.5em]
    \end{matrix}
\end{equation}
%
where 
$x_k \in \mathbb{R}^n$ is the state of the system, 
$u_k \in \mathbb{R}^m$ is the control input, 
$w_k \in \mathbb{R}^n$ is a bounded process disturbance, 
$A(k) \in \mathbb{R}^{n \times n}$ is state matrix, 
$B(k)\in \mathbb{R}^{n \times m}$ is control matrix, 
$d(k) \in \mathbb{R}^n$ is additive term of the affine dynamical model,
and 
$\Omega_k = \text{convexhull}(\mathcal{V}_k)$, 
is a polytopic set of models, defined as a convex hull of its $L$ vertices $\mathcal{V}_k = \left\{
    [A_1(k) \ B_1(k) \ d_1(k)], ..., 
    [A_L(k) \ B_L(k) \ d_L(k)]
    \right\}$. We call the models in the set $\mathcal{V}_k$ \emph{vertex models}.
    
First, we note that the polytopic set $\Omega_k$ is itself time-varying, which well reflects the scenario when $\Omega_k$ is obtained from a linearization of an uncertain non-linear system along a given trajectory. Second, while both $d(k)$ and $w_k$ appear in the expression \eqref{eq:LTV} as additive terms, we avoid grouping them into a single term, since they belong to different sets and will be handled differently in set propagation in the case of uncertain dynamics. Now we can formulate the problem that this paper aims to solve:
    
\begin{problem}
\label{p:problem1}
    For the system \eqref{eq:LTV} find a trajectory $\bar{u}_k$, $\bar{x}_k$ and control policy $u_k = (\bar{u}_k - K_k (x_k - \bar{x}_k) ) \in \mathbb{H}_k^u$, such that for any initial condition $x_0 \in \mathbb{X}_0$, the intermediate values of the state $x_k$ are bounded by $x_k \in \mathbb{H}_k^x$, for any disturbance $w_k \in \mathbb{W}_k$, and for any $[A \ B \ d] \in \Omega_k$, $\forall k$, where $\mathbb{H}_k^u$, $\mathbb{H}_k^x$, $\mathbb{X}_0$ and $\mathbb{W}_k$ are zonotopes.
\end{problem}

\subsection{Known parameters case}
%
If the matrices $A$ and $B$ are known exactly, the problem \ref{p:problem1} can be described as an evolution (propagation) of the initial zonotope $\mathbb{X}_0$, subject to dynamics \eqref{eq:LTV}:
%
\begin{equation}
\label{eq:LTV_zonotopes}
    \mathbb{X}_{k+1}  = (A(k) \mathbb{X}_k + B(k) \mathbb{U}_k + d(k)) \oplus \mathbb{W}_k
\end{equation}
%
where 
$\mathbb{X}_k = \Zo{\bar{x}_k}{G_k}$, 
$\mathbb{U}_k = \Zo{\bar{u}_k}{\theta_k}$, 
and 
$\mathbb{W}_k = \Zo{0_{n \times 1}}{W(k)}$ are zonotopes representing state, control actions, and process noise, 
$G_k      \in \mathbb{R}^{n \times p}$, 
$\theta_k \in \mathbb{R}^{m \times p}$, and 
$W(k)      \in \mathbb{R}^{n \times n_w}$ are generators or these zonotopes, 
$\bar{x}_k      \in \mathbb{R}^{n}$ and 
$\bar{u}_k \in \mathbb{R}^{m}$ 
are their centers. 
Using \eqref{eq_single_vector_Minkowski} we could combine $ d(k)$ and $\mathbb{W}_k$ into a single zonotope $\Zo{d(k)}{W(k)}$, arriving at a standard linear dynamics representation. On each time step any admissible disturbance $w_k \in \mathbb{W}_k$ can act on the system, which is the reason for the use of Minkowski sum; therefore the order of the zonotopes $\mathbb{X}_k$ grows on each time step unless order reduction techniques are employed. 

As it was discussed in \cite{sadraddini2019sampling, sadraddini2019linear}, propagation of zonotopes can be decomposed into separate equations describing the evolution of their centers and generators. With that \eqref{eq:LTV_zonotopes} can be re-written as:
%
\begin{equation}
\label{eq:generator_prop}
    \begin{cases}
    G_{k+1} = (A(k) G_{k} + B(k) \theta_k, \ W(k)) \\
    \bar{x}_{k+1} = A(k) \bar{x}_{k} + B(k) \bar{u}_k + \bar{d}(k) 
    \end{cases}
\end{equation}

For the case when $A$ and $B$ are known exactly, the following linear control law was proposed in \cite{sadraddini2019sampling}:
%
\begin{equation}
\label{eq:lin_feedback}
    u = \bar{u}_k - \theta_k G_k^\dagger(x_k - \bar{x}_k)
\end{equation}
%
where $(\cdot)^\dagger$ denotes Moore-Penrose pseudoinverse. However, the problem \ref{p:problem1} does not allow precise knowledge of $A$ and $B$; our solution to this problem is discussed in the next section.