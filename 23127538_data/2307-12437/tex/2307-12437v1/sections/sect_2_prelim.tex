\section{Notation and Preliminaries}
% 
A zonotope can be defined as a point-symmetric set in $n$-dimensional space \cite{kopetzki2017methods}, described as a center $c \in \mathbb{R}^n $ and $p$ generators $g^{(i)} \in \mathbb{R}^n, \ i \in \{1,...,p\}$; the latter can be presented as columns of matrix $G \in \mathbb{R}^{n \times p}$:
%
\begin{equation} 
\label{eq_zonotope_def}
    \mathbb{Z} = \Zo{c}{G} = \{c + \sum_{i=1}^p \beta_i g^{(i)} :  -1 \leq \beta_i \leq 1 \}
\end{equation}
%
where $\beta_i$ are scalar multipliers. Eq. \eqref{eq_zonotope_def} defines a \emph{vector zonotope} (it represents a set whose elements are vectors), as oppose to a \emph{matrix zonotope} which we introduce later. 
In the following discussion, we will also use the concept of \emph{zonotope order}, which is a ratio between the number of columns and rows of the generator. The order of zonotope \eqref{eq_zonotope_def} is given as $p/n$.

\subsection{Algebraic Operations on Vector Zonotopes}
Vector zonotopes are closed under addition and linear transformation, defined as follows:
%
\begin{equation} 
\label{eq_zonotope_addition}
    \Zo{x}{X}+\Zo{y}{Y} = \Zo{x+y}{X+Y}
\end{equation}
\begin{equation} 
\label{eq_zonotope_linear_tranform}
    A\Zo{c}{G} = \Zo{Ac}{AG},
\end{equation}
%
where $A$ is a linear operator.

Minkowski sum for vector zonotopes is defined as:
%
\begin{equation}
\label{eq:MinkowskiDef}
    \Zo{x}{X} \oplus \Zo{y}{Y} = \Zo{x + y}{(X, Y)}
\end{equation}
%
where notation $(X, Y)$ refers to horizontal matrix concatenation. 


We can define the addition of a vector zonotope and a vector as follows:
%
\begin{equation} 
\label{eq_zonotope_vector_addition}
    \Zo{c}{G}+v = \Zo{c+v}{G}
\end{equation}

That definition implies the following property of the vector addition and Minkowski sum:
%
\begin{equation}
\label{eq_single_vector_Minkowski}
    \Zo{x + v}{X} \oplus \Zo{y}{Y} = \Zo{x}{X} \oplus \Zo{y + v}{Y} 
\end{equation}

\subsection{Zonotopes with positive-semidefinite diagonal blocks}

Given zonotopes $\Zo{x}{(D_x, X)}$ and $\Zo{y}{(D_y, Y)}$, where $D_x \succeq 0$ and $D_y \succeq 0$ are diagonal matrices, their Minkowski sum can be described by the following zonotope:
%
\begin{equation}
\label{diagMinkowski}
    \Zo{x}{(D_x, X)} \oplus \Zo{y}{(D_y, Y)} \equiv
    \Zo{x + y}{(D_x+D_y, X, Y)},
\end{equation}
%
where equivalence is understood in the sense that the set to the left of the sign $\equiv$ contains all vectors that are contained in the set to the right, and no others. Note that this definition of Minkowski sum results in zonotopes with fever columns in the generator matrix; however, this definition only works for zonotopes with positive semidefinite diagonal blocks in their generators.

%
\subsection{Zonotope Containtment}
%
To check if a zonotope is contained in another zonotope, we can use the method proposed in \cite{sadraddini2019linear}.
Given two zonotopes $\mathbb{X}$ = $\langle x,X \rangle$ and $\mathbb{Y}$ = $\langle y,Y \rangle$, where $X \in \mathbb{R}^{n \times n_x}$, $Y \in \mathbb{R}^{n \times n_y}$, if there exists $\Gamma \in \mathbb{R}^{n_y \times n_x}$ and $\beta \in \mathbb{R}^{n_y}$, such that:
%
\begin{equation}
    \label{eq:containment}
    X = Y\Gamma, \ y - x = Y\beta, \  ||(\Gamma,\beta)||_\infty \leq 1
\end{equation}
%
then zonotope $\mathbb{X}$ is contained in zonotope $\mathbb{Y}$. This method is convenient, as it requires solving a single linear program.
%
\subsection{Approximating convex hull of two zonotopes}
%
In \cite{girard2005reachability} it is proposed to use the following approximation of the convex hull of two zonotopes $\mathbb{X} = \langle x,X \rangle$ and $\mathbb{Y} = \langle y,Y \rangle$:
%
%
\begin{equation}
    \label{eq_convhull}
    \text{Co}(\mathbb{X}, \mathbb{Y}) \subseteq 
    \Zo{\frac{x + y}{2}}
    {\left(
    \frac{X + Y}{2},\frac{x - y}{2},\frac{X - Y}{2}
    \right)}
\end{equation}

This method can be used to approximate a convex hull of a set of $2^p$, dividing the set into $2^{p-1}$ pairs and applying the method to each pair, and repeating the same step on the resulting set of zonotopes, iterating $p$ times. 

This method is conservative and computationally inexpensive. Most importantly for our purpose, it can be incorporated in a convex optimization problem formulation as a linear equality constraint. 

\subsection{Matrix zonotopes}
Zonotopes can be used to represent symmetric polytopic sets of matrices; in that case they are referred to as \emph{matrix zonotopes} \cite{althoff2011reachable}. A matrix zonotope is defined analogous to a vector zonotope, as $\mathbb{A} = 
    \Zo{A^{(0)}}{ \{ A^{(1)}, ..., A^{({n_a})} \} }$:
%
%
\begin{equation} 
\label{eq_matrix_zonotope_def}
    \mathbb{A} = 
    \{A^{(0)} + \sum_{i=1}^{n_a} \beta_i A^{(i)} :  -1 \leq \forall\beta_i \leq 1 \},
\end{equation}
%
where $A^{(i)}$ are matrices. One can define multiplication of matrix zonotopes $\mathbb{A} \otimes \mathbb{B}$ as:
%
\begin{equation}
\begin{aligned}
     \mathbb{A} \otimes \mathbb{B} = \{ AB, \ A \in \mathbb{A}, \ B \in \mathbb{B} \}
\end{aligned}
\end{equation}

In \cite{althoff2011reachable} this operation is defined for sets of square matrices; however, given appropriate dimensions of generator matrices, it can be extended to matrix-vector multiplication without changes in the formulations. Following \cite{althoff2011reachable}, an overapproximation of the matrix zonotope multiplication can be introduced as follows. Given a matrix zonotope \eqref{eq_matrix_zonotope_def} and a vector zonotope \eqref{eq_zonotope_def}, their product is approximated as:
%
\begin{equation}
\label{eq_matrix_zonotope_mult}
\begin{aligned}
     \mathbb{A} \otimes \mathbb{Z} \ \approx
     \Bigl\langle A^{(0)} c, \  
     \Bigl(
     A^{(0)}g^{(1)}, \ ..., A^{(0)}g^{(n_p)}, \ 
     A^{(1)} c, \ A^{(1)}g^{(1)}, \ ..., A^{(1)}g^{(n_p)}, \ 
     ..., \ 
     A^{(n_a)} c, \ A^{(n_a)}g^{(1)}, \ ..., A^{(n_a)}g^{(n_p)} 
     \Bigr)\Bigr\rangle
\end{aligned}
\end{equation}

Implementation of the operations presented in this section can be found in the software package CORA \cite{althoff2015introduction}.

