\section{Case-Study: Pendubot}
%
% Figure environment removed
%
We would like to demonstrate the performance of our algorithm for an underactuated system performing a non-trivial motion. To demonstrate this, we will take pendubot, a classic example of an underactuated system discussed in \cite{freidovich2008periodic, fantoni2000energy}
%
\subsection{Experimental setup}
Pendubot is a two-link planar manipulator with an actuator in the first link and a passive second link. We use a BLDC motor (T-Motor U8 Lite KV85) with an ODrive controller, allowing current control. Positions of the links are measured with Renishaw MHA7 encoders. Control commands are updated at the 100Hz frequency. A diagram of the robot is shown in Fig. \ref{fig:pendubot}
%
\subsection{Mathematical description}
%
% Figure environment removed
%
Pendubot dynamics can be described in the following form:
%
\begin{equation}\label{eq:pendubot_dynamics}
D(q) \ddot q + C(q,\dot q) \dot q + g(q) = B u - f_f
\end{equation}	
%
where $q = [q_1,q_2]^\top$ defines orientation of the links, $D$ is the generalized inertia matrix, $C$ is the Cariolis and inertial force matrix, $B$ is the control matrix, and $f_f$ is the generalized bearing friction. Except for the bearing friction, analytical expressions for these quantities can be found in the literature \cite{freidovich2008periodic, fantoni2000energy}. Bearing friction can be described as:
%
\begin{equation}
    f_f = b_s\text{sgn}(\dot q) + b_v \dot q
\end{equation}
%
where $b_s$ and $b_v$ are constants and $\text{sgn}$ is element-wise sign function. 

The lengths and masses of the system parts were measured directly, the moments of inertia were taken from the design documentation, and the friction coefficients were obtained by identification. However, to demonstrate the robustness of the algorithm, in some experiments, an additional mass $m_e = 0.06$ kg is attached to the middle of the first link.

The trajectory of the system was obtained via direct collocation as a solution to non-convex optimization, followed by linearization and discretization along the found trajectory. Initial and final sets $\mathbb{X}_0$ and $\mathbb{X}_f$ are constrained as: $\mathbb{X}_0 = \mathbb{X}_f \in \Zo{0_{4 \times 1}}{\text{diag}(0.2, 0.2, 2, 2)}$. Additive disturbance $\mathbb{W}$ is chosen as $\mathbb{W} = \Zo{0_{4 \times 1}}{9.5 \cdot 10^{-4} I}$, where $I$ is identity matrix. Torque limits are set implicitly as $\mathbb{H}^u = \Zo{0}{10}$. Parameter $z$ (zonotope order) in the order reduction algorithm is chosen as 62.5, much higher than in the previous example.

The control law was applied to the experimental setup. The first experiment was performed with additional mass $m_e = 0.06$ kg on the first link, and the second - without the additional mass. Results of the control design and the experiments are shown together in Fig. \ref{fig:phase_pendubot}. The method allowed us to design a control law that was able to stabilize the trajectory, which was shown both in simulation and via experimental study.

Control design for this and previous experiments can be replicated using our code, distributed under open source license \cite{Balakhnov2021}.