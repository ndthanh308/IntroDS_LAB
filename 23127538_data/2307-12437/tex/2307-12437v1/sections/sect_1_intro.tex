\section{Introduction}

Model-based control is an essential tool in many areas of engineering, especially in Robotics. Its general weakness relates to our limited ability to exactly model the plant; examples of that are parametric uncertainty, unmodeled dynamics, unmodeled disturbances, etc. When a sufficiently precise model is not available the model uncertainty needs to be explicitly addressed in a model-based control framework, for which families of methods have been designed. Examples of the latter include adaptive and robust control \cite{ortega2019adaptive, karimi2007robust}; the latter is well represented by linear-matrix inequality (LMI)-based methods for systems with polytopic, interval, and norm-bounded uncertainties \cite{karimi2007robust, rosinova2003robust, amato2011robust}, as well as for systems with unknown inputs, and others. 

Among the model-based methods that have seen a lot of progress in the last three decades is control design based on set propagation. Advantages of this type of control design include the ability to directly account for the set of initial conditions and to place constraints on the final and intermediate sets of control actions, states, and/or outputs. Limiting ourselves to two examples of control design, we can mention the propagation of ellipsoidal sets employed in Finite-Time Stability (FTS) control \cite{amato2014finite} and propagation of zonotopes employed in reachability analysis and recently in control design for piecewise affine (PWA) dynamical systems \cite{sadraddini2019sampling}. Zonotopes are symmetric polytopes, described as an affine transformation of a unit cube; the linear part of the transformation is called \emph{generator}, and the additive part is called \emph{center}. Their properties will be discussed in the later sections. Both types of set propagation allow casting control design problem as a single convex optimization program, which is one of the chief appeals of these approaches. In the case of linear dynamical systems, propagation of ellipsoidal sets in FTS control usually leads to linear matrix inequalities (LMI) \cite{dorato1997robust, amato2006finite, amato2014finite} and hence semidefinite programs (SDP) \cite{boyd1994linear}, while the propagation of zonotopes can lead to linear programs (LP), quadratic programs (QP) or second-order cone programs (SOCP), depending on the objective function and set containment criteria used \cite{sadraddini2019sampling}. For piecewise affine dynamical systems, control design requires integer variables which leads to mixed-integer programs. Among mentioned types of optimization problems, SDP is the most challenging, having less mature solvers compared with SOCP, QP, and LP. This is especially apparent with mixed-integer problems \cite{sadraddini2019sampling}. This justifies the interest in control design based on zonotope propagation.

Robust control design methods for FTS and are well developed, taking advantage of the results in LMI-based control for linear systems \cite{dorato1997robust, amato2014finite}. The latter offer methods for handling norm-bounded uncertainties, polytopic and interval uncertainties, and unknown inputs, casting all of those problems as a single LMI \cite{kothare1996robust, amato2006robust, amato2011robust, li2008improved, ramos2002lmi}. The same methods are currently lacking for control design based on the propagation of zonotopes. We should note that this statement refers only to feedback control design, as robust state estimation methods based on the use of zonotopes for set representation have been studied previously \cite{combastel2015zonotopes, wang2018set}. It is also limited to zonotope propagation cast as a set of linear transformations and Minkowski sums of zonotopes presented as centers and generators, which leads to linear, quadratic, and second-order cone programs; there are works that implement, e.g., stabilization of systems with uncertainties based on invariant sets represented as zonotopes, where computations are cast in the traditional LMI framework leading to SDP problems \cite{hamdi2017stabilization}. The goal of this work is to provide a control design framework based on zonotope propagation, robust to additive and polytopic multiplicative model uncertainties in both the state and control matrices.

An important limitation, characteristic of a number of robust control methods for linear systems with multiplicative time-varying uncertainties is that the set of all possible uncertain parameters is itself time-invariant. Examples of control methods developed for such systems include the aforementioned LMI-based robust control for LTI systems, as well as a number of robust control methods for linear parameter-varying (LPV) systems \cite{apkarian1995self, amato2006robust}. For models obtained as a linearization of an uncertain nonlinear system along a nominal trajectory, the set of uncertain parameters of the resulting model may itself be time-variant. We show that this type of model can be handled in the proposed framework; the proposed method can also be extended to hybrid linear dynamical systems, as we show in the paper.

Thus, this paper proposes a robust explicit MPC method for time-varying linear and hybrid linear dynamical systems based on the propagation of zonotopes, allowing to directly handle time-varying polytopic multiplicative uncertainties, as well as additive uncertainties and constraints on state and control. To facilitate the practical application of the proposed method, we introduce a new zonotope order reduction method, as well as a time-agnostic approach to feedback control. To the best of our knowledge, this is the first time a zonotope set propagation-based explicit MPC robust to time-varying polytopic parametric uncertainty was proposed, and the first time a zonotope-based robust MPC was experimentally validated.




