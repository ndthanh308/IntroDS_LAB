\documentclass{article}


% if you need to pass options to natbib, use, e.g.:
%     \PassOptionsToPackage{numbers, compress}{natbib}
% before loading neurips_2023


% ready for submission

\usepackage[final,nonatbib]{neurips_2023}


% to compile a preprint version, e.g., for submission to arXiv, add add the
% [preprint] option:
%     \usepackage[preprint]{neurips_2023}


% to compile a camera-ready version, add the [final] option, e.g.:
%     \usepackage[final]{neurips_2023}


% to avoid loading the natbib package, add option nonatbib:
%    \usepackage[nonatbib]{neurips_2023}
     % hyperlinks
\usepackage{graphicx}
\usepackage{hyperref}  
\usepackage[utf8]{inputenc} % allow utf-8 input
\usepackage[T1]{fontenc}    % use 8-bit T1 fonts
\usepackage{url}            % simple URL typesetting
\usepackage{booktabs}       % professional-quality tables
\usepackage{amsfonts}       % blackboard math symbols
\usepackage{nicefrac}       % compact symbols for 1/2, etc.
\usepackage{microtype}      % microtypography
\usepackage{xcolor}         % colors
\usepackage{algorithm}
\usepackage{algorithmic}
\usepackage{tikz}
\usepackage{wrapfig}
\usepackage{adjustbox}

% Attempt to make hyperref and algorithmic work together better:
% \newcommand{\theHalgorithm}{\arabic{algorithm}}

% For theorems and such
\usepackage{amsmath}
\usepackage{amssymb}
\usepackage{mathtools}
\usepackage{amsthm}

\usetikzlibrary{bayesnet}  
\usetikzlibrary{positioning}  

% if you use cleveref..
\usepackage[nameinlink,capitalize,noabbrev]{cleveref}


%%%%%%%%%%%%%%%%%%%%%%%%%%%%%%%%
% THEOREMS
%%%%%%%%%%%%%%%%%%%%%%%%%%%%%%%%
\theoremstyle{plain}
\newtheorem{theorem}{Theorem}[section]
\newtheorem{proposition}[theorem]{Proposition}
\newtheorem{lemma}[theorem]{Lemma}
\newtheorem{corollary}[theorem]{Corollary}
\theoremstyle{definition}
\newtheorem{definition}[theorem]{Definition}
\newtheorem{assumption}[theorem]{Assumption}
\theoremstyle{remark}
\newtheorem{remark}[theorem]{Remark}


% Todonotes is useful during development; simply uncomment the next line
%    and comment out the line below the next line to turn off comments
%\usepackage[disable,textsize=tiny]{todonotes}
\usepackage[textsize=tiny]{todonotes}

\usepackage{amsthm, amssymb}
\usepackage{hyperref}
% \usepackage[nameinlink,capitalize]{cleveref}
\usepackage{booktabs}
\usepackage{comment}
\usepackage{graphicx}
% Define others
\newif\ifshowcomments
\showcommentsfalse
\usepackage{marginnote}
\usepackage{bbm}
\usepackage{wrapfig}
\usepackage{caption}
\usepackage{subcaption}
\usepackage{adjustbox}
%\usepackage{subfig}
\newcommand{\wg}[1]{\textcolor{red}{#1}}
\newcommand{\WG}[1]{\ifshowcomments \textcolor{red}{WG: #1} \fi}
\newcommand{\rev}[1]{\textcolor{blue}{#1}}

\newcommand{\np}[1]{\ifshowcomments \textcolor{red}{#1} \fi}
\newcommand{\NP}[1]{\ifshowcomments \textcolor{red}{NP: #1} \fi}

\newcommand{\cz}[1]{ \textcolor{blue}{#1} }
\newcommand{\CZ}[1]{ \textcolor{blue}{CZ: #1}}

% \def\sectionautorefname{Section}
% \theoremstyle{definition}
% \newtheorem{theorem}{Theorem}
% \newtheorem{corollary}{Corollary}[theorem]

% \theoremstyle{definition}
% \newtheorem{assumption}{Assumption}

\newcommand{\ModelName}{\texttt{BayesDAG}} % SaBS

% Additional math notation
\newcommand{\multitimeX}[1]{\mX_{0:T}^{(#1)}}
\newcommand{\singletimeX}[1][0]{\mX_{#1:T}}
\newcommand{\singletimeY}[1][0]{\mY_{#1:T}}
\newcommand{\singletimeXsource}[1][0]{\mX_{#1:\sourcelength}}
\newcommand{\singletimeYsource}[1][0]{\mY_{#1:\sourcelength}}

\newcommand{\singletimeXfull}[2]{\mX_{#1:#2}}
\newcommand{\singletimeYfull}[2]{\mY_{#1:#2}}

\newcommand{\Graph}[1][0]{\mG_{#1:K}}
\newcommand{\singlegraph}{\mG_{0}}
\DeclareMathOperator{\tr}{tr}
\DeclareMathOperator{\CATE}{CATE}
\DeclareMathOperator{\doop}{do}
\def\vardist{{q_\phi(\mG)}}
\def\sourcelength{{\mathcal{S}}}
\def\sourcenodes{{\mX_{0:\sourcelength}}}

% \newcommand{\PaGst}[1]{\Pa_{G}(#1;<t)}
\newcommand{\PaGst}[1][i]{\Pa_{G}^{#1}(<t)}
% \newcommand{\PaGt}[1]{\Pa_{G}(#1;t)}
\newcommand{\PaGt}[1][i]{\Pa_{G}^{#1}(t)}

% \newcommand{\PaGdst}[1]{\Pa_{G'}(#1;<t)}
\newcommand{\PaGdst}[1][i]{\Pa_{G'}^{#1}(<t)}
% \newcommand{\PaGdt}[1]{\Pa_{G'}(#1;t)}
\newcommand{\PaGdt}[1][i]{\Pa_{G'}^{#1}(t)}

 \newcommand{\ind}{\perp\!\!\!\!\perp} 

\newcommand{\PaGstXY}[1][X,Y]{\Pa_{G}^{#1}(<t)}
\newcommand{\PaGtXY}[1][X,Y]{\Pa_{G}^{#1}(t)}

\newcommand{\bPaGstXY}[1][X,Y]{\overline{\Pa}_{G'}^{#1}(<t)}
\newcommand{\bPaGtXY}[1][X,Y]{\overline{\Pa}_{G'}^{#1}(t)}
\newcommand{\TODO}[1]{\textcolor{red}{**TODO**-{#1}}}

\def\historyGX{{\vh_G^X}}
\def\historyGY{{\vh_G^Y}}

\def\bhistoryGX{{\bar{\vh}_{G'}^X}}
\def\bhistoryGY{{\bar{\vh}_{G'}^Y}}

\def\bnu{{\bar{\nu}}}

\def\alphat{{\alpha_t}}
\def\betat{{\beta_t}}

\newenvironment{sproof}{%
  \renewcommand{\proofname}{Sketch of proof}\proof}{\endproof}
  
  
\usepackage{wrapfig}
\usepackage{alphalph}
\usepackage[shortlabels]{enumitem}
% \newenvironment{sproof}{%
%   \renewcommand{\proofname}{Sketch of proof}\proof}{\endproof}
% \makeatletter
% \newtheorem*{rep@theorem}{\rep@title}
% \newcommand{\newreptheorem}[2]{%
% \newenvironment{rep#1}[1]{%
%  \def\rep@title{#2 \ref{##1}}%
%  \begin{rep@theorem}}%
%  {\end{rep@theorem}}}
% \makeatother
%%%%% NEW MATH DEFINITIONS %%%%%
\newtheorem{property}{Property}
\newtheorem{definition}{Definition}
\newtheorem{theorem}{Theorem}
\newtheorem{lemma}{Lemma}
\newtheorem{corollary}{Corollary}
\DeclarePairedDelimiter\abs{\lvert}{\rvert}
\DeclarePairedDelimiter\norm{\lVert}{\rVert}
\makeatletter
\let\oldabs\abs
\def\abs{\@ifstar{\oldabs}{\oldabs*}}
\let\oldnorm\norm
\def\norm{\@ifstar{\oldnorm}{\oldnorm*}}
\makeatother

% Mark sections of captions for referring to divisions of figures
\newcommand{\figleft}{{\em (Left) }}
\newcommand{\figcenter}{{\em (Center) }}
\newcommand{\figright}{{\em (Right)}}
\newcommand{\figtop}{{\em (Top) }}
\newcommand{\figbottom}{{\em (Bottom) }}
\newcommand{\captiona}{{\em (a) }}
\newcommand{\captionb}{{\em (b) }}
\newcommand{\captionc}{{\em (c) }}
\newcommand{\captiond}{{\em (d) }}

% Highlight a newly defined term
\newcommand{\newterm}[1]{{\bf #1}}


\def\figref#1{figure~\ref{#1}}
\def\Figref#1{Figure~\ref{#1}}
\def\twofigref#1#2{figures \ref{#1} and \ref{#2}}
\def\quadfigref#1#2#3#4{figures \ref{#1}, \ref{#2}, \ref{#3} and \ref{#4}}
\def\secref#1{section~\ref{#1}}
\def\Secref#1{Section~\ref{#1}}
\def\twosecrefs#1#2{sections \ref{#1} and \ref{#2}}
\def\secrefs#1#2#3{sections \ref{#1}, \ref{#2} and \ref{#3}}
\def\eqref#1{equation~\ref{#1}}
\def\Eqref#1{Equation~\ref{#1}}
% A raw reference to an equation---avoid using if possible
\def\plaineqref#1{\ref{#1}}
% Reference to a chapter, lower-case.
\def\chapref#1{chapter~\ref{#1}}
% Reference to an equation, upper case.
\def\Chapref#1{Chapter~\ref{#1}}
% Reference to a range of chapters
\def\rangechapref#1#2{chapters\ref{#1}--\ref{#2}}
% Reference to an algorithm, lower-case.
\def\algref#1{algorithm~\ref{#1}}
% Reference to an algorithm, upper case.
\def\Algref#1{Algorithm~\ref{#1}}
\def\twoalgref#1#2{algorithms \ref{#1} and \ref{#2}}
\def\Twoalgref#1#2{Algorithms \ref{#1} and \ref{#2}}
% Reference to a part, lower case
\def\partref#1{part~\ref{#1}}
% Reference to a part, upper case
\def\Partref#1{Part~\ref{#1}}
\def\twopartref#1#2{parts \ref{#1} and \ref{#2}}

% Random variables
\def\reta{{\textnormal{$\eta$}}}
\def\ra{{\textnormal{a}}}

% Random vectors
\def\rvepsilon{{\mathbf{\epsilon}}}
\def\rvtheta{{\mathbf{\theta}}}
\def\rva{{\mathbf{a}}}

% Elements of random vectors
\def\erva{{\textnormal{a}}}
\def\ervb{{\textnormal{b}}}

% Random matrices
\def\rmA{{\mathbf{A}}}
\def\rmB{{\mathbf{B}}}

% Elements of random matrices
\def\ermA{{\textnormal{A}}}
\def\ermB{{\textnormal{B}}}

\def\fvec{{\mathbf{f}}}
\def\bff{{\mathbf{f}}}
\def\bfg{{\mathbf{g}}}
% Vectors
\def\vzero{{\bm{0}}}
\def\vone{{\bm{1}}}
\def\vmu{{\bm{\mu}}}
\def\vtheta{{\bm{\theta}}}
\def\va{{\bm{a}}}
\def\vb{{\bm{b}}}
\def\vc{{\bm{c}}}
\def\vd{{\bm{d}}}
\def\ve{{\bm{e}}}
\def\vf{{\bm{f}}}
\def\vg{{\bm{g}}}
\def\vh{{\bm{h}}}
\def\vi{{\bm{i}}}
\def\vj{{\bm{j}}}
\def\vk{{\bm{k}}}
\def\vl{{\bm{l}}}
\def\vm{{\bm{m}}}
\def\vn{{\bm{n}}}
\def\vo{{\bm{o}}}
\def\vp{{\bm{p}}}
\def\vq{{\bm{q}}}
\def\vr{{\bm{r}}}
\def\vs{{\bm{s}}}
\def\vt{{\bm{t}}}
\def\vu{{\bm{u}}}
\def\vv{{\bm{v}}}
\def\vw{{\bm{w}}}
\def\vx{{\bm{x}}}
\def\vy{{\bm{y}}}
\def\vz{{\bm{z}}}

% Matrix
\def\mA{{\bm{A}}}

% Tensor
\DeclareMathAlphabet{\mathsfit}{\encodingdefault}{\sfdefault}{m}{sl}
\SetMathAlphabet{\mathsfit}{bold}{\encodingdefault}{\sfdefault}{bx}{n}
\newcommand{\tens}[1]{\bm{\mathsfit{#1}}}
\def\tA{{\tens{A}}}
\def\tB{{\tens{B}}}
\def\tC{{\tens{C}}}
\def\tD{{\tens{D}}}
\def\tE{{\tens{E}}}
\def\tF{{\tens{F}}}
\def\tG{{\tens{G}}}
\def\tH{{\tens{H}}}
\def\tI{{\tens{I}}}
\def\tJ{{\tens{J}}}
\def\tK{{\tens{K}}}
\def\tL{{\tens{L}}}
\def\tM{{\tens{M}}}
\def\tN{{\tens{N}}}
\def\tO{{\tens{O}}}
\def\tP{{\tens{P}}}
\def\tQ{{\tens{Q}}}
\def\tR{{\tens{R}}}
\def\tS{{\tens{S}}}
\def\tT{{\tens{T}}}
\def\tU{{\tens{U}}}
\def\tV{{\tens{V}}}
\def\tW{{\tens{W}}}
\def\tX{{\tens{X}}}
\def\tY{{\tens{Y}}}
\def\tZ{{\tens{Z}}}


% Graph
\def\gA{{\mathcal{A}}}
\def\gB{{\mathcal{B}}}
\def\gC{{\mathcal{C}}}
\def\dataset{{\mathcal{D}}}
\def\gE{{\mathcal{E}}}
\def\gF{{\mathcal{F}}}
\def\fourier{{\mathcal{F}}}
\def\gG{{\mathcal{G}}}
\def\gH{{\mathcal{H}}}
\def\gI{{\mathcal{I}}}
\def\gJ{{\mathcal{J}}}
\def\gK{{\mathcal{K}}}
\def\gL{{\mathcal{L}}}
\def\loss{{\mathcal{L}}}
\def\gM{{\mathcal{M}}}
\def\gN{{\mathcal{N}}}
\def\normal{{\mathcal{N}}}
\def\gaussian{{\mathcal{N}}}
\def\gO{{\mathcal{O}}}
\def\gP{{\mathcal{P}}}
\def\gQ{{\mathcal{Q}}}
\def\gR{{\mathcal{R}}}
\def\gS{{\mathcal{S}}}
\def\gT{{\mathcal{T}}}
\def\gU{{\mathcal{U}}}
\def\uniform{{\mathcal{U}}}
\def\gV{{\mathcal{V}}}
\def\gW{{\mathcal{W}}}
\def\gX{{\mathcal{X}}}
\def\gY{{\mathcal{Y}}}
\def\gZ{{\mathcal{Z}}}

\def\algebra{{\mathscr{A}}}
\def\borel{{\mathscr{B}}}
\def\manifold{{\mathscr{M}}}

% Sets
\def\sA{{\mathbb{A}}}
\def\sB{{\mathbb{B}}}
\def\complex{{\mathbb{C}}}
\def\sD{{\mathbb{D}}}
\def\expectation{{\mathbb{E}}}
\newcommand{\E}{\mathbb{E}}
\def\sF{{\mathbb{F}}}
\def\sG{{\mathbb{G}}}
\def\sH{{\mathbb{H}}}
\def\sI{{\mathbb{I}}}
\def\sJ{{\mathbb{J}}}
\def\sK{{\mathbb{K}}}
\def\sL{{\mathbb{L}}}
\def\sM{{\mathbb{M}}}
\def\natural{{\mathbb{N}}}
\def\sO{{\mathbb{O}}}
\def\sP{{\mathbb{P}}}
\def\rational{{\mathbb{Q}}}
\def\real{{\mathbb{R}}}
\newcommand{\R}{\mathbb{R}}
\def\sS{{\mathbb{S}}}
\def\sphere{{\mathbb{S}}}
\def\sT{{\mathbb{T}}}
\def\sU{{\mathbb{U}}}
\def\sV{{\mathbb{V}}}
\def\sW{{\mathbb{W}}}
\def\sX{{\mathbb{X}}}
\def\sY{{\mathbb{Y}}}
\def\integer{{\mathbb{Z}}}
\def\indicator{{\mathbbm{1}}}

% Entries of a matrix
\def\emLambda{{\Lambda}}
\def\emA{{A}}
\def\emB{{B}}
\def\emC{{C}}
\def\emD{{D}}
\def\emE{{E}}
\def\emF{{F}}
\def\emG{{G}}
\def\emH{{H}}
\def\emI{{I}}
\def\emJ{{J}}
\def\emK{{K}}
\def\emL{{L}}
\def\emM{{M}}
\def\emN{{N}}
\def\emO{{O}}
\def\emP{{P}}
\def\emQ{{Q}}
\def\emR{{R}}
\def\emS{{S}}
\def\emT{{T}}
\def\emU{{U}}
\def\emV{{V}}
\def\emW{{W}}
\def\emX{{X}}
\def\emY{{Y}}
\def\emZ{{Z}}
\def\emSigma{{\Sigma}}

% entries of a tensor
% Same font as tensor, without \bm wrapper
\newcommand{\etens}[1]{\mathsfit{#1}}
\def\etLambda{{\etens{\Lambda}}}
\def\etA{{\etens{A}}}
\def\etB{{\etens{B}}}
\def\etC{{\etens{C}}}
\def\etD{{\etens{D}}}
\def\etE{{\etens{E}}}
\def\etF{{\etens{F}}}
\def\etG{{\etens{G}}}
\def\etH{{\etens{H}}}
\def\etI{{\etens{I}}}
\def\etJ{{\etens{J}}}
\def\etK{{\etens{K}}}
\def\etL{{\etens{L}}}
\def\etM{{\etens{M}}}
\def\etN{{\etens{N}}}
\def\etO{{\etens{O}}}
\def\etP{{\etens{P}}}
\def\etQ{{\etens{Q}}}
\def\etR{{\etens{R}}}
\def\etS{{\etens{S}}}
\def\etT{{\etens{T}}}
\def\etU{{\etens{U}}}
\def\etV{{\etens{V}}}
\def\etW{{\etens{W}}}
\def\etX{{\etens{X}}}
\def\etY{{\etens{Y}}}
\def\etZ{{\etens{Z}}}

\def\ceil#1{\lceil #1 \rceil}
\def\floor#1{\lfloor #1 \rfloor}
\def\eps{{\epsilon}}

\newcommand{\pder}[1]{\frac{\partial}{\partial #1}}

\newcommand{\half}{\frac{1}{2}}
\newcommand{\limNinf}{\lim_{N \to \infty}}
\newcommand{\limTzero}{\lim_{\tau \to 0}}


\newcommand{\cmark}{\ding{51}}
\newcommand{\xmark}{\ding{55}}

\newcommand{\layer}{\mathcal{H}}
\newcommand{\defeq}{\triangleq}
%\newcommand{\defeq}{vcentcolon=}
\newcommand{\domain}{\Omega}
\newcommand{\grad}{\nabla}

\newcommand{\cin}{c_{\rm{in}}}
\newcommand{\cout}{c_{\rm{out}}}
\newcommand{\intdomain}{\int_{\domain}}
\newcommand{\network}{\gT}
\newcommand{\subnet}{\gK}
\newcommand{\map}{\gR} %\gR

\newcommand{\innerproduct}[2]{\langle #1, #2 \rangle}
\newcommand{\mcsum}[1][j]{\frac{1}{N}\sum_{#1=1}^N}

\newcommand{\inrspace}[1][c]{\gF_{#1}}

\DeclareMathOperator*{\argmax}{arg\,max}
\DeclareMathOperator*{\argmin}{arg\,min}

\let\ab\allowbreak


\title{BayesDAG:  Gradient-Based Posterior Inference for Causal Discovery}

\makeatletter
\newcommand{\printfnsymbol}[1]{%
  \textsuperscript{\@fnsymbol{#1}}%
}
\makeatother
% The \author macro works with any number of authors. There are two commands
% used to separate the names and addresses of multiple authors: \And and \AND.
%
% Using \And between authors leaves it to LaTeX to determine where to break the
% lines. Using \AND forces a line break at that point. So, if LaTeX puts 3 of 4
% authors names on the first line, and the last on the second line, try using
% \AND instead of \And before the third author name.

\author{%
\textbf{Yashas Annadani}$^{\printfnsymbol{2}\thanks{Equal contribution. \printfnsymbol{2} Work done during internship at Microsoft Research. Correspondence to \texttt{wenbogong@microsoft.com}. Code: \url{https://github.com/microsoft/Project-BayesDAG}}\,\,\,1,3,4}$
\quad
\textbf{Nick Pawlowski}$^{2}$
\quad
\textbf{Joel Jennings}$^{2}$
\quad
\textbf{Stefan Bauer}$^{3, 4}$
\\
\textbf{Cheng Zhang}$^{2}$
\quad
\textbf{Wenbo Gong}$^{\printfnsymbol{1}2}$
\\
$^1$ KTH Royal Institute of Technology, Stockholm \quad $^2$ Microsoft Research\\ $^3$ Helmholtz AI, Munich $^4$ TU Munich\\
}

\begin{document}


\maketitle


\begin{abstract}
\looseness=-1 Bayesian causal discovery aims to infer the posterior distribution over causal models from observed data, quantifying epistemic uncertainty and benefiting downstream tasks. However, computational challenges arise due to joint inference over combinatorial space of Directed Acyclic Graphs (DAGs) and nonlinear functions. Despite recent progress towards efficient posterior inference over DAGs,  existing methods are either limited to variational inference on node permutation matrices for linear causal models, leading to compromised inference accuracy, or continuous relaxation of adjacency matrices constrained by a DAG regularizer, which cannot ensure resulting graphs are DAGs. In this work, we introduce a scalable Bayesian causal discovery framework based on a combination of stochastic gradient Markov Chain Monte Carlo (SG-MCMC) and Variational Inference (VI) that overcomes these limitations. Our approach directly samples DAGs from the posterior without requiring any DAG regularization, simultaneously draws function parameter samples and is applicable to both linear and nonlinear causal models. To enable our approach, we derive a novel equivalence to the permutation-based DAG learning, which opens up possibilities of using any relaxed gradient estimator defined over permutations. To our knowledge, this is the first framework applying gradient-based MCMC sampling for causal discovery. Empirical evaluation on synthetic and real-world datasets demonstrate our approach's effectiveness compared to state-of-the-art baselines.
\end{abstract}

The problem of the presence or absence of phase transition is central in statistical mechanics. To prove the existence of phase transition, the standard idea is to define a notion of contour and use \textit{Peierls' argument} \cite{Peierls.1936}. In the usual Ising model \cite{Ising_25}, particles of the system interact only with their nearest-neighbors. On ferromagnetic long-range Ising models \cite{Anderson_Yuval_69}, there is interaction between each pair of spins in the lattice. The Hamiltonian of the model is given formally by
\begin{equation*}
    H(\sigma) = - \sum_{x,y\in \Z^d}J_{xy}\sigma_x\sigma_y,
\end{equation*}
where $J_{xy}=J|x-y|^{-\alpha}$, $J>0$, $\alpha > d$. It is well-known that the Peierls' argument in dimension 2 implies phase transition for Ising models with nearest-neighbors or long-range interactions when $d\geq 2$, using correlation inequalities. For the unidimensional lattice, it was known that short-range models do not present phase transition. In the long-range case, a different behavior was expected depending on the exponent $\alpha$ (see \cite{Kac_Thompson_69}), but the problem was challenging since contours were first created as multidimensional objects.

In dimension $d=1$, phase transition was proved first in 1969 by Dyson \cite{Dyson.69}, for $\alpha \in (1,2)$, by proving phase transition in an auxiliary model and then using correlation inequalities. In 1982, Fr{\"o}hlich and Spencer \cite{Frohlich.Spencer.82} introduced a notion of one-dimensional contours and then applied the Peierls' argument to show phase transition for the critical value $\alpha = 2$. These contours were inspired by the multiscale techniques previously introduced to study the Berezinskii-Kosterlitz-Thouless transition in two-dimensional continuous spin systems \cite{FS81}. Later, Cassandro, Ferrari, Merola and Presutti  \cite{Cassandro.05} extended the contour argument previously available for $\alpha=2$ to exponents $\alpha\in (3-\frac{\ln 3}{\ln 2}, 2)$, with the additional restriction that the nearest-neighbor interaction is strong, i.e.,  ${J(1)\gg 1}$; this restriction was removed for a subclass of interactions in \cite{Bissacot.Endo.18}. Further results were obtained using contour arguments, such as the decay of correlations, cluster expansions, phase transition with random interactions, etc; some references with these results are \cite{ Cassandro.Merola.Picco.17, Cassandro.Merola.Picco.Rozikov.14, Imbrie.82, Imbrie.Newman.88, Johansson.91}. 

In the multidimensional setting ($d\geq 2$), Ginibre, Grossmann, and Ruelle, in \cite{Ginibre.Grossmann.Ruelle.66}, proved the phase transition for $\alpha > d+1$, using an enhanced version of Peierls' argument and the usual contours. Park proposed a different notion of contour for long-range systems in \cite{Park.88.I, Park.88.II}, extending the Pirogov-Sinai theory available for short-range interactions assuming $\alpha > 3d+1$, although he can also consider Potts models with his methods. Some results in the literature suggest that truly long-range effects appear only when $d < \alpha \leq d+1$, see for instance, \cite{Biskup_Chayes_Kivelson_07}. Recently, Affonso, Bissacot, Endo and Handa \cite{Affonso.2021}, inspired by the ideas from Fr{\"o}hlich and Spencer in \cite{FS81, Frohlich.Spencer.82}, introduced a version of multiscale multidimensional contour and proved phase transition by a contour argument in the whole region $\alpha > d$. They can consider long-range Ising models with deterministic decaying fields, first introduced in the context of nearest-neighbor interactions in \cite{Bissacot_Cioletti_10}. For these models, the lack of analyticity of the free energy does not imply phase transition since these models have the same free energy as the models with zero field. It is expected that fields decaying slowly imply uniqueness. In this setting, a contour argument is useful for proofs of phase transitions as well for uniqueness, some papers with models with deterministic decaying fields are \cite{Aoun_Ott_Velenik_23, Bissacot_Cass_Cio_Pres_15, Bissacot.Endo.18, Cioletti_Vila_2016}.

The Random Field Ising model (RFIM) \cite{Imry.Ma.75} is the nearest-neighbor Ising model with an additional external field acting on each site $(h_x)_{x\in\Z^d}$ that is a family of i.i.d. Gaussian random variable with mean 0 and variance 1. Formally, the Hamiltonian of the model is given by
\begin{equation*}
    H(\sigma) = - \sum_{\substack{x,y\in \Z^d \\|x-y|=1}}J\sigma_x\sigma_y  - \varepsilon\sum_{x\in\Z^d}h_x\sigma_x,
\end{equation*}
where $J>0$, $\varepsilon>0$, $\alpha > d$ and $d \geq 1$. A detailed account of the history of the phase transition problem for this model, as well as detailed proofs, was given in \cite{Bovier.06}. Here we present a brief overview.

During the 1980s, the question of the specific dimension where phase transition for the RFIM should happen attracted much attention and was a topic of heated debate. Two convincing arguments were dividing the physics community. One of them, due to Imry and Ma \cite{Imry.Ma.75}, was a non-rigorous application of the Peierls' argument together with the use of the isoperimetric inequality. The key idea of Peierls' argument is to define a notion of contour and calculate the energy cost of "erasing" each contour, i.e., the energy cost of flipping all spins inside the contour. When there is no external field, that energy necessary to flip the spins in a region $A\subset \Z^d$ is of the order of the boundary $|\partial A|$. When we add an external field, we get an extra cost depending on this field. Imry and Ma argued that this cost should be approximately $\sqrt{|A|}$, which is smaller than $|\partial A|$ for all regions only when $d\geq 3$, so this should be the region where phase transition occurs. The other argument, due to Parisi and Sourlas \cite{Parisi.Sourlas.79}, based on dimensional reduction, predicted that the $d$-dimensional RFIM would behave like the $d-2$-dimensional nearest-neighbor Ising model, therefore presenting phase transition only when $d\geq 4$. 

The question was settled by two celebrated papers showing that Imry and Ma's prediction was correct. First, in 1988, Bricmont and Kupiainen \cite{Bricmont.Kupiainen.88} showed that there is phase transition almost surely in $d\geq3$, for low temperatures and variance $\varepsilon$ small enough. Their proof uses a rigorous renormalization group analysis for the short-range case and it is considered involved. Still, they claimed that the result works for any model with a suitable contour representation and centered sub-gaussian external field. Later on, Aizenman and Wehr \cite{Aizenman.Wehr.90} proved uniqueness for $d\leq 2$. For detailed proofs of these results, we refer the reader to \cite{Bovier.06} (see also \cite{Berretti.85, Camia.18, Frohlich.Imbre.84,  Klein.Masooman.97} for more uniqueness results). 

Recently, Ding and Zhuang, see \cite{Ding2021}, provided a simpler proof of the phase transition, not using RGM. And in  \cite{Ding.Liu.Xia.22}, Ding, Liu and Xia proved that if $\beta_c(d)$ is the critical inverse of the temperature of the Ising model with no field, for all $\beta>\beta_c(d)$ there exists a critical value $\varepsilon_0(d, \beta)$ such that the RFIM with $\varepsilon \leq \varepsilon_0$ presents phase transition. 

In the present paper, we are considering a long-range Ising model with a random field, whose Hamiltonian is given formally by
\begin{equation*}
    H(\sigma) = - \sum_{x,y\in \Z^d}J_{xy}\sigma_x\sigma_y - \varepsilon\sum_{x\in\Z^d}h_x\sigma_x,
\end{equation*}
where $J_{xy}=J|x-y|^{-\alpha}$, $J, \varepsilon>0$, $\alpha > d$ and $h_x\in\mathbb{R}$, $d\geq 3$.
Until now, the only known result in the long-range setting is for the one-dimensional long-range Ising model with a random field, by Cassandro, Orlandi, and Picco \cite{Cassandro.Picco.09}. They used the contours of \cite{Cassandro.05} to show the phase transition for the model when $\alpha\in (3-\frac{\ln 3}{\ln 2}, \frac{3}{2})$, under the assumption $J(1) \gg 1$. We stress that, as remarked by Aizenman, Greenblatt, and Lebowitz \cite{Aizenman_Greenblatt_Lebowitz_2012}, although their argument does not work for the whole region for the exponent $\alpha$, the phase transition holds for values close to the critical value $\alpha=3/2$, since by the Aizenman-Wehr theorem we know that there is uniqueness for $\alpha>3/2$.

The argument from Ding and Zhuang in \cite{Ding2021}, for $d\geq3$, involves controlling the probability of a bad event, which is closely related to controlling the quantity $$\sup_{\substack{0\in A\subset\Z^d \\ A \text{ connected }}}\frac{\sum_{x\in A}h_x}{|\partial A|},$$ known as the greedy animal lattice normalized by the boundary. The greedy animal lattice normalized by the size, instead of the boundary, was extensively studied for general distributions of $(h_x)_{x\in\Z^d}$, see \cite{Cox_Gandolfi_Griffin_Kesten_93, Gandolfi_Kesten_94, Hammond_06, Martin_02}. When we normalize by the boundary, an argument by Fisher, Fr\"{o}hlich and Spencer \cite{FFS84} shows that the expected value of the greedy animal lattice is constant. In dimension $d=2$, the expected value is not finite, see \cite{Ding.Wirth.20}. The supremum is taken over connected regions containing the origin since the interiors of the usual Peierls contours are of this form.


For the long-range model, the interior of contours is not necessarily connected. In fact, long-range contours may have considerably large diameters with respect to their size, so their interiors can be very sparse. To avoid this, we define contours, strongly inspired by the $(M,a,r)$-partition in \cite{Affonso.2021}, using a multiscaled procedure that assures that the contours have no cluster with small density.  With them, we generalize the arguments by Fisher-Fr\"{o}hlich-Spencer \cite{FFS84}, and prove that the expected value of the greedy animal lattice is constant, even considering regions not necessarily connected in the supremum. Then, we prove the phase transition for $d\geq 3$. The main result of this paper is the following.
\begin{theorem*}Given $d\geq 3$, $\alpha>d$, there exists $\beta_c\coloneqq\beta(d, \alpha)$ and $\varepsilon_c\coloneqq\varepsilon(d, \alpha)$ such that, for $\beta >\beta_c$ and $\varepsilon\leq \varepsilon_c$, the extremal Gibbs measures $\mu_{\beta, \varepsilon}^+$ and $\mu_{\beta, \varepsilon}^-$ are distinct, that is, $\mu_{\beta, \varepsilon}^+ \neq \mu_{\beta, \varepsilon}^-$ $\mathbb{P}$-almost surely. Therefore the long-range random field Ising model presents phase transition.
\end{theorem*}

This paper is divided as follows. In Section 2, we define the model and the contours, and suitable generalizations to the constructions in \cite{Affonso.2021} are introduced.  In Section 3, we define two bad events of the external field and prove that they occur with a small probability.  In Section 4, we present the proof of the phase transition.
\section{Background}
\label{sec: Prerequisite}
% In this section, we provide a brief overview of the necessary prerequisite knowledge. Specifically, we focus on fundamental terminologies, including the causal graph and structural causal model (SCM), the Bayesian causal discovery formulation \cite{friedman2003being}, and a gentle introduction to the NoCurl \cite{yu2021dags} characterization of the causal graph.


\paragraph{Causal Graph and Structural Causal Model}
Consider a data generation process with $d$ variables $\mX\in\sR^{d}$. The causal relationships among these variables is  represented by a Structural Causal Model (SCM) which consists of a set of structural equations~\cite{peters2017elements} where each variable $X_i$ is a function of its direct causes $\mX_{\Pa^i}$ and an exogenous noise variable $\epsilon_i$ with distribution $P_{\epsilon_i}$: 
\begin{equation}
X_i\coloneqq f_i(\mX_{\Pa^i},\epsilon_i)
\label{eq: CM}
\end{equation}
These equations induce a 
causal graph $\mG=(\mV,\mE)$, comprising a node set $\mV$ with $\vert\mV\vert=d$ indexing the variables $\mX$ and a directed edge set $\mE$. If a directed edge $e_{ij}\in\mE$ exists between a node pair $v_i,v_j\in\mV$ (i.e., $v_i\rightarrow v_j$), we say that $X_i$ causes $X_j$ or $X_i$ is the parent of $X_j$. We use the binary adjacency matrix $\mG\in\binaryset$ to represent the causal graph, where the entry $G_{ij}=1$ denotes $v_i \rightarrow v_j$. %Given $\mG$, the functional relationship between variables can be described using a structural causal model (SCM) defined as
A standard assumption in causality is that the structural assignments are acyclic and the induced causal graph is a DAG~\cite{bongers2021foundations, pearl2009causality}, which we adopt in this work.
We further assume that the SCM is causally sufficient i.e. all variables are measurable and exogenous noise variables $\epsilon_i$ are mutually independent. Throughout this work, we consider a special form of SCM called Gaussian additive noise model (ANM):
\begin{equation}
X_i\coloneqq f_i(\mX_{\Pa^i}) + \epsilon_i~~~~~~\text{where}~~~\epsilon_i\sim\mathcal{N}(0,\sigma^2_i)
\label{eq: ANM}
\end{equation}
% We refer to \cref{subsect: model formulation} for a detailed model formulation of \ModelName{}.
If the functions are not linear or constant in any of its arguments, the Gaussian ANM is structurally identifiable~\cite{hoyer2008nonlinear,peters2014causal}.

\paragraph{Bayesian Causal Discovery}
Given a dataset $\mD=\{\vx^{(1)},\ldots,\vx^{(N)}\}$ with i.i.d observations, underlying graph $\mG$ and SCM parameters $\mTheta$%\footnote{In this work, $\mTheta$ corresponds to the parameters of functions $f_i$ and the variance of Gaussian exogenous noise variables $P_{\epsilon_i}=\mathcal{N}(0,\sigma^2_i)$.}
, they induce a unique joint distribution $p(\mD,\mTheta, \mG) = p(\mD\vert \mG,\mTheta)p(\mG,\mTheta)$ with the prior $p(\mG,\mTheta)$ and likelihood $p(\mD\vert \mG,\mTheta)$ \cite{friedman2003being}. Under finite data and/or limited identifiability of SCM (e.g upto MEC), it is desirable to have accurate uncertainty estimation for downstream decision making rather than inferring a single SCM and its graph (for e.g. with a maximum likelihood estimate). Bayesian causal discovery therefore aims to infer the posterior $p(\mG,\mTheta\vert \mD)=p(\mD,\mTheta, \mG)/p(\mD)$. However, this posterior is intractable due to the super-exponential growth of the possible DAGs $\mG$~\cite{robinson1973counting} and continuously valued model parameters $\mTheta$ in nonlinear functions. VI \cite{zhang2018advances} or SG-MCMC \cite{gong2022advances,ma2015complete} are two types of methods developed to tackle general Bayesian inference problems, but adaptations are required for Bayesian causal discovery. 

\paragraph{NoCurl Characterization}
Inferring causal graphs is challenging due to the DAG constraint. Previous works \cite{geffner2022deep, gong2022rhino,lachapelle2019gradient,lorch2021dibs,yu2019dag} directly infer adjacency matrix with the DAG regularizer~\cite{zheng2018dags}. However, it requires an annealing schedule, resulting in slow convergence, and no guarantees on generating DAGs. Recently, \cite{yu2021dags} introduced NoCurl, a novel characterization of the \textbf{weighted DAG} space. They define a potential $p_i\in \sR$ for each node $i$, grouped as potential vector $\vp\in\sR^d$ . Further, a gradient operator on $\vp$ mapping it to a skew-symmetric matrix is introduced:
\begin{equation}
    (\grad \vp)(i,j) = p_i-p_j
    \label{eq: grad operator}
\end{equation}
Based on the above operation, a mapping that directly maps from the augmented space $(\mW,\vp)$ to the DAG space $\gamma(\cdot,\cdot):\sR^{d\times d}\times \sR^{d}\rightarrow \sR^{d\times d}$ was proposed:
\begin{equation}
\gamma(\mW,\vp) = \mW\odot \relu(\grad \vp)
\label{eq: NoCurl mapping}
\end{equation}
where $\relu(\cdot)$ is the ReLU activation function and $\mW$ is a skew-symmetric \textbf{continuously weighted} matrix. This formulation is complete (Theorem 2.1 in \cite{yu2021dags}), as any continuously weighted DAG can be represented by a $(\mW,\vp)$ pair and vice versa.
NoCurl translates the learning of a single weighted DAG to a corresponding $(\mW,\vp)$ pair. However, direct gradient-based optimization is challenging due to a highly non-convex loss landscape, which leads to the reported failure in \cite{yu2021dags}. 

Although NoCurl appears suitable for our purpose, the failure in directly learning suggests non-trivial optimizations. We hypothesize that this arises from the continuously weighted matrix $\mW$. In the following, we introduce our proposed parametrization inspired by NoCurl to characterize the \textbf{binary} DAG adjacency matrix.





\section{Methodology}
\label{sec:method}
% Figure environment removed


%\vspace*{0.2cm}\noindent\textbf{Problem formulation.} 
% Let $O$ be the set of Objects,  $S$ be the set of
%  States and $\textit{I}$ the set of
%   Images which consists of the disjoint sets ${I^{S}}$ and ${I^{U}}$ that are used during the training and testing phase respectively. 
%    Each image $i_{k} \in \textit{I}$  contains an object $o_{i} \in O$ which  is situated in a state $s_{j} \in S$. The OSC task deals with the yielding  of a  predicted state label $sp_{j} \in S$ for an image $i_{k} \in {I^U}$ that has been given as an input. In the zero-shot variation of OSC, ${S^{S}} 
%   \not \supseteq {S^{U}}$, i.e. some of the states contained in the testing images do not appear in the training images. 
% Let $O$ denote a set of objects, 
% \textcolor{red}{$S^S$ as the set of known object states found in the training images, $S^U$ as the test, yet unknown, object state labels} and $I$ the set of images, which is partitioned into the training set $I^T$ and the test set $I^U$. 
% % Each image $i \in I$ contains an object $o \in O$ in a state $s \in S$. 
% \textcolor{red}{Each image $i \in I^T$ contains an object $o \in O$ in a state $s \in S^S$, while an image $i \in I^U$ contains an object $o \in O$ in any state $s \in S = \{S^S \cup S^U\}$. }
% The goal of OSC is to predict the state $s \in S$, given the object $o$ in $i \in I^U$. In the zero-shot variation of OSC, the set of states observed in the test images $S^U$ is not a subset of the set of states observed in the training images $S^S$, i.e., there exists some states in the test image set that do not appear in the training set. Furthermore, the task should be addressed in an object-agnostic manner, i.e. no information concerning the object classes is to be utilized explicitly.  However,  although the set of object classes does not directly affect the task of OaSC, its size is proportional to the complexity of the problem. 
% The workflow of the proposed method is shown in \autoref{fig:pipeline}.

Let $O$ denote a set of objects, $S$ denote the set of states and $I$ denote the set of images, which is partitioned into the training set $I^T$ and the testing set $I^U$. Each image $i \in I$ contains an object $o \in O$ in a state $s \in S$. 
The goal of OSC is to predict the state $s \in S$, given the object $o$ in $i \in I^U$. In the zero-shot variation of OSC, the set of states observed in the test images $S^U$ is not a subset of the set of states observed in the training images $S^S$, i.e., there exists some states in the test image set that do not appear in the training set. Furthermore, the task should be addressed in an object-agnostic manner, i.e. no information concerning the object classes is to be utilized explicitly.  However,  although the set of object classes does not directly affect the task of OaSC, its size is proportional to the complexity of the problem. 
The workflow of the proposed method is shown in \autoref{fig:pipeline}.




% and will be analyzed in the following sections.

% \vspace*{0.2cm}\noindent\textbf{Approach.}

\subsection{Overview}
% Our method is inspired by works that address the problem of zero-shot object classification \cite{}. The main idea behind this line of work is that the necessary information for the classification of the unseen classes can be found in a Knowledge Graph (KG) if processed appropriately by a Graph Neural Network (GNN). Obviously, the most crucial component of this approach lies in the combination of the visual information stemming from the training images and referring to the seen classes with the semantic information stemming from the KG  and referring to the unseen classes.

We are inspired by prior research on zero-shot object classification and leverage the potential of KGs and GNNs to classify previously unseen objects~\cite{Kampffmeyer2019,nayak:tmlr22}. 
The core idea is that semantic information that is stored in the KG can be used by GNNs to learn graph embeddings that can be utilized jointly with visual information extracted from training images. 
This enables the model to generalize to new object classes by leveraging the semantic and contextual information encoded in the graph embeddings of the KG.

% More in detail, the GNN architecture is adopted to the architecture of the Classifier  that is used for the training on seen classes, the GNN last layer has the same size  with the Classifier last layer. This way the GNN can produce semantic embedding features that correspond to all the classes, both seen and unseen, that will be encountered during the inference. These embedding features  replace the last layer of the Classifier. Holding this layer fixed, the body of the Classifier is then fine-tuned with the training images.

GNNs are designed to operate on graph-structured data, such as KGs~\cite{kipf2016semi,Monka2022}. KGs are typically represented as labeled multi-graphs, where nodes correspond to entities, and edges represent entity relationships. GNNs process this graph by iteratively aggregating information from neighboring nodes, using neural network-based operations.

At each iteration, a GNN receives a feature vector for each graph node, which is initially set to the node's embedding vector. Then, the GNN performs a message-passing step that aggregates information from neighboring nodes, based on the edge weights and the features of the nodes. This message-passing operation can be formulated as a neural network layer, which applies a learnable function to the features of the neighboring nodes and returns an aggregated message for each node. After the message-passing step, the GNN updates the node features by applying a learnable transformation that takes into account the original features of the node and the received messages from its neighbors. This updated feature vector is then passed to the next iteration of the message-passing step. The process continues until a fixed number of epochs or convergence.
%%%AAA: Endexetai na mas rethrown gia tis times aytwn twn parametrwn?
% KP edw anaferetai genika mia diadikasia GNN training. Na anaferoume edw times parametrwn h sto 4 - see implementation details ?

The proposed method leverages GNN training using a visual classifier that is trained on seen state classes as supervision. In particular, the last layer of the GNN is designed to have the same size as the last layer of the classifier. This enables the GNN to generate semantic embedding features that correspond to all classes, including both seen and unseen classes that will be encountered during inference. Subsequently, the semantic embedding features replace the last layer of the classifier while this layer is kept fixed. The body of the classifier is then fine-tuned with the training images to optimize the overall model for state recognition.

% \vspace*{0.2cm}\noindent\textbf{GNN Details.} 
Overall, we experimented with four different model architectures and opted for the Transformer Graph Convolutional network (Tr-GCN)~\cite{nayak:tmlr22}. Further details are provided in Section~\ref{sec:abl} and the supplementary material of this work. 
The Tr-GCN mode is capable of combining input sets non-linearly by utilizing multilayer perceptrons and self-attention. Tr-GCN refers to an inductive model that can learn node representations by aggregating local neighborhood features allowing the trained model to make predictions on new graph structures without retraining. 
We leverage the aforementioned property of the Tr-GCN to train a permutation invariant non-linear aggregator that captures the intricate structure of a common sense knowledge graph. 
% , rendering it well-suited for zero-shot learning. 
% It is worth noting that a similar network architecture has been effectively employed for zero-shot object classification~\cite{nayak:tmlr22}.

% A critical aspect of the proposed method involves calibrating the weights of the GNN in a manner that its predictions in the semantic space are useful for the classifier deployed in the visual space. To accomplish this, we adopt an approach based on prior research \cite{Kampffmeyer2019, Wang2018b, nayak:tmlr22} that involves learning the semantic class representations by minimizing the L2 distance between the learned class representations and the weights of a fully connected layer in a ResNet classifier pre-trained on the ILSVRC 2012 dataset \cite{russakovsky2015imagenet}. Once the class representations are learned, we fix them and fine-tune the ResNet backbone using the training images from the dataset.




% \vspace*{0.2cm}\noindent\textbf{Building of the KG.}
% The KG is created by the querying  of a common sense repository. The repositories that we are ConceptNet \cite{} and WordNet\cite{}. The procedure takes place as follows. Initially we create a set of nodes that correspond to the target stace classes. Subsequently, the repository is queried for each of these nodes and its neighbours in the repository of  added to the KG if  certain criteria are met (see ablation section for more details). This procedure is repeated for the newly added nodes and henceforth until a number of hops has been reached.  

\subsection{The proposed OaSC approach}
\label{sec:pipeline}
Overall, the proposed method consists of four stages, as shown in \autoref{fig:pipeline}: (1) construction of the KG, (2) GNN training and learning of semantic graph embeddings, (3) fine-tuning of the visual classifier and (4) deployment of the fine-tuned state classifier.

\vspace*{0.0cm}\noindent\textbf{Construction of the KG (Stage 1)}:
To create the KG, we query a common sense repository to compile a generic solution and to avoid the construction of a task-specific KG, tailored to the entities at hand and their relationships. First, a set of nodes that correspond to the words of the target state classes $S^U$ and $S^S$ is generated. Then, we query the repository for each of these nodes and add their neighbors in the KG, if they meet specific criteria (see also Section~\ref{sec:abl}). This process is repeated for the newly added nodes until a specified number of node hops is reached.

This technique for building a generic KG offers several advantages in comparison to other problem-specific approaches. First, it allows the same KG to be used for different variations of the task. It also enables transfer learning since KGs can be reused to tackle other related problems. Moreover, the construction of such a KG does not rely on expert knowledge. Besides, the structured representation of relationships between entities and concepts that KGs provide can be leveraged to generate robust embeddings for zero-shot learning.
% which is expensive and time-consuming.  
The trade-off is that such KGs are prone to noisy information in the used repositories. 

% In comparison, language models, such as BERT~\cite{devlin2018bert}, often rely on large amounts of unstructured text data to generate embeddings. While language models are highly effective at capturing semantic relationships between words and phrases, they can also be prone to create associations between concepts that are not actually related. This can lead to noisy or unreliable embeddings, which can in turn degrade the performance of zero-shot learning models. By contrast, the structured nature of KGs allows for more accurate and precise capture of relationships between entities and concepts, leading to more robust embeddings that can improve the accuracy and reliability of zero-shot learning models~\cite{brown2020language}.


\vspace*{0.0cm}\noindent\textbf{Computation of  Graph Embeddings (Stage 2)}:
% Given the KG constructed in Stage 1, a word features embedding matrix corresponding to the KG nodes is created by utilizing the pre-computed word features of GloVe~\cite{pennington2014glove}. 
% % Subsequently,  random walks are performed in the KG and a sample of neighbors for each node is obtained.
% By taking the word features embedding matrix, the KG topology, and a target node as inputs, the GNN estimates the node's embeddings: the features of the node and its neighbors are  aggregated and submitted to a series of convolutions and pooling operations before the  output is produced in the form of a feature vector, the length of which is tailored to be the same as the size dimension of the last layer of a ResNet-101 classifier. 
% This procedure is repeated for all KG nodes and results in the computation of the semantic embeddings for all target state classes with each embedding being a feature vector of length equal to 2048. By combining these embeddings for the \mathcal{d} target classes  a  $ d \times 2048$ features matrix is created which serves as the last layer of a CNN classifier that is utilized during Stages 3 and 4.
% which serves as the last layer of a CNN classifier that is utilized during Stages 3 and 4 .
% We employ an established approach~\cite{Kampffmeyer2019, Wang2018b} that involves training of a transformer-based Graph Convolutional Model using graph embeddings of a set of semantic entities acquired by a common sense repository by minimizing the L2 distance between the learned class representations and the weights of a fully connected layer in a ResNet classifier, pre-trained on the ILSVRC 2012 dataset~\cite{russakovsky2015imagenet}, ensuring that the semantic class representations are meaningfully embedded.
We employ an established approach~\cite{Kampffmeyer2019, Wang2018b} that involves the training of a transformer-based Graph Convolutional Network (GCN)
 \textcolor{black}{ that utilizes a KG as input  %Training is performed %using features of a set of semantic entities acquired by a common sense repository, \textcolor{red}{(e.g. the ConceptNet, CSKG, or other)}  
 and generates an embedding vector for each node of the  KG. %. For the production of the embeddings vectors the GCM employs a sequence of transformations to the semantic features that correspond to the concepts linked to each node.
This process defines pre-computed GloVe word, i.e. semantic features~\cite{pennington2014glove}, for the KG nodes with each node representing a concept class.
% To compute node embeddings, the GNN is applied to encode the KG topology and the word feature embedding matrix. 
The GNN  aggregates each node's and its neighbors' features through a sequence of convolutions and pooling operations. %This results in the generation of a feature vector having a length equal to the dimension of the last layer in a visual CNN-based classifier that is instantiated using a ResNet-101 model. 
%By pre-training the visual classifier in a set of target classes 
The visual classifier is pre-trained on a set of target classes and using the weights of its fully connected layer, the GCN learns to produce visual feature representations, i.e. visual embeddings,  corresponding to the concept classes of the KG`s nodes.}
\textcolor{black}{
Formally,  the training involves the minimization of the L2 distance   $\mathcal{L_G}$ between the generated visual embeddings and the ground truth visual embeddings stemming from the visual classifier.} 
\textcolor{black}{In notation, 
\begin{equation}
\mathcal{L_G} = \frac{1}{2N} \sum_{n \in N} \sum_{p \in P} (W_{n,p} - \tildea{W}_{n,p})^2,
 \end{equation}
where $\tildea{W} \in \mathbb{R}^{|N|xP}$ denotes the weights of the GCN for the set of known concept classes $N$ and the dimensionality $P$ of the weight vector. Similar to~\cite{Kampffmeyer2019}, the ground truth weights, denoted as $W \in \mathbb{R}^{|N|xP}$, are obtained by extracting the last layer weights of a pre-trained CNN.}
% This process is repeated for all KG nodes corresponding to $S^U$ and $S^S$, generating semantic graph embeddings for all target state classes. 
%Each embedding comes in the form of a feature vector of length 2048. 


%By combining these embeddings for the $d$ target classes, a  $d \times 2048$ features matrix is defined that is integrated as the final layer of the visual CNN-based classifier that is employed in Stages~3-4.
%A critical aspect of this process is adjusting the GNN weights to align its predictions with the semantic space. This ensures that the semantic embeddings effectively aid the classifier used in Stages 3 and 4, operating in the visual space. 



\textcolor{black}{ 
The KG  given as an input to the GCN model is a hierarchical graph created for the requirements of the   ILSVRC 2012 dataset~\cite{russakovsky2015imagenet} and represents the WordNet hierarchical structure of the $1,000$ classes comprising the dataset. These 1,000 concept labels constitute the set of classes upon which the visual classifier used for the extraction of the ground truth visual embeddings is pre-trained.
}
% A critical aspect of this process is adjusting the GNN weights to align its predictions with the semantic space. This ensures that the semantic embeddings effectively aid the classifier used in Stages 3 and 4, operating in the visual space. 
% The concepts  that are used for the training refer to a set of 1K object classes of the ILSVRC 2012 dataset~\cite{russakovsky2015imagenet}, while the pre-trained ResNet101-based classifier is used for supervision to ensure that the GNN outputs, thus the semantic object class representations, are meaningfully embedded into the visual feature space. 
After the training is completed, the GCN model is employed to process the KG (constructed in Stage 1) and generate visual embeddings for the KG nodes that correspond to the object state classes,  by taking as input the  KG that was constructed during Stage 1. Each embedding comes in the form of a feature vector of length 2048, i.e. dimension of the last layer of the  pre-trained visual CNN-based classifier.
By combining these embeddings for the $d$ target classes, a  $d \times 2048$ features matrix is defined that is integrated as the final layer of the visual CNN-based classifier that is employed in Stages~3-4.
% First, using their pre-computed GloVe word features~\cite{pennington2014glove}, a matrix of word, i.e. semantic, features embeddings is defined for each of the KG nodes.
% % To compute node embeddings, the GNN is applied to encode the KG topology and the word feature embedding matrix. 
% Subsequently,  the GNN takes as input every target node that corresponds to any class in $S^U$ and $S^S$ and aggregates the features about the node and its neighbors through a sequence of convolutions and pooling operations. This results in the generation of a feature vector having a length equal to the dimension of the last layer in the visual CNN-based classifier that is instantiated using a ResNet-101 model.
% % This process is repeated for all KG nodes corresponding to $S^U$ and $S^S$, generating semantic graph embeddings for all target state classes. 
% Each embedding comes in the form of a feature vector of length 2048. By combining these embeddings for the $d$ target classes, a  $d \times 2048$ features matrix is defined that is integrated as the final layer of the visual CNN-based classifier that is employed in Stages~3-4.
% A critical aspect of this process is adjusting the GNN weights to align its predictions with the semantic space. This ensures that the semantic embeddings effectively aid the classifier used in Stages 3 and 4, operating in the visual space. 

% A critical aspect of this procedure involves calibrating the weights of the GNN to embed its predictions in the semantic space, i.e. semantic embeddings, are useful for the classifier deployed in the visual space during Stages 3 and 4. To accomplish this, we adopt an approach based on prior research~\cite{Kampffmeyer2019, Wang2018b} that involves learning the semantic class representations by minimizing the L2 distance between the learned class representations and the weights of a fully connected layer in a ResNet classifier pre-trained on the ILSVRC 2012 dataset~\cite{russakovsky2015imagenet}.  

\vspace*{0.0cm}\noindent\textbf{Fine-tuning of the Visual Classifier (Stage 3)}:
The estimated semantic embeddings are integrated into a visual CNN classifier that relies on the ResNet backbone and is initially pre-trained for object classification. The embeddings serve as the final layer of the network, encapsulating the representations essential for predicting the train state classes $S^S$. To enable this adaptation, the visual classifier undergoes re-training, specifically tailored to the classification of the train classes. 
During this fine-tuning process, input images $I^T$ contain states sourced exclusively from the training set $S^S$, i.e. ``seen states''. The primary objective is to harness the classifier capabilities to classify these familiar states, accurately. Notably, fine-tuning involves keeping the weights of the last layer fixed, safeguarding the integrity of the acquired semantic representations from Stage 2. Consequently, adjustments are only applied to the weights of preceding layers to ensure they effectively match the ``frozen'' last-layer weights.
% Apart from this detail, the procedure takes place in the same manner as the training of a CNN classifier.
% in every training epoch a loss is computed the value of which guides the update of all layers weights except the last one. 
Following the notation introduced 
\textcolor{black}{in the beginning of Section~\ref{sec:method}, the loss function is defined as:}
\begin{equation}
% \mathcal{L} = -\sum_{i \in S^{S}} y_i \cdot \log(P(y=i|X))
\mathcal{L_V} = -\sum_{s \in S^S, i \in I^{T}} y_s \cdot \log(P(s|i)),
 \end{equation}
\textcolor{black}{for the predicted \textit{$y_s$} state label in the \textit{$S^S$} set of state labels. $P(s|i)$ denotes the probability of state label \textit{s} based on the softmax vector given an image \textit{i} from the $I^T$ training set.}

\noindent\textbf{Zero-shot OaSC (Stage 4)}:
Upon the completion of fine-tuning, the visual state classifier can be utilized for  prediction by choosing the most likely class
\begin{equation}
% \^y = \arg\max_{i \in S} \left( P(y=i|X) \right)
\hat{y} = \arg\max_{s \in S^U i \in I^{U}} \left( P(s|i) \right),
\end{equation}
\textcolor{black}{where $I^U$ denotes the test image set and $S^U$ the test state classes respectively.} 
We highlight that the classifier is well-suited for predicting either only unseen classes, i.e. zero-shot classification, or both seen and unseen classes, i.e. generalized zero-shot classification.
\vspace{-.15cm}
% \subsection{Pipeline}

% Overall, the pipeline of our method consists of four stages (\autoref{fig:pipeline}}). During \textbf{Stage 1}, the KG is constructed.

% \vspace*{0.2cm}\noindent\textbf{Construction of the KG (Stage 1)}:
% The KG creation process involves querying a common sense repository to enable generalization instead of creating a custom KG tailored to specific entities and relationships. Initially, nodes corresponding to the target state classes are generated. The repository is then queried for each node, and neighbors meeting specific criteria are added to the knowledge graph. This process continues for the newly added nodes until a specified number of hops is reached. More details can be found in the ablation section.


% \vspace*{0.2cm}\noindent\textbf{Computation of semantic embeddings (Stage 2)}:


% \vspace*{0.2cm}\noindent\textbf{Finetuning of the Classfier (Stage 3)}:

% \vspace*{0.2cm}\noindent\textbf{Deployment  (Stage 4)}:
\section{Related Work}
\label{appsec: related work}
Bayesian causal discovery literature has primarily focused on inference in linear models with closed-form posteriors or marginalized parameters. Early works considered sampling directed acyclic graphs (DAGs) for discrete~\cite{cooper1992bayesian, madigan1995bayesian, heckerman2006bayesian} and Gaussian random variables~\cite{friedman2003being, tong2001active} using Markov chain Monte Carlo (MCMC) in the DAG space. However, these approaches exhibit slow mixing and convergence~\cite{eaton2012bayesian,grzegorczyk2008improving}, often requiring restrictions on number of parents~\cite{kuipers2017partition}. %Alternative exact dynamic programming methods are limited to small settings~\cite{koivisto2012advances}. 

Recent advances in variational inference~\cite{zhang2018advances} have facilitated graph inference in DAG space, with gradient-based methods employing the NOTEARS DAG penalty \cite{zheng2018dags}.\cite{annadani2021variational} samples DAGs from autoregressive adjacency matrix distributions, while \cite{lorch2021dibs} utilizes Stein variational approach \cite{liu2016stein} for DAGs and causal model parameters. \cite{cundy2021bcd} proposed a variational inference framework on node orderings using the gumbel-sinkhorn gradient estimator \cite{mena2018learning}. \cite{deleu2022bayesian,nishikawa2022bayesian} employ the GFlowNet framework \cite{bengio2021gflownet} for inferring the DAG posterior. Most methods, except\cite{lorch2021dibs} are restricted to linear models, while \cite{lorch2021dibs} has high computational costs and lacks DAG generation guarantees compared to our method.
% at least quadratic scaling complexity, both with respect to the number of nodes (due to the DAG penalty) as well as number of posterior samples. Our proposed approach instead has linear complexity with respect to number of posterior samples and does not require any additional DAG penalty.     

In contrast, \emph{quasi-Bayesian} methods, such as DAG bootstrap \cite{friedman2013data}, demonstrate competitive performance. DAG bootstrap resamples data and estimates a single DAG using PC \cite{spirtes2000causation}, GES \cite{chickering2002optimal}, or similar algorithms, weighting the obtained DAGs by their unnormalized posterior probabilities. Recent neural network-based works employ variational inference to learn DAG distributions and point estimates for nonlinear model parameters \cite{charpentier2022differentiable,geffner2022deep}.
\section{Experiments}
% \haizhou{Follow the same way of introduction as we did in Section2.}
% \noindent In this section, we will introduce datasets and experimental setups that we used. Then we evaluate our method, other self-supervised methods, and supervised methods under different distribution shifts (\ie, concept shifts and covariate shifts) under common settings (\ie, transductive, inductive settings). It has to note that we focus on node-level tasks (\eg, node classification) in this work. As for graph-level tasks, we leave it as our future work and some simple experiments can be found in Appendix~\ref{app:graph_classification}. 
In this section, we first introduce the experimental setup including datasets, training, and evaluation protocol in Section~\ref{sec:dataset}~and~\ref{sec:unsupervised}. 
% Next, we present our experimental setup and conduct extensive experiments to evaluate our method in Section~\ref{sec:unsupervised}. 
We then perform an ablation study to demonstrate the effectiveness of each proposed component in Section~\ref{sec:ablation}. 
Additionally, we analyze the impact of important hyper-parameters in Section~\ref{sec:sensitivity}. 
Subsequently, we integrate our method with various encoding models, showcasing the model-agnostic nature of our recipe in Section~\ref{sec:other_models}. 
Finally, we provide some qualitative results such as feature visualization in Section~\ref{sec:vis}.
It is important to note that we focus on node-level tasks (\eg, node classification) in this work. As for graph-level tasks, we leave it as our future work, while some simple experiments are also provided in Appendix~\ref{app:graph_classification}.

\subsection{Datasets}\label{sec:dataset}
There exist some benchmarks for evaluating graph out-of-distribution generalization~\cite{good,ji2022drugood,gds}. 
Among them, GOOD~\cite{good} is the most representative and comprehensive benchmark that curates more diverse graph datasets with diverse tasks, including single/multi-task graph classification, graph regression, and node classification involving more distribution shifts (\ie, concept shifts and covariate shifts). Hence in this work, we follow the evaluation protocol proposed in \cite{good}. Furthermore, we validate the effectiveness of our method in the datasets (\ie, Amazon-Photo, Elliptic) that are used in EERM~\cite{eerm}. The statistics and detailed introduction to these datasets can be found in Table~\ref{tab:dataset} and Appendix~\ref{app:datasets}.

\begin{table*}[htp]
\caption{The descriptions of datasets. ``Domain-Level'' means splitting by graphs, ``Time-Aware'' denotes splitting according to chronological order.``Word'' and ``Degree'' represent splitting according to word diversity and node degree respectively. ``Language'' means splitting by user language, suggesting the prediction should not be impacted by the language the user use. ``University'' denotes splitting according to the domain university, implying that the prediction of webpages should be based on word contents and link connections rather than university features. ``Color'' means that nodes are split according to node differences in covariate shift and color-label correlations in concept shift.}
\label{tab:dataset}
\centering
\begin{tabular}{cccccccc}
\toprule
Datasets     & Network Type        & \#Nodes & \#Edges & \#Attributes &\#Classes& Train/Val/Test Split     & Metric   \\
% Cora         & Artificial Transformation & 2,703   &         &              &         &                      & Accuracy \\
Amazon-Photo\footnotemark
             & Co-purchasing network      & 7,650   & 119,081   & 755          & 10      & Domain-Level         & Accuracy \\
Elliptic\footnotemark  
             & Bitcoin transactions       & 203,769 & 234,355   & 165          & 2       & Time-Aware           & F1-Score \\
GOOD-Cora    & Scientific publications    & 19,793  & 126,842   & 8,710         & 70      & Word/Degree          & Accuracy \\
% GOOD-Arxiv   & arXiv papers               & 169,343 & 2,315,598 & 128          & 40      & Time/Degree          & Accuracy \\
GOOD-Twitch  & Gamer network              & 34,120  & 892,346   & 128          & 2       & Language             & ROC-AUC  \\
GOOD-CBAS    & A BA-house graph           & 700     & 3,962     & 4             & 4       & Color                & Accuracy \\
GOOD-WebKB   & Webpage network            & 617     & 1,138     & 1,703         & 5       & University           & Accuracy \\
\bottomrule
\end{tabular}
\end{table*}
\footnotetext[5]{This dataset is adopted from~\cite{yang2016revisiting}. \cite{eerm} constructs ten graphs with different environment id’s for each graph.} 
\footnotetext[6]{The original is available on \hyperlink{https://www.kaggle.com/ellipticco/elliptic-data-set}{https://www.kaggle.com/ellipticco/elliptic-data-set}}

\subsection{Unsupervised Representation Learning}\label{sec:unsupervised}
\subsubsection{Transductive Setting}~\label{sec:trans}
% \noindent\textbf{Baselines.}\quad We conduct experiments with 12 baselines which consist of three categories: supervised methods and self-supervised generative methods, self-supervised contrastive methods. Specifically, we compare with three supervised baselines: empirical risk minimization~(ERM)~\cite{erm}, invariant risk minimization (IRM)~\cite{irm}, and a recent proposed graph OOD method dubbed EERM~\cite{eerm}. We also compare various unsupervised node-level representation learning methods: three self-supervised generative methods including GAE~\cite{gae}, VGAE~\cite{gae}, GraphMAE~\cite{gmae} and seven self-supervised contrastive methods: DGI~\cite{dgi}, MVGRL~\cite{mvgrl}, GRACE~\cite{grace}, RoSA~\cite{rosa}, BGRL~\cite{bgrl}, COSTA~\cite{costa}, SwAV~\cite{swav}. The descriptions of these methods can be found in Appendix~\ref{app:baselines}.
In this subsection, we focus on validating our proposed algorithm under the transductive setting, where the test nodes will participate in message passing~\cite{gilmer2017neural} during training following~\cite{good}. 

\noindent\textbf{Baselines.} We conduct experiments with 12 baselines from three categories: (i)~supervised methods, including empirical risk minimization~(\textbf{ERM})~\cite{erm}, invariant risk minimization (\textbf{IRM})~\cite{irm}, and a recent proposed graph OOD method \textbf{EERM}~\cite{eerm}; (ii)~self-supervised generative methods including Graph Autoencoder (\textbf{GAE})~\cite{gae}, Variational Graph Autoencoder (\textbf{VGAE})~\cite{gae}, Self-Supervised Masked Graph Autoencoders (\textbf{GraphMAE})~\cite{gmae}; (iii)~self-supervised contrastive methods including Deep Graph Infomax (\textbf{DGI})~\cite{dgi}, Contrastive Multi-View Representation Learning on Graphs (\textbf{MVGRL})~\cite{mvgrl}, Deep Graph Contrastive Representation Learning (\textbf{GRACE})~\cite{grace}, A Robust Self-Aligned Framework for Node-Node Graph Contrastive Learning (\textbf{RoSA})~\cite{rosa}, Bootstrapped Representation Learning on Graphs (\textbf{BGRL})~\cite{bgrl}, Covariance-Preserving Feature Augmentation for Graph Contrastive Learning (\textbf{COSTA})~\cite{costa}, Unsupervised Learning of Visual Features by Contrasting Cluster Assignments (\textbf{SwAV})~\cite{swav}. The detailed descriptions of these baselines can be found in Appendix~\ref{app:baselines}.

\noindent\textbf{Experimental setup.} We use the same graph encoder across different datasets for a fair comparison following~\cite{good}. We use grid search to find other hyper-parameters (\eg, learning rate, epochs) for different methods. For all experiments, we select the best checkpoints for ID and OOD tests according to results on ID and OOD validation sets following~\cite{good}, respectively. Experimental details and hyper-parameter selections are provided in Appendix~\ref{app:hyper}. For evaluating unsupervised methods, a linear classifier will be built on the frozen trained encoder after finishing pre-training. The reported results are the mean performance with standard deviation after 10 runs following~\cite{good}.

\noindent\textbf{Analysis.}\quad Based on the experimental results listed in Table~\ref{tab:trans_concept} and \ref{tab:trans_covariate}, we can draw the following conclusions: firstly, we find strong self-supervised methods (\eg, GRACE, BGRL, COSTA) are more robust to distribution shifts (concept shift in Table~\ref{tab:trans_concept} and covariate shift in Table~\ref{tab:trans_covariate}) compared to supervised methods. For instance, on GOOD-CBAS and GOOD-WebKB datasets, GRACE surpasses the best supervised method by large margins (over 6\% absolute improvement). Interestingly, we find the methods designed for OOD generalization (\ie, IRM) and graph OOD generalization (\ie, EERM) do not attain superior performance than the standard ERM on most of the datasets. For example, EERM shows superior OOD performance compared to ERM in only one experiment, and IRM outperforms ERM in four out of ten experiments across the conducted evaluations. This phenomenon is also observed in \cite{good,ahuja2020empirical,rosenfeld2021risks}, showcasing the challenge of achieving invariant prediction in non-Euclidean graph settings. 

Furthermore, our method surpasses other SOTA self-supervised methods on the OOD test set of all datasets by a considerable margin while achieving comparable performance in the in-distribution test set. For instance, on small datasets such as GOOD-CBAS and GOOD-WebKB, our method outperforms GRACE\footnote{MARIO is built up on GRACE according to our recipe. So, we make a comparison with GRACE here.} by over 2\% absolute accuracy on the OOD test set. On larger datasets such as GOOD-Cora and GOOD-Twitch, our method still outperforms other methods which shows its superiority. For instance, under covariate shift, MARIO surpasses other methods by over 7\% absolute accuracy on the GOOD-Twitch OOD test set. These statistics prove the effectiveness of our design.


\begin{table*}[htp]
\caption{Experimental results of all methods under concept shift. The bold font means the top-1 performance and the underline represents the second performance across the unsupervised methods. 'ID' represents in-distribution test performance and 'OOD' means out-of-distribution test performance. (OOM: out-of-memory on a GPU with 24GB memory)}
\label{tab:trans_concept}
\centering
\scalebox{0.95}{
\begin{tabular}{l|cc|cc|cc|cc|cc}
\toprule
\toprule
\multirow{3}{*}{concept shift} & \multicolumn{4}{c|}{GOOD-Cora}                   & \multicolumn{2}{c|}{GOOD-CBAS} & \multicolumn{2}{c|}{GOOD-Twitch} & \multicolumn{2}{c}{GOOD-WebKB} \\
                           & \multicolumn{2}{c}{word} & \multicolumn{2}{c|}{degree}& \multicolumn{2}{c|}{color}    & \multicolumn{2}{c|}{language}   & \multicolumn{2}{c}{university} \\
                           & ID         & OOD         & ID          & OOD          & ID            & OOD           & ID             & OOD            & ID            & OOD            \\
\midrule
ERM                        & 66.38±0.45 & 64.44±0.18  & 68.60±0.40  & 60.76±0.34   & 89.79±1.39    & 83.43±1.19    & 80.80±1.00     & 56.92±0.92     & 62.67±1.53    & 26.33±1.09     \\
IRM                        & 66.42±0.41 & 64.29±0.31  & 68.57±0.35  & 61.45±0.24   & 89.64±1.21    & 82.29±1.14    & 78.87±1.04     & 59.30±1.79     & 62.67±1.10    & 26.88±1.42     \\
EERM                       & 65.10±0.44 & 62.45±0.19  & 66.95±0.44  & 56.58±0.25   & 79.07±2.12    & 64.50±1.01    & OOM            & OOM            & 62.50±2.01    & 28.07±3.23      \\
\midrule
% Random-Init                & 37.53±1.74 & 32.12±1.24  & 37.82±1.71  & 27.74±1.14   &               &               &                &                & 60.33±2.21    & 27.07±1.70     \\
GAE                        & 60.65±0.89 & 58.00±0.55  & 62.59±1.11  & 53.44±0.80   & 75.28±1.36    & 68.07±2.05    & 81.25±0.81     & 51.51±1.05     & 62.17±3.34    & 25.78±1.85     \\
VGAE                       & 63.19±0.53 & 60.35±0.47  & 61.65±0.66  & 54.28±0.28   & 76.50±0.50    & 59.07±0.56    & 80.46±0.53     & 55.56±4.53     & 62.50±2.38    & 24.40±2.57     \\
GraphMAE                   & \underline{66.44±0.46} & \underline{64.87±0.30}  & 67.95±0.46  & 59.41±0.39   & 89.14±0.89    & 82.93±0.93    & 80.05±0.64     & 59.38±1.49     & 61.83±3.37    & 29.27±2.15     \\
DGI                        & 63.33±0.56 & 60.71±0.49  & 65.93±1.02  & 55.83±0.53   & 91.22±1.47    & 85.00±1.66    & 80.05±0.87     & 59.16±1.88     & 61.83±2.83    & 28.63±1.92      \\
MVGRL                      & OOM        & OOM         & OOM         & OOM          & 88.57±1.15    & 76.50±1.17    & OOM            & OOM            & 62.00±3.79    & 28.26±4.20     \\
GRACE                      & 65.61±0.61 & 63.92±0.44  & \textbf{68.59±0.35}  & 60.15±0.45   & 92.00±1.39    & 88.64±0.67    & \textbf{83.43±0.63}     & \underline{60.45±1.46}     & 64.00±3.43    & \underline{34.86±3.43}  \\
RoSA                       & 64.06±0.67 & 62.44±0.39  & 67.07±0.65  & 57.68±0.44   & 90.78±2.27    & 85.93±2.14    & 82.39±0.42     & 57.45±2.16     & 64.17±4.10    & 32.20±2.15     \\
BGRL                       & 65.18±0.43 & 63.43±0.45  & 66.83±0.80  & 59.63±0.38   & 92.36±1.16    & 87.14±1.60    & 82.52±0.60     & 55.48±1.48     & 63.67±2.33    & 31.47±3.43     \\
COSTA                      & 65.05±0.80 & 62.37±0.45  & 66.76±0.87  & 55.73±0.36   & \underline{93.50±2.62}    & \underline{89.29±3.11}    & 83.15±0.30 & 55.03±3.22     & 61.66±2.58    & 32.39±2.13 \\
% ArCL                       &            &             & 67.64±0.57  & 59.71±0.44   &               &               &                &                & 65.00±3.94    & 35.41±1.97 \\      
SwAV                       & 62.22±0.53 & 59.79±0.53  & 64.65±0.94  & 55.06±0.39   & 89.00±0.79    & 81.72±0.66    & \underline{83.32±0.15}     & 59.69±1.97     & \underline{65.17±3.76}    & 29.36±2.01    \\
\midrule
MARIO                       & \textbf{67.11±0.46} & \textbf{65.28±0.34}  & \underline{68.46±0.40}  & \textbf{61.30±0.28}   & \textbf{94.36±1.21}    & \textbf{91.28±1.10}    & 82.31±0.54     & \textbf{63.33±1.72}     & \textbf{65.67±2.81}    & \textbf{37.15±2.37}     \\
\bottomrule
\end{tabular}}
\end{table*}

\begin{table*}[htp]
\caption{Experimental results of all methods under covariate shift. The bold font means the top-1 performance and the underline represents the second performance across the unsupervised methods. 'ID' represents in-distribution test performance and 'OOD' means out-of-distribution test performance. (OOM: out-of-memory on a GPU with 24GB memory)}
\label{tab:trans_covariate}
\centering
\scalebox{0.95}{
\begin{tabular}{l|cc|cc|cc|cc|cc}
\toprule
\toprule
\multirow{3}{*}{covariate shift} & \multicolumn{4}{c|}{GOOD-Cora}                                   & \multicolumn{2}{c|}{GOOD-CBAS} & \multicolumn{2}{c|}{GOOD-Twitch} & \multicolumn{2}{c}{GOOD-WebKB} \\
                           & \multicolumn{2}{c}{word} & \multicolumn{2}{c|}{degree}& \multicolumn{2}{c|}{color}    & \multicolumn{2}{c|}{language}   & \multicolumn{2}{c}{university} \\
                           & ID         & OOD         & ID          & OOD          & ID            & OOD           & ID             & OOD            & ID            & OOD            \\
\midrule
ERM                        & 70.50±0.41 & 64.69±0.33  & 72.46±0.49  & 55.53±0.50   & 92.00±3.08    & 77.57±1.29    & 70.98±0.41     & 49.35±5.09     & 39.34±1.79    & 14.52±3.14   \\
IRM                        & 70.48±0.26 & 64.53±0.57  & 71.98±0.34  & 53.72±0.46   & 90.86±2.41    & 78.86±1.67    & 69.81±0.95     & 49.11±2.82     & 38.52±3.30    & 13.97±2.80     \\
EERM                       & OOM        & OOM         & OOM         & OOM          & 65.00±2.57    & 57.43±3.60    & OOM            & OOM            & 46.07±4.55    & 27.40±7.65     \\
\midrule
GAE                        & 56.63±0.79 & 48.93±0.93  & 66.30±0.88  & 34.01±0.87   & 73.00±2.16    & 60.86±3.01    & 67.24±1.23     & 47.65±2.49     & 45.08±6.32    & 28.02±6.29    \\
VGAE                       & 62.02±0.66 & 54.12±0.86  & 69.41±0.57  & 44.20±1.29   & 62.29±2.04    & 63.29±1.11    & 66.99±1.43     & \underline{50.48±4.58}     & 48.85±4.68    & 20.87±6.69     \\
GraphMAE                   & 68.14±0.43 & 64.00±0.33  & \textbf{73.36±0.56}  & 53.75±0.55   & 67.28±3.03    & 67.28±1.49    & 68.84±1.20     & 48.02±2.79     & 48.03±4.34    & 30.00±8.09     \\
DGI                        & 60.85±0.75 & 57.03±0.67  & 68.97±0.41  & 41.75±0.88   & 69.57±4.09    & 59.71±3.43    & 68.43±1.05     & 44.83±1.61     & 48.52±5.04    & 21.11±7.50     \\
MVGRL                      & OOM        & OOM         & OOM         & OOM          & 65.00±1.94    & 64.15±0.77    & OOM            & OOM           & \textbf{54.10±5.39}    & 16.59±6.51     \\
GRACE                      & \underline{68.77±0.33} & \underline{64.21±0.41}  & 72.69±0.34  & \underline{56.10±0.63}   & \underline{93.57±1.83}    & \underline{89.29±3.40}    & \underline{71.12±0.87} & 46.21±1.54 & 49.67±5.82    & 28.10±4.68    \\
RoSA                       & 68.19±0.56 & 62.48±0.61  & 71.04±0.62  & 52.72±0.79   & 84.71±4.14    &79.14±3.51     & 70.58±0.36     & 45.83±1.72     & 52.30±4.24    & \underline{34.24±7.92}     \\
BGRL                       & 67.23±0.43 & 61.33±0.36  & 72.11±0.39  & 49.15±0.73   & 89.00±2.56    & 79.86±3.29    & \textbf{71.43±0.53}     & 43.86±0.94     & 51.80±5.55    & 30.32±7.61    \\
COSTA                      & 65.28±0.60 & 60.33±0.53  & 70.65±0.62  & 54.03±0.28   & 92.29±1.59    & 82.71±2.74    & 69.29±1.37     & 49.07±2.13     & 50.49±3.01    & 29.84±4.75   \\
SwAV                       & 63.29±1.01 & 56.98±0.94  & 70.27±0.73  & 43.00±0.52   & 89.57±1.12    & 81.43±1.69    & 69.19±0.93     & 49.37±2.96     & 49.84±4.82    & 30.55±6.72   \\
\midrule
MARIO                       & \textbf{69.99±0.54} & \textbf{65.06±0.34}  & \underline{72.73±0.43}  & \textbf{57.73±0.45}  & \textbf{94.57±2.46}    & \textbf{91.00±2.48}     & 68.31±0.78 & \textbf{57.37±1.37}     & \underline{53.94±3.23}    & \textbf{35.24±4.98}   \\
\bottomrule
\end{tabular}}

\end{table*}

\subsubsection{Inductive Setting}
In this subsection, we conduct experiments under the inductive settings, where the test nodes are kept unseen during training. This setting is more suitable for domain generalization.
% But we think it is more convincing that conduct experiments under inductive settings which means test nodes are unseen during training. This setting is more appropriate for domain generalization.

\noindent\textbf{Baselines:} For GOOD-WebKB and GOOD-CBAS datasets, we adopt ERM, IRM, GraphMAE, and GRACE as our baselines. And for Amazon-Photo and Elliptic datasets, we select ERM, EERM, and GRACE as our baselines.

\noindent\textbf{Experimental setup:} For GOOD-WebKB and GOOD-CBAS datasets, we use the same model configuration in Section~\ref{sec:trans}.
% Besides, we add experiments on Amazon-Photo dataset~\cite{yang2016revisiting} and Elliptic~\cite{elliptic} dataset in this subsection. 
For Amazon-Photo dataset~\cite{yang2016revisiting} and Elliptic~\cite{elliptic} dataset, they consist of many snapshots (training data and testing data use different snapshots) which are naturally inductive. For Amazon-Photo dataset, we use 2-layer GCN~\cite{gcn} as the encoder and for elliptic dataset, we use 5-layer GraphSAGE~\cite{sage} as encoder following~\cite{eerm}.

% Figure environment removed

\noindent\textbf{Analysis:}
According to Figure~\ref{fig:amazon},\ref{fig:elliptic},\ref{fig:ind_con},\ref{fig:ind_cov}, we can draw following conclusions:
firstly, based on Figure~\ref{fig:amazon}, it is evident that our method outperforms other representative supervised and self-supervised methods on all test graphs (T1$\sim$T8). This superiority is reflected in the larger median value of our method compared to others. For instance, MARIO achieves over a 3\% absolute improvement compared to ERM in terms of the mean value of eight median values. Additionally, our method demonstrates higher stability across different random initializations, as indicated by the closer proximity of the first and third quartile values to the median value~(\eg, the difference of first and third quartile values of ERM, EERM, GRACE and MARIO are 4.2, 3.3, 6.7 and 1.0 on T8 respectively which indicates MARIO is much more stable than other methods). Furthermore, our method exhibits consistent performance across different graphs (\eg, The standard deviation of median values on T1$\sim$T8 for ERM, EERM, GRACE, and MARIO are 0.4, 1.1, 1.2, and 0.3, respectively.), indicating its robustness to environmental variations and its ability to extract invariant features: $g(G^e) \approx g(G^{e'})$ for all $e, e' \in \mathcal{E}^\text{train}$. In summary, our method showcases enhanced OOD generalization capabilities.
% $g(G^e)g(G^e^\prime)$ where $any e, e^\prime in \mathcal{E}^{train}$

Secondly, from the results presented in Figure~\ref{fig:elliptic}, we can observe that our method averagely harvests 10.9\% absolute improvement over GRACE and 12.5\% absolute improvement over EERM in terms of F1 scores on Elliptic dataset. This demonstrates the effectiveness of our method in handling distribution shifts and improving performance compared to existing approaches. It is worth noting that GRACE's performance worsens over time, indicating its inability to handle distribution shifts effectively. In contrast, our method consistently achieves better F1 scores, except for T9, which is caused by the dark market shutdown occurred after T7~\cite{elliptic}. The emergence of such an event introduces significant variations in data distributions, which subsequently results in performance degradation for all methods. Indeed, this event serves as an unpredictable external factor that introduces significant challenges for models trained on limited training data. The results indicate that the performance heavily depends on available training data. Nonetheless, our approach outperforms other methods even in such an extreme case. This highlights the effectiveness of our method in addressing distribution shifts and improving generalization performance.

Finally, based on the observations from Figure~\ref{fig:ind_con} and Figure~\ref{fig:ind_cov} MARIO demonstrates the best performances on both ID and OOD test sets for GOOD-WebKB and GOOD-CBAS datasets, under both concept shift and covariate shift. Notably, MARIO outperforms other methods by more than 3\% and 10\% absolute improvement on GOOD-WebKB and GOOD-CBAS, respectively, under covariate shift. We can draw similar conclusions as discussed in Section~\ref{sec:trans}. Even under the inductive setting, our method continues to demonstrate excellent OOD generalization capabilities and achieves comparable or even improved in-distribution test performance. These statistical results further validate the effectiveness of our method in handling distribution shifts and enhancing generalization performance.

Overall, the observations we have made provide strong evidence of the great capacity of our method for handling distribution shifts, validating its effectiveness and potential for real-world applications.



% Figure environment removed

% Figure environment removed


% Figure environment removed


\subsection{Ablation Studies}\label{sec:ablation}
\noindent Table~\ref{tab:aba} provides a detailed analysis of the effect of each component according to our proposed recipe for improving OOD generalization in graph contrastive learning. Let's examine the different variants of our method and their impact on performance.
Specifically, MARIO~(w/o ad) represents MARIO without  adversarial augmentation. MARIO~(w/o cmi) denotes we only maximize the mutual information between positive pairs without considering conditional mutual information. MARIO~(w/o cmi, ad) means a vanilla graph contrastive method that is similar to GRACE. 

From Table~\ref{tab:aba}, we can find MARIO~(w/o cmi) lags far behind MARIO on OOD test set which demonstrates appropriately minimizing the redundant information (\ie, conditional mutual information) is essential to improve OOD generalization of GCL methods. And adversarial augmentation can also boost OOD generalization because it can approximately serve as a supermum operator to learn more invariant features  discussed in Section~\ref{sec:aug}. Based on the analysis of these variants, it is evident that the proposed improvements on data augmentation and contrastive loss in the recipe are both effective in enhancing graph OOD generalization. Each component contributes to the overall performance improvement, and their combination leads to a stronger self-supervised graph learner in terms of graph OOD generalization. 

In short, the findings from Table~\ref{tab:aba} support the rationale behind your proposed recipe and provide empirical evidence of the effectiveness of each proposed component. By incorporating these enhancements, our method achieves superior performance in handling distribution shifts and improving graph OOD generalization in graph contrastive learning.
\begin{table*}[htp]
\caption{Ablation studies for MARIO by masking each component.}
\label{tab:aba}
\centering
\scalebox{0.9}{
\begin{tabular}{l|cc|cc|cc|cc|cc}
\toprule
\toprule
\multirow{3}{*}{concept shift} & \multicolumn{4}{c|}{GOOD-Cora}                       & \multicolumn{2}{c|}{GOOD-CBAS} & \multicolumn{2}{c|}{GOOD-Twitch} & \multicolumn{2}{c}{GOOD-WebKB} \\
                           & \multicolumn{2}{c}{word} & \multicolumn{2}{c|}{degree}& \multicolumn{2}{c|}{color}    & \multicolumn{2}{c|}{language}   & \multicolumn{2}{c}{university} \\
                           & ID         & OOD         & ID          & OOD          & ID            & OOD           & ID             & OOD            & ID            & OOD            \\
\midrule
MARIO                      & \textbf{67.11±0.46} & \textbf{65.28±0.34}  & \textbf{68.46±0.40}  & \textbf{61.30±0.28}      & \textbf{94.36±1.21}  & \textbf{91.28±1.10}    & 82.31±0.54     & \textbf{63.33±1.72}     & \textbf{65.67±2.81}    & \textbf{37.15±2.37}     \\
MARIO(w/o ad)              & 66.23±0.53 & 64.02±0.18  & 67.88±0.38  & 60.46±0.29   & 93.21±1.25    & 90.29±0.91    & 82.42±0.73     & 60.50±1.02     & 64.83±2.83    & 36.51±3.25    \\
MARIO(w/o cmi)             & 65.32±0.60 & 63.51±0.32  & 68.14±0.32  & 61.19±0.34   & 94.15±1.23    & 90.57±1.96    & \textbf{82.51±0.56}     & 61.41±2.63     & 64.50±4.35    & 35.78±2.53     \\
MARIO(w/o cmi, ad)         & 64.67±0.55 & 63.11±0.32  & 67.95±0.65  & 60.01±0.57   & 93.36±1.66    & 89.64±1.73    & 81.90±0.75     & 60.12±1.60     & 64.17±3.67    & 34.13±2.38     \\
\bottomrule
\end{tabular}}
\end{table*}
% & 65.32±0.60 & 63.51±0.32 exchange 64.67±0.55 & 63.11±0.32
% 68.14±0.32       id ood test: 60.95±0.43       ood ood test: 61.19±0.34


\subsection{Sensitivity Analysis}\label{sec:sensitivity}
\noindent In this subsection, we will analyze some important hyper-parameters of our method. We conduct sensitivity analysis on GOOD-WebKB dataset with concept shift, we chose two sensitive hyper-parameters (\ie, the coefficient $\gamma$ of condition mutual information in Equation~\ref{equ:cmi} and the number of prototypes $|C|$ in Equation~\ref{equ:pq}). The coefficient of CMI range in $[0.001, 0.01, 0.1, 0.5, 1]$ and the number of prototypes $|C|$ ranges in $[10, 50, 100, 200, 300]$. From Figure~\ref{fig:sensitivity}, we can observe that $\gamma$ reaches 0.1 and $|C|$ reaches 100 or 200 can achieve the best OOD test accuracy. Both higher and lower values of $\gamma$ result in suboptimal performance. This finding aligns with previous research such as DIB~\cite{dib}, indicating that an appropriate compression level is crucial for achieving optimal performance. Extremely high or low compression values are not ideal. 

Regarding the number of prototypes $|C|$, based on the results shown in Figure~\ref{fig:sensitivity}, it is found that setting $|C|=100$ leads to the best performance in terms of OOD test accuracy. This choice provides a moderate number of pseudo labels, which is beneficial for the learning process. 

Based on the sensitivity analysis, we determined that setting $\gamma=0.1$ and $|C|=100$ on most datasets. These hyperparameter values strike a balance between compression level and the number of prototypes, resulting in improved graph OOD generalization.
% Figure environment removed


\subsection{Integrated with Other Models}\label{sec:other_models}
% Figure environment removed

\begin{table}[htp]
\caption{Results of different learning approaches with different encoding models (\ie, GCN, GraphSAGE, GAT).}
\label{tab:others}
\centering
\scalebox{0.9}{
\begin{tabular}{cc|cc|cc}
\toprule
\toprule
\multirow{3}{*}{Model}& \multirow{3}{*}{Method} & \multicolumn{2}{c|}{GOOD-CBAS} & \multicolumn{2}{c}{GOOD-WebKB} \\
                & & \multicolumn{2}{c|}{color}    & \multicolumn{2}{c}{university} \\
                &   & ID          & OOD         & ID          & OOD            \\
\midrule
\multirow{3}{*}{GCN} 
&ERM               & 89.79±1.39 & 83.43±1.19  &  62.67±1.53 & 26.33±1.09         \\
&GRACE             & 92.00±1.39 & 88.64±0.67  &  64.00±3.43 & 34.86±3.43        \\
&MARIO             & 94.36±1.21 & 91.28±1.10  &  65.67±2.81 & 37.15±2.37        \\ \bottomrule
\multirow{3}{*}{SAGE} 
&ERM               & 95.07±1.51 & 75.14±1.19  & 73.67±2.08  & 46.33±3.42       \\
&GRACE             & 95.29±1.11 & 74.43±2.36  & 70.50±5.06  & 49.54±3.83        \\
&MARIO             & 96.00±1.07 & 76.29±3.01  & 71.00±3.82  & 51.74±4.63        \\ \bottomrule
\multirow{3}{*}{GAT} 
&ERM               & 78.64±3.63 & 72.93±2.64  & 61.33±3.71  & 28.99±2.63        \\
&GRACE             & 84.57±1.79 & 78.36±1.60  & 59.50±2.36  & 35.78±3.26        \\
&MARIO             & 84.93±1.95 & 80.43±1.89  & 62.17±4.78  & 38.17±3.10        \\
\bottomrule
\end{tabular}}
\end{table}



\noindent In the subsection, we demonstrate the model-agnostic nature of the recipe by integrating it with various graph neural network (GNN) models, including GCN, GraphSAGE, and GAT.

From Table~\ref{tab:others}, it can be observed that regardless of the specific GNN model used as the encoder, our method consistently achieves the best performance on the OOD test set. This indicates the effectiveness and robustness of our method across different GNN models.
By achieving superior performance across different GNN models, MARIO demonstrates its versatility and ability to improve the OOD generalization of various graph neural models. This highlights the broad applicability and effectiveness of our recipe in enhancing the performance of different GNN encoders.

Furthermore, we integrate our recipe with other GCL methods in Appendix~\ref{app:other_methods}. The results demonstrate our recipe can boost the OOD generalization ability of various GCL methods which means our recipe can serve as a plug-in for many current classical GCL methods.

% Figure environment removed

\subsection{Visualization}\label{sec:vis}
\subsubsection{Metric Score Curves}
We present metric score curves for ERM and MARIO, including training, ID validation, ID testing, OOD validation, and OOD testing accuracy, in Figure~\ref{fig:curve2}. Notably, MARIO demonstrates superior convergence with approximately 10\% absolute improvement on the OOD test set compared to ERM. Furthermore, MARIO effectively narrows the performance gap between in-distribution and out-of-distribution performance, showcasing its efficacy in enhancing OOD generalization for graph data. More metric score curves can be found in Appendix~\ref{app:curves}.


\subsubsection{Feature Visualization}
In order to assess the quality of learned embeddings, we adopt t-SNE~\cite{tsne} to visualize the node embedding on GOOD-Cora dataset (concept shift in word domain) using random-init of GCN, EERM, GRACE, and MARIO, where different classes have different colors in Figure~\ref{fig:vis}. For clarity, we select eight classes with the largest number of nodes to enhance the informativeness and interpretability of the visualization. We can observe that the 2D projection of node embeddings learned by MARIO has a better separation of clusters, which indicates the model can help learn representative features for downstream tasks. It has to note that we depict both ID nodes and OOD nodes in the same figure. 

Besides, we also separately visualize ID nodes and OOD nodes in the different figures in the Appendix~\ref{app:feature}. And we can find MARIO performs a clearer separation of clusters whether on ID nodes or OOD nodes compared to other methods.



\section{Conclusion}
\label{sec:conclusion}


We presented DCA, an algorithm to address disparity in outcomes of ranking processes using compensatory bonus points. We showed that DCA, by relying on a sampling-based approach, successfully reduces disparity in a wide range of settings, while being significantly more efficient than state-of-the-art approaches, running in sub-linear time. This makes DCA a good candidate for iterative processes that would allow users to identify the ranking function that best fits their needs while checking for its fairness impacts and the required compensatory bonus points.  


Our approach relies on the use of compensatory bonus points, a departure from previous work, which has mostly focused on modifying the ranking function directly, or on the use of quotas. A significant advantage of compensatory bonus points is that they are transparent, interpretable, and easily explainable to all stakeholders.




% reference
\bibliography{reference}
\bibliographystyle{plain}

% App
\newpage
\appendix
%\section{Related Work}
%\subsection{Cost Volume based Deep Stereo Matching}
%Stereo matching is a typical problem that has been studied for decades and a well-known four-step pipeline \cite{scharstein2002taxonomy} has been established, where cost volume construction is an indispensable step. Current state-of-the-art stereo matching methods are all cost volume based methods and they can be categorized into two types. Typically, a cost volume is a 4D tensor of height, width, disparity, and features. The first category just uses a full correlation to generate a single-feature cost volume. Such methods are usually efficient but lose much information because of the decimation of feature channels. Many previous work, including Dispnet \cite{dispnet}, MADNet \cite{madnet}, IResNet \cite{iresnet} and AANet \cite{aanet}, belong to this category. The second category usually uses concatenation \cite{gcnet} or group-wise correlation \cite{gwcnet} to generate a multi-feature 4D cost volume. Such a method can achieve better performance while requiring higher computational complexity and memory consumption. Actually, a majority of the top-performing networks in public leaderboards belong to this category, such as GANet \cite{ganet}, CSPN \cite{cspn} and ACFNet \cite{acfnet}. These methods generally employ multiple 3D convolution layers to constantly regularize the 4D cost volume and then apply softmax over the disparity dimension to produce a discrete disparity probability distribution. The final predicted disparity is obtained by softly weighting indices according to their probability, which is also called soft argmin in GCNet \cite{gcnet}. However, soft argmin leaves the output susceptible to multi-modal disparity probability distributions. ACFNet \cite{acfnet} observes this problem and proposes to directly supervise the cost volume with unimodal ground truth distributions. In contrast, we define an uncertainty estimation to quantify the degree to which the cost volume tends to be multi-modal distribution, higher implies the higher possibility of estimation error.

\subsection{Multi-scale Cost Volume based Stereo Matching}
Cost volume construction is an indispensable step in the well-known four-step pipeline for stereo matching \cite{scharstein2002taxonomy, pamisurvey1, pamisurvey2}. Typically, current state-of-the-art stereo matching methods can be categorized into two types of cost volume-based methods, where the cost volume is a 4D tensor of height, width, disparity, and features. The first category usually uses the single-feature 3D cost volume generated by full correlation, which is efficient while losing much information due to the decimation of feature channels. Many real-time methods, such as Dispnet \cite{dispnet}, MADNet \cite{madnet, madnet_pami} and AANet \cite{aanet}, belongs to the category. Moreover, two-stage refinement \cite{mcvmfc} and pyramidal towers \cite{madnet} are commonly applied in the single-feature cost volume based network to construct multi-scale cost volume. The second category usually uses the multi-feature 4D cost volume generated by concatenation \cite{gcnet} or group-wise correlation \cite{gwcnet}, which can achieve better performance with higher computational complexity and memory consumption. Most top-performing networks, including GANet \cite{ganet}, CSPN \cite{cspn} and ACFNet \cite{acfnet} belong to this category. 
% In these methods, the 4D cost volume is constantly regularized by multiple 3D convolution layers and then a discrete disparity probability distribution can be produced by softmax. Next, the final predicted disparity can be obtained by softly weighting indices according to their probability \cite{gcnet}. However, such output is susceptible to multimodal disparity probability distributions and ACFNet \cite{acfnet} gives a solution by directly supervising the cost volume with unimodal ground truth distributions to alleviate this problem. 
Recently, to alleviate the high computational complexity and memory consumption when employing multi-feature 4D cost volumes, \cite{cvpmvsnet, cascade, uscnet} propose to use cascade cost volume representation in multi-view stereo. These methods usually first predict an initial disparity at the coarsest resolution of the image and then gradually refine the disparity by narrowing down the disparity search space. More closely related to our approach is Casstereo \cite{cascade}, which first extended such representation to stereo matching. It selected to uniform sample a pre-defined range to generate the next stage’s disparity search range. Instead, we employ pixel-level uncertainty estimation to adaptively adjust the next stage disparity searching range and generate pseudo-labels for subsequent domain adaptation. Our method also shares similarities with UCSNet \cite{uscnet}, which constructs uncertainty-aware cost volume in multi-view stereo while it doesn’t employ uncertainty estimation to generate pseudo-labels.

%\subsection{Multi-scale Cost Volume based Deep Stereo Matching} 
% \subsection{Multi-scale Cost Volume based Stereo Matching} 
%Multi-scale cost volume firstly was applied in the single-feature cost volume based network with the form of two-stage refinement \cite{mcvmfc} and pyramidal towers \cite{madnet}. Recently, cascade cost volume representation \cite{cvpmvsnet, cascade, uscnet} was proposed in multi-view stereo to alleviate the high computational complexity and memory consumption when employing multi-feature 4D cost volumes. These methods generally predict an initial disparity at the coarsest resolution of the image. Then, they will narrow down the disparity search space and gradually refine the disparity. More closely related to our approach is Casstereo \cite{cascade}, which first extended such representation to stereo matching. It selected to uniform sample a pre-defined range to generate the next stage’s disparity search range. Instead, we employ uncertainty estimation to adaptively adjust the next stage pixel-level disparity searching range and push the next stage's cost volume to be predominantly unimodal.

% The single-feature cost volume based network with the form of two-stage refinement \cite{mcvmfc} and pyramidal towers \cite{madnet} first employ multi-scale cost volume for stereo matching. Recently, to alleviate the high computational complexity and memory consumption when employing multi-feature 4D cost volumes, \cite{cvpmvsnet, cascade, uscnet} propose to use cascade cost volume representation in multi-view stereo, which generally predict an initial disparity at the coarsest resolution of the image. Then, the disparity search space is narrowed down and the disparity is gradually refined. More closely related to our approach is Casstereo \cite{cascade}, which first extended such representation to stereo matching. It selected to uniform sample a pre-defined range to generate the next stage’s disparity search range. Instead, we employ uncertainty estimation to adaptively adjust the next stage pixel-level disparity searching range and push the next stage's cost volume to be predominantly unimodal.

% Figure environment removed

\subsection{Robust Stereo Matching} 
There exist three categories of generalization definitions for robust stereo matching. 1) Cross-domain Generalization: the network’s ability to perform well on unseen scenes (cannot see the image pairs of the target domain in advance). Towards this end, Jia et al \cite{sungeneralizaiton} propose to incorporate scene geometry priors into an end-to-end network. Zhang et al \cite{dsmnet} introduce a domain normalization and a trainable non-local graph-based filter to construct a domain-invariant stereo matching network. 2) Adapt Generalization: the network’s ability to adapt pre-trained models to the new domain with unlabeled target data. Previous work usually pre-trains the models on synthetic data and then adapts it to new target domains with Graph Laplacian regularization \cite{zoom}, non-adversarial progressive color transfer \cite{adastereo}, and Knowledge Reverse Distillation \cite{aohnet}. More closely related to our approach are \cite{aohnet, unsuperviseddomainadaptation} in stereo matching and Monoresmatch \cite{monoresmatch} in monocular depth estimation, which also proposes to generate a pseudo-label for domain adaptation. However, these methods all select to employ classical stereo matching methods \cite{sgm} alongside with confidence estimators, e.g., left-right consistency check to generate pseudo-labels. That is all these methods need an independent method to generate corresponding pseudo-labels. Instead, the proposed method is an end-to-end network that can generate the predicted disparity map, corresponding uncertainty map and pseudo-labels jointly, which is a more simple, yet efficient way. 
% Instead, our proposed method can employ pixel-level and area-level uncertainty estimation to self-distill the predicted disparity maps of our pre-training model and generate sparse while reliable pseudo-labels to align the domain gap, which is a more simple, yet efficient way. 
3) Joint Generalization: the network’s ability to perform well on a variety of datasets with the same model parameters. MCV-MFC \cite{mcvmfc} introduces a two-stage finetuning scheme to achieve a good trade-off between generalization and fitting capability on multiple datasets. However, it doesn’t touch the inner difference between diverse datasets, e.g, the unbalanced disparity distribution. To further address this problem, we propose a cascade cost volume to adaptively the next stage disparity searching space, where the pixel-level uncertainty estimation is at the core.

% \subsection{Monocular Depth Estimation}
% Monocular depth estimation aims to estimate depth values from a single image, instead of stereo images or multiple frames in a video. This problem is ill-posed because of the ambiguity of object sizes. However, humans could estimate the depth from a single image with prior knowledge of the scenes. Recently, learning based methods were explored to learn depth values by supervised or unsupervised learning. Eigen et al. first employed Convolutional Neural Networks (CNN) to predict depth in a coarse-to-fine manner and further improved its performance by multi-task learning. Liu et al. presented deep convolutional neural fields model by combining deep model with continuous CRF. Li et al. [22] refined deep CNN outputs with a hierarchical CRF. Multi-scale continuous CRF was formulated into a deep sequential network by Xu et al. [45] to refine depth estimation. Unsupervised methods tried to train monocular depth estimation with stereo
% image pairs or image sequences and test on single images. Garg et al. [9] used novel image view synthesis loss to train a depth estimation network in an unsupervised way. Godard et al. [11] introduced left-right consistency regularization to improve the performance of view synthesis loss. Recently, some work also propose to use the stereo matching network as a proxy to learn depth from synthetic data or directly employ traditional stereo matching methods to distill proxies labels from the target domain, which proves the feasibility of distilling stereo matching networks to learn monocular depth estimation.



\section*{Appendix -- BayesDAG: Gradient-Based Posterior Sampling for Causal Discovery}
\section{Joint Inference with SG-MCMC}
\label{appsec: joint inference SG-MCMC}
\begin{wrapfigure}[12]{r}{0.4\textwidth}
   \usetikzlibrary{bayesnet}
\scalebox{0.7}{
\begin{tikzpicture}[scale=3]

  % Define nodes
  \node[latent] (Wtilde) {$\tilde{W}$};
  \node[latent, right=of Wtilde] (W) {$W$};
  \node[latent, above=of W] (p) {$p$};
  \node[latent, right=of W] (Theta) {$\Theta$};
  \node[obs, below=of Theta] (x) {$\vx^{(i)}$};
  
  % Connect nodes
  \edge {Wtilde} {W};
  \edge {p} {x};
  \edge {W} {Theta};
  \edge {W} {x};
  \edge {Theta} {x};
  \edge {p} {Theta};
  
  % Define plate
  \plate {x_plate} {(x)} {$i = 1, \dots, N$};

\end{tikzpicture}
}
   \caption{Graphical model with latent variable $\tmW$.}
   \label{fig: graphical model with latent}
\end{wrapfigure}

In this section, we propose an alternative formulation that enables a joint inference framework for $\vp,\mW,\mTheta$ using SG-MCMC, thereby avoiding the need for variational inference for $\mW$.

We adopt a continuous relaxation of $\mW$, similar to \cite{lorch2021dibs}, by introducing a latent variable $\tmW$. The graphical model is illustrated in \cref{fig: graphical model with latent}.  We can define 
\begin{equation}
   p(\mW|\tmW) = \prod_{i,j} p(W_{ij}|\tilde{W}_{ij})
   \label{eq: W and tilde W forward prob}
\end{equation}
with $p(W_{ij}=1|\tilde{W}_{ij})=\sigma(\tilde{W}_{ij})$ where $\sigma(\cdot)$ is the sigmoid function. In other words, $\tilde{W}{ij}$ defines the existence logits of $W{ij}$.

With the introduction of $\tmW$, the original posterior expectations of $p(\vp,\mW,\mTheta\vert \mD)$, e.g.~during evaluation, can be translated using the following proposition. 
\begin{proposition}[Equivalence of posterior expectation]
   Under the generative model \cref{fig: graphical model with latent}, we have
   \begin{equation}
       \E_{p(\vp,\mW,\mTheta|\mD)}\left[
       \vf(\mG=\tau(\vp,\mW),\mTheta)
       \right] = \E_{p(\vp,\tmW,\mTheta)}\left[
       \frac{
       \E_{p(\mW\vert \tmW)}\left[f(\mG,\mTheta)p(\mD,\mTheta\vert \vp,\mW)\right]
       }{
       \E_{p(\mW\vert \tmW)}\left[p(\mD,\mTheta\vert \vp, \mW)\right]
       }
       \right]
       \label{eq: evaluation equivalence}
   \end{equation}
where $\vf$ is the target quantity.
\label{prop: equivalence of evaluation}
\end{proposition}

This proof is in \cref{subapp: proof of prop equivalence evaluation}.

With this proposition, instead of sampling $\mW$, use SG-MCMC to draw $\tmW$ samples. 
Similar to \cref{subsubsec: combined inference}, to use SG-MCMC for $\vp,\tmW,\mTheta$, we need their gradient information. The following proposition specifies the required gradients.
\begin{proposition}
   With the generative model defined as \cref{fig: graphical model with latent}, we have
   \begin{align}
       \nabla_{\vp,\mTheta,\tmW} U(\vp,\tmW,\mTheta) = &-\nabla_{\vp} \log p(\vp) - \nabla_{\mTheta}\log p(\mTheta) \nonumber \\
       &-\nabla_{\tmW}\log p(\tmW) - \nabla_{\vp,\mTheta,\tmW} \log \E_{p(\mW\vert \tmW)}[p(\mD\vert \mW,\vp,\mTheta)]
       \label{eq: joint inference gradient}
   \end{align}
   \label{prop: joint inference gradient}
\end{proposition}
The proof is in \cref{subapp: joint inference gradient}.

With these gradients, we can directly plug in existing SG-MCMC samplers to draw samples for $\vp, \tmW$, and $\mTheta$ in joint inference (\cref{alg: joint inference}). 

\begin{algorithm}[tb]
\caption{Joint inference}
\label{alg: joint inference}
\begin{algorithmic}
\STATE {\bfseries Input:} dataset $\mD$, prior $p(\vp,\tmW,\mTheta)$, SG-MCMC sampler update $\sampler(\cdot)$; sampler hyperparameter $\Psi$; training steps $T$.
\STATE {\bfseries Output:} posterior samples $\{\vp,\tmW,\mTheta\}$
\STATE Initialize $\vp_0,\tmW_0, \mTheta_0$
\FOR{$t=1,\ldots, T$}
   \STATE Evaluate gradient $\nabla_{\vp_{t-1},\tmW_{t-1},\mTheta_{t-1}} U$ based on \cref{eq: joint inference gradient}.
   \STATE Update samples $\vp_t,\tmW_t,\mTheta_t = \sampler(\nabla_{\vp_{t-1},\tmW_{t-1},\mTheta_{t-1}} U; \Psi)$
   \IF{storing condition met}
       \STATE $\{\vp,\tmW,\mTheta\}\leftarrow \vp_t,\tmW_t, \mTheta_t$
   \ENDIF
\ENDFOR
\end{algorithmic}
\end{algorithm}
\section{Theory}
\label{app: Theory}
\subsection{Proof of \cref{thm: equivalence of bayesian inference}}
\label{subapp: proof of equivalence of bayesian inference}
For completeness, we recite the theorem here.
\begin{reptheorem}{thm: equivalence of bayesian inference}[Equivalence of inference in $(\mW,\vp)$ and binary DAG space]
Assume graph $\mG$ is a binary adjacency matrix representing a DAG and node potential $\vp$ does not contain the same values, i.e.~$p_i\neq p_j$ $\forall i,j$. Then, with the induced joint observational distribution $p(\mD,\mG)$, dataset $\mD$ and a corresponding prior $p(\mG)$, we have
\begin{align}
    p(\mG\vert \mD) = \int p_\tau(\vp,\mW\vert \mD)\indicator(\mG=\tau(\mW,\vp))d\mW d\vp
    \label{eq: equivalence of bayesian inference}
\end{align}
if $p(\mG)=\int p_\tau(\vp,\mW)\indicator(\mG=\tau(\mW,\vp))d\mW d\vp$, where $p_\tau(\mW,\vp)$ is the prior, $\indicator(\cdot)$ is the indicator function and $p_\tau(\vp,\mW\vert D)$ is the posterior distribution over $\vp,\mW$. 
\end{reptheorem}


To prove this theorem, we first prove the following lemma stating the equivalence of $\tau$ (\cref{eq: Binary NoCurl}) to binary DAG space. 
\begin{lemma}[Equivalence of $\tau$ to DAG space]
Consider $d$ random variables, a node potential vector $\vp\in\sR^d$ and a binary matrix $\mW\in\binaryset$. Then the following holds:
\begin{enumerate}[(a)]
\item For any $\mW\in\binaryset$, $\vp\in\sR^d$, $\mG=\tau(\mW,\vp)$ is a DAG.
    \item For any DAG $\mG\in \sD$, where $\sD$ is the space of all DAGs, there exists a corresponding $(\mW,\vp)$ such that $\tau(\mW,\vp)=\mG$.
\end{enumerate}
\label{lemma: Equivalence of DAG space}
\end{lemma}
\begin{proof}
The main proof directly follows the theorem 2.1 in \cite{yu2021dags}. For (a), we show the output from $\tau(\mW,\vp)$ must be a DAG. 
By leveraging the Lemma 3.4 in \cite{yu2021dags}, we can easily obtain that $\step(\grad \vp)$ emits a binary adjacency matrix representing a DAG. The only difference is that we replace the $\relu(\cdot)$ with $\step(\cdot)$ but the conclusion can be directly generalized.

For (b), we show that for any DAG $\mG$, there exists a $(\mW,\vp)$ pair s.t. $\tau(\mW,\vp)=\mG$. To see this, we can observe that $\vp$ implicitly defines a topological order in the mapping $\tau$. For any $p_i>p_j$, we have $j\rightarrow i$ after the mapping $\step(\grad \vp)$. Thus, by leveraging Theorem 3.7 in \cite{yu2021dags}, we obtain that there exists a potential vector $\vp\in\sR^d$ for any DAG $\mG$ such that 
$$
(\grad\vp)(i,j)>0 \;\;\;\;\;\text{when}\;G_{ij}=1
$$
Thus, we can choose $\mW$ in the following way:
$$
\mW=\begin{cases}
&W_{ij}=0\;\;\;\;\; \text{if}\; G_{ij}=0\\
&W_{ij}=1\;\;\;\;\; \text{if}\; G_{ij}=1
\end{cases}
$$
\end{proof}
Next, let's prove the \cref{thm: equivalence of bayesian inference}.

\begin{proof}[Proof of \cref{thm: equivalence of bayesian inference}]
    From \cref{lemma: Equivalence of DAG space}, we see that the mapping is complete. Namely, the $(\mW,\vp)$ space can represent the entire DAG space. Next, we show that performing Bayesian inference in $(\mW,\vp)$ space can also correspond to the inference in DAG space. 

    Assume we have the prior $p_\tau(\mW,\vp)$. Then through mapping $\tau$, we implicitly define a prior over the DAG $\mG$ in the following:
    \begin{equation}
        p_\tau(\mG) = \int p_\tau(\mW,\vp)\indicator(\mG=\tau(\mW,\vp))d\mW d\vp
        \label{eq: accumulated prior}
    \end{equation}
    This basically states that the corresponding prior over $\mG$ is an accumulation of the corresponding probability associated with $(\mW,\vp)$ pairs.

    Similarly, we can define a corresponding posterior $p_\tau(\mG\vert \mD)$:
    \begin{equation}
        p_\tau(\mG\vert \mD) = \int p_\tau(\mW,\vp\vert \mD)\indicator(\mG=\tau(\mW,\vp))d\mW d\vp
        \label{eq: accumulated posterior}
    \end{equation}
    Now, let's show that this posterior $p_\tau(\mG\vert \mD) = p(\mG\vert \mD)$ if prior matches, i.e.~$p(\mG)=p_\tau(\mG)$.
    From Bayes's rule, we can easily write down
    \begin{equation}
        p_\tau(\mW,\vp\vert \mD) = \frac{p(\mD\vert \mG=\tau(\mW,\vp))p(\vp,\mW)}{\sum_{\mG'\in\sD}p(\mD,\mG')}
        \label{eq:def of W p posterior}
    \end{equation}
    Then, by substituting \cref{eq:def of W p posterior} into \cref{eq: accumulated posterior}, we have
    \begin{align}
        p_\tau(\mG\vert \mD) &= \int \frac{p(\mD\vert \mG)p_\tau(\mW,\vp)}{\sum_{\mG'\in\sD}p(\mD,\mG')}\indicator(\mG=\tau(\mW,\vp))d\mW d\vp\nonumber \\
        &=\frac{\int p(\mD\vert \mG)p_\tau(\mW,\vp)\indicator(\mG=\tau)d\mW d\vp}{\sum_{\mG'\in\sD}p(\mD,\mG')} \label{eq: proof 3.1 eq 1}\\
        &=\frac{p(\mD\vert \mG)\int p_\tau(\mW,\vp)\indicator(\mG=\tau)d\mW d\vp}{\sum_{\mG'\in\sD}p(\mD,\mG')}\label{eq: proof 3.1 eq 2}\\
        &=\frac{p(\mD\vert \mG)p_\tau(\mG)}{\sum_{\mG'\in\sD}p(\mD\vert\mG')p_\tau(\mG')}\nonumber\\
        &=p(\mG\vert \mD) \label{eq: proof 3.1 eq 3}
    \end{align}
where \cref{eq: proof 3.1 eq 1} is from the fact that $\sum_{\mG'\in\sD}p(\mD,\mG')$ is independent of $(\mW,\vp)$ due to marginalization. \cref{eq: proof 3.1 eq 2} is obtained because $p(\mD\vert \mG)$ is also independent of $(\mW,\vp)$ due to (1) $\indicator(\mG=\tau(\mW,\vp))$ and (2) $p(\mD\vert \mG)$ is a constant when fixing $\mG$. \cref{eq: proof 3.1 eq 3} is obtained by applying Bayes's rule and $p_\tau(\mG) = p(\mG)$. 
\end{proof}



\subsection{Proof of \cref{thm: equivalence of alternative formulation}}
\label{subapp: proof of theorem alternative formulation}

\begin{reptheorem}{thm: equivalence of alternative formulation}[Equivalence of NoCurl formulation]
Assuming the conditions in \cref{thm: equivalence of bayesian inference} are satisfied. Then, for a given $(\mW,\vp)$, we have
$$
\mG=\mW\odot \step(\grad \vp) = \mW\odot \left[ \perm^*(\vp)\mL\perm^*(\vp)^T\right]
$$
where $\mG$ is a DAG and $\perm^*(\vp)$ is defined in \cref{eq: alternative formulation}. 
\end{reptheorem}

To prove this theorem, we need to first prove the following lemma. 

\begin{lemma}
    For any permutation matrix $\mM\in\bm{\Sigma}_d$, we have
    $$
    \grad(\mM \vp) = \mM^T\grad(p) \mM
    $$
    where $\grad$ is the operator defined in \cref{eq: grad operator}.
    \label{lemma: grad with permutation matrix}
\end{lemma}
\begin{proof}
    By definition of $\grad(\cdot)$, we have
    \begin{align*}
        \grad(\mM\vp) &= (\mM\vp)_i - (\mM\vp)_j\\
        &=\bm{1}(i)^T\mM\vp - \bm{1}(j)^T\mM\vp\\
        &=\mM_{i,:}\vp - \mM_{j,:}\vp\\
    \end{align*}
    where $\bm{1}(i)$ is a one-hot vector with $i^{\text{th}}$ entry $1$, and $\mM_{i,:}$ is the $i^{\text{th}}$ row of matrix $\mM$. The above is equivalent to computing the $\grad$ with new labels obtained by permuting $\vp$ with $\mM$. Therefore, we can see that $\grad(\mM\vp)$ can be computed by permuting the original $\grad(\vp)$ by matrix $\mM$.
    $$
    \grad(\mM\vp)=\mM^T\grad(\vp)\mM
    $$
\end{proof}

\begin{proof}[Proof of \cref{thm: equivalence of alternative formulation}]
Since $\mW$ plays the same role in both formulations, we focus on the equivalence of $\step(\grad(\cdot))$.

Define a sorted $\tilde{\vp}= \perm \vp$, where $\perm\in\bm{\Sigma}_d$, such that for $i<j$, we have $\tilde{p}_i > \tilde{p}_j$. Namely, $\perm$ is a permutation matrix.
Thus, we have 
$$
\grad(\vp) = \grad(\perm^T\tilde{\vp}).
$$
By \cref{lemma: grad with permutation matrix}, we have
$$
\grad(\perm^T\tilde{\vp}) = \perm\grad(\tilde{\vp}) \perm^T.
$$
Since $\tilde{\vp}$ is an ordered vector. Therefore, $\grad(\tilde{\vp})$ is a skew-symmetric matrix with a positive lower half part. 

Therefore, we have
$$
\step(\grad(\vp)) = \step(\perm\grad(\tilde{\vp})\perm^T) = \perm\step(\grad(\tilde{p}))\perm^T =  \perm\mL\perm^T
$$
This is true because $\perm$ is just a permutation matrix that does not alter the sign of $\grad(\tilde{\vp})$.

Since $\perm$ is a permutation matrix that sort $\vp$ value in a ascending order, from Lemma 1 in \cite{blondel2020fast}, we have
$$
\perm = \argmax_{\perm'\in\bm{\Sigma}_d}\vp^T(\perm'\vo)
$$
\end{proof}

\subsection{Proof of \cref{prop: equivalence of evaluation}}
\label{subapp: proof of prop equivalence evaluation}
\begin{proof}
    \begin{align*}
        &\E_{p(\vp, \mW,\mTheta\vert \mD)}\left[f(\mG=\tau(\vp,\mW), \mTheta)\right]\\
        =& \int p(\vp,\mW,\mTheta, \tmW\vert \mD)f(\mG,\mTheta)d\vp d\mW d\mTheta d\tmW\\
        =&\int p(\vp, \tmW,\mTheta\vert \mD)p(\mW\vert \vp,\mTheta,\tmW,\mD)f(\mG,\mTheta)d\vp d\mW d\mTheta d\tmW\\
        =& \E_{p(\vp,\tmW,\mTheta\vert \mD)}\left[
        \frac{
        \int p(\mD\vert \vp,\mTheta,\mW)p(\vp)p(\tmW)p(\mW\vert \tmW)p(\mTheta\vert \vp,\mW)f(\mG,\mTheta)d\mW
        }{\int p(\mD\vert \vp,\mTheta,\mW)p(\vp)p(\tmW)p(\mW\vert \tmW)p(\mTheta\vert \vp,\mW)d\mW}
        \right]\\
        =&\E_{p(\vp,\tmW,\mTheta)}\left[
        \frac{
        \E_{p(\mW\vert \tmW)}\left[f(\mG,\mTheta)p(\mD,\mTheta\vert \vp,\mW)\right]
        }{
        \E_{p(\mW\vert \tmW)}\left[p(\mD,\mTheta\vert \vp, \mW)\right]
        }
        \right]
    \end{align*}
\end{proof}

\subsection{Proof of \cref{prop: joint inference gradient}}
\label{subapp: joint inference gradient}
\begin{proof}
    \begin{align*}
\nabla_\vp U(\vp,\tmW,\mTheta) &= -\nabla_{\vp} \log p(\vp,\tmW,\mTheta, \mD)\\
&=-\nabla_{\vp}\log p(\vp) - \nabla_{\vp} \log p(\tmW,\mTheta,\mD\vert \vp)\\
&=-\nabla_{\vp}\log p(\vp) - \frac{
\nabla_{\vp}\int p(\mD\vert \mW,\vp,\mTheta)p(\mTheta\vert \vp,\mW)p(\mW\vert \tmW)p(\tmW)d\mW
}{
\int p(\mD\vert \mW,\vp,\mTheta)p(\mTheta\vert \vp,\mW)p(\mW\vert \tmW)p(\tmW)d\mW
}\\
&=-\nabla_{\vp}\log p(\vp) - \frac{
\nabla_{\vp}\E_{p(\mW\vert\tmW)}\left[p(\mD\vert \mW,\vp,\mTheta)\right]
}{\E_{p(\mW\vert\tmW)}\left[p(\mD\vert \mW,\vp,\mTheta)\right]}\\
&=-\nabla_{\vp}\log p(\vp) - \nabla_{\vp} \log \E_{p(\mW\vert\tmW)}\left[p(\mD\vert \mW,\vp,\mTheta)\right]
    \end{align*}

Other gradient $\nabla_{\tmW} U$ and $\nabla_{\mTheta} U$ can be derived using the similar approach, which concludes the proof.
\end{proof}

\subsection{Proof of \cref{prop: gradient computation}}
\label{subapp: proof of proposition gradient computation}
\begin{proof}[Proof of \cref{prop: gradient computation}]
    By definition, we have easily have
    \begin{align*}
        \nabla_{\vp} U &= -\nabla_{\vp} \log p(\vp,\mW,\mTheta,\mD)\\
        &=-\nabla_{\vp}\log p(\vp,\mW) - \nabla_{\vp}\log p(\mD,\mTheta\vert \tau(\mW,\vp))\\
        &=-\nabla_{\vp}\log p(\vp,\mW) - \nabla_{\vp}\log p(\mD\vert \mTheta, \tau(\mW,\vp)) + \underbrace{\nabla_{\vp} \log p(\mTheta\vert \tau(\vp,\mW))}_{0}
    \end{align*}
Similarly, we have
\begin{align*}
    \nabla_{\mTheta} U &=  -\nabla_{\mTheta} \log p(\vp,\mW,\mTheta, \mD)\\
    &=-\nabla_{\mTheta} \log p(\mD\vert \mTheta,\tau(\mW,\vp)) - \nabla_{\mTheta}\log p(\mTheta\vert \vp,\mW) - \underbrace{\nabla_{\mTheta} \log p(\vp,\mW)}_{0}\\
    &= -\nabla_{\mTheta} \log p(\mD\vert \mTheta,\tau(\mW,\vp)) - \nabla_{\mTheta}\log p(\mTheta)
\end{align*}
\end{proof}

\subsection{Derivation of ELBO}
\label{subapp: derivation of ELBO}

\begin{align}
    \log p(\vp,\mTheta,\mD)&= \log \int p(\vp,\mTheta,\mD,\mW)d\mW \nonumber \\
    &= \log \int \frac{q_\phi(\mW\vert \vp)}{q_\phi(\mW\vert \vp)}p(\vp,\mTheta,\mD,\mW)d\mW \nonumber\\
    &\geq \int q_\phi(\mW\vert \vp)\log p(\vp,\mTheta,\mD\vert \mW)d\mW + \int q_\phi(\mW\vert \vp)\log \frac{p(\mW)}{q_\phi(\mW\vert \vp)}d\mW \label{eq: derivation of ELBO}\\
    &=\E_{q_\phi(\mW\vert \vp)}\left[\log p(\vp,\mTheta,\mD\vert \mW)\right]-\KL\left[q_\phi(\mW\vert\vp)\Vert p(\mW)\right]\nonumber
\end{align}
where the \cref{eq: derivation of ELBO} is obtained by Jensen's inequality.
\section{SG-MCMC Update}
\label{app: SG-MCMC update}

Assume we want to draw samples $\vp\sim p(\vp\vert \mD,\mW,\mTheta)\propto \exp(-U(\vp,\mW,\mTheta))$ with $U(\vp,\mW,\mTheta)=-\log p(\vp,\mW,\mTheta)$, we can compute $U$ by
\begin{equation}
    U(\vp,\mW,\mTheta) = -\sum_{n=1}^N\log p(\vx_n|\mG=\tau(\mW,\vp),\mTheta) - \log p(\vp,\mW,\mTheta)
    \label{eq: def of U}
\end{equation}
In practice, we typically use mini-batches $\mathcal{S}$ instead of the entire dataset $\mD$. Therefore, an approximation is 
\begin{equation}
    \tilde{U}(\vp,\mW,\mTheta) = -\frac{\vert \mD\vert}{\vert \mathcal{S}\vert}\sum_{n\in\mathcal{S}} \log p(\vx_n\vert \mG=\tau(\mW,\vp),\mTheta) -\log p(\vp,\mW,\mTheta)
    \label{eq: def of minibatch U}
\end{equation}
where $\vert \mathcal{S}\vert$ and $\vert \mD\vert$ are the minibatch and dataset sizes, respectively.

\cite{gong2019icebreaker} uses the preconditioning technique on \emph{stochastic gradient Hamiltonian Monte Carlo}(SG-HMC), similar to the preconditioning technique in \cite{li2016preconditioned}. In particular, they use a moving-average approximation of diagonal Fisher information to adjust the momentum.
The transition dynamics at step $t$ with EM discretization is 
\begin{align}
    &B = \frac{1}{2}l \nonumber\\
    &\mV_t = \beta_2 \mV_{t-1} + (1-\beta_2)\nabla_{\vp}\tilde{U}(\vp,\mW,\mTheta)\odot \nabla_{\vp}\tilde{U}(\vp,\mW,\mTheta)\nonumber\\
    &g_t=\frac{1}{\sqrt{1+\sqrt{\mV_t}}}\nonumber\\
&\vr_t = \beta_1\vr_{t-1} - lg_t\nabla_{\vp}\tilde{U}(\vp,\mW,\mTheta) + l\frac{\partial g_t}{\partial \vp_t} + s\sqrt{2l(\frac{1-\beta_1}{l}-B)}\eta\nonumber \nonumber \\
&\vp_t=\vp_{t-1}+lg_{t}\vr_t
\label{eq: SGMCMC update rule}
\end{align}
where $l^2$ is the learning rate; $(\beta_1,\beta_2)$ controls the preconditioning decay rate, $\eta$ is the Gaussian noise with $0$ mean and unit variance, and $s$ is the hyperparameter controlling the level of injected noise to SG-MCMC.
Throughout the paper, we use $(\beta_1, \beta_2)= (0.9,0.99)$ for all experiments. 




\section{Experiments}
% \haizhou{Follow the same way of introduction as we did in Section2.}
% \noindent In this section, we will introduce datasets and experimental setups that we used. Then we evaluate our method, other self-supervised methods, and supervised methods under different distribution shifts (\ie, concept shifts and covariate shifts) under common settings (\ie, transductive, inductive settings). It has to note that we focus on node-level tasks (\eg, node classification) in this work. As for graph-level tasks, we leave it as our future work and some simple experiments can be found in Appendix~\ref{app:graph_classification}. 
In this section, we first introduce the experimental setup including datasets, training, and evaluation protocol in Section~\ref{sec:dataset}~and~\ref{sec:unsupervised}. 
% Next, we present our experimental setup and conduct extensive experiments to evaluate our method in Section~\ref{sec:unsupervised}. 
We then perform an ablation study to demonstrate the effectiveness of each proposed component in Section~\ref{sec:ablation}. 
Additionally, we analyze the impact of important hyper-parameters in Section~\ref{sec:sensitivity}. 
Subsequently, we integrate our method with various encoding models, showcasing the model-agnostic nature of our recipe in Section~\ref{sec:other_models}. 
Finally, we provide some qualitative results such as feature visualization in Section~\ref{sec:vis}.
It is important to note that we focus on node-level tasks (\eg, node classification) in this work. As for graph-level tasks, we leave it as our future work, while some simple experiments are also provided in Appendix~\ref{app:graph_classification}.

\subsection{Datasets}\label{sec:dataset}
There exist some benchmarks for evaluating graph out-of-distribution generalization~\cite{good,ji2022drugood,gds}. 
Among them, GOOD~\cite{good} is the most representative and comprehensive benchmark that curates more diverse graph datasets with diverse tasks, including single/multi-task graph classification, graph regression, and node classification involving more distribution shifts (\ie, concept shifts and covariate shifts). Hence in this work, we follow the evaluation protocol proposed in \cite{good}. Furthermore, we validate the effectiveness of our method in the datasets (\ie, Amazon-Photo, Elliptic) that are used in EERM~\cite{eerm}. The statistics and detailed introduction to these datasets can be found in Table~\ref{tab:dataset} and Appendix~\ref{app:datasets}.

\begin{table*}[htp]
\caption{The descriptions of datasets. ``Domain-Level'' means splitting by graphs, ``Time-Aware'' denotes splitting according to chronological order.``Word'' and ``Degree'' represent splitting according to word diversity and node degree respectively. ``Language'' means splitting by user language, suggesting the prediction should not be impacted by the language the user use. ``University'' denotes splitting according to the domain university, implying that the prediction of webpages should be based on word contents and link connections rather than university features. ``Color'' means that nodes are split according to node differences in covariate shift and color-label correlations in concept shift.}
\label{tab:dataset}
\centering
\begin{tabular}{cccccccc}
\toprule
Datasets     & Network Type        & \#Nodes & \#Edges & \#Attributes &\#Classes& Train/Val/Test Split     & Metric   \\
% Cora         & Artificial Transformation & 2,703   &         &              &         &                      & Accuracy \\
Amazon-Photo\footnotemark
             & Co-purchasing network      & 7,650   & 119,081   & 755          & 10      & Domain-Level         & Accuracy \\
Elliptic\footnotemark  
             & Bitcoin transactions       & 203,769 & 234,355   & 165          & 2       & Time-Aware           & F1-Score \\
GOOD-Cora    & Scientific publications    & 19,793  & 126,842   & 8,710         & 70      & Word/Degree          & Accuracy \\
% GOOD-Arxiv   & arXiv papers               & 169,343 & 2,315,598 & 128          & 40      & Time/Degree          & Accuracy \\
GOOD-Twitch  & Gamer network              & 34,120  & 892,346   & 128          & 2       & Language             & ROC-AUC  \\
GOOD-CBAS    & A BA-house graph           & 700     & 3,962     & 4             & 4       & Color                & Accuracy \\
GOOD-WebKB   & Webpage network            & 617     & 1,138     & 1,703         & 5       & University           & Accuracy \\
\bottomrule
\end{tabular}
\end{table*}
\footnotetext[5]{This dataset is adopted from~\cite{yang2016revisiting}. \cite{eerm} constructs ten graphs with different environment id’s for each graph.} 
\footnotetext[6]{The original is available on \hyperlink{https://www.kaggle.com/ellipticco/elliptic-data-set}{https://www.kaggle.com/ellipticco/elliptic-data-set}}

\subsection{Unsupervised Representation Learning}\label{sec:unsupervised}
\subsubsection{Transductive Setting}~\label{sec:trans}
% \noindent\textbf{Baselines.}\quad We conduct experiments with 12 baselines which consist of three categories: supervised methods and self-supervised generative methods, self-supervised contrastive methods. Specifically, we compare with three supervised baselines: empirical risk minimization~(ERM)~\cite{erm}, invariant risk minimization (IRM)~\cite{irm}, and a recent proposed graph OOD method dubbed EERM~\cite{eerm}. We also compare various unsupervised node-level representation learning methods: three self-supervised generative methods including GAE~\cite{gae}, VGAE~\cite{gae}, GraphMAE~\cite{gmae} and seven self-supervised contrastive methods: DGI~\cite{dgi}, MVGRL~\cite{mvgrl}, GRACE~\cite{grace}, RoSA~\cite{rosa}, BGRL~\cite{bgrl}, COSTA~\cite{costa}, SwAV~\cite{swav}. The descriptions of these methods can be found in Appendix~\ref{app:baselines}.
In this subsection, we focus on validating our proposed algorithm under the transductive setting, where the test nodes will participate in message passing~\cite{gilmer2017neural} during training following~\cite{good}. 

\noindent\textbf{Baselines.} We conduct experiments with 12 baselines from three categories: (i)~supervised methods, including empirical risk minimization~(\textbf{ERM})~\cite{erm}, invariant risk minimization (\textbf{IRM})~\cite{irm}, and a recent proposed graph OOD method \textbf{EERM}~\cite{eerm}; (ii)~self-supervised generative methods including Graph Autoencoder (\textbf{GAE})~\cite{gae}, Variational Graph Autoencoder (\textbf{VGAE})~\cite{gae}, Self-Supervised Masked Graph Autoencoders (\textbf{GraphMAE})~\cite{gmae}; (iii)~self-supervised contrastive methods including Deep Graph Infomax (\textbf{DGI})~\cite{dgi}, Contrastive Multi-View Representation Learning on Graphs (\textbf{MVGRL})~\cite{mvgrl}, Deep Graph Contrastive Representation Learning (\textbf{GRACE})~\cite{grace}, A Robust Self-Aligned Framework for Node-Node Graph Contrastive Learning (\textbf{RoSA})~\cite{rosa}, Bootstrapped Representation Learning on Graphs (\textbf{BGRL})~\cite{bgrl}, Covariance-Preserving Feature Augmentation for Graph Contrastive Learning (\textbf{COSTA})~\cite{costa}, Unsupervised Learning of Visual Features by Contrasting Cluster Assignments (\textbf{SwAV})~\cite{swav}. The detailed descriptions of these baselines can be found in Appendix~\ref{app:baselines}.

\noindent\textbf{Experimental setup.} We use the same graph encoder across different datasets for a fair comparison following~\cite{good}. We use grid search to find other hyper-parameters (\eg, learning rate, epochs) for different methods. For all experiments, we select the best checkpoints for ID and OOD tests according to results on ID and OOD validation sets following~\cite{good}, respectively. Experimental details and hyper-parameter selections are provided in Appendix~\ref{app:hyper}. For evaluating unsupervised methods, a linear classifier will be built on the frozen trained encoder after finishing pre-training. The reported results are the mean performance with standard deviation after 10 runs following~\cite{good}.

\noindent\textbf{Analysis.}\quad Based on the experimental results listed in Table~\ref{tab:trans_concept} and \ref{tab:trans_covariate}, we can draw the following conclusions: firstly, we find strong self-supervised methods (\eg, GRACE, BGRL, COSTA) are more robust to distribution shifts (concept shift in Table~\ref{tab:trans_concept} and covariate shift in Table~\ref{tab:trans_covariate}) compared to supervised methods. For instance, on GOOD-CBAS and GOOD-WebKB datasets, GRACE surpasses the best supervised method by large margins (over 6\% absolute improvement). Interestingly, we find the methods designed for OOD generalization (\ie, IRM) and graph OOD generalization (\ie, EERM) do not attain superior performance than the standard ERM on most of the datasets. For example, EERM shows superior OOD performance compared to ERM in only one experiment, and IRM outperforms ERM in four out of ten experiments across the conducted evaluations. This phenomenon is also observed in \cite{good,ahuja2020empirical,rosenfeld2021risks}, showcasing the challenge of achieving invariant prediction in non-Euclidean graph settings. 

Furthermore, our method surpasses other SOTA self-supervised methods on the OOD test set of all datasets by a considerable margin while achieving comparable performance in the in-distribution test set. For instance, on small datasets such as GOOD-CBAS and GOOD-WebKB, our method outperforms GRACE\footnote{MARIO is built up on GRACE according to our recipe. So, we make a comparison with GRACE here.} by over 2\% absolute accuracy on the OOD test set. On larger datasets such as GOOD-Cora and GOOD-Twitch, our method still outperforms other methods which shows its superiority. For instance, under covariate shift, MARIO surpasses other methods by over 7\% absolute accuracy on the GOOD-Twitch OOD test set. These statistics prove the effectiveness of our design.


\begin{table*}[htp]
\caption{Experimental results of all methods under concept shift. The bold font means the top-1 performance and the underline represents the second performance across the unsupervised methods. 'ID' represents in-distribution test performance and 'OOD' means out-of-distribution test performance. (OOM: out-of-memory on a GPU with 24GB memory)}
\label{tab:trans_concept}
\centering
\scalebox{0.95}{
\begin{tabular}{l|cc|cc|cc|cc|cc}
\toprule
\toprule
\multirow{3}{*}{concept shift} & \multicolumn{4}{c|}{GOOD-Cora}                   & \multicolumn{2}{c|}{GOOD-CBAS} & \multicolumn{2}{c|}{GOOD-Twitch} & \multicolumn{2}{c}{GOOD-WebKB} \\
                           & \multicolumn{2}{c}{word} & \multicolumn{2}{c|}{degree}& \multicolumn{2}{c|}{color}    & \multicolumn{2}{c|}{language}   & \multicolumn{2}{c}{university} \\
                           & ID         & OOD         & ID          & OOD          & ID            & OOD           & ID             & OOD            & ID            & OOD            \\
\midrule
ERM                        & 66.38±0.45 & 64.44±0.18  & 68.60±0.40  & 60.76±0.34   & 89.79±1.39    & 83.43±1.19    & 80.80±1.00     & 56.92±0.92     & 62.67±1.53    & 26.33±1.09     \\
IRM                        & 66.42±0.41 & 64.29±0.31  & 68.57±0.35  & 61.45±0.24   & 89.64±1.21    & 82.29±1.14    & 78.87±1.04     & 59.30±1.79     & 62.67±1.10    & 26.88±1.42     \\
EERM                       & 65.10±0.44 & 62.45±0.19  & 66.95±0.44  & 56.58±0.25   & 79.07±2.12    & 64.50±1.01    & OOM            & OOM            & 62.50±2.01    & 28.07±3.23      \\
\midrule
% Random-Init                & 37.53±1.74 & 32.12±1.24  & 37.82±1.71  & 27.74±1.14   &               &               &                &                & 60.33±2.21    & 27.07±1.70     \\
GAE                        & 60.65±0.89 & 58.00±0.55  & 62.59±1.11  & 53.44±0.80   & 75.28±1.36    & 68.07±2.05    & 81.25±0.81     & 51.51±1.05     & 62.17±3.34    & 25.78±1.85     \\
VGAE                       & 63.19±0.53 & 60.35±0.47  & 61.65±0.66  & 54.28±0.28   & 76.50±0.50    & 59.07±0.56    & 80.46±0.53     & 55.56±4.53     & 62.50±2.38    & 24.40±2.57     \\
GraphMAE                   & \underline{66.44±0.46} & \underline{64.87±0.30}  & 67.95±0.46  & 59.41±0.39   & 89.14±0.89    & 82.93±0.93    & 80.05±0.64     & 59.38±1.49     & 61.83±3.37    & 29.27±2.15     \\
DGI                        & 63.33±0.56 & 60.71±0.49  & 65.93±1.02  & 55.83±0.53   & 91.22±1.47    & 85.00±1.66    & 80.05±0.87     & 59.16±1.88     & 61.83±2.83    & 28.63±1.92      \\
MVGRL                      & OOM        & OOM         & OOM         & OOM          & 88.57±1.15    & 76.50±1.17    & OOM            & OOM            & 62.00±3.79    & 28.26±4.20     \\
GRACE                      & 65.61±0.61 & 63.92±0.44  & \textbf{68.59±0.35}  & 60.15±0.45   & 92.00±1.39    & 88.64±0.67    & \textbf{83.43±0.63}     & \underline{60.45±1.46}     & 64.00±3.43    & \underline{34.86±3.43}  \\
RoSA                       & 64.06±0.67 & 62.44±0.39  & 67.07±0.65  & 57.68±0.44   & 90.78±2.27    & 85.93±2.14    & 82.39±0.42     & 57.45±2.16     & 64.17±4.10    & 32.20±2.15     \\
BGRL                       & 65.18±0.43 & 63.43±0.45  & 66.83±0.80  & 59.63±0.38   & 92.36±1.16    & 87.14±1.60    & 82.52±0.60     & 55.48±1.48     & 63.67±2.33    & 31.47±3.43     \\
COSTA                      & 65.05±0.80 & 62.37±0.45  & 66.76±0.87  & 55.73±0.36   & \underline{93.50±2.62}    & \underline{89.29±3.11}    & 83.15±0.30 & 55.03±3.22     & 61.66±2.58    & 32.39±2.13 \\
% ArCL                       &            &             & 67.64±0.57  & 59.71±0.44   &               &               &                &                & 65.00±3.94    & 35.41±1.97 \\      
SwAV                       & 62.22±0.53 & 59.79±0.53  & 64.65±0.94  & 55.06±0.39   & 89.00±0.79    & 81.72±0.66    & \underline{83.32±0.15}     & 59.69±1.97     & \underline{65.17±3.76}    & 29.36±2.01    \\
\midrule
MARIO                       & \textbf{67.11±0.46} & \textbf{65.28±0.34}  & \underline{68.46±0.40}  & \textbf{61.30±0.28}   & \textbf{94.36±1.21}    & \textbf{91.28±1.10}    & 82.31±0.54     & \textbf{63.33±1.72}     & \textbf{65.67±2.81}    & \textbf{37.15±2.37}     \\
\bottomrule
\end{tabular}}
\end{table*}

\begin{table*}[htp]
\caption{Experimental results of all methods under covariate shift. The bold font means the top-1 performance and the underline represents the second performance across the unsupervised methods. 'ID' represents in-distribution test performance and 'OOD' means out-of-distribution test performance. (OOM: out-of-memory on a GPU with 24GB memory)}
\label{tab:trans_covariate}
\centering
\scalebox{0.95}{
\begin{tabular}{l|cc|cc|cc|cc|cc}
\toprule
\toprule
\multirow{3}{*}{covariate shift} & \multicolumn{4}{c|}{GOOD-Cora}                                   & \multicolumn{2}{c|}{GOOD-CBAS} & \multicolumn{2}{c|}{GOOD-Twitch} & \multicolumn{2}{c}{GOOD-WebKB} \\
                           & \multicolumn{2}{c}{word} & \multicolumn{2}{c|}{degree}& \multicolumn{2}{c|}{color}    & \multicolumn{2}{c|}{language}   & \multicolumn{2}{c}{university} \\
                           & ID         & OOD         & ID          & OOD          & ID            & OOD           & ID             & OOD            & ID            & OOD            \\
\midrule
ERM                        & 70.50±0.41 & 64.69±0.33  & 72.46±0.49  & 55.53±0.50   & 92.00±3.08    & 77.57±1.29    & 70.98±0.41     & 49.35±5.09     & 39.34±1.79    & 14.52±3.14   \\
IRM                        & 70.48±0.26 & 64.53±0.57  & 71.98±0.34  & 53.72±0.46   & 90.86±2.41    & 78.86±1.67    & 69.81±0.95     & 49.11±2.82     & 38.52±3.30    & 13.97±2.80     \\
EERM                       & OOM        & OOM         & OOM         & OOM          & 65.00±2.57    & 57.43±3.60    & OOM            & OOM            & 46.07±4.55    & 27.40±7.65     \\
\midrule
GAE                        & 56.63±0.79 & 48.93±0.93  & 66.30±0.88  & 34.01±0.87   & 73.00±2.16    & 60.86±3.01    & 67.24±1.23     & 47.65±2.49     & 45.08±6.32    & 28.02±6.29    \\
VGAE                       & 62.02±0.66 & 54.12±0.86  & 69.41±0.57  & 44.20±1.29   & 62.29±2.04    & 63.29±1.11    & 66.99±1.43     & \underline{50.48±4.58}     & 48.85±4.68    & 20.87±6.69     \\
GraphMAE                   & 68.14±0.43 & 64.00±0.33  & \textbf{73.36±0.56}  & 53.75±0.55   & 67.28±3.03    & 67.28±1.49    & 68.84±1.20     & 48.02±2.79     & 48.03±4.34    & 30.00±8.09     \\
DGI                        & 60.85±0.75 & 57.03±0.67  & 68.97±0.41  & 41.75±0.88   & 69.57±4.09    & 59.71±3.43    & 68.43±1.05     & 44.83±1.61     & 48.52±5.04    & 21.11±7.50     \\
MVGRL                      & OOM        & OOM         & OOM         & OOM          & 65.00±1.94    & 64.15±0.77    & OOM            & OOM           & \textbf{54.10±5.39}    & 16.59±6.51     \\
GRACE                      & \underline{68.77±0.33} & \underline{64.21±0.41}  & 72.69±0.34  & \underline{56.10±0.63}   & \underline{93.57±1.83}    & \underline{89.29±3.40}    & \underline{71.12±0.87} & 46.21±1.54 & 49.67±5.82    & 28.10±4.68    \\
RoSA                       & 68.19±0.56 & 62.48±0.61  & 71.04±0.62  & 52.72±0.79   & 84.71±4.14    &79.14±3.51     & 70.58±0.36     & 45.83±1.72     & 52.30±4.24    & \underline{34.24±7.92}     \\
BGRL                       & 67.23±0.43 & 61.33±0.36  & 72.11±0.39  & 49.15±0.73   & 89.00±2.56    & 79.86±3.29    & \textbf{71.43±0.53}     & 43.86±0.94     & 51.80±5.55    & 30.32±7.61    \\
COSTA                      & 65.28±0.60 & 60.33±0.53  & 70.65±0.62  & 54.03±0.28   & 92.29±1.59    & 82.71±2.74    & 69.29±1.37     & 49.07±2.13     & 50.49±3.01    & 29.84±4.75   \\
SwAV                       & 63.29±1.01 & 56.98±0.94  & 70.27±0.73  & 43.00±0.52   & 89.57±1.12    & 81.43±1.69    & 69.19±0.93     & 49.37±2.96     & 49.84±4.82    & 30.55±6.72   \\
\midrule
MARIO                       & \textbf{69.99±0.54} & \textbf{65.06±0.34}  & \underline{72.73±0.43}  & \textbf{57.73±0.45}  & \textbf{94.57±2.46}    & \textbf{91.00±2.48}     & 68.31±0.78 & \textbf{57.37±1.37}     & \underline{53.94±3.23}    & \textbf{35.24±4.98}   \\
\bottomrule
\end{tabular}}

\end{table*}

\subsubsection{Inductive Setting}
In this subsection, we conduct experiments under the inductive settings, where the test nodes are kept unseen during training. This setting is more suitable for domain generalization.
% But we think it is more convincing that conduct experiments under inductive settings which means test nodes are unseen during training. This setting is more appropriate for domain generalization.

\noindent\textbf{Baselines:} For GOOD-WebKB and GOOD-CBAS datasets, we adopt ERM, IRM, GraphMAE, and GRACE as our baselines. And for Amazon-Photo and Elliptic datasets, we select ERM, EERM, and GRACE as our baselines.

\noindent\textbf{Experimental setup:} For GOOD-WebKB and GOOD-CBAS datasets, we use the same model configuration in Section~\ref{sec:trans}.
% Besides, we add experiments on Amazon-Photo dataset~\cite{yang2016revisiting} and Elliptic~\cite{elliptic} dataset in this subsection. 
For Amazon-Photo dataset~\cite{yang2016revisiting} and Elliptic~\cite{elliptic} dataset, they consist of many snapshots (training data and testing data use different snapshots) which are naturally inductive. For Amazon-Photo dataset, we use 2-layer GCN~\cite{gcn} as the encoder and for elliptic dataset, we use 5-layer GraphSAGE~\cite{sage} as encoder following~\cite{eerm}.

% Figure environment removed

\noindent\textbf{Analysis:}
According to Figure~\ref{fig:amazon},\ref{fig:elliptic},\ref{fig:ind_con},\ref{fig:ind_cov}, we can draw following conclusions:
firstly, based on Figure~\ref{fig:amazon}, it is evident that our method outperforms other representative supervised and self-supervised methods on all test graphs (T1$\sim$T8). This superiority is reflected in the larger median value of our method compared to others. For instance, MARIO achieves over a 3\% absolute improvement compared to ERM in terms of the mean value of eight median values. Additionally, our method demonstrates higher stability across different random initializations, as indicated by the closer proximity of the first and third quartile values to the median value~(\eg, the difference of first and third quartile values of ERM, EERM, GRACE and MARIO are 4.2, 3.3, 6.7 and 1.0 on T8 respectively which indicates MARIO is much more stable than other methods). Furthermore, our method exhibits consistent performance across different graphs (\eg, The standard deviation of median values on T1$\sim$T8 for ERM, EERM, GRACE, and MARIO are 0.4, 1.1, 1.2, and 0.3, respectively.), indicating its robustness to environmental variations and its ability to extract invariant features: $g(G^e) \approx g(G^{e'})$ for all $e, e' \in \mathcal{E}^\text{train}$. In summary, our method showcases enhanced OOD generalization capabilities.
% $g(G^e)g(G^e^\prime)$ where $any e, e^\prime in \mathcal{E}^{train}$

Secondly, from the results presented in Figure~\ref{fig:elliptic}, we can observe that our method averagely harvests 10.9\% absolute improvement over GRACE and 12.5\% absolute improvement over EERM in terms of F1 scores on Elliptic dataset. This demonstrates the effectiveness of our method in handling distribution shifts and improving performance compared to existing approaches. It is worth noting that GRACE's performance worsens over time, indicating its inability to handle distribution shifts effectively. In contrast, our method consistently achieves better F1 scores, except for T9, which is caused by the dark market shutdown occurred after T7~\cite{elliptic}. The emergence of such an event introduces significant variations in data distributions, which subsequently results in performance degradation for all methods. Indeed, this event serves as an unpredictable external factor that introduces significant challenges for models trained on limited training data. The results indicate that the performance heavily depends on available training data. Nonetheless, our approach outperforms other methods even in such an extreme case. This highlights the effectiveness of our method in addressing distribution shifts and improving generalization performance.

Finally, based on the observations from Figure~\ref{fig:ind_con} and Figure~\ref{fig:ind_cov} MARIO demonstrates the best performances on both ID and OOD test sets for GOOD-WebKB and GOOD-CBAS datasets, under both concept shift and covariate shift. Notably, MARIO outperforms other methods by more than 3\% and 10\% absolute improvement on GOOD-WebKB and GOOD-CBAS, respectively, under covariate shift. We can draw similar conclusions as discussed in Section~\ref{sec:trans}. Even under the inductive setting, our method continues to demonstrate excellent OOD generalization capabilities and achieves comparable or even improved in-distribution test performance. These statistical results further validate the effectiveness of our method in handling distribution shifts and enhancing generalization performance.

Overall, the observations we have made provide strong evidence of the great capacity of our method for handling distribution shifts, validating its effectiveness and potential for real-world applications.



% Figure environment removed

% Figure environment removed


% Figure environment removed


\subsection{Ablation Studies}\label{sec:ablation}
\noindent Table~\ref{tab:aba} provides a detailed analysis of the effect of each component according to our proposed recipe for improving OOD generalization in graph contrastive learning. Let's examine the different variants of our method and their impact on performance.
Specifically, MARIO~(w/o ad) represents MARIO without  adversarial augmentation. MARIO~(w/o cmi) denotes we only maximize the mutual information between positive pairs without considering conditional mutual information. MARIO~(w/o cmi, ad) means a vanilla graph contrastive method that is similar to GRACE. 

From Table~\ref{tab:aba}, we can find MARIO~(w/o cmi) lags far behind MARIO on OOD test set which demonstrates appropriately minimizing the redundant information (\ie, conditional mutual information) is essential to improve OOD generalization of GCL methods. And adversarial augmentation can also boost OOD generalization because it can approximately serve as a supermum operator to learn more invariant features  discussed in Section~\ref{sec:aug}. Based on the analysis of these variants, it is evident that the proposed improvements on data augmentation and contrastive loss in the recipe are both effective in enhancing graph OOD generalization. Each component contributes to the overall performance improvement, and their combination leads to a stronger self-supervised graph learner in terms of graph OOD generalization. 

In short, the findings from Table~\ref{tab:aba} support the rationale behind your proposed recipe and provide empirical evidence of the effectiveness of each proposed component. By incorporating these enhancements, our method achieves superior performance in handling distribution shifts and improving graph OOD generalization in graph contrastive learning.
\begin{table*}[htp]
\caption{Ablation studies for MARIO by masking each component.}
\label{tab:aba}
\centering
\scalebox{0.9}{
\begin{tabular}{l|cc|cc|cc|cc|cc}
\toprule
\toprule
\multirow{3}{*}{concept shift} & \multicolumn{4}{c|}{GOOD-Cora}                       & \multicolumn{2}{c|}{GOOD-CBAS} & \multicolumn{2}{c|}{GOOD-Twitch} & \multicolumn{2}{c}{GOOD-WebKB} \\
                           & \multicolumn{2}{c}{word} & \multicolumn{2}{c|}{degree}& \multicolumn{2}{c|}{color}    & \multicolumn{2}{c|}{language}   & \multicolumn{2}{c}{university} \\
                           & ID         & OOD         & ID          & OOD          & ID            & OOD           & ID             & OOD            & ID            & OOD            \\
\midrule
MARIO                      & \textbf{67.11±0.46} & \textbf{65.28±0.34}  & \textbf{68.46±0.40}  & \textbf{61.30±0.28}      & \textbf{94.36±1.21}  & \textbf{91.28±1.10}    & 82.31±0.54     & \textbf{63.33±1.72}     & \textbf{65.67±2.81}    & \textbf{37.15±2.37}     \\
MARIO(w/o ad)              & 66.23±0.53 & 64.02±0.18  & 67.88±0.38  & 60.46±0.29   & 93.21±1.25    & 90.29±0.91    & 82.42±0.73     & 60.50±1.02     & 64.83±2.83    & 36.51±3.25    \\
MARIO(w/o cmi)             & 65.32±0.60 & 63.51±0.32  & 68.14±0.32  & 61.19±0.34   & 94.15±1.23    & 90.57±1.96    & \textbf{82.51±0.56}     & 61.41±2.63     & 64.50±4.35    & 35.78±2.53     \\
MARIO(w/o cmi, ad)         & 64.67±0.55 & 63.11±0.32  & 67.95±0.65  & 60.01±0.57   & 93.36±1.66    & 89.64±1.73    & 81.90±0.75     & 60.12±1.60     & 64.17±3.67    & 34.13±2.38     \\
\bottomrule
\end{tabular}}
\end{table*}
% & 65.32±0.60 & 63.51±0.32 exchange 64.67±0.55 & 63.11±0.32
% 68.14±0.32       id ood test: 60.95±0.43       ood ood test: 61.19±0.34


\subsection{Sensitivity Analysis}\label{sec:sensitivity}
\noindent In this subsection, we will analyze some important hyper-parameters of our method. We conduct sensitivity analysis on GOOD-WebKB dataset with concept shift, we chose two sensitive hyper-parameters (\ie, the coefficient $\gamma$ of condition mutual information in Equation~\ref{equ:cmi} and the number of prototypes $|C|$ in Equation~\ref{equ:pq}). The coefficient of CMI range in $[0.001, 0.01, 0.1, 0.5, 1]$ and the number of prototypes $|C|$ ranges in $[10, 50, 100, 200, 300]$. From Figure~\ref{fig:sensitivity}, we can observe that $\gamma$ reaches 0.1 and $|C|$ reaches 100 or 200 can achieve the best OOD test accuracy. Both higher and lower values of $\gamma$ result in suboptimal performance. This finding aligns with previous research such as DIB~\cite{dib}, indicating that an appropriate compression level is crucial for achieving optimal performance. Extremely high or low compression values are not ideal. 

Regarding the number of prototypes $|C|$, based on the results shown in Figure~\ref{fig:sensitivity}, it is found that setting $|C|=100$ leads to the best performance in terms of OOD test accuracy. This choice provides a moderate number of pseudo labels, which is beneficial for the learning process. 

Based on the sensitivity analysis, we determined that setting $\gamma=0.1$ and $|C|=100$ on most datasets. These hyperparameter values strike a balance between compression level and the number of prototypes, resulting in improved graph OOD generalization.
% Figure environment removed


\subsection{Integrated with Other Models}\label{sec:other_models}
% Figure environment removed

\begin{table}[htp]
\caption{Results of different learning approaches with different encoding models (\ie, GCN, GraphSAGE, GAT).}
\label{tab:others}
\centering
\scalebox{0.9}{
\begin{tabular}{cc|cc|cc}
\toprule
\toprule
\multirow{3}{*}{Model}& \multirow{3}{*}{Method} & \multicolumn{2}{c|}{GOOD-CBAS} & \multicolumn{2}{c}{GOOD-WebKB} \\
                & & \multicolumn{2}{c|}{color}    & \multicolumn{2}{c}{university} \\
                &   & ID          & OOD         & ID          & OOD            \\
\midrule
\multirow{3}{*}{GCN} 
&ERM               & 89.79±1.39 & 83.43±1.19  &  62.67±1.53 & 26.33±1.09         \\
&GRACE             & 92.00±1.39 & 88.64±0.67  &  64.00±3.43 & 34.86±3.43        \\
&MARIO             & 94.36±1.21 & 91.28±1.10  &  65.67±2.81 & 37.15±2.37        \\ \bottomrule
\multirow{3}{*}{SAGE} 
&ERM               & 95.07±1.51 & 75.14±1.19  & 73.67±2.08  & 46.33±3.42       \\
&GRACE             & 95.29±1.11 & 74.43±2.36  & 70.50±5.06  & 49.54±3.83        \\
&MARIO             & 96.00±1.07 & 76.29±3.01  & 71.00±3.82  & 51.74±4.63        \\ \bottomrule
\multirow{3}{*}{GAT} 
&ERM               & 78.64±3.63 & 72.93±2.64  & 61.33±3.71  & 28.99±2.63        \\
&GRACE             & 84.57±1.79 & 78.36±1.60  & 59.50±2.36  & 35.78±3.26        \\
&MARIO             & 84.93±1.95 & 80.43±1.89  & 62.17±4.78  & 38.17±3.10        \\
\bottomrule
\end{tabular}}
\end{table}



\noindent In the subsection, we demonstrate the model-agnostic nature of the recipe by integrating it with various graph neural network (GNN) models, including GCN, GraphSAGE, and GAT.

From Table~\ref{tab:others}, it can be observed that regardless of the specific GNN model used as the encoder, our method consistently achieves the best performance on the OOD test set. This indicates the effectiveness and robustness of our method across different GNN models.
By achieving superior performance across different GNN models, MARIO demonstrates its versatility and ability to improve the OOD generalization of various graph neural models. This highlights the broad applicability and effectiveness of our recipe in enhancing the performance of different GNN encoders.

Furthermore, we integrate our recipe with other GCL methods in Appendix~\ref{app:other_methods}. The results demonstrate our recipe can boost the OOD generalization ability of various GCL methods which means our recipe can serve as a plug-in for many current classical GCL methods.

% Figure environment removed

\subsection{Visualization}\label{sec:vis}
\subsubsection{Metric Score Curves}
We present metric score curves for ERM and MARIO, including training, ID validation, ID testing, OOD validation, and OOD testing accuracy, in Figure~\ref{fig:curve2}. Notably, MARIO demonstrates superior convergence with approximately 10\% absolute improvement on the OOD test set compared to ERM. Furthermore, MARIO effectively narrows the performance gap between in-distribution and out-of-distribution performance, showcasing its efficacy in enhancing OOD generalization for graph data. More metric score curves can be found in Appendix~\ref{app:curves}.


\subsubsection{Feature Visualization}
In order to assess the quality of learned embeddings, we adopt t-SNE~\cite{tsne} to visualize the node embedding on GOOD-Cora dataset (concept shift in word domain) using random-init of GCN, EERM, GRACE, and MARIO, where different classes have different colors in Figure~\ref{fig:vis}. For clarity, we select eight classes with the largest number of nodes to enhance the informativeness and interpretability of the visualization. We can observe that the 2D projection of node embeddings learned by MARIO has a better separation of clusters, which indicates the model can help learn representative features for downstream tasks. It has to note that we depict both ID nodes and OOD nodes in the same figure. 

Besides, we also separately visualize ID nodes and OOD nodes in the different figures in the Appendix~\ref{app:feature}. And we can find MARIO performs a clearer separation of clusters whether on ID nodes or OOD nodes compared to other methods.



\section{Additional Results}
\label{app:additional results}
\subsection{Walltime Comparison}
\cref{tab:walltime} presents walltime comparison of different methods. Our method converges faster while being scalable w.r.t.~DIBS, the nonlinear Bayesian causal discovery baseline. Other methods like BGES and DDS, while faster, perform much worse in terms of uncertainty quantification. In addition BGES is limited to linear model and DDS is not a fully Bayesian method.
\begin{table}[]
\centering
\caption{Walltime results (in minutes, rounded to the nearest minute) of the runtime of different approaches on a single 40GB A100 NVIDIA GPU. The N/A fields indicate that the corresponding method cannot be run within the memory constraints of a single GPU.}
\label{tab:walltime}
\begin{tabular}{l|cccc}
\hline
             & \multicolumn{1}{c|}{d=30} & \multicolumn{1}{c|}{d=50} & \multicolumn{1}{c|}{d=70} & d=100 \\ \hline
BaDAG (\textbf{Ours})(Bayesian, Nonlinear) & 171                       & 238                       & 261                       & 448   \\
DIBS (Bayesian, Nonlinear)        & 187                       & 350                       &      N/A                     &     N/A  \\ \hline
BGES (Quasi-Bayesian, Linear)       & 2                         & 3                         & 6                         & 11    \\
BCD (Bayesian, Linear)         & 252                       & 328                       & 418                       & 600   \\
DDS  (Quasi-Bayesian, Nonlinear)        & 92                        & 130                       & 174                       &  N/A   \\
\hline
\end{tabular}
\end{table}
\subsection{Performance with higher dimensional datasets}
\label{appsubsec: higher dimensional datasets}
Full results for all the metrics for settings $d=20$, $d=70$ and $d=100$ for nonlinear settings are presented in \cref{fig:nonlinear_20_40}, \cref{fig:nonlinear_70_140} and \cref{fig:nonlinear_100_200}.
% Figure environment removed
% Figure environment removed
% Figure environment removed
\subsection{Performance of SG-MCMC with Continuous Relaxation}
\label{appsubsec: fully SG-MCMC performance}
We compare the performance of SG-MCMC+VI and SG-MCMC with $\tmW$ on $d=10$ ER and SF graph settings. \cref{fig:sg-mcmc+vi vs sg-mcmc 10} shows the performance comparison. We can observe that SG-MCMC+VI generally outperforms its counterpart in most of the metrics. We hypothesize that this is because VI network $\mu_\phi$ couples $\vp$ and $\mW$. This coupling effect is crucial since the changes in $\vp$ results in the change of permutation matrix, where the $\mW$ can immediately respond to this change through $\mu_\phi$. On the other hand, $\tmW$ can only respond to this change through running SG-MCMC steps on $\tmW$ with fixed $\vp$. In theory, this is the most flexible approach since this coupling do not requires parametric form like $\mu_\phi$. However in practice, we cannot run many SG-MCMC steps with fixed $\vp$ for convergence, which results in the inferior performance. 
% Figure environment removed

%\subsection{Ablation Study}
%\label{appsubsec: ablation study}
%We conducts several ablation studies to investigate the %properties of \ModelName{}. \cref{fig: ablation init p scale} demonstrates the performance comparison between different initialization scale of $\vp$ value.
%\cref{fig: ablation num particles} shows the performances with different number of parallel SG-MCMC chains. \cref{fig: ablation scale,fig: ablation scale_p} shows the performance differences with different levels of injected SG-MCMC noise. For this setting, we fix the injected noise level for either $\vp$ or $\mTheta$ and vary the other. All ablation studies are conducted using ER $d=30$ datasets with the same hyperparameters reported in \cref{appsubsec: hyperparameter selection}. 


%% Figure environment removed
%% Figure environment removed

%% Figure environment removed

%% Figure environment removed


\section{Code and License}
For the baselines, we use the code from the following repositories:
\begin{itemize}
    \item BGES: We use the code from \cite{agrawal2019abcd} from the repository \href{https://github.com/agrawalraj/active_learning}{https://github.com/agrawalraj/active\_learning} (No license included).
    \item DDS: We use the code from the official repository \href{https://github.com/sharpenb/Differentiable-DAG-Sampling}{https://github.com/sharpenb/Differentiable-DAG-Sampling} (No license included).
    \item BCD: We use the code from the official repository \href{https://github.com/ermongroup/BCD-Nets}{https://github.com/ermongroup/BCD-Nets} (No license included).
    \item DIBS: We use the code from the official repository \href{https://github.com/larslorch/dibs}{https://github.com/larslorch/dibs} (MIT license).
\end{itemize}
Additionally for the Syntren~\cite{van2006syntren} and Sachs Protein Cells~\cite{sachs2005causal} datasets, we use the data provided with repository \href{https://github.com/kurowasan/GraN-DAG}{https://github.com/kurowasan/GraN-DAG} (MIT license).

\section{Broader Impact Statement}

This work is concerned with understanding cause and effects from data, which has potential applications in empirical sciences, economics and machine perception. Understanding causal relationships can improve fairness in decision making, understand biases which might be present in the data and answering causal queries. As such, we envision this line of work to not have any significant negative impact.
\end{document}