\section{Sampling the DAGs}
\label{sec: sampling the DAGs}
In this section, we focus on the Bayesian inference over binary DAGs through a novel mapping, $\tau(\mW,\vp)$, a modification of NoCurl. We establish the validity of performing Bayesian inference within $(\mW,\vp)$ space utilizing $\tau$ (\cref{subsec: Bayesian inference W p space}). However, $\tau$ yields uninformative gradient during back-propagation, a challenge we overcome by deriving an equivalent formulation based on permutation-based DAG learning, thereby enabling the use of relaxed gradient estimators (\cref{subsec: equivalent formulation}).


% In this section, instead of learning a single continuously weighted DAG, we focus on the Bayesian inference over the binary DAGs by introducing a new mapping $\tau(\mW,\vp)$ and deriving equivalent formulations. In particular, we will first prove that performing Bayesian inference in $(\mW,\vp)$ space with $\tau$ is equivalent to being in the binary DAG space (\cref{subsec: Bayesian inference W p space}). However, $\tau$ involves a piecewise constant $\step$ function, which lacks useful gradient information during backprop. To resolve the issue, we derive an equivalent formulation based on inferring the permutation matrix (\cref{subsec: equivalent formulation}). 
\subsection{Bayesian Inference in $W,p$ Space}
\label{subsec: Bayesian inference W p space}
The NoCurl formulation (\cref{eq: NoCurl mapping}) focuses on learning \emph{a single weighted} DAG, which is not directly useful for our purpose. We need to address two key questions: (1) considering only binary adjacency matrices without weights; (2) ensuring Bayesian inference in $(\mW,\vp)$ is valid.

First, we introduce a modification $\tau:\binaryset\times\sR^d\rightarrow \binaryset$:
\begin{equation}
\tau(\mW,\vp) = \mW\odot \step(\grad \vp)
\label{eq: Binary NoCurl}
\end{equation}
where we abuse the term $\mW$ for binary matrices, and replace $\relu(\cdot)$ with $\step(\cdot)$. $\mW$ acts as mask to disable the edge existence. Thus, due to the $\step$, $\tau$ can only output a binary adjacency matrix.

Next, we show that performing Bayesian inference in such augmented $(\mW,\vp)$ space is valid, i.e., using the posterior $p(\mW,\vp\vert \mD)$ to replace $p(\mG\vert \mD)$. This differs from NoCurl, which focuses on a single graph rather than the validity for Bayesian inference, requiring a new theory for soundness.

\begin{theorem}[Equivalence of inference in $(\mW,\vp)$ and binary DAG space]
Assume graph $\mG$ is a binary adjacency matrix representing a DAG and node potential $\vp$ does not contain the same values, i.e.~$p_i\neq p_j$ $\forall i,j$. Then, with the induced joint observational distribution $p(\mD,\mG)$, dataset $\mD$, and a corresponding prior $p(\mG)$, we have
\begin{align}
p(\mG\vert \mD) = \int p_\tau(\vp,\mW\vert \mD)\indicator(\mG=\tau(\mW,\vp))d\mW d\vp
\label{eq: equivalence of bayesian inference}
\end{align}
if $p(\mG)=\int p_\tau(\vp,\mW)\indicator(\mG=\tau(\mW,\vp))d\mW d\vp$, where $p_\tau(\mW,\vp)$ is the prior, $\indicator(\cdot)$ is the indicator function, and $p_\tau(\vp,\mW\vert D)$ is the posterior distribution over $\vp,\mW$.
\label{thm: equivalence of bayesian inference}
\end{theorem}

Refer to \cref{subapp: proof of equivalence of bayesian inference} for detailed proof.


This theorem guarantees that instead of performing inference directly in the constrained space (i.e.~DAG space), we can apply Bayesian inference in a less complex $(\mW,\vp)$ space where $\mW\in\binaryset$ and $\vp\in\sR^d$ without explicit constraints. 


For inference of $\vp$, we adopt a sampling-based approach, which is asymptotically accurate \cite{ma2015complete}. In particular, we consider SG-MCMC (refer to \cref{sec: SGMCMC sampling framework}), which avoids the expensive Metropolis-Hastings acceptance step and scales to large datasets. We emphasize that any other suitable sampling algorithms can be directly plugged in, thanks to the generality of the framework.  


However, the mapping $\tau$ does not provide meaningful gradient information for $\vp$ due to the piecewise constant $\step(\cdot)$ function, which is required by SG-MCMC.

% However, one major problem with this mapping $\tau$ is that it does not provide meaningful gradient information for $\vp$ due to the piecewise constant $\step(\cdot)$, which is required by SG-MCMC methods.




\subsection{Equivalent Formulation}
\label{subsec: equivalent formulation}
\looseness=-1 In this section, we address the above issue by deriving an equivalence to a permutation learning problem. This alternative formulation enables various techniques that can approximate the gradient of $\vp$. 

\paragraph{Intuition} The node potential $\vp$ implicitly defines a topological ordering through the mapping $\step(\grad(\cdot))$. In particular, $\grad(\cdot)$ outputs a skew-symmetric adjacency matrix, where each entry specifies the potential difference between nodes. $\step(\grad(\cdot))$ zeros out the negative potential differences (i.e.~$p_i\leq p_j$), and only permits the edge direction from higher potential to the lower one (i.e.~$p_i>p_j$). This implicitly defines a sorting operation based on the descending node potentials, which can be cast as a particular $\argmax$ problem \cite{blondel2020fast,kuhn1955hungarian, mena2018learning, niculae2018sparsemap,zantedeschi2023dag} involving a permutation matrix.

\paragraph{Alternative formulation} We define $\mL\in\binaryset$ as a matrix with lower triangular part to be $1$, and vector $\vo=[1,\ldots, d]$. We propose the following formulation:
\begin{align}
&\mG = \mW \odot \left[\perm(\vp)\mL \perm(\vp)^T\right] \
&\text{where}\; \perm(\vp) = \argmax_{\perm'\in\bm{\Sigma}_d} \vp^T(\perm' \vo)
\label{eq: alternative formulation}
\end{align}
Here, $\bm{\Sigma}_d$ represents the space of all $d$ dimensional permutation matrices. The following theorem states the equivalence of this formulation to  \cref{eq: Binary NoCurl}.
\begin{theorem}[Equivalence to NoCurl formulation]
Assuming the conditions in \cref{thm: equivalence of bayesian inference} are satisfied. Then, for a given $(\mW,\vp)$, we have
$$
\mG=\mW\odot \step(\grad \vp) = \mW\odot \left[ \perm(\vp)\mL\perm(\vp)^T\right]
$$
where $\mG$ is a DAG and $\perm(\vp)$ is defined in \cref{eq: alternative formulation}.
\label{thm: equivalence of alternative formulation}
\end{theorem}
Refer to \cref{subapp: proof of theorem alternative formulation} for details.


This theorem translates our proposed operator $\step(\grad(\vp))$ into finding a corresponding permutation matrix $\perm(\vp)$. Although this does not directly solve the uninformative gradient, it opens the door for approximating this gradient with the tools from the differentiable permutation literature \cite{blondel2020fast,mena2018learning, niculae2018sparsemap}. For simplicity, we adopt the Sinkhorn approach \cite{mena2018learning}, but we emphasize that this equivalence is general enough that any past or future approximation methods can be easily applied.


\paragraph{Sinkhorn operator} The Sinkhorn operator $\Sinkhorn(\mM)$ on a matrix $\mM$ \cite{adams2011ranking} is defined as a sequence of row and column normalizations, each is called Sinkhorn iteration.

\cite{mena2018learning} showed that the non-differentiable $\argmax$ problem
\begin{equation}
    \perm = \argmax_{\perm'\in\Sigma_d}\left\langle\perm', \mM\right\rangle
    \label{eq: permutation argmax}
\end{equation}
\looseness=-1 can be relaxed through an entropy regularizer with its solution being expressed by $\Sinkhorn(\cdot)$.
% \cite{mena2018learning} showed that the $\argmax$ problem of learning permutation matrix in \cref{eq: permutation argmax} can be relaxed through an entropy regularizer, and its solution can be obtained using $\Sinkhorn(\cdot)$.
In particular, they showed that $\Sinkhorn(\mM/t)=\argmax_{\perm'\in \polytope}\left\langle\perm',\mM\right\rangle+th(\perm')$, where $h(\cdot)$ is the entropy function. This regularized solution converges to the solution of \cref{eq: permutation argmax} when $t\rightarrow 0$, i.e.~$\lim_{t\rightarrow 0}\Sinkhorn(\mM/t)$. 
Since the Sinkhorn operator is differentiable, $\Sinkhorn(\mM/t)$ can be viewed as a differentiable approximation to \cref{eq: permutation argmax}, which can be used to obtain the solution of \cref{eq: alternative formulation}. Specifically, we have
\begin{equation}
    \argmax_{\perm'\in\bm{\Sigma}_d} \vp^T(\perm'\vo) = \argmax_{\perm'\in\bm{\Sigma}_d}\langle\perm',\vp\vo^T\rangle = \lim_{t\rightarrow 0}\Sinkhorn(\frac{\vp\vo^T}{t})
    \label{eq: Sinkhorn solution}
\end{equation}
In practice, we approximate it wth $t>0$, resulting in a doubly stochastic matrix. To get the binary permutation matrix, we apply the Hungarian algorithm \cite{munkres1957algorithms}. During the backward pass, we use a straight-through estimator~\cite{bengio2013estimating} for $\vp$. 

Some of the previous works \cite{charpentier2022differentiable,cundy2021bcd} have leveraged the Sinkhorn operator to model variational distributions over permutation matrices. However, they start with a full rank $\mM$, which has been reported to require over \textbf{1000} Sinkho    rn iterations to converge \cite{cundy2021bcd}. However, our formulation, based on explicit node potential $\vp\vo^T$, generates a rank-1 matrix, requiring much fewer Sinkhorn steps (around \textbf{300}) in practice, saving two-thirds of the computational cost.






\section{Bayesian Causal Discovery via Sampling}
\label{sec: SGMCMC sampling framework}
In this section, we delve into two specific methodologies that are derived from the proposed framework. The first one, which will be our main focus, combines SG-MCMC and VI in a Gibbs sampling manner. The second one, which is based entirely on SG-MCMC with continuous relaxation, is also derived, but we include its details in \cref{appsec: joint inference SG-MCMC} due to its inferior empirical performance.
% In this section, we aim to consoludate  

% two general frameworks for non-linear Bayesian structure learning by utilizing SG-MCMC and VI for $\vp, \mW$ and SEM parameters $\Theta$. First, we propose a SEM formulation based on the additive noise model (ANM). Then, we show how to perform the inference by (1) joint inference with SG-MCMC, and (2) SG-MCMC+VI.
\subsection{Model Formulation}
\label{subsect: model formulation}
We build upon the model formulation of \cite{geffner2022deep}, which combines the additive noise model with neural networks to describe the functional relationship. Specifically, $X_i \coloneqq f_i(\mX_{\Pa^i}) + \epsilon_i$, where $f_i$ adheres to the adjacency relation specified by $\mG$, i.e.~$\partial f_i(\vx) / \partial x_j = 0$ if no edge exists between nodes $i$ and $j$. We define $f_i$ as 
\begin{equation}
    f_i(\vx) = \zeta_i\left(\sum_{j=1}^dG_{ji}l_j(x_j)\right),
\end{equation}
where $\zeta_i$ and $l_i$ are neural networks with parameters $\mTheta$, and $\mG$ serves as a mask disabling non-parent values. To reduce the number of neural networks, we adopt a weight-sharing mechanism: $\zeta_i(\cdot) = \zeta(\vu_i,\cdot)$ and $l_i(\cdot) = l(\vu_i,\cdot)$, with trainable node embeddings $\vu_i$.

\paragraph{Likelihood of SCM}
The likelihood can be evaluated through the noise $\bm{\epsilon} = \vx - \vf(\vx;\mTheta)$. \cite{geffner2022deep} showed that if $\mG$ is a DAG, then the mapping from $\bm{\epsilon}$ to $\vx$ is invertible with a Jacobian determinant of 1. Thus, the observational data likelihood is:
\begin{equation}
    p(\vx \vert \mG) = p_\epsilon(\vx-\vf(\vx;\mTheta)) = \prod_{i=1}^d p_{\epsilon_i}(x_i-f_i(\vx_{\Pa_G^i}))
    \label{eq: SEM likelihood}
\end{equation}
\paragraph{Prior design}
We implicitly define the prior $p(\mG)$ via $p(\vp,\mW)$. We propose the following for the joint prior:
\begin{equation}
    p(\mW,\vp, \mTheta) \propto \nonumber \mathcal{N}(\mTheta;\bm{0},\bm{I})  \mathcal{N}(\vp;\bm{0},\alpha\bm{I})\mathcal{N}(\mW;\bm{0},\bm{I})\exp(-\lambda_s\Vert\tau(\mW,\vp)\Vert^2_F)
    \label{eq: Prior p w}
\end{equation}
where $\alpha$ controls the initialization scale of $\vp$ and $\lambda_s$ controls the sparseness of $\mG$. 

\subsection{Bayesian Inference of $W,p,\Theta$}
%\subsubsection{Combined inference: SG-MCMC + VI}
\label{subsubsec: combined inference}

The main challenge lies in the binary nature of $\mW \in \{0,1\}^{d\times d}$, which requires a discrete sampler. Although recent progress has been made \cite{grathwohl2021oops,sun2022discrete,zanella2020informed,zhang2022langevin}, these methods either involve expensive Metropolis-Hasting (MH) steps or require strong assumptions on the target posterior when handling batched gradients. To address this, we propose a combination of SG-MCMC for $\vp,\mTheta$ and VI for $\mW$. It should be noted that our framework can incorporate any suitable discrete sampler if needed. 
\begin{wrapfigure}[13]{r}{0.25\textwidth}
    \vspace{2em}
    \usetikzlibrary{bayesnet}
\scalebox{0.6}{
\begin{tikzpicture}[scale=3]

  % Define nodes
  \node[latent] (W) {$\mW$};
  \node[latent, left=of W] (p) {$\vp$};
  \node[latent, below=of W, xshift=1.2cm] (Theta) {$\mTheta$};
  \node[obs, below=of Theta] (x) {$\vx^{(i)}$};
  
  % Connect nodes
  \edge {p} {x};
  % \edge {p} {W};
  \edge {W} {Theta};
  \edge {W} {x};
  \edge {Theta} {x};
  \edge {p} {Theta};
  
  % Define plate
  \plate {x_plate} {(x)} {$i = 1, \dots, N$};

\end{tikzpicture}
}
    \caption{Graphical model of the inference problem.}
    \label{fig: graphical model with latent}
    %\vspace{-1em}
\end{wrapfigure}
% Since the joint inference requires the introduction of the additional latent variable $\tmW$ as a surrogate for $\mW$, we can sidestep it by proposing an alternative procedure, where we use SG-MCMC for $\vp, \mTheta$ but VI for $\mW$. 

We employ a Gibbs sampling procedure \cite{casella1992explaining}, which iteratively applies (1) sampling $\vp,\mTheta\sim p(\vp,\mTheta|\mD,\mW)$ with SG-MCMC; (2) updating the variational posterior $q_\phi(\mW|\vp,\mD)\approx p(\mW|\vp,\mTheta,\mD)$.

We define the posterior $p(\vp,\mTheta \vert \mD,\mW)\propto \exp(-U(\vp,\mW,\mTheta))$, where $U(\vp,\mW,\mTheta) = -\log p(\vp,\mD,\mW,\mTheta)$. SG-MCMC in continuous time defines a specific form of It\^o diffusion that maintains the target distribution invariant \cite{ma2015complete} without the expensive computation of the MH step. We adopt the Euler-Maruyama discretization for simplicity. Other advanced discretization can be easily incorporated \cite{chen2015convergence,platen2010numerical}. 

Preconditioning techniques have been shown to accelerate SG-MCMC convergence \cite{chen2014stochastic, gong2019icebreaker,li2016preconditioned,welling2011bayesian,ye2017langevin}. We modify the sampler based on \cite{gong2019icebreaker}, which is inspired by Adam \cite{kingma2014adam}. Detailed update equations can be found in \cref{app: SG-MCMC update}.

The following proposition specifies the gradients required by SG-MCMC: $\nabla_{\vp,\mTheta} U(\vp,\mW,\mTheta)$.

% For $\vp$ and $\mTheta$, their sampling procedures are identical as the joint inference (\cref{subsubsec: joint inference}). The following proposition specifies the required gradient $\nabla_{\vp,\mTheta}U(\vp,\mW,\mTheta)$

\begin{proposition}
Assume the model is defined as above, then we have the following:
\begin{equation}
    \nabla_{\vp} U=-\nabla_{\vp}\log p(\vp) - \nabla_{\vp}\log p(\mD\vert \mTheta, \tau(\mW,\vp))
    \label{eq: p gradient}
\end{equation}
and 
\begin{equation}
    \nabla_{\mTheta} U =-\nabla_{\mTheta}\log p(\mTheta) - \nabla_{\mTheta}\log p(\mD\vert \mTheta,\tau(\vp,\mW))
    \label{eq: mTheta gradient}
\end{equation}
\label{prop: gradient computation}
\end{proposition}
Refer to \cref{subapp: proof of proposition gradient computation} for details.

\paragraph{Variational inference for $\mW$}
We use the variational posterior $q_\phi(\mW\vert \vp)$ to approximate the true posterior $p(\mW\vert \vp,\mTheta,\mD)$. Specifically, we select an independent Bernoulli distribution with logits defined by the output of a neural network $\mu_\phi(\vp)$:
\begin{equation}
    q_\phi(\mW\vert \vp)=\prod_{ij}Ber(\mu_\phi(\vp)_{ij})
    \label{eq:VI bernoulli}
\end{equation}
To train $q_\phi$, we derive the corresponding \emph{evidence lower bound} (ELBO):
\begin{equation}
    \ELBO(\phi) = \E_{q_\phi(\mW|\vp)}\left[
    \log p(\mD,\vp,\mTheta\vert \mW)]-\KL\left[q_\phi(\mW\vert \vp)\Vert p(\mW)\right]
    \right].
    \label{eq: ELBO for W}
\end{equation}
where $\KL$ is the Kullback-Leibler divergence. 
The derivation is in \cref{subapp: derivation of ELBO}.
\cref{alg: combined inference} summarizes this inference procedure. 
\begin{algorithm}[tb]
\caption{\ModelName~ SG-MCMC+VI Inference}
\label{alg: combined inference}
\begin{algorithmic}
\STATE {\bfseries Input:} dataset $\mD$; prior $p(\vp, \mW),p(\mTheta)$; SG-MCMC sampler $\sampler$; sampler hyperparameters $\Psi$; network $\mu_\phi(\cdot)$; training iteration $T$.
\STATE {\bfseries Output:} samples $\{\mTheta,\vp\}$ and variational posterior $q_\phi$
\STATE Initialize $\mTheta^{(0)}, \vp^{(0)},\phi$
\FOR{$t=1\ldots,T$}
    \STATE Sample $\mW^{(t-1)}\sim q_{\phi}(\mW\vert \vp^{(t-1)})$
    \STATE Evaluate $\nabla_{\vp,\mTheta} U$ (\cref{eq: p gradient,eq: mTheta gradient}) with $\mTheta^{(t-1)},\vp^{(t-1)},\mW^{(t-1)}$
    \STATE $\mTheta^{(t)},\vp^{(t)} = \sampler(\nabla_{\vp,\mTheta}U;\Psi)$
    \IF{storing condition met}
        \STATE $\{\vp,\mTheta\}\leftarrow \vp^{(t)},\mTheta^{(t)}$
    \ENDIF
    \STATE Maximize ELBO (\cref{eq: ELBO for W}) w.r.t. $\phi$ with $\vp^{(t)}, \mTheta^{(t)}$
\ENDFOR
\end{algorithmic}
\end{algorithm}
\paragraph{SG-MCMC with continuous relaxation}
Furthermore, we explore an alternative formulation that circumvents the need for variational inference. Instead, we employ SG-MCMC to sample $\tmW$, a continuous relaxation of $\mW$, facilitating a fully sampling-based approach. For a detailed formulation, please refer to \cref{appsec: joint inference SG-MCMC}. We report its performance in \cref{appsubsec: fully SG-MCMC performance}, which surprisingly is inferior to SG-MCMC+VI. We hypothesize that coupling $\mW, \vp$ through $\mu_\phi$ is important since changes in $\vp$ results in changes of the permutation matrix $\perm(\vp)$, which should also influence $\mW$ accordingly during posterior inference. However, through sampling $\tmW$ with few SG-MCMC steps, this change cannot be immediately reflected, resulting in inferior performance. Thus, we focus only on the performance of SG-MCMC+VI for our experiments. 

\paragraph{Computational complexity}
Our proposed SG-MCMC+VI offers a notable improvement in computational cost compared to existing approaches, such as DIBS \cite{lorch2021dibs}. 
The computational complexity of our method is $O(BN_p+N_pd^3)$, where $B$ represents the batch size and $N_p$ is the number of parallel SG-MCMC chains. This former term stems from the forward and backward passes, and the latter comes from the Hungarian algorithm, which can be parallelized to further reduce computational cost. In comparison, DIBS has a complexity of $O(N_p^2N+N_pd^3)$ with $N\gg B$ being the full dataset size. This is due to the kernel computation involving the entire dataset and the evaluation of the matrix exponential in the DAG regularizer \cite{zheng2018dags}. As a result, our approach provides linear scalability w.r.t. $N_p$ with substantially smaller batch size $B$. Conversely, DIBS exhibits quadratic scaling in terms of $N_p$ and lacks support for mini-batch gradients.



