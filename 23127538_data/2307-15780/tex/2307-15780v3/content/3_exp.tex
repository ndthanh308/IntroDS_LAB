\section{Experiments}\label{sec:exp}

\begin{table*}[htpb]
    \centering
    \setlength{\abovecaptionskip}{0.cm}
    % \setlength{\belowcaptionskip}{-0.1cm}
    \caption{Long-term traffic forecasting on the METR-LA, PEMS-BAY, PEMS04, and PEMS08 datasets. Methods with $^*$ are implemented with patch embedding. Details can be found in Section 5.1.}
    \label{tab:lstf_traffic}
    % \scalebox{1.065}{
    \begin{tabular}{cccc|cc|cc|cc|cc|cc}
      \toprule
      \hline
      \multirow{2}*{\textbf{Data}} &
      \multirow{2}*{\textbf{Methods}} & 
      \multicolumn{2}{c}{\textbf{@Horizon 12}}& 
      \multicolumn{2}{c}{\textbf{@Horizon 48}}& 
      \multicolumn{2}{c}{\textbf{@Horizon 96}}& 
      \multicolumn{2}{c}{\textbf{@Horizon 144}}& 
      \multicolumn{2}{c}{\textbf{@Horizon 192}}& 
      \multicolumn{2}{c}{\textbf{@Horizon 288}} \\
      \cmidrule(r){3-4} \cmidrule(r){5-6} \cmidrule(r){7-8} \cmidrule(r){9-10} \cmidrule(r){11-12} \cmidrule(r){13-14}
      &  & MAE & MAPE & MAE & MAPE & MAE & MAPE & MAE & MAPE & MAE & MAPE & MAE & MAPE\\
      \midrule

    \hline
    \multirow{12}*{\rotatebox{90}{\textbf{METR-LA}}}
      & HI & 10.44 & 23.21\% & 10.42 & 23.19\% & 10.43 & 23.23\% & 10.43 & 23.32\% & 10.40 & 23.34\% & 10.22 & 22.81\%\\
      & DLinear & 7.61 & 16.19\% & 12.86 & 23.79\% & 12.99 & 23.11\% & 12.90 & 23.48\% & 12.89 & 23.15\% & 13.07 & 23.33\%\\
      & Informer & 4.65 & 15.52\% & 4.86 & 16.54\% & 4.98 & 17.16\% & 5.07 & 17.41\% & 5.07 & 17.30\% & 5.06 & 17.14\%\\
      & Autoformer & 7.23 & 19.25\% & 7.27 & 19.73\% & 7.45 & 20.23\% & 7.83 & 21.49\% & 7.74 & 20.98\% & 8.41 & 22.43\%\\
      & FEDFormer & 8.78 & 22.29\% & 9.11 & 22.69\% & 9.12 & 22.75\% & 9.54 & 24.18\% & 9.81 & 24.76\% & 10.13 & 25.56\%\\
      & Pyraformer & 4.22 & 12.84\% & 4.55 & 14.93\% & 4.75 & 15.81\% & 4.80 & 15.89\% & 4.81 & 15.68\% & 4.62 & 14.79\%\\
      \cmidrule{2-14}
      & DCRNN$^*$ & 4.07 & 12.74\% & 4.39 & 14.08\% & 4.44 & 14.02\% & 4.46 & 14.16 \% & 4.51 & 14.41\% & 4.71 & 15.59\%\\
      & GWNet$^*$ & 3.87 & 12.18\% & 4.19 & 13.60\% & 4.25 & 13.62\% & 4.42 & 14.56\% & 4.58 & 15.40\% & 4.51 & 15.09\%\\
      & MTGNN$^*$ & 4.01 & 12.31\% & 4.31 & 13.84\% & 4.53 & 14.85\% & 4.59 & 14.77\% & 4.57 & 15.18\% & 4.75 & 15.93\%\\
      & STID & 3.84 & 12.17\% & 4.13 & 14.11\% & 4.04 & 13.05\% & 4.11 & 13.65\% & 4.15 & 14.07\% & 4.17 & 13.83\%\\
      & STEP & 00.00 & 00.00\% & 00.00 & 00.00\% & 00.00 & 00.00\% & 00.00 & 00.00\% & 00.00 & 00.00\% & 00.00 & 00.00\%\\
      & D$^2$STGNN$^*$ & 3.71 & 11.24\% & 3.96 & 12.84\% & 3.99 & 13.26\% & 4.05 & 13.17\% & 4.05 & 13.36\% & 4.09 & 12.78\%\\
      \cmidrule{2-14}
      & HUTFormer & \textbf{3.58} & \textbf{11.11\%} & \textbf{3.77} & \textbf{11.76\%} & \textbf{3.78} & \textbf{11.98\%} & \textbf{3.81} & \textbf{12.12\%} & \textbf{3.83} & \textbf{12.16\%} & \textbf{3.84} & \textbf{12.27\%}\\
    \midrule

    \hline
    \multirow{12}*{\rotatebox{90}{\textbf{PEMS04}}}
      & HI & 41.73 & 28.46\% & 41.16 & 28.61\% & 41.38 & 28.62\% & 41.28 & 28.42\% & 30.99 & 27.34\% & 39.58 & 26.49\%\\
      & DLinear & 27.29 & 19.83\% & 37.20 & 26.51\% & 37.50 & 26.78\% & 37.57 & 26.87\% & 37.17 & 25.27\% & 36.87 & 25.21\%\\
      & Informer & 25.94 & 17.56\% & 25.72 & 18.05\% & 25.60 & 18.27\% & 25.98 & 17.81\% & 26.42 & 17.67\% & 27.42 & 18.57\%\\
      & AutoFormer & 29.94 & 28.00\% & 31.30 & 27.41\% & 31.47 & 27.73\% & 31.95 & 27.89\% & 32.03 & 28.03\% & 33.34 & 29.82\%\\
      & FEDFormer & 34.94 & 34.33\% & 32.24 & 37.23\% & 33.90 & 34.33\% & 35.12 & 41.26\% & 35.16 & 34.08\% & 41.83 & 51.01\%\\
      & PyraFormer & 23.40 & 17.18\% & 25.40 & 18.80\% & 26.45 & 19.89\% & 26.22 & 19.01\% & 26.51 & 19.18\% & 26.58 & 20.57\%\\
      \cmidrule{2-14}
      & DCRNN$^*$ & 22.25 & 16.59\% & 24.42 & 18.89\% & 25.20 & 19.17\% & 26.31 & 19.61\% & 27.32 & 19.74\% & 28.04 & 21.02\%\\
      & GWNet$^*$ & 22.24 & 16.51\% & 23.50 & 18.29\% & 24.08 & 18.07\% & 24.85 & 18.21\% & 25.83 & 18.98\% & 31.17 & 21.00\%\\
      & MTGNN$^*$ & 21.75 & 15.93\% & 23.04 & 17.81\% & 24.33 & 17.80\% & 25.56 & 17.68\% & 25.80 & 17.85\% & 26.78 & 20.64\%\\
      & STID & 21.01 & 15.24\% & 22.77 & 16.61\% & 23.39 & 16.87\% & 24.06 & 17.08\% & 24.43 & 17.22\% & 25.19 & 17.49\%\\
      & STEP & 00.00 & 00.00\% & 00.00 & 00.00\% & 00.00 & 00.00\% & 00.00 & 00.00\% & 00.00 & 00.00\% & 00.00 & 00.00\%\\
      & D$^2$STGNN$^*$ & 21.55 & 16.03\% & 22.98 & 17.04\% & 24.16 & 17.57\% & 24.50 & 17.93\% & 24.59 & 17.19\% & 24.79 & 17.97\%\\
      \cmidrule{2-14}
      & HUTFormer & \textbf{19.61} & \textbf{13.59\%} & \textbf{21.54} & \textbf{14.95\%} & \textbf{21.96} & \textbf{15.22\%} & \textbf{22.66} & \textbf{15.30\%} & \textbf{23.10} & \textbf{15.35\%} & \textbf{23.43} & \textbf{15.71\%}\\
    \midrule

    \hline
    \multirow{12}*{\rotatebox{90}{\textbf{PEMS-BAY}}}
      & HI & 3.37 & 7.84\% & 3.36 & 7.80\% & 3.36 & 7.77\% & 3.36 & 7.76\% & 3.36 & 7.74\% & 3.38 & 7.79\%\\
      & DLinear & 2.70 & 6.28\% & 3.14 & 7.75\% & 3.13 & 7.77\% & 3.15 & 7.76\% & 3.15 & 7.78\% & 3.23 & 7.90\%\\
      & Informer & 2.77 & 6.65\% & 2.80 & 6.88\% & 2.84 & 7.06\% & 2.83 & 7.07\% & 2.82 & 6.98\% & 2.92 & 7.16\%\\
      & AutoFormer & 3.15 & 7.48\% & 3.24 & 7.85\% & 3.30 & 8.00\% & 3.37 & 8.10\% & 3.39 & 8.15\% & 4.35 & 11.25\%\\
      & FEDFormer & 3.04 & 7.55\% & 3.14 & 7.61\% & 3.13 & 7.58\% & 3.32 & 8.00\% & 3.42 & 8.45\% & 3.67 & 9.33\%\\
      & Pyraformer & 2.53 & 6.21\% & 2.71 & 6.72\% & 2.64 & 6.39\% & 2.74 & 6.65\% & 2.75 & 6.68\% & 2.77 & 6.81\%\\
      \cmidrule{2-14}
      & DCRNN$^*$ & 2.18 & 5.49\% & 2.52 & 6.49\% & 2.54 & 6.43\% & 2.66 & 6.79\% & 2.67 & 6.80\% & 2.66 & 6.62\%\\
      & GWNet$^*$ & 2.01 & 5.11\% & 2.35 & 5.91\% & 2.40 & 5.98\% & 2.47 & 6.35\% & 2.46 & 6.24\% & 2.46 & 6.09\%\\
      & MTGNN$^*$ & 2.17 & 5.40\% & 2.45 & 6.11\% & 2.51 & 6.04\% & 2.52 & 6.13\% & 2.57& 6.19\% & 2.70 & 6.40\%\\
      & STID & 2.02 & 5.02\% & 2.29 & 5.66\% & 2.32 & 5.69\% & 2.33 & 5.72\% & 2.32 & 5.67\% & 2.38 & 5.81\%\\
      & STEP & 00.00 & 00.00\% & 00.00 & 00.00\% & 00.00 & 00.00\% & 00.00 & 00.00\% & 00.00 & 00.00\% & 00.00 & 00.00\%\\
      & D$^2$STGNN$^*$ & 2.04 & 4.97\% & 2.26 & 5.44\% & 2.29 & 5.60\% & 2.34 & 5.55\% & 2.31 & 5.50\% & 2.38 & 5.64\%\\
      \cmidrule{2-14}
      & HUTFormer & \textbf{1.93} & \textbf{4.62\%} & \textbf{2.18} & \textbf{5.16\%} & \textbf{2.21} & \textbf{5.24\%} & \textbf{2.22} & \textbf{5.24\%} & \textbf{2.23} & \textbf{5.25\%} & \textbf{2.28} & \textbf{5.35\%}\\
    \midrule

    \hline
    \multirow{12}*{\rotatebox{90}{\textbf{PEMS08}}}
      & HI & 37.33 & 25.01\% & 37.31 & 25.07\% & 37.23 & 25.05\% & 37.09 & 25.02\% & 36.94 & 24.98\% & 36.40 & 24.76\%\\
      & DLinear & 22.91 & 17.23\% & 34.13 & 24.15\% & 34.34 & 25.54\% & 34.44 & 23.80\% & 34.52 & 23.91\% & 35.11 & 23.71\%\\
      & Informer & 24.55 & 14.76\% & 24.80 & 15.03\% & 24.72 & 15.03\% & 25.07 & 15.11\% & 24.82 & 14.91\% & 25.09 & 15.61\%\\
      & AutoFormer & 31.36 & 25.44\% & 32.29 & 27.13\% & 33.19 & 27.45\% & 32.98 & 26.15\% & 33.57 & 25.78\% & 36.75 & 28.82\%\\
      & FEDFormer & 24.62 & 20.01\% & 26.76 & 21.85\% & 28.56 & 23.02\% & 30.33 & 24.47\% & 29.11 & 23.14\% & 29.91 & 24.47\%\\
      & Pyraformer & 21.92 & 14.43\% & 23.00 & 14.70\% & 23.80 & 15.46\% & 24.45 & 16.88\% & 24.34 & 16.17\% & 22.71 & 14.79\%\\
      \cmidrule{2-14}
      & DCRNN & 18.64 & 13.47\% & 20.42 & 14.92\% & 20.97 & 15.11\% & 21.63 & 15.51\% & 22.45 & 16.23\% & 22.95 & 16.72\%\\
      & GWNet & 17.07 & 11.57\% & 19.55 & 11.93\% & 20.38 & 14.33\% & 20.49 & 14.82\% & 20.00 & 14.68\% & 20.29 & 15.20\%\\
      & MTGNN & 17.75 & 12.61\% &19.27 & 13.35\% & 19.99 & 13.85\% & 20.68 & 15.00\% & 20.95 & 14.65\% & 22.16 & 15.68\%\\
      & STID & 16.40 & 11.42\% & 18.53 & 13.26\% & 19.17 & 13.66\% & 19.59 & 13.78\% & 19.59 & 14.03\% & 20.23 & 15.35\%\\
      & STEP & 00.00 & 00.00\% & 00.00 & 00.00\% & 00.00 & 00.00\% & 00.00 & 00.00\% & 00.00 & 00.00\% & 00.00 & 00.00\%\\
      & D$^2$STGNN$^*$ & 17.27 & 11.47\% & 18.45 & 12.35\% & 18.97 & 12.63\% & 19.33 & 12.81\% & 19.09 & 12.34\% & 19.55 & 12.93\%\\
      \cmidrule{2-14}
      & HUTFormer & \textbf{15.18} & \textbf{10.09\%} & \textbf{16.72} & \textbf{11.26\%} & \textbf{17.23} & \textbf{11.55\%} & \textbf{17.59} & \textbf{11.74\%} & \textbf{17.83} & \textbf{11.84\%} & \textbf{18.44} & \textbf{12.20\%}\\
    \midrule

    \bottomrule
    \end{tabular}
    % }
  \end{table*}

\subsection{Experiment Setup}\label{sec:exp_setup}

\noindent\textbf{Datasets and Baslines.} Two widely adopted recommendation benchmarks are used, Movielens-1M~\citep{harper2015movielens} for movie recommendation, and Recipe~\citep{majumder2019generating} for recipe recommendation.  
To assess \model's efficacy, we compare it against two distinct categories of baselines. The first category includes content-based baselines that takes solely the original item descriptions as input. The second category includes different text augmentation methods.  Details including dataset statistics, preprocessing specifics, baselines, model training, hyper-parameter settings and implementation are discussed in Appendix~\ref{sec:detailed_exp_setting}. 

\noindent\textbf{Language Models.} Two large language models are selected for experiments. The first is {\sc GPT-3}~\citep{brown2020language}, particularly its variant {\tt text-davinci-003}. This variant is an advancement over the InstructGPT models~\citep{ouyang2022training}. We select this variant due to its ability to consistently generate high-quality writing, effectively handle complex instructions, and demonstrate enhanced proficiency in generating longer form content~\citep{raf2023davinci}. The second is {\sc Llama-2}~\citep{touvron2023llama2}, which is an open-sourced model that has shown superior performance across various external benchmarks in reasoning, coding, proficiency, and knowledge tests. Specifically, we use the {\sc Llama-2-Chat} variant of 7B parameters.


\noindent\textbf{Evaluation Protocols.} %To assess the recommendation performance, 
We follow the same evaluation methodology of \citet{wei2019mmgcn}. We randomly divide the dataset into training, validation, and test sets using an 8:1:1 ratio. Negative training samples are created by pairing users and items without any recorded interactions (note that these are pseudo-negative samples). For the validation and test sets, we pair each observed user-item interaction with $n$ items that the user has not previously interacted with. Here we follow the methodology outlined in the previous study~\cite{wei2019mmgcn} and set $n$ to $1,000$. It is important to note that there is \textit{no} overlap between the negative samples in the training set and the unobserved user-item pairs in the validation and test sets. This ensures the independence of the evaluation data. We use metrics such as Precision@K, Recall@K and NDCG@K to evaluate the performance of top-K recommendations, where $K$ is set to $10$. We report the average scores across five different splits of the testing sets. The recommendation module of \model is the combination of an MLP model and a dot product.






















