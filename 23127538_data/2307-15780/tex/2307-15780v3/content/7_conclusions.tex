
\section{Discussions and Conclusions}\label{sec:discussions_and_conclusions}

In this study, we have investigated the effectiveness of \model as a simple yet impactful mechanism for improving recommendation through LLMs. Our approach is among those early attempts~\cite{lin2023can, chen2023large} that leverage LLMs for text augmentation in recommendation. There are three key contributions that distinguish our work from the concurrent ones. 
First, while previous work, such as KAR~\cite{xi2023towards}, focuses on design augmentation algorithm for a specific recommendation model, our model focuses on input text augmentation with LLMs, which is suitable for any content-based backbone recommendation models, demonstrating the flexibility of our approach. 
Second, in addition to our recommendation-driven augmentation using LLMs, we also design engagement-guided prompts to augment the input, which contains more personalized item characteristics. 
Overall, we conduct comprehensive experiments, with different combinations of prompting strategies, to not only illustrate the superior performance of our approach but also uncover the underlying rationale of the improvements.

We introduce \model, which enhances recommendation by augmenting the original item descriptions which often contains incomplete information for effective recommendations using large language models. % Through extensive experiments, 
We observed from extensive experiments that combining augmented input text and original item descriptions yields notable improvements in recommendation quality. These findings show the potential of using LLMs and strategic prompting techniques to enhance the accuracy and relevance of recommendation with an easier training process. By incorporating additional context, we enable the recommendation algorithms to capture more nuanced information and generate recommendations that better align with user preferences.