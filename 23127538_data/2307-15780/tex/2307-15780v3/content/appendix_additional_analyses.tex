\clearpage
\section{Additional Analysis}\label{appendix: additional_analysis}

\subsection{Additional Discussions on Augmented Text}\label{appendix:augmented_text_understanding}
Overall, we observe that \model effectively elicits LLMs to enrich item descriptions with additional information. This augmentation results in descriptions that are not only more detailed but also more expressive, enhancing the original item descriptions.  Tables~\ref{tab:example_gpt_tag} and \ref{tab:example_gpt_infer} show example responses generated by {\sc GPT-3} comparing the recommendation-driven and basic promptings in terms of {\tt tag} and {\tt infer}. Tables~\ref{tab:example_llama2_para}, \ref{tab:example_llama2_tag}, and \ref{tab:example_llama2_infer} show example responses generated by {\sc Llama-2} comparing the recommendation-driven and basic promptings. Responses from the recommendation prompting strategies provides several additional pieces of information and context compared to the responses from the basic prompting strategies. Take the responses in Table~\ref{tab:example_gpt_tag} as an example, the recommendation prompting strategy introduces new themes like ``Supernatural'', ``Paranormal'', and ``Psychological Thriller'', which are not present in the responses from the basic promptings. These themes suggest a broader and more specific context for the story, indicating not just communication with the dead, but also elements of horror and suspense. The term ``Troubled Child'' adds a new dimension to the ``Young Boy'' mentioned in the first sentence, suggesting that the child's character may face internal conflicts or challenges.



\noindent\textbf{What extra information does engagement-guided strategy prompt LLMs to augment?} Consistent with our previous experiments, we curate exemplary responses obtained from $p^{eng}$ for closer examination (Figure~\ref{fig:case_eng}). Our analysis reveals a distinct pattern compared to what we have observed with recommendation-driven prompting. There are primarily two scenarios to consider. First, if the descriptions of the important neighbor items and the target items exhibit high similarity, the impact of $p^{eng}$ resembles that of $p_{para}$, as exemplified in the second Recipe example in Figure~\ref{fig:case_eng}. Second, $p^{eng}$ guides LLMs to generate additional information, which may be derived from the descriptions of the important neighbor items. Consequently, how the engagement-guided strategy influences LLMs' text generation—whether it aligns with one of the behaviors we have described, both of them, or even other unexplored patterns—largely depends on the composition of the important neighbor items. This composition, in turn, is contingent on the neighbor sampling method which is out of the scope of our study. We leave a more in-depth exploration of this topic to future research endeavors.

Interestingly, the recommendation-driven + engagement-guided prompting strategy is able to generate text that shares similar characteristics with both sub-strategies. How they quantitatively form the final generation remains an open question. Examples can be found in Table~\ref{tab:example_gpt_rec_eng}.




Table~\ref{tab:example_llama2_eng} shows example responses of {\sc Llama-2} to the engagement-guided prompting strategy. Table~\ref{tab:example_gpt_rec_eng} shows example responses of {\sc GPT-3} to the recommendation-driven and engagement-guided prompting strategy. Overall, the 7B {\sc Llama-2-Chat} performs poorly compared to {\sc GPT-3}. In some cases, there is no generated content as we have also observed in Appendix~\ref{sec:example_responses_appendix}.






\subsection{Additional Discussions on Applicable Item Domains and Available Textual Information}\label{appendix:applicability}


The applicability of \model beyond movies and recipes, particularly in domains with sparse textual information, remains a question. To address this, we conduct an analysis to determine \model's efficacy in enriching text across various domains and text lengths. We use item descriptions from ten distinct domains in the Amazon review dataset~\cite{ni2019justifying}, which includes product metadata like descriptions, category, price, brand, and image features. The selected domains are all beauty, appliances, automotive, digital music, grocery and gourmet food, pet supplies, sports and outdoors, video games, magazine subscriptions, and industrial and scientific. For each domain, we sample 50 items and prompt {\sc GPT-3} using $p^{rec}_{para}$ (\ie recommendation-driven prompting).

Our previous discussions highlight that the most valuable information for improving recommendation performance typically aligns with expressive words pertinent to item characteristics. While no single metric directly quantifies this added information, we use the increase in the number of adjectives as a proxy. Additionally, the total word count serves as a straightforward metric to approximate the volume of augmented information. By comparing the augmented texts with the original item descriptions, we calculate the percentage increase in the number of adjectives. Note that the adjective increase is computed as a ratio of the difference in adjective count to the original word count.

As Figure~\ref{fig:applicable_domain} demonstrates, \model effectively enriches item descriptions across multiple domains, including those lacking in rich textual content. For instance, the average word counts in movie and digital music descriptions are only 20.34 and 30.18 words, respectively. \model enhances expressiveness, with a notable increase in the use of adjectives. 


\subsection{Additional Experiments on Applying LLM-Rec to Other Baselines}\label{appendix_sec: llm-rec_to_other_baselines}
We extend the application of \model to other text-based recommendation systems and replicate the experiments. The results, as presented in Table~\ref{tab:other_baselines_with_llm_rec}, indicate that \model can be easily adapted to various text-based recommendation systems and generally enhances recommendation performance compared to using the original text.



\begin{table*}[t]
    \centering
    \adjustbox{max width=\textwidth}{
    \begin{tabular}{lccccccc}
    \toprule[1.1pt]
   & & \multicolumn{3}{c}{\textbf{Movielens-1M}}   & \multicolumn{3}{c}{\textbf{Recipe}}  \\
  &    & Precision@10  & Recall@10      & NDCG@10       & Precision@10      & Recall@10    &NDCG@10         \\ \midrule
\multicolumn{2}{l}{AutoInt~\citep{song2019autoint}}                                                & 0.2149  \scriptsize{$\textcolor{gray}{\pm 0.0078}$} &  0.1706  \scriptsize{$\textcolor{gray}{\pm 0.0075}$}    & 0.2698 \scriptsize{$\textcolor{gray}{\pm 0.0092}$} & 0.0351 \scriptsize{$\textcolor{gray}{\pm 0.0032}$}  & 0.0772 \scriptsize{$\textcolor{gray}{\pm 0.0102}$}& 0.0658 \scriptsize{$\textcolor{gray}{\pm 0.0089}$} \\


\multicolumn{2}{l}{\quad - Basic} & 0.2590  \scriptsize{$\textcolor{gray}{\pm 0.0038}$} &  0.2193  \scriptsize{$\textcolor{gray}{\pm 0.0049}$}    & 0.3224 \scriptsize{$\textcolor{gray}{\pm 0.0052}$} & 0.0361 \scriptsize{$\textcolor{gray}{\pm 0.0030}$}  & 0.0797 \scriptsize{$\textcolor{gray}{\pm 0.0097}$}& 0.0664 \scriptsize{$\textcolor{gray}{\pm 0.0076}$} \\

\multicolumn{2}{l}{\quad - Rec} & 0.2593  \scriptsize{$\textcolor{gray}{\pm 0.0035}$} &  0.2197  \scriptsize{$\textcolor{gray}{\pm 0.0068}$}    & 0.3242 \scriptsize{$\textcolor{gray}{\pm 0.0059}$} & 0.0357 \scriptsize{$\textcolor{gray}{\pm 0.0029}$}  & 0.0794 \scriptsize{$\textcolor{gray}{\pm 0.0096}$}& 0.0660 \scriptsize{$\textcolor{gray}{\pm 0.0079}$} \\

\multicolumn{2}{l}{\quad - Eng} & 0.2323  \scriptsize{$\textcolor{gray}{\pm 0.0011}$} &  0.1857  \scriptsize{$\textcolor{gray}{\pm 0.0032}$}    & 0.2899 \scriptsize{$\textcolor{gray}{\pm 0.0031}$} & 0.0349 \scriptsize{$\textcolor{gray}{\pm 0.0028}$}  & 0.0764 \scriptsize{$\textcolor{gray}{\pm 0.0090}$}& 0.0642 \scriptsize{$\textcolor{gray}{\pm 0.0072}$} \\

\multicolumn{2}{l}{\quad - Rec+Eng} & 0.2620  \scriptsize{$\textcolor{gray}{\pm 0.0021}$} &  0.2230  \scriptsize{$\textcolor{gray}{\pm 0.0037}$}    & 0.3270 \scriptsize{$\textcolor{gray}{\pm 0.0022}$} & 0.0349 \scriptsize{$\textcolor{gray}{\pm 0.0029}$}  & 0.0759 \scriptsize{$\textcolor{gray}{\pm 0.0099}$}& 0.0647 \scriptsize{$\textcolor{gray}{\pm 0.0075}$} \\\midrule



\multicolumn{2}{l}{DCN-V2~\citep{wang2021dcn}}   & 0.2961  \scriptsize{$\textcolor{gray}{\pm 0.0050}$}   & 0.2433 \scriptsize{$\textcolor{gray}{\pm 0.0057}$}   & 0.3689   \scriptsize{$\textcolor{gray}{\pm 0.0033}$}      & 0.0360 \scriptsize{$\textcolor{gray}{\pm 0.0036}$} & 0.0786 \scriptsize{$\textcolor{gray}{\pm 0.0104}$} & 0.0653  \scriptsize{$\textcolor{gray}{\pm 0.0085}$}\\

\multicolumn{2}{l}{\quad - Basic}   & 0.3083  \scriptsize{$\textcolor{gray}{\pm 0.0033}$}   & 0.2601 \scriptsize{$\textcolor{gray}{\pm 0.0051}$}   & 0.3842   \scriptsize{$\textcolor{gray}{\pm 0.0054}$}      & 0.0365 \scriptsize{$\textcolor{gray}{\pm 0.0028}$} & 0.0802 \scriptsize{$\textcolor{gray}{\pm 0.0093}$} & 0.0658  \scriptsize{$\textcolor{gray}{\pm 0.0084}$}\\

\multicolumn{2}{l}{\quad - Rec}   & 0.3062  \scriptsize{$\textcolor{gray}{\pm 0.0031}$}   & 0.2572 \scriptsize{$\textcolor{gray}{\pm 0.0049}$}   & 0.3831   \scriptsize{$\textcolor{gray}{\pm 0.0041}$}      & 0.0362 \scriptsize{$\textcolor{gray}{\pm 0.0035}$} & 0.0794 \scriptsize{$\textcolor{gray}{\pm 0.0108}$} & 0.0670  \scriptsize{$\textcolor{gray}{\pm 0.0095}$}\\

\multicolumn{2}{l}{\quad - Eng}   & 0.2990  \scriptsize{$\textcolor{gray}{\pm 0.0024}$}   & 0.2496 \scriptsize{$\textcolor{gray}{\pm 0.0020}$}   & 0.3725   \scriptsize{$\textcolor{gray}{\pm 0.0021}$}      & 0.0356 \scriptsize{$\textcolor{gray}{\pm 0.0032}$} & 0.0786 \scriptsize{$\textcolor{gray}{\pm 0.0094}$} & 0.0647  \scriptsize{$\textcolor{gray}{\pm 0.0076}$}\\

\multicolumn{2}{l}{\quad - Rec+Eng}   & 0.3114  \scriptsize{$\textcolor{gray}{\pm 0.0021}$}   & 0.2641 \scriptsize{$\textcolor{gray}{\pm 0.0038}$}   & 0.3882   \scriptsize{$\textcolor{gray}{\pm 0.0028}$}      & 0.0357 \scriptsize{$\textcolor{gray}{\pm 0.0034}$} & 0.0793 \scriptsize{$\textcolor{gray}{\pm 0.0104}$} & 0.0654  \scriptsize{$\textcolor{gray}{\pm 0.0083}$}\\


             
             \bottomrule[1.1pt]
\end{tabular}}
\caption{Average recommendation performance by applying \model to other text-based recommendation modules across five different train/test splits.}
    \label{tab:other_baselines_with_llm_rec}
\end{table*}




\subsection{Additional Discussions on Integration Process}\label{appendix_sec: integration}
 In our setup, the text encoder is frozen (not fine-tuned), with a fixed output dimension for all vectors. The fundamental difference between the {\tt Concat-All} and Text Concatenation methods lies in their processing sequence. The {\tt Concat-All} method initially transforms individual text segments into embeddings and subsequently concatenates these embeddings. In contrast, the Text Concatenation method first concatenates the text segments and then converts this combined text into a single embedding.

The observed superiority of the {\tt Concat-All} method can be attributed to how these processes handle information density. When lengthy text segments are concatenated before encoding, there is a higher likelihood of \textbf{information loss}, particularly given the constraints of a frozen text encoder. This encoder, \textbf{not} being fine-tuned for the specific nuances of our data, may struggle to effectively capture and retain crucial information from longer text inputs. Therefore, processing shorter text segments individually before concatenation (as in {\tt Concat-All}) may help in preserving important features and nuances in the embeddings.





\subsection{Additional Discussions on Prompt Design}\label{appendix:prompt_design}

To investigate whether modifying the word choice in the designed prompts significantly affects the augmented output, we construct one variant prompt for each of \model's prompts, ensuring they convey the same meaning but with different word choices. Take $p_{para}$ as an example, $p_{para}$ is {\it ``The description of an item is as follows `\{{\tt description}\}', paraphrase it.''}. One variant is {\it ``Summarize the given item description, `\{{\tt description}\}', using different words.''} Next, we randomly sample 50 items from Movielens-1M, and prompt {\sc GPT-3} with these variants. The cosine similarity between the responses generated from the variant prompt and \model's prompt is computed and shown in Table~\ref{tab:appendix_variant_quantitative}. Tables~\ref{tab:variant_basic_para}-\ref{tab:variant_rec_eng} shows the example responses.


\subsection{Additional Discussions on Dynamic Prompts}\label{appendix:dynamic_prompt}
The concept of dynamic prompts in recommendation systems is an intriguing area that holds the potential for enhancing personalization. By incorporating descriptions of a user's most recently interacted items into prompts, the system can generate item descriptions on-the-fly that are more closely aligned with the user's current interests and preferences. This approach could lead to more precise and tailored recommendations, as the generated descriptions would reflect the user's evolving tastes.

One of the primary considerations is the computational cost associated with generating dynamic prompts. Each user interaction would require real-time processing to update the prompt, which could be resource-intensive, especially for large-scale systems with many users and items.

To mitigate computational costs, several strategies can be employed.
Developing efficient algorithms for prompt generation and item description generation can help mitigate computational costs. 
Implementing caching mechanisms for frequently accessed data can reduce the processing time required for updating prompts.
Instead of completely regenerating prompts after each interaction, the system could employ incremental updates to modify prompts based on recent changes in user behavior.


While the implementation of dynamic prompts presents several challenges, it also offers a promising avenue for enhancing personalization in recommendation systems. With careful consideration, this approach has the potential to cater more effectively to individual user needs.




\section{Extended Related Work}\label{sec:detailed_related_work}
\textbf{Augmentation in Text-based Recommendation.} Traditionally, researchers have advocated the augmentation of item descriptions through the incorporation of external knowledge sources~\citep{di2012linked,musto2018semantics,sachdeva2020useful}. Notably, \citet{di2012linked} harnesse data from external databases such as {\tt DBpedia}~\citep{bizer2009dbpedia}, {\tt Freebase}~\citep{bollacker2008freebase}, and {\tt LinkedMDB}~\citep{hassanzadeh2009linked} to gather comprehensive information pertaining to movies, including details about actors, directors, genres, and categories. This approach aimed to enrich the background knowledge available to movie recommender systems. The explicit semantics embedded in these external knowledge sources have demonstrated a discernible enhancement in recommendation performance~\citep{musto2017introducing}. However, this process necessitates a profound domain expertise to effectively and efficiently select and leverage the precise database, ensuring the incorporation of genuinely valuable information into item descriptions~\citep{dumitru2011demand}.


\noindent\textbf{LLM for Recommendation.} The use of large language models in recommender systems has garnered significant attention in recent research~\cite{lin2023can, chen2023large}. Many studies have explored the direct use of LLMs as recommender models. The underlying principle of these approaches involves constructing prompts that encompass the recommendation task, user profiles, item attributes, and user-item interactions. These task-specific prompts are then presented as input to the LLMs, which is instructed to predict the likelihood of interaction between a given user and item~\citep{dai2023uncovering, gao2023chat,geng2022recommendation,li2023exploring,liu2023chatgpt,zhang2023recommendation}. For instance, \citet{wang2023zero} designed a three-step prompting strategy to directly guide LLMs to capture users' preferences, select representative previously interacted items, and recommend a ranked list of 10 items. While these works demonstrate the potential of LLMs as powerful recommender models, the focus primarily revolves around utilizing the LLMs directly for recommendation purposes. However, a potential issue of these methods is that LLMs may generate predictions merely from memorizing training samples which poses a challenge for conducting effective evaluations. In this study, we approach the problem from a different perspective. Rather than using LLMs as recommender models, this study explores diverse prompting strategies to \textit{augment input text} with LLMs for personalized content recommendation. The actual recommendation process still relies on existing recommendation methodologies.

\section{Example Responses}\label{sec:example_responses_appendix}

Tables~\ref{tab:example_response_ml1m} and \ref{tab:example_response_recipe} show example responses by {\sc GPT-3} and the 7B {\sc Llama-2-Chat} on Movielens-1M~\citep{harper2015movielens} and Recipe~\citep{majumder2019generating}. Augmented components are highlighted (recommendation-driven: \purpletext{blue}; engagement-guided: \greentext{green}; rec+eng: \orangetext{orange}). In summary, both {\sc GPT-3} and {\sc Llama-2} exhibit the capability to enrich item descriptions with supplementary information. Nevertheless, the {\sc Llama-2-Chat} model with its 7B parameters demonstrates comparatively poorer performance, which could be attributed to its limited parameter scale. This limitation offers insight into the diminished recommendation quality when using {\sc Llama-2} responses in contrast to {\sc GPT-3}. Future research endeavors should focus on optimizing the \model framework, particularly concerning the selection of different large language models as backbones, to enhance recommendation outcomes.


\begin{table*}[t]
    \centering
    \small
    \adjustbox{max width=\textwidth}{
    \begin{tabular}{p{\textwidth}}
        \toprule[1.1pt]

        \textbf{Description}: A troubled child psychologist helps a young boy who is able to see and communicate with the dead.
\\
\vspace{-1mm}
$p_{tag}$: \#Psychology \#Child \#YoungBoy \#Dead \#Communication \\
$p^{rec}_{tag}$: Tags: Child Psychology, Troubled Child, Young Boy, Communicate with Dead, \purpletext{Supernatural}, \purpletext{Paranormal}, \purpletext{Ghost Story}, Psychological \purpletext{Thriller} \\
\midrule
        \textbf{Description}: This unique Tex-Mex recipe was created by a distant relative who was stationed at the Alamo almost 160 years ago. Family stories say that he was searching for quality tomatoes when Santa Ana launched an attack on the now-iconic Texas landmark. Fortunately, our ancestor had hidden this recipe in his coon-skin cap's lining, ensuring that both he and his avocado dip survived to see another day. Years later, as he lay dying from a chihuahua bite infection, he gazed deeply into his son's eyes and, with his final breath, whispered the memorable words, ``Remember... the... guacamole.''
\\
\vspace{-1mm}
$p_{tag}$: \#TexMex \#Alamo \#AvocadoDip \#CoonSkinCap \#SantaAna \#Chihuahua \#Guacamole \\
$p^{rec}_{tag}$: Tags: Tex-Mex, Alamo, Santa Ana, Avocado Dip, Chihuahua, Guacamole, \purpletext{Recipe}, \purpletext{Family Lore} \\

        \bottomrule[1.1pt]
    \end{tabular}}
    \caption{Example responses to $p_{tag}$ and $p^{rec}_{tag}$ generated by {\sc GPT-3}. The additional information augmented via the
recommendation-driven prompting is highlighted in \purpletext{blue}. }
    \label{tab:example_gpt_tag}
\end{table*}

\begin{table*}[t]
    \centering
    \small
    \adjustbox{max width=\textwidth}{
    \begin{tabular}{p{\textwidth}}
        \toprule[1.1pt]

        \textbf{Description}: A former Roman General is forced to become a gladiator and seek revenge against the corrupt emperor who murdered his family.
\\
\vspace{-1mm}
$p_{infer}$: This description can evoke a range of emotions, including anger, sadness, revenge, determination, and courage. \\
$p^{rec}_{infer}$: This item is sure to evoke strong emotions in its viewers. It tells the story of a former Roman General who is forced to become a gladiator and seek revenge against the corrupt emperor who murdered his family. It is a tale of tragedy, courage, and justice that will leave you feeling \purpletext{inspired} and \purpletext{moved}. \\
\midrule
\textbf{Description}: Our garden's potatoes are ripe and ready, and this recipe is always the first one I go for! They're incredibly delicious!
\\
\vspace{-1mm}
$p_{infer}$: This description can evoke feelings of excitement, anticipation, joy, and satisfaction. \\
$p^{rec}_{infer}$: This recipe is sure to bring a smile to your face! The potatoes are \purpletext{fresh} from the garden, and the \purpletext{flavor is unbeatable}. Enjoy the feeling of \purpletext{accomplishment} as you savor the deliciousness of these potatoes. You won't regret it!
 \\

        \bottomrule[1.1pt]
    \end{tabular}}
    \caption{Example responses to $p_{infer}$ and $p^{rec}_{infer}$ generated by {\sc GPT-3}. The additional information augmented via the
recommendation-driven prompting is highlighted in \purpletext{blue}. }
    \label{tab:example_gpt_infer}
\end{table*}

\begin{table*}[h]
    \centering
    \small
    \adjustbox{max width=\textwidth}{
    \begin{tabular}{p{\textwidth}}
        \toprule[1.1pt]

        \textbf{Description}: Two agents from a secret organization protect the Earth from alien threats while maintaining the public's ignorance of extraterrestrial life.
\\
\vspace{-1mm}
$p_{para}$: `` Two government agents keep Earth safe from intergalactic dangers while concealing the truth about alien life from the public.'' \\
$p^{rec}_{para}$: ``If you're a fan of \purpletext{action-packed sci-fi with a dash of humor}, then this is the show for you! The agents are \purpletext{hilarious} and the alien threats they face are so \purpletext{creative and thrilling}. Plus, the way they manage to keep the public in the dark about alien life is just genius. It's like a \purpletext{sci-fi comedy version of 'Men in Black' with a lot more explosions. Give it a try and let me know what you think!}'' \\
\midrule
        \textbf{Description}: Indulge in something unique tonight. It's tasty and relatively simple to make. Plus, I usually have all the ingredients on hand.
\\
\vspace{-1mm}
$p_{para}$: Sure! Here's an example paraphrase of the original description \\
$p^{rec}_{para}$: If you're looking for something different and easy to make, try this recipe! \purpletext{It's good and has everything you need.} I usually have these ingredients on hand, so it's a great option when you're feeling \purpletext{spontaneous}. \purpletext{Give it a try and let me know what you think!}\\

        \bottomrule[1.1pt]
    \end{tabular}}
    \caption{Example responses to $p_{para}$ and $p^{rec}_{para}$ generated by {\sc Llama-2}. The additional information augmented via the
recommendation-driven prompting is highlighted in \purpletext{blue}. }
    \label{tab:example_llama2_para}
\end{table*}



\begin{table*}[h]
    \centering
    \small
    \adjustbox{max width=\textwidth}{
    \begin{tabular}{p{\textwidth}}
        \toprule[1.1pt]

        \textbf{Description}: An ant embarks on a journey to save his colony from the oppressive rule of a group of grasshoppers.
\\
\vspace{-1mm}
$p_{tag}$: Sure! Here is an example of how you could summarize the description of the item with tags \\
$p^{rec}_{tag}$: One tag you could use to recommend this book to others is \purpletext{``insects''} or \purpletext{``animals in fiction.''} This tag would \purpletext{help others find the book if they are interested in stories that feature insects or animals as main characters.} \\
\midrule
        \textbf{Description}: I adore this crock-pot chicken recipe for a couple of reasons: first, my family can't get enough of it, and second, it's incredibly simple to prepare! My sister shared this recipe with me. She has two kids, and they're big fans as well! Plus, if there are any leftovers, they're the best you'll ever have!
\\
\vspace{-1mm}
$p_{tag}$: Tags \\
$p^{rec}_{tag}$: If you want to recommend the recipe to others, you could use tags such as \purpletext{``chicken recipe'', ``easy recipe'', ``family-friendly recipe'', ``leftovers recipe'', and ``sister's recipe''}. Explanation \\
        \bottomrule[1.1pt]
    \end{tabular}}
    \caption{Example responses to $p_{tag}$ and $p^{rec}_{tag}$ generated by {\sc Llama-2}. The additional information augmented via the
recommendation-driven prompting is highlighted in \purpletext{blue}. }
    \label{tab:example_llama2_tag}
\end{table*}




\begin{table*}[h]
    \centering
    \small
    \adjustbox{max width=\textwidth}{
    \begin{tabular}{p{\textwidth}}
        \toprule[1.1pt]

        \textbf{Description}: A young police officer must race against time to save a bus full of passengers from a bomb planted by a vengeful criminal.
\\
\vspace{-1mm}
$p_{infer}$: Suspense  Explanation \\
$p^{rec}_{infer}$: I would recommend this movie to others because it evokes a \purpletext{sense of urgency and suspense that is both thrilling and emotionally resonant}. The scene where the young police officer races against time to defuse the bomb is \purpletext{heart-pumping and intense}, and the \purpletext{emotional stakes are high as the officer works to save the passengers}. The movie also explores themes of \purpletext{justice and revenge}, making it a \purpletext{thought-provoking and emotionally impactful viewing experience}. Overall, I would \purpletext{highly recommend this movie to anyone looking for a gripping and emotional thriller}.\\
\midrule
\textbf{Description}: Bored with the same old ground beef recipes? Give this spicy dish a try! If you prefer it extra saucy, feel free to double the sauce amount. Update: Since hoisin sauce is quite sweet, you might want to start with a small amount and gradually increase it to your taste!
\\
\vspace{-1mm}
$p_{infer}$: 1. Desire \\
$p^{rec}_{infer}$: ``Hey fellow foodies! Have you tried this Spicy Beef and Noodles recipe? It's a \purpletext{game changer}! The hoisin sauce adds a depth of flavor that will leave you wanting more, and the spicy kick is perfect for those who like a little heat in their meals. \purpletext{I made it for my family last night and they all loved it! Give it a try and let me know what you think! \#foodie \#delicious \#homemade}''
 \\

        \bottomrule[1.1pt]
    \end{tabular}}
    \caption{Example responses to $p_{infer}$ and $p^{rec}_{infer}$ generated by {\sc Llama-2}. The additional information augmented via the
recommendation-driven prompting is highlighted in \purpletext{blue}. }
    \label{tab:example_llama2_infer}
\end{table*}







\input{tab_fig/case_eng}

\begin{table*}[h]
    \centering
    \small
    \adjustbox{max width=\textwidth}{
    \begin{tabular}{p{\textwidth}}
        \toprule[1.1pt]

        \textbf{Description}: A troubled child psychologist helps a young boy who is able to see and communicate with the dead.
\\
\vspace{-1mm}
$p^{eng}$:  Despite their different subjects and genres, the descriptions share a common theme of \greentext{personal transformation and empowerment.} Each story revolves around a \greentext{character's journey towards self-discovery and self-acceptance, as they navigate challenges and obstacles that help them grow and develop as individuals.}\\
        \bottomrule[1.1pt]
    \end{tabular}}
    \caption{Example responses to $p^{eng}$ generated by {\sc Llama-2}. The additional information augmented via the
engagement-guided prompting is highlighted in \greentext{green}. }
    \label{tab:example_llama2_eng}
\end{table*}








\begin{table*}[h]
    \centering
    \small
    \adjustbox{max width=\textwidth}{
    \begin{tabular}{p{\textwidth}}
        \toprule[1.1pt]

        \textbf{Description}: The Paper Chase follows a first-year Harvard Law student as he struggles to balance his studies and personal life while facing the demanding expectations of his professor.
\\
\vspace{-1mm}
$p^{rec+eng}$:  If you're looking for a movie that will keep you on the edge of your seat, The Paper Chase is a great choice. It follows a first-year Harvard Law student as he navigates the pressures of his studies and personal life while facing the expectations of his professor. It's a story of \orangetext{determination} and \orangetext{resilience} in the face of adversity, and it will leave you feeling \orangetext{inspired} and \orangetext{motivated}.\\
\midrule
\textbf{Description}: Our garden's potatoes are ripe and ready, and this recipe is always the first one I go for! They're incredibly delicious!
\\
\vspace{-1mm}
$p^{rec+eng}$: This potato recipe is a \orangetext{must-try}! It's \orangetext{packed with flavor and so easy to make}. \orangetext{Plus, you don't need to add any extra salt - the ranch dressing mix does the trick}. It's a great way to get your \orangetext{family} to enjoy potatoes \orangetext{in a new way}. Plus, it's a great way to use up \orangetext{leftovers if you have any}. \orangetext{Give it a try - you won't be disappointed!}\\
        \bottomrule[1.1pt]
\end{tabular}}
\caption{Example responses to $p^{rec+eng}$ generated by {\sc GPT-3}. The additional information augmented via the
engagement-guided prompting is highlighted in \orangetext{orange}. }
    \label{tab:example_gpt_rec_eng}
\end{table*}








\begin{table*}[h]
    \centering
    \small
    \adjustbox{max width=\textwidth}{
    \begin{tabular}{p{\textwidth}}
        \toprule[1.1pt]

        \textbf{Description}: In this Cold War-era thriller, a Soviet submarine captain attempts to defect to the United States with his state-of-the-art vessel.
\\ \midrule
$p_{para}$: The description of an item is as follows: `\{description\}', paraphrase it. \\
\vspace{-1mm}
\textbf{Response:} In this suspenseful movie set during the Cold War, a captain of a Soviet submarine attempts to defect to the US with his advanced submarine.\\ \midrule
\textbf{Variant Prompt:} Summarize the given item description, `\{description\}', using different words. \\
\vspace{-1mm}
\textbf{Response:} This suspenseful movie from the Cold War period follows a Soviet submarine commander who attempts to defect to America with his advanced submarine.\\ 
        \bottomrule[1.1pt]
    \end{tabular}}
    \caption{Example responses generated by {\sc GPT-3} for $p_{para}$ and its variant that shares the same meaning but differs in word choices.}
    \label{tab:variant_basic_para}
\end{table*}





\begin{table*}[h]
    \centering
    \small
    \adjustbox{max width=\textwidth}{
    \begin{tabular}{p{\textwidth}}
        \toprule[1.1pt]

        \textbf{Description}: In this Cold War-era thriller, a Soviet submarine captain attempts to defect to the United States with his state-of-the-art vessel.
\\ \midrule
$p_{tag}$: The description of an item is as follows: `\{description\}', summarize it with tags.\\
\vspace{-1mm}
\textbf{Response:} Cold War, Soviet, submarine, defect, United States, state-of-the-art.\\ \midrule
\textbf{Variant Prompt:} Condense the provided item description `\{description\}' using tags. \\
\vspace{-1mm}
\textbf{Response:} \#ColdWar \#Thriller \#SovietSubmarine \#Defect \#US \#StateOfTheArtVessel\\ 
        \bottomrule[1.1pt]
    \end{tabular}}
    \caption{Example responses generated by {\sc GPT-3} for $p_{tag}$ and its variant that shares the same meaning but differs in word choices.}
    \label{tab:variant_basic_tag}
\end{table*}




\begin{table*}[h]
    \centering
    \small
    \adjustbox{max width=\textwidth}{
    \begin{tabular}{p{\textwidth}}
        \toprule[1.1pt]

        \textbf{Description}: In this Cold War-era thriller, a Soviet submarine captain attempts to defect to the United States with his state-of-the-art vessel.
\\ \midrule
$p_{infer}$: The description of an item is as follows: `\{description\}', what kind of emotions can it evoke? \\
\vspace{-1mm}
\textbf{Response:} This description can evoke feelings of suspense, tension, and excitement. It can also evoke feelings of patriotism and loyalty as the Soviet submarine captain attempts to defect to the United States.\\ \midrule
\textbf{Variant Prompt:} Based on the description `\{description\}', what emotions might it elicit? \\
\vspace{-1mm}
\textbf{Response:} -Tension-Suspense-Fear-Anxiety-Dread-Unease-Paranoia\\ 
        \bottomrule[1.1pt]
    \end{tabular}}
     \caption{Example responses generated by {\sc GPT-3} for $p_{infer}$ and its variant that shares the same meaning but differs in word choices.}
    \label{tab:variant_basic_infer}
\end{table*}






\begin{table*}[h]
    \centering
    \small
    \adjustbox{max width=\textwidth}{
    \begin{tabular}{p{\textwidth}}
        \toprule[1.1pt]

        \textbf{Description}: In this Cold War-era thriller, a Soviet submarine captain attempts to defect to the United States with his state-of-the-art vessel.
\\ \midrule
$p^{rec}_{para}$: The description of an item is as follows: `\{description\}', what else should I say if I want to recommend it to others? \\
\vspace{-1mm}
\textbf{Response:} This Cold War-era thriller is a gripping and suspenseful story that will keep you on the edge of your seat. It follows the daring journey of a Soviet submarine captain as he attempts to defect to the United States with his state-of-the-art vessel. With its thrilling plot and captivating characters, this movie is sure to be an unforgettable experience.\\ \midrule
\textbf{Variant Prompt:} In addition to the description `\{description\}', what other details would you include to endorse it to others? \\
\vspace{-1mm}
\textbf{Response:} This Cold War-era thriller follows the story of a Soviet submarine captain, Marko Ramius, who attempts to defect to the United States with his state-of-the-art vessel. Along the way, he must outwit and outmaneuver a relentless CIA analyst, Jack Ryan, who is determined to stop him. With intense action sequences and a gripping plot, this movie will keep you on the edge of your seat as the two men battle for control of the submarine. Featuring an all-star cast including Sean Connery, Alec Baldwin, and Scott Glenn, The Hunt for Red October is a thrilling and suspenseful ride that will leave you wanting more.\\ 

        \bottomrule[1.1pt]
    \end{tabular}}
    \caption{Example responses generated by {\sc GPT-3} for $p^{rec}_{para}$ and its variant that shares the same meaning but differs in word choices.}
    \label{tab:variant_rec_para}
\end{table*}









\begin{table*}[h]
    \centering
    \small
    \adjustbox{max width=\textwidth}{
    \begin{tabular}{p{\textwidth}}
        \toprule[1.1pt]

        \textbf{Description}: In this Cold War-era thriller, a Soviet submarine captain attempts to defect to the United States with his state-of-the-art vessel.
\\ \midrule
$p^{rec}_{tag}$: The description of an item is as follows: `\{description\}', what tags should I use if I want to recommend it to others?\\
\vspace{-1mm}
\textbf{Response:} Tags: Cold War, Thriller, Soviet Submarine, Defection, United States, State-of-the-Art Vessel\\ \midrule
\textbf{Variant Prompt:} If you aim to recommend `\{description\}' to others, what tags or terms would you use? \\
\vspace{-1mm}
\textbf{Response:} Tags/Terms: Cold War, Thriller, Soviet Submarine, Defection, United States, State-of-the-Art Vessel.\\ 
        \bottomrule[1.1pt]
    \end{tabular}}
    \caption{Example responses generated by {\sc GPT-3} for $p^{rec}_{tag}$ and its variant that shares the same meaning but differs in word choices.}
    \label{tab:variant_rec_tag}
\end{table*}



\begin{table*}[h]
    \centering
    \small
    \adjustbox{max width=\textwidth}{
    \begin{tabular}{p{\textwidth}}
        \toprule[1.1pt]

        \textbf{Description}: In this Cold War-era thriller, a Soviet submarine captain attempts to defect to the United States with his state-of-the-art vessel.
\\ \midrule
$p^{rec}_{infer}$: The description of an item is as follows: `\{description\}', recommend it to others with a focus on the emotions it can evoke.\\
\vspace{-1mm}
\textbf{Response:} This Cold War-era thriller will take you on an intense journey of suspense and intrigue. Follow a Soviet submarine captain as he attempts to defect to the United States with his state-of-the-art vessel, and experience the tension and drama of a high-stakes game of cat and mouse. Feel the fear and anticipation as the captain navigates a dangerous course of political and personal risk.\\ \midrule
\textbf{Variant Prompt:} Using the description `\{description\}', endorse this item to others while emphasizing the emotions it can inspire. \\
\vspace{-1mm}
\textbf{Response:} This Cold War-era thriller will have you on the edge of your seat! Follow the gripping story of a Soviet submarine captain as he attempts to defect to the United States with his state-of-the-art vessel. Feel the tension and suspense as you watch this thrilling tale unfold. Don't miss out on this captivating story of courage and determination!\\ 
        \bottomrule[1.1pt]
    \end{tabular}}
     \caption{Example responses generated by {\sc GPT-3} for $p^{rec}_{infer}$ and its variants that shares the same meaning but differs in word choices.}
    \label{tab:variant_rec_infer}
\end{table*}








\begin{table*}[h]
    \centering
    \small
    \adjustbox{max width=\textwidth}{
    \begin{tabular}{p{\textwidth}}
        \toprule[1.1pt]

        \textbf{Description}: The Paper Chase follows a first-year Harvard Law student as he struggles to balance his studies and personal life while facing the demanding expectations of his professor.\\
        \vspace{-1mm}
        \textbf{Descriptions of other important neighbors}: `The Paper Chase follows a first-year Harvard Law student as he struggles to balance his studies and personal life while facing the demanding expectations of his professor.'; `The Shawshank Redemption is a story of hope, friendship, and resilience in the face of despair.'; `A criminal fakes insanity to escape prison, only to find himself in a mental institution where he must battle the oppressive head nurse.'
\\ \midrule
$p^{eng}$: Summarize the commonalities among the following descriptions: `\{description\}'; `\{descriptions of other important neighbors\}'.\\
\vspace{-1mm}
\textbf{Response:} The commonalities among the descriptions are that they all involve characters facing difficult challenges and having to find a way to overcome them. In The Paper Chase, the protagonist must balance his studies and personal life while facing the demanding expectations of his professor. In The Shawshank Redemption, the characters must find hope, friendship, and resilience in the face of despair. In the other description, the criminal must fake insanity to escape prison and then battle the oppressive head nurse in the mental institution.\\ \midrule
\textbf{Variant Prompt:}  Outline the shared characteristics of the following descriptions: `\{description\}' and `\{descriptions of other important neighbors\}'.\\
\vspace{-1mm}
\textbf{Response:} The shared characteristics of the descriptions are: - A protagonist facing a difficult situation - A struggle to balance personal life and studies - Overcoming adversity - Hope, friendship, and resilience - Oppressive forces to be overcome.\\ 

        \bottomrule[1.1pt]
    \end{tabular}}
    \caption{Example responses generated by {\sc GPT-3} for $p^{eng}$ and its variant that shares the same meaning but differs in word choices.}
    \label{tab:variant_eng}
\end{table*}



\begin{table*}[h]
    \centering
    \small
    \adjustbox{max width=\textwidth}{
    \begin{tabular}{p{\textwidth}}
        \toprule[1.1pt]

        \textbf{Description}: The Paper Chase follows a first-year Harvard Law student as he struggles to balance his studies and personal life while facing the demanding expectations of his professor.\\
        \vspace{-1mm}
        \textbf{Descriptions of other important neighbors}: `The Paper Chase follows a first-year Harvard Law student as he struggles to balance his studies and personal life while facing the demanding expectations of his professor.'; `The Shawshank Redemption is a story of hope, friendship, and resilience in the face of despair.'; `A criminal fakes insanity to escape prison, only to find himself in a mental institution where he must battle the oppressive head nurse.'
\\ \midrule
$p^{rec+eng}$: The description of an item is as follows: `\{description\}'. What else should I say if I want to recommend it to others? This content is considered to hold some similar attractive characteristics as the following descriptions: `\{descriptions of other important neighbors\}'.\\
\vspace{-1mm}
\textbf{Response:} If you're looking for a movie that will keep you on the edge of your seat, The Paper Chase is a great choice. It follows a first-year Harvard Law student as he navigates the pressures of his studies and personal life while facing the expectations of his professor. It's a story of determination and resilience in the face of adversity, and it will leave you feeling inspired and motivated.\\ \midrule
\textbf{Variant Prompt:} If you wish to recommend `\{description\}' to others, what other details would you provide? This content possesses certain akin attractive attributes as the following descriptions: `\{descriptions of other important neighbors\}'. \\
\vspace{-1mm}
\textbf{Response:} The Paper Chase is a classic drama that follows the struggles of a first-year Harvard Law student as he navigates the demanding expectations of his professor while trying to balance his studies and personal life. It is a story of perseverance, determination, and resilience in the face of adversity. The film features an all-star cast, including John Houseman, Timothy Bottoms, and Lindsay Wagner, and is sure to captivate viewers with its gripping plot and powerful performances.\\
        \bottomrule[1.1pt]
    \end{tabular}}
    \caption{Example responses generated by {\sc GPT-3} for $p^{rec+eng}$ and its variant that shares the same meaning but differs in word choices.}
    \label{tab:variant_rec_eng}
\end{table*}




\begin{table*}[h]
    \centering
    \small
    \resizebox{\textwidth}{!}{
    \begin{tabular}{lcccccc}
    \toprule
    & \multicolumn{3}{c}{\textbf{Movielens-1M}}   & \multicolumn{3}{c}{\textbf{Recipe}}  \\ %\cline{2-13} 
      & Precision@10  & Recall@10      & NDCG@10       & Precision@10      & Recall@10    & NDCG@10         \\ \midrule
      Original Item Description & 0.2922 \scriptsize{$\textcolor{gray}{\pm 0.0019}$} & 0.2455 \scriptsize{$\textcolor{gray}{\pm 0.0031}$}& 0.3640 \scriptsize{$\textcolor{gray}{\pm 0.0039}$} & 0.0325 \scriptsize{$\textcolor{gray}{\pm 0.0021}$}& 0.0684 \scriptsize{$\textcolor{gray}{\pm 0.0066}$}& 0.0580 \scriptsize{$\textcolor{gray}{\pm 0.0054}$}\\
  Duplicating Original Description Embeddings & 0.2858 \scriptsize{$\textcolor{gray}{\pm 0.0020}$}&0.2417 \scriptsize{$\textcolor{gray}{\pm 0.0032}$}&0.3567 \scriptsize{$\textcolor{gray}{\pm 0.0043}$}& 0.0327 \scriptsize{$\textcolor{gray}{\pm 0.0030}$}&0.0694 \scriptsize{$\textcolor{gray}{\pm 0.0082}$} &0.0590 \scriptsize{$\textcolor{gray}{\pm 0.0068}$}\\
  Text Concatenation & 0.3075 \scriptsize{$\textcolor{gray}{\pm 0.0015}$}& 0.2636 \scriptsize{$\textcolor{gray}{\pm 0.0035}$}& 0.3853 \scriptsize{$\textcolor{gray}{\pm 0.0027}$} & 0.0332 \scriptsize{$\textcolor{gray}{\pm 0.0026}$}& 0.0714 \scriptsize{$\textcolor{gray}{\pm 0.0083}$}& 0.0591 \scriptsize{$\textcolor{gray}{\pm 0.0065}$} \\

             {\tt Concat-All} (ours)     & \textbf{0.3150}  \scriptsize{$\textcolor{gray}{\pm 0.0023}$} &\textbf{0.2766}  \scriptsize{$\textcolor{gray}{\pm 0.0030}$} & \textbf{0.3951}  \scriptsize{$\textcolor{gray}{\pm 0.0035}$}& \textbf{0.0394}  \scriptsize{$\textcolor{gray}{\pm 0.0033}$} & \textbf{0.0842}  \scriptsize{$\textcolor{gray}{\pm 0.0098}$}&  \textbf{0.0706}  \scriptsize{$\textcolor{gray}{\pm 0.0084}$}\\ 
             \bottomrule
\end{tabular}}
\caption{Concatenating embeddings of the responses augmented by \model outperforms concatenating duplicate embeddings of original description. It also achieves a superior performance than concatenating the raw text.}
\label{tab:concat}
\end{table*}
