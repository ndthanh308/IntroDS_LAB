\begin{table*}[h]
    \centering
    \small
    \adjustbox{max width=\textwidth}{
    \begin{tabular}{p{\textwidth}}
        \toprule[1.1pt]

        \textbf{Description}: An ant embarks on a journey to save his colony from the oppressive rule of a group of grasshoppers.
\\
\vspace{-1mm}
$p_{tag}$: Sure! Here is an example of how you could summarize the description of the item with tags \\
$p^{rec}_{tag}$: One tag you could use to recommend this book to others is \purpletext{``insects''} or \purpletext{``animals in fiction.''} This tag would \purpletext{help others find the book if they are interested in stories that feature insects or animals as main characters.} \\
\midrule
        \textbf{Description}: I adore this crock-pot chicken recipe for a couple of reasons: first, my family can't get enough of it, and second, it's incredibly simple to prepare! My sister shared this recipe with me. She has two kids, and they're big fans as well! Plus, if there are any leftovers, they're the best you'll ever have!
\\
\vspace{-1mm}
$p_{tag}$: Tags \\
$p^{rec}_{tag}$: If you want to recommend the recipe to others, you could use tags such as \purpletext{``chicken recipe'', ``easy recipe'', ``family-friendly recipe'', ``leftovers recipe'', and ``sister's recipe''}. Explanation \\
        \bottomrule[1.1pt]
    \end{tabular}}
    \caption{Example responses to $p_{tag}$ and $p^{rec}_{tag}$ generated by {\sc Llama-2}. The additional information augmented via the
recommendation-driven prompting is highlighted in \purpletext{blue}. }
    \label{tab:example_llama2_tag}
\end{table*}


