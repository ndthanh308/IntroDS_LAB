\section{Related Work}

Modern large-scale search systems are tiered~\cite{Wang2011SIGIR} with at least two layers. The \emph{candidate retrieval} layer generates a small subset of potentially relevant documents from a corpus many orders of magnitude larger in size, while emphasizing efficiency and recall. The \emph{re-ranking} layer uses more computationally expensive methods to re-rank the candidates generated by the retrieval stage to produce a high-precision final result list. Better recall in candidate retrieval leads to better overall accuracy. In this paper, we focus on the candidate retrieval layer.

Candidate retrieval has traditionally used lexical retrieval approaches such as BM25~\cite{robertson_okapi_1995}. Recently, neural dense retrievers, which generate candidates by finding documents similar to the query in embedding space, have risen in popularity \cite{lee_latent_2019,karpukhin_dense_2020,xiong_approximate_2021}.
Dense retrievers have complementary strengths to lexical retrieval and combining the two approaches together have shown strong results~\cite{chen_out--domain_2022,wang_bert-based_2021}. XWalk is complementary to \emph{both} lexical and dense retrieval.

Random-walk based approaches are frequently used in large, real-time recommendation systems due to their effectiveness and efficiency \cite{paudel_updatable_2016,eksombatchai_pixie_2018}. In addition, when using implicit feedback (e.g. logged interaction data such as user clicks) Park et al. \cite{park_comparative_2017} showed that random walk based approaches can perform better than matrix factorization approaches. 

Historically, graph-based approaches in information retrieval were used to create features (e.g. PageRank, click graphs~\cite{jiang_learning_2016,zhang_neural_2019}) for ranking. Recently, graph neural networks (GNNs) have achieved state of the art performance in recommendation tasks and are being adapted for search \cite{zamani_learning_2020,xia_searchgcn_2021,zhao_joint_2022}. However, large-scale GNNs are often complex and cumbersome to train. XWalk is simple, efficient graph-based approach to search that trains in a fraction of the time and is highly efficient in inference. 

% Cite papers about:
% \begin{itemize}
%     \item recsys using random walk based approaches.
%     \item IR that use graphs: click graph features, page rank/personalized page rank, joint search/recs learning.
%     \item old data fusion techniques: TODO go digging in Alistair/Shane's recent data fusion papers for citations
%     \item some seminal query log analysis/user interaction mining papers?
%     \item general neural IR approaches?
%     \item work on combining different retrieval engines
% \end{itemize}
