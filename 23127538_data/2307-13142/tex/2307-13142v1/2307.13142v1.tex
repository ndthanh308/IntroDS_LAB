% ----------------------------------------------------------------
% AMS-LaTeX Paper ************************************************
% **** -----------------------------------------------------------
\documentclass[11pt]{amsart}
\usepackage{graphicx}
\usepackage{hyperref}
%\usepackage{pdfpages}
% ----------------------------------------------------------------
\vfuzz2pt % Don't report over-full v-boxes if over-edge is small
\hfuzz2pt % Don't report over-full h-boxes if over-edge is small
% THEOREMS -------------------------------------------------------
\newtheorem{thm}{Theorem}[section]
\newtheorem{cor}[thm]{Corollary}
\newtheorem{cond}[thm]{Condition}
\newtheorem{lem}[thm]{Lemma}
\newtheorem{prop}[thm]{Proposition}
\theoremstyle{definition}
\newtheorem{defn}[thm]{Definition}
\theoremstyle{remark}
\newtheorem{rem}[thm]{Remark}
%\numberwithin{equation}{}
% MATH -----------------------------------------------------------
\newcommand{\norm}[1]{\left\Vert#1\right\Vert}
\newcommand{\abs}[1]{\left\vert#1\right\vert}
\newcommand{\set}[1]{\left\{#1\right\}}
\newcommand{\Real}{\mathbb R}
\newcommand{\eps}{\varepsilon}
\newcommand{\To}{\longrightarrow}
\newcommand{\BX}{\mathbf{B}(X)}
\newcommand{\A}{\mathcal{A}}
\renewcommand{\baselinestretch}{1.0}
\setcounter{page}{1}

\begin{document}
\title[The convergence of power matrices]{The convergence of power matrices}%
\author{Vyacheslav M. Abramov}%
\address{24 Sagan Drive, Cranbourne North, Vic-3977, Australia}%

\email{vabramov126@gmail.com}%

%\thanks{$^*$ \url{https://orcid.org/0000-0002-9859-100X}}%
\subjclass{30A99, 15A16, 15B51}%
\keywords{Markov chains, matrices with complex entries; power matrices; stochastic matrices}

\begin{abstract}
For the class of $d\times d$ matrices $B=[b_{i,j}]$ with complex nonzero entries satisfying $\sum_{i=1}^{d}|b_{i,j}|=1$, we provide the conditions for the convergence of power matrices $B^n$ to a nonzero limit matrix. In particular, for $d\geq3$ we prove that the sequence of matrices $B^n$ converges to a nonzero limit matrix if and only if all the entries are positive and real.
\end{abstract}

\maketitle


\section{Introduction}

The basic results on the theory of real square matrices  with nonnegative entries are based on the Perron--Frobenius (PF) theorem \cite{F,P}. The PF theorem has various applications that include probability theory, theory of dynamical systems, economy etc \cite{L, S}. One of the widely used applications of PF theorem that is related to the theory of discrete Markov chains is as follows.
Let $B$ be a real square matrix with nonnegative entries $b_{i,j}$ satisfying $\sum_{j=1}^{d}b_{i,j}=1$ or $\sum_{i=1}^{d}b_{i,j}=1$, $i=1,2,\ldots,d$. If $B$ is aperiodic and irreducible (see e.g. \cite[pp. 11--13]{T} for the definitions), then the positive limit of $B^n$ (i.e. a limit matrix with nonnegative entries in which some of the entries are strictly positive), as $n$ tends to infinity, exists (see \cite[p. 15]{T}). In particular, if all the entries of the matrix $B$ are strictly positive, then it is irreducible and aperiodic, and $B^n$ converges to a limit matrix, the entries of which are strictly positive. A matrix with nonnegative entries satisfying $\sum_{j=1}^{d}b_{i,j}=1$ or $\sum_{i=1}^{d}b_{i,j}=1$, $i=1,2,\ldots,d$, is called stochastic matrix. Stochastic matrices were originally used by Russian mathematician Andrey Markov \cite{M} to describe transition probabilities in certain probability problems with outcomes depending on the experiment conditions and then widely developed covering many areas of application of Markov chains.

Whereas the theory of stochastic matrices, the entries of which are real is well-developed and well-known, it is not much known about its analogue, in which the entries are complex. Sequences of power matrices with complex entries can arise in probability problems, in the situations when the entries of a matrix are generating functions or Laplace-Stieltjes transforms of some probability distributions. As well, the problem of the convergence of a sequence of power matrices with complex entries can arise in the theory of differential equations. The material of this paper can be a subject for further more deepen study of matrices with complex entries as well.
For some elementary facts about matrices with complex entries see \cite[Ch. 7]{AH}.

The aim of this paper is to obtain an extension of the aforementioned known result on matrices with real entries to the case of matrices with complex entries.
Namely, we consider a $d\times d$ matrix $B$ with nonzero complex entries $b_{i,j}$ satisfying $\sum_{i=1}^{d}|b_{i,j}|=1$, $j=1,2,\ldots, d$, and
provide the conditions under which the sequence $B^n$ converges to a nonzero matrix. Surprisingly, in the cases $d=2$ and $d\geq3$ the obtained results in the form of necessary and sufficient conditions are distinct, and it turns out that the result for the case $d\geq3$ is simpler.

The rest of the article is structured into two sections and appendix. In Section \ref{S2} we study the case $d=2$, and then in Section \ref{S3} we study the case $d\geq3$. The appendix contains auxiliary statements that are used to prove the required theorems.


\section{The case $d=2$}\label{S2}
In this section, we consider $2\times2$ complex matrices $B$ that satisfy the property $|b_{1,1}|+|b_{2,1}|=|b_{1,2}|+|b_{2,2}|=1$. We prove the following theorem.

\begin{thm}\label{t1}
Let $B=[b_{i,j}]$ be a $2\times2$ complex matrix with nonzero entries, $|b_{1,1}|+|b_{2,1}|=|b_{1,2}|+|b_{2,2}|=1$. Assume also that $b_{1,2}$, $b_{2,1}$ do not take negative real values.  Then the sequence $B^n$ converges to a nonzero matrix if and only if (i) $b_{i,i}$, $i=1,2$, are positive real,  and (ii) $b_{1,2}b_{2,1}=|b_{1,2}b_{2,1}|$.
\end{thm}

\begin{proof}
We first introduce the class of equivalent matrices and lemma that are used to prove the required theorem.

\subsection{Definition and lemma}

\begin{defn}
Two matrices $A=[a_{i,j}]$ and $\tilde{A}=[\tilde{a}_{i,j}]$ with complex entries are called \textit{likewise}, if $|a_{i,j}|=|\tilde{a}_{i,j}|$ for all $i$, $j$.
\end{defn}

\begin{lem}\label{lem1}
Let $A=[a_{i,j}]$ be a $2\times2$ real matrix in which $a_{1,2}\geq0$ and $a_{2,1}\geq0$, and let $\tilde{A}=[\tilde{a}_{i,j}]$ be the matrix, the entries of which satisfy the relations $\tilde{a}_{i,i}=a_{i,i}$, $i=1,2$, and $\tilde{a}_{1,2}=a_{1,2}\mathrm{e}^{\mathbf{i}\varphi}$,
$\tilde{a}_{2,1}=a_{2,1}\mathrm{e}^{-\mathbf{i}\varphi}$, $\varphi\neq\pi$.
Then, for any $n\geq1$, the matrices $A^n$ and $\tilde{A}^n$ are likewise.
\end{lem}
 \begin{proof} The proof of the lemma is given by induction.
Denote the entries of $A^n$ by $a_{i,j}^{(n)}$ and the entries of $\tilde{A}^n$ by $\tilde{a}_{i,j}^{(n)}$. Apparently,
\[
\tilde{a}_{i,i}^{(2)}=\tilde{a}_{i,i}^2+\tilde{a}_{1,2}\tilde{a}_{2,1}=a_{i,i}^2+a_{1,2}a_{2,1}=a_{i,i}^{(2)}, \quad i=1,2,
\]
\[
\tilde{a}_{1,2}^{(2)}=\tilde{a}_{1,1}\tilde{a}_{1,2}+\tilde{a}_{1,2}\tilde{a}_{2,2}=(a_{1,1}a_{1,2}+a_{1,2}a_{2,2})\mathrm{e}^{\mathbf{i}\varphi}=a_{1,2}^{(2)}\mathrm{e}^{\mathbf{i}\varphi},
\]
\[
\tilde{a}_{2,1}^{(2)}=\tilde{a}_{2,1}\tilde{a}_{1,1}+\tilde{a}_{2,2}\tilde{a}_{2,1}=(a_{2,1}a_{1,1}+a_{2,2}a_{2,1})\mathrm{e}^{-\mathbf{i}\varphi}=a_{2,1}^{(2)}\mathrm{e}^{-\mathbf{i}\varphi}.
\]
Now, assuming that $\tilde{a}_{i,i}^{(n)}=a_{i,i}^{(n)}$, $i=1,2$, $\tilde{a}_{1,2}^{(n)}=a_{1,2}^{(n)}\mathrm{e}^{\mathbf{i}\varphi}$ and $\tilde{a}_{2,1}^{(n)}=a_{2,1}^{(n)}\mathrm{e}^{-\mathbf{i}\varphi}$, by the similar derivations we obtain: $\tilde{a}_{i,i}^{(n+1)}=a_{i,i}^{(n+1)}$, $i=1,2$, $\tilde{a}_{1,2}^{(n+1)}=a_{1,2}^{(n+1)}\mathrm{e}^{\mathbf{i}\varphi}$ and $\tilde{a}_{2,1}^{(n+1)}=a_{2,1}^{(n+1)}\mathrm{e}^{-\mathbf{i}\varphi}$.
The proof of the lemma is completed.
 \end{proof}

\subsection{Proof of the sufficient condition}
Using the polar system, write $b_{i,j}=|b_{i,j}|\mathrm{e}^{\mathbf{i}\theta_{i,j}}$, $i,j=1,2$; $\mathbf{i}=\sqrt{-1}$. According to the assumption $b_{1,1}$ and $b_{2,2}$ are real positive, $b_{1,2}=|b_{1,2}|\mathrm{e}^{\mathbf{i}\theta}$, $b_{2,1}=|b_{2,1}|\mathrm{e}^{-\mathbf{i}\theta}$, $\theta\in[0,2\pi)\setminus\{\pi\}$. (The value $\theta=\pi$ is excluded due to the made convention on the entries of the matrix $B$.) Let $|B|$ denote the $2\times2$ real matrix, the entries of which are $|b_{i,j}|$. According to Lemma \ref{lem1} the matrices $B^n$ and $|B|^n$ are likewise, and due to \cite[p. 15]{T}, the sequence $|B|^n$ converges to the nonzero limit. Then, following the proof of Lemma \ref{lem1}, the sequence $B^n$ converges to a nonzero limit too.

\subsection{Proof of the necessary condition}
Assume that at least one of the two assumptions (i) or (ii) is violated. We are to prove that then the sequence $B^n$ does not converge to a nonzero matrix.
Indeed, using the notation $X^{(n)}=B^n$ write $X^{(n+1)}=X^{(n)}B$. Then denoting the entries of $X^{(n)}$ by $x_{i,j}^{(n)}$ we have
\begin{equation}\label{0}
x_{i,k}^{(n+1)}=x_{i,1}^{(n)}|b_{1,k}|\mathrm{e}^{\mathbf{i}\theta_{1,k}}+x_{i,2}^{(n)}|b_{2,k}|\mathrm{e}^{\mathbf{i}\theta_{2,k}}, \quad i,k=1,2.
\end{equation}
Assume in contrary that there exists the nonzero limiting matrix $X^{(*)}=\lim_{n\to\infty}X^{(n)}$, the entries of which are denoted $x_{i,j}^{(*)}$.  Then we obtain $X^{(*)}=X^{(*)}B$, and the system of the equations for the entries is
\begin{equation}\label{0.5}
x_{i,k}^{(*)}=x_{i,1}^{(*)}|b_{1,k}|\mathrm{e}^{\mathbf{i}\theta_{1,k}}+x_{i,2}^{(*)}|b_{2,k}|\mathrm{e}^{\mathbf{i}\theta_{2,k}}, \quad i,k=1,2.
\end{equation}
%Assuming that $X^{(*)}$ is a nonzero matrix we in fact assumed that all the entries $x_{i,j}^{(*)}$ in this system of the equations are nonzero. If at least one of $x_{i,j}^{(*)}$ is zero, then all other ones become zero automatically, and the assumption is violated.

Taking the absolute values, we have:
\begin{equation}\label{1}
|x_{i,k}^{(*)}|\leq|x_{i,1}^{(*)}b_{1,k}|+|x_{i,2}^{(*)}b_{2,k}|, \quad i,k=1,2.
\end{equation}
If at least one of these four inequalities is strict, then, according to Lemma \ref{lemA} (see Appendix), we obtain $|x_{i,j}^{(*)}|=0$ and consequently, $x_{i,j}^{(*)}=0$ for all $i$ and $j$. For the nonzero matrix-solution $X^{(*)}$, we need to have the equalities
\begin{equation}\label{1.3}
x_{i,i}^{(*)}b_{i,i}=|x_{i,i}^{(*)}b_{i,i}|, \quad i=1,2,
\end{equation}
as well as
\begin{equation}\label{1.5}
x_{1,2}^{(*)}b_{2,1}=|x_{1,2}^*b_{2,1}| \quad \text{and} \quad x_{2,1}^{(*)}b_{1,2}=|x_{2,1}^*b_{1,2}|.
\end{equation}
All these equalities are achievable in the only cases when $b_{i,i}$, $i=1,2$, are positive real, and $b_{1,2}b_{2,1}= |b_{1,2}b_{2,1}|$.

Indeed, note first that $b_{1,2}b_{2,1}= |b_{1,2}b_{2,1}|$ holds if and only if $b_{1,2}=|b_{1,2}|\mathrm{e}^{\mathbf{i}\theta}$ and $b_{2,1}=|b_{2,1}|\mathrm{e}^{-\mathbf{i}\theta}$ for all $\theta\in[0, 2\pi)$. The case $\theta=\pi$ should be excluded according to the made convention on the entries of the matrix $B$. The entries $b_{1,2}$ and $b_{2,1}$  in this case are negative, and according to Lemma \ref{corA} (see Appendix) the nonzero limit matrix $X^{(*)}$ does not exist. However, for $\theta\in[0, 2\pi)\setminus\{\pi\}$ the equalities $b_{1,2}b_{2,1}= |b_{1,2}b_{2,1}|$ together with the inequalities $b_{1,1}>0$ and $b_{2,2}>0$ imply \eqref{1.3} and \eqref{1.5}.

%\begin{defn}\label{defn1}
%Let $A$ be a $2\times2$ real matrix with entries $a_{i,j}$, where $a_{1,2}$ and $a_{2,1}$ are positive, and let $\tilde{A}$ be the $2\times2$ matrix with the entries $\tilde{a}_{i,j}$ defined as follows: $\tilde{a}_{1,1}=a_{1,1}$, $\tilde{a}_{2,2}=a_{2,2}$, $\tilde{a}_{1,2}=a_{1,2}\mathrm{e}^{\mathbf{i}\theta}$ and $\tilde{a}_{2,1}=a_{2,1}\mathrm{e}^{-\mathbf{i}\theta}$ for some $\theta\in(0, 2\pi)\setminus\{\pi\}$. Then the matrices $A$ and $\tilde{A}$ are called \textit{similar}.
%\end{defn}
%\begin{rem}\label{rem1}
%In the degenerate case $a_{1,2}=a_{2,1}=0$, the matrices $A$ and $\tilde{A}$ are also may be considered as similar matrices.
%\end{rem}



%\begin{lem}\label{lem1}
%For any integer $n\geq1$, the matrices $A^n$ and $\tilde{A}^n$ are similar.
%\end{lem}
%
%\begin{proof}
%The proof of this lemma follows directly by induction. The degenerate case mentioned in Remark \ref{rem1} may appear in this proof while calculating $A^n$ and $\tilde{A}^n$.
%\end{proof}

%Let us adapt this lemma to our case. Consider the recurrence relations given by \eqref{0}, in which $b_{1,1}$ and $b_{2,2}$ are positive real, and $b_{1,2}=|b_{1,2}|\mathrm{e}^{\mathbf{i}\theta}$, $b_{2,1}=|b_{2,1}|\mathrm{e}^{-\mathbf{i}\theta}$ ($\theta\neq\pi$). We obtain the following interesting properties: $x_{1,1}^{(n)}$ and $x_{2,2}^{(n)}$ are positive real, $x_{1,2}^{(n)}=|x_{1,2}^{(n)}|\mathrm{e}^{\mathbf{i}\theta}$, $x_{2,1}^{(n)}=|x_{2,1}^{(n)}|\mathrm{e}^{-\mathbf{i}\theta}$, and $|x_{1,1}^{(n)}|+|x_{2,1}^{(n)}|=|x_{1,2}^{(n)}|+|x_{2,2}^{(n)}|=1$ for all $n$. The last means that the limits $x_{i,j}^{(*)}$ exists, and $|x_{1,1}^{(*)}|+|x_{2,1}^{(*)}|=|x_{1,2}^{(*)}|+|x_{2,2}^{(*)}|=1$.

Assume first that (ii) is violated, i.e. $b_{1,2}$ and $b_{2,1}$ take complex values such that $|b_{1,2}b_{2,1}|\neq b_{1,2}b_{2,1}$. Then, it follows from \eqref{0} for $n=1$ that
%
%To finish the proof of the necessary condition, we are to prove that if at least one of the conditions $b_{1,2}b_{2,1}= |b_{1,2}b_{2,1}|$, $b_{1,1}>0$ or $b_{2,2}>0$ is not satisfied, then the sequence $B^n$ does not converge to a nonzero matrix. If $b_{1,2}$ and $b_{2,1}$ take complex values such that $|b_{1,2}b_{2,1}|\neq b_{1,2}b_{2,1}$, then it follows from \eqref{0} for $n=1$ that
\[
|x_{i,k}^{(2)}|\leq|b_{i,1}b_{1,k}|+|b_{i,2}b_{2,k}|, \quad i,k=1,2,
\]
and at least one of these four inequalities must be strict. Then, according to Lemma \ref{lemA} (see Appendix), the sequence $B^{2n}$ converges to the zero matrix, that in turn implies that the sequence $B^n$ does not converge to a nonzero matrix.

Assume now that (i) is violated, i.e. at least one of $b_{1,1}$ and $b_{2,2}$ is not real positive. We now prove that then the sequence $B^n$ does not converge to a nonzero matrix. Because of the similarity, it is enough to provide the proof assuming only that $b_{1,1}$ is not real positive.

%Let us now prove that if $b_{1,1}$ and $b_{2,2}$ are not real positive, then the sequence $B^n$ does not converge to a nonzero matrix. Because of the similarity, it is enough to provide the proof for $b_{1,1}$ only.

Assume first that $b_{1,1}$ is not real, i.e. $b_{1,1}=|b_{1,1}|\mathrm{e}^{\mathbf{i}\theta_{1,1}}$ with $\theta_{1,1}\in(0, 2\pi)\setminus\{\pi\}$. Then $|b^2_{1,1}|<|b_{1,1}|^2$, and we arrive at the system of two inequalities $|x_{1,1}^{(2)}|+|x_{2,1}^{(2)}|<1$ and $|x_{1,2}^{(2)}|+|x_{2,2}^{(2)}|\leq1$. From this, according to Lemma \ref{lemA} (see Appendix), we arrive at the conclusion that the sequence $B^{2n}$ converges to the zero matrix, that in turn implies that the sequence $B^n$ does not converge to a nonzero matrix.


%It will be shown later that for $\theta\in[0, 2\pi)\setminus\{\pi\}$ the required limit matrix exists if in addition $b_{1,1}$ and $b_{2,2}$ are real and strictly positive.

%Let us prove below that $b_{1,1}$ and $b_{2,2}$ must be positive real. The proof will be provided only for $b_{1,1}$, since for $b_{2,2}$ it is similar.



Consider now the case when $b_{1,1}$ is real negative. If the matrix $B$ is real, then the result follows from Lemma \ref{corA} (see Appendix). Therefore, we are to consider the case when $b_{1,2}=|b_{1,2}|\mathrm{e}^{\mathbf{i}\theta}$ and $b_{2,1}=|b_{2,1}|\mathrm{e}^{-\mathbf{i}\theta}$, where $\theta\in(0, 2\pi)\setminus\{\pi\}$. Now, by Lemma \ref{lem1} one can replace the original sequence of matrices $B^n$ by the corresponding sequence of likewise matrices containing the only real entries, i.e. we are to replace $b_{1,2}$ and $b_{2,1}$ in the matrix $B$ with $|b_{1,2}|$ and $|b_{2,1}|$, respectively.  In other words, the matrix $B$ with the real entries is the only case, and this case have already been considered. The required statement is proved.

Therefore, if the nonzero limit of the sequence of matrices $B^n$ exists, then $b_{1,1}$ and $b_{2,2}$ must be real positive, and $b_{1,2}b_{2,1}= |b_{1,2}b_{2,1}|$.
\end{proof}
%
%
%\subsection{Second-level heading.}
%
%The same goes for second-level headings.  It is not necessary to add font commands to make the math within heads bold and sans serif; this change will occur automatically when the production style is applied.
\section{The case $d\geq3$}\label{S3}
In this section, we study a matrix $B$ with nonzero complex entries $b_{i,j}$, $i,j=1,2,\ldots,d$, where $d\geq3$.  The following theorem holds true.

\begin{thm}
Assume that $B$ is $d\times d$ complex matrix, $d\geq3$, with nonzero entries, and $\sum_{i=1}^{d}|b_{i,j}|=1$ for each $j$. Then the sequence of matrices $B^n$ converges to a nonzero matrix if and only if all $b_{i,j}$ are positive and real.
\end{thm}

\begin{proof}
%Note first that if the sequence $B^n$ converges to the nonzero limit, the similar is true for the transpose matrix $B^\prime$, since $(B^\prime)^n=(B^n)^\prime$ for any $n$.
If $B$ is a matrix with real positive entries, then under the made assumptions the sufficient condition follows from \cite{F, P, S} or \cite[p. 15]{T}.

Therefore, we need to prove the necessary condition only. We use the arguments similar to those given in the proof of Theorem \ref{t1}. Assume that at least one of $b_{i,j}$ has imaginary part. We are to prove that a nonzero limit of $X^{(n)}=B^n$, as $n\to\infty$, does not exist. Write $X^{(n+1)}=X^{(n)}B$, and assume in contrary that there exists a nonzero limiting matrix $X^{(*)}=\lim_{n\to\infty}X^{(n)}$, with the entries $x_{i,j}^{(*)}$.  Then $X^{(*)}=X^{(*)}B$, and we need to have
$x_{i,k}^{(*)}=\sum_{j=1}^{d}x_{i,j}^{(*)}b_{j,k}$, $j,k=1,2,\ldots,d$. Similarly to \eqref{1} given in the proof of Theorem \ref{t1}, we have the system of inequalities:
\begin{equation}\label{2}
|x_{i,k}^{(*)}|\leq\sum_{j=1}^{d}|x_{i,j}^{(*)}b_{j,k}|, \quad i,k=1,2,\ldots,d.
\end{equation}
However, compared to the case of \eqref{1} the number of summand-terms on the right-hand side of \eqref{2} is more than two. If at least one of the inequalities in \eqref{2} is strict, then according to Lemma \ref{lemA} there is no nonzero limit of $B^n$ as $n\to\infty$.

%Therefore, in order to have all the equalities in \eqref{2}, we must have the system $x_{i,j}^{(*)}b_{j,k}=|x_{i,j}^{(*)}b_{j,k}|$ for all combinations of $i,j,k$ that appear in the expansion. 
Note first that the diagonal elements $b_{i,i}$ and consequently (according to \eqref{2}) $x_{i,i}^{(*)}$ must be real. Indeed, we have
\begin{equation}\label{2.5}
|x_{i,i}^{(2)}|\leq\sum_{j=1}^{d}|b_{i,j}b_{j,i}|, \quad i=1,2,\ldots,d.
\end{equation}
Then, if at least one of $b_{i,i}$ is not real, then one of the inequalities of \eqref{2.5} must be strict. Specifically, if $b_{i,i}=|b_{i,i}|\mathrm{e}^{\mathbf{i}\theta}$, $\theta\in(0, 2\pi)\setminus\{\pi\}$, then $|b_{i,i}^2|<|b_{i,i}|^2$, and according to Lemma \ref{lemA} a nonzero limit matrix $X^{(*)}$ does not exist.

Then we are to consider the system $x_{i,j}^{(*)}b_{j,k}=|x_{i,j}^{(*)}b_{j,k}|$ for all $i\neq j$ and $j\neq k$.
We are to prove that the
 aforementioned equalities can be satisfied in the only case when all the entries $b_{i,j}$ are real and positive. That is, we are to prove that if at least one of $b_{i,j}$, $i\neq j$ has an imaginary part or at least one of $b_{i,j}$ is real negative, then the system of equalities becomes inconsistent.

Assume in contrary that $b_{j,k}=|b_{j,k}|\mathrm{e}^{\mathbf{i}\theta}$ for a certain pair $j, k$ ($j\neq k)$, and some $\theta\in(0,2\pi)$. Then $x_{i,j}^{(*)}=|x_{i,j}^{(*)}|\mathrm{e}^{-\mathbf{i}\theta}$ for all $j\neq i$. It turns out from this, that the set of entries of the matrix $B$ having the presentation similar to that $b_{j,k}$ is extended, and for all of the non-diagonal entries of the matrix $B$ we must have $b_{i,j}=|b_{i,j}|\mathrm{e}^{\mathbf{i}\theta}$. Then for all of the non-diagonal entries of the matrix $X^{(*)}$ we must have $x_{i,j}^{(*)}=|x_{i,j}^{(*)}|\mathrm{e}^{-\mathbf{i}\theta}$. It leads to the contradiction. If $\theta\neq\pi$, then the equality $X^{(*)}=X^{(*)}B$ with nonzero matrix $X^{(*)}$ is impossible, since the arguments of the entries on the left- and right-hand sides are different. If $\theta=\pi$, then a nonzero limit matrix $X^{(*)}$ does not exist according to Lemma \ref{corA}. Finally, if at least one of $b_{i,i}$ is negative, then a nonzero limit matrix $X^{(*)}$ does not exist according to Lemma \ref{corA} either.
\end{proof}


\subsection*{Data availability statement}
Data sharing not applicable to this article as no datasets were generated or analysed during the current study.

\subsection*{Disclosures and declarations}
No conflict of interest was reported by the author

\subsection*{Authorship clarified}
The author, who conducted this work has no relation to any institution.




%\subsubsection*{Acknowledgment}
%The author thanks all people, who helped in preparation of this note officially or privately.

\appendix
\section{Auxiliary statements}

We prove the following results.

\begin{lem}\label{lemA}
Let $B$ be a $d\times d$ matrix with positive real entries $b_{i,j}$ satisfying the system of inequalities $\sum_{i=1}^{d}b_{i,j}\leq1$. If at least one of these inequalities is strict, then the sequence $B^n$ converges to zero matrix.
\end{lem}

\begin{proof}
Let $X^{(n)}=B^n$. Denote the limit of $X^{(n)}$ as $n\to\infty$ by $X^{(*)}$ and the entries of the limit matrix by $x_{i,j}^{(*)}$. (Since $B$ contains only positive entries, then the required limit exists \cite{F,P,S}.) Then we have the system of equations
\begin{equation}\label{A1}
x_{i,k}^{(*)}=\sum_{j=1}^{d}x_{i,j}^{(*)}b_{j,k}, \quad i,k=1,2,\ldots,d.
\end{equation}
Apparently that if $x_{i,j}^{(*)}=0$ for all $i,j$, then the system of equations \eqref{A1} is satisfied. Prove that this is the only solution. Assume in contrary that the matrix $X^{(*)}$ is not zero matrix, but having positive entries. Then, summing up in $i$ and $k$ both sides of \eqref{A1}, we obtain $\sum_{i=1}^{d}\sum_{k=1}^{d}x_{i,k}^{(*)}>\sum_{i=1}^{d}\sum_{k=1}^{d}x_{i,k}^{(*)}$. This contradiction indicates that the only solution is $x_{i,j}^{(*)}=0$ for all $i,j$.
\end{proof}

\begin{lem}\label{corA}
Let $B$ be a $d\times d$ matrix with nonzero real entries $b_{i,j}$ satisfying the inequalities $\sum_{i=1}^{d}|b_{i,j}|\leq1$ for each $j$. If at least one of the entries $b_{i,j}$ is negative, then the sequence $B^n$ does not converge to a nonzero matrix.
\end{lem}
\begin{proof} In the case when all entries of $B$ are negative, a nonzero limit of $B^n$ does not exist, since the entries of $B^{2n}$ are positive, while the entries of $B^{2n-1}$ are negative, $n\geq1$. Hence, we consider the case when the matrix $B$ has both positive and negative entries.
For $X^{(n)}=B^n$ we have $X^{(n+1)}=X^{(n)}B$. Let $x_{i,j}^{(n)}$ denote the entries of $X^{(n)}$. We arrive at the inequality
\[
|x_{i,k}^{(n+1)}|\leq\max\left\{\sum_{j=1}^{d}|x_{i,j}^{(n)}|\max\{b_{j,k},0\}, -\sum_{j=1}^{d}|x_{i,j}^{(n)}|\min\{b_{j,k},0\}\right\}.
\]
Consequently,
\[
\begin{aligned}
&\sum_{i=1}^{d}\sum_{k=1}^{d}|x_{i,k}^{(n+1)}|\\
&\leq\sum_{i=1}^{d}\sum_{k=1}^{d}\max\left\{\sum_{j=1}^{d}|x_{i,j}^{(n)}|\max\{b_{j,k},0\}, -\sum_{j=1}^{d}|x_{i,j}^{(n)}|\min\{b_{j,k},0\}\right\}.
\end{aligned}
\]
For the entries of $B$ we have the inequalities $-1\leq\sum_{i=1}^{d}b_{i,j}\leq1$, at least one of which is strong. Hence,
\[
\sum_{i=1}^{d}\sum_{k=1}^{d}|x_{i,k}^{(n+1)}|\leq\alpha\sum_{i=1}^{d}\sum_{k=1}^{d}|x_{i,k}^{(n)}|,
\]
for some $\alpha<1$. The fixed point theorem (e.g. \cite[p. 1]{KVZRS}) enables us to conclude that $\lim_{n\to\infty}|x_{i,j}^{(n)}|=0$, and consequently $\lim_{n\to\infty}x_{i,j}^{(n)}=0$.
\end{proof}


\begin{thebibliography}{2}
\bibitem{AH} Andrilli, S.; Hecker, D.: \emph{Elementary Linear Algebra}. 5th ed. Academic Press, Boston, 2016.
\bibitem{F} Frobenius, G.: \"{U}ber Matrizen aus nicht negativen Elementen. Sitzungsberichte der K\"{o}niglich Preussischen Academie der Wissenschaften: (5) (2012), 456--477. \url{https://ia800501.us.archive.org/26/items/mobot31753002089602/mobot31753002089602.pdf}
\bibitem{KVZRS}  Krasnosel'skii, M. A.; Vainikko, G. M.; Zabreiko,  P. P.; Rutitskii, Y. B.;  Stetsenko,  V. Y.: \emph{Approximate Solutions of Operator Equations}. Wolters-Noordhoff, Groningen, 1972.


\bibitem{L} Lankaster, K.: \emph{Mathematical Economics}. Macmillan, New York, 1968.

\bibitem{M} Markov, A. A.: Extension of the law of large numbers to quantities, depending on each other (1906). Reprint. \emph{Journal \'{E}lectronique d'Histoire des Probabilit\'{e}s et de la Statistique} [electronic only] 2(1b) (2006), article 10, 1--12. (In old Russian.) \url{eudml.org/doc/128778}.

\bibitem{P}
Perron, O.: Zur Theorie der Matrices. \emph{Mathematische Annalen}, 64(2) (1907), 248--263. \url{https://doi.org/10.1007/BF01449896}

\bibitem{S}
Seneta, E.: \emph{Nonnegative Matrices and Markov Chains}. Springer, New York, 1981.

\bibitem{T}
Tak\'{a}cs, L.: \emph{Stochastic Processes: Problems and Solutions}. John Wiley \& Sons Inc., New York, 1960.

\end{thebibliography}

\end{document} 