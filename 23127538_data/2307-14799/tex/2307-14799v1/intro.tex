Scheduling semiconductor manufacturing processes imposes a complex challenge due to the variety of products, operations, and high-tech machines with diverse capabilities and characteristics.
% to produce a diverse range of products. 
Effective scheduling aims at allocating jobs to machines in a manner that satisfies production needs, optimizes factory throughput, and guarantees punctual delivery~\cite{upasani2006problem}.
% Adding to the intricacy of the scheduling process is the job's flexibility, as each job can be carried out using several machines.
In view of the steadily increasing demand~\cite{lesliePandemicScramblesSemiconductor2022},
semiconductor manufacturers are forced to optimize their throughput, decrease cycle times, and enhance the on-time delivery of products to customers~\cite{pfund2008multi}.
Corresponding approaches to provide decision support can be categorized into planning at the strategic level and operating a factory at the tactical or execution level. In this paper, we present an approach for scheduling realistic semiconductor manufacturing processes, taking multiple optimization objectives such as throughput or makespan as well as setup and batching criteria into account.
Our work builds on the recent SMT2020 simulation scenario~\cite{kopp2020smt2020},
providing datasets that model the production processes of modern wafer fabs. 

A typical wafer fabrication plant encompasses a variety of process flows, which are designated production routes for wafer lots within the factory. Each route consists of several hundred operations to be processed by machines belonging to about one hundred separate tool groups with specific functionalities and characteristics. % which can be quite expensive \cite{kopp2020smt2020}. 
% In many cases, identical machine groups process lots in parallel, forming a batch.
To reduce the required investments into costly machines~\cite{kopp2020smt2020}, constant utilization and idleness prevention are important process goals.
% such costly machines are often shared among all lots that require a specific processing operation, even if they are at different stages of their manufacturing process flow.
Moreover, sophisticated process steps are iterated in several stages, which results in a re-entrant flow where wafer lots
revisit machines in the same tool group multiple times.
Hence, the manufacturing environment is different from traditional flow-shop and job-shop scenarios~\cite{DBLP:journals/mor/GareyJS76}.
A crucial consequence of this re-entrant flow is that wafers at different stages in their manufacturing cycle can compete for the same machines, and dispatching strategies to resolve such competing demands have a noticeable impact on the overall production efficiency.

% Additionally, the operations along a production route vary substantially in duration, leading to a highly non-linear process flow. While some operations run quickly and take half an hour or less, the completion of others may need half a day. Usually such long running operations involve batching, where a machine is capable of processing multiple wafer lots simultaneously.
% In particular, about one third of the machines featured by the SMT2020 scenario
% are batching tools, at which wafer lots queue in front and are then processed in groups.
% The need to perform periodic maintenance procedures further affects the availability of machines, and product-specific setups required for some of the tools, e.g., ion implanters, take equipping time and may induce bottlenecks if operations are scheduled improperly.
% The simulation and real-world processes also involve stochastic uncertainties and unscheduled events, such as varying processing times or sudden machine disruptions, which limit the accuracy and time horizon of predicting the progression of semiconductor manufacturing processes, thus necessitating frequent re-scheduling to deal with the dynamics of a wafer fab.

As a consequence of the complexity and dynamicity faced in practice,
the wafer production is typically controlled by
handcrafted~\cite{pfundSemiconductorManufacturingScheduling2006} or 
machine-learned~\cite{waschneckDeepReinforcementLearning2018}
dispatching rules at the execution level,
or (re-)scheduling is localized to specific tool groups~\cite{monchSurveyProblemsSolution2011}, e.g., for optimizing the allocation of lots queuing in front of a group of batching machines.
While such local decision making approaches are tuned to specific fab settings,
their scope is generally too narrow to guarantee overall efficiency in terms of optimization objectives.
Unlike that, our work makes a step towards large-scale scheduling by modeling the production processes of a modern wafer fab, represented by the SMT2020 scenario, using Answer Set Programming (ASP) with difference logic \cite{gebser2016theory,janhunen2017clingo}.
% Our declarative approach incorporates
We significantly extend our preliminary approach introduced in~\cite{ali2023flexible} and incorporate crucial features of realistic semiconductor fabs, including flexible machine processing, setup, batching and maintenance operations, as well as multiple optimization objectives reflecting the factory throughput, setup and batching criteria.
% The main contributions of our work are the following:%

The paper is organized as follows.
Section~\ref{sec:review} surveys related literature on scheduling in traditional job-shop scenarios and particular challenges encountered in semiconductor manufacturing.
In Section~\ref{sec:smsp}, we introduce the scheduling problem including crucial features of the SMT2020 scenario as well as the necessary background on ASP with difference logic.
Our hybrid ASP with difference logic model enabling the large-scale scheduling of semiconductor manufacturing processes is presented in Section~\ref{sec:asp}.
In Section~\ref{sec:results}, we perform an experimental evaluation examining the potentials of large-scale scheduling subject to multiple optimization objectives.
Section~\ref{sec:conclusion} concludes the paper with a brief summary and outlook on future work.

% The semiconductor manufacturing scheduling problem (SMSP) can be considered as NP-hard, which means that finding an optimal solution is computationally intractable for large problem instances. This complexity arises due to the large number of possible schedules, the number of machines involved, and the intricate relationships between jobs and machines. 

% To address this complexity, various heuristic and metaheuristic approaches have been proposed in the literature. These methods aim to find near-optimal solutions within a reasonable amount of time, often sacrificing optimality for computational efficiency. Despite these efforts, SMSP remains a challenging and active area of research, with new approaches and algorithms continually being developed to tackle its complexity. 

% A new approach to solve SMSP is presented that leverages recent advances in optimization and machine learning. Our approach aims to improve scheduling efficiency and reduce completion time of lots while meeting other practical constraints. We believe that our approach represents a significant step forward in this field and will contribute to the ongoing efforts to address the complexity of the SMSP.