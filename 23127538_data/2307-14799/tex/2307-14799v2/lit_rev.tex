Semiconductor fab scheduling is a highly complex task %, mainly because of the typical features, such as complex 
due too sophisticated producting routes,
diverse machine characteristics, and
rapidly changing demands~\cite{ellis2004scheduling}. 
% To develop effective scheduling methods in wafer fabrication is thus challenging. The combined used of a simulation and an optimization algorithms provides an efficient solution to scheduling problems.
While lacking specific features of the semiconductor manufacturing process such as, e.g., re-entrant flow, batching, setup and maintenance operations, as well as varying processing times and sudden machine disruptions, the Flexible Job-Shop Scheduling Problem (FJSP) \cite{brusch90a,taillard93a} along with the optimization methods devised for it are related approaches.

% SMSP can be considered as flexible job shop problem (FJSP) due to the similarities in the structure and complexity of both problems. Multiple operations can be performed at on multiple machines. This flexibility in machine assignment is a characteristic feature of FJSP, where each operation can be performed on one of several available machines. Both SMSP and FJSP involve constraints related to machine availability, limited buffer space, and tooling requirements and complex objectives, such as minimizing cycle time, minimize tardiness, and maximizing machine utilization. All these features need to be considered when scheduling operations to avoid bottlenecks and ensure smooth production flow. The methodologies and algorithms developed for FJSP can be applied to SMSP, such as dispatching rules, metaheuristic optimization algorithms, simulation, and artificial intelligence, has been employed to address the complexity and challenges of FJSP.  

% FJSP is combinatorial optimization problem in which each operation can be processed by a set of identical machines. Several methods have been developed in the literature for solving flexible job shop scheduling problems. These methods can be broadly classified into exact and heuristic methods. 

% Exact methods attempt to find the optimal solution to the scheduling problem by examining all possible combinations of job sequences and machine assignments. Exact methods include mathematical programming techniques such as linear programming, integer programming, and mixed-integer programming. These methods have been used to solve smaller instances of the flexible job shop scheduling problem.

% Heuristic methods, on the other hand, aim to find a good solution to the problem in a reasonable amount of time, without examining all possible solutions. Heuristic methods are often faster and more practical than exact methods for solving larger instances of the flexible job shop scheduling problem.

A wide range of techniques have been proposed in the literature to tackle combinatorial optimization for FJSP solving.
Meta-heuristic algorithms incorporate local search methods,
such as Genetic Programming \cite{li2016effective,wang2001effective}, Tabu Search~\cite{li2016effective}, Simulated Annealing~\cite{wang2001effective}, Harmony Search~\cite{sahraeian2017new}, Particle Swarm Optimization~\cite{hassanzadeh2016two}, and Ant Colony Optimization~\cite{xing2010knowledge}.
% are also extensively applied to FJSP.
Exact solving methods are based on FJSP models in 
Mixed Integer Programming (MIP) \cite{ceylan2021coordinated,gran2015mixed,ham2021energy},
Constraint Programming (CP) \cite{da2019industrial,ham2021energy}, or 
ASP with difference logic \cite{el2022problem,janhunen2017clingo}.

Beyond FJSP, ASP~\cite{lifschitz19a} has been successfully used
to schedule
printing devices~\cite{balduccini2011industrial},
specialist teams~\cite{rigralmaliiile12a},
work shifts~\cite{abseher2016shift},
course timetables~\cite{bainkaokscsotawa18a},
medical treatments~\cite{dogagrmamopo21a},
and
aircraft routes~\cite{tassel2021multi}.
The hybrid framework of ASP with difference logic~\cite{janhunen2017clingo} particularly supports a compact representation
and reasoning with quantitative resources like time,
which has been exploited in domains such as
lab resource~\cite{francescutto2021solving},
train connection~\cite{abels2021train}, and
parallel machine~\cite{eiter2022answer} scheduling, as well as for FJSP solving
\cite{el2022problem,janhunen2017clingo}.
In this work, we extend our preliminary ASP with difference logic approach~\cite{ali2023flexible} to semiconductor fab scheduling with support for batching machines, partially flexible machine allocation strategies, and multi-objective optimization functionalities.

% A wide range of techniques have been proposed in literature to deal with FJSP. A mixed integer programming for FJSP is developed by \cite{gran2015mixed}. Another model with Mixed Integer Programming (MIP) was given by \cite{ham2021energy,ceylan2021coordinated}. Meta heuristics algorithms hybrid with local search are also extensively used for FJSP such as Genetic Programming \cite{li2016effective,wang2001effective}, Tabu Search \cite{li2016effective}, Simulated Annealing \cite{wang2001effective}, Harmony Search \cite{sahraeian2017new}, Particle Swarm \cite{hassanzadeh2016two} and Ant Colony Optimization \cite{xing2010knowledge}. Constraint Programming (CP) formulation for FJSP is presented by \cite{ham2021energy,da2019industrial}. Answer set programming (ASP) based approach for machine scheduling is presented by many studies such as \cite{balduccini2011industrial,eiter2022answer,el2022decomposition,francescutto2021solving}. 

% To improve efficiency of methods, researches have developed hybrid languages and and solvers such as ASP+CP has been used in industrial size applications \cite{balduccini2011industrial}. Similarly, MIP+CP is studied by \cite{ham2021energy}.

% Despite the similarities, SMSP is more complex then FJSP due to process complexity, with hundreds of operations and numerous workstations involved in production of a single semiconductor wafer. A re-entrant flow adds complexity to this as it requires managing job in a re-entrant loops. Additional feature include the batch processing where some operations are performed on batch tools, which consist of multiple processing chambers sharing a common wafer handling system. Semiconductor manufacturing equipments are highly sensitive and prone to failures, hence require regular maintenance. This leads to machine downtime and uncertainties in machine availability, which must be factored into the scheduling process. Therefore, scheduling algorithms need to be adapted and extended to address the specific challenges and constraints of SMSP, which is the aim of this paper.