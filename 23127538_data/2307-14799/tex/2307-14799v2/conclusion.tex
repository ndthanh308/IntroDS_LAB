This work extends our preliminary SMSP approach~\cite{ali2023flexible}
with crucial features, namely, scalable and informed preallocation strategies to reduce the instantiation size and search complexity, as well as batch
processing and multiple optimization objectives.
While we enhance the scheduling scalability and coverage of real-world features,
our mid-term goal is to incorporate scheduling into the real or simulated management of semiconductor manufacturing processes.
As next step into this direction, we aim to use scheduling for improving the decision making in the PySCFabSim simulator~\cite{kotaalelgese22a}, where
methods available so far, i.e., handcrafted dispatching rules or black-box machine
learning models, function locally and do not take the global impact of their decisions into account.

% In this work, we proposed a model for solving a semiconductor scheduling problem using Answer Set Programming supported by the difference logic. Our multi-objective optimization model considers machine maintenance while optimizing the makespan, setup and batching. We propose strategies for assigning steps to machines and investigating their performance. We created different instances based on the SMT2020 dataset to asses our model, and the results showed that assigning the steps requiring the same setup to the same machine had a significant impact on the schedule. On the other side, assigning the steps of the same lot to the same machine had an unfavorable impact on the quality of the schedule. In addition, it is crucial to partially assign steps to machines by splitting the machines into subgroups. In future work, we will consider the cascading machines that execute more than one lot simultaneously and integrate the model with a simulator to schedule a work center for a specific period. We aim to improve the batching method by assigning each step to any batch w.r.t the execution time.