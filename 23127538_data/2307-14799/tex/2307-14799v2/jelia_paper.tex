\documentclass[runningheads]{llncs}

\usepackage{helvet,times,courier}
\usepackage[T1]{fontenc}
\usepackage{graphicx}
\usepackage{listings}
\usepackage{enumitem}
\usepackage{xspace}
\usepackage{amssymb}

\section{Code and License}
For the baselines, we use the code from the following repositories:
\begin{itemize}
    \item BGES: We use the code from \cite{agrawal2019abcd} from the repository \href{https://github.com/agrawalraj/active_learning}{https://github.com/agrawalraj/active\_learning} (No license included).
    \item DDS: We use the code from the official repository \href{https://github.com/sharpenb/Differentiable-DAG-Sampling}{https://github.com/sharpenb/Differentiable-DAG-Sampling} (No license included).
    \item BCD: We use the code from the official repository \href{https://github.com/ermongroup/BCD-Nets}{https://github.com/ermongroup/BCD-Nets} (No license included).
    \item DIBS: We use the code from the official repository \href{https://github.com/larslorch/dibs}{https://github.com/larslorch/dibs} (MIT license).
\end{itemize}
Additionally for the Syntren~\cite{van2006syntren} and Sachs Protein Cells~\cite{sachs2005causal} datasets, we use the data provided with repository \href{https://github.com/kurowasan/GraN-DAG}{https://github.com/kurowasan/GraN-DAG} (MIT license).

\section{Broader Impact Statement}

This work is concerned with understanding cause and effects from data, which has potential applications in empirical sciences, economics and machine perception. Understanding causal relationships can improve fairness in decision making, understand biases which might be present in the data and answering causal queries. As such, we envision this line of work to not have any significant negative impact.

\lstset{basicstyle=\ttfamily\scriptsize,keywordstyle=\ttfamily\scriptsize\bfseries}

\newcommand{\midrule}{\hline}

\renewcommand{\textfraction}{-0.1}
\renewcommand{\topfraction}{1.1}
\renewcommand{\bottomfraction}{1.1}
\addtolength{\textfloatsep}{-10.5pt}

\sloppy

\begin{document}

\title{Hybrid ASP-based multi-objective scheduling of semiconductor manufacturing processes\\(Extended version)} % \thanks{}}
\titlerunning{ASP-based multi-objective scheduling of semiconductor manufacturing processes}

\author{Mohammed~M.~S.~El-Kholany\inst{1,2}\orcidID{0000-0002-1088-2081} \and
Ramsha~Ali\inst{1}\orcidID{0000-0002-4794-6560} \and
Martin~Gebser\inst{1,3}\orcidID{0000-0002-8010-4752}}
\authorrunning{M. M. S. El-Kholany et al.}

\institute{University of Klagenfurt, Austria \and
Cairo University, Egypt \and
Graz University of Technology, Austria
\email{\{mohammed.el-kholany,ramsha.ali,martin.gebser\}@aau.at}}
%
\maketitle

\begin{abstract}
Modern semiconductor manufacturing involves intricate production processes consisting of hundreds of operations, which can take several months from lot release to completion. The high-tech machines used in these processes are diverse, operate on individual wafers, lots, or batches in multiple stages, and necessitate product-specific setups and specialized maintenance procedures. This situation is different from traditional job-shop scheduling scenarios, which have less complex production processes and machines, and mainly focus on solving highly combinatorial but abstract scheduling problems. In this work, we address the scheduling of realistic semiconductor manufacturing processes by modeling their specific requirements using hybrid Answer Set Programming with difference logic, incorporating flexible machine processing, setup, batching and maintenance operations. Unlike existing methods that schedule semiconductor manufacturing processes locally with greedy heuristics or by independently optimizing specific machine group allocations, we examine the potentials of large-scale scheduling subject to multiple optimization objectives.

\keywords{Hybrid Answer Set Programming \and Semiconductor manufacturing scheduling \and Difference logic \and Multi-objective optimization}
\end{abstract}

\section{Introduction}
% Figure environment removed

\section{Introduction}
Automatic 3D reconstruction of clothed humans using image inputs has gained increasing significance due to its potential applications in a wide array of AR/VR scenarios. High-fidelity reconstructions typically depend on sophisticated capture systems, which are developed with dense camera arrays~\cite{collet2015high,joo2015panoptic,joo2018total}, programmable light-stages~\cite{Vlasic2009, guo2019relightables}, and depth sensors~\cite{newcombe2011kinectfusion,DoubleFusion,BodyFusion,dou2016fusion4d,newcombe2015dynamicfusion}. However, stringent capture environments equipped with complex hardware pose significant challenges for consumer-level applications.


In this context, considerable research effort has been dedicated to developing methods that allow for more flexible capture configurations, such as utilizing a few RGB inputs. Among these works, learning implicit functions \cite{iccv2020PIFu, saito2020pifuhd, hong2021stereopifu} has proven effective in achieving highly detailed reconstructions by integrating the advancements of deep neural networks. These methods employ large multi-layer perceptrons (MLPs) to predict the occupancy probability or truncated signed distance function (TSDF) value of every queried 3D point based on its associated local feature, which is extracted from images. They can recover a continuous surface at arbitrary resolutions without topology restrictions.


However, in typical MLP-based implicit networks, the occupancy or TSDF value at each location is solved independently with planar image features, rendering them less capable of addressing challenging cases such as occlusions. Consequently, these methods suffer from generalization and robustness issues, particularly when tackling strong occlusions caused by large motion or multiple interacting humans. 
Some follow-up studies  \cite{zheng2021deepmulticap,zheng2021pamir,huang2020arch} utilize an extra geometric model, SMPL~\cite{Loper2015}, to improve robustness by introducing strong shape priors. 
Their success typically relies on the assumption of geometrical similarity \cite{huang2020arch} between the shape prior and target reconstruction, making them intractable for handling complex cases with loose clothes and sensitive to errors in SMPL model fitting.



%\ping{this paragraph sounds like `TSDF is better than MLP/SMPL, and we use TSDF to solve the problem'. But in Sec 3, we are telling a different story, saying `MLP needs a 3D convolutional encoder'. We need to make these two sections consistent.}\sicong{I think in this paragraph we claim that the TSDF}


%We opt for Trucated Signed Distance Funtion (TSDF) volumetric representations as they are naturally suitable for convolution operations, which have shown remarkable performance for learning hierarchical features on 2D visual perception tasks \cite{SunXLW19}. 
%Meanwhile, TSDF also describes the gradual geometry change around shape surface, which is not reflected by occupancy volume. 

We instead revisit the 3D volumetric representation and resort to 3D convolutional neural networks (CNNs) for feature learning, due to their impressive performance in feature learning and the ability to incorporate spatial context. However, volumetric methods and 3D convolution involve discretization, which might raise concerns regarding whether a discretized volume can preserve subtle geometric details as continuous representations learned in implicit functions. We investigate the relationship between volume resolution and quantization error on synthetic data by converting target mesh objects to TSDF volumes, as shown in Figure~\ref{fig:quantization_error}. We observe that the quantization errors are significantly reduced by increasing volume resolution and become nearly negligible when reaching a relatively high resolution (e.g., 512 or higher). In other words, achieving fine-detailed reconstruction is not supposed to be restricted by the use of volume representations as long as a proper volume resolution is utilized. Therefore, we present a method with high-resolution feature volumes, e.g., 256 and 512, while traditional volumetric methods \cite{varol18_bodynet,gilbert2018volumetric} are often limited to much lower resolutions, such as 32 or 128.



On the other hand, an increase in volume resolution may lead to a cubic growth of memory overhead \cite{8100085}. Reducing memory costs while guaranteeing the granularity of volumetric representations is necessary for pursuing high-quality reconstruction. Thus, we adopt a coarse-to-fine approach and cull away irrelevant voxels to build a sparse high-resolution feature volume. At the coarse level, the network computes an initial TSDF by applying a U-Net with sparse 3D CNN \cite{3DSemanticSegmentationWithSubmanifoldSparseConvNet} on the sparse feature volume, which is carved by a visual hull. Through our experiments, it turns out that more than 95\% of the volume grids are discarded by the visual hull culling, making the sparse 3D CNN efficient. At the fine level, the network focuses on a narrow band near the zero-level set of the initial TSDF and discretizes the narrow band with smaller voxels. By employing this narrow-band culling, we further shrink the sampling space, resulting in a relatively small range of grid numbers (usually 300K--500K in our experiments) even with a high volume resolution of 512. The remaining voxels in the narrow band are associated with features that fuse high-frequency information from the computed normal maps upon the low-frequency shape from the coarse level to compute the TSDF at high resolution. The final mesh is then extracted from the TSDF using the Marching-Cube algorithm ~\cite{Lorensen87marchingcubes}.
% Different from the u-net sturcture to preserve global topology context, we then apply a shallow 3dcnn to compute the final TSDF $D_{final}$ which contain more local geometry detail.




% \ping{this paragraph can be expanded. It is an important contribution and often ignored by other works. stress on the novel idea of regressing blending weights instead of colors}

In addition to geometry, high-quality mesh texture is also a crucial factor contributing to visual appearance. Directly computing a color field in 3D space, as in \cite{iccv2020PIFu}, struggles to capture high-frequency texture details, while the neural radiance field (NeRF) \cite{yu2020pixelnerf} or the DoubleField~\cite{shao2022doublefield} require expensive per-instance optimization and are often unstable for sparse input images. In contrast, we adopt an image-based rendering approach to compute a texture atlas map, which is efficient and widely supported in existing computer graphics tools. 
Specifically, we compute a blending weight at each 3D point on the mesh surface to determine its color as a weighted average of the colors at its image projections. The blending weights can be computed at a relatively coarse resolution, e.g., 512 volume resolution in our case, and leave texture details to the high-resolution images, such as 1K or 2K. Unlike previous methods that generate blurry texturing results under sparse input, our method generalizes well on both synthetic and real data with just a few input views. 
Figure~\ref{fig:teaser} shows two examples reconstructed by our method. Despite the challenging garment, pose, and occlusion, our method recovers faithful shape, normal, and texture on the right.

%with a wide variety of poses and clothing styles, and it is also adaptive to handle input image with arbitrary resolutions.
%\sicong{For this concern we claim that when the resolution of dicretized volume meets certain threshold (which is 256 in our experiment), the quantization error can be neglected.} 



In summary, the main contributions of this paper are as follows:
\begin{itemize}
\vspace{-0.1in}
  \item 
  We revisit the 3D volumetric representation and demonstrate that it can support clothed human reconstruction with equal or even better performance compared to implicit representation. 
  \item 
  We develop a memory and computation-efficient method for high-resolution volumetric reconstruction using sophisticated sparse 3D CNN, coarse-to-fine estimation, and voxel culling by visual hull and narrow bands. 
  \item 
  We introduce a novel method to compute a texture atlas map, which captures rich appearance details from high-resolution input images.
  \item 
  We achieve impressive results on standard benchmark datasets Twindom and MultiHuman, significantly reducing the point-2-surface (P2S) precision to approximately 0.2cm from just six input views, with more than $50\%$ error reduction compared to the state-of-the-art methods, including DoubleField~\cite{shao2022doublefield} and PIFuHD~\cite{saito2020pifuhd}.
\end{itemize}

\section{Literature review}\label{sec:review}
Semiconductor fab scheduling is a highly complex task %, mainly because of the typical features, such as complex 
due too sophisticated producting routes,
diverse machine characteristics, and
rapidly changing demands~\cite{ellis2004scheduling}. 
% To develop effective scheduling methods in wafer fabrication is thus challenging. The combined used of a simulation and an optimization algorithms provides an efficient solution to scheduling problems.
While lacking specific features of the semiconductor manufacturing process such as, e.g., re-entrant flow, batching, setup and maintenance operations, as well as varying processing times and sudden machine disruptions, the Flexible Job-Shop Scheduling Problem (FJSP) \cite{brusch90a,taillard93a} along with the optimization methods devised for it are related approaches.

% SMSP can be considered as flexible job shop problem (FJSP) due to the similarities in the structure and complexity of both problems. Multiple operations can be performed at on multiple machines. This flexibility in machine assignment is a characteristic feature of FJSP, where each operation can be performed on one of several available machines. Both SMSP and FJSP involve constraints related to machine availability, limited buffer space, and tooling requirements and complex objectives, such as minimizing cycle time, minimize tardiness, and maximizing machine utilization. All these features need to be considered when scheduling operations to avoid bottlenecks and ensure smooth production flow. The methodologies and algorithms developed for FJSP can be applied to SMSP, such as dispatching rules, metaheuristic optimization algorithms, simulation, and artificial intelligence, has been employed to address the complexity and challenges of FJSP.  

% FJSP is combinatorial optimization problem in which each operation can be processed by a set of identical machines. Several methods have been developed in the literature for solving flexible job shop scheduling problems. These methods can be broadly classified into exact and heuristic methods. 

% Exact methods attempt to find the optimal solution to the scheduling problem by examining all possible combinations of job sequences and machine assignments. Exact methods include mathematical programming techniques such as linear programming, integer programming, and mixed-integer programming. These methods have been used to solve smaller instances of the flexible job shop scheduling problem.

% Heuristic methods, on the other hand, aim to find a good solution to the problem in a reasonable amount of time, without examining all possible solutions. Heuristic methods are often faster and more practical than exact methods for solving larger instances of the flexible job shop scheduling problem.

A wide range of techniques have been proposed in the literature to tackle combinatorial optimization for FJSP solving.
Meta-heuristic algorithms incorporate local search methods,
such as Genetic Programming \cite{li2016effective,wang2001effective}, Tabu Search~\cite{li2016effective}, Simulated Annealing~\cite{wang2001effective}, Harmony Search~\cite{sahraeian2017new}, Particle Swarm Optimization~\cite{hassanzadeh2016two}, and Ant Colony Optimization~\cite{xing2010knowledge}.
% are also extensively applied to FJSP.
Exact solving methods are based on FJSP models in 
Mixed Integer Programming (MIP) \cite{ceylan2021coordinated,gran2015mixed,ham2021energy},
Constraint Programming (CP) \cite{da2019industrial,ham2021energy}, or 
ASP with difference logic \cite{el2022problem,janhunen2017clingo}.

Beyond FJSP, ASP~\cite{lifschitz19a} has been successfully used
to schedule
printing devices~\cite{balduccini2011industrial},
specialist teams~\cite{rigralmaliiile12a},
work shifts~\cite{abseher2016shift},
course timetables~\cite{bainkaokscsotawa18a},
medical treatments~\cite{dogagrmamopo21a},
and
aircraft routes~\cite{tassel2021multi}.
The hybrid framework of ASP with difference logic~\cite{janhunen2017clingo} particularly supports a compact representation
and reasoning with quantitative resources like time,
which has been exploited in domains such as
lab resource~\cite{francescutto2021solving},
train connection~\cite{abels2021train}, and
parallel machine~\cite{eiter2022answer} scheduling, as well as for FJSP solving
\cite{el2022problem,janhunen2017clingo}.
In this work, we extend our preliminary ASP with difference logic approach~\cite{ali2023flexible} to semiconductor fab scheduling with support for batching machines, partially flexible machine allocation strategies, and multi-objective optimization functionalities.

% A wide range of techniques have been proposed in literature to deal with FJSP. A mixed integer programming for FJSP is developed by \cite{gran2015mixed}. Another model with Mixed Integer Programming (MIP) was given by \cite{ham2021energy,ceylan2021coordinated}. Meta heuristics algorithms hybrid with local search are also extensively used for FJSP such as Genetic Programming \cite{li2016effective,wang2001effective}, Tabu Search \cite{li2016effective}, Simulated Annealing \cite{wang2001effective}, Harmony Search \cite{sahraeian2017new}, Particle Swarm \cite{hassanzadeh2016two} and Ant Colony Optimization \cite{xing2010knowledge}. Constraint Programming (CP) formulation for FJSP is presented by \cite{ham2021energy,da2019industrial}. Answer set programming (ASP) based approach for machine scheduling is presented by many studies such as \cite{balduccini2011industrial,eiter2022answer,el2022decomposition,francescutto2021solving}. 

% To improve efficiency of methods, researches have developed hybrid languages and and solvers such as ASP+CP has been used in industrial size applications \cite{balduccini2011industrial}. Similarly, MIP+CP is studied by \cite{ham2021energy}.

% Despite the similarities, SMSP is more complex then FJSP due to process complexity, with hundreds of operations and numerous workstations involved in production of a single semiconductor wafer. A re-entrant flow adds complexity to this as it requires managing job in a re-entrant loops. Additional feature include the batch processing where some operations are performed on batch tools, which consist of multiple processing chambers sharing a common wafer handling system. Semiconductor manufacturing equipments are highly sensitive and prone to failures, hence require regular maintenance. This leads to machine downtime and uncertainties in machine availability, which must be factored into the scheduling process. Therefore, scheduling algorithms need to be adapted and extended to address the specific challenges and constraints of SMSP, which is the aim of this paper.

\section{Background}\label{sec:smsp}

\vspace{-0.15cm}

\section{Strategy Templates}\label{sec:templates}

In this section, we introduce a formalization of player \Odd strategies in \Odd-fair parity games via \emph{strategy templates}.
% 
In contrast to player \Even, player \Odd winning strategies are no longer positional in \Odd-fair parity games, as illustrated by the following example. %that requires the same number of symbolic steps as the algorithm computing winning strategies for \Even in \enquote{normal} parity games.
% \vspace{-0.5em}
\begin{example}\label{ex:strategytemplates}
Consider the three different parity games depicted in Fig.~\ref{fig:Oddstrategies1}. %, three \Odd-fair parity games are depicted, with circles indicating \Ve and squares indicating \Vo. Edges in $E^\ell$ are shown by dashed lines. All nodes are labeled with their priorities.
   In all three games, \Odd has a winning strategy from all vertices, i.e., $\mathcal{W}_{Odd}=V$. %The red-colored edges indicate \Odd's strategy: if \Odd takes the red edges alternatingly from the source nodes, it wins from all nodes. 
  However, in order to win, the vertex $3$ has to be seen infinitely often in game (a) and (b), which forces \Odd to use its live edge\textbackslash s infinitely often. This prevents the existence of a positional strategy for \Odd in games (a) and (b): In (a) it needs to somehow alternate between (it's only) live edge to $4$ and a \enquote{normal} edge to $7$ (both indicated in red) in order to win, and in (b) it needs to somehow alternate between all its live edges (also indicated in red). In the game (c), \Odd can win by 'escaping' its live vertex $3$ to a \enquote{normal} vertex $5$, and thereby has a positional strategy. % (again indicated in red).
   
  Now consider the subgraph of each game formed by all colored edges (red and blue), which include the strategy choices from \Vo and \emph{all} outgoing edges from \Ve. As we have seen that \Odd needs to play all red edges repeatably, this subgraph represents the paths that \emph{can} be seen in the game depending on the \Even strategy. Hence, a node $v\in\Vl\subseteq\Vo$ can be seen infinitely often in a play (compliant with \Odd's strategy), if it lies on a cycle in this subgraph. We observe that, in games (a) and (b), node $3$ lies on cycles in this subgraph, whereas in game (c), it does not. 
  We further see that whenever a vertex  $v\in\Vl$ lies on a cycle, \Odd needs to take all its outgoing live edges (as for vertex $3$ in example (b)) and possibly one more edge (as for vertex $3$ in example (a)), for all other vertices in $\Vo$ a positional strategy suffices (as for vertex $5$ in all examples, and for vertex $3$ in example (c)). This shows that \Odd strategies are intuitively still \enquote{almost positional}.
% 
\end{example}

% Figure environment removed


\vspace*{-0.2cm}

The intuitions conveyed by Ex.~\ref{ex:strategytemplates} are formalized by the following definitions. % for \Odd strategy templates.


\begin{definition}[\Odd Strategy Template]\label{def:Oddstrategytemplate}
 Given an \Odd-fair parity game $\mathcal{G}^\ell = \ltup{\mathcal{G}, E^\ell}$ with \newline $\mathcal{G} = \langle V, \Ve, \Vo, E, \chi\rangle$, an \Odd \emph{strategy template} $\mathcal{S}$ over $\mathcal{G}^\ell$ is a subgraph of $\mathcal{G}$ given as follows: $\mathcal{S}:=\tup{V',E'}$ where $V'\subseteq V$ and $E'\subseteq E \cap (V' \times V')$ such that the following hold,
\begin{compactitem}\label{item:Oddstrtemprules}
 \item if $v \in \Vo \cap V'$ does not lie on a cycle in $(V',E')$, then $|E'(v)|=1$,
 \item if $v \in \Vo \cap V'$ lies on a cycle in $(V',E')$ then $E^\ell(v) \subseteq E'(v)$ and  $1\leq |E'(v)|\leq |E^\ell(v)| + 1$,
 \item if $v \in \Ve \cap V'$, then  $E'(v) = E(v)$.
\end{compactitem}
\end{definition}
%
\begin{definition}\label{def:compliantstrat}
 Let  $\mathcal{G}^\ell = \ltup{\mathcal{G}, E^\ell}$ be an \Odd-fair parity game with \Odd strategy template $\mathcal{S}=\tup{V',E'}$, and $V'_\Odd := V' \cap V_\Odd$. Then an
\Odd strategy $\rho$ is said to be \textbf{compliant} with $\mathcal{S}$ if  
it is a winning strategy in the game $\ltup{\gamegraph,\alpha'}$ where $\gamegraph= \tup{V,\Ve,\Vo,E}$ and 
\begin{subequations}
 \begin{align}
 \alpha':= &\textstyle\bigwedge_{v\in\Vo'}(\,\square\, (\,v \implies \bigvee_{(v,w)\in E'} \bigcirc\, w\,))\,\label{equ:alpha:a}\\
 & \textstyle\wedge \bigwedge_{v\in\Vo'} (\,\square \,\diamondsuit\, v \implies \bigwedge_{(v,w)\in E'}\square\, \diamondsuit\, (\,v \wedge \bigcirc \,w\,)).\label{equ:alpha:b}
\end{align}
\end{subequations}
\end{definition}

Intuitively, for all \Odd vertices in $\mathcal{S}$, the strategy $\rho$ compliant with $\mathcal{S}$ takes only their outgoing edges in $\mathcal{S}$ \eqref{equ:alpha:a}, and if a play visits an \Odd node $v$ infinitely often, then $\rho$ takes each of $v$'s outgoing edges in $\mathcal{S}$ infinitely often \eqref{equ:alpha:b}.
% 
For an \Odd strategy template $\mathcal{S}$, if $v \in V'_\Odd$ lies on a cycle in $\mathcal{S}$, then by Def. \ref{def:Oddstrategytemplate}, $\mathcal{S}$ contains all live outgoing edges of $v$. By \eqref{equ:alpha:b} any \Odd strategy $\rho$ compliant with $\mathcal{S}$ satisfies the fairness condition in \eqref{eq:fairness-ltl} for $v$. 
On the other hand, if $v \in V'_\Odd$ does not lie on a cycle in $\mathcal{S}$, then by \eqref{equ:alpha:a} any such $\rho$ sees $v$ at most once. Thus $\rho$ trivially satisfies \eqref{eq:fairness-ltl} for $v$. 
This observation is stated in the following proposition.


\begin{proposition}
 Given the premisses of Def.~\ref{def:compliantstrat} let $\pi$ be a play starting from a node in $V'$ that complies with $\rho$. Then $\pi \models \alpha$ where $\alpha$ if the LTL formula in~\eqref{eq:fairness-ltl}.%\vspace{-2mm}
\end{proposition}

Next, we define \Even strategy templates. Each \Even strategy template encodes a unique \Even positional strategy, which is known to exist in \Odd-fair parity games \cite{banerjee2022fast}, due to the lack of fair edges defined on \Even vertices. %, \Even strategy templates are very simple\footnote{In fact, \Even strategy templates simply encode a positional strategy and are only re-defined to make further arguments more symmetric for both players.}.
\begin{definition}\label{def:Evenstrategytemplate}
    Given an \Odd-fair parity game $\mathcal{G}^\ell = \ltup{\mathcal{G}, E^\ell}$ with \newline $\mathcal{G} = \langle V, \Ve, \Vo, E, \chi\rangle$, an \Even \emph{strategy template} $\mathcal{S}$ over $\mathcal{G}^\ell$ is a subgraph of $\mathcal{G}$ given as $\mathcal{S}:=\tup{V', E'}$ where $V'\subseteq V$ and $E'\subseteq E \cap (V' \times V')$ such that,    \begin{compactitem}\label{item:Evenstrtemprules}
     \item if $v \in \Ve \cap V'$, then $|E'(v)|=1$,
     \item if $v \in \Vo \cap V'$, then  $E'(v) = E(v)$.
    \end{compactitem}
\end{definition}

\vspace*{-0.1cm}

An \Even strategy $\rho$ is compliant with the \Even strategy template $\mathcal{S} = \tup{V', E'}$ if for all $v \in V'_\Even$, $\rho(v) = E'(v)$. In other words, $\rho$ is the positional strategy defined by $\mathcal{S}$.

Let $\rho$ be an \Odd (\Even) strategy, compliant with the \Odd (\Even) strategy template $\mathcal{S}$ and let $\pi$ be a play compliant with $\rho$. Then we call $\pi$ a play \emph{compliant with $\mathcal{S}$}.

\vspace*{-0.1cm}

\begin{definition}
An \Odd (\Even) strategy template $\mathcal{S}=\ltup{V', E'}$ is \emph{winning} in the \Odd-fair parity game $\mathcal{G}^\ell$ if all \Odd (\Even) strategies $\rho$ compliant with $\mathcal{S}$ are winning for player \Odd (\Even) in $\mathcal{G}^\ell$ from $V'$. A winning \Odd (\Even) strategy template $\mathcal{S}$ is called \emph{maximal} if $V'=\Wo$ ($\We$).%\vspace{-2mm}
\end{definition}

\vspace*{-0.2cm}
We note that maximal winning \Odd (\Even) strategy templates $\mathcal{S}$ immediately imply that for every vertex $v\in \Wo$ ($\We$) there exists a winning strategy for player \Odd (\Even) from $v$ that is compliant with $\mathcal{S}$.
% 
The existence of maximal winning \Even strategy templates follows from the existence of positional \Even strategies \cite{banerjee2022fast}. 
% 
The first main contribution of this paper is a constructive proof showing the existence of maximal winning \Odd strategy templates given in the next section. 
This result is then used in Sec.~\ref{sec:zielonka} to prove the correctness of \Odd-fair Zielonka's algorithm, which is introduced there.





\section{Hybrid ASP encoding}\label{sec:asp}
In the following, we present our hybrid ASP with difference logic encoding
of SMSP supplying multi-objective optimization functionalities.
We start by describing the fact format of problem instances
(Section~\ref{subsec:instance}), followed by static preallocation strategies to
limit the number of assignable machines for each operation
(Section~\ref{subsec:partial}), then
the main encoding part to generate schedules incorporating batches, setup and maintenance operations (Section~\ref{subsec:schedule}), 
and finally optimization by multi-shot ASP solving
(Section~\ref{subsec:optimization}).

\subsection{Problem instance}\label{subsec:instance}

% The input predicates used to specify lots, machines, setup and operations are utilized in Listing~\ref{prg:facts}, which presents a condensed portion of the SMT2020 HV/LM setting. The input facts are explained below:
The example SMSP instance for which an (optimal) schedule is shown in
Figure~\ref{fig:schedule} is represented by the facts in Listing~\ref{prg:facts}. The structure of the used predicates is as follows:
%
\begin{description}[font=\normalfont\ttfamily\footnotesize]
\item[route($p$,$i$,$g$,$t$,$m$,$n$,$s$).]
The $i$-th operation for lots of product $p$ takes the processing time $t$ on a machine in tool group $g$ equipped with the setup $s$, where $m$ and $n$ provide the minimum and maximum batch size.
E.g., the fact in line~\ref{prg:facts:prd:begin} of Listing~\ref{prg:facts} states that the first operation for lots of product $1$ needs to be performed by a machine in the \emph{diffusion\_fe\_120} tool group, taking the processing time~$20$ for batches of (preferably) at least $2$ and at most $4$ lots with an arbitrary setup in view of $s=0$.
%
\item[setup($g$,$s$,$t$,$m$).]
Changing to setup $s\neq 0$ takes time $t$ on machines of the tool group $g$,
and at least $m$ operations should be processed in the setup $s$ before performing
another setup change.
The facts in lines~\ref{prg:facts:set:begin} and~\ref{prg:facts:set:mid} of Listing~\ref{prg:facts} express that the setups \emph{su128\_1} and \emph{su128\_2} need $20$ or $18$ time units, respectively, to be equipped on machines in the \emph{implant\_128} tool group,
where each of them ought to be maintained for $4$ production operations at minimum before changing.
Unlike that, the \emph{su450\_3} setup, taking $22$ units for equipping \emph{lithotrack\_fe\_95} machines with it, can be changed freely, as declared by the fact in line~\ref{prg:facts:set:end}.
%
\item[pm($g$,$l$,$e$,$m$,$n$,$t$).]
Machines in the tool group $g$ need to undergo a periodic maintenance operation labeled $l$,
whose type $e$ is either \lstinline{lots} or \lstinline{time}, the parameters $m$ and $n$
specify a minimum and maximum amount of lots or processing time, respectively, after which the
maintenance operation taking $t$ time units needs to repeated.
The facts in lines \ref{prg:facts:pm:begin}-\ref{prg:facts:pm:end} of Listing~\ref{prg:facts}
specify one maintenance operation per tool group, where \emph{implant\_128\_mn} is based on processed lots, while \emph{lithotrack\_fe\_95\_wk} and \emph{diffusion\_fe\_120\_mn}
need to repeated according to accumulated processing times.
%
\item[tool($g$,$l$).]
A machine labeled $l$ belongs to the tool group~$g$,
where the facts in line~\ref{prg:facts:tool:begin} of Listing~\ref{prg:facts}
introduce one machine per tool group.
%
\item[lot($l$,$p$).]
A lot labeled~$l$ of product~$p$ needs to produced,
and two lots of the (single) product~\lstinline{1} are
declared by the facts in line~\ref{prg:facts:wip:begin} of Listing~\ref{prg:facts}.
\end{description}%
%
\lstinputlisting[float=t,label=prg:facts,caption={Facts for an SMSP instance with two lots and three tool groups with one machine each},linerange={2-14}]{listing/facts.lp}
%
% \begin{itemize}
% 	\item \emph{route($p,r,g,t,u,min,max,s$).}
% 	This defines the route for a given product \emph{p} and its corresponding step number \emph{r}. The step can be processed on a machine group \emph{g} and takes a processing time of \emph{t} units. The processing unit representation is indicated by \emph{u}, which can be either "batch" or "wafer". In case of batch, the minimum and maximum batch sizes are specified by \emph{min} and \emph{max}, respectively, $0$ otherwise. Additionally, the setup required for the step is denoted by \emph{s}.
	
% 	\item \emph{setup($g,s,t,min\_r$).}
% 	This defines the setup of a given tool group \emph{g} with setup name \emph{s}, which takes certain amount of time \emph{t} and minimum runs \emph{min\_r}. The parameter minimum runs indicates that the machine with a particular setup should perform a minimum of \emph{min\_r} operations before switching to a different setup. This parameter plays a crucial role in the optimization criteria. 
	
% 	\item \emph{pm($g,a,x,y,z,w$).}
% 	This defines the maintenance procedures of type ${x \in \{lots, time\}}$ with duration \emph{w} applied to machines in the tool group \emph{g} after processing between \emph{y} and \emph{z} operations or accumulating as much processing time, respectively.
	
% 	\item \emph{tool($g,m$).}
% 	This defines a machine \emph{m} belongs to tool group \emph{g}.
	
% 	\item \emph{lot($l,p$).}
% 	This defines the scheduling of lot \emph{l} of product \emph{p}. The lot could range from $ 1 \dots n$.
% \end{itemize}

% % \input{schedule_example.tex}

% An optimal schedule for example problem instance shown in Listing~\ref{prg:facts} is shown in Figure~\ref{fig:schedule}.

\subsection{Partially flexible machine assignment}\label{subsec:partial}

\lstinputlisting[float=t,label=prg:groups,caption={Encoding part for partitioning tool groups into subgroups and preallocation by setups},linerange={1-45,65-69}]{listing/encoding_labeled.lp}
%
Experiments with our prototypical SMSP encoding \cite{ali2023flexible} showed that fixing the machine assignment of operations upfront sacrifices optimality, while a fully flexible assignment leads to plenty ground rules slowing down the optimization when a tool group contains many machines.
To enable trade-offs between the fixed and fully flexible machine allocation strategies, the novel encoding part in Listing~\ref{prg:groups} introduces a constant
\lstinline{sub_size} that allows for limiting the number of assignable machines per operation.
That is, when \lstinline{sub_size} is \lstinline{0},
the machine assignment remains fully flexible, gets fixed if the value is~\lstinline{1},
or is limited to some \emph{subgroup} of a tool group with at most \lstinline{sub_size} many machines for values greater than one.
In the latter case, the rule in lines \ref{prg:encoding:10}-\ref{prg:encoding:12}
partitions \lstinline{N} machines of a tool group \lstinline{G} into
$\lceil\text{\lstinline{N}}\div\text{\lstinline{sub_size}}\rceil$ many subgroups,
% with
%$\left(\text{\lstinline{sub_size}} - (\text{\lstinline{N}} \mathbin{\scriptstyle{\%}} \text{\lstinline{sub_size}})\right)\mathbin{\scriptstyle{\%}}\text{\lstinline{sub_size}}$
each gathering \lstinline{sub_size} or $\text{\lstinline{sub_size}}-1$ of the machines in \lstinline{G} when $\text{\lstinline{N}}\geq \text{\lstinline{sub_size}}$.
For example, we derive the atoms
\lstinline{subgroup(implant_128,1,1,3)},
\lstinline{subgroup(implant_128,2,4,5)}, and
\lstinline{subgroup(implant_128,}\linebreak[1]\lstinline{3,}\linebreak[1]\lstinline{6,7)}
by the rule in lines~\ref{prg:encoding:13}-\ref{prg:encoding:14},
giving the subgroups
$\{\text{\lstinline{1}},\text{\lstinline{2}},\text{\lstinline{3}}\}$,
$\{\text{\lstinline{4}},\text{\lstinline{5}}\}$, and
$\{\text{\lstinline{6}},\text{\lstinline{7}}\}$
when
seven machines in the \emph{implant\_128} tool group are partitioned for
the \lstinline{sub_size} value~\lstinline{3}.

The rules in lines \ref{prg:encoding:18}-\ref{prg:encoding:24} determine the subgroup
to which an operation is allocated, based on a lexicographical index
for operations to be processed by machines in the same tool group.
This allocation can be configured by the constant \lstinline{lot_step}: if its value is \lstinline{0}, all operations of a lot are mapped to a common index, or to successive indexes in case of value~\lstinline{1}.
The rationale for these two strategies is that operations performed on the same lot
succeed one another and will thus never compete for a machine.
On the other hand, the operations may require different setups so that changes are needed when
reusing the same machine.
In fact, the latter indexing strategy is likely to map operations of a lot to separate subgroups,
as the rule in lines \ref{prg:encoding:23}-\ref{prg:encoding:24} allocates them in a round robin fashion.

As subordinate machine allocation criterion within each subgroup, the setups of operations can be inspected by means of the rules in lines \ref{prg:encoding:30}-\ref{prg:encoding:50} when the constant \lstinline{by_setup} is set to a value other than~\lstinline{0}.
The idea of the rules in lines \ref{prg:encoding:30}-\ref{prg:encoding:38}
is to order setups by the sum of processing times for their operations, where setups requiring more processing time come first.
Then the rules in lines \ref{prg:encoding:40}-\ref{prg:encoding:50} follow this order to map setups and the respective operations to specific machines, always picking the machine with the least load so far for the next setup to allocate.
The rules to determine the least loaded machine for a setup are omitted in Listing~\ref{prg:groups} to save space, and our full encoding is available online.%
\footnote{\url{https://github.com/prosysscience/FJSP-SMT2020}\label{foo:online}}
For example, if two machines of the \emph{implant\_128} tool group belong to the same subgroup for
the lots specified by the facts in Listing~\ref{prg:facts}, the allocation by setups yields the atoms
\lstinline{assignable((1,1,3,8,}\linebreak[1]\lstinline{su128_1),}\linebreak[1]\lstinline{implant_128,}\linebreak[1]\lstinline{1)},
\lstinline{assignable((2,1,3,8,}\linebreak[1]\lstinline{su128_1),}\linebreak[1]\lstinline{implant_128,}\linebreak[1]\lstinline{1)},
\lstinline{assignable((1,1,5,7,}\linebreak[1]\lstinline{su128_2),}\linebreak[1]\lstinline{implant_128,}\linebreak[1]\lstinline{2)}, and
\lstinline{assignable((2,1,5,7,}\linebreak[1]\lstinline{su128_2),}\linebreak[1]\lstinline{implant_128,}\linebreak[1]\lstinline{2)},
thus mapping the third operation in the route of both lots, requiring the setup \emph{su128\_1}, to the first 
and the fifth operation for both with the setup \emph{su128\_2} to the second machine of the subgroup.

\subsection{Schedule generation}\label{subsec:schedule}

While the previous encoding part specifies preallocation strategies to
statically limit the machines to which each operation may be assigned,
Listing~\ref{prg:assign} describes the actual, combinatorial scheduling task,
including the machine assignment, setup and maintenance operations, as well as
the aggregation of \emph{batches}.
The latter feature was not yet incorporated in our prototypical SMSP encoding
\cite{ali2023flexible} and is newly introduced by the rules in lines
\ref{prg:encoding:54}-\ref{prg:encoding:64}.
To begin with, (ordered) pairs of operations processed by machines with a
maximum batch size beyond one are determined in lines
\ref{prg:encoding:54}-\ref{prg:encoding:58}.
Then, batches are generated by applying the choice rule in line~\ref{prg:encoding:60}, which represents batch processing of the first operation
for both lots given by the facts in Listing~\ref{prg:facts} in terms of the derivable atom
\lstinline{batch(diffusion_fe_120,(1,1,1,20,0),(2,1,1,20,0))}.
That is, a batch is identified by the operation on its lexicographically smallest lot, to which lots with greater identifiers are linked via the \lstinline{batch/3} predicate.
Any such linked lots are indicated by \lstinline{batched/1},
and \lstinline{batched/2} provides a symmetric version of \lstinline{batch/3},
where both of the \lstinline{batched} predicates are derived by the rule in
line~\ref{prg:encoding:63}.
The integrity constraint in line~\ref{prg:encoding:64} makes sure that batches
partition the lots of a product, as it rules out that the lot identifying a batch is itself linked to another (lexicographically smaller) lot.
This unambiguous batch representation is exploited by the integrity constraint in line~\ref{prg:encoding:61}, where it suffices to count the linked lots to assert that the maximum batch size for an operation is not exceeded.
%
\lstinputlisting[float=t,label=prg:assign,caption={Encoding part to assign batches, machines, as well as setup and maintenance operations},linerange={71-109,120-129,150-156},firstnumber=52]{listing/encoding_labeled.lp}

The choice rule in line~\ref{prg:encoding:68} continues with the \emph{machine assignment} by selecting exactly one machine, among those determined by a preallocation strategy from the previous subsection, for processing an operation.
Operation pairs assigned to the same machine are brought into an ordered
representation in terms of the \lstinline{step_assign/3} predicate via the rules in lines
\ref{prg:encoding:70}-\ref{prg:encoding:71}.
These pairs are filtered in lines \ref{prg:encoding:72}-\ref{prg:encoding:73} to
indicate the operations on different lots by \lstinline{lots_assign/3}.
Only for the latter an execution order needs to be guessed by applying the rules
in lines \ref{prg:encoding:77}-\ref{prg:encoding:78},
provided that the operations do not belong to the same batch,
which is checked in line~\ref{prg:encoding:74}.
However, the operations in a batch must share a common machine, as asserted by the integrity constraint in line~\ref{prg:encoding:75}.

The execution order of operations sharing a machine must be inspected further to
allocate required \emph{setup and maintenance} operations.
As several kinds of periodic maintenance may need to be applied to machines of the same tool group and their durations add up when they are performed in sequence,
the rules in lines \ref{prg:encoding:82}-\ref{prg:encoding:84} associate
maintenance operations with (positive) indexes in decreasing order of their
durations, with the additional index \lstinline{0} used for operation setups other
than (don't care) setup~$0$.
E.g., the facts in Listing~\ref{prg:facts} yield the atoms
\lstinline{main_setup(implant_128,implant_128_mn,1,13)},
\lstinline{main_setup(implant_128,su128_1,0,20)}, and
\lstinline{main_setup(implant_128,su128_2,0,18)} in view of the
periodic \emph{mn} maintenance along with the \emph{su128\_1} and
\emph{su128\_2} setups of operations processed by machines in the
\emph{implant\_128} tool group.
For tracking the exact execution order of operations on a machine, also if they
involve the same lot,
the \lstinline{step_order/3} predicate determined by 
the rules in lines \ref{prg:encoding:86}-\ref{prg:encoding:88} augments the
guessed predicate \lstinline{lots_order/3} with atoms reflecting the production route of a lot revisiting the same machine.
While we omit the details to save space,
let us mention that the necessity of a setup change before performing an operation
is a consequence of the execution sequence on a machine, i.e., the rule in lines
\ref{prg:encoding:92}-\ref{prg:encoding:93} derives an atom of the
\lstinline{equip/3} predicate whenever the setup required for an operation is not
already in place.
Unlike that, maintenance procedures are subject to a range, either in terms of
processed lots or accumulated processing time, after which they have to be repeated.
Hence, the rule in line~\ref{prg:encoding:95} introduces the choice to perform a
specific maintenance before the next production operation, the integrity constraint in line~\ref{prg:encoding:96} distributes such a choice over all lots in a batch, and (auxiliary) maintenances before the first production operation on a machine are asserted in lines \ref{prg:encoding:97}-\ref{prg:encoding:98}.
The resulting maintenance and setup times needed before the next
production operation can be processed are then added up by the rules in
lines \ref{prg:encoding:116}-107, where additional rules and constraints ensuring the compliance of maintenance procedures to the specified repetition ranges are part of our full encoding.%
\footref{foo:online}
For example, the \emph{mn} maintenance performed before the fifth operation for % in the route of 
lot~$1$ in Figure~\ref{fig:schedule} %, identified by the tuple \lstinline{(1,1,5,7,su128_2)},
is expressed by the atoms
\lstinline{delay((1,1,5,7,su128_2),implant_128,1,13)} and
\lstinline{delay((1,1,5,7,su128_2),implant_128,0,13)},
the latter providing \lstinline{13} as the sum of
all maintenance and setup times required before the production operation can be processed.

\subsection{Multi-objective optimization}\label{subsec:optimization}

Our multi-objective optimization approach combines minimization at the level of 
difference logic variable values, as already used in \cite{ali2023flexible,el2022problem},
with native ASP optimization capacities, as applied in \cite{abels2021train,eiter2022answer,francescutto2021solving} w.r.t.\ the satisfaction of difference logic constraints,
by means of multi-shot solving functionalities~\cite{gekakasc17a}.
To this end, the rules in lines \ref{prg:encoding:125}-\ref{prg:encoding:138} of Listing~\ref{prg:optimize} assert difference logic constraints on the completion times of operations,
beginning with processing times of the first operations in production routes
(lines \ref{prg:encoding:125}-\ref{prg:encoding:126}) or processing plus setup times 
(lines \ref{prg:encoding:127}-\ref{prg:encoding:128}) for all operations to which the latter apply.
These lower bounds are propagated along the production route of each lot
(lines \ref{prg:encoding:131}-\ref{prg:encoding:132}) and
the processing order of operations 
on machines
(lines \ref{prg:encoding:133}),
where the times required for maintenance and setup are incorporated in addition
(lines \ref{prg:encoding:134}-\ref{prg:encoding:135}).
Notably, batches are handled by synchronizing the completion time between the
operations on involved lots in line~\ref{prg:encoding:130},
so that the predecessor operation (if any) finishing latest among all lots in the batch is decisive for
the entire batch.
The rule in lines
\ref{prg:encoding:137}-\ref{prg:encoding:138}
asserts the completion time of the last operation in each lot's production route as a
lower bound on the difference logic variable \lstinline{makespan}, thus enabling plain
\emph{makespan minimization} by supplying \lstinline{--minimize-variable=makespan} as an option to \clingodl.
%
\lstinputlisting[float=t,label=prg:optimize,caption={Encoding part for determining lot completion times and multi-objective optimization},linerange={158-191},firstnumber=109]{listing/encoding_labeled.lp}

However, to incorporate the minimization of \emph{setup and batch violations} as additional optimization criteria beyond makespan,
we utilize a custom control script on top of the Python interface of \clingodl.
Its first stage concerns makespan minimization,
where the \lstinline{opt(b)} subprogram in lines
\ref{prg:encoding:142}-\ref{prg:encoding:145} of Listing~\ref{prg:optimize}
is instantiated with the value $t-1$ for the parameter \lstinline{b} and then solved with the external atom \lstinline{bound(}$t-1$\lstinline{)} set to true
whenever an answer set such that \lstinline{makespan}${}=t$ has been found.
This makes sure that each answer set provides a schedule with strictly shorter
makespan until an unsatisfiable solving attempt yields that the makespan~$t$ of the last schedule is optimal.
In the latter case, the subprogram \lstinline{weak(b)} in lines
\ref{prg:encoding:147}-\ref{prg:encoding:156} gets instantiated with the 
value $t$ for \lstinline{b} (and possibly also \lstinline{opt(b)} if the value~$t$ has not been supplied for \lstinline{b} before),
which fixes the makespan of any subsequently found schedule to the optimum~$t$.
With the weak constraints in lines
\ref{prg:encoding:152}-\ref{prg:encoding:156} as well as the rule in
line~\ref{prg:encoding:150} at hand for indicating operations whose setup is
reinstalled after some temporary change,
the second stage consists of native ASP optimization for minimizing setup and batch violations.
Here we take setup violations, where a setup gets changed before performing the intended minimum number of production operations using it, as strictly more
significant (optimization level \lstinline{@2}) than violations of the minimum
batch size (optimization level \lstinline{@1}), considering that equipping a machine with a setup takes extra time and effort.
For example, the schedule in Figure~\ref{fig:schedule} involves one
setup violation due to changing from the setup \emph{su128\_1} to \emph{su128\_2}
before performing the intended minimum number of four operations with this setup
on the \emph{implant\_128} machine.
Since avoiding the setup violation would require a second machine in the \emph{implant\_128} tool group, the schedule is nevertheless optimal.

% % Figure environment removed%

% \subsection{Sub-group strategy}
% \lstinputlisting[float=t,label=prg:subgroup,caption={Sub-group strategy},linerange={1-12}]{listing/subgroup.lp}

% An SMSP model with fixed and flexible assignment is presented in \cite{ali2023flexible}. We extend our model by adopting different assignment strategies while incorporating the batch tool management. We aim to optimize weak constraints additionally to the main optimization.

% Semiconductor manufacturing involves numerous machines within tool groups, which can make scheduling a complex task.
% To address this challenge, this paper presents a novel, sub-group strategy that divides machines in each tool group into variable-sized subgroups. By adjusting the parameter $size$, the number of subgroups can be controlled, allowing for improved manageability and adaptability in the scheduling process. By breaking down large tool groups into smaller sub-groups, the scheduling problem becomes more manageable. This reduces the complexity of the scheduling process and allows for easier implementation of scheduling algorithms and heuristics.

% The first section of base subprogram is presented in Listing~\ref{prg:subgroup} that describes the rules for sub-group strategy developed using own heuristics.
% The rule in line~\ref{sub} states that if a tool group $G$ exists, then it has a subgroup of size $1$, with index $1$.
% The rule from line~\ref{sub2:begin}--\ref{sub2:end} recursively defines the subgroups of tool group $G$, starting with a subgroup of size $M$ and index $I$. It uses the previously defined subgroups of smaller size $(M-1)$ to define the current subgroup, and ensures that the subgroup size is not evenly divisible by the $sub{\_}size{\_}grp$ variable.
% The rule from line~\ref{sub3:begin}--\ref{sub3:end} is similar to the previous one, but it ensures that the subgroup size is evenly divisible by the $sub{\_}size{\_}grp$ variable.



% \subsection{Assignment of steps to sub-group}
% \lstinputlisting[float=t,label=assign,caption={Assigning steps to subgroups},linerange={1-14}]{listing/optosubgroup.lp}

% We now propose an approach to assign operations to the subgroups to further enhance the scheduling process. 
% Here we proposed two different assignment approaches, job-based and step-based, for allocating operations to subgroups.  
% The effectiveness of the job-based and step-based assignment methods is evaluated through experiments with different instances. This approach is designed to optimize resource allocation. 

% In the job-based assignment approach, the primary focus is on assigning entire jobs to appropriate subgroups within a tool group. This method considers the overall requirements of a job, including its operation sequence, processing times, and precedence constraints. The job-based approach takes a global view of all jobs, ensuring that all steps that belongs to a job are considered when making assignment decisions.  

% In the step-based operation assignment approach, individual operations are assigned to subgroups within a tool group. This approach takes a local view of steps, considering only the current step requirements when making assignment decisions.

% The assigning of steps to sub group makes the problem partially flexible which can be determined from rules in Listing~\ref{assign}. 
% The rule from line~\ref{step:begin}--\ref{step:end} calculates the index $I$ for each lot $L$, product $P$, step $S$, tool group $G$, and processing time $Pro\_t$. It is calculated by counting the number of tuples (L, P, S) that belong to same tool group and meet the specified conditions such that $(R1, S1 * idx, Pro\_t1 * idx, P1, L1) < (R, S * idx, Pro\_t * idx, P, L)$. In this rule, the $step{\_}pro, step{\_}prio, and lot{\_}release$ predicates represent the process step information, step priority, and lot release time, respectively. The variable $idx$ is the controller for job-based and step-based approach with $0$ and $1$ values, respectively. 
% The rule from line~\ref{asg0:begin}--\ref{asg0:end} calculates the subgroup $D$ in tool group $G$, to which a step with attributes $(L, P, S, Prio)$ should be assignment based on index $I$. The division operator $"\textbackslash"$ is used to obtain the integer quotient of $I$ divided by $N$, which gives the subgroup index, and one is added to adjust the sub group index to start from 1 instead of 0. 



% \subsection{Setup-strategy within sub-group}
% \section{Background and Problem Statement}
\label{sec:setup}
We consider the problem of an agent interacting with an SCM for $T$ rounds in order to maximize the value of a reward variable. We start by introducing SCMs, the soft intervention model used in this work, and then define the adversarial sequential decision-making problem we study. In the following, we denote with $[m]$ the set of integers $\{0, \dots, m\}$. \looseness-1

\paragraph{Structural Causal Models}
Our SCM is described by a tuple $\langle \G,  Y, \bX, \fs, \snoiserv \rangle$ of the following elements: $\G$ is a \emph{known} DAG; $Y$ is the reward variable; $\bX = {\{X_i\}_{i=0}^{m-1}}$ is a set of observed scalar random variables; the set $\fs = \{\fofi\}_{i=0}^m$ defines the \emph{unknown} functional relations between these variables; and $\snoiserv = \{\snoiserv_i \}_{i=0}^{m}$ is a set of independent noise variables with zero-mean and known distribution. % \looseness-1
 We use the notation $Y$ and $X_m$ interchangeably and assume the elements of $\bX$ are topologically ordered, i.e., $X_0$ is a root and $X_m$ is a leaf.  We denote with $\pa_i \subset \{0, \dots, m\}$ the indices of the parents of the $i$th node, and use the notation $\bZi = \{ X_j\}_{j \in \pa_i}$ for the parents this node. We sometimes use $X_i$ to refer to both the $i$th node and the $i$th random variable. \looseness-1\looseness-1

Each $X_i$ is generated according to the function $\fofi: \calZ_i \rightarrow \calX_i$, taking the parent nodes $\bZi$ of $X_i$ as input: $\si =\fofi(\zi) + \noisei$, where lowercase denotes a realization of the corresponding random variable. The reward is a scalar $x_m \in [0,1]$ while observation $X_i$ is defined over a compact set $\si \in \calX_i \subset \R$, and its parents are defined over $\calZ_i = \prod_{j \in pa_i} \calX_j$ for $i\in [m-1]$.\footnote{Here we consider scalar observations for ease of presentation, but we note that the methodology and analysis can be easily extended to vector observations as in \citet{sussex2022model}}  \looseness-1

\paragraph{Interventions}

\looseness -1 In our setup, an agent and an adversary both perform \emph{interventions} on the SCM~\footnote{Our framework allows for there to be potentially multiple adversaries, but since we consider everything from a single player's perspective, it is sufficient to combine all the other agents into a single adversary.}. 
We consider a soft intervention model \citep{eberhardt2007interventions} where interventions are parameterized by controllable \emph{action variables}. A simple example of a soft intervention is a shift intervention, where actions affect their outputs additively \citep{zhang2021matching}.

First, consider the agent and its action variables $\bm a = {\{ \ai\}_{i=0}^{m}}$. Each action $a_i$ is a real number chosen from some finite set. That is, the space $\calA_i $  of action $a_i$ is   $\calA_i \subset \R_{[0, 1]}$ where $\abs{\calA_i} = K_i$  for some $K_i \in \nN$. Let $\calA$ be the space of all actions $\bm a = {\{ \ai\}_{i=0}^{m}}$. 
% Let $\calA$ be the space of all actions $\bm a = {\{ \ai\}_{i=0}^{m}}$.
We represent the actions as additional nodes in $\G$ (see \cref{fig:overview}): $\ai$ is a parent of only $X_i$, and hence an additional input to $\fofi$. Since $\fofi$ is unknown, the agent does not know apriori the functional effect of $\ai$ on $X_i$. Not intervening on a node $X_i$ can be considered equivalent to selecting $\ai = 0$. For nodes that cannot be intervened on by our agent, we set $K_i = 1$ and do not include the action in diagrams, meaning that without loss of generality we consider the number of action variables to be equal to the number of nodes $m$.
\footnote{There may be constraints on the actions our agent can take. We refer the reader to \citet{sussex2022model} for how our setup can be extended to handle constraints.}

For the adversary we consider the same intervention model but denote their actions by $\a'$ with each $\ai'$ defined over $\calA_i' \subset \R_{[0, 1]}$ where $\abs{\calA_i'} = K_i'$ and $K_i'$ is not necessarily equal to $K_i$. 

According to the causal graph, actions $\a, \a'$ induce a realization of the graph nodes: 
\begin{align}
\label{eq:groud_truth}
& \si = \fofi(\zi, \ai, \ai') + \noisei, \ \ \forall i \in [m].
\end{align}
 
If an index $i$ corresponds to a root node, the parent vector $\zi$ denotes an empty vector, and the output of $\fofi$ only depends on the actions.

\looseness-1

\paragraph{Problem statement}
Over multiple rounds, the agent and adversary intervene simultaneously on the SCM, with known DAG $\calG$ and fixed but unknown functions $\fs = \{\fofi\}_{i=1}^m$ with $\fofi: \calZ_i \times \A_i \times \A_i' \rightarrow \calX_i$. \looseness-1
At round $t$ the agent selects actions $\at = \{\ait\}_{i=0}^m$ and obtains observations $\st = \{\sit\}_{i=0}^m$, where we add an additional subscript to denote the round of interaction. When obtaining observations, the agent also observes what actions the adversary chose $\at' = \{\ait'\}_{i=0}^m$.  We assume the adversary does not have the power to know $\at$ when selecting $\at'$, but only has access to the history of interactions until round $t$. The agent obtains a reward given by \looseness-1
\begin{align}
\label{eq:groud_truth_target}
& y_t = f_m(\bm z_{m, t}, a_{m, t}, a_{m, t}') + \noise_{m, t},
\end{align}
which implicitly depends on the whole action vector $\at$ and adversary actions $\at'$. 

The agent's goal is to select a sequence of actions that maximizes their cumulative expected reward $\sum_{t=1}^T 
r(\at, \at')$ where $r(\at, \at') = \E{y_t\mid \at, \at'}$ and expectations are taken over $\snoise$ unless otherwise stated. The challenge for the agent lies in not knowing a-priori neither the causal model (i.e., the functions $\fs = \{\fofi\}_{i=1}^m$), nor the sequence of adversarial actions $\{\at'\}_{t=1}^{\cdots}$.

\paragraph{Performance metric} 

After $T$ timesteps, we can measure the performance of the agent via the notion of regret:
\begin{align}
    R(T) = \max_{\a \in \A} \sum_{t=1}^T r(\a, \at') - \sum_{t=1}^T r(\at, \at'),
    \label{eq:regret}
\end{align}
\ie, the difference between the best cumulative expected reward obtainable by playing a single fixed action if the adversary's action sequence and $\fs$ were known in hindsight, and the agent's cumulative expected reward. We seek to design algorithms for the agent that are \emph{no-regret}, meaning that $R(T)/T \rightarrow 0$ as $T\rightarrow \infty$, for any sequence $\at'$. We emphasize that while we use the term `adversary', our regret notion encompasses all strategies that the adversary could use to select actions. This might include cooperative agents or mechanism non-stationarities. \looseness -1


 For simplicity, we consider only adversary actions observed after the agent chooses actions. Our methods can be extended to also consider adversary actions observed \emph{before} the agent chooses actions, i.e., a \textit{context}. This results in learning a policy that returns actions depending on the context, rather than just learning a fixed action. This extension is straightforward and we briefly discuss it in~\Cref{app:contextual}. \looseness-1

\textbf{Regularity assumptions} We consider standard smoothness assumptions for the unknown functions $\fofi:\mathcal{S} \rightarrow \X_i$ defined over a compact domain $\mathcal{S}$ \citep{srinivas10}. In particular, for each node $i \in [m]$, we assume that $\fofi(\cdot)$ belongs to a reproducing kernel Hilbert space (RKHS) $\mathcal{H}_{k_i}$, a space of smooth functions defined on $\calS = \calZ_i \times \calA_i \times \calA_i'$.
This means that $\fofil \in \mathcal{H}_{k_i}$ is induced by a kernel function $k_i: \calS \times  \calS \rightarrow \mathbb{R}$. 
We also assume that $k_i(s,s') \leq 1$ for every $s, s' \in \calS$\footnote{This is known as the bounded variance property, and it holds for many common kernels.}. Moreover, the RKHS norm of $\fofi(\cdot)$ is assumed to be bounded $\|\fofi\|_{k_i} \leq \mathcal{B}_i$ for some fixed constant $\mathcal{B}_i>0$.  Finally, to ensure the compactness of the domains $\Z_i$, we assume that the noise $\snoise$ is bounded, i.e., $\noisei \in \left[-1,1\right]^{d}$. \looseness-1


% \subsection{Assigning steps to machines}
% \lstinputlisting[float=t,label=assign_machine,caption={Assigning steps to machines},%
% linerange={1-12}]{listing/optomachine.lp}

% After implementing a flexible sub-group and setup strategy, the next step is to allocate steps to machines within these subgroups. 

% The assignment of steps to machines are determined from rules in Listing~\ref{assign_machine}. 
% The two rules from line~\ref{asg1:begin}--\ref{asg2:end} determine whether a tuple $(L, P, S, Prio)$ of operation can be assigned to a machine $M$ of sub group $D$ in a tool group $G$. Here the setup strategy is not applicable $(setup{\_}strg = 0)$. In the second rule, the assignment of a tuple of operation holds same as in the previous rule. However, this rule is valid for the setup strategy which is determined by atoms $setup{\_}to{\_}machines\slash4$ and $step{\_}setup\slash3$. The $setup{\_}to{\_}machines(G, M, D, Setup)$ rule maps machine $M$ in subgroup with index $D$ based on the setup strategy for tool group $G$. We aim to ensure that if number of machines in subgroup is greater then the number of setups, all machines are equipped with a setup.  
% The $step{\_}setup(P, S, Setup)$ rule checks whether the $Setup$ can perform required for step $S$. These atoms are explained in setup-strategy section.  
% The last rule in line~\ref{asg:begin}--\ref{asg:end} assigns an operation $(L, P, S, Prio)$ in tool group $G$ to a machine $M$ (mach{\_}assign) within a valid subgroup, given that its subgroup index is $D$ (which is calculated using the previous rule discussed in sub-group strategy) and the subgroup is assignable. The rule uses a constraint logic programming (CLP) language construct to choose one valid $mach{\_}assign\slash6$ predicate that satisfies the $assignable\slash3$ condition.

% \subsection{Batching}
% \input{batch.tex}

% \subsection{Difference logic constraints}
% \lstinputlisting[float=t,label=dl,caption={Difference Logic},linerange={1-24}]{listing/dl.lp}

% The difference logic constraints are represented in Listing~\ref{dl} which determines the start time of steps. Due to space limitation the rules that define the sequencing of operation, maintenance and setup are not included in the paper. However the explanation of their predicates are as follows:

% \begin{itemize}
% 	\item $first\_step((L, P, Cur\_s, Prio), R)$. This states the first step $(L, P, Cur\_s, Prio)$ of job must be executed at the release time $R$. If there is a setup associated with the operation, then the start time of first step is the release time plus the setup time of the step. 
	
% 	\item $tool\_order(O, \_, Pro\_t)$. This defines the order of two operations $O$ and $O'$ that are executed on the same machine, and $Pro\_t$ is the processing time of the $O$ on that machine. In other words, this helps to establish a sequence of steps that use the same tool, then they must be executed in a certain order.
	
% 	\item $nofirst(O)$. The operation $O$ is not the first in the sequence of steps processing on its machine.
	
% 	\item $setup\_duration(O, G, Setup\_t)$. The setup duration $Setup\_t$ is calculated based on the changeover times between the current setup and the previous setup on the same machine group if required for operation $O$.
	
% 	\item $order(O1, O2, Pro_t1)$. This specifies the order of two steps $O1$ and $O2$ following the precedence constraint when belong to same job. 
	
% 	\item $pm\_and\_setup(O, Dur)$. This determines the total duration $Dur$ that sum up the setup and preventive maintenance time of a step $O$.   
	
% \end{itemize}


% \subsection{Optimization}    
% \lstinputlisting[float=t,label=prg:opt,caption={Optimization},linerange={1-17}]{listing/opt.lp}

% Our main objective is to minimize the makespan of all jobs which is taken as an external bound while optimizing weak constraints, such as setup and batching. By prioritizing setup over batching, we aim to achieve a better balance of constraints, leading to improve overall scheduling performance. 

% The optimization of makespan is represented in Listing~\ref{prg:opt}. Hence, we aim to optimize multi-objective criteria. Therefore the optimization consists of two parts: $\#program opt(b)$ and $\#program weak(b)$.

% \begin{itemize}
% 	\item $\#program opt(b)$ and $\#program weak(b)$:
	
% 	The line~\ref{opt} and line~\ref{weak} is the subprogram named $opt$ and $weak$ with parameter $b$. The $\#external bound(b)$ directive in both subprograms declares that $bound(b)$ is an external predicate, meaning that its truth value can be set from outside the program.
	
% 	\item $\&diff{makespan - 0} <= b :- bound(b).$:
	
% 	This rule in line~\ref{opt1} and line~\ref{weak1} is present in both the $opt$ and $weak$ subprograms. It states that the difference between the makespan (i.e., the total time taken for the manufacturing process) and 0 must be less than or equal to $b$ if $bound(b)$ is true. This difference logic constraint ensures that the makespan stays within the specified bound.
	
% 	\item Weak constraints:
	
% 	The rules from line~\ref{w:begin}--\ref{w:end} are soft constraints that the solver will try to minimize the violations of while still satisfying the hard constraints. In this case, there are two weak constraints:
	
% 	\begin{itemize}
% 		\item Setup:
		
% 		The constraint from line line~\ref{asgsetup:begin}--line~\ref{asgsetup:end}
% 		tries to minimize the number of setup changes that do not satisfy a certain minimum requirement $(Min_r1 - 1)$. The weight of this constraint is 1 and the priority is 2.
		
% 		\item Batch:
		
% 		The constraint from line~\ref{bat:begin}--line~\ref{bat:end} attempts to minimize the situations where the number of wafers $W$ in the $last\_step\_batch\slash5$ atom is less than the minimum batch size $(Min\_b)$ in the $batch\_info\slash4$ atom. The weight of this constraint is 1 and the priority is 1.
% 	\end{itemize}
	
% \end{itemize}

% In summary, these rules define optimization constraints for the makespan and two weak constraints to minimize setup changes and batch sizes that do not meet certain requirements.

% % Figure environment removed%


\section{Experiments}\label{sec:results}
\begin{table}[t]
	\centering
	\caption{Preallocation strategy results with $3$ machines per tool group and $10$ operations per lot}
	\label{tab:table}
	\figspace\scriptsize
	%	\resizebox{15.5cm}{!}{
		\begin{tabular}{|l%r
				cl||rr|rr|rr|rr|}
			%			\hline
			%			&                    &                      & %        &
			%			 \multicolumn{8}{c}{\textbf{M = 9}} \\
			\hline
			& \multicolumn{1}{@{\hspace{-3mm}}c@{\hspace{-3mm}}}{\textbf{9 Machines}}                   &                      & % &
			\multicolumn{2}{r|}{\textbf{70 Operations}}                 & \multicolumn{2}{r|}{\textbf{80 Operations}}                 & \multicolumn{2}{r|}{\textbf{90 Operations}}                 & \multicolumn{2}{r|}{\textbf{100 Operations}}                 \\
			& Size % \multicolumn{2}{c}{\textbf{Parameters}}            
			&        &
			Lot                         & Step                        & Lot                         & Step                        & Lot          & Step         & Lot          & Step         \\
			%			& size              % & setup % idx
			%			                  &         & 0                           & 1                           & 0                           & 1                           & 0            & 1            & 0            & 1            \\
			%			&                    & setup                &         &                             &                             &                             &                             &              &              &              &              \\
			\hline\hline
			\multirow{3}{*}{\textbf{Fixed}}    & \multirow{3}{*}{1} & % \multirow{3}{*}{0/1} &
			Makespan    & 483                         & 428                         & 489                         & 440                         & 486          & 531          & 592          & 553         \\
			&                    & %                     &
			Setup/Batch & 6/12                        & 2/12                        & 5/14                        & 0/13                        & 5/14         & 3/12         & 3/12         & 0/16         \\
			&                    & %                     &
			1\ts{st}/2\ts{nd} Stage & 2/1                         & TO/27                          & 6/2                        & TO/13                          & 11/13         & TO           & TO/78           & TO           \\
			\midrule
			\multirow{6}{*}{\textbf{Flexible}} & \multirow{3}{*}{2} & % \multirow{6}{*}{0}   &
			Makespan    & 483                         & 475                         & 592                         & 592                         & 592          & 539          & 745          & 698          \\
			&                    & %                     &
			Setup/Batch & 2/8                        & 0/9                        & 1/8                        & 1/8                        & 1/10         & 0/11          & 0/12          & 0/15          \\
			&                    & %                     &
			1\ts{st}/2\ts{nd} Stage & 5/1                         & TO                          & TO/114                          & TO/1                          & TO/130           & TO           & TO           & TO          \\
			\cline{2-11}
			%			& & & & & & & & & & &   \\
			& \multirow{3}{*}{3} & %                     &
			Makespan    & 559                         & --                          & 815                         & --                          & 1357 & -- & 1486 & -- \\ % \multicolumn{4}{c|}{\multirow{3}{*}{Assignment issue}}     \\
			&                    & %                     &
			Setup/Batch & 0/8                         & --                          & 0/8                        & --                          & 0/10 & -- & 10/18 & -- \\ %\multicolumn{4}{c|}{}                                      \\
			&                    & %                     &
			1\ts{st}/2\ts{nd} Stage & TO                       & --                          & TO/140                          & --                          & TO/79 & -- & TO & -- \\ %\multicolumn{4}{c|}{}                                      \\
			\midrule
			\multirow{6}{*}{\textbf{Setup}}    & \multirow{3}{*}{2} & % \multirow{6}{*}{1}   &
			Makespan    & 483                         & 475                         & 592                         & 592                         & 592          & 536          & 745          & 683          \\
			&                    & %                     &
			Setup/Batch & 2/8                        & 0/9                        & 1/8                        & 1/8                        & 1/10         & 0/12          & 0/13          & 0/16          \\
			&                    & %                     &
			1\ts{st}/2\ts{nd} Stage & 2/1                        & TO                          & TO/21                          & TO/25                          & TO/22           & TO           & TO/76           & TO           \\
			%			& & & & & & & & & & &   \\
			\cline{2-11}
			& \multirow{3}{*}{3} & %                     &
			Makespan    & \textbf{334}                         & --                          & \textbf{345}                         & --                          & \textbf{434}          & --           & \textbf{555}          & --           \\
			&                    & %                     &
			Setup/Batch & 0/8                         & --                          & 0/8                         & --                          & 0/11          & --           & 0/12          & --           \\
			&                    & %                     &
			1\ts{st}/2\ts{nd} Stage & TO/20                       & --                          & TO/123                          & --                          & TO           & --           & TO/73           & --           \\
			\hline
		\end{tabular}
		%	}
\end{table}
%
We constructed a scalable set of benchmark instances, focusing on sub-routes of
$10$ production operations for two product types from the SMT2020 simulation scenario~\cite{kopp2020smt2020}.
The $10$ operations in both sub-routes are processed by machines
belonging to three tool groups and do thus involve re-entrant flow,
as a lot visits the same tool group multiple times.
Moreover, the operations incorporate batching and specific setups, and machines undergo periodic maintenance operations.
In the following, we concentrate on instances with $9$ machines, i.e., $3$ per
tool group, and gradually increasing number of lots.
Further smaller- and larger-scale instances along with our implementation are
available online.\footref{foo:online}

We ran our experiments with \clingodl\ (version 1.4.0) on an Intel® Core™i7-8650U CPU Dell Latitude 5590 machine under Windows 10, imposing two time limits per run:
the first stage for makespan minimization is aborted at $450$ seconds, in which case the best schedule found so far % (if any) 
is taken as upper bound on the makespan for proceeding to minimize setup and batch violations with 
another $150$ seconds time limit.

Table~\ref{tab:table} reports the quality of best schedules obtained within the time limits for both optimization stages, split into `Makespan' and `Setup/Batch'
values, while two runtimes or `TO' for a timeout, respectively, are given in the
`1\ts{st}/2\ts{nd} Stage' rows, only listing a single `TO' entry in case both stages timed out.
The `Size' column provides the value taken for the constant \lstinline{sub_size},
limiting the number of machines in subgroups to which the operations are preallocated.
For the latter, the `Lot' columns include results with value \lstinline{0} for the constant \lstinline{lot_step}, where a common subgroup takes all operations for a lot, or for value \lstinline{1} in the `Step' columns, leading to their distribution among subgroups.

The `Size' value 1 necessarily leads to a fixed machine assignment, for which the
quality indicators clearly show that the `Step' strategy yields better schedules,
although it incurs more timeouts and thus fewer certain optima because operations on different lots increase the flexibility of execution sequences and thus search complexity.
While flexibility within subgroups by setting their `Size' to 2 or 3 in principle allows for improved schedules, we observe a deterioration due to sharply increasing instantiation size and search effort, as already observed in \cite{ali2023flexible}.
The setup strategy to differentiate operations and machines within subgroups,
activated by changing the constant \lstinline{by_setup},
aims to cut down the scheduling complexity in line with the optimization objectives by reducing the need for setup changes.
This leads to significantly improved schedules with `Size' 3, where the
`Lot' and `Step' preallocation strategies are indifferent and redundant results for the latter are omitted, up to a critical size reached with $100$~operations.

With our preliminary approach~\cite{ali2023flexible}, using a more naive and less feature-rich encoding of either fixed or fully flexible machine assignments, the
threshold at which problem size and combinatorics get prohibitive was reached at less than $50$ operations already.
Despite gearing up to double that size, our benchmark instances still represent small excerpts of the large-scale semiconductor fabs with more than $100$ tool groups and from $242$ to $543$ production operations per lot modeled by~\cite{kopp2020smt2020}.
%
The elevated complexity in comparison to basic settings like the traditional FJSP is mainly due to sophisticated setup and maintenance operations, requiring a detailed analysis of execution sequences on machines for SMSP.
We conjecture that similar scalability limits would also be encountered with MIP or CP encodings, yet the first-order modeling language of ASP with difference logic facilitates rapid prototyping and experimentation.
In fact, our performance evaluation aims to explore the feasibility of search and optimization, in order to come up with strategies for breaking down large SMSP instances into more manageable portions, e.g., focusing on some bottleneck tool groups or re-entrant flow of operations.

% This section will show the experimental results performed by applying the machine assignment strategies mentioned before, with several instances ranging from $30$ to $130$ steps and $6$ to $12$ machines. All experiments are run using an Intel\textsuperscript{\textregistered} Core\texttrademark{} i7-8650U CPU Dell Latitude 5590 machine under Windows 10. Our timeout limit is $600$ seconds, splitted to $450$ seconds for the makespan and $150$ seconds for the setup and batching. 

% We considered three tool groups for all generated instances in which batch processing, time/counter-based maintenance, and setup are considered. For generating the instances, we started with a small instance containing $30$ steps and $6$ machines where each tool group has $2$ machines and then we generate the next instance by adding one more lot, which has $10$ steps. We kept the tool group size till the fixed machine assignment strategy could not reach the optimum within the time limit. We created $3$ parameters \textit{size, idx} and \textit{setup} to activate a specific machine assignment strategy. The size determines the size of a sub-group in each tool group. The $idx$ defines the Job/Step-based indexing of all steps in the same tool group where all steps of the same lot will have the same index if the $idx = 0$ and Hence, they are assigned to the same sub-group/machine. If $idx = 1$, then each step in the tool group will have an identical index. The last parameter setup is to activate the setup strategy or not. If the $setup = 1$, then the setup strategy is applied; if $setup = 0$ then it's not applied.

% % To continue tomorrow isA :)
% Table \ref{tab:table01} shows the results of the instances with $2$ machines in each toll group. The first column refers to the strategy applied for the machine assignment. The second and third columns show the parameters for selecting a particular strategy. The assignment is fully flexible if the \textit{size} is greater than or equal to the number of machines in a tool group. Otherwise, the assignment is partially flexible. In the fourth column, we list our optimization criteria and the time limit for the makespan and setup/batching represented by 1st/2nd call. Each following two consecutive columns illustrate the results of an instance when the Job/Step-based indexing is selected. From the \ref{tab:table01}, we observed that the best-obtained results were achieved by the full flexible assignment in the first three instances and for the last instance, the setup strategy was the best. The fixed/setup strategies terminated within the time limit except for only one case.

% \begin{table}[h]
% 	\centering
% 	\caption{Comparison between the allocation strategies with 2 machines per tool group}
% 	\label{tab:table01}
% %	\resizebox{15.5cm}{!}{
% 		\begin{tabular}{|l%r
% 			cl||rr|rr|rr|rr|}
% 			\hline
% %			&                    &                      &         & \multicolumn{8}{c}{\textbf{M = 6}} \\
% %			\hline
% 			& \textbf{M = 6}                   & %                     &
% 			  & \multicolumn{2}{r|}{\textbf{Instance 01}}                 & \multicolumn{2}{r|}{\textbf{Instance 02}}                 & \multicolumn{2}{r|}{\textbf{Instance 03}}                 & \multicolumn{2}{r|}{\textbf{Instance 04}}                 \\
% 			& Size % \multicolumn{2}{c}{\textbf{Parameters}}            
% 			 &			         & Job                         & Step                        & Job                         & Step                        & Job          & Step         & Job          & Step         \\
% 			\hline
% %			& size               & setup %idx
% %			                  &         & 0                           & 1                           & 0                           & 1                           & 0            & 1            & 0            & 1            \\
% %			&                    & setup                &         &                              &                             &                             &                             &              &              &              &              \\
% 			\hline
% 			\multirow{3}{*}{\textbf{Fixed}}    & \multirow{3}{*}{1} & % \multirow{3}{*}{0/1} &
% 			 Makespan    & 409                         & 353                         & 409                         & 409                         & 525          & 424          & 525          & 493          \\
% 			&                    & %                     &
% 			 Setup/Batch & 5/6                         & 4/6                         & 4/8                         & 4/8                         & 4/9          & 1/9          & 3/11          & 2/10          \\
% 			&                    & %                     &
% 			 1\ts{st}/2\ts{nd}-Call & \textless{}1/\textless{}1 & \textless{}1/\textless{}1 & \textless{}1/\textless{}1 & \textless{}1/\textless{}1 & 31/1         & 137/6        & 37/11          & TO/53           \\
% 			\midrule
% 			\multirow{3}{*}{\textbf{Flexible}} & \multirow{3}{*}{2} & % \multirow{3}{*}{0}   &
% 			 Makespan   & \textbf{233}                         & --                          & \textbf{281}                         & --                          & \textbf{365}          & --           & 587          & --           \\
% 			&                    & %                     &
% 			 Setup/Batch & 0/5                         & --                          & 0/6                         & --                          & 0/8          & --           & 3/9          & --           \\
% 			&                    & %                     &
% 			 1\ts{st}/2\ts{nd}-Call & 7/0                         & --                          & TO/6                          & --                          & TO/83           & --           & TO           & --           \\
% 			\midrule
% 			\multirow{3}{*}{\textbf{Setup}}    & \multirow{3}{*}{2} & % \multirow{3}{*}{1}   &
% 			 Makespan  & 277                         & --                          & 321                         & --                          & 381          & --           & \textbf{419}          & --           \\
% 			&                    & %                     &
% 			 Setup/Batch & 0/4                         & --                          & 0/6                         & --                          & 0/8          & --           & 0/9          & --           \\
% 			&                    & %                     &
% 			 1\ts{st}/2\ts{nd}-Call & \textless{}1/\textless{}1 & --                          & 25/1                         & --                          & TO/12        & --           & TO/122           & -- \\
% 			 \hline
% 		\end{tabular}
% %	}
% \end{table}

% Table~\ref{tab:table02} summarizes the results of the subsequent $4$ instances where each tool group has $3$ machines. In this instances group, we can split the machines into sub-group by setting the \textit{size} parameter to $2$; in that case, we have two sub-groups in each tool group. The experiments demonstrated that the fixed strategy has the same or better performance than the flexible. In addition, the flexible strategy could not find a feasible solution for instances $7$ and $8$ when all machines were in the same group. On the other hand, the setup strategy performed better than the other two strategies when all machines were in one group, in addition to reaching the optimal value of the setup for all instances. 

% \begin{table}[h]
% 	\centering
% 	\caption{Comparison between the allocation strategies with 3 machines per tool group}
% 	\label{tab:table02}
% %	\resizebox{15.5cm}{!}{
% 		\begin{tabular}{|l%r
% 			cl||rr|rr|rr|rr|}
% %			\hline
% %			&                    &                      & %        &
% %			 \multicolumn{8}{c}{\textbf{M = 9}} \\
% 			\hline
% 			& \textbf{M = 9}                   &                      & % &
% 			 \multicolumn{2}{r|}{\textbf{Instance 05}}                 & \multicolumn{2}{r|}{\textbf{Instance 06}}                 & \multicolumn{2}{r|}{\textbf{Instance 07}}                 & \multicolumn{2}{r|}{\textbf{Instance 08}}                 \\
% 			& Size % \multicolumn{2}{c}{\textbf{Parameters}}            
% 			&        &
% 			 Job                         & Step                        & Job                         & Step                        & Job          & Step         & Job          & Step         \\
% %			& size              % & setup % idx
% %			                  &         & 0                           & 1                           & 0                           & 1                           & 0            & 1            & 0            & 1            \\
% %			&                    & setup                &         &                             &                             &                             &                             &              &              &              &              \\
% 			\hline\hline
% 			\multirow{3}{*}{\textbf{Fixed}}    & \multirow{3}{*}{1} & % \multirow{3}{*}{0/1} &
% 			 Makespan    & 525                         & 433                         & 525                         & 452                         & 525          & 521          & 643          & \textbf{559}          \\
% 			&                    & %                     &
% 			 Setup/Batch & 6/13                        & 1/13                        & 5/15                        & 0/14                        & 5/16         & 6/16         & 6/12         & 3/12         \\
% 			&                    & %                     &
% 			 1\ts{st}/2\ts{nd}-Call & 30/3                         & TO/153                          & 24/8                        & TO/63                          & 231/81         & TO           & TO           & TO           \\
% 			\midrule
% 			\multirow{6}{*}{\textbf{Flexible}} & \multirow{3}{*}{2} & % \multirow{6}{*}{0}   &
% 			 Makespan    & 525                         & 475                         & 650                         & 650                         & 650          & 595          & 745          & 742          \\
% 			&                    & %                     &
% 			 Setup/Batch & 2/11                        & 0/11                        & 1/12                        & 1/12                        & 6/13         & 4/14          & 3/17          & n/a          \\
% 			&                    & %                     &
% 			 1\ts{st}/2\ts{nd}-Call & 26/7                         & TO                          & TO/12                          & TO                          & TO           & TO           & TO           & TO           \\
% 			\cline{2-11}
% %			& & & & & & & & & & &   \\
% 			& \multirow{3}{*}{3} & %                     &
% 			 Makespan    & 744                         & --                          & 1206                         & --                          & 1698 & -- & n/a & -- \\ % \multicolumn{4}{c|}{\multirow{3}{*}{Assignment issue}}     \\
% 			&                    & %                     &
% 			 Setup/Batch & 2/12                         & --                          & n/a                        & --                          & 8/15 & -- & n/a & -- \\ %\multicolumn{4}{c|}{}                                      \\
% 			&                    & %                     &
% 			 1\ts{st}/2\ts{nd}-Call & TO                       & --                          & TO                          & --                          & TO & -- & TO & -- \\ %\multicolumn{4}{c|}{}                                      \\
% 			\midrule
% 			\multirow{6}{*}{\textbf{Setup}}    & \multirow{3}{*}{2} & % \multirow{6}{*}{1}   &
% 			 Makespan    & 525                         & 475                         & 650                         & 650                         & 643          & 553          & 745          & 642          \\
% 			&                    & %                     &
% 			 Setup/Batch & 2/11                        & 0/11                        & 1/12                        & 1/12                        & 1/14         & 0/13          & 1/14          & 1/16          \\
% 			&                    & %                     &
% 			 1\ts{st}/2\ts{nd}-Call & 44/2                        & TO                          & TO/4                          & TO/2                          & TO           & TO/7           & TO           & TO           \\
% %			& & & & & & & & & & &   \\
% 			\cline{2-11}
% 			& \multirow{3}{*}{3} & %                     &
% 			 Makespan    & \textbf{346}                         & --                          & \textbf{373}                         & --                          & \textbf{429}          & --           & 820          & --           \\
% 			&                    & %                     &
% 			 Setup/Batch & n/a                         & --                          & n/a                         & --                          & n/a          & --           & n/a          & --           \\
% 			&                    & %                     &
% 			 1\ts{st}/2\ts{nd}-Call & TO                       & --                          & TO                          & --                          & TO           & --           & TO           & --           \\
% 			\hline
% 		\end{tabular}
% %	}
% \end{table}

% Table~\ref{tab:table03} considers $4$ machines in each tool group and the flexible strategy obtained the best result for the first instance. However, it had the same feasibility issue when all machines were in the same group. For the rest instances, the setup strategy dominated when the machines were equally distributed into sub-groups. 

% From the conducted experiments, we can conclude that 
% \begin{itemize}
% 	\item The flexible assignment performed well on the small-scale.
% 	\item While increasing the scale, the setup strategy dominates in the most cases
% 	\item Assigning the steps of the same lot independently with the fixed assignment leads to better performance
% 	\item The Setup strategy has a significant impact in minimizing the setup objective through all instances
% 	\item The full flexible assignment has an assignment issue while increasing the number of machines
% \end{itemize}

% \begin{table}[h]
% 	\centering
% 	\caption{Comparison between the allocation strategies with 4 machines per tool group}
% 	\label{tab:table03}
% %	\resizebox{15.5cm}{!}{%
% 		\begin{tabular}{|l%r
% 			cl||rr|rr|rr|rr|}
% 			\hline
% %			&                    &                      &  &  \multicolumn{8}{c}{\textbf{M = 12}} 
% %			\\ \hline
% 			& \textbf{M = 12}                   & %                     & 
% 			 & \multicolumn{2}{r|}{\textbf{Instance 09}}                 & \multicolumn{2}{r|}{\textbf{Instance 10}}                 & \multicolumn{2}{r|}{\textbf{Instance 11}}                 & \multicolumn{2}{r|}{\textbf{Instance 12}}                 \\
% 			& Size % \multicolumn{2}{l}{\textbf{Parameters}}            
% 			 &			 &			 Job                    & Step                   & Job                    & Step                   & Job                    & Step                   & Job                    & Step                   \\
% %			& Size               & setup % idx
% %			                  &  & 0                      & 1                      & 0                      & 1                      & 0                      & 1                      & 0                      & 1                      \\
% %			&                    & setup                &  &  &                        &                        &                        &                        &                        &                        &                                               \\
% 			\hline\hline
% 			\multirow{3}{*}{\textbf{Fixed}}    & \multirow{3}{*}{1} & % \multirow{3}{*}{0/1} &
% 			 Makespan                 & 525                    & 453                    & 525                    & 452                    & 525                    & 493                    & 643                    & 561                    \\
% 			&                    & %                     &
% 			 Setup/Batch              & 7/19                   & 3/20                   & 7/20                  & n/a                   & 6/22                   & 4/20                   & 4/22                   & n/a                   \\
% 			&                    & %                     &
% 			 1\ts{st}/2\ts{nd}-Call              & 124/5                 & TO & 25/17                 & TO & 25/53                 & TO/142 & TO & TO \\
% 			\midrule
% 			\multirow{9}{*}{\textbf{Flexible}} & \multirow{3}{*}{2} & % \multirow{9}{*}{0}   &
% 			 Makespan                 & \textbf{373}                    & 503                    & 491                    & 778                    & 569                    & 569                    & 765                    & 1673                   \\
% 			&                    & %                     &
% 			 Setup/Batch              & n/a                    & 6/17                    & n/a                   & n/a                    & n/a                    & n/a                   & n/a                    & 12/24                  \\
% 			&                    & %                     &
% 			 1\ts{st}/2\ts{nd}-Call              & TO & TO & TO & TO & TO & TO & TO & TO \\
% 			\cline{2-11}
% %			& & & & & & & & & & &   \\
% 			& \multirow{3}{*}{3} & %                     &
% 			 Makespan                 & 709                    & 688                    & 800                    & 907                    & 876                    & 876                    & 905                    & 1643                   \\
% 			&                    & %                     &
% 			 Setup/Batch              & 5/17                    & n/a                   & 3/18                   & 5/19                   & n/a                   & n/a                   & n/a                  & 15/24                    \\
% 			&                    & %                     &
% 			 1st/2nd              & TO & TO & TO & TO & TO & TO & TO & TO \\
% 			\cline{2-11}
% %			& & & & & & & & & & &   \\
% 			& \multirow{3}{*}{4} & %                     &
% 			 Makespan                 & n/a & -- & n/a & -- & n/a & -- & n/a & -- \\ %\multicolumn{8}{c|}{\multirow{3}{*}{Assignment issue}}                                                                                                                                                 \\
% 			&                    & %                     &
% 			 Setup/Batch              & n/a & -- & n/a & -- & n/a & -- & n/a & -- \\ %\multicolumn{8}{c|}{}                                                                                                                                                                                  \\
% 			&                    & %                     &
% 			 1\ts{st}/2\ts{nd}-Call              & TO & -- & TO & -- & TO & -- & TO & -- \\ %\multicolumn{8}{c|}{}                                                                                                                                                                                  \\
% 			\midrule
% 			\multirow{9}{*}{\textbf{Setup}}    & \multirow{3}{*}{2} & % \multirow{9}{*}{1}   &
% 			 Makespan                 & 401                    & 396                    & 419                    & \textbf{416}                    & \textbf{419}                    & \textbf{419}                    & \textbf{457}                    & 471                    \\
% 			&                    & %                     &
% 			 Setup/Batch              & 0/15                   & 0/14                   & 0/16                   & 0/16                   & n/a                   & n/a                   & 0/21                    & n/a                    \\
% 			&                    & %                     &
% 			 1\ts{st}/2\ts{nd}-Call              & TO & TO/92 & TO & TO & TO & TO & TO & TO \\
% 			\cline{2-11}
% %			& & & & & & & & & & &   \\
% 			& \multirow{3}{*}{3} & %                     &
% 			 Makespan                 & 706                    & 642                    & 792                    & 753                    & 942                    & 942                    & 939                    & 894                    \\
% 			&                    & %                     &
% 			 Setup/Batch              & 1/14                    & n/a                    & 2/16                    & n/a                   & n/a                   & n/a                    & n/a                    & 1/22                    \\
% 			&                    & %                     &
% 			 1\ts{st}/2\ts{nd}-Call              & TO & TO & TO & TO & TO & TO & TO & TO \\
% 			\cline{2-11}
% %			& & & & & & & & & & &   \\
% 			& \multirow{3}{*}{4} & %                     &
% 			 Makespan                 & 679                    & -- & 1725                    & -- & n/a                    & -- & n/a                    & -- \\
% 			&                    & %                     &
% 			 Setup/Batch              & n/a                   & -- & n/a                    & -- & n/a                   & -- & n/a                   & -- \\
% 			&                    & %                     &
% 			 1st/2nd              & TO & -- & TO & -- & TO & -- & TO & -- \\
% 			\hline
% 		\end{tabular}%
% %	}
% \end{table}

\section{Conclusion}\label{sec:conclusion}
\section{Conclusion and Future Work}
In this work, I design corruption-robust algorithms for the Lipschitz contextual search problem. I present the \emph{agnostic checking} technique and demonstrate its effectiveness in designing corruption-robust algorithms. There are several open problems for future research. First, in the algorithm I propose for pricing loss, the schedule for agnostic checks is fixed upfront. Can the learner design an adaptive checking schedule for the pricing loss? Second, this work assumes the learner has knowledge of the Lipschitz constant $L$. Can the learner design efficient no-regret algorithms without knowledge of $L$? 

\paragraph{Acknowledgments}
This work was funded by 
FFG project 894072 (SwarmIn) as well as
KWF project 28472, cms electronics GmbH, FunderMax GmbH, Hirsch Armbänder GmbH, incubed IT GmbH, Infineon Technologies Austria AG, Isovolta AG, Kostwein Holding GmbH, and Privatstiftung Kärntner Sparkasse.
%
We are greatful to the anonymous reviewers for their helpful comments.

\bibliographystyle{splncs04}
\bibliography{reference}

\end{document}