We proposed a machine-learning based method to approximate diagonal as well as non-diagonal elements of the Hessian of a molecule. The representation used is specific for every internal coordinates, and takes explicitly into account the bond order, which is sensible because a single point DFT calculation is computationally considerably less expensive that the explicit calculation of the Hessian.
We trained our ML model on a relatively small dataset (subset of QM7) of less than 7000 molecules. The Hessian was computed at the B3LYP/cc-pVDZ level of theory. 
The agreement between ML and DFT was satisfactory. In particular, the calculated MAPE for bond stretching force constant was below 2\%, and was particularly small for bonds involving hydrogen atoms because they point outwards and are less affected by the chemical environment. The MAPE for bending and torsion was of 5\% and 10\%, respectively. 
From the ML model trained on QM7 we were also able to predict the Hessian of some molecules representative of the QM9 dataset. The Hessian predicted in internal coordinates was then transformed into the mass-weighted Cartesian Hessian, the diagonalization of which yields the harmonic vibrational frequencies and normal modes, that can be compared to the ones calculated  explicitly from DFT.

High frequency vibrations and normal modes were predicted accurately, while lower frequency ones were not. This behaviour is analogous to the IR spectroscopy theory, where stretchings and bendings can be identified accurately, while torsion and delocalized vibrations are more difficult to be interpreted.

The approximate Hessian obtained with ML is computational inexpensive and can be used as an initial Hessian guess for geometry optimization, or in the context of alchemical geometry relaxation \cite{Domenichini2020,domenichini2022alchemical, shiraogawa2022exploration,shiraogawa2023optimization}. 
A good starting Hessian may speed up the convergence of the geometrical optimization. An in detail assessment of the performance of the ML Hessian proposed is not yet provided, but should carefully take into account many parameters on which the optimization depends, \textit{e.g.} the type of molecule, the initial geometry, the optimization algorithm, and the Hessian update scheme.

