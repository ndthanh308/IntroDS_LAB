\documentclass[aip, jmp, amsmath,amssymb, reprint,floatfix,]{revtex4-1}

\usepackage{import}
\usepackage{graphicx}
\graphicspath{ {./Figures/} }
\usepackage{mathtools}
\usepackage{dcolumn}
\usepackage{bm}
\usepackage{braket}
\usepackage{enumitem}
\usepackage{float}
\usepackage[section]{placeins} 
\usepackage{booktabs}
\usepackage{array}
\usepackage{hyperref} 

\begin{document}


\title{Molecular Hessian matrices from a machine learning random forest regression algorithm}
\author{Giorgio Domenichini} 
\email{giorgio.domenichini@univie.ac.at}
\affiliation{Faculty of Physics,
University of Vienna,
Kolingasse 14-16,
1090 Vienna,
Austria}

\author{Christoph Dellago} 
\email{christoph.dellago@univie.ac.at}
\affiliation{Faculty of Physics,
University of Vienna,
Kolingasse 14-16,
1090 Vienna,
Austria}

\begin{abstract} In this article we present a machine learning model to obtain fast and accurate estimates of the molecular Hessian matrix. In this model, based on a random forest, the second derivatives of the energy with respect to redundant internal coordinates are learned individually. The internal coordinates together with their specific representation guarantee rotational and translational invariance. The model is trained on a subset of the QM7 data set, but is shown to be applicable to larger molecules picked from the QM9 data set. From the predicted Hessian it is also possible to obtain reasonable estimates of the vibrational frequencies, normal modes and zero point energies of the molecules.
\end{abstract}
\maketitle
\section{Introduction}

\import{./}{1_Introduction.tex}

\section{Methods}
\import{./}{2A_Redundant_internal_coordinates}
\import{./}{2B_Representations}
\import{./}{2C_ML_method}
\import{./}{2D_Computational_details}

\section{Numerical Results}
\import{./}{3A_CrossValidation}
\import{./}{3B_nondiagonal}
\import{./}{3C_Vibrations}


\section{Conclusion}
\import{./}{5_conclusion.tex}

\section{Acknowledgements}

We acknowledge financial support of the Austrian Science Fund (FWF) through the SFB TACO, Grant number F 81-N. The computational results presented have been achieved in part using the Vienna Scientific Cluster (VSC).

We also would like to thank Jan Weinreich, Michail Sahre, and Luigi Ranalli, from the University of Vienna, as well as Danish Khan and Anatole von Lilienfeld from the University of Toronto, for the productive discussions and feedback received.


\section{References}
%\bibliographystyle{IEEetran}
\bibliography{bib_files/BasisRef,bib_files/Alchemy_vLg,bib_files/Alchemy_Others, bib_files/Alchemy_Cardenas, bib_files/Alchemy_Geerlings, bib_files/Alchemy_Keith, bib_files/software, bib_files/Other_works ,bib_files/QML, bib_files/derivatives,bib_files/Geomopt,bib_files/Dellago,bib_files/behler_parrinello,bib_files/randomforest,bib_files/IRspectroscopy,bib_files/Geometryoptimizations_review,bib_files/schlegel,bib_files/Barone_ML,bib_files/Heiniswork,bib_files/MLhessianspectra,bib_files/Ceotto_ML,bib_files/Domenichini,bib_files/Alchemy_Hasegawa, bib_files/datasets}

\end{document}