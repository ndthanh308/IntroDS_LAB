\subsection{Non-diagonal terms}
\label{sec:nondiag}
While the non-diagonal elements of the Hessian are usually small and often assumed zero in many approximations \cite{schlegel1984_hess_est}, some of them can be successfully predicted using the machine learning framework presented in this article. We applied machine learning to some of the most important non-diagonal elements, assuming that the geometrical proximity of the internal coordinates is correlated with the magnitude of the Hessian term.
In particular, we predicted the following terms of the Hessian: consecutive (sharing one atom) bond-bond terms; included (the bond is a side of the angle) and adjacent (sharing one vertex)  bond-angle terms; adjacent (sharing one side and the vertex), consecutive (sharing one side), and opposite (sharing the vertex) angle-angle terms; and the mixed terms between a dihedral angle and the bonds (internal and external) between the atoms which defining the dihedral. The combinations of internal coordinates, for which we predicted the non-diagonal Hessian elements, are schematically shown in the various panels of Fig. \ref{fig:Nondiags}.
For each type of these non-diagonal terms we constructed a specific representation, built from the enumeration of the type and the positions of the atoms surrounding the involved internal coordinates, similar to what was done for the diagonal terms in Sec.\ref{sec:Representations}. As for the diagonal terms of the Hessian, coordinates pairs are now classified according to the chemical elements which constitute them, so that different chemical elements are predicted with different machine learning models. We refer the reader to the code published on the Zenodo repository for a detailed definition of the representation \cite{domenichini_dataset_hessian}.

% Figure environment removed
  
\begin{table}[h]
    \centering
    \begin{tabular}{l|c}
    Coordinate Pair & MAE (  10$^{-3}$ a.u.) \\
    \hline
     Bond-bond consecutive & 0.54  \\
     Bond-angle included & 0.86  \\
     Bond-angle  adjacent & 0.47 \\
     Angle-angle adjacent & 0.99  \\
     Angle-angle consecutive & 1.66 \\
     Angle-angle opposite & 0.89  \\
     Bond-dihedral external & 0.69  \\
     Bond-dihedral internal & 0.87  \\
    \end{tabular}
    \caption{Mean absolute error for the prediction of the non-diagonal Hessian terms, averaged over 10 train-test splits of the dataset.}
    \label{tab:nondiag_MAE}
\end{table}

The scatter plots shown in Figure \ref{fig:Nondiags} are for one train test split of our data. In the figure, data obtained for every possible combination of atomic species, are shown together. Values are expressed in atomic units (a.u.), which correspond to Hartree/Bohr$^2$, Hartree/(Bohr*Radians), Hartree/Radians$^2$, depending on the type of coordinate pairs involved.
 
As can be inferred from Figure \ref{fig:Nondiags}, the correlation between true (calculated) and predicted values is particularly good for panels a, b and c, which are related to bond-bond and bond-angle terms. The errors are largest in the prediction of consecutive angle-angle terms (panel e), and bond-dihedral elements (panels g and h). All of these non-diagonal elements are small in magnitude and can be zero. 
For this reason, instead of the percentage error, we report in Table \ref{tab:nondiag_MAE} the mean absolute error (MAE), averaged over 10 train test splits. This error is on the order of $10^{-3}$ a.u., which is small compared to the range $\pm 0.1$ a.u. of possible values shown in Fig. \ref{fig:Nondiags}.