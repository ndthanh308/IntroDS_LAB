\subsection{Prediction of QM9 Hessians}
%which thousand separator to use?
% Figure environment removed

Given the locality of our representation, it is possible to predict the Hessian matrix for molecules of any given size. While we trained the machine learning model on the QM7 data set, we will now test it by predicting the Hessian for molecules of the QM9 dataset. As an example, we picked the molecules of QM9 whose index is a multiple of 10,000, from molecule 10,000 to 130,000. These 13 molecules possess 8 or 9 heavy atoms, so they are bigger than any molecule in the training set.

At first, we optimized the molecular geometries using the B3LYP/cc-pVDZ level of theory, and  calculated the Hessians explicitly, in order to compared them with the ML predictions.
For the ML Hessian expressed in RICs, we calculated all the diagonal elements as well as the non-diagonal elements listed in section \ref{sec:nondiag}.
The transformation from internal ($H_q$) to Cartesian ($H_x$) coordinates was performed according to Equation \ref{eq:hess_to_cartesian}.
After the transformation to Cartesian coordinates, it is possible to calculate the expected harmonic vibrational frequencies, as the eigenvalues of the mass-weighted Hessian matrix while the corresponding eigenvectors correspond to the vibrational normal modes\cite{MCMURRY1959203_mwh,Wilson1941_mwh,li2002partial_mwh} (Eq.\ref{eq:red_mass_H}). 
\begin{equation} \label{eq:red_mass_H}
  H^\mathrm{MW}_{x_Ix_J} := \frac{H_{x_Ix_J}}{\sqrt{m_I {m_J}}}
\end{equation}
Here, $x_I$ is a Cartesian coordinate of atom $I$ and $m_I$ is its atomic mass. 


While the simulation of an IR spectrum should include the explicit calculation absorption intensities and peak broadenings \cite{herman1955influence,segal1967calculation,IRabsorb_vibrfreq_Ziegler},
here we applied a Gaussian convolution with a fixed intensity and broadening to graphically compare the calculated and the ML-predicted vibrational frequencies.
As an example, Fig.\ref{fig:QM9_vibrfreq} shows a comparison between the harmonic frequencies calculated using B3LYP with the cc-pVDZ basis set and the ML predictions for molecule Nr 10,000 of QM9 ((1-cyano-2-oxoethyl)formate) . 
From the cross-validation scores, shown in section \ref{sec:CV}, it is no surprise that the frequencies associated with bond stretching are predicted fairly accurately by the machine learning method proposed. Around 3000 $cm^{-1}$ one clearly recognizes the C-H stretching modes, which differ from the calculated ones by less than 5$cm^{-1}$. The vibrational stretchings of the cyano group(2360 $cm^{-1}$) and the carbonyl groups(1840-1890 $cm^{-1}$), as well as the scissoring of the carbonyl O=C-H angles, are predicted with an error lower than 10$cm^{-1}$.
For lower frequency vibrations the predictions are not as accurate for mainly two reasons: the error in the prediction of Hessian elements associated with angles and dihedrals is larger than for bonds, and, more importantly, such vibrations are strongly non-local, and involve larger parts of the molecule. It is not easy to predict these vibrations using just a local representation.

Within the harmonic approximation, the sum of the vibrational frequencies is the vibrational zero point energy (ZPE) of the molecules $ZPE= (1/2) \hbar  \sum_i \omega_i $. This quantity represents the ground state vibrational energy of a molecule and can be easily computed from both the calculated and the ML predicted Hessian. In table \ref{tab:ZPEs} we summarize the results for the 13 QM9 molecules analyzed: the ML approximations always underestimate the ZPE with an error which ranges from 1.3\% to 14.5\%.
\begin{table}[h!]
\begin{tabular*}{.95\linewidth} {@{\extracolsep{\fill}}r|c|c|r}
QM9 Index   &   Calculated ZPE & Predicted ZPE & Error \\
\hline
10,000   &   68.2198  &  67.3312  &  1.3 \%  \\
20,000   &   93.9252  &  89.7762  &  4.42 \%  \\
30,000   &   126.745  &  123.158  &  2.83 \%  \\
40,000   &   208.222  &  178.019  &  14.5 \%  \\
50,000   &   122.161  &  108.193  &  11.4 \%  \\
60,000   &   203.131  &  186.773  &  8.05 \%  \\
70,000   &   160.905  &  139.64  &  13.2 \%  \\
80,000   &   170.515  &  162.541  &  4.68 \%  \\
90,000   &   157.959  &  144.444  &  8.56 \%  \\
100,000   &   154.283  &  149.215  &  3.28 \%  \\
110,000   &   167.916  &  155.576  &  7.35 \%  \\
120,000   &   169.058  &  157.315  &  6.95 \%  \\
130,000   &   92.2275  &  88.5746  &  3.96 \%  \\
\end{tabular*}
\label{tab:ZPEs}
\caption[width=\linewidth]{Vibrational ZPE for 13 QM9 molecules whose indices are multiples of 10,000 predicted by ML predictions and calculated explicit with B3LYP-DFT. Energy values are expressed in atomic units (milli Hartree ).}
\end {table}