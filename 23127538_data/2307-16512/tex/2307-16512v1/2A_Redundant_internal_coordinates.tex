\subsection{Redundant internal Coordinates} \label{sect:RICs}

%RICs as initial guess for geommopt  

The choice of internal coordinates is not unique, and many types of internal coordinates have been developed, such as the Z-matrix\cite{baker_z_matrix}, the normal coordinates \cite{sellers1978normal}, the natural internal coordinates developed by Pulay and coworkers \cite{pulay1979systematic,pulay1992geometry,fogarasi1992calculation}, the delocalized internal coordinates developed by Baker \textit{et al.} \cite{baker_delocalized}, and others.\cite{reveles2004_DFT_geomopt}

The redundant internal coordinates introduced by Schlegel \cite{schlegel1982_Optimization}, which we use in this work, describe the geometry of a molecule in terms of bonds, bond angles and dihedrals, which can be easily and uniquely defined.
The number of internal coordinates exceeds the number of degrees of freedom of the molecule. This redundancy, however, guarantees a complete description of molecular distortions.
In the context of geometry optimization, RICs are commonly used to build the initial guess for the Hessian matrix. \cite{schlegel1984_hess_est,Geom_opt_largemols_Schegel,schlegel2011geometry}

\subsubsection{Definition} \label{sect:RIC_defin}

The RICs used in this work are defined as follows \cite{Peng}:
\begin{enumerate} 
    \item A bond $I–J$ is established if the distance between two atoms $I$ and $J$ is shorter than 1.3 times the sum of their covalent radii. 
    \item An angle $I–J–K$ spanned by atoms $I$, $J$ and $K$ is established if atom $I$ and $J$ as well as $J$ and $K$ are bonded and the angle is larger than 45°.  
    \item A dihedral is established from two consecutive angles $I–J–K$ and $J-K-L$. If one of the two angles is 180°, i.e., three atoms lie on a line, they are used as a new base for a dihedral ($I–J=K=L-M$).
\end{enumerate}
Molecules with dihedrals defined using 5 atoms (\textit{e.g.} $ I–J=K=L-M$ ) will be excluded from the dataset of this article, because they generate inconsistencies in the machine learning representation. A separate treatment would be required for them. 
 
%%%%%%%%%%%%%%%%%%%%%%%%%%%%%%%%%%%%%%%%%%%%%%%%%%%%%%
\subsubsection{Derivatives expression}
In this work, Hessians were calculated using DFT in Cartesian coordinates, then transformed to RICs to be used as training data for the ML algorithm.
The ML model predicts Hessians in RICs, which have to be transformed back to Cartesian coordinates, to calculate vibrational frequencies and zero point energies.
In this section we recall the Hessian transformations used here, which follow the formalism developed by Pulay, Fogarasi and Schlegel. \cite{fogarasi1992calculation,pulay1992geometry,Peng,schlegel_1998_vibr_analisis,allen1993_coord_transformations}

For a given set of Cartesian coordinates $x$ and RICs $q$, the Wilson $B$ matrix,\cite{wilson1980molecular, wilson1955molecular} is defined as follows: 
\begin{equation}
    B_{ij} :=\frac{\partial q_i}{\partial x_j}. 
\end{equation}
The transformation of the gradient from internal ($\mathbf{g}_q$) to Cartesian coordinates ($\mathbf{g}_x$) is then given by:
\begin{equation}
    \mathbf{g}_x=B^T \mathbf{g}_q.
\end{equation}
Since in general $B$ is not a square matrix, we define for $B$ a right pseudo-inverse matrix $B_{\rm inv}$, 
\begin{equation}
\begin{aligned}
    B_{\rm inv} &= B^T (B B^T)^{-1}, \\
    B_{\rm inv}^T &= (B B^T)^{-1} B.
\end{aligned}  
\end{equation}
The transformation of the gradient from Cartesian coordinates back to RICs is:
\begin{equation}
    \mathbf{g}_q=B_{\rm inv}^T \mathbf{g}_x. 
\end{equation}
Defining $B'$ as the derivative of $B$ with respect to the Cartesian coordinates, \
\begin{equation}
  B'_{ijk}=\partial^2 q_i /\partial x_j \partial x_k,
\end{equation}
the transformation of the Hessian matrix $H$ from RICs to Cartesian coordinates can be written as:
\begin{equation}
    H_x=B^T H_q B+(B')^T\mathbf{g}_q.
\end{equation} 
Accordingly, the back transformation from Cartesian coordinates to RICs reads
\begin{equation} \label{eq:hess_to_cartesian}
    H_q=B_{inv}^T (H_x-(B')^T\mathbf{g}_q)B_{\rm inv}.
\end{equation}