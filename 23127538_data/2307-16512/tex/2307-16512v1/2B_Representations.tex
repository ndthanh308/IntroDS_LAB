\subsection{Representations}   
 \label{sec:Representations}
It is a convenient choice to predict the Hessian in RICs element by element using a specific local representation for every matrix element of interest.
Since predicting all elements of the Hessian can be a challenging task, we will focus only on some of the most important terms while we set to zero the less relevant elements. 
The diagonal elements are the most important ones and they will be treated in detail, but we note our machine learning method can be extended also to non-diagonal terms (see Sec.\ref{sec:nondiag}).

Every diagonal element of the RIC Hessian corresponds to the second derivative of the energy with respect to an internal coordinate (bond, angle or dihedral).
For each internal coordinate we construct a specific representation describing it and its chemical environment. These representations contain lists of molecular parameters, such as nuclear charges, bond lengths, bond orders, bond angles, and dihedrals. 

Inside the representations, nuclear charges and bond lengths are expressed in atomic units (proton charges and Bohr), bond angles and dihedrals are expressed using the periodic form $1+\text{cos}(\alpha)$, which prevents a discontinuity at $\alpha =\pi$, similarly to previously developed symmetry functions. \cite{dellago_NN_polymorphic,Behler_symmfuns}
Bond orders were calculated using the formula from Ref. \cite{Mayer_bond_order_matrix}, which for a closed shell system has the form
\begin{equation}
  O_{IJ}=\sum_{\mu \in I, \nu \in J} (PS)_{\mu\nu} (PS)_{\nu\mu}
\end{equation}
where he indices $I$ and $J$ refer to atoms, $P$ is the monoelectronic density matrix and $S$ is the atomic orbital overlap matrix.
In order to improve precision, we imposed a model selection, prior to the actual regression, based on two criteria.
The first is a classification based on the type of atoms which define the coordinate, {\textit {e.g.}}, to predict the C-O bond component of the Hessian, we use a model trained only on other C-O bonds.
The second classification is whether the internal coordinate is part a ring system. Different machine learning models are associated with internal coordinates which have or have not two or more atoms belonging to the same ring system (we call them ring coordinates and acyclic coordinates, respectively). 

\subsubsection{Bond-bond elements}
To represent the bond between two atoms $I$ and $J$ correctly, it is important to create a rule to label $I$ and $J$ uniquely. The choice was to label $I$ and $J$ according to their:
\begin{enumerate} 
    \item Nuclear charge:  $Z_I\geq Z_J$.
    \item Hybridization: if $Z_I = Z_J$, $I$ must have an $s$ character higher than $J$.  
    \item Nuclear charges of the bonded atoms: if $I$ and $J$ have the same nuclear charge and hybridization, the atoms bonded to $I$ should have a higher nuclear charge than the ones bonded to $J$.
  \end{enumerate}

The representation is a 26-element vector, which collects the parameters of $I-J$ and the spatial arrangement of the atoms bonded to $I$ and $J$.
The first two elements of the representation are the bond length and order of $I-J$. 
After that, for every atom $I_n$ bonded to $I$ and different from $J$ the following parameters are added:
\begin{enumerate} 
    \item its nuclear charge,
    \item the bonding order with $I$,
    \item the distance from $I$, and
    \item the width of the angle $I_n-I-J$ with $I$ and $J$
\end{enumerate}
Then, the analogous parameters are added or every atom $J_n$ bonded to $J$. The atoms $I_n$ ($J_n$) are included in decreasing order of nuclear charge and bond order with $I$ ($J$). 
Since molecules in the QM7 dataset do not have atoms with more than 4 bonds, $n$ has values form 1 to 3; if $I$ or $J$ have less than 4 bonded atoms, null values are inserted to maintain the same representation size.
% Figure environment removed

%%%%%%%%%%%%%%%%%%%%%%%%%  Angles   %%%%%%%%%%%%%%%%%%%%%%%%

\subsubsection{Angle-angle elements}

In order to give a unique representation to the angle $I-J-K$, the atoms $I$ and $K$ need to be labelled uniquely according to the following criteria:
\begin{enumerate}
    \item Nuclear charge:  $Z_I\geq Z_K$
    \item Bond order with $J$: ${O}_{IJ}>{O}_{JK} $
    \item Hybridization:  $I$ must have an $s$ character higher than $K$ 
    \item Nuclear charges of the bonded atoms: the atoms bonded to $I$ should have a higher nuclear charge than the ones bonded to $K$
\end{enumerate}

Each angle representation consists of 46 elements. The first elements are the magnitude of the angle, the bond lengths $I-J$, $J-K$, and the bond orders $O_{IJ}$,$O_{JK}$,$O_{IK}$.
After these six parameters, for every atom $I_n$,$J_n$,$K_n$ bonded to $I$,$J$, or $K$, respectively, the several other quantities are listed. More specifically, for an atom $I_n$ bonded to atom $I$ (and different from $J$) the list include
\begin{enumerate}
    \item the nuclear charge of $I_n$
    \item the bond order of $I-I_n$
    \item the distance $I-I_n$
    \item the angle $J-I-I_n$ 
    \item the dihedral $K-J-I-I_n$ 
\end{enumerate}
%For an atom $J_n$ bonded to $J5$, the quantities are
%\begin{enumerate}
%    \item the nuclear charge of $J_n$
%    \item the bond order of $J_n-J$
%    \item the distance $J_n-J$
%    \item the angle $J_n-J-I$ 
%    \item the dihedral $J_n-J-I-K$ 
%\end{enumerate}
For an atom $J_n$ bonded to $J5$ (different from $I$,$K$), the quantities are: the nuclear charge of $J_n$,the bond order of $J-J_n$,the distance $J-J_n$,the angle $K-J-J_n$ ,the dihedral $I-J-K-J_n$ .

%Finally, for an atom $K_n$ bonded to $K$, one adds
%\begin{enumerate}
%    \item the nuclear charge of $K_n$
%    \item the bond order of $K_n-K$
%    \item the distance $K_n-K$
%    \item the angle $K_n-K-J$ 
%    \item the dihedral $K_n-K-J-I$ 
%\end{enumerate}

Finally, for an atom $K_n$ bonded to $K$ (different from $J$), are added to the representation: the nuclear charge of $K_n$, the bond order of $K-K_n$, the distance $K-K_n$, the angle $J-K-K_n$, and the dihedral $I-J-K-K_n$. 
Since each atom has forms no more than four bonds, there are at most eight atoms bonded to $I$, $J$ or $K$ and in total the list includes at most 40 elements, which together with the previous 6 elements gives a representation of length 46. (Note, that if there are less then 8 atoms bonded to $I$, $J$, or $K$ the remaining elements are filled with zeros.) 

If $K$ is a hydrogen atom, it does not have any bonded atoms ($K_n$) other than $J$ and the representation defined above would have many zeros. To compensate for this lack of descriptors, we include the five parameters of the lists above for the atoms $J1_n$ bonded to $J_1$ as shown in the second panel of Fig \ref{fig:angle_representation}. If both $K$ and $I$ are hydrogens, we include the five parameters of the lists above for the atoms $J1_n$ bonded to $J_1$, and also for the $J2_n$ bonded to $J_2$.

% Figure environment removed


%%%%%%%%%%%%%%%%%%%%%%%%   Dihedrals     %%%%%%%%%%%%%%%%%%%%%%%%%%%%%%%%%%

\subsubsection{Dihedral-dihedral elements}
The representation of a dihedral, used to predict the diagonal dihedral-dihedral terms of the Hessian, was built in a similar way to what was done for bonds and angles.
The label ordering of the dihedral $I-J-K-L$ was chosen such that $Z_J \ge Z_K$, and if $Z_J = Z_K$, than $Z_I \ge Z_L$. In case of a symmetrical atom arrangement, where $Z_J = Z_K$ and $Z_I = Z_L$, it is imposed that the bond orders $O_{IJ}>O_{KL}$.
The representation for a dihedral consists 65 elements.
The magnitude of the dihedral, the bond lengths and bond orders for the bonds $I-J$,$J-K$ and $K-L$, as well as the angles  $I-J-K$ and $J-K-L$ constitute the first 9 elements. 
Then other 6 elements describe how and if the dihedral is part of a ring system.
Finally, 50 more elements are given by all the up to 10 atoms bonded to $I,J,K$ or $L$, of which we list nuclear charges, the orders and lengths of the bonds to $I,J,K$ or $L$, the angles and the dihedrals formed with $I,J,K$ or $L$.

 For further clarification on the definition of the representations the reader is encouraged to consult the code and the dataset published on Zenodo\cite{domenichini_dataset_hessian}.