\subsection{Cross validation of the Hessian's diagonal terms} \label{sec:CV}
The dataset for training and testing was built from the 6810 QM7 molecules, whose dihedrals are defined only by consecutive atoms (Section \ref{sect:RIC_defin}). The structures of this subset were relaxed and the Hessians were calculated at the B3LYP/cc-pVDZ level of theory. 
The molecules were then distributed randomly between training set and the test set. Since internal coordinates of the same molecule are in general not independent quantities, it is important that the train-test splits are performed on molecules and not on individual coordinates, so that coordinates of the same molecule either all belong to the training set, or they all belong to the test set.
The internal coordinates were classified according to their chemical composition and to the inclusion in a ring system (as explained in section \ref{sec:Representations}).
For each of these groups, different machine-learning models were trained independently.

% Figure environment removed

The learning can be visualized as the decrease of the prediction error with increasing training set size. 

Focusing on some of the most frequently occurring RICs in the training set (acyclic C-C bonds, C-C-C angles, and C-C-C-C dihedrals), in Figure \ref{fig:learning_curves} we show the Mean Absolute Percentage Error (MAPE) for the predictions of the Hessian elements as a function of the number of molecules in the training set.

In an ideal machine learning model the learning curves \cite{learningcurves_bing,learningcurves_krr_muller} should exhibit, in a log-log plot, a linearly decaying trend. In our cases, this linear trend is approximated very well for bonds; for angles and dihedrals there is a small deviation, but still learning is achieved throughout the chosen training set. 

% Figure environment removed

Diagonal elements of the Hessian predicted by the machine learning models for all bonds, angle and dihedrals are compared to the reference data in Figure \ref{fig:Diag_Scatter}.
As can be inferred from the figure, the accuracy is higher for bonds than for angles and and particularly for dihedrals. 
For bond derivatives the only outlier is a single C-C bond inside a conjugated atom chain. The reason is that the machine learning model treats this bond as a single bond while the true value of this Hessian element is closer to that of a double bond. 

% Figure environment removed

The performance of our method was estimated for all coordinates and all atomic element combinations within our QM7 dataset. The MAPEs resulting from ten train-test splits are summarized in Figure \ref{fig:Bar_mape}. 

The prediction error is around 2\% for bond-bond, around 5\% for angle-angle, and around 10\% for dihedral-dihedral terms.
Bond-bond Hessian elements can be predicted more accurately, probably because bond stretching is a more localized motion than angle scissoring or dihedral torsion, which can be associated with wider molecular motions and may depend more strongly on the non-bonded atomic surroundings.  
For similar reasons, the most accurate predictions are those for bonds between hydrogens and a heavier element (C, N, O), the MAPEs of which are below 0.2\%. 

Hessian elements related to C-N, C-C, and C-O bonds are also well predicted (with an error of about 0.5\% if not inside a ring), because they occur frequently in the training set.

On the contrary, the biggest errors were made for S-C, O-N, and N-N bonds, because their occurrence in the training set is limited.

Also for angles and dihedrals, those including hydrogen atoms have the lowest prediction errors. In fact, the rotation of a hydrogen atom is usually not correlated with other molecular motions and thus it is more consistently repeated in chemical space. The highest errors were obtained for dihedrals and angles which have a limited number of occurrences in the dataset, mostly involving S, O, and N angles.

% correct 7-7-23
