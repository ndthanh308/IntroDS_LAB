% Hessian is important
The molecular Hessian matrix, namely the matrix of second derivatives of the potential energy with respect to nuclear positions, has an important role in quantum chemistry. It describes the curvature of the Potential Energy Surface (PES), and the knowledge of it is important for geometry relaxation as well as for the determination of transition states\cite{schlegel2011geometry,reveles2004_DFT_geomopt}.
In the context of molecular geometry optimization, all quasi-Newton methods require the knowledge of an approximate Hessian matrix, as an inexpensive initial guess for the first step.
The knowledge of the Hessian matrix also helps to interpret infrared (IR) spectra of molecules: in the harmonic approximation, the vibrational frequencies and the vibrational normal modes are calculated from the eigenvalues and eigenvectors of the mass-weighted Hessian matrix \cite{pople_vibrations,Wilson1941_mwh,MCMURRY1959203_mwh}.

% Vibrations are local 
In general, molecular vibrations are motions of the whole molecule, but some normal modes are strongly localized on particular functional groups. For instance, bond stretching and some angle bending modes often involve only few atoms. Dihedral torsions, on the other hand, are more collective in character and are often mixed with other degrees of freedom in low-frequency collective mode, making it more difficult to identify them. The frequencies of localized vibrations depend on the atom species involved, on the bond order which connects the atoms, and to a smaller extent on the chemical environment. These frequency changes very little among different molecules, and tabulated values are available for them\cite{McMurry1952_correlation_IR}.

%the hessian is approximated in internal coordinates 

The \textit{ab initio} calculation of the Hessian is computationally expensive and therefore several methods to approximate the Hessian have been proposed.
For geometry optimization, the Hessian is often calculated using approximate quantum mechanical methods \cite{HEAD1986359,pople1988_optimization_HF_hessian, fischer_almlof1992_geomopt} or approaches based on force fields expressed in internal coordinates \cite{schlegel1984_hess_est,schlegel2005_hessian_update,LINDH1995423}.
%or even on empirical values from spectroscopy data.\cite{sellers1978normal}
To approximate the Hessian matrix in internal coordinates is particularly convenient, because they are rotationally and translationally invariant, and because they are less coupled than Cartesian coordinates (the Cartesian Hessian may have many non-diagonal elements comparable in magnitude with the diagonal ones).


%Machine learning 

An alternative way to compute an approximate Hessian is to use Machine Learning (ML), which is currently having a significant impact in many ares of 
computational chemistry.\cite{qml_ccs_anatole2018, qml_nutshell_rupp2015}
Many molecular properties can be predicted directly \cite{von2020exploring,qml_rupp_atomization, faber_qml_lower_DFT,christensen2019operators,weinreich2021machine,qml_properties,qml_optimization_hammer, qml_lemm2021energy, Keith2021QML} from ML models and recent works showed that it is possible to predict the shape of a PES using Redundant Internal Coordinates (RICs). \cite{mancini2020unsupervised,falbo2022integration} 
A common way to calculate the Hessian with ML methods is by differentiation of the potential energy expressed using kernel in Kernel Ridge Regression (KRR) models \cite{response_inccs,von2020thousands,heinen2021toward,Heinen2022_transition}, or neural networks. \cite{Westermayr2021_ml_exctstates} However, recent methods have been developed for the direct prediction of vibrational spectra \cite{Zhang2020_vibrational_nn,Han2022_MLvibrations_review}, and Hessian matrices, but not across chemical space . \cite{ML_DCVS_Gandolfi_Ceotto, ML_Ngas_Gandolfi_Ceotto}

% Paper's outline
 
In this paper we present an alternative ML approach to predict directly, element by element, the Hessian matrix in RICs, across chemical compound space.\cite{ccs} The central idea of the method is to learn the elements of the Hessian using a random forest of decision trees. Trained on Hessians computed quantum mechanically for subset of small molecules (up to 7 non-hydrogen atoms) extracted from the QM7 dataset\cite{QM7_blum,QM7_rupp}, the model succeeds in predicting vibrational frequencies and Zero point vibrational energies (ZPE) for some larger molecules, sampled from the QM9 \cite{QM9_Ruddigkeit,QM9_ramakrishnan2014quantum} dataset (molecules up to 9 non-hydrogen atoms).

The remainder of this article is organized as follows. In Section II we briefly summarize the definitions of RICs, and the transformations of the Hessian between coordinate systems. We also present a new set of coordinate-specific representations that can be used to train a Random Forest Regression (RFR) model for the diagonal elements of the Hessian matrix. In Section III we validate the prediction for the diagonal Hessian elements, as well as for some of the most important non-diagonal elements. Some conclusions are provided in Section IV.







