%chktex-file 8
\documentclass[a4paper,titlepage,runningheads]{llncs}
\usepackage{hyperref}
\hypersetup{%
    hidelinks,
    pdftitle={SusTrainable: Promoting Sustainability as a Fundamental Driver in Software Development Training and Education},
    pdfauthor={Tihana Galinac Grbac, Csaba Szab\'o and João Paulo Fernandes (Eds.)},
    pdfsubject={2nd Teacher Training Proceedings},
}

\usepackage{url}
%\usepackage[subtle]{savetrees}
%\usepackage[margin=2cm]{geometry}
\usepackage{tikz,amsmath, amssymb,bm,color, amsthm,amsfonts}
\usetikzlibrary{positioning, calc,chains,fit,shapes}
%\usetikzlibrary{circuits.logic.US,circuits.logic.IEC,fit}
\usepackage{enumerate}
\usepackage{comment}
\usepackage{tikz}
\usepackage{graphics}
%\usepackage[cm]{fullpage}
\usepackage{longtable}
\usepackage{mdframed}
\usepackage{caption}
\usepackage{subcaption}
\usepackage{slashbox}
\usepackage{url}
\usepackage{framed}
\usepackage{array}
\usepackage{tabu}
\usepackage{lscape}
\usepackage{multirow}
\usepackage{ulem}
\usepackage{multicol}
\usepackage{placeins}
\usepackage{cite}
\usepackage{enumitem}
\usepackage{mathtools}
%\usepackage[numbers]{natbib}
%\usepackage{mathtools}
%\usepackage{authblk}

\mdfsetup{skipabove=2pt,skipbelow=2pt}
%\setlenght {\marginparwidth }{2cm}
%\usepackage{todonotes}

%\usepackage{floatrow}
%\usepackage{adjustbox}
%\setlength{\extrarowheight}{.05ex}
%\renewcommand\thesubfigure{\roman{subfigure}}


%\newtheorem{theorem}{Theorem}[section]
%\newtheorem{lemma}[theorem]{Lemma}
%\newtheorem{observation}[theorem]{Observation}
%\newtheorem{corollary}[theorem]{Corollary}
%\newtheorem{proposition}[theorem]{Proposition}
%\newtheorem{definition}[theorem]{Definition}
\newtheorem{construction}{Construction}
%\newtheorem{conjecture}{Conjecture}
%\newtheorem{remark}[theorem]{Remark}

\newcommand{\pname}[1]{\textnormal{\textsc{#1}}}
\newcommand{\cclass}[1]{\textnormal{\textsf{#1}}}
\newcommand{\nog}{nine} % no of members in the gang!
\newcommand{\nogd}{nineteen} % no of members in the gang - for deletion/completion
\newcommand{\nogl}{eighteen} % no of members in the larger gang - for editing
\newcommand{\nogld}{thirty eight} % no of members in the larger gang - for deletion/completion
\newcommand{\diffnog}{ten} %
%\newcommand{\dominatedby}{dominated by} %
%\newcommand{\dominatingset}{dominating set} %
%\newcommand{\dominates}{dominates} %
\newcommand{\simulates}{simulates} %
\newcommand{\baseset}{base} %
\newcommand{\issimulatedby}{is simulated by} %

\newcommand{\StarSAT}{\pname{8-SAT$_{\geq 6}$}}
\newcommand{\FSAT}{\pname{4-SAT$_{\geq 2}$}}
\newcommand{\FISAT}{\pname{5-SAT$_{\geq 3}$}}
\newcommand{\SIXSAT}{\pname{6-SAT$_{\geq 4}$}}
\newcommand{\ESAT}{\pname{8-SAT$_{\geq 6}$}}
\newcommand{\KSAT}{\pname{$k$-SAT$_{\geq {k-2}}$}}
\newcommand{\KSATO}{\pname{$k$-SAT}}
\newcommand{\ESATO}{\pname{8-SAT}}
\newcommand{\FSATO}{\pname{4-SAT}}
\newcommand{\FISATO}{\pname{5-SAT}}
\newcommand{\TSAT}{\pname{3-SAT}}
\newcommand{\HED}{\pname{${H}$-free Edge Deletion}}
\newcommand{\AEE}{\pname{${A}$-free Edge Editing}}
\newcommand{\AED}{\pname{${A}$-free Edge Deletion}}
\newcommand{\TSED}{\pname{$t$-star-free Edge Deletion}}
\newcommand{\ATSED}{\pname{Annotated $t$-star-free Edge Deletion}}
\newcommand{\AFSED}{\pname{Annotated $4$-star-free Edge Deletion}}
\newcommand{\FSED}{\pname{$4$-star-free Edge Deletion}}
\newcommand{\FVSED}{\pname{$5$-star-free Edge Deletion}}
\newcommand{\HEE}{\pname{${H}$-free Edge Editing}}
\newcommand{\HEC}{\pname{${H}$-free Edge Completion}}
\newcommand{\HDEE}{\pname{${H'}$-free Edge Editing}}
\newcommand{\HDDEE}{\pname{${H''}$-free Edge Editing}}
\newcommand{\HDED}{\pname{${H'}$-free Edge Deletion}}
\newcommand{\HDEC}{\pname{${H'}$-free Edge Completion}}
\newcommand{\HBEE}{\pname{${\overline{H}}$-free Edge Editing}}
\newcommand{\HBED}{\pname{${\overline{H}}$-free Edge Deletion}}
\newcommand{\HBEC}{\pname{${\overline{H}}$-free Edge Completion}}
\newcommand{\HOEDCE}{\pname{${H_1}$-free Edge Deletion(Completion/Editing)}}
\newcommand{\HEDCE}{\pname{${H}$-free Edge Deletion(Completion/Editing)}}
\newcommand{\HEEDC}{\pname{${H}$-free Edge Editing(Deletion/Completion)}}
\newcommand{\HDEEDC}{\pname{${H'}$-free Edge Editing(Deletion/Completion)}}
\newcommand{\BFED}{\pname{Bow-free Edge Deletion}}
\newcommand{\ABFED}{\pname{Annotated Bow-free Edge Deletion}}
\newcommand{\DTIS}{\pname{Distance-3 Independent Set}}
\newcommand{\SVC}{\pname{Strong Vertex Cover}}
\newcommand{\CLIQUE}{\pname{Clique}}
\newcommand{\IS}{\pname{Independent Set}}
\newcommand{\PFS}{\pname{Propagational-$f$ Satisfiability}}
\newcommand{\RHED}{\pname{Restricted ${H}$-free Edge Deletion}}
\newcommand{\RHEC}{\pname{Restricted ${H}$-free Edge Completion}}
\newcommand{\RHDED}{\pname{Restricted ${H'}$-free Edge Deletion}}
\newcommand{\RHDEC}{\pname{Restricted ${H'}$-free Edge Completion}}
\newcommand{\RHEE}{\pname{Restricted ${H}$-free Edge Editing}}
\newcommand{\PH}{$\cclass{NP} \subseteq \cclass{coNP/poly}$}
\newcommand{\NOPH}{$\cclass{NP} \not\subseteq \cclass{coNP/poly}$}
\newcommand{\LG}{\mathcal{W}}
\newcommand{\LGD}{\mathcal{W}'}
\newcommand{\LGDD}{\mathcal{W}''}


%\let\oldvee\vee
\renewcommand\vee{\boxtimes}

\newcommand\addvmargin[1]{
  \node[fit=(current bounding box),inner ysep=#1,inner xsep=0]{};
}
\setlength{\fboxrule}{0pt}

\newcommand{\defstage}[2]{% PGD Version
  \hfill\\\smallskip\noindent%
  \begin{tabularx}{\textwidth}{|l X|}%
    \hline%
    \multicolumn{2}{|l|}{\textbf{#1}}\\%
    &#2\\\hline%
  \end{tabularx}%
%  \smallskip%
}
\setlength\extrarowheight{15pt}

\newcounter{rowcntr}[table]
\renewcommand{\therowcntr}{\thetable.\arabic{rowcntr}}

% A new columntype to apply automatic stepping
\newcolumntype{N}{>{\refstepcounter{rowcntr}\therowcntr}c}

% Reset the rowcntr counter at each new tabular
\AtBeginEnvironment{longtabu}{\setcounter{rowcntr}{0}}

\newcounter{rowcntra}[table]
\renewcommand{\therowcntra}{\arabic{rowcntra}}

% A new columntype to apply automatic stepping
\newcolumntype{M}{>{\refstepcounter{rowcntra}\therowcntra}c}

% Reset the rowcntr counter at each new tabular
\AtBeginEnvironment{tabular}{\setcounter{rowcntra}{0}}

\newcommand{\NPC}{NP-Complete}


\newcommand{\highlight}[1]{\textcolor{blue}{#1}}
\newcommand{\dhanya}[1]{\textcolor{blue}{dhanya: #1}}


%\newcommand{\XCD1}[1]{\pname{$\chi_{cd}$\ensuremath{(#1)}}}
\newcommand{\XCD}{\pname{$\chi_{cd}$}}
\newcommand{\SC}{\pname{$\omega_{s}$}}

\newcommand{\CDC}{\textsc{CD-coloring}}
\newcommand{\SCP}{\textsc{Separated-Cluster}}
\newcommand{\TD}{\textsc{Total Domination}}
\newcommand{\ISP}{\textsc{Independent Set}}
\newcommand{\CC}{\textsc{Clique Cover}}
\newcommand{\TETHS}{Further, the problem cannot be solved in time \ensuremath{2^{o(|V(G)|)}}, unless the ETH fails}
%\usetikzlibrary{positioning,chains,shapes,calc}
\usetikzlibrary{fit}
\thispagestyle{empty}
\usetikzlibrary{
  graphs,
  graphs.standard
}
\renewcommand{\orcidID}[1]{\href{https://orcid.org/#1}{\textsuperscript{% Figure removed}}}

\bibliographystyle{splncs04}

\title{
SusTrainable: Promoting Sustainability as a Fundamental Driver in Software Development Training and Education\\[2ex]
2nd Teacher Training\\
January 23--27, 2023, Juraj Dobrila University of Pula, Pula, Croatia\\
Revised lecture notes
}
\author{Tihana Galinac Grbac, Csaba Szab\'o and João Paulo Fernandes (Eds.)}
\date{June 2023}

\begin{document}

%\pagestyle{empty}

%Cannot use maketitle because llncs overrides that
\begin{titlepage}
  \thispagestyle{empty}
  \let\footnotesize\small
  \let\footnoterule\relax
  \let \footnote \thanks
  \null\vfil
  \vskip 60pt
  \begin{center}%
    {\LARGE
        SusTrainable: Promoting Sustainability as a Fundamental Driver in Software Development Training and Education\\[2ex]
        \large
        2nd Teacher Training\\
        January 23--27, 2023, Juraj Dobrila University of Pula, \\
        Pula, Croatia\\
        Revised lecture notes
    \par}%
    \vskip 3em%
    {\large
     \lineskip .75em%
      \begin{tabular}[t]{c}%
        Tihana Galinac Grbac\and Csaba Szab\'o\and Jo\~ao Paulo Fernandes (Eds.)
      \end{tabular}\par}%
      \vskip 1.5em%
    {\large June 2023 \par}%       % Set date in \large size.
  \end{center}\par
  \vfil\null
\end{titlepage}

\thispagestyle{empty}

\frontmatter

\chapter*{Preface}
\markboth{Preface}{Preface}

This volume exhibits the revised lecture notes of the second teacher training organized
as part of the project \emph{Promoting Sustainability as a Fundamental Driver in Software Development Training and Education},
and held at the Juraj Dobrila University of Pula, Croatia, in the week January 23--27, 2023.
The project is the Erasmus+ project \emph{No. 2020{-}1{-}PT01{-}KA203{-}078646 --- Sustrainable}.
More details can be found at the official project web site

\begin{center}
\url{https://sustrainable.github.io/}
\end{center}

\noindent but for convenience of the reader, the brief description of the project is quoted as follows.

``Sustainability as a key driver for the development of modern society and the future of the planet has never achieved as much consensus worldwide as today.

Ultimately, it is the hardware of ICT systems that consumes energy, but it is software that controls this hardware. Thus, controlling the software is crucial to reduce the ever-growing energy footprint of ICT systems. The Sustrainable project advocates the introduction of all facets of sustainability as a primary concern into software engineering practice.

This project aims to actively promote educating the next generation of software engineers to consider sustainability in all aspects of the software engineering process: SusTrainable means Training for Sustainability. We aim to provide future software engineers with essential skills to develop software that is not only functionally correct, but also easy to maintain and evolve, that is durable, has a low impact and uses the hardware it is running on in the most energy-efficient way.

Our objective is to train the future avant-garde software engineers for the sustainable software and ICT that the knowledge-based, environmentally concerned societies of the 21st century demand. The majority of summer school participants are close to their transition from learning to doing and will soon join the European software engineering workforce. They will carry the ideas, concepts and methods they learned at the summer schools into industrial software engineering practice across Europe and worldwide. Furthermore, they will act as facilitators and multipliers alike for a sustainable future software-driven Europe. Conversely, we provide the ubiquitous enthusiasm of the younger generation for sustainable development a constructive, meaningful and high-impact route forward.

For this purpose we assemble a broad and diverse consortium of researchers and educators from 10 selected universities and 7 countries from across Europe.''

\bigskip

One of the most important contributions of the project are two summer schools. The 2nd SusTrainable Summer School (SusTrainable -- 23) will be organized
at the University of Coimbra, Portugal, in the week July 10--14, 2023. The summer school
will consist of lectures and practical work for master and PhD students in the area of computing science and closely related fields.
There will be contributions from
Babe\textcommabelow{s}-Bolyai University,
E\"{o}tv\"{o}s Lor\'{a}nd University,
Juraj Dobrila University of Pula,
Radboud University Nijmegen,
Roskilde University,
Technical University of Ko\v{s}ice,
University of Amsterdam,
University of Coimbra,
University of Minho,
University of Plovdiv,
University of Porto,
University of Rijeka.

To prepare and streamline the summer school, the consortium organized a teacher training in Pula, Croatia.
This was an event of five full days in the week January 23--27, 2023, organized by Tihana Galinac Grbac and Neven Grbac at the Juraj Dobrila University of Pula, Croatia.
The Juraj Dobrila University of Pula is very concerned with the sustainability issues. The education, research and management is conducted with sustainability goals in mind. The university
participates in the UI GreenMetric World University Rankings.

The goal of this teacher training is to improve the efficiency and effectiveness of all the educational activities at the forthcoming
2nd SusTrainable Summer School at the University of Coimbra.

An effective and efficient summer school consists of a series of educational and teaching activities by a number of invited researchers and/or educators. It is important that these
activities are of high quality, but also that they are not totally unrelated and independent. The thing is that a summer school is usually very demanding for students, thus, a well
balanced combination of forms of activities and teaching methodologies would improve students' satisfaction with the knowledge and techniques acquired at the school. Besides that, it
is very convenient that different activities explain their part at the same or similar examples, the tasks are related and different aspects well explained.

At the teacher training, all the proposed summer school teachers are asked to present their plan for the activities at the summer school. This includes an outline and content for their series of
activities (course), the teaching methodology and the form of activities, prerequisites assumed the students will be familiar with (including their expected previous knowledge, practical
skills, programming tools).
The final outcome of the training should be a detailed plan for each teacher's series of activities at the summer school, orchestrated with all other teachers in the sense explained
above. The plans should include the list of prerequisites required by the students (with references and possibly some pre-school reading assignments), the required hardware and
software for the summer school (for the organizers), and an overview of the lecture material, examples and student assignments.


As in the first teacher training, the contents of the planned summer school lectures were discussed and fine-tuned.
The topics discussed in regard of each contribution are the following:
\begin{itemize}
\item
    the contents of the intended lecture, but not the entire lecture itself;
\item
    what is the intended audience;
\item
    what is the required knowledge and skills of the audience;
\item
    what are required materials (hardware and software) for the exercises;
\item
    what will students know after this lecture and can this be used in subsequent lectures;
\item
    what is the relation with other contributions, other dependencies and what are the lessons learned during the teacher training.
\end{itemize}


Apart from the discussion regarding the contributions of the summer school and other outcomes of the teacher training mentioned above and listed in the proceedings,
there were invited lectures focused on best practices in teaching IT-related topics in view of sustainability goals of the summer school:
\begin{itemize}
\item
    Dr.\ Patricia Lago from the Vrije Universiteit Amsterdam: \emph{VU Master-level education in software engineering, green IT, and digital sustainability}.
\item
    Dr.\ Darko Etinger and Dr.\ Sini\v{s}a Mili\v{c}i\'{c} from the Juraj Dobrila University of Pula: \emph{Virtual study of informatics: experiences and best practices}.
\end{itemize}
Those presentations provided useful background knowledge for the teaching team.
See

\begin{center}
\url{https://www.unipu.hr/en/international-cooperation/projects/sustrainable/teacher_training_pula}
\end{center}

\noindent
for the complete schedule of the teacher training.

The contributions in the proceedings were reviewed and provide a good overview of the range of topics that will be covered at the summer school.
The papers in the proceedings, as well as the very constructive and cooperative teacher training, guarantee the highest quality and
beneficial summer school for all participants. We are looking forward to contributing to a sustainable future.

\medskip

\begin{flushright}
\noindent June 2023\hfill
{Tihana Galinac Grbac\footnote{Juraj Dobrila University of Pula, Croatia}}\\
{Csaba Szab\'o\footnote{Technical University of Ko\v{s}ice, Slovakia}}\\
{Jo\~ao Paulo Fernandes\footnote{University of Porto, Portugal}}\\
\end{flushright}


\mainmatter%

\tableofcontents



%
% \title{Code optimization by matrix operations in data mining}
\title{Code optimization with vectorization in data mining and machine learning}
%
\titlerunning{Code optimization for sustainable data mining}
% If the paper title is too long for the running head, you can set
% an abbreviated paper title here
%
% \author{Lehel Csató\inst{1}\orcidID{0000-0002-4857-878X} \and
% Zalán Bodó\inst{1}\orcidID{0000-0003-1052-1849}
% }
\author{%
	Zalán Bodó\orcidID{0000-0003-1052-1849}  %
	\and
	Lehel Csató\orcidID{0000-0002-4857-878X} %
}
%
\authorrunning{Z. Bodó, L. Csató}
% First names are abbreviated in the running head.
% If there are more than two authors, 'et al.' is used.
%
\institute{Faculty of Mathematics and Computer Science, Babeș-Bolyai University,\\
str. M. Kogălniceanu, nr. 1, RO-400084, Cluj\\
\email{\{zalan.bodo,lehel.csato\}@ubbcluj.ro}
}
%
\maketitle              % typeset the header of the contribution
%
\begin{abstract}
%
% The abstract should briefly summarize the contents of the paper in
% 15--250 words.
%
Interpreted languages are preferred over compiled languages when considering scientific computing -- in this article, we focus on the MATLAB, Python, and Julia languages.
%
Whilst running interpreted code usually means longer runtime, the source code for interpreted languages is usually more compact and easier to understand.
%
A second significant advantage of interpreted languages is that they are easier to learn.
%
These languages, moreover, often provide built-in constructs or third-party libraries for efficient numerical computations.
%
In this course, we give a short introduction to basic machine learning algorithms and show how to implement these algorithms with vectorization in the different languages and modules -- with the scope of creating a more efficient and more understandable code.
%
Besides vectorization, the need for optimized data structures in scientific computing will also be discussed.

\keywords{Code optimization  \and Vectorization \and Data mining.}
\end{abstract}
%

%%%%%%%%%%%%%%%%%%%%%%%%%%%%%%%%%
\section{Contents of the lecture}

The motivation of the lecture is the rise in popularity -- and indeed in the added value towards the teaching process -- of the interpreted languages, specifically when the focus of the application is an area related to data analysis, machine learning or -- more broadly -- artificial intelligence.
%
These languages are easier to code, easier to debug, but they are inefficient, for example, when executing loops, typically used when the same operations have to be applied to several elements -- mostly due to the higher level of abstraction due to the presence of the virtual machine that interprets the code \cite{birkbeck2007dimension}.\footnote{Loops are highlighted here because in vectorization we actually optimize cycles, but, of course, this inefficiency applies to almost all language constructs.}
%
In this respect, the compiled versions of the same code -- evidently written in a compiled language -- is at least one order of magnitude faster \cite{aho2007compilers}.

When numerical computations are required, however, some of these languages offer built-in features or libraries to support these.
%
For example, in MATLAB or more recently in the Julia language~\cite{bezanson2017julia,Sherrington2015}, the initial type for all variables is the multidimensional array\footnote{\url{https://www.mathworks.com/help/matlab/learn_matlab/matrices-and-arrays.html}}, and these languages provide specialized operations to make computations on these data structures more efficient than those realized by programmers and using loops.
%
Similarly, Python's NumPy\footnote{\url{https://numpy.org/}} package offers functions to perform optimized matrix computations \cite{van2011numpy}.

In this course, we will discuss the different approaches/methods to rewrite numerical computations using matrices to optimize the program code, if such a rewriting is possible/efficient.
%
This procedure is called \emph{vectorization}.
%
Besides vectorization, the need for using optimized data structures -- e.g. sparse vectors/matrices -- will also be discussed.
% During the lecture we will illustrate the ``classical'' computation of resulting vectors in python, then will switch to the transcription to the

For the sake of example, we consider the problem of computing the Euclidean distance matrix of a set of items in a dataset -- usually needed to calculate nearest neighbors, the RBF kernel matrix when applying kernel methods \cite{scholkopf2018learning}, or in simple clustering algorithms -- i.e. we need to compute a matrix $\mathbf{D}$ of size $N\times N$, containing the pairwise Euclidean distances over data items:
$$
    D_{i,j} = \|\mathbf{x}_{i} - \mathbf{x}_{j}\|_{2}^{2}, \quad i,j\in \{1,2,\ldots,N\}
$$
Suppose the data $\mathbf{x}_{i}\in \mathbb{R}^d$, $i=1,2,\ldots,N$ -- represented as \emph{column} vectors -- is arranged into the $N\times d$ matrix $\mathbf{X}$, as below (left), then with only matrix operations we can compute $\mathbf{D}$ as below (right):
\begin{equation*}
  \mathbf{X} = \begin{bmatrix}
        \mathbf{x}_{1}^{T}\\
        \mathbf{x}_{2}^{T}\\
        \vdots\\
        \mathbf{x}_{N}^{T}
        \end{bmatrix}
  \qquad\qquad\qquad\qquad
  \begin{array}{ll}
    \mathbf{A} = \mathbf{X}\mathbf{X}^{T}\\
    \mathbf{B} = \diag(\diag(\mathbf{A}))\\
    \mathbf{D} = -2\mathbf{X}\mathbf{X}^{T} + \mathbf{B}\mathbf{1}_{NN} + \mathbf{1}_{NN}\mathbf{B},\\
  \end{array}
\end{equation*}
where $\diag(\cdot)$ is the diagonal function, implemented under the same name -- \verb!diag! -- in MATLAB/Octave\footnote{\url{https://www.mathworks.com/help/matlab/ref/diag.html}} which either keeps only the diagonal elements of a matrix -- the inner \verb!diag! function --, or makes a diagonal matrix out of a vector of elements -- the outer function\footnote{In the language Julia there is a distinction: \texttt{diag} takes a matrix and returns the vector of diagonals, for the other direction one uses \texttt{diagm}.}, and $\mathbf{1}_{NN}$ is the matrix containing all ones.

The above simplification of computing the pairwise distances highlights (1) the simplicity of the code -- which is extremely close to the mathematical notations in MATLAB and Julia -- and (2) the efficiency of the implementation, which originates in the optimized versions of the matrix multiplication and the matrix notation itself.
%
Using vectors and matrices in these methods allows the straightforward application of some formulae for more efficient computations when needed (e.g. Searle's set of identities) \cite{matrix_cookbook,golub2013matrix}.

We will explore the approaches to vectorization within the data processing and data analysis methods to
compute dot products, matrix products, weighted averages, etc., and use these in different algorithms.
The examples will be written for real-world data in order to keep the audience motivated.

%%%%%%%%%%%%%%%%%%%%%%%%%%%%%%%%%
\section{Intended audience}

The target audience of this course includes everybody who is interested in data mining, applied mathematics, numerical computing, and mathematical modeling within data analysis.
%
In the present, it is inevitable for someone to encounter applications of artificial intelligence, more specifically machine learning methods, since these can be found almost everywhere, from spam filtering, through intelligent image enhancement, to machine translation -- just to name a few.

Consequently, we recommend the course for everyone who is interested in artificial intelligence; specifically in understanding machine learning models and the crux of deep learning methodology.
%
They will acquire knowledge and skills that will help a quicker and ``deeper'' understanding of machine learning methods and their implementation details.
%
The use of vectorization and related concepts will help present and future PhD students improve the quality of their code and produce a more \emph{sustainable} code basis -- in every aspect of the word.
%
Sustainable coding/programming -- from the perspective of scientific computing -- reduces to the following three things: (i) using interpreted languages facilitating fast development and easy debugging; (ii) using optimized data representations, data structures; (iii) exploiting the benefits of vectorization.

%%%%%%%%%%%%%%%%%%%%%%%%%%%%%%%%%
\section{Prerequired knowledge}

The required knowledge to be able to follow the present course efficiently can be summarized as follows:
\begin{itemize}
    \item basic linear algebra knowledge (linear equations, vector spaces, matrix operations);
    \item basic probability theory (probability distributions, conditional probability, Bayes' theorem);
    \item programming skills, not necessarily in the languages mentioned above.
\end{itemize}
However, if some of the above-mentioned knowledge/skill is missing or is only partially available, the participants will be able to fill in the gaps and catch up during the theoretical and practical parts as explanations will be added upon questions from the audience.

%%%%%%%%%%%%%%%%%%%%%%%%%%%%%%%%%
\section{Required materials}

As a practical part, we will analyze basic machine learning/data mining algorithms (for example, centroid, $k$-means, kernel $k$-means, $k$-nearest neighbors, naive Bayes \cite{bishop2006pattern}, the PageRank  algorithm \cite{LangvilleMeyer2006}) and implement these using matrix operations, working in Python/Julia/Octave.
%
We will provide the participants with the materials needed for the course, including the presentation slides, bibliography, links to additional materials, datasets, as well as sample code snippets to facilitate the practice.
%
For effective cooperation, however, we would like to ask the participants to have Python\footnote{\url{https://www.python.org/}} and Julia\footnote{\url{https://julialang.org/}} installed on their computers, as well as an integrated development environment or code editor in which they can work comfortably -- we suggest the \emph{Visual Studio Code}\footnote{\url{https://code.visualstudio.com/}} system.

%%%%%%%%%%%%%%%%%%%%%%%%%%%%%%%%%
\section{Desired outcome}

After finishing the course the participants will have a basic knowledge of machine learning/data mining methods, will understand how to apply these and in which situations and they will gain practical knowledge of how to efficiently implement some of these algorithms using vectorization as well.

%%%%%%%%%%%%%%%%%%%%%%%%%%%%%%%%%
\section{Relation to other contributions and lectures}

This lecture is a continuation of the \emph{Sustainable functional programming and applications} lecture given by \emph{Lehel Csat\'o} at the Rijeka Summer school in 2022 -- where the Julia language was used to solve machine learning problems.
%
At the same summer school, the lecture \emph{Applying Soft Computing for Sustainability} by \emph{Goran Mauša} is also related to vectorization; since one computes the same functions for the whole dataset.
%
The 2023 Sust(r)ainable Summer School will host the \emph{Efficient evolutionary computing} lecture by \emph{Goran Mauša}, where the data mining problems require vectorization to function efficiently.

\section{Acknowledgments}

This work acknowledges the support of the ERASMUS+ project ``SusTrainable -- Promoting Sustainability as a Fundamental Driver in Software Development Training and Education'', no. 2020–1–PT01–KA203–078646.

%\bibliographystyle{splncs04}
%\bibliography{bib}

%\end{document}

\input{bbls/bodo_csato_vectorization.bbl}




\title{Sustainability through Green Gamification}%
\titlerunning{Sustainability through Green Gamification}% explicit
%
\author{Denitza Charkova\orcidID{0000-0003-0873-4415} \and Elena Somova\orcidID{0000-0003-3393-1058} }
\authorrunning{D. Charkova, E. Somova}% explicit
%
\institute {University of Plovdiv "Paisii Hilendarski", 24 Tzar Assen St., 4000 Plovdiv, Bulgaria,\\%
	\email{  dcharkova@uni-plovdiv.bg;eledel@uni-plovdiv.bg}\\%
	\url{https://www.uni-plovdiv.bg/} }%
%
\maketitle% typeset the header of the contribution
%
\begin{abstract}%
Among the many trials that our contemporary society faces there is one which stands out as a priority to cultivate a healthy lifestyle on our planet. This issue is climate change and to tackle it, urgent actions are needed not only on the societal, but also on a personal level. The exponential growth of environmental problems stimulates us, as educators, to invent new strategies in the field of Green Education to teach our children how to maintain a more sustainable future. This article discusses the integration and application of sustainable practices through educating Green Gamification at the Plovdiv University “Paisii Hilendarski” (PU). The experiment was performed on first- and fourth-year students majoring in three different IT majors. The study aimed at raising awareness about climate change, the burning of fossil fuels and the crucial importance of developing green behaviours on a both personal and business level.%
\keywords{Environmental Sustainability \and Technical Sustainability \and Sustainable Behavior}%
\end{abstract}%

\section{Sustainable UI and UX}%
\par%
The economic, societal, and educational realities we currently face reveal how vulnerable communities, countries and the entire world can be in all aspects of life. Many alterations are needed for our future generations to live a healthier and more sustainable existence. These ideas should be implemented in the educational systems at an early age to have long lasting effects and change behaviours for a greener future.%
\par%
One interesting method for achieving these goals is teaching sustainability through Green Gamification. This method can be integrated in the educational systems and accompany students in their entire course of study. Furthermore, integrating practices such as developing sustainable UI (user interface) and sustainable UX (user experience) we aim to have a greener and cleaner experience when using the availability of the Internet. The design of the sustainable UI/UX will be the main topic of our SusTrainable Summer School’2023 lecture.%
\par%
There are four key areas where work can be done in terms of building a more sustainable UI/UX~\cite{BIB2}: findability, performance optimisation, design and UX, and green hosting. Findability focuses on making the content as accessible and easy to locate as possible, thus saving energy because users will have to load fewer pages in order to find the desired information. Performance optimization focuses on speeding up the website and thus using less processing power. The design and UX focus on approaches allowing websites to be accessible to all users, regardless of the hardware they use. In turn, green hosting reduces the carbon footprint of a website by switching to a green web hosting company.%

\section{Theories of action}\label{section:overview}%
\par%
There are existent theories of action which can further encourage users to practise green behaviours through models of conduct. Such examples are~\cite{BIB1}: Nudge theory, Hooked model, Mindful design, Theory of interpersonal behaviour, and Transtheoretical model of behaviour change. Likewise, the integration of Green Gamification in our software applications, classrooms and even community models is another method of stimulating and rooting sustainable behavioural practices. %
\par%
To begin with, the Nudge theory~\cite{BIB9} ~\cite{BIB14} suggests it is capable of affecting human behaviour - choices made by individuals can easily be affected in a predictable way without forbidding any options. Such a sustainable design example can be increasing the visibility to affect human behaviour - by giving one of two buttons (options) extra colour and contrast. %
\par%
The second theory – Hooked model~\cite{BIB10} is also called Model for building habit-forming products (sustainable habits), where the 4 key words are Trigger, Action, Variable reward and Investment. The model starts to act with some trigger (e.g. the need), then there is a need for action (to eliminate the trigger). For example, here the UI/UX designer can increase the probability for sustainable action. On the next step action is rewarded (e.g. through likes, levelling up or praising the choice of a sustainable option). The reward should not be predictable, since variability will keep the user’s interest over time. In the end, the reward encourages the user to invest in the choosen action/product in waiting of more rewards. %
\par%
The Mindful design~\cite{BIB11} ~\cite{BIB15} is another way to shape behaviours. Mindfulness is an optimal interaction between attention and awareness. There are 3 designated steps to Mindful design: 1. Identification of the lack of mindfulness (e.g. user intensively scrolling an app), 2. Identification of the mindful solutions (e.g. reminds the users of their sustainable intent), 3. Implementation of the mindful solutions (makes the users mindful they must be disrupted and forced to reflect). %
\par%
The Theory of Interpersonal Behaviour~\cite{BIB12}~\cite{BIB16} is a cognitive model that claims: the individual’s intention is the core factor behind how humans act. The intention is created by the individual's beliefs, social norms and emotions. Habits trigger the intention and facilitating conditions restrict what behaviours are possible. So, the designed solutions should focus on habits.%
\par%
The Transtheoretical model of behaviour change~\cite{BIB13} defines the stages people pass from the moment they start thinking about changing their behaviour to the moment the change is durably achieved – Precontemplation, Contemplation, Preparation, Action, Maintenance and Termination. After the user has gained insight, the design must empower the user to create an action plan: by facilitating information of what is sustainable action the design prepares the user to take action. The design should support the user to take correct actions: by only suggesting options that are considered sustainable. In addition, the design should encourage the user to keep up with the new behaviour and prevent relapse. In the end, the users have to sustain the new behaviour themselves.%
\section{Green Gamification}%
\par%
The second focal point in the theoretical models is the application of Green Gamification on a both community and educational level. By definition~\cite{BIB3}, Gamification is the application of game principles and design elements in non-game contexts. It uses game elements in non-gaming systems to improve user experience and user engagement, as it applies game design to make otherwise boring tasks more engaging~\cite{BIB4}. It hooks us by meeting our basic human needs for achievement, appreciation, reciprocity, and a sense of control over our little corner of life. In today’s competitive battle for attention, games are the most effective tool for leveraging technology, rising above marketing noise and engaging the socially networked consumer. The European Commission Acknowledges that the challenge is to mainstream the application of gaming technologies, design and aesthetics to non-leisure contexts, for social and economic benefits.%
\par%
Gamification consists of elements, techniques, and actions, which all come together to build an educational experience through games. Examples of game elements~\cite{BIB5}~\cite{BIB6} can be avatars, bonuses, badges, combos, rewards, leaderboards, progress, status, teams etc. The game techniques can also vary depending on the material, topic, class and aim of the specific class. Such examples~\cite{BIB5}~\cite{BIB6} can be identity shift, reward system, tracking progress, current status, rules, teamwork, time limit, hidden treasure, feedback etc. The game actions can also differ depending on the educational aim and circumstance e.g., role playing, receiving a bonus or award, gaining an advantage, retrieving resources, following progress level, completing missions etc. %
\par%
According to the Bartle Player quiz~\cite{BIB7}, the author designates 4 specific player types, which possess unique characteristics when engaging in a game: achiever, killer, explorer and socialiser. By nature, the achiever focuses on elements such as: level, badge, bonus, reward, resource, progress and techniques such as a reward system, feedback, challenge, mission, adventure and progress tracking. On the other hand, the profile of the socializer focuses on elements such as teamwork, the usage of avatars, chat and messaging and applies techniques like teamwork, communication etc.%
\par%
Taking a step further into the topic of Green Gamification~\cite{BIB8} we can define the term as the use of game mechanics to engage people and change behaviour, and apply it to sustainability issues.%

\section{PU green experiment}%
\par%
The Green Gamification experiment, performed at the university level consisted of two smaller experiments – an individual and team experiment. The experiment began with a lecture on Sustainability, Sustainable Development and Sustainable Education. After the acquaintance with the core concepts, students were introduced to the idea of Green Gamification. The aim was to highlight the importance of the issues and addressing them through meaningful education on the subject matter. Students were encouraged first to share their prior knowledge on Sustainability and were informed about legislations and sustainable goals both in Europe and worldwide. %
\par%
The first task was performed on an individual level and students were responsible for “creating a green log” each day. This was done using a shared file, where in the course of 1 week, students had to perform green actions, fill in the log which should not repeat previous logs (no matter if the logs were made by other students or were their own). They were in charge of discovering a green activity or behaviour that contributes to Environmental Sustainability. They also had to describe the essence of this action in 100 characters, using relevant vocabulary and explain why their action is sustainable. The game would be won by the one who manages to publish the most activities. This experiment showed that students were very actively motivated and most of them had multiple logs, explained their action’s impact and provided evidence. They shared that they found the task quite interesting and used a lot of humour in their explanations. Students were excited to share their actions in the logs and followed up in class. Students noted they were not so much motivated by the badges rather the new idea they had to generate. %
\par%
The second experiment was titled Team Green Business and was intended to be a group experiment. Students were in charge of building their own business which supports Green Behaviour and Sustainability. Students were encouraged to mind map their ideas in class, filter them and create a project of a business they thought would support green behaviour. Students were provided with helper questions to scaffold and guide them through the task. Such questions were: What does your company do? What kind of specialists do you have? What does your office environment look like? What green habits do you want to introduce in your company? Which unsustainable practices do you want to eliminate from your work environment? What is your position on fossil fuels? Would you work for clients who work for fossil fuel companies?%

\section{Conclusion}%
\par%
The core goals were to motivate sustainable behaviour through games and to provoke students to maintain the sustainable practices in their own urban environment. It is used to incentivise positive changes in behaviour and help young IT professionals to improve their overall quality of life. In addition, through the power of gamification we can make that experience predictable, repeatable, and financially rewarding. %
\par%
The plans for the SusTrainable Summer School'2023 are for the students to acquire and practice skills which will help them create a sustainable UI/UX design through the application of different approaches (incl. Green Gamification). The lecture will present suitable theories, models and approaches that can be used in sustainable UI/UX design. During the lab session students will design a green gamified mobile application that has to encourage/impact some type of sustainability, has to be sustainable from the technical point of view and has to include gamification techniques. In a few days of ongoing collaborative work in preset teams, students will present their ideas on UI/UX design. The training will finish with the evaluation of the ideas by all students and teachers and a subsequent award ceremony for the best works will be held. The evaluation questions for the students' work will be: Do they use Gamification (elements x techniques)?, Do they provide technical sustainable decisions? and Do they impact/encourage the user to adopt sustainable behavior (environmental/economic/social)?. Evaluators also will choose the most original and innovative idea. After completing the training, students will have gained knowledge on many approaches on how to design a sustainable UX/UI and will be able to develop sustainable websites and mobile applications.%

\par%
\textbf{Acknowledgements}%
\par%
This paper acknowledges the support of the Erasmus+ Key Action 2 (Strategic partnership for higher education) project No 2020-1-PT01-KA203-078646: “SusTrainable – Promoting Sustainability as a Fundamental Driver in Software Development Training and Education”. The information and views set out in this paper are those of the authors and do not necessarily reflect the official opinion of the European Union. Neither European Union institutions and bodies, nor any person acting on their behalf may be held responsible for the use which maybe made of the information contained therein.

%\printbibliography%
%\end{document}%
% 
\input{bbls/Paper_PU_2_USE_BIBTEX.bbl}




%
\title{Large project sustainability: the role of code comprehension\thanks{Prepared with the professional support of the Doctoral Student Scholarship Program of the Co-operative Doctoral Program of Ministry of Innovation and Technology financed by the National Research, Development and Innovation Fund.}}
%
%\titlerunning{Abbreviated paper title}
% If the paper title is too long for the running head, you can set
% an abbreviated paper title here
%
\author{Anett Fekete\orcidID{0000-0001-8466-7096} \and
Zoltán Porkoláb\orcidID{0000-0001-6819-0224}
}
%
\authorrunning{A. Fekete and Z. Porkoláb}
% First names are abbreviated in the running head.
% If there are more than two authors, 'et al.' is used.
%
\institute{Eötvös Loránd University, Faculty of Informatics,\\ Pázmány Péter stny. 1/C, 1117 Budapest, Hungary\\
\email{\{afekete, gsd\}@inf.elte.hu}
}
%
\maketitle              % typeset the header of the contribution
%
\begin{abstract}
Code comprehension is a daily challenge in a software developer's life. A substantial amount of time is spent with comprehension activities at the expense of productivity. Junior programmers are in the most dire need of proper code comprehension support, since they lack the amount of work experience which facilitates comprehension tasks. There is a wide variety of comprehension supporting software, however, developers are usually not aware of their options, and settle for the insufficient support in code editors. We aspire to raise awareness among programmers about the importance of proper code comprehension support by introducing the CodeCompass code comprehension framework to students. We expect that by showing the practical usage and benefits of a multi-purpose comprehension tool through solving a C++ programming task will instill the need in young programmers for such support who will later indicate their need in the workplace.

\keywords{Code comprehension  \and Sustainability \and Experiment.}
\end{abstract}
%
%
%
\section{Introduction}

Understanding the source code and architecture of a program is part of the daily challenges in a software developer's life, no matter how long they have been working as a programmer. Code comprehension is made difficult by several circumstances: incomplete or missing documentation, scattered knowledge among developers, insufficient comprehension support in code editors, etc. According to various studies, developers spend at least half of their working hours with code comprehension activities, and this amount of time has just been growing as computer science became a more and more paramount part of our lives \cite{corbi1989program, cherubini2007let, minelli2015know, siegmund2016program, xia2017measuring}. The statistics in these studies suggest that programmers are in need of more efficient code comprehension support. More time spent on comprehension activities mean slower task execution and more resource consumption. There are many resource types whose costs can be greatly reduced by using effective code comprehension support: working hours, money and actual energy used by computers are all included.

In this paper, we present the role of code comprehension in resource management by emphasizing the usage of comprehension tools in everyday programming tasks. We also present our plans for the summer school of 2023 in Coimbra, where we intend to popularize code comprehension tools among young developers with the hope that the culture of needing proper code comprehension support in the workplace will become more widespread.

\section{Code comprehension models}

There has been extensive research concerning the source code comprehension strategies of software developers. Program comprehension procedures form well-defined workflows that can be categorized into various code comprehension models which serve as unified abstractions of the comprehension process. Several models have been identified throughout the last couple decades, with two major categories that identify the code comprehension process based on the directions from where programmers approach the source code: the top-down and the bottom-up direction.

Some studies presented reviews of these categories \cite{storey2005theories}, with Mayrhauser and Vans also defining a new comprehension model \cite{von1995program}. Siegmund also summarized the categories in their study \cite{siegmund2016program}, along with an insight of the present and the possible future of code comprehension. We have also published a comprehensive review of the above mentioned code comprehension directions, putting the focus on classifying the function sets of modern code editors and integrated development environments into the categories \cite{fekete2020comprehensive}.

A top-down approach is built when the programmer first tries to build a mental image (hypothesis) of the purpose of the program, then moves on to finer details by comparing a the code to similar known programs. In this case, the programmer is initially familiar with the program domain. The finer details serve as further information which may help accepting, rejecting or modifying the hypothesis \cite{brooks1977towards}. The program domain elements (e.g. programming language syntac, algorithms, coding conventions etc.) provide a basis for the evaluation of the initial assumptions \cite{soloway1984empirical}.

A bottom-up strategy is applied when a programmer does not possess enough domain knowledge, so they will start understanding the code by searching for "pivot points" in the source code. The programmer relies on syntactic and semantic knowledge of a programming language to build a high level mental model of the program \cite{shneiderman1979syntactic}. In other cases, observation does not start at the code itself, but at the control flow of the program \cite{pennington1987stimulus}. Then the programmer considers the programming goals and the control flow to build a hypothesis of the program's operation.

There are comprehension models that include elements from both top-down and bottom-up approaches \cite{levy2019understanding} which are called integrated approaches. These models pay the most attention to the discursive way of human thinking. In this case, the programmer switches between the above discussed directions in an arbitrary way based on the previously existing and recently collected information~\cite{letovsky1987cognitive}.

\section{Code comprehension and sustainability}

In the realm of green computing and sustainability, code comprehension support holds paramount significance. As the world strives to mitigate the environmental impact of computing systems, it becomes imperative to develop software solutions that are energy-efficient, resource-conscious, and eco-friendly. However, achieving such objectives necessitates a comprehensive understanding of the underlying code base. Code comprehension support enables developers to navigate intricate software architectures, identify energy-intensive code segments, and optimize resource utilization. By comprehending the details of the code, developers can make informed decisions regarding algorithmic efficiency, memory management, and power consumption. Furthermore, code comprehension support facilitates the identification of areas for code refactoring, eliminating redundant or inefficient operations, and promoting cleaner, streamlined code. Ultimately, by enhancing code comprehension, developers can significantly contribute to the development of energy-efficient software systems, reducing the carbon footprint of computing and advancing the goals of sustainability in the digital age.

\section{CodeCompass}

CodeCompass \cite{porkolab2018codecompass} is an open-source\footnote{GitHub: \url{https://github.com/Ericsson/CodeCompass}} code comprehension framework developed by Eötvös Loránd University and Ericsson. It consists of a parser and a webserver binary. The CodeCompass parser applies static analysis to the given, ideally compiled source code and the corresponding build commands generated during compilation. Various information is stored afterwards about the project regarding structural data, code metrics, version control information, etc. This information is stored in the workspace database which is then accessed by the CodeCompass webserver. The webserver provides several different textual and graphic services, such as detailed searching, structural and code-level visualizations, and Git blame data.

CodeCompass is a pluginable framework. Each plugin is independent of every other, thus a certain plugin can be easily skipped from the parsing process if not needed. Plugins consist of a \emph{model}, a \emph{parser}, a \emph{service}, and a \emph{web GUI} component. Currently, CodeCompass fully supports C and C++ programs, and is in part capable of parsing C\#, Java, and Python projects. Apart from language parsing, CodeCompass provides code metrics analysis, advanced search functionalities, and Git repository visualizations.

\subsection{Role in education}

CodeCompass is mainly developed in the Model C++ Software Technology laboratory at Eötvös Loránd University, Faculty of Informatics. Students have the opportunity to join the project by solving issues (e.g. bug fixing, small improvements, refactoring), or by taking up a large subproject, such as creating a new plugin, or complementing an already existing one with various new features. Students who join the lab learn about working with various programming languages and APIs, using version control, reviewing other people's code, designing software architecture, etc. \cite{fekete2022building}

The software is integrated with the TMS assignment management system which is developed in our faculty and is used to handle programming assignments. This feature is supposed to facilitate the work of teachers: every submitted assignment can be parsed by CodeCompass in a containerized environment, and the teacher is able to browse the code by launching the webserver. This way the teacher does not have to download each assignment to their computer.

\section{Course information}

In our part of the summer school we plan to show the participants the importance of proper code comprehension support. For this purpose, during the lecture we will introduce the theoretical background of code comprehension along with the feature set of CodeCompass. This way, the students will be familiar with the wide variety of code and program comprehension supporting features that the tool offers; this lecture also serves as a thought provoking session in which we motivate the students to identify their specific needs in code comprehension tasks, e.g. come up with new functionalities based on real life examples from school or workplace.

In the practical session we will present a small C++ project, \emph{TinyXML2}\footnote{GitHub: \url{https://github.com/leethomason/tinyxml2}} which is readily parsed and available on the demo site\footnote{Demo website: \url{https://codecompass.net/}} of CodeCompass. TinyXML2 is a distinguished project that we use to test and present CodeCompass. It consists of three source files which contain hundreds of source code which makes it a perfect test project for code comprehension activities. After a short walk-through of all available functions, we will give the students a minor programming task to students in TinyXML2: they will have to find and modify one specific line in the source code in order to make the tag parsing feature of TinyXML2 case-insensitive. The students will get approximately 60 minutes to solve the task. Based on the results of a previous experiment which we conducted in the spring semester of 2022 with the participation of 27 Computer Science MSc students \cite{fekete2022report}, cc. 60 minutes should be enough for everyone to solve the task.

During the practice the students will be only allowed to use CodeCompass for code browsing and understanding. The software will be available on the demo site with TinyXML2 already parsed. We will anonymously collect information about the students' activities in CodeCompass via Google Analytics. They will be able to modify the code using Visual Studio Code, and compile the code with CMake and Make. The focus of this task is to correctly identify the exact line which has to be modified instead of writing new code; for this reason, we will readily provide the C++ method which has to be invoked to make the parsing case-insensitive. The students need will need fair understanding of C++ or any other imperative language, and a basic computer with 2 to 4 GB of RAM and Internet access. They will not need any knowledge in build systems as the exact commands will be given to them beforehand.

After our session we will analyze and evaluate the activity data. We will also publish various statistics for the students: best, average and medium solution times, most used functionalities in CodeCompass, etc.

\section{Conclusion}

As we discussed, an average programmer spends at least half of their work hours with code comprehension activities. Proper and effective code comprehension support is a very important aspect of software development: it reduces development costs in human resources, finances, and energy consumption. However, standalone comprehension supporting software is not widespread enough to make significant difference in resource usage. Our purpose in the summer school is to popularize code comprehension tools by showing the various functionality they serve. Our plan is to give the participating students a small C++ programming task which they have to solve by using a code comprehension supporting software. Our demo comprehension tool is CodeCompass which is equipped with several features which may help reducing the time spent with comprehension activities, and thus, resource consumption.

\section{Acknowledgement}

This paper acknowledges the support of the Erasmus+ Key Action 2 (Strategic partnership for higher education) project No. 2020-1-PT01-KA203-078646: “SusTrainable - Promoting Sustainability as a Fundamental Driver in Software Development Training and Education”.

The information and views set out in this paper are those of the author(s) and do not necessarily reflect the official opinion of the European Union. Neither the European Union institutions and bodies nor any person acting on their behalf may be held responsible for the use which may be made of the information contained therein.

%\bibliographystyle{splncs04}
%\bibliography{references}
%\end{document}

\input{bbls/main2_USE_BIBTEX.bbl}




%
\title{Energy Consumption and Optimization of Software}

%\thanks{Supported by organization x.}}
%
%\titlerunning{Abbreviated paper title}
% If the paper title is too long for the running head, you can set
% an abbreviated paper title here
%
\author{João Paulo Fernandes\inst{1}\orcidID{0000-0002-1952-9460} \and
Bernardo Santos\inst{1}\orcidID{0000-0002-6172-3830} \and
Maja H. Kirkeby\inst{2}\orcidID{0000-0003-0033-2438} \and
Luís Paquete\inst{3}\orcidID{0000-0001-7525-8901}
}
%
\authorrunning{J. P. Fernandes et al.}
% First names are abbreviated in the running head.
% If there are more than two authors, 'et al.' is used.
%
\institute{LIACC \& DEI-FEUP, University of Porto, Porto, Portugal\\
\email{\{jpaulo,up201706534\}@fe.up.pt}
\and
Roskilde University, Roskilde, Denmark\\
\email{majaht@ruc.dk}
\and
University of Coimbra, Coimbra, Portugal\\
\email{paquete@dei.uc.pt}}
%
\maketitle              % typeset the header of the contribution
%
\begin{abstract}
We describe a tutorial to be delivered in the second Summer School organized with the Sustrainable - Promoting Sustainability as a Fundamental Driver in Software Development
Training and Education, an Erasmus+ Strategic Partnership.

The tutorial aims to provide software engineers with the knowledge and tools to measure software energy consumption. This is regarded as a first step towards the more ambitious goal of optimizing such consumption. Reducing energy consumption within ICT systems is paramount in realizing a more sustainable exploration of the technology that ever more rules the world.



\keywords{Energy Consumption \and Software Analysis  \and Software Optimization}
\end{abstract}
%
%
%
\section{Introduction}

Sustainability is a crucial driver for the development of modern society and the planet's future.
While groundbreaking progresses make of our age an exciting period for humanity, there exists a growing awareness that progress needs to consider the resources the world can provide.
If we continue over-exploiting such resources, the ability of the next generations to inhabit the planet is endangered.

However, the challenges towards sustainability are tremendous, as they are complex and multifaceted and require expertise from a wide range of disciplines to be addressed. For example, addressing climate change requires expertise in atmospheric science, ecology, engineering, economics, and policy analysis. Similarly, addressing social sustainability challenges such as poverty, inequality, and social justice requires sociology, psychology, political science, and public policy expertise.

Sustainability also entails understanding the interconnectedness of social, environmental, and economic systems. For example, environmental degradation can significantly impact human health and well-being, and economic development can significantly impact environmental sustainability. Understanding and addressing these interconnections requires interdisciplinary collaboration and a holistic approach to problem-solving.


Our work focuses on addressing and promoting sustainability within software development and engineering.

As a broad discipline, Sustainable Software Development/Engineering encompasses various aspects, ranging from economic to social, including technological, technical, and environmental factors~\cite{lago15}.
The social perspective explores the use of software to enhance people's quality of life, while the economic perspective emphasizes the development of products that can endure for an extended period. Additionally, from a software development standpoint, technical sustainability encourages the production of high-quality software, which in turn promotes reusability and reduces future development efforts and resource consumption.

One significant aspect that must be considered is environmental sustainability. It focuses on producing and utilizing software with minimal environmental impact, aiming to minimize resource use and maximize energy efficiency. This is a critical concern.

The emergence of large-scale cloud deployment models has led to the establishment of massive data centers with significant energy consumption, raising environmental sustainability concerns.
A 2020 study by the EU Commission revealed that data centers in the former 28 EU countries experienced a substantial increase in energy consumption. In 2010, they consumed 53.9 terawatt-hours, which escalated to 76.8 terawatt-hours by 2018. This significant rise accounted for 2.7\% of the total electricity demand in the EU.

While, in practice, the hardware components of ICT systems consume energy, their software counterparts govern how to run the hardware and how and when energy is consumed. This means that achieving the goals of Sustainability, particularly under its environmental lenses, is only possible by targeting Sustainable Software Development.

The relevance of addressing research in this line is confirmed by Software Developers themselves: it has been shown that Software Developers are indeed keen on developing energy-efficient software~\cite{pinto2014mining,pang2016programmers}. And studies have shown that different design patterns~\cite{DBLP:books/sp/21/Feitosa00F0S21}, sorting algorithms~\cite{bunse2009exploring}, software version changes~\cite{DBLP:journals/ese/Hindle15}, refactorings and transformations~\cite{park2014investigation,DBLP:conf/wcre/0001SF20}, and different Java based collections~\cite{Pereira:2016:IJC:2896967.2896968,hasan2016energy} have a statistically significant impact on energy usage.
Studies have also shown how even the choice of the programming languages to use influences energy usage~\cite{pereira2017sle,DBLP:journals/scp/PereiraCRRCFS21}.

While there is promising evidence of the feasibility of improving energy efficiency by targeting software components of ICT systems, the fact is that this topic is clearly under-represented in the education of modern software engineers.
This paper describes a summer school tutorial that aims to provide junior software engineers with the motivation, knowledge, and tools to measure, analyze and ultimately analyze the energy consumption of software. The structure and content of the tutorial we propose are described in detail in the next section.


\section{Structure and Contents of the Proposed Tutorial}

The tutorial will cover the essential knowledge and skills needed to understand and optimize software with energy consumption in mind. It is divided into two parts, a theoretical lecture to introduce key concepts and a hands-on laboratory session to become familiar with collecting and comparing energy consumption of software.
At the end of this tutorial, the participants will have acquired:

\begin{enumerate}
    \item [$\mathbf{O1}$] knowledge about what affects the energy consumption of software;
    \item [$\mathbf{O2}$] skills in measuring the energy consumption of software.
\end{enumerate}

\subsection{Theoretical Lecture}
A great way of demonstrating the importance of keeping energy efficiency in mind when developing software is to look at the energy consumption trends of the ICT sector, of which software is a very large part. Thus, showing the participants this sector's past consumption and forecasts for the near future is the first matter addressed in the lecture.

In order to successfully design and develop energy-efficient software, it is crucial to possess more than just a basic familiarity with energy consumption and efficiency concepts. It is essential to have a deep and comprehensive understanding of the various factors that influence consumption. Furthermore, one must be well-versed in the techniques and tools that are at our disposal for accurately measuring and optimizing energy efficiency in software development.

In software applications, energy consumption refers to the amount of energy the application consumes during runtime. Regarding energy efficiency, it is defined in~\cite{energy_efficiency_2} as "(...) using less energy for the same output or producing more with the same energy input (...)". Ultimately energy efficiency is the ratio between energy consumed and outputs produced, as it can be improved by reducing consumption and maintaining the output, maintaining consumption and increasing the output, or simultaneously reducing consumption and increasing the output.

To better understand how different factors influence the energy consumption of software we can abstract a computer into 7 layers, presented in Figure~\ref{fig:stack}, providing us with a framework to understand the different levels of software and hardware involved in computing. At the top of the stack, algorithms represent the highest level of abstraction and are the most energy-agnostic. As we move down the stack, we encounter more specific and concrete layers, which can be optimized with some effort. And at the bottom of the stack are the most hardware-specific layers where energy efficiency can be most effectively optimized. Understanding this stack, and relating it to the factors identified as influential in the energy consumption in~\cite{cloud_survey}, can help identify opportunities to improve efficiency at various system levels.

% Figure environment removed

The participants will be introduced to energy measurement and modeling, two different approaches to studying the energy consumption of software, each with its advantages and limitations. While energy measuring is considered the ground truth, as it provides accurate and precise measurements, it requires specialized hardware which may be expensive to acquire.
On the other hand, energy modeling approaches estimate energy consumption, resulting in values that may not be totally accurate but instead estimates that are relatively accurate, i.e., the relative distance between two measurements is correct. Still, the actual values may differ from the ground truth. Energy estimators are a more cost-effective approach that does not require specialized hardware.

The lecture will introduce Intel's Running Average Power Limit, or RAPL, an energy estimator that has proven itself to be reliable and accurate~\cite{pl:rank2017,tec:rapl1,tec:rapl2}, and has been used in multiple studies~\cite{pl:rank2017,ds:haskell,pereira2017sle,DBLP:journals/scp/PereiraCRRCFS21,Pereira:2016:IJC:2896967.2896968}. However, this tool has its limitations and the authors of~\cite{rapl_in_action} do a great job of pointing them out and presenting mitigation strategies when possible. Some of these include a lack of granularity, register overflows, and non-atomic updates. Therefore, it is important to be aware of some good practices when taking measurements, like disabling non-required services or WiFi connection, to reduce the noise in the system. Alternatively, one could use other tools, such as pTop~\cite{tec:ptop} or PowerTop\footnote{\url{https://github.com/fenrus75/powertop}, accessed 13/06/2023}, which overcome some of RAPL's limitations even though they might introduce some of their own. Nonetheless, for this tutorial, an overview of RAPL will be provided, so that the participants may realize their own experiments in the hands-on session.


\subsection{Practical Lecture}
The purpose of the hands-on lecture is to provide participants with an opportunity to put their recently acquired knowledge into practice. Each participant will be furnished with a laptop equipped with all the necessary materials and software for seamless execution.

The initial task involves two parts. Firstly, participants need to choose between programs with diverse functionality, such as sorting, provided in various programming languages that encompass different programming paradigms, including functional, object-oriented, and imperative. This allows participants to select programs and languages they are familiar with. Secondly, participants are required to utilize RAPL to measure the energy consumption of these selected programs. Once completed, the participants will have their first experience in gathering information about energy consumption.

In the second task, participants will be required to modify the chosen program while preserving its functionality, aiming to improve performance or make changes to implementation details. Following these modifications, participants will once again measure energy consumption. The number of measurements to be performed is up to the participant. Still, they should collect enough samples to assess whether or not the modifications influenced energy consumption.


\subsection{Reaching the Learning Outcomes}
To ensure that the participants acquire the learning outcomes of this tutorial, we have designed a comprehensive program that integrates theory and practice.

Regarding $\mathbf{O1}$, the theoretical lecture will introduce the key concepts of energy consumption in the software context, as well as some factors that affect energy consumption. To consolidate these concepts, the participants will be given the opportunity to experiment with software and modify them to observe the resulting energy consumption, on the hands-on laboratory session. This will enable the participants to develop a deeper understanding of how the various factors affect energy consumption in software.

When it comes to $\mathbf{O2}$, the participants will learn the difference between energy modeling and measuring, including the advantages and drawbacks of each method, during the theoretical lecture. Furthermore, they will also be introduced to RAPL and good practices for using tools such as this. In the practical session, participants will be required to measure the energy consumption of some programs, putting into practice what they have learned in the theoretical lecture.


\section{Conclusion}

This paper describes a summer school tutorial that aims to promote sustainability in software development and engineering. The tutorial provides participants with the knowledge and skills to optimize software with energy consumption in mind. Through a theoretical lecture and a hands-on laboratory session, participants will gain an understanding of the factors affecting energy consumption in software and how to measure it effectively.

By the end of the tutorial, participants will have acquired knowledge about what affects the energy consumption of software and the skills needed to measure and optimize it. This tutorial has the potential to make a significant impact in promoting sustainability in the field of software development and engineering. By equipping junior software engineers with the tools they need to make informed decisions about energy consumption, we can contribute to a more sustainable future.


\section*{Acknowledgements}

This work acknowledges the support of the ERASMUS+ project “SusTrainable —
Promoting Sustainability as a Fundamental Driver in Software Development
Training and Education”, no. 2020–1–PT01–KA203–078646.

%
% ---- Bibliography ----
%
% BibTeX users should specify bibliography style 'splncs04'.
% References will then be sorted and formatted in the correct style.
%
%\bibliographystyle{splncs04}
%\bibliography{mybibliography}
%

%\end{document}

% This is samplepaper.tex, a sample chapter demonstrating the
% LLNCS macro package for Springer Computer Science proceedings;
% Version 2.20 of 2017/10/04
%
\documentclass[runningheads]{llncs}
%
\usepackage{graphicx}

\usepackage[dvipsnames]{xcolor}

\usepackage{hyperref}
\urlstyle{same}
\hypersetup{colorlinks=true,
    linkcolor=black,
    filecolor=black,      
    urlcolor=black,
    citecolor=black}

% Used for displaying a sample figure. If possible, figure files should
% be included in EPS format.
%
% If you use the hyperref package, please uncomment the following line
% to display URLs in blue roman font according to Springer's eBook style:
\renewcommand\UrlFont{\color{blue}\rmfamily}
\usepackage{amsmath}
% Make sure refs are sorted numerically when citing multiple papers together [chris]
\usepackage[sort]{cite}

% NOTE: is this allowed?
\usepackage{tcolorbox}
%%%% Shorthands
\newcommand{\etc}{etc.\ }
\newcommand{\eg}{e.g., }
\newcommand{\cf}{cf.\ }
\newcommand{\ie}{i.e., }
\newcommand{\vs}{vs.\ }
\newcommand{\etal}{\textit{et al.}\ }

%%% referencing shorthands  [chris]
\usepackage[nameinlink]{cleveref}
\crefname{figure}{Fig.\hspace{-1pt}}{Figs.\hspace{-1pt}}
\Crefname{figure}{Figure}{Figures}
\crefname{equation}{Eq.\hspace{-1pt}}{Eqs.\hspace{-1pt}}
\Crefname{equation}{Equation}{Equations}
\crefname{section}{Section}{Sections}
\Crefname{section}{Section}{Sections}
\crefname{table}{Table}{Tables}
\crefname{appendix}{Appendix}{Appendices}
% \newcommand{\figref}[1]{Fig.~\ref{#1}}    % within sentence
% \newcommand{\figsref}[2]{Figs.~\ref{#1}--\ref{#2}}    % within sentence
% \newcommand{\Figref}[1]{Figure~\ref{#1}}  % start of sentence
% \newcommand{\Figsref}[2]{Figures~\ref{#1}--\ref{#2}} % start of sentence
% \newcommand{\tabref}[1]{Table~\ref{#1}}
% \newcommand{\Tabref}[1]{Table~\ref{#1}}
% \newcommand{\secref}[1]{Section~\ref{#1}}
% \newcommand{\equref}[1]{Eq.~(\ref{#1})}
% \newcommand{\eqsref}[2]{Eq.~(\ref{#1})--(\ref{#2})}
% \newcommand{\appref}[1]{Appendix~\ref{#1}} % Appendix is NOT prepended automatically in IEEEtran

\newcommand{\chris}[1]{\textcolor{ForestGreen}{#1}}
\renewcommand{\chris}[1]{\textcolor[rgb]{0,0.0,0.0}{#1}}
\newcommand{\thms}[1]{\textcolor[rgb]{.8,0.3,0.1}{#1}}
\renewcommand{\thms}[1]{\textcolor[rgb]{0,0.0,0.0}{#1}}
\newcommand{\semere}[1]{\textcolor[rgb]{1,0.2,0.4}{#1}}
\renewcommand{\semere}[1]{\textcolor[rgb]{0,0.0,0.0}{#1}}
\newcommand{\johannes}[1]{\textcolor[rgb]{0.5,0.5,0.7}{#1}}
%\renewcommand{\johannes}[1]{\textcolor[rgb]{0,0.0,0.0}{#1}}
\newcommand{\semererev}[1]{\textcolor[rgb]{1,0.6,0.8}{#1}}
\renewcommand{\semererev}[1]{\textcolor[rgb]{0,0.0,0.0}{#1}}
\newcommand{\colortemplate}[1]{\textcolor{CadetBlue}{#1}}

\usepackage{booktabs}
\usepackage{amssymb}
\usepackage[inline,shortlabels]{enumitem}

\usepackage{multirow}
\usepackage{enumitem}
\usepackage{amsmath} % for the equation* environment

\usepackage[colorinlistoftodos,prependcaption,textsize=normalsize]{todonotes}

\usepackage{todonotes}
%\newcommand{\comment}[1]{\todo[inline]{#1}}

\usepackage{graphicx}
\graphicspath{ {./figures/} }

\newcommand\dqsim{{\textsf{DQ-SIM}}}
\newcommand\mtfive{{\textsf{mT5}}}
\newcommand\zerochatgpt{{\textsf{Zero-ChatGPT}}}
\newcommand\staticchatgpt{{\textsf{Static-Demo-ChatGPT}}}
\newcommand\demochatgpt{{\textsf{Dynamic-Demo-ChatGPT}}}

\newcommand\roberta{{\textsf{XLM-RoBERTa}}}
\newcommand\CLS{{\sc{[cls]}}}
\newcommand\FFNN{{\textsf{FFNN}}}


\newcommand{\h}{\mathbf{h}}











\begin{document}
%
%\title{Gap prediction for Grammar Exercises:\\
%Learning from partially annotated Real-world Data}

%\title{Dynamically retrieved in-context examples to Generate Distractors using ChatGPT}
%\title{Improving Distractor Generation in Generative Large Language Models: 
%\title{Effective distractor generation with large language models:Leveraging dynamically retrieved in-context examples}
\title{Distractor generation for multiple-choice questions with predictive prompting and large language models }
%\title{Retrieve dynamic in-context examples and generate distractors using ChatGPT}
% leveraging exercise corpora for the automated exercise creation without additional supervision: 
% Automated Exercise Creation from  Educational Sources:  


%What can be Learned from Real-World Data with Incomplete Annotations?} 
%\thanks{Supported by organization x.}}
%\title{\thms{Learning from Incomplete Annotations: \\}}
%
\titlerunning{Distractor generation via predictive prompting and LLMs}
% If the paper title is too long for the running head, you can set
% an abbreviated paper title here
%

\author{Semere Kiros Bitew \and
Johannes Deleu \and
\\
Chris Develder \and
Thomas Demeester}

%
\authorrunning{Bitew et al.}
%\authorrunning{}
% First names are abbreviated in the running head.
% If there are more than two authors, 'et al.' is used.
%
\institute{
IDLab, Ghent University -- imec, Ghent, Belgium\\
\email{\{semerekiros.bitew, johannes.deleu, chris.develder, thomas.demeester\}@ugent.be}}
%
\maketitle              % typeset the header of the contribution
%
\begin{abstract}
   
Large Language Models (LLMs) such as ChatGPT have demonstrated remarkable performance across various tasks and have garnered significant attention from both researchers and practitioners.
%However, there exists a research gap concerning the ability of generative LLMs to generate distractors.
\chris{However, in an educational context, we still observe a performance gap in generating distractors --- \ie plausible yet incorrect answers --- with LLMs for multiple-choice questions (MCQs).} In this study, we propose a strategy for guiding LLMs such as ChatGPT, in \chris{generating} relevant distractors by prompting them with question items automatically retrieved from a question bank as well-chosen in-context examples.
\chris{We evaluate our LLM-based solutions using a quantitative assessment on an existing test set, as well as through quality annotations by human experts, \ie teachers.}
%To assess the effectiveness of our proposed approach, we solicited feedback from teachers who evaluated the quality of the generated distractors.
%The evaluation revealed that, on average,
\chris{We found that on average 53\% of the} %5.5 out of the 10 
\chris{generated} distractors presented to the teachers were rated as high-quality% distractors
\chris{, \ie} suitable for immediate use \chris{as is}, outperforming the state-of-the-art model. 
%%% Furthermore, %in general, 9 of the distractors shown to the teachers 
%%% \chris{on average 90\%} were found to be relevant to the topic albeit not being used as is.
We also show the gains of our approach\footnote{{\url{https://github.com/semerekiros/distractGPT/} }} in generating high-quality distractors by comparing it with a zero-shot ChatGPT and a few-shot ChatGPT prompted with static examples. 
%These results highlight the promise of our strategy in enhancing the capability of LLMs in the context of distractor generation, which has implications for educational assessment and automated question generation systems.


\keywords{Distractor generation  \and natural language processing \and large language models \and predictive prompting \and language learning \and neural networks.}
\end{abstract}
%
%  Call for papers
%-----------------------------------------------------------------------------
%By offering a large number of highly diverse resources, learning platforms have been attracting lots of participants, and the interactions with these systems have generated a vast amount of learning-related data. Their collection, processing and analysis have promoted a significant growth of machine learning and knowledge discovery approaches and have opened up new opportunities for supporting and assessing educational experiences in a data-driven fashion. Being able to understand students' behavior and devise models able to provide data-driven decisions pertaining to the learning domain is a primary property of learning platforms, aiming at maximizing learning outcomes.

%However, the use of knowledge discovery in education also raises a range of ethical challenges including transparency, reliability, fairness, and inclusiveness. The purpose of RKDE, Responsible Knowledge Discovery in Education, is to encourage principled research that will lead to the advancement of explainable, transparent, ethical and fair data mining and machine learning in the context of educational data. RKDE is an event organized into two moments: a tutorial to introduce the audience to the topic, and a workshop to discuss recent advances in the research field. The tutorial will provide a broad overview of the state of the art on the major applications for responsible approaches and their relationship with the educational context. Moreover, it will present hands-on case studies that practically shows how knowledge discovery tasks can be responsibly addressed in education. The workshop will seek top-quality submissions addressing uncovered important issues related to ethical, fair, explainable and transparent data mining and machine learning in education. Papers should present research results in any of the topics of interest for the workshop as well as application experiences, tools and promising preliminary ideas. RKDE asks for contributions from researchers, academia and industries, working on topics addressing these challenges primarily from a technical point of view, but also from a legal, ethical or sociological perspective.
% ---------------------------End of call for papers-----------------------------------------
\section{Introduction}

%\semere{Second try of writing introduction:}

%\paragraph{\textbf{Advancements in AI and LLMs and its contribution to education:}}
The rapid advancement in artificial intelligence (AI) and large language models (LLMs) have paved the way for transformative applications across various domains, including the education domain. Since several \semere{LLMs (\eg GPT-3~\cite{brown2020language}, InstructGPT~\cite{ouyang2022training}, GPT-4~\cite{openai2023gpt4})} have been pretrained on massive amounts of data across multiple domains and languages, they are capable of solving natural language processing (NLP) tasks with little training examples (\ie few-shot learning) or no additional training (\ie zero-shot learning). This opens up new opportunities for adopting LLMs in the development of many educational technological solutions that aim to automate time-consuming and laborious educational tasks such as generating questions~\cite{kurdi2020systematic} and exercises%~\cite{bitew2023learning}, 
~\cite{bitew-etal-2023-learning}, essay scoring~\cite{ramesh2022automated}, and automated feedback~\cite{cavalcanti2021automatic}.  

%\paragraph{\textbf{ChatGPT and its educational tasks so far. Giving examples too} }
In particular, the recent release of ChatGPT, an LLMs-based generative AI model that requires only natural language prompts without additional model training or fine-tuning, has demonstrated diverse potential in automating various educational tasks. For example, ChatGPT has achieved the equivalent of a passing score for a third-year medical student (above 60\%) in the United States Medical Licence Examination (USMLE) Step 1 exam, and provided logical justification and informational context across the majority of answers~\cite{gilson2023does}. Likewise, ChatGPT's performance on four real exams (containing 95 MCQs and 12 essay writing questions), at the University of Minnesota Law School was equivalent to C+ students implying a pass in the course~\cite{choi2023chatgpt}. \semere{Li~\etal~\cite{li2023can} show the capability} of ChatGPT in generating high-quality reflective responses in writing assignments administered for pharmacy courses.

%\paragraph{\textbf{General introduction to MCQ and distractor generation and how LLMs can benefit MCQ generation}} 
One important educational task is the generation of multiple-choice questions (MCQs). MCQs have long been a popular form of formative and summative assessment in education due to their automatic scoring capability and the potential they hold for delivering timely and targeted feedback, which is crucial for facilitating effective learning~\cite{ramsden2003learning}. However, the process of crafting high-quality MCQs with effective distractors (\ie plausible yet incorrect answers) has traditionally been both a challenging and time-consuming task for educators (\eg teachers, content creators \etc) as poorly prepared distractors undermine the quality of MCQs~\cite{gierl2017developing}. This is where LLMs offer substantial benefits as they can be leveraged to automate the MCQ construction process, thus saving educators' time and effort while maintaining the quality and validity of the assessment items. For instance, teachers could employ LLMs to not only create different variants of the same MCQ questions but also develop different MCQs of comparable difficulty levels, facilitating targeted assessment for students with similar proficiency levels. Furthermore, students can benefit from the availability of several MCQs, enabling them to engage in regular practice, which is a well-established and highly effective learning strategy~\cite{roediger2006test}. Additionally, such models could be used for large-scale testing contexts (\eg licensure and certification testing) in which it is necessary to have multiple forms of a test and to introduce new question items regularly to minimize security concerns related to item exposure.    

%\paragraph{\textbf{Bridge : between old paper -- research question and current work}} 
In a recent study~\cite{bitew2022learning} conducted around the same time as the release of ChatGPT, researchers used local language models to automatically retrieve and reuse distractors to create new MCQs for education by leveraging existing pools of question items. In a user study they conducted with teachers, 3 out of 10 distractors proposed by their system were found to be high-quality, which is generally sufficient for creating an MCQ, as an average MCQ typically contains 3 distractors. However, they also report a staggering 50\% production of distractors that were entirely out of context given a question (so-called ``nonsense distractors''). With the emergence of ChatGPT, the question arises: \emph{does this previous approach become obsolete?} In our current study, we aim to address this question by examining the ability of out-of-the-box ChatGPT to generate effective distractors to be measured on the same scale as the previous study and evaluated by experts. Moreover, we study how both approaches could be combined into an even more effective approach. We also delve into the reliability issue, specifically in decreasing the production of nonsense distractors, which has implications for teachers' trust in the distractor generation tools.  
%LLMs are known to hallucinate by generating plausible-sounding but incorrect texts
To guide our investigation, we formulate the following research questions (RQs):


\begin{enumerate}
    \item \label{item:rq1} \textbf{RQ1}: In comparison to ranking-based models, does ChatGPT generate high-quality distractors for educational MCQs?
    \item \label{item:rq2} \textbf{RQ2: }To what extent can we rely on ChatGPT-generated distractors, and how can we measure their trustworthiness? 
    \item \label{item:rq3} \textbf{RQ3: } Is it possible to enhance the capability of distractor generation by combining ranking-based models with LLMs?
\end{enumerate}

To answer the RQs, we designed ChatGPT prompting strategies and we solicited feedback from human experts, \ie teachers, to evaluate the quality of generated distractors. We also compared the different strategies in terms of the reliability of generating less nonsensical distractors. In general, we found ChatGPT-driven solutions produced high-quality distractors compared to ranking-based models. They are also more reliable than the ranking-based model as they produce significantly less number of nonsense distractors. 
\thms{We also combined the rank-based approach with ChatGPT, through the automatic composition of an example-based prompt from the output of the rank-based model.  We found that this}
%We also found combining local ranking models with ChatGPT 
leads to a more reliable and effective generation of distractors. The contribution of this paper can be summarized as follows:

\begin{itemize}

    \item We proposed a strategy to guide LLMs, specifically ChatGPT, to generate effective distractors for MCQs across various subjects by prompting the model with question items automatically retrieved from existing question banks. 
    \item We performed a user study with teachers to evaluate the quality of distractors proposed by our strategy.
    \item The evaluation of our approach unveils its dual capability to generate valuable distractors while simultaneously minimizing the occurrence of nonsensical options. 
    %This reduction in nonsensical distractors contributes to the enhancement of user trust in the reliability and competence of the model.  
\end{itemize}




%\paragraph{\semere{Structure of the paper: }} 
The remainder of the paper is organized as follows: \cref{sec:related} describes the relevant work in distractor generation and LLM prompting strategies. \Cref{sec:method} explains the details of the baselines and the proposed method, while \cref{sec:experiments} introduces the test dataset and the evaluation setup of the user study with teachers. In \cref{sec:expertevaluation}, we report the results and provide some insights. Finally, in \cref{sec:conclusion}, we present the conclusion by summarizing the key findings and implications of our study.

%Finally, in Section 6, we present the concluding remarks summarizing the key findings and implications of our study.

%%%%%%%%%%%%%%%%%%%%%%%%%%%%%%%%% Begin of old introduction %%%%%%%%%%%%%%%%%%%%%%%%%%%%%%%%

%\todo[inline]{Three sentences answering the three research questions.}

%Now ChatGPT is there vs. there's this very recent work (same month as ChatGPT??) on using local LMs for distractor selection; is this made obsolete?  To be measured on the same scale, evaluated by experts; also: can both approaches (local model + ChatGPT prompting) be combined into even more effective approach? 
%HOw good is naive ChatGPT? and how could we improve it?

%\paragraph{Objective of this paper: }

%\paragraph{Importance of MCQ:} The recent shift towards online learning at scale has presented new challenges to educators, including the need to generate questions and exercises at scale for formative and summative assessment. Among the various question formats utilized on online platforms, multiple-choice questions (MCQs) are the most commonly used questions due to their automatic scoring capability and the potential they hold for delivering timely and targeted feedback, which is crucial for facilitating effective learning~\cite{ramsden2003learning}.

%\paragraph{Role of LLM in MCQ construction:} The rapid rise of large language models (LLMs) and advancements in artificial intelligence (AI) presents an opportunity to automate the MCQ construction process in education with enhanced quality and efficiency. For instance, teachers could employ LLMs to not only create different variants of the same MCQ questions but also develop different MCQs of comparable difficulty levels, facilitating targeted assessment for students with similar proficiency levels. Furthermore, students can benefit from the availability of several MCQs, enabling them to engage in regular practice, which is a well-established and highly effective learning strategy~\cite{roediger2006test}. Additionally, such models could be used for large-scale testing contexts (\eg licensure and certification testing) in which it is necessary to have multiple forms of a test and to introduce new question items regularly to minimize security concerns related to item exposure. %Licensure and certification testing may have these requirements, as my other large-scale progress testing and self-assessment programs.   

%\paragraph{ChatGPT in Education:} In particular, ChatGPT, a recently released AI model, has demonstrated diverse potential in automating various educational tasks. For example, ChatGPT has achieved the equivalent of a passing score for a third-year medical student (above 60\%) in the United States Medical Licence Examination (USMLE) Step 1 exam, and provided logical justification and informational context across the majority of answers~\cite{gilson2023does}. Likewise, ChatGPT's performance on four real exams (containing 95 MCQs and 12 essay writing questions), at the University of Minnesota Law School was equivalent to C+ students implying a pass in the course~\cite{choi2023chatgpt}. \semere{~\cite{li2023can}, shown the} capability of ChatGPT in generating high-quality reflective responses in writing assignments administered for pharmacy courses.    

%\todo{Include mitte's information on exposing students to incorrect information/statements could backfire.}
%\todo{- Less Nonsense distractor potentially means trustworthy AI models\\
%- To make education safer by building efficient and effective models. Such models should be deployed  rather than replacing teachers. For example,  distractor generation tools, for example, pieter less
%\\
%- Trust of teachers/reliability on such models could be improved as well}


%\paragraph{Objective of this paper: }In this paper, we seek to investigate the capabilities of ChatGPT, to generate distractors -- \ie plausible yet incorrect answers -- for MCQs in education. Although generating distractors might seem like a minor part of the full MCQ construction process, item teachers (\ie teachers, content creators \etc) often report that it is both a challenging and time-consuming task. Poorly prepared distractors can undermine the quality of MCQs~\cite{gierl2017developing}. 

%\semere{We aim to leverage ChatGPT to generate effective distractors ultimately creating high-quality MCQs, thus providing educators and researchers with new insights. In addition, it remains unknown whether/how AI-generated MCQs can be made safe for students because wrong information could have a lifelong effect on students' learning progress. This may be particularly important for educational stakeholders aiming to integrate LLMs such as ChatGPT in their learning/assessment platforms and trust on the models by its users.}

%Understanding these challenges is essential for translating research findings into educational technologies that stakeholders (e.g., students, teachers, and institutions) can use in authentic teaching and learning practices [1]
%A recent variant of the GPT model developed by OpenAI, i.e., ChatGPT, has become extraordinarily popular since its launch in November 2022. Compared to its predecessors, the significant step forward with ChatGPT hinges on the extra human-guided fine-tuning for the conversational context. This specific training allows ChatGPT to generate more natural-sounding and context-specific responses. Therefore, we posit that ChatGPT holds the potential to advance question generation and distractor generation.

%ChatGPT is a successor of the large language model (LLM) InstructGPT (Ouyang et al., 2022) with a dialog interface that is fine-tuned using the Reinforcement Learning with Human Feedback (RLHF) (Christiano et al., 2017) approach. 1
%%%%%%%%%%%%%%%%%%%%%%%%%%%%%%%%% end of old introduction %%%%%%%%%%%%%%%%%%%%%%%%%%%%%%%%





%---------------------------------------
\section{Related Work}
\label{sec:related}
%=======================================
%add disclaimer: that general chatgpt in education was discussed in the intro.

Since we briefly covered the broad application of LLMs in education in the introduction, in this section we only focus on describing prior works on distractor generation (\cref{sec:distractorgeneration}) and discussing LLMs' prompting strategies (\cref{sec:promptingstrategies}) and their relevance to our work. %The introduction covers the broad application of LLMs in education, which we have revised and omitted from this section for brevity.


%---------------------------------------
\subsection{Distractor Generation}
\label{sec:distractorgeneration}
%---------------------------------------

%Our work is related to the task of generating incorrect options (\ie distractors) for  multiple-choice question (MCQ) items. Generating plausible but inherently wrong distractors is time-consuming and may also determine the quality of the MCQ, and thus has been subject to research and study by researchers.The main approaches to generating distractors can be broadly divided into two strands; retrieval-based and generation-based methods. 
%The present study pertains to the issue of generating erroneous options, commonly referred to as distractors, for multiple-choice question (MCQ) items. The task of generating viable yet incorrect distractors is not only time-intensive but also plays a pivotal role in determining the quality of MCQs, thereby necessitating extensive inquiry by researchers. Broadly speaking, the generation of distractors is approached through two main methods, namely retrieval-based and generation-based techniques.
% This study focuses
\chris{We focus} on generating incorrect options (\ie distractors) for multiple-choice questions (MCQs)%. It's
\chris{, which is} a time-consuming task that impacts MCQ quality and has been extensively researched. Broadly speaking, the main methods for generating distractors can be categorized into retrieval-based and generation-based techniques. %Retrieval-based methods assume access to existing corpus of question items or ontologies.  use the answer (or answer and question stem) as the query to retrieve distractors from existing ontologies or pool of question items. 

\emph{Retrieval-based methods} generate distractors by selecting the most similar alternative answers in existing knowledge bases or question item corpora.
To approximate the similarity between distractors and the answer key (and question stem), several approaches are used %\chris{:} %including 
based on
\begin{enumerate*}[(i)]
\item embedding space proximity~\cite{bitew2022learning,jiang2017distractor,guo2016questimator},
\item similarity in lexical databases such as WordNet~\cite{miller1995wordnet}, which is of particular importance in language and vocabulary learning~\cite{mitkov2009semantic,pino2008selection}, and
\item the semantic distance within domain-specific ontologies, which is critical in factoid-type questions~\cite{leo2019ontology,faizan2018automatic,alsubait2014generating,papasalouros2008automatic}.
\end{enumerate*}
This ultimately leads to the selection of candidate distractors based on a ranking strategy~\cite{liang2018distractor}.
%similarity in lexical databases such as WordNet~\cite{} important in language and vocabulary learning, semantic distance in domain specific ontologies, which lead to selection of candidate distractors according to a ranking strategy.

%Retrieval-based methods are employed to generate distractors by selecting alternative answers that closely resemble those found within existing knowledge bases or question item corpora. To approximate the similarity between the distractors and the answer key, as well as the question stem, several approaches are used, including closeness on embedding spaces, lexical databases such as WordNet, which is of particular importance in language and vocabulary learning, and the semantic distance within domain-specific ontologies, which is critical in factoid-type questions. This ultimately leads to the selection of candidate distractors based on a ranking strategy.

\emph{Generation-based methods} make use of deep learning models to directly generate distractors.
Pioneering research~\cite{gao2019generating,yeung2019difficulty,zhou2020co} demonstrated the feasibility of using sequence-to-sequence models to generate distractors%.
\chris{, while more recently, solutions based on BERT~\cite{chung-etal-2020-bert,kalpakchi-boye-2021-bert} or T5~\cite{rodriguez2022end} have been explored.}
%employed BERT encoders for the task, while~\cite{rodriguez2022end} fine-tuned pretrained T5 model to generate distractors.
\chris{Rather than directly (auto-regressively) generate a distractor, the technique of back translation has shown to be relatively \thms{effective} (beating a BERT-based baseline) for fill-in-the-blank language assessment tests~\cite{panda-etal-2022-automatic}.}
%\cite{panda-etal-2022-automatic} used back-translation for generating distractors in fill-in-the-blank questions.


In this work, we investigate the potential of ChatGPT\footnote{\url{https://chat.openai.com/}}, a large and autoregressive language model, in creating distractors.
%We aim to combine retrieval-based and generative-based approaches by incorporating the automatic retrieval of similar question items as in-context examples while generating distractors using ChatGPT.
%We aim to combine in a pipeline the retrieval-based and the generative-based approaches by prompting ChatGPT with dynamic in-context examples, which are automatically retrieved from a question bank.
We aim to combine retrieval-based and generative-based approaches by \begin{enumerate*}[(i)]
\item automatically retrieving similar question items from pre-existing question banks to compose an example prompt and \item using this example prompt to guide ChatGPT to generate relevant distractors.
    
\end{enumerate*}  %that guides ChatGPT to generate relevant distractors. 

%We aim to combine the retrieval-based and generative-based approaches by first automatically retrieving question items that are similar to the question we want to generate distractor for. Then, we use 

%\textbf{New papers since 2022}
%\begin{itemize}
 %   \item 
%\end{itemize}


%General content
%\begin{itemize}
%    \item Distractor Generation in terms of multilingualism/domain/techniques
%\end{itemize}
%\subsection{Large Language Models}

%\begin{itemize}
%chris: NLP in Education- general overiew

%probmles - general solution - how they applied --> domain then zoom in to distractor generation.


%    \item Commercial EdTech products using ChatGPT 
%    \item Scientific publications in the context of language learning/ MCQ generation/ other educational domains
%    \item Prompting LLMs -- few-shot/zero-shot, chain-of-thought
%\end{itemize}


%------------------------------------------
\subsection{Prompting strategies}
\label{sec:promptingstrategies}
%------------------------------------------
% Language models have been shown to perform various downstream tasks without having to %without necessitating fine-tuning on the target task 
% fine-tune them on the target task~\cite{NEURIPS2020_1457c0d6,radford2019language}. 
% This is achieved by prompting the models with either textual instruction (zero-shot) or a few examples (few-shot) that %requires no gradient update.
% \chris{does not require a gradient update.}
% The test example's input is shown to the model in zero-shot prompting.
% In contrast, in few-shot learning (a.k.a\chris{.} in-context learning), few examples of the target task are prepended along with a test example's input.
% These few examples are also called demonstrations.
\chris{Recent instruction-based large language models (LLMs) have been a game-changer for various tasks, showing remarkable performance without any task-specific training (\eg through finetuning) of the LLM~\cite{brown2020language,radford2019language}.
A specific task is solved through phrasing an instruction (zero-shot), possibly including a few input/output examples (few-shot) for the task at hand, as the so-called prompt that serves as input to the LLM.
The few-shot setting, including some examples, is commonly referred to as in-context learning (ICL).}
Another prompting strategy, chain-of-thought, induces language models to generate intermediate steps before predicting the final response~\cite{wei2022chain}.

In this paper, we introduce a variant of %in-context learning
\chris{ICL} wherein the examples presented to the %model
\chris{LLM} are %dynamically 
determined \chris{dynamically,} based on the test example \chris{(\ie the question to generate distractors for, in our case)}.    
%Zero-shot learning aims to solve unseen tasks without labeled training instances. 


%-- Provide a minimal description of what are we talking about? 
%topic types/ how teachers are annotating/ 


%------------------------------------------
\section{Methods}
\label{sec:method}
%==========================================
% In this section, we describe in detail our baselines and the proposed strategy for generating distractors. 
%\chris{We now describe our
%\begin{enumerate*}[(i)]
%    \item finetuned T5-based model (\cref{sec:t5-distractor-generation}),
%    and out-of-the-box ChatGPT-based solutions in a 
%    \item zero-shot setting (\cref{sec:zero-shot-chatgpt}), as well as using
%    \item in-context learning (\cref{sec:demonstration-chatgpt}).    
%\end{enumerate*}}
We now describe our finetuned T5-based model (\cref{sec:t5-distractor-generation}), and out-of-the-box ChatGPT-based solutions in a zero-shot setting (\cref{sec:zero-shot-chatgpt}), as well as using in-context learning (\cref{sec:demonstration-chatgpt}).    


%------------------------------------------
\subsection{T5-based Distractor Generation}
\label{sec:t5-distractor-generation}
%------------------------------------------

%As our first baseline, we
\chris{We} fine-tuned a multilingual T5 (mT5) model~\cite{xue-etal-2021-mt5} to generate distractors.
%To accomplish this, we utilized the Televic dataset~\cite{bitew2022learning}, a comprehensive educational resource consisting of approximately 62K multiple-choice question items in the form of question, answer and distractors triplets.
\chris{To this end, we use a private dataset (\ie the Televic dataset from ~\cite{bitew2022learning}) of 62K multiple-choice question items in the form of triplets comprising a question, answer and distractors.}
These question items are diverse in terms of language, domain, subject and question type. \semererev{On average, a question item has more than 2 distractors and contains exactly one answer. Additionally, the distractors in the dataset are not limited to single-word distractors. } %, which made them ideal for our purpose. 
% chriscom: sounds more like a mask than a sentinel? (you also use 'mask' in the fig; so I'd rather also use that term in the text. 'sentinel' suggests more of a separator/marker...)

Following the unsupervised pre-training objective used in the mT5 model, we rearranged our fine-tuning data into input and output sequences as illustrated in \cref{fig:mt5template}.
Our mT5 model's input sequence is constructed by copying the question stem and answer from the original question item and inserting the sentence \emph{``Which of the following are incorrect answers''} (or its translation depending on the language of the question item) between them.
Furthermore, we masked each distractor (\ie distractors could be multi-word spans) in the question item using a sentinel \chris{token}\footnote{Each sentinel token is assigned a token ID that is unique to the sequence.
The sentinel IDs are special tokens added to the model's vocabulary and do not correspond to any wordpiece} %token
and separated them by %consecutive counting numbers, which denote the total number of distractors present.
\chris{increasing item numbers}.
%Each distractor (\ie distractors could be multi-word spans) in the question item first prepended with a number (\ie 1,2,3 \etc, depending on the number of distractors in the question item) and is masked by replacing it with a sentinel\footnote{Each sentinel token is assigned a token ID that is unique to the sequence. The sentinel IDs are special tokens added to the model's vocabulary and do not correspond to any wordpiece} token. 
The target sequence corresponds to all the dropped-out distractors and the objective is to predict the distractors.

% Figure environment removed

%\semere{Our fine-tuning setup is designed to make generating multiple distractors easy because all required distractors per question are generated in a single decoding step as a number-separated list. } 
The fine-tuning configuration that we have devised is intended to simplify the generation of multiple distractors. Specifically, all the necessary distractors for each question are generated as a list separated by numbers in a single decoding step.
%Our fine-tuning setup is designed to make it easy to generate multiple distractors because all required distractors per question are generated in a single decoding step as a number-separated list. 

%\begin{tcolorbox}[title=mT5 fine-tuning template, center title]

%{\small \{Question\} \textsf{</s>} Which of the following are incorrect answers? \textsf{</s>}  \\ 
%1. \{Answer\} \\
%2. \{Mask 1\} \\
%3. \{Mask 2\} \\
%}
%\end{tcolorbox}
%\begin{tcolorbox}[colback=white, colframe=black, coltitle=white, fonttitle=\bfseries\large, title={My template}, center title]
%    Hello world
%\end{tcolorbox}

%------------------------------------------
\subsection{Zero-shot ChatGPT}
\label{sec:zero-shot-chatgpt}
%------------------------------------------
%As the second baseline, we use ChatGPT in a zero-shot setting where
\chris{To use ChatGPT in a zero-shot setting ({\zerochatgpt}),} we %condition it on a concatenation of 
\chris{construct a prompt that concatenates} a fixed instruction sentence and the test example\chris{,} as shown in \chris{\cref{fig:zero-shot-ChatGPT}.} %the template below.
%Specifically, we first prepend the question and the answer by the tokens \emph{question:} and \emph{answer:}, respectively.
%Then we add the instruction sentence \emph{``Generate 10 plausible but inherently incorrect answers.''} (or its translation depending on the language of the question item, \ie Dutch or French) at the beginning of the prompt. 
%ChatGPT is a successor of the large language model (LLM) InstructGPT (Ouyang et al., 2022) with a dialog interface that is fine-tuned using the Reinforcement Learning with Human Feedback (RLHF) (Christiano et al., 2017) approach. 1
Note that each time a new query is made to ChatGPT, we clear conversations to avoid the influence of previous samples through independent API calls. We use a Python ChatGPT wrapper\footnote{Note that all the calls to the API were made between 06/04/2023 and 11/04/2023. Link to wrapper: \url{https://github.com/mmabrouk/chatgpt-wrapper}} to call the ChatGPT API automatically.  %(\ie we open a new session).
%\todo[inline]{Add API details}
%Note that each time a new query is made to ChatGPT, we clear conversations to avoid the influence of previous samples.
%We used a Python ChatGPT wrapper\footnote{\url{https://github.com/mmabrouk/chatgpt-wrapper}} to automatically call the ChatGPT API. All calls to the API were made between 06/04/2023 and 11/04/2023.



%\subsection{In-context learning of ChatGPT with retrieved demonstrations}
% Figure environment removed



%------------------------------------------
\subsection{Demonstration-based ChatGPT }
\label{sec:demonstration-chatgpt}
%------------------------------------------

%In this study, we assess the performance of ChatGPT in a few-shot scenario by subjecting it to carefully selected demonstrations. To this end, we suggest retrieving the most relevant question items from the Televic dataset (as described in Section 3.1) and using them as demonstrations for a particular test instance. We accomplish this by leveraging the question similarity (Q-SIM) model introduced by Bitew et al. (2022) to automatically identify the most similar question items for the given test instance. Subsequently, we incorporate these examples to generate the distractors.
Finally, we evaluate ChatGPT in a few-shot setting by probing it with smartly chosen demonstrations ({\demochatgpt}).
We propose to retrieve the most relevant question items from the Televic dataset (see \cref{sec:t5-distractor-generation}) and use them as demonstrations for a given test instance. We accomplish this by leveraging the question similarity (Q-SIM) model proposed by~\cite{bitew2022learning} to automatically select the top similar question items for the given test instance. \semererev{The Q-SIM model is a BERT-based ranking model that returns a ranked list of question items according to their similarity to a given test question.}
%We augment the test instance with these examples and an instruction sentence \emph{``Generate 10 incorrect answers for the following question''} (or its translation depending on the language of the test instance), before sending it as a query to ChatGPT to generate distractors as illustrated in \cref{fig:fewshot}.
\chris{\Cref{fig:fewshot} illustrates how we combine the original question (to generate distractors for) with the retrieved examples into a prompt to ChatGPT.}

% Figure environment removed


%\begin{tcolorbox}[title= Few-shot prompt template, center title]
%{\small question: What is the capital of Belgium? \\ 
%answer: Brussels  \\ 
%distractors: 1. Ghent 2. Antwerp 3. Amsterdam \\
%Generate 20 plausible but inherently incorrect answers.}
%\end{tcolorbox}


%----------------------------------------
\section{Experiments}
\label{sec:experiments}
%========================================

%----------------------------------------
%\subsection{Evaluation Dataset}
\subsection{Test Dataset}
%----------------------------------------

%The proposed strategies are evaluated using the Wezooz test data, as introduced by Bitew et al. (2022). This test data comprises 300 multiple-choice questions (MCQs) designed for language and factual knowledge learning, aimed at secondary school students and teachers. The test includes French and English questions intended for language learning, while Natural sciences, Geography, History, and Biology constitute the factoid questions. Each subject category has 50 MCQs. Notably, the factoid questions in this test data have a different distribution from the Televic dataset, as mentioned in Section 3.1, which was utilized to fine-tune our mT5 model. However, the language learning questions in the Wezooz test data are drawn from the same distribution.
\chris{To quantitatively evaluate our distractor generating models introduced in \cref{sec:method}, we}
%We
use the Wezooz test data introduced by \cite{bitew2022learning}\chris{, which} % to evaluate the proposed strategies. %This test data contains 300 multiple-choice questions (MCQ) for language and factual knowledge learning targeting secondary school students and teachers.
%This test data
comprises 300 multiple-choice questions (MCQs) designed for language and factual knowledge learning %,
\chris{and is} aimed at secondary school students and teachers.
It includes French and English questions for language learning purposes, while Natural sciences, Geography, History and Biology constitute the factoid questions.
Each subject has 50 MCQs.
Note that the data distribution of the factoid questions is different from the Televic dataset (see \cref{sec:t5-distractor-generation} for details)% used 
\chris{, which we use to}
\begin{enumerate*}[(i)]
\item fine-tune our mT5 model, and
\item %in retrieving
\chris{retrieve} similar examples in our demonstration-based ChatGPT model.
\end{enumerate*}
However, the language learning questions are drawn from the same distribution, in a similar design setup as~\cite{bitew2022learning}.


%--------------------------------------
%\subsection{Expert Evaluation Setup}
\subsection{Human Expert Quality Assessment}
\label{sec:expert-eval-setup}
%--------------------------------------
%In \cref{sec:method}, we presented various models for generating distractors. 
%To evaluate %their
% chris{the} quality \chris{of our distractor generating models introduced in \cref{sec:method}}, we solicited feedback from teachers.
\chris{We also investigated our models' output quality using human assessors, by collecting feedback from teachers.}
For each %question in the test set, we used each model to produce 10 distractors.
\chris{of the 300 questions in the aforementioned WeZooz test set, we generated 10 distractors with each of our 3 models.}
The teachers were then presented with a randomized list of all %the
\chris{30} generated distractors for each question.
They were explicitly instructed to rate each distractor independent of the other distractors in the list\chris{,} based on how much they thought it would help them if they were given the task of preparing distractors for that specific question. %its perceived usefulness in preparing distractors for that specific question.
We used %a
\chris{the} four-level annotation scheme proposed by \cite{bitew2022learning} to assign quality labels to each distractor\chris{:} %. The labels are:
\begin{enumerate*}[(1)]
    \item \textbf{True Answer}: the distractor partially or completely overlaps with the answer key.
    \item \textbf{Good distractor}: the distractor is viable and could be used in an MCQ as is.
    \item \textbf{Poor distractor}: the distractor is on topic but could easily be ruled out by students. %This could happen due to one or both of the following reasons.
    \item \textbf{Nonsense distractor}:  distractor is completely out of context.
\end{enumerate*}
%, which the teachers assessed independently of the other distractors in the list.

%We asked teachers to rate the quality of distractors generated by the different models introduced in \cref{sec:method}. Using each mode, we generate 10 distractors for each question in the test set. Teachers were then shown the unified distractor predictions coming from all models in a random order for each test question. Teachers were explicitly instructed to rate each distractor based on how much they thought it would help if they were given the task of preparing distractors for that specific question. Specifically, we asked them to assign quality labels to each distractor independent of the other distractors in the list according to a four-level annotation scheme proposed in \cite{bitew2022learning}. The labels are: 
%\begin{itemize}
 %   \item \textbf{True Answer}: specifies that the distractor partially or completely overlaps with the answer key.
%\item \textbf{Good distractor}: specifies that the distractor is viable and could be used in an MCQ as is.
%\item \textbf{Poor distractor}: specifies that the distractor is on topic but could easily be ruled out by students. This could happen due to one or both of the following reasons.

%\item \textbf{Nonsense distractor}: specifies that the proposed distractor is completely out of context
%\end{itemize}







%- how well the two are aligned and different in terms of distribution (wezooz and Televic) 
%- homogeneity/ format type , wezooz well-curated, the other coming from different sources (kinda) 


%This data is a small-scale test set of questions gathered from WeZooz Academy, 7 which is a Flanders based company providing an online platform with digital teaching materials for secondary school students and teachers. We selected four subjects; Natural sciences, Geography, Biology and History. Each subject was made to contain a fixed list of 50 questions that were randomly selected, and augmented with distractor annotations by teachers for these respective subjects (see Section 4). Note that this is an external test set, in the sense that the data distribution in the training set is not necessarily representative for this test set. This serves as a proof-of-concept for the general validity of our proposed method and models to specific use cases.



%------------------------------------------
\section{Results and Discussion}
\label{sec:resultsanddiscussion}
%==========================================

\begin{table}[t]
\small
\begin{center}
\caption{Inter-annotation agreement of experts% in terms of
\chris{, measured by the} Jaccard similarity coeffic\chris{i}ent\chris{.}}% (\%)\chris{.}}
\label{tab:interannotation_jaccard}
\setlength{\tabcolsep}{6pt}
\begin{tabular}{lccccc}
\toprule
\textbf{Subjects} & 
\multicolumn{1}{c}{\textbf{True}} &
\multicolumn{1}{c}{\textbf{Good}} &
\multicolumn{1}{c}{\textbf{Poor}} &
\multicolumn{1}{c}{\textbf{Nonsense}} & 
\multicolumn{1}{c}{\textbf{Overall}} 
\\\midrule

English & 50.0 & 37.6 & 8.9 & 40.0 & 49.4\\
Geography & 33.3 & 75.7 & 34.1 & 35.0  & 74.0\\
%\midrule
%Cohen's Kappa  & 52& & & & \\
\bottomrule
\end{tabular}
\end{center}

\end{table}

In this section, we provide evidence of the effectiveness and reliability of our approach by reporting the experimental results and discussing the insights obtained. In \cref{sec:interannotatoragreement}, we explain the annotation agreement among the teachers, followed by the evaluation results in \cref{sec:expertevaluation}. %Note that all the numerical results reported in this section are in percentage points. 

%------------------------------------------
\subsection{Inter-annotator agreement}
\label{sec:interannotatoragreement}
%------------------------------------------

Following the annotation scheme introduced in \cref{sec:expert-eval-setup}, a total of 12,860 ratings for distractor quality were collected from the annotation by teachers (see \cref{tab:ratings-data-description} in \cref{sec:appendix} for details of rating statistics). These ratings come from 10 distractors generated by each of the models (\ie all presented simultaneously to teachers as randomly shuffled list). \semererev{In total, 10 teachers participated in our quality assessment study.}

We adopt two strategies to determine the level of agreement between annotators. First, we ask teachers to rate the same set of distractors using the four-level annotation scale. We selected the subjects English, from language category, and Geography, from factoids, for annotations by at least two teachers. \Cref{tab:interannotation_jaccard} shows the inter-annotator agreement of teachers using the Jaccard similarity coefficient. The Jaccard similarity measures the similarity between two sets of data by calculating what fraction of the union of those datasets is covered by their intersection. In our case, it is calculated as the number of times the teachers agreed on a distractor quality label (\ie one of the four labels), divided by the total number of distractors that were annotated (by either annotator) with that label. \semere{In general, we note a higher agreement on what is considered a good distractor compared to the other distractor categories. Moreover, the overall agreement between the Geography teachers is higher than the English teachers.} 

\semere{Second, we employed the widely utilized Cohen's kappa coefficient \cite{mchugh2012interrater}. Our analysis substantiates the previously mentioned observation that annotators have a greater consensus when evaluating factoid questions compared to language-related queries as \cite{bitew2022learning}. Specifically, among English teachers, the calculated Cohen's kappa value stands at 28.9, signifying a ``fair agreement'' level. Similarly, Geography teachers exhibit a higher level of agreement with a Cohen's kappa value of 52, indicating a level of agreement categorized as ``moderate.'' %This is in line with the findings of Bitew \etal~\cite{bitew2022learning} that teachers agree more on factoid type questions than language distractors.
}





%------------------------------------------
\subsection{Evaluation of models}
\label{sec:expertevaluation}
%------------------------------------------

\begin{table*}[t]
\centering
\caption{Expert evaluation of distractors (\%). GDR: good distractor rate, NDR: nonsense distractor rate; $\uparrow$: higher is better, $\downarrow$: lower is better; evaluation on WeZooz test set. The markers $\star$ and $\ddag$ respectively denote the one-tailed significance levels of the bootstrap-based $p-$value, \ie $p<$ 0.1 and $p<$ 0.01 with respect to the best model {\demochatgpt} in each column. }
\label{tab:results_expert_eval}
\setlength{\tabcolsep}{4pt}
\small
\begin{tabular}{lcccc}
\toprule
\textbf{Models} & 
\multicolumn{2}{c}{\textbf{Language learning}} &
\multicolumn{2}{c}{\textbf{Factoid learning }} 
%\\\midrule
\\
\cmidrule(lr){2-3} \cmidrule(lr){4-5}
& \multicolumn{1}{c}{\textbf{GDR@10\,$\uparrow$}} &
\multicolumn{1}{c}{\textbf{NDR@10\,$\downarrow$}} &
\multicolumn{1}{c}{\textbf{GDR@10\,$\uparrow$}} &
\multicolumn{1}{c}{\textbf{NDR@10\,$\downarrow$}} 
\\
\midrule
{\dqsim} \cite{bitew2022learning} & 27.9$^\ddag$ & 44.6$^\ddag$ & 28.9$^\ddag$ & 50.1$^\ddag$   \\ 
{\mtfive} & 24.5$^\ddag$ & 42.3$^\ddag$ & 27.8$^\ddag$ & 36.6$^\ddag$ \\ 
{\zerochatgpt} & 30.2$\ddag$ & 34.6$\ddag$ & 57.6$\star$ & 17.5$\star$ \\ 
%\staticchatgpt{} & 43.3 & 16.2 & - & - \\ 
{\demochatgpt} & \textbf{46.7} & \textbf{15.5} &  \textbf{58.8} & \textbf{16.4} \\ 
\bottomrule
\end{tabular}

\end{table*}







\Cref{tab:results_expert_eval} shows the expert evaluation of distractors in terms of \emph{good distractor rate} (GDR@10), and \emph{nonsense distractor rate} (NDR@10). GDR@10 is calculated as the percentage of distractors that were rated `good' among the proposed 10 distractor for each model. Similarly, NDR@10 is calculated as the percentage of distractors that were rated `nonsense' among the 10 candidate distractors proposed by each model. We are interested in reporting the NDR metric %mainly because it is indicative of educational model's reliability and frequent occurrence of nonsense distractors may scare away users by eroding their trust in the model.
because it could be used as a measure of the reliability of educational models, as a high occurrence of nonsense distractors may undermine users' trust in the model. 
The reported metrics are averages of all the subjects in each category (\ie French and English for language learning, and Biology, Natural Sciences, History and Geography for factoids). In the table, the upward arrow ($\uparrow$) indicates larger values are desired, while the downward arrow$\downarrow$ indicates smaller values are preferred. 

In general, the ChatGPT-based solutions (\ie {\zerochatgpt}, and {\demochatgpt}) were rated better in proposing plausible distractors than the baselines. They also produced fewer nonsense distractors. Particularly, the {\demochatgpt} outperformed all the other models. On average, approximately 5 of its 10 proposed distractors were rated high-quality distractors and only 1.5 distractors were rated nonsense. Moreover, on average 8.5 distractors were generally found to be on-topic (\ie distractors rated as either good or poor distractors) for our best model {\demochatgpt}.

All the models are better at generating effective distractors for factoids than for language questions as shown by the higher GDR@10 results for factoids than languages. \semere{We hypothesize this is because, for factoid questions, our models are mainly tasked with generating accurately composed distractors that are contextually incorrect. In contrast, when faced with language questions, the intended distractors may possess ungrammatical attributes, posing a challenge for our models to generate text that is intentionally ungrammatical.}


%In contrast, when dealing with language questions, the desired distractors may exhibit ungrammatical characteristics making it difficult for LLMs to generate ungrammatical text.}% that are in themselves neither correct nor incorrect but rather more or less appropriate for the given context. Whereas in the languages distractors could be ungrammatical.}  

%On the other hand, when faced with language questions, the intended distractors may possess ungrammatical attributes, posing a challenge for Language Models (LLMs) to generate text that is intentionally ungrammatical.


Our purely generative local {\mtfive} model does not improve the {\dqsim} model (\ie previous state-of-the-art model on the test set) at proposing good distractors (\ie GDR@10 of 24.5 \vs 27.9 and 28.9 \vs 27.8). However, it is a more reliable model as it produces fewer nonsense distractors as illustrated by its lower NDR@10 values of 42.3 and 36.6 for languages and factoids, respectively, in contrast to the corresponding values of 44.6 and 50.1 for the {\dqsim} model. The relatively high number of nonsense distractors in {\dqsim} is partly attributed to its inherent limitation of only ranking pre-existing distractors according to their relevance to a given question, thereby lacking the ability to generate brand-new distractors.%the generative models.



%\paragraph{\textbf{Results bulletpoints}}
%\begin{enumerate}
%    \item \demochatgpt{} outperforms a ranking-based model \dqsim{}, which was the state-of-the-art model in the dataset
%    \item A purely generative local model \mtfive{} doesn't improve the \dqsim{} model on producing good distractors, however, it is better at generating fewer nonsense distractors.
%    \item Performance is better for factoids than languages for all the compared models. This is in line with what's perceived as a good distractor in language learning as shown by the less agreement between the English teachers.
%    \item The high number of nonsense distractors in \dqsim{} is not unexpected because it is a ranking model and is limited to sorting the available distractors in order of relevance to the given question. It is not capable of generating brand new distractors. 
%\end{enumerate}

\semere{In addition, in order to ensure the validity of the differences between the models, we carry out a bootstrap significance analysis~\cite{sakai2007evaluating} by sampling with replacement the annotation results {\dqsim}, {\mtfive}, {\zerochatgpt}, and {\demochatgpt} models 1000 times. The resulting one-tailed significance levels ($p$ values) are indicated in \cref{tab:results_expert_eval} by markers $\star$ and $\ddag$ which respectively denote $p<0.1$ and $p<0.01$ with respect to our best model {\demochatgpt{}} in each column.}


%\paragraph{\textbf{Effect of dynamically retrieved in-context examples}} To study the effect of using dynamically retrieved in-context examples, we randomly select in-context examples from the Televic question bank, and fix it (\ie {\staticchatgpt}). Particularly, we study this for the language learning category because of the huge jump in performance from {\zerochatgpt} to {\demochatgpt}. 

\paragraph{\textbf{Effect of dynamically retrieved in-context examples}} We replace the dynamically retrieved examples with randomly selected \semere{language} in-context examples from the Televic question bank, and we keep this selection constant (\ie {\staticchatgpt}) to generate distractors. Similar to the other models, we generated 10 distractors using the {\staticchatgpt} model and asked teachers to annotate the quality of the distractors. We focused on the language learning category as it showed a huge performance improvement when transitioning from {\zerochatgpt} to {\demochatgpt}.  %To investigate the impact of utilizing dynamically retrieved in-context examples, we replaced them with randomly selected $k$ in-context examples from the Televic question bank, and we keep this selection constant (\ie {\staticchatgpt}) to prompt ChatGPT. The focus of this study is specifically directed towards the language learning category, as it showed a huge performance improvement when transitioning from {\zerochatgpt} to {\demochatgpt}. We generated 10 distractors using {\staticchatgpt} methods and asked one English and one French to annotate the quality of distractors. 

We observe that the {\demochatgpt{}} model significantly outperforms the {\staticchatgpt{}} model in generating high-quality distractors as indicated by the GDR@10 metric in \cref{tab:ablation}. However, the difference in generating less nonsense distractor (\ie NDR@10) is not significant. See \cref{tb:generated_examples} in \cref{sec:appendix} for an example of generated distractors using the approaches.  
\begin{table}[t]
\small
\begin{center}
\caption{Effect of using dynamically retrieved in-context examples: \demochatgpt{} \vs \staticchatgpt{} that uses static in-context examples for language learning. The markers $\ddag$ denotes the one-tailed significance level of the bootstrap-based $p-$value, \ie $p<0.01$ with respect to {\demochatgpt}}
\label{tab:ablation}
\setlength{\tabcolsep}{6pt}
\begin{tabular}{lcc}
\toprule
\textbf{Models} & 
\textbf{GDR@10\,$\uparrow$} &
\textbf{NDR@10\,$\downarrow$} 
\\\midrule
{\staticchatgpt} & 43.3$\ddag$ & 16.2  \\ 
{\demochatgpt} & \textbf{46.7} & 15.5 \\ 
\bottomrule
\end{tabular}
\end{center}

\end{table}


\subsection{Discussion of Research questions}
To answer \textbf{RQ\ref{item:rq1}}, we compare the ChatGPT-based solutions (\ie {\zerochatgpt}, {\staticchatgpt{}} and {\demochatgpt{}}) with the previous state-of-the-art ranking-based model, {\dqsim{}} in generating distractors. All the ChatGPT-based distractor generation strategies significantly outperform the {\dqsim}.

To address \textbf{RQ\ref{item:rq2}}, we employ the NDR@10 metric as a proxy to measure the trustworthiness of models. Our best model produces an average of only 16\% nonsense distractors, which is a remarkable improvement compared to the previously reported state-of-the-art performance of 50\% NDR@10. This significant reduction of nonsense distractors can be expected to inspire more trust in the approach by teachers. 
%significantly decreasing the previously reported state-of-the-art 50\% NDR@10. This leads to better trust in the approach by teachers.   

To answer \textbf{RQ\ref{item:rq3}}, we compare {\demochatgpt{}}, which combines a local ranking model with ChatGPT, against {\zerochatgpt} and {\staticchatgpt{}}. As shown in \cref{tab:results_expert_eval} and \cref{tab:ablation}, combining local models with ChatGPT leads to a better quality distractor generation, highlighting the effectiveness of this combined approach. 

\section{Conclusion}
\label{sec:conclusion}
%\semere{This paper introduced and evaluated a strategy to guide LLMs such as ChatGPT to generate effective and reliable distractors for MCQ creation in education. Our proposed \demochatgpt{} model combines a rank-based approach with ChatGPT, by automatically predicting in-context examples that are used by ChatGPT to generate distractors. Importantly, our \demochatgpt{} showed a considerably reduced production of nonsense distractors (\ie only 16\% rated as nonsense) compared to \zerochatgpt{} (\ie out-of-the-box ChatGPT), which we consider a useful asset in terms of trust in the model by teachers. Moreover, on average, 5 out of the 10 distractors suggested by our approach were rated as high-quality by teachers, to be readily used. } 

This research paper introduced and evaluated a novel strategy designed to guide LLMs, such as ChatGPT, in generating reliable and effective distractors for the creation of MCQs in educational contexts. Our proposed approach, \demochatgpt{} model combines a rank-based approach with ChatGPT. This involves the dynamic retrieval of relevant question items through the ranker that are then presented as in-context examples to ChatGPT for generating distractors. Our results indicated that the \demochatgpt{} showed a considerably reduced production of nonsense distractors (\ie only 16\% rated as nonsense) compared to \zerochatgpt{} (\ie out-of-the-box ChatGPT), which we consider a useful asset in terms of trust in the model by teachers. Moreover, on average, 5 out of the 10 distractors suggested by our approach were rated as high-quality by teachers, to be readily used. 


\semererev{For future work, we aim to investigate designing a fine-grained evaluation setup for distractors that takes into account various factors such as the level of the student, the difficulty of the questions etc. There is also a potential to explore alternative prompting strategies for LLMS, when generating distractors. For example, the utilization of self-correcting mechanism~\cite{wang2023self}, which involves revising the initial output of an LLM by evaluating certain aspects of the text, could be explored in the context of distractor generation.}


%In our future research, our objective is to delve into the development of a detailed evaluation method for distractors, one that considers various factors, including the student's proficiency level and the difficulty of the questions. Additionally, there exists an opportunity to explore alternative strategies for Language Learning Management Systems (LLMS) when it comes to generating distractors. For instance, we could investigate the implementation of a self-correcting mechanism [33], which entails revising the initial output of an LLMS by evaluating specific aspects of the text. This approach holds promise for enhancing the generation of distractors.
%More specifically, starting with an initial output, CRITIC interacts with appropriate tools to evaluate certain aspects of the text, and then revises the output based on the feedback obtained during this validation process
%For future work, we first aim to adapt seq2seq models for our task particularly text-to-text models such as T5 (Raffel et al., 2020). There is also potential to explore different prompting strategies for large language models (LLMs), when generating gap-filling grammar exercises. For instance, the utilization of chainof-thought prompting (Wei et al., 2022), which involves generating intermediate steps before producing the final response, could be explored for generating grammar exercises. Additionally, an interesting future study would involve investigating the number of example demonstrations that LLMs require in order to accurately mimic example gap exercises.

%A key finding of our study was that the \demochatgpt{} model exhibited a significant decrease in generating nonsensical distractors, with only 16\% of the distractors rated as nonsensical, in contrast to the Zero-ChatGPT approach (i.e., the default, out-of-the-box ChatGPT). This reduction in nonsensical distractors is of paramount importance in establishing confidence and reliance in the model among educators.

%Furthermore, the results demonstrated that, on average, 5 out of the 10 distractors suggested by our approach were regarded as high-quality by teachers, making them readily suitable for implementation. This indicates the potential of our proposed methodology to enhance the MCQ creation process by providing meaningful distractors that align with educational objectives and contribute to the overall effectiveness and accuracy of assessments.

%We also combined the rank-based approach with ChatGPT, through the automatic composition of an example-based prompt from the output of the rank-based model.



%\todo[inline]{- Less Nonsense distractor potentially means trustworthy AI models\\
%- To make education safer by building efficient and effective models. Such models should be deployed  rather than replacing teachers. For example,  distractor generation tools, for example, pieter less
%\\
%- Trust of teachers/reliability on such models could be improved as well}


%\todo[inline]{For evaluating language distractors, a fine-grained class category needs to be designed that takes into account the level of the student, the difficulty of the questions etc}





















%
% BibTeX users should specify bibliography style 'splncs04'.
% References will then be sorted and formatted in the correct style.
%
\bibliographystyle{splncs04}
\bibliography{mybibliography}
\appendix
%\newpage
\section{User Study Details}
\label{sec:appendix}

This section contains the user study details. \Cref{tab:ratings-data-description} describes the data gathered from the annotations provided by teachers. Every subject contains 50 questions, except English which has 48 questions. We collected 12,860 annotations for the proposed candidate distractors (\ie 10 distractors by each of the three models). A total of 10 teachers participated in the study. English (\ie from languages) and Geography (\ie from factoids) were annotated twice by two different teachers to calculate inter-annotator agreement. \semere{Additionally, to study the effect of dynamic retrieval of in-context examples, we asked 1 English and 1 French teacher to annotate the distractor predictions from the {\staticchatgpt{}} model.} \semererev{The second column (\ie \emph{Item count}), shows the number of question items for each subject in the Wezooz dataset. Alongside, the \emph{distractors count} column provides two distinct values: the gold truth distractors count within the dataset, and the count of unique distractors generated by our models. It is important to note that different models may produce identical distractors for a given question, resulting in varying numbers of newly generated distractors across the different subjects.} %The \emph{distractors count} column shows the gold truth distractors count in the dataset and the unique number of the newly generated distractors using our models (\ie two models can generate the same distractor for a given question which is why we see different number of newly generated distractor for the different subjects.).} 

%Alongside, the \emph{distractors count} column provides two distinct values: the gold truth distractors count within the dataset, and the count of unique distractors generated by our models. It is important to note that different models may produce identical distractors for a given question, resulting in varying numbers of newly generated distractors across the different subjects.

\begin{table}[ht!]
\centering
\footnotesize
\begin{center}
\caption{Ratings Data Description }
\label{tab:ratings-data-description}

%\label{tab:annotationsdatadescription}
\setlength{\tabcolsep}{5pt}

\begin{tabular}{lccccccc}

\toprule
\textbf{Subjects} & 
\multicolumn{1}{c}{\textbf{Item count}} &
\multicolumn{2}{c}{\textbf{Distractors count }} &
\multicolumn{1}{c}{\textbf{Ratings count}} &
\multicolumn{1}{c}{\textbf{No of Raters}} &
\\\cmidrule(lr){3-4} %\cmidrule(lr){5-6}
& &
\multicolumn{1}{c}{\textbf{Gold}} &
\multicolumn{1}{c}{\textbf{Generated}} &
%\multicolumn{1}{c}{\textbf{Gold}} &
%\multicolumn{1}{c}{\textbf{New}} &
%\\

\\\midrule
English         &  48   & 130   & 1324      & 3360  & 3 \\ 
French          &  50   & 92    & 1396      & 2000  & 2 \\ 
Geography       &  50   & 145   & 1248      & 3000  & 2 \\ 
History         &  50   & 130   & 1354     & 1500  & 1 \\ 
Biology         &  50   & 88    & 1391      & 1500  & 1 \\ 
Nat. Sciences   & 50    & 100   & 1449     & 1500  & 1 \\
\midrule
Total           & 298   & 685   & 8162  & 12860   & 10  \\ 
\bottomrule
\end{tabular}
\end{center}

\end{table}

\section{Example Generated Distractors}
\label{sec:generatedexamples}

\semere{\Cref{tb:generated_examples} shows a cherry-picked example of our ChatGPT-based models. }%For \staticchatgpt and \demochatgpt{}, we only show two in-context examples in the prompt, while in reality, we use ten in-context examples for each case. }

\begin{table*}[t]

    \caption{Some generated examples from {\zerochatgpt}, {\demochatgpt}, {\staticchatgpt} models for English. High-quality distractors are shown in \textbf{boldface}, while on-topic and nonsense distractors are \textit{italicized} and \underline{underlined}, respectively. We only show 2 in-context examples for the \staticchatgpt and \demochatgpt models as part of the prompt but in practice, we use 5 of such examples.}
    \label{tb:generated_examples}
    \centering
    \setlength{\tabcolsep}{5pt}

    \begin{tabular}{p{0.1\linewidth} p{0.8\linewidth}}
\toprule

\multicolumn{1}{l}{\textbf{Model}} &  
\multicolumn{1}{c}{\textbf{Example}} \\

\midrule
\zerochatgpt & \textbf{Prompt}: Generate 10 plausible but incorrect answers for the following question. \newline
question: They do n't speak English and can not make ... understood. \newline 
answer: themselves      \newline
    %\textbf{Ground}: [] \newline
      \textbf{Ground}: [them, us,you]     \newline
      \textbf{Prediction}: [1. \emph{each other}, 2. \underline{any sense}, 3. \underline{a sentence}, 4. \underline{appropriate gestures}, 5. \underline{complete thoughts}, 6. \underline{proper communication}, 7. \underline{accurate translations}, 8. \underline{meaningful conversation}, 9. \underline{clear statements}, 10. \underline{intelligible sounds}]
       \\
\midrule
\staticchatgpt &\textbf{Prompt}: Generate 10 plausible but incorrect answers for the following question. \newline
question: Why do n't we ask Jane to come over and talk ? She 's an ... \newline
answer:  alumna \newline 
incorrect answers: 1. alumnus  2. alumni  3. alumn \newline 
question: Choose the opposite of the word in bold . This company is loss-making . \newline
answer:  profit-making \newline 
incorrect answers: 1. loose  2. lucrative  3. unprofitable \newline 

. . . \newline 


question: They do n't speak English and can not make ... understood. \newline 
answer: themselves      \newline
    %\textbf{Ground}: [] \newline
      \textbf{Ground}: [them, us,you]     \newline
      \textbf{Prediction}: [1. \emph{themselvess}, 2. \textbf{themselves} 3. \textbf{themselfs}, 4. \textbf{themselfes}, 5. \emph{themselfis}, 6. \emph{themselfies}, 7. \textbf{themselfves}, 8. \emph{themselfselves}, 9. \underline{themselfiness}, 10. \underline{themselfulated}
]
      \\
\midrule
\demochatgpt &\textbf{Prompt}: Generate 10 plausible but incorrect answers for the following question. \newline
question: They should choose ... would be best suited for the job \newline
answer:  which \newline 
incorrect answers: 1. whom  2. that  3. him who \newline 
question: 36 . He does not speak English and can not make ... understood .\newline
answer:  himself \newline 
incorrect answers: 1. you  2. us  3. herself \newline 
question: The new e-ticketing system , ... ... ... ... was installed last month , is working well . \newline
answer:  which \newline 
incorrect answers: 1. what  2. him  3. where \newline 


... \newline


question: They do n't speak English and can not make ... understood. \newline 
answer: themselves      \newline
    %\textbf{Ground}: [] \newline
      \textbf{Ground}: [them, us,you]     \newline
      \textbf{Prediction}: [1.\textbf{theirself}, 
2. \textbf{theirselves},
3. \textbf{themself},
4. \textbf{themselfs}, 
5. \textbf{theirs}, 
6. \textbf{us},
7. \underline{they}, 
8. \underline{them is}, 
9. \textbf{theyself}, 
10. \textbf{themselfes}
]
      \\
    \bottomrule
    
    
    \end{tabular}
\end{table*}



\end{document}





%
\title{Sustainable and adaptive structures in networked software systems}
%
%\titlerunning{Abbreviated paper title}
% If the paper title is too long for the running head, you can set
% an abbreviated paper title here
%
\author{Tihana Galinac Grbac\orcidID{0000-0002-4351-4082}
\and
Neven Grbac\orcidID{0000-0001-6657-6297}
}
%
\authorrunning{T. Galinac Grbac and N. Grbac}
% First names are abbreviated in the running head.
% If there are more than two authors, 'et al.' is used.
%
\institute{Juraj Dobrila Univeristy of Pula,
Zagreba\v{c}ka 30, HR-52100 Pula, Croatia \\
\email{\{tgalinac,neven.grbac\}@unipu.hr}
%\url{http://www.springer.com/gp/computer-science/lncs} \and
%ABC Institute, Rupert-Karls-University Heidelberg, Heidelberg, Germany\\
%\email{\{abc,lncs\}@uni-heidelberg.de}
}
%
\maketitle              % typeset the header of the contribution
%
\begin{abstract}
Sustainability is a globally shared goal and is used to drive technology evolution for the prosperity of people and the planet in the future. Software is central to modern technology solutions. The way how we develop, design, and deliver software solutions significantly impacts technology operation sustainability. Therefore, in this lecture, we aim to describe the challenges to reaching sustainable goals while developing, designing, and delivering software in the new network architecture, and explain the relationship between software autonomy and technology sustainability. In the central focus of the lecture are design principles for building autonomous software architectures as a key driver for sustainable technology evolution. Finally, we discuss practical examples and software engineering practices in relation to these challenges. Note that our particular interests are related to software delivered in Clouds, as blockchains of microservices within future networks.

\keywords{design principles \and autonomy \and sustainability \and software \and future networks}
\end{abstract}
%

\section{Introduction}
\label{sect:Intro}

Software is interconnected and highly integrated within telecommunication networks and is viewed as a key enabler for future digitalization, which is one of the main prerequisites for achieving sustainable goals in various domains (e.g. e-health, e-government, smart water, smart city, etc) \cite{DigitSust}. Future networks are preparing for autonomous machines (cars, drones, robots) and their continuous communication, in order to accommodate the continuous flow of massive amounts of data and their use for effective management solutions \cite{Ericsson5G}. Furthermore, there is a progressive development of applications that can make decisions with the help of artificial intelligence models. The idea is to converge to a global and highly interconnected network of things that continuously share massive amounts of data and autonomously adapt to environmental conditions \cite{AInetwork}. However, future networks should be designed not only to meet technical requirements for meeting people's purposes and achieving economic goals in various domains, but network designs should also look beyond these technical issues to foresee their impacts on environmental conditions aiming to co-exist on Earth over a long time.


Further evolution of information-communication technologies (ICT) should secure its sustainable operation in every domain of application. Network, as a central part of modern ICT technologies, is evolving by developing network autonomy in which sustainable network designs would be based on dynamic network adaptations fulfilling the aforementioned goals. In our previous lectures \cite{GalinacRole1,GalinacRole2}, we presented the main enabler technologies for network autonomy and its sustainable behavior, such as software-defined networking and virtual network functions. In order to achieve its full potential of optimized infrastructure and time-effective response to customer, networks have implemented these main architectural elements as enablers for future networks. The network has prepared its technology for rising its level of autonomy. More precisely, network autonomy rises with the wide implementation of self–management functions such as self–configuration, self–healing, self–optimization, self–protection, \cite{Autonomicbook}. Recent research trends focus on developing adequate software algorithms and solutions to support the aforementioned self-management functions, \cite{AInetwork,AICloudRM}. In this lecture, we will introduce key design principles that are necessary to guide the designs of self-management functions.


Evolution is driven by the idea to standardize and automatize processes within the systems so that we can systematically monitor and measure their behavior and minimize human intervention. One of the main goals of autonomic systems is to manage system complexity and introduce the system's self-organized capability. For such a purpose IBM has introduced a simple control loop frequently referred as Monitor, Analyze, Plan, and Control, MAPE, \cite{IBMMAPE}. The MAPE loop model assumes the existence of sensors within the system that can measure operations and the effectors that can issue system commands on the system's managed entity. This sequential autonomic control loop makes two key assumptions. Firstly, it assumes the existence of directly controlled system entities, and secondly, it assumes the standardized interfaces for system entity monitor and control. However, in most technical systems these conditions are not satisfied. Nowadays, software is often delivered and deployed over the underlying network infrastructure, in Cloud environments and it is often impossible to achieve direct monitoring and control of particular system entities. There is currently a vast amount of different software engineering frameworks and technologies that support developers in their system engineering activities to deliver software products within Network and Cloud environments (e.g. various Web application development frameworks like for example React, Angular, Vue, etc.). Software products are concurrently assessing the shared pool of network and Cloud resources.  The main problems arise with performance aspects under their concurrent operation when software products are offered to the numerous end users,\cite{PerfomanceReact,PerformanceWeb}. In such environments, the software product business goals are balanced between the number of users concurrently using these services and service performances achieved during concurrent software access to shared resources. Thus, in order to optimize their business goals, some in-service monitor and measure solutions are usually proposed, \cite{Webinserviceperf}. However, it is often impossible to implement in-service control of the underlying shared resource pool. From the network perspective, the coexistence of a variety of these technologies in the runtime environment makes significant operational problems in terms of their autonomic and sustainable management. It becomes challenging to automatically manage numerous technologies and find the balance between network and service layer self-adaptive behaviors. One of the main problems for future network evolution would be to find new architectural models, in which compromise between network and service operation would be achieved by satisfying their combined sustainable behavior. In the next decade, our goal should be to move from the \emph{Cloud computing} paradigm to the \emph{Responsive computing} paradigm, in which we should develop software engineering skills and tools that would enable engineering of sustainable technology systems.

One of the main problems is that we still engineer these systems from the technological vision perspective and not from the operational behavior, reliability, or sustainability perspective. We still lack software engineering technology for engineering these so-called *ility attributes into our systems. The same is valid for recently identified vital software characteristic that is related to its sustainable operation. Software architecture is the main artifact we engineer during the software development process and the results of our engineering approach is reflecting the operational characteristics of our final software product. For example, the way how we design software parts and their interactions (that we refer to as system structure design) has a significant impact on system performance, reliability, security, etc. Our approach is to reverse the traditional top-down deductive systems engineering approach into a bottom-up inductive way, and towards framing inside-out design patterns \cite{Design4finSus}. We would like here to bring attention to considering autonomic networking design principles when studying software structure designs for software sustainable operation. Architecture designs should introduce light system simulations that would observe system behavior through a sequence of accumulating events and concurrent executions. Moreover, designing systems architecture should aim to implement mechanisms that would be able to adapt at runtime. The essence of sustainability design is to prioritize internal needs over external needs while keeping the system execution within its limits. Networking design principles are biologically inspired principles that try to translate these nature's adaptivity driven by the instinct to survive into the network. Here sustainable skills would refer to students' abilities of systems thinking in these adaptive designs.


In this lecture, we will teach students about design principles as key skills that would be needed to shape future network and software delivery architectures. The lectures would be supported by the tools that we developed to enable the design of sustainable software structures. Therefore, in the next section Sect.~\ref{sect:dp}, we introduce the main sustainable design principles that are used to implement autonomic network behavior. In Sect.~\ref{sect:ss} we explain our approach to teaching and promoting sustainable design principles within software architecture design.


\section{Design principles for autonomic systems}
\label{sect:dp}

As a starting point of discussion, we need to clarify the basic terms and try to make clear distinctions among them. There are three frequently mixed terms: automatic, autonomous, and autonomic, \cite{Autonomicbook}.

The term automatic refers to an action that is a spontaneous reaction of the system not related to any guided rule but some stimulants are always resulting in the same automatically produced action. From the computer science and programming perspective, we can interpret this term in the sense of functional programming in which we want that functions are consistent and always return the same result on the given input.

An autonomic system is defined as a system that exhibits some degree of self-governance, however, the self--governance is achieved solely from the system's inside processes and has no relation to outside stimuli. The concept of the autonomic system is reused from biology (e.g. autonomic nervous system) and is representing only a part of the nervous system that is responsible for the control of spontaneous or autonomic functions of the human body that are preconditions to survive, like breathing, heart beating etc. These processes are executed outside the human conscience and represent continuous background human activity. It is important to understand that the human body spends a lot of energy to execute solely these processes, without any additional activity.

The autonomous system is viewed as a self-governed system based on some internally system-defined policies and principles. Moreover, this system is self-controlled and functionality independent of the outside world. Note that this concept assumes the system sensing ability which enables the autonomous system to develop its own knowledge and thus make further decisions and make control over the system. All living systems have some autonomous behavior and inherently built-in mechanisms to support such autonomous behavior.

The aforementioned concept of the autonomic system is derived into the communication network context with the help of autonomic design principles. These autonomic design principles have been widely studied while planning the network evolution and as a core philosophy for developing new network architectures with high network management autonomy. Although there are numerous viewpoints on sustainability principles, here we will reflect on sustainable principles for autonomic network behavior. The following principles were recognized as valid in networks context \cite{Autonomicbook} that we will briefly discuss in our lecture.
\begin{itemize}
    \item Living systems inspired design
    \item Policy-based design
    \item Context awareness design
    \item Self–similarity
    \item Adaptive design
    \item Knowledge-based design
\end{itemize}



\subsection{Living system inspired design}
\label{subsect:livingsys}
Software design should follow the behavior of living systems. All living systems exhibit high levels of autonomy and the designs of systems we engineer can be highly inspired from these biological examples. In particular, there are two perspectives to explore in the context of living systems. These are survivability and collective behavior.

Observation of the instinct to survive in the living systems has always concluded with the fact that living systems work toward keeping the equilibrium state and any deviations caused by environmental conditions are forcing the system to bring back to this initial system state \cite{LivingSyst}. There are numerous adaptation mechanisms of living systems that can be reused also for human engineered systems such as software architecture.

The other interesting mechanism of living systems is their collective behavior, which is about system social characteristics while aiming to adapt to a group to which it belongs. There are some interesting studies on the influence of networking technologies like Facebook on social collective behavior, \cite{Loc2glob}. It is interesting how some local information has gained significant importance from the numerous local pieces of information and taken dominance in guiding system global behavior. Some of these concepts may be useful while engineering sustainable software system structures.

\subsection{Policy based design}
\label{subsect:policy}
Policy-based design means that there exists a predefined rule that governs the system's behavior. This design principle has been already used in software system design in which system behavior must adhere to different rules of behavior and these behaviors are specified during the system design phase. Then, based on some parameter settings within the system, the system behavior is chosen at system runtime.

This approach has two drawbacks. Firstly, it requires policy definition during the design time, thus limiting system dynamic adaptation to environmental conditions. Furthermore, we have to secure adequate conflict resolution mechanisms that may be needed in the intersection of system behaviors driven by different policies. One example of such design is software that can serve the same purpose in different conditions, i.e., some standard protocol implementation that has variant implementations in different markets and for these markets, some standard protocol behavior involves some specific feature.



\subsection{Context awareness design}
\label{subsect:context}
Context-awareness has been already used within computer science and is a concept related to the ability of the system to characterize the environment and adapt its behavior to the current situation parameters and knowledge gained from its historic behavior under the same conditions. The main challenge in applying this design principle is in the selection of appropriate variables that may be measured, and sampling strategies to adequately model and capture all relevant states and their interaction in order to effectively and efficiently recognize relevant environment and situation conditions.


\subsection{Self–similarity}
\label{subsect:similarity}
Self-similarity refers to the similarity of system organization on different scales. More precisely, the self–similarity design principle is related to a characteristic that system organization persists as the various system scales and thus guarantees its global properties. Here, it is the main point to develop system functions that preserve system scalability. The same functions may be used as the system is growing in hierarchy and functionality. Thus, further system evolution may be systematically enabled by a proper system architecture that minimizes the need for system change during the system evolution. Moreover, the system properties and system behavior at a large scale have to be the same as the system properties and system behavior at a low scale.


\subsection{Adaptive design}
\label{subsect:adaptive}
Adaptive design is related to the ability of the system to adapt its inner behavior as a reaction to various environmental conditions. Such a system is able to learn from its experience in operation and react accordingly by adapting its actions based on collected information and knowledge gained. Over time the system acquires new knowledge and is better adapting to the local environmental conditions of its operation.

\subsection{Knowledge-based design}
\label{subsect:knowledge}

Design based on the knowledge extracted from big data gathered from the complex system (using AI and other models) differs from the previous adaptive design concept in terms that it assumes some global data collection and artificial global intelligence that can be used to guide local decisions.



\section{Content of the summer school lecture and educational goals}
\label{sect:ss}

Education in software architecture should prepare students for their profession in a new sustainable world. One of the primary steps is to integrate sustainability principles into architecture design courses.

We will formally start the lecture by revising traditional system design skills, and traditional teaching approaches. Here we will introduce some basic definitions. System design is concerned with system decomposition into its components. The design decisions are fundamental for successful implementation and evolution of the system, \cite{Vliet}. The main output of the design phase is the software structure that is defined as a set of components and their mutual dependencies, \cite{Vliet}. The system structure is usually depicted by the graph that represents 'the uses' relation among the system components, which is called the call graph. During the system design phase, there are no clear rules on how to perform successful system designs. The system design rather offers design principles that we reviewed in \cite{GalinacRole1} such as abstraction, modularity, information hiding, layering and hierarchy. Traditional designing system architecture involves exploring various system structure designs in terms of requirements defined in the requirements phase. However, here we aim to rethink this approach and explore widening this narrow design principle list with design principles defined in the autonomous systems concept and explained in Sect.~\ref{sect:dp}. We will face here two challenging issues.

Firstly, we need to introduce novel learning approaches to engage students' abstraction abilities. Some studies \cite{RoleofFluid} have identified that fluid intelligence has a significant role in biological phenomena abstraction and that understanding of bioinspired design is an essential part for design creativity.

Secondly, we promote here a system thinking design approach \cite{systdynamics} that is reflecting on migration from a top-down system design analysis to a bottom-up approach, in which the system structure is examined from the system dynamics perspective. The system designer has to take here a more active role, more as a system experimenter than just a passive critical structure analyst's observer. In such a changed role, the system designer should develop skills that are related to the following: modeling of system simulation by formulating simulation formulas and models, coping with the understanding of system behavior, evaluation of system policies by the development of a control group, development and selection of appropriate treatment cases, establishing a hypothesis, ability to monitor and compare the results for different treatments, ability to judge the results, develop models that may represent intervention actions. Software structure designs are usually governed by the technical vision and usually rely solely on human unreliable intuition. In this lecture, we will try to present our ideas on a simple toy example by using popular among the Academy Python technology.


At the summer school, we plan to structure lectures as follows:

\begin{itemize}
    \item Introduction to System Thinking and System Dynamics
    \item Software as a concurrent system: Python skeleton
    \item Simulating software behavior
    \item Development of simulation scripts
    \item Observing system dynamics within system boundaries
    \item Python GUI for user monitor
    \item Autonomic system design principles
    \item Implementation of the student version of sustainable design principle
\end{itemize}
Firstly, we plan to explain the meaning of control MAPE loop and present its implementation within the Python environment. Some implementations of MAPE loop in Python environment are available publicly\footnote{\texttt{https://github.com/elbowz/PyMAPE}}. We will demonstrate how to implement Python functions for monitoring and control of software behavior by using public Python libraries.

In the second part of the lecture, we will describe system thinking approaches and system dynamics perspective. As a practical example, we plan to ask students to implement a simplistic version of a Python application. Then, we will explain how to implement a Python simulation script that can be used to monitor Python application behavior. Furthermore, we present several practical software implementations of bioinspired design principles and provoke students to play with simulations and observe software behavior. Finally, we motivate students to develop their own solutions, so that students can easily experiment with their versions of implementation of autonomic design principles in a predefined context.

The intended audience for the lectures is undergraduate or graduate students of Computing study programs that have previous experience in Python programming. For active exercising, it is required that students have PC or laptops with installed Python IDE.

The key learning outcomes will be on understanding how to measure software behavior, the ability to develop its own simulation models for capturing software behavior and thus understand how to use a bottom-up approach to system design, and understanding how bioinspired design principles may improve self-system management and thus affects sustainable goals.



\section{Conclusion}
\label{sect:conc}
In this lecture we provide a bottom--up approach to software design integrating system thinking and system dynamics skills into the software engineering curriculum. We focus our system design on the implementation of autonomic system concepts within software applications and provide autonomic design principles to as guidance for students. Throughout the lecture, we undertake an exercise-driven approach to learning autonomic design principles and orientation to system thinking and system dynamics design.



\section*{Acknowledgments}

This paper acknowledges the support of the Erasmus+ Key Action 2 (Strategic partnership for higher education) project No. 2020–1–PT01–KA203–078646: “SusTrainable - Promoting Sustainability as a Fundamental Driver in Software Development Training and Education” and the support of the Croatian Science Foundation under the project HRZZ-IP-2019-04-4216. The information and views set out in this paper are those of the author(s) and do not necessarily reflect the official opinion of the European Union. Neither the European Union institutions and bodies nor any person acting on their behalf may be held responsible for the use which may be made of the information contained therein.

\begin{thebibliography}{99}

\bibitem{Autonomicbook}
N. Agoulmine (ed.), \textit{Autonomic Network Management Principles: from Concepts to Applications}, Elsevier Inc. (2011). \doi{10.1016/C2009-0-62958-5}.

\bibitem{Webinserviceperf}
Y. Amannejad, D. Krishnamurthy, B. Far, Managing performance interference in Cloud-based web services,
\textit{IEEE Transactions on Network and Service Management} 12 (2015), no. 3,
320-333. \doi{10.1109/TNSM.2015.2456172}.

\bibitem{LagoSurvey}
N. Condori-Fernandez, P. Lago,
Characterizing the contribution of quality requirements to software sustainability,
\textit{Journal of Systems and Software} 137 (2018), 289-305.

\bibitem{TDA}
H. Edelsbrunner, J. L. Harer,
\textit{Computational Topology: An Introduction},
American Mathematical Society, Providence RI, 2010.

\bibitem{systdynamics}
J. W. Forrester, System dynamics, systems thinking, and soft OR,
\textit{Syst. Dyn. Rev.} 10 (1994), 245-256. \doi{10.1002/sdr.4260100211}.

\bibitem{GalinacRole1}
T. Galinac Grbac, The Role of Functional Programming in Management and Orchestration of Virtualized Network Resources Part I. System structure for Complex Systems and Design Principles. CoRR abs/2107.12136 (2021)

\bibitem{GalinacRole2}
T. Galinac Grbac, N. Domazet, The Role of Functional Programming in Management and Orchestration of Virtualized Network Resources Part II. Network Evolution and Design Principles. CoRR abs/2107.12227 (2021)

\bibitem{ICT4S2}
L. M. Hilty, B. Aebischer,
ICT for sustainability: an emerging research field,
In: L. M. Hilty, B. Aebischer (eds.),
\textit{ICT Innovations for Sustainability}, pp. 3-36,
\textit{Advances in Intelligent Systems and Computing}
vol. 310,
Springer, Cham, 2015.

\bibitem{ICT4S1}
L. M. Hilty, P. Arnfalk, L. Erdmann, J. Goodman, M. Lehmann, P. A. W\"{a}ger,
The relevance of information and communication technologies for environmental sustainability: A prospective simulation study,
\textit{Environmental Modelling \& Software} 21 (2006), 1618–1629.

\bibitem{IBMMAPE}
IBM Corporation,
An architectural blueprint for autonomic computing,
Technical Report, 2005.

\bibitem{PerfomanceReact}
A. Javeed, Performance optimization techniques for ReactJS, in: 2019 IEEE International Conference on Electrical, Computer and Communication Technologies (ICECCT), Coimbatore, India, 2019, pp. 1-5. \doi{10.1109/ICECCT.2019.8869134}.

\bibitem{FramingSustainability}
P. Lago, S. A. Ko\c{c}ak, I. Crnkovi\'{c}, B. Penzenstadler,
Framing sustainability as a property of software quality,
\textit{Commun. ACM} 58 (2015), 70–78.

\bibitem{LagoGreen}
P. Lago, N. Meyer, M. Morisio, H. A. M\"{u}ller, G. Scanniello,
Leveraging ``energy efficiency to software users'': summary of the second GREENS workshop,
\textit{SIGSOFT Softw. Eng. Notes}
39 (2014), no. 1, 36–38.

\bibitem{Petric}
J. Petri\'{c}, T. Galinac Grbac,
Software structure evolution and relation to system defectiveness,
In: EASE 2014, pp. 34:1-34:10.

\bibitem{Design4finSus}
J. Thomas, P. Mantri, Design for financial sustainability,
\textit{Patterns} 3 (2022), no. 9, Art. no. 100585 (32 pages).

\bibitem{SoftwareGraph}
S. Valverde, R. Sole,
Network motifs in computational graphs: A case study in software architecture,
\textit{Physical review. E, Statistical, nonlinear, and soft matter physics} 72 (2005), Art. no. 026107 (8 pages).

\bibitem{Vliet}
H. van Vliet,
\textit{Software Engineering: Principles and Practice},
John Wiley \& Sons, 2008.

\bibitem{PerformanceWeb}
Y. Yao and J. Xia, Analysis and research on the performance optimization of Web application system in high concurrency environment, in: 2016 IEEE Information Technology, Networking, Electronic and Automation Control Conference, Chongqing, China, 2016, pp. 321-326. \doi{10.1109/ITNEC.2016.7560374}.

\bibitem{Ericsson5G}
Dohler, M., ; Nakamura, T. (2016). 5G Mobile and Wireless Communications Technology (A. Osseiran, J. Monserrat, P. Marsch, Eds.). Cambridge: Cambridge University Press.

\bibitem{DigitSust}
Mondejar, Maria; Avtar, Ram; Baños Diaz, Heyker ;Dubey, Rama ; Esteban, Jesús ; Gómez-Morales, Abigail, Hallam, Brett, Mbungu, Nsilulu; Okolo, Chukwuebuka ; Kumar, Arun ; She, Qianhong ; Garcia-Segura, Sergi. (2021). Digitalization to achieve sustainable development goals: Steps towards a Smart Green Planet. Science of The Total Environment. 794. 148539. 10.1016/j.scitotenv.2021.148539.

\bibitem{AInetwork}
B. Mao, F. Tang, Y. Kawamoto and N. Kato, "AI Models for Green Communications Towards 6G," in IEEE Communications Surveys and Tutorials, vol. 24, no. 1, pp. 210-247, Firstquarter 2022

\bibitem{AICloudRM}
Shreshth Tuli, Sukhpal Singh Gill, Minxian Xu, Peter Garraghan, Rami Bahsoon, Schahram Dustdar, Rizos Sakellariou, Omer Rana, Rajkumar Buyya, Giuliano Casale, Nicholas R. Jennings, HUNTER: AI based holistic resource management for sustainable cloud computing, Journal of Systems and Software, Volume 184, 2022, 111124, ISSN 0164-1212.

\bibitem{LivingSyst}
James G. Miller, I: The nature of living systems, Biosystems, Volume 4, Issue 2, 1972, Pages 55-77, ISSN 0303-2647.

\bibitem{Loc2glob}
Carmen Leong, Isam Faik, Felix T.C. Tan, Barney Tan, Ying Hooi Khoo,
Digital organizing of a global social movement: From connective to collective action, Information and Organization, Volume 30, Issue 4, 2020, 100324, ISSN 1471-7727.

\bibitem{RoleofFluid}
Hung-Hsiang Wang, Xiaotian Deng,
The role of fluid intelligence in creativity: The case of bio-inspired design,
Thinking Skills and Creativity, Volume 45, 2022, 101059, ISSN 1871-1871.

\end{thebibliography}
%\end{document}





\title{Teaching Sustainable IoT programming}
\author{Pieter Koopman \orcidID{0000-0002-3688-0957} \and
Mart Lubbers \orcidID{0000-0002-4015-4878}}
%
\institute{Institute for Computing and Information Sciences, Radboud University Nijmegen, The Netherlands\\
\email{\{pieter,mart\}@cs.ru.nl}}
%
\maketitle
%
\begin{abstract}
This paper explains how we want to teach sustainable programming for the Internet of Things, IoT, in the upcoming SusTrainable summer school.
Two aspects of sustainable IoT programming are discussed.
There is green computing, limiting the energy consumption of IoT applications.
Another important aspect is the efficient creation and maintenance of such IoT applications.
The lecture will show that both aspects can be served by Task-Oriented programming, TOP.
The students will experience the better maintenance of TOP based IoT applications by extending a given program in the practical session.

\keywords{IoT-programming \and TOP \and Green Computing \and Software Maintenance.}

\end{abstract}

\section{Introduction}

This paper discusses how to teach two aspects of sustainable Internet of Things, IoT, programming.

The first aspect is known as green computing.
It concerns the energy consumption of the IoT, especially the power used by the edge nodes of the IoT.
This is important since many edge nodes are battery powered, and we want to stretch the interval between battery charges.
Due to the enormous amount of edge nodes,  also reducing the energy consumption of other nodes is worthwhile.
These edge nodes together consume several percents of the worldwide total of electricity.

The second aspect of sustainable IoT computing is the creation and maintenance of IoT system software.
The rapidly increasing number of applications for the IoT makes it important to produce and update this software efficiently and reliable.
However, traditional tiered IoT applications are notoriously hard to produce and maintain because they are built using a layered architecture.
The IoT system consists typically of a user interface in the presentation layer.
This is often a web page, but it can also be an app tailored for smartphones or a wall mounter tablet like devices.
Next, there is an application layer containing web servers, various data storages and collectors to fill these storages with fresh data from the edge nodes.
The communication between application layer and the edge nodes is done in the network layer.
Apart from the well-known TCP connections, there are also tailored network services like MQTT~\cite{mqtt} and Protobuf~\cite{protobuf} used.
Finally, the bottom layer contains the edge devices.
These are single board computers, like a Raspberry Pi, or microcontrollers, like an ESP8266, equipped with sensors and actuators to measure and control phenomena in the real world.

Each of the subsystems in an IoT application is programmed in its own programming language.
By various communication protocols, these subsystems co-operate to establish the goals of the IoT application.
This makes IoT software a multi-lingual distributed heterogeneous system.
The various languages and protocols used give rise to semantic friction~\cite{ireland_classification_2009}.
There is limited support to ensure that the diverse subsystems cooperate smoothly.
Combined with the physical distribution of the edge nodes, this makes the development and maintenance of IoT applications a challenging and error-prone endeavour.

The lecture shows that Task-Oriented Programming, TOP, provides solutions for those problems.
In the iTask system, a TOP system for distributed interactive applications, the programmer specifies tasks to be done by users and the system itself at a high level of abstraction~\cite{plasmeijer_task-oriented_2012}.
All communication between subsystems, the storage of data, and even a web interface tailored to the task of specific users is generated automatically.
This single program replaces the various components of an IoT system mentioned above.
The iTask system is embedded in the strongly typed functional programming language Clean~\cite{Clean:language}.
The strong type-system prevents run time errors by checking programs at compile time.
Hence, run time type errors cannot occur in iTask programs.

Recent results show that TOP programs for the IoT are considerably shorter and simpler than their traditional counterparts~\cite{LubbersTIOT}.
The TOP programs use less programming languages and paradigms.
Together, this makes those programs easier to develop and maintain.

\section{Green Computing}

To reduce the energy consumption of IoT nodes, several strategies are applied.

The communication between the edge node and the server can consume a significant fraction of the energy consumption.
Edge computing limits this energy consumption by processing as much data as possible on the nodes where it is collected, instead of sending all collected data to the server, processing it there, and communicating the results.

Another strategy to reduce the energy consumption of the IoT is by using energy efficient microcontrollers, like a ESP8266, instead of single board computers, like a Raspberry Pi, to drive the edge nodes.
This reduces the energy consumption of such a device by another of magnitude, but this comes at a price.
The microcontrollers have a very limited amount of memory and processing power.
This often forces us to use domain-specific programming languages, like MicroPython, on these devices.
These nodes typically run either no operating system, or a limited real-time operating system.
This implies that the user program has to take care of the interleaving of multiple subtasks.
The iTask ecosystem offers the mTask language to run tasks on such microcontrollers~\cite{lubbers_writing_2019}.


Most microcontrollers have some sleeping modes on top of the basic low energy consumption.
The system temporarily shuts down part of the system to save energy in such a sleeping mode.
Obvious candidates for parts to switch off are the Wi-Fi radio and peripheral controllers that are not needed for some time.
Even the main processor and the RAM can be put in sleep mode when it is known that nothing has to be done for some time.
The system wakes up and starts working again, and hence consuming energy, after a specified sleeping time, or when an external interrupt wakes the system.
In the 2022 Sustrainable summer school in Rijeka we have shown how the mTask system can achieve such energy savings automatically~\cite{TFP22}.
For the upcoming summer school, some results are reused, but do not require any knowledge of this lecture.

\section{Task-Oriented IoT Programming by Example}

A complete introduction to TOP is outside the scope of this paper.
We illustrate the basic concepts with a few variations of a program reading a temperature sensor and showing this value in a web browser.
This example is concise, but contains all layers of an IoT application mentioned in the introduction.

The examples use a Shared Data Source, SDS, to store the latest temperature.
Such an SDS is a memory location that various tasks can read and update.
Within a single program, this behaves like a global variable.
Between various programs, an SDS is more like a database.
Here, the SDS \prog{tempSDS} contains just a real number to represent the temperature.
But, SDSs can contain values of any first order datatype, including for instance large lists of records.

\subsection{Local Temperature Sensor}

In the first version of our program, all tasks run on the same server.
The sensor is connected directly to this server, the SDS lives here and the web-server also runs on this machine.

\begin{lstlisting}[language=Clean,caption={An iTask program to read a local temperature sesnsor and display the value.},label={lst:itask1}]
tempSDS :: SimpleSDSLens Real
tempSDS = sharedStore "tempSDS" -273.15 // brrr

localSensor :: Task Real
localSensor =
    withDHT TempID \dht ->
    get tempSDS >>- \old ->
        devTask dht old -|| viewSharedInformation [] tempSDS <<@ Label "temperature"

devTask :: DHT Real -> Task Real
devTask dht old =
    temperature dht >>~ \new ->
        if (old <> new)
            (set new tempSDS >-| devTask dht new)
            (waitForTimer False delayTime >-| devTask dht old)

Start world = doTasks localSensor world

delayTime = 5
\end{lstlisting}

The \prog{localSensor} task first creates a sensor object, \prog{dht}, with \prog{withDHT}.
The argument \prog{TempID} tells how to access the physical sensor, we skip details.
Next, the task reads the current value of the SDS by \prog{get}.
The combinator \prog{>>-} bind the result of this task to the lambda function with \prog{old} as argument.
This function uses the combinator \prog{-||} for the parallel composition of the tasks \prog{devTask} and the \prog{viewSharedInformation}.
The view of the SDS is automatically updated when a change occurs.
This task keeps displaying the latest temperature as long as it runs.
The \prog{devTask} controls the sensor.
With \prog{temperature dht} we read the current value of the sensor.
Next, we check if the \prog{new} value is differs from the \prog{old} one.
When they are different, we update the SDS with the new value and call the function recursively.
Otherwise, we wait some time and start this task recursively.

This simple program contains two energy optimizations.
First, we update the SDS only when the temperature is changed.
This prevents that the web page is updated to display the same value with all associated computations and network traffic.
Future versions will execute this computation on the edge node to obtain edge-computation.
Next, the \prog{waitForTimer} tasks impose a delay.
During this time, the task does not require energy.

\subsection{Remote Temperature Sensor}

The temperature sensor is connected to a remote node in a more realistic IoT example.
We assume that the remote machine is sufficiently powerful to run an iTask program and that we have access to it.
A Raspberry Pi is perfectly suited for this job.

Our example program remains largely unchanged.
We only have to indicate which tasks runs on which machines and where the SDS is stored.
We chose to store the SDS on the remote machine.
We replace \prog{localSensor} by \prog{remoteSensor}.
The differences with the previous task is that it asks the user of the program for the address of the remote machine to run \prog{devTask}.
The function \prog{asyncTask} takes care of moving the task to the remote machine.
We tell the \prog{viewSharedInformation} that it has to display the value of this remote SDS instead of a local SDS by \prog{remoteShare}.
All other code is unchanged.

\begin{lstlisting}[language=Clean,caption={The iTask task to read a remote sensor and display its value.},label={lst:itask2}]
remoteSensor :: Task Real
remoteSensor =
    withDHT TempID \dht ->
    get tempSDS >>- \old ->
    enterInformation [] <<@ Title "device" >>? \dev ->
        asyncTask dev.domain dev.port (devTask dht old)
    -|| viewSharedInformation [] (remoteShare tempSDS dev) <<@ Label "temperature"
\end{lstlisting}

In general, it is hard to decide which part of the code is executed where.
The code executed remotely can use all functions and data types specified in the program.
To ensure that the desired code is available, the iTask system makes the entire code available on every machine executing some part of the program.
This holds even for a browser displaying the user interface.

\subsection{Remote Temperature Sensor on Microcontroller}

The introduction explains that microcontrollers are attractive hardware for edge nodes since they consume less energy and are cheaper.
Unfortunately, we cannot execute the code from the previous section on common microcontrollers.
The generated code is just too large and requires too much processing power.

The mTask system allows us to identify what parts of the program must be executed on the microcontroller.
The simple data structures and first-order strict evaluation of mTask ensure that these tasks can be executed on the restricted hardware.
The mTask system is powered by an embedded domain-specific language, DSL, offering task definitions very similar to iTask.
In contrast to standalone DSLs, embedded DSLs are expressed in terms of the host language~\cite{hudak_modular_1998}.
The strong type system of the host language ensures that the mTask parts of a program will co-operate smoothly within iTask programs.

In \prog{mTaskSensor}, the user enters the remote device.
Next, it starts the mTask task on the specified device with \prog{withDevice} in parallel with displaying the value of the temperature SDS.

All code of the \prog{devTask} is compiled dynamically to byte code that is executed by the mTask run time system running on that device.
First, the \prog{liftsds} makes a copy of the \prog{tempSDS} on the device.
The mTask machinery takes care of synchronization between those shares.
The \prog{DHT} definition creates a sensor object.
The \prog{fun} primitive defines a mTask function called \prog{measure}.
This function has the old temperature as its argument.
Just like the iTask version of this task, it reads the temperature sensor and updates the SDS when the \prog{new} value differs from the \prog{old} one.
The \prog{main} expression reads the old temperature from the SDS and calls the \prog{measure} function.

\begin{lstlisting}[language=Clean,caption={An iTask\slash{}mTask program to read a remote sensor and display its value.},label={lst:mtask}]
mTaskSensor :: Task Real
mTaskSensor =
    enterInformation [] <<@ Title "device" >>? \dev ->
        withDevice dev (liftmTask devTask)
    -|| viewSharedInformation [] tempSDS <<@ Label "temperature"

devTask :: v Real | mTask v
devTask =
  liftsds \rSDS -> tempSDS In
  DHT DHT_I2C \dht ->
  fun \measure = (\old ->
    temperature dht >>~. \new ->
    If (new  !=. old) (setSds rSDS new >>|. measure new) (measure old) In
  {main = getSds rSDS >>~. measure}
\end{lstlisting}

The reader might notice that this mTask code does not contain an explicit delay.
The mTask system is equipped with a scheduler that will automatically put the device into sleep mode in orde to save energy when there are no tasks that need execution~\cite{TFP22}.

\section{Sustainable Programming}

To verify that the TOP code is indeed more sustainable than traditional tiered code, we implemented a real world IoT example~\cite{smartSensors} in four ways~\cite{lubbers20tiered,LubbersTIOT}.
For the traditional implementation, we use a Raspberry Pi 3 as edge node and Python, JSON, HTML, PHP, Redis and MongoDB to construct the software.
This is called Python Raspberry Sensor, PRS.
Its direct counterpart uses just Clean and the iTask library and is called Clean Raspberry Sensor, CRS.
We made a variant with a WEMOS D1 mini microcontroller instead of the Raspberry Pi for both implementations.
The microcontroller runs MicroPython in the tiered approach.
This is called Python WEMOS Sensor, PWS.
The microcontroller runs mTask in the tierless TOP approach.
This variant is called Clean WEMOS Sensor, CWS.
Table~\ref{table:t1} give some key figures about these implementations.

\begin{table}[h!]
\centering
\caption{Key figures about the four implementations of the smart sensor.}%
\label{table:t1}
\begin{tabular}{l r r r r}
\toprule
   & PRS & CRS & PWS & CWS \\
 \midrule
 number of languages used & 6 & 1 & 7 & 2 \\
 total SLOC & 576 & 155 & 562 & 155 \\
 files & 38 & 3 & 35 & 3 \\
 memory residence (KiB) & 3557 & 2726 & 20 & 0.9 \\
 power consumption edge node (W) & 1-2 & 1-2 & 0.2 & 0.2 \\
 \bottomrule
\end{tabular}
\end{table}

The number of languages and files used in these TOP variants is significant smaller.
In addition, these are single source files that are checked by the strong static type system of Clean.
The Clean based versions use considerable less Source Lines of Code, SLOC, than the Python based variants.
Checking the details about the source code reveals that the TOP code is in most case 95\% shorter than their tiered counterpart.
The interface to MongoDB is only 26\% smaller and consumes almost half of the TOP code.
We can achieve a huge reduction of code by using a native SDS instead of the external MongoDB.

The code size differs not much by replacing the Raspberry Pi to a WEMOS microcontroller.
However, the restrictions of the microcontroller require an additional programming language and an appropriate distribution of tasks that obeys the limitations of the microcontroller languages.
This might be tricky for complex programs.

The microcontroller based software has the required small memory residence and achieve the desired power saving of about one order of magnitude.

\section{Teaching}

The lecture in the upcoming SusTrainable summer school will tell this story in some more detail.
We will focus on easier maintenance and its support in the TOP approach.
Other lectures in the 2023 summer school will argue the importance of maintenance by showing that up to 90\% of the total project effort is allocated to this phase of the software life cycle.

We will give the students an initial project that works with a microcontroller in the practical part of the tutorial.
The plan is to start with a project like the one in this paper.
Students will experience the support for changing such a project.
This project will be extended in steps by controlling the measurement interval dynamically, adding humidity measurements, storing historical data, and extending the user interface.
% \section{conclusion}

% bla bla

\section*{Acknowledgements}
This work acknowledges the support of the ERASMUS\raisebox{.25ex}{+} project ``SusTrainable---Promoting Sustainability as a Fundamental Driver in Software Development Training and Education'', no.\ 2020--1--PT01--KA203--078646.

%\bibliographystyle{splncs04}
%\bibliography{bibliography}
%
%\end{document}


\input{bbls/pulaTTiotProgramming.bbl}




\title{Distributed Systems Architectures with \\Green Labels}
%
\titlerunning{Distributed Systems Architectures with Green Labels}

\author{Jianhao Li \and Vikt\'oria Zs\'ok}

\authorrunning{J. Li \and V. Zs\'ok }

\institute{E\"otv\"os Lor\'and University, Faculty of Informatics\\
Department of Programming Languages and Compilers\\
H-1117 Budapest, P\'azm\'any P\'eter s\'et\'any 1/C., Hungary\\
\email{lijianhao288@hotmail.com, zsv@inf.elte.hu}\\
}
\maketitle
%
\begin{abstract}
Communication middleware is critical in distributed systems. However, many existing distributed middleware only focus on performance. This tutorial concentrates on distributed messaging middleware that balances sustainability, scalability, and performance. Through comparison, let students understand the energy-saving benefits each design detail can bring. Additionally, through the analysis of the overall architecture of the middleware, students will have a deeper understanding of sustainable distributed system architecture. The goal is to give students the ability and awareness to consider sustainability and energy efficiency when developing distributed systems.

\keywords{Distributed communication \and \textsc{Go}.}
\end{abstract}


\section{Introduction}
Distributed systems are increasingly widespread in our daily lives and play an important role.
2 \% of the global CO2 emissions are generated by the ICT (Information and Communication Technology) infrastructure~\cite{en10101470}. Therefore, besides the energy efficiency optimization of the hardware, more and more researchers realized that reducing software energy consumption (SEC) is also essential to improve the sustainability of distributed systems of various data centers.


The energy-aware programmers can use energy analysis and modeling techniques with software engineering tools to reduce the energy consumption of code~\cite{Gallagher17}. The measurement of the energy consumption of software is essential when we teach about green software implementation~\cite{GreenCurriculum}. However, finding the best measurement tool for the distributed system in \textsc{Go} is not easy. Hardware energy measurement tools are very precise and may involve an additional financial cost, while many studies have shown that software energy measurements are unstable~\cite{phdthesis}. The on-chip power sensor is a sub-category of hardware measurement tools that generally does not require extra boards or devices. The RAPL (Running Average Power Limit) is an on-chip power sensor that provides power-limiting features and accurate energy readings for CPUs and DRAM~\cite{rapl1}.

Many important energy consumption studies have been conducted with the help of RAPL~\cite{EnergyWar,PEREIRA2021102609} and Joulemeter~\cite{joulemeterR,GreenRefactorJoulemeterJ}. At the summer school, we will introduce the best practices of using RAPL to measure the energy consumption of distributed systems in \textsc{Go}, and we will also evaluate measurements of the distributed communication system.


The power model using CPU utilization, as the primary signal of machine-level activity, tracks the dynamic power usage behavior extremely well~\cite{cpuu}.
There are many built-in profilers of \textsc{Go} runtime. For example, the CPU profiler~\cite{GolangPerf} shows which functions consume what percent of CPU time. However, it is a software measurement tool that may need to be more stable and accurate, maybe it is only suitable for performance profiling. Therefore, we need to assess its usage in green computing, i.e. can it provide valuable information to locate the energy consumption problems of a program?

The software energy consumption can be reduced at different levels: execution environment level and code level~\cite{phdthesis}. At the code level, code refactoring can significantly impact the energy usage of an application~\cite{coder1,coder2}, so developing energy-efficient software through code refactoring is meaningful~\cite{refactor1}. Therefore, the energy consumption reduction of our distributed communication project starts with sustainable code refactoring.

The energy consumption of small code snippets of \textsc{Java}~\cite{url_Java} using RAPL is reported in~\cite{javaCode1}. It can guide the developer in building energy-efficient software in \textsc{Java}. The book~\cite{refactorBook} also guides the refactoring of object-oriented code. \textsc{Go} is a compiled programming language with efficient built-in concurrency constructs~\cite{url_Golang}. Based on the studies of refactoring, we research the energy consumption of the \textsc{Go} distributed communication system from the code refactoring point of view by evaluating the code snippets summarized from different implementation decisions. We will also evaluate the energy consumption of the essential elements of \textsc{Go}, such as channel, map, slice, and mutex.


There are general energy efficiency strategies that guide green computing implementations~\cite{Conceptual}. However, more practical details and suggestions are needed, which we will provide during the school to acquire green computing in \textsc{Go} distributed system.


Beside the sustainable code refactoring, we also study the sustainability of the distributed systems architecture. The author of~\cite{EnterArch} doubts the transparency of distributed objects based on performance considerations.

The Erlang distributed communication system achieved location transparency by unifying local and remote communication interfaces~\cite{url_Erlang}. It is convenient for the user that does not need to think about whether the destination actor is in the same node or a remote node when sending a message. However, we think location transparency is not a proper sustainability consideration because the users cannot intentionally reduce the number or size of the remote messages if they cannot tell the difference between local and remote sending. We suggest implementing different interfaces for local (in the same node) and remote communications (among different nodes) for sustainability so that users can make sustainable design choices. For example, in case of the \textsc{Go} programming language, the built-in channel construct deals with local communication among different lightweight threads, while other distributed communication systems focus on providing distributed remote communication interfaces.



In the lecture, we will share with students our considerations and measurements of sustainability when designing a distributed system.
We will split different parts of the implementation design of Uactor, such as various group joining algorithms, abstract them into standalone programs, and attempt to make more sustainable design choices for distributed systems by measuring the energy consumption of programs representing different design decisions. We refer to these programs that abstract the energy consumption of distributed systems as "distributed system energy evaluation abstraction programs", which have simplified the distributed system design implementations, not encompassing all the details but reflecting the overall design approach and highlighting some energy-relevant aspects. Then, a comparative analysis with controlled variables will be conducted. Ultimately, after the initial design, before the formal implementation, we can consider which design approach may be more environmentally friendly.


The accuracy of energy consumption evaluation for different designs mainly depends on the design of the abstract evaluation programs, which should contain certain features that can manifest the energy consumption differences among different design decisions or algorithms. We will discuss some experiences designing abstract energy evaluation programs with students in the course.


\section{Uactor project}

The Uactor is a distributed communication system that we are currently developing. We have presented and submitted a tutorial about it at the SusTrainable Summer School 2022~\cite{SummerSchool22}. It integrates our research on distributed systems, including exploring distributed communication models and the coordination of underlying protocols, investigating scalable peer-to-peer networks, and analyzing novel distributed group membership algorithms. Currently, we are designing this project's group membership protocol and connection mechanisms under sustainability and scalability considerations.

Uactor is a connectionless and brokerless distributed messaging middleware that balances sustainability, scalability, and performance. It follows a modified actor model and provides group communication service. The actors are organized in a dynamic overlay of connected non-hierarchical structured overlays. Additionally, this model has unified communication constructs, which means we can effortlessly use different communication mechanisms without connecting different constructs.

We are further improving this project's scalability, so we must redesign its group service mechanisms and architecture. In addition, we will design the architecture of this project with more sustainability features in the next versions. Finally, the practical green design and implementation experience will be shared with the students.

\section{Course information}

This course guides the students in implementing sustainable distributed systems by teaching energy consumption measurement and comparing the energy efficiency of different implementation code snippets. Additionally, we provide sustainable distributed system architecture suggestions and introduce sustainable distributed communication middleware.

Here are some prerequisites for students to understand the course properly: Go programming language, TLA+, and RaPL.

The students should have general knowledge of the essential elements of the programming language \textsc{Go}: variable, functions, if statement, for loop, for range loop, slice, struct, pointer, method, map, goroutine, wait group, mutex, select, and channel~\cite{goBook}.
%The comprehensive introduction to the programming language \textsc{Go} can be found in the book.

TLA+ can assist system architects in modeling the designs or algorithms of distributed systems~\cite{tla}. To describe the design more explicitly, we will summarize a part of the Uactor design using TLA+ specification.

The lecture will first introduce how to measure the energy consumption of \textsc{Go} programs. For example, we teach students how to configure RAPL in the distributed system and evaluate the energy efficiency of their code. The built-in profilers of \textsc{Go} runtime will be used in cases where they provide essential information about energy consumption, such as locating the functions that cause the energy leakage.

After that, we will introduce the energy efficiency suggestions for a distributed system in \textsc{Go} and guide the students to compare the energy efficiency of different implementing decisions. Many decisions in distributed systems are summarized into small code snippets that are easier to understand and test.

Finally, we will introduce the green design of the Uactor project and provide suggestions for designing the green architecture of distributed systems.
The students can modify the prepared \textsc{Go} programs at the practice session according to the green computing suggestions and compare the energy consumption.

After this course, the students will gain more experience in energy consumption measurement and implementing energy-efficient distributed systems in \textsc{Go}. In addition, they will have an insight into the sustainable design of the distributed communication system.


\section*{Acknowledgements}
This  paper  acknowledges  the  support  of  the  Erasmus+ Key Action 2 (Strategic partnership for  higher education) project $No$ 2020-1-PT01-KA203-078646: “SusTrainable – Promoting Sustainability as a Fundamental Driver in Software Development Training and Education”. The information and views set out in this paper are those of the authors and do not necessarily reflect the official opinion of the European Union. Neither European Union institutions and bodies, nor any person acting on their behalf may be held responsible for the use which maybe made of the information contained therein.

%\bibliographystyle{splncs04}
%\bibliography{mybibliography}

%\end{document}

\documentclass[a4paper,11pt]{article}
\pdfoutput=1 % if your are submitting a pdflatex (i.e. if you have
             % images in pdf, png or jpg format)

%\usepackage[utf8]{inputenc}
%\usepackage{mathrsfs, amssymb, amsmath}  
%\usepackage{comment}
%\usepackage{dcolumn}
%\usepackage{multirow}
%\usepackage{color}
%\usepackage{amsfonts,amssymb,amsmath, txfonts}
%\usepackage{float}

\usepackage{jcappub} % for details on the use of the package, please
                     % see the JCAP-author-manual

\usepackage[T1]{fontenc} % if needed

\hypersetup{ linktoc=all,
    colorlinks=true, linkcolor={blue},  
       citecolor={red}, urlcolor={darkred}
}
\definecolor{Redgreen}{RGB}{153,76,0}
\definecolor{vividviolet}{rgb}{0.62, 0.0, 1.0}
\definecolor{green}{RGB}{11,98,17}
\definecolor{darkgreen}{RGB}{40,150,65}
\definecolor{darkblue}{rgb}{0,0,0.3}
\definecolor{darkred}{rgb}{0.7,0,0}

\def\blue{\textcolor{blue}}
\def\red{\textcolor{red}}
\def\be{\begin{equation}}
\def\ee{\end{equation}}
\def\bea{\begin{eqnarray}}
\def\eea{\end{eqnarray}}


\title{MCMC Marginalisation Bias and $\Lambda$CDM tensions}
%\title{Overcoming bias in MCMC marginalisation to elucidate $\Lambda$CDM tensions}
%\title{Temp}

%%Markov Chain Monte Carlo

%% %simple case: 2 authors, same institution
%% \author{A. Uthor}
%% \author{and A. Nother Author}
%% \affiliation{Institution,\\Address, Country}

% more complex case: 4 authors, 3 institutions, 2 
\author[a]{Eoin \'O Colg\'ain}
\author[b]{Saeed Pourojaghi}
\author[b, c]{M. M. Sheikh-Jabbari}
\author[a]{Darragh Sherwin}

% The "\note" macro will give a warning: "Ignoring empty anchor..."
% you can safely ignore it.

\affiliation[a]{Atlantic Technological University, Ash Lane, Sligo, Ireland}
\affiliation[b]{School of Physics, Institute for Research in Fundamental Sciences (IPM), P.O.Box 19395-5531, Tehran, Iran}
\affiliation[c]{The Abdus Salam ICTP, Strada Costiera 11, I-34014 Trieste, Italy}

% e-mail addresses: one for each author, in the same order as the authors
\emailAdd{eoin.ocolgain@atu.ie}
\emailAdd{pourojaghi@ipm.ir}
\emailAdd{jabbari@theory.ipm.ac.ir}
\emailAdd{darragh.sherwin@research.atu.ie}




\abstract{Probability distributions become non-Gaussian when the flat $\Lambda$CDM model is fitted to redshift binned data in the late Universe. We explain mathematically why this non-Gaussianity arises and confirm that Markov Chain Monte Carlo (MCMC) marginalisation leads to biased inferences in observational Hubble data (OHD). In particular, in high redshift bins we find that $\chi^2$ minima, as identified from both least squares fitting and the MCMC chain, fall outside of the $1 \sigma$ confidence intervals. We resort to profile distributions to correct this bias. Doing so, we observe that $z \gtrsim 1$ cosmic chronometer (CC) data currently prefers a non-evolving (constant) Hubble parameter over a Planck-$\Lambda$CDM cosmology at $\sim 2 \sigma$. We confirm that both mock simulations and profile distributions agree on this significance. Moreover, on the assumption that the Planck-$\Lambda$CDM cosmological model is correct, using profile distributions we confirm  a $> 2 \sigma$ discrepancy with Planck-$\Lambda$CDM in a combination of  CC and baryon acoustic oscillations (BAO) data beyond $ z \sim 1.5$ that was noted earlier through comparison of least square fits of observed and mock data.}



\begin{document}
\maketitle
\flushbottom

\section{Introduction}
\label{sec:intro}
The flat $\Lambda$CDM model is the minimal model that fits Cosmic Microwave Background (CMB) data. Remarkably, CMB data from the Planck satellite \cite{Planck:2018vyg} constrains the $\Lambda$CDM model to sub-percent errors, thereby not only providing the strongest constraints, but also a concrete prediction for cosmological probes in the late Universe. The unmitigated success of the $\Lambda$CDM model is that CMB, Type Ia supernovae (SN) \cite{Riess:1998cb, Perlmutter:1998np} and baryon acoustic oscillations (BAO) \cite{Eisenstein:2005su} agree on a $\Lambda$CDM Universe that is approximately $30 \%$ matter. Thus, one key prediction of the Planck-$\Lambda$CDM model agrees across early and late Universe cosmological probes. Given this non-trivial agreement, any discrepancies that arise elsewhere constitute challenging puzzles. 

Nevertheless, one cannot define any \textit{model} for a dynamical system, especially a complicated system like the Universe, using data from a cosmic snapshot.\footnote{Here, we mean CMB data with an effective redshift $z \sim 1100$.} At best, one has a \textit{prediction} and not a model. In recent years, key predictions of Planck data have been challenged by late Universe determinations of the Hubble constant $H_0$ \cite{Riess:2021jrx, Freedman:2021ahq, Pesce:2020xfe, Blakeslee:2021rqi, Kourkchi:2020iyz} and the $S_8:= \sigma_8 \sqrt{\Omega_m/0.3}$ parameter \cite{HSC:2018mrq, KiDS:2020suj, DES:2021wwk, Boruah:2019icj, Said:2020epb}. Given the diversity of the late Universe probes (see reviews \cite{Perivolaropoulos:2021jda, Abdalla:2022yfr}), it is highly unlikely that any single systematic can be found to explain the discrepancies. That being said, in astrophysics one can never preclude systematics; 3 decades after Phillips' seminal paper \cite{Phillips:1993ng}, we are still debating an ad hoc correction for the mass of the host galaxy in Type Ia SN \cite{NearbySupernovaFactory:2018qkd, Kang:2019azh, Brout:2020msh, Lee:2021txi}. Bearing in mind that Type Ia SN are one of our best understood cosmological probes, one quickly understands that any systematics debate may be endless. 

Thus, it is far more expedient to assume that the $\Lambda$CDM model is breaking down and to look for tell-tale signatures of model breakdown. If signatures cannot be found, one arrives at a contradiction, and revisits the assumption that the model is breaking down. For physicists, \textit{model breakdown comes about when model fitting parameters return discrepant values at different time slices or epochs}. Translated into astronomy, this equates to discrepant cosmological parameters in different redshift ranges. The usual $H_0, S_8$ tensions  may also be viewed in the same light: a discrepancy between high and low redshift inferences/measurements of the parameters \cite{Perivolaropoulos:2021jda, Abdalla:2022yfr}. Nevertheless, early and late Universe observables are typically not the same, so one is confronted with a rich set of potential systematics. 

Within the context of $\Lambda$CDM tensions, it was recently observed that the integration constant from the Friedmann equations, aka the Hubble constant $H_0$, picks up redshift dependence whenever our model assumption - required to close the Friedmann equations - disagrees with the Hubble parameter $H(z)$ extracted from observations \cite{Krishnan:2020vaf, Krishnan:2022fzz}. %\footnote{One is free to speculate about the nature of the missing physics \cite{Liao:2020zko, Montani:2023xpd}.} 
Similarly, $\rho_{m0}=H_0^2\Omega_m$, an integration constant of the matter continuity equation, implies matter density $\Omega_m$ is a mathematically constant quantity. 
These are irrefutable predictions from mathematics, i. e. a prediction that is \textit{robust to systematics}. However, observationally $H_0$ and $\Omega_{m}$ are model fitting parameters and nothing precludes them picking up redshift dependence (except of course if one assumes they do not!), and providing a signature of model breakdown. If this happens in the late Universe within the $\Lambda$CDM model, $H_0$ is correlated with matter density $\Omega_m$, 
while $\Omega_m$ is correlated with $S_8 \propto \sigma_8 \sqrt{\Omega_m}$. Thus, there is at least one simple scenario, namely redshift evolution of cosmological parameters in the late Universe, where ``$H_0$ tension'' and ``$S_8$ tension'' are not independent and simply symptoms of $\Lambda$CDM model breakdown. 

The next relevant question is, where is the evidence for evolving cosmological parameters in the late Universe? Starting with strong lensing time delay \cite{Wong:2019kwg, Millon:2019slk},\footnote{Systematics are explored in \cite{Millon:2019slk} and the descending trend is not an obvious systematic. The lensed system RXJ1131-1231 \cite{Sluse:2003iy}, which partly drives the trend, has recently been re-analysed using spatially resolved stellar kinematics of the host galaxy \cite{Shajib:2023uig}, and the higher $H_0$ value remains robust, admittedly with inflated errors. As TDCOSMO project to analyse 40 lenses, the prospect of a discovery of a descending $H_0$ trend assuming the $\Lambda$CDM model remain strong.} descending trends of $H_0$ with redshift have been reported in Type Ia SN \cite{Dainotti:2021pqg, Colgain:2022nlb, Colgain:2022rxy,  Malekjani:2023dky, Hu:2022kes, Jia:2022ycc} and combinations of data sets \cite{Krishnan:2020obg, Dainotti:2022bzg}. On the other hand, larger values of $\Omega_m$ have been noted in high redshift observables, primarily quasars (QSOs) \cite{Risaliti:2015zla, Risaliti:2018reu, Lusso:2020pdb, Yang:2019vgk, Khadka:2020vlh, Khadka:2020tlm, Khadka:2021xcc, Pourojaghi:2022zrh},\footnote{Just as with Type Ia SN, the systematics of QSOs are being investigated \cite{Zajacek:2023qjm}.} but also Type Ia SN \cite{Colgain:2022nlb, Colgain:2022rxy, Malekjani:2023dky, Pasten:2023rpc} (see also \cite{Wagner:2022etu, Sakr:2023hrl}). Note, as emphasised earlier, if $H_0$ evolves at the background level, correlated fitting parameters are expected to also evolve. Moreover, mock analysis within the $\Lambda$CDM setting reveals that evolution of best fit $(H_0, \Omega_m)$ parameters cannot be precluded, and conversely possesses a finite likelihood, in either observational Hubble data (OHD) \textit{or} angular diameter distance data \textit{or} luminosity distance data \cite{Colgain:2022tql}. We stress that this result \textit{rests on mock analysis}; it represents a purely mathematical statement about the $\Lambda$CDM model that is independent of systematics. 

Separately, at the perturbative level, redshift evolution of $S_8$ or $\sigma_8$ has been reported in galaxy cluster number counts and Lyman-$\alpha$ spectra \cite{Esposito:2022plo}, $f \sigma_8$ constraints from peculiar velocities and redshift space distortions (RSD) 
 \cite{Adil:2023jtu}, comparison between weak \cite{HSC:2018mrq, KiDS:2020suj, DES:2021wwk} and CMB lensing \cite{ACT:2023dou, ACT:2023kun}. What is important here is that these observations appear to restrict the evolution in $S_8$ to the late Universe. In \cite{ACT:2023ipp} the possibility was raised that \textit{``tracers at higher redshift and probing larger scales prefer higher $S_8$''}.\footnote{There are also conflicting observations of high redshift $\sigma_8$ or $S_8$ values that are lower than Planck in the late Universe \cite{Miyatake:2021qjr, Alonso:2023guh}, so either this trend is not universal, or systematics are at play.} Nevertheless, one can argue against evolution with scale on the grounds that cosmic shear \cite{HSC:2018mrq, KiDS:2020suj, DES:2021wwk}, which is sensitive to smaller scales (larger $k$), and peculiar velocity constraints \cite{Boruah:2019icj, Said:2020epb}, which are sensitive to larger scales (smaller $k$), both prefer lower values of $S_8$. Moreover, both galaxy clusters and Lyman-$\alpha$ spectra are expected to probe similar scales.\footnote{We thank Matteo Viel for correspondence on this point.} Thus, if systematics are not impacting results, then redshift evolution is the only point of agreement in the observations \cite{Esposito:2022plo, Adil:2023jtu, HSC:2018mrq, KiDS:2020suj, DES:2021wwk, ACT:2023dou, ACT:2023kun, ACT:2023ipp}. Note also that redshift is more fundamental than scale in FLRW cosmology; one must solve the Friedmann equations in either time or redshift before one contemplates any discussion of scale.  

 The purpose of this letter is to revisit the analysis presented in \cite{Colgain:2022rxy,Colgain:2022tql}, where the evidence for evolution was quantified on the basis of mock simulations and not Markov Chain Monte Carlo (MCMC), the technique most familiar in cosmology. The fundamental problem is that once one bins low redshift data and studies evolution of cosmological parameters with bin redshift, one quickly encounters projection effects in MCMC analyses. These effects are not just the preserve of exotic models \cite{Herold:2021ksg, Gomez-Valent:2022hkb, Meiers:2023gft}, such as Early Dark Energy (EDE) \cite{Poulin:2018cxd, Niedermann:2019olb}, and happen in the simplest model when one bins data. The most striking demonstration of the resulting bias is that the peaks of MCMC posteriors no longer coincide with the minimum of the likelihood (see \cite{Gomez-Valent:2022hkb}). Ultimately, this bias is expected  because one is working in a regime of the $\Lambda$CDM model with non-Gaussian probability distributions   \cite{Colgain:2022tql}  (see also \cite{Colgain:2022rxy}).

 The structure of this paper is as follows. In section \ref{sec:MCMC_bias} we confirm the bias in MCMC marginalisation. In section \ref{sec:PD} we introduce profile distributions (PDs) \cite{Gomez-Valent:2022hkb} as a means of addressing the bias and confirm that the statistical significance of discrepancies from mock simulations agree well with PD analysis. In section \ref{sec:tension}, we revisit and confirm the high redshift OHD tensions reported in \cite{Colgain:2022rxy}. We end in section \ref{sec:discussion} with concluding remarks. 
 %A short appendix is also added on Fisher matrix for $\Lambda$CDM mdoel. 

\section{A bias in MCMC marginalisation}
\label{sec:MCMC_bias}
In this section we illustrate a bias in MCMC marginalisation that arises in the (flat) $\Lambda$CDM model when data is binned by redshift. This bias can be traced to a regime of the $\Lambda$CDM model with non-Gaussian distributions and is independent of systematics  \cite{Colgain:2022rxy, Colgain:2022tql}. 

\subsection{Mathematical Foundations}
\label{sec:math}
Consider an exercise where one bins OHD and confronts it to the $\Lambda$CDM Hubble Parameter $H(z)$ in the late Universe, a setting where the radiation sector can be safely decoupled. In high redshift bins ($z \gg 0$) in the matter-dominated regime, the Hubble parameter becomes insensitive to the dark energy (DE) sector: 
\be
\label{eq:lcdm}
H(z) = H_0 \sqrt{1-\Omega_m + \Omega_m (1+z)^3} \xrightarrow[z \gg 0]{} H_0 \sqrt{\Omega_m} (1+z)^{\frac{3}{2}}.  
\ee
More concretely, taking $z \rightarrow \infty$ we see that data can only constrain the combination $\rho_{m0}=H_0^2{\Omega_m}$. For \textit{hypothetical} data in a redshift bin with effective redshift $z = \infty$, this means that one can only constrain the combination $\Omega_m h^2$ ($h:= H_0/100)$, but $H_0$ and $\Omega_m$ remain unconstrained. Alternatively put, for any given $\Omega_m h^2$ constraint, there is an infinite number of corresponding $(H_0, \Omega_m)$ pairs. Translated into a probability density function (PDF), this is simply the statement that in a very high redshift bin at $z = \infty$, one expects uniform or flat distributions for $H_0$ and $\Omega_m$ with the model (\ref{eq:lcdm}).  

Of course, observed data resides at finite $z$ and not $z = \infty$. As a result, one does not encounter \textit{exactly} flat PDFs in $H_0$ and $\Omega_m$ at high redshift, but \textit{almost} flat PDFs. More important to us is the observation that these PDFs must flatten in a non-Gaussian manner. To appreciate this fact, we observe that high redshift OHD only constrains $\Omega_m h^2$ well.\footnote{Note that observables like SN or QSO that measure $D_L(z)=c (1+z)\int_0^z \textrm{d} z'/H(z')$ are mainly sensitive to the low redshift part of $H(z)$, i. e. the combination $H_0^2 (1-\Omega_m)$, and in this sense they are complementary to the OHD data which is more sensitive to high redshift part of $H(z)$, $H_0^2\Omega_m$. The complementarity can be demonstrated by combining $H(z)$ and $D_{L}(z)$ constraints and checking that one recovers mock data input parameters in all redshift bins \cite{Colgain:2022tql}. } For this reason, best fit parameters are constrained to a $\Omega_m h^2 = \textrm{constant}$ curve in the $(H_0, \Omega_m)$-plane. The almost flat $H_0$ and $\Omega_m$ PDFs can only arise if this curve stretches in the $(H_0, \Omega_m)$-plane. As a result of this stretching, one ends up with a relatively uniform distribution on a curve. At the extremes of the curve, one finds a distribution of large $H_0$ values, which do not differ greatly in $\Omega_m$, and they get projected to a peak at small values on the $\Omega_m$ axis. Conversely, at the other end of the curve, one finds a distribution of small $\Omega_m$ values, which do not differ greatly in $H_0$, and they get projected onto a peak at large values on the $H_0$ axis.  This is a ``projection effect'' in common cosmology parlance.  It is driven by the irrelevance of the DE sector at high redshift and the constraint $\Omega_m h^2 = \textrm{constant}$ from the $\Lambda$CDM model (\ref{eq:lcdm}). Together these features distort the distribution away from a Gaussian configuration. 

Thus, simply by binning and fitting OHD to the $\Lambda$CDM model one enters a non-Gaussian regime as the effective redshift of the bin increases. This effect, which is expected from the purely mathematical arguments above, has been confirmed in mock data \cite{Colgain:2022rxy, Colgain:2022tql}, and in line with expectations, we demonstrate that it impacts MCMC inferences with observed data in the next subsection.  

% Figure environment removed

\subsection{Cosmic Chronometer (CC) Data}
\label{sec:CCbias}
Here we work with OHD from the cosmic chronometer (CC) program \cite{Jimenez:2001gg}. Concretely, we work with 34 $H(z)$ constraints spanning the redshift range $0.07 \leq z \leq 1.965$ \cite{Stern:2009ep, Moresco:2012jh, Zhang:2012mp, Moresco:2016mzx, Ratsimbazafy:2017vga, Borghi:2021rft, Jiao:2022aep, Tomasetti:2023kek}. We illustrate the data in Fig.~\ref{fig:CC}, where it is consistent with Fig. 9 of \cite{Tomasetti:2023kek} {modulo the fact that we have an additional data point at $z = 0.8$, which is not independent. See Table 1.1 of \cite{Moresco:2023zys}. While CC data may eventually be good enough to arbitrate on Hubble tension \cite{Moresco:2023zys}, the data is not good enough on its own to do cosmology. To put this comment in context, we observe that the errors in Fig.~\ref{fig:CC} do not include systematic errors (see \cite{Moresco:2020fbm} for an account of the systematics). As a result the constraints we get on cosmological parameters will be underestimated. Thus, from our perspective the data in Fig.~\ref{fig:CC} is simply some representative cosmological data in the OHD class.}

\paragraph{Methodology:} We impose a low redshift cut-off on the OHD $z_{\textrm{min}}$, removing all data points with redshifts $z_i < z_{\textrm{min}}$, and then extremising the $\chi^2$ likelihood, 
\be
\label{eq:chi2}
\chi^2 = Q^{T} \cdot C^{-1} \cdot Q, 
\ee
where $C$ is the covariance matrix, which is simply the square of the $H_i$ errors on the diagonal, and $Q$ is the vector, 
\be
\label{eq:Q}
Q_i = H_i - H_{\textrm{model}}(z_i), 
\ee
where $H_i:=H(z_i)$ denotes OHD and $H_{\textrm{model}}(z)$ is the model (\ref{eq:lcdm}) without the high redshift limit. The best fit $(H_0, \Omega_m)$ parameters correspond to the minumum of the $\chi^2$, while on the assumption of Gaussian errors, we estimate the errors from a Fisher matrix (appendix \ref{sec:fisher}). In parallel, we perform MCMC marginalisation through \textit{emcee} \cite{Foreman-Mackey:2012any}. More concretely, subject to the priors $H_0 \in [0, 200 ]$ and $\Omega_m \in [ 0, 1]$, the latter restricting us to a physical regime, we record $16^{\textrm{th}}$, $50^{\textrm{th}}$ and $84^{\textrm{th}}$ percentiles for MCMC posteriors, as is common practice with Gaussian distributions. Thus, both techniques are tailored to Gaussian posteriors, yet non-Gaussianities will be evident in MCMC posteriors. By comparing the output from these two techniques in Table \ref{tab:LCDM_CC} for different values of $z_{\textrm{min}}$ we observe that error estimates from Fisher matrix and MCMC quickly disagree as $z_{\textrm{min}}$ increases. 

From Table \ref{tab:LCDM_CC}, we see that MCMC inferences lead to non-Gaussian $1 \sigma$ confidence intervals, where in line with the expectations from \cite{Colgain:2022tql}, $H_0$ errors are larger for smaller values, and $\Omega_m$ errors are larger for larger values, respectively. This is expected if the $H_0$ and $\Omega_m$ posteriors are peaked at larger and smaller values, respectively, in line with our earlier mathematical argument. Only for the full data set with $z_{\textrm{min}} = 0$  do we find reasonable agreement between the Fisher matrix and MCMC $1 \sigma$ confidence intervals. As can be seen from the lopsided MCMC confidence intervals, the non-Gaussianity becomes more pronounced with increasing $z_{\textrm{min}}$. Interestingly, beyond $z_{\textrm{min}} = 1$, the minimum of the $\chi^2$ falls outside of the MCMC $1 \sigma$ confidence intervals. Nevertheless, by evaluating the MCMC chains on the $\chi^2$ likelihood (\ref{eq:chi2}), we confirm that the parameters corresponding to the minimum $\chi^2$ value are tracking the best fit. Note, the peak of the MCMC posterior is no longer a measure of goodness of fit and inferences have become biased in a regime of model parameter space where distributions are expected to be inherently non-Gaussian. Our analysis here underscores potential problems with a blind MCMC analysis with the traditional $16^{\textrm{th}}$, $50^{\textrm{th}}$ and $84^{\textrm{th}}$ percentiles.       



\begin{table}[htb]
    \centering
    \begin{tabular}{c|c|c|c|c|c}
    \rule{0pt}{3ex} $z_{\textrm{min}}$ & \# CC & \multicolumn{2}{c}{Fisher Matrix}  & \multicolumn{2}{|c}{MCMC} \\
    \hline
    \rule{0pt}{3ex} & & $H_0$ (km/s/Mpc) & $\Omega_m$ & $H_0$ (km/s/Mpc) & $\Omega_m$ \\
    \hline
    \rule{0pt}{3ex} $0$ & $34$ & $68.14 \pm 3.07$ & $0.320 \pm 0.059$ & $67.76^{+3.03}_{-3.09}$  ($68.12$) & $0.328^{+0.065}_{-0.055}$ ($0.321$) \\
    \hline 
    \rule{0pt}{3ex} $0.2$ & $27$ & $65.03 \pm 6.65$ & $0.368 \pm 0.118$ & $63.05^{+6.64}_{-7.23}$ ($64.98$) & $0.405^{+0.170}_{-0.111}$ ($0.369$) \\
    \hline 
    \rule{0pt}{3ex} $0.4$ & $22$ & $62.42 \pm 8.38$ & $0.411 \pm 0.161$ & $59.54^{+8.30}_{-8.22}$ ($62.39$) & $0.470^{+0.229}_{-0.151}$ ($0.411$)\\
    \hline 
    \rule{0pt}{3ex} $0.6$ & $15$ & $59.83 \pm 17.21$ & $0.454 \pm 0.338$ & $56.45^{+13.16}_{-9.33}$ ($59.86$) & $0.526^{+0.288}_{-0.225}$ ($0.453$) \\
    \hline 
    \rule{0pt}{3ex} $0.7$ & $14$ & $79.11 \pm 19.40$ & $0.222 \pm 0.162$ & $67.59^{+19.19}_{-16.57}$ ($79.18$) & $0.344^{+0.344}_{-0.178}$ ($0.222$) \\
    \hline 
    \rule{0pt}{3ex} $0.8$ & $11$ & $103.97 \pm 24.94$ & $0.097 \pm 0.088$ & $82.43^{+28.33}_{-27.03}$ ($104.02$) & $0.206^{+0.357}_{-0.131}$ ($0.096$) \\
    \hline 
    \rule{0pt}{3ex} $1$ & $8$ & $150.37 \pm 31.21$ & $0.010 \pm 0.035$ & $108.92^{+33.94}_{-44.47}$ ($150.38$) & $0.087^{+0.304}_{-0.068}$ ($0.010$) \\
    \hline 
    \rule{0pt}{3ex} $1.2$ & $7$ & $154.35 \pm 42.95$ & $0.006 \pm 0.042$ & $83.07^{+48.52}_{-32.19}$ ($154.47$) & $0.194^{+0.439}_{-0.159}$ ($0.006$) \\
    \hline 
    \rule{0pt}{3ex} $1.4$ & $4$ & $125.41 \pm 79.55$ & $0.039 \pm 0.132$ & $65.32^{+44.88}_{-20.30}$ ($125.44$) & $0.320^{+0.423}_{-0.250}$ ($0.039$) \\
    \hline 
    \rule{0pt}{3ex} $1.5$ & $3$ & $36.12 \pm 72.69$ & $1.000 \pm 4.269$ & $55.19^{+34.64}_{-14.73}$ ($36.16$) & $0.393^{+0.387}_{-0.283}$ ($0.999$)
    \end{tabular}
    \caption{Comparison between Fisher matrix and MCMC analysis for CC data with a low redshift cut-off $z_{\textrm{min}}$. We record the number of data points, the extremum of the $\chi^2$ and $1 \sigma$ confidence interval estimated from the Fisher matrix,  $16^{\textrm{th}}$, $50^{\textrm{th}}$ and $84^{\textrm{th}}$ percentiles from MCMC posteriors corresponding to $1 \sigma$ confidence intervals, and the minimum $\chi^2$ from the MCMC chain in brackets. MCMC marginalisation exhibits non-Gaussian $1 \sigma$ confidence intervals, and for $z_{\textrm{min}} > 1$, the minimum value of the $\chi^2$ from the MCMC chain falls outside of this interval. The latter tracks the best fit up to small numbers in line with expectations. }
    \label{tab:LCDM_CC}
\end{table}

\subsection{Features in CC Data}
\label{sec:features}
Once one accounts for biases, it is clear from Table \ref{tab:LCDM_CC} that there are trends in CC data when it is binned. Starting from $z_{\textrm{min}} = 0$ through to $z_{\textrm{min}} = 0.6$ we see a decreasing trend in best fit values of $H_0$ (also central $H_0$ values from MCMC), which is compensated by a increasing trend in $\Omega_m$ best fit values. From Fig.~\ref{fig:CC} it is difficult to visibly discern any trend from the raw data. From $z_{\textrm{min}} = 0.7$ through to $z_{\textrm{min}} = 1.4$, there is in contrast a preference for larger $H_0$ and smaller $\Omega_m$ values. This trend is evident from the raw data, where at higher redshifts one sees large scatter and large fractional errors in the data. For $z_{\textrm{min}} = 1$, it is clear that the best fit line in magenta corresponding to $(H_0, \Omega_m) = (150.4, 0.01)$ (Table \ref{tab:LCDM_CC}) is closer to horizontal line than the Planck-$\Lambda$CDM cosmology in red. To be more explicit, for $z_{\textrm{min}} = 0$, $\rho_{m0}:=H_0^2\Omega_m\simeq 1500$ which is close to the Planck value, whereas for $z_{\textrm{min}} = 1$, $\rho_{m0}\simeq 225$. The sharp drop in $\rho_{m0}$ means the magenta line should be almost horizontal. For $z_{\textrm{min}} = 1.5$, we switch to an opposite regime of parameter space with unexpectedly low and high values of $H_0$ and $\Omega_m$, respectively, a trend which is evident in the data, but there are only three data points. Despite, the small number of data points, the tendency for smaller $H_0$ and larger $\Omega_m$ inferences within $\Lambda$CDM cosmology at high redshifts has been documented across three independent observables \cite{Colgain:2022rxy}. We will come back to this claim in section \ref{sec:tension}. Finally, it is worth noting that for large $z_{\textrm{min}}$ and samples with few data points, one expects broad MCMC posteriors. These posteriors are severely impacted by the prior on $\Omega_m$, as is evident from Table \ref{tab:LCDM_CC}. 

For the moment we leave physical speculations to the discussion and return to the trend in CC data above $z=1$ favouring less evolution in the Hubble parameter than the Planck-$\Lambda$CDM model. We would like to quantify the significance of this trend, but since we are working in a non-Gaussian regime of the model, we can expect both Fisher matrix and MCMC to give biased results. In Fig.~\ref{fig:CCsplit1} we show MCMC posteriors for $z>1$ CC data in blue alongside posteriors for low redshift ($z < 1$) CC data, which is simply added to aid comparison and also highlight the Gaussianity of the low redshift posteriors. One notes that the peaks of the $z > 1$ distributions are a little displaced from to the values minimising the $\chi^2$. However, the emergence of the lower peak in the $H_0$ posterior at $H_0 \sim 50$ km/s/Mpc has the hallmarks of a projection effect. To appreciate this, note that the configurations in the blue curve in the top left corner of the 2D posterior are projected onto the lower $H_0$ peak. Moreover, if one shifts the $H_0$ peak from $H_0 \sim 150$ to $H_0 \sim 50$ km/s/Mpc while maintaining $\Omega_m \sim 0$, this shifts the magenta curve in Fig. \ref{fig:CC} outside of all the data points, so the lower $H_0$ peak is a phantom artefact unrelated to the goodness of fit. We also observe a shift in the higher $H_0$ peak away from the minimum of the $\chi^2$.

Ignoring these features, one could attempt to interpret the overlap in the 2D posteriors in Fig. \ref{fig:CCsplit1}. Doing so, one may conclude that low and high redshift CC data are consistent within $1 \sigma$. However, since Hubble tension is a 1D problem (local $H_0$ determinations are insensitive to other parameters), to compare with locally observed values of $H_0$ one needs to project onto the $H_0$ axis. Alternatively put, Hubble tension is a problem in 1D posteriors. Projecting onto the $H_0$ axis by determining $16^{\textrm{th}}$, $50^{\textrm{th}}$ and $84^{\textrm{th}}$ percentiles, one sees from Table \ref{tab:LCDM_CC} that the $z_{\textrm{min}} = 1$ MCMC confidence interval encloses the $z_{\textrm{min}} = 0$ central values within $1 \sigma$,\footnote{Note, removing the eight high redshift data points from the $z_{\textrm{min}} = 0$ sample will not shift the central values much.} but not the point in parameter space that best fits the data!


% Figure environment removed



Evidently, given the non-Gaussian posteriors, care is required when interpreting the significance of the trend towards a non-evolving (horizontal) $H(z)$ at higher redshifts in Fig.~\ref{fig:CC}. We cannot use the errors from the Fisher matrix as we are clearly in a non-Gaussian regime, whereas MCMC inferences are impacted by projection effects to the extent that the minimum of the $\chi^2$ (confirmed from the MCMC chain) falls outside of the $1 \sigma$ confidence interval. For this reason, we resort to mock simulations. While this may seem a little redundant if we are going to employ profile distributions in section \ref{sec:PD}, there is motivation for this exercise. In \cite{Colgain:2022rxy} the significance of a descending $H_0$/increasing $\Omega_m$ trend with effective redshift in OHD, Type Ia SN and QSOs was estimated to be a $\sim 3 \sigma$ effect on the basis of combining $\sim 2 \sigma$ effects in each of the \textit{independent} data sets using Fisher's method. Here, working with the same data throughout, we can directly compare the significance of a discrepancy estimated through mock simulations from the significance of a discrepancy estimated through profile distributions. In particular, we will address the question: how significant is a constant $H(z)$ with $z_{\textrm{min}}=1$ (8 data points) against the Planck consistent cosmology favoured by the full data set ($z_{\textrm{min}}=0$ entry in Table \ref{tab:LCDM_CC})? Note, the significance will be overestimated due to missing systematic uncertainties (see \cite{Moresco:2020fbm}), but we can still make comparison between the two techniques.

\paragraph{{Mock simulations:}} To address this question using mock simulations, we begin with the MCMC chains for the full sample. For each entry in the MCMC chain (approximately 15,000 entries in total), we generate a new realisation of the 8 high redshift data points $(z > 1)$ that are by construction statistically consistent with both the best fits from the full sample and also the Planck-$\Lambda$CDM values \cite{Planck:2018vyg}. More concretely, for each $(H_0, \Omega_m)$ entry in our MCMC chain, we displace the data points to the corresponding $\Lambda$CDM Hubble parameter before generating new data points in a normal distribution where the errors serve as standard deviations. We then fit back the $\Lambda$CDM model to each realisation of the mock data and record the best fit $(H_0, \Omega_m)$ values, which give us a distribution of expected $(H_0, \Omega_m)$ best fits. The distributions are presented in Fig.~\ref{fig:CCsims} alongside the best fits from observed data. Throughout, we assume canonical values $(H_0, \Omega_m) = (70, 0.3)$ for the initial guess of the fitting algorithm. Best fits can saturate our bounds, i. e. $\Omega_m = 0$ and $\Omega_m = 1$, and this leads to an unsightly pile up of best fits at $\Omega_m = 0$ and $\Omega_m = 1$ in Fig.~\ref{fig:CCsims} \cite{Colgain:2022rxy}. It is important to retain all the configurations, otherwise one is not accounting for the probability that a best fit falls outside our priors. As a consistency check, we see that the median or 50$^{\textrm{th}}$ percentile, $(H_0, \Omega_m) = (68.32, 0.321)$ agrees well with the mock input parameters, thereby demonstrating that there are an equal number of best fits with values above and below the injected parameters in the mocks. We find that probability of a more extreme (larger) $H_0$ value to be $p = 0.022$, while the probability of a more extreme (smaller) $\Omega_m$ value to be $p = 0.035$, respectively. Converted into a Gaussian statistic, these correspond to $2 \sigma$ and $1.8 \sigma$, respectively, for a one-sided normal distribution. Thus, on the basis of mock simulations, we estimate the non-evolving constant $H(z)$ with $z_{\textrm{min}} = 1$ as a $\sim 2 \sigma$ effect. In the next section we will recover this number more or less from the profile distribution analysis. 

% Figure environment removed


\section{Profile Distributions}
\label{sec:PD}
Having explained the mathematics behind the bias, which gives rise to a projection effect, in subsection \ref{sec:math}, and having illustrated how it affects MCMC inferences in subsection \ref{sec:CCbias} - the minimum of the $\chi^2$ may fall outside of $1 \sigma$ confidence intervals - we turn to profile distributions (PDs) \cite{Gomez-Valent:2022hkb}, an extension of the profile likelihood, e. g. \cite{Trotta:2017wnx}, in order to address the bias. Consider two sets of parameters $\theta_1$ and $\theta_2$ and a normalised distribution $\mathcal{P}(\theta_1, \theta_2)$. The basic idea \cite{Gomez-Valent:2022hkb} is to study the ratio 
\be
\label{R}
R(\theta_1) = \frac{\tilde{\mathcal{P}}(\theta_1)}{\max_{\theta_1} \tilde{\mathcal{P}}(\theta_1) } = \frac{\tilde{\mathcal{P}}(\theta_1)}{\max_{\theta_1, \theta_2} \mathcal{P}(\theta_1, \theta_2) },  
\ee
where $\tilde{\mathcal{P}}(\theta_1)$ is the PD, defined to be the maximum of $\mathcal{P}$ for each $\theta_1$ along the $\theta_2$ direction: 
\be
\label{PD}
\tilde{\mathcal{P}} (\theta_1) = \max_{\theta_2} \mathcal{P}(\theta_1, \theta_2). 
\ee
The advantage of this approach is that $R(\theta_1)$ can serve as a probability distribution function (up to an overall normalization), however we do not need to perform any integration, so $R(\theta_1)$ is not prone to volume or projection effects. At this juncture, given the simplicity of our setup with only two parameters $(H_0, \Omega_m)$, we can be more explicit. Consider the probability distribution,   
\be
\mathcal{P}(\theta_1, \theta_2) = \exp \left( - \frac{1}{2} \chi^2(\theta_1, \theta_2) \right), 
\ee
where $\theta_i \in \{H_0, \Omega_m \}$  and $\chi^2(H_0, \Omega_m)$ is our earlier likelihood (\ref{eq:chi2}). The maximum value of $\mathcal{P}$ occurs for the minimum value of $\chi^2$ from the MCMC chain, $\mathcal{P}_{\textrm{max}} = e^{-\frac{1}{2} \chi^2_{\textrm{min}}}$. In this concrete setting, the PD becomes 
\be
\tilde{\mathcal{P}}(\theta_1) = e^{-\frac{1}{2} \chi^2_{\textrm{min}}(\theta_1)}, 
\ee
where $\chi^2_{\textrm{min}}(\theta_1)$ denotes the minimum value of the $\chi^2$ along the $\theta_2$ direction for a fixed $\theta_1$ value. It should not be confused with the overall minimum $\chi^2_{\textrm{min}}$, which can be extracted easily from the MCMC chain. In practice, one can also determine $\chi^2_{\textrm{min}}(\theta_1)$ from the MCMC chain by breaking the $\theta_1$ direction up into bins and finding the minimum of the $\chi^2$ for each bin. Having done so, we are in a position to define a PDF \cite{Gomez-Valent:2022hkb}: 
\be
\label{eq:w}
w(\theta_1) = \frac{e^{-\frac{1}{2} \chi^2_{\textrm{min}}(\theta_1)}}{\int e^{-\frac{1}{2} \chi^2_{\textrm{min}}(\theta_1)} \, \textrm{d} \theta_1} = \frac{R(\theta_1)}{\int R(\theta_1) \, \textrm{d} \theta_1}, 
\ee
where in the second equality we have divided top and bottom by $\mathcal{P}_{\textrm{max}} = e^{-\frac{1}{2} \chi^2_{\textrm{min}}}$. As a result, $R(\theta_1) = e^{-\frac{1}{2} \Delta \chi_{\textrm{min}}^2}$, where $\Delta \chi^2_{\textrm{min}} := \chi_{\textrm{min}}^2(\theta_1) - \chi^2_{\textrm{min}}$, so that $R(\theta_1)$ peaks at $R(\theta_1) = 1$. Note that $\int_{-\infty}^{+\infty} w(\theta_1) \, \textrm{d} \theta_1 = 1$ by construction, so $w(\theta_1)$ describes a properly normalised PDF. Thus we can identify the $1 \sigma, 2 \sigma$ and $3 \sigma$ confidence intervals corresponding to the 68\%, 95\% and 99.7\% confidence level, respectively, by simply identifying $\theta_1^{(1)}$ and $\theta_1^{(2)}$ such that \cite{Gomez-Valent:2022hkb}
\be
\label{eq:wsigma}
\int_{\theta_1^{(1)}}^{\theta_1^{(2)}} w(\theta_1) \, \textrm{d} \theta_1 = I, \quad w(\theta_1) = w(\theta_2), \quad I \in \{0.68, 0.95, 0.997\}. 
\ee
We will outline how these conditions can most easily be satisfied when we turn to explicit examples. 

Our first port of call is making sure that the PD methodology gives sensible results. This can be best judged by applying it to the CC data with $z_{\textrm{min}} = 0$, since this is where we expect a distribution closest to a Gaussian distribution, as is evident from the agreement between Fisher matrix and MCMC results in Table \ref{tab:LCDM_CC}. In particular, we will be interested in a comparison between $1 \sigma$ confidence intervals to make sure that (\ref{eq:wsigma}) is not underestimating or overestimating the $1 \sigma$ confidence interval. 

% Figure environment removed

We start by running a long MCMC chain (100,000 iterations) in order to ensure bins are well populated, and begin by analysing $\theta_1 = H_0$ with $\theta_2 = \Omega_m$. From the MCMC chain we identify the smallest and largest value of $H_0$ in the chain and break up this range into approximately 200 uniform bins, which we label using the $H_0$ value at the centre of the bin. We omit any empty bins. One can increase the number of bins by simply running a longer MCMC chain. In each $H_0$ bin we identify the minimum value of the $\chi^2$, $\chi^2_{\textrm{min}}(H_0)$, and calculate $R(H_0)$. One then repeats the steps for $\Omega_m$. In Fig.~\ref{fig:R_zmin0} we plot $R(H_0)$ against $H_0$ and $R(\Omega_m)$ against $\Omega_m$, noting that the distributions are Gaussian to first approximation. 

Since the distributions from the MCMC chain are sparse in the tails, empty bins are evident in Fig.~\ref{fig:R_zmin0}. Nevertheless, with 200 bins, modulo any empty bins, we have sufficient density of points to calculate the total area under the $R(H_0)$ and $R(\Omega_m)$ curve using Simpson's rule. Any concern about precision can simply be mitigated by running a longer MCMC chain and increasing the number of bins. 
One may directly use $R(H_0)\leq 1$ and $R(\Omega_m)\leq 1$   to find $68$, $95$ and $99.7$ percentiles,  respectively corresponding to $1 \sigma, 2 \sigma$ and $3 \sigma$ confidence intervals. Consider $F_\kappa:= \int_{R\geq \kappa} R (\theta_1) \, \textrm{d} \theta_1$, where $\kappa \leq 1$. Observe that $F_{\kappa=1}=0$ and $F_{\kappa=0}:=F_0=\int R(\theta_1) \textrm{d} \, \theta_1$. Then move $\kappa$ through and terminate the process when $F_\kappa/F_0$ is equal to $0.68$, $0.95$ and $0.997$. This gives the corresponding range for $\theta_1$ that defines the confidence interval.
Working with the precision afforded to us by approximately 200 bins, the $H_0$ and $\Omega_m$ $1 \sigma$ confidence intervals are presented in Fig.~\ref{fig:R_zmin0} and the first entry in Table \ref{tab:LCDM_CC_PD}. The outcome is in excellent agreement with both Fisher matrix and MCMC analysis. In particular, a mild non-Gaussianity in $\Omega_m$ is evident in both Fig.~\ref{fig:R_zmin0} and the errors. 
Thus, we have succeeded in recovering results in the (almost) Gaussian regime that are consistent with Fisher matrix and MCMC analysis and this provides an important check of the methodology.  

% Figure environment removed

We now apply the same PD methodology to the non-Gaussian regime where MCMC marginalisation leads to biased results. To be concrete, we focus on the eight data points in the range $1 < z < 2$ where a non-evolving $H(z)$ trend is evident in the raw data in Fig.~\ref{fig:CC}. Our goal here is to quantify the disagreement with the full data set, where one infers $H_0 \sim 68$ km/s/Mpc and $\Omega_m \sim 0.32$. A similar exercise was performed in subsection \ref{sec:features} with mock simulations and the disagreement was estimated to be approximately $2 \sigma$. Repeating the steps outlined above for the CC data with $z_{\textrm{min}} = 1$ we find the distributions in Fig.~\ref{fig:R_zmin1}. The first observation is that the distributions are non-Gaussian, but a comparison to the MCMC posteriors from the same data in blue in Fig.~\ref{fig:CCsplit1} reveals that there is no secondary $H_0$ peak at $H_0 \sim 50$ km/s/Mpc. Thus, we confirm the secondary peak to be a projection effect. That being said, the primary $H_0$ peak from Fig.~\ref{fig:CCsplit1} has shifted to the dashed line corresponding to the minimum of the $\chi^2$, since the peak of the distribution and $\chi^2$ minimum agree by construction. Comparing the blue $\Omega_m$ distribution from Fig.~\ref{fig:CCsplit1} to the $R(\Omega_m)$ distribution in Fig.~\ref{fig:R_zmin1}, we see that the peak is close to $\Omega_m = 0$ and that the tails continue to $\Omega_m = 1$. In both plots we see that there is a non-zero probability of inferring $\Omega_m = 1$. In some sense, this is not so surprising, the reason being that one is free to adopt generous priors for $H_0$, so that probability of large and small $H_0$ values is zero, but the priors on $\Omega_m$ in the flat $\Lambda$CDM model are restricted. For this reason, as a distribution spreads one invariably finds that distributions are impacted by the $\Omega_m$ priors.\footnote{Note, this is a problem for the flat $\Lambda$CDM model. In particular, one may easily find that the peak of the $\Omega_m$ distribution is larger than $\Omega_m=1$, as is the case with Hubble Space Telescope SN with redshifts $z > 1$ in the Pantheon+ sample \cite{Malekjani:2023dky}.}

It is evident from Fig.~\ref{fig:R_zmin1} that any tension that exists is confined to the $H_0$ parameter. Moreover, since there may be only one binned value of $\Omega_m$ below the $R(\Omega_m)$ peak, at the precision afforded to us by 200 bins, the $R(\Omega_m)$ distribution in Fig.~\ref{fig:R_zmin1} is essentially one-sided and the $1 \sigma$ confidence interval stretches beyond $\Omega_m \sim 0.32$, so there is no disagreement in the $\Omega_m$ parameter. Nevertheless, in the $H_0$ parameter we see that $H_0 \sim 68$ km/s/Mpc, the value favoured by the full data set is just under $2 \sigma$ removed from the peak. The main point here is that, as is obvious from the raw data, current CC data with $z > 1$ has a preference for a non-evolving Hubble parameter $H(z)$ with a large constant $H_0 \sim 150$ km/s/Mpc. The disagreement is just under $2 \sigma$, more accurately $1.9 \sigma$ from $R(H_0)$, and only $0.9 \sigma$ from $R(\Omega_m)$. Although this may not be a serious discrepancy, essentially because of the poor data quality (8 data points), this disagreement supports the $\sim 2 \sigma$ discrepancy seen in the mock simulations. It should be borne in mind that systematic uncertainties have been omitted and these will reduce this discrepancy once properly propagated. Given the agreement between the PD and mock simulation analysis, there is nothing to suggest that the three independent trends highlighted in \cite{Colgain:2022rxy} across OHD, Type Ia SN and QSOs are not \textit{bona fide} disagreements and that redshift evolution is present in the sample. The task remains to combine them at the level of a $\chi^2$ likelihood instead of combining them using Fisher's method on the basis that they are independent probabilities. We leave this exercise for a forthcoming paper, but revisit the tension in OHD data in the following section.  %\ref{sec:tension}. 
For completeness, in Table \ref{tab:LCDM_CC_PD} we perform a reanalysis of CC data subsets with the PD approach and record the $1 \sigma$ intervals.  

\begin{table}[htb]
    \centering
    \begin{tabular}{c|c|c|c}
    \rule{0pt}{3ex} $z_{\textrm{min}}$ & \# CC & \multicolumn{2}{c}{PD}  \\
    \hline
    \rule{0pt}{3ex} & & $H_0$ (km/s/Mpc) & $\Omega_m$ \\
    \hline
    \rule{0pt}{3ex} $0$ & $34$ & $68.15^{+3.04}_{-3.11}$ & $0.320^{+0.065}_{-0.055}$ \\
    \hline 
    \rule{0pt}{3ex} $0.2$ & $27$ & $65.03^{+6.52}_{-7.03}$ & $0.368^{+0.167}_{-0.110}$ \\
    \hline 
    \rule{0pt}{3ex} $0.4$ & $22$ & $62.42^{+7.78}_{-8.74}$ & $0.411^{+0.236}_{-0.113}$ \\
    \hline
    \rule{0pt}{3ex} $0.6$ & $15$ & $59.75^{+11.73}_{-13.97}$ & $0.455^{+0.355}_{-0.160}$ \\
    \hline
    \rule{0pt}{3ex} $0.7$ & $14$ & $79.10^{+16.42}_{-20.56}$ & $0.222^{+0.386}_{-0.117}$ \\
    \hline
    \rule{0pt}{3ex} $0.8$ & $11$ & $103.94^{+22.88}_{-28.54}$ & $0.097^{+0.378}_{-0.074}$ \\
    \hline
    \rule{0pt}{3ex} $1$ & $8$ & $150.35^{+17.12}_{-35.95}$ & $ < 0.339$ \\
    \hline
    \rule{0pt}{3ex} $1.2$ & $7$ & $154.26^{+14.88}_{-54.82}$ & $ < 0.570$ \\
    \hline
    \rule{0pt}{3ex} $1.4$ & $4$ & $124.81^{+35.38}_{-52.60}$ & $ < 0.661$ \\
    \hline
    \rule{0pt}{3ex} $1.5$ & $3$ & $36.11^{+72.87}_{-2.43}$ & $ > 0.354$
    \end{tabular}
    \caption{Same as Table \ref{tab:LCDM_CC} but with the PD methodology in lieu of Fisher matrix and MCMC analysis. The high redshift $R(\Omega_m)$ distributions are typically one-sided, so one encounters $1 \sigma$ upper and lower bounds.}
    \label{tab:LCDM_CC_PD}
\end{table}




\section{A tension with Planck}
\label{sec:tension}
A $2 \sigma$ ($p = 0.021$) tension with Planck has been reported in OHD through best fits and mock simulations in \cite{Colgain:2022rxy}. In particular, it was noted that a combination of 7 CC and BAO data points above $z = 1.45$ resulted in a $(H_0, \Omega_m) = (37.8, 1)$ best fit, where in line with analysis here, an $\Omega_m \in [0, 1]$ uniform prior was assumed. Based on mock simulations, the probability of such a best fit configuration arising by chance in mocks assuming input parameters consistent with Planck was estimated to be $p = 0.021$ \cite{Colgain:2022rxy}. A similar best fit appears in the last entry of Table \ref{tab:LCDM_CC} and Table \ref{tab:LCDM_CC_PD}, but there is no tension with Planck within the errors, even with our PD analysis, because CC data is inherently of poorer quality than BAO data. One further difference between the analysis is that \cite{Colgain:2022rxy} imposes a Gaussian Planck prior $\Omega_m h^2 = 0.1430 \pm 0.0011$ \cite{Planck:2018vyg} \footnote{This prior essentially prevents high redshift CC data from tracking a non-evolving $H(z)$.} to fix the high redshift behaviour of $H(z)$, whereas our analysis here so far has not introduced a prior. 

% Figure environment removed

Nevertheless, armed with a new PD methodology, we are in a position to revisit the earlier result and see if we can recover the $2 \sigma$ tension with Planck. Since \cite{Colgain:2022rxy} made use of older BAO data, here we replace QSO and Lyman-$\alpha$ BAO with the latest eBOSS results \cite{Hou:2020rse, Neveux:2020voa, duMasdesBourboux:2020pck}. Moreover, we work directly with the $D_{H}/r_d$ constraints and do not invert them. This entails assuming a value for the radius of the sound horizon, which we take to be the Planck value, $r_d = 147.09 \pm 0.26$ Mpc \cite{Planck:2018vyg}. In addition, we reinstate the prior $\Omega_m h^2 = 0.1430 \pm 0.0011$, so that the only difference with \cite{Colgain:2022rxy} is simply to update OHD BAO to the latest constraints. We stress that the priors we introduce are consistent with the Planck cosmology, so \textit{they cannot be driving any disagreement}. Moreover, the $\Omega_m h^2$ prior restricts one to a curve in the $(H_0, \Omega_m)$, but it cannot dictate where one is on the curve, this is done by the remaining 3 CC and 3 BAO data points.  

We again marginalise over the free parameters $(H_0, \Omega_m, r_d)$ with MCMC. In Fig.~\ref{fig:CC_BAO_MCMC} we present the posteriors. While $r_d$ is Gaussian and peaked on our Planck prior, as expected, the $\Omega_m$ posterior is peaked at $\Omega_m \sim 0.6$ and the fact that the fall off in the distribution is gradual beyond the peak leads to a pile up of configurations in the top left corner of the $(H_0, \Omega_m)$-plane. This fall off continues beyond $\Omega_m = 1$ and if the prior is relaxed, the $H_0$ peak shifts to smaller values. So,  once again all the hallmarks of projection effects are present. That being said, given the sharp fall off in the $\Omega_m$ distribution to smaller $\Omega_m$ values, some tension appears to be evident with the Planck values (dashed lines). 

% Figure environment removed

We now run the MCMC chain through our PD methodology. From Fig.~\ref{fig:CC_BAO}, we can see that the $R(H_0)$ and $R(\Omega_m)$ distributions prefer smaller values of $H_0$ and larger values of $\Omega_m$. The peak of the distributions occurs at $H_0 = 42.40$ km/s/Mpc and $\Omega_m = 0.795$.  The lone dot in the $R(H_0)$ distribution at low values of $H_0$ tells us that the distribution falls off sharply below $H_0 = 40$ km/s/Mpc. Note, since we employed generous uniform priors $H_0 \in [0, 200]$, the priors are not impacting the $R(H_0)$ distribution, so it is expected that the distribution falls off to zero on both sides. In contrast, the $R(\Omega_m)$ distribution is one-sided and fails to fall off in the direction of larger values within the uniform priors $\Omega_m \in [0, 1]$. The tension with Planck falls between $2 \sigma$ and $3 \sigma$. By integrating the PDF as far as the black lines corresponding to the Planck values in Fig.~\ref{fig:CC_BAO}, we estimate that the Planck $H_0$ is located at $2.1 \sigma$ from the peak, while the Planck $\Omega_m$ value is $2.5 \sigma$ from the peak.

The main take-away from this section is that OHD data comprising CC and BAO data points beyond $z=1.45$ is inconsistent with the Planck cosmology at in excess of $2 \sigma$. We have employed Planck priors to arrive at this result, but these priors cannot drive the disagreement. Moreover, independent analysis based on least squares fitting and mock simulations presented in \cite{Colgain:2022rxy} also points to a $2 \sigma$ tension, albeit with less up-to-date high redshift BAO data. In summary, different methodologies agree on a $2 \sigma$ discrepancy with Planck, which is robust to interchanging older and newer BAO data. 

\section{Concluding remarks}
\label{sec:discussion}
A $\chi^2$ likelihood is a metric or measure of how well a model fits data. The point in model parameter space that fits the data the best possesses the lowest $\chi^2$. Once one has identified this point, the problem remains to establish $1 \sigma$, $2 \sigma$, etc, confidence intervals in parameter space. In cosmology and astrophysics, MCMC is the prevailing technique for estimating confidence intervals. Its great advantage is that it allows one to i) globally sample the parameter space and ii) arrive at posteriors that serve as an estimate of the errors even with non-Gaussian distributions. In contrast, if one minimises the $\chi^2$ by gradient descent, there is always a risk that one ends up in a local minimum, i. e. the global minimum is missed, while error estimation through Fisher matrix assumes any distribution is Gaussian. The appeal of MCMC marginalisation is that it is widely applicable. However, the point of this paper is that limitations exist, even in the simplest model. 

Indeed, what happens when the MCMC posterior no longer tracks points in parameter space that fit the data better? Traditionally, volume effects are seen as the preserve of higher-dimensional models, e. g. \cite{Herold:2021ksg, Gomez-Valent:2022hkb, Meiers:2023gft}, but projection effects also occur in the minimal $\Lambda$CDM model when one fits the model to data binned by redshift in the late Universe \cite{Colgain:2022tql}. As explained in \cite{Colgain:2022tql}, this ``projection effect'' is driven by OHD, $H(z_i)$, and angular diameter or luminosity distance data, $D_{A}(z_i)$ or $D_{L}(z_i)$, {respectively} only constraining the combinations $\Omega_m h^2$ and $ (1-\Omega_m) h^2$ well, with high redshift data $z_i \gg 0$. In practice, this restricts MCMC configurations to constant $\Omega_m h^2$ and constant $(1-\Omega_m) h^2$ curves in the $(H_0, \Omega_m)$ plane, and as the curves stretch due to DE or matter being less well constrained in high redshift bins, projection effects lead to shifts in the peaks of MCMC posteriors and the emergence of non-Gaussian tails \cite{Colgain:2022tql}. We stress that one sees the same effect in PDFs of best fit $(H_0, \Omega_m)$ parameters in a large number of mock data realisations \cite{Colgain:2022tql}, so the problem is more general than MCMC; there is an inherent bias in the $\Lambda$CDM model when one fits it to redshift binned $H(z)$ \textit{or} $D_{A}(z)$ \textit{or} $D_{L}(z)$ data. Within MCMC, one sees this effect in the errors, but also in the drift of the parameters corresponding to the $\chi^2$ minimum outside of the $1 \sigma$ confidence intervals. Highlighting this (expected) bias in MCMC using OHD is the opening salvo (result) of this paper.     

Why should one care? This is evidently only a problem if one bins data and confronts the $\Lambda$CDM model. First, note that some data sets are inherently binned. For example, effective redshifts are assigned to CC and BAO analysed in a given redshift bin, while each strongly lensed system constitutes its own bin. Working with binned data is unavoidable. Secondly, $\Lambda$CDM tensions point to a problem with the $\Lambda$CDM model once the tensions become widespread and persistent. As explained in \cite{Krishnan:2020vaf}, if the minimal $\Lambda$CDM model is too simple, one expects redshift evolution of $\Lambda$CDM cosmological parameters as it is confronted to redshift binned data. Hints of these trends are now evident in $H_0$ \cite{Wong:2019kwg, Millon:2019slk, Dainotti:2021pqg, Colgain:2022nlb, Colgain:2022rxy, Malekjani:2023dky, Hu:2022kes, Jia:2022ycc, Krishnan:2020obg, Dainotti:2022bzg}, $\Omega_m$ \cite{Risaliti:2015zla, Risaliti:2018reu, Lusso:2020pdb, Yang:2019vgk, Khadka:2020vlh, Khadka:2020tlm, Khadka:2021xcc, Pourojaghi:2022zrh, Colgain:2022nlb, Colgain:2022rxy, Malekjani:2023dky, Pasten:2023rpc, Sakr:2023hrl} and $S_8$/$\sigma_8$ \cite{Esposito:2022plo, Adil:2023jtu, ACT:2023dou, ACT:2023kun} (also \cite{Miyatake:2021qjr, Alonso:2023guh}) across a host of different observables. This evolution is an expected hallmark of model breakdown, which must happen at some redshift if systematics are not universally at play. 

The main problem with redshift dependent $\Lambda$CDM cosmological parameters\footnote{There is a separate interpretation problem as the cosmology literature works with  parameters ``defined today''. In more mathematical language, this is simply the statement that one solves an ordinary differential equation (ODE), namely the Friedmann equation or equivalent, by specifying an integration constant, e.g. $H_0 = H(z=0)$ or $\rho_m(z=0)=\rho_{m0}=H_0^2\Omega_{m}$. However, this is a mathematical statement and it still needs to be confirmed observationally that $H_0$ or $\rho_{m0}$ are \textit{bona fide} constants. This cannot be \textit{a priori} assumed, because it is mathematical prediction of the model. If the model is correct, a constant $H_0$ and $\Omega_m$  will be supported by the data. See \cite{Krishnan:2020vaf} for further discussion.} is one needs to assign a statistical significance to any trend. At a purely practical level, this entails constructing bins centered on different redshifts and identifying discrepancies in $\Lambda$CDM parameters between bins, \textit{ideally in the same observable}, so that the potential systematics are under greatest control. As demonstrated both mathematically and observationally with the CC data in section \ref{sec:MCMC_bias}, MCMC marginalisation leads to biased inferences when one bins the data. In this paper we have resorted to profile distributions \cite{Gomez-Valent:2022hkb} to overcome this bias and have applied the technique to a setting where $\Lambda$CDM distributions are expected to be non-Gaussian for the reasons outlined above and in section \ref{sec:MCMC_bias}. This new technique, provides a complementary perspective that confirms the least square fits of observed and mock data presented in \cite{Colgain:2022nlb, Colgain:2022rxy, Malekjani:2023dky}, where evidence for redshift evolution in $H_0$ and $\Omega_m$ was presented. Regardless of the methodology, the objective is to drill down on the prevailing \textit{assumption} that cosmological parameters are constants. \textit{In the era of tensions in cosmology, nothing can be assumed, especially noting that the tensions are in essence showing an example of evolution of these parameters with redshift.}

More concretely, in this paper with both mock simulations and profile distributions we have shown that high redshift CC data has a preference for a non-evolving $H(z)$ over Planck-$\Lambda$CDM at approximately $\sim 2 \sigma$. This trend, which constitutes the second result of the paper, is unquestionable, as it is visible in the data. Note, we have not propagated systematic uncertainties, so the significance will be less when these are properly propagate. Nevertheless, low and high redshift CC data currently have a preference for different $\Lambda$CDM cosmological parameters. This is important because if the CC program is claiming an 8\% constraint on the Hubble constant, $H_0 = 66.7 \pm 5.5$ km/s/Mpc \cite{Moresco:2023zys}, it is imperative that \textit{all subsets of the data are consistent with this result}. If they are not, then we are staring at either systematics or model breakdown. Admittedly, demanding self-consistency of subsets of a data set confronted to a model is a high bar, but it is important that data sets result in overlapping constraints on $\Lambda$CDM parameters, otherwise this makes cosmological inferences moot. Note, the $\Lambda$CDM model is largely only well tested in the DE dominated regime $z \lesssim 1$ and at very high redshifts $z \sim 1100$, which leaves a wide expanse of redshifts to be explored in order to confirm or refute the model. Given the existing $\Lambda$CDM tensions \cite{Perivolaropoulos:2021jda, Abdalla:2022yfr}, and the hints of evolution in $H_0$, $\Omega_m$ and $S_8$ across assorted probes in the late Universe $z \lesssim 5$, it would be surprising if all discrepancies could be explained away by systematics.\footnote{We are open to the possibility, we just consider it a bad bet at the moment. The odds can of course change as observations improve.}

As an aside, it is intriguing that CC data has a preference for larger best fit values of $H_0$ and smaller best fit values of $\Omega_m$ beyond $z_{\textrm{min}} = 0.7$, as this is traditionally the transition redshift between decelerated and accelerated expansion. % where $\ddot{a} = 0$. 
Moreover, at higher redshifts $z \sim 2.3$, there is not only a longstanding anomaly in Lyman-$\alpha$ BAO \cite{duMasdesBourboux:2020pck}, but QSOs also show a preference for a lower luminosity distance, $D_{L}(z)$, relative to Planck-$\Lambda$CDM \cite{Risaliti:2015zla, Risaliti:2018reu}. Translated into $\Lambda$CDM parameters, this corresponds to conversely larger $\Omega_m$ values, e. g.  \cite{Yang:2019vgk, Khadka:2020vlh, Khadka:2020tlm, Khadka:2021xcc, Pourojaghi:2022zrh}, and consequently smaller $H_0$ values. Thus, the emerging probes CC and QSOs  \cite{Moresco:2022phi} do not appear to be in sync on high redshift $\Lambda$CDM inferences. Nevertheless, neither may be inconsistent with the anomaly in Lyman-$\alpha$ BAO. Relative to Planck-$\Lambda$CDM, Lyman-$\alpha$ BAO prefers \textit{smaller} values of $D_{M}(z) := c \int_{0}^z 1/H(z^{\prime}) \, \textrm{d} z$ and \textit{smaller} values of $H(z)$ (larger values of $D_{H}(z) := c/H(z)$).\footnote{In this statement we assumed the Planck value $r_d \sim 147$ Mpc \cite{Planck:2018vyg} If we reinstate the radius of the sound horizon in these expressions, one recognises that changing the sound horizon, as advocated by early Universe resolutions to Hubble tension, cannot consistently address the Lyman-$\alpha$ BAO anomaly. In general, even for the Planck-$\Lambda$CDM sound horizon, one cannot get both a smaller $D_{M}(z)$ and smaller $H(z)$ from a strictly increasing function, such as the $\Lambda$CDM $H(z)$. As a result, deviations from the Planck-$\Lambda$CDM model that address this anomaly are expected to lead to wiggles in $H(z)$ \cite{Akarsu:2022lhx}, which are unsurprisingly seen in data reconstructions \cite{Zhao:2017cud, Wang:2018fng, Escamilla:2021uoj}. Finally, evolution in $H_0, \Omega_m$ discussed here cannot be explained or accommodated by early resolutions to Hubble tension relying on a change in the $r_d$ at very high $z$.}. If CC data prefer less evolution in $H(z)$ in the matter-dominated regime, then this is consistent with the preference for a smaller $H(z)$ from Lyman-$\alpha$ BAO. Furthermore, QSO data prefers smaller luminosity distances $D_{L}(z)$ relative to Planck, which are consistent with the smaller $D_{M}(z) \propto D_{L}(z)$ values preferred by Lyman-$\alpha$ BAO. Thus, even if CC and QSOs appear to be showing diverging behaviour in the cosmological parameters $(H_0, \Omega_m)$, this may still turn out to be consistent with Lyman-$\alpha$ BAO. We await future DESI \cite{DESI:2023ytc} data releases to ascertain if the non-evolving $H(z)$ trend in high redshift CC data is physical or not. 

Finally, we come to our third and main result outlined in section \ref{sec:tension}. We have revisited a $\sim 2 \sigma$ tension between high redshift CC and BAO data reported in \cite{Colgain:2022rxy}, where the significance was estimated through mock simulations. Here, we have upgraded the BAO data to the latest constraints and again  recover a $>2 \sigma$ discrepancy in $(H_0, \Omega_m)$ with different methodology. This provides a consistency check that there is evolution in OHD between low and high redshifts in the late Universe. Note, this evolution runs contrary to the non-evolving $H(z)$ seen in high redshift CC data because it assumes Planck has accurately constrained the high redshift behaviour of the Hubble parameter in (\ref{eq:lcdm}). Nevertheless, both with and without a Planck prior on $\Omega_m h^2$, evolution at $ \gtrsim 2 \sigma$ is evident in OHD data. It should be stressed that evolution is evident in PDFs of best fit $\Lambda$CDM parameters fitted to a large number of Planck-$\Lambda$CDM mocks \cite{Colgain:2022tql}, so evolution in observed data can be expected. It is imperative to revisit the remaining observations in \cite{Colgain:2022rxy, Malekjani:2023dky} in order to confirm the significance of $\sim 2 \sigma$ hints of evolution found separately in Type Ia SN and QSO data sets. 




\acknowledgments
We would like to thank Adri\`a G\'omez-Valent for discussions and comments on the draft. We thank Gabriela Marques, Mike Hudson and Matteo Viel for related discussions on late Universe evolution in $S_8$. E\'OC thanks Yonsei University and Asia Pacific Center for Theoretical Physics for hospitality. 
This article/publication is based upon work from COST Action CA21136 – “Addressing observational tensions in cosmology with systematics and fundamental physics (CosmoVerse)”, supported by COST (European Cooperation in Science and Technology). SP and MMShJ acknowledge SarAmadan grant No. ISEF/M/401332. MMShJ thanks the support from ICTP associates office (under Senior Associate program) and ICTP HECAP section for hospitality.  


\appendix
\section{Fisher Matrix}
\label{sec:fisher}
Consider the $\chi^2$ (\ref{eq:chi2}). 
Defining $H_{\textrm{model}}(z) = H_0 \sqrt{1-\Omega_m + \Omega_m (1+z)^3}$ and $Q_i$ as in \eqref{eq:Q}, we can now work out the derivatives
\begin{equation}
    \begin{split}
\partial_{H_0} Q_i &= -\sqrt{1-\Omega_m + \Omega_m (1+z_i)^3}, \\  \partial_{\Omega_m} Q_i &= - \frac{1}{2} H_0 (z_i^3 + 3 z_i^2 + 3 z_i)/\sqrt{1-\Omega_m + \Omega_m (1+z_i)^3}, \\
\partial^2_{H_0} Q_i &= 0, \\
\partial_{H_0} \partial_{\Omega_m} Q_i &= - \frac{1}{2} (z_i^3 + 3 z_i^2 + 3 z_i)/\sqrt{1-\Omega_m + \Omega_m (1+z_i)^3}, \\
\partial^2_{\Omega_m} Q_i =& \frac{1}{4} H_0 (z_i^3 + 3 z_i^2 + 3 z_i)^2/(1-\Omega_m + \Omega_m (1+z_i)^3)^{\frac{3}{2}}.      
    \end{split}
\end{equation}
We can then define the Fisher matrix 
\be
F_{ij} = \frac{1}{2} \frac{\partial^2 \chi^2(H_0, \Omega_m)}{\partial p_i \partial p_j}
\ee
where $p_i \in \{ H_0, \Omega_m \}$. Note that the Fisher matrix is evaluated on the best fit parameters. The result is a $2 \times 2$ matrix, which one inverts and the estimated errors are the square root of the diagonal entries. 








\begin{thebibliography}{99}

\bibitem{Planck:2018vyg}
N.~Aghanim \textit{et al.} [Planck],
``Planck 2018 results. VI. Cosmological parameters,''
Astron. Astrophys. \textbf{641} (2020), A6
% doi:10.1051/0004-6361/201833910
%[arXiv:1807.06209 [astro-ph.CO]].

\bibitem{Riess:1998cb}
A.~G.~Riess \textit{et al.} [Supernova Search Team],
``Observational evidence from supernovae for an accelerating universe and a cosmological constant,''
Astron. J. \textbf{116} (1998), 1009-1038
% doi:10.1086/300499
%[arXiv:astro-ph/9805201 [astro-ph]].
%13031 citations counted in INSPIRE as of 02 Feb 2021

\bibitem{Perlmutter:1998np}
S.~Perlmutter \textit{et al.} [Supernova Cosmology Project],
``Measurements of $\Omega$ and $\Lambda$ from 42 high redshift supernovae,''
Astrophys. J. \textbf{517} (1999), 565-586
% doi:10.1086/307221
%[arXiv:astro-ph/9812133 [astro-ph]].
%13057 citations counted in INSPIRE as of 02 Feb 2021

\bibitem{Eisenstein:2005su}
D.~J.~Eisenstein \textit{et al.} [SDSS],
``Detection of the Baryon Acoustic Peak in the Large-Scale Correlation Function of SDSS Luminous Red Galaxies,''
Astrophys. J. \textbf{633} (2005), 560-574
%doi:10.1086/466512
%[arXiv:astro-ph/0501171 [astro-ph]].
%3380 citations counted in INSPIRE as of 08 Oct 2020

\bibitem{Riess:2021jrx}
A.~G.~Riess, W.~Yuan, L.~M.~Macri, D.~Scolnic, D.~Brout, S.~Casertano, D.~O.~Jones, Y.~Murakami, L.~Breuval and T.~G.~Brink, \textit{et al.}
``A Comprehensive Measurement of the Local Value of the Hubble Constant with 1 km s$^{?1}$ Mpc$^{?1}$ Uncertainty from the Hubble Space Telescope and the SH0ES Team,''
Astrophys. J. Lett. \textbf{934} (2022) no.1, L7
%doi:10.3847/2041-8213/ac5c5b
%[arXiv:2112.04510 [astro-ph.CO]].
%370 citations counted in INSPIRE as of 09 Jan 2023

\bibitem{Freedman:2021ahq}
W.~L.~Freedman,
``Measurements of the Hubble Constant: Tensions in Perspective,''
Astrophys. J. \textbf{919} (2021) no.1, 16
%doi:10.3847/1538-4357/ac0e95
%[arXiv:2106.15656 [astro-ph.CO]].
%179 citations counted in INSPIRE as of 09 Jan 2023

\bibitem{Pesce:2020xfe}
D.~W.~Pesce, J.~A.~Braatz, M.~J.~Reid, A.~G.~Riess, D.~Scolnic, J.~J.~Condon, F.~Gao, C.~Henkel, C.~M.~V.~Impellizzeri and C.~Y.~Kuo, \textit{et al.}
%``The Megamaser Cosmology Project. XIII. Combined Hubble constant constraints,''
Astrophys. J. Lett. \textbf{891} (2020) no.1, L1
%doi:10.3847/2041-8213/ab75f0
%[arXiv:2001.09213 [astro-ph.CO]].
%96 citations counted in INSPIRE as of 12 Jul 2021

\bibitem{Blakeslee:2021rqi}
J.~P.~Blakeslee, J.~B.~Jensen, C.~P.~Ma, P.~A.~Milne and J.~E.~Greene,
%``The Hubble Constant from Infrared Surface Brightness Fluctuation Distances,''
Astrophys. J. \textbf{911} (2021) no.1, 65
%doi:10.3847/1538-4357/abe86a
%[arXiv:2101.02221 [astro-ph.CO]].
%11 citations counted in INSPIRE as of 12 Jul 2021

\bibitem{Kourkchi:2020iyz}
E.~Kourkchi, R.~B.~Tully, G.~S.~Anand, H.~M.~Courtois, A.~Dupuy, J.~D.~Neill, L.~Rizzi and M.~Seibert,
%``Cosmicflows-4: The Calibration of Optical and Infrared Tully\textendash{}Fisher Relations,''
Astrophys. J. \textbf{896} (2020) no.1, 3
%doi:10.3847/1538-4357/ab901c
%[arXiv:2004.14499 [astro-ph.GA]].
%15 citations counted in INSPIRE as of 12 Jul 2021

\bibitem{HSC:2018mrq}
C.~Hikage \textit{et al.} [HSC],
``Cosmology from cosmic shear power spectra with Subaru Hyper Suprime-Cam first-year data,''
Publ. Astron. Soc. Jap. \textbf{71}, 43  (2019).
%doi:10.1093/pasj/psz010

\bibitem{KiDS:2020suj}
M.~Asgari \textit{et al.} [KiDS],
``KiDS-1000 Cosmology: Cosmic shear constraints and comparison between two point statistics,''
Astron. Astrophys. \textbf{645} (2021), A104
%doi:10.1051/0004-6361/202039070
%[arXiv:2007.15633 [astro-ph.CO]].
%113 citations counted in INSPIRE as of 18 Aug 2021

\bibitem{DES:2021wwk}
T.~M.~C.~Abbott \textit{et al.} [DES],
``Dark Energy Survey Year 3 results: Cosmological constraints from galaxy clustering and weak lensing,''
Phys. Rev. D \textbf{105} (2022) no.2, 023520
%doi:10.1103/PhysRevD.105.023520
%[arXiv:2105.13549 [astro-ph.CO]].
%519 citations counted in INSPIRE as of 14 Jul 2023

\bibitem{Boruah:2019icj}
S.~S.~Boruah, M.~J.~Hudson and G.~Lavaux,
``Cosmic flows in the nearby Universe: new peculiar velocities from SNe and cosmological constraints,''
Mon. Not. Roy. Astron. Soc. \textbf{498} (2020) no.2, 2703-2718
%doi:10.1093/mnras/staa2485
%[arXiv:1912.09383 [astro-ph.CO]].
%54 citations counted in INSPIRE as of 14 Jul 2023

\bibitem{Said:2020epb}
K.~Said, M.~Colless, C.~Magoulas, J.~R.~Lucey and M.~J.~Hudson,
``Joint analysis of 6dFGS and SDSS peculiar velocities for the growth rate of cosmic structure and tests of gravity,''
Mon. Not. Roy. Astron. Soc. \textbf{497} (2020) no.1, 1275-1293
%doi:10.1093/mnras/staa2032
%[arXiv:2007.04993 [astro-ph.CO]].
%49 citations counted in INSPIRE as of 14 Jul 2023

\bibitem{Perivolaropoulos:2021jda}
L.~Perivolaropoulos and F.~Skara,
``Challenges for \ensuremath{\Lambda}CDM: An update,''
New Astron. Rev. \textbf{95}, 101659  (2022).
%doi:10.1016/j.newar.2022.101659
%\href{https://arxiv.org/abs/2105.05208}{2105.05208}

\bibitem{Abdalla:2022yfr}
E.~Abdalla, G.~Franco Abell\'an, A.~Aboubrahim, A.~Agnello, O.~Akarsu, Y.~Akrami, G.~Alestas, D.~Aloni, L.~Amendola and L.~A.~Anchordoqui, \textit{et al.}
``Cosmology intertwined: A review of the particle physics, astrophysics, and cosmology associated with the cosmological tensions and anomalies,''
JHEAp \textbf{34}, 49  (2022).
%doi:10.1016/j.jheap.2022.04.002
%\href{https://arxiv.org/abs/2203.06142}{2203.06142}

\bibitem{Phillips:1993ng}
M.~M.~Phillips,
``The absolute magnitudes of Type IA supernovae,''
Astrophys. J. Lett. \textbf{413} (1993), L105-L108
%doi:10.1086/186970
%1245 citations counted in INSPIRE as of 24 Aug 2021

\bibitem{NearbySupernovaFactory:2018qkd}
M.~Rigault \textit{et al.} [Nearby Supernova Factory],
``Strong Dependence of Type Ia Supernova Standardization on the Local Specific Star Formation Rate,''
Astron. Astrophys. \textbf{644} (2020), A176
%doi:10.1051/0004-6361/201730404
%[arXiv:1806.03849 [astro-ph.CO]].
%143 citations counted in INSPIRE as of 20 Jul 2023

\bibitem{Kang:2019azh}
Y.~Kang, Y.~W.~Lee, Y.~L.~Kim, C.~Chung and C.~H.~Ree,
``Early-type Host Galaxies of Type Ia Supernovae. II. Evidence for Luminosity Evolution in Supernova Cosmology,''
Astrophys. J. \textbf{889} (2020) no.1, 8
%doi:10.3847/1538-4357/ab5afc
%[arXiv:1912.04903 [astro-ph.GA]].
%56 citations counted in INSPIRE as of 20 Jul 2023

\bibitem{Brout:2020msh}
D.~Brout and D.~Scolnic,
``It\textquoteright{}s Dust: Solving the Mysteries of the Intrinsic Scatter and Host-galaxy Dependence of Standardized Type Ia Supernova Brightnesses,''
Astrophys. J. \textbf{909} (2021) no.1, 26
%doi:10.3847/1538-4357/abd69b
%[arXiv:2004.10206 [astro-ph.CO]].
%82 citations counted in INSPIRE as of 20 Jul 2023

\bibitem{Lee:2021txi}
Y.~W.~Lee, C.~Chung, P.~Demarque, S.~Park, J.~Son and Y.~Kang,
``Evidence for strong progenitor age dependence of type Ia supernova luminosity standardization process,''
Mon. Not. Roy. Astron. Soc. \textbf{517} (2022) no.2, 2697-2708
%doi:10.1093/mnras/stac2840
%[arXiv:2107.06288 [astro-ph.GA]].
%5 citations counted in INSPIRE as of 20 Jul 2023


\bibitem{Krishnan:2020vaf}
C.~Krishnan, E.~\'O~Colg\'ain, M.~M.~Sheikh-Jabbari and T.~Yang,
``Running Hubble Tension and a H0 Diagnostic,''
Phys. Rev. D \textbf{103} (2021) no.10, 103509
%doi:10.1103/PhysRevD.103.103509
%[arXiv:2011.02858 [astro-ph.CO]].
%65 citations counted in INSPIRE as of 14 Jul 2023 

\bibitem{Krishnan:2022fzz}
C.~Krishnan and R.~Mondol,
``$H_0$ as a Universal FLRW Diagnostic,''
[arXiv:2201.13384 [astro-ph.CO]].
%12 citations counted in INSPIRE as of 14 Jul 2023

%\bibitem{Liao:2020zko}
%K.~Liao, A.~Shafieloo, R.~E.~Keeley and E.~V.~Linder,
%``Determining Model-independent H 0 and Consistency Tests,''
%Astrophys. J. Lett. \textbf{895} (2020) no.2, L29
%doi:10.3847/2041-8213/ab8dbb
%[arXiv:2002.10605 [astro-ph.CO]].
%51 citations counted in INSPIRE as of 14 Jul 2023

%\bibitem{Montani:2023xpd}
%G.~Montani, M.~De Angelis, F.~Bombacigno and N.~Carlevaro,
%``Metric $f(R)$ gravity with dynamical dark energy as a paradigm for the Hubble Tension,''
%[arXiv:2306.11101 [gr-qc]].
%1 citations counted in INSPIRE as of 14 Jul 2023

\bibitem{Wong:2019kwg}
K.~C.~Wong, S.~H.~Suyu, G.~C.~F.~Chen, C.~E.~Rusu, M.~Millon, D.~Sluse, V.~Bonvin, C.~D.~Fassnacht, S.~Taubenberger and M.~W.~Auger, \textit{et al.}
``H0LiCOW \textendash{} XIII. A 2.4 per cent measurement of H0 from lensed quasars: 5.3\ensuremath{\sigma} tension between early- and late-Universe probes,''
Mon. Not. Roy. Astron. Soc. \textbf{498} (2020) no.1, 1420-1439
%doi:10.1093/mnras/stz3094
%[arXiv:1907.04869 [astro-ph.CO]].
%804 citations counted in INSPIRE as of 18 May 2023

\bibitem{Millon:2019slk}
M.~Millon, A.~Galan, F.~Courbin, T.~Treu, S.~H.~Suyu, X.~Ding, S.~Birrer, G.~C.~F.~Chen, A.~J.~Shajib and D.~Sluse, \textit{et al.}
``TDCOSMO. I. An exploration of systematic uncertainties in the inference of $H_0$ from time-delay cosmography,''
Astron. Astrophys. \textbf{639} (2020), A101
%doi:10.1051/0004-6361/201937351
%[arXiv:1912.08027 [astro-ph.CO]].
%114 citations counted in INSPIRE as of 18 May 2023

\bibitem{Sluse:2003iy}
D.~Sluse, J.~Surdej, J.~F.~Claeskens, D.~Hutsemekers, C.~Jean, F.~Courbin, T.~Nakos, M.~Billeres and S.~V.~Khmil,
``A Quadruply imaged quasar with an optical Einstein ring candidate: 1RXS J113155.4-123155,''
Astron. Astrophys. \textbf{406} (2003), L43-L46
%doi:10.1051/0004-6361:20030904
%[arXiv:astro-ph/0307345 [astro-ph]].
%83 citations counted in INSPIRE as of 14 Jul 2023

\bibitem{Shajib:2023uig}
A.~J.~Shajib, P.~Mozumdar, G.~C.~F.~Chen, T.~Treu, M.~Cappellari, S.~Knabel, S.~H.~Suyu, V.~N.~Bennert, J.~A.~Frieman and D.~Sluse, \textit{et al.}
``TDCOSMO. XIII. Improved Hubble constant measurement from lensing time delays using spatially resolved stellar kinematics of the lens galaxy,''
Astron. Astrophys. \textbf{673} (2023), A9
%doi:10.1051/0004-6361/202345878
%[arXiv:2301.02656 [astro-ph.CO]].
%3 citations counted in INSPIRE as of 18 May 2023

\bibitem{Dainotti:2021pqg}
M.~G.~Dainotti, B.~De Simone, T.~Schiavone, G.~Montani, E.~Rinaldi and G.~Lambiase,
``On the Hubble constant tension in the SNe Ia Pantheon sample,''
Astrophys. J. \textbf{912}, 150  (2021).
%doi:10.3847/1538-4357/abeb73


\bibitem{Colgain:2022nlb}
E.~\'O~Colg\'ain, M.~M.~Sheikh-Jabbari, R.~Solomon, G.~Bargiacchi, S.~Capozziello, M.~G.~Dainotti and D.~Stojkovic,
``Revealing intrinsic flat \ensuremath{\Lambda}CDM biases with standardizable candles,''
Phys. Rev. D \textbf{106}, L041301  (2022).
%doi:10.1103/PhysRevD.106.L041301

\bibitem{Colgain:2022rxy}
E.~\'O~Colg\'ain, M.~M.~Sheikh-Jabbari, R.~Solomon, M.~G.~Dainotti and D.~Stojkovic,
``Putting Flat $\Lambda$CDM In The (Redshift) Bin,''
[arXiv:2206.11447 [astro-ph.CO]].
%42 citations counted in INSPIRE as of 14 Jul 2023

%\cite{Colgain:2022tql}
%\bibitem{Colgain:2022tql}
%E.~\'O.~Colg\'ain, M.~M.~Sheikh-Jabbari and R.~Solomon,
%``High redshift \ensuremath{\Lambda}CDM cosmology: To bin or not to bin?,''
%Phys. Dark Univ. \textbf{40} (2023), 101216
%doi:10.1016/j.dark.2023.101216
%[arXiv:2211.02129 [astro-ph.CO]].
%10 citations counted in INSPIRE as of 25 Jul 2023


\bibitem{Malekjani:2023dky}
M.~Malekjani, R.~M.~Conville, E.~\'O.~Colg\'ain, S.~Pourojaghi and M.~M.~Sheikh-Jabbari,
``Negative Dark Energy Density from High Redshift Pantheon+ Supernovae,''
[arXiv:2301.12725 [astro-ph.CO]].
%13 citations counted in INSPIRE as of 17 Jul 2023

\bibitem{Hu:2022kes}
J.~P.~Hu and F.~Y.~Wang,
``Revealing the late-time transition of H0: relieve the Hubble crisis,''
Mon. Not. Roy. Astron. Soc. \textbf{517}, 576  (2022).

\bibitem{Jia:2022ycc}
X.~D.~Jia, J.~P.~Hu and F.~Y.~Wang,
``Evidence of a decreasing trend for the Hubble constant,''
Astron. Astrophys. \textbf{674} (2023), A45
%doi:10.1051/0004-6361/202346356
%[arXiv:2212.00238 [astro-ph.CO]].
%10 citations counted in INSPIRE as of 17 Jul 2023

\bibitem{Krishnan:2020obg}
C.~Krishnan, E.~\'O~Colg\'ain, Ruchika, A.~A.~Sen, M.~M.~Sheikh-Jabbari and T.~Yang,
``Is there an early Universe solution to Hubble tension?,''
Phys. Rev. D \textbf{102} (2020) no.10, 103525
%doi:10.1103/PhysRevD.102.103525
%[arXiv:2002.06044 [astro-ph.CO]].
%69 citations counted in INSPIRE as of 17 Jul 2023

\bibitem{Dainotti:2022bzg}
M.~G.~Dainotti, B.~De Simone, T.~Schiavone, G.~Montani, E.~Rinaldi, G.~Lambiase, M.~Bogdan and S.~Ugale,
``On the Evolution of the Hubble Constant with the SNe Ia Pantheon Sample and Baryon Acoustic Oscillations: A Feasibility Study for GRB-Cosmology in 2030,''
Galaxies \textbf{10}, 24  (2022).
%doi:10.3390/galaxies10010024

\bibitem{Risaliti:2015zla}
G.~Risaliti and E.~Lusso,
``A Hubble Diagram for Quasars,''
Astrophys. J. \textbf{815} (2015), 33
%doi:10.1088/0004-637X/815/1/33
%[arXiv:1505.07118 [astro-ph.CO]].
%146 citations counted in INSPIRE as of 16 Jun 2023

\bibitem{Risaliti:2018reu}
G.~Risaliti and E.~Lusso,
``Cosmological constraints from the Hubble diagram of quasars at high redshifts,''
Nature Astron. \textbf{3}, 272  (2019).

\bibitem{Lusso:2020pdb}
E.~Lusso, G.~Risaliti, E.~Nardini, G.~Bargiacchi, M.~Benetti, S.~Bisogni, S.~Capozziello, F.~Civano, L.~Eggleston and M.~Elvis, \textit{et al.}
``Quasars as standard candles III. Validation of a new sample for cosmological studies,''
Astron. Astrophys. \textbf{642}, A150  (2020).


\bibitem{Yang:2019vgk}
T.~Yang, A.~Banerjee and E.~\'O~Colg\'ain,
``Cosmography and flat $\Lambda$CDM tensions at high redshift,''
Phys. Rev. D \textbf{102}, 123532  (2020).

\bibitem{Khadka:2020vlh}
N.~Khadka and B.~Ratra,
``Using quasar X-ray and UV flux measurements to constrain cosmological model parameters,''
Mon. Not. Roy. Astron. Soc. \textbf{497}, 263  (2020).


\bibitem{Khadka:2020tlm}
N.~Khadka and B.~Ratra,
``Determining the range of validity of quasar X-ray and UV flux measurements for constraining cosmological model parameters,''
Mon. Not. Roy. Astron. Soc. \textbf{502}, 6140  (2021).


\bibitem{Khadka:2021xcc}
N.~Khadka and B.~Ratra,
``Do quasar X-ray and UV flux measurements provide a useful test of cosmological models?,''
Mon. Not. Roy. Astron. Soc. \textbf{510}, 2753  (2022).
%doi:10.1093/mnras/stab3678

\bibitem{Pourojaghi:2022zrh}
S.~Pourojaghi, N.~F.~Zabihi and M.~Malekjani,
``Can high-redshift Hubble diagrams rule out the standard model of cosmology in the context of cosmography?,''
Phys. Rev. D \textbf{106}, 123523  (2022).


\bibitem{Zajacek:2023qjm}
M.~Zaja\v{c}ek, B.~Czerny, N.~Khadka, R.~Prince, S.~Panda, M.~L.~Mart\'\i{}nez-Aldama and B.~Ratra,
``Extinction biases quasar luminosity distances determined from quasar UV and X-ray flux measurements,''
[arXiv:2305.08179 [astro-ph.GA]].
%0 citations counted in INSPIRE as of 17 Jul 2023

\bibitem{Pasten:2023rpc}
E.~Past\'en and V.~H.~C\'ardenas,
``Testing \ensuremath{\Lambda}CDM cosmology in a binned universe: Anomalies in the deceleration parameter,''
Phys. Dark Univ. \textbf{40} (2023), 101224
%doi:10.1016/j.dark.2023.101224
%[arXiv:2301.10740 [astro-ph.CO]].

\bibitem{Wagner:2022etu}
J.~Wagner,
``Casting the $H_0$ tension as a fitting problem of cosmologies,''
[arXiv:2203.11219 [astro-ph.CO]].
%5 citations counted in INSPIRE as of 28 Jul 2023

\bibitem{Sakr:2023hrl}
Z.~Sakr,
``One matter density discrepancy to alleviate them all or further trouble for $\Lambda$CDM model,''
[arXiv:2305.02846 [astro-ph.CO]].
%0 citations counted in INSPIRE as of 24 Jul 2023


\bibitem{Colgain:2022tql}
E.~\'O~Colg\'ain, M.~M.~Sheikh-Jabbari and R.~Solomon,
``High redshift \ensuremath{\Lambda}CDM cosmology: To bin or not to bin?,''
Phys. Dark Univ. \textbf{40} (2023), 101216
%doi:10.1016/j.dark.2023.101216
[arXiv:2211.02129 [astro-ph.CO]].
%10 citations counted in INSPIRE as of 28 Jun 2023

\bibitem{Esposito:2022plo}
M.~Esposito, V.~Ir\v{s}i\v{c}, M.~Costanzi, S.~Borgani, A.~Saro and M.~Viel,
``Weighing cosmic structures with clusters of galaxies and the intergalactic medium,''
Mon. Not. Roy. Astron. Soc. \textbf{515}, 857  (2022).
%doi:10.1093/mnras/stac1825
[arXiv:2202.00974 [astro-ph.CO]].

\bibitem{Adil:2023jtu}
S.~A.~Adil, \"O.~Akarsu, M.~Malekjani, E.~\'O~Colg\'ain, S.~Pourojaghi, A.~A.~Sen and M.~M.~Sheikh-Jabbari,
``$S_8$ increases with effective redshift in $\Lambda$CDM cosmology,''
[arXiv:2303.06928 [astro-ph.CO]].
%1 citations counted in INSPIRE as of 14 Jul 2023

\bibitem{ACT:2023dou}
F.~J.~Qu \textit{et al.} [ACT],
``The Atacama Cosmology Telescope: A Measurement of the DR6 CMB Lensing Power Spectrum and its Implications for Structure Growth,''
[arXiv:2304.05202 [astro-ph.CO]].
%10 citations counted in INSPIRE as of 14 Jul 2023

\bibitem{ACT:2023kun}
M.~S.~Madhavacheril \textit{et al.} [ACT],
``The Atacama Cosmology Telescope: DR6 Gravitational Lensing Map and Cosmological Parameters,''
[arXiv:2304.05203 [astro-ph.CO]].
%10 citations counted in INSPIRE as of 14 Jul 2023

\bibitem{ACT:2023ipp}
G.~A.~Marques \textit{et al.} [ACT and DES],
``Cosmological constraints from the tomography of DES-Y3 galaxies with CMB lensing from ACT DR4,''
[arXiv:2306.17268 [astro-ph.CO]].
%0 citations counted in INSPIRE as of 14 Jul 2023

\bibitem{Miyatake:2021qjr}
H.~Miyatake, Y.~Harikane, M.~Ouchi, Y.~Ono, N.~Yamamoto, A.~J.~Nishizawa, N.~Bahcall, S.~Miyazaki and A.~A.~Plazas Malag\'on,
``First Identification of a CMB Lensing Signal Produced by 1.5~Million Galaxies at z\ensuremath{\sim}4: Constraints on Matter Density Fluctuations at High Redshift,''
Phys. Rev. Lett. \textbf{129} (2022) no.6, 061301
%doi:10.1103/PhysRevLett.129.061301
[arXiv:2103.15862 [astro-ph.CO]].
%7 citations counted in INSPIRE as of 25 Jul 2023

\bibitem{Alonso:2023guh}
D.~Alonso, G.~Fabbian, K.~Storey-Fisher, A.~C.~Eilers, C.~Garc\'\i{}a-Garc\'\i{}a, D.~W.~Hogg and H.~W.~Rix,
``Constraining cosmology with the Gaia-unWISE Quasar Catalog and CMB lensing: structure growth,''
[arXiv:2306.17748 [astro-ph.CO]].
%0 citations counted in INSPIRE as of 25 Jul 2023


\bibitem{Herold:2021ksg}
L.~Herold, E.~G.~M.~Ferreira and E.~Komatsu,
``New Constraint on Early Dark Energy from Planck and BOSS Data Using the Profile Likelihood,''
Astrophys. J. Lett. \textbf{929} (2022) no.1, L16
%doi:10.3847/2041-8213/ac63a3
%[arXiv:2112.12140 [astro-ph.CO]].
%43 citations counted in INSPIRE as of 17 Jul 2023

\bibitem{Gomez-Valent:2022hkb}
A.~G\'omez-Valent,
``Fast test to assess the impact of marginalization in Monte~Carlo analyses and its application to cosmology,''
Phys. Rev. D \textbf{106} (2022) no.6, 063506
%doi:10.1103/PhysRevD.106.063506
%[arXiv:2203.16285 [astro-ph.CO]].
%20 citations counted in INSPIRE as of 11 Jul 2023

\bibitem{Meiers:2023gft}
M.~Meiers, L.~Knox and N.~Sch\"oneberg,
``Exploration of the Pre-recombination Universe with a High-Dimensional Model of an Additional Dark Fluid,''
[arXiv:2307.09522 [astro-ph.CO]].
%0 citations counted in INSPIRE as of 22 Jul 2023

\bibitem{Poulin:2018cxd}
V.~Poulin, T.~L.~Smith, T.~Karwal and M.~Kamionkowski,
``Early Dark Energy Can Resolve The Hubble Tension,''
Phys. Rev. Lett. \textbf{122} (2019) no.22, 221301
%doi:10.1103/PhysRevLett.122.221301
%[arXiv:1811.04083 [astro-ph.CO]].
%608 citations counted in INSPIRE as of 17 Jul 2023

\bibitem{Niedermann:2019olb}
F.~Niedermann and M.~S.~Sloth,
``New early dark energy,''
Phys. Rev. D \textbf{103} (2021) no.4, L041303
%doi:10.1103/PhysRevD.103.L041303
[arXiv:1910.10739 [astro-ph.CO]].
%140 citations counted in INSPIRE as of 24 Jul 2023

\bibitem{Jimenez:2001gg}
R.~Jimenez and A.~Loeb,
``Constraining cosmological parameters based on relative galaxy ages,''
Astrophys. J. \textbf{573} (2002), 37-42
%doi:10.1086/340549
%[arXiv:astro-ph/0106145 [astro-ph]].
%598 citations counted in INSPIRE as of 28 Jun 2023

\bibitem{Stern:2009ep}
D.~Stern, R.~Jimenez, L.~Verde, M.~Kamionkowski and S.~A.~Stanford,
``Cosmic Chronometers: Constraining the Equation of State of Dark Energy. I: H(z) Measurements,''
JCAP \textbf{02} (2010), 008
%doi:10.1088/1475-7516/2010/02/008
%[arXiv:0907.3149 [astro-ph.CO]].
%740 citations counted in INSPIRE as of 20 May 2022

\bibitem{Moresco:2012jh}
M.~Moresco, A.~Cimatti, R.~Jimenez, L.~Pozzetti, G.~Zamorani, M.~Bolzonella, J.~Dunlop, F.~Lamareille, M.~Mignoli and H.~Pearce, \textit{et al.}
``Improved constraints on the expansion rate of the Universe up to z\textasciitilde{}1.1 from the spectroscopic evolution of cosmic chronometers,''
JCAP \textbf{08} (2012), 006
%doi:10.1088/1475-7516/2012/08/006
%[arXiv:1201.3609 [astro-ph.CO]].
%508 citations counted in INSPIRE as of 20 May 2022

\bibitem{Zhang:2012mp}
C.~Zhang, H.~Zhang, S.~Yuan, T.~J.~Zhang and Y.~C.~Sun,
``Four new observational $H(z)$ data from luminous red galaxies in the Sloan Digital Sky Survey data release seven,''
Res. Astron. Astrophys. \textbf{14} (2014) no.10, 1221-1233
%doi:10.1088/1674-4527/14/10/002
%[arXiv:1207.4541 [astro-ph.CO]].
%425 citations counted in INSPIRE as of 20 May 2022

\bibitem{Moresco:2016mzx}
M.~Moresco, L.~Pozzetti, A.~Cimatti, R.~Jimenez, C.~Maraston, L.~Verde, D.~Thomas, A.~Citro, R.~Tojeiro and D.~Wilkinson,
``A 6\% measurement of the Hubble parameter at $z\sim0.45$: direct evidence of the epoch of cosmic re-acceleration,''
JCAP \textbf{05} (2016), 014
%doi:10.1088/1475-7516/2016/05/014
%[arXiv:1601.01701 [astro-ph.CO]].
%505 citations counted in INSPIRE as of 17 May 2022

\bibitem{Ratsimbazafy:2017vga}
A.~L.~Ratsimbazafy, S.~I.~Loubser, S.~M.~Crawford, C.~M.~Cress, B.~A.~Bassett, R.~C.~Nichol and P.~V\"ais\"anen,
``Age-dating Luminous Red Galaxies observed with the Southern African Large Telescope,''
Mon. Not. Roy. Astron. Soc. \textbf{467} (2017) no.3, 3239-3254
%doi:10.1093/mnras/stx301
%[arXiv:1702.00418 [astro-ph.CO]].
%162 citations counted in INSPIRE as of 17 May 2022

\bibitem{Borghi:2021rft}
N.~Borghi, M.~Moresco and A.~Cimatti,
``Toward a Better Understanding of Cosmic Chronometers: A New Measurement of H(z) at z \ensuremath{\sim} 0.7,''
Astrophys. J. Lett. \textbf{928} (2022) no.1, L4
%doi:10.3847/2041-8213/ac3fb2
%[arXiv:2110.04304 [astro-ph.CO]].
%10 citations counted in INSPIRE as of 17 May 2022

\bibitem{Jiao:2022aep}
K.~Jiao, N.~Borghi, M.~Moresco and T.~J.~Zhang,
``New Observational H(z) Data from Full-spectrum Fitting of Cosmic Chronometers in the LEGA-C Survey,''
Astrophys. J. Suppl. \textbf{265} (2023) no.2, 48
%doi:10.3847/1538-4365/acbc77
%[arXiv:2205.05701 [astro-ph.CO]].
%14 citations counted in INSPIRE as of 17 Jul 2023

\bibitem{Tomasetti:2023kek}
E.~Tomasetti, M.~Moresco, N.~Borghi, K.~Jiao, A.~Cimatti, L.~Pozzetti, A.~C.~Carnall, R.~J.~McLure and L.~Pentericci,
``A new measurement of the expansion history of the Universe at z=1.26 with cosmic chronometers in VANDELS,''
[arXiv:2305.16387 [astro-ph.CO]].
%1 citations counted in INSPIRE as of 28 Jun 2023

\bibitem{Moresco:2023zys}
M.~Moresco,
``Addressing the Hubble tension with cosmic chronometers,''
[arXiv:2307.09501 [astro-ph.CO]].
%0 citations counted in INSPIRE as of 24 Jul 2023

\bibitem{Moresco:2020fbm}
M.~Moresco, R.~Jimenez, L.~Verde, A.~Cimatti and L.~Pozzetti,
``Setting the Stage for Cosmic Chronometers. II. Impact of Stellar Population Synthesis Models Systematics and Full Covariance Matrix,''
Astrophys. J. \textbf{898} (2020) no.1, 82
%doi:10.3847/1538-4357/ab9eb0
[arXiv:2003.07362 [astro-ph.GA]].
%57 citations counted in INSPIRE as of 28 Jul 2023

\bibitem{Foreman-Mackey:2012any}
D.~Foreman-Mackey, D.~W.~Hogg, D.~Lang and J.~Goodman,
``emcee: The MCMC Hammer,''
Publ. Astron. Soc. Pac. \textbf{125} (2013), 306-312
%doi:10.1086/670067
%[arXiv:1202.3665 [astro-ph.IM]].
%3393 citations counted in INSPIRE as of 17 Jul 2023


\bibitem{Hou:2020rse}
J.~Hou, A.~G.~S\'anchez, A.~J.~Ross, A.~Smith, R.~Neveux, J.~Bautista, E.~Burtin, C.~Zhao, R.~Scoccimarro and K.~S.~Dawson, \textit{et al.}
``The Completed SDSS-IV extended Baryon Oscillation Spectroscopic Survey: BAO and RSD measurements from anisotropic clustering analysis of the Quasar Sample in configuration space between redshift 0.8 and 2.2,''
Mon. Not. Roy. Astron. Soc. \textbf{500} (2020) no.1, 1201-1221
%:10.1093/mnras/staa3234
%[arXiv:2007.08998 [astro-ph.CO]].
%135 citations counted in INSPIRE as of 28 Jun 2023

\bibitem{Neveux:2020voa}
R.~Neveux, E.~Burtin, A.~de Mattia, A.~Smith, A.~J.~Ross, J.~Hou, J.~Bautista, J.~Brinkmann, C.~H.~Chuang and K.~S.~Dawson, \textit{et al.}
``The completed SDSS-IV extended Baryon Oscillation Spectroscopic Survey: BAO and RSD measurements from the anisotropic power spectrum of the quasar sample between redshift 0.8 and 2.2,''
Mon. Not. Roy. Astron. Soc. \textbf{499} (2020) no.1, 210-229
%doi:10.1093/mnras/staa2780
%[arXiv:2007.08999 [astro-ph.CO]].
%133 citations counted in INSPIRE as of 28 Jun 2023

\bibitem{duMasdesBourboux:2020pck}
H.~du Mas des Bourboux, J.~Rich, A.~Font-Ribera, V.~de Sainte Agathe, J.~Farr, T.~Etourneau, J.~M.~Le Goff, A.~Cuceu, C.~Balland and J.~E.~Bautista, \textit{et al.}
``The Completed SDSS-IV Extended Baryon Oscillation Spectroscopic Survey: Baryon Acoustic Oscillations with Ly\ensuremath{\alpha} Forests,''
Astrophys. J. \textbf{901} (2020) no.2, 153
%doi:10.3847/1538-4357/abb085
%[arXiv:2007.08995 [astro-ph.CO]].
%172 citations counted in INSPIRE as of 28 Jun 2023

\bibitem{Trotta:2017wnx}
R.~Trotta,
``Bayesian Methods in Cosmology,''
[arXiv:1701.01467 [astro-ph.CO]].
%96 citations counted in INSPIRE as of 18 Jul 2023

\bibitem{Moresco:2022phi}
M.~Moresco, L.~Amati, L.~Amendola, S.~Birrer, J.~P.~Blakeslee, M.~Cantiello, A.~Cimatti, J.~Darling, M.~Della Valle and M.~Fishbach, \textit{et al.}
``Unveiling the Universe with emerging cosmological probes,''
Living Rev. Rel. \textbf{25} (2022) no.1, 6
%doi:10.1007/s41114-022-00040-z
%[arXiv:2201.07241 [astro-ph.CO]].
%71 citations counted in INSPIRE as of 16 Jun 2023

\bibitem{DESI:2023ytc}
G.~Adame \textit{et al.} [DESI],
``The Early Data Release of the Dark Energy Spectroscopic Instrument,''
%doi:10.5281/zenodo.7964161
[arXiv:2306.06308 [astro-ph.CO]].
%13 citations counted in INSPIRE as of 26 Jul 2023

\bibitem{Akarsu:2022lhx}
O.~Akarsu, E.~\'O~Colg\'ain, E.~\"Ozulker, S.~Thakur and L.~Yin,
``Inevitable manifestation of wiggles in the expansion of the late Universe,''
Phys. Rev. D \textbf{107} (2023) no.12, 123526
%doi:10.1103/PhysRevD.107.123526
%[arXiv:2207.10609 [astro-ph.CO]].
%6 citations counted in INSPIRE as of 17 Jul 2023

\bibitem{Zhao:2017cud}
G.~B.~Zhao, M.~Raveri, L.~Pogosian, Y.~Wang, R.~G.~Crittenden, W.~J.~Handley, W.~J.~Percival, F.~Beutler, J.~Brinkmann and C.~H.~Chuang, \textit{et al.}
``Dynamical dark energy in light of the latest observations,''
Nature Astron. \textbf{1} (2017) no.9, 627-632
%doi:10.1038/s41550-017-0216-z
%[arXiv:1701.08165 [astro-ph.CO]].
%356 citations counted in INSPIRE as of 17 Jul 2023

\bibitem{Wang:2018fng}
Y.~Wang, L.~Pogosian, G.~B.~Zhao and A.~Zucca,
``Evolution of dark energy reconstructed from the latest observations,''
Astrophys. J. Lett. \textbf{869} (2018), L8
%doi:10.3847/2041-8213/aaf238
%[arXiv:1807.03772 [astro-ph.CO]].
%92 citations counted in INSPIRE as of 17 Jul 2023

\bibitem{Escamilla:2021uoj}
L.~A.~Escamilla and J.~A.~Vazquez,
``Model selection applied to reconstructions of the Dark Energy,''
Eur. Phys. J. C \textbf{83} (2023) no.3, 251
%doi:10.1140/epjc/s10052-023-11404-2
%[arXiv:2111.10457 [astro-ph.CO]].
%13 citations counted in INSPIRE as of 17 Jul 2023

\end{thebibliography}
\end{document}






%
\title{Efficient evolutionary computing}
%
%\titlerunning{Abbreviated paper title}
% If the paper title is too long for the running head, you can set
% an abbreviated paper title here
%
\author{Goran Mau\v{s}a\inst{1,2}\orcidID{0000-0002-0643-4577}}
%
\authorrunning{G. Mau\v{s}a}
% First names are abbreviated in the running head.
% If there are more than two authors, 'et al.' is used.
%
\institute{University of Rijeka, Faculty of Engineering, Vukovarska 58, 51000 Rijeka, Croatia \and
University of Rijeka, Center for Artificial Intelligence and Cybersecurity, Radmile Matejčić 2, 51000 Rijeka, Croatia \\
\email{gmausa@riteh.hr}\\
%\url{http://www.springer.com/gp/computer-science/lncs} \and
%ABC Institute, Rupert-Karls-University Heidelberg, Heidelberg, Germany\\
%\email{\{abc,lncs\}@uni-heidelberg.de}
}
%
\maketitle              % typeset the header of the contribution
%
\begin{abstract}
%The abstract should briefly summarize the contents of the paper in 15--250 words.

In computer science, evolutionary computing is a family of nature-inspired algorithms for solving complex search-based and optimization problems.
The idea of evolutionary computing is to find the best solution to a given problem in a smaller number of steps than traditional and computationally demanding approaches like exhaustive or grid search.
This tutorial opens the question whether this strategy can be made even greener and analyzes this issue through hyper-parameter tuning, selecting the appropriate optimization algorithm and programming language, building surrogate models and neuroevolution.
The goal is to present theoretical background of evolutionary computing and genetic algorithm, provide a practical exercise as a showcase of its potential and to demonstrate its usage in a sustainable manner.


\keywords{Evolutionary Computing \and Genetic Algorithm \and Energy Consumption.}
\end{abstract}


%%%%%%%%%%%%%%%%%%%%%%%%%%%%%%%
\section{Introduction}

Evolutionary computation is a sub-field of artificial intelligence (AI) and it is used extensively in complex combinatorial problems and for continuous optimization.
In technical terms, it is a family of population-based trial and error problem solvers with a metaheuristic and stochastic character.
Although these algorithms can be used to solve modelling and simulation problems, their main application is optimization~\cite{eiben2015introduction}.

Evolutionary computation is used to solve problems that have too many variables for traditional algorithms and when optimal solution cannot be derived in (desirable) polynomial time \cite{borah2022applied}.
Although inherently efficient in solving complex problems, there are several efficiency enhancement techniques: parallelization, hybridization, time continuation, and evaluation relaxation~\cite{goldberg2002design}.
The task of parallelization is to distribute the computational load among processor units, hybridization combines local and global search techniques to reach a ballance between exploration and exploitation~\cite{sinha2005designing}, time continuation exploits the trade-off between the population size and the number of convergence epochs~\cite{srivastava2002time}, while evaluation relaxation promotes substitution of computationally expensive fitness with inexpensive approximation, also know as the surrogate model~\cite{smith1995fitness}.


In this tutorial, the aim is to provide the means of making evolutionary computing more sustainable, present a set of case studies designed to improve their energy efficiency and give a practical session for the second Summer School organized with the SusTrainable project.




%%%%%%%%%%%%%%%%%%%%%%%%%%%%%%%
\section{Evolutionary algorithms}

Darwinian's theory of evolution, governed by \textit{the survival of the fittest} principle, inspired automated problem solving family of algorithms that form evolutionary computing~\cite{fogel1998evolutionary}.
The most popular representative is the Genetic Algorithm (GA), named by Holland in the '70s~\cite{holland1973genetic}.
Individuals in a population, whose size is constrained by the limited amount of available resources, reproduce to prolong their species and the ones that are better adapted to environmental conditions have greater chances for survival.
In GA, individuals represent candidate solutions to a given problem, their genomes are sets of characteristics, i.e. parameters that define them and their quality is defined by a fitness function that plays the role of \textit{adaptation to the environment}~\cite{eiben2015introduction}.


The main requirements to solve an optimization problem using a GA is to have an adequate \textbf{representation} for candidate solutions and a \textbf{fitness function} that can numerically evaluate their quality.
The usual representation types, given in Figure \ref{fig1:evolutionary}, include bit string for binary variables, vector of real numbers for continuous variables, permutations for arranging the order of events and tree-based structures for creating programs, mathematical functions or similar more complex entities.
Each of the classical deductive optimization tasks like knapsack problem, finding minima of a complex mathematical function, travelling salesman and symbolic regression require one of these representation types.
In knapsack problem we are selecting the best subset of items to maximize their weight without exceeding its carrying capacity and the best representation is the bit string.
Finding extrema of a mathematical function such as the Rosenbrock function~\cite{rosenbrock1960automatic}, presented in Figure \ref{fig1:evolutionary}, requires the candidate solutions to be expressed through a vector of real numbers.
A travelling salesman needs to find the shortest path to visit a number of predetermined number of places of his route only once and permutations are the obvious choice for describing the candidate solutions.


% Figure environment removed


The schematic overview of GA workflow, as presented in Figure \ref{fig1:evolutionary}, stars from initialization of the population, usually by random generation of genomes.
The quality of each individual within the population is estimated numerically by the fitness function.
%The fitness function is then put to use to numerically evaluate the quality of every individual within the population.
Better solutions have greater chances for becoming parents of new individuals produced in the reproduction phase by the operator of genome crossover.
To introduce small random changes in the genes inherited from their parents, a mutation operator is applied in the offspring population.
Final phase of this iterative process is the quality-oriented selection of the individuals that will form the next generation of individuals.
Most of these functions are of stochastic nature, and their overall goal is to gradually drive the evolution in the direction of the best solutions~\cite{eiben2015introduction}.


Drawing inspiration from swarm intelligence, biology, physics, chemistry or even social phenomena and grammar, nowadays there is a great number of evolutionary algorithms~\cite{fister2013brief}.
Several exotic algorithms, like the music-inspired Harmony Search Algorithm, demonstrated the ability to enhance computational efficiency with simpler implementation and a lower number of setting parameters~\cite{geem2009music}.
Although often used as black box solution, understanding the underlying subtleties of evolutionary computing may lead to their cost-effective and more sustainable utilization.


%%%%%%%%%%%%%%%%%%%%%%%%%%%%%%%
\section{Training Course}

The tutorial will start by introducing the basic concepts of evolutionary computing, explain the main operators and functions within the GA and what is its purpose.
Although GA is known to be a more efficient alternative to computationally demanding approaches like exhaustive or grid search, the tutorial will discuss the strategies which may be implemented to make it even more sustainable.
Besides the obvious hyper-parameter tuning of numerous selection, crossover and mutation operators, we will analyze which is the appropriate choice of programming language for the implementation of GA and compare the popular languages like java, R and python.
We will also compare several exotic optimization algorithms like Firefly, Artificial Bee Colony and Flower Pollination Algorithm in terms of execution time and accuracy of the prediction model after performing feature selection.
To present more elaborate ways of making GA greener, we will introduce the concept of surrogate models, whose aim is to replace expensive fitness function with simpler approximate solutions.
The practical exercise will demonstrate the potential of evolutionary computing applied to evolving neural network (NN)-based classifiers, known as the neuroevolution.

The key take-away messages of this course will be:
\begin{enumerate}
    \item Understanding the fundamental operators of evolutionary computing;
    \item Know-how for estimating and improving the efficiency of GA;
    \item Practical guidelines for developing a sustainable neuroevolution framework.
\end{enumerate}



\subsection{Prerequisite Knowledge and Skills}

This lecture is designed to fit the expertise of a master or PhD student enrolled in the study programme of computer science or related studies.
The intended audience should have theoretical knowledge of algorithms and data structures to follow the training materials on evolutionary computing.
To fully benefit from the training materials related to surrogate modelling and neuroevolution, a basic understanding of machine learning and its application is recommended, along with the following concepts:
\begin{itemize}
    \item Understanding the complexity of neural networks;
    \item Familiarity with prediction model training pipeline;
    \item Evaluation metrics for estimating the classification performance.
\end{itemize}


\subsection{Materials and Methods}

The sustainability aspect of the practical exercise will require energy measurements for executing the optimization of NN architecture and the training of the prediction model.
Running Average Power Limit (RAPL) and its Python version (pyRAPL) offer an estimated energy consumption of CPUs and DRAM~\cite{RAPL2019} with minimal performance overhead~\cite{khan2018rapl} and we will be used it for that purpose.
Energy consumed by the grid search and by the neuroevolution will be compared to demonstrate the benefit of using the evolutionary computing paradigm for the task of evolving low-complexity prediction models.


The software packages and libraries that are going to be needed for the implementation and execution of examples in the practical session are:
\begin{itemize}
    \item tensorflow - model construction and training;
    \item pyRAPL - energy consumption measurement;
    \item scikit-learn - machine learning library containing grid search implementation and various benchmark datasets;
    \item matplotlib - visualizing the results;
    \item numpy - supporting mathematical operations;
    \item pandas - manipulating data;
    \item neat-python - implementation of neuroevolution algorithm in python;
    \item seqprops - encoding for peptides dataset
    \item scikit-bio - various tools for dealing with peptides datasets
    \item scikit-optimize - hyperparameter optimization
    \item scipy - optimization, statistics
    \item dask - parallelization library
\end{itemize}


Datasets available in scikit-learn library, such as California Housing dataset, Diabetes dataset or Iris plants dataset and/or publicly available peptide datasets like DRAMP 2.0~\cite{kang2019dramp}, will be used to demonstrate the principles of training NN-based prediction models.
The goal is to examine the benefit of using neuroevolution for simultaneous arcitecture optimization and training process in terms of time and energy consumption.



%%%%%%%%%%%%%%%%%%%%%%%%%%%%%%%
\section{Conclusion}

This paper presents a tutorial on basic concepts of evolutionary computing, with the aim to minimize energy consumption when solving optimization tasks.
It is a continuation of our lecture entitled Soft Computing for Sustainability Science, which dealt with the problem of data pre-processing and feature selection in particular, its application on dimensionality reduction and impact on reducing the time and energy when training predictive models.
This tutorial tackles the issue of developing NN-based models, which are emerging as the most popular choice for solving complex classification problems.
A lack of clear guidelines for choosing the appropriate NN architecture impedes their wider applicability and often leads to implementation of computationally demanding optimization procedures like grid-search.
Not only do these methods present an unsustainable means of solving the problem, but they also often result with architectures whose complexity is higher than actually needed, which increases computational cost for training and using them.

Participants of this tutorial will go through a theoretical lecture and a practical exercise, after which they will understand the importance of evolutionary algorithms and gain skills required to train NN-based prediction model in a sustainable manner.
Aligned with the aim of our SusTrainable project, the tutorial strives to provide scientific and technical support in identifying new development approaches, promote sustainability as an important topic in AI and to disseminate these concepts to a wider audience.




%%%%%%%%%%%%%%%%%%%%%%%%%%%%%%%




% \begin{table}
% \caption{Table captions should be placed above the
% tables.}\label{tab1}
% \begin{tabular}{|l|l|l|}
% \hline
% Heading level &  Example & Font size and style\\
% \hline
% Title (centered) &  {\Large\bfseries Lecture Notes} & 14 point, bold\\
% 1st-level heading &  {\large\bfseries 1 Introduction} & 12 point, bold\\
% 2nd-level heading & {\bfseries 2.1 Printing Area} & 10 point, bold\\
% 3rd-level heading & {\bfseries Run-in Heading in Bold.} Text follows & 10 point, bold\\
% 4th-level heading & {\itshape Lowest Level Heading.} Text follows & 10 point, italic\\
% \hline
% \end{tabular}
% \end{table}


%%%%%%%%%%%%%%%%%%%%%%%%%%%%%%%

\section*{Acknowledgement}

This paper acknowledges the support of the Erasmus+ Key Action 2 (Strategic partnership for higher education) project No. 2020-1-PT01-KA203-078646: “SusTrainable - Promoting Sustainability as a Fundamental Driver in Software Development Training and Education”.

The information and views set out in this paper are those of the author(s) and do not necessarily reflect the official opinion of the European Union. Neither the European Union institutions and bodies nor any person acting on their behalf may be held responsible for the use which may be made of the information contained therein.



%%%%%%%%%%%%%%%%%%%%%%%%%%%%%%%
% ---- Bibliography ----
%
% BibTeX users should specify bibliography style 'splncs04'.
% References will then be sorted and formatted in the correct style.
%\bibliographystyle{splncs04}
%\bibliography{References}


%\end{document}

%Version 2.1 April 2023
% See section 11 of the User Manual for version history
%
%%%%%%%%%%%%%%%%%%%%%%%%%%%%%%%%%%%%%%%%%%%%%%%%%%%%%%%%%%%%%%%%%%%%%%
%%                                                                 %%
%% Please do not use \input{...} to include other tex files.       %%
%% Submit your LaTeX manuscript as one .tex document.              %%
%%                                                                 %%
%% All additional figures and files should be attached             %%
%% separately and not embedded in the \TeX\ document itself.       %%
%%                                                                 %%
%%%%%%%%%%%%%%%%%%%%%%%%%%%%%%%%%%%%%%%%%%%%%%%%%%%%%%%%%%%%%%%%%%%%%

%%\documentclass[referee,sn-basic]{sn-jnl}% referee option is meant for double line spacing

%%=======================================================%%
%% to print line numbers in the margin use lineno option %%
%%=======================================================%%

%%\documentclass[lineno,sn-basic]{sn-jnl}% Basic Springer Nature Reference Style/Chemistry Reference Style

%%======================================================%%
%% to compile with pdflatex/xelatex use pdflatex option %%
%%======================================================%%

%%\documentclass[pdflatex,sn-basic]{sn-jnl}% Basic Springer Nature Reference Style/Chemistry Reference Style


%%Note: the following reference styles support Namedate and Numbered referencing. By default the style follows the most common style. To switch between the options you can add or remove �Numbered� in the optional parenthesis. 
%%The option is available for: sn-basic.bst, sn-vancouver.bst, sn-chicago.bst, sn-mathphys.bst. %  
 
\documentclass[sn-mathphys,Numbered,iicol]{sn-jnl}% Style for submissions to Nature Portfolio journals
%%\documentclass[sn-basic]{sn-jnl}% Basic Springer Nature Reference Style/Chemistry Reference Style
% \documentclass[sn-mathphys,Numbered]{sn-jnl}% Math and Physical Sciences Reference Style
%%\documentclass[sn-aps]{sn-jnl}% American Physical Society (APS) Reference Style
%%\documentclass[sn-vancouver,Numbered]{sn-jnl}% Vancouver Reference Style
%%\documentclass[sn-apa]{sn-jnl}% APA Reference Style 
%%\documentclass[sn-chicago]{sn-jnl}% Chicago-based Humanities Reference Style
%%\documentclass[default]{sn-jnl}% Default
%%\documentclass[default,iicol]{sn-jnl}% Default with double column layout

%%%% Standard Packages
%%<additional latex packages if required can be included here>

\usepackage{graphicx}%
\usepackage{multirow}%
\usepackage{amsmath,amssymb,amsfonts}%
\usepackage{amsthm}%
\usepackage{mathrsfs}%
\usepackage[title]{appendix}%
\usepackage{xcolor}%
\usepackage{textcomp}%
\usepackage{manyfoot}%
\usepackage{booktabs}%
\usepackage{algorithm}%
\usepackage{algorithmicx}%
\usepackage{algpseudocode}%
\usepackage{listings}%
\usepackage{subfigure}
\usepackage{bm}
\usepackage{multicol}

%%%%

%%%%%=============================================================================%%%%
%%%%  Remarks: This template is provided to aid authors with the preparation
%%%%  of original research articles intended for submission to journals published 
%%%%  by Springer Nature. The guidance has been prepared in partnership with 
%%%%  production teams to conform to Springer Nature technical requirements. 
%%%%  Editorial and presentation requirements differ among journal portfolios and 
%%%%  research disciplines. You may find sections in this template are irrelevant 
%%%%  to your work and are empowered to omit any such section if allowed by the 
%%%%  journal you intend to submit to. The submission guidelines and policies 
%%%%  of the journal take precedence. A detailed User Manual is available in the 
%%%%  template package for technical guidance.
%%%%%=============================================================================%%%%

%\jyear{2021}%

%% as per the requirement new theorem styles can be included as shown below
\theoremstyle{thmstyleone}%
\newtheorem{theorem}{Theorem}%  meant for continuous numbers
%%\newtheorem{theorem}{Theorem}[section]% meant for sectionwise numbers
%% optional argument [theorem] produces theorem numbering sequence instead of independent numbers for Proposition
\newtheorem{proposition}[theorem]{Proposition}% 
%%\newtheorem{proposition}{Proposition}% to get separate numbers for theorem and proposition etc.

\theoremstyle{thmstyletwo}%
\newtheorem{example}{Example}%
\newtheorem{remark}{Remark}%

\theoremstyle{thmstylethree}%
\newtheorem{definition}{Definition}%

\raggedbottom
%%\unnumbered% uncomment this for unnumbered level heads

\begin{document}

\title[Article Title]{Non-Markovian Quantum Gate Set Tomography}

%%=============================================================%%
%% Prefix	-> \pfx{Dr}
%% GivenName	-> \fnm{Joergen W.}
%% Particle	-> \spfx{van der} -> surname prefix
%% FamilyName	-> \sur{Ploeg}
%% Suffix	-> \sfx{IV}
%% NatureName	-> \tanm{Poet Laureate} -> Title after name
%% Degrees	-> \dgr{MSc, PhD}
%% \author*[1,2]{\pfx{Dr} \fnm{Joergen W.} \spfx{van der} \sur{Ploeg} \sfx{IV} \tanm{Poet Laureate} 
%%                 \dgr{MSc, PhD}}\email{iauthor@gmail.com}
%%=============================================================%%

\author[1,3,4]{\fnm{Ze-Tong} \sur{Li}}

\author[1,3,4]{\fnm{Cong-Cong} \sur{Zheng}}

\author[5]{\fnm{Fan-Xu} \sur{Meng}}

\author[2,3,4,6]{\fnm{Zai-Chen} \sur{Zhang}}

\author*[1,3,4,6]{\fnm{Xu-Tao} \sur{Yu}}\email{yuxutao@seu.edu.cn}

% \equalcont{These authors contributed equally to this work.}
\affil[1]{\orgdiv{State Key Laboratory of Millimeter Waves}, \orgname{Southeast University}, \orgaddress{\city{Nanjing}, \postcode{210096}, \country{China}}}

\affil[2]{\orgdiv{National Mobile Communications Research Laboratory}, \orgname{Southeast University}, \orgaddress{\city{Nanjing}, \postcode{210096}, \country{China}}}

\affil[3]{\orgdiv{Frontiers Science Center for Mobile Information Communication and Security}, \orgname{Southeast University}, \orgaddress{\city{Nanjing}, \postcode{210096}, \country{China}}}

\affil[4]{\orgdiv{Quantum Information Center}, \orgname{Southeast University}, \orgaddress{\city{Nanjing}, \postcode{210096}, \country{China}}}

\affil[5]{\orgdiv{College of Artificial Intelligence}, \orgname{Nanjing Tech University,}, \orgaddress{\city{Nanjing}, \postcode{211800}, \country{China}}}

\affil[6]{\orgname{Purple Mountain Lab}, \orgaddress{\city{Nanjing}, \postcode{211111}, \country{China}}}


%%==================================%%
%% sample for unstructured abstract %%
%%==================================%%

\abstract{Engineering quantum devices requires reliable characterization of the quantum system including qubits, quantum operations (aka instruments) and the quantum noise. Recently, quantum gate set tomography (GST) has emerged as a promissing technique to self-consistently describe the quantum states, gates and measurements. However, non-Markovian correlations between the quantum system and environment cause the reliability regression of GST. It is essential to simultaneously describe the gate set and non-Markovian correlations. To this end, we first propose a self-consistent operational method, named instrument set tomography (IST), for non-Markovian GST. Based on the stochastic quantum process, the instrument set is defined to describe instruments, the initial state, and non-Markovian system-environment (SE) correlations. First, we propose a linear inversion IST (LIST) to detect and describe the disharmony of linear relationship of instruments and SE correlations with gauge freedom. 
However, LIST cannot always determine physical implementable instrument set because of the absence of constraints. Then, a physically constrained statistical method based on the miximum likelihood estimation for IST (MLE-IST) is proposed with polynomial number of parameters with respect to the Markovian order. It shows significant flexibility that suit for different types of device, e.g. noisy intermediate-scale quantum (NISQ) devices, by adjusting the model and constraints. The experimental results show the effectiveness of describing instruments and the non-Markovian quantum system. As a result, the IST provides an essential method for benchmarking and developing quantum devices in the aspect of instrument set.}

%%================================%%
%% Sample for structured abstract %%
%%================================%%

% \abstract{\textbf{Purpose:} The abstract serves both as a general introduction to the topic and as a brief, non-technical summary of the main results and their implications. The abstract must not include subheadings (unless expressly permitted in the journal's Instructions to Authors), equations or citations. As a guide the abstract should not exceed 200 words. Most journals do not set a hard limit however authors are advised to check the author instructions for the journal they are submitting to.
% 
% \textbf{Methods:} The abstract serves both as a general introduction to the topic and as a brief, non-technical summary of the main results and their implications. The abstract must not include subheadings (unless expressly permitted in the journal's Instructions to Authors), equations or citations. As a guide the abstract should not exceed 200 words. Most journals do not set a hard limit however authors are advised to check the author instructions for the journal they are submitting to.
% 
% \textbf{Results:} The abstract serves both as a general introduction to the topic and as a brief, non-technical summary of the main results and their implications. The abstract must not include subheadings (unless expressly permitted in the journal's Instructions to Authors), equations or citations. As a guide the abstract should not exceed 200 words. Most journals do not set a hard limit however authors are advised to check the author instructions for the journal they are submitting to.
% 
% \textbf{Conclusion:} The abstract serves both as a general introduction to the topic and as a brief, non-technical summary of the main results and their implications. The abstract must not include subheadings (unless expressly permitted in the journal's Instructions to Authors), equations or citations. As a guide the abstract should not exceed 200 words. Most journals do not set a hard limit however authors are advised to check the author instructions for the journal they are submitting to.}

\keywords{non-Markovian correlation, gate set tomography, quantum tomography}

%%\pacs[JEL Classification]{D8, H51}

%%\pacs[MSC Classification]{35A01, 65L10, 65L12, 65L20, 65L70}

\maketitle

\section{Introduction}
Quantum computing requires engineering reliable and controllable quantum devices that manipulate the quantum states with high fidelity. However, recent quantum devices suffer the non-ignorable quantum noise introduced by the imperfect implementations of quantum gates and the system-environment (SE) correlations \cite{papic2023Error}. Characterization of qubits, operations, and entire processors to analyse the influence of quantum noise plays a significant role in the quantum characterization, verification, and validation (QCVV) and offers basic information for the device manufacturing and calibration. 

Based on different assumptions, many protocols have been proposed for this task under the common skeleton of quantum tomography \cite{banaszek2013Focus,smolin2012Efficient,blume-kohout2010Optimal,koutny2022Neuralnetwork,riebe2006Process,mohseni2008Quantumprocess,surawy-stepney2022Projected,greenbaum2015Introduction,nielsen2021Gate}: (1) prepare a set of experiments described by quantum states, circuits and measurements; (2) gather data by executing the prepared experiments; (3) yeild the target result of quantum states, processes and/or measurements by performing estimation algorithms. Among these tomographic methods, gate set tomography (GST) \cite{greenbaum2015Introduction,nielsen2021Gate} is the most powerful and comprehensive method to operationally and self-consistently characterize quantum gates, 
state preparations and measurements (SPAM) without assuming any component of the experiments to be known previously, while the quantum state tomography (QST) \cite{banaszek2013Focus,smolin2012Efficient,blume-kohout2010Optimal,koutny2022Neuralnetwork} and quantum process tomography (QPT) \cite{riebe2006Process,mohseni2008Quantumprocess,surawy-stepney2022Projected} generally require the full knowledge of not-target parts in the experiments. The GST successfully describes two-time noisy quantum gates by completely positive trace-preserving (CPTP) maps under the Markovian assumption. However, no system is isolated \cite{pollock2018NonMarkovian}. There is sufficient evidence that the non-Markovian multiple time correlation nonnegligibly impacts current generation quantum devices \cite{blume-kohout2017Demonstration,proctor2022Measuring,white2020Demonstration,sarovar2020Detecting}. It not only disturbs the tomography under Markovian model that operations in the past influence the behavior of current operation and result in the theoretical violation of CPTP constraints \cite{proctor2022Measuring,milz2021Quantum}. Moreover, the effectiveness of quantum error-correcting codes can degrade or vanish with the appearance of the non-Markovian correlation \cite{nickerson2019Analysing,clader2021Impact}. Therefore, Markovian two-time CPTP maps are not sufficient to describe entire dynamics of the quantum device. Correlations across multiple time scales should be considered while characterizing the device. 

Based on the quantum stochastic process \cite{milz2021Quantum} representing the multiple time correlation, the non-Markovian system dynamics can be modeled by the system, environment, instruments act on the system, and unitaries act on the system and environment simultaneously. For an experimenter, the only accessible part is the instruments representing interventions on the system including quantum gates and measurements. Hence, an instrument can be represented by completely positive trace-non-increasing (CPTNI) maps. Aiming at operationally describing the time-dependent SE correlations, process tensor tomography (PTT) \cite{pollock2018NonMarkovian,guo2022Reconstructing,milz2018Reconstructing,white2022NonMarkovian} relaxes the Markovian constraint to perform the non-Markovian quantum process tomography. It construct well defined CP process tensor with unit trace by interventions of known instruments. However, the differences between the knowledge and the practical performance of instruments may disturb the reconstruction of the process tensor \cite{white2022NonMarkovian}. A simple example is that PTT may generate inconsistent two process tensors using two set of faulty state-informationally complete instruments (that are sufficient to span the space of quantum state). Consequently, the characterization of real quantum devices requires a self-consistent method to tomographically describe the non-Markovian SE correlation and faulty instruments, which directly motivate this work.

To tackle these issues, we first propose a self-consistent method to perform GST under non-Markovian situation. We call the method instrument set tomography (IST). We first propose the linear inversion IST (LIST) a simple, closed-form algorithm to estimate the instruments as well as the SE correlations represented by the process tensor. Unsurprisingly, the IST still exhibits the gauge freedom as GST. Hence the gauge optimization is required at the end of LIST. Although the estimated result may not satisfy the physical constraints since we introduce no constraint in the gauge optimization, it is consistent to the probability measurement data.
Then, we propose a statistical IST method based on the maximum likelihood estimation (MLE) trying to extract more information from overcomplete measurement data. The MLE-IST models the instruments and SE correlations via a flexible way that can suite for different assumption. By introduce constraints, the results are guaranteed to be physical. It also enables the explicit estimation of unitaries representing the non-Markovian SE correlation and evolution instead of the process tensor. Particularly, we also demonstrate how to implement IST on the current noisy quantum intermediate-scale quantum (NISQ) devices. The experimental results show the effectiveness of characterization of instruments, initial states, and non-Markovian correlations. As a result, the IST provides an essential, self-consistent, and reliable method for benchmarking and developing a quantum device under non-Markovian situation in the aspect of instrument set.

\section{Result}

\subsection{Quantum Stochastic Process and Instrument Set}
Before moveing on to present the IST, we first recall the quantum stochastic process representing the non-Markovian quantum correlation and give definitions for instrument set. For a $d$-dimensional quantum system with non-Markovian correlations, the experimenter intervenes the quantum system at $k$ time steps by CPTNI instruments from 
\begin{align}
  \mathcal{J}^{(t)} := \left\{\mathcal{A}^{(t)}_{0}, \mathcal{A}^{(t)}_{1},\dots, \mathcal{A}^{(t)}_{m_t-1}\right\},
\end{align}
where $t=1,\dots,k$ and $m_t$ is the number of valid instruments at time step $t$. Each intervention of the instrument output a value and transform the quantum state for the next time step. The available instruments at different time steps may be different. Then, the operational open quantum process can be described by a $d$-dimensional system and a $d$-dimensional environment with interventions of instruments on the system at $k$ time steps and SE unitaries evolutions between time steps as depicted in Fig.~\ref{fig:qsp} \cite{pollock2018NonMarkovian}. Note that there is a boundary between the accessible and inaccessible parts of the open quantum dynamics to an experimenter. Specifically, an experimenter can not access the quantum state directly. All information of the quantum state the experimenter obtained should with the help of output values of interventions of instruments. The probability to get a sequence of output values $\bm{x}$ is
\begin{equation}\label{eq:se_evo_prob_tr}
  p_{\bm{x}}=\mathrm{Tr}\left[\mathcal{A}^{(k)}_{x_{k-1}}\bigcirc_{t=0}^{k-2}\left(\mathcal{U}_{t:t+1}\mathcal{A}^{(t)}_{x_t}\right)\left(\rho_{SE}^{(0)}\right)\right],
\end{equation}
where $x_t$ is the output value at time step $t$. Without loss of generality, we refer the output value $x_t$ to be the indexes of intruments instead of the actual output value in the following text. Besides, we use the note $\mathcal{A}^{(t)}_{x_t}$ instead of $\mathcal{A}^{(t)}_{x_t}\otimes \mathcal{I}$ for simplicity without confusing. 

From Eq.~\eqref{eq:se_evo_prob_tr}, it is quite clear that the probability can be determined when the instruments, the SE unitary dynamics, and the initial state are given. Therefore, the instrument set describing the operational open quantum dynamics of the quantum device can be defined as
\begin{align}\label{eq:instrument_set_full}
  \mathfrak{I}_{\mathrm{full}} := \left\{\mathcal{J}, \mathcal{U},\rho^{(0)}_{SE}\right\},
\end{align}
where $ \mathcal{J}:=\left\{\mathcal{J}^{(t)}\right\}_{t=0}^{k-1}$ and $\mathcal{U}:=\left\{\mathcal{U}_{t:t+1}\right\}_{t=0}^{k-2}$.
This full definition explicitly depends on the inaccessible initial state and the SE unitary evolution between time steps in which the experimenter may be insterested. However, explicitly characterization of the initial state and the SE unitarie is difficult.

Benifit from the process tensor $\mathcal{T}$ representing the inaccessible parts \cite{pollock2018NonMarkovian,milz2021Quantum,white2022NonMarkovian}, the probability to get $\bm{x}$ can be described as 
\begin{align}\label{eq:pt_def}
  p_{\bm{x}} = \mathcal{T}\left(\mathcal{A}^{(0)}_{x_0},\dots,\mathcal{A}^{(k-1)}_{x_{k-1}}\right),
\end{align}
implying the sufficiency to determine the measurement probability by given $\mathcal{T}$ and $\mathcal{J}$. Therefore, the reduced instrument set can be defined as 
\begin{align}\label{eq:instrument_set_reduced}
  \mathfrak{I}_{\mathrm{reduced}}:=\left\{\mathcal{J}, \mathcal{T}\right\}.
\end{align}

These two definitions of instrument set will be used to propose the IST with clear declaration. In the following, we always use the pauli transfer matrix (PTM) representation to describe instruments, quantum states and process tensors. Particularly, notations $A$ and $\vert \rho\rangle\!\rangle$ are used to indicate the instrument $\mathcal{A}$ and the quantum state $\rho$. Moreover, the PTM representation of the process tensor $\Upsilon_{\mathcal{T}}$ is defined as
\begin{align}\label{eq:se_evo_prob_pt}
  p_{\bm{x}} = \mathrm{Tr}\left[\Upsilon_{\mathcal{T}}^\dagger\left(\begin{matrix}A^{(0)}_{x_0}\\\vdots\\A^{(k-1)}_{x_{k-1}}\end{matrix}\right)\right],
\end{align}
where terms in parentheses are defined as
\begin{align}
  \left(\begin{matrix}X_1\\ \vdots \\X_n\end{matrix}\right) = \left(\begin{matrix}X_1, \dots, X_n\end{matrix}\right):=X_1\otimes\dots\otimes X_n
\end{align}
for clearness and simplicity, instead of directly applying the Choi-Jamiołkowski isomorphism (CJI) representation for the notation consistency. It is easy to verify that the PTM and CJI representation of process tensor is equivalent.

% $\chi^{(t)}_{x_t}$ and $\Upsilon_{\mathcal{T}}$ are the Choi-Jamiołkowski isomorphism (CJI) of $\mathcal{A}^{(t)}_{x_t}$ and $\mathcal{T}$ respectively. Therefore, the reduced instrument set can be defined as 
% \begin{align}\label{eq:instrument_set_reduced}
%   \mathfrak{I}_{\mathrm{reduced}}=\left\{\mathcal{J}, \mathcal{T}\right\}.
% \end{align}
% The reduced instrument set is sufficient to describe the non-Markovian quantum dynamics with the interventions of instruments as the full instrument set, since the process tensor are sufficient to describe the initial state and SE unitary evolution that 
% \begin{align}\label{eq:pt_u_convert}
%   \Upsilon_{\mathcal{T}} = \mathrm{Tr}_E \left[\mu_{k-2:k-1}\star \dots \star \mu_{0:1} \star \rho_{SE}^{(0)}\right],
% \end{align}
% where $\mu_{t:t+1}$ is the CJI of $\mathcal{U}_{t:t+1}$ and $\star$ is the link product defined in \cite{chiribella2009Theoretical}. These two definitions of instrument set will be used to propose the IST with clear declaration. 

% Figure environment removed

\subsection{Linear Inversion IST}\label{sec:LIST}
We first propose the linear inversion IST (LIST) based on the reduced instrument set as defined in Eq.~\eqref{eq:instrument_set_reduced}. Focusing on the time step $t$, the measurement probability can be described as
\begin{equation}\label{eq:inst_decomp_tr_basis}
  p^{(t)}_{\alpha,{x_t}} = \mathrm{Tr}\left[B_\alpha^{(t)} A^{(t)}_{x_t}\right],
\end{equation}
where $A^{(t)}_{x_t}$ is the PTM a $d^2\times d^2$ matrix that completely represent the instrument $\mathcal{A}^{(t)}_{x_t}$, $B_\alpha^{(t)}$ is a $d^2\times d^2$ real basis matrix indexed by $\alpha$. Let $\bm{x}^+$ and $\bm{x}^-$ denote the output values before and after time step $t$ in a $k$-time step non-Markovian experiment, respectively. The LIST constructs a bijection $\alpha = f(\bm{x}^+, \bm{x}^-)$ between integer $\alpha$ and the concatenation of vectors $(\bm{x}^+,\bm{x}^-)$ by the adjustment of $\bm{x}^+$ and $\bm{x}^-$ such that $\mathbb{B}^{(t)} = \{B_0^{(t)}, B_1^{(t)},\dots, B_{d^4-1}^{(t)}\}$ is a linear independent basis set. See Method for detail. 

This implies the decomposition of $A^{(t)}_{x_t}$ on the non-orthogonal process-informationally complete basis $\mathbb{B}^{(t)}$, 
\begin{equation}\label{eq:inst_decomp_vec}
  \bm{p}_{x_t}^{(t)} = \begin{bmatrix}
    (\bm{b}_0^{(t)})^\dagger\\
    (\bm{b}_1^{(t)})^\dagger\\
    \vdots\\
    (\bm{b}_{d^4-1}^{(t)})^\dagger\\ 
  \end{bmatrix}\bm{a}^{(t)}_{x_t} = B^{(t)}\bm{a}^{(t)}_{x_t},
\end{equation}
where $\bm{a}^{(t)}_{x_t}$ and $\bm{b}_\alpha^{(t)}$ represent the vectorization of the $A^{(t)}_{x_t}$ and $B_\alpha^{(t)}$, respectively. Note that instruments at time step $t$ share the same $B^{(t)}$. If $B^{(t)}$ is invertible, we can get instruments
\begin{gather}\label{eq:list_recover}
    \Xi^{(t)} =\left(B^{(t)}\right)^{-1} \Gamma^{(t)},
\end{gather}
where $\Xi^{(t)} = [\bm{a}^{(t)}_{0},\bm{a}^{(t)}_{1}, \dots, \bm{a}^{(t)}_{m_t-1}]$ and $\Gamma^{(t)} = [\bm{p}_{0}^{(t)},\bm{p}_{1}^{(t)},\dots,\bm{p}_{m_t-1}^{(t)}]$. PTMs of instruments can be recovered by devectorization of determined $\bm{a}^{(t)}_{x_t}$. 

The instruments are reconstructed by repeating this for each time step. Then, we choose the maximum linear independent set of the instruments at each time step to formulate the process tensor
\begin{align}
  \Upsilon_\mathcal{T} =\sum_{\bm{x}}p_{\bm{x}}\left(\begin{matrix}D^{(0)}_{x_0}\\\vdots\\D^{(k-1)}_{x_{k-1}}\end{matrix}\right),
\end{align}
where $\left\{D^{(t)}_{x_{t}}\right\}$ is the dual set of maximum linear independent set $\left\{A^{(t)}_{x_t}\right\}$ such that $\mathrm{Tr}\left[\left(D^{(t)}_{i}\right)^\dagger A^{(t)}_{j}\right] = \delta_{ij}$. 

The tomography of the instrument set shows gauge freedom up to a set of invertible matrices $\{B^{(t)}\}$ because of the inaccessible initial state and SE unitaries. We can not distinguish the quantum operations up to $\{B^{(t)}\}$ by the probability measurement, because we can obtain a set of instruments and process tensor without violations of measurement probabilities $p_{\bm{x}}$ for each given set of gauge matrices $\{B^{(t)}\}$. See Method for detail.

The gauge optimization is required to provide a reasonable gauge matrices set to determine the tomographic result of instrument set. We assume that the quantum instruments are implemented well that are close to the ideal instruments. Then, the gauge matrix can be optimized by
\begin{equation}\label{eq:gauge_opt_obj_fn}
  B^{(t)} = \arg\min_X \sum_{t} \left\|X\Gamma^{(t)} - \Xi^{(t)}_{\mathrm{knowledge}}\right\|_F,
\end{equation}
where $\Xi^{(t)}_{\mathrm{knowledge}}$ is the knowledge of instruments to the experimenter. Consequently, the tomographic result is
\begin{gather}
  \hat{\mathfrak{I}}=\left\{\hat{\mathcal{J}}, \hat{\mathcal{T}}\right\},\label{eq:result_list}\\
  \hat{\mathcal{J}}=\left\{\left\{A^{(t)}_{0},\dots,A^{(t)}_{m_t-1}\right\}\right\}_{t=0}^{k-1},\\
  \hat{\mathcal{T}}=\Upsilon_\mathcal{T}.
\end{gather}

A few points are worth mentioning. First, it can be seen from Eq.\eqref{eq:list_recover} and Eq.\eqref{eq:gauge_opt_obj_fn} that the tomographic result of instruments is always the prior knowledge at the time step the instruments are linear independent and not overcomplete. In this case, the LIST degrades to the linear inversion PTT and all imperfect implementation of instruments are represented by the process tensor. The LIST shows the power detecting the disharmony of linear relationship when the instruments at a time step are not linear independent. Moreover, the linear inverse method actually determines a self-consistent tomographic result of instruments and the process tensor, but they may not physically implementable. These characteristics result from the absence of constraints in the gauge optimization. We actually can constrain each $B_\alpha^{(t)}$ and/or $A_{x_t}^{(t)}$ to be CPTNI\footnote{The constraints to the instruments are corresponding to the completely positive trace non-increasing assumption of instruments. This can be adjusted along with the instruments' assumptions (CPTP at intermediate time steps on NISQ devices, for example).}. Nevertheless, this will increase computational complexity. Instead, we optimize $B^{(t)}$ over the entire group of real, invertible matrices to strenuously fit the data. This is similar to the linear inverse GST (LGST) \cite{greenbaum2015Introduction} under Markovian situation.

Second, the objective function in Eq.\eqref{eq:gauge_opt_obj_fn} is not convex and may has nonunique global minima especially the instruments at time step $t$ are not process-informationally complete. This indeterminacy is generic in quantum tomography, and appears in QST, QPT and GST as well. Therefore, we adopt the reduced definition of instrument set using the process tensor to avoid explicit introducing of the SPAM gauge freedom at each time step. See Method for detail. However, this gauge freedom objectively exists that we cannot determine the initial state and actual SE unitaries by the LIST but a set of consistent ones. In this case, fixing the gauge provides no additional information about the initial state and S-E unitaries. Therefore, the LIST also derives the consequence that a initialization error can not be distinguished from a faulty measurement (as described in GST) at each time step $t$ \cite{greenbaum2015Introduction}. Moreover, the non-Markovian SE correlation before the intervention of the instrument can not be distinguished from the non-Markovian SE correlation after.

Third, the LIST at each time step requires the a process-informationally complete basis by combining the instrument at other time steps rather than that the system state and the measurement simultaneously and respectively form state-informationally complete basis. This is because the environment carries the information by non-Markovian SE evolutions. See Method for detail. However, it is difficult to confirm that the entanglements before and after a time step are enough to carry the information such that the specified set of time steps has the ability to construct process-informationally complete basis for tomography at the time step, because the SE dynamics are inaccessible for experimenters. In other word, the non-Markovian effect may not so severe that has high possibility to satisfy the condition of composing process-informationally complete basis. Therefore, we still recommend constructing state-informationally complete basis before and after the time step, respectively. Note that this challenge becomes intractable when conducting tomography at time steps closed to the time edge, especially $0$ and $k$, that the former or the later instruments can not form a state-informationally complete basis. The proposed LIST do not tackle this problem. However, the tomographic result are still compatible with the measurement probabilities.

\subsection{Maximum Likelihood Estimation based IST}

LIST provide a quick method to estimate the instruments and the non-Markovian quantum system. However, it may not always give a physical result and is incompatible of working with overcomplete data for constructing basis of decomposition, which could be used to improve the estimate. Moreover, experimenters may interested in more characteristics, for example, the S-E evolutions themself, requiring high flexibility of the model. To tackle these issue, we propose a statistical framework for IST via maximum likelihood estimation (MLE-IST). As a result, the likelihood function of instrument set is derived as
\begin{align}
  l(\hat{\mathfrak{I}})=\sum_{\bm{x}}{\left(\tilde{p}_{\bm{x}}-\hat{p}_{\bm{x}}\right)^2}/{\sigma_{\bm{x}}^2},
\end{align}
where $\tilde{p}_{\bm{x}}$ denote the measurement probability of getting $\bm{x}$ obtained by the experiment, $\sigma_{\bm{x}}^2$ is the sampling variance of $\tilde{p}_{\bm{x}}$, and $\hat{p}_{\bm{x}}$ is the estimator of measurement probability which is modeled by parameters. The MLE-IST exhibit high flexibility estimating the instrument set with physical constraints based on the various assumptions, such as CPTNI for generality or CPTP on NISQ.

Based on the full definition of instrument set in Eq.\eqref{eq:instrument_set_full}, each instrument $\mathcal{A}_{x_t}^{(t)}$ is modeled by a real matrix $\hat{R}_{x_t}^{(t)}\in[-1,1]^{d^2\times d^2}$ as the PTM representation with the CPTNI constraints. More specifically, the CP requires the Choi state of $\hat{R}_{x_t}^{(t)}$ to be positive semidefinite as 
\begin{align}
  \hat{\rho}_{x_t} = \frac{1}{d^2}\sum_{i,j=0}^{d^2-1}[\hat{R}_{x_t}^{(t)}]_{i,j} \left(\begin{matrix}P_j^T\\ P_i\end{matrix}\right) \succcurlyeq 0,
\end{align}
where $P_i$ represents the $i$-th Pauli matrix. The TNI requires the first entry to be $0\le[\hat{R}_{x_t}^{(t)}]_{0,0} \le 1$. The initial S-E state $\vert\hat{\rho}^{(0)}_{SE}\rangle\!\rangle\in[-1,1]^{d^2\times 1}$ is modeled as a real vector with CP and unit-trace constraints. In other word, the corresponding density matrix is positive semidefinite, and the first entry of $\vert\hat{\rho}^{(0)}_{SE}\rangle\!\rangle$ is $1/\sqrt{d}$. Without loss of generality, we assume that the S-E evolutions do not include the operation on the system dimension, which means any evolution on the system only are absorbed into the instruments. Hence, we can use $\bm{\alpha}^{(t:t+1)} \in [-\pi,\pi]^{d^4(d^4-1)}$ to model each $U_{t}$ corresponding to rotation angles of Pauli operators in $\mathbb{P} = \{P^{S}_i \otimes P^{E}_j|i=1,2,\dots,d^4-1,~j=0,1,\dots,d^4-1\}$, where $P_0 = I$ is the identity matrix. Letting $\bm{\sigma}$ represent the vector of Pauli operators, the recovered unitary $\hat{V}_{t:t+1}(\bm{\alpha})$ is defined as the PTM of unitary $\exp\left(\iota (\bm{\alpha}^{(t:t+1)})^T\bm{\sigma}\right)$, where $\iota^2 = -1$. We use the notation $\hat{V}_{{t:t+1}}$ to indicate $\hat{V}_{{t:t+1}}(\bm{\alpha}^{(t:t+1)})$ in the following for simplicity. Hence, the estimator of the probability is given by
\begin{align}
  \hat{p}_{\bm{x}}=\langle\!\langle 0_{SE}\rvert\!\left(\!\begin{matrix}\hat{R}^{(k-1)}_{x_{k-1}}\\I\end{matrix}\!\right) \!\!\prod_{t=0}^{k-2} \!{\hat{V}_{t:t+1}\left(\!\begin{matrix}\hat{R}^{(t)}_{x_t}\\ I\end{matrix}\right)} \!\lvert\hat{\rho}^{(0)}_{SE}\rangle\!\rangle.
\end{align}

Then, the optimization problem describing MLE-IST based on the full definition of instrument set is given by
\begin{align}\label{eq:mle_ist_opt_prob_full}
  \min&_{\substack{\lvert \hat{\rho}^{(0)}_{SE}\rangle\!\rangle,\hat{R}^{(t)}_{x_t}, \bm{\alpha}^{(t:t+1)},
  \forall x_t, t}}~l(\hat{\mathfrak{I}}),\\
  s.t.~& \hat{\rho}_{x_t} = \frac{1}{d^2}\sum_{i,j=0}^{d^2-1}[\hat{R}_{x_t}^{(t)}]_{i,j}\left(\begin{matrix}P_j^T\\ P_i\end{matrix}\right) \succcurlyeq 0, \forall x_t,\tag{C1} \label{eq:full_def_cp_costraint}\\
  &0\le[\hat{R}_{x_t}^{(t)}]_{1,1} \le 1, \forall x_t,\tag{C2} \label{eq:full_def_tni_costraint}\\
  &-1\le[\hat{R}_{x_t}^{(t)}]_{i,j} \le 1, \forall x_t,i,j,\tag{C3} \label{eq:full_def_ptm_val_constraint}\\
  &\hat{\rho}_{SE}^{(0)} = \frac{1}{\sqrt{d}} \sum_{i=0}^{d^4-1} \langle\!\langle i \vert\hat{\rho}^{(0)}_{SE}\rangle\!\rangle P_i \succcurlyeq 0, \tag{C4} \label{eq:full_def_init_state_cp_constraint}\\
  &[\vert\hat{\rho}^{(0)}_{SE}\rangle\!\rangle]_0 = 1/\sqrt{d}, \tag{C5} \label{eq:full_def_init_state_tr1_constraint}\\
  &-\pi \le [\bm{\alpha}^{(t)}]_i \le \pi, \tag{C6} \label{eq:full_def_upval_constraint}
\end{align}
where \eqref{eq:full_def_cp_costraint} and \eqref{eq:full_def_tni_costraint} constraint the instruments to be CP and TNI, respectively, \eqref{eq:full_def_ptm_val_constraint} defines the range of PTM entries, \eqref{eq:full_def_init_state_cp_constraint} and \eqref{eq:full_def_init_state_tr1_constraint} restrict the initial state to be CP and with unit trace, respectively, and \eqref{eq:full_def_upval_constraint} limits the range of paramters of SE unitaries. Consequently, the MLE-IST estimate the insturment set as 
\begin{gather}\label{eq:result_mleist_full}
  \hat{\mathfrak{I}}:= \left\{\hat{\mathcal{J}}, \hat{\mathcal{U}},\vert\hat{\rho}^{(0)}_{SE}\rangle\!\rangle\}\right\}\\
  \hat{\mathcal{J}}= \left\{\left\{\hat{R}^{(t)}_{0},\dots, \hat{R}^{(t)}_{m_t-1}\right\}\right\}_{t=0}^{k-1},\\
  \hat{U} = \left\{\hat{V}_{t:t+1}\right\}_{t=0}^{k-2}
\end{gather}
with $\sum_{t=0}^{k-1}m_t d^4 + (k-1)d^4(d^4-1) + d^4-1$ parameters which is linear with respect to the non-Markovian order $k$.

The model described above makes an isolation of the instruments and SE unitary dynamics that the S-E unitaries include nothing act on the system dimension only. All evolutions on the local system dimension are absorbed into the instruments. Therefore, the result data explicitly link the instruments and the transformaton of the state on system dimension. This may helps the calibration of quantum operations. Moreover, the models of instruments and the SE unitaries are flexible to be manipulated depending on assumptions the experimenter takes and the characteristics of the instrument set the experimenter interested in. For example, the constraints of instruments can be replaced by the CPTP for quantum gates on NISQ devices, while the instruments of measurements are assumed to be vectors. The SE unitary can also be modeled as a CPTP real orthonormal matrix.

It is obvious that the optimization problem is non-convex and may have multiple global optima, because each estimator $\hat{p}_{\bm{x}}$ consists of mulplications of variable matrices resulting in $(k+2)$-order of polynomial with $k$-order of exponential parameters. Hence, a reasonable initialization of parameters is significant for the optimization. We recommend conducting the LIST (or regular MLE-GST under the Markovian assumption if the LIST generates a nonphysical result) for the initialization of the MLE-IST with identity initialization of $\hat{V}_t$.

Additionally, the MLE-IST can also work with reduced instrument set. However, it requires $\mathcal{O}(d^{4k})$ parameters which is exponential with respect to the non-Markovian order $k$. It is intractable to solve the problem with exponentially increasing number of parameters. Therefore, we propose the reduced instrument set MLE-IST framework but do not implement it for simulations and experiments.

% \subsubsection*{MLE-IST with reduced instrument set}

% While performing MLE-IST with reduced instrument set (reduced MLE-IST), an instrument $\mathcal{A}_{x_t}^{(t)}$ is modeled as it in the full instrument set MLE-IST with CPTNI constraints. The process tensor is modeled by $\hat{\Upsilon}_\mathcal{T} \in [-1,1]^{d^{2k}\times d^{2k}}$ with CP and casuality constraints. Specifically, the casuality requires $\langle\!\langle \hat{\Upsilon} \vert 0\rangle\!\rangle = 1$ and
% \begin{gather}
%   \langle\!\langle \hat{\Upsilon} \vert P_{\mathrm{ban}}\rangle\!\rangle = 0, \\
%   \forall P_{\mathrm{ban}} := I^{\otimes{2t+1}} \otimes \left(\begin{matrix} \tilde{Q}_{2t+2} \\Q_{2t+3}\\\vdots\\ Q_{2k-1}\end{matrix}\right), \forall t,\\
%   \tilde{Q} \in \left\{P_1,\dots,P_{d^2-1}\right\},\\
%   Q \in \left\{P_0,P_1,\dots,P_{d^2-1}\right\}.
% \end{gather}
% Then, the estimator is modeled by
% \begin{align}
%   \hat{p}_{\bm{x}}=\langle\!\langle \hat{\Upsilon} \vert\left(\begin{matrix}\vert\hat{\chi}^{(0)}_{x_0}\rangle\!\rangle\\\vdots\\\vert\hat{\chi}^{(k-1)}_{x_{k-1}}\rangle\!\rangle\end{matrix}\right).
% \end{align}

% Consequently, the optimization problem of the reduced MLE-IST is given by
% \begin{align}\label{eq:mle_ist_opt_prob_reduced}
%   \min&_{\substack{\langle\!\langle \hat{\Upsilon} \vert, \vert\hat{\chi}^{(t)}_{x_t}\rangle\!\rangle,\forall x_t, t
%   }}~l(\hat{\mathfrak{I}})\\
%   s.t.~& \hat{\chi}^{(t)}_{x_t} = \frac{1}{\sqrt{d}} \sum_{i=0}^{d^4-1} \langle\!\langle i \vert\hat{\chi}^{(t)}_{x_t}\rangle\!\rangle P_i \succcurlyeq 0, ~ \forall{t,x_t},\tag{C7} \label{eq:reduced_def_inst_cp}\\
%   & 0 \le [\vert\hat{\chi}^{(t)}_{x_t}\rangle\!\rangle]_0 \le 1, ~\forall{t,x_t} \tag{C8} \label{eq:reduced_def_inst_tni}\\
%   & \langle\!\langle \hat{\Upsilon} \vert P_{\mathrm{ban}}\rangle\!\rangle = 0, ~\forall P_{\mathrm{ban}},\tag{C9} \label{eq:reduced_def_casuality_1}\\
%   & \langle\!\langle \hat{\Upsilon} \vert 0\rangle\!\rangle = 1, \tag{C10} \label{eq:reduced_def_casuality_2}
% \end{align}
% where \eqref{eq:reduced_def_inst_cp} and \eqref{eq:reduced_def_inst_tni} constraint the CPTNI of instruments, \eqref{eq:reduced_def_casuality_1} and \eqref{eq:reduced_def_casuality_2} guarantee the casuality of process tensor. See Method for detail. The reduced MLE-IST requires $\sum_{t=0}^{k-1}m_t d^4 + d^{4k} - \frac{d^{2k}-d^2}{d+1}$ parameters to estimate the instrument set as described in Eq.~\eqref{eq:result_list} after the devectorization of instruments.

% Obviously, this formulation constraints the instrument set in a quite simple form that all inaccessible initial state and SE unitary dynamics are modeled in a vector represented process tensor. However, it requires $\mathcal{O}(d^{4k})$ parameters which is exponential with respect to the non-Markovian order $k$. It is intractable to solve the problem with exponentially increasing number of parameters. Therefore, we propose the reduced instrument set MLE-IST framework but do not implement it for simulations and experiments. The flexibility to adjust constraints to fit the various assumptions still holds. Moreover, this optimization problem is still non-convex and may have multiple global optima, since each estimator consists of $(k+1)$-order of polynomial parameters.

\subsection{Performing IST on NISQ Devices}
A typical NISQ device execute a given quantum circuit consisting of CPTP intermidiate operations and a measurement at the end. Hence, instruments at time step $t$ consist of a set of CPTP operations and a set of measurements,
\begin{align}\label{eq:nisq_inst}
  \mathcal{J}^{(t)} := \left\{\left\{\mathcal{A}^{(t)}_{x_t}\right\}, \left\{\mathcal{M}_{x_t}^{(t)}\right\}\right\}.
\end{align}

Hence, the measurement probability for a $k$-time step non-Markovian quantum circuit is given by
\begin{align}\label{eq:nisq_exp}
  p_{\bm{x}} = \left(\begin{matrix}\langle\!\langle M^{(k-1)}_{x_{k-1}}\vert\\\langle\!\langle 0_E\vert\end{matrix}\right)\prod_{t=0}^{k-2} {U_{t:t+1}\left(\begin{matrix}A^{(t)}_{x_t}\\ I\end{matrix}\right)} \vert\rho^{(0)}_{SE}\rangle\!\rangle.
\end{align}
Associating with Eq.\eqref{eq:nisq_exp} and Eq.\eqref{eq:se_evo_prob_tr}, the measurement $\langle\!\langle M_{x_t}^{(t)}\rvert$ is the first row of the PTM of a CP and trace decreasing (TD) instrument with other entries $0$ at the end. Therefore, LIST can be performed as regular non-Markovian situation by representing measurements as regular instruments at the last time step and takes the first row as the result, when the non-Markovian correlation is sufficient to construct process-informationally complete basis by justing the instruments before the time step. However, it is intractable to perform ordinary LIST in the other case. This results from that the measurement should be the last instrument of the circuit. Hence, every time step are considered as the last time step in LIST.

Based on the observation that the PTM matrix of a (CPTD) measurement is always linear independent with the CPTP maps, the LIST can be performed by separately conducting the LIST subroutine for CPTP maps and measurements. The first row of CPTP maps are omitted in the vectorization in Eq.~\eqref{eq:inst_decomp_vec}, leading to the requirement of $d^2(d^2-1)$ measured probabilities per CPTP map and $d^2(d^2-1)\times d^2(d^2-1)$ dimensional gauge matrix. Then, the tomography of measurements are conducted by $d^2$ measured probabilities per measurement and $d^2\times d^2$ dimensional gauge matrix. Probability data and gauge matrix are measured and optimized in the two subroutine independently. Other steps are the same as ordinary LIST.

As for MLE-IST, the model can be simplified to enhance the efficiency. Each measurement can be modeled by a $d^2$ dimensional real row vector $\langle\!\langle \hat{E}_{x_t}\vert\in[-1, 1]^{1\times d^2}$ with positive constraints, i.e., both the matrix $\hat{E}_{x_t}$ the $\langle\!\langle \hat{E}_{x_t}\vert$ represents and $I-\hat{E}_{x_t}$
are positive semidefinite. Each intermediate instrument can be modeled by $d^2\times (d^2-1)$ parameters with CP constraint, since the TP constraint implies that the first row of $\hat{R}_{x_t}^{(t)}$ is $[1,0,0,\dots,0]$. Then, the estimator of probability is given by
\begin{align}\label{eq:nisq_estimator}
  \hat{p}_{\bm{x}} = \left(\begin{matrix}\langle\!\langle \hat{E}^{(k-1)}_{x_{k-1}}\vert\\\langle\!\langle 0_E\vert\end{matrix}\right)\prod_{t=0}^{k-2} {\hat{V}_{t:t+1}\left(\begin{matrix}\hat{R}^{(t)}_{x_t}\\ I\end{matrix}\right)} \vert\hat{\rho}^{(0)}_{SE}\rangle\!\rangle.
\end{align}

Consequently, the optimization problem for MLE-IST on NISQ devices can be described as
\begin{align}
  \min&_{\substack{\lvert \rho^{(0)}_{SE}\rangle\!\rangle, \langle\!\langle E_{x_t}^{(t)} \vert,R^{(t)}_{x_t},V_{t:t+1}, 
  \forall x_t, t}}~l(\hat{\mathcal{I}}),\\
  s.t.~& \eqref{eq:full_def_cp_costraint}, \eqref{eq:full_def_ptm_val_constraint},\eqref{eq:full_def_init_state_cp_constraint},\eqref{eq:full_def_init_state_tr1_constraint}, \eqref{eq:full_def_upval_constraint}\notag\\
  &[R_{x_t}^{(t)}]_{0,i} = \delta_{0,i}, \forall x_t, i,\tag{C7}\label{eq:cptp_constraint}\\
  &\hat{E}^{(t)}= \frac{1}{\sqrt{d}} \sum_{i=0}^{d^2-1} \langle\!\langle E_{x_t}^{(t)} \vert i\rangle\!\rangle P_i \succcurlyeq 0,~ \forall t,\tag{C8}\label{eq:mea_positive}\\
  &I-\hat{E}^{(t)} \succcurlyeq 0,~ \forall t,\tag{C9}\label{eq:comp_mea_positive}
\end{align}
where Eq.~\eqref{eq:cptp_constraint} is the CPTP constraint, Eq.~\eqref{eq:mea_positive} and Eq.~\eqref{eq:comp_mea_positive} are positive constraints of measurements.

\subsection{Experiment Result}

We first conduct a $5$-time-step single-qubit LIST simulation with overcomplete instruments and SE unitaries $R_{ZZ}(0.2)$ for all time step. Details of instruments and SE unitaries are given in the Method and depicted in Fig.~\ref{fig:ideal_inst_set}. Note that the knowledge and the basic implementation of $\mathcal{A}_4$ and $\mathcal{A}_5$ are not identical. 

The tomographic result of CPTP maps are depicted in the Fig.~\ref{fig:u1_list_ptm}. The difference between knowledge and implementation of $\mathcal{A}_4$ and $\mathcal{A}_5$ are detected. However, the disharmony not only influence the tomographic results of $\mathcal{A}_4$ and $\mathcal{A}_5$, since the LIST cannot distinguish which instrument is correctly implemented. Besides, the nonunique global optima leads to the results consistent with the probability but different from corresponding PTMs in Fig.~\ref{fig:ideal_inst_set}. 

% Figure environment removed

% Figure environment removed


Then, a $5$-time-step single-qubit MLE-IST is simulated with perfect and imperfect implemented complete instruments and SE unitaries $R_{ZZ}(0.2)$ for all time step. As depicted in Fig.~\ref{fig:perfect_inst_set} and Fig.~\ref{fig:imperfect_inst_set}, the tomographic results shows that the IST methods effectively reconstruct the instrument set. However, there are non-ignorable differences between the setups and the results of unitaries, initial state and measurement at the start and the end time step in the imperfect scenario. The sufficiency of constructing process-informationally complete decomposition basis influences the tomographic result in a subtle way. There are more results of unitary evolutions and initial quantum states that fit the probability data when the sufficiency is not satisfied. As a consequence, the output may not meet the experimenter's expectation, but is loyal to the data. 

Moreover, we demonstrate the instrument set of $4$-time-step single-qubit MLE-IST on the real quantum device of IBM Quantum Experience (QX) with complete instruments in Fig~\ref{fig:ibm_lima_inst_set}. Each time step consists of 10 time slots of single qubit gate depending on the quantum hardware. The result shows the potentiality guiding the quantum device engineering.

% Figure environment removed
% Figure environment removed
% Figure environment removed

\section{Method}

\subsection{Decomposition of Instruments for LIST}

By Pauli transfer matrix (PTM) representation defined in \cite{greenbaum2015Introduction}, the probability of getting $\bm{x}$ as described in Eq.\eqref{eq:se_evo_prob_tr} can be reformed as
\begin{align}
  p_{\bm{x}} =& \langle\!\langle0_{SE}\vert \!\left(\!\!\begin{matrix}A^{(k-1)}_{x_{k-1}}\\ I\!\end{matrix}\right)\! \prod_{t=0}^{k-2}U_{t:t+1}\!\left(\begin{matrix}\!\!A^{(t)}_{x_t}\\ I\end{matrix}\right) \!\vert \rho^{(0)}_{SE}\rangle\!\rangle,
  \label{eq:nm_exp_ptm}
\end{align}
where $\vert \bullet \rangle\!\rangle$ and $\langle\!\langle \bullet\vert$ are the superoperators of a quantum state and a positive operator-valued measurement (POVM) operator, $A^{(t)}_{x_t}$ and $U_{t:t+1}$ are the PTM representations of $\mathcal{A}^{(t)}_{x_t}$ and $\mathcal{U}_{t:t+1}$, respectively, and $I$ is the identity. When performing tomography at time step $t$, the probability of getting $x_t$ with $\bm{x}^{+}$ and $\bm{x}^{-}$ is
\begin{align}
  &p_{\bm{x^+}\bm{x^-}}(x_t)\\
  &\begin{aligned}=&\langle\!\langle 0_{SE}\vert \left[\prod_{i=1}^{k-t-2}\left(\begin{matrix}A^{(t+i)}_{x^{+}_i}\\ I\end{matrix}\right)U_{t+i-1:t+i}\right]  \\
    &\left(\begin{matrix}A^{(t)}_{x_t}\\ I\end{matrix}\right)\left[\prod_{j=1}^{t-1}U_{j:j+1}(A^{(j)}_{x^{-}_j}\otimes I)\right] \vert \rho^{(0)}_{SE}\rangle\!\rangle\end{aligned}\\
  =&\langle\!\langle 0_{SE}\vert F_{\bm{x}^+} \left(\begin{matrix}A^{(t)}_{x_t}\\ I\end{matrix}\right)F_{\bm{x}^-}\vert \rho^{(0)}_{SE}\rangle\!\rangle\\
  =&\langle\!\langle F_{\bm{x}^+}^{SE}\vert  \left(\begin{matrix}A^{(t)}_{x_t}\\ I\end{matrix}\right)\vert F_{\bm{x}^-}^{SE}\rangle\!\rangle\\
  =&\sum_{ij}\langle\!\langle F^S_{\bm{x}^+,i}\vert A^{(t)}_{x_t}\vert F^S_{\bm{x}^-,j}\rangle\!\rangle \langle\!\langle F^E_{\bm{x}^+,i}\vert F^E_{\bm{x}^-,j}\rangle\!\rangle\\
  =&\mathrm{Tr}\left[\!\sum_{ij}\langle\!\langle F^E_{\bm{x}^+\!,i}\vert F^E_{\bm{x}^-\!,j}\rangle\!\rangle \vert F^S_{\bm{x}^-\!,j}\rangle\!\rangle\langle\!\langle F^S_{\bm{x}^+\!,i}\vert A^{(t)}_{x_t}\!\right],\label{eq:inst_decomp_detail}
\end{align}
which corresponds to Eq.~\eqref{eq:inst_decomp_tr_basis} implying the decomposition of $A^{(t)}_{x_t}$ on the non-orthogonal basis $\mathbb{B}^{(t)} = \left\{B_{f(\bm{x}^+,\bm{x}^-)} := \sum_{ij}\langle\!\langle F^E_{\bm{x}^+,i}\vert F^E_{\bm{x}^-,j}\rangle\!\rangle \vert F^S_{\bm{x}^-,j}\rangle\!\rangle\langle\!\langle F^S_{\bm{x}^+,i}\vert\right\}$. Reconstruction of $A^{(t)}_{x_t}$ requires $\mathbb{B}^{(t)}$ to be process-informationally complete that there exist at least $d^4$ linear independent basis matrices in it. Then we can obtain the decomposition in Eq.\eqref{eq:inst_decomp_vec} via the vectorization of the matrices. 

% The vectorization method is specified as the superoperator of the Choi state of the operand without loss of generality for the simplicity of process tensor representation. Specifically, let 
% \begin{align}
%   \vert \chi_{x_t}^{(t)}\rangle\!\rangle = \sum_i s_i A_{x_t}^{(t)} \vert i \rangle\!\rangle \otimes \vert i \rangle\!\rangle,\\
%   \langle\!\langle B_{\alpha}^{(t)}\vert = \sum_j s_j \langle\!\langle j \vert B_{\alpha}^{(t)} \otimes \langle\!\langle j\vert,
% \end{align}
% where $\sum_i s_i \vert i \rangle\!\rangle \otimes \vert i \rangle\!\rangle$ is the superoperator of maximum entangled state, $s_i \in \{-1, 1\}$. Then, the measurement probability is represented by 
% \begin{align}
%   \left(\vert p_{x_t}^{(t)}\rangle\!\rangle\right)_\alpha =& \langle\!\langle B_{\alpha}^{(t)}\vert \chi_{x_t}^{(t)}\rangle\!\rangle \\
%   =& \sum_{i,j} s_i s_j \langle\!\langle j \vert B_{\alpha}^{(t)} A_{x_t}^{(t)} \vert i \rangle\!\rangle \otimes \langle\!\langle j\vert i \rangle\!\rangle\\
%   =& \sum_{i} \langle\!\langle i \vert B_{\alpha}^{(t)} A_{x_t}^{(t)} \vert i \rangle\!\rangle\\
%   =& \mathrm{Tr}\left[B_{\alpha}^{(t)} A_{x_t}^{(t)}\right] = p_{\alpha,x_t}^{(t)},
% \end{align}
% which brings about the Eq.\eqref{eq:inst_decomp_ptm_choi}.


\subsection{Gauge Freedom}
The tomography of the instrument set shows gauge freedom up to a set of invertible matrices $\{B^{(t)}\}$ because of the inaccessible initial state and SE unitaries. We can not distinguish the quantum operations by probability measurement up to $\{B^{(t)}\}$. This is because the probability is given by
\begin{align}
  p_{\bm{x}} &= \mathrm{Tr}\left[\Upsilon_{\mathcal{T}}^\dagger\left(\begin{matrix}A^{(0)}_{x_0}\\\vdots\\A^{(k-1)}_{x_{k-1}}\end{matrix}\right)\right]\\
  &=\sum_{\bm{x}}p_{\bm{x}}\prod_{t=0}^{k-1}\mathrm{Tr}\left[D^{(t)\dagger}_{x_t}A^{(t)}_{x_t}\right]\\
  &=\sum_{\bm{x}}p_{\bm{x}}\prod_{t=0}^{k-1}{\bm{d}}^{(t)\dagger}_{x_t}\bm{a}^{(t)}_{x_t}\\
  &=\sum_{\bm{x}}p_{\bm{x}}\prod_{t=0}^{k-1}{\bm{q}}^{(t)\dagger}_{x_t}\left(B^{(t)}\right)^{-1} B^{(t)}\bm{p}^{(t)}_{x_t},
\end{align}
where $\left\{ \bm{q}^{(t)}_{x_{t}}\right\}$ is the dual set of $\left\{\bm{p}^{(t)}_{x_{t}}\right\}$ corresponding to $\left\{A^{(t)}_{x_{t}}\right\}$. This indicates that, for any gauge $\{B^{(t)}\}$, we can obtain a set of instruments and process tensor without violations of measurement probabilities $p_{\bm{x}}$.

Another kind of gauge freedom is the indeterminacy of SE unitaries and initial states. Specifically, we can not distinguish $\langle\!\langle 0_{SE}\vert F_{\bm{x}^+} U(A^{(t)}_{x_t}, I)F_{\bm{x}^-}\vert\rho^{(0)}_{SE}\rangle\!\rangle$ and $\langle\!\langle 0_{SE}\vert F_{\bm{x}^+} ( A^{(t)}_{x_t}, I ) F_{\bm{x}^-} U\vert\rho^{(0)}_{SE}\rangle\!\rangle$, where U is an arbitrary operation that commutes with $(A, I)$. For example, the depolarizing noise on the system with an arbitrary operation on the environment. This generally results in different sets $\{F_{\bm{x}^+}, F_{\bm{x}^-}, \vert\rho^{(0)}_{SE}\rangle\!\rangle\}$ that consistent with the data. We adopt the reduced definition of instrument set using process tensor instead of discussing $F_{\bm{x}^+}$, $F_{\bm{x}^-}$ and $\vert\rho^{(0)}_{SE}\rangle\!\rangle$ themselves to avoid explicit introducing of this type of gauge freedom.

\subsection{Likelihood Function}

Specifying a sequence $\bm{x}$, the probability defined in Eq.~\eqref{eq:se_evo_prob_tr} is measured by repeating the experiment $n_s$ times and record $n_{\bm{x}}$ how many times the desired outputs occur. Therefore, we use the general likelihood function of instrument set
\begin{align}
  \mathcal{L}(\hat{\mathfrak{I}}) = \prod_{\bm{x}}(\hat{p}_{\bm{x}})^{n_{\bm{x}}}(1-\hat{p}_{\bm{x}})^{n_{s}-n_{\bm{x}}},
\end{align}
where $\hat{p}_{\bm{x}}$ is the probability estimator modeled by parameters.

By exploiting the central limit theorem, each term of the likelihood can be rewritten as a normal distribution,
\begin{align}
  \mathcal{L}(\hat{\mathfrak{I}}) = \prod_{\bm{x}}\exp\left[-\frac{\left(\tilde{p}_{\bm{x}}-\hat{p}_{\bm{x}}\right)^2}{\sigma_{\bm{x}}^2}\right],
\end{align}
where $\tilde{p}_{\bm{x}}=n_{bm{x}}/n_s$ represents the measured probability, $\sigma_{\bm{x}}^2=\tilde{p}_{\bm{x}}(1-\tilde{p}_{\bm{x}})/n_s$ is the sampling variance in the measurement $m_{\bm{x}}$. Exploiting the the monotonic logarithm function, maximizing $\mathcal{L}$ is equivalent to minimizing the weighted mean square error (MSE)
\begin{align}
  l(\hat{\mathfrak{I}})=& -\log(\mathcal{L}(\hat{\mathfrak{I}})) = \sum_{\bm{x}}\frac{\left(\tilde{p}_{\bm{x}}-\hat{p}_{\bm{x}}\right)^2}{\sigma_{\bm{x}}^2}.
\end{align}

\subsection{Detail of Numerical Simulation}
We conduct the numerical simulations on classical computers by Python with Pennylane package. The probability $\tilde{p}$ is analytically computed without sampling error appling the model as shown in Fig.~\ref{fig:qsp}. We specify the number of qubits to simulating the environment is the same as the system. Moreover, we introduce the $U_{\textnormal{-}1:0}$ to generate the initial state as $\vert\rho^{(0)}_{SE}\rangle\!\rangle=U_{\textnormal{-}1:0}\vert \rho^{(\textnormal{-}1)}_{SE}\rangle\!\rangle$, where $\vert \rho^{(\textnormal{-}1)}_{SE}\rangle\!\rangle$ is the superoperator of $\vert0_{SE}\rangle\langle0_{SE}\vert$.

Instruments consist of a measurement $\mathcal{M}$ and CPTP operations $\mathcal{A}_i$, $i=0,1,\dots,5$. 
The knowledge of measurements and CPTP maps known to the experimenter is defined as
\begin{align}
  \mathcal{M} &:=\vert 0 \rangle\langle 0\vert,~ \mathcal{A}_0 := I,~ \mathcal{A}_1 := X,\\
  \mathcal{A}_{2} &:=\sqrt{X}R_Z\left(\frac{\pi}{2}\right),~
  \mathcal{A}_3 := R_Z\left(\frac{\pi}{2}\right)\sqrt{X},\\
  \mathcal{A}_4 &:=\sqrt{X}R_Z\left(\frac{\pi}{3}\right),~ \mathcal{A}_5 := R_Z\left(\frac{\pi}{3}\right)\sqrt{X}.
\end{align}
However, the basic implementation of the $\mathcal{A}_4$ and $\mathcal{A}_5$ are $\sqrt{X}R_Z\left(\frac{\pi}{4}\right)$ and $R_Z\left(\frac{\pi}{4}\right)\sqrt{X}$, respectively.

The perfect implementations of instruments are the basic implementations described above. For the imperfect implementations, there is a depolarizing channel and an amplitude damping channel after the basic implementation of each CPTP operation. The parameters of depolarizing and amplitude damping are set increasingly with respect to the time step as $0.05(t+1)$. 

Simulations uses both the complete instruments and overcomplete instruments, where the complete instruments are given by
\begin{equation}\label{eq:complete_instruments}
  \mathcal{J}^{(t)} := \left\{\begin{aligned}&\left\{\mathcal{A}_0, ..., \mathcal{A}_3, \mathcal{M}\right\}, & t\ne k-1,\\
    &\left\{\mathcal{M}\right\}, & t=k-1,\end{aligned}\right.\\
\end{equation}
and the overcomplete instruments are defined as
\begin{equation}\label{eq:overcomplete_instruments}
  \mathcal{J}^{(t)} := \left\{\begin{aligned}&\left\{\mathcal{A}_0, ..., \mathcal{A}_5, \mathcal{M}\right\}, & t\ne k-1,\\
    &\left\{\mathcal{M}\right\}. & t=k-1.\end{aligned}\right.\\
\end{equation}


% For all simulations, SE unitary evolutions are selected from unitaries defined as
% \begin{align}
%   U_0 &:= R_{ZZ}(0.2)R_{YY}(0.2)R_{XX}(0.2),\\
%   U_1 &:= R_{ZZ}(0.2),\\
%   U_2 &:= R_{IX}(0.2)R_{ZZ}(0.2),\\
%   U_3 &:=  R_{IX}(0.2)R_{ZZ}(0.2)R_{XI}(0.2),
% \end{align}
% where $R_{P^SP^E}(\theta):=\exp(\frac{-\iota \theta P^SP^E}{2})$. Specifically, $U_1$, $U_2$, and $U_3$ introduce the non-Markovian quantum correlations that are insufficient to construct process-informationally complete decomposition basis for the first and last time steps but sufficient for intermediate time steps. On the contrary, the non-Markovian correlations of $U_0$ are sufficient to construct the basis for all time steps. Besides, there is no component of $U_0$ and $U_1$ on the system or the environment only. $U_2$ and $U_3$ have the components on the environment and both the system and environment, respectively.


\section{Discussion}\label{sec:discussion}

In this paper, we proposed a framework the instrument set tomography (IST) for quantum gate set tomography under the non-Markovian situation. Based on the quantum stochastic process operationally representing the non-Markovian quantum correlation and evolution, the instrument set is defined in the full and reduced formation. We first proposed a quick linear inversion method based on the reduced instrument set for IST, aka LIST. Consequently, both the disharmony of linear relationship of instruments and the non-Markovian quantum correlations are detected and described with gauge freedom by LIST. However, because of the absence of constraints in the gauge optimization, the result of linear independent instruments is always the prior knowledge when the probability matrix is full rank. Moreover, the result of LIST is not guaranteed to be physical implementable. Then, a statistical method based on the miximum likelihood estimation for IST is proposed as MLE-IST with the ability utilizing overcomplete data. Based on the full instrument set, the MLE-IST tries to explicitly describe the detail of SE correlations with polynomial number of parameters with respect to the Markovian order. The results of MLE-IST is guaranteed to be physical implementable with constraints based on the assumptions of the quantum device. Specifically, we demonstrate how to implement IST on the current noisy quantum intermediate-scale quantum (NISQ) devices. The results of simulations and experiments shows the effectiveness of describing instruments and the non-Markovian quantum system including the initial state and the SE correlations. The IST provide an essential method for benchmarking and developing a quantum device under non-Markovian situation in the aspect of instrument set.
\backmatter

\bmhead{Acknowledgments}

This work is supported by NSFC projects 61960206005 and the Fundamental Research Funds for the Central Universities 2242022k60001, and in part by the National Science Foundation of China under Grant 61871111.


%%===========================================================================================%%
%% If you are submitting to one of the Nature Portfolio journals, using the eJP submission   %%
%% system, please include the references within the manuscript file itself. You may do this  %%
%% by copying the reference list from your .bbl file, paste it into the main manuscript .tex %%
%% file, and delete the associated \verb+\bibliography+ commands.                            %%
%%===========================================================================================%%

\bibliography{citations}% common bib file
%% if required, the content of .bbl file can be included here once bbl is generated
%%\input sn-article.bbl


\end{document}




\title{Energy- and Time-aware Coordination\\ for Multi-core Cyber-physical Systems}
% Author list symbols commands
\newcommand{\correspondingauthor}{ (\raisebox{-0.25em}{\Envelope})}
%\newcommand{\orcid}[1]{\href{https://orcid.org/#1}{\textcolor[HTML]{A6CE39}{ORCID}}}
%
\author{Lukas Miedema\inst{1}\orcidID{0000-0002-7295-6568}
  \and
  Clemens Grelck\inst{1,2}\orcidID{0000-0003-3003-1388}\thanks{Corresponding author}
}
%
\authorrunning{L.~Miedema, C.~Grelck}
\titlerunning{Energy- and time-aware coordination for multi-core cyber-physical systems}
%
\institute{University of Amsterdam, Netherlands\\
  \email{\{l.miedema, c.grelck\}@uva.nl}
  \and
  Friedrich Schiller University Jena, Germany\\
  \email{clemens.grelck@uni-jena.de}
}
%
\maketitle
%
\begin{abstract}
We present the coordination language TeamPlay for the high-level specification of cyber-physical systems with strong requirements on non-functional properties such as execution time and energy consumption, both as a constraint (deadline, energy budget) and as an optimisation objective (save energy, run quickly).
TeamPlay makes trade-offs between time, energy and (optional) redundancy for fault-tolerance visible in early system design and implementation stages.

In this paper we describe the essential concepts of the TeamPlay coordination language, mostly in an abstract way.
Eventually, we sketch out our programme for the SusTrainable summer school to be held in Coimbra, Portugal, in July 2023.
\end{abstract}

\section{Introduction}
\label{sec:summary}

Cyber-physical systems (CPS) combine hardware/software (cyber) systems with the physical word seen through sensors and impacted through actuators.
In cyber-physical systems the functional correctness of software, thus, is of pa\-ra\-mount importance as erroneous behaviour is not ``just'' an annoyance, but may have disastrous consequences in the physical world including loss of life.
Beyond functional correctness, non-functional properties such as execution time and energy consumption play an equally important role.
Typically, reactions to sensor inputs must occur within some given deadline.
Energy consumption is not only relevant as a general environmental concern, but many cyber-physical devices are in fact battery-driven.
As such energy consumption is in a direct relation to the usefulness of a cyber-physical system.

From a software engineering point of view the question is how confidence in functional correctness, meeting of execution deadlines and energy consumption can all be addressed at an equal level of importance without creating an unmaintainable software mess.
We proposed the coordination language \emph{TeamPlay} \cite{RoedRouxAltm+20} to address these concerns.
TeamPlay embraces the concept of \emph{exogenous} coordination \cite{Arbab98,Arbab04} and organises a concurrent application as a set of independent, identifiable black-box \emph{components} (aka \emph{tasks}) that communicate with each other solely via data-flow channels.
Components are meaningful, sequential application building blocks, implemented in a general-purpose programming language.

TeamPlay enforces a stringent two-layer software architecture where components are simple enough to be implemented correctly for a given input-output relation.
Furthermore, components have a defined dynamic behaviour regarding worst-case/average-case execution time and energy consumption for any given execution platform under consideration.
These data could be obtained through static code analysis or through experimentation, depending on the compute architecture.
A system integrator then takes these (in coding terms) black-box components and synthesise a systems by means of the TeamPlay coordination language.

We primarily target high-performance embedded system architectures, like the Odroid or Jetson families.
These combine heterogeneous multi-core CPUs with DVFS and highly parallel GPUs for bulk computing.
Mapping TeamPlay coordination programs to high-performance embedded systems under time and energy constraints creates a complex scheduling (in time) and mapping (in space) problem that we address with various scheduling/mapping schemes \cite{RoedRouxAltm+20b,RoedRouxGrel21,RoedPimeGrelAIAI23} and a runtime middleware for said high-performance embedded systems \cite{RouxAltmGrel21}.

During the SusTrainable summer school in Coimbra, Portugal, in July 2023 we will motivate the use of exogenous coordination in the field of cyber-physical systems, introduce the essential features of the TeamPlay coordination language and explain the ins and outs of the scheduling and mapping problem in the presence of heterogeneous high-performance embedded systems.
During the labs students are supposed to implement a simple (mock) cyber-physical system using TeamPlay.
We will provide the latest implementation via a virtual machine to avoid installation hassle and uncertainties.


%In the remainder of this document, we describe our overall objectives as well as our on-going work with respect to these three pillars in more detail.


\section{Components and contracts}
\label{components}

As pointed out before, a TeamPlay application is organised as a collection of interacting components.
Components are meaningful, sequential application building blocks, implemented in a general-purpose programming language.
As is common in the domain of cyber-physical systems, we exclusively work with the system programming language C in practice.
We illustrate TeamPlay components in Figure~\ref{fig:components3}.


% Figure environment removed

\noindent
A TeamPlay component consists of four entities:
\begin{description}
\item[Name:] a unique name: \textit{ImageCapture} and \textit{ObjectDetection} in the example;
\item[Code:] a component implementation in the form of a callable, linkable C function object code;
\item[Functional contracts:] define the component's functional interaction with the outside world;
\item[Non-functional contracts:] define the component's non-functional properties, namely with respect to energy and time.
\end{description}
We further refine the functional contracts of a component into:
\begin{description}
\item[Input ports:] a set of pairs specifying type and name of each input port;
\item[Output ports:] a set of pairs specifying type and name of each output port;
\item[State ports:] a set of pairs specifying type and name of each internal state, which in this way is externalised to the coordination layer.
\end{description}
If the set of input ports is empty, we speak of a \textit{source component} that controls one or more sensors.
If the set of output ports is empty, we speak of a \textit{sink component} that controls one or more actuators.
And, if the set of state ports is empty, we speak of a \textit{state-less component}.
Information regarding the internal state of a component is crucial for its automatic replication or migration between multiple execution units.

We further distinguish two kinds of non-functional contracts:
\begin{description}
\item[Time contract:] the timing behaviour of the component, namely average and/or worst-case execution time;
\item[Energy contract:] the energy behaviour of the component, namely average and/or worst-case energy consumption.
%\item[Security contract:] the security level of one specific implementation of the component, for the time being as a discrete number.
\end{description}
Functional and non-functional contracts differ from each other in one crucial aspect: any implementation of a component must obey the functional contracts, but the non-functional contracts are specific to one particular execution platform (system architecture).
Hence the functional contracts of a component are not stored in the coordination source file, but in a separate data base.
In a simplistic view this data base could be seen as a matrix with all components of an application on one axis and all potential execution platforms and their individual processing units (e.g.~Arm big.LITTLE) on the other axis.
Determining a component's non-functional properties on a specific processing unit is beyond the scope of our work.
For relatively simple architectures this could be done with static analysis \cite{WegeNikoNune+23}, but for more complex architectures (deep caches, out-of-order execution, etc.) this is usually accomplished by dynamic profiling \cite{Seewald2019coarse} and some configurable safety margin.

% Figure environment removed

Our observation is that usually a component can be implemented in various different ways.
Furthermore, component implementations can be compiled into executable code with different compilation objectives (e.g.~speed, energy, code size, etc.).
Hence, we further refine our component model to support \emph{multi-version components}, as illustrated in Figure~\ref{fig:components4}.
Multi-version components have multiple implementations that all share the component's functional contracts, but behave differently with respect to the non-functional contracts, even on the very same execution platform and processing unit.

The genuine rationale behind multi-version components is to offer a choice of different non-functional behaviours as described by the non-functional contracts.
Consequently, the coordination layer, or more precisely the component space/time scheduler, may choose among the various versions according to global application-specific objectives.
Multiple component versions can further be used to support multiple binary-incompatible execution units or to optimise one (source code) implementation of a component for one specific type of execution unit, even if they are binary-compatible in principle.
The coordination run-time system \cite{RouxAltmGrel21} may even react on changing environmental conditions, as for example the battery status of some device.

%Of course, components may only have a single version only, and in the following we use the terms \textit{contract}, \textit{behaviour}, etc, with respect to components and their versions interchangeably.

\ignore{
In essence, (non-functional) contracts define the ETS properties of the component. Hence, each component has a defined run-time behaviour, amount of energy required and a security level obtained from the analysis tools of the other partners. Additionally, each component can have multiple implementations with equivalent functional, but different extra-functional behaviour. Besides the guaranteed performance of the individual components, we need to specify the component-specific and system-wide ETS-constraints, expressed as, for instance, deadlines on the response-time of a component or energy budgets of the entire system or minimum security levels.
}

\ignore{
We use the very same annotations explained in Deliverable D7.4 to define the ETS-characteristics of each component, e.g.:\\
\textit{ \_\_teamplay\_\{qualifier\}\_metric(name\{,default, confidence\});stmt;} \\
These annotations form the central glue between the technical work packages: while some work packages are primarily concerned with deriving these non-functional properties from component implementations, others, among them work package 2, are concerned with interpreting and exploiting this additional information on the application level.
}

\section{Component interaction}

As illustrated in Figure~\ref{fig:coordination}, several components together form an application (or system).
They interact in the form of a streaming network or component workflow.
Such streaming networks may have arbitrary shape, but must be free of cycles.
In other words, our streaming networks form \emph{acyclic directed graphs}, or \emph{DAGs} for short.
In the example we see a source component \textit{ImageCapture} that feeds data into the streaming network.
The rationale behind the example is that this component is connected to some kind of camera, but in principle any sensor input is a typical case of such a source component.

% Figure environment removed


The data is passed on to two subsequent components, called \textit{ObjectDetection} and \textit{OpticalFlow}. This could be characterised as a broadcast of one data item to two components, as is the intention behind the example, or the \textit{ImageCapture} component could send two different data items to \textit{ObjectDetection} and \textit{OpticalFlow}, respectively.
Components \textit{ObjectDetection} and \textit{OpticalFlow} are stateless and process the data received from \textit{ImageCapture} in different ways and output their results to the subsequent component \textit{DecisionMaking}.

The \textit{DecisionMaking} component in the given example synchronises the data received from both incoming
streams in a pair-wise manner, derives a decision (as the name suggests) and forwards it to the final component of the workflow \textit{DecisionRec} for recording.
%In other scenarios a component with multiple incoming streams may well be activated by the presence of data on one of its input streams instead of all.
The final component \textit{DecisionRec} is a sink component and as such does not have any output port.

The whole execution regime is entirely data-driven: components are activated by the availability of data items on their input ports. Components process the data received and in turn produce data items on their output ports, which trigger subsequent components further down the workflow or streaming network.
Both the external behaviour of components as well as their orderly interaction in a streaming network or work flow are described in the TeamPlay coordination language \cite{RoedRouxAltm+20}.
The (static) type system of TeamPlay ensures the soundness of streaming networks.




\section{Component scheduling and mapping}

Eventually, componentised applications and their coordination glue must be mapped to execution hardware.
We specifically target heterogeneous parallel systems with multiple different types of execution units.
These types of execution units may or may not be binary-compatible, but they (almost) always expose different energy/time trade-offs.
Together with our multi-version component model we have a created a rich design space for the crucial question where to run what and when to meet application-specific global constraints and to achieve application-specific global objectives.
We illustrate this plethora of options and opportunities in Figure~\ref{fig:scheduling}.

% Figure environment removed

It is the combination of multi-version components with heterogeneous parallel hardware that in essence causes this explosion of options.
For example one may want to switch from a fast implementation to a more energy-efficient implementation as the battery status of the executing device degrades.
Different versions of the same component could implement different algorithms, could be compiled using different compiler flags (or even different compilers) use different parameters (frame-rate) or could make use of different execution units of a heterogeneous system, such as sketched out in Figure~\ref{fig:scheduling}.

Depending on application requirements we consider both energy and time as either a budget limitation or an optimisation problem.
For time both variations have classical role models and solutions: the real-time community works with time budgets while the high-performance computing community sees time as an object of optimisation.
Energy can be addressed analogously as a budget constraint or an optimisation objective.
The most typical combination in a our target application domain of cyber-physical systems, however, is to combine time budgets, i.e.~real-time deadlines, with energy optimisation objectives.
In other words we aim to run an application within the deadline with the least energy possible.
The energy reduction objective could be motivated by either environmental concerns or by battery life times, but that again is beyond the scope of our work.

The scheduling problem as such is NP-complete, and so optimal solutions can effectively only be computed for very small concrete problems.
Consequently, heuristics are used in practice, and we have integrated a number of such scheduling heuristics into our TeamPlay compiler \cite{seewald2019component,RoedRouxAltm+20b,RoedRouxGrel21,RoedPimeGrelAIAI23} that explore different aspects of the general subject matter.
Our energy-aware schedulers make use of an elaborate energy model that distinguishes between static and dynamic energy consumption and take full advantage of platform-specific DVFS capabilities (i.e.~dynamic voltage and frequency scaling).
Static energy consumption refers to the energy consumption of a processing unit in idle status, i.e.~without executing any code.
Static energy consumption, as the name suggests, is constant over time and only affected by application makespan, i.e.~the time it takes an application to run to completion or to complete one application cycle.
In other words, static energy consumption is a property of the type of execution unit, but application-independent.
Dynamic energy consumption depends on the instruction mix of an application component, the type of execution unit and the choice of voltage and frequency in case the hardware supports DVFS, which is a common case nowadays.

Execution time and energy consumption are not independent of each other, but they are not the same either.
In this sense, the fastest solution is typically not the most energy-efficient one.
Running an application faster cuts down the static energy consumption, but this is often over-compensated by running code on higher clock frequency and voltage or by even running it on a more powerful but likewise more energy-hungry execution unit.
For example, on the ARM big.LITTLE architecture we have used so far for illustration, moving a component from a LITTLE core to a big core typically increases platform energy consumption.
Should the compute power of a big core be required to meet the deadline, this additional energy is well invested, but otherwise most likely not.




\section{Related work}
\label{related}

Coordination itself is a well established computing paradigm with a plethora of languages, abstractions and approaches~\cite{CML+18}.
Yet, we are neither aware of any explicit adoption of the principle in the domain of cyber-physical systems, nor are we aware of energy- and time-aware approaches to coordination as a paradigm.

We have previously worked on the coordination language and component technology S-Net~\cite{GSS10,GrelJulkPencCCGRID12}, from which we draw both inspiration and experience for the design of TeamPlay.
However, like other coordination approaches S-Net merely addresses the functional aspects of coordination programming and has left out any non-functional requirements, not to mention energy or time specifically.

A notable exception in the otherwise pretty much uncharted territory of energy/time-aware coordination is the functional language Hume~\cite{HaM03} that was specifically designed with real-time systems in mind. Thus, guarantees on time (and space) consumption are key to Hume.
The main motivation behind Hume was to explore how far high-level functional programming features, such as automatic memory management, higher-order functions, polymorphism, recursion and currying, eventually up to full-scale Haskell, can be supported while still providing accurate real-time guarantees.

Pradhan et al.~\cite{Pradhan15} proposed \textit{CHARIOT}, a \textit{Domain Specific Language} for CPSoSes. While they do not describe their language as a \emph{coordination language}, it provides capabilities such that it could be considered a coordination language. The authors also do not explicitly describe utility for systems-of-systems and, instead, focus on \textit{fractionated} systems. A fractionated system is an architecture within which one system is composed of a set of networked systems at the same location (e.g.~a satellite swarm). The advantage of such a system is the ease of extendibility: a system providing a new capability (e.g.~sensor or processing platform) can be added to the general area of the other systems, thus enhancing the capabilities of the composite system.

\ignore{
The CHARIOT language aims at aiding the development of these complex, extensible cyber-physical systems-of-systems, primarily the integration between middleware produced by different vendors and managing complexity. A user specifies components, their communication channels (similar to inports and outports in TeamPlay), as well as their mapping to hardware in the coordination language. The presence of this mapping makes the final conversion step, where a CHARIOT specification is converted into instructions for a given middleware, rather straightforward. Their main contribution is in offering a consistent programming interface to address fractionated systems.

The composed system can switch between predefined modes referred to as \textit{objectives}. Objectives are somewhat similar to TeamPlay modes, given that they are discrete system states between which the system may switch. However, TeamPlay modes attempt to serve as a front-end to a \textit{coordination compiler} and further downstream tools, where the conversion between modes is non-trivial, yet handled automatically by design-time algorithms. At the same time, the CHARIOT objectives rather form a set of configuration points that are each written to a database. These points are available to the runtime, which handles them when they are selected. The contribution of Pradhan et al.~is primarily in the management of software development complexity, where the conversion of their DSL to a mapping is rather trivial (from an academic point of view). In contrast, the TeamPlay coordination language aims not only to reduce software development complexity, but also to optimise for objectives (such as energy, time or robustness) that are not feasible by traditional means.
}

More work published under the \emph{DSL} label shows considerable overlap with coordination languages. The contribution by Berger~\cite{Berger14} proposes a DSL for the integration of sensors into a cyber-physical system. The platform is supplied in a \textit{board specification file}. However, while the board can be changed, the work is limited to a single board. The DSL encodes the requirements for sensors attached to the board. Each sensor is attached by pins, which must be set to a particular mode (e.g.~analog or use i2c). Not all pins support all modes, which forms the basis of a \textit{constraint-satisfaction problem} to perform the assignment of sensors to pins. As such, this work is an example of using a language to formulate a problem that cannot be solved manually, similar to the TeamPlay language. However, the scope and focus on real-time of our coordination language remains unique.


\section{Summer school}
\label{school}

Our contribution to the planned summer school targets an audience of MSc and PhD students with a broad understanding of computer science in general and solid programming skills in particular.
We do not expect prior exposure to engineering cyber-physical systems or mapping/scheduling problems nor any prior knowledge of the topics described in this document.

The learning objectives for the summer school are as follows:
\begin{itemize}
\item obtain a basic understanding of the requirements of cyber-physical systems and resource-aware computing;
\item understand the concept of exogenous coordination;
\item familiarise with the TeamPlay coordination language;
\item solve a (simple) scheduling/mapping problem for a (mock) application;
\item optimise the energy consumption of said cyber-physical system by using the TeamPlay methodology \cite{RouxPaghAkes+20}
\end{itemize}

We will provide any subject-specific background required during the lecture.
However, we are convinced that just telling the story is not sufficient for a deeper and lasting learning experience.
Thus, a hands-on practical session is essential to meet the learning objectives.
Such practical sessions making use of research (prototypical) software incur the risk of losing considerable time in overcoming installation and compatibility issues.
We will address and avoid such problems by providing a docker container with a light-weight virtual machine and all required non-standard software pre-installed and tested.

The concrete task to be solved by the summer school participants has not yet been finalised entirely, but our working hypothesis circles around a fork-join component/task graph, where the fork component reads wifi signal strength (from some imaginary sensor) and forwards the data to a fixed number of tasks \emph{representing students in a room}.
These tasks determine the approximate locations of the students in the room, which are forwarded to a join task that takes some decision based on the locations, e.g.~raising an alarm should the 1.5m Covid-19 distance rule be violated.

We plan to provide the summer school participants with a monolithic solution to the problem and let them identify the logical components and their interaction using the TeamPlay coordination technology.
We provide non-functional properties, i.e.~time and energy, for the (rather obvious) building blocks of our application.
As a result the students will find out that the monolithic solution cannot be scheduled with the given deadline.
With the explicit componentisation and the TeamPlay methodology the application not only becomes schedulable, but as an extra incentive we will have a small competition who of the participants manages to solve the problem with the least energy demand.

Time-permitting, we can also illustrate the fault-tolerance features of TeamPlay \cite{LoevGrel20} that permit to easily run selected components under a variety of fault-tolerance regimes, such as double or triple modular redundancy.
Here, the students can experiment with how much redundancy we can afford on a given hardware platform before the application becomes unschedulable.
Moreover, the price in terms of energy consumption that needs to be paid for redundancy and fault-tolerance becomes evident.


\ignore{

  Lukas:

> = Sustrainable
> - Goal: play a bit with the coordination language. NFP and source code provided, slight changes to the configuration
>   - Needs: a container / VM with everything ready to go (Cecile on path, Yasmin, build-essentials)
>   - [!] this will be tricky on ARM macs (need an ARM mac)
>     -> solution: docker container with bash as start shell and all tools on path
> - Concrete plan:
>   - Application: fork-join esq graph where
>     Fork task reads wifi signal strength to a fixed number of students
>     Multiple process task locates each student
>     Join task makes some decision based on locations (e.g. 1.5m rule violations)
>   - Excercise: students get a monolithic version + code, nfp for parts of the code
>     Monolithic version is not schedulable
>     Students rewrite the task graph using SDF to enable hardware parallism
>     Run join task with fault tolerance (replicas = no. of replicas, (ignore) votingReplicas = no. of voters)
>     [!] Fix spelling mistake in homogenize (from homogeneise)
>     [!] Finish expand-ft-compiler-pass
> - Other:
>     [Clemens] will send slide deck regarding other instructions

After this I realized it actually needs to be about energy-aware scheduling -- the monolithic version should be schedulable as well, just use a lot of energy. But we can still take the same basic idea.
}

\section*{Acknowledgments}
\label{ack}

This project has received funding from the European Union's Horizon-2020 research and innovation programme under grant agreement No.~779882 (TeamPlay: Time, Energy and Security Analysis for Multi/Many-core Heterogeneous Platforms) and under grant agreement No.~871259 (ADMORPH: Towards Adaptively Morphing Embedded Systems).
Our work has received further funding from the European Union via the Erasmus Plus Key Action 2 (Strategic partnership for higher education) under grant agreement No.~2020-1-PT01-KA203-078646 (SusTrainable: Promoting Sustainability as a Fundamental Driver in Software Development Training and Education).
Last not least, this work has been supported by the COST Association through COST Action CA19135 (CERCIRAS: Connecting Education and Research Communities for an Innovative Resource Aware Society).

Special thanks go to all who have contributed to the design and implementation of the TeamPlay coordination language:
Julius Roeder,
Benjamin Rouxel,
Steven Swatman,
Wouter Loeve and
Sebastian Altmeyer.

%We thank the reviewers for their suggestions on improving this paper.

%\bibliography{bibliography}
%\bibliographystyle{splncs}
%\bibliographystyle{splncs04}

%\end{document}

\documentclass[twocolumn,hyperpdf,amsmath,amssymb,aps,prd,10pt,superscriptaddress,nofootinbib,noeprint,preprintnumbers,floatfix]{revtex4-2}

%% \usepackage[utf8]{inputenc} % allow utf-8 input
%\usepackage[T1]{fontenc}    % use 8-bit T1 fonts
%\usepackage{hyperref}       % hyperlinks
\usepackage{url}            % simple URL typesetting
\usepackage{booktabs}       % professional-quality tables
\usepackage{multirow}    
\usepackage{amsfonts}       % blackboard math symbols
\usepackage{nicefrac}       % compact symbols for 1/2, etc.
\usepackage{microtype}      % microtypogrhy
% \usepackage{natbib}
\usepackage{enumerate}
%\usepackage{enumitem}
\usepackage{hhline}
\usepackage{makecell}
\usepackage{pifont}

% use Times
%\usepackage{times}
% For figures
\usepackage{graphicx} % more modern
%\usepackage{epsfig} % less modern
%\usepackage{subfigure}
\usepackage{caption}
\usepackage{subcaption}
% For citations
\usepackage{amsmath}
\usepackage{amsthm}
\usepackage{amssymb}
\usepackage{tikz}
\usepackage{xcolor}
\usetikzlibrary{arrows}

\allowdisplaybreaks

%for fonts
\usepackage{mathrsfs}

% For algorithms
\usepackage{algorithm}
\usepackage{algorithmic}
% \usepackage{algpseudocode}
% \usepackage[noend]{algpseudocode}
\usepackage{hyperref}
\usepackage{bm}
%\usepackage{todonotes}

%For theorems
\allowdisplaybreaks

%for convinience
\newcommand{\RR}{\mathbb{R}}
\newcommand{\vct}{\boldsymbol }
%\newcommand{\mat}{\mathbf}
\newcommand{\rnd}{\mathsf}
\newcommand{\ud}{\mathrm d}
\newcommand{\nml}{\mathcal{N}}
\newcommand{\loss}{\mathcal{L}}
\newcommand{\hinge}{\mathcal{R}}
\newcommand{\kl}{\mathrm{KL}}
\newcommand{\cov}{\mathrm{cov}}
\newcommand{\dir}{\mathrm{Dir}}
\newcommand{\mult}{\mathrm{Mult}}
\newcommand{\err}{\mathrm{err}}
\newcommand{\sgn}{\mathrm{sgn}}
%\renewcommand{\span}{\mathrm{span}}
% \newcommand{\argmin}{\mathrm{argmin}}
% \newcommand{\argmax}{\mathrm{argmax}}
\newcommand{\poly}{\mathrm{poly}}
% \newcommand{\rank}{\mathrm{rank}}
% \newcommand{\conv}{\mathrm{conv}}
%\newcommand{\E}{\mathbb{E}}
% \newcommand{\diag}{\mat{diag}}
\newcommand{\acc}{\mathrm{acc}}

\newcommand{\labs}{\left\vert}
\newcommand{\rabs}{\right\vert}
\newcommand{\lnorm}{\left\Vert}
\newcommand{\rnorm}{\right\Vert}

\newcommand{\aff}{\mathrm{aff}}
% \newcommand{\range}{\mathrm{Range}}
\newcommand{\Sgn}{\mathrm{sign}}

\newcommand{\hit}{\mathrm{hit}}
\newcommand{\cross}{\mathrm{cross}}
\newcommand{\Left}{\mathrm{left}}
\newcommand{\Right}{\mathrm{right}}
\newcommand{\Mid}{\mathrm{mid}}
\newcommand{\bern}{\mathrm{Bernoulli}}
\newcommand{\ols}{\mathrm{ols}}
\newcommand{\tr}{\operatorname{tr}}
\newcommand{\opt}{\mathrm{opt}}
%\newcommand{\ridge}{\mathrm{ridge}}
\newcommand{\unif}{\mathrm{Unif}}
\newcommand{\Image}{\mathrm{im}}
\newcommand{\Kernel}{\mathrm{ker}}
\newcommand{\supp}{\mathrm{supp}}
\newcommand{\pred}{\mathrm{pred}}
\newcommand{\distequal}{\stackrel{\mathbf{P}}{=}}
%\newcommand{\gege}{\textcircled{1}}
\newcommand{\gege}{{A(\vect{w},\vect{w}_*)}}
\newcommand{\gele}{{A(\vect{w},-\vect{w}_*)}}
\newcommand{\lele}{{A(-\vect{w},-\vect{w}_*)}}
\newcommand{\lege}{{A(-\vect{w},\vect{w}_*)}}
\newcommand{\firstlayer}{\mathbf{W}}
\newcommand{\firstlayerWN}{v}
\newcommand{\secondlayer}{a}
\newcommand{\inputvar}{\vect{x}}
\newcommand{\anglemat}{\mathbf{\Phi}}
\newcommand{\holder}{H\"{o}lder }
\newcommand{\real}{\mathbb{R}}
\newcommand{\approxerr}{\delta}

\def\R{\mathbb{R}}
\def\Z{\mathbb{Z}}
\def\cA{\mathcal{A}}
\def\cB{\mathcal{B}}
\def\cD{\mathcal{D}}
\def\cE{\mathcal{E}}
\def\cF{\mathcal{F}}
\def\cG{\mathcal{G}}
\def\cH{\mathcal{H}}
\def\cS{\mathcal{S}}
\def\cI{\mathcal{I}}
\def\cL{\mathcal{L}}
\def\cM{\mathcal{M}}
\def\cN{\mathcal{N}}
\def\cP{\mathcal{P}}
\def\cS{\mathcal{S}}
\def\cT{\mathcal{T}}
\def\cV{\mathcal{V}}
\def\cW{\mathcal{W}}
\def\cZ{\mathcal{Z}}
\def\SS{\mathbb{S}}
\def\NN{\mathbb{N}}
\def\bP{\mathbf{P}}
\def\TV{\mathrm{TV}}
\def\MSE{\mathrm{MSE}}

\def\vw{\mathbf{w}}
\def\va{\mathbf{a}}
\def\vZ{\mathbf{Z}}

\newcommand{\mat}[1]{#1}
\newcommand{\vect}[1]{#1}
\newcommand{\norm}[1]{\left\|#1\right\|}
\newcommand{\normop}[1]{\left\|#1\right\|_{\mathrm{op}}}
\newcommand{\simplex}{\triangle}
\newcommand{\abs}[1]{\left|#1\right|}
\newcommand{\expect}{\mathbb{E}}
\newcommand{\prob}{\mathbb{P}}
\newcommand{\proj}{\gP}
% \newcommand{\prox}[2]{\textbf{Prox}_{#1}\left\{#2\right\}}
\newcommand{\event}[1]{\mathscr{#1}}
\newcommand{\set}[1]{#1}
\newcommand{\diff}{\text{d}}
\newcommand{\difference}{\triangle}
\newcommand{\inputdist}{\mathcal{Z}}
\newcommand{\indict}{\mathbb{I}}
\newcommand{\rotmat}{\mathbf{R}}
\newcommand{\normalize}[1]{\overline{#1}}
\newcommand{\vectorize}[1]{\text{vec}\left(#1\right)}
\newcommand{\vclass}{\mathcal{G}}
\newcommand{\pclass}{\Pi}
\newcommand{\qclass}{\mathcal{Q}}
\newcommand{\rclass}{\mathcal{R}}
\newcommand{\classComplexity}[2]{N_{class}(#1,#2)}
\newcommand{\cclass}{\mathcal{F}}
\newcommand{\gclass}{\mathcal{G}}
\newcommand{\pthres}{p_{thres}}
\newcommand{\ethres}{\epsilon_{thres}}
\newcommand{\eclass}{\epsilon_{class}}
\newcommand{\states}{\mathcal{S}}
\newcommand{\trans}{P}
\newcommand{\lowprobstate}{\psi}
\newcommand{\actions}{\mathcal{A}}
\newcommand{\contexts}{\mathcal{X}}
\newcommand{\edges}{\mathcal{E}}
\newcommand{\variance}{\text{Var}}
\newcommand{\params}{\vect{w}}

\newcommand{\relu}[1]{\sigma\left(#1\right)}
\newcommand{\reluder}[1]{\sigma'\left(#1\right)}
\newcommand{\act}[1]{\sigma\left(#1\right)}

\newtheorem{thm}{Theorem}
% \newtheorem{thm}{Theorem}
\newtheorem{lem}{Lemma}
% Thm -> corollary 
\newtheorem{cor}{Corollary}
\newtheorem{prop}{Proposition}
\newtheorem{asmp}{Assumption}
\newtheorem{defn}{Definition}
\newtheorem{oracle}{Oracle}
\newtheorem{fact}{Fact}
\newtheorem{conj}{Conjecture}
\newtheorem{rem}{Remark}
\newtheorem{example}{Example}
\newtheorem{condition}{Condition}
\newtheorem{exercise}{Exercise}
\newtheorem{mess}{Message}
\newtheorem{claim}{Claim}
\newtheorem{ec}{Empirical Conclusion}






\usepackage[capitalize,noabbrev]{cleveref}
% \usepackage{cleveref}
\crefname{thm}{Theorem}{Theorems}
\crefname{lem}{Lemma}{Lemmas}
\crefname{cor}{Corollary}{Corollaries}
\crefname{prop}{Proposition}{Propositions}
\crefname{asmp}{Assumption}{Assumptions}
\crefname{defn}{Definition}{Definitions}
\crefname{oracle}{Oracle}{Oracles}
\crefname{fact}{Fact}{Facts}
\crefname{conj}{Conjecture}{Conjectures}
\crefname{rem}{Remark}{Remarks}
\crefname{claim}{Claim}{Claims}
\crefname{ec}{Empirical Observation}{Empirical Observations}


\renewcommand{\algorithmicrequire}{\textbf{Input:}}
\renewcommand{\algorithmicensure}{\textbf{Output:}}


\definecolor{red}{rgb}{1, 0, 0}
\newcommand{\RED}[1]{{\color{red} #1}}

\definecolor{green}{rgb}{0, 1, 0}
\definecolor{darkgreen}{rgb}{0.0, 0.2, 0.13}
\definecolor{darkseagreen}{rgb}{0.56, 0.74, 0.56}
\definecolor{officegreen}{rgb}{0.0, 0.5, 0.0}


\newcommand{\GREEN}[1]{{\color{green} #1}}

\definecolor{blue}{rgb}{0, 0, 1}
\newcommand{\BLUE}[1]{{\color{blue} #1}}

\definecolor{orange}{rgb}{1, 0.4, 0.0}
\newcommand{\ORANGE}[1]{{\color{orange} #1}}


\usepackage{graphicx, color}
\usepackage[dvipsnames]{xcolor}

%% text %%
\usepackage[letterspace=-10]{microtype} 

%% math, tables %%
\usepackage{bm, amsmath, amsfonts, amssymb,xfrac}
\usepackage{multirow, tabularx, dcolumn}
\usepackage{mathtools, leftidx, braket, slashed, cancel, bigdelim}
\usepackage{blkarray}
\usepackage[figures]{rotating}

\usepackage{tikz}
\usetikzlibrary[plotmarks]
\usepackage{anyfontsize}

%% referencing %%
\usepackage[utf8]{inputenc} 
\usepackage{hyperref}
\pdfstringdefDisableCommands{ \renewcommand{\bm}[1]{#1} }

%% colors %%
\definecolor{jlab_red}{RGB}{192,39,45}
\definecolor{jlab_orange}{RGB}{249,102,0}
\definecolor{jlab_blue}{RGB}{47,122,121}
\definecolor{jlab_green}{RGB}{65,125,10}
\definecolor{jlab_gray}{gray}{0.6}
\definecolor{magenta}{rgb}{0.5, 0, 0.5}

\newcommand\bef{% Figure environment removed}
\newcommand\beq{\begin{equation}}
\newcommand\eeq[1]{\label{#1}\end{equation}}
\newcommand\beqa{\begin{eqnarray}}
\newcommand\eeqa[1]{\label{#1}\end{eqnarray}}
\newcommand\bet{\begin{table}}
\newcommand\eet[1]{\label{tb:#1}\end{table}}
\newcommand\fgn[1]{Figure \ref{fg:#1}} 
\newcommand\eqn[1]{Eq.\ (\ref{#1})}
\newcommand\tbn[1]{Table \ref{tb:#1}} 
\newcommand\pmn[1]{\textcolor{red}{#1}} 
\newcommand\pmnc[1]{\textcolor{red}{\it Comment: #1}}
\newcommand\nma[1]{\textcolor{blue}{#1}} 
\newcommand\nmac[1]{\textcolor{blue}{\it NM Comment: #1}}
\newcommand\arad[1]{\textcolor{green}{#1}} 
\newcommand\aradc[1]{\textcolor{green}{\it AR Comment: #1}}

%% editing macros %%
\newcommand{\cm}{\ensuremath{\mathsf{cm}}}


%% pdf hypertext links
\hypersetup{%
pdftitle = {title},
pdfsubject = {},
pdfkeywords = {},
%pdfauthor = {Hadron Spectrum Collaboration},
colorlinks = {true},
filecolor = {black},
linkcolor = {jlab_blue},
menucolor = {black},
citecolor = {jlab_green},
urlcolor = {jlab_green},
}{}

\begin{document}

%%%%%%%%%%%%%%%%%%%%%%%%%%%%%%%%%%%%%%%%%%%%%%%%%%%%%%%%%%%%%%%%%%%%%%
%\preprint{JLAB-THY-22-xxxx}
%
\title{Bound isoscalar axial-vector $bc\bar u\bar d$ tetraquark $T_{bc}$ in QCD}
%
\author{M. Padmanath}
\email{padmanath@imsc.res.in}
\affiliation{The Institute of Mathematical Sciences, a CI of Homi Bhabha National Institute, Chennai, 600113, India}
%
\author{Archana Radhakrishnan}
\email{archana.radhakrishnan@tifr.res.in}
\affiliation{Department of Theoretical Physics, Tata Institute of Fundamental Research, \\ Homi Bhabha Road, Mumbai 400005, India }

%
\author{Nilmani Mathur}
\email{nilmani@theory.tifr.res.in}
\affiliation{Department of Theoretical Physics, Tata Institute of Fundamental Research, \\ Homi Bhabha Road, Mumbai 400005, India }
%
% \collaboration{for the Hadron Spectrum Collaboration}
%

\preprint{IMSc/23/05, TIFR/TH/23-14}

\date{\today}
\begin{abstract}

The Fast Reciprocal Square Root Algorithm is a well-established approximation technique consisting of two stages: first, a coarse approximation is obtained by manipulating the bit pattern of the floating point argument using integer instructions, and second, the coarse result is refined through one or more steps, traditionally using Newtonian iteration but alternatively using improved expressions with carefully chosen numerical constants found by other authors. The algorithm was widely used before microprocessors carried built-in hardware support for computing reciprocal square roots. At the time of writing, however, there is in general no hardware acceleration for computing other fixed fractional powers. This paper generalises the algorithm to cater to all rational powers, and to support any polynomial degree(s) in the refinement step(s), and under the assumption of unlimited floating point precision provides a procedure which automatically constructs provably optimal constants in all of these cases. It is also shown that, under certain assumptions, the use of monic refinement polynomials yields results which are much better placed with respect to the cost/accuracy tradeoff than those obtained using general polynomials. Further extensions are also analysed, and several new best approximations are given.

\end{abstract}

\maketitle
%%%%%%%%%%%%%%%%%%%%%%%%%%%%%%%%%%%%%%%%%%%%%%%%%%%%%%%%%%%%%%%%%%%%%%%%%%%%%%%%%

% Figure environment removed

\section{Introduction}
Automatic 3D reconstruction of clothed humans using image inputs has gained increasing significance due to its potential applications in a wide array of AR/VR scenarios. High-fidelity reconstructions typically depend on sophisticated capture systems, which are developed with dense camera arrays~\cite{collet2015high,joo2015panoptic,joo2018total}, programmable light-stages~\cite{Vlasic2009, guo2019relightables}, and depth sensors~\cite{newcombe2011kinectfusion,DoubleFusion,BodyFusion,dou2016fusion4d,newcombe2015dynamicfusion}. However, stringent capture environments equipped with complex hardware pose significant challenges for consumer-level applications.


In this context, considerable research effort has been dedicated to developing methods that allow for more flexible capture configurations, such as utilizing a few RGB inputs. Among these works, learning implicit functions \cite{iccv2020PIFu, saito2020pifuhd, hong2021stereopifu} has proven effective in achieving highly detailed reconstructions by integrating the advancements of deep neural networks. These methods employ large multi-layer perceptrons (MLPs) to predict the occupancy probability or truncated signed distance function (TSDF) value of every queried 3D point based on its associated local feature, which is extracted from images. They can recover a continuous surface at arbitrary resolutions without topology restrictions.


However, in typical MLP-based implicit networks, the occupancy or TSDF value at each location is solved independently with planar image features, rendering them less capable of addressing challenging cases such as occlusions. Consequently, these methods suffer from generalization and robustness issues, particularly when tackling strong occlusions caused by large motion or multiple interacting humans. 
Some follow-up studies  \cite{zheng2021deepmulticap,zheng2021pamir,huang2020arch} utilize an extra geometric model, SMPL~\cite{Loper2015}, to improve robustness by introducing strong shape priors. 
Their success typically relies on the assumption of geometrical similarity \cite{huang2020arch} between the shape prior and target reconstruction, making them intractable for handling complex cases with loose clothes and sensitive to errors in SMPL model fitting.



%\ping{this paragraph sounds like `TSDF is better than MLP/SMPL, and we use TSDF to solve the problem'. But in Sec 3, we are telling a different story, saying `MLP needs a 3D convolutional encoder'. We need to make these two sections consistent.}\sicong{I think in this paragraph we claim that the TSDF}


%We opt for Trucated Signed Distance Funtion (TSDF) volumetric representations as they are naturally suitable for convolution operations, which have shown remarkable performance for learning hierarchical features on 2D visual perception tasks \cite{SunXLW19}. 
%Meanwhile, TSDF also describes the gradual geometry change around shape surface, which is not reflected by occupancy volume. 

We instead revisit the 3D volumetric representation and resort to 3D convolutional neural networks (CNNs) for feature learning, due to their impressive performance in feature learning and the ability to incorporate spatial context. However, volumetric methods and 3D convolution involve discretization, which might raise concerns regarding whether a discretized volume can preserve subtle geometric details as continuous representations learned in implicit functions. We investigate the relationship between volume resolution and quantization error on synthetic data by converting target mesh objects to TSDF volumes, as shown in Figure~\ref{fig:quantization_error}. We observe that the quantization errors are significantly reduced by increasing volume resolution and become nearly negligible when reaching a relatively high resolution (e.g., 512 or higher). In other words, achieving fine-detailed reconstruction is not supposed to be restricted by the use of volume representations as long as a proper volume resolution is utilized. Therefore, we present a method with high-resolution feature volumes, e.g., 256 and 512, while traditional volumetric methods \cite{varol18_bodynet,gilbert2018volumetric} are often limited to much lower resolutions, such as 32 or 128.



On the other hand, an increase in volume resolution may lead to a cubic growth of memory overhead \cite{8100085}. Reducing memory costs while guaranteeing the granularity of volumetric representations is necessary for pursuing high-quality reconstruction. Thus, we adopt a coarse-to-fine approach and cull away irrelevant voxels to build a sparse high-resolution feature volume. At the coarse level, the network computes an initial TSDF by applying a U-Net with sparse 3D CNN \cite{3DSemanticSegmentationWithSubmanifoldSparseConvNet} on the sparse feature volume, which is carved by a visual hull. Through our experiments, it turns out that more than 95\% of the volume grids are discarded by the visual hull culling, making the sparse 3D CNN efficient. At the fine level, the network focuses on a narrow band near the zero-level set of the initial TSDF and discretizes the narrow band with smaller voxels. By employing this narrow-band culling, we further shrink the sampling space, resulting in a relatively small range of grid numbers (usually 300K--500K in our experiments) even with a high volume resolution of 512. The remaining voxels in the narrow band are associated with features that fuse high-frequency information from the computed normal maps upon the low-frequency shape from the coarse level to compute the TSDF at high resolution. The final mesh is then extracted from the TSDF using the Marching-Cube algorithm ~\cite{Lorensen87marchingcubes}.
% Different from the u-net sturcture to preserve global topology context, we then apply a shallow 3dcnn to compute the final TSDF $D_{final}$ which contain more local geometry detail.




% \ping{this paragraph can be expanded. It is an important contribution and often ignored by other works. stress on the novel idea of regressing blending weights instead of colors}

In addition to geometry, high-quality mesh texture is also a crucial factor contributing to visual appearance. Directly computing a color field in 3D space, as in \cite{iccv2020PIFu}, struggles to capture high-frequency texture details, while the neural radiance field (NeRF) \cite{yu2020pixelnerf} or the DoubleField~\cite{shao2022doublefield} require expensive per-instance optimization and are often unstable for sparse input images. In contrast, we adopt an image-based rendering approach to compute a texture atlas map, which is efficient and widely supported in existing computer graphics tools. 
Specifically, we compute a blending weight at each 3D point on the mesh surface to determine its color as a weighted average of the colors at its image projections. The blending weights can be computed at a relatively coarse resolution, e.g., 512 volume resolution in our case, and leave texture details to the high-resolution images, such as 1K or 2K. Unlike previous methods that generate blurry texturing results under sparse input, our method generalizes well on both synthetic and real data with just a few input views. 
Figure~\ref{fig:teaser} shows two examples reconstructed by our method. Despite the challenging garment, pose, and occlusion, our method recovers faithful shape, normal, and texture on the right.

%with a wide variety of poses and clothing styles, and it is also adaptive to handle input image with arbitrary resolutions.
%\sicong{For this concern we claim that when the resolution of dicretized volume meets certain threshold (which is 256 in our experiment), the quantization error can be neglected.} 



In summary, the main contributions of this paper are as follows:
\begin{itemize}
\vspace{-0.1in}
  \item 
  We revisit the 3D volumetric representation and demonstrate that it can support clothed human reconstruction with equal or even better performance compared to implicit representation. 
  \item 
  We develop a memory and computation-efficient method for high-resolution volumetric reconstruction using sophisticated sparse 3D CNN, coarse-to-fine estimation, and voxel culling by visual hull and narrow bands. 
  \item 
  We introduce a novel method to compute a texture atlas map, which captures rich appearance details from high-resolution input images.
  \item 
  We achieve impressive results on standard benchmark datasets Twindom and MultiHuman, significantly reducing the point-2-surface (P2S) precision to approximately 0.2cm from just six input views, with more than $50\%$ error reduction compared to the state-of-the-art methods, including DoubleField~\cite{shao2022doublefield} and PIFuHD~\cite{saito2020pifuhd}.
\end{itemize}
\section{Ensembles and fermion actions}\label{sec:lattice}

We use the same computational setup as in several of our previous publications \cite{Junnarkar:2019equ,
Junnarkar:2018twb,Basak:2014kma,Padmanath:2017lng,Basak:2012py,Basak:2013oya,Mathur:2016hsm,
Mathur:2018epb,Mathur:2018rwu,Junnarkar:2022yak,Mathur:2022ovu}, which we briefly summarize below for completeness. Four 
$N_f=2+1+1$ lattice QCD ensembles generated by the MILC collaboration are used in this study \cite{MILC:2012znn}, where
the dynamical quark flavors were simulated using Highly Improved Staggered Quark (HISQ) action on gauge fields 
that respect one-loop, tadpole-improved Symanzik gauge action with tuned coefficients through 
$\mathcal{O}(\alpha_sa^2, n_f\alpha_sa^2)$ \cite{Follana:2006rc}. The charm and strange quark masses are 
tuned to their respective physical values, whereas the dynamical light quarks are chosen such 
that $m_s/m_l\sim 5$. We list the relevant details of various lattice QCD ensembles used in \tbn{lattice}.

\bet[tbh]
  \begin{center}
	  \begin{tabular}{p{1.5cm}p{1.5cm}p{1.5cm}>{\hfill\arraybackslash}p{1.5cm}}
      \hline
Label & Symbol & $a~[fm]$     & $N_s^3\times N_t$ \\ \hline
$S_1$ & \pmb{\textcolor{red}{\tikz{\pgfsetplotmarksize{0.8ex}\pgfuseplotmark{diamond}}}} & 0.1207(11)   & $24^3\times64$ \\
$S_2$ & \pmb{\textcolor{magenta}{\tikz{\pgfsetplotmarksize{0.8ex}\pgfuseplotmark{pentagon}}}} & 0.0888(8)    & $32^3\times96$ \\
$S_3$ & \pmb{\textcolor{blue}{\tikz{\pgfsetplotmarksize{0.7ex}\pgfuseplotmark{o}}}} & 0.0582(4)    & $48^3\times144$ \\
$L_1$ & \pmb{\textcolor{OliveGreen}{\pgfsetplotmarksize{0.7ex}\tikz{\pgfuseplotmark{square}}}} & 0.1189(9)    & $40^3\times64$ \\   \hline
  \end{tabular}
  \end{center}
\caption{Relevant details of the lattice QCD ensembles used. The lattice spacing estimates 
are measured using the $r_1$ parameter \cite{MILC:2012znn}. $L$ in $L_1$ refers to large spatial volume, 
and $S$ in $S_1,~S_2$, and $S_3$ refer to small spatial volume. }
\eet{lattice}

The valence quark fields for the light, strange and charm flavors are realized using an overlap 
fermion action that is $\mathcal{O}(am)$ improved. To this end, we utilize the numerical 
implementation of the overlap action following Refs. \cite{Chen:2003im,xQCD:2010pnl}. Following 
the Fermilab prescription \cite{El-Khadra:1996wdx}, the bare charm quark mass on each ensemble was tuned 
using the kinetic mass of spin averaged $1S$ charmonia $\{a\overline M_{kin}^{\bar cc} = 0.75 aM_{kin}(J/\psi) + 0.25 aM_{kin}(\eta_c)\}$
determined for the respective ensembles. Further details on the tuning of charm quark mass, 
the tuned bare quark mass, and resulting discretization effects are discussed in Refs. \cite{Basak:2012py,Basak:2013oya}.
The bare strange quark mass is set by equating the lattice estimate for the fictitious pseudoscalar $\bar ss$ 
meson mass to 688.5 MeV \cite{Chakraborty:2014aca}. Additionally, we perform the quark propagator 
measurements in the valence sector using overlap fermion action for three other quark masses in 
all the ensembles corresponding to pseudoscalar masses of approximately 0.5, 0.6 and 1.0 GeV. 

We employ a nonrelativistic QCD (NRQCD) Hamiltonian \cite{Lepage:1992tx} for the bottom quark. 
We tuned the bottom quark mass using the Fermilab prescription \cite{El-Khadra:1996wdx}, by equating 
the lattice extracted kinetic mass of the spin averaged 1S bottomonia $\{\overline M_{kin}^{\bar bb} = 0.75 M_{kin}(\Upsilon) + 0.25 M_{kin}(\eta_b)\}$
to its experimental value, where the kinetic mass is evaluated from the dispersion relation 
$aM_{kin}^2 = ((ap)^2 - (a\Delta E)^2)/2a\Delta E$. The details of NRQCD Hamiltonian, the improvement 
coefficients, and bottom quark mass tuning on our setup are discussed in Ref. \cite{Mathur:2016hsm}.

\bef[tbh]
% Figure removed
\caption{A landscape plot of the pseudoscalar masses corresponding to the quark mass that we have utilized 
in this work for different lattice ensembles used. The horizontal gray bands indicate a representative 
$M_{ps}$ estimate to guide the eye for a similar pseudoscalar meson mass across all four ensembles.} 
\eef{mpiVslat}

In this work, we assume isospin symmetry ($m_u = m_d$), and then for the channel that study here, 
involves three quark masses: the bottom ($b$), the charm ($c$), and the light ($u/d$) quarks.
For the light quark mass, we investigate five 
different cases: three unphysical quark masses discussed above [referred in terms of the 
corresponding approximate pseudoscalar meson masses $M_{ps}\sim$0.5, 0.6, and 1.0 GeV], the 
strange quark mass [$M_{ps}\sim$0.7 GeV] and the charm quark mass [$M_{ps}\sim$3.0 GeV]. In 
\fgn{mpiVslat}, we present the landscape of the five light quark masses studied in terms of the 
corresponding $M_{ps}$ versus the ensembles used. Using this setup, we evaluate the finite-volume spectrum in the 
isoscalar axialvector channel with $bc\bar u\bar d$ flavor for all these five quark masses on all 
four ensembles, next investigate the scattering of $D$ and $B^*$ mesons in all five scenarios and then 
extract the $m_{u/d}$ (otherwise $M_{ps}$) dependence of the scattering parameters. We utilize a wall-smearing procedure for all 
our quark propagator measurements (see Refs. \cite{Mathur:2018epb,Junnarkar:2018twb,Mathur:2022ovu} 
for details), and our primary focus on the finite-volume spectrum is on the ground state in each case. 






\section{Measurements and interpolators}\label{sec:2ptIO}

Lattice determination of finite-volume spectrum follows through an evaluation or measurement of Euclidean 
two-point correlation functions $\mathcal{C}_{ij}(t)$, of interpolating operators 
$\mathcal{O}_i(\mathbf{x},t)$ with desired quantum numbers, given by
\beq
\mathcal{C}_{ij}(t) = \sum_{\mathbf{x}}\left<\mathcal{O}_i(\mathbf{x},t)\mathcal{O}_j^{\dagger}(\mathbf{0},0)\right> = \sum_n \frac{Z_i^nZ_j^{n\dagger}}{2E^n} e^{-E^nt}.
\eeq{c2pt}
Here the second equality suggests that $\mathcal{C}_{ij}(t)$ can be expressed as a sum of exponentials 
following a spectral decomposition. $Z_i^n = \bra{0}\mathcal{O}_i\ket{n}$ is the operator-state overlap that 
quantifies the efficacy of the interpolator $\mathcal{O}_i$ in determining the time evolution of the state $n$. 
The utilization of wall smearing for the quark sources effectively kills all the high-momentum modes
at the source, whereas a zero momentum projection at the sink time slice ($\sum_{\mathbf{x}}$), as shown
in \eqn{c2pt}, efficiently projects the correlation function to the rest frame.  

Our main focus is on the ground state in the $T_{1}^+$ irreducible representation (irrep) in the rest frame, 
which is the only relevant rest frame finite-volume irrep for studying states in the infinite-volume continuum 
with quantum numbers ($J^P = 1^+$). To this end, we use a similar set of operators in the $T_{1}^+$ irrep as 
was utilized in Ref. \cite{Francis:2018jyb} and we briefly discuss them below for completeness. Assuming isospin symmetry, 
the relevant low-lying two-meson thresholds in the order of increasing energy are $E_{DB^*} = M_{B^*}+M_{D}$, 
$E_{BD^*}=M_B+M_{D^*}$, and $E_{D^*B^*}=M_B^*{}+M_{D^*}$. Hence, we consider the following low-lying two-meson 
interpolators 
\beqa
\mathcal{O}_1(x) &=& [\bar u(x) \gamma_i b(x)][\bar d(x) \gamma_5 c(x)]  \nonumber \\&& - [\bar d(x) \gamma_i b(x)][\bar u(x) \gamma_5 c(x)] \nonumber \\
\mathcal{O}_2(x) &=& [\bar u(x) \gamma_5 b(x)][\bar d(x) \gamma_i c(x)]  \nonumber \\&& - [\bar d(x) \gamma_5 b(x)][\bar u(x) \gamma_i c(x)] \nonumber \\
\mathcal{O}'(x) &=& \epsilon_{ijk} [\bar u(x) \gamma_i b(x)][\bar d(x) \gamma_j c(x)] \nonumber \\&& - [\bar d(x) \gamma_i b(x)][\bar u(x) \gamma_j c(x)].
\eeqa{mmops}
We utilize $\mathcal{O}_1(x)$ and $\mathcal{O}_2(x)$ in the computation of correlation functions. 
$\mathcal{O}'(x)$ has its associated two-meson threshold sufficiently higher up in the energy 
spectrum compared to the other two thresholds and it was found to have no effects in the low-lying 
energy spectrum. Hence we disregard this operator
from the rest of our analysis. Note that the lowest three particle threshold $DB\pi$ is above $E_{BD^*}$
for all the considered heavier-than-physical light quark masses. At $m_{u/d}^{phys}$, the $BD\pi$ threshold is 
immediately below $E_{BD^*}$, yet it remains sufficiently above $E_{DB^*}$ 
to have any significant effects on the ground states that we extract. We also compute two-point 
correlation functions for $B$, $B^*$, $D$, and $D^*$ mesons, using standard local quark 
bilinear interpolators ($\overline Q~\Gamma~q$) with spin structures $\Gamma\sim\gamma_5$ and 
$\gamma_i$ for pseudoscalar and vector quantum numbers, respectively. 

Phenomenologically, doubly bottom tetraquark in the axialvector channel is expected to be deeply 
bound. Such a state is expected to be quite compact owing to its doubly heavy flavor content 
and deeply bound nature \cite{Francis:2016hui,Czarnecki:2017vco}. Consequently, a local 
diquark-antidiquark interpolator is naturally interesting. Such an operator has already 
been utilized in all lattice QCD studies of the doubly bottom as well as bottom charm tetraquarks 
in the past \cite{Bicudo:2015kna,Francis:2016hui,Bicudo:2017szl,Junnarkar:2018twb,Leskovec:2019ioa,
Francis:2018jyb,Hudspith:2020tdf,Meinel:2022lzo,Hudspith:2023loy} and we follow the same strategy. 
Along with operators in \eqn{mmops}, we employ a local diquark-antidiquark interpolator 
\beq
\mathcal{O}_3(x) = (\bar u(x)^T \Gamma_5 \bar d(x) - \bar d(x)^T \Gamma_5 \bar u(x))( b(x) \Gamma_i c(x)),
\eeq{dadops}
where $\Gamma_k = C\gamma_k$ with $C=i\gamma_y\gamma_t$ being the charge conjugation matrix and 
the diquarks (antidiquarks) in the color antitriplet (triplet) representations. 

Our final basis is composed of the above-mentioned three interpolators $\{\mathcal{O}_1(x), \mathcal{O}_2(x), \mathcal{O}_3(x)\}$, 
which is diverse enough to reliably determine the ground state in the energy spectra that we are interested in.. Using this basis we determine 
the correlation matrices, with elements evaluated as prescribed in \eqn{c2pt}. Then the correlation matrices $\mathcal{C}$
are analyzed following a variational approach \cite{Michael:1985ne} to determine the energy estimates for 
low-lying levels in the spectrum. In this procedure, we look for the solutions of the generalized eigenvalue 
problem (GEVP) given by 
\beq
\mathcal{C}(t)v^n(t) = \lambda^n(t) \mathcal{C}(t_0)v^n(t),
\eeq{gevp}
where $t_0$ is a reference timeslice at which the eigenvalues $\lambda^n$s are identically unity. 
\bef[h]
% Figure removed
\caption{Effective energy plot for the eigenvalue correlation function $\lambda^0(t)$ (square) and for the product 
of single-meson correlators (circle) representing the noninteracting two-meson correlation function 
($\mathcal{C}_{D}(t)\mathcal{C}_{B^*}(t)$). The data correspond to $M_{ps} \sim 700$ MeV in the finest ensemble. 
The bands shown are the energy fit estimates for the final chosen time intervals.}
\eef{effmass}
The eigensolutions in the large time limit represent the lowest $N$ eigenstates $E^n$, for which the time 
evolution is dictated by the eigenvalues as $\lim_{t\to\infty}\lambda^n(t) \sim A_ne^{-E^nt}$. 
The corresponding eigenvectors are represented by $v^n(t)$, which are related to the operator-state-overlaps as
\beq
Z_i^{n}=\bra{0}\mathcal{O}_i \ket{n} = \sqrt{2E^n}(V^{-1})_i^n e^{E^{n}(t_0)/2},
\eeq{overlaps}
where $V$ is a matrix built out of $v^n(t)$. $v^n(t)$ is expected to be time independent in the limit, 
where the signal in $\mathcal{C}$ is saturated by the lowest $N$ eigenstates of the system.  

Conventionally the signal in the two point correlator data $C(t)$ is first assessed based on the 
large time plateauing in effective energies defined as $aE_{eff} = [ln(C(t)/C(t+\delta t))]/\delta t$. In 
\fgn{effmass}, we present the effective energies as a function of time for the eigenvalue correlation 
function (squares) and the noninteracting two-meson ($\mathcal{C}_{D}(t)\mathcal{C}_{B^*}(t)$) 
correlation function (circles). These effective energies can be seen to saturate around timeslices 
24 to 28 in the example shown. The results presented correspond to the lowest eigenvalue correlator 
$\lambda^0(t)$ at the strange quark mass ($M_{ps}\sim0.7$ GeV) in the finest ensemble we study. 
Evidently, there is a negative shift in the energies in $\lambda^0(t)$ with respect to the 
noninteracting energies at all times, except at very large times where the signal-to-noise ratio 
degrades substantially. 

Extraction of the energy spectra proceeds via fitting the eigenvalue correlators, $\lambda_{n}(t)$, 
with the expected asymptotic exponential behaviour. Alternatively, one can fit the asymptotic time 
estimates for the ratio of correlators given by 
\beq
R^n(t)=\frac{\lambda^n(t)}{\mathcal{C}_{m_1}(t) \mathcal{C}_{m_2}(t)}, 
\eeq{ratio}
to a single exponential form ($Ae^{-\Delta E^nt}$), where $\Delta E^n$ is expected to saturate to 
$E^n-M_{m_1}-M_{m_2}$ at large times. Here, $\mathcal{C}_{m_i}$ is the correlation function for 
the meson $m_i$, and $M_{m_i}$ is its mass. Being a ratio, $R^n(t)$ is empirically known to efficiently 
mitigate correlated noise between the product of two meson correlators and the interacting correlator 
for the two-meson system \cite{Green:2021qol}. Note that the automatic cancellation of the additive 
mass renormalization, inherent to NRQCD formulation, is an added advantage in using \eqn{ratio} for 
the fits. In \fgn{fitcompare}, we present a representative plot showing the $t_{min}$ dependence of 
the $\Delta E^n$ fit estimates determined from the fits to $\lambda^n(t)$ and $R^n(t)$, respectively, 
where $t_{min}$ is the lower boundary of the time interval used for these fits for a fixed upper boundary 
timeslice for the time interval. The energy differences are evaluated from $\lambda^n(t)$ using the relation 
$\Delta E^n = E^n-M_{m_1}-M_{m_2}$, where $M_{m_1}$ and $M_{m_2}$ are mass estimates for individual 
mesons determined from separate fits to $\mathcal{C}_{m_1}(t)$ and $\mathcal{C}_{m_2}(t)$, respectively. 
The estimates from different procedures can be seen to agree asymptotically in time, based on which 
optimal $t_{min}$ values are chosen. Our final results are based on fitting the ratio correlators 
defined in \eqn{ratio}.

    
\bef[h]
% Figure removed
\caption{$t_{min}$ dependence of the $\Delta E^0$ fit estimates determined from the fits to $\lambda^0$
and $R^0(t)$ for the case $M_{ps} \sim 700$ MeV in the finest ensemble. Here the superscript 0 refers 
to the ground state. }
\eef{fitcompare}



%%%%%%%%%%%%%%%%%%%%%%%%%%%%%%%%%%%%%%%%%%%%%%%%%%%%%%%%%%%%%%
\section{Energy spectra in finite-volume}\label{fvresults}
%%%%%%%%%%%%%%%%%%%%%%%%%%%%%%%%%%%%%%%%%%%%%%%%%%%%%%%%%%%%%%
In this section, we present our results that we obtain from the finite-volume correlators. 
After presenting the energy spectrum extracted using variational techniques, we discuss 
the operator-state-overlaps and the operator basis dependence. In the final subsection, 
we describe our strategy for rebuilding the ground state energies that are corrected for 
the additive NRQCD offset and for using them in further amplitude fits. 

\subsection{Details of energy spectra}
% Figure environment removed
In \fgn{spectrum}, we present the finite-volume energy spectra of the isoscalar 
axialvector $bc{\bar{u}}{\bar{d}}$ channel that we extract on the four ensembles 
listed in \tbn{lattice}, at the five different $m_{u/d}$ values corresponding to 
$M_{ps}\sim$ 0.5, 0.6, 0.7, 1.0, and 3.0 GeV. The energy spectrum is shown in 
lattice units. Note that these levels are shown with unaccounted additive 
renormalization measures related to the NRQCD-based dynamics of the heavy bottom quarks. 
The noninteracting two-meson energy levels corresponding to $DB^*$ and $BD^*$ thresholds 
are indicated as dotted horizontal line segments for each lattice and each $M_{ps}$. 
A clear trend for negative energy shifts can be observed in all the cases, indicating 
a possible attractive interaction between the scattering particles involved \cite{scalarbc}.
The $B^*D^*$ threshold in each case is also shown in the figure by dashed lines. 

From the energy spectra in the lattices $L_1$, $S_2$ and $S_3$, it can be observed 
that a consistent pattern emerges with respect to the two-meson thresholds.
The relative positioning of the ground state energy with the elastic threshold in the 
$S_1$ ensemble is also consistent with the other three ensembles. This is an encouraging 
feature in the finite-volume spectrum, as our main interest is on reliable extraction of the 
ground state energies. It is this ground state energy from each ensemble that we 
later on employ to constrain the $DB^*$ scattering amplitude.  

The excited states in the $S_1$ ensemble for $M_{ps}$, other than at the charm point, 
indicate enhanced negative shifts compared to that on the other ensembles. This could be 
related to a combination of effects arising from various less attractive features of the 
$S_1$ lattice, which includes the coarsest lattice spacing, small spatial volume and 
possible insufficient statistics for the study at lighter $M_{ps}$. To this end, we 
perform two additional checks. First, we make an associated study of the $S_1$ and the 
$L_1$ ensembles at the level of variational analysis and fitting procedures to determine 
the low-lying spectra with emphasis on the ground and the first excited states. We discuss 
this part of the investigation in Appendix \ref{app:S1L1}. Secondly, we perform amplitude 
fits with and without results from the $S_1$ ensemble to verify the robustness in our estimates 
for the scattering length. We discuss this in detail in Section \ref{Ampfits}.

\subsection{Operator-state overlaps}\label{sec:OSO}
\bef[hbt!]
% Figure removed
\caption{Normalized operator-state overlaps $\tilde{Z}_i^n$ for a state indicated by $n={0, 1, 2}$ 
and an operator represented by $\mathcal{O}_i$, where $i={1, 2, 3}$ on the $L_1$ ensemble. 
The errors in the normalized overlap factors are smaller than the symbols and hence are 
suppressed. The five horizontal panes stand for the five different $M_{ps}$ values we 
investigate. The two vertical lines in each horizontal pane separate $\tilde{Z}_i^n$ for 
different operators $\mathcal{O}_i$. }
\eef{Zratiosl40}
Now we investigate the operator-state overlaps $Z_i^n$, as in \eqn{overlaps}, to evaluate 
the efficacy of the interpolators in determining the low-lying spectra. To this end, 
we define normalized operator-state overlaps $\tilde{Z}_i^n$ such that its largest value 
for any given operator $\mathcal{O}_i$ across all the states $\{n\}$ is unity \cite{Dudek:2009qf,Padmanath:2013zfa}.
$\tilde{Z}_i^n$ quantifies the relative relevance of any given operator across all the 
states. In \fgn{Zratiosl40}, we present $\tilde{Z}_i^n$ at all $M_{ps}$ values we have used
on the $L_1$ ensemble. Each square marker corresponds to the $\tilde{Z}_i^n$ for a given operator 
$\mathcal{O}_i$ on to a given state $n$. Each horizontal pane stands for an $M_{ps}$ indicated on the 
right-hand side, whereas the vertical lines in each horizontal pane part $\tilde{Z}_i^n$ 
for different operators indicated on the top pane. The $x$-axis ticks refer to the three low-lying 
states we have extracted. $\mathcal{O}_1$, the two-meson operator related to $DB^*$ threshold, 
can be seen to have the largest overlap with the ground state and has significantly small 
overlaps with the excited states. $\mathcal{O}_2$, the two-meson operator related to $BD^*$ 
threshold, has the largest overlap with the first excited state and a very small overlap with 
the ground state. $\mathcal{O}_2$ also have nonnegligible overlap factors with the second 
excited state indicating $BD^*$-type two-meson Fock component, which decreases with increasing 
$M_{ps}$. On the other hand, $\mathcal{O}_3$, the diquark-antidiquark type operator, have 
substantial overlap factors with all states at the two lightest $M_{ps}$ values, whereas with 
an increased $M_{ps}$ its largest overlap is with the second excited state. Note that 
$\mathcal{O}_3$ is Fierz related to two-meson interpolators \cite{Padmanath:2015era}, and 
the large $\tilde{Z}_3^n$ values of $\mathcal{O}_3$ for all $n$ could be related to this 
underlying connection between two-meson and diquark-antidiquark operators. 


A summary from the above observations on overlap factors is as follows. $\mathcal{O}_1$
predominantly determines the ground state, whereas it has a significantly small coupling with 
the excited states. Similar patterns of overlap factors are also observed for other ensembles, 
all of which indicate that $\mathcal{O}_1$ predominantly determines the ground state. 
The two excited states have strong two-meson and diquark-antidiquark Fock components 
in the two lightest $M_{ps}$ values. The two-meson Fock components in the second excited state 
and the diquark-antidiquark Fock components in the first excited state decreases with 
increasing $M_{ps}$. This is consistent with the phenomenological expectation, which suggests 
that the binding energy in doubly heavy tetraquarks increases with increasing 
relative heaviness for the heavy quarks with respect to the light quarks \cite{Francis:2016hui,
Czarnecki:2017vco,Junnarkar:2018twb}. A deeply bound state could be significantly compact 
and hence could have large Fock components of a compact object such as that of a 
diquark-antidiquark. In other words, the relevance of compact diquark-antiquark operators for 
the low-lying spectrum increases with decreasing light quark mass, as is evident from \fgn{Zratiosl40}. 

\subsection{Operator basis dependence}
\bef[tbh!]
% Figure removed
\caption{Operator basis dependence of the low energy spectra of the $L_1$ ensemble and 
$M_{ps}\sim$700 MeV for all possible operator basis that can be built out of the three operators 
discussed in Section \ref{sec:2ptIO}. The basis is presented in digital notation ($x$-axis tick labels) 
where the operators are arranged in the order $\{\mathcal{O}_1, \mathcal{O}_2, \mathcal{O}_3\}$.
The horizontal lines refer to the $DB^*$, $BD^*$, and $D^*B^*$ thresholds. The bands indicate the 
bounds of the ground and first excited state energy estimates from the full three-operator basis. }
\eef{basisdep}
Next, we look into the basis dependence of the finite-volume energy spectra presented in 
\fgn{spectrum}. In \fgn{basisdep}, we show this basis dependence as determined for $M_{ps}\sim$ 
700 MeV in the $L_1$ ensemble, for various operator basis build out of $\mathcal{O}_1$, 
$\mathcal{O}_2$, and $\mathcal{O}_3$ operators as defined in \eqn{mmops} and \eqn{dadops}. The digital 
indexing on the $x$-axis tick labels refers to various operator basis in the order 
$\{\mathcal{O}_1, \mathcal{O}_2, \mathcal{O}_3\}$, with an overline on the third index 
as a visual aid within the plot to highlight the diquark-antidiqaurk interpolator. 1 (0) 
indicates an operator is included in (excluded from) the basis. The horizontal 
lines refer to the $DB^*$, $BD^*$ and $B^*D^*$ thresholds. The gray horizontal bands 
refer to the two lowest levels in the full basis indicated by $11\overline{1}$. A level 
below the threshold appears only when $\mathcal{O}_1$ is present in the basis. The first 
excited state in the full basis $11\overline{1}$ is faithfully reproduced in those bases 
where $\mathcal{O}_2$ is included. $\mathcal{O}_3$ alone does not precisely determine 
any level in the energy spectrum using full basis. Similar observations are also made 
on other ensembles. 

In summary, the ground state in the full basis $11\overline{1}$ is reliably determined 
with $\mathcal{O}_1$ and is unaffected by the inclusion of $\mathcal{O}_2$ and 
$\mathcal{O}_3$ operators. The excited states have nonnegligible overlap factors with 
$\mathcal{O}_2$ and $\mathcal{O}_3$ operators. Given our setup with only a few energy 
levels, any assumption more complicated than a simple elastic $DB^*$ assumption for 
the amplitude fits is beyond the scope of this work. Such an assumption is justified within the 
isosymmetric limit as the lowest inelastic threshold ($BD^*$ at unphysically heavy 
$m_{u/d}$ or $BD\pi$ for $m_{u/d}^{phys}$) is significantly high. In light of all 
these observations above, we limit ourselves to using only ground states determined 
from all ensembles at various $M_{ps}$ values to constrain the elastic $S$-wave
$DB^*$ scattering amplitude.

\subsection{The ground state energies}

\begin{figure}[h]
% Figure removed
\caption{The ground state energies in units of the elastic threshold ($DB^*$) on all 
ensembles (see \tbn{lattice} for color-symbol conventions) for all $M_{ps}$ values
(different vertical panes).}
\eef{gsspectrum}

In this subsection, we discuss how we obtain ground state energies after adjusting the 
additive correction that is inherent to an NRQCD calculation. The set of numbers that 
we extract from our variational analysis and eigenvalue correlator fitting procedures 
are the single meson masses $M_{B}$, $M_{B^*}$, $M_{D}$, and $M_{D^*}$ and the energy 
splittings $\Delta E_n = E_n - M_{m_1} - M_{m_2}$ (see \eqn{ratio}). %for the interacting system from the reference 
%two meson level $m_1m_2$ determined from the ratio of correlators defined in \eqn{ratio}. 
First, we account for the NRQCD corrections in single meson masses (involving a bottom quark) as 
\beq
\tilde{M}_{B^{(*)}} = M_{B^{(*)}} - 0.5\overline M^{\bar bb}_{lat} + 0.5 \overline M^{\bar bb}_{phys},
\eeq{msnNRcor}
where $\overline M^{\bar bb}_{lat}(\overline M^{\bar bb}_{phys})$ refers to the spin averaged mass 
of the $1S$ bottomonium measured on the lattice (experiments). We follow this procedure as we 
have tuned the bottom quark mass through the spin average bottomonia at each ensemble.

For the interacting energy spectrum, the NRQCD offset is automatically canceled in the energy splittings
$\Delta E^n$. One can then build the energy estimates $\tilde{E}^n$ of interacting spectrum by adding 
the noninteracting level energy ($M_{m_1} + M_{m_2}$) with the energy splittings $\Delta E^n$ as,
\beq 
\tilde{E}^n = \Delta E^n + M_{m_1} + M_{m_2}.
\eeq{intNRcor}
If either $m_1$ or $m_2$ is a bottom meson, we use the corresponding corrected $\tilde{M}_{m_i}$\footnote{From 
the next section, for brevity we suppress the $\tilde{}$ notation indicating corrected masses and energies.} 
determined using \eqn{msnNRcor}, instead of $M_{m_i}$. 

In \fgn{gsspectrum}, we present the corrected ground state energy estimates, 
in units of the energy of elastic threshold $E_{DB^*}$, at various $M_{ps}$ and 
for all the ensembles we have employed. The spectrum clearly shows a trend of decreasing 
energy spitting, hence decreasing interaction strength, with increasing $M_{ps}$. 
Another feature worth noting here is that the lattice spacing dependence of the 
ground state energies on similar volume ensembles ($S_1$, $S_2$, $S_3$) for the 
non-charm $M_{\pi}$ are opposite to that at the charm point. We will revisit this 
point when we discuss extraction of $DB^*$ scattering amplitude using these energy levels. 



%%%%%%%%%%%%%%%%%%%%%%%%%%%%%%%%%%%%%%%%%%%%%%%%%%%%%%%%%%%%%%
\section{$\mathbf{DB^*}$ scattering amplitude}\label{Ampfits}
%%%%%%%%%%%%%%%%%%%%%%%%%%%%%%%%%%%%%%%%%%%%%%%%%%%%%%%%%%%%%%

\subsection{Strategy}

The finite-volume energy splittings determined in the previous section are related to 
the infinite-volume scattering physics via L\"uscher's finite-volume prescription 
\cite{Luscher:1990ux} and its generalizations, e.g. \cite{Briceno:2014oea}. Assuming 
these energy splittings are purely described by an elastic scattering in the $DB^*$ 
system, we utilize them to constrain the associated $S$-wave scattering amplitude. Here 
we consider only the ground states in all ensembles for all quark mass scenarios, as 
the excited states are found to be affected by the inelastic $BD^*$ channel. 

It is interesting that even the excited states are also found to have statistically 
significant shifts with respect to the inelastic $BD^*$ threshold (see \fgn{spectrum}), 
possibly indicating nontrivial interactions between $B$ and $D^*$ mesons. If the $DB^*$ 
and $BD^*$ channels were totally decoupled, such shifts point to equivalent interactions 
in both channels \cite{bdsbc}. However, independent elastic analysis for the excited 
states is not well justified. On the other hand, the inclusion of excited states in our 
analysis demands an inelastic treatment involving more parameters than the available 
degrees of freedom in the amplitude fits, which is beyond the scope of this work. We 
also assume only negligible effects from higher partial waves or any off-shell pion 
exchange interactions that can induce coupling between $DB^*$ and $BD^*$ channels 
\cite{Du:2023hlu}, for the same reason. 

\subsection{Amplitude fits and continuum extrapolations}
For the scattering of a $D$ and a $B^*$ meson in the $S$-wave leading to total angular 
momentum and parity $J^P=1^+$, the scattering phase shifts $\delta_{l=0}(k)$ are related to 
the finite-volume energy spectrum through \cite{Luscher:1990ux}:
\beq
kcot[\delta_0(k)] = \frac{2Z_{00}[1;(\frac{kL}{2\pi})^2)]}{L\sqrt{\pi}},
\eeq{luscher}
where $k$ is the momentum of either mesons in the center of momentum frame corresponding 
to the center of momentum energy $E_{cm}=\sqrt{s}$. $k$ and $E_{cm}$ are related to each other through 
\beq
4sk^2 = (s-(M_{D}+M_{B^*})^2)(s-(M_{D}-M_{B^*})^2).
\eeq{k2cm}
A sub-threshold pole singularity in the $S$-wave scattering amplitude $t = ({\mathrm{cot}}\delta_0 - i)^{-1}$ 
occurs when $k{\mathrm{cot}}\delta_0 = \pm\sqrt{-k^2}$ 
\bet[hb]
  \begin{center}
          \begin{tabular}{p{2.0cm}p{2.0cm}p{2.0cm}>{\hfill\arraybackslash}p{2.cm}}
      \hline
      \hline
$M_{ps}$ [GeV] & $\chi^2/d.o.f$ & $A^{[0]}/E_{DB^*}$ & $A^{[1]}/E_{DB^*}$ \\\hline
\multirow{2}{*}{0.5} & 2.1/2 & $-0.05(1)$ & $~0.17(_{-11}^{+13})$ \\\cline{2-4} 
                     & 1.3/1 & $-0.05(1)$ & $~0.13(_{-12}^{+13})$ \\ \hline
\multirow{2}{*}{0.6} & 0.5/2 & $-0.044(_{-8}^{+9})$ & $~0.10(_{-9}^{+9})$ \\ \cline{2-4} 
                     & 0.3/1 & $-0.043(_{-8}^{+9})$ & $~0.09(_{-10}^{+9})$ \\ \hline
\multirow{2}{*}{0.7} & 3.0/2 & $-0.042(_{-6}^{+8})$ & $~0.09(_{-7}^{+6})$ \\ \cline{2-4} 
                     & 1.5/1 & $-0.040(_{-6}^{+8})$ & $~0.06(_{-8}^{+6})$ \\ \hline
\multirow{2}{*}{1.0} & 2.9/2 & $-0.043(4)$ & $~0.11(_{-5}^{+5})$ \\ \cline{2-4} 
                     & 0.4/1 & $-0.041(4)$ & $~0.14(_{-4}^{+5})$ \\ \hline
\multirow{2}{*}{3.0} & 3.6/2 & $~0.006(_{-5}^{+6})$ & $-0.20(_{-5}^{+4})$ \\ \cline{2-4} 
                     & 1.9/1 & $~0.010(_{-5}^{+6})$ & $-0.25(_{-5}^{+4})$ \\ \hline
      \hline
  \end{tabular}
  \end{center}
\caption{Results from amplitude fits for different light quark mass scenarios indicated 
in terms of $M_{ps}$ in the first column. For each $M_{ps}$, two independent fits are 
performed with (top row) and without (bottom row) the level from $S_1$ ensemble. All fits 
are performed with the parameterization in \eqn{linparam}, where the optimized parameter 
values in the table are presented in units of the $DB^*$ threshold, $E_{DB^*}$. }
\eet{Ampfits1}
for scattering in $S$-wave. We follow the procedure outlined in Appendix B of Ref. 
\cite{Padmanath:2022cvl} in constraining the amplitude, such that the parametrization of 
$k{\mathrm{cot}}\delta_0$ is tuned to satisfy Eq. (\ref{luscher}). The parametrized 
$k{\mathrm{cot}}\delta_0$ is then investigated for poles of $t$ in the complex energy plane.  

\begin{figure}[h]
% Figure removed
\caption{$k{\mathrm{cot}}\delta_0$, in units of the elastic threshold $E_{DB^*}$, versus $a$ 
(lattice spacing) for all $M_{\pi}$ values. We follow the marker/color coding in \tbn{lattice} 
for the data points referring to the simulated data. The colored/gray bands indicate the fit 
results to the continuum extrapolation fit form in \eqn{linparam} with/without the data from 
$S_1$ ensemble.} 
\eef{alatdep}
Since we use only the ground states for amplitude fits, we limit ourselves to a scattering 
amplitude parametrization that is completely described by scattering length $a_0$ in an effective 
range expansion near the threshold. Additionally, we also consider a lattice spacing dependence 
on the parametrization of $k{\mathrm{cot}}\delta_0$. We find that a linear functional form
given by 
\beq
k{\mathrm{cot}}\delta_0 = A^{[0]} + aA^{[1]}
\eeq{linparam}
provides acceptable fits to the scattering amplitudes. Such an $a$ dependence was also found to 
be necessary in our previous investigations using NRQCD framework as well \cite{Mathur:2022ovu}, 
and is consistent with the leading $a$ dependence of observables
involving an NRQCD evolution. In this form, $A^{[0]}=-1/a_0$, where $a_0$ is the scattering 
length in the continuum limit. 
We list the results from different amplitude fits in \tbn{Ampfits1}. In \fgn{alatdep}, we present 
the quality of these fits by comparing the fit results with the data points. The colored/gray 
bands indicate the fit results including/excluding results from the $S_1$ ensemble to the respective 
fits. It can be clearly seen that fit results are less affected by inputs from $S_1$ ensemble, 
which is obvious given the large uncertainties associated with them, in contrast to inputs from 
$L_1$, $S_2$, and $S_3$. In \fgn{pcotdelta_summary}, we present $k{\mathrm{cot}}\delta_0$ versus 
$k^2$ based on the ground state energies presented in \fgn{gsspectrum} following \eqn{luscher}. 
The colored/gray bands indicate continuum extrapolated results including/excluding results from 
the $S_1$ ensemble to the respective fits. Clearly, there are no statistically significant effects
from the inclusion/exclusion of the energy levels from the $S_1$ ensemble observed.
\begin{figure}[h]
% Figure removed
\caption{$k{\mathrm{cot}}\delta_0$ versus $k^2$ for all $M_{\pi}$ values studied in units of 
the elastic threshold $E_{DB^*}$. The data points refer to the simulated data and follow 
the color coding in \tbn{lattice}. The dashed orange (cyan) curve indicates the constraint 
for the existence of a sub-threshold pole in the scattering amplitude. The horizontal bands
are the continuum extrapolated estimates of $k{\mathrm{cot}}\delta_0$ for the respective
$M_{\pi}$ (see \fgn{alatdep}). }
\eef{pcotdelta_summary}

Our main aim is to reliably determine $A^{[0]}=-1/a_0$, the sign of which determines the fate 
of the near threshold pole, if there exists one. A negative (positive) value of $A^{[0]}$($a_0$) 
indicates that the interaction potential is strong enough to form a real bound state\cite{Landau:1991wop}. 
It can be seen from \tbn{Ampfits1} and \fgn{alatdep} that for the non-charm light quark masses, 
$A^{[0]}$, the continuum extrapolated value for $k{\mathrm{cot}}\delta_0$ is negative, which 
indicates a possibly strong attractive interaction sufficient enough to host a real bound state. 
Whereas at the charm point, despite the unambiguous negative energy shifts in the finite-volume 
ground state energies with respect to the elastic threshold, the attraction is weak to host any 
real bound state as suggested by the positive value of $k{\mathrm{cot}}\delta_0$ in the continuum 
limit. This observation goes in line with the phenomenological expectation for doubly heavy four 
quark ($QQ'l_1l_2$) systems with $m_{l_1}=m_{l_2}$ that the binding increases with increased 
relative heaviness of the heavy quarks with respect to its light quark content
\cite{Francis:2016hui,Czarnecki:2017vco,Junnarkar:2017sey}. 

Another interesting observation is related to the lattice spacing dependence of $k{\mathrm{cot}}\delta_0$
values. At the charm point, $A^{[1]}$ (see \eqn{linparam}) acquires a different signature in contrast 
to that for the light quark masses. This suggests that for a doubly heavy four quark ($QQ'l_1l_2$) system 
with $(m_{l_1} = m_{l_2}, ~m_{Q},m_{Q'}>>m_{l})$, the cut off effects weaken the finite-volume energy 
splitting of the ground state with the elastic threshold. On the other hand, at the charm point (where 
$m_{Q},m_{Q'}\sim m_{l}$) such effects enhance this energy splitting in the $QQ'l_1l_2$ system determined 
in a finite-volume. Relatively large errors at the noncharm $M_{ps}$ values partially obscure these effects, 
if any exist, while at the charm point such effects are clearly reflected. 

\subsection{Light quark mass dependence}
Following the individual amplitude fits to different light quark mass cases, now we investigate the light 
quark mass ($m_{u/d}$) or $M_{ps}$ dependence of the parameters $A^{[0]}$ and $A^{[1]}$. Due to leading order 
$M_{ps}^2$ terms in the chiral expansion, we assume the $M_{ps}$ dependence of hadron masses for 
light $m_{u/d}$ values ($m_q\lesssim\Lambda_{QCD}$) to be linear in $M_{ps}^2$. Whereas towards the heavy 
$m_{u/d}$ regime ($m_q>>\Lambda_{QCD}$) heavy hadron masses are expected to be proportional to the quark mass, 
hence to $M_{ps}$ \cite{Neubert:1993mb}. With these assumptions, we work with three following fit forms that 
could be useful. 
\beqa
	f_l(M_{ps}) &=& \alpha_c + \alpha_l M_{ps}, \nonumber \\
	f_s(M_{ps}) &=& \beta_c + \beta_s M_{ps}^2, \mbox{~~~and} \nonumber \\
	f_q(M_{ps}) &=& \theta_c + \theta_l M_{ps} + \theta_s M_{ps}^2.
\eeqa{mqdep}
Fits to determine the $M_{ps}$ dependence were made by minimizing a single cost function 
defined combinedly for $A^{[0]}$ and $A^{[1]}$ as %\cite{FullLuscher}
\beq
	\chi^2 =\sum_{\substack{x, y \\ \in \{A^{[j]}_{i}\}}}\left(f_x-f_{px}(M_{ps})\right)\tilde{\mathcal{C}}^{-1}_{xy}\left(f_y-f_{py}(M_{ps})\right),
\eeq{chi2mqdep}
where the summation runs over all fitted parameters $\{A^{[j]}_{i}\}$ with $j\in\{0, 1\}$ and $i$ 
referring to the five different light quark masses studied. In \tbn{Ampfits1}, we list the fit results 
for $f_{x,y}$. $\tilde{\mathcal{C}}_{ij}$ is the associated data covariance determined following Ref. \cite{Prelovsek:2020eiw}. 
$f_{pn}(M_{ps})$ are the fit forms incorporating the $m_{u/d}$ dependence in parameters $\{A^{[j]}_{i}\}$.  
In \fgn{a0a1_separate}, we show the fit results for $A^{[0]}=-1/a_0$ to the fit forms in \eqn{mqdep}. The large 
circles represent the $A^{[0]}$ values at different $M_{ps}$, the bands represent the fit results 
with different fit forms in \eqn{mqdep}, and the two stars represent $A^{[0]}$ at the physical 
$M_{ps}$ (equivalently the physical scattering length $a_0^{phys}$) and the critical $M_{ps}$ at which 
$A^{[0]}$ changes its sign (positive to negative), in other words, the system becomes unbound. It is 
indeed desired to have more points in the intermediate mass regime between the charm and the strange 
\begin{figure}[h]
% Figure removed
\caption{Continuum extrapolated $k{\mathrm{cot}}\delta_0$ or $A^{[0]}=-1/a_0$ estimates of the $DB^*$ system 
as a function of $M_{ps}^2$ in units of $E_{DB^*}$. The band indicates fit results to the simulated results. 
The legend carries info on the fit forms presented (see also \eqn{mqdep}) and the quality of fits. The dotted 
vertical line close to the $y$-axis indicates the physical $M_{ps}$. The two star symbols represent the 
amplitude at the physical $M_{ps}$ and the critical $M_{ps}$ at which the system becomes unbound.}
\eef{a0a1_separate}
quark masses to further constrain the dependence. Yet, our fits in this work demonstrate near independence in 
the fit forms as can be observed from the consistency between the error bands from different fit 
forms. 

\begin{figure}[h]
% Figure removed
\caption{The landscape of the continuum scattering length $A^{[0]}$ versus $A^{[1]}$ (see \eqn{linparam}) 
for all $M_{ps}$ values (indicated in the legend) studied. The central values are represented by black 
edged circles with color fillings, whereas the scattered points are the bootstrap samples. The band 
represents the correlated $M_{ps}$ dependence of the fitted parameters.} 
\eef{a0a1_combined}
Next we look at the correlated pion mass dependence in the parameters $A^{[0]}$ and $A^{[1]}$ (see 
\eqn{chi2mqdep} for the definition of the cost function) presented in \fgn{a0a1_combined}. The black 
bordered symbols are the central values of parameters determined for each $M_{ps}$, whereas scattered 
small circles indicate the bootstrap sample distribution in the $[A^{[0]},~A^{[1]}]$ landscape. The bands 
in the figure represent the uncertainty in the parameters, with the inner band quantifying the statistical 
errors, while the outer band also incorporates the systematic uncertainty arising from different fit forms 
added in quadrature symmetrically. A negative correlation can clearly be observed between the parameters 
across different quark masses studied, which is accounted in the fits through the data covariance matrix 
entering the cost function. This correlation can also be observed within the distribution of the bootstrap 
samplings at all quark masses. This observation clearly demonstrates the need for a careful treatment of 
cutoff errors, particularly in heavy hadron systems with interesting near threshold features, such as this. 

In the chiral regime ($m_{u/d}\lesssim\Lambda_{QCD}$), leading $m_{u/d}$ dependence in hadronic observables 
is assumed to go as linear in $M_{ps}^2$. Based on the fit form $f_s(M_{ps})$, we find that the scattering length 
of the $DB^*$ system at the physical light quark mass ($m_{u/d}^{phys}$) to be
\beq
a_0^{phys} = 0.57(^{+4}_{-5})(17) \mbox{~fm}.
\eeq{scatlen}
The asymmetric errors indicate the statistical uncertainties, whereas the second parenthesis quotes 
the systematic uncertainties with the most dominant contribution arising from the chiral extrapolation 
fit forms. We elaborate on various systematic uncertainties towards end of this section. The positive 
value of the scattering length is an unambiguous evidence for the ability/strength of the hadron-hadron 
interaction potential to host a real bound state (when $k~{\mathrm{cot}}\delta_0 = -\sqrt{-k^2}$). 
The observed scattering length at physical light quark mass suggests the presence of a real $bc\bar u\bar d$ 
tetraquark bound state $T_{bc}$ with binding energy 
\beq
\delta m_{T_{bc}} = -43(^{+6}_{-7})(^{+14}_{-24}) \mbox{~MeV},
\eeq{betbc}
with respect to $E_{DB^*}$. The systematic effects on the $a_0^{phys}$ and $\delta m_{T_{bc}}$ estimates 
of ignoring the charm point in the fits to the $m_{u/d}$ dependence are found to be very small, 
compared to the number quoted for systematic uncertainties in \eqn{scatlen}. 

Towards the heavy quark regime ($m_{u/d}>>\Lambda_{QCD}$), the heavy hadron masses can have 
leading linear dependence in $M_{ps}$ as $M_{ps}\propto m_{u/d}\sim m_{Q}$ \cite{Neubert:1993mb}. 
Following the fit form $f_l(M_{ps})$, which is linear in quark mass, the critical light quark mass 
$m_{u/d}^*$ at which the scattering length diverges, then changes its signature such that the 
interaction potential is not able host a real bound state, corresponds to the critical pseudoscalar 
meson mass given by 
\beq
M^{*}_{ps} = 2.73(21)(14) \mbox{~GeV}.
\eeq{unitary}
This corresponds to the star symbol at the zero crossing in the $x$-axis ($A^{[0]}=0$) in \fgn{a0a1_separate}. 
Once again the first parenthesis indicates the statistical errors and the second one quantifies various 
systematic uncertainties added in quadrature. 


Now we briefly comment on other possible sources of systematic uncertainties in this calculation. Our lattice setup, 
discussed in Section \ref{sec:lattice}, together with the bare bottom and charm quark mass tuning procedure 
has been demonstrated to reproduce the $1S$ hyperfine splittings in quarkonia with uncertainties less than 6 MeV
\cite{Mathur:2022ovu,Mathur:2016hsm}. Additionally, our strategy of evaluating the energy differences 
and working with mass ratios has also been shown to significantly mitigate the systematic uncertainties related 
heavy quark masses \cite{Mathur:2018epb,Mathur:2022ovu}. Our fitting procedure discussed in Section \ref{sec:2ptIO} 
involves careful and conservative determination of statistical errors, and uncertainties related to the 
excited-state-contamination and fit-window errors. The amplitude determination and followed extrapolations
are performed with results from varying the fit-windows to evaluate the uncertainties propagated to our final 
results. The uncertainties related to the fit forms used in chiral extrapolations are observed to be dominant,  
and the number in the second parenthesis in Eqs. \ref{scatlen}, \ref{betbc}, and \ref{unitary} are the total 
systematic uncertainties added in quadrature. Uncertainty related to scale setting are also found to be negligibly 
small in comparison to the statistical uncertainties \cite{Mathur:2018epb,Mathur:2022ovu}. 


%!TEX root = ../Schur indices and line operators.tex


\section{Discussion}












\begin{table*}[t]
\caption{Summary of the top-performing teams in each track of the RoboDepth Challenge.}
\centering\scalebox{1}{
\begin{tabular}{c|p{5cm}|p{5cm}}
\toprule
\textbf{Rank} & \textbf{\#1: Robust Self-Supervised MDE} & \textbf{\#2: Robust Supervised MDE}
\\\midrule\midrule
\multirow{13}{*}{\textcolor{robo_blue}{\textbf{1st Place}}} & \textbf{Team Name} & \textbf{Team Name}
\\
& \textcolor{robo_blue}{OpenSpaceAI} & \textcolor{robo_blue}{USTCxNetEaseFuxi}
\\
\cmidrule{2-3}
& \textbf{Team Members} & \textbf{Team Members}
\\
& Ruijie Zhu$^1$, Ziyang Song$^1$, Li Liu$^1$, Tianzhu Zhang$^{1,2}$ & Jun Yu$^1$, Mohan Jing$^1$, Pengwei Li$^1$, Xiaohua Qi$^1$, Cheng Jin$^2$, Yingfeng Chen$^2$, Jie Hou$^2$
\\
\cmidrule{2-3}
& \textbf{Affiliations} & \textbf{Affiliations}
\\
& $^1$University of Science and Technology of China, $^2$Deep Space Exploration Lab & $^1$University of Science and Technology of China, $^2$NetEase Fuxi
% \\
% \cmidrule{2-3}
% & \textbf{Approach} & \textbf{Approach}
% \\
% & IRUDepth with MPViT as depth encoder and PoseNet for camera poses and depth maps with AugMix& <...>
\\\cmidrule{2-3}
& \textbf{Contact} $\textrm{\Letter}$ & \textbf{Contact} $\textrm{\Letter}$
\\
& \texttt{ruijiezhu@mail.ustc.edu.cn} & \texttt{USTC\_IAT\_United@163.com}
\\\midrule\midrule
\multirow{17}{*}{\textcolor{robo_red}{\textbf{2nd Place}}} & \textbf{Team Name} & \textbf{Team Name}
\\
& \textcolor{robo_red}{USTC-IAT-United} & \textcolor{robo_red}{OpenSpaceAI}
\\
\cmidrule{2-3}
& \textbf{Team Members} & \textbf{Team Members}
\\
& Jun Yu$^1$, Xiaohua Qi$^1$, Jie Zhang$^2$, Mohan Jing$^1$, Pengwei Li$^1$, Zhen Kan$^1$, Qiang Ling$^1$, Liang Peng$^3$, Minglei Li$^3$, Di Xu$^3$, Changpeng Yang$^3$ & Li Liu$^1$, Ruijie Zhu$^1$, Ziyang Song$^1$, Tianzhu Zhang$^{1,2}$
\\
\cmidrule{2-3}
& \textbf{Affiliations} & \textbf{Affiliations}
\\
& $^1$University of Science and Technology of China, $^2$Central South University, $^3$Huawei Cloud Computing Technology Co., Ltd & $^1$University of Science and Technology of China, $^2$Deep Space Exploration Lab
\\
\cmidrule{2-3}
& \textbf{Contact} $\textrm{\Letter}$ & \textbf{Contact} $\textrm{\Letter}$
\\
& \texttt{USTC\_IAT\_United@163.com} & \texttt{liu\_li@mail.ustc.edu.cn}
\\\midrule\midrule
\multirow{11}{*}{\textcolor{robo_green}{\textbf{3rd Place}}} & \textbf{Team Name} & \textbf{Team Name}
\\
& \textcolor{robo_green}{YYQ} & \textcolor{robo_green}{GANCV}
\\
\cmidrule{2-3}
& \textbf{Team Members} & \textbf{Team Members}
\\
& Yuanqi Yao$^1$, Gang Wu$^1$, Jian Kuai$^1$, Xianming Liu$^1$, Junjun Jiang$^1$ & Jiamian Huang$^1$, Baojun Li$^1$
\\
\cmidrule{2-3}
& \textbf{Affiliations} & \textbf{Affiliations}
\\
& $^1$Harbin Institute of Technology & $^1$Individual Researcher
\\
\cmidrule{2-3}
& \textbf{Contact} $\textrm{\Letter}$ & \textbf{Contact} $\textrm{\Letter}$
\\
& \texttt{yuanqiyao@stu.hit.edu.cn} & \texttt{huang176368745@gmail.com}
\\\bottomrule
\end{tabular}
}
\label{tab:summary}
\end{table*}
%%%%%%%%%%%%%%%%%%%%%%%%%%%%%%%%%%%%%%%%%%%%%%%%%%%%%%%%%%%%%%%%%%%%%%%%%%%%%%%%
%% ACKNOWLEDGMENTS
\begin{acknowledgments}
%
This work is supported by the Department of Atomic Energy, Government of India, under Project Identification Number RTI 4002. We are thankful to the MILC collaboration and in particular to S. Gottlieb for providing us with the HISQ lattice ensembles. We thank Sara Collins for a careful reading of the manuscript. We thank the authors of Ref. \cite{Morningstar:2017spu} for making the {\it TwoHadronsInBox} package utilized in this work. We also thank Gunnar Bali, Parikshit Junnarkar and Sayantan Sharma for discussions. Computations were carried out on the Cray-XC30 of ILGTI, TIFR. Amplitude analyses were performed on Nandadevi computing cluster at IMSc Chennai. N. M. would also like to thank A. Salve and K. Ghadiali for computational support.
\end{acknowledgments}



\begin{comment}
\section{System Architecture}
\label{appendix:architecture}
\system has a novel modularized system architecture with three key components: 
\emph{StreamManager}, 
\emph{TxnManager} and \emph{TxnScheduler}. 
These components are instantiated in each thread locally.
The execution outline of \system is presented in Algorithm~\ref{alg:algo}.
Transactional stream processing is continuous and potentially never ends (Line 1$\sim$8).
The dependency resolution and execution of state transactions are separated into two non-overlapping phases by punctuations~\cite{Tucker:2003:EPS:776752.776780} (Line 2 and 5), which guarantees that no subsequent input event will have a smaller timestamp. 
Effectively, a batch of state transactions is collected during the first phase, and processed during the second phase.

In the first phase (i.e., stream processing phase), 
the \emph{StreamManager} conducts preprocessing for every input event ($e$). Similar to some prior works~\cite{tstream}, state transactions may be issued but not immediately processed during preprocessing (Line 3).
The \emph{pre\_processing} and \emph{post\_processing} functions are exposed as APIs to users.
The \emph{TxnManager} handles dependency resolution (Line 4) among state transactions and insert decomposed operations to construct a \tpg. We discuss the detailed two-phase \tpg construction process in Section~\ref{subsec:construction}.

In the second phase  (i.e., transaction processing phase), 
the \emph{TxnManager} is first involved again to refine (Line 6) the constructed \tpg with further dependency resolution.
The \emph{TxnScheduler} 
schedules operations for concurrent execution based on the constructed \tpg according to the three dimensions of scheduling decisions (Line 7). 
In particular, a scheduling decision model $M$ is instantiated based on the constructed \tpg (Line 14).
\textbf{\circled{1}} Guided by $M$, execution threads adopt an exploration strategy (Section~\ref{subsec:explore}) to explore the constructed \tpg for operations available to be scheduled constrained by dependencies. 
\textbf{\circled{2}} 
During exploration, one or multiple operations may be treated as the 
% basic 
unit of scheduling (Section~\ref{subsec:granularity}). 
Subsequently, \textbf{\circled{3}} every thread executes operation(s) in the unit of scheduling with various abort handling mechanisms (Section~\ref{subsec:abort_handling}).
Only when state transactions are processed (i.e., committed or aborted) can the associated input events be postprocessed (Line 8) by the \emph{StreamManager} based on transaction processing results.
\end{comment}

\begin{comment}
\begin{algorithm}
\footnotesize
    \KwData{$e$ \tcp{Input event}}
    \KwData{$txn_{ts}$ \tcp{State transaction}}
    \KwData{$G$ \tcp{The currently constructed TPG}}
    \While{!finish processing of input streams}{
        \eIf(\tcp*[h]{Phase 1}){\text{$e$ is not a $punctuation$}}{
                $txn_{ts}$ $\gets$ PRE\_Processing($e$)\;
                \textbf{TPG\_Construction}($G$, $txn_{ts}$)\; 
          }(\tcp*[h]{Phase 2}){
                \textbf{TPG\_Refinement}($G$)\; 
                \textbf{TXN\_Scheduling}($G$)\; 
                POST\_Processing()\;
          }
    }
    
    \SetKwFunction{FMain}{TPG\_Construction}
    \SetKwProg{Fn}{Function}{:}{}
    \Fn{\FMain{$G$, $txn_{ts}$}}{
        $O_{1..k}$ $\gets$ \textbf{Partition} $txn_{ts}$\;
        \ForEach{\text{operation $O_{i}$ $\in$ $O_{1..k}$}}{
            \textbf{Identify} its \ld\;
            $G$ $\gets$ $G$ + $O_{i}$ \;
        }
    }
    \SetKwFunction{FMain}{TPG\_Refinement}
    \SetKwProg{Fn}{Function}{:}{}
    \Fn{\FMain{$G$}}{
        \ForEach{\text{vertex $e_{i}$ $\in$ $G$}}{
            \textbf{Identify} its \td, \pd\;
        }
    }
    
    \SetKwFunction{FMain}{TXN\_Scheduling}
    \SetKwProg{Fn}{Function}{:}{}
    \Fn{\FMain{$G$}}{
        $M$ $\gets$ Instantiated with $G$;\tcp{A decision model}
        \While{!finish scheduling of $G$
        }{
          \textbf{\circled{2}} $Scheduling Unit$ $\gets$ \textbf{\circled{1}} \emph{Explore}($G$, $M$)\; 
            \textbf{\circled{3}} \emph{Execute with Abort Handling} ($Scheduling Unit$)\; 
        }
    }
  \caption{Execution Outline of \system}
  \label{alg:algo}
\end{algorithm}
\end{comment}



%%%%%%%%%%%%%%%%%%%%%%%%%%%%%%%%%%%%%%%%%%%%%%%%%%%%%%%%%%%%%%%%%%%%%%%%%%%%%%%%%
%% BIBILOGRAPHY
%\bibliographystyle{apsrev4-1}
%\bibliography{bib}
\bibliography{paper}

%%%%%%%%%%%%%%%%%%%%%%%%%%%%%%%%%%%%%%%%%%%%%%%%%%%%%%%%%%%%%%%%%%%%%%%%%%%%%%%%%


\end{document}






%
\title{Sustainable Behaviour of IT Practitioners}%
\titlerunning{Sustainable Behaviour of IT Practitioners}% explicit
%
\author{Elena Somova\orcidID{0000-0003-3393-1058} \and Denitza Charkova\orcidID{0000-0003-0873-4415}  }
\authorrunning{E. Somova, D. Charkova}% explicit
%
\institute {University of Plovdiv "Paisii Hilendarski", 24 Tzar Assen St., 4000 Plovdiv, Bulgaria,\\%
	\email{ eledel@uni-plovdiv.bg; dcharkova@uni-plovdiv.bg}\\%
	\url{https://www.uni-plovdiv.bg/} }%
%
\maketitle% typeset the header of the contribution
%
\begin{abstract}%
Environmental protection is a key topic nowadays. Many world and European organizations and bodies (with their respective documents) fight against the global negative impact of industries such as fossil fuels. The work of IT specialists could influence other sectors of sustainability in both positive and negative ways. The paper presents a questionnaire, exploring students' attitude to sustainability and the place of sustainability in higher education and future employment. The survey will be conducted during the Second SusTrainable Summer school.%
\keywords{Environmental Sustainability \and Technical Sustainability \and Sustainable Behavior}%
\end{abstract}%
\section{Introduction}%
\par%
Sustainability has many dimensions - environmental, economic, social, technical, educational, cultural, etc., as environmental sustainability will lead to the most long-term and positive global consequences. For example, the negative effects of burning fossil fuels (coal, oil and natural gas) are well-known: air pollution, water pollution, climate change and health problems.%
\par%
The work of IT specialists could also influence sustainability in other sectors. Something more, IT professionals can impact environmental sustainability, both positively and negatively – positively by reducing the impact of IT on the environment (Green IT) and applying IT to improve sustainability (\#Tech4Good), and negatively by supporting and working for fossil fuel companies and other unsustainable sectors (\#Tech4Bad)~\cite{1BIB1}.%
\par%
This work presents the survey aimed at exploring IT specialists (students and teachers)’ attitude towards sustainability (in particular to the fossil fuel industry) and the place of sustainability in higher education and future employment. The survey is going to be carried out with the participants of the Second SusTrainable Summer School.
%

\section{Global and European political and legislation issues}\label{section:overview2}%
\par%
The global and European policy is directed to environmental protection incl. first reduction and late cessation of the fossil fuel industry:%
\begin{itemize}%
\item The United Nations Framework Convention on Climate Change~\cite{1BIB2}, signed at the United Nations Conference on Environment and Development in 1992;
\item The Paris Agreement~\cite{1BIB3}, adopted at COP 21 in 2015;%
\item The United Nations 2030 Agenda for Sustainable Development, adopted in 2015, with its 17 Sustainable Development Goals (SDGs)~\cite{1BIB4};%
\item The Global Coal to Clean Power Transition Statement, signed at the 26th United Nations Climate Change Conference (COP 26) in 2021;%
\item The European Green Deal~\cite{1BIB5}, approved in 2020;%
\item The set of proposals to reduce net greenhouse gas emissions, adopted by the European Commission in 2021;%
\item The priority areas of the Energy Transition Council, established in 2020;%
\item The International Energy Agency (IEA) Report~\cite{1BIB6} in 2021, etc.%
\end{itemize}%
\par%
All these reductions in environmentally harmful industries will inevitably affect the IT sector.%
\section{Sustainable behavior of IT practitioners}
\par%
When IT practitioners make their behavioral decisions, the choose how sustainable their behavior is depending on some pragmatic and ethical issues. The paper~\cite{1BIB1} explores in detail these pragmatic and ethical concerns that IT practitioners have to take into account when they want to start working or already work in a fossil fuel company or its supplier company.%
\par%
From a pragmatic point of view, the fossil fuel sector is decreasing and will end in the next few decades, according to the global politics (see Section 2), therefore IT specialist should move to another sector. A lot of jobs will be lost according to IEA~\cite{1BIB6} very soon, because companies will not invest in new fossil fuel exploration and development. Because of the "net-zero emissions by 2050" pathway, some industries will end by 2050 or earlier, such an example being  the transport fossil fuel industry. In recent years, there has been a rapid increase in net zero emissions announcements by companies with target 2050 or earlier~\cite{1BIB6}. A common classification system – EU Taxonomy for sustainable activities~\cite{1BIB7}, establishing a list of environmentally sustainable economic activities has been created, where the fossil fuels are excluded. Some organizations provide lists that can be used for recognition of sustainable companies – for examples, MSCI launched Global Fossil Fuels Exclusion Indexes~\cite{1BIB8}, used by institutional investors to to eliminate or reduce some or all fossil fuel reserves exposure from their investments, and Investopedia’s Environmental, social, and governance (ESG) criteria~\cite{1BIB9}, giving a set of standards for a company’s behavior used by investors to screen potential investments based on corporate sustainable policies.%

\par%
From an ethical standpoint, to work for a company producing or supplying fossil fuels is not illegal, but the ethical point of view should be considered. Whether to work for an environmentally damaging sector or not, is a personal decision. Ethical principles, connected to the issue of sustainability, are reflected in the main documents of a number of IT organizations – IEEE Code of Ethics~\cite{1BIB10}, BCS Code of Conduct~\cite{1BIB11}, ACM Code of Ethics and Professional Conduct~\cite{1BIB12}, etc. The environmentally damaging fossil fuels sector directly hinders the two of the 17 SDGs of the UN – SDG13. Climate Action and SDG7. Affordable and Clean Energy, and indirectly – SDG3. Good Health and Well-Being, SDG11. Sustainable Cities and Communities, etc.%
\par%
All these pragmatic and ethical considerations~\cite{1BIB1} that IT practitioners should take into account when deciding which company to work for show that this decision is very important and with consequences.%
%
\section{Survey questionnaire}
\par%
The objectives of the questionnaire are:%
\begin{itemize}%
\item to investigate young IT professionals’ and teachers’ attitude to sustainability (incl. the fossil fuel industry);%
\item to explore IT students’ and teachers’ opinion about the place of sustainability in higher education;%
\item to study their opinion about future employment in relation to sustainability.%
\end{itemize}%
\par%
The survey consists of 20 questions (see the Survey Questionnaire): 5 demographic questions and 15 questions (presented as statements) on issues of interest. The survey uses 7 of the UK survey questions~\cite{1BIB13}~\cite{1BIB15}~\cite{1BIB16} to enable comparison with existing results. The statements are divided into 6 groups (see Table 1).%

\begin{table}[]
	\centering
	\begin{tabular}{ll}
		\textbf{Group} & \textbf{Questions}  \\
		Sustainability awareness & 1-4  \\
		Sustainability at university &   5-7  \\
		Sustainability in employers &   8-10 \\
		Ethical concerns &  11-13 \\
		Practical concerns &  14-16 \\
		Community behavior &   17-20
	\end{tabular}
\caption{\label{demo-table2}Survey question groups}
\end{table}

\par%
The majority of the questions in the survey use a 5-step Likert scale to give students a better opportunity to express their opinion to the fullest: Strongly Disagree, Disagree, Undecided, Agree and Strongly Agree. The remaining questions provide students with situational cases from which they need to choose the option best suited for them.%

\par%
\textbf{SURVEY QUESTIONNAIRE}%
\par%
\textbf{Part 1. Personal data}%
\par%
1. Sex – Male/Female%
\par%
2. Year of study/Teacher – 1/2/3/4/teacher%
\par%
3. Working now/have worked in the ICT sector? – Yes/No%
\par%
4. If yes, for how long?%
\par%
5. Country of study/teaching%
\par%
\textbf{Part 2. Personal opinion}%
\par%
1. I know of the United Nations’ Sustainable Development Goals.%
\par%
2. I know that globally we must reach net zero carbon emissions by 2050.%
\par%
3. Fossil fuels are at the root of the climate crisis.%
\par%
4. I personally carry out the skill “understand people’s relationship to nature”.%
\par%
5. The university practices and promotes good social and environmental skills.%
\par%
6. The University is obliged to develop students’ social and environmental skills as part of the courses.%
\par%
7. The university courses consider the ethical (in the direction of environmental and social aspects) implications of the course subject.%
\par%
8. My employer should consider the environmental and social impacts of its products and/or services.%
\par%
9. My employer should actively work for the reduction of carbon emissions (incl. by refusing to provide products/services to the fossil fuels sector).%
\par%
10. The employee’s skill of “understanding people’s relationship to nature” should be important for my employers (current or future).%
\par%
11. I am concerned about environmental climate change.%
\par%
12. I will refuse to do any work at my company that supports the fossil fuel industry.%
\par%
13. I will leave my company if I learn that it produces software/ provides services for the fossil fuel industry.%
\par%
14. I share my position about not supporting the Fossil Fuel Industry with other ICT practitioners.%
\par%
15. I participate in conversations/events about the negative impact of the ICT sector on the environment when supporting the Fossil Fuel Industry.%
\par%
16. I show support to others who are speaking up and saying no to working for the Fossil Fuel Industry.%
\par%
17. I know which organization (Trade union, Non Government Organizations) to contact with to help me find an employer who doesn’t support the fossil fuel industry or to help me leave a company that supports the environmentally damaging fossil fuel sector.%
\par%
18. Which option would you choose?
Assuming all other factors are equal, I would choose a graduate position with a starting salary of 100 Euro higher than average in a company with a poor environmental and social record.
Assuming all other factors are equal, I would choose a graduate position with a starting salary of 100 Euro lower than average in a company with a strong environmental and social record.%
\par%
19. Which option would you choose?
Assuming all other factors are equal, I would choose a graduate position with a starting salary of 300 Euro higher than average in a company with a poor environmental and social record.
Assuming all other factors are equal, I would choose a graduate position with a starting salary of 300 Euro lower than average in a company with a strong environmental and social record.%
\par%
20. Which option would you choose?
Assuming all other factors are equal, I would accept a graduate position with a starting salary of 300 Euro higher than average in a role that does not contribute to positive environmental and social change.
Assuming all other factors are equal, I would accept a graduate position with a starting salary of 300 Euro lower than average in a role that contributes to positive environmental and social change.%

\par%
The survey was completed by 260 young IT professionals – IT students from University of Plovdiv “Paisii Hilendarski”, Bulgaria.%
\par%
During the summer school, a survey will be conducted among IT students and teachers, and the results of the survey will be able to be compared across participating countries. The survey will be voluntary and anonymous.%
\section{Conclusion}
\par%
IT specialists have to take into account that a career in an environmentally damaging sector like the fossil fuel industry will be short-term and in near future they have to change their job in other sectors. Working for the fossil fuel industry, through supporting or developing software that continues the production and sales of fossil fuels, is a legal job, but it is incompatible with the ethical principles of the ICT professional organizations and ethical job decisions are very important.%
\par%
During the Second SusTrainable Summer School, a survey will be conducted among IT students and teachers to explore IT specialists’ attitude to sustainability (in particular to the fossil fuel industry) and the place of sustainability in higher education and future employment.%
%
\par%
\textbf{Acknowledgements}%
\par%
This paper acknowledges the support of the Erasmus+ Key Action 2 (Strategic partnership for higher education) project No 2020-1-PT01-KA203-078646: “SusTrainable – Promoting Sustainability as a Fundamental Driver in Software Development Training and Education”. The information and views set out in this paper are those of the authors and do not necessarily reflect the official opinion of the European Union. Neither European Union institutions and bodies, nor any person acting on their behalf may be held responsible for the use which maybe made of the information contained therein.

%\bibliographystyle{splncs04}
%\bibliography{Paper_PU_1.bib}
%\printbibliography%

%\end{document}%
% 
\input{bbls/Paper_PU_1_BIBTEX.bbl}




%
\title{Energy Consumption Related Questions of Developing and Evolving Dynamic Web Applications}
%
\titlerunning{Energy Consumption of Web Applications}
% If the paper title is too long for the running head, you can set
% an abbreviated paper title here
%
\author{Csaba Szab\'o\orcidID{0000-0001-5147-2452}}
%
\authorrunning{Cs. Szab\'o}
% First names are abbreviated in the running head.
% If there are more than two authors, 'et al.' is used.
%
\institute{Technical University of Ko\v sice, Letn\'a 9, 042 00 Ko\v sice, Slovakia\\
\email{csaba.szabo@tuke.sk}}
%
\maketitle              % typeset the header of the contribution
%
\begin{abstract}
The aim of this paper is to shortly introduce the need of considering energy consumption when developing dynamic web applications. The underlying problem is presented from the users' point of view, namely that a web page that consumes too much energy might be the main reason of unwanted browser restarts or worse responsibility. Later, similarities and differences to related work are shown to contrast specifics of web applications, in both contexts of development and evolution. Then, we define an example development project using React with Typescript as frontend technology and Tensorflow with Python as backend technology. Energy consumption measurement will be distributed between the Firefox browser running the frontend and PyRAPL library at backend. After the initial measurements based on selected scenarios, we focus on analysis of energy consumption changes caused by directed evolution of the web application. All presented content aims to contribute to the topic of software engineering education by focusing on sustainable software solutions.

\keywords{Energy consumption \and Evolving web applications \and Python \and RAPL \and React \and Typescript.}
\end{abstract}
%
%
%
\section{Motivation}
"Everything is on the Internet" is a very common phrase of the last two decades. All industries consider this medium as the most important one to inform (potential) customers, clients etc. Common types of Internet-based communication include e-mail, chat and web pages. Using amount of information as classification criteria, we could say that web pages aim to provide much more information and functionality than e-mail or chat\cite{similarweb}. Usual implementations often include both chat and e-mail option as a form of one-on-one communication. Current webs offer lots of functionalities in form of rich web applications.

Nowadays, web browsers obviously offer the possibility to open more than one web page at the same time. So, the user does not need to run many browsers. With increasing number of open web pages, memory and CPU load from browsing also increases. Modern web browsers are optimised for a huge number of simultaneously opened pages, but the browser developers used selected webs when they optimised these products\cite{mdntesting}.

Despite this kind of usage focused optimisations, sometimes the users experience browser failures. Opened pages require restart (become slow) due to high resource usage, servers stop responding or more precisely the request is timed out. Mobile devices such as laptops and phones are more sensitive to such intensive resource usage since they are not continuously charged. Servers might have a shorter lifetime under higher workload.

Web pages also have their creators: designers and developers. In a world close to perfect, systematic testing process\cite{mdntesting} is also executed before deployment. Testing can uncover design and implementation weaknesses, but some types of weaknesses are usage specific, which is impossible to completely cover by testing. Users will also report some system failures they experience during application use.

User feedback\cite{swevolutionbook} points to further application improvements. This is what we call evolution of web applications. We have two sides. On the one, there are the users requesting changes, on the other one, there are the developers trying to implement the best improvement.

In this paper, we focus on a kind of web application evolution that is addressing energy consumption optimisation. First, we present related work and tools. Then, we introduce an example project using selected (measurable) technologies\cite{raplzhang,raplinaction}. To improve the measured stats of the example web application, we implement selected evolution techniques. To make it more interesting, we take a look at a different type of evolution: extending functionality; and its impact on the energy consumption profile of the application. Result analysis and what-if discussions will close the teaching class. We only present the process in this paper.

\section{Server/Client-side Green Computing and Evolution}
Web applications usually follow the same architecture. The logic and data of the application are distributed into a number of components, which are also distributed and/or grouped into specific virtual or physical locations\cite{architecture}.

The roles of these system modules are clear: presentation of information, storing underlaying data and implementation of the logic of the application. Logical distribution is much simpler in terms that there are only two parts: the modules run from within the browser (a.k.a. frontend) and everything else (a.k.a. backend). As the application is growing during its evolution, the architecture might be subject of change as well, but there also exist generic architectural patterns preparing the system for long-term evolution that do not require significant changes in the system architecture, such as the microservices' or microfrontends' architecture\cite{architecture}. In all of these architectures, front- and backends are clearly defined. Thus, the architecture comprehension could be minimised to understanding of the historical client/server architecture, considering frontend the client and backend the server.

Evolution of software is necessary\cite{swevolutionbook}. Since there is almost impossible to provide a final software from the first try so that software will not require any further modifications. The literature defines these necessary and/or user-requested modifications as software evolution. Sure, there exist solutions that do not require any improvements after implementation as they are complete and perfect (S-type software by Lehman, perfection is not subject of change over time) or, at least there is no better solution (P-type software by Lehman, where a minimal chance of further one-time modification exists, but that requires progress in theory and practical state of the art).

Majority of software is of E-type, meaning we expect a kind of evolution. This evolution is guided by the feedback from the system usage, which includes direct user feedback, usage data analysis etc. Part of this is profiling in different production environments.

With web applications, the problem of profiling a distributed application raises\cite{architecture}. But, a partial profiling can be done by separately analysing the selected nodes the system is being distributed to.

When we focus on energy profiling\cite{balkishanprofiling}, a significant problem is the limited ability to separate the effects of the artefacts under profiling from the side-effects of measuring and effects of other system components. Invasive and non-invasive measurements are used at different level of error\cite{raplzhang,raplinaction}. Non-invasive is the way, when one is measuring the whole system, and the artefact selection is done by scenario design (test setup in automation). An invasive way is to introduce changes into the system to be able to measure with a lower error. Some techniques emit signals to external measuring only, other might be derived from Unit-testing techniques\cite{unit}.

RAPL\cite{raplzhang,raplinaction} supports both, since there exists its language library integrations such as PyRAPL integration, but it could be also used as an external measurement tool such as the power gadget. Getting inside interpreters is much more complicated, therefore browser or operating system level measurements are kind different. The browser itself needs to include a development support interface for data collection, which data could be then used in further analysis, e.g. using RAPL again. Firefox web browser includes energy profiling ability since version 104. The energy model is not included in the browser, collected data are sent to a web service that processes them to a readable form.

Our preliminary profiling results in technologies show that even React and Angular web application frameworks are already significantly different when developing standard news applications (fetching news data from a database and displaying them in an equally formatted form). While React seems to be slower in execution (see Fig.~\ref{RvA1}), Fig.~\ref{RvA2} shows a significantly lower workload.
% Figure environment removed

% Figure environment removed

\section{Our Example Web Application}
We propose a web application containing user profiles and allowing chat messaging between these users. We select React and Typescript as frontend technologies, while a very basic backend will be implemented using Python. Two development projects as building blocks of one application.

Repair and evolution tasks for the example:
\begin{enumerate}
\item In the application, there will be a programming error causing increased energy consumption of the browser (too many re-renders), which will be eliminated "based on user feedback".
\item According to energy saving design guides, we will evolve the style of the application to decrease frontend energy consumption (i.e. moving from light to dark theme style).
\end{enumerate}

Later, we will focus on the Python backend, measuring it using the pyRAPL library:
\begin{verbatim}
import pyRAPL

pyRAPL.setup( ... )

@pyRAPL.mesureit()
def Application(params):
\end{verbatim}

Here, the task will be to extend the backend. E.g. by an infantry filter, translator or even more advanced AI NLP procedure (supported by the Tensorflow library).

\section{Expected outputs}
We expect the increase of understanding of RAPL reports and profiles, especially their usage during software evolution. Practical experience with using the selected technologies in web application development will point out the similarity between different programming languages when using the same (object or component oriented) pragmas.

Certain level of competition between student groups (we plan to apply pair programming) will be set up based on comparing level of sustainability of the provided versions of the web application. Applications covering the same subset of functionalities will be sorted according to their energy requirements. The more classic competition in functional size will be also applied, thus, the ultimate winners will be the members of the team that implements the most functionalities by reaching a local minimum in energy consumption. We consider this factor as the evaluation of sustainability of the provided software solution:
\begin{itemize}
\item best functionality,
\item lowest energy consumption,
\item best of both worlds.
\end{itemize}


Practically, we expect that the teams will measure the frontends in Firefox and, separately, the Python backends as well. Then, the sum of the measurement results over a 5 minute interval will serve as the value entering the evaluation. Proof of correctness and function coverage of the measurements will be checked based on the submitted source codes and measurement logs.

\section*{Conclusion}
This paper aimed to present topics of the identically entitled lecture at the SusTrainable Summer School in Coimbra. These topics are web application architecture, development and evolution, with an emphasis on energy consumption measurement of selected components of the application architecture. Topics' presentation is enriched by using the illustrative example described in Section 3, more precisely its setup and first evolutionary steps.

The following practical session will provide discussion and competition environment to the attendees to evolutionary improve both frontend and backend functionality and energetic performance.

To be able to fulfil all described tasks, the attendees must have at least basic programming skills in the presented technologies (Typescript, React, Python), and basic soft skills to "survive" in a team of three developers. From technical perspective, there would be the usage of low-edge configurations pointing to more significant improvements, since top developer computers will be not running out of any resources in the lab exercises.

\subsubsection{Acknowledgements} This work acknowledges the support of the ERASMUS+ project “SusTrainable -- Promoting Sustainability as a Fundamental Driver in Software Development Training and Education”, no. 2020–1–PT01–KA203–078646.

%
% ---- Bibliography ----
%
% BibTeX users should specify bibliography style 'splncs04'.
% References will then be sorted and formatted in the correct style.
%
% \bibliographystyle{splncs04}
% \bibliography{mybibliography}
%
\begin{thebibliography}{99}
\bibitem{similarweb}
Similarweb: App Intelligence; A $360^{\circ}$ View Of The Digital World, \url{https://www.similarweb.com/corp/research/mobile-app-intelligence/}. Last accessed 18
Jun 2023

\bibitem{mdntesting}
MDN Web Docs: Tools and testing, \url{https://developer.mozilla.org/en-US/docs/Learn/Tools_and_testing}. Last accessed 18
Jun 2023

\bibitem{swevolutionbook}
Software evolution and feedback: Theory and practice, Edited by Nazim H. Madhavji ... [et al.]. John Wiley \& Sons Ltd, The Atrium, Southern Gate, Chichester, West Sussex PO19 8SQ, England (2006). ISBN-13: 978-0-470-87180-5

\bibitem{raplzhang}
Zhang, H, Hoffmann, H.: A Quantitative Evaluation of the RAPL Power Control System (2014)

\bibitem{raplinaction}
Khan, K. N., Hirki, M., Niemi, T., Nurminen, J. K., Ou, Z.: RAPL in Action: Experiences in Using RAPL for Power Measurements. ACM Trans. Model. Perform. Eval. Comput. Syst. 3, 2, Article 9 (June 2018), 26 pages. \url{https://doi.org/10.1145/3177754}

\bibitem{architecture}
Smith, S.: Architecting Modern Web Applications with ASP.NET Core and Azure. Microsoft Developer Division, .NET, and Visual Studio product teams, A division of Microsoft Corporation, One Microsoft Way, Redmond, Washington (2023)

\bibitem{balkishanprofiling}
Sharma, B.: Web Front End Profiling Client Side Performance Testing, \url{https://www.linkedin.com/pulse/web-front-end-profiling-client-side-performance-testing-sharma}, Published 29 Dec 2020

\bibitem{unit}
Noureddine, A., Rouvoy, R., Seinturier, L.: Unit Testing of Energy Consumption of Software Libraries. International Symposium On Applied Computing (SAC), March 2014, Gyeongju, South Korea. pp.1200-1205.

\end{thebibliography}

\end{document}





\end{document}
