\begin{abstract}%
Among the many trials that our contemporary society faces there is one which stands out as a priority to cultivate a healthy lifestyle on our planet. This issue is climate change and to tackle it, urgent actions are needed not only on the societal, but also on a personal level. The exponential growth of environmental problems stimulates us, as educators, to invent new strategies in the field of Green Education to teach our children how to maintain a more sustainable future. This article discusses the integration and application of sustainable practices through educating Green Gamification at the Plovdiv University “Paisii Hilendarski” (PU). The experiment was performed on first- and fourth-year students majoring in three different IT majors. The study aimed at raising awareness about climate change, the burning of fossil fuels and the crucial importance of developing green behaviours on a both personal and business level.%
\keywords{Environmental Sustainability \and Technical Sustainability \and Sustainable Behavior}%
\end{abstract}%

\section{Sustainable UI and UX}%
\par%
The economic, societal, and educational realities we currently face reveal how vulnerable communities, countries and the entire world can be in all aspects of life. Many alterations are needed for our future generations to live a healthier and more sustainable existence. These ideas should be implemented in the educational systems at an early age to have long lasting effects and change behaviours for a greener future.%
\par%
One interesting method for achieving these goals is teaching sustainability through Green Gamification. This method can be integrated in the educational systems and accompany students in their entire course of study. Furthermore, integrating practices such as developing sustainable UI (user interface) and sustainable UX (user experience) we aim to have a greener and cleaner experience when using the availability of the Internet. The design of the sustainable UI/UX will be the main topic of our SusTrainable Summer School’2023 lecture.%
\par%
There are four key areas where work can be done in terms of building a more sustainable UI/UX~\cite{BIB2}: findability, performance optimisation, design and UX, and green hosting. Findability focuses on making the content as accessible and easy to locate as possible, thus saving energy because users will have to load fewer pages in order to find the desired information. Performance optimization focuses on speeding up the website and thus using less processing power. The design and UX focus on approaches allowing websites to be accessible to all users, regardless of the hardware they use. In turn, green hosting reduces the carbon footprint of a website by switching to a green web hosting company.%

\section{Theories of action}\label{section:overview}%
\par%
There are existent theories of action which can further encourage users to practise green behaviours through models of conduct. Such examples are~\cite{BIB1}: Nudge theory, Hooked model, Mindful design, Theory of interpersonal behaviour, and Transtheoretical model of behaviour change. Likewise, the integration of Green Gamification in our software applications, classrooms and even community models is another method of stimulating and rooting sustainable behavioural practices. %
\par%
To begin with, the Nudge theory~\cite{BIB9} ~\cite{BIB14} suggests it is capable of affecting human behaviour - choices made by individuals can easily be affected in a predictable way without forbidding any options. Such a sustainable design example can be increasing the visibility to affect human behaviour - by giving one of two buttons (options) extra colour and contrast. %
\par%
The second theory – Hooked model~\cite{BIB10} is also called Model for building habit-forming products (sustainable habits), where the 4 key words are Trigger, Action, Variable reward and Investment. The model starts to act with some trigger (e.g. the need), then there is a need for action (to eliminate the trigger). For example, here the UI/UX designer can increase the probability for sustainable action. On the next step action is rewarded (e.g. through likes, levelling up or praising the choice of a sustainable option). The reward should not be predictable, since variability will keep the user’s interest over time. In the end, the reward encourages the user to invest in the choosen action/product in waiting of more rewards. %
\par%
The Mindful design~\cite{BIB11} ~\cite{BIB15} is another way to shape behaviours. Mindfulness is an optimal interaction between attention and awareness. There are 3 designated steps to Mindful design: 1. Identification of the lack of mindfulness (e.g. user intensively scrolling an app), 2. Identification of the mindful solutions (e.g. reminds the users of their sustainable intent), 3. Implementation of the mindful solutions (makes the users mindful they must be disrupted and forced to reflect). %
\par%
The Theory of Interpersonal Behaviour~\cite{BIB12}~\cite{BIB16} is a cognitive model that claims: the individual’s intention is the core factor behind how humans act. The intention is created by the individual's beliefs, social norms and emotions. Habits trigger the intention and facilitating conditions restrict what behaviours are possible. So, the designed solutions should focus on habits.%
\par%
The Transtheoretical model of behaviour change~\cite{BIB13} defines the stages people pass from the moment they start thinking about changing their behaviour to the moment the change is durably achieved – Precontemplation, Contemplation, Preparation, Action, Maintenance and Termination. After the user has gained insight, the design must empower the user to create an action plan: by facilitating information of what is sustainable action the design prepares the user to take action. The design should support the user to take correct actions: by only suggesting options that are considered sustainable. In addition, the design should encourage the user to keep up with the new behaviour and prevent relapse. In the end, the users have to sustain the new behaviour themselves.%
\section{Green Gamification}%
\par%
The second focal point in the theoretical models is the application of Green Gamification on a both community and educational level. By definition~\cite{BIB3}, Gamification is the application of game principles and design elements in non-game contexts. It uses game elements in non-gaming systems to improve user experience and user engagement, as it applies game design to make otherwise boring tasks more engaging~\cite{BIB4}. It hooks us by meeting our basic human needs for achievement, appreciation, reciprocity, and a sense of control over our little corner of life. In today’s competitive battle for attention, games are the most effective tool for leveraging technology, rising above marketing noise and engaging the socially networked consumer. The European Commission Acknowledges that the challenge is to mainstream the application of gaming technologies, design and aesthetics to non-leisure contexts, for social and economic benefits.%
\par%
Gamification consists of elements, techniques, and actions, which all come together to build an educational experience through games. Examples of game elements~\cite{BIB5}~\cite{BIB6} can be avatars, bonuses, badges, combos, rewards, leaderboards, progress, status, teams etc. The game techniques can also vary depending on the material, topic, class and aim of the specific class. Such examples~\cite{BIB5}~\cite{BIB6} can be identity shift, reward system, tracking progress, current status, rules, teamwork, time limit, hidden treasure, feedback etc. The game actions can also differ depending on the educational aim and circumstance e.g., role playing, receiving a bonus or award, gaining an advantage, retrieving resources, following progress level, completing missions etc. %
\par%
According to the Bartle Player quiz~\cite{BIB7}, the author designates 4 specific player types, which possess unique characteristics when engaging in a game: achiever, killer, explorer and socialiser. By nature, the achiever focuses on elements such as: level, badge, bonus, reward, resource, progress and techniques such as a reward system, feedback, challenge, mission, adventure and progress tracking. On the other hand, the profile of the socializer focuses on elements such as teamwork, the usage of avatars, chat and messaging and applies techniques like teamwork, communication etc.%
\par%
Taking a step further into the topic of Green Gamification~\cite{BIB8} we can define the term as the use of game mechanics to engage people and change behaviour, and apply it to sustainability issues.%

\section{PU green experiment}%
\par%
The Green Gamification experiment, performed at the university level consisted of two smaller experiments – an individual and team experiment. The experiment began with a lecture on Sustainability, Sustainable Development and Sustainable Education. After the acquaintance with the core concepts, students were introduced to the idea of Green Gamification. The aim was to highlight the importance of the issues and addressing them through meaningful education on the subject matter. Students were encouraged first to share their prior knowledge on Sustainability and were informed about legislations and sustainable goals both in Europe and worldwide. %
\par%
The first task was performed on an individual level and students were responsible for “creating a green log” each day. This was done using a shared file, where in the course of 1 week, students had to perform green actions, fill in the log which should not repeat previous logs (no matter if the logs were made by other students or were their own). They were in charge of discovering a green activity or behaviour that contributes to Environmental Sustainability. They also had to describe the essence of this action in 100 characters, using relevant vocabulary and explain why their action is sustainable. The game would be won by the one who manages to publish the most activities. This experiment showed that students were very actively motivated and most of them had multiple logs, explained their action’s impact and provided evidence. They shared that they found the task quite interesting and used a lot of humour in their explanations. Students were excited to share their actions in the logs and followed up in class. Students noted they were not so much motivated by the badges rather the new idea they had to generate. %
\par%
The second experiment was titled Team Green Business and was intended to be a group experiment. Students were in charge of building their own business which supports Green Behaviour and Sustainability. Students were encouraged to mind map their ideas in class, filter them and create a project of a business they thought would support green behaviour. Students were provided with helper questions to scaffold and guide them through the task. Such questions were: What does your company do? What kind of specialists do you have? What does your office environment look like? What green habits do you want to introduce in your company? Which unsustainable practices do you want to eliminate from your work environment? What is your position on fossil fuels? Would you work for clients who work for fossil fuel companies?%

\section{Conclusion}%
\par%
The core goals were to motivate sustainable behaviour through games and to provoke students to maintain the sustainable practices in their own urban environment. It is used to incentivise positive changes in behaviour and help young IT professionals to improve their overall quality of life. In addition, through the power of gamification we can make that experience predictable, repeatable, and financially rewarding. %
\par%
The plans for the SusTrainable Summer School'2023 are for the students to acquire and practice skills which will help them create a sustainable UI/UX design through the application of different approaches (incl. Green Gamification). The lecture will present suitable theories, models and approaches that can be used in sustainable UI/UX design. During the lab session students will design a green gamified mobile application that has to encourage/impact some type of sustainability, has to be sustainable from the technical point of view and has to include gamification techniques. In a few days of ongoing collaborative work in preset teams, students will present their ideas on UI/UX design. The training will finish with the evaluation of the ideas by all students and teachers and a subsequent award ceremony for the best works will be held. The evaluation questions for the students' work will be: Do they use Gamification (elements x techniques)?, Do they provide technical sustainable decisions? and Do they impact/encourage the user to adopt sustainable behavior (environmental/economic/social)?. Evaluators also will choose the most original and innovative idea. After completing the training, students will have gained knowledge on many approaches on how to design a sustainable UX/UI and will be able to develop sustainable websites and mobile applications.%

\par%
\textbf{Acknowledgements}%
\par%
This paper acknowledges the support of the Erasmus+ Key Action 2 (Strategic partnership for higher education) project No 2020-1-PT01-KA203-078646: “SusTrainable – Promoting Sustainability as a Fundamental Driver in Software Development Training and Education”. The information and views set out in this paper are those of the authors and do not necessarily reflect the official opinion of the European Union. Neither European Union institutions and bodies, nor any person acting on their behalf may be held responsible for the use which maybe made of the information contained therein.

%\printbibliography%
%\end{document}%
% 