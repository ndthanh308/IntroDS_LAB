\begin{abstract}
Communication middleware is critical in distributed systems. However, many existing distributed middleware only focus on performance. This tutorial concentrates on distributed messaging middleware that balances sustainability, scalability, and performance. Through comparison, let students understand the energy-saving benefits each design detail can bring. Additionally, through the analysis of the overall architecture of the middleware, students will have a deeper understanding of sustainable distributed system architecture. The goal is to give students the ability and awareness to consider sustainability and energy efficiency when developing distributed systems.

\keywords{Distributed communication \and \textsc{Go}.}
\end{abstract}


\section{Introduction}
Distributed systems are increasingly widespread in our daily lives and play an important role.
2 \% of the global CO2 emissions are generated by the ICT (Information and Communication Technology) infrastructure~\cite{en10101470}. Therefore, besides the energy efficiency optimization of the hardware, more and more researchers realized that reducing software energy consumption (SEC) is also essential to improve the sustainability of distributed systems of various data centers.


The energy-aware programmers can use energy analysis and modeling techniques with software engineering tools to reduce the energy consumption of code~\cite{Gallagher17}. The measurement of the energy consumption of software is essential when we teach about green software implementation~\cite{GreenCurriculum}. However, finding the best measurement tool for the distributed system in \textsc{Go} is not easy. Hardware energy measurement tools are very precise and may involve an additional financial cost, while many studies have shown that software energy measurements are unstable~\cite{phdthesis}. The on-chip power sensor is a sub-category of hardware measurement tools that generally does not require extra boards or devices. The RAPL (Running Average Power Limit) is an on-chip power sensor that provides power-limiting features and accurate energy readings for CPUs and DRAM~\cite{rapl1}.

Many important energy consumption studies have been conducted with the help of RAPL~\cite{EnergyWar,PEREIRA2021102609} and Joulemeter~\cite{joulemeterR,GreenRefactorJoulemeterJ}. At the summer school, we will introduce the best practices of using RAPL to measure the energy consumption of distributed systems in \textsc{Go}, and we will also evaluate measurements of the distributed communication system.


The power model using CPU utilization, as the primary signal of machine-level activity, tracks the dynamic power usage behavior extremely well~\cite{cpuu}.
There are many built-in profilers of \textsc{Go} runtime. For example, the CPU profiler~\cite{GolangPerf} shows which functions consume what percent of CPU time. However, it is a software measurement tool that may need to be more stable and accurate, maybe it is only suitable for performance profiling. Therefore, we need to assess its usage in green computing, i.e. can it provide valuable information to locate the energy consumption problems of a program?

The software energy consumption can be reduced at different levels: execution environment level and code level~\cite{phdthesis}. At the code level, code refactoring can significantly impact the energy usage of an application~\cite{coder1,coder2}, so developing energy-efficient software through code refactoring is meaningful~\cite{refactor1}. Therefore, the energy consumption reduction of our distributed communication project starts with sustainable code refactoring.

The energy consumption of small code snippets of \textsc{Java}~\cite{url_Java} using RAPL is reported in~\cite{javaCode1}. It can guide the developer in building energy-efficient software in \textsc{Java}. The book~\cite{refactorBook} also guides the refactoring of object-oriented code. \textsc{Go} is a compiled programming language with efficient built-in concurrency constructs~\cite{url_Golang}. Based on the studies of refactoring, we research the energy consumption of the \textsc{Go} distributed communication system from the code refactoring point of view by evaluating the code snippets summarized from different implementation decisions. We will also evaluate the energy consumption of the essential elements of \textsc{Go}, such as channel, map, slice, and mutex.


There are general energy efficiency strategies that guide green computing implementations~\cite{Conceptual}. However, more practical details and suggestions are needed, which we will provide during the school to acquire green computing in \textsc{Go} distributed system.


Beside the sustainable code refactoring, we also study the sustainability of the distributed systems architecture. The author of~\cite{EnterArch} doubts the transparency of distributed objects based on performance considerations.

The Erlang distributed communication system achieved location transparency by unifying local and remote communication interfaces~\cite{url_Erlang}. It is convenient for the user that does not need to think about whether the destination actor is in the same node or a remote node when sending a message. However, we think location transparency is not a proper sustainability consideration because the users cannot intentionally reduce the number or size of the remote messages if they cannot tell the difference between local and remote sending. We suggest implementing different interfaces for local (in the same node) and remote communications (among different nodes) for sustainability so that users can make sustainable design choices. For example, in case of the \textsc{Go} programming language, the built-in channel construct deals with local communication among different lightweight threads, while other distributed communication systems focus on providing distributed remote communication interfaces.



In the lecture, we will share with students our considerations and measurements of sustainability when designing a distributed system.
We will split different parts of the implementation design of Uactor, such as various group joining algorithms, abstract them into standalone programs, and attempt to make more sustainable design choices for distributed systems by measuring the energy consumption of programs representing different design decisions. We refer to these programs that abstract the energy consumption of distributed systems as "distributed system energy evaluation abstraction programs", which have simplified the distributed system design implementations, not encompassing all the details but reflecting the overall design approach and highlighting some energy-relevant aspects. Then, a comparative analysis with controlled variables will be conducted. Ultimately, after the initial design, before the formal implementation, we can consider which design approach may be more environmentally friendly.


The accuracy of energy consumption evaluation for different designs mainly depends on the design of the abstract evaluation programs, which should contain certain features that can manifest the energy consumption differences among different design decisions or algorithms. We will discuss some experiences designing abstract energy evaluation programs with students in the course.


\section{Uactor project}

The Uactor is a distributed communication system that we are currently developing. We have presented and submitted a tutorial about it at the SusTrainable Summer School 2022~\cite{SummerSchool22}. It integrates our research on distributed systems, including exploring distributed communication models and the coordination of underlying protocols, investigating scalable peer-to-peer networks, and analyzing novel distributed group membership algorithms. Currently, we are designing this project's group membership protocol and connection mechanisms under sustainability and scalability considerations.

Uactor is a connectionless and brokerless distributed messaging middleware that balances sustainability, scalability, and performance. It follows a modified actor model and provides group communication service. The actors are organized in a dynamic overlay of connected non-hierarchical structured overlays. Additionally, this model has unified communication constructs, which means we can effortlessly use different communication mechanisms without connecting different constructs.

We are further improving this project's scalability, so we must redesign its group service mechanisms and architecture. In addition, we will design the architecture of this project with more sustainability features in the next versions. Finally, the practical green design and implementation experience will be shared with the students.

\section{Course information}

This course guides the students in implementing sustainable distributed systems by teaching energy consumption measurement and comparing the energy efficiency of different implementation code snippets. Additionally, we provide sustainable distributed system architecture suggestions and introduce sustainable distributed communication middleware.

Here are some prerequisites for students to understand the course properly: Go programming language, TLA+, and RaPL.

The students should have general knowledge of the essential elements of the programming language \textsc{Go}: variable, functions, if statement, for loop, for range loop, slice, struct, pointer, method, map, goroutine, wait group, mutex, select, and channel~\cite{goBook}.
%The comprehensive introduction to the programming language \textsc{Go} can be found in the book.

TLA+ can assist system architects in modeling the designs or algorithms of distributed systems~\cite{tla}. To describe the design more explicitly, we will summarize a part of the Uactor design using TLA+ specification.

The lecture will first introduce how to measure the energy consumption of \textsc{Go} programs. For example, we teach students how to configure RAPL in the distributed system and evaluate the energy efficiency of their code. The built-in profilers of \textsc{Go} runtime will be used in cases where they provide essential information about energy consumption, such as locating the functions that cause the energy leakage.

After that, we will introduce the energy efficiency suggestions for a distributed system in \textsc{Go} and guide the students to compare the energy efficiency of different implementing decisions. Many decisions in distributed systems are summarized into small code snippets that are easier to understand and test.

Finally, we will introduce the green design of the Uactor project and provide suggestions for designing the green architecture of distributed systems.
The students can modify the prepared \textsc{Go} programs at the practice session according to the green computing suggestions and compare the energy consumption.

After this course, the students will gain more experience in energy consumption measurement and implementing energy-efficient distributed systems in \textsc{Go}. In addition, they will have an insight into the sustainable design of the distributed communication system.


\section*{Acknowledgements}
This  paper  acknowledges  the  support  of  the  Erasmus+ Key Action 2 (Strategic partnership for  higher education) project $No$ 2020-1-PT01-KA203-078646: “SusTrainable – Promoting Sustainability as a Fundamental Driver in Software Development Training and Education”. The information and views set out in this paper are those of the authors and do not necessarily reflect the official opinion of the European Union. Neither European Union institutions and bodies, nor any person acting on their behalf may be held responsible for the use which maybe made of the information contained therein.

%\bibliographystyle{splncs04}
%\bibliography{mybibliography}

%\end{document}
