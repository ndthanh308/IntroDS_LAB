\begin{abstract}
Code comprehension is a daily challenge in a software developer's life. A substantial amount of time is spent with comprehension activities at the expense of productivity. Junior programmers are in the most dire need of proper code comprehension support, since they lack the amount of work experience which facilitates comprehension tasks. There is a wide variety of comprehension supporting software, however, developers are usually not aware of their options, and settle for the insufficient support in code editors. We aspire to raise awareness among programmers about the importance of proper code comprehension support by introducing the CodeCompass code comprehension framework to students. We expect that by showing the practical usage and benefits of a multi-purpose comprehension tool through solving a C++ programming task will instill the need in young programmers for such support who will later indicate their need in the workplace.

\keywords{Code comprehension  \and Sustainability \and Experiment.}
\end{abstract}
%
%
%
\section{Introduction}

Understanding the source code and architecture of a program is part of the daily challenges in a software developer's life, no matter how long they have been working as a programmer. Code comprehension is made difficult by several circumstances: incomplete or missing documentation, scattered knowledge among developers, insufficient comprehension support in code editors, etc. According to various studies, developers spend at least half of their working hours with code comprehension activities, and this amount of time has just been growing as computer science became a more and more paramount part of our lives \cite{corbi1989program, cherubini2007let, minelli2015know, siegmund2016program, xia2017measuring}. The statistics in these studies suggest that programmers are in need of more efficient code comprehension support. More time spent on comprehension activities mean slower task execution and more resource consumption. There are many resource types whose costs can be greatly reduced by using effective code comprehension support: working hours, money and actual energy used by computers are all included.

In this paper, we present the role of code comprehension in resource management by emphasizing the usage of comprehension tools in everyday programming tasks. We also present our plans for the summer school of 2023 in Coimbra, where we intend to popularize code comprehension tools among young developers with the hope that the culture of needing proper code comprehension support in the workplace will become more widespread.

\section{Code comprehension models}

There has been extensive research concerning the source code comprehension strategies of software developers. Program comprehension procedures form well-defined workflows that can be categorized into various code comprehension models which serve as unified abstractions of the comprehension process. Several models have been identified throughout the last couple decades, with two major categories that identify the code comprehension process based on the directions from where programmers approach the source code: the top-down and the bottom-up direction.

Some studies presented reviews of these categories \cite{storey2005theories}, with Mayrhauser and Vans also defining a new comprehension model \cite{von1995program}. Siegmund also summarized the categories in their study \cite{siegmund2016program}, along with an insight of the present and the possible future of code comprehension. We have also published a comprehensive review of the above mentioned code comprehension directions, putting the focus on classifying the function sets of modern code editors and integrated development environments into the categories \cite{fekete2020comprehensive}.

A top-down approach is built when the programmer first tries to build a mental image (hypothesis) of the purpose of the program, then moves on to finer details by comparing a the code to similar known programs. In this case, the programmer is initially familiar with the program domain. The finer details serve as further information which may help accepting, rejecting or modifying the hypothesis \cite{brooks1977towards}. The program domain elements (e.g. programming language syntac, algorithms, coding conventions etc.) provide a basis for the evaluation of the initial assumptions \cite{soloway1984empirical}.

A bottom-up strategy is applied when a programmer does not possess enough domain knowledge, so they will start understanding the code by searching for "pivot points" in the source code. The programmer relies on syntactic and semantic knowledge of a programming language to build a high level mental model of the program \cite{shneiderman1979syntactic}. In other cases, observation does not start at the code itself, but at the control flow of the program \cite{pennington1987stimulus}. Then the programmer considers the programming goals and the control flow to build a hypothesis of the program's operation.

There are comprehension models that include elements from both top-down and bottom-up approaches \cite{levy2019understanding} which are called integrated approaches. These models pay the most attention to the discursive way of human thinking. In this case, the programmer switches between the above discussed directions in an arbitrary way based on the previously existing and recently collected information~\cite{letovsky1987cognitive}.

\section{Code comprehension and sustainability}

In the realm of green computing and sustainability, code comprehension support holds paramount significance. As the world strives to mitigate the environmental impact of computing systems, it becomes imperative to develop software solutions that are energy-efficient, resource-conscious, and eco-friendly. However, achieving such objectives necessitates a comprehensive understanding of the underlying code base. Code comprehension support enables developers to navigate intricate software architectures, identify energy-intensive code segments, and optimize resource utilization. By comprehending the details of the code, developers can make informed decisions regarding algorithmic efficiency, memory management, and power consumption. Furthermore, code comprehension support facilitates the identification of areas for code refactoring, eliminating redundant or inefficient operations, and promoting cleaner, streamlined code. Ultimately, by enhancing code comprehension, developers can significantly contribute to the development of energy-efficient software systems, reducing the carbon footprint of computing and advancing the goals of sustainability in the digital age.

\section{CodeCompass}

CodeCompass \cite{porkolab2018codecompass} is an open-source\footnote{GitHub: \url{https://github.com/Ericsson/CodeCompass}} code comprehension framework developed by Eötvös Loránd University and Ericsson. It consists of a parser and a webserver binary. The CodeCompass parser applies static analysis to the given, ideally compiled source code and the corresponding build commands generated during compilation. Various information is stored afterwards about the project regarding structural data, code metrics, version control information, etc. This information is stored in the workspace database which is then accessed by the CodeCompass webserver. The webserver provides several different textual and graphic services, such as detailed searching, structural and code-level visualizations, and Git blame data.

CodeCompass is a pluginable framework. Each plugin is independent of every other, thus a certain plugin can be easily skipped from the parsing process if not needed. Plugins consist of a \emph{model}, a \emph{parser}, a \emph{service}, and a \emph{web GUI} component. Currently, CodeCompass fully supports C and C++ programs, and is in part capable of parsing C\#, Java, and Python projects. Apart from language parsing, CodeCompass provides code metrics analysis, advanced search functionalities, and Git repository visualizations.

\subsection{Role in education}

CodeCompass is mainly developed in the Model C++ Software Technology laboratory at Eötvös Loránd University, Faculty of Informatics. Students have the opportunity to join the project by solving issues (e.g. bug fixing, small improvements, refactoring), or by taking up a large subproject, such as creating a new plugin, or complementing an already existing one with various new features. Students who join the lab learn about working with various programming languages and APIs, using version control, reviewing other people's code, designing software architecture, etc. \cite{fekete2022building}

The software is integrated with the TMS assignment management system which is developed in our faculty and is used to handle programming assignments. This feature is supposed to facilitate the work of teachers: every submitted assignment can be parsed by CodeCompass in a containerized environment, and the teacher is able to browse the code by launching the webserver. This way the teacher does not have to download each assignment to their computer.

\section{Course information}

In our part of the summer school we plan to show the participants the importance of proper code comprehension support. For this purpose, during the lecture we will introduce the theoretical background of code comprehension along with the feature set of CodeCompass. This way, the students will be familiar with the wide variety of code and program comprehension supporting features that the tool offers; this lecture also serves as a thought provoking session in which we motivate the students to identify their specific needs in code comprehension tasks, e.g. come up with new functionalities based on real life examples from school or workplace.

In the practical session we will present a small C++ project, \emph{TinyXML2}\footnote{GitHub: \url{https://github.com/leethomason/tinyxml2}} which is readily parsed and available on the demo site\footnote{Demo website: \url{https://codecompass.net/}} of CodeCompass. TinyXML2 is a distinguished project that we use to test and present CodeCompass. It consists of three source files which contain hundreds of source code which makes it a perfect test project for code comprehension activities. After a short walk-through of all available functions, we will give the students a minor programming task to students in TinyXML2: they will have to find and modify one specific line in the source code in order to make the tag parsing feature of TinyXML2 case-insensitive. The students will get approximately 60 minutes to solve the task. Based on the results of a previous experiment which we conducted in the spring semester of 2022 with the participation of 27 Computer Science MSc students \cite{fekete2022report}, cc. 60 minutes should be enough for everyone to solve the task.

During the practice the students will be only allowed to use CodeCompass for code browsing and understanding. The software will be available on the demo site with TinyXML2 already parsed. We will anonymously collect information about the students' activities in CodeCompass via Google Analytics. They will be able to modify the code using Visual Studio Code, and compile the code with CMake and Make. The focus of this task is to correctly identify the exact line which has to be modified instead of writing new code; for this reason, we will readily provide the C++ method which has to be invoked to make the parsing case-insensitive. The students need will need fair understanding of C++ or any other imperative language, and a basic computer with 2 to 4 GB of RAM and Internet access. They will not need any knowledge in build systems as the exact commands will be given to them beforehand.

After our session we will analyze and evaluate the activity data. We will also publish various statistics for the students: best, average and medium solution times, most used functionalities in CodeCompass, etc.

\section{Conclusion}

As we discussed, an average programmer spends at least half of their work hours with code comprehension activities. Proper and effective code comprehension support is a very important aspect of software development: it reduces development costs in human resources, finances, and energy consumption. However, standalone comprehension supporting software is not widespread enough to make significant difference in resource usage. Our purpose in the summer school is to popularize code comprehension tools by showing the various functionality they serve. Our plan is to give the participating students a small C++ programming task which they have to solve by using a code comprehension supporting software. Our demo comprehension tool is CodeCompass which is equipped with several features which may help reducing the time spent with comprehension activities, and thus, resource consumption.

\section{Acknowledgement}

This paper acknowledges the support of the Erasmus+ Key Action 2 (Strategic partnership for higher education) project No. 2020-1-PT01-KA203-078646: “SusTrainable - Promoting Sustainability as a Fundamental Driver in Software Development Training and Education”.

The information and views set out in this paper are those of the author(s) and do not necessarily reflect the official opinion of the European Union. Neither the European Union institutions and bodies nor any person acting on their behalf may be held responsible for the use which may be made of the information contained therein.

%\bibliographystyle{splncs04}
%\bibliography{references}
%\end{document}
