\begin{abstract}
We describe a tutorial to be delivered in the second Summer School organized with the Sustrainable - Promoting Sustainability as a Fundamental Driver in Software Development
Training and Education, an Erasmus+ Strategic Partnership.

The tutorial aims to provide software engineers with the knowledge and tools to measure software energy consumption. This is regarded as a first step towards the more ambitious goal of optimizing such consumption. Reducing energy consumption within ICT systems is paramount in realizing a more sustainable exploration of the technology that ever more rules the world.



\keywords{Energy Consumption \and Software Analysis  \and Software Optimization}
\end{abstract}
%
%
%
\section{Introduction}

Sustainability is a crucial driver for the development of modern society and the planet's future.
While groundbreaking progresses make of our age an exciting period for humanity, there exists a growing awareness that progress needs to consider the resources the world can provide.
If we continue over-exploiting such resources, the ability of the next generations to inhabit the planet is endangered.

However, the challenges towards sustainability are tremendous, as they are complex and multifaceted and require expertise from a wide range of disciplines to be addressed. For example, addressing climate change requires expertise in atmospheric science, ecology, engineering, economics, and policy analysis. Similarly, addressing social sustainability challenges such as poverty, inequality, and social justice requires sociology, psychology, political science, and public policy expertise.

Sustainability also entails understanding the interconnectedness of social, environmental, and economic systems. For example, environmental degradation can significantly impact human health and well-being, and economic development can significantly impact environmental sustainability. Understanding and addressing these interconnections requires interdisciplinary collaboration and a holistic approach to problem-solving.


Our work focuses on addressing and promoting sustainability within software development and engineering.

As a broad discipline, Sustainable Software Development/Engineering encompasses various aspects, ranging from economic to social, including technological, technical, and environmental factors~\cite{lago15}.
The social perspective explores the use of software to enhance people's quality of life, while the economic perspective emphasizes the development of products that can endure for an extended period. Additionally, from a software development standpoint, technical sustainability encourages the production of high-quality software, which in turn promotes reusability and reduces future development efforts and resource consumption.

One significant aspect that must be considered is environmental sustainability. It focuses on producing and utilizing software with minimal environmental impact, aiming to minimize resource use and maximize energy efficiency. This is a critical concern.

The emergence of large-scale cloud deployment models has led to the establishment of massive data centers with significant energy consumption, raising environmental sustainability concerns.
A 2020 study by the EU Commission revealed that data centers in the former 28 EU countries experienced a substantial increase in energy consumption. In 2010, they consumed 53.9 terawatt-hours, which escalated to 76.8 terawatt-hours by 2018. This significant rise accounted for 2.7\% of the total electricity demand in the EU.

While, in practice, the hardware components of ICT systems consume energy, their software counterparts govern how to run the hardware and how and when energy is consumed. This means that achieving the goals of Sustainability, particularly under its environmental lenses, is only possible by targeting Sustainable Software Development.

The relevance of addressing research in this line is confirmed by Software Developers themselves: it has been shown that Software Developers are indeed keen on developing energy-efficient software~\cite{pinto2014mining,pang2016programmers}. And studies have shown that different design patterns~\cite{DBLP:books/sp/21/Feitosa00F0S21}, sorting algorithms~\cite{bunse2009exploring}, software version changes~\cite{DBLP:journals/ese/Hindle15}, refactorings and transformations~\cite{park2014investigation,DBLP:conf/wcre/0001SF20}, and different Java based collections~\cite{Pereira:2016:IJC:2896967.2896968,hasan2016energy} have a statistically significant impact on energy usage.
Studies have also shown how even the choice of the programming languages to use influences energy usage~\cite{pereira2017sle,DBLP:journals/scp/PereiraCRRCFS21}.

While there is promising evidence of the feasibility of improving energy efficiency by targeting software components of ICT systems, the fact is that this topic is clearly under-represented in the education of modern software engineers.
This paper describes a summer school tutorial that aims to provide junior software engineers with the motivation, knowledge, and tools to measure, analyze and ultimately analyze the energy consumption of software. The structure and content of the tutorial we propose are described in detail in the next section.


\section{Structure and Contents of the Proposed Tutorial}

The tutorial will cover the essential knowledge and skills needed to understand and optimize software with energy consumption in mind. It is divided into two parts, a theoretical lecture to introduce key concepts and a hands-on laboratory session to become familiar with collecting and comparing energy consumption of software.
At the end of this tutorial, the participants will have acquired:

\begin{enumerate}
    \item [$\mathbf{O1}$] knowledge about what affects the energy consumption of software;
    \item [$\mathbf{O2}$] skills in measuring the energy consumption of software.
\end{enumerate}

\subsection{Theoretical Lecture}
A great way of demonstrating the importance of keeping energy efficiency in mind when developing software is to look at the energy consumption trends of the ICT sector, of which software is a very large part. Thus, showing the participants this sector's past consumption and forecasts for the near future is the first matter addressed in the lecture.

In order to successfully design and develop energy-efficient software, it is crucial to possess more than just a basic familiarity with energy consumption and efficiency concepts. It is essential to have a deep and comprehensive understanding of the various factors that influence consumption. Furthermore, one must be well-versed in the techniques and tools that are at our disposal for accurately measuring and optimizing energy efficiency in software development.

In software applications, energy consumption refers to the amount of energy the application consumes during runtime. Regarding energy efficiency, it is defined in~\cite{energy_efficiency_2} as "(...) using less energy for the same output or producing more with the same energy input (...)". Ultimately energy efficiency is the ratio between energy consumed and outputs produced, as it can be improved by reducing consumption and maintaining the output, maintaining consumption and increasing the output, or simultaneously reducing consumption and increasing the output.

To better understand how different factors influence the energy consumption of software we can abstract a computer into 7 layers, presented in Figure~\ref{fig:stack}, providing us with a framework to understand the different levels of software and hardware involved in computing. At the top of the stack, algorithms represent the highest level of abstraction and are the most energy-agnostic. As we move down the stack, we encounter more specific and concrete layers, which can be optimized with some effort. And at the bottom of the stack are the most hardware-specific layers where energy efficiency can be most effectively optimized. Understanding this stack, and relating it to the factors identified as influential in the energy consumption in~\cite{cloud_survey}, can help identify opportunities to improve efficiency at various system levels.

% Figure environment removed

The participants will be introduced to energy measurement and modeling, two different approaches to studying the energy consumption of software, each with its advantages and limitations. While energy measuring is considered the ground truth, as it provides accurate and precise measurements, it requires specialized hardware which may be expensive to acquire.
On the other hand, energy modeling approaches estimate energy consumption, resulting in values that may not be totally accurate but instead estimates that are relatively accurate, i.e., the relative distance between two measurements is correct. Still, the actual values may differ from the ground truth. Energy estimators are a more cost-effective approach that does not require specialized hardware.

The lecture will introduce Intel's Running Average Power Limit, or RAPL, an energy estimator that has proven itself to be reliable and accurate~\cite{pl:rank2017,tec:rapl1,tec:rapl2}, and has been used in multiple studies~\cite{pl:rank2017,ds:haskell,pereira2017sle,DBLP:journals/scp/PereiraCRRCFS21,Pereira:2016:IJC:2896967.2896968}. However, this tool has its limitations and the authors of~\cite{rapl_in_action} do a great job of pointing them out and presenting mitigation strategies when possible. Some of these include a lack of granularity, register overflows, and non-atomic updates. Therefore, it is important to be aware of some good practices when taking measurements, like disabling non-required services or WiFi connection, to reduce the noise in the system. Alternatively, one could use other tools, such as pTop~\cite{tec:ptop} or PowerTop\footnote{\url{https://github.com/fenrus75/powertop}, accessed 13/06/2023}, which overcome some of RAPL's limitations even though they might introduce some of their own. Nonetheless, for this tutorial, an overview of RAPL will be provided, so that the participants may realize their own experiments in the hands-on session.


\subsection{Practical Lecture}
The purpose of the hands-on lecture is to provide participants with an opportunity to put their recently acquired knowledge into practice. Each participant will be furnished with a laptop equipped with all the necessary materials and software for seamless execution.

The initial task involves two parts. Firstly, participants need to choose between programs with diverse functionality, such as sorting, provided in various programming languages that encompass different programming paradigms, including functional, object-oriented, and imperative. This allows participants to select programs and languages they are familiar with. Secondly, participants are required to utilize RAPL to measure the energy consumption of these selected programs. Once completed, the participants will have their first experience in gathering information about energy consumption.

In the second task, participants will be required to modify the chosen program while preserving its functionality, aiming to improve performance or make changes to implementation details. Following these modifications, participants will once again measure energy consumption. The number of measurements to be performed is up to the participant. Still, they should collect enough samples to assess whether or not the modifications influenced energy consumption.


\subsection{Reaching the Learning Outcomes}
To ensure that the participants acquire the learning outcomes of this tutorial, we have designed a comprehensive program that integrates theory and practice.

Regarding $\mathbf{O1}$, the theoretical lecture will introduce the key concepts of energy consumption in the software context, as well as some factors that affect energy consumption. To consolidate these concepts, the participants will be given the opportunity to experiment with software and modify them to observe the resulting energy consumption, on the hands-on laboratory session. This will enable the participants to develop a deeper understanding of how the various factors affect energy consumption in software.

When it comes to $\mathbf{O2}$, the participants will learn the difference between energy modeling and measuring, including the advantages and drawbacks of each method, during the theoretical lecture. Furthermore, they will also be introduced to RAPL and good practices for using tools such as this. In the practical session, participants will be required to measure the energy consumption of some programs, putting into practice what they have learned in the theoretical lecture.


\section{Conclusion}

This paper describes a summer school tutorial that aims to promote sustainability in software development and engineering. The tutorial provides participants with the knowledge and skills to optimize software with energy consumption in mind. Through a theoretical lecture and a hands-on laboratory session, participants will gain an understanding of the factors affecting energy consumption in software and how to measure it effectively.

By the end of the tutorial, participants will have acquired knowledge about what affects the energy consumption of software and the skills needed to measure and optimize it. This tutorial has the potential to make a significant impact in promoting sustainability in the field of software development and engineering. By equipping junior software engineers with the tools they need to make informed decisions about energy consumption, we can contribute to a more sustainable future.


\section*{Acknowledgements}

This work acknowledges the support of the ERASMUS+ project “SusTrainable —
Promoting Sustainability as a Fundamental Driver in Software Development
Training and Education”, no. 2020–1–PT01–KA203–078646.

%
% ---- Bibliography ----
%
% BibTeX users should specify bibliography style 'splncs04'.
% References will then be sorted and formatted in the correct style.
%
%\bibliographystyle{splncs04}
%\bibliography{mybibliography}
%

%\end{document}
