\begin{abstract}
Sustainability is a globally shared goal and is used to drive technology evolution for the prosperity of people and the planet in the future. Software is central to modern technology solutions. The way how we develop, design, and deliver software solutions significantly impacts technology operation sustainability. Therefore, in this lecture, we aim to describe the challenges to reaching sustainable goals while developing, designing, and delivering software in the new network architecture, and explain the relationship between software autonomy and technology sustainability. In the central focus of the lecture are design principles for building autonomous software architectures as a key driver for sustainable technology evolution. Finally, we discuss practical examples and software engineering practices in relation to these challenges. Note that our particular interests are related to software delivered in Clouds, as blockchains of microservices within future networks.

\keywords{design principles \and autonomy \and sustainability \and software \and future networks}
\end{abstract}
%

\section{Introduction}
\label{sect:Intro}

Software is interconnected and highly integrated within telecommunication networks and is viewed as a key enabler for future digitalization, which is one of the main prerequisites for achieving sustainable goals in various domains (e.g. e-health, e-government, smart water, smart city, etc) \cite{DigitSust}. Future networks are preparing for autonomous machines (cars, drones, robots) and their continuous communication, in order to accommodate the continuous flow of massive amounts of data and their use for effective management solutions \cite{Ericsson5G}. Furthermore, there is a progressive development of applications that can make decisions with the help of artificial intelligence models. The idea is to converge to a global and highly interconnected network of things that continuously share massive amounts of data and autonomously adapt to environmental conditions \cite{AInetwork}. However, future networks should be designed not only to meet technical requirements for meeting people's purposes and achieving economic goals in various domains, but network designs should also look beyond these technical issues to foresee their impacts on environmental conditions aiming to co-exist on Earth over a long time.


Further evolution of information-communication technologies (ICT) should secure its sustainable operation in every domain of application. Network, as a central part of modern ICT technologies, is evolving by developing network autonomy in which sustainable network designs would be based on dynamic network adaptations fulfilling the aforementioned goals. In our previous lectures \cite{GalinacRole1,GalinacRole2}, we presented the main enabler technologies for network autonomy and its sustainable behavior, such as software-defined networking and virtual network functions. In order to achieve its full potential of optimized infrastructure and time-effective response to customer, networks have implemented these main architectural elements as enablers for future networks. The network has prepared its technology for rising its level of autonomy. More precisely, network autonomy rises with the wide implementation of self–management functions such as self–configuration, self–healing, self–optimization, self–protection, \cite{Autonomicbook}. Recent research trends focus on developing adequate software algorithms and solutions to support the aforementioned self-management functions, \cite{AInetwork,AICloudRM}. In this lecture, we will introduce key design principles that are necessary to guide the designs of self-management functions.


Evolution is driven by the idea to standardize and automatize processes within the systems so that we can systematically monitor and measure their behavior and minimize human intervention. One of the main goals of autonomic systems is to manage system complexity and introduce the system's self-organized capability. For such a purpose IBM has introduced a simple control loop frequently referred as Monitor, Analyze, Plan, and Control, MAPE, \cite{IBMMAPE}. The MAPE loop model assumes the existence of sensors within the system that can measure operations and the effectors that can issue system commands on the system's managed entity. This sequential autonomic control loop makes two key assumptions. Firstly, it assumes the existence of directly controlled system entities, and secondly, it assumes the standardized interfaces for system entity monitor and control. However, in most technical systems these conditions are not satisfied. Nowadays, software is often delivered and deployed over the underlying network infrastructure, in Cloud environments and it is often impossible to achieve direct monitoring and control of particular system entities. There is currently a vast amount of different software engineering frameworks and technologies that support developers in their system engineering activities to deliver software products within Network and Cloud environments (e.g. various Web application development frameworks like for example React, Angular, Vue, etc.). Software products are concurrently assessing the shared pool of network and Cloud resources.  The main problems arise with performance aspects under their concurrent operation when software products are offered to the numerous end users,\cite{PerfomanceReact,PerformanceWeb}. In such environments, the software product business goals are balanced between the number of users concurrently using these services and service performances achieved during concurrent software access to shared resources. Thus, in order to optimize their business goals, some in-service monitor and measure solutions are usually proposed, \cite{Webinserviceperf}. However, it is often impossible to implement in-service control of the underlying shared resource pool. From the network perspective, the coexistence of a variety of these technologies in the runtime environment makes significant operational problems in terms of their autonomic and sustainable management. It becomes challenging to automatically manage numerous technologies and find the balance between network and service layer self-adaptive behaviors. One of the main problems for future network evolution would be to find new architectural models, in which compromise between network and service operation would be achieved by satisfying their combined sustainable behavior. In the next decade, our goal should be to move from the \emph{Cloud computing} paradigm to the \emph{Responsive computing} paradigm, in which we should develop software engineering skills and tools that would enable engineering of sustainable technology systems.

One of the main problems is that we still engineer these systems from the technological vision perspective and not from the operational behavior, reliability, or sustainability perspective. We still lack software engineering technology for engineering these so-called *ility attributes into our systems. The same is valid for recently identified vital software characteristic that is related to its sustainable operation. Software architecture is the main artifact we engineer during the software development process and the results of our engineering approach is reflecting the operational characteristics of our final software product. For example, the way how we design software parts and their interactions (that we refer to as system structure design) has a significant impact on system performance, reliability, security, etc. Our approach is to reverse the traditional top-down deductive systems engineering approach into a bottom-up inductive way, and towards framing inside-out design patterns \cite{Design4finSus}. We would like here to bring attention to considering autonomic networking design principles when studying software structure designs for software sustainable operation. Architecture designs should introduce light system simulations that would observe system behavior through a sequence of accumulating events and concurrent executions. Moreover, designing systems architecture should aim to implement mechanisms that would be able to adapt at runtime. The essence of sustainability design is to prioritize internal needs over external needs while keeping the system execution within its limits. Networking design principles are biologically inspired principles that try to translate these nature's adaptivity driven by the instinct to survive into the network. Here sustainable skills would refer to students' abilities of systems thinking in these adaptive designs.


In this lecture, we will teach students about design principles as key skills that would be needed to shape future network and software delivery architectures. The lectures would be supported by the tools that we developed to enable the design of sustainable software structures. Therefore, in the next section Sect.~\ref{sect:dp}, we introduce the main sustainable design principles that are used to implement autonomic network behavior. In Sect.~\ref{sect:ss} we explain our approach to teaching and promoting sustainable design principles within software architecture design.


\section{Design principles for autonomic systems}
\label{sect:dp}

As a starting point of discussion, we need to clarify the basic terms and try to make clear distinctions among them. There are three frequently mixed terms: automatic, autonomous, and autonomic, \cite{Autonomicbook}.

The term automatic refers to an action that is a spontaneous reaction of the system not related to any guided rule but some stimulants are always resulting in the same automatically produced action. From the computer science and programming perspective, we can interpret this term in the sense of functional programming in which we want that functions are consistent and always return the same result on the given input.

An autonomic system is defined as a system that exhibits some degree of self-governance, however, the self--governance is achieved solely from the system's inside processes and has no relation to outside stimuli. The concept of the autonomic system is reused from biology (e.g. autonomic nervous system) and is representing only a part of the nervous system that is responsible for the control of spontaneous or autonomic functions of the human body that are preconditions to survive, like breathing, heart beating etc. These processes are executed outside the human conscience and represent continuous background human activity. It is important to understand that the human body spends a lot of energy to execute solely these processes, without any additional activity.

The autonomous system is viewed as a self-governed system based on some internally system-defined policies and principles. Moreover, this system is self-controlled and functionality independent of the outside world. Note that this concept assumes the system sensing ability which enables the autonomous system to develop its own knowledge and thus make further decisions and make control over the system. All living systems have some autonomous behavior and inherently built-in mechanisms to support such autonomous behavior.

The aforementioned concept of the autonomic system is derived into the communication network context with the help of autonomic design principles. These autonomic design principles have been widely studied while planning the network evolution and as a core philosophy for developing new network architectures with high network management autonomy. Although there are numerous viewpoints on sustainability principles, here we will reflect on sustainable principles for autonomic network behavior. The following principles were recognized as valid in networks context \cite{Autonomicbook} that we will briefly discuss in our lecture.
\begin{itemize}
    \item Living systems inspired design
    \item Policy-based design
    \item Context awareness design
    \item Self–similarity
    \item Adaptive design
    \item Knowledge-based design
\end{itemize}



\subsection{Living system inspired design}
\label{subsect:livingsys}
Software design should follow the behavior of living systems. All living systems exhibit high levels of autonomy and the designs of systems we engineer can be highly inspired from these biological examples. In particular, there are two perspectives to explore in the context of living systems. These are survivability and collective behavior.

Observation of the instinct to survive in the living systems has always concluded with the fact that living systems work toward keeping the equilibrium state and any deviations caused by environmental conditions are forcing the system to bring back to this initial system state \cite{LivingSyst}. There are numerous adaptation mechanisms of living systems that can be reused also for human engineered systems such as software architecture.

The other interesting mechanism of living systems is their collective behavior, which is about system social characteristics while aiming to adapt to a group to which it belongs. There are some interesting studies on the influence of networking technologies like Facebook on social collective behavior, \cite{Loc2glob}. It is interesting how some local information has gained significant importance from the numerous local pieces of information and taken dominance in guiding system global behavior. Some of these concepts may be useful while engineering sustainable software system structures.

\subsection{Policy based design}
\label{subsect:policy}
Policy-based design means that there exists a predefined rule that governs the system's behavior. This design principle has been already used in software system design in which system behavior must adhere to different rules of behavior and these behaviors are specified during the system design phase. Then, based on some parameter settings within the system, the system behavior is chosen at system runtime.

This approach has two drawbacks. Firstly, it requires policy definition during the design time, thus limiting system dynamic adaptation to environmental conditions. Furthermore, we have to secure adequate conflict resolution mechanisms that may be needed in the intersection of system behaviors driven by different policies. One example of such design is software that can serve the same purpose in different conditions, i.e., some standard protocol implementation that has variant implementations in different markets and for these markets, some standard protocol behavior involves some specific feature.



\subsection{Context awareness design}
\label{subsect:context}
Context-awareness has been already used within computer science and is a concept related to the ability of the system to characterize the environment and adapt its behavior to the current situation parameters and knowledge gained from its historic behavior under the same conditions. The main challenge in applying this design principle is in the selection of appropriate variables that may be measured, and sampling strategies to adequately model and capture all relevant states and their interaction in order to effectively and efficiently recognize relevant environment and situation conditions.


\subsection{Self–similarity}
\label{subsect:similarity}
Self-similarity refers to the similarity of system organization on different scales. More precisely, the self–similarity design principle is related to a characteristic that system organization persists as the various system scales and thus guarantees its global properties. Here, it is the main point to develop system functions that preserve system scalability. The same functions may be used as the system is growing in hierarchy and functionality. Thus, further system evolution may be systematically enabled by a proper system architecture that minimizes the need for system change during the system evolution. Moreover, the system properties and system behavior at a large scale have to be the same as the system properties and system behavior at a low scale.


\subsection{Adaptive design}
\label{subsect:adaptive}
Adaptive design is related to the ability of the system to adapt its inner behavior as a reaction to various environmental conditions. Such a system is able to learn from its experience in operation and react accordingly by adapting its actions based on collected information and knowledge gained. Over time the system acquires new knowledge and is better adapting to the local environmental conditions of its operation.

\subsection{Knowledge-based design}
\label{subsect:knowledge}

Design based on the knowledge extracted from big data gathered from the complex system (using AI and other models) differs from the previous adaptive design concept in terms that it assumes some global data collection and artificial global intelligence that can be used to guide local decisions.



\section{Content of the summer school lecture and educational goals}
\label{sect:ss}

Education in software architecture should prepare students for their profession in a new sustainable world. One of the primary steps is to integrate sustainability principles into architecture design courses.

We will formally start the lecture by revising traditional system design skills, and traditional teaching approaches. Here we will introduce some basic definitions. System design is concerned with system decomposition into its components. The design decisions are fundamental for successful implementation and evolution of the system, \cite{Vliet}. The main output of the design phase is the software structure that is defined as a set of components and their mutual dependencies, \cite{Vliet}. The system structure is usually depicted by the graph that represents 'the uses' relation among the system components, which is called the call graph. During the system design phase, there are no clear rules on how to perform successful system designs. The system design rather offers design principles that we reviewed in \cite{GalinacRole1} such as abstraction, modularity, information hiding, layering and hierarchy. Traditional designing system architecture involves exploring various system structure designs in terms of requirements defined in the requirements phase. However, here we aim to rethink this approach and explore widening this narrow design principle list with design principles defined in the autonomous systems concept and explained in Sect.~\ref{sect:dp}. We will face here two challenging issues.

Firstly, we need to introduce novel learning approaches to engage students' abstraction abilities. Some studies \cite{RoleofFluid} have identified that fluid intelligence has a significant role in biological phenomena abstraction and that understanding of bioinspired design is an essential part for design creativity.

Secondly, we promote here a system thinking design approach \cite{systdynamics} that is reflecting on migration from a top-down system design analysis to a bottom-up approach, in which the system structure is examined from the system dynamics perspective. The system designer has to take here a more active role, more as a system experimenter than just a passive critical structure analyst's observer. In such a changed role, the system designer should develop skills that are related to the following: modeling of system simulation by formulating simulation formulas and models, coping with the understanding of system behavior, evaluation of system policies by the development of a control group, development and selection of appropriate treatment cases, establishing a hypothesis, ability to monitor and compare the results for different treatments, ability to judge the results, develop models that may represent intervention actions. Software structure designs are usually governed by the technical vision and usually rely solely on human unreliable intuition. In this lecture, we will try to present our ideas on a simple toy example by using popular among the Academy Python technology.


At the summer school, we plan to structure lectures as follows:

\begin{itemize}
    \item Introduction to System Thinking and System Dynamics
    \item Software as a concurrent system: Python skeleton
    \item Simulating software behavior
    \item Development of simulation scripts
    \item Observing system dynamics within system boundaries
    \item Python GUI for user monitor
    \item Autonomic system design principles
    \item Implementation of the student version of sustainable design principle
\end{itemize}
Firstly, we plan to explain the meaning of control MAPE loop and present its implementation within the Python environment. Some implementations of MAPE loop in Python environment are available publicly\footnote{\texttt{https://github.com/elbowz/PyMAPE}}. We will demonstrate how to implement Python functions for monitoring and control of software behavior by using public Python libraries.

In the second part of the lecture, we will describe system thinking approaches and system dynamics perspective. As a practical example, we plan to ask students to implement a simplistic version of a Python application. Then, we will explain how to implement a Python simulation script that can be used to monitor Python application behavior. Furthermore, we present several practical software implementations of bioinspired design principles and provoke students to play with simulations and observe software behavior. Finally, we motivate students to develop their own solutions, so that students can easily experiment with their versions of implementation of autonomic design principles in a predefined context.

The intended audience for the lectures is undergraduate or graduate students of Computing study programs that have previous experience in Python programming. For active exercising, it is required that students have PC or laptops with installed Python IDE.

The key learning outcomes will be on understanding how to measure software behavior, the ability to develop its own simulation models for capturing software behavior and thus understand how to use a bottom-up approach to system design, and understanding how bioinspired design principles may improve self-system management and thus affects sustainable goals.



\section{Conclusion}
\label{sect:conc}
In this lecture we provide a bottom--up approach to software design integrating system thinking and system dynamics skills into the software engineering curriculum. We focus our system design on the implementation of autonomic system concepts within software applications and provide autonomic design principles to as guidance for students. Throughout the lecture, we undertake an exercise-driven approach to learning autonomic design principles and orientation to system thinking and system dynamics design.



\section*{Acknowledgments}

This paper acknowledges the support of the Erasmus+ Key Action 2 (Strategic partnership for higher education) project No. 2020–1–PT01–KA203–078646: “SusTrainable - Promoting Sustainability as a Fundamental Driver in Software Development Training and Education” and the support of the Croatian Science Foundation under the project HRZZ-IP-2019-04-4216. The information and views set out in this paper are those of the author(s) and do not necessarily reflect the official opinion of the European Union. Neither the European Union institutions and bodies nor any person acting on their behalf may be held responsible for the use which may be made of the information contained therein.

\begin{thebibliography}{99}

\bibitem{Autonomicbook}
N. Agoulmine (ed.), \textit{Autonomic Network Management Principles: from Concepts to Applications}, Elsevier Inc. (2011). \doi{10.1016/C2009-0-62958-5}.

\bibitem{Webinserviceperf}
Y. Amannejad, D. Krishnamurthy, B. Far, Managing performance interference in Cloud-based web services,
\textit{IEEE Transactions on Network and Service Management} 12 (2015), no. 3,
320-333. \doi{10.1109/TNSM.2015.2456172}.

\bibitem{LagoSurvey}
N. Condori-Fernandez, P. Lago,
Characterizing the contribution of quality requirements to software sustainability,
\textit{Journal of Systems and Software} 137 (2018), 289-305.

\bibitem{TDA}
H. Edelsbrunner, J. L. Harer,
\textit{Computational Topology: An Introduction},
American Mathematical Society, Providence RI, 2010.

\bibitem{systdynamics}
J. W. Forrester, System dynamics, systems thinking, and soft OR,
\textit{Syst. Dyn. Rev.} 10 (1994), 245-256. \doi{10.1002/sdr.4260100211}.

\bibitem{GalinacRole1}
T. Galinac Grbac, The Role of Functional Programming in Management and Orchestration of Virtualized Network Resources Part I. System structure for Complex Systems and Design Principles. CoRR abs/2107.12136 (2021)

\bibitem{GalinacRole2}
T. Galinac Grbac, N. Domazet, The Role of Functional Programming in Management and Orchestration of Virtualized Network Resources Part II. Network Evolution and Design Principles. CoRR abs/2107.12227 (2021)

\bibitem{ICT4S2}
L. M. Hilty, B. Aebischer,
ICT for sustainability: an emerging research field,
In: L. M. Hilty, B. Aebischer (eds.),
\textit{ICT Innovations for Sustainability}, pp. 3-36,
\textit{Advances in Intelligent Systems and Computing}
vol. 310,
Springer, Cham, 2015.

\bibitem{ICT4S1}
L. M. Hilty, P. Arnfalk, L. Erdmann, J. Goodman, M. Lehmann, P. A. W\"{a}ger,
The relevance of information and communication technologies for environmental sustainability: A prospective simulation study,
\textit{Environmental Modelling \& Software} 21 (2006), 1618–1629.

\bibitem{IBMMAPE}
IBM Corporation,
An architectural blueprint for autonomic computing,
Technical Report, 2005.

\bibitem{PerfomanceReact}
A. Javeed, Performance optimization techniques for ReactJS, in: 2019 IEEE International Conference on Electrical, Computer and Communication Technologies (ICECCT), Coimbatore, India, 2019, pp. 1-5. \doi{10.1109/ICECCT.2019.8869134}.

\bibitem{FramingSustainability}
P. Lago, S. A. Ko\c{c}ak, I. Crnkovi\'{c}, B. Penzenstadler,
Framing sustainability as a property of software quality,
\textit{Commun. ACM} 58 (2015), 70–78.

\bibitem{LagoGreen}
P. Lago, N. Meyer, M. Morisio, H. A. M\"{u}ller, G. Scanniello,
Leveraging ``energy efficiency to software users'': summary of the second GREENS workshop,
\textit{SIGSOFT Softw. Eng. Notes}
39 (2014), no. 1, 36–38.

\bibitem{Petric}
J. Petri\'{c}, T. Galinac Grbac,
Software structure evolution and relation to system defectiveness,
In: EASE 2014, pp. 34:1-34:10.

\bibitem{Design4finSus}
J. Thomas, P. Mantri, Design for financial sustainability,
\textit{Patterns} 3 (2022), no. 9, Art. no. 100585 (32 pages).

\bibitem{SoftwareGraph}
S. Valverde, R. Sole,
Network motifs in computational graphs: A case study in software architecture,
\textit{Physical review. E, Statistical, nonlinear, and soft matter physics} 72 (2005), Art. no. 026107 (8 pages).

\bibitem{Vliet}
H. van Vliet,
\textit{Software Engineering: Principles and Practice},
John Wiley \& Sons, 2008.

\bibitem{PerformanceWeb}
Y. Yao and J. Xia, Analysis and research on the performance optimization of Web application system in high concurrency environment, in: 2016 IEEE Information Technology, Networking, Electronic and Automation Control Conference, Chongqing, China, 2016, pp. 321-326. \doi{10.1109/ITNEC.2016.7560374}.

\bibitem{Ericsson5G}
Dohler, M., ; Nakamura, T. (2016). 5G Mobile and Wireless Communications Technology (A. Osseiran, J. Monserrat, P. Marsch, Eds.). Cambridge: Cambridge University Press.

\bibitem{DigitSust}
Mondejar, Maria; Avtar, Ram; Baños Diaz, Heyker ;Dubey, Rama ; Esteban, Jesús ; Gómez-Morales, Abigail, Hallam, Brett, Mbungu, Nsilulu; Okolo, Chukwuebuka ; Kumar, Arun ; She, Qianhong ; Garcia-Segura, Sergi. (2021). Digitalization to achieve sustainable development goals: Steps towards a Smart Green Planet. Science of The Total Environment. 794. 148539. 10.1016/j.scitotenv.2021.148539.

\bibitem{AInetwork}
B. Mao, F. Tang, Y. Kawamoto and N. Kato, "AI Models for Green Communications Towards 6G," in IEEE Communications Surveys and Tutorials, vol. 24, no. 1, pp. 210-247, Firstquarter 2022

\bibitem{AICloudRM}
Shreshth Tuli, Sukhpal Singh Gill, Minxian Xu, Peter Garraghan, Rami Bahsoon, Schahram Dustdar, Rizos Sakellariou, Omer Rana, Rajkumar Buyya, Giuliano Casale, Nicholas R. Jennings, HUNTER: AI based holistic resource management for sustainable cloud computing, Journal of Systems and Software, Volume 184, 2022, 111124, ISSN 0164-1212.

\bibitem{LivingSyst}
James G. Miller, I: The nature of living systems, Biosystems, Volume 4, Issue 2, 1972, Pages 55-77, ISSN 0303-2647.

\bibitem{Loc2glob}
Carmen Leong, Isam Faik, Felix T.C. Tan, Barney Tan, Ying Hooi Khoo,
Digital organizing of a global social movement: From connective to collective action, Information and Organization, Volume 30, Issue 4, 2020, 100324, ISSN 1471-7727.

\bibitem{RoleofFluid}
Hung-Hsiang Wang, Xiaotian Deng,
The role of fluid intelligence in creativity: The case of bio-inspired design,
Thinking Skills and Creativity, Volume 45, 2022, 101059, ISSN 1871-1871.

\end{thebibliography}
%\end{document}
