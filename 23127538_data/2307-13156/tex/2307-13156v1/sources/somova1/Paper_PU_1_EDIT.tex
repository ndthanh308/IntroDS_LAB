\begin{abstract}%
Environmental protection is a key topic nowadays. Many world and European organizations and bodies (with their respective documents) fight against the global negative impact of industries such as fossil fuels. The work of IT specialists could influence other sectors of sustainability in both positive and negative ways. The paper presents a questionnaire, exploring students' attitude to sustainability and the place of sustainability in higher education and future employment. The survey will be conducted during the Second SusTrainable Summer school.%
\keywords{Environmental Sustainability \and Technical Sustainability \and Sustainable Behavior}%
\end{abstract}%
\section{Introduction}%
\par%
Sustainability has many dimensions - environmental, economic, social, technical, educational, cultural, etc., as environmental sustainability will lead to the most long-term and positive global consequences. For example, the negative effects of burning fossil fuels (coal, oil and natural gas) are well-known: air pollution, water pollution, climate change and health problems.%
\par%
The work of IT specialists could also influence sustainability in other sectors. Something more, IT professionals can impact environmental sustainability, both positively and negatively – positively by reducing the impact of IT on the environment (Green IT) and applying IT to improve sustainability (\#Tech4Good), and negatively by supporting and working for fossil fuel companies and other unsustainable sectors (\#Tech4Bad)~\cite{1BIB1}.%
\par%
This work presents the survey aimed at exploring IT specialists (students and teachers)’ attitude towards sustainability (in particular to the fossil fuel industry) and the place of sustainability in higher education and future employment. The survey is going to be carried out with the participants of the Second SusTrainable Summer School.
%

\section{Global and European political and legislation issues}\label{section:overview2}%
\par%
The global and European policy is directed to environmental protection incl. first reduction and late cessation of the fossil fuel industry:%
\begin{itemize}%
\item The United Nations Framework Convention on Climate Change~\cite{1BIB2}, signed at the United Nations Conference on Environment and Development in 1992;
\item The Paris Agreement~\cite{1BIB3}, adopted at COP 21 in 2015;%
\item The United Nations 2030 Agenda for Sustainable Development, adopted in 2015, with its 17 Sustainable Development Goals (SDGs)~\cite{1BIB4};%
\item The Global Coal to Clean Power Transition Statement, signed at the 26th United Nations Climate Change Conference (COP 26) in 2021;%
\item The European Green Deal~\cite{1BIB5}, approved in 2020;%
\item The set of proposals to reduce net greenhouse gas emissions, adopted by the European Commission in 2021;%
\item The priority areas of the Energy Transition Council, established in 2020;%
\item The International Energy Agency (IEA) Report~\cite{1BIB6} in 2021, etc.%
\end{itemize}%
\par%
All these reductions in environmentally harmful industries will inevitably affect the IT sector.%
\section{Sustainable behavior of IT practitioners}
\par%
When IT practitioners make their behavioral decisions, the choose how sustainable their behavior is depending on some pragmatic and ethical issues. The paper~\cite{1BIB1} explores in detail these pragmatic and ethical concerns that IT practitioners have to take into account when they want to start working or already work in a fossil fuel company or its supplier company.%
\par%
From a pragmatic point of view, the fossil fuel sector is decreasing and will end in the next few decades, according to the global politics (see Section 2), therefore IT specialist should move to another sector. A lot of jobs will be lost according to IEA~\cite{1BIB6} very soon, because companies will not invest in new fossil fuel exploration and development. Because of the "net-zero emissions by 2050" pathway, some industries will end by 2050 or earlier, such an example being  the transport fossil fuel industry. In recent years, there has been a rapid increase in net zero emissions announcements by companies with target 2050 or earlier~\cite{1BIB6}. A common classification system – EU Taxonomy for sustainable activities~\cite{1BIB7}, establishing a list of environmentally sustainable economic activities has been created, where the fossil fuels are excluded. Some organizations provide lists that can be used for recognition of sustainable companies – for examples, MSCI launched Global Fossil Fuels Exclusion Indexes~\cite{1BIB8}, used by institutional investors to to eliminate or reduce some or all fossil fuel reserves exposure from their investments, and Investopedia’s Environmental, social, and governance (ESG) criteria~\cite{1BIB9}, giving a set of standards for a company’s behavior used by investors to screen potential investments based on corporate sustainable policies.%

\par%
From an ethical standpoint, to work for a company producing or supplying fossil fuels is not illegal, but the ethical point of view should be considered. Whether to work for an environmentally damaging sector or not, is a personal decision. Ethical principles, connected to the issue of sustainability, are reflected in the main documents of a number of IT organizations – IEEE Code of Ethics~\cite{1BIB10}, BCS Code of Conduct~\cite{1BIB11}, ACM Code of Ethics and Professional Conduct~\cite{1BIB12}, etc. The environmentally damaging fossil fuels sector directly hinders the two of the 17 SDGs of the UN – SDG13. Climate Action and SDG7. Affordable and Clean Energy, and indirectly – SDG3. Good Health and Well-Being, SDG11. Sustainable Cities and Communities, etc.%
\par%
All these pragmatic and ethical considerations~\cite{1BIB1} that IT practitioners should take into account when deciding which company to work for show that this decision is very important and with consequences.%
%
\section{Survey questionnaire}
\par%
The objectives of the questionnaire are:%
\begin{itemize}%
\item to investigate young IT professionals’ and teachers’ attitude to sustainability (incl. the fossil fuel industry);%
\item to explore IT students’ and teachers’ opinion about the place of sustainability in higher education;%
\item to study their opinion about future employment in relation to sustainability.%
\end{itemize}%
\par%
The survey consists of 20 questions (see the Survey Questionnaire): 5 demographic questions and 15 questions (presented as statements) on issues of interest. The survey uses 7 of the UK survey questions~\cite{1BIB13}~\cite{1BIB15}~\cite{1BIB16} to enable comparison with existing results. The statements are divided into 6 groups (see Table 1).%

\begin{table}[]
	\centering
	\begin{tabular}{ll}
		\textbf{Group} & \textbf{Questions}  \\
		Sustainability awareness & 1-4  \\
		Sustainability at university &   5-7  \\
		Sustainability in employers &   8-10 \\
		Ethical concerns &  11-13 \\
		Practical concerns &  14-16 \\
		Community behavior &   17-20
	\end{tabular}
\caption{\label{demo-table2}Survey question groups}
\end{table}

\par%
The majority of the questions in the survey use a 5-step Likert scale to give students a better opportunity to express their opinion to the fullest: Strongly Disagree, Disagree, Undecided, Agree and Strongly Agree. The remaining questions provide students with situational cases from which they need to choose the option best suited for them.%

\par%
\textbf{SURVEY QUESTIONNAIRE}%
\par%
\textbf{Part 1. Personal data}%
\par%
1. Sex – Male/Female%
\par%
2. Year of study/Teacher – 1/2/3/4/teacher%
\par%
3. Working now/have worked in the ICT sector? – Yes/No%
\par%
4. If yes, for how long?%
\par%
5. Country of study/teaching%
\par%
\textbf{Part 2. Personal opinion}%
\par%
1. I know of the United Nations’ Sustainable Development Goals.%
\par%
2. I know that globally we must reach net zero carbon emissions by 2050.%
\par%
3. Fossil fuels are at the root of the climate crisis.%
\par%
4. I personally carry out the skill “understand people’s relationship to nature”.%
\par%
5. The university practices and promotes good social and environmental skills.%
\par%
6. The University is obliged to develop students’ social and environmental skills as part of the courses.%
\par%
7. The university courses consider the ethical (in the direction of environmental and social aspects) implications of the course subject.%
\par%
8. My employer should consider the environmental and social impacts of its products and/or services.%
\par%
9. My employer should actively work for the reduction of carbon emissions (incl. by refusing to provide products/services to the fossil fuels sector).%
\par%
10. The employee’s skill of “understanding people’s relationship to nature” should be important for my employers (current or future).%
\par%
11. I am concerned about environmental climate change.%
\par%
12. I will refuse to do any work at my company that supports the fossil fuel industry.%
\par%
13. I will leave my company if I learn that it produces software/ provides services for the fossil fuel industry.%
\par%
14. I share my position about not supporting the Fossil Fuel Industry with other ICT practitioners.%
\par%
15. I participate in conversations/events about the negative impact of the ICT sector on the environment when supporting the Fossil Fuel Industry.%
\par%
16. I show support to others who are speaking up and saying no to working for the Fossil Fuel Industry.%
\par%
17. I know which organization (Trade union, Non Government Organizations) to contact with to help me find an employer who doesn’t support the fossil fuel industry or to help me leave a company that supports the environmentally damaging fossil fuel sector.%
\par%
18. Which option would you choose?
Assuming all other factors are equal, I would choose a graduate position with a starting salary of 100 Euro higher than average in a company with a poor environmental and social record.
Assuming all other factors are equal, I would choose a graduate position with a starting salary of 100 Euro lower than average in a company with a strong environmental and social record.%
\par%
19. Which option would you choose?
Assuming all other factors are equal, I would choose a graduate position with a starting salary of 300 Euro higher than average in a company with a poor environmental and social record.
Assuming all other factors are equal, I would choose a graduate position with a starting salary of 300 Euro lower than average in a company with a strong environmental and social record.%
\par%
20. Which option would you choose?
Assuming all other factors are equal, I would accept a graduate position with a starting salary of 300 Euro higher than average in a role that does not contribute to positive environmental and social change.
Assuming all other factors are equal, I would accept a graduate position with a starting salary of 300 Euro lower than average in a role that contributes to positive environmental and social change.%

\par%
The survey was completed by 260 young IT professionals – IT students from University of Plovdiv “Paisii Hilendarski”, Bulgaria.%
\par%
During the summer school, a survey will be conducted among IT students and teachers, and the results of the survey will be able to be compared across participating countries. The survey will be voluntary and anonymous.%
\section{Conclusion}
\par%
IT specialists have to take into account that a career in an environmentally damaging sector like the fossil fuel industry will be short-term and in near future they have to change their job in other sectors. Working for the fossil fuel industry, through supporting or developing software that continues the production and sales of fossil fuels, is a legal job, but it is incompatible with the ethical principles of the ICT professional organizations and ethical job decisions are very important.%
\par%
During the Second SusTrainable Summer School, a survey will be conducted among IT students and teachers to explore IT specialists’ attitude to sustainability (in particular to the fossil fuel industry) and the place of sustainability in higher education and future employment.%
%
\par%
\textbf{Acknowledgements}%
\par%
This paper acknowledges the support of the Erasmus+ Key Action 2 (Strategic partnership for higher education) project No 2020-1-PT01-KA203-078646: “SusTrainable – Promoting Sustainability as a Fundamental Driver in Software Development Training and Education”. The information and views set out in this paper are those of the authors and do not necessarily reflect the official opinion of the European Union. Neither European Union institutions and bodies, nor any person acting on their behalf may be held responsible for the use which maybe made of the information contained therein.

%\bibliographystyle{splncs04}
%\bibliography{Paper_PU_1.bib}
%\printbibliography%

%\end{document}%
% 