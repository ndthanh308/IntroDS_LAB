\documentclass[twoside]{article}
% picins: misssing
\usepackage{amsfonts,amssymb,amsbsy,textcomp,marvosym,amsmath,caption,threeparttable,amsthm,subfigure,float,lastpage,lscape}
\usepackage{eurosym,mathrsfs,fancyhdr,CJK,multicol,graphics,indentfirst,color,bm,upgreek,booktabs,graphicx,multirow,warpcol}
\usepackage{epstopdf}
%\usepackage[noend]{algorithm}
%\usepackage[noend]{algorithmic}
%\usepackage[lined,algonl,boxed]{algorithm2e}
\usepackage{wrapfig}

\usepackage{hyperref}
\newcommand\nnfootnote[1]{%
  \begin{NoHyper}
  \renewcommand\thefootnote{}\footnote{#1}%
  \addtocounter{footnote}{-1}%
  \end{NoHyper}
}
\renewcommand*{\figureautorefname}{Fig.}
\renewcommand*{\sectionautorefname}{Sec.}
\renewcommand*{\subsectionautorefname}{Sec.}
\renewcommand*{\equationautorefname}{Eq.}

\usepackage{xspace,xpunctuate}
\newcommand{\aka}{{a.k.a.},\xspace}
\newcommand{\ie}{\textit{i.e.},\xspace}
\newcommand{\etal}{\xspace\textit{et al.}\xspace}
\newcommand{\eg}{\textit{e.g.},\xspace}

\newcommand{\rc}[1]{\textcolor{blue}{#1}}


\looseness=-1
%------------Page layout and margin and Headrule-------------
\headsep=5mm \headheight=4mm \topmargin=0cm \oddsidemargin=-0.5cm
\evensidemargin=-0.5cm \marginparwidth=0pt \marginparsep= 0pt
\marginparpush=0pt \textheight=23.1cm \textwidth=17.5cm \footskip=8mm
\columnsep=7mm \setlength{\doublerulesep}{0.1pt}
\footnotesep=3.5mm\arraycolsep=2pt
\font\tenrm=cmr10
%===========================================================
\def\footnoterule{\kern 1mm \hrule width 10cm \kern 2mm}
\def\rmd{{\rm d}} \def\rmi{{\rm i}} \def\rme{{\rm e}}
\def\sj#1{$^{[#1]}$}\def\lt{\left}\def\rt{\right}
\renewcommand{\captionfont}{\footnotesize}
\renewcommand\tablename{\bf \footnotesize Table}
\renewcommand\figurename{\footnotesize Fig.\!\!}
\captionsetup{labelsep=period}%
% \captionsetup[longtable]{labelsep=period}%
\allowdisplaybreaks
\sloppy
\renewcommand{\headrulewidth}{0pt}
\catcode`@=11
\def\title#1{\vspace{3mm}\begin{flushleft}\vglue-.1cm\Large\bf\boldmath\protect\baselineskip=18pt plus.2pt minus.1pt #1
\end{flushleft}\vspace{1mm} }
\def\author#1{\begin{flushleft}\normalsize #1\end{flushleft}\vspace*{-4pt} \vspace{3mm}}
\def\address#1#2{\begin{flushleft}\vglue-.35cm${}^{#1}$\small\it #2\vglue-.35cm\end{flushleft}\vspace{-2mm}\par}


\def\jz#1#2{{$^{\footnotesize\textcircled{\tiny #1}}$\let\thefootnote\relax\footnotetext{\!\!$^{\footnotesize\textcircled{\tiny #1}}$#2}}}
\catcode`@=11
\def\section{\@startsection{section}{1}{\z@}%
 %{-3.5ex \@plus -1ex \@minus -.2ex}%
 {-3ex \@plus -.3ex \@minus -.2ex}%
 {2.2ex \@plus.2ex}%
{\normalfont\normalsize\protect\baselineskip=14.5pt plus.2pt minus.2pt\bfseries}}
\def\subsection{\@startsection{subsection}{2}{\z@}%
 %{-3.25ex\@plus -1ex \@minus -.2ex}%
 {-3ex\@plus -.2ex \@minus -.2ex}%
 {2ex \@plus.2ex}%
{\normalfont\normalsize\protect\baselineskip=12.5pt plus.2pt minus.2pt\bfseries}}
\def\subsubsection{\@startsection{subsubsection}{3}{\z@}%
 %{-3.25ex\@plus -1ex \@minus -.2ex}%
 {-2.2ex\@plus -.21ex \@minus -.2ex}%
 {1.4ex \@plus.2ex}
{\normalfont\normalsize\protect\baselineskip=12pt plus.2pt minus.2pt\sl}}
\def\proofname{{\indent \it Proof.}}
%===========================================================ÒÔÉϲ»¶¯

\pagestyle{fancy}
\fancyhf{}% Çå¿Õҳüҳ½Å
% \fancyhead[LO]{\small\sl Shortened Title Within 45 Characters}%
%\fancyhead[LO]{\small\sl Computational Approaches for TCP: Six Principles}%
\fancyhead[RO]{\small\thepage}
\fancyhead[LE]{\small\thepage}
%\fancyhead[RE]{\small\sl J. Comput. Sci. \& Technol.}
\setcounter{page}{1}
\begin{document}
\begin{CJK*}{GBK}{song}
\thispagestyle{empty}
\vspace*{-13mm}
%\noindent {\small Journal of computer science and technology: Instruction for authors.
%JOURNAL OF COMPUTER SCIENCE AND TECHNOLOGY}
%===========================================================
\vspace*{2mm}

\nnfootnote{}







\title{Computational Approaches for Traditional Chinese Painting: From the ``Six Principles of Painting'' Perspective}


\author{Wei Zhang$^{1}$, Jian-Wei Zhang$^{1}$, Kam Kwai Wong$^{2}$, Yifang Wang$^{3}$, Yingchaojie Feng$^{1}$, Luwei Wang$^{1}$, and Wei Chen$^{1,4,*}$}

\address{1}{State Key Lab of CAD\&CG, Zhejiang University, Hangzhou 310058, China}
\address{2}{Hong Kong University of Science and Technology, Hong Kong 999077, China}
\address{3}{Kellogg School of Management, Northwestern University, Evanston 60208, U.S.A}
\address{4}{Laboratory of Art and Archaeology Image, Zhejiang University, Hangzhou 310058, China}


\vspace{2mm}

\noindent E-mail: zw\underline{~~}yixian@zju.edu.cn; zjw.cs@zju.edu.cn; kkwongar@cse.ust.hk; yifang.wang@kellogg.northwestern.edu; fycj@zju.edu.cn; ppwlwpp@zju.edu.cn; chenvis@zju.edu.cn \\[-1mm]





%\let\thefootnote\relax\footnotetext{{}\\[-4mm]\indent\ Regular Paper}

\noindent {\small\bf Abstract} \quad  
{\small {Traditional Chinese Painting (TCP) is an invaluable cultural heritage resource and a unique visual art style. In recent years, increasing interest has been placed on digitalizing TCPs to preserve and revive the culture. The resulting digital copies have enabled the advancement of computational methods for structured and systematic understanding of TCPs. To explore this topic, we conducted an in-depth analysis of 92 pieces of literature. We examined the current use of computer technologies on TCPs from three perspectives, based on numerous conversations with specialists. First, in light of the ``Six Principles of Painting" theory, we categorized the articles according to their research focus on artistic elements. Second, we created a four-stage framework to illustrate the purposes of TCP applications. Third, we summarized the popular computational techniques applied to TCPs. The framework also provides insights into potential applications and future prospects, with professional opinion. The list of surveyed publications and related information is available online at https://ca4tcp.com.}}

%{\small \textcolor{blue}{Please provide an abstract of 100 to 250 words. The abstract should clearly state the nature and significance of the paper. It must not include undefined abbreviations, mathematical expressions or bibliographic references.}}

\vspace*{3mm}

\noindent{\small\bf Keywords} \quad {\small Traditional Chinese Painting, Digital humanity, Cultural heritage, Computer vision, Deep learning}
%[\textcolor{blue}{Keywords should closely reflect the topic and should optically characterize the paper. Please use about 3$\sim $5 keywords or phrases in alphabetical order separated by commas.}]}

\vspace*{4mm}

\end{CJK*}
\baselineskip=18pt plus.2pt minus.2pt
\parskip=0pt plus.2pt minus0.2pt
\begin{multicols}{2}


\section{Introduction}

% Figure environment removed

Reinforcement Learning from Human Feedback (RLHF) has recently been used to great effect to align pretrained large language models (LLMs) to human preferences, optimizing for desirable qualities like harmlessness and helpfulness~\citep{bai2022training} and achieving state-of-the-art results across a variety of natural language tasks~\citep{openai2023gpt4}. %RLHF approaches fundamentally rely on collecting pairs of LLM outputs $(o_1, o_2)$ from a shared prompt $p$, with a human indicating which output in each pair is better on a specified attribute.
% A fundamental component of RLHF is a preference model derived from human labels, typically formatted as pairs of LLM outputs $(o_1, o_2)$ generated from a shared prompt $p$.

A standard RLHF procedure fine-tunes an initial unaligned LLM using an RL algorithm such as PPO~\citep{schulman2017proximal}, optimizing the LLM to align with human preferences. %\violet{not sure whether we need to provide this detail in the intro, especially this has nothing to do with our contribution.} % i feel like this context is useful later when e.g. explaining that context distillation is SFT
RLHF is thus critically dependent on a reward model derived from human-labeled preferences, typically \textit{pairwise preferences} on LLM outputs $(o_1, o_2)$ generated from a shared prompt $p$. % and labeled by humans. 

However, collecting human pairwise preference data, especially high-quality data, may be expensive and time consuming at scale. To address this problem, approaches have been proposed to obtain labels without human annotation, such as Reinforcement Learning from AI Feedback (RLAIF) and context distillation. 

\iffalse
raising the question of whether we can generate high-quality data for RLHF without using human labeling. %accurately-labeled preference pairs $(o_1, o_2)$
%, motivating model alignment approaches that aim to generate accurately-labeled preference pairs $(o_1, o_2)$ without human involvement. 
Two major categories of such approaches are . 
\fi

RLAIF approaches (e.g.,~\citet{bai2022constitutional}) simulate human pairwise preferences by scoring $o_1$ and $o_2$ with an LLM (Figure \ref{fig:rlcd_differences} center); the scoring LLM is often the same as the one used to generate the original pairs $(o_1, o_2)$. Of course, the resulting LLM pairwise preferences will be somewhat noisier compared to human labels. However, this problem is exacerbated by using the same prompt $p$ to generate both $o_1$ and $o_2$, causing $o_1$ and $o_2$ to often be of very similar quality and thus hard to differentiate (e.g., Table~\ref{tab:rlaif_bad_example}). Consequently, training signal can be overwhelmed by label noise, yielding lower-quality preference data. 

% While it avoids human labeling efforts, it has weakness. First, LLM preference labels will naturally be somewhat noisier compared to human labels. Furthermore, since the same prompt $p$ is used to generate both $o_1$ and $o_2$, their quality is often very similar and hard to differentiate (See Table~\ref{tab:rlaif_bad_example}). As a result, training signals can be overwhelmed by label noise, yielding lower-quality preference data. 

Meanwhile, context distillation methods (e.g., \citet{sun2023principle}) create more training signal by modifying the initial prompt $p$. 
%to create more significant training signal. 
The modified prompt $p_+$ typically contains additional context encouraging a \textit{directional attribute change} in the output $o_+$ (Figure \ref{fig:rlcd_differences} right). However, context distillation methods only generate a single output $o_+$ per prompt $p_+$, which is then used for supervised fine-tuning, losing the pairwise preferences which help RLHF-style approaches to 
%rather than using a RLHF-style preference model to 
derive signal from the contrast between outputs. 
Multiple works have observed that RL approaches using preference models for pairwise preferences can substantially improve over supervised fine-tuning by itself when aligning LLMs~\citep{ouyang2022training,dubois2023alpacafarm}. 

% conduct alignment by running supervised fine-tuning on model outputs $o_+$ generated from a modified prompt $p_+$. $p_+$ typically contains additional context encouraging desirable attributes (Figure \ref{fig:rlcd_differences} right), such as in \citet{sun2023principle}. However, multiple works have observed that RLHF-style approaches can substantially improve over supervised fine-tuning by itself when aligning LLMs~\citep{ouyang2022training,dubois2023alpacafarm}. 

Therefore, while both RLAIF and context distillation approaches have already been successfully applied in practice to align language models, we posit that it may be even more effective to combine the key advantages of both. That is, we will use RL with \textit{pairwise preferences}, while also using modified prompts to encourage \textit{directional attribute change} in outputs. %In particular, we will adapt the RLAIF data generation process with two different prompts rather than a single $p$, modifying both prompts similarly to context distillation. %\violet{this motivation is a little unexciting. I think we can more specifically discuss the potential benefits of our approach, like the benefits from RL: exploration/data generation; benefits from contrast. I don't think we get too much benefits from context distillation since we switched to the RL framework.} 

Concretely, we propose \oursfull{} (\ours{}). 
\ours{} generates preference data as follows. Rather than producing two i.i.d.\ model outputs $(o_1, o_2)$ from the same prompt $p$ as in RLAIF, \ours{} creates two variations of $p$: a \textit{positive prompt} $p_+$ similar to context distillation which encourages directional change toward a desired attribute, and a \textit{negative prompt} $p_-$ which encourages directional change \textit{against} it (Figure \ref{fig:rlcd_differences} left). We then generate model outputs $(o_+, o_-)$ respectively, and automatically label $o_+$ as preferred---that is, \ours{} automatically ``generates'' pairwise preference labels by construction. %, without further post hoc labeling.\violet{should make it clearer that our approach `generates' labels by construction} 
We then follow the standard RL pipeline of training a preference model followed by PPO. 

Compared to RLAIF-generated preference pairs $(o_1, o_2)$ from the same input prompt $p$, there is typically a clearer difference in the quality of $o_+$ and $o_-$ generated using \ours{}'s directional prompts $p_+$ and $p_-$, which may result in less label noise. %which may result in better training signal for the preference model. 
That is, intuitively, \ours{} exchanges having examples be \textit{closer to the classification boundary} for much more \textit{accurate labels} on average. Compared to standard context distillation methods, on top of leveraging pairwise preferences for RL training, \ours{} can derive signal not only from the positive prompt $p_+$ which improves output quality, but also from the negative prompt $p_-$ which degrades it. %\ours{} is not learning to imitate $o_+$, but to distill the \textit{contrast} between $o_+$ and $o_-$. 
Positive outputs $o_+$ don't need to be perfect; they only need to contrast with $o_-$ on the desired attribute while otherwise following a similar style.

% \todo{discuss our method and why intuitively it may be better.}

We evaluate the practical effectiveness of \ours{} through both human and automatic evaluations on three tasks, aiming to improve the ability of LLaMA-7B~\citep{touvron2023llama} to generate harmless outputs, helpful outputs, and high-quality story outlines. %\ours{} outperforms both RLAIF and context distillation baselines in pairwise comparisons on 
As shown in Sec. \ref{sec:experiments}, \ours{} substantially outperforms both RLAIF and context distillation baselines in pairwise comparisons when simulating preference data with LLaMA-7B, while still performing equal or better when simulating with LLaMA-30B. 
%On all three tasks, \ours{} substantially outperforms both RLAIF and context distillation baselines in pairwise comparisons---by a margin of at least 9\% and often more than 30\%---validating our method's efficacy. 
We will release all code at a later date, although in any case \ours{} is fairly easy to implement by modifying any reference RLAIF codebase. %We release all code at \todo{github link}.
\section{Methodology}
\label{sec:meth}

This section describes the search methodology, corpus construction, and collaboration with experts.

\subsection{Methods and Corpus}

We constructed a literature corpus based on keyword- and relation-search methods. We focused on TCP-related keywords (\eg ``Chinese landscape painting'', ``Chinese ink wash painting'', and ``Chinese brush painting''), resulting in an initial 88 papers.
To enlarge our literature corpus, we further used the relation-search method. We identified eight influential papers from the initial corpus and exhaustively traversed their references and citations, which expanded the corpus to 112 papers.
We thoroughly assessed the corpus based on relevance, focusing on publications that probe methodologies and applications. Papers on theory~\cite{wu2013modeling} and evaluation~\cite{bo2018computational} were excluded. 
Despite extensive research on brushes, we only included those relevant to TCP, discarding the calligraphy-related articles~\cite{mi2002droplet, bai2007efficient}. Finally, the corpus comprised 92 papers.



\subsection{Cooperate with Domain Expert}

Over the past year, we have been working closely with two experts to strengthen our understanding of the specialized and unique domain of TCP.
The experts include a professor with over 20 years of expertise in Chinese painting, and a doctoral candidate in Chinese painting theory with five years of experience.
Our collaboration consists of the following stages:
Firstly, we consulted the experts about the domain knowledge of TCP and the domain interpretation of some exemplar papers.
Secondly, we iteratively refined the analysis framework for the application of computer technology to TCP.
Finally, we explored research challenges and opportunities by discussing the findings and proposing future research projects.



\subsection{Coding and Classification}

Through iterative discussions with experts, we have analyzed the corpus from three perspectives: research scope in TCP~\autoref{sec:background}, specifically-targeted problem~\autoref{sec:ana_fmwk}, and the use of computer-based methods~\autoref{sec:tech}. 
To differentiate the TCP research from image data analysis properly, the characteristics of TCP should be taken into account. 
We noticed that the ``Six Principles of Painting'' has summarized the most important considerations for drawing and appreciating TCP. 
We used it as the coding scheme for categorizing the literature with different targeted problems, \ie the concerned artistic elements. 
In addition, we evaluated the current analysis of TCP, and concluded with a paradigm which outlines the components and purposes of TCP analysis that are supported by modern computer techniques. 
Specifically, the paradigm includes digitalization, interpretation, creation, and exhibition of TCP.
Lastly, we coded the papers in the corpus according to the types of computational techniques they utilized, rather than the concrete algorithms that are largely interchangeable. 


 

During the paper analysis, three authors independently coded 92 papers over a period of four weeks (\autoref{fig:code}). 
The classification criteria were refined during the coding process. 
In cases where there were disputes regarding categorization, all authors were involved in debates to reach a consensus.
For instance, we first restricted the classification of the papers using the ``Resonance of the Spirit'' in the Six Principles to the subjective evaluation of static Chinese painting images. 
After deliberation, we decided that this idea should be expanded to include animations, as this was deemed to be a more expressive art form which could be evaluated by the principle of ``Resonance of the Spirit''. More discussions are given in \autoref{sec:spirit}. 


\setcounter{figure}{1}
% Figure environment removed
% \baselineskip=18pt plus.2pt minus.2pt
% \parskip=0pt plus.2pt minus0.2pt

\section{The Six Principles of Painting}
\label{sec:background}
The \textit{Six Principles of Painting} were proposed in the sixth century to serve as the grading standards of TCP~\cite{lin1967art}.
% Xie He used these principles to rate previous paintings in the book \textit{Gu Hua Pin Lu} (Notes on the Criticism of Old Paintings).
They have remained influential to this day and shaped the ways in which TCPs are drawn and appreciated.
We selected the translations collected in~\cite{van1962way} to clarify these principles in the following sections.

\subsection{Resonance of the Spirit, Movement of Life}
\label{sec:spirit}

\hspace{0.1pt}
\vspace{-25pt}
\begin{wrapfigure}[2]{l}{0.02\textwidth}
 \begin{center}
  \vspace{-25pt}
  % Figure removed
 \end{center}
\end{wrapfigure}
\noindent
The first principle emphasizes the delivery of vitality in the objects and emotions.
This principle is considered the most important, upon which the remaining principles were developed~\cite{lin1967art}.
While the resonated spirits and lively movements are difficult to evaluate objectively, we adopted their semantic meaning and defined them as techniques that aim to induce emotional arousal and engage audiences.

Papers within this category analyze the conveyed emotions and use this information to recreate captivating paintings on various devices.
For example, Zheng\etal~\cite{zheng2017chinese} utilized machine learning methods to extract relevant features and classified TCPs with the conveyed emotions.
To enhance the emotional expressions, other works~\cite{lianginstance,liu2020animating,zhang2009video} attempted to make the objects in the paintings `alive' (\eg animations), so that the paintings can be interpreted vividly.
In recent years, significant amount of efforts~\cite{jin2022immersive,jin2020reconstructing,zhao2020shadowplay2,ma2012annotating,jin2007real} have utilized mixed reality technology to display TCPs, giving viewers new perspectives on appreciating the paintings.

\subsection{Bone Manner, Structural Use of the Brush}
\hspace{0.1pt}
\vspace{-25pt}
\begin{wrapfigure}[2]{l}{0.02\textwidth}
 \begin{center}
  \vspace{-25pt}
    % Figure removed
 \end{center}
\end{wrapfigure}
\noindent
The bone method corresponds to the use of the brush.
Calligraphy and paintings highly influenced one other, with each brush stroke having its own structure, texture, and meaning.
We defined this principle as rendering techniques that involve brush strokes.
% Papers that fall into this category concentrate on rendering brush strokes.
There are 36 articles discussing the bone method from the following aspects:

\emph{Stroke extraction.} Some articles focus on identifying and replicating the distinctive brushstrokes of TCP~\cite{chan2002two,jiang2021mtffnet,li2004studying,sun2016monte,sun2015brushstroke,sheng2014,sheng2014recognition,sheng2013style,xu2006animating,yeh2002non,yu2003image}. 
Frequent subjects in TCP, including mountains~\cite{way2002synthesis}, rocks~\cite{way2001synthesis}, and trees~\cite{zhang1999simple}, were imitated.

\textbf{Ink simulation.} TCPs are drawn with brushes dipped in black ink or Chinese pigments, which diffuse on rice papers or silk. Therefore, the simulation of ink effect is heavily studied~\cite{xu2012stroke,way2006computer,10.1145/1073204.1073221,chu2004real,mi2004droplet,guo2003nijimi,way2003physical,yu2002model,lee2001diffusion,kunii1995diffusion,10.1145/15886.15911}. On the other hand, the physical model of the brush itself also deserves in-depth analysis~\cite{bai2009chinese,xu2005virtual,yeh2002effects,lee1999simulating}.

\textbf{Style recognition.} Xieyi and Gongbi are two representation styles of TCP. Xieyi uses techniques that privilege the spontaneity of the line (\hyperref[fig:gongbixieyi]{Fig.~3}A). 
% Gongbi uses highly detailed brushstrokes that precisely delimit details (\autoref{fig:g_x}B). 
Gongbi uses highly detailed brushstrokes that precisely delimit details (\hyperref[fig:gongbixieyi]{Fig.~3}B). 
These stroke characteristics were used to classify different painting styles~\cite{jiang2019dct,yang2019easy,jiang2004categorizing}.

\textbf{Painting process reproduction.} Understanding the painting process of TCP is particularly useful for painting practice and education~\cite{8113507,yang2013animating,xie2013artist,yao2005painting}. It also provides a practical basis for painting generation.

\subsection{Conform with the Objects, Obtain their Likeness}
% depicting the forms of things as they are [1]
% The drawing of forms which answer to natural forms[2]
\hspace{0.1pt}
\vspace{-25pt}
\begin{wrapfigure}[2]{l}{0.02\textwidth}
    \begin{center}
    \vspace{-25pt}
    % Figure removed
    \end{center}
\end{wrapfigure}
\noindent
Chinese painters have developed a distinct way of depicting objects in the world.
Compared with their Western counterparts, objects in Chinese paintings are more surreal.
This principle is reinterpreted as extracting objects from TCP and depicting and sketching natural objects in the TCP artistic style.
Six articles~\cite{dong2020feature,liong2020automatic,li2020multi,gu2019deep,lu2008content,zhang2004modelling} have studied the TCP classification according to the extracted objects, and two articles~\cite{chen2021poemgeneration,feng2022ipoet} produce text descriptions (\ie instance-level captions) for the objects in TCP. There are additional 16 articles that conduct the style transfer for natural objects with TCP art styles~\cite{xue2021end,li2021immersive,9413063,zhang2020detail,le2019walking,meng2019elements,he2018chipgan,wu2018research,shi2017generative,lai2016data,guo2015novel,li2014writing,dong2014real,liang2013image,amati2010modeling,wang2007image}.



\setcounter{figure}{2}
\begin{center}
% Figure removed\\
\vspace{3mm}
\parbox[c]{8.3cm}{\footnotesize{Fig.3.~}  Liang Kai's ``Immortal in Splashed Ink" (A)~\cite{xy2022painting} exemplifies the Xieyi style, using wet strokes of monochromatic ink to create the immortal's cloth (A1), while Ma Lin's ``King Yu of Xia" (B)~\cite{gb2022painting} exemplifies the more elaborate Gongbi style for clothing patterns (B1).}
\label{fig:gongbixieyi}
%\vspace*{.2mm}
\end{center}




\subsection{According to the Species, Apply the Colors}
% appropriate colouring [1]
% Appropriate distribution of the colours[2]
\hspace{0.1pt}
\vspace{-25pt}
\begin{wrapfigure}[2]{l}{0.02\textwidth}
 \begin{center}
  \vspace{-25pt}
  % Figure removed
 \end{center}
\end{wrapfigure}
\noindent
The suitability to type often appears to judge the correct use of colors.
In the drawing process, inks have to be applied in multiple layers to arrive at the desired tone.
Since inks and papers react to the environment and receive damages, ancient paintings have lost their original states and need to be conserved properly.
This principle corresponds to color analysis and restoration of paintings.
Papers that fall into this category concentrate on understanding the object classes~\cite{zhan2019,hung2018study,meng2018classification,liu2014Classification,bao2009effective,guan2005automatic}%(13,20,24、28,41、61、70) 
and performing further actions (\eg color enhancement~\cite{hu2015object,xu2007generic,jiang2006effective}). %(38,63,66)
Additionally, understanding the color of TCP is crucial for repairing them. Some papers employ computer technologies to address the issues of deteriorating paper and fading color~\cite{guo2013image,chen2012simulating,zhang2011multispectral,pei2006background,pei2004virtual,ding2012research}.

\subsection{Plan and Design, Place and Position}
\hspace{0.1pt}
\vspace{-25pt}
\begin{wrapfigure}[2]{l}{0.02\textwidth}
 \begin{center}
  \vspace{-25pt}
  % Figure removed
 \end{center}
\end{wrapfigure}
\noindent
Division and planning refer to the positioning and arrangement of objects.
A unique layout characteristic in Chinese painting is the concept of ``void'' (white space).
It is believed that leaving some part of the paper blank could induce more imagination in readers' minds~\cite{fan2019evaluation,fan2017visual}.
The seals, preface, and postscript are also significant components of the composition of TCP~\cite{bao2010novel,liang2010simple}.
Papers in this category concentrate on analyzing and enhancing the composition.

\subsection{To Transmit Models by Drawing}
\hspace{0.1pt}
\vspace{-25pt}
\begin{wrapfigure}[2]{l}{0.02\textwidth}
 \begin{center}
  \vspace{-25pt}
  % Figure removed
 \end{center}
\end{wrapfigure}
\noindent
Prior to the invention of printers, the manual replication of paintings was required to facilitate their distribution and use as commodities.
Copying the classics and antique masterpieces also help amateurs improve their skills by closely observing the techniques.
Papers in this area has concentrated on digitalizing Chinese paintings and showing them on different devices, such as high relief~\cite{8419282}, virtual reality reconstruction~\cite{yuan2016tunable}, and interactive devices~\cite{subramonyam2015sigchi,hsieh2013viewing}.
The primary distinction between this principle and others is that these methods focus on precise reproduction and minimal modification from the original copies.
\section{Analytical Framework}
\label{sec:ana_fmwk}

In this section, we propose a framework for applying computational techniques in TCP based on the state-of-the-art and the expertise of domain experts. 
As shown in \autoref{fig:datapipeline}, the framework involves four typical stages, from the \textbf{digitalization} and \textbf{interpretation} of existing paintings to the \textbf{creation} of new artworks. 
These three stages will also serve the purpose of \textbf{exhibition}. 



\setcounter{figure}{3}
% Figure environment removed
\baselineskip=18pt plus.2pt minus.2pt
\parskip=0pt plus.2pt minus0.2pt






\subsection{Digitalization}
The \textbf{digitalization} stage involves transforming physical raw TCP into digital signals or codes, primarily as digital images~\cite{warwick2012digital}. 
These images could form a large corpus of TCP, comprising hundreds and thousands of artworks that can hardly be accessed physically in one place. 
Therefore, this stage presents technical requirements for \textbf{storage}, \textbf{retrieval}, and \textbf{restoration}. 

\textbf{Storage} requires storing a large amount of TCP digital images in the database. 
Many image-related techniques are developed to store and show paintings smoothly with different levels of detail~\cite{iiif}. 
However, there is little research targeting the storage of TCP digital images, which incorporates specific query and analytical requirements. 
For these large databases, information \textbf{retrieval} searches interested series of paintings efficiently from different dimensions. 
In addition to the metadata of TCP, such as authors and themes, content-based image retrieval utilizes similarities of paintings in terms of visual features~\cite{dong2020feature,hung2018study,wang2007image}. 
\textbf{Restoration} of TCP deals with the pigment fading and paper aging~\cite{guo2013image} in the digitalization stage.
Chen\etal\cite{chen2012simulating} simulated the aging and reverse-aging phenomena. 
Several studies apply image recovery techniques to restore the electronic forms of TCP in terms of stroke and brush~\cite{guo2013image} and colors~\cite{guo2013image,chen2012simulating,pei2006background,pei2004virtual,ding2012research}. 

\subsection{Interpretation} 
After obtaining the digital formats of original TCP, the next stage is to \textbf{interpret} them by applying computational approaches. 
The aspects of interpretation vary from micro-level analysis (\eg colors and objects~\cite{feng2022ipoet,chen2021poemgeneration}), meso-level analysis (\eg emotion extraction~\cite{feng2022ipoet}), to macro-level analysis (\eg the layout of the painting and white space analysis~\cite{fan2019evaluation}). 

In terms of techniques, the majority of the current work falls into two categories: local and global feature extraction. 
The former emphasizes the identification of relevant features, including handcrafted features (\eg brushwork~\cite{zhan2019,lai2016data} and color~\cite{guo2015novel,liu2014Classification}) and learned features with deep learning technologies (\eg object detection~\cite{meng2019elements,gu2019deep}). 
The latter focuses on the effectiveness of classification such as accuracy, in which most work utilizes learned features~\cite{liong2020automatic,meng2019elements}.
More technical details will be introduced in \autoref{sec:tech_tasks} and \autoref{sec:tech_feature}. 

\subsection{Creation} 
After obtaining useful features and insights from the analysis of existing paintings, another large proportion of studies focuses on the \textbf{creation} of new artworks. 
Image generation and video generation are two mainstream creation outcomes. 

\textbf{Image Generation.} 
The majority of studies focuses on generating new paintings, considering the unique characteristics in style and the artistic elements of TCP based on \textit{The Six Principles}. 
Style transfer takes existing inputs (\eg photos and prepared sketches) and output generated TCP artworks~\cite{li2021immersive,9413063,zhang2020detail}. 
Several studies also modify the content of the original input such as face replacement~\cite{li2021immersive}. 
In addition to generating from existing materials, a few works have experimented with creating artworks from scratch, including both content and style generation~\cite{xue2021end}. 

\textbf{Video Generation.} 
With the development of computer animation, a line of work has explored creating videos in a TCP style, aiming to express the \textit{Spirit Resonance} in \textit{The Six Principles} from a new perspective. 
They could be classified into two categories according to the input and animation entities. 
The first category is to input an existing video and apply video style transfer to transform the whole frame~\cite{lianginstance,zhang2009video}. 
The second category is to input a TCP and animate entities such as characters and animals on the painting~\cite{lianginstance,lai2016data,xu2006animating}.
In addition, 2.5D artworks~\cite{8419282,amati2010modeling} and video scribing showing the construction of TCP for educational purposes~\cite{8113507} are explored.

\subsection{Exhibition} 
As an essential type of artwork, the exhibition is a typical stage for promoting TCP to the general audience. 
In addition to the traditional approach of arranging items one by one in the museum, studies have been exploring interactive approaches to engage audiences in the exhibition. 
Existing work could be classified into three categories according to the interactive platforms, namely, touch-screen-based, XR-based, and others. 

\textbf{Touch screens} are commonly applied in today's museums. 
Hsieh\etal~\cite{hsieh2013viewing} presented an interactive tabletop for audiences to view detailed regions of TCP. 
Subramonyam\etal~\cite{subramonyam2015sigchi} developed an iPad application, ``Rice Paper," for artists to highlight and annotate key information of the TCP for the general public. 
They also printed a tangible booklet based on this application to guide audiences. 
CalliPaint~\cite{li2014writing} is a system that allows audiences or artists to conveniently create TCP-based digital artworks.

With the development of immersive devices, \textbf{XR (Mixed Reality)} has become a new creation platform for curators and artists. 
Several studies (\cite{jin2020reconstructing,zhao2020shadowplay2,yuan2016tunable}) reconstruct TCP in the VR (Virtual Reality) environment with 3D or 2.5D characters and objects. 
They are intended to provide an immersive experience that the static TCPs cannot fulfill.
Jin\etal~\cite{jin2022immersive}) evaluated the engagement of audiences when showing TCPs on the touch screen and the VR platform. 

In addition to the touch screen and XR, other \textbf{interactive installations} are also studied, such as using sensors to capture audiences' walking in the 3D space to generate Chinese Shanshui Paintings (\cite{le2019walking}) and applying real-time projector-camera system for audiences to interact with TCP (\cite{jin2007real}). 
\section{Computational Techniques}
\label{sec:tech}

In this section, we will discuss computational techniques that have been applied to TCP. Although TCP and natural images are similar in their modality as pictures, they differ in terms of technique used for interpretation and creation. We organize and elaborate the techniques from three perspectives as follows: 

\textbf{tasks} for which computational techniques are used (\autoref{sec:tech_tasks}), 
\textbf{extracted features} which the models use for these tasks (\autoref{sec:tech_feature}), 
and \textbf{rendering techniques} in which the model generate new paintings (\autoref{sec:tech_renderin}).
% These categories of each aspect are listed in \autoref{fig:tech_classes}. 
These categories of each aspect are listed in \hyperref[fig:techClasses]{Fig.~5}. 



\begin{center}
% Figure removed\\
\vspace{3mm}
\parbox[c]{8.3cm}{\footnotesize{Fig.5.~} Summary of computational techniques applied to TCP.}
\label{fig:techClasses}
%\vspace*{.2mm}
\end{center}



% ===========================================
\subsection{Tasks}\label{sec:tech_tasks}

Previous works on TCP mainly focus on tasks that resemble those in computer vision. Nevertheless, considering that TCP have distinct characteristics (as presented in \autoref{sec:background}) compared to natural images and videos, handling these tasks requires more TCP-specific designs and contributions.


\textbf{Image Classification.} TCP can be classified according to multiple attributes (\eg artists, painting techniques, and painting subjects). Annotating TCP with attributes can improve the retrieval experience and help understand the painting. Distinguishing TCP typically requires expert knowledge, which is time-consuming and expensive. Therefore, it is necessary to train automatic models for accurate TCP classification.


Many works~\cite{li2004studying,sheng2014recognition,liu2014Classification,sun2015brushstroke,sun2016monte,jiang2021mtffnet} classify TCP according to the artists in that the painting styles of different artists tend to be distinct. 
Specifically, for describing artists' painting styles, \cite{li2004studying} adopt a mixture of multiresolution hidden Markov models, and \cite{liu2014Classification} adopt various algorithms, such as Bayes, FLD, and SVM classifiers. \cite{sun2016monte} propose artistic descriptors with Monte Carlo Convex Hull for feature selection and use SVM for classification. 
Previous methods typically utilize traditional image processing techniques for classification. In contrast, \cite{sheng2014recognition}, \cite{sun2015brushstroke}, and \cite{jiang2021mtffnet} utilize MLPs or CNNs for distinguishing artists' styles. 


TCP have two mainstream painting techniques, {\it Gongbi}~(a meticulous style, focusing on details) and {\it Xieyi}~(an ideographic style, expressing artists' feelings). \cite{jiang2019dct} apply discrete cosine transformation and CNNs for classifying {\it Gongbi} and {\it Xieyi} paintings, achieving promising performance.
Some other works focus on classifying painting subjects, including mountains-and-waters~(landscape), flowers-and-birds, and human figures. \cite{meng2018classification} apply a modified VGG~\cite{simonyanVery2015} network for painting subject classification, achieving 93.8\% accuracy. Since TCP is closely related to calligraphy, \cite{liang2010simple} distinguish TCP from calligraphy according to the Chinese characters' structures and the differences in image composition. \cite{li2020multi} propose an LSTM-based model to classify TCP into five categories: ancient trees, people, flowers-and-birds, Jiangnan water-bound town, and ink paintings. However, these categories overlap with each other in the TCP concept, which inevitably limits the model's generalization ability.



\textbf{Image Segmentation and Object Detection.} 
Image segmentation was studied in TCP with traditional morphological methods, yet recent neural network-based methods have not been explored. There are two reasons: (1) It is hard to collect large-scale training datasets of TCP, which require domain knowledge for annotation; (2) the object boundaries of TCP (specifically a key category, {\it Xieyi} painting) are hard to determine, as shown in \hyperref[fig:gongbixieyi]{Fig.~3}A. In spite of these difficulties, \cite{hu2015object} try to extract the foreground objects from a human-designed saliency map, which has a smaller dependency on the scale of data. Some works \cite{8419282,zhang2011multispectral,chen2012simulating} decompose the painting into multiple layers to obtain foreground objects or stroke segmentation. On the other hand, prefaces and postscripts are vital components of TCP~(as shown in \autoref{fig:tcp_w}A1), thus \cite{bao2010novel} propose a rule-based method to extract these scripts. 

Directly adopting natural image-tailored deep learning models for detecting objects (\eg figure, plant, flower) in TCP tends to have poor performance.~\cite{gu2019deep}. \cite{meng2019elements} utilize modified YOLOv3~\cite{redmon2018YOLOv3} and RetinaNet~\cite{lin2017Focal}, and \cite{gu2019deep} propose a modified RPN~\cite{ren2017Faster} by assembling low-level visual information and high-level semantic information. Apart from the categories that also appeared in natural images, a traditional Chinese painting may contain many seals that identify the owners and collectors in a long history, automatically detecting seals can greatly help understand the artwork~\cite{bao2009effective}. 

\textbf{Image Generation.} 
TCP have their own styles (\eg ink wash painting, white space) compared with other painting types, such as oil painting. Early works try to transfer a natural image into ink wash paintings by adjusting colors and textures~\cite{guo2015novel,dong2014real,zhang2011multispectral,yu2003image} based on tuned hyper-parameters. These early methods are learning-free, thus typically requiring tuning hyper-parameters for each image. Recent works~\cite{xue2021end,zhang2020detail,he2018chipgan,li2021immersive,9413063} have taken efforts to create TCP with Generative Adversarial Networks~\cite{goodfellow2014Generative} that transfer noises or natural images into paintings by adversarial training. Some other works~\cite{wu2018research} perform style transfer methods with CNNs by separately learning semantic information and styles from two source images and generating a blending image. These methods are machine learning models, requiring a number of training samples for learning millions of parameters. 

Apart from regarding the painting as a whole to generate, another group of works considers that Chinese paintings employ brush strokes and ink to depict objects on the paper or silk. Specifically, some works~\cite{xie2013artist,xu2005virtual,mi2004droplet,way2001synthesis,lee1999simulating,10.1145/15886.15911} model either the brush or various stroke shapes, pursuing better texture simulation of real brush strokes. With the specifically modeled brushes, users can draw Chinese paintings stroke by stroke on the screen, instead of drawing on papers with a real brush~\cite{yang2019easy,le2019walking,li2014writing}. Considering the characteristic of rice paper and silk, a large number of early works~\cite{liang2013image,xu2007generic,wang2007image,10.1145/1073204.1073221,guo2003nijimi,way2003physical,lee2001diffusion} model the ink diffusion on the rice paper and silk, seeking to improve the realism of paintings. 
In addition, previous methods focus on creating digital Chinese paintings. Yao\etal~\cite{yao2005painting} build a painting robot to handle the brushes and draw real paintings by simulating human actions. 

\textbf{Video Generation.} We divide the works on TCP video generation into three classes according to their targets: (1) displaying the painting process, (2) animating objects, and (3) natural video style transfer. For the first target, a few works~\cite{yang2013animating,8113507,yang2013animating} focus on the creation of Chinese painting, proposing to display the painting process of brush strokes for TCP. In this way, brush trajectory can be animated for both education and appreciation purposes. For the second target, some other works~\cite{liu2020animating,lai2016data,zhang2009video,xu2006animating} present methods to animate figures, flowers, and water for a vivid representation of elements in Chinese paintings. Zhao\etal~\cite{zhao2020shadowplay2} build a visualization system to build 2.5-dimensional stories about Chinese poetry, displayed by 360-degree videos, which is expected to provide an immersive appreciation of poetry in Chinese painting styles. For the third target, Liang\etal~\cite{lianginstance} display a deep learning-based multi-frame fusion framework to stylize natural videos with ink wash styles. In the process of transferring, object coherence between adjacent frames is specifically considered for semantic consistency. 

\textbf{Human-Computer Interaction.} Compared with videos, devices supporting human interactions typically have a well immersive experience to appreciate TCP. For instance, the 360-degree space in VR can better satisfy the demands of displaying handscroll. As building the scenery on the virtual reality platform is labor-intensive and expensive, existing works only focus on a single painting. Specifically, Yuan\etal~\cite{yuan2016tunable} reconstruct a painting ``Listening to a Guqin'' in the mode of virtual reality, and Jin\etal~\cite{jin2020reconstructing} build the 3D scene of the painting ``Spring Morning in the Han Palace'' using a head-mounted platform. Moreover, the immersive multi-touch tabletop is also a promising interactive method for facilitating learning and appreciating TCP~\cite{jin2022immersive,subramonyam2015sigchi,hsieh2013viewing}. Ma\etal~\cite{ma2012annotating} embed the audio explanation into the local area of the Chinese painting, thus enabling the user to move the focus to get the audio explanation of the corresponding area while enjoying the painting. Jin\etal~\cite{jin2007real} develop a real-time projector-camera system that allows users to interact with Chinese ink cartoons (\eg interacting with water can create ripples). 

\textbf{Others.} Apart from the discussed tasks in CV and HCI above, there are various tasks involving TCP, such as color recovery~\cite{ding2012research}, poet generation from TCP~\cite{feng2022ipoet,chen2021poemgeneration}, Chinese painting retrieval~\cite{dong2020feature,hung2018study,zhang2004modelling}, white space understanding~\cite{fan2019evaluation}, and digital image enhancement~\cite{guo2013image,chen2012simulating,pei2006background,pei2004virtual}. 



\subsection{Feature Extraction}\label{sec:tech_feature}
There are abundant features in TCP which distinguish them from many other painting genres.
To help users analyse and learn from TCP, many researches have extracted features for downstream tasks, such as painting classification and creation.
We summarized the features of TCP into two categories, \ie handcrafted features and learned features, according to the methodology of feature extraction.



\textbf{Handcrafted features.}
The handcrafted features are extracted by rule-based methods and reflect the specific aspects of TCP.

% Brushwork
\emph{Brushwork} is an important feature in depicting the bone method of the paintings. % \cite{zhanying2019}
Typically, the TCP are created with brushes dipped in ink, and the ink permeates through the rice paper, creating the unique shape of the brush strokes.
To automatically generate the TCP, a wide range of studies \cite{zhan2019,lai2016data,dong2014real,xu2012stroke,amati2010modeling,bai2009chinese,xu2007generic,yao2005painting} focusing on simulating the diffusion effect of color ink.
Wang\etal~\cite{wang2007image} proposed a physically-based model with texture synthesis method to simulate the color ink diffusion.
Chu\etal~\cite{10.1145/1073204.1073221} introduced a fluid flow model to calculate the percolation in the paper.
In addition, the brushwork is related to the visual complexity of the paintings. Dense thin strokes can increase the complexity while sparse thick strokes lower the complexity.
Fan\etal~\cite{fan2017visual} measured stroke thickness based on the calculation of color change.
Combining the analysis of stroke structures with the ink dispersion densities and placement densities, \cite{lai2016data} generated animations for water flow in the TCP according to the stroke pattern groups of the flow field.

% Color
\emph{Color} is another significant factor and implies the types of the TCP style \cite{guo2015novel,liu2014Classification,chen2012simulating,feng2022ipoet,lu2008content,wang2007image}.
Liu\etal~\cite{liu2014Classification} extracted the color information of the paintings by calculating the mean and variance values of the image pixels, and used them to support painting classification tasks.
Color can also be used in painting retrieval \cite{hung2018study}, painting style modeling \cite{feng2022ipoet}, and painting enhancement \cite{chen2012simulating}.
Over time, ancient Chinese paintings have faded and aged, requiring human restoration. 
Pei\etal~\cite{pei2004virtual,pei2006background} design color enhancement schemes to improve the image contrast, making the paintings more vivid and bright.

% Object
\emph{Objects}, such as the scenery in the paintings, are the basic elements of the painting composition and contain semantic information. Ding\etal~\cite{zhang2004modelling} extract objects by labeling pixels according to their connectivity in a pre-processed image, and use them for image retrieval. 
Feng\etal~\cite{feng2022ipoet} extract the objects in the TCP and use them to describe the painting content and create the painting poetry.
Zhao\etal~\cite{zhao2020shadowplay2} built a TCP style image repository for basic objects, supporting users to create immersive videos for poetry appreciation.

% Script
\emph{Scripts} are written in the empty space of the paintings and serve as complementary expression of creators' artistic ideas.
Bao\etal~\cite{bao2010novel} automatically identify and extract the scripts from the paintings according to their colors and regions.
Several studies also focus on other feature of the TCP, such as the white space~\cite{fan2017visual}, composition~\cite{sun2016monte}, and seal images~\cite{bao2009effective}.

\textbf{Learned features.}
With the fast development of deep learning technology, many studies have introduced deep learning models to learn the features of TCP.
Based on the labeled data of TCP, supervised learning methods (\eg CNN~\cite{krizhevsky2017imagenet}, VGG-16~\cite{simonyan2014very}, and YOLOv3~\cite{redmon2018YOLOv3}) are applied in object detection \cite{meng2019elements,gu2019deep,feng2022ipoet}, image classification \cite{liong2020automatic,meng2019elements,meng2018classification}.
As the stylistic features of the TCP are unique from other paintings, it is valuable to learn the stylistic features to transfer neural images into TCP.
A range of studies \cite{xue2021end,lianginstance,9413063,he2018chipgan} focus on capturing the stylistic features of TCP with adversarial training, a classical learning strategy in unsupervised learning.
In contrast, Li\etal~\cite{li2020multi} introduce weakly-supervised learning for semantic classification in the scenario with limited number of training images.
\subsection{Rendering} \label{sec:tech_renderin}
TCP rendering techniques are adopted in the process of image and animation generation. According to the focus of used techniques in rendering, we classify the TCP rendering methods into three classes: stroke-based, image-based, and geometry-based. 

\noindent\paragraph{Stroke-Based Rendering.} Users can create TCP through simulated paint brushes to be rendered on a digital canvas. Strassmann\etal~\cite{10.1145/15886.15911} propose a realistic model of painting including Brush (a series of bristles with ink supply and positions), Stroke (a set of parameters like position and pressure), Dip (a procedure to assign states to each bristle of the brush), and Paper (the carrier of ink as it comes off the brush). With the four elements, one can build an interactive or automatic painting software on the computer. 

Typically, there are two types of methods to generate brush strokes. The first method models the stroke boundaries with B'{e}zier or B-spline curves and then fills the closed curves with designed textures~\cite{chua1990bezier,nishita1993display}. The other method directly models the two-dimensional brush, such as the work of Strassmann\etal~\cite{10.1145/15886.15911}. However, the brush bristles are visually fixed in shape, users cannot apply such an e-brush with their realistic painting skills. Some works~\cite{lee1999simulating,yeh2002effects} develop ``soft'' brushes in which the shape of bristle bundle varies in response to the forces given by users. Furthermore, Xu\etal~\cite{xu2005virtual} model brushes with writing primitives (a bundle of hair bristles), instead of each single brush bristle, to improve the simulating realism. To further simplify the model complexity, Bai\etal~\cite{bai2009chinese} propose a geometry model to simulate the entire brushes, instead of large amount of bristles. A dynamic model is also introduced to simulate the brush deformation under the internal and external forces. 
Previous methods focus on modeling general brush strokes, some methods propose tailored algorithms to model specific object shapes and textures such as rocks~\cite{way2001synthesis}, trees~\cite{way2002synthesis}, bamboos~\cite{yao2005painting}, and water~\cite{zhang2009video}. Instead of small brush strokes, Fu\etal~\cite{8419282} decompose the painting into image layers with each layer representing a class of specific strokes. With these stroke layers, they can create a new high relief, an art form between 3D sculpture and 2D painting. 

For animation generation, Xu\etal~\cite{xu2006animating} build a brush stroke library obtained from painting experts, and animate the paintings by decomposing them into brush strokes and changing these strokes. Zhang\etal~\cite{zhang2009video} create running water animations with a novel proposed painting structure generation method, which is used to estimate water flow line positions. Previous methods create brush trajectory relying on manual inputs, Yang\etal~\cite{yang2013animating} automatically estimate such trajectory from the paintings by modeling the brush footprint. Considering the automation process of extracting brush trajectory, some methods~\cite{8113507} try to reconstruct the drawing process by estimating and animating the drawing order of brush strokes.

Apart from the brush strokes, ink diffusion in paper fibers structure is also a critical characteristic, which has been studies in literature~\cite{kunii1995diffusion,zhang1999simple,lee2001diffusion,yu2002model,guo2003nijimi,10.1145/1073204.1073221,wang2007image,xu2007generic,liang2013image}. Specifically, Kunii\etal~\cite{kunii1995diffusion} propose a multidimensional diffusion model to simulate the ink density distribution as in real paper. Some works~\cite{lee2001diffusion,way2003physical,way2006computer,wang2007image} further simulate the ink of brush strokes on various types of paper based on physical-based models. Considering the potential blending of multiple strokes, Yeh\etal~\cite{yeh2002effects} and Yu\etal~\cite{yu2002model} build the ink diffusion simulation model with multi-layered structures of brush and paper. Chu\etal~\cite{chu2004real} develop a system for creating painting with more complicated ink diffusion effects based on lattice Boltzmann equation, and accelerate the algorithm for real-time process by utilizing both CPU and GPU. 

\noindent\paragraph{Image-Based Rendering.} Previous methods mainly focus on modeling the brush strokes and ink diffusion for interactive painting creation. From another technical route, one can directly synthesize Chinese painting from existing images. For instance, Yu\etal~\cite{yu2003image} propose a framework for image-based painting synthesis. Specifically, the authors build a brush stroke texture primitive collection, and map those texture primitives to a constructed mask image (named as control picture in \cite{yu2003image}). Apart from blending strokes, some works propose style transfer methods to transform natural pictures to paintings, involving handcrafted feature-based flow~\cite{liang2013image,dong2014real,guo2015novel} or deep neural networks. Typically, deep neural network based style transfer methods~\cite{wu2018research,he2018chipgan,zhang2020detail,9413063,li2021immersive,xue2021end} are data-driven, characterised by training the model with tailored training program and large scale data sets. For instance, ChipGAN~\cite{he2018chipgan} consists of a generator and a discriminator, where the generator is trained to transfer photos into paintings while the discriminator is trained to discriminate the generated paintings and real paintings. Meanwhile, ChipGAN requires thousands of images for training the model due to the large scale trainable parameters. 

For animating Chinese paintings, Liu\etal~\cite{liu2020animating} propose a sample point processing method to preserve the style of brush strokes and determine control bones, and a skeleton-based deformation method for animation generation. Liang\etal~\cite{lianginstance} leverage deep neural networks for transferring natural videos into ink wash painting-style videos. In order to enhance temporal consistency between video frames, the authors introduce multi-frame fusion and implement instance-aware style transfer, which help generate paintings with proper white space. 

\noindent\paragraph{Geometry-Based Rendering.} Stroke-based and image-based rendering are both from the view of computer vision. Instead, geometry-based methods adopt the view of computer graphics. Chan\etal~\cite{chan2002two} decompose the brush stroke-like features into layers of procedural shaders, and then mix different layers to construct desired effects in 3D models. Some works~\cite{way2002synthesis,yeh2002non,xu2012stroke} synthesis objects with ink painting styles by building polygonal models or extracting silhouette, and then mapping specific textures on the models. Amati\etal~\cite{amati2010modeling} develop a webcam-based system to capture the process of user drawing plants and build corresponding 2.5-dimensional digital models. Different from previous methods, Shi~\cite{shi2017generative} build landscape paintings from a 3-dimensional city model by generating mountains from buildings and assigning ink painting styles. 








\section{Challenges and Oppurtunities}
\label{sec:challenge}

\subsection{Lack of large-scale and high-quality datasets.}
The lack of large and high-quality open-access datasets is a crucial reason hindering the further development of Traditional Chinese Painting research.
According to our paper, some works have announced that they have produced a few Chinese painting datasets~\cite{hung2018study, xue2021end, 9413063, liong2020automatic, dong2020feature}. For example, Liong et al.\cite{liong2020automatic} constructed an unlabeled dataset containing more than 1,000 Chinese paintings, and Dong et al.\cite{dong2020feature} collected a labeled dataset. However, these datasets are limited in size and have not been made open-source.
Building a large and high-quality Chinese painting dataset faces several challenges, including:

\begin{itemize}
\setlength{\itemsep}{0pt}
    \item \textit{Data availability}. As most Chinese paintings are held in museums and private collections all over the world, there is a problem of copyright ownership. It is particularly difficult to collect online resources of Chinese paintings.
    \item \textit{Data quality}. Many famous Chinese paintings are large in size, rich in details, and difficult to preserve, leading to the high cost of digitizing Chinese painting and a high technical barrier for generating high-definition pictures.
    \item \textit{Data diversity}. Chinese paintings contain relatively independent items, such as colophons and seals that can be used for analyzing historical events, collection paths, etc. However, only a few articles discussed the extraction of colophons and seals~\cite{bao2010novel,bao2009effective} without further exploration.
    \item \textit{Data annotation}. Due to the domain professionalism, annotating Chinese painting data requires high-level expertise, especially for systematic annotations based on the \textit{Six Principles of Painting}, which can be extremely expensive.
\end{itemize}

\subsection{Insufficient consideration on the TCP uniqueness.}
The analysis of TCP research tendencies based on the \textit{Six Principles of Painting} in \autoref{sec:background} reveals that most articles focus on \textit{Bone Manner, Structural Use of the Brush} (36/92), and \textit{Conform with the Objects, Obtain their Likeness} (24/92). This is mainly because the recognition, segmentation, and classification of strokes and objects in paintings can be formulated into CV tasks and solved with mature CV models.
However, researchers have paid little attention to the unique stroke system in traditional Chinese painting (3/92)~\cite{way2002synthesis,way2001synthesis,zhang1999simple}. For example, there are eighteen unique drawing methods in TCP techniques for depicting portraits (\autoref{fig:tcp_w}), and different wrinkling techniques for depicting mountains and rocks (\autoref{tbl:tcp_oil}). A detailed analysis of the stroke system sheds lights on the painter's style and the mentoring relationships between painters. Therefore, it is an issue that deserves attention in future computer fields.

Moreover, the \textit{Place and Position} aspect is not given as much attention in the current work (4/92). According to experts, the composition of Chinese paintings is crucial. Painters often use white space to convey mood, inscriptions, and seals to balance the picture's composition. Therefore, a computer-based systematic examination of the composition of Chinese paintings can aid specialists in understanding the compositional traits of paintings across time.

Regarding the study of \textit{Movement of Life} (9/92), it is important to note that in recent years, the fusion of AR and VR technology into TCP has risen, allowing audiences to experience Chinese painting from a new perspective. The development of new technology has increased the opportunities for the study and presentation of TCP, but more work of this type is required, and it may be reinforced in the future.

\subsection{Disregard for the data-linking in TCP analysis.}
Cultural heritage has various types of data, including paintings, ancient books, sculptures, architecture, and more. 
As a form of cultural heritage, TCP has garnered attention in recent years. 
However, most current research has focused solely on TCP data, and rarely combines other datasets for cross-analysis. 
From a historical research perspective, TCP collections represent only a snapshot of a certain time period. It is possible that snapshots of different artifacts describe the same social landscape. To gain a more comprehensive historical understanding, different collections of cultural relics should be viewed together. 
Therefore, it is worthwhile to pay attention to how to integrate different data related to paintings in order to restore a more accurate historical picture.

On the other hand, the use of multi-modal data to construct deep learning models is a growing trend. Multi-modal data enables better feature representation construction in the latent space, which improves the fusion of textual, visual, and other forms of information like videos and knowledge graphs. This ultimately strengthens the model's performance and enhances its generalization capacity.

\subsection{Insufficient exploration of ML methods and large models on TCP.}
TCP image data has unique characteristics compared with natural images, such as cross-domain, few annotated training samples, imbalanced classes, and variable sizes of objects. Advanced ML methods have taken profound discussions on related topics such as transfer learning, domain adaptation, domain generalization, few-shot learning, and learning with long-tailed data distribution. Therefore, applying these advanced methods to TCP can promote Chinese painting analysis from a computational view. Meanwhile, these methods can effectively reduce the demand for data annotations and alleviate the burden of collecting large-scale and high-quality annotated datasets.

In addition, large language models (\eg GPT-4~\cite{openai2023gpt4}) and large vision models (\eg CLIP~\cite{radford2021Learninga}, Stable-Diffusion~\cite{rombach2022HighResolution}) are becoming the new foundations of advanced research. Current models are not specifically adapted to TCP data, tending to generate images that ignore the Six Principles of Paintings, as well as textual descriptions that often lack detail and do not capture the essence of the painting. It is inevitable that unimodal or multimodal large models will be adopted in traditional Chinese painting research. 



\subsection{Inadequate applications for artwork creation and promotion.} 
Although a line of work has explored the generation of TCP-styled paintings and videos, the quality of these AI-generated artworks could be doubtful. 
Involving artists in the creation process with a semi-automated creation style would be a promising direction in the future. 
In addition, advanced display and interaction techniques (\eg immersive techniques) should be applied to promote TCP to the general public. New storytelling approaches should also be constructed to enhance the appreciation and understanding of TCP.  
\section{Conclusion and Future Work}
\label{sec: Conclusion and Future Work}
This paper explores formal method-based reachability analysis of variable-length time series regression neural networks (NNs) using approximate Star methods in the context of predictive maintenance, which is crucial with the rise of Industry 4.0 and the Internet of Things. The analysis considers sensor noise introduced in the data. Evaluation is conducted on two datasets, employing a unified reachability analysis that handles varying features and variable time sequence lengths while analyzing the output with acceptable upper and lower bounds. Robustness and monotonicity properties are verified for the TEDS dataset. Real-world datasets are used, but further research is needed to establish stronger connections between practical industrial problems and performance metrics. The study opens new avenues for exploring perturbation contributions to the output and extending reachability analysis to 3-dimensional time series data like videos. Future work involves verifying global monotonicity properties as well, and including more predictive maintenance and anomaly detection applications as case studies. \newblue{The study focuses solely on offline data analysis and lacks considerations for real-time stream processing and memory constraints, which present fascinating avenues for future research.}
\paragraph{\textbf{Acknowledgements.}}
The material presented in this paper is based upon work supported by the National Science Foundation (NSF) through grant numbers 1910017, 2028001, 2220418, 2220426, and 2220401, and the Defense Advanced Research Projects Agency (DARPA) under contract number FA8750-18-C-0089 and FA8750-23-C-0518, and the Air Force Office of Scientific Research (AFOSR) under contract number FA9550-22-1-0019 and FA9550-23-1-0135. Any opinions, findings, conclusions, or recommendations expressed in this paper are those of the authors and do not necessarily reflect the views of AFOSR, DARPA, or NSF. We also want to thank our colleagues, Tianshu and Barnie for their valuable feedback.
 
 


\begin{thebibliography}{100}
\footnotesize
\itemsep=-3pt plus.2pt minus.2pt
\baselineskip=13pt plus.2pt minus.2pt
%\begin{thebibliography}{100}

\bibitem{iiif}
{International Image Interoperability Framework (IIIF)}.
\newblock \url{https://iiif.io/get-started/how-iiif-works/}.

\bibitem{amati2010modeling}
Cristina Amati and Gabriel~J Brostow.
\newblock Modeling 2.5 d plants from ink paintings.
\newblock In {\em Proceedings of the Seventh Sketch-Based Interfaces and
  Modeling Symposium}, pages 41--48, 2010.

\bibitem{bai2007efficient}
Bendu Bai, Kam-Wah Wong, and Yanning Zhang.
\newblock An efficient physically-based model for chinese brush.
\newblock In {\em International Workshop on Frontiers in Algorithmics}, pages
  261--270. Springer, 2007.

\bibitem{bai2009chinese}
Bendu Bai, Yanning Zhang, Kam-Wah Wong, and Ying Li.
\newblock Chinese hairy brush: A physically-based model for calligraphy.
\newblock {\em Chinese Journal of Electronics}, 18(2):302--306, 2009.

\bibitem{bao2010novel}
Hong Bao, Ye~Liang, Hong-Zhe Liu, and De~Xu.
\newblock A novel algorithm for extraction of the scripts part in traditional
  chinese painting images.
\newblock In {\em 2010 2nd International Conference on Software Technology and
  Engineering}, volume~2, pages V2--26. IEEE, 2010.

\bibitem{bao2009effective}
Hong Bao, De~Xu, and Songhe Feng.
\newblock An effective method to detect seal images from traditional chinese
  paintings.
\newblock In {\em 2009 International Conference on Wireless Communications \&
  Signal Processing}, pages 1--4. IEEE, 2009.

\bibitem{bo2018computational}
Yihang Bo, Jinhui Yu, and Kang Zhang.
\newblock Computational aesthetics and applications.
\newblock {\em Visual Computing for Industry, Biomedicine, and Art},
  1(1):1--19, 2018.

\bibitem{bradley2018visualization}
Adam~James Bradley, Mennatallah El-Assady, Katharine Coles, Eric Alexander, Min
  Chen, Christopher Collins, Stefan J{\"a}nicke, and David~Joseph Wrisley.
\newblock Visualization and the digital humanities.
\newblock {\em IEEE computer graphics and applications}, 38(6):26--38, 2018.

\bibitem{chan2002two}
Ching Chan, Ergun Akleman, and Jianer Chen.
\newblock Two methods for creating chinese painting.
\newblock In {\em 10th Pacific Conference on Computer Graphics and
  Applications, 2002. Proceedings.}, pages 403--412. IEEE, 2002.

\bibitem{chen2021poemgeneration}
Jiazhou Chen, Keshu Huang, Yingchaojie Feng, Wei Zhang, Siwei Tan, and Wei
  Chen.
\newblock Automatic poetry generation based on ancient chinese paintings.
\newblock {\em Journal of Computer-Aided Design \& Computer Graphics}, 33(7):7,
  2021.

\bibitem{chen2012simulating}
Lieu-Hen Chen, Meng-Feng TSAI, Chien-Hui HSU, and Yu-Sheng CHEN.
\newblock Simulating aging and reverse-aging phenomena of traditional chinese
  paintings.
\newblock pages 4M1IOS3c5--4M1IOS3c5, 2012.

\bibitem{cheng2018essential}
Maria Cheng, Tang~Wai Hung, et~al.
\newblock {\em Essential terms of Chinese painting}.
\newblock City University of HK Press, 2018.

\bibitem{10.1145/1073204.1073221}
Nelson S.-H. Chu and Chiew-Lan Tai.
\newblock Moxi: Real-time ink dispersion in absorbent paper.
\newblock {\em ACM Trans. Graph.}, 24(3):504-511, jul 2005.

\bibitem{chu2004real}
Nelson~SH Chu and Chiew-Lan Tai.
\newblock Real-time painting with an expressive virtual chinese brush.
\newblock {\em IEEE Computer Graphics and applications}, 24(5):76--85, 2004.

\bibitem{chua1990bezier}
Yap~Siong Chua.
\newblock Bezier brushstrokes.
\newblock {\em Computer-Aided Design}, 22(9):550--555, 1990.

\bibitem{ding2012research}
Haiyan Ding and Huaidong Ding.
\newblock Research on computer color recovery system for traditional chinese
  painting.
\newblock In {\em 2012 International Conference on Systems and Informatics
  (ICSAI2012)}, pages 1985--1988. IEEE, 2012.

\bibitem{diverdi2015modular}
Stephen DiVerdi.
\newblock A modular framework for digital painting.
\newblock {\em IEEE Transactions on Visualization and Computer Graphics},
  21(7):783--793, 2015.

\bibitem{dong2014real}
Lixing Dong, Shufang Lu, and Xiaogang Jin.
\newblock Real-time image-based chinese ink painting rendering.
\newblock {\em Multimedia tools and applications}, 69(3):605--620, 2014.

\bibitem{dong2020feature}
Zhenhao Dong, Jing Wan, Chaoyue Li, Han Jiang, Yingge Qian, and Wenxie Pan.
\newblock Feature fusion based cross-modal retrieval for traditional chinese
  painting.
\newblock In {\em 2020 International Conference on Culture-oriented Science \&
  Technology (ICCST)}, pages 383--387. IEEE, 2020.

\bibitem{fan2017visual}
Zhen~Bao Fan, Yi-Na Li, Jinhui Yu, and Kang Zhang.
\newblock Visual complexity of chinese ink paintings.
\newblock In {\em Proceedings of the ACM Symposium on Applied Perception},
  pages 1--8, 2017.

\bibitem{fan2019evaluation}
ZhenBao Fan, Kang Zhang, and XianJun~Sam Zheng.
\newblock Evaluation and analysis of white space in wu guanzhong's chinese
  paintings.
\newblock {\em Leonardo}, 52(2):111--116, 2019.

\bibitem{feng2022ipoet}
Yingchaojie Feng, Jiazhou Chen, Keyu Huang, Jason~K Wong, Hui Ye, Wei Zhang,
  Rongchen Zhu, Xiaonan Luo, and Wei Chen.
\newblock ipoet: interactive painting poetry creation with visual multimodal
  analysis.
\newblock {\em Journal of Visualization}, 25(3):671--685, 2022.

\bibitem{8419282}
Yunfei Fu, Hongchuan Yu, Chih-Kuo Yeh, Jianjun Zhang, and Tong-Yee Lee.
\newblock High relief from brush painting.
\newblock {\em IEEE Transactions on Visualization and Computer Graphics},
  25(9):2763--2776, 2019.

\bibitem{goodfellow2014Generative}
Ian Goodfellow, Jean {Pouget-Abadie}, Mehdi Mirza, Bing Xu, David
  {Warde-Farley}, Sherjil Ozair, Aaron Courville, and Yoshua Bengio.
\newblock Generative {{Adversarial Nets}}.
\newblock In Z.~Ghahramani, M.~Welling, C.~Cortes, N.~D. Lawrence, and K.~Q.
  Weinberger, editors, {\em Advances in {{Neural Information Processing
  Systems}} 27}, pages 2672--2680. 2014.

\bibitem{gu2019deep}
Qianqian Gu and Ross King.
\newblock Deep learning does not generalize well to recognizing cats and dogs
  in chinese paintings.
\newblock In {\em International Conference on Discovery Science}, pages
  166--175. Springer, 2019.

\bibitem{guan2005automatic}
Xiaohui Guan, Gang Pan, and Zhaohui Wu.
\newblock Automatic categorization of traditional chinese painting images with
  statistical gabor feature and color feature.
\newblock In {\em International Conference on Computational Science}, pages
  743--750. Springer, 2005.

\bibitem{guo2015novel}
Fan Guo, Hui Peng, and Jin Tang.
\newblock A novel method of converting photograph into chinese ink painting.
\newblock {\em IEEJ Transactions on Electrical and Electronic Engineering},
  10(3):320--329, 2015.

\bibitem{guo2013image}
Fan Guo, Jin Tang, and Hui Peng.
\newblock Image recovery for ancient chinese paintings.
\newblock 2013.

\bibitem{guo2003nijimi}
Qinglian Guo and Tosiyasu~L Kunii.
\newblock "nijimi" rendering algorithm for creating quality black ink
  paintings.
\newblock In {\em Proceedings Computer Graphics International 2003}, pages
  152--159. IEEE, 2003.

\bibitem{hxz2022painting}
Xizai Han.
\newblock {The Night Revels of Han Xizai}.
\newblock \url{https://www.dpm.org.cn/collection/paint/228200.html}.

\bibitem{he2018chipgan}
Bin He, Feng Gao, Daiqian Ma, Boxin Shi, and Ling-Yu Duan.
\newblock Chipgan: A generative adversarial network for chinese ink wash
  painting style transfer.
\newblock In {\em Proceedings of the 26th ACM international conference on
  Multimedia}, pages 1172--1180, 2018.

\bibitem{hsieh2013viewing}
Chun-ko Hsieh, Yi-Ping Hung, Moshe Ben-Ezra, and Hsin-Fang Hsieh.
\newblock Viewing chinese art on an interactive tabletop.
\newblock {\em IEEE computer graphics and applications}, 33(3):16--21, 2013.

\bibitem{hu2015object}
Zhengkun Hu and Tingmei Wang.
\newblock Object extraction in chinese painting base on visual saliency.
\newblock In {\em 2015 8th International Symposium on Computational
  Intelligence and Design (ISCID)}, volume~2, pages 493--496. IEEE, 2015.

\bibitem{huang2019learning}
Zhewei Huang, Wen Heng, and Shuchang Zhou.
\newblock Learning to paint with model-based deep reinforcement learning.
\newblock In {\em Proceedings of the IEEE/CVF International Conference on
  Computer Vision}, pages 8709--8718, 2019.

\bibitem{hung2018study}
Chia-Ching Hung.
\newblock A study on a content-based image retrieval technique for chinese
  paintings.
\newblock {\em The Electronic Library}, 2018.

\bibitem{jiang2006effective}
Shuqiang Jiang, Qingming Huang, Qixiang Ye, and Wen Gao.
\newblock An effective method to detect and categorize digitized traditional
  chinese paintings.
\newblock {\em Pattern Recognition Letters}, 27(7):734--746, 2006.

\bibitem{jiang2004categorizing}
Shuqiang Jiang and Tiejun Huang.
\newblock Categorizing traditional chinese painting images.
\newblock In {\em Pacific-Rim Conference on Multimedia}, pages 1--8. Springer,
  2004.

\bibitem{jiang2021mtffnet}
Wei Jiang, Xiaoyu Wang, Jinchang Ren, Sen Li, Meijun Sun, Zheng Wang, and
  Jesse~S Jin.
\newblock Mtffnet: a multi-task feature fusion framework for chinese painting
  classification.
\newblock {\em Cognitive Computation}, 13(5):1287--1296, 2021.

\bibitem{jiang2019dct}
Wei Jiang, Zheng Wang, Jesse~S Jin, Yahong Han, and Meijun Sun.
\newblock Dct--cnn-based classification method for the gongbi and xieyi
  techniques of chinese ink-wash paintings.
\newblock {\em Neurocomputing}, 330:280--286, 2019.

\bibitem{jin2007real}
Ming Jin, Hui Zhang, Xubo Yang, and Shuangjiu Xiao.
\newblock A real-time procam system for interaction with chinese ink-and-wash
  cartoons.
\newblock In {\em 2007 IEEE Conference on Computer Vision and Pattern
  Recognition}, pages 1--2. IEEE, 2007.

\bibitem{jin2022immersive}
Sheng Jin, Min Fan, and Aynur Kadir.
\newblock Immersive spring morning in the han palac e: Learning traditional
  chinese art via virtual reality and multi-touch tabletop.
\newblock {\em International Journal of Human--Computer Interaction},
  38(3):213--226, 2022.

\bibitem{jin2020reconstructing}
Sheng Jin, Min Fan, Yongchao Wang, and Qi~Liu.
\newblock Reconstructing traditional chinese paintings with immersive virtual
  reality.
\newblock In {\em Extended Abstracts of the 2020 CHI Conference on Human
  Factors in Computing Systems}, CHI EA '20, page 1-8, New York, NY, USA,
  2020. Association for Computing Machinery.

\bibitem{xy2022painting}
Liang Kai.
\newblock {Immortal in Splashed Ink}.
\newblock
  \url{https://digitalarchive.npm.gov.tw/Painting/Content?pid=1819\&Dept=P}.

\bibitem{kotovenko2021rethinking}
Dmytro Kotovenko, Matthias Wright, Arthur Heimbrecht, and Bjorn Ommer.
\newblock Rethinking style transfer: From pixels to parameterized brushstrokes.
\newblock In {\em Proceedings of the IEEE/CVF Conference on Computer Vision and
  Pattern Recognition}, pages 12196--12205, 2021.

\bibitem{krizhevsky2017imagenet}
Alex Krizhevsky, Ilya Sutskever, and Geoffrey~E Hinton.
\newblock Imagenet classification with deep convolutional neural networks.
\newblock {\em Communications of the ACM}, 60(6):84--90, 2017.

\bibitem{kunii1995diffusion}
Tosiyasu~L Kunii, Gleb~V Nosovskij, and Takafumi Hayashi.
\newblock A diffusion model for computer animation of diffuse ink painting.
\newblock In {\em Proceedings Computer Animation'95}, pages 98--102. IEEE,
  1995.

\bibitem{kyprianidis2012state}
Jan~Eric Kyprianidis, John Collomosse, Tinghuai Wang, and Tobias Isenberg.
\newblock State of the ``art'': A taxonomy of artistic stylization techniques
  for images and video.
\newblock {\em IEEE transactions on visualization and computer graphics},
  19(5):866--885, 2012.

\bibitem{lai2016data}
Yu-Chi Lai, Bo-An Chen, Kuo-Wei Chen, Wei-Lin Si, Chih-Yuan Yao, and Eugene
  Zhang.
\newblock Data-driven npr illustrations of natural flows in chinese painting.
\newblock {\em IEEE transactions on visualization and computer graphics},
  23(12):2535--2549, 2016.

\bibitem{le2019walking}
Aven Le~Zhou.
\newblock Walking through shanshui: Generating chinese shanshui paintings via
  real-time tracking of human position.
\newblock In {\em Proceedings of the IEEE/CVF International Conference on
  Computer Vision Workshops}, pages 0--0, 2019.

\bibitem{lee1999simulating}
Jintae Lee.
\newblock Simulating oriental black-ink painting.
\newblock {\em IEEE Computer Graphics and Applications}, 19(3):74--81, 1999.

\bibitem{lee2001diffusion}
Jintae Lee.
\newblock Diffusion rendering of black ink paintings using new paper and ink
  models.
\newblock {\em Computers \& Graphics}, 25(2):295--308, 2001.

\bibitem{li2020multi}
Daxiang Li and Yue Zhang.
\newblock Multi-instance learning algorithm based on lstm for chinese painting
  image classification.
\newblock {\em IEEE Access}, 8:179336--179345, 2020.

\bibitem{li2004studying}
Jia Li and James~Ze Wang.
\newblock Studying digital imagery of ancient paintings by mixtures of
  stochastic models.
\newblock {\em IEEE transactions on image processing}, 13(3):340--353, 2004.

\bibitem{li2014writing}
Jiajia Li, Grace Ngai, Stephen~CF Chan, Kien~A Hua, Hong~Va Leong, and Alvin
  Chan.
\newblock From writing to painting: A kinect-based cross-modal chinese painting
  generation system.
\newblock In {\em Proceedings of the 22nd ACM international conference on
  Multimedia}, pages 57--66, 2014.

\bibitem{li2021immersive}
Jiayue Li, Qing Wang, Shiji Li, Qiang Zhong, and Qian Zhou.
\newblock Immersive traditional chinese portrait painting: Research on style
  transfer and face replacement.
\newblock In {\em Chinese Conference on Pattern Recognition and Computer Vision
  (PRCV)}, pages 192--203. Springer, 2021.

\bibitem{li2022computing}
Meng Li, Yun Wang, and Ying-Qing Xu.
\newblock Computing for chinese cultural heritage.
\newblock {\em Visual Informatics}, 6(1):1--13, 2022.

\bibitem{lianginstance}
Hao Liang, Shuai Yang, Wenjing Wang, and Jiaying Liu.
\newblock Instance-aware coherent video style transfer for chinese ink wash
  painting.
\newblock 2021.

\bibitem{liang2013image}
Lingyu Liang and Lianwen Jin.
\newblock Image-based rendering for ink painting.
\newblock In {\em 2013 IEEE International Conference on Systems, Man, and
  Cybernetics}, pages 3950--3954. IEEE, 2013.

\bibitem{liang2010simple}
Ye~Liang, Hong Bao, and Hong-Zhe Liu.
\newblock A simple method for classification of traditional chinese painting
  and calligraphy images.
\newblock In {\em 2010 International Conference on Educational and Information
  Technology}, volume~3, pages V3--340. IEEE, 2010.

\bibitem{gb2022painting}
Ma~Lin.
\newblock {King Yu of Xia}.
\newblock
  \url{https://digitalarchive.npm.gov.tw/Painting/Content?pid=14658\&Dept=P}.

\bibitem{lin2017Focal}
Tsung-Yi Lin, Priya Goyal, Ross Girshick, Kaiming He, and Piotr Dollar.
\newblock Focal {{Loss}} for {{Dense Object Detection}}.
\newblock In {\em Proceedings of the {{IEEE International Conference}} on
  {{Computer Vision}} ({{ICCV}})}, October 2017.

\bibitem{lin1967art}
Yutang Lin.
\newblock {\em The Chinese Theory of Art}.
\newblock G.P. Putnam's Sons, NY, 1967.

\bibitem{liong2020automatic}
Sze-Teng Liong, Yen-Chang Huang, Shumeng Li, Zhongkai Huang, Jingyang Ma, and
  Yee~Siang Gan.
\newblock Automatic traditional chinese painting classification: A benchmarking
  analysis.
\newblock {\em Computational Intelligence}, 36(3):1183--1199, 2020.

\bibitem{litwinowicz1997processing}
Peter Litwinowicz.
\newblock Processing images and video for an impressionist effect.
\newblock In {\em Proceedings of the 24th annual conference on Computer
  graphics and interactive techniques}, pages 407--414, 1997.

\bibitem{liu2014Classification}
Chi Liu and He~Jiang.
\newblock Classification of traditional chinese paintings based on supervised
  learning methods.
\newblock In {\em 2014 IEEE International Conference on Signal Processing,
  Communications and Computing (ICSPCC)}, pages 641--644, 2014.

\bibitem{liu2020animating}
Damon Shing-Min Liu, Ching-I Cheng, and Mei-Lin Liu.
\newblock Animating characters in chinese painting using two-dimensional
  skeleton-based deformation.
\newblock {\em Multimedia Tools and Applications}, 79(27):20343--20371, 2020.

\bibitem{liu2021paint}
Songhua Liu, Tianwei Lin, Dongliang He, Fu~Li, Ruifeng Deng, Xin Li, Errui
  Ding, and Hao Wang.
\newblock Paint transformer: Feed forward neural painting with stroke
  prediction.
\newblock In {\em Proceedings of the IEEE/CVF international conference on
  computer vision}, pages 6598--6607, 2021.

\bibitem{lu2008content}
Guanming Lu, Zhong Gao, Danni Qin, Xin Zhao, and Mengjue Liu.
\newblock Content-based identifying and classifying traditional chinese
  painting images.
\newblock In {\em 2008 Congress on Image and Signal Processing}, volume~4,
  pages 570--574. IEEE, 2008.

\bibitem{ma2012annotating}
Wei Ma, Yizhou Wang, Ying-Qing Xu, Qiong Li, Xin Ma, and Wen Gao.
\newblock Annotating traditional chinese paintings for immersive virtual
  exhibition.
\newblock {\em Journal on Computing and Cultural Heritage (JOCCH)}, 5(2):1--12,
  2012.

\bibitem{meng2019elements}
Qingyu Meng, Kaiyue Li, Mingquan Zhou, and Huanhuan Zhang.
\newblock The elements extraction on traditional chinese paintings based on
  object detection.
\newblock In {\em Proceedings of the 2019 2nd Artificial Intelligence and Cloud
  Computing Conference}, pages 111--116, 2019.

\bibitem{meng2018classification}
Qingyu Meng, Huanhuan Zhang, Mingquan Zhou, Shifeng Zhao, and Pengbo Zhou.
\newblock The classification of traditional chinese painting based on cnn.
\newblock In {\em International Conference on Cloud Computing and Security},
  pages 232--241. Springer, 2018.

\bibitem{mi2004droplet}
Xiao-Feng Mi, Min Tang, and Jin-Xiang Dong.
\newblock Droplet: a virtual brush model to simulate chinese calligraphy and
  painting.
\newblock {\em Journal of Computer Science and Technology}, 19(3):393--404,
  2004.

\bibitem{mi2002droplet}
Xiaofeng Mi, Jie Xu, Min Tang, and Jinxiang Dong.
\newblock The droplet virtual brush for chinese calligraphic character
  modeling.
\newblock In {\em Sixth IEEE Workshop on Applications of Computer Vision,
  2002.(WACV 2002). Proceedings.}, pages 330--334. IEEE, 2002.

\bibitem{nishita1993display}
Tomoyuki Nishita, Shinichi Takita, and Eihachiro Nakamae.
\newblock A display algorithm of brush strokes using bezier functions.
\newblock In {\em Communicating with virtual worlds}, pages 244--257. Springer,
  1993.

\bibitem{pei2006background}
S~Pei and Y~Chiu.
\newblock Background adjustment and saturation enhancement in ancient chinese
  paintings.
\newblock {\em IEEE transactions on image processing}, 15(10):3230--3234, 2006.

\bibitem{pei2004virtual}
Soo-Chang Pei, Yi-Chong Zeng, and Ching-Hua Chang.
\newblock Virtual restoration of ancient chinese paintings using color contrast
  enhancement and lacuna texture synthesis.
\newblock {\em IEEE transactions on image processing}, 13(3):416--429, 2004.

\bibitem{redmon2018YOLOv3}
Joseph Redmon and Ali Farhadi.
\newblock {{YOLOv3}}: {{An Incremental Improvement}}, April 2018.

\bibitem{ren2017Faster}
Shaoqing Ren, Kaiming He, Ross Girshick, and Jian Sun.
\newblock Faster {{R-CNN}}: {{Towards Real-Time Object Detection}} with
  {{Region Proposal Networks}}.
\newblock {\em IEEE Trans. Pattern Anal. Mach. Intell.}, 39(6):1137--1149, June
  2017.

\bibitem{sheng2014}
Jiachuan Sheng.
\newblock Automatic categorization of traditional chinese paintings based on
  wavelet transform.
\newblock {\em Computer Science}, 41(2):317--319, 2014.

\bibitem{sheng2013style}
Jiachuan Sheng and Jianmin Jiang.
\newblock Style-based classification of chinese ink and wash paintings.
\newblock {\em Optical Engineering}, 52(9):093101, 2013.

\bibitem{sheng2014recognition}
Jiachuan Sheng and Jianmin Jiang.
\newblock Recognition of chinese artists via windowed and entropy balanced
  fusion in classification of their authored ink and wash paintings (iwps).
\newblock {\em Pattern Recognition}, 47(2):612--622, 2014.

\bibitem{shi2017generative}
Weili Shi.
\newblock A generative approach to chinese shanshui painting.
\newblock {\em IEEE Computer Graphics and Applications}, 37(1):15--19, 2017.

\bibitem{simonyan2014very}
Karen Simonyan and Andrew Zisserman.
\newblock Very deep convolutional networks for large-scale image recognition.
\newblock {\em arXiv preprint arXiv:1409.1556}, 2014.

\bibitem{simonyanVery2015}
Karen Simonyan and Andrew Zisserman.
\newblock Very {{Deep Convolutional Networks}} for {{Large-Scale Image
  Recognition}}.
\newblock In {\em 3rd {{International Conference}} on {{Learning
  Representations}}, {{ICLR}} 2015, {{San Diego}}, {{CA}}, {{USA}}, {{May}}
  7-9, 2015, {{Conference Track Proceedings}}}, 2015.

\bibitem{10.1145/15886.15911}
Steve Strassmann.
\newblock Hairy brushes.
\newblock {\em SIGGRAPH Comput. Graph.}, 20(4):225-232, aug 1986.

\bibitem{subramonyam2015sigchi}
Hariharan Subramonyam, Yuncheng Shen, and Samantha~Lauren Jones.
\newblock Sigchi: Enabling context for traditional chinese paintings with
  "rice paper".
\newblock In {\em Proceedings of the 33rd Annual ACM Conference Extended
  Abstracts on Human Factors in Computing Systems}, CHI EA '15, page 49-54,
  New York, NY, USA, 2015. Association for Computing Machinery.

\bibitem{sun2015brushstroke}
Meijun Sun, Dong Zhang, Jinchang Ren, Zheng Wang, and Jesse~S Jin.
\newblock Brushstroke based sparse hybrid convolutional neural networks for
  author classification of chinese ink-wash paintings.
\newblock In {\em 2015 IEEE International Conference on Image Processing
  (ICIP)}, pages 626--630. IEEE, 2015.

\bibitem{sun2016monte}
Meijun Sun, Dong Zhang, Zheng Wang, Jinchang Ren, and Jesse~S Jin.
\newblock Monte carlo convex hull model for classification of traditional
  chinese paintings.
\newblock {\em Neurocomputing}, 171:788--797, 2016.

\bibitem{8113507}
Fan Tang, Weiming Dong, Yiping Meng, Xing Mei, Feiyue Huang, Xiaopeng Zhang,
  and Oliver Deussen.
\newblock Animated construction of chinese brush paintings.
\newblock {\em IEEE Transactions on Visualization and Computer Graphics},
  24(12):3019--3031, 2018.

\bibitem{van1962way}
Fritz Van~Briessen.
\newblock {\em The Way of the Brush: Painting Techniques of China and Japan}.
\newblock Tuttle Publishing, 1962.

\bibitem{nw2022painting}
Rembrandt van Rijn.
\newblock {The Night Watch}.
\newblock \url{https://hart.amsterdam/collectie/object/amcollect/38543}.

\bibitem{wang2007image}
Chung-Ming Wang and Ren-Jie Wang.
\newblock Image-based color ink diffusion rendering.
\newblock {\em IEEE Transactions on Visualization and Computer Graphics},
  13(2):235--246, 2007.

\bibitem{9413063}
Rui Wang, Huaibo Huang, Aihua Zheng, and Ran He.
\newblock Attentional wavelet network for traditional chinese painting
  transfer.
\newblock In {\em 2020 25th International Conference on Pattern Recognition
  (ICPR)}, pages 3077--3083, 2021.

\bibitem{warwick2012digital}
Claire Warwick, Melissa Terras, and Julianne Nyhan.
\newblock {\em Digital humanities in practice}.
\newblock Facet Publishing, 2012.

\bibitem{way2003physical}
De-Lor Way, Shen-Wen Huang, and Zen-Chung Shih.
\newblock Physical-based model of ink diffusion in chinese paintings.
\newblock 2003.

\bibitem{way2002synthesis}
De-Lor Way, Yu-Ru Lin, and Zen-Chung Shih.
\newblock The synthesis of trees in chinese landscape painting using silhoutte
  and texture strokes.
\newblock 2002.

\bibitem{way2006computer}
Der-Lor Way, Wei-Jin Lin, and Zen-Chung Shih.
\newblock Computer-generated chinese color ink paintings.
\newblock {\em Journal of the Chinese Institute of Engineers},
  29(6):1041--1050, 2006.

\bibitem{way2001synthesis}
Der-Lor Way and Zen-Chung Shih.
\newblock The synthesis of rock textures in chinese landscape painting.
\newblock In {\em Computer Graphics Forum}, volume~20, pages 123--131. Wiley
  Online Library, 2001.

\bibitem{wu2018research}
Bing Wu and Qingshuang Dong.
\newblock Research on the synthetic method of ink painting based on
  convolutional neural network.
\newblock In {\em Tenth International Conference on Digital Image Processing
  (ICDIP 2018)}, volume 10806, page 108062B. International Society for Optics
  and Photonics, 2018.

\bibitem{wu2013modeling}
Xinming Wu, Guofeng Li, and Ye~Liang.
\newblock Modeling chinese painting images based on ontology.
\newblock In {\em 2013 International Conference on Information Technology and
  Applications}, pages 113--116. IEEE, 2013.

\bibitem{xie2013artist}
Ning Xie, Hirotaka Hachiya, and Masashi Sugiyama.
\newblock Artist agent: A reinforcement learning approach to automatic stroke
  generation in oriental ink painting.
\newblock {\em IEICE TRANSACTIONS on Information and Systems},
  96(5):1134--1144, 2013.

\bibitem{xu2005virtual}
Songhua Xu, CM~Francis Lau, Congfu Xu, and Yunhe Pan.
\newblock Virtual hairy brush for digital painting and calligraphy.
\newblock {\em Science in China Series F: Information Sciences}, 48:285--303,
  2005.

\bibitem{xu2007generic}
Songhua Xu, Haisheng Tan, Xiantao Jiao, Francis~CM Lau, and Yunhe Pan.
\newblock A generic pigment model for digital painting.
\newblock In {\em Computer Graphics Forum}, volume~26, pages 609--618. Wiley
  Online Library, 2007.

\bibitem{xu2006animating}
Songhua Xu, Yingqing Xu, Sing~Bing Kang, David~H Salesin, Yunhe Pan, and
  Heung-Yeung Shum.
\newblock Animating chinese paintings through stroke-based decomposition.
\newblock {\em ACM Transactions on Graphics (TOG)}, 25(2):239--267, 2006.

\bibitem{xu2012stroke}
Tian-Chen Xu, Li-Jie Yang, and En-Hua Wu.
\newblock Stroke-based real-time ink wash painting style rendering for
  geometric models.
\newblock In {\em SIGGRAPH Asia 2012 Technical Briefs}, pages 1--4. 2012.

\bibitem{xue2021end}
Alice Xue.
\newblock End-to-end chinese landscape painting creation using generative
  adversarial networks.
\newblock In {\em Proceedings of the IEEE/CVF Winter Conference on Applications
  of Computer Vision}, pages 3863--3871, 2021.

\bibitem{yang2013animating}
LiJie Yang and TianChen Xu.
\newblock Animating chinese ink painting through generating reproducible brush
  strokes.
\newblock {\em Science China Information Sciences}, 56(1):1--13, 2013.

\bibitem{yang2019easy}
Lijie Yang, Tianchen Xu, Jixiang Du, and Enhua Wu.
\newblock Easy drawing: Generation of artistic chinese flower painting by
  stroke-based stylization.
\newblock {\em Ieee Access}, 7:35449--35456, 2019.

\bibitem{yao2005painting}
Fenghui Yao and Guifeng Shao.
\newblock Painting brush control techniques in chinese painting robot.
\newblock In {\em ROMAN 2005. IEEE International Workshop on Robot and Human
  Interactive Communication, 2005.}, pages 462--467. IEEE, 2005.

\bibitem{yeh2002effects}
Jeng-sheng Yeh, Ting-yu Lien, and Ming Ouhyoung.
\newblock On the effects of haptic display in brush and ink simulation for
  chinese painting and calligraphy.
\newblock In {\em 10th Pacific Conference on Computer Graphics and
  Applications, 2002. Proceedings.}, pages 439--441. IEEE, 2002.

\bibitem{yeh2002non}
Jun-Wei Yeh and Ming Ouhyoung.
\newblock Non-photorealistic rendering in chinese painting of animals.
\newblock {\em Journal of System Simulation 14 (6): 1220-1224}, (1262), 2002.

\bibitem{yu2003image}
JinHui Yu, GuoMing Luo, and QunSheng Peng.
\newblock Image-based synthesis of chinese landscape painting.
\newblock {\em Journal of Computer Science and Technology}, 18(1):22--28, 2003.

\bibitem{yu2002model}
Young~Jung Yu, Young~Bok Lee, Hwan~Gue Cho, and Do~Hoon Lee.
\newblock A model based technique for realistic oriental painting.
\newblock In {\em 10th Pacific Conference on Computer Graphics and
  Applications, 2002. Proceedings.}, pages 452--453. IEEE, 2002.

\bibitem{yuan2016tunable}
Chen Yuan and Ze~Yun.
\newblock Tunable, a vr reconstruction of "listening to a guqin" from emperor
  zhao ji.
\newblock In {\em SIGGRAPH ASIA 2016 VR Showcase}, pages 1--2. 2016.

\bibitem{zhan2019}
Ying Zhan, Yan Gao, and Linyun Xie.
\newblock Aesthetic feature analysis and classification of chinese traditional
  painting.
\newblock {\em Journal of Beijing University of Aeronautics and Astronautics},
  45(12):2514--2522, 2019.

\bibitem{zhang2004modelling}
Danging Zhang, Binh Pham, and Yuefeng Li.
\newblock Modelling traditional chinese paintings for content-based image
  classification and retrieval.
\newblock In {\em 10th International Multimedia Modelling Conference, 2004.
  Proceedings.}, pages 258--264. IEEE, 2004.

\bibitem{zhang2020detail}
Fengquan Zhang, Huaming Gao, and Yuping Lai.
\newblock Detail-preserving cyclegan-adain framework for image-to-ink painting
  translation.
\newblock {\em IEEE Access}, 8:132002--132011, 2020.

\bibitem{zhang2021comprehensive}
Jiajing Zhang, Yongwei Miao, and Jinhui Yu.
\newblock A comprehensive survey on computational aesthetic evaluation of
  visual art images: Metrics and challenges.
\newblock {\em IEEE Access}, 9:77164--77187, 2021.

\bibitem{zhang2011multispectral}
Jiawan Zhang, Yi~Zhang, Shengping Zhang, Lixia Yan, and Jinyan Chen.
\newblock Multispectral image matting of ancient chinese paintings.

\bibitem{zhang1999simple}
Qing Zhang, Youetsu Sato, Jun-ya Takahashi, Kazunobu Muraoka, and Norishige
  Chiba.
\newblock Simple cellular automaton-based simulation of ink behaviour and its
  application to suibokuga-like 3d rendering of trees.
\newblock {\em The Journal of Visualization and Computer Animation},
  10(1):27--37, 1999.

\bibitem{zhang2009video}
SongHai Zhang, Tao Chen, YiFei Zhang, ShiMin Hu, and Ralph Martin.
\newblock Video-based running water animation in chinese painting style.
\newblock {\em Science in China Series F: Information Sciences},
  52(2):162--171, 2009.

\bibitem{zhao2020shadowplay2}
Zhenjie Zhao and Xiaojuan Ma.
\newblock Shadowplay2. 5d: A 360-degree video authoring tool for immersive
  appreciation of classical chinese poetry.
\newblock {\em Journal on Computing and Cultural Heritage (JOCCH)},
  13(1):1--20, 2020.

\bibitem{zheng2018strokenet}
Ningyuan Zheng, Yifan Jiang, and Dingjiang Huang.
\newblock Strokenet: A neural painting environment.
\newblock In {\em International Conference on Learning Representations}, 2018.

\bibitem{zheng2017chinese}
Wang Zheng, Li~Haoyue, Xu~Hongshan, and Sun Meijun.
\newblock Chinese painting emotion classification based onconvolution neural
  network and svm.
\newblock {\em Journal of Nanjing Normal University (Natural Science Edition)},
  2017.

\bibitem{zou2021stylized}
Zhengxia Zou, Tianyang Shi, Shuang Qiu, Yi~Yuan, and Zhenwei Shi.
\newblock Stylized neural painting.
\newblock In {\em Proceedings of the IEEE/CVF Conference on Computer Vision and
  Pattern Recognition}, pages 15689--15698, 2021.


\bibitem{openai2023gpt4}  
OpenAI.  
\newblock GPT-4 Technical Report.  
\newblock Technical report, arXiv:2303.08774, cs.CL, 2023.  

\bibitem{radford2021Learninga}  
Alec Radford, Jong Wook Kim, Chris Hallacy, Aditya Ramesh, Gabriel Goh, Sandhini Agarwal, Girish Sastry, Amanda Askell, Pamela Mishkin, Jack Clark, Gretchen Krueger, and Ilya Sutskever.  
\newblock Learning Transferable Visual Models From Natural Language Supervision.  
\newblock In {\em Proceedings of the 38th International Conference on Machine Learning}, pages 8748--8763, Jul. 2021, PMLR.  

\bibitem{rombach2022HighResolution}  
Robin Rombach, Andreas Blattmann, Dominik Lorenz, Patrick Esser, and Bj{\"o}rn Ommer.  
\newblock High-Resolution Image Synthesis With Latent Diffusion Models.  
\newblock In {\em CVPR}, pages 10684--10695, 2022.  

%\end{thebibliography}




% {\footnotesize
% \bibliographystyle{plain}
% \bibliography{reference.bib}}



\end{thebibliography}

\end{multicols}
\label{last-page}
\end{document}




%\section{Introduction}

%Journal of Computer Science and Technology (JCST) is an international forum for scientists and engineers involved in all aspects of computer science and technology to publish high quality, refereed papers. It is an international research journal sponsored by Institute of Computing Technology (ICT), Chinese Academy of Sciences (CAS), and China Computer Federation (CCF). The journal is jointly published by Science Press of China and Springer on a bimonthly basis in English.

%The journal offers survey and review articles from experts in the field, promoting insight and understanding of the state of the art, and trends in technology. The contents include original research and innovative applications from all parts of the world. The journal presents mostly previously unpublished materials.

%The coverage of JCST includes computer architecture and systems, artificial intelligence and pattern recognition, computer networks and distributed computing, computer graphics and multimedia, software systems, data management and data mining, theory and algorithms, emerging areas, and more.

%Enhanced versions of papers previously published in conference proceedings may be considered provided:

%1) The version submitted to JCST has at least 30\% new kernel contribution (not including more related work, detailed experimental data, etc.) against the conference version.

%2) The conference version should be cited as a reference, and the new kernel contribution of the version submitted to JCST against the conference version should be explained explicitly in both the cover letter and the main document of the submission.

%All the authors should follow JCST's Guidelines for Authors\jz{1}{https://jcst.ict.ac.cn/EN/column/column107.shtml, May 2020.}, and especially, the authors must fulfill the Ethical Responsibilities of Authors and comply with the Referencing Guidelines.

%\section{Content}

%\subsection{Text}

%{\it Text Formatting}. Please refer to JCST Submit/Publish Template (LATEX, WORD) at:  http://jc\-st.ict.ac.cn/EN/column/column111.shtml.

%Manuscripts submitted for reviews should follow the JCST Submit Template, and those that have passed the review and are going to be accepted should use the JCST Publish Template.

%All elements of formulae should be type-written whenever possible. Use the automatic page numbering function to number the pages. Do not use field functions. Use the table function, not spreadsheets, to make tables. Save your file in TeX or LaTeX files, or docx format (Word 2007 or higher) or doc format (older Word versions). For Word files, please do use the MathType included in the Word template .rar file for equations.

%{\it Abbreviations.} Abbreviations should be defined at first mention and used consistently thereafter.

%{\it Footnotes}. Footnotes can be used to give additional information, which may include the citation of a reference included in the reference list. They should not consist solely of a reference citation, and they should never include the bibliographic details of a reference. They should also not contain any figures or tables. Footnotes to the text are numbered consecutively. Always use footnotes instead of endnotes.

%{\it Acknowledgments}. Upon acceptance of the paper, authors may add acknowledgement of people, grants, funds, etc., which should be placed in a separate section. The names of funding organizations should be written in full.

%{\it Biography and Photo}. Upon acceptance of the paper, authors will be asked to provide a short biography and a photo (with resolution = 600 dpi) of each author, to be included at the end of the manuscript.

%{\it Scientific Style}. Please always use internationally accepted signs and symbols for units (SI units).

%\subsection{References}
%\subsubsection{Citation}

%At times, it may be necessary for authors to include another author's material or to reuse portions of their own previously published work.

%When an author uses text, charts, photographs, or other graphics from another author's material, the author shall:

%1) clearly indicate reused material and provide a full reference to the origin (publication, person, etc.) of the material and

%2) obtain written permission from the publisher or, if the reused material has not been published, obtain written permission from the original source.

%When an author reuses text, charts, photographs, or other graphics from his/her own previously published material, the author shall:

%1) clearly indicate all reused material and provide a full reference to the original publication of the material and

%2) if the previously published or submitted material is used as a basis for a new submission, clearly indicate how the new submission differs from the previously published work(s).

%Reference citations in the text should be identified by numbers in square
%brackets. Some examples:

%1) Negotiation research spans many disciplines [3].

%2) This effect has been widely studied [1-3, 7].

%\subsubsection{Reference List}

%The list of references should only include articles that are cited in the text and that have been published or accepted for publication. Personal communications and unpublished work should only be mentioned in the text using footnotes to give more information. Do not use footnotes or endnotes as a substitute for a reference list.

%The references should be listed at the end of the manuscript and numbered in the order they are referred to in the text. For journals the following information should appear: names (including initials of the first names) of all authors, full title of the paper, and journal name, volume, pages and year of publication. For books the following should be listed: author(s), full title, edition, publisher, place of publication and year.

%\subsection{Tables}

%All tables are numbered using Arabic numerals in the order they are referred to in the text.

%Tables should be cited in text in consecutive numerical order. For each table, please supply a table caption (title) explaining the components of the table. Identify any previously published material by giving the original source in the form of a reference at the end of the table caption.

%Footnotes to tables should be indicated by ``Note:'' and included beneath the table body.

%\subsection{Definitions and Theorems}

%{\bf Definition 1} (Name of the Definition). {\it All definitions are numbered using Arabic numerals in the order they are presented in the text.}

%{\bf Theorem 1.}  {\it All theorems are numbered using Arabic numerals in the order they are presented in the text.}

%\begin{proof}
%Example for a proof.
%\end{proof}

%\subsection{Artwork and Illustrations Guidelines}
%\subsubsection{Electronic Figure Submission}

% $\bullet$ Supply all figures electronically.

% $\bullet$ For vector graphics, the preferred format is EPS; for halftones, please use
% TIFF format. MSOffice files are also acceptable.

% $\bullet$ Vector graphics containing fonts must have the fonts embedded in the files.

% $\bullet$ Name your figure files with ``Fig'' and the figure number, e.g., Fig1.eps.

% \subsubsection{Line Art}

% {\bf Definition 2} (Line Art). {\it Lines are black and white graphic with no shading.}

% Do not use faint lines and/or lettering and check that all lines and
% lettering within the figures are legible at final size. All lines should be
% at least 0.1 mm (0.3 pt) wide. Scanned line drawings and line drawings in
% bitmap format should have a minimum resolution of 1200 dpi. Vector graphics
% containing fonts must have the fonts embedded in the files.

% \subsubsection{Halftone Art}

% {\bf Definition 3} (Halftone Art). {\it Halftones include photographs, drawings, or paintings with fine shading, etc.}

% If any magnification is used in the photographs, indicate this by using
% scale bars within the figures themselves. Halftones should have a minimum
% resolution of 600 dpi.

% \subsubsection{Combination Art}

% {\bf Definition 4} (Combination Art). {\it Combination art is combination of halftone and line art, e.g., halftones containing line drawing, extensive lettering, color diagrams, etc. Combination artwork should have a minimum resolution of 600 dpi.}

% \subsubsection{Color Art}

% If black and white will be shown in the print version, make sure that the main information will still be visible. Many colors are not distinguishable from one another when converted to black and white. A simple way to check this is to make a xerographic copy to see if the necessary distinctions between the different colors are still apparent. If the figures will be printed in black and white, do not refer to color in the captions and text.

% Color illustrations should be submitted as RGB (8 bits per channel).

% \subsubsection{Figure Lettering}

% To add lettering, it is best to use Times New Roman. Please keep lettering consistently sized throughout your final-sized artwork, usually about 8 pt.

% Variance of type size within an illustration should be minimal, e.g., do not use 8-pt type on an axis and 20-pt type for the axis label.

% Avoid effects such as shading, outline letters, etc. Do not include titles or captions within your illustrations.

% \subsubsection{Figure Numbering}

% All figures are to be numbered using Arabic numerals in the order they are referred to in the text. Figures should always be cited in text in consecutive numerical order.

% Figure parts should be denoted by lowercase letters: (a), (b), (c), etc.

% If an appendix appears in your article and it contains one or more figures, number the appendix figures: A1, A2, A3, etc.

% \subsubsection{Figure Captions}

% Each figure should have a concise caption describing accurately what the figure depicts. Include the captions in the text file of the manuscript, not in the figure file.

% Figure captions begin with the term Fig. in bold type, followed by the figure number, also in bold type. No punctuation is to be included after the number, nor is any punctuation to be placed at the end of the caption.

% Identify all elements found in the figure in the figure caption; and use boxes, circles, etc., as coordinate points in graphs.

% Identify previously published material by giving the original source in the form of a reference citation at the end of the figure caption.

% \vspace{4mm}

% \begin{center}
% % Figure removed\\
% \vspace{3mm}
% \parbox[c]{8.3cm}{\footnotesize{Fig.1.~}  Example for inserting a one-column wide figure. }%\vspace*{.2mm}
% \end{center}

% \setcounter{figure}{1}
% % Figure environment removed
% \baselineskip=18pt plus.2pt minus.2pt
% \parskip=0pt plus.2pt minus0.2pt

% \vspace{2mm}

% \tabcolsep 12pt
% %\cmidrule(l){2-4}%
% \renewcommand\arraystretch{1.3}
% \begin{center}
% {\footnotesize{\bf Table 1.} \textcolor{red}{C}aption of \textcolor{red}{T}his \textcolor{red}{O}ne-\textcolor{red}{C}olumn \textcolor{red}{W}ide \textcolor{red}{T}able}\\
% \vspace{2mm}
% \footnotesize{
% \begin{tabular*}{\linewidth}{c}\hline\hline\hline
% \\\hline
% \\
% \\
% \\\hline\hline\hline
% \end{tabular*}%\vspace*{.2mm}
% \\\vspace{1mm}\parbox{8.3cm}{Note: You may explain the meaning of some special format, e.g., in bold, and/or give the full names of the abbreviations used in the table whose full names have not presented in the text.}
% }
% \end{center}

% \vspace{1mm}

% \setcounter{table}{1}
% \tabcolsep 9pt
% %\cmidrule(l){2-4}
% \renewcommand\arraystretch{1.3}
% \begin{table*}[!htb]
% \centering
% \caption{\label{3} \textcolor{red}{C}aption of \textcolor{red}{T}his \textcolor{red}{T}able}\vspace{-2mm}
% {\footnotesize
% \begin{tabular*}{\linewidth}{c}\hline\hline\hline
% \\\hline
% \\
% \\
% \\\hline\hline\hline
% \end{tabular*}%\vspace*{.2mm}
% %\\\vspace{1mm}\parbox{17.5cm}{}
% }
% \end{table*}
% \baselineskip=18pt plus.2pt minus.2pt
% \parskip=0pt plus.2pt minus0.2pt

% \subsubsection{Placement and Size}

% When preparing your figures, size figures to fit in the column width (one-column or two-column as needed).

% \subsubsection{Permissions}

% If you include figures that have already been published elsewhere, you must obtain permission from the copyright owner(s) for both the print and online format. Please be aware that some publishers do not grant electronic rights for free and that Springer will not be able to refund any costs that may have occurred to receive these permissions. In such cases, material from other sources should be used.

% \subsubsection{Accessibility}

% In order to give people of all abilities and disabilities access to the content of your figures, please make sure that:

% $\bullet$ All figures have descriptive captions;

% $\bullet$ Patterns are used instead of or in addition to colors for conveying
% information (colorblind users would then be able to distinguish the visual
% elements);

% $\bullet$ Any figure lettering has a contrast ratio of at least 4.5:1.

% \section{Electronic Supplementary Material}

% Springer accepts electronic multimedia files (animations, movies, audio, etc.) and other supplementary files to be published online along with an article or a book chapter. This feature can add dimension to the author's article, as certain information cannot be printed or is more convenient in electronic form.

% We encourage research data to be archived in data repositories wherever
% possible.

% \subsection{Submission}

% Please supply all supplementary material in standard file formats.

% To accommodate user downloads, please keep in mind that larger-sized files may require very long download times and that some users may experience other problems during downloading.

% {\it Audio, Video, and Animations.} Aspect ratio: 16:9 or 4:3; maximum file size: 25 GB; minimum video duration: 1 sec; supported file formats: avi, wmv, mp4, mov, m2p, mp2, mpg, mpeg, flv, mxf, mts, m4v, 3gp.

% {\it Text and Presentations.} Submit your material in PDF format; .doc or .ppt files are not suitable for long-term viability. A collection of figures may also be combined in a PDF file.

% {\it Spreadsheets}. Spreadsheets should be converted to PDF if no interaction with the data is intended. If the readers should be encouraged to make their own calculations, spreadsheets should be submitted as .xls files (MS Excel).

% {\it Specialized Formats}. Specialized format such as .pdb (chemical), .wrl (VRML), .nb (Mathematica notebook), and .tex can also be supplied.

% {\it Collecting Multiple Files}. It is possible to collect multiple files in a .rar or .gz file.

% {\it Numbering}. If supplying any supplementary material, the text must make specific mention of the material as a citation, similar to that of figures and tables. 1) Refer to the supplementary files as ``Online Resource'', e.g., "... as shown in the animation (Online Resource 3)", ``... additional data are given in Online Resource 4''. 2) Name the files consecutively, e.g. ``ESM{\_}3.mpg'', ``ESM{\_}4.pdf''.

% {\it Captions}. For each supplementary material, please supply a concise caption describing the content of the file.

% {\it Accessibility}. In order to give people of all abilities and disabilities access to the content of your supplementary files, please make sure that: 1) The manuscript contains a descriptive caption for each supplementary material. 2) Video files do not contain anything that flashes more than three times per second (so that users prone to seizures caused by such effects are not put at risk).

% \subsection{Highlight}

% Upon acceptance of the paper, authors will be asked to provide highlight of the paper. It is a short collection of information (e.g., text and graphics), in $4\sim 5$-pages PPT (with the first page presenting the title and the authors), to convey the research problem and the kernel findings, to provide readers with a quick overview of the article. The highlights describe the essence of the research (e.g., research problem, kernel contribution, results or conclusions) and highlight what is distinctive about it.

% Highlights may be displayed online in http://www.springer.com/journal/11390, but will not appear in the article PDF file or print.

% \section{After Acceptance}

% {\it Copyright Transfer}. Authors will be asked to transfer copyright of the article to the Publisher (or grant the Publisher exclusive publication and dissemination rights). This will ensure the widest possible protection and dissemination of information under copyright laws.

% {\it Proof Reading}. The purpose of the proof is to check for typesetting or conversion errors and the completeness and accuracy of the text, tables and figures. Substantial changes in content, e.g., new results, corrected values, title and authorship, are not allowed without the approval of the Editor.

% \section{Ethical Responsibilities of Authors}

% {\it Important Note}. The journal uses software to screen for plagiarism.

% The journal is committed to upholding the integrity of the scientific record. It follows the Committee on Publication Ethics (COPE) guidelines to deal with potential acts of misconduct.

% Authors should refrain from misrepresenting research results which could damage the trust in the journal, the professionalism of scientific authorship, and ultimately the entire scientific endeavour. Maintaining integrity of the research and its presentation can be achieved by following the rules of good scientific practice, which include:

% $\bullet$ The manuscript has not been submitted to more than one journal for simultaneous consideration.

% $\bullet$ The manuscript has not been published previously (partly or in full), unless the new work concerns an expansion of previous work (please provide transparency on the re-use of material to avoid the hint of text-recycling (``self-plagiarism'')).

% $\bullet$ A single study is not split up into several parts to increase the quantity of submissions and submitted to various journals or to one journal over time (e.g. ``salami-publishing'').

% $\bullet$ No data have been fabricated or manipulated (including images) to support your conclusions

% $\bullet$ No data, text, or theories by others are presented as if they were the author's own (``plagiarism''). Proper acknowledgements to other works must be given (this includes material that is closely copied (near verbatim), summarized and/or paraphrased), quotation marks are used for verbatim copying of material, and permissions are secured for material that is copyrighted.

% $\bullet$ Consent to submit has been received explicitly from all co-authors, as well as from the responsible authorities --- tacitly or explicitly --- at the institute/organization where the work has been carried out, before the work is submitted.

% $\bullet$ Authors whose names appear on the submission have contributed sufficiently to the scientific work and therefore share collective responsibility and accountability for the results.

% $\bullet$ Authors are strongly advised to ensure the correct author group, corresponding author, and order of authors at submission. Changes of authorship or in the order of authors are not accepted after acceptance of a manuscript.

% $\bullet$ Adding and/or deleting authors at revision stage may be justifiably warranted. A letter must accompany the revised manuscript to explain the role of the added and/or deleted author(s). Further documentation may be required to support your request.

% $\bullet$ Upon request authors should be prepared to send relevant documentation or data in order to verify the validity of the results. This could be in the form of raw data, samples, records, etc. Sensitive information in the form of confidential proprietary data is excluded.

% If there is a suspicion of misconduct, the journal will carry out an investigation following the COPE guidelines. If, after investigation, the allegation seems to raise valid concerns, the accused author will be contacted and given an opportunity to address the issue. If misconduct has been established beyond reasonable doubt, this may result in the Editor-in-Chief's implementation of the following measures, including, but not limited to:

% $\bullet$ If the article is still under consideration, it may be rejected and returned
% to the author.

% $\bullet$ If the article has already been published online, depending on the nature
% and severity of the infraction, either an erratum will be placed with the
% article or in severe cases complete retraction of the article will occur.
% The reason must be given in the published erratum or retraction note. Please
% note that retraction means that the paper is maintained on the platform,
% watermarked "retracted" and explanation for the retraction is provided in a
% note linked to the watermarked article.

% $\bullet$ The author's institution may be informed.

% \section{English Language Editing}

% For editors and reviewers to accurately assess the work presented in your manuscript you need to ensure the English language is of sufficient quality to be understood. If you need help with writing in English you should consider:

% $\bullet$ asking a colleague who is a native English speaker to review your manuscript for clarity;

% $\bullet$ visiting the English language tutorial which covers the common mistakes when writing in English;

% $\bullet$ using a professional language editing service where editors will improve the English to ensure that your meaning is clear and  identify problems that require your review.

% Please note that the use of a language editing service is not a requirement for publication in this journal and does not imply or guarantee that the article will be selected for peer review or accepted.

% If your manuscript is accepted it will be checked by our editors for spelling and formal style before publication.

% \section{[\textcolor{blue}{last section}] Conclusions}

% Although a conclusion may review the main points of the paper, do not replicate the abstract as the conclusion. A conclusion might elaborate on the importance and results of the work, and/or suggest applications and extensions.

% \vspace{2mm}

% [\textcolor{blue}{The references should be listed at the end of the manuscript and numbered in the order they are referred to in the text.}]



\begin{thebibliography}{99}
\footnotesize
\itemsep=-3pt plus.2pt minus.2pt
\baselineskip=13pt plus.2pt minus.2pt
\bibitem{1}Sayah J Y, Kime C R. Test scheduling in high performance VLSI system implementations. {\it IEEE Trans. Computers}, 1992, 41(1): 52-67. [\textcolor{blue}{example for journal paper}]

\bibitem{2} Gordon Plotkin. A semantics for type checking. In {\it Lecture Notes in Computer Science 526,} Ito T, Meyer A R (eds.), Springer-Verlag, 1991, pp.1-17. [\textcolor{blue}{example for book chapter}]

\bibitem{3} Geddes K O, Czapor S R, Labahn G. Algorithms for Computer Algebra. Boston: Kluwer, 1992. [\textcolor{blue}{example for book}]

\bibitem{4} Kwan A W, Bic L. Distributed memory computers. In {\it Proc. the 6th Int. Parallel Processing Symp.}, March 1992, pp.10-17. [\textcolor{blue}{example for conference}]

\bibitem{5} Harris M J. Real-time cloud simulation and rendering [Ph.D. Thesis]. Department of Computer Science, The University of North Carolina at Chapel Hill, 2003. [\textcolor{blue}{example for thesis}]

\bibitem{6} Jurczyk M, Coldwind G. Identifying and ex-ploiting windows kernel race conditions via mem-ory access patterns. Technical Report, Google Re-search, 2013. http://pdfs.semanticscholar.org/ca60/2e7193f159a56a3559-f08b677abfba60beb2.pdf, Mar. 2018. [\textcolor{blue}{example for technical report}]

\bibitem{7} Gipp B, Meuschke N, Gernandt A. Decentra-lized trusted timestamping using the crypto cur-rency Bitcoin. arXiv:1502.04015, 2015. https://arxiv.org/abs/1502.04015, May. 2018. [\textcolor{blue}{example for ar-Xiv document}]

\bibitem{8} Tong Y, Chen L, Zhou Z, JagadishH V, Shou L, Lv W. SLADE: A smart large-scale task decomposer in crowdsourcing. {\it IEEE Transactions on Knowledge and Data Engineering}. doi:10.1109/TKDE.2018.2797962. (preprint) [\textcolor{blue}{example for preprint}]

\end{thebibliography}

\label{last-page}
\end{multicols}
\label{last-page}
\end{document}

