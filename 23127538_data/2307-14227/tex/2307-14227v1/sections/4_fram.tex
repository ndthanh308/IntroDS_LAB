\section{Analytical Framework}
\label{sec:ana_fmwk}

In this section, we propose a framework for applying computational techniques in TCP based on the state-of-the-art and the expertise of domain experts. 
As shown in \autoref{fig:datapipeline}, the framework involves four typical stages, from the \textbf{digitalization} and \textbf{interpretation} of existing paintings to the \textbf{creation} of new artworks. 
These three stages will also serve the purpose of \textbf{exhibition}. 



\setcounter{figure}{3}
% Figure environment removed
\baselineskip=18pt plus.2pt minus.2pt
\parskip=0pt plus.2pt minus0.2pt






\subsection{Digitalization}
The \textbf{digitalization} stage involves transforming physical raw TCP into digital signals or codes, primarily as digital images~\cite{warwick2012digital}. 
These images could form a large corpus of TCP, comprising hundreds and thousands of artworks that can hardly be accessed physically in one place. 
Therefore, this stage presents technical requirements for \textbf{storage}, \textbf{retrieval}, and \textbf{restoration}. 

\textbf{Storage} requires storing a large amount of TCP digital images in the database. 
Many image-related techniques are developed to store and show paintings smoothly with different levels of detail~\cite{iiif}. 
However, there is little research targeting the storage of TCP digital images, which incorporates specific query and analytical requirements. 
For these large databases, information \textbf{retrieval} searches interested series of paintings efficiently from different dimensions. 
In addition to the metadata of TCP, such as authors and themes, content-based image retrieval utilizes similarities of paintings in terms of visual features~\cite{dong2020feature,hung2018study,wang2007image}. 
\textbf{Restoration} of TCP deals with the pigment fading and paper aging~\cite{guo2013image} in the digitalization stage.
Chen\etal\cite{chen2012simulating} simulated the aging and reverse-aging phenomena. 
Several studies apply image recovery techniques to restore the electronic forms of TCP in terms of stroke and brush~\cite{guo2013image} and colors~\cite{guo2013image,chen2012simulating,pei2006background,pei2004virtual,ding2012research}. 

\subsection{Interpretation} 
After obtaining the digital formats of original TCP, the next stage is to \textbf{interpret} them by applying computational approaches. 
The aspects of interpretation vary from micro-level analysis (\eg colors and objects~\cite{feng2022ipoet,chen2021poemgeneration}), meso-level analysis (\eg emotion extraction~\cite{feng2022ipoet}), to macro-level analysis (\eg the layout of the painting and white space analysis~\cite{fan2019evaluation}). 

In terms of techniques, the majority of the current work falls into two categories: local and global feature extraction. 
The former emphasizes the identification of relevant features, including handcrafted features (\eg brushwork~\cite{zhan2019,lai2016data} and color~\cite{guo2015novel,liu2014Classification}) and learned features with deep learning technologies (\eg object detection~\cite{meng2019elements,gu2019deep}). 
The latter focuses on the effectiveness of classification such as accuracy, in which most work utilizes learned features~\cite{liong2020automatic,meng2019elements}.
More technical details will be introduced in \autoref{sec:tech_tasks} and \autoref{sec:tech_feature}. 

\subsection{Creation} 
After obtaining useful features and insights from the analysis of existing paintings, another large proportion of studies focuses on the \textbf{creation} of new artworks. 
Image generation and video generation are two mainstream creation outcomes. 

\textbf{Image Generation.} 
The majority of studies focuses on generating new paintings, considering the unique characteristics in style and the artistic elements of TCP based on \textit{The Six Principles}. 
Style transfer takes existing inputs (\eg photos and prepared sketches) and output generated TCP artworks~\cite{li2021immersive,9413063,zhang2020detail}. 
Several studies also modify the content of the original input such as face replacement~\cite{li2021immersive}. 
In addition to generating from existing materials, a few works have experimented with creating artworks from scratch, including both content and style generation~\cite{xue2021end}. 

\textbf{Video Generation.} 
With the development of computer animation, a line of work has explored creating videos in a TCP style, aiming to express the \textit{Spirit Resonance} in \textit{The Six Principles} from a new perspective. 
They could be classified into two categories according to the input and animation entities. 
The first category is to input an existing video and apply video style transfer to transform the whole frame~\cite{lianginstance,zhang2009video}. 
The second category is to input a TCP and animate entities such as characters and animals on the painting~\cite{lianginstance,lai2016data,xu2006animating}.
In addition, 2.5D artworks~\cite{8419282,amati2010modeling} and video scribing showing the construction of TCP for educational purposes~\cite{8113507} are explored.

\subsection{Exhibition} 
As an essential type of artwork, the exhibition is a typical stage for promoting TCP to the general audience. 
In addition to the traditional approach of arranging items one by one in the museum, studies have been exploring interactive approaches to engage audiences in the exhibition. 
Existing work could be classified into three categories according to the interactive platforms, namely, touch-screen-based, XR-based, and others. 

\textbf{Touch screens} are commonly applied in today's museums. 
Hsieh\etal~\cite{hsieh2013viewing} presented an interactive tabletop for audiences to view detailed regions of TCP. 
Subramonyam\etal~\cite{subramonyam2015sigchi} developed an iPad application, ``Rice Paper," for artists to highlight and annotate key information of the TCP for the general public. 
They also printed a tangible booklet based on this application to guide audiences. 
CalliPaint~\cite{li2014writing} is a system that allows audiences or artists to conveniently create TCP-based digital artworks.

With the development of immersive devices, \textbf{XR (Mixed Reality)} has become a new creation platform for curators and artists. 
Several studies (\cite{jin2020reconstructing,zhao2020shadowplay2,yuan2016tunable}) reconstruct TCP in the VR (Virtual Reality) environment with 3D or 2.5D characters and objects. 
They are intended to provide an immersive experience that the static TCPs cannot fulfill.
Jin\etal~\cite{jin2022immersive}) evaluated the engagement of audiences when showing TCPs on the touch screen and the VR platform. 

In addition to the touch screen and XR, other \textbf{interactive installations} are also studied, such as using sensors to capture audiences' walking in the 3D space to generate Chinese Shanshui Paintings (\cite{le2019walking}) and applying real-time projector-camera system for audiences to interact with TCP (\cite{jin2007real}). 