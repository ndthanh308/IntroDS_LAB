\section{Methodology}
\label{sec:meth}

This section describes the search methodology, corpus construction, and collaboration with experts.

\subsection{Methods and Corpus}

We constructed a literature corpus based on keyword- and relation-search methods. We focused on TCP-related keywords (\eg ``Chinese landscape painting'', ``Chinese ink wash painting'', and ``Chinese brush painting''), resulting in an initial 88 papers.
To enlarge our literature corpus, we further used the relation-search method. We identified eight influential papers from the initial corpus and exhaustively traversed their references and citations, which expanded the corpus to 112 papers.
We thoroughly assessed the corpus based on relevance, focusing on publications that probe methodologies and applications. Papers on theory~\cite{wu2013modeling} and evaluation~\cite{bo2018computational} were excluded. 
Despite extensive research on brushes, we only included those relevant to TCP, discarding the calligraphy-related articles~\cite{mi2002droplet, bai2007efficient}. Finally, the corpus comprised 92 papers.



\subsection{Cooperate with Domain Expert}

Over the past year, we have been working closely with two experts to strengthen our understanding of the specialized and unique domain of TCP.
The experts include a professor with over 20 years of expertise in Chinese painting, and a doctoral candidate in Chinese painting theory with five years of experience.
Our collaboration consists of the following stages:
Firstly, we consulted the experts about the domain knowledge of TCP and the domain interpretation of some exemplar papers.
Secondly, we iteratively refined the analysis framework for the application of computer technology to TCP.
Finally, we explored research challenges and opportunities by discussing the findings and proposing future research projects.



\subsection{Coding and Classification}

Through iterative discussions with experts, we have analyzed the corpus from three perspectives: research scope in TCP~\autoref{sec:background}, specifically-targeted problem~\autoref{sec:ana_fmwk}, and the use of computer-based methods~\autoref{sec:tech}. 
To differentiate the TCP research from image data analysis properly, the characteristics of TCP should be taken into account. 
We noticed that the ``Six Principles of Painting'' has summarized the most important considerations for drawing and appreciating TCP. 
We used it as the coding scheme for categorizing the literature with different targeted problems, \ie the concerned artistic elements. 
In addition, we evaluated the current analysis of TCP, and concluded with a paradigm which outlines the components and purposes of TCP analysis that are supported by modern computer techniques. 
Specifically, the paradigm includes digitalization, interpretation, creation, and exhibition of TCP.
Lastly, we coded the papers in the corpus according to the types of computational techniques they utilized, rather than the concrete algorithms that are largely interchangeable. 


 

During the paper analysis, three authors independently coded 92 papers over a period of four weeks (\autoref{fig:code}). 
The classification criteria were refined during the coding process. 
In cases where there were disputes regarding categorization, all authors were involved in debates to reach a consensus.
For instance, we first restricted the classification of the papers using the ``Resonance of the Spirit'' in the Six Principles to the subjective evaluation of static Chinese painting images. 
After deliberation, we decided that this idea should be expanded to include animations, as this was deemed to be a more expressive art form which could be evaluated by the principle of ``Resonance of the Spirit''. More discussions are given in \autoref{sec:spirit}. 


\setcounter{figure}{1}
% Figure environment removed
% \baselineskip=18pt plus.2pt minus.2pt
% \parskip=0pt plus.2pt minus0.2pt
