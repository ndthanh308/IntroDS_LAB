\section{The Six Principles of Painting}
\label{sec:background}
The \textit{Six Principles of Painting} were proposed in the sixth century to serve as the grading standards of TCP~\cite{lin1967art}.
% Xie He used these principles to rate previous paintings in the book \textit{Gu Hua Pin Lu} (Notes on the Criticism of Old Paintings).
They have remained influential to this day and shaped the ways in which TCPs are drawn and appreciated.
We selected the translations collected in~\cite{van1962way} to clarify these principles in the following sections.

\subsection{Resonance of the Spirit, Movement of Life}
\label{sec:spirit}

\hspace{0.1pt}
\vspace{-25pt}
\begin{wrapfigure}[2]{l}{0.02\textwidth}
 \begin{center}
  \vspace{-25pt}
  % Figure removed
 \end{center}
\end{wrapfigure}
\noindent
The first principle emphasizes the delivery of vitality in the objects and emotions.
This principle is considered the most important, upon which the remaining principles were developed~\cite{lin1967art}.
While the resonated spirits and lively movements are difficult to evaluate objectively, we adopted their semantic meaning and defined them as techniques that aim to induce emotional arousal and engage audiences.

Papers within this category analyze the conveyed emotions and use this information to recreate captivating paintings on various devices.
For example, Zheng\etal~\cite{zheng2017chinese} utilized machine learning methods to extract relevant features and classified TCPs with the conveyed emotions.
To enhance the emotional expressions, other works~\cite{lianginstance,liu2020animating,zhang2009video} attempted to make the objects in the paintings `alive' (\eg animations), so that the paintings can be interpreted vividly.
In recent years, significant amount of efforts~\cite{jin2022immersive,jin2020reconstructing,zhao2020shadowplay2,ma2012annotating,jin2007real} have utilized mixed reality technology to display TCPs, giving viewers new perspectives on appreciating the paintings.

\subsection{Bone Manner, Structural Use of the Brush}
\hspace{0.1pt}
\vspace{-25pt}
\begin{wrapfigure}[2]{l}{0.02\textwidth}
 \begin{center}
  \vspace{-25pt}
    % Figure removed
 \end{center}
\end{wrapfigure}
\noindent
The bone method corresponds to the use of the brush.
Calligraphy and paintings highly influenced one other, with each brush stroke having its own structure, texture, and meaning.
We defined this principle as rendering techniques that involve brush strokes.
% Papers that fall into this category concentrate on rendering brush strokes.
There are 36 articles discussing the bone method from the following aspects:

\emph{Stroke extraction.} Some articles focus on identifying and replicating the distinctive brushstrokes of TCP~\cite{chan2002two,jiang2021mtffnet,li2004studying,sun2016monte,sun2015brushstroke,sheng2014,sheng2014recognition,sheng2013style,xu2006animating,yeh2002non,yu2003image}. 
Frequent subjects in TCP, including mountains~\cite{way2002synthesis}, rocks~\cite{way2001synthesis}, and trees~\cite{zhang1999simple}, were imitated.

\textbf{Ink simulation.} TCPs are drawn with brushes dipped in black ink or Chinese pigments, which diffuse on rice papers or silk. Therefore, the simulation of ink effect is heavily studied~\cite{xu2012stroke,way2006computer,10.1145/1073204.1073221,chu2004real,mi2004droplet,guo2003nijimi,way2003physical,yu2002model,lee2001diffusion,kunii1995diffusion,10.1145/15886.15911}. On the other hand, the physical model of the brush itself also deserves in-depth analysis~\cite{bai2009chinese,xu2005virtual,yeh2002effects,lee1999simulating}.

\textbf{Style recognition.} Xieyi and Gongbi are two representation styles of TCP. Xieyi uses techniques that privilege the spontaneity of the line (\hyperref[fig:gongbixieyi]{Fig.~3}A). 
% Gongbi uses highly detailed brushstrokes that precisely delimit details (\autoref{fig:g_x}B). 
Gongbi uses highly detailed brushstrokes that precisely delimit details (\hyperref[fig:gongbixieyi]{Fig.~3}B). 
These stroke characteristics were used to classify different painting styles~\cite{jiang2019dct,yang2019easy,jiang2004categorizing}.

\textbf{Painting process reproduction.} Understanding the painting process of TCP is particularly useful for painting practice and education~\cite{8113507,yang2013animating,xie2013artist,yao2005painting}. It also provides a practical basis for painting generation.

\subsection{Conform with the Objects, Obtain their Likeness}
% depicting the forms of things as they are [1]
% The drawing of forms which answer to natural forms[2]
\hspace{0.1pt}
\vspace{-25pt}
\begin{wrapfigure}[2]{l}{0.02\textwidth}
    \begin{center}
    \vspace{-25pt}
    % Figure removed
    \end{center}
\end{wrapfigure}
\noindent
Chinese painters have developed a distinct way of depicting objects in the world.
Compared with their Western counterparts, objects in Chinese paintings are more surreal.
This principle is reinterpreted as extracting objects from TCP and depicting and sketching natural objects in the TCP artistic style.
Six articles~\cite{dong2020feature,liong2020automatic,li2020multi,gu2019deep,lu2008content,zhang2004modelling} have studied the TCP classification according to the extracted objects, and two articles~\cite{chen2021poemgeneration,feng2022ipoet} produce text descriptions (\ie instance-level captions) for the objects in TCP. There are additional 16 articles that conduct the style transfer for natural objects with TCP art styles~\cite{xue2021end,li2021immersive,9413063,zhang2020detail,le2019walking,meng2019elements,he2018chipgan,wu2018research,shi2017generative,lai2016data,guo2015novel,li2014writing,dong2014real,liang2013image,amati2010modeling,wang2007image}.



\setcounter{figure}{2}
\begin{center}
% Figure removed\\
\vspace{3mm}
\parbox[c]{8.3cm}{\footnotesize{Fig.3.~}  Liang Kai's ``Immortal in Splashed Ink" (A)~\cite{xy2022painting} exemplifies the Xieyi style, using wet strokes of monochromatic ink to create the immortal's cloth (A1), while Ma Lin's ``King Yu of Xia" (B)~\cite{gb2022painting} exemplifies the more elaborate Gongbi style for clothing patterns (B1).}
\label{fig:gongbixieyi}
%\vspace*{.2mm}
\end{center}




\subsection{According to the Species, Apply the Colors}
% appropriate colouring [1]
% Appropriate distribution of the colours[2]
\hspace{0.1pt}
\vspace{-25pt}
\begin{wrapfigure}[2]{l}{0.02\textwidth}
 \begin{center}
  \vspace{-25pt}
  % Figure removed
 \end{center}
\end{wrapfigure}
\noindent
The suitability to type often appears to judge the correct use of colors.
In the drawing process, inks have to be applied in multiple layers to arrive at the desired tone.
Since inks and papers react to the environment and receive damages, ancient paintings have lost their original states and need to be conserved properly.
This principle corresponds to color analysis and restoration of paintings.
Papers that fall into this category concentrate on understanding the object classes~\cite{zhan2019,hung2018study,meng2018classification,liu2014Classification,bao2009effective,guan2005automatic}%(13,20,24、28,41、61、70) 
and performing further actions (\eg color enhancement~\cite{hu2015object,xu2007generic,jiang2006effective}). %(38,63,66)
Additionally, understanding the color of TCP is crucial for repairing them. Some papers employ computer technologies to address the issues of deteriorating paper and fading color~\cite{guo2013image,chen2012simulating,zhang2011multispectral,pei2006background,pei2004virtual,ding2012research}.

\subsection{Plan and Design, Place and Position}
\hspace{0.1pt}
\vspace{-25pt}
\begin{wrapfigure}[2]{l}{0.02\textwidth}
 \begin{center}
  \vspace{-25pt}
  % Figure removed
 \end{center}
\end{wrapfigure}
\noindent
Division and planning refer to the positioning and arrangement of objects.
A unique layout characteristic in Chinese painting is the concept of ``void'' (white space).
It is believed that leaving some part of the paper blank could induce more imagination in readers' minds~\cite{fan2019evaluation,fan2017visual}.
The seals, preface, and postscript are also significant components of the composition of TCP~\cite{bao2010novel,liang2010simple}.
Papers in this category concentrate on analyzing and enhancing the composition.

\subsection{To Transmit Models by Drawing}
\hspace{0.1pt}
\vspace{-25pt}
\begin{wrapfigure}[2]{l}{0.02\textwidth}
 \begin{center}
  \vspace{-25pt}
  % Figure removed
 \end{center}
\end{wrapfigure}
\noindent
Prior to the invention of printers, the manual replication of paintings was required to facilitate their distribution and use as commodities.
Copying the classics and antique masterpieces also help amateurs improve their skills by closely observing the techniques.
Papers in this area has concentrated on digitalizing Chinese paintings and showing them on different devices, such as high relief~\cite{8419282}, virtual reality reconstruction~\cite{yuan2016tunable}, and interactive devices~\cite{subramonyam2015sigchi,hsieh2013viewing}.
The primary distinction between this principle and others is that these methods focus on precise reproduction and minimal modification from the original copies.