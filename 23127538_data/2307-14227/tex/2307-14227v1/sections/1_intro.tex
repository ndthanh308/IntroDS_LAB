\section{Introduction}
\label{sec:Introduction}
Originating from the Han Dynasty, Traditional Chinese Painting (\textbf{TCP}) has been a primary art form in China~\cite{cheng2018essential}, characterized by artistic expressions depicted on paper and silk with brushes dipped in black ink and Chinese pigments.
TCP has been used to express the author's artistic creativity and to insinuate criticism of the society, philosophy, and politics of the time (\autoref{fig:tcp_w}A). 
Existing studies on TCP mainly focus on its history, explanation, and appreciation of paintings, as well as various styles and techniques. 
However, these are typically conducted thorough a close-reading and case study approach based on painting theories~\cite{bradley2018visualization}, which while insightful, is time-consuming and unscalable for studying the patterns and evolution trends of TCP that have emerged over the centuries.


% Figure environment removed
\baselineskip=18pt plus.2pt minus.2pt
\parskip=0pt plus.2pt minus0.2pt







Recent advances in computer technology have greatly improved the efficiency of TCP research. For example, Convolutional Neural Networks (CNNs) detect and segment the elements in TCP~\cite{8419282,zhang2011multispectral,chen2012simulating}, while Generative Adversarial Networks (GANs) generate TCPs and perform style transfer~\cite{xue2021end,zhang2020detail,he2018chipgan,li2021immersive,9413063}. However, applying these deep learning techniques on TCP requires a comprehensive understanding of their characteristics. For instance, although AI-generated TCPs may resemble the originals when viewed from afar, the brushstroke details can be far from natural.
Moreover, the visual objects are often mislocated with regard to the TCP composition style. Oil paintings have similar challenges and have been studied with regard to their specific characteristics, such as stroke composition~\cite{zheng2018strokenet,litwinowicz1997processing,huang2019learning,zou2021stylized,kotovenko2021rethinking,liu2021paint}.
Nevertheless, such analyses and conclusions cannot be directly applied to TCPs, since they differ greatly in terms of material, format, and techniques (see \autoref{tbl:tcp_oil}). 
They have different requirements for ink diffusion effects, point of view considerations, and painting techniques to depict textures.

\begin{table*}[tb]
    \renewcommand\arraystretch{1.5}
	\centering 
	\caption{Differences between traditional Chinese painting and oil painting.}  %
	\label{tbl:tcp_oil}
	\begin{tabular}{p{1.5cm}|p{7cm}|p{7.6cm}}
		\toprule
		\textbf{Aspects} & \textbf{Traditional Chinese painting} & \textbf{Oil painting} \\
		\midrule
        Material   & Painting black ink and Chinese pigments (similar to gouache paint) on paper and silk (\autoref{fig:tcp_w}A4). & Applying the mixture of pigments and drying oils with diverse plasticity on wood and canvas (\autoref{fig:tcp_w}B1).\\
        Format    & Contain several sections (\autoref{fig:tcp_w}A1); The point of views can come from different scenes (\autoref{fig:tcp_w}A2).  & Depict the scene from a focal perspective (\autoref{fig:tcp_w}B2).\\
        Technique & Objects are depicted by different brushwork systems, \eg wrinkling techniques for landscapes and calligraphic-line techniques for figures. (\autoref{fig:tcp_w}A3). & Focus on realism, using light and dark tones to express the texture of objects (\autoref{fig:tcp_w}B).\\
		\bottomrule
	\end{tabular}
\vspace{-0.2in}
\end{table*}

Various papers on computer vision techniques have been conducted mainly from the perspectives of aesthetic judgment and stylization.
DiVerdi~\cite{diverdi2015modular} investigated the modular framework for digital paintings and loosely addressed TCP by calligraphy.
Zhang\etal\cite{zhang2021comprehensive} surveyed systems of photographs and paintings from the perspective of aesthetic evaluation.
Kyprianidis\etal\cite{kyprianidis2012state} reviewed methods for transforming photos into aesthetically stylized renderings.
Li\etal\cite{li2022computing} conducted a review of the computer methods used during various stages of production and preservation of Chinese cultural heritage.
TCP, however, is only viewed as a common format of image data in these papers, which overlook the peculiarities of Chinese paintings' data attributes and analysis tasks.

To bridge the gap between general image data and TCP, we conducted a systematic review of the literature on key areas such as data visualization (VIS), Computer Vision (CV), Computer Graphics (CG), and Human-Computer Interaction (HCI).
We drew inspirations from the TCP appreciation theory and adapted the ``Six Principles of Painting" to modern concepts for categorizing the collected literature (\autoref{sec:background}). 

In collaboration with Chinese painting specialists, we proposed an four-stage framework to review the purposes of using computational techniques in TCP (\autoref{sec:ana_fmwk}). 
We then classified the recent computer-aided techniques from the perspectives of task, feature, and rendering (\autoref{sec:tech}). 
Lastly, we reported the discussions with specialists about the current drawbacks and potential future applications of computer technology to TCP (\autoref{sec:challenge}). We believe this paper can offer an explanation, insights, and examples into every aspect of TCP, thus making way for more comprehensive appreciation. An interactive browser of this paper is available at https://ca4tcp.com.

