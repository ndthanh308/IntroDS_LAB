\subsection{Feature Extraction}\label{sec:tech_feature}
There are abundant features in TCP which distinguish them from many other painting genres.
To help users analyse and learn from TCP, many researches have extracted features for downstream tasks, such as painting classification and creation.
We summarized the features of TCP into two categories, \ie handcrafted features and learned features, according to the methodology of feature extraction.



\textbf{Handcrafted features.}
The handcrafted features are extracted by rule-based methods and reflect the specific aspects of TCP.

% Brushwork
\emph{Brushwork} is an important feature in depicting the bone method of the paintings. % \cite{zhanying2019}
Typically, the TCP are created with brushes dipped in ink, and the ink permeates through the rice paper, creating the unique shape of the brush strokes.
To automatically generate the TCP, a wide range of studies \cite{zhan2019,lai2016data,dong2014real,xu2012stroke,amati2010modeling,bai2009chinese,xu2007generic,yao2005painting} focusing on simulating the diffusion effect of color ink.
Wang\etal~\cite{wang2007image} proposed a physically-based model with texture synthesis method to simulate the color ink diffusion.
Chu\etal~\cite{10.1145/1073204.1073221} introduced a fluid flow model to calculate the percolation in the paper.
In addition, the brushwork is related to the visual complexity of the paintings. Dense thin strokes can increase the complexity while sparse thick strokes lower the complexity.
Fan\etal~\cite{fan2017visual} measured stroke thickness based on the calculation of color change.
Combining the analysis of stroke structures with the ink dispersion densities and placement densities, \cite{lai2016data} generated animations for water flow in the TCP according to the stroke pattern groups of the flow field.

% Color
\emph{Color} is another significant factor and implies the types of the TCP style \cite{guo2015novel,liu2014Classification,chen2012simulating,feng2022ipoet,lu2008content,wang2007image}.
Liu\etal~\cite{liu2014Classification} extracted the color information of the paintings by calculating the mean and variance values of the image pixels, and used them to support painting classification tasks.
Color can also be used in painting retrieval \cite{hung2018study}, painting style modeling \cite{feng2022ipoet}, and painting enhancement \cite{chen2012simulating}.
Over time, ancient Chinese paintings have faded and aged, requiring human restoration. 
Pei\etal~\cite{pei2004virtual,pei2006background} design color enhancement schemes to improve the image contrast, making the paintings more vivid and bright.

% Object
\emph{Objects}, such as the scenery in the paintings, are the basic elements of the painting composition and contain semantic information. Ding\etal~\cite{zhang2004modelling} extract objects by labeling pixels according to their connectivity in a pre-processed image, and use them for image retrieval. 
Feng\etal~\cite{feng2022ipoet} extract the objects in the TCP and use them to describe the painting content and create the painting poetry.
Zhao\etal~\cite{zhao2020shadowplay2} built a TCP style image repository for basic objects, supporting users to create immersive videos for poetry appreciation.

% Script
\emph{Scripts} are written in the empty space of the paintings and serve as complementary expression of creators' artistic ideas.
Bao\etal~\cite{bao2010novel} automatically identify and extract the scripts from the paintings according to their colors and regions.
Several studies also focus on other feature of the TCP, such as the white space~\cite{fan2017visual}, composition~\cite{sun2016monte}, and seal images~\cite{bao2009effective}.

\textbf{Learned features.}
With the fast development of deep learning technology, many studies have introduced deep learning models to learn the features of TCP.
Based on the labeled data of TCP, supervised learning methods (\eg CNN~\cite{krizhevsky2017imagenet}, VGG-16~\cite{simonyan2014very}, and YOLOv3~\cite{redmon2018YOLOv3}) are applied in object detection \cite{meng2019elements,gu2019deep,feng2022ipoet}, image classification \cite{liong2020automatic,meng2019elements,meng2018classification}.
As the stylistic features of the TCP are unique from other paintings, it is valuable to learn the stylistic features to transfer neural images into TCP.
A range of studies \cite{xue2021end,lianginstance,9413063,he2018chipgan} focus on capturing the stylistic features of TCP with adversarial training, a classical learning strategy in unsupervised learning.
In contrast, Li\etal~\cite{li2020multi} introduce weakly-supervised learning for semantic classification in the scenario with limited number of training images.