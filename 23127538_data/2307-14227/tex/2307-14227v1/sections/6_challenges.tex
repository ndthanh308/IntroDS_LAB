\section{Challenges and Oppurtunities}
\label{sec:challenge}

\subsection{Lack of large-scale and high-quality datasets.}
The lack of large and high-quality open-access datasets is a crucial reason hindering the further development of Traditional Chinese Painting research.
According to our paper, some works have announced that they have produced a few Chinese painting datasets~\cite{hung2018study, xue2021end, 9413063, liong2020automatic, dong2020feature}. For example, Liong et al.\cite{liong2020automatic} constructed an unlabeled dataset containing more than 1,000 Chinese paintings, and Dong et al.\cite{dong2020feature} collected a labeled dataset. However, these datasets are limited in size and have not been made open-source.
Building a large and high-quality Chinese painting dataset faces several challenges, including:

\begin{itemize}
\setlength{\itemsep}{0pt}
    \item \textit{Data availability}. As most Chinese paintings are held in museums and private collections all over the world, there is a problem of copyright ownership. It is particularly difficult to collect online resources of Chinese paintings.
    \item \textit{Data quality}. Many famous Chinese paintings are large in size, rich in details, and difficult to preserve, leading to the high cost of digitizing Chinese painting and a high technical barrier for generating high-definition pictures.
    \item \textit{Data diversity}. Chinese paintings contain relatively independent items, such as colophons and seals that can be used for analyzing historical events, collection paths, etc. However, only a few articles discussed the extraction of colophons and seals~\cite{bao2010novel,bao2009effective} without further exploration.
    \item \textit{Data annotation}. Due to the domain professionalism, annotating Chinese painting data requires high-level expertise, especially for systematic annotations based on the \textit{Six Principles of Painting}, which can be extremely expensive.
\end{itemize}

\subsection{Insufficient consideration on the TCP uniqueness.}
The analysis of TCP research tendencies based on the \textit{Six Principles of Painting} in \autoref{sec:background} reveals that most articles focus on \textit{Bone Manner, Structural Use of the Brush} (36/92), and \textit{Conform with the Objects, Obtain their Likeness} (24/92). This is mainly because the recognition, segmentation, and classification of strokes and objects in paintings can be formulated into CV tasks and solved with mature CV models.
However, researchers have paid little attention to the unique stroke system in traditional Chinese painting (3/92)~\cite{way2002synthesis,way2001synthesis,zhang1999simple}. For example, there are eighteen unique drawing methods in TCP techniques for depicting portraits (\autoref{fig:tcp_w}), and different wrinkling techniques for depicting mountains and rocks (\autoref{tbl:tcp_oil}). A detailed analysis of the stroke system sheds lights on the painter's style and the mentoring relationships between painters. Therefore, it is an issue that deserves attention in future computer fields.

Moreover, the \textit{Place and Position} aspect is not given as much attention in the current work (4/92). According to experts, the composition of Chinese paintings is crucial. Painters often use white space to convey mood, inscriptions, and seals to balance the picture's composition. Therefore, a computer-based systematic examination of the composition of Chinese paintings can aid specialists in understanding the compositional traits of paintings across time.

Regarding the study of \textit{Movement of Life} (9/92), it is important to note that in recent years, the fusion of AR and VR technology into TCP has risen, allowing audiences to experience Chinese painting from a new perspective. The development of new technology has increased the opportunities for the study and presentation of TCP, but more work of this type is required, and it may be reinforced in the future.

\subsection{Disregard for the data-linking in TCP analysis.}
Cultural heritage has various types of data, including paintings, ancient books, sculptures, architecture, and more. 
As a form of cultural heritage, TCP has garnered attention in recent years. 
However, most current research has focused solely on TCP data, and rarely combines other datasets for cross-analysis. 
From a historical research perspective, TCP collections represent only a snapshot of a certain time period. It is possible that snapshots of different artifacts describe the same social landscape. To gain a more comprehensive historical understanding, different collections of cultural relics should be viewed together. 
Therefore, it is worthwhile to pay attention to how to integrate different data related to paintings in order to restore a more accurate historical picture.

On the other hand, the use of multi-modal data to construct deep learning models is a growing trend. Multi-modal data enables better feature representation construction in the latent space, which improves the fusion of textual, visual, and other forms of information like videos and knowledge graphs. This ultimately strengthens the model's performance and enhances its generalization capacity.

\subsection{Insufficient exploration of ML methods and large models on TCP.}
TCP image data has unique characteristics compared with natural images, such as cross-domain, few annotated training samples, imbalanced classes, and variable sizes of objects. Advanced ML methods have taken profound discussions on related topics such as transfer learning, domain adaptation, domain generalization, few-shot learning, and learning with long-tailed data distribution. Therefore, applying these advanced methods to TCP can promote Chinese painting analysis from a computational view. Meanwhile, these methods can effectively reduce the demand for data annotations and alleviate the burden of collecting large-scale and high-quality annotated datasets.

In addition, large language models (\eg GPT-4~\cite{openai2023gpt4}) and large vision models (\eg CLIP~\cite{radford2021Learninga}, Stable-Diffusion~\cite{rombach2022HighResolution}) are becoming the new foundations of advanced research. Current models are not specifically adapted to TCP data, tending to generate images that ignore the Six Principles of Paintings, as well as textual descriptions that often lack detail and do not capture the essence of the painting. It is inevitable that unimodal or multimodal large models will be adopted in traditional Chinese painting research. 



\subsection{Inadequate applications for artwork creation and promotion.} 
Although a line of work has explored the generation of TCP-styled paintings and videos, the quality of these AI-generated artworks could be doubtful. 
Involving artists in the creation process with a semi-automated creation style would be a promising direction in the future. 
In addition, advanced display and interaction techniques (\eg immersive techniques) should be applied to promote TCP to the general public. New storytelling approaches should also be constructed to enhance the appreciation and understanding of TCP. 