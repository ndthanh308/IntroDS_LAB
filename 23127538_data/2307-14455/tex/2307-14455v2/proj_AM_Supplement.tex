\documentclass[prc,aps,nofootinbib,onecolumn]{revtex4-2}   
\usepackage{epsfig}  
\usepackage{graphicx}  
\usepackage{amssymb}
\usepackage{amsmath}
\usepackage{color}  
\usepackage{ulem}
\newcommand{\gs}[1]{\textcolor{blue}{#1}}


\begin{document}  


\title{Spin orientation of Fission Products Intrinsic Spins and their Correlations}

\author{G. Scamps$^{1}$, I. Abdurrahman$^{2}$, M. Kafker$^{1}$, A. Bulgac$^{1}$, and I. Stetcu$^{2}$}
\address{$^{1}$Department of Physics, University of Washington, Seattle, Washington 98195-1560, USA}
\address{$^{2}$Theoretical Division, Los Alamos National Laboratory, Los Alamos, New Mexico 87545, USA}

\begin{abstract} 
Here we provide some pertinent details to the main text.
\end{abstract}
%
\maketitle

\section{Derivation of the Eq. (5) in the main text} 

 The distribution of fission fragments (FFs) intrinsic spins $S_H,S_L,$ and of the orbital angular momentum ${\bf \Lambda} = {\bf R}\times {\bf P}$,
 where ${\bf R}$ is the FF relative separation and ${\bf P}$ their relative linear momentum, is obtained using the triple projection~\cite{Bulgac:2022b}, 
%
\begin{align}
P(\Lambda,\Lambda_z,S_H,S_L)&= 
\sum_{K_H K_L} \langle \Psi | \hat P^{\Lambda }_{\Lambda_z\Lambda_z} \hat P^{H S_H}_{K_HK_H} \hat P^{L S_L}_{K_LK_L} |\Psi\rangle,\label{eq:Triple_proj}
\end{align}
%
where the symmetry projectors are defined as~\cite{Ring:2004,Varshalovich:1988,Bally:2021}
\begin{align}
& \hat P^{F J_F}_{K_F K_F'}= \frac{(2J+1)}{16\pi^2} \iiint \hat R_F (\Omega_F) {\cal D}^{J_F*}_{K_FK_F'} (\Omega_F) d\Omega_F, \label{eq:PJMK} \\
&   \hat R_H (\Omega_H)  \hat R_L (\Omega_L)=\hat R_L (\Omega_L)  \hat R_H (\Omega_H),\\
& \hat P^{\Lambda}_{\Lambda_z \Lambda_z}= \frac{(2\Lambda+1)}{8\pi^2} 
\iiint \hat R (\Omega) {\cal D}^{\Lambda*}_{\Lambda_z,\Lambda_z} (\Omega) d\Omega, \label{eq:PL}\\
&\hat R (\Omega) = e^{i\alpha \hat J_z} e^{i\beta \hat J_y} e^{i\gamma \hat J_z},
\end{align}
and $\hat R (\Omega)$  rotates the fission direction and  $\hat R_F (\Omega_F)$ rotate the FFs respectively. 
The rotation of each FF should be performed with respect to their center of mass and in their moving framework.
Since the operator 
\begin{align}
\hat R(\Omega)\hat R_H(\Omega) \hat R_L(\Omega)
\end{align} 
rotates the entire system in Eq.~\eqref{eq:Triple_proj}  $\hat R(\Omega)$ can be replaced by $\hat R_H(-\Omega) \hat R_L(-\Omega)$, 
which has the effect of rotating simultaneously both FFs. Note that $\Lambda_z = 0$ in the intrinsic system of the fissioning nucleus. 
Introducing for each FF $F = H, L$, 
\begin{align} 
& \hat R_F (\Omega_F')    =  \hat R_F (-\Omega)  \hat R_F (\Omega_F) \\
& {\cal D}^{J_F*}_{M_F,K_F} (\Omega_F) = \sum_{M_F'} {\cal D}^{J_F'}_{M_F,M_F'} (- \Omega)  {\cal D}^{J_F'*}_{K_F,M_F'} (\Omega_F') 
\end{align}
 we obtain,
\begin{align}
P(\Lambda,S_H,S_L) &= \sum_{M_H,M_L} \frac{ (2\Lambda+1)  (S_H+1/2) (S_L+1/2)}{(8 \pi^2 )^3} 
\iiint d\Omega \iiint d\Omega_H'  \iiint d\Omega_L'   \langle \Psi |  \hat R_{H} (\Omega_H')   \hat R_{L} (\Omega_L')   | \Psi \rangle \nonumber \\
& \times {\cal D}^{\Lambda*}_{0,0} (\Omega)   {\cal D}^{S_H}_{0,-M_H} (\Omega) {\cal D}^{S_H*}_{0,M_H} (\Omega_H')  
{\cal D}^{S_L}_{0,-M_L} (\Omega)  {\cal D}^{S_L*}_{0,M_L} (\Omega_L') (-1)^{M_H+M_L} 
\end{align}
and we used the property Eq.~(5), page 97 in Ref.~\cite{Varshalovich:1988}),
\begin{align}
\iiint  d\Omega  {\cal D}^{J*}_{M,K} (\Omega)    {\cal D}^{J_1}_{M_1,K_1} (\Omega)  {\cal D}^{J_2}_{M_2,K_2} (\Omega) = \frac{8 \pi^2}{2J+1}
C^{JM}_{J_1M_1J_2M_2} C^{JM}_{J_1K_1J_2K_2}
\end{align}
where $C_{J_1,M_1,J_2,M_2}^{JM}$ is a Clebsch-Gordan coefficient,
\begin{align}
 {\cal D}^{J}_{M,K} (- \Omega) = (-1)^{K-M} {\cal D}^{J}_{-M,-K} (\Omega).
\end{align}
The expression \eqref{eq:Triple_proj} can be simplified exactly to the following projection formula,
\begin{align}
P(\Lambda,S_H,S_L) &= \sum_{ K K' }  C_{S_H,-K,S_L,K}^{\Lambda,0}  C_{S_H,-K',S_L,K'}^{\Lambda,0}  
\langle \Psi | \hat P^{H S_H}_{K K'} \hat P^{L S_L}_{-K -K'} | \Psi \rangle. \label{eq:Triple_proj2}
\end{align}
The use of Eq.~\eqref{eq:Triple_proj2} requires a projection on six Euler angles, which as far as we are aware of has never been attempted in practice. 
In the case when $\Lambda_z = K_H+K_L=0$ the number of needed Euler angles can be reduced to four, a number which is quite large and rarely used in practice.


% Figure environment removed

\section{Quasiparticle, canonical wave functions,  and Distributions $P(\Lambda,S_H,S_L)$}

We performed simulations in a box $N_x\times N_y\times N_z = 30\times 30\times 60$ with a lattice constant 1 fm, see Ref.~ \cite{Shi:2020}, with which we evolved 
$16\times30^2\times60= 864,000$ coupled nonlinear complex partial differential equations in 3D+time.  
The total number of quasiparticle wave functions being too large to perform 
the evaluations of the overlaps in Eqs.~(\ref{eq:PJMK}, \ref{eq:Triple_proj2}), we introduced canonical wave functions as described in Ref.~\cite{Bulgac:2022c}, which reduced 
significantly the dimension of the overlap matrix to be evaluated with the Pfaffian method described in Ref.~\cite{Bertsch:2012}.  
While the use of quasiparticle wave functions is  typical in treatment of nuclear systems with pairing, since the generalized density matrix commutes with the 
generalized quasiparticle Hamiltonian and one can define then in a natural manner the quasiparticle energies~\cite{Ring:2004}, 
the canonical wave functions are not eigenfunctions of the generalized pairing Hamiltonian and are very rarely used in practice. 
Since the canonical occupation probabilities are not conserved during time evolution, the set of canonical wave functions has to be introduced 
only in the final state, see Ref.~\cite{Bulgac:2022c}.

In Refs.~\cite{Bulgac:2022b,Bulgac:2022e}, we were forced to use a very large set of quasiparticle 
wave functions in order to evaluate the FF intrinsic spins, since the technology to determine the 
canonical wave functions and many of their properties described in Ref.~\cite{Bulgac:2022c} were not known. 
That prevented us from performing full symmetry projection and 
we have adopted at the time a reasonable approximation described here as well. 
We evaluated the triple distribution 
\begin{align}
\tilde{P}_0(\Lambda,S^L,S^H) &= \int_{-\pi/2}^{\pi/2}\frac{d\beta_0}{\pi}
\int_{-\pi/2}^{\pi/2}\frac{d\beta^L}{\pi}
\int_{-\pi/2}^{\pi/2}\frac{d\beta^H}{\pi}
e^{i\beta_0\Lambda + i\beta^LS^L +i\beta^HS^H} \nonumber \\
&\times \langle \Psi | 
e^{-i\beta_0(\hat{J}^L_x+J^H_x) -i\beta^L\hat{J}^L_x-i\beta^H\hat{J}^H_x}
|\Psi\rangle ,    
\end{align}
which technically depends on two angles only as an approximation to the projection 
\begin{align}
P(\Lambda,S_L,S_H)= \langle \Psi | \hat P^{\Lambda }_{0,0} \hat P^{H S_H}_{0,0} \hat P^{L S_L}_{0,0} |\Psi\rangle,
\end{align}
which unfortunately, as explained in Refs.~\cite{Bulgac:2022b,Bulgac:2022e} 
does not enforce the triangle constraint.
\begin{align}
\Delta = \Theta (\Lambda \ge |S_l-S_h|)\Theta (\Lambda \le S_L+S_H).
\end{align} 
For that reason we have used the approximate projector
\begin{align}
\tilde{P}(\Lambda, S_L,S_H) = {\cal N}
\tilde{P}_0(\Lambda,S_L,S_H)\Theta (\Lambda \ge |S_L-S_H|)\Theta (\Lambda \le S_L+S_H)    \label{eq:2}
\end{align}
with the appropriate normalization ${\cal N}$. The differences between 
the exact triple distribution with Eq.~\eqref{eq:Triple_proj2} and
the approximate triple distribution 
with Eq.~\eqref{eq:2} are illustrated in the 3D profiles in Fig.~\ref{fig:spat}.%~\ref{fig:1}.

At the level of one spin and two spins distributions these triple 
distributions lead to very similar results however. 
From the triple distributions $P(\Lambda,S_H,S_L)$ or $\tilde{P}(\Lambda,S_H,S_L)$ 
one can determine the single and double FF intrinsic spin distributions
\begin{align}
&P_1(S_L)= \sum_{S_H,\Lambda} P(\Lambda,S_H,S_L), \quad P_1(S_H)= \sum_{S_L,\Lambda} P(\Lambda,S_H,S_L),\\
& P_2(S_H,S_L) = \sum_\Lambda   P(\Lambda,S_H,S_L).
\end{align}
Our results show that 
\begin{align} 
P_1(S_F)\approx (2S_F+1)\exp \left [ -\frac{S_F^2}{2\sigma_F^2}\right ] ,\label{eq:P1}\\
\end{align}
a result derived in a Fermi gas model, apart from odd-even effects, see Refs.~ \cite{Bethe:1936,Ericson:1960}, 
and that separately for half-integer and integer spins
\begin{align}
P_2(S_H,S_L) \approx P_1(S_H)P_1(S_L),\label{eq:P2}
\end{align}
also observed experimentally by \textcite{Wilson:2021} and theoretically in
Refs.~\cite{Vogt:2021,Randrup:2021,Bulgac:2022e}. 


In the present study, we were able to perform angular momentum projection with four Euler angles
(instead of formally six Euler angles) after introducing 
the set of canonical wave functions and taking into account the symmetry of the FFs.
The use of the approximate two-angle projector Eq.~\eqref{eq:2} is not necessary anymore. 
As a result, we were able to evaluate the intrinsic spins and their correlations for both 
even and odd mass FFs, as demonstrated in Fig. 1 in the main text and also evaluated 
the distribution of the projection of the 
FF intrinsic spins on the fission direction, see Figs. 1 and 2 in the main text. 

Due to these two technical developments, we were able to  prove 
that in fission dynamics all possible FF spin modes (conjectured to be allowed since
1965~\cite{Nix:1965,Moretto:1980,Moretto:1989},
but never incontrovertibly shown to exist) are excited at scission and beyond, 
where the nuclei are already hot. The possible FF spin modes are two 
bending and two wriggling modes 
and one twisting and another tilting~\cite{Nix:1965}.  
The tilting mode is excited only if the total spin of the compound nucleus ${\bf S_0} > 0$.
Surprisingly, the twisting mode is present 
with a probability larger than 0.5 for both  induced 
fission of $^{236}$U and spontaneous fission of $^{252}$Cf. 
This aspect is in stark contrast with the 
conjecture in Ref.~\cite{Nix:1965},  adopted subsequently in numerous data analyses, and the 
predictions of the phenomenological model introduced almost 15 years ago
FREYA~\cite{Randrup:2009,Vogt:2021,Randrup:2021,Randrup:2022}, 
where the twisting mode is deemed irrelevant and thus 
suppressed.  One should add that the generation of each single point 
in Figs. 1-2 in the main text require a separate angular 
momentum projection for each set of values $ S_F, K_F$ 
for the single intrinsic spin distribution  $\hat{P}^{F S_F}_{K_FK_F}$, see 
in Eq.~\eqref{eq:PJMK}, and similarly for each set of values $S_H, S_L, K, K', -K, -K'$ 
for the triple angular momentum distribution  
$P(\Lambda,S_H,S_L)$, see Eq.~\eqref{eq:Triple_proj2}, 
therefore leading to many hundreds of angular momentum projections performed in this study.





\bibliography{local_fission1.bib}
\end{document}