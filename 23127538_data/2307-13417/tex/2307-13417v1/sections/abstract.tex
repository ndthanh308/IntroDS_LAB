\begin{abstract}
% Do Not Use Symbols, Special Characters, Footnotes, or Math in Paper Title or Abstract.
    Ambiguity is ubiquitous in natural language.
    Resolving ambiguous meanings is especially important in information retrieval tasks.
    While word embeddings carry semantic information, they fail to handle ambiguity well.
    Transformer models have been shown to handle word ambiguity for complex queries, but they cannot be used to
    identify ambiguous words, e.g. for a 1-word query.
    Furthermore, training these models is costly in terms of time, hardware resources, and training data,
    prohibiting their use in specialized environments with sensitive data.
    Word embeddings can be trained using moderate hardware resources.
    This paper shows that applying DBSCAN clustering to the latent space can identify ambiguous words and
    evaluate their level of ambiguity.
    An automatic DBSCAN parameter selection leads to high-quality clusters, which are semantically coherent and
    correspond well to the perceived meanings of a given word.
    \keywords{natural language processing \and word-sense-disambiguation \and word embedding \and clustering \and DBSCAN \and silhouette score}
\end{abstract}

%Dieser Konferenzbeitrag präsentiert eine neue Methode zur automatischen Identifizierung von Kontexten,
%in denen ein Wort oder Begriff eine eindeutige Bedeutung annimmt.
%Durch Kombination des Word2Vec Modells mit dem DBSCAN Clustering-Verfahren sowie der automatisierten Auswahl der
%Clustering-Parameter können so die verschiedenen Bedeutungen eines einzelnen Wortes oder Begriffs in einem gegebenen
%Datensatz identifiziert werden.
%Diese Methode ist durch den Verzicht auf Large Language Models (LLMs) auf regulären Workstations in annehmbarer Zeit
%ausführbar.

