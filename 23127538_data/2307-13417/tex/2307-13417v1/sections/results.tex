\section{Results and Discussion}\label{sec:results and Discussion}
In this section we qualitatively analyse the clustering results as well as the clustering score.
We show a small excerpt of the clustering results for the selected words.

\subsection{Parameter Score Results}\label{subsec:wikipedia-ambiguity-page-as-reference}
Table~\ref{tab:param-score-results} summarizes the metrics for the best parameter scores for every selected word.
The number of clusters consistently exceed the manually counted meanings in wikipedia pages with homonymous contents,
for every word.

\begin{table}[h]
    \centering
    \caption{The metrics for the top parameter score result for every selected word.}
    \label{tab:param-score-results}
    \begin{tabular}{l|r|r|l|l|r|r|l|l}
        \toprule
        \thead[c]{Word}  & Domains & Meanings & $ \thead[c]{p(c)}$ & $\thead[c]{\epsilon}$ & \thead[c]{$m$} & \thead[c]{$n(c)$} & \thead[c]{$\overline{s}(c)$} & \thead[c]{$noise(c)$} \\
        \midrule
        Mars             & 9       & 86       & 0.165              & 0.21                  & 5              & 104               & -0.40                        & 0.211                 \\
        Maus (en. mouse) & 14      & 51       & 0.209              & 0.2                   & 5              & 78                & -0.35                        & 0.233                 \\
        \makecell[lt]{Kleeblatt \\(en. clover leaf)}            & 6       & 14        & 0.271              & 0.23                  & 5              & 19                & -0.13                        & 0.269                 \\
        Pepsi & 2 & 4 & 0.218 & 0.32 & 5 & 11 & -0.11 & 0.078 \\
        \makecell[lt]{Gabelbein \\(en. wishbone)} & 2 & 2 & 0.596 & 0.2 & 5 & 3 & 0.212 & 0.196 \\
        \makecell[lt]{Datenstruktur \\(en. data structure)} & 1 & 1 & 0.544 & 0.25 & 5 & 8 & 0.302 & 0.084 \\
        \bottomrule
    \end{tabular}
\end{table}


The parameter score results visible in figure~\ref{fig:parameter-score} clearly propose the parameters $m = 5$ with
$\epsilon = {0.19, 0.21}$ for the word `mars', corresponding to a number of clusters of 110 and 104, respectively.
Clusterings for the word `mars' with $0.43 \geq \epsilon \geq 0.99$ all hold a parameter score of 0,
as the number of clusters for all clusterings equal one, resulting in the default silhouette score of -1,
thus in a parameter score of 0.

% Figure environment removed

\citeauthor{schubertDBSCANRevisitedRevisited2017} (\cite{schubertDBSCANRevisitedRevisited2017}) note that a sudden
increase or decrease in the number of clusters may indicate to have reached the resolution limits of the dataset.
We noticed that our parameter score is highest for the clustering with the last, smallest epsilon value right before
a sudden decrease in the number of clusters.
We interpret this sudden drop, that is also visible in the exemplary heatmap~\ref{fig:cluster_count},
as the resolution limit of the context vectors, and conclude that the proposed parameter score complies to
this heuristic.

\subsection{Context Evaluation}\label{subsec:context-evaluation}
Table~\ref{tab:words} presents the unmodified In2In and In2Out context tokens for the first four clusters of all
selected words by example.
We added the presumed meaning to every cluster.
When looking at the 10 words closest to the mean context vector, it is quite simple to manually assign a single topic
word that describes the words within a single context.
Even if not, we can easily find the meaning of a cluster with a single query in a search engine, simply combining the
target word and a context word.
For example, cluster '2' from the In2In similar words contain the words 'amiga', 'lp', and 'langspielplatte', with 7
dates, ranging from 1981 to 1990, in between.
The single query 'kleeblatt amiga' showed that amiga is a label that had active releases in the given years.

\begin{table}
    \centering
    \caption{The top 3 words from the the first four clusters of the selected words.}
    \label{tab:words}
    \scriptsize
    \begin{tabular}{|c|ll|ll|ll|}
        % Mars
        \toprule
        \emph{Mars} & \multicolumn{2}{c|}{Cluster 1} & \multicolumn{2}{|c|}{Cluster 2} & \multicolumn{2}{|c|}{Cluster 3} \\ \cmidrule{1-7}
        Word & In2In       & In2Out        & In2In        & In2Out      & In2In      & In2Out      \\
        \midrule
        1 & mars & apollo & ausrüsten & \makecell[lt]{maschinen- \\ gewehre}    & commune      & commune         \\
        2 & planet & venus & feldhaubitz & batterie & anctoville & déléguée \\
        3 & jupiter & planet & haubitzen & waffensystem & \makecell[lt]{saint-caprais- \\ de-blaye} & \makecell[lt]{gemeinde- \\ verband}    \\
        \midrule
        Meaning & \multicolumn{2}{c|}{Planets} & \multicolumn{2}{|c|}{Tactical System} & \multicolumn{2}{|c|}{\makecell[ct]{Admin. units \\ `mars' in france}} \\
        \bottomrule

        % Maus
        \toprule
        \emph{Maus} & \multicolumn{2}{c|}{Cluster 1} & \multicolumn{2}{|c|}{Cluster 2} & \multicolumn{2}{|c|}{Cluster 3} \\
        \midrule
        1 & burg & burg & maus       & maus & orthologe & uniprot \\
        2    & \makecell[lt]{burg\_      \\neuhaus} & burghügel & normal & hund & \makecell[lt]{eutel- \\ eostomi}    & orthologe    \\
        3 & \makecell[lt]{burg- \\ anlage} & deidesheim & mauspad & \makecell[lt]{beispiels- \\ weise}    & entrez       & lebewesen         \\
        \midrule
        Meaning & \multicolumn{2}{c|}{Castles} & \multicolumn{2}{|c|}{\makecell[ct]{Computer Periphery \\ and Animals}} & \multicolumn{2}{|c|}{Genetics} \\
        \bottomrule

        % Kleeblatt
        \toprule
        \emph{Kleeblatt} & \multicolumn{2}{c|}{Cluster 1} & \multicolumn{2}{|c|}{Cluster 2} & \multicolumn{2}{|c|}{Cluster 3} \\
        \midrule
        1 & balken & kreuz & fcn.de    & kik & amiga & cd \\
        2 & schild & rot & \makecell[lt]{fc\_st. \\ \_pauli}    & pizarro & lp    & box    \\
        3 & hufeisen & wappe & bvb & 2008 & 1982 & amiga \\
        \midrule
        Meaning & \multicolumn{2}{c|}{Heraldry} & \multicolumn{2}{|c|}{Soccer} & \multicolumn{2}{|c|}{\makecell[ct]{`Kleeblatt' LP \\ Amiga record label}} \\
        \bottomrule

        % Pepsi
        \toprule
        \emph{Pepsi} & \multicolumn{2}{c|}{Cluster 1} & \multicolumn{2}{|c|}{Cluster 2} & \multicolumn{2}{|c|}{Cluster 3} \\
        \midrule
        1    & pepsi    & pepsi & blockflötist      & komponist    & 2001–2003      & 2001–2002      \\
        2 & show & jell-o & \makecell[lt]{volksmusik- \\ forscher} & 1959 & 1998–2001 & 1997–2001 \\
        3 & dallas & live & \makecell[lt]{kultur- \\ manager} & \makecell[lt]{kultur- \\manager} & 2000–2003 & 2002–2005 \\
        \midrule
        Meaning & \multicolumn{2}{c|}{\makecell[ct]{Superbowl \\ Commercial}} & \multicolumn{2}{|c|}{\makecell[ct]{Composer \\ David Lucas}} & \multicolumn{2}{|c|}{\makecell[ct]{Dates of \\ Cola-Wars}} \\
        \bottomrule

        % Gabelbein
        \toprule
        \emph{Gabelbein} & \multicolumn{2}{c|}{Cluster 1} & \multicolumn{2}{|c|}{Cluster 2} & \multicolumn{2}{|c|}{Cluster 3} \\
        \midrule
        1 & \makecell[lt]{becken- \\ knoche} & unterkiefer & radnabe & hydraulisch & auslegen & \makecell[lt]{schraub- \\verbindung} \\
        2 & gelenkfläch & oberkiefer & zahnkranz & verstellbar & reduzieren & steifigkeit \\
        3 & osteoderme & schädel & tauchrohr & antriebswelle & \makecell[lt]{konstruktions- \\ bedingen} & hydraulisch \\
        \midrule
        Meaning & \multicolumn{2}{c|}{Bones} & \multicolumn{2}{|c|}{\makecell[ct]{Telescopic \\ Fork}} & \multicolumn{2}{|c|}{\makecell[ct]{Pros and Cons of \\ Telescopic Forks}} \\
        \bottomrule

        % Datenstruktur
        \toprule
        \makecell[lt]{\emph{Daten-} \\ \emph{struktur}} & \multicolumn{2}{c|}{Cluster 1} & \multicolumn{2}{|c|}{Cluster 2} & \multicolumn{2}{|c|}{Cluster 3} \\
        \midrule
        1 & \makecell[lt]{sinnvoller- \\ weise} & \makecell[lt]{implemen- \\ tierung} & verlag & vieweg & iabot & uni-kassel.de \\
        2 & sinnvoll & datenstruktur & lineare & lehrbuch & toter & online \\
        3 & definiert & prozessor & teubner & verlag & \makecell[lt]{internet\_ \\ archive)} & 2008 \\
        \midrule
        Meaning & \multicolumn{2}{c|}{\makecell[ct]{Technical \\ terms}} & \multicolumn{2}{|c|}{Literature} & \multicolumn{2}{|c|}{\makecell[ct]{Outdated \\ References}} \\
        \bottomrule
    \end{tabular}
\end{table}

The first three words of all in2in and in2out clusters can not be clearly divided into
similar (In2IN) and thematic (In2Out) words.
Yet, the meaning, i.e.\ context that every clusters represents are simple to derive.
Only few examples are difficult to decipher, like the second cluster of `pepsi',
where the In2In words would suggest the broader topic `musician', while
the In2Out words clearly suggest a composer, which was simple to find in the 2-word query `komponist pepsi'.
Some tokens are incomplete, such as `Burg Maus' (en.\ castle `Maus') or `beckenknoche' (ref.\ to ger.\ `Beckenknochen'),
while others are special characters, as the pipe `|' in the fourth cluster of `mars'.
