% \section{Introduction}

% Lens flare \cite{holladay1926fundamentals,seibert1985removal,goodman2005introduction,kingslake1992optics} is a common optical artifact that occurs when non-image-forming light enters a camera's lens system and interacts with the imaging sensor.  This phenomenon can degrade image quality and adversely affect the visual appeal of photographs, especially in mobile computational imaging. Lens flare is more prevalent in this field due to factors such as the widespread use of plastic lenses in mobile camera systems, resulting in lower lens quality compared to professional cameras, and the lack of costly anti-reflective (AR) coatings \cite{blahnik2016reduction}.


% Previous approaches to flare removal have relied on traditional image processing techniques \cite{chabert2015automated, asha2019auto,vitoria2019automatic,hullin2011physically}. Recently, given flare datasets are proposed \cite{dai2022flare7k,dai2023nighttime,dai2023flare7k++,wu2021train} learning-based approaches are proposed to address this problem. In previous works \cite{dai2022flare7k,zhou2023improving} researchers have identify tone-mapping as a non-linear and non-invertable processing step in the ISP that significantly affects the restoration performance and try to reverse the tone-mapping operation for better restoration. 

% However, there are more than one non-invertable transformation in the ISP pipeline, which all leads to lossing originical image information. For example, denoising algorithms are widely used in the ISP, and denoising problem is known as a trade-off problem between removing image details and removing the noise.
% With smaller sensor size on the mobile phone, the mobile camera system tends to use higher ISO setting to increase the sensitivity of the sensor but also amplifies the noise. This is especially true for nigh-time images. As a results, it is necessary to perform noise reduction before outputting the image. In addition, due to user preference, mobile phone manufactures also incorporate sharpen operation to amplify details of the images to provide a more present visual output.
% At the end of the ISP, compression will be used to convert the images into more usable format, such as JPEG, which means the data from 12-bit raw format is saved into a 8-bit lossy format. 
% Based on the above facts, natural questions arise: 

% \vspace{1mm}
% \textit{Is there any other non-invertable steps in the ISP, such as denoising, sharpening or the compression, obstruct the restoration of the flare removal task? Is it the same for different types of flares? }
% \vspace{1mm}

% Without understanding the pros and cons of these operations, we will not be able to construct the ISP in a the correct order of operations to resolve the flare removal problem. 
% However, given there is not raw image dataset available for this problem, the previous appoaches are limited to mimic one operation instead of performing a full investigation for ISP operations.
% In this paper, we address these challenges by proposing a new dataset and methodology for lens flare removal. we utilize the unique property of raw images, and decompose these non-linear non-invertable post-processing into multiple steps for investigation. 
% Our dataset encompass more than $2,000$ real-world high-quality, full-resolution raw image pairs for scattering flare, and 1,100 for reflective flare.
% The image data is captured with mobile phone cameras and saving them into both raw image format along with the image processed by the internal ISP, providing rich and diverse data for training and evaluation. To our best knowledge, we are the first dataset for providing raw image data for the flare removal problem for both scattering flare and reflective. Through our experiments, we notice that the existing approaches are mostly trained with local corrupted flare images. In our dataset we inlcudes both local and global flare corrupted image and the experiments shows that the generalization of exisiting appoaches can be further improved with our data. 


% In summary, the paper's contributions are manifold:
% \begin{itemize}[leftmargin=0cm, itemindent=0.5cm]
%     \item Introduction of the first comprehensive raw image dataset for flare removal, consisting of over 3,000 real-world examples of scattering and reflective flares captured with mobile phone cameras. This dataset not only addresses the limitations of existing datasets but also provides a richer basis for training and evaluation.
%     \item Provision of both local and global flare corruption data, addressing the limitation of current datasets that focus primarily on localized flare effects. Our experiments demonstrate that this broader dataset improves the generalizability of flare removal algorithms.
%     \item Detailed exploration of non-invertible ISP operations such as denoising, compression, and sharpening, assessing their impact on flare removal. Our findings indicate that while denoising is crucial for both types of flares, compression and sharpening affect them differently, leading to varying degrees of degradation.
%     \item A comprehensive qualitative and quantitative analysis of the flare removal challenge within the ISP, offering new insights and perspectives on integrating flare removal steps effectively into the ISP pipeline.
% \end{itemize}

% \begin{itemize}[leftmargin=0cm, itemindent=0.5cm]
%     \item We contribute the first high-quality raw image dataset for flare removal problem, addressing the limitations of existing datasets and methods. Our dataset comprises more than $3000$ real-world examples of both scattering and reflective flares captured with mobile phone cameras, saved in raw image format and the ones processed by the internal processing pipeline of the mobile phone.
%     \item As most of existing dataset only include local degradation of the flare corruption, the networks trained with these models may not generalize well to the large-scale global artifacts. In this work we provide both local and global flare corruption data. Through experiments we shows that we improve the generalization of existing approaches with new data. 
%     \item We investigate the non-invertable ISP operations through detailed experiments, such as denoising, compression, and sharpening, may either improve or obstruct flare removal. We show that denoising in both types of flares are important for addressing the flare issue. While the compression and sharpening operation leads to different levels of degradation for different types of flares. 
%     \item We systematically analyze the flare removal problem in ISP in both qualitatively and qualitatively, providing a new understanding and perspective for the flare removal problem in incorporating flare removal steps into the ISP.
% \end{itemize}
 % ------ revise ------
 
% \section{Introduction}

% Lens flare significantly undermines image quality in mobile computational imaging, where it is particularly pronounced due to the ubiquity of lower-quality lenses and the absence of sophisticated anti-reflective coatings. Efforts to mitigate this issue have historically leveraged a range of image processing techniques, both traditional and learning-based. Recent research has pinpointed the complicating role of non-invertible tone-mapping operations within the image signal processing (ISP) pipeline on the restoration efficacy.

% However, the challenge extends beyond tone-mapping. The ISP pipeline incorporates several non-invertible transformations — namely, denoising, sharpening, and compression — each contributing to the loss of original image detail. The reliance on higher ISO settings in mobile cameras to offset limited sensor sizes amplifies noise, necessitating aggressive noise reduction techniques. Simultaneously, manufacturers employ sharpening algorithms to enhance detail, and images are often compressed into lossy formats for storage, all of which can distort the raw image data. This complex web of modifications raises pivotal questions: Do other non-invertible steps in the ISP, such as denoising, sharpening, or compression, hinder the restoration efforts in flare removal? And, is their impact consistent across different types of flares?

% Addressing these questions is hampered by a lack of suitable raw image datasets, limiting previous research to simulating isolated ISP operations rather than a holistic analysis. In response, this paper presents a novel dataset and methodology tailored for comprehensively tackling lens flare removal. By harnessing the intrinsic qualities of raw images, we dissect the non-linear, non-invertible post-processing steps of the ISP pipeline. Our dataset, unprecedented in scope, includes over 2,000 real-world high-quality raw image pairs for scattering flare and 1,100 for reflective flare, captured via mobile phone cameras. This compilation not only facilitates a deeper exploration into the flare removal challenge but also marks the first endeavor to offer a raw image dataset specifically designed for this purpose.

% In summary, the paper's contributions are manifold:
% \begin{itemize}
%     \item We introduce the first extensive raw image dataset for flare removal, containing more than 3,000 examples of both scattering and reflective flares, significantly expanding the boundaries of existing datasets by including images afflicted by both local and global flare artifacts.
%     \item Through rigorous experimentation, we evaluate the effects of non-invertible ISP operations such as denoising, compression, and sharpening on flare removal, uncovering that denoising is crucial across flare types, whereas compression and sharpening affect flare removal differently depending on the flare's nature.
%     \item Our analysis not only enriches the understanding of the flare removal problem within the ISP framework but also suggests strategic integration of flare mitigation steps into the ISP to enhance image restoration.
% \end{itemize}

% This investigation highlights the critical role of identifying and addressing various non-invertible ISP steps that may obstruct flare removal, underlining the importance of tailoring restoration strategies to the specific type of flare encountered.

\section{Introduction}

Lens flare \cite{holladay1926fundamentals,seibert1985removal,goodman2005introduction,kingslake1992optics} is a common optical artifact degrading image quality and visual appeal. It occurs when stray light enters the camera lens and interacts with the imaging sensor. This phenomenon is particularly prevalent in mobile computational imaging due to several factors.  Firstly, mobile cameras often utilize plastic lenses, which generally exhibit lower quality compared to professional-grade glass lenses. Secondly, the cost constraints of mobile devices often preclude the inclusion of expensive anti-reflective (AR) coatings \cite{blahnik2016reduction}, further exacerbating the issue of lens flare.

Early approaches to flare removal relied on image processing techniques \cite{chabert2015automated, asha2019auto,vitoria2019automatic,hullin2011physically}. However, recent advancements in deep learning and the availability of flare datasets \cite{dai2022flare7k,dai2023nighttime,dai2023flare7k++,wu2021train} have spurred the development of learning-based methods for tackling this problem. Notably, researchers have identified tone-mapping as a critical factor affecting restoration performance \cite{dai2022flare7k,zhou2023improving}. Tone-mapping is a non-linear and non-invertible process within the image signal processing (ISP) pipeline that can hinder accurate flare removal.

However, the ISP pipeline encompasses multiple non-invertible transformations beyond tone-mapping, each contributing to the loss of original image information, as exemplified in Fig.~\ref{fig: raw vs isp}. For instance, denoising algorithms, while essential for mitigating noise amplified by high ISO settings and small sensor sizes (especially in nighttime photography), inevitably remove some image details as a trade-off. Additionally, user preferences often drive manufacturers to incorporate sharpening operations that enhance image details, potentially introducing artifacts. Finally, image compression, typically into lossy formats like JPEG, discards information to achieve manageable file sizes.
These observations raise crucial questions regarding the impact of various ISP steps on flare removal:


% Figure environment removed




\vspace{1mm}
\textit{Do other non-invertible ISP operations, such as denoising, sharpening, and compression, hinder or benefit the effectiveness of flare removal? Does their influence vary for different types of flares (e.g., scattering vs. reflective)?}
\vspace{1mm}

Understanding the interplay between these operations is vital for optimizing the ISP pipeline and effectively tackling lens flare. Unfortunately, the lack of readily available raw image datasets has limited previous research to mimicking individual operations rather than conducting comprehensive investigations.

This paper addresses these challenges by introducing a novel dataset and methodology specifically designed for lens flare removal. Our approach leverages the unique properties of raw images to decompose the non-linear, non-invertible processing steps within the ISP for detailed analysis. The dataset comprises over $2,000$ high-quality, full-resolution raw image pairs exhibiting scattering flare and $1,200$ pairs with reflective flare, all captured using mobile phone cameras. Each pair includes both the raw image and its processed counterpart from the internal ISP, providing rich and diverse data for training and evaluation. To the best of our knowledge, this is the first dataset offering raw image data for both scattering and reflective flare removal. 
Our experiments reveal that existing approaches, primarily trained on images with localized flare artifacts, may not generalize well to large-scale, global flare effects. By including both local and global flare corruptions, our dataset enhances the generalizability of these methods.
This paper makes the following key contributions:




\begin{itemize}[leftmargin=0.3cm, itemindent=0.0cm]
    \item Introduce a unique raw image dataset for lens flare removal, featuring over 3,000 real-world examples that address the limitations of existing datasets by including diverse lighting and environmental conditions.
    \item Enhances the generalizability of lens flare removal techniques by incorporating a comprehensive range of both local and global flare effects.
    \item The impact of non-invertible ISP operations (denoising, compression, sharpening) on flare removal is thoroughly investigated, revealing their varying effects.
    \item A comprehensive analysis of the interplay between flare removal and the ISP pipeline is provided, offering valuable insights for effective integration. 
\end{itemize}

