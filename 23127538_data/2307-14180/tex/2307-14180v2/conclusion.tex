% \section{Conclusion}
% In conclusion, this paper introduces a novel raw image dataset specifically tailored for mobile camera systems, focusing on both scattering and reflective flare removal. This dataset, encompassing a broad range of real-world scenarios captured using various mobile devices and camera settings, lays a solid foundation for developing advanced flare removal algorithms by exploiting the unique properties of raw images. We anticipate that this dataset will catalyze further research in flare removal and contribute to significant enhancements in mobile image quality, benefiting mobile photographers and end-users alike. Experimental results underscore the limitations of networks trained with synthetic data, as they grapple with complex lighting conditions present in our real image dataset. Moreover, we showcase the considerable benefits of utilizing raw image data over processing data through a mobile phone's internal ISP, which adversely affects image quality.

% In conclusion, our work introduces a novel raw image dataset specifically tailored for mobile camera systems, focusing on both scattering and reflective flare removal. This dataset, encompassing a broad range of real-world scenarios captured using various mobile devices and camera settings, lays a solid foundation for developing advanced flare removal algorithms by exploiting the unique properties of raw images. Experimental results underscore the limitations of networks trained with synthetic data, as they grapple with complex lighting conditions present in our real image dataset. Moreover, we showcase the considerable benefits of utilizing raw image data over processing data through a mobile phone's internal ISP, which adversely affects image quality.
% We anticipate that our dataset will catalyze further research in flare removal and contribute to significant enhancements in mobile image quality, benefiting mobile photographers and end-users alike.
% In conclusion, our work presents a novel raw image dataset specifically designed for mobile camera systems, focusing on both scattering and reflective flare removal. This dataset, which encompasses a wide variety of real-world scenarios captured with diverse mobile devices and camera settings, provides a solid foundation for developing advanced flare removal algorithms by leveraging the distinct properties of raw images. Experimental results highlight the limitations of networks trained with synthetic data, as they struggle to cope with complex lighting settings present in our real image dataset. Furthermore, we demonstrate the significant advantages of using raw image data over processing data through a mobile phone's internal ISP, which compromises image quality.
% We expect that our dataset will spur further research in flare removal and contribute to substantial improvements in mobile image quality, benefiting mobile photographers and end-users alike.
% Our dataset, comprising over 2,000 high-quality full-resolution raw image pairs for scattering flare and 1,100 for reflective flare, ensures adaptability across various imaging conditions and can be further segmented into a larger number of paired patches. By addressing the limitations of existing datasets and methods, our work aims to significantly advance the field of lens flare removal in mobile computational imaging, ultimately leading to improved image quality and enhanced visual experiences for mobile phone camera users. 
% \section{Conclusion}

% In this work, we have addressed the pervasive issue of lens flare in mobile computational photography, a challenge that degrades image quality and diminishes the user experience. By introducing a novel dataset comprising over 3,000 raw and processed image pairs capturing both scattering and reflective flares, we have established a robust foundation for advancing flare removal research. This dataset, distinguished by its focus on high-quality, full-resolution images from mobile phone cameras, not only fills a gap in the current research landscape but also enhances the applicability and generalizability of flare removal techniques.

% Our comprehensive evaluation of flare removal models across different processing conditions—utilizing both internally processed images (ISPRGB) and those subjected to an external processing pipeline (RAW2RGB)—has shed light on the critical impact of various ISP operations on flare removal efficacy. By dissecting the ISP pipeline into its constituent non-invertible steps, including denoising, sharpening, and compression, our study reveals the nuanced ways in which these operations influence the restoration process, varying significantly between scattering and reflective types of flare.

% Key findings from our experiments underscore the importance of the initial denoising step in enhancing image quality and the detrimental effects of subsequent sharpening and compression operations, which can introduce artifacts and reduce detail, complicating flare removal efforts. Moreover, our analysis indicates that the placement of the flare removal step within the ISP pipeline is crucial, with the best outcomes achieved when it follows denoising but precedes other processing steps.

% The contributions of this paper are significant, offering valuable insights into the complex interplay between flare removal and the ISP pipeline. By providing a comprehensive raw image dataset specifically designed for lens flare research, we enable more nuanced investigations into the effects of ISP operations on flare removal. Furthermore, our findings not only enhance the understanding of flare dynamics but also inform the development of more effective image restoration techniques.

% In conclusion, our work not only bridges a critical gap in the available resources for lens flare research but also paves the way for future advancements in image restoration technologies. As mobile computational photography continues to evolve, the insights and resources provided in this paper will undoubtedly contribute to the development of more sophisticated and effective flare removal algorithms, ultimately leading to improved image quality and enriched visual experiences for users worldwide.
\section{Conclusion}

This paper tackled the significant challenge of lens flare in mobile computational photography by introducing a novel raw image dataset tailored for the analysis and removal of both scattering and reflective flares. 
 By utilizing RAW data, we can investigate those non-invertible ISP operations, and provide new and critical insights for addressing the flare problem on real data, which is not feasible with previous methods.
% Our dataset, comprising over 3,000 raw and processed image pairs captured with mobile phone cameras, marks a significant advancement in the field, offering unprecedented detail and variety to support the development of more effective flare removal techniques. 
% The contributions of this work extend beyond the introduction of a comprehensive dataset.
Through rigorous experimentation, we've illuminated the complex effects of various non-invertible Image Signal Processing (ISP) operations—namely denoising, sharpening, and compression—on the performance of flare removal algorithms. We've provided a nuanced understanding of how different ISP operations impact flare removal, particularly emphasizing the importance of denoising as a foundational step for enhancing image quality. This study's insights into the interplay between ISP steps and flare removal offer valuable guidance for optimizing image processing pipelines, ultimately facilitating better image restoration techniques and improving the visual quality of photographs captured with mobile devices.
Our findings reveal the delicate balance between enhancing image quality and preserving essential details necessary for effective flare mitigation, highlighting the critical role of processing sequence in achieving optimal restoration results.

% The contributions of this work extend beyond the introduction of a comprehensive dataset; we've provided a nuanced understanding of how different ISP operations impact flare removal, particularly emphasizing the importance of denoising as a foundational step for enhancing image quality. This study's insights into the interplay between ISP steps and flare removal offer valuable guidance for optimizing image processing pipelines, ultimately facilitating better image restoration techniques and improving the visual quality of photographs captured with mobile devices. As the field of mobile computational photography continues to evolve, the resources and findings presented in this paper will undoubtedly serve as a cornerstone for future research, driving innovations in image processing and restoration that enhance user experiences worldwide.
