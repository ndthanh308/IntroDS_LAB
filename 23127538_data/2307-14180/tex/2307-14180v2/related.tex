

% Figure environment removed




% Figure environment removed


% \section{Related Works}
% In this section we first introduce the current flare image dataset. Followed by the discussion about how a raw image datasets can be used for enhancing user experimence in mobile compulational photography. 

% \subsection{Flare Removal Problem and Dataset}

% There are various causes of lens flare, including light scattering within the lens system, reflections between lens elements, and the influence of dust, contaminants, or scratches on lens surfaces. Lens flare can be broadly classified into two types: scattering flares and reflective flares, each exhibiting distinct characteristics and shapes.

% Scattering flares \cite{gu2009removing, talvala2007veiling, mccann2007camera, raskar2008glare} arise from the interaction of light with microscopic imperfections and defects within the lens system. As an example shown in Fig.~\ref{fig: reflect and scatter examples}, these imperfections cause light to scatter in various directions, resulting in a visible haze such as veiling glare \cite{talvala2007veiling} or a series of artifacts in the captured image. The shape and appearance of scattering flares depend on the nature and distribution of the defects within the lens. Dust particles on the lens surface can cause small, localized bright spots or streaks, while scratches can produce more elongated, linear artifacts. The presence of multiple defects may lead to a complex pattern of overlapping flares, further degrading image quality.

% Reflective flares \cite{hullin2011physically, lee2013practical, chabert2015automated,vitoria2019automatic}, in contrast, are caused by reflections between lens elements, particularly in multi-element lens systems, as illustrated in Fig.~\ref{fig: reflect and scatter examples}. When light enters the lens, it can reflect off the internal surfaces of the lens elements, bouncing between them before eventually reaching the image sensor. These internal reflections can create a series of concentric rings, polygons, or other geometric shapes in the image, depending on the lens design and the relative position of the light source. Reflective flares are often more pronounced when the light source is close to the optical axis or when the lens system comprises a large number of elements.
% Moreover, reflective flares can exhibit different shapes and appearances depending on whether they are in-focus or out-of-focus. In-focus reflective flares tend to form sharp, well-defined geometric patterns, such as white spots \cite{asha2019auto, chabert2015automated,vitoria2019automatic}. Out-of-focus reflective flares can appear more diffuse and irregular, often taking the form of circular or elliptical blobs, known as bokeh. Factors influencing out-of-focus flares include lens design, aperture shape, and the degree of defocus.

% For the flare removal problem, there are several datasets available. The initial flare image dataset, proposed by Wu \etal   \ \cite{wu2021train}, consists of 2,000 captured flare-only images and 3,000 flare images simulated using their physics-based model. These flares are superimposed on flare-free base images to create synthesized flare corruption. However, their real image data exhibits similar lens settings, resulting in comparable flare-only images across different scenes and unrealistic simulation outcomes.
% Qiao \etal \   \cite{qiao2021light} collect unpaired flare-corrupted and flare-free images for training Cycle-GAN-like networks \cite{zhu2017unpaired}, but the lack of paired data precludes its use for training pixel-to-pixel neural networks such as U-Net \cite{ronneberger2015u}.
% Flare7K \cite{dai2022flare7k} synthesizes 5,000 scattering flare-only images in various colors and 2,000 reflective flare-only images. 
% The dataset also includes 100 real images with a resolution of $512\times512$ pixels for evaluation. Similar to \cite{wu2021train}, flare-only images are added to flare-free base images for simulation. \cite{dai2023flare7k++} extends this work and construct a real flare image dataset by capturing the flare in the darkroom. 

% As a more prevalent flare corruption in daily life, reflective flare is insufficiently addressed in these synthetic datasets. Given the nature that the reflective flare is always exist in the optical system, it is difficult to provide groundtruth data for this type of flare. Flare7K \cite{dai2022flare7k}  and Wu \etal \cite{wu2021train} offer simulated reflective flares at specific angles without considering image content, such as light source shape. Bracket Flare \cite{dai2023nighttime} proposed a night-time reflective flare image dataset, by composing an image with normal exposure and lower-exposure using a symmetrical prior, focusing on eliminating the in-the-focus reflective flare area. 

% In our constructed dataset, we present both reflective flare and scattering flare. For reflective flare we provide in-the-focus flare and out-of-focus flare.

\section{Related Works}

This section presents an overview of the flare removal problem and current available  datasets, followed by a discussion on the utility of raw image datasets in mobile computational photography.

% \subsection{Flare Removal Problem and Dataset}
\subsection{Flare Removal Problem}

Lens flare, a common issue in photography, arises from various sources such as light scattering within the lens system, reflections between lens elements, and the impact of dust, contaminants, or scratches on lens surfaces. It can significantly affect the quality of photographs by introducing unwanted artifacts. Lens flare is broadly categorized into two types: scattering flares and reflective flares, each with distinct characteristics and appearances.

Scattering flares occur due to light interacting with microscopic imperfections within the lens system, leading to light scattering in different directions and resulting in visible artifacts such as veiling glare or a series of distortions in the image \cite{gu2009removing, talvala2007veiling, mccann2007camera, raskar2008glare}. The appearance of these flares depends on the distribution and nature of the imperfections. For example, dust on the lens surface may create bright spots or streaks, while scratches can lead to elongated artifacts. Multiple defects can result in complex overlapping flare patterns, degrading the image further.

Reflective flares, on the other hand, are the result of light reflecting between lens elements in multi-element lens systems \cite{hullin2011physically, lee2013practical, chabert2015automated,vitoria2019automatic}. These internal reflections can produce geometric shapes such as concentric rings or polygons, depending on the lens design and the positioning of the light source. Reflective flares are especially noticeable when the light source is near the optical axis or the lens comprises many elements. Furthermore, these flares can vary in shape and clarity based on their focus. In-focus reflective flares appear as sharp, defined patterns, whereas out-of-focus flares often manifest as diffuse, irregular shapes, influenced by lens design and aperture shape.

\subsection{Overview of Flare Datasets}
Several significant datasets have been developed for flare removal research. Wu \etal \  introduced a dataset featuring both captured flare-only images and simulated scattering flare images \cite{wu2021train}. Qiao \etal \ \cite{qiao2021light} collected unpaired flare-corrupted and flare-free images, suitable for networks that capture distribution differences but unsuitable for pixel-to-pixel neural networks due to the lack of paired data . The Flare7K dataset \cite{dai2022flare7k} includes synthesized scattering and reflective flare images, alongside real images for evaluation. Dai \etal \  later expanded this dataset to include real scattering flare images captured in a darkroom, addressing some limitations of their earlier datasets \cite{dai2023flare7k++}. Zhou \etal \  proposed a dataset collected using electronic devices for evaluation purposes \cite{zhou2023improving}.

Despite these contributions, the issue of reflective flare, which is common in everyday photography, remains underexplored. Creating ground truth data for reflective flare is challenging due to its inherent presence in optical systems. Flare7K and Wu \etal’s datasets simulate reflective flares at specific angles without considering the image content context, such as the shape of the light source \cite{dai2022flare7k, wu2021train}. The Bracket Flare dataset by Dai \etal \  addresses in-focus reflective flare with a night-time dataset using a novel composition method \cite{dai2023nighttime}.

Our dataset offers a comprehensive resource for studying both reflective and scattering flares, covering in-focus and out-of-focus reflective flares, as well as local and global corruption in scattering flares. This diverse collection aims to deepen and broaden the scope of flare removal research.

% \subsection{Overview of Flare Datasets}
% Several significant datasets have been developed for flare removal research. Wu \etal introduced a dataset featuring both captured flare-only images and simulated scattering flare images \cite{wu2021train}. Qiao \etal collected unpaired flare-corrupted and flare-free images, suitable for networks for capturing distribution difference but unsuitable for pixel-to-pixel neural networks due to the lack of paired data \cite{qiao2021light}. The Flare7K dataset \cite{dai2022flare7k} includes synthesized scattering and reflective flare images, alongside real images for evaluation. Dai \etal later expanded this dataset to include real scattering flare images captured in a darkroom, addressing some limitations of their earlier datasets \cite{dai2023flare7k++}. Zhou \etal proposed a dataset collected using electronic devices for evaluation purposes \cite{zhou2023improving}.

% Despite these contributions, the issue of reflective flare, which is common in everyday photography, remains underexplored. Creating ground truth data for reflective flare is challenging due to its inherent presence in optical systems. Flare7K and Wu \etal’s datasets simulate reflective flares at specific angles without considering the image content context, such as the shape of the light source \cite{dai2022flare7k, wu2021train}. The Bracket Flare dataset by Dai \etal addresses in-focus reflective flare with a night-time dataset using a novel composition method \cite{dai2023nighttime}.

% Our dataset offers a comprehensive resource for studying both reflective and scattering flares, covering in-focus and out-of-focus reflective flares, as well as local and global corruption in scattering flares. This diverse collection aims to deepen and broaden the scope of flare removal research.
% \subsection{Review of Flare Datasets}
% Regarding datasets for flare removal research, there are a few notable contributions. Wu \etal \  introduced a dataset consisting of flare-only images captured and simulated scattering flare images \cite{wu2021train}. Qiao \etal \ collected unpaired flare-corrupted and flare-free images, suitable for Cycle-GAN-like networks \cite{zhu2017unpaired} but not for pixel-to-pixel neural networks due to the absence of paired data \cite{qiao2021light}. The Flare7K \cite{dai2022flare7k} dataset includes synthesized scattering and reflective flare images, as well as real images for evaluation. It was later expanded by Dai \etal \ to include real scattering flare images captured in a darkroom, addressing some of the limitations of their earlier datasets \cite{dai2023flare7k++}. Zhou \etal \ \cite{zhou2023improving} proposed a dataset collected with electronic devices for evaluation.

% However, the issue of reflective flare, particularly prevalent in everyday photography, remains insufficiently explored in existing datasets. Given the inherent presence of reflective flare in optical systems, creating ground truth data for this type of flare is challenging. Flare7K and the dataset by Wu \etal \ simulate reflective flares at specific angles without considering the context of the image content, such as the shape of the light source \cite{dai2022flare7k, wu2021train}. The Bracket Flare dataset, proposed by Dai \etal, introduces a night-time reflective flare dataset designed to address in-focus reflective flare through a novel composition method \cite{dai2023nighttime}. 

% In our dataset, we provide a comprehensive resource for the study of both reflective and scattering flares, encompassing in-focus and out-of-focus reflective flares, as well as local and global corruption in scattering flares. 
% This diverse collection aims to enhance the depth and breadth of flare removal studies.


\begin{table*}[h!]
\caption{Detailed specifications of the mobile phones used for constructing the dataset.}
\label{tab: devices}
\resizebox{1.6\columnwidth}{!}{%
\begin{tabular}{ccclc}
\hline
Model & Manufacturer & \begin{tabular}[c]{@{}c@{}}CMOS sensor \\ (Main camera)\end{tabular} & \multicolumn{1}{c}{Specification} & Released year \\ \hline
iPhone 13 & Apple & IMX603 & \begin{tabular}[c]{@{}l@{}}12 MP sensor, 1/1.7-inch sensor, 1.7$\mu$m pixels, \\ 26 mm equivalent f/1.6-aperture lens\end{tabular} & 2021 \\
Pixel 7 & Google & GN1 & \begin{tabular}[c]{@{}l@{}}50MP sensor, 1/1.31-inch sensor, 1.2$\mu$m pixels, \\ 24mm equivalent f/1.85-aperture lens\end{tabular} & 2022 \\
iQoo Neo 7 & Vivo & IMX766V & \begin{tabular}[c]{@{}l@{}}50MP sensor, 1/1.56-inch sensor, 1$\mu$m pixels, \\ 23mm equivalent f/1.88-aperture lens\end{tabular} & 2022 \\
Find X6 Pro & OPPO & IMX989 & \begin{tabular}[c]{@{}l@{}}50MP sensor, 1-inch sensor, 1.5$\mu$m pixels,  \\ 23mm equivalent f/1.8-aperture lens\end{tabular} & 2023 \\ \hline
\end{tabular}
}
% \vspace{-5mm}
\end{table*}

% \subsection{Image Signal Processing and RAW Data}
\subsection{Image Signal Processing and RAW Data}

The Image Signal Processing (ISP) Pipeline is a critical component in digital cameras, specifically engineered to process the complex data captured by camera sensors. The main objective of an ISP is to transform raw sensor data into a visually appealing image format, such as JPEG \cite{wallace1992jpeg}. This transformation process encompasses several steps, starting with a linear transform that includes demosaicing \cite{gunturk2005demosaicking}, white balancing \cite{afifi2020deep}, and color correction \cite{zeng2020learning}. Subsequent steps involve non-linear transformations, such as denoising \cite{chen2022simple}, tone mapping \cite{debevec2002tone}, and JPEG compression \cite{wallace1992jpeg}. Additionally, in the realm of mobile photography, ISPs are tasked with advanced image enhancement techniques, like sharpening, to compensate for the limitations imposed by smaller sensors and the inherent processing pipeline, as illustrated in Fig.~\ref{fig: post-processing on isp}.


% \subsection{Comparison with Existing Datasets}
While existing datasets serve as essential resources for addressing the flare removal challenge, they fall short in providing the raw image data necessary to fully comprehend the impact of various ISP operations. This gap is particularly noticeable in operations like tone mapping, which significantly influence performance. Zhou \etal \ highlighted this issue, pointing out that methods like those employed in the Flare7K dataset \cite{dai2022flare7k}, which combine flare and scene images in a gamma-corrected space, fail to account for the non-linear nature of tone mapping, leading to synthetic images with unrealistic contrast and color distortions, especially around bright light sources \cite{zhou2023improving}. In the absence of paired raw data for flare-affected and clean images, these researchers have resorted to simulating tone-mapping effects within their synthesis pipeline through a pixel illuminance-based weighting scheme. While this approach represents a creative attempt to mimic the non-linear effects of tone mapping, it underscores the critical need for access to genuine raw data. This need is not only for improving the accuracy and generalizability of flare removal models but also because raw images are subject to a wider range of operations beyond tone mapping. Our proposed raw image dataset, therefore, presents a unique and valuable resource for examining the influence of tone mapping and other ISP operations on the manifestation of lens flare artifacts, setting the stage for more advanced research in the field.



% \subsection{Image Signal Processing and RAW Data}
% An Image Signal Processing Pipeline (ISP) is a specialized processing pipline in cameras designed to handle complex tasks involved in processing data captured by sensors. The primary function of an ISP is to convert raw data from the raw camera sensor data into a usable image format, such as JPEG \cite{wallace1992jpeg}. A typical processing pipeline involves a linear transform that includes demosaicing \cite{gunturk2005demosaicking}, white balance \cite{afifi2020deep}, color manipulation \cite{zeng2020learning}, and non-linear transform such as denosing, tone-mapping \cite{debevec2002tone}, and JPEG compression \cite{wallace1992jpeg}. Modern mobile phones also incorporate advanced processing algorithms, such as image sharpening for compensating the missing details due to smaller sensor and the previous pipeline, as demonstrated in Fig.~\ref{fig: post-processing on isp}. 

% For these reasons, a high-quality raw image dataset from mobile phones is versatile and desirable for image restoration studies. Unfortunately, the availability of raw datasets captured by mobile phones is still limited. The SIDD dataset \cite{abdelhamed2018high} presents real noisy images from smartphone cameras with high-quality ground truth. The Fujifilm UltraISP dataset \cite{ignatov2022pynet} and the ETH dataset \cite{ignatov2020replacing} aim to enhance learning an ISP for better quality on mobile phones by providing data captured with mobile phones and professional high-end DSLR cameras. 

% While current datasets offer valuable resources for flare removal problem, they  lack the raw image data crucial for understanding the true impact of image processing operations. This limitation becomes particularly evident in the context of the key operations that may affects the performance, such as  tone-mapping as highlighted by Zhou \etal\cite{zhou2023improving}. They argued that existing methods such as Flare7K \cite{dai2022flare7k} directly add flare and scene images in a gamma-corrected space , assuming it's equivalent to the RAW image domain. However, this ignores the non-linear nature of tone-mapping, resulting in synthetic images with inaccurate contrast and color, especially around bright light sources. Due to the unavailability of real raw data pairs for flare-corrupted and clean images, they resort to mimicking the effects of tone-mapping in their synthesis pipeline. By employing a weighting scheme based on pixel illuminance, they attempt to approximate the non-linear transformations introduced by tone-mapping during image capture for approximation of raw image. This approach, while innovative, underscores the importance of having access to actual raw data for developing more accurate and generalizable flare removal models, as raw image has undergone many other opearations than tone-mapping. In contrast to relying on approximations, our proposed raw image dataset offers a unique opportunity to directly study the influence of tone-mapping and other ISP operations on lens flare artifacts, paving the way for further advancements in this domain.


% Given the intricate processes involved, a high-quality raw image dataset from mobile phones becomes an invaluable asset for research in image restoration. However, the availability of such datasets remains scarce. The SIDD dataset \cite{abdelhamed2018high} offers a collection of real noisy images from smartphone cameras alongside their high-quality ground truth counterparts. Similarly, the Fujifilm UltraISP dataset \cite{ignatov2022pynet} and the ETH dataset \cite{ignatov2020replacing} provide data captured using both mobile phones and high-end DSLR cameras, aiming to improve the learning of ISPs for enhanced mobile image quality.
% \subsection{Image Signal Processing and RAW Data}

% The Image Signal Processing (ISP) Pipeline is crucial in digital cameras, converting complex sensor data into visually appealing images, like JPEGs \cite{wallace1992jpeg}. This complex process involves linear transformations such as demosaicing \cite{gunturk2005demosaicking}, white balancing \cite{afifi2020deep}, and color correction \cite{zeng2020learning}, followed by non-linear operations including denoising, tone mapping \cite{debevec2002tone}, and JPEG compression \cite{wallace1992jpeg}. Moreover, in mobile photography, ISPs employ advanced techniques like image sharpening to overcome sensor limitations, enhancing the final image's detail and quality as illustrated in Fig.~\ref{fig: post-processing on isp}.

% High-quality raw datasets from mobile phones are invaluable for image restoration research; however, they are scarce. The SIDD dataset \cite{abdelhamed2018high} and others like the Fujifilm UltraISP \cite{ignatov2022pynet} and ETH dataset \cite{ignatov2020replacing} attempt to fill this gap, providing raw and processed images from both mobile and DSLR cameras to improve ISP learning for mobile image enhancement. Despite the progress, the challenge remains in adequately simulating ISP operations like tone mapping, crucial for realistic flare removal. Many existing datasets fail to capture the non-linear effects of these operations, leading to unrealistic contrast and color in synthetic images, particularly around bright light sources \cite{zhou2023improving}. Our proposed raw image dataset aims to provide a comprehensive basis for studying the effects of tone mapping and other ISP operations on lens flare, enabling more accurate and generalizable flare removal models.