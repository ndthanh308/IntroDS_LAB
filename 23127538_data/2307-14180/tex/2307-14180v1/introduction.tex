\section{Introduction}
% TODO:
% \begin{enumerate}

% \item Prevalence of lens flare in mobile computational imaging: Explain the factors that contribute to lens flare being prevalent in mobile phone cameras, such as the use of plastic lenses, lower lens quality compared to professional cameras, and the lack of expensive coatings like Zeiss.

% \item Causes of lens flare: Discuss the different causes of lens flare, including light scattering within the lens system, reflections between lens elements, and the influence of dust, dirt, or scratches on lens surfaces.

% \item Types of lens flare: Explain the different types of lens flare, such as scattering flares and reflective flares, and describe their distinct characteristics and shapes.

% \item Symmetric properties of lens flare: Introduce the symmetric properties of lens flare, which will be utilized in the proposed method for capturing ground truth data.

% \item Previous approaches to flare removal and their limitations: Briefly review existing methods for flare removal, such as traditional image processing techniques, and highlight their limitations.

% \item Limitations of existing datasets for lens flare removal: Explain the limitations of existing datasets for lens flare removal, such as the lack of real-world examples, limited diversity, and low quality of synthesized flare images.

% \item Motivation for a new dataset and methodology: Describe how the paper will address the limitations mentioned above by proposing a new dataset and methodology for lens flare removal.

% \end{enumerate}

% \section{Introduction}

% Lens flare is a common optical artifact that occurs when non-image-forming light enters the camera's lens system and interacts with the imaging sensor. This artifact can degrade image quality and negatively affect the visual appearance of photographs, especially in mobile computational imaging, where lens flare is more prevalent due to factors such as the use of plastic lenses, lower lens quality compared to professional cameras, and the lack of expensive anti-reflective (AR) coatings.

% Different causes of lens flare include light scattering within the lens system, reflections between lens elements, and the influence of dust, dirt, or scratches on lens surfaces. Lens flare can be categorized into two main types: scattering flares and reflective flares, each with their distinct characteristics and shapes.
% Scattering flares are caused by the interaction of light with microscopic imperfections and defects within the lens system, such as dust, dirt, or scratches. When light passes through the lens, these imperfections cause the light to scatter in various directions, resulting in a visible haze or a series of artifacts in the captured image. The shape and appearance of scattering flares depend on the nature and distribution of the defects within the lens. For instance, dust particles on the lens surface can cause small, localized bright spots or streaks, whereas scratches can produce more elongated, linear artifacts. Moreover, the presence of multiple defects may result in a complex pattern of overlapping flares, which can further degrade image quality.
% Reflective flares, on the other hand, are caused by reflections between lens elements, particularly in multi-element lens systems. When light enters the lens, it can reflect off the internal surfaces of the lens elements, bouncing between them before eventually reaching the image sensor. These internal reflections can create a series of concentric rings, polygons, or other geometric shapes in the image, depending on the lens design and the relative position of the light source. Reflective flares are often more pronounced when the light source is close to the optical axis or when the lens system includes a large number of elements.
% In addition, reflective flares can exhibit different shapes and appearances depending on whether they are in-focus or out-of-focus. In-focus reflective flares tend to form sharp, well-defined geometric patterns, such as a white spot. Out-of-focus reflective flares, on the other hand, can appear as more diffuse and irregular shapes, often taking on the form of circular or elliptical blobs, known as bokeh. The characteristics of out-of-focus flares are influenced by factors such as the lens design, aperture shape, and the degree of defocus.
% The symmetric properties of lens flare, particularly in the case of reflective flares, can be utilized in our proposed method for capturing ground truth data. By carefully controlling the position and orientation of the light source relative to the lens system, we can exploit the symmetry of the flare patterns to obtain accurate and consistent ground truth data for training and evaluating lens flare removal algorithms.

% Previous approaches to flare removal have relied on traditional image processing techniques and synthesized datasets for training. However, these methods suffer from various limitations, such as the inability to handle complex real-world examples, limited diversity, and low quality of synthesized flare images.
% Existing datasets for lens flare removal also have their limitations. For instance, they often lack real-world examples, feature limited diversity in flare types, and rely on low-quality synthesized flare images. These limitations make it challenging to train and evaluate robust lens flare removal algorithms that can handle diverse real-world scenarios.

% In this paper, we address these challenges by proposing a new dataset and methodology for lens flare removal. Our dataset consists of real-world examples of lens flare captured with mobile phone cameras and saving them into raw image format and we also save the image processed by the internal ISP, providing rich and diverse data for training and evaluation. To our best knowledge, we are the first dataset for providing real image data for supervised training for the flare removal problem for both scattering flare and reflective. Moreover, for reflective flare, in the previous works there is no real image pairs that can be used to resolve this problem. Our methodology leverages the symmetric properties of lens flare to capture ground truth data, allowing for supervised training of state-of-the-art deep learning models.
% By addressing the limitations of existing datasets and methods, our work aims to significantly advance the field of lens flare removal in mobile computational imaging, ultimately leading to improved image quality and enhanced visual experiences for users of mobile phone cameras.
% % Figure environment removed

% Figure environment removed



% Lens flare is a common optical artifact that occurs when non-image-forming light enters a camera's lens system and interacts with the imaging sensor. This artifact can degrade image quality and negatively impact the visual appearance of photographs, particularly in mobile computational imaging. Lens flare is more prevalent in this domain due to factors such as the widely use of plastic lenses on mobile camera systems, which lowers lens quality compared to professional cameras, and the absence of expensive anti-reflective (AR) coatings.

Lens flare \cite{holladay1926fundamentals,seibert1985removal,goodman2005introduction,kingslake1992optics} is a common optical artifact that occurs when non-image-forming light enters a camera's lens system and interacts with the imaging sensor.  This phenomenon can degrade image quality and adversely affect the visual appeal of photographs, especially in mobile computational imaging. Lens flare is more prevalent in this field due to factors such as the widespread use of plastic lenses in mobile camera systems, resulting in lower lens quality compared to professional cameras, and the lack of costly anti-reflective (AR) coatings \cite{blahnik2016reduction}.

There are various causes of lens flare, including light scattering within the lens system, reflections between lens elements, and the influence of dust, contaminants, or scratches on lens surfaces. Lens flare can be broadly classified into two types: scattering flares and reflective flares, each exhibiting distinct characteristics and shapes.

Scattering flares \cite{gu2009removing, talvala2007veiling, mccann2007camera, raskar2008glare} arise from the interaction of light with microscopic imperfections and defects within the lens system. As an example shown in Fig.~\ref{fig: reflect and scatter examples}, these imperfections cause light to scatter in various directions, resulting in a visible haze such as veiling glare \cite{talvala2007veiling} or a series of artifacts in the captured image. The shape and appearance of scattering flares depend on the nature and distribution of the defects within the lens. Dust particles on the lens surface can cause small, localized bright spots or streaks, while scratches can produce more elongated, linear artifacts. The presence of multiple defects may lead to a complex pattern of overlapping flares, further degrading image quality.

Reflective flares \cite{hullin2011physically, lee2013practical, chabert2015automated,vitoria2019automatic}, in contrast, are caused by reflections between lens elements, particularly in multi-element lens systems, as illustrated in Fig.~\ref{fig: reflect and scatter examples}. When light enters the lens, it can reflect off the internal surfaces of the lens elements, bouncing between them before eventually reaching the image sensor. These internal reflections can create a series of concentric rings, polygons, or other geometric shapes in the image, depending on the lens design and the relative position of the light source. Reflective flares are often more pronounced when the light source is close to the optical axis or when the lens system comprises a large number of elements.

Moreover, reflective flares can exhibit different shapes and appearances depending on whether they are in-focus or out-of-focus. In-focus reflective flares tend to form sharp, well-defined geometric patterns, such as white spots \cite{asha2019auto, chabert2015automated,vitoria2019automatic}. Out-of-focus reflective flares can appear more diffuse and irregular, often taking the form of circular or elliptical blobs, known as bokeh. Factors influencing out-of-focus flares include lens design, aperture shape, and the degree of defocus.

% % Figure environment removed
The symmetric properties of lens flare, particularly in the case of reflective flares, can be utilized in our proposed method for capturing ground truth data. For a camera lens with rotational symmetry, meaning its elements are centered along the optical axis, the flare chain stretches in a straight line from the light source through the center of the image. Each individual ghost image's shape exhibits symmetry with respect to this axis. This occurs because light rays originating from a specific source point travel symmetrically concerning the tangential plane \cite{blahnik2016reduction}. By carefully controlling the position and orientation of the light source relative to the lens system, we can exploit the symmetry of flare patterns to obtain accurate and consistent ground truth data for training and evaluating lens flare removal algorithms.

Previous approaches to flare removal have relied on traditional image processing techniques or synthetic datasets \cite{chabert2015automated, asha2019auto,vitoria2019automatic, wu2021train, dai2022flare7k,hullin2011physically} for training. However, these methods suffer from various limitations, such as the inability to handle complex real-world examples, limited diversity, and low quality of synthesized flare images. Existing datasets for lens flare removal also have their shortcomings, as they often lack real-world examples, feature limited diversity in flare types, and rely on low-quality synthesized flare images. These limitations make it challenging to train and evaluate robust lens flare removal algorithms capable of handling diverse real-world scenarios.

% In this paper, we address these challenges by proposing a new dataset and methodology for lens flare removal. Our dataset consists of real-world examples of lens flare captured with mobile phone cameras, saved in raw image format and processed by the internal ISP, providing rich and diverse data for training and evaluation. To the best of our knowledge, our dataset is the first to provide real image data for supervised training for both scattering and reflective flare removal. Furthermore, for reflective flare, our dataset presents the first real image pairs available for addressing this problem. Our methodology leverages the symmetric properties of lens flare to capture ground truth data, allowing for supervised training of state-of-the-art deep learning models.
% By addressing the limitations of existing datasets and methods, our work aims to significantly advance the field of lens flare removal in mobile computational imaging, ultimately leading to improved image quality and enhanced visual experiences for users of mobile phone cameras. Our novel dataset, which includes both scattering and reflective flares in real-world scenarios, offers a valuable resource for researchers and practitioners working on lens flare removal techniques. Moreover, our proposed methodology for capturing ground truth data based on the symmetric properties of lens flare enables more accurate and consistent training and evaluation of lens flare removal algorithms. Together, these contributions have the potential to drive significant progress in the development of more robust and effective lens flare removal solutions, ensuring better image quality across a wide range of mobile computational imaging applications.

In this paper, we present the construction of a novel dataset for lens flare removal in mobile computational imaging, addressing the limitations of existing datasets and methods. Our dataset comprises real-world examples of both scattering and reflective flares captured with mobile phone cameras, saved in raw image format and processed by the internal processing pipline of the mobile phone. This dataset, the first of its kind, enables supervised training of state-of-the-art deep learning models and provides valuable ground truth data by leveraging the symmetric properties of lens flare. Our findings reveal that networks trained with synthetic data struggle on this real image dataset, and that internal processing pipline processed data are more challenging to restore compared to raw image data, possibly due to aggressive post-processing algorithms and heavy compression. These insights confirm the importance of real-world data and raw image formats for more accurate and reliable lens flare removal, ultimately leading to improved image quality and enhanced visual experiences for mobile phone camera users.


