\documentclass[12pt]{article}

\usepackage{epsfig} 
\usepackage{amsfonts, amssymb, amsthm, graphicx}
\usepackage{amsmath, accents, bbold}
%\usepackage{amsbsy}
\usepackage{authblk}
\usepackage{cmll}
\usepackage{dsfont}
\usepackage{array}
\usepackage{textgreek}
\usepackage{cite}
\usepackage{listings}
\usepackage{color}
\usepackage[colorlinks=true,allcolors=red]{hyperref}
%\usepackage[notext]{stix2}

\newtheorem{tw}{Theorem}
\newtheorem{de}{Definition}
\newtheorem{co}{Corollary}

\renewcommand{\thesection}{\arabic{section}}
\renewcommand{\theequation}{\thesection.\arabic{equation}}
\renewcommand{\theenumi}{\thesection.\roman{enumi}}
\renewcommand{\thede}{\thesection.\arabic{de}}
\renewcommand{\thetw}{\thesection.\arabic{tw}}

\renewcommand{\theco}{\thesection.\arabic{co}}
\newcommand{\be}{\begin{equation}}
\newcommand{\ee}{\end{equation}}
\newcommand{\bea}{\begin{eqnarray}}
\newcommand{\eea}{\end{eqnarray}}
\newcommand{\dd}{\mathrm{d}}
\DeclareMathOperator{\Tr}{Tr}

\newcounter{orange} \renewcommand{\theorange}{\alph{orange}}

\parskip=1ex
\oddsidemargin= 0.5cm
\evensidemargin= 0.5cm
\parindent=1.5em
\textheight=23.0cm
\textwidth=16.2cm
\topmargin=-1.0cm

\begin{document}

\title{The continuity equation and its applications  in phase space quantum mechanics}



\author{ Jaromir Tosiek \footnote{E-mail address: jaromir.tosiek@p.lodz.pl}}
\author{Luca Campobasso \footnote{E-mail address: luca.campobasso@p.lodz.pl }}
\affil[1]{Institute of Physics, Lodz University of Technology, ul. W\'olcza\'nska 217/221, 93-005 \L\'od\'z, Poland}
\date{\today} 
 \maketitle 

\begin{abstract}
The continuity equation for quantum systems with internal degrees of freedom and  represented by density operators is derived. A phase space description of such systems is presented. 
The $1$ -- D Dirac equation is discussed and its phase space counterpart is found.
 The phase space  representation of free motion and of scattering in a nonrelativistic and a relativistic case is discussed and illustrated.
Properties of Wigner functions of unbound states are analysed.
\end{abstract}

\section{Introduction}

Quantum mechanics describes the part of physical reality inaccessible to our senses. Thus the only one way to deal with the quantum reality is mathematics. The formalism most commonly used by researchers is based on a separable Hilbert space. On the other hand, classical physics is  a limit of the quantum world. Then it is natural, that these two cases should be represented in a similar mathematical frame. Since classical mechanics in its Lagrange or Hamilton approaches works perfectly even in general relativity, we believe that the unified language of quantum and classical mechanics should be based on the phase space formalism rather than an algebraic one. Therefore we are trying to develop the so called phase space version of quantum physics. 

 The basic ingredients of this formulation of quantum theory are contained in \cite{WY31, WI32, GW46, MO49}. As milestones in development of  phase space quantum mechanics one can consider a formal series calculus \cite{bayen78}, a generalisation of the Moyal product on arbitrary symplectic manifolds \cite{fedosov94, fedosov96} and a construction of phase space for integral degrees of freedom \cite{gracia88, varilly89}. For readers interested in a holistic approach to this formalism we recommend looking into \cite{YK91, FS94, WS01, CZ05, tat, cd, lee, dit, SW07, tosiek21}. 

Our contribution focuses on unbound states of quantum systems with internal degrees of freedom in the nonrelativistic case and the relativistic one represented by the $1$ --D Dirac equation. Thus we begin with derivation of continuity equation for structures represented by density operators. Then we remind some facts about scattering and  introduce the phase space formalism referring to particles with discrete variables. That part is based on our earlier works \cite{przanowski14, przanowski19}. In the next step the phase space version of nonrelativistic and relativistic continuity equation is discussed. Finally  the free motion of the nonrelativistic and the relativistic particles as well as the scattering on step potential are analysed.





 
 \section{The time evolution of spatial density of probability in the Hilbert space  quantum mechanics}
 
The continuity equation applied to the spatial density of probability expresses the conservation of probability. It plays a crucial role e.g. in analysis
of  scattering processes. The continuity equation is usually written in terms of a wave function. However, since our goal is construction of the phase space counterpart of it, we
will derive its more general version based on a density operator $\hat{\varrho}(t)$. 
  
Assume our object of interest is a quantum particle living in space ${\mathbb R}^3.$ This particle has an internal structure enabling $s+1, \, s \in {\mathcal N}$
possible eigenstates. Thus the states of it are represented by density operators acting in the tensor product of Hilbert spaces $L^2 ({\mathbb R}^3) \otimes
{\mathbb C}^{s+1}.$

Under the aforementioned assumptions the spatial density of probability $\rho({\vec r}_0,t)$ of finding the particle at a point ${\vec r}_0$ at
instant of time $t$ in an arbitrary internal state is equal to
 \[
\rho({\vec r}_0,t)={\rm Tr} \left\{ \hat{\varrho}(t)  \left(|{\vec r}_0\big> \big<{\vec r}_0|\otimes \hat{\bf 1}\right) \right\}.
 \]
 The speed  of change of the  spatial density of probability  at ${\vec r}_0$ is
 \[
 \frac{\partial }{\partial t}\rho({\vec r}_0,t)=
 \frac{\partial }{\partial t}
{\rm Tr} \left\{ \hat{\varrho}(t)  \left(|{\vec r}_0\big> \big<{\vec r}_0|\otimes \hat{\bf 1}\right) \right\}
= {\rm Tr} \left\{\frac{\partial }{\partial t}\hat{\varrho}(t) \left( |{\vec r}_0\big> \big<{\vec r}_0|\otimes \hat{\bf 1}\right)
\right\}.
\]

As it is known, the time evolution  of the density operator is  represented by the Liouville -- von Neumann equation 
\be
\label{1}
 \frac{\partial \hat{\varrho}(t)}{\partial t} + \frac{1}{i \hbar} [\hat{\varrho}(t), \hat{H}]=0.
\ee
It implies  that the speed of change of spatial density of probability satisfies the condition
\[
\frac{\partial}{ \partial t}\rho({\vec r}_0,t) =-
\frac{1}{i \hbar}  {\rm Tr} \left\{ \big[\hat{\varrho}(t), \hat{H} \big] (|{\vec r}_0\big> \big<{\vec r}_0|\otimes \hat{\bf 1})\right\}.\]
Calculating the trace in the position representation we get
\be
\label{nn28}
\frac{\partial}{ \partial t}\rho({\vec r}_0,t)=
- \frac{1}{i \hbar} \sum_{k=0}^s \int_{{\mathbb R}^3} d {\vec r} \; \big< {\vec r},k| \big[\hat{\varrho}(t), \hat{H} \big] |{\vec
r}_0\big> \big<{\vec r}_0|\otimes \hat{\bf 1}| {\vec r},k\big>
=- \frac{1}{i \hbar} \sum_{k=0}^s \big< {\vec r}_0,k|  \big[\hat{\varrho}(t), \hat{H} \big] |{\vec r}_0,k\big>.
\ee
We restrict to the Hamilton operators  represented  by sums of self -- adjoint terms $\hat{T}$ and $\hat{V}$ such that
\be
\label{dod1}
\hat{H} =\hat{T} + \hat{V}\otimes \hat{V}_{\rm int}
\ee
where
\[
[\hat{T}, \hat{\vec{p}}\,]= \hat{0}  \;\;\;
{\rm and } \;\;\;
[\hat{V}, \hat{\vec{r}}\,]= \hat{0}.
\]
 Please notice that the postulated form of Hamiltonian excludes e.g. the presence of a magnetic field or the spin -- orbit interaction. As usual, the
operator of momentum is $\hat{\vec{p}}=(\hat{p}_x,\hat{p}_y, \hat{p}_z),$ and the operator of position equals $\hat{\vec{r}}=(\hat{x},
\hat{y}, \hat{z}).$
Moreover, let $\hat{\sigma}: {\mathbb C}^{s+1} \rightarrow {\mathbb C}^{s+1} $ be an operator representing the internal degree of freedom, which eigenvectors are kets $\big| k \big>$ indexed
by $ k=0, \ldots s.$ The set of vectors $\{\big| k \big>\}_{k=0}^{s+1}$ constitutes a basis of the space $ {\mathbb C}^{s+1}.$

Of course one can discuss the Hamilton operators more general than  \eqref{dod1} but in  our opinion it would obscure 
presentation of 
  the phase space the description of unbound states with internal degrees of freedom.

Since for the operator \eqref{dod1} the sum in formula \eqref{nn28}
\be
\label{nn2}
\sum_{k=0}^s \big< {\vec r}_0,k|  \big[\hat{\varrho}(t), \hat{V} \otimes \hat{V}_{\rm int} \big] |{\vec r}_0,k\big>=0
\ee
disappears we conclude 
that the time derivative of the spatial density of probability is equal to
\be
\label{001}
\frac{\partial }{\partial t}\rho({\vec r}_0,t)= - \frac{1}{i \hbar} \sum_{k=0}^s \big< {\vec r}_0,k| [\hat{\varrho}(t), \hat{T} ]|{\vec r}_0,k\big>.
\ee
This formula resembles  very much the continuity equation
\be
\label{01}
\frac{\partial }{\partial t}\rho({\vec r}_0,t) + {\rm div} {\vec j}({\vec r}_0,t)=0.
\ee
Indeed the expression \eqref{001} would be the continuity equation
if there existed a vector ${\vec j}({\vec r}_0,t)$ such that
\be
\label{101}
{\rm div} {\vec j}({\vec r}_0,t)= \frac{1}{i \hbar} \sum_{k=0}^s \big< {\vec r}_0,k| [\hat{\varrho}(t), \hat{T} ]
|{\vec r}_0,k\big>.
\ee
For the fixed operators $\hat{\varrho}(t)$ and $\hat{T}$ one can usually introduce several vectors ${\vec j}({\vec r}_0,t)$ fulfilling that condition. In
the next two subsections we discuss the problem of choice of vector ${\vec j}({\vec r}_0,t)$ in a $3$ -- D nonrelativistic case and then for a
$1$ -- D relativistic particle satisfying the Dirac equation.

\subsection{The nonrelativistic particle in the $3$ -- D space}

Let us derive an explicit form of the continuity equation for a nonrelativistic particle when   when $\hat{T}= \frac{\hat{\vec p}^{\,2}}{2m} \otimes \hat{\bf 1}.$ 
Substituting this operator into \eqref{101} we obtain that
\be
\label{12071}
 \big< {\vec r}_0,k|  \big[ \hat{\varrho}(t), \hat{T} \big] |{\vec r}_0,k\big>
=  \frac{1}{2m} 
\big< {\vec r}_0,k| \big( \hat{\varrho}(t) \hat{\vec p} + \hat{\vec p} \hat{\varrho}(t) \big) \cdot \hat{\vec p} - \hat{\vec p} \cdot
\big(\hat{\varrho}(t) \hat{\vec p} +
\hat{\vec p}  \hat{\varrho}(t) \big) |{\vec r}_0,k\big>.
\ee
To simplify notation we have omitted the identity operator $\hat{\bf 1}.$ Applying the observations that in the position representation
$\hat{\vec p}= - i \hbar \left( \frac{\partial}{\partial x}, \frac{\partial}{\partial y}, \frac{\partial}{\partial z}\right) $ and
$\hat{\vec p}= \hat{\vec p}^{\; \dagger},$ we deduce that formula \eqref{12071} is equal to
\[
- \frac{i \hbar}{2m} {\rm div} \big< {\vec r}_0,k| \hat{\varrho}(t) \hat{\vec p} + \hat{\vec p} \hat{\varrho}(t) |{\vec r}_0,k\big>.
 \]
 Therefore the most natural choice of the current density vector is 
 \be
\label{011}
{\vec j}({\vec r}_0,t):= \frac{1}{2m} \sum_{k=0}^s \big< {\vec r}_0,k| \hat{\varrho}(t) \hat{\vec p} + \hat{\vec p} \hat{\varrho}(t)
|{\vec r}_0,k\big>.
\ee
This result is valid for both: pure and mixed states.
 When the  system is a pure state 
 $
 \hat{\varrho}(t)= \big| \Psi(t) \big> \big< \Psi(t) \big|,
 $
 the  current density vector equals
 \be
 \label{11}
\vec{j}({\vec r}_0,t)= - \frac{\hbar}{2mi}\sum_{k=0}^s \left(\Psi({\vec r}_0,k,t) {\rm grad} \overline{\Psi}({\vec r}_0,k,t) -
\overline{\Psi}({\vec r}_0,k,t)
 {\rm grad} \Psi({\vec r}_0,k,t)
 \right),
 \ee
 as expected.



%%%%%%%%%%%%%%%%%%%%%%%%%%%%%%%%%%%%%%%%%%%%%%%%%%%%%%%%%%%%%%%%%%%%%%%%%%%%%%%%%%%%%%%%%%%%%%%%%%%%%%%%%%%%%%%%%%%%%%%%%%%%%%%%%%%%%%%%%%%%%%%%%%%%%%%%%%%%%%%%%%%%%%%%%%%%%%%%%%%%%%%%%%%%%%%%%%%%%%%%%%%%%%%%%%%%%%%%%%%%%%%%%%%%%%%%%%%%%%%
 
 \subsection{The relativistic case for the free $1$ -- D Dirac equation}
 \label{sub2.2}

Let us consider the continuity equation for a  particle described by the $1$ -- D Dirac equation. 
In this case the interpretation of
the conservation law changes, because it refers to the electric charge rather than the spatial probability but the general idea of its construction
remains the same.

 At the beginning the $1$ -- D Dirac equation was treated as a toy model but now one can observe a growing interest in it (see \cite{JK20} and works quoted therein). This relation is used e.g. in modelling graphene. The main advantage of the $1$ -- D Dirac equation in comparison to its more dimensional versions is the absence of some interactions e.g. the spin -- orbit interaction.

The general shape of the free Hamilton operator for the Dirac particle is
\be
\label{2.1}
\hat{H}= \alpha (c \hat{p}) + \beta  m c^2 
\ee
where $\alpha$ and $\beta$ are some square matrices. Due to the requirement
\[
\hat{H}^2= c^2 \hat{p}^2 + m^2 c^4 
\]
we can see that there must be
\be
\label{nn27}
\alpha^2= {\mathbf 1} \;\;, \;\; \alpha \cdot \beta + \beta \cdot \alpha = 0 \;\; , \;\;
\beta^2= {\mathbf 1}
\ee
where the symbol ${\mathbf 1}$ represents the identity matrix.

On the contrary to the $3$ -- D case the system of conditions \eqref{nn27} can be fulfilled by $2 \times 2$ square matrices e.g. 
 \[
 \alpha = \sigma_x = \left[\begin{array}{cc}
 0 & 1 \\
 1 & 0 
 \end{array}
 \right] \;\;\; {\rm and} \;\;\;  \beta= \sigma_z = \left[\begin{array}{cc}
 1 & 0 \\
 0 & -1 
 \end{array}
 \right]
 \]
where $\sigma_x$ and $\sigma_z$ are the respective Pauli matrices.

 Since the time-dependent Dirac equation is of the form
 \[
 i \hbar \frac{\partial}{\partial t}|\Psi(t) \big>= \hat{H}|\Psi(t) \big>
 \]
we easily deduce that vector $|\Psi(t) \big>$ is a two -- component object $|\Psi(t) \big>= \left[ \begin{array}{c}|\Psi_1(t) \big> \\
|\Psi_{0}(t) \big>\end{array}\right]$ belonging to the tensor product of Hilbert spaces $ {\mathcal H} \otimes {\mathbb C}^2, $ where
the space ${\mathcal H}$ is isomorphic to $L^2({\mathbb R}).$

 The space ${\mathbb C}^2$ is  spanned by eigenvectors of operator $\hat{\sigma}_z$ represented by the Pauli matrix $\sigma_z$ and its eigenvectors $|1 \rangle $,   $ |0 \rangle$ obey the equations
\be
\label{nn4}
\hat{\sigma}_z |1 \rangle = 1 \cdot |1 \rangle \;\;\; , \;\;\; \hat{\sigma}_z |0 \rangle = -1 \cdot |0 \rangle.
\ee
Notice that the operators $\hat{\sigma}_x$ and $\hat{\sigma}_z$ do not commute with the Hamilton operator \eqref{2.1}. 
 The
operators $\hat{\sigma}_x$ and $\hat{\sigma}_z$ can be alternatively written as
\[
\hat{\sigma}_x=   |0\rangle\langle1| + |1\rangle\langle 0 | \;\;\; , \;\;\; \hat{\sigma_z}= |1\rangle\langle1| - |0 \rangle\langle 0|.
\]
Possible eigenvalues of the Hamilton operator \eqref{2.1}  belong to the sum of intervals
\[
E \in (-\infty, -mc^2) \cup (mc^2, + \infty)
\]
and satisfy the dispersion relation:
\begin{equation}
    E^2 = (cp)^2 + (mc^2)^2,\qquad E_\pm = \pm\sqrt{(c p)^2 + (mc^2)^2}.
\end{equation}


The normalised eigenfunctions of the relativistic free particle are indexed by the value of momentum ${\tt p} \in {\mathbb R}$ 
\begin{equation}
    \psi_{{\tt p} \pm} (x) =  \frac{1}{\sqrt{2\pi \hbar}}\begin{pmatrix}
    \frac{c{\tt p}}{E_{\pm}- mc^2} \\ 1
    \end{pmatrix}\left(\frac{(c{\tt p})^2}{(E_{\pm}- mc^2)^2} + 1\right)^{-\frac{1}{2}} \exp \left(\frac{i{\tt p}x}{  \hbar} \right).
\end{equation}
They fulfil the  standard orthonormality conditions 
\[
   \int_\mathbb{R} \psi^{\dagger}_{{\tt p'}+}(x)\psi_{{\tt p}+}(x)\dd x =  \delta({\tt p}-{\tt p'})
   \;\; , \;\;
   \int_\mathbb{R} \psi^{\dagger}_{{\tt p'}-}(x)\psi_{{\tt p}-}(x)\dd x =  \delta({\tt p}- {\tt p'})   
\]
together with
\be
\int_\mathbb{R} \psi^{\dagger}_{{\tt p'}+}(x)\psi_{{\tt p}-}(x)\dd x =0.
\ee
We observe that in the $1$ -- D case of the Dirac equation there is no spin and solutions are parametrised exclusively by the value of momentum ${\tt p}$
and the sign of energy. {\it Indeed, there is no nontrivial operator represented by a $2 \times 2$ matrix, which would commute with the
momentum and the Hamilton operators}. A physical explanation of the phenomenon is that the spin is related to a rotation which in the $1$
-- D case has no sense.
But  for a fixed momentum we still have  the particle `$+$' and the antiparticle `$-$' solutions. 

Let us consider the $1$ -- D Dirac equation with a potential
\be
\label{nn5}
\hat{H}=  c \hat{p}  \sigma_x +  m c^2 \sigma_z + \hat{V}(x) \otimes \hat{V}_{\rm int}.
\ee
A density operator for a $1$ -- D Dirac particle acts in the tensor product of Hilbert spaces
 \[
 \hat{\varrho}(t):   L^2({\mathbb R}) \otimes {\mathbb C}^2 \rightarrow   L^2({\mathbb R}) \otimes {\mathbb C}^2.
 \]
Thus the density operator can be treated as a matrix
 \[
 \hat{\varrho}(t)= \left[\begin{array}{cc}
 \hat{\varrho}_{11}(t) & \hat{\varrho}_{1 0}(t) \\
 \hat{\varrho}_{01}(t) & \hat{\varrho}_{00}(t)
 \end{array}
 \right] .
 \]
 The time evolution of the density operator in the Schroedinger picture is given by the general formula \eqref{1}.
 The projection operator on the state representing a particle localised in the configuration space at a point $x$  equals
 \[
 \hat{\Pi}_{x}=
| x \rangle \langle x | \otimes \hat{\mathbf 1}
=| x, 1 \rangle \langle x,1 | + | x, 0 \rangle \langle x,0 |= 
 \left[ 
 \begin{array}{cc}
  | x \rangle \langle x | & 0 \\
 0 &  | x \rangle \langle x | 
 \end{array}
 \right].
 \]
 Hence the electric charge density  at a point $x_0$ is calculated as
 \[
 \rho(x_0,t) = {\rm Tr } \left\{ q\hat{\varrho}(t) \hat{\Pi}_{x_0} \right\} 
 =  q \langle x_0 | \hat{\varrho}_{11}(t) | x_0 \rangle + q\langle x_0 | \hat{\varrho}_{00}(t) | x_0 \rangle,
 \]
where by `$q$' we denote a charge of the Dirac particle.
The time evolution of the charge density $\rho(x,t)$ is given by the expression
\be
\label{201}
\frac{\partial \rho(x_0,t)}{\partial t}= q \big< x_0 |\frac{\partial \hat{\varrho}_{11}(t)}{\partial t} | x_0 \big> + q \big< x_0
|\frac{\partial \hat{\varrho}_{00}(t)}{\partial t} | x_0 \big>.
\ee
As it was shown before  \eqref{001}, neither the potential nor  the term $m c^2 \sigma_z$  influence the time evolution of the spatial density of probability. Therefore it is sufficient to calculate 
the commutator 
\be
\label{202}
[\hat{\varrho}(t), c\hat{p}\sigma_x]= 
\left[
\begin{array}{cc}
c \hat{\varrho}_{10}(t)\hat{p}- c\hat{p} \hat{\varrho}_{01}(t) & c \hat{\varrho}_{11}(t)\hat{p}- c\hat{p} \hat{\varrho}_{00}(t)
 \\
c \hat{\varrho}_{00}(t)\hat{p}- c\hat{p} \hat{\varrho}_{11}(t) & c \hat{\varrho}_{01}(t)\hat{p}- c\hat{p}
\hat{\varrho}_{10}(t)
\end{array}
\right].
\ee
Now comparing formulas \eqref{1} and \eqref{202} one can see easily that
\[
 \frac{\partial  \hat{\varrho}_{11}(t)}{\partial t}
=- \frac{c}{i \hbar} \Big(  \hat{\varrho}_{10}(t)\hat{p}- \hat{p} \hat{\varrho}_{01}(t)\Big)
\;\;\;
{\rm and}\;\;\;
\frac{\partial  \hat{\varrho}_{00}(t)}{\partial t}
 = -  \frac{c}{i \hbar} \Big(  \hat{\varrho}_{01}(t)\hat{p}- \hat{p} \hat{\varrho}_{10}(t)\Big).
\]
Relation \eqref{201} is  a   continuity equation.
Indeed we can postulate that
\[
{\rm div }\vec{j}(x_0,t)= \frac{qc}{i \hbar} \langle x_0 | - \hat{p} (\hat{\varrho}_{10}(t)+\hat{\varrho}_{01}(t) ) +
(\hat{\varrho}_{10}(t)+\hat{\varrho}_{01}(t) ) \hat{p} | x_0 \rangle
\]
so the charge current density equals
\be
\label{130722}
\vec{j}(x_0,t)= qc \langle x_0 |  \hat{\varrho}_{10}(t)+\hat{\varrho}_{01}(t)  | x_0 \rangle.
\ee

For a pure state $ \left[ \begin{array}{c}|\Psi_1(t) \big> \\
|\Psi_{0}(t) \big>\end{array}\right]$
 the unique component of current density  is expressed as
\[
j(x_0,t)=qc(  \Psi_1(x_0,t)\overline{\Psi_{0}(x_0,t)} + \Psi_{0}(x_0,t)\overline{\Psi_1(x_0,t)}).
\]
In that situation  the spatial charge density at the point $x_0$ is given by the relation
\[
{\rho}(x_0,t)= q \Big( |\Psi_1(x_0,t)|^2 + |\Psi_{0}(x_0,t)|^2 \Big).
\]

\vspace{0.5cm}
At the end of this section we present a few general remarks about the current density for stationary states. These observations are valid in the spaces ${\mathbb R} $ and  ${\mathbb R}^3 $ for nonrelativistic and relativistic systems.

Let us consider a stationary state   $\frac{\partial
\hat{\varrho}(t)}{\partial t}=0$ . Therefore at every point ${\vec r}_0$ the time derivative of the spatial density of probability  charge density  vanishes
 $
  \frac{\partial }{\partial t}\rho({\vec r}_0,t)=0
 $  
 and the continuity equation implies that
$
 {\rm div} {\vec j}(\vec{r}_0,t)=0.
$
Applying the nonrelativistic definition \eqref{011} as well as the relativistic one \eqref{130722} we deduce that $ \frac{\partial
}{\partial t} {\vec j}(\vec{r}_0,t)=0$ so in every stationary case the current density is independent from the time. 

Moreover, since for every instant of time the equality $
 {\rm div} {\vec j}(\vec{r}_0)=0
$ is satisfied, from the Gauss theorem we deduce that even at points of discontinuity of the potential the current density is continuous.



%%%%%%%%%%%%%%%%%%%%%%%%%%%%%%%%%%%%%%%%%%%%%%%%%%%%%%%%%%%%%%%%%%%%%%%%%%%%%%%%%%%%%%%%%%%%%%%%%%%%%%%%%%%%%%%%%%%%%%%%%%%%%%%%%%%%%%%%%%%%%%%%%%%%%%%%%%%%%%%%%%%%%%%%%%%%%%%%%%%%%%%%%%%%%%%%%%%%%%%%%%%%%%%%%%%%%%%%%%%%%%%%%%%%%%%%%%%%%%%%%%%%%%%%%%%%%%%%%%%%%%%%%%%%%%%%%%%%%%%%%%%%%%%%%%%%%%%%%%%%%%%%%%%%%%%%%%%%%%%%%%%%%%%%%%%%%%%%%%%%%%%%%%%%%%%%%%%%%%%%%%%%%%%%%%%%%%%%%%%%%%%%%%%%%%%%%%%%%%%


 \section{Scattering on a potential barrier}
 
 \setcounter{equation}{0}

Theory of scattering is the part of quantum mechanics in which the continuity equation plays the fundamental role. 
In this section we present  general remarks about the systems in which scattering is observed. We focus on  cases   with   the potentials \eqref{dod1}  constant in infinity i.e. such that in the position representation 
the limit
\be
\label{02}
\lim_{|\vec{r}\,|\rightarrow \infty}{\rm grad} V(\vec{r}\,)=0.
\ee

The process of scattering of particles on  the barrier represented by a function $V(\vec{r}\,)$ is stationary and characterised with the use of coefficients 
of transmission $T$ and of reflection $R.$  
 In the  situation, when the potential depends exclusively on one Cartesian coordinate, these coefficients take the form
 \[
 T= \frac{|\vec{j}_{trans}(\vec{r}_2)|}{|\vec{j}_{inc}(\vec{r}_1)|} \;\; , \;\;R = \frac{|\vec{j}_{ref}(\vec{r}_1)|}{|\vec{j}_{inc}(\vec{r}_1)|}. 
 \] 
where the applied symbols mean respectively: $\vec{j}_{trans}(\vec{r}_2)$ -- the  current density of transmitted particles, $\vec{j}_{ref}(\vec{r}_1)$ -- the 
current density of reflected flow and finally $\vec{j}_{inc}(\vec{r}_1)$ -- the  current density of incoming beam. The beams $\vec{j}_{inc}(\vec{r}_1)$ and $\vec{j}_{ref}(\vec{r}_1)$ are measured at an arbitrary  point $\vec{r}_1$ in front of  the barrier and $\vec{j}_{trans}(\vec{r}_2)$ at some point $\vec{r}_2$ behind the barrier. From the continuity equation we deduce that
 \be
\label{301}
 |\vec{j}_{trans}(\vec{r}_2)|=|\vec{j}_{inc}(\vec{r}_1)| - |\vec{j}_{ref}(\vec{r}_1)|
 \ee
 so the sum of coefficients $T+R=1.$
 
 % Figure environment removed
 
Let us look at Pic. \ref{wyk1}. The sketched potential  satisfies the  conditions 
\[
{\rm for}\; x<x_L\; V(x)=0 \;\;\; {\rm and} \;\;\; {\rm for}\; x>x_R\;  V(x)=V_0
\]
where $x_L, x_R$ are the boundaries of the potential barrier. This potential can be treated as one dimensional or living in a more dimensional space but depending exclusively on one Cartesian variable $x.$

We assume that the source of particles is localised at minus infinity and their energy exceeds the value of potential $V_0.$
In both the nonrelativistic and the relativistic case
 at the left -- hand side of the barrier a general solution of the stationary Schroedinger or the Dirac equation is a linear combination of functions
\setcounter{orange}{1}
\renewcommand{\theequation} {\arabic{section}.\arabic{equation}\theorange}
 \be
\label{a}
\Psi(x)= \left[
\begin{array}{c}A_s \\ A_{s-1} \\ \vdots \\ A_0
\end{array}
\right]
 \exp\left( \frac{i{\tt p}x}{\hbar} \right) + 
 \left[
\begin{array}{c}B_s \\ B_{s-1} \\ \vdots \\ B_0
\end{array}
\right]
  \exp\left(- \frac{i{\tt p}x}{\hbar} \right)
 \ee
referring to the eigenvalue of energy $E=\frac{{\tt p}^2}{2m}$ or $E =  \sqrt{{\tt p}^2c^2 + m^2 c^4}$ respectively 
 and at the right -- hand side of the barrier  we obtain
\addtocounter{orange}{1}
\addtocounter{equation}{-1}
 \be
\label{b}
 \Psi(x)=\left[
\begin{array}{c}C_s \\ C_{s-1} \\ \vdots \\ C_0
\end{array}
\right] \exp\left(\frac{i\tilde{\tt p}x}{\hbar} \right), 
 \ee
where $E=\frac{\tilde{\tt p}^2}{2m}+V_0$ or $E =  \sqrt{\tilde{\tt p}^2 c^2 + m^2 c^4}+ V_0$ respectively. 

In the nonrelativistic case the parametre $s$ usually refers to the number of possible projections of spin of the particle. For the $1$ -- D Dirac equation $s=1$ and  is related to the discrete degree of freedom particle -- antiparticle.

\renewcommand{\theequation} {\arabic{section}.\arabic{equation}}

Substituting these expressions into formula \eqref{11} we see that at the right -- hand side of the barrier the total current density decomposes
into the sum of two current densities $\vec{j}_{inc}(x_1)$ and $ \vec{j}_{ref}(x_1)$ determined by functions $ \exp\left( \frac{i{\tt p}x}{\hbar}
\right)$ and $ \exp\left(- \frac{i{\tt p}x}{\hbar} \right)$ respectively. The transmitted current density $\vec{j}_{trans}(x_2)$ follows from exclusively from function
\eqref{b}.

The pure eigenstate of energy $E$ in a basis free form can be decomposed into the sum of four elements  
\be
\label{o1}
|\Psi \big> = |\Psi_{inc} \big> + |\Psi_{ref} \big>+|\Psi_{barrier} \big> + |\Psi_{trans} \big>
\ee
The part $|\Psi_{barrier} \big>$ representing the particle interacting with the barrier does not come directly into the coefficients $R$ and $T.$

Representation of processes of reflection and transmission done with the use of the wave function  of the system is straightforward. Some difficulties appear when we solve the same problem but with the stationary density operator
\[
\hat{\varrho}= | \Psi \rangle \langle \Psi |
\] 
\be
\label{o0}
=\Big(|\Psi_{inc} \big> + |\Psi_{ref} \big>+|\Psi_{barrier} \big> + |\Psi_{trans} \big>\Big)
\Big(\langle \Psi_{inc} | + \langle \Psi_{ref} |+ \langle \Psi_{barrier} | + \langle \Psi_{trans} |\Big).
\ee
The reason is that it  contains  terms referring to  different regions like e.g. $ | \Psi_{ref} \rangle \langle \Psi_{barrier} | $.

Let us see which components of the density operator of the state \eqref{o1}
really
contribute to the current density. We start from the nonrelativistic case so the current density of probability is given by formula \eqref{011}.

At the point  $x_1$ we can see that due to the spatial separation between functions $\langle x|\Psi_{inc}\rangle + \langle
x|\Psi_{ref}\rangle$, $ \langle x|\Psi_{barrier}\rangle$ and $ \langle x|\Psi_{trans} \rangle$ in formula \eqref{011} only four elements
survive. Thus 
\[
\vec{j}(x_1)= \frac{1}{2m} \sum_{k=0}^{s}\Big(  \langle x_1,k| \Psi_{inc} \rangle \langle \Psi_{inc} | \hat{p}| x_1,k \rangle 
+\langle x_1,k|\hat{p} |\Psi_{inc} \big> \langle \Psi_{inc} |  x_1,k \rangle 
\]
\[
+
\langle x_1,k| \Psi_{ref} \big> \langle \Psi_{ref} | \hat{p}| x_1,k \rangle 
+\langle x_1,k|\hat{p} |\Psi_{ref} \big> \langle \Psi_{ref} |  x_1,k \rangle
\]
\[
+
\langle x_1,k| \Psi_{inc} \big> \langle \Psi_{ref} | \hat{p}| x_1,k \rangle 
+\langle x_1,k|\hat{p} |\Psi_{inc} \big> \langle \Psi_{ref} |  x_1,k \rangle 
\]
\be
\label{o2}
+
\langle x_1,k| \Psi_{ref} \big> \langle \Psi_{inc} | \hat{p}| x_1,k \rangle 
+\langle x_1,k|\hat{p} |\Psi_{ref} \big> \langle \Psi_{inc} |  x_1,k \rangle \Big).
\ee
Moreover, the components  $| \Psi_{ref} \big> \langle \Psi_{inc} |$ and $| \Psi_{inc} \big> \langle \Psi_{ref} |$ do
not give a contribution to $\vec{j}(x_1).$ Thus finally
\be
\label{o3}
\vec{j}(x_1)=\vec{j}_{inc}(x_1) + \vec{j}_{ref}(x_1)
\ee
where
\[
\vec{j}_{inc}(x_1):= \frac{1}{2m} \sum_{k=0}^{s}\Big(  \langle x_1,k| \Psi_{inc} \rangle \langle \Psi_{inc} | \hat{p}| x_1,k \rangle 
+\langle x_1,k|\hat{p} |\Psi_{inc} \big> \langle \Psi_{inc} |  x_1,k \rangle \Big)
\]
and
\[
\vec{j}_{ref}(x_1):= \frac{1}{2m} \sum_{k=0}^{s}\Big( \langle x_1,k| \Psi_{ref} \big> \langle \Psi_{ref} | \hat{p}| x_1,k \rangle 
+\langle x_1,k|\hat{p} |\Psi_{ref} \big> \langle \Psi_{ref} |  x_1,k \rangle \Big).
\]
At the point $x_2$ at  the right-hand side  the barrier we obtain that 
\be
\label{o4}
\vec{j}(x_2)=\vec{j}_{trans}(x_2):= \frac{1}{2m} \sum_{k=0}^{s}\Big( \langle x_1,k| \Psi_{trans} \big> \langle \Psi_{trans} | \hat{p}|
x_1,k \rangle
+\langle x_1,k|\hat{p} |\Psi_{trans} \big> \langle \Psi_{trans} |  x_1,k \rangle \Big).
\ee
We get $\vec{j}_{trans}(x_2)$ directly by application of expression \eqref{011} to the density operator \eqref{o0}. 
Summing up  we can see that the current densities $\vec{j}_{trans}(x), \; \vec{j}_{ref}(x)$ and $\vec{j}_{inc}(x)$
are completely determined by the following components of the density
operator: $|\Psi_{inc} \big> \big<\Psi_{inc} |, $
$|\Psi_{ref} \big> \big<\Psi_{ref} | $ and $|\Psi_{trans} \big> \big<\Psi_{trans} |. $ 

The analogous observation is valid for the $1$ -- D Dirac equation.

But technically the way to distinguish between the projection operator  $|\Psi_{inc} \big> \big<\Psi_{inc} | $ and $|\Psi_{ref} \big> \big<\Psi_{ref} | $ is not trivial.  The reason is that the incoming wave and
the reflected one are not mutually orthogonal as
\[
\langle \Psi_{ref}|\Psi_{inc}\rangle \sim 
  - i \exp \left( \frac{2ix_{L}{\tt p}}{\hbar}\right) \,{\rm vp} \frac{1}{\tt  p} + \pi\,\delta({\tt p}). 
 \]
 Therefore an additional procedure leading to separation the total current density \eqref{o2} into $\vec{j}_{inc}(x_1)$ and 
 $\vec{j}_{ref}(x_1)$ is required.
 
 Let us introduce a projection operator
\be
\label{1101}
\hat{I}_{-}:= \sum_{k=0}^s\int_{- \infty}^{x_L} |x,k \rangle dx \langle x,k|.
\ee
Then 
\be
\label{1102}
\hat{I}_{-} \hat{\varrho} \hat{I}_{-}= |\Psi_{ref} \big> \langle \Psi_{ref} | + |\Psi_{inc} \big> \langle \Psi_{inc} |+ |\Psi_{inc}
\big> \langle \Psi_{ref} |+ |\Psi_{ref} \big> \langle \Psi_{inc} | .
\ee
We discuss the nonrelativistic case first.

Since the density operator $\hat{\varrho}$ represents projection on the eigenstate of the Hamilton operator $\hat{H},$ referring to the
eigenvalue $\sqrt{2mE},$  the  limit
\be
\label{1103a}
\lim_{G \rightarrow + \infty}\frac{1}{G} \sum_{k=0}^s \int_{- G}^{+ \infty } \langle x,k | \hat{I}_{-} \sqrt{2m\hat{H}} \hat{\varrho}
\hat{I}_{-} | x, k \rangle dx= 2 {\tt p} \sum_{k=0}^s \Big( |A_k|^2 + |B_k|^2 \Big).
\ee
From the direct wave function considerations we know that 
\be
\label{1.77}
|\vec{j}_{inc}(x_1)|= \frac{2{\tt p}}{m}  \sum_{k=0}^s |A_k|^2 \;\;\; {\rm and}\;\;\; |\vec{j}_{ref}(x_1)|=  \frac{2 {\tt p}}{m}  \sum_{k=0}^s |B_k|^2. 
\ee

Therefore finally the incoming current density of probability equals
\be
\label{1104}
\vec{j}_{inc}(x_1)= \frac{1}{2m}\lim_{G \rightarrow + \infty}\frac{1}{G} \sum_{k=0}^s \int_{- G}^{+ \infty } \langle x,k | \hat{I}_{-}
\sqrt{2m\hat{H}}\hat{\varrho} \hat{I}_{-} | x, k \rangle dx +\frac{\vec{j}(x_1)}{2}
\ee
and the reflected  current density of probability can be calculated as
\be
\label{1105}
\vec{j}_{ref}(x_1)= -\frac{1}{2m}\lim_{G \rightarrow + \infty}\frac{1}{G} \sum_{k=0}^s \int_{- G}^{+ \infty } \langle x,k | \hat{I}_{-}
\sqrt{2m\hat{H}} \hat{\varrho} \hat{I}_{-} | x, k \rangle dx + \frac{\vec{j}(x_1)}{2}.
\ee

 If one  introduces a self -- adjoint operator
\[
\hat{|p|}_{-}:=\frac{1}{2} \left( \hat{I}_{-}\sqrt{2m\hat{H}} + \sqrt{2m\hat{H}} \hat{I}_{-} \right),
\]
then the term \eqref{1103a} represents the main value of $\hat{|p|}_{-}$. Thus alternatively
\be
\label{1104a}
\vec{j}_{inc}(x_1)= \frac{\langle \hat{|p|}_{-} \rangle }{2m} +\frac{\vec{j}(x_1)}{2}
\;{\rm
as \; well \; as}\;\;
\vec{j}_{ref}(x_1)= - \frac{\langle \hat{|p|}_{-} \rangle }{2m} +\frac{\vec{j}(x_1)}{2}.
\ee
Notice that the terms $\frac{\langle \hat{|p|}_{-} \rangle }{2m}$ and $\frac{\vec{j}(x_1)}{2}$ do not depend on a choice of the point $x_1$ if $x_1 < x_L$.

The case of $1$ --D Dirac equation is similar. Instead of operator $ \hat{I}_{-}
\sqrt{2m\hat{H}}\hat{\varrho} \hat{I}_{-}$ in formula \eqref{1103a} we simply put  
\be
\label{1104b}
\frac{1}{c}\hat{I}_{-}
\sqrt{\hat{H}^2-m^2c^4}\,\hat{\varrho} \hat{I}_{-}.
\ee

%%%%%%%%%%%%%%%%%%%%%%%%%%%%%%%%%%%%%%%%%%%%%%%%%%%%%%%%%%%%%%%%%%%%%%%%%%%%%%%%%%%%%%%%%%%%%%%%%%%%%%%%%%%%%%%%%%%%%%%%%%%%%%%%%%%%%%%%%%%%%%%%%%%%%%%%%%%%%%%%%%%%%%%%%%%%%%%%%%%%%%%%%%%%%%%%%%%%%%%%%%%%%%%%%%%%%%%%%%%%%%%%%%%%%%%%%%%%%%%%%%%%%%%%%%%%%%%%%%%%%%%%%%%%%%%%%%%%%%%%%%%%%%%%%%%%%%%%%%%%%%%%%%%%%%%%%%%%%%%
 
 \section{The phase space formulation of quantum theory}

\setcounter{equation}{0}

In this section we remind some basic facts about the phase space quantum mechanics.  We emphasise that this attempt to quantum theory  is
equivalent to its Hilbert space version and can be applied without any references to it. However, since e.g. quantum internal
degrees of freedom do not have their classical counterparts, it is much easier to use the bridge between phase space quantum mechanics
and the Hilbert space formulation in order to construct their phase space representation. Ideas described below were presented in full in
\cite{ przanowski19, tosiek21, przanowski17}.

The quantum system under considerations is modelled on the Hilbert space
$
L^2({\mathbb R}^3) \otimes {\mathbb C}^{s+1}.
$
In the space of square integrable functions $L^2({\mathbb R}^3)$ we have two families of operators: operators of position
$\hat{\vec{r}}$ and operators of momentum $\hat{\vec{p}}.$
Using them we build a family of  unitary operators called displacement operators  
\[
 \widehat{\mathcal{U}}(\vec{\lambda},\vec{\mu}):=\exp\{i(\vec{\lambda}\cdot\hat{\vec{p}}+\vec{\mu}\cdot\hat{\vec{r}}\,)\} 
\]
\be
\label{402}
= \exp\left\{-i \frac{\hbar \vec{\lambda}\cdot
\vec{\mu}}{2}\right\}\exp\{i\vec{\lambda}\cdot\hat{\vec{p}}\,\}\exp\{i\vec{\mu}\cdot\hat{\vec{r}}\,\}
= \exp\left\{i \frac{\hbar \vec{\lambda}\cdot \vec{\mu}}{2}\right\}\exp\{i \vec{\mu}\cdot\hat{\vec{r}}\,\}\exp\{i
\vec{\lambda}\cdot\hat{\vec{p}}\,\}
\ee
where vectors
$\vec{\lambda}, \; \vec{\mu} \in {\mathbb R}^3$,
 and the symbol `$\cdot$' denotes the scalar product.

Alternatively these operators can be written as
\be
\label{215}
\widehat{\mathcal{U}}(\vec{\lambda},\vec{\mu})= \int\limits_{\mathbb{R}^{3}}\exp\{i\vec{\mu}\cdot
\vec{r}\,\}\Big|\vec{r}-\frac{\hbar\vec{\lambda}}{2}\Big> d\vec{r} \,\Big< \vec{r}+\frac{\hbar\vec{\lambda}}{2}\Big|
= \int\limits_{\mathbb{R}^{3}}\exp\{i\vec{\lambda}\cdot \vec{p}\,\}\Big| \vec{p}+\frac{\hbar\vec{\mu}}{2}\Big> d\vec{p} \,\Big<
\vec{p}-\frac{\hbar \vec{\mu}}{2} \Big|,
\ee
where by definition  $d\vec{r}=dxdydz\, , \, d\vec{p}:=dp_x dp_y dp_z$ and the kets are equal to\\
$\Big|\vec{r}-\frac{\hbar\vec{\lambda}}{2}\Big>:= \Big|x-\frac{\hbar\lambda_x}{2}\Big> \otimes \Big|y-\frac{\hbar\lambda_y}{2}\Big>
\otimes
\Big|z-\frac{\hbar\lambda_z}{2}\Big>, \;\; \Big| \vec{p}+\frac{\hbar\vec{\mu}}{2}\Big>:= \Big| p_x+\frac{\hbar \mu_x}{2}\Big> \otimes
\Big| p_y+\frac{\hbar \mu_y}{2}\Big> \otimes \Big| p_z+\frac{\hbar \mu_z}{2}\Big>.$

In the finite dimensional Hilbert space ${\mathbb C}^{s+1}$ we do not have canonically conjugated operators $\hat{\vec{p}}$ and
$\hat{\vec{r}}.$ Thus we start from an arbitrary orthonormal basis $\{|n\rangle\}^{s}_{n=0}$ and build another complete orthonormal
system of vectors $\{|\phi_{m}\rangle\}^{s}_{m=0}$ according to the rule
\[
 |\phi_{m}\rangle\:=\frac{1}{\sqrt{s+1}}\sum_{n=0}^{s}\exp\{i n \phi_{m}\} |n\rangle, \,\,\, m=0,...,s
\]
where the numbers 
$
 \phi_{m}=\frac{2\pi}{s+1}m, \,\,\, m=0,...,s.
$

Applying the  projection operators 
\[
|n\rangle\langle n|,\;n=0,\ldots s \;\;\; {\rm and} \;\;\; |\phi_{m}\rangle\langle \phi_{m}|\; m=0,\ldots s
\]

 we introduce two self -- adjoint operators
\begin{equation}\label{25}
  \hat{n}:=\sum_{n=0}^{s}n|n\rangle\langle n|\,\,\, , \,\,\, \hat{\phi}:=\sum_{m=0}^{s}\phi_{m}|\phi_{m}\rangle\langle \phi_{m}|
\end{equation} 
and then the Schwinger operators 
\begin{equation}\label{26}
  \hat{V}:=\exp\left\{i \frac{2\pi}{s+1}\hat{n}\right\}, \,\,\, \hat{R}:=\exp\{i\hat{\phi}\}.
\end{equation}
With the use of operators $\hat{V}$ and $\hat{R}$ we construct a
 family of unitary operators 
\begin{equation}\label{29}
  \hat{\mathcal{D}}(k,l):=\exp\left\{-i\frac{\pi k l}{s+1}\right\}\hat{R}^{k}\hat{V}^{l}, \,\,\, k,l \in \mathbb{Z}
\end{equation}
parametrised by numbers $k$ and $l$ and playing in the Hilbert space ${\mathbb C}^{s+1}$ the role analogous to the operators \eqref{402}
in the space $L^2({\mathbb R}^3).$

The displacement operators $\hat{\mathcal{U}}(\vec{\lambda},\vec{\mu})$ together with $\hat{\mathcal{D}}(k,l)$ enable us to establish a one -- to -- one
relationship between the Hilbert space version and a phase space formulation of quantum mechanics.
But first we need to propose an appropriate quantum phase space. The counterpart of the Hilbert space $ L^2({\mathbb R}^3) $ is well
known. This is the classical symplectic space ${\mathbb R}^3 \times {\mathbb R}^3$ of the system. The internal states are represented on
the $(s+1) \times (s+1)$ grid being a discrete phase space denoted as $\Gamma^{s+1}$. Thus phase space quantum mechanics is built on the
set
\[
\Gamma={\mathbb R}^3 \times {\mathbb R}^3 \times \Gamma^{s+1} 
\]
 and the coordinates of points belonging to $\Gamma$ are $(\vec{p},\vec{r},\phi_m,n).$

A correspondence between a linear operator $\hat{f}$ acting in the Hilbert space $
L^2({\mathbb R}^3) \otimes {\mathbb C}^{s+1}$ and its respective function $f(\vec{p},\vec{r},\phi_{m},n)$ on the space $\Gamma$ is
established as
\[
f(\vec{p},\vec{r},\phi_{m},n)=\left(\frac{\hbar}{2\pi}\right)^{3}(s+1)^{-1}\sum_{k,l=0}^{s}\int_{\mathbb{R}^{3}\times\mathbb{R}^{3}}d\vec{\lambda}
d\vec{\mu}\left(\mathcal{P}\left(\frac{\hbar \lambda_x\mu_x}{2}, \frac{\hbar \lambda_y\mu_y}{2}, \frac{\hbar
\lambda_z\mu_z}{2}\right)\mathcal{K}\left(\frac{\pi k l}{s+1}\right)\right)^{-1}
\]
\be
\label{231}
\exp\{i(\vec{\lambda}\cdot \vec{p} +\vec{\mu}\cdot \vec{r}\,)\}\exp\left\{ i\frac{2\pi}{s+1}(km+ln)\right\}
\textrm{Tr}\left\{\hat{f}\hat{\mathcal{U}}^{\dag}(\vec{\lambda},\vec{\mu})\hat{\mathcal{D}}^{\dag}(k,l)\right\} 
\ee
and depends on two additional functions $\mathcal{P} \left(\frac{\hbar \lambda_x\mu_x}{2}, \frac{\hbar \lambda_y\mu_y}{2}, \frac{\hbar
\lambda_z\mu_z}{2}\right)$ and $\mathcal{K}\left(\frac{\pi k l}{s+1}\right)$ known as kernels.
On the continuous component ${\mathbb R}^3 \times {\mathbb R}^3$ of the phase space we  choose the Weyl ordering so  we put
\[
\mathcal{P} \left(\frac{\hbar \lambda_x\mu_x}{2}, \frac{\hbar \lambda_y\mu_y}{2}, \frac{\hbar \lambda_z\mu_z}{2}\right)=1.
\]
A decision about kernel $\mathcal{K}\left(\frac{\pi k l}{s+1}\right)$ is not so simple because one cannot say that
$\mathcal{K}\left(\frac{\pi k l}{s+1}\right)=1$ (see \cite{przanowski17}). Since in our further considerations the parameter $s=1,$ the best admissible option seems to be
\[
\mathcal{K}\left(\frac{\pi k l}{s+1}\right)=(-1)^{kl}.
\]
Defining the family of operators called the Fano operators or the Stratonovich -- Weyl quantiser
$$
\widehat{\Omega}[\mathcal{P},\mathcal{K}](\vec{p},\vec{r},\phi_{m},n):=\left(\frac{\hbar}{2\pi}\right)^{3}(s+1)^{-1}\sum_{k,l=0}^{s}\int_{\mathbb{R}^{3}\times\mathbb{R}^{3}}d\vec{\lambda}
\,d\vec{\mu}\,
\mathcal{P} \left(\frac{\hbar \lambda_x\mu_x}{2}, \frac{\hbar \lambda_y\mu_y}{2}, \frac{\hbar
\lambda_z\mu_z}{2}\right)\mathcal{K}\left(\frac{\pi k l}{s+1}\right)
$$
\begin{equation}\label{232}
\times \exp\{-i(\vec{\lambda}\cdot \vec{p} +\vec{\mu}\cdot \vec{r})\}\exp\left\{-i\frac{2\pi}{s+1}(km+ln)\right\}
\hat{\mathcal{U}}(\vec{\lambda},\vec{\mu})\hat{\mathcal{D}}(k,l)
\end{equation}
we may reduce the correspondence \eqref{231} to a compact form
\be
\label{2321}
f(\vec{p},\vec{r},\phi_{m},n)= \textrm{Tr}\left\{ \hat{f} \, \hat{\Omega}[\mathcal{P},\mathcal{K}](\vec{p},\vec{r},\phi_{m},n) \right\}.\ee


A phase space counterpart of the density operator $\hat{\varrho}$ is known as the Wigner function $W(\vec{p},\vec{r},\phi_{m},n).$ For
the choice of kernels proposed above we introduce the Wigner function via formula
\be
\label{2322}
W(\vec{p},\vec{r},\phi_{m},n,t)= \frac{1}{(2\pi\hbar)^{3}(s+1)} \textrm{Tr}\left\{\hat{\varrho}(t)\,
\hat{\Omega}[1,(-1)^{kl}](\vec{p},\vec{r},\phi_{m},n)\right\}.
\ee
Thus the mean value of an observable represented by a function $f(\vec{p},\vec{r},\phi_{m},n)$ equals
\begin{equation}
\label{44}
\langle f(\vec{p},\vec{r},\phi_{m},n) \rangle(t)=\sum_{m,n=0}^{s}\int_{\mathbb{R}^{3}\times\mathbb{R}^{3}}d\vec{p} \,d\vec{r}\,
f(\vec{p},\vec{r},\phi_{m},n) W(\vec{p},\vec{r},\phi_{m},n,t).
\end{equation}
The straightforward consequence of formula \eqref{44} is observation that for normalisable states
\be
\sum_{m,n=0}^{s}\int_{\mathbb{R}^{3}\times\mathbb{R}^{3}}d\vec{p} \,d\vec{r} \,W(\vec{p},\vec{r},\phi_{m},n,t)= \langle 1 \rangle(t) =1.
\ee

 
 The time evolution of Wigner function is given by the Liouville -- von Neumann -- Wigner equation
 \be
\label{x2}
\frac{\partial}{\partial t}W(\vec{p},\vec{r},\phi_{m},n,t) + \{W(\vec{p},\vec{r},\phi_{m},n,t), H(\vec{p},\vec{r},\phi_{m},n,t)\}_M=0, \ee
  where the Moyal bracket is calculated as
 \[
\{W(\vec{p},\vec{r},\phi_{m},n,t), H(\vec{p},\vec{r},\phi_{m},n,t)\}_M
\]
\[
:= \frac{1}{i \hbar} \Big(W(\vec{p},\vec{r},\phi_{m},n,t) \ast H(\vec{p},\vec{r},\phi_{m},n,t)- H(\vec{p},\vec{r},\phi_{m},n,t) \ast
W(\vec{p},\vec{r},\phi_{m},n,t) \Big).
 \]
By $H(\vec{p},\vec{r},\phi_{m},n,t)$ the Hamilton function is denoted. The asterix `$\ast$' symbolises the famous star product. We give its
form for the Weyl ordering and the discrete kernel $\mathcal{K}\left(\frac{\pi k l}{s+1}\right)=(-1)^{kl}$. Then
\[
(f*g)(\vec{p},\vec{r},\phi_{m},n,t)=
\]
$$
\frac{1}{(2\pi\hbar)^{6}(s+1)^{2}}\sum_{m',n', m'',n''=0}^{s}
\int_{\mathbb{R}^{12}} d\vec{p}\,' d\vec{r}\,' d\vec{p}\,'' d\vec{r}\,'' f(\vec{p}\,',\vec{r}\,',\phi_{m'},n',t)
g(\vec{p}\,'',\vec{r}\,'',\phi_{m''},n'',t)
$$
\begin{equation}\label{33}
\textrm{Tr}\left\{\widehat{\Omega}[1,(-1)^{kl}](\vec{p},\vec{r},\phi_{m},n)\widehat{\Omega}[1,(-1)^{kl}](\vec{p}\,',\vec{r}\,',\phi_{m'},n')
\widehat{\Omega}[1,(-1)^{kl}](\vec{p}\,'',\vec{r}\,'',\phi_{m''},n'')\right\}.
\end{equation}
Since we are interested exclusively in the situation when $s=1,$
we simplify formula  \eqref{33} to the expression
\begin{multline}\label{610}
(f*g)(\vec{p},\vec{r},\phi_{m},n,t)=\frac{4}{(2\pi\hbar)^{6}}\sum_{m',n', m'',n''=0}^{1}
\int_{\mathbb{R}^{12}} d\vec{p}\,' d\vec{r}\,' d\vec{p}\,'' d\vec{r}\,'' \\
\times  f(\vec{p}\,',\vec{r}\,',\phi_{m'},n',t) g(\vec{p}\,'',\vec{r}\,'',\phi_{m''},n'',t)
\exp\left\{\frac{2i}{\hbar}[(\vec{r}-\vec{r}\,')\cdot(\vec{p}-\vec{p}\,'')-(\vec{r}-\vec{r}\,'')\cdot(\vec{p}-\vec{p}\,')]\right\} \\
\times
\left\{(1+(-1)^{m'+m''})(1+(-1)^{n'+n''})+(-1)^{m}((-1)^{m'}+(-1)^{m''})+(-1)^{m+n}((-1)^{m'+n'}+(-1)^{m''+n''})\right.\\
+(-1)^{n}((-1)^{n'}+(-1)^{n''})+i\left[(-1)^{m}(-1)^{n'+n''}((-1)^{m'}-(-1)^{m''})\right.\\
\left.\left.+(-1)^{m+n}((-1)^{m''+n'}-(-1)^{m'+n''})+(-1)^{n}(-1)^{m'+m''}((-1)^{n''}-(-1)^{n'})\right]\right\}.
\end{multline}


%%%%%%%%%%%%%%%%%%%%%%%%%%%%%%%%%%%%%%%%%%%%%%%%%%%%%%%%%%%%%%%%%%%%%%%%%%%%%%%%%%%%%%%%%%%%%%%%%%%%%%%%%%%%%%%%%%%%%%%%%%%%%%%%%%%%%%%%%%%%%%%%%%%%%%%%%%%%%%%%%%%%%

\section{The continuity equation in the phase space quantum mechanics}

\setcounter{equation}{0}

In order to find a phase space counterpart of the continuity equation we use the following marginal distribution \cite{przanowski19}
 \[
\sum_{m,n=0}^{s} \int_{{\mathbb R}^3} d\vec{p}\, W(\vec{p},\vec{r}_0,\phi_{m},n,t)= {\rm Tr} \left( \hat{\varrho}(t) |{\vec r}_0\big>
\big<{\vec r}_0|\otimes \hat{\mathbf 1}\right)=\rho({\vec r}_0,t).
 \]
 Therefore in terms of the phase space objects the speed of change of the spatial density of probability equals
 \be
\label{x1}
\frac{\partial }{ \partial t}\rho({\vec r}_0,t)= \sum_{m,n=0}^{s} \int_{{\mathbb R}^3} d\vec{p}\, \frac{\partial }{ \partial
t}W(\vec{p},\vec{r}_0,\phi_{m},n,t).
 \ee
 Since now we will omit the index at the vector ${\vec r}_0.$
 
 Thus the change of spatial density of probability $\rho({\vec r},t)$ with the time expressed by \eqref{x1} combined with the Liouville -- von
Neumann -- Wigner equation \eqref{x2} leads to the relation
 \[
\frac{\partial}{\partial t} \sum_{m,n=0}^{s} \int_{{\mathbb R}^3} d\vec{p} \,W(\vec{p},\vec{r},\phi_{m},n, t) + \sum_{m,n=0}^{s}
\int_{{\mathbb R}^3} d\vec{p} \,\{W(\vec{p},\vec{r},\phi_{m},n,t), H(\vec{p},\vec{r},\phi_{m},n,t)\}_M
 =0.
 \]
Comparing this formula with the continuity equation \eqref{01} we see that
 the term
 \[
\sum_{m,n=0}^{s}
 \int_{{\mathbb R}^3} d\vec{p} \, \{W(\vec{p},\vec{r},\phi_{m},n,t), H(\vec{p},\vec{r},\phi_{m},n,t)\}_M
 \]
 must represent the element $ {\rm div} \vec{j}(\vec{r},t).$ In order to find an explicit expression for the current density or the charge density we need to declare the form of the Hamilton function  $ H(\vec{p},\vec{r},\phi_{m},n,t).$
 
 \subsection{The nonrelativistic current density of probability }
 \label{subsec51}
 
First let us look at a nonrelativistic 
system, which Hamilton operator is of the form \eqref{dod1} with $\hat{T}=\frac{\hat{\vec p\,}^2}{2m}$ an the number of possible internal states $s+1=2.$
 In order to construct a respective Hamilton function on the quantum
phase space ${\mathbb R}^6 \times \Gamma^2$ we need to apply the correspondence rule \eqref{2321} for $s=1.$

Two states referring to the internal degree of freedom we denote by $|0\rangle $ and $|1\rangle.$ They are chosen as eigenstates of 
the internal potential operator $\hat{V}_{\rm int}.$ Thus in the basis $\{|1\rangle , \, |0\rangle \}$ 
\be
\label{nn1}
\hat{V}_{\rm int}= \left[
\begin{array}{cc}
V_{11} & 0 \\
0 & V_{00}
\end{array}
\right]
\ee
where $V_{11}$ and $V_{00}$ are some real numbers.

Using this set of vectors as a basis
of the Hilbert space ${\mathbb C}^2$ we find that the elements
\[
|\phi_0 \rangle= \frac{1}{\sqrt{2}} \left( |0 \rangle + |1 \rangle \right) \;\;\; , \;\;\; |\phi_1 \rangle= \frac{1}{\sqrt{2}} \left( |0
\rangle - |1 \rangle \right) .
\]
Therefore the operators
\be
\label{n1}
\hat{n}=|1 \rangle \langle 1|\;\;\; , \;\;\; \hat{\phi}= \pi |\phi_1 \rangle \langle \phi_1 |
\ee
are proportional to the projection operators on the directions $|1 \rangle $ and $|\phi_1 \rangle$ respectively. From \eqref{26} we
obtain that the Schwinger operators are
\be
\label{n2}
\widehat{V}= |0 \rangle \langle 0 | - |1\rangle \langle 1| \;\;\; {\rm and } \;\;\; \widehat{R}= |\phi_0 \rangle \langle \phi_0 | -
|\phi_1\rangle \langle \phi_1|=
|0 \rangle \langle 1 | + |1 \rangle \langle 0 |.
\ee
Thus we see that the family of unitary operators $\widehat{D}(k,l)$ (see \eqref{29}) consists of four elements
\bea
\label{n3}
\widehat{D}(0,0)= \hat{\mathbf 1}=|0 \rangle \langle 0 | + |1\rangle \langle 1|, & \widehat{D}(1,0)= \widehat{R}=|0 \rangle \langle 1 |
+ |1 \rangle \langle 0 | \nonumber \\
\widehat{D}(0,1)= \widehat{V}= |0 \rangle \langle 0 | - |1\rangle \langle 1|, & \widehat{D}(1,1)= -i \widehat{R} \widehat{V}= i |0
\rangle \langle 1 | -i |1 \rangle \langle 0 | 
\eea
and the discrete part of the Stratonovich -- Weyl quantiser is determined by
\bea
\label{nn5}
\hat{\Omega}[(-1)^{kl}](\phi_0,0)&=&\frac{1}{2} \Big( 2|0 \rangle \langle 0 | +(1-i)  |0
\rangle \langle 1 | +(1+i) |1 \rangle \langle 0 | \Big),  \nonumber \\
\hat{\Omega}[(-1)^{kl}](\phi_1,0)&=&\frac{1}{2} \Big( 2|0 \rangle \langle 0 | +(-1+i)  |0
\rangle \langle 1 | +(-1-i) |1 \rangle \langle 0 | \Big),  \nonumber \\
\hat{\Omega}[(-1)^{kl}](\phi_0,1)&=&\frac{1}{2} \Big( 2|1 \rangle \langle 1 | +(1+i)  |0
\rangle \langle 1 | +(1-i) |1 \rangle \langle 0 | \Big),  \nonumber \\
\hat{\Omega}[(-1)^{kl}](\phi_1,1)&=&\frac{1}{2} \Big( 2|1 \rangle \langle 1 | +(-1-i)  |0
\rangle \langle 1 | +(-1+i) |1 \rangle \langle 0 | \Big).
\eea
The continuous part of the Stratonovich -- Weyl quantiser is given by 
\be
\label{nn51}
\hat{\Omega}[1](\vec{p},\vec{r})=\hbar^3  \int\limits_{\mathbb{R}^{3}}\exp\{-i\vec{\lambda}\cdot
\vec{p}\,\}\Big|\vec{r}-\frac{\hbar\vec{\lambda}}{2}\Big> d\vec{\lambda} \,\Big< \vec{r}+\frac{\hbar\vec{\lambda}}{2}\Big|.
\ee
 
From the rule \eqref{2321} applied to the nonrelativistic Hamilton operator \eqref{dod1} with the internal potential \eqref{nn1} we get
\bea
\label{x3}
H({\vec p},\vec{r}, \phi_{0},0 )  =  \frac{{\vec p\,}^2}{2m} + V_{00} \cdot V({\vec{r}}\,), &&
H({\vec p},\vec{r}, \phi_{1},0 )  =  \frac{{\vec p\,}^2}{2m} +V_{00} \cdot V({\vec{r}}\,), \nonumber \\
H({\vec p},\vec{r}, \phi_{0},1 )  =  \frac{{\vec p\,}^2}{2m} +V_{11} \cdot V({\vec{r}}\,), &&
H({\vec p},\vec{r}, \phi_{1},1 )  =  \frac{{\vec p\,}^2}{2m} +V_{11} \cdot V({\vec{r}}\,).
\eea

 Using linearity of the correspondence \eqref{231} we conclude that the phase space counterpart of Eq. \eqref{nn2} is
 \[
  \sum_{m,n=0}^1 \int_{{\mathbb R}^3} d\vec{p} \, \{W(\vec{p'},\vec{r'},\phi_{m'},n',t), V_{n''\,n''} \cdot V({\vec{r''}}\,) \}_M=
    0.
 \]
Symbols $\vec{p'},\vec{r'},\vec{r''},\phi_{m'},n',n''$ denote auxiliary variables in the $*$ -- product \eqref{610} so the Moyal bracket finally depends on variables $\vec{p},\vec{r},\phi_{m},n.$
  
  Therefore 
  \[
{\rm div} \vec{j}(\vec{r},t)= \sum_{m,n=0}^1 \int_{{\mathbb R}^3} d\vec{p} \, \left\{W(\vec{p},\vec{r},\phi_{m},n,t), \frac{{\vec
p\,}^2}{2m}\right\}_M=
   {\rm div} \frac{1}{m} \sum_{m,n=0}^1  \int_{{\mathbb R}^3} d\vec{p} \,{\vec p\,} \,W(\vec{p},\vec{r},\phi_{m},n,t)
  \]
where divergence is calculated with respect to the spatial coordinates.
  Thus finally the nonrelativistic current density of probability equals
  \be
  \label{1.5}
  \vec{j}(\vec{r},t) = \frac{1}{m} \sum_{m,n=0}^1 \int_{{\mathbb R}^3} d\vec{p} \,{\vec p\,} \,W(\vec{p},\vec{r},\phi_{m},n,t).
  \ee
  A straightforward consequence of formula \eqref{1.5} is the  fact that the average value of momentum is related to the current density of probability by the integral
\[
\langle \vec{p}(t) \rangle = m \int_{{\mathbb R}^3} d \vec{r} \; \vec{j}(\vec{r},t).
\]  



%%%%%%%%%%%%%%%%%%%%%%%%%%%%%%%%%%%%%%%%%%%%%%%%%%%%%%%
%%%%%%%%%%%%%%%%%%%%%%%%%%%%%%%%%%%%%%%%%%%%%%%%%%%%%%%%
 \subsection{The relativistic current density  for the $1$ -- D Dirac equation}
 
This subsection contains derivation of the current density for the  $1$ -- D Dirac equation in the phase space quantum mechanics. The first part of construction is devoted to building the phase space counterpart of the  Hamilton operator
\[
\hat{H} = c \hat{p }  (| 0 \rangle \langle 1 | + | 1 \rangle \langle 0 | ) + mc^2  (| 1 \rangle \langle 1| - | 0 \rangle \langle 0 | )
\]
\be
\label{n5}
+ \hat{V}(x)   \Big(V_{11}| 1 \rangle \langle 1| +V_{10} | 1 \rangle \langle 0|+ V_{01}| 0 \rangle \langle 1 |+V_{00}| 0 \rangle \langle 0 | \Big) .
\ee
 As vectors $|1 \rangle$ and $ |0 \rangle$ we take 
eigenvectors of the Pauli matrix $\sigma_z$  according to formula \eqref{nn4}. Since $\hat{V}_{\rm int}$ is a self -- adjoint operator which in principle does not commute with the operator $\hat{\sigma}_z$, numbers $V_{00}, V_{11} \in {\mathbb R}$ and $V_{01}= \overline{V}_{10}.$

Then we build the operators $ \hat{\Omega}[1](p,x), \;\hat{n},\; \hat{\phi}, \;\widehat{V}, \; \widehat{R}, \;\widehat{D}(k,l), \; \hat{\Omega}[(-1)^{kl}](\phi_m,n)$ exactly as we did in the
nonrelativistic case in Subsection \ref{subsec51}.


After simple although long calculations based on the rule \eqref{2321} we see that the Hamilton function on the quantum phase space ${\mathbb R} \times \Gamma^2$
equals
\be
\label{nn10}
 \begin{array}{rcl}
H(p,x,\phi_0,0) &=& cp - mc^2+V(x)\big( V_{00} + \Re((1+i)V_{01}) \big), \\
 H(p,x,\phi_1,0) &=& -cp - mc^2 +V(x)\big( V_{00} - \Re((1+i)V_{01}) \big), \\
H(p,x, \phi_0,1)& =&  cp + mc^2 +V(x)\big( V_{11} + \Re((1-i)V_{01}) \big),\\
 H(p,x,\phi_1,1) &=&-cp + mc^2 +V(x)\big( V_{11} - \Re((1-i)V_{01}) \big),
\end{array} 
\ee
where the symbol $\Re$ denotes the real part.

Analogously like in the nonrelativistic case we observe that only components of the Hamilton function $H(p,x,\phi_m,n)$ containing $p$  contribute to the current density. From the direct calculus we obtain that
\[
{\rm div} \vec{j}(x,t)=
\]
\be
\label{1.6}
 \frac{\partial }{\partial x } \int_{\mathbb R} dp \, q c \left(W(p,x,\phi_0,0,t) + W(p,x,\phi_0,1,t) -
W(p,x,\phi_1,0,t)- W(p,x,\phi_1,1,t)\right).
\ee
Thus
 the current density for the $1$ -- D Dirac equation equals
\be
\label{1.7}
\vec{j}(x,t)=  q c \int_{\mathbb R} dp \, \left(W(p,x,\phi_0,0,t) + W(p,x,\phi_0,1,t) - W(p,x,\phi_1,0,t)- W(p,x,\phi_1,1,t)\right).
\ee
%%%%%%%%%%%%%%%%%%%%%%%%%%%%%%%%%%%%%%%%%%%%%%%%%%%%%%%%%%%%%%%%%%%%%%%%%%%%%%%%%%%%%%%%%%%%%%%%%%%%%%%%%%%%%%%%%%%%%%%%%%%%%%%%%%%%%%%%%%%%%%%%%%%%%%%%%%%%%%%%%%%%%%%%%%%%%%%%%%%%%%%%%%%%%%%%%%%%%%%%%%%%%%%%%%%%%%%%%%%%%%%%%%%%%%%%%%%%%%%%%%%%%%%%%%%%%%%
  
\section{Examples}

\setcounter{equation}{0}

In this section we discuss a few illustrative examples of stationary quantum systems characterised by the current density of probability.
The common starting point for them is the eigenvalue equation for the Hamilton function
 \be
\label{nn7}
  H(\vec{p},\vec{r},\phi_m,n) * W_E(\vec{p},\vec{r},\phi_m,n)= E W_E(\vec{p},\vec{r},\phi_m,n).
  \ee
  We focus on Hamiltonians do not depending on time so also respective Wigner eigenfunctions are independent from it. 
For physical reasons we look for real valued functions  $W_E(\vec{p},\vec{r})$ assigned to real values of energy $E.$ These requirements imply that for every $E$ the Hamilton function $H(\vec{p},\vec{r})$ and the Wigner eigenfunction $W_E(\vec{p},\vec{r})$ commute
\[
\big\{ H(\vec{p},\vec{r},\phi_m,n), W_E(\vec{p},\vec{r},\phi_m,n)\big\}_{M}=0. 
\]


\subsection{A  nonrelativistic free particle with spin $\frac{1}{2}$}

Let us consider a nonrelativistic free particle living in $3$ -- D space with the spin $\frac{1}{2}.$ 
As the internal states of the particle $|1 \rangle, |0 \rangle$ we choose the eigenstates of the third component of spin applying the convention $|\uparrow \rangle \rightsquigarrow |1 \rangle $ and $|\downarrow \rangle \rightsquigarrow |0 \rangle. $ The  system is modelled on the grid ${\mathbb R}^6 \times \Gamma^{2}.$

In this case the energy all four components of the Hamilton function are equal to 
\[
H(\vec{p},\vec{r}, \phi_{0},0 )  = H(\vec{p},\vec{r}, \phi_{1},0 )=  H(\vec{p},\vec{r}, \phi_{0},1 )= H(\vec{p},\vec{r}, \phi_{1},1 ) = \frac{\vec{p}^{\;2}}{2m}
\]
and the eigenvalue equation \eqref{nn7} is of the form
\be
\label{nn8}
\left(\frac{\vec{p}^{\;2}}{2m}*W_E \right)(\vec{p},\vec{r},\phi_m,n)= \frac{\vec{p}^{\;2}}{2m}W_E(\vec{p},\vec{r},\phi_m,n) +\frac{i \hbar \vec{p}}{2m} \cdot \nabla W_E(\vec{p},\vec{r},\phi_m,n)
-  \frac{\hbar^2}{8m} \Delta W_E(\vec{p},\vec{r},\phi_m,n)
\ee
\[
= E \,W_E(\vec{p},\vec{r},\phi_m,n), \;\;\; m,n=0,1.
\]
The operators $\nabla$ and $\Delta$ act exclusively on spatial coordinates.

Please notice that we deal with four conditions indexed by the parametres $m$ and $n$ but in this example the eigenvalue formula does not mix terms with different values of   $\phi_m$ and $n.$

The degeneration with respect to the momentum and the spin can be observed. As the eigenstates of energy  in the domain ${\mathbb R}^6$ we choose eigenstates of the momentum function $\vec{p}.$ Assuming that   
 $E= \frac{{\tt p}_x^2+{\tt p}_y^2+{\tt p}_z^2}{2m}$ we obtain 
\be
\label{nn9}
W_{{\tt p}_x\, {\tt p}_y\, {\tt p}_z} (\vec{p},\vec{r},\phi_m,n)= \frac{1}{(2 \pi \hbar)^{3}} C_{mn} \delta(p_x- {\tt p}_x) \delta(p_y- {\tt p}_y) \delta(p_z- {\tt p}_z) \;\;\; m,n=0,1
\ee
and the coefficients $C_{11}, C_{12},C_{21},C_{22}$ are arbitrary real numbers fulfilling the condition  
\[
C_{11}+ C_{12}+C_{21} +C_{22}=1.
\]
Eliminating the degeneration with respect to the spin we finally get the components of the upper spin Wigner function are
\[
W_{\uparrow \, {\tt p}_x\, {\tt p}_y\, {\tt p}_z} (\vec{p},\vec{r},\phi_0,0)= W_{\uparrow \, {\tt p}_x\, {\tt p}_y\, {\tt p}_z} (\vec{p},\vec{r},\phi_1,0)=0,
\]
\be
\label{y1}
W_{\uparrow \, {\tt p}_x\, {\tt p}_y\, {\tt p}_z} (\vec{p},\vec{r},\phi_0,1)= W_{\uparrow \, {\tt p}_x\, {\tt p}_y\, {\tt p}_z} (\vec{p},\vec{r},\phi_1,1)=  \frac{1}{2(2 \pi \hbar)^{3}}  \delta(p_x- {\tt p}_x) \delta(p_y- {\tt p}_y) \delta(p_z- {\tt p}_z)
\ee
and the down spin Wigner energy eigenfunction equals
\[
W_{\downarrow \, {\tt p}_x\, {\tt p}_y\, {\tt p}_z} (\vec{p},\vec{r},\phi_0,0)= W_{\downarrow \, {\tt p}_x\, {\tt p}_y\, {\tt p}_z} (\vec{p},\vec{r},\phi_1,0)=  \frac{1}{2(2 \pi \hbar)^{3}}  \delta(p_x- {\tt p}_x) \delta(p_y- {\tt p}_y) \delta(p_z- {\tt p}_z)
\]
\be
\label{y2}
W_{\downarrow \, {\tt p}_x\, {\tt p}_y\, {\tt p}_z} (\vec{p},\vec{r},\phi_0,1)= W_{\downarrow \, {\tt p}_x\, {\tt p}_y\, {\tt p}_z} (\vec{p},\vec{r},\phi_1,1)=0.
\ee
Wigner functions \eqref{y1} and \eqref{y2}, as representing unbound states, are not normalisable.

The current density of probability \eqref{1.5} is constant in the space and at every point  equals
\[
\vec{j}_{\uparrow}(\vec{r}\,)=\vec{j}_{\downarrow}(\vec{r}\,)= \frac{1}{(2 \pi \hbar)^3} \frac{\vec{\tt p}}{m}
\]
and is independent from the orientation of spin.


%%%%%%%%%%%%%%%%%%%%%%%%%%%%%%%%%%%%%%%%%%%%%%%%%%%%%%%%%%%%%%%%%%%%%%%%%%%%%%%%%%%%%%%%%%%%%%%%%%%%%%%%%%%%%%%%%%%%%%%%%%%%%%%%%%%%%%%%%%%%%%%%%%%%%%%%%%%%%%%%%%%%%
\subsection{A $1$ -- D free Dirac particle}
This example is devoted to the eigenstates of the $1$ -- D free Dirac particle.
Since there is  no potential energy, the Hamilton function \eqref{nn10} reduces to 
\be
\label{nn11}
 \begin{array}{rl}
H(p,x,\phi_0,0) = cp - mc^2, &
H(p,x, \phi_0,1) =  cp + mc^2
 , \\
 H(p,x,\phi_1,0) = -cp - mc^2,&
 H(p,x,\phi_1,1) =-cp + mc^2 .
\end{array} 
\ee
The explicit form of the eigenvalue equation \eqref{nn7} consisting of four different formulas is really long so as an illustration we present  one them indexed by the discrete variables $m=1,\; n=1$:
\[
\frac{1}{2}\Big(  mc^2 W_E(p,x,\phi_1,1) - cp  W_E(p,x,\phi_1,1)                                                                                                                                           - cp  W_E(p,x,\phi_1,0) +mc^2 W_E(p,x,\phi_0,1) \Big)
\]
\[
+\frac{i}{2}\Big(  mc^2 W_E(p,x,\phi_1,0) + cp  W_E(p,x,\phi_0,1)                                                                                                                                           - cp  W_E(p,x,\phi_0,0) -mc^2 W_E(p,x,\phi_0,0) \Big)
\]
\[
+\frac{i\hbar}{4} \Big(c \frac{\partial W_E(p,x,\phi_1,1)}{\partial x}+ c \frac{\partial W_E(p,x,\phi_1,0)}{\partial x}
\Big)
+\frac{\hbar}{4} \Big(c \frac{\partial W_E(p,x,\phi_0,1)}{\partial x}- c \frac{\partial W_E(p,x,\phi_0,0)}{\partial x}\Big)
\]
\be
\label{nn30}
=E W_E(p,x,\phi_1,1).
\ee
In order to eliminate degeneration we parametrise the solutions by the momentum ${\tt p}$ and the sign of energy. Therefore $E = \pm \sqrt{c^2 {\tt p}^2 + m^2 c^4  }.$

Since the Wigner eigenfunction of the momentum $p$ does not depend on $x,$ the expression \eqref{nn30} does not contain any terms with $\hbar.$ Moreover, as the Wigner eigenfunction is real, the formula \eqref{nn30} divides in two linear functional equations.

 The  Wigner  energy eigenfunction  is proportional to the Dirac delta and has the form 
\[ 
\left\{
\begin{array}{rcl}
W_{{\tt p} \pm}(p,x,\phi_1,1)& =& \frac{1}{4 \pi \hbar} \delta(p- {\tt p}) \frac{c{\tt p}(c{\tt p}-E_{\pm}+ mc^2)}{(c{\tt p})^2+(E_{\pm}- mc^2)^2}
,
\vspace{0.2cm}\\
W_{{\tt p} \pm}(p,x,\phi_1,0)&= & \frac{1}{4 \pi \hbar}\delta(p- {\tt p}) \frac{(E_{\pm}- mc^2)(E_{\pm}- mc^2-c{\tt p})}{(c{\tt p})^2+(E_{\pm}- mc^2)^2},
\vspace{0.2cm}\\
W_{{\tt p} \pm}(p,x, \phi_0,1)&=&\frac{1}{4 \pi \hbar}\delta(p- {\tt p}) \frac{c{\tt p}(c{\tt p}+E_{\pm}- mc^2)}{(c{\tt p})^2+(E_{\pm}- mc^2)^2},
\vspace{0.2cm}\\
W_{{\tt p} \pm}(p,x,\phi_0,0)& =& \frac{1}{4 \pi \hbar} \delta(p- {\tt p}) \frac{(E_{\pm}- mc^2)(c{\tt p}+E_{\pm}- mc^2)}{(c{\tt p})^2+(E_{\pm}- mc^2)^2}.
\end{array} \right.
\]
Substituting this result into definition \eqref{1.7} we obtain that the relativistic  current density is stationary and equals
\be
\label{nn12}
\vec{j}_{{\tt p} \pm}(x)= \frac{q c^2 {\tt p}}{2 \pi \hbar E_{\pm}}= \pm  \frac{q c^2 {\tt p}}{2 \pi \hbar \sqrt{c^2 {\tt p}^2+m^2 c^4}}
\ee
as expected.
%%%%%%%%%%%%%%%%%%%%%%%%%%%%%%%%%%%%%%%%%%%%%%%%%%%%%%%%%%%%%%%%%%%%%%%%%%%%%%%%%%%%%%%%%%%%%%%%%%%%%%%%%%%%%%%%%%%%%%%%%%%%%%%%%%%%%%%%%%%%%%%%%%%%%%%%%%%%%%%%%%%%%
\subsection{A nonrelativistic scattering of spin $\frac{1}{2}$ particles on  the step potential }
\label{subsection6.3}

In this paragraph we discuss the $3$ -- D  nonrelativistic motion  of quantum particles with spin $\frac{1}{2}$ with the step barrier  $V= V(x) \otimes \hat{\mathbf 1},$
where
\begin{equation}
\label{nn23.0}
    V(x) = \begin{cases} 
     0 & \mbox{for } x < 0,
     \\ V_0 & \mbox{for } x \geq 0 
    \end{cases}
\end{equation}
and $V_0>0.$ Because of the shape of the barrier we assume that particles are moving exclusively in the $x$ direction. Thus all considerations are done like for $1$ -- D case but  the spin exists. This kind of potential does not interact with the internal angular momentum. We focus on the situation when energy of particles in the incoming beam are greater than $V_0.$ 

In principle the problem can be solved exclusively in frames of phase space quantum mechanics. However, since the final result is well known and in order to solve the energy eigenvalue equation  \eqref{nn7} we would have to apply the integral definition \eqref{610} of the star product, we prefer to find the Wigner energy eigenfunction directly from  the formula \eqref{2322}. More explanatory is analysis of its properties.

The beam of scattered particles consists of molecules of the fixed momentum ${\tt p}= \sqrt{2mE}, \; E>V_0$ before the barrier ($x<0$) and ${ \tilde{\tt p}}=\sqrt{2m(E-V_0)}$ over the barrier ($x>0$).
Therefore the respective wave function is given by the expression
\[
\Psi_E(x)= \left[ 
\begin{array}{c}A_1 \\A_0
\end{array}
\right] \psi_E(x) =  \frac{Y(-x)}{2}
\left[ 
\begin{array}{c}A_1 \\A_0
\end{array}
\right] 
\left( \left( 1+ \frac{\tilde{\tt p}}{\tt p}\right) \exp\left( \frac{i{\tt p}x}{\hbar}\right)
+ \left( 1- \frac{\tilde{\tt p}}{\tt p}\right) \exp\left(- \frac{i {\tt p}x}{\hbar}\right) \right)
\]
\be
\label{nn12}
+Y(x) 
\left[
\begin{array}{c}A_1 \\ A_0
\end{array}
\right]  \exp\left( \frac{i\tilde{\tt p}x}{\hbar}\right), \;\;\; |A_1|^2 + |A_0|^2=1.
\ee
In formula \eqref{nn12} we recognise immediately the incoming beam, the transmitted one as well as the reflected one.
The  density operator $\hat{\varrho}$ in the position representation equals
\[
\langle x' |\hat{\varrho} | x \rangle = \left[ 
\begin{array}{cc}|A_1|^2 & A_1 \overline{A_0} \\ \overline{A_1} A_0 & |A_0|^2
\end{array}
\right] \psi_E(x') \overline{\psi_E(x)}.
\]
Thus from \eqref{2322} we get that the components of the Wigner function are
\be
\label{nn13} 
\left\{
\begin{array}{rcl}
W_E(p,x,\phi_0,0)& =& \frac{1}{2 \sqrt{2 \pi} \hbar} (|A_0|^2 + \Re((1+i)\overline{A_0}A_1) ){\cal F}_{\xi} \left[ \overline{\psi_E}\left( x+
\frac{{\xi}}{2}\right) \psi_E\left( x- \frac{{\xi}}{2}\right) \right] \left( \frac{p}{\hbar}\right),
\vspace{0.2cm}\\
W_E(p,x,\phi_1,0)&= & \frac{1}{2 \sqrt{2 \pi} \hbar} (|A_0|^2 - \Re((1+i)\overline{A_0}A_1)){\cal F}_{\xi} \left[ \overline{\psi_E}\left( x+
\frac{{\xi}}{2}\right) \psi_E\left( x- \frac{{\xi}}{2}\right) \right] \left( \frac{p}{\hbar}\right),
\vspace{0.2cm}\\
W_E(p,x, \phi_0,1)&=&\frac{1}{2 \sqrt{2 \pi} \hbar} (|A_1|^2 + \Re((1-i)\overline{A_0}A_1)){\cal F}_{\xi} \left[ \overline{\psi_E}\left( x+
\frac{{\xi}}{2}\right) \psi_E\left( x- \frac{{\xi}}{2}\right) \right] \left( \frac{p}{\hbar}\right),
\vspace{0.2cm}\\
W_E(p,x,\phi_1,1)& =& \frac{1}{2 \sqrt{2 \pi} \hbar} (|A_1|^2 - \Re((1-i)\overline{A_0}A_1)){\cal F}_{\xi} \left[ \overline{\psi_E}\left( x+
\frac{{\xi}}{2}\right) \psi_E\left( x- \frac{{\xi}}{2}\right) \right] \left( \frac{p}{\hbar}\right)
\end{array} 
\right.
\ee
where the Fourier transform is defined as
 \[
  {\cal F}[\phi(z)](t)={\cal F}_z[\phi](t)   := \frac{1}{\sqrt{2 \pi}} \int_{{\mathbb R}} \phi(z) \exp (i z t) dz.
  \]


The fact that the relationship between the wave function $\psi_E(x)$ and the Wigner function \eqref{nn13} is based on the Fourier transform of the product of translated functions, leads to an interesting observation that the value of the Wigner function at a fixed point $(p_0,x_0)$ depends on the values of  wave function $\psi_E(x)$ in the whole space. This effect is caused by different ways in which the wave function and the Wigner function encode the information about the momentum.

Let us  focus at the component 
\[
 \frac{1}{ \sqrt{2 \pi} \hbar} {\cal F}_{\xi} \left[ \overline{\psi_E}\left( x+
\frac{{\xi}}{2}\right) \psi_E\left( x- \frac{{\xi}}{2}\right) \right] \left( \frac{p}{\hbar}\right)=
\]
\[
 \left( 1+\frac{\tilde{\tt p}}{\tt p}\right) \cos \left( \frac{({\tt p}- \tilde{\tt  p})x}{\hbar}\right) \delta(2 p-{\tt p}- \tilde{\tt p})
+ \left( 1-\frac{\tilde{\tt p}}{\tt p}\right) \cos \left( \frac{({\tt p}+ \tilde{\tt p})x}{\hbar}\right) \delta(2 p+{\tt p}- \tilde{\tt p})
\]
\[
+ \frac{1}{ \pi} \left( 1-\frac{\tilde{\tt p}}{\tt p}\right) \sin \left( \frac{({\tt p} + \tilde{\tt p})x-(2p+{\tt p}-\tilde{\tt p})|x|}{\hbar}\right) {\rm vp} \frac{1}{2p+{\tt p}-\tilde{\tt p}}
\]
\[
+ \frac{1}{ \pi} \left( 1+\frac{\tilde{\tt p}}{\tt p}\right) \sin \left( \frac{({\tt p} - \tilde{\tt p})x-(2p-{\tt p}-\tilde{\tt p})|x|}{\hbar}\right) {\rm vp} \frac{1}{2p-{\tt p}-\tilde{\tt p}}
\]
\[
-\frac{1}{4 \pi} \left( 1-\frac{\tilde{\tt p}}{\tt p}\right)^2 \sin\left( \frac{2x(p+{\tt p})}{\hbar}\right) \frac{Y(-x)}{p+{\tt p}} 
-\frac{1}{4 \pi} \left( 1+\frac{\tilde{\tt p}}{\tt p}\right)^2 \sin\left( \frac{2x(p-{\tt p})}{\hbar}\right) \frac{Y(-x)}{p-{\tt p}} 
\]
\be
\label{nn14}
-\frac{1}{2 \pi} \left( 1-\left(\frac{\tilde{\tt p}}{\tt p}\right)^2\right) \cos\left( \frac{2{\tt p}x}{\hbar}\right)\sin\left( \frac{2px}{\hbar}\right) \frac{Y(-x)}{p} 
+ \frac{1}{ \pi} \sin\left( \frac{2x(p-\tilde{\tt p})}{\hbar}\right)  \frac{Y(x)}{p-\tilde{\tt p}}
\ee
of the Wigner function \eqref{nn13}. 

The purely transmitted term (the counterpart of the component $Y(x) \exp\left( \frac{i\tilde{\tt p}x}{\hbar}\right)$ of the incident wave function) is represented by the function 
\[
W_{E\,trans}(p,x)= \frac{1}{ \pi} \sin\left( \frac{2x(p-\tilde{\tt p})}{\hbar}\right)  \frac{Y(x)}{p-\tilde{\tt p}},
\]
the incident part equals 
\[
W_{E\,inc}(p,x)=
-\frac{1}{4 \pi} \left( 1+\frac{\tilde{\tt p}}{\tt p}\right)^2 \sin\left( \frac{2x(p-{\tt p})}{\hbar}\right) \frac{Y(-x)}{p-{\tt p}} 
\]
and the reflected one is
\[
W_{E\,ref}(p,x)=
-\frac{1}{4 \pi} \left( 1-\frac{\tilde{\tt p}}{\tt p}\right)^2 \sin\left( \frac{2x(p+{\tt p})}{\hbar}\right) \frac{Y(-x)}{p+{\tt p}}. 
\]
Other parts of expression \eqref{nn14} origin from interference between the incident element of the wave function \eqref{nn12}, its reflected component and the transmitted one under the Fourier transform.

As one can notice easily, a direct application of formula \eqref{1.5} to calculating components of the current density of probability leads to divergent expressions. To avoid this obstacle we introduce a family of auxiliary functions
for which the integral $\int_{\mathbb R} dp$ is convergent (see  \cite{schwartz65}). For example for the transmitted beam we put 
\[
W_{E \, trans}(p,x, \alpha):= \exp \left( - \alpha |p|\right)W_{E\, trans}(p,x), \;\; \alpha >0.
\]
Of course now the integral  $\int_{\mathbb R}pW_{E \, trans}(p,x, \alpha)  dp$ is convergent for every admissible value of the parametre $\alpha.$ We assume that
\[
\int_{\mathbb R}pW_{E \, trans}(p,x)  dp = \lim_{\alpha \rightarrow 0^+} \int_{\mathbb R} pW_{E \, trans}(p,x, \alpha)dp.
\]
From \eqref{1.5} for the Wigner function \eqref{nn13}  on the grid we get that in the $X$ axis  direction the components of the the current density are
\[
j_{trans}(x)=  \frac{ \tilde{\tt p}}{m}Y(x)\;,\; j_{inc}(x)= \frac{1}{4 } \left( 1+\frac{\tilde{\tt p}}{\tt p}\right)^2  \frac{\tt p}{m}Y(-x),
\]
\be
\label{nn20}
 j_{ref}(x)= -\frac{1}{4 } \left( 1-\frac{\tilde{\tt p}}{\tt p}\right)^2  \frac{\tt p}{m}Y(-x)
\ee
as expected. We remind that $|A_1|^2 + |A_0|^2=1.$

A very interesting question is a contribution of interference terms to spatial density of probability and the current density of probability (see also \cite{tosiek16}). Let us consider two components a wave function 
\be
\label{nn21}
Y(x-a_1)\psi_1(x)Y(b_1-x)\;{\rm and }\; Y(x-a_2)\psi_2(x)Y(b_2-x),
\ee
where the numbers $a_1, b_1, a_2, b_2$ fulfill the requirements 
\[
  - \infty \leq a_1 < b_1 < a_2 <b_2 \leq \infty.
\]
The contribution of these two terms \eqref{nn21} to the total Wigner function equals 
\[
W_{12}(p,x)=\frac{2}{ \sqrt{2 \pi} \hbar} \Re \left\{{\cal F}_{\xi} \left[ \overline{\psi}_1\left( x+
\frac{{\xi}}{2}\right) \psi_2\left( x- \frac{{\xi}}{2}\right) \right] \left( \frac{p}{\hbar}\right) \right\}.
\]
But 
\be
\label{nn22}
\forall \;
n \in {\mathcal N} \;\;\forall \; x \in {\mathbb R}
\int_{\mathbb R} dp \,p^n \, W_{12}(p,x)=0 \;\;\; 
\ee
so the interference terms do not influence neither the spatial density of probability nor the current density. Moreover, this observation does not depend on the shape of the potential barrier. 

And for any arbitrary analytical function $f(p,x)$ on ${\mathbb R}^2$ the term $W_{12}(p,x)$ does not contribute to the mean value $\langle f(p,x) \rangle.$

The role of the interference component of the Wigner function originated from an incident beam and the reflected one requires a separate analysis because the incident wave and the reflected one are defined on the same domain. As before, since an internal degree  is not involved in calculations, we assume that
\be
\label{nn22}
\psi_{E\, inc}(x) \sim Y(b-x) \exp \left( \frac{i {\tt p}x}{\hbar}\right)\; , \; 
\psi_{E\, ref}(x) \sim Y(b-x) \exp \left(- \frac{i {\tt p}x}{\hbar}\right), \;\; b \in {\mathbb R}.
\ee
Therefore for arbitrary $x<b$ one can see that
\[
\forall  \;
n \in {\mathcal N} / \{ 0 \} \;\; 
\int_{\mathbb R} dp  \,p^n \, W_{E \,inc \, ref}(p,x)=0
\]
so this part of the Wigner function does not influence  the current density of probability.
%%%%%%%%%%%%%%%%%%%%%%%%%%%%%%%%%%%%%%%%%%%%%%%%%%%%%%%%%%%%%%%%%%%%%%%%%%%%%%%%%%%%%%%%%%%%%%%%%%%%%%%%%%%%%%%%%%%%%%%%%%%%%%%%%%%%%%%%%%%%%%%%%%%%%%%%%%%%%%%%%%%%%%%%%%%%%%%%%%%%%%%%%%%%%%%%%%%%%%%%%%%%%%%%%%%%%%%%%%%%%%%%%%%%%%%%%%%%%%%%%%%%%%%%%%%%%%%%%%%%%%%%%%%%%%%%%%%%%%%%%%%%%%%%%%%%%%%%%%%%%%%%%%%%%%%%%%%%%%%%%%%%%%%%%%%%%%%%%%%%%%%%%%%%%%%%%%%%%%%%%%%%%%%%%%%%%%%%%%%%%%%%%%%%%%%%%%%%%%%%%%%%%%%%%%%%%%%%%%%%




  



\subsection{The $1$ -- D relativistic step potential}

This last example presents  discussion of motion of a $1$ -- D Dirac particle in the step potential \eqref{nn23.0}. 
From \eqref{n5} we can see that its Hamilton operator equals
\be
\label{nn24} 
\hat{H} = c \hat{p }  (| 0 \rangle \langle 1 | + | 1 \rangle \langle 0 | ) + mc^2  (| 1 \rangle \langle 1| - | 0 \rangle \langle 0 | )
+ \hat{V}(x)   \Big(| 1 \rangle \langle 1| +| 0 \rangle \langle 0 | \Big) 
\ee
with the potential $V(x)$ given by formula \eqref{nn23.0}.

Moreover, we assume that  a source of particles of energy $E>mc^2 + V_0 $ is localised in minus infinity. Therefore in the position representation  the solution of the energy 
eigenvalue equation
\[
 \hat{H} \Psi_E(x) = E \Psi_E(x)
\]
is the function
\[
\Psi_{E}(x)= Y(-x)    \left[ \begin{array}{c}
    \frac{c{\tt p}}{E- mc^2} \\ 1
    \end{array} \right]  \exp\left( \frac{i{\tt p}x}{ \hbar } \right) +  Y(-x)  N_{ref}    \left[ \begin{array}{c}
    \frac{-c{\tt p}}{E- mc^2} \\ 1
    \end{array} \right]  
    \exp\left(- \frac{i{\tt p}x}{ \hbar } \right) 
\]
\be
\label{nn23}
 + Y(x)  N_{trans}  \left[  \begin{array}{c}
    \frac{c \tilde{\tt p}}{E-V_0 - mc^2} \\ 1
    \end{array} \right]
    \exp\left( \frac{i \tilde{\tt p}x}{ \hbar } \right)
    \ee
with 
\[
E = \sqrt{{\tt p}^2 c^2 + m^2 c^4}= \sqrt{\tilde{\tt p}^2 c^2 + m^2 c^4}+ V_0.
\]
Because of the requirement $E>mc^2 + V_0$ we can see that in the analysed case both parametres  ${\tt p}$ and $\tilde{\tt p}$ are real,  positive and $\tilde{\tt p} < {\tt p}$.



The continuity of function $\Psi_E(x)$ at $x=0$ implies that the coefficient
\setcounter{orange}{1} \renewcommand{\theequation}{\arabic{section}.\arabic{equation}\theorange} 
\be
\label{nn23.1}
N_{trans}= 1 + \frac{E}{mc^2}+ \frac{mc^2}{V_0} + \frac{\sqrt{E^2 - m^2 c^4} \sqrt{(E-V_0)^2-m^2c^4}}{mc^2 V_0} - \frac{E^2}{mc^2 V_0}
\ee
and
\addtocounter{orange}{1} \addtocounter{equation}{-1} 
\be
\label{nn23.2}
N_{ref}=N_{trans}-1.
\ee
 \renewcommand{\theequation}{\arabic{section}.\arabic{equation}} 
As it can be checked easily, the coefficient $N_{trans}$ is a monotonic function of energy. It varies from $0$ for $E=V_0 + mc^2$ to $1$ when the
energy tends to infinity.

 In the previous example we proved, that only elements $W_{inc}(p,x), \, W_{ref}(p,x), \, W_{trans}(p,x),$ contribute in the current density. This observation about the lack of contribution of interference terms can be adapted immediately for the interference Wigner function between the spatially separated components. 
 
 But the interference term between the incoming beam and the reflected one requires some analysis. By direct calculation from \eqref{1.7} we see that this part of the total Wigner function indeed does not influence the current density.
Thus only the elements
\setcounter{orange}{1} \renewcommand{\theequation}{\arabic{section}.\arabic{equation}\theorange} 
\be
\label{nn24} 
\left\{
\begin{array}{rcl}
W_{inc}(p,x,\phi_0,0)& =& -\frac{1}{2 \pi} \frac{E-mc^2+c{\tt p}}{E-mc^2}  \sin\left( \frac{2x(p-{\tt p})}{\hbar}\right) \frac{Y(-x)}{p-{\tt p}} ,
\vspace{0.2cm}\\
W_{inc}(p,x,\phi_1,0)&= & -\frac{1}{2 \pi} \frac{E-mc^2-c{\tt p}}{E-mc^2}  \sin\left( \frac{2x(p-{\tt p})}{\hbar}\right) \frac{Y(-x)}{p-{\tt p}} ,
\vspace{0.2cm}\\
W_{inc}(p,x, \phi_0,1)&=&-\frac{1}{2 \pi} \frac{c {\tt p}(E-mc^2+c{\tt p})}{(E-mc^2)^2}  \sin\left( \frac{2x(p-{\tt p})}{\hbar}\right) \frac{Y(-x)}{p-{\tt p}} ,
\vspace{0.2cm}\\
W_{inc}(p,x,\phi_1,1)& =& -\frac{1}{2 \pi}\frac{-c {\tt p}(E-mc^2-c{\tt p})}{(E-mc^2)^2}  \sin\left( \frac{2x(p-{\tt p})}{\hbar}\right) \frac{Y(-x)}{p-{\tt p}} 
\end{array} 
\right.,
\ee
\addtocounter{orange}{1} \addtocounter{equation}{-1} 
\be
\label{nn25} 
\left\{
\begin{array}{rcl}
W_{ref}(p,x,\phi_0,0)& =& -\frac{(1-N_{trans})^2}{2 \pi} \frac{E-mc^2-c{\tt p}}{E-mc^2}  \sin\left( \frac{2x(p+{\tt p})}{\hbar}\right) \frac{Y(-x)}{p+{\tt p}} ,
\vspace{0.2cm}\\
W_{ref}(p,x,\phi_1,0)&= & -\frac{(1-N_{trans})^2}{2 \pi} \frac{E-mc^2+c{\tt p}}{E-mc^2}  \sin\left( \frac{2x(p+{\tt p})}{\hbar}\right) \frac{Y(-x)}{p+{\tt p}} ,
\vspace{0.2cm}\\
W_{ref}(p,x, \phi_0,1)&=&-\frac{(1-N_{trans})^2}{2 \pi} \frac{-c {\tt p}(E-mc^2-c{\tt p})}{(E-mc^2)^2}  \sin\left( \frac{2x(p+{\tt p})}{\hbar}\right) \frac{Y(-x)}{p+{\tt p}} ,
\vspace{0.2cm}\\
W_{ref}(p,x,\phi_1,1)& =& -\frac{(1-N_{trans})^2}{2 \pi}\frac{c {\tt p}(E-mc^2+c{\tt p})}{(E-mc^2)^2}  \sin\left( \frac{2x(p+{\tt p})}{\hbar}\right) \frac{Y(-x)}{p+{\tt p}} 
\end{array} 
\right.
\ee
\addtocounter{orange}{1} \addtocounter{equation}{-1} 
and
\be
\label{nn26} 
\left\{
\begin{array}{rcl}
W_{trans}(p,x,\phi_0,0)& =& \frac{N_{trans}^2}{2 \pi} \frac{E-V_0-mc^2+c\tilde{\tt p}}{E-V_0-mc^2}  \sin\left( \frac{2x(p-\tilde{\tt p})}{\hbar}\right) \frac{Y(x)}{p-\tilde{\tt p}} ,
\vspace{0.2cm}\\
W_{trans}(p,x,\phi_1,0)&= & \frac{N_{trans}^2}{2 \pi} \frac{E-V_0-mc^2-c \tilde{\tt p}}{E-V_0-mc^2} \sin\left( \frac{2x(p-\tilde{\tt p})}{\hbar}\right) \frac{Y(x)}{p-\tilde{\tt p}} ,
\vspace{0.2cm}\\
W_{trans}(p,x, \phi_0,1)&=&\frac{N_{trans}^2}{2 \pi} \frac{c \tilde{\tt p}(E-V_0-mc^2+c\tilde{\tt p})}{(E-V_0-mc^2)^2}  \sin\left( \frac{2x(p-\tilde{\tt p})}{\hbar}\right) \frac{Y(x)}{p-\tilde{\tt p}} ,
\vspace{0.2cm}\\
W_{trans}(p,x,\phi_1,1)& =& \frac{N_{trans}^2}{2 \pi} \frac{-c \tilde{\tt p}(E-V_0-mc^2-c\tilde{\tt p})}{(E-V_0-mc^2)^2}  \sin\left( \frac{2x(p-\tilde{\tt p})}{\hbar}\right) \frac{Y(x)}{p-\tilde{\tt p}} 
\end{array} 
\right.
\ee
\renewcommand{\theequation}{\arabic{section}.\arabic{equation}} 
give contribution to the current density.

From the formula \eqref{1.7} we obtain that
\[
j_{inc}(x)= Y(-x)\frac{2c^2q {\tt p}}{E-mc^2}\;\; , \;\;j_{ref}(x)= -Y(-x)\frac{2c^2q {\tt p}(1-N_{trans})^2}{E-mc^2},
\]
\be
\label{nn27}
j_{trans}(x)= Y(x)\frac{2c^2q \tilde{\tt p}N_{trans}^2}{E-V_0-mc^2}.
\ee
Thus the transmission coefficient
\be
\label{nn28}
T= \frac{|j_{trans}(x_2)|}{|j_{inc}(x_1)|}, \;\; x_1<0< x_2
\ee
increases from $0$ for $E=V_0 + mc^2$ to $1$ for energy tending to infinity as expected.

% Figure environment removed 
The reflection coefficient and the transmission one fulfill the standard relation
\[
T+R=1.
\]

An amazing effect can be observed when  $mc^2 < E $ and $E+ mc^2 <V_0.$ Instead of the probability tending rapidly to $0$ inside the barrier we obtain an oscillating solution representing a wave running to infinity. The solution of $1$ -- D Dirac equation is the function similar to \eqref{nn23}
\[
\Psi_{E}(x)= Y(-x)    \left[ \begin{array}{c}
    \frac{c{\tt p}}{E- mc^2} \\ 1
    \end{array} \right]  \exp\left( \frac{i{\tt p}x}{ \hbar } \right) +  Y(-x)  N_{ref}    \left[ \begin{array}{c}
    \frac{-c{\tt p}}{E- mc^2} \\ 1
    \end{array} \right]  
    \exp\left(- \frac{i{\tt p}x}{ \hbar } \right) 
\]
\be
\label{nn29}
 + Y(x)  N_{trans}  \left[  \begin{array}{c}
    \frac{c \tilde{\tt p}}{E-V_0 - mc^2} \\ 1
    \end{array} \right]
    \exp\left( \frac{i \tilde{\tt p}x}{ \hbar } \right)
    \ee
but now
\[
E = \sqrt{{\tt p}^2 c^2 + m^2 c^4}= - \sqrt{\tilde{\tt p}^2 c^2 + m^2 c^4}+ V_0.
\]
As before the coefficients ${\tt p}, \; \tilde{\tt p} $ are positive. However, this time 
 $\tilde{\tt p}> {\tt p} $ if $V_0>2E.$ Moreover, the coefficient $  \frac{c \tilde{\tt p}}{E-V_0 - mc^2}$ is negative.
 
 From the continuity of function \eqref{nn29} at $x=0$ we obtain exactly the same formulas for the coefficients $N_{trans}$ and $N_{ref} $ as before \eqref{nn23.1}, \eqref{nn23.2}. The range of $N_{trans}$ changes from $2$ for $V_0= E + mc^2$ till $1 +\frac{E}{mc^2}+ \frac{\sqrt{E^2-m^2 c^4}}{mc^2}$ for $V_0 \rightarrow \infty.$
 
 The components of the Wigner function look like \eqref{nn24}, \eqref{nn25}, \eqref{nn26} as well as formulas for the current densities  \eqref{nn27}. But now the transmitted current density is negative and the transmission coefficient \eqref{nn28} may exceed 1. 
% Figure environment removed 
The fact that the transmitted current is negative, changes the relation between the reflection coefficient and the transmission coefficient.
Now 
\[
R-T=1.
\]
This result is called  {\it the Klein paradox}. Its formulation and  detailed discussion  can be found in \cite{Kl29, AH81, WG90, PS98}.
%%%%%%%%%%%%%%%%%%%%%%%%%%%%%%%%%%%%%%%%%%%%%%%%%%%%%%%%%%%%%%%%%%%%%%%%%%%%%%%%%%%%%%%%%%%%%%%%%%%%%%%%%%%%%%%%%%%%%%%%%%%%%%%%%%%%%%%%%%%%%%%%%%%%%%%%%%%%%%%%%%%%%%%%%%%%%%%%%%%%%%%%%%%%%%%%%%%%%%%%%%%%%%%%%%%%%%%%%%%%
\section{Final Remarks}
Since the phase space formulation of quantum mechanics is an approach equivalent to the Hilbert space version of quantum theory, all final predictions obtained in these two models must be identical. What is encouraging, this equivalence can be achieved also with respect to purely nonclassical degrees of freedom. However, mathematical tools involved in considerations are different. The traditional Hilbert space attempt is based on theory of partial differential equations and can be applied even to potentials with discontinuities. The phase space calculations done for scattering use integral equations and generalised functions. Thus seem to be more complex and may require some tricks (see Subsection \ref{subsection6.3}).

The eigenstates considered in the Examples are unbound. Thus their Wigner  eigenfunctions  are not normalisable. Therefore the average value of a function $f(\vec{p},\vec{r},\phi_m,n)$ in such a state determined by a Wigner function $W(\vec{p},\vec{r},\phi_m,n)$ must 
be calculated with the use of the modified formula \eqref{44}
\[
\langle f(\vec{p},\vec{r},\phi_m,n) \rangle = \frac{\sum_{m,n=0}^s\int_{\Omega}d\vec{p} \,d\vec{r}\,
f(\vec{p},\vec{r},\phi_{m},n) W(\vec{p},\vec{r},\phi_{m},n)}{\sum_{m,n=0}^s\int_{\Omega}d\vec{p} \,d\vec{r}\,
 W(\vec{p},\vec{r},\phi_{m},n)}
\]
where the region $\Omega$ expands to ${\mathbb R}^3 \times {\mathbb R}^3.$

An interesting question which arose in our current publication and which will be discussed  in another paper is information contained in the interference components of Wigner function. 





%%%%%%%%%%%%%%%%%%%%%%%%%%%%%%%%%%%%%%%%%%%%%%%%%%%%%%%%%%%%%%%%%%%%%%%%%%%%%%%%%%%%%%%%%%%%%%%%%%%%%%%%%%%%%%%%%%%%%%%%%%%%%%%%%%%%%%%%%%%%%%%%%%%%%%%%%%%%%%%%%%%%%%%%%%%%%%%%%%%%%%%%%%%%%%%%%%%%%%%%%%%%%%%%%%%%%%%%%%%%




\begin{thebibliography}{55555555}



\bibitem{WY31} H. Weyl, {\it The Theory of Groups and Quantum Mechanics}, Methuen, London 1931 (reprinted Dover, New York 1950).

\bibitem{WI32} E. P. Wigner,  {\it Phys. Rev.} {\bf 40}, 749 (1932). 
 
\bibitem{GW46} H. J. Groenewold, {\it  Physica} {\bf 12}, 405 (1946).

\bibitem{MO49} J. E. Moyal, {\it Proc. Camb. Phil. Soc.} {\bf 45}, 99 (1949). 




\bibitem{bayen78} F. Bayen, M.Flato, C. Fronsdal, A. Lichnerowicz and D. Sternheimer, {\it Ann. Phys. NY} {\bf 111}, 61 (1978); F.
Bayen, M.Flato, C. Fronsdal, A. Lichnerowicz and D. Sternheimer, {\it Ann.
Phys. NY} {\bf 111}, 111 (1978).

\bibitem{fedosov94} B. Fedosov, {\it J. Diff. Geom.} {\bf 40}, 213 (1994).

\bibitem{fedosov96} B. Fedosov, {\it Deformation Quantization and Index Theory}, Akademie Verlag, Berlin 1996.

 \bibitem{gracia88}
  J. M. Gracia -- Bond\'{i}a and  J. C. V\'{a}rilly, {\it J. Phys. A: Math. Gen.} {\bf 21}, L879 (1988). 
  
  \bibitem{varilly89}
 J. C. V\'{a}rilly and J. M. Gracia -- Bond\'{i}a, {\it Ann. Phys.}  {\bf 190}, 107 (1989).

\bibitem{YK91}
Y. S. Kim and M. E. Noz, {\it Phase Space Picture of Quantum Mechanics}, World Scientific, Singapore 1991.

\bibitem{FS94}
F. E. Schroeck, Jr., {\it Quantum Mechanics on Phase Space}, Kluwer Academic Publishers, London 1994.

\bibitem{WS01} W. Schleich, {\it Quantum Optics in Phase Space}, Wiley -- VCH, Berlin 2001.

\bibitem{CZ05} C. K. Zachos, D. B. Fairlie and T. L. Curtright (Eds.), {\it Quantum Mechanics in Phase Space}, World Scientific, London
2005.

\bibitem{tat}
W.I. Tatarskij, {\it Usp. Fiz. Nauk} {\bf 139}, 587 (1983).

\bibitem{cd}
M. Hillery, R.F. O'Connell, M.O. Scully and E.P. Wigner, {\it
Phys. Rep.} {\bf 106}, 121 (1984).

\bibitem{lee}
H. W. Lee, {\it Phys. Rep.} {\bf 259}, 147 (1995).

\bibitem{dit}
G. Dito and D. Sternheimer, {\it Deformation Quantization: Genesis, Developments and Metamorphoses}, in {\it Deformation Quantization},
ed. G. Halbout, IRMA Lectures Maths. Theor. Phys., Walter de Gruyter, Berlin 2002, p. 9.

\bibitem{SW07} S. Waldmann, {\it Poisson -- Geometrie und Deformationsquantisierung}, Springer, Berlin 2007 (in German).

 \bibitem{tosiek21} 
J. Tosiek and M. Przanowski, {\it Entropy} {\bf 23}, 581 (2021).

 \bibitem{przanowski14}
 M. Przanowski, P. Brzykcy and J. Tosiek, {\it Ann. Phys.} {\bf 351}, 919 (2014); 
M. Przanowski, P. Brzykcy and J. Tosiek, {\it Ann. Phys.} {\bf 363}, 559 (2015).

\bibitem{przanowski19}
 M. Przanowski, J. Tosiek  and F. J. Turrubiates, {\it Fortschritte der Physik} {\bf 67}, 1900080 (2019).
 
 \bibitem{JK20}
 J. Karwowski, A. Ishkhanyan and A. Poszwa, {\it Theor. Chemistry Accounts} {\bf 139}, 178 (2020).

 \bibitem{przanowski17}  
M. Przanowski and J. Tosiek, {\ it J. Math. Phys.} {\bf 58}, 102106 (2017).
 
 \bibitem{tosiek16}
J. Tosiek, R. Cordero and F. J. Turrubiates, {\it J. Math. Phys.} {\bf 57}, 062103 (2016). 

 \bibitem{schwartz65} L. Schwartz, {\it M\'{e}thodes math\'{e}matiques pour les sciences physiques}, Hermann, Paris 1965 (in French).

\bibitem{Kl29}
O. Klein, {\it Z. Physik} {\bf 53}, 157 (1929).

\bibitem{AH81}
A. Hansen and F. Ravndal {\it Phys. Scr.} {\bf 23}, 1036 (1981).

\bibitem{WG90}
W. Greiner, {\it Relativistic Quantum Mechanics}, Springer -- Verlag, Berlin 1990.

\bibitem{PS98}
P. Strange, {\it Relativistic Quantum Mechanics}, Cambridge University Press, Cambridge 1998.



 



\end{thebibliography}
\end{document}