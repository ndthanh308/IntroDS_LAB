%\documentclass{article}
\documentclass[twocolumn]{aastex631}
%\usepackage{graphicx} % Required for inserting images

%\usepackage{graphicx} % Required for inserting images
\usepackage{amsmath}
\usepackage{amssymb}
\usepackage{hyperref}
\usepackage{soul}
\usepackage{graphicx}
%\usepackage{subfig}
%\usepackage{subcaption}

\newcommand{\vdag}{(v)^\dagger}
\newcommand\aastex{AAS\TeX}
\newcommand\latex{La\TeX}


\begin{document}
\title{Constraining the jet composition of GRB 221009A with the prompt TeV emission limit}
%\author{Cui-Yuan Dai, Xiang-Yu Wang, Ruo-Yu Liu, Bing Zhang}
%\affiliation{School of Astronomy and Space Science, Nanjing University, Nanjing 210093, China}
%\affiliation{Key Laboratory of Modern Astronomy and Astrophysics (Nanjing University), Ministry of Education, Nanjing
%210093, China}
\author[0000-0002-0170-0741]{Cui-Yuan Dai}
\affiliation{School of Astronomy and Space Science, Nanjing University, Nanjing 210093, China; xywang@nju.edu.cn; ryliu@nju.edu.cn}
\affiliation{Key Laboratory of Modern Astronomy and Astrophysics (Nanjing University), Ministry of Education, Nanjing
210093, China}

\author[0000-0002-5881-335X]{Xiang-Yu Wang}
\affiliation{School of Astronomy and Space Science, Nanjing University, Nanjing 210093, China; xywang@nju.edu.cn; ryliu@nju.edu.cn}
\affiliation{Key Laboratory of Modern Astronomy and Astrophysics (Nanjing University), Ministry of Education, Nanjing
210093, China}

\author[0000-0003-1576-0961]{Ruo-Yu Liu}
\affiliation{School of Astronomy and Space Science, Nanjing University, Nanjing 210093, China; xywang@nju.edu.cn; ryliu@nju.edu.cn}
\affiliation{Key Laboratory of Modern Astronomy and Astrophysics (Nanjing University), Ministry of Education, Nanjing
210093, China}

\author[0000-0002-9725-2524]{Bing Zhang}
\affiliation{Nevada Center for Astrophysics, University of Nevada, Las Vegas, NV 89154, USA}
\affiliation{Department of Physics and Astronomy, University of Nevada Las Vegas, Las Vegas, NV 89154, USA}


\begin{abstract}
Recent LHAASO observations of the prompt emission phase of the brightest-of-all-time GRB 221009A  imposes a stringent limit on the flux ratio between the TeV and MeV emissions, $F_{\rm TeV}/F_{\rm MeV}\le   2\times10^{-5}$. Within the framework of internal shocks, we study the internal $\gamma\gamma$ absorption in GRB 221009A by generating a set of synthetic bursts in a simulation that reproduces the observed feature of GRB 221009A. We find that the $\gamma\gamma$ absorption does not lead to an exponential cutoff, but rather a power-law spectrum, consistent with previous works. We further find that the attenuation due to $\gamma\gamma$ absorption alone cannot explain the flux limit ratio of GRB 221009A, suggesting a low ratio of synchrotron self-Compton (SSC) to synchrotron emission outputs. This requires that the magnetic field energy density is much larger than the synchrotron photon energy density so that the SSC flux is greatly suppressed. This indicates that the jet composition of GRB 221009A is likely Poynting-flux-dominated.


%Then we analyze the time-resolved spectra and lightcurve to testify if the early GeV emission can be contributed to secondary photons rather than afterglow, however,no $\gamma$-ray signal is found to be associated with the cascaded emission.
\end{abstract}

%\begin{document}

%\maketitle

\section{Introduction}
The traditional “fireball” model of gamma-ray bursts (GRBs) assumes that a fireball is formed due to the initial energy release in a catastrophic event (see \cite{Kumar2015} for a review). Then the fireball expands under its own thermal pressure and gets accelerated to a relativistic speed. Most of the thermal energy is converted to the kinetic energy
of the outflow. 
A fraction of the kinetic energy is then dissipated in the so-called “internal shocks”, where the internal collisions among the GRB ejecta produce shocks, leading to particle acceleration and producing the prompt GRB.  An alternative model is that the GRB ejecta carries a dynamically important magnetic field
component \citep{Meszaros1997, Lyutikov2003, Narayan2009,zhang2011}, i.e., $\sigma>1$, where $\sigma$ is the ratio between the Poynting flux and
matter (baryonic) flux.  In this model, the GRB radiation is thought to be powered by the
dissipation of the magnetic field energy in the ejecta. 




The most popular emission mechanism of the prompt emission is synchrotron radiation from relativistic electrons accelerated in internal shocks or magnetic dissipation \citep{Meszaros1994,Uhm2014,ZhangBing2014}, which radiate in the magnetic field within the shocked plasma. 
In the optically-thin synchrotron scenario for the prompt emission, the SSC emission produced by the same population of relativistic electrons is naturally expected to generate GeV to TeV gamma-rays \citep{Guetta2003, Gupta2007}.
The intensity of the inverse Compton component depends on the intensity of the magnetic field \citep{Bosnjak2009},   thus the SSC component can be used to diagnose the magnetic field strength and hence the composition of the GRB jet. 

On the other hand, the observed flux of high-energy emission from the SSC component is also sensitive to the  $\gamma\gamma$ opacity, as high-energy gamma-rays suffer from pair production absorption with low-energy photons in the prompt emission.  It is usually thought that the internal absorption leads to an exponential drop in the flux at the high-energy end, and thus a moderate optical depth would largely suppress the high-energy emission flux. 
%If this is the case, it would be hard to observe the SSC component of the prompt emission. 
However,  \cite{Granot2008} argued that, due to the time evolution of the $\gamma\gamma$ opacity, the spectral shape of the $\gamma\gamma$ cutoff in the time-integrated spectrum is not exponential, but rather power-law like. \cite{Aoi2010} studied the time-integrated spectrum from  internal shocks
with a numerical approach and find that a broken power law spectrum is formed in practical observations that integrate emissions from
different internal shocks.  \cite{Hascoet} considered a more realistic
calculation of the $\gamma\gamma$ opacity, which takes into account the time, space and direction dependent
photon field existing in the outflow, and find that the $\gamma\gamma$ opacity shows strong variations for a multiple-pulse GRB. Consequently, the $\gamma\gamma$ cutoff is much closer to a power-law steepening. In this case, the suppression due to  $\gamma\gamma$ opacity may not be extreme, and it may be still possible to observe the SSC component, especially for bright GRBs.

%Previously only loose upper limits on the TeV flux during the prompt emission phase were obtained \citep{HAWC-prompt}. 
The brightest-of-all-time GRB 221009A serendipitously occurred within the field of view of the Large High Altitude Air Shower Observatory (LHAASO) \citep{LHAASO2023}.  GRB 221009A was observed by LHAASO during the main burst phase, yielding a differential flux limit of $\sim 6\times 10^{-8}\,{\rm erg\,cm^{-2}\,s^{-1}}$ (after the EBL correction) at $\sim 1$ TeV from $T_0+220\,\textrm{s}$ to $T_0+230\,\textrm{s}$, where $T_0$ is the trigger time \citep{LHAASO2023}. Compared to the averaged MeV flux during the same period,  the flux ratio is $\bar{R}\equiv F_{\rm TeV}/F_{\rm MeV}\le   2\times10^{-5}$ \citep{LHAASO2023}. This represents the strongest constraint on the flux ratio between TeV and MeV fluxes of GRBs during the prompt emission, thus providing an unprecedented opportunity to study the SSC emission and constrain the magnetic field in the prompt emission phase. 

In this paper, we consider the simplest internal shock scenario to study this problem. By constraining the magnetic field strength with the prompt TeV emission limit in the internal shock model, we test the validity of this model. 
%We perform a numerical simulation of the internal shock and calculate the gamma-ray emission produced in the internal shock scenario.   We will show that the attenuation due
%to $\gamma\gamma$ absorption alone cannot explain this flux limit ratio and a low ratio between the SSC and synchrotron outputs is required. This could then put constraints on the intensity of the magnetic field in the emitting region of the prompt emission of GRB 221009A. 
The paper is organized as follows. In Sect.\ref{sec_internal_shock_modeling}, through the simulation of the internal shock,  we generate a synthetic burst that can reproduce the observed feature of GRB 221009A. Then we study the $\gamma\gamma$ opacity in GRB 221009A in Sect.\ref{sec_opacity}. In Sect.\ref{sec_constrain}, by comparing the SSC emission after the $\gamma\gamma$ absorption with the TeV flux limit of GRB 221009A measured by LHAASO, we put constraints on the intensity of the magnetic field in the emitting region of the prompt emission. In Sect.\ref{mangetic dissipation}, we briefly discuss the implication of the TeV limit for the magnetic dissipation model. Finally, we give discussions and conclusions in Sect.\ref{summary}.




\section{Internal shock simulation of GRB 221009A} \label{sec_internal_shock_modeling}
We model the internal shock process following the approach in \cite{Kobayashi1997}. We assume that the ejection by the central engine lasts for a duration $T$ and consider the energy content of the relativistic outflow is dominated by the kinetic energy. The ejecta shells are then described by a distribution of the Lorentz factor $\Gamma$, width $l$, and mass $m$. Both quantities can vary on a timescale $\Delta t_{\rm c}$ (variability timescale of the central engine). We model the dynamic of these internal shocks via a multiple-shell model where the successive collisions between shells mimic the propagation of shock waves within the relativistic outflow (see Appendix Sect\ref{Asec_internal shock modeling}). After modeling internal shock, we can get a series of photon fields, each of which is emitted isotropically in the comoving frame of the thin spherical shell expanding ultra-relativistically with a finite duration. 
%We consider the fast cooling case, where almost all the energy of electrons accelerated in each merged shell is released instantaneously as radiation. We assume that the energy spectrum for each collision is expressed by the Band function \citep{Band1993}, which is characterized by low-energy power-law index $\alpha$, high-energy power-law index $\beta$, and the break energy $E_{\rm p}$. 

\subsection{Parameters setting} \label{p set}




%% Figure environment removed





% Figure environment removed

In internal shock modeling, we assume the Lorentz factors of relativistic shells following the normal distribution ranging from $100\operatorname{-}1000$ with mean value $\mu = \Gamma_0$ and variance $\sigma = \Gamma_0$. A bulk Lorentz factor of $\Gamma_0=440$ is derived from the deceleration time of the ejecta in GRB 221009A\citep{LHAASO2023}. This value represents the  Lorentz factor of the final merged ejecta after internal shock collisions. Since the main peak of GRB 221009A locates within $T_0+220\operatorname{-}T_0+230\, \rm s$, a choice of $\Gamma_0=440$ during this period is reasonable. We set initial distance $L = c\Delta t_{\rm c}$ between shells to $5l$ and the mass of shells is the same. We fix the duration of the central engine to $10\,\rm s$. We generate a synthetic GRB according to the main features of the observation \citep{GECAM2023} of GRB 221009A: we normalized the mean luminosity during $T_0+220\operatorname{-}T_0+230\, \rm s$ to be $10^{54} \, \rm erg \, s^{-1}$, and use the Band function for the spectrum ($\alpha = -0.75$, $\beta = -2.2$ and $E_{\rm p} = 1\,\rm MeV$) of all pulses \citep{GECAM2023, GBM2023}. During the rising phase of the main burst emission  ($T_0+210\,{\rm s}$ to $T_0+219\,{\rm s}$), when the detector is not saturated,  the shortest variability time is found to be $\Delta t_{\rm v}=0.082\, {\rm s}$~\citep{Liu2023}.  During the main burst, the variability could be shorter, so we consider two cases of variability timescale of the central engine, i.e., case I: $\Delta t_{\rm c}=82 \, {\rm ms}$, and case II: $\Delta t_{\rm c}=10 \, {\rm ms}$.   %The collisions between the shells typically occur at  radii $R_{\rm in}\sim 2\Gamma_0^2 c  \Delta t_{\rm v}=10^{15}\,{\rm cm}\,(\Gamma_0/440)^2 \, ( \Delta t_{\rm v}/0.082\, {\rm s})$. 

%In internal shock scenario, $\Delta t_{\rm c}$ determines the collision radius and the collision radius determines the angular spreading time $\Delta t_{\rm a}$, so $\Delta t_{\rm a} \simeq \Delta t_{\rm c} \simeq R/(2\Gamma^2c)$. In thin shells scenario ($L>l$), $\Delta t_{\rm v}$ is comparable to $\Delta t_{\rm a}$, so we have $\Delta t_{\rm c} \simeq \Delta t_{\rm v}$. During the main burst, the variability could be shorter, so we consider two cases of variability timescale of the central engine, i.e., case I: $\Delta t_{\rm c}=82 \, {\rm ms}$, and case II: $\Delta t_{\rm c}=10 \, {\rm ms}$.




\subsection{Simulating the MeV emission of GRB 221009A} \label{Simulate}
The distribution of merged shells' Lorentz factor (centered at $440$) and the resulting collision radii $R$ are shown in panels (a) and (b) of Fig.\ref{fig1}. The resulting collision radii $R$ is centering at $R\sim 10^{15} {\, \rm cm}$ with a broad wing in case I and centering at $R\sim 10^{14} {\, \rm cm}$ in case II, which is consistent with the analytic estimate $R \sim 2\Gamma_0^2 c  \Delta t_v$. 

Our modeling generates a synthetic GRB which is integrated over the equal-arrival time surfaces (EATS) (see Appendix Sect\ref{Asec_eats}). We use the parameters in Sect.\ref{p set} to reproduce the main spectral features of GRB 221009A \citep{GECAM2023, GBM2023}, which are shown in Fig.\ref{fig1}: (1) in panel (c), we show the spectrum of the synthetic GRB, which is consistent with the observed spectrum (a Band function: $\alpha = -0.75$, $\beta = -2.2$ and $E_{\rm p} = 1\,\rm MeV$). (2) The light curve of the MeV emission is shown in panel (d), from which we can see that the variability time scale $\Delta t_{\rm v}$ of the light curve is comparable to $\Delta t_{\rm c}$ of relativistic shells. The mean flux in $20\operatorname{-}200 \, \rm keV$ between $T_0+220\operatorname{-}T_0+230\, \rm s$ is about $10^{-3} \, \rm erg\, cm^{-2}\, s^{-1}$.

%Distribution of merged shells' Lorentz factor (centered at $440$) and the resulting collision radii $R$ are shown in panels (c) and (d) of Fig.\ref{fig1}. The resulting collision radii $R$ is centering at $R\sim 10^{15} {\, \rm cm}$ with a broad wing in case I and centering at $R\sim 10^{14} {\, \rm cm}$ in case II, which is consistent with the analytic estimate $R \sim 2\Gamma_0^2 c  \Delta t_v$. 



%The absorption ratio $f_{\gamma\gamma}$ is ? in case I() and ? in case II considering the $\gamma\gamma$ opacity.





\section{$\gamma\gamma$ absorption of TeV emission} \label{sec_opacity}
In the internal shock scenario, a series of pulses are generated from collisions. High-energy photons cannot escape from the source because of the pair-creation processes $\gamma\gamma\rightarrow{e^+e^-}$. The absorption of photons comes from both the photons in the same pulse and that in other pulses, and we defined these two kinds of absorption as internal absorption and external absorption, respectively. An important quantity to describe the degree of absorption is the absorption ratio $f_{\gamma\gamma}$, which is defined as the average ratio of the absorbed flux to that without absorption:
\begin{equation}
    f_{\gamma\gamma}(E_{\rm HE}) = \frac{\sum\limits_{k = 1}^{k_{\rm{tot}}} F_{\nu}(E_{\rm HE}, k)\times e^{-\tau_{\gamma\gamma}(E_{\rm HE}, k)}}{\sum\limits_{k = 1}^{k_{\rm{tot}}} F_{\nu}(E_{\rm HE}, k)},
\end{equation}
where $E_{\rm HE}$ is the energy of a high-energy photon, $F_{\nu}(E_{\rm HE}, k)$ is the flux density of the $k$th pulse, $k_{\rm tot}$ is the number of pulses and $\tau_{\gamma\gamma}(E_{\rm HE}, k)$ is the $\gamma\gamma$ optical depth for the $k$th pulse  (see Eq.\ref{eq_in_abs} and Eq.\ref{eq_ex_abs}).

Before dealing with more complex dynamical configurations within the internal shock framework, we first study a simple single-pulse case and two-pulse case to compare with previous studies and understand the underlying physics.

%For a given pulse, the $\gamma \gamma$ opacity mainly depends on the energy density and the degree of anisotropy of local photon field (strong anisotropy leading to a small interaction angel between photons, so the $\gamma\gamma$ cross section is small in strong anisotropy photon field). At large radius, the photons energy density is small and the degree of anisotropy of photon field is strong, so the $\gamma\gamma$ opacity becomes small. Fig.\ref{fig_tau_R} shows how the $\gamma\gamma$ opacity (including both internal and external absorption) and total energy change with collision radius for pulses, in which both $\gamma\gamma$ opacity and total energy decrease as collision radius increases. When $\tau < 0$, photons can escapes from the source almost without absorption.



% % Figure environment removed


\subsection{Single-pulse case}  \label{sect single-pulse}
In single pulse scenario, we assume that photons are emitted isotropically in the comoving frame of a thin spherical shell with a finite duration, which turns on radiation at $R = 10^{15}\, \rm cm$ and turns off at $1.5\times 10^{15}\, \rm cm$ with isotropic luminosity $L_\gamma = 10^{54} \, \rm erg \, s^{-1}$ and Lorentz factor $\Gamma = 440$. In this case, we only need to consider internal absorption within the pulse. The absorption ratio $f_{\gamma\gamma}$ in single-pulse case is calculated by Eq.\ref{eq_in_abs}, which is power-law like (see the black solid line in Fig.\ref{in_ex_result}) instead of exponential. This is in agreement with the result in \cite{Granot2008}, which argues that high-energy photons can escape before the building-up of the target photon field and the time-integrated spectrum is power-law like.



\subsection{Two-pulse case} \label{sect two-pulse}
For two-pulse case, we consider two photon fields: the back (inner) photon field absorbed by the front (outer) photon field. Since we only can receive the photons emitted along the line of sight, and the back photons can never catch the front photons emitted on the line of sight, we do not need consider the absorption of the back photons field to the front photons field. The external absorption of the back photon field can be calculated by Eq.\ref{eq_ex_abs}. We fix the Lorentz factor $\Gamma_{\rm f} = 440$, the total energy $E_{\rm f} = 10^{53} \, \rm erg$ of the front photon field, and the emission radius $R_{\rm b} = 10^{15} \, \rm cm$ of the back photon field. The distance $d_{\rm bf}$ between two photon fields is fixed to $3\times 10^9 \, \rm cm$. Then we vary the emission radius $R_{\rm f}$ of the front photon field and calculate the  $\gamma\gamma$ absorption. The $\gamma\gamma$ opacity is shown in Fig.\ref{in_ex_result} and we find the following results for the external absorption:

%(1) The external absorption in outer region is also weaker than that in inner region like internal absorption.
(1) Even though the spectrum of an individual pulse is power-law like, the external absorption by the other pulse makes the spectrum exponential except for the first received pulse that does not undergo external absorption. This is because the photon field (i.e., the front photon field) has been built up and the absorption is important.

% (2) The absorption of $\rm TeV$ photons in back photon field is strongest when $R_{\rm f}\approx R_{\rm b}$, which can be seen from Fig.\ref{in_ex_result}: (i) if $R_{\rm f} < R_{\rm b}$,  the absorption region is larger than $R_{\rm b}$. Therefore if $R_{\rm f}$ is small, the anisotropy of the front photon field in the absorption region will be strong, and the $\gamma\gamma$ opacity will be small; (ii) if $R_{\rm f} > R_{\rm b}$, then the absorption region is larger than $R_{\rm f}$. So when $R_{\rm f}$ is large, the energy density of the front photon field in the absorption region will be lower, and thus the $\gamma\gamma$ opacity will also be smaller.

(2) The absorption of TeV photons in the background photon field is strongest when $R_{\rm f}\approx R_{\rm b}$, as shown in Fig.\ref{in_ex_result}. (i) If $R_{\rm f} < R_{\rm b}$, the absorption region exceeds $R_{\rm b}$. Therefore, when $R_{\rm f}$ is small, the anisotropy of the front photon field in the absorption region will be significant, resulting in a small $\gamma\gamma$ opacity. (ii) if $R_{\rm f} > R_{\rm b}$, then the absorption region exceeds $R_{\rm f}$, large $R_{\rm f}$ leading to a lower energy density in the absorption region. Consequently, the $\gamma\gamma$ opacity will also be reduced.






% Figure environment removed


\subsection{The $\gamma\gamma$ absorption in internal shocks}
% Figure environment removed
In this section, the model is applied to the dynamical evolution expected in the internal shock framework, where the whole prompt gamma-ray emissions is treated as a collection of multiple sub-pulses (see Sect.\ref{Simulate}).
%We get a power-law like absorption ratio including both internal and external absorption after integrating on EATS, which is shown in Fig.\ref{abs_result}. 
%Absorption is weaker in Case I ($\Delta t_{\rm c} = 82\, \rm ms$) since the whole emission radii are smaller leading to a more compact photon field in case II ($\Delta t_{\rm c} = 10\, \rm ms$).
In Fig.\ref{abs_result}, we show the spectra of the first six observed pulses from simulation in case I ($\Delta t_{\rm c} = 82 \, \rm ms$): the first pulse (pulse 1 in Fig.\ref{abs_result}) does not undergo external absorption and its spectrum is power-law like, but all other pulses have an exponential cutoff at the high energy end because of external absorption, which is consistent with the result in Sect.\ref{sect single-pulse} and Sect.\ref{sect two-pulse}. However, the time-integrated spectrum of all pulses becomes much flatter and resembles a power-law due to the superposition of spectra with different cut-off energies $E_{\rm cut}$ ($\tau_{\gamma\gamma}(E_{\rm cut}) = 1$).  $E_{\rm cut}$ mainly depends on two factors: the energy density of the local photon field and the degree of anisotropy of the photon field. At large radii, the photon energy density is small and the degree of anisotropy of the photon field is strong, so the $\gamma\gamma$ opacity becomes small (see Fig.\ref{fig_tau_R}), leading to a larger $E_{\rm cut}$. The finding that the superposition of spectra with different cut-off energies leads to a power-law spectrum is similar to that in \cite{Aoi2010} (but note that \cite{Aoi2010} did not consider the external absorption).% {\bf If the energy density of local photon field is small due to the intrinsic variability of the outflow, the $\gamma\gamma$ opacity still can be small in relative small radius leading to a larger $E_{\rm cut}$ (and this is the main reason why $\gamma\gamma$ opacity can be different for pulses emission at the same radius, see Fig.\ref{fig_tau_R}). ???} The finding that the superposition of spectra with different cut-off energies leads to a power-law spectrum  is quite similar to that in \cite{Aoi2010} (but note that \cite{Aoi2010} did not consider the external absorption).

% Figure environment removed





In the right panel of Fig.\ref{abs_result}, we show the absorption ratio $f_{\gamma\gamma}$ as a function of photon energy. The absorption is stronger in the case of smaller time variability. The absorption ratio is about $10^{-2}$ at $1 {\, \rm TeV}$ for the case of $\Delta t_c=10 {\, \rm ms}$. 








\section{Constraints on the jet composition in the internal shock scenario} \label{sec_constrain}
The obtained absorption ratio $f_{\gamma\gamma}$ is insufficient to explain the flux ratio $F_{\rm TeV}/F_{\rm MeV}\le   2\times10^{-5}$ if the SSC emission component (i.e., TeV emission) has a comparable flux to that of the synchrotron component (i.e., MeV emission). This suggests a low ratio between the SSC and synchrotron outputs. Below we discuss its implication for the magnetic field intensity in the emitting region. 





We assume that in GRB internal shocks, fractions of $\epsilon_{B}$ and $\epsilon_{e}$ of the shock internal energy are converted into the
energy in the magnetic field and electrons, respectively. It is
usually assumed that the electrons are rapidly cooling, so the
energy density in gamma-ray emission $U_\gamma$ is equal to the
electron energy density $U_e$. The magnetic field is given by:

\begin{equation}
\frac{B^2}{8\pi}=\left(\frac{ \epsilon_{B}}{ \epsilon_{e}}\right)U_\gamma=\left(\frac{ \epsilon_{B}}{ \epsilon_{e}}\right)\frac{L_\gamma}{4\pi
R_{\rm in}^2 c \Gamma_0^2},
\end{equation}
where $R_{\rm in}$ is the radius of the internal shock/dissipation, $L_\gamma$ is the luminosity
in gamma-ray emission and $\Gamma_0$ is the bulk Lorentz factor. In the synchrotron model for the prompt MeV
emission, by use of $h\nu_{
m}=\phi_{\nu}\frac{3hqB}{4\pi m_e c}\gamma_m^2\Gamma$,  one can
derive the Lorentz factor of electrons that radiate at the GRB
peak energy $h\nu_m$~\citep{Wang2009},

\begin{equation}
\label{eq:gammam}
\begin{array}{ll}
\gamma_m &=\left(\frac{4\pi m_e c}{3\phi_\nu hq}\right)^{1/2}
\left(\frac{ \epsilon_{}}{ \epsilon_{B}}\right)^{1/4}\left(\frac{2 L_\gamma}{R^2
c}\right)^{-1/4} \varepsilon_p^{1/2}\\
&=1.5\times 10^3 \left(\frac{ \epsilon_{e}}{ \epsilon_{B}}\right)^{1/4}  \left(\frac{L_{\gamma}}
{3\times10^{53}}\right)^{-1/4} R_{\rm in,15}^{1/2} \left(\frac{h\nu_m}{1\, {\rm
MeV}}\right)^{1/2},
\end{array}
\end{equation}
where $\phi_\nu\simeq0.5$ is the coefficient defined in \cite{Wijers} and $q$ is the electron charge.


The peak energy of the prompt emission during the initial phase of GRB 221009A is ${\sim 1}$~MeV \citep{GCN-KW}, so we take $h\nu_m\sim 1\, {\rm MeV}$. The KN limit introduces a new critical Lorentz factor $\hat\gamma_m$, defined as $\hat\gamma_m\equiv \Gamma m_e c^2/h\nu_m$, corresponding to the critical electrons that up-scattering photons at energy $h\nu_m$ in the Thompson scattering regime \citep{Nakar2009}. We obtain $\hat\gamma_m=220 (\Gamma_0/440) (h\nu_m/ 1\, {\rm MeV})^{-1} $.  Unless $ \epsilon_{e}\le 10^{-4} \epsilon_{B}$, which is unreasonable in
terms of the burst energetics, we have $\gamma_m>\hat\gamma_m$, so the IC scattering between
$\gamma_m$-electrons and the bulk of the gamma-ray emission should
be in the Klein-Nishina (KN) regime. In the KN regime,  the  SSC spectrum is given by \cite{Nakar2009}:

\begin{equation}
F_{\nu}= \left\{
\begin{array}{lll}
\nu^{-\frac{1}{2}},  \,\,\,\, 2\gamma_c^2\nu_c<\nu<2\nu_m\gamma_m{\hat\gamma_m} \\
\nu^{-p+\frac{1}{2}}, \,\,\,\, \nu> 2\nu_m\gamma_m{\hat\gamma_m}.  \\
\end{array}
\right.
\end{equation}
The SSC emission flux peaks at
\begin{equation}
\begin{array}{ll}
\varepsilon_p^{\rm IC}&=2\nu_m\gamma_m{\hat\gamma_m}=2\Gamma_0\gamma_m m_e c^2\\
&=0.7\,{\rm TeV}\, \left(\frac{L_{\gamma}}
{3\times10^{53}}\right)^{-1/4} \left(\frac{\Gamma_0}{440}\right) \left(\frac{ \epsilon_{e}}{ \epsilon_{B}}\right)^{1/4} R_{\rm in,15}^{1/2} \left(\frac{\varepsilon_p}{1\, {\rm
MeV}}\right)^{1/2},
\end{array}
\end{equation}
where we have used Eq.\ref{eq:gammam}. Thus the SSC spectral peak is expected to be located within or close to the observed energy range of LHAASO.  Defining $f_{\rm spec}$ as a suppression factor due to SSC peak deviation from the LHAASO energy range,  we expect $f_{\rm spec}\ge 0.3$ in a conservative way.

In the case of $\gamma_m>\hat\gamma_m$, the SSC to synchrotron energy output is well approximated by \cite{Nakar2009}:
\begin{equation}
\label{Y}
Y=\frac{L_{\rm SSC}}{L_{\rm syn}}=\frac{ \epsilon_{e}}{ \epsilon_{B}}f_{\rm KN}\sim (p-2)\frac{ \epsilon_{e}}{ \epsilon_{B}}\left(\frac{\gamma_m}{\hat\gamma_m}\right)^{-1/2},
\end{equation}
where the suppression due to the KN effect is $f_{\rm KN}=(p-2)\left(\frac{\gamma_m}{\hat\gamma_m}\right)^{-1/2}\sim 0.1$ for typical parameter values. 



During $T_0+220\,{\rm s}$ to $T_0+230\,{\rm s}$,  the flux ratio between TeV flux and the MeV flux is $\bar{R}\equiv F_{\rm TeV}/F_{\rm MeV}\le 2\times 10^{-5}$.
Considering the $\gamma\gamma$ absorption, the SSC to synchrotron luminosity ratio should be $Y\le \bar{R}/(f_{\rm spec}f_{\gamma\gamma})$. From Eq.\ref{Y}, we obtain roughly
\begin{equation}
\left\{
\begin{array}{l}
    \epsilon_{B}\ga 50 \epsilon_{e} , \,\rm case\, I \, (\Delta t_{\rm c} = 82\, \rm ms)\\
    \epsilon_{B}\ga 30 \epsilon_{e} , \,\rm case\, II \, (\Delta t_{\rm c} = 10\, \rm ms)
\end{array}
\right.
\end{equation}
for typical value of $p=2.2\operatorname{-}2.5$. This means that the magnetic field energy density is much larger than the electron energy density, so the SSC flux can be greatly suppressed. This is inconsistent with the usual assumption that energy equipartition between the ion, electron and magnetic energy is obtained at the internal shocks \citep{Murphy2010}. $ \epsilon_{B}\gg \epsilon_{e}$  implies a small $\epsilon_e$ for GRB 221009A, which leads to a small radiation efficiency for internal shock emission. Therefore, the prompt TeV limit imposed by LHAASO observations seems to challenge the standard internal shock model for GRB 221009A. 



\section{The magnetic field dissipation model} \label{mangetic dissipation}

In the above, we have assumed a matter-dominated composition, where the internal shock scenario applies.  $\epsilon_{B}\gg \epsilon_{e}$ points to a magnetic-field dominated composition of the jet. If the GRB ejecta carries a dynamically important magnetic field
component, i.e., $\sigma>1$,  the GRB radiation is thought to be powered by the
dissipation of the magnetic field energy in the ejecta \citep{Meszaros1997, Lyutikov2003, Narayan2009}.


In the Internal-Collision-Induced Magnetic Reconnection and Turbulence (ICMART) model \citep{zhang2011}, magnetically-dominated winds are injected from the central engine intermittently. At first, the field lines are ordered and have the same orientation, so the reconnection is suppressed. But the collisions trigger an “avalanche” of magnetic reconnection/turbulence events, causing efficient radiation. So the ICMART model needs multiple collisions to dissipate magnetic energy and cause radiation, which tends to occur at larger radii. In the ICMART scenario, we consider a single main pulse emitted with an observed duration of about $5\,\rm s$). We assume that the pulse turns on radiation at $\sim R_{\rm ICMART} = 10^{16} \, \rm cm$ or $10^{17} \, \rm cm$, respectively, and other parameters are the same as those in the back pulse of Fig.\ref{in_ex_result}. In this case, we find the TeV absorption ratio decreases significantly, with a value of $f_{\gamma\gamma} = 0.34$ and $0.82$ for $R_{\rm ICMART} = 10^{16}$ and $10^{17} \, \rm cm$, respectively, at 1 TeV.  

The suppression ratio due to the KN effect is  $f_{\rm KN}\propto R^{-1/4}$ according to Eq.\ref{eq:gammam} and $f_{\rm KN}\propto \gamma_m^{-1/2}$. We obtain $f_{\rm KN} \simeq 0.06$ and $0.03$ for $R_{\rm ICMART}=10^{16}\, \rm cm$ and $10^{17}\, \rm cm$, respectively. In the ICMART scenario, the gamma-ray luminosity is $L_\gamma=L_w \eta \epsilon_e$, where   $L_w$ is the total isotropic luminosity of the wind and $\eta$ is the energy dissipation efficiency \citep{zhang2011}. Then the energy density of photons in the comoving-frame of the jet is $U'_{ph}=\frac{L_\gamma}{4\pi R^2 \Gamma^2 c}$ and the energy density of the magnetic field is $U'_B=\frac{L_w}{4\pi \Gamma^2 R^2 c} \frac{\sigma}{1+\sigma}$. Then we have:
\begin{equation} \label{Y_ICMART}
    \frac{L_{\rm SSC}}{L_{\rm syn}} = \frac{U'_{ph}}{U'_B}f_{\rm KN} = \frac{L_w\eta \epsilon_e}{L_w\left( \frac{\sigma}{1+\sigma} \right)} f_{\rm KN}.
\end{equation}
Using $\bar{R}\equiv F_{\rm TeV}/F_{\rm MeV}\le 2\times 10^{-5}$ and assuming $\epsilon_e = 0.1$, we get $\eta\le 0.033$ for $R_{\rm ICMART} = 10^{16} \, \rm cm$  and $\eta\le 0.027$ for  $R_{\rm ICMART} = 10^{17} \, \rm cm$  when $\sigma\gg 1$. 

Note that the above estimate of the efficiency $\eta$ applies only to $T_0+220\operatorname{-}T_0+230\, \rm s$, during which about $1/3$ of the whole prompt radiation energy is radiated according to prompt emission observations \citep{GECAM2023, GBM2023}. The efficiency at later time could be higher as $\sigma$ decreases. In addition, the above calculation has assumed that the property of the emitting region of ICMART is the same as that in the internal shock, which is not the case. In the ICMART model, the observed flux is the superposition of the many mini-jets with different orientations.  Each reconnection event is a fundamental mini-jet in the ICMART model, and the direction of the mini-jets can be isotropic \citep{Zhang2014, Shao2022}. This would lead to a larger interaction angle for high-energy photons and a stronger $\gamma\gamma$ absorption. Therefore the absorption factor $f_{\gamma\gamma}$ could be lower than the above estimate and the inferred radiation efficiency will increase correspondingly.   Detailed calculation of the absorption ratio including mini-jet structure is beyond the scope of this work.

%However, the radiation energy in $T_0+220\operatorname{-}T_0+230\, \rm s$ is about $1/3$ of the whole prompt radiation energy according to observation \citep{GECAM2023, GBM2023}, which means nearly $1/3$ of magnetic energy has dissipation in this stage. So $\eta = 1/(1+\sigma) \sim 1/3$, which is larger than the above calculation. This is because we ignore the mini-jet structure in the ICMART model: each reconnection event is a fundamental mini-jet in the ICMART model, and the direction of the mini-jet can be isotropic \citep{Zhang2014, Shao2022}. This leads to a larger interaction angle and stronger absorption. In this case, low flux ratio $\bar{R}$ should be caused by small $f_{\gamma\gamma}$ instead of small $\eta$. However, detailed calculations including mini-jet structure are beyond the range of this work.





%Radiation efficiency will be higher in larger radius in the ICMART model as the magnetic field energy is continuously transformed into radiation energy, but the TeV flux in the later stage is dominated by the early afterglow emission \citep{LHAASO2023}, making it difficult to constrain $\sigma$ at later stage.



%So in ICMART model, the $\gamma\gamma$ absorption are weaker due to larger emission radius. Since $f_{\rm KN}$ is insensitive with radius (see Eq.\ref{eq:gammam} and Eq.\ref{Y}), the jet is still magnetically dominated in radiation region in ICMART model.

\section{Conclusions and Discussions} \label{summary}
\cite{LHAASO2023} estimated the $\gamma\gamma$ optical depth in the prompt emission of GRB 221009A to be $\tau_{\gamma\gamma}\sim 190$ assuming an isotropic radiation field in the single-zone model.  
In this work, we find that although the optical depth is large, it does not lead to exponential attenuation of TeV photons in the framework of the internal shock scenario. We find that the superposition of spectra of different pulses with different cut-off energy results in a power-law-like spectrum for the time-integrated emission (see Fig.\ref{abs_result}). The absorption ratio is only about $\sim 10^{-2}$ for TeV photons, assuming a bulk Lorentz factor of $440$ and the variability time in the range of $10\operatorname{-}82 {\, \rm ms}$. This absorption ratio cannot explain the low  TeV flux limit of GRB 221009A and requires a low ratio between the SSC and synchrotron emission outputs. Then we find that  $\epsilon_{B}\ga 50 \epsilon_{e}$ in case I ($\Delta t_{\rm c} = 82\, \rm ms)$ and $\epsilon_{B}\ga 30 \epsilon_{e}$ in case II ($\Delta t_{\rm c} = 10\, \rm ms)$) for GRB 221009A.

The inference  $\epsilon_{B}\gg \epsilon_{e}$ is inconsistent with the usual assumption that energy equipartition between the ion, electron and magnetic energy is obtained in internal shocks \citep{Murphy2010}. $\epsilon_{B}\gg \epsilon_{e}$  also leads to a small radiation efficiency for internal shock emission. On the other hand, $\epsilon_{B}\gg \epsilon_{e}$ seems to point to a Poynting-flux dominated outflow in which the radiation is powered by the magnetic dissipation. We discussed the possibility of interpreting the prompt emission of GRB 221009A with the ICMART model. The $\gamma\gamma$ absorption could be larger due to the configuration of mini-jets, although a detailed calculation is needed to verify this. If this is the case, then a mild magnetization with $\sigma \sim {\rm a \, few}$ would be sufficient to explain the TeV limit and also lead to a high radiation efficiency. 














\appendix \label{Asec_appendix}
%\section*{APPENDIX} \label{Asec_appendix}
\setcounter{equation}{0}
\renewcommand\theequation{A\arabic{equation}}
\subsection{Internal shock modeling} \label{Asec_internal shock modeling}
We model the internal shock process following the approach in \cite{Kobayashi1997}. At $t = 0$, all the shells are at initial positions $R_i$ and characterized by following parameters: width $l_i$, mass $m_i$, and the Lorentz factor $\Gamma_i$. So the distance $L_i$ between shell $i$ and shell $i+1$ is $R_i-R_{i+1}-l_{i+1}$. The internal collisions keep to occur until all the shells merge into one shell or  the velocities of all inner shells are  slower than that of outer shells. We note that the relative value of the mass of shells determines the dynamic evolution rather than the absolute value. The absolute value of the mass of shells only determines the total energy of radiation. For the $j$th collision, we can calculate and record the following physical quantities in the source frame: collision radius $r_j$, collision time $t_j$, duration time $\delta t_j$ of emission and energy $E_j$ of emission and Lorentz factor of the merged shell $\Gamma_{{\rm m},j}$. By recording these quantities, we can calculate both internal and external absorption. 

\subsection{Internal absorption} \label{Asec:_internal absorption}
As mentioned above, photons are emitted isotropically in the comoving frame with finite duration by a thin spherical shell. We use numerical method to calculate $\gamma \gamma$ opacity as in \cite{Hascoet}:
\begin{equation}
\begin{array}{ll} \label{eq_abs}
    \tau_{\gamma\gamma}(E_{\rm{HE}}, j, k) = \sum\limits_{k'=1}^k \frac{\sigma_{\rm{T}} {\epsilon}_{{\rm{rad}},jk'}}{4\pi R_{jk'}^2\Gamma_{jk'}E_{{\rm{p}},jk'}'} \\
    \times \int \, d\ell \, \mathcal{F} [\ell; E_{\rm{HE}}, R_{jk}, t_{jk}; R_{jk'}, t_{jk'}, E_{{\rm{p}}, jk'}', {\mathcal{B}_{jk'}}].
\end{array}
\end{equation}
Here subscript $j$ indicates the quantities of the $j$th collision. We divide the duration of the $j$th emission into a series of time grids so that in every grid the emission can be approximated as a “flash” instead of continuous radiation. $t_{jk}$, $R_{jk}$, $\Gamma_{jk}$, $\epsilon_{{\rm{rad}}, jk}$ and $E'_{{\rm{p}}, jk}$ $(k=1,\cdots, k_{\rm{max}})$ are the emission time, emission radius, Lorentz factor, total radiation energy and spectra peak energy of the $k$th flash in the $j$th emission. The emission turns on at $t_{j1} = t_j$ and turns off at $t_{jk_{\rm{max}}} = t_j+\delta t_j$, where $t_j$ and $\delta t_j$ is the collision time and duration of emission of $j$th collision. We assume $\Gamma_{jk}$, $E'_{{\rm{p}},jk}$, normalized spectra $\mathcal{B}_{jk'}$ ($\int_0^{\infty} \mathcal{B}(x)\,dx = 1$, $x=E_{\rm HE}/E_{\rm p}$) and luminosity $L_{{\rm{m}},j}$ ($L_{{\rm{m}},j} = \sum \limits_{k'=1}^{k_{\rm max}} \epsilon _{{\rm rad},jk'}/{\delta t_j}$) are constant during the shell spreading in single collision event. All quantities in Eq.\ref{eq_abs} are in source frame besides $E'_{{\rm{p}}, jk}$, which is in comoving frame, and normalized spectra $\mathcal{B}$ are the same in source frame and comoving frame.

The internal $\gamma \gamma$ opacity of the $j$th pulse can be written as:

% \begin{equation}
%     e^{-\bar{\tau}_{\gamma \gamma}(E_{\rm{HE}}, j)} = \frac{\sum\limits_{k = 1}^{k_{\rm{max}}} F_{\nu}(E_{\rm HE}, j, k)\times e^{-\tau(E_{\rm HE}, j, k)}}{\sum\limits_{k = 1}^{k_{\rm{max}}} F_{\nu}(E_{\rm HE}, j, k)}
% \end{equation}

\begin{equation} \label{eq_in_abs}
    \tau_{\gamma \gamma,\rm in}(E_{\rm{HE}}, j) = -ln\left[ f_{\gamma\gamma}(E_{\rm HE}) \right] = -ln \left[ \frac{\sum\limits_{k = 1}^{k_{\rm{max}}} F_{\nu}(E_{\rm HE}, j, k)\times e^{-\tau_{\gamma\gamma}(E_{\rm HE}, j, k)}}{\sum\limits_{k = 1}^{k_{\rm{max}}} F_{\nu}(E_{\rm HE}, j, k)} \right].
\end{equation}


All the quantities requiring astrophysical input are $R_j$, $\delta t_j$, $L_{{\rm{m}},j}$, $\Gamma_{{\rm m},j}$ and photon index of spectra to calculate $\tau_{\gamma \gamma,\rm in}$. $R_j$, $\delta t_j$, $L_{{\rm{m}},j}$ and $\Gamma_{{\rm m},j}$ can be calculated from internal shock modeling. The Photon index should be determined by observation and we assume the photon index of the spectrum is the same for all collisions.

\subsection{External absorption} \label{Asec: external absorption}
In order to calculate $\gamma \gamma$ opacity between different shells and reduce computing time, we make an approximation that every emission of collision is a flash. If the shell's distance $L_i$ is larger than the shell's width $l_i$, then the pulse width $\delta t_j$ is determined by angular spreading time instead of hydrodynamic time \citep{Kobayashi1997}. So flash approximation is effective since we assume $L_i > l_i$ in the internal shock modeling. By using this approximation, we can calculate the external $\gamma \gamma$ opacity of the $j$th pulse:

\begin{equation} \label{eq_ex_abs}
\begin{array}{ll}
    \tau_{\gamma\gamma, \rm ex}(E_{\rm{HE}}, j) = \sum\limits_{j'} \frac{\sigma_{\rm{T}} {\epsilon}_{{\rm{rad}},j'}}{4\pi R_{j'}^2\Gamma_{j'}E_{{\rm{p}},j'}'} \\
    \times \int \, d\ell \, \mathcal{F} [\ell; E_{\rm{HE}}, R_{j}, t_{j}; R_{j'}, t_{j'}, E_{{\rm{p}}, j'}', {\mathcal{B}_{j'}}].
\end{array}
\end{equation}

All the pulses outside of the $j$th pulse should be included in Eq.\ref{eq_ex_abs}, since only photons from which outside of the $j$th pulse can catch the photons from the $j$th pulse in the external scenario as that mentioned in Sect.\ref{sect two-pulse}. The quantities requiring astrophysical input in Eq.\ref{eq_ex_abs} are collision radius $R_j$, collision time $t_j$, Lorentz factor of merged shell $\Gamma_{{\rm m},j}$, energy of emission $\epsilon_{{\rm rad}, j}$ and spectrum $\mathcal{B}_{j}$, all have been calculated in the internal shock modeling besides $\mathcal{B}_{j}$, which has been determined by observation as mentioned above.












% Figure environment removed


\subsection{Integrating the flux on EATS} \label{Asec_eats}

The radiation caused by single collision can be approximated as a flash as mentioned above. For simplicity, we first integrate the flux on the equal-arrival time surface of a single “flash” and ignore the difference between the source frame and observer frame. Fig.\ref{Ashow_fig} shows a single spherical flash emitted from the $j$th collision. The number of photons emitted between $\theta \sim \theta + d\theta, \nu \sim \nu+d\nu$ received by detector at point O in Fig.\ref{Ashow_fig} with area $dA$ in source frame is:

\begin{equation} \label{eq_dN}
    dN_{\nu} = f_{\nu}dSd\Omega d\nu,
\end{equation}

where $f_{\nu}$ is the photon flux emitted from the annular area of the sphere, $dS$ is the annular area between $\theta \sim \theta + d\theta$ on the sphere, and $d\Omega = dA/(D)^2$ is the solid angle of $dA$ seen by the location of radiation, where $D$ is the proper distance, all of these quantities are in source frame. Eq.\ref{eq_dN} can also be expressed in comoving frame: $dN'_{\nu '} = f'_{\nu '}dS'd\Omega ' d\nu'$. Using Lorentz transformation: $dN_{\nu} = dN'_{\nu'}$, $dS = dS'$, $d\Omega = d\Omega '/{\delta_{\rm D}^2}$ and $d\nu = \delta_{\rm D} d\nu '$, we have:

\begin{equation} \label{eq_f_nu}
    f_{\nu} = \delta_{\rm D} f'_{\nu'}, 
\end{equation}

where $\delta_{\rm D} = 1/\Gamma/(1-\beta cos\theta)$ is the doppler factor.




The total radiation energy in the comoving frame can be written as:


\begin{equation}
\begin{array}{ll} \label{eq_E'}
    \epsilon'_{\rm rad} &= \int \, d\frac{\nu'}{\nu'_p} \left[ \epsilon'_{\rm rad} \mathcal{B}(\frac{\nu'}{\nu'_p})  \right] \\
       &= \int \, d\nu'dS'd\Omega'(h\nu' f'_{\nu'}).
\end{array}
\end{equation}

We have assumed that the photon field is isotropic in the comoving frame, so $f'_{\nu'}$ is independent of $d\Omega'$. Then using Eq.\ref{eq_E'}, we have:

\begin{equation} \label{eq_f_nu'}
    f'_{\nu'} = \frac{\epsilon'_{\rm rad} \mathcal{B} \left( \frac{\nu'}{\nu'_{p}} \right)}{(4\pi R)^2h\nu' \nu'_{p}}.
\end{equation}



Similarly, the total radiation energy in the source frame is:

\begin{equation}
    \epsilon_{\rm rad} = \int \, d\nu dSd\Omega (h\nu f_{\nu}).
\end{equation}

Using Eq.\ref{eq_f_nu}, we have:

\begin{equation} \label{eq_E2}
    \epsilon_{\rm rad} = \int \, d\nu dSd\Omega (h\nu \delta_{\rm D} f'_{\nu'}).
\end{equation}

Using the expression of $f'_{\nu'}$ (Eq.\ref{eq_f_nu'}), $dS$ ($dS = R^2sin\theta d\theta d\phi$) and $d\Omega$ ($d\Omega = dA/(D)^2$), \ref{eq_E2} can be written as:
\begin{equation} \label{eq_E3}
    \epsilon'_{\rm rad} = \frac{2\nu'_p\epsilon_{\rm rad}}{\int \, \frac{1}{[\Gamma(1-\beta cos\theta)]^2} \mathcal{B}\left( \frac{\nu}{\nu_p} \right) d\cos\theta d\nu} .
\end{equation}






Using Eq.\ref{eq_f_nu}, Eq.\ref{eq_f_nu'} and Eq.\ref{eq_E3}, $f_{\nu}$ can be calculated once $E$ has been determined in the internal shock modeling:

\begin{equation}
    f_{\nu} = \frac{2\delta_{\rm D}^2 \epsilon_{\rm rad} \mathcal{B} \left( \frac{\nu}{\nu_p} \right)}{(4\pi R)^2h\nu \int \, \frac{1}{[\Gamma(1-\beta cos\theta)]^2} \mathcal{B}\left( \frac{\nu}{\nu_p} \right) d\cos\theta d\nu}.
\end{equation}




\par

By using energy conservation, the received flux $F_{\nu}$ is:

\begin{equation}
    F_{\nu}dtdA = h\nu f_{\nu}dSd\Omega.
\end{equation}

The difference of arrival time of two photons emitted on $\theta$ and $\theta + d\theta$ is:

\begin{equation}
    dt = -\frac{Rdcos\theta}{c}.
\end{equation}

If we assume the photon emitted at origin corresponding to $t_{\rm e} = 0$, then the equal arrival time is \citep{Granot2008}:
\begin{equation} \label{eq_eats}
    \frac{t_{\rm{obs}}}{1+z} = t_{\rm{e}}-\frac{R}{c}cos\theta, 
\end{equation}
where $t_{\rm{e}}$ is the emission time.

Considering the difference of flux between the source frame and observer frame, we should introduce a factor $1+z$, then the final expression of $F_{\nu}$ is:

\begin{equation}
    F_{\nu}(t_{\rm{obs}}) = \frac{(1+z)c\delta_{\rm D}^2 \epsilon_{\rm rad} \mathcal{B} \left( \frac{\nu}{\nu_p} \right)}{4\pi d^2_LR\int \, \frac{1}{[\Gamma(1-\beta cos\theta)]^2} \mathcal{B} \left( \frac{\nu}{\nu_p} \right)dcos\theta d\nu},
\end{equation}

where $d_L$ is the luminosty distance, $\delta_{\rm D} = 1/\Gamma/(1-\beta cos\theta_0)$ and $cos\theta_0$ can be determined by Eq.\ref{eq_eats}.

In the case of multi-flash with $\gamma\gamma$ absorption, we include the contribution of all flashes:

\begin{equation}
    F_{\nu}(t_{\rm{obs}}) = \sum\limits_{j = 1}^{j = j_{\rm{max}}} \frac{(1+z)c\delta_{{\rm D}, j}^2 \epsilon_{{\rm rad}, j} \mathcal{B} \left( \frac{\nu}{\nu_p} \right)}{4\pi d^2_L R_j\int \, \frac{1}{[\Gamma_j(1-\beta_j cos\theta)]^2} \mathcal{B} \left( \frac{\nu}{\nu_p} \right)dcos\theta d\nu}\times e^{-[\tau_{\gamma\gamma, \rm in}(\nu, j)+\tau_{\gamma\gamma, \rm ex}(\nu, j)]}.
\end{equation}


















\bibliography{ref}{}
\bibliographystyle{aasjournal}

\end{document}
