\subsection{\texorpdfstring{$SO(8)$}{SO(8)} characters}
The $SO(8)$ characters are written in terms of theta function with characteristics, defined as
\begin{equation}\vartheta \left[
\begin{array}{c}
 a \\
 b \\
\end{array}
\right](0,\tau )=\overset{\infty }{\sum _{n=-\infty } }e^{2 \pi  b i (a+n)+\pi  i \tau  (a+n)^2} \ .
\end{equation}
They are given by
\begin{equation} \small\label{characters}
\begin{split}
    {\bar O}_8 &= \frac{1}{2 \bar{\eta }^4}\left(\bar{\vartheta }^4\left[
\begin{array}{c}
 0 \\
 0 \\
\end{array}
\right](0,\tau )+\bar{\vartheta }^4\left[
\begin{array}{c}
 0 \\
 \frac{1}{2} \\
\end{array}
\right](0,\tau )\right)  \\
& =  \frac{1}{2 \bar{\eta }^4}\left(\left(\overset{\infty }{\sum _{n=-\infty } }\bar{q}^{\frac{n^2}{2}}\right)^4+\left(\overset{\infty }{\sum _{n=-\infty } }(-1)^n \bar{q}^{\frac{n^2}{2}}\right)^4\right)\\
{\bar V}_8 &= \frac{1}{2 \bar{\eta }^4}\left(\bar{\vartheta }^4\left[
\begin{array}{c}
 0 \\
 0 \\
\end{array}
\right](0,\tau )-\bar{\vartheta }^4\left[
\begin{array}{c}
 0 \\
 \frac{1}{2} \\
\end{array}
\right](0,\tau )\right)
= \frac{1}{2 \bar{\eta }^4} \bar{\vartheta }^4\left[
\begin{array}{c}
 \frac{1}{2} \\
 0 \\
\end{array}
\right](0,\tau ) \\
&= \frac{1}{2 \bar{\eta }^4} \left(\overset{\infty }{\sum _{n=-\infty } }(-1)^n \bar{q}^{\frac{n^2}{2}}\right)^4 \\
%&=\frac{1}{2 \bar{\eta }^4}\left(\left(\overset{\infty }{\sum _{n=-\infty } }\bar{q}^{\frac{n^2}{2}}\right)^4-\left(\overset{\infty }{\sum _{n=-\infty } }(-1)^n \bar{q}^{\frac{n^2}{2}}\right)^4\right) \\
{\bar S}_8 &= \frac{1}{2 \bar{\eta }^4}\left(\bar{\vartheta }^4\left[
\begin{array}{c}
 \frac{1}{2} \\
 0 \\
\end{array}
\right](0,\tau )+\bar{\vartheta }^4\left[
\begin{array}{c}
 \frac{1}{2} \\
 \frac{1}{2} \\
\end{array}
\right](0,\tau )\right) 
= \frac{1}{2 \bar{\eta }^4} \bar{\vartheta }^4\left[
\begin{array}{c}
 \frac{1}{2} \\
 0 \\
\end{array}
\right](0,\tau ) \\
&= \frac{1}{2 \bar{\eta }^4} \left(\overset{\infty }{\sum _{n=-\infty } }(-1)^n \bar{q}^{\frac{n^2}{2}}\right)^4 \\
%&=\frac{1}{2 \bar{\eta }^4}\left(\left(\overset{\infty }{\sum _{n=-\infty } }\bar{q}^{\frac{1}{2} \left(n+\frac{1}{2}\right)^2}\right)^4+\left(\overset{\infty }{\sum _{n=-\infty } }(-1)^n \bar{q}^{\frac{1}{2} \left(n+\frac{1}{2}\right)^2}\right)^4\right) \\
   {\bar C}_8 &= \frac{1}{2 \bar{\eta }^4}\left(\bar{\vartheta }^4\left[
\begin{array}{c}
 \frac{1}{2} \\
 0 \\
\end{array}
\right](0,\tau )-\bar{\vartheta }^4\left[
\begin{array}{c}
 \frac{1}{2} \\
 \frac{1}{2} \\
\end{array}
\right](0,\tau )\right)
= \frac{1}{2 \bar{\eta }^4} \bar{\vartheta }^4\left[
\begin{array}{c}
 \frac{1}{2} \\
 0 \\
\end{array}
\right](0,\tau ) \\
&= \frac{1}{2 \bar{\eta }^4} \left(\overset{\infty }{\sum _{n=-\infty } }(-1)^n \bar{q}^{\frac{n^2}{2}}\right)^4 
%&= \frac{1}{2 \bar{\eta }^4}\left(\left(\overset{\infty }{\sum _{n=-\infty } }\bar{q}^{\frac{1}{2} \left(n+\frac{1}{2}\right)^2}\right)^4-\left(\overset{\infty }{\sum _{n=-\infty } }(-1)^n \bar{q}^{\frac{1}{2} \left(n+\frac{1}{2}\right)^2}\right)^4\right)
\end{split}
\end{equation}

The $q$ expansion of the theta series for the lattices $\Gamma_{v,s,c,0}$ defined in \eqref{conjdef} is
\beq \label{Thetas}
\begin{split}
\theta_{v} &= 1+224 q+31200 q^2+522880 q^3+ O(q^4) \\
\theta_{s} = \theta_{c} &= 256 q+30720 q^2+527360 q^3+ O(q^4)\\
\theta_{0} &= 4096 q^{\tfrac32} + 147456 q^{\tfrac52} + O(q^{\tfrac72})\, ,
\end{split}
\eeq
while that of the functions $\sigma_{0,2}$ defined in \eqref{sigma} is
\beq \label{powersigma}
\begin{split}
   \sigma_0(q,\bar q)  = & 1 + 2 \bar{q} + 4 \bar{q}^{\frac14} q^{\frac14} + 4 \bar{q}^{\frac94} q^{\frac14} + 2 q + 4 \bar{q} q + 
 4 \bar{q}^{\frac14} q^{\frac94} + 4 \bar{q}^{\frac94} q^{\frac94} + O(q^{\frac{13}{4}}) + O({\bar q}^{\frac{13}{4}})\\
\sigma_1(q,\bar q) 
 = & 2 \bar{q} + 2 \bar{q}^{\frac14} q^{\frac14} + 10 \bar{q}^{\frac94} q^{\frac14} + 2 q + 8 \bar{q} q + 
 10 \bar{q}^{\frac14} q^{\frac94} + 18 \bar{q}^{\frac94} q^{\frac94} + O(q^{\frac{13}{4}}) + O({\bar q}^{\frac{13}{4}})
\end{split}
\eeq

\subsection{Light fermions in \texorpdfstring{$E_6 \times SU(12)$}{E6 x SU(12)} enhancement}
\label{App:lightfermions}

The enhacement to $E_6 \times SU(12)$ is the best counter-example to the common lore that the cosmological constant is roughly the number of massless fermions  minus the number of massless bosons. This enhancement has no massless fermions, yet the cosmological constant is positive. The reason is that there are many (more precisely 2176) very light fermions ($m^2=\frac12$), half in class (s) and half in class (c) that give about a contribution to the cosmological constant that is in order of magnitude about 1.5 that of the massless states.  In the following we list these spinors, which have all  $p_R^2 = \frac{1}{4}$ and $p_L^2 = \frac{9}{4}$.

There are 1088 class (s) spinors
\bea 
\left(-w, n, \pi_1,\dots,\pi_8, -\pi_9,\pi_{10},\dots,\pi_{16} \right)
= \begin{cases}
%\pm\left({0,0},{0^3},{0^5},\underline{-\tfrac12^3,\tfrac12^5}\right) \quad & \rightarrow \quad 112 \\
%\pm\left(\underline{1,0},{0^3},{0^5},\underline{-\tfrac12,\tfrac12^7}\right) \quad & \rightarrow \quad 32 \\
\pm\left(\underline{k,0},{0^3},{0^5},\underline{-\tfrac12^{3-2k},\tfrac12^{5+2k}}\right) \quad & \rightarrow \quad 144 \\
%\pm\left({1,1},{0^3},\underline{0^3,1^2},\underline{-\tfrac12,\tfrac12^7}\right) \quad & \rightarrow \quad 160 \\
%\pm\left(\underline{2,1},{0^3},\underline{0,1^4},\underline{-\tfrac12,\tfrac12^7}\right) \quad & \rightarrow \quad 160 \\
%\pm\left({2,2},{0^3},\underline{1^4,2},\underline{-\tfrac12,\tfrac12^7}\right) \quad & \rightarrow \quad 80 \\
\pm\left( \underline{k,1},{0^3},\underline{0,1^{2k}},\underline{-\tfrac12,\tfrac12^7}\right) \quad & \rightarrow \quad 400 \\
\pm\left({2,2},\underline{0,0,\pm 1},1^5,\underline{\tfrac12^7,\tfrac32}\right) \quad & \rightarrow \quad 96 \\
%\pm\left({0,0},\underline{\pm\tfrac12^3}_{\text{even}},\underline{-\tfrac12^2,\tfrac12^3},{0^8}\right) \quad & \rightarrow \quad 80 \\
%\pm\left(\underline{-1,0},\underline{\pm\tfrac12^3}_{\text{even}},\underline{-\tfrac12^4,\tfrac12},{0^8}\right) \quad & \rightarrow \quad 80 \\
%\pm\left(\underline{1,0},\underline{\pm\tfrac12^3}_{\text{even}},{\tfrac12^5},{0^8}\right) \quad & \rightarrow \quad 16 \\
\pm\left(\underline{k-1,0},\underline{\pm\tfrac12^3}_{\text{even}},\underline{-\tfrac12^{4-2k}},\tfrac12^{1+2k},{0^8}\right) \quad & \rightarrow \quad 176 \\
\pm\left({1,1},\underline{\pm\tfrac12^3}_{\text{even}},{\tfrac12^5},\underline{0^6,1^2}\right) \quad & \rightarrow \quad 224 \\
%\pm\left({1,1},\underline{\pm\tfrac12^3}_{\text{odd}},\underline{\tfrac12^4,\tfrac32},{0^8}\right) \quad & \rightarrow \quad 40 \\
%\pm\left({3,3},\underline{\pm\tfrac12^3}_{\text{odd}},{\tfrac32^5},{1^8}\right) \quad & \rightarrow \quad 8 
\pm\left((1+2k)^2,\underline{\pm\tfrac12^3}_{\text{odd}},\underline{\tfrac12^{4-4k},\tfrac32^{1+4k}},{k^8}\right) \quad & \rightarrow \quad 48 \\
\end{cases}\nonumber
%\pm\left({k,1},\underline{\pm\tfrac12^3}_{\text{even}},{\tfrac12^5},\underline{0^{8-2k},1^{2k}}\right)_{k=0,\,1} \quad & \rightarrow \quad 240 \\
\eea 
and 1088 class (c) spinors:
\bea 
\left(-w, n, \pi_1,\dots,\pi_8, -\pi_9,\pi_{10},\dots,\pi_{16} \right)
= \begin{cases}
%\pm\left(0,0,0^3,\underline{0^4,-1},\underline{0^7,1}\right) \quad & \rightarrow \quad 80 \\
%\pm\left(0,0,\underline{0^2, \pm 1},0^5,\underline{0^7,1}\right) \quad & \rightarrow \quad 96 \\
%\pm\left(0,0,\underline{0^2,-1},0^5,\underline{0^7,1}\right) \quad & \rightarrow \quad 48 \\
\pm\left(0,0,\underline{0^7,-1},\underline{0^7,1}\right) \quad & \rightarrow \quad 128 \\
%\pm\left(0,0,\underline{0^2,1},0^5,\underline{0^7,1}\right) \quad & \rightarrow \quad 48 \\
%\pm\left(2,2,0^3,1^5,\underline{0^3,1^5}\right) \quad & \rightarrow \quad 112 \\
\pm\left(2k,2k,\underline{0^{2+k},1^{1-k}},k^5,\underline{0^{7-4k},1^{1+4k}}\right) \quad & \rightarrow \quad 160 \\
%\pm\left(\underline{1,0},0^3,\underline{0^4,1},\underline{0^7,1}\right) \quad & \rightarrow \quad 160 \\
%\pm\left(1,1,0^3,\underline{0^2,1^3},\underline{0^7,1}\right) \quad & \rightarrow \quad 160 \\
%\pm\left(\underline{2,1},0^3,1^5,\underline{0^7,1}\right) \quad & \rightarrow \quad 32 \\
\pm\left(\underline{k,1},0^3,\underline{0^{4-2k},1^{1+2k}},\underline{0^7,1}\right) \quad & \rightarrow \quad 352 \\
%\pm \left(\underline{2,1},(\pm\tfrac12)^3_{\text{even}},\underline{\tfrac12^4,\tfrac32},\tfrac12^8\right) \quad & \rightarrow \quad 80 \\
%\pm \left(\underline{3,2},(\pm\tfrac12)^3_{\text{even}},\tfrac32^5,\tfrac12^8\right) \quad & \rightarrow \quad 16 \\
%\pm \left(\underline{2,2},(\pm\tfrac12)^3_{\text{even}},\underline{\tfrac12^2,\tfrac32^3},\tfrac12^8\right) \quad & \rightarrow \quad 80 \\
\pm \left(\underline{k+1 ,2},(\pm\tfrac12)^3_{\text{even}},\underline{\tfrac12^{4-2k},\tfrac32^{2k+1}},\tfrac12^8\right) \quad & \rightarrow \quad 176 \\
\pm \left({1,1},(\pm\tfrac12)^3_{\text{even}},\underline{-\tfrac12,\tfrac12^4},
\tfrac12^8\right) \quad & \rightarrow \quad 40 \\
\pm \left({1,1},(\pm\tfrac12)^3_{\text{odd}},\tfrac12^5,
\underline{-\tfrac12^2,\tfrac12^7}\right) \quad & \rightarrow \quad 224 \\
\pm \left({1,1},(\pm\tfrac12)^3_{\text{odd}},\tfrac12^5,
\tfrac12^8\right) \quad & \rightarrow \quad 8 \nonumber
\end{cases}
\eea 
where underline means any permutation of the entries, and on the right we count the total number of states for each $Z$. 


Each state has a contribution to $z_s$ or $z_c$ of $q^{\tfrac18}\bar{q}^{\tfrac18} = 
%(q \bar{q})^{\tfrac18} = 
e^{-\frac{\pi \tau_2}{2}}$.
After integrating over the fundamental region, we get a contribution to $\Lambda$ of $0.177$ for each state. The $2176$ total fermionic states at this mass level increase $\Lambda$ by $385.2$, gaining over the massless states negative contribution of $-243.7$. As a result we get a positive cosmological constant even though there are no massless fermions. Note that there is still a piece of the cosmological constant coming from heavier states, as the one of the massless states plus the one of these fermionic states is equal to 141.5, while the total cosmological constant at this enhancement point is 180.4.