\documentclass{article}
\usepackage[utf8]{inputenc}

\title{Statement on the Revision of JOSS-D-23-00099 Based on Reviewer's reports}
\author{M. Gil de Oliveira, A.M. Ozorio de Almeida}

\begin{document}

\maketitle

In view of the constructive criticism provided by both referees, we have stressed the Wigner-Weyl representation of the density operator, in contrast to the
more usual density matrix, to the extent of adding the Wigner function to the title of this paper.

\section{Response to Reviewer 1}

We thank the reviewer for his mindful comments. Following his suggestions, we have included the calculation of the Wigner function for the Morse system, as well as the expectation value of the operators $\left| q \right> \left< q \right|$ and $\left| p \right> \left< p \right|$, to illustrate different kinds of quantities that can be calculated within our formalism. We also included a Poincaré section for the Nelson system, to display its generic mixture of chaotic and regular behaviour. Lastly, we have included an appendix in which we describe in detail the computational part of our work.

\section{Response to Reviewer 2}

We thank the reviewer for bringing to our attention references of which we were unaware, but that are closely related to our work. On the whole, we came to the conclusion that the reviewer has already alluded to: The main advantage of our method is its being applicable to arbitrary hamiltonians, not only those that are quadratic in momentum. Indeed, in the latter case, the Wick rotation for the thermal density matrix is actually simpler, but further advantages of a direct calculation of a semiclassical calculation of the thermal Wigner function are now presented in additions to the Introduction and the final Discussion, in order to make this point more clear. 

We also stressed the fact that we did apply our method for a system which does not fall under this category: the Kerr system, which is an essential element of the many-body Bose-Hubbard hamiltonian. In this case, the numerical technique based on the double hamiltonian was not needed, because all quantities could be obtained analytically. We also included an appendix detailing the computational part of our work, as requested.

A discussion about the extra difficulty of approximating the heat capacity, which may even display a spurious negative regions, is inserted in its first presentation, for Kerr Hamiltonian.

\end{document}