%%%%%%%%%%%%%%%%%%%%%%%%%%%%%%%%%%%%%%%%%%%%%%%%%%%%%%%%%%%%%%%%%%%%%%%%%%
% 
% PLEASE, PREPARE YOUR PAPER ACCORDING TO THIS FILE
% 
%%%%%%%%%% REQUIRED PACKAGES %%%%%%%%%%%%%%%%%%%%%%%%%%%%%%%%%%%%%%%%%%%%%
% The following packages are used in production. You can comment some of them, 
% if you do not have some of required files. Also, you can add additional 
% packages to handle pictures etc.

\documentclass[oneside,10pt]{article}          % please do not change
\usepackage[b5paper]{geometry}	    % your paper can be easily printed on a4 or letter paper with enlargenment      
                                    % comment if you have problem with print     
\usepackage{amsfonts,amsmath,latexsym,amssymb} % these packages are required
\usepackage{theorem}                % please use one of this two options for theorems
%\usepackage{amsthm}                % please use one of this two options for theorems
\usepackage{mathrsfs,upref}         % not so essential, but part of journal style
\usepackage{mathptmx}		    % Journal is printed with poscript fonts: 
	                            % this package is essential for exact line and page breaks
		                    % comment if you do not have this package
	               	            % 
\usepackage{oam}	            % Journal style, comment only if you find a bug
                	            % 
%%%%%%%%%% THEOREMS, LEMMAS, DEFINITIONS, ETC %%%%%%%%%%%%%%%%%%%%%%%%%%%%
% Theorem system is the standard one, you can use the one which is familiar to you
% The style of numbering of theorems is left to author(s)
% Change the following lines accordingly, but left definitions and remarks behind
% \theoremstyle{definition}:

\newtheorem{theorem}{Theorem}           
\newtheorem{lemma}{Lemma}               
\newtheorem{corollary}{Corollary}
\newtheorem{proposition}{Proposition}
%\newtheorem{theorem}{Theorem}[section]         % alternative set of theorem numbering  
%\newtheorem{lemma}{theorem}{Lemma}             
%\newtheorem{corollary}{theorem}{Corollary}

\theoremstyle{definition}

\newtheorem{definition}{Definition}
\newtheorem{example}{Example}
\newtheorem{exercise}{Exercise}
\newtheorem{remark}{Remark}

%\newtheorem{definition}{Definition}[section]
%\newtheorem{remark}{Remark}[section]

%\numberwithin{equation}{section}       
        % use this style if you prefer numbering of equations with sections

%%%%%%%%%% LOCAL MACROS, like the following ones. Be careful with names... %%%%%%%%
%
% \let\a\alpha
% \let\b\beta
% \define\eR{\mathbf{R}}
% \def\e{\varepsilon}
% \newcommand\q{\quad}
%
%%%%%%% START OF THE PAPER %%%%%%%%%%%%%%%%%%%%%%%%%%%%%%%%%%%%%%%%%%%%

\begin{document}

\title[Soft $g$-frames in \cdots]{Soft $g$-frames in soft Hilbert spaces}

% Short title is optional, it will appear in running heads.
% It is necessary only if the title is to long to be used in running heads

\author{Sayyed Mehrab Ramezani}

\address{Faculty of Technology and Mining, Yasouj University, Choram, Iran.\\
\email{m.ramezani@yu.ac.ir}}

%\address{Faculty of Technology and Mining, Yasouj University, Choram, Iran.,\\
%\email{email of the Second Author}}
%
%\address{Third Author, Full postal Address of the Third Author\\
%\email{email of the Third Author}}

\CorrespondingAuthor{Sayyed Mehrab Ramezani}

%\dedicated{Dedicated to...}                    % Optional

\date{DD.MM.YYYY}                               % Please, write the date of submission

\keywords{frames; soft Hilbert spaces; soft frames;  soft $g$-frames}

\subjclass{42C15, 42C99}
        % AMS-2010 subj class. The list can be found on http://www.ams.org/mathscinet/msc/msc2010.html

%\thanks{This research is supported by ...\par This paper is a lecture that was given at...} 
        % Optional. Only one command thanks is allowed, use \par inside text if you need multiple thanks.       

\begin{abstract}
      In this paper, we define the concept of soft $g$-frame in soft Hilbert spaces
 by extending the concept of soft from frame to $g$-frame. We then  show some properties of the soft $g$-frames in soft Hilbert spaces.
  Among other results, we show that the $g$-frame operator is associated with a soft, finite, self-adjoining, reversible, and finite inverse  $g$-frame and get the dual g-frames. In addition, we prove that every element in Hilbert's soft space satisfies the theorem of $g$-frame decomposition.
\end{abstract}

\maketitle

%%%%% END OF TITLE PAGE %%%%%%%%%%%%%%%%%%%%%%%%%%%%%%%%%%%%%%%%%%%%%%

%%%%% BODY OF THE PAPER %%%%%%%%%%%%%%%%%%%%%%%%%%%%%%%%%%%%%%%%%%%%%%
% You should eventually delete (after reading) the rest of the text below %%
\section{Introduction}
Frames for Hilbert space were formally defined by Duffin and Schaeffer \cite{duf} in 1952 while studying some problems in non-harmonic Fourier series.
Recall that for a Hilbert space ${H}$ and a countable index set $J$,
a collection $\lbrace f_{j}\rbrace_{j\in J}\subset {H}$ is called a frame for the Hilbert space
${H}$ if there exist two positive constants $a$,  $b$ such that for all $f\in H$
\begin{equation}\label{eq1}
a\Vert f\Vert^2\leq\sum_{j\in J}\vert\langle f , f_{j}\rangle\vert^2\leq b\Vert f\Vert^2;
\end{equation}
$a$ and $b$ are called the lower and upper frame bounds, respectively.
If only the right-hand inequality in (\ref{eq1}) is satisfied, we call $\lbrace f_{j}\rbrace_{j\in J}$ a Bessel sequence for ${H}$ with Bessel bound $b$.
Wenchang Sun \cite{sun} developed a generalization of frame ($g$-frame) including more other  concepts and proved that many basic properties can be derived from  this more general context and Ramezani  \cite{meh} presented a generalization of orthonormal bases or simply $g$-orthonormal bases.

A sequence $ \lbrace \Lambda_{j}\in \mathcal{B}(\mathcal{H} , \mathcal{ K}_{j} ): j\in J\rbrace $ is called a $g$-frame for $ \mathcal{H}$ with respect to $\lbrace  \mathcal{K}_{j}\rbrace_{j\in J}$ if there exist two positive constants $A$ and $B$ such that: 
\begin{equation}\label{eq2}
A\Vert f\Vert^2\leq\sum_{j\in J}\Vert\Lambda_{j}f\Vert^2\leq B\Vert f\Vert^2,
\hspace{3cm}
(f\in \mathcal{H}).
\end{equation}
The bounded linear operator $T_{\Lambda}$  is defined by:
$$T_\Lambda:\bigoplus_{j\in J} \mathcal{K}_{j}\longrightarrow{\mathcal{H}},\hspace{1cm}T_\Lambda(\lbrace g_{j}\rbrace_{j\in J})=\sum_{j\in J}\Lambda^*_j g_j,$$
is called the pre $g$-frame operator of $\lbrace\Lambda_{j}\rbrace_{j\in J}$. Also, the bounded linear operator $S_{\Lambda}$
is defined by:
$$S_{\Lambda}: \mathcal{H}\longrightarrow \mathcal{H},\hspace{1cm}S_{\Lambda}(f)=\sum_{j\in J}\Lambda^*_j\Lambda_j f,$$
is called the $g$-frame  operator of $\lbrace\Lambda_{j}\rbrace_{j\in J}$.
\begin{example}\label{ex1}
If $\mathcal{K}_{j}$=$\mathbb{C}$ for any $j\in J$ and $\Lambda_{j}f = \langle f , f_{j}\rangle$ for any 
$ f\in \mathcal{H}$, in this case, the $g$-frame is just a frame for $\mathcal{H}$.
\end{example}
G-frames are generalized frames which include ordinary frames, bounded invertible linear operators,
as well as many recent generalizations of frames.

Soft set theory was introduced by Molodtsov \cite{mol} in 1999 as a new mathematical tool
for dealing with uncertainties while modeling problems in engineering, physics, computer
science, economics, social science, and medical sciences. This theory began to receive special
attention in 2002 when Maji et al. \cite{maj} applied the soft sets to decision-making problems,
using rough mathematics, and later in 2003, with the definitions of various operations
of soft sets \cite{roy}.

In \cite{soft},
Osmin Ferrer , Arley Sierra and José Sanabria
  used soft linear operators to introduce the notion of discrete frames on
soft Hilbert spaces, which extends the classical notion of frames on Hilbert spaces to the context of
algebraic structures on soft sets. Also, they  showed that the frame operator associated to
a soft discrete frame is bounded, self-adjoint, invertible and with a bounded inverse. Furthermore,
they proved that every element in a soft Hilbert space satisfies the frame decomposition theorem.

In this paper, we extend the concept of discrete frames in soft Hilbert spaces to generalized frames, and define the $g$-frame concept in terms of soft Hilbert spaces. Among other results, we show that the $g$-frame operator is associated with a soft, finite, self-adjoining, reversible, and finite inverse  $g$-frame. In addition, we prove that every element in Hilbert's soft space satisfies the theorem of $g$-frame decomposition.
\section{Preliminaries}
Throughout this paper, $\mathcal{X}$ denotes a non-empty set (possibly without algebraic structure),
$\mathcal{P}(\mathcal{X})$ the power set of $\mathcal{X}$, and $A$ a non-empty set of parameters.
\begin{definition}[\cite{mol}]
A soft set on $\mathcal{X}$ is a pair $(F,A)$ where $F$ is a mapping given by $F:A\longrightarrow \mathcal{P}(\mathcal{X})$.
In this way, we can see a soft set as the following:
$$(F,A)=:\lbrace(\lambda,F(\lambda))\ \ :\ \ \lambda\in A,\ F(\lambda)\in\mathcal{P}(\mathcal{X})\rbrace.$$
\end{definition}
\begin{definition}[\cite{das}]
A soft set $(F,A)$  on $\mathcal{X}$ is said to be a null soft set if $F(\lambda)=\emptyset$ for all $\lambda\in A$,
and in this case we write $(F,A)=\Phi$.
If $F(\lambda)\neq\emptyset$ for some $\lambda\in A$, then the soft set $(F,A)$   is said to be a non-null soft set
on $\mathcal{X}$.
\end{definition}
\begin{definition}[\cite{das}]
A soft set $(F,A)$  on $\mathcal{X}$ is said to be an absolute soft set if $F(\lambda)=\mathcal{X}$ for
all $\lambda\in A$; in this case, we write $(F,A)=\check{\mathcal{X}}$.
\end{definition}
\begin{definition}[\cite{das}]
A soft element on $\mathcal{X}$ is a function $\varepsilon:A\longrightarrow \mathcal{X}$. Now if $\varepsilon(\lambda)\in F(\lambda)$ for all
$\lambda\in A$, the soft element $\varepsilon$ is said to belong to the soft set $(F,A)$ on $\mathcal{X}$, which we denote by $\varepsilon\tilde{\in}F$.
\end{definition}
We denote the collection of all the soft elements of a soft set $(F,A)$ by $SE\big((F, A)\big)$; this
is
$$SE\big((F, A)\big):=\lbrace\varepsilon\ :\ \varepsilon\tilde{\in}F\rbrace=\lbrace\varepsilon\ :\ \varepsilon(\lambda)\in F(\lambda),\ \ \forall\lambda\in A\rbrace,$$
and  equivalently,
$$F(\lambda)=\lbrace\varepsilon(\lambda)\ :\ \varepsilon\tilde{\in}F\rbrace.$$
\begin{definition}[\cite{das1}]
Let $\mathbb{R}$ be the set of real numbers and $\mathcal{P}(\mathbb{R})$  be the collection
of all non-empty bounded subsets of $\mathbb{R}$ and A be a set of
parameters. Then a mapping $F:A\longrightarrow \mathcal{P}(\mathbb{R})$ is called a soft real  set. It is denoted
by $\mathbb{R}(A)$.
\end{definition}
\begin{definition}[\cite{das2}]
Let $\mathbb{C}$ be the set of complex numbers and $\mathcal{P}(\mathbb{C})$  be the collection
of all non-empty bounded subsets of $\mathbb{C}$. A be a set of
parameters. Then a mapping $F:A\longrightarrow \mathcal{P}(\mathbb{C})$ is called a soft complex set. It is denoted
by $\mathbb{C}(A)$.
\end{definition}
\begin{definition}[\cite{das}]
For two soft real numbers $\alpha\!:A\!\longrightarrow\!\mathbb{R}$ and $\beta\!:A\!\longrightarrow\!\mathbb{R}$ we define the following:
\begin{enumerate}
\item
$\alpha\tilde{\leq}\beta,\ \text{if} \ \alpha(\lambda)\leq\beta(\lambda),\ \text{for all}\ \lambda\in A$,
\vspace*{0.25cm}
\item
$\alpha\tilde{\geq}\beta,\ \text{if} \ \alpha(\lambda)\geq\beta(\lambda),\ \text{for all}\ \lambda\in A$,
\vspace*{0.25cm}
\item
$\alpha\tilde{<}\beta,\ \text{if} \ \alpha(\lambda)<\beta(\lambda),\ \text{for all}\ \lambda\in A$,
\vspace*{0.25cm}
\item
$\alpha\tilde{>}\beta,\ \text{if} \ \alpha(\lambda)>\beta(\lambda),\ \text{for all}\ \lambda\in A$.
\end{enumerate}
\end{definition}
\begin{definition}[\cite{das3,das4}]
Let $\mathcal{X}$ be a $\mathbb{K}$-vector space (typically $\mathbb{K}=\mathbb{R}$ or $\mathbb{K}=\mathbb{C}$), $A$ be a non-empty
set of parameters and $(F, A) $ be a soft set on $\mathcal{X}$. The soft set $(F, A) $ is said to be a soft $\mathbb{K}$-vector space on  $\mathcal{X}$ if $F(\lambda)$ is a vector subspace of  $\mathcal{X}$ for all $\lambda\in A$.
\end{definition}
\begin{definition}[\cite{das3,das4}]
Let $(F, A) $ be a soft $\mathbb{K}$-vector space on $\mathcal{X}$. A soft vector $x$ of $(F, A) $ is said to be a null soft vector if $x(\lambda)=\theta$, $\text{for all}\ \lambda\in A$, where $\theta$ is the zero element of $\mathcal{X}$. This is denoted by
$\Theta$.
\end{definition}
\begin{definition}[\cite{Tak}]
Let $\check{\mathcal{X}}$ be an absolute soft $\mathbb{K}$-vector space, then a mapping $\Vert\cdot\Vert: SE\big((F, A)\big)\longrightarrow\mathbb{R}(A)$ is said to be a soft norm on $\check{\mathcal{X}}$ if $\Vert\cdot\Vert$ satisfies the following conditions:
\begin{enumerate}
\item
$\Vert x\Vert\tilde{\geq}\overline{0},\ \text{for all}\ x \tilde{\in}\check{\mathcal{X}}$,
\vspace*{0.25cm}
\item
$\Vert x\Vert=\overline{0},\ \text{if and only if}\ x =\Theta$,
\vspace*{0.25cm}
\item
$\Vert \alpha\cdot x\Vert=\vert \alpha\vert\Vert x\Vert,\ \text{for all}\  x\tilde{\in}\check{\mathcal{X}}\ 
\text{and for all soft scalar}\ \alpha,
\vspace*{0.25cm}$
\item
$\Vert x+y\Vert \tilde{\leq}\Vert x\Vert +\Vert y\Vert,\ \text{for all}\  x,y\tilde{\in}\check{\mathcal{X}}$.
\end{enumerate}
\end{definition}
The soft $\mathbb{K}$-vector space $\check{\mathcal{X}}$ with a soft norm $\Vert\cdot\Vert$ on $\check{\mathcal{X}}$ is said to be a soft $\mathbb{K}$-normed
space and is denoted by $(\check{\mathcal{X}},\Vert\cdot\Vert,A)$ or $(\check{\mathcal{X}},\Vert\cdot\Vert)$.
\begin{theorem}[\cite{Tak}Decomposition theorem]
 If $(\check{\mathcal{X}},\Vert\cdot\Vert,A)$ is a soft normed space, then for each
$\lambda\in A$, the mapping $\Vert\cdot\Vert:{\mathcal{X}}\longrightarrow\mathbb{R}^+$ defined by $\Vert\xi\Vert_\lambda=\Vert x\Vert(\lambda)$ where $x$ is such that $x(\lambda)=\xi$
is a norm on $\mathcal{X}$ for all $\lambda\in A$.
\end{theorem}
\begin{definition}[\cite{das3}]
If $\check{\mathcal{X}}$ is an absolute soft vector space, then a binary operation $\langle\cdot,\cdot\rangle: SE\big(\check{\mathcal{X}}\big)\times SE\big(\check{\mathcal{X}}\big)\longrightarrow\mathbb{C}(A)$ is said to be a soft inner product on $\check{\mathcal{X}}$ if it satisfies the following conditions:
\begin{enumerate}
\item
$\langle x ,x\rangle\tilde{\geq}\overline{0},\ \text{for all}\ x \tilde{\in}\check{\mathcal{X}} 
\ \text{ and }\ \langle x ,x\rangle=\overline{0}, \ \text{if and only if}\ x =\Theta$,
\vspace*{0.25cm}
\item 
$\langle x ,y\rangle=\overline{\langle y ,x\rangle},\ \text{for all}\ x ,y \tilde{\in}\check{\mathcal{X}}$,
\vspace*{0.25cm}
\item
$\langle \alpha\cdot x ,y\rangle=\alpha\cdot{\langle x ,y\rangle},\ \text{for all}\  x,y\tilde{\in}\check{\mathcal{X}}\ 
\text{and for all soft scalar}\ \alpha,$
\vspace*{0.25cm}
\item
$\langle x+y,z\rangle =\langle x,z\rangle +\langle y,z\rangle ,\ \text{for all}\  x,y,z\tilde{\in}\check{\mathcal{X}}$.
\end{enumerate}
\end{definition}
The soft vector space $\check{\mathcal{X}}$ with a soft inner product $\langle\cdot,\cdot\rangle$ on $\check{\mathcal{X}}$ is said to be a soft inner
product space and is denoted by $(\check{\mathcal{X}},\langle\cdot,\cdot\rangle,A)$ or $(\check{\mathcal{X}},\langle\cdot,\cdot\rangle)$.
\begin{definition}[\cite{das5}]
A soft inner product space is said to be complete if it is complete with respect
to the soft metric defined by the soft inner product. A complete soft inner product space is said to be a soft Hilbert space.
\end{definition}
\begin{example}
Let $\mathcal{X}\!:=\!\bigoplus_{j\in J}\!\mathcal V_{j}\!=\!\lbrace \lbrace g_{j}\rbrace_{j\in J}:\ g_{j}\!\!\in \!\!\mathcal V_{j},\ \sum_{j\in J}\!|\langle  g_{j},  g_{j}\rangle|\!\!<\!\infty\rbrace$. It is well known that $\mathcal{X}$ is a Hilbert space with respect to the
inner product $\langle f,g\rangle=\sum_{j\in J}\langle  f_{j},  g_{j}\rangle_{\mathcal V_{j}}$ for $f=\lbrace f_{j}\rbrace_{j\in J}$, $g=\lbrace g_{j}\rbrace_{j\in J}$
in $\bigoplus_{j\in J}\mathcal V_{j}$. Now, let $\tilde{f}$, $\tilde{g}$ be
soft elements of the absolute soft vector space $\check{\mathcal{X}}$. Then, 
$\tilde{f}(\lambda)=\lbrace f_{j}^{\lambda}\rbrace_{j\in J}$,
$\tilde{g}(\lambda)=\lbrace g_{j}^{\lambda}\rbrace_{j\in J}$
 are
elements of $\bigoplus_{j\in J}\mathcal V_{j}$. Thus, the mapping 
$\langle\cdot,\cdot\rangle:SE(\check{\mathcal{X}})\times SE(\check{\mathcal{X}})\longrightarrow\mathbb{C}(A)$, defined by, $ \langle\tilde{f},\tilde{g}\rangle(\lambda)=\sum_{j\in J}\langle  f_{j}^{\lambda},  g_{j}^{\lambda}\rangle_{\mathcal V_{j}}$
 for all $\lambda\in A$, is a soft inner product on $\check{\mathcal{X}}$. Therefore, $(\check{\mathcal{X}},\langle\cdot,\cdot\rangle,A)$,
with $A$ being a finite set of parameters, is a soft Hilbert space.
\end{example}
\section{Soft $g$-frames in soft Hilbert spaces}\label{sec3}
In this section, we define the concept of soft $g$-frame in soft Hilbert spaces
 by extending the concept of soft from frame to $g$-frame. We then  show some properties of the soft $g$-frames in soft Hilbert spaces.
 \begin{definition} 
 Let $ \lbrace \Lambda_{j}\rbrace_{j\in J} $ be a sequence of soft operator of a soft Hilbert space $\check{\mathcal{U}}$ to
 soft Hilbert spaces ${\check{\mathcal{V}}}_{j}$ having a finite
sets of parameters.
We call a sequence $ \lbrace \Lambda_{j}\in \mathcal{B}(\check{\mathcal{U}} , {\check{\mathcal{V}}}_{j} ): j\in J\rbrace $ a 
soft $g$-frame for $\check{\mathcal{U}}$ with respect to $\lbrace \check{\mathcal{V}}_{j}\rbrace_{j\in J}$ if
 there exist positive soft real numbers
  $\overline{D}\tilde{\geq}\overline{C}\tilde{>}\overline{0}$ such that: 
\begin{equation}\label{eq1}
\overline{C}\Vert f\Vert^2\tilde{\leq}\sum_{j\in J}\Vert\Lambda_{j}f\Vert^2\tilde{\leq} \overline{D}\Vert f\Vert^2,
\hspace{3cm}
(f\in\check{\mathcal{U}}).
\end{equation}
 The soft real numbers $\overline{C}$ and $\overline{D}$ are called bounds of the soft $g$-frame. These are not
unique, as the optimal bounds are the largest possible value of a and the smallest possible
value of $\overline{D}$ that satisfy \eqref{eq1}. In the case that $\overline{C}=\overline{D}$, the soft $g$-frame is called tight.
We call $ \lbrace \Lambda_{j}\in \mathcal{B}(\check{\mathcal{U}}  , \check{\mathcal{V}} _{j} ): j\in J\rbrace $ an exact soft $g$-frame if it ceases to be a soft $g$-frame whenever any one of its
elements is removed.
\end{definition}
\begin{example}
Let  $ \lbrace f_{j}\rbrace_{j\in J} $ be a soft frame for   a soft Hilbert space $\check{\mathcal{U}}$
 Let $\Lambda_{f_j}$ be
the functional induced by $f_j$, i.e.,
$$\Lambda_{f_j}=\langle f,f_j\rangle \hspace*{1.5cm}(f\in\check{\mathcal{U}} )$$
It is easy to check that $ \lbrace \Lambda_{f_j}\rbrace_{j\in J} $ is a soft $g$-frame for  $\check{\mathcal{U}}$ with respect to $ \mathbb{C}(A) $.
 \end{example}
 \section{soft $g$-frame operators and dual g-frames }
 Next, we introduce the notion of 
the soft pre $g$-frame operator and the soft $g$-frame operator of a soft  $g$-frame. In addition, we
establish the most important result, called the decomposition theorem of soft $g$-frames.
\begin{definition}
Given a soft $g$-frame  $ \lbrace \Lambda_{j}\rbrace_{j\in J} $ 
 for soft Hilbert space $\check{\mathcal{U}}$ with respect to soft Hilbert space $\lbrace \check{\mathcal{V}}_{j}\rbrace_{j\in J}$ having a finite set of
parameters, we define the soft pre $g$-frame operator by the following:
\begin{align*}
&T:SE(\check{\bigoplus_{j\in J}\mathcal V_{j}})\longrightarrow SE(\check{\mathcal{U}}),\\
&T(\lbrace {g_j}\rbrace_{j\in J})=
\sum_{j\in J}\Lambda^*_{j}g_{j}.
\end{align*}
\end{definition}
\begin{proposition}
 The soft pre $g$-frame operator associated to soft $g$-frame  $ \lbrace \Lambda_{j}\rbrace_{j\in J} $  is well defined and bounded.
 \begin{proof}
 Let  $g=\lbrace g_{j}\rbrace_{j\in J}$ be a soft element of $\check{\bigoplus_{j\in J}\mathcal V_{j}}$ then
 
 \begin{align*}
 \Vert T(g)\Vert&=\sup_{\Vert f\Vert=\overline{1}}\vert\langle T(\lbrace {g_j}\rbrace_{j\in J}), f\rangle\vert\\
 &=\sup_{\Vert f\Vert=\overline{1}}\vert\langle \sum_{j\in J}\Lambda_{j}^*g_j, f\rangle\vert\\
 &=\sup_{\Vert f\Vert=\overline{1}}\vert\sum_{j\in J}\langle g_j, \Lambda_{j}f\rangle\vert\\
 &\tilde{\leq}\sup_{\Vert f\Vert=\overline{1}}\sum_{j\in J}\vert\langle g_j, \Lambda_{j}f\rangle\vert\\
 &\tilde{\leq}\left( \sum_{j\in J}\vert\langle g_j, g_{j}\rangle\vert^2\right) ^\frac{1}{2}\cdot
 \sup_{\Vert f\Vert=\overline{1}}\left( \sum_{j\in J}\vert\langle \Lambda_{j}f, \Lambda_{j}f\rangle\vert^2\right) ^\frac{1}{2}\\
 &=\Vert g\Vert
 \sup_{\Vert f\Vert=\overline{1}}\left( \sum_{j\in J}\Vert \Lambda_{j}f\Vert^2\right) ^\frac{1}{2}\\
 &\tilde{\leq}\Vert g\Vert \sqrt{\overline{D}}\sup_{\Vert f\Vert=\overline{1}}\Vert f\Vert\\
 &\tilde{\leq}\Vert g\Vert \sqrt{\overline{D}}\Vert f\Vert,
 \end{align*}
 so
  \begin{align*}
 \Vert T\Vert
 &\tilde{\leq}\Vert g\Vert \sqrt{\overline{D}},
 \end{align*}
 and therefore
 the soft $g$-frame operator   $ T $  is well defined and bounded.
 \end{proof}
\end{proposition}
\begin{proposition}
The adjoint soft operator of $T$ is given by
\begin{align*}
&T^*:SE(\check{\mathcal{U}})\longrightarrow SE(\check{\bigoplus_{j\in J}\mathcal{V}_{j}}),
 \\
&T^*(f)=\lbrace\Lambda_{j}f\rbrace_{j\in J}.
\end{align*}
 \begin{proof}
 Let 
 $f\in SE(\check{\mathcal{U}})$
 and 
 $g\in SE(\check{\bigoplus_{j\in J}\mathcal{V}_{j}})$, then 
 $g=\lbrace g_{j}\rbrace_{j\in J}$
 with $g(\lambda)=\lbrace g^\lambda_{j}\rbrace_{j\in J}$.
Additionally, for all $\lambda\in A$, the following holds:
\begin{align*}
\langle T^*(f),\lbrace g_{j}\rbrace_{j\in J}\rangle_{\bigoplus}(\lambda)&=
\langle f,T(\lbrace g_{j}\rbrace_{j\in J})\rangle_{\check{\mathcal{U}}}(\lambda)\\
&=\langle f,\sum_{j\in J}\Lambda^*_{j}g_{j}\rangle_{\check{\mathcal{U}}}(\lambda)\\
&=\sum_{j\in J}\langle \Lambda_{j}f,g_{j}\rangle_{\bigoplus}(\lambda)\\
&=\langle \lbrace\Lambda_{j}f\rbrace_{j\in J},\lbrace g_{j}\rbrace_{j\in J}\rangle_{\bigoplus}(\lambda).
\end{align*}
 \end{proof}
\end{proposition}
\begin{definition}
We define the associated soft $g$-frame operator to the soft $g$-frame as follow 
\begin{align*}
&S:SE(\check{\mathcal{U}})\longrightarrow SE(\check{\mathcal{U}})
 \\
&S(f)=TT^*(f),
%=T(\lbrace\Lambda_{j}f\rbrace_{j\in J})=\sum_{j\in J}\Lambda^*_{j}\Lambda_{j}f
\end{align*}
where $ \Lambda^*_{j} $ is the adjoint operator of $ \Lambda_{j} $. First of all, $ S $ is well defined on $ SE(\check{\mathcal{U}}) $. To see this, let $ n_1<n_2 $
be integers. Then we have 
\end{definition}
%The associated  $ g $-frame operator to the soft $ g $-frame $ \lbrace \Lambda_{j}\rbrace_{j\in J} $ 
%for $\check{\mathcal{U}}$ with respect to $\lbrace \check{\mathcal{V}}_{j}\rbrace_{j\in J}$ 
%with bounds
%$\overline{D}\tilde{\geq}\overline{C}\tilde{>}\overline{0}$ is bounded and self-adjoint.
\begin{align*}
\Vert\sum_{j=n_1}^{j=n_2}\Lambda^*_{j}\Lambda_{j}f\Vert&=\sup_{g\in SE(\check{\mathcal{U}}),\Vert g\Vert=\overline{1} }\left|\left\langle \sum_{j=n_1}^{j=n_2}\Lambda^*_{j}\Lambda_{j}f,g\right\rangle \right|\\
&=\sup_{g\in SE(\check{\mathcal{U}}),\Vert g\Vert=\overline{1} }\left|\left\langle \sum_{j=n_1}^{j=n_2}\Lambda_{j}f,\Lambda_{j}g\right\rangle \right|\\
&\tilde{\leq}\sup_{g\in SE(\check{\mathcal{U}}),\Vert g\Vert=\overline{1} }
\left( \sum_{j\in J}\Vert\Lambda_{j}f\Vert^2\right) ^{\frac{1}{2}}\left( \sum_{j\in J}\Vert\Lambda_{j}g\Vert^2\right) ^{\frac{1}{2}}\\
&\tilde{\leq}
D ^{\frac{1}{2}}\left( \sum_{j\in J}\Vert\Lambda_{j}f\Vert^2\right) ^{\frac{1}{2}}.
\end{align*}
Therefore, $ Sf $ is well defined for
any $ f\in SE(\check{\mathcal{U}}) $
and,
\begin{align*}
\Vert S\Vert&=\sup_{f\in SE(\check{\mathcal{U}}),\Vert f\Vert=\overline{1} }\left\langle Sf,f\right\rangle 
=\sup_{f\in SE(\check{\mathcal{U}}),\Vert f\Vert=\overline{1} }\sum_{j\in J}\Vert\Lambda_{j}f\Vert^2 
\tilde{\leq}
D.
\end{align*}
Hence $ S $ is a bounded self-adjoint operator.
\begin{corollary}
Let $ S:SE(\check{\mathcal{U}})\longrightarrow SE(\check{\mathcal{U}}) $ be the associated soft $ g $-frame operator to the soft $ g $-frame $ \lbrace \Lambda_{j}\rbrace_{j\in J} $ 
for $\check{\mathcal{U}}$ with respect to $\lbrace \check{\mathcal{V}}_{j}\rbrace_{j\in J}$. Then
 \begin{align*}
 S(f)=\sum_{j\in J}\Lambda^*_{j}\Lambda_{j}f.
\end{align*}
 \begin{proof}
 Let 
 $ f\in SE(\check{\mathcal{U}}) $
  then
 \begin{align*}
 S(f)=TT^*(f)=T(\lbrace\Lambda_{j}f\rbrace_{j\in J})=\sum_{j\in J}\Lambda^*_{j}\Lambda_{j}f.
\end{align*}
 \end{proof}
\end{corollary}
\begin{remark}
Note that for all $ f\in SE(\check{\mathcal{U}}) $ we have the following:
\begin{align*}
 \left\langle Sf,f\right\rangle =\left\langle \sum_{j\in J}\Lambda^*_{j}\Lambda_{j}f,f\right\rangle 
= \sum_{j\in J}\left\langle\Lambda^*_{j}\Lambda_{j}f,f\right\rangle
= \sum_{j\in J}\left\langle\Lambda_{j}f,\Lambda_{j}f\right\rangle
= \sum_{j\in J}\Vert\Lambda_{j}f\Vert^2.
\end{align*}
\end{remark}
Hence, the soft $ g $-frame condition can be written in the form 
\begin{align*}
\overline{C}\Vert f\Vert^2\tilde{\leq}\left\langle Sf,f\right\rangle \tilde{\leq} \overline{D}\Vert f\Vert^2,
\hspace{3cm}
(f\in\check{\mathcal{U}}).
\end{align*}
The above mentioned cases can be summarized in the following proposition:
\begin{proposition}\label{por1}
Given a sequence
$ \lbrace \Lambda_{j}\rbrace_{j\in J} $ of soft operator of a soft Hilbert space $\check{\mathcal{U}}$ to
 soft Hilbert spaces ${\check{\mathcal{V}}}_{j}$ having a finite
sets of parameters, the following
statements are equivalent:
\begin{enumerate}
\item
 $ \lbrace \Lambda_{j}\in \mathcal{B}(\check{\mathcal{U}} , {\check{\mathcal{V}}}_{j} ): j\in J\rbrace $ is a 
soft $g$-frame for $\check{\mathcal{U}}$ with respect to $\lbrace \check{\mathcal{V}}_{j}\rbrace_{j\in J}$ with bounds $\overline{D}\tilde{\geq}\overline{C}\tilde{>}\overline{0}$.
\item
$ S(f)=\sum_{j\in J}\Lambda^*_{j}\Lambda_{j}f $,  is a positive and bounded soft linear operator of $ SE(\check{\mathcal{U}}) $ to $ SE(\check{\mathcal{U}}) $,
which satisfies $\overline{C}Id\tilde{\leq}S\tilde{\leq}\overline{D}Id$. 
\end{enumerate}
\end{proposition}
Here we give an application of soft $g$-frame.
Let   $ \lbrace \Lambda_{j}\rbrace_{j\in J} $ be a soft $g$-frame
 for soft Hilbert space $\check{\mathcal{U}}$ with respect to soft Hilbert spaces $\lbrace \check{\mathcal{V}}_{j}\rbrace_{j\in J}$. Suppose that $ \lbrace \tilde{\Lambda}_{j}\rbrace_{j\in J} $ is
the canonical dual soft $g$-frame. Then for any $ f\in\check{\mathcal{U}} $, we have
\begin{align*}
 f=\sum_{j\in J}\Lambda^*_{j}\tilde{\Lambda}_{j}f=\sum_{j\in J}\tilde{\Lambda}^*_{j}\Lambda_{j}f.
\end{align*}
 Let $ T $ be a bounded soft
linear   operator on $ \check{\mathcal{U}} $. We see that
\begin{align}\label{eq3}
 T(f)=\sum_{j\in J}\Lambda^*_{j}\tilde{\Lambda}_{j}T(f)=\sum_{j\in J}\tilde{\Lambda}^*_{j}\Lambda_{j}T(f).
\end{align}
We call \eqref{eq3} soft atomic resolutions of a soft operator T.
%
%In general, the Bessel sequence $ \lbrace g_{j}\rbrace_{j\in J} $ is called an alternate dual of the frame $ \lbrace f_{j}\rbrace_{j\in J} $ if the
%following formula holds, for all $ f\in\check{\mathcal{U}} $: 
%\begin{align*}
% f=\sum_{j\in J}\langle f_{j},f\rangle g_{j}
%\end{align*}
\begin{theorem}
 Let $ \lbrace \Lambda_{j}\rbrace_{j\in J} $ and $ \lbrace \Gamma_{j}\rbrace_{j\in J} $  be a pair of dual soft $g$-frame for
  for soft Hilbert space $\check{\mathcal{U}}$ with respect to soft Hilbert spaces $\lbrace \check{\mathcal{V}}_{j}\rbrace_{j\in J}$.
 Suppose $ \lbrace f_{j,k}:k\in K_j\rbrace $
 and $ \lbrace { g}_{j,k}:k\in K_j\rbrace $ be a pair of dual soft frames for $ \check{\mathcal{V}}_{j} $, respectively.
Then
 $ \lbrace \Lambda^*_{j}(f_{j,k}):j\in J\ k\in K_j\rbrace $
 and $ \lbrace \Gamma^*_{j}({ g}_{j,k}):j\in J, \ k\in K_j\rbrace $ 
   are a pair of dual soft   frames for $\check{\mathcal{U}}$
provided the soft frame bounds for $ \lbrace f_{j,k}:k\in K_j\rbrace $ satisfying
   $\overline{C}\tilde{\leq}\overline{A_j}
   \tilde{\leq}\overline{B_j}
   \tilde{\leq}\overline{D}$
   for some constants
$\overline{C},\overline{D}
   \tilde{>}\overline{0}$.
   \begin{proof}
  Assume that $ \lbrace f_{j,k}:k\in K_j\rbrace $ be a soft frame for $ \check{\mathcal{V}}_{j} $ then for  soft element $ \Lambda_{j}f $ in $ \check{\mathcal{V}}_{j} $   we have 
   \begin{equation*}
\overline{A_j}\Vert \Lambda_{j}f\Vert^2\tilde{\leq}\sum_{k\in K_j}\vert\langle \Lambda_{j} f,(f_{j,k})\rangle\vert^2\tilde{\leq} \overline{B_j}\Vert \Lambda_{j}f\Vert^2,
\hspace{4.3cm}
(f\in\check{\mathcal{U}}),
\end{equation*}
so
\begin{equation*}
\overline{A}\Vert \Lambda_{j}f\Vert^2\tilde{\leq}
\overline{A_j}\Vert \Lambda_{j}f\Vert^2\tilde{\leq}\sum_{k\in K_j}\vert\langle \Lambda_{j} f,(f_{j,k})\rangle\vert^2\tilde{\leq} \overline{B_j}\Vert \Lambda_{j}f\Vert^2\tilde{\leq}
\overline{B}\Vert \Lambda_{j}f\Vert^2,
\hspace{0.5cm}
(f\in\check{\mathcal{U}}).
\end{equation*}
therefore
 \begin{equation*}
\sum_{j\in J}\overline{A}\Vert \Lambda_{j}f\Vert^2\tilde{\leq}\sum_{j\in J}\sum_{k\in K_j}\vert\langle \Lambda_{j} f,(f_{j,k})\rangle\vert^2\tilde{\leq} \sum_{j\in J}\overline{B}\Vert \Lambda_{j}f\Vert^2,
\hspace{2.4cm}
(f\in\check{\mathcal{U}}).
\end{equation*}
$ \lbrace \Lambda_{j}\rbrace_{j\in J} $   is a  soft $g$-frame for
  for soft Hilbert space $\check{\mathcal{U}}$ with respect to soft Hilbert spaces $\lbrace \check{\mathcal{V}}_{j}\rbrace_{j\in J}$, then for all soft element $ f $ in $ \check{\mathcal{U}} $   we have 
   \begin{equation*}
   \overline{A}\  \overline{C}\Vert f\Vert^2\tilde{\leq}
\sum_{j\in J}\overline{A}\Vert \Lambda_{j}f\Vert^2\tilde{\leq}\sum_{j\in J}\sum_{k\in K_j}\vert\langle \Lambda_{j} f,(f_{j,k})\rangle\vert^2\tilde{\leq} \sum_{j\in J}\overline{B}\Vert \Lambda_{j}f\Vert^2
\tilde{\leq} \overline{B}\ \overline{D}\Vert f\Vert^2,
\end{equation*}
consequently
\begin{equation*}
   \overline{A} \ \overline{C}\Vert f\Vert^2
\tilde{\leq}\sum_{j\in J}\sum_{k\in K_j}\vert\langle \Lambda_{j} f,(f_{j,k})\rangle\vert^2
\tilde{\leq} \overline{B}\ \overline{D}\Vert f\Vert^2,
\hspace{3cm}
(f\in\check{\mathcal{U}}).
\end{equation*}
Therefore, $ \lbrace \Lambda^*_{j}(f_{j,k}):j\in J\ k\in K_j\rbrace $ is a soft frame for $\check{\mathcal{U}}$
 and  similarly we can get that $ \lbrace \Gamma^*_{j}({ g}_{j,k}):j\in J, \ k\in K_j\rbrace $ is a soft frame for $\check{\mathcal{U}}$.
On the other hand for any $ f\in\check{\mathcal{U}} $, we have
\begin{align*}
 \sum_{j\in J}\!\sum_{k\in K_j}\langle \Lambda^*_{j}(f_{j,k}),f\rangle \Gamma^*_{j}({ g}_{j,k})
\! =\!\sum_{j\in J} \Gamma^*_{j}\!\left( \sum_{k\in K_j}\langle (f_{j,k}),\Lambda_{j}f\rangle({ g}_{j,k})\right) 
\!  =\!\sum_{j\in J} \Gamma^*_{j}\Lambda_{j}f=f.
\end{align*}
Similarly we can get that
\begin{align*}
 \sum_{j\in J}\sum_{k\in K_j}\langle \Gamma^*_{j}({ g}_{j,k}) ,f\rangle\Lambda^*_{j}(f_{j,k})
 =f.
\end{align*}
Hence $ \lbrace \Lambda^*_{j}(f_{j,k}):j\in J\ k\in K_j\rbrace $
 and $ \lbrace \Gamma^*_{j}({ g}_{j,k}):j\in J, \ k\in K_j\rbrace $ 
   are  dual soft frames for $\check{\mathcal{U}}$.
   \end{proof}
\end{theorem}
\begin{corollary}
Suppose that $ \lbrace \Lambda_{j}\rbrace_{j\in J} $ and $ \lbrace \tilde{\Lambda}_{j}\rbrace_{j\in J} $  are canonical dual soft g-frames,
$ \lbrace f_{j,k}:k\in K_j\rbrace $
 and $ \lbrace \tilde{ f}_{j,k}:k\in K_j\rbrace $ are canonical dual soft frames, and that $ \lbrace f_{j,k}:k\in K_j\rbrace $ is a
tight soft frame with soft frame bounds
  $ \overline{A_j}
   =\overline{B_j}=\overline{A} $. 
   Then $ \lbrace \Lambda^*_{j}(f_{j,k}):j\in J\ k\in K_j\rbrace $
 and $ \lbrace \tilde{\Lambda}^*_{j}(\tilde{ f}_{j,k}):j\in J, \ k\in K_j\rbrace $  are canonical dual soft  frames. 
 \begin{proof}
 Assume that $ \lbrace \Lambda_{j}\rbrace_{j\in J} $ and $ \lbrace \tilde{\Lambda}_{j}\rbrace_{j\in J} $  are canonical dual soft g-frames and
$ \lbrace f_{j,k}:k\in K_j\rbrace $ is a tight frame with frame bounds 
 $ \overline{A}_j=\overline{B}_j=\overline{A} $. Then $ \tilde{ f}_{j,k}=\dfrac{1}{\overline{A}} f_{j,k}$. Let
$ S_\Lambda $ and $ S_ {\Lambda^*}$ be the soft  frame operators associated with $ \lbrace \Lambda_{j}\rbrace_{j\in J} $ and $ \lbrace \Lambda^*_{j}(f_{j,k}):j\in J\ k\in K_j\rbrace $,
respectively. Then we have
\begin{align*}
 S_ {\Lambda^*}f&=
 \sum_{j\in J}\sum_{k\in K_j}\langle \Lambda^*_{j}({ f}_{j,k}) ,f\rangle\Lambda^*_{j}(f_{j,k})\\
 &=\sum_{j\in J}\Lambda^*_{j}\left( \sum_{k\in K_j}\langle { f}_{j,k} ,\Lambda_{j}f\rangle f_{j,k}\right) \\
 &=\overline{A}\sum_{j\in J}\Lambda^*_{j}\Lambda_{j}f\\
 &=\overline{A}S_\Lambda f.
\end{align*}
Hence
\begin{align*}
S_ {\Lambda^*}^{-1}\Lambda^*_{j}(f_{j,k})=
\dfrac{1}{\overline{A}}S_\Lambda \Lambda^*_{j}(f_{j,k})=
\tilde{\Lambda}^*_{j}(\tilde{ f}_{j,k}).
\end{align*}
This completes the proof.
 \end{proof}
\end{corollary}
\begin{theorem}
If $ \lbrace \Lambda_{j}\in \mathcal{B}(\check{\mathcal{U}} , {\check{\mathcal{V}}}_{j} ): j\in J\rbrace $ is a 
soft $g$-frame for $\check{\mathcal{U}}$ with respect to $\lbrace \check{\mathcal{V}}_{j}\rbrace_{j\in J}$  with bounds  $\overline{D}\tilde{\geq}\overline{C}\tilde{>}\overline{0}$ , then the
following statements are satisfied: 
\begin{enumerate}
\item
The soft linear operator $ S:SE(\check{\mathcal{U}})\longrightarrow SE(\check{\mathcal{U}}) $ is invertible and
$\overline{D}\tilde{\geq}\overline{C}\tilde{>}\overline{0}$
\begin{align*}
\overline{D}^{-1}I\tilde{<}S^{-1}\tilde{<}\overline{C}^{-1}I,
\end{align*}
where $ \overline{C}^{-1}(\lambda)=\dfrac{1}{\overline{C}} $ , for all $ \lambda\in A $.
\item
$ \lbrace\Lambda_j S^{-1}\rbrace_{j\in J} $ is 
a 
soft $g$-frame for $\check{\mathcal{U}}$ with respect to $\lbrace \check{\mathcal{V}}_{j}\rbrace_{j\in J}$  with bounds  $\overline{C}^{-1}\tilde{\geq}\overline{D}^{-1}\tilde{>}\overline{0}$, called the dual soft  $g$-frame
of $ \lbrace \Lambda_jS^{-1}\rbrace_{j\in J} $. 
\end{enumerate}
\begin{proof}
\begin{enumerate}
\item
The proof is the same as the proof of Theorem $ 18 $ in reference \cite{soft}.
\item
Since $ S^{-1} $ is positive and self-adjoint, we have the following: 
\begin{align*}
\sum_{j\in J}(\Lambda_{j}S^{-1})^*\Lambda_{j}S^{-1}(f)\!=\!
S^{-1}(\sum_{j\in J}\Lambda^*_{j}\Lambda_{j}(S^{-1}f))\!=\!
S^{-1}(S(S^{-1}f))\!=\!
 S^{-1}f.
\end{align*}
 As 
$\overline{D}^{-1}Id\tilde{\leq}S^{-1}{\leq}\overline{C}^{-1}Id$,
 by Proposition \ref{por1}, we obtain that $ \lbrace \Lambda_jS^{-1}\rbrace_{j\in J} $ is a soft
$g$-frame for $\check{\mathcal{U}}$ with respect to $\lbrace \check{\mathcal{V}}_{j}\rbrace_{j\in J}$  with bounds  $\overline{C}^{-1}\tilde{\geq}\overline{D}^{-1}\tilde{>}\overline{0}$. 
\end{enumerate}
\end{proof}
\end{theorem}


\begin{theorem}[Decomposition Theorem for soft frames]
 Let  $ \lbrace \Lambda_{j}\in \mathcal{B}(\check{\mathcal{U}} , {\check{\mathcal{V}}}_{j} ): j\in J\rbrace $ a 
soft $g$-frame for $\check{\mathcal{U}}$ with respect to $\lbrace \check{\mathcal{V}}_{j}\rbrace_{j\in J}$  with bounds  $\overline{D}\tilde{\geq}\overline{C}\tilde{>}\overline{0}$ and $ S $ be the associated $g$-frame operator. Then, for all $ f\in SE(\check{\mathcal{U}}) $, we
have the following: 
\begin{align*}
f=\sum_{j\in J}\Lambda^*_{j}\Lambda_{j}S^{-1}f=\sum_{j\in J}S^{-1}\Lambda^*_{j}\Lambda_{j}f.
\end{align*}
%\begin{align*}
% f=\sum_{j\in J}\langle f,S^{-1}\Lambda_j f\rangle g_{j}.
%\end{align*}
\begin{proof}
If  $ f\in SE(\check{\mathcal{U}}) $, then the
following statements are satisfied: 
\begin{align*}
 f=S(S^{-1}f)=\sum_{j\in J}\Lambda^*_{j}\Lambda_{j}(S^{-1}f)
 =\sum_{j\in J}\Lambda^*_{j}\Lambda_{j}S^{-1}f,
\end{align*}
and
\begin{align*}
 f=S^{-1}(Sf)=S^{-1}(\sum_{j\in J}\Lambda^*_{j}\Lambda_{j}f)
 =\sum_{j\in J}S^{-1}\Lambda^*_{j}\Lambda_{j}f.
\end{align*}
\end{proof}
\end{theorem}
\begin{thebibliography}{20}
\bibitem{das2}
Das, S.; Samanta, S.K. On soft complex sets and soft complex numbers. J. Fuzzy Math. 2013, 21, 195–216.
\bibitem{das5}
Das, S.; Samanta, S.K. On soft inner product spaces. Ann. Fuzzy Math. Inform. 2013, 6, 151–170.
\bibitem{das}
Das, S.; Samanta, S.K. On soft metric spaces. J. Fuzzy Math. 2013, 21, 707–734.
\bibitem{das3}
Das, S.; Samanta, S.K. Operators on soft inner product spaces. Fuzzy Inf. Eng. 2014, 6, 435–450.
\bibitem{das4}
Das, S.; Samanta, S.K. Soft linear operators in soft normed linear spaces. Ann. Fuzzy Math. Inform. 2013, 6, 295–314.
\bibitem{das1}
S. Das and S. K. Samanta, Soft real sets, soft real numbers and their properties, The journal of fuzzy
math. 20(3) (2012), 551-576.
\bibitem{duf}
{ R. Duffin,  A. Schaeffer }. A class of non-harmonic Fourier series. {\small\it Trans. Amer. Math. Soc.} {\bf 72}, (1952), 341-366.
\bibitem{soft}
O. Ferrer, A. Sierra,  J. Sanabria. Soft Frames in Soft Hilbert Spaces. Mathematics. 2021; 9(18):2249.  https://doi.org/10.3390/math9182249
\bibitem{maj}
 P.K. Maji, R. Biswas, R. Roy,  An application of soft sets in a decision making problem. Comput. Math. Appl. 2002, 44, 1077–1083.
\bibitem{roy}
 P.K. Maji, R. Biswas, R. Roy, Soft set theory. Comput. Math. Appl. 2003, 45, 555–562.
\bibitem{mol}
D. Molodtsov,  Soft set theory-First results. Comput. Math. Appl. 1999, 37, 19–31.
%\bibitem{mehrab}
%{Ramezani S M}. G-duals of continuous frames and their perturbations in Hilbert spaces.{\small\it U.P.B. Sci. Bull.} {\bf 82}, (2020), 75-82.
\bibitem{meh}
{Ramezani S M, Nazari A}. g-orthonormal bases, g-Riesz bases and $g$-dual of g-frames.{\small\it U.P.B. Sci. Bull.} {\bf 78}, (2016), 91-98.
\bibitem{sun}
{W. Sun }. Stability of g-frames.{\small\it J. Math. Anal. Appl.}{\bf 326}(2), (2007), 858-868.
\bibitem{Tak}
Thakur, R.; Samanta, S.K. Soft Banach Algebra. Ann. Fuzzy Math. Inform. 2015, 10, 397–412.
\end{thebibliography}
%
%        Editorial office prefers use of the AMS commands 
%        (which can be cited like ordinary bibitem references):
%
%\Refs
%
%\bibitem{AS2}
%        \by M. Abramowitz and I. A. Stegun (Eds)
%        \book Handbook of Mathematical Functions with Formulas, Graphs, and Mathematical Tables
%        \publ National Bureau of Standards, Applied Mathematics Series {\bf 55}, 9th printing
%        \publaddr Washington 
%        \yr 1970
%\endref
%  
%\bibitem{Burn}
%        \by W. Burnside
%        \paper A rapidly convergent series for $\log N!$
%        \jour Messenger Math.
%        \vol 46
%        \issue 1
%        \yr 1917
%        \pages 157--159
%\endref
%
%\endRefs

\end{document}
