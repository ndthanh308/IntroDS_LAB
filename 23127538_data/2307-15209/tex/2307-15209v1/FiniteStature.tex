     \documentclass[12pt, oneside,usenames,dvipsnames]{amsart}

\usepackage{amsmath,ifthen, amsfonts, amssymb, yhmath,
srcltx, % COMMENT OUT
   amsopn,tikz, tkz-euclide, url}
\usetikzlibrary{decorations.markings,arrows, cd}
\usepackage{tikz-cd}
 \usepackage{dutchcal}
\usepackage{xfrac} 
\usepackage{graphicx, ifsym}
\usepackage{fullpage}
%\usepackage{geometry}\geometry{margin=1in}
\tikzset{>=latex}

\usepackage{thmtools}
\usepackage{thm-restate}
\usepackage{pgfplots}
\usetikzlibrary{intersections, pgfplots.fillbetween}

\usepackage{hyperref}

\usepackage{cleveref}

%\declaretheorem[name=Theorem,numberwithin=section]{thmx}


\newcommand{\showcomments}{yes}
\renewcommand{\showcomments}{no}

\newsavebox{\commentbox}
\newenvironment{com}%
% begin comment
{\ifthenelse{\equal{\showcomments}{yes}}%
% then begin comment in margin
{\footnotemark
        \begin{lrbox}{\commentbox}
        \begin{minipage}[t]{1.25in}\raggedright\sffamily\tiny
        \footnotemark[\arabic{footnote}]}
% else eat contents of the environment
{\begin{lrbox}{\commentbox}}}%
% end comment
{\ifthenelse{\equal{\showcomments}{yes}}%
% then end comment
{\end{minipage}\end{lrbox}\marginpar{\usebox{\commentbox}}}
% else finish eating
{\end{lrbox}}}

\usepackage{amsmath,amsfonts, amssymb, graphicx, tikz, bbding,ifthen,srcltx,amsopn, url}
%\usepackage[latin2]{inputenc}
\usepackage{bm}
\usetikzlibrary{decorations.markings,arrows, intersections}
\usetikzlibrary{decorations.pathreplacing}

\newcommand{\angled}[1]{\langle#1\rangle}
\newcommand{\Z}{\mathbb Z}
\newcommand{\R}{\mathbb R}
\newcommand{\C}{\mathbb C}
\newcommand{\Hyp}{\mathbb H^2}
\newcommand{\h}{\mathbcal h}
\newcommand{\mH}{\mathbcal H}
\newcommand{\kk}{\mathcal k}
\newcommand{\bb}{\bar b}
\newcommand{\inv}[1]{#1^{-1}}
\newcommand{\conj}[2]{#1#2\inv{#1}}
\newcommand{\conjinv}[2]{\inv{#1}#2#1}
\newcommand{\embeds}{\hookrightarrow}
\newcommand{\immerses}{\looparrowright}
\newcommand{\Rep}{\mathrm{Rep}}
%\newcommand{\parallel}{||}
\newcommand{\no}{|\!|}
\DeclareMathOperator{\Fix}{Fix}
\DeclareMathOperator{\Aut}{Aut}
\DeclareMathOperator{\Stab}{Stab}
%\DeclareMathOperator{\Min}{Min}
\DeclareMathOperator{\sk}{sk}
\DeclareMathOperator{\Art}{Art}
\DeclareMathOperator{\canc}{canc}
\DeclareMathOperator{\rank}{rk}
\DeclareMathOperator{\im}{Im}
\DeclareMathOperator{\core}{Core}
\newcommand{\dist}{\textup{\textsf{d}}}
\newcommand{\Ast}{\mathop{\scalebox{1.5}{\raisebox{-0.2ex}{$\ast$}}}}%
\newcommand{\hookdownarrow}{\mathrel{\rotatebox[origin=t]{90}{$\hookleftarrow$}}}



\newcommand{\euler}{\chi}

\theoremstyle{definition}
\newtheorem{thm}{Theorem}[section]
\newtheorem{lem}[thm]{Lemma}
\newtheorem{prop}[thm]{Proposition}
\newtheorem{claim}{Claim}
\newtheorem{exa}[thm]{Example}
\newtheorem*{cond}{Condition}
\newtheorem{question}[thm]{Question}
\newtheorem{remark}[thm]{Remark}
\newtheorem{defn}[thm]{Definition}
\newtheorem{cor}[thm]{Corollary}
\newtheorem{obs}[thm]{Observation}
\newtheorem{notn}[thm]{Notation}
\newtheorem{thmx}{Theorem}
\renewcommand{\thethmx}{\Alph{thmx}} % "letter-numbered" theorems
\newtheorem{corx}[thmx]{Corollary}





\author{Kasia Jankiewicz}
\address{Department of Mathematics, University of California, Santa Cruz, CA 95064}
\email{kasia@ucsc.edu}
\title{Finite Stature in Artin groups}

\setcounter{tocdepth}{3}

\begin{document}
\begin{com}
{\bf \normalsize COMMENTS\\}
ARE\\
SHOWING!\\
\end{com}

\begin{abstract} We give criteria for a graph of groups to have finite stature with respect to its collection of vertex groups, in the sense of Huang-Wise. We apply it to the triangle Artin groups that were previously shown to split as a graph of groups. This allows us to deduce residual finiteness, and expands the list of Artin groups known to be residually finite.
\end{abstract}

\subjclass[2010]{20F36, 20F65, 20E26}
\keywords{Artin groups, Finite stature, Graphs of groups, Residual finiteness}
%\date{\today}

%%%
\maketitle

%\tableofcontents
\section{Introduction}
A group $G$ has \emph{finite stature} with respect to a collection of subgroups $\Omega$, if for every $H\in \Omega$ there are only finitely many $H$-conjugacy classes of subgroups of the form $H\cap\bigcap_{i\in I}H_i^{g_i}$ where $H_i^{g_i}$ is a $G$-conjugate of an element $H_i\in\Omega$. Finite stature was introduced by Huang-Wise in \cite{HuangWiseStature19} where they proved that under certain assumptions the fundamental group $G$ of a graph of groups has certain separability properties, provided that $G$ has finite stature with respect to its collection of vertex groups. In \cite{HuangWiseSpecial19} the same authors showed that a graph of nonpositively curved cube complexes $X$ with word hyperbolic fundamental group is virtually special, provided that $\pi_1X$ has finite stature with respect to the vertex groups in the corresponding splitting as a graph of groups. Finite stature is closely related to the more classical notion of finite \emph{height}, introduced and studied in \cite{GMRS98}.

In this article we provide explicit examples of very different nature, which are well-studied groups arising naturally in topology and geometric group theory. Our examples are not hyperbolic and not virtually compact special. Specifically we show that the splittings of certain Artin groups obtained by the author in \cite{JankiewiczArtinRf, JankiewiczArtinSplittings} have finite stature with respect to the vertex groups.
A \emph{triangle Artin group} is an Artin group on three generators, given by the presentation
$$G_{MNP} = \langle a, b, c\mid (a, b)_{M} = (b, a)_{M}, \,\,\,(b,c)_{N} = (c,b)_{N}, \,\,\,(c,a)_{P} = (a,c)_{P}\rangle,$$
where $(a,b)_M$ denotes the alternating word $aba\dots$ of length $M$. The value of $M$ can be $\infty$, in which case there is no relation of the form $(a, b)_{M} = (b, a)_{M}$. 
%An Artin group has \emph{large type}, if all $M\geq 3$ (possibly $M=\infty$).

\begin{thm}\label{thm: main}
A triangle Artin group $G_{MNP}$ splits as graphs of free groups with finite stature with respect to its collection of vertex groups, provided that either $M>2$ or $N>3$, where we assume that $M\leq N\leq P$.
%Triangle Artin groups $G_{2MN}$ where $M,N\geq 4$ admit splittings as graphs of free groups, and have finite stature with respect to a collection of their vertex groups.
\end{thm}
As a consequence (using results of \cite{HuangWiseStature19}) we obtain the following.
\begin{cor}
A triangle Artin group $G_{MNP}$, where $M\leq N\leq P$ and either $M>2$ or $N>3$, is residually finite.
\end{cor}

The condition on $M,N,P$ in Theorem~\ref{thm: main} excludes the cases  $(M,N,P)=(2,2,P)$ and $(M,N,P)=(2,3,P)$. In the first case, the corresponding Artin group $G_{MNP}$ is isomorphic to $Z\times A_P$ where $A_P$ denotes a dihedral Artin group, and consequently $G_{MNP}$ does not split as a graph of free groups, but is well-known to be residually finite.
However, when $(M,N,P)=(2,3,P)$ and $P\geq 7$ we do not know whether $G_{MNP}$ splits as a graphs of free groups with finite stature with respect to its collection of vertex groups, or if $G_{MNP}$ is residually finite,


There are a few other classes of Artin group that are known to be residually finite. 
 In the case of spherical type Artin groups, residual finiteness follows from linearity \cite{Krammer2002,Bigelow2001,CohenWales2002,Digne2003}. 
The linearity of a few other Artin groups was established as a consequence of being virtually special \cite{LiuGraphManifolds, PrzytyckiWiseGraphManifolds}, but none of the triangle Artin groups considered in Thereom~\ref{thm: main} admit virtual geometric actions on CAT(0) cube complexes \cite{HuangJankiewiczPrzytycki16, HaettelArtin}. Residual finiteness of some other Artin groups was proven in \cite{BlascoGarciaJuhaszParis18, BlascoGarciaMartinezPerezParis19}. There are also more examples provided in \cite{JankiewiczArtinRf}.

Some, but not all, of the groups considered in the above corollary were proven to virtually split as \emph{algebraically clean} graphs of free groups, i.e.\ graphs of finite rank free groups where all inclusions of edge groups in the adjacent vertex groups are inclusions as free factors, in \cite{JankiewiczArtinRf, JankiewiczArtinSplittings}. Such groups are known to be residually finite \cite{WisePolygons}.
Our method allows us to deduce residual finiteness of new Artin groups, but also recover the residual finiteness of the Artin groups treated in \cite{JankiewiczArtinRf, JankiewiczArtinSplittings}.


Group virtually splitting as algebraically clean graphs of groups satisfy some stronger profinite properties than residual finiteness, 
some of which are discussed in the forthcoming paper \cite{JankiewiczSchreveProfiniteAlgebraicallyClean}. 
% e.g.\ they are cohomologically good, for every $p$ they are virtually residually $p$-finite and virtually cohomologically $p$-complete \cite{JankiewiczSchreveProfiniteAlgebraicallyClean}. 
We do not know whether all the groups considered in this paper are in fact virtually algebraically clean. More generally, the following is open.
\begin{question}
Let $G$ be a graph of finite rank free groups with finite stature with respect to its collection of vertex groups. Does $G$ have a finite index subgroup whose induced splitting is algebraically clean?
\end{question}
The converse is known to be false, as there are examples of algebraically clean graphs of free groups that do not have finite stature \cite[Exmp 3.31]{HuangWiseStature19}. On the other hand, we do not know whether there exists a group $G$ splitting as an algebraically clean graph of groups such that $G$ does not have finite stature with respect to \emph{any} splitting with free vertex groups. 


This paper is organized as follows. In Section~\ref{sec: maps} we state some facts about maps between graphs and free groups, and fix the notation and terminology. Section~\ref{sec: finite stature} discusses the notion of finite stature, and we prove some facts used later in the text. Section~\ref{sec: graphs of groups} studies certain families of graphs of free groups. Finally, Section~\ref{sec: Artin} is devoted to Artin groups, and contains computations that allow us to apply the results from earlier sections to prove Theorem \ref{thm: main}.


\subsection*{Acknowledgements}
The author thanks Jingyin Huang and Dani Wise for fruitful discussions. This material is based upon work supported by the National Science Foundation grant DMS-2203307.

\section{Preliminaries}\label{sec: maps}
\subsection{Maps between graphs}
A \emph{combinatorial graph} $\Gamma$ is a disjoint union $V(\Gamma)\sqcup E(\Gamma)$ together with the operation $E(\Gamma)\to E(\Gamma), e\mapsto \bar e$ of taking the \emph{opposite edge} (i.e.\ the same edge with opposite orientation), and the operation $E(\Gamma)\to V(\Gamma), e\mapsto \tau(e)$ of taking the \emph{endpoint} of an oriented edge.

A \emph{metric graph} is a combinatorial graph that can also be viewed as a $1$-dimensional CW-complex, with a path metric in which each $1$-cell has length $1$. 
Later, we will be considering graphs of free groups and corresponding graphs of spaces where the spaces are graphs as well. We will denote the underlying graph of the graph of groups/graphs by $\Gamma$, while the vertex and edge spaces will be denoted by letters such as $X, Y$ and will be viewed as metric graphs. The following definitions will be applied to graphs arising as vertex and edge spaces.


A continuous map $\phi:Y\to X$ between two metric graphs is \emph{combinatorial}, if the image of each $0$-cell of $Y$ is a $0$-cell of $X$, and while
restricted to an open $1$-cell with endpoints $x_1, x_2$, $\phi$ is an isometry onto an edge with endpoints $\phi(x_1), \phi(x_2)$. A \emph{combinatorial immersion} is a combinatorial map $f:Y\to X$ which is locally injective. It is a well-known fact that every combinatorial immersion is $\pi_1$-injective.


A \emph{Stalling's fold} is a combinatorial map $f:Y\to X$ where 
\begin{itemize}
\item there exist distinct edges $y_1, y_2\in E(Y)$ such that $\tau(\bar y_1) = \tau(\bar y_2)$, and $E(X) = E(Y)/ {y_1\sim y_2}$,
\item $V(X) = V(Y)/ \tau(y_1) \sim \tau(y_2)$, and
\item $f$ is the natural quotient map, where $f(y_1) = f(y_2)$.
\end{itemize}
We note that $f$ is a homotopy equivalence if and only if $ \tau(y_1) \neq \tau(y_2)$.

We will also consider more general maps between graphs than combinatorial.
\begin{defn}
A continuous map $\phi:Y\to X$ between two metric graphs is \emph{monotone}, if the image of each $0$-cell of $Y$ is a $0$-cell of $X$, and while restricted to each 1-cell $y$ of $Y$, $\phi$ is either constant and its image is a $0$-cell $x$ in $X$, or $\phi$ is a combinatorial map after possibly subdividing $y$ into $n$ nontrivial subintervals.
 \end{defn}
 
 Here are two important examples of monotone maps.
 An \emph{edge-subdivision} is a monotone map $f:Y\to X$ where 
\begin{itemize}
\item there exists an edge $y\in E(Y)$ and edges $y_1, \dots, y_k \in E(X)$ where $k\geq 2$ such that $f(y)$ is equal the path $y_1\cdots y_k$,
\item $E(Y)-\{y\} = E(X) -\{y_1, \dots, y_k \}$, and $f$ is the identity map on $E(Y)-\{y\}$,
\item $V(X) = V(Y) \sqcup \{\tau(y_1), \dots, \tau(y_{k-1})\}$, and $\tau(y_i) = \tau(\bar y_{i+1})$ for all $i=1, \dots, k-1$ (i.e. $y_1\cdots y_k$ is a path in $X$),
\end{itemize}
An edge-subdivision is always a homotopy equivalence.
An \emph{edge-collapse} is a monotone map $f:Y\to X$ where 
\begin{itemize}
\item there exist an edge $y \in E(Y)$ such that $E(X) = E(Y)-\{y\}$,
\item $V(X) = V(Y)/ \tau(y) \sim \tau(\bar y)$, and
\item $f$ is the natural quotient map, where $f_{|y}$ is constant.
\end{itemize}
Similarly, an edge-collapse is a homotopy equivalence if and only if $ \tau(y) \neq \tau(\bar y)$, i.e.\ if $y$ is not a loop.

The following proposition provides a useful factorization of every monotone map.
\begin{prop}\label{prop:factorization}
Every monotone map $\phi:Y\to X$ factors as $Y\xrightarrow{\sigma} \overline Y\xrightarrow{\iota} X$ where
\begin{itemize}
\item $\sigma:Y\to \overline Y$ is obtained by a sequence of edge-subdivisions, Stalling's folds, edge-collapses,
\item $\iota: \overline Y\to X$ is a combinatorial immersion.
\end{itemize}
\end{prop}

\begin{proof} Every combinatorial map factors as a sequence of Stallings-folds followed by a combinatorial immersion. By definition, a monotone map $\phi$ restricted to an edge is either an edge-collapse, or an edge-subdivision post-composed with a combinatorial map. The statement follows.
\end{proof}


\subsection{Subgroups of free groups}
Let $X$ be a metric graph with a basepoint $x\in X$, and let $F=\pi_1(X,x)$.
A \emph{precover} of $X$ is a combinatorial immersion $Y  \to X$, such that there exists a covering map $\hat X\to X$ where $Y \subseteq \hat X$.

\begin{defn} %Let $X$ be a metric graph.
%A \emph{precover} of $X$ is a combinatorial immersion $Y  \to X$, such that there exists a covering $\hat X\to X$ where $Y \subseteq \hat X$. 
Given a subgroup $H\subseteq F$, the \emph{core of $H$ with respect to $X$} is a based precover $i:(Y , \hat x)\to (X, x)$ where $Y$ is the minimal subgraph of the covering space of $\hat X\to X$ corresponding to $H$ with $\pi_1(Y, \hat x)$. More generally, we also say a subgroup $H$ of $F$ is \emph{represented} by a combinatorial immersion $(Y,\hat x)  \to (X, x)$, if $H = \pi_1(Y, \hat x) \hookrightarrow F$.
\end{defn}
%\begin{com}
%Is this needed?
%\end{com}
Clearly, a precover $Y\to X$ is an embedding if and only if $\hat X = X$. 
%We note that for a subgroup $H$ of $G$, the core of $H$ with respect to $X$ is an embedding if and only if for every $g\in G-H$ there is no $k\geq 1$ such that $g^k\in H$.  
%\begin{com}
%is that iff?
%\end{com}
%In particular, this is an algebraic property of $H$ as a subgroup of $G$, and is independent of the choice of a graph $X$.


Let $\phi_i: Y_i\to X$ be a combinatorial immersion for $i=1,2$.
The \emph{fiber product of $Y_1$ and $Y_2$ over $X$} is the graph $Y_1\otimes_X Y_2$ with the vertex set
\[
\{(v_1,v_2)\in V(Y_1)\times V(Y_2): \phi_1(v_1) = \phi_2(v_2)\}
\] 
and the edge set
\[
\{(e_1, e_2)\in E(Y_1)\times E(Y_2): \phi_1(e_1) = \phi_2(e_2)\}.
\]
There is a natural combinatorial immersion $Y_1\otimes_X Y_2 \to X$, given by $(y_1, y_2)\mapsto \phi_1(y_1) = \phi_2(y_2)$. %This combinatorial immersion induces the embedding $\pi_1(Y_1\otimes_X Y_2, (v_1, v_2))\to \pi_1(X,v)$ (where $v\in V(X), v_1\in V(Y_1), v_2\in V(Y_2)$ satisfy $v = \phi_1(v_1) = \phi_2(v_2)$). 

\begin{lem}[{\cite{Stallings83}}]\label{lem:fiber product}
Let $H_1, H_2\subseteq G = \pi_1 (X, v)$ where $X$ is a finite metric graph, and for $i=1,2$ let $(Y_i, \hat x_i)\to (X, x)$ be the core of $H_i$ with respect to $X$. Then the intersection $H_1\cap H_2$ is represented by $(Y_1 \otimes_{X} Y_2, (\hat x_1, \hat x_2)) \to (X,x)$.
\end{lem}

\begin{lem}\label{lem: finitely many conjugacy classes of intersections}
Suppose $X$ is a finite graph, and $H_1, H_2\subseteq G= \pi_1(X, v)$ are finite rank subgroups. Then there are only finitely many conjugacy classes of the intersections of conjugates of $H_1$ and $H_2$, and any such intersection has finite rank.
\end{lem}
\begin{proof} 
By Lemma~\ref{lem:fiber product} each conjugacy class of the intersection of conjugates $H_1$ and $H_2$ is represented by  the connected component of the fiber product $Y_1\otimes_X Y_2$ where $Y_1, Y_2$ are cores of $H_1, H_2$ with respect to $X$. Since $H_1, H_2$ have finite ranks, $Y_1, Y_2$ are finite graphs. Thus $Y_1\otimes_X Y_2$ is finite, and in particular, $Y_1\otimes_X Y_2$ has finitely many connected components (each representing a conjugacy class of the intersections of conjugates of $H_1$ and $H_2$).
\end{proof}


\section{Finite stature}\label{sec: finite stature}
 Let $G$ be a group and let $\Omega = \{H_{\lambda}\}_{\lambda\in \Lambda}$ be a collection of subgroups
of $G$. Then $(G, \Omega)$ has \emph{finite stature} if for each $H \in\Omega$, there are finitely many $H$-conjugacy classes of infinite subgroups of form $H \cap C$, where $C$ is an intersection
of (possibly infinitely many) $G$-conjugates of elements of $\Omega$.
The main result of \cite{HuangWiseStature19} is the following.
\begin{thm}[{\cite[Thm 1.3]{HuangWiseStature19}}]
Let $G$ be the fundamental group of a graph of groups with finite underlying graph $\Gamma$. Suppose that
\begin{enumerate}
\item each  $G_v$ for $v\in V(\Gamma)$ is a hyperbolic, virtually compact special group,
\item each $G_e$ for $e\in E(\Gamma)$ is quasiconvex in its vertex groups,
\item $(G, \{G_v\}_{v\in V(\Gamma)})$ has finite stature.
\end{enumerate}
Then $G$ is residually finite.
\end{thm}

In particular, the first two conditions are automatically satisfied for any finite graph of finite rank free groups. 
We also note the following characterization of finite stature in terms of edge stabilizers in the action of $G$ on the Bass-Serre tree associated to the splitting. All the stabilizers considered in this paper are \emph{pointwise} stabilizers.

\begin{lem}[{\cite[Lem 3.19]{HuangWiseStature19}}]\label{lem:HuangWise stabilizer lemma}
Let $T$ be the Bass-Serre tree of the splitting of $G$ as a graph of groups with the underlying graph $\Gamma$. Then $(G, \{G_v\}_{v\in V(\Gamma)})$ has finite stature if and only if for each $v\in V(\Gamma)$, there are only finitely many conjugacy classes of groups of the form $G_v\cap\bigcap_{e\in E} \Stab(e)$ where $E\subseteq E(T)$. 

Moreover, if all the vertex groups are hyperbolic and edge groups are quasiconvex, then it suffices to only consider finite subsets $E\subseteq E(T)$.
\end{lem}

We note that we can identify $G_v$ with $\Stab(\tilde v)$ for some $\tilde v\in V(T)$. In fact, every conjugate of a vertex group of $G$ can be identified with $\Stab(\tilde v)$ for some $\tilde v\in V(T)$. We explain in more detail, how one can think of the intersections of conjugates of vertex groups. 

We will denote the pointwise stabilizer of a path $\rho$ in $T$ by $\Stab(\rho)$, i.e.\ $\Stab(\rho) = \bigcap_{e\in \rho}\Stab(e)$. Using the identification of $G_v$ with $\Stab(\tilde v)$, we can view $\Stab(\rho)$ as a subgroup of $G_v$ if $\tilde v$ is a vertex of $\rho$. To emphasize that, we will denote such a subgroup by $G_v\cap \Stab(\rho)$.
If $\rho, \rho'$ both pass through $\tilde v$ and $\rho\subseteq \rho'$, then $G_v\cap\Stab(\rho')\subseteq G_v\cap\Stab(\rho)$. 


\begin{prop}\label{prop: conjugacy classes of path stabilizers} Let $G$ be a graph of finite rank free groups with finite rank edge groups, and let $T$ be its Bass-Serre tree. Then $(G, \{G_v\}_{v\in V(\Gamma)})$ has finite stature if and only if there are only finitely many $G$-conjugacy classes of groups of the form $G_{\tilde v}\cap \Stab(\rho)$ where $\rho$ is a finite path in $T$ passing through $\tilde v$. 
\end{prop}

\begin{proof} Since free groups are hyperbolic and locally quasiconvex, Lemma~\ref{lem:HuangWise stabilizer lemma} implies that it suffices to show that there are only finitely many conjugacy classes of groups of the form $G_{\tilde v}\cap\bigcap_{e\in E} \Stab(e)$ for $v\in V(\Gamma)$, and finite $E\subseteq E(T)$ if and only if there are only finitely many conjugacy classes of groups of the form $G_v\cap \Stab(\rho)$ where $\rho$ is a finite path that passes through $v$.

The forward implication is immediate. Let us assume there are only finitely many conjugacy classes of groups of the form $G_v\cap \Stab(\rho)$ where $\rho$ is a finite path passing through $\tilde v$.
  %First, we note that every group of the form $G_v\cap \Stab(\rho)$ is a finite rank subgroup of $G_v$. Indeed, intersection of any finite number of finite rank subgroups of a free group has finite rank. If $G_v\cap \Stab(\rho)$ had infinite rank for some infinite path $\rho$, the subgroups $G_v\cap \Stab(\rho_n)$ where $\rho _n$ is the initial subpath of $\rho$ of length $n$ would belong to infinitely many conjugacy classes, as their rank would tend to infinity as $n$ goes to infinity.
The group $G_v\cap\bigcap_{e\in E} \Stab(e)$ is exactly the subgroup of $G$ stabilizing the vertex and all the edges in $E$. In particular, it can be realized as the subgroup of $G$ stabilizing the union of paths $\{\rho_e\}_{e\in E}$ where $\rho_e$ is the minimal path containing $v$ and the edge $e$, i.e.\ $G_v\cap\bigcap_{e\in E} \Stab(e) = G_v \cap \bigcap_{e\in E}\Stab (\rho_e) =  \bigcap_{e\in E}\left(G_v \cap \Stab (\rho_e)\right)$. Since there are only finitely many conjugacy classes of group of the form $G_v \cap \Stab (\rho)$, there are also only finitely many conjugacy classes of their intersections by Lemma~\ref{lem: finitely many conjugacy classes of intersections} .\end{proof}

%\begin{lem}
%Let $v, v'\in T$. Then $\Stab(v) \cap \Stab(v') = \bigcap_{e\in \rho}\Stab(e)$ where $\rho$ is the unique geodesic from $v$ to $v'$ in $T$, viewed as a collection of edges. 
%\end{lem}
%\begin{proof}
%If $g\in G$ stabilizes vertices $v,v'$, then it must also pointwise stabilize the geodesic path $\rho$ between them. 
%\end{proof}
%\begin{com} Do we need this lemma?
%\end{com}
%
%We note that 
%
%\begin{thm}
%Suppose that there exists a uniform bound $N$ such that for all 
%\end{thm}

%In the next proposition, we continue to denote the initial length $n$ subpath of a path $\rho$ by $\rho_n$.
%
%\begin{prop} Let $G$ be the fundamental group of a finite graph of finite rank free groups with finite rank edge groups, and let $T$ be the associated Bass-Serre tree of $G$. Suppose there exists a constant $k$ such that for every path $\rho$ starting at $v$ in $T$ the sequence of group $G_v\cap\Stab(v) \supseteq G_v\cap\Stab(\rho_{1}) \supseteq G_v\cap\Stab(\rho_{2}) \supseteq \dots \supseteq G_v\cap\Stab(\rho_{n}) \supseteq \dots $ stabilizes in at most $k$ steps. Then $G$ has finite stature with respect to its vertex groups.
%\end{prop}
%
%\begin{proof}
%We use Lemma~\ref{prop: conjugacy classes of path stabilizers} to deduce that $G$ has finite stature. By assumption, $G_{\tilde v}\cap \Stab(\rho) = G_{\tilde v}\cap \Stab(\rho_k)$ so it suffices to show that there are only finitely many conjugacy classes of the groups of the form $G_{\tilde v}\cap \Stab(\rho)$ where $\rho$ is a path passing through $\tilde v$ of length at most $k$.
%
%We prove by induction on the length of $\rho$ that there are only finitely many conjugacy classes of the groups of the form $G_{\tilde v}\cap \Stab(\rho)$ and each such subgroups has finite rank. If the length of $\rho$ is $1$, then $G_{\tilde v}\cap \Stab(\rho) = \Stab(e)$ where $e$ is the only edge in $\rho$, so clearly there are only finitely many conjugacy classes of such groups since we assume that the underlying graph in the graph of groups structure of $G$ is finite. Also by assumption $G_{\tilde v}\cap \Stab(\rho)$ is a finite rank subgroup of $G_v$.
%
%Suppose that for every path $\rho'$ of length $i-1$, there are finitely many conjugacy classes of groups of the form $G_{\tilde v}\cap \Stab(\rho')$. Let $\rho$ be a path starting at $\tilde v$ that have length $i$, and let $\tilde v'$ be its second vertex. Let $\rho'$ be the subpath of $\rho$ of length $i-1$ starting at $\tilde v'$, and let $e$ be the remaining edge of $\rho$. By inductive assumption, there are only finitely many conjugacy classes of the groups of the form $G_{\tilde v'}\cap \Stab(\rho')$, and each such subgroup has finite rank. By Lemma~\ref{lem: finitely many conjugacy classes of intersections} there are only finitely many conjugacy classes of the intersections of conjugates $G_{\tilde v'}\cap \Stab(\rho')$ and $G_{\tilde v'}\cap \Stab(e)$. Therefore, there are only finitely many conjugacy classes of the subgroups pf the form $G_{\tilde v'}\cap \Stab(\rho)$ and each such subgroups has finite rank. Since the group $G_{\tilde v'}\cap \Stab(\rho)$ is contained in $\Stab(e)$ we can also view it as a subgroup of $G_v$ and denote it by $G_{\tilde v}\cap \Stab(\rho)$.
%
%\begin{com} everything right?
%\end{com}
%\end{proof}


We finish this section with the following observation that will allow us to work with certain finite index subgroups of the considered groups.

\begin{prop}[Passing to finite index supergroups]\label{prop:passing to finite index}
Let $G$ split as a graph of groups. %and $T$ be the corresponding Bass-Serre tree. 
If $G'$ is a finite index subgroup of $G$ such that $G'$ has finite stature with respect to the vertex groups in the induced graph of groups decomposition, then $G$ has finite stature with respect to its vertex groups.
 \end{prop}
\begin{proof}
This follows immediately from the characterization of finite stature in terms of the of number of orbits of based big trees in the sense of \cite[Def 3.7]{HuangWiseStature19}, see \cite[Lem 3.9]{HuangWiseStature19}.
\end{proof}



\section{Graphs of free groups}\label{sec: graphs of groups}
\subsection{Amalgamated products $A*_CB$ where $[B:C]=2$}
Let $G = A*_CB$ be an amalgamated product of finite rank free groups, where $[B:C] = 2$. Let $b\in B-C$, i.e.\ $bC$ is the nontrivial coset of $C/B$.
%%%


Let $T$ be the Bass-Serre tree of $G$ (metrized so that each edge of $T$ has length $1$).
The vertices of $T$ are of two kinds: infinite valence \emph{$A$-vertices}, corresponding to conjugates of $A$, and valence two  \emph{$B$-vertices} corresponding to conjugates of $B$. 
The edges of $T$ correspond to conjugates of $C$. We use the convention where $C^g$ denotes the conjugate $gCg^{-1}$. We emphasize that using this notation, we have $(C^g)^h = C^{hg}$, but on the other hand it makes the statement of Lemma~\ref{lem:stabilizers as intersections} below slightly simpler.

We start with the following observation.

\begin{lem}\label{lem:stabilizers of adjacent edges}
 An element $g\in G$ stabilizes an edge $e$ of $T$ if and only if $g$ stabilizes an adjacent edge $e'$ meeting $e$ at a $B$-vertex. 
\end{lem}
\begin{proof} 
Since the vertex incident to both $e$ and $e'$ has valence $2$, any element stabilizing one of the edges must stabilize the other one as well.
\end{proof}

As a consequence of the above lemma, for every path $\rho'$ in $T$ $\Stab(\rho') = \Stab(\rho)$ where $\rho$ is the minimal path containing $\rho'$ that starts and ends at $A$-vertices.
Thus we will only consider paths in $T$ starting and ending at $A$-vertices. 
We continue measuring the length of paths with respect to the original metric on the tree, i.e.\ any two $A$-vertices are even distance away.

In the following lemma, we describe all the stabilizers of paths in $T$ joining two $A$-vertices. In our application, we will only need the statement for the paths of length at most $8$, so we give explicit description in those cases, but for completeness we also give the general statement for paths of arbitrary length. 

\begin{lem}\label{lem:stabilizers as intersections}
Let $\rho$ be a length $2\ell$ path in $T$ starting and ending at $A$-vertices. Then $\Stab(\rho)$ is a conjugate of a subgroup $K_\ell\subseteq C$ of the form: 
\begin{itemize} 
\item $K_1 = C$
\item $K_2 = C^{a_1} \cap C$ for some $a_1\in A$
\item $K_3 = C^{a_1}\cap C\cap C^{bd_1}$ for some $a_1, d_1\in A$
\item $K_4 = C^{a_1ba_2}\cap C^{a_1}\cap C \cap C^{bd_1}$ for some $a_1, a_2, d_1\in A$
\end{itemize}
and more generally, 
\begin{itemize} 
\item for $\ell = 2k+1$:
$$K_{2k+1} =C^{a_1b\dots ba_{k-1}ba_k}\cap C^{a_1b\dots ba_{k-1}}\cap \dots\cap C^{a_1}\cap C \cap C^{bd_1}\cap C^{bd_1bd_2}\cap \dots \cap C^{bd_1\dots bd_{k}}$$
for some $a_1, \dots, a_k, d_1, \dots, d_k\in A$;
%$$K_{\ell} =C^{a_{\ell-2}ba_{\ell-4}b\dots ba_1}\cap C^{a_{\ell-4}b\dots ba_1}\cap \dots\cap C^{a_1}\cap C \cap C^{a_2b}\cap C^{a_4ba_2b}\cap \dots \cap C^{a_{\ell-1}b\dots ba_{2}b}$$
\item for $\ell = 2k$:
 $$K_{2k} =C^{a_1b\dots ba_{k-1}ba_k}\cap C^{a_1b\dots ba_{k-1}}\cap \dots \cap C^{a_1}\cap C \cap C^{bd_1}\cap C^{bd_1bd_2}\cap \dots \cap C^{bd_1\dots bd_{k-1}}$$
 for some $a_1, \dots, a_k, d_1, \dots, d_{k-1}\in A$.
% $$K_{\ell} =C^{a_{\ell-1}ba_{\ell-3}b\dots ba_1}\cap C^{a_{\ell-3}b\dots ba_1}\cap \dots \cap C^{a_1}\cap C \cap C^{a_2b}\cap C^{a_4ba_2b}\cap \dots \cap C^{a_{\ell-2}b\dots ba_{2}b}$$
\end{itemize}
Additionally, we have the following, where $K_{\ell}$ and $K_{\ell}'$ denote two groups of the form as above (for possibly different choices of elements $a_i$'s and $d_i$'s.).
%\begin{itemize}
%\item $K_2 = K_1\cap (K_1')^{a_1}$
%\item $K_3 = C^{a_1}\cap C \cap C^{bd_1} = (C^{a_1}\cap C) \cap  (C^{d_1}\cap C)^b = K_2\cap (K_2')^b$
%\item $K_4 = C^{a_1ba_2}\cap C^{a_1}\cap C \cap C^{bd_1} = (C^{a_1}\cap C \cap C^{bd_1}) \cap  (C^{a_1^{-1}} \cap C \cap C^{ba_2})^{a_1} = K_3\cap (K_3')^{a_1}$
%\end{itemize}
%and more generally,
\begin{itemize}
\item  if $\ell$ is odd, then $K_\ell = K_{\ell-1}\cap (K'_{\ell-1})^b$
\item  if $\ell$ is even, then $K_\ell = K_{\ell-1}\cap (K'_{\ell-1})^{a}$
\end{itemize}
\end{lem}

\begin{proof} Since the length of $\rho$ is $2\ell$, so always even, the middle point of $\rho$ is always a vertex in $T$. Depending on parity of $\ell$, the middle vertex can be an $A$-vertex or a $B$-vertex. If $\ell$ is even, the middle vertex of $\rho$ is an $A$-vertex, and by conjugating $\Stab(\rho)$, we can assume that the middle vertex of $\rho$ is stabilized by $A$, and that the following vertex is stabilized by $B$ (see initial subpath of length $4$ of the path in Figure~\ref{fig: length 6 path} for an example with $\ell=2$). If $\ell$ is odd, the middle vertex of $\rho$ is a $B$-vertex, and by conjugating $\Stab(\rho)$, we can assume that it is stabilized by $B$, and the vertex before is stabilized by $A$ (see Figure~\ref{fig: length 6 path} for an example where $\ell=3$). 
% Figure environment removed

Since $\Stab(\rho) = \bigcap_{e\in \rho}\Stab(e)$, b analyzing the stabilizers of edges in $\rho$, we get the description of $\Stab(\rho)$ as required.

Let us now prove the second part of the statement. First assume that $\ell=2k+1$. Then 
\begin{align*}
K_{2k+1} = &(C^{a_1b\dots ba_{k-1}ba_k}\cap C^{a_1b\dots ba_{k-1}}\cap \dots\cap C^{a_1}\cap C \cap C^{bd_1}\cap C^{bd_1bd_2}\cap \dots \cap C^{bd_1\dots bd_{k-1}})\cap \\
&\cap (C^{d_1b\dots bd_{k-1}bd_k}\cap \dots\cap C^{d_1}\cap C \cap C^{b(b^{-2}a_1)}\cap C^{b(b^{-2}a_1)ba_2}\cap \dots \cap C^{b(b^{-2}a_1)\dots ba_{k-1}})^b
\end{align*}
%\begin{align*}
%K_{\ell} = &(C^{a_{\ell-2}ba_{\ell-4}b\dots ba_1}\cap C^{a_{\ell-4}b\dots ba_1}\cap \dots\cap C^{a_1}\cap C \cap C^{a_2b}\cap C^{a_4ba_2b}\cap \dots \cap C^{a_{\ell-3}b\dots ba_{2}b}) \cap \\
%&(C^{a_{\ell-1}ba_{\ell-3}b\dots ba_2}\cap C^{a_{\ell-3}b\dots ba_2}\cap \dots \cap C^{a_2}\cap C \cap C^{(a_1b^{-2})b}\cap C^{a_3b(a_1b^{-2})b}\cap \dots \cap C^{a_{\ell-3}b\dots b(a_1b^{-2})b})^b
%\end{align*}
We note that $b^{-2}a_1\in A$ since $b^2\in C$, so the expression above is indeed of the form $K_{2k}\cap (K'_{2k})^b$.
Similarly, when $\ell = 2k$, we get
\begin{align*}
K_{2k} = &(C^{a_1b\dots ba_{k-1}}\cap C^{a_1b\dots ba_{k-2}}\cap \dots\cap C^{a_1}\cap C \cap C^{bd_1}\cap C^{bd_1bd_2}\cap \dots \cap C^{bd_1\dots bd_{k-1}}) \cap \\
&\cap (C^{a_1^{-1}bd_1\dots bd_{k-2}}\cap \dots\cap C^{a_1^{-1}}\cap C\cap C^{ba_2}\cap \dots \cap C^{ba_2b\dots ba_{k}} )^{a_1}
\end{align*}
%\begin{align*}
%K_{\ell} = &(C^{a_{\ell-3}ba_{\ell-5}b\dots ba_1}\cap C^{a_{\ell-5}b\dots ba_1}\cap \dots \cap C^{a_1}\cap C \cap C^{a_2b}\cap C^{a_4ba_2b}\cap \dots \cap C^{a_{\ell-2}b\dots a_{2}b}) \cap\\
%&(C^{a_{\ell-4}b\dots a_{2}ba_1^{-1}}\cap \dots\cap C^{a_2ba_1^{-1}}\cap C^{a_1^{-1}}\cap C\cap C^{a_3b} \cap \dots \cap C^{a_{\ell-3}b\dots ba_3b} \cap C^{a_{\ell-1}b\dots ba_3b})^{a_1}
%\end{align*}
which gives as $K_{\ell} = K_{\ell-1}\cap (K'_{\ell-1})^{a}$ for $a=a_1$ as required.
\end{proof}

We emphasize that subgroups $K_{\ell}$ in the above statement are not uniquely defined, i.e.\ they depend on the choice of elements $a_i$ and $d_i$. 

%We now turn into geometric way of representing stabilizers $\Stab(\rho)$. 
%We will make some additional assumption on the splitting $G = A*_CB$ that, as we will later see, are satisfied in our application. Let $X_A, X_B, X_C$ be graphs with $\pi_1 X_A = A$,  $\pi_1 X_B = B$, and $\pi_1 X_C = C$. We assume that $X_C\to X_B$ is a (two-sheeted since $[B:C]=2$) cover. Let $\beta: X_C\to X_C$ be an order two Deck transformation corresponding to an element $b\in B\setminus C$.
%Let $\phi:X_C\to X_A$ be a monotone map that induces the inclusion $C\to A$. By Lemma~\ref{lem:factorization} $\phi$ factors as $\iota\cdot \sigma$. Since






%
%Now we state a criterion which will allow to deduce that certain graph of groups with well-behaved stabilizers of finite paths have finite stature. Let $v$ be an $A$-vertex of $T$, and identify $G_v$ with $A$. For any path $\rho$ in $T$ whose second vertex is $v$, let $\overline X_{\rho}$ be a graph mapping to $X_A$ such that $\overline X_{\rho}\to X_A$ is a combinatorial immersion inducing the inclusion $\Stab(\rho) \to A$.
%
%\begin{prop} Fix $k\geq 3$.
%Suppose that for every length $k$ path $\rho$ in $T$ starting at $v$, the natural map $\overline X_{\rho}\to X_{\rho'}$ is an inclusion, where $\rho'$ is the initial subpath of $\rho$ of length $k-1$. Then $G$ has finite stature.
%\end{prop}
%
%\begin{proof}
%
%We prove that for every path $\rho$ of length $\geq k$ the map $\overline X_{\rho}\to X_{\rho'}$ is an inclusion where $\rho'$ is the initial subpath of $\rho$ of length $k-1$. 
%%By assumption, this holds when the length of $\rho$ is exactly $k$. If 
%
%The proof is by induction on the length of $\rho$. The base case corresponds to the case where $\rho$ has length $k$, and follows directly from the inductive assumption. 
%Suppose that the statement is true for any path of length $n-1$. Let $\rho$ be a path of length $n$, and denote the subpath of $\rho$ of length $n-1$ starting at the second vertex $\widetilde v'$ by $\rho"$. By induction,  
%
%
%
%\begin{com}
%{\color{red}
%incomplete}
%\end{com}
%\end{proof}
%
%\begin{prop}
%Suppose that for every length $4$ path $\rho$ in $T$ starting at $v$, the natural map $\overline X_{\rho}\to X_{\rho'}$ is an inclusion, where $\rho'$ is the initial subpath of $\rho$ of length $2$. Then $G$ has finite stature.
%\end{prop}
%\begin{proof}
%By Lemma~\ref{lem:}
%\end{proof}
%

\subsection{Monochrome cycles preserving splittings}\label{sec:color respecting splittings}

We start with recalling the definition of a graph of spaces in the special cases where all the vertex and edge spaces are graphs.
A \emph{graph of graphs} $X(\Gamma)$ consist of the following data:
\begin{itemize}
\item a combinatorial graph $\Gamma$,
\item for every $v\in V(\Gamma)$, a metric graph $X_v$,
\item for every edge $e\in E(\Gamma)$, a metric graph $X_e$ such that $X_e \stackrel{\beta}{\simeq} X_{\bar e}$,
and a monotone map $\phi_e: X_e\to X_{\tau(e)}$.
\end{itemize}

We emphasize that we do not require $\phi_e$ to be a combinatorial map. 

Let $[n]$ denote the set $\{1,\dots, n\}$. 
An \emph{edge coloring} of a metric graph $X$ is a maps $c:\{\text{1-cells of } X\}\to [n]$. 
We refer to $i\in [n]$ as \emph{colors}.
A cycle in a graph $X$ is \emph{monochrome} if each edge in the cycle has the same color. 
Suppose graphs $X, X'$ admit edge colorings $c,c'$ with colors $[n]$ respectively. A monotone map $\phi:X\to X'$ is \emph{color-preserving}, if $c'(\phi(e)) = c(e)$ for every $1$-cell $e$ of $X$. A \emph{color-preserving isomorphism} is a combinatorial map which is bijective on both vertex-sets and edge-sets, which is color-preserving.


\begin{defn}[Monochrome cycles preserving graph of graphs]
Fix $n\geq 1$ and for each $i\in [n] =\{1,\dots, n\}$ let $\ell_i\geq 1$. %We will refer to $i\in [n]$ as a \emph{color}, and $\ell_i$ the \emph{length associated to the color $i$}.
Let $X(\Gamma)$ be a graph of graphs, where for each $y\in V(\Gamma)\cup E(\Gamma)$ there exists a coloring $c_y: \{\text{1-cells of }X_y\}\to [n]$, and if $y\in E(\Gamma)$ then $c_y = c_{\bar y}$.
A graph of graphs $X(\Gamma)$ is \emph{monochrome cycles preserving} if 
\begin{itemize}
\item for every $e\in E(\Gamma)$, $\phi_e$ is color-preserving, and 
\item for each $i\in [n]$ and each $y\in V(\Gamma)\sqcup E(\Gamma)$, the preimage $c_y^{-1}(i)\subseteq X_y$ is a disjoint union of embedded cycles,
\item for $e\in E(\Gamma)$, the map $\phi_e$ restricted to each cycle of color $i$ factors through a cycle of length $\ell_i$ in the factorization provided by Proposition~\ref{prop:factorization}.
\end{itemize}
\end{defn}


We can visualize such graphs of groups as having edges in vertex and edge graphs colored in a way that the induced colorings of edges in the edge graphs is consistent with respect to both adjacent vertex graphs. 
Note that in particular, each vertex and edge graph in a monochrome cycles preserving graph of graphs is a union of monochrome cycles.
The third condition can be thought of stating that each cycle of a given color in an edge graph has length $\ell_i$ in the metric induced by each vertex group. We note that this length does not need to correspond to the combinatorial length of that cycle, as the attaching maps $\phi_e$ do not need to be combinatorial. We make this (and more general) statement more precise in Lemma~\ref{lem:preserved length of colored cycles}. 
Instead of providing any examples now, we refer the reader to Section~\ref{sec: Artin} and splittings of Artin groups, induced by monochrome cycles preserving graph of graphs. They are the motivation for the above definition.



We will denote the associated graph of group by $G(\Gamma)$.


In the next couple of Lemmas, we assume that $\rho\subseteq T$ is a path in the Bass-Serre tree of $G(\Gamma))$ passing through the vertex $\tilde v$, and an edge $\tilde e$ containing $\tilde v$. 
We identify the stabilizer $\Stab{\tilde v}$ with $G_v$ for some $v\in V(\Gamma))$, and the $\Stab(\tilde e)$ with $G_e$ for some $e \in E(\Gamma))$. 
We view the stabilizer $\Stab(\rho) =\bigcap_{e\subseteq \rho}\Stab(e)$ as a subgroup of $\Stab(\tilde e) = G_e$.

Since we are assuming that $G_v$ is the fundamental group of the graph $X_v$, the inclusion of $\Stab(\rho)$ in $G_v$ can be represented by the precover $\phi:Y_\rho \to X_v$, where $Y_\rho$ is the core of $\Stab(\rho)$ with respect to $X_v$. %Let $\overline \Gamma_\rho$ be the folded and collapsed image of $\Gamma_\rho$ with respect to $\phi_\rho$.
Let $\phi:Y_\rho \xrightarrow{\sigma} \overline Y_\rho\xrightarrow{\iota} X_v$ be a factorization of $\phi$ provided by Proposition~\ref{prop:factorization}.

\begin{lem}\label{lem:preserved length of colored cycles}
The graph $\overline Y_\rho$ with the coloring induced from $X_v$, is a union of monochrome cycles, where each cycle of color $i$ has length $\ell_i$. 
\end{lem}

\begin{proof} By definition, $\phi_e:X_e\to X_{\tau(e)}$ factors as $X_e\xrightarrow{\sigma} \overline X_e\xrightarrow{\iota} X_{\tau(e)}$ where $\overline X_e$ is a disjoint union of monochrome cycles, with each cycle of color $i$ having length $\ell_i$. 
For any subgroup of $H\subseteq G_e$ whose corresponding graph $Y$ is a disjoint union of monochrome cycles, we have that each cycle of color $i$ in $\overline Y = \sigma(Y)$ has length $\ell_i$ if and only if each cycle of color $i$ in $\sigma\beta(Y)$ has length $\ell_i$ (here we abuse the notation and extend the isomorphism $\beta:X_e\to X_{\overline{e}}$ to all precovers of $X_e$). In other words the precovers induced by $Y$ either have each cycle of color $i$ having length $\ell_i$ in both vertex groups $G_{\tau(e)}$, $G_{\tau(\overline e)}$, or do not have this property in both vertex groups.

An intersection of any two adjacent edge groups corresponds to the fiber product of the corresponding graphs. The fiber product of two graphs that are both unions of monochrome cycles, with each cycle of color $i$ having length $\ell_i$, is also a union of monochrome cycles, with each cycle of color $i$ having length $\ell_i$.
Since $\Stab(\rho) = \bigcap_{e\in\rho} \Stab(e)$, the core $\overline Y_\rho$ is obtained in a multiple steps by taking fiber products of graphs that are unions of monochrome cycles, where each cycle of color $i$ has length $\ell_i$. The conclusion follows.
\end{proof}

As a consequence of Lemma~\ref{lem:preserved length of colored cycles}, we can view every $\overline Y_\rho$ as the $1$-skeleton of a $2$-complex $\widetriangle Y_\rho$ obtained by attaching $\ell_i$-gons of color $i$ along each monochrome cycle of color $i$. %Indeed, $\widetriangle\Gamma_\rho$ is obtained from $\overline\Gamma_\rho$ by attaching $2$-cells along all cycles of color $i$ and length $\ell_i$. 

If $\rho\subseteq \rho'$, then $\Stab(\rho')\subseteq \Stab(\rho)$ and so there is a map of precovers $\overline Y_{\rho'}\to \overline Y_{\rho}$ of $X_v$.

\begin{lem}\label{lem:simply connected Y triangle}
 Suppose $\rho\subseteq \rho'$ and $\widetriangle Y_\rho$ is simply connected. Then the precover $\overline Y_{\rho'}\to \overline Y_{\rho}$ is  an embedding of a subgraph.
\end{lem}
\begin{proof} %Since $\rho\subseteq \rho'$, the group $\Stab(\rho') = \bigcap_{e\subseteq \rho'}\Stab(e)\subseteq \bigcap_{e\subseteq \rho}\Stab(e) = \Stab(\rho)$, and so the graph $\bar\Gamma_{\rho'}$ is a precover of $\bar\Gamma_{\rho}$. 
Since $\widetriangle Y_{\rho}$ is simply connected, it follows that $\widetriangle Y_{\rho'}\subseteq \widetriangle Y_{\rho}$, and so $\overline Y_{\rho'}\subseteq \overline Y_{\rho}$.
\end{proof}




\section{Finite stature in triangle Artin groups}\label{sec: Artin}
\subsection{The statement of the main result}
A \emph{triangle Artin group} is given by the presentation
$$G_{MNP} = \langle a, b, c\mid (a, b)_{M} = (b, a)_{M}, (b,c)_{N} = (c,b)_{N}, (c,a)_{P} = (a,c)_{P}\rangle,$$
where $(a,b)_M$ denote the alternating word $aba\dots$ of length $M$. 


%The \emph{double of $A$ along $C$ twisted by $\beta$}, 
%denoted by $D(A,C, \beta)$ 
%is a free product with amalgamation $A*_CA$ 
%where $C$ is mapped to the first factor via the natural inclusion $C\hookrightarrow A$, 
%and to the second factor via the natural inclusion precomposed with $\beta$.
The following theorem describes a splitting of $G_{MNP}$ as an amalgamated product of free groups, where the map from the amalgamating subgroup to the vertex groups is described in terms of maps between graphs.

\begin{thm}[{\cite[Cor 4.13]{JankiewiczArtinRf}}]\label{thm:splitting}
Let $G_{MNP}$ be an Artin group where $M,N,P\geq 3$.  Then $G_{MNP} = A*_CB$ where $A\simeq F_3$, $B\simeq F_4$ and $C\simeq F_7$, and $[B:C] = 2$. The map $C\to A$ is induced by the map $\phi: X_C \to X_A$ pictured in Figure~\ref{fig:mapCtoA}, and the map $C\to B$ is induced by the quotient of the graph $X_C$ by a $\pi$ rotation.
% Figure environment removed
%Moreover, $A_{MNP}$ has an index two subgroup that is isomorphic to the twisted double $D(A,C,\beta)$ where $\beta:C\to C$ is given by $\beta(h) = g^{-1}hg$ for some (equivalently any) $g\in B-C$.
\end{thm}

\begin{thm}[{\cite[Prop 2.8]{JankiewiczArtinSplittings}}]\label{thm:splitting2}
Let $G_{MNP}$ be an Artin group where $M,N\geq 4$  and $P=2$. 
\begin{itemize}
\item If at least one of $M,N$ is odd, then $G_{MNP} = A*_CB$ where $A\simeq F_2$, $B\simeq F_3$ and $C\simeq F_5$, and $[B:C] = 2$. The map $C\to A$ is induced by the map $\phi: X_C \to X_A$ pictured in Figure~\ref{fig:2MN}, and the map $C\to B$ is induced by the quotient of the graph $X_C$ by a $\pi$ rotation.
% Figure environment removed
\item If both $M,N$ are even then $G_{MNP} = A*_B$ where $A\simeq F_2$, $B\simeq F_3$. The two maps $B\to A$ are induced by the maps $\phi_1, \phi_2: X_B \to X_A$ pictured in Figure~\ref{fig:2eveneven}.
% Figure environment removed
\end{itemize}
\end{thm}

Here is a precise statement of the main theorem of this paper (Theorem~\ref{thm: main}).
\begin{thm}
Let $G_{MNP}$ be a triangle Artin group where $M\leq N\leq P$ and either $M>2$, or $N>3$. Then $G_{MNP}$ has finite stature with respect to $\{A\}$, where $A$ is as described in Theorem~\ref{thm:splitting} or Theorem~\ref{thm:splitting2} respectively. All finitely generated subgroups of $A$ are separable in $G_{MNP}$, and in particular $G_{MNP}$ is residually finite.
\end{thm}
In subsections~\ref{sec:at least one even}, \ref{sec:two 4}, \ref{sec:all odd} and \ref{sec:2} we will prove groups $G_{MNP}$ as above have finite stature with respect to $\{A\}$ by analyzing various cases (see Proposition~\ref{prop: at least one even and not 2m+1,4,4}, Corollary~\ref{cor:442m+1}, Corollary~\ref{cor: all odd}, and Corollary~\ref{cor: 2MN}). The separability of finitely generated subgroups of $A$ is a consequence of \cite[Thm 1.3]{HuangWiseStature19}.

\subsection{Some facts about the splittings of Artin groups}
We start with some facts that will be used in the next sections.
We first focus on the cases where $G_{MNP}$ splits as $A*_CB$.
Let $\beta: X_{C}\to X_C$ be the $\pi$-rotation, as in Theorem~\ref{thm:splitting} or Theorem~\ref{thm:splitting2} respectively. 
A choice of a path between $x\in X_C$ and $\beta(x)\in X_C$ determines an element $b\in B-C$, such that the induced homomorphism $C \to C$ is the conjugation by $b$.
Figure~\ref{fig:mapCtoA} and Figure~\ref{fig:2MN} illustrate the factorization $\phi = \iota \circ \sigma$ from Proposition~\ref{prop:factorization}.
We denote $\sigma(X_C) = \overline X_C$. 

We will also extend the definition of $\sigma$ to any precover of $X_C$ (and abuse the notation) in the following way. Given a precover $Y\to X_C$, let $\overline Y\to \overline X_C$ be a precover, and let $\sigma:Y\to \overline Y$ be a composition of edge-subdivisions, Stallings' folds, and edge-collapses, which locally coincides with $\sigma: X_C\to \overline X_C$. In particular, the following diagram commutes.
\[\begin{tikzcd}
Y \arrow{r}{\sigma} \arrow{d}{} & \overline Y \arrow{d}{} \\
X_C\arrow{r}{\sigma}& \overline X_C.
\end{tikzcd}
\]



We note the following. 
%For any precover $Y\to X_C$, there is a corresponding precover $\overline Y\to \overline X_C$, and an induced map $Y\to \overline Y$ which locally (on each edge) coincides with the map $\sigma: X_C\to \overline X_C$. We will abuse the notation and denote the induced map by $\sigma$ as well.
\begin{lem}\label{lem:1-to-1 correspondence between cores}
The map $\sigma: Y\to \overline Y$ is a homotopy equivalence for every precover $Y\to X_C$.

For each subgroup $H\subseteq C$ there is a one-to-one correspondence between the core $Y\to X_C$ of $H$ with respect to $X_C$ and the core $\overline Y\to \overline X_C$ of $H$ with respect to $\overline X_C$, where $\overline Y = \sigma(Y)$ as above.
\end{lem}

\begin{proof} The map $\sigma: X_C\to \overline X_C$ is obtained as a sequence of edge-subdivisions and edge-collapses of the edges. By analyzing each of the cases in Figure~\ref{fig:mapCtoA} and Figure~\ref{fig:2MN}, we note that we never collapse a loop. Thus, by discussion in Section~\ref{sec: maps}, $\sigma: X_C\to \overline X_C$ is a homotopy equivalence. Similarly, any induced map $Y\to \overline Y$ is also obtained as a sequence of edge-subdivisions and edge-collapses of the edges that are not loops, and hence $\sigma: Y\to \overline Y$ is a homotopy equivalence. By construction $Y\to X_C$ is the core of some subgroup $H\subseteq \pi_(X_C)$ with respect to $X_C$ if and only if $\overline Y\to \overline X_C$ is the core of $H$ with respect to $\overline X_C$.
\end{proof}

We will use the notation $\sigma^{-1}(\overline Y)$ to denote $Y$ such $\overline Y = \sigma(Y)$. 

\begin{lem}\label{lem: conjugate core}
Let $H\subseteq C$ be a subgroup, and let $Y\to X_C$ be its core with respect to $X_C$. Then $Y\to X_C\xrightarrow{\beta} X_C$ is the core of $H^b\subseteq C$
\end{lem}
\begin{proof}
Indeed, $Y\to X_C$ induces the inclusion $H\to C$ and $X_C\xrightarrow{\beta} X_C$ induces the conjugation by $b$.
\end{proof}

The following lemma will allow us to apply Proposition~\ref{prop: conjugacy classes of path stabilizers}.

\begin{lem}\label{lem:computing stabilizers}
Let $T$ be the Bass-Serre tree of the splitting $G_{MNP} = A*_CB$, and $\rho$ be a finite path in $T$.
The stabilizer $\Stab(\rho)$ of a path $\rho$ of length $2\ell$ between a pair of  $A$-vertices, is conjugate to a subgroup of $C$, represented by a precover $\overline Y_{\ell}\to \overline X_C$, where the corresponding $Y_\ell$ is defined recursively:
\begin{itemize}
\item $Y_1 = X_C$,
\item $Y_{\ell} = \sigma^{-1} (\overline{Y}_{\ell-1}\otimes_{X_A} \overline{Y}_{\ell-1})$ for even $\ell$, 
\item $Y_{\ell} = \sigma^{-1} (\overline Y_{\ell-1}\otimes_{X_A} \sigma\cdot \beta(Y_{\ell-1}))$ for odd $\ell$
\end{itemize}
The map $\overline Y_{\ell-1}\to X_A$ in the recursive definition above is obtained by composing the map $\overline Y_{\ell-1}\to \overline X_C$ with the map $\overline X_C\to X_A$.
%The intersection $C\cap C^a$ for $a\in A$, viewed as a subgroup of $A$, is represented by the combinatorial map from the fiber product $F=\overline X_{C}\otimes _{X_A}\overline X_C$ to $X_A$. The intersection $C^a\cap C^{ba'}$ where $a, a'\in A$, viewed as a subgroup of $A$, is represented by the combinatorial map $\sigma\cdot \beta\cdot \sigma^{-1}(F)\otimes_{X_A} F \to X_A$.
\end{lem}
\begin{proof}
Let $\rho$ be a path of length $2\ell$. By Lemma~\ref{lem:stabilizers as intersections}, $\Stab(\rho)$ is conjugate to a group $K_{\ell}$ defined recursively as 
\begin{itemize}
\item $K_1 = C$,
\item  if $\ell$ is even, then $K_\ell = K_{\ell-1}\cap (K'_{\ell-1})^{a_1}$.
\item  if $\ell$ is odd, then $K_\ell = K_{\ell-1}\cap (K'_{\ell-1})^b$,
\end{itemize}
Clearly, $Y_1 = X_C\to X_C$ is the core of $L_1 = C$ with respect to $X_C$. 
For even $\ell$, $K_\ell = K_{\ell-1}\cap (K'_{\ell-1})^{a}$, so by Lemma~\ref{lem:fiber product} the core of $K_{\ell}$ with respect to $\overline X_C$ is $\overline Y_{\ell-1}\otimes_{X_A} \overline Y_{\ell-1}$, and by Lemma~\ref{lem:1-to-1 correspondence between cores} the core of $K_{\ell}$ with respect to $X_C$ is $\sigma^{-1}( \overline Y_{\ell-1}\otimes_{X_A} \overline Y_{\ell-1})$.
For odd $\ell$, $K_\ell = K_{\ell-1}\cap (K'_{\ell-1})^b$, so by Lemma~\ref{lem:fiber product} and Lemma~\ref{lem: conjugate core} the core $K_{\ell}$ with respect to $\overline X_C$ is $\overline Y_{\ell-1}\otimes_{X_A} \sigma\cdot \beta (Y_{\ell-1})$, and by Lemma~\ref{lem:1-to-1 correspondence between cores} the core of $K_{\ell}$ with respect to $X_C$ is $\sigma^{-1} (\overline Y_{\ell-1}\otimes_{X_A} \sigma\cdot \beta (Y_{\ell-1}))$.
\end{proof}

Again, we emphasize that graphs $Y_{\ell}$ are not uniquely determined, as they are associated to non-unique $K_{\ell}$'s. Since a sequence of group $C = K_1\supseteq K_2\supseteq K_3 \supseteq \dots$ form a descending chain, we have a corresponding sequence of precovers $\dots \xrightarrow{} Y_3 \to Y_2\to Y_1  = X_C$.

\begin{lem}\label{lem: finitely many colored graphs}
Suppose that there are only finitely many 
%color-preserving 
isomorphism types of graphs $Y_{\ell}$ for any $\ell\geq 1$. Then $G_{MNP}$ has finite stature with respect to $\{A\}$.

In particular if there exists $k\geq 1$ such that every map $\overline Y_{k+2}\to \overline Y_k$ is an embedding of a subgraph, then $G_{MNP}$ has finite stature with respect to $\{A\}$.
\end{lem}
\begin{proof}
We first prove the first statement. 
%Consider an index $2$ subgroup $G'$ of $G_{MNP}$ which is isomorphic to $A*_CA$ (see \cite[Cor 4.13]{JankiewiczArtinRf}). 
%Then all the vertex groups in the splitting of $G'$ are $G_{MNP}$-conjugates of $A$.
Let $T$ be the Bass-Serre tree $T$ of $G_{MNP} = A*_CB$, and $\tilde v\in T$ be a vertex whose stabilizer is $A$.
By Lemma~\ref{lem:computing stabilizers} and Lemma~\ref{lem:stabilizers as intersections} there are finitely many conjugacy classes of $\Stab(\rho)$ (viewed as subgroups of $A$) for finite paths $\rho$ in $T$ joining two $A$-vertices and passing through $\tilde v$. Since every finite path passing through $\tilde v$ is contained in such a path joining two $A$-vertices, we conclude that the assumptions of Proposition~\ref{prop: conjugacy classes of path stabilizers} are satisfied. 
 %$G'$ has finite stature with respect to both copies of $A$ in $G'$. 
%Finally, by Proposition~\ref{prop:passing to finite index} 
We deduce that $G_{MNP}$ has finite stature with respect to $\{A\}$.

Now let $k\geq 1$ such that $Y_{k+2}\to Y_k$ is an inclusion. Since $Y_{k+2}$ is obtained from $Y_k$ in two steps as described in Lemma~\ref{lem:computing stabilizers}, we deduce that $Y_{k+2(i+1)}\to Y_{k+2i}$ is an inclusion for each $i\geq 0$. In particular, there can only be finitely many color-preserving isomorphism types of graphs $Y_{k+2i}$ since $Y_{k}$, as a finite graph, has only finitely many subgraphs. Using the formula for $Y_{k+2i+1}$ from Lemma~\ref{lem:computing stabilizers} we deduce that there are finitely many isomorphism types of graphs $Y_{\ell}$ for any $\ell\geq 1$. The conclusion follows from the first part of the lemma.
\end{proof}




\subsection{Monochrome cycle preserving structure of splitting of Artin groups}

\begin{prop}\label{prop:artin monchrome cycles preserving}
Let $G_{MNP}$ be an Artin group where $M\leq N\leq P$ and either $M>2$ or $N>3$. Then $G_{MNP}$ has a subgroup $G'$ of index at most $2$ that is the fundamental group of a monochrome cycles preserivng graph of graphs $X_A\xleftarrow{\phi} X_C\xrightarrow{\beta\cdot\phi } X_A$.
\end{prop}

\begin{proof} Let $G_{MNP} = A*_CB$ as in Theorem~\ref{thm:splitting} or Theorem~\ref{thm:splitting2}. Then $G_{MNP}$ has an index $2$ subgroup $G'$ which splits as $A*_CA$. The associated graph of graphs has two vertices with each vertex graph being a copy of $X_A$, and one edge graph $X_C$. We choose the coloring of $c_A:X_A\to \{\text{red}, \text{green}, \text{blue}\}$, where each loop has distinct color, as in Figure~\ref{fig:mapCtoA} or Figure~\ref{fig:2MN}. Those figure also show how the coloring is $c_C:X_C\to \{\text{red}, \text{green}, \text{blue}\}$ is defined. The two maps $X_C\to X_A$ differ by precomposing one with the automorphism $\beta$ of $X_C$. In particular, both maps $X_C\to X_A$ are color-preserving, and the preimage of each color in $X_C$ is a union of disjoint embedded cycles. Moreover, the maps $X_C\to X_A$ both factor through $\overline X_C$, and in particular, both maps restricted to each cycle factors through a cycle of length $Q$ if $Q$ is odd, and $Q/2$ is $Q$ is even, for $Q = M,N,P$ respectively.
Thus the graphs of graphs $X_A\xleftarrow{\phi} X_C\xrightarrow{\beta\cdot\phi } X_A$ is monochrome cycles preserving.
\end{proof}

%In the next proposition, given a combinatorial immersion $Y\to X_A$, we denote the $2$- complex obtained from $Y$ by attaching $2$-cells whose boundaries have color $i$ and length $\ell_i$ by $\widetriangle Y$, as in Section~\ref{sec:color respecting splittings}.

Every finite path $\rho$ in the Bass-Serre tree of $G_{MNP} = A*_CB$ joining a pair of $A$-vertices can be also thought of as a path in the Bass-Serre tree of the index $2$ subgroup $G' = A*_CA$ of $G_{MNP}$. By Proposition~\ref{prop:artin monchrome cycles preserving} above and Lemma~\ref{lem:preserved length of colored cycles}, for the precover $Y_\rho \to X_A$ of $\Stab(\rho)$ the associated graph $\overline Y_\rho$ is a union of monochrome cycles, where each cycle of color $i$ has length $\ell_i$. We can denote the the $2$- complex obtained from $Y_\rho$ by attaching $2$-cells whose boundaries have color $i$ and length $\ell_i$ by $\widetriangle Y_\rho$, as in Section~\ref{sec:color respecting splittings}. 
\begin{notn}\label{notn}
We now switch to the use of notation of Lemma~\ref{lem:computing stabilizers}, where the graph $Y_\rho$ is denoted by $Y_{\ell}$ where $2\ell = |\rho|$, and the associated $K_{\ell}$ is the stabilizer $\Stab(\rho)$. We will also write $\widetriangle Y_{\ell}$ for $\widetriangle Y_\rho$. Once again, we remind that $Y_{\ell}$, $K_{\ell}$ depend not only on $\ell$, but also the choice of parameters $a_i, d_i$ in their definition, which are equivalent to the choice of $\rho$.
\end{notn}

\begin{lem}\label{lem: finite stature of twisted doubles}
%Let $G=A*_CA$ be an amalgamated product of finite rank free groups, with an index $2$ subgroup that is the fundamental group of a monochrome cycles preserving graph of groups $X_A\xleftarrow{\phi} X_C\xrightarrow{\beta\cdot\phi }$, as in Proposition~\ref{prop:artin monchrome cycles preserving}. 


%Let graphs $Y_{\ell}$ be defined as in Lemma~\ref{lem:computing stabilizers}. 
If for some $\ell\geq 1$ a complex $\widetriangle Y_{\ell}$ is simply connected, then for every $\overline Y_{\ell+2}$, the precover $\overline Y_{\ell+2}\to \overline Y_\ell$ is an embedding of a subgraph. In particular, if there exists $\ell\geq 1$ such that every $\widetriangle Y_\ell$ is simply connected, then $G_{MNP}$ has finite stature with respect to $\{A\}$.
%Suppose that $C\cap C^g$ for $g\in A$ corresponds to a combinatorial immersion of graphs $Y_2\to X_A$ such that $\widetriangle Y_2$ is simply connected. Then $G$ has finite stature with respect to the conjugates of $A$.
\end{lem}
\begin{proof}
The first statement follows directly from Lemma~\ref{lem:simply connected Y triangle}.
%By Lemma~\ref{lem:preserved length of colored cycles}, for each path $\rho$ in the Bass-Serre tree of $G$, in the core $\overline Y_{\rho}$ of $\Stab(\rho)$ (viewed as a subgroup of $A$) with respect to $X_A$, all cycles of color $i$ have length $\ell_i$. 
Since there are only finitely many color-preserving isomorphism types of $\overline Y_k$, there are also only finitely many color-preserving isomorphism types of their subgraphs. Thus if all $Y_k$ are simply-connected, there are only finitely many  color-preserving isomorphism types of graphs that $\overline Y_{\rho}$ might have. It follows that there are only finitely many conjugacy classes of the groups of the form $G_{\tilde v}\cap \Stab(\rho)$. By Proposition~\ref{prop: conjugacy classes of path stabilizers} $G'$ has finite stature with respect to both copies of $A$. By Proposition~\ref{prop:passing to finite index} $G_{MNP}$ also has finite stature with respect to $\{A\}$.
\end{proof}


In the next subsections we apply Lemma~\ref{lem: finitely many colored graphs} or Lemma~\ref{lem: finite stature of twisted doubles} to prove that all the large type triangle Artin group have finite stature. We consider three cases:
\begin{enumerate}
\item[(\ref{sec:at least one even})] at least one $M,N,P\geq 3$ is even and $\{M,N,P\}  \neq \{2m+1,4,4\}$ for $m\geq 1$,
\item[(\ref{sec:two 4})]  $\{M,N,P\}  = \{2m+1,4,4\}$ where $m\geq 1$,
\item[(\ref{sec:all odd})] all $M,N,P$ are odd and $\geq 3$.
\end{enumerate}
We also consider the case where one of the exponents is $2$, and the other two are both strictly greater than $3$:
\begin{enumerate}\setcounter{enumi}{3}
\item[(\ref{sec:2})] $\{M,N,P\}$ where $M,N\geq 4$ and $P=2$. 
\end{enumerate}

\subsection{Case where at least one of $M,N,P\geq 3$ is even and $\{M,N,P\} \neq \{2m+1,4,4\}$}\label{sec:at least one even}
In the next proof, we continue to use Notation~\ref{notn}.

\begin{prop}\label{prop: at least one even and not 2m+1,4,4}
Suppose 
$M, N, P\geq 3$ and at least one of them is even, but $\{M,N,P\} \neq \{2m+1,4,4\}$.
Then $G_{MNP}$ has finite stature with respect to $\{A\}$, where $A$ is as in Theorem~\ref{thm:splitting}.
\end{prop}
\begin{proof} By Theorem~\ref{thm:splitting} in all the cases listed in the statement, $G_{MNP}$ splits as an amalgamated product $A*_C B$ of finite rank free groups where $[B:C] = 2$, which by Proposition~\ref{prop:artin monchrome cycles preserving} is virtually the fundamental group of a monochrome cycles preserving graph of graphs. By \cite[Lem 5.2, 5.3, 5.4]{JankiewiczArtinRf} (see also \cite[Rem 5.5]{JankiewiczArtinRf}) $\widetriangle Y_2$ is simply-connected, where $Y_2\to X_C$ if the core of $C\cap C^g$ with respect to $X_C$, as in Lemma~\ref{lem:computing stabilizers}. By Lemma~\ref{lem: finite stature of twisted doubles} $G_{MNP}$ has finite stature with respect to $\{A\}$.
\end{proof}
We note that the residual finiteness of the Artin groups considered above was also proven in \cite{JankiewiczArtinRf}.

\subsection{Case where $\{M,N,P\}  = \{2m+1,4,4\}$}\label{sec:(2m+1,4,4)}\label{sec:two 4}
We continue to use Notation~\ref{notn}.

\begin{lem}\label{lem:44(2p+1)}
Let $\{M,N,P\}  = \{2m+1,4,4\}$. Every graph $\overline Y_2$ is either the left graph in Figure~\ref{fig:44(2p+1)}, or has simply connected $\widetriangle Y_2$.
Every graph $\overline Y_3$ is either the right graph in Figure~\ref{fig:44(2p+1)}, or has simply connected $\widetriangle Y_3$.
The map $\overline Y_4\to \overline Y_2$ is always an embedding of a subgraph.
% Figure environment removed
\end{lem}

\begin{proof}
By Theorem~\ref{thm:splitting} in all the cases listed in the statement, $G_{MNP}$ splits as an amalgamated product $A*_C B$ of finite rank free groups where $[B:C] = 2$, which by Proposition~\ref{prop:artin monchrome cycles preserving} is virtually the fundamental group of a monochrome cycles preserving graph of graphs.

By Lemma~\ref{lem:computing stabilizers}, $\overline Y_2$ is computed as a connected component of the fiber product $\overline Y_1\otimes_{X_A}\overline Y_1$, which has been done in \cite[Lem 5.1]{JankiewiczArtinRf}. If $\widetriangle Y_2$ is simply-connected, then $\overline Y_4\to \overline Y_2$ is an embedding of a subgraph for every $Y_4$, by Lemma~\ref{lem: finite stature of twisted doubles}.

In the case where $\widetriangle Y_2$ is not simply connected, the resulting map $\overline Y_2\to \overline X_C$ is illustrated as the first vertical arrow in Figure~\ref{fig:proof 44M}. 
Lemma~\ref{lem:1-to-1 correspondence between cores} ensures that $Y_2\to X_C$ can be computed, which is done in the second  vertical arrow in Figure~\ref{fig:proof 44M}. 
Then the rest of Figure~\ref{fig:proof 44M} represent the computation of $\sigma\beta(Y_2)\to \overline X_C$.
Finally, by Lemma~\ref{lem:computing stabilizers}, $\overline Y_3$  is computed as the fiber product  $\sigma\beta(Y_2)\otimes_{X_A} \overline Y_2$, i.e.\ the fiber product of the left top and the right top graphs in Figure~\ref{fig:proof 44M}. 
We deduce that $\overline Y_3$ either has simply connected $\widetriangle Y_3$, or it is the right graph in Figure~\ref{fig:44(2p+1)}.
% Figure environment removed

If $\widetriangle Y_3$ is simply-connected, then so is $\widetriangle Y_4$ and $\overline Y_4\to \overline Y_2$ is  an embedding of a subgraph, as required.
%
Otherwise, $\overline Y_4$ is a connected component of $\overline Y_3\otimes_{X_A} \overline Y_3$ by Lemma~\ref{lem:computing stabilizers}. Note that each connected component $\overline Y_4$ is either equal to $\overline Y_3$, or has simply connected $\widetriangle Y_4$, and in particular, the map $\overline Y_4\to \overline Y_2$ is an embedding of a subgraph. 
\end{proof}

Combining Lemma~\ref{lem:44(2p+1)} and Lemma~\ref{lem: finitely many colored graphs} yields the following.

\begin{cor}\label{cor:442m+1}
The Artin group $G_{MNP}$ where $M = 2m+1\geq 3$ and $N= P=4$ has finite stature with respect to $\{A\}$, where $A$ is as described in Theorem~\ref{thm:splitting}. 
\end{cor}



\subsection{Case where $M,N,P\geq 3$ are all odd}\label{sec:all odd}
First consider the case where $M=N=P=3$. 

\begin{prop}\label{lem:333} 
Let $(M,N,P) =(3,3,3)$, and let $T$ be the Bass-Serre tree of the splitting $G_{333} = A*_CB$. Then for every path $\rho$ in $T$, $\Stab(\rho) = C$.\end{prop}

\begin{proof} Indeed, in this case $C$ is normal in both $A$ and $B$, so all $G_{333}$-conjugates of $C$ are equal $C$. This proves that all edge stabilizers in the action of $G_{333}$ on $T$ are equal $C$.
\end{proof}

For the remaining cases, we will apply Lemma~\ref{lem: finitely many colored graphs} to deduce that $G_{MNP}$ has finite stature with respect to $\{A\}$, similarly as in Section~\ref{sec:(2m+1,4,4)} 
We now consider the case where $M,N,P$ are all at least $5$. We continue to use Notation~\ref{notn}.
\begin{lem}\label{lem:three odd}
Let $M,N,P\geq 5$ be all odd. Every graph $\overline Y_2$ is either the left graph in Figure~\ref{fig:three odd}, or has simply connected $\widetriangle Y_2$.
Also, every graph $\overline Y_3$ is either the right graph in Figure~\ref{fig:three odd}, or has simply connected $\widetriangle Y_3$.
The map $\overline Y_4\to \overline Y_2$ is always an embedding of a subgraph.
 
% Figure environment removed
\end{lem}
\begin{proof}
We write $M=2m+1$, $N=2n+1$, and $P=2p+1$. The first part of the lemma was proven in \cite[Lem 5.1]{JankiewiczArtinRf}. In order to prove the second part we start with computing $\sigma\beta(Y_2)$, which is illustrated in Figure~\ref{fig:three odd proof}. 
% Figure environment removed
We note that there are two connected components $Y_2$ of the fiber product $\overline X_C\otimes_{X_A}\overline X_C$ for which $\widetriangle Y_3$ is not simply connected. They are both color-preserving isomorphic to the left graph in Figure~\ref{fig:three odd}, but their maps to $X_C$ are different. The first column of Figure~\ref{fig:three odd proof} shows the two precovers $\overline Y_2\to \overline X_C$ (they are determined by the coloring of the vertices). For each $Y_2$, we compute $\sigma\beta(Y_2)$, in a similar manner as in Lemma~\ref{lem:44(2p+1)}, see the rest of Figure~\ref{fig:three odd proof}. In each case, we deduce that each connected component $\overline Y_3$ of $\overline Y_2\otimes_{X_A}\sigma\beta(Y_2)$ either has simply connected $\widetriangle Y_3$, or it is the right graph in Figure~\ref{fig:three odd}. In either case, we every map $\overline Y_4\to \overline Y_2$ is an embedding of a subgraph by a reasoning similar to one in  Lemma~\ref{lem:44(2p+1)}.
\end{proof}

We now move to the case where one or two of $M,N,P$ are equal to $3$.
Unlike in the previous case, the computation of the fiber product $\overline X_C\otimes_{X_A}\overline X_C$ in such cases was not included in \cite{JankiewiczArtinRf}. We start with that computation.
\begin{lem}\label{lem:fiberprod}
Suppose one or two of $M,N,P$ are equal to $3$. Every connected component $Y_2$ of $\overline X_C\otimes_{X_A}\overline X_C$ either has simply connected $\widetriangle Y_2$ or is 
\begin{itemize}
\item the left graph in Figure~\ref{fig:fiber product with some 3s}, when $M=3$ and $N, P\geq 5$,
\item the right graph in Figure~\ref{fig:fiber product with some 3s}, when $M=N=3$ and $P\geq 5$,
\end{itemize}
% Figure environment removed
\end{lem}

\begin{proof}
This is a direct computation. We remind that the graph $\overline X_C$ is the middle graph in the first row of Figure~\ref{fig:mapCtoA}. In Figure~\ref{fig:fiber product with some 3s} we bi-colored the vertices of the graphs (i.e.\ colored with a pair of colors) to make it easier for the reader to verify the computation.
\end{proof}

Now our goal is to show that $\overline Y_{\ell+2}\to \overline Y_{\ell}$ is an embedding of a subgraph for some $\ell$, so we can apply Lemma~\ref{lem: finitely many colored graphs}. The case where exactly one of $M,N,P$ is equal $3$ is considered first.

\begin{lem}\label{lem:3(2n+1)(2p+1)}
Let $N,P \geq 5$ be odd, and $M=3$. 
Every $\overline Y_3$ is either a single monochrome cycle or one of the graphs in Figure~\ref{fig:3(2n+1)(2p+1)proof2}, and in particular has simply connected $\widetriangle Y_3$.
   % Figure environment removed
%In particular, every graph $\overline Y_3$ has simply connected  $\widetriangle Y_3$.
 \end{lem}
 \begin{proof}
 We write $N=2n+1$ and $P=2p+1$. By Lemma~\ref{lem:fiberprod}, every $Y_2$ either has simply connected $\widetriangle Y_2$ or is the left graph in Figure~\ref{fig:fiber product with some 3s}. 
There are two components of $\overline X_C\otimes_{X_A}\overline X_C$ color-preserving isomorphic to the left graph in Figure~\ref{fig:fiber product with some 3s}. 
We compute $\sigma\beta(Y_2)$ similarly as in Lemma~\ref{lem:three odd} and Lemma~\ref{lem:3(2n+1)(2p+1)}. This is illustrated in Figure~\ref{fig:3(2n+1)(2p+1)proof}. 
   % Figure environment removed
Next, for each of the two choices of $\sigma\beta(Y_2)$ (as illustrated in Figure~\ref{fig:fiber product with some 3s}) we compute the fiber product $\overline Y_2\otimes \sigma\beta(Y_2)$. The labelling of the vertices in the top left, top right and the bottom right graph in Figure~\ref{fig:fiber product with some 3s}, will help the reader to verify that all the connected components of those fiber products are either pictured in Figure~\ref{fig:3(2n+1)(2p+1)proof2} or consist of a single monochrome cycle.
Finally, we conclude that for every $\overline Y_3$, the complex $\widetriangle Y_3$ is simply connected.
\end{proof}


In the remaining case exactly two of $M,N,P$ are equal $3$.
\begin{lem}\label{lem:33(2p+1)}
Let $P \geq 5$ be odd, and $M=N=3$. 
Every $\overline Y_3$ either has simply connected $\widetriangle Y_3$, or is one of the graphs in Figure~\ref{fig:33(2p+1)proof2}.
% Figure environment removed
Moreover, the map $\overline Y_5\to \overline Y_3$ is always an embedding of a subgraph.
\end{lem}
 \begin{proof}
  We write $P=2p+1$. By Lemma~\ref{lem:fiberprod}, every $Y_2$ either has simply connected $\widetriangle Y_2$ or is color-preserving isomorphic to the right graph in Figure~\ref{fig:fiber product with some 3s}. 
 
We compute $\sigma\beta(Y_2)$ similarly as in Lemma~\ref{lem:3(2n+1)(2p+1)}. This is illustrated in Figure~\ref{fig:33(2p+1)proof}.
  % Figure environment removed
Once again, for each of the two choices of $\sigma\beta(Y_2)$ we compute the fiber product $\overline Y_2\otimes \sigma\beta(Y_2)$. As a result we obtain that $\overline Y_3$ is either a monochrome (blue) cycle, or it is color-preserving isomorphic to one of the graphs in Figure~\ref{fig:33(2p+1)proof2}.


We now note that the collection of graphs in Figure~\ref{fig:33(2p+1)proof2}:
\begin{itemize}
\item has the property that the fiber product of any two graphs is a subgraph of one of the graphs in the collection, and 
\item is invariant under $\beta$, see Figure~\ref{fig:33(2p+1)proof3}.
\end{itemize}
The first fact implies that every $\overline Y_4$ is a subgraph of some $\overline Y_3$. The second fact implies that this is also the case for $\overline Y_5$.
In particular, every $\overline Y_5\to \overline Y_3$ is an embedding.
% Figure environment removed
\end{proof}

%Lemma~\ref{lem:33(2p+1)} and Lemma~\ref{lem: finitely many colored graphs} yield the following. Again, the residual finiteness of the Artin groups considered below is a new result.
We now summarize what we have proven in this subsection.
\begin{cor}\label{cor: all odd}
The Artin group $G_{MNP}$ where $M,N,P\geq 3$ are odd has finite stature with respect to $\{A\}$, where $A$ is as described in Theorem~\ref{thm:splitting}. 
%In particular, all subgroups of $A$ are separable in $G_{MNP}$, and $G_{MNP}$ is residually finite.
\end{cor}

\begin{proof}
When $M=N=P=3$, the statement follows from Proposition~\ref{lem:333}. 
The case where $M=N=3$, and $P=2p+1\geq 5$ follows from Lemma~\ref{lem:33(2p+1)} and Lemma~\ref{lem: finitely many colored graphs}.
The case where $M=3$, $N=2n+1\geq 5$, and $P=2p+1\geq 5$ follows from Lemma~\ref{lem:3(2n+1)(2p+1)} and Lemma~\ref{lem: finitely many colored graphs}.
Finally, the case where $M=2m+1\geq 5$, $N=2n+1\geq 5$, and $P=2p+1\geq 5$ is a consequence of Lemma~\ref{lem:three odd} and Lemma~\ref{lem: finitely many colored graphs}.
\end{proof}
We note that the residual finiteness of $G_{333}$ follows from \cite{Squier87}. The residual finiteness of $G_{MNP}$ where $M,N,P\geq 5$ was proven in \cite{JankiewiczArtinRf}. However, the methods of \cite{JankiewiczArtinRf} do not cover the cases where one or two of $M,N,P$ are equal $3$.

\subsection{The case where $\{M,N,2\}$ where $M,N\geq 4$}\label{sec:2}
We first focus on the case where $M,N$ are both even. We recall that, unlike in the previous cases, $G_{MNP}$ splits as an HNN-extension $A*_B$, as in Theorem~\ref{thm:splitting2}.
\begin{lem}\label{lem:eveneven2 proof}
Let $M=2m, N=2n$ and $P=2$. The graphs $\overline{\phi_1X_B}$ and $\overline{\phi_2 X_B}$ are (unbased) color-preserving isomorphic. In particular, the stabilizer of every finite path in the Bass-Serre tree of the splitting of $G_{MNP} = A*_B$ is conjugate to a subgroup of $A$ represented by $\overline{\phi_1X_B}$ or a wedge of monochrome cycles.
\end{lem}
\begin{proof}
The graphs  $\overline{\phi_1X_B}$ and $\overline{\phi_2 X_B}$ are computed in Theorem~\ref{thm:splitting2}, and it is easy to see that the two graphs are color-preserving isomorphic. Every connected component $Y$ of the fiber product $\overline{\phi_1X_B}\otimes_{X_A} \overline{\phi_1X_B}$ is either color-preserving isomorphic to $\overline{\phi_1X_B}$ or is a wedge of monochrome cycles.
\end{proof}

Next, we consider the cases where at both $M,N$ are odd.
\begin{lem}\label{lem:2oddodd}
Let $P=2$ and $M=2m+1,N=2n+1\geq 5$. Every graph $\overline Y_2$ either is color-preserving isomorphic  to the left graph in Figure~\ref{fig:2andsomeodd}(a) or it is a wedge of monochrome cycles. If $Y$ is the left graph in Figure~\ref{fig:2andsomeodd}(a), then $\sigma\beta(Y)$ is (unbased) isometric to $Y$. Therefore, every graph $\overline Y_i$ either one of the two graphs in Figure~\ref{fig:2andsomeodd}(a), or it is a wedge of monochrome cycles. 
\end{lem}
% Figure environment removed
\begin{proof} The first statement was proven in \cite[Rem 3.5]{JankiewiczArtinSplittings}. The proof of the second statement is illustrated in Figure~\ref{fig:invarianceofY2}(a). Let $\overline Y_2$ be the left graph in Figure~\ref{fig:2andsomeodd}(a). Then  every connected component $\overline Y_3$ of the fiber product $\overline{Y}_2\otimes_{X_A} \sigma\beta(Y_2) = \overline{Y}_2\otimes_{X_A} \overline{Y}_2$ is a wedge of monochrome cycles, is isomorphic to $\overline Y_2$ or to the right graph in Figure~\ref{fig:2andsomeodd}(a). 
We also note that if $Y$ is the right graph in Figure~\ref{fig:2andsomeodd}(b), then $\sigma\beta(Y)$ is isometric to $Y$. We conclude that every graph $\overline Y_\ell$ either one of the two graphs in Figure~\ref{fig:2andsomeodd}(a), or it is a wedge of monochrome cycles. 
% Figure environment removed
\end{proof}

Finally, we consider the cases where exactly one of $M,N$ is odd.

\begin{lem}\label{lem:2oddeven}
Let $P=2$, $M=2m+1\geq 5$, and $N=2n\geq 4$. Every graph $\overline Y_2$ either is isometric to the graph in Figure~\ref{fig:2andsomeodd}(b) or it is a wedge of monochrome cycles. If $Y$ is the graph in Figure~\ref{fig:2andsomeodd}(b), then $\sigma\beta(Y)$ is (unbased) isometric to $Y$. Therefore, every graph $\overline Y_i$ either one of the graph in Figure~\ref{fig:2andsomeodd}(b), or it is a wedge of monochrome cycles. 
\end{lem}
\begin{proof} The first statement was proven in \cite[Prop 3.4]{JankiewiczArtinSplittings}. The proof of the second statement is illustrated in Figure~\ref{fig:invarianceofY2}(b). Let $Y$ denote the graph in Figure~\ref{fig:invarianceofY2}(b). Every connected component of the fiber product $Y\otimes_{X_A} Y$ is either isometric to $Y$ or it is a wedge of monochrome cycles.
\end{proof}


\begin{cor}\label{cor: 2MN}
The Artin group $G_{MN2}$ where $M,N\geq 4$ has finite stature with respect to $\{A\}$, where $A$ is as described in Theorem~\ref{thm:splitting}.
% In particular, all subgroups of $A$ are separable in $G_{MNP}$, and $G_{MNP}$ is residually finite.
\end{cor}

\begin{proof}All the cases can be deduced from Lemma~\ref{lem: finitely many colored graphs} together with 
\begin{itemize}
\item Lemma~\ref{lem:eveneven2 proof} when $M,N$ are both even;
\item Lemma~\ref{lem:2oddeven} when exactly one of $M,N$ is odd;
\item Lemma~\ref{lem:2oddodd} when both $M,N$ are odd.\qedhere
\end{itemize}
\end{proof}

Residual finiteness of $G_{MN2}$ where at least one of $M,N$ is even was proven in \cite{JankiewiczArtinSplittings}, but the case of both $M,N$ odd is a new result.

\subsection{Triangle Artin groups with label $\infty$} Note that all  of the above proofs are valid if any of the labels $M,N,P$ are equal to $\infty$.


\bibliographystyle{alpha}
%\bibliographystyle{plain}
\bibliography{../../../kasia}

\end{document}



















































