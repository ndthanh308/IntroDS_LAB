%\documentclass[sigconf,authordraft]{acmart}
%\documentclass[sigconf,prologue,dvipsnames]{acmart}
%\documentclass[10pt,conference,prologue,dvipsnames]{IEEEtran}
\documentclass[11pt]{article}

% Remove the "review" option to generate the final version.
\usepackage{EMNLP2023}

%%
%% \BibTeX command to typeset BibTeX logo in the docs
\AtBeginDocument{%
  \providecommand\BibTeX{{%
    \normalfont B\kern-0.5em{\scshape i\kern-0.25em b}\kern-0.8em\TeX}}}

%% Rights management information.  This information is sent to you
%% when you complete the rights form.  These commands have SAMPLE
%% values in them; it is your responsibility as an author to replace
%% the commands and values with those provided to you when you
%% complete the rights form.
%\setcopyright{acmcopyright}
%\copyrightyear{2018}
%\acmYear{2018}
%\acmDOI{10.1145/1122445.1122456}

%% These commands are for a PROCEEDINGS abstract or paper.
%\acmConference[Woodstock '18]{Woodstock '18: ACM Symposium on Neural
%  Gaze Detection}{June 03--05, 2018}{Woodstock, NY}
%\acmBooktitle{Woodstock '18: ACM Symposium on Neural Gaze Detection,
%  June 03--05, 2018, Woodstock, NY}
%\acmPrice{15.00}
%\acmISBN{978-1-4503-XXXX-X/18/06}


%%
%% Submission ID.
%% Use this when submitting an article to a sponsored event. You'll
%% receive a unique submission ID from the organizers
%% of the event, and this ID should be used as the parameter to this command.
%%\acmSubmissionID{123-A56-BU3}

%%
%% The majority of ACM publications use numbered citations and
%% references.  The command \citestyle{authoryear} switches to the
%% "author year" style.
%%
%% If you are preparing content for an event
%% sponsored by ACM SIGGRAPH, you must use the "author year" style of
%% citations and references.
%% Uncommenting
%% the next command will enable that style.
%%\citestyle{acmauthoryear}

% \usepackage[utf8]{inputenc}
% \usepackage[T1]{fontenc}

\usepackage{times}
\usepackage{latexsym}
\usepackage{microtype}
\usepackage{inconsolata}
%%
%% end of the preamble, start of the body of the document source.
\usepackage{float}
%\usepackage[dvipsnames]{xcolor}
\PassOptionsToPackage{usenames,dvipsnames}{xcolor}
%\usepackage[usenames,dvipsnames,rgb]{xcolor}
\usepackage[utf8]{inputenc} % allow utf-8 input
\usepackage[T1]{fontenc}    % use 8-bit T1 fonts
\usepackage{hyperref}       % hyperlinks
\usepackage{url}            % simple URL typesetting
\usepackage{booktabs}       % professional-quality tables
\usepackage{amsfonts}       % blackboard math symbols
\usepackage{nicefrac}       % compact symbols for 1/2, etc.
\usepackage{microtype}      % microtypography
\usepackage{tcolorbox}
\usepackage{adjustbox}
\usepackage[frozencache,cachedir=.]{minted}
%\usepackage[finalizecache,cachedir=.]{minted}
%\usepackage{minted}
%\usepackage{xcolor}
\usepackage{array}
\usepackage{siunitx}
%\setminted{fontsize=\scriptsize}
%\setminted{fontsize=\tiny}
\setminted{fontsize=\scriptsize}
%\setminted{fontsize=\small}
\definecolor{bg}{HTML}{282828}
\usepackage{pmboxdraw}
\usepackage{multirow}
%\usepackage{balance}
% \usepackage[keeplastbox]{flushend}
\usepackage{amsmath}
\usepackage{xspace}
\usepackage{amssymb}
\usepackage{multirow}
\usepackage{multicol}
\usepackage{booktabs}
\usepackage{graphicx}
%\usepackage[justification=centering]{caption}
% \usepackage{listings}
% \lstset{extendedchars=true}



\definecolor{ForestGreen}{RGB}{34,139,34}
\definecolor{RoyalBlue}{RGB}{85,118,209}

\definecolor{Gray}{gray}{0.9}
\newcommand{\txtmint}[1]{\mintinline[fontsize=\scriptsize, bgcolor=Gray]{text}{#1}}

\title{Predicting Code Coverage without Execution}


\author{Michele Tufano, Shubham Chandel, Anisha Agarwal, Neel Sundaresan, Colin Clement \\
  Microsoft \\
  Redmond, WA, USA \\
  \texttt{\{mitufano, schandel, anisagarwal, neels, coclement\}@microsoft.com}}



\begin{document}

\newcommand{\dataset}{{\sc CoverageEval}\xspace}

\newcommand{\ie}{\textit{i.e.,}~}
\newcommand{\eg}{\textit{e.g.,}~}
\newcommand{\etc}{\textit{etc.}~}
\newcommand{\etal}{\textit{et al.}~}

%Comments
\newcommand{\nb}[2]{
    \fbox{\bfseries\sffamily\scriptsize#1}
    {\sf\small$\blacktriangleright$\textit{#2}$\blacktriangleleft$}
}

\newcommand\MICHELE[1]{\textcolor{blue}{\nb{MICHELE}{#1}}}
\newcommand\COLIN[1]{\textcolor{green}{\nb{COLIN}{#1}}}
\newcommand\SHUBHAM[1]{\textcolor{red}{\nb{SHUBHAM}{#1}}}





% DOUBLE BLIND

% \author{\IEEEauthorblockN{Michele Tufano, Dawn Drain, Alexey Svyatkovskiy, Neel Sundaresan}
% \IEEEauthorblockA{Microsoft\\
% Redmond, WA, USA\\
% Email: \{mitufano, dadrain, alsvyatk, neels\}@microsoft.com}\vspace{-5ex}}



% \settopmatter{authorsperrow=4}

% \author{Michele Tufano}
% \affiliation{%
%   \institution{Microsoft}
%   \city{Redmond}
%   \state{WA}
%   \country{USA}
% }

% \email{mitufano@microsoft.com}

% \author{Shubham Chandel}
% \affiliation{%
%   \institution{Microsoft}
%   \city{Redmond}
%   \state{WA}
%   \country{USA}
% }
% \email{schandel@microsoft.com}

% \author{Anisha Agarwal}
% \affiliation{%
%   \institution{Microsoft}
%   \city{Redmond}
%   \state{WA}
%   \country{USA}
% }
% \email{anisagarwal@microsoft.com}

% \author{Neel Sundaresan}
% \affiliation{%
%   \institution{Microsoft}
%   \city{Redmond}
%   \state{WA}
%   \country{USA}
% }
% \email{neels@microsoft.com}

% \author{Colin Clement}
% \affiliation{%
%   \institution{Microsoft}
%   \city{Redmond}
%   \state{WA}
%   \country{USA}
% }
% \email{coclement@microsoft.com}

% \renewcommand{\shortauthors}{Tufano, et al.}


%%
%% The code below is generated by the tool at http://dl.acm.org/ccs.cfm.
%% Please copy and paste the code instead of the example below.
%%
% \begin{CCSXML}
% <ccs2012>
%   <concept>
%       <concept_id>10011007.10011074.10011099.10011102.10011103</concept_id>
%       <concept_desc>Software and its engineering~Software testing and debugging</concept_desc>
%       <concept_significance>500</concept_significance>
%       </concept>
%   <concept>
%       <concept_id>10010147.10010178.10010179.10010180</concept_id>
%       <concept_desc>Computing methodologies~Machine translation</concept_desc>
%       <concept_significance>300</concept_significance>
%       </concept>
%  </ccs2012>
% \end{CCSXML}

% \ccsdesc[500]{Software and its engineering~Software testing and debugging}
% \ccsdesc[300]{Computing methodologies~Machine translation}

% \keywords{software testing, unit test, neural networks}

\maketitle

\begin{abstract}
Code coverage is a widely used metric for quantifying the extent to which program elements, such as statements or branches, are executed during testing. %It provides valuable insights into the amount of code exercised by a test suite and serves as a standard measure of test suite quality. %Higher code coverage percentages indicate a lower risk of undiscovered software bugs. 
%While code coverage is not a panacea for eliminating bugs, it remains a reliable proxy metrics for assessing code quality. 
Calculating code coverage is resource-intensive, requiring code building and execution with additional overhead for the instrumentation. Furthermore, computing coverage of any snippet of code requires the whole program context. Using Machine Learning to amortize this expensive process could lower the cost of code coverage by requiring only the source code context, and the task of code coverage prediction can be a novel benchmark for judging the ability of models to understand code.
We propose a novel benchmark task called Code Coverage Prediction for Large Language Models (LLMs). We formalize this task to evaluate the capability of LLMs in understanding code execution by determining which lines of a method are executed by a given test case and inputs. We curate and release a dataset we call \dataset by executing tests and code from the HumanEval dataset and collecting code coverage information. We report the performance of four state-of-the-art LLMs used for code-related tasks, including OpenAI's GPT-4 and GPT-3.5-Turbo, Google's BARD, and Anthropic's Claude, on the Code Coverage Prediction task. Finally, we argue that code coverage as a metric and pre-training data source are valuable for overall LLM performance on software engineering tasks.

%Our ultimate goal is to leverage LLMs for predicting code coverage, offering a viable alternative to execution-based coverage in various scenarios. This approach proves advantageous when program build and execution costs are prohibitive, code coverage needs to be invoked multiple times, only code snippets are available (e.g., server-side scenarios), or errors in the project prevent complete builds. Additionally, this task serves as a novel metric for code understanding and a valuable (pre-)training objective. Training models to excel at this task could enhance their overall performance on code-related tasks.
\end{abstract}









% Code coverage measures which program elements -- such as statements or branches -- are executed by a (set of) test case(s).
% Code coverage provides a useful metric to quantify the amount of code that is executed during testing, and is a standard method to estimate the quality of a test suite. Intuitively, the higher the percentage of statements and branches covered by a test suite, the lower the risk of undetected software bugs lingering in the program. While code coverage is not a silver bullet against bugs, it represents one of the best available proxy metrics for code quality, often required for safety standards (such as FAA and automotive software).

% Code coverage is computed by instrumenting the code, and running the test suite while monitoring the code execution. This process is expensive, since it requires building and executing code, with additional overhead for the instrumentation. Additionally, it is not currently possible to measure code coverage for a snippet of code without the availability of the entire program which contains the given snippet.

% In this paper we present an approach which aims at predicting code coverage without building or executing the target program. Our approach relies on a Transformer-based model, which takes two inputs: (i) a method or code snippet for which code coverage is needed; (ii) a test case testing the given method (or an invocation such as a \texttt{main} function). The model is then trained to generate as output the coverage-annotated method, where statements and branches that would be executed by the test case are marked appropriately.

% This approach could represent a viable alternative to execution-based coverage in several circumstances, such as: (i) program build and execution is too expensive; (ii) code coverage needs to be invoked multiple times; (iii) only a snippet of code is available, but not the entire program (e.g. server-side scenario); (iv) there are errors somewhere in the project preventing complete builds. This approach can also serve as a novel code understanding metric and training objective. Large language models should be able to perform well on this task if they truly `understand' code, and training models to perform this task should improve their general performance on code-related tasks.




%% Figure environment removed
% ---------- END Code Example ----------------


For example, coverage is one of the metrics considered by the Federal Aviation Administration (FAA) for safety certification of avionic equipment, as documented in DO-178B \cite{johnson1998178b} and DO-178C \cite{rierson2017developing}. Test coverage is also a requirement in the automotive safety standard ISO 26262 Road Vehicles - Functional Safety \cite{palin2011iso}.
 
Given a focal method $m$, which is executed \textit{directly} by the test case $t$, code coverage measures the number of statements that have been executed (\ie covered) by the test $t$. Figure \ref{fig:coverage} shows an example of a focal method $m$ (method under test) tested by $t$. The coverage obtained by $t$ on $m$ is represented in the coverage-annotated method $\mathrm{cov}(m,t)$, where executed statements are marked with \colorbox{green}{>} while missed (\ie uncovered statements) with \colorbox{red}{!} and unreachable code (\ie dead code) with \colorbox{gray}{-}. From this representation, several quantitative coverage metrics can be computed, such as functional, statement, branch, and path coverage. 
%We will discuss these coverage metrics in details in the Sec \ref{sec:background}.

Code coverage is computed by instrumenting the code and running the test suite while monitoring the code execution. This process is expensive, since it requires building and executing code, especially for large software projects or when code coverage is computed multiple times. Additionally, it is not possible to measure code coverage for a snippet of code without the availability of the entire program which contains the given snippet. This situation happens when only partial code is available, for example within a commit log/diff, or when only partial code is transmitted to a server, for security and/or networking reasons.

While Large Language Models (LLMs) have gained prominence in code-related tasks and demonstrated impressive results in areas such as code generation and test generation, it remains unclear to what extent these models truly understand code execution~\cite{liu2023code}. The task of accurately determining which lines of a method are executed based on a given test case and its inputs requires a deep understanding of the underlying code execution dynamics. This motivates the need for a dedicated task, referred to as Code Coverage Prediction, which specifically evaluates the capability of LLMs in comprehending code execution. Further, a model capable of this task is independently useful as it can amortize the expensive code coverage computation process, or function in cases where normal code coverage is not possible to compute.

In this paper we formalize the Code Coverage Prediction task, with the primary objective of evaluating the capability of LLMs in understanding code execution by accurately determining which lines of a method are executed based on a given test case. To facilitate evaluation, we have curated a comprehensive dataset named \dataset, consisting of coverage-annotated methods. This dataset is created by executing tests and code from the HumanEval dataset, allowing us to collect valuable code coverage information. We have organized and made this curated dataset available on GitHub,
%and HuggingFace Datasets
 enabling researchers to explore and advance code coverage prediction techniques and LLM code understanding.

We evaluate the performance of four state-of-the-art LLMs widely employed for code-related tasks: OpenAI's GPT-4 and GPT-3.5, Google's BARD, and Anthropic's Claude. Our ultimate goal is to gain insights into the capabilities of LLMs in predicting code coverage, offering a promising alternative to execution-based coverage measurement in various scenarios. This approach proves advantageous when the costs associated with program building and execution are prohibitive, when code coverage needs to be invoked multiple times, when only code snippets are available (e.g., in server-side scenarios), or when errors in the project prevent complete builds. Additionally, this task introduces a novel metric for assessing code understanding and serves as a valuable (pre-)training objective. By training models to excel in this task, we believe we can enhance their overall performance on code-related tasks.

This paper makes the following contributions:
\begin{itemize}
\item \textit{Code Coverage Prediction Task}: We propose a novel task to assess the capability of LLMs in understanding code execution by accurately predicting executed lines of a method based on a given test case and inputs.

\item \textit{Evaluation of State-of-the-Art LLMs}: We evaluate four prominent LLMs (GPT-4, GPT-3.5, BARD, and Claude) on the Code Coverage Prediction task, providing insights into their performance and understanding of code execution.

\item \textit{Curated Dataset}: We curate a comprehensive dataset (\dataset) of coverage-annotated methods and test cases, derived from the HumanEval dataset. This dataset is openly available on GitHub\footnote{\url{https://github.com/microsoft/coverage-eval}}~\cite{CoverageDataset_github}
%\footnote{\url{https://anonymous.4open.science/r/coverage-eval-467C/}} ~\cite{CoverageDataset_github},
%and HuggingFace\footnote{\url{To-be-released}}~\cite{CoverageDataset_huggingface}, 
enabling further research and advancement in code coverage prediction techniques.
\end{itemize}

\section{Background}
\label{sec:background}

Code coverage is a measure of the degree to which a test suite exercises a software system \cite{ivankovic2019code}. Code coverage is commonly computed by means of instrumentation. This technique inserts instrumentation code in various locations within the code or binaries of the program under test, in order to monitor its execution. This inserted code provides counters to record which function or statement of the program have been executed by the test suite. Inserting these additional statements within the original code leads to execution overhead, which can be significant especially for large software programs \cite{tikir2002efficient}.

The most common coverage metric is computed at statement level, where statement refers to a syntactic unit of code (\eg assignment, invocation, assertion), often matching a single line of code. The coverage indicates whether a statement has been executed or not, and aggregated metrics can be computed at function/program level to measure the amount of statements covered by a test suite. In the example in Figure \ref{fig:coverage}, the test case $t$ executes four statements in $m$, which constitutes $\sim44$\% statement coverage for the method $m$.


Given statement coverage information, other coverage criteria and metrics can be obtained by means of static analysis. Statement coverage information regarding control structure (\eg \texttt{if-else} and \texttt{case} statements) can be used to compute branch coverage, which measure how many logical branches in the program have been executed. In the example in Figure \ref{fig:coverage} only one branch is executed (\ie \texttt{else if (x > 0)} ), while the other two branches are missed by the test case $t$.

In the remainder of this paper we will focus on statement coverage, from which other coverage criteria can be obtained.


\section{Code Coverage Prediction Task}
\label{sec:problem}
Given a method under test (focal method) $m$, composed of $n$ statements $S_m = s_1, s_2, \dots, s_n$, and a test case $t$ which exercises the method $m$, the coverage-annotated focal method $\mathrm{cov}(m,t)$ is composed of a sequence of $n$ statements $S_{m}^{t} = s_1^*, s_2^*, \dots, s_n^*$, where each statement  $s_i^*$ represents the coverage-annotated statement of $s_i$ in $m$. Specifically, $s_i^*$ is marked with one of the three possible coverage symbols $c \in \{>, !, - \}$, where the symbol $>$ identifies statements that have been executed by $t$, the symbol $!$ identifies statements that have been missed by $t$, and the symbol $-$ identifies statements that are unreachable. This defines a sequence of $n$ coverage symbols $C_m^t = c_1, c_2, \dots, c_n$, where $c_i \in \{>, !, - \}$. %This sequence can be combined with the original sequence of statements $S_m = s_1, s_2, \dots, s_n$, to obtain the coverage-annotated sequence of statements $S_m^t = s_1^*, s_2^*, \dots, s_n^*$ comprising the coverage $\mathrm{cov}(m,t)$.

We define the Code Coverage Prediction Task as the problem of predicting the coverage-annotated sequence of statements $S_m^t$ given the focal method $m$ and a test case $t$. Formally, this problem can be defined in terms of inputs and expected output:

\textbf{Input}
        \begin{itemize}
        \item Focal Method: $m$
        \item Test Case: $t$
    \end{itemize}
    
\textbf{Output}
    \begin{itemize}
        \item $S_m^t = s_1^*, s_2^*, \dots, s_n^*$ \\
        or
        \item $C_m^t = c_1, c_2, \dots, c_n$
    \end{itemize}

Specifically, the output can be either the coverage-annotated sequence of statements $S_m^t$, or the sequence of coverage symbols $C_m^t$, which can then combined with the original sequence of statements $S_m = s_1, s_2, \dots, s_n$, to obtain the coverage-annotated sequence of statements $S_m^t = s_1^*, s_2^*, \dots, s_n^*$ comprising the coverage $\mathrm{cov}(m,t)$. This final step is performed by aligning the two sequences and obtaining $s_i^* = c_i + s_i$, where the $+$ operation refers to string concatenation. 

Let us take as example the focal method $m$ and test case $t$ in Figure \ref{fig:coverage}. The model is expected to predict either the coverage-annotated sequence of statements $S_m^t$ or the sequence of coverage symbols: \texttt{> > ! > > ! ! ! -}. 
%This sequence of symbols is then aligned with the original sequence of statements, and the coverage-annotated focal method $c$ is obtained by concatenating each coverage symbol $c_i$ to the corresponding statement $s_i$.\COLIN{Is this still the case or do we just directly predict the annotated statements?} \MICHELE{Good point. Clarified this.}


\subsection{Coverage Prediction for Pre-Training}

We propose that the code coverage prediction task introduced in our paper can serve as a valuable pre-training task for LLMs focused on code generation. While current pre-training tasks, such as Masked Language Modeling (MLM) help models understand code syntax and semantics by analyzing vast amounts of raw text representing code, our proposed task enables the model to learn about code execution, which is not technically discoverable by source code text alone.

To accomplish this pre-training, we suggest augmenting the training data with extensive coverage logs obtained from Continuous Integration/Continuous Deployment (CI/CD) pipelines. These logs contain valuable information about code coverage from regression tests executed during pull requests or commits.

By exposing the models to these coverage logs during pre-training, they can learn to associate test cases and inputs with the specific lines of code that are executed. This pre-training approach enhances the models' understanding of how different parts of the code are exercised by various test scenarios. Consequently, the models can acquire a deeper comprehension of the relationships between inputs, tests, and code execution, leading to improved code generation capabilities.

Integrating coverage prediction as a pre-training task could enable models to learn from real-world test scenarios, capturing the nuances of code execution in practical settings. This real-world exposure should enhances the models' ability to generate code that aligns with actual testing practices.

Furthermore, incorporating coverage prediction as a pre-training task opens up possibilities for transfer learning. Models pre-trained on coverage prediction can be fine-tuned on downstream tasks, such as bug detection or test case generation, where understanding code execution is crucial. The models' pre-existing knowledge of code coverage can provide a solid foundation for these related tasks, potentially improving their overall performance.

\section{\dataset Dataset}
In addition to proposing the code coverage prediction task, this paper also introduces \dataset, a dataset specifically designed for evaluating LLMs on this task. This section outlines the process of curating this dataset, which begins with the HumanEval dataset \cite{chen2021codex}. By executing test cases from the HumanEval dataset, we gather code coverage information. To create \dataset, we parse the code coverage logs generated during the execution of the test cases. This parsing step enables us to extract the relevant coverage annotations. We then carefully structure and export the dataset in a format that facilitates its use and evaluation by researchers and practitioners alike.

By curating this dataset, we aim to provide a standardized benchmark for evaluating LLMs on the code coverage prediction task. The availability of \dataset enables researchers to explore and advance code understanding, fostering innovation and enabling the development of more effective models.

\subsection{HumanEval}
The HumanEval dataset consists of 164 hand-written problems and their code solutions, where each problem is a programming task involving language comprehension, reasoning, algorithms and/or simple mathematics \cite{chen2021codex}. Each code solution in the dataset includes a function signature, a docstring containing the problem description, a function body, and several unit tests. We extend the HumanEval dataset to include coverage, calculated using the function body and the respective unit tests.



\subsection{Coverage Analysis}
In this section, we describe the steps taken to analyze the code coverage on the HumanEval dataset and create our \dataset dataset.

Each code solution in the HumanEval dataset is accompanied by a single test case, which includes multiple asserts designed to test the correctness of the code solution based on the given problem's functional requirements. These asserts cover various inputs, scenarios, and code statements/branches. To enhance the dataset and increase the complexity of each data point, we split the single test case into multiple test cases, each containing a single assert. This splitting process allows us to generate additional method-test pairs, as well as making each data point more challenging. The original test case may cover most of the lines and branches in the method, but each individual assert covers only a subset of them.

By performing this split, we create a more diverse set of method-test pairs within the dataset. Each individual test case invokes the focal method once and covers a subset of the statements and branches within the method. This enables us to evaluate the LLMs' ability to predict code coverage at a more granular level, going beyond the overall coverage of the method. It also adds complexity to the task, as predicting coverage for each assert requires a deeper understanding of the code and its potential execution paths.

Subsequently, we execute the extracted test cases individually with \texttt{pytest}. During the execution, we also enable the coverage computation using \texttt{coverage.py}. To do so, we run the following command: \texttt{coverage run -m pytest <test\_name>} where \texttt{<test\_name>} is each individual test in the dataset.

Next, for each test case $t$, we analyze the corresponding coverage report obtained by the test execution in order to extract the annotated coverage $\mathrm{cov}(m,t)$. The coverage report marks each source code line in the file with coverage information, specifying whether the statement has been executed or not.

We automatically parse this report and extract the corresponding annotated coverage $\mathrm{cov}(m,t)$. At the end of this process, we obtained a dataset where each data point is formed by a triplet $d = \{ m, t, \mathrm{cov}(m,t) \}$.

\subsection{Data Format}
The \dataset dataset maintains the structure of the HumanEval dataset, with the addition of coverage information for each test. Each record corresponds to a unique problem and contains the following fields:

\begin{itemize}
  \item Problem ID: A unique ID for the problem
  \item Problem: The name of the method written to solve the problem
  \item Method: The method contents, including a function signature, a docstring with the details of the problem, and the function body.
  \item Tests: A list of unit tests for the problem. Each item in the list includes the unique ID of the test and the code of the test. We have also added coverage information for each test in the following two forms:
    \begin{enumerate}
      \item Coverage: The code of the method, with each line annotated with \colorbox{green}{>}, \colorbox{red}{!} or \colorbox{gray}{-} for code that is executed, missed or unreachable by the given test.
    \item Coverage Sequence: A list of equal length to the number of lines in the method, where each value in the list is \colorbox{green}{>}, \colorbox{red}{!} or \colorbox{gray}{-}, depending on the status of the respective line of code in the method.
    \end{enumerate}
\end{itemize}

Figure \ref{fig:coval_data_example} (Appendix) shows a sample record from the \dataset dataset. \dataset is available to the public via GitHub~\cite{CoverageDataset_github}.
%and HuggingFace~\cite{CoverageDataset_huggingface}.

Table \ref{tab:dataset} reports the statistics for the \dataset dataset in terms of number of problems, code solutions, tests, and coverage symbols. The discrepancy between number of problems and solutions is explained by the fact that some problems have multiple solutions. It is also worth noting that while our dataset currently does not contain any unreachable code (-), we have proactively considered the potential presence of unreachable code while designing the task.

\begin{table}[]
\resizebox{0.5\textwidth}{!}{
\begin{tabular}{@{}cccccc@{}}
\toprule
\multirow{2}{*}{Problems} & \multirow{2}{*}{Solutions} & \multirow{2}{*}{Tests} & \multicolumn{3}{c}{Coverage Symbols}                     \\ \cmidrule(l){4-6} 
                          &                            &                        & Executed (\textgreater{}) & Missed (!) & Unreachable (-) \\ \midrule
                158       &                 164     &           1160            &                      20037 & 1734      & 0               \\
%\multicolumn{1}{l}{}      & \multicolumn{1}{l}{}       &                        &                           &            &                 \\ 
\bottomrule
\label{tab:dataset}
\end{tabular}
}
\vspace{-0.5cm}
\caption{\dataset statistics.}
\label{tab:dataset}
\vspace{-0.5cm}
\end{table}



\section{Evaluating LLMs}

In this section, we present our evaluation of state-of-the-art Language Models (LLMs) for the proposed task of Code Coverage Prediction. We selected four highly regarded LLMs that are not only popular for code generation but also widely used for other Natural Language (NL) tasks. The LLMs we employed for this evaluation are OpenAI's GPT-4 and GPT-3.5, Google's BARD, and Anthropic's Claude.

GPT-3.5 \cite{brown2020language} and GPT-4 \cite{openai2023gpt4} are large language models developed by OpenAI which are Transformer-style models \cite{DBLP:journals/corr/VaswaniSPUJGKP17} pre-trained to predict the next token in a document. Both models were then fine-tuned using Reinforcement Learning from Human Feedback (RLHF) \cite{christiano2017deep}. GPT-4 improves over the predecessor by accepting as input both images and text (multimodal model) and producing text as output. BARD is a conversational AI developed by Google based on LaMDA\cite{thoppilan2022lamda} a Transformer-based language models trained on dialogue \cite{adiwardana2020towards}. Anthropic Claude is a 52-billion-parameter LLM developed by Anthropic. Claude was pretrained on a large text corpus and finetuned with "RL from AI Feedback" (RLAIF), where AI feedback are steered by a small set of principles drawn from a "constitution" defined by humans \cite{bai2022constitutional}.

\subsection{Experimental Design}
%\SHUBHAM{Updated the datapoints. Looks good to me.} \MICHELE{Great!}
When evaluating the LLMs on the code coverage prediction task, we designed the experiments to assess their performance on non-trivial coverage sequences while progressively providing more information and examples.

First, we filtered out data points $d = \{ m, t, \mathrm{cov}(m,t) \}$ where the coverage sequence is \textit{trivial} consisting exclusively of the symbol \colorbox{green}{>}. These cases represent methods with no branches or where the test case covers every statement in the focal method. Although these data points are included in the \dataset dataset, we excluded them from this specific evaluation. The subset of data points containing only trivial symbols is reported in our online appendix. It's important to note that no data points in the dataset has a coverage sequence consisting solely of \colorbox{red}{!} or \colorbox{gray}{-} symbols. After this filtering step, we were left with 478 data points on which we evaluated the LLMs.

The prompt used to evaluate the LLMs was designed to include the following sections:
\begin{itemize}
\item System NL prompt: a prompt providing a natural language description of the task, aimed at conveying the task to the LLM.
\item Examples: zero, one, or multiple examples of the task.
\item Focal Method $m$ and Test Case $t$.
\end{itemize}

In terms of the System NL prompt, our evaluation involved experimenting with various prompts and descriptions. We achieved the most favorable outcomes by utilizing a system prompt that emulates a terminal environment (e.g., python terminal). Within this prompt, we instructed the LLM to generate the code coverage output based on a given test case and method. For OpenAI models, we included this prompt in the specific system prompt section, while for BARD and Claude, we incorporated it as the initial part of the prompt.

To comprehensively assess the LLMs' performance, we conducted evaluations using different numbers of examples for the code coverage prediction task. Specifically, we employed zero-shot, one-shot, and multi-shot prompting approaches. This allowed us to examine the impact of example availability on the models' performance and their ability to generalize the task across various methods.

When selecting examples for evaluating coverage on a particular method $m_i$, we took care to prevent data leakage and encourage the LLMs to generalize their predictions to other methods. To achieve this, we randomly sampled a data point $\{m_j, t, \mathrm{cov}(m,t)\}$ where $m_j \neq m_i$ when providing examples.

Finally, the prompt provides a focal method $m$ and a corresponding test case $t$ for which we expected the model to predict the code coverage. Figure \ref{fig:prompt} shows an example of the prompt we designed.


Inference is performed on all the LLMs with temperature and topp set to 0, and generating one sample.



% ---------- Code Example ----------------
% Figure environment removed
% ---------- END Code Example ----------------


\subsection{Evaluation Metrics}
In this section we describe the evaluation metrics.

Given the method $m$, the test case $t$, and the sequence of coverage symbols $C_m^t = c_1, c_2, \dots, c_n$, where $c_i \in \{>, !, - \}$, the model generates a predicted sequence of coverage symbols $\hat{C}_m^t = \hat{c}_1, \hat{c}_2, \dots, \hat{c}_n$. We consider the following metrics to evaluate the performances of our proposed approach.




\begin{table*}[]
\resizebox{\textwidth}{!}{%
\begin{tabular}{@{}lccccccccc@{}}
\toprule
\multirow{2}{*}{Model} & \multicolumn{3}{c}{zero-shot}                                 & \multicolumn{3}{c}{one-shot}                                 & \multicolumn{3}{c}{multi-shot}                                 \\ \cmidrule(l){2-10} 
                       & \multicolumn{1}{c}{Match} & \multicolumn{1}{c}{Stmt} & Branch & \multicolumn{1}{c}{Match} & \multicolumn{1}{c}{Stmt} & Branch & \multicolumn{1}{c}{Match} & \multicolumn{1}{c}{Stmt} & Branch \\ \midrule
OpenAI GPT-4   (gpt-4)        & \textbf{25.75} & \textbf{84.47} & \textbf{20.16} &\textbf{ 22.85} & \textbf{90.71} & \textbf{22.65} & \textbf{30.04} & \textbf{90.5} & \textbf{22.5} \\
OpenAI GPT-3.5 (gpt-3.5-turbo)         &0 & 39.87 & 8.33 & 8.17 & 76.53 & 17.17 & 11.03 & 82.29 & 17.9 \\
Google BARD (text-bison-001)            &0 & 81.27 & 17.21 & 1.87 & 86.93 & 19.63 & 21.56 & 85.66 & 20.52 \\
Anthropic Claude (claude-1.3)       &3.9 & 84.47 & 20.07 & 4.83 & 83.21 & 19.16 & 6.88 & 55.7 & 12.23 \\
 \bottomrule
\end{tabular}
}
\caption{LLMs performances on the Code Coverage Prediction Task. The table reports the percentages of predicted coverage sequences that match the ground truth (Match), the percentage of correct coverage symbols for statements (Stmt), and specifically for branches (Branch). Evaluation performed for zero-shot, one-shot, and multi-shot.}
\label{tab:results}
\end{table*}









\subsubsection{Perfect Sequence Match}
The perfect sequence match metric counts the number of times that the predicted sequence $\hat{C}_m^t$ exactly matches (symbol-by-symbol) the target coverage sequence $C_m^t$. This represents the case where the model predicts the coverage with perfect accuracy for all the statements and branches.

\subsubsection{Statement Correctness}
The statement correctness metric measures the percentage of statements for which the execution prediction is correct. This is equivalent to the percentage of symbols in the predicted sequence that match the target sequence.


\subsubsection{Branch Correctness}
The branch correctness metric measures the percentage of branch-specific statements for which the execution prediction is correct.
The branch correctness only considers the symbols associated with branch statements. It measures the percentage of symbols in the predicted sequence (associated with branches) that match the symbols in the target sequence. 



\section{Results}

Table \ref{tab:results} presents the performance of different LLMs on the Code Coverage Prediction task. The table showcases the percentage of predicted coverage sequences that match the ground trught (Match), the percentage of correct coverage symbols for all the statements (Stmt), and the percentage of correct coverage symbols when only considering branch statements (Branch). Evaluation performances are computed using zero-shot, one-shot, and multi-shot prompting.

OpenAI GPT-4 demonstrates the highest performance on this task, achieving 24.75\% exact match with zero-shot prompting and improving to 30\% with multi-shot prompting, where up to 6 examples are provided in the prompt. Notably, the other LLMs achieve low exact matches with zero-shot prompting (between 0 and 4\%), suggesting that these foundational models may not have been exposed to coverage logs during their training or that. The second best-performing model is Google BARD, with an exact sequence match reaching 21.5\% with multi-shot prompting.

Regarding the percentage of correct coverage statements (see Stmt), most models demonstrate improvement as more examples are included in the prompt. OpenAI GPT-4 obtain the overall best scores between 84\% and 90\% of statement correctness.

When considering only statements involved in branches (\eg \texttt{if-else}, \texttt{while}), it becomes evident that there is a significant drop in correct predictions. In fact, the best performing model, OpenAI GPT-4, accurately predicts a modest 22\% of these symbols when one- and multi-shot is used for prompting. It is important to note that this subset of statements, which are intricately connected to branches, presents a greater challenge for evaluation because the LLM must reason about the boolean conditions that determine which branch is covered. Consequently, accurately predicting coverage symbols within this context requires the model to possess a profound understanding of the conditional logic that guides program execution.

Despite the surprisingly strong results of OpenAI GPT-4 on the Code Coverage Prediction task, it should be noted that the model still fails to generate the correct coverage for more than 70\% of the method-test pairs in the \dataset dataset. This emphasizes that LLMs have a long way to go in developing a deep understanding of code execution.

We believe that in order to enhance code generation results, these LLMs should gain a comprehensive understanding of code execution under different inputs and test cases. Therefore, we assert that our dataset and proposed task can contribute to the advancement of LLMs towards this goal.














\section{Discussion\& Applications}

LLMs trained to excel on the Code Coverage Prediction task could offer a promising alternative to traditional execution-based code coverage measurement in various scenarios. In this section, we discuss several use case scenarios where this approach can be valuable and beneficial.

\subsection{Expensive Build \& Execution}

For large software projects with millions of lines of code and numerous dependencies, the build and execution process can be time-consuming and expensive. In such cases, developers may want to analyze the code coverage obtained by newly written tests without waiting for the lengthy build phase. By leveraging LLMs trained on the Code Coverage Prediction task, developers can predict the coverage obtained by the new tests on existing methods without the need to build the entire project or execute the tests. This enables developers to quickly assess whether additional tests are required to cover missed lines or branches in the methods, saving valuable time and resources.

\subsection{Limited Code Availability}

Traditional code coverage computation requires the complete source code of the codebase to be available for instrumentation and execution. However, there are scenarios where only a partial view of the code is accessible, making code coverage computation impossible using traditional methods. 

In cases where limited code availability poses a challenge, the Code Coverage Prediction approach can be employed. For example, when utilizing an AI code generation service from an IDE, developers may transmit only a partial view of the code to the server where the AI model resides. In this scenario, the server can use the proposed approach to predict the code coverage of the AI-generated test cases on the given method. This enables estimation of the code coverage without the need for the entire codebase, addressing privacy concerns and network limitations. The predicted code coverage can then be used to make informed decisions, such as generating additional tests if coverage is insufficient or transmitting the generated tests to the user if coverage is satisfactory.

\subsection{Live Coverage}

Live Unit Testing, integrated into various IDEs, allows developers to receive real-time feedback on the impact of code changes on existing tests and identifies whether newly added or modified code is covered by existing tests. In this scenario, the Code Coverage Prediction approach can be applied by replacing the actual execution of test cases with an AI inference call to predict the coverage on the modified or newly added methods. This provides developers with immediate feedback on code coverage without the need for executing the entire test suite. By utilizing LLM-based models for code coverage prediction, developers can streamline the testing process and receive timely insights into the coverage of their code changes.

\section{Conclusion}
In this paper, we introduced the novel task of Code Coverage Prediction, which aims to assess the capabilities of Large Language Models (LLMs) in understanding code execution by accurately predicting the lines of code that are executed based on given test cases. We curated a comprehensive dataset named \dataset, consisting of coverage-annotated methods derived from the HumanEval dataset. This dataset enables researchers to explore and advance code coverage prediction techniques and LLM code understanding.

We evaluated the performance of four state-of-the-art LLMs, namely OpenAI's GPT-4 and GPT-3.5, Google's BARD, and Anthropic's Claude, on the Code Coverage Prediction task. The results demonstrated that GPT-4 achieved the highest performance, with 10.46\% exact match with zero-shot prompting and 24.48\% with multi-shot prompting. However, none of the models, including GPT-4, achieved high accuracy in predicting code coverage, indicating that LLMs still have a long way to go in developing a deep understanding of code execution.

The Code Coverage Prediction task serves as a valuable metric for assessing code understanding and can potentially contribute to the enhancement of LLMs' overall performance on code-related tasks. By training models to excel in this task, we can improve their ability to comprehend code execution dynamics, which is crucial for tasks such as code generation and test generation.










%% The next two lines define the bibliography style to be used, and
%% the bibliography file.
%\bibliographystyle{ACM-Reference-Format}
%\bibliography{main}


%\bibliographystyle{IEEEtran}
% \bibliographystyle{ACM-Reference-Format}
\bibliography{main}
\bibliographystyle{acl_natbib}


\newpage


\appendix

\section{\dataset Example}

% ---------- Data Example ----------------
% Figure environment removed
% ---------- END Code Example ----------------


% \section{Data Collection}
% \MICHELE{This section will probably be removed entirely. However, I'm thinking whether we could use some of these paragraphs to describe future work for the dataset expansion. Like data augmentation, mutation, etc.}
% In order to train our models to predict code coverage, we require a large amount of data including methods, tests, and corresponding coverage information. To collect this data, we first extract test cases for open source GitHub repositories, then we execute test cases individually while measuring their coverage. We also employ data augmentation techniques to expand our dataset.


% \subsection{Test Extraction}
% We begin by cloning GitHub repositories locally and analyze them in order to discover and extract test cases. Our approach can be applied to any programming languages, as long as sufficient data is extracted. In this paper we provide details for extracting tests in Java, Csharp, and python, but similar steps can be performed on other programming languages. We automatically analyze the source code of the repositories to identify test cases. The identification relies on language-specific rules. For example, in Java test cases are marked with the \texttt{@Test} annotation, in Csharp tests can be marked using annotations such as \texttt{[Test]}, \texttt{[TestMethod]}, \texttt{[Fact]}, \texttt{[Theory]} based on the testing framework used by the developers. In python tests are usually defined using the prefix \texttt{test\_} in their signature. We use these rules to identify the tests, extract their source code and fully qualified signatures.


% \subsection{Test Execution \& Coverage}
% In this stage, execute the extracted test cases individually using the corresponding testing framework (\eg JUnit, MSTest, pytest). During the execution, we also enable the coverage computation using existing tools such as Cobertura for Java, Coverage.py for python, and the included coverage tool for MSTest.

% Next, for each test case $t$, we analyze the corresponding coverage report obtained by the test execution in order to extract the corresponding focal method $m$ as well as the annotated coverage $\mathrm{cov}(m,t)$. The coverage report is often organized as an xml file where each source code line in the repository is annotated with coverage information, specifying whether the line has been executed or not.

% Given the test case $t$, we analyze the coverage report and we define as focal method $m$ the method which has the following properties: (i) has a direct invocation in the test $t$; (ii) the first statement (signature) is marked as executed in the coverage report.

% Once the focal method $m$ is determined for the test case $t$, we extract the corresponding annotated coverage $\mathrm{cov}(m,t)$ from the coverage report. At the end of this process, we obtained a dataset where each data point is formed by a triplet $d = \{ m, t, \mathrm{cov}(m,t) \}$.


% \subsection{Data Augmentation}
% We employ two data augmentation techniques to increase the number of data points in our dataset.

% \subsubsection{Mutation}
% We perform random mutation (\ie small modifications in a piece code according to mutation operators) at statement- and branch-level of tests and methods. Given the pair of focal method and test $(m, t)$ we can randomly mutate the method to obtain the new pair $(m', t)$ or similarly mutate the test case and obtain a new pair $(m, t')$. Note that multiple types of mutations can be applied to obtain more pairs. Each new pair is then executed and coverage is obtained. These mutations may or may not affect the coverage obtained by the test, however they are still meaningful data points that can be used to train the model to discern whether specific differences in source code (mutations) influence the coverage.


% \subsubsection{Automated Test Generation}
% We can generate new tests by relying on automated test generation techniques, thus increasing our dataset.

% To do so, we employ classic test generation techniques that are guided by coverage, such as EvoSuite \cite{fraser2011evosuite}, as well as novel AI-based techniques such as AthenaTest \cite{DBLP:journals/corr/abs-2009-05617}.

% Both classes of techniques provide meaningful data for our model. The classic test generation technique, while creating simpler, and machine-alike tests, can be used to train our model in predicting coverage for those types of tests, which are often used by large organization. AI-based techniques are able to generate more realistic and developer-alike tests, which can augment the existing dataset extracted from open source repository.


% \section{Use Case Scenarios}

% \label{sec:scenarios}
% This approach could represent a viable alternative to execution-based coverage in several circumstances, such as: (i) program build and execution is too expensive; (ii) code coverage needs to be invoked multiple times iteratively; (iii) only a snippet of code is available, but not the entire program (e.g. server-side scenario).

% In the following subsections we'll provide some examples of these use case scenarios.

% \subsection{Expensive Build \& Execution}
% Large software projects containing millions of lines of code and hundreds 
%  of dependencies may require a long and expensive build. It is fairly normal for these project to have a build that requires hours to complete.

% Let's consider the scenario where a developer is working on creating new tests for an existing set of methods, belonging to a large codebase. The developer would like to analyze the code coverage obtained by these newly written tests, without waiting for the expensive and time consuming build phase. The developer relies on the proposed approach to predict the coverage obtained by the new tests on the existing methods, without needing to build the project and execute the tests. This allows the developers to quickly understand whether they need more tests to cover missed lines/branches in the methods.


% \subsection{Fine-Grained Code Coverage Analysis}
% To compute code coverage, the software project needs to be instrumented and tests executed on the instrumented codebase. However, when a group of tests are executed in parallel on the instrumented codebase, the obtained coverage represents the aggregated coverage for all the executed tests. Specifically, the coverage accounts for all the lines/branches covered by the tests, without fine-grained details on which test covers a specific line/branch.

% In order to obtain fine-grained code coverage for each individual test, the tests need to be executed individually one-at-a-time (in sequence, not in parallel) and code coverage shall be saved and reset after each test execution. This process is very time consuming, since test executions' cannot be parallelized and additional overhead is introduced. 

% Let's consider the scenario where a developer wants to obtain fine-grained code coverage for each individual test case. This fine-grained code coverage would allow the developer to only run a subset of tests based on what has been changed in future commits (\ie this problem is referred to as Test Selection problem in literature). To do so, instead of running each individual test one-at-a-time and saving/resetting the coverage, the developers decides to use our proposed approach to predict the code coverage of each test case. This allows the developer to quickly obtain fine-grained code coverage.


% \subsection{Limited Code Availability}
% Classic code coverage computation requires the entire codebase's source code to be available, instrumented, and executable. There are several scenarios where only a partial view of the code is available and code coverage is required. This is currently not possible with the existing code coverage computation but is feasbile with our approach.

% Let's consider the scenario where a developer is using an AI code generation service from the developer's IDE (\eg Visual Studio Code). The developer asks the AI service to generate test cases for an existing method. The code of the method, including some surrounding context, is transmitted to the server where the AI model resides. Note that only a partial view of the code is transmitted to the server, this is due to privacy reasons as well as network considerations. The AI model (on the server-side) received the input method and generates a set of candidate test cases. These test cases cannot be executed on the server-side since they require the entire codebase (available only at the user-side). In order to estimate the code coverage obtained by the AI-generated tests on the given method, the server invokes our proposed approach. The infrastructure on the server-side analyzes the code coverage results and decides whether to generate additional tests (\ie tests are missing lines/branches) or transmit the generated tests to the user (\ie the code coverage is satisfactory).

% Our approach allows code coverage computation even with limited code availability.


%  \subsection{Live Coverage}
% Live Unit Testing is a technique, integrated in several IDEs such as VS, which involves the execution of tests in background while the developer is making changes to an application. This is intended to provide feedback to developers on how the changes impacted existing tests and whether the new added code is covered by one or more existing tests.

% Our approach can be used in this scenario by replacing the actual execution of test cases with an AI inference call that predicts the new coverage on modified or newly added methods. 


\newpage

\section{Deployed Systems}
We deploy our approach in two systems covering some of the use cases described in the paper.

\subsection{System A - Live Coverage}
Figure \ref{fig:system_A} shows the deployment of System A, which provides live coverage prediction for developers directly into their IDE. System A supports the scenario where a developer is writing tests for a given method (\eg \texttt{Fibonacci(n)}) in their codebase. System A provides live coverage information (bottom of Figure \ref{fig:system_A}) where lines covered by the tests are marked with \colorbox{green}{>} and highlighted in green and the line missed are marked with \colorbox{red}{!} and highlighted in red.

The benefits provided by System A are the following: (i) no need to build the entire codebase; (ii) no need to execute the tests; (iii) live and lightweight coverage prediction.


% Figure environment removed


\subsection{System B - Test Generation with Coverage}
Figure \ref{fig:system_B} shows the deployment of System B, which provides Test Suites with a coverage guarantee. System B supports the scenario where a developer is requesting test cases for a given method and would like to obtain a certain degree of coverage on the method under test. Once the method is transmitted to the Test Generation Service, the Test Generation Model (\ie an AI-based test generation tool or any other tool) outputs a first batch of test case candidates. The Coverage Prediction Model analyzes these tests and the method under test, and predicts the coverage that these tests achieve on the method. If the coverage is satisfactory (w.r.t. a given criteria and threshold) the tests are transmitted to the IDE and shown to the developer. If the tests do not meet the criteria in terms of coverage, the Test Generation Service requests additional tests from the Test Generation Model (optionally, providing the specific lines/branches which still need to be covered). 

The benefits provided by System B are the following: (i) automated test generation with coverage guarantees; (ii) lightweight generation without need of build and test execution on the user side.


% Figure environment removed










% % ---------- Code Example ----------------
% % Figure environment removed
% % ---------- END Code Example ----------------




% \balance
%\begin{appendix}
%\section{Examples}
%\section{\add{Assembly-level Verification for Page Table Traversal and Mapping}}
\label{sec:experiment_appendix}
%To both validate and demonstrate the value of the modal approach to reasoning about virtual memory management, 
% we study several
% We validate our logic by studying
% distillations of key VMM functionality.
% real concerns of virtual memory managers.
% Recall from Section \ref{sec:logic} that virtual points-to assertions work just like regular points-to assertions, by design.

\replace{
In this section we verify several critical and challenging pieces of VMM code.
First, in several stages, we work up to mapping a new page in the current address space.
This requires a number of independently challenging substeps: dynamically traversing a page table to find
the appropriate L1 entry to update; inserting additional levels of the page table if necessary (updating
the VMM invariants along the way);
converting the physical addresses found in intermediate entries into the corresponding virtual addresses
that can be used for memory access;
installing the new mapping;
and collecting sufficient resources to form a virtual points-to assertion.
Of these, only the second-to-last step (installing the correct mapping into the
current address space) has previously been formally verified with respect to a machine model with address translation.
Second, we formally verify a switch into a new address space as part of a task switch,
the first such verification handling both old and new processes' assertions (in different address spaces) at the time of the switch.
}{
While our logic was developed and proven sound for x86-64 assembly,
Section \ref{sec:traversingC} described verification of software page table walking code (\lstinline|pte_get_next_table| and \lstinline|walkpgdir|)
as if at the level of C for improved readability.
This appendix describes the actual assembly-level verification carried out in Rocq.
Careful readers of both Section \ref{sec:traversingC} and this appendix will notice
strong similarities in the assertions and and reasoning, for good reason:
The C code in Section \ref{sec:traversingC} was the original kernel code that was compiled
(with no optimizations) to x86-64 assembly and verified with our logic, and the proof outlines
in that section largely back-port the assembly proofs back to C.
\looseness=-1
}

\add{
 This section describes the assembly proofs without reference to the C outlines given in Section \ref{sec:traversingC}.
 The main additional details of note at the assembly level are:
 \begin{itemize}
 \item Accurate treatment of register management (particularly the AMD64 System V calling convention) leading to more direct correspondence
       with our logic
 \item The assembly is naturally more verbose than the C, so the proof outlines are relatively more sparse, with assertions written
       only for key updates.
 \item Bitwise manipulations of page table entries are harder to follow than C's bitfield access support.
       Multiple manipulations which are each explicit in C become adjacent (or sometimes non-adjacent) bitwise operations.
       The critical ones are commented in the assembly figures.
 \item And compared to the C-based presentation earlier, there are differences in logical variable names. For example,
       the assembly proofs use \textsf{entry} as the name for the \emph{physical} address of the entry modified by
       \lstinline|pte_get_next_table| in Figure \ref{fig:calltopteinitialize}, whereas to make sense of the C code
       in Figure \ref{fig:calltopteinitializeC} we used \textsf{entry} consistently with the C variable name and introduced
       separate logical names for physical addresses. This propagates to figures presenting larger invariants separately,
       as they also refer to the logical names from the proofs.
 \end{itemize}
}

%\begin{comment}
%\todo[inline]{Identity mappings are difficult, and our current approach won't quite work. Consider trying to have a virtual pointsto for an actual page table entry (i.e., that one could use to update a page table mapping), while also having a virtual pointsto for an address that entry mapped. With the current (let's call it v1) solution, we can't actually have both of those simultaneously!  That's because the PTE pointsto will assert full ownership of the physical memory cell holding the PTE as its data value, while the virtual pointsto for the data mapped by that entry will \emph{also} assert (fractional) ownership of all entries a page table walk would traverse.
%}
%\todo[inline,color=violet]{This doesn't seem to cause issues with the mapping/unmapping examples, only with changing intermediate page table pointers. The mapping example requires a virtual pointsto for the blank PTE, and once filled in that ownership can be immediately split to create the 512 new virtual pointsto assertions for the newly mapped page. Conversely, for unmapping we'd assume ownership of all the relevant virtual pointsto assertions for the page we're unmapping, at which point we can (with a bit of work) show that they all correspond to the same L1 PTE, and extract the 512 fractional shares of that entry from the pointsto assertions.  But changing intermediate page tables, as one would do for coallescing or splitting a superpage while preserving the virtual-to-physical mappings, couldn't be done without some really complicated separating implication tricks.}
%\todo[inline,color=green]{One possible approach to resolving this, which we came up with in our Tuesday meeting, is to recognize that the current (v1) virtual points-to is too strong, because it really doesn't care about \emph{owning} those fractional resources, it only cares that \emph{something} ensures the correct page table walk exists. Iris has a ghost map resource where authoritative ownership of an individual key-value pair can be handled as a resource.  (Colin was using this in the filesystem cache.)
%We can use that mechanism to separate the virtual-to-physical translation from the physical memory involved (Kolanski and Klein may have done something similar for different reasons): (fractional) virtual points-to assertions can be defined in terms of (fractional) ownership of these authoritative ghost map entry assertions, plus sharing an invariant that the current installed page table respects all entries of the mapping. Unmapping collects the authoritative map kvpairs from collecting the assertions, and then can remove them from the ghost map and update the page tables. Critically, physical ownership of the page tables then lives in the invariant on the current page table, so some virtual pointsto assertions can refer to memory in those page tables.
%This still works with the modality, since that invariant is also semantically a predicate on a page table root.
%Let's call this v2.
%}
%\end{comment}
\subsection{Traversing Live Page Tables}
\label{sec:traversing}
We build up to the main task of mapping a new page after traversing page tables in software.
The mapping operation of Figure \ref{fig:mapping_code} assumes an operation \textsf{walkpgdir} which must traverse the page tables
in order to locate the address of the L1 entry to update --- 
% possibly allocating tables for levels 3, 2, and 1 in the process,
% installing them into levels 4, 3, and 2, along the way.
possibly allocating new L3, L2, and L1 tables as necessary.
Traversing the page tables is itself challenging functionality to verify: loading the current table root from \lstinline|cr3| is straightforward
(a \lstinline|mov| instruction), however this produces the physical address of \lstinline|cr3|, not the virtual address the kernel code would use to access that memory.
This problem repeats at each level of the page table: assuming the code has \emph{somehow} read the appropriate L4 (or L3, or L2) entry, those entries again
yield physical addresses, not virtual.

\subsubsection{Loading Page-Table Address Value}
We will discuss access to the level 4 table later (Section \ref{wlkpgdir}). But for subsequent levels, the base address of level $n$ must be
fetched from the appropriate entry in the level $n+1$ table.
This is the role of \lstinline|pte_get_next_table| (Figures \ref{fig:calltopteinitialize} and \ref{fig:p2v}):
it is passed the virtual address of the page table entry in level $n+1$, and should return the \emph{virtual} 
address of the \emph{base} of the level $n$ table
indicated by that entry.
If the entry is empty (i.e., this is a sparse part of the page table representation),
the code also allocates a page for the level $n$ table, installs it in the level $n+1$ entry, and establishes appropriate invariants.
Figure \ref{fig:calltopteinitialize} presents the initial part of the function, which performs the allocation if necessary.
Figure \ref{fig:p2v} (discussed in Section \ref{sec:p2v}) deals with the cases where no allocation is necessary \emph{or} the allocation has already
been performed by the code in this figure.
\looseness=-1

Note that the specification does \emph{not} assume a specific page table level --- logical parameter \textsf{v} represents the level
of the entry passed as an argument, and this code
is used for all three level transitions when traversing page tables (4 to 3, 3 to 2, 2 to 1).
This comes into play with a subtlety of the specification of \lstinline|pte_get_next_table| that we will
revisit several times: \lstinline|pte_get_next_table|'s specification
assumes it is given a virtual \emph{vpte-pointsto}
(a virtual points-to exposing the underlying physical address instead of existentially quantifying it;
 see Section \ref{sec:mapnew}) granting access to the specified entry,
but its postcondition does not yield new virtual points-to assertions!
Instead it merely computes the base virtual address of the next table, and returns adequate capabilities (discussed in Section \ref{subsec:identitymappings})
for the \emph{caller} to construct a vpte-pointsto for any entry of the next table level (if this is not an L1 entry ---
the caller knows which level of the table this is for).
\looseness=-1

Within \textsf{get\_next\_table}, after a standard function prologue, the code 
loads the entry pointed to by the argument (logical variable \textsf{entry} in the proof outline).
This is a page table entry: a 64-bit word divided into bit-fields for
the physical address of the next table, and control bits like the valid bit, as discussed in 
Section \ref{sec:backgroundonmachinemodel}.



\ifPLDI
Line \ref{line:mask_present} checks % In the condensed figure, it's all on one line
\else
Lines \ref{line:mask_present}--\ref{line:check_entry_present} check
\fi
if the entry's ``present'' bit is set.
If it is zero, a new page must be allocated for the next level of the table --- which is done by the fall-through
from Line \ref{line:check_entry_present_jump}'s conditional jump. Otherwise the code jumps ahead to
the case for the next level already existing, which is discussed in Section \ref{sec:p2v} and Figure \ref{fig:p2v}.
First, we must discuss another refinement of the address space invariant, establishing
enough structure on the page tables themselves to allow the traversal.
The code for allocating a new level of the page table must establish this extended invariant.

%wshiftll (wshiftll (natToWord 64 entry) (WordImpl.concat (WordImpl.zero 56) (WordImpl.from_nat 8 12 ^& WordImpl.concat (WordImpl.zero 2) WO~1~1~1~1~1~1)) ^& constf)
%(WordImpl.concat (WordImpl.zero 56) (natToWord 8 12 ^& WordImpl.concat (WordImpl.zero 2) WO~1~1~1~1~1~1))
%
%wshiftll
 %      (wshiftll
%          ((((natToWord 64 entry ^& WordImpl.concat (WordImpl.zero 32) consta ^| WordImpl.concat (WordImpl.zero 32) (natToWord 32 2))
%             ^& WordImpl.concat (WordImpl.zero 32) constb ^| WordImpl.concat (WordImpl.zero 32) (natToWord 32 4)) ^& constd
%            ^| wshiftll
%                 (wshiftll (nextpaddr ^+ ^~ (natToWord 64 KERNBASE))
%                    (WordImpl.concat (WordImpl.zero 56) (WordImpl.from_nat 8 12 ^& WordImpl.concat (WordImpl.zero 2) WO~1~1~1~1~1~1))
%                  ^& constf)
%                 (WordImpl.concat (WordImpl.zero 56) (WordImpl.from_nat 8 12 ^& WordImpl.concat (WordImpl.zero 2) WO~1~1~1~1~1~1)))
%           ^& WordImpl.concat (WordImpl.zero 32) conste ^| wone 64)
%          (WordImpl.concat (WordImpl.zero 56) (WordImpl.from_nat 8 12 ^& WordImpl.concat (WordImpl.zero 2) WO~1~1~1~1~1~1)) ^& constf)
%       (WordImpl.concat (WordImpl.zero 56) (natToWord 8 12 ^& WordImpl.concat (WordImpl.zero 2) WO~1~1~1~1~1~1)) 
% Figure environment removed

\subsubsection{Identity Mappings}
\label{subsec:identitymappings}
Kernels need to convert between physical and virtual addresses, in both directions.
Traversing the page tables in software is the simplest way to convert a virtual address to a physical address; this is the context we are working up to.
However, implementing this virtual-to-physical (V2P) translation in this way ironically requires physical-to-virtual (P2V) translation,
because the addresses stored in page table entries are physical, but memory accesses issued by the OS code use virtual addresses.
% There is no universal way to convert physical addresses to virtual --- doing so relies on the kernel maintaining careful invariants or
% additional data structures to enable P2V translation.
\looseness=-1

Because VMM operations are performance-critical for many workloads, most kernels 
maintain invariants that enable very fast P2V conversions (rather than adding another data structure).
Most kernels maintain an invariant on their page tables that the virtual address of any page used for a page table 
% lives at a virtual address whose value 
is \emph{a constant offset from the physical address} --- a practice sometimes referred to as \emph{identity mapping} 
(even though the physical-to-virtual translation
is typically not literally the identity function, but adding a non-zero constant offset).\footnote{Some kernels do this for all physical memory on the machine, simplifying interaction
with DMA devices.
On newer platforms like RISC-V, this sometimes truly is an identity mapping ---
x86-64 machines are forced into offsets by backwards compatibility with bootloaders that cannot access the full memory space of the
machine.
}

For this reason we extend the per-address-space invariant as in Figure \ref{fig:peraspaceinvariant_with_p2v_extension}, to also track which
addresses we can perform a P2V conversion on by a adding a constant offset.
$\Xi$ is another ghost map, from physical addresses to the level of the page table they represent (1--4).
\emph{Only} physical addresses in $\Xi$ can undergo P2V conversion. 
Section \ref{sec:p2v} describes the actual conversion,
but we describe the invariant here 
because adding new level 3/2/1 tables must maintain the invariant.

% Figure environment removed

For each $\paddr\mapsto \textsf{v} \in\Xi$, the invariant tracks a virtual points-to justifying that virtual address $\paddr+\textsf{KERNBASE}$ maps to physical address $\paddr$
(the ``Ghost translation'' in Figure \ref{fig:peraspaceinvariant_with_p2v_extension});
fractional ownership of the physical memory for that page table entry;
and for valid entries (with the present bit set) above L1, ghost map tokens for every entry in the table pointed to by the entry, which can be used
to repeat the process one level down. 
% (L1 entries point to data pages, whose physical memory ownership resides in some virtual points-to).
The assertion on Line \ref{line:conditional_children} of Figure \ref{fig:calltopteinitialize} comes from the invariant one level up; 
if the valid bit is set,
the code can return those child tokens without the conditional guard.
\looseness=-1

The fractional ownership of the entry's physical memory is subtle. Recall that $\textsf{L}_{4}\_\textsf{L}_{1}\_\textsf{PointsTo}$ retains some physical
ownership of each page table entry that is traversed (proportional to how many virtual addresses share the entry).
So in general the invariant cannot keep full permission to the memory in this part of the invariant, or it would overlap the page table walk for virtual points-to
assertions. But in the case where the entry is invalid, we may need to write to it (e.g., to install a reference to a next-level table, as we do in Figure \ref{fig:calltopteinitialize}),
which requires full permission. Fortunately, the entry can only be in use if its valid bit is set; if the valid bit is not set we know
that no virtual points-to entry in $\delta$/$\theta$ holds any partial ownership.
Thus we use the invariant portion annotated as ``Entry validity'' in Figure \ref{fig:peraspaceinvariant_with_p2v_extension} to capture this:
if the entry is invalid the invariant holds full ownership of the entry, so it can be updated; while if the entry is valid,
the invariant owns only a constant non-zero fragment sufficient to read the entry, but not modify it (which would invalidate some virtual points-to assertions):
\begin{equation*}
 \ulcorner \textsf{qfrac} = 1 \leftrightarrow \; \lnot\textsf{entry\_present }(\vale) \urcorner \tag{*}
\end{equation*}
Thus the fractional ownership of the physical location is enough for Line \ref{line:read_entry_contents} in Figure \ref{fig:calltopteinitialize} to access the entry, though in \lstinline|get_next_table|
the caller has pulled that piece of information out of the invariant and passed it for the entry at hand.
This removal appears explicitly in assertions,
as the argument to the invariant is $\Xi\setminus\{\mathsf{entry}\}$ (indexing by the set $\Xi$ allows us to borrow the physical resources
for a specific page table entry out of the invariant, and later put them back).
Line \ref{line:check_entry_present_jump}'s conditional then determines in the fall-through case that the bit is not set, which 
together with other facts entails $\textsf{qfrac} = 1$ at Line \ref{line:after_concluding_qfrac1},
and permits storing a new entry (in ellided code around Line \ref{line:install_new_entry}).
\looseness=-1

This seemingly-simple piece of code has a highly non-trivial correctness argument, which depends critically on detailed invariants on how access to page table
entries is shared between parts of the kernel. No prior work has engaged with this problem.

% Concretely speaking, going back to Line 15 in Figure \ref{fig:calltopteinitialize}, to read the value referenced by physical address \textsf{entry} while preserving the soundness of memory mappings, our extended invariant introduces the side condition (*)
% \begin{equation*}
%  \ulcorner \textsf{qfrac} = 1 \leftrightarrow \; \lnot\textsf{entry\_present }(\vale) \urcorner \tag{*}
% \end{equation*}
% assuring that looking the identity mapping for \textsf{entry} is safe under the subtle justification which equates the full ownership to the non/presence of the entry which can only be known when investigated in Line 21 in Figure \ref{fig:calltopteinitialize}.
\begin{comment}
 % Figure environment removed
\end{comment}


 \subsubsection{Installing a New Table}
 After obtaining the identity mapping for \textsf{entry}, we are able to load the \textsf{entry\_val} into \textsf{rdi}, and check the presence bit through
\ifPLDI
Line \ref{line:mask_present} % in condensed version, all on same line
\else
Lines \ref{line:mask_present}--\ref{line:check_entry_present} 
\fi
in Figure \ref{fig:calltopteinitialize}.
Accessing the presence bit and checking the value allows us to exploit the condition (*) that was just discussed when verifying the allocation
path (i.e., when the entry is invalid  and Lines \ref{line:alloc_path_start}--\ref{line:alloc_path_end} in Figure \ref{fig:calltopteinitialize}
must allocate the next level of tables).
This operation is subtle. To reiterate: the operation requires that the relevant table entry is readable, but the exact portion of ownership 
returned must be determined by inspecting the valid bit of the value in memory --- so full ownership is returned only for unused entries.
When the bit is not set, that entails full ownership of the entry's memory ($\textsf{qfrac} = 1$) and justifies writing to that memory.
Otherwise, the code jumps past the end of this listing, to the following code at the top of Figure \ref{fig:p2v} (which is also the
continuation of this code).

% Figure environment removed

If the entry is not set, \textsf{pte\_initialize} (Line \ref{line:call_to_pte_initialize} in Figure \ref{fig:calltopteinitialize}) 
allocates a physical page (internally utilizing the only unverified (trusted) code in our case studies, the page-allocator's \textsf{kalloc},\footnote{
  This is an allocator for regions of pre-zeroed physical memory that is mapped, but not accessed by the allocator itself,
  as is typical for slab allocators~\cite{bonwick1994slab}.
  Its verification would be similar to verifying a usermode \textsf{malloc} verifications~\cite{Chlipala2013Bedrock,wickerson2010explicit},
  just with additional invariants on the memory pool.
} 
on Line \ref{line:call_to_kalloc} in Figure \ref{pteinitializespec}). 
Since we are using \textsf{pte\_initialize} for page-table address allocation, we must relate this newly
allocated physical address to the identity mapping map $\Xi$ --- 
see Line \ref{line:page_of_caps} in Figure \ref{fig:calltopteinitialize}, where
\texttt{kalloc}'s specification guarantees it has returned memory from a designated memory
pool that is already mapped
\ifPLDI
\else
\footnote{A reasonable reader might wonder where this pool
initially comes from, and how it might grow when needed. Typically an initial mapping subject to this identity mapping
constraint is set up prior to transition to 64-bit kernel code (notably,
a page table must exist \emph{before} virtual memory is enabled during boot, as part of enabling it is setting
a page table root).
Growing this pool later requires cooperation of physical memory range allocation and virtual memory range allocation,
typically by starting general virtual address allocation at the highest physical memory address plus the identity mapping offset.
This reserves the virtual addresses corresponding to all physical addresses plus the offset for later use in this pool,
as needed.
} 
\fi
and satisfies the offset invariants.
% \todo[inline,color=blue]{colin frontier.
% Stuck with line 31 onwards in Figure 7. rax holds nextpaddr, but I think that should be entrypfn, and 
% the explicit entrypfn id token assertion should go away, as its covered by the forall assertion.
% then the postcondition for pte-initialize should have a specific level now for the entries,
% like 0, which can be updated in the view shift on line 42.
% }
% Focusing on the specification of \textsf{pte\_initialize} separately in Figure \ref{fig:pteinitializespec}, 
% we right immediately realize that instead of seeing see a physical pointsto for the fresly page-table address 
% (e.g. $\mathsf{nextpaddr} \mapsto_{\mathsf{p}} \mathsf{w64\_0}$) deliberately in the post-conditoin in Lines 15-16,
%  we observe a full-ownership token representing the knowledge that a frame and all the entries indexed from this 
% frame are freshly allocated with full-ownership to be a part of the identity map, $\Xi$. 
The soundness argument of this specification relies on the fact that these freshly allocated resources are part 
of an entry construction that has not been completed yet: the presence bit is set 
(Line \ref{line:install_new_entry} in Figure \ref{fig:calltopteinitialize}) after these freshly allocated resources are incorporated to the 
entry construction via the page-frame portion of the PTE. In other words, the side condition, (*),
 formalizes that any access to the entry with these resources is \textit{invalid} (in the sense of not necessarily
having accompanying resources) until the entry is marked present (and thus the memory returned from \textsf{kalloc}
moves into the page table invariant.

\add{Note that the C presentation in Figure \ref{fig:calltopteinitializeC}
omitted the precondition on the implication of Figure \ref{fig:calltopteinitialize}'s Line \ref{line:page_of_caps},
which is logically equivalent to \textsf{True} since \textsf{entry\_present} checks if the present bit is set in an entry,
and \textsf{pte\_initialize} sets that bit. The actual invariant has this form here, and in the postcondition
of \lstinline|pte_initialize| (Figure \ref{pteinitializespec}), to match the conditional form from earlier in
\lstinline|pte_get_next_table| (which is also provably true when the check of the present bit
determines that the entry was already valid/present).
Our proof discharges the conditional at the join point, rather than eagerly in each branch.
}

\subsubsection{Physical-to-Virtual Conversion with \textsf{P2V}}
\label{sec:p2v}
Once we know the entry refers to a physical address in the identity mapping range ($\Xi$)
(via the branch at Line \ref{line:check_entry_present_jump}, or  by allocating and installing a new entry
as just discussed for Lines \ref{line:check_entry_present_jump}--\ref{line:end_of_allocation_path}), 
we can convert this frame address to a corresponding virtual address via the identity mappings
discussed in Section \ref{subsec:identitymappings} and Figure \ref{fig:peraspaceinvariant_with_p2v_extension}.
in the last lines of \lstinline|pte_get_next_table| shown in Figure \ref{fig:p2v} (the continuation of Figure \ref{fig:calltopteinitialize}).
This is a critical piece of the full page table walk verification.
In our small kernel (Line \ref{line:p2v} in Figure \ref{fig:p2v}), as in larger kernels, the C macro \texttt{P2V} common to many kernels
is actually just addition by the constant offset mentioned in Section \ref{subsec:identitymappings}.
But the correctness of this simple instruction is quite subtle.
%  and cannot be proven 
% without the extended invariant (Figure \ref{fig:peraspaceinvariant_with_p2v_extension})
% worked out Section \ref{subsec:identitymappings}.

% Figure environment removed
Figure \ref{fig:p2v} shows the verification of the end of \lstinline|pte_get_next_table| specialized to the case where 
where no allocation was necessary (i.e., the conditional on Line \ref{line:check_entry_present} of Figure \ref{fig:calltopteinitialize} was taken).
In this case, the true present bit allows access to the child tokens from Line \ref{line:conditional_children} of Figure \ref{fig:calltopteinitialize},
which is then refined to the assertion on Line \ref{line:children} of Figure \ref{fig:p2v}.
The code loads \lstinline|rcx| with the offset value \textsf{KERNBASE}, which gives us the value of the virtual address ($\textsf{entry}_{\textsf{pfn}}$ \textsf{+KERNBASE})
of the \emph{base} of the next level of the page table.
% \todo[inline]{the next sentence depends on having figure 10 updated to reflect the page-worth of tokens}
While we could now convert this address to a virtual points-to, this is not necessarily the correct thing to do.
The caller \lstinline|walkpgdir| (discussed next) uses \lstinline|pte_get_next_table| to retrieve just the base address,
because only the caller knows which entry in the subsequent table will be accessed (it depends on the corresponding bits from the virtual
address being translated). So instead we pass back the per-address-space invariant with the identity mapping resources for \lstinline|entry|
pulled out. The caller determines which entry in that table must actually
be accessed --- by selecting the appropriate index into the 512 ghost map tokens returned in the postcondition,
and using the ghost translation and physical location portions of the invariant to assemble a vpte-pointsto
that justifies the caller's subsequent access to a particular entry of the returned table.
% in the identity map ($\Xi\setminus\{entry\}$) of the kernel invariant.
% the logical update in Specification  Lines 5-10 to 10-14 for obtaining virtual-pointsto resource for the frame 
% ($\textsf{entry}_{\textsf{pfn}}$) by removing it from the ghost map ($\Xi\setminus\{entry\}\cup \{\textsf{entry}_{\textsf{pfn}}) \}$) 
% in Line 5 and compute the identity mapping for this physical frame address in Line 13 in Figure \ref{fig:p2v}).

\subsubsection{Walking Page-Table Tree: Calling \textsf{pte\_get\_next\_table} for Each Level}
\label{wlkpgdir}
% Figure environment removed

% Figure environment removed
Implementing a software page-table walk amounts to calling \textsf{pte\_get\_next\_table} for each level as shown in Figure \ref{walkpgdir}. 
The key part of the specification and proof for a page table walk is accumulation of memory mappings for the page-table entries 
visited and frame addresses for page-tables. 
For example, Lines \ref{line:ex_l4_vpte} and \ref{line:ex_l3_vpte} in Figure \ref{walkpgdir} show the virtual pte-pointsto assertions for L4 and L3 entries.
In the final post-condition, we expect the accumulation of these resources from each level -- $\textsf{R}_{\textsf{walk}}$ -- 
which allows us to construct and return the path to the L1 entry in the tree to insert a new page.  

This is the code which performs most actual physical-to-virtual conversions using the identity mapping portion of the per-address-space invariant.
\lstinline|walkpgdir| accepts a \emph{virtual} pointer to the base of the L4 table, and the address to translate.
The precondition provides knowledge that the virtual base of the L4 is at the appropriate offset from the current \lstinline|cr3| value,
but does not provide a virtual points-to assertion --- because the function must calculate (Lines \ref{line:start_pml4_calc}--\ref{line:end_pml4_calc})
which entry it needs access to.
Instead the precondition has 512 identity map tokens, guaranteeing that every entry on the page is subject to the identity mapping invariant.
Line \ref{line:end_pml4_calc} calculates the virtual address of the relevant entry, and the subsequent view shift
pulls that entry out of the identity mapping ($\Xi$) and fetches its corresponding resources as
described by Figure \ref{fig:peraspaceinvariant_with_p2v_extension} and Section \ref{subsec:identitymappings}.
The ghost translation and physical location are used to form the virtual pte-pointsto for the L4 entry
(Line \ref{line:first_pte_pointsto}), with the entry validity and next-level indexing
satisfying the rest of the precondition for \lstinline|pte_get_next_table|.
\lstinline|pte_get_next_table| then, as described earlier, checks the valid bit in the indicated
entry and either returns the (unconditional) tokens for the L3 entry physical addresses (if valid), or
allocates into the entry and returns new (also unconditional) tokens for the L3 entry physical addresses.
\lstinline|pte_get_next_table|'s first call (Line \ref{line:first_getnext_call}) returns
the virtual address of the base of the L3 table (a \emph{page directory pointer}, so PDP, in official
x86-64 terminology). Then the situation to move from that pointer to the base of the L2
is just like the process just followed: the proof calculates the address of the relevant
L3 entry, uses the appropriate L3 identity mapping token to construct a virtual pte-pointsto to that entry,
and passes that along with additional resources pulled out of the invariant to another call to
\lstinline|pte_get_next_table|. That call then returns the base of an L2 table, and the process
repeats until the function returns the virtual address of the relevant L1 entry.
That will then be used in the next section by the caller of \lstinline|walkpgdir|
to install a new mapping.


% \textsf{walkpgdir}, as a client, holds the knowledge that there exists an identity mapping for the physical entry address (\textsf{entry})
%  in the root page table ($\textsf{L}_{4}$):  $\mathsf{entry} \mapsto_{\textsf{id}} \textsf{\_}$ in Specification Line 3 is a partially owned
%  token for accessing and looking up the resources in the identity map, $\Xi$, to construct the \textit{virtual-to-physical} pointsto relation 
% $\textsf{entry+KERNBASE} \mapsto_{\textsf{vpte,qfrac}} \textsf{entry \entry\_val}$ with the virtual address (\textsf{entry+KERNBASE}) obtained 
% by offsetting the physical address (\textsf{entry}). With this knowledge on the root-page-table-entry, we can start traversing the page-table 
% tree which requires locating the address of the next table -- a call to \textsf{pte\_get\_next\_table} shown in Figure \ref{fig:calltopteinitialize}. 
% Beyond a frame, the precondition before Line 15 requires the current address space invariant, and knowledge that \textsf{entry} is mapped to a 
% random entry value, subtly, 
% the operation also, at least, requires that the relevant table entry is readable, but the exact portion of ownership 
% returned must be determined by inspecting the valid bit
% of the value in memory --- so full ownership is returned only for unused entries.
% This is a simple piece of code whose functionality is critical and whose correctness is highly non-trivial. No prior work engages with this problem.



%% Figure environment removed



%\caption{Traversing page-tables, and allocating entries as needed while mapping-a-page in Figure \ref{fig:mappingcode}.}
% \citet{kolanski08vstte,kolanski09tphols} verified a single code block with their logic which was roughly Figure \ref{fig:mapping_code} for a 2-level ARM
% page table, but several critical complexities our work deals with were not addressed.
% First, beyond the limitations discussed in Section \ref{sec:overly-restrictive}, Kolanski and Klein assumed that virtual addresses
% for page tables at each level were given as parameters rather than verifying any conversion from physical addresses to virtual addresses (or even axiomatizing their lookup).
% In contrast, our verification articulates the address space invariant from which the physical-to-virtual translation can be implemented.
% Second, our proof deals with the construction of a valid virtual points-to \emph{to the PTE to update} in mapping, which Kolanski and Klein also
% assumed was given.
% \todo{some of this is really an argument for our verification being more thorough, rather than being about our logic}

% Reasoning about the page table walk in their logic would have required 
% could reason about the walk, but would need to explicitly prove that all other invariants
% of the kernel, the current address space, and all other address spaces of interest were preserved by each update, because their model
% only supports separation within a single address space. In our model, this follows for free from making
% our separation logic directly aware of address translation and internalizing assumptions about other address spaces as further separable assertions.
% Kolanski and Klein did address part of the walk information for a 2-level page table (a possible ARM configuration), but 

% \textsc{seL4} currently still trusts address translations; it models page tables as a data structure in regular memory, thus not capturing the possibility that even
% temporarily destroying the mappings and restoring them can actually crash the OS. \textsc{CertiKOS} papers share little in the way of precise details about
% their virtual memory management, but because their core technology is based on a fork of \textsc{CompCert}, whose model of memory is
% a set of unordered block allocations, we can infer their proofs must also trust these translations.


\subsection{Mapping a New Page}
\label{sec:mapnew}
One of the key tasks of a page fault handler in a general-purpose OS kernel is
to map new pages into an address space by writing into an existing page table via a call\\
\centerline{\textsf{vaspace\_mappage(pte\_t *pml4, void *va,uintptr\_t fpaddr)}}\\
in Figure \ref{fig:mapping_code}.
To do so, with a given allocated a fresh page (\textsf{fpaddr}), then calculate the appropriate
known-valid page table walks (via \textsf{walkpgdir} Line \ref{line:call_walkpgdir} in Figure \ref{fig:mapping_code})  and update 
the appropriate L1 page table entry (Line 35 in Figure \ref{fig:mapping_code});
unmapping is the reverse of the logic we discuss here.
\looseness=-1
%\lstset{
%  columns=fullflexible,
%  numbers=left,
%  basicstyle=\ttfamily,
%  keywordstyle=\color{blue}\bfseries,
%  morekeywords={mov,add,call},
%  emph={rsp,rdx,rax,rbx,rbp,rsi,rdi,rcx,r8,r9,r10,r11,r12,r13,r14,r15},
%  emphstyle=\color{green},
%  emph={[2]cr3},
%  emphstyle={[2]\color{violet}},
%  morecomment=[l]{;;},
%  mathescape
%}
% Figure environment removed

In Figure \ref{fig:mapping_code}, we see an address ($\vaddr$) currently not
mapped to a page ($\theta \; !!\; \vaddr = \texttt{None}$). Mapping a fresh
physical page to back the desired virtual page first requires ensuring
the existence of a memory location for an appropriate L1 table entry.
The code uses a helper function \lstinline{walkpgdir} (discussed again in Section \ref{sec:traversing}).
\textsf{walkpgdir}'s postcondition contains virtual \emph{PTE} pointsto assertions ($\mapsto_{\textsf{vpte}}$)
both for ensuring partial page table walk reaching the
L1 entry (l1e) by asserting that higher levels of the page table exist (R$_{\textsf{walk}}$ in Figure \ref{fig:rwalk}), 
and for allowing access to the memory of the L1 entry via virtual address (R$_{\textsf{l1e}}$ in Figure \ref{fig:rwalk}).

% After obtaining a virtual address \textsf{pte\_addr} in \textsf{rax} backed 
% by the physical memory for the L1 entry that will be used to translate the virtual addresses
% we are mapping, we save it to \textsf{r14} to be updated later in Line 9.

%In the precondition, we see Line 12 allocates a fresh page-aligned, zero-initialized page  (at \textsf{fpaddr}),
%returning a pre-filled PTE entry in \textsf{rax} ($+3$ sets the lower 2 bits).

% , to hold the freshly
% allocated physical page address (\textsf{fpaddr}) in Line X.

We already discussed for the upper level page-tables how the entry-present checks are handled.
However, for L1 entries this check is left to the caller of the 
traversal function \textsf{walkpgdir}. In other words, unlike what we see in R$_{\textsf{walk}}$ for the upper levels where all entry-present
checks have already been performed, the specification in R$_{\textsf{l1e}}$ ensures that page table entry for L1 needs to be checked at the caller site. 
By doing so, as we see in Figure \ref{fig:mapping_code}, the page reference \textsf{fpaddr} is linked to back the virtual address \textsf{va} 
only if it is not already referring to a physical resource (Lines \ref{line:mappage_pte_present_start}--\ref{line:mappage_pte_present_end} in Figure \ref{fig:mapping_code}). 

The crucial step in addition to traversing the page table in Figure \ref{walkpgdir} is actually updating the L1 entry (Line \ref{line:updatepfn} in Figure \ref{fig:mapping_code}),
via the virtual address (\textsf{pt\_entry+KERNBASE}) known to translate to the appropriate physical address, in our example the L1
table entry address ($\textsf{PTE\_ADDR\_TO\_PFN(fpaddr)}$).

Unlike the only prior work verifying analogous code for mapping a new page~\cite{kolanski08vstte,kolanski09tphols}, our proof above
does \emph{not} need to reason directly over the operational semantics,
making this the first verification we know of for mapping a virtual memory page that 
stays entirely at the program logic level.
\looseness=-1
% By incorporating verification of the
% \lstinline|ensure_L1| function (see Section \ref{sec:traversing}), our verification also directly handles several subtle aspects which
% were axiomatized in prior work.
\ifPLDI
\else
\subsection{Unmapping a Page}
\todo[inline]{update (esp. line refs) for new mapping code}
The reverse operation, unmapping a designated page that is currently mapped,
would essentially be the reverse of
the reasoning around line 22 above: given the virtual points-to assertions for all 512
machine words of memory that the L1 entry would map,
and information about the physical location, 
full permission on the L1 entry could be obtained, allowing the construction of a
full virtual PTE pointer for it, setting to 0, and reclaiming the now-unmapped physical memory.
\fi


% % Figure environment removed

% \subsection{Change of Address Space}
% A critical piece of \emph{trusted} code in current verified OS kernels is the assembly code to change the current address space; current verified OS kernels currently 
% lack effective ways to specify and reason about this low-level operation, for reasons outlined in Section \ref{sec:relwork}.

% Figure \ref{fig:swtch} gives simplified code for a basic task switch, the heart of an OS scheduler implementation. This is code that saves the context (registers and stack)
% of the running thread (here in a structure pointed to by \lstinline|rdi|'s value shown in Lines \ref{line:start_save}--\ref{line:end_save} of Figure \ref{fig:swtch}) and restores the context of 
% an existing thread (from \lstinline|rsi| shown in abbreviated Lines \ref{line:start_restore}--\ref{line:end_restore}), including the corresponding change of address space for a target thread in another process.
% This code assumes the System V AMD64 ABI calling convention, where the normal registers not mentioned are caller-save, and therefore saved on the stack of the thread
% that calls this code, as well as on the new stack of the thread that is restored, thus only the callee-save registers and \texttt{cr3} must be 
% restored.\footnote{We are simplifying by ignoring non-integer registers (e.g., floating point, vector registers),
% and the caller-save registers should be initialized to 0 to avoid leaking information across processes, but this captures the key challenges.}
% With the addition of a return instruction, this code would satisfy the C function signature\footnote{This is the function in UNIX 6th Edition 
% with the infamous ``You are not expected to understand this'' comment~\cite{lions1996lions}.}
% \centerline{\lstinline[language=C]|void swtch(context_t* save, context_t* restore);|}\\
% A call to this code begins executing one thread (until just before Line \ref{line:end_save}) in one address space ($\rtv$), whose information will be saved in a structure at address $old$,
% and finishes execution executing a different thread in a different address space (Line \ref{line:end_restore} on) whose information is initially in $new$.

% Because this code does not directly update the instruction pointer, it is worth explaining \emph{how} this switches threads: by switching address spaces and stacks. 
% This is meant to be called with a return address for the current thread stored on the current stack when called. 
% The precondition of the return address on the initial stack requires the callee-save register values at the time of the call: those stored in the first 
% half of the code.
% Likewise, part of the invariant of the stack of the second thread, the one being restored, is that the return address on \emph{that} stack requires the saved 
% callee-save registers stored in that context to be in registers as its precondition.

% The wrinkle, and the importance of the modal treatment of assertions, is that the target thread's precondition is \emph{relative to its address space}, 
% not the address space of the calling thread, which is reflected by
% the other-space modality 
% $[\rtv']( I\texttt{ASpace}(\theta,\Xi,m) \ast \texttt{Pother})$
% in the specfication. 
% The precondition of this code,
% in context, would include that the initial stack pointer (before \lstinline|rsp| is updated)
% has a return address expecting the then-current callee-save register values and 
% suitably updated (i.e., post-return) stack in the \emph{current} (initial) address space;
% this would be part of \textsf{P} in the precondition.
% The specification also requires that
% the stack pointer saved in the context to restore expects the same of the saved registers and stack 
% \emph{in the other address space}. 
% The other-space modality plays a critical role here; \textsf{Pother} would contain these assumptions in the other
% address space.
% \looseness=-1

% % Lines 10--16 save the current context into memory (in the current address space).
% % Line 22 saves the initial page table root.
% % Lines 33--38 begin restoring the target context, including the stack pointer (line 33),
% % which may not be mapped in the address space at that time: it is the stack for the context being
% % loaded into the CPU.
% % The actual address switch occurs on line 45, which is verified with our modal rule for updating \lstinline|cr3|,
% % and thus shifts resources in and out of other-space modalities as appropriate.

% The postcondition is analagous to the precondition, but interpreted \emph{in the new address space}: the then-current (updated) stack would have a return address expecting the new (restored) register values (again, in \textsf{Pother}),
% and the saved context's invariant captures the precondition for restoring its execution \emph{in the previous address space} (as part of \textsf{P}). 

% Immediately after the page table switch, assertions about the saved and restored contexts are
% guarded by a modality for the retiring
% address space \rtv{} (Line \ref{line:modality_switch}), per
% \textsc{WriteToRegCtlFromRegModal} (Figure \ref{fig:wpdamd}),
% because
% there is no guarantee that the data structures of the previous address space are mapped in the new address space.
% The ability to transfer that points-to information out of that modality is specific to a given kernel's design. 
% Kernels that map kernel memory into all address spaces would need invariants
% that justified moving those assertions out of the other-space modality.
% % Following Spectre and Meltdown, this kernel design became less prevalent because speculative execution of accesses to kernel addresses could leak information even if the access did eventually cause a fault (the user/kernel mode permission check was done after fetching data from memory). Thus many modern kernels have reverted to the older kernel design where the kernel inhabits its own unique address space, and user processes have only enough extra material mapped in their address spaces to switch into the kernel (CPUs do not speculate past updates to \texttt{cr3}).
% \looseness=-1

% While prior work has verified context switches within a single address space~\cite{ni2007contexts}, and context switches
% without any code before or after~\cite{syeda2020formal} (i.e., not reasoning about the impact of address space change
% on what data was accessible), this is the first verification handling both.
% \looseness=-1

% \begin{comment}
% \[  
% $\specline{\exists (\entryf ,\;\entrytr,\; \entrytw,\; \entryo,\;\textsf{pte\_addr },\paddr) \; \ldotp\textsf{P} \ast  I\texttt{ASpace}(\theta,m) \ast  \texttt{r14}\mapsto_{\textsf{r}} \_ \ast \texttt{rdi}\mapsto_{r} \vaddr \ast \texttt{rax}\mapsto_{\textsf{r}} \textsf{ pte\_addr} \; \ast }_{\rtv}$
% $\specline{ \ulcorner  \texttt{addr\_L1 }(\vaddr, \entryo) = \paddr \urcorner \ast \ulcorner \texttt{entry\_present } \entryf \land \texttt{entry\_present } \entrytr \land  \texttt{entry\_present } \entrytw \urcorner \; \ast}_{\rtv}$
% $\specline{\nfpointsto{\mask\vaddr\maskfour\rtv}{\mask\vaddr\maskfouroff\rtv}\entryf\qone\naddr \; \ast \nfpointsto{\mask\vaddr\maskthree\entryf}{\mask\vaddr\maskthreeoff\entryf}\entrytr\qtwo\naddr \ast}_{\rtv}$ 
% $\specline{  \nfpointsto{\mask\vaddr\masktwo\entrytr}{\mask\vaddr\masktwooff\entrytr}\paddr\qthree\entryo \;\ast \texttt{pte\_addr} \mapsto_{\texttt{vpte}} \paddr \;(\texttt{wzero 64}) \ast \texttt{rax}\mapsto_{\textsf{r}} \texttt{pte\_addr}  }_{\rtv}$
% mov r14, rax ;; Save that before another call
% $\specline{\textsf{P} \ast  I\texttt{ASpace}(\theta,m) \ast  \texttt{r14}\mapsto_{\textsf{r}} \texttt{pte\_addr} \ast \texttt{rdi}\mapsto_{\textsf{r}} \vaddr \ast \texttt{rax}\mapsto_{\textsf{r}} \textsf{ pte\_addr} \; \ast }_{\rtv}$
% $\specline{ \nfpointsto{\mask\vaddr\maskfour\rtv}{\mask\vaddr\maskfouroff\rtv}\entryf\qone\naddr \ast \ulcorner \texttt{entry\_present } \entryf \land \texttt{entry\_present } \entrytr \land  \texttt{entry\_present } \entrytw \urcorner \ast}_{\rtv}$ 
% $\specline{  \nfpointsto{\mask\vaddr\maskthree\entryf}{\mask\vaddr\maskthreeoff\entryf}\entrytr\qtwo\naddr \ast \nfpointsto{\mask\vaddr\masktwo\entrytr}{\mask\vaddr\masktwooff\entrytr}\paddr\qthree\entryo \;\ast}_{\rtv}$
% $\specline{\texttt{pte\_addr} \mapsto_{\texttt{vpte}} \paddr \;(\texttt{wzero 64}) \ast \texttt{rax}\mapsto_{\textsf{r}} \texttt{pte\_addr}  }_{\rtv}$
% call alloc_phys_page_or_panic
% $\specline{\textsf{P} \ast  I\texttt{ASpace}(\theta,m) \ast  \texttt{r14}\mapsto_{\textsf{r}} \texttt{pte\_addr} \ast \texttt{rdi}\mapsto_{\textsf{r}} \vaddr \;\ast \nfpointsto{\mask\vaddr\maskfour\rtv}{\mask\vaddr\maskfouroff\rtv}\entryf\qone\naddr \ast}_{\rtv}$ 
% $\specline{  \nfpointsto{\mask\vaddr\maskthree\entryf}{\mask\vaddr\maskthreeoff\entryf}\entrytr\qtwo\naddr \ast \nfpointsto{\mask\vaddr\masktwo\entrytr}{\mask\vaddr\masktwooff\entrytr}\paddr\qthree\naddr \;\ast}_{\rtv}$
% $\specline{\texttt{pte\_addr} \mapsto_{\texttt{vpte}} \paddr\; (\texttt{wzero 64})  \ast \ulcorner \texttt{entry\_present } \entryf \land \texttt{entry\_present } \entrytr \land  \texttt{entry\_present } \entrytw \urcorner}_{\rtv}$
% $\specline{\exists \texttt{ fpaddr} \ldotp \ulcorner \texttt{aligned fpaddr} \urcorner \ast \texttt{rax}\mapsto_{\textsf{r}} \texttt{fpaddr+3} \ast \texttt{fpaddr} \mapsto_{\textsf{p}} (\texttt{wzero 64}) \ast \ulcorner \texttt{entry\_present (fpaddr+3)}\urcorner}_{\rtv}$
% ;; Calculate new L1 entry
% mov [r14], rax ;; store the page table entry, mapping the page
% $\specline{\textsf{P} \ast  I\texttt{ASpace}(\theta,m) \ast  \texttt{r14}\mapsto_{\textsf{r}} \texttt{pte\_addr} \ast \texttt{rdi}\mapsto_{\textsf{r}} \vaddr \;\ast \nfpointsto{\mask\vaddr\maskfour\rtv}{\mask\vaddr\maskfouroff\rtv}\entryf\qone\naddr \ast}_{\rtv}$ 
% $\specline{  \nfpointsto{\mask\vaddr\maskthree\entryf}{\mask\vaddr\maskthreeoff\entryf}\entrytr\qtwo\naddr \ast \nfpointsto{\mask\vaddr\masktwo\entrytr}{\mask\vaddr\masktwooff\entrytr}\paddr\qthree\entryo \;\ast}_{\rtv}$
% $\specline{\texttt{pte\_addr} \mapsto_{\texttt{vpte}} \paddr \;(\texttt{fpaddr+3}) \; \ast \ulcorner \texttt{entry\_present } \entryf \land \texttt{entry\_present } \entrytr \land  \texttt{entry\_present } \entrytw \urcorner }_{\rtv}$
% $\specline{\ulcorner \texttt{aligned fpaddr} \urcorner \ast \texttt{rax}\mapsto_{\textsf{r}} \texttt{fpaddr+3} \ast \texttt{fpaddr} \mapsto_{\textsf{p}} (\texttt{wzero 64}) \ast \ulcorner \texttt{entry\_present fpaddr+3}\urcorner}_{\rtv}$
% $\;\;\;\;\;\;\;\;\;\;\;\;\;\;\;\;\;\;\;\;\;\;\;\;\;\;\;\;\;\;\;\;\;\;\;\;\;\;\;\;\;\;\;\; \sqsubseteq $
% $\specline{\textsf{P} \ast  I\texttt{ASpace}(\theta,m) \ast  \texttt{r14}\mapsto_{\textsf{r}} \texttt{pte\_addr} \ast \texttt{rdi}\mapsto_{\textsf{r}} \vaddr \ast }_{\rtv}$
% $\specline{\textsf{L}_{4}\_\textsf{L}_{1}\_\textsf{PointsTo}(\vaddr,\entryf,\entrytr,\entrytw,\fpaddr+3) \ast \ulcorner \theta \;!!\;\vaddr = \texttt{None}\urcorner \; \ast}_{\rtv}$
% $\specline{\ulcorner \texttt{aligned fpaddr} \urcorner \ast \texttt{rax}\mapsto_{\textsf{r}} \texttt{fpaddr+3} \ast \texttt{fpaddr} \mapsto_{\textsf{p}} (\texttt{wzero 64}) }_{\rtv}$
% $\;\;\;\;\;\;\;\;\;\;\;\;\;\;\;\;\;\;\;\;\;\;\;\;\;\;\;\;\;\;\;\;\;\;\;\;\;\;\;\;\;\;\;\; \sqsubseteq $
% $\specline{\textsf{P} \ast  I\texttt{ASpace} (<[\vaddr:=\texttt{fpaddr}]> \theta,m) \ast}_{\rtv}$
% $\specline{\ulcorner \texttt{aligned fpaddr} \urcorner \ast \texttt{fpaddr} \mapsto_{\textsf{p}} \textsf{ wzero 64} \ast \ghostmaptoken{\delta{}s}{\rtv}{\delta}  \ast\sumwalkabs\vaddr\qfrac\fpaddr}_{\rtv}$
% $\;\;\;\;\;\;\;\;\;\;\;\;\;\;\;\;\;\;\;\;\;\;\;\;\;\;\;\;\;\;\;\;\;\;\;\;\;\;\;\;\;\;\;\; \sqsubseteq $
%   $\specline{\textsf{P} \ast  I\texttt{ASpace} (<[\vaddr:=\texttt{fpaddr}]> \theta,m) \ast \vaddr \mapsto_{\textsf{vpte}}\; \{\qfrac\} \;\fpaddr \textsf{ wzero 64}}_{\rtv}$
% $\;\;\;\;\;\;\;\;\;\;\;\;\;\;\;\;\;\;\;\;\;\;\;\;\;\;\;\;\;\;\;\;\;\;\;\;\;\;\;\;\;\;\;\; \sqsubseteq $
% $\specline{\textsf{P} \ast  I\texttt{ASpace} (<[\vaddr:=\texttt{fpaddr}]> \theta,m) \ast \vaddr \mapsto_{\textsf{v}}\; \{\qfrac\} \textsf{wzero 64}}_{\rtv}$
% \end{comment}

%\end{appendix}



\end{document}
\endinput