%2multibyte Version: 5.50.0.2952 CodePage: 65001
\documentclass
[preprint,prd,a4paper,amsfonts,amssymb,nobibnotes,nofootinbib,superscriptaddress,fleqn]{revtex4}%
\usepackage{amsfonts}
\usepackage{amsmath}
\usepackage{amssymb}
\usepackage{graphicx}
\usepackage{comment}
\usepackage{bm}
\usepackage{mathrsfs}
\usepackage{color}%
\setcounter{MaxMatrixCols}{30}
%TCIDATA{OutputFilter=latex2.dll}
%TCIDATA{Version=5.50.0.2952}
%TCIDATA{Codepage=65001}
%TCIDATA{CSTFile=revtex4.cst}
%TCIDATA{Created=Wednesday, July 19, 2023 17:20:43}
%TCIDATA{LastRevised=Wednesday, July 26, 2023 18:33:52}
%TCIDATA{<META NAME="GraphicsSave" CONTENT="32">}
%TCIDATA{<META NAME="SaveForMode" CONTENT="1">}
%TCIDATA{BibliographyScheme=Manual}
%TCIDATA{<META NAME="DocumentShell" CONTENT="Articles\SW\REVTeX 4">}
%BeginMSIPreambleData
\providecommand{\U}[1]{\protect\rule{.1in}{.1in}}
%EndMSIPreambleData
\begin{document}
\preprint{CHIBA-EP-258}
\preprint{KEK Preprint 2023-24}
\title{Gauge-independent transition dividing the confinement phase in the lattice
SU(2) gauge-adjoint scalar model.}
\author{Akihiro Shibata}
\affiliation{Computing Research Center, High Energy Accelerator Research Organization
(KEK), Tsukuba 305-0801,Japan}
\affiliation{Department of Accelerator Science, SOKENDAI (The Graduate Univercity for
Advanced Studies), Tsukuba 305-0801, Japan}
\author{Kei-Ichi Kondo}
\affiliation{Department of Physics, Graduate School of Science, Chiba University, Chiba
263-8522, Japan}
\keywords{gauge-scalar model, gauge-independent BEH mechanism, confinement }
\pacs{PACS number}

\begin{abstract}
The lattice SU(2) gauge-scalar model with the scalar field in the adjoint
representation of the gauge group has two completely separated confinement and
Higgs phases according to the preceding studies based on numerical simulations
which have been performed in the specific gauge fixing based on the
conventional understanding of the Brout-Englert-Higgs mechanism.

In this paper, we re-examine this phase structure in the gauge-independent way
based on the numerical simulations performed without any gauge fixing. This is
motivated to confirm the recently proposed gauge-independent
Brout-Englert-Higgs mechanism for generating the mass of the gauge field
without relying on any spontaneous symmetry breaking. For this purpose we
investigate correlation functions between gauge-invariant operators obtained
by combining the original adjoint scalar field and the new field called the
color-direction field which is constructed from the gauge field based on the
gauge-covariant decomposition of the gauge field due to Cho-Duan-Ge-Shabanov
and Faddeev-Niemi.

Consequently, we reproduce gauge-independently the transition line separating
confinement phase and Higgs phase, and show surprisingly the existence of a
new transition line that divides completely the confinement phase into two
parts. Finally, we discuss the physical meaning of the new transition and
implications to confinement mechanism.

\end{abstract}
\maketitle


\section{Introduction}

In this paper, we investigate the gauge-scalar model to clarify the mechanism
of confinement in the Yang-Mills theory in the presence of matter fields and
also non-perturbative characterization of the Brout-Englert-Higgs (BEH)
mechanism \cite{Higgs1} providing the gauge field with the mass, in the
gauge-independent way.
%In conventional studies on the lattice, gauge fixing %and color (global) symmetry fixing
%is necessary to obtain a signal due to Elizer's theorem.
%However, this does not satisfy the requirement of color confinement.
%To overcome this problem, we propose the lattice formulation of the gauge-independent Brout-Englert-Higgs (BEH) mechanism, which allows simulations and analysis without breaking the gauge symmetry.


For concreteness, we reexamine the lattice $SU(2)$ gauge-scalar model with a
radially-fixed scalar field (no Higgs mode) which transforms according to the
adjoint representation of the gauge group $SU(2)$ without any gauge fixing. In
fact, this model was investigated long ago in \cite{Brower82} by taking a
specific gauge, say unitary gauge, based on the traditional characterization
for the BEH mechanism to identify the Higgs phase. It is a good place to
recall the traditional characterization of the BEH mechanism: If the original
continuous gauge group is spontaneously broken, the resulting massless
Nambu-Goldstone particle is absorbed into the gauge field to provide the gauge
field with the mass. In the perturbative treatment, such a spontaneous
symmetry breaking is signaled by the non-vanishing vacuum expectation value of
the scalar field. However, this is impossible to realize on the lattice unless
the gauge fixing condition is imposed, since gauge non-invariant operators
have vanishing vacuum expectation value on the lattice without gauge fixing
due to the Elitzur theorem \cite{Elitzur75}. This traditional characterization
of the BEH mechanism prevents us from investigating the Higgs phase in the
gauge-invariant way.

This difficuty can be avoided by using the \textit{gauge-independent
description of the BEH mechanism} proposed recently by one of the authors
\cite{Kondo16,Kondo18}, which \textit{needs neither the spontaneous breaking
of gauge symmetry},
%$G \rightarrow H$,
nor the \textit{non-vanishing vacuum expectation value of the scalar field}.
%$\langle0|\phi(x)|0\rangle:=v\neq 0$.
Then we can give a \textit{gauge-invariant definition of the mass for the
gauge field} resulting from the BEH mechanism. Therefore, we can study the
Higgs phase in the gauge-invariant way on the lattice without gauge fixing
based on the lattice construction of gauge-independent description of the BEH
mechanism. Consequently, we can perform numerical simulations without any
gauge-fixing and compare our results with those of the preceding result
\cite{Brower82} obtained in a specific gauge. Indeed, our gauge-independent
study reproduces the transition line separating Higgs and confinement phases
obtained by \cite{Brower82} in a specific gauge.

%To explain it, we need to introduce a specific gauge-scalar model (\textit{complementary gauge-scalar model}) which reduces to the \textit{Yang-Mills theory with a gauge-invariant gluon mass term} (\textit{massive Yang-Mills theory}). The gauge-invariant gluon mass term simulates the dynamically generated mass to be extracted in the low-energy effective theory of the Yang-Mills theory and plays the role of a new probe to study confinement mechanism through the phase structure (Confinement, Higgs, deconfinement) in the gauge-invariant way. In this talk we give preliminary studies in this direction.%


Moreover, we investigate the phase structure of this model based on the
gauge-independent (invariant) procedure to look into the mechanism for
confinement. For this purpose we introduce the \textit{gauge-covariant decomposition
of the gauge field} originally due to Cho-Duan-Ge-Shabanov and Faddeev-Niemi
\cite{Cho80,Duan-Ge79,Shabanov99,FN98}, which we call CDGSFN decomposition for
short. It has been confirmed that this method is quite efficient to extract
the dominant mode responsible for quark confinement in a gauge-independent way
\cite{Exactdecomp09,CFNdccomp07,KKSS15}, even if we expect the dual superconductor picture for quark confinement \cite{dualsuper}.

To discriminate and characterize the phases among confinement phase, Higgs
phase, and the other possible phases, we investigate the correlation functions
between the gauge-invariant composite operators constructed from the scalar
field and the the color-direction field obtained through the CDGSFN
decomposition. As a result of the gauge-independent analysis, we find
suprisingly a new transition line that divides the conventional confinement
phase into two parts.
%Note that this finding owes much to gauge-independent simulations without any gauge fixing, and analysis and can only be established through this.
Finally, we discuss the physical meaning of this transition and the
implications to confinement.%



This paper is organized as follows.
%In Sec.2 we give a brief review of the gauge-independent BEH mechamism for a SU(2) gauge-adjoint scalar model.
In Sec.II we define the lattice $SU(2)$ gauge-scalar model with a
radially-fixed scalar field in the adjoint representation of the gauge group
and introduce the gauge-covariant CDGSFN decomposition of the gauge field
variable on the lattice. We explain the method of numerical simulations in the
new framework of the lattice gauge theory. In Sec.III we present the results
of the numerical simulations. We give an analysis in view of the the
gauge-covariant CDGSFN decomposition. By measuring the correlation function
between the gauge-invariant composite operators composed of the original
adjoint scalar field and the color-direction field obtained from the
decomposition, we find a new phase that divides the confinement phase
completely into two parts. The final section is devoted to conclusion and discussion.%


%\newpage

\section{Lattice $SU(2)$ gauge-scalar model with a scalar field in the adjoint
representation}

%We now discuss the numerical simulations for the proposed gauge-scalar model on the lattice, where \textit{the gauge-independent description for the BEH mechanism}.%


\subsection{Lattice action} % and integration measure}

The $SU(2)$ gauge-scalar model with a radially-fixed scalar field in the
adjoint representation is given on the lattice with a lattice spacing
$\epsilon$ by the following action with two parameters $\beta$ and $\gamma$:
\begin{align}
S_{\text{GS}}  &  :=S_{g}[U]+S_{\phi}[U,\mathbf{\phi}]\text{ }%
,\label{eq:GS,odel}\\
S_{g}[U]  &  :=\sum_{x}\sum_{\mu<\nu}\frac{\beta}{2}\mathrm{tr}\left(
\mathbf{1}-U_{x,\mu}U_{x+\mu,\nu}U_{x+\nu,\mu}^{\dag}U_{x,\nu}^{\dag}\right)
+c.c.\text{ ,}\label{eq:GaugeAction}\\
S_{\phi}[U,\phi]  &  :=\sum_{x,\mu}\frac{\gamma}{2}\mathrm{tr}\left(  (D_{\mu
}^{\epsilon}[U]\mathbf{\phi}_{x})^{\dag}(D_{\mu}^{\epsilon}[U]\mathbf{\phi
}_{x})\right)  \text{ , } \label{eq:ScalarAction}%
\end{align}
where $U_{x,\mu}=\exp(-ig\epsilon\mathscr{A}_{x,\mu}) \in SU(2)$ represents a
gauge variable on a link $<x,\mu>$, $\mathbf{\phi}_{x}\mathcal{\mathbf{=}}%
\phi_{x}^{A}\sigma^{A}$ $\in su(2)$ ($A=1,2,3$) represents a scalar field on a
site $x$ in the adjoint representation subject to the radially-fixed
condition: $\mathbf{\phi}_{x}\cdot\mathbf{\phi}_{x}=\phi_{x}^{A}\phi_{x}%
^{A}=1$, and $D_{\mu}^{\epsilon}[U]\mathbf{\phi}_{x}$ represents the covariant
derivative in the adjoint representation defined as%
\begin{equation}
D_{\mu}^{\epsilon}[U]\mathbf{\phi}_{x}=U_{x,\mu}\mathbf{\phi}_{x+\epsilon
\hat{\mu}}-\mathbf{\phi}_{x}U_{x,\mu}\text{ .} \label{eq:CovaritDriv}%
\end{equation}
This action reproduces in the naive continum limit $\epsilon\to0$ the continum
gauge-scalar theory with a radiallly-fixed scalar field $|\phi(x)|=v$ and a
gauge coupling constant $g$ where $\beta=4/g^{2}$ and $\gamma=v^{2}/2$.
%\marginpar{CHECK :: $\gamma=v^{2}/2$. }


\subsection{Gauge-covariant decomposition}

To investigate gauge-independently the phase structure of the gauge-scalar
model, we introduce the lattice version \cite{Exactdecomp09,CFNdccomp07} of
change of variables based on the idea of the \textit{gauge-covariant
decompostion of the gauge field}, so called the CDGSFN decomposition
\cite{Cho80,Duan-Ge79,Shabanov99,FN98}. For a review, see \cite{KKSS15}.

We introduce the site variable $\mathbf{n}_{x}:=n_{x}^{A}\sigma_{A}\in
SU(2)/U(1)$ which is called the color-direction (vector) field, in addition to
the original link variable $U_{x,\mu}\in SU(2)$. The link variable $U_{x,\mu}$
and the site variable $\mathbf{n}_{x}$ transforms under the gauge
transformation $\Omega_{x}\in SU(2)$ as
\begin{equation}
U_{x,\mu}\rightarrow\Omega_{x}U_{x,\mu}\Omega_{x+\mu}^{\dagger}=U_{x,\mu
}^{\prime},\quad\mathbf{n}_{x}\rightarrow\Omega_{x}\mathbf{n}_{x}\Omega
_{x}^{\dagger}=\mathbf{n}_{x}^{\prime}.
\end{equation}
In the decomposition, a link variable $U_{x,\mu}$ is decomposed into two
parts:
\begin{equation}
U_{x,\mu}:=X_{x,\mu}V_{x,\mu}.
\end{equation}
We identify the lattice variable $V_{x,\mu}$ with a link variable which
transforms in the same way as the original link variable $U_{x,\mu}$:
\begin{equation}
V_{x,\mu}\rightarrow\Omega_{x}V_{x,\mu}\Omega_{x+\mu}^{\dagger}=V_{x,\mu
}^{\prime}.
\end{equation}
On the other hand, we define the lattice variable $X_{x,\mu}$ such that it
transforms in just the same way as the site variable $\mathbf{n}_{x}$:
\begin{equation}
X_{x,\mu}\rightarrow\Omega_{x}X_{x,\mu}\Omega_{x}^{\dagger}=X_{x,\mu}^{\prime
},
\end{equation}
which automatically follows from the above definition of the decomposition.
Such decomposition is obtained by solving the defining equations:
\begin{align}
&  D_{\mu}[V]\mathbf{n}_{x}:=V_{x,\mu}\mathbf{n}_{x+\mu}-\mathbf{n}%
_{x}V_{x,\mu}=0,\\
&  \mathrm{tr}(\mathbf{n}_{x}X_{x,\mu})=0.
\end{align}
This defining equation has been solved exactly and the resulting link variable
$V_{x,\mu}$ and site varible $X_{x,\mu}$ are of the form
%(up to thenormalization)
\cite{Exactdecomp09}:
\begin{align}
V_{x,\mu}  &  :=\tilde{V}_{x,\mu}/\sqrt{\mathrm{tr}[\tilde{V}_{x,\mu}%
^{\dagger}\tilde{V}_{x,\mu}]/2}, \ \tilde{V}_{x,\mu} :=U_{x,\mu}%
+\mathbf{n}_{x}U_{x,\mu}\mathbf{n}_{x+\mu},\\
X_{x,\mu}  &  :=U_{x,\mu}V_{x,\mu}^{-1}\text{ .}%
\end{align}
This decomposion is obtained uniquly for given set of link variable $U_{x,\mu
}$ once the site variable $\mathbf{n}_{x}$ is given. The configurations of the
color-direction field $\{\mathbf{n}_{x}\}$ are obtained by minimizing the
functional:
\begin{equation}
F_{\text{red}}[\{\mathbf{n}_{x}\}|\{U_{x,\mu}\}]:=\sum_{x,\mu} \text{tr}%
\left\{  \left(  D_{x,\mu}[U]\mathbf{n}_{x}\right)  ^{\dag}\left(  D_{x,\mu
}[U]\mathbf{n}_{x}\right)  \right\}  , \label{reduction}%
\end{equation}
which we call the \textit{reduction condition}. Note that this functional has
the same form as the action of the scalar field:%
\begin{align}
S_{\phi}= \frac{\gamma}{2}F_{\text{red}}[\{\mathbf{\phi}_{x}\}|\{U_{x,\mu}\}].
\end{align}


\subsection{Numerical simulations}

The numerical simulation can be performed by updating link varables and scalar fields alternatly. For link variable $U_{x,\mu}$ we can apply the standard HMC algorithm. Whlie for scalar field we reparametrized the variable $\mathbf{\phi}_{x}$ $\in su(2)$ according to the adjoint-orbit representation:%
\begin{equation}
\mathbf{\phi}_{x}:=Y_{x}\sigma^{3}Y_{x}^{\dag}, \ Y_{x} \in SU(2) ,
\label{eq:fhi2}%
\end{equation}
which satisfies the normalization condition $\mathbf{\phi}_{x}\cdot \mathbf{\phi}_{x}=1$ automatically. Therefore, the Haar measure is replaced by $D[\mathbf{\phi}]$ $\prod_{x}\delta(\mathbf{\phi}_{x}\cdot\mathbf{\phi}_{x}-1)$ to $D[Y]$, and we can apply the standard HMC algorithm for $Y_{x}$,
to update configurations of the scalar fields $\phi_{x}$.

We perform Monte Carlo simulations on the $16^{4}$ lattice with periodic boudary condition in the gauge-independent way (without gauge fixing). 
In each Monte Carlo step (sweep), we update link variables $\{U_{x,\mu}\}$ and scalar fields $\{\mathbf{\phi}_{x}\}$ alternatly by using the HMC algorithm with integral interval $\Delta\tau=1$ as explained in the previous section.
We take thermalization for $5000$ sweeps and store 800 configurations for measurements every 25 sweeps. 
Fig.\ref{fig:simulation} shows data sets of the simulation parameters in the $\beta$--$\gamma$ plane.

\section{Lattice result and gauge-independent analyses }

\subsection{Action densities for the plaquette and scalar parts}%

%TCIMACRO{\TeXButton{Bfig}{% Figure environment removed}}%
%BeginExpansion
\end{figure}%
%EndExpansion

The search for the phase boundary is performed by measuring the expectation value $\left\langle \mathcal{O} \right\rangle$ of a chosen operator $\mathcal{O}$ by changing $\gamma$ (or $\beta$) along the $\beta=$const.  (or $\gamma=$const.) lines. 
In order to identify the boundary, we used the bent, step, and gap observed in the graph of the plots for $\left\langle \mathcal{O} \right\rangle$. 
%The existence and the magnitude of the gap is to be related to the existence and the strength of the first-order phase transition.


First of all, in order to determine the phase boundary of the model, we
measure the Wilson action per plaquette (plaquette-action density),
\begin{equation}
P=\frac{1}{6N_{\text{site}}}\sum_{x}\sum_{\mu<\nu}\frac{1}{2}\mathrm{tr}%
(U_{x,\mu\nu}), 
\text{\qquad}U_{x,\mu\nu}=U_{x,\mu}U_{x+\hat{\mu},\nu}%
U_{x+\hat{\nu},\mu}^{\dag}U_{x,\nu}^{\dag}\text{, } \label{eq:P}%
\end{equation}
and that of the scalar action per link (scalar-action density),
\begin{equation}
M=\frac{1}{4N_{\text{site}}}\sum_{x}\sum_{\mu}\frac{1}{2}\mathrm{tr}\left(
\left(  D_{\mu}[U_{x,\mu}]\mathbf{\phi}_{x}\right)  ^{\dag}\left(  D_{\mu
}[U_{x,\mu}]\mathbf{\phi}_{x}\right)  \right)  \text{, } \label{eq:S}%
\end{equation}
as Brower et al. have done in \cite{Brower82}. \ %

%TCIMACRO{\TeXButton{Bfig}{% Figure environment removed}}%
%BeginExpansion
\end{figure}%
%EndExpansion
%

%TCIMACRO{\TeXButton{Bfig}{% Figure environment removed}}%
%BeginExpansion
\end{figure}%
%EndExpansion


First, we try to determine the phase boundary from the plaquette-action density. 
Fig.\ref{fig:mesure<P>} shows the results of measurements of the plaquette-action density
$\left\langle P\right\rangle $ in the $\beta$--$\gamma$ plane. 
The left panel shows the plots of $\left\langle P\right\rangle $\ along $\beta=$const. lines as functions of $\gamma$, where error bars are not shown because they are smaller than the size of the plot points. 
On the other hand, the right panel shows the plots of $\left\langle P\right\rangle $\ along $\gamma=$const. lines as functions of $\beta$. 


Next, in the same way, we try to determine the phase boundary from the scalar-action density. 
Fig.\ref{fig:mesurement<M>} shows the results of measurement of the scalar-action density $\left\langle M\right\rangle $ in the $\beta$--$\gamma$ plane. 
The left pannel of Fig.\ref{fig:mesurement<M>} shows the plots of $\left\langle M\right\rangle $ along $\beta$-const. lines as functions of $\gamma$,
while the right panel of Fig.\ref{fig:mesurement<M>} shows the plots of $\left\langle M\right\rangle $ along $\gamma$-const. lines as functions of $\beta$.


In Fig.\ref{fig:Critical_action}, the phase boundary determined from the plaquette-action density $\left\langle P\right\rangle $ is given in the left panel of Fig.\ref{fig:Critical_action}. 
The phase boundary determined from the scalar-action density $\left\langle M\right\rangle $ is given in the right panel of Fig.\ref{fig:Critical_action}. 
%The right panel of Figure \ref{fig:Critical_action} shows the phase boundary detemined from scalar-action density. 
The interval between the two simulation points corresponds to the short line with ends. 
The error bars in the phase boudary are due to the spacing of the simulation points.
It should be noticed that the two phase boundaries determined from $\left\langle P\right\rangle$ and $\left\langle M\right\rangle$ are consistent within accuracy of numerical calculations. 
Thus we find that the gauge-independent numerical simualtions reproduce the critical line obtained by Brower et al. \cite{Brower82}.%


%TCIMACRO{\TeXButton{Bfig}{% Figure environment removed}}%
%BeginExpansion
\end{figure}%
%EndExpansion


\subsection{Susceptibilities for $P$ and $M$}%


To find out more about phase boundary, we next measure \textquotedblleft susceptibility" for the action densities:%
\begin{align}
\left\langle \chi(P)\right\rangle  &  :=\left\langle P^{2}\right\rangle
-\left\langle P\right\rangle ^{2}\text{ ,}\label{eq:Chi(P)}\\
\left\langle \chi(M)\right\rangle  &  :=\left\langle M^{2}\right\rangle
-\left\langle M\right\rangle ^{2}\text{ .}\label{eq:Chi(S)}%
\end{align}
%These are kinds of the heat capacity rato(?ratio) for action densities.%

%TCIMACRO{\TeXButton{Bfig}{% Figure environment removed}}%
%BeginExpansion
\end{figure}%
%EndExpansion
%
%TCIMACRO{\TeXButton{Bfig}{% Figure environment removed}}%
%BeginExpansion
\end{figure}%
%EndExpansion

Figure \ref{fig:susceptiblity<P>} shows the measurements of $\left\langle
\chi(P)\right\rangle$. 
The upper panels show plots of $\left\langle \chi(P)\right\rangle $ versus $\gamma$ along $\beta=$ const. lines, while the lower panels show plots of $\left\langle \chi(P)\right\rangle $  versus $\beta$ along the $\gamma=$ const. lines. 

%We find that some plots of  $\left\langle \chi(P)\right\rangle $ indicate peaks or blowup shapes, and plot these location in $\beta$--$\gamma$ plane as the left panel of Figure \ref{fig:criticalChi(P)VChi(M)>}. 

Figure \ref{fig:susceptibility<M>} shows the measurements of $\left\langle \chi(M)\right\rangle$. 
The upper panels show plots of $\left\langle \chi(M)\right\rangle $ versus $\gamma$ along $\beta=$ const. lines, 
while the lower panels show plots of $\left\langle \chi(M)\right\rangle $  versus $\beta$ along the $\gamma=$ const. lines. 


%TCIMACRO{\TeXButton{Bfig}{% Figure environment removed}}%
%BeginExpansion
\end{figure}%
%EndExpansion

Fig.\ref{fig:criticalChi(P)VChi(M)>}  is the phase boundary determined by the susceptibility (specific heat) as a function of $\beta$ or $\gamma$. The green boundary is determined from the position of the peak in the susceptibility graph. The black boundary was
determined from the position of the bend in the susceptibility graph.
The orange boundary in the left panel of Fig. \ref{fig:criticalChi(P)VChi(M)>}
is determined from the peak position of the plaquette-action susceptibility.

The left panel of Fig.\ref{fig:criticalChi(P)VChi(M)>} gives same boundary as that determined by $\left\langle \chi(P)\right\rangle $ in Fig.\ref{fig:Critical_action} for relatively large $\gamma$.
 However, the phase bounday in Fig.\ref{fig:criticalChi(P)VChi(M)>} and Fig. \ref{fig:Critical_action} do not neccesarily coinside:
in Fig.\ref{fig:Critical_action}  the boundary line in the region $\beta>2$ extends along the holizontal line $\gamma\simeq1$ towards the pure scalar axis at $\beta=\infty$, while in the left panel of Fig.\ref{fig:criticalChi(P)VChi(M)>} the boundary line extends also to the point $(\beta\simeq2.2,\gamma=0)$ on the pure gauge axis at $\gamma=0$.
This orange part of the phase boundary could be identified with the $SU(2)$ cross over in pure $SU(2)$ gauge theory which discriminates the weak coupling asymptotic scaling region from the strong coupling region, as seen by Bhanot and Creutz in their model \cite{BhanotCreutz81}.


The right panel of Fig.\ref{fig:criticalChi(P)VChi(M)>} shows the phase boundary determined by  measurements of $\left\langle \chi (M)\right\rangle $ given in Fig.\ref{fig:susceptibility<M>}. 
The phase boundary determined from $\left\langle P\right\rangle$ and $\left\langle \chi(M)\right\rangle$ coinside. 
Note that  graphs along  $\gamma\approx0.75$ and  $\beta\approx$ $1.2$
corresponds to the crross sections along the ridge section of the phase boundary.%

%This indicates the nature of the first-order phase transition.%




%\subsection{Analysis in view of the gauge covariant decomposition}

\subsection{Correlations between the scalar field and the color-direction field through the gauge covariant decomposition}%


We measure the scalar-color correlation detected by the scalar-color composite operator:%
\begin{equation}
Q=\frac{1}{N_{\text{site}}}\sum_{x}\frac{1}{2}\mathrm{tr}(\mathbf{n}%
_{x}\mathbf{\phi}_{x}),
\end{equation}
%\ref{fig:critical<Q>}
where $\mathbf{n}_{x}$ is the color-direction field in the gauge-covariant decomposition for the gauge link variable. 
For this purpose, we need  to solve the reduction condition (\ref{reduction}) to obtain the color-direction field $\mathbf{n}_{x}$, which however has two kinds of ambiguity. 
One comes form so-called the Gribov copies that are  the local minimal solutions of the reduction condition.  
In order to avoid the local minimal solutions and to obtain the absolute minima, the reduction condition is solved by changing the initial values to search for the absolute minima of the functional. 
Another comes from the choice of a global sign factor, which originates from the fact that whenever a configuration $\{\mathbf{n}_{x}\}$ is a solution, the flipped one $\{-\mathbf{n}_{x}\}$ is also a solution, since the reduction functional is quadratic in the color fields. 
To avoid these issues, we propose to use $\left\langle \left\vert Q\right\vert \right\rangle$ and $\left\langle Q^{2}\right\rangle$, which are examined as the order parameters that determine the phase boundary.%

The phase boundary is searched for based on two ways:

\begin{description}
\item[(i)] 
the location at which $\left\langle \left\vert Q\right\vert
\right\rangle $ changes from $\left\langle \left\vert Q\right\vert
\right\rangle \simeq0$ to $\left\langle \left\vert Q\right\vert \right\rangle
>0$.
This is also the case for $\left\langle Q^{2}\right\rangle$.

\item[(ii)] 
the location at which $\left\langle \left\vert
Q\right\vert \right\rangle $ changes abruptly, as was done for
$\left\langle P\right\rangle $ and $\left\langle M\right\rangle$.
This is also the case for $\left\langle Q^{2}\right\rangle$.

\end{description}

Fig.\ref{fig:measurement<Q>} shows the measurements of $\left\langle \left\vert Q\right\vert \right\rangle$. The left panel shows plots of $\left\langle \left\vert Q\right\vert \right\rangle $ versus $\gamma$ along various $\beta=$const. lines, while the right panel shows plots of $\left\langle \left\vert Q\right\vert \right\rangle $ versus $\beta$ along various $\gamma=$const. lines. 


Fig.\ref{fig:measurement<Q2>}, on the other hand, shows the measurements of
$\left\langle Q^{2}\right\rangle $ in the same manner as $\left\langle \left\vert Q\right\vert \right\rangle $. 


%TCIMACRO{\TeXButton{Bfig}{% Figure environment removed}}%
%BeginExpansion
\end{figure}%
%EndExpansion
%

%TCIMACRO{\TeXButton{Bfig}{% Figure environment removed}}%
%BeginExpansion
\end{figure}%
%EndExpansion




%TCIMACRO{\TeXButton{Bfig}{% Figure environment removed}}%
%BeginExpansion
\end{figure}%
%EndExpansion

Fig.\ref{fig:critical<Q>} shows the phase boundary (critical line) determined by
$\left\langle \left\vert Q\right\vert \right\rangle $ and $\left\langle Q^{2}\right\rangle$.
The left panel of Fig.\ref{fig:critical<Q>} shows the phase boundary determined from $\left\langle \left\vert Q\right\vert \right\rangle $. 




The purple boundary indicates that (i) $\left\langle \left\vert Q\right\vert \right\rangle $ changes from $\left\langle \left\vert Q\right\vert \right\rangle \simeq0$ to $\left\langle \left\vert Q\right\vert \right\rangle >0$ 
(or $\left\langle Q^{2}\right\rangle $ changes from $\left\langle Q^{2}\right\rangle \simeq0$ to $\left\langle Q^{2}\right\rangle >0$). 
The black boundary corresponds to the location  at which $\left\langle \left\vert Q\right\vert \right\rangle $ (or $\left\langle Q^{2}\right\rangle $) has gaps. 
The orange boundary corresponds to the location at which $\left\langle \left\vert Q\right\vert \right\rangle $ (or $\left\langle Q^{2}\right\rangle $) bends.
The right panel of Fig.\ref{fig:critical<Q>} shows the phase boundary determind from $\left\langle Q^{2}\right\rangle $.
The results in Fig.\ref{fig:critical<Q>} are consistent with each other. 


Fig.\ref{fig:critical<Q>} shows not only the phase boundary that divides the phase diagram into two phases, so-called the Higgs phase and the confiment phase, but also the new boundary that divides the confinement phase into two different parts. 
It should be remarked that this finding owes much to gauge-independent numerical simulations and their analyses, and this new results can only be established through our framework.






\subsection{Susceptibility for the scalar and color-direction field  }%

We finally investigate the ``susceptibility" of the scalar-color local
correlation:%
\begin{equation}
\left\langle \chi(\left\vert Q\right\vert )\right\rangle =\left\langle
Q^{2}\right\rangle -\left\langle \left\vert Q\right\vert \right\rangle ^{2}%
\end{equation}
%

%TCIMACRO{\TeXButton{Bfig}{% Figure environment removed}}%
%BeginExpansion
\end{figure}%
%EndExpansion


Figure \ref{fig:susceptibility<Q>} shows the measurement of $\left\langle \chi(\left\vert Q\right\vert )\right\rangle$. 
The upper panel of Fig.\ref{fig:susceptibility<Q>} show plots of $\left\langle \chi(\left\vert Q\right\vert )\right\rangle $ versus $\gamma$ along the $\beta=$ const. lines  and the lower panels show plots of $\left\langle \chi(\left\vert Q\right\vert )\right\rangle $ versus $\beta$  along the $\gamma=$const. lines. 


%TCIMACRO{\TeXButton{Bfig}{% Figure environment removed}}%
%BeginExpansion
\end{figure}%
%EndExpansion
%


First, we search for the transition along the vertical lines with fixed values of $\beta$ in a phase diagram. 
For relatively small fixed value of $\beta$ ($0\le \beta \le 1.6$), $\left\langle \chi(\left\vert Q\right\vert )\right\rangle$ is nearly equal to zero for small $\gamma$, but reaches a large but finite value across a critical value $\gamma_c(\beta)$, showing a peak as $\gamma$ increases. 
For larger values of $\beta$ ($1.8\le \beta \le 4.0$), $\left\langle \chi(\left\vert Q\right\vert )\right\rangle$ increases from zero to a finite value, which shows however no peak and increases monotonically as $\gamma$ increase. 
These observations yield the existence of a new transition line $\gamma=\gamma_c(\beta)$.
%%, which seems to be the first order transition line. 

Next, we search for the transition along the horizontal line  with fixed values of $\gamma$ in a phase diagram. 
For small fixed value of $\gamma$ ($0\le \gamma \le 1.5$), $\left\langle \chi(\left\vert Q\right\vert )\right\rangle$ shows nonzero value for small $\beta$, and decreases monotonically as $\beta$ increases. 

Figure \ref{fig:critical<Chi(|Q|)>} shows the phase boundary (critical line) determined from $\left\langle \chi(\left\vert Q\right\vert )\right\rangle$.
 The blue boundary is obtained from the the location of the rapid change. 
 The red and black intervals are obtained from the location of bends.
This result agrees with the critical line already obtained in the above. 


\subsection{Understanding the new phase structure obtained from numerical simulations}


Finally, we discuss why the above phase structure should be obtained and how the respective phase is characterized from the physical point of view. 

(i) 
Below the new critical line $\gamma<\gamma_c(\beta)$, there could be a confinement phase (I) where the effect of the scalar field would be relatively small and confinement would occur  in the way similar to the pure $SU(2)$ gauge theory which is expected to be in a single phase with no phase transition on the $\beta$ axis and has a mass gap. 
The confinement phase (I)  is regarded as a disordered phase in the sense that the color direction field $\bm{n}_{x}$ takes various directions with no specific direction in color space.
This can be estimated in relation to the direction of the adjoint scalar field $\bm{\phi}_{x}$ through $Q$, which yields the  very small or vanishing value of the average $\langle Q \rangle=0$. 
Confinement is expected to occur due to vacuum condensations of non-Abelian magnetic monopoles \cite{dualsuper}.  Here the non-Abelian magnetic monopole should be carefully defined in the gauge-independently in the gauge-invariant method, which is actually realized by extending the gauge-covariant decomposition of the gauge field, see \cite{KKSS15} for a review.  


(ii)
Above the new critical line $\gamma>\gamma_c(\beta)$, on the other hand, $\langle Q \rangle$ takes the non-vanishing value $\langle Q \rangle>0$.
%the correlator between the original adjoint scalar field $\bm{\phi}_{x}$ and the color-direction field $\bm{n}_{x}$ take the non-vanishing value $\langle Q \rangle>0$.  
This means that the color-direction field $\bm{n}_{x}$ correlates strongly with the given scalar field $\bm{\phi}_{x}$ which tends to align to a specific direction in this region $\gamma>\gamma_c(\beta)$ as expected from the spontaneous symmetry breaking 
leading to an ordered phase.

In this region the gauge fields become massive due to different physical origins.  
In the right region $\beta>\beta_c(\gamma)$ to be called the Higgs phase (II), the off-diagonal gauge fields for the modes $SU(2)/U(1)$ become massive due to the BEH mechanism, which is a consequence of the (partial) spontaneous symmetry breaking $SU(2) \to U(1)$ according to the conventional understanding of the BEH mechanism, although this phenomenon is understood gauge-independently based on the new understanding of the  BEH mechanism without the spontaneous symmetry breaking \cite{Kondo16}. 
The diagonal gauge field  for the mode $U(1)$ always remains massless everywhere in the phase (II).  
This is not the case in the other phases. 
Therefore, the Higgs phase (II) can be clearly distinguished from the other phases. 

(iii)
In the left region (III):$\beta<\beta_c(\gamma)$  above the new critical line $\gamma>\gamma_c(\beta)$, indeed, the gauge fields become massive due to self-interactions among the gauge fields, as in the phase (I).  
The difference between (II) and (III) can be understood as follows. 
First, consider the limit $\gamma \to \infty$ in (II) and (III).  Then the off-diagonal gauge fields become infinitely heavy and decouple from the theory, while the diagonal gauge field survives the limit both in (II) and (III). 
Consequently, the $SU(2)$ gauge-scalar model reduces to the pure compact $U(1)$ gauge model. 
The pure compact $U(1)$ gauge model in four spacetime dimensions has two phases: confinement phase with massive $U(1)$ gauge field in the strong gauge coupling region $\beta<\beta_*$ and the Coulomb phase with massless $U(1)$ gauge field in the weak gauge coupling region $\beta>\beta_*$, which has been proved rigorously \cite{Guth80,FS82}. 
For large but finite $\gamma$, furthermore, the critical line $\beta=\beta_c(\gamma)$ extends into the interior of the phase diagram from the critical point $(\beta_*=\beta_c(\infty),\gamma=\infty)$ as shown in \cite{Brower82}. 
This result is consistent with the above consideration for the right region (II):$\beta>\beta_c(\gamma)$ Higgs phase (II) for the existence of the massless $U(1)$ gauge field. 
In the right region (III):$\beta<\beta_c(\gamma)$ to be identified with a confinement phase (III), no massless gauge field exists and the gauge fields for all the modes become massive, which is consistent with the belief that the original gauge symmetry $SU(2)$ is kept intact and not spontaneously broken. 
This phenomenon can be explained as a cosequence of vacuum condensations of magnetic monopoles \cite{BMK77}, which is expected to occur also in the phase (I). 



\section{Conclusion and discussion}

In this paper, we have investigated the lattice $SU(2)$ gauge-scalar model
with the scalar field in the adjoint representation of the gauge group in a
gauge-independent way.
%We first explain the lattice construction of the gauge-independent description of the BEH mechanism, which does not rely on the spontaneous breaking of gauge symmetry.
This model was considered to have two completely separated confinement and Higgs phases according to the preceding studies \cite{Brower82} based on numerical simulations which
have been performed in the specific gauge fixing based on the conventional
understanding of the Brout-Englert-Higgs mechanism \cite{Higgs1}.

We have re-examined this phase structure in the gauge-independent way based on
the numerical simulations performed without any gauge fixing, which should be compared with  the preceding studies \cite{Brower82}. This is motivated to confirm the recently
proposed gauge-independent Brout-Englert-Higgs mechanics for giving the mass of the
gauge field without relying on any spontaneous symmetry breaking \cite{Kondo16,Kondo18}. For this purpose we have investigated correlation functions between gauge-invariant operators
obtained by combining the original adjoint scalar field and the new field
called the color-direction field which is constructed from the gauge field
based on the gauge-covariant decomposition of the gauge field due to
Cho-Duan-Ge-Shabanov and Faddeev-Niemi.

Consequently, we have reproduced gauge-independently the transition line separating
confinement and Higgs phase obtained in \cite{Brower82}, and show surprisingly
the existence of a new transition line that divides completely the confinement
phase into two parts. 
We have discussed the physical meaning of the new transition and implications to confinement mechanism. 
More discussions on the physical properties of the respective phase will be given in subsequent papers. 

The result obtained in this paper should be compared with the lattice $SU(2)$
gauge-scalar model with the scalar field in the fundamental representation of
the gauge group in a gauge-independent way. This model has a single
confinement-Higgs phase where two confinement and Higgs regions are
analytically continued according to the preceding studies \cite{Ostewlder78,FradkinShenker79}. Even in this case, it is shown \cite{Ikeda23} that 
the composite operator constructed from the original fundamental scalar field
and the color-direction field can discriminate two regions and indicate the
existence of the transition line seperating the confinement-Higg phase into
completely different two phases, Confinement phase and Higgs phase.


\subsubsection*{Acknowledgement}

This work was supported by Grant-in-Aid for Scientific Research, JSPS KAKENHI
Grant Number (C) No.23K03406. The numerical simulation is supported by the
Particle, Nuclear and Astro Physics Simulation Program No.2022-005 (FY2022) of
Institute of Particle and Nuclear Studies, High Energy Accelerator Research
Organization (KEK).

\begin{thebibliography}{99}                                                                                               %


\bibitem {Higgs1}P.W. Higgs,
%Broken symmetries, massless particles and gauge fields,
Phys. Lett.\textbf{12}, 132
%--133
(1964).
%
%\\
%P.W. Higgs,
%Broken Symmetries and the Masses of Gauge Bosons,
Phys. Rev. Lett. \textbf{13}, 508 (1964).
%--509
%\\
%P.W. Higgs,
%Spontaneous Symmetry Breakdown without Massless Bosons
%Phys. Rev. \textit{145}, 1156--1163 (1966).


%\bibitem{Higgs2}
F. Englert and R. Brout,
%Broken Symmetry and the Mass of Gauge Vector Mesons,
Phys. Rev. Lett.\textbf{13}, 321
%--323
(1964).

\bibitem {Brower82}R.C. Brower, D.A. Kessler, T. Schalk, H. Levine, M.
Nauenberg, Phys. Rev. D\textbf{25}, 3319 (1982).

\bibitem {Elitzur75}S. Elitzur,
%Impossibility of spontaneously breaking local symmetries,
Phys. Rev. D\textbf{12}, 3978 (1975).

\bibitem {Kondo16}K.-I. Kondo,
%Gauge-invariant description of Higgs phenomenon and quark confinement,
Phys. Lett. B \textbf{762}, 219
%--224
(2016).
%CHIBA-EP-219
%DOI: 10.1016/j.physletb.2016.09.026
arXiv:1606.06194 [hep-th]

\bibitem {Kondo18}K.-I. Kondo,
%Gauge-independent Brout-Englert-Higgs mechanism and Yang-Mills theory with a gauge-invariant gluon mass term,
%CHIBA-EP-230,
Eur. Phys. J. C \textbf{78}, 577 (2018). arXiv:1804.03279 [hep-th]

\bibitem {Cho80}Y.M. Cho, Phys. Rev. D\textbf{21}, 1080 (1980). Phys. Rev.
D\textbf{23}, 2415
%--
(1981).

\bibitem {Duan-Ge79}Y.S. Duan and M.L. Ge, Sinica Sci. \textbf{11}, 1072
%--1081
(1979).

\bibitem {Shabanov99}S.V. Shabanov, Phys. Lett. B\textbf{463}, 263
%--272
(1999). [hep-th/9907182]

%S.V. Shabanov, Phys. Lett. B\textbf{458}, 322 %--330
%(1999). [hep-th/0608111];

\bibitem {FN98}L.D. Faddeev and A.J. Niemi, Phys. Rev. Lett. \textbf{82}, 1624
(1999). [hep-th/9807069];

L.D. Faddeev and A.J. Niemi, Nucl. Phys. B\textbf{776}, 38
%--65
(2007). [hep-th/0608111]

\bibitem {Exactdecomp09}A. Shibata, K.-I.Kondo, T.Shinohara, Phys.
Lett.B\textbf{691}, 91
%--98
(2010). arXiv:0706.2529 [hep-lat]

\bibitem {CFNdccomp07}A. Shibata, S. Kato, K.-I. Kondo, T. Murakami, T.
Shinohara, S. Ito,
%Compact lattice formulation of Cho-Faddeev-Niemi decomposition: gluon mass generation and infrared Abelian dominance
Phys. Lett. B\textbf{653}, 101
%--108
(2007). arXiv:0706.2529 [hep-lat]

\bibitem {KKSS15}K.-I. Kondo, S. Kato, A. Shibata and T. Shinohara, Phys.
Rept. \textbf{579}, 1--226 (2015). arXiv:1409.1599 [hep-th]

\bibitem {dualsuper}
%\bibitem{Nambu74}
Y. Nambu,
%Strings, monopoles, and gauge fields,
Phys. Rev. D\textbf{10}, 4262 (1974). \newline G. 't Hooft, in: High Energy
Physics, edited by A. Zichichi (Editorice Compositori, Bologna, 1975).
\newline S. Mandelstam,
%Vortices and quark confinement in non-abelian gauge theories,
Phys. Rept. \textbf{23}, 245 (1976).

%\bibitem {tHooft81}G. 't Hooft,
%Topology of the gauge condition and new confinement phases in non-Abelian gauge theories,
%Nucl. Phys. B\textbf{190} [FS3], 455 (1981).

%\bibitem {EI82}Z.F. Ezawa and A. Iwazaki,
%Abelian dominance and quark confinement in Yang-Mills theories,
%Phys. Rev. D\textbf{25}, 2681 (1982).

%\bibitem {SY90}T. Suzuki and I. Yotsuyanagi,
%Possible evidence of abelian dominance in quark confinement,
%Phys. Rev. D\textbf{42}, 4257 (1990).

%\bibitem {AS99}K. Amemiya and H. Suganuma,
%Off diagonal gluon mass generation and infrared Abelian dominance in the maximally Abelian gauge in lattice QCD,
%Phys. Rev. D\textbf{60}, 114509 (1999). [hep-lat/9811035]


\bibitem {BhanotCreutz81}G. Bhanot and M. Creutz Phys. Rev. D\textbf{24}, 3212 (1981).

\bibitem{BMK77} T. Banks, R. Myerson, and J.B. Kogut, Nucl. Phys. B\textbf{129},  493 (1977).

\bibitem {Guth80} A.H. Guth, Phys. Rev. D\textbf{21}, 2291 (1980).


\bibitem {FS82} J. Fröhlich and T. Spencer, Commun. Math. Phys. \textbf{83}, 411 %-–454 
(1982).

\bibitem {Ostewlder78}K. Ostewalder and E. Seiler, Annls. Phys \textbf{110},
440 (1978).

\bibitem {FradkinShenker79}E. Fradkin and S.H. Shenker, Phys. Rev.
D\textbf{19}, 3682 (1979).

%\bibitem {PhysRep15}K.-I. Kondo, S. Kato, A.Shibata, T. Shinohara, Phys. Rept. \textbf{579}, 1--226 (2015) . e-Print:1409.1599 [hep-th]


\bibitem {Ikeda23}R. Ikeda, S. Kato, K.-I. Kondo, and A. Shibata, Preprint
CHIBA-EP-259, in preparation.



\end{thebibliography}

\end{document}

