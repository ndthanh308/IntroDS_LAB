% This is samplepaper.tex, a sample chapter demonstrating the
% LLNCS macro package for Springer Computer Science proceedings;
% Version 2.20 of 2017/10/04
%
\documentclass[runningheads]{llncs}
%
\usepackage{graphicx}
\usepackage{amsmath}
\usepackage{amssymb}
\usepackage{booktabs}
\usepackage{rotating}
\usepackage{makecell}
\usepackage{xspace}
\usepackage{xcolor}
\usepackage[export]{adjustbox}
\usepackage{multirow}
\usepackage{float}
\usepackage{pifont}
\usepackage{caption}
\usepackage{subcaption}
\usepackage{multirow}
% It is strongly recommended to use hyperref, especially for the review version.
% hyperref with option pagebackref eases the reviewers' job.
% Please disable hyperref *only* if you encounter grave issues, e.g. with the
% file validation for the camera-ready version.
%
% If you comment hyperref and then uncomment it, you should delete
% ReviewTempalte.aux before re-running LaTeX.
% (Or just hit 'q' on the first LaTeX run, let it finish, and you
%  should be clear).
\usepackage[pagebackref,breaklinks,colorlinks]{hyperref}


\newcommand{\masoud}[1]{{\color{green}{[Masoud: #1]}}}
\newcommand{\placeholder}[1]{{\color{red}{[X]}}}
\newcommand{\neda}[1]{{\color{blue}{[Neda: #1]}}}

\makeatletter
\DeclareRobustCommand\onedot{\futurelet\@let@token\@onedot}
\def\@onedot{\ifx\@let@token.\else.\null\fi\xspace}
\def\iid{\emph{i.i.d}\onedot} \def\IID{\emph{I.I.D}\onedot}
\def\eg{\emph{e.g}\onedot} \def\Eg{\emph{E.g}\onedot}
\def\ie{\emph{i.e}\onedot} \def\Ie{\emph{I.e}\onedot}
\def\cf{\emph{c.f}\onedot} \def\Cf{\emph{C.f}\onedot}
\def\etc{\emph{etc}\onedot} \def\vs{\emph{vs}\onedot}
\def\wrt{w.r.t\onedot} \def\dof{d.o.f\onedot}
\def\aka{\emph{a.k.a}\onedot}
\def\etal{\emph{et al}\onedot}
\makeatother

\newcommand*\samethanks[1][\value{footnote}]{\footnotemark[#1]}

% Support for easy cross-referencing
\usepackage[capitalize]{cleveref}
\crefname{section}{Sec.}{Secs.}
\Crefname{section}{Section}{Sections}
\Crefname{table}{Table}{Tables}
\crefname{table}{Tab.}{Tabs.}
\crefname{subtable}{Fig.}{Figures}

\begin{document}

\title{Supplementary Material}

% If the paper title is too long for the running head, you can set
% an abbreviated paper title here

\author{Hooman Vaseli et al.}

\authorrunning{Hooman Vaseli et al.}
% First names are abbreviated in the running head.
% If there are more than two authors, 'et al.' is used.

\institute{Department of Electrical and Computer Engineering, The University of British Columbia, Vancouver, BC, Canada}

\maketitle              % typeset the header of the contribution

% Figure environment removed

% Figure environment removed


\begin{table}[h]
\caption{Diagnostic performance on the test set of the TMED-2 dataset.
Image predictions are derived with  probabilities with prioritized view averaging and thresholding.}
\label{tab:tmed_results}
\resizebox{\textwidth}{!}{
\begin{tabular}{c|cccc|cc}
\multirow{2}{*}{Method} & \multicolumn{4}{c|}{Aggregation Method}
& \multicolumn{2}{c}{Study-level}                      
\\ 
& \multicolumn{1}{c}{Proto Layer} 
& \multicolumn{1}{c}{Prioritized View} 
& \multicolumn{1}{c}{View Thresh} 
& \multicolumn{1}{c|}{Abs Thresh} 
& \multicolumn{1}{c}{View BACC} 
& \multicolumn{1}{c}{AS BACC}  \\ 
\hline 
\hline
Huang et al
& x
& \checkmark 
& \checkmark 
& x 
& 96.2\%
& 74.6\% \\ 
Ours 
& \checkmark 
& \checkmark 
& \checkmark 
& \checkmark 
& 96.1\%  
& \textbf{79.7\%}   \\ 
\end{tabular}
}
\end{table}

\begin{table}[h!]
    \settowidth\rotheadsize{Example 1}
    \begin{tabular}
    {@{\hspace{0mm}}c@{\hspace{1mm}}c@{\hspace{1mm}}c@{\hspace{1mm}}c}
    \centering
        \rothead{\centering Healthy} &
        % Figure removed & 
        % Figure removed & 
        % Figure removed 
        \\ \addlinespace[0.5mm]
        \rothead{\centering Significant} &
        % Figure removed & 
        % Figure removed & 
        % Figure removed 
        \\ \addlinespace[0.5mm]
        \rothead{\centering Uncertainty} &
        % Figure removed & 
        % Figure removed & 
        % Figure removed 
        \\ \addlinespace[0.5mm]
    \end{tabular}
    \caption{ Most prototypes focus on areas around the aortic valve. Three example prototypes for healthy and significant AS have been demonstrated. Three uncertainty prototypes are also shown. Most uncertainty prototypes capture views other than PLAX and PSAX or images that contain extra information such as EKG or patient information.}
     \label{tab:tmed_representations}
\end{table}


\end{document}

