%!TEX TS-program = pdflatex
%!TEX spellcheck {"en-us": "Packages/Language - English/en_US.dic"}

% \documentclass[cha,twocolumn, amsmath,amssymb,reprint,aip,floatfix,nofootinbib]{revtex4-2}  
\documentclass[aps,prab,reprint,twocolumn,nofootinbib,amsmath,amssymb]{revtex4-2}
% \documentclass[11pt,a4paper,USenglish]{scrartcl}
% \AtBeginDocument{
%     \hypersetup{hidelinks}
% }

\usepackage[utf8]{inputenc}
\usepackage{lmodern}
% \usepackage[scaled]{helvet}
% \usepackage[defaultsans,scale=0.95]{opensans}
\usepackage[scale=0.97]{sourcesanspro}
% \usepackage[T1]{fontenc}

% load math befor setup of fonts (ams-thinggie)!
% \usepackage{amsmath,amssymb}
% \usepackage{array} % improved formatting and options for equation arrays
\DeclareMathAlphabet{\mathup}{OT1}{\familydefault}{m}{n} % only for non-lualatex
\usepackage{dakmath}

\usepackage[ngerman,USenglish]{babel}
\makeatletter
\adddialect\l@en\l@USenglish
\adddialect\l@Englisch\l@USenglish
\adddialect\l@de\l@ngerman
\makeatother

\usepackage{siunitx}
\sisetup{
  exponent-product = \ensuremath{\cdot}, 
  per-mode = symbol,
}
\DeclareSIUnit{\permille}{\text{\textperthousand}}
\usepackage{daktikz}
\usepackage{pgfplotstable}
\usepgfplotslibrary{polar}
\tikzexternaldisable

\usepackage{graphicx}
\DeclareGraphicsExtensions{.pdf,.jpeg,.png}
%% positioning of floats/figures
%% more info: https://robjhyndman.com/hyndsight/latex-floats/
% \setcounter{topnumber}{2} % maximum number at top
% \setcounter{bottomnumber}{2} % maximum number at bottom
% \setcounter{totalnumber}{4} % maximum number on page
% \renewcommand{\topfraction}{0.85} % maximum page fraction to use at top
% \renewcommand{\bottomfraction}{0.85} % maximum page fraction to use at bottom
% \renewcommand{\textfraction}{0.1} % minimum fraction for text
% \renewcommand{\floatpagefraction}{0.7} % minimum fraction for a float page (page with only floats)

\usepackage[hidelinks]{hyperref}

\hyphenation{wave-gui-de wave-gui-des eval-u-a-tion an-iso-tro-pic}

\begin{document}

\title[Quasi-ZGVs and beating resonances]{Beating resonance patterns and orthogonal wave propagation due to zero-group-velocity guided elastic waves}

\author{Daniel~A.~Kiefer}
\email{daniel.kiefer@espci.fr}
\affiliation{Institut Langevin, ESPCI Paris, Universit\'e PSL, CNRS, 75005 Paris, France}

\author{Sylvain~Mezil}
\affiliation{Institut Langevin, ESPCI Paris, Universit\'e PSL, CNRS, 75005 Paris, France}

\author{Claire~Prada}
\affiliation{Institut Langevin, ESPCI Paris, Universit\'e PSL, CNRS, 75005 Paris, France}

\date{\today} 

\begin{abstract}\noindent
Elastic waves in anisotropic media can exhibit a power flux that is not collinear with the wave vector. We show that this has remarkable consequences for the zero-group-velocity (ZGV) resonances that appear in an infinite plate. True ZGV resonances, in the sense that the guided wave's overall power flux vanishes while its wavelength remains finite, are only found when the wave vector is oriented along a principal axis of the material. At other propagation angles, ``quasi-ZGV'' waves are found for which the power flux is orthogonal to the wave vector. We present original measurements of such a wave field in a single crystal silicon wafer. As a consequence of the nonzero power flux of qZGV waves, time acts as a filter in the wave vector domain that selects precisely the eight plane waves corresponding to true ZGV resonances. This intuitively explains the highly symmetric resonance pattern that emerges naturally on the surface of the plate after a pulsed point source excitation. We showcase a direct measurement of this beating pattern. The developed physical understanding paves the way towards novel designs of surface acoustic wave devices and is of high relevance for ultrasonic nondestructive evaluation.
\end{abstract}
% \begin{keyword}
% elasticity; guided elastic waves; zero-group-velocity (ZGV) resonances; waveguides; ultrasound
% \end{keyword}

\maketitle

\section{Introduction}
Common engineering materials often exhibit anisotropic stiffness, in particular, single crystals and composites. Monocrystalline silicon is the single most important material for fabrication of integrated circuits, microelectromechanical systems (MEMS) and photovoltaic cells~\cite{hopcroft_what_2010,belyaev_crack_2006}. These are commonly produced from a wafer, i.e., a thin plate of this material. Understanding the intricate mechanics of elastic wave propagation in these plates is of particular importance for the functional design of MEMS devices such as surface acoustic wave filters and sensors, as well as for their quality evaluation~\cite{chakrapani_crack_2012,song_crack_2013,masserey_defect_2019}.

As the structures in the mentioned applications are usually thin, guided elastic wave propagation is of great relevance. These waves propagate dispersively~\cite{auld_acoustic_1990,royer_elastic_2022}, i.e., their angular frequency~$\omega$ is non-linearly related to their wavenumber~$k$ through the dispersion relation~$\omega(k)$. Of particular importance are solutions where the group velocity~${\partial \omega / \partial k}$ vanishes while the wavenumber remains finite~\cite{prada_laser-based_2005,clorennec_laser_2006,clorennec_local_2007,balogun_simulation_2007,prada_local_2008,xie_imaging_2019}. These zero-group-velocity (ZGV) points represent local resonances that are due to the finite thickness of the semi-infinite structure. At sufficiently high frequencies they usually dominate the overall mechanical response of the structure. They are rather simple to excite and measure with contactless laser-ultrasonic techniques. 
Nondestructive evaluation techniques exploiting ZGV resonances are capable of retrieving various structural properties such as material parameters, thickness or bonding state with very high precision in a spatially resolved manner~\cite{ces_characterization_2012,grunsteidl_inverse_2016,grunsteidl_determination_2018,mezil_non_2014,ryzy_determining_2023}. 

% Figure environment removed

ZGV resonances in anisotropic plates were the object of some theoretical~\cite{kiefer_computing_2023,glushkov_multiple_2021,hussain_lamb_2012} as well as experimental studies~\cite{prada_influence_2009,thelen_laser-excited_2021}. Prada et al.~\cite{prada_influence_2009} revealed the directional dependence of the ZGV resonance frequency in a silicon wafer and we reproduce this result in Fig.~\ref{fig:qZGV_resonance_time}a. This behavior is somewhat astonishing given that the resonance is due to the plate's thickness, which is clearly rotationally invariant. In fact, such a directional dependence is impossible for common thickness resonances characterized by ${k = 0}$. It is striking that the extrema in Fig.~\ref{fig:qZGV_resonance_time} seem to play a special role, but this effect has remained unexplored. Furthermore, in Ref.~\cite{prada_influence_2009} the response to the point-source excitation was observed to be dominated by the two ZGV frequencies associated to principal directions of the crystal. This fact was confirmed by theoretical computations in Ref.~\cite{glushkov_multiple_2021}. Here again, an explanation is missing in the literature. 

The aim of the present contribution is twofold. Firstly, the special role of ZGV resonances along principal directions of the material is explained. At the same time, the role and very particular characteristics of ``resonances'' along other directions is assessed. Secondly, we predict and confirm the formation of a complex resonance pattern on the surface of the plate after it has been excited by an impulsive point load. We show that this beating pattern can be explained by the interference of eight ZGV modes associated to the crystal's principal directions. 

% The article is structured as follows: the sample and the required theoretical background on guided elastic waves and their power flux is presented in Sec.~\ref{sub:guided_waves_in_silicon}. The differences between ZGV resonances along principal and non-principal directions are discussed in detail in Sec.~\ref{sub:quasi_zgv_resonances}. Subsequently, in Sec.~\ref{sub:resonance_pattern} we model and present evidence of the hypothesized resonance pattern that forms on the surface of the plate. Lastly, a conclusion is given in Sec.~\ref{sec:Discussion}.


\section{Results} 
\label{sec:results}


\subsection{Guided waves in a silicon wafer}
\label{sub:guided_waves_in_silicon}
We study guided elastic waves propagating in a monocrystalline silicon wafer as depicted in Fig.~\ref{fig:geometry_wafer}. The plate has stress-free surfaces and is considered to be of infinite lateral dimensions, i.e., we disregard waves reflected from the border. The material's stiffness is of cubic anisotropy (Voigt-notated stiffness $C_{11} = \SI{165.6}{\giga\pascal}$, $C_{12} = \SI{63.9}{\giga\pascal}$, $C_{44} = \SI{79.5}{\giga\pascal}$, density of \SI{2330}{\kilo\gram\per\meter\cubed}). The $\dirvec{X}\dirvec{Y}\dirvec{Z}$ coordinate system is aligned with the crystal axes, while $\dirvec{r}\dirvec{\theta}\dirvec{z}$ is oriented with the wave vector. Taking the point of view of a plane wave, $\dirvec{r}$ denotes the \emph{axial direction}, while $\dirvec{\theta}$ refers to the \emph{transverse direction}. 

% Figure environment removed

The guided waves are characterized by the angular frequency~$\omega$, the wave vector~${\ten{k} = k \dirvec{r}(\theta)} = k_X \dirvec{X} + k_Y \dirvec{Y}$, as well as the through-thickness displacement distribution~$\ten{u}(z)$. Only certain combinations of frequency and wave vector can propagate, which is described by the \emph{dispersion relation}~\cite{auld_acoustic_1990,royer_elastic_2022,langenberg_theoretische_2009,kiefer_elastodynamic_2022}. The solutions form surfaces $\omega(k_X, k_Y)$, or equivalently $\omega(k, \theta)$. We use a semi-analytical method to obtain these solutions, i.e., the associated eigenvalue problem is solved numerically. Our implementation is made available as \texttt{GEWtool}~\cite{kiefer_gewtool_2023}. Therewith, we are also able to compute ZGV points (and the qZGV points that will be introduced later on) by employing the numerical methods that we have recently developed for this purpose~\cite{kiefer_computing_2023,plestenjak_gew_2023}. For details on the model and the numerical methods, see Section~\ref{sub:computing_guided_waves_and_power_flux}. 

The result of the computation is depicted in Fig.~\ref{fig:S1S2b_ZGV_surf}. Only the dispersion surface of the first modes that exhibit ZGV points are shown. Note that due to the cubic anisotropy of silicon, the dispersion surfaces are \SI{90}{\degree}-periodic in $\theta$. Furthermore, we observe reflection symmetry across the directions $\theta = \SI{0}{\degree}$ and $\theta = \SI{45}{\degree}$, which correspond to the [110] and [010] crystallographic axes, respectively.
% Note that rotating from $\theta = \SI{45}{\degree}$ to $\theta = \SI{90}{\degree}$ makes the curves transition back to the initial behavior at $\theta = \SI{0}{\degree}$. 

% Figure environment removed

Most commonly, dispersion curves along a fixed propagation direction $\theta$ are inspected and they correspond to cuts across the aforementioned ${\omega(k, \theta)}$-surfaces. The results in three different propagation directions are depicted in Fig.~\ref{fig:dispersion}a. Note that \SI{0}{\degree} and \SI{45}{\degree} are the directions where the curves are most different. Although pure Lamb/SH waves do not generally exist in the silicon plate because they are coupled, we label the mode branches consistently to the commonly used notation. In particular, following the notation in Ref.~\cite{prada_local_2008}, the positive-slope branch shown in Fig.~\ref{fig:dispersion}a will be denoted as S1, while the negative-slope section is called S2b. 

% Figure environment removed

The power flux of the waves is proportional to the group velocity~\cite{langenberg_theoretische_2009,langenberg_energy_2010,auld_acoustic_1990,royer_elastic_2022}. The latter is defined as the gradient of $\omega$ with respect to the components of the wave vector, i.e.,
\begin{equation}\label{eq:group_velocity}
   \ten{c}_\mup{g} := \nabla_{\ten{k}} \omega = \frac{\partial \omega}{\partial k} \dirvec{r} + \frac{1}{k} \frac{\partial \omega}{\partial \theta} \dirvec{\theta} \,.
\end{equation} 
It is worth remarking that these vectors are–per definition–orthogonal to the iso-frequency lines drawn in Fig.~\ref{fig:S1S2b_ZGV_surf}a.

The power flux of waves is discussed by comparing propagation at the principal direction $\theta = \SI{0}{\degree}$ and the nonprincipal direction $\theta = \SI{26.6}{\degree}$. We explicitly computed both components of the group velocity and the result is depicted in Fig.~\ref{fig:dispersion}b. Due to the material's reflection symmetry across principal axes, the transverse power flux $c_{\mup{g}\theta}$ is identically zero on principal directions. This is not the case for other propagation directions, as evidenced in Fig.~\ref{fig:dispersion}b. Furthermore, when the axial component~$c_{\mup{g}r}$ is positive we speak of \emph{forward waves}, otherwise of \emph{backward waves}~\cite{prada_local_2008,shuvalov_backward_2008}. With respect to a source, a backward wave exhibits outgoing flux but incoming phase fronts. This is the case for the S2b wave. Note that the phenomena investigated hereinafter are strongly related to the existence of such waves.

The transverse group velocity component~$c_{\mup{g}\theta}$ is due to the anisotropy $\partial \omega / \partial \theta$ and induces a transverse power flux. We remark that it is the coupling of Lamb- and SH-polarizations that implies this (usually) nonzero transverse power flux~\cite[Appendix~B]{kiefer_computing_2023}. Due to this transverse component, the overall power flux (or group velocity) is not collinear with the wave vector. The angle between the two is denoted as \emph{steering angle} or \emph{skew angle}~$\alpha$ \cite{langenberg_theoretische_2009,chapuis_excitation_2010,karmazin_study_2013}. The frequency-dependent skew angles of the S1/S2b modes are shown in Fig.~\ref{fig:dispersion}c. It is remarkable that they cover almost \SI{180}{\degree} in a rather narrow frequency range. 




\subsection{ZGV and quasi-ZGV resonances}
\label{sub:quasi_zgv_resonances}

% Figure environment removed

Resonances appear where the power flux of the waves vanishes and such points are denoted as ZGV points. They are commonly identified by local extrema in the wavenumber dispersion curves, such as in Fig.~\ref{fig:dispersion}a. According to equation (\ref{eq:group_velocity}) this is not appropriate for anisotropic media because these curves only reveal the axial component $c_{\mup{g}r} = \partial \omega / \partial k$ of the group velocity vector. Hence, while the marked points on the curves of Fig.~\ref{fig:dispersion}a appear to be ZGV points in the conventional sense that $\partial \omega / \partial k = 0$, their overall power flux might not vanish. Albeit their axial power flux is zero, their transverse power flux can be nonzero. For this reason we distinguish between 
\begin{enumerate}
  \item ZGV points: waves with overall zero power flow, i.e., $\ten{c}_\mup{g} = \ten{0}$ and
  \item quasi-ZGV (qZGV) points: waves with vanishing \emph{axial} power flux, i.e., $\partial \omega / \partial k = 0$.
\end{enumerate}
Note that this distinction is not necessary in isotropic media because all qZGV points are also ZGV points. This is a consequence of the decoupling of Lamb and SH polarizations, which is no longer the case in anisotropic media.

A \emph{true ZGV point} is an extremum or saddle point of the dispersion surface depicted in Fig.~\ref{fig:S1S2b_ZGV_surf}. Our cubic material exhibits eight isolated ZGV points on the principal directions. The four minima in the $\langle 010 \rangle$ directions are denoted as ZGV1 and are characterized by $\omega/2\pi = \SI{7.7079}{\mega\hertz}$ and $k = \SI{3.421}{\radian\per\milli\meter}$. The ZGV2 are saddle points at $\langle 110 \rangle$ with $\omega/2\pi = \SI{7.7551}{\mega\hertz}$ and $k = \SI{3.142}{\radian\per\milli\meter}$. The wavenumber and frequency of the \emph{qZGV points} also depend on the wave vector orientation~$\theta$ and form the continuous red curve drawn in Fig.~\ref{fig:S1S2b_ZGV_surf}a and Fig.~\ref{fig:S1S2b_ZGV_surf}b. The closed qZGV-curve separates the forward wave region (outside) from the backward wave region (inside). 



\subsubsection{qZGV radiated by a line source}
The existence of qZGV waves is quite remarkable as their power flux is \emph{orthogonal} to the wave vector! While this fact is obvious from their definition, the evolution of power flux when approaching the qZGV-frequency can be assessed in Fig.~\ref{fig:dispersion}c: when approaching the qZGV point from high frequencies, the S1 and S2b skew angles both tend towards \SI{-90}{\degree}, which they reach when meeting at the qZGV point. Note that the weaker the coupling between Lamb and SH modes, the sharper we expect the transition towards \SI{-90}{\degree} to take place. Ultimately, when these wave families decouple (i.e., on a principal direction or in an isotropic plate), we obtain a discontinuity as observed in Fig.~\ref{fig:dispersion}c for $\theta = \SI{0}{\degree}$. The combination of orthogonal propagation and wide skew-angle spectrum lead to extraordinary diffraction effects.

Waves usually radiate in normal direction to the line source and diffract (approximately) circularly around its edges. This is not the case for the qZGV wave, as can be observed in measurements. For this end, we performed a 2D point-by-point scan with laser-ultrasonic techniques to obtain the wave field on the surface of the silicon wafer, for details see Sec.~\ref{sub:measurement_setup}. Fig.~\ref{fig:qzgv_line_intensity}a depicts the response to a \SI{6.4}{\milli\meter} line source preferentially exciting wave vectors at \SI{26.6}{\degree} and \SI{-153.4}{\degree}. The associated wavenumber spectrum at the computationally obtained qZGV-frequency is compared to the theoretical wavenumbers in Fig.~\ref{fig:qzgv_line_intensity}a. In order to observe the qZGV wave, we apply a Gaussian filter in the wavenumber plane. Its full width at half maximum is delineated in Fig.~\ref{fig:qzgv_line_intensity}a. A subsequent inverse Fourier transform into the spatial domain yields the instantaneous intensity distributions depicted in Fig.~\ref{fig:qzgv_line_intensity}b. Two wave packets propagating along the line source can clearly be discerned, i.e., their skew angle is indeed $\SI{\pm 90}{\degree}$ as expected. From another point of view, this wave ``diffracts'' around the line's edge while maintaining the orientation of the phase fronts aligned with the line source. For the best of our knowledge, this is the first measurement of a transversely/orthogonally propagating wave packet.

A further validation is done by comparing the group velocity of the wave packets to the theory. After \SI{40}{\micro\second} the maxima of the envelops are \SI{6.81}{\milli\meter} from the source's center, yielding a transverse velocity of \SI{\pm0.170}{\milli\meter\per\micro\second}. This is in very good agreement with the computationally obtained transverse group velocity of \SI{-0.177}{\milli\meter\per\micro\second} for the qZGV wave at $\theta = \SI{26.6}{\degree}$, as Fig.~\ref{fig:dispersion}b shows. Note that the wave vectors oriented at $\theta = \SI{-153.4}{\degree}$ lead to the wave packet with positive transverse group velocity.

Propagation of energy purely orthogonal to the wave vector is a strictly monochromatic process, both in frequency and wavenumber. The neighboring spectral components exhibit a nonzero axial power flux and these waves are responsible for the packets' lateral extend.
Wave packets and their power flux can only be observed when considering a spectrum of finite width. This is ensured by our Gaussian filter in Fig.~\ref{fig:qzgv_line_intensity}a, which has a half-width of \SI{0.7}{\radian\per\milli\meter}. While all spectral components of the wave packets possess very similar wave vectors, their power flux is along very different directions. In fact, the skew-angle spectrum contained in a packet spans almost \SI{180}{\degree}, as Fig.~\ref{fig:qzgv_line_intensity}a shows. Remember that the power flux is orthogonal to the constant-frequency curves drawn therein.

% Figure environment removed

Lastly, it is worth observing that the two wave packet's phases propagate in opposite directions, as indicated in Fig.~\ref{fig:qzgv_line_intensity}b. It is remarkable that the phase fronts of one packet are all in the same direction. 
This means that the phases propagate \emph{towards the source} on one side, and \emph{away from the source} on the other side, as indicated by the wave vector arrows in Fig.~\ref{fig:qzgv_line_intensity}b. Accordingly, each wave packet clearly consists of the S1 forward wave and the S2b backward wave, separated spatially by the qZGV component. This behavior is in accordance with Fig.~\ref{fig:qzgv_line_intensity}a as well as Fig.~\ref{fig:dispersion}c. The counter-propagating phases of the wave packets lead to quickly varying interference in the regions where the two superpose.



\subsubsection{Decay of qZGV resonances}
The decay observed for some spectral components in the introducing Fig.~\ref{fig:qZGV_resonance_time} can now be explained. In contrast to proper ZGVs, the qZGVs exhibit a small but nonzero overall power flux. 
Note that the excitation was performed with a line source, hence, selecting a propagation direction and, consequently, either a ZGV or a qZGV mode. The latter disappear completely after about \SI{40}{\micro\second}, while the expected eight ZGV resonances at $\theta = n \times \SI{45}{\degree}, n \in \mathbb{Z}$, perdure. 

We conclude that, while ZGV modes are resonances of semi-infinite structures, the qZGV points can be interpreted as corresponding \emph{leaky resonances}. More concretely, they are resonant in one direction along the plate (axial direction) but radiate waves in the orthogonal direction (transverse direction). Consequently, they can only exist in structures that are bounded in exactly one direction (infinite plates/cylinders/shells). Albeit their losses, the qZGV resonances evidently play an important role in the presented measurements, which underlines the relevance of the concept.




\subsection{Resonance pattern of ZGV modes}
\label{sub:resonance_pattern}

% Figure environment removed

The previous results demonstrate that time acts as a filter in the wave vector-frequency domain. After sufficient time, only the ZGV resonances remain in the finite spatial region of observation. As a consequence, a resonance pattern develops in the silicon crystal plate. For the frequency region of interest, this pattern is explained by the interference of the eight ZGV resonances sketched in Fig.~\ref{fig:resonance_pattern_synth}a and Fig.~\ref{fig:resonance_pattern_synth}c. 

The interference of the four wave vectors pertaining to a given ZGV leads to the periodic square wave patterns depicted in Fig.~\ref{fig:resonance_pattern_synth}b and Fig.~\ref{fig:resonance_pattern_synth}d. Note that for each wave vector $\ten{k}$, there exists a counter-propagating wave $-\ten{k}$. As a consequence, the square patterns are standing wave fields~\cite{prada_local_2008}. It is notable that this standing wave field is not due to the edges of the plate but actually emerges in the \emph{infinite plate}. The ZGV1 and ZGV2 waves interfere to form the overall resonance pattern shown in Fig.~\ref{fig:resonance_pattern_synth}f. As each of the wave components is flux free, no energy is propagated. Nonetheless, this is no longer a standing wave field. Instead, the phase fronts move radially out or into the center of the pattern. This can be observed in the supplementary video. 

For the field synthesis shown in Fig.~\ref{fig:resonance_pattern_synth}f we arbitrarily assumed identical amplitudes and a zero phase shift between ZGV1 and ZGV2. Due to the similarity of the displacement eigenfunctions $\ten{u}_\mup{ZGV1}(z)$ and $\ten{u}_\mup{ZGV2}(z)$, both modes are excited almost identically by the point source. This is confirmed in Fig.~\ref{fig:qZGV_resonance_time}, where the measured amplitude at the frequency extrema is very similar. For the same reason, we expect both ZGVs to be in phase at the moment of generation by the pulse.

The phase shift between ZGV1 and ZGV2, on the other hand, is modulated in both time and space due to their somewhat different frequencies and wavenumbers. In other words, temporal and spatial beating is expected as a consequence of their superposition. Hence, the instantaneously observed phase shift loops $2\pi$ in a period of 
\begin{equation}\label{eq:beating_period}
  \Delta T = \frac{2\pi}{\omega_\mup{ZGV2} - \omega_\mup{ZGV1}} = \SI{21.2}{\micro\second}\,.
\end{equation}

The described resonance pattern is directly observable by measuring the wave field excited by a point-like source. For this end, it is sufficient to avoid other resonances as well as reflections from the border of the plate. We achieve this with a band-pass filter between \SI{7}{\mega\hertz} and \SI{8}{\mega\hertz} that is applied on the experimental response, see Sec.~\ref{sub:measurement_setup} for details. The filtered displacement fields are depicted in Fig.~\ref{fig:resonance_pattern} for selected time instances. To our knowledge, this is the first experimental evidence of ZGV resonance patterns in anisotropic plates.

After one beating period $\Delta T$, the two ZGVs should be in-phase for the first time and, hence, we expect a pattern similar to Fig.~\ref{fig:resonance_pattern_synth}f. Indeed, at \SI{21.25}{\micro\second} we can see the ZGV resonance pattern in the inner region marked by the rectangle in Fig.~\ref{fig:resonance_pattern}. Note that it is difficult to read out the exact time due to the beating behavior of the field. But overall, this timing is in good agreement with equation (\ref{eq:beating_period}). The pattern requires a long time to form due to the low power flux of the participating waves. Nonetheless, after two beating periods, at \SI{42.4}{\micro\second}, we recognize a fully developed field in the region of observation. Discrepancies to Fig.~\ref{fig:resonance_pattern_synth}f can mostly be attributed to imprecise timing of the snapshot, as we have only about 13 samples per period.

% Figure environment removed

A quantitative validation is performed by comparing measurement and theory in the spectral domain. First, a time-Fourier transform yields the measured wave fields at the ZGV frequencies. The result is depicted in Fig.~\ref{fig:measurement_fields}b and Fig.~\ref{fig:measurement_fields}d and both confirm the wave fields expected in Figs.~\ref{fig:resonance_pattern_synth}b and \ref{fig:resonance_pattern_synth}d. Second, we additionally perform a spatial 2D Fourier transform into the wave vector-frequency domain. The obtained spectral amplitudes are displayed in Fig.~\ref{fig:measurement_fields}a and Fig.~\ref{fig:measurement_fields}c together with the ZGV wave vectors expected from theory. We observe that the energy confines closely to the regions predicted by the computed wave vectors. While the wave field at $\omega_\mup{ZGV1}$ consists almost purely of the ZGV resonances, propagating modes exist at $\omega_\mup{ZGV2}$ and are also excited. Note that spatio-temporal gating can be used to remove the propagating waves as was done in Fig.~\ref{fig:qZGV_resonance_time}. Overall, this confirms that the resonances observed in an infinite plate of anisotropic elasticity are formed by the superposition of a discrete set of proper ZGV modes.



\section{Discussion}
\label{sec:Discussion}
ZGV resonances have been used very successfully for ultrasonic nondestructive testing and material characterization. We expect the same principles to be exploitable for novel microelectromechanical systems, in particular sensors. They have the potential of outperforming current designs while at the same time being simpler to implement. The present work overcomes the main difficulty in designing such systems: properly accounting for anisotropy, such as is encountered for silicon.

Our analysis demonstrates that ZGV resonances directly reveal information about the principal axes of the material, independent of the actual orientation of the sample. Measuring the ZGV1 and ZGV2 resonance frequencies as well as the frequency of a regular thickness resonance is sufficient to fully characterize a material of cubic symmetry. These frequencies are readily obtained with a single pulse-echo measurement as they all manifest as sharp peaks in the spectrum~\cite{prada_influence_2009}. This avoids the technically complicated scan of the wave field, as would need to be done for widespread techniques based on guided waves. Overall, this enables nondestructive, contactless and real-time material characterization. 

The discussed phenomena are very particular to two-dimensional wave propagation in anisotropic media. Analogous phenomena should be expected when the material pertains to a different anisotropy class than the one studied here. Moreover, the same effects are expected for waves in phononic crystals or meta-materials, as dispersion surfaces with similar features haven been reported in this case~\cite{bossart_extreme_2023,maznev_mapping_2011,kaina_negative_2015,yves_crystalline_2017}. 

The present contribution restricted to ZGVs due to the S1/S2b waves. It is well known that other ZGV resonances exist at higher-frequencies. These superpose with the resonance that was isolated and studied in this work. Note that anisotropic plates can also support various ZGV resonances pertaining to the same modes~\cite{kiefer_computing_2023}. In this case, not only local minima, but also local maxima appear in the dispersion surface. The analysis and mechanics presented here apply analogously. 




\section{Methods}
\label{sec:methods}

\subsection{Computing guided waves and power flux}
\label{sub:computing_guided_waves_and_power_flux}

The procedure that we use to compute guided elastic waves is sketched in the following. The waves propagate in a plate as depicted in Fig.~\ref{fig:geometry_wafer} that is assumed to be infinite in the $X$-$Y$-plane.
Its material is of anisotropic stiffness~$\ten{c}$ and mass density~$\rho$. The plane waves of wavenumber~$k$ shall be harmonic with angular frequency~$\omega$. Hence, their wave field is of the form 
\begin{equation}
  \ten{\tilde{u}}(r,\theta,z,t) = \ten{u}(z) \e^{\iu k r} \e^{-\iu \omega t} \,.
\end{equation}
We proceed by arbitrarily fixing the propagation vector $\ten{k} = k \dirvec{r}$ and computing the corresponding frequency~$\omega$ and modal displacements~$\ten{u}(z)$. These are obtained as the solutions of a differential eigenvalue problem~\cite{adamou_spectral_2004,kiefer_elastodynamic_2022,kiefer_computing_2023}. They represent the eigenvalues and the eigenfunctions, respectively. For a concise derivation of the concrete problem formulation see Ref.~\cite{kiefer_computing_2023}. 

We compute solutions with a \emph{semi-analytical procedure} that consists in two steps: (i) discretize the differential eigenvalue problem and (ii) solve the resulting algebraic eigenvalue problem numerically. Many variants of this procedure have been discussed in the literature~\cite{adamou_spectral_2004,gravenkamp_numerical_2012,finnveden_evaluation_2004}. A concise implementation based on spectral collocation is \texttt{GEW dispersion script}~\cite{kiefer_gew_2022}. For the current work, we perform the discretization using \emph{spectral elements}, i.e., one finite element of high polynomial order~\cite{gravenkamp_high-order_2021}. Our implementation is available as \texttt{GEWtool}~\cite{kiefer_gewtool_2023} and includes all required post-processing methods to reproduce the results of this contribution. The spectral element method leads to a regular Hermitian eigenvalue problem~\cite{gravenkamp_notes_2023}, which allows us to reliably compute the (quasi)-ZGV points in general elastic waveguides~\cite{kiefer_computing_2023}. These computational methods are included in \texttt{GEWtool} and area also separately available~\cite{plestenjak_gew_2023}.

The coupling and decoupling of wave families is of importance~\cite{kiefer_elastodynamic_2022,kiefer_computing_2023,hernando_quintanilla_symmetry_2017,gravenkamp_notes_2023}. Symmetric and anti-symmetric waves decouple due to the plate's symmetry across its mid-plane. We compute only the symmetric waves by modeling the top half of the plate and fixing the $u_z$ displacement component at the mid-plane.
Moreover, the shear-horizontal (SH) polarization ($u_\theta$) only decouples form the Lamb polarization ($u_r, u_z$) for $\theta = n \times \SI{45}{\degree}, n \in \mathbb{Z}$. For this reason, we always compute the \emph{fully-coupled waves}, meaning that all three displacement components are accounted for in the displacement vector $\ten{u}(z)$. 

After computing guided wave solutions, their power flux and group velocity can be computed in a post-processing step. For this end, we exploit the fact that for waves with real-valued $\ten{k}$ and $\omega$ in nondissipative media, the group velocity is equivalent to the \emph{energy velocity}~\cite{langenberg_energy_2010,auld_acoustic_1990,biot_general_1957}, which is given by
\begin{equation}\label{eq:energy_vel}
  \ten{c}_\mup{g} = \ten{c}_\mup{e} = \frac{\int \ten{p} \diff z}{\int  \mathcal{H} \diff z} \,.
\end{equation}
It is defined through the power flux density vector~$\ten{p}$ and the average total stored energy density~$\mathcal{H}$. Using the particle velocity~$\ten{v} = -\iu\omega \ten{u}$ and stress $\ten{T} = { \ten{c}:\nabla \ten{u} } = {\ten{c} : (\iu k \dirvec{r} + \dirvec{z} \partial_z) \ten{u} }$, the power flux density can be computed as 
\begin{align}
  \ten{p} &= -\frac{1}{2} \Re\{ \ten{v}^* \cdot \ten{T} \} \nonumber\\
  \label{eq:power_flux}
  &= -\frac{1}{2} \Re\{ \iu\omega \ten{u}^* \cdot \ten{c} : (\iu k \dirvec{r} + \dirvec{z} \partial_z) \ten{u} \} \,,
\end{align}

where ``$^*$'' denotes complex conjugation, ``$\Re$'' the real-part operator and $\partial_z = \partial/\partial z$. Furthermore, due to equipartition of energy~\cite{langenberg_theoretische_2009}, we can compute the average total stored energy density by
\begin{equation}\label{eq:stored_energy}
  \mathcal{H} = \frac{1}{2} \rho \omega^2 \ten{u}^* \cdot \ten{u} \,.
\end{equation}

Our previously outlined procedure to solve for guided waves yields all quantities required to compute equation (\ref{eq:power_flux}) and equation (\ref{eq:stored_energy}). The differentiation in equation (\ref{eq:power_flux}) and the integration in equation (\ref{eq:energy_vel}) can be performed numerically. In this way, we can compute the energy velocity of each point of the dispersion curves independently. Most importantly, proceeding in this way explicitly provides both the axial and the transverse components of the energy velocity vector. The $z$-component is identically zero due to the power-flux-free surfaces of the plate.

Lastly, the \emph{skew angle}~$\alpha$ can be computed from the energy velocity vector as 
\begin{equation}
  \alpha = 
  \begin{cases}
    \arctan (c_{\mup{e}\theta}/c_{\mup{e}r}) & \text{for } c_{\mup{e}r} \ge 0 \,,\\
    \arctan (c_{\mup{e}\theta}/c_{\mup{e}r}) - \pi & \text{otherwise} \,.
  \end{cases}
\end{equation}
Note that this is valid because the wave vector is $\ten{k} = k \dirvec{r}$.






\subsection{Measurement setup}
\label{sub:measurement_setup}

Measurements are achieved all-optically and are presented schematically in Fig.~\ref{fig:schema}. Guided waves are generated with a Q-switched  Nd:YAG (yttrium aluminum garnet) laser (Quantel Laser, France, Centurion, optical wavelength of 1064\,nm, 100\,Hz repetition rate) that delivers 10\,ns pulses of 9\,mJ. The laser output beam is expanded through a beam expander and then focused onto the silicon plate with a lens (focal length 100\,mm). The focal spot is kept rather large (beam width {$\approx$1\,mm}) and excites wavenumbers up to {$\approx$7\,rad/mm}. This favors the generation of the first ZGV resonance while limiting the generation of the fundamental wave modes in the frequency range of interest~\cite{bruno16}. The wafer has a thin metallic coating on the side which is associated with the excitation, which reduces the optical penetration depth. However, similar results were obtained when exciting on the side without the coating. The coating is sufficiently thin as not to perturb the wave propagation in the silicon plate. 

% Figure environment removed

The normal surface displacement is detected on the opposite side of the silicon plate with a heterodyne interferometer of 532\,nm with a power of 100\,mW and a focal spot of {$\approx$50\,\textmu m}.  Signals are recorded by a digital sampling oscilloscope connected to a computer to process the data. Each signal is averaged 128 times. The 2D scan of {25.05\,mm\,$\times$\,25.05\,mm} is performed by moving the excitation unit with a two-axes translation stage along $\mathbf{e}_X$ and $\mathbf{e}_Y$ with steps of 150\,{\textmu m}. Note that scanning the field is very time consuming. For this reason we exploited the cubic symmetry of the material and scanned only one quadrant of the field, as indicated in the inset of Fig.~\ref{fig:schema}. The full fields depicted in Fig.~\ref{fig:resonance_pattern} have been reconstructed by repeating the measured quadrant appropriately.

While the measurements directly provide the point-source response, the line-source response shown in Fig.~\ref{fig:qzgv_line_intensity} is synthesized a posteriori. For this end, we properly shift and superpose 20 times the acquired point-source response. The $\theta = \SI{26.6}{\degree}$ orientation is obtained by shifting twice as much vertically than horizontally.

Fast waves outside the ZGV region are also excited and their reflections from the border of the plate disturb the long-time observation of the resonance pattern as well as the qZGV wave. To avoid this, the experimental response is band-pass filtered between \SI{7}{\mega\hertz} and \SI{8}{\mega\hertz}, restricting attention to the slow waves close to the ZGV points. Note that propagating waves with large wavenumbers also exist in this frequency range but they are not excited by the source due to its rather large size.





\appendix
\section*{Acknowledgements}
This work was supported by LABEX WIFI (Laboratory of Excellence within the French Program “Investments for the Future”) under Reference Nos. ANR-10-LABX-24 and ANR-10-IDEX-0001-02 PSL.

\section*{Author contributions}% bigskip need due to bug in template
D.A.K. and C.P. conceived and designed the study. S.M. conceived and performed the experiment and the data curation. D.A.K. performed the modeling and implemented the numerical model. D.A.K. and C.P. analyzed and interpreted the experimental results. D.A.K. and S.M. wrote the manuscript. All authors proofread and revised the manuscript.


\section*{Competing interests}
The authors declare no competing interests.



\bibliographystyle{naturemag}
\bibliography{ZGVpartial}
\end{document}
