\documentclass[10pt,twocolumn,letterpaper]{article}

\usepackage{iccv}
\usepackage{times}
\usepackage{epsfig}
\usepackage{graphicx}
% \usepackage{amsmath}
% \usepackage{amssymb}
\usepackage{booktabs}
\usepackage{amsmath, amsthm, amssymb}

\usepackage{algorithm}
\usepackage{algorithmic}
\renewcommand{\algorithmicrequire}{\textbf{Input:}} 
\renewcommand{\algorithmicensure}{\textbf{Output:}}
% \renewcommand{\algorithmicensure}{\textbf{Parameter:}}
\usepackage{graphicx}
\usepackage{bbm}
\usepackage{bbding}
\usepackage{url}
\usepackage{threeparttable}

\usepackage{caption}
\usepackage{balance}
\usepackage{subcaption}
\usepackage{colortbl}
\usepackage{multirow}
% Include other packages here, before hyperref.

\usepackage{color}
\newcommand{\red}[1]{\textcolor{red}{#1}}

\def\ie{\textit{i.e.}}
\def\eg{\textit{e.g.}}
\usepackage[pagebackref,breaklinks,colorlinks]{hyperref}

\usepackage{lipsum}
% \usepackage[symbol]{footmisc}
% \renewcommand{\thefootnote}{\fnsymbol{footnote}}

\newcommand\blfootnote[1]{%
  \begingroup
  \renewcommand\thefootnote{}\footnote{#1}%
  \addtocounter{footnote}{-1}%
  \endgroup
}
\usepackage{authblk}
% Support for easy cross-referencing
\usepackage[capitalize]{cleveref}
\crefname{section}{Sec.}{Secs.}
\Crefname{section}{Section}{Sections}
\Crefname{table}{Table}{Tables}
\crefname{table}{Tab.}{Tabs.}

% Include other packages here, before hyperref.

% If you comment hyperref and then uncomment it, you should delete
% egpaper.aux before re-running latex.  (Or just hit 'q' on the first latex
% run, let it finish, and you should be clear).
% \usepackage[breaklinks=true,bookmarks=false]{hyperref}

\iccvfinalcopy % *** Uncomment this line for the final submission

\def\iccvPaperID{****} % *** Enter the ICCV Paper ID here
\def\httilde{\mbox{\tt\raisebox{-.5ex}{\symbol{126}}}}

% Pages are numbered in submission mode, and unnumbered in camera-ready
\ificcvfinal\pagestyle{empty}\fi

\begin{document}

%%%%%%%%% TITLE
\title{Downstream-agnostic Adversarial Examples}


\author{Ziqi Zhou\textsuperscript{\rm $*$}\textsuperscript{1 2 3 4}, Shengshan Hu\textsuperscript{\rm $*$}\textsuperscript{1 2 3 4},  Ruizhi Zhao\textsuperscript{\rm $*$}\textsuperscript{1 2 3 4}\\ \vspace{-0.5em} Qian Wang\textsuperscript{\rm $\ddagger$}, Leo Yu Zhang\textsuperscript{\rm $\mathsection$},   Junhui Hou\textsuperscript{\rm $\mathparagraph$}, Hai Jin\textsuperscript{\rm $\dagger$}\textsuperscript{1 2 5}\\\vspace{0.5em}

\textsuperscript{\rm $*$}School of Cyber Science and Engineering, Huazhong University of Science and Technology\\
\textsuperscript{\rm $\dagger$}School of Computer Science and Technology, Huazhong University of Science and Technology\\
\textsuperscript{\rm $\ddagger$}School of Cyber Science and Engineering, Wuhan University\\
\textsuperscript{\rm $\mathsection$}School of Information and Communication Technology, Griffith University\quad\\
\textsuperscript{\rm $\mathparagraph$}Department of Computer Science, City University of Hong Kong \\
{\tt\small \{zhouziqi, hushengshan, zhaoruizhi, hjin\}@hust.edu.cn}\\ 
{\tt\small qianwang@whu.edu.cn}, 
{\tt\small leo.zhang@griffith.edu.au}, 
{\tt\small jh.hou@cityu.edu.hk}
}


\maketitle
\ificcvfinal\thispagestyle{empty}\fi

%%%%%%%%% ABSTRACT
\begin{abstract}
Self-supervised learning usually uses a large amount of unlabeled data to pre-train an encoder which can be used as a general-purpose feature extractor, such that downstream  users only need to perform fine-tuning operations to enjoy the benefit of ``large model". Despite this promising prospect, the security of pre-trained encoder has not been thoroughly investigated yet, especially when the pre-trained encoder is publicly available for commercial use. 
\vspace{0.2em}
\blfootnote{
\textsuperscript{1}National Engineering Research Center for Big Data Technology and System
\textsuperscript{2}Services Computing Technology and System Lab \
\textsuperscript{3}Hubei Key Laboratory of Distributed System Security \ 
\textsuperscript{4}Hubei Engineering Research Center on Big Data Security \ \textsuperscript{5}Cluster and Grid Computing Lab
}

In this paper, we propose AdvEncoder, the first framework for generating downstream-agnostic universal adversarial examples based on the pre-trained encoder. 
AdvEncoder aims to construct a universal adversarial perturbation or patch for a set of natural images that can fool all the downstream tasks inheriting the victim pre-trained encoder. 
Unlike traditional adversarial example works, the pre-trained encoder only outputs feature vectors rather than classification labels. Therefore, we first exploit the high frequency component information of the image to guide the generation of adversarial examples. 
Then we design a generative attack framework to construct adversarial perturbations/patches by learning the distribution of the attack surrogate dataset to improve their attack success rates and transferability.
Our results show that an attacker can successfully attack downstream tasks without knowing either the pre-training dataset or the downstream dataset.
We also tailor four defenses for pre-trained encoders, the results of which further prove the attack ability of AdvEncoder.
Our codes are available at: \url{https://github.com/CGCL-codes/AdvEnocder}.

\end{abstract}
\vspace{-0.6em}



%%%%%%%%% BODY TEXT


\section{Introduction}
\label{sec:introduction}

The recent surge of Large Language Models (LLMs), such as GPT-3.5/4~\cite{bubeck_sparks_2023}, PaLM~\cite{chowdhery_palm_2022}, FLAN-T5~\cite{chung_scaling_2022}, and Alpaca~\cite{taori_stanford_2023}, has shown a promising trend of large pre-trained models to do a variety of tasks in a zero-shot setting (\ie without any new training data). Example tasks include question answering~\cite{omar2023chatgpt,robinson2023leveraging}, logic reasoning~\cite{wei_chain--thought_2023,zhou_least--most_2023}, machine translation~\cite{brants2007large,gulcehre2017integrating} \etc\ 
A number of experiments have revealed that, built on hundreds of billions of parameters, these LLMs have started to show the capability to understand the human common sense beneath the natural language and do proper reasoning and inference accordingly~\cite{bubeck_sparks_2023,nori_capabilities_2023}.

Among different applications, one particular question yet to be answered is how well LLMs can understand human mental health states through natural language.
Mental health problems represent a significant burden for individuals and societies worldwide.
A recent report suggested that more than 20\% of adults in the U.S. would experience at least one mental disorder in their lifetime~\cite{mental2022state} and 5.6\% of adults experienced a serious psychotic disorder that significantly impairs functioning~\cite{mental2023stats}. The global economy loses around \$1 trillion annually in productivity due to depression and anxiety alone~\cite{mentalcost2023}.

In the past decade, there has been a plethora of research in natural language processing (NLP) and computational social science on detecting mental health issues via online text data such as social media~(\eg \cite{guntuku_detecting_2017,eichstaedt2018facebook,coppersmith_clpsych_2015,de_choudhury_social_2013,de_choudhury_mental_2014}). However, most of these studies have focused on building domain-specific machine learning (ML) models (\ie one model for one particular task, such as stress detection~\cite{nijhawan2022stress,guntuku2019understanding}, depression prediction~\cite{eichstaedt2018facebook,tadesse2019detection,xu_leveraging_2019}, or suicide risk assessment~\cite{de_choudhury_discovering_2016,coppersmith2018natural}). Even for traditional pre-trained language models such as BERT, it needs to be finetuned for specific downstream tasks~\cite{devlin_bert_2019,liu_roberta_2019}.
Since natural language is a major component of mental health assessment and treatment~\cite{sharma2018mental,gkotsis2016language}, LLMs might be a potentially powerful tool to understand end-users' mental states based on the language users' wrote. These instruction-finetuned and general-purpose models can understand a variety of inputs and obviate the need to train multiple models for different tasks. Thus, we can envision using one LLM for a variety of mental-health-related tasks, such as multiple question-answering, reasoning, and inference.
Such a vision opens up a wide range of opportunities for UbiComp, Human-Computer Interaction (HCI), and mental health communities, such as online public health monitoring systems~\cite{patel2018psyheal,graham2019artificial}, intelligent assistants for mental counselors and supporters~\cite{sharma_towards_2021,sharma_humanai_2023}, mental-health-aware personal chatbots~\cite{abd2021perceptions,denecke2020mental}, to just name a few.
However, there is a lack of investigation into understanding, evaluating, and improving the capability of LLMs for mental health prediction tasks.

There are few very recent studies on the evaluation of LLMs (\eg ChatGPT) on mental-health-related tasks, most of which are in zero-shot settings with simple prompt engineering~\cite{yang_evaluations_2023,amin_will_2023,lamichhane_evaluation_2023}. Researchers have shown preliminary results that LLMs have some initial capability of predicting mental health disorders with natural language with some promising but still limited performance compared to state-of-the-art domain-specific NLP models~\cite{yang_evaluations_2023,lamichhane_evaluation_2023}.
This remaining gap is expected since existing general-purpose LLMs are not specifically trained on mental health tasks.
However, to achieve our vision of leveraging LLMs for mental health support and assistance, we need to answer the research question: \textbf{How to empower LLMs with more mental health domain knowledge and become an expert}?

We conducted a series of experiments with multiple LLMs, including Alpaca~\cite{noauthor_stanford_2023}, Alpaca-LoRA~\cite{hu_lora_2021}, and GPT-3.5~\cite{noauthor_introducing_2022}.
Considering the data availability, we focused on online social media data with high-quality human-generated mental health labels.
Our experiments contained three stages: (1) zero-shot prompting, where we experimented with various prompts related to mental health, (2) few-shot prompting, where we inserted examples into prompt inputs, and (3) instruction-finetuning, where we finetuned LLMs on multiple mental-health datasets with various tasks.

Our results indicate that zero-shot obtained promising but limited performance on multiple mental health prediction tasks across all models. GPT-3.5 had relatively better results since it has a larger scale. But their performance is still far from task-specific models. 
Meanwhile, providing a few shots in the prompt can improve the model performance to some extent ($\overline{\Delta}$ = 4.7\%), but the advantage is limited.
Finally and most importantly, we found that instruction-finetuning can significantly improve the model performance across multiple mental-health-related tasks at the same time. Our finetuned Alpaca, namely \textbf{Mental-Alpaca}, significantly outperforms the original GPT-3.5 ($\times$25 times of model size) by an average of 16.7\% on balance accuracy. 
Meanwhile, Mental-Alpaca can further perform on par with the task-specific state-of-the-art Mental-RoBERTa~\cite{ji_mentalbert_2021}. It is noteworthy that Mental-RoBERTa needs to be trained on each task individually, 
while our Mental-Alpaca can solve different tasks off the shelf. 
% We open-source our training code and model at [github link].
Our experiments present the first comprehensive evaluation of various techniques to enhance LLMs' capability in the mental health domain.

The contribution of our paper can be summarized as follows:
\begin{s_enumerate}
\item We present the first comprehensive evaluation of prompt engineering, few-shot, and finetuning techniques on multiple LLMs in the mental health domain.
\item With online social media data, our results reveal that finetuning on a variety of datasets can significantly improve LLM's capability on multiple mental-health-specific tasks simultaneously.
% We release our model \textbf{Mental-Alpaca} as the first open-source LLM targeted at mental health prediction tasks.
\item We provide a few technical guidelines for future researchers and developers on turning LLMs into experts in specific domains.
\end{s_enumerate}

\section{Background and Related Work}
\subsection{Self-supervised Learning}
Self-supervised learning seeks to utilize the oversight signals within the unlabeled data itself to pre-train encoders that can convert complex inputs into generic representations. The pre-trained encoder that learned generally valuable domain knowledge can be used as a universal feature extractor to transfer knowledge to solve different specific downstream tasks. In this paper, we concentrate on image encoders.

Based on~\cite{fini2022self, tao2022exploring},  self-supervised learning schemes can be divided into the following categories: \emph{(1) contrastive learning methods} (\eg, MoCo~\cite{chen2020improved, chen2021empirical}, SimCLR~\cite{chen2020simple}) train representations such that dissimilar negative pairs are widely apart and comparable positive pairs are shown to be near to one another. \emph{(2) negative-free methods} (\eg, BYOL~\cite{grill2020bootstrap}, Sim-Siam~\cite{chen2021exploring}, and ReSSL~\cite{zheng2021ressl}) achieve better representation without the use of negative samples by maintaining the consistency between positive samples and ignoring negative ones. \emph{(3) clustering-based methods} (\eg, SwAV~\cite{caron2020unsupervised}, DeepCluster v2~\cite{caron2020unsupervised}, and DINO~\cite{caron2021emerging}) group  similar samples into the same class using conventional clustering methods. \emph{(4) redundancy reduction-based methods} (\eg, Barlow Twins~\cite{zbontar2021barlow}, W-MSE~\cite{ermolov2021whitening}, VICReg~\cite{bardes2021vicreg}, and VIbCReg~\cite{lee2021vibcreg}) enhance the connection in the same dimension of the representation while attempting to decoupling in distinct dimensions. Concurrently, the use of nearest-neighbor retrieval has been investigated in NNCLR~\cite{dwibedi2021little}. These approaches start from different motivations, design different loss functions, and use different network structures and tricks, which also make them have different defense abilities against adversarial attacks.

\subsection{Attacks on Pre-trained Encoders}

Recently, a growing number of works began to investigate the privacy and security issues of the pre-trained encoders in self-supervised learning. 
Some efforts investigated privacy risks against pre-trained encoders, such as membership inference attacks~\cite{he2021quantifying, liu2021encodermi}, model extraction ~\cite{dziedzic2022difficulty, liu2022stolenencoder,sha2022can}.
At the same time, backdoor attacks and poisoning attacks, the common security threats that usually occur in the training phase, have been shown to be deleterious to pre-trained encoders~\cite{carlini2021poisoning,jia2022badencoder,liu2022poisonedencoder,saha2022backdoor}. 
In contrast, \textit{adversarial examples}, which appear during the testing phase and pose great threat against neural networks, have not been thoroughly investigated yet. 
A concurrent work, PAP~\cite{ban2022pre}, produced a pre-trained perturbation by lifting the feature activations of low-level layers, but the generated adversarial examples lack semantics and rely heavily on the pre-training dataset.
On the contrary, our work aims to achieve effective attacks from the perspective of directly changing the intrinsic texture features of the samples under more demanding conditions that better reflect 
realistic scenarios.


\subsection{Universal Adversarial Examples}
It is well known that deep neural networks are vulnerable to adversarial examples, where an attacker can fool the model by adding minor noise to the image, usually in the form of perturbation~\cite{carlini2017towards, goodfellow2014explaining, hu2022protecting, hu2023pointca} and patch \cite{hu2021advhash, liu2020bias, yang2020design}.
Universal adversarial attack \cite{moosavi2017universal} was proposed to fool the target model by imposing a single adversarial noise vector on all the images.
Existing works can be divided into optimization-based universal adversarial attacks~\cite{moosavi2017universal, mopuri2018generalizable, mopuri2017fast} and generative universal adversarial attacks~\cite{hayes2018learning, mopuri2018nag, mopuri2018ask}.
Compared with  optimization-based solution, generative universal adversarial attacks can generate more generalized and natural-looking adversarial examples by learning the distribution of samples.
% UAP ~\cite{moosavi2017universal}, UAPEPGD~\cite{deng2020universal}, SSP~\cite{naseer2020self}.
However, existing generative universal adversarial attacks in supervised learning can only fool a single model and require the label  information of the model output. 
Since pre-trained encoders can only output the feature vector corresponding to the image, exiting attacks cannot be directly applied  to  the pre-trained encoders, let alone having no knowledge about the downstream tasks.
Some works also proposed different defenses against adversarial examples, such as data pre-processing, adversarial training~\cite{madry2017towards,tramer2019adversarial}, pruning~\cite{ zhu2017prune}, and fine-tuning~\cite{peng2022fingerprinting}. These methods can defend against adversarial samples at different phases.

\section{Methodology}

% Figure environment removed

% This section is for: How we conduct our study to answer these RQs?
% Problem scope declaration: effective fuzz driver generation, ...

\subsection{Overview of Study}

\noindent
\textbf{Overall Design} \tab 
Figure~\ref{fig:overview-for-study} shows the overall workflow of this study.
To understand the effectiveness of different LLM-based fuzz driver generation strategies, we first constructed a quiz, which contains a set of evaluation questions and the effectiveness criteria.
Each question requires the LLMs to generate a fuzz driver according to a given API and the effectiveness of the generated driver will be evaluated based on the criteria.
% An API is qualified if its project is significant, \textit{i.e.}, tested by OSS-Fuzz, and it is the core API inside the existing fuzz drivers.
% The effectiveness criteria of an API are the conditions or checks to distinguish ineffective drivers and effective ones. 
Then we built an evaluation framework to maximize the automation of the evaluation.
% The framework will add general contents of the prompts, launch LLM queries, validate the effectiveness of replies, and classify the failed validation.
Upon these, we designed and evaluated the effectiveness of different query strategies from basic to enhanced (the first two \textbf{RQs}).
% the first two RQs are studied by comparing and analyzing the evaluation results of different query strategies.
In \textbf{RQ1}, we designed and explored the effectiveness of basic strategies which use fundamental API information and have simple interactions with LLMs;
In \textbf{RQ2}, enhanced strategies which leverage extended API usage information and interactive queries are studied.
Necessary in-depth analysis were conducted to explain the benefits and limitations of these strategies.
Lastly, in \textbf{RQ3}, we compared LLM-generated fuzz drivers with OSS-Fuzz drivers to understand their positions, advantages, and disadvantages.
% The comparison involves the static metrics of the code such as the number of unique APIs and dynamic metrics including the testing coverage and unique crashes.

\noindent
\textbf{Evaluated Language Models} \tab 
As shown in Table~\ref{tab:evaluated_models}, two state-of-the-art language models of OpenAI have been studied~\cite{openai-models}.
% All the LLM-generated fuzz drivers evaluated in this study are retrieved via ChatGPT web interfaces~\cite{chatgpt-website} (release version 23 Mar 2023) based on \texttt{chatgpt-wrapper}\cite{chatgpt-wrapper} v0.7.1 (\texttt{dabe72101b}).
All the LLM-generated fuzz drivers evaluated in this study are retrieved via \texttt{chatgpt-wrapper}~\cite{chatgpt-wrapper} v0.7.1 (\texttt{dabe72101b}).
The parameters of the models are kept same as the default values in the ChatGPT official website~\cite{chatgpt-default-config}.
% , \textit{e.g.}, temperature is set as 0.9 and top\_p is 1.

% Figure environment removed

\begin{table}[!t]
\centering
\caption{LLMs Evaluated in This Work.}
\label{tab:evaluated_models}
\resizebox{0.8\linewidth}{!}{
\begin{tabular}{lllll}
\toprule
 Model & Abbr & ChatGPT Version & Max Tokens & Training Data \tabularnewline
\midrule 
\rowcolor{black!10}
gpt3.5-turbo & \texttt{gpt3.5} & 23 Mar 2023 & 4,096 tokens & Up to Sep 2021 \tabularnewline
gpt4 & \texttt{gpt4} & 23 Mar 2023 & 8,192 tokens & Up to Sep 2021 \tabularnewline
\bottomrule
\end{tabular}
}
\end{table}

\subsection{Quiz Construction}
\label{sec:quiz-construction}

\noindent
\textbf{Qualified APIs Collection} \tab 
% What is a question for fuzz driver generation
The question in quiz is designed as generating fuzz drivers for one given API.
This is based on the intuition that any fuzz driver can be divided into one or more simpler fuzz drivers targeting different APIs.
In other words, complicate, multi-purpose fuzz drivers are essentially the combination code which fuzzes multiple API targets simultaneously.
Therefore, as the first practical evaluation on LLM-based fuzz driver generation, we focused on the more fundamental scenario.

% why selecting an appropriate question is not naive
Not all APIs are suitable to be set as questions.
Naively setting all APIs as questions will lead to the creation of meaningless or confusing questions which influences the evaluation result.
Specifically, some APIs, such as \texttt{void libxxx\_init(void)}, are meaningless fuzz targets since the code executed behind the APIs can not be affected by any input data.
Some APIs can only be supplemental APIs rather than the main fuzzing target due to the nature of their functionalities.
For example, given two APIs \texttt{object *parse\_from\_str(char *input)} and \texttt{void free\_object(object *obj)}, feeding the mutated data into \texttt{input} is apparently a better choice than feeding a meaningless pointer to the \texttt{obj} argument.
However, calling the latter API when fuzzing the former is meaningful since 
\ding{182} it may uncover the hidden bug when the former does not correctly initialize the \texttt{object *} by freeing the members of \texttt{object};
\ding{183} it releases the resources allocated in this iteration of fuzzing, which prevents the falsely reported memory leak issue.

% how did we select qualified APIs
To build a high quality quiz, we need to collect a set of qualified APIs which are both representative and suitable to be set as fuzzing targets.
Our intuition is to collect core APIs of the existing fuzz drivers from popular projects fuzzed by OSS-Fuzz.
A driver's core APIs are located based on the following criteria:
\ding{182} they are the target APIs explicitly pointed out by the author in its driver file name or the code comments, \textit{e.g.}, \texttt{dns\_name\_fromwire} API is the core API of driver \textit{dns\_name\_fromwire.c};
\ding{183} otherwise, we pick the basic APIs as the core rather than the supplemental ones.
For example, we pick the former between parse and use/free APIs.
Some fuzz drivers are composite drivers which fuzzes multiple APIs simultaneously, we identified multiple core APIs from them.
Specifically, we randomly selected 30 projects from OSS-Fuzz (commit hash \texttt{135b000926}) C projects, manually extracted 86 core APIs from 51 fuzz drivers.
More details are listed in \appe~\ref{sec:quiz-questions-detail}.

\noindent
\textbf{Effectiveness Criteria Establishment} \tab 
% What is an effective fuzz driver
% general validation method is hard
% can greatly influence the evaluation result
% our approach
% -> semi-automatic criteria: automatic + manually built checkers
% An effective fuzz driver represents the drivers which have correct API usage and produce no false positives.
% Precisely validating the effectiveness of fuzz drivers is crucial for evaluating fuzz driver generation methods.
% However, it is hard to propose a general validation technique since the effectiveness
% general validation techniques do not work well due to the diverse semantics on the API usages.
An effective fuzz driver should use the API effectively while producing no false positives.
Determining the effectiveness of a fuzz driver is important but challengeable since it requires the correct classification of its false positives (bugs caused by the driver code) and negatives (ineffective usage).
To validate precisely, we adopted a semi-automatic validation method.
As shown in Figure~\ref{fig:validation-checker}, it has four checkers.
The first two checks require manual configuration for each project while the rest two need to be configured per API, \textit{a.k.a.} per question in our quiz.
The first check examines the grammatical correctness of a driver using compiler, while the second one checks the existence of abnormal fuzzing behaviours via short-term fuzzing.
It checks whether the driver reports any bugs, \textit{i.e.}, crashes or timeout, or does not have any coverage progress in a short time period with a default fuzzing setup (empty seed, no dictionary, etc).
The intuition behind this is that, under a default setup, neither the zero coverage progress nor the quick identification of bugs are common cases.
In our study, we set the time period as one minute.
Obviously, the second check can make both false positive and negative decisions.
The rest two checks are introduced to refine the check results.
By fuzzing the OSS-Fuzz provided drivers, we collected the signatures of real bugs that can be found quickly.
We filtered these bugs in the third check.
Lastly, based on the examination of the API usage, we summarized semantic constraints that a correct driver should obey on a given API.
These constraints are written as tests to the fuzz drivers.
For example, assuming an API require the mutated input be stored in a file, one semantic test will be hook the API and check whether the filename argument represents a valid file and contains the mutated input.
Our validation framework can taxonomize the failures (Section~\ref{subsec:evaluation-framework}), which can help us iteratively improve the inexact checks when validating more fuzz drivers.

\subsection{Evaluation Framework}
\label{subsec:evaluation-framework}

% what the evaluation framework is and how to use
To boost the evaluation of a large amount of fuzz drivers, we developed a framework to maximize the automation.
As shown in Figure~\ref{fig:overview-for-study}, the framework takes a prompt generated by a query strategy as input, and outputs the classified validation result.
It also includes a website to ease the manual analysis.
In total, the framework is written in 9,342/1,542 lines of Python/HTML, and 3,857 lines of JSON, YAML, and Bash scripts.
Instead of illustrating every detail, we discussed significant design choices as follows.

\noindent
\textbf{Prompt Standardization} \tab 
% controlling the output
% extract code part only
% preparing headers
% What are already assumed to be correctly provided, and how we postprocessing the answers
Unless specifically mentioned in the strategy, a sentence is inserted at the beginning of the prompt to help standardize the output format of LLMs.
The sentence is listed as follows (Python literal), which instructs LLMs to reply in code.
\begin{tcolorbox}[size=title, opacityfill=0.1]
\small  \texttt{// The following is a fuzz driver written in C language, complete the implementation. Output the continued code in reply only.{\textbackslash}n{\textbackslash}n}
\end{tcolorbox}

\noindent
\textbf{LLM Query} \tab 
% token limitation, how many left for query and how many for answer
% perhaps discuss this in iteration section is better
% one query one conversation
% restful deisgn?
LLMs limit the maximum token numbers for the sum of tokens in query and answer.
We set 6,000 tokens for the prompt of \texttt{gpt4}, and 3,600 tokens for \texttt{gpt3.5}.
Prompts exceeding the limit need to be adjusted in strategy.

\noindent
\textbf{Effectiveness Validation} \tab 
Due to the non-deterministic nature of the LLM, some replies still can contain both code and text.
We summarized the patterns and extracted the code in replies.
Besides, to focus on evaluating the effectiveness of the generated code, our framework automatically configures the required header files and build options for a driver.
This is done by manually configuring the rules of header inclusion, and build options for the selected 30 projects, which guarantees all compilation and link errors are caused by the incorrect code rather than the unsuitable configurations.
Besides, each validation of a fuzz driver is in a fresh, isolated container, excluding the environment disturbances to the results.

\noindent
\textbf{Failure Taxonomy} \tab 
% automatic taxonomy
% semi-automatically taxonomy
The taxonomy of root causes for ineffective results also follows a semi-automatic approach.
We first did a rough categorization of the compilation, link, and fuzz errors based on the string patterns of the errors outputted by \texttt{clang}~\cite{clang-lex-diagnostics, clang-parse-diagnostics, clang-sema-diagnostics} and \texttt{libfuzzer}~\cite{libfuzzer, sanitizers}.
Then for each API, we manually identified the root causes per category and mapped each category into the final categories. 
The manual mappings were written in code and will be iteratively improved once we found a new unclassified failure. 

\section{Empirical Evaluation}
\label{sec:exp}

In this section, we first evaluate the accuracy and efficiency of the analytical estimators. Then, we analyze the relationship between local robustness and softmax probability. Lastly, we demonstrate the usefulness of local robustness and its analytical estimators in real-world applications. Key results are discussed in this section and full results are in Appendix~\ref{app:experiments}.

%datasets and models
%\subsection{Datasets and Models}
\textbf{Datasets and Models.}
We evaluate the estimators on four datasets: MNIST \citep{deng2012mnist}, FashionMNIST \citep{xiao2017fashion}, CIFAR10 \citep{krizhevsky2009learning}, and CIFAR100 \citep{krizhevsky2009learning}. For MNIST and FashionMNIST, we train linear models and CNNs to perform classification. For CIFAR10 and CIFAR100, we train ResNet18 \citep{he2016deep} models to perform classification. We train the ResNet18 models using varying levels of gradient norm regularization ($\lambda$) to obtain models with varying levels of robustness. The experiments below use each dataset's full test set, each consisting of 10,000 points. Additional details about the datasets and models are described in Appendix~\ref{app:datasets} and \ref{app:models}.


\subsection{Evaluation of the accuracy of analytical estimators}
\label{sec:exp_correctness}

\textbf{The analytical estimators accurately compute local robustness.}
To confirm that the analytical estimators accurately compute \probust{}, we calculate \probust{} for each model and test set using \pmc{}, \ptaylor{}, \pmmse{}, \ptaylormvs{}, \pmmsemvs{}, and \psoftmax{} for different $\sigma$'s. For \pmc{}, \pmmse{}, and \pmmsemvs{}, we use a sample size at which these estimators have converged ($n=10000, 500, \text{and } 500$, respectively). (Convergence analyses are in Appendix~\ref{app:experiments}.) Then, we measure the absolute and relative difference between \pmc{} and the other estimators. The smaller these differences, the more accurately the estimator computes \probust{}. 

%pmmse family = best estimator
The performance of the estimators for the FashionMNIST CNN model is shown in Figure~\ref{fig1a:method-works-over-sigma}. The results indicate that \pmmsemvs{} and \pmmse{} are the best estimators of \probust{}, followed closely by \ptaylormvs{} and \ptaylor{}, trailed by \psoftmax{}. Consistent with the theory in Section~\ref{sec:methods}, the MMSE estimators outperform the Taylor ones because the former obtains better estimates of $\grad g_i(\X)$, and \psoftmax{} performs poorly in general settings because of its multiple levels of approximation.

%smaller noise neighborhood, better approximation
The results also confirm that the smaller the noise neighborhood $\sigma$, the more accurately the estimators compute \probust{}. For the MMSE and Taylor estimators, this is because their linear approximation of the model around the input is more faithful for smaller $\sigma$'s. As expected, when the model is linear, \ptaylor{} and \pmmse{} accurately compute \probust{} for all $\sigma$'s (Appendix~\ref{app:experiments}). For the softmax estimator, \psoftmax{} values are constant over $\sigma$'s and this particular model has high \psoftmax{} values for most points. Thus, for small $\sigma$'s where \probust{} is near one, \psoftmax{} happens to approximate \probust{} for this model. Examples of images with varying levels of noise ($\sigma$) are in Appendix~\ref{app:experiments}.

\textbf{For robust models, the analytical estimators compute local robustness more accurately over a larger noise neighborhood.} 
The performance of \pmmse{} for CIFAR10 ResNet18 models of varying levels of robustness is shown in Figure~\ref{fig1b:method-works-robust}. The results indicate that for more robust models (larger $\lambda$), the estimator is more accurate over a larger $\sigma$. This is because gradient norm regularization leads to models that are more locally linear, making the estimator's linear approximation of the model around the input more accurate over a larger $\sigma$, making its \probust{} values more accurate.


\textbf{The mv-sigmoid function approximates the multivariate Normal CDF well in practice.} To examine \emph{mv-sigmoid}'s approximation of \emph{mvn-cdf}, we compute both functions using the same inputs ($z~=~\left[  \frac{g_1(\X)}{\sigma \|\grad g_1(\X)\|_2}, ..., \frac{g_C(\X)}{\sigma \|\grad g_C(\X) \|_2} \right]$, as described in Proposition~\ref{eqn:taylor-estimator}) for the CIFAR10 ResNet18 model for different $\sigma$'s. The plot of \emph{mv-sigmoid(z)} against \emph{mvn-cdf(z)} for $\sigma=0.05$ is shown in Figure~\ref{fig2:mvsig-mvncdf}. The results indicate that the two functions are strongly positively correlated with low approximation error, suggesting that \emph{mv-sigmoid} approximates the \emph{mvn-cdf} well in practice.

% \clearpage

%fig1 -- method properly approximates p_empirical
%fig1a: method works
%02d_pemp_vs_pothers_over_sigma/rel/fmnist_cnn.png
%fig1b: method works better for robust models
%02e_pemp_vs_pmmse_over_sigma_robust_models/rel/cifar10_resnet18.png
% Figure environment removed


%fig2: mvsigmoid is a good approximator for mvncdf
%correlation for sigma=0.1 
% 03_mvncdf_vs_mvsigmoid/cifar10_resnet18/cifar10_resnet18_gnorm0.0_sigma0.05.png

%fig3: p_mc takes many samples to converge
% 02a_p_emp_convergence_n50000_baseline/rel/cifar10_resnet18_sigma0.1.png
% Figure environment removed


%table: naive method is inefficient, analytical method is efficient
\begin{table}[ht!]
\centering
\begin{tabular}{l|l|l|l|l|l}
    \multicolumn{2}{c}{}   & \multicolumn{2}{|c|}{CPU: Intel x86\_64}   & \multicolumn{2}{|c}{GPU: Tesla V100-PCIE-32GB} \\
    \midrule
    Estimator   & \# samples ($n$)   & Serial   & Batched   & Serial   & Batched \\
    \midrule
    \pmc{}   & \begin{tabular}[c]{@{}l@{}}  $n=10000$\end{tabular}               
             & \begin{tabular}[c]{@{}l@{}}  1:41:11\end{tabular}                                               
             & \begin{tabular}[c]{@{}l@{}}  1:14:38\end{tabular}                                                
             & \begin{tabular}[c]{@{}l@{}}  0:19:56\end{tabular}                                                
             & \begin{tabular}[c]{@{}l@{}}  0:00:35\end{tabular} \\
    \ptaylor{}   & N/A
                 & 0:00:08                                                                                                                     
                 & 0:00:07                                                                                                                      
                 & 0:00:02                                                                                                                      
                 & $<$ 0:00:01 \\
    \pmmse{}   & \begin{tabular}[c]{@{}l@{}} $n=5$\end{tabular} 
               & \begin{tabular}[c]{@{}l@{}} 0:00:41\end{tabular} 
               & \begin{tabular}[c]{@{}l@{}} 0:00:31\end{tabular} 
               & \begin{tabular}[c]{@{}l@{}} 0:00:06\end{tabular} 
               & \begin{tabular}[c]{@{}l@{}} 0:00:02\end{tabular} \\              
\end{tabular}
\vspace{0.2cm}
\caption{Runtimes of \probust{} estimators. Each estimator computes \probustwsigma{0.1} for the CIFAR10 ResNet18 model for 50 data points. Estimators that use sampling use the minimum number of samples necessary for convergence. Runtimes are in the format of hour:minute:second. The analytical estimators (\ptaylor{} and \pmmse{}) are more efficient than the naïve estimator (\pmc{}).}
\vspace{-0.5cm}
\label{table:runtimes}
\end{table}



%fig4 -- p_robust and p_softmax
%fig4a: scatterplot, non-robust model
%02i_pemp_vs_pmmse_corr_robust_models_scatterplots/cifar10_resnet18_sigma0.1_gnormreg0.png
%fig4b: the more robust the model, the more the two are related
%02h_pemp_vs_pmmse_corr_robust_models_lineplots/cifar10_resnet18_cifar100_resnet18_sigma0.1.png
%fig4c: scatterplot, robust model
%02i_pemp_vs_pmmse_corr_robust_models_scatterplots/cifar10_resnet18_sigma0.1_gnormreg0.01.png
% Figure environment removed

%fig5: local robustness bias
% 02f_p_distr_vs_classes/p_all_over_classes/cifar10_resnet18_sigma0.09.png
\begin{SCfigure}
  \vspace{1cm}
  \centering
  % Figure removed
  \vspace{-2.5cm}
  \caption{Local robustness bias among classes for the ResNet18 CIFAR10 model. \probust{} reveals that the model is less locally robust for some classes than for others. The analytical estimator \pmmse{} properly captures this model bias.}
  \label{fig5:robustness-bias}
\end{SCfigure}


%fig4 -- 2x4 images
%top-k and bottom-k images
% 02g_topk_bottomk_images/cifar10_resnet18/p_mmse/
% - cifar10_resnet18_p_mmse_sigma0.1_class9_bottomk.png
% - cifar10_resnet18_p_mmse_sigma0.1_class9_topk.png
% - ... class0 x 2
% 02g_topk_bottomk_images/cifar10_resnet18/p_sm
% - cifar10_resnet18_p_sm_sigma0.1_class9_bottomk.png
% - cifar10_resnet18_p_sm_sigma0.1_class9_topk.png
% - ... class0 x 2
% Figure environment removed



% %fig4 -- 3x4 images
% %top-k and bottom-k images
% % 02g_topk_bottomk_p/resnet18_cifar10/p_mmse/
% % - resnet18_cifar10_p_mmse_sigma0.1_class9_topk.png
% % - resnet18_cifar10_p_mmse_sigma0.1_class9_bottomk.png
% % - ...class8... x 2
% % - ...class0... x 2
% % 02g_topk_bottomk_p/resnet18_cifar10/p_sm/
% % - resnet18_cifar10_p_sm_sigma0.1_class9_topk.png
% % - resnet18_cifar10_p_sm_sigma0.1_class9_bottomk.png
% % Figure environment removed







\subsection{Evaluation of the efficiency of analytical estimators}

\textbf{The naïve estimator is statistically inefficient.} To examine the efficiency of \pmc{}, we calculate \pmc{} for each model and test set using different sample sizes ($n$) over different $\sigma$'s, and measure the absolute and relative difference between \pmc{} at a given $n$ and \pmc{} at $n=50,000$. Results for the CIFAR10 ResNet18 model are shown in Figure~\ref{fig3:pmc-convergence}. The results indicate that \pmc{} requires around 10,000 samples per point to converge, which is impractical.

\textbf{The analytical estimators are more efficient than the naïve estimator.}
Next, we examine the efficiency of the estimators by measuring their runtimes when calculating \probustwsigma{0.1} for the CIFAR10 ResNet18 model for 50 points. Runtimes are displayed in Table~\ref{table:runtimes}. They indicate that \ptaylor{} and \pmmse{} perform 35x and 17x faster than \pmc{}, respectively. Additional runtimes are in Appendix~\ref{app:experiments}.
% Thus, the analytical estimators are more efficient than the naïve estimator.

\subsection{Comparison of local robustness and softmax probability}

\textbf{Local robustness and softmax probability are two distinct measures.} To examine the relationship between \probust{} and \psoftmax{}, we calculate \pmmse{} and \psoftmax{} for CIFAR10 and CIFAR100 models of varying levels of robustness, and measure the correlation of their values and ranks using Pearson and Spearman correlations. Results are in Figure~\ref{fig4:probust-and-psoftmax}. For a non-robust model, \probust{} and \psoftmax{} are not strongly correlated (Figure~\ref{fig4a:ps-nonrob-model}). As model robustness increases, the two quantities become more correlated (Figures~\ref{fig4b:ps-rob-models-lineplot} and~\ref{fig4c:ps-rob-model}). However, even for robust models, the relationship between the two quantities is mild (Figure~\ref{fig4c:ps-rob-model}). That \probust{} and \psoftmax{} are not strongly correlated is consistent with the theory in Section~\ref{sec:methods}: in general settings, \psoftmax{} is not a good estimator for \probust{}.

% that the two probabilities are conceptually different: \probust{} measures the uncertainty of a model’s prediction with respect to input noise (i.e., the probability that the prediction will change upon adding noise to the input) while \psoftmax{} is an uncalibrated uncertainty of the model's prediction being correct with respect to a calibration set. \textcolor{red}{[check interp of raw \psoftmax]}.

\subsection{Applications of local robustness}

\textbf{\boldmath \probust{} detects local robustness bias.} We demonstrate that \probust{} can detect bias in local robustness by examining its distribution for each class for each model and test set over different $\sigma$'s. Results for the CIFAR10 ResNet18 model are in plotted in Figure~\ref{fig5:robustness-bias}. The results show that different classes have different \probust{} distributions, i.e., the model is more locally robust for some classes (e.g., frog) than for others (e.g., airplane). The results also show that \pmc{} and \pmmse{} have very similar distributions, further indicating that the latter well-approximates the former. Thus, \probust{} can be applied to detect local robustness bias, which is critical when models are deployed in high-stakes, real-world settings.

\textbf{\boldmath \probust{} identifies images that are robust to and images that are vulnerable to random noise.} We demonstrate that \probust{} can also distinguish between images that are robust to and images that are vulnerable to random noise in a way that is superior to \psoftmax{}. For each dataset, we train a simple CNN to distinguish between images with high and low \pmmse{} and the same CNN to also distinguish between images with high and low \psoftmax{} (additional setup details described in Appendix~\ref{app:experiments}). Then, we compare the performance of the two models. For CIFAR10, the test set accuracy for the \pmmse{} CNN is 0.92 while that for the \psoftmax{} CNN is 0.58. These results indicate that \probust{} better identifies images that are robust to and vulnerable to random noise than \psoftmax{}.

We also visualize images with the highest and lowest \pmmse{} in each class for each model. For comparison, we do the same with \psoftmax{}. Example CIFAR10 images are displayed in Figure~\ref{fig6:topk-vs-bottomk}. Images with low \probust{} tend to have neutral colors, with the object being a similar color as the background (making the prediction likely to change when the image is slightly perturbed), while images with high \probust{} tend to be brightly-colored, with the object strongly contrasting with the background (making the prediction likely to stay constant when the image is slightly perturbed). These differences are not as evident for images with the highest and lowest \psoftmax{}. Thus, in addition to detecting local robustness bias, \probust{} can also be applied to identify images that are robust to and images that are vulnerable to random noise.



% For all the experiments above, we observe consistent results across datasets and models (Appendix).

% --- OLD STUFF BELOW ---


%\ptaylormvs{} and \pmmsemvs{} perform better than \ptaylor{} and \pmmse{}, respectively, because... \textcolor{blue}{[How to explain why \pmmsemvs and \ptaylormvs perform better than \pmmse and \ptaylor?]} 


% %exp1
% \textbf{Experiment 1. As the noise neighborhood increases, local robustness deteriorates.}
% First, we examine the behavior of \probust{} as the noise neighborhood increases. For a given model, we calculate \pmc{} for different values of $\sigma$ for 1,000 randomly-selected points from the test set (hereafter referred to as the “test set”). To calculate \pmc{} for a given image, we use 10,000 noisy samples (a value at which \pmc{} converged; convergence analyses are described in Appendix~\ref{app:exp-convergence-pmc-pmmse}).

% Results for the linear model and CNN trained on FashionMNIST are shown in Figure~\ref{fig1:pmc-vs-noise}. In the figure, the distribution of \pmc{} is concentrated at one for small values of $\sigma$ and increasingly shifts towards zero as $\sigma$ increases. Thus, as expected, the results indicate that the models are locally robust for small noise neighborhoods, and as the noise neighborhood increases, local robustness deteriorates. This is expected because as more noise is added to the original image, it becomes more likely for the prediction of the noisy image to differ from that of the original image, causing local robustness to deteriorate (i.e., causing \probust{} to decrease). We observe similar results across datasets and models (Appendix~\ref{app:exp-pmc-vs-noise}).


% %exp3
% \textbf{Experiment 3. \boldmath \probust{} and \boldmath \psoftmax{} measure different types of uncertainty.}
% Next, we show that \probust{} and \psoftmax{} measure different types of uncertainty. For a given model, test set, and noise neighborhood ($\sigma$), we calculate \pmmse{} (a close estimate of \probust{}) and \psoftmax{} and examine their relationship. 

% Results for the ResNet18 model trained on CIFAR10 at $\sigma=0.1$ are shown in Figure~\ref{fig3:correlation-pmmse-psm}. The results indicate that \probust{} and \psoftmax{} are not strongly correlated: points which have high \psoftmax{} values have high or low \probust{} values, and points which have low \probust values have high or low \psoftmax{} values. This finding is corroborated by the low Pearson correlation and Spearman correlation coefficients, indicating that the values and ranks, respectively, of \probust{} and \psoftmax{} are not strongly correlated. We observe similar results across datasets and models (Appendix~\ref{app:exp-correlation-pmmse-psm}).

% That \probust{} and \psoftmax{} are not strongly correlated is consistent with the understanding that the two probabilities measure different types of uncertainty: \probust{} measures the uncertainty of a model’s prediction with respect to input noise (i.e., the probability that the prediction will change upon adding noise to the input) while \psoftmax{}... \textcolor{blue}{[what is the interpretation of raw \psoftmax, if any?]}.


% %exp4

% \textbf{Experiment 4. \boldmath \probust{} can be used to detect differences in local robustness among classes.}
% Next, we demonstrate that \probust{} can be used to detect bias in local robustness among classes. We examine two setups. First, for a given model, test dataset, and noise neighborhood ($\sigma$), we calculate \probust{} using \pmmse{} and examine the distribution of \probust{} for each class. This setup examines local robustness based on confidence level for fixed $\sigma$. Second, for a given model, test dataset, and confidence level, we calculate the noise neighborhood that meets the specified confidence level. To do so, we optimize $\sigma$ such that \ptaylor{} equals the specified confidence level (we use \ptaylor{} because it is easier to optimize than \pmmse{}; however, the same idea applies to \pmmse{}). This setup examines local robustness based on $\sigma$ for a fixed confidence level.

% Results for both setups for the ResNet18 model trained on CIFAR10 are shown in Figure~\ref{fig4:robustness-bias}. Different classes have different confidence level distributions when $\sigma$ is fixed (Figure~\ref{fig4a:sigma-fixed}) and different $\sigma$ distributions when confidence level is fixed (Figure~\ref{fig4b:confidence-fixed}), indicating that the model is more locally robust for some classes than for others. We observe similar results across datasets and models (Appendix~\ref{app:exp-robustness-bias}).

% ADD LATER: We also examine local robustness bias among classes by fixing the same confidence level for each point and calculating the $\sigma$ that yields that confidence level, and we obtain consistent results indicating that models are not equally robust for all classes.

% %exp5
% \textbf{Experiment 5. \probust{} can distinguish between clear and ambiguous images.}
% Lastly, we demonstrate that \probust{} can distinguish between clear and ambiguous images. For a given model, test set, and $\sigma$, we calculate \probust{} (using \pmmse{}) and \psoftmax{}. Then, for each class, we measure the difference between images with the highest and lowest \probust{} and the difference between images with the highest and lowest \psoftmax{} by… \textcolor{blue}{[need to do this experiment, see if we get desired result]}.

% Results for the ResNet18 model trained on CIFAR10 are shown in Figure~\ref{fig5:topk-vs-bottomk}. As seen in Figure~\ref{fig5a:differences}, differences between the top and bottom images based on \probust{} are larger than those based on \psoftmax{} \textcolor{blue}{[currently wrote desired results, fill in real results later]}. Visual inspection of the images, such as in Figures~\ref{fig5b:bottomk-probust}-\ref{fig5e:topk-psoftmax}, suggests that top and bottom images based on \probust{} tend to be clear and ambiguous images for each class, respectively, while this distinction is not evident for top and bottom images based on \psoftmax{}. We observe similar results across datasets and models (Appendix~\ref{app:exp-topk-vs-bottomk}). Taken together, these results indicate that \probust{} better distinguishes between clear and ambiguous images than \psoftmax{}, suggesting that \probust{} better reflects image differences in the latent feature space \textcolor{blue}{[check statement after the comma]}. 

\section{Conclusion}

This paper investigated the $(\bm{\alpha}, \bm{\beta})$-proportionally fair normalized cut graph partitioning problem.
We proposed a novel algorithm, FNM, consisting of an extended spectral embedding method and a $k$-means-based rounding scheme to provide a node partitioning with a small Ncut value on a graph while strictly following the proportional fairness constraints.
The comprehensive experimental findings confirmed the superior performance of FNM in terms of partition quality, fairness, and efficiency.
In future work, we will generalize our algorithm to handle other notions of fairness, e.g., individual fairness~\cite{MahabadiV20, gupta2022consistency}, in graph partitioning problems.



{\small
\bibliographystyle{ieee_fullname}
\bibliography{egbib}
}

\appendix
\section{Appendix Contents}
\setcounter{table}{0}
\setcounter{figure}{0}
\renewcommand{\thetable}{A\arabic{table}}
\renewcommand{\thefigure}{A\arabic{figure}}

\begin{itemize}

\item Sec.~\ref{dataset}: 
Details of the datasets. 
% We provide details of the datasets and an explanation of how to use them in the main body.
\item Sec.~\ref{optimization}: 
Optimization procedure of AdvEncoder. 
We describe the optimization of our attack in detail.

\item Sec.~\ref{attack_performance}: 
Supplemental results of AdvEncoder's attack performance on \textbf{Image Classification \& Retrieval}, \textbf{Object Detection}, and  \textbf{Semantic Segmentation} tasks.


\item Sec.~\ref{ablation}: 
Supplemental ablation study about the effect of different backbones and random seeds on AdvEncoder.

\item Sec.~\ref{transferability}: 
Supplemental transferability results.


\item Sec.~\ref{Visualization}: 
Visualization of perturbations/patches generated by AdvEncoder.
\end{itemize} 



\section{Datasets}\label{dataset}

Our experiments are based on the following four datasets: CIFAR10~\cite{krizhevsky2009learning}, STL10~\cite{coates2011analysis}, GTSRB~\cite{stallkamp2012man}, and ImageNet~\cite{russakovsky2015imagenet}. 
Specifically, we use the above four datasets as the attacker’s surrogate datasets and the downstream datasets, respectively.
Following~\cite{jia2022badencoder}, we resize all examples to 64x64x3. The details of the datasets are as follows:

\noindent\textbf{CIFAR10.} CIFAR10 contains 50,000 training images and 10,000 testing images. Each image has a size of 32×32×3 and belongs to one of 10 classes. 

\noindent\textbf{STL10.} STL10 contains 5,000 labeled training images and 8,000 labeled testing images, each of which has a size of 96×96×3. Moreover, the dataset contains 10 classes and each image belongs to one of them.  

\noindent\textbf{GTSRB.} GTSRB contains 51,800 traffic sign images in 43 categories. Each image has a size of 32×32×3. The dataset is divided into 39,200 training images and 12,600 testing images. 

\noindent\textbf{ImageNet.} ImageNet contains $1.2M$ training samples and $50,000$ testing samples with $1000$ classes. Each image has a size of 256×256×3. We randomly select $100$ classes from ImageNet to build our dataset. 


\section{Optimization}\label{optimization}
In this section, we describe the optimization process of AdvEncoder in detail.
Given a random noise $z$, the generator generates a universal adversarial noise $\delta$ of the same size as the input image. 
We need to first crop the universal adversarial noise $\delta$ to within the imperceptibility constraint $\epsilon$. Then the universal adversarial noise can be converted into two forms of adversarial perturbation and adversarial patch by using \cref{eq:7} and \cref{eq:8}, respectively. 
By minimizing the objective function mentioned in \cref{eq:3}, we can optimize the generator to generate more generalized and transferable universal adversarial  perturbations or patches.
The whole optimization process is outlined in Algorithm.~\ref{optimization_advencoder}. 




% Table generated by Excel2LaTeX from sheet 'Sheet1'
\begin{table}[htbp]
\setlength{\abovecaptionskip}{2pt}
  \centering
   \caption{The clean retrieval accuracy (\%) of different downstream models based on pre-trained encoders over two datasets. PD denotes the pre-training dataset and DD represents the downstream dataset.}
      \scalebox{0.6375}{
    \begin{tabular}{c|c|c|c|c|c|c|c|c}
    \toprule[1.5pt]
    PD & DD & Model & top1  & top5  & top10 & top20 & top50 & top100 \\
    \hline
    \multirow{12}[4]{*}{CIFAR10} & \multirow{6}[2]{*}{STL10} & Barlow & 99.33  & 93.31  & 80.22  & 60.36  & 36.86  & 24.67 \\
          &       & BYOL  & 99.00  & 93.32  & 81.33  & 63.17  & 36.82  & 25.25 \\
          &       & DINO  & 100.00  & 93.62  & 82.79  & 64.45  & 38.72  & 25.67 \\
          &       & MoCo2+  & 99.67  & 93.74  & 82.74  & 63.56  & 37.21  & 25.4 \\
          &       & NNCLR & 100.00  & 94.21  & 82.83  & 63.99  & 36.34  & 25.01 \\
          &       & SimCLR & 99.67  & 94.45  & 80.55  & 62.16  & 38.06  & 25.28 \\
\cline{2-9}          & \multirow{6}[2]{*}{GTSRB} & Barlow & 96.67  & 97.43  & 96.39  & 94.52  & 92.38  & 90.42 \\
          &       & BYOL  & 95.33  & 94.65  & 92.24  & 89.66  & 85.99  & 82.54 \\
          &       & DINO  & 94.33  & 94.86  & 92.80  & 89.51  & 84.63  & 80.41 \\
          &       & MoCo2+  & 85.00  & 87.19  & 86.17  & 84.45  & 81.72  & 78.62 \\
          &       & NNCLR & 90.33  & 91.43  & 89.58  & 87.57  & 82.85  & 79.92 \\
          &       & SimCLR & 94.00  & 94.19  & 92.30  & 90.58  & 87.73  & 85.43 \\
    \hline
    \multirow{12}[4]{*}{ImageNet} & \multirow{6}[2]{*}{STL10} & Barlow & 99.67  & 94.07  & 81.03  & 62.26  & 36.67  & 25.02 \\
          &       & BYOL  & 93.33  & 90.53  & 79.02  & 61.63  & 37.58  & 24.92 \\
          &       & DINO  & 94.00  & 91.38  & 79.58  & 61.00  & 36.33  & 24.6 \\
          &       & MoCo2+  & 65.67  & 70.93  & 63.24  & 49.50  & 32.28  & 23.38 \\
          &       & NNCLR & 86.67  & 86.63  & 76.15  & 59.27  & 35.72  & 24.67 \\
          &       & SimCLR & 83.33  & 84.48  & 73.58  & 56.99  & 35.11  & 24.34 \\
\cline{2-9}          & \multirow{6}[2]{*}{GTSRB} & Barlow & 97.67  & 97.02  & 93.92  & 91.14  & 87.06  & 83.85 \\
          &       & BYOL  & 83.33  & 87.52  & 85.68  & 82.21  & 77.18  & 74.78 \\
          &       & DINO  & 87.33  & 90.45  & 88.49  & 86.14  & 83.17  & 80.59 \\
          &       & MoCo2+  & 82.00  & 85.24  & 84.37  & 82.06  & 78.10  & 75.37 \\
          &       & NNCLR & 77.67  & 85.13  & 83.83  & 81.75  & 78.23  & 75.37 \\
          &       & SimCLR & 90.67  & 91.81  & 88.53  & 84.78  & 80.20  & 77.26 \\
    \bottomrule[1.5pt]
    \end{tabular}%
    }
  \label{tab:clean_retrieval}%
     \vspace{-4mm}
\end{table}%


% Table generated by Excel2LaTeX from sheet '补充材料'
\begin{table*}[htbp]
\setlength{\abovecaptionskip}{2pt}
  \centering
  \caption{The clean accuracy (\%) of different downstream models based on pre-trained encoders over four datasets. PD denotes the pre-training dataset and DD represents the downstream dataset.}
  \scalebox{0.6855}{
    \begin{tabular}{c|c|cccccccccccccc}
    \toprule[1.5pt]
    PD & DD & Barlow & BYOL  & DeepC2 & DINO  & MoCo2+ & MoCo3 & NNCLR & ReSSL & SimCLR & SupCon & SwAV  & VIbCReg & VICReg & W-MSE \\
    \hline
    \multirow{4}[2]{*}{CIFAR10} & CIFAR10 & 93.61  & 94.47  & 90.59  & 91.60  & 95.10  & 94.67  & 93.91  & 92.67  & 93.63  & 96.23  & 92.32  & 92.35  & 93.17  & 90.13  \\
        & STL10 & 83.98  & 83.18  & 77.99  & 83.30  & 84.61  & 83.92  & 83.26  & 81.65  & 81.64  & 84.47  & 81.32  & 81.97  & 82.18  & 78.26  \\
          & GTSRB & 97.57  & 93.66  & 95.99  & 97.04  & 91.90  & 92.09  & 96.28  & 95.97  & 96.59  & 97.39  & 97.77  & 96.96  & 97.11  & 88.63  \\
          & ImageNet & 67.58  & 52.94  & 45.44  & 60.16  & 51.15  & 48.92  & 57.00  & 51.01  & 54.23  & 50.67  & 58.82  & 61.00  & 61.83  & 41.00  \\
    \hline
    \multirow{4}[2]{*}{ImageNet} & CIFAR10 & 72.31  & 72.44  & 69.06  & 73.14  & 73.12  & 72.86  & 73.89  & 72.22  & 69.51  & 70.71  & 71.48  & 72.90  & 73.21  & 68.64  \\
          & STL10 & 65.09  & 64.36  & 61.77  & 65.16  & 65.23  & 63.10  & 65.95  & 65.04  & 62.49  & 63.31  & 64.05  & 65.23  & 64.70  & 61.17  \\
          & GTSRB & 95.49  & 92.18  & 95.16  & 92.28  & 90.28  & 87.88  & 93.93  & 90.82  & 93.71  & 95.81  & 96.18  & 95.13  & 94.51  & 92.66  \\
          & ImageNet & 56.87  & 46.10  & 46.04  & 49.21  & 42.84  & 43.26  & 49.15  & 45.19  & 48.79  & 52.44  & 48.83  & 55.50  & 52.29  & 38.13  \\
    \bottomrule[1.5pt]
    \end{tabular}%
    }
  \label{tab:clean}%
\end{table*}%


\begin{algorithm}[H]
    \caption{Frequency-based Generative Attack Framework}
    \label{optimization_advencoder}

    \begin{algorithmic}[1] % 此处的[1]控制一下算法中的每句前面都有标号 
        \REQUIRE Attacker’s surrogate data point $ x \in \mathcal{D}_{a}$, a pre-trained encoder $g_{\theta}$, max-perturbation constraint $\epsilon$, a fixed noise $z$, adversarial generator parameters  $\theta_{\mathcal{G}}$, hyper-parameters $\alpha$, $\beta$, $\lambda$, temperature parameter $\tau$.
        \ENSURE a universal adversarial noise $\delta$. 
        \STATE {Initiate learning rate $\eta$, batch size $n$.}
        \STATE {Sample a vector $z$ from $\mathcal{N}(0, 1)^{100}$}
        % \STATE {$\delta \longleftarrow \mathcal{G}(z)$}
        % \STATE {Clip $\delta$ to satisfy imperceptibility constraint $\epsilon$}
        % \FOR {$ i = 1 $; $ i < n $; $ i ++ $ }
        %     \STATE {Step C}
        % \ENDFOR
        \WHILE {max iterations or not  converge}
            \STATE {$\delta \longleftarrow \mathcal{G}(z)$}
            \STATE {Clip $\delta$ to satisfy imperceptibility constraint $\epsilon$}
            \IF {Adv-PER}
            \STATE {$x^{adv} \longleftarrow x + \delta$}
            \ENDIF
             \IF {Adv-PAT}
            \STATE {$x^{adv} \longleftarrow  x  \odot (1- m) + \delta  \odot m$}
            \ENDIF
            % \STATE {$x^{adv} \longleftarrow x + \delta$ through Eq.7}
            \STATE {Calculate loss mentioned in \cref{eq:3}} 
            % \STATE {Calculate the Euclidean distance between the HFC of $x^{adv}$ and $x$ through Eq.~(\ref{eq:7})}
            % \STATE {Calculate the adv loss through Eq.~(\ref{eq:6})}
            \STATE {Update $\mathcal{G}$ through backprop}
        \ENDWHILE
        
        % \FORALL {...}
        %     \STATE {Step E}
        % \ENDFOR
        % \IF {condition}
        %     \STATE {Step F}
        % \ENDIF
    \end{algorithmic} 
\end{algorithm}

% \begin{algorithm}[H]
%     \caption{algorithm 1}
%     \begin{algorithmic}[1] % 此处的[1]控制一下算法中的每句前面都有标号 
%         \REQUIRE input
%         \ENSURE output 
%         \STATE {Step A}
%         \STATE {Step B}
%         \FOR {$ i = 1 $; $ i < n $; $ i ++ $ }
%             \STATE {Step C}
%         \ENDFOR
%         \WHILE {$ |E_n| \leq L_1 $}
%             \STATE {Step D}
%         \ENDWHILE
%         \FORALL {...}
%             \STATE {Step E}
%         \ENDFOR
%         \IF {condition}
%             \STATE {Step F}
%         \ENDIF
%     \end{algorithmic} 
% \end{algorithm}
% \vspace{-0.2cm}
\vspace{-4mm}

\section{Supplemental Attack Performance}\label{attack_performance}
In this section, we further investigate the attack performance of AdvEncoder in two types of downstream tasks, classification and retrieval.
We keep all the following experimental settings consistent with the main body.
% For the classification task,



\subsection{Attack Performance on Classification} \label{attack_performance_classification}
We first provide the normal accuracy of downstream models based on fourteen SSL pre-trained encodes on four different datasets. 
The results in \cref{tab:clean} prove that users using pre-trained encoders can achieve excellent performance on different datasets just by fine-tuning the linear layer.
Specifically, we evaluate AdvEncoder on fourteen victim pre-trained encoders over four downstream tasks using two attacker's surrogate datasets, \textbf{STL10} and \textbf{GTSRB}, respectively.
The performance of our attack using two additional attacker's surrogate datasets on the classification task further confirms that downstream tasks based on pre-trained encoders are exposed to significant security risks. The experimental results  in \cref{tab:attack_performance_per1} and \cref{tab:attack_performance_pat1} demonstrate that an attacker can achieve successful attacks even without prior knowledge of the pre-training dataset and the downstream dataset. 
These findings are in line with the results presented in the main body of the paper.


% Table generated by Excel2LaTeX from sheet '补充材料'
\begin{table*}[htbp]
\setlength{\abovecaptionskip}{2pt}
  \centering
  \caption{The attack success rate (\%) of Adv-PER under different settings.
  $\mathcal{S}_{1}$ - $\mathcal{S}_{4}$ denote the settings where the downstream datasets are CIFAR10, STL10, GTSRB, ImageNet, respectively, and all the attacker’s surrogate dataset is \textbf{STL10}. $\mathcal{S}_{5}$ -  $\mathcal{S}_{8}$ use \textbf{GTSRB} as the attacker’s surrogate dataset, with the downstream datasets remained the same as $\mathcal{S}_{1}$ - $\mathcal{S}_{4}$. Barlow Twins and DeepCluster v2 are abbreviated as Barlow and DeepC2, respectively.}
   \scalebox{0.7}{
    \begin{tabular}{c|c|cccccccccccccc}
    \toprule[1.5pt]
    Dataset & Setting & Barlow & BYOL  & DeepC2 & DINO  & MoCo2+ & MoCo3 & NNCLR & ReSSL & SimCLR & SupCon & SwAV  & VIbCReg & VICReg & W-MSE \\
    \hline
    \multirow{9}[2]{*}{CIFAR10} & $\mathcal{S}_{1}$    & 89.34  & 88.36  & 86.17  & 89.85  & 78.39  & 87.49  & 90.95  & 88.79  & 70.10 & 90.21  & 51.27  & 89.02  & 81.11  & 62.29  \\
          & $\mathcal{S}_{2}$    & 53.58  & 72.57  & 71.96  & 70.74  & 37.07  & 60.83  & 72.16  & 63.67  & 29.11  & 83.02  & \textcolor[RGB]{169,169,169}{28.52}  & 72.33  & 44.42  & 35.68  \\
          & $\mathcal{S}_{3}$    & \textbf{91.92}  & \textbf{92.05}  & \textbf{89.50}  & 92.96  & \textbf{83.93}  & 82.08  & 91.28  & \textbf{94.46}  & \textbf{70.72}  & 91.99  & 66.02  & 90.76  & \textbf{82.97}  & \textbf{72.55}  \\
          & $\mathcal{S}_{4}$    & 88.09  & 88.50  & 85.74  & 86.43  & 82.08  & \textbf{88.25}  & 87.55  & 85.76  & 68.86  & \textbf{94.48}  & 61.45  & 90.31  & 75.44  & 72.22  \\
          & $\mathcal{S}_{5}$    & 87.81  & 88.85  & 82.75  & 89.71  & 51.37  & 55.67  & 89.22  & 86.83  & 45.38  & 83.38  & 61.40  & 85.78  & 73.91  & 52.28  \\
          & $\mathcal{S}_{6}$    & \textcolor[RGB]{169,169,169}{50.83}  & \textcolor[RGB]{169,169,169}{67.82}  & \textcolor[RGB]{169,169,169}{52.01}  & \textcolor[RGB]{169,169,169}{64.50}  & \textcolor[RGB]{169,169,169}{30.79}  & \textcolor[RGB]{169,169,169}{25.16}  & \textcolor[RGB]{169,169,169}{60.46}  & \textcolor[RGB]{169,169,169}{52.69}  & \textcolor[RGB]{169,169,169}{26.54}  & \textcolor[RGB]{169,169,169}{48.44}  & 33.14  & \textcolor[RGB]{169,169,169}{57.67}  & \textcolor[RGB]{169,169,169}{35.60}  & \textcolor[RGB]{169,169,169}{29.48}  \\
          & $\mathcal{S}_{7}$    & 91.29  & 91.19  & 89.05  & \textbf{94.73}  & 78.57  & 61.55  & \textbf{91.86}  & 94.21  & 64.48  & 80.67  & \textbf{76.86}  & \textbf{91.32}  & 77.77  & 71.33  \\
          & $\mathcal{S}_{8}$    & 82.55  & 90.27  & 79.19  & 83.55  & 66.19  & 65.62  & 83.81  & 79.47  & 61.73  & 79.30  & 66.14  & 80.56  & 64.43  & 63.74  \\
          & AVG   & 79.43  & 84.95  & 79.55  & 84.06  & 63.55  & 65.83  & 83.41  & 80.74  & 54.62  & 81.44  & 55.60  & 82.22  & 66.96  & 57.44  \\
    \hline
    \multirow{9}[2]{*}{ImageNet} & $\mathcal{S}_{1}$    & 61.52  & 77.28  & 62.96  & 67.81  & 68.41  & 61.67  & \textbf{74.31}  & 77.48  & 69.80  & 67.26  & 69.17  & 68.42  & 64.65  & 79.24  \\
          & $\mathcal{S}_{2}$    & 58.01  & 54.82  & \textcolor[RGB]{169,169,169}{46.96}  & 49.93  & 52.23  & \textcolor[RGB]{169,169,169}{52.81}  & 52.77  & 60.24  & 57.91  & 51.44  & 52.33  & 48.33  & 53.47  & 67.53  \\
          & $\mathcal{S}_{3}$    & 62.95  & 72.76  & 63.47  & 71.51  & 71.24  & 68.84  & 65.11  & \textbf{80.42}  & 62.89  & \textbf{71.20}  & 61.69  & 59.40  & 68.19  & 76.03  \\
          & $\mathcal{S}_{4}$    & 69.63  & 71.88  & 69.26  & 66.18  & 67.59  & 64.52  & 67.89  & 73.20  & 72.20  & 68.19  & 71.24  & 64.83  & 69.47  & \textbf{80.28}  \\
          & $\mathcal{S}_{5}$    & \textbf{79.51}  & \textbf{83.14}  & 65.30  & 61.31  & 64.59  & 68.47  & 65.01  & 69.87  & 70.72  & 62.82  & 65.01  & 67.89  & 65.25  & 78.97  \\
          & $\mathcal{S}_{6}$    & \textcolor[RGB]{169,169,169}{55.21}  & \textcolor[RGB]{169,169,169}{50.13}  & 48.68  & \textcolor[RGB]{169,169,169}{46.01}  & \textcolor[RGB]{169,169,169}{49.08}  & 53.54  & \textcolor[RGB]{169,169,169}{49.67}  & \textcolor[RGB]{169,169,169}{49.94}  & \textcolor[RGB]{169,169,169}{51.10}  & \textcolor[RGB]{169,169,169}{44.63}  & \textcolor[RGB]{169,169,169}{49.34}  & \textcolor[RGB]{169,169,169}{46.33}  & \textcolor[RGB]{169,169,169}{48.85}  & \textcolor[RGB]{169,169,169}{58.65}  \\
          & $\mathcal{S}_{7}$    & 79.38  & 77.71  & \textbf{72.36}  & \textbf{77.26}  & \textbf{73.52}  & \textbf{76.74}  & 71.66  & 77.47  & \textbf{75.46}  & 69.55  & \textbf{71.38}  & \textbf{74.10}  & \textbf{70.98}  & 70.72  \\
          & $\mathcal{S}_{8}$    & 72.18  & 68.43  & 68.18  & 65.24  & 62.74  & 64.47  & 64.66  & 65.66  & 68.02  & 64.11  & 68.57  & 67.59  & 66.77  & 73.32  \\
          & AVG   & 67.30  & 69.52  & 62.15  & 63.16  & 63.68  & 63.88  & 63.88  & 69.28  & 66.01  & 62.40  & 63.59  & 62.11  & 63.45  & 73.10  \\
    \bottomrule[1.5pt]
    \end{tabular}%
    }
  \label{tab:attack_performance_per1}%
\end{table*}%

% Table generated by Excel2LaTeX from sheet '补充材料'
\begin{table*}[htbp]
\setlength{\abovecaptionskip}{2pt}
  \centering
  \caption{The attack success rate (\%) of Adv-PAT under different settings. $\mathcal{S}_{1}$ - $\mathcal{S}_{8}$ represent the same settings as mentioned in \cref{tab:attack_performance_per1}.}
   \scalebox{0.7}{
    \begin{tabular}{c|c|cccccccccccccc}
    \toprule[1.5pt]
    Dataset & Setting & Barlow & BYOL  & DeepC2 & DINO  & MoCo2+ & MoCo3 & NNCLR & ReSSL & SimCLR & SupCon & SwAV  & VIbCReg & VICReg & W-MSE \\
    \hline
    \multirow{9}[2]{*}{CIFAR10} & $\mathcal{S}_{1}$    & \textcolor[RGB]{169,169,169}{83.47}  & 79.55  & 90.88  & 88.61  & \textcolor[RGB]{169,169,169}{81.60}  & 88.83  & \textcolor[RGB]{169,169,169}{65.55}  & 73.08  & 89.91  & 80.03  & 89.30  & 57.27  & 87.58  & 88.61  \\
          & $\mathcal{S}_{2}$    & 88.78  & 80.98  & \textcolor[RGB]{169,169,169}{87.66}  & 79.87  & 82.51  & \textcolor[RGB]{169,169,169}{77.50}  & 75.83  & 73.19  & 89.32  & 69.42  & \textcolor[RGB]{169,169,169}{81.41}  & 56.82  & 82.36  & 81.32  \\
          & $\mathcal{S}_{3}$    & 93.14  & 89.95  & 95.44  & 86.40  & \textbf{99.08}  & 92.43  & 89.84  & 88.62  & 94.98  & 88.27  & 97.09  & 86.43  & 94.47  & 88.97  \\
          & $\mathcal{S}_{4}$    & 93.41  & \textbf{98.03}  & 99.53  & \textbf{98.16}  & 98.55  & 97.15  & 94.25  & \textbf{97.85}  & \textbf{98.97}  & \textbf{96.29}  & \textbf{98.53}  & \textbf{94.82}  & \textbf{97.76}  & \textbf{96.33}  \\
          & $\mathcal{S}_{5}$    & 83.58  & 87.64  & 90.88  & 82.59  & 86.05  & 89.83  & 67.80  & \textcolor[RGB]{169,169,169}{64.38}  & 89.90  & 77.01  & 89.30  & 53.89  & \textcolor[RGB]{169,169,169}{75.92}  & 87.02  \\
          & $\mathcal{S}_{6}$    & 83.79  & \textcolor[RGB]{169,169,169}{78.71}  & 89.67  & \textcolor[RGB]{169,169,169}{71.72}  & 82.41  & 83.64  & 66.58  & 75.40  & \textcolor[RGB]{169,169,169}{88.42}  & \textcolor[RGB]{169,169,169}{56.32}  & 83.92  & \textcolor[RGB]{169,169,169}{52.26}  & {78.69}  & \textcolor[RGB]{169,169,169}{79.55}  \\
          & $\mathcal{S}_{7}$    & 92.84  & 88.83  & 94.53  & 87.63  & 98.27  & 92.41  & 87.88  & 88.94  & 95.23  & 84.60  & 97.09  & 84.20  & 90.52  & 86.22  \\
          & $\mathcal{S}_{8}$    & \textbf{93.64}  & 96.68  & \textbf{99.70}  & 95.97  & 98.25  & \textbf{97.32}  & \textbf{94.45}  & 97.80  & 98.94  & 94.43  & 98.17  & 94.10  & 94.59  & 95.01  \\
          & AVG   & 89.08  & 87.55  & 93.53  & 86.37  & 90.84  & 89.89  & 80.27  & 82.41  & 93.21  & 80.80  & 91.85  & 72.47  & 87.74  & 87.88  \\
    \hline
    \multirow{9}[2]{*}{ImageNet} & $\mathcal{S}_{1}$    & 89.33  & 88.45  & \textcolor[RGB]{169,169,169}{89.22}  & 89.41  & \textcolor[RGB]{169,169,169}{87.28}  & \textcolor[RGB]{169,169,169}{88.80}  & 88.76  & 92.01  & 90.31  & 90.50  & \textcolor[RGB]{169,169,169}{90.06}  & 89.04  & 89.18  & 91.15  \\
          & $\mathcal{S}_{2}$    & \textcolor[RGB]{169,169,169}{83.53}  & 88.11  & 89.98  & \textcolor[RGB]{169,169,169}{89.09}  & 90.65  & 91.22  & 88.86  & 91.11  & \textcolor[RGB]{169,169,169}{89.26}  & 90.92  & 90.28  & \textcolor[RGB]{169,169,169}{86.13}  & 87.55  & \textcolor[RGB]{169,169,169}{89.84}  \\
          & $\mathcal{S}_{3}$    & 93.37  & \textbf{99.19}  & 97.04  & 94.95  & 95.54  & \textbf{98.47}  & 98.36  & 90.67  & 94.33  & 94.52  & 97.07  & 95.42  & 95.31  & 98.30  \\
          & $\mathcal{S}_{4}$    & \textbf{98.60}  & 98.63  & 99.21  & \textbf{98.79}  & 98.06  & 98.30  & \textbf{98.40}  & 98.17  & \textbf{99.02}  & \textbf{99.15}  & 98.66  & \textbf{98.59}  & \textbf{98.79}  & \textbf{98.49}  \\
          & $\mathcal{S}_{5}$    & 86.45  & 88.95  & \textcolor[RGB]{169,169,169}{89.22}  & 89.41  & 91.26  & 89.04  & 88.72  & 92.00  & 90.30  & 90.50  & \textcolor[RGB]{169,169,169}{90.06}  & 89.03  & 89.49  & 91.19  \\
          & $\mathcal{S}_{6}$    & 87.00  & \textcolor[RGB]{169,169,169}{87.15}  & 89.98  & 89.48  & 90.58  & 91.50  & \textcolor[RGB]{169,169,169}{88.15}  & 91.19  & 89.60  & \textcolor[RGB]{169,169,169}{90.21}  & 90.15  & 89.88  & \textcolor[RGB]{169,169,169}{87.54}  & 89.86  \\
          & $\mathcal{S}_{7}$    & 94.05  & 97.37  & 96.72  & 94.65  & 96.54  & 97.43  & 98.36  & \textcolor[RGB]{169,169,169}{90.65}  & 94.30  & 92.31  & 97.09  & 96.29  & 96.02  & 98.33  \\
          & $\mathcal{S}_{8}$    & 98.54  & 98.61  & \textbf{99.22}  & 98.43  & \textbf{98.30}  & 98.19  & 98.25  & \textbf{98.32}  & \textbf{99.02}  & 98.38  & \textbf{98.88}  & 98.56  & 98.77  & 98.43  \\
          & AVG   & 91.36  & 93.31  & 93.82  & 93.03  & 93.52  & 94.12  & 93.48  & 93.01  & 93.27  & 93.31  & 94.03  & 92.87  & 92.83  & 94.45  \\
    \bottomrule[1.5pt]
    \end{tabular}%
    }
  \label{tab:attack_performance_pat1}%
\end{table*}%


% Table generated by Excel2LaTeX from sheet 'Sheet1'
\begin{table*}[htbp]
\setlength{\abovecaptionskip}{2pt}
  \centering
   \caption{The retrieval attack performance (\%) of AdvEncoder under different settings on the pre-training dataset CIFAR10. PD denotes the pre-training dataset, SD indicates the attacker’s surrogate dataset  and DD represents the downstream dataset. }
        \scalebox{0.675}{
    \begin{tabular}{c|c|c|c|cccccccccccc}
    \toprule[1.5pt]
    \multirow{2}{*}{PD} & \multirow{2}{*}{SD} & \multirow{2}{*}{DD} & \multirow{2}{*}{Model} & \multicolumn{2}{c}{top1} & \multicolumn{2}{c}{top5} & \multicolumn{2}{c}{top10} & \multicolumn{2}{c}{top20} & \multicolumn{2}{c}{top50} & \multicolumn{2}{c}{top100} \\
          &       &       &       & p\_mAP & pat\_mAP & p\_mAP & pat\_mAP & p\_mAP & pat\_mAP & p\_mAP & pat\_mAP & p\_mAP & pat\_mAP & p\_mAP & pat\_mAP \\
     \hline
    \multirow{24}[8]{*}{CIFAR10} & \multirow{12}[4]{*}{CIFAR10} & \multirow{6}[2]{*}{STL10} & Barlow & 8.33  & 11.00  & 21.03  & 23.80  & 22.90  & 24.97  & 21.75  & 21.54  & 16.53  & 16.89  & 14.13  & 14.10  \\
          &       &       & BYOL  & 10.00  & 10.00  & 19.89  & 18.15  & 21.96  & 20.65  & 20.79  & 19.82  & 16.83  & 16.44  & 14.26  & 13.96  \\
          &       &       & DINO  & 10.00  & 9.00  & 19.13  & 19.26  & 20.85  & 21.39  & 20.07  & 20.22  & 16.50  & 16.45  & 13.61  & 14.27  \\
          &       &       & MoCo2+  & 12.00  & 9.33  & 21.88  & 21.00  & 22.74  & 23.28  & 20.96  & 23.05  & 16.61  & 17.08  & 14.22  & 14.25  \\
          &       &       & NNCLR & 9.00  & 9.33  & 18.29  & 20.31  & 20.49  & 21.85  & 19.46  & 21.51  & 16.13  & 16.99  & 13.85  & 14.13  \\
          &       &       & SimCLR & 10.67  & 9.67  & 21.55  & 17.14  & 23.39  & 20.40  & 21.16  & 19.69  & 17.05  & 16.22  & 14.22  & 14.13  \\
\cline{3-16}          &       & \multirow{6}[2]{*}{GTSRB} & Barlow & 18.33  & 10.00  & 21.57  & 10.00  & 22.21  & 10.00  & 21.83  & 10.05  & 21.30  & 10.06  & 20.92  & 10.07  \\
          &       &       & BYOL  & 14.33  & 14.33  & 15.64  & 17.41  & 15.56  & 17.43  & 15.61  & 17.66  & 15.17  & 17.91  & 14.83  & 17.77  \\
          &       &       & DINO  & 12.67  & 17.33  & 18.13  & 25.29  & 18.22  & 24.75  & 17.82  & 24.06  & 17.13  & 23.57  & 16.17  & 23.36  \\
          &       &       & MoCo2+  & 30.00  & 12.33  & 34.01  & 12.34  & 33.86  & 12.24  & 32.81  & 12.21  & 30.93  & 12.17  & 29.37  & 12.08  \\
          &       &       & NNCLR & 16.33  & 18.67  & 18.16  & 22.88  & 17.64  & 23.18  & 17.28  & 22.86  & 16.94  & 22.77  & 16.87  & 22.79  \\
          &       &       & SimCLR & 46.00  & 10.00  & 50.10  & 10.33  & 49.21  & 10.38  & 48.37  & 10.49  & 46.68  & 10.43  & 45.26  & 10.41  \\
\cline{2-16}          & \multirow{12}[4]{*}{ImageNet} & \multirow{6}[2]{*}{STL10} & Barlow & 11.00  & 11.00  & 20.03  & 21.01  & 23.10  & 22.43  & 21.75  & 20.77  & 17.43  & 16.64  & 14.45  & 14.20  \\
          &       &       & BYOL  & 12.67  & 10.00  & 23.25  & 17.04  & 24.60  & 18.83  & 22.69  & 20.95  & 17.37  & 17.14  & 14.55  & 14.25  \\
          &       &       & DINO  & 8.00  & 10.00  & 18.41  & 18.76  & 20.01  & 20.51  & 19.00  & 20.58  & 15.78  & 16.34  & 13.43  & 14.00  \\
          &       &       & MoCo2+  & 10.67  & 8.33  & 20.19  & 16.30  & 22.34  & 17.62  & 20.48  & 18.27  & 16.85  & 15.63  & 14.29  & 13.62  \\
          &       &       & NNCLR & 8.33  & 11.33  & 18.97  & 22.09  & 22.26  & 24.21  & 20.94  & 22.41  & 16.14  & 17.40  & 13.83  & 14.19  \\
          &       &       & SimCLR & 9.33  & 11.33  & 19.62  & 16.52  & 21.24  & 19.95  & 20.30  & 18.47  & 16.10  & 16.27  & 13.82  & 14.24  \\
\cline{3-16}          &       & \multirow{6}[2]{*}{GTSRB} & Barlow & 12.67  & 16.00  & 17.05  & 17.79  & 17.08  & 17.76  & 15.48  & 16.79  & 14.21  & 16.37  & 13.63  & 16.08  \\
          &       &       & BYOL  & 13.00  & 10.33  & 17.36  & 12.83  & 16.71  & 13.37  & 16.31  & 13.33  & 15.93  & 13.58  & 15.39  & 14.03  \\
          &       &       & DINO  & 14.00  & 26.00  & 19.37  & 32.69  & 19.17  & 33.77  & 18.78  & 33.33  & 18.57  & 32.10  & 18.39  & 31.49  \\
          &       &       & MoCo2+  & 27.67  & 15.00  & 31.15  & 15.04  & 31.58  & 14.81  & 31.37  & 14.76  & 30.86  & 14.54  & 30.42  & 14.24  \\
          &       &       & NNCLR & 8.67  & 20.33  & 10.18  & 24.39  & 10.32  & 24.19  & 10.55  & 23.32  & 11.17  & 22.67  & 11.22  & 22.30  \\
          &       &       & SimCLR & 13.00  & 10.00  & 21.82  & 10.00  & 22.07  & 10.04  & 22.14  & 10.04  & 22.05  & 10.04  & 21.58  & 10.11  \\
    \bottomrule[1.5pt]
    \end{tabular}%
    }
  \label{tab:retrieval_attack_cifar10}%
\end{table*}%
 
% Table generated by Excel2LaTeX from sheet 'Sheet1'
\begin{table*}[htbp]
\setlength{\abovecaptionskip}{2pt}
  \centering
   \caption{The retrieval attack performance (\%) of AdvEncoder under different settings on the pre-training dataset ImageNet}
     \scalebox{0.675}{
    \begin{tabular}{c|c|c|c|cccccccccccc}
    \toprule[1.5pt]
    \multirow{2}{*}{PD} & \multirow{2}{*}{SD} & \multirow{2}{*}{DD} & \multirow{2}{*}{Model} & \multicolumn{2}{c}{top1} & \multicolumn{2}{c}{top5} & \multicolumn{2}{c}{top10} & \multicolumn{2}{c}{top20} & \multicolumn{2}{c}{top50} & \multicolumn{2}{c}{top100} \\
          &       &       &       & p\_mAP & pat\_mAP & p\_mAP & pat\_mAP & p\_mAP & pat\_mAP & p\_mAP & pat\_mAP & p\_mAP & pat\_mAP & p\_mAP & pat\_mAP \\
    \hline
    \multirow{24}[8]{*}{ImageNet} & \multirow{12}[4]{*}{CIFAR10} & \multirow{6}[2]{*}{STL10} & Barlow & 11.67  & 9.00  & 21.70  & 19.44  & 23.26  & 21.15  & 21.55  & 20.68  & 17.02  & 16.36  & 14.27  & 13.76  \\
          &       &       & BYOL  & 9.67  & 8.00  & 20.25  & 18.01  & 21.76  & 19.64  & 21.03  & 19.26  & 16.36  & 16.84  & 13.97  & 14.03  \\
          &       &       & DINO  & 9.67  & 9.33  & 22.10  & 19.82  & 22.95  & 21.12  & 22.01  & 19.93  & 17.18  & 16.10  & 14.53  & 14.15  \\
          &       &       & MoCo2+  & 11.00  & 14.00  & 20.73  & 23.04  & 22.77  & 26.89  & 21.19  & 25.05  & 16.36  & 20.26  & 13.90  & 14.68  \\
          &       &       & NNCLR & 12.33  & 10.67  & 23.82  & 22.85  & 24.99  & 26.22  & 22.92  & 24.89  & 17.90  & 17.20  & 14.98  & 13.60  \\
          &       &       & SimCLR & 9.67  & 11.33  & 19.80  & 21.07  & 21.76  & 23.59  & 20.13  & 21.46  & 16.62  & 17.04  & 14.31  & 13.99  \\
\cline{3-16}          &       & \multirow{6}[2]{*}{GTSRB} & Barlow & 38.67  & 10.00  & 43.57  & 11.16  & 42.81  & 11.63  & 41.57  & 11.67  & 39.82  & 11.22  & 38.66  & 11.21  \\
          &       &       & BYOL  & 40.67  & 10.00  & 46.60  & 10.07  & 45.81  & 10.07  & 43.88  & 10.07  & 41.21  & 10.24  & 39.33  & 10.15  \\
          &       &       & DINO  & 26.67  & 10.00  & 30.54  & 10.07  & 30.72  & 10.17  & 29.85  & 10.24  & 29.45  & 10.55  & 28.69  & 10.60  \\
          &       &       & MoCo2+  & 32.33  & 7.00  & 39.30  & 11.07  & 38.93  & 11.75  & 37.79  & 12.25  & 35.51  & 11.55  & 34.81  & 11.23  \\
          &       &       & NNCLR & 30.67  & 9.00  & 36.07  & 11.96  & 36.14  & 11.66  & 35.22  & 11.97  & 33.33  & 12.04  & 32.50  & 12.16  \\
          &       &       & SimCLR & 40.33  & 10.00  & 45.80  & 10.42  & 45.56  & 13.63  & 44.11  & 15.75  & 42.12  & 14.76  & 40.88  & 14.68  \\
\cline{2-16}          & \multirow{12}[4]{*}{ImageNet} & \multirow{6}[2]{*}{STL10} & Barlow & 13.00  & 9.67  & 22.27  & 19.99  & 23.02  & 21.75  & 21.45  & 20.18  & 16.92  & 16.28  & 14.09  & 13.76  \\
          &       &       & BYOL  & 11.33  & 11.33  & 20.36  & 20.41  & 22.14  & 23.85  & 21.46  & 20.93  & 16.94  & 15.90  & 14.03  & 13.51  \\
          &       &       & DINO  & 9.00  & 9.33  & 19.75  & 20.21  & 21.46  & 22.75  & 21.04  & 20.28  & 16.44  & 16.29  & 13.90  & 14.45  \\
          &       &       & MoCo2+  & 9.67  & 8.00  & 20.40  & 19.23  & 21.12  & 22.28  & 20.02  & 23.67  & 16.33  & 17.06  & 13.82  & 13.80  \\
          &       &       & NNCLR & 11.00  & 11.33  & 20.59  & 19.96  & 22.82  & 24.91  & 20.74  & 23.10  & 16.89  & 16.33  & 14.17  & 13.62  \\
          &       &       & SimCLR & 9.33  & 9.00  & 19.36  & 21.31  & 21.43  & 23.91  & 20.83  & 20.59  & 16.84  & 16.37  & 14.15  & 13.64  \\
\cline{3-16}          &       & \multirow{6}[2]{*}{GTSRB} & Barlow & 28.67  & 9.33  & 34.08  & 10.68  & 34.65  & 11.10  & 34.38  & 11.61  & 34.14  & 11.99  & 33.92  & 12.08  \\
          &       &       & BYOL  & 35.33  & 10.00  & 41.06  & 10.00  & 41.40  & 10.00  & 38.94  & 10.12  & 36.42  & 10.30  & 34.38  & 10.22  \\
          &       &       & DINO  & 27.00  & 10.00  & 31.89  & 10.12  & 31.57  & 10.05  & 30.65  & 10.09  & 29.95  & 10.11  & 29.18  & 10.06  \\
          &       &       & MoCo2+  & 22.33  & 12.33  & 33.48  & 13.18  & 32.40  & 13.73  & 30.93  & 13.29  & 29.12  & 12.48  & 27.94  & 12.07  \\
          &       &       & NNCLR & 31.67  & 10.00  & 36.82  & 12.05  & 37.07  & 13.04  & 37.21  & 13.03  & 35.99  & 13.29  & 35.20  & 12.54  \\
          &       &       & SimCLR & 19.00  & 10.00  & 21.68  & 11.83  & 21.58  & 12.41  & 21.42  & 12.72  & 20.90  & 12.93  & 20.36  & 13.37  \\
    \bottomrule[1.5pt]
    \end{tabular}%
    }
  \label{tab:retrieval_attack_imagenet}%
\end{table*}%



\subsection{Attack Performance on Retrieval}  \label{attack_performance_retrieval}
% \noindent\textbf{Implementation Details.}

 We aim to investigate the impact of AdvEncoder on the retrieval accuracy of downstream tasks under different settings. 
 We select Barlow Twins\cite{zbontar2021barlow}, MoCo v2+\cite{chen2020improved}, BYOL\cite{grill2020bootstrap}, DINO\cite{caron2021emerging}, NNCLR\cite{dwibedi2021little}, and SimCLR \cite{chen2020simple} as pre-trained encoders to evaluate the performance of AdvEncoder in retrieval downstream tasks.
We use CIFAR10 and ImageNet as pre-training datasets for the victim models. As shown in \cref{tab:clean_retrieval}, we provide the precision (mAP) corresponding to their different settings on STL10 and GTSRB downstream retrieval tasks. We then use CIFAR10 and ImageNet as the surrogate datasets for the attacker to launch attacks on the downstream tasks.
We use per-mAP and pat-mAP metrics, where lower values indicate better attack performance. 
The results in ~\cref{tab:retrieval_attack_cifar10} and ~\cref{tab:retrieval_attack_imagenet} illustrate that AdvEncoder can successfully attack the downstream retrieval task without any knowledge of the pre-training and downstream datasets.

% Figure environment removed


% Figure environment removed



\subsection{Attack Performance on Object Detection \& Semantic Segmentation}  
We provide the attack performance of AdvEncoder in  \textbf{Object Detection} and \textbf{Semantic Segmentation} tasks using ImageNet as the surrogate dataset in \cref{fig:retriveal1}(a) - (b). 
We employe the official MOCOv2 model based on ResNet50 and fine-tune it on the COCO dataset using Mask R-CNN for the above two types of tasks.
The results in~\cref{fig:retriveal1}(a) - (b) illustrate that AdvEncoder can successfully attack the two types of downstream tasks.



\section{Supplemental Ablation Study} \label{ablation}
In this section, we explore the effect of different random seeds and backbones on the attack performance of AdvEncoder. 
The following experimental settings are consistent with the main body. 
We choose CIFAR10 as the surrogate dataset and GTSRB as the downstream dataset.

\noindent\textbf{The Effect of Random Seed.} 
The default random number seed for our experiments is 100, and we further provide results for different randomized seeds in \cref{fig:retriveal1}(c).

\noindent\textbf{The Effect of Backbone.}
We provide ASRs for downstream GTSRB tasks for five architectures of CLIP (ResNet50, ResNet101, ViT-B/16, ViT-B/32, ViT-L/14).
The results in \cref{fig:retriveal1}(d) show that AdvEncoder can successfully attack downstream tasks based on the pre-trained encoders with different backbones.
 \vspace{-4mm}


\section{Supplemental Transferability Study}  \label{transferability}

In this section, we aim to investigate the transferability of AdvEncoder from two distinct perspectives: pre-training datasets and crossing SSL methods. The experimental settings in this analysis are consistent with those outlined in the main body of the paper. 
From~\cref{fig:transfer}, we explore the attack performance of Adv-PAT and Adv-PER in different transportability scenarios.
C2I-GTS-PAT represents the two encoders we trained using CIFAR10 and ImageNet, on which we made adversarial examples of Adv-PAT and downstream tasks of GTSRB, respectively. I2I-STL-PER represents the two encoders we trained using ImageNet and ImageNet, on which we made adversarial examples of Adv-PER and downstream tasks of STL10, respectively.
The other captions have the same definition.
We can see that Adv-Encoder has good transferability between different downstream tasks based on different encoders.



\section{Visualization}\label{Visualization}
In this section, as shown in the \cref{fig:visualization_advencoder}, we show adversarial perturbations and patches generated by AdvEncoder using the attacker’s surrogate dataset CIFAR10 for each of the fourteen SSL encoders trained with ImageNet.



% Figure environment removed

\end{document}
