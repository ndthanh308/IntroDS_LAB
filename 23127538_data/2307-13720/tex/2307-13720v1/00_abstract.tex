\begin{abstract}
For an artist or a graphic designer, the spatial layout of a scene is a critical design choice. However, existing text-to-image diffusion models provide limited support for incorporating spatial information. This paper introduces \textbf{Composite Diffusion} as a means for artists to generate high-quality images by composing from the sub-scenes. The artists can specify the arrangement of these sub-scenes through a flexible free-form segment layout. They can describe the content of each sub-scene primarily using natural text and additionally by utilizing reference images or control inputs such as line art, scribbles, human pose, canny edges, and more.

We provide a comprehensive and modular method for Composite Diffusion that enables alternative ways of generating, composing, and harmonizing sub-scenes. Further, we wish to evaluate the composite image for effectiveness in both image quality and achieving the artist's intent. We argue that existing image quality metrics lack a holistic evaluation of image composites. To address this, we propose novel quality criteria especially relevant to composite generation. 

We believe that our approach provides an intuitive method of art creation. Through extensive user surveys, quantitative and qualitative analysis, we show how it achieves greater spatial, semantic, and creative control over image generation. In addition,  our methods do not need to retrain or modify the architecture of the base diffusion models and can work in a plug-and-play manner with the fine-tuned models.
\end{abstract}

