\section{Artworks exercise}
\label{appendix_sec:artworks}
Here, we present the outcomes of a brief collaboration with an artist. The aim of this exercise was to expose the modalities of our system to an external artist and gather early-stage user feedback. To achieve this, we explained the workings of the Composite Diffusion system to the artist and asked her to provide us with 2-3 specific scenarios of artwork that she would normally like to work on. The artist's inputs were given to us on plain sheets of paper in the form of rough drawings of the intended paintings, with clear labels for various objects and sections.

We converted these inputs into the bimodal input - the free-form segment layouts and text descriptions for each segment.  We did not create any additional control inputs.  We then supplied these inputs to our Composite Diffusion algorithm and performed many iterations with base and a few fine-tuned models, and also at different scaffolding values. The outputs were first shown to a few internal artists for quick feedback, and the final selected outputs were shared with the original artist.  For the final shared outputs, refer to Figures \ref{fig:artwork-picnic-a} and \ref{fig:artwork-picnic-b}  for input 1, Figures \ref{fig:artwork-bear-a} and \ref{fig:artwork-bear-b} for input 2, and Figures \ref{fig:artwork-fairy-a}, \ref{fig:artwork-fairy-b}, and \ref{fig:artwork-fairy-c} for input 3. Please note that the objective here was to produce an artwork with artistically satisfying outputs. So, for some of our generations, we even allowed the outputs which were creative and caught the overall intent of the artist, but did not strictly conform to the prescribed inputs by the artist.

The feedback that we received from the artist at the end of the exercise (as received on Jan 25, 2023), is presented here verbatim:

\textbf{For Artwork 1} (refer to Figures \ref{fig:artwork-picnic-a} and \ref{fig:artwork-picnic-b}):
% \begin{quote}
\textit{``The intended vision was of a lively scene with bright blue skies, picnic blossoms blooming, soft green grass with fallen pink petals, and a happy meal picnic basket. All the images are close enough to the description. Colors are bright and objects fit harmoniously with each other."}
% \end{quote}

\textbf{For Artwork 2} (refer to Figures \ref{fig:artwork-bear-a} and \ref{fig:artwork-bear-b}):
% \begin{quote}
\textit{``The intended vision was of bears in their natural habitat, surrounded by forest trees and snow-clad mountains, catching fish in the stream. Overall, the bears, mountains, trees, rocks, and streams are quite realistic. However, not a single bear could catch a fish. Bear 4 looks like an Afgan hound (dog breed with long hair) and bear 5 itself became a mountain. In image 10, the objects have merged into one another, having ill-defined margins and boundaries."}
% \end{quote}

\textbf{For Artwork 3}(refer to Figures \ref{fig:artwork-fairy-a}, \ref{fig:artwork-fairy-b}, and \ref{fig:artwork-fairy-c}):
% \begin{quote}
\textit{``The intended vision was of an angel - symbolic of hope \& light, salvaging a dejected man, bound by the shackles of hopelessness and despair. Each and every angel is a paragon of beauty. Compared to the heavenly angels, the desolate men look a bit alienish. There is a slight disproportion seen between man and angel in some images. My personal best ones are no. 1,3,6,10."}
% \end{quote}

In another case, we wanted to check if our system can be effectively used for artistic pattern generation. Here we gave the system an abstract pattern in the form of a segment layout and specified the objects that we want to fill within those segments. Figure \ref{fig:artwork-flowers} shows a sample output of such an exercise where we fill in the pattern with different forms of flowers.

While the exercise of interactions with the artist and the application of the system for creating the artwork was informal and done in a limited manner, still, it demonstrated to us at an early stage the effectiveness of Composite Diffusion in creating usable art in real-life scenarios. It also validated that the art workflow was simple and intuitive enough to be adopted by people with varied levels of art skills. 



% Figure environment removed

% Figure environment removed

% Figure environment removed

% Figure environment removed

% Figure environment removed

% Figure environment removed

% Figure environment removed

% Figure environment removed



